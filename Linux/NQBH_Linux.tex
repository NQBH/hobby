\documentclass{article}
\usepackage[backend=biber,natbib=true,style=authoryear]{biblatex}
\addbibresource{/home/hong/1_NQBH/reference/bib.bib}
\usepackage{tocloft}
\renewcommand{\cftsecleader}{\cftdotfill{\cftdotsep}}
\usepackage[colorlinks=true,linkcolor=blue,urlcolor=red,citecolor=magenta]{hyperref}
\usepackage{algorithm,algpseudocode,amsmath,amssymb,amsthm,float,graphicx,mathtools}
\allowdisplaybreaks
\numberwithin{equation}{section}
\newtheorem{assumption}{Assumption}[section]
\newtheorem{conjecture}{Conjecture}[section]
\newtheorem{corollary}{Corollary}[section]
\newtheorem{definition}{Definition}[section]
\newtheorem{example}{Example}[section]
\newtheorem{lemma}{Lemma}[section]
\newtheorem{notation}{Notation}[section]
\newtheorem{principle}{Principle}[section]
\newtheorem{problem}{Problem}[section]
\newtheorem{proposition}{Proposition}[section]
\newtheorem{question}{Question}[section]
\newtheorem{remark}{Remark}[section]
\newtheorem{theorem}{Theorem}[section]
\usepackage[left=0.5in,right=0.5in,top=1.5cm,bottom=1.5cm]{geometry}
\usepackage{fancyhdr}
\pagestyle{fancy}
\fancyhf{}
\lhead{\small \textsc{Sect.} ~\thesection}
\rhead{\small \nouppercase{\leftmark}}
\renewcommand{\sectionmark}[1]{\markboth{#1}{}}
\cfoot{\thepage}
\def\labelitemii{$\circ$}

\title{Linux}
\author{Nguyen Quan Ba Hong\footnote{Independent Researcher, Ben Tre City, Vietnam\\e-mail: \texttt{nguyenquanbahong@gmail.com}}}
\date{\today}

\begin{document}
\maketitle
\begin{abstract}
	Some notes on Unix\texttt{/}Linux.
\end{abstract}
\tableofcontents

I have used the following Linux distributions: \textsc{Ubuntu, Kubuntu, SUSE, OpenSUSE}, where the 1st one is install in my personal notebooks\texttt{/}laptops, the 2nd one is install in my WIAS Dell XPS 15 (KDE provides a lot of good stuffs), while the other ones are install in computer servers at WIAS Berlin. Note that SUSE \& \textsc{OpenSUSE} are produced by Germans\footnote{In my own experiences, I believe that Germans \& Frenches prefer building their own facilities over using those provided by the rest of the world.}.

\begin{question}
	Which resources I should use to learn deeply Linux in a fast but complete way?
\end{question}
William Shotts. \textit{The Linux Command Line: A Complete Introduction} (2e) seems a good start.

%------------------------------------------------------------------------------%

\printbibliography[heading=bibintoc]
	
\end{document}