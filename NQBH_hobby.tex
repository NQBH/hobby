\documentclass{article}
\usepackage[backend=biber,natbib=true,style=alphabetic,maxbibnames=50]{biblatex}
\addbibresource{/home/nqbh/reference/bib.bib}
\usepackage[utf8]{vietnam}
\usepackage{tocloft}
\renewcommand{\cftsecleader}{\cftdotfill{\cftdotsep}}
\usepackage[colorlinks=true,linkcolor=blue,urlcolor=red,citecolor=magenta]{hyperref}
\usepackage{amsmath,amssymb,amsthm,enumitem,float,graphicx,mathtools}
\newtheorem{question}{Question}
\usepackage[left=1cm,right=1cm,top=5mm,bottom=5mm,footskip=4mm]{geometry}
\setlist[itemize]{leftmargin=*}
\setlist[enumerate]{leftmargin=*}

\title{Anime, Manga, Movie, Music, Quote, {\it\&} Sport}
\author{Nguyễn Quản Bá Hồng\footnote{e-mail: {\sf nguyenquanbahong@gmail.com}, website: \url{https://nqbh.github.io}, Bến Tre, Việt Nam.}}
\date{\today}

\begin{document}
\maketitle
\begin{abstract}
	A personal collection of animes, mangas, movies, music (lyrics, instrumental music, etc.), legends, \& quotes.
	
	Latest version:
	\begin{itemize}
		\item {\it Anime, Manga, Movie, Music, Quote, \& Sport}.
		
		PDF: {\sc url}: \url{https://github.com/NQBH/hobby/blob/master/anime_manga_movie_music_quote/NQBH_anime_manga_movie_music_quote.pdf}.
		
		\TeX: {\sc url}: \url{https://github.com/NQBH/hobby/blob/master/anime_manga_movie_music_quote/NQBH_anime_manga_movie_music_quote.tex}.
	\end{itemize}
\end{abstract}
\setcounter{secnumdepth}{4}
\setcounter{tocdepth}{4}
\tableofcontents

%------------------------------------------------------------------------------%

\section{Anime, Manga, {\it\&} Manhwa}
I have a habit to watch animes with the speed 1.5x, 1.75x, or 2x, not at the normal speed so that I can have more free time \& watch much more animes \& others also.
\begin{enumerate}
	\item 2.5-jigen no Ririsa $\star$ 2.5 Dimensional Seduction (2024--)\hfill[S1.E22--]
	\item 86 (2021--2022)\hfill[S1.E11][S2.E12]
    \item {\sc Akira} (1988)
    \item {\sc Alice to Therese no Maboroshi Kôjô $\star$ Maboroshi} (2023)
    \item {\sc Ano hi mita hana no namae o bokutachi wa mada shiranai $\star$\\Anohana: The Flower We Saw That Day} (2011)\hfill[S1.E11]
    \item {\sc Anazâ $\star$ Another} (2012)\hfill[S1.E12]
    \item Ao no Hako $\star$ Blue Box (2024)\hfill[S1.E10--]
    \item Aoashi (2022--)\hfill[S1.E24][Chap. 327--]
    
    
    \item {\sc Arcane: League of Legends $\star$ Arcane} (2021--)\hfill[S1.E9][S2.E1--]
    \item  Arifureta Shokugyou de Sekai Saikyou $\star$ Arifureta: From Commonplace to World's Strongest
    
    (2019--2024)\hfill[S1.E13][SP2][S2.E12][SP][S3.E8--]
    \item {\sc Bakemono no ko $\star$ The Boy \& the Beast} (2015)
    \item {\sc Bakemonogatari} (2009--2013)\hfill[S1.E15][Chap. 10--]
    \item Black Lagoon (2006)\hfill[S1.E2--]
    \item Bleach: Sennen Kessen-hen $\star$ Bleach: Thousand-Year Blood War\hfill[S1.E35--]
    \item {\sc Blood-C} (2011)\hfill[S1.E13]
    \item Blood-C: The Last Dark (2012)
    \item {\sc Blue Eye Samurai} (2023)\hfill[S1.E12]
    \item Blue Lock (2022)\hfill[S1.E24][S2.E9--][Chap. 1--]
    \item Bocchi the Rock! (2022--)\hfill[S1.E12]
    \item {\sc BoJack Horseman} (2014--2020)\hfill[S1.E12][S2.E12][S3.E12][S4.E12][S5.E12][S6.E16]
    
    {\sf BoJack Horseman.}
    \begin{itemize}
    	\item ``He's so stupid he doesn't realize how miserable he should be. I envy that.''
    	\item ``Ow, crap. I hate this. Running is terrible. Everything is the worst.''
    	\item ``You know sometimes I feel like I was born with a leak, \& any goodness I started with just slowly spilled out of me, \& now it's all gone. And I'll never get it back in me. It's too late. ... Life is a series of closing doors, isn't it?''
    	\item``You are a horrible person, \& you not understanding that you're a horrible person, doesn't make you less of a horrible person.''
    	\item ``I'm responsible for my own happiness? I can't even be responsible for my own breakfast!''
    	\item ``You sleep on my couch \& you don't pay rent. I've had tapeworms that are less parasitic. I don't even remember why I let you stay with me in the 1st place.''
    	\item ``Man, I really regret buying those muffins \& then eating them all on the drive home.''
    	\item ``I feel like my life is just a series of unrelated wacky adventures''
    	\item ``I don't understand how people... live. It's amazing to me that people wake up every morning \& say: `Yeah, another day, let's do it.' How do people do it? I don't know how.''
    	\item ``Look, for a lot of people, life is just one long hard kick in the urethra.''
    	\item ``Same thing that always happens. You didn't know me \& then you fell in love with me. And now you know me.''
    	\item ``Now if you excuse me, I need to go take a shower so I can't tell if I'm crying or not.''
    	\item ``I spend a lot of time with the real me \& believe me, nobody's gonna love that guy.''
    	\item ``Yeah, I ate them all in one sitting because I have no self-control \& I hate myself.''
    	\item ``nothing on the outside, nothing on the inside'' as he runs his hand over the oven
    	\item ``Settle. Because otherwise you're just gonna get older \& harder, \& more alone. And you're gonna do everything you can to fill that hole, with friends, \& your career, \& meaningless sex, but the hole doesn't get filled. One day, you're gonna look around \& you're going to realize that everybody loves you, but nobody likes you. And that is the loneliest feeling in the world.''
    	\item ``I want to feel good about myself. The way you do. And I don't know how. I don't know if I can.''
    	\item ``Fool me once, shame on you. But teach a man to fool me \& I'll be fooled for the rest of my life'' 
    	\item ``Dead on the inside, dead on the outside.''
    	\item ``I have half a mind...''
    	\item ``I can't say no to people because I want everyone to like me.''
    	\item ``It's made up by Steven Spielberg to sell movie tickets. It's like the happiness or the Munich Olympics, it doesn't exist in the real world. The only thing to do now is just to keep living forward.''
    	\item ``We're just two lonely people trying to hate ourselves a little less.''
    	\item ``your girlfriend has a stripper name''
    	\item ``Kelsey, in this terrifying world, all we have are the connections that we make. I'm sorry I got you fired, I'm sorry I never called you after.''
    	\item ``Being a movie star is the hardest job, \& we get no recognition!''
    	\item ``Todd, your good-hearted naivety has once again conspired with outrageous happenstance to completely dick me over!''
    	\item ``Why 12 steps? Nobody wants to do 12 of anything. Did you see `12 Years A Slave' \& think `that's a short number of years to be a slave?'''
    	\item ``When your father dies, you ask yourself a lot of questions, like `Wait, did you say he died in a duel?' \& `Who dies in a duel?'''
    	\item ``Is it possible that this whole time, I've been an amazing feminist hero \& nobody knew it?''
    	\item ``Diane, I am a famous -- everyone gives me everything I want all the time. It is an existential curse, but a huge day-to-day convenience.''
    	\item ``You've got a nice set of pipes. You're like a Josh Groban who doesn't also think he's funny.''
    	\item ``I want to do things with you. Fully clothed, sober, in daylight hours.''
    	\item ``How can I put this... imagine if the Holocaust happened every four years like the Olympics. I would rather THAT happen than your rock opera.''
    	\item ``Goddammit, Honeydew? Jesus, why does Cantaloupe think every time it gets invited to a party it can bring along its dumb friend Honeydew? You don't get a plus one Cantaloupe.''
    	\item ``There's that old saying: Liquor before beer, never fear, don't do heroin.''
    	\item ``Why put the skip ad button so late? I'm not skipping now. I'm invested.''
    	\item ``I drove, but I moved my arm a bunch so the Fitbit counted the miles.''
    	\item ``That was, \& I don't say this lightly, worse than 100 September 11ths.''
    	\item ```Here's the thing about most long distance plans.' I hate that. `Here's the thing.' It's so stupid! Just say the thing! You don't need to introduce the concept that there's going to be a thing.''
    	\item ``Todd! Get in the car. It is time to get serious about autoerotic asphyxiation.''
    	\item ``Also, are you wearing comfortable shoes? Because I feel like that was a long road to walk to get to that punchline.''
    	\item ``I'm sorry I accused you of murder, American TV legend Henry Winkler.''
    	\item ``Slap my salami, the guy's a commie.''
    	\item ``I told you, that's not what was happening that time! I was masturbating to what the picture represented! And you came in at the worst possible time!''
    	\item ```Neigh way, Jose!' I improvised that line. I mean, it was scripted, but I gave it the ol' BoJack spin.''
    	\item ``Well, that was another in a long series of regrettable life choices.''
    	\item ``It's so sad that when you see someone as they really are, it ruins them.''
    	\item ``I've had a lot of what I thought were rock bottoms, only to discover another, rockier bottom underneath.''
    	\item ``Just pretend you are happy, \& eventually you'll forget you're pretending.''
    	\item ``I have poison inside me, \& I destroy everything I touch.''
    	\item ``The only thing to do now is just to keep living forward.''
    	\item ``My darling mother gave the eulogy. My entire life, I never heard her say a kind word to OR about my father, but at his funeral, she said, ``My husband is dead, \& everything is worse now.''
    	\item ``I'm gonna say ``Hello. I am BoJack Horseman. Obviously, you know who I am, because I'm very famous, \& also because we called ahead. And I am here because ... I need help.''
    	\item ``I don't hate the troops, I just hate one specific troop. I don't even hate him, I just think he's wrong about the muffins.''
    	\item ``All I learned about being good, I learned from TV. And in TV, flawed characters are constantly showing people they care with these surprising grand gestures. And I think that part of me still believes that's what love is.''
    	\item ``I told you, that's not what was happening that time. I was masturbating to what the picture represented! You walked in at the worst possible moment.''
    	\item ``You turn yourself around. That's what it's all about.''
    	\item ``I don't cry in front of other people.''
    	\item ``But in real life, the big gesture isn't enough. You need to be consistent, you need to be dependably good. You need to do it every day, which is so ... hard.''
    	\item ``What more do you want?! What else could the Universe possibly owe you?''
    \end{itemize}
    {\bf Diane Nguyen.}
    \begin{itemize}
    	\item ``It's not about being happy, that is the thing. I'm just trying to get through each day. I can't keep asking myself `Am I happy?' It just makes me more miserable. I don't know If I believe in it, real lasting happiness, All those perky, well-adjusted people you see in movies \& TV shows ? I don't think they exist.''
    	\item ``Idea for a new app: an undo button that can undo long amounts of time. Three months. A year. A life.''
    	\item ``Even if no one appreciates you, it's important that you don't stop being good.''
    	\item ``When I was a kid, I used to watch you on TV, \& you know I didn't have the best family. Things weren't that great for me. But, for half an hour, every week, I got to watch this show about four people who had nobody, who came together \& became a family, \& for half an hour, every week, I had a home, \& it helped me survive.''
    	\item ``Well, that's the problem with life, right? Either you know that you want \& then you don't get what you want, or you get what you want, \& then you don't know what you want.''
    	\item ``Most people don't even get to do the brady bunch version of the thing they want to do with their lives.''
    	\item ``You're responsible for your own happiness, you know?''
    	\item ``You learn that you can survive being alone.''
    	\item ``Sometimes life's a bitch \& you keep living.''
    	\item ``There are people in your life that help you become the person you end up being, \& you can be grateful for them, even if they were never meant to be in your life forever.''
    	\item ``Every happy ending has the day after the happy ending.''
    	\item ``There's no deep down, I believe that all we are is what we do.''
    	\item ``There's no such thing as ``bad guys'' or ``good guys''. We're all just... guys, who do good stuff sometimes \& bad stuff sometimes. And all we can do is try to do less bad stuff \& more good stuff, but you're never going to be good because you're not bad.''
    \end{itemize}
    {\bf Todd Chavez.}
    \begin{itemize}
    	\item ``Things don't become traditions because they're good, BoJack, they become good because they're traditions.''
    	\item ``You can't keep doing shitty things, \& then feel bad about yourself like that makes it okay! You need to be better!''
    	\item ``You are all the things that are wrong with you.''
    \end{itemize}
    {\bf Princess Carolyn.}
    \begin{itemize}
    	\item ``Oh fish. Of course! Why would you ever make things easy for me when instead you could make things incredibly difficult. Laura! Clear our my schedule! I have to push a boulder up a hill \& then have it roll over me time \& time again with no regard for my well-being!''
    	\item ``That woman can knock a drink back like a Kennedy at a wake for another Kennedy, but I'll be damned if she doesn't get shit done!''
    	\item ``BoJack, I'm gonna level with you, honey. This whole you-hating-the-troops thing is not great.''
    	\item ``Because my life is a mess right now \& I compulsively take care of other people.''
    	\item ``I got into this business because I love stories. They comfort us, they inspire us, they create a context for how we experience the world. But also, you have to be careful, because if you spend a lot of time with stories, you start to believe that life is just stories, \& it's not. Life is life, \& that's so sad because there's so little time \& ... what are we doing with it?''
    	\item ``So, are you available for Tuesday, or are you gonna be too busy masturbating to old pictures of yourself.``
    	\item ``Stop pissing off the orphans. A lot of them grow up to be serial killers.''
    \end{itemize}
    {\bf Mr. Peanutbutter.}
    \begin{itemize}
    	\item ``Sweetie, you know I support you, whatever you want to do, but you're not gonna find what you're looking for in these awful made-up places. The universe is a cruel, uncaring void. The key to being happy isn't the search for meaning; it's just to keep yourself busy with unimportant nonsense, \& eventually, you'll be dead.''
    	\item ``Everybody deserves to be loved''
    	\item ``Crack an egg on your head. Let the yolk drip down.''
    	\item ``All I ever wanted was to be your friend ... \& you treat me like a big joke. You think I don't notice? Why don't you like me?''
    	\item ``I told you I don't know where it is. Don't put things in my butt if you want them back.''
    \end{itemize}
    {\bf Herb Kazzaz.}
    \begin{itemize}
    	\item ``You know what your problem is? You want to think of yourself as the good guy. Well, I know you better than anyone, \& I can tell you that you're not. In fact, you'd probably sleep a lot better at night if you just admitted to yourself that you're a selfish goddamn coward who just takes whatever he wants \& doesn't give a shit about who he hurts. That's you. That's BoJack Horseman.''
    	\item ``There is no other side.''
    \end{itemize}
    {\bf Beatrice Horseman.}
    \begin{itemize}
    	\item ``I'm punishing you for being alive.''
    	\item ``I just wanted to tell you that I know. I know you want to be happy, but you won't be... \& I'm sorry.''
    	
    	BoJack: ``... What?''
    	
    	``It's not just you, you know. Your father \& I, we, well... you come by it honestly, the ugliness inside you. You were born broken, that's your birthright. And now you can fill your life with projects your books \& your movies \& your little girlfriends but... that won't make you whole. You're BoJack Horseman. There's no cure for that.''
    	\item ``Don't throw your dreams away for this child. Don't let that man poison your life the way he did mine. You are going to finish your schooling \& become a nurse. You'll meet a man, a good man, \& you'll have a family, but please, believe me, you don't want this. Please, Henrietta, you have to believe me. Please ... don't do what I did.''
    	\item ``Henrietta, don't use a foreign language in front of the child, she'll get ideas!''
    \end{itemize}
    {\bf Henry Winkler.}
    \begin{itemize}
    	\item ``There is no shame in dying for nothing. That's why most people die.''
    \end{itemize}
    {\bf Hollyhock.}
    \begin{itemize}
    	\item ``That voice, the one that tells you you're stupid, worthless, \& ugly... it goes away right? It's just a dumb teenage girl thing, but... then it goes away?'', ``Yeah''
    \end{itemize}
    {\bf Secretariat.}
    \begin{itemize}
    	\item ``BoJack, when I was your age, I got sad. A lot. I didn't come from such a great home, but one day, I started running, \& that seemed to make sense, so then I just kept running. BoJack, when you get sad you run, straight ahead \& you keep running forward no matter what there are people in your life that who are going to try to hold you back , slow you down , but you don't let them. Don't you ever stop running, don't ever look behind you there is nothing for you behind you. All that exists is whats ahead.'' 
    \end{itemize}
    {\bf Wanda Pierce.}
    \begin{itemize}
    	\item ``When you look at someone through rose-colored glasses, all the red flags just look like flags.''
    \end{itemize}
    {\bf Amanda Hannity.}
    \begin{itemize}
    	\item ''When we know what we know about a monster like that \& we still put him on TV every week, we're teaching a generation of young boys \& girls that a man's reputation is more important than the lives of the women he's ruined.''
    \end{itemize}
    {\bf Fuzzy Whiskers.}
    \begin{itemize}
    	\item ``I don't know what to tell you. I'm happy for the 1st time in my life \& I'm not gonna feel bad about it. It takes a long time to realise how truly miserable you are \& even longer to see it doesn't have to be that way. Only after you give up everything can you begin to find a way to be happy.''
    \end{itemize}
    {\bf Sextina Aquafina.}
    \begin{itemize}
    	\item ``I think about my child's heartbeat \& oh, it makes me weep. I hope \& pray to God my little fetus has a soul. Because I want it to feel pain when I eject it from my hole.''
    \end{itemize}
    {\bf Miscellaneous.}
    \begin{itemize}
    	\item ``There has been another mass shooting. I am totally unqualified to cover a news story this important, but as a straight white male, I will plow forward with confidence \& assume I'm doing fine!''
    	\item ``It gets easier. Every day it gets a little easier. But you gotta do it every day -- that's the hard part. But it does get easier''
    	\item ``When you do bad things, you have something you can point to when people eventually leave you. It's not you, you tell yourself, it's that bad thing you did. Do you often keep people at arm's length? Are you afraid of being known \& knowing others?''
    	\item ``He's probably just razzin' ya. But he's a good dog. All bark, no bite. Oh, sorry! That's a labrador expression. I guess in human terms it would be: he's all talk, no shooting you with an assault rifle.''
    	\item ``Do you ever get the feeling that to know you more is to love you less?''
    	\item ``It doesn't matter where you are, it's who you are, that's not gonna change whether you're in California or Maine or New Mexico. You know, you can't escape you.''
    \end{itemize}
    \item {\sc Boku dake ga inai machi $\star$ Erased} (2016)\hfill[S1.E12]
    \item Boku no hîrô akademia $\star$ My Hero Academia (2016--)\hfill[S1.E13][S2.E1--]
    \item {\sc Boushoku no Berserk $\star$ Berserk of Gluttony} (2023--)\hfill[S1.E12]
    \item {\sc Buttobi Itto}\hfill[Chap. 102]
    \item {\sc By\^osoku 5 senchimêtoru $\star$ 5 Centimeters Per Second} (2007)
    \item {\sc Castlevania} (2017--2021)\hfill[S1.E4][S2.E8][S3.E10][S4.E10]
    \item {\sc Chainsaw Man}\hfill[Chap. 183--][S1.E12]
    \item Chi. Chikyuu no Undou ni Tsuite $\star$ Orb: On the Movements of the Earth (2024--)\hfill[S1.E8--]
    \item {\sc Chiyu Mahô no Machigatta Tsukai-kata: Senjô o Kakeru Kaifuku Yôin $\star$ The Wrong Way to Use Healing Magic} (2024--)[S1.E13]
    \item {\sc Clannad} (2007--2008)\hfill[S1.E23]
    \item {\sc Clannad: After Story} (2008--2009)\hfill[S1.E22][OVA3]
    \item {\sc Cyberpunk: Edgerunners} (2022)\hfill[S1.E10]
    \item Dandadan (2024--)\hfill[S1.E9--][Chap. 175--]
    
    Chap. 18: Sad -- kẻ nhào lộn mượt mà.
    \item {\sc Dark Gathering} (2023--)\hfill[S1.E25]
    \item {\sc Darling in the Franxx} (2018)\hfill[S1.E24]
    
    \href{https://darling-in-the-franxx.fandom.com/wiki/List_of_Quotes}{Fandom. {\it A list of quotes from \textsc{Darling} in the \textsc{Franxx}}}
    
    {\sc Hiro $\star$ 016.}
    \begin{itemize}
    	\item ``Our sole reason to exist is to become \textsc{Franxx} \& fight. We were born only for that purpose, raised only for that purpose. But to me, the failure of the group, that's something I can't forgive. With nowhere I can feel like I belong... that's when I met you.'' - Hiro's 1st narrative in Episode 01 preview
    	\item ``It was like I was put under a spell. Her two alluring horns, \& my 1st ever look at a naked girl's body, left me transfixed, unable to take my eyes off of her.'' - Hiro to himself, upon his 1st meeting with Zero Two
    	\item ``I feel myself going deeper inside you!'' - Hiro to Zero Two, piloting Strelizia for the second time 
    	\item ``Getting mocked for my lack of a partner would be one thing. But the fact that there was still a part of me that still hoped for one made me feel pathetic.'' - Hiro to himself after 1st meeting Zero Two
    	\item ``The act of bringing two pairs of lips together, which we'd never heard of, was called a kiss, she said.'' - Hiro remembering what a kiss means from Zero Two
    	\item  ''My wings exist for you. I'm your partner. I'm not going to leave you alone.'' - Hiro promising to be Zero Two's partner
    	\item ``It's only I met you, that I can stand here right now.'' - Hiro to Zero Two
    	\item ``When we 1st met, I couldn't take my eyes off you. You were confident, held your head high, \& I found that beautiful.'' - Hiro as he confesses to Zero Two
    	\item ``You \& I; alone \& lonesome. With the wings that I tore apart with my own hands, it will never be possible again to fly away. The days that I could spend as your wing, the sky that I dreamed we could take to some day. It has all started fading away, far, far away.'' - Hiro about wanting to fly with Zero Two
    	\item ``And I called you a monster! Now we're even!'' - Hiro to Zero Two when they reconcile
    	\item ``I feel the same! Zero Two, I love you too!'' - Hiro declaring his love to Zero Two
    	\item ``If Zero Two can't smile for me, than I might as well be dead.'' - Hiro to Goro after announcing he is going to space 
    	\item ``Liar. If that were true, than why is your last page left blank? Don't fly off on your own. Please, Zero Two, let me stay with you. Together... let's rewrite that story.'' - Hiro to Zero Two before going into the warp gate
    	\item ``The distant skies. Beyond time \& distance. An overwhelmingly long journey just for the two of us. You're a part of me. I'm a part of you. I'll remember your warmth, along with the memories we've made together. I'll never let you go again!'' - Hiro \& Zero Two about their bond
    	\item ``I love you!'' - Hiro confessing to Zero Two
    	\item ``Zero Two... we're becoming one. Now you are me... We're probably going to disappear soon but the path we walked on, others will take it from here... I love you, Zero two'' - Hiro's last words, to Zero Two before his death
    \end{itemize}
    {\sc Zero Two $\star$ 002.}
    \begin{itemize}
    	\item ``The Jian, also known as ``the bird that shares wings,'' only possesses one wing. Unless a male \& female pair lean on each other \& act as one, they're incapable of flight. They're imperfect, incomplete creatures. But, for some reason, their way of life, struck me as profoundly beautiful. It was beautiful, I felt.'' - Zero Two's narrative debut
    	\item ``I'm always alone, too. Thanks to these horns.''
    	\item ``I'm always alone, too. Thanks to these horns.'' - Zero Two talking to Hiro
    	\item ``If you don't belong here, just build a place where you do. If you don't have a partner, find one you. And if you can't, take one by force.'' - Zero Two after Hiro explains his lack of partner
    	\item ``Your taste makes my heart race. It bites \& lingers... the taste of danger.'' - Zero Two after licking Hiro
    	\item ``I think I've taken a liking to you. How would you feel about being my darling?'' - Zero Two to Hiro
    	\item ``Let me get a taste of you. After all... you are now my darling!'' - Zero Two declaring Hiro as her darling
    	\item ``Its been a long time since I last saw a human cry.'' - Zero Two after seeing Hiro's tears
    	\item ``Found you, my darling.'' - Zero Two about Hiro
    	\item ``Once we die, we'll just become a statistic. It won't matter what we were called. Just look at this lifeless city. There are no skies or oceans here. It's a one-way street to nowhere. A dead-end life.'' - Zero Two to Hiro while life looking at the city
    	\item ``Darling, wanna run away with me? I can get you out of here.'' - Zero Two to Hiro as they look at the inner city
    	\item ``You're mine. I love the way you taste. It's true: all the weaklings before died. That's to be expected. But you're special. Believe in me, okay, Darling?'' - Zero Two about Hiro
    	\item ``What is human to you people?'' - Zero Two to Ichigo
    	\item ``A kiss is something you share with your special someone. Is the one you kissed special to you?'' - Zero Two talking about a kiss with Ichigo
    	\item ``Don't worry. We'll always be together until we die.'' - Zero Two to Hiro after the partner shuffle
    	\item ``The weak ones die. Big deal.'' - Zero Two to 090 after his confrontation
    	\item ``If you have anything you wanna say, you better spit it out while you can. Because you're all going to die sooner or later.'' - Zero Two to Ichigo, Miku, Kokoro \& Ikuno during the Boys x Girls fight
    	\item ``They're tiny fragments of memories. It's white, cold \& filled with things unknown, but it's beautiful. In the outside world that I felt for the 1st time, I heard a voice calling to me from somewhere. And before my eyes, a warm hand was reaching out for me.'' - Zero Two about her childhood memories
    	\item ``The leaving something behind part. My body can't do that. It's wonderful. You're all wonderful. You have the ability to decide your futures with your own hearts.'' - Zero Two to Squad 13
    	\item ``The distant skies. Beyond time \& distance. An overwhelmingly long journey just for the two of us. You're a part of me. I'm a part of you. I'll remember your warmth, along with the memories we've made together. I'll never let you go again!'' - Zero Two \& Hiro about their bond
    	\item ``And you are me... It doesn't matter how long it takes, as long as we have souls, I'm sure I will meet you again on Earth \& we'll pick up where we left off... I feel the same. I love you, darling.'' - Zero Two's last words before her death 
    \end{itemize}
    {\sc Ichigo $\star$ 015.}
    \begin{itemize}
    	\item ``Maybe we can't win alone, but the two of us together can!'' - Ichigo to Goro after swimming in the klaxosaur to rescue him
    	\item ``She puts a curse to drain the life out of stamen, the one who's not human. I can't consider her as one of us anymore. Even if she's considered as the key to save the world, even if as a result that I'm hated by the person that I consider the most important.'' - Ichigo about Zero Two devouring her partners
    	\item ``I... I love you, Hiro!'' - Ichigo to Hiro when she confessed
    	\item ``I'm happy for them.'' - Ichigo about Hiro \& Zero Two
    	\item ``But, you know, no one can swim in the same river water twice. We must choose our own path. Until then, our lives have just begun.'' - Ichigo to Goro
    \end{itemize}
    {\sc Goro $\star$ 056.}
    \begin{itemize}
    	\item ``You are such a complete, utter fool!'' - Goro to Hiro about Hiro choosing to keep riding with Zero Two
    	\item ``Presents bring out special feelings. The feeling inspired by receiving something from someone. The feeling you get when you want to give something to someone. And then, there's that feeling when in the end, you couldn't give. I have been wondering for a long time what that emotion was, but it has finally hit me.'' - Goro about his feelings for Ichigo
    	\item ``Things keep changing. Until now, food, shelter, a reason to live, even a place to die, everything was only given to us. But, that's over too. Now, we have woken up from that long dream \& we will never go back to being CHILDREN. That choice, we made it on our own.'' - Goro after the parasites are abandoned
    	\item ``What do you mean, you feel bad? What do you mean, you want us to understand? Have you ever spared a thought for how the people close to you feel? You've always been like this: you run off, make your own decisions, \& never stop to think about us. Papa \& the adults are gone, but forget freedom, we've got our hands tied on every single thing. Have you stopped thinking about how worried we are about this? When someone collapses, all we can do is watch. And yet we're trying our damnest hard to live on! But here you are, running off to die. There are people doing all they can to support you because they want you to live. And you're trampling their feelings as you leave! What the hell do you want us to understand about you?!'' - Goro snapping at Hiro for wanting to go on a death mission
    	\item ``That was never about how everyone felt. It was all about me. I'm the one scared of the world that I chose myself.'' - Goro to Ichigo, about his rant earlier on
    	\item ``This is the path we've chosen in order to live. You don't get to fight it, either.'' - Goro to Hiro, after forgiving him
    \end{itemize}
    {\sc Zorome $\star$ 666.}
    \begin{itemize}
    	\item ``I've never directly spoken with an adult. But they are always looking after us. If we do our best, they praise us \& even give us rewards. If I continue fighting for them, I'm sure that someday I can also become an adult. I've been dreaming forever for this day to come.'' - Zorome about his dream of becoming an adult
    	\item ``Hey! Nobody asked for a calm \& composed analysis!'' - Zorome to Mitsuru
    	\item ``Whoa, Whoa! Did you guys just try to press your bodies together? Are you imitating Hiro \& Zero Two?'' - To Kokoro \& Mitsuru
    	\item ``We won't become ADULTS. We just keep on fighting until we die. That's what we were born \& raised for. But, we're about to decide our own future. We're finally ready to take flight. Now is the time for us, nestlings, to leave the nest.'' - Zorome about the future
    \end{itemize}
    {\sc Miku $\star$ 039.}
    \begin{itemize}
    	\item ``No ordinary person has horns.'' - Miku to Kokoro about Zero Two
    	\item ``Better than the boys we're stuck with'' - To Kokoro about the stamen of Squad 26
    	\item ``The beginning of adolescence, it's just a small realization. You can't understand the others, \& they misunderstand you. It seems to me there's such a huge divide between us. A change of heart like this makes a lot of things that were obvious, not obvious anymore.'' - Miku prior to the Boys vs Girls conflict
    	\item ``Nana... Please help us?!'' - To Nana when the crops die
    	\item ``Do you think we will find that? Something that means the world to us, that we'd choose it over everything else?'' - To Zorome, Futoshi, Mitsuru, \& Ikuno
    \end{itemize}
    {\sc Mitsuru $\star$ 326.}
    \begin{itemize}
    	\item ``I say Hiro's showing real integrity. It must be hard for him to face us after what happened, too. Spare a thought for how he feels.'' - Mitsuru speaking sarcastically about Hiro
    	\item ``If you place your hopes in anything, they will be betrayed. Promises will go unfulfilled \& faith will let you down.'' - Mitsuru to himself after dreaming about his broken childhood promise
    	\item ``In a name, there's power. While we were still young, when all we had was code numbers, you could say we lacked individuality. The one who gave us names was the young Hiro. At that time, you were a guidepost to us. But that's in the past. Right now, I don't expect anything from you.'' - Mitsuru about Hiro
    	\item ``Sure, Hiro was special. Even among the double-digits, he was a cut above. We were all certain that he'd be our leader \& show us the way. But reality had other ideas. Hiro couldn't become a parasite. He's not who he used to be. We must give up \& cut our losses. I don't want to see this pathetic side of Hiro anymore.'' - Mitsuru talking about Hiro to Ichigo
    	\item ``Doesn't it scare you to trust someone so much? You're unbelievable.'' - Mitsuru to Kokoro during their 1st sortie together in Genista
    	\item ``I promise to protect Kokoro from now on.'' - Mitsuru's promise to Futoshi to protect Kokoro
    	\item ``That's what this is... I love Kokoro?'' - Mitsuru realizing his feelings for Kokoro
    	\item ``Don't decide things on your own. I'm here. Lean on me more, Kokoro. I want to make you happy.'' - Mitsuru to Kokoro
    	\item ``If I look up, I see thousands of lives shining deep in the sky. So far away \& although I try, I just can't reach out. Yet, even with broken wings, we'll fly away once again. For the promise we must keep. For the future we must pass on.'' - Mitsuru \& Kokoro about their future
    	\item ``Please don't say you have nothing. If you don't have anything, neither do I! But that's not true. I've found my reason to live... in you. I want to protect you \& that baby's future. I'm a weakling! And I don't really understand what love is, I don't. But if I'm with you, I can keep walking. That's what I've come to believe. Even without our memories, we can start over.'' - Mitsuru reconciling with Kokoro
    	\item ``I read an old book that said how you two feel means you love each other. We want our child, our Ai, to inherit that bond.'' - Mitsuru to Hiro \& Zero Two about Ai
    \end{itemize}
    {\sc Kokoro $\star$ 556.}
    \begin{itemize}
    	\item ``We were raised in an institution \& now we live in the Bird Cage. There are so many things we don't know. Take the sea, for example. How it actually smells, the sound of the washing waves, the fact that sea water tastes salty. All those things that we'd only seen in books \& all those scenes, it was all kept intact there.'' - Kokoro about going to the beach for the 1st time
    	\item ``Do you know the language of flowers?'' - Kokoro to Mitsuru during the Boys x Girls fight
    	\item ``In the past, having a baby was the most natural thing.'' - Kokoro to Mitsuru about why humans stopped having children
    	\item ``Mitsuru, you can rely on others more, you know. I'll believe in you, so believe in me too.'' - Kokoro trying to comfort Mitsuru during their 1st sortie in Genista together
    	\item ``We lived our lives believing our only purpose was to ride the \textsc{Franxx} into battle. But you know what? That might not be true! We could carry new lives \& leave them for the future. When I learned about that, I was very happy.'' - Kokoro to Squad 13
    	\item ``Is it wrong to create a new life? Are we not allowed to think of the future?'' - Kokoro to Mitsuru about having a child
    	\item ``Together, we'll find happiness.'' - Kokoro to Mitsuru on their wedding day
    	\item ``f I look up, I see thousands of lives shining deep in the sky. So far away \& although I try, I just can't reach out. Yet, even with broken wings, we'll fly away once again. For the promise we must keep. For the future we must pass on.'' - Kokoro \& Mitsuru about their future
    	\item ``Why do you call my name even when it hurts you?'' - Kokoro to Mitsuru during the thunderstorm
    	\item ``Look, there's your papa.'' - Kokoro to Ai, when Mitsuru arrives
    	\item ``Thanks to Hiro \& Zero Two, this planet go go back \& start over from square one.'' - Kokoro after Hiro \& Zero Two's sacrifice
    \end{itemize}
    {\sc Futoshi $\star$ 214.}
    \begin{itemize}
    	\item ``If the Plantation has no magma-fuel, the city can't maintain function. That's why there's mining facilities everywhere, to keep digging out the fuel. It's the same with us \& food. When we eat lots, it gives us strength. So, let's eat!'' - Futoshi about magma energy providing necessities 
    	\item ``Will you promise to be my partner forever?!'' - Futoshi to Kokoro
    	\item ``The birdcage is still here. And so are we. The people we were supposed to protect aren't there anymore \& our wings are still being repaired, but we believe that the ten of us will all be able to fly again soon. Until then, we'll make it through on our own.'' - Futoshi after Squad 13 is left to fend for themselves
    \end{itemize}
    {\sc Ikuno $\star$ 196.}
    \begin{itemize}
    	\item ``We're always fighting our fears. We might end up dragging down our partner. We might end up failing to move our \textsc{Franxx} correctly. We might become unnecessary. Will we be able to overcome our fears someday?'' - Ikuno about failing as a parasite
    	\item ``Everything comes to an end. The only difference is whether it comes sooner or later. And even though we sense the end is near, we spend another day, idling in the cradle of our lives.'' - Ikuno about life
    	\item ``A pain. Even if it is, so what?!'' - Ikuno to Alpha after he berated Kokoro
    	\item ``I love you, Ichigo.'' - Ikuno confessing her love to Ichigo
    	\item ``We will leave the birdcage when the sakura blooms. Together, we had our laughs \& we had our clashes. All these memories are as delicate as glass, but they're also the beautiful \& invaluable testimony that we were here.'' - Ikuno before leaving Mistilteinn
    	\item ``It's different for each person'' - To Miku about finding the most important thing in the world to them
    	\item ``I've decided to make most of the time I have left'' - To Hiro \& Zero Two about her accelerated aging process
    \end{itemize}
    {\sc Nana.}
    \begin{itemize}
    	\item ``These children... they're not like the others.'' - Nana describing the squad after the bathroom fight
    	\item ``Don't think you \& your squad are special'' - To Kokoro about her breaking the rules
    	\item ``CHILDREN are usually obedient, docile \& rational. They only live to fulfil their mission: fight the Klaxosaur \& protect the ADULTS. Not these kids. Too many times, they don't know what to do with their emotions \& get carried away. That's definitely not CHILD-like.'' - Nana talking about Squad 13 
    \end{itemize}
    {\sc Hachi.}
    \begin{itemize}
    	\item ``The numbers conclusively prove Code 016's compatibility with Strelizia'' - Upon Hiro suffering no damage from his second sortie
    	\item ``Code 016 is hereby appointed Code 002's partner'' - Declaring Hiro as Zero Two's partner
    	\item ``We call the training facility for Parasites, the Garden. There, CHILDREN are brought up to develop a sound mind \& learn how to behave properly as CHILDREN, they are then pruned \& only the Parasites who leave the Garden's nest with their heads high have the honor of defending ADULTS against the Klaxosaur threat.'' - Hachi prior to taking Squad 13 to the Garden
    	\item ``That feeling... is having it that wrong? The children's sole purpose in life was to fight \& if this is truly their final battle, \& in their lives after, Don't you think they'll need someone like you.'' - Hachi to Nana about the children's future
    \end{itemize}
    {\sc Papa.}
    \begin{itemize}
    	\item ``My children, you have had the good fortune to be chosen as parasites. Your great predecessors defended out cities \& our people, \& they splendidly took flight as our representatives. Turn your life into a blaze of glory \& shed every last drop of blood you have. I pray that you will become a shining ray of hope for humanity.'' - Papa to Squad 13 During the graduation ceremony
    	\item ``VIRM will not die. I'm certain we will face each other again in the apex of evolution. As long as this universe contains a flicker of life.'' - Papa to Hiro \& Zero Two after the VIRM planet is destroyed
    \end{itemize}
    {\sc 9'a.}
    \begin{itemize}
    	\item ``Humans have evolved \& cast their reproductive functions aside in the process. Reject that, \& we'll all have to go back to conforming to one gender'' - Alpha to Kokoro about her `leaving a mark'
    	\item ``The gender is a pain, an annoyance that's only tolerated to operate the \textsc{Franxx}? And that's all it really is.'' - Alpha criticizing Kokoro
    	\item ``Humans have also cast away their tendency to be ruled by emotions like that because it serves zero purpose in life.'' - Alpha to Ikuno after she slaps him
    	\item ``Meeting you guys helped me a little about being human, but my home is the battlefield \& nowhere else. Tell Iota hi for me. It's been kinda fun, I guess.'' - Alpha's last words before his death
    \end{itemize}
    {\sc Dr. \textsc{Franxx}.}
    \begin{itemize}
    	\item ``Fate is cruel. In order to obtain something, one must lose something else. Humanity has searched \& sacrificed a lot for what lies beyond life which is our limit. And then we met her, such a beautiful \& perfect being.'' - Dr.\textsc{Franxx} about humanity \& 001
    	\item ``You two become the children's new adults'' - Dr. \textsc{Franxx}'s final wish upon his death
    \end{itemize}
    {\sc New Nana.}
    \begin{itemize}
    	\item ``It is considered impossible for a pregnant woman to pilot a \textsc{Franxx}. Past records indicate abortion, that is the removal of the fetus as one of the possible treatments. Please let me know what you decide to do.'' - To Kokoro upon informing her of her pregnancy
    \end{itemize}
    {\sc Naomi.}
    \begin{itemize}
    	\item ``I hope you find a good partner, Hiro'' - To Hiro when she leaves
    \end{itemize}
    {\sc Klaxosaur Princess.}
    \begin{itemize}
    	\item ``I smell my brethren on you. It seems your sins are far too grave to be punished with mere death.''
    	\item ``Ah, the fake that the humans created. A duplicate that does not realize it is being used by the invaders.''
    	\item ``I shall devour your body \& soul.''
    	\item ``Damned human wannabes.'' - 001 about VIRM after killing Tarsier
    	\item ``We are the defenders of this planet. Once upon a time, during a long battle, we turned ourselves into an immortal weapon. Except fighting, all was lost, so we went to sleep at the bottom of death, our bond was to prepare for the returning invader.'' - 001 about the war with VIRM
    	\item ``Perhaps some lives only shine when in unison with others.'' - 001 about Hiro \& Zero Two's bond
    	\item ``Is this what living is for you? We believed that by abandoning our ties \& embracing solitude, we could perfect ourselves, make ourselves stronger. Very well. I shall give you every ounce of strength that remains within me. Whether or not you can take over the controls will be up to you. I stake this planet's future on you two.''
    	\item ``Decide whether you want to fight or accept your ruin.'' - 001's last words before she sacrifices herself \& entrusts the world's fate to Hiro \& Zero Two
    \end{itemize}
    {\sc Ai.}
    \begin{itemize}
    	\item ``Papa, Darling, darling!'' - Ai to Mitsuru
    	\item ``Mama?'' - Ai to Kokoro
    \end{itemize}    
    \item Dead Mount Death Play (2023--)\hfill[S1.E7--]
    \item {\sc Death Note: Desu n\^oto $\star$ Death Note} (2006--2007)\hfill[S1.E37]
    \item {\sc Death Parade} (2015)\hfill[S1.E12]
    \item {\sc Dekiru Neko wa Kyou mo Yuuutsu $\star$ The Masterful Cat Is Depressed Again Today} (2023)\hfill[S1.E13]
    \item {\sc Devilman: Crybaby} (2018)\hfill[S1.E10]
    \item {\sc Dorohedoro} (2020)\hfill[S1.E12][OVA]
    \item {\sc Dororo} (2019)\hfill[S1.E24]
    \item Dr. Stone (2019--)\hfill[S1.E5--]
    \item {\sc Dungeon Meshi $\star$ Delicious in Dungeon} (2024--)\hfill[S1.E24]
    \item {\sc Erufen r\^{\i}to $\star$ Elfen Lied} (2004)\hfill[S1.E14]
    \item Fate{\tt/}stay night (2006)\hfill[S1.E5--]
    \item {\sc Food Wars!: Shokugeki no Soma $\star$ Food Wars} (2015--2020)\hfill[S1.E24][S2.E13][OVA4][S3.E12][S4.E12][S5.E13]
    \item Fumetsu no Anata e $\star$ To Your Eternity (2021--)\hfill[S1.E8][Chap. 144--]
    \item {\sc Gantz}\hfill[Chap. 383]
    \item {\sc Gekijouban Jujutsu Kaisen 0 $\star$ Jujutsu Kaisen 0: The Movie} (2021)
    \item {\sc Gimai Seikatsu $\star$ Days with My Stepsister} (2024--)\hfill[S1.E12]
    \item Gintama (2005--2021)\hfill[S1.E2--]
    \item Girls Band Cry (2024--)\hfill[S1.E2--]
    \item {\sc Goblin Slayer} (2018--)\hfill[S1.E14][S2.E12][Chap. 83--]
    \item {\sc Goblin Slayer: Goblin's Crown} (2020)
    \item {\sc Golden Boy: Sasurai no o-benky\^o yar\^o $\star$ Golden Boy} (1995--1996)\hfill[S1.E6][Chap. 1{\tt/}94]
    \item Grand Blue (2018--)\hfill[S1.E12][Chap. 2--]
    \item GTO $\star$ GTO: Great Teacher Onizuka (1999--2000)\hfill[S1.E28--]
    \item {\sc Hagane no renkinjutsushi $\star$ Fullmetal Alchemist: Brotherhood} (2009--2012)\hfill[S1.E64][Chap. 1{\tt/}108.2]
    \item Haigakura (2024--)\hfill[S1.E1--]
    \item {\sc Haikyuu!!} (2014--2020)\hfill[S1.E26][S2.E26][S3.E11][S4.E25]
    \item {\sc Hajime no ippo $\star$ Fighting Spirit}\hfill[S1.E75]
    \item {\sc Hametsu no Oukoku $\star$ The Kingdoms of Ruin} (2023--)\hfill[S1.E12]
    \item {\sc Hanamonogatari} (2014)\hfill[S1.E5]
    \item {\sc Handyman Saitou in Another World} (2023--)\hfill[S1.E12]
    \item {\sc Helck} (2023--)\hfill[S1.E24]
    \item {\sc Hige Wo Soru. Soshite Joshikosei Wo Hirou} (2021--)\hfill[S1.E13][Chap. 29--]
    \item {\sc Hikikomari Kyuuketsuki no Monmon $\star$ The Vexations of a Shut-In Vampire Princess} (2023--)\hfill[S1.E12]
    \item {\sc Homunculus}\hfill[Chap. 158]
    \item {\sc Horimiya} (2021--)\hfill[S1.E13]
    \item {\sc Horimiya: Piece $\star$ Horimiya: The Missing Pieces} (2023)\hfill[S1.E13]
    \item {\sc Hoshi o ou kodomo $\star$ Children Who Chase Lost Voices} (2011)
    \item {\sc Hunter $\times$ Hunter} (2011--2014)\hfill[S1.E148]
    \item {\sc Inuyashiki} (2017)\hfill[S1.E11]
    \item Invincible (2021--)\hfill[S1.E1--]
    \item Isekai Ojisan $\star$ Uncle from Another World (2022--)\hfill[S1.E13]
    \item Isekai Shikkaku $\star$ No Longer Allowed in Another World (2024--)\hfill[S1.E1--]
    \item {\sc Isle of Dogs} (2018)
    \item Jigoku shôjo $\star$ Hell Girl (2005--2006)\hfill[S1.E3--]
    \item Jigokuraku (2023--)\hfill[S1.E13]
    \item {\sc Jiisan Baasan Wakagaeru $\star$ Grandpa \& Grandma Turn Young Again} (2024--)\hfill[S1.E11]
    \item {\sc Jujutsu Kaisen} (2020--)\hfill[S1.E24][S2.E23][SP01][Chap. 271]
    \item Kaguya-sama wa kokurasetai $\sim$Tensai tachi no ren'ai zun\^o sen$\sim$ $\star$ Kaguya-sama: Love Is War (2019--)\hfill[S1.E12][S2.E12][S3.E13]
    \item {\sc Kaguya-sama wa Kokurasetai: First Kiss wa Owaranai $\star$ Kaguya-sama: Love Is War - The First Kiss That Never Ends} (2022)\hfill[E4]
    \item Kaiju No. 8 (2024--)\hfill[S1.E12][Chap. 111--]
    \item Kaiko sareta Ankoku Heishi (30-dai) no Slow na Second Life (2023--)\hfill[S1.E12]
    \item {\sc Kaminaki Sekai no Kamisama Katsudou $\star$ KamiKatsu: Working for God in a Godless World} (2023--)\hfill[S1.E12]
    \item {\sc Kamonohashi Ron no Kindan Suiri $\star$ Ron Kamonohashi's Forbidden Deductions} (2023--)\hfill[S1.E13][S2.E9--]
    \item {\sc Kattobi Itto}\hfill[Chap. 102]
    \item {\sc Kaub\^oi bibappu $\star$ Cowboy Bebop} (1998--1999)\hfill[S1.E26]
    \item Kawaisugi Crisis (2023--)\hfill[S1.E3--]
    \item {\sc Kengan Ashura} (2019--2024)\hfill[S1.E12][S2.E28]
    \item {\sc Kenp\^u Denki Berserk $\star$ Berserk} (1997--1998)\hfill[S1.E25][Chap. 373--]
    
    See \href{https://en.wikiquote.org/wiki/Berserk_(anime)}{Wikiquote{\tt/}Berserk (anime)}.
    
    {\bf Narrator.}
   	\begin{itemize}
   		\item ``In this world, is the destiny of mankind controlled by some transcendental entity or law? Is it like the hand of god hovering above? At least it is true that man has no control; even over his own will.''
   		\item ``It was much too big to be called a sword. Massive, thick, heavy, \& far too rough. Indeed, it was like a heap of raw iron.''
   		\item ``Dreams, ambition, love, hope; in this world, could the glories of a youthful heart be $\ldots$ forbidden?''
   		\item ``Dreams. Each man longs to pursue his dream. Each man is tortured by this dream, but the dream gives meaning to his life. Even if the dream ruins his life, man cannot allow himself to leave it behind. In this world, is man ever able to possess anything more solid, than a dream?''
   		\item ``In this world, there is a time that one is forever unable to retrieve. In pursuit for forgiveness one is destined to atone by living through agony \& letting time slip away. The sorrow in the furthest reaches of one's memory$\ldots$ The sorrow just beyond one's memory$\ldots$ Which is more heartbreaking?''
   		\item ``Providence may guide a man to meet 1 specific person, even if such guidance eventually leads him to darkness. Man simply cannot forsake the beauty of his own chosen path. When will man learn a way to control his soul?''
   		\item ``A man draws his sword in order to protect the small wound deep in his heart, it was inflicted in the days long past at the farthest reach of his memories, a man wields his sword in order to depart life with a smile$\ldots$''
   	\end{itemize}
    {\bf Guts.}
    \begin{itemize}
    	\item ``Even if we painstakingly piece together something lost, it doesn't mean things will ever go back to how they were.''
    	\item ``You're going to be all right. You just stumbled over a stone in the road. It means nothing. Your goal lies far beyond this. Doesn't it? I'm sure you'll overcome this. You'll walk again$\ldots$ soon.''
    	\item ``One who does something he hates just because he's told to.. is called an errand boy.''
    	\item ``You're right, we are mortal \& fragile. But even if we are tortured or wounded, we'll fight to survive. You should feel the pain we feel \& understand. I am the messenger that will deliver you to that pain \& understanding.''
    	\item ``When you meet your God tell him to leave me alone.''
    	\item ``Throughout my life, the moments, \& people who have defined me$\ldots$ they have all been illuminated by sparks.''
    	\item ``If you're alone$\ldots$ if it's just your life, you can use it however you please. Wear yourself out, get cut to ribbons, doesn't matter. But when there's two, the blade grows heavy. Fighting like death doesn't concern you becomes a thing of the past. It's no longer just you. I threw away my way of life, relied on the strength of others, \& somehow pushed on.''
    	\item ``In the end the winner is still the last man standing.''
    	\item ``I don't want what another man can give me. If he grants me anything, then it's his to give \& not my own.''
    	\item ``Look, look around carefully. Strain your eyes. At the darkness around us$\ldots$ At the darkness$\ldots$ around me. You said, ``anywhere but here''. This is where. Here at the border. Gathered by the winds. Those who've met their final destiny hanging between jealousy \& regret. Those who failed, swept together here. You say ``It doesn't matter where''. If you follow me, this is where you'll go! This is your Eden! You run from it. It is no Eden. If you follow me$\ldots$ to this place$\ldots$ the entire world$\ldots$ is a battlefield.''
    	\item ``FUCK YOU. I'm human, the real deal, right down to the fuckin' marrow of my bones. Don't lump me together with you faggot-ass monsters.''
    	\item ``The reward for ambition too great$\ldots$ is self destruction.''
    	\item ``People bring the small flames of their wishes together$\ldots$  since they don't want to extinguish the small flame$\ldots$  they'll bring that small flame to a bigger fire. A big flame named Griffith. But you know$\ldots$  I didn't bring a flame with me. I think I just stopped by to warm myself by the bonfire.''
    	\item ``My place really was here. I was too foolish \& stubborn to notice. But, what I truly hoped for then was here. Why do I always realize it$\ldots$ when I've already lost it.''
    	\item ``If you're always worried about crushing the ants beneath you$\ldots$ you won't be able to walk.''
    	\item ``I'd rather fight for my life than live it.''
    	\item ``People who perish in other's battles are worms$\ldots$ If one can't live their life the way they want, they might as well die.''
    	\item ``You have the strongest armor, because you are the weakest!''
    	\item ``In this world is the destiny of mankind controlled by some transcendental entity or law? Is it like the hand of god hovering from above? Perhaps men have no control even over their own will.''
    	\item ``Dreams, ambition, love, hope; in this world, could the glories of a youthful heart be.. forbidden?''
    	\item ``DO NOT PRAY! If you pray, your hands will close together. You will not be able to fight!''
    	\item ``That thing was too big to be called a sword. Too big, too thick, too heavy, \& too rough, it was more like a large hunk of iron.'' 
    	\item ``I will never draw my sword for another man again, or be dangled by another mans dream. From now on, I will fight my own battles.''
    	\item ``Look around you carefully. Strain your eyes at the darkness around us$\ldots$ At the darkness around me. You said anywhere but here. This is where, here, at the border. Gathered by the winds. Those who have met their final destiny hanging between jealousy \& regret. Those who failed, swept together here. You say it doesn't matter where. If you follow me, this is where you'll go. This is your eden.''
    	\item ``If you're always worried about crushing the ants beneath you$\ldots$ you won't be able to walk.''
    	\item ``Do whatever you want now. But if you disturb me, I'll kill you.''
    	\item ``My place really was here. I was too foolish \& stubborn to notice. But, what I truly hoped for then was here$\ldots$ Why do I always realize it$\ldots$ When I've already lost it$\ldots$''
    	\item ``You have the strongest armor, because you are the weakest!''
    	\item ``When you meet your God, tell him to leave me alone.''
    	\item ``God shows himself on the ground. These are his miracles.''
    	\item ``I don't have time for miracles. They make me puke.''
    	\item ``He appeared right in front of me, \& he wasn't a demon$\ldots$ but what looked like a human. As if he'd been yanked from before into the present unchanged. I gazed at him \& for a second$\ldots$ {\bf I forgot to kill him}.''
    	\item ``I'd rather fight for my life than live it.''
    	\item ``Couldn't you settle for a smile, \& a fond farewell?''
    	\item ``I don't want what another man can give me. If he grants me anything, then it's his to give \& not my own.''
    	\item ``In the end the winner is still the last man standing.''
    	\item ``I can't compare to you$\ldots$ Sure I can kill a hundred enemies but$\ldots$ \& not only you, but Griffith too. You both hold on to something, gambling with your very lives. It's amazing to me, I can't compare. I just kill for the sake of killing. Everyday, hundreds, thousands$\ldots$ There's nothing$\ldots$ honorable$\ldots$ about it.''
    	\item ``Looking from up here, it's as if each flame were a small dream, for each person. They look like a bonfire of dreams, don't they?$\ldots$ - But, there's not flame for me here. I'm just$\ldots$ a temporary visitor, taking comfort from the flame.''
    	\item ``Monsters shouldn't call themselves warriors, you self-centered bastard.''
    	\item ``Humans are weak$\ldots$ but we want to live$\ldots$ even if we're wounded$\ldots$ or tortured$\ldots$ we feel the pain$\ldots$''
    	\item ``A beast, a monster? Heh$\ldots$ Don't make me laugh. I$\ldots$ am me. \& nothing on earth. Can change me. No matter who chases me.''
    	\item ``So I guess you're like a mad dog that bit his master's hand. Not bad. I can't say that I don't like stuff like that. But you know$\ldots$ I hate to say this$\ldots$ But I don't have the time for that shit right now, like I would care about pissing contests between monsters. Go find someone else.''
    	\item ``Revenge$\ldots$ War$\ldots$ Maybe that's enough reason. I'm all alone, I think. For now. \& within me is the black berserker rage$\ldots$ \& only that$\ldots$ Will keep me on my feet. As I head toward you$\ldots$ Pressing on.''
    	\item ``Why do you shrink back from these sights? If you were in a place of worship, you'd call them angels. Talk to ``God.''''
    	\item ``People who perish in other's battles are worms$\ldots$ If one can't live their life the way they want, they might as well die.''
    	\item ``The reward for ambition too great$\ldots$ is self destruction.''
    	\item ``People bring the small flames of their wishes together$\ldots$ since they don't want to extinguish the small flame$\ldots$ they'll bring that small flame to a bigger fire. A big flame named Griffith. But you know$\ldots$ I didn't bring a flame with me. I think I just stopped by to warm myself by the bonfire.''
    	\item ``Throughout my life, the moments, \& people who have defined me$\ldots$ they have all been illuminated by sparks.''
    	\item ``Where am I going? If I just collapsed back there I would be better off. If I were dead back there, there would be nothing. Now there's only bad.''
    \end{itemize}
	{\bf Griffith.}
	\begin{itemize}
		\item ``While many can pursue their dreams in solitude, other dreams are like great storms blowing hundreds, even thousands of dreams apart in their wake. Dreams breathe life into men \& can cage them in suffering. Men live \& die by their dreams. But long after they have been abandoned they still smolder deep in men's hearts. Some see nothing more than life \& death. They are dead, for they have no dreams.''
		\item ``It is my perception, that a true friend never relies on another's dream. A person with the potential to be my true friend, must be able to find his reason for life without my help. And, he would have to put his heart \& soul into protecting his dream. He would never hesitate to fight for his dream, even against me. For me, a true friend is one who stands equal on those terms$\ldots$''
		\item ``Is he living his dreams in death? Or$\ldots$ is death the end of dreams?''
		\item ``A friend would not just follow another's dream… a friend would find his own reason to live.''
		\item ``Death on the battlefield comes regardless of class, royal or common. The loser must die!''
		\item ``Among thousands of comrades \& ten thousand enemies, only you$\ldots$ only you made me forget my dream. I see it!''
		\item ``A dream$\ldots$ It's something you do for yourself, not for others.''
		\item ``I feel no responsibility to comrades who've lost lives under my command. Because they chose to fight in each battle.. Just as I chose this. But if there is something that… I can do for them. Something I can do for the dead… Then it is to win! I must keep winning to attain my dream. The same one they clung to, \& risked their lives for!! To realize my dream, I will perch on top of their corpses.. It is a blood-smeared dream, after all. I don't regret or feel guilty about it.. But to risk thousands of lives while never getting myself dirty. It's not a dream that can be so easily realized!''
		\item ``A friend would not just follow another's dream$\ldots$ a friend would find his own reason to live$\ldots$''
		\item ``A dream$\ldots$ It's something you do for yourself, not for others.''
		\item ``It is my perception that a true friend never relies on another's dream. A person with the potential to be my true friend must be able to find his reason for life without my help. \&, he would have to put his heart \& soul into protecting his dream. He would never hesitate to fight for his dream, even against me. For me, a true friend is one who stands equal on those terms.''
		\item ``A dream can make a man feel alive, or it can kill him instead. But to simply exist$\ldots$ just because one's been born is the sort of notion that I hate$\ldots$ I can't stand it.''
	\end{itemize}
	{\bf Judeau.}
	\begin{itemize}
		\item ``Whether it's good or bad, it's so unfortunate to wake up during a dream.''
		\item ``I realized that I could never be the best, so I decided to find the man that could \& serve by his side.''
		\item ``I'm a jack of all trades$\ldots$ a little better than most at everything, but I don't shine at anything.''
		\item ``It's okay to cry$\ldots$''
		\item ``You cry a lot when you're alone, don't you Casca?''
	\end{itemize}
	{\bf Skull Knight.}
	\begin{itemize}
		\item ``God gave them this destiny. This encounter.''
		\item ``What you want$\ldots$ may not be what she wants.''
		\item ``The world is as moonlight reflected on the water's surface. The moons light will not be extinguished. So long as the moon exists in the sky, moonlight will remain on the water$\ldots$ $\ldots$ \& this is a thing that already was. What will follow now is a shadow$\ldots$ No more than a shadow cast high above the earth$\ldots$ By light from a distant dying sun. We already subsist$\ldots$ Within the current of causality. We who exist beyond the physical are still merely shadows on the water. Maybe you aren't a shadow on the water$\ldots$ But instead, a fish that breaches water's surface.''
	\end{itemize}
	{\bf Schierke.}
	\begin{itemize}
		\item ``No matter how strong, for a human to fight a monster means he has submerged his humanity \& transformed himself into a greater monster.''
	\end{itemize}
	{\bf Godo.}
	\begin{itemize}
		\item ``Hate is a place, where a man who can't stand sadness goes.''
		\item ``If you desire one thing for so long, it's a given that you'll miss other things along the way. That's how it is$\ldots$ that's life.''
	\end{itemize}
	{\bf Judeau.}
	\begin{itemize}
		\item ``Whether it's good or bad, it's so unfortunate to wake up during a dream.''
	\end{itemize}
	{\bf Void.}
	\begin{itemize}
		\item ``Dreams. Each man longs to pursue his dream. Each man is tortured by this dream, but the dream gives meaning to his life. Even if the dream ruins his life, man cannot allow himself to leave it behind. In this world, is man ever able to possess anything more solid, than a dream?''
		\item ``If fate is a principle beyond Human comprehension which capriciously torments man, then it is karma that man confront fate by embracing sorcery.''
	\end{itemize}
	{\bf Kentaro Miura.}
	\begin{itemize}
		\item ``Living for the future is more important than trying to avenge the past.'' - Kentaro Miura, Berserk, Vol. 2
		\item ``Don't forget$\ldots$ when you gaze into the darkness$\ldots$ the darkness gazes back into you.'' - Kentaro Miura, Berserk, Vol. 26
		\item ``Dreams. Win or lose$\ldots$ I'm sure you could spend your whole life chasing one.'' - Kentaro Miura, Berserk, Vol. 7
		\item ``From where I stand$\ldots$ you're the same as that idol you worship. Completely hollow.'' - Kentaro Miura, Berserk, Vol. 16
		\item ``You went alone. You were right beside those irreplacable things$\ldots$ yet you couldn't bear to immerse yourself together in sorrow with them. So instead$\ldots$ you ran away so that your own malice could burn within you.'' - Kentaro Miura, Berserk, Vol. 17
		\item ``Things you have now, things you've lost. People who're near by, people who've gone far away. No matter what you choose, truth is, both regret \& reluctance are going to follow you around. You just have to make sure you don't make excuses to yourself down the road.'' - Kentaro Miura, Berserk, Vol. 38
		\item ``Even these thoughts will slip my mind in time. And then$\ldots$ only$\ldots$ the beat of my heart still remains.'' - Kentaro Miura, Berserk, Vol. 7
		\item ``Beneath an unsinking black sun$\ldots$ through the boundless gloom$\ldots$ our journey continues.''
	\end{itemize}
	\item Kikansha no Mahou wa Tokubetsu desu $\star$ {\sc A Returner's Magic Should Be Special} (2023--)\hfill[S1.E12]
    \item Kimetsu No Yaiba $\star$ Demon Slayer\hfill[S1.E26][S2.E7][S3.E11][S4.E11][S4.E8][Chap. 205]
    
    {\sf Tanjiro Kamado.}
    
    {\sf Inosuke Hashibira.}
    
    {\sf Zenitsu Agatsuma.}
    
    {\sf Nezuko Kamado.}
    
    {\sf Akaza.}
    \begin{itemize}
    	\item ``{\it I hate weak people. Weak people$\ldots$ never fight face to face. And poison well. Despicable. Weak people$\ldots$ They lack patience. They will quickly reap what they sow. I killed people with these `protecting fists'. I blooded my master's previous Soryuu style. I couldn't respect my father's last words. That's right. That's who I wanted to kill.}'' - Akaza, before committing suicide, Chap. 155.
    \end{itemize}
    \item {\sc Kimetsu no Yaiba: Mugen Ressha-Hen $\star$ Demon Slayer: Mugen Train} (2020)
    \item {\sc Kimi no Koto ga Daidaidaidaidaisuki na 100-nin no Kanojo $\star$ The 100 Girlfriends Who Really, Really, Really, Really, REALLY Love You} (2023--)\hfill[S1.E12]
    \item {\sc Kimi no na wa $\star$ Your Name} (2016)
    \item {\sc Kimi no suiz\^o o tabetai $\star$ I Want to Eat Your Pancreas} (2018)
    \item {\sc Kiseij\^u: Sei no kakuritsu $\star$ Parasyte: The Maxim} (2015--2015)\hfill[S1.E24]
    \item {\sc Kizumonogatari I: Tekketsu-hen $\star$ Kizumonogatari Part 1: Tekketsu} (2016)
    \item {\sc Kizumonogatari II: Nekketsu-hen $\star$ Kizumonogatari Part 2: Nekketsu} (2016)
    \item {\sc Kizumonogatari III: Reiketsu-hen $\star$ Kizumonogatari Part 3: Reiketsu} (2017)
    \item {\sc Kôdo giasu - Hangyaku no rurûshu: Code Geass - Lelouch of the Rebellion $\star$ Code Geass} (2006--2008)\hfill[S1.E25][S2.E25]
    \item {\sc Koe no katachi $\star$ A Silent Voice: The Movie} (2016)
    \item Kôkaku kidôtai: Stand Alone Complex $\star$ Ghost in the Shell: Stand Alone Complex (2002--2005)\hfill[S1.E5--]
    \item Komi-san wa, Komyushou Desu $\star$ Komi Can't Communicate (2021--)\hfill[S1.E1--]
    \item {\sc Konyaku Haki sareta Reijou wo Hirotta Ore ga, Ikenai Koto wo Oshiekomu $\star$ I'm Giving the Disgraced Noble Lady I Rescued a Crash Course in Naughtiness} (2023--)\hfill[S1.E12]
    \item Koroshi Ai $\star$ Love of Kill (2022--)\hfill[S1.E3][Chap. 31--]
    \item {\sc Koto no ha no niwa $\star$ The Garden of Words} (2013)
    \item {\sc K\^okaku Kid\^otai $\star$ Ghost in the Shell} (1995)
    \item {\sc Koyomimonogatari} (2016)\hfill[S1.E12]
    \item {\sc Kureimoa $\star$ Claymore} (2007)\hfill[S1.E26]
    \item {\sc Kusuriya no Hitorigoto $\star$ The Apothecary Diaries} (2023--)\hfill[S1.E24]
    \item Link Click (2021--)\hfill[S1.E12][S2.E12]
    \item Love, Death \& Robots (2019--)\hfill[S1.E18][S2.E8][S3.E9]
    \item Lycoris Recoil (2022--)\hfill[S1.E13]
    \item Made in Abyss (2017--)\hfill[S1.E13][S2.E12]
    \item {\sc Majo to Yajû $\star$ The Witch \& the Beast} (2024--)\hfill[S1.E12]
    \item {\sc Make Hiroin ga Ôsugiru! $\star$ Makeine: Too Many Losing Heroines!} (2024--)\hfill[S1.E12]
    \item {\sc Mashle: Magic \& Muscles} (2023--)\hfill[S1.E12][S1.E12]
    \item {\sc Mato Seihei no Slave $\star$ Chained Soldier} (2024--)\hfill[S1.E12]
    \item {\sc Me!Me!Me!} (2014)
    \item Meiji Gekken: 1874 (2024--)\hfill[S1.E9--]
    \item Mieruko-chan (2021--)\hfill[S1.E12][Chap. 40--]
    \item Migi to Dali $\star$ Migi \& Dali (2023--)\hfill[S1.E11--]
    \item Mob Psycho 100 (2016--2019)\hfill[S1.E12][S2.E13][S3.E1--]
    \item {\sc Monogatari Series: Second Season} (2013)\hfill[S1.E23]
    
    \href{https://www.cbr.com/monogatari-series-best-quotes/}{CBR{\tt/}Monogatari: 10 Most Iconic Quotes From The Entire Franchise}: ``The Monogatari Series is one of the most dialogue heavy anime out there. Fans are attracted to the series for its narration, character interaction \& character development over anything else. While it has some amazing animation, character design \& music, what really sells it is the dialogue. The series is filled with amazing quotes, \& all sorts of different character types to attribute them to. The series deals heavily in character relations, \& how different personalities are perceived by those around them as well as the people themselves. With so many great lines to go through, some might get left out of the conversation. Regardless, these are the most iconic quotes from the massive anime franchise.''
    
    {\sf Uncategorized.}
    \begin{itemize}
    	\item ``If you consider yourself unlucky, blame it on your habitual actions.\\
    	\item ``Unfortunately, I don't know any apparition stories that surpass what I experienced.''
    \end{itemize}
    
    {\sf Koyomi Araragi.} Koyomi Araragi serves as the series narrator, \& as a result, a lot of what happens throughout the Monogatari Series narrative is seen from his perspective. Koyomi has had some great lines over the years, but this is one of his best. A lot of the Monogatari series revolves around themes of identity \& change. There are all sorts of character type that are introduced throughout the series, all of whom have their own sort of youthful naiveté. Koyomi's quote about optimism speaks to his understanding \& helpful nature.
    \begin{itemize}
    	\item ``But being optimistic isn't a bad thing, is it? It's not like you're doing anything bad. It's also not like you're cheating either.''
    	\item ``We, who wounded each other, now lick each other's wounds.
    	
    	We, who were wounded, now need each other to heal.
    	
    	If tomorrow you die, tomorrow my life will end.
    	
    	If today you live, then today i too will live on.
    	
    	Thus, a tale of the wounded begins.
    	
    	A tale of blood.
    	
    	Red when wet, black when dry.
    	
    	A tale of our precious wounds that will never disappear.
    	
    	I will not tell it to anyone.''
    	\item ``Even if mistaken, even if cruel, even if stupid, if many people acknowledge it, I found out that it can become the right thing. I found out that righteousness was able to be infinitely mass-produced. I found out that righteousness was established by the number of people. I found out that maneuvering for a majority was everything.''
    	\item ``It may have been impossible, it may have been unreasonable, but it wasn't useless.''
    \end{itemize}
    {\sf Mother Araragi.}
    \begin{itemize}
    	\item ``People can run from the things they don't like all they want. But if they're just averting their eyes, they're not running. As long as you think the current situation is okay, no one can help.''
    \end{itemize}
    {\sf Kaiki Deishu.}
    \begin{itemize}
    	\item ``Have suspicions, not faith.''
    	\item ``No character looks the same from all angles.''
    	\item ``The fake is of far greater value. In its deliberate attempt to be real, it's more real than the real thing.''
    	\item ``A woman I know, a woman I know very well always treats her current romance like it's her 1st. She always looks like she's never fallen in love before. \& that's the way to go. That's how it should be. There is no peerless person, there is nothing that cannot be replaced. Because humans, as humans can redo anything however many times they want.''
    \end{itemize}
    {\sf Mayoi Hachikuji.}
    \begin{itemize}
    	\item ``I know love. The convenience store was selling it. For 298 yen.''
    \end{itemize}
    {\sf Tsubasa Hanekawa.}
    \begin{itemize}
    	\item ``I don't know everything. I just know what I know.''
    \end{itemize}
    {\sf Senjougahara Hitagi.} Senjougahara isn't the type of person to let all of her emotions show when she says or does something. That's in good part thanks to her traumatizing past experiences, \& though it may seem a little harsh, this quote speaks volumes about Senjougahara \& her own experiences above anything else. Senjougahara is often shown to be a no-nonsense type of character. She's empathetic of others, but cuts right to the heart of the issue when it comes to the problems the characters in the series have faced -- including herself.
    \begin{itemize}
    	\item ``Those who get fooled are partially at fault.''
    	\item ``I am not a friend of justice. I am an enemy of evil.''
    	\item ``It's like I'd want to ask what justice is doing right now if it existed in this world. Well, justice is empty \& ineffectual.''
    	\item ``The last thing I can offer you is this starry sky.''
    	\item ``If it's because of this misfortune that caught your eye, then I'm glad that it happened.''
    \end{itemize}
    {\sf Suruga Kanbaru.} Like most of the other characters in the series, Kanbaru goes through quite a bit of soul searching before she finally comes to terms with herself. Though outwardly friendly \& easygoing, Kanbaru has her own problems that she has to deal with. This is a pretty loaded line from a character who often gets straight to the point. It would make sense for Kanbaru to feel somewhat alone given her situation, but she luckily overcomes that with the help of Araragi \& those around her as the series progresses.
    \begin{itemize}
    	\item ``I start chasing all the rabbits I see \& end up catching none of them.''
    	\item ``I'm a bit too empty-headed for thinking, a little too dull for feeling. There's only one thing I'm much good at, \& that's running. When I run, I can leave everything else behind. They say the legs are like a second brain. I imagine that comes from people often having a flash of insight while they're out for a stroll, but that only applies to walking. While they're running, humans don't do any thinking at all. We may not be able to walk without looking back -- but we can run without looking back. Our minds, our worries. We leave it all on the starting line. That said I do usually have my course planned out beforehand when I go for my early morning jog, but that night I left even that up to chance. Whenever I came to a corner, I turned it. Traversing roads in my own town that I'd never been down gave me just the slightest feeling of freshness, but I left that feeling behind too. It felt good. It felt good to run with every ounce of strength I had. Come to think of it, isn't running really the only chance we have to use every ounce of our strength? Most of the time, people have a limiter in place. Whatever they're doing, frankly they're not giving it everything they've got because if they don't regulate their strength, they'll end up breaking something. Themselves or their surroundings -- something gets broken. So they look at their watches, keep tabs on how many lives they have left before game over, \& try to avoid leaning too far towards industry or sloth. To avoid using their full strength. In that sense, I guess people regulate themselves while they're running as well -- not a person alive can complete a marathon at the speed they would run a sprint. It's always important to pace yourself, no matter what you're doing. But that night, I even left all thoughts of pacing myself behind -- \& ran with every ounce of strength I had. Push it too far \& your pace drops. But even then, give it everything you've got. Run to the breaking point. Run until you run out. It was an ugly run, without proper form or anything. My gait \& breathing were all over the place. The appropriate expression to describe it was probably less ``mad dash'' than ``running blind'' -- or more likely ``running around like a chicken with its head cut off.'' But I ran like that until dawn, all night long. I ran for over ten hours without a rest -- I don't know how many circuits of the town I made, but I must have run over sixty miles. I was probably in for worse than just a few sore muscles. I could very easily have pulled the muscles in my thighs or, yes, suffered a stress fracture. Given that I slammed down hard onto the asphalt after pushing myself to the point that my legs literally buckled under me. But it didn't feel like a forfeit, it felt like I'd crossed some invisible finish line. I had that feeling of elation. Like I'd completed the race. No one had told me to run, \& I hadn't actually resolved a damn thing with Numachi, but I nevertheless felt like my slate had been wiped clean. ``My legs…are killing me.'' Not just my legs, my whole body was killing me. It was a struggle even to blink. But it was probably nothing compared to the pain Numachi had felt--according to Higasa, she'd been dealing with a lot of other stuff too, but it was hard for me to believe that she'd chosen death for any reason other than that pain. What besides that suffering would have driven her to die -- since her emotional pain seemed to be eased to some degree by her unhappiness collecting, the foundation for which she laid even before transferring. But maybe that was just what I wanted to believe. At this point, I couldn't really know how much of her story was true \& how much of it was a lie. Common sense dictated that she was nothing but a hallucination, something I saw at a particularly sensitive moment in my life with my seniors gone \& my environment altered -- including the devil's arm. ``I guess I should have at least paid some attention to my form…'' I muttered as I lifted my head slightly. It felt like lifting a ten-ton weight, \& once I got it up I saw that the soles of my brand-new Reeboks had worn down to nothing. ``But if I did, I doubt I would've made it.'' Only after the words got out did I realize that I had no idea what I'd made, \& I looked up at the sky with a wry smile on my face.''
    \end{itemize}
    {\sf Yotsugi Ononoki.}
    \begin{itemize}
    	\item ``But still $\ldots$ Not trying to undo misunderstanding is the same as telling a lie.''
    \end{itemize}
    {\sf Meme Oshino.} Meme Oshino might've not stuck around for very long, but he certainly made an impact on the series \& some of its most important characters. Initially introduced as an oddity problem solver, Meme became a quick fan favorite for his aloof personality \& no-nonsense attitude.
    \begin{itemize}
    	\item ``People have to save themselves. 1 person saving another is impossible.''
    	\item ``You look energetic, did something good happen?''
    \end{itemize}
    {\sf Oshino Ougi.}
    \begin{itemize}
    	\item ``I don't know anything. You're the one that knows everything.''
    	\item ``When the world is filled with red lights signaling danger, the world is safer than usual. But when it's filled with green lights signaling safety, it creates a place more dangerous than anywhere.''
    \end{itemize}
    {\sf Nadeko Sengoku.}
    \begin{itemize}
    	\item ``Since that isn't good, let's try this.''
    \end{itemize}
    {\sf Gaen Tooe.}
    \begin{itemize}
    	\item ``If you can't be medicine, be poison. Otherwise you're nothing but water.''
    	\item ``Don't let your opposite side become your opposing side.''
    \end{itemize}
    
    \item Monogatari: Off \& Monster Season (2024)\hfill[S1.E14--]
    \item {\sc Môsô dairinin $\star$ Paranoia Agent} (2004)\hfill[S1.E11]
    \item Mushoku-tensei $\sim$Isekai ittara honki dasu$\sim$ $\star$ Mushoku Tensei: Jobless Reincarnation (2021--)\hfill[S1.E11][S2.E24][Chap. 87--]
    \item My Balls\hfill[Chap. 41]
    \item My Home Hero (2023--)\hfill[S1.E1]
    \item {\sc Nekomonogatari (Kuro)} (2012)[S1.E4]
    \item {\sc N.H.K ni yôkoso! $\star$ Welcome to the N.H.K.} (2006)\hfill[S1.E24]
    \item {\sc Nier: Automata Ver1.1a}\hfill[S1.E24]
    \item {\sc Ninja Kamui} (2024)\hfill[S1.E13]
    \item {\sc Nozomanu Fushi no Bôkensha $\star$ The Unwanted Undead Adventurer} (2024--)\hfill[S1.E12]
    \item {\sc Odd Taxi} (2021)\hfill[S1.E13]
    \item {\sc Orange} (2016)\hfill[S1.E13]
    \item {\sc Ore dake Level Up na Ken $\star$ Solo Leveling} (2024--)\hfill[S1.E12][Chap. 200]
    \item {\sc Ore wa Subete wo [Parî] Suru ~Gyaku Kanchigai no Sekai Saikyô wa Bôkensha ni Naritai~ $\star$ I Parry Everything} (2024--)\hfill[S1.E12]
    \item {\sc Oshi no Ko} (2023--)\hfill[S1.E11][S2.E13][S3.E1--][Chap. 166]
    \item {\sc \^Okami kodomo no Ame to Yuki $\star$ Wolf Children} (2012)
    \item One Punch Man: Wanpanman $\star$ One Punch Man (2015--)\hfill[S1.E12][S2.E12]
    \item \^Osama Ranking $\star$ Ranking of Kings (2021--)\hfill[S1.E23][S2.E3--]
    \item Otonari no tenshi-sama ni itsu no ma ni ka dame-ningen ni sareteita ken $\star$ The Angel Next Door Spoils Me Rotten (2023--)\hfill[S1.E12]
    \item {\sc Owarimonogatari} (2015--2017)\hfill[S1.E13][S2.E7]
    \item {\sc Papurika $\star$ Paprika} (2006)
    \item Paradox Live the Animation (2023--)\hfill[S1.E3--]
    \item {\sc Pâfekuto burû $\star$ Perfect Blue} (1997)
    \item Perfect Half\hfill[Chap. 145{\tt/}145--]
    \item {\sc Pluto} (2023)\hfill[S1.E8]
    \item {\sc Psycho-Pass} (2012--)\hfill[S1.E22][S2.E11][S3.E8]
    \item {\sc Puss in Boots: The Last Wish} (2022)
    \item {\sc Ragna Crimson} (2023--)\hfill[S1.E24]
    \item {\sc Ratatouille} (2007)
    \item {\sc Re: Monster} (2024--)\hfill[S1.E12]
    \item Re:Zero kara Hajimeru Isekai Seikatsu $\star$ Re: Zero, Starting Life in Another World (2016--)\hfill[S1.E25][S2.E25][S3.E6--]
    \item Rick \& Morty (2013--)\hfill[S1.E11][S2.E10][S3.E10][S4.E10][S5.E10][S6.E10]
    \item {\sc Rukku Bakku $\star$ Look Back} (2024)
    \item {\sc Saiki Kusuo no Psi Nan $\star$ The Disastrous Life of Saiki K.} (2016--2018)\hfill[S1.E24][S2.E24][S3.E2]
    \item Saikyô no Shienshoku [Wajutsushi] Dearu ore wa Sekai Saikyô Kuran o Shitagaeru $\star$ The Most Notorious ``Talker'' Runs the World's Greatest Clan (2024--)\hfill[S1.E10--]
    \item {\sc Saikyou Onmyouji no Isekai Tenseiki $\star$ The Reincarnation of the Strongest Exorcist in Another World} (2023)\hfill[S1.E13]
    \item {\sc Sama Taimu Renda $\star$ Summer Time Rendering} (2022--)\hfill[S1.E25][Chap. 139]
    \item {\sc Samâ uôzu $\star$ Summer Wars} (2009)
    \item Samurai chanpurû $\star$ Samurai Champloo (2004--2005)\hfill[S1.E8{\tt/}E26]
    \item {\sc Sangatsu no Lion $\star$ March Comes in Like a Lion} (2016--2018)\hfill[.E23][S2.E22]
    \item {\sc Sasaki to Pi-chan $\star$ Sasaki \& Peeps} (2024--)\hfill[S1.E12]
    \item {\sc Seiken Gakuin no Makentsukai $\star$ The Demon Sword Master of Excalibur Academy} (2023--)\hfill[S1.E12]
    \item {\sc Seishun Buta Yaro wa Bunny Girl-senpai no Yume wo Minai $\star$ Rascal Does Not Dream of Bunny Girl Senpai} (2018--)\hfill[S1.E14]
    \item {\sc Seishun Buta Yaro wa Yumemiru Shoujo no Yume wo Minai $\star$ Rascal Does Not Dream of Bunny Girl Senpai The Movie} (2019)
    \item {\sc Seishun buta yaro ha Odekake sisuta no yume wo minai $\star$ Rascal Does Not Dream of a Sister Venturing Out} (2023)
    \item {\sc Seishun buta yaro wa ransel girl no yume o minai $\star$ Rascal Does Not Dream of a Knapsack Kid} (2023)
    \item {\sc Serial Experiments Lain} (1998)\hfill[S1.E13]
    \item {\sc Shangri-La Frontier: Kusoge Hunter, Kamige ni Idoman to su $\star$ Shangri-La Frontier} (2023)\hfill[S1.25][S2.E8--]
    \item {\sc Shigatsu wa kimi no uso $\star$ Your Lie in April} (2014--2015)
    
    {\sc Arima Kousei.}
    \begin{itemize}
    	\item ``Spring will be here soon. Spring, the season I met you, is coming. A Spring without you... is coming.''
    	\item ``Everything you say \& do... it all sparkles so brightly. It's too blinding for me, \& I end up closing my eyes. But I can't help aspiring to be like you.''
    	\item ``Maybe... just maybe, the light can reach even the bottom of a dark ocean.''
    	\item ``Music speaks louder than words.''
    	\item ``For you, I am casting about for an excuse again.''
    	\item ``This silence belongs to us... \& every single person out there, is waiting for us to fill it with something.''
    	\item ``This silence belongs to us. Every single person here... is waiting for us to start producing sounds.''
    	\item ``Isn't it funny how the most unforgettable scenes can be so trivial?''
    	\item ``A lump of steel, like a shooting star. Just seeing the same sky as you makes familiar scenery look different. I swing between hope \& despair at your slightest gesture, \& my heart starts to play a melody. What kind of feeling is this again? What do they call this kind of feeling? I think it's probably... called love. I'm sure this is what they call love.''
    	\item ``Sure I'm okay. Because that's how I was built, after all.''
    	\item ``The piano is meant to be played like you're embracing it, right?''
    	\item ``You know, I discovered something. Everyone has something... Something deep inside their hearts. For some, it might have been enmity. For others, admiration. Wishes, a craving for the spotlight, feelings that one wants to deliver, feelings for one's mother. Everyone was supported by their own feelings. I realize now that, perhaps, no one can stand alone on stage.''
    	\item ``You're like a cat. If I get close, you'll ignore me \& go far away. If I get hurt, you'll play around to share the pain.''
    	\item ``The more I concentrate, the more I get consumed by my performance. The sounds I play fade away from my reach, tangling up like flowers seized by the spring wind \& vanish.''
    	\item ``But I wasn't slacking at all. I practiced till I passed out. I gave my all. And if I still turned in a sloppy performance after all that, then... That's who I am now. It's the current me, playing with everything I've got.''
    	\item ``The moment I met her, my life changed. Everything I saw, everything I heard, everything I felt, all the scenery around me... started to take on color.''
    	\item ``I want to hear it again, yet I don't want to hear it. I want to see her, yet I don't want to see her. What do you call this kind of feeling again?''
    	\item ``As if you can see right through me, into my heart... Always, out of nowhere, you... just show up.''
    	\item ``She's merciless. That unbending gaze even from the back, she won't let me give up. The one who was being supported... was me. Thank you. Thank you.''
    	\item ``A single petal that drifted into my life. The worst 1st impression ever. The girl who likes my best friend. Will it reach her? I hope it reaches her.''
    	\item ``You exist inside a spring that can't be replaced.''
    	\item ``Naïve... Bizarre... it's like I'm on a rollercoaster. I'm being jerked around \& around. It's like this girl herself is the journey with no clear destination. You're freedom itself.''
    	\item ``That moment, when my music reached them... there's no way I could ever forget that. Because I'm a musician, just like you.''
    	\item ``She moves me. With such power, like the pounding of my heart. I can hear your sound. You're here.''
    	\item ``Because of music, I was given the chance to meet others. I was moved by those encounters. There are people I got to meet. I got to discover emotion. These are all... memories that my mother, who taught me how to play the piano, left me.''
    	\item ``Just one person matters to me. Only you matter.''
    	\item ``You're in love with food, you're in love with the violin, you're in love with music. I guess that's why you sparkle.''
    	\item ``After struggling, losing my way, \& suffering... the answer I arrived at was so laughably simple...''
    	\item ``This is all your fault. Because you put me back on the stage. Always... you move me. I'm going to prove it. That I'm incredible. That Kaori Miyazono, who's named me to be her accompanist... is even more incredible.''
    	\item ``I'm... going on a journey. The applause raining down. Pursuing that moment when my music reached them. Pursuing that sight of her with her back to me. until one day, for sure, I've pulled even with her... until that day comes.''
    	\item ``There's no-one who'd ever fall in love with me.''
    	\item ``To me it all looks like it's in monotone. Just like music scores... just like a keyboard.''
    	\item ``Setbacks come with the territory of becoming a superstar.''
    	\item ``Even it the depths of the darkest oceans, some light always pierces through.''
    	\item ``I knew all along. The ghost of my mother was a shadow of my own creation. An excuse for me to run away. My own weakness. Mom isn't there anymore. Mom... is inside me.''
    	\item ``The way I touch the keys, the way I move my fingers, my habit of squeezing the pedals, my tastes, the order that I eat... Mom's in every little gesture of mine. We're... Mom \& I... are connected.''
    	\item ``I'm a guy who hurled my precious sheet music away. I don't deserve to be a performer.''
    	\item ``I look like I'm suffering, huh? That's not good... but of course I'd be suffering. I mean, I'm gonna sail in charted waters, right? Both, taking on a challenge \& creating something. It is painful, but it's fulfilling. So thank you. For sweeping away the dust that had collected on my body. .. For encountering me... ever since that day... my world, even the keyboard... became colorful.''
    	\item ``Spring will be here soon. Spring, the season I met you, is coming. A Spring without you... is coming.''
    	\item ``That smile of yours, who flew threw the window \& died. I'll never forget it.''
    	\item ``I couldn't ask her the reason for her tears.''
    	\item ``How can I forgot about you, when everything about you, already became a part of me?''
    	\item ``Hold on. Don't go! Let's argue again. I'll bribe you with a canalé. I'll call you to kill time. I don't mind being Friend A. Please don't go. Please don't go. Please don't go...please don't leave me behind...''
    	\item ``Nothing better than memories \& there's nothing worse than them.''
    	\item ``You gave me forever within the numbered days. And I'm grateful.''
    	\item ``I just really wish that you were her. To compose another memory. I promise when I'm missing you. I'll play your symphony.''
    	\item ``What did you have in your heart? What did you lean on?''
    	\item ``One day in April, I met a really weird violist. Totally outrageous. Self-righteous. But the smile she shows to people, she is like angelic.''
    	\item ``I have no idea what lies ahead, but I've taken the 1st steps.''
    	\item ``Ever since that day I met you, the world became colorful.''
    \end{itemize}
    {\sc Kaori Miyazono.}
    \begin{itemize}
    	\item ``Was I able to live inside someone's heart? Was I able to live inside your heart? Do you think you'll remember me at least a little? You'd better not hit ''reset!'' Don't forget me, okay? That's a promise, okay? I'm glad it's you, after all. Will I reach you? I hope I can reach you.''
    	\item ``If you can't move with your hands then play with your feet! If you don't have enough fingers, then use your nose as well. Whether you're sad, you're a mess, or you've hit rock bottom, you still have to play. That's how people like us survive.''
    	\item ``Maybe there's only a dark road ahead. But you still have to believe \& keep going. Believe that the stars will light your path, even a little bit. Come on... Let's go on a journey!''
    	\item ``Mozart's telling us from up the sky... ``Go on a journey,'' he's saying.''
    	\item So ephemeral \& weak. But it's shining with all its might. Thump, Thump, like a heartbeat. This is the light of life.''
    	\item ``Such a cruel boy. Telling me to dream one more time. I thought I was satisfied because my dream had come true... And I'd told myself it was enough... Yet here you are, watering this withered heart again.''
    	\item ``We're all afraid, you know.. to get up on stage. Maybe you'll mess up. Maybe they'll totally reject you. Even so, you grit your teeth \& get up on stage anyway.''
    	\item ``Music is freedom.''
    	\item ``By exchanging notes, you get to know one another, to understand one another. As if your souls were connected \& your hearts were overlapping. It's a conversation through instruments. A miracle that creates harmony. In that moment, music transcends words.''
    	\item ``Do you think you will be able to forget?''
    	\item ``Memories are always special. Sometimes we smile by remembering the days we cried. And cry by remembering the days we smile.''
    	\item ``Even if you know what's coming you're never prepared for the feels.''
    	\item ``You were my greatest hello. My saddest goodbye \& the biggest what if. I will question it for the rest of my life.''
    	\item ``Don't worry if you're rejected. They maybe the people you want not the people you need.''
    	\item ``I want you to smile everytime you think about me \& in your smile I will live forever.''
    	\item ``Sometimes laughter is not the best medicine it can also be a disguise.''
    	\item ``You \& I, we have music in our bones.''
    \end{itemize}
    {\sc Tsubaki Sawabe.}
    \begin{itemize}
    	\item ``Even though I'm bitter over losing, even though I'm depressed, even though my ankle hurts, \& my eyes are smeared with tears... even though I've never felt worse... I wonder why the stars are sparkling like this.'' 
    	\item ``Sure, I know that I have no right to be feeling this way. But I still don't like it! I just don't like it. We were always together. I was always by his side. During time of joy, \& grief as well. But... I realize he's far away from me now... I'm not by his side... there's somebody else there.''
    	\item ``The boy I took for granted would always be my side, the boy I want to be by my side forever. I'm such an idiot.''
    	\item ``I wish time would just stand still.''
    \end{itemize}
    {\sc Hiroko Seto.}
    \begin{itemize}
    	\item ``We're all connected. Just like the notes are intermittently connected. It's shared by us all. Through music, with the people you know, with the people you don't know, with all the people in this world.''
    	\item ``Defiance toward parents is an establishment of one's self; it's a sign of independence.''
    	\item It's not just allies who support each other. From your enemies, you learn so much \& gain so much. Until the day you meet again... Just knowing they exist helps you to withstand the loneliness. Those who compete, even if they're enemies, help each other out.''
    	\item ``As a musician, in the process of learning from a teacher, your differences breed a sense of discomfort which you should cherish. It's because of those differences that we have individuality.''
    	\item ``When you say you ``can't hear the sound'', doesn't it really mean you ``aren't restrained by the sound''? Rather than the sound you hear with your ears, an image inside you is boiling up from the depths of yourself \& taking over without you even knowing. The sound inside, the landscape in your heart, your wishes, a sound loaded with your thoughts; didn't you feel it, even for a moment? ``Not being able to hear the sound.'' That is a gift.''
    \end{itemize}
    {\sc Watari Ryouta.}
    \begin{itemize}
    	\item ``Setbacks are inevitable to superstars. Adversity is what separates the good from the great. After all, stars can only shine during the night.''
    	\item ``It's only natural for the girl you're crushing on to be in love with someone else. Since you're in love with her, she sparkles in your eyes. That's why people fall so irrationally in love.''
    \end{itemize}
    {\sc Emi Igawa.}
    \begin{itemize}
    	\item``The music felt like April.''
    \end{itemize}
    \item {\sc Shimoneta to Iu gainen ga sonzai shinai taikutsu na sekai $\star$ Shimoneta: A Boring World Where the Concept of Dirty Jokes Doesn't Exist} (2015--)\hfill[S1.E12]
    \item {\sc Shin seiki evangerion $\star$ Neon Genesis Evangelion} (1995--1996)\hfill[S1.E26]
    
    {\bf Kaworu Nagisa.}
	\begin{itemize}
   		\item ``Humans cannot create anything out of nothingness. Humans cannot accomplish anything without holding onto something. After all, humans are not gods.''
   		\item ``The fact that you have a place where you can return home, will lead you to happiness. That is a good fact.''
   	\end{itemize}
    {\bf Shinji Ikari.}
    \begin{itemize}
    	\item ``I still don't know where to find happiness. But I'll continue to think about whether it's good to be here $\ldots$ whether it was good to have been born. But in the end, it's just realizing the obvious over \& over again. Because I am myself.''
    	\item ``No one can justify life by linking happy moments into a rosary\footnote{{\bf rosary} [n] (plural {\bf rosaries}) {\bf 1.} [countable] a string of beads that are used by some Roman Catholics for counting prayers as they say them; {\bf 2.} {\bf the Rosary} [singular] the set of prayers said by Roman Catholics while counting rosary beads.}.''
    	\item ``I didn't have a choice! They made me pilot the stupid thing!''
    	\item ``I mustn't run away! I mustn't run away! I mustn't run away!''
    \end{itemize}
	{\bf Ritsuko Akagi.}
	\begin{itemize}
		\item ``This is man's ultimate fighting machine the synthetic life form know as Evangelion, Unit 1. But here in secret, it is mankind's last hope.''
		\item ``Even though a hedgehog may want to become close with another hedgehod. The closer they get the more they injure each other with their spines.''
	\end{itemize}
	{\bf Misato Katsuragi.}
	\begin{itemize}
		\item ``This city is a fortress designed to stand against the angels. This is Tokyo3, this is our city \& it's the city that you saved.''
		\item ``If getting into the Eva means nothing but pain to him, I don't think he should pilot again. Ugh, otherwise he'll be killed.''
	\end{itemize}
	{\bf Asuka Langley.}
	\begin{itemize}
		\item ``My mind is being eaten away $\ldots$ Kaji-san, it's unraveling my mind! What do I do? It's defiling my mind.''
	\end{itemize}
	{\bf Yui Ikari.}
	\begin{itemize}
		\item ``Anywhere can be paradise as long as you have the will to live. After all, you are alive, so you will always have the chance to be happy.''
	\end{itemize}
    \item {\sc Shin seiki Evangelion Gekijô-ban: Air/Magokoro wo, kimi ni $\star$ Neon Genesis Evangelion: The End of Evangelion} (1997)
    \item {\sc Shingeki no kyojin $\star$ Attack on Titan} (2013--2022)\hfill[S1.E25][S2.E12][S3.E22][S4.E30][Chap. 139]
    
    {\bf Eren Jaeger.}
    \begin{itemize}
    	\item ``If you think reality is just living comfortably \& following your own whims, can you seriously dare to call yourself a soldier?''
    	\item ``What is the point if those with the means \& power do not fight?''
    	\item ``I don't have time to worry if it's right or wrong, you can't hope for a horror story with a happy ending.''
    	\item ``I want to see \& understand the world outside. I don't want to die inside these walls without knowing what's out there!''
    	\item ``Nothing can suppress a human's curiosity.''
    	\item ``I'll slaughter you all $\ldots$ \& take back what you stole $\ldots$ All of it!''
    	\item ``If you win you live. If you lose you die. If you don't fight, you can't win.''
    	\item ``I knew $\ldots$ you were more of a hero than anyone else.''
    	\item ``I can do this. No, we can do this! Because we've all been special since the day we were born. We're free!''
    	\item ``No matter how messed up things get, you can always figure out the best solution.''
    	\item ``I disposed of some dangerous beasts. Mere animals that just happened to resemble humans.''
    \end{itemize}
	{\bf Hange Zoe.}
	\begin{itemize}
		\item ``Ever since I joined the survey corps, I've had people dying on me everyday. But you understand, don't you? 1 day or another, everyone you care about eventually dies. It's something we simply can't accept. It's a realization that could drive you insane.''
		\item ``Even in moments of the deepest despair $\ldots$ I guess we can still find hope, huh?''
	\end{itemize}
	{\bf Armin Arlert.}
	\begin{itemize}
		\item ``People are crazy for believing that these walls will protect us forever. Even though the walls have been intact for the past 100 years, there's nothing that can guarantee they won't be broken down today.''
		\item ``You're only resorting to physical abuse because you can't prove that I'm wrong.''
		\item ``I think there are times people have to die $\ldots$ even if I don't like it.''
		\item ``When people are faced with a situation they don't understand, it's easy for fear to take hold.''
		\item ``People who can't throw something important away, can never hope to change anything.''
		\item ``I'm leaving it all with Eren. My dream, my life, everything. I have nothing else left to lose. I'm sure Eren will be able to reach the ocean. He'll have to see it for both of us.''
		\item ``To surpass monsters, you must be willing to abandon your humanity.''
		\item ``Willpower alone isn't enough in battle.''
		\item ``We're going to explore the outside world someday, right? Far beyond these walls, there's flaming water \& made of ice, \& fields of sand spread wide. It's the world my parents wanted to go to.''
		\item ``I don't like the terms good person or bad person because it's impossible to be entirely good to everyone, or entirely bad to everyone. To some, you are a good person, while to others you are a bad person.''
		\item ``The strong feed upon the weak. It's such an obligingly simple rule. Except in this world, my friends tried to be strong.''
		\item ``Endure it. Don't let go.''
		\item ``Everyone can make a choice after they have learned what it will result in. It is so easy to say we should have done in this way afterwards. But you can't know what your choice will result in before actually choosing.''
		\item ``I was, I am, \& I remain a soldier, sworn to devote my heart \& soul to the restoration of humankind. There is no greater glory than dying for that belief!''
		\item ``I'd rather die than become a burden.''
	\end{itemize}
	{\bf Ymir.}
	\begin{itemize}
		\item ``I want to survive $\ldots$ \& see her again. As a person, I'm really lower than shit. But she knows that, \& she smiles kindly at me anyway.''
		\item ``Do you always want to live hiding behind the mask you put up for the sake of others? You're you, \& there's nothing wrong with that.''
		\item ``Living this way is my way of getting revenge. I'm going to be living proof that your fate isn't decided at birth!''
		\item ``I too used to believe that the world would be a better place if I hadn't been born. I was hated merely for the fact that I existed, \& I died for the happiness of many people. But there was 1 thing I wished for with all my heart. If I'm ever given a 2nd chance in life, I want to live for only myself. That is my sincere wish.''
	\end{itemize}
	{\bf Erwin Smith.}
	\begin{itemize}
		\item ``If we only focus on making the best moves, we will never get the better of our opponent. When necessary, we must be willing to take big risks, \& be prepared to lose everything. Unless we change how we fight, we cannot win.''
		\item ``If you begin to regret, you'll dull your future decisions \& let others make your choices for you. All that's left for you then is to die. Nobody can foretell the outcome. Each decision you make holds meaning only by affecting your next decision.''
		\item ``It's us who gives meaning to our comrades lives.''
		\item ``They want to know what became of the heart they gave. Because the fighting isn't over yet.''
	\end{itemize}
	{\bf Historia Reiss.}
	\begin{itemize}
		\item ``We need to stop living for others. From now on $\ldots$ Let's live for ourselves.''
		\item ``Even if you have your reasons \& there are things you can't tell me, no matter what, I'm on your side.''
	\end{itemize}
	{\bf Levi Ackerman.}
	\begin{itemize}
		\item ``A lot of the time, you're going into a situation you know nothing about. So what you need is to be quick to act $\ldots$ \& make tough decisions in worst-case scenarios.''
		\item ``I don't know which option you should choose. I could never advise you on that $\ldots$ No matter what kind of wisdom dictates you the option you pick, no one will be able to tell if it's right or wrong until you arrive to some sort of outcome from your choice.''
		\item ``Some scouts'' lives are more valuable than others, only those dumb enough to acknowledge that join us.''
		\item ``Whether you have the body, dead is dead.''
		\item ``The lesson you need to learn right now can't be taught with words, only with action.''
		\item ``The only thing we're allowed to do is believe that we won't regret the choice we made.''
		\item ``Don't get me wrong. It's not like I trust him. If he betrays us or goes berserk, I'll put him down without hesitation.''
		\item ``I think pain is the best discipline.''
		\item ``It's good to see that someone has the balls to go. But don't forget to do your damnedest to stay alive.''
		\item ``If you don't want to die, think!''
		\item ``The difference between your decision \& ours is experience. But you don't have to rely on that.''
		\item ``No casualties, Don't you dare Die!''
	\end{itemize}
	{\bf Hannes.}
	\begin{itemize}
		\item ``You couldn't save your mom because you weren't strong enough. I didn't face the Titan $\ldots$ because I wasn't brave enough.''
	\end{itemize}
	{\bf Mikasa Ackerman.}
	\begin{itemize}
		\item ``This world is merciless, \& it's also very beautiful.''
		\item ``I don't want to lose what little family I have left.''
		\item ``There are only so many lives I can value. \& $\ldots$ i decided who those people were 6 years ago. So you shouldn't try to ask for my pity. Because right now, I don't have time to spare or room in my heart.''
		\item ``That's right $\ldots$ This world $\ldots$ is cruel. It hit me that living was like a miracle.''
		\item ``I'm sorry Eren. I won't give up. I'll never give up again. So I'll win, no matter what! I'll survive no matter what.''
		\item ``The world is crammed with cruelty.''
		\item ``I am strong, real strong. None of you come close.''
		\item ``Only victors are allowed to live. This world is merciless like that.''
		\item ``Believe in your own power.''
		\item ``Asking me for compassion is mistaken. After all I have no heart or time to spare.''
		\item ``My speciality is lacerating flesh. Anyone interested in experiencing my skill firsthand, step right up.''
		\item ``Once I'm dead I won't be able to remember you. So I'll win no matter what. I'll live no matter what.''
		\item ``You don't stand a single chance to win unless you fight.''
	\end{itemize}
	{\bf Annie Leonhart.}
	\begin{itemize}
		\item ``Going against the flow takes a lot of courage. I respect that. Maybe people who can do it are just stupid, but $\ldots$ Well, what I'm sure of is that people like that are rare.''
	\end{itemize}
	{\bf Marco Bott.}
	\begin{itemize}
		\item ``You're not a strong person, so you can really understand how weak people feel. I mean $\ldots$ Most humans are weak, including me. But if I got an order from someone who saw things like I do, no matter how tough it was, I'd do my damnedest to carry it out.''
	\end{itemize}
	{\bf Jean Kirstein.}
	\begin{itemize}
		\item ``The future of humanity will be doomed. Having said that, I'm not about to sit around while we all get slaughtered.''
		\item ``Right now we've got no choice but to try. We gotta believe there's a way to beat him!''
	\end{itemize}
	\item {\sc Shinmai Ossan Bôkensha, Saikyô Party ni Shinu Hodo Kitaerarete Muteki ni Naru $\star$ The Ossan Newbie Adventurer, Trained to Death by the Most Powerful Party, Became Invincible} (2024--)\hfill[S1.E12]
	\item {\sc Shôshimin Shirîzu $\star$ Shoshimin: How to Become Ordinary} (2024--)\hfill[S1.E10]
	\item Skip \& Loafer (2023--)\hfill[S1.E11--]
	\item Sokushi Cheat ga Saikyôsugite, Isekai no Yatsura ga Marude Aite ni Naranain Desu ga $\star$ My Instant Death Ability is Overpowered (2024--)\hfill[S1.E11--]
    \item Sono Bisque Doll wa Koi wo Suru $\star$ My Dressing-Up Darling\hfill[S1.E12][Chap. 73--]
    \item {\sc Sousou no Frieren $\star$ Frieren: Beyond Journey's End} (2024)\hfill[S1.E28]
    \item {\sc Spy $\times$ Family} (2022--)\hfill[S1.E25][S2.E12][Chap. 1--]
    \item {\sc Spy x Family Code: White} (2023)
    \item {\sc Stein;Gate} (2011--2015)\hfill[S1.E26]
    \item {\sc Steins;Gate: The Movie - Load Region of Déjà Vu} (2013)
    \item {\sc Stein;Gate 0} (2018)\hfill[S1.E24]
    \item {\sc Suzume no Tojimari $\star$ Suzume} (2022)
    \item {\sc Sweet Guy}\hfill[Chap. 75]
    \item Tengoku Daimakyou $\star$ Heavenly Delusion (2023--)\hfill[S1.E13]
    \item {\sc Tenki no ko $\star$ Weathering with You} (2019)
    \item The Eminence in Shadow (2022--)\hfill[S1.E20][S2.E12][Chap. 64--]
    \item {\sc Toki o kakeru sh\^ojo $\star$ The Girl Who Leapt Through Time} (2006)
    \item {\sc Tokidoki Bosotto Russia-go de Dereru Tonari no Alya-san $\star$ Alya Sometimes Hides Her Feelings in Russian} (2024--)\hfill[S1.E12]
    \item {\sc Tomodachi Game} (2022--)\hfill[S1.E12]
    \item Tondemo Skill de Isekai Hourou Meshi $\star$ Campfire Cooking in Another World with My Absurd Skill (2023--)\hfill[S1.E12]
    \item {\sc Tsuki ga michibiku isekai-dôchû $\star$ Tsukimichi: Moonlit Fantasy} (2021--2024)\hfill[S1.E12][S2.E25]
    \item {\sc Tsukimonogatari} (2014)\hfill[S1.E4]
    \item {\sc Undead Girl Murder Farce $\star$ Undead Murder Farce} (2023--)\hfill[S1.E13]
    \item Undead Unluck (2023--)\hfill[S1.E24--]
    \item {\sc Up} (2009)
    \item {\sc Uzumaki $\star$ Uzumaki: Spiral Into Horror} (2024)\hfill[S1.E4]
    \item {\sc Vinland Saga} (2019--)\hfill[S1.E24][S2.E24]
    \item {\sc Violet Evergarden} (2014)\hfill[S1.E14]
    \item {\sc Vivy: Fluorite Eye's Song} (2021--)\hfill[S1.E13]
    \item {\sc WALL$\cdot$E} (2008)
    \item Watashi no Oshi wa Akuyaku Reijou $\star$ I'm in Love with the Villainess (2023--)\hfill[S1.E12--]
    \item {\sc Watashi no Shiawase na Kekkon $\star$ My Happy Marriage} (2023--)\hfill[S1.E12][OVA]
    \item {\sc Yahari ore no seishun rabukome wa machigatteiru. $\star$ My Teen Romantic Comedy SNAFU}\hfill[S1.E14][S2.E14][S3.E12]
    \item {\sc Yakusoku no Neverland $\star$ The Promised Neverland}\hfill[S1.E12][S2.E11][Chap. 1{\tt/}181.5]
    \item Yamada-kun to Lv999 no Koi wo Suru $\star$ My Love Story with Yamada-kun at Lv999 (2023--)\hfill[S1.E13]
    \item Yôkoso jitsuryoku shijô shugi no kyôshitsu e $\star$ Classroom of the Elite (2017--)\hfill[S1.E12][S2.E13][S1.E12--]
    \item {\sc Yosuga No Sora} (2010--)\hfill[S1.E12]
    \item Yozakura-san Chi no Daisakusen $\star$ Mission: Yozakura Family, Mission of Yozakura family (2024--)\hfill[S1.E1--]
    \item {\sc Yûsha ga shinda! $\star$ The Legendary Hero Is Dead!} (2023--)\hfill[S1.E12]
    \item {\sc Yuukoku no Moriarty $\star$ Moriarty the Patriot} (2020)\hfill[S1.E26]
    \item {\sc Zankyô no teroru $\star$ Terror in Resonance} (2014)\hfill[S1.E11]
    \item {\sc Zoku Owarimonogatari} (2018)[S1.E6]
    \item {\sc Zom 100: Zombie ni Naru made ni Shitai 100 no Koto $\star$ Zom 100: Bucket List of the Dead} (2023--)\hfill[S1.E12]
\end{enumerate}

%------------------------------------------------------------------------------%

\section{Movie}

\begin{enumerate}
	\item {\sc 3 Idiots} (2009)
	\item {\sc A Beautiful Mind} (2001)
	\item {\sc Alita: Battle Angel} (2019)
	\item {\sc American Psycho} (2000)
	\item {\sc Ant-Man} (2015)
	\item {\sc Ant-Man \& The Wasp} (2018)
	\item {\sc Aquaman} (2018)
	\item {\sc Arrival} (2016)
	\item {\sc Avatar} (2009)
	\item {\sc Avengers: Age of Ultron} (2015)
	\item {\sc Avengers: Endgame} (2019)
	\item {\sc Avengers: Infinity War} (2018)
	\item {\sc Batman Begins} (2005)
	\item {\sc Better Call Saul} (2015--2022)\hfill[S1.E10][S2.E10][S3.E10][S4.E10][S5.E10][S6.E13]
	\item {\sc Birdman or (The Unexpected Virtue of Ignorance)} (2014)
	\item Black Mirror (2011--2019)\hfill[S1.E3][S2.E4][S3.E6][S4.E6][S5.E1{\tt/}E3]
	\item {\sc Black Panther} (2018)
	\item {\sc Black Swan} (2010)
	\item {\sc Blade Runner 2049} (2017)
	\item {\sc Blood Diamond} (2006)
	\item {\sc Breaking Bad} (2008--2013)\hfill[S1.E7][S2.E13][S3.E13][S4.E13][S5.E16]
	
	{\sc Walter White.}
	\begin{itemize}
		\item ``You clearly don't know who you're talking to, so let me clue you in. I am not in danger, Skyler. I am the danger. A guy opens his door \& gets shot, \& you think that of me? No! I am the one who knocks!''
		\item ``Right now, what I need, is for you to climb down out of my ass. Can you do that? Will you do that for me honey? Will you please, just once, get off my ass? You know? I'd appreciate it. I really would.''
		\item ``Smoking marijuana, eating Cheetos, \& masturbating do not constitute plans in my book.''
		\item ``Fuck you, \& your eyebrows.''
		\item {\bf On the importance of caution.} ``If you don't know who I am, then maybe your best course would be to tread lightly.'' - Walter White, Breaking Bad, Season 5, Blood Money
		\item ``Is this just a genetic thing with you? Is it congenital? Did your, did your mother drop you on your head when you were a baby?''
		\item {\bf On overcoming your fears.} ``I have spent my whole life scared, frightened of things that could happen, might happen, might not happen, 50 years I spent like that. Finding myself awake at three in the morning. But you know what? Ever since my diagnosis, I sleep just fine. What I came to realize is that fear, that's the worst of it. That's the real enemy. So, get up, get out in the real world \& you kick that bastard as hard you can right in the teeth.'' - Walter White, Breaking Bad, Season 2, Better Call Saul
		\item {\bf On selfishness.} ``I did it for me. I liked it. I was good at it. And, I was really$\ldots$ I was alive.''
		\item ``Jesse, you asked me if I was in the meth business, or the money business$\ldots$ Neither. I'm in the empire business.''
		\item ``We tried to poison you. We tried to poison you because you're an insane, degenerate piece of filth, \& you deserve to die.''
		\item ``I watched Jane die. I was there. And I watched her die. I watched her overdose \& choke to death. I could have saved her. But I didn't.''
		\item ``I told you Skyler, I warned you for a solid year: You cross me, \& there will be consequences.''
		\item ``Say my name.''
		\item ``Stay out of my territory.''
		\item {\bf On trauma.} ``You need to stop focusing on the darkness behind you. The past is the past. Nothing can change what we've done.'' - Walter White, Breaking Bad, Season 5, Blood Money
		\item {\bf On living life on your own terms.} ``I've been living with cancer for the better part of a year. Right from the start it's a death sentence. That's what they keep telling me. Well, guess what? Every life comes with a death sentence. So, every few months I come in here for my regular scan knowing full well that one of these times - hell, maybe even today - I'm gonna hear some bad news. But until then, who's in charge? Me. That's how I live my life.'' - Walter White, Breaking Bad, Season 4, Hermanos
		\item {\bf On change.} ``Electrons - they change their energy levels. Molecules change their bonds. Elements - they combine \& change into compounds. Well, that's all of life, right? It's the constant. It's the cycle. It's solution, dissolution, just over \& over \& over. It is growth, then decay, then transformation.'' - Walter White, Breaking Bad, Season 1, Pilot
		\item {\bf On seizing opportunity.} ``There is gold in the streets just waiting for someone to come \& scoop it up.'' - Walter White, Breaking Bad, Season 5, Madrigal
		\item ``Well, technically, chemistry is the study of matter. But I prefer to see it as the study of change.''
		\item ``We're done when I say we're done.''
		\item ``To all law enforcement entities, this is not an admission of guilt.''
		\item ``I am speaking to my family now. Skyler, you are the love of my life. I hope you know that.''
		\item ``I won.''
		\item ``I have lived under the threat of death for a year now. And because of that, I've made choices.''
		\item ``All I can do is wait $\ldots$ For the cancer to come back.''
		\item ``If you believe that there's a hell.''
		\item ``My name is Walter Hartwell White. I live at 308 Negra Aroya Lane, Albuquerque, New Mexico, 87104. To all law enforcement entities, this is not an admission of guilt. I am speaking to my family now. Skyler, you are the love of my life. I hope you know that. Walter Jr., you're my big man. There are going to be some things that you'll come to learn about me in the next few days. But just know that no matter how it may look, I only had you in my heart. Goodbye.''
	\end{itemize}
	{\sc Jesse Pinkman.}
	\begin{itemize}
		\item {\bf On personal freedom.} ``This is my own private domicile \& I will not be harassed$\ldots$ bitch!'' - Jesse Pinkman, Breaking Bad, Season 3, Sunset
		\item ``Ah, like I came to you, begging to cook meth. Oh, hey, nerdiest old dude I know, you wanna come cook crystal? Please. I'd ask my diaper-wearing granny, but her wheelchair wouldn't fit in the RV.''
		\item ``Did you know that there's an acceptable level of rat turds that can go into candy bars? It's the government, jack. Even government doesn't care that much about quality. You know what is okay to put in hot dogs? Huh? Pig lips \& assholes. But I say, hey, have at it bitches 'cause I love hot dogs.''
		\item ``Some straight like you, giant stick up his ass, age what, 60? He's just gonna break bad?''
		\item ``So you do have a plan! Yeah Mr. White! Yeah Science!''
		\item ``You don't need a criminal lawyer. You need a \emph{criminal} lawyer.''
		\item ``Look$\ldots$ look, you two guys are just$\ldots$ guys, OK? Mr. White$\ldots$ he's the devil. You know, he is$\ldots$ he is smarter than you, he is luckier than you. Whatever$\ldots$ whatever you think is supposed to happen$\ldots$ I'm telling you, the exact reverse opposite of that is gonna happen, OK?''
		\item ``You got me riding shotgun to every dark anal recess of this state. It'd be nice if you clued me in a little.''
		\item ``Right on. New Zealand. That's where they made Lord of the Rings. I say we just move there, yo. I mean, you can do your art, right? Like, you can paint the local castles \& s**t. And I can be a bush pilot.''
		\item ``Look, I like making cherry product, but let's keep it real, alright? We make poison for people who don't care. We probably have the most unpicky customers in the world.''
		\item {\bf On keeping promises.} ``Oh well, heil Hitler, bitch. And let me tell you something else. We flipped a coin, okay? You \& me. You \& me! Coin flip is sacred! Your job is waiting for you in that basement, as per the coin!'' - Jesse Pinkman, Breaking Bad, Season 1, And the Bag's in the River
		\item ``What good is being an outlaw when you have responsibilities?''
		\item ``Yeah, bitch! Magnets!''
		\item ``I uh$\ldots$ I eat a lot of frozen stuff$\ldots$ It's usually pretty bad, I mean the pictures are always so awesome, you know? It's like ``hell yeah, I'm starved for this lasagna!'' \& then you nuke it \& the cheese gets all scabby on top \& it's like$\ldots$ it's like you're eating a scab$\ldots$ I mean, seriously, what's that about?''
		\item ``What if this is like math, or algebra? And you add a plus douchebag to a minus douchebag, \& you get, like, zero douchebags?''
		\item ``I got two dudes that turned into raspberry slushie then flushed down my toilet. I can't even take a proper dump in there. I mean, the whole damn house has got to be haunted by now.''
		\item ``I am not turning down the money! I am turning down you! You get it? I want nothing to do with you! Ever since I met you, everything I ever cared about is gone! Ruined, turned to shit, dead, ever since I hooked up with the great Heisenberg!£
		\item ``Possum. Big, freaky, lookin' bitch. Since when did they change it to opossum? When I was comin' up it was just possum. Opossum makes it sound like he's irish or something. Why do they gotta go changing everything?''
		\item {\bf On the need for consequences.} ``The thing is, if you just do stuff \& nothing happens, what's it all mean? What's the point?'' - Jesse Pinkman, Breaking Bad, Season 4, Problem Dog
		\item ``We're all on the same page. The one that says, if I can't kill you, you'll sure as shit wish you were dead.''
		\item ``So you do have a plan! Yeah, Mr. White! Yeah, Science!''
		\item ``You're my free pass$\ldots$ bitch.''
	\end{itemize}
	{\sc Gustavo Fring.}
	\begin{itemize}
		\item {\bf On responsibilities.} ``When you have children, you always have family. They will always be your priority, your responsibility. And a man - a man provides. And he does it even when he's not appreciated, or respected, or even loved. He simply bears out \& he does it.'' - Gustavo Fring, Breaking Bad, Season 3, Más
		\item ``I hide in plain sight, same as you.''
		\item ``If you try to interfere, this becomes a much simpler matter. I will kill your wife. I will kill your son. I will kill your infant daughter.''
	\end{itemize}
	{\sc Hank Schrader.}
	\begin{itemize}
		\item {\bf On regret.} ``Been thinking about a summer job I used to have marking trees in the woods. Tagging trees is a lot better than chasing monsters.'' - Hank Schrader, Breaking Bad, Season 5, Gliding Over All
		\item ``You're the smartest guy I've ever met. And you're too stupid to see$\ldots$ he made up his mind ten minutes ago.''
		\item ``No, they're minerals, Jesus Marie!''
		\item ``My name is ASAC Schrader, \& you can go fuck yourself.''
	\end{itemize}
	{\sc Mike Ehrmantraut.}
	\begin{itemize}
		\item {\bf On hubris.} ``Just because you shot Jesse James, don't make you Jesse James.'' - Mike Ehrmantraut, Breaking Bad, Season 5, Hazard Pay
		\item {\bf On learning from failure.} ``The moral of the story is I chose a half measure when I should have gone all the way. I'll never make that mistake again. No more half measures, Walter.'' - Mike Ehrmantraut, Breaking Bad, Season 3, Half Measures
		\item {\bf On toxic relationships.} ``You are a time bomb tick, tick, ticking. And I have no intention of being around for the `boom'.'' - Mike Ehrmantraut, Breaking Bad, Season 5, Madrigal
		\item ``You are not the guy. You're not capable of being the guy. I had a guy, but now I don't. You are not the guy.''
		\item ``Shut the fuck up \& let me die in peace''
	\end{itemize}
	{\sc Skyler White.}
	\begin{itemize}
		\item {\bf On knowing when to quit.} ``There is more money here than we could spend in 10 lifetimes. Please tell me: how much is enough? How big does this pile have to be?'' - Skyler White, Breaking Bad, Season 5, Gliding Over All
		\item ``You know what Walt? Someone needs to protect this family from the man who protects this family.''
	\end{itemize}
	{\sc Saul Goodman.}
	\begin{itemize}
		\item {\bf On self-confidence.} ``I once told a woman I was Kevin Costner, \& it worked because \emph{I believed it}.'' - Saul Goodman, Breaking Bad, Season 3, Abiquiu
		\item {\bf On choosing your battles.} ``Some people are immune to good advice.'' - Saul Goodman, Breaking Bad, Season 5, Confessions
		\item {\bf On breaking the law.} ``As to your dead guy: occupational hazard. Drug dealer getting shot? I'm gonna go out on a limb here \& say it's been known to happen.'' - Saul Goodman, Breaking Bad, Season 2, Mandala
		\item ``Scientists love lasers.''
		\item ``If you're committed enough, you can make any story work.''
		\item ``Sending him on a trip to Belize.''
		\item ``The fun's over. From here on out, I'm Mr. Low Profile. Just another douche bag with a job \& three pairs of Dockers. If I'm lucky, month from now, best-case scenario, I'm managing a Cinnabon in Omaha.''
		\item ``I'm not saying it's not bad. It's bad. But it could be worse.''
		\item ``Hey, I'm a civilian! I'm not your lawyer anymore. I'm nobody's lawyer.''
		\item ``Congratulations, you've just left your family a second-hand Subaru.''
	\end{itemize}
	{\sc Jane Margolis.}
	\begin{itemize}
		\item {\bf On appreciating the little things.} ``Why should we do anything more than once? Should I just smoke this one cigarette? Maybe we should only have sex once if it's the same thing? Should we just watch one sunset? Or live just one day? Because it's new every time. Each time is a new experience.'' - Jane Margolis, Breaking Bad, Season 3, Abiquiu
	\end{itemize}
	{\sc Tuco Salamanaca.}
	\begin{itemize}
		\item ``This kicks like a mule with its balls wrapped in duct tape!''
	\end{itemize}
	{\sc Badger.}
	\begin{itemize}
		\item ``Darth Vader had responsibilities. He was responsible for the Death Star.''
	\end{itemize}
	\item {\sc Captain America: Civil War} (2016)
	\item {\sc Captain America: The First Avenger} (2011)
	\item {\sc Captain America: The Winter Soldier} (2014)
	\item {\sc Casino Royale} (2006)
	\item {\sc Catch Me If You Can} (2002)
	\item {\sc Chernobyl} (2019)\hfill[S1.E5]
	
	{\bf Boris Schcherbina.}
	\begin{itemize}
		\item ``You'll do it because it must be done. You'll do it because nobody else can. \& if you don't, millions will die. If you tell me that's not enough, I won't believe you.''
		\item ``This is what has always set people apart. A thousand years of sacrifice in our veins. \& every generation must know its own suffering.''
		\item ``Have you ever spent time with miners?''
		
		``No.''
		
		``My advice: Tell the truth. These men work in the dark they see everything.''
		\item ``The science is strong, but a test is only as good as the men carrying it out.''
		\item ``When it's your life \& the lives of everyone you love, your moral conviction doesn't mean anything.''
		\item ``You came off like a naive idiot. \& naive idiots are not a threat.''
	\end{itemize}
	{\bf Charkov.}
	\begin{itemize}
		\item ``Why worry about something that isn't going to happen?'' ``Why worry about something that isn't going to happen? Oh, that's perfect. They should put that on our money.''
	\end{itemize}
	{\bf Mikhail Gorbachev.}
	\begin{itemize}
		\item ``Well all victories come at a cost.''
		\item ``All victories inevitably come at a cost.''
		\item ``Our power comes from the perception of our power.''
	\end{itemize}
	{\bf Valery Legasov.}
	\begin{itemize}
		\item ``The truth doesn't care about our needs or wants -- it doesn't care about our governments, our ideologies, our religions -- to lie in wait for all time. This, at last, is the gift of Chernobyl.''
		\item ``What is the cost of lies? It's not that we'll mistake them for the truth. The real danger is that if we hear enough lies, then we no longer recognize the truth at all.''
		\item ``There was nothing sane about Chernobyl. What happened there, what happened after, even the good we did, all of it $\ldots$ all of it, madness.''
		\item ``If we don't find out how this happened, it will happen again.''
		\item ``Where I once would fear the cost of truth, now I only ask, what is the cost of lies?''
		\item ``You think the right question will get you the truth? There is no truth.''
		\item ``It means the core is open. It means the fire we're watching with our own eyes is giving nearly twice the radiation released by the bomb in Hiroshima. \& that's every single hour. Hour after hour, 20 hours since the explosion, so 40 bombs worth by now. 48 more tomorrow. \& it will not stop. Not in a week, not in a month. It will burn \& spread its poison until the entire continent is dead!''
		\item ``To be a scientist is to be naive. We are so focused on our search for truth, we fail to consider how few actually want us to find it. But it is always there, whether we see it or not, whether we choose to or not.''
		\item ``Every lie we tell incurs a debt to the truth. Sooner or later, that debt is paid.''
		\item ``When the truth offends, we lie \& lie until we can no longer remember it is even there but it is still there.''
		\item ``In a just world, I'd be shot for my lies, but not for this, not for the truth.''
	\end{itemize}
	{\bf Uncategorized.}
	\begin{itemize}
		\item ``You know the old Russian proverb: `Trust But Verify'. \& the Americans think that Ronald Reagan thought that up.''
		\item ``You scientists, when there's a disease you're in a lab somewhere, with your nose buried in a book. But when there's not a disease, you're out in public, causing a panic.''
		\item ``We live in a world where children have to die to save their mothers $\ldots$ someone has to start telling the truth.''
		\item ``You put a bullet in someone, you are not you anymore. You will never be you again. But then you wake up the next morning \& you are still you. \& you realize, that was you all along.''
	\end{itemize}
	See also, e.g., \href{https://en.wikiquote.org/wiki/Chernobyl_(miniseries)}{Wikiquote{\tt/}Chernobyl (miniseries)}, \href{https://tvtropes.org/pmwiki/pmwiki.php/Quotes/Chernobyl}{tvtropes{\tt/}quotes{\tt/}Chernobyl}.
	\item {\sc Coherence} (2013)
	\item {\sc Cosmos: A Spacetime Odyssey} (2014)\hfill[S1.E13]
	\item {\sc Creed} (2015)
	\item {\sc Creed II} (2018)
	\item Dahmer - Monster: The Jeffrey Dahmer Story (2022)\hfill[S1.E2{\tt/}E10]
	\item {\sc Daredevil} (2015--2018)\hfill[S1.E13][S2.E13][S3.E13]
	\item {\sc Dark} (2017--2020)\hfill[S1.E10][S2.E8][S3.E8]
	
	{\sc Adam.}
	\begin{itemize}
		\item ``Only when we've freed ourselves of emotion can we be truly free. Only when you're willing to sacrifice what you hold dearest.''
		\item ``What we know is a drop. What we do not know$\ldots$ is an ocean.'' - Adam (Jonas)
		\item ``Life is a labyrinth. Some wander around until their death in search of a way out of it.'' - Adam (Jonas)
		\item ``Death is incomprehensible. But one can reconcile themselves with it.'' - Adam (Jonas)
		\item ``All is cause \& effect.'' - Adam (Jonas)
		\item ``Every pain tends us to act, forms our will.'' - Adam (Adam)
		\item ``Life is a gift$\ldots$ for those who know how to use it.'' - Adam (Jonas)
		\item ``The end is the beginning, \& the beginning is the end.''
		\item ``A man$\ldots$ lives three lives.
		
		The first, ends with the loss of naivety,
		
		the second, with the loss of innocence
		
		and the third$\ldots$ with the loss of life itself.
		
		It's inevitable that we go through all three stages.''
	\end{itemize}
	{\sc Eva.}
	\begin{itemize}
		\item ``But in the end, every death is just a new beginning.'' - Eva (Martha)
		\item ``The mistake in all of our thinking is that we each believe ourselves to be an independent entity. While in reality, we're all just fractions of an infinite whole.'' - Eva (Martha)
	\end{itemize}
	{\sc Jonas Kahnwald.}
	\begin{itemize}
		\item ``Two days ago I kissed my aunt!!!''
		\item ``The big things \& the little things don't follow the same rules. We won't be able to change the grand scheme of things$\ldots$ but the details. We change a grain of sand, \& with that, the whole world.''
		\item ``You \& I are made for each other, never believe anything else.'' - Jonas \& Martha
	\end{itemize}
	{\sc Martha Nielsen.}
	\begin{itemize}
		\item ``We all face the same end. Those above have long forgotten us. They do not judge us. In death I am all alone, \& my only judge$\ldots$ is me.''
	\end{itemize}
	{\sc Noah.}
	\begin{itemize}
		\item ``Most people are nothing but pawns on a chessboard led by an unknown hand.''
		\item ``We are all full of sin. No pure human being exists. But no matter what we do, we never fall any lower than into God's hands.''
		\item ``But every now \& then it's good to question those who question things.''
		\item ``Fear is the worst enemy of progress.''
		\item ``Your pain defines who you are but it no longer holds power on you.''
		\item ``There's nothing but chaos out there.''
		\item ``Life is nothing but a spiral of pain.''
		\item ``There was this sadness in his eyes. The kind you sometimes see in those who want to die, but life won't let them.''
	\end{itemize}
	{\sc Claudia.}
	\begin{itemize}
		\item ``There are moments when we must understand that the decisions we make influence more than just our own fate.''
	\end{itemize}
	{\sc Mikkel Nielsen.}
	\begin{itemize}
		\item ``Things only change when we change them. But you have to do it.''
		\item ``There is no such thing as magic, just illusion. Things only change when we change them. But you have to do it skillfully, in secret. Then it seems like magic.''
		\item ``Good \& evil are a question of perspective.''
	\end{itemize}
	{\sc H.G. Tannhaus.}
	\begin{itemize}
		\item``Our thinking is shaped by dualism. Entrance, exit. Black, white. Good, evil. Everything appears as opposite pairs. But that's wrong.''
		\item ``There are things out there that our little minds will never comprehend.''
		\item ``What if everything that came from the past was influenced by the future.''
		\item ``Dreams change. Other things become important.''
	\end{itemize}
	{\sc Old Tannhaus's Father.}
	\begin{itemize}
		\item ``But everything that once lived, lives on forever. In the eternity of time.''
	\end{itemize}
	{\sc Michael Kahnwald.}
	\begin{itemize}
		\item ``The truth is a strange thing. You can try to suppress it, but it will always find its way to the surface.''
	\end{itemize}
	{\sc The Stranger.}
	\begin{itemize}
		\item ``In the end, we will all get just what we deserve.''
		\item ``If I now change my past, I will change who I am right now.''
		\item ``We trust that time is linear. That it proceeds eternally, uniformly. Into infinity. But the distinction between past, present \& future is nothing but an illusion. Yesterday, today \& tomorrow are not consecutive, they are connected in a never-ending circle. Everything is connected.''
		\item ``I'm hoping that by tomorrow, it'll already be different from today.''
		\item ``We're not free in what we do, because we're not free in what we want. We can't overcome what's deep within us.''
	\end{itemize}
	{\sc Eva's Son.}
	\begin{itemize}
		\item ``Hell is empty \& all devils are here.''
	\end{itemize}
	{\sc Helge Doppler.}
	\begin{itemize}
		\item ``Tick Tack. Tick Tack.''
	\end{itemize}
	{\sc Albert Einstein.}
	\begin{itemize}
		\item ``The distinction between past, present, \& future is only a stubbornly persistent illusion.''
	\end{itemize}
	
	{\bf Adam.}
	\begin{itemize}
		\item ``Only when we've freed ourselves of emotion can we be truly free. Only when you're willing to sacrifice what you hold dearest.''
		\item ``Man is a strange creature. All his actions are motivated by desire, his character forged by pain. As much as he may try too suppress that pain, to repress the desire, he cannot free himself from the eternal servitude to his feelings. For as long as the storm rages within him, he cannot find peace. Not in life, not in death. \& so he will do what he must, day in, day out. The pain is his vessel, desire his compass. It is all that man is capable of.''
		\item ``A man lives 3 lives. The 1st one ends with the loss of naivety, the 2nd, with the loss of innocence \& the 3rd $\ldots$ with the loss of life itself. It's inevitable that we go through all 3 stages.''
		\item ``No matter how much we fight it we are connected by our blood. We can feel estranged from our families \& not understand what they do. \& still, in the end we will do anything for them.''
	\end{itemize}
	{\bf Claudia.}
	\begin{itemize}
		\item ``There are moments when we must understand that the decisions we make influence more than just our own fates.''
	\end{itemize}
	{\bf H.G. Tannhaus.}
	\begin{itemize}
		\item ``What if everything that came from the past was influenced by the future.''
		\item ``Our thinking is shaped by dualism. Entrance, exit. Black, white. Good, evil. Everything appears as opposite pairs. But that's wrong.''
		\item ``Black holes are considered to be the hellmouths of the universe. Those who fall inside disappear. Forever. But where to? What lies behind a black hole? Along with things, do space \& time also vanish there? Or would space \& time be tied together \& be part of an endless cycle? What if everything that came from the past were influenced by the future?''
	\end{itemize}
	{\bf Ines Kahnwald.}
	\begin{itemize}
		\item ``Have you heard of Master Zhuang's paradox? `I dreamt I was a butterfly. Now I've woken up \& I no longer know if I'm a person who dreamed he's a butterfly or if I'm a butterfly who's dreaming it's a person.''
	\end{itemize}
	{\bf Jonas.}
	\begin{itemize}
		\item ``2 days ago I kissed my aunt!!!''
		\item ``You \& I are perfect for each other. Never believe anything else.''
	\end{itemize}
	{\bf Martha.}
	\begin{itemize}
		\item ``We all face the same end. Those above have long forgotten us. They do not judge us. In death I am all alone, \& my only judge $\ldots$ is me.''
	\end{itemize}
	{\bf Michael.}
	\begin{itemize}
		\item ``The truth is a strange thing. You can try to suppress it, but it will always find its way to the surface. We make a lie into our truth in order to survive. We try to forget, until we can't anymore. We don't even know half of the mysteries of this world. We're wanderers in the darkness.''
	\end{itemize}
	{\bf Mikkel Nielsen.}
	\begin{itemize}
		\item ``There is no such thing as magic, just illusion. Things only change when we change them. But you have to do it skillfully, in secret. Then it seems like magic.''
		\item ``Good \& evil are a question of perspective.''
	\end{itemize}
	{\bf Noah.}
	\begin{itemize}
		\item ``There was this sadness in his eyes. The kind you sometimes see in those who want to die, but life won't let them.''
		\item ``But every now \& then it's good to question those who question things.''
		\item ``Most people are nothing but pawns on a chessboard led by an unknown hand.''
		\item ``We are all full of sin. No pure human being exists. But no matter what we do, we never fall any lower than into God's hands.''
		\item ``Your pain defines who you are but it no longer holds power on you.''
		\item ``God doesn't have a plan. There is no plan at all. There's nothing but chaos out there. Pain $\ldots$ \& chaos! People are bad. Malicious, evil. Life is nothing but a spiral of pain. \& the world is doomed to be destroyed.''
	\end{itemize}
	{\bf The Stranger.}
	\begin{itemize}
		\item ``We trust that time is linear. That it proceeds eternally, uniformly. Into infinity. But the distinction between past, present \& future is nothing but an illusion. Yesterday, today \& tomorrow are not consecutive, they are connected in a never-ending circle. Everything is connected.''
		\item ``We're not free in what we do because we're not free in what we want. We can't overcome what's deep within us.''
		\item ``If I now change my past, I will change who I am right now.''
		\item ``But every decision for something is a decision against something else.''
	\end{itemize}
	{\bf Uncategorized.}
	\begin{itemize}
		\item ``There are things out there that our little minds will never comprehend.''
		\item ``What we know is a drop. What we don't know is an ocean.''
		\item ``The end is the beginning, \& the beginning is the end.''
		\item ``Everything is connected to everything else.''
		\item ``Fear is the worst enemy of progress.''
		\item ``Dreams change. Other things become important.''
		\item ``The distinction between past, present \& future is only a stubbornly persistent illusion.''
		\item ``Tick Tack. Tick Tack.''
		\item ``The big things \& the little things don't follow the same rules. We won't be able to change the grand scheme of things $\ldots$ but the details. We change a grain of sand, \& with that, the whole world.''
		\item ``Maybe the Big Bang is nothing more than God's act of creation.''
		\item ``God has a plan for every human being.''
		\item ``Nothing is normal in this town. \& we're all a part of it.''
		\item ``That everything's repeating That this has all happened before Like a massive d\'eja vu.''
		\item ``Some people wander around their whole lives looking for a way out, but there's only 1 path \& it leads you ever deeper.''
		\item ``What if God doesn't know what he's doing? If the plan is wrong? If God is wrong?''
		\item ``You came in the door like thunder. Then hit the floor like thunder. Laying me down you wonder. Shaking the walls like thunder.''
		\item ``My only aim is to take many lives. The more the better I feel.''
		\item ``The world is full of such paradoxes we simply choose to ignore them most of the time.''
		\item ``In the end we will all get just what we deserve.''
		\item ``In short, the god mankind has prayed to for thousands of years the god that everything is bound with, this god exists as nothing other than time itself.''
		\item ``Death is always inevitable. Destiny is nothing but the connection of cause \& effect. In light, in shadow.''
		\item ``I'm just a tiny section of a huge tumor that is much bigger than any of us can imagine.''
		\item ``There are moments when we must understand that the decisions we make influence more than just our own fates.''
		\item ``We're wanderers in the darkness.''
		\item ``The pain is his vessel, desire his compass. It is all that man is capable of.''
		\item ``Fold out your hands. Give me a sign. Put down your lies. Lay down next to me. Don't listen when I scream. Bury your doubts \& fall asleep. For neither ever. Nor never.''
		\item ``Time is God.''
	\end{itemize}	
	\item {\sc Dead Poets Society} (1989)
	\begin{itemize}
		\item ``So avoid using the word `very' because it's lazy. A man is not very tired, he is exhausted. Don't use very sad, use morose. Language was invented for 1 reason, boys -- to woo women -- \&, in that endeavor, laziness will not do. It also won't do in your essays.'' --  N.H. Kleinbaum, Dead Poets Society
		\item ``We don't read \& write poetry because it's cute. We read \& write poetry because we are members of the human race. \& the human race is filled with passion. \& medicine, law, business, engineering, these are noble pursuits \& necessary to sustain life. But poetry, beauty, romance, love, these are what we stay alive for. To quote from Whitman, ``O me! O life! $\ldots$ of the questions of these recurring; of the endless trains of the faithless $\ldots$ of cities filled with the foolish; what good amid these, O me, O life?'' Answer. That you are here - that life exists, \& identity; that the powerful play goes on \& you may contribute a verse. That the powerful play \emph{goes on} \& you may contribute a verse. What will your verse be?'' --  N.H. Kleinbaum, Dead Poets Society
		\item ``No matter what anybody tells you, words \& ideas can change the world.'' -- Tom Schulman, Dead Poets Society
		\item ``When you read, don't just consider what the author thinks, consider what you think.'' --  Tom Schulman, Dead Poets Society: The Screenplay
		\item ``If you listen real close, you can hear them whisper their legacy to you. Go on, lean in. Listen, you hear it? -- Carpe -- hear it? -- Carpe, Carpe Diem, seize the day boys, make you lives extraordinary.'' -- N.H. Kleinbaum, Dead Poets Society
		\item ``Only in their dreams can men be truly free. `Twas always thus, \& always thus will be.'' -- Tom Schulman, Dead Poets Society
		\item ``I close my eyes, \& this image floats beside me.
		
		A sweaty toothed mad man with a stare that pounds my brain.
		
		His hands reach out \& choke me, \& all the time he's mumbling.
		
		``Truth, truth.''
		
		Like a blanket that always leaves your feet cold.
		
		You push it, stretch it, but it'll never be enough.
		
		You kick at it, beat it, it'll never cover any of us.
		
		From the moment we enter crying,
		
		to the moment we leave dying,
		
		it'll just cover your face,
		
		as you wail \& cry \& scream.'' -- Tom Schulman, Dead Poets Society
		\item ``Sucking the marrow out of life doesn't choking on the bone.'' -- Tom Schulman, Dead Poets Society
	\end{itemize}
	\item {\sc Deadpool} (2016)
	\item {\sc Deadpool 2} (2018)
	\item {\sc Deadpool \& Wolverine} (2024)
	\item Detachment (2011)
	
	{\bf Henry Barthes.}
	\begin{itemize}
		\item ``Whatever is on my mind, I say it as I feel it, I'm truthful to myself; I'm young \& I'm old, I've been bought \& I've been sold, so many times. I am hard-faced, I am gone. I am just like you.''
		\item ``How are you to imagine anything if the images are always provided for you?''
		\item ``Doublethink. To deliberately believe in lies, while knowing they're false.''
		\item DOUBLETHINK is on the blackboard, from Orwell's ``1984''. When none of the students knows what it means he tells them
		
	    It's deliberately believing in lies while knowing they're false.
		\item ``Examples of this in everyday life: ``Oh, I need to be pretty to be happy. I need surgery to be pretty. I need to be thin, famous, fashionable.'' Our young men today are being told that women are whores, bitches, things to be screwed, beaten, shit on, \& shamed. This is a marketing holocaust. 24 hours a day for the rest of our lives, the powers that be are hard at work dumbing us to death.''
		\item ``So to defend ourselves, \& find against assimilating this dullness into our thought processes, we must learn to read. To stimulate our own imagination, to cultivate our own consciousness, our own belief systems. We all need skills to defend, to preserve our own minds.''
		\item ``I realized something. I'm a non-person, Sarah. You shouldn't be here, I'm not here. You may see me, but I'm hollow.''
		\item ``A child's intelligent heart can fathom the depth of many dark places, but can it fathom the delicate moment of its own detachment?''
		\item ``We have such a responsibility to guide our young so that they don't end up falling apart, falling by the wayside, becoming insignificant.''
		\item ``I am money, I change hands like the dollar bill, that has been rubbed by a lamp; Then a genie appeared \& cried loudly, with volume; But the tears were all for myself, \& that's where it all went wrong.''
		\item ``Y'know it's funny, I spend a lot of time trying to not have to deal $\ldots$ to not really commit. I'm a substitute teacher, there's no real responsibility to teach. Your responsibility is to maintain order, make sure nobody kills anybody in your classroom, \& then they get to their next period.''
		\item Henry Barthes: Y'know you can't $\ldots$ you can't keep living on the street $\ldots$
		
		Erica: I'm not, I mean, I'm staying here with you.
		
		Henry Barthes: Well, you can't continue to stay here with me. I'm not good for you $\ldots$
		
		Erica: That's not true. You're like, the only family I've ever had $\ldots$
		
		Henry Barthes: Well, I can't be your family, I can't give you what you need. You have to understand, you should be $\ldots$
		
		Erica: You're good \& gentle, you're the most kind $\ldots$ I love you Henry. Don't let them take me, please nooooooooo, you're all I have, please don't let me go.
		
		[Social workers take her away still protesting]
		\item It doesn't take strength Meredith, you've gotta understand that, unfortunately, most people lack self awareness.
		\item Henry Barthes: [In nursing home] Grampa, you doing any writing in your journal?
		
		[Thumbs through empty journal]
		
		Grampa: I don't remember much, I lost the habit. You can't think in this place, you can't make new memories.
		\item
		
		[agitated at assisted living nurse] ``Let me be very clear here, you stop neglecting his needs, or I will start fucking with yours! I will have you fired! Then it's going to be your family! Your children are gonna be at risk! You got it?
	\end{itemize}
	{\bf Meredith.}
	\begin{itemize}
		\item ``Suicide is a permanent solution to a temporary problem.''
	\end{itemize}
	{\bf Mr. Wiatt.}
	\begin{itemize}
		\item ``I was in my room for 2 hours \& saw 1 parent. Where are they? Where is everybody? It's uncanny, no air raid sirens, not bombs. It doesn't happen that way. It starts with a whisper, \& then nothing.''
	\end{itemize}
	\item {\sc Dexter} (2006--2013)\hfill[S1.E12][S2.E12][S3.E12][S4.E12][S5.E12][S6.E12][S7.E12][S8.E12]
	\item {\sc District 9} (2009)
	\item {\sc Divergent} (2014)
	\item {\sc Django Unchained} (2012)
	\item {\sc Doctor Strange} (2016)
	\item {\sc Doctor Strange in the Multiverse of Madness} (2022)
	\item {\sc Donnie Darko} (2001)
	\item {\sc Edge of Tomorrow} (2014)
	\item {\sc Everything Everywhere All at Once} (2022)
	\item {\sc Fight Club} (1999)
	\item {\sc Finding Forrester} (2000)
	
	{\bf William Forrester}: ``You don't know a goddamn thing about reason. There are no reasons! Reasons why some of us live \& why some of us don't! Fortunately for you, you have decades to figure that out!''
	
	{\bf Jamal Wallace}: ``Yeah, \& what's the reason in having a file cabinet full of writing \& keep the shit locked so nobody can read it? What is that man? I'm done with this shit.'' [William \& Jamal are arguing \& William throws a glass against wall \& breaks it]
	
	{\bf Jamal Wallace}: `Women will sleep with you if you write a book?''
	
	{\bf William Forrester}: ``Women will sleep with you if you write a bad book.''
	
	{\bf William Forrester.}
	\begin{itemize}
		\item ``Someone I once knew wrote that we walk away from our dreams afraid that we may fail or worse yet, afraid we may succeed. You need to know that while I knew so very early that you would realize your dreams, I never imagined I would once again realize my own.''
		\item ``Why is it that the words that we write for ourselves are always so much better than the words we write for others?''
		\item ``Writers write things to give readers something to read.''
		\item ``No thinking -- that comes later. You must write your 1st draft with your heart. You rewrite with your head. The 1st key to writing is $\ldots$ to write, not to think!''
		\item ``The key to a woman's heart is an unexpected gift at an unexpected time.''
		\item ``An expression is worth a thousand words. Perhaps in your case, just two.''
	\end{itemize}	
	{\bf Jamal Wallace.}
	\begin{itemize}
		\item ``The rest of those who have gone before us cannot steady the unrest of those to follow.''
		\item ``Be sure to write.''
	\end{itemize}
	\item {\sc Forrest Gump} (1994)
	\item {\sc Fracture} (2007)
	\item {\sc Fury} (2014)
	\item {\sc Game of Thrones} (2011--2019)\hfill[S1.E10][S2.E10][S3.E10][S4.E10][S5.E10][S6.E10][S7.E7][S8.E6]
	\item {\sc Get Out} (2017)
	\item {\sc Gisaengchung $\star$ Parasite} (2019)
	\item {\sc Gladiator} (2000)
	\item {\sc Gone Baby Gone} (2007)
	\item {\sc Gone Girl} (2014)
	\item {\sc Good Will Hunting} (1997)
	\item {\sc Gravity} (2013)
	\item {\sc Green Book} (2018)
	\item {\sc Guardians of The Galaxy} (2014)
	\item {\sc Guardians of The Galaxy Vol. 2} (2017)
	\item {\sc Hachi: A Dog's Tale} (2009)
	\item {\sc Hacksaw Ridge} (2016)
	\item {\sc Hannibal} (2013--2015)\hfill[S1.E13][S2.E13][S3.E13]
	\item {\sc Harry Potter \& The Chamber of Secrets} (2002)
	\item {\sc Harry Potter \& The Deathly Hallows: Part 1} (2010)
	\item {\sc Harry Potter \& The Deathly Hallows: Part 2} (2011)
	\item {\sc Harry Potter \& The Goblet of Fire} (2005)
	\item {\sc Harry Potter \& The Half-Blood Prince} (2009)
	\item {\sc Harry Potter \& The Order of the Phoenix} (2007)
	\item {\sc Harry Potter \& The Prisoner of Azkaban} (2004)
	\item {\sc Harry Potter \& The Sorcerer's Stone} (2001)
	\item {\sc I Am Legend} (2007)
	\item {\sc Inception} (2010)
	\item {\sc Inglourious Basterds} (2009)
	\item {\sc Inside Man} (2006)
	\item {\sc Interstellar} (2014)
	\item {\sc Iron Man} (2008)
	\item {\sc Iron Man 2} (2010)
	\item {\sc Iron Man 3} (2013)
	\item {\sc John Wick} (2014)
	\item {\sc John Wick: Chapter 2} (2017)
	\item {\sc John Wick: Chapter 3 - Parabellum} (2019)
	\item {\sc Joker} (2019)
	\item {\sc L\'eon: The Professional} (1994)
	\item {\sc Limitless} (2011)
	\item {\sc Logan} (2017)
	\item {\sc Lucy} (2014)
	\item {\sc Mad Max: Fury Road} (2015)
	\item Marguerite's Theorem $\star$ Marguerite's Theorem (2023)
	\item {\sc Memento} (2000)
	\item {\sc Mindhunter} (2017--2019)\hfill[S1.E10][S2.E9]
	\item {\sc Minority Report} (2002)
	\item {\sc Monster} (2004--2010)\hfill[S1.E74]
	\item Moon Knight (2022--)\hfill[S1.E6{\tt/}E6]
	\item {\sc Moonlight} (2016)
	\item {\sc Mr. Nobody} (2009)
	\item {\sc Mystic River} (2003)
	\item Narcos (2015--2017)\hfill[S1.E1{\tt/}E10][S2.E1{\tt/}E10][S3.E1{\tt/}E10]
	\item Naui Haebangilji $\star$ My Liberation Diary{\tt/}My Liberation Notes (2022--)\hfill[S1.E16]
	
	For its quotes categorized by topics, not by characters, see, e.g., \href{https://korean-binge.com/2022/04/12/100-quotes-from-my-liberation-notes/}{Korean Binge{\tt/}100+ Quotes From ``My Liberation Notes''}	
	
	{\bf Bo-Ram.}
	\begin{itemize}
		\item ``I think you're unhappy because you don't know how good you are.''
		\item ``There will be people who frustrate you anywhere you go, \& those people will never change. Then that means it's me who has to change.''
	\end{itemize}
	{\bf Cho Tae-Hun.}
	\begin{itemize}
		\item ``I found out why I was suffering, but other than that $\ldots$''
	\end{itemize}
	{\bf Choi Jun-Ho.}
	\begin{itemize}
		\item ``What do you do in the Liberation Club? What are you being liberated from? Work?'' -- Choi Jun-Ho
		
		``From people. From tedious people.'' -- Yeom Mi-Jeong
	\end{itemize}
	{\bf Ji Hyeon-A.}
	\begin{itemize}
		\item ``I only feel alive when I exhaust myself completely. If I have energy left, I feel heavy.''
		\item ``I once read a book about how to be a good writer to become a writer, \& it is said that a good drama is one where the main character tries hard to achieve something but can't do it. So I gave up.''
		\item ``Why would I write something that's like life? It's so boring.''
		\item ``There are a lot of crazy jerks who do bad things \& blame you for it.''
		\item ``You could use a little self-awareness. Everyone else knows you except for you.''
		\item ``Once you reach a certain point, you start playing with your words. \& once you start to enjoy drawing attention to yourself with your words, there's no turning back.''
		\item ``I like that you don't try to get attention from people with your words. That's why each \& every word you utter is so special.''
	\end{itemize}
	{\bf Kwak Hye-Suk.}
	\begin{itemize}
		\item ``\& it's not like I have any days off. It's 365 days a year.''
		\item ``Fate is nothing more than a person's outlook on life.''
	\end{itemize}
	{\bf Mr. Gu.}
	\begin{itemize}
		\item ``Do you want a part-time job?'' -- Mr. Gu
		
		``What kind of part-time job? Cleaning?'' -- Yeom Mi Jeong
		
		``No.'' -- Mr. Gu
		
		``Then what?'' -- Yeom Mi Jeong
		
		``Listening to me talk.'' -- Yeom Mi Jeong
		\item ``I get irritable when I'm in places with a lot of people. Even someone sitting alone at a table next to mine in a cafe irritates me.''
		\item ``You should get paid to listen to someone else talking.''
		\item ``How have you been? Have you managed to liberate yourself?''
		\item ``I'm not sad at all. But why do I keep tearing up?''
		\item ``Does anyone live without pretending?''
		\item ``The weaker you are, the more evil you get.''
		\item ``That's how life is. It seems to go well \& then stabs you in the back. Did you think it was always going to be peachy?''
		\item ``Just a few seconds ago, they thought dying was the only way out \& jumped. But in just a few seconds, that feeling started to feel like nothing.''
		\item ``Misfortunes should come in small doses, but you keep stopping them \& making them bigger.''
		\item ``Women with sharp instincts can be scary.''
		\item ``Women always asks for things like I owe them something.''
		\item ``I was practically a dead man walking. I was barely alive. But you saved my life by stabbing me in the back.''
		\item ``When I drink, it feels like the puzzle pieces that have been floating around in my head fall into place.''
		\item ``You should know who you are.''
		\item ``When I am drunk, I'm more human than I am when I'm sober.''
		\item ``It felt as if I was walking out of my own grave. Completely hopeless.''
		\item ``Why do you act like you've done something wrong when you're asking for what's rightfully yours?''
		\item ``Let's go together. Step by step, trudging on.''
	\end{itemize}
	{\bf Mr. Park.}
	\begin{itemize}
		\item ``The feeling of holding something in your heart. My liberation.''
		\item ``So why do I keep looking at my watch? I think I feel a compulsion to live a productive day, but there's not much to show for it. I'm just constantly looking at my watch \& being chased by time.''
		\item ``I go to work, finish work, eat, \& sleep. Everyday is the same.''
		\item ``Do not give advice. Do not try to comfort. Those are the rules of our club.''
		\item ``I may not be able to be completely liberated from time, but resting when I've done enough, \& waking up when I've slept enough. Finding my own rhythm like that might be the liberation I need the most.''
		\item ``Are we under special care or something? Can't they just leave introverted people alone?''
	\end{itemize}
	{\bf Office Staff.}
	\begin{itemize}
		\item ``No matter what job is, after 6 months, it becomes mundane. But it's a lot better when you get along with others. It also increases productivity.''
	\end{itemize}
	{\bf Oh Du-Hwan.}
	\begin{itemize}
		\item ``Deleting it won't make me forget what happened. I just have to bear it.''
	\end{itemize}
	{\bf So Hyang-Gi.}
	\begin{itemize}
		\item ``We can't exactly say we accomplish nothing, don't you think? Well, some days, I feel like I am. \& some days, I feel like I'm back to square 1. But I still feel I've been liberated even just a little.''
		\item ``When I see someone in front of me, my face automatically makes this expression. Even though I'm not happy at all.''
	\end{itemize}
	{\bf The Liberation Club.}
	\begin{itemize}
		\item ``I will not pretend to be happy. I will not pretend to be unhappy. I will be honest.''
	\end{itemize}
	{\bf Yeom Chang-Hee.}
	\begin{itemize}
		\item ``It might not be because they're shameless. It might be because they don't have the money.''
		\item ``I don't think I was a 1-coin won. I think I've been that mountain all along. I think I'll return to that mountain.''
		\item ``That's why they say you can get away with anything if you have love.''
		\item ``Do I have to have a goal? Can't I just live my life without one? I can't force myself to live for something I don't really desire.''
		\item ``I think I've come far enough. This isn't the right path for me. I don't have to force myself to keep walking on it.''
		\item ``You've worked really hard. Take a break. Can't you just say that?''
		\item ``Nothing goes my way.''
		\item ``I guess I have been putting up a facade with people. Now that I'm alone, I've become so calm \& gentle.''
		\item ``Life is a series of embarrassments. It's embarrassing from the moment you're born. You are born naked.''
		\item ``When you desperately long for something, your soul already knows deep down that it's not yours. You want it but you know it's not yours. That's what drives you crazy.''
		\item ``If you hesitate to say something but then actually say it, you're guaranteed to regret it. You hesitate because you know you shouldn't say it. But you still end up saying it \& making yourself miserable. Humans are really nonsensical animals, you know?''
		\item ``I've never felt real joy, pleasure, or excitement in my life.''
		\item ``How did you end up moving here?'' -- Yeom Chang-Hee
		
		``I got off at the wrong stop.'' -- Mr. Gu
		\item ``Do you know what she said about Gyeonggi-do? She said Gyeonggi-do's like an egg white. An egg white that wraps around Seoul.''
	\end{itemize}
	{\bf Yeom Chang-Hee's friend.}
	\begin{itemize}
		\item ``I wish I could work in an office. I want to sit inside a building where it's quiet even if there's a thunderstorm going on outside.''
	\end{itemize}
	{\bf Yeom Gi-Jeong.}
	\begin{itemize}
		\item ``How do you know what suits you better? You have to try different workplaces.''
		\item ``This comes \& goes in cycles. 3 days of the week are so tiring, the other 3 are just barely manageable, \& I don't even know how the last day goes.''
		\item ``There are just too many people. So it takes forever for my turn to come. I can't get anything when I want them. I have to wait for everything. For food, to get back home, \& even men.''
		\item ``I think I would be liberated if I shaved my head.''
		\item ``Let's run away. Telling myself that, I got on the train in a hurry.''
		\item ``I want to face the mountains 1 by 1 instead of avoiding them.''
		\item ``If I die, it will be because of commuting to Seoul for work.''
		\item ``I'm hungry but there's nothing I want to eat.''
		\item ``I've been so impatient lately. I just want to die already. After 14 years, my job is the same, the meetings are the same, \& the people are the same. I curse \& get mad the same way. It's all the same endless repetition.''
		\item ``You can afford to have a good outlook if you have money.''
		\item ``The saying that love makes you kind has some truth to it. Whether it's money or a man, if you have something, you become positive but do I have a man or any money? Neither.''
		\item ``I don't want to just say things to feel like I exist. I want to talk to relax. Words that make you feel relaxed.''
	\end{itemize}
	{\bf Yeom Gi-Jeong's friend.}
	\begin{itemize}
		\item ``Even if you live alone, it's fine. You can be perfectly happy. You can eat whatever you want, whenever you want, \& sleep whenever you want.''
	\end{itemize}
	{\bf Yeom Je-Ho.}
	\begin{itemize}
		\item ``Even when you don't know how you'll go onrun, if you pull yourself together, you can still find things that are bearable.''
	\end{itemize}
	{\bf Yeom Mi-Jeong.}
	\begin{itemize}
		\item ``I think that's what this is all about. Finding out what my issue is.''
		\item ``5 minutes a day. If you have 5 minutes of peace, it's bearable.''
		\item ``Even if I thought I did a good job, if they said no, all the work I did would go down the drain.''
		\item ``When I get frustrated, I go out for a walk at night thinking, `I don't care if I die tonight.' I walk through a pitch-dark mountain.''
		\item ``Tonight, I have nothing to fear. I'll become a warrior.''
		\item ``When I wander around aimlessly, I can catch the briefest glimpse. `So that's what is in my head.' The feeling of being abandoned.''
		\item ``I think humans are only sane when they're lonely. So I think I'm saner at night.''
		\item ``I'm hungry. My face is burning up. I feel like I'll collapse.''
		\item ``He may be somebody at work, but he's a nobody outside.''
		\item ``I've never felt better after getting angry. It would take me 2 to 3 days to forget about it if I didn't get angry, but if I get angry, it lasts more than 10 days.''
		\item ``When I was a child, I was asked to hand in what we had prayed about at church. Looking at what my friends wrote, I thought, `Why would they pray for that? Grades, the school they want to get into, friends. They're seriously praying for that? To God? But it's God.' There was only 1 thing I was curious about. `What am I? Why am I here?'''
		\item ``I didn't exist before 1991 \& I won't exist in 50 years, but I feel like I existed before that \& will still exist after that. The feeling that I'll exist forever. I've been frustrated by that feeling \& I've never, in my heart, ever, felt settled.''
		\item ``I feel uneasy in bed, I feel uneasy around people. `Why can't I laugh happily like other people? Why am I sad all the time? Why am I always nervous? Why is everything so boring?''
		\item ``It feels like people are all scarecrows. They don't really know what they are. They're just acting as if they do.''
		\item ``People who say they live healthily \& happily maybe the people who decided to put all these questions behind them. People who have decided to lie \& say, `This is just how life is.' I'll never do that.''
		\item ``I don't care about where I'll go after I die. I want to see heaven while I'm alive.''
		\item ``It's bizarre when you see a thing somewhere it doesn't belong. A bird lying on the ground, a man hanging on a tree, even a dog on a farm.''
		\item ``I drink to feel high.'' -- Yeom Mi-Jeong
		
		``I drink to feel calm.'' -- Mr. Gu
		\item ``Why am I feeling sad? Why am I sad?''
		\item ``Then it's not food you want.''
		\item ``Out of the 24 hours in a day, I only feel okay for about a couple. \& it's not like I even feel good, I just feel okay. I just try to get through the rest.''
		\item ``I want liberation. I want to be liberated. I don't know where I'm trapped but I feel trapped. There's nothing in my life that relaxes me. I feel cramped \& stifled. I want to break free.''
		\item ``I'm not unhappy but I'm not happy either.''
		\item ``If you think about it, all the assholes in my life started with that same look in their eyes. Eyes that seemed to say, `You're not good enough.' It makes you feel so small. Like you're insignificant.''
		\item ``People are scared of thunder \& lightning but strangely, I find them calming.''
		\item ``It feels like I'm stuck but I don't know how to get out. That's probably why I hope everything ends all at once.''
		\item ``Everyone is on their way to their graves, so why is everyone so happy \& excited?''
		\item ``Sometimes, I think that people who are damaged are much more honest than those who live their lives happily.''
		\item ``I don't know where I'm stuck but I want to break free.''
		\item ``I wish I was genuinely happy \& be able to say things like `Yes, this is life,' `This is what life is all about.'''
		\item ``If we had lived in Seoul, would we have been any different?''
		\item ``If I imagine that I'm sitting here working next to you, even awful tasks like these turn into something beautiful. Work becomes bearable.''
		\item ``Rather than going through exhausting, difficult times without you, isn't it more admirable that I'm finding strength thinking of you?''
		\item ``No matter where I live, I think I would have been the same. I'd be living the same mundane life \& no on would ever be interested in me. I felt like if I lived like this for too long, I'd shrivel up \& die.''
		\item ``People are so good with words.''
		\item ``I want all of us to be happy. As bright \& cheery as a sunny day. Without so much as a crease in our hearts.''
		\item ``Maybe it's just me who's worth 20 points.''
		\item ``I'm exhausted. I don't know when it all started to go wrong, but I'm exhausted.''
	\end{itemize}	
	\item {\sc Nightcrawler} (2014)
	\item {\sc Nisemonogatari} (2012)\hfill[S1.E11]
	\item {\sc No Country for Old Men} (2007)
	\item No Time To Die (2021)
	\item {\sc Nocturnal Animals} (2016)
	\item {\sc Noruwei no mori $\star$ Norwegian Wood} (2010)
	\item {\sc Now You See Me} (2013)
	\item {\sc Now You See Me 2} (2016)
	\item {\sc Oblivion} (2013)
	\item {\sc Oppenheimer} (2023)
	\item {\sc Passengers} (2016)
	\item {\sc PK} (2014)
	\item {\sc Predestination} (2014)
	\item {\sc Primal Fear} (1996)
	\item {\sc Prisoners} (2013)
	\item {\sc Pulp Fiction} (1994)
	\item {\sc Quantum of Solace} (2008)
	\item {\sc Rango} (2011)
	\item {\sc Ready Player One} (2018)
	\item {\sc Red Dragon} (2002)
	\item {\sc Rush} (2013)
	\item {\sc Saving Private Ryan} (1998)
	\item {\sc Se7en} (1995)
	\item {\sc Serbuan maut $\star$ The Raid: Redemption} (2011)
	\item {\sc Serbuan maut 2: Berandal $\star$ The Raid 2} (2014)
	\item {\sc Sherlock} (2010--2017)\hfill[S1.E3][S2.E3][S3.E3][S4.E1{\tt/}E3]
	\item Shin seiki evangerion $\star$ Neon Genesis Evangelion (1995--1996)\hfill[S1.E26]
	\item {\sc Shutter Island} (2010)
	\item {\sc Sicario} (2015)
	\item {\sc Skyfall} (2012)
	\item {\sc Sorido Eopsi $\star$ Voice of Silence} (2020)
	\item {\sc Source Code} (2011)
	\item {\sc Southpaw} (2015)
	\item {\sc Spectre} (2015)
	\item {\sc Spider-Man} (2002)
	\item {\sc Spider-Man 2} (2004)
	\item {\sc Spider-Man 3} (2007)
	\item {\sc Spider-Man: Far from Home} (2019)
	\item {\sc Spider-Man: Homecoming} (2017)
	\item {\sc Spider-Man: Into the Spider-Verse} (2018)
	
	{\sc Peter B. Parker.}
	\begin{itemize}
		\item {\bf Miles Morales}: ``{\it When will I know I'm ready?}''
		
		{\bf Peter B. Parker}: ``{\it You won't. It's a leap of faith. That's all it is, Miles. A leap of faith}.''
	\end{itemize}
	\item {\sc Spider-Man: No Way Home} (2021)
	\item {\sc Split} (2016)
	\item Stranger Things (2016--)\hfill[S1.E8][S2.E9][S3.E8][S4.E]
	\item {\sc The Amazing Spider-Man} (2012)
	\item {\sc The Amazing Spider-Man 2} (2014)
	\item {\sc The Avengers} (2012)
	\item The Boys (2019--)\hfill[S1.E8][S2.E8][S3.E8][S4.E7--]
	\item {\sc The Butterfly Effect} (2004)
	\item {\sc The Curious Case of Benjamin Button} (2008)
	\item {\sc The Dark Knight} (2008)
	\item {\sc The Dark Knight Rises} (2012)
	\item {\sc The Departed} (2006)
	\item {\sc The Divergent Series: Insurgent} (2015)
	\item {\sc The Game} (1997)
	\item {\sc The Girl with The Dragon Tattoo} (2011)
	\item {\sc The Godfather} (1972)
	\item {\sc The Godfather: Part II} (1972)
	\item {\sc The Grand Budapest Hotel} (2014)
	\item {\sc The Great Gatsby} (2013)
	\item {\sc The Green Mile} (1999)
	\item {\sc The Hunger Games} (2012)
	\item {\sc The Hunger Games; Catching Fire} (2013)
	\item {\sc The Hunger Games: Mockingjay - Part 1} (2014)
	\item {\sc The Hunger Games: Mockingjay - Part 2} (2015)
	\item {\sc The Hunt} (2012)
	\item {\sc The Imitation Game} (2014)
	\item {\sc The Incredible Hulk} (2008)
	\item {\sc The Invisible Guest} (2016)
	\item {\sc The Lord of The Rings: The Fellowship of The Ring} (2001)
	\item {\sc The Lord of The Rings: The Two Towers} (2002)
	\item {\sc The Lord of The Rings: The Return of The King} (2003)
	\item {\sc The Machinist} (2004)
	\item {\sc The Martian} (2015)
	\item {\sc The Matrix} (1999)
	\item {\sc The Matrix Reloaded} (2003)
	\item {\sc The Matrix Revolutions} (2003)
	\item {\sc The Maze Runner} (2014)
	\item {\sc The Prestige} (2006)
	\item {\sc The Punisher} (2017--2019)\hfill[S1.E13][S2.E13]
	\item {\sc The Pursuit of Happyness} (2006)
	\item {\sc The Revenant} (2015)
	\item {\sc The Shawshank Redemption} (1994)
	\item {\sc The Silence of The Lambs} (1991)
	\item {\sc The Social Network} (2010)
	\item {\sc The Theory of Everything} (2014)
	\item {\sc The Usual Suspects} (1995)
	\item {\sc The Wolf of Wall Street} (2013)
	\item {\sc Thor} (2011)
	\item {\sc Thor: Ragnarok} (2017)
	\item {\sc Thor: The Dark World} (2013)
	\item The Witcher (2019--)\hfill[S1.E8][S2.E8]
	\item {\sc Time Lapse} (2014)
	\item {\sc Titanic} (1997)
	\item Tomorrow (2022)\hfill[S1.E5{\tt/}E16]
	\item {\sc Triangle} (2009)
	\item True Detective (2014--2019)\hfill[S1.E8][S2.E8][S3.E1{\tt/}E8]
	\item {\sc Warrior} (2011)
	\item West World (2016--)\hfill[S1.E10][S2.E5{\tt/}E10][S3.E1{\tt/}E8]
	\item {\sc Whiplash} (2014)
	\item {\sc Wind River} (2017)
	\item {\sc World War Z} (2013)
	\item {\sc Zodiac} (2007)
\end{enumerate}

%------------------------------------------------------------------------------%

\section{Music}

{\bf Anime Music}

\begin{enumerate}
	\item \href{https://www.youtube.com/watch?v=21nlozmIbP0}{Shatter Me [AMV] $\sim$ [SEIZURE WARNING!]}
	\item \href{https://www.youtube.com/watch?v=80fcg329zBY}{SAVAGE - AMV - [Anime MV]}
	\item \href{https://www.youtube.com/watch?v=i0Q7T_9vNNE}{Shigatsu wa Kimi no Uso - Boku to Kimi to no Ongakuchou}
	\item \href{https://www.youtube.com/watch?v=Gm-ywPzO0NA}{AMV | Tanjiro (Demon Slayer) - Gasoline (Halsey)}
	\item \href{https://www.youtube.com/watch?v=ZyCPOWdudiw}{When Giorno's theme gets stuck in your head}
	\item \href{https://www.youtube.com/watch?v=oxkgOEiE_HY}{Dynasty - AMV - [Anime MV]}
	\item \href{https://www.youtube.com/watch?v=TWEFFbh7rn0}{Violet Evergarden - Young \& Beautiful AMV}
\end{enumerate}


\begin{enumerate}
	\item {\sc 2Pac.}
	\begin{itemize}
		\item {\sc 2Pac $+$ Eminem.} Mr Lucifer.
	\end{itemize}
	\item {\sc 24kGoldn.}
	\begin{itemize}
		\item {\sc 24kGoldn $+$ iann dior.} Mood.
	\end{itemize}
	\item {\sc 7!!.} Orange [Shigatsu wa Kimi no Uso ED 2].
	\item {\sc Aaron Smith.} Dancin (KRONO Remix).
	\item {\sc Abo Takeshi.} Someday (Believe me) [Steins;Gate OST].
	\item {\sc Adam Lambert.} Runnin'. Whataya Want From Me.
	\item {\sc Adele.} Rolling In The Deep. Set Fire To The Rain. Skyfall. Someone Like You.
	\item {\sc Aerosmith.} Dream On.
	\item {\sc Ái Phương.} Tôi Thấy Hoa Vàng Trên Cỏ Xanh.
	\item {\sc Alizée.} La Isla Bonita.
	\item {\sc Akira Phan.} Điều Ước Giản Đơn. Lời Nguyền.
	\item {\sc Alan Walker.} Alone. Sing Me To Sleep. The Spectre. Unity.
	\begin{itemize}
		\item {\sc Alan Walker $+$ Ina Wroldsen.} Blue.
		\item {\sc Alan Walker $+$ K-391 $+$ Emelie Hollow.} Lily.
		\item {\sc Alan Walker $+$ K-391 $+$ Tungevaag $+$ Mangoo.} Play.
		\item {\sc Alan Walker $+$ Sofia Carson $+$ K-391 $+$ CORSAK.} Different World.
		\item {\sc Alan Walker $+$ Sophia Somajo.} Diamond Heart.
	\end{itemize}
	\item {\sc Alec Benjamin.} If We Have Each Other. Let Me Down Slowly. The Way You Felt. Water Fountain.
	\item {\sc Alessia Cara.} Scars To Your Beautiful.
	\item {\sc Allie X.} Downtown. Lifted. Paper Love.
	\item {\sc Anna Blue.} Silent Scream.
	\begin{quotation}\it
		I'm caught up in your expectations
		
		You're trying to make me live your dream
		
		But I'm causing you so much frustration
		
		And you only want the best for me
		\\
		
		You wanted me to show more interests
		
		To always keep a big bright smile
		
		Be that pinky little perfect princess
		
		But I'm not that type of child
		\\
		
		And this storm is rising inside of me
		
		Don't you feel that our whole worlds collide?
		
		It's getting harder to breathe
		
		It hurts deep inside
		\\
		
		Just let me be
		
		Who I am
		
		It's what you really need to understand
		
		And I hope so hard for the pain to go away
		\\
		
		And it's torturing me
		
		But I can't break free
		
		So I cry \& cry but just won't get it out
		
		The silent scream
		\\
		
		Tell me why you're putting pressure on me
		
		And everyday you 'cause me harm
		
		That's the reason why I feel so lonely
		
		Even though you hold me in your arms
		\\
		
		Wanna put me in a box of glitter
		
		But I'm just trying to get right out
		
		And now you're feeling so so bitter
		
		Because I've let you down
		\\
		
		And this storm is rising inside of me
		
		Don't you feel that our whole worlds collide?
		
		It's getting harder to breathe
		
		It hurts deep inside
		\\
		
		Just let me be
		
		Who I am
		
		It's what you really need to understand
		
		And I hope so hard for the pain to go away
		\\
		
		And it's torturing me
		
		But I can't break free
		
		So I cry \& cry but just won't get it out
		
		The silent scream
		\\
		
		Can't you see how I cry for help
		
		'Cause you should love me just for being myself
		
		I'll drown in an ocean
		
		Of pain \& emotion
		
		If you don't save me right away
		\\
		
		Just let me be
		
		Who I am
		
		It's what you really need to understand
		
		And I hope so hard for the pain to go away
		\\
		
		And it's torturing me
		
		But I can't break free
		
		So I cry \& cry but just won't get it out
		
		The silent scream
		
		My silent scream
	\end{quotation}	
	\item {\sc Anne-Marie.} 2002. Beautiful. Problems.
	\begin{itemize}
		\item {\sc Anne-Marie $+$ KSI $+$ Digital Farm Animals.} Don't Play.
		\item {\sc Anne-Marie $+$ Little Mix.} Kiss My (Uh Oh).
		\item {\sc Anne-Marie $+$ Nathan Dawe $+$ MoStack.} Way Too Long.
		\item {\sc Anne-Marie $+$ Niall Horan.} Everywhere (BBC Children In Need). Our Song.
	\end{itemize}
	\item {\sc Antonio Vivaldi.} Four Seasons. Storm. Winter.
	\item {\sc Arash.} Broken Angel.
	\item {\sc Ariana Grande.} 7 Rings. Breathin. Focus. Into You. One Last Time.
	
	\begin{quotation}\it
		Yeah, breakfast at Tiffany's \& bottles of bubbles
		
		Girls with tattoos who like getting in trouble
		
		Lashes \& diamonds, ATM machines
		
		Buy myself all of my favorite things (yeah)
		
		Been through some bad shit, I should be a sad bitch
		
		Who woulda thought it'd turn me to a savage?
		
		Rather be tied up with calls \& not strings
		
		Write my own checks like I write what I sing, yeah (yeah)
		
		My wrist, stop watchin', my neck is flossy
		
		Make big deposits, my gloss is poppin'
		
		You like my hair? Gee, thanks, just bought it
		
		I see it, I like it, I want it, I got it (yeah)
		
		I want it, I got it, I want it, I got it
		
		I want it, I got it, I want it, I got it
		
		You like my hair? Gee, thanks, just bought it
		
		I see it, I like it, I want it, I got it (yeah)
		
		Wearing a ring, but ain't gon' be no ``Mrs.''
		
		Bought matching diamonds for six of my bitches
		
		I'd rather spoil all my friends with my riches
		
		Think retail therapy my new addiction
		
		\textbf{Whoever said money can't solve your problems}
		
		\textbf{Must not have had enough money to solve 'em}
		
		They say, ``Which one?'' I say, ``Nah, I want all of 'em''
		
		Happiness is the same price as red bottoms
		
		My smile is beamin', my skin is gleamin'
		
		The way it shine, I know you've seen it (you've seen it)
		
		I bought a crib just for the closet
		
		Both his \& hers, I want it, I got it, yeah
		
		I want it, I got it, I want it, I got it
		
		I want it, I got it, I want it, I got it (baby)
		
		You like my hair? Gee, thanks, just bought it (oh yeah)
		
		I see it, I like it, I want it, I got it (yeah)
		
		Yeah, my receipts, be lookin' like phone numbers
		
		If it ain't money, then wrong number
		
		Black card is my business card
		
		The way it be settin' the tone for me
		
		I don't mean to brag, but I be like, ``Put it in the bag,'' yeah
		
		When you see them racks, they stacked up like my ass, yeah
		
		Shoot, go from the store to the booth
		
		Make it all back in one loop, give me the loot
		
		Never mind, I got the juice
		
		Nothing but net when we shoot
		
		Look at my neck, look at my jet
		
		Ain't got enough money to pay me respect
		
		Ain't no budget when I'm on the set
		
		If I like it, then that's what I get, yeah
		
		I want it, I got it, I want it, I got it (yeah)
		
		I want it, I got it, I want it, I got it (oh yeah, yeah)
		
		You like my hair? Gee, thanks, just bought it
		
		I see it, I like it, I want it, I got it (yeah)\hfill$\square$
	\end{quotation}
	\begin{itemize}
		\item {\sc Ariana Grande $+$ The Weeknd.} Love Me Harder.
	\end{itemize}
	\item {\sc Astrid S.} Hurts So Good.
	\item {\sc AURORA.} A Different Kind Of Human. A Little Place Called The Moon. A Temporary High. Artemis. Blood In The Wine. Churchyard. Conqueror. Cure For me. Daydreamer. Exhale Inhale. Exist For Love. Forgotten Love. Gentle Earthquakes. Giving In To The Love. Golden (Harry Styles  cover, Radio 1 Piano Sessions). Half The World Away. Heathens. Queendom. Runaway. Running With The Wolf. The River. The Seed. The Woman I Am (live at Gullruten 2022). Warrior.
	\begin{itemize}
		\item {\sc AURORA $+$ Pomme.} Everything Matters.
	\end{itemize}
	\item {\sc Ava Max.} So Am I. Sweet but Psycho.	
	\item {\sc Avril Lavigne.} When You're Gone.
	\item {\sc Axel Johansson.} Wonderland.
	\item {\sc B Ray.}
	\begin{itemize}
		\item {\sc B Ray $+$ ĐạtG $+$ Masew $+$ K-ICM.} Cao Ốc 20.
		\item {\sc B Ray $+$ Sofia $+$ Châu Đăng Khoa.} Thiêu Thân.
	\end{itemize}
	\item {\sc Backstreets Boys.} As Long As You Love Me. I Want It That Way. Show Me The Meaning Of Being Lonely.
	\item {\sc Bahjat.} Hometown Smile.
	\item {\sc Bảo Anh.} Anh Muốn Em Sống Sao. Trái Tim Em Cũng Biết Đau. Yêu Một Người Vô Tâm.
	\begin{itemize}
		\item {\sc Bảo Anh $+$ Mr Siro.} Sống Xa Anh Chẳng Dễ Dàng. Trái Tim Em Cũng Biết Đau.
	\end{itemize}
	\item {\sc Bảo Thy.}
	\begin{itemize}
		\item {\sc Bảo Thy $+$ Quang Vinh.} Ngôi Nhà Hoa Hồng.
		\item {\sc Bảo Thy $+$ Vương Khang.} Please Tell Me Why. Xin Lỗi Anh (Sorry).
	\end{itemize}
	\item {\sc Bằng Cường.}
	\begin{itemize}
		\item {\sc Bằng Cường $+$ Khánh Phương.} Tôn Thờ Một Tình Yêu.
	\end{itemize}
	\item {\sc Basshunter.} DotA.
	\item {\sc Beethoven.} Moonlight Sonata.
	\item {\sc Bella Poarch.} Build a B*tch.
	\item {\sc Bệt.} Nhẹ.
	\item {\sc Beyonc\'e.} Halo.
	\item {\sc B.I.}
	\begin{itemize}
		\item {\sc B.I $+$ DeVita.} BTBT.
	\end{itemize}
	\item {\sc Bích Phương.} Bùa Yêu. Rằng Em Mãi Ở Bên.
	\item {\sc Billie Eilish.} No Time To Die. Ocean Eyes.
	\begin{itemize}
		\item {\sc Billie Eilish $+$ Khalid.} Lovely.
	\end{itemize}
	\item {\sc BlackPink.} As If It's Your Last. Boombayah. Crazy Over You. Ddu-Du Ddu-Du. Forever Young. How You Like That. Kill this Love. Lovesick Girls. Pink Venom. Playing With Fire. Savage. Sure Thing. You Never Know. Whistle.
	\item {\sc Blue.} All Rise. One Love.
	\begin{itemize}
		\item {\sc Blue $+$ Elton John.} Sorry Seems To Be The Hardest Word.
	\end{itemize}
	\item {\sc Britney Spears.} Criminal. Everytime.
	\item {\sc Bruno Mars.} Grenade. Talking To The Moon. That's What I Like.
	\item {\sc Buitruonglinh.} Đường Tôi Chở Em Về.
	\item {\sc Camila Cabello.} Havana. Never Be The Same.
	\item {\sc Cẩm Ly.} Em Sẽ Là Người Ra Đi.
	\item {\sc Carly Rae Jepsen.} Call Me Maybe.
	\item {\sc C\'eline Dion.} Ashes. Lying Down. My Heart Will Go On.
	\item {\sc Charly Luske.} It's A Man's Man's Man's World.
	\item {\sc Charlie Puth.} Attention. Dangerously. How Long. Light Switch. One Call Away.
	\begin{itemize}
		\item {\sc Charlie Puth $+$ Selena Gomez.} We Don't Talk Anymore.
	\end{itemize}
	\item {\sc Chi Pu.} Shh! Chỉ Ta Biết Thôi (Chị Chị Em Em OST).
	\item {\sc Childish Gambino.} This Is America.
	\item {\sc Ckay.} Love Nwantiti.
	\item {\sc Clean Bandit.}
	\begin{itemize}
		\item {\sc Clean Bandit $+$ Sean Paul $+$ Anne-Marie.} Rockabye.
	\end{itemize}
	\item {\sc Coldplay.} Hymn For The Weekend. The Scientist.
	\begin{itemize}
		\item {\sc Coldplay $+$ Selena Gomez.} Let Somebody Go.
	\end{itemize}
	\item {\sc Coolio.}
	\begin{itemize}
		\item {\sc Coolio $+$ L.V.} Gangsta's Paradise.
	\end{itemize}
	\item {\sc Da LAB.} Thanh Xuân. Thức Giấc. Từ Ngày Em Đến.
	\begin{itemize}
		\item {\sc Da LAB $+$ Miu Lê.} Gác Lại Lo Âu.
	\end{itemize}
	\item {\sc Daniel Powter.} Bad Day. Free Loop.
	\item {\sc David Guetta.}
	\begin{itemize}
		\item {\sc David Guetta $+$ Sia.} Titanium.
	\end{itemize}
	\item {\sc Demi Lovato.} Cool For The Summer. Heart Attack.
	\item {\sc Dido.} Thank You.
	\item {\sc Dire Straits.} Sultans Of Swing. Walk Of Life.
	\item {\sc DJ Gimi-O.} Habibi.
	\item {\sc DJ Snake.}
	\begin{itemize}
		\item {\sc DJ Snake $+$ Selena Gomez $+$ Cardi B $+$ Ozuna.} Taki Taki.
	\end{itemize}
	\item {\sc Don Omar.}
	\begin{itemize}
		\item {\sc Don Omar $+$ Lucenzo.} Danza Kuduro.
	\end{itemize}
	\item {\sc Double Noize.}
	\begin{itemize}
		\item {\sc Double Noize $+$ CM1X $+$ Trung I.U $+$ Hồng JP.} Âm Bản.
	\end{itemize}
	\item {\sc Dr. A.}
	\begin{itemize}
		\item {\sc Dr. A $+$ Yun $+$ Vercynus.} Hồn Trôi.
	\end{itemize}
	\item {\sc Dr. Dre.}
	\begin{itemize}
		\item {\sc Dr. Dre $+$ Eminem $+$ Skylar Grey.} I Need A Doctor.
		\item {\sc Dr. Dre $+$ Snoop Dogg.} Still D.R.E.
		\item {\sc Dr. Dre $+$ Snoop Dogg $+$ Kurupt $+$ Nate Dogg.} The Next Episode.
	\end{itemize}
	\item {\sc Dua Lipa.} Homesick.
	\begin{itemize}
		\item {\sc Dua Lipa $+$ BlackPink.} Kiss \& Make Up.
	\end{itemize}
	\item {\sc Duncan Laurence.} Loving You Is A Losing Game.
	\begin{itemize}
		\item {\sc Duncan Laurence $+$ FLETCHER.} Arcade.
	\end{itemize}
	\item {\sc Đa Sắc.}
	\begin{itemize}
		\item {\sc Đa Sắc $+$ JGKiD $+$ Đen $+$ Thảo Phương.} Chạy Trốn Mặt Trời.
	\end{itemize}
	\item {\sc Đạt G.}
	\begin{itemize}
		\item {\sc Đạt G $+$ Du Uyên.} Khó Vẽ Nụ Cười.
		\item {\sc Đạt G $+$ Ngọc Haleyy.} Điều Khác Lạ (Masew Mix).
	\end{itemize}
	\item {\sc Đẳng Thậm Ma Quân.} Từ Cửu Môn Hồi Ức.
	\item {\sc Đen.} Lộn Xộn I. Lộn Xộn II. Trời Hôm Nay Nhiều Mây Cực!. 
	\begin{itemize}
		\item {\sc Đen $+$ Biên.} Cảm Ơn.
		\item {\sc Đen $+$ Chi Pu $+$ Lynk Lee.} Nếu Mình Gần Nhau.
		\item {\sc Đen $+$ Giang Pham $+$ Triple D.} Ngày Khác Lạ.
		\item {\sc Đen $+$ Linh Cáo.} Đưa Nhau Đi Trốn. Ta Cứ Đi Cùng Nhau.
		\item {\sc Đen $+$ Lynk Lee.} Đừng Gọi Anh Là Idol.
		\item {\sc Đen $+$ MIN.} Bài Này Chill Phết.
		\item {\sc Đen $+$ MTV Band.} Trốn Tìm.
		\item {\sc Đen $+$ Ngọc Linh.} Mười Năm (Lộn Xộn 3).
		\item {\sc Đen $+$ Nguyên Thảo.} Mang Tiền Về Cho Mẹ.
		\item {\sc Đen $+$ Phương Anh Đào.} Lối Nhỏ.
		\item {\sc Đen $+$ Thành Đồng.} Một Triệu Like.
		\item {\sc Đen $+$ Trần Tiến.} Đi Trong Mùa Hè.
		\item {\sc Đen $+$ Worm JB $+$ Sol'bass $+$ LongMin.} Hoàng Hôn.
	\end{itemize}
	\item {\sc Đinh Đại Vũ.} Em Đâu Hay.
	\item {\sc Đông Nhi.} Khóc.
	\item {\sc Đức Phúc.} Ánh Nắng Của Anh (Chờ Em Đến Ngày Mai OST). Còn Yêu, Đâu Ai Rời Đi. Cũng Đành Thôi. Hết Thương Cạn Nhớ. Hơn Cả Yêu. Năm Ấy. Ngày Đầu Tiên. Ta Còn Yêu Nhau.
	\item {\sc Ed Sheeran.} Perfect. Shape Of You.
	\begin{itemize}
		\item {\sc Ed Sheeran $+$ Taylor Swift.} The Joker And The Queen.
	\end{itemize}
	\item {\sc Eiffel 65.} Blue (Da Ba Dee).
	\item {\sc Ellie Goulding.} Love Me Like You Do.
	\item {\sc Ember Island.} Umbrella.
	\item {\sc Emily.}
	\begin{itemize}
		\item {\sc Emily $+$ Lil' Knight $+$ JustaTee.} Xin Anh Đừng.
	\end{itemize}
	\item {\sc Eminem.} Fall. Lose Yourself. Mockingbird. Not Afraid. Rap God. The Real Slim Shady. Till I Collapse. Venom. When I'm Gone. Without Me.
	\begin{itemize}
		\item {\sc Eminem $+$ Dido.} Stan (long version).
		\item {\sc Eminem $+$ Rihanna.} Love The Way You Lie. The Monster.
	\end{itemize}
	\item {\sc ERIK.} Có Tất Cả Nhưng Thiếu Anh. Em Không Sai, Chúng Ta Sai. Sau Tất Cả.
	\begin{itemize}
		\item {\sc ERIK $+$ Mr. Siro.} Chạm Đáy Nỗi Đau.
	\end{itemize}
	\item {\sc Europe.} The Final Countdown.
	\item {\sc Eurythmics.}
	\begin{itemize}
		\item {\sc Eurythmics $+$ Annie Lennox $+$ Dave Stewart.} Sweet Dreams (Are Made Of This).
	\end{itemize}
	\item {\sc Evanescence.} Bring Me To Life. My Immortal.
	\item {\sc Fall Out Boy.} Immortals (from ``Big Hero 6'').
	\item {\sc FBBOIZ.} Để Em Rời Xa.
	\item {\sc Fiona Fung.} Proud Of You.
	\item {\sc Fleetwood Mac.} Dreams. Little Lies.
	\item {\sc Flo Rida.}
	\begin{itemize}
		\item {\sc Flo Rida $+$ Ke\$ha.} Right Round.
		\item {\sc Flo Rida $+$ T-Pain.} Low.
	\end{itemize}
	\item {\sc Fort Minor.} Remember The Name.
	\item {\sc Gary Jules.} Mad World.
	\item {\sc G-Eazy.}
	\begin{itemize}
		\item {\sc G-Eazy $+$ Halsey.} Him \& I.
		\item {\sc G-Eazy $+$ Bebe Rexha.} Me, Myself \& I.
	\end{itemize}	
	\item {\sc GD $+$ Taeyang.} Good Boy.
	\item {\sc Gill Chang $+$ Danni Carra.} Why Do I Try.
	\item {\sc Gotye.}
	\begin{itemize}
		\item {\sc Gotye $+$ Kimbra.} Somebody That I Used To Know.
	\end{itemize}
	\item {\sc Grain in Ear.} Mang Chủng.
	\item {\sc Green Day.} 21 Days. Boulevard Of Broken Dreams.
	\item {\sc Hà Anh Tuấn.} Tháng Tư Là Lời Nối Dối Của Em.
	\item {\sc Halsey.} Colors. Gasoline. Nightmare. Without Me.
	
	\begin{quotation}\it
		\textbf{Are you insane like me?}
		
		\textbf{Been in pain like me?}
		
		Bought a hundred dollar bottle of champagne like me?
		
		\textbf{Just to pour that motherfucker down the drain like me?}
		
		Would you use your water bill to dry the stain like me?
		\\
		
		Are you high enough without the Mary Jane like me?
		
		\textbf{Do you tear yourself apart to entertain like me?}
		
		Do the people whisper 'bout you on the train like me?
		
		Saying that you shouldn't waste your pretty face like me?
		\\
		
		And all the people say
		
		\textbf{You can't wake up, this is not a dream}
		
		\textbf{You're part of a machine, you are not a human being}
		
		With your face all made up, living on a screen
		
		Low on self-esteem, so you run on gasoline
		\\
		
		I think there's a flaw in my code
		
		(Oh, ooh-oh, ooh-oh, oh)
		
		These voices won't leave me alone
		
		Well, my heart is gold \& my hands are cold
		\\
		
		Are you deranged like me?
		
		Are you strange like me?
		
		Lighting matches just to swallow up the flame like me?
		
		Do you call yourself a fucking hurricane like me?
		
		\textbf{Pointing fingers 'cause you'll never take the blame like me?}
		\\
		
		And all the people say
		
		You can't wake up, this is not a dream
		
		You're part of a machine, you are not a human being
		
		With your face all made up, living on a screen
		
		\textbf{Low on self-esteem, so you run on gasoline}
		\\
		
		\textbf{I think there's a flaw in my code}
		
		(Oh, ooh-oh, ooh-oh, oh)
		
		These voices won't leave me alone
		
		Well, \textbf{my heart is gold} \& \textbf{my hands are cold}\hfill$\square$
	\end{quotation}
	\item {\sc Hạnh Sino.} Em Mây.
	\item {\sc Hans Zimmer.} A Dark Knight. Leave No Man Behind. Mountains. S.T.A.Y. The Da Vinci Code. Time (Inception).
	\item {\sc Haro.}
	\begin{itemize}
		\item {\sc Haro $+$ Phong Max $+$ Masew.} Lừa Tình.
	\end{itemize}
	\item {\sc Hawk Nelson.} Sold Out.
	\item {\sc Hiền Thục.} Nhật Ký Của Mẹ. Yêu Dấu Theo Gió Bay.
	\item {\sc Hiroyuki Sawano.} Project [emU] ``Attack on Titan'' suite.
	\item {\sc Hoàng Yến Chibi.} Đồi Hoa Mặt Trời. No Boyfriend.
	\item {\sc Hoa Vinh.} Đừng Quên Tên Anh.
	\item {\sc Hoàng Thuỳ Linh.} Bánh Trôi Nước (Woman). Duyên Âm (Love of Ghost). Để Mị Nói Cho Mà Nghe (Le Mi tell). Em Đây Chẳng Phải Thúy Kiều (I Am Not Thuy Kieu). Kẻ Cắp Gặp Bà Già (Diamond Cut Diamond). Kẽo Cà Kẽo Kẹt (The Creeking). Lắm Mối Tối Ngồi Không (Run After Two Hares, Catch Nones). See Tình.
	\begin{itemize}
		\item {\sc Hoàng Thuỳ Linh $+$ Binz.} Kẻ Cắp Gặp Bà Già (Diamond Cut Diamond) (VisconC Remix). 
		\item {\sc Hoàng Thuỳ Linh $+$ Đen.} Gieo Quẻ (Casting Coins).
		\item {\sc Hoàng Thuỳ Linh $+$ Hồ Hoài Anh $+$ TripleD.} Tứ Phủ.
		\item {\sc Hoàng Thuỳ Linh $+$ Thanh Lam, Tùng Dương.} Đánh Đố.
	\end{itemize}
	\item {\sc Hoaprox.}
	\begin{itemize}
		\item {\sc Hoaprox $+$ Nick Strand $+$ Mio.} With You (Ngẫu Hứng).
		\item {\sc Hoaprox $+$ Xesi.} Vô Tình.
	\end{itemize}
	\item {\sc Hozier.} Take Me To Church.
	
	\begin{quotation}\it
		My lover's got humor
		
		She's the giggle at a funeral
		
		Knows everybody's disapproval
		
		I should've worshiped her sooner
		
		If the Heavens ever did speak
		
		She's the last true mouthpiece
		
		Every Sunday's getting more bleak
		
		A fresh poison each week
		
		"We were born sick", you heard them say it
		
		My church offers no absolutes
		
		She tells me, "Worship in the bedroom"
		
		The only Heaven I'll be sent to
		
		Is when I'm alone with you
		
		I was born sick, but I love it
		
		Command me to be well
		
		A-, Amen, Amen, Amen
		\\
		
		Take me to church
		
		I'll worship like a dog at the shrine of your lies
		
		I'll tell you my sins \& you can sharpen your knife
		
		Offer me that deathless death
		
		Good God, let me give you my life
		
		Take me to church
		
		I'll worship like a dog at the shrine of your lies
		
		I'll tell you my sins \& you can sharpen your knife
		
		Offer me that deathless death
		
		Good God, let me give you my life
		\\
		
		If I'm a pagan of the good times
		
		My lover's the sunlight
		
		To keep the Goddess on my side
		
		She demands a sacrifice
		
		Drain the whole sea
		
		Get something shiny
		
		Something meaty for the main course
		
		That's a fine looking high horse
		
		What you got in the stable?
		
		We've a lot of starving faithful
		
		That looks tasty
		
		That looks plenty
		
		This is hungry work
		\\
		
		Take me to church
		
		I'll worship like a dog at the shrine of your lies
		
		I'll tell you my sins so you can sharpen your knife
		
		Offer me my deathless death
		
		Good God, let me give you my life
		
		Take me to church
		
		I'll worship like a dog at the shrine of your lies
		
		I'll tell you my sins so you can sharpen your knife
		
		Offer me my deathless death
		
		Good God, let me give you my life
		\\
		
		No masters or kings when the ritual begins
		
		There is no sweeter innocence than our gentle sin
		
		In the madness \& soil of that sad earthly scene
		
		Only then I am human
		
		Only then I am clean
		
		Oh, oh, Amen, Amen, Amen
		\\
		
		Take me to church
		
		I'll worship like a dog at the shrine of your lies
		
		I'll tell you my sins \& you can sharpen your knife
		
		Offer me that deathless death
		
		Good God, let me give you my life
		
		Take me to church
		
		I'll worship like a dog at the shrine of your lies
		
		I'll tell you my sins \& you can sharpen your knife
		
		Offer me that deathless death
		
		Good God, let me give you my life
	\end{quotation}
	\item {\sc Hồ Ngọc Hà.}
	\begin{itemize}
		\item {\sc Hồ Ngọc Hà $+$ Noo Phước Thịnh.} Nỗi Nhớ Đầy Vơi.
	\end{itemize}
	\item {\sc Hồ Quỳnh Hương.} Hoang Mang.
	\item {\sc Huang Ling.}
	\begin{itemize}
		\item {\sc Huang Ling $+$ Tăng Duy Tân $+$ Phong Max.} Ngây Thơ Chinese version.
	\end{itemize}
	\item {\sc Huỳnh Tú.} Thinking Of You.
	\item {\sc HuyR.} Anh Thanh Niên.
	\begin{itemize}
		\item {\sc HuyR $+$ Tùng Viu $+$ Quang Đăng.} Cô Gái Vàng.
	\end{itemize}
	\item {\sc Hứa Lam Tâm.} Hồng Mã.
	\item {\sc Hương Ly.} Thế Thái.
	\item {\sc Hương Tràm.} Cho Em Gần Anh Thêm Chút Nữa. Duyên Mình Lỡ.
	\item {\sc Indila.} Love Story.
	\item {\sc Idina Menzel.}
	\begin{itemize}
		\item {\sc Idina Menzel $+$ AURORA.} Into the Unknown (from ``Frozen 2'').
	\end{itemize}
	\item {\sc iKON.} Love Scenario. 
	\item {\sc Imagine Dragons.} Bad Liar. Believer. I'm Happy. 6Whatever It Takes.
	\item {\sc ITZY.} Wannabe.
	\item {\sc Jade.} Control (Zoe Wees cover). Homesick (Dua Lipa cover). Lately. Straw House. Your Type.
	\item {\sc James Brown.} It's A Man's Man's Man's World.
	\item {\sc JayKii.} Chiều Hôm Ấy.
	\item {\sc Jaymes Young.} Infinity.
	\item {\sc Jennie.} Solo.
	\item {\sc Jennifer Lopez.} Papi.
	\begin{itemize}
		\item {\sc Jennifer Lopez $+$ Pitbull.} Live It Up. On The Floor.
	\end{itemize}
	\item {\sc Jessie J.} Flashlight (from Pitch Perfect 2).
	\begin{itemize}
		\item {\sc Jessie J $+$ Ariana Grande $+$ Nicki Minaj.} Bang Bang.
	\end{itemize}
	\item {\sc JGKiD.}
	\begin{itemize}
		\item {\sc JGKiD $+$ Đen.} Ta Và Nàng.
	\end{itemize}
	\item {\sc Jim Croce.} Time In A Bottle.
	\item {\sc Joe Hisaishi.} Kiki's Delivery Service. Merry-Go-Round of Life (from Howl's Moving Castle). ``Princess Mononoke'' Suite. The Wind Forest (from My Neighbor Totoro).
	\item {\sc John Legend.} All of Me.
	\item {\sc John Newman}. Love Me Again.
	\item {\sc JustaTee.} Bâng Khuâng. Đã Lỡ Yêu Em Nhiều. Forever Alone. Người Nào Đó. She Neva Knows.
	\begin{itemize}
		\item {\sc JustaTee $+$ Binz.} Crying Over You.
		\item {\sc JustaTee $+$ Hoàng Thùy Linh $+$ Đen.} Làm Gì Phải Hốt.
		\item {\sc JustaTee $+$ Kimmese.} Lời Nói Dối Chân Thật.
	\end{itemize}
	\item {\sc Justin Bieber.} Boyfriend. Love Yourself. What Do You Mean?.
	\begin{itemize}
		\item {\sc Justin Bieber $+$ Ludacris.} Baby.
	\end{itemize}
	\item {\sc Justine Skye.}
	\begin{itemize}
		\item {\sc Justine Skye $+$ Tyga.} Collide.
	\end{itemize}
	\item{\sc JVKE.} Golden Hour.
	\item {\sc K-391.} Summertime [Sunshine].
	\begin{itemize}
		\item {\sc K-391 $+$ Alan Walker $+$ Julie Bergan $+$ Seungri.} Ignite.
		\item {\sc K-391 $+$ R{\O}RY.} Aurora.
	\end{itemize}
	\item {\sc KALEO.} Way Down We Go.
	\item {\sc Katy Perry.} Firework. Harleys In Hawaii. The One That Got Away.
	\item {\sc Karik.} Lần Cuối. Nơi Những Cảm Xúc Nối Dài. Từng Là Tất Cả.
	\begin{itemize}
		\item {\sc Karik $+$ Daniel Mastro.} Anh Là Sinh Viên.
		\item {\sc Karik $+$ Emma.} Tất Cả Tại Anh.
		\item {\sc Karik $+$ Orange.} Vô Thường.
		\item {\sc Karik $+$ Thái Trinh.} Cạn Cả Nước Mắt.
		\item {\sc Karik $+$ Vũ Phụng Tiên.} Chưa Từng Vì Nhau. Đau Vậy Đủ Rồi.
	\end{itemize}	
	\item {\sc Katy Perry.} The One That Got Away.
	\item {\sc Kendrick Lamar.}
	\begin{itemize}
		\item {\sc Kendrick Lamar $+$ SZA.} All The Stars.
	\end{itemize}
	\item {\sc Kenshi Yonezu.} Lemon.
	\item {\sc Ke\$ha.} Die Young. TiK ToK.
	\item {\sc Khalid.} Young Dumb \& Broke.
	\item {\sc Khắc Hưng.}
	\begin{itemize}
		\item {\sc Khắc Hưng $+$ MIN $+$ ERIK.} Ghen.
	\end{itemize}
	\item {\sc Khổng Tú Quỳnh.}
	\begin{itemize}
		\item {\sc Khổng Tú Quỳnh $+$ RIN9.} Mãi Mãi Là Một Lời Nói Dối.
	\end{itemize}
	\item {\sc Kimmese.}
	\begin{itemize}
		\item {\sc Kimmese $+$ Đen.} Loving You Sunny (Prod.by Touliver).
		\item {\sc Kimmese $+$ JustaTee.} Real Love.
	\end{itemize}
	\item {\sc Lady Gaga.} Always Remember Us This Way. Bad Romance. Poker Face.
	\begin{itemize}
		\item {\sc Lady Gaga $+$ BlackPink.} Sour Candy.
		\item {\sc Lady Gaga $+$ Colby O'Donis.} Just Dance.
	\end{itemize}
	\item {\sc Lana Del Rey.} Born To Die. Summertime Sadness.
	\item {\sc Le Sserafim.} Easy. Perfect Night. Smart.
	\item {\sc Lil Nas X.}
	\begin{itemize}
		\item {\sc Lil Nas X $+$ Billy Ray Cyrus.} Old Town Road.
	\end{itemize}
	\item {\sc Lil Tjay.}
	\begin{itemize}
		\item {\sc Lil Tjay $+$ 6LACK.} Calling My Phone.
	\end{itemize}
	\item {\sc Limp Bizkit.} Behind Blue Eyes.
	\item {\sc Lindsey Stirling.} Boulevard of Broken Dreams (Green Day cover). Carol of The Bells. Crystallize. Elements. My Immortal (Evanescence cover). Phantom Of The Opera. River Flows In You. Senbonzakura (cover). Shadows. Take Flight. Til The Light Goes Out.
	\begin{itemize}
		\item {\sc Lindsey Stirling $+$ Lzzy Hale.} Shatter Me.
		\item {\sc Lindsey Stirling $+$ Pentatonix.} Radioactive (Imagine Dragons cover).
	\end{itemize}
	\item {\sc Linkin Park.} Breaking The Habit. Burn It Down. Burning In The Skies. Castle Of Glass. Faint. Final Masquerade. From The Inside. Leave Out All The Rest. New Divine. Numb. Somewhere I Belong. Until It's Gone. What I've Done.
	\item {\sc Lisa.} Lalisa. Money.
	\item {\sc Lil'Knight (LK).} At Last. Cơn Mưa Qua. Dị Mộng. Hà Nội Xịn. K 2. Không Tin Một Sớm Mai Bình Yên. Lip.
	\begin{itemize}
		\item {\sc LK $+$ Eddy Việt.} Ánh Sáng Nơi Cuối Con Đường.
		\item {\sc LK $+$ JustaTe.} Người Lạ Nơi Cuối Con Đường.
		\item {\sc LK $+$ JustaTee $+$ Andree $+$ Emily.} Ngọn Nến Trước Gió.
		\item {\sc LK $+$ Mc Ill $+$ Wowy.} Thầy Hiệu Trưởng.
	\end{itemize}
	\item {\sc Little Mix.}
	\begin{itemize}
		\item {\sc Little Mix $+$ Ty Dolla \$ign.} Think About Us.
	\end{itemize}
	\item {\sc Lou Hoàng.} Mình Là Gì Của Nhau. Yêu Em Dại Khờ.
	\begin{itemize}
		\item {\sc Lou Hoàng $+$ Miu Lê.} Yêu Một Người Có Lẽ.
		\item {\sc Lou Hoàng $+$ OnlyC Pro.} Bắt Cóc Con Tim.
	\end{itemize}
	\item {\sc Lynk Lee.} Buồn Thì Cứ Khóc Đi.
	\item {\sc M2M.} Pretty Boy. The Day You Went Away.
	\item {\sc MAGIC!.} Rude.
	\item {\sc Major Lazer.}
	\begin{itemize}
		\item {\sc Major Lazer $+$ DJ Snake $+$ M\O.} Lean On.
	\end{itemize}
	\item {\sc M\r{a}neskin.} Beggin'.
	\item {\sc Maroon 5.} Animals. Memories. One More Night.
	\begin{itemize}
		\item {\sc Maroon 5 $+$ Megan Three Stallion.} Beautiful Mistakes.
		\item {\sc Maroon 5 $+$ Wiz Khalifa.} Payphone.
	\end{itemize}
	\item {\sc Marshmello.}
	\begin{itemize}
		\item {\sc Marshmello $+$ Anne-Marie.} Friends.
		\item {\sc Marshmello $+$ Halsey.} Be Kind.
	\end{itemize}
	\item {\sc Martin Garrix.}
	\begin{itemize}
		\item {\sc Martin Garrix $+$ David Guetta.} So Far Away.
	\end{itemize}
	\item {\sc Masew.}
	\begin{itemize}
		\item {\sc Masew $+$ Khoi Vu.} Ái Nộ.
		\item {\sc Masew $+$ Khoi Vu $+$ Yến Tatoo.} Ái Nộ.
		\item {\sc Masew $+$ Pháo.} Điêu Toa.
		\item {\sc Masew $+$ Tuấn Cry.} Mời Trầu.
	\end{itemize}
	\item {\sc Max Oazo.}
	\begin{itemize}
		\item {\sc Max Oazo $+$ Moonessa.} Once Upon A Time.
	\end{itemize}
	\item {\sc MC Mong.}
	\begin{itemize}
		\item {\sc MC Mong $+$ Mellow.} Sick Enough To Die.
		\item {\sc MC Mong $+$ Sweden Laundry.} Sick Enough To Die (Part 2).
	\end{itemize}
	\item {\sc Michael Learns to Rock.} Take Me To Your Heart.
	\item {\sc Miguel.} Sure Thing.
	\item {\sc Miley Cyrus.} Jolene. Mother's Daughter. Wrecking Ball.
	\item {\sc MIN.} Đừng Yêu Nữa, Em Mệt Rồi. Trên Tình Bạn Dưới Tình Yêu.
	\begin{itemize}
		\item {\sc MIN $+$ Mr. A.} Có Em Chờ.
		\item {\sc MIN $+$ Hứa Kim Tuyền $+$ Veyo.} Tìm X.
	\end{itemize}
	\item {\sc Minh Vương M4U.} Em Ơi Lên Phố. Nỗi Đau Xót Xa.
	\item {\sc Miu Lê.} Giả Vờ Nhưng Em Yêu Anh.
	\begin{itemize}
		\item {\sc Miu Lê $+$ Karik $+$ Châu Đăng Khoa.} Vì Mẹ Anh Bắt Chia Tay.
	\end{itemize}
	\item {\sc Mr. T.}
	\begin{itemize}
		\item {\sc Mr. T $+$ Yanbi $+$ Bảo Thy.} Nothing In Your Eyes 2.
		\item {\sc Mr. T $+$ Yanbi $+$ Hà Bi.} Nothing In Your Eyes.
		\item {\sc Mr. T $+$ Yanbi $+$ Hằng Bingboong.} Thu Cuối.
	\end{itemize}
	\item {\sc Mỹ Tâm.} Chuyện Như Chưa Bắt Đầu (Pretend We Had No Start). Đâu Chỉ Riêng Em. Như Một Giấc Mơ (Like A Dream).
	\item {\sc Natalie Taylor.} Surrender.
	\item {\sc Nelly.} Just A Dream.
	\item {\sc Ngải Thần.} Thời Không Sai Lệch.
	\item {\sc Ngọc Dolil.} Cùng Anh.
	\item {\sc Nguyễn Đình Vũ.} Chúng Ta Dừng Lại Ở Đây Thôi. Cứ Thế Mong Chờ. Em Của Quá Khứ.
	\item {\sc Nguyên Hà.} Chờ Ngày Lời Hứa Nở hoa. Nhắm Mắt Thấy Mùa Hè. Sau Này Hãy Gặp Lại Nhau Khi Hoa Nở. Ta Có Hẹn Với Tháng 5. Xin Lỗi.
	\item {\sc Nguyễn Kiều Anh.} Độc Ẩm.
	\item {\sc Nguyễn Trọng Tài.}
	\begin{itemize}
		\item {\sc Nguyễn Trọng Tài $+$ San Ji $+$ Double X.} HongKong1.
	\end{itemize}
	\item {\sc Niccolo Paganini.} La Campanella.
	\item {\sc Nightwish.} Amaranth. \'Elan. Nemo. Over The Hills And Far Away. The Islander. While Your Lips Are Still Red. Wish I Had An Angel.
	\item {\sc Noo Phước Thịnh.} Cause I Love You. Chờ Ngày Mưa Tan.
	\begin{itemize}
		\item {\sc Noo Phước Thịnh $+$ Tonny Việt.} Gạt Đi Nước Mắt.
	\end{itemize}
	\item {\sc OneRepublic.} Counting Stars.
	\item {\sc Only C.}
	\begin{itemize}
		\item {\sc Only C $+$ Lou Hoàng.} Đếm Ngày Xa Em.
	\end{itemize}
	\item {\sc Orange.} Em Hát Ai Nghe. Ok Anh Đúng.
	\begin{itemize}
		\item {\sc Orange $+$ Khói $+$ Châu Đăng Khoa.} Chân Ái.
	\end{itemize}
	\item {\sc Outr3ach.}
	\begin{itemize}
		\item {\sc Outr3ach $+$ J-Marin $+$ Kaitlin Grace.} Worth It.
	\end{itemize}
	\item {\sc Passenger.} Let Her Go.
	\item {\sc Phạm Trưởng.} Không Được Khóc.
	\item {\sc Phan Đình Tùng.} Cào Cào Lá Tre. Kiếp Dã Tràng.
	\item {\sc Phan Mạnh Quỳnh.} Anh Ghét Làm Bạn Em. Có Chàng Trai Viết Lên Cây. Hãy Ra Khỏi Người Đó Đi. Hồi Ức. Huyền Thoại $|$ The Legend. Khi Người Mình Yêu Khóc. Khi Phải Quên Đi. Nhạt. Nước Ngoài. Tri Kỷ. Từ Đó (Mắt Biếc OST). Xa Kỷ Niệm.
	\item {\sc Pháo.}
	\begin{itemize}
		\item {\sc Pháo $+$ Masew.} 2 Phút Hơn.
	\end{itemize}
	\item {\sc Phùng Khánh Linh.} Hôm Nay Tôi Buồn.
	\item {\sc Phương Ly.}
	\begin{itemize}
		\item {\sc Phương Ly $+$ JustaTee.} Mặt Trời Của Em.
	\end{itemize}
	\item {\sc Pitpull.}
	\begin{itemize}
		\item {\sc Pitpull $+$ Marc Anthony.} Rain Over Me.
	\end{itemize}
	\item {\sc PLVTINUM $+$ Tarro.} Champagne \& Sunshine.
	\item {\sc Queen.} Bohemian Rhapsody.
	\item {\sc R. City.}
	\begin{itemize}
		\item {\sc R. City $+$ Adam Levine.} Locked Away.
	\end{itemize}
	\item {\sc Rachel Platten.} Fight Song.
	\item {\sc Rag'n'Bone Man.} Skin.
	\item {\sc Redfoo.} New Thang.
	\item {\sc Rhymastic.} Nến \& Hoa.
	\item {\sc Richard Marx.} I Will Be Right Here Waiting For You.
	\item {\sc Rixton.} Me \& My Broken Heart.
	\item {\sc Ros\'e (BlackPink).} A Little Girl. Can't Help Falling In Love. Coming Home. Don't Look Back In Anger (Oasis). Eyes Closed.Eyes, Nose, Lips. Fallin' All In You. Fix You. Gone. If It Is You. Irreplaceable. Let It Be. Let It Go. Not For Long. Officially Missing You. On The Ground. Only Look At Me. Read My Mind. Somebody Else. Someone You Loved. The Only Exception. Until I Found You. Viva La Vida (Coldplay cover). You \& I (Park Bom).
	\begin{itemize}
		\item {\sc Ros\'e (BlackPink) $+$ G-Dragon.} Without You.
		\item {\sc Ros\'e (BlackPink) $+$ Jiso (BlackPink).} Love Yourself.
		\item {\sc Ros\'e (BlackPink) $+$ Lisa (BlackPink).} L.O.V.E.
		\item {\sc Ros\'e (BlackPink) $+$ Millenium.} Just Dance.
	\end{itemize}
	\item {\sc Ruelle.} Madness.
	\item {\sc Sadboixx.} I Don't Wanna Be Me.
	\item {\sc Sam Smith.} Diamonds. Fire On Fire. Writing's On The Wall.
	\item {\sc Sam Tinnesz.} Bloodshot. Carry On. Even If It Hurts. Far From Home (The Raven). Fight On. Hold On For Your Life. Leading The Pack. Savage. When The Truth Hunts You Down.
	\begin{itemize}
		\item {\sc Sam Tinnesz $+$ Silverberg.} Wolves.
		\item {\sc Sam Tinnesz $+$ Super Duper.} Babel.
		\item {\sc Sam Tinnesz $+$ Yacht Money.} Play With Fire.
	\end{itemize}
	\item {\sc Sasha Sloan.} Older.
	\item {\sc SAYGRACE $+$ G-Eazy.} You Don't Own Me.
	\item {\sc Sean Kingston.} Beautiful Girls.
	\item {\sc Sean Paul.}
	\begin{itemize}
		\item {\sc Sean Paul $+$ Dua Lipa.} No Lie.
	\end{itemize}
	\item {\sc Selena Gomez.} Come \& Get It. Kill Them With Kindness. Lose You To Love Me. Same Old Love. Slow Down. The Heart Wants What It Wants.
	\begin{itemize}
		\item {\sc Selena Gomez $+$ Marshmello.} Wolves.
	\end{itemize}
	\item {\sc Serena.} Safari.
	\item {\sc Shawn Mendes.} Imagination. In My Blood. It'll Be Okay. Kid In Love. Memories. Mercy. Never Be Alone. Running Low. Stitches. Treat You Better.
	\begin{itemize}
		\item {\sc Shawn Mendes $+$ Camila Cabello.} I Know What You Did Last Summer. Se\~norita.
	\end{itemize}
	\item {\sc Shayne Ward.} Breathless. No Promises. Until You.
	\item {\sc Shontelle.} Impossible.
	\item {\sc Sia.} California Dreamin'. Chandelier. Elastic Heart. Unstoppable.
	
	\begin{quotation}\it
		All smiles, I know what it takes to fool this town
		
		I'll do it 'til the sun goes down \& all through the night time
		
		Oh yeah
		
		Oh yeah, I'll tell you what you wanna hear
		
		Leave my sunglasses on while I shed a tear
		
		It's never the right time
		
		Yeah, yeah
		\\
		
		I put my armor on, show you how strong how I am
		
		I put my armor on, I'll show you that I am
		\\
		
		I'm unstoppable
		
		I'm a Porsche with no brakes
		
		I'm invincible
		
		Yeah, I win every single game
		
		I'm so powerful
		
		I don't need batteries to play
		
		I'm so confident
		
		Yeah, I'm unstoppable today
		
		Unstoppable today
		
		Unstoppable today
		
		Unstoppable today
		
		I'm unstoppable today
		\\
		
		Break down, only alone I will cry on out
		
		You'll never see what's hiding out
		
		Hiding out deep down
		
		Yeah, yeah
		
		I know, I've heard that to let your feelings go
		
		Is the only way to make friendships grow
		
		But I'm too afraid now
		
		Yeah, yeah
		\\
		
		I put my armor on, show you how strong how I am
		
		I put my armor on, I'll show you that I am
		\\
		
		I'm unstoppable
		
		I'm a Porsche with no breaks
		
		I'm invincible
		
		Yeah, I win every single game
		
		I'm so powerful
		
		I don't need batteries to play
		
		I'm so confident
		
		Yeah, I'm unstoppable today
		
		Unstoppable today
		
		Unstoppable today
		
		Unstoppable today
		
		I'm unstoppable today
		
		Unstoppable today
		
		Unstoppable today
		
		Unstoppable today
		
		I'm unstoppable today
		\\
		
		I put my arm around, show you how strong I am
		
		I put my arm around, I'll show you that I am
		\\
		
		I'm unstoppable
		
		I'm a Porsche with no breaks
		
		I'm invincible
		
		Yeah, I win every single game
		
		I'm so powerful
		
		I don't need batteries to play
		
		I'm so confident
		
		Yeah, I'm unstoppable today
		
		Unstoppable today
		
		Unstoppable today
		
		Unstoppable today
		
		I'm unstoppable today
		
		Unstoppable today
		
		Unstoppable today
		
		Unstoppable today
		
		I'm unstoppable today\hfill$\square$
	\end{quotation}
	\item {\sc Skylar Grey.} Coming Home. Everything I Need (Aquaman soundtrack).
	\item {\sc Sofia.} Rồi Một Ngày Mình Nói Về Tình Yêu.
	\begin{itemize}
		\item {\sc Sofia $+$ Addy Trần.} Ai Chung Tình Được Mãi (Cover).
		\item {\sc Sofia $+$ Khói $+$ Châu Đăng Khoa.} Là Do Em Xui Thôi. Nhớ Người Hay Nhớ.
		\item {\sc Sofia $+$ Superbrother.} Ai Chung Tình Được Mãi (Remake).
	\end{itemize}
	\item {\sc Soobin Hoàng Sơn.} Phía Sau Một Cô Gái.
	\item {\sc Sơn Tùng M-TP.} Buông Đôi Tay Nhau Ra. Chạy Ngay Đi (Run Now). Chắc Ai Đó Sẽ Về. Chúng Ta Không Thuộc Về Nhau. Cơn Mưa Ngang Qua. Hãy Trao Cho Anh. Muộn Rồi Mà Sao Còn. Nắng Ấm Xa Dần.
	\item {\sc Stephen Sanchez.} Until I Found You.
	\item {\sc Stevie Wonder.} Pastime Paradise.
	\item {\sc Sting.} Shape of My Heart.
	\begin{quotation}
		``He deals the cards to find the answer
		
		The sacred geometry of chance
		
		The hidden law of a probable outcome
		
		The numbers lead a dance'' -- Sting, {\it Shape of My Heart}
	\end{quotation}
	\item {\sc Suni Hạ Linh.} Em Đã Biết.
	\begin{itemize}
		\item {\sc Suni Hạ Linh $+$ Lou Hoàng.} Không Sao Mà Em Đây Rồi.
	\end{itemize}
	\item {\sc Survivor.} Burning Heart. Eye Of The Tiger.
	\item {\sc Taeyang.} Eyes, Nose, Lips.
	\item {\sc T-ara.} Cry Cry. Don't Leave. Day By Day.
	\begin{itemize}
		\item {\sc T-ara $+$ Supernova.} TTL (Time To Love).
	\end{itemize}
	\item {\sc Tate McRae.} You Broke Me First.
	\item {\sc Taylor Swift.} 22 (Taylor's version). All Too Well (Sad Girl Autumn Version; Taylor's version; 10 minute version, Taylor's version, From The Vault). All Too Well: The Short Film. Babe (Taylor's version, From The Vault). Back To December. Bad Blood. Begin Again (Taylor's version). Better Man (Taylor's version, From The Vault). Breath (Taylor's version). Bye Bye Baby (Taylor's version, From The Vault). Change (Taylor's version). Cardigan. Champagne Problems. Closure. Come Back $\ldots$ Be Here (Taylor's version). Come In With The Rain (Taylor's version). Cowboy Like Me. Don't You (Taylor's version, From The Vault). Dorothea. Everything Has Changed (Taylor's version). Fearless (Taylor's version). Fifteen (Taylor's version). Forever \& Always (Taylor's version; Piano version, Taylor's version). Forever Winter (Taylor's version, From The Vault). Girl At Home (Taylor's version). Gold Rush. Happiness. Hey Stephen (Taylor's version). Holy Ground (Taylor's version). I Almost Do (Taylor's version). I Bet You Think About Me (Taylor's version, From The Vault). I Knew You Were Trouble. Love Story (Taylor's version; Elvira remix, Taylor's version, From The Vault). It's Time To Go. Ivy. Jump Then Fall (Taylor's version). Long Short Story. Marjorie. Message In A Bottle (Taylor's version, From The Vault). Mr. Perfectly Fine (Taylor's version). Nothing New (Taylor's version, From The Vault). Right Where You Left Me. Red (Taylor's version). Ronan (Taylor's version). Run (Taylor's version, From The Vault). Sad Beautiful Tragic (Taylor's version). Safe \& Sound. State Of Grace (Taylor's version; Acoustic version). Star Light (Taylor's version). Stay Stay Stay (Taylor's version). Superstar (Taylor's version). Tell Me Why (Taylor's version). That's When (Taylor's version, From The Vault). The Best Day (Taylor's version). The Last Time (Taylor's version). The Lucky One (Taylor's version). The Moment I Knew (Taylor's version). The Other Side Of The Door (Taylor's version). The Very 1st Night (Taylor's version, From The Vault). The Way I Loved You (Taylor's version). 'Tis The Damn Season. Today Was A Fairytale (Taylor's version). Tolerate It. Treacherous (Taylor's version). Untouchable (Taylor's version). We Are Never Ever Getting Back Together (Taylor's version). We Were Happy (Taylor's version, From The Vault). White Horse (Taylor's version). Willow (lonely witch version; dancing witch version; moonlit witch version; dancing witch version, Elvira remix; moonlit witch version, yule log; lonely witch version, yule log; dancing witch version, yule log). You're Not Sorry (Taylor's version). You Belong With Me (Taylor's version).
	\begin{itemize}
		\item {\sc Taylor Swift $+$ Bon Iver.} Evermore.
		\item {\sc Taylor Swift $+$ Chris Stapleton.} I Bet You Think About Me (Taylor's version).
		\item {\sc Taylor Swift $+$ HAIM.} No Body, No Crime.
		\item {\sc Taylor Swift $+$ Maren Morris.} You All Over Me (From The Vault).
		\item {\sc Taylor Swift $+$ the National.} Coney Island.
	\end{itemize}
	\item {\sc Tăng Duy Tân.} Bên Trên Tầng Lầu. Dạ Vũ. Mắt Nâu. Tình Đầu.
	\begin{itemize}
		\item {\sc Tăng Duy Tân $+$ Phong Max.} 05 (Không Phai). Ngây Thơ.
	\end{itemize}	
	\item {\sc TeamV.}
	\begin{itemize}
		\item {\sc TeamV $+$ NhatNguyen.} Bình Yên Nơi Anh.
	\end{itemize}
	\item {\sc Thanh Bùi.}
	\begin{itemize}
		\item {\sc Thanh Bùi $+$ Hồ Ngọc Hà.} Lặng Thầm Một Tình Yêu ( Đề Mai Tính OST).
		\item {\sc Thanh Bùi $+$ Tata Young.} Tình Về Nơi Đâu - Where Do We Go.
	\end{itemize}
	\item {\sc The Animals.} House of the Rising Sun.
	\item {\sc The Chainsmokers $+$ Halsey.} Closer.
	\item {\sc The Cranberries.} Zombie.
	\item {\sc TheFatRat.} Afterlife. Arcadia [Chapter 2]. Ascendancy. Electrified. Elevate. Envelope. Fire [Chapter 8]. Jackpot (Jackpot EP Track 1). Kingdom Come. Mad Moon Falling. Nemesis. Never Be Alone. No No No. Origin. Origin Reprise. Reminiscence. Rise Up. Rise Up (Orchestra Version). Telescope. Threnody. Time Lapse. Unity. Upwind [Chapter 4]. Warrior Songs (DOTA 2 music pack). Windfall. Xenogenesis.
	\begin{itemize}
		\item {\sc TheFatRat $+$ AleXa.} Rule The World.
		\item {\sc TheFatRat $+$ Anjulie.} Close To The Sun. Fly Away. Let Love Win [Chapter 10]. Love It When You Hurt Me [Chapter 9].
		\item {\sc TheFatRat $+$ Anna Yvette $+$ Laura Brehm.} Chosen.
		\item {\sc TheFatRat $+$ Cecilia Gault.} Our Song [Chapter 5]. Violet Sky [Chapter 6].
		\item {\sc TheFatRat $+$ Everen Maxwell $+$ Lindsey Stirling.} Warbringer [Chapter 7].
		\item {\sc TheFatRat $+$ Laura Brehm.} MAYDAY. Monody. The Calling. We'll Meet Again.
		\item {\sc TheFatRat $+$ Lola Blanc.} Oblivion.
		\item {\sc TheFatRat $+$ Maisy Kay.} The Storm.
		\item {\sc TheFatRat $+$ RIELL.} Hiding In The Blue [Chapter 1]. Pride \& Fear [Chapter 3].
		\item {\sc TheFatRat $+$ Slaydit.} Solitude.
		\item {\sc TheFatRat $+$ Slaydit $+$ Anjulie.} Stronger [Monstercat Release].
		\item {\sc TheFatRat $+$ Stasia Estep.} Warrior Song.
	\end{itemize}
	\item {\sc The Mamas \& The Papas.} California Dreamin'.
	\item {\sc The Men.} Chờ Em Trong Đêm.
	\item {\sc The Score.} Born For This. Fire. Higher. In My Bones. Legend. Miracle. Never Going Back. On And On. Only One. Revolution. Rush. Shakedown. The Fear. The Heat. Tightrope. Who I Am.
	\begin{itemize}
		\item {\sc The Score $+$ XYL\O.} Bulletproof.
	\end{itemize}
	\item {\sc The Weeknd.} False Alarm. Starboy. The Hills.
	\item {\sc Thu Thủy.}
	\begin{itemize}
		\item {\sc Thu Thủy $+$ Lương Bằng Quang.} Anh Tin Mình Đã Cho Nhau Một Kỷ Niệm.
	\end{itemize}
	\item {\sc Thùy Chi.} Con Đường Hạnh Phúc. Giấc Mơ Trưa. Giữ Em Đi.
	\begin{itemize}
		\item {\sc Thùy Chi $+$ M4U.} Mưa. Phía Cuối Con Đường. Xe Đạp.
		\item {\sc Thùy Chi $+$ Phạm Hồng Phước.} Anh Sẽ Tốt Mà.
		\item {\sc Thùy Chi $+$ Phan Đinh Tùng.} Thần Thoại (OST The Myth 2005).
		\item {\sc Thùy Chi $+$ Tiên Tiên.} Giữ Em Đi.
		\item {\sc Thùy Chi $+$ Trung Quân Idol.} Cô Bé Mùa Đông.
		\item {\sc Thùy Chi $+$ Wanbi Tuấn Anh.} Cho Em.
	\end{itemize}
	\item {\sc Tiên Cookie.}
	\begin{itemize}
		\item {\sc Tiên Cookie $+$ JustaTee $+$ BigDaddy.} Thời Gian Sẽ Trả Lời.
	\end{itemize}
	\item {\sc Thu Minh.} Bay.
	\item {\sc Tiên Cookie.} Tâm Sự Với Người Lạ.
	\item {\sc Tiên Tiên.} Đi Về Đâu. Em Không Thể. My Everything.
	\begin{itemize}
		\item {\sc Tiên Tiên $+$ Touliver.} Em Không Thể.
	\end{itemize}
	\item {\sc Timbaland.}
	\begin{itemize}
		\item {\sc Timbaland $+$ OneRepublic.} Apologize.
	\end{itemize}
	\item {\sc Timebelle.} Apollo.
	\item {\sc Tóc Tiên.}
	\begin{itemize}
		\item {\sc Tóc Tiên $+$ Touliver.} Có Ai Thương Em Như Anh.
	\end{itemize}
	\item {\sc Touliver.}
	\begin{itemize}
		\item {\sc Touliver $+$ Lê Hiếu $+$ Soobin Hoàng Sơn.} Ngày Mai Em Đi.
	\end{itemize}
	\item {\sc Trấn Thành.}
	\begin{itemize}
		\item {\sc Trấn Thành $+$ Thùy Chi.} Đêm Cô Đơn.
	\end{itemize}
	\item {\sc Trịnh Đình Quang.} Nếu Em Còn Tồn Tại.
	\item {\sc Trung Quân.} Chưa Bao Giờ. Trót Yêu.
	\item {\sc Tưởng Tuyết Nhi.} Yến Vô Hiết.
	\item {\sc Văn Mai Hương.} Nếu Như Anh Đến.
	\item {\sc Vicetone.}
	\begin{itemize}
		\item {\sc Vicetone $+$ Cozi Zuehlsdorff.} Nevada. Way Back.
		\item {\sc Vicetone $+$ Meron Ryan.} Walk Thru Fire.
	\end{itemize}
	\item {\sc Vincent Vinet.} Lose Yourself.
	\item {\sc Vitas.} Lucia Di Lammermoor. Opera \#2. The 7th Element.
	\item {\sc Vivaldi.} Winter.
	\item {\sc Vũ Cát Tường.} Vết Mưa.
	\item {\sc Wanbi Tuấn Anh.} Đôi Mắt. Vụt Mất.
	\item {\sc Westlife.} I Lay My Love On You. If I Let You Go. My Love. Nothing Gonna Change My Love For You. Soledad. You Raise Me Up.
	\item {\sc Whitney Houston.} I Will Always Love You.
	\item {\sc Wiz Khalifa.} See You Again.
	\item {\sc Xesi.}
	\begin{itemize}
		\item {\sc Xesi $+$ Masew $+$ NhatNguyen.} Túy Âm.
	\end{itemize}
	\item {\sc Yanni.} A Love for Life. Butterfly Dance. Dance With a Stranger. Deliverance. Felitsa. For All Seasons. If I Could Tell You. Into The Deep Blue. Keys To Imagination. Love Is All. Marching Season. Never Too Late. Nightingale. Niki Nana. Nostalgia. On Sacred Ground. Opening $+$ Desire. Playtime. Prelude. Rainmaker. Reflections of Passion. Renegade. Southern Exposure. Standing in Motion. Swept Away. The Mermaid. The Rain Must Fall. The Storm. Thirst For Life. Tribute. Waltz in 7{\tt/}8. Within Attraction. World Dance.
	\item {\sc Yến Tatoo.} Cứ Thế Rời Xa. Đắng Nồng Cay.
	\item {\sc Yun.}
	\begin{itemize}
		\item {\sc Yun $+$ Dr. A $+$ Deniro $+$ Dizzy.} Cố Nhân. Cố Nhân 2.
		\item {\sc Yun $+$ Vercynus $+$ Dizzy $+$ Dr. A.} Quỷ Tha Ma Bắt.
	\end{itemize}
	\item {\sc Zara Larsson.} Ain't My Fault. Lush Life.
	\item {\sc ZAYN.}
	\begin{itemize}
		\item {\sc ZAYN $+$ Sia.} Dusk Till Dawn.
	\end{itemize}
	\item {\sc Zedd.}
	\begin{itemize}
		\item {\sc Zedd $+$ Elley Duh\'e.} Happy Now.
	\end{itemize}
	\item {\sc Zella Day.} East of Eden.
\end{enumerate}

%------------------------------------------------------------------------------%

\section{Quotes}

\begin{enumerate}
	\item {\sc Gustave Le Bon.}
	\begin{itemize}
		\item ``It is better to understand many things than to know many things.''
	\end{itemize}
	\item {\sc Cervantes.}
	\begin{itemize}
		\item ``Tis holier to journey than to arrive'' -- Vinh quang là hành trình chứ không phải đích đến \cite[Suy ngẫm 2000, p. 269]{Leighton_Feyman_last_journey_VN}
	\end{itemize}
	\item {\sc Văn Như Cương.}
	\begin{itemize}
		\item ``Biển học là mênh mông, trong đó sách vở tuy quan trọng nhưng chỉ là vùng biển gần bờ mà thôi.''
		\item ``Hãy trung thực đừng dối trá, hãy vị tha đừng vị kỷ, hãy hòa đồng đừng đố kỵ, hãy cao thượng đừng thấp hèn, hãy độc lập suy nghĩ đừng adua bầy đàn, hãy nói lời thanh cao đừng buông lời tục tĩu.''
		\item ``Tự học là phương pháp tốt nhất để phát huy trí tuệ, để nắm vững kiến thức \& linh hoạt áp dụng. Còn học thêm là con đường ngắn nhất làm cho trí tuệ trở thành ``thiểu năng''.''
		\item ``Tất cả mọi biện pháp muốn thành công, đều phụ thuộc vào cái tâm thực hiện nó.''
		\item ``Các em có thể trở thành những người lao động chân chính, những nhà kỹ thuật có chuyên môn giỏi, những nhà nghiên cứu thành công, những doanh nghiệp tầm cỡ, những nhà lãnh đạo xuất sắc, $\ldots$ nhưng trước hết phải là những người tử tế.''
		\item ``Công việc gì có thể làm ngày hôm nay thì đừng để đến ngày mai.''
		\item ``Không lao động, không có sáng tạo. Người lười lao động chắc chắn không làm việc gì thành công.''
		\item ``Ai cũng vào đại học là lạc hậu.''
		\item ``Mỗi đưa trẻ đều có điểm mạnh \& điểm yếu. Nghệ thuật làm cha làm mẹ là biết cách khuyến khích, khen ngợi, nhưng không đề cao quá đáng những điểm mạnh của con mình, mặt khác cần khắc phục mà không vùi dập những điểm yếu của nó.''
		\item ``1 đứa trẻ kiêu căng, tự phụ hoặc 1 đứa trẻ tự ti, sợ hãi không phải là mục tiêu giáo dục của chúng ta.''
	\end{itemize}
	\item {\sc John Dryden.}
	\begin{itemize}
		\item ``Lương tâm chính là sức mạnh của con người.''
	\end{itemize}
	\item {\sc W. Faulkner.}
	\begin{itemize}
		\item ``Don't bother to just be better than your comtemporaries or predecessors. Try to be better than yourself.''
	\end{itemize}
	\item {\sc Mahatma Gandhi.}
	\begin{itemize}
		\item ``You must be the change you wish to see in the world.''
	\end{itemize}
	\item {\sc Vincent van Gogh.}
	\begin{itemize}
		\item ``The way to succeed is to keep your courage \& patience, \& to work energetically.''
	\end{itemize}
	\item {\sc Isaac Newton.}
	\begin{itemize}
		\item Letter to Robert Hooke, 1675: ``If I have seen further it is by standing on the shoulders of giants.''
	\end{itemize}	
	\item {\sc Georgia O'Keeffe.}
	\begin{itemize}
		\item ``Details are confusing. It is only by selection, by elimination, by emphasis that we get at the real meaning of things.''
	\end{itemize}
	\item {\sc M.L. Parashar.}
	\begin{itemize}
		\item ``If God is not in mathematics, then where is He?''\footnote{``Tat Twam Asi'', Isha Upanishad.}
	\end{itemize}
	\item {\sc Samuel Paterson.}
	\begin{itemize}
		\item ``Books, like friends, should be few \& well-chosen.'' -- Samuel Paterson, {\it Joineriana}
	\end{itemize}
	\item {\sc Sim\'eon Poisson.}
	\begin{itemize}
		\item ``Life is good for only 2 things, discovering mathematics \& teaching mathematics.'' -- \cite[p. v]{Gelca_Andreescu2017}
	\end{itemize}
	\item {\sc George P\'olya.}
	\begin{itemize}
		\item ``To understand mathematics means to be able to do mathematics. \& what does it mean doing mathematics? In the 1st place it means to be able to solve mathematical problems.''
	\end{itemize}
	\item {\sc Sartre.}
	\begin{itemize}
		\item ``Every word has consequences. Every silence, too.''
	\end{itemize}	
	\item {\sc William Shakespeare.}
	\begin{itemize}
		\item ``There is divinity in odd numbers, either in nativity, chance, or death.''
	\end{itemize}
	\item {\sc Rabindranath Tagore.}
	\begin{itemize}
		\item ``We come nearest to the great when we are great in humility.''
	\end{itemize}
	\item {\sc Luc Tartar.}
	\begin{itemize}
		\item ``As for me, once the doubt had entered my mind, what other choice did it leave me but to search for the truth, in all fields?'' -- \cite{Tartar2006}
		\item ``To an uninformed observer, it may seem that there is more interest in the Navier--Stokes equation nowadays, but many who claim to be interested show such a lack of knowledge about continuum mechanics that one may wonder about such a superficial attraction. {\it Could one of the Clay Millennium Prizes be the reason behind this renewed interest?} Reading the text of the conjectures to be solved for winning that particular prize leaves the impression that the subject was not chosen by people interested in continuum mechanics, as the selected questions have almost no physical content.'' -- \cite[Preface, p. vii]{Tartar2006}		
		\item ``One should learn to distinguish between a mathematical property of an equation \& a conjecture that some property holds which one guesses from the belief that the equation corresponds to a physical problem. One should learn about which defects are already known concerning how a mathematical model describes physical reality, but one should not forget that a mathematical model which is considered obsolete from the physical point of view may still be useful for mathematical reasons. I often wonder why so many forget to mention the defects of the models that they study.'' -- \cite[Preface, p. vii]{Tartar2006}		
		\item ``If one had used the word ``turbulence'' to make the donator believe that he would be giving one million dollars away for an important realistic problem in continuum mechanics, why has attention been restricted to unrealistic domains without boundary (the whole space $\mathbb{R}^3$, or a torus for periodic solutions), as if one did not know that vorticity is created at the boundary of the domain? The problems seem to have been chosen in the hope that they will be solved by specialists of harmonic analysis, \& it has given the occasion to some of these specialists to help others in showing the techniques that they use, as in a recent book by \cite{Lemarie-Rieusset2016}'' -- \cite[Preface, p. viii]{Tartar2006}		
		\item ``Being a mathematician interested in science, \& having learnt more than most mathematicians about various aspects of mechanics \& physics, one reason for teaching various courses \& writing lecture notes is to help isolated researchers to learn about some aspects unknown to most mathematicians whom they could meet, or read. A consequence of this choice is then to make researchers aware that some who claim to work on problems of continuum mechanics or physics have forgotten to point out known defects of the models that they use.'' -- \cite[Preface, pp. viii--ix]{Tartar2006}
		\item ``Of course, I also suffer from the same disease of not having learnt enough\footnote{NQBH: i.e., imposter syndrome obviously.}, but my hope is that by explaining what I have already understood \& by showing how to analyze \& criticize classical models, many will acquire my understanding \& a few will go much further than I have on the path of discovery.'' -- \cite[Preface, p. ix]{Tartar2006}
		\item ``I once heard my advisor, Jacques-Louis LIONS, mention that once the detailed plan of a book is made, the book is almost written, \& he was certainly speaking of experience as he had already written a few books at the time. He gave me the impression that he could write directly a very reasonable text, which he gave to a secretary for typing; maybe he then gave chapters to one of his students, as he did with me for one of his books, \& very few technical details had to be fixed. His philosophy seemed to be that there is no need to spend too much time polishing the text or finding the best possible statement, as the goal is to take many readers to the front of research, or to be more precise to one front of research, because in the beginning he changed topics every two or three years.'' -- \cite[Preface, p. ix]{Tartar2006}
		\item ``As for myself, I have not yet written a book, \& the main reason is that I am quite unable to write in advance a precise plan of what I am going to talk about, \& I have never been very good at writing even in my mother tongue (French). When I write, I need to read again \& again what I have already written until I find the text acceptable (and that notion of acceptability evolves with time \& I am horrified by my style of twenty years ago), so this way of writing is quite inefficient, \& makes writing a book prohibitively long. One solution would be not to write books, \& when I go to a library I am amazed by the number of books which have been written on so many subjects, \& which I have not read, because I never read much. {\it Why then should I add a new book?} However, I am even more amazed by the number of books which are not in the library, \& although I have access to a good inter-library loan service myself, I became concerned with how difficult it is for isolated students to have access to scientific knowledge (and I do consider mathematics as part of science, of course).'' -- \cite[Preface, p. ix]{Tartar2006}
		\item ``It is clear that fewer \& fewer students in industrialized countries are interested in studying mathematics, for various reasons, \& as a consequence more \& more mathematicians are likely to come from developing countries. It will therefore be of utmost importance that developing countries should not simply become a reservoir of good students that industrialized countries would draw upon, but that these countries develop a sufficiently strong scientific environment for the benefit of their own economy \& people, so that only a small proportion of the new trained generations of scientists would become interested in going to work abroad.'' -- \cite[Preface, p. ix]{Tartar2006}
		\item ``I have seen the process of decolonization at work in the early 1960s, \& I have witnessed the consequences of too hasty a transition, which was not to the benefit of the former colonies, \& certainly the creation of a scientific tradition is not something that can be done very fast. I see the development of mathematics as a good way to start building a scientific infrastructure, \& inside mathematics the fields that I have studied should play an important role, where mathematics interacts with continuum mechanics \& physics.'' -- \cite[Preface, pp. ix--x]{Tartar2006}
		\item ``I decided at that time to add some information that one rarely finds in courses of mathematics, something about the people who have participated in the creation of the knowledge related to the subject of the course. I had the privilege to study in Paris in the late 1960s, to have great teachers like Laurent SCHWARTZ \& Jacques-Louis LIONS, \& to have met many famous mathematicians. This has given me a different view of mathematics than the one that comes from reading books \& articles, which I find too dry, \& I have tried to give a little more life to my story by telling something about the actors; for those mathematicians whom I have met, I have used their 1st names in the text, \& I have tried to give some simple biographical data for all people quoted in the text, in order to situate them, both in time \& in space.'' -- \cite[Preface, p. x]{Tartar2006}
		\item ``I observe that there have been efficient schools in some areas of mathematics at some places \& at some moments in time, \& when I was a student in Paris in the late 1960s, Jacques-Louis LIONS had mentioned that Moscow was the only other place comparable to Paris for its concentration of mathematicians. Although the conditions might be less favorable outside important centers, I want to think that a lot of good work could be done elsewhere, \& my desire is that my lecture notes may help isolated researchers participate more in the advance of scientific knowledge. A few years ago, an Italian friend, GianPietro DEL PIERO, told me that he had taught for a few months in Somalia, \& he mentioned that one student had explained to him that he should not be upset if some of the students fell asleep during his lectures, because the reason was not their lack of interest in the course, but the fact that sometimes they had eaten nothing for a few days. It was by thinking about these courageous students who, despite the enormous difficulties that they encounter in their everyday life, are trying to acquire some precious knowledge about mathematics, that I devised my plan to write lecture notes \& make them available to all, wishing that they could arrive freely to isolated students \& researchers, working in much more difficult conditions than those having access to a good library, or in contact with good teachers. I hope that publishing this revised version will have the effect that it will reach many libraries scattered around the world, where isolated researchers have access.'' -- \cite[Preface, p. xi]{Tartar2006}
		\item ``I hope that my lack of organizational skills will not bother the readers too much. I consider teaching courses like leading groups of newcomers into countries which are often unknown to them, but not unknown to me, as I have often wandered around; some members of a group who have already read about the region or have been in other expeditions with guides more organized than me might feel disoriented by my choice of places to visit, \& indeed I may have forgotten to show a few interesting places, but my goal is to familiarize the readers with the subject \& encourage them to acquire an open \& scientific point of view, \& not to write a definitive account of the subject.'' -- \cite[Preface, p. xii]{Tartar2006}
		\item ``There are results which are repeated, but it is inevitable in a real course that one should often recall results which have already been mentioned. There are also results which are mentioned without proof, \& sometimes they are proven later but sometimes they are not, \& if no references are given, one should remember that I have been trained as a mathematician, \& that my statements without proofs have indeed been proven in a mathematical sense, because if they had not I would have called them conjectures instead;\footnote{Some people like to talk of pure mathematics versus applied mathematics, but I do not think that such a distinction is accurate, as I mentioned in the introduction of an article for a conference at \'Ecole Polytechnique (Palaiseau, France) in the fall of 1983, but because that introduction was cut by political censors, it is worth repeating that for what concerns different parts of mathematics there are those which I know, those which I do not know well but think that they could be useful to me, \& those which I do not know well but do not see how they could be useful to me, \& all this evolves with time, so I finally wonder if it is reasonable to classify mathematics as being pure or applied. I consider myself as an ``applied'' mathematician, although I give it a French meaning (a mathematician interested in other fields of science), opposed to a British meaning (a specialist of continuum mechanics, allowed to use an incomplete mathematical proof without having to call the result a conjecture), \& in French universities, applied mathematicians in the British style are found in departments of m\'ecanique. Probably for the reason of funding, which strangely enough is given more easily to people who pretend to do applied research, some who have studied to become mathematicians practice the art of using words which make naive people wrongly believe that they know continuum mechanics or physics, \& I find this attitude potentially dangerous for the university system.} however, I am also human \& my memory is not perfect \& I may have made mistakes. I think that the right attitude in mathematics is to be able to explain all the statements that one makes, but in a course one has to assume that the reader already has some basic knowledge of mathematics, \& some proofs of a more elementary nature are omitted. Here \& there I mention a result that I have heard of, but for which I never read a proof or did not make up my own proof, \& I usually say so. If many proofs are mine it does not necessarily mean that I was the 1st to prove the corresponding result, but that I am not aware of a prior proof, maybe because I never read much. Actually, my advisor mentioned to me that it is useful to read only the statement of a theorem \& one should read the proof only if one cannot supply one.'' -- \cite[Preface, pp. xii--xiii]{Tartar2006}
		\item ``My personal mathematical training has been in functional analysis \& PDEs, starting at \'Ecole Polytechnique, Paris, France, where I had two great teachers, Laurent SCHWARTZ \& Jacques-Louis LIONS. Having studied there in order to become an engineer, but having had to change my orientation once I had been told that such a career required administrative skills (which I lack completely), I opted for doing research in mathematics with an interest in other sciences \& I asked Jacques-Louis LIONS to be my advisor, \& it was normal that once I had been taught enough on the mathematical side, I would apply my improved understanding to investigating questions of continuum mechanics \& physics which I had heard about as a student, \& to developing the new mathematical tools which are necessary for that.'' -- \cite[Preface, p. xiii]{Tartar2006}
		\item ``In my lectures I also try to teach mathematicians about the defects of the models used, but I want to apologize for some of the words which I use, which may have offended some. I have a great admiration for the achievements of physicists \& engineers\footnote{I am not mentioning biologists \& chemists because biology was not part of my studies, \& although I have learnt some chemistry, I only hope to understand it in a better way once my program for understanding continuum mechanics \& physics has progressed enough.} during the last century, \& a lot of the improvements in our lives result from their understanding, which is so different than the type of understanding that mathematicians are trained to achieve. If I write that something that they say does not make any sense, it is not a criticism towards physicists or engineers, who are following the rules of their profession, but it is a challenge to my fellow mathematicians that there is something there that mathematicians ought to clarify.'' -- \cite[Preface, p. xiii]{Tartar2006}
		\item ``Most problems are much too academic from the point of view of continuum mechanics, because the model used by Jean LERAY is too crude to be meaningful, \& the difficulties of the open questions are merely of a technical mathematical nature. Also, Jean LERAY unfortunately called turbulent the weak solutions that he was seeking, \& it must be stressed that turbulence is certainly not about regularity or lack of regularity of solutions, nor about letting time go to infinity either.'' -- \cite[Introduction, p. xv]{Tartar2006}
		\item ``It seems to have become my trade mark among mathematicians, that I do not want to lie about the usefulness of models when some of their defects have already been pointed out. This is obviously the way that any scientist is supposed to behave, but in explaining why I have found myself so isolated \& stubborn in maintaining that behavior, I have often invoked a question of religious training.'' -- \cite[Introduction, p. xvi]{Tartar2006}
		\item ``I am not good at following plans.'' -- \cite[Introduction, p. xvi]{Tartar2006}
		\item ``I opted for describing the general techniques for nonlinear partial differential equations that I had developed, {\it homogenization, compensated compactness} \& {\it H-measures}; there are obviously many important situations where they should be useful, \& I found it more important to teach them than to analyze in detail some particular models for which I do not feel yet how good they are (which means that I suspect them to be quite wrong). Regularly, I was trying to explain why what I was teaching had some connection with questions about {\it fluids}.'' -- \cite[Introduction, pp. xvi--xvii]{Tartar2006}
		\item ``It goes with my philosophy to {\it explain the origin of mathematical ideas} when I know about them, \& as my ideas are often badly attributed, I like to mention {\it why \& when I had introduced an idea}.'' -- \cite[Introduction, p. xvii]{Tartar2006}
		\item ``I have also tried to {\it encourage mathematicians to learn more about continuum mechanics \& physics}, listening to the specialists \& then trying to put these ideas into a sound mathematical framework.'' -- \cite[Introduction, p. xvii]{Tartar2006}
		\item ``The framework of functional analysis is not just a change of language, because it is crucial for understanding the point of view that I developed in the 1970s for relating what happens at a macroscopic level from the description at a microscopic{\tt/}mesoscopic level, using convergences of weak type (and not just weak convergences), which is quite a different idea than the game of using ensemble averages, which destroys the physical meaning of the problems considered.]'' -- \cite[Introduction, p. xvii]{Tartar2006}
	\end{itemize}
	\item {\sc Desmond Tutu.}
	\begin{itemize}
		\item ``My father always used to say, ``Don't raise your voice. Improve your argument.''
	\end{itemize}
	\item {\sc Mike Tyson.}
	\begin{itemize}
		\item ``Another thing that freaks me out is time. Time is like a book. You have a beginning, a middle \& an end. It's just a cycle.''
		\item ``As long as we persevere \& endure, we can get anything we want.''
		\item ``Everybody's got plans $\ldots$ until they get hit.''
		\item ``Everyone has a plan 'till they get punched in the mouth.''
		\item ``God lets everything happen for a reason. It's all a learning process, \& you have to go from 1 level to another.''
		\item ``Greatness is not guarding yourself from the people; greatness is being accepted by the people.''
		\item ``I'm a dreamer. I have to dream \& reach for the stars, \& if I miss a star then I grab a handful of clouds.''
		\item ``I'm a Muslim, but I think Jesus would have a drink with me. He would be cool. He would talk to me.''
		\item ``I'm in trouble because I'm normal \& slightly arrogant. A lot of people don't like themselves \& I happen to be totally in love with myself.''
		\item ``I'm not Mother Teresa, but I'm not Charles Manson, either.''
		\item ``I'm not much for talking. You know what I do. I put guys in body bags when I'm right.''
		\item ``I'm just like you. I enjoy the forbidden fruits in life, too.''
		\item ``I could feel his muscle tissues collapse under my force. It's ludicrous these mortals even attempt to enter my realm.''
		\item ``I don't react to a tragic happening anymore. I took so many bad things as a kid \& some people think I don't care about anything. It's just too hard for me to get emotional. I can't cry no more.''
		\item ``I don't try to intimidate anybody before a fight. That's nonsense. I intimidate people by hitting them.''
		\item ``I don't understand why people would want to get rid of pigeons. They don't bother no one.''
		\item ``I feel like sometimes that I was not meant for this society.''
		\item ``I've lived places these guys can't defecate\footnote{{\bf defecate} [v] ({\it British English also} {\bf defaecate}) [intransitive] ({\it formal}) to get rid of solid waste from your body through your bowels.} in.''
		\item ``I love to hit people. I love to.''
		\item ``I'll go back \& take what the people own me.''
		\item ``Real freedom is having nothing. I was freer when I didn't have a cent.''
		\item ``I just want to be humble at all times.''
		\item ``I just want to conquer people \& their souls.''
		\item ``I try to catch them right on the tip of this nose, because I try to punch the bone into the brain.''
		\item ``I think I'll take a bath in his blood.''
		\item ``It's good to know how to read, but it's dangerous to know how to read \& not how to interpret what you're reading.''
		\item ``My biggest weakness is my sensitivity. I am too sensitive a person.''
		\item ``When I fight someone, I want to break his will. I want to take his manhood. I want to rip out his heart \& show it to him.''
		\item ``When I was in prison, I was wrapped up in all those deep books. That Tolstoy crap -- people shouldn't read that stuff.''
		\item ``When Jesus comes back, these crazy, greedy, capitalistic men are gonna kill him again.''
		\item ``When you see me smash somebody's skull, you enjoy it.''
	\end{itemize}
	\item {\sc Lao Tzu.}
	\begin{itemize}
		\item ``The journey of a thousand miles begins with one step.''
	\end{itemize}	
	\item {\sc Voltaire.}
	\begin{itemize}
		\item ``Mặt người khác nhau, tâm người còn khác hơn.''
	\end{itemize}
\end{enumerate}

%------------------------------------------------------------------------------%

Source: \href{http://www.greatexpectations.org/}{Great  Expectations}.

\subsection{Quotes on Citizenship}

\begin{enumerate}
	\item ``A house divided against itself cannot stand.'' -- Abraham Lincoln
	\item ``This city is what it is because our citizens are what they are.'' -- Plato
	\item ``United we stand, divided we fall.'' -- Aesop
	\item ``We must all live together as brothers or perish alone as fools.'' -- Martin Luther King Jr.
	\item ``Never doubt that a small group of thoughtful, committed people can change the world. Indeed, it is the only thing that ever has.'' -- Margaret Mead
	\item ``Citizenship is the chance to make a difference to the place where you belong.'' -- Charles Handy
	\item ``There can be no daily democracy without daily citizenship.'' -- Ralph Nader
	\item ``\& as we let our light shine, we unconsciously give other people permission to do the same.'' -- Nelson Mandela
	\item ``Citizenship consists in the service of the country.'' -- Jawaharlal Nehru
\end{enumerate}

\subsection{Quotes on Compassion}

\begin{enumerate}
	\item ``People will forget what you said, people will forget what you did, but people will never forget how you made them feel.'' -- Maya Angelou
	\item ``What do we live for, if it is not to make life less difficult for each other?'' -- George Eliot (Mary Ann Evans)
	\item ``Small deeds done are better than great deeds planned.'' -- Peter Marshall
	\item ``Never look down on anybody unless you're helping them up.'' -- The Reverend Jesse Jackson
	\item ``A good heart is better than all the heads in the world.'' -- Edward George Bulwer-Lytton
	\item ``No act of kindness, however small, is ever wasted.'' -- Aesop
\end{enumerate}

\subsection{Quotes on Cooperation}

\begin{enumerate}
	\item ``Teamwork divides the task \& multiples the success.'' -- Author Unknown
	\item ``Team work
	
	Coming together is a beginning.
	
	Keeping together is progress.
	
	Working together is success.'' -- Henry Ford
	\item ``If you want to go fast, go alone. If you want to go far, go together.'' -- Author Unknown
	\item ``United we stand, divided we fall.'' -- Author Unknown
	\item ``Only strength can cooperate.'' -- Dwight D. Eisenhower
	\item ``Alone we can do little; together we can do so much.'' -- Helen Keller
	\item ``If we don't all row the boat won't go.'' -- Author Unknown
	\item ``Together ordinary people can achieve extraordinary results.'' -- Becka Schoettle
	\item ``No man is an island, entire of itself; every man is a piece of the continent.'' -- John Donne
\end{enumerate}

\subsection{Quotes on Courtesy}

\begin{enumerate}
	\item ``Politeness costs nothing \& gains everything.'' -- Lady Montague
	\item ``Courtesy is as much a mark of a gentleman as courage.'' -- Theodore Roosevelt
	\item ``The true greatness of a person, in my view, is evident in the way he or she treats those with whom courtesy \& kindness are not required.'' -- Joseph B. Wirthlin
	\item ``All doors open to courtesy.'' -- Thomas Fuller
	\item ``A tree is known by its fruit; a man by his deeds. A good head is never lost; he who sows courtesy reaps friendship, \& he who plants kindness gathers love.'' -- Saint Basil
	\item ``Courtesies of a small \& trivial character are the ones which strike deepest in the grateful \& appreciating heart.'' -- Henry Clay
	\item ``As we are, so we do; \& as we do, so is it done to us; we are the builders of our fortunes.'' -- Ralph Waldo Emerson
	\item ``Talk to strangers politely $\ldots$ Every friend you have now was once a stranger, although not every stranger becomes a friend.'' -- Israelmore Ayivor
	\item ``Not only the footwear, wear also the courtesy, respect, \& gratitude in your heart while stepping out of home.'' -- Rupali Desai
	\item ``Politeness is a desire to be treated politely, \& to be esteemed polite oneself.'' -- Francois de La Rochefoucauld
\end{enumerate}

\subsection{Quotes on Curiosity}

\begin{enumerate}
	\item ``Judge a man by his questions, rather than his answers.'' -- Voltaire
	\item ``Better to ask a question than to remain ignorant.'' -- Author Unknown
	\item ``I never learned anything talking. I only learn things when I ask questions.'' -- Lou Holz
\end{enumerate}

\subsection{Quotes on Effort}

\begin{enumerate}
	\item ``Happiness lies in the joy of achievement \& the thrill of creative effort.'' -- Theodore Roosevelt
	\item ``Continuous effort -- not strength or intelligence -- is the key to unlocking our potential.'' -- Winston Churchill
	\item ``All things are difficult before they are easy.'' -- Thomas Fuller
	\item ``Knowing is not enough; we must apply. Willing is not enough; we must do.'' -- Johann Wolfgang von Goethe
	\item ``Hope is wishing for a thing to come true. Faith is believing it will come true. Work is making it come true.'' -- Dr. Norman Vincent Peale
	\item ``A man of words, \& not of deeds, is like a garden full of weeds.'' -- English Proverb
	\item ``Thinking well is wise; planning well wiser; doing well wisest \& best of all.'' -- Persian Proverb
	\item ``Enthusiasm is the mother of effort, \& without it nothing great was ever achieved.'' -- Ralph Waldo Emerson
	\item ``Just try to be the best you can be; never cease trying to be the best you can be. That's in your power.'' -- John Wooden
	\item ``If a task is once begun, never leave it till it's done. Be the labor great or small, do it well or not at all.'' -- Author Unknown
	\item ``If we were supposed to talk more than we listen, we would have 2 mouths \& 1 ear.'' -- Mark Twain
	\item ``No bees, No honey, No work, No money.'' -- Author Unknown
\end{enumerate}

\subsection{Quotes on ESPRIT de CORPS}

\begin{enumerate}
	\item ``Associate yourself with men of good quality if you esteem your own reputation, for `tis better to be alone than in bad company.'' -- George Washington
	\item ``We must all live together as brothers or perish alone as fools.'' -- Martin Luther King Jr.
	\item ``Stand with anybody that stands right. Stand with him while he is right \& part with him when he goes wrong.'' -- Abraham Lincoln
	\item ``Alone we can do little; together we can do so much.'' -- Helen Keller
	\item ``Teamwork divides the task \& multiples the success.'' -- Author Unknown
	\item ``Individual commitment to a group effort -- that is what makes a team work, a company work, a society work, a civilization work.'' -- Vince Lombardi
	\item ``Regardless of differences, we strive shoulder to shoulder $\ldots$ Teamwork can be summed up in 5 short words: ``We believe in each other.'' -- Author Unknown
	\item ``Never doubt that a small group of thoughtful, committed people can change the world. Indeed, it is the only thing that ever has.'' -- Margaret Mead
	\item ``In union there is strength.'' -- Aesop
\end{enumerate}

\subsection{Quotes on Friendship}

\begin{enumerate}
	\item ``Don't walk behind me; I may not lead.
	
	Don't walk in front of me; I may not follow.
	
	Just walk beside me \& be my friend.'' -- Albert Camus
	\item ``Friendship is not a big thing $\ldots$ it's a million little things.'' -- Author Unknown
	\item ``Truly great friends are hard to find, difficult to leave, \& impossible to forget.'' -- G. Randolf
	\item ``A friend is one that knows who you are, understands where you have been, accepts what you have become, \& still, gently allows you to grow.'' -- William Shakespeare
	\item ``True friends are those who dare to let you know your mistakes rather than agreeing with you over the wrong things.'' -- Author Unknown
	\item ``Don't expect your friend to be a perfect person. But, help your friend to become a perfect person. That's true friendship.'' -- Mother Teresa
	\item ``Be genuinely interested in people. Just try, \& you can like almost everyone.'' -- Author Unknown
	\item ``Treat people as if they were what they ought to be \& you help them become what they are of capable of being.'' -- Johann Wolfgang von Goethe
	\item 
\end{enumerate}

\subsection{Quotes on Honesty}

\begin{enumerate}
	\item ``Honesty is the 1st chapter in the book of wisdom.'' -- Thomas Jefferson
	\item ``A half truth is a whole lie.'' -- Yiddish Proverb
	\item ``Every lie is 2 lies, the lie we tell others \& the lie we tell ourselves to justify it.'' -- Robert Brault
	\item ``Integrity is telling myself the truth. \& honesty is telling the truth to other people.'' -- Spencer Johnson
	\item ``Honesty is the best policy. If I lose mine honor, I lose myself.'' -- William Shakespeare
	\item ``Honesty is more than not lying. It is truth telling, truth speaking, truth living, \& truth loving.'' -- James E. Faust
	\item ``No legacy is so rich as honesty.'' -- William Shakespeare, {\it All's Well That Ends Well}
	\item ``It takes strength \& courage to admit the truth.'' -- Rick Riordan, {\it The Red Pyramid}
	\item ``Honest people don't hide their deeds.'' -- Emily Bront\"e, {\it Wuthering Heights}
	\item ``When you tell a lie, you steal someone's right to the truth.'' -- Khaled Hosseini, {\it The Kite Runner}
\end{enumerate}

\subsection{Quotes on Integrity}

\begin{enumerate}
	\item ``In the matters of style, swim with the current. In matters of principle, stand like a rock.'' -- Thomas Jefferson
	\item ``If it is not right, do not do it; if it is not true do not say it.'' -- Marcus Aurelius
	\item ``Integrity is telling myself the truth. \& honesty is telling the truth to other people.'' -- Spencer Johnson
	\item ``Our deeds determine us, as much as we determine our deeds.'' -- George Eliot
	\item ``There can be no friendship without confidence, \& no confidence without integrity.'' -- Samuel Johnson
	\item ``Have the courage to say no. Have the courage to face the truth. Do the right thing because it is right. These are the magic keys to living your life with integrity.'' -- W. Clement Stone
	\item ``Dependability, integrity, the characteristic of never knowingly doing anything wrong, that you would never cheat anyone, that you would give everybody a fair deal. Character is a sort of all-inclusive thing. If a man has character, everyone has confidence in him.'' -- Omar Nelson Bradley
	\item ``Lead your life so you wouldn't be ashamed to sell the family parrot to the town gossip.'' -- Will Rogers
	\item ``Day by day, your choices, your thoughts, your actions fashion the person you become. Your integrity determines your destiny.'' -- Heraclitus
	\item ``Wisdom is knowing the right path to take $\ldots$ Integrity is taking it.'' -- Author Unknown
\end{enumerate}

\subsection{Quotes on Perseverance}

\begin{enumerate}
	\item ``It's not where you start, it's where you end up. You haven't failed until you've stopped trying.'' -- Unknown
	\item ``The man who removes mountains begins by carrying away small stones.'' -- Chinese Proverb
	\item ``We are never in the land of done.'' -- Dr. Thomas Dooley's letter to a young doctor
	\item ``The difficult we do immediately; the impossible takes a little longer.'' -- Air Force motto
	\item ``Some people see things \& say, ``Why?'' I see things \& ask, ``Why not?'' -- Robert F. Kennedy
	\item ``Many of life's failures are people who did not realize how close they were to success when they gave up.'' -- Thomas Edison
	\item ``It's not that I'm so smart, it's just that I stay with problems longer.'' -- Albert Einstein
	\item ``If you can't fly, then run. If you can't run, then walk. If you can't walk, then crawl. But whatever you do, you have to keep moving forward.'' -- Martin Luther King Jr.
	\item ``Never, never, never give up.'' -- Winston Churchill
	\item ``It always seems impossible until it's done.'' -- Nelson Mandela
	\item ``The right angle to approach a difficult problem is the TRY-angle.'' -- Unknown
\end{enumerate}

\subsection{Quotes on Problem--Solving}

\begin{enumerate}
	\item ``A problem is a chance for you to do your best.'' -- Duke Ellington
	\item ``We cannot solve our problems with the same level of thinking that created them.'' -- Albert Einstein
	\item ``A problem well stated is a problem half solved'' -- John Dewey
	\item ``A sum can be put right: but only by going back till you find the error \& working it afresh from that point, never by simply going on.'' -- C. S. Lewis, {\it The Great Divorce}
	\item ``If your only tool is a hammer then every problem looks like a nail.'' -- Abraham Maslow
	\item ``Too often we give our children answers to remember rather than problems to solve.'' -- Roger Lewin
	\item ``There's no use talking about the problem unless you talk about the solution.'' -- Betty Williams
\end{enumerate}

\subsection{Quotes on Respect}

\begin{enumerate}
	\item ``You can demand courtesy, but you have to earn respect.'' -- Lawrence Goldstone
	\item ``I will speak ill if no man \& speak the good I know of everybody.'' -- Benjamin Franklin
	\item ``I'm not concerned with your liking me or disliking me $\ldots$ All I ask is that you respect me as a human being.'' -- Jackie Robinson
	\item ``Respect for ourselves guides our morals, respect for others guides our manners.'' -- Laurence Sterne
	\item ``A person's a person, no matter how small.'' -- Dr. Seuss
	\item ``That you may retain your self-respect, it is better to displease the people by doing what you know is right, than to temporarily please them by doing what you know is wrong.'' -- William J. H. Boetcker
	\item ``Respect yourself \& others will respect you.'' -- Confucius
	\item ``I cannot conceive of a greater loss than the loss of one's self-respect.'' -- Mahatma Gandhi
	\item ``Set the standard! Stop expecting others to show you love, acceptance, commitment, \& respect when you don't even show that to yourself.'' -- Steve Maraboli
	\item ``Be a reflection of what you'd like to see in others! If you want love, give love. If you want honesty, give honesty. If you want respect, give respect. You get in return what you give!'' -- Author Unknown
\end{enumerate}

\subsection{Quotes on Responsibility}

\begin{enumerate}
	\item ``You cannot escape the 
\end{enumerate}

%------------------------------------------------------------------------------%

\section{Video}

\subsection{Mathematics}

\begin{enumerate}
	\item \href{https://www.youtube.com/watch?v=yEVlCZTqht8}{Fields Medal -- Peter Scholze -- ICM2018}.
	\item \href{https://www.youtube.com/watch?v=G0rrnx8SaDI}{Fields Medal -- Alessio Figalli -- ICM2018}. 
	\item \href{https://www.youtube.com/watch?v=p7iXVQxvM00}{Fields Medal -- Akshay Venkatesh -- ICM2018}.
	\item \href{https://www.youtube.com/watch?v=mKLkzQqxmFE}{Fields Medal Winner 2018 Caucher Birkar}.
	\item \href{https://www.youtube.com/watch?v=yalkCpigHTA}{Fields Medal: Artur Avila}.
	\item \href{https://www.youtube.com/watch?v=yO8lQWb6TZ4}{Simons Foundation{\tt/}Fields Medal: June Huh}
\end{enumerate}

\subsection{\href{https://www.ted.com/}{TED}}

\begin{enumerate}
	\item {\sc Adam Grant.} \href{https://www.youtube.com/watch?v=YyXRYgjQXX0}{YouTube{\tt/}Are you a giver or a taker?} \href{https://www.ted.com/talks/adam_grant_are_you_a_giver_or_a_taker}{TED{\tt/}Are you a giver or a taker?} \href{https://www.youtube.com/watch?v=fxbCHn6gE3U}{YouTube{\tt/}The surprising habits of original thinkers}
	\item \href{https://www.youtube.com/watch?v=lEXBxijQREo}{YouTube{\tt/}Nicole Avena, How sugar affects the brain}. \href{https://ed.ted.com/lessons/how-sugar-affects-the-brain-nicole-avena}{TED{\tt/}Nicole Avena, How sugar affects the brain}
	\item \href{https://www.youtube.com/watch?v=KUKgJsvoDUk}{Adam Benn. Write Well. Start Writing Now}. TEDxVitoriaGasteiz
	\item \href{https://www.youtube.com/watch?v=c0KYU2j0TM4}{Susan Cain: The power of introverts}
\end{enumerate}

%------------------------------------------------------------------------------%

\section{Cooking -- Nấu Ăn}

\subsection{Cooking Tools}

\begin{question}
	Nên mua bếp nướng điện loại nào?
\end{question}

\subsection{Cooking recipes}
{\sc Donald Erwin Knuth}'s favorite quote in AoCP is from a cooking book.
\begin{quote}
	{\it``Here is your book, the one your thousands of letters have asked us to publish. It has taken us years to do, checking \& rechecking countless recipes to bring you only the best, only the interesting, only the perfect. Now we can say, without a shadow of a doubt, that every single 1 of them, if you follow the directions to the letter, will work for you exactly as well as it did for us, even if you have never cooked before.''} -- {\sc McCall}'s Cookbook (1963)
\end{quote}

%------------------------------------------------------------------------------%

\section{Game}
Actually, I only play 1 game, DotA 2, which is good for training strategies \& concentration.

\subsection{DotA 2}
I have used DotA 2 as an ``intermediate break'': When I feel tired with words \& research, I will take some notes \& pause my work, I play some DotA 2 matches until I lose (if I win, I continue to find another {\sc mmr} match). The reason is simple: The toxicity of DotA 2 community \& stupidity of some DotA 2 players, especially in some matches I have done all my best to carry them, make me intellectually bored with gaming \& thus motivate me to get back to my research work. The boring--stimulating loop{\tt/}cycle will continue until the end of the day everyday.

\begin{itemize}
	\item \href{https://www.youtube.com/watch?v=d9vmHp-4f3k}{YouTube{\tt/}NQBH{\tt/}NQBH's 4th Night Stalker DotA 2 Rampage}.
\end{itemize}

%------------------------------------------------------------------------------%

\section{Miscellaneous}

\subsection{Joke}

\begin{enumerate}
	\item ``My penis was in the Guinness Book of World Records. But then the librarian asked me to take it out.'' \href{https://9gag.com/gag/aPZPm6q}{9gag{\tt/}funny content}
	\item Never stick your dick in any toxic crazy bitch. Because if you stick, your dick will be poisoned. Love your own dick!
\end{enumerate}

\subsection{Jump Rope -- Dây Nhảy}
{\sf Categorization -- Phân loại.} Dây nhảy tốc độ (speed rope for boosting agility), dây nhảy boxing (heavy rope for building \& boosting strength).
\begin{enumerate}
	\item \href{https://www.rushathletics.co.uk}{Rush Athletics UK}. I bought these items via Amazon when I was in Germany \& Austria, hard to buy them from Vietnam.
	\begin{itemize}
		\item \href{https://www.rushathletics.co.uk/collections/ropes/products/legacy-jump-rope}{Legacy Jump Rope}.
		\item \href{https://www.rushathletics.co.uk/collections/ropes/products/icon-freestyle-rope}{Icon Freestyle Rope}.
		\item \href{https://www.rushathletics.co.uk/collections/ropes/products/rush-athletics-speed-rope-ghost-white}{Rush Athletics Speed Rope -- Limited Edition -- Ghost White}.
	\end{itemize}	
	\item \href{https://shopee.vn/matvunhayday?categoryId=100637&entryPoint=ShopByPDP&itemId=14803106033}{Shopee{\tt/}Mật Vụ Nhảy Dây}.
	\begin{itemize}
		\item Dây nhảy PVC 5 mm Mật Vụ Nhảy Dây cao cấp -- Speed rope.
		\item Dây nhảy PVC 6 mm Mật Vụ Nhảy Dây cao cấp -- Speed rope.
	\end{itemize}
\end{enumerate}

\subsection{Laptop}

\begin{enumerate}
	\item {\sc Dell XPS 15 i7.} 15.6 inches.
	\item {\sc MSI Katana GF76 11UC i7.} 17.3 inches.
\end{enumerate}
I really want to have a Workstation, especially the latest {\sc Dell Precision}, \& 1 of the (stupid) reasons is that I love the term ``precision''.

\subsection{Software}

\begin{enumerate}
	\item Sublime Text 4
	\item Sublime Merge (I prefer to use Git on Terminal.)
	\item Vanilla \TeX Live 2022
\end{enumerate}

%------------------------------------------------------------------------------%

%------------------------------------------------------------------------------%

\section{Psychology}

\subsection{\href{https://www.youtube.com/channel/UCLb_x8TboHIoO0F7uYSl4Ww}{YouTube{\tt/}WordToTheWise}}

\subsubsection{\href{https://www.youtube.com/playlist?list=PLi3n6z1voQ1arCpQ89izFMwk7lsFrFnq0}{YouTube{\tt/}Jordan Peterson Motivational Videos \& Life Advice}}
\begin{enumerate}
	\item \href{https://www.youtube.com/watch?v=MUmKnW1j2Xs}{\textsc{Look up, Move ahead} - Powerful Motivational Video $|$ Jordan Peterson}    
	\begin{quotation}
		``{\it What should move forward in time with me? And what should be let go as if it's deadwood?} - Jordan Peterson
	\end{quotation}
	\item \href{https://www.youtube.com/watch?v=PrIrZRd0pGE}{\textsc{Stumble towards the Light} - Powerful Motivational Video $|$ Jordan Peterson}    
	\begin{quotation}
		``{\it Maybe the star that Geppetto wished on was the wrong damn star. But at least it was a star. At least it was in the sky. At least it moved him forward}.'' - Jordan Peterson
	\end{quotation}
	\item \href{https://www.youtube.com/watch?v=L0FCwYKkZ4o}{You Need To Pay Attention! $|$ Jordan Peterson $|$ Best Life Advice}    
	\begin{quotation}
		``{\it Habitable order is generated by spoken truth. I think that's the truest thing I know}.'' - Jordan Peterson
	\end{quotation}
	\item \href{https://www.youtube.com/watch?v=LjIAzKo62MQ}{\textsc{Look Where You Least Want to} - Powerful Life Advice $|$ Jordan Peterson}    
	\begin{quotation}
		``{\it That which you most need will be found where you least want to look}.'' - Carl Jung
	\end{quotation}
	\item \href{https://www.youtube.com/watch?v=_g6Aqrl3fFM}{Sanity $|$ Jordan Peterson $|$ Best Life Advice}
	\item \href{https://www.youtube.com/watch?v=3EGIlQyEwvE}{Dealing With Dark Times $|$ Jordan Peterson $|$ 12 Rules for Life $|$ Best Life Advice}    
	\begin{quotation}
		``{\it There's a big gap between tragedy \& hell}.'' - Jordan Peterson
	\end{quotation}
	\item \href{https://www.youtube.com/watch?v=zRYjVFFsi38}{The World Shifts Itself Around Your Aim $|$ Jordan Peterson $|$ Best Life Advice}    
	\begin{quotation}
		``{\it If you dare to do the most difficult thing that you can conceptualize, your life will work out better than it will if you do anything else}.'' - Jordan Peterson
	\end{quotation}
	\item \href{https://www.youtube.com/watch?v=bb9g9mtDHZo}{How To Beat Fear And Anxiety $|$ Jordan Peterson $|$ Powerful Life Advice}    
	\begin{quotation}
		``{\it To be fully self-conscious means that you're perfectly aware of your limitations \& how you might be hurt. And then to make the decisions to move forward into the unknown \& the land of the stranger anyways$\ldots$ That's one of the secrets to a good life}.'' - Jordan Peterson
	\end{quotation}
\end{enumerate}

%------------------------------------------------------------------------------%

\subsection{\href{https://www.youtube.com/channel/UCrBsO0_kjefBUQPNgaELGGw}{YouTube{\tt/}Zala Films}}
\begin{enumerate}
	\item \href{https://www.youtube.com/watch?v=ONvYPldXoZs}{I Want to Be a Mathematician: A conversation with Paul Halmos - trailer}
\end{enumerate}

%------------------------------------------------------------------------------%

\subsection{\href{https://www.youtube.com/channel/UCDok3oTo2SSWZt6I8sHuzVg}{YouTube{\tt/}astudyofeverything}}
\begin{enumerate}
	\item \href{https://www.youtube.com/watch?v=i0UTeQfnzfM}{Beauty Is Suffering [Part 1 - The Mathematician]}
	\begin{quotation}
		``{\it Suffering becomes beautiful when anyone bears great calamities with cheerfulness, not through insensibility but through greatness of mind}.'' - Aristotle
	\end{quotation}
\end{enumerate}

%------------------------------------------------------------------------------%

\subsection{\href{https://www.youtube.com/channel/UCnMLdlOoLICBNcEzjMLOc7w}{YouTube{\tt/}Rio ICM2018}}
\begin{enumerate}
	\item \href{https://www.youtube.com/watch?v=yEVlCZTqht8}{Fields Medal - Peter Scholze - ICM2018}
\end{enumerate}

%------------------------------------------------------------------------------%

\subsection{\href{https://www.youtube.com/channel/UCrN1lcGgsCB9axGjZjpOqiQ}{YouTube{\tt/}Web of Stories - Life Stories of Remarkable People}}
\begin{enumerate}
	\item \href{https://www.youtube.com/watch?v=75Ju0eM5T2c}{Donald Knuth - My advice to young people (93/97)}
\end{enumerate}

\subsection{\href{https://www.youtube.com/channel/UCtUId5WFnN82GdDy7DgaQ7w}{YouTube{\tt/}Better Ideas}}
\begin{enumerate}
	\item \href{https://www.youtube.com/watch?v=EQf8w8Ed5k0}{Why you can't get anything done?}
	\item \href{https://www.youtube.com/watch?v=I0biBk4Y8Qg}{Why you feel so stuck in life}
	\item \href{https://www.youtube.com/watch?v=wLzteVkyReA}{The worst self improvement mistake}
	\item \href{https://www.youtube.com/watch?v=Rim2rXIbVoA}{How video games are changing my life}
	\item \href{https://www.youtube.com/watch?v=W9qsxhhNUoU}{How to be miserable for the rest of your life}
	\item \href{https://www.youtube.com/watch?v=akRLUpQp5xo}{My morning routine}
	\item \href{https://www.youtube.com/watch?v=8XO53urLcvY}{How the algorithm controls your life}
	\item \href{https://www.youtube.com/watch?v=B-L8CYwglj4}{The mindset that's changing my life}
	\item \href{https://www.youtube.com/watch?v=DbxPgd9DKEY}{3 habits that boost mental clarity}
	\item \href{https://www.youtube.com/watch?v=HHCifdfI47M}{Why your ego is (slowly) ruining your life}
	\item \href{https://www.youtube.com/watch?v=rQoS_S9K464}{Why it's so hard to be happy}
	\item \href{https://www.youtube.com/watch?v=js2vfr96iAQ}{Why you're always tired}
	\item \href{https://www.youtube.com/watch?v=SHFAZv8PR_c}{Why you don't have enough money}
	\item \href{https://www.youtube.com/watch?v=nm7OMGjbCgc}{How I learned to make more friends}
	\item \href{https://www.youtube.com/watch?v=DEz7oJy37lI}{Why self improvement is ruining your life}
	\item \href{https://www.youtube.com/watch?v=yYWvUoN4yt8}{How overstimulation is ruining your life}
	\item \href{https://www.youtube.com/watch?v=6U06IfrhN6k}{The simple idea that changed my life}
	\item \href{https://www.youtube.com/watch?v=b8SRAKfP2J4}{How to stop quarantine from ruining your life}
	\item \href{https://www.youtube.com/watch?v=0h-IAlNjd4Q}{How the comfort zone is ruining your life}
	\item \href{https://www.youtube.com/watch?v=CkJGBM3rLLM}{The brain's hidden superpower}
	\item \href{https://www.youtube.com/watch?v=SIsCy663mz4}{The most important skill for improving your life}
	\item \href{https://www.youtube.com/watch?v=GtNEf9UyOLg}{A method for sticking to habits}
	\item \href{https://www.youtube.com/watch?v=-dYgnvrvQ3M}{Why your life is so boring}
\end{enumerate}

%------------------------------------------------------------------------------%

\section{Book \& Articles}

%------------------------------------------------------------------------------%

\section*{Library}
{\sf Websites to download, respectively, books \& scientific articles freely:} \url{https://libgen.is/}, \url{https://sci-hub.se/}.

Có rất nhiều sách liệt kê ở đây nhưng mình không mua. Đơn giản là hứng lên  thì liệt kê vào danh sách những sách \emph{có tiềm năng} để mua nhưng 1 thời gian sau phát hiện hướng viết không cần những sách đó nên thôi. Cứ liệt kê đã, mài dũa sau. Do it 1st, sharpen it later.

\subsection{Elementary STEM Book -- Sách STEM Sơ Cấp}

\subsubsection{Elementary Mathematics Book -- Sách Toán Sơ Cấp}

\paragraph{Grade 6}

\begin{enumerate}
	\item \cite{Binh_Toan_6_tap_1}. {\sc Vũ Hữu Bình}. {\it Nâng Cao \& Phát Triển Toán 6. Tập 1}.\hfill{\sf[done]}
	
	\item \cite{Binh_Toan_6_tap_2}. {\sc Vũ Hữu Bình}. {\it Nâng Cao \& Phát Triển Toán 6. Tập 2}.\hfill{\sf[done]}
	
	\item \cite{Binh_boi_duong_Toan_6_tap_1}. {\sc Vũ Hữu Bình, Đặng Văn Quân, Bùi Văn Tuyên}. {\it Bồi Dưỡng Toán 6. Tập 1}.\hfill{\sf[done]}
	
	\item \cite{Binh_boi_duong_Toan_6_tap_2}. {\sc Vũ Hữu Bình, Nguyễn Thị Quỳnh Anh, Phan Thanh Hồng, Bùi Văn Tuyên, Đặng Văn Tuyến, Nguyễn Thị Thanh Xuân}. {\it Bồi Dưỡng Toán 6. Tập 2}.\hfill{\sf[done]}
	
	\item \cite{TLCT_THCS_Toan_6_so_hoc}. {\sc Vũ Hữu Bình, Nguyễn Tam Sơn}. {\it Tài Liệu Chuyên Toán THCS Toán 6. Tập 1: Số Học}.\hfill{\sf[done]}
	
	\item \cite{TLCT_THCS_Toan_6_hinh_hoc}. {\sc Vũ Hữu Bình, Đàm Hiếu Chiến}. {\it Tài Liệu Chuyên Toán THCS Toán 6. Tập 2: Hình Học}.\hfill{\sf[done]}
	
	\item \cite{SGK_Toan_6_Canh_Dieu_tap_1}. {\sc Đỗ Đức Thái, Đỗ Tiến Đạt, Nguyễn Sơn Hà, Nguyễn Thị Phương Loan, Phạm Sỹ Nam, Phạm Đức Quang}. {\it Toán 6 Tập 1. Cánh Diều}.\hfill{\sf[done]}
	
	\item \cite{SGK_Toan_6_Canh_Dieu_tap_2}. {\sc Đỗ Đức Thái, Đỗ Tiến Đạt, Nguyễn Sơn Hà, Nguyễn Thị Phương Loan, Phạm Sỹ Nam, Phạm Đức Quang}. {\it Toán 6 Tập 2. Cánh Diều}.\hfill{\sf[done]}
	
	\item \cite{SBT_Toan_6_Canh_Dieu_tap_1}. {\sc Đỗ Đức Thái}. {\it Bài Tập Toán 6 Tập 1. Cánh Diều}.\hfill{\sf[done]}
	
	\item \cite{SBT_Toan_6_Canh_Dieu_tap_2}. {\sc Đỗ Đức Thái}. {\it Bài Tập Toán 6 Tập 2. Cánh Diều}.\hfill{\sf[done]}
	
	\item \cite{Trong_Toan_6}. {\sc Đặng Đức Trọng, Nguyễn Đức Tấn, Phạm Lê Quốc Thắng, Nguyễn Phúc Trường, Cao Hoàng Lợi}. {\it Bồi Dưỡng Năng Lực Tự Học Toán 6}.\hfill{\sf[reading]}
	
	\item \cite{Tuyen_Toan_6}. {\sc Bùi Văn Tuyên}. {\it Bài Tập Nâng Cao \& 1 Số Chuyên Đề Toán 6}.\hfill{\sf[done]}
\end{enumerate}

\paragraph{Grade 7}

\begin{enumerate}
	\item \cite{Binh_Toan_7_tap_1}. Vũ Hữu Bình. {\it Nâng Cao \& Phát Triển Toán 7. Tập 1}.\hfill{\sf[done]}
	
	\item \cite{Binh_Toan_7_tap_2}. Vũ Hữu Bình. {\it Nâng Cao \& Phát Triển Toán 7. Tập 2}.\hfill{\sf[done]}
	
	\item Vũ Hữu Bình. {\it Tài Liệu Chuyên Toán THCS Toán 7. Tập 1: Đại Số}.
	
	\item Vũ Hữu Bình. {\it Tài Liệu Chuyên Toán THCS Toán 7. Tập 2: Hình Học}.
	
	\item \cite{Binh_boi_duong_Toan_7_tap_1}. Vũ Hữu Bình, Nguyễn Xuân Bình, Đàm Hiếu Chiến, Phan Thanh Hồng, Nguyễn Thị Thanh Xuân. {\it Bồi Dưỡng Toán 7 Tập 1}.\hfill{\sf[reading]}
	
	\item \cite{Binh_boi_duong_Toan_7_tap_2}. Vũ Hữu Bình, Nguyễn Xuân Bình, Đàm Hiếu Chiến. {\it Bồi Dưỡng Toán 7 Tập 2}.\hfill{\sf[reading]}
	
	\item \cite{Hung_Mai_Toan_7_hinh_hoc}. Trần Quang Hùng, Đào Thị Hoa Mai. {\it Tuyển Chọn Các Chuyên Đề Bồi Dưỡng Học Sinh Giỏi Toán 7 Hình Học}.
	
	\item \cite{SGK_Toan_7_Canh_Dieu_tap_1}. Đỗ Đức Thái, Đỗ Tiến Đạt, Nguyễn Sơn Hà, Nguyễn Thị Phương Loan, Phạm Sỹ Nam, Phạm Đức Quang. {\it Toán 7 Tập 1. Cánh Diều}.\hfill{\sf[done]}
	
	\item \cite{SGK_Toan_7_Canh_Dieu_tap_2}. Đỗ Đức Thái, Đỗ Tiến Đạt, Nguyễn Sơn Hà, Nguyễn Thị Phương Loan, Phạm Sỹ Nam, Phạm Đức Quang. {\it Toán 7 Tập 2. Cánh Diều}.\hfill{\sf[done]}
	
	\item \cite{Trong_Toan_7}. Đặng Đức Trọng, Nguyễn Đức Tấn, Phạm Lê Quốc Thắng, Nguyễn Phúc Trường, Cao Hoàng Lợi, Nguyễn Thị Kiều Anh. {\it Bồi Dưỡng Năng Lực Tự Học Toán 7}.\hfill{\sf[reading]}
	
	\item \cite{Tuyen_Toan_7}. Bùi Văn Tuyên. {\it Bài Tập Nâng Cao \& 1 Số Chuyên Đề Toán 7}.\hfill{\sf[done]}
\end{enumerate}

\paragraph{Grade 8}

\begin{enumerate}
	\item \cite{Binh_Toan_8_tap_1}. Vũ Hữu Bình. {\it Nâng Cao \& Phát Triển Toán 8. Tập 1}.\hfill{\sf[reading]}
	
	\item \cite{Binh_Toan_8_tap_2}. Vũ Hữu Bình. {\it Nâng Cao \& Phát Triển Toán 8. Tập 2}.\hfill{\sf[reading]}
	
	\item Vũ Hữu Bình, Tôn Thân, Đỗ Quang Thiều. {\it Toán Bồi Dưỡng Học sinh Lớp 8 Đại Số}.
	
	\item Vũ Hữu Bình, Tôn Thân, Đỗ Quang Thiều. {\it Toán Bồi Dưỡng Học sinh Lớp 8 Hình Học}.
	
	\item \cite{Binh_boi_duong_Toan_8_tap_1}. Vũ Hữu Bình, Nguyễn Xuân Bình, Phan Thanh Hồng, Phạm Thị Bạch Ngọc, Nguyễn Thị Thanh Xuân. {\it Bồi Dưỡng Toán 8 Tập 1}.\hfill{\sf[reading]}
	
	\item \cite{Binh_boi_duong_Toan_8_tap_2}. Vũ Hữu Bình, Đàm Hiếu Chiến, Nguyễn Bá Đang, Phạm Thị Bạch Ngọc. {\it Bồi Dưỡng Toán 8 Tập 2}.\hfill{\sf[reading]}
	
	\item \cite{TLCT_THCS_Toan_8_dai_so}. Vũ Hữu Bình, Trần Hữu Nam, Phạm Thị Bạch Ngọc, Nguyễn Tam Sơn. {\it Tài Liệu Chuyên Toán THCS Toán 8. Tập 1: Đại Số}.\hfill{\sf[reading]}
	
	\item \cite{TLCT_THCS_Toan_8_hinh_hoc}. Vũ Hữu Bình, Văn Như Cương, Nguyễn Ngọc Đạm, Nguyễn Bá Đang, Trương Công Thành. {\it Tài Liệu Chuyên Toán THCS Toán 8. Tập 2: Hình Học}.\hfill{\sf[reading]}
	
	\item \cite{SGK_Toan_8_tap_1}. Phan Đức Chính, Tôn Thân, Vũ Hữu Bình, Trần Đình Châu, Ngô Hữu Dũng, Phạm Gia Đức, Nguyễn Duy
	Thuận. {\it Toán 8 Tập 1}.\hfill{\sf[done]}
	
	\item \cite{SGK_Toan_8_tap_2}. Phan Đức Chính, Tôn Thân, Nguyễn Huy Đoan, Lê Văn Hồng, Trương Công Thành, Nguyễn Hữu Thảo. {\it Toán 8 Tập 2}.\hfill{\sf[done]}
	
	\item \cite{SGK_Toan_8_Canh_Dieu_tap_1}. Đỗ Đức Thái, Lê Tuấn Anh, Đỗ Tiến Đạt, Nguyễn Sơn Hà, Nguyễn Thị Phương Loan, Phạm Sỹ Nam, Phạm Đức Quang. {\it Toán 8 Cánh Diều Tập 1}.\hfill{\sf[reading]}
	
	\item \cite{SGK_Toan_8_Canh_Dieu_tap_2}. Đỗ Đức Thái, Lê Tuấn Anh, Đỗ Tiến Đạt, Nguyễn Sơn Hà, Nguyễn Thị Phương Loan, Phạm Sỹ Nam, Phạm Đức Quang. {\it Toán 8 Cánh Diều Tập 2}.\hfill{\sf[reading]}
	
	\item \cite{Tuyen_Toan_8}. Bùi Văn Tuyên. {\it Bài Tập Nâng Cao \& 1 Số Chuyên Đề Toán 8}.\hfill{\sf[reading]}
	
	\item \cite{Tuyen_Toan_8_old}. Bùi Văn Tuyên. {\it Bài Tập Nâng Cao \& 1 Số Chuyên Đề Toán 8}.\hfill{\sf[done]}
\end{enumerate}

\paragraph{Grade 9}

\begin{enumerate}
	\item \cite{Binh_Toan_9_tap_1}. Vũ Hữu Bình. {\it Nâng Cao \& Phát Triển Toán 9. Tập 1}.\hfill{\sf[done]}
	
	\item \cite{Binh_Toan_9_tap_2}. Vũ Hữu Bình. {\it Nâng Cao \& Phát Triển Toán 9. Tập 2}.\hfill{\sf[done]}
	
	\item \cite{Binh_boi_duong_Toan_9_tap_1}. Vũ Hữu Bình, Nguyễn Xuân Bình, Phạm Thị Bạch Ngọc. {\it Bồi Dưỡng Toán 9. Tập 1}.\hfill{\sf[done]}
	
	\item \cite{Binh_boi_duong_Toan_9_tap_2}. Vũ Hữu Bình, Nguyễn Xuân Bình, Phạm Thị Bạch Ngọc. {\it Bồi Dưỡng Toán 9. Tập 2.}\hfill{\sf[done]}
	
	\item \cite{TLCT_THCS_Toan_9_dai_so}. Vũ Hữu Bình, Phạm Thị Bạch Ngọc, Đàm Văn Nhỉ. {\it Tài Liệu Chuyên Toán THCS Toán 9. Tập 1: Đại Số}.\hfill{\sf[done]}
	
	\item \cite{TLCT_THCS_Toan_9_hinh_hoc}. Vũ Hữu Bình, Nguyễn Ngọc Đạm, Nguyễn Bá Đang, Lê Quốc Hán, Hồ Quang Vinh. {\it Tài Liệu Chuyên Toán THCS Toán 9. Tập 2: Hình Học}.\hfill{\sf[done]}
	
	\item \cite{Dung_Can_Anh_bdt_8_9}. Nguyễn Văn Dũng, Võ Quốc Bá Cẩn, Trần Quốc Anh. {\it Phương Pháp Giải Toán Bất Đẳng Thức \& Cực Trị Dành Cho Học Sinh 8, 9}.\hfill{\sf[reading]}
	
	\item \cite{Tuyen_Toan_9_old}. Bùi Văn Tuyên. {\it Bài Tập Nâng Cao \& 1 Số Chuyên Đề Toán 9}.\hfill{\sf[done]}
	
	\item Vũ Dương Thụy, Nguyễn Ngọc Đạm. {\it Toán Nâng Cao \& Các Chuyên Đề Hình Học 9}.
\end{enumerate}

\paragraph{Secondary School -- Trung Học Cơ Sở [THCS]}

\begin{enumerate}
	\item Vũ Hữu Bình. {\it 9 Chuyên Đề Đại Số THCS}.
	
	\item Vũ Hữu Bình. {\it 9 Chuyên Đề Số Học THCS}.
	
	\item Vũ Hữu Bình. {\it 9 Chuyên Đề Hình Học THCS}.
	
	\item \cite{Dang2018}. Nguyễn Bá Đang. {\it Phát Triển Kỹ Năng Giải Toán Hình Học Phẳng Dành Cho Bậc THCS}.\hfill{\sf[reading]}
	
	\item \cite{Dong_23_1001_toan_I}. Nguyễn Đức Đồng. {\it 23 Chuyên Đề Giải 1001 Bài Toán Sơ Cấp. Tập 1}.\hfill{\sf[reading]}
	
	\item \cite{Dong_23_1001_toan_II}. Nguyễn Đức Đồng. {\it 23 Chuyên Đề Giải 1001 Bài Toán Sơ Cấp. Tập 2}.\hfill{\sf[reading]}
	
	\item \cite{Hung_Dung_Mai_Qua_Long_Toan_9_hinh_hoc}. Trần Quang Hùng, Nguyễn Tiến Dũng, Đào Thị Hoa Mai, Nguyễn Đăng Quả, Đỗ Xuân Long. {\it Tuyển Chọn Các Chuyên Đề Bồi Dưỡng Học Sinh Giỏi Toán 9 Hình Học}.\hfill{\sf[reading]}
	
	\item \cite{Kien_Trung_Khuong_Hanh_Bon}. Nguyễn Trung Kiên, Đặng Thành Trung, Nguyễn Duy Khương, Bùi Hồng Hạnh, Vũ Trung Bồn. {\it Một Số Chủ Đề Hay \& Khó Trong Kỳ Thi Tuyển Sinh Vào Lớp 10}.\hfill{\sf[reading]}
	
	\item \cite{Lam_An_Tuan_Toan_9_dai_so}. Nguyễn Tiến Lâm, Trương Quang An, Trịnh Khắc Tuân. {\it Tuyển Chọn Các Chuyên Đề Bồi Dưỡng Học Sinh Giỏi Toán 9 Đại Số}.\hfill{\sf[reading]}
	
	\item Phạm Minh Phương, Trần Văn Tấn, Nguyễn Thị Thanh Thủy. {\it Bồi Dưỡng Học Sinh Giỏi Toán THCS: Số Học}.
	
	\item \cite{Son_Nghiep_Trung_Can_bdt}. Nguyễn Ngọc Sơn, Chu Đình Nghiệp, Lê Hải Trung, Võ Quốc Bá Cẩn. {\it Các Chủ Đề Bất Đẳng Thức Ôn Thi Vào Lớp 10}.\hfill{\sf[reading]}
	
	\item \cite{Son_Tinh_Trung_Cau_rgbt}. Nguyễn Ngọc Sơn, Trần Văn Tình, Lê Hải Trung, Vũ Văn Cầu. {\it Luyện Thi Vào Lớp 10 Môn Toán Chuyên Đề Rút Gọn Biểu Thức}.\hfill{\sf[reading]}
	
	\item \cite{Tan_Han_Dung_Duc_Hieu_Phuong_Hau_Nga_Long_Tuan_tap_2}. Nguyễn Đức Tấn, Nguyễn Ngọc Hân, Cao Văn Dũng, Phí Trung Đức, Tạ Minh Hiếu, Thái Nhật Phượng, Hoàng Công Hậu, Trần Thị Phi Nga, Phùng Văn Long, Nguyễn Quang Tuấn. {\it Ôn Luyện Thi Vào Lớp 10 Chuyên Môn Toán Tập 2}.
	
	\item \cite{Thu_Viet_Minh_ptb2}. Nguyễn Tất Thu, Đoàn Quốc Việt, Vũ Công Minh. {\it Tự Luyện Giải Toán THCS Theo Chuyên Đề. Tập 3: Phương Trình Bậc 2}.\hfill{\sf[done]}
	
	\item \cite{Thu_Chung_Viet_Minh_circ}. Nguyễn Tất Thu, Đào Quốc Chung, Đoàn Quốc Việt, Vũ Công Minh. {\it Tự Luyện Giải Toán THCS Theo Chuyên Đề. Tập 8: Các Bài Toán Chứng Minh Hệ Điểm Nằm Trên Đường Tròn}.\hfill{\sf[done]}
\end{enumerate}

\paragraph{Grade 10}

\begin{enumerate}
	\item \cite{Hai_Hung_Thu_Tung_ncpt_Toan_10_tap_1}. Phạm Việt Hải, Trần Quang Hùng, Ninh Văn Thu, Phạm Đình Tùng. {\it Nâng Cao \& Phát Triển Toán 10 Tập 1}.
	
	\item \cite{Hai_Hung_Thu_Tung_ncpt_Toan_10_tap_2}. Phạm Việt Hải, Trần Quang Hùng, Ninh Văn Thu, Phạm Đình Tùng. {\it Nâng Cao \& Phát Triển Toán 10 Tập 2}.
\end{enumerate}

\paragraph{Grade 11}

\begin{enumerate}
	\item \cite{Hung_nang_cao_phat_trien_Toan_11_tap_1}. Trần Quang Hùng, Lê Thị Việt Anh, Phạm Việt Hải, Khiếu Thị Hương, Tạ Công Sơn, Nguyễn Xuân Thọ, Ninh Văn Thu, Phạm Đình Tùng. {\it Nâng Cao \& Phát Triển Toán 11 Tập 1}.\hfill{\sf[reading]}
	
	\item \cite{Hung_nang_cao_phat_trien_Toan_11_tap_2}. Trần Quang Hùng, Lê Thị Việt Anh, Phạm Việt Hải, Khiếu Thị Hương, Tạ Công Sơn, Nguyễn Xuân Thọ, Ninh Văn Thu, Phạm Đình Tùng. {\it Nâng Cao \& Phát Triển Toán 11 Tập 2}.\hfill{\sf[reading]}
	
	\item \cite{Liem_Thang2020}. Nguyễn Xuân Liêm, Đặng Hùng Thắng. {\it Bài Tập Nâng Cao \& 1 Số Chuyên Đề Đại Số \& Giải Tích 11}.\hfill{\sf[reading]}
	
	\item \cite{Tan2017}. Trần Văn Tấn. {\it Bài Tập Nâng Cao \& Một Số Chuyên Đề Hình Học 11}.\hfill{\sf[reading]}
	
	\item \cite{TLCT_hinh_hoc_11}. Đoàn Quỳnh, Phạm Khắc Ban, Văn Như Cương, Nguyễn Đăng Phất, Lê Bá Khánh Trình. {\it Tài Liệu Chuyên Toán Hình Học 11}.\hfill{\sf[reading]}	
\end{enumerate}

\paragraph{Grade 12}

\begin{enumerate}
	\item \cite{Kiselev_hhkg}. A. P. Kiselev. {\it Hình Học Không Gian}.\hfill{\sf[reading]}
	
	\item \cite{TLCT_giai_tich_12}. Đoàn Quỳnh (CB), Trần Nam Dũng, Hà Huy Khoái, Đặng Hùng Thắng, Nguyễn Trọng Tuấn. {\it Tài Liệu Chuyên Toán Giải Tích 12}.\hfill{\sf[reading]}
	
	\item \cite{TLCT_hinh_hoc_12}. Đoàn Quỳnh (CB), Hạ Vũ Anh, Phạm Khắc Ban, Văn Như Cương, Vũ Đình Hòa. {\it Tài Liệu Chuyên Toán Hình Học 12}.\hfill{\sf[reading]}
\end{enumerate}

\paragraph{Miscellaneous}

\begin{enumerate}
	\item \cite{Binh_HHTH}. Vũ Hữu Bình. {\it Hình Học Tổ Hợp}.\hfill{\sf[reading]}
	
	\item \cite{Binh_PTNN}. Vũ Hữu Bình. {\it Phương Trình Nghiệm Nguyên \& Kinh Nghiệm Giải}.\hfill{\sf[reading]}
	
	\item Võ Quốc Bá Cẩn, Trần Quốc Anh. {\it Sử Dụng Phương Pháp AM--GM Để Chứng Minh Bất Đẳng Thức}.
	
	\item Võ Quốc Bá Cẩn, Trần Quốc Anh. {\it Sử Dụng Phương Pháp Cauchy--Schwarz Để Chứng Minh Bất Đẳng Thức}.
	
	\item \cite{Dong_23_1001_toan_I}. Nguyễn Đức Đồng. {\it 23 Chuyên Đề Giải 1001 Bài Toán Sơ Cấp I: 12 Chuyên Đề Về Đại Số Sơ Cấp}.\hfill{\sf[reading]}
	
	\item \cite{Dong_23_1001_toan_II}. Nguyễn Đức Đồng. {\it 23 Chuyên Đề Giải 1001 Bài Toán Sơ Cấp II: 11 Chuyên Đề Về Toán Rời Rạc \& Hình Học Sơ Cấp}.\hfill{\sf[reading]}
	
	\item \cite{Khai_Huong_bdt}. Phan Huy Khải, Đoàn Thanh Hương. {\it Các Phương Pháp Hiệu Quả Giải Bài Toán Về Bất Đẳng Thức \& Giá Trị Lớn Nhất Nhỏ Nhất}.\hfill{\sf[reading]}
	
	\item \cite{Viet2014}. Dương Quốc Việt. {\it Những Tư Tưởng Cơ Bản Ẩn Chứa Trong Toán Học Phổ Thông}.\hfill{\sf[done]}
	
	\item \cite{Dung_cac_phuong_phap_giai_toan_qua_cac_ky_thi_olympic_2022}. Trần Nam Dũng, Nguyễn Văn Huyện, Lê Phúc Lữ, Tống Hữu Nhân, Lương Văn Khải, Bùi Khánh Vĩnh, Nguyễn Công Thành, Nguyễn Nam, Trang Sĩ Trọng, Trần Bình Thuận, Trần Nguyễn Nam Hưng, Trương Tuấn Nghĩa, Đặng Cao Minh, Đào Trọng Toàn. {\it Các Phương Pháp Giải Toán Qua Các Kỳ Thi Olympic}.\hfill{\sf[reading]}
	
	\item \cite{Chinh2021_tap_1}. Phan Đức Chính. {\it Tuyển Tập Những Bài Toán Sơ Cấp Đại Số Tập 1}.\hfill{\sf[reading]}
	
	\item \cite{Chinh2021_tap_2}. Phan Đức Chính. {\it Tuyển Tập Những Bài Toán Sơ Cấp Đại Số Tập 2}.\hfill{\sf[reading]}
	
	\item \cite{Ha_huong}. {\sc Nguyễn Minh Hà}. {\it Hướng Trong Hình Học Phẳng}.\hfill{\sf[done]}
	
	\item \cite{Hung_weekly_geometry}. {\sc Trần Quang Hùng}. {\it Mỗi Tuần Một Bài Toán Hình Học}.\hfill{\sf[reading]}
	
	\item \cite{Huy_so_hoc}. Nguyễn Nhất Huy. {\it Một Số Chủ Đề Số Học Hướng Tới Kỳ Thi HSG \& Chuyên Toán}.\hfill{\sf[reading]}
	
	\item \cite{Linh_topic_geometry}. Nguyễn Văn Linh. {\it 1 Số Chủ Đề Hình Học Phẳng}.\hfill{\sf[reading]}
	
	\item \cite{Linh_108_geometry}. Nguyễn Văn Linh. {\it 108 Bài Toán Hình Học Sơ Cấp}.\hfill{\sf[reading]}
	
	\item \cite{Nhan_8_geometry_theorem}. Tống Hữu Nhân. {\it 8 Định Lý Chọn Lọc Trong Hình Học Phẳng}.\hfill{\sf[reading]}
	
	\item \cite{Polya2014}. {\sc Georg Polya}. {\it How to Solve It: A New Aspect of Mathematical Method}. {\sf[1207 Amazon ratings][4873 Goodreads ratings]}
	
	{\sf Amazon review.} ``The bestselling book that has helped millions of readers solve any problem. A must-have guide by eminent mathematician {\sc G. Polya}, {\it How to Solve It} shows anyone in any field how to think straight. In lucid \& appealing prose, {\sc Polya} reveals how the mathematical methods of demonstrating a proof or finding an unknown can help you attack any problem that can be reasoned out -- from building a bridge to winning a game of anagrams. {\it How to Solve it} includes a heuristic dictionary with dozens of entries on how to make problems more manageable -- from analogy \& induction to the heuristic method of starting with a goal \& working backward to something you already know.
	
	The disarmingly elementary book explains how to harness curiosity in the classroom, bring the inventive faculties of students into play, \& experience the triumph of discovery. But it's not just for the classroom. Generations of readers from all walks of life have relished {\sc Polya}'s brilliantly deft instructions on stripping away irrelevancies \& going straight to the heart of a problem.''
	
	{\sf Editorial reviews.}
	\begin{itemize}
		\item ``Every prospective teacher should read it. In particular, graduate students will find it invaluable. The traditional mathematics professor who reads a paper before 1 of the Mathematical Societies might also learn something from the book: `He writes a, he says b, he means c; but it should be d.''' -- {\sc E. T. Bell}, {\it Mathematical Monthly}
		\item ``[This] elementary textbook on heuristic reasoning, shows anew how keen its author is on questions of method \& the formulation of methodological principles. Exposition \& illustrative material are of a disarmingly elementary character, but very carefully thought out \& selected.'' --- {\sc Herman Weyl}, {\it Mathematical Review}
		\item ``I recommend it highly to any person who is seriously interested in finding out methods of solving problems, \& who does not object to being entertained while he does it.'' -- {\it Scientific Monthly}
		\item ``Any young person seeking a career in the sciences would do well to ponder this important contributions to the teacher's art.'' -- {\sc A. C. Schaeffer}, {\it American Journal of Psychology}
		\item ``Every mathematics student should experience \& live this book.'' -- {\it Mathematics Magazine}
		\item ``In an age that all solutions should be provided with the least possible effort, this book brings a very important message: mathematics \& problem solving in general needs a lot of practice \& experience obtained by challenging creative thinking, \& certainly not by copying predefined recipes provided by others. Let's hope this classic will remain a source of inspiration for several generations to come.'' -- {\sc A. Bultheel}, {\it European Mathematical Society}
	\end{itemize}
	{\sf About the Author.} {\sc G. Polya} (1887--1985) was one of the most influential mathematicians of the twentieth century. His basic research contributions span complex analysis, mathematical physics, probability theory, geometry, \& combinatorics. He was a teacher par excellence who maintained a strong interest in pedagogical matters throughout his long career. Even after his retirement from Stanford University in 1953, he continued to lead an active mathematical life. He taught his final course, on combinatorics, at the age of ninety. John H. Conway (1937--2020) was professor emeritus of mathematics at Princeton University. He was awarded the London Mathematical Society's Polya Prize in 1987. He was interested in many branches of mathematics \& invented a successor to {\sc Polya}'s notation for crystallographic groups.
	
	{\bf Preface to 1st Printing.} ``A great discovery solves a great problem but there is a grain of discovery in the solution of any problem. Your problem may be modest; but if it challenges your curiosity \& brings into play your inventive faculties, \& if you solve it by your own means, you may experience the tension \& enjoy the triumph of discovery. Such experiences at a susceptible age may create a taste for mental work \& leave their imprint on mind \& character for a lifetime.
	
	Thus, a teacher of mathematics has a great opportunity. If he fills his allotted time with drilling his students in routine operations he kills their interest, hampers their intellectual development, \& misuses his opportunity. But if he challenges the curiosity of his students by setting them problems proportionate to their knowledge, \& helps them to solve their problems with stimulating questions, he may give them a taste for, \& some means of, independent thinking.
	
	Also a student whose college curriculum includes some mathematics has a singular opportunity. This opportunity is lost, of course, if he regards mathematics as a subject in which he has to earn so \& so much credit \& which he should forget after the final examination as quickly as possible. The opportunity may be lost even if the student has some natural talent for mathematics because he, as everybody else, must discover his talents \& tastes; he cannot know that he likes raspberry pie if he has never tasted raspberry pie. He may manage to find out, however, that a mathematics problem may be as much fun as a crossword puzzle, or that vigorous mental work may be an exercise as desirable as a fast game of tennis. Having tasted the pleasure in mathematics he will not forget it easily \& then there is a good chance that mathematics will become something for him: a hobby, or a tool of his profession, or his profession, or a great ambition.
	
	The author remembers the time when he was a student himself, a somewhat ambitious student, eager to understand a little mathematics \& physics. He listened to lectures, read books, tried to take in the solutions \& facts presented, but there was a question that disturbed him again \& again: ``Yes, the solution seems to work, it appears to be correct; but how is it possible to invent such a solution? Yes, this experiment seems to work, this appears to be a fact; but how can people discover such facts? \& how could I invent or discover such things by myself?'' Today the author is teaching mathematics in a university; he thinks or hopes that some of his more eager students ask similar questions \& he tries to satisfy their curiosity. Trying to understand not only the solution of  this or that problem but also the motives \& procedures of the solution, \& trying to explain these motives \& procedures to others, he was finally led to write the present book. He hopes that it will be useful to teachers who wish to develop their students' ability to solve problems, \& to students who are keen on developing their own abilities.
	
	Although the present book pays special attention to the requirements of students \& teachers of mathematics, it should interest anybody concerned with the ways \& means of invention \& discovery. Such interest may be more widespread than one would assume without reflection. The space devoted by popular newspapers \& magazines to crossword puzzles \& other riddles seems to show that people spend some time in solving unpractical problems. Behind the desire to solve this or that problem that confers no material advantage, there may be a deeper curiosity, a desire to understand the ways \& means, the motives \& procedures, of solution.
	
	The following pages are written somewhat concisely, but as simply as possible, \& are based on a long \& serious study of methods of solution. This sort of study, called {\it heuristic} by some writers, is not in fashion nowadays but has a long past \&, perhaps, some future.
	
	Studying the methods of solving problems, we perceive another face of mathematics. Yes, mathematics has 2 faces; it is the rigorous science of {\sc Euclid} but it is also something else. Mathematics presented in the Euclidean way appears as a systematic, deductive science; but mathematics in the making appears as an experimental, inductive science. Both aspects are as old as the science of mathematics itself. But the 2nd aspect is new in 1 respect; mathematics ``in statu nascendi,'' in the process of being invented, has never before been presented in quite this manner to the student, or to the teacher himself, or to the general public.
	
	The subject of heuristic has manifold connections; mathematicians, logicians, psychologists, educationalists, even philosophers may claim various parts of it as belonging to their special domains. The author, well aware of the possibility of criticism from opposite quarters \& keenly conscious of his limitations, has 1 claim to make: he has some experience in solving problems \& in teaching mathematics on various levels.
	
	The subject is more fully dealt with in a more extensive book by the author which is on the way to completion.'' -- Stanford University, Aug 1, 1944
	
	{\bf From Preface to 7th Printing.} ``I am glad to say that I have now succeeded in fulfilling, at least in part, a promise given in the preface to the 1st printing: The 2 volumes {\it Induction \& Analogy in Mathematics} \& {\it Patterns of Plausible Inference} which constitute my recent work {\it Mathematics \& Plausible Reasoning} continue the line of thinking begun in {\it How to Solve It}.'' Zurich, Aug 30, 1954
	
	{\bf Preface to 2nd edition.} ``The present 2e adds, besides a few minor improvements, a new 4th part, ``Problems, Hints, Solutions.''
	
	As this edition was being prepared for print, a study appeared (Educational Testing Service, Princeton, N.J.; cf. {\it Time}, Ju 18, 1956) which seems to have formulated a few pertinent observations -- they are not new to the people in the know, but it was high time to formulate them for the general public--: ``$\ldots$ mathematics has the dubious honor of being the least popular subject in the curriculum $\ldots$ Future teachers pass through the elementary schools learning to detest mathematics $\ldots$ They return to the elementary school to teach a new generation to detest it.''
	
	I hope that the present edition, designed for wider diffusion, will convince some of its readers that mathematics, besides being a necessary avenue to engineering jobs \& scientific knowledge, may be fun \& may also open up a vista of mental activity on the highest level.'' -- Zurich, Jun 30, 1956
	\begin{itemize}
		\item {\sf ``How to Solve It'' list.}
		\begin{itemize}
			\item {\sc Understanding the problem.} 1st: You have to {\it understand} the problem. {\it What is the unknown? What are the data? What is the condition?} Is it possible to satisfy the conditions? Is the condition sufficient to determine the unknown? Or is it insufficient? Or redundant? Or contradictory? Draw a figure. Introduce suitable notation. Separate the various parts of the condition. Can you write them down?
			\item {\sc Devising a plan.} 2nd. Find the connection between the data \& the unknown. You may be obliged to consider auxiliary problems if an immediate connection cannot be found. You should obtain eventually a {\it plan} of the solution. Have you seen it before? Or have you seem the same problem in a slightly different form? {\it Do you know a related problem?} Do you know a theorem that could be useful? {\it Look at the unknown!} \& try to think of a familiar problem having the same or a similar unknown.
			
			{\sf Here is a problem related to yours \& solved before. Could you use it?} Could you use its result? Could you use its method? Should you introduce some auxiliary element in order to make its use possible? Could you restate the problem? Could you restate it still differently? Go back to definitions.
			
			If you cannot solve the proposed problem try to solve 1st some related problem. Could you imagine a more accessible related problem? A more general problem? A more special problem? An analogous problem? Could you solve a part of the problem? Keep only a part of the condition, drop the other part; how far is the unknown then determined, how can it vary? Could you derive something useful from the data? Could you think of other data appropriate to determine the unknown? Could you change the unknown or the data, or both if necessary, so that the new unknown \& the new data are nearer to each other?
			
			Did you use all the data? Did you see the whole condition? Have you taken into account all essential notions involved in the problem?
			\item {\sc Carrying out the plan.} 3rd. {\it Carry out} your plan. Carrying out your plan of the solution, {\it check each step}. Can you see clearly that the step is correct? Can you prove that it is correct?
			\item {\sc Looking back}. 4th. {\it Examine} the solution obtained. Can you {\it check the results?} Can you check the argument? Can you derive the result differently? Can you see it at a glance? Can you use the result, or the method, for some other problem?
		\end{itemize}
		\item {\sf Foreword by {\sc John H. Conway}.} {\it How to Solve It} is a wonderful book! This I realized when I 1st read right through it as a student many years ago, but it has taken me a long time to appreciate just {\it how} wonderful it is. Why is that? 1 part of the answer is that the book is unique. In all my years as a student \& teacher, I have never seen another that lives up to {\sc George Polya}'s title by teaching you how to go about solving problems. {\sc A. H. Schoenfeld} correctly described its importance in his 1987 article ``{\sc Polya}, Problem Solving, \& Education'' in {\it Mathematics Magazine}. ``For mathematics education \& the world of problem solving it marked a line of demarcation between 2 eras, problem solving before \& after {\sc Polya}.''
		
		It is 1 of the most successful mathematics books ever written, having sold over a million copies \& been translated into 17 languages since it 1st appeared in 1945. {\sc Polya} later wrote 2 more books about the art of doing mathematics, {\it Mathematics \& Plausible Reasoning} (1954) \& {\it Mathematical Discovery} (2 volumes, 1962 \& 1965).
		
		The book's title makes it seem that it is directed only toward students, but in fact it is addressed just as much to their teachers. Indeed, as {\sc Polya} remarks in his introduction, the 1st part of the book takes the teacher's viewpoint more often than the student's.
		
		Everybody gains that way. The student who reads the book on his own will find that overhearing {\sc Polya}'s comments to his non-existent teacher can bring that desirable person into being, as an imaginary but very helpful figure leaning over one's shoulder. This is what happened to me, \& naturally I made heavy use of the remarks I'd found most important when I myself started teaching a few years later.
		
		But it was some time before I read the book again, \& when I did, I suddenly realized that it was even more valuable than I'd thought! Many of {\sc Polya}'s remarks that hadn't helped me as a student now made me a better teacher of those whose problems had differed from mine. {\sc Polya} had met many more students than I had, \& had obviously thought very hard about how to best help all of them learn mathematics. Perhaps his most important point is that learning must be active. As he said in a lecture on teaching, ``Mathematics, you see, is not a spectator sport. To understand mathematics means to be able to do mathematics. \& what does it mean [to be] doing mathematics? In the 1st place, it means to be able to solve mathematical problems.''
		
		It is often said that to teach any subject well, one has to understand it ``at least as well as one's students do.'' It is a paradoxical truth that to teach mathematics well, one must also know how to misunderstand it at least to the extent one's students do! If a teacher's statement can be parsed in 2 or more ways, it goes without saying that some students will understand it 1 way \& others another, with results that can vary from the hilarious to the tragic. {\sc J. E. Littlewood} gives 2 amusing examples of assumptions that can easily be made unconsciously \& misleadingly. 1st, he remarks that the description of the coordinate axes (``$Ox$ \& $Oy$ as in 2D, $Oz$ vertical'') in {\sc Lamb}'s book {\it Mechanics} is incorrect for him, since he always worked in an armchair with his feet up! Then, after asking how his reader would present the picture of a closed curve lying all on 1 side of its tangent, he states that there are 4 main schools (to left or right of vertical tangent, or above or below horizontal one) \& that by lecturing without a figure, presuming that the curve was to the right of its vertical tangent, he had unwittingly made nonsense for the other 3 schools.
		
		I know of no better remedy for such presumptions than {\sc Polya}'s counsel: before trying to solve a problem, the students should demonstrate his or her understanding of its statement, preferably to a real teacher, but in lieu of that, to an imagined one. Experienced mathematicians know that often the hardest part of researching a problem is understanding precisely what that problem says. They often follow {\sc Polya}'s wise advice: ``If you can't solve a problem, then there is an easier problem you can't solve: find it.''
		
		Readers who learn from this book will also want to learn about its author's life.
		\item {\sf Introduction.}
		
		Part I: In The Classroom
		\item {\sf Purpose}
		\begin{itemize}
			\item {\sf Helping the student.}
			\item {\sf Questions, recommendations, mental operations.}
			\item {\sf Generality.}
			\item {\sf Common sense.}
			\item {\sf Teacher \& student. Imitation \& practice.}
		\end{itemize}
		\item {\sf Main divisions, main questions.}
		\begin{itemize}
			\item {\sf4 phases.}
			\item {\sf Understanding the problem.}
			\item {\sf Example.}
			\item {\sf Devising a plan.}
			\item {\sf Example.}
			\item {\sf Carrying out the plan.}
			\item {\sf Example.}
			\item {\sf Looking back.}
			\item {\sf Example.}
			\item {\sf Various approaches.}
			\item {\sf The teacher's method of questioning.}
			\item {\sf Good questions \& bad questions.}
		\end{itemize}
		\item {\sf More examples}
		\begin{itemize}
			\item {\sf A problem of construction.}
			\item {\sf A problem to prove.}
			\item {\sf A rate problem.}
		\end{itemize}
		Part II: How To Solve It
		\item {\sf A dialogue.}
		
		Part III. Short Dictionary of Heuristic.
		\item {\sf Analogy}
		\item {\sf Auxiliary elements.}
		\item {\sf Auxiliary problem.}
		\item {\sf Bolzano.}
		\item {\sf Bright idea.}
		\item {\sf Can you check the result.}
		\item {\sf Can you derive the result differently?}
		\item {\sf Can you use the result?}
		\item {\sf Carrying out.}
		\item {\sf Condition.}
		\item {\sf Contradictory.}
		\item {\sf Corollary.}
		\item {\sf Could you derive something useful from the data?}
		\item {\sf Could you restate the problem?}
		\item {\sf Decomposing \& recombining.}
		\item {\sf Definition.}
		\item {\sf Descartes.}
		\item {\sf Determination, hope, success.}
		\item {\sf Diagnosis.}
		\item {\sf Did you use all the data?}
		\item {\sf Do you know a related problem?}
		\item {\sf Draw a figure.}
		\item {\sf Examine your guess.}
		\item {\sf Figures.}
		\item {\sf Generalization.}
		\item {\sf Have you seen it before?}
		\item {\sf Here is a problem related to yours \& solved before.}
		\item {\sf Heuristic.}
		\item {\sf Heuristic reasoning.}
		\item {\sf If you cannot solve the proposed problem.}
		\item {\sf Induction \& mathematical induction.}
		\item {\sf Inventor's paradox.}
		\item {\sf Is it possible to satisfy the condition?}
		\item {\sf Leibniz.}
		\item {\sf Lemma.}
		\item {\sf Look at the unknown.}
		\item {\sf Modern heuristic.}
		\item {\sf Notation.}
		\item {\sf Pappus.}
		\item {\sf Pedantry \& mastery.}
		\item {\sf Practical problems.}
		\item {\sf Problems to find, problems to prove.}
		\item {\sf Progress \& achievement.}
		\item {\sf Puzzles.}
		\item {\sf Reductio \& absurdum \& indirect proof.}
		\item {\sf Redundant.}
		\item {\sf Routine problem.}
		\item {\sf Rules of discovery.}
		\item {\sf Rules of style.}
		\item {\sf Rules of teaching.}
		\item {\sf Separate the various parts of the condition.}
		\item {\sf Setting up equations.}
		\item {\sf Signs of progress.}
		\item {\sf Specialization.}
		\item {\sf Subconscious work.}
		\item {\sf Symmetry.}
		\item {\sf Terms, old \& new.}
		\item {\sf Test by dimension.}
		\item {\sf The future mathematician.}
		\item {\sf The intelligent problem-solver.}
		\item {\sf The intelligent reader.}
		\item {\sf The traditional mathematics professor.}
		\item {\sf Variation of the problem.}
		\item {\sf What is the unknown?}
		\item {\sf Why proofs?}
		\item {\sf Wisdom of proverbs.}
		\item {\sf Working backwards.}
		
		Part IV. Problems, Hints, Solutions.
		\item {\sf Problems.}
		\item {\sf Hints.}
		\item {\sf Solutions.}
	\end{itemize}
		
	\item \cite{Quy2022}. Bùi Quỹ. {\it TikZ \& Vẽ Hình \LaTeX\ Vẽ Hình Toán Phổ Thông}.\hfill{\sf[reading]}
	
	\item \cite{Tao2006}. {\sc Terence Tao}. {\it Solving Mathematical Problems: A Personal Perspective}. {\sf[125 Amazon ratings][257 Goodreads ratings]}
	
	{\sf Amazon review.} Authored by a leading name in mathematics, this engaging \& clearly presented text leads the reader through the various tactics involved in solving mathematical problems at the Mathematical Olympiad level. Covering number theory, algebra, analysis, Euclidean geometry, \& analytic geometry, Solving Mathematical Problems includes numerous exercises \& model solutions throughout. Assuming only a basic level of mathematics, the text is ideal for students of 14 years \& above in pure mathematics.
	\begin{itemize}
		\item ``There are a handful of really wonderful books that can introduce a young high-school student to the beauty of mathematics. This is definitely 1 of them. Besides, this book is probably going to be known as the 1st book written by 1 of the best mathematicians of the 21 century.'' -- {\it The Mathematical Asociation of America}
	\end{itemize}
	{\sf Author the Author.} {\sc Terence Tao} was born in Adelaide, Australia, in 1975. In 1987, 1988, \& 1989 he competed in the International Mathematical Olympiad for the Australian team, winning a bronze, silver, \& gold medal respectively, \& being the youngest competitor ever to win a gold medal at this event. Since 2000, Terence has been a full professor of mathematics at the University of California, Los Angeles. He now lives in Los Angeles with his wife \& son.
	
	\item {\sc Đặng Hùng Thắng, Nguyễn Văn Ngọc, Vũ Kim Thùy}. {\it Bài Giảng Số Học}.
	
	\item \cite{Son2006}. {\sc Đỗ Thanh Sơn}. {\it Chuyên Đề Bồi Dưỡng Học Sinh Giỏi Toán Trung Học Phổ Thông: Phép Biến Hình Trong Mặt Phẳng}.\hfill{\sf[reading]}	
\end{enumerate}

\subsubsection{Elementary Physics Book}

\paragraph{Grade 7}

\begin{enumerate}
	\item \cite{Thinh_Lua_ncpt_Vat_Ly_7}. Bùi Gia Thịnh, Lê Thị Lụa, Nguyễn Thị Tâm. {\it Nâng Cao \& Phát Triển Vật Lý 7}.\hfill{\sf[reading]}
\end{enumerate}

\paragraph{Grade 8}

\begin{enumerate}
	\item \cite{SGK_Vat_Ly_8}. Vũ Quang, Bùi Gia Thịnh, Dương Tiến Khang, Vũ Trọng Rỹ, Trịnh Thị Hải Yến. {\it Vật Lý 8}.\hfill{\sf[reading]}
	
	\item \cite{SBT_Vat_Ly_8}. Bùi Gia Thịnh, Dương Tiến Khang, Vũ Trọng Rỹ, \& Trịnh Thị Hải Yến. {\it Bài Tập Vật Lý 8}.\hfill{\sf[reading]}
	
	\item \cite{Thinh_Lua_ncpt_Vat_Ly_8}. Bùi Gia Thịnh, Lê Thị Lụa. {\it Nâng Cao \& Phát Triển Vật Lý 8}.\hfill{\sf[reading]}
\end{enumerate}

\paragraph{Grade 9}

\begin{enumerate}
	\item \cite{SGK_Vat_Ly_9}. Vũ Quang, Đoàn Duy Hinh, Nguyễn Văn Hòa, Vũ Quang, Ngô Mai Thanh, Nguyễn Đức Thâm. {\it Vật Lý 9}.\hfill{\sf[reading]}
	
	\item \cite{SBT_Vat_Ly_9}. Đoàn Duy Hinh, Nguyễn Văn Hòa, Vũ Quang, Ngô Mai Thanh, Nguyễn Đức Thâm. {\it Bài Tập Vật Lý 9}.\hfill{\sf[reading]}
	
	\item \cite{Hoe_Vat_Ly_9}. Nguyễn Cảnh Hòe. {\it Nâng Cao \& Phát Triển Vật Lý 9}.\hfill{\sf[reading]}
	
	\item \cite{Hoe_Hoach_Vat_Ly_nang_cao_9}. Nguyễn Cảnh Hòe, Lê Thanh Hoạch. {\it Vật Lý Nâng Cao 9 Bồi Dưỡng Học Sinh Giỏi Thi Vào Lớp 10}.\hfill{\sf[reading]}
\end{enumerate}

\paragraph{Secondary School -- Trung Học Cơ Sở [THCS]}

\begin{enumerate}
	\item \cite{Van_500_BT_Vat_Ly_THCS}. Phan Hoàng Văn. {\it 500 Bài Tập Vật Lý Trung Học Cơ Sở}.\hfill{\sf[reading]}
	
	\item \cite{Van_Quyen_Hanh_Nhu_10_chuyen_Ly}. Nguyễn Văn, Phan Thị Quyên, Bùi Thị Lý Hạnh, Phạm Thị Quỳnh Như. {\it Giải Thích Chuyên Đề Thi Vào 10 Chuyên Lý}.\hfill{\sf[reading]}
	
	\item \cite{Vuong_10_chuyen_Ly}. Phạm Hồng Vương. {\it Giải Thích Bộ Đề Thi Vào 10 Chuyên Lý}.\hfill{\sf[reading]}
\end{enumerate}

\paragraph{Grade 10}

\begin{enumerate}
	\item \cite{Giang_Hang_Trung_ncpt_Vat_Ly_10}. Tô Giang, Trần Thúy Hằng, Lê Minh Trung. {\it Nâng Cao \& Phát Triển Vật Lý 10}.\hfill{\sf[reading]}
	
	\item Tô Giang. {\it Tài liệu chuyên Vật lý. Vật lý 10. Tập 1}.
	
	\item Phạm Quý Tư, Nguyễn Đình Noãn. {\it Tài liệu chuyên Vật lý. Vật lý 10. Tập 2}.
\end{enumerate}

\paragraph{Grade 11}

\begin{enumerate}
	\item Vũ Thanh Khiết, Nguyễn Thế Khôi. {\it Tài liệu chuyên Vật lý. Vật lý 11. Tập 1}.
	
	\item Vũ Quang. {\it Tài liệu chuyên Vật lý. Vật lý 11. Tập 2}.
\end{enumerate}

\paragraph{Grade 12}

\begin{enumerate}
	\item \cite{SGK_Vat_Ly_12_CD}. {\sc Nguyễn Văn Khánh, Phạm Thùy Giang, Đoàn Thị Hải Quỳnh, Trần Bá Trình, Trương Anh Tuấn}. {\it Vật Lý 12 Cánh Diều}.\hfill{\sf[done]}
	
	\item \cite{SBT_Vat_Ly_12_CD}. {\sc Nguyễn Văn Khánh, Phạm Thùy Giang, Đoàn Thị Hải Quỳnh, Đỗ Hương Trà, Mai Văn Túc, Trương Anh Tuấn}. {\it Bài Tập Vật Lý 12 Cánh Diều}.\hfill{\sf[done]}
	
	\item Tô Giang, Vũ Thanh Khiết, Nguyễn Thế Khôi. {\it Tài liệu chuyên Vật lý. Vật lý 12. Tập 1}.
	
	\item Vũ Quang, Vũ Thanh Khiết. {\it Tài liệu chuyên Vật lý. Vật lý 12. Tập 2}.
	
	\item Tô Giang, Đặng Đình Tới, Bùi Trọng Tuân. {\it Tài liệu chuyên Vật lý -- Bài tập Vật lý 10}.
	
	\item Lưu Hải An, Nguyễn Hoàng Kim, Vũ Thanh Khiết, Nguyễn Thế Khôi, Lưu Văn Xuân. {\it Tài liệu chuyên Vật lý -- Bài tập Vật lý 11}.
	
	\item Tô Giang, Vũ Thanh Khiết, Đặng Đình Tới. {\it Tài liệu chuyên Vật lý -- Bài tập Vật lý 12}.
	
	\item Đàm Trung Đồn. {\it Tài liệu chuyên Vật lý -- Thực hành Vật lý Trung học phổ thông}.
	
	\item Tô Giang. {\it Bồi Dưỡng Học Sinh Giỏi Vật Lý THPT: Cơ học 1}.\hfill{\sf[reading]}
	
	\item Tô Giang. {\it Bồi Dưỡng Học Sinh Giỏi Vật Lý THPT: Cơ học 2}.\hfill{\sf[reading]}
	
	\item Tô Giang. {\it Bồi Dưỡng Học Sinh Giỏi Vật Lý THPT: Cơ học 3}.\hfill{\sf[reading]}
	
	\item Vũ Thanh Khiết, Nguyễn Thế Khôi. {\it Bồi Dưỡng Học Sinh Giỏi Vật Lý THPT: Điện học 1}.\hfill{\sf[reading]}
	
	\item Vũ Thanh Khiết, Tô Giang. {\it Bồi Dưỡng Học Sinh Giỏi Vật Lý THPT: Điện học 2}.\hfill{\sf[reading]}
	
	\item Phạm Quý Tư. {\it Bồi Dưỡng Học Sinh Giỏi Vật Lý THPT: Nhiệt Học \& Vật Lý Phân Tử}.
	
	\item Ngô Quốc Quỳnh. {\it Bồi Dưỡng Học Sinh Giỏi Vật Lý THPT: Quang học 1}.\hfill{\sf[reading]}
	
	\item Vũ Quang. {\it Bồi Dưỡng Học Sinh Giỏi Vật Lý THPT: Quang học 2}.
	\item \cite{Khiet_Vat_Ly_hien_dai}. Vũ Thanh Khiết. {\it Bồi Dưỡng Học Sinh Giỏi Vật Lý Trung Học Phổ Thông: Vật Lý Hiện Đại}.\hfill{\sf[reading]}
	
	\item Phạm Văn Thiều. Đoàn Văn Ro, Nguyễn Văn Phán. {\it Bồi Dưỡng Học Sinh Giỏi Vật Lý THPT: Phương pháp giải 1 số bài toán điển hình}.
	
	\item Phạm Văn Thiều. {\it Bồi Dưỡng Học Sinh Giỏi Vật Lý THPT: Những bài toán tổng hợp: phân tích \& lời giải}.
	
	\item Bùi Quang Hân, Nguyễn Duy Hiền, Nguyễn Tuyến. {\it Giải Toán \& Trắc Nghiệm Vật Lý 10. Tập 1: Cơ học}.
	
	\item Bùi Quang Hân, Nguyễn Duy Hiền, Nguyễn Tuyến. {\it Giải Toán \& Trắc Nghiệm Vật Lý 10. Tập 2: Nhiệt học}.
	
	\item Bùi Quang Hân, Nguyễn Duy Hiền, Nguyễn Tuyến. {\it Giải Toán \& Trắc Nghiệm Vật Lý 11. Tập 1: Tĩnh điện \& Dòng điện không đổi}.
	
	\item Bùi Quang Hân, Nguyễn Duy Hiền, Nguyễn Tuyến. {\it Giải Toán \& Trắc Nghiệm Vật Lý 11. Tập 2: Điện từ \& Quang học}.
	
	\item Bùi Quang Hân, Nguyễn Duy Hiền, Nguyễn Tuyến. {\it Giải Toán \& Trắc Nghiệm Vật Lý 12. Tập 1: Động lực học vật rắn, Dao động cơ, Sóng cơ}.
	
	\item Bùi Quang Hân, Nguyễn Duy Hiền, Nguyễn Tuyến. {\it Giải Toán \& Trắc Nghiệm Vật Lý 12. Tập 2: Dao động \& sóng điện từ, Dòng điện xoay chiều}.
	
	\item Bùi Quang Hân, Nguyễn Duy Hiền, Nguyễn Tuyến. {\it Giải Toán \& Trắc Nghiệm Vật Lý 12 -- Tập 3: Sóng ánh sáng, Lượng tử ánh sáng, Thuyết tương đối hẹp, Hạt nhân nguyên tử, Từ vi mô đến vĩ mô}.
	
	\item Vũ Thanh Khiết, Lưu Hải Ân, Phạm Vũ Kim Hoàng, Nguyễn Đức Hiệp, Nguyễn Hoàng Kim. {\it Bồi Dưỡng Học Sinh Giỏi Vật Lý THPT: Bài Tập Điện Học -- Quang Học Vật Lý Hiện Đại}.
\end{enumerate}

\subsubsection{Elementary Chemistry Book}

\paragraph{Grade 7}

\begin{enumerate}
	\item \cite{SGK_KHTN_7_Canh_Dieu}. Mai Sỹ Tuấn, Đinh Quang Báo, Nguyễn Văn Khánh, Đặng Thị Oanh, Nguyễn Văn Biên, Đào Tuấn Đạt, Phan Thị Thanh Hội, Ngô Văn Hưng, Đỗ Thanh Hữu, Đỗ Thị Quỳnh Mai, Phạm Xuân Quế, Trương Anh Tuấn, Ngô Văn Vụ. {\it KHTN 7. Cánh Diều}.\hfill{\sf[reading]}
\end{enumerate}

\paragraph{Grade 8}

\begin{enumerate}
	\item \cite{An_Hoa_Hoc_nang_cao_8_9}. Ngô Ngọc An. {\it Hóa Học Nâng Cao Bồi Dưỡng Học Sinh Giỏi Các Lớp 8, 9}.\hfill{\sf[done]}
	
	\item \cite{SBT_Hoa_Hoc_8}. Nguyễn Cương, Ngô Ngọc An, Đỗ Tất Hiển, Lê Xuân Trọng. {\it Bài Tập Hóa Học 8}.\hfill{\sf[reading]}
	
	\item \cite{Giac2021}. Cao Cự Giác. {\it Bồi Dưỡng Học Sinh Giỏi Hóa Học 8}.\hfill{\sf[reading]}
	
	\item Nguyễn Xuân Trường, Quách Văn Long, Hoàng Thị Thúy Hương. {\it Các Chuyên Đề Bồi Dưỡng Học Sinh Giỏi Hóa Học 8}.
	
	\item \cite{Truong_BTNC_Hoa_Hoc_8_2022}. Nguyễn Xuân Trường. {\it Bài Tập Nâng Cao Hóa Học 8}.\hfill{\sf[reading]}
	
	\item \cite{SGK_KHTN_8_Canh_Dieu}. Mai Sỹ Tuấn, Đinh Quang Báo, Nguyễn Văn Khánh, Đặng Thị Oanh, Nguyễn Thị Hồng Hạnh, Đỗ Thị Quỳnh Mai, Lê Thị Phượng, Phạm Xuân Quế, Dương Xuân Quý, Đào Văn Toàn, Trương Anh Tuấn, Lê Thị Tuyết, Ngô Văn Vụ. {\it KHTN 8. Cánh Diều}.\hfill{\sf[reading]}
	
	\item \cite{SGK_KHTN_8_KNTTVCS}. Vũ Văn Hùng, Mai Văn Hưng, Lê Kim Long, Vũ Trọng Rỹ, Nguyễn Văn Biên, Nguyễn Hữu Chung, Nguyễn Thu Hà, Lê Trọng Huyền, Nguyễn Thế Hưng, Nguyễn Xuân Thành, Bùi Gia Thịnh, Nguyễn Thị Thuần, Mai Thị Tình, Vũ Thị Minh Tuyến, Nguyễn Văn Vịnh. {\it KHTN 8. Kết Nối Tri Thức Với Cuộc Sống}.\hfill{\sf[reading]}
	
	\item \cite{SGK_Hoa_Hoc_8}. Lê Xuân Trọng, Nguyễn Cương, Đỗ Tất Hiển. {\it Hóa Học 8}.\hfill{\sf[reading]}
\end{enumerate}

\paragraph{Grade 9}

\begin{enumerate}	
	\item \cite{SGK_Hoa_Hoc_9}. Lê Xuân Trọng, Cao Thị Thặng, Ngô Văn Vụ. {\it Hóa Học 9}.\hfill{\sf[reading]}
	
	\item \cite{SBT_Hoa_Hoc_9}. Lê Xuân Trọng, Ngô Ngọc An, Ngô Văn Vụ. {\it Bài Tập Hóa Học 9}.\hfill{\sf[reading]}
	
	\item \cite{Truong_BTNC_Hoa_Hoc_9_2021}. Nguyễn Xuân Trường. {\it Bài Tập Nâng Cao Hóa Học 9}.\hfill{\sf[reading]}
	
	\item \cite{Vu_Hoa2021}. Ngô Văn Vụ, Phạm Hồng Hoa. {\it Nâng Cao \& Phát Triển Hóa Học 9}.\hfill{\sf[reading]}
\end{enumerate}

\paragraph{Secondary School -- Trung Học Cơ Sở [THCS]}

\begin{enumerate}
	\item \cite{Tuan2022}. Vũ Anh Tuấn. {\it Bồi Dưỡng Hóa Học THCS}.\hfill{\sf[reading]}
	
	\item \cite{Ninh_Chi_Khu_Lien_Thanh2019}. Trần Trung Ninh, Khiếu Thị Hương Chi, Lê Văn Khu, Trần Thị Kim Liên, Nguyễn Thị Kim Thành. {\it $500$ Bài Tập Hóa Học Chuyên Trung Học Cơ Sở (Bồi Dưỡng Học Sinh Giỏi)}.\hfill{\sf[reading]}
	
	\item Nguyễn Đình Hành, Nguyễn Hữu Thọ. {\it 22 Chuyên Đề Hay \& Khó Bồi Dưỡng Học Sinh Giỏi Hóa Học THCS. Tập 1}.
\end{enumerate}

\paragraph{Grade 10}

\begin{enumerate}
	\item \cite{Ha_Hai_Huyen_Tuan2022}. Nguyễn Thu Hà, Nguyễn Văn Hải, Lê Trọng Huyền, Vũ Anh Tuấn. {\it Nâng Cao \& Phát Triển Hóa Học 10}.\hfill{\sf[reading]}
	
	\item \cite{Truong_Long_Huong_bdhsg_Hoa_Hoc_10}. Nguyễn Xuân Trường, Quách Văn Long, Hoàng Thị Thúy Hương. {\it Bồi Dưỡng Học Sinh Giỏi Hóa Học 10 Theo Chuyên Đề}.\hfill{\sf[reading]}
	\item \cite{An_Hoa_Hoc_co_ban_nang_cao_10}. Ngô Ngọc An. {\it Hóa Học Cơ Bản \& Nâng Cao 10}.\hfill{\sf[reading]}
	
	\item Đào Hữu Vinh, Nguyễn Duy Ái. {\it Tài liệu chuyên Hóa học 10. Tập 2}.\hfill{\sf[reading]}
\end{enumerate}

\paragraph{Grade 11}

\begin{enumerate}
	\item \cite{An_Hoa_Hoc_nang_cao_11}. Ngô Ngọc An. {\it Hóa Học Nâng Cao 11}.\hfill{\sf[reading]}
	
	\item \cite{An_400_BT_Hoa_Hoc_11}. Ngô Ngọc An. {\it 400 Bài Tập Hóa Học 11}.\hfill{\sf[reading]}
\end{enumerate}

\paragraph{Grade 12}

\begin{enumerate}	
	\item \cite{Son2021}. Trần Quốc Sơn. {\it Tài Liệu Chuyên Hóa Học 11--12. Tập 1: Hóa Học Hữu Cơ}.\hfill{\sf[reading]}
	
	\item \cite{Ai2022}. Nguyễn Duy Ái. {\it Tài Liệu Chuyên Hóa Học 11--12. Tập 2: Hóa Học Vô Cơ}.\hfill{\sf[reading]}
	
	\item \cite{SGK_Hoa_Hoc_12_co_ban}. Lê Xuân Trọng, Nguyễn Hữu Đĩnh, Từ Vọng Nghi, Đỗ Đình Răng, Cao Thị Thặng. {\it Hóa Học 12}.\hfill{\sf[reading]}
	
	\item \cite{SGK_Hoa_Hoc_12_nang_cao}. Nguyễn Xuân Trường, Phạm Văn Hoan, Từ Vọng Nghi, Đỗ Đình Răng, Nguyễn Phú Tuấn. {\it Hóa Học 12 Nâng Cao}.
	
	\item \cite{Truong_Long_Huong_bdhsg_Hoa_Hoc_12}. Nguyễn Xuân Trường, Quách Văn Long, Hoàng Thị Thúy Hương. {\it Bồi Dưỡng Học Sinh Giỏi Hóa Học 12 Theo Chuyên Đề}.\hfill{\sf[reading]}
\end{enumerate}

\paragraph{High School -- THPT}

\begin{enumerate}
	\item \cite{An_chuoi_PUHH}. Ngô Ngọc An. {\it Giúp Trí Nhớ Chuỗi Phản Ứng Hóa Học}.\hfill{\sf[reading]}
	
	\item Trần Quốc Sơn. {\it Tài Liệu Chuyên Hóa Học THPT: Bài Tập Hữu Cơ. Tập 1}.
	
	\item Trần Quốc Sơn. {\it Tài Liệu Chuyên Hóa Học THPT: Bài Tập Hữu Cơ. Tập 2}.
	
	\item Nguyễn Duy Ái, Nguyễn Tinh Dung, Trần Quốc Sơn, Nguyễn Văn Tòng. {\it Bồi Dưỡng Học Sinh Giỏi Hóa Học THPT. Tập 3}.
	
	\item \cite{Lovebook2022}. Gia Đình Lovebook. {\it Chinh Phục Đỉnh Cao Hóa Học Quốc Gia -- Quốc Tế}.\hfill{\sf[reading]}
\end{enumerate}

\subsubsection{Elementary Natural Science Book -- Sách Khoa Học Tự Nhiên Sơ Cấp}

\paragraph{Elementary Natural Science Grade 6}

\begin{enumerate}
	\item \cite{SGK_KHTN_6_CD}. {\sc Mai Sỹ Tuấn, Nguyễn Văn Khánh, Đặng Thị Oanh, Lê Minh Cầm, Ngô Ngọc Hoa, Lê Thị Phương Hoa, Phan Thị Thanh Hội, Đỗ Thanh Hữu, Cao Tiến Khoa, Lê Thị Thanh, Nguyễn Đức Trường, Trương Anh Tuấn}. {\it Khoa Học Tự Nhiên 6 Cánh Diều}.\hfill{\sf[done]}
	
	\item \cite{SBT_KHTN_6_CD}. {\sc Nguyễn Văn Khánh, Đặng Thị Oanh, Mai Sỹ Tuấn, Lê Minh Cầm, Ngô Ngọc Hoa, Phan Thị Thanh Hội, Ngô Văn Hưng, Đỗ Thanh Hữu, Cao Tiến Khoa, Lê Thị Thanh, Nguyễn Đức Trường}. {\it Bài Tập Khoa Học Tự Nhiên 6 Cánh Diều}.\hfill{\sf[done]}
	
	\item {\it Khoa Học Tự Nhiên 6. Chân Trời Sáng Tạo}.
	
	\item {\it Bài Tập Khoa Học Tự Nhiên 6. Chân Trời Sáng Tạo}.
	
	\item {\it Khoa Học Tự Nhiên 6. Kết Nối Tri Thức Với Cuộc Sống}.
	
	\item {\it Bài Tập Khoa Học Tự Nhiên 6. Kết Nối Tri Thức Với Cuộc Sống}.
	
	\item \cite{ncpt_KHTN_6_tap_1}. {\sc Nguyễn Thu Hà, Trần Thúy Hằng, Lê Trọng Huyền, Nguyễn Thị Thu Hương}. {\it Nâng Cao \& Phát Triển KHTN 6 Tập 1}.\hfill{\sf[done]}
	
	\item \cite{ncpt_KHTN_6_tap_2}. {\sc Hoàng Thị Đào, Trần Thúy Hằng, Vũ Thị Minh Tuyến}. {\it Nâng Cao \& Phát Triển KHTN 6 Tập 2}.\hfill{\sf[done]}
\end{enumerate}

\paragraph{Grade 7}

\begin{enumerate}	
	\item \cite{SGK_KHTN_7_Canh_Dieu}. {\it Khoa Học Tự Nhiên 7}. Cánh Diều.\hfill{\sf[done]}
	
	\item \cite{SBT_KHTN_7_Canh_Dieu}. {\it Bài Tập  Khoa Học Tự Nhiên 7}. Cánh Diều.\hfill{\sf[done]}
	
	\item {\it Khoa Học Tự Nhiên 7. Chân Trời Sáng Tạo}.
	
	\item {\it Bài Tập Khoa Học Tự Nhiên 7. Chân Trời Sáng Tạo}.
	
	\item {\it Khoa Học Tự Nhiên 7. Kết Nối Tri Thức Với Cuộc Sống}.
	
	\item {\it Bài Tập Khoa Học Tự Nhiên 7. Kết Nối Tri Thức Với Cuộc Sống}.
	
	\item \cite{ncpt_KHTN_7_tap_1}. Nguyễn Thị Thanh Chi, Trần Thúy Hằng, Vũ Thị Minh Tuyến. {\it Nâng Cao \& Phát Triển KHTN 7 Tập 1}. (Hóa Học $+$ Vật Lý).\hfill{\sf[done]}
	
	\item \cite{ncpt_KHTN_7_tap_2}. Nguyễn Thanh Loan, Trương Thị Nhàn. {\it Nâng Cao \& Phát Triển KHTN 7 Tập 2} (Sinh Học).\hfill{\sf[done]}
	
	{\sf Note.} I do not teach Elementary Biology, except some computational biology problems, so I skip this book \& give it to some of my best students who need to learn Elementary Biology.
\end{enumerate}

\paragraph{Grade 8}

\begin{enumerate}
	\item \cite{SGK_KHTN_8_Canh_Dieu}. {\it Khoa Học Tự Nhiên 8 Cánh Diều}.\hfill{\sf[done]}
	
	\item \cite{SBT_KHTN_8_Canh_Dieu}. {\it Bài Tập Khoa Học Tự Nhiên 8 Cánh Diều}.\hfill{\sf[done]}
	
	\item {\it Khoa Học Tự Nhiên 8. Chân Trời Sáng Tạo}.
	
	\item {\it Bài Tập Khoa Học Tự Nhiên 8. Chân Trời Sáng Tạo}.
	
	\item \cite{SGK_KHTN_8_KNTTVCS}. {\it Khoa Học Tự Nhiên 8 Kết Nối Tri Thức với Cuộc Sống}.\hfill{\sf[done]}
	
	\item {\it Bài Tập Khoa Học Tự Nhiên 8. Kết Nối Tri Thức Với Cuộc Sống}.
\end{enumerate}

\paragraph{Grade 9}

\subsubsection{Elementary Computer Science}

\begin{enumerate}
	\item \cite{Duc_200_BT_Python}. Nguyễn Tiến Đức. {\it Tuyển Tập 200 Bài Tập Lập Trình Bằng Ngôn Ngữ Python}.\hfill{\sf[reading]}
	\begin{itemize}
		\item Python source code $+$ input{\tt/}output files:\\\href{https://github.com/NQBH/hobby/tree/master/elementary_computer_science/Python/Duc_200_BTLT_Python}{GitHub{\tt/}NQBH{\tt/}hobby{\tt/}elementary computer science{\tt/}Python{\tt/}NTD 200 BTLT Python}.\footnote{\textsc{url}: \url{https://github.com/NQBH/hobby/tree/master/elementary_computer_science/Python/Duc_200_BTLT_Python}.}
	\end{itemize}
	
	\item \cite{Olympic30-4_2010_Tin_Hoc}. {\it Tuyển Tập Đề Thi Olympic 30 Tháng 4, Lần Thứ XVI -- 2010 Tin học}.\hfill{\sf[reading]}
	
	\item \cite{SGK_Tin_Hoc_11}. Hồ Sĩ Đàm, Hồ Cẩm Hà, Trần Đỗ Hùng, Nguyễn Đức Nghĩa, Nguyễn Thanh Tùng, Ngô Ánh Tuyết. {\it Tin Học 11}.\hfill{\sf[done]}
	
	\item \cite{TL_chuyen_Tin_quyen_1}. Hồ Sĩ Đàm, Đỗ Đức Đông, Lê Minh Hoàng, Nguyễn Thanh Hùng. {\it Tài Liệu Chuyên Tin Học Quyển 1}.\hfill{\sf[reading]}
	
	\item \cite{TL_chuyen_Tin_quyen_2}. Hồ Sĩ Đàm, Đỗ Đức Đông, Lê Minh Hoàng, Nguyễn Thanh Hùng. {\it Tài Liệu Chuyên Tin Học Quyển 2}.\hfill{\sf[reading]}
	
	\item \cite{TL_chuyen_Tin_quyen_3}. Hồ Sĩ Đàm, Đỗ Đức Đông, Lê Minh Hoàng, Nguyễn Thanh Hùng. {\it Tài Liệu Chuyên Tin Học Quyển 3}.\hfill{\sf[reading]}
	
	\item \cite{TL_chuyen_Tin_BT_quyen_1}. Hồ Sĩ Đàm, Đỗ Đức Đông, Lê Minh Hoàng, Nguyễn Thanh Hùng. {\it Tài Liệu Chuyên Tin Học Bài Tập Quyển 1}.
	
	\item \cite{TL_chuyen_Tin_BT_quyen_2}. Hồ Sĩ Đàm, Đỗ Đức Đông, Lê Minh Hoàng, Nguyễn Thanh Hùng. {\it Tài Liệu Chuyên Tin Học Bài Tập Quyển 2}.
	
	\item \cite{TL_chuyen_Tin_BT_quyen_3}. Hồ Sĩ Đàm, Đỗ Đức Đông, Lê Minh Hoàng, Nguyễn Thanh Hùng. {\it Tài Liệu Chuyên Tin Học Bài Tập Quyển 3}.
	
	\item \cite{Giang_sang_tao_lap_trinh}. {\sc Nguyễn Ngọc Giang}. {\it Sáng Tạo Trong Toán Lập Trình}.	
	\begin{itemize}
		\item ``Năng lực sáng tạo là 1 trong những năng lực quan trọng nhất hiện nay. Đây là vấn đề sống còn, tồn tại của cá nhân, tập thể, \& quốc gia. Tất cả mọi người đều đặt ra câu hỏi: ``Làm thế nào để có được năng lực sáng tạo?'' Nói về lĩnh vực này, nhà vật lý người Pháp F. Balibar nói: {\it``Thiên tài sáng tạo là bằng ý thức đổi mới, không lệ thuộc nếp cũ đem nhân với bình phương của trí tưởng tượng \& khả năng trừu tượng hóa.''}. Trong khi đó, nhà bác học Poincar\'e viết: {\it``Trong sáng tạo khoa học ý tưởng chỉ là ánh chớp, nhưng ánh chớp đó là tất cả.''}'' -- \cite[Lời nói đầu]{Giang_sang_tao_lap_trinh}
	\end{itemize}
	
	\item \cite{Trung_THCS_Tin}. Vương Thành Trung. {\it Tuyển Tập Đề Thi Học Sinh Giỏi Cấp Tỉnh Trung Học Cơ Sở \& Đề Thi Vào Lớp 10 Chuyên Tin Môn Tin Học}. {\sc url}: \url{https://github.com/NQBH/elementary_STEM_beyond/tree/main/elementary_computer_science/VTT_THCS}.\hfill{\sf[reading]}
	
	\item \cite{Trung_THPT_Tin}. Vương Thành Trung. {\it Tuyển Tập Đề Thi Học Sinh Giỏi Trung Học Phổ Thông Môn Tin Học}. {\sc url}: \url{https://github.com/NQBH/elementary_STEM_beyond/tree/main/elementary_computer_science/VTT_THPT}.\hfill{\sf[reading]}
	
	\item \cite{Trung_HSG_THPT_Tin}. Vương Thành Trung. {\it Tuyển Tập Đề Thi Học Sinh Giỏi Cấp Tỉnh Trung Học Phổ Thông Tin Học}.\hfill{\sf[reading]}
	
	\item \cite{VietSTEM2021}. Học Viện VietSTEM. {\it Sách Luyện Thi Hội Thi Tin Học Trẻ  với Python Bảng B: Thi Kỹ Năng Lập Trình Cấp Trung Học Cơ Sở}.\hfill{\sf[reading]}
	
	\item \cite{VietSTEM2022}. Học Viện VietSTEM. {\it Lập Trình với Python (Hành Trang Cho Tương Lai)}.\hfill{\sf[done]}
\end{enumerate}

%------------------------------------------------------------------------------%

\subsection{Advanced STEM Book}

\begin{enumerate}
	\item \cite{Feibelman2011}. {\sc Peter J. Feibelman}. {\it A PhD Is Not Enough!: A Guide to Survival in Science}. {\sf[460 Amazon ratings][1762 Goodreads ratings]}
	
	{\sf Amazon review.} Everything you ever need to know about making it as a scientist.
	
	Despite your graduate education, brainpower, \& technical prowess, your career in scientific research is far from assured. Permanent positions are scarce, science survival is rarely part of formal graduate training, \& a good mentor is hard to find.
	
	In {\it A Ph.D. Is Not Enough!}, physicist {\sc Peter J. Feibelman} lays out a rational path to a fulfilling long-term research career. He offers sound advice on selecting a thesis or postdoctoral adviser; choosing among research jobs in academia, government laboratories, \& industry; preparing for an employment interview; \& defining a research program. The guidance offered in {\it A Ph.D. Is Not Enough!} will help you make your oral presentations more effective, your journal articles more compelling, \& your grant proposals more successful.
	
	A classic guide for recent \& soon-to-be graduates, {\it A Ph.D. Is Not Enough!} remains required reading for anyone on the threshold of a career in science. This new edition includes 2 new chapters \& is revised \& updated throughout to reflect how the revolution in electronic communication has transformed the field.
	
	{\sf Editorial reviews.}
	\begin{itemize}
		\item ``It took me $> 40$ years to learn from experience what can be learned in 1 hour from this guide.'' -- {\sc Carl Djerassi}
		\item ``Breezily written, irreverent, \& filled with useful information. I wish something like it had been available when I was starting out.'' -- {\sc Michael Weber}, Cancer Center Director, University of Virginia, Charlottesville
		\item ``I loved {\it A Ph.D. Is Not Enough!} I couldn't put it down. His writing is delightful, \& he is on targed with virtually all of his advice.'' -- {\sc Steven H. Strogatz}, author of {\it The Joy of X}
	\end{itemize}
	{\sf About the Author.}  A Senior Scientist at Sandia National Laboratories, {\sc Peter J. Feibelman} received a Ph.D. in Physics from the University of California at San Diego, did postdoctoral research at the C.E.N. Saclay (France) \& the University of Illinois (Urbana), \& taught for 3 years at Stony Brook University. {\sc Feibelman} lives in Albuquerque, New Mexico.
	\begin{itemize}
		\item {\sf Preface: What This Book Is About.} ``My scientific career almost never happened. I emerged from graduate school with a PhD \& excellent technical skills but with little understanding of how to survive in science. In this, I was not unusual. Survival skills are rarely part of the graduate curriculum. Many professional scientists believe that ``good'' students find their way on their own, while the remainder cannot be helped. This justifies neglect \&, perhaps not incidentally, reduces work load. There may be some sense to the Darwinian selection process implicit in ``benign neglect,'' but on the whole, failing to teach science survival results in wasting a great deal of student talent \& time, \& not infrequently makes a mess of students' lives.
		
		Because science survival skills are rarely taught in a direct way, most young scientists need a mentor. Some will find one in graduate school, or as a postdoctoral researcher, or perhaps as an assistant professor. Those who do not have an excellent chance of moving from graduate study to scientific retirement without passing through a career. The unmentored can only succeed by being considerably more astute than the naive, idealistic, \& very bright young persons who generally choose a science major.
		
		These thoughts have been on my mind ever since I almost had to tell Mom \& Dad that their gold boy was not good enough to find a permanent (or any!) job in physics, a job for which his qualifications included 8 years of higher education \& 4 more of postdoctoral work. The agony of those days is not easily forgotten -- the boy with the high IQ, who had skipped a grade, graduated from the Bronx High School of Science at 16 \& from Columbia summa cum laude at 20, found himself in a muddle at 28. How do you choose a research problem? How do you give a talk? What do you do to persuade a university or a national or industrial lab to hire \& keep you? I hadn't a clue until, midway through my 2nd postdoctoral job, I had the good fortune to spend some months collaborating with a young professor who cared whether I survived as a scientist. Although this mentoring relationship was brief, it helped me acquire a set of skills that graduate education did not, skills without which my lengthy training in physics would have been wasted.
		
		This book is meant for those who will not be lucky enough to find a mentor early, for those who naively suppose that getting through graduate school, doing a postdoc, etc., are enough to guarantee a scientific career. I want you to see what stands between you \& a career, to help you prepare for the inevitable obstacles before they overwhelm you. In short, I hope to enable you to use your exceptional brainpower in the way that you \& those who put you through school have dreamed about.
		
		I begin with some brief case histories. This may help to put your own early career in better perspective. At least I hope it will give you a feeling for how important mentoring can be.
		
		Important or not, you are likely to wonder whether an elder who emerged into the scientific marketplace when times were flush, \& advanced technology looked very different from today's, can possibly offer you useful advice. Chap. 2 argues that one can.
		
		Succeeding chapters are arranged in parallel with a career trajectory. Skip ahead to whichever may be relevant to your situation. Chap. 3 deals with choosing a thesis or a postdoctoral adviser. My choice of thesis adviser was based on 2 criteria: Who is the most eminent professor in the department? \& whose students finish soonest? Was this intelligent, or did it represent a 1st mistake? Chap. 4 concerns oral presentation of your work. However brilliant your insights, they will be of little use if you cannot make them appear interesting to others. If no one pays attention, what difference does it make if your results are clever? There are of course Nobel prize -- winners whose orations are Delphic, whose visuals look as though they were put together during a particularly turbulent flight, \& so on. But you are not 1 of them yet, \& if that is how your talks are prepared, you never will be either. There is more to Chap. 4, though, than advice on preparing appealing slides. It contains a range of important ideas on making your oral presentations effective.
		
		In Chap. 5, you will find a discussion of paper writing. Through your scholarly articles, you can make yourself known nationally \& internationally. I.e., your reputation in science does not just depend on what your boss says about you but also on documentation that is readily available on the Internet. You should therefore view publishing as a means to attaining job security \& take the task of writing compelling journal articles very seriously.
		
		Chap. 6 is devoted to career choices, mainly the merits \& defects of positions in academia \& in government or industrial labs. The focus is on being reflective \& rational rather than naive or romantic about key decisions in your scientific life. In Chap. 7, I discuss job interviews. There is more to an interview than wearing your Sunday best \& having a firm handshake. Doing your homework \& persuading your potential employers that you have a sense of direction are the most important issues. Incidentally, this is not a matter of deception -- knowing who your colleagues will be \& developing an idea of what you want to know, scientifically, are keys to having a productive career. There are also a few choice words in chis chapter about negotiations, once you do get an offer. Negotiating for what you will need when your leverage is maximal can make a large difference to your happiness \& to your success.
		
		In Chap. 8, I discuss what -- to many -- is the bane of scientific life namely, getting money. This used to be the exclusive headache of those in academia, but nowadays it is also a significant part of the lives of government \& industrial scientists. I suggest that you view the preparation of a proposal as an important scientific exercise. Coming to see \& being able to articulate how your work fits into ``the big picture'' is essential not only to winning financial support but also to being a 1st-class researcher. Learning to distinguish extravagant ``pie in the sky'' from promises that you have a chance of fulfilling is also very valuable.
		
		The most difficult problem in being a scientist is selecting what to work on, \& it is even more difficult when you are just launching your career. Therefore, in Chap. 9, I venture a few comments on establishing a research program. Jumping into the hottest research area may not be a very good idea, nor is taking on a project that you have no realistic hope of completing before you short-term employment comes to an end. The main idea is to establish a program that simultaneously maximizes your chances of continuing employment {\it\&} of scientific achievement. The focus is on strategic thinking.
		
		As this book is written, economic times are tough worldwide, \& funding for scientific research is contracting. I hardly need to emphasize that when resources become scarce, competition intensifies for what remains available. To win a permanent position in scientific research, \& the funds to carry on serious work, you will have to be exceptionally thoughtful about your career choices. My hope is that this ``pocket mentor'' will help you to become more introspective about what it will take to succeed. -- Albuquerque, Aug 1993
		
		The past 17 years have seen revolutionary changes in how we communicate information. Virtually all journals are available electronically. Preprints can be published on the Internet before or without ever being refereed. Overhead projectors have disappeared from scientific meetings in favor of LCD projectors \& laptop computers. R\'esum\'es are often distributed electronically. This update of {\it A PhD Is Not Enough!} comes abreast of these changes, though the basic content of the 1993 original remains timely. The communications revolution cannot be ignored but has not made it less important to be thoughtful about choosing your career path or to respect audiences \& readers. I still attend talks that make me squirm \& struggle to read sleep-inducing scientific articles. I hope attentive readers of this book will reap the rewards of doing better.'' -- Albuquerque, Jan 2010		
		
		\item {\sf Chap. 1: Do You See Yourself in This Picture?} {\it A set of nonfiction vignettes illustrating some of the ways that young scientists make their lives more unpleasant than necessary or fail entirely to establish themselves in a research career.}
		
		The brief stories in this chapter have a common theme: that understanding \& dealing rationally with the realities of a life in science are as important to science survival as being bright. Once you leave graduate school, the clock is ticking. Unlike a fine wine, you do not have many years to mature. As a young professional, you must be able to select appropriate research problems, you have to finish projects in a timely manner, \& you ought to be giving compelling tasks \& publishing noteworthy papers. When job opportunities present themselves, you should be able to assess their value realistically. Romanticizing your prospects is a major mistake \& is likely to have serious consequences, not excluding dropping out of scientific life prematurely. The 1st story is an excerpt from my own scientific beginnings. The others are also nonfiction, though I have altered locations \& personal characteristics to avoid invading the privacy of the protagonists. I have deliberately identified the various characteristics with initials, rather than names, to avoid any ethnic implications.
		\begin{itemize}
			\item {\sf What Do Scientists Do? Technique vs. Problem Orientation.} Virtually all classroom work \& much of what happens in a typical thesis project is aimed at developing a student's technical skills. But although the success of your research efforts may depend heavily on designing a piece of apparatus or a computer code, \& on making it work properly, {\it no technical skill is worth more than knowing how to select exciting research projects}. Regrettably, this vital ability is almost never taught. When I signed on with a research adviser in my 1st year of graduate school, I was thrilled to be given a problem to work in the physics of the upper atmosphere. That I had no idea what motivated the problem did not prevent me from carrying out an analysis, on a supercomputer of the day, \& publishing my 1st paper at the age of 22. For my thesis, I consciously switched to a project that would require learning the tools of modern quantum physics, but again I found myself assimilating technical skills without ever grasping the significance of the problem, without understanding how or whether it was at the cutting edge of science. This way of working became a habit, one that seriously threatened my career. My 1st 7 publications were in 7 different areas of physics. In each case, I relied on a senior scientist to tell me what would be an interesting problem to work on; then I would carry out the task. I assume it was my ability to complete projects that impressed my superiors sufficiently to keep me employed. It certainly wasn't my depth in any field.
			
			4 years \& 2 postdoctoral positions after earning a PhD -- still having little sense of what I wanted to learn as a scientist -- I was on the job market. More than anything else, I needed good recommendations from faculty at the university where I was employed. I was asked to give the weekly solid-state physics seminar \& realized, at best dimly, that my performance in this venue was either going to make or break me as a scientist.
			
			The talks I was giving at this point in my career reflected my approach to science. There was little in the way of introductory material. Much of the presentation was technical. I would describe a few ``interesting'' problems I had worked on \& explain the methods I had used but would give little idea of context because I really didn't know what it was. For the seminar at hand, I prepared my usual hodgepodge of this project \& that, with no introduction, no theme, \& ultimately no meaning to anyone but an expert. Fortunately, the professor supervising my research, C., understood what was about to happen to me, \& asked for a preview of my seminar in his office. Thank goodness I accepted this invitation. C. expressed surprise at how poorly I had prepared my talk (though I don't think he was surprised at all), how little grasp I seemed to have of the reasons that the problems we had worked out were meaningful, \& consequently how uninterestingly I was going to present them to my audience. But, he told me, he thought I was too good technically to be allowed to fail in the way I was about to, \& he gave me the lesson I needed. His most important advice was:
			\begin{enumerate}
				\item There has to be a theme to your work -- some objective -- something you want to know. There has to be a story line. (Do not start with, ``I have been trying to explain the interesting wavelength dependence of light scattering from small particles,'' but rather ``There is a widespread need to explain to one's kids why the sky is blue.'')
				\item If you know why you have chosen to work on a particular problem, it is easy to present an absorbing seminar. Start out by telling your story, why the field you are working in is an important one, \& what the main problems are. Give some historical material showing where the field is, the relative advantages of different methods, \& so on. Then outline what you did, \& describe your results. Conclude with a statement of how your results have advanced our understanding of nature, \& perhaps give an inkling of the new directions that your work opens up. Do not assume that your audience comprises experts only. There may be a couple of them, but even experts like to hear things that they understand \& particularly to have their colleagues hear (from someone else) why their field is an important one.
				\item Lastly, rehearse your talk in front of 1 or 2 of your peers or professional supporters. Choose listeners who will not be shy about asking questions \& offering constructive suggestions. Giving a seminar is serious business. Your future depends on the strong recommendations of your senior colleagues. If your talk is a hodgepodge of techniques or experiments or equations, if you seem to have no idea where you are headed, if you reek of deference to the experts in the audience, you will not be perceived as a rising star, a budding scientific leader. You will fail.
			\end{enumerate}
			The wonderful result of C.'s mentoring was that I finally learned what it means to be a scientist. In making my work meaningful to others, I had also made it compelling to myself. No longer was I just working on somebody else's problems. I was part of an intellectual enterprise with relatively well-defined goals, which might actually make a difference to humanity. I scrapped most of the equations I had planned to show \& refocused my talk using thematic material I had garnered from C. I gave an excellent seminar -- people I scarcely knew complimented me afterward on my choice of an exciting research area \& remarked on the clarity of my presentation. In science, the reinforcement doesn't get much more positive than that. I had learned a key lesson \& was on my way.
			
			-- Kết quả tuyệt vời từ sự hướng dẫn của C. là cuối cùng tôi đã học được ý nghĩa của việc trở thành một nhà khoa học. Khi làm cho công việc của mình có ý nghĩa với người khác, tôi cũng đã làm cho nó trở nên hấp dẫn đối với chính mình. Tôi không còn chỉ làm việc trên các vấn đề của người khác nữa. Tôi là một phần của một doanh nghiệp trí tuệ với các mục tiêu được xác định tương đối rõ ràng, điều này thực sự có thể tạo ra sự khác biệt cho nhân loại. Tôi đã loại bỏ hầu hết các phương trình mà tôi đã lên kế hoạch trình bày \& tập trung lại bài nói chuyện của mình bằng cách sử dụng tài liệu chuyên đề mà tôi đã thu thập được từ C. Tôi đã có một buổi hội thảo tuyệt vời -- những người mà tôi hầu như không quen biết đã khen ngợi tôi sau đó về sự lựa chọn lĩnh vực nghiên cứu thú vị của tôi \& nhận xét về sự rõ ràng trong bài thuyết trình của tôi. Trong khoa học, sự củng cố không có gì tích cực hơn thế. Tôi đã học được một bài học quan trọng \& đang trên đường thực hiện.
			\item {\sf Timing Is Everything.} Having completed a respectable thesis problem \& having acquired a reputation in graduate school as an excellent sounding board \& scientific consultant, T. accepted a postdoctoral position with a leading science at a 1st-rate government laboratory. There, he was offered \& began to work on a computational research project that 1st involved arriving at a numerically practical mathematical formulation of a problem \& then required a considerable computer programming effort. As the months passed, \& with the necessity on the horizon of finding a permanent job, T. absorbed himself totally in his very challenging work. Whereas in graduate school, under little time pressure, he would have spent a few hours each week visiting labs \& contributing to projects other than his own, as a postdoc, T. became utterly single-minded.
			
			-- {\sf Thời gian là tất cả.} Sau khi hoàn thành một vấn đề luận án đáng kính \& đã có được danh tiếng trong trường sau đại học là một người lắng nghe tuyệt vời \& cố vấn khoa học, T. đã chấp nhận một vị trí sau tiến sĩ với một khoa học hàng đầu tại một phòng thí nghiệm của chính phủ hạng nhất. Ở đó, anh ấy đã được mời \& bắt đầu làm việc trên một dự án nghiên cứu tính toán đầu tiên liên quan đến việc đưa ra một công thức toán học thực tế về mặt số của một vấn đề \& sau đó đòi hỏi một nỗ lực lập trình máy tính đáng kể. Khi những tháng trôi qua, \& với nhu cầu tìm kiếm một công việc cố định đang ở phía trước, T. đã đắm mình hoàn toàn vào công việc rất đầy thử thách của mình. Trong khi ở trường sau đại học, dưới áp lực thời gian ít ỏi, anh ấy sẽ dành một vài giờ mỗi tuần để đến thăm các phòng thí nghiệm \& đóng góp cho các dự án khác ngoài dự án của mình, thì với tư cách là một tiến sĩ sau tiến sĩ, T. đã trở nên hoàn toàn tập trung.
			
			Working 12 hours a day \& more, he managed to complete his computer program soon enough to be able to run test calculations. The results were promising but not far enough along to yield a persuasive ``story.'' Accordingly, neither T. nor his audiences found his job seminar very exciting. What is more, since he had not taken time to meet \& consult with scientists at his lab, his only strong recommendation was from his postdoctoral adviser. The lab itself was unwilling to promote T. to a permanent position, which it sometimes did, because he had not made himself useful, or even known, to a spectrum of its staff members.
			
			On the outside, his job offers were a cut below what his thesis adviser had expected for him. In the competition for the best positions, T. did not persuade potential employers that he would ever derive useful results from his postdoctoral project, even though T. believed that he would have them within 6 months to a year. Other job candidates whose postdoctoral work had been far less ambitious, but had resulted in 2 or 3 finished projects, appeared much more impressive. Moreover, they had obtained excellent recommendations from the experimental colleagues whose data they had analyzed.
			
			On the whole, it is hard to blame potential employers for their view of T. To them he was ``a pig in a poke,'' an unknown quantity. His thesis work might just have been done by his thesis adviser, \& his postdoctoral project, though in principle a worthy one was unfinished. Would T. be able to complete projects on his own? Was he a self-starter? The information simply was not there, in the eyes of the interviewers.
			
			To some extent, T.'s fate was the fault of his adviser. Assigning a long-term project to a postdoctoral researcher who will be on the job market in 18 months is a clear risk to the postdoc's future. But, had T. been as reflective about his career as he was in carrying out his research, he himself would have realized the dangerous path he was taking. As exciting as his assigned project seemed, he would have recognized that his postdoctoral years were the wrong time for such a large effort. At the very least, he would have reversed time each day or week to establish contact with other researchers at the lab \& involved himself in 1 or 2 short-term projects with a clear chance for success. Many a graduate student or postdoc spends time trying to understand what his adviser wants \& getting it done. In fact, it is the young scientists who define \& carry out what {\it they} want, who learn to be scientific leaders, who find the best jobs \& have the most productive \& satisfying careers. Making your thesis or postdoctoral adviser happy is sensible, \& worth doing, but not more so than acting in your own best interests.
			\item {\sf Know Thyself -- A Sweet Job Turns Sour.} B. obtained a PhD from a top-flight university in the Midwest. He had 2 different thesis advisers during the course of his 4 years as a graduate student. The 1st was a Nobel prizewinner, a theoretician whose name is a household word to chemists. The 2nd was an experimentalist, also a very widely respected scientist. Having completed his degree, \& cognizant of the scarcity of real jobs, B. accepted a ``permanent'' position at a major laboratory instead of a postdoctoral, temporary slot. It did not take him long to realize that this apparently wonderful opportunity was a trap. On arrival at his new location, B. was presented with 2 options. A senior staff member, who was involved in a major experiment, suggested that B. begin his tenure by working in his lab. That way, B.'s knowledge of the experimental aspects of his field would deepen, \& after a couple of years, he would be much better prepared to work on his own. Objectively, one would say that this was a wonderful opportunity, effectively a postdoctoral job, but at a regular staff salary \& with a reasonable approximation to regular staff job security. B.'s alternative option was to begin independent work immediately. Talking to his younger colleagues, he heard that, in the eyes of management, a full staff member was supposed to run his own research program \& that at the annual performance review, if he was perceived to be working as someone else's ``assistant,'' his rating, salary, \& job security would suffer, perhaps irretrievably.
			
			One does not have to be a rocket scientist, as they say, to appreciate that B.'s 2-year stint as a graduate student in experimental physics was inadequate preparation for him to perform at the level of his supposed peers. Nevertheless, unmentored, B. was not willing to risk his all-too-sweet regular staff position by choosing the training that he badly needed. This was a mistake. After 3 years of buying equipment \& setting up a lab, B. had still not established a research program, \& indeed had little idea of what he wanted to accomplish as a scientist. Thus, despite its investment in his laboratory equipment, \& despite his nominally very impressive pedigree, B.'s employer moved him out of basic research. In an environment where goals were clearly defined from above, he eventually matured into a real contributor \& is reasonably happy. On the other hand, he is not doing basic research any more, \& he went through several very stressful years as a result of his bad start. Sadly, his failure at work coincided with the breakup of his marriage, an unhappy fate shared by many whose scientific careers flounder.
			\item {\sf The PhD Technician.}
		\end{itemize}
		
		\item {\sf Chap. 2: Advice from a Dinosaur?} {\it Can you expect someone to be an effective mentor who emerged into the scientific marketplace in a world that looked very different?}
		
		\item {\sf Chap. 3: Important Choices: A Thesis Adviser, a Postdoctoral Job.} {\it A discussion of what to consider: young adviser vs. an older one, a superstar vs. a journeyman, a small group vs. a ``factory.'' Understanding \& attending to your interests as a postdoc.}
		
		As a young graduate student, I selected a thesis adviser on the bases of his prominence in the world of physics \& his reputation as one who would not require me to spend too much time in graduate school. As with other aspects of my early career, I now see these criteria as reasonable but insufficient.
		\begin{itemize}
			\item {\sf A Prominent Scientist as a Thesis Adviser.} Choosing a prominent thesis adviser makes a lot of sense, but not because brilliance is transferable. It is not, as I have witnessed more than once. Trying to be another Linus Pauling, Roald Hoffmann, James Watson,or P. W. Anderson is a common road to failure. What a prominent adviser \textit{can} offer is: 1. being part of the ``oldboy network'' (he or she can help you survive if times are tough, sometimes even if you don't deserves to); \& 2. not competing with you. Point 1 is self-evident upon a moment's thought. Point 2 is not so obvious to the naive.
			
			A young adviser, only recently on the road to a permanent research position, has a lot to prove, is understandably leery of being shown up by a student or postdoc, \& is correspondingly unlikely to be generous with credit for ideas or progress. By contrast, advisers who have already made their mark view the accomplishments of their students, in effect their research ``children,'' with pride, even joy. Thus, other things being equal, an established (tenured) professor is a superior choice for an adviser. This recommendation is a simple corollary of the way universities are organized. It is not an indictment of young professors to recognize that they are likely to view their own scientific survival as more important than that of their students.
			
			A more senior adviser also offers you better prospects of finishing the thesis project that you start \& of spending your entire graduate career at 1 university. Many assistant professors fail to win promotion to tenure. If this happens to your adviser, he or she will either have to move to another university or may drop out of academic science entirely. In either case, you will face unwanted, difficult choices: whether or not to move with your adviser, or whom to choose as a new one; whether to select a new dissertation topic or to try to find another professor who is willing \& able to help you proceed in your initial direction.
			
			Although a senior professor may also move to another job while you are a student, the probability is lower. 1 reason is that the bother involved in moving an established, large group is substantial. Another is that universities will offer what it takes, if the money is available, to retain their top staff. If your senior professorial adviser does decide to move, the consequences for your thesis project are unlikely to be dire. A senior scientist relocates by choice, usually because the funding situation in the new location is, or perhaps other aspects of scientific life are, better. Moving with your adviser is thus likely to be both financially possible \& scientifically desirable. \& if you do decide to move, the delay in your progress toward a PhD should be minimal.
			
			Obviously, an older professor has a better chance of becoming seriously ill or dying while you are a student. Otherwise, the chances of a senior scientist's dropping out of research entirely are rather remote.
			
			\textit{Tenure \& prominence are not enough:} Although signing on as the student of an established scientist has many clear advantages, choosing a good advise is not as easy as finding out who has won the most important prizes, gives the most invited talks, or brings in the largest research grants. Is the professor you are considering available to consult with students on a reasonably frequent basis \& able to convey real guidance? Is your intended adviser comfortable talking to people who are not scientific peers (i.e., beginners such as yourself)? Does the group you wish to join have a sense of purpose? Do its members interact with each other? \& does Professor Eminent teach survival skills? These are important questions. Making a mistake in choosing your adviser can mean years of frustration. If you can learn the answers to the important questions in advance, by talking to current or former students, you may save yourself a lot of grief.
			
			\textit{Do group members see the big picture?} Prof. E. was obsessive. He was obnoxious. I have heard it said that he didn't know quantum mechanics. But his contributions to material science were manifold -- \& his students have done wonderfully well. They knew what they wanted to learn, \& they learned from each other. Thus, even if E. was often away consulting at industrial labs, his students thrived.
			
			How do you find out in advance whether the group you are considering will be like E.'s? Visit the members. Ask them what they are doing. See if they can explain the big picture. If they cannot, find a different adviser.
			
			Often a prominent scientist will lead a big group with, say, 15 or 20 experimental systems, enabling an equal number of graduate students to study trends. These students are guaranteed to finish their degrees in a reasonable period of time. In total contrast to my own graduate student experience, they are assigned very specific problems. They take their data, report their results, \& get their degrees. It all seems so easy. Should you be part of this kind of group? Again, the issue is whether the students have an inkling of the big picture. Is it only the adviser who knows what trend is being studied, while student A. is looking at rhodium, B. has a sample of ruthenium, \& C. has some palladium? If the students cannot tell a good story, move on.
			\item {\sf Choosing a Postdoctoral Position.} How should you be rational about the choice of a postdoctoral position? It is essential to understand what your interests are \& how they differ from the employer's. To begin, you should realize that what you actually achieved in your thesis is not especially important to your postdoctoral adviser. If you are 1 of the few whose thesis represents a major breakthrough, you will probably be much in demand, \& will likely have few problems finding a permanent job. You probably won't ever have a postdoctoral position. Your problem may be that you will spend the next several years trying to show that your initial triumph was not a fluke. This kind of thinking has paralyzed more than a few young ``geniuses'' but is not an important consideration for the majority, for whom this chapter is written.
			
			If your thesis, as is more likely, has not attracted much interest, despite your worries, you will probably find a postdoctoral slot. Employers generally feel that a postdoctoral employee is not a big risk. Unlike a graduate student, who has to be shown the ropes \& whose education may absorb so much time that his or her net contribution to the progress of a project may be slight, or negative, a postdoc is a trained researcher who can be expected to be reasonably competent \& not terribly demanding of supervision.
			
			For the typical employer, a postdoc is cheap labor. At the laboratory where I work, \& this is common, a postdoctoral employee receives minimal benefits. The lab pays for medical insurance but makes no contributions to a pension plan. Paid vacation is only 2 weeks per year, \& a postdoc salary is not loaded with substantial overhead or indirect costs.
			
			A postdoc will also be gone in 2--4 years. A helpful \& productive one will be a blessing, no doubt, \& a postdoctoral sojourn leading to a successful career can be counted a noteworthy success. But a failure by those standards is only assessed as unfortunate -- not unusual, \& not disastrous. Acquiring a postdoc, in short, is much like buying a piece of laboratory equipment. One assumes it will work for a while, helping to produce results. Then it will be replaced with a newer model. From the postdoctoral employer's viewpoint, signs of a candidate's viability are, accordingly: 1. an excellent thesis-research presentation -- this implies that the candidate will be a good spokesperson for the supervisor's research program; 2. not having taken overly long to finish the PhD -- supporting the hope that after a sojourn lasting no more than a few years, the postdoc will have produced several publications; \& 3. seriousness, knowledge, engagement, \& interactivity -- indications that the new hire will make for a livelier, more productive, \& collaborative research group.
			
			If a postdoc candidate wants to change fields, that is not a problem but a common practice. If the candidate's thesis work did not produce a major piece of new knowledge, that is not a problem either because a postdoc is hired fundamentally to further the supervisor's research program. If a postdoc breaks new ground or does something important during his postdoctoral period, he may be offered a permanent job. If not, he will go away, \& not much will have been lost. This is the employer's perspective. What should yours be?
			
			You have 3 important tasks in your postdoctoral years: You must decide in what area of science to make your name. You must \textit{finish} at least 1 significant project. \&, you must establish your identity in the research community sufficiently to land an assistant professorship or a junior position in an industrial or government laboratory. You have little time to waste because it will not be long after you begin your postdoctoral work that you will be back on the job market.
			
			These considerations imply that: 1. you do not want a position where your field of research is undefined. You want to get to work on a significant research project on arrival or shortly thereafter; 2. you do not want a position in which a complex technique is being perfected (which means that your chance of producing results in time for your job hunt is minimal). You want to be involved in 1 or several short-term projects.
			
			If you are changing fields, you want to start your reading \& learning \textit{before} you arrive at your postdoc site. The clock starts ticking when you get to your new location. Whatever you do before you leave the nest of graduate school doesn't count, for all practical purposes. Generally, it would be wise to find a mature scientist for a postdoctoral supervisor rather than a relative novice. The reasons are the same as for a thesis professor. You do not want to be in competition for resources or credit for results. If there is only 1 experimental apparatus in the laboratory, or if the group computer budget is relatively thin, do you think you will be allowed to use whichever resource as much as you need? Will an adviser who has less than 6 years before tenure review be capable of recognizing the importance of your achieving recognition after only a year or so? There is more than a little chance not, logic dictates. Thus, unless you can find an assistant professor or junior industrial researcher who is a superstar, or at the very least, unless you can satisfy yourself that the young scientist you want to work with understands \& agrees to accommodate your needs, you would probably be better off working with someone established.
			
			\textit{Keys to success as a postdoc:} Once you do take a postdoctoral position, the keys to success are: 1. \textit{finish something}; \& 2. make yourself known \& useful. Your 1st priority as a postdoc is to have something to talk about when you go job hunting. No employer wants to hire a person who starts but cannot finish projects. Even if you have put a year \& a half into developing a \textit{very} promising method, you will lose out in the job market to your competitor whose methods may be less adventurous but who has produced a kernel of new knowledge, who has written it up \& published it.
			
			I do not recommend that you be careless in your research endeavors. Nevertheless, you should be aware that it is possible \& may be desirable to publish an exciting result before the last \textit{i}'s are dotted \& \textit{t}'s are crossed. It is possible, \& relatively risk-free, if you are honest in your manuscript about the work that remains to be done. It may be desirable because someone who has a provocative story to tell, even if it is only supported by admittedly plausible evidence, will win out in the job market over someone whose very thorough effort is not far enough along to allow conclusions to be drawn. Although attention to detail is important, \& publishing results that later turn out to be incorrect is anything but desirable, \textit{finishing} projects \& having a story to tell are essential. As a postdoc, under time pressure, you may have to sacrifice your desire for perfection, you may have to live with the fear that you haven't got everything just right, in order to develop a story that you can use to sell yourself. This is not cynicism but realism, \& worth remembering for your entire career. The famous physicist Wolfgang Pauli is remembered for complaining ironically that the work of a young colleague ``isn't even wrong.'' Think about \textit{that!}
			
			\textit{Do not be a slave to your postdoctoral adviser:} If you just sit in your office working, while you are a postdoc, your supervisor will know you, but no one else will. You will get 1 good recommendation letter, assuming you have performed well, \& that is all. If you chose a thesis adviser with good connections, he may still be able to help you find a real job after your postdoc. But what you accomplished as a graduate student does not count for much in later life, unless it is very exceptional. If your thesis adviser helps you find a job via his connections, it may be looked on as being \textit{despite} your performance as a postdoc, \& the burden on you to prove yourself in a junior, continuing position may be greater than otherwise.
			
			What you really want to achieve as a postdoctoral researcher is to gain the respect of 3 or 4 staff members where you work who will write you good recommendations. If you are a theorist, plan on spending 2 or 3 hours weekly talking to experimentalists, \& vice versa. Barge into people's labs, politely, \& find out what kind of work is going on. Discover whether there are other research programs to which you can contribute. Get copies of your lab's preprints. Read them, \& if you have criticisms, questions, or contributions, make them known. Every lab is eager to employ \& to recommend interactive people.
			
			If you are congenitally shy, you have a real problem, one that it would be helpful to overcome. Try to focus on the idea that positive feedback from the people you help will help you psychologically, \& the recognition that their positive comments to others will advance your career.
			
			Above all, during your postdoc years, work hard. You have only a short time to prove yourself. Do not slack off now. There is no time to waste. Your postdoctoral years represent the most intensely important period in determining whether you will have a career.
		\end{itemize}
		
		\item {\sf Chap. 4: Giving Talks.} {\it Preparing talks that will make people want to hire \& keep you \& that will make the information you present easy to assimilate.}
		
		\item {\sf Chap. 5: Writing Papers: Publishing Without Perishing.} {\it Why it is important to write good papers. When to write up your work, how to draw the reader in, how to draw attention to your results.}
		
		The negative connotation of the clich\'e  \textit{publish or perish} is seriously misplaced. Publication is a key component of your research efforts. It is widely accepted that a scientific endeavor is not complete until it has been written up. The exercise of putting your reasoning down on paper will frequently lead you to refine your thoughts, to detect flaws in your arguments, \& perhaps to realize that your work has wide significance than you had originally imagined. Publication also has strategic significance. As a beginning scientist, not only do you work long hours for long pay, but your job security is anything but assured. To succeed, you must make your talents well known \& widely appreciated. Publishing provides you with an important way to accomplish that. Your papers, on public view around the world, represent not only your product but also your r\'esum\'e. Compelling, thoughtful, well-written articles are timeless advertisements for yourself. You can imagine that a sloppy r\'esum\'e is not worth preparing. A premature or slapdash publication is far worse. It will remain available to readers indefinitely. These thoughts raise the 2 basic questions addressed in the present chapters: \textit{When} should one write a paper, \& \textit{how} should one write it?
		\begin{itemize}
			\item {\sf Timing.} Generally, articles are written too soon in response to the fear that one's competitors will publish 1st or as a result of intellectual laziness (i.e., inattention to important details). Papers are written too late because of the fear of publishing a blunder or because of writer's block. Overcoming these fears \& frailties is necessary for \textit{everyone} in science. At the very least, the knowledge that they are not yours alone may help you deal with them. (Read Carl Djerassi's novel \textit{Cantor's Dilemma} [New York: Penguin Books, 1991] for a poignant exposition of the problem of when \& what to publish.)
			
			Planning your research as a series of relatively short, complete projects (cf. Chap. 9) is the best way to achieve a disciplined publication schedule, one that serves your interests in scientific priority, self-advertisement, \& job security. Even though you are working toward an important long-term goal, you report each project as an independent piece of work that has produced a new kernel of knowledge (only half-jokingly a ``publon,'' a quantum of publication\footnote{The concept of the ``publon'' emerged from the graduate student minds of M. J. Weber, now at the University of Virginia, \& W. Eckhart, now at the Salk Institute.}). In the introduction to each paper of a series, you place the work reported in the context of the long-term goal, to which you thereby lay claim, \& you explain how the present results take you a step closer. If your project turns out to be as significant as you had hoped, after you have published several papers in the series, no doubt you will be asked to write a review. \textit{This} will provide you with an appropriate forum for a long, definitive article, one that will be widely referred to \& will help to make your name in science.
			
			There are many advantages to writing up your work as a series of short papers. Managers \& funding agencies need concrete evidence that they have hired personnel \& spent money wisely. Nothing is more helpful in this regard than the list of publications their wisdom has fostered. Of course, they will be pleased if you eventually realize a long-term research goal. However, funding cycles are typically 2 or 3 years (cf. Chap. 8), \& renewal of junior scientific positions occurs on a similar time scale. Therefore, deans, research directors, \& contract managers cannot wait for your long-term dreams to come true. They need published evidence of your progress on an ongoing basis.
			
			By writing numerous, relatively short articles, you can keep your name in the spotlight. The titles, abstracts, \& authorship of your new papers will show up in electronic databases, generally updated weekly. Such search engines as \url{scholar.google.com}, \url{www.osti.gov/eprints},
			\& \url{www.scirus.com} will readily lead the community to manuscripts you have posted on \url{arXiv.org}, \url{precedings.nature.com}, or any of a host of other preprint servers. The number of citations of a long publication list increases more rapidly than that of a short list.
			
			You mustn't be overly cynical about these facts of scientific life. If you attempt to achieve name-recognition by padding your publication list with repetitive papers, your efforts will soon reap scorn rather than admiration. Still, the little admiration you gain for publishing an awesome magnum opus in a single paper is surely not worth the risk that this publication strategy poses to your job security.
			
			If you publish frequently, you are less likely to be ``scooped.'' The longer you hold back reporting your results, particularly if they are important, the greater the chance some other group will beat you into print. You do need to develop an appreciation for when a piece of work is complete enough to be written up. If the logic of a manuscript is clearly missing an important piece of confirmatory evidence, submitting it to a journal is likely to cause you endless, painful interactions with referees. This is the time to hold back. (Among other problems, the referees may very well be your competitors. Their own publication strategy is likely to be affected by their appreciation of where your incomplete work stands.) On the other hand, if you \textit{have} completed a project, the sooner you get it into the hands of a journal, the better the chances are that you will get credit for your accomplishment.
			
			Writing a paper that presents 1 new idea or result is much easier than writing a long, complex article. This is a reasonable way to address the problem of writer's block. Much of the introduction to a shorter paper can be prepared, at least mentally, when the long-term research project is originally proposed. The organization of a paper is simpler if there is not so much material to present, \& it is also relatively easy to explain the conclusions in that case.
			
			Referees are generally busy people \& prefer to review short papers. You are likely to receive a more thoughtful \& positive report on a short manuscript than on a long one. Shorter papers are of course not only easier on referees. They also can be read \& assimilated more easily by the scientific community at large.
			
			Writing up individual kernels of new research should have some appeal for the perfectionist. It is easier to get everything right when one is dealing with a small project than when publishing the results of a major, complex effort.
			
			Eventually, of course, all the significant details of a research project need to be reported in an archival journal so that others may repeat \& confirm the validity of the new science. Writing such technical papers is an important exercise, \& one that will win you credit from your peers if you do it well. On the other hand, in most cases the writing of such papers can be carried out at leisure.
			\item {\sf Writing Compelling Papers.} A journal article should present a careful \& relatively complete account of your research. However, it is all too easy to write an accurate description of your work that attracts no attention \& that adds little to your scientific reputation, \textit{even when your results are significant}. Learning to write articles that people will read \& remember will make you a more effective scientist. It will also enhance your chances for survival as a researcher.
			
			The structure of a news article is a good model to follow in preparing a scientific publication. Newspaper readers, like your research colleagues, rarely have much time for acquiring new information. This is just the reason that news articles present a story several times, in increasing levels of detail. Their headlines, equivalent to the titles of your scientific papers, are there to draw readers in by providing a succinct description of what is noteworthy. Scientists attempting to keep up in a world of information overload often do no more than skim the tables of contents of the leading journals in their field or conduct electronic keyword searches. You can help direct them to your new paper by taking the time to prepare an accurate \& compelling title, concise yet incorporating the most important keywords. (``Cute'' should be avoided, as a rule.)
			
			The abstract of a paper corresponds to the 1st paragraph of a news item. It summarizes the main information, what the important results are, \& what methods you used to obtain them. Numerous journals place a word limit (e.g., 75 words) on the abstract. It is a good idea to impose such a limit on yourself whether or not the journal does. An abstract that is brief \& to the point has a better chance of being read. A wordy one which reads like the introduction to or the body of a paper, will lose readers.
			
			As in the case of titles, it is worth remembering that abstracts circulate more widely than the papers they summarize. They are the 1st item to pop up when one searches journal content \& generally available without charge, even when seeing a full article requires a subscription. A well-written abstract may thus make the difference between someone's downloading your full text or emailing you for a copy, rather than just moving on.
			
			The introduction to a paper is where you tell your story, possibly illustrating the text with an important figure or some key results, but without going into great detail. Here is where you want to explain why your project was an important one to undertake \& how your results make a difference to the way we understand the world. Many busy scientists read only the introduction \& conclusion sections of papers, leaving the technical details for another time. Therefore, it is a good idea to highlight your results -- e.g., by placing your most important figure in the introduction. Even if your readers never take the time to plow through the complete description of your work in the body of your paper, they may think enough of the information in your introduction to make sure to catch your talk at the next scientific meeting.
			
			Virtually everyone finds that writing the introduction to a paper is the most difficult task. It is easy to report the procedures you followed \& to describe the data you obtained. The hard part of paper writing is drawing the reader in. My solution to this problem is to start thinking about the 1st paragraph of an article \textit{when I begin a project rather than when I complete it}. I would not embark on a scientific effort if I didn't think it was important \& that my work would answer a question of rather wide interest. The reasons that I found the project in question interesting enough to work on provide half the material I need for my introduction. The remainder is a summary of my key results. The decision to start writing a paper is generally based on recognizing that a kernel of knowledge has been produced. In my introduction, I want to let my reader know what this new information is, in a nutshell, \& why it is worth reading about. Sitting at the word processor, I imagine I am on the phone with a scientist friend whom I haven't spoken to in sometime. He asks me what I have been doing recently. I write down my imagined response. If, when you try this, you feel an attack of writer's block coming on, turn on a recording device \& actually call a friend. It works.
			
			Incidentally, if you know why you have carried out a scientific project \& what makes your results interesting, there is no reason that your paper should start with an inane clich\'e, such as, ``Recently there has been a resurgence of interest in $\ldots$ (whatever the topic),'' which bothers me every time I see it. If you have been working on a project for several months or a year solely because \textit{other} people are interested in it, you have a lot to learn about problem selection. (In this case, see Chap. 9 for some help. Do not pass go. Do not collect your next paycheck.) Before you start on a research effort, you must understand why it is important, \& in the introduction to your publication on the subject, this is just what you need to explain.
			
			In writing your introduction, as well as the body of your paper, it is essential to place your work in context, not only by explaining what you did \& why but also by citing the relevant literature. This is important, not only to provide your readers with a way of understanding your area of research, but also because your scientific colleagues are very eager to get credit for their achievements. (This is not just vanity. Scientists' careers are built on the perceived importance or usefulness of their research results.) You have much to gain \& little to lose by scrupulously citing your competitor's work. I said above that many busy scientists read only the introduction \& conclusion sections of papers. Even more move directly from the title \& abstract to the references, to see if their work is cited. If someone's papers are not mentioned there but should be, you risk losing a potential friend or at least some respect.
			
			I would add that an excellent way to keep up with developments in your field is to check, from time to time, who is citing your own papers. A ``citation index,'' such as is available on the ISI Web of Science${}^{\rm SM}$, makes this an easy task. Bear in mind as you do this that if checking citations is how people in your field keep up, an article you have written that fails to cite their work is more likely to go unnoticed.
			
			In revising \& editing your article before submitting it, you should constantly be asking yourself if you have dealt with all the loose ends in your logic. \textit{Are there arguments you have thought about \& used but not written into your text? Are you wishy-washy about inferences you have drawn, instead of forceful, because there are missing links in the logic?} If so, you either need to work a little longer before writing your paper, or be forthright about what is conjecture \& what has actually been proven. Even if the referee does not catch the weak points of your manuscript, you must not forget that your paper will be on public view for a long time. Intellectual honesty is accordingly a very good policy. This is not to say you should be such a perfectionist that you never feel comfortable declaring a project done \& ready to be published, but rather that you should own up, in print, to what you think might be weak links in your reasoning. This is a service to the community, in that it points to further research directions. It shows the world that you are a thoughtful \& forthright individual. Importantly, it also provides you an out if your reasoning is later shown to be incorrect.
			
			The format of the body of a paper is often dictated by the journal where it will be submitted. Within the journal's constraints, however, the key to organizing your work is to make your text read like a story. Often it is a good idea to relegate detailed discussion of a technical aspect of the work to an appendix. That way, experts or interested parties can try to understand your arguments in full detail, whereas others do not have to guess how much of the text to skip to move on to the next idea.
			
			Keep in mind that the function of a journal article is to communicate, not simply to indicate how wonderful your results are. In principle, a paper should provide enough information that an interested reader would be able to reproduce your work. It is your responsibility to ensure that the necessary information is made available, at the same time as you try to make your paper as snappy \& readable as you can.
			\item {\sf Snappy Papers.} In archaic times, say 30 years ago, you generally had to write your papers as though the work had actually been done by someone else. You were discouraged from using the personal pronoun ``I'' in favor of ``we'' or, even worse, ``one.'' Journals seemed to require writing papers in the passive mood, as in ``the data were obtained using the following novel method'' rather than ``I developed the following novel method to obtain the data.'' More recently, it has become possible to drop the phoniness of this style \& to reveal in your writing that \textit{you} actually did the work you are reporting. I greatly prefer the more straightforward style \& recommend that you use it.
			
			People of a mathematical bent often connect the sentences in their papers with such words as \textit{now, then, thus, however, therefore, whence, hence}, \& so forth. If you want your text to be readable to the non-pedantic, you should be very sparing in using them. Go over your 1st draft \& challenge yourself to see how many of these connectives you can remove without undermining the logic of your argument.
			
			In this era of speedy desktop computers \& full-featured graphics programs, there are few excuses for omitting evocative figures from a paper. A picture may be worth more than a thousand words in a scientific article, particularly if the thousand words are not read, but the thoughtfully prepared figure is examined \& the information it reports absorbed. This does mean it is important not to prepare figures that are too cluttered. If they offend the eye, they may be ignored along with the thousands of words.
			
			Some journals restrict the length of articles. This typically forces one to go back through the 1st draft of a manuscript to rewrite more economically. In preparing the 1st draft, it is a good idea to be as generous as possible with words. You should write down everything that comes to mind as relevant. This may not be easy but helps get all the logic on paper. (Again, get out the voice recorder if you tend to be stingy with words.) If you have written a copious text, the exercise of cutting back may be more difficult but is less likely to lead to a paper whose flow is compromised by the absence of something important. I recommend the approach of writing generously \& then editing severely in all cases -- i.e., whether or not the journal in question imposes restrictions on manuscript length. The exercise of rewriting as concisely as possible leads to more readable text \& thus to text that is read more widely.
			
			As in the preparation of a seminar, the last section of a paper should provide not just a summary of the results reported but also some idea of how they might affect the direction of future research. The goal of the conclusions section is to leave your reader thinking about how your work affects his or her own research plans. Good science opens new doors.
			\item {\sf Referees.} Last, because arguments with journal referees can take many months to settle, \& can be very frustrating, it is a good idea to forestall them by having your manuscripts reviewed locally, by 1 or 2 of your colleagues, before submission. If you have chosen your local reviewer well, you may discover the weak points in your article in a matter of days rather than months. If English is not your mother tongue (\& if you are writing for an English-language journal) it is even more important to have your paper reviewed \& edited by a colleague, one whose English is near perfect. Your readers, including your journal's referees, are human \& thus impatient to some degree. The easier you can make their task, the better will be their response to your efforts.
			
			Incidentally, as one who referees many papers, I much prefer receiving a cogent, well-written manuscript that I can learn from than the other kind. A paper that I enjoy reading disposes me favorably toward the author. Your referee may be your paper's most careful reader ever. Making a good impression on this anonymous potential employer is not a bad idea!
			
			If your referee does have serious complaints about your article, getting angry is not a productive response. A better idea is to consider why this thoughtful expert did not follow your argument \& agree with it. If on reflection you believe that your results are correct \& that the referee has simply misunderstood them, it is likely that spending some time revising your text will not only persuade the referee to recommend that your paper be published but will also ultimately make your ideas less confusing to your journal's general readership.
			\item {\sf Additional Reading.}
			\begin{itemize}
				\item Carter, Sylvester P. \textit{Writing for Your Peers: The Primary Journal Paper}. New York: Praeger, 1987.
				\item Alley, Michael. \textit{The Craft of Scientific Writing}. 3rd ed. New York: Springer Science \& Business Media, 1996.
				\item Booth, Vernon. \textit{Communicating in Science: Writing a Scientific Paper \& Speaking at Scientific Meetings}. 2nd ed. New York: Cambridge University Press, 1993.
			\end{itemize}
		\end{itemize}		
		
		\item {\sf Chap. 6: From Here to Tenure: Choosing a Career Path.} {\it An unsentimental comparison of the merits of jobs in academia, industry, \& in government laboratories.}
		
		As a scientist, your goals are to make exciting discoveries, to change the way your colleagues \& maybe even the public at large view the world, \& generally to improve people's lives. However, need I remind you, you will remain a human being, with human needs, even while you are pushing back the frontiers of ignorance. No matter how romantically you view your role in research, you will not be happy without a secure, well-paid job. You will want help in accomplishing your research goals \& recognition for your achievements. You will probably want to see your family on a regular basis \&, more generally, to have enough free time to engage in activities outside your professional life.
		
		It is all too easy to lock yourself into a situation where 1 or more of such basic desires will not be satisfied. This may adversely affect your productivity, your family life, \& your ability to enjoy yourself. Thus it is important to consider rationally, \& in advance, not only the benefits \& disadvantages of the various kinds of scientific positions -- academic, industrial, \& governmental -- but also the merits of the different roads to permanent employment.
		
		Economic conditions may limit your choices, but if you are fortunate enough to have more than 1 job possibility, this exercise will save you considerable stress. It may have a significant effect on your financial well-being. It may save your marriage. I harbor a secret hope: If enough of you start to act rationally, the system may eventually be rationalized.
		
		It is only natural to adopt as role models the people one encounters in one's formative years. For this reason, many -- perhaps most of us -- finish graduate school dreaming of an academic career. For some, the academic life may be ideal. For many, it is not. Even if being a professor \textit{is} the right goal, however, it is far from clear that rising up the academic ladder is the most desirable way to get there. My recommendations \& the reasons for them are the subject of what follows.
		\begin{itemize}
			\item {\sf The Pluses \& Minuses of a Job in Academia.} The idea that a university is an ivory tower is commonplace. The academic freedom embodied in the granting of tenure was originally supposed to protect the professoriat from political repercussions against expressions of minority views of the world. However, tenure is in itself a uniquely desirable \& economically significant benefit.
			
			\textit{Who wouldn't want the ultimate in job security?} As a tenured professor, if you fulfill minimal performance requirements (e.g., teaching a class every semester) \& maintain at least minimal moral standards (love affairs with your students are sometimes frowned upon), \& if your university doesn't shut down your department entirely in response to severe economic stress, you have a guaranteed paycheck. In face, universities have long since recognized the economic significance of tenure. University salaries would certainly have to be higher if professors were subject to being laid off.
			
			Tenure is a form of financial independence \& thus conveys corollary benefits. A university professor chooses research topics \& collaborators at will. No boss is empowered to say what to work on or to decide who will work with whom. In principle, the pace of research is also up to the professor. If energetic \& ambitious, an established professor, together with a group of students \& postdocs, may produce a dozen publications a year, or more. A ``scholar'' may publish many fewer, might be poorly funded, \& may not have much of a group. The department chair or the dean may complain, but the scholarly professor will still receive a paycheck.
			
			Although tenure \& its corollaries are the unique benefits of a professorship, they are far from the only attractive features of the job. Professors can anticipate the respect not only of class after class of students, who pay a great deal of money to be exposed to what they have to say, but also of the community at large.
			
			Typically, professors are free to sell their services as consultants, perhaps 1 day per week, to supplement their salary. Many science professors find private companies to develop the fruits of their research \& sell them for their own profit. Others write textbooks on university time \& pay, \& then are allowed to reap the royalties for themselves.
			
			Because classes are held only 9 months of the year, the remaining 3 are in principle a very long annual vacation or at worst, unprogrammed time. Sabbaticals are typically part of a university contract. Every several years, professors can look forward to 6 months or a year at a distant \& often exciting location where they can recharge their intellectual batteries, learn a new field, write a book, or basically do what they please -- \& get paid for it!
			
			Given that the job has all these wonderful benefits, you might be surprised that many professors complain about the demands of their work \& that many scientists are happy not to be members of the professoriat. What, then, are the disadvantages of living in the ivory tower?
			
			Probably the most widespread complaint is that a professor rarely has time to set foot in the lab \& to do the scientific research that used to be so much fun. Professors have so many responsibilities \& have to work so hard to fulfill them that their scientific work is mostly vicarious -- it's the students \& postdocs who do the hands-on research. To say the least, professors end up with little time for themselves. There are thankfully few tenured individuals who cynically view their permanent slot as an opportunity to do nothing (although there is generally more than enough ``dead wood'' in a department to embitter the assistant professor not promoted to tenure.) The professors I know work many more than 8 hours a day \& rarely take more than a week or 2 of vacation each year, even though in principle they could take much more.
			
			A professorship is effectively several jobs rolled into one. A professor is of course a teacher. Although there are many stories of professors whose lecture noes are yellowed with age, taking the job of teaching seriously means devoting considerable effort to making classes coherent, informative, \& up-to-date. One needs to prepare homework sets \& exams \& to develop meaningful lab exercises. One must also spend time with students during office hours. A professor is expected to be a good departmental citizen. This means attending a significant number of meetings to decide policies \& to discuss hiring \& promotions. The ambitious professor spends a great deal of time as a manager. This means writing grant proposals, traveling to Washington to meet with grant administrators, fighting for lab space, hiring \& firing students \& postdocs, \& so forth. Being an active scientific citizen, which includes refereeing manuscripts \& grant proposals, preparing \& giving lectures at other institutions, \& attending conferences, also absorbs hours. Consulting \& textbook writing come on top of that. It does not take a genius to see what professors have little time for reading a novel or playing with the kids.
			
			A job with many demands provides many opportunities for frustration. When economic times are tough, the chances of getting a proposal funded or renewed are reduced. If you have no grant money, you cannot afford to pay students \& postdocs. If you cannot spare much time to do research yourself, this means your research program will grind to a halt. Your ensuing lack of productivity will then make it harder for you to acquire funding in the future, a most unpleasant feedback mechanism. Apart from keeping yourself alive as a researcher, if your funding dries up, you may find yourself struggling to make ends meet. Typically, a university salary is only paid during the academic year, \& if you are not bringing in substantial outside money, your 9 months' pay will not be particularly generous. (The university reasons that you are unlikely to give up your sinecure for less than a major pay increase, something a poorly funded professor is unlikely to be offered elsewhere.) Your application for a research contract will therefore generally include a request for ``summer salary''; most universities allow you to receive 2 months' pay from grants. This makes getting funded intensely important to your pocketbook. If you succeed, your annual pay can increase by better than 20\%. If you don't, you may wonder why you are working so hard.
			
			Interacting with students can be a great pleasure but is often very stressful. As a teacher, you will have to deal with insistent people who want to know why their exam grades were so poor \& who want private help to understand the material you have been presenting. You will have to deal with students who cheat on tests \& with premeds who have no interest in anything but grades. Only some of your graduate students will really contribute to your research. Others will break your equipment, contaminate your samples, \& install bugs in your computer programs. Some postdocs (particularly those who haven't read this book!) will flounder for a year or 2, will be bitter about their inability to find a job, \& will complain publicly about your guidance.
			
			\textit{Your} academic freedom is certainly a great benefit, but what about that of your colleagues? In some departments, the various groups talk to each other. However, this situation is far from guaranteed. Because there is effectively no management in a university, professors tend to work independently. There is no particular reward for collaboration. This is very different from a national or industrial lab, where the job description includes helping to promote the efforts of one's professional colleagues.
			
			\textit{Assistant professorhood:} If after this litany of disadvantages, you still want to be a tenured professor, there remains the question of how to attain such a position. The most direct route is to work your way up from the bottom, i.e., to start as an assistant professor \& to be promoted. I heartily recommend that you avoid this path if at all possible.
			
			As an assistant professor, you suffer most of the disadvantages \& have few of the benefits of a tenured academic position. Not only do you have to teach, but unlike your senior colleagues, you haven't got sheaves of lecture notes from yesteryear. You start from scratch -- which means devoting many, many hours of preparation for each hour you spend in the classroom. The same is true when it comes to preparing homework assignments \& exam questions.
			
			Although being responsible about your teaching duties is necessary for you to win promotion to tenure, at a research-oriented university, it is far from sufficient. You will certainly be judged on your ability to bring in grant money. Although you will have to publish to avoid perishing, you will also have to get funded to survive. This means you will be learning the ropes of grant writing at the same time as you are trying to establish a research effort \& desperately need to produce some results.
			
			Your salary as an assistant professor, as for all professors, will not only reflect your seniority, or in this case your lack of it, but also your success at bringing in outside money. Since you are just starting out, you will have had no such success. Therefore, your salary will be miserly to poor. If you are such an exciting prospect that you have managed to land an assistant professorship at a major private university with a fancy reputation, your salary may be even worse. Such a university can expect you to accept lower pay in return for the snob appeal of its name on your r\'esum\'e. It can also offer significantly reduced opportunity if any for promotion to tenure, on the perhaps correct assumption that its name is worth more to you than job security.
			
			Unhappily, whereas full professors might accept lower pay in return for the grant of tenure, assistant professors are expected to take the low pay without the compensation of a secure position. Responding to the American Association of University Professors' (AAUP) efforts to protect you against exploitation, most schools adhere to the policy that an assistant professor who hasn't been granted tenure after 7 years must be fired. Thus, ironically, thanks to a labor organization that purports to represent your interests, you will lose your job if you are not promoted!
			
			There \textit{are} pleasures to working as an assistant professor. Teaching \& interacting with students can be exciting. The university environment is in itself very stimulating. There are certainly more kinds of people with more diverse interests than in any industrial lab. You do get respect from the community. On the other hand, the price of being an assistant professor is much too high. The hours are long, the pay is terrible, \& the job security is bad. After your years of study for a PhD \& further years as a postdoctoral apprentice, you will probably be about 30 years old. You'll probably be starting a family. Your former colleagues who went to engineering or business school will be making their way in the world, earning good salaries, \& having time to participate in activities outside their jobs. Do you want to be working 16 hours a day for half what they are earning, on the chance that after 5 or 6 years your department may give you tenure? If enough of you answers ``no,'' maybe the job conditions will improve. Until then, I recommend that you find a position in an industrial or government research lab. There you can establish a reputation with much less pain, as discussed below, \&, reputation in hand, can start at the top in a university job, if that is still what you want.
			\item {\sf Industrial \& Government Research Positions.} Research jobs in industry or at government labs have some serious disadvantages but many benefits relative to university professorships. At some of the national labs, there are tenured research positions, but for the most part tenure is not offered outside the framework of a university. You can be laid off for a variety of reasons if you work for private industry, of course, but also if you are employed at a government lab.
			
			There is no doubt that tenure is a valuable benefit. However, you should remember that your real job security as a scientist is the recognition \& approval of your peers around the world. If your published research is admired \& used by fellow scientists everywhere, you have little to fear. 1 day you may have to change job locations, but unemployment should not be a worry. Industrial \& government labs provide an environment where it is relatively easy to establish a scientific r\'esum\'e. Thus, if you are competent, the issue of tenure ends up being relatively insignificant. (Incidentally, the reluctance of the managers who hired you to admit that they made a mistake provides an additional, if melancholy, form of job security at a research lab. Firing you after 6 or 7 years if you are not promoted is not built into the system as at a university.)
			
			The most important advantage of working in a research lab, whether industrial or governmental, is that your job description is relatively simple. You are expected to be a scientific leader, to advance knowledge in 1 or more areas of importance to your employer, \& to make yourself useful to your fellow employees. The modern world being what it is, you can also anticipate being asked to help bring in funding. Because your main task is to produce results that will sooner or later benefit stockholders or the taxpayer, your lab will \textit{want} to provide you with the necessary hardware (within budgetary constraints, of course), \& if your work has a high priority, this hardware will be in the form of the latest \& highest power models. E.g., while your university colleagues are writing lengthy proposals to buy a work station, at a research lab you will be struggling to keep up with the latest upgrade to the multiteraflop, massively parallel processor. You get the idea.
			
			Because your job description at a research lab is simple, you can perform up to expectations without working unusually long hours. As a professional, you will certainly find yourself working long days occasionally, when you are on the threshold of an exciting result, or when you have to submit an article by a certain deadline. However, you will not be spending half your time doing work that is necessary but not sufficient for your survival (i.e., teaching, explaining to students why they got a D on your last exam, etc.). You will therefore have time to help your spouse with dinner, to read a novel, to see your kids' school play, or to be a soccer coach. You won't have historians, specialists in Russian literature, or bassoon professors for colleagues, so you will have to make more effort to enhance your cultural life than at a university. On the other hand, you will have more time to spend with friends from outside the workplace.
			
			A research lab is a \textit{managed} environment. We'll consider the downside of living with managers momentarily. The advantages are that management monitors the functioning of the lab \& has the power to make it work better, \& also that management is paid to do bureaucratic dirty work that would otherwise find its way to your in-box. At a government or industrial lab, significant portions of annual pay raises are awarded for merit rather than for having been employed 1 more year. There is unavoidably some arbitrariness \& subjectivity in the annual performance reviews by which merit pay is determined. Nevertheless, the fact that a group seriously considers whether your work is achieving recognition \& deserves a special reward, whether you \& your colleagues are interactive, \& whether support personnel are doing their jobs makes the atmosphere at an industrial or government lab enormously different from a university's. Employees who know that their attitudes \& performance will make a difference to their paychecks take collaboration more seriously. At a research lab, you will find librarians who offer to photocopy articles for you \& who will do electronic literature searches; you will find computer support personnel who want to advance their own careers by helping you make your computer programs more efficient, \& who will hold your hand while you are learning a new system. You will find groups of professional scientists addressing the same complex problem from several different perspectives, groups who meet to share new results \& think up succeeding experiments. At a university, such collegiality is rarer.
			
			There are many ways that management can make your life less rather than more pleasant. Abrupt changes in corporate or congressional priorities may be imposed on you if you work at a commercial or government lab. You may have to redirect your research plans, or even terminate a project before it is completed, because of your company's poor earnings or because of political changes in Washington. Your research progress may be impeded by incessant demands to take Internet or live courses -- on protection of intellectual property, ``export control,'' shop safety, types of fire extinguishers, \& you name it. Heavy-handed scientific managers may insist that it is more important for you to work on their latest (harebrained?) idea than your own. They may reinforce this by refusing to buy the equipment you want for your own purposes. They may insist that you put their name on your papers or patent applications. Or, conversely, your supervisor may have little knowledge of your field \& try to compensate by requiring you to write reports on a too-frequent basis. Management may badger you with the latest buzzwords or theories to emerge from business schools\footnote{``Empowerment,'' interpreted by many in the trenches as the ability to be blamed rather than heard, \& ``thinking outside the box'' are recent ones. Howe often do managers who never take risks themselves or think outside whatever box, urge their technical staff to do just that? \&, when a high-risk, outside-the-box project proves fruitless, who do you suppose suffers the consequences?} instead of inspiring you with rewards in the form of new instruments for your lab \& more money in your bank account. Lastly, personality conflicts with someone who has the power to fire you, to determine whether you can give an invited paper in a faraway place, \& to control the size of your paycheck can cause you plenty of grief.
			
			Obviously, if you work in a managed lab, you need to have some feeling that you will not be subject to a too-heavy hand. A bigger lab, e.g., will provide you more freedom to correct a bad situation than a smaller one would. At a large lab, if you just can't get along with your supervisor, there may be several other groups who would be happy to benefit from your wisdom \& whose supervisors would be easier to deal with. As you reputation grows, of course, your management will look to you for new ideas \& be less likely to suggest that you change directions. In a sense, this is another aspect of the reward system in a managed environment. The more credibly you play the role of a scientific leader, the more freedom you will have to follow your own research ideas. This is a real incentive, I can assure you.
			
			Management suggestions of an important research project or area, incidentally, need not always be bad. Michelangelo was asked by the pope to paint the Sistine Chapel. He didn't write his own proposal to an ``Arts Council of Rome.'' Although research driven by applications is often viewed with some disdain, the desire to fulfill a real need can \& has led to extremely important basic science -- e.g., the Nobel prize -- winning invention of the transistor -- \& has changed the world. You can \& should judge your superiors' suggested research ideas thoughtfully \& on a case-by-case basis.
			
			If you are considering a job in a commercial or government lab with the idea in mind that you will make a name for yourself \& then return in style to academic life, you must be careful to determine whether your projected position \& laboratory policies are consistent with your plan. If the research group you are considering works in an area that is important to the company in question but is of little basic scientific significance, you will very likely not be a viable competitor for an academic position several years down the track. You will have attended the wrong meetings, \& your papers will not have been read in the academic world. If your scientific results are going to be treated as proprietary information, i.e., are not going to be published, to protect commercial advantage, or if they are going to be hidden from the outside world as ``classified data,'' you will not be able to achieve recognition comparable to that of many of your contemporaries. Thus, even though their scientific competence may be no greater than yours, many of your peers will have a significant advantage over you in the competition for tenured academic positions.
			
			Apart from problems in dealing with management, 1 of the worst features of scientific life in many industrial \& government labs is a lack of helpers. Whereas a well-funded university professor can enlist an army of students \& postdocs to bring projects to fruition faster, a staff member at a research lab is lucky to have a technician \& an occasional postdoc. (This is much less of a problem in the biotech industry than in companies that perform physical research, according to my sources.) There are opportunities to alleviate such a shortage, e.g., by collaborating with a university research group. However, such opportunities must be aggressively pursued \& are unlikely in unfavorable geographic situations. Scientists who have dreams of attacking a problem from many sides at once will not be able to fulfill them at a government or industrial lab unless they can persuade colleagues to help.
			\item {\sf Money.} In deciding what kind of scientific position to aim for, you will certainly want to consider relatively pay scales. There are dramatic differences between universities \& research labs in this regard. Whereas the salary distribution for government or commercial labs is a relatively narrow bell curve whose peak is in the realm of the upper-middle class, the histogram for the professoriat is much broader.\footnote{At state-funded universities, salaries are typically public information, making it possible to compile a histogram in the campus library. In some cases, publication on the Internet makes life easier. E.g., an Internet search for ``faculty salaries cavalier'' turns up a list compiled by \textit{The Cavalier Daily} of faculty members \& their salaries at the University of Virginia at Charlottesville. In 2008, the ratio of highest to lowest pay among biologists, chemists, \& physicists on the list was about 6:1.} The university pay scale starts lower than in industry, \& the median university salary is also lower. On the other hand, the incentives for senior scientists at a university are substantially greater than at a national or commercial lab. If as a professor you bring in substantial grant money, you are very valuable to your university \&, not surprisingly, you reap big rewards. The ratio of highest to lowest salaries in a physics department might be 3:1 or 4:1, or more. In an industrial lab it is likely to be less than 2:1. In addition, at a university you can supplement your income by consulting \& by writing textbooks on university time.
			
			Financial priorities thus dictate the same career path as the scientific ones. Entry-level salaries are better in the research labs, \& the merit pay increases they provide can keep you earning more than your university colleagues until you reach the somewhat poorly defined level of ``senior scientists.'' After that, if you want to maximize your salary in industry or in a government lab, there is no alternative but to move into a management position. (1 thing managers seem to do very well is reward themselves.) If you want a high salary while keeping a hand in research, the best alternative is a full professorship. Having established an outstanding scientific reputation working 8 hours a day at a commercial or government lab, you will know what a good contract proposal looks like; you will be relatively successful at bringing in money; \& so you will have a good salary, many students \& postdocs, \& all the good things a university has to offer.
			
			Circumstances -- economic, family, or other -- may prevent you from following the optimal career trajectory. But at least I hope you will now go into the job market with a clear idea of how you would like to arrange your career \& why.
			\item {\sf Additional Reading.} Browse \url{sciencecareers.sciencemag.org}, the careers website of Science magazine.
		\end{itemize}
		
		\item {\sf Chap. 7: Job Interviews.} {\it What will happen on your interview trip; the questions you had better be prepared to answer.}
		
		\item {\sf Chap. 8: Getting Funded.} {\it What goes into an effective grant proposal; how \& when to start writing one.}
		
		\item {\sf Chap. 9: Establishing a Research Program.} {\it Tuning your research efforts to your own capabilities \& your situation in life; e.g., why not to start a 5-year project when you have a 2-year postdoctoral appointment.}
		
		``I wish I could tell you how to go about winning a Nobel prize. (I wish I could tell myself!) However, goal: considerably more modest. Want to help you see how the research program you establish will affect your chances not only of producing important science but of staying in science at all.
		
		To succeed, you will have to make a rather cold-blooded analysis of your capabilities. I.e., planning not just scientifically exciting projects but ones you can complete in good time. You need to consider how your present activities will affect your long-term interests. This may lead you to broaden your effects well beyond the field you were hired to work in. On the other hand, you should recognize when your experience gives you an advantage relative to your competitors -- a special perspective based on your work in another field, or an unusual technical capability -- \& choose projects that exploit your advantage.
		
		Although it is a good idea to build on your experience, whether by using novel techniques you have developed, complex ones that you have mastered, special reagents you have purified, or organisms that you have isolated, you will greatly improve your chances for long-term productivity \& survival in research if you can teach yourself to be problem- rather than technique-oriented. Problem-orientation means keeping clearly in mind the scientific problems you want to solve \& working toward their solution even if it means learning or developing a new technique from time to time. You want to be more than simply the master of a particular technique, uninterested in any scientific issue to which it is not applicable. If you operate in the technique-oriented mode, you are unlikely to be a scientific leader for long, \& your freedom to pursue personal research interests will probably not last. Being problem-oriented does not mean you need to master {\it every} technique necessary to solve a problem of interest -- often it will make more sense to take on a collaborator than to learn yet another method. What it does mean is that you will be primarily a {\it scientific} leader \& only secondarily a {\it technical} one.
		
		Some fields of research are riskier than others. E.g., if you work in an area sufficiently developed that there is just 1 ``big problem'' to solve, the chances that you will be the one to solve it may be rather slim. Starting your career off in an area where your contributions have a better chance of gaining recognition would seem more sensible, of somewhat less exciting.
		\begin{itemize}
			\item {\sf Timing Is Everything.} Timing is 1 of the most important issues in establishing your research direction. A problem that will take 2 years to finish must not be the main focus of your activities if you are a postdoc \& will be looking for a permanent position in a year \& a half. If your postdoctoral adviser suggests that you work on a major, long-term project, you should at the very least ask for an estimate of what you will have to show for your efforts by the time your job hunt is to begin. You might also ask whether you will continue to receive financial support if your results are still several months off when your postdoctoral term is due to end. If you hold a 2-year position \& your adviser cannot persuade you that your project has a reasonable chance of yielding publishable, noteworthy output within 18 months, say, respectfully, that you need to start on some short-term research efforts 1st, or perhaps simultaneously. If your adviser insists that you devote yourself wholly to the long-term endeavor, remember that ultimately {\it you} are responsible for your success or failure as a scientist. If your adviser (especially your young adviser) places his or her interests above your own, do not be too surprised. Seek a different group to work in, one that offers you a more realistic opportunity to produce short-term, publishable output.
			
			-- {\sf Thời gian là tất cả.} Thời gian là 1 trong những vấn đề quan trọng nhất trong việc xác định hướng nghiên cứu của bạn. Một vấn đề mất 2 năm để hoàn thành không được là trọng tâm chính trong các hoạt động của bạn nếu bạn là nghiên cứu sinh sau tiến sĩ \& sẽ tìm kiếm một vị trí cố định trong một năm \& rưỡi. Nếu cố vấn sau tiến sĩ của bạn đề xuất rằng bạn làm việc trên một dự án lớn, dài hạn, bạn ít nhất nên yêu cầu ước tính những gì bạn sẽ phải thể hiện cho những nỗ lực của mình vào thời điểm bắt đầu tìm việc. Bạn cũng có thể hỏi liệu bạn có tiếp tục nhận được hỗ trợ tài chính hay không nếu kết quả của bạn vẫn còn cách xa vài tháng khi học kỳ sau tiến sĩ của bạn kết thúc. Nếu bạn giữ một vị trí 2 năm \& cố vấn của bạn không thể thuyết phục bạn rằng dự án của bạn có cơ hội hợp lý để tạo ra đầu ra đáng chú ý, có thể xuất bản trong vòng 18 tháng, hãy nói một cách tôn trọng rằng bạn cần bắt đầu một số nỗ lực nghiên cứu ngắn hạn trước tiên hoặc có thể đồng thời. Nếu cố vấn của bạn khăng khăng rằng bạn phải cống hiến hết mình cho nỗ lực dài hạn, hãy nhớ rằng cuối cùng {\it bạn} phải chịu trách nhiệm cho thành công hay thất bại của mình với tư cách là một nhà khoa học. Nếu cố vấn của bạn (đặc biệt là cố vấn trẻ của bạn) đặt lợi ích của mình lên trên lợi ích của bạn, đừng quá ngạc nhiên. Hãy tìm một nhóm khác để làm việc, một nhóm cung cấp cho bạn cơ hội thực tế hơn để tạo ra sản phẩm ngắn hạn, có thể xuất bản.
			
			In looking for an alternative research group, do not whine about adviser number 1 to prospective adviser number 2. Your goal in interviewing for a new opportunity is to persuade the new group leader that you are mature enough to understand what is necessary to launch your career. Without complaining, you can make clear that although your initial adviser's project is one you would have liked to pursue, you fear that you are not going to be around when the important results are obtained \& published, that you will get little credit for your contributions, \& that you want to avoid living on an unemployment check 2 years hence.
			\item {\sf Timeliness vs. Importance.} Apropos ``coldblooded analysis,'' the idea that the importance of a project justifies a long-term effort is worth a critical look. Experience teaches that, important or not, a research endeavor becomes {\it timely} only once it can be approached with suitable technical infrastructure. Before then, a proposed long-term effort is likely to translate into fruitless weeks, months, or even years of struggling to make headway with inadequate tools.
			
			Because beating your head against a wall is neither satisfying nor productive, you should be wary of embarking on long-term efforts, whether formulated by yourself or suggested by a mentor or collaborator. It may make better sense to put off work on that important problem until new techniques have been developed -- perhaps by you, perhaps by somebody else -- than pushing ahead, on the assumption that brute force will eventually lead to success.
			
			Apart from whether you will be able to obtain significant results before your return to the job market or your consideration by a tenure committee, a serious peril of the brute-force approach is that a competitor will develop a labor-saving new technique \& race to the goal while you are still struggling.
			\item {\sf Technique- vs. Problem Orientation.} Most young scientists emerge from graduate school having learned a set of technical skills. Many are tempted to try to build a research program around them. This frequently leads to an unfortunate mode of thinking about what to do next, which I call, with apologies to {\sc Luigi Pirandello}, {\it6 Techniques in Search of a Problem}. The institutions that hire young scientists often reinforce the technique-oriented approach to research planning by looking for new PhD's or postdocs who have worked with a particular instrument -- e.g., at a synchrotron radiation facility -- or who have experience with a hot new technique -- e.g. scanning tunneling microscopy or transgenic organisms. If a new hire swallows the idea that he is to be ``the man at the synchrotron,'' \& particularly if he feels that he must reject ay project that does not involve synchrotron radiation, he is likely to have little impact on the world of science, with corresponding consequences to his career.
			
			When a remarkable new instrument, e.g. the laser, or a technique, like nuclear magnetic resonance spectrometry, becomes available, it is often profitable to ask how its capabilities can be applied to solving outstanding problems. Few scientists, however, are able to make a long-term success of applying their favorite technique to 1 problem after another. Eventually the well runs dry. It is the researchers who focus on a significant problem \& are willing to bring to it whatever resources are necessary who give the most absorbing talks, write the most significant papers, \& win grant support most easily. I strongly recommend that you try to teach yourself to be problem-oriented, to plan your research projects so that they address important scientific issues regardless of what techniques you \& your coworkers will need to use.
			
			The people who hired you because of a certain technical expertise may be somewhat to very disappointed when you 1st announce that you will not be spending {\it all} your time working with the synchrotron, scanning tunneling microscope, or whatever. On the other hand, they will not be pleased, some years later, if you have become obsolete along with your particular technique. If \& when you decide you need to branch out or move away from your initial technical role, you must make certain to fulfill your commitments to ongoing projects. Assuming that you do this gracefully, your group's disappointment at your change in technical focus will be tempered as your broadened effort leads you to the solution of an important science problem, enables you to win new research funding, \& maintains or enhances your standing in the research community.
			\item {\sf Strategic Thinking.} There are several strategies for establishing a record of accomplishment that will help make you more salable or will enhance your chances of winning promotion to a continuing scientific job. The most obvious is to aim at an important long-term goal by planning your work as a sequence of short-term projects. Each of the latter should yield an identifiable \& publishable milestone (a ``publon''; see Chap. 5). Your papers \& oral presentations can then begin by identifying you \& your work with an exciting research area, while the new kernel of knowledge that you describe will give confidence that you are a person who completes projects \& who will be a credit to the department that hires or keeps you.
			
			Planning \& publishing the results of short-term projects minimizes your chances of being scooped. No matter how clever you are, \& particularly if you choose to work in a fashionable research area, you will have some very clever competitors. Packaging your ideas in publishable bundles \& getting them out into the literature is important if you are to get credit (to ``establish priority'') for your work. Apart from enhancing your personal scientific reputation, this is important to the people who pay for your research \& want recognition for that.
			
			Each time you lengthen your publication list by publishing the results of a short-term project, you lower your risk factor in a potential employer's eyes. A proven producer is always preferred to a pig in a poke, \& a substantial publication list is the best evidence that you have been \& will be productive. Although professionals rightly scorn colleagues whose publication list is padded by repeated articles on the same work, you win no brownie points for writing long, multifaceted papers (cf. Chap. 5). Each time you publish the results of 1 of your short-term efforts, you advertise your productivity \& that of the institution you work in to your fellow scientists, your contract managers, \& your potential future employers. You also perform an estimable services to the research community because the timely introduction of new ideas speeds up the development of a field \& prevents duplication of effort. There is always an opportunity to write a comprehensive review when several small projects add up to a major accomplishment or discovery.
			
			Incidentally, publishing more papers rather than fewer will help you in several ways with the bean counters among those who judge you. They will not only look at the number of pages you have published but will also consult a citations database (e.g., the ISI Web of Science) to see how many pages of citations your papers have garnered. If you have published twice as many articles, this ``objective measure'' of their impact will be roughly twice as great. You may find this idea crass. I do. But it is safe to assume that there will be bean counters among those who determine your future, \& it certainly does you no harm to please them.
			
			Another important strategy for establishing a successful scientific career is to work on $> 1$ project at a time. This has several advantages: It means that when you temporarily run out of ideas related to project A, you need not waste the rest of the day, week, or month but can simply turn to project B. When a project has been completed, you do not have to spend entire days wondering what to do next but rather can budget some time to push ahead on another, one hopes publishable, piece of science.
			
			-- Một chiến lược quan trọng khác để thiết lập sự nghiệp khoa học thành công là làm việc trên $> 1$ dự án cùng một lúc. Điều này có một số lợi thế: Điều đó có nghĩa là khi bạn tạm thời hết ý tưởng liên quan đến dự án A, bạn không cần phải lãng phí phần còn lại của ngày, tuần hoặc tháng mà chỉ cần chuyển sang dự án B. Khi một dự án đã hoàn thành, bạn không phải dành cả ngày để tự hỏi phải làm gì tiếp theo mà thay vào đó có thể dành thời gian để thúc đẩy một dự án khoa học khác, hy vọng có thể xuất bản.
			
			Working on $> 1$ project is the only way a young (or any!) scientist should undertake an inherently long-term project. I spent 10 years (!!) writing a computer program to model the energetics of atoms \& molecules on metal crystal surfaces. Although I was able to publish several pieces of technical progress along the way (e.g., mathematical tricks that made portions of the computation more efficient), the really significant science output could only be produced when the computer code was substantially complete. I survived this project scientifically by establishing collaborations in which the tools required to generate results were either completely or almost completely developed. By devoting $\approx50\%$ of my time to short-term projects using these tools, I maintained a publication record -- several new papers a year -- adequate to persuade my peers \& my employer that I was not brain-dead.
			
			I do not, by the way, recommend 10-year projects as a good idea for young scientists. I waited until I had established a strong scientific reputation before risking it. But even if you want to carry out a 3-year project, having something else going on is highly recommended.
			
			Working on 2 or 3 projects simultaneously has $\ge2$ other advantages. 1 is that it forces you to be broader than otherwise. There is a strong tendency to become narrower \& deeper as you progress scientifically, particularly if you work in an industrial or government laboratory. At a university, teaching requirements counteract this tendency. Without at all wanting to argue that you should strive to be broad \& shallow or that you should spread yourself so thin that you are unable to make progress in any area, I suggest that by having your fingers in several pies, you are more likely to prosper scientifically. As 1 area loses its scientific appeal, another wit which you are already familiar many increase in importance. The clever ideas you learn or develop in 1 area may be applicable in another. This can be an extraordinarily efficient way to make progress.
			
			The 2nd advantage of having $> 1$ project underway is that it will lessen the impact on your career should you be scooped. This is something to worry about if you have chosen to work in a hot area.
			
			\item {\sf Establishing a Name for Yourself.} It is particularly important that a young researcher establish an identity in the community. Collaborating with other scientists is certainly an effective way to build up a publication record. However, except under special circumstances -- e.g., if you bring a unique \& identifiable skill to the collaboration -- most of the credit for the papers you write will go to the senior partner. Instead of your work's being referred to as ``Young Postdoc, et al.'' it will be the paper published by ``Honcho's group.'' This is independent of the fact that your name came 1st on the paper.
			
			For this reason, it is important for you to start thinking up, working on, \& publishing the results of projects where you are the sole author or perhaps the only theorist in collaboration with an experimental group. In the latter case, it is not enough just to act as the house theorist, the data analyst who performed regressions on demand. You must perceptibly contribute new ideas -- ones that your experimental colleagues would be unlikely to have produced on their own.
			\item {\sf Risky Business.} Although working in a hot area is exciting -- major meetings are mob scenes, the scent of a prize is in the air -- it is a risky business. Before moving into a fashionable field, you must ask yourself whether you have a realistic chance of emerging from the mob as someone who has made an important advance. If the problem is solved \& this hot area is the only one you know well, how long will it take you to establish yourself in another one? Are your ideas sufficiently different from others' that you can hope to beat the competition to the answer?
			
			A less risky course is to try to lead rather than follow fashion. 1 way is to think how a recent technical advance may have made a problem ripe for solution that had previously been untimely \& therefore pushed to the back burner. Another is to make the needed technical advance yourself. That may require hard work. But in compensation, you will likely not have to race to outdo competitors; few will want to invest the labor. If in the end you make a distinct advance in teh technical state of the art, you will deserve, \& win, considerable recognition.
			
			Aside from working hard, you can reduce the risk inherent in undertaking a major project by making sure that enough money is spent on it. After a research department or funding agency has invested heavily in your goals, it has a real stake in your success. It is correspondingly reluctant to admit that your project is going awry.
			
			-- Ngoài việc làm việc chăm chỉ, bạn có thể giảm thiểu rủi ro vốn có khi thực hiện một dự án lớn bằng cách đảm bảo rằng đã chi đủ tiền cho dự án đó. Sau khi một phòng nghiên cứu hoặc cơ quan tài trợ đã đầu tư mạnh vào mục tiêu của bạn, họ có cổ phần thực sự trong thành công của bạn. Tương ứng, họ miễn cưỡng thừa nhận rằng dự án của bạn đang đi chệch hướng.
			
			No one ever got ahead in science by saving money. In my own area of research, e.g., great algorithmic advances have made it possible to compute the properties of solids in a fraction of the time that was previously required. Does this mean people are requesting smaller computer budgets? Not on your life! They have scaled up the size of the problems they propose to solve. They are asking for bigger computers than currently available \& for more computer time.
			
			-- Không ai từng tiến xa hơn trong khoa học bằng cách tiết kiệm tiền. Ví dụ, trong lĩnh vực nghiên cứu của riêng tôi, những tiến bộ lớn về thuật toán đã giúp tính toán các đặc tính của chất rắn trong một phần nhỏ thời gian cần thiết trước đây. Điều này có nghĩa là mọi người đang yêu cầu ngân sách máy tính nhỏ hơn không? Không đời nào! Họ đã mở rộng quy mô của các vấn đề mà họ đề xuất để giải quyết. Họ đang yêu cầu những chiếc máy tính lớn hơn so với những chiếc máy tính hiện có \& nhiều thời gian sử dụng máy tính hơn.
			
			Ambition is rewarded in scientific life. Lack of it leads to the exit. Let your management worry about pinching pennies. That is not your job. Let the people who pay the bills know you are scientifically alive not only by publishing exciting results but also by keeping up your requests for support.
			
			-- Tham vọng được đền đáp trong cuộc sống khoa học. Thiếu nó sẽ dẫn đến sự ra đi. Hãy để ban quản lý của bạn lo lắng về việc thắt chặt chi tiêu. Đó không phải là công việc của bạn. Hãy để những người trả hóa đơn biết rằng bạn vẫn còn sống trong khoa học không chỉ bằng cách công bố những kết quả thú vị mà còn bằng cách duy trì các yêu cầu hỗ trợ của bạn.
		\end{itemize}
		
		\item {\sf Chap. 10: A Survival Checklist.} {\it Do not attempt a takeoff before being sure the flaps are down.}
		
		-- {\sf Danh sách kiểm tra sinh tồn.} {\it Không cố cất cánh trước khi chắc chắn rằng cánh tà đã hạ xuống.}
		\begin{quote}
			Your surgeon should use one but may not. Your airline pilot, thankfully, has no choice.
		\end{quote}
		Your have a lot at stake in your quest for a permanent research position, the investment of ``the best years of your life,'' 8 or 9 of them, in science-oriented higher education, \& likely several more in postdoctoral research. Referring to a checklist may help you stay on the track to success.\footnote{{\sc Atul Gawande}, {\it The Checklist Manifesto: How to Get Things Right} (New York: Metropolitan Books, 2009).} Here is what it should say:
		
		-- Bạn có rất nhiều thứ để đánh cược trong hành trình tìm kiếm vị trí nghiên cứu lâu dài, đầu tư ``những năm tháng đẹp nhất trong cuộc đời bạn,'' 8 hoặc 9 năm trong số đó, vào giáo dục đại học định hướng khoa học, \& có thể là nhiều năm nữa vào nghiên cứu sau tiến sĩ. Tham khảo danh sách kiểm tra có thể giúp bạn đi đúng hướng đến thành công.
		\begin{itemize}
			\item {\it Put yourself in the shoes of your audience}: In whatever aspect of your scientific life -- deciding how to spend your time at work, preparing a seminar, or writing \& editing a manuscript -- step outside yourself to imagine how your department, management, listeners, or readers will respond to your effort. To win a permanent research position is to seal a contract with the scientific community. That will not happen unless both sides are satisfied with the terms. You should not accept a job offer without guarantees of enough time to conduct your research, enough finding to get started, freedom to ``get a life'' outside your work, \& an adequate paycheck. Now ask yourself what terms your paymaster(s) \& scientific audience might require. To begin, ask the number-1 question: ``Would I recommend hiring a candidate who has completely taken data with high-tech instruments or learned to run a sophisticated computer program but not produced a publication?'' The answer, ``No,'' is invaluable guidance. It is a reminder that finishing projects, writing them up, \& sending them off to journals are prerequisites for winning the job of your dreams.
			
			-- {\it Hãy đặt mình vào vị trí của khán giả}: Trong bất kỳ khía cạnh nào của cuộc sống khoa học của bạn -- quyết định cách dành thời gian cho công việc, chuẩn bị hội thảo hoặc viết \& biên tập bản thảo -- hãy bước ra khỏi bản thân để tưởng tượng cách phòng ban, ban quản lý, người nghe hoặc người đọc của bạn sẽ phản ứng với nỗ lực của bạn. Để giành được một vị trí nghiên cứu lâu dài là ký kết hợp đồng với cộng đồng khoa học. Điều đó sẽ không xảy ra trừ khi cả hai bên đều hài lòng với các điều khoản. Bạn không nên chấp nhận một lời đề nghị làm việc mà không có sự đảm bảo về đủ thời gian để tiến hành nghiên cứu của mình, đủ phát hiện để bắt đầu, tự do ``có một cuộc sống'' bên ngoài công việc của bạn, \& một mức lương thỏa đáng. Bây giờ hãy tự hỏi những điều khoản nào mà người trả lương của bạn \& khán giả khoa học có thể yêu cầu. Để bắt đầu, hãy đặt câu hỏi số 1: ``Tôi có nên thuê một ứng viên đã hoàn toàn thu thập dữ liệu bằng các thiết bị công nghệ cao hoặc học cách chạy một chương trình máy tính phức tạp nhưng chưa xuất bản không?'' Câu trả lời là ``Không'' là lời khuyên vô giá. Đây là lời nhắc nhở rằng việc hoàn thành dự án, viết chúng ra và gửi chúng đến các tạp chí là điều kiện tiên quyết để giành được công việc mơ ước của bạn.
			
			Ask yourself next, ``How would I react to a poorly prepared interview talk? If the slides were cluttered \& confusing, if the arguments were unconvincing, or worse, would I be excited about hiring the speaker?'' The obvious answer is ample impetus to make your oral presentations, all of them, engaging, informative, \& persuasive. (Return to Chap. 4 for details.)
			
			-- Tiếp theo, hãy tự hỏi, ``Tôi sẽ phản ứng thế nào với một bài phát biểu phỏng vấn được chuẩn bị kém? Nếu các slide lộn xộn \& khó hiểu, nếu các lập luận không thuyết phục, hoặc tệ hơn, liệu tôi có hào hứng thuê diễn giả không?'' Câu trả lời rõ ràng là động lực dồi dào để khiến bài thuyết trình của bạn, tất cả đều hấp dẫn, nhiều thông tin, \& thuyết phục.
			\item Last, ask how you would react to a badly written research statement or paper on a candidate's publication list. Not well? Again putting yourself in the shoes of a potential employer, you will realize that becoming a merciless editor of your own writing is an excellent investment of your time. (Return to Chap. 5 for a review of specifics.)
			
			-- Cuối cùng, hãy tự hỏi bạn sẽ phản ứng thế nào với một tuyên bố nghiên cứu hoặc bài báo viết tệ trong danh sách xuất bản của ứng viên. Không ổn? Một lần nữa, hãy đặt mình vào vị trí của một nhà tuyển dụng tiềm năng, bạn sẽ nhận ra rằng trở thành một biên tập viên tàn nhẫn cho bài viết của chính mình là một khoản đầu tư thời gian tuyệt vời.
			
			In short, to {\it get} the research job of your dreams, you must learn to {\it give} what your audience wants. Putting yourself in their shoes is the best way to understand what that is.
			\item {\it Getting your priorities straight!} Should an opportunity arise to embark on a new activity, do not say ``yes'' before considering a key question: ``What is my job?'' For a common example, imagine that several months into a 2-year postdoctoral stint, while striving to complete your 1st research project, you notice an announcement, or your research adviser tells you of a competition for a grant. Should you complete? Generally, no. Your answer to ``What is my job?'' should be, ``1st \& foremost, to complete my research project,'' including writing it up \& submitting it for publication. It is {\it not} to bring in money. That is your adviser's or manager's responsibility. Once far enough along in your postdoctoral sojourn that you have submitted a paper for publication (\& $> 1$ would be better), \& confident that you have a compelling story to tell in your job hunt, {\it then} you might consider spending time preparing a grant proposal. Otherwise, for a postdoc, grant-writing is a diversion. Single-mindedness is a more likely prescription for success.
			
			Here is another example: Suppose, as a junior faculty member or starting lab scientist, you are asked to serve on a committee of a national scientific society. Should you say yes? The answer depends on anticipated work load. Decline if the job will be so burdensome that it stands in the way of your producing new science at an acceptable rate. You may win esteem by serving on a national committee \& may well build a network of influential colleagues there, but at crunch time (e.g., when you come up for tenure), you will be judged by your scientific output before anything else. Once you have won permanent employment in the research community, you can serve on all the committees you want. Not before.
			\item {\it Learn when to say no}: Can you survive as a cheerful scientist without being somewhat selfish? Likely not. If you are perceived as someone incapable of saying no to committee work, or to becoming associate editor of a journal, or to being the lead investigator -- the one responsible for collating all the many contributions -- on 1 joint project after another, you are likely to join the many scientists who are perpetually under stress \& who often seem irritable or angry. Your goal, whether in a university department or a research lab, is to win respect for your scientific product, not love for taking on whatever extra job comes your way. Once you are established \& correspondingly experienced, you will have the freedom \& ability to multitask. Even then, however, declining more than a little extraneous work is likely to make you (\& your family) happier.
			\item {\it Be thoughtful about networking opportunities}: Beyond being scientifically productive, is there a surer way to the job of your dreams than through connections? How do you become a member of the old-boy or old-girl network? Not by learning a secret handshake, but by taking advantage of opportunities to make yourself known.
			
			Begin at your desk. Have you read a stimulating paper related to your work? Has it raised compelling questions? Engage the author in an email dialogue. When you start looking for a job, he or she might recall your thoughtful queries, or your critique, \& be willing to help.
			
			At the lab where you now work, budget time to learn what people beyond your research mentor's labs are doing. Attend their seminars. Engage them in dialogue.
			
			Are you about to attend a conference? Read the abstracts. Pick out the talks most relevant to your interests, then download \& peruse their authors' recent papers. Go to the conference with questions. Meet selected authors after their sessions end.
			
			Will you be traveling? Is there a lab near your destination where you might like to work 1 day? See if you can arrange a lab visit while you are in the neighborhood. If you are still a student, apply to spend a week or a month there during the summer. In the mode of ``putting yourself in their shoes,'' think how much easier it is for an employer to hire someone he or she has met \& sized up, compared with another who has come for a brief interview visit -- \& whose recommendation letters may be inflated. Thus, aim to be the person your hoped-for employer already knows.
		\end{itemize}
		After giving a career day lecture at a midwestern state university several years ago, I was asked whether the ideas I had presented wouldn't lose their advantage if everyone adopted them. ``That's true,'' I replied, ``in the sense of the theory of the efficient marketplace -- but I'm not holding my breath.'' This chapter's checklist largely amounts to common-sense ideas. But common sense is in shorter supply than you might imagine, \& the market for permanent positions in research is correspondingly far from efficient. Thus, mind the checklist to stay on track; many others won't.
		
		-- Sau khi thuyết trình về ngày nghề nghiệp tại một trường đại học tiểu bang miền Trung Tây cách đây vài năm, tôi được hỏi liệu những ý tưởng tôi trình bày có mất đi lợi thế nếu mọi người đều áp dụng chúng không. ``Đúng vậy,'' tôi trả lời, ``theo nghĩa của lý thuyết về thị trường hiệu quả -- nhưng tôi không nín thở chờ đợi đâu.'' Danh sách kiểm tra của chương này phần lớn là những ý tưởng hợp lý. Nhưng lý lẽ hợp lý đang khan hiếm hơn bạn có thể tưởng tượng, \& thị trường cho các vị trí cố định trong nghiên cứu cũng không hiệu quả. Do đó, hãy chú ý đến danh sách kiểm tra để duy trì đúng hướng; nhiều người khác sẽ không làm vậy.
		
		\item {\sf Chap. 11: Afterthoughts.} {\it A behaviorist approach to professional success.}
		\begin{quote}
			Experience is the best teacher (but only when the experience isn't fatal).
		\end{quote}
		The tacit premise of this book is that behaviors appropriate to launching a scientific career can be learned. Many of my colleagues doubt this, throw up their hands, \& propound the Darwinian approach. They say that scientific maturity comes with experience \& cannot be taught. The fittest students will survive. The rest will not, according to the law of the science jungle. As I mentioned at the outset, adopting this fatalistic, laissez-faire viewpoint does have the advantage that busy professors need not spend time trying to teach their students science survival strategies. On the other hand, if they are wrong, then they are guilty of avoiding an important responsibility.
		
		-- Tiền đề ngầm của cuốn sách này là những hành vi phù hợp để bắt đầu sự nghiệp khoa học có thể học được. Nhiều đồng nghiệp của tôi nghi ngờ điều này, giơ tay đầu hàng, \& đề xuất cách tiếp cận của Darwin. Họ nói rằng sự trưởng thành trong khoa học đến từ kinh nghiệm \& không thể dạy được. Những sinh viên khỏe mạnh nhất sẽ sống sót. Những người còn lại thì không, theo quy luật của khu rừng khoa học. Như tôi đã đề cập ngay từ đầu, việc áp dụng quan điểm định mệnh, tự do này có lợi thế là các giáo sư bận rộn không cần phải dành thời gian cố gắng dạy cho sinh viên của mình các chiến lược sinh tồn trong khoa học. Mặt khác, nếu họ sai, thì họ có tội vì đã trốn tránh một trách nhiệm quan trọng.
		
		I take a behaviorist viewpoint. Although the inner feelings \& thoughts that go along with scientific maturity may be real \& may only come with experience, what is needed to make the transition from graduate student to professional researcher is to learn certain behaviors. It is not important whether a student prepares an adequate introduction to a seminar because my book suggests it would be a good idea, rather than because of a deep inner conviction based on experience. What {\it } important is whether the seminar ends up stimulating \& enlightening listeners. Arguments over the possibility of teaching students to be mature should not stand in the way of teaching the skills involved in giving good talks, writing excellent papers, succeeding in job interviews, \& so forth. They are not all that hard to learn, \& the underlying ideas do not tax one's intellectual powers greatly. It should be obvious that the problem with waiting for experience to dictate appropriate behaviors is that one is very likely to fail as a result of the bad experiences that are supposed to produce the appropriate feelings. {\it It is far better to learn from the bad experiences of others than from your own}.
		
		-- Tôi theo quan điểm của chủ nghĩa hành vi. Mặc dù cảm xúc bên trong \& suy nghĩ đi kèm với sự trưởng thành trong khoa học có thể là có thật \& chỉ có thể đến từ kinh nghiệm, nhưng điều cần thiết để chuyển đổi từ sinh viên sau đại học sang nhà nghiên cứu chuyên nghiệp là học một số hành vi nhất định. Không quan trọng là sinh viên có chuẩn bị phần giới thiệu đầy đủ cho hội thảo hay không vì cuốn sách của tôi gợi ý rằng đó là một ý tưởng hay, thay vì vì niềm tin sâu sắc bên trong dựa trên kinh nghiệm. Điều {\it } quan trọng là liệu hội thảo có kích thích \& khai sáng cho người nghe hay không. Những lập luận về khả năng dạy sinh viên trở nên trưởng thành không nên cản trở việc dạy các kỹ năng liên quan đến việc thuyết trình hay, viết bài báo xuất sắc, thành công trong các cuộc phỏng vấn xin việc, \& v.v. Chúng không quá khó để học, \& những ý tưởng cơ bản không đòi hỏi nhiều sức mạnh trí tuệ của một người. Rõ ràng là vấn đề khi chờ đợi kinh nghiệm quyết định hành vi phù hợp là rất có thể người ta sẽ thất bại do những trải nghiệm tồi tệ được cho là sẽ tạo ra những cảm xúc phù hợp. {\it Tốt hơn nhiều là học hỏi từ những trải nghiệm tồi tệ của người khác hơn là từ chính bạn}.
		
		The result I have hoped for in writing this book is that you will become more reflective about your career \& act in a way that is appropriate to being successful \& productive. If you stop to think about whether that talk you have been working on is well organized, whether the paper you are writing is one you will be proud of in 5 years, or whether the research program you have developed is appropriate to your station in scientific life, I will have succeeded. No matter how well you do in these regards, you will certainly still experience difficult times, have regrets about some of your choices, \& possibly fail anyway. Nevertheless, your chances for having a scientific career will be greatly improved. I wish you every success!
		
		-- Kết quả mà tôi hy vọng khi viết cuốn sách này là bạn sẽ trở nên suy ngẫm hơn về sự nghiệp của mình \& hành động theo cách phù hợp để thành công \& hiệu quả. Nếu bạn dừng lại để suy nghĩ xem bài nói chuyện mà bạn đang làm có được tổ chức tốt không, liệu bài báo bạn đang viết có phải là bài báo mà bạn sẽ tự hào sau 5 năm nữa không, hay liệu chương trình nghiên cứu bạn đã phát triển có phù hợp với vị trí của bạn trong cuộc sống khoa học hay không, thì tôi đã thành công. Cho dù bạn làm tốt như thế nào trong những khía cạnh này, bạn chắc chắn vẫn sẽ trải qua những thời điểm khó khăn, hối tiếc về một số lựa chọn của mình, \& có thể vẫn thất bại. Tuy nhiên, cơ hội để bạn có được sự nghiệp khoa học sẽ được cải thiện rất nhiều. Tôi chúc bạn mọi thành công!
	\end{itemize}
	{\sf Reader's Suggestions Are Welcome.} My view of the world of science is inevitably framed by my own experiences \& those of my colleagues. You can help subsequent editions of this book reflect a broader view of what it takes to establish a scientific career. Send anecdotes, suggestions, criticisms, \& comments.
\end{enumerate}

\subsubsection{Advanced Mathematics Book}

\begin{enumerate}
	\item \cite{Gessen2009}. Masha Gessen. {\it Perfect Rigor: A Genius \& the Mathematical Breakthrough of the Century}.\hfill{\sf[done]}
	
	\item \cite{Gessen2022}. Masha Gessen. {\it Perfect Rigor: A Genius \& the Mathematical Breakthrough of the Century -- Thiên Tài Kỳ Dị \& Đột Phá Toán Học Của Thế Kỷ}.\hfill{\sf[reading]}
	\item \cite{Giang_tich_hop_Toan_Tin_Ly}. Nguyễn Ngọc Giang. {\it Tích Hợp Toán, Tin, \& Vật Lý}.\hfill{\sf[reading]}
	
	\item \cite{Launay2022}. Micka\"el Launay. {\it Toán Học: Một Thiên Tiểu Thuyết -- Lịch Sử Toán Học Kể Từ Thời Tiền Sử Đến Nay}.\hfill{\sf[done]}
	
	\item \cite{Viet_Chua2022}. Dương Quốc Việt, Lê Văn Chua. {\it Cơ Sở Lý Thuyết Galois}.\hfill{\sf[reading]}
	
	\item \cite{Viet_Ha_Thanh_Dang_Loc2022}. Dương Quốc Việt, Lê Thị Hà, Trương Thị Hồng Thanh, Nguyễn Đạt Đăng, Nguyễn Quang Lộc. {\it Bài Tập Lý Thuyết Galois}.\hfill{\sf[reading]}
	\item Dương Quốc Việt. {\it Cở Sở Lý Thuyết Module}.
	
	\item Nguyễn Xuân Liêm. {\it Giải Tích Hàm}.
	
	\item Nguyễn Văn Khuê, Lê Mậu Hải. {\it Giáo Trình Giải Tích Hàm}.
	
	\item Lê Mậu Hải, Tăng Văn Long. {\it Bài Tập Giải Tích Hàm}.
	
	\item \cite{Viet_Nhi_number_theory_polynomial}. Dương Quốc Việt, Đàm Văn Nhỉ. {\it Cơ Sở Lý Thuyết Số \& Đa Thức}.\hfill{\sf[reading]}
	
	\item \cite{Viet_Dang_Dinh_Ha_Hanh_Minh_Thanh_Thuy2022}. Dương Quốc Việt, Nguyễn Đạt Đăng, Lê Văn Đinh, Lê Thị Hà, Đặng Đình Hanh, Đào Ngọc Minh, Trương Thị Hồng Thanh, Phan Thị Thủy. {\it Bài Tập Cơ Sở Lý Thuyết Số \& Đa Thức}.\hfill{\sf[reading]}
	
	\item Nguyễn Doãn Tuấn, Sĩ Đức Quang, Nguyễn Thị Thảo. {\it Giáo Trình Hình Học Vi Phân}.
	
	\item Trần Văn Tấn. {\it Hình Học của Nhóm Biến Đổi}.
	
	\item Nguyễn Văn Đoành. {\it Đa Tạp Khả Vi}.
	
	\item \cite{Halmos1985}. {\sc Paul Halmos}. {\it I Want To Be A Mathematician}.
	
	``The book is about the career of a professional mathematician from the 1930's--1980's. It is presented, more or less, in chronological order, from high school to retirement, but its sections are organized by substance rather than time. [$\ldots$] It expresses prejudices, it tells anecdotes, it gossips about people, \& it preaches sermons. It tells about taking prelims, looking for a job, writing a book, traveling, teaching, \& editing.
	
	50 years ago I was cocky, iconoclastic, eager, ambitious, in a hurry, ignorant, insecure. I have slowed down, mellowed (?), \& learned a few things. To some extent the book is from the me of today to the me of yore, revealing some of the secrets that I desperately wanted to know then.'' -- \cite[Overture]{Halmos1985}
	
	-- Cuốn sách nói về sự nghiệp của một nhà toán học chuyên nghiệp từ những năm 1930--1980. Nó được trình bày ít nhiều theo thứ tự thời gian, từ trung học đến nghỉ hưu, nhưng các phần của nó được sắp xếp theo nội dung hơn là theo thời gian. [$\ldots$] Nó thể hiện những thành kiến, nó kể những giai thoại, nó nói xấu về mọi người, \& nó rao giảng. Nó kể về việc tham gia các buổi sơ tuyển, tìm việc làm, viết sách, đi du lịch, giảng dạy, biên tập.
	
	50 năm trước, tôi là người tự mãn, theo chủ nghĩa bài trừ biểu tượng, háo hức, đầy tham vọng, vội vàng, ngu dốt, bất an. Tôi đã sống chậm lại, êm dịu (?), \& học được vài điều. Ở một mức độ nào đó, cuốn sách là từ tôi của ngày hôm nay đến tôi của ngày xưa, tiết lộ 1 số bí mật mà lúc đó tôi vô cùng muốn biết. 
	
	[time-traveling back to self-teach how to self-study]
	
	``6 people read every word of a typescript version of this book. Their comments (cut this out -- who was he? -- not so bad -- tone it down -- are you sure?) cheered me up, or made me mad, but, in either case, spurred me on.'' -- \cite[Thanks]{Halmos1985}
	
	-- 6 người đọc từng chữ trong phiên bản đánh máy của cuốn sách này. Những nhận xét của họ (bỏ câu này đi -- anh ấy là ai? -- không tệ lắm -- nói nhỏ lại -- bạn có chắc không?) đã cổ vũ tôi hoặc khiến tôi tức giận, nhưng, trong cả hai trường hợp, đều đã thúc đẩy tôi tiếp tục.
	
	``My expository style relies heavily on the exemplary singular, \& the construction ``everybody $\ldots$ his'' therefore comes up frequently. This ``his'' is	generic, not gendered. ``His or her'' becomes clumsy with repetition \& suggests that ``his'' alone elsewhere is masculine, which it isn't. ``Her'' alone draws attention to itself \& distracts from the topic at hand. ``Their'' solves the problem neatly but substitutes another. ``Ter'' is bolder than I am ready for. ``One's'' defeats the purpose of the construction, which is meant to be vivid \& particular. ``Its'' is too harsh a joke. Rather than play hob with the language, we feminists might adopt the position of pitying men for being forced to share their pronouns around.'' -- A note on pronouns from {\it A Handbook for Scholars}, by {\sc Mary-Claire Van Leunen} 1978.
	
	Phong cách trình bày của tôi chủ yếu dựa vào số ít mẫu mực, \& do đó, cấu trúc ``mọi người $\ldots$ his'' xuất hiện thường xuyên. ``his'' này mang tính chung chung, không phân biệt giới tính. ``His or her'' trở nên vụng về khi lặp lại \& gợi ý rằng chỉ riêng ``của anh ấy '' ở nơi khác là nam tính, nhưng không phải vậy. ``Ter'' táo bạo hơn mức tôi sẵn sàng. ``One's'' đánh bại mục đích của việc xây dựng, vốn có ý nghĩa sống động \& đặc biệt. ngôn ngữ, chúng ta, những nhà hoạt động vì nữ quyền, có thể có lập trường thương hại đàn ông vì bị buộc phải chia sẻ đại từ của họ với mọi người.
	
	\item \cite{Hung_linear_algebra}. {\sc Nguyễn Hữu Việt Hưng}. {\it Đại Số Tuyến Tính}.\hfill{\sf[reading]}
	
	\item Trần Diên Hiền, Nguyến Tiến Tài, Nguyễn Văn Ngọc. {\it Giáo Trình Lý Thuyết Số}.
	
	\item \cite{Quy_Liem2012}. Nguyễn Mạnh Quý, Nguyễn Xuân Liêm. {\it Giáo Trình Phép Tính Vi Phân \& Tích Phân của Hàm 1 Biến Số: Phần Lý Thuyết}.\hfill{\sf[reading]}
	
	\item Đoàn Quỳnh. {\it Hình Học Vi Phân}.
	
	\item Bùi Duy Hiền. {\it Bài Tập Đại Số Đại Cương}.
	
	\item \cite{Nhi_Chin_Dung_Dung_Tinh_Dung_Son_Tuan2017}. Đàm Văn Nhỉ, Văn Đức Chín, Trần Thị Hồng Nhung, Lê Xuân Dũng, Trần Trung Tình, Đào Ngọc Dũng, Đặng Xuân Sơn, Nguyễn Anh Tuấn. {\it Đa Thức -- Chuỗi \& Chuyên Đề Nâng Cao}.
	
	\item \cite{Hardy1940, Hardy1992, Hardy2022}. G. H. Hardy. {\it A Mathematician's Apology}. [\href{https://github.com/NQBH/hobby/blob/master/advanced_mathematics/Hardy2017/NQBH_Hardy2017.pdf}{pdf}]. [\href{https://github.com/NQBH/hobby/blob/master/advanced_mathematics/Hardy2017/NQBH_Hardy2017.tex}{\TeX}].\hfill{\sf[done]}
	
	\item \cite{Villani2015}. {\sc C\'edric Villani}. {\it Birth of a Theorem: A Mathematical Adventure}. {\sf[379 Amazon ratings][1586 Goodreads ratings]}
	
	{\sf Amazon review.} ``In 2010, the French mathematician {\sc C\'edric Villani} received Fields Metal, the most coveted prize in mathematics, in recognition of a proof that he devised with his close collaborator {\sc Cl\'ement Mouhot} to explain 1 of the most surprising theories in classical physics. {\it Birth of a Theorem} is {\sc Villani}'s own account of the years leading up to the award. It invites readers inside the mind of a great mathematician as he wrestles with the most important work of his career.
	
	But you don't have to understand nonlinear Landau damping to love {\it Birth of a Theorem}. It doesn't simplify or overexplain; rather, it invites readers into a collaboration. {\sc Villani}'s diaries, e-mails, \& musings enmesh you in the process of discovery. You join him in unproductive lulls \& late-night breakthroughs. You're privy to dining-hall conversations at the world's greatest research institutions. {\sc Villani} shares his favorite songs, his love of manga, \& the imaginative stories he tells his children. In mathematics, as in any creative work, it is the thinker's whole life that propels discovery -- \& with {\it Birth of a Theorem}, {\sc C\'edric Villani} welcomes you into this.''
	
	{\sf Editorial reviews.} Winner of the French-American Foundation Translation Prize in Nonfiction
	\begin{itemize}
		\item ``Riveting! A gem.'' -- {\sc Nassim Nicholas Taleb}, author of {\it The Black Swan}
		\item ``{\sc Villani} has written probably the most unlikely unputdownable thriller of the decade.'' -- {\sc Richard Morrison}, {\it The Times}
		\item ``Combining poetry, music, \& formidable sleuthing, the charismatic {\sc C\'edric Villani} skilfully unfolds the complex yet wondrous world of mathematics. {\it Birth of a Theorem} inspires \& entertains!'' -- {\sc Patti Smith}
		\item ``[{\it Birth of a Theorem}] is less about math than about mathematicians -- how they live, how they work, \& how they talk to one another.'' -- {\sc Thomas Lin}, {\it The New Yorker}
		\item ``{\it Birth of a Theorem} is a remarkable book \& I urge everyone to buy it.'' -- {\sc Alexander Masters}, {\it The Spectator}
		\item ``A fine book from a brilliant man.'' -- {\sc Ron Liddle}, {\it Sunday Times}
		\item ``[{\sc Villani}] is widely regarded as 1 of the most talented mathematicians of his generation $\ldots$ Ultimately, this is a story about the limits of what can be achieved. \& in that respect it has everything: partnership, courage, doubt, \& anxiety, elation \& despair. {\sc Villani}'s path to success was not always easy, \& he writes vividly of his setbacks \& obstacles, detailing the inner monologue of self-doubt that we all experience, regardless of our ability.'' -- {\sc Hannah Fry}, {\it The Guardian}
		\item ``{\it Birth of a Theorem} succeeds in giving us a glimpse $\ldots$ of what it feels like to be {\sc C\'edric Villani}.'' -- {\sc Evelyn Lamb}, {\it Scientific American} Blog Network
		\item ``[Provides] a view of the math community not often seen by the general public $\ldots$ {\sc Villani}'s book eloquently humanizes mathematicians \& is inexplicably fascinating even for the layperson.'' -- {\it Publishers Weekly}
		\item ``{\sc C\'edric Villani}'s {\it Birth of a Theorem} is like no other book about math: an unfiltered view into the daily life, \& the soul, of a great mathematician, as he approaches \& finally conquers a major result.'' -- {\sc Jordan Ellenberg}, author of {\it How Not to Be Wrong: The Power of Mathematical Thinking}
		\item ``[{\sc Villani}] could plainly do for mathematics what {\sc Brian Cox} has done for physics $\ldots$ [{\it Birth of a Theorem}] is 1 of the most peculiar \& entertaining science books you will ever read $\ldots$ He realizes that what seems too obvious to him -- the beauty of maths -- is baffling to almost everybody else, \& he wants to break down the barrier this creates, not by condescendingly trying to be normal, by by being {\sc C\'edric Villani}. As maths is, as I say, the language that can make or break us, this is an urgent task that only {\sc Villani} \& only this book are addressing.'' -- {\sc Bryan Appleyard}, {\it Sunday Times}
		\item ``{\sc Villani}'s flair for storytelling, drawing on fables, metaphors, \& anecdotes, ensures that [{\it Birth of a Theorem}] is never boring.'' -- {\sc Stephen Muirhead}, {\it Chalkdust}
		\item ``{\it Birth of a Theorem} should not be read as a book about mathematics or a mathematician. It is a book about life \& a man whose zest for life is insatiable. Read it if you enjoy knowing that when approached in the right spirit by someone of sufficient energy \& talent, life can be beautiful.'' -- {\sc Daniel W. Stroock}, {\it Notices of the American Mathematical Society}
		\item ``Compellingly readable $\ldots$ I am not aware of any other account that so lucidly describes the desolation felt by mathematicians when a solution simply refuses to be found \ldots But as {\it Birth of a Theorem} shows, the exhilaration when a breakthrough occurs is beyond compare.'' -- {\sc Noel-Ann Bradshaw}, {\it Times Higher Education}
		\item ``A refreshing alternative to most pop-maths books $\ldots$ {\sc Villani} pours you inside his mind \& swirls you around, leaving you with nothing to hold on to \& breathlessly wondering what you'll encounter next.'' -- {\sc Jacob Aron}, {\it New Scientist}
	\end{itemize}
	{\sf About the Author.} {\sc C\'edric Villani} is the director of the Institut Henri Poincaré in Paris \& a professor of mathematics at the Université de Lyon. His work on PDEs \& various topics in mathematical physics has been honored by a number of awards, including the Fermat Prize \& the Henri Poincar\'e Prize. He received the Fields Medal in 2010 for results concerning Landau damping \& the Boltzmann equation.
	
	``{\sc C\'edric Villani} is a French mathematician, professor at Lyon University \& director of the Institut Henri Poincaré in Paris. He has made groundbreaking contributions in PDEs, calculations of variations \& mathematical physics, working on such subjects as the evolution of gases \& plasmas, non-Euclidean geometry, \& entropy.
	
	His work has gained him many international awards including the Jacques Herbrand Prize, the Prize of European Mathematical Society, the Fermat Prize, the Henri Poincar\'e Prize, \& most importantly the Fields Medal, attributed every 4 years to a maximum of 4 young mathematicians, \& generally viewed as `the mathematicians' Nobel Prize'.
	
	Since winning the Fields Medal {\sc Villani} has been extremely active as a spokesperson for the mathematical community, through public lectures, media, books, \& movies. His book {\it Th\'eor\`em vivant}, in English {\it Birth of a Theorem}, was an international hit -- it has been translated into 12 languages \& sold $> 100000$ copies.''
	
	{\bf Preface.} ``I am often asked what it's like to be a mathematician -- what a mathematician's daily life is like, how a mathematician's work gets done. In the pages that follow I try to answer these questions.
	
	This book tells the story of a mathematical journey, a quest, from the moment when the decision is made to venture forth into the unknown until the moment when the article announcing a new result -- a new {\it theorem} -- is accepted for publication in an international journal.
	
	Far from moving swiftly between these 2 points, in a straight line, the mathematician moves forward haltingly, along a long \& winding road. He meets with obstacles, suffers setbacks, sometimes loses his way. As we all do from time to time.
	
	-- Thay vì di chuyển nhanh chóng giữa 2 điểm này, trên một đường thẳng, nhà toán học tiến về phía trước một cách ngập ngừng, dọc theo một con đường dài \& quanh co. Anh ta gặp trở ngại, thất bại, đôi khi lạc lối. Như tất cả chúng ta thỉnh thoảng vẫn làm.
	
	Apart from a few insignificant details, the story I have told here is in agreement with reality, or at least with reality as I experienced it. [$\ldots$] {\sc C\'edric Villani}, Paris, Dec 2011.'' -- \cite[p. 5]{Villani2015}
	
	\fbox{1} Lyon, Mar 23, 2008. ``1 o'clock on a Sunday afternoon. Normally the laboratory would be deserted, were it not for 2 busy mathematicians in need of a quiet place to talk -- the office that I've occupied for  8 years now on the 3rd floor of a building on the campus of the \'Ecole Normale Sup\'erieure in Lyon.
	
	I'm seated in a comfortable armchair, insistently tapping my fingers on the large desk in front of me. My fingers are spread apart like the legs of a spider. Just as my piano teacher trained me to do, years ago.
	
	To my left, on a separate table, a computer workstation. To my right a cabinet containing several hundred works of mathematics \& physics. Behind me, neatly arranged on long shelves, thousands \& thousands of pages of articles, lawfully photocopied back in the days when scientific journals were  still printed on paper, \& a great many mathematical monographs, unlawfully photocopied back in the days when I didn't make enough money to buy all of the books I wanted. There are also a good 3 feet of rough drafts of my own work, meticulously achieved over many years, \& quite as many feet of handwritten notes, the legacy of hours \& hours spent listening to research talks. In front of me, Gaspard, my laptop computer, named in honor of Gaspard Monge, the great mathematician \& revolutionary. \& a stack of pages covered with mathematical symbols -- more notes from every 1 of the 8 corners of the world, assembled especially for this occasion.
	
	-- Bên trái tôi, trên một chiếc bàn riêng, một máy tính. Bên phải tôi là một chiếc tủ chứa hàng trăm tác phẩm toán học \& vật lý. Phía sau tôi, được xếp ngay ngắn trên những kệ dài, hàng nghìn \& hàng nghìn trang bài báo, được sao chụp hợp pháp từ thời các tạp chí khoa học vẫn còn được in trên giấy, \& rất nhiều chuyên khảo toán học, được sao chép bất hợp pháp vào thời tôi chưa làm vậy. không kiếm đủ tiền để mua tất cả những cuốn sách tôi muốn. Ngoài ra còn có khoảng 3 feet bản thảo thô dài về tác phẩm của chính tôi, được hoàn thiện một cách tỉ mỉ trong nhiều năm, \& khá nhiều feet ghi chú viết tay, di sản của hàng giờ \& giờ dành để nghe các cuộc nói chuyện nghiên cứu. Trước mặt tôi là Gaspard, chiếc máy tính xách tay của tôi, được đặt tên để vinh danh Gaspard Monge, nhà toán học \& nhà cách mạng vĩ đại. \& một chồng trang đầy các ký hiệu toán học -- thêm nhiều ghi chú từ mỗi nơi trong số 8 nơi trên thế giới, được tập hợp đặc biệt cho dịp này.
	
	My partner, Cl\'ement Mouhot, stands to 1 side of the great whiteboard that takes up the entire wall in front of me, marker in hand, eyes sparkling.
	
	``So what's up? Your message was pretty vague.''
	
	``My old demon's back again -- regularity for the inhomogeneous Boltzmann.''
	
	``Conditional regularity? You mean, modulo minimal regularity bounds?''
	
	``No, unconditional.''
	
	``Completely? Not even in a perturbative framework? You really think it's possible?''
	
	``Yes, I do. I've been working on it again for a while now \& I've made pretty good progress. I have some ideas. But now I'm stuck. I broke the problem down using a series of scale models, but even the simplest one baffles me. I thought I'd gotten a handle on it with a maximum principle argument, but everything fell apart. I need to talk.''
	
	-- Tôi đồng ý. Tôi đã làm việc lại được một thời gian \& tôi đã đạt được tiến bộ khá tốt. Tôi có một số ý tưởng. Nhưng bây giờ tôi bị mắc kẹt. Tôi đã giải quyết vấn đề bằng cách sử dụng một loạt mô hình tỷ lệ, nhưng ngay cả mô hình đơn giản nhất cũng khiến tôi bối rối. Tôi tưởng mình đã giải quyết được vấn đề bằng lập luận mang tính nguyên tắc tối đa, nhưng mọi thứ đã sụp đổ. Tôi cần nói chuyện.
	
	``Go on, I'm listening $\ldots$''
	
	I went on for a long time. About the result I have in mind, the attempts I've made so far, the various pieces I can't fit together, the logical puzzle that so far has defeated me. The Boltzmann equation remains intractable.
	
	Ah, the Boltzmann! The most beautiful equation in the world, as I once described it to a journalist. I fell under its spell when I was young -- when I was writing my doctoral thesis. Since then I've studied every aspect of it. It's all there in Boltzmann's equation: statistical physics, time's arrow, fluid mechanics, probability theory, information theory, Fourier analysis, \& more. Some people say that I understand the mathematical world of this equation better than anyone alive.
	
	-- À, Boltzmann! Phương trình đẹp nhất trên thế giới, như tôi đã từng mô tả nó với một nhà báo. Tôi đã bị mê hoặc bởi nó khi còn trẻ - khi tôi đang viết luận án tiến sĩ. Kể từ đó tôi đã nghiên cứu mọi khía cạnh của nó. Tất cả đều có trong phương trình Boltzmann: vật lý thống kê, mũi tên thời gian, cơ học chất lỏng, lý thuyết xác suất, lý thuyết thông tin, phân tích Fourier, \& hơn thế nữa. Một số người nói rằng tôi hiểu thế giới toán học của phương trình này hơn bất kỳ ai còn sống.
	
	7 years ago I initiated {\sc Cl\'ement} into this mysterious world when he began his own thesis under my direction. He was eager to learn. Certainly he's the only person who has read everything I've written on Boltzmann's equation. Now {\sc Cl\'ement} is a respected member of the profession, a mathematician in his own right, brilliant, eager to get on with his own research.
	
	7 years ago I helped him get started; today I'm the one who needs help. The problem I've chosen to work on is exceedingly difficult. I'll never solve it by myself. I've got to be able to explain what I've done so far to someone who knows the theory inside out.
	
	``Let's assume grazing collisions, okay? A model without cutoff. Then the equation behaves like a fractional diffusion, degenerate, of course, but a diffusion just the same, \& as soon as you've got bounds on density \& temperature you can apply a Moser-style iteration scheme, modified to take nonlocality into account.''
	
	-- Hãy giả sử va chạm khi sượt qua, được chứ? Một mô hình không có điểm cắt. Sau đó, phương trình hoạt động giống như một sự khuếch tán phân đoạn, suy biến, tất nhiên, nhưng một sự khuếch tán giống nhau, \& ngay khi bạn đạt được giới hạn về mật độ \& nhiệt độ, bạn có thể áp dụng sơ đồ lặp kiểu Moser, được sửa đổi để lấy tính phi định xứ tính đến.
	
	``A Moser scheme? Hmmmmm $\ldots$ Hold on a moment, I need to write this down.''
	
	``Yes, a Moser-style scheme. The key is that the Boltzmann operator $\ldots$ true, the operator is bilinear, it's not local, but even so it's basically in divergence form -- that's what makes the Moser scheme work. You make a nonlinear function change, you raise the power $\ldots$ You need a little more than temperature, of course, there's a matrix of moments of order 2 that have to be controlled. But the positivity is the main thing.''
	
	-- Vâng, một sơ đồ kiểu Moser. Điều quan trọng là toán tử Boltzmann $\ldots$ true, toán tử này là song tuyến tính, nó không cục bộ, nhưng dù vậy về cơ bản nó vẫn ở dạng phân kỳ -- đó là điều khiến sơ đồ Moser hoạt động. Bạn thực hiện một thay đổi hàm phi tuyến, bạn tăng công suất $\ldots$. Tất nhiên, bạn cần nhiều hơn một chút ngoài nhiệt độ, có một ma trận các khoảnh khắc cấp 2 phải được kiểm soát. Nhưng tính dương là điều chính.
	
	``Sorry, I don't follow -- why isn't temperature enough?''
	
	I paused to explain why, at some length. We discussed. We argued. Before long the board was flooded with symbols. Cl\'ement was still unsure about the positivity. How can strict positivity be proved without any regularity bound? Is such a thing even imaginable?
	
	-- Tôi dừng lại để giải thích tại sao, hơi dài dòng. Chúng tôi đã thảo luận. Chúng tôi đã thảo luận. Chẳng bao lâu, bảng tràn ngập các biểu tượng. Cl\'ement vẫn chưa chắc chắn về tính tích cực. Làm thế nào có thể chứng minh tính tích cực nghiêm ngặt mà không có bất kỳ ràng buộc đều đặn nào? Chuyện như vậy có thể tưởng tượng được không?
	
	``It's not so shocking, when you think about it: collisions produce lower bounds; so does transport, in a confined system. So it makes sense. Unless we're completely missing something, the 2 effects ought to reinforce each other. Bernt tried a while ago, he gave up. A whole bunch of people have tried, but no one's had any luck so far. Still, it's plausible.''
	
	-- Sẽ không quá sốc khi bạn nghĩ về điều đó: va chạm tạo ra giới hạn thấp hơn; vận chuyển cũng vậy, trong một hệ thống hạn chế. Trừ khi chúng ta hoàn toàn thiếu thứ gì đó, nếu không thì 2 hiệu ứng này sẽ củng cố lẫn nhau. Bernt đã thử cách đây không lâu, anh ấy đã bỏ cuộc. Rất nhiều người đã thử, nhưng cho đến nay vẫn chưa có ai gặp may mắn. Tuy nhiên, nó vẫn hợp lý.
	
	``You're sure that the transport is going to turn out to be positive without regularity? \& yet without collisions, you bring over the same density value, it doesn't become more positive--''
	
	-- Bạn có chắc chắn rằng việc vận chuyển sẽ trở nên tích cực mà không đều đặn? \& nhưng không có va chạm, bạn mang lại cùng một giá trị mật độ, nó không trở nên tích cực{\tt/}dương tính hơn--
	
	``I know, but when you average the velocities, it strengthens the positivity -- a little like what happens with the averaging lemmas for kinetic equations. But here we're dealing with positivity, not regularity. No one's really looked at it from this angle before. Which reminds me $\ldots$ when was it? That's it! 2 years go, at Princeton, a Chinese postdoc asked me a somewhat similar question. You take a transport equation, in the torus, say. Assuming zero regularity, you want to show that the spatial density becomes strictly positive. Without regularity! He could do it for free transport, \& for something more general on small time scales, but for larger times he was stymied $\ldots$ I remember asking other people about it at the time, but no one had a convincing answer.''
	
	-- Tôi biết, nhưng khi bạn lấy trung bình các vận tốc, nó sẽ củng cố tính dương -- hơi giống với những gì xảy ra với các bổ đề lấy trung bình cho các phương trình động học. Nhưng ở đây chúng ta đang nói đến sự tích cực chứ không phải sự đều đặn. Trước đây chưa có ai thực sự nhìn nó từ góc độ này. Điều này làm tôi nhớ $\ldots$ là khi nào? Thế thôi! 2 năm sau, tại Princeton, một postdoc người Trung Quốc đã hỏi tôi một câu hỏi tương tự. Bạn lấy một phương trình vận chuyển, ví dụ như trong hình xuyến. Giả sử độ đều đặn bằng 0, bạn muốn chứng minh rằng mật độ không gian trở nên dương hoàn toàn. Không đều đặn! Anh ấy có thể làm điều đó để được vận chuyển miễn phí, \& cho một điều gì đó tổng quát hơn trong khoảng thời gian nhỏ, nhưng đối với những thời điểm lớn hơn, anh ấy bị cản trở $\ldots$. Tôi nhớ đã hỏi người khác về điều đó vào thời điểm đó, nhưng không ai có câu trả lời thuyết phục.
		
	``Back up. How did he handle the simple free transport case?''
	
	``Free transport'' is a piece of jargon that refers to an ideal gas in which the particles do not interact. The model is too simplified to be at all realistic, but you can still learn a lot from it.''
	
	-- Vận chuyển tự do'' là một thuật ngữ dùng để chỉ một loại khí lý tưởng trong đó các hạt không tương tác. Mô hình này quá đơn giản nên không thể thực tế được, nhưng bạn vẫn có thể học được nhiều điều từ nó.
	
	``Not sure -- but it should be obvious from an explicit solution. Let's try to figure it out, right now $\ldots$''
	
	Each of us set about reconstructing the argument that this postdoc, Dong Li, must have developed. No big deal, more like a minor exercise in problem solving. But maybe it will help us resolve that great enigma, who knows? \& besides, it's a contest -- who can come up with the answer 1st? We scribbled away in silence for a few minutes. I won.
	
	``I think I've got it.''
	
	I got up \& went over to the board, just like in school when the teacher shows the class how to solve a problem.
	
	``You break down the solution in terms of the replicas of the torus $\ldots$ you change the variables in each piece $\ldots$ a Jacobian drops out, you use the Lipschitz regularity $\ldots$ \& finally you end up with convergence in $\frac{1}{t}$. Slow, but it looks about right.''
	
	``But then you don't have regularization $\ldots$ you get convergence by averaging $\ldots$ by averaging $\ldots$''
	
	Cl\'ement was thinking out loud, staring at my calculation. Suddenly his face lit up. In a state of great excitement, he jabbed at the board with his index finger: ``But then you'd have to check to see whether that helps with Landau damping!''
	
	-- Cl\'ement đang suy nghĩ lung tung, nhìn chằm chằm vào phép tính của tôi. Đột nhiên mặt anh ta sáng bừng lên. Trong trạng thái phấn khích tột độ, anh ta dùng ngón trỏ chọc mạnh vào bảng: ``Nhưng sau đó bạn phải kiểm tra xem liệu điều đó có giúp giảm chấn Landau không!''
	
	I was at a loss for words. 3 seconds of silence. A vague feeling this could be important.
	
	Now it was my turn to ask Cl\'ement to explain. He didn't know what to say either. He hemmed \& hawed, shifting his weight from 1 foot to the other. Then he said that my solution reminded him of a conversation he'd had 3 years ago with a Chinese-born mathematician in the United States, Yan Guo, at Brown.
	
	-- Bây giờ đến lượt tôi yêu cầu Cl\'ement giải thích. Anh cũng không biết phải nói gì. Anh ta gập người lại, chuyển trọng lượng của mình từ chân này sang chân kia. Sau đó, anh ấy nói rằng giải pháp của tôi khiến anh ấy nhớ đến cuộc trò chuyện cách đây 3 năm với một nhà toán học gốc Hoa ở Hoa Kỳ, Yan Guo, tại Brown.
	
	``In Landau damping you want to have relaxation for a reversible equation--''
	
	``Yes, yes, I know. But doesn't interaction play a role? We're not dealing with the Vlasov here, it's just free transport!''
	
	``Okay, maybe you're right, interaction must play a role -- in which case $\ldots$ the convergence should be exponential. Do you think $\frac{1}{t}$ is optimal?''
	
	``Sounds right to me. What do you think?''
	
	``But what if the regularity was stronger? Wouldn't it be better if it was?''
	
	I groaned. Doubt mixed with concentration, interest with frustration.
	
	-- Tôi rên rỉ. Nghi ngờ xen lẫn sự tập trung, sự quan tâm xen lẫn sự thất vọng.
	
	We stood in silence, staring at each other, wondering where to go from here. After a while conversation resumed. As fascinating as it is, the weird (\& possibly mythical) phenomenon of Landau damping has nothing to do with what we've set out to accomplish. A few more minutes passed \& we'd moved on to something else. We talked for a long time. 1 topic led to another. We took notes, we argued, we got annoyed with each other, we reached agreement about a few things, we prepared a plan of attack. When we left my office a few hours later, Landau damping was nevertheless on our long list of homework assignments.
	
	The {\it Boltzmann equation},
	\begin{equation*}
		\frac{\partial f}{\partial t} + {\bf v}\cdot\nabla_{\bf x}f = \int_{\mathbb{R}^3}\int_{\mathbb{S}^2} |{\bf v} - {\bf v}_\star|\left[f({\bf v}')f({\bf v}_\star') - f({\bf v})f({\bf v}_\star)\right]\,{\rm d}{\bf v}_\star\,{\rm d}\sigma,
	\end{equation*}
	discovered around 1870, models the evolution of a rarefied gas made of billions \& billions of particles that collide with one another. The statistical distribution of the positions \& velocities of these particles is represented by a function $f(t,{\bf x},{\bf v})$, which at time $t$ indicates the density of particles whose position is (roughly) $\bf x$ \& whose velocity is (roughly) $\bf v$.
	
	Ludwig Boltzmann was the 1st to express the statistical notion of \fbox{entropy, or disorder}, in a gas:
	\begin{equation*}
		S = -\iint f\log f\,{\rm d}{\bf x}\,{\rm d}{\bf v}.
	\end{equation*}
	By means of this equation he was able to prove that, moving from an initial arbitrarily fixed state, entropy can only increase over time, never decrease. Left to its own devices, in other words, {\it the gas spontaneously becomes more \& more disordered}. He also proved that this process is {\it irreversible}.
	
	-- Bằng phương trình này, ông đã có thể chứng minh rằng, khi chuyển từ trạng thái cố định tùy ý ban đầu, entropy chỉ có thể tăng theo thời gian chứ không bao giờ giảm. Nói cách khác, hãy để nó tự vận hành, {\it khí tự phát trở nên \& rối loạn hơn}. Ông cũng chứng minh rằng quá trình này là {\it không thể đảo ngược}.
	
	In stating the principle of entropy increase, Boltzmann reformulated a law that had been discovered a few decades earlier, the {\it 2nd law of thermodynamics}. But he did several things that enriched it immeasurably from the conceptual point of view. 1st, by providing a rigorous proof, he placed an experimentally observed regularity that had been elevated to the status of a natural law on a secure theoretical foundation; next, he introduced an extraordinarily fruitful mathematical interpretation of a mysterious phenomenon; finally, he \fbox{reconciled microscopic physics} -- unpredictable, chaotic, \& reversible -- \fbox{with macroscopic physics} -- predictable, stable, \& irreversible. These achievements earned Boltzmann a place of honor in the pantheon of theoretical physicists \& stimulated an enduring interest in his work among epistemologists \& philosophers of science.
	
	-- Khi phát biểu nguyên lý tăng entropy, Boltzmann đã xây dựng lại một định luật đã được phát hiện vài thập kỷ trước đó, {\it định luật thứ 2 của nhiệt động lực học}. Nhưng anh ấy đã làm một số điều làm phong phú thêm nó một cách vô cùng từ quan điểm khái niệm. Thứ nhất, bằng cách đưa ra một bằng chứng chặt chẽ, ông đã đặt một quy luật được quan sát bằng thực nghiệm đã được nâng lên thành quy luật tự nhiên trên nền tảng lý thuyết chắc chắn; tiếp theo, ông đưa ra một cách giải thích toán học cực kỳ hiệu quả về một hiện tượng bí ẩn; cuối cùng, ông đã dung hòa được vật lý vi mô -- không thể đoán trước, hỗn loạn, \& thuận nghịch -- với vật lý vĩ mô -- có thể dự đoán được, ổn định, \& không thể đảo ngược. Những thành tựu này đã mang lại cho Boltzmann một vị trí danh dự trong đền thờ các nhà vật lý lý thuyết \& đã kích thích sự quan tâm lâu dài đến công việc của ông trong giới các nhà nhận thức luận \& các triết gia khoa học.
	
	Additionally, Boltzmann defined the equilibrium state of a statistical system as the state of maximum entropy, thus founding a vast field of research known as {\it equilibrium statistical physics}. In so doing, he demonstrated that {\it the most disordered state is the most natural state of all}.
	
	-- Ngoài ra, Boltzmann đã định nghĩa trạng thái cân bằng của một hệ thống thống kê là trạng thái entropy cực đại, từ đó thành lập một lĩnh vực nghiên cứu rộng lớn được gọi là {\it vật lý thống kê cân bằng}. Khi làm như vậy, ông đã chứng minh rằng {\it trạng thái rối loạn nhất lại là trạng thái tự nhiên nhất}.
	
	The triumphant young Boltzmann turned into a tormented old man who took his own life, in 1906. His treatise on the theory of gases appears in retrospect to have been 1 of the most important scientific works of the 19th century. \& yet its predictions, though they have been repeatedly confirmed by experiment, still await a satisfactory mathematical explanation. 1 of the missing pieces of the puzzle is an understanding of the regularity of solutions to the Boltzmann equation. Despite this persistent uncertainty, or perhaps because of it, the Boltzmann equation is now the object of intensive theoretical investigation by an international community of mathematicians, physicists, \& engineers who gather by the hundreds at conferences on rarefied gas dynamics \& many other meetings every year.'' -- \cite[pp. 3--9]{Villani2015}
	
	-- Chàng trai trẻ chiến thắng Boltzmann đã biến thành một ông già đau khổ \& tự kết liễu đời mình vào năm 1906. Luận thuyết về lý thuyết chất khí của ông nhìn lại có vẻ là một trong những công trình khoa học quan trọng nhất của thế kỷ 19. \& tuy nhiên những dự đoán của nó, mặc dù đã được xác nhận nhiều lần bằng thực nghiệm, vẫn đang chờ một lời giải thích toán học thỏa đáng. Một trong những mảnh ghép còn thiếu của câu đố là sự hiểu biết về tính quy luật của nghiệm của phương trình Boltzmann. Bất chấp sự không chắc chắn dai dẳng này, hoặc có lẽ vì nó, phương trình Boltzmann hiện là đối tượng nghiên cứu lý thuyết chuyên sâu của một cộng đồng quốc tế gồm các nhà toán học, nhà vật lý, \& kỹ sư tập hợp hàng trăm người tại các hội nghị về động lực khí loãng \& nhiều cuộc họp khác hàng năm.
	
	\fbox{2} Lyon, Last week of Mar 2008. Landau damping! In the days following our working session, a confused series of recollections came to me -- snatches of conversation, discussions begun but never finished $\ldots$ Plasma physicists have long been used to the idea of Landau damping. But as far as mathematicians are concerned, the phenomenon remains a mystery.
	
	In Dec 2006 I was visiting Oberwolfach, the legendary institute for mathematical research deep in the heart of the Black Forest, a retreat where mathematicians come \& go in an unending ballet of the the mind, giving talks on every subject imaginable. No locks on the doors, an open bar, cakes \& pastries galore, small wooden cash boxes in which you put payment for food \& drinks, tables at which your seat is determined by drawing lots.
	
	-- Vào tháng 12 năm 2006, tôi đến thăm Oberwolfach, viện nghiên cứu toán học huyền thoại nằm sâu trong lòng Rừng Đen, nơi các nhà toán học đến \& đi trong một vở ballet không ngừng nghỉ của tâm trí, thuyết trình về mọi chủ đề có thể tưởng tượng được. Không có ổ khóa trên cửa, một quầy bar mở, vô số bánh ngọt \& bánh ngọt, những hộp đựng tiền nhỏ bằng gỗ để bạn thanh toán đồ ăn \& đồ uống, những chiếc bàn nơi chỗ ngồi của bạn được xác định bằng cách rút thăm.
	
	1 day chance placed me at the same table with 2 Americans, {\sc Robert Glassey \& Eric Carlen}, both of them authorities on the kinetic theory of gases. The evening before, at the opening of that week's seminar, I had proudly presented a whole batch of new results, \& that same morning Eric had given a truly memorable performance, bursting with energy \& jam-packed with ideas. The 2 events, coming one right after the other, were a bit overwhelming for Robert, who confessed to feeling old \& worn out. ``Time to retire,'' he sighed, ``Retire?'' Eric exclaimed in disbelief. There's never been a more exciting time in the theory of gases! ``Retire?'' I cried. Just when we are so urgently in need of the wisdom this man has accumulated in his 35 yeas as a professional mathematician!
	
	-- Một ngày nọ, tình cờ đã đặt tôi vào cùng bàn với 2 người Mỹ, {\sc Robert Glassey \& Eric Carlen}, cả hai đều là chuyên gia về lý thuyết động học của chất khí. Tối hôm trước, khi khai mạc buổi hội thảo tuần đó, tôi đã tự hào trình bày một loạt kết quả mới, \& ngay buổi sáng hôm đó Eric đã có một màn trình diễn thực sự đáng nhớ, tràn đầy năng lượng \& ngập tràn ý tưởng. Hai sự kiện diễn ra nối tiếp nhau khiến Robert hơi choáng ngợp, anh thú nhận rằng mình cảm thấy già nua \& kiệt sức. ``Đã đến lúc nghỉ hưu,'' anh thở dài, ``Nghỉ hưu à?'' Eric kêu lên đầy hoài nghi. Chưa bao giờ có thời gian thú vị hơn trong lý thuyết về chất khí! ``Nghỉ hưu à?'' Tôi kêu lên. Đúng lúc chúng ta đang rất cần sự khôn ngoan mà người đàn ông này đã tích lũy được trong 35 năm làm nhà toán học chuyên nghiệp!
	
	``Robert, what can you tell me about the mysterious Landau damping effect? Do you think it's real?''
	
	The words ``weird'' \& ``strange'' stood out in Robert's reply. Yes, Maslov worked on it; yes, there is a paradox of reversibility that seems incompatible with Landau damping; no, it isn't at all clear what's going on. Eric suggested that the effect was chimerical -- a product of physicists' fertile imaginations that had no hope of being rigorously formulated in mathematical terms. None of this meant much to me at the time, but I did manage to make a mental note \& file it away in a corner of my brain.
	
	-- Những từ ``kỳ lạ'' \& ``lạ'' nổi bật trong câu trả lời của Robert. Đúng, Maslov đã làm việc đó; vâng, có một nghịch lý về khả năng đảo ngược dường như không tương thích với hệ thống giảm chấn Landau; không, không rõ chuyện gì đang xảy ra. Eric cho rằng hiệu ứng này là ảo tưởng - 1 sản phẩm của trí tưởng tượng phong phú của các nhà vật lý không có hy vọng được hình thành một cách chặt chẽ dưới dạng toán học. Những điều này không có ý nghĩa gì nhiều với tôi vào thời điểm đó, nhưng tôi đã cố gắng ghi nhớ \& cất nó vào một góc trong não.
	
	Now here we are in 2008, \& I don't know anything more about Landau damping than I did 2 years ago. Cl\'ement, on the other hand, had a chance to discuss the matter at length with Yan Guo, 1 of Robert's younger brothers in mathematics (they both had the same thesis director, 20 years apart). The heart of the difficulty, according to Yan, is that Landau didn't work on Vlasov's original model but on a simplified, {\it linearized} version. No one knows if what he found also applies to the ``true'' nonlinear model. Yan is fascinated by this problem -- \& he's not alone.
	
	Could Cl\'ement \& I tackle it? Sure, we could try. But in order to solve a problem, you've got to know at the outset exactly what the problem is! In mathematical research, clearly identifying what is you are trying to do is a crucial, \& often very tricky, 1st step.
	
	-- Cl\'ement \& tôi có thể giải quyết nó được không? Chắc chắn, chúng ta có thể thử. Nhưng để giải quyết một vấn đề, bạn phải biết chính xác vấn đề là gì ở đầu ra! Trong nghiên cứu toán học, việc xác định rõ ràng những gì bạn đang cố gắng làm là bước đầu tiên quan trọng, \& thường rất khó khăn.
	
	\& no matter what our objective might turn out to be, the only thing we'd be sure of to begin with is the Vlasov equation,
	\begin{equation}
		\label{Vlasov}
		\partial_tf + {\bf v}\cdot\nabla_{\bf x}f - \left(\nabla W*\int f\,{\rm d}v\right)\cdot\nabla_{\bf v}f = 0,
	\end{equation}
	which determines the statistical properties of plasmas with exquisite precision. Mathematicians, like the poor Lady of Shalott in Tennyson's Arthurian ballad, cannot look at the world directly, only at its reflection -- a mathematical reflection. It is therefore in the world of mathematical ideas, governed by logic alone, that we will have to track down Landau.
	
	Neither C\'ement nor I have ever worked on this equation. But equations belong to everybody. We're going to roll up our sleeves \& give it our best shot.
	
	{\sc Lev Davidovich Landau}, a Russian Jew born in 1908, winner of the Nobel Prize in 1962, was 1 of the greatest theoretical physicists of the 20th century. Persecuted by the Soviet regime \& finally freed from prison through the devoted efforts of his colleagues, he survived to become a towering, almost tyrannical figure in the world of science. With {\sc Evgeny Lifshitz} he wrote the magisterial 10-volume {\it Course of Theoretical Physics}, still a standard reference today, \& made 2 fundamental contributions to the study of plasma physics in particular: the Landau equation, a sort of little sister to the Boltzmann equation (I studied both in preparing my thesis), \& Landau damping, a spontaneous phenomenon of stabilization in plasmas -- i.e., a return to equilibrium without any increase in entropy, in contrast to the mechanisms described by the {\sc Boltzmann}.
	
	-- {\sc Lev Davidovich Landau}, một người Do Thái gốc Nga sinh năm 1908, đoạt giải Nobel năm 1962, là một trong những nhà vật lý lý thuyết vĩ đại nhất thế kỷ 20. Bị chế độ Xô Viết đàn áp \& cuối cùng được giải thoát khỏi nhà tù nhờ nỗ lực tận tụy của các đồng nghiệp, ông đã sống sót để trở thành một nhân vật cao chót vót, gần như chuyên chế trong thế giới khoa học. Cùng với {\sc Evgeny Lifshitz} ông đã viết {\it Khóa học Vật lý lý thuyết} dày 10 tập, vẫn là một tài liệu tham khảo tiêu chuẩn cho đến ngày nay, \& đã có 2 đóng góp cơ bản cho việc nghiên cứu vật lý plasma nói riêng: phương trình Landau, một loại của em gái với phương trình Boltzmann (tôi đã nghiên cứu cả hai khi chuẩn bị luận án của mình), \& Giảm chấn Landau, một hiện tượng ổn định tự phát trong plasma -- tức là sự trở lại trạng thái cân bằng mà không tăng entropy, trái ngược với các cơ chế được mô tả bởi {\sc Boltzmann}.
	
	With the physics of gases we are in the realm of {\sc Boltzmann}: entropy increases, information is lost, the arrow of time points toward the future, the initial state is forgotten; gradually the statistical distribution of neural particles approaches a state of maximum entropy, the most disordered state possible.
	
	-- Với vật lý học về chất khí, chúng ta đang ở trong phạm vi của {\sc Boltzmann}: entropy tăng, thông tin bị mất, mũi tên thời gian hướng về tương lai, trạng thái ban đầu bị lãng quên; Dần dần sự phân bố thống kê của các hạt thần kinh tiến tới trạng thái entropy cực đại, trạng thái mất trật tự nhất có thể.
	
	With plasma physics, on the other hand, we are in the realm of Vlasov: entropy is constant, information is conserved, there is no arrow of time, the initial state is always remembered; disorder does not increase, \& there is no reason for the system to approach 1 state rather than another.
	
	-- Mặt khác, với vật lý plasma, chúng ta đang ở thế giới của Vlasov: entropy không đổi, thông tin được bảo toàn, không có mũi tên thời gian, trạng thái ban đầu luôn được ghi nhớ; sự rối loạn không tăng lên, \& không có lý do gì để hệ thống tiến tới trạng thái này mà không phải trạng thái khác.
	
	{\sc Landau} had a low opinion of {\sc Vlasov}, even going so far as to say that almost all of {\sc Vlasov}'s results were wrong. \& yet he adopted {\sc Vlasov}'s model. {\sc Landau} drew from it a conclusion that {\sc Vlasov} had completely overlooked, namely, that the electrical forces weakened spontaneously over time without any corresponding increase in entropy or any friction whatsoever. Heresy?
	
	-- {\sc Landau} không đánh giá cao {\sc Vlasov}, thậm chí còn đi xa hơn khi nói rằng hầu hết các kết quả của {\sc Vlasov} đều sai. \& tuy nhiên anh ấy đã áp dụng mô hình của {\sc Vlasov}. {\sc Landau} từ đó rút ra kết luận rằng {\sc Vlasov} đã hoàn toàn bỏ qua, tức là, các lực điện suy yếu một cách tự phát theo thời gian mà không có bất kỳ sự gia tăng tương ứng nào về entropy hay bất kỳ ma sát nào. Dị giáo?
	
	{\sc Landau}'s ingeniously complex mathematical calculation satisfied most physicists, \& the so-called {\it damping phenomenon} soon came to be named after him. But not everyone was convinced.
	
	-- Phép tính toán học phức tạp khéo léo của {\sc Landau} đã làm hài lòng hầu hết các nhà vật lý, \& cái gọi là {\it hiện tượng giảm chấn} nhanh chóng được đặt theo tên ông. Nhưng không phải ai cũng bị thuyết phục.
	
	\fbox{3} Lyon, Apr 2, 2008. In the hallway, a low table strewn with pages of hastily scribbled notes \& a blackboard covered with little drawings. Through the great picture window, a view of a gigantic long-legged black cubist spider, the famous P4 laboratory where experiments are conducted on the most dangerous viruses in the world.
	
	-- Ở hành lang, một chiếc bàn thấp ngổn ngang những trang ghi chú viết nguệch ngoạc \& một tấm bảng đen phủ đầy những hình vẽ nhỏ. Qua cửa sổ hình ảnh tuyệt vời, hình ảnh một con nhện lập thể màu đen chân dài khổng lồ, phòng thí nghiệm P4 nổi tiếng, nơi tiến hành các thí nghiệm về loại virus nguy hiểm nhất thế giới.
	
	My guest, {\sc Freddy Bouchet}, gathered up his notes \& put them in his bag. We'd spent a good hour talking about his research on the numerical simulation of galaxy formation \& the mysterious power of stars to spontaneously organize themselves in stable clusters.
	
	-- Vị khách của tôi, {\sc Freddy Bouchet}, đã thu thập các ghi chú của mình \& bỏ chúng vào túi của anh ấy. Chúng tôi đã dành một giờ đồng hồ để nói về nghiên cứu của anh ấy về mô phỏng số học về sự hình thành thiên hà \& sức mạnh bí ẩn của các ngôi sao trong việc tự tổ chức thành các cụm ổn định một cách tự nhiên.
	
	This phenomenon is not contemplated by {\sc Isaac Newton}'s law of universal gravitation, discovered $> 300$ years ago. \& yet when one observes a cluster of stars governed by {\sc Newton}'s law, it does indeed seem that the entire cloud settles into a stable state after a rather long time -- an impression that has been confirmed by a great many calculations performed on very powerful computers.
	
	Is it possible, then, to {\it deduce} this property from the law of universal gravitation? The English astrophysicist {\sc Donald Lynden-Bell} had no doubt whatsoever about the reality of dynamic stabilization in star clusters. He thought it was a ``hard'' phenomenon -- as hard as, well, an iron meteorite -- \& gave it the name {\it violent relaxation}. A splendid oxymoron!
	
	``Violent relaxation, C\'edric, is like Landau damping. Except that Landau damping is a perturbative regime \& violent relaxation is a highly nonlinear regime.''
	
	{\sc Freddy} was trained in both mathematics \& physics, \& he has devoted a good part of his professional life to studying such problems. Today he had come to talk to me about 1 question in particular.
	
	``When you model galaxies, you treat the stars as a fluid -- as a gas of stars, in effect. You go from the discrete to the continuous. But how great an error does this approximation entail? Does it depend on the number of stars? In a gas there are a billion billion particles, but in a galaxy there are only a hundred billion stars. How much of a difference does that make?''
	
	{\sc Freddy} went on in this vein for a long while, raising further questions, explaining recent results, drawing figures \& diagrams on the board, noting references. I pointed out the connection between his research \& 1 of my hobbyhorses, the theory of optimal transport inaugurated by {\sc Monge}. {\sc Freddy} seemed pleased; it was a profitable conversation for him. For my part, I was thrilled to see Landau damping suddenly make another appearance, scarcely $> 1$ week after my discussion with Cl\'ement.
	
	Just as I was coming back to my office after saying goodbye to {\sc Freddy}, my neighbor {\sc\'Etienne}, who until then had been bustling about, noiselessly filing papers, popped his head into the hallway. With his long gray hair cut in a bob, he looks like an elderly teenager, anticonformist but hardly threatening.
	
	-- Ngay khi tôi đang quay trở lại văn phòng của mình sau khi tạm biệt {\sc Freddy}, người hàng xóm của tôi {\sc\'Etienne}, người cho đến lúc đó vẫn đang hối hả, im lặng sắp xếp giấy tờ, thò đầu ra hành lang. Với mái tóc dài màu xám cắt theo kiểu bob, anh ấy trông giống như một thiếu niên lớn tuổi, theo chủ nghĩa phản tuân thủ nhưng hầu như không có tính đe dọa.
	
	``I didn't really want to say anything, C\'edric, but those figures there on the board -- I've seen them before.''
	
	A plenary speaker at the last International Congress of Mathematicians, member of the French Academy of Sciences, often (\& probably rightly) described as the world's best lecturer on mathematics. \'Etienne Ghys is an institution unto himself. As a staunch advocate of promoting research outside the Paris region, he has spent the past 20 years developing the mathematics laboratory at ENS-Lyon. More than anyone else, he is responsible for turning it into 1 of the leading centers for geometry in the world. {\sc\'Etienne}'s charisma is matched only by his grumpiness: he has something to say about everything -- \& nothing will stop him from saying it.
	
	-- Một diễn giả toàn thể tại Đại hội Toán học Quốc tế vừa qua, thành viên của Viện Hàn lâm Khoa học Pháp, thường (\& có lẽ đúng) được mô tả là giảng viên giỏi nhất thế giới về toán học. \'Etienne Ghys là một tổ chức cho riêng mình. Là người ủng hộ nhiệt tình việc thúc đẩy nghiên cứu bên ngoài khu vực Paris, ông đã dành 20 năm qua để phát triển phòng thí nghiệm toán học tại ENS-Lyon. Hơn ai hết, ông có nhiệm vụ biến nơi đây thành 1 trong những trung tâm hình học hàng đầu thế giới. Sức thu hút của {\sc\'Etienne} chỉ phù hợp với tính cách gắt gỏng của anh ấy: anh ấy có điều gì đó để nói về mọi thứ -- \& không có gì có thể ngăn cản anh ấy nói điều đó.
	
	``You've seen these figures?''
	
	``Yes, that one's from KAM theory. \& this one, I know it from somewhere $\ldots$''
	
	``Where should I look?''
	
	``Well, KAM is found almost everywhere. You start from a completely integrable, quasi-periodic dynamical system \& you introduce a small perturbation. There's a problem with small divisors that eliminate certain trajectories, but even so, probabilistically speaking, you've got long-term stability.''
	
	-- Chà, KAM được tìm thấy ở hầu hết mọi nơi. Bạn bắt đầu từ một hệ thống động lực gần như tuần hoàn, hoàn toàn có thể tích hợp và bạn đưa ra một nhiễu loạn nhỏ. Có một vấn đề với các ước số nhỏ loại bỏ những quỹ đạo nhất định, nhưng ngay cả như vậy, về mặt xác suất mà nói, bạn vẫn có được sự ổn định lâu dài.
	
	``Yes, I know. But what about the figures?''
	
	``Hold on, I'm going to find a good book on the subject for you. But a lot of the figures you see in works on cosmology are usually found in dynamical systems theory.''
	
	-- Đợi đã, tôi sẽ tìm cho bạn một cuốn sách hay về chủ đề này. Nhưng rất nhiều số liệu bạn thấy trong các công trình về vũ trụ học thường được tìm thấy trong lý thuyết hệ thống động lực.
	
	Very interesting, I'll have to talk a look. Maybe it will help me figure out what stabilization is really all about.
	
	That's what I love most of all about our small but very productive laboratory -- the way conversation moves from 1 topic to another, especially when you're talking with someone whose mathematical interests are different from yours. With no disciplinary barriers to get in the way, there are so many new paths to explore!
	
	-- Đó là điều tôi yêu thích nhất ở phòng thí nghiệm nhỏ nhưng rất hiệu quả của chúng tôi -- cách cuộc trò chuyện chuyển từ chủ đề này sang chủ đề khác, đặc biệt là khi bạn nói chuyện với một người có sở thích toán học khác với bạn. Không có rào cản kỷ luật cản trở, có rất nhiều con đường mới để khám phá!
	
	I didn't have the patience to wait for \'Etienne to rummage through his vast collection of books, so I rooted around in my own library \& came up with a monograph by {\sc Alinhac \& G\'erard} on Nash--Moser theorem. As it happens, I'd made a careful study of this work a few years ago, so I was well aware that the method developed by {\sc John Nash \& J\"urgen Moser} is 1 of the pillars of the Kolmogorov--Arnold--Moser (KAM) theory that \'Etienne had mentioned. I also knew that Nash--Moser relies on Newton's extraordinary iteration schemes for finding successively better approximations to the roots of real-valued equations -- a method that converges unimaginably fast, exponentially exponentially fast! -- \& that {\sc Kolmogorov} was able to exploit it with remarkable ingenuity. Frankly, I couldn't see any connection whatever between these things \& Landau damping. But who knows, I muttered to myself, perhaps {\sc\'Etienne}'s intuition will turn out to be correct $\ldots$
	
	Enough daydreaming! I wedged the book into my backpack \& rushed off to pick up my kids from school, got on the m\'etro \& immediately took out a manga from my coat pocket. For a few brief \& precious moments life around me disappeared, giving way to a world of supernaturally skilled physicians with surgically reconstructed faces, hardened yakuza who lay down their lives for their children, little girls with huge doe yes, cruel monsters who suddenly turn into tragic heroes, little boys with blond curls who gradually turn into cruel monsters $\ldots$ A skeptical \& tender world, passionate, disillusioned, devoid of either prejudice or Manichaean certainties; a world of emotions that strike deep down in the soul \& bring tears to the eyes of anyone innocent enough to surrender himself to them--
	
	-- Mơ mộng đủ rồi! Tôi nhét cuốn sách vào ba lô \& lao đi đón con tan trường, lên m\'etro \& lập tức lấy trong túi áo khoác ra một cuốn truyện tranh. Trong một vài khoảnh khắc ngắn ngủi \& quý giá, cuộc sống xung quanh tôi biến mất, nhường chỗ cho một thế giới của những bác sĩ có tay nghề siêu phàm với khuôn mặt được phẫu thuật tái tạo, những yakuza cứng cỏi hy sinh mạng sống vì con mình, những cô bé với con nai to lớn, vâng, những con quái vật độc ác đột nhiên trở mặt thành những anh hùng bi thảm, những cậu bé tóc vàng dần dần biến thành những con quái vật độc ác $\ldots$ 1 thế giới hoài nghi \& dịu dàng, đam mê, vỡ mộng, không có thành kiến hay những điều chắc chắn của người Manichaean; 1 thế giới của những cảm xúc chạm sâu vào tâm hồn \& khiến những người vô tội phải rơi nước mắt--
	
	H\^otel de Ville! My stop! During the time it took to get here the story had flowed through my brain \& through my veins, a small torrent of ink \& paper. I felt cleansed through \& through.
	
	While I'm reading manga all thoughts of mathematics are suspended. It's like hitting a pause button: manga \& mathematics don't mix. But what about later, when I'm dreaming at night? What if {\sc Landau}, after the terrible accident that should have cost him his life, had been operated on by {\sc Black Jack}? Surely the fiendishly gifted surgeon would have fully restored his powers, \& {\sc Landau} would have resumed his superhuman labors $\ldots$
	
	For at least a brief time anyway, I was able to forget {\sc\'Etienne}'s remark \& this business about KAM theory. What connection could there possibly be between {\sc Kolmogorov \& Landau}? The moment I got off the m\'etro, the question echoed through my mind over \& over again. If there really is a connection, I'll find it.
	
	At the time I had no way of knowing that it would take me $> 1$ year to find the link between the 2. Nor could I have suspected the fantastic irony that would finally emerge: the figure that caught {\sc\'Etienne}'s attention, that put him in mind of {\sc Kolmogorov}, actually illustrates a situation where {\sc Landau} damping \& KAM theory have nothing to do with each other! {\sc\'Etienne}'s intuition was right, but for the wrong reason -- as though {\sc Darwin} had guessed correctly about the evolution of species by comparing bats \& pterodactyls, mistakenly supposing that the 2 were closely related.
	
	-- Vào thời điểm đó, tôi không thể nào biết được rằng tôi sẽ phải mất $>1$ năm để tìm ra mối liên hệ giữa 2 điều đó. Tôi cũng không thể ngờ rằng điều trớ trêu tuyệt vời cuối cùng sẽ xuất hiện: hình ảnh{\tt/}biểu đồ{\tt/}lược đồ đã bắt được {\sc\'Etienne } khiến anh nhớ đến {\sc Kolmogorov}, thực tế minh họa một tình huống trong đó lý thuyết giảm chấn {\sc Landau} \& KAM không liên quan gì đến nhau! Trực giác của {\sc\'Etienne} là đúng, nhưng vì lý do sai -- như thể {\sc Darwin} đã đoán đúng về sự tiến hóa của các loài bằng cách so sánh loài dơi \& pterodactyls, vì nhầm tưởng rằng cả hai loài này có quan hệ họ hàng gần gũi.
	
	10 days after the unexpected turn taken by my working session with Cl\'ement, a 2nd miraculous coincidence had occurred -- \& on the same subject! The timing could not have been more fortuitous.
	
	-- 10 ngày sau bước ngoặt bất ngờ trong buổi làm việc của tôi với Cl\'ement, một sự trùng hợp kỳ diệu thứ 2 đã xảy ra -- \& về cùng một chủ đề! Thời điểm không thể ngẫu nhiên hơn.
	
	Now to take advantage of it.
	
	What was the name of that Russian physicist? Just like what happened to me, everyone thought he was dead when they pulled him out from the wreckage. Medically, he {\it was} dead. An extraordinary story. The Soviet authorities mobilized every resource in order to save an irreplaceable scientist. An appeal for help was even issued to physicians in other countries. The dead man was revived. For weeks the greatest surgeons in this world took turns at his bedside. 4 times the man died. 4 times life was artificially breathed into him. I've forgotten the details, but I do remember how fascinating it was to read about this struggle against an inadmissible fatality. His tomb was opened up \& he was forcibly removed. He resumed his post at the university in Moscow. [{\sc Paul Guimard}, {\it Les choses de la vie}]
	
	-- Tên của nhà vật lý người Nga đó là gì? Giống như những gì đã xảy ra với tôi, mọi người đều nghĩ rằng anh ấy đã chết khi kéo anh ấy ra khỏi đống đổ nát. Về mặt y tế, anh ấy {\it đã} chết. Một câu chuyện phi thường. Chính quyền Liên Xô đã huy động mọi nguồn lực để cứu một nhà khoa học không thể thay thế. Lời kêu gọi giúp đỡ thậm chí còn được gửi đến các bác sĩ ở các nước khác. Người chết đã sống lại. Trong nhiều tuần, các bác sĩ phẫu thuật vĩ đại nhất trên thế giới này đã thay phiên nhau túc trực bên giường bệnh của ông. 4 lần người đàn ông chết. 4 lần sự sống được thổi vào anh một cách nhân tạo. Tôi đã quên chi tiết, nhưng tôi nhớ thật thú vị biết bao khi đọc về cuộc đấu tranh chống lại cái chết không thể chấp nhận được này. Ngôi mộ của anh ta đã được mở ra \& anh ta bị buộc phải di dời. Ông tiếp tục công việc của mình tại trường đại học ở Moscow. [{\sc Paul Guimard}, {\it Les chooses de la vie}]
	
	{\sc Newton}'s law of universal gravitation states that any 2 bodies are attracted to each other by a force proportional to the product of their masses \& inversely proportional to the square of the distance between them: $F = \frac{GM_1M_2}{r^2}$. In its classical form, this law does a good job of accounting for the motion of stars in galaxies. But even if {\sc Newton}'s law is simple, the immense number of stars in a galaxy makes it difficult to apply. After all, just because we understand the behavior of individual atoms doesn't mean that we understand the behavior of a human being $\ldots$
	
	A few years after formulating the law of gravitation, {\sc Newton} made another extraordinary discovery: an iterative method for calculating the solutions of any equation of the form $F(x) = 0$. Starting from an approximate solution $x_0$, you replace the function $F$ by its tangent $T_{x_0}$ at the point $(x_0,F(x_0))$ (more precisely, the equation is linearized around $x_0$) \& solve the approximate equation $T_{x_0}(x) = 0$. This gives a new approximate equation $x_1$, \& you now repeat the same procedure: replace $F$ by its tangent $T_{x_1}$ at $x_1$, obtain $x_2$ as the solution of $T_{x_1}(x_2) = 0$, \& so on. In exact mathematical notation, the relation that associates $x_n$ with $x_{n+1}$ is $x_{n+1} = x_n - [DF(x_n)]^{-1}F(x_n)$.
	
	The successive approximations $x_1,x_2,x_3,\ldots$ obtained in this fashion are incredibly good: they approach the ``true'' solution with phenomenal swiftness. It is often the case that 4 or 5 tries are all that is needed to achieve a precision great than that of any modern pocket calculator. The Babylonians are said to have used this method 4000 years ago to extract square roots; {\sc Newton} discovered that it can be used to find not only square roots but the roots of any real-valued equation.
	
	Much later, the same preternatually rapid convergence made it possible to demonstrate some of the most striking theoretical results of the 20th century, among them {\sc Kolmogorov}'s stability theorem \& {\sc Nash}'s isometric embedding theorem. Single-handedly, {\sc Newton}'s diabolical scheme transcends the artificial distinction between pure \& applied mathematics.
	
	The Russian mathematician {\sc Andrei Kolmogorov} is a legendary figure in the history of 20th-century science. In the 1930s, {\sc Kolmogorov} founded modern probability theory. His theory of turbulence in fluid dynamics, worked out in 1941, remains the starting point for research on this subject today, both for those who seek to corroborate it \& for those who seek to disconfirm it. His theory of complexity prefigured the development of artificial intelligence.
	
	{\sc Henri Poincar\'e} had convinced his fellow mathematicians that the solar system is intrinsically unstable, \& that uncertainty about the position of the planets, however small, makes any prediction of the position of the planets in the distant future impossible. But some 70 years later, 1954, at the International Congress of Mathematicians in Amsterdam, {\sc Kolmogorov} presented an astonishing result. Harnessing probabilities \& the deterministic equations of mechanics with breath-taking audacity, he argued that the solar system {\it probably} is stable. Instability is possible, as {\sc Poincar\'e} correctly saw -- but if it occurs, it should occur only rarely.
	
	{\sc Kolmogorov}'s theorem asserts that if one assumes an exactly soluble mechanical system (in this case, the solar system as {\sc Kepler} imagined it to be, with the planets endlessly revolving around the sun in regular \& unchanging elliptical orbits), \& if one then disturbs it ever so slightly (taking into account the gravitational force of attraction, neglected by {\sc Kepler}), the resulting system remains stable for the great majority of initial conditions.
	
	{\sc Kolmogorov}'s argument was not widely accepted at 1st. This was mainly because of its complexity, but {\sc Kolmogorov}'s own elliptical style of exposition didn't help matters. $< 1$ decade later, however, the Russian mathematician {\sc Vladimir Arnold} \& German mathematician {\sc J\"urgen Moser}, using different approaches, succeeded in providing a complete demonstration, {\sc Arnold} proving {\sc Kolmogorov}'s original statement of the theorem \& {\sc Moser} a more general variant of it. Thus was born KAM theory, which in its turn has given birth to some of the most powerful \& surprising results in classical mechanics.
	
	The singular beauty of this theory silenced skeptics, \& for the next 25 years the solar system was believed to be stable, even if the technical constraints of {\sc Kolmogorov}'s theorem did not correspond exactly to reality. With the work of the French astrophysicist {\sc Jacques Laskar} in the late 1980s, however, opinion reversed itself once more. But that's another story $\ldots$
	
\end{enumerate}

\paragraph{Mathematical Analysis}

\begin{enumerate}
	\item \cite{Brezis2011}. Ha\"\i m Brezis. {\it Functional Analysis, Sobolev Spaces, \& PDEs}.\hfill{\sf[reading]}
	
	\item \cite{Evans2010}. Lawrence C. Evans. {\it Partial Differential Equations}.\hfill{\sf[reading]}
	
	\item \cite{Hardy_Littlewood_Polya1952}. {\sc G. H. Hardy, J. E. Littlewood, G. P\'{o}lya}. {\it Inequalities}. {\sf[21 Amazon ratings]}
	
	{\sf Amazon review.} This classic of the mathematical literature forms a comprehensive study of the inequalities used throughout mathematics. 1st published in 1934, it presents clearly \& lucidly both the statement \& proof of all the standard inequalities of analysis. The authors were well-known for their powers of exposition \& made this subject accessible to a wild audience of mathematicians.
	
	{\sf Editorial reviews.}
	\begin{itemize}
		\item ``1 of the classics of 20th Century mathematical literature $\ldots$ it covers lucidly \& exhaustively both statement \& proof of all the standard inequalities of mathematical analysis.'' -- New Technical Books
	\end{itemize}
	\begin{quote}
		``Oh! the little more, \& how much it is! \& the little less, \& what worlds away!'' -- {\sc Robert Browning}
	\end{quote}
	{\bf Preface to 1st edition.} ``This book was planned \& begun in 1929. Our original intention was that it should be 1 of the {\it Cambridge Tracts}, but it soon became plain that a tract would be much too short for our purpose.
	
	Our objects in writing the book are explained sufficiently in the introductory chapter, but we add a note here about history \& bibliography. Historical \& bibliographical questions are particularly troublesome in a subject like this, which has applications in every part of mathematics but has never been developed systematically.
	
	It is often really difficult to trace the origin of a familiar inequality. It is quite likely to occur 1st as an auxiliary proposition, often without explicit statement, in a memoir on geometry or astronomy; it may have been rediscovered, many years later, by half a dozen different authors; \& no accessible statement of it may be quite complete. We have almost always found, even with the most famous inequalities, that we have a little new to add.
	
	Have done our best to be accurate \& have given all references we can, but we have never undertaken systematic bibliographical research. Follow common practice, when a particular inequality is habitually associated with a particular mathematician's name; speak of inequalities of {\sc Schwarz, H\"older, \& Jensen}, though all these inequalities can be traced further back; \& we do not enumerate explicitly all the minor additions which are necessary for absolute completeness. Chap. III has been very largely rewritten as result of Dr. {\sc Jensen}'s suggestions. Hope that the book may now be reasonably free from error, in spite of the mass of detail which it contains.
	\begin{itemize}
		\item {\sf Chap. I: Introduction.}
		\begin{itemize}
			\item {\sf Finite, infinite, \& integral inequalities.} Cauchy's inequality, variables of inequality. The number of variables is finite, \& Cauchy inequality states a relation between certain finite sums. Call such an inequality an {\it elementary{\tt/}finite} inequality. The most fundamental inequalities are finite, but shall also be concerned with inequalities which are not finite \& involve generalizations of notion of a sum. The most important of such generalizations are the infinite sums $\sum_{i=1}^\infty a_i,\sum_{i=-\infty}^\infty a_i$ \& the integral $\int_a^b f(x)\,{\rm d}x$ where $a,b$ may be finite or infinite. Infinite inequality \& integral inequality.
		\end{itemize}
		\item {\sf Chap. II: Elementary mean Value.}
		\item {\sf Chap. III: Mean Values With An Arbitrary Function \& The Theory of Convex Functions.}
		\item {\sf Chap. IV: Various Applications of The Calculus.}
		\item {\sf Chap. V: Infinite Series.}
		\item {\sf Chap. VI: Integrals.}
		\item {\sf Chap. VII: Some Applications of Calculus of Variations.}
		\item {\sf Chap. VIII: Some Theorems Concerning Bilinear \& Multilinear Forms.}
		\item {\sf Chap. IX: Hilbert's Inequality \& Its Analogues \& Extensions.}
		\item {\sf Chap. X: Rearrangements.}
	\end{itemize}
	
	\item \cite{Rudin1976}. Walter Rudin. {\it Principles of Mathematical Analysis}.\hfill{\sf[done]}
\end{enumerate}

\paragraph{Hyperbolic System.}

\begin{enumerate}
	\item \cite{Tartar2008}. {\sc Luc Tartar}. {\it From Hyperbolic Systems to Kinetic Theory: A Personalized Quest}.
	
	{\sf Amazon review.} Equations of state are not always effective in continuum mechanics. {\sc Maxwell \& Boltzmann} created a kinetic theory of gases, using classical mechanics. How could they derive the irreversible Boltzmann equation from a reversible Hamiltonian framework? By using probabilities, which destroy physical reality! Forces at distance are non-physical as we know from {\sc Poincar\'e}'s theory of relativity. Yet {\sc Maxwell \& Boltzmann} only used trajectories like hyperbolas, reasonable for rarefied gases, but wrong without bound trajectories if the ``mean free path between collisions'' tends to 0. {\sc Tartar} relies on his H-measures, a tool created for homogenization, to explain some of the weaknesses, e.g., from quantum mechanics: there are no ``particles'', so the Boltzmann equation \& the 2nd principle, cannot apply. He examines modes used by energy, proves which equation governs each mode, \& conjectures that the result will not look like the Boltzmann equation, \& there will be more modes than those indexed by velocity!
	
	{\sf Editorial reviews.}
	\begin{itemize}
		\item ``The book is well organized $\ldots$ {\sc Tartar} is excellent in bringing out the essence of ideas \& methods $\ldots$ The monograph is an interesting read from $> 1$ point of view. 1st, the mathematically literate reader finds an extensive $\ldots$ review of the relevant subjects; reading the sketches of proofs offers a bird's eye view of the underlying concepts as well as the exceptional mind of the author $\ldots$ All in all, a remarkable book.'' -- {\sc Reinhard Illner}, SIAM Reviews, Vol. 51 (2), June, 2009
		\item ``This is a very nice book devoted to hyperbolic systems of conservation laws $\ldots$ This book can be recommended to all researchers in respective areas, especially to graduate students.'' -- {\sc Mitsuru Yamazaki}, athematical Reviews, Issue 2010 j
	\end{itemize}
\end{enumerate}

\paragraph{Finite Volume Method FVM}

\begin{enumerate}
	\item \cite{Eymard_Gallouet_Herbin2019}. Robert Eymard, Thierry Gallou\"et, Rapha\`ele Herbin. {\it Finite Volume Methods}.\hfill{\sf[reading]}
	
	\item T. Gallou\"et, Rapha\`ele Herbin, J.-C. Latch\'e. {\it Convergence of the Marker-\&-Cell Scheme for the Incompressible Navier--Stokes Equations on Non-uniform Grids}.\hfill{\sf[reading]}
\end{enumerate}

\paragraph{Optimal Control}

\begin{enumerate}
	\item \cite{Hintermueller_Kroener2023}. Michael Hinterm\"uller, Axel Kr\"oner. {\it Differentiability properties for boundary control of fluid-structure interactions of linear elasticity with Navier-Stokes equations with mixed-boundary conditions in a channel}.\hfill{\sf[done]}
\end{enumerate}

\paragraph{Shape Optimization}

\begin{enumerate}
	\item \cite{Bandle_Wagner2023}. Catherine Bandle, Alfred Wagner. {\it Shape Optimization: Variations of Domains \& Applications}.\hfill{\sf[reading]}
	
	\item \cite{Haubner_Siebenborn_Ulbrich2021}. Johannes Haubner, Martin Siebenborn, Michael Ulbrich. {\it A continuous perspective on shape optimization via domain transformations}.\hfill{\sf[done]}
	
	\item \cite{Haubner_Ulbrich_Ulbrich2020}. Johannes Haubner, Michael Ulbrich, Stefan Ulbrich. {\it Analysis of shape optimization problems for unsteady fluid-structure interaction}.\hfill{\sf[done]}
	
	\item \cite{Hiptmair_Paganini_Sargheini2015}. R. Hiptmair, A. Paganini, S. Sargheini. {\it Comparison of approximate shape gradients}.\hfill{\sf[done]}
\end{enumerate}

\paragraph{Turbulence}

\begin{enumerate}
	\item \cite{Perron_Boivin_Herard2004}. {\it A FVM to solve the 3{D} NSEs on Unstructured Collocated Meshes}.\hfill{\sf[reading]}
\end{enumerate}

%------------------------------------------------------------------------------%

\subsubsection{Advanced Physics Book}

\begin{enumerate}
	\item Lương Duyên Bình, Nguyễn Hữu Hồ, Lê Văn Nghĩa, Nguyễn Quang Bình. {\it Bài Tập Vật Lý Đại Cương. Tập 2: Điện -- Dao Động -- Sóng}.
	
	\item \cite{Einstein_Infeld_tien_hoa_Vat_Ly}. Albert Einstein, Leopold Infeld. {\it The Evolution of Physics: From Early Concepts to Relativity \& Quanta -- Sự Tiến Hóa Của Vật Lý: Từ Những Khái Niệm Ban Đầu Đến Thuyết Tương Đối \& Lượng Tử}.\hfill{\sf[done]}
	
	\item \cite{Feyman_Leighton_Sands_6_easy_pieces}. {\sc Richard P. Feynman}. {\it Six Easy Pieces: Essentials of Physics Explained by Its Most Brilliant Teacher}. {\sf[3332 Amazon ratings][28807 Goodreads ratings]}
	
	{\sf Amazon review.} Learn how to think like a physicist from a Nobel laureate \& ``1 of the greatest minds of the 20th century'' ({\it New York Review of Books}) with these 6 classic \& beloved lessons.
	
	It was {\sc Richard P. Feynman}'s outrageous \& scintillating method of teaching that earned him legendary status among students \& professors of physics. From 1961 to 1963, {\sc Feynman} delivered a series of lectures at the California Institute of Technology that revolutionized the teaching of physics around the world. {\it6 Easy Pieces}, taken from these famous Lectures on Physics, represent the most accessible material from the series.
	
	In these classic lessons, {\sc Feynman} introduces the general reader to the following topics: atoms, basic physics, energy, gravitation, quantum mechanics, \& the relationship of physics to other topics. With his dazzling \& inimitable wit, {\sc Feynman} presents each discussion with a minimum of jargon. Filled with wonderful examples \& clever illustrations, {\it6 Easy Pieces} is the ideal introduction to the fundamentals of physics by 1 of the most admired \& accessible physicists of modern times.
	
	{\sf Editorial reviews.}
	\begin{itemize}
		\item ``1 of the greatest minds of the 20th century.'' -- {\it New York Review of Books}
		\item ``The essence of physics \& {\sc Feynman}. No jargon\footnote{words or expressions that are used by a particular profession or group of people, \& are difficult for others to understand. tiếng lóng.}, just ideas, excitement, \& the straight dope. \& real answers, like `we don't know.''' -- {\sc Stephen Wolfram}
		\item ``The most original mind of his generation.'' -- {\sc Freeman Dyson}
		\item ``If 1 book was all that could be passed onto the next generation of scientists it would undoubtedly have to be {\it6 Easy Pieces}.'' -- {\sc John Gribbin}, {\it New Scientist}
	\end{itemize}
	{\sf About the Author.} {\sc Richard P. Feynman} (1918--1988) was the Richard Chace Tolman Professor of Theoretical Physics at the California Institute of Technology. He was awarded the 1965 Nobel Prize for his work on the development of quantum field theory. He was also 1 of the most famous \& beloved figures of the 20th century, both in physics \& in the public arena.
	
	\item \cite{Feyman_Leighton_Sands_lecture_physics}. {\sc Richard P. Feynman, Robert B. Leighton, Matthew Sands}. {\it The Feynman Lectures on Physics}. {\sf[1078 Amazon ratings][7848 Goodreads ratings]}
	
	{\sf Amazon review.} The legendary introduction to physics from the subject's greatest teacher. ``The whole thing was basically an experiment,'' {\sc Richard Feynman} said late in his career, looking back on the origins of his lectures. The experiment turned out to be hugely successful, spawning a book that has remained a definitive introduction to physics for decades. Ranging from the most basic principles of Newtonian physics through such formidable theories as general relativity \& quantum mechanics, {\sc Feynman}'s lectures stand as a monument of clear exposition \& deep insight. Now, we are reintroducing the printed books to the trade, fully corrected, for the 1st time ever, \& in collaboration with Caltech. Timeless \& collectible, the lectures are essential reading, not just for students of physics but for anyone seeking an introduction to the field from the inimitable {\sc Feynman}.
	
	\item {\sc Richard P. Feynman}. {\it``Surely You're Joking, Mr. Feynman!'': Adventures of a Curious Character}.
	\begin{quotation}
		{\it``I learned there that innovation is a very difficult thing in the real world.''}
		
		{\it``They had wasted all their time memorizing stuff like that, when it could be looked up in 15 minutes.''}
		
		{\it``Of course, you only live 1 life, \& you make all your mistakes, \& learn what not to do, \& that's the end of you.''}
	\end{quotation}
	{\sf Review.}
	\begin{quotation}
		``A storyteller in the tradition of {\sc Mark Twain}. {\sf Feynman} proves once again that it is possible to laugh out \& scratch your head at the same time.'' -- K. C. Cole, {\it New York Times Book Review}
		
		{\it``Quintessential\footnote{representing the perfect example of something, tinh túy.} Feynman -- funny, brilliant, bawdy\footnote{(of jokes, songs, etc.) dealing with sex in a way that is slightly rude \& makes people laugh, dâm đãng.} $\ldots$ enormously entertaining.''} -- {\it Los Angeles Times Book Review}
	\end{quotation}	
	
	\item Vũ Văn Hùng. {\it Cơ Học Lượng Tử}.
	
	\item Vũ Văn Hùng. {\it Bài Tập Cơ Học Lượng Tử}.
	
	\item Nguyễn Quang Học, Đinh Quang Vinh. {\it Bài Tập Vật Lý Lý Thuyết 2. Tập 2: Vật Lý Thống Kê}.
	
	\item Nguyễn Quang Học, Vũ Văn Hùng. {\it Giáo Trình Vật Lý Thống Kê \& Nhiệt Động Lực Học. Tập 1: Nhiệt Động Lực Học}.
	
	\item \cite{Hawking_bbc}. Stephen Hawking. {\it Black Holes: The BBC Reith Lectures -- Lỗ Đen: Các Bài Giảng Trên Đài}.\hfill{\sf[done]}
	
	\item {\sc Stephen Hawking}. {\it A Brief History of Time}. {\sf[26210 Amazon ratings][447348 Goodreads ratings]}
	
	\item \cite{Hawking_lstg}. {\sc Stephen Hawking}. {\it A Brief History of Time -- Lược Sử Thời Gian}.\hfill{\sf[done]}
	
	\item \cite{Hawking_vttvhd}. {\sc Stephen Hawking}. {\it The Universe In A Nutshell -- Vũ Trụ Trong Vỏ Hạt Dẻ}.\hfill{\sf[done]}
	
	\item \cite{Leighton_Feyman_last_journey}. {\sc Ralph Leighton}. {\it Tuva or Bust! Richard Feynman's Last Journey}.
	
	\item \cite{Leighton_Feyman_last_journey_VN}. {\sc Ralph Leighton}. {\it Tuva or Bust! Richard Feynman's Last Journey -- Cuộc Phiêu Lưu Cuối Cùng Của Feynman}. Tủ Sách Khoa Học Khám Phá.\hfill{\sf[done]}
	
	{\it``Tis holier to journey than to arrive.''} -- {\sc Cervantes.}
	
	-- Vinh quang là hành trình chứ không phải đích đến. \cite[Suy ngẫm 2000, p. 269]{Leighton_Feyman_last_journey_VN}
	\item Đào Văn Phúc. {\it Lịch Sử Vật Lý Học}.
\end{enumerate}

%------------------------------------------------------------------------------%

\subsubsection{Advanced Chemistry Book}

\begin{enumerate}
	\item Hoàng Nhâm. {\it Hóa Học Vô Cơ Cơ Bản. Tập 1: Lý Thuyết Đại Cương về Hóa Học}.
	
	\item Hoàng Nhâm. {\it Hóa Học Vô Cơ Cơ Bản. Tập 2: Các Nguyên Tố Hóa Học Điển Hình}.
	
	\item Hoàng Nhâm. {\it Hóa Học Vô Cơ Cơ Bản. Tập 3: Các Nguyên Tố Chuyển Tiếp}.
	
	\item Hoàng Nhâm. {\it Bài Tập Hóa Học Vô Cơ}.
	
	\item Hoàng Nhâm, Hoàng Nhuận. {\it Bài Tập Hóa Học Vô Cơ. Quyển I $+$ II: Lý Thuyết Đại Cương về Hóa Học}.
	
	\item Hoàng Nhâm, Hoàng Nhuận. {\it Bài Tập Hóa Học Vô Cơ. Quyển III: Hóa Học Các Nguyên Tố}.
	
	\item Hoàng Nhâm. {\it Hóa Học Vô Cơ Nâng Cao. Tập 1: Lý Thuyết Đại Cương về Hóa Học}.
	
	\item Hoàng Nhâm. {\it Hóa Học Vô Cơ Nâng Cao. Tập 2: Các Nguyên Tố Hóa Học Tiêu Biểu}.
	
	\item Hoàng Nhâm. {\it Hóa Học Vô Cơ Nâng Cao. Tập 3: Các Nguyên Tố Chuyển Tiếp}.
	
	\item Đào Đình Thức. {\it Cấu Tạo Nguyên Tử \& Liên Kết Hóa Học. Tập 1}.
	
	\item Đào Đình Thức. {\it Cấu Tạo Nguyên Tử \& Liên Kết Hóa Học. Tập 2}.
	
	\item Đỗ Đình Răng, Đặng Đình Bạch, Lệ Thị Anh Đào, Nguyễn Mạnh Hà, Nguyễn Thị Thanh Phong. {\it Hóa Học Hữu Cơ 3}.
	
	\item Trần Thành Huế, Nguyễn Ngọc Hà. {\it Đối Xứng Phân Tử \& Lý Thuyết Nhóm Trong Hóa Học}.
\end{enumerate}

%------------------------------------------------------------------------------%

\subsubsection{Advanced Computer Science}

\begin{enumerate}
	\item \cite{Cormen_Leiserson_Rivest_Stein_algorithm}. {\sc Thomas H. Cormen, Charles E. Leiserson,  Ronald L. Rivest, Clifford Stein}. {\it Introduction to Algorithms}. {\sf[630 Amazon ratings][9071 Goodreads ratings]}
	
	{\sf Amazon review.} ``A comprehensive update of the leading algorithms text, with new material on matchings in bipartite graphs, online algorithms, ML, \& other topics. Some books on algorithms are rigorous but incomplete; others cover masses of material but lack rigor. {\it Introduction to Algorithms} uniquely combines rigor \& comprehensiveness. It covers a broad range of algorithms in depth, yet makes their design \& analysis accessible to all levels of readers, with self-contained chapters \& algorithms in pseudocode. Since the publication of 1e, {\it Introduction to Algorithms} has become the leading algorithms text in universities worldwide as well as the standard reference for professionals. This 4e has been updated throughout. New for 4e:
	\begin{itemize}
		\item New chapters on matching in bipartite graphs, online algorithms, \& ML
		\item New material on topics including solving recurrence equations, hash tables, potential functions, \& suffix arrays
		\item 140 new exercises \& 22 new problems
		\item Reader feedback-informed improvements to odd problems
		\item Clearer, more personal, \& gender-neutral writing style
		\item Color added to improve visual presentation
		\item Notes, bibliography, \& index updated to reflect developments in the field
		\item Website with new supplementary material
	\end{itemize}
	This book is a comprehensive update of the leading algorithms text, covering a broad range of algorithms in depth, yet making their design \& analysis accessible to all levels of readers, with self-contained chapters \& algorithms in pseudocode.''
	\begin{itemize}
		\item ``A data structure is a way to store \& organize data in order to facilitate access \& modifications.''
		\item ``Machine learning can be thought of as a method for performing algorithmic tasks without explicitly designing an algorithm, but instead inferring patterns from data \& thereby automatically learning a solution.''
		\item ``The running time of an algorithm on a particular input is the number of instructions \& data accesses executed.''
	\end{itemize}
	{\sf About the Author.} {\sc Thomas H. Cormen} is Emeritus Professor of Computer Science at Dartmouth College. {\sc Charles E. Leiserson} is Edwin Sibley Webster Professor in Electrical Engineering \& Computer Science at MIT. {\sc Ronald L. Rivest} is Institute Professor at MIT. {\sc Clifford Stein} is Wai T. Chang Professor of Industrial Engineering \& Operations Research, \& of Computer Science at Columbia University.
	
	\item \cite{Chacon_Straub2014}. Scott Chacon, Ben Straub. {\it Pro Git: Everything You Need to Know About Git}.\hfill{\sf[reading]}
	
	\item \cite{Durr_Vie2021}. Christoph D\"urr, Jill-J\^enn Vie. {\it Competitive Programming in Python: 128 Algorithms to Develop Your Coding Skills}.
	
	\item \cite{Ha_Python_co_ban}. Bùi Việt Hà. {\it Python Cơ Bản}.\hfill{\sf[done]}
	
	\item \cite{Ha_loi_giai_BT_Python_co_ban}. Bùi Việt Hà. {\it Lời Giải Bài Tập Python Cơ Bản}.\hfill{\sf[reading]}
	
	\item \cite{Ha_Python_nang_cao}. Bùi Việt Hà. {\it Python Nâng Cao}.\hfill{\sf[done]}
	
	\item \cite{Hien_DevUp}. Nguyễn Hiền. {\it DevUP}.\hfill{\sf[done]}
	
	\item \cite{Hoang_code_dao_ky_su}. Phạm Huy Hoàng. {\it Code Dạo Ký Sự: Lập Trình Viên Đâu Phải Chỉ Biết Code}.\hfill{\sf[done]}
	
	\item \cite{Hoang_toi_di_code_dao}. Phạm Huy Hoàng. {\it Hello Các bạn, Mình Là Tôi Đi Code Dạo: Chuyện Code, Chuyện Nghề, Chuyện Đời}.\hfill{\sf[done]}
	
	\item \cite{Knuth1997}. Donald E. Knuth. {\it The Art of Computer Programming. Volume 1: Fundamental Algorithms}.\hfill{\sf[reading]}
	
	\item \cite{Knuth1998}. Donald E. Knuth. {\it The Art of Computer Programming. Volume 3: Sorting \& Searching}.\hfill{\sf[reading]}
	
	\item \cite{Laaksonen2020}. Antti Laaksonen. {\it Guide to Competitive Programming: Learning \& Improving Algorithms Through Contests}.\hfill{\sf[reading]}
	
	\item \cite{LeCun_Bengio_Hinton2015}. Yann LeCun, Yoshua Bengio, Geoffrey Hinton. {\it Deep Learning}.\hfill{\sf[reading]}
	
	\item \cite{Matthes2019}. Eric Matthes. {\it Python Crash Course: A Hands-on, Project-Based Introduction to Programming}. 2e.\hfill{\sf[reading]}
	
	\item \cite{Matthes2023}. Eric Matthes. {\it Python Crash Course: A Hands-on, Project-Based Introduction to Programming}. 3e.\hfill{\sf[reading]}
	
	\item \cite{Ngoc_Pascal}. Quách Tuấn Ngọc. {\it Ngôn Ngữ Lập Trình Pascal}.\hfill{\sf[reading]}
	
	\item \cite{Ngoc_BT_Pascal}. Quách Tuấn Ngọc. {\it Bài Tập Ngôn Ngữ Lập Trình Pascal}.\hfill{\sf[reading]}
	
	\item \cite{Ngoc_C}. Quách Tuấn Ngọc. {\it Ngôn Ngữ Lập Trình C}.\hfill{\sf[reading]}
	
	\item \cite{Ngoc_C++}. Quách Tuấn Ngọc. {\it Ngôn Ngữ Lập Trình C++}.\hfill{\sf[done]}
	
	\item \cite{Press_Teukolsky_Vetterling_Flannery_recipe_C++}. {\sc William H. Press, Saul A. Teukolsky, William T. Vetterling, Brian P. Flannery}. {\it Numerical Recipes: The Art of Scientific Computing}. {\sf[158 Amazon ratings][155 Goodreads ratings]}
	
	{\sf Amazon review.} Co-authored by 4 leading scientists from academia \& industry, Numerical Recipes 3e starts with basic mathematics \& computer science \& proceeds to complete, working routines. Widely recognized as the most comprehensive, accessible \& practical basis for scientific computing, this new edition incorporates $> 400$ Numerical Recipes routines, many of them new or upgraded. The executable C++ code, now printed in color for easy reading, adopts an object-oriented style particularly suited to scientific applications. The whole book is presented in the informal, easy-to-read style that made earlier editions so popular. New key features:
	\begin{itemize}
		\item 2 new chapters, 25 new sections, 25\% longer than 2e.
		\item Thorough upgrades throughout the text
		\item $> 100$ completely new routines \& upgrades of many more.
		\item New Classification \& Inference chapter, including Gaussian mixture models, HMMs, hierarchical clustering, Support Vector Machines
		\item New Computational Geometry chapter covers KD trees, quad- \& octrees, Delaunay triangulation, \& algorithms for lines, polygons, triangles, \& spheres
		\item New sections include interior point methods for linear programming, Monte Carlo Markov Chains, spectral \& pseudospectral methods for PDEs, \& many new statistical distributions
		\item An expanded treatment of ODEs with completely new routines
	\end{itemize}
	Plus comprehensive coverage of
	\begin{itemize}
		\item linear algebra, interpolation, special functions, random numbers, nonlinear sets of equations, optimization, eigensystems, Fourier methods \& wavelets, statistical tests, ODEs \& PDEs, integral equations, \& inverse theory
	\end{itemize}
	The essential text \& reference for modern scientific computing now also covers computational geometry, classification \& inference, \& much more.
	
	{\sf Editorial reviews.}
	\begin{itemize}
		\item ``This monumental \& classic work is beautifully produced \& of literary as well as mathematical quality. It is an essential component of any serious scientific or engineering library.'' -- Computing Reviews
		\item ``$\ldots$ an instant `classic,' a book that should be purchased \& read by anyone who uses numerical methods $\ldots$'' -- American Journal of Physics
		\item ``$\ldots$ replete with the standard spectrum of mathematically pretreated \& coded{\tt/}numerical routines for linear equations, matrices \& arrays, curves, splines, polynomials, functions, roots, series, integrals, eigenvectors, FFT, \& other transforms, distributions, statistics, \& on to ODE's \& PDE's $\ldots$ delightful.'' -- Physics in Canada
		\item ``$\ldots$ if you were to have only a single book on numerical methods, this is the one I would recommend.'' -- EEE Computational Science \& Engineering
		\item ``This encyclopedic book should be read (or at least owned) not only by those who must roll their own numerical methods, but by all who mus use prepackaged programs.'' -- New Scientist
		\item ``These books are a must for anyone doing scientific computing.'' -- Journal of the American Chemical Society
		\item ``The authors are to be congratulated for providing the scientific community with a valuable resource.'' -- The Scientist
		\item ``I think this is an incredibly valuable book for both learning \& reference \& I recommend it for any scientists or student in a numerate discipline who need to understand \&{\tt/}or program numerical algorithms.'' -- International Association for Pattern Recognition
		\item ``The attractive style of the text \& the availability of the codes ensured the popularity of the previous editions \& also recommended this recent volume to different categories of readers, more or less experienced in numerical computation.'' -- {\sc Octavian Pastravanu}, Zentralblatt MATH
	\end{itemize}
	{\sf About the Author.} {\sc William H. Press} holds the Raymer Chair in Computer Sciences \& Integrative Biology at the University of Texas at Austin.
	
	{\sc Saul A. Teukolsky} is H. A. Bethe Professor in Physics in the Radiophysics \& Space Research Department of Cornell University.
	
	{\sc William Vetterling} is a Research Fellow \& Director of the Image Science Laboratory at ZINK Imaging, LLC in Waltham, MA. His career includes eight years on the physics faculty at Harvard \& 20 years of numerical modeling \& laboratory research on digital imaging at Polaroid Corporation.
	
	{\sc Brian P. Flannery} is Science, Strategy \& Programs Manager at Exxon Mobil Corporation.
	
	\item {\sc William H. Press}. {\it More Than Curious: A Science Memoir}.
	
	\item \cite{Press_Teukolsky_Vetterling_Flannery_recipe_C}. {\sc William H. Press, Saul A. Teukolsky, William T. Vetterling, Brian P. Flannery}. {\it Numerical Recipes in C: The Art of Scientific Computing}. {\sf[76 Amazon ratings][255 Goodreads ratings]}
	
	{\sf Amazon review.} The product of a unique collaboration among 4 leading scientists in academic research \& industry, Numerical Recipes is a complete text \& reference book on scientific computing. In a self-contained manner it proceeds from mathematical \& theoretical considerations to actual practical compute routines. With $> 100$ new routines bringing the total to well $> 300$, plus upgraded versions of the original routines, the new edition remains the most practical, comprehensive handbook of scientific computing available today.
	
	{\bf Editorial reviews.}
	\begin{itemize}
		\item ``$\ldots$ an instant `classic,' a book that should be purchased \& read by anyone who uses numerical methods $\ldots$'' -- American Journal of Physics
		\item ``No matter what language you program in, these packages are classics, both as a textbook or reference. They are an essential \& valuable addition to the academic, professional, or personal library.'' -- Internet
		\item ``The new book exceeds, if possible, the excellence of its predecessor: it is $\approx50\%$ longer \& has been thoroughly updated $\ldots$ The bibliographical material has been considerably extended \& updated $\ldots$ For new users, it is sufficient to say that practically every aspect of numerical analysis is covered $\ldots$ This monumental \& classic work is beautifully produced \& of literary as well as mathematical quality. It is an essential component of any serious scientific or engineering library.'' -- {\sc A. D. Booth}, Computing Reviews
		\item ``If you already have the 1st edition, will you want or need the 2nd? The answer is a definitive yes $\ldots$ a book that should be on your desk (not your shelf) if you have any interest in the analysis of data or the formulation of models.'' -- {\sc  Lyle W. Konigsberg}, Human Biology
		\item ``$\ldots$ the 2nd [edition] expands the scope of coverage \& continues the standard of excellence achieved in the 1st. If you were to have only a single book on numerical methods, this is the one I would recommend.'' -- {\sc Edmund Miller}, IEEE Computational Science \& Engineering
		\item ``$\ldots$ remarkably complete $\ldots$ it contains many more routines than many commercial mathematics packages $\ldots$'' -- Byte
		\item ``The authors are to be congratulated for providing the scientific community with a valuable resource.'' -- The Scientist
		\item ``$\ldots$ replete with the standard spectrum of mathematically pretreated \& coded{\tt/}numerical routines for linear equations, matrices, \& arrays, curves, splines, polynomials, functions, roots, series, integrals, eigenvectors, FFT, \& other transforms, distributions, statistics, \& on to ODE's \& PDE's $\ldots$ such an education $\ldots$ is delightful $\ldots$'' -- Physics in Canada
	\end{itemize}
	
	\item \cite{Press_Teukolsky_Vetterling_Flannery_recipe_Fortran77}. {\sc William H. Press, Saul A. Teukolsky, William T. Vetterling, Brian P. Flannery}. {\it Numerical Recipes in Fortran 77: The Art of Scientific Computing}. {\sf[41 Amazon ratings][38 Goodreads ratings]}
	
	{\sf Amazon review.} This is the greatly revised \& greatly expanded 2e of the hugely popular Numerical Recipes: The Art of Scientific Computing. The product of a unique collaboration among 4 leading scientists in academic research \& industry Numerical Recipes is a complete text \& reference book on scientific computing. In a self-contained manner it proceeds from mathematical \& theoretical considerations to actual practical computer routines. With $> 100$ new routines bringing the total to well $> 300$, plus upgraded versions of the original routines, this new edition remains the most practical, comprehensive handbook of scientific computing available today. Highlights of the new material include:
	\begin{itemize}
		\item A new chapter on integral equations \& inverse methods
		\item Multigrid \& other methods for solving PDEs
		\item Improved random number routines
		\item Wavelet transforms
		\item The statistical bootstrap method
		\item A new chapter on ``less-numerical'' algorithms including compression coding \& arbitrary precision arithmetic.
	\end{itemize}
	The book retains the informal easy-to-read style that made the 1st edition so popular, while introducing some more advanced topics. It is an ideal textbook for scientists \& engineers \& an indispensable reference for anyone who works in scientific computing. The 2e is available in FORTRAN, the traditional language for numerical calculations \& in the increasingly popular C language.
	
	{\sf Editorial reviews.}
	\begin{itemize}
		\item ``This is a phenomenal effort. Virtually anyone involved in scientific computing, from engineers, to physicists, to social scientists, will find information \& methods applicable to their specific needs, or helpful subroutines that can be inserted into the reader's existing programs $\ldots$ No matter what language you program in, these packages are classics, both as a textbook or reference. They are an essential \& valuable addition to the academic, professional, or personal library.'' Internet
		\item ``Anyone who writes (or is curious about) computer codes to solve many of the common numerical problems in science \& engineering will want to own this large book. The writing is authoritative (2 of the authors have published 1st-rate research in writing code for astrophysics problems), but never dull. Flashes of humor appear at regular intervals, in the appropriate places, \& as hard as it may be to believe, this book is interesting even as casual reading! I recommend this book highly, \& both the authors \& the publisher are to be commended for an outstanding piece of work.'' -- {\sc Paul J. Nahin}, Science Books \& Films
		\item ``This encyclopedic book should be read (or at least owned) not only by those who must roll their own numerical methods, but by all who must use prepackaged programs.'' -- {\sc Mike Holderness}, New Scientist
		\item ``This reviewer knows of no other single source of so much material of this nature. Highly recommended.'' -- {\sc R. J. Wernick}, Choice
		\item ``$\ldots$ will be appreciated by anyone involved in the numerical solution of engineering problems $\ldots$ the authors have successfully blended tutorial discussion, fundamental mathematics, explanation of algorithms, \& working computer programs into neatly packaged chapters covering all of the basic topics in numerical methods. What sets this book apart, in the reviewer's opinion, is the versatility of the book $\ldots$ indispensable.'' -- {\sc Ben H. Thacker}, Applied Mechanics Review
		\item ``If you already have the 1e, will you want or need the 2nd? The answer is a definitive yes $\ldots$ a book that should be on your desk (not your shelf) if you have any interest in the analysis of data or the formulation of models $\ldots$ The 2e contains numerous additions of important material, e.g., a section on Cholesky decomposition (which is critical for simulating multivariate distributions), discussions of the boostrap method, \& the addition \& expansion of other numerical methods too numerous to mention here.'' -- {\sc Lyle W. Konigsberg}, Human Biology
		\item ``$\ldots$ a valuable resource for those with a specific need for numerical software. The routines are prefaced with lucid, self-contained explanations $\ldots$ highly recommended for those who require the use \& understanding of numerical software.'' -- {\sc Elizabeth Greenwell Yanik}, SIAM Review
		\item ``$\ldots$ the 2nd [edition] expands the scopes of covereage \& continues the standard of excellence achieved in the 1st. If you were to have only a single book on numerical methods, this is the one I would recommend.'' -- {\sc Edmund Miller}, IEEE Computational Science \& Engineering
	\end{itemize}
	
	\item \cite{Press_Teukolsky_Vetterling_Flannery_recipe_Fortran90}. {\sc William H. Press, Saul A. Teukolsky, William T. Vetterling, Brian P. Flannery}. {\it Numerical Recipes in Fortran 90: Volume 2, Volume 2 of Fortran Numerical Recipes: The Art of Parallel Scientific Computing}. {\sf[16 Amazon ratings]}
	
	{\sf Amazon review.} The 2nd volume of Fortran Numerical Recipes series, Numerical Recipes in Fortran 90 contains a detailed introduction to the Fortran 90 language \& to the basic concepts of parallel programming, plus source code for all routines from 2e of Numerical Recipes. This volume does not repeat any of the discussion of what individual programs actually do, the mathematical methods they utilize, or how to use them.
	
	This book gives a detailed introduction to Fortran 90 \& to parallel programming, with all 350+ routines from the second edition of Numerical Recipes.
	
	{\sf Editorial reviews.}
	\begin{itemize}
		\item ``$\ldots$ this present volume will contribute decisively to a significant breakthrough, as it provides models not only of the numerical algorithms for which previous editions are already famed, but also of an excellent Fortran 90 style $\ldots$'' -- From the Foreword by {\sc Michael Metcalf}, 1 of Fortran 90's original designers \& author of FORTRAN 90 Explained
		\item ``As with previous volumes in this series, this book is a classical \& is essential reading for anyone concerned with the future of numerical calculation. It is beautifully produced, inexpensive for its content, \& a must for any serious worker or student.'' -- {\sc A. D. Booth}, Computing Reviews
		\item ``$\ldots$ very useful for graduate \& postgraduate courses \& also for all specialists who are interested in parallel scientific computing.'' -- {\sc Mikhail P. Levin}, IEEE Concurrency
	\end{itemize}
	
	\item \cite{Press_Teukolsky_Vetterling_Flannery_recipe_Fortran90}. {\sc William H. Press, Saul A. Teukolsky, William T. Vetterling, Brian P. Flannery}. {\it Numerical Recipes in Pascal: The Art of Scientific Computing}. {\sf[22 Amazon ratings]}
	
	{\sf Amazon review.} This version of the 1st edition of Numerical Recipes contains the original 200 routines translated into Pascal along with the tutorial text. In a single volume, Numerical Recipes in Pascal provides lucid, easy-to-read discussions of the most important practical numerical techniques for science \& engineering. The programs contained in this book are also available as machine-readable code on the Numerical Recipes Code CD-ROM with Windows{\tt/}Macintosh Single Screen License. Numerical Recipes in Pascal fills a long-recognized need for a practical, comprehensive handbook of scientific computing in the Pascal language.
	
	{\sf Editorial reviews.}
	\begin{itemize}
		\item ``This book is indispensable to anyone who wishes to employ numerical techniques quickly \& confidently without becoming an expert in applied mathematics.'' --  
		Journal of Nuclear Medicine
		\item ``Anyone who is numerate will read it with profit; anyone who is literate will read it with satisfaction; \& anyone who has a sense of humor will read it with real enjoyment.'' -- Observatory
		\item ``Truly impressive is the insight the authors offer throughout into various aspects of the numerical methods presented.'' -- Physics Today
		\item ``There is no way that this review can even begin to convey the truly immense amount of material that is in this book.'' -- Technometrics
		\item ``Splendid text \& reference on the art of scientific programming. As a sourcebook of important algorithms \& their background information, proceeds effortlessly from the math \& theory to the $> 200$ practical Pascal routines.'' -- Computer Book Review
		\item ``It really still remains a gold mine for those who need ready solutions to computing problems $\ldots$ You might well ask, `do we really need another version whose text is unchanged but where the algorithms are presented in Pascal rather than in FORTRAN?' The answer is `yes' if you, like me, are a Pascal programmer.'' -- Geological Magazine
		\item ``$\ldots$ a `toolbox' for people in the sciences \& engineering who frequently work with numbers.'' -- American Scientist
		\item ``I would recommend this text, or any of the series, to scientific programmers.'' -- Physics in Canada
		\item ``The book's virtues are that it lists important topics from numerical analysis that may be of interest to scientists \& engineers. It gives a summary of the philosophy behind each of the methods discussed \& a bibliography so that one can find out more.'' -- Computing Reviews
		\item ``This is a phenomenal effort. Virtually anyone involved in scientific computing, from engineers, to physicists, to social scientists, will find information \& methods applicable to their specific needs, or helpful subroutines that can be inserted into the reader's existing programs $\ldots$ No matter what language you program in, these packages are classics, both as a textbook or reference. They are an essential \& valuable addition to the academic, professional, or personal library.'' -- Internet
	\end{itemize}
	
	\item \cite{Shotts2019}. {\sc William Shotts}. {\it The Linux Command Line: A Complete Introduction}.\hfill{\sf[reading]}
	
	\item \cite{Stroustrup2013}. {\sc Bjarne Stroustrup}. {\it The C++ Programming Language, 4th edition}.\hfill{\sf[reading]}
	
	\item \cite{Stroustrup2018}. Bjarne Stroustrup. {\it A Tour of C++, 2nd edition}.\hfill{\sf[reading]}
	
	\item \cite{Thu_Phuong_Tien_Triet_NMLT}. Trần Đan Thư, Nguyễn Thanh Phương, Đinh Bá Tiến, Trần Minh Triết. {\it Nhập Môn Lập Trình}.\hfill{\sf[reading]}
	
	\item \cite{Thu_Phuong_Tien_Triet_Phuong_KTLT}. Trần Đan Thư, Nguyễn Thanh Phương, Đinh Bá Tiến, Trần Minh Triết, Đặng Bình Phương. {\it Kỹ Thuật Lập Trình}.\hfill{\sf[reading]}
	
	\item \cite{Thu_Tien_Khang_PPLTHDT}. Trần Đan Thư, Đinh Bá Tiến, Nguyễn Tấn Trần Minh Khang. {\it Phương Pháp Lập Trình Hướng Đối Tượng}.\hfill{\sf[reading]}
\end{enumerate}

\paragraph{Artificial Intelligence AI}

\begin{enumerate}
	\item \cite{Bac_Viet_AI}. Lê Hoài Bắc, Tô Hoài Việt. {\it Cơ Sở Trí Tuệ Nhân Tạo}.\hfill{\sf[done]}
	
	\item Trieu H. Trinh, Yuhuai Wu, Quoc V. Le, He He, Thang Luong. {\it Solving olympiad geometry without human demonstrations}. Nature.\hfill{\sf[reading]}
\end{enumerate}

%------------------------------------------------------------------------------%

\subsection{Education Book}

\begin{enumerate}
	\item \cite{Brown_Roediger_McDaniel_stick}. {\sc Peter C. Brown, Henry L. Roediger III, Mark A. McDaniel}. {\it Make It Stick: The Science of Successful Learning}.
	
	{\sf Amazon review.} The international bestseller that has helped millions of students, teachers, \& lifelong learners use proven approaches to learn better \& remember longer.
	
	``We have made {\it Make It Stick} a touchstone for our instructors $\ldots$ to gain a real advantage for our learners as they tackle some of the toughest work in the world.'' -- {\sc Carl Czech}, former Senior Instructional Systems Specialist{\tt/}Advisor, US Navy SEALs
	
	Are you tired of forgetting what you learn? This groundbreaking book, based on the latest research in cognitive science, offers powerful strategies to boost memory \& learning.
	
	To most of us, learning something ``the hard way'' means wasted time \& effort. Good teaching, many believe, should be tailored to the different learning styles of students \& should use strategies that make learning easier. {\it Make It Stick} turns fashionable ideas like these on their head. Drawing on recent discoveries in cognitive psychology \& a 10-year collaboration among some of the world's leading experts on human learning \& memory, the authors explain what {\it really} drives successful learning. With clear, real-world examples, they show how we can confidently hone our skills \& learn more effectively.
	
	Many common study habits simply don't work. Underlining, highlighting, rereading, cramming, \& single-minded repetition of new skills create the illusion of mastery, but gains fade quickly. Science shows that more durable learning comes from self-testing, introducing certain difficulties in practice, waiting to re-study new material until a little forgetting has occurred, \& interleaving the practice of 1 skills or topic with another. {\it Make It Stick} breaks down these proven approaches in compelling ways \& offers concrete techniques for becoming more productive learners.
	
	Full of eye-opening \& inspiring stories for students, educators, \& parents, {\it Make It Stick} is an indispensable guide for all those interested in the challenge of lifelong learning \& self-improvement.
	\begin{quotation}
		{\it``Elaboration is the process of giving new material meaning by expressing it in your own words \& connecting it with what you already know.''}
		
		-- Xây dựng là quá trình tạo ra ý nghĩa vật chất mới bằng cách diễn đạt nó bằng từ ngữ của riêng bạn \& kết nối nó với những gì bạn đã biết.
		
		{\it``Trying to solve a problem before being taught the solution leads to better learning, even when errors are made in the attempt.''}
		
		-- Cố gắng giải quyết một vấn đề trước khi được dạy giải pháp sẽ dẫn đến việc học tập tốt hơn, ngay cả khi mắc lỗi trong nỗ lực đó.
		
		{\it``Learning is stronger when it matters, when the abstract is made concrete \& personal.''}
		
		-- Việc học tập sẽ mạnh mẽ hơn khi nó quan trọng, khi phần tóm tắt được cụ thể hóa \& mang tính cá nhân.
		
		{\it``Mastery requires both the possession of ready knowledge \& the conceptual understanding of how to use it.''}
		
		-- Sự thành thạo đòi hỏi cả việc sở hữu kiến thức đã có sẵn \& sự hiểu biết mang tính khái niệm về cách sử dụng nó.
	\end{quotation}
	{\sf Editorial reviews.}
	\begin{itemize}
		\item ``If you want to read a lively \& engaging book on the science of learning, this is a must $\ldots$ {\it Make It Stick} benefits greatly from its use of stories about people who have achieved mastery of complex knowledge \& skills. Over the course of the book, the authors weave together stories from an array of learners -- surgeons, pilots, gardeners, \& school \& university students -- to illustrate their arguments about how to successful learning takes place $\ldots$ This is a rich \& resonant book \& a pleasurable read that will leave you pondering the processes through which you, \& your students, acquire new knowledge \& skills.'' -- {\sc Hazel Christie}, {\it Times Higher Education}
		\item ``Many educators are interested in making use of recent findings about the human brain \& how we learn $\ldots$ {\it Make It Stick} [is] the single best work I have encountered on the subject. Anyone with an interest in teaching or learning will benefit from reading this book, which not only presents thoroughly grounded research but does so in an eminently readable way that is accessible even to students.'' -- {\sc James M. Lang}, {\it Chronicle of Higher Education}
		\item ``We have made {\it Made It Stick} a touchstone for our instructors $\ldots$ to gain a real advantage for our learners as they tackle some of the toughest work in the world.'' -- {\sc Carl Czech}, former Senior Instructional Systems Specialist{\tt/}Advisor, US Navy SEALs
		\item ``It is surprising to me [that] we have such highly educated people coming to medical school who haven't thought that deeply about learning. I feel like we are teaching the gospel of {\it Make It Stick} during our 1st weeks with the students $\ldots$ With the immense time pressure you have as a medical student, the importance of these principles becomes very clear to them.'' -- {\sc Randall King}, Harry C. McKenzie Professor of Cell Biology, Harvard Medical School
		\item ``It's an illuminating read $\ldots$ Learning ability is probably the most important skill you can have. Unfortunately, lots of the techniques for learning that we pick up in school don't help with long-term recall -- like cramming or highlighting $\ldots$ For a deeper dig into the science of learning, make sure to pick up {\it Make It Stick}.'' -- {\sc Drake Baer}, {\it Business Insider}
		\item ``Aimed primarily at students, parents, \& teachers, {\it Make It Stick} also offers practical advice for learners of all ages, at all stages of life $\ldots$ With its credible challenge to conventional wisdom, {\it Make It Stick} does point the way forward, with a very real prospect of tangible \& enduring benefits.'' -- {\sc Glenn C. Altschuler}, {\it Psychology Today}
		\item ``{\it Make It Stick} will help you become a much more productive learner. [It] presents a compelling case fo why we are attracted to the wrong strategies for learning \& teaching -- \& what we can do to remedy our approaches $\ldots$ In clear language, {\it Make It Stick} explains the science underlying how people learn. But the authors don't simply recite the research; they show readers how it is applied in real-life learning scenarios, with engaging stories of real people in academic, professional, \& sports environments $\ldots$ The learning strategies proposed in this book can be implemented immediately, at no cost, \& to great effect.'' -- {\sc Stephanie Castellano}, {\it TD Magazine}
		\item ``If I could, I would assign all professors charged with teaching undergraduates 1 book: {\it Make It Stick: The Science of Successful Learning} $\ldots$ It lays out what we know about the science of learning in clear, accessible prose. Every educator -- \& parent, \& student, \& professional -- ought to have it on their own personal syllabus.'' -- {\sc Annie Murphy Paul}, author of {\it The Extended Mind}
		\item ``The authors have provided a great service for educators by capturing the important lessons from decades of research in the learning sciences $\ldots$ It should be highly recommended reading for anyone in the teaching, learning, \& training professions.'' -- {\sc Robert H. Bruininks}, Professor \& President Emeritus, University of Minnesota
		\item ``This is a quite remarkable book. It describes important research findings with startling implications for how we can improve our own learning, teaching, \& coaching. Even more, it shows us how more positive attitudes toward our own learning, teaching, \& coaching. Even more, it shows us how more positive attitudes toward our own abilities -- \& the willingness to tackle the hard stuff -- enables us to achieve our goals. The compelling stories bring the ideas out of the lab \& into the real world.'' -- {\sc Robert Bjork}, University of California, Los Angeles
		\item ``Learning is essential \& life-long. Yet as these authors argue convincingly, people often use exactly the wrong strategies \& don't appreciate the ones that work. We've learned a lot in the last decade about applying cognitive science to real-world learning, \& this book combines everyday examples with clear explanations of the research. It's easy to read -- \& should be easy to learn from, too!'' -- {\sc Daniel L. Schacter}, author of {\it The 7 Sins of Memory}
		\item ``Anyone who teaches anything would benefit from reading this book: coaches, tutors, classroom teachers, parents, even corporate trainers. Instead of dong what we've always done \& wondering why some learners just don't get it, we can take a different approach that's based on research, even if it seems counterintuitive.'' -- {\sc Jennifer Gonzales}, {\it Cult of Pedagogy}		
	\end{itemize}
	{\sf About the Author.} {\sc Peter C. Brown} is a writer \& former management consultant. {\sc Henry L. Roediger III} is James S. McDonnell Distinguished University Professor of Psychology at Washington University in St. Louis. {\sc Mark A. McDaniel} is Professor of Psychology \& Director of the Center for Integrative Research on Cognition, Learning, \& Education (CIRCLE) at Washington University in St. Louis.
	
	\item \cite{SuperSummary_Brown_Roediger_McDaniel_stick}. {\sc SuperSummary}. {\it Study Guide: Make It Stick by Peter C. Brown, Henry L. Roediger III, Mark A. McDaniel (SuperSummary)}.
	
	\item \cite{Oakley_mind_number}. Barbara Oakley. {\it A Mind for Numbers: How to Excel at Math \& Science (Even If You Flunked Algebra)}.\hfill{\sf[reading]}
	
	\item \cite{Oakley_mind_number}. Barbara Oakley. {\it A Mind for Numbers: How to Excel at Math \& Science (Even If You Flunked Algebra) -- Cách Chinh Phục Toán \& Khoa Học (Ngay Cả Khi Bạn Vừa Trượt Môn Đại Số)}.\hfill{\sf[done]}
	
	\item \cite{Oakley_Sejnowski_McConville_learn_how_learn}. Barbara Oakley, Terrence Sejnowski, Alistair McConville. {\it Learning How to Learn: How to Succeed in School Without Spending All Your Time Studying; A Guide for Kids \& Teens}.\hfill{\sf[reading]}
	
	\item \cite{Oakley_Sejnowski_McConville_learn_how_learn}. Barbara Oakley, Terrence Sejnowski, Alistair McConville. {\it Learning How to Learn: How to Succeed in School Without Spending All Your Time Studying; A Guide for Kids \& Teens -- Học Cách Học: Công Cụ Trí Tuệ Mạnh Mẽ Chinh Phục Mọi Môn Học}.\hfill{\sf[done]}
	
	\item \cite{Oakley_Rogowsky_Sejnowski_McConville_uncommon_sense_teaching}. Barbara Oakley, Beth Rogowsky, Terrence J.Sejnowski. {\it Uncommon Sense Teaching: Practical Insights in Brain Science to Help Students Learn -- Dạy Học Không Theo Lối Mòn: Hiểu Đúng Về Trí Nhớ \& Khoa Học Não Bộ Để Dạy Học Hiệu Quả Trong Mọi Hoàn Cảnh}.\hfill{\sf[done]}
	
	\item \cite{Yukichi_encourage_learn}. {\sc Fukuzawa Yukichi}. {\it An Encouragement of Learning}.\hfill{\sf[reading]}
	
	\item \cite{Yukichi_khuyen_hoc}. {\sc Fukuzawa Yukichi}. {\it An Encouragement of Learning -- Khuyến Học}.\hfill{\sf[done]}
\end{enumerate}

%------------------------------------------------------------------------------%

\subsection{Literary Book}

\begin{enumerate}
	\item \cite{Can_oc_sang_suot}. {\sc Nguyễn Duy Cần -- Thu Giang}. {\it Óc Sáng Suốt}.\hfill{\sf[done]}
	
	\item \cite{Can_thuat_tu_tuong}. {\sc Nguyễn Duy Cần -- Thu Giang}. {\it Thuật Tư Tưởng}.\hfill{\sf[done]}
	
	\item \cite{Can_tu_hoc}. {\sc Nguyễn Duy Cần -- Thu Giang}. {\it Tôi Tự Học}.\hfill{\sf[done]}
	
	\item \cite{Can_nha_van}. {\sc Nguyễn Duy Cần -- Thu Giang}. {\it Để Thành Nhà Văn}.\hfill{\sf[done]}
	
	\item \cite{Can_thuat_xu_the}. 	{\sc Nguyễn Duy Cần -- Thu Giang}. {\it Thuật Xử Thế Của Người Xưa}.\hfill{\sf[done]}
	
	\item \cite{Can_dung_thanh_nhan}. {\sc Nguyễn Duy Cần -- Thu Giang}. {\it Cái Dũng Của Thánh Nhân}.\hfill{\sf[done]}
	
	\item \cite{Can_thuat_yeu_duong}. {\sc Nguyễn Duy Cần -- Thu Giang}. {\it Thuật Yêu Đương}.\hfill{\sf[done]}
	
	\item \cite{Can_nghe_thuat_song}. {\sc Nguyễn Duy Cần -- Thu Giang}. {\it Một Nghệ Thuật Sống}.\hfill{\sf[done]}
	
	\item {\sc Nguyễn Duy Cần -- Thu Giang}. {\it Cái Cười Của Thánh Nhân}.
	
	\item {\sc Nguyễn Duy Cần -- Thu Giang}. {\it Trang Tử Tinh Hoa}.
	
	\item {\sc Nguyễn Duy Cần -- Thu Giang}. {\it Dịch Học Nhập Môn}.
	
	\item {\sc Nguyễn Duy Cần -- Thu Giang}. {\it Dịch Học Tinh Hoa}.
	
	\item {\sc Nguyễn Duy Cần -- Thu Giang}. {\it Thanh Dạ Văn Chung}.

	\item {\sc Nguyễn Duy Cần -- Thu Giang}. {\it Chu Dịch Huyền Giải}.
	
	\item {\sc Nguyễn Duy Cần -- Thu Giang}. {\it Phật Học Tinh Hoa}.
	
	\item {\sc Nguyễn Duy Cần -- Thu Giang}. {\it Lão Tử Tinh Hoa}.
	
	\item {\sc Nguyễn Duy Cần -- Thu Giang}. {\it Lão Tử Đạo Đức Kinh}.
	
	\item {\sc Nguyễn Duy Cần -- Thu Giang}. {\it Trang Tử Nam Hoa Kinh}.
	
	\item \cite{Chip_US_dream}. {\sc Huyền Chip}. {\it Giấc Mơ Mỹ -- Đường Đến Stanford}.\hfill{\sf[done]}
	
	\item \cite{Coelho_alchemist_VN}. {\sc Paul Coelho}. {\it Nhà Giả Kim}.\hfill{\sf[done]}
	
	\item \cite{Giang_nature}. {\sc Đặng Hoàng Giang}. {\it Vẻ Đẹp Của Cảnh Sắc Tầm Thường Hay Vì Sao Chúng Ta Cần Thay Đổi Cách Thưởng Thức Thiên Nhiên?}\hfill{\sf[done]}
	
	\item \cite{Herring2016}. {\sc Peter Herring}. {\it The Farlex Grammar Book: Complete English Grammar Rules: Examples, Exceptions, Exercises, \& Everything You Need to Master Proper Grammar}.\hfill{\sf[reading]}
	
	\item {\sc Han Kang}. {\it Human Acts: A Novel}.
	
	\item {\sc Han Kang}. {\it The Vegetarian: A Novel}.
	
	\item {\sc Han Kang}. {\it Greek Lessons: A Novel}.
	
	\item {\sc Han Kang}. {\it We Do No Part: A Novel}.
	
	\item {\sc Han Kang}. {\it The White Book}. {\sf[665 Amazon ratings][20872 Goodreads ratings]}
	
	{\sf Amazon review.} From {\sc Han Kang}, winner of the 2024 Nobel Prize in Literature.
	\begin{itemize}
		\item ``[{\sc Han Kang} writes in] intense poetic prose that $\ldots$ exposes the fragility of human life.'' -- from the Nobel Prize citation
		\item Stunning beautiful $\ldots$ 1 of the smartest reflections on what it means to remember whose we've lost.'' -- NPR
	\end{itemize}
	Shortlisted for the International Booker Prize. A ``formally daring, emotionally devastating, \& deeply political'' ({\it The New York Times Book Review}) exploration of personal grief through the prism of the color white, from the internationally bestselling author of {\it The Vegetarian}. Shortlisted for the International Booker Prize, {\sc Han Kang}'s {\it The White Book} is a meditation on color, as well as an attempt to make sense of her older sister's death, who died in her mother's arms just a few hours after she was born. In captivating, starkly beautiful language, {\it The White book} is a letter from {\sc Kang} to her sister, offering a multilayered exploration of color \& its absence, \& of the tenacity \& fragility of human spirit.
	
	{\sf Editorial review.}
	\begin{itemize}
		\item ``A brilliant psychogeography of grief, moving as it does between place, history \& memory $\ldots$ Poised \& never flinches from serene dignity $\ldots$ {\it The White Book} is a mysterious text, perhaps in part a secular prayer book $\ldots$ Translated peerlessly by {\sc Smith}, [it] succeeds in reflecting {\sc Han}'s urgent desire to transcend pain with language.'' -- {\it The Guardian}
		\item ``With eloquence \& grace, {\sc Han} breathes life into loss \& fills the emptiness with this new work.'' -- {\it Library Journal}
		\item ``Everything I ever thought about the color white has been profoundly altered by reading {\sc Han Kang}'s brilliant exploration of its meaning \& the ways in which white shapes her world, from birth to death -- including the death of {\it The White Book}'s narrator's older sister, who died just a few hours after she was born, in her mother's arms. This is an unforgettable meditation on grief \& memory, resilience \& acceptance, all offered up in {\sc Han}'s luminous, intimate prose.'' -- {\it Nylon}
		\item ``{\sc Han}'s 1st 2 English-language translations were instant sensations, establishing her as a riveting practitioner of the surreal \& of historical fiction alike. Her latest $\ldots$ is told by a woman haunted by the death of her elder sister just after birth -- a contemplation of life, death, resilience \&, as the title hints, color.'' -- {\it HuffPost}
		\item ``[{\it The White Book}] promises to be equal parts {\sc talo Calvino, Angela Carter}, \& something entirely {\sc Han Kang}'s own $\ldots$ A quieter, yet just as intensely symbolic, follow-up to the startling violence of her 1st 2 books.'' -- {\it LibHub}
		\item ``A quietly gripping contemplation on life, death, \& the existential impact of those who have gone before.'' -- {\sc Eimear McBride}, author of {\it The Lesser Bohemians}
		\item ``{\it The White Book} is a profound \& precious thing, its language achingly intimate, each image haunting \& true. It is a remarkable achievement. {\sc Han Kang} is a genius.'' -- {\sc Lisa Mclnerney}, author of {\it The Glorious Heresies}
		\item ``{\sc Kang}'s masterful voice is captivating \& nothing short of brilliant.'' -- {\it Booklist}, starred review
	\end{itemize}
	{\sf About the Author.} {\sc Han Kang} was born in 1970 in South Korea. She is the author of {\it The Vegetarian}, winner of the International Booker Prize, as well as {\it Human Acts, The White Book, Greek Lessons, \& We Do Not Part}. In 2024, she was awarded the Nobel Prize in Literature.
	
	\item \cite{King2000, King2010}. Stephen King. {\it On Writing: On Writing: A Memoir of the Craft}.\hfill{\sf[done]}
	
	\item \cite{Krakauer_wild}. {\sc Jon Krakauer}. {\it Into The Wild}.
	
	\item \cite{Murakami_bird_VN}. {\sc Haruki Murakami}. {\it The Wind-Up Bird Chronicle: A Novel -- Biên Niên Ký Chim Vặn Dây Cót}.\hfill{\sf[done]}
	
	\item \cite{Murakami_Kafka_VN}. {\sc Haruki Murakami}. {\it Kafka on the Shore -- Kafka Bên Bờ Biển}.\hfill{\sf[done]}
	
	\item \cite{Murakami_run}. {\sc Haruki Murakami}. {\it What I Talk about When I Talk about Running: A Memoir}. {\sf[8677 Amazon ratings][179832 Goodreads ratings]}
	
	{\sf Amazon review.} An intimate look at writing, running, \& the incredible way they intersect, {\it What I Talk About When I Talk About Running} is an illuminating glimpse into the solitary passions of 1 of our greatest artists. While training for the New York City Marathon, {\sc Haruki Murakami} decided to keep a journal of his progress. The result is a memoir about his intertwined obsessions with running \& writing, full of vivid recollections \& insights, including the eureka moment when he decided to become a writer. By turns funny \& sobering, playful \& philosophical, here is a rich \& revelatory work that elevates the human need for motion to an art form. A memoir about the author's intertwined obsessions with running \& writing, full of vivid recollections \& insights.
	\begin{itemize}
		\item ``Emotional hurt is the price a person has to pay in order to be independent.''
		\item ``The most important thing we ever learn at school is the fact that the most important things can't be learned at school.''
		\item ``I'm struck by now, except when you're young, you really need to prioritize in life, figuring out in what order you should divide up your time \& energy. If you don't get that sort of system set by a certain age, you'll lack focus \& you life will be out of balance.''
		\item ``No matter how mundane some action might appear, keep at it long enough \& it becomes a contemplative, even meditative act.''
	\end{itemize}
	{\sf Editorial review.}
	\begin{itemize}
		\item ``A fascinating portrait of {\sc Murakami}'s working mind \& how he works his magic on the page.'' -- {\it The Plain Dealer}
		\item ``A brilliant meditation on how his running \& writing nurture \& sustain each other $\ldots$ With spare, engaging prose $\ldots$ {\sc Murakami} shares his runner's high.'' -- {\it Sports Illustrated}
		\item ``Enthralling $\ldots$ A quirky, brilliant gem.'' -- {\it Time Out New York}
		\item ``{\sc Murakami}'s descriptive eye is as acute as ever $\ldots$ Fascinating $\ldots$ A glimpse into the creative process of 1 of the world's great writers.'' -- {\it The Hartford Courant}
		\item ``A genuine memoir, filled with gentle minutiae that truly communicates the rhythm of {\sc Murakami}'s daily life \& work $\ldots$ {\sc Murakami} actually offers himself {\it whole}.'' -- {\sc Jesse Jarnow}, {\it Paste Magazine}
		\item ``A felicitous, casual series of reflections \& anecdotes $\ldots$ [{\sc Murakami}] has a Warholian way of tinting the mundane with mystery \& restrained humor $\ldots$ Do still waters run deep? This paean to a runner's life keeps us, pleasurably, wondering.'' -- {\sc Joel Rice}, {\it The Tennessean}
		\item ``[{\it What I Talk About When I Talk About Running} is] a graceful explanation of Mr. {\sc Murakami}'s intertwining obsessions, conveyed with his characteristic ability to draw unexpected connections. Running may be a matter of placing 1 foot in front of the other on the ground, but, as is so often the case with Mr. {\sc Murakami}, terrestrial objects have a tendency to take flights.'' -- {\sc Chloë Schama}, {\it New York Sun}
		\item ``Beautifully written \& full of great running aphorisms $\ldots$ Anyone who knows perseverance can appreciate this work.'' -- {\sc Helen Montoya}, {\it San Antonio Express-News}
		\item ``Engaging, insightful $\ldots$ {\it What I Talk About When I Talk About Running} extends [{\sc Murakami}'s] winning streak.'' -- {\sc Jenny Shank}, {\it Sunday Camera}
		\item ``{\sc Murakami} constructs this piecemeal narrative with the same masterful, accessible prose marked by humor \& streaks of magic which has made him a household name, the same staggering insights, the same fascinating connections $\ldots$ this is exactly what makes {\sc Murakami} so special: his ability to render everything a part of everything else, \& to end with monumental poignancy $\ldots$ In an extremely personal, candid \& moving way, the book makes one want to read \& run at the same time.'' -- {\sc Reynard Seifert}, {\it Austin Fit Magazine}
		\item ``[{\it What I Talk About When I Talk About Running}] provides a fascinating portrait of {\sc Murakami}'s working mind \& how he works his magic on the page $\ldots$ [a] charming, sober little book.'' -- {\sc John Freeman}, Neward {\it Star-Ledger}
		\item ``Highly recommended $\ldots$ Practical philosophy from a man whose insight into his own character, \& how running both suits \& shapes that character, is revelatory \& can provide tools for readers to examine \& improve their own lives.'' -- {\it Library Journal}
	\end{itemize}
	
	\item \cite{Murakami_run_VN}. {\sc Haruki Murakami}. {\it What I Talk about When I Talk about Running: A Memoir -- Tôi Nói Gì Khi Nói Về Chạy Bộ}.\hfill{\sf[done]}
	
	\item \cite{Murakami_Norwegian_wood_VN}. {\sc Haruki Murakami}. {\it Norwegian Wood -- Rừng Na Uy}.\hfill{\sf[done]}
	
	\item \cite{Murakami_Tokyo_VN}. {\sc Haruki Murakami}. {\it Tokyo Kitan-Shu -- Những Chuyện Lạ Ở Tokyo}.\hfill{\sf[done]}
	
	\item \cite{Murakami_Sputnik_VN}. Haruki Murakami. {\it Sputnik Sweetheart -- Người Tình Sputnik}.\hfill{\sf[done]}
	
	\item \cite{Murakami_Tsukuru_Tazaki_VN}. Haruki Murakami. {\it Colorless Tsukuru Tazaki \& His Years of Pilgrimage -- Tazaki Tsukuru Không Màu \& Những Năm Tháng Hành Hương}.\hfill{\sf[done]}
	
	\item \cite{Murakami_1st_person}. {\sc Haruki Murakami}. {\it First Person Singular: Stories}. {\sf[3360 Amazon ratings][44632 Goodreads ratings]}
	
	{\sf Amazon review.} National Best Seller. A mind-bending new collection of short stories from the internationally acclaimed, best-selling author.
	\begin{itemize}
		\item ``Some novelists hold a mirror up to the world \& some, like {\sc Haruki Murakami}, use the mirror as a portal to a universe hidden beyond it.'' -- {\it The Wall Street Journal}
	\end{itemize}
	The 8 stories in this new book are all told in the 1st person by a classic {\sc Murakami} narrator. From memories of youth, meditations on music, \& an ardent love of baseball, to dreamlike scenarios \& invented jazz albums, together these stories challenge the boundaries between our minds \& the exterior world. Occasionally, a narrator may or may not be {\sc Murakami} himself. Is it memoir or fiction? The reader decides.
	
	Philosophical \& mysterious, the stories in {\it1st Person Singular} all touch beautifully on love \& solitude, childhood \& memory $\ldots$ all with a signature {\sc Murakami} twist.
	
	{\sf Editorial reviews.}
	\begin{itemize}
		\item ``{\it All fiction is magic}. That's the thought that occurred to me often as I read {\it1st Person Singular}, the brilliant new book of stories by {\sc Haruki Murakami} $\ldots$ Whatever you want to call {\sc Murakami}'s work -- magic realism, supernatural realism -- he writes like a mystery tramp, exposing his global readership to the essential \& cosmic (yes, cosmic!) questions that only art can provoke: What does it mean to carry the baggage of identity? Who is this inside my head in relation to the external, so-called real world? Is the person I was years ago the person I am now? Can a name be stolen by a monkey? $\ldots$ Describing how these stories succeed is like trying to describe exactly why, $> 50$ years later, a Beatles song still sounds fresh.'' -- {\sc David Means}, {\it The New York Times Book Review}
		\item ``{\it1st Person Singular} marks a blazing \& brilliant return to form $\ldots$ Here we have a taut \& tight, suspenseful \& spellbinding, witty \& wonderful group of 8 stories $\ldots$ All are told in the 1st person, most by narrators looking back from the vantage point of middle age on youthful experiences, obsessions, or encounters. \& there isn't a weak one in the bunch. The stories echo with {\sc Murakami}'s preoccupations. Nostalgia \& longing for the charged, evocative moments of young adulthood. Memory's power \& fragility; how identity forms $\ldots$ the at once intransigent \& fragile nature of the ``self.'' Guilt, shame, \& regret for mistakes made $\ldots$ Music's power to make indelible impressions $\ldots$ The themes become a kind of meter against which all the stories make their particular, chiming rhythms $\ldots$ This mesmerizing collection would make a superb introduction to {\sc Murakami} for anyone who hasn't yet fallen under his spell; his legion of devoted fans will gobble it up \& beg for more.'' -- {\sc Priscilla Gilman}, {\it The Boston Globe}
		\item ``{\sc Haruki Murakami} is a master of mesmerizing head-scratcher. His fiction, whether long or short, highlights life's essential strangeness \& unfathomability $\ldots$ The 8 stories in {\it1st Person Singular} [$\ldots$] are classic {\sc Murakami}, filled with multiple recurrent obsessions -- jazz, classical music, Beatles, baseball, \& memories of perplexing young love $\ldots$ {\sc Murakami}'s plainspoken short stories, like his more complex novels, raise existential questions about perception, memory, \& the meaning of it all -- though he's the opposite of heavy-handed, \& rarely proposes answers $\ldots$ What is it all about, his frequently awestruck \& befuddled characters wonder repeatedly -- \& contagiously $\ldots$ ``Confessions of a Shinagawa Monkey'' is a standout that will appeal especially to readers enchanted by {\sc Murakami}'s surrealist turns, which blur the line between dreams \& reality $\ldots$ [A] winning collection.'' -- {\sc Heller McAlpin}, {\it NPR}
		\item ``{\sc Haruki Murakami} often seems most at home in his short-story collections, cycling through his various fixations unburdened by the narrative mechanics of his novels. {\it1st Person Singular} is no exception, offering ruminations on the fickleness of memory while fleeing from baseball to Beatlemania to Kafka-inspired talking monkey.'' -- {\sc Chris Stanton}, {\it Vulture}
		\item ``For new readers, {\it1st Person Singular} is a crash course in appreciating {\sc Murakami} $\ldots$ [These stories] are steeped in the love of music -- especially of jazz, classical \& the Beatles -- that reverberates throughout his work. There is a piece on his famous passion for baseball (it was supposedly while watching a game that he was inspired to become a writer) \& another that includes the return of a talking monkey he 1st wrote about 15 years ago. Most of all, though, these stories are unmistakably {\sc Murakami}'s for the way they traffic in his signature themes of time \& memory, nostalgia \& young love. They are characterized, like so much of his writing, by the collision of everyday realism with the surreal \& the sublime.'' -- {\sc Alexander Nurnberg}, {\it The Times (UK)}
		\item ``{\sc Murakami} has woven a lifelong obsession with music into his writing, including in his stunning {\it1st Person Singular} $\ldots$ The pieces here tap the author's infatuations with the Beatles \& {\sc Mozart}, baseball \& poetry, transgressive sex \& fleeting romance, served up with dollops of American pop culture. It's all here, narrated in a range of voices, from deadpan poet to magical realist to song critic. {\sc Duke Ellington, Charlie Parker}, \& other jazz greats pop up throughout $\ldots$ But his tastes are wide-ranging: the Beatles make a cameo as well, \& the author's passion for classical music fuels the subtle, stirring ``Carnaval'' $\ldots$ {\sc Murakami}'s encyclopedic knowledge of music surges to the force, echoed in vivid imagery.'' -- {\sc Hamilton Cain}, {\it Oprah Daily}
		\item ``{\it1st Person Singular} will satisfy [{\sc Murakami's}] fans \& serve as a fine introduction to neophytes, echoing many of the uncanny scenarios of his earlier work $\ldots$ In ``Cream,'' the opening story of the collection, a lovesick young man goes to a piano recital located in the mountains of Kobe, only to find no one there. In unsettling episodes that one might find in a Flannery O'Connor story, he encounters a car broadcasting a Christian message that everyone will die \& be judged harshly for their sins $\ldots$ The collection's Kafkaesque titular story is the strongest because of its notable timeliness $\ldots$ These 8 stories, all told in 1st person, are unapologetically {\sc Murakami} $\ldots$ [\&] will remind readers why {\sc Murakami}'s work is singular.'' -- {\sc Leland Cheuk}, {\it The Washington Post}
		\item ``To step into {\it1st Person Singular} is to cross from our present moment \& into a lost country demarcated by old memories $\ldots$ [The story] ``Confessions of a Shinagawa Monkey'' is as fun as anything I've read during this pandemic lockdown $\ldots$ The collection ends, brilliantly, with an interrogation. A man sits at a bar \& a stranger begins to berate him about an event he has no memory of $\ldots$ For all our reminiscing, {\sc Murakami} seems to say, it's the things we don't remember that might haunt us the most. After all, memory is itself another liminal space, on where we experience both now \& then at the same time. Likewise, finishing {\it1st Person Singular} requires thinking back to everything we've just read about these characters' lives, \& to everything we didn't.'' -- {\sc Andrew Ervin}, {\it The Brooklyn Rail}
		\item ``[{\sc Murakami} is] 1st \& foremost a remarkably accessible storyteller. His books are an intimate to revel in his perpetually unpredictable, yet remarkably convincing, imagination $\ldots$ {\sc Murakami} writes with such assurance as to turn the implausible credible, the outlandish engrossing. Each story enthralls.'' -- {\sc Terry Hong}, {\it Christian Science Monitor}
		\item ``I'm 4 stories into the 8 that make up {\it1st Person Singular}, \& I can't stop thinking about their beauty, their charm, \& their weirdness.'' -- {\sc Patrick Rapa}, {\it The Philadelphia Inquirer}
		\item ``The stories in {\sc Haruki Murakami}'s new collection, {\it1st Person Singular}, have a sort of fractal nature -- you're reading a story by a middle-aged Japanese man in which a middle-aged Japanese man is telling you a story (\& sometimes that story involves him telling other stories). You get drawn into the spiral, \& soon you're in that strange world where many of his stories exist, a place full of his favorite things (jazz, baseball, the Beatles, though surprisingly few cats this time) \& yet unmistakably odd, existing at a slight, unexplained angle to reality.'' -- {\sc Petra Mayer}, {\it NPR}
		\item ``{\sc Murakami}'s engrossing collection offers a crash course in his singular style \& vision, blending passion for music \& baseball \& nostalgia for youth with portrayals of young love \& moments of magical realism $\ldots$ {\sc Murakami}'s gift for evocative, opaque magical realism shines in ``Charlie Parker Plays Bossa Nova,'' in which a review of a fictional album breathes new life into the ghost of the jazz great, \& ``Confessions of a Shinagawa Monkey,'' wherein a talking monkey ruminates with a traveler on love \& belonging. {\sc Murakami} finds ample material in young love \& sex, showcased in ``On a Stone Pillow,'' in which a young man's brief tryst with a coworker, unremarkable in itself, takes on a degree of immortality after she mails him her poetry $\ldots$ These shimmering stories are testament to {\sc Murakami}'s talent \& enduring creativity.'' -- {\it Publishers Weekly (Starred Review)}
		\item ``Whether in his epic-scale novels or in his shorter works, much of {\sc Murakami}'s appeal has always come from the beguiling way in which his characters react to wildly fantastical events in the most matter-of-fact manner, ever ready to accept how the twists \& turns of everyday life can blend into more audacious alternate realities. In these 8 stories, we see that phenomenon most disarmingly in ``Confessions of a Shinagawa Monkey,'' in which a monkey strides into a sauna at a remote hotel \& asks the narrator if he would like to have his back scrubbed $\ldots$ The glue that holds together {\sc Murakami}'s blending realities -- in these stories \&, indeed, in all of his fiction -- is always the narrator's love for something (a woman, a song, a baseball team, a moment in the past) that is both life-giving \& deeply melancholic. Masterful short fiction.'' -- {\sc Bill Ott}, {\it Booklist (Starred Review)}
		\item ``You can't have a conversation about literary fiction of the past 50 years without mentioning {\sc Haruki Murakami}, \& {\it1st Person Singular} reminds us why $\ldots$ As 1 of the standard-bearers of contemporary magical realism, {\sc Murakami} has traveled deep into the hearts \& minds of both his characters \& his readers. In {\it1st Person Singular}, he offers 8 new stories, all told in 1st person -- hence the title -- as perhaps memoir, perhaps fiction. E.g., ``The Yakult Swallows Poetry Collection'' finds a baseball-loving writer named {\sc Haruki Murakami} musing on his favorite team \& the ties that bind us together. {\sc Murakami} is always blurring lines, \& here it's left up to the reader to decide what's real. By distorting reality, the author creates a special closeness to his audience, \& he acknowledges this relationship with intelligence \& grace.'' -- {\sc Eric Ponce}, {\it BookPage}
		\item ``A new collection of stories from the master of the strange, enigmatic twist of plot $\ldots$ Music is never far from a {\sc Murakami} yarn, though always with an unexpected turn: {\sc Charlie Parker} comes in a dream to tell 1 young man that death is pretty boring \& meaningless $\ldots$ {\sc Murakami}'s characters are typically flat of affect, protesting their ugliness \& ordinariness, \& puzzled or frightened by things as they are. But most are also philosophical even about those ordinary things, as is the narrator of that fine Beatles-tinged tale, who ponders why it is that pop songs are important \& informative in youth, when our lives are happiest $\ldots$ An essential addition to any {\sc Murakami} fan's library.'' -- {\it Kirkus (Starred Review)}
		\item ``The versatile \& prolific {\sc Murakami} collects 8 1st-person stories that affirm his obsessions -- American pop music \& magical realism, baseball \& sex -- yet break new literary ground. From a messy hookup to an imaginary {\sc Charlie Parker} album to a monkey masseur, the Japanese maestro taps the weirdness of the everyday, exposing conflicts that simmer within us all.'' -- {\sc Oprah Daily}
	\end{itemize}
	{\sf About the Author.} {\sc Haruki Murakami} was born in Kyoto in 1949 \& now lives near Tokyo. His work has been translated into $> 50$ languages, \& the most recent of his many international honors is the Hans Christian Andersen Literature Award, whose previous recipients include Karl Ove Knausgård, Isabel Allende, \& Salman Rushdie.
	
	\item \cite{Murakami_ngoi_1}. {\sc Haruki Murakami}. {\it First Person Singular: Stories -- Ngôi Thứ Nhất Số Ít}\hfill{\sf[done]}
	\begin{itemize}
		\item ``Thật khó hình dung cô gái nhỏ nhắn, gầy gò, nhợt nhạt đang ngồi trước mặt lại chính là cô gái đã ở trong vòng tay tôi đêm qua, hét lên những tiếng hân hoan nhục cảm dưới ánh trăng mùa đông rọi vào từ cửa sổ.'' -- \cite[Trên Gối Đá, p. 14]{Murakami_ngoi_1}
		\item ``Nhưng liệu con người ta có thể làm 1 vụ quấy rối kỳ công đến như vậy chỉ vì ghét không? Việc in bưu thiếp chắc chắc là tốn kém. Con người có thể xấu tính đến mức ấy ư? Tôi không nhớ mình đã làm gì để bị cô ghét. Nhưng đôi khi, ta chà đạp lên cảm xúc của người khác, xúc phạm danh dự hoặc khiến họ khó chịu mà ta không nhận ra.'' -- \cite[Kem, pp. 34--35]{Murakami_ngoi_1}
		\item ``Kỳ lạ thay, khi mở tập thơ, lướt qua những bài thơ được in bằng chữ màu đen cỡ lớn, sau đó đọc lên thành tiếng, cơ thể cô mà tôi đã nhìn thấy dưới ánh nắng vào buổi sáng hôm sau mà là cơ thể với làn da sáng mịn nằm trong vòng tay tôi dưới ánh trăng. Bầu vú tròn trịa, đầu vú nhỏ \& cứng, lông mu đen \& thưa, âm đạo ướt đẫm. Lúc lên đỉnh, cô nghiến chặt chiếc khăn, mắt nhắm nghiền, liên tục gọi da diết tên 1 người đàn ông bên tai tôi. 1 cái tên rất đỗi bình thường mà tôi không tài nào nhớ ra.'' -- \cite[Trên Gối Đá, p. 19]{Murakami_ngoi_1}
		\item ````Nghe này, hãy tưởng tượng bằng năng lực của mình. Vắt kiệt trí thông minh để hình dung ra. 1 hình tròn có nhiều tâm, đã vậy lại không có đường ngoại biên. Nỗ lực 1 cách nghiêm túc như thể phải trầy da tróc vẩy vậy, khi đó cậu mới dần nhận ra hình tròn đó trông như thế nào.''
		
		``Có vẻ khó nhỉ,'' tôi nói.
		
		``Hiển nhiên rồi,'' ông nói như vừa nhổ đi 1 vật cứng. ``Trên đời này chẳng có thứ gì giá trị mà lại có được dễ dàng cả.'' Rồi ông khẽ đằng hắng, rành rọt như thể ngắt câu xuống dòng. ``Nhưng, khi đạt được thứ khó khăn đó với nhiều thời gian \& công sức, thứ đó sẽ trở thành kem của cuộc đời.''
		
		$\ldots$ trong tiếng Pháp có câu `cr\`eme de la cr\`eme' $\ldots$
		
		``Kem của kem, nghĩa là thứ tốt nhất trong những thứ tốt nhất. Tinh hoa quan trọng nhất của cuộc đời $\ldots$ đó chính là `cr\`eme de la cr\`eme'. Cậu hiểu không? Còn lại thì toàn là những thứ nhạt nhẽo, vô giá trị'''' -- \cite[Kem, pp. 39--40]{Murakami_ngoi_1}
		
		``Có lúc, tôi cảm giác như mình gần như đã lý giải được chuyện đó, nhưng khi suy nghĩ sâu hơn, tôi lại không hiểu gì nữa. Chuyện đó cứ lặp đi lặp lại. Nhưng có lẽ, đó không phải là hình tròn với hình dạng cụ thể mà là hình tròn chỉ tồn tại trong tâm trí của con người. Tôi nghĩ vậy. Rằng khi chúng ta thật lòng yêu ai đó, cảm thông sâu sắc với điều gì đó, có lý tưởng về thế giới này, tìm thấy niềm tin (hoặc thứ giống như niềm tin), chúng ta sẽ hiểu \& chấp nhận sự tồn tại của hình tròn đó như lẽ đương nhiên. Tất nhiên đây chỉ là suy diễn mơ hồ của tôi.
		
		Đầu óc của cậu là để suy nghĩ những thứ khó. Biến điều không biết thành biết. Thứ đó sẽ trở thành kèm của cuộc đời. Còn lại thì toàn là những thứ nhạt nhẽo, vô giá trị.'' -- \cite[Kem, p. 44]{Murakami_ngoi_1}
		\item ``Trong suốt nhiều năm, tôi quên hẳn mình đã từng viết bài luận đó hồi sinh viên. Phần vì đời sống của tôi sau này hối hả hơn tôi tưởng, vả lại, bài phê bình âm nhạc giả tưởng đó rốt cuộc cũng chỉ là 1 trò đùa tắc trách vô ưu thời tuổi trẻ. Tuy nhiên, khoảng 15 năm sau, bài luận đó đã trở về với tôi theo 1 cách đầy bất ngờ. Hệt như cái boomerang ta quên bẵng mình đã ném đi bỗng nhiên bay về vào lúc ta không ngờ đến.'' -- \cite[Charlie Parker Plays Bossa Nova, p. 55]{Murakami_ngoi_1}
		\item ``Khi tiếp tục lang thang tản bộ sau đó, đột nhiên cảm giác hối hận dâng trào trong tôi. Lẽ ra tôi nên mua đĩa hát đó. Dù đó là đĩa hát giả mạo vô nghĩa, dù giá của nó quá đắt đi nữa thì vẫn nên mua. Coi như 1 món đồ kỷ niệm kỳ quặc trong cuộc đời nhiều ngã rẽ của tôi.'' -- \cite[Charlie Parker Plays Bossa Nova, p. 57]{Murakami_ngoi_1}
		\item ``Tôi chỉ có thể nói đó là thứ âm nhạc chạm được tới nơi sâu thẳm nhất trong tâm hồn. Thứ âm nhạc mà trước \& sau khi nghe, ta dường như thấy cấu tạo cơ thể mình đã khác đi đôi chút $\ldots$ Quả thật, trên thế giới có tồn tại thứ âm nhạc như vậy.'' -- \cite[Charlie Parker Plays Bossa Nova, p. 61]{Murakami_ngoi_1}
	\end{itemize}
	
	\item \cite{Rand_fountainhead}. Ayn Rand. {\it The Fountainhead -- Suối Nguồn}.\hfill{\sf[done]}
	
	\item \cite{Rosie_travel}. Rosie Nguyễn. {\it Ta Ba Lô Trên Đất Á}.\hfill{\sf[done]}
	
	\item \cite{Rosie_self_study}. Rosie Nguyễn. {\it Trên Hành Trình Tự Học}.\hfill{\sf[done]}
	
	\item \cite{Rosie_happy}. Rosie Nguyễn. {\it Mình Nói Gì Khi Nói Về Hạnh Phúc?}.\hfill{\sf[done]}
	
	\item \cite{Rosie_youth}. Rosie Nguyễn. {\it Tuổi Trẻ Đáng Giá Bao Nhiêu?}.\hfill{\sf[done]}
	
	\item \cite{Salinger_catcher_in_rye}. {\sc J. D. Salinger}. {\it The Catcher In The Rye}.
	
	\item \cite{Salinger_btdx}. {\sc J. D. Salinger}. {\it The Catcher In The Rye -- Bắt Trẻ Đồng Xanh}.\hfill{\sf[done]}
	
	\item \cite{Shapiro2014}. Dani Shapiro. {\it Still Writing: The Perils \& Pleasures of a Creative Life}.\hfill{\sf[reading]}
	
	\item \cite{Strunk_element_style}. William Strunk Jr. {\it The Elements of Style}.\hfill{\sf[done]}
	
	\item \cite{Strunk_White_element_style}. William Strunk Jr, E. B. White. {\it The Elements of Style}.\hfill{\sf[done]}
	
	\item \cite{Van_mat_troi_suoi_lanh}. Nguyễn Phương Văn. {\it Mặt Trời Trong Suối Lạnh}.\hfill{\sf[done]}
	
	\item \cite{VanVu2022}. Vũ Hà Văn. {\it Giáo Sư Phiêu Lưu Ký: Tản Mạn với 1 Nhà Toán Học}.\hfill{\sf[done]}
	
	\item \cite{Vasconcelos_orange_tree}. Jos\'e Mauro de Vasconcelos. {\it My Sweet Orange Tree -- Cây Cam Ngọt Của Tôi}.\hfill{\sf[done]}
	
	\item \cite{Wallace_water}. David Foster Wallace. {\it This Is Water: Some Thoughts, Delivered on a Significant Occasion, about Living a Compassionate Life}.\hfill{\sf[done]}
	
	\item \cite{Wallace_jest}. David Foster Wallace. {\it Infinite Jest}.\hfill{\sf[reading]}
	
	\item \cite{Wiest_101_essays}. {\sc Brianna Wiest}. {\it 101 Essays That Will Change The Way You Think}.
	
	\item \cite{Wiest_101_essays_VN}. {\sc Brianna Wiest}. {\it 101 Essays That Will Change The Way You Think -- Sống Khai Vấn, Sống Tỉnh Thức}.\hfill{\sf[done]}
	
	\item \cite{Zinsser2005}. William Zinsser. {\it Writing About Your Life: A Journey into the Past}.
	
	\item \cite{Zinsser2016}. William Zinsser. {\it On Writing Well: The Classic Guide to Writing Nonfiction}.\hfill{\sf[reading]}
\end{enumerate}

%------------------------------------------------------------------------------%

\subsection{Psychology Book}
Về các cuốn sách tâm lý, mình có nên liệt kê chúng theo thứ tự hay dần{\tt/}theo chiều tăng của sự tâm đắc cá nhân, riêng những cuốn đang mua chưa đọc sẽ tạm để ở cuối danh sách, sau khi đọc 1 phần{\tt/}xong đủ để đánh giá mức độ hay của những cuốn sách đó thì mình sẽ sắp thứ tự sau. Chỉ riêng sách Văn Học, Tâm Lý \& Triết Học mới được áp dụng cách liệt kê này, đặc biệt không áp dụng (được) cho các sách STEM -- đơn giản vì chúng hay theo nhiều lĩnh vực khác nhau, nên không thể nào sắp duy nhất 1 thứ tự trên 1 tập hợp bán thứ tự được (poset -- partial ordering set)?
\begin{enumerate}
	\item \cite{Adler_science_living}. Alfred Adler. {\it The Science of Living}.\hfill{\sf[done]}
	
	\item \cite{Adler_human_nature}. {\sc Alfred Adler}. {\it Understanding Human Nature}.\hfill{\sf[done]}
	
	\item \cite{Adler_human_nature_VN}. {\sc Alfred Adler}. {\it Understanding Human Nature -- Hiểu Về Bản Chất Con Người}.\hfill{\sf[done]}
		
	\item \cite{Ariely_reasonably_irrational}. Dan Ariely. {\it Phi Lý Trí Một Cách Hợp Lý: Câu Trả Lời Hài Hước Cho Những Hiện Tượng Tâm Lý Kỳ Quặc}.\hfill{\sf[done]}
	
	\item \cite{Ariely_predictably_irrational}. Dan Ariely. {\it Predictably Irrational: The Hidden Forces That Shape Our Decisions -- Phi Lý Trí: Khám Phá Những Động Lực Vô Hình Ẩn Sau Những Quyết Định Của Con Người}.\hfill{\sf[done]}
	
	\item \cite{Ariely_upside_rationality}. Dan Ariely. {\it The Upside of Irrationality: The Unexpected Benefits of Defying Logic -- Lẽ Phải Của Phi Lý Trí: Lợi Ích Bất Ngờ Của Việc Phá Bỏ Những Quy Tắc Logic Trong Công Việc \& Cuộc Sống}.\hfill{\sf[done]}
	
	\item \cite{Ariely_dishonesty}. Dan Ariely. {\it The Honest Truth About Dishonesty: How We Lie to Everyone--Especially Ourselves -- Bản Chất Của Dối Trá: Chúng Ta Đã Dối Gạt Mọi Người \& Chính Mình Như Thế Nào?}.\hfill{\sf[done]}
	
	\item \cite{Aron_HSP}. Elaine N. Aron. {\it The Highly Sensitive Person: How to Thrive When the World Overwhelms You}.\hfill{\sf[done]}
	
	\item \cite{Bancroft_why_he_do}. Lundy Bancroft. {\it Why Does He Do That?}.\hfill{\sf[reading]}
	\begin{quotation}
		{\it``1 of the basic human rights he takes away from you is the right to be angry with him.''}
		
		-- 1 trong những quyền cơ bản của con người mà hắn tước đi của bạn là quyền được tức giận với hắn.
		
		{\it``Abuse \& respect are diametric opposites: You do not respect someone whom you abuse, \& you do not abuse someone whom you respect.''}
		
		-- Lạm dụng \& tôn trọng hoàn toàn trái ngược nhau: Bạn không tôn trọng người mà bạn lạm dụng, \& bạn không lạm dụng người mà bạn tôn trọng.
		
		{\it``Abuse grows from attitudes \& values, not feelings. The roots are ownership, the trunk is entitlement, \& the branches are control.''}
		
		-- Sự lạm dụng phát triển từ thái độ \& giá trị chứ không phải cảm xúc. Gốc là quyền sở hữu, thân là quyền, \& nhánh là quyền kiểm soát.
		
		{\it``Their value system is unhealthy, not their psychology.''}
		
		-- Hệ thống giá trị của họ không lành mạnh, không phải tâm lý của họ.
	\end{quotation}
	
	\item \cite{Bancroft_why_he_do_VN}. Lundy Bancroft. {\it Why Does He Do That? -- Tại Sao Anh Ta Làm Thế? Giải Mã Tâm Lý Kẻ Bạo Hành}.\hfill{\sf[done]}
	
	\item \cite{Bon_crowd_psychology}. {\sc Gustave Le Bon}. {\it Psychology of Crowds}.
	
	\item \cite{Bon_crowd_psychology_VN}. {\sc Gustave Le Bon}. {\it Psychology of Crowds -- Tâm Lý Học Đám Đông}.\hfill{\sf[done]}
	
	\item \cite{Cain_quiet}. Susan Cain. {\it Quiet: The Power of Introverts in a World That Can't Stop Talking}.\hfill{\sf[reading]}
	
	\item \cite{Cain_quiet_VN}. Susan Cain. {\it Quiet: The Power of Introverts in a World That Can't Stop Talking -- Hướng Nội: Sức Mạnh của Sự Yên Lặng Trong 1 Thế Giới Nói Không Ngừng}.\hfill{\sf[done]}
	
	\item \cite{Cain_Mone_Moroz_quiet_power}. Susan Cain, Gregory Mone, Erica Moroz. {\it Quiet Power: The Secret Strengths of Introverted Kids}.\hfill{\sf[reading]}
	
	\item \cite{Cain_Mone_Moroz_quiet_power_VN}. Susan Cain, Gregory Mone, Erica Moroz. {\it Quiet Power: The Secret Strengths of Introverted Kids -- Trầm Lặng: Sức Mạnh Tiềm Ẩn Của Người Hướng Nội}.\hfill{\sf[done]}
	
	\item \cite{Carnegie2021}. Dale Carnegie. {\it How to Win Friends \& Influence People -- Đắc Nhân Tâm}.\hfill{\sf[done]}
	
	\item \cite{Chi2022}. Chi, Nguyễn (The Present Writer). {\it Một Cuốn Sách về Chủ Nghĩa Tối Giản}. \hfill{\sf[done]}
	
	\item \cite{Clear_habit}. James Clear. {\it Atomic Habits; An Easy \& Proven Way to Build Good Habits \& Break Bad Ones}.\hfill{\sf[reading]}
	
	\item \cite{Clear_habit_VN}. James Clear. {\it Atomic Habits; An Easy \& Proven Way to Build Good Habits \& Break Bad Ones -- Thay Đổi Tí Hon, Hiệu Quả Bất Ngờ: Tạo Thói Quen Tốt, Bỏ Thói Quen Xấu Bằng Phương Pháp Đơn Giản mà Hiệu Quả}.\hfill{\sf[done]}
	
	\item \cite{Duhigg_habit}. {\sc Charles Duhigg}. {\it The Power of Habit: Why We Do What We Do in Life \& Business}.
	
	Website: \url{https://www.charlesduhigg.com/the-power-of-habit}.
	
	\item \cite{Duhigg_habit_VN}. {\sc Charles Duhigg}. {\it The Power of Habit: Why We Do What We Do in Life \& Business -- Sức Mạnh Của Thói Quen}.\hfill{\sf[done]}
	
	\item \cite{Csikszentmihalyi_creativity}. {\sc Mihaly Csikszentmihalyi}. {\it Creativity: Flow \& the Psychology of Discovery \& Invention}.\hfill{\sf[reading]}
	
	\item \cite{Csikszentmihalyi_flow}. {\sc Mihaly Csikszentmihalyi}. {\it Flow: The Psychology of Optimal Experience}.\hfill{\sf[reading]}
	
	\item \cite{Csikszentmihalyi_flow_VN}. {\sc Mihaly Csikszentmihalyi}. {\it Flow: The Psychology of Optimal Experience -- Dòng Chảy: Tâm Lý Học Hiện Đại Trải Nghiệm Tối Ưu}.\hfill{\sf[done]}
	
	\item \cite{Dweck_mindset}. {\sc Carol S. Dweck}. {\it Mindset: The New Psychology of Success}. {\sf[21978 Amazon ratings][159678 Goodreads ratings]}
	
	{\sf Amazon review.} ``From the renowned psychologist who introduced the world to ``growth mindset'' comes this updated edition of the million-copy bestseller -- featuring transformative insights into redefining success, building lifelong resilience, \& supercharging self-improvement.
	
	``Through clever research studies \& engaging writing, {\sc Dweck} illuminates how our beliefs about our capabilities exert tremendous influence on how we learn \& which paths we take in life.'' -- {\sc Bill Gates}, {\it GatesNotes}
	
	``It's not always the people who start out he smartest who end up the smartest.''
	
	After decades of research, world-renowned Stanford University psychologist {\sc Carol S. Dweck}, PhD.D., discovered a simple but groundbreaking idea: the power of mindset. In this brilliant book, she shows how success in school, work, sports, the arts, \& almost every area of human endeavor can be dramatically influenced by how we think about our talents \& abilities. People with a {\it fixed mindset} -- those who believe that abilities are fixed -- are less likely to flourish than those with a {\it growth mindset} -- those who believe that abilities can be developed. {\it Mindset} reveals how great parents, teachers, managers, \& athletes can put this idea to use to foster outstanding accomplishment.
	
	In this edition, {\sc Dweck} offers new insights into her now famous \& broadly embraced concept. She introduces a phenomenon she calls false growth mindset \& guides people toward adopting a deeper, truer growth mindset. She also expands the mindset concept beyond the individual, applying it to the cultures of groups \& organizations. With the right mindset, you can motivate those you lead, teach, \& love -- to transform their lives \& your own. This is is about how your mindset can influence your success in various areas of life, \& how you can develop a growth mindset to achieve greater success.''
	\begin{itemize}
		\item ``Why waste time proving over \& over how great you are, when you could be getting better?''
		\item ``Believing that your qualities are carved in stone -- the fixed mindset -- creates an urgency to prove yourself over \& over.''
		\item ``People with the growth mindset know that it takes time for potential to flower.''
	\end{itemize}
	{\sf Editorial reviews.}
	\begin{itemize}
		\item ``A good book is one whose advice you believe. A great book is one whose advice you follow. This is a book that can change your life, as its ideas have changed mine.'' -- {\sc Robert J. Sternberg}, co-author of {\it Teaching for Wisdom, Intelligence, Creativity, \& Success}
		\item ``An essential read for parents, teachers [\&] coaches $\ldots$ as well as for those who would like to increase their own feelings of success \& fulfillment.'' -- {\it Library Journal} (starred review)
		\item ``Everyone should read this book.'' -- {\sc Chip Heath \& Dan Heath}, authors of {\it Made to Stick}
		\item ``1 of the most influential books ever about motivation.'' -- {\sc Po Bronson}, author of {\it NurtureShock}
		\item ``If you manage people or are a parent (which is a form of managing people), drop everything \& read {\it Mindset}.'' -- {\sc Guy Kawasaki}, author of {\it The Art of the Start 2.0}
	\end{itemize}
	{\sf About the Author.} {\sc Carol S. Dweck} , Ph.D., is widely regarded as one of the world's leading researchers in the fields of personality, social psychology, \& developmental psychology. She is the Lewis \& Virginia Eaton Professor of Psychology at Stanford University, has been elected to the American Academy of Arts \& Sciences \& the National Academy of Sciences, \& has won nine lifetime achievement awards for her research. She addressed the United Nations on the eve of their new global development plan \& has advised governments on educational \& economic policies. Her work has been featured in almost every major national publication, \& she has appeared on {\it Today, Good Morning America, \& 20{\tt/}20}. She lives with her husband in Palo Alto, California.
	
	\item \cite{Dweck_mindset_VN}. {\sc Carol S. Dweck}. {\it Mindset: The New Psychology of Success -- Tâm Lý Học Thành Công: Sức Mạnh Của Niềm Tin Phát Huy Tiềm Năng Của Chúng Ta Như Thế Nào}.\hfill{\sf[done]}
	
	\item {\sc Carol S. Dweck}. {\it Mindset-- Updated Edition: Changing The Way You think To Fulfil Your Potential}. {\sf[7726 Amazon ratings][159678 Goodreads ratings]}
	
	{\sf Amazon review.} ``World-renowned Stanford University psychologist {\sc Carol S. Dweck}, in decades of research on achievement \& success, has discovered a truly groundbreaking idea -- the power of our mindset.
	
	{\sc Dweck} explains why it's not just our abilities \& talent that bring us success -- but whether we approach them with a fixed or growth mindset. She makes clear why praising intelligence \& ability doesn't foster self-esteem \& lead to accomplishment, but may actually jeopardize success. With the right mindset, we can motivate our kids \& help them to raise their grades, as well as reach our own goals -- personal \& professional. {\sc Dweck} reveals what all great parents, teachers, CEOs, \& athletes already know: how a simple idea about the brain can create a love of learning \& a resilience that is the basis of great accomplishment in every area. This book is about how our mindset can affect our success in life, \& how we can change our mindset to achieve more.''
	\begin{itemize}
		\item ``Believing that your qualities are carved in stone -- the fixed mindset -- creates an urgency to prove yourself over \& over.''
		\item ``The fixed mindset makes you concerned with how you'll be judged; the growth mindset makes you concerned with improving.''
		\item ````Becoming is better than being.'' The fixed mindset does not allow people the luxury of becoming. They have to already be.''
	\end{itemize}
	
	\item \cite{DK2018}. DK. {\it How Psychology Works: The Facts Visually Explained (How Things Work)}.\hfill{\sf[reading]}
	
	\item \cite{Eun-Jung_hurt_VN}. Yoo Eun-Jung. {\it Không Ai Có Thể Làm Bạn Tổn Thương Trừ Khi Bạn Cho Phép}.\hfill{\sf[done]}
		
	\item \cite{Garcia_Miralles_ikigai}. {\sc H\'ector Garc\'ia, Francesc Miralles}. {\it Ikigai: The Japanese Secret To A Long And Happy Life by Francesc Miralles}. {\sf[60710 Amazon ratings][80344 Goodreads ratings]}
	
	{\sf Amazon review.} Los Angeles Times bestseller. ``If hygge\footnote{the quality of being warm \& comfortable that gives a feeling of happiness.} is the art of doing nothing, ikigai is the art of doing something -- \& doing it with supreme focus \& joy.'' -- New York Post Bring meaning \& joy to all your days with this internationally bestselling guide to the Japanese concept of ikigai (pronounced ee-key-guy) -- the happiness of always being busy -- as revealed by the daily habits of the world's longest-living people. \& from the same authors, don't miss The Book of Ichigo Ichie -- about making the most of every moment in your life.
	\begin{quotation}
		{\it``The grand essentials to happiness in this life are something to do, something to love, \& something to hope for.''}
		
		-- Những yếu tố thiết yếu nhất để có được hạnh phúc trong cuộc sống này là có việc gì đó để làm, có điều gì đó để yêu thương, có điều gì đó để hy vọng.
		
		{\it``Concentrating on 1 thing at a time may be the single most important factor in achieving flow.''}
		
		-- Tập trung vào 1 việc tại một thời điểm có thể là yếu tố quan trọng nhất để đạt được dòng chảy.
		
		{\it``We are what we repeatedly do. Excellence, then, is not an act but a habit.''}
		
		-- Chúng ta là những gì chúng ta làm đi làm lại nhiều lần. Vì vậy, sự xuất sắc không phải là một hành động mà là một thói quen.
		
		{\it````He who has a why to live for can bear with almost any how.''''}
		
		-- ''Người có lý do để sống có thể chịu đựng được hầu hết mọi việc.''
	\end{quotation}	
	
	\item \cite{Giang_death}. {\sc Đặng Hoàng Giang}. {\it Điểm Đến Của Cuộc Đời: Đồng Hành Với Người Cận Tử \& Những Bài Học Cho Cuộc Sống}.\hfill{\sf[done]}
	
	\item \cite{Giang_buc_xuc}. {\sc Đặng Hoàng Giang}. {\it Bức Xúc Không Làm Ta Vô Can}.\hfill{\sf[done]}
	
	\item \cite{Giang_smartphone}. {\sc Đặng Hoàng Giang}. {\it Thiện, Ác \& Smart Phone}.\hfill{\sf[done]}
	
	\item \cite{Giang_after_childhood}. {\sc Đặng Hoàng Giang}. {\it Tìm Mình Trong Thế Giới Hậu Tuổi Thơ}.\hfill{\sf[done]}
	
	\item \cite{Giang_dai_duong_den}. {\sc Đặng Hoàng Giang}. {\it Đại Dương Đen: Những Câu Chuyện Từ Thế Giới Của Trầm Cảm}.\hfill{\sf[done]}
	
	\item \cite{Gladwell2007}. Malcolm Gladwell. {\it Blink: The Power of Thinking Without Thinking}.\hfill{\sf[reading]}
	
	\item \cite{Gladwell_blink}. Malcolm Gladwell. {\it Blink: The Power of Thinking Without Thinking -- Trong Chớp Mắt: Sức Mạnh Của Việc Nghĩ Mà Không Cần Suy Nghĩ}.\hfill{\sf[done]}
	
	\item \cite{Gladwell2008}. Malcolm Gladwell. {\it Outliers: The Story of Success}.\hfill{\sf[reading]}
	
	\item \cite{Gladwell_outlier}. Malcolm Gladwell. {\it Outliers: The Story of Success -- Những Kẻ Xuất Chúng: Cái Nhìn Mới Lạ Về Nguồn Gốc Của Thành Công}.\hfill{\sf[done]}
	
	\item \cite{Gladwell2009}. Malcolm Gladwell. {\it What The Dog Saw: And Other Adventures}.\hfill{\sf[reading]}
	
	\item \cite{Gladwell_dog}. Malcolm Gladwell. {\it What The Dog Saw: And Other Adventures -- Chú Chó Nhìn Thấy Gì?: Lật Tẩy Những Góc Khuất Trong Cuộc Sống Xã Hội}.\hfill{\sf[done]}
	
	\item \cite{Gladwell2019}. Malcolm Gladwell. {\it Talking to Strangers: What We Should Know about the People We Don't Know}.\hfill{\sf[reading]}
	
	\item \cite{Gladwell_stranger}. Malcolm Gladwell. {\it Talking to Strangers: What We Should Know about the People We Don't Know -- Đọc Vị Người Lạ: Điều Ta Nên Biết Về Những Người Không Quen Biết}.\hfill{\sf[done]}
	
	\item \cite{Gladwell2021}. Malcolm Gladwell. {\it The Bomber Mafia: A Dream, a Temptation, \& the Longest Night of the 2nd World War}.\hfill{\sf[reading]}
	
	\item \cite{Gladwell_bomber_mafia}. Malcolm Gladwell. {\it The Bomber Mafia: A Dream, a Temptation, \& the Longest Night of the Second World War -- The Bomber Mafia: Giấc Mơ, Cám Dỗ, \& Đêm Dài Nhất Trong Thế Chiến II}.\hfill{\sf[done]}
	
	\item \cite{Gladwell2022}. Malcolm Gladwell. {\it The Tipping Point: How Little Things Can Make a Big Difference}.\hfill{\sf[reading]}
	
	\item \cite{Gladwell_tipping_point}. {\it The Tipping Point: How Little Things Can Make a Big Difference -- Điểm Bùng Phát: Làm Thế Nào Những Điều Nhỏ Bé Tạo Nên Sự Khác Biệt Lớn Lao?}.\hfill{\sf[done]}
	
	\item \cite{Goggins_hurt}. {\sc David Goggins}. {\it Can't Hurt Me: Master Your Mind \& Defy the Odds}. -- Không thể làm tổn thương tôi: Làm chủ tâm trí và thách thức số phận {\sf[97542 Amazon ratings][259533 Goodreads ratings]}
	
	{\sf Amazon review.} For {\sc David Goggins}, childhood was a nightmare -- poverty, prejudice, \& physical abuse colored his days \& haunted his nights. But through self-disciplined, mental toughness, \& hard work, {\sc Goggins} transformed himself from a depressed, overweight young man with no future into a U.S. Armed Forces icon \& 1 of the world's top endurance athletes. The only man in history to complete elite training as a Navy SEAL, Army Ranger, \& Air Force Tactical Air Controller, he went on to records in numerous endurance events, inspiring Outside magazine to name him ``The Fittest (Real) Man in America.''
	
	In {\it Can't Hurt Me}, he shares his astonishing life story \& reveals that most of us tap into only 40\% of our capabilities. {\sc Goggins} calls this The 40\% rule, \& his story illuminates a path that anyone can follow to push past pain, demolish fear, \& reach their full potential.
	\begin{quote}
		\item ``Everything in life is a mind game! Whenever we get swept under by life's dramas, large \& small, we are forgetting that no matter how bad the pain gets, no matter how harrowing the torture, all bad things end.''
		\item ``If you want to master the mind \& remove the governor, you'll have to become addicted to hard work. Because passion \& obsession, even talent, are only useful tools if you have the work ethic to back them up.''
		\item ``By the time I graduated, I knew that the confidence I'd managed to develop didn't come from a perfect family or God-given talent. It came from personal accountability which brought me self respect, \& self respect will always light a way forward.''
	\end{quote}
	{\sf Editorial reviews.}
	\begin{itemize}
		\item ``{\sc David Goggins} is a being of pure will \& inspiration. Just listening to this guy talk makes you want to run up a mountain. I firmly believe people like him can change the course of the world just by inspiring us to push harder \& dig deeper in everything we do. His goal to be `uncommon amongst uncommon people' is something we can all use to propel ourselves to fulfill our true potential. I'm a better man having met him.'' -- {\sc Joe Rogan}, Standup Comedian \& Host of the {\it Joe Rogan Experience} Podcast
		\item ``{\sc David Goggins} lives out every goal, every dream no matter what. PERIOD. He's unstoppable. There's no limit to him because he doesn't live in a comfort zone. His mental \& physical capacity are equal. {\sc Goggins} proves that your body can handle anything if you let your mind keep up. There's no way to stop something or someone that doesn't understand the concept of being beat.'' -- {\sc Marcus Luttrell}, retired Navy SEAL, Author of {\it New York Times} best seller {\it Lone Survivor}
		\item ``Modern neuroscience is teaching us that the path to courage \& success arrives through embracing pain \& fear, not by avoiding them. If ever there was a real-life example of this, it is the story of {\sc David Goggins}. In his unrelenting pursuit to self-conquer, {\sc Goggins} taught himself how to tap into that elusive holy grail of human existence: the ability to rewire one's own brain in order to continually do better \& actually become better, regardless of feelings, external conditions, or motivational state. {\it Can't Hurt Me} is the remarkable description of that journey \& the capacity to leverage \& better the mind. More importantly, it also teaches you how.'' -- {\sc Andrew D. Huberman}, PhD, Professor of Neurobiology, Stanford University School of Medicine
		\item ``{\sc David Goggins} throws the door open on pain, evil, darkness, the worst \& yes, the best of humanity, \& the strength of the human soul $\ldots$ \& that's just in Chap. 1. If you are looking for a book that will heal, stretch, inspire, \& dig into the corners of what it takes to persevere \& overcome in a messed-up world, this is your book.'' -- {\sc Taya Kyle}, Widow of American Sniper Chris Kyle, Author of {\it New York Times} Best Seller {\it American Wife}
		\item ``By the time you finish {\sc David Goggins}'s new book, you'll have kicked your victim mentality in the butt. Where you go from there is entirely up to you -- as {\sc Goggins} makes clear in this entertaining \& poignant memoir cum inspirational how-to. As the man with a hole in his heart tells you, there are no excuses in life, only reasons to try harder.'' -- {\sc Jim DeFelice}, Author of {\it American Sniper}
		\item ``{\sc David Goggins}'s book is not the 1st about overcoming severe hardships to achieve success, but it is certainly 1 of the most compelling. His story of beating the odds, of achieving athletic greatness, of serving his country \& his charities, \& of mastering his own destiny will inspire all of us to reach a little higher \& give a little more. `I will never quit' is a tenet of the Navy SEAL ethos, \& one that {\sc David Goggins} applies to everything he does.'' -- {\sc Admiral Eric Olson},  U.S. Navy (Retired); Former Commander, United States Special Operations Command; Chairman, Special Operations Warrior Foundation
		\item ``I'm inspired that people like this guy exist. Not everyone will live a life like {\sc David Goggins}, but he is proof that anyone could if given the right headspace within.'' -- {\sc Kelly Slater}, 11-Time World Champion Surfer
		\item ``Guaranteed to galvanize more than a few couch potatoes in action.'' -- {\it Kirkus Reviews}
	\end{itemize}
	{\sf About the Author.} {\sc David Goggins} is a retired Navy SEAL \& the only member of the U.S. Armed Forces ever to complete SEAL training, U.S. Army Ranger School, \& Air Force Tactical Air Controller training. {\sc Goggins} has competed in more than sixty ultra-marathons, triathlons, \& ultra-triathlons, setting new course records \& regularly placing in the top 5. A former Guinness World Record holder for completing 4,030 pull-ups in 17 hours, he's a much-sought-after public speaker who's shared his story with the staffs of Fortune 500 companies, professional sports teams, \& hundreds of thousands of students across the country.
	
	\item {\sc David Goggins}. {\it Never Finished: Unshackle Your Mind \& Win the War Within}. {\sf[11884 Amazon ratings][34397 Goodreads ratings]}
	
	{\sf Amazon review.} This is not a self-help book. It's a wake-up call! Can't Hurt Me, {\sc David Goggins}' smash hit memoir, demonstrated how much untapped ability we all have but was merely an introduction to the power of the mind. In Never Finished, {\sc Goggins} takes you inside his Mental Lab, where he developed the philosophy, psychology, \& strategies that enabled him to learn that what he thought was his limit was only his beginning \& that the quest for greatness is unending.
	
	The stories \& lessons in this raw, revealing, unflinching memoir offer the reader a blueprint they can use to climb from the bottom of the barrel into a whole new stratosphere that once seemed unattainable. Whether you feel off-course in life, are looking to maximize your potential or drain your soul to break through your so-called glass ceiling, this is the only book you will ever need.
	\begin{quote}
		\item ``I'm haunted by my future goals, not my past failures. I'm haunted by what I may still become. I'm haunted by my own continued thirst for evolution.''
		\item ``It's about constant effort, learning, \& adaptation, which demands unwavering discipline \& belief.''
		\item ``It's not an emotion to be shared or an intellectual concept, \& nobody else can give it to you. It must bubble up from within.''
	\end{quote}
	{\sf Editorial reviews.}
	\begin{itemize}
		\item ``There are levels to mental strength, \& the undisputed gold standard is my friend {\sc David Goggins}. The combination of the superhuman spectacle of his accomplishments \& the immense gravity of his words serves as 1 of the most potent motivational drugs that exist on God's green earth. He's a man who came from a humble \& troubled childhood \&, through the force of sheer will, forged himself into 1 of the hardest motherfucker that's ever lived. I believe there are people that are put here to elevate our expectations \& redefine what's possible for the rest of us, \& {\sc David Goggins} is the best example of that idea that I've ever come across in my life.'' -- {\sc Joe Rogan}
		\item ``{\sc David Goggins} is a rare breed of human being. His commitment to the philosophy of ``achieving your greatness by giving life all you got'' through mental toughness \& self discipline has been an anchor of inspiration \& motivation to millions around the world -- myself included. To me, what makes {\sc David} rare isn't his elite U.S. Navy SEAL career or his record-breaking endurance events as an athlete. What makes him rare is, from the day we connected, he's always been a real guy who shoots from the hip \& speaks from the heart. \& that's what matters most. Hardest workers in the room.'' -- {\sc Dwayne ``The Rock'' Johnson}
		\item ``I did not pick {\sc David} up when he fell, but he did pick me up every time I fell. {\sc David} has purposefully \& meticulously scrutinized every morbidly ugly aspect of our lives \& of his life with raw, brutal, \& oftentimes painful honesty. By sharing his life experiences, he has given all of us a blueprint on not only how to pick ourselves up but also how to excel in the face of overwhelming adversity \& ``stay hard'' in the process.'' -- {\sc Jacqueline Gardner, David}'s mom
	\end{itemize}
	
	\item \cite{Grant_give_take}. {\sc Adam Grant}. {\it Give \& Take: A Revolutionary Approach to Success}.\hfill{\sf[reading]}
	
	\item \cite{Grant2020}. Adam Grant. {\it Originals: How Non-Conformists Move the World -- Tư Duy Ngược Dịch Chuyển Thế Giới}.\hfill{\sf[done]}
	
	\item \cite{Grant_give_take_VN}. Adam Grant. {\it Give \& Take: Why Helping Others Drives Our Success -- Cho \& Nhận: Vì Sao Giúp Người Đưa Ta Đến Thành Công?}.\hfill{\sf[done]}
	
	\item \cite{Grant2022b}. Adam Grant. {\it Think Again: The Power of Knowing What You Don't Know -- Dám Nghĩ Lại: Sức Mạnh của Việc Biết Mình Không Biết}.\hfill{\sf[done]}
	
	\item \cite{Greene_laws_power}. {\sc Robert Greene}. {\it The 48 Laws of Power}. {\sf[80696 Amazon ratings][182673 Goodreads ratings]}
	
	{\sf Amazon review.} Amoral, cunning, ruthless, \& instructive, this multi-million-copy {\it New York Times} bestseller is the definitive manual for anyone interested in gaining, observing, or defending against ultimate control -- from the author of {\it The Laws of Human Nature}.
	
	In the book that {\it People} magazine proclaimed ``beguiling'' \& ``fascinating,'' {\sc Robert Greene} \& {\sc Joost Elffers} have distilled 3000 years of the history of power into 48 essential laws by drawing from the philosophies of {\sc Machiavelli, Sun Tzu}, \& {\sc Carl Von Clausewitz}, \& also from the lives of figures ranging from {\sc Henry Kissinger} to {\sc P.T. Barnum}.
	
	Some laws teach the need for prudence (``Law 1: Never Outshine the Master''), others teach the value of confidence (``Law 28: Enter Action with Boldness''), \& many recommend absolute self-preservation (``Law 15: Crush Your Enemy Totally''). Every law, though, has 1 thing in common: an interest in total domination. In a bold \& arresting 2-color package, {\it The 48 Laws of Power} is ideal whether your aim is conquest, self-defense, or simply to understand the rules of the game.
	\begin{quotation}
		{\it``Never take your position for granted \& never let any favors you receive go to your head.''}
		
		{\it``Never waste valuable time, or mental peace of mind, on the affairs of others -- that is too high a price to pay.''}
		
		{\it``When it comes to power, outshining the master is perhaps the worst mistake of all.''}
		
		{\it``Impatience, on the other hand, only makes you look weak. It is a principal impediment to power.''}
	\end{quotation}
	{\sf Editorial reviews.}
	\begin{itemize}
		\item ``{\sc Machiavelli} has a new rival. \& {\sc Sun Tzu} had better watch his back. {\sc Green} $\ldots$ has put together a checklist of ambitious behavior. Just reading the table of contents is enough to stir a little corner-office lust.'' -- {\it New York} magazine
		\item ``Beguiling $\ldots$ literate $\ldots$ fascinating. A wry primer for people who desperately want to be on top.'' -- {\it People} magazine
		\item ``An heir to {\sc Machiavelli}'s {\it Prince} $\ldots$ gentler souls will find this book frightening, those whose moral compass is oriented solely to power will have a perfect {\it vade mecum}.'' -- {\it Publishers Weekly}
		\item ``Satisfyingly dense $\ldots$ literary, with fantastic examples of genius power-game players. It's {\it The Rules} meets {\it In Pursuit of Wow!} with a degree in comparative literature.'' -- {\it Allure}
	\end{itemize}
	
	\item \cite{Greene_laws_power_VN}. {\sc Robert Greene}. {\it The 48 Laws of Power -- Nguyên Tắc Chủ Chốt Của Quyền Lực}.\hfill{\sf[done]}
	
	\item \cite{Greene_laws_human_nature}. {\sc Robert Greene}. {\it The Laws of Human Nature}. {\sf[15502 Amazon ratings][22993 Goodreads ratings]}
	
	{\sf Amazon review.} From the \#1 New York Times-bestselling author of {\it The 48 Laws of Power} comes the definitive new book on decoding the behavior of the people around you.{\sc Robert Greene} is a master guide for millions of readers, distilling ancient wisdom \& philosophy into essential texts for seekers of power, understanding \& mastery. Now he turns to the most important subject of all -- understanding people's drives \& motivations, even when they are unconscious of them themselves.
	
	We are social animals. Our very lives depend on our relationships with people. Knowing why people do what they do is the most important tool we can possess, without which our other talents can only take us so far. Drawing from the ideas \& examples of {\sc Pericles, Queen Elizabeth I, Martin Luther King Jr.}, \& many others, {\sc Greene} teaches us how to detach ourselves from our own emotions \& master self-control, how to develop the empathy that leads to insight, how to look behind people's masks, \& how to resist conformity to develop your singular sense of purpose. Whether at work, in relationships, or in shaping the world around you, {\it The Laws of Human Nature} offers brilliant tactics for success, self-improvement, \& self-defense.
	\begin{quotation}
		{\it``Your 1st impulse should always be to find the evidence that disconfirms your most cherished beliefs \& those of others. That is true science.''}
		
		-- Động lực đầu tiên của bạn phải luôn là tìm ra bằng chứng bác bỏ niềm tin ấp ủ nhất của bạn \& của người khác. Đó là khoa học đích thực.
		
		{\it``The truth is that we humans live on the surface, reacting emotionally to what people say \& do. We form opinions of others \& ourselves that are rather simplified. We settle for the easiest \& most convenient story to tell ourselves.''}
		
		-- Sự thật là con người chúng ta sống bề nổi, phản ứng theo cảm xúc với những gì mọi người nói \& làm. Chúng ta hình thành quan điểm của người khác \& chính mình khá đơn giản. Chúng ta chọn câu chuyện dễ dàng \& thuận tiện nhất để kể cho chính mình.
		
		{\it``1st, the laws will work to transform you into a calmer \& more strategic observer of people, helping to free you from all the emotional drama that needlessly drains you.''}
		
		-- Thứ nhất, luật pháp sẽ có tác dụng biến bạn thành một người quan sát mọi người bình tĩnh hơn \& có chiến lược hơn, giúp giải phóng bạn khỏi tất cả những bi kịch cảm xúc đang làm bạn kiệt sức một cách không cần thiết.
	\end{quotation}
	{\sf Editorial reviews.}
	\begin{itemize}
		\item ``The writing in engaging \& the ideas are fascinating $\ldots$ we could all use the insights {\sc Greene} provides $\ldots$ a hopeful book that advocates freedom \& creativity.'' -- {\it Quartz}
		\item ``The lessons have profound implications. There's a chapter on reading body language that is absolutely profound; each ``law'' has stunningly vivid descriptions of an historical figure.'' --Inc.
		\item ``{\it The Laws of Human Nature} provides some 1st-rate comprehensive \& in-depth information about how to deal with our fellow human beings effectively. {\sc Greene}'s intense curiosity about the inner workings of humanity is contagious, as he invites us to join him as fellow sleuths on his investigation of why people, including ourselves, do what we do. He rightly (\& frequently) reminds us that in order to understand others, we must 1st \& foremost understand what makes ourselves tick.'' -- New York Journal of Books
		\item ``In this detailed \& expansive guide, {\sc Green} ({\it Mastery}) seeks to $\ldots$ transform the reader into a `calmer \& more strategic observer', immune to `emotional drama.' Those are lofty promises, but even skeptics will become believers after diving into {\sc Greene}'s well-organized text. Overcoming the ``law of irrationality,'' e.g., leads to the ability to ``open your mind to what is really happening, as opposed to what you are feeling.'' {\sc Greene}'s thoughtful examination of self \& society will, for the committed reader, deliver a refreshing \& revitalizing perspective.'' -- {\it Publishers Weekly}
		\item ``{\sc Greene}'s specialty is analyzing the lives \& philosophies of historical figures like {\sc Sun Tzu} \& {\sc Napoleon}, \& extracting from them tips on how to manipulate people \& situations -- a cutthroat worldview that has earned him a devoted following among a like-minded readership of rappers, drug dealers \& corporate executives.'' -- {\it The New York Times}
		\item ``Compelling.'' -- Forbes
		\item ``Illuminating.'' -- The Guardian
	\end{itemize}
	{\sf About the Author.} {\sc Robert Greene} is the author of the New York Times bestsellers The 48 Laws of Power, The Art of Seduction, The 33 Strategies of War, \& The 50th Law. His highly anticipated fifth book, Mastery, examines the lives of great historical figures such as Charles Darwin, Mozart, Paul Graham \& Henry Ford \& distills the traits \& universal ingredients that made them masters. In addition to having a strong following within the business world \& a deep following in Washington, DC, Greene's books are hailed by everyone from war historians to the biggest musicians in the industry (including Jay-Z \& 50 Cent).
	
	Greene attended U.C. Berkeley \& the University of Wisconsin at Madison, where he received a degree in classical studies. He currently lives in Los Angeles.
	
	\item \cite{Greene_laws_human_nature_VN}. {\sc Robert Greene}. {\it The Laws of Human Nature -- Những Quy Luật Của Bản Chất Con Người}.\hfill{\sf[done]}
	
	\item {\sc Robert Greene}. {\it The Art of Seduction}.
	
	\item {\sc Robert Greene}. {\it Mastery}.
	
	\item {\sc Robert Greene}. {\it The Daily Laws: 366 Meditations on Power, Seduction, Mastery, Strategy, \& Human Nature}.
	
	\item {\sc Robert Greene}. {\it The 33 Strategies Of War}.
	
	\item {\sc Robert Greene}. {\it The 50th Law}.
	
	\item \cite{Hare1999}. {\sc Robert D. Hare}. {\it Without Conscience: The Disturbing World of the Psychopaths Among Us}.\hfill{\sf[reading]}
	
	\item \cite{Harper_unfuck_brain}. {\sc Faith G. Harper}. {\it Unfuck Your Brain: Getting Over Anxiety, Depression, Anger, Freak-Outs, \& Triggers with science (5-Minute Therapy)}.
	\begin{quotation}
		{\it``Emotions last longer than 90 seconds because we continue to fuel them with our thoughts. We do this by telling ourselves the same stories about the triggering situation over \& over. This is when they stop being emotions \& start becoming moods.''}
		
		-- Cảm xúc tồn tại lâu hơn 90 giây vì chúng ta tiếp tục tiếp thêm năng lượng cho chúng bằng suy nghĩ của mình. Chúng ta làm điều này bằng cách tự kể cho mình những câu chuyện tương tự về tình huống kích hoạt lặp đi lặp lại. Đây là lúc chúng không còn là cảm xúc \& bắt đầu trở thành tâm trạng.
		
		{\it``Taking care of {\sc ourselves} often becomes a luxury we can't afford, rather than a necessity we can't ignore.''}
		
		-- Việc chăm sóc {\sc bản thân mình} thường trở thành một điều xa xỉ mà chúng ta không thể mua được, hơn là một điều cần thiết mà chúng ta không thể bỏ qua.
		
		{\it``Most of the time, it takes about 3 months to reestablish equilibrium after a trauma. That is, after about $90$ days, our emotional sensors are no longer operating at hyper warp speed mode, \& return to normal.''}
		
		-- Thông thường, phải mất khoảng 3 tháng để thiết lập lại trạng thái cân bằng sau chấn thương. Tức là, sau khoảng $90$ ngày, các cảm biến cảm xúc của chúng ta không còn hoạt động ở chế độ siêu tốc độ nữa, \& trở lại bình thường.
	\end{quotation}
	{\sf Amazon review.} ``A no-nonsense \& helpful guide on how to cope with a slew of mental-health issues that are hellbent on ruining the lives of millions of people worldwide.
	
	Our brains do their best to help us out, bu every so often they can be real assholes -- having melt downs, getting addicted to things, or shutting down completely at the worst possible moments. Your brain knows it's not good to do these things, but it can't help it sometimes -- especially if it's obsessing about trauma it can't overcome. That's where this life-changing book comes in.
	
	With humor, patient, science, \& lots of good-ole swearing, Dr. Faith explains what's going on in your skull, \& talks you through the process of retraining your brain to respond appropriately to the non-emergencies of everyday life, \& to deal effectively with old, or newly acquired, traumas (particularly post-traumatic stress disorder.
	
	``As a passionate professor, counselor, \& follower of neuroscience research, I strongly recommend {\it Unfuck Your Brain: Using Science to Get Over Anxiety, Depression, Anger, Freak-outs, \& Triggers}. Dr {\sc Harper}'s writing style definitely held my attention \& made me laugh many times while still informing me about some complicated neuroscience \& health related topics. This book is a wonderful change from all the dry, dull, writing I usually read on a daily basis. I encourage everyone dealing with any of these issues or who is interested in becoming updated in the recent neuroscience research to purchase a copy \& start reading it today.'' -- {\sc Allen Novian}, PhD, LMFT, LPC-S, Adjunct Professor at St. Mary's University
	
	{\sf About the Author.} Dr. {\sc Faith G. Harper}, ACS, ACN, holds postdoctoral certifications in sexology \& applied clinical nutrition \& is trained in yoga, meditation, breathwork, mindful movement, \& all of those other forms of care that make most people avoid her at parties. In the past, she has worked in academia, community mental health, \& private practice as a licensed professional counselor. She maintains a connection with academia through he work with the Society of Indian Psychologists. She lives in San Antonio, TX, with her amazing friends \& family \& terrible rescue cats. She can be reached through her website, \url{faithgharper.com}.
	\item \cite{Harper_unfuck_anger}. {\sc Faith G. Harper}. {\it Unfuck Your Anger: Using Science to Understand Frustration, Rage, \& Forgiveness (5-Minute Therapy)}.
	
	{\sf Amazon review.} ``If you've ever been so pissed off that you did things that you regretted, or ruined your own day \& some other people's too, this book is for you. Or if you feel angry every single day \& it's affecting your health \& sleep \& love of life. Or if you've got very good reasons to be mad as hell, \& you're aren't going to take it anymore. Or if you've repressed your anger all your life \& now it's all coming out at once. Microcosm Publishing bestseller Dr. {\sc Faith Harper} explains here what the hell is going on in your brain \& how to retrain yourself to deal with enraging situations more productively \& without torpedoing your relationships. This is Your Brain on Anger gives you a heady dose of neuroscience \& cultural explanation of what anger is \& what it does to you, \& then gives you a handy 4-step checklist to help you deal with maddening situations after (or before) the fact, guidance on getting over things, \& a chapter on forgiveness. Your brain actually knows what it's doing, \& anger can be a good thing sometimes -- just not if it's ruining your life.''
	
	[Endorsement] ``Dr. {\sc Faith Harper} has done it again. Using the foulest of language, Faith has written a book that explains anger, gives you tools to turn down the volume on anger \& presents it in a way where you can actually read the whole fucking book (because it isn't boring, constipated\footnote{unable to get rid of waste material from the bowels easily, bị táo bón.} \& dry like most academic tomes). If you think you might be irritable, or if your wife says you are irritable, do yourself a big ass favor, buy this book, read it, put the tools in it into practice. You can thank me later -- when you are happier, more connected \& less fucking irritable!'' -- Dr. {\sc John Schinnerer}, creator of the Ultimate Anger Management Course, host of {\it The Evolved Caveman Podcast}, \& High Performance Coach at Guide to Self.
	\item \cite{Headlee_do_nothing}. {\sc Celeste Headlee}. {\it Do Nothing: How to Break Away from Overworking, Overdoing, \& Underliving}. {\sf[1321 Amazon ratings][8928 Goodreads ratings]}
	
	{\sf Amazon review.} ``A welcome antidotes to our toxic hustle culture of burnout.'' -- {\sc Arianna Huffington}
	
	``This book is so important \& could truly save lives.'' -- {\sc Elizabeth Gilbert}
	
	``A clarion call to work smarter [\&] accomplish more by doing less.'' -- {\sc Adam Grant}
	
	We work feverishly\footnote{1. in a way that shows strong feelings of excitement or worry, often with a lot of activity or quick movements; 2. in a way that is caused by a fever ($=$ a high temperature).} to make ourselves happy. So why are we so miserable?
	
	Despite our constant search for new ways to optimize our bodies \& minds for peak performance, human beings are working more instead of less, living harder not smarter, \& becoming more lonely \& anxious. We strive for the absolute best in every aspect of our lives, ignoring what we do well naturally \& reaching for a bar that keeps rising higher \& higher. Why do we measure our time in terms of efficiency instead of meaning? Why can't we just take a break?
	
	In {\it Do Nothing}, award-winning journalist {\sc Celeste Headlee} illuminates a new path ahead, seeking to institute a global shift in our thinking so we can stop sabotaging our well-being, put work aside, \& start living instead of doing. As it turns out, we're searching for external solutions to an internal problem. We won't find what we're searching for in punishing diets, productivity apps, or the latest self-improvement schemes. Yet all is not lost -- we just need to learn how to take time for ourselves, without agenda or profit, \& redefine what is truly worthwhile.
	
	Pulling together threads from history, neuroscience, social science, \& even paleontology (cổ sinh vật học). {\sc Headlee} examines long-held assumptions about time use, idleness, hard work, \& even our ultimate goals. Her research reveals that the habits we cling to are doing us harm; they developed recently in human history, which means they are habits that can, \& must, be broken. It's time to reverse the trend that's making us all sadder, sicker, \& less productive, \& return to a way of life that allows us to thrive.
	\begin{quotation}
		{\it``The key to well-being is shared humanity, even though we are pushing further \& further toward separation.''}
		
		-- Chìa khóa của hạnh phúc là sự chia sẻ của nhân loại, mặc dù chúng ta đang tiến xa hơn nữa đến sự chia ly.
		
		{\it``Far too many of us have been lured into the cult of efficiency. We are driven, but we long ago lost sight of what we were driving toward. We judge our days based on how efficient they are, not how fulfilling.''}
		
		-- Quá nhiều người trong chúng ta đã bị lôi cuốn vào sự sùng bái tính hiệu quả. Chúng ta bị thúc đẩy, nhưng từ lâu chúng ta đã đánh mất những gì chúng ta đang hướng tới. Chúng ta đánh giá một ngày của mình dựa trên mức độ hiệu quả của chúng chứ không phải mức độ thỏa mãn.
	\end{quotation}
	{\sf Editorial review.}
	\begin{itemize}
		\item ``A welcome antidote to our toxic hustle culture burnout.'' -- {\sc Arianna Huffington}, founder \& CEO of Thrive Global
		\item Do Less, Live More
		\item ``If you've ever felt compelled to work harder, this book is a clarion call to work smarter instead. Sometimes you accomplish more by doing less.'' -- {\sc Adam Grant}
		\item ``Through deep research \& evocative storytelling, {\sc Celeste Headlee} shows us how to break free from constant pressure \& live the life we truly want.'' -- {\sc Arianna Huffington}, founder \& CEO of Thrive Global
		\item ``Despite working harder than ever, people have never been more depressed, anxious, \& unhappy. Without a doubt, our modern way of life is not working. In fact, it's killing us. But what is to be done? With intelligence \& compassion, {\sc Headlee} presents realistic solutions for how we can reclaim our health \& our humanity from a technological revolution that seems hell-bent on destroying both. I'm so grateful to have read this book. It delivers on its promise of a better life.'' -- {\sc Elizabeth Gilbert}, author of {\it Big Magic} \& {\it Eat Pray Love}
		\item ``{\sc Celeste Headlee} makes a powerful case that productivity is not an inherent virtue -- if you're not careful, it can become a vice. If you've ever felt compelled to work harder, this book is a clarion call to work smarter instead. Sometimes you accomplish more by doing less.'' -- {\sc Adam Grant}, {\it New York Times} bestselling author of {\it Originals, Give \& Take}, \& host of the chart-topping TED postcast {\it WorkLife}
		\item ``At a time when so many people are feeling overworked, overwhelmed, \& addicted to busyness, work, \& ever-present technology, {\sc Celeste Headlee} offers a pathway out. Drawing on extensive research \& her own experience, {\it Do Nothing} is a powerful reminder that taking the time to stop, connect with others, \& forge real bonds is vital for building community, fostering empathy, \& ultimately leads to joy.'' -- {\sc Brigid Schulte}, author of the {\it New York Times} bestselling {\it Overwhelmed}, \& director of The Better Life Lab at New America
		\item ``I needed this book. \& chances are you need it, too. {\sc Celeste Headlee} does something amazing in {\it Do Nothing}. She battles this hectic, stressful time \& highlights the things that makes our lives better. Connection. Experience. Self-care. \&, above all, she reminds us to get busy living.'' -- {\sc Jared Yates Sexton}, author of {\it The Man They Wanted Me to Be}
		\item ``In this thought-provoking, well-researched book, {\sc Celeste} invites readers to push back against the I'm-too-busy narrative \& discover what it means to be truly successful.'' -- {\sc Laura Vanderkam}, author of {\it Off the Clock} \& {\it I Know How She Does It}
		\item ``This book is honest, heartbreaking, \& hopeful. It's that kind of gem that you read \& know you need to hear, know you need to embrace, even if it's challenging. Incredibly well-researched \& yet never preachy or dull, this book will help us all reclaim a bit of our humanness if we allow it.'' -- {\sc Nataly Kogan}, author of {\it Happier Now}
		\item ``[{\it Do Nothing}'s] conversational tone draws readers in, \& it will appeal to those looking beyond self-help to something more meaningful.'' -- {\it Booklist}
		\item ``This is neither a self-help book nor a how-to for people looking for a guide for different working habits. Rather, {\sc Headlee} systematically deconstructs the toxicity of hustle culture with historical \& scientific research to help readers question their habits \& impulses surrounding overwork.'' -- {\it Shelf Awareness}
	\end{itemize}
	{\sf About the Author.} {\sc Celeste Headlee} is an award-winning journalist \& professional speaker, \& is the bestselling author of {\it We Need To Talk: How to Have Conversations That Matter}. She is cohost of the new weekly series {\it Retro Report} on PBS \& season 3 of the {\it Scene on Radio} podcast -- {\it MEN}. {\sc Celeste} serves as an advisory board member for Procon \& the Listen 1st Project. In her 20-year career in public ratio, {\sc Celeste} has been the executive producer of {\it On 2nd Thought} at Georgia Public Radio \& has anchored programs including, {\it Tell Me More, Talk of the Nation, All Things Considered, \& Weekend Edition}. She also cohosted of the national morning news show {\it The Takeaway} for PRI \& WNYC, anchored World Channel's presidential coverage in 2012, \& received the 2019 Media Changemaker Award. {\sc Celeste} lives in Washington, DC.
	
	\item \cite{Headlee_do_nothing_VN}. {\sc Celeste Headlee}. {\it Do Nothing: How to Break Away from Overworking, Overdoing, \& Underliving -- Lười: 1 Lần Lười Bằng 10 Thang Thuốc Bổ}.
	
	\item \cite{Headlee_talk}. {\sc Celeste Headlee}. {\it We Need to Talk: How to Have Conversations That Matter}. {\sf[1115 Amazon ratings][4883 Goodreads ratings]}
	
	{\sf Amazon review.} ``WE NEED TO TALK.'' Now in paperback, public radio journalist {\sc Celeste Headlee}'s insightful \& urgent book on how to bridge what divides us -- by having real conversations. Based on the TED Talk with $> 10\cdot10^6$ views. NPR's Best Books of 2017.
	
	``{\it We Need to Talk} is an important read for a conversationally-challenged, disconnected age. {\sc Headlee} is a talented, honest storyteller, \& her advice has helped me become a better spouse, friend, \& mother.'' -- {\sc Jessica Lahey}, author of {\it New York Times} bestseller {\it The Gift of Failure}
	
	Today most of us communicate from behind electronic screens, \& studies show that Americans feel less connected \& more divided than ever before. The blame for some of this disconnect can be attributed to our political landscape, but the erosion of our conversational skills as a society lies with us as individuals.
	
	\& the only way forward, says {\sc Headlee}, is to start talking to each other. In {\it We Need to Talk}, she outlines the strategies that have made her a better conversationalist -- \& offers simple tools that can improve anyone's communication. E.g.:
	\begin{itemize}
		\item BE THERE OR GO ELSEWHERE. Human beings are incapable of multitasking, \& this is especially true of tasks that involve language. Think you can type up a few emails while on a business call, or hold a conversation with your child while texting your spouse? Think again.
		\item CHECK YOUR BIAS. The belief that your intelligence protects you more vulnerable to them. We are have blind spots that affect the way we view others. Check your bias before you judge someone else.
		\item HIDE YOUR PHONE. Don't just put down your phone, put it away. New research suggests that the mere presence of a cell phone can negatively impact the quality of a conversation.
	\end{itemize}
	Whether you're struggling to communicate with your kid's teacher at school, an employee at work, or the people you love the most -- {\sc Headlee} offers smart strategies that can help us all have conversations that matter.
	
	{\sf Editorial review.}
	\begin{itemize}
		\item ``Civil discourse is 1 of humanity's founding institutions \& it faces an existential threat: We, the people, need to talk about how we talk to one another. {\sc Celeste Headlee} shows us how.'' -- {\sc Ron Fournier}, {\it New York Times} bestselling author of {\it Love That Boy} \& Publisher of {\it Crain's} Detroit.
		\item ``{\it We Need To Talk} is an important read for a conversationally-challenged, disconnected age. {\sc Headlee} is a talented, honest storyteller, \& her advice has helped me become better spouse, friend, \& mother.'' -- {\sc Jessica Lahey}, author of {\it New York Times} bestseller {\it The Gift of Failure}
		\item ``This powerful debut offers 10 strategies for improving conversational skills. Tidbits (mẩu tin nhỏ) from sociological studies \& anecdotes from history, including from civil rights activist {\sc Xernona Clayton}'s groundbreaking conversations with KKK leader {\sc Calvin Craig}, round out a book that takes its own advice \& has much to communicate.'' -- {\it Publishers Weekly}
		\item ``In the course of her career, {\sc Headlee} has interviewed thousands of people from all walks of life \& learned that sparking a great conversation is really a matter of a few simple habits that anyone can learn.'' -- {\sc Jessica Stillman}, Inc.
		\item ``This book is necessary $\ldots$ {\sc Headlee}'s treatise on creating space for valuable mutual reciprocity is one that should become a handbook in any school, business or even a doctor's office where the everyday person visits.'' -- {\sc George Elerick}, {\it Buzzfeed}
		\item ``A well-researched \& careful analysis of how \& why we talk with one another -- our strengths \& (myriad) weakness $\ldots$ A thoughtful discussion \& sometimes-passionate plea for civility \& consideration in conversation.'' -- {\it Kirkus Reviews}
		\item ``Refreshing honest $\ldots$ In the era of the lost art of conversation, {\sc Headlee} helps us find our voice.'' -- {\sc Henry Bass}, Essence
		\item ``The perfect pre-Thanksgiving read to head off family squabbles \& turn the holiday meal into a feast of ideas instead of a political fracas.'' -- {\sc Karin illespie, Augusta Karin Gillespie, Augusta Chronicle}
	\end{itemize}
	
	\item \cite{Housel_money}. {\sc Morgan Housel}. {\it The Psychology of Money: Timeless lessons on wealth, greed, \& happiness}. {\sf[56424 Amazon ratings][221313 Goodreads ratings]}
	
	{\sf Amazon review.} Doing well with money isn't necessarily about what you know. It's about how you behave. \& behavior is hard to teach, even to really smart people.
	
	Money -- investing, personal finance, \& business decisions -- is typically taught as a math-based field, where data \& formulas tell us exactly what to do. But in the real world people don't make financial decisions on a spreadsheet. They make them at the dinner table, or in a meeting room, where personal history, your own unique view of the world, ego, pride, marketing, \& odd incentives are scrambled together.
	
	In {\it The Psychology of Money}, award-winning author {\sc Morgan Housel} shares 19 short stories exploring the strange ways people think about money \& teaches you how to make better sense of 1 of life's most important topics.
	\begin{quotation}
		{\it``There is no reason to risk what you have \& need for what you don't have \& don't need.''}
		
		-- Không có lý do gì để mạo hiểm những gì bạn có \& cần cho những gì bạn không có \& không cần.
		
		{\it``Having a strong sense of controlling one's life is a more dependable predictor of positive feelings of wellbeing than any of the objective conditions of life we have considered.''}
		
		-- Có ý thức mạnh mẽ về việc kiểm soát cuộc sống của mình là một yếu tố dự báo đáng tin cậy hơn về cảm giác hạnh phúc tích cực hơn bất kỳ điều kiện khách quan nào của cuộc sống mà chúng ta đã xem xét.
		
		{\it``Financial success is not a hard science. It's a soft skill, where how you behave is more important than what you know.''}
		
		-- Thành công về mặt tài chính không phải là một môn khoa học khó. Đó là một kỹ năng mềm, trong đó cách bạn cư xử quan trọng hơn những gì bạn biết.
		
		{\it``Therefore, focus less on specific individuals \& case studies \& more on broad patterns.''}
		
		-- Do đó, hãy tập trung ít hơn vào các cá nhân cụ thể \& nghiên cứu trường hợp \& nhiều hơn vào các mô hình chung.
	\end{quotation}
	{\sf Editorial review.}
	\begin{itemize}
		\item ``It's 1 of the best \& most original finance books in years.'' -- {\sc Jason Zweig}, {\it The Wall Street Journal}
		\item ``{\it The Psychology of Money} is bursting with interesting ideas \& practical takeaways. Quite simply, it is essential reading for anyone interested in being better with money. Everyone should own a copy.'' -- {\sc James Clear}, author, million-copy bestseller, {\it Atomic Habits}
		\item ``{\sc Morgan Housel} is that rare writer who can translate complex concepts into gripping, easy-to-digest narrative. {\it The Psychology of Money} is a fast-paced, engaging read that will leave you with both the knowledge to understand why we make bad financial decisions \& the tools to make better ones.'' -- {\sc Annie Duke}, author, {\it Thinking in Bets}
		\item ``{\sc Housel}'s observations often hit the daily double: they say things that haven't been said before, \& they make sense.'' -- {\sc Howard Marks}, Director \& Co-Chairman, Oaktree Capital \& Author, {\it The Most Important Thing} \& {\it Mastering the Market Cycle}
		\item ``{\sc Morgan Housel} is 1 of the brightest new lights among financial writers. He is accessible to everyone wanting to learn more about the psychology of money. I highly recommend this book.'' -- {\sc James P. O'Shaughnessy}, author, {\it What Works on Wall Street}
		\item ``Few people write about finance with the graceful clarity of {\sc Morgan Housel}. {\it The Psychology of Money} is an essential read for anyone who wants to make wiser decisions or live a richer life.'' -- {\sc Daniel H. Pink}, \#1 New York Times Bestselling Author of {\it When, To Sell Is Human}, \& {\it Drive} Review		
	\end{itemize}
	{\sf About the Author.} {\sc Morgan Housel} is a partner at The Collaborative Fund \& a former columnist at The Motley Fool \& The Wall Street Journal. He is a 2-time winner of the Best in Business Award from the Society of American Business Editors \& Writers, winner of the New York Times Sidney Award, \& a 2-time finalist for the Gerald Loeb Award for Distinguished Business \& Financial Journalism. He serves on the board of directors at Markel. He lives in Seattle with his wife \& 2 kids.
	
	\item \cite{Ichiro_Fumitake_disliked}. {\sc Kishimi Ichiro, Koga Fumitake}. {\it The Courage to Be Disliked: The Japanese Phenomenon That Shows You How to Change Your Life \& Achieve Real Happiness}. {\sf[17559 Amazon ratings][90979 Goodreads ratings]}
	
	{\sf Amazon review.} An international bestselleer \& TikTok sensation with $> 10^7$ copies sold worldwide, {\it The Courage to Be  Disliked} is a transformative \& practical guide to personal happiness \& self-fulfillment.
	
	Now you can unlock your full potential \& free yourself from the shackles of past traumas \& societal expectations to find true personal happiness. Based on the theories of renowned psychologist {\sc Alfred Adler}, this book guides you through the principles of self-forgiveness, self-care, \& mind decluttering in a straightforward, easy-to-digest style that's accessible to all.
	
	{\it The Courage to Be Disliked} unfolds as a dialogue between a philosopher \& a young man, who, over the course of 5 enriching conversations, realizes that each of us is in control of our life's direction, independent of past burdens \& expectations of others.
	
	Wise, empowering, \& profoundly liberating, this book is a life-changing experience that shows you a path to lasting happiness \& how to finally be the person you truly want to be. Millions are already benefiting from its teachings -- \& you can be next.
	\begin{quotation}
		{\it``Your life is not something that someone gives you, but something you choose yourself, \& you are the one who decides how you live.''}
		
		-- Cuộc sống của bạn không phải là thứ ai đó trao cho bạn mà là thứ bạn tự chọn, \& bạn là người quyết định cách mình sống.
		
		{\it```No matter what has occurred in your life up to this point, it should have no bearing at all on how you live from now on.' That you, living in here \& now, are the one who determines your own life.''}
		
		-- Cho dù điều gì đã xảy ra trong cuộc sống của bạn cho đến thời điểm này, nó sẽ không ảnh hưởng gì đến cách bạn sống kể từ bây giờ.' Rằng bạn, đang sống ở đây \& bây giờ, chính là người quyết định cuộc sống của chính mình.
		
		{\it``You were so afraid of interpersonal relationships that you came to dislike yourself. You've avoided interpersonal relationships by disliking yourself.''}
		
		-- Bạn sợ các mối quan hệ giữa các cá nhân đến mức bạn không thích chính mình. Bạn đã tránh xa các mối quan hệ giữa các cá nhân bằng cách không thích chính mình.
		
		{\it``The important thing is not what one is born with, but what use one makes of that equipment.''}
		
		-- Điều quan trọng không phải là người ta sinh ra đã có tài năng gì mà là người ta sử dụng thiết bị đó vào việc gì.
	\end{quotation}
	{\sf Editorial reviews.}
	\begin{itemize}
		\item ``{\sc Marie Kondo}, but for your brain.'' -- {\it Hello Giggles}
		\item ``Adlerian psychology meets Stoic philosophy in Socratic dialogue. Compelling from front to back. Highly recommended.'' -- {\sc Marc Andreessen}, venture capitalist \& founder of Andreessen Horowitz
		\item ``A nuanced discussion of a complex theory, with moments of real philosophical insights $\ldots$ [It's] refreshing \& useful to read a philosophy that goes against many contemporary orthodoxies. More than a century since {\sc Adler} founded his school of psychology, there's still insight \& novelty in his theories.'' -- {\it Quartzy}
		\item ``[{\it The Courage to be Disliked} guides] readers toward achieving happiness \& lasting change $\ldots$ For those seeking a discourse that helps explain who they are in the world, {\sc Kishimi, Koga} provide an illuminating conversation.'' -- {\it Library Journal}
	\end{itemize}
	{\sf About the Author.} {\sc Ichiro Kishimi} was born in Kyoto, where he currently resides. He writes \& lectures on Adlerian psychology \& provides counseling for youths in psychiatric clinics as a certified counselor \& consultant for the Japanese Society of Adlerian Psychology. He is the translator, into Japanese, of selected writings by {\sc Alfred Adler} -- {\it The Science of Living} \& {\it Problems of Neurosis} -- \& he is the author of {\it Introduction to Adlerian Psychology}, in addition to numerous other books.
	
	{\sc Fumitake Koga} is an award-winning professional writer \& author. He was released numerous bestselling works of business-related \& general non-fiction. He encountered Adlerian psychology in his late 20s \& was deeply affected by its conventional wisdom -- defying ideas. Thereafter, {\sc Koga} made numerous visits to {\sc Ichiro Kishimi} in Kyoto, gleaned from him the essence of Adlerian psychology, \& took down the notes for the classical ``dialogue format'' method of Greek philosophy that is used in this book.
	
	\item \cite{Ichiro_Fumitake_disliked_VN}. {\sc Kishimi Ichiro, Koga Fumitake}. {\it The Courage to Be Disliked: The Japanese Phenomenon That Shows You How to Change Your Life \& Achieve Real Happiness -- Dám Bị Ghét}.\hfill{\sf[done]}
	
	\item \cite{Ichiro_Fumitake_happy}. {\sc Kishimi Ichiro, Koga Fumitake}. {\it The Courage to Be Happy: Discover the Power of Positive Psychology \& Choose Happiness Every Day}. {\sf[2357 Amazon ratings][7732 Goodreads ratings]}
	
	{\sf Amazon review.} In this follow-up to the international bestseller \& TikTok sensation {\it The Courage to Be Disliked}, discover how to reconnect with your true self, experience true happiness, \& live the life you want.
	
	{\it What if 1 simple choice could unlock your destiny?}
	
	Already a major international bestseller, this eye-opening \& accessible follow-up to the ``compelling'' ({\sc Marc Andreessen}) worldwide phenomenon {\it The Courage to Be Disliked} shares the powerful teachings of {\sc Alfred Adler}, 1 of the giants of 19th-century psychology, through another illuminating dialogue between the philosopher \& the young man.
	
	3 years after their 1st conversation, the young man finds himself disillusioned \& disappointed, convinced {\sc Adler}'s teachings only work in theory, not in practice. But through further discussions between the philosopher \& the young man, they deepen their own understandings of {\sc Adler}'s powerful teachings, \& learn the tools needed to apply {\sc Adler}'s teachings to the chaos of everyday life.
	
	To be read on its own or as a companion to the bestselling 1st book, {\it The Courage to Be Happy} reveals a bold new way of thinking \& living, empowering you to let go of the shackles of past trauma \& th expectations of others, \& to use this freedom to create the life you truly desire.
	
	Plainspoken yet profoundly moving, reading {\it The Courage to Be Happy} will light a torch with the power to illuminate your life \& brighten the world as we know it. Now you can discover the courage to choose happiness.
	\begin{quotation}
		{\it``With regard to memory, think of it like this: from the innumerable events that have happened in a person's past, that person chooses only those events that are compatible with the present goals, gives meaning to them \& turns them into memories. \& conversely, events that run counter to the present goals are erased.''}
		
		-- Về trí nhớ, hãy nghĩ như thế này: trong vô số sự kiện đã xảy ra trong quá khứ của một người, người đó chỉ chọn những sự kiện phù hợp với mục tiêu hiện tại, gán cho chúng ý nghĩa \& biến chúng thành ký ức. \& ngược lại, những sự kiện đi ngược lại mục tiêu hiện tại sẽ bị xóa.
		
		{\it``Was one bitten by a dog? Or was one helped by another person? The reason Adlerian psychology is considered a `psychology of use' is this aspect of `being able to choose one's own life'. The past does not decide `now'. It is your `now' that decides the past.''}
		
		-- Có bị chó cắn không? Hay một người đã được người khác giúp đỡ? Lý do tâm lý học Adlerian được coi là ``tâm lý học sử dụng'' là do khía cạnh ``có thể lựa chọn cuộc sống của chính mình''. Quá khứ không quyết định ‘bây giờ'. Chính cái ‘bây giờ' của bạn mới quyết định quá khứ.
		
		{\it``The 2nd stage of problem behavior is `attention drawing'.''}
		
		-- Giai đoạn thứ 2 của hành vi có vấn đề là 'thu hút sự chú ý'.
	\end{quotation}
	{\sf Author's Note.}  ``{\sc Alfred Adler}, the thinker who was 100 years ahead of his time. Though he stands beside {\sf Sigmund Freud} \& {\sc Carl Gustav Jung} as 1 of the most important figures in the world of psychology, {\sc Adler} was for many years a ``forgotten giant.'' Since the publication of {\it The Courage to Be Disliked}, the context of {\sc Adler} \& his school of thought has gone through a remarkable transformation. {\sc Adler} has been widely known in Europe \& America for some time. But now, after our book spent a record-setting 51 weeks as a \#1 bestsheller -- having sold milliosn of copies in South Korea \& Japan -- I have a strong sense that {\sc Adler} is present within many people, \& no longer needs introduction. There is something deeply moving about his ideals being accepted in Asia after the passage of 100 years.
	
	{\it The Courage to Be Disliked} is a kind of map for informing people of the existence of Adlerian psychology, \& for giving an overview of {\sc Adler}'s ideas. It is a grand map that we put together over the course of several years, with the intention of creating a definitive introduction to Adlerian psychology.
	
	{\it The Courage to Be Happy}, once more we find the philosopher engaged in a dialogue with the pessimistic youth. 3 years after the conclusion of {\it The Courage to Be Disliked}, the youth, who has become a teacher with the intention of putting {\sc Adler}'s ideas into practice, calls on the philosopher 1 last time. Frustrated with Adlerian psychology \& angry with the philosopher for introducing him to {\sc Adler}'s ideals, the youth has returned to the philosopher's study to challenge everything the philosopher taught him \& insist that he cease to corrupt other young minds with ideals that don't hold up in the real world when interacting with real people. Calmly, the philosopher invites the youth to join him for 1 final conversation about having courage not only to take the 1st step toward happiness, but to continue walking along the path of self-improvement in order to love, be self-reliant, \& nurture community feeling.
	
	In what way can we make concrete progress on the path to happiness shown in the preceding volume, {\it The Courage to Be Disliked?} How can we put Adlerian psychology into practice in our everyday lives? \& what is that conclusion arrived at by {\sc Adler}, ``the biggest choice in life,'' that everyone must make in order to live in happiness?
	
	The curtain opens once more on this strong-medicine philosophical dialogue. Do you have the {\it courage} to climb the stairway of understanding with the youth?
		
	\item \cite{Ichiro_Fumitake_happy_VN}. Kishimi Ichiro, Koga Fumitake. {\it The Courage to Be Happy: Discover the Power of Positive Psychology \& Choose Happiness Every Day -- Dám Hạnh Phúc}.\hfill{\sf[done]}
	
	\item \cite{Jung_man_symbol}. {\sc Carl Gustav Jung}. {\it Man \& His Symbols}. [3809 Amazon ratings]. [30082 Goodreads ratings]
	
	{\sf Amazon review.} Explores {\sc Jung}'s psychological concepts regarding the nature, function, \& importance of man's symbols as they appear on both the conscious \& subconscious level.
	
	{\sf About the Author.} {\sc Carl Gustav Jung} (1875--1961) was a Swiss psychiatrist, an influential thinker \& the founder of analytical psychology (also known as Jungian psychology). Jung's radical approach to psychology has been influential in the field of depth psychology \& in counter-cultural movements across the globe. {\sc Jung}'s radical approach to psychology has been influential in the field of depth psychology \& in counter-cultural movements across the globe. {\sc Jung} is considered the 1st modern psychologist to state that the human psyche is ``by nature religious'' \& to explore it in depth. His major works include ``Analytic Psychology: Its Theory \& Practice'', ``Man \& His Symbols,'' ``Memories, Dreams, Reflections'', ``The Collected Works of Carl G. Jung'', \& ``The Red Book''.
	\item \cite{Jung_man_symbol_VN}. {\sc Carl Gustav Jung}. {\it Man \& His Symbols -- Con Người \& Biểu Tượng: Sự Thông Đạt Từ Những Biểu Tượng Trong Giấc Mơ}.\hfill{\sf[done]}
	\item {\sc Carl Gustav Jung}. {\it Memories, Dreams, Reflections: An Autobiography}.
	\item {\sc Carl Gustav Jung}. {\it Modern Man in Search of a Soul}. {\sf[2875 Amazon ratings][12714 Goodreads ratings]} Recommended by {\sc Jordan B. Peterson}.
	
	{\sf Amazon review.} 2017 Reprint of 1933 U.S. Edition. Full facsimile of the original edition, not reproduced with Optical Recognition software. Considered by many to be 1 of the most important books in the field of psychology, {\it Modern Man in Search of a Soul} is a comprehensive introduction to the thought of {\sc Carl Gustav Jung}. The writing covers a broad array of subjects e.g. gnosticism, theosophy, Eastern philosophy \& spirituality in general. The 1st part of the book deals with dream analysis in its practical application, the problems \& aims of modern psychotherapy, \& also his own theory of psychological types. The middle section addresses {\sc Jung}'s beliefs about the stages of life \& Archaic man. He also contrasts his own theories with those of {\sc Sigmund Freud}. In the latter parts of the book {\sc Jung} discusses psychology \& literature \& devotes a chapter to the basic postulates of analytical psychology. The last 2 chapters are devoted to the spiritual problem of modern man in aftermath of World War I. He compares it to the flowering of gnosticism in the 2nd century \& investigates how psychotherapists are like the clergy.
	
	\item {\sc Carl Gustav Jung}. {\it The Collected Works of C. G. Jung, Volume 9 (Part 1): Archetypes \& the Collective Unconscious}.
	
	\item {\sc Carl Gustav Jung}. {\it Synchronicity: An Acausal Connecting Principle}.
	
	\item {\sc Carl Gustav Jung}. {\it Psychology of the Unconscious: A Study of the Transformations \& Symbolisms of the Libido (The Collected Works of C. G. Jung -- Supplements)}.
	
	\item {\sc Carl Gustav Jung}. {\it The Undiscovered Self: With Symbols \& the Interpretation of Dreams (Jung Extracts Book 31)}.
	
	\item \cite{Kahneman2022}. Daniel Kahneman. {\it Thinking, Fast \& Slow -- Tư Duy Nhanh \& Chậm: Nên Hay Không Nên Tin Vào Trực Giác?}.\hfill{\sf[done]}
	
	\item \cite{Kahnweiler2022}. Jennifer B. Kahnweiler. {\it Quiet Influence -- Sức Mạnh của Sự Trầm Lắng -- The Introvert's Guide to Making a Difference}.\hfill{\sf[done]}
	
	\item \cite{Kushner_bad_things_good_people}. {\sc Harold S. Kushner}. {\it When Bad Things Happen to Good People}. {\sf[4711 Amazon ratings][19007 Goodreads ratings]}
	
	{\sf Amazon review.} The \#1 bestselling inspirational classic from the nationally known spiritual leader; a source of solace \& hope for over 4 million readers.
	
	When {\sc Harold Kushner}'s 3-year-old son was diagnosed with a degenerative disease that meant the boy would only live until his early teens, he was faced with 1 of life's most difficult questions: Why, God? Years later, {\sc Rabi Kushner} wrote this straightforward, elegant contemplation of the doubts \& fears that arise when tragedy strikes. In these pages, {\sc Kushner} shares his wisdom as a rabbi, a parent, a reader, \& a human being. Often imitated but never superseded, {\it When Bad Things Happen to Good People} is a classic that offers clear thinking \& consolation\footnote{sự an ủi.} in times of sorrow.
	
	{\sf Editorial reviews.}
	\begin{itemize}
		\item ``Whether religious or not, this book will speak because it touches -- profoundly, but simply -- on questions no parent \& no person can avoid.'' -- {\sc Harvey Cox}, Harvard Divinity School
		\item ``{\it When Bad Things Happen to Good People} offers a moving \& humane approach to understanding life's windstorms.'' -- {\sc Elisabeth KŸbler-Ross}
		\item ``A touching, heartwarming book for those of us who must content with suffering, \& that, of course, is all of us.'' -- {\sc Andrew M. Greeley}
		\item ``This is a book all humanity needs. It will help you understand the painful vicissitudes of this life \& enable you to stand up to them creatively.'' -- {\sc Norman Vincent Peale}
	\end{itemize}
	{\sf About the Author.} {\sc Harold S. Kushner} is rabbi laureate of Temple Israel in Natick, Massachusetts, having long served that congregation. He is best known as the author of {\it When Bad Things Happen to Good People}.
		
	\item \cite{Lembke_dopamine}. {\sc Anna Lembke}. {\it Dopamine Nation: Finding Balance in the Age of Indulgence}.\hfill{\sf[reading]}
	\begin{quotation}
		{\it``This book is about pleasure. It's also about pain. Most important, it's about how to find the delicate balance between the two, \& why now more than ever finding balance is essential.''}
		
		-- ``Cuốn sách này nói về lạc thú. Nó cũng nói về nỗi đau. Nhưng trên hết, nó nói về mối quan hệ giữa lạc thú \& nỗi đau, cũng như tầm quan trọng của việc hiểu được mối quan hệ đó để sống 1 cuộc đời đúng nghĩa.'' -- \cite[p. 9]{Lembke_dopamine_VN}
		
		{\it``The paradox is that hedonism, the pursuit of pleasure for its own sake, leads to anhedonia, which is the inability to enjoy pleasure of any kind.''}
		
		-- Điều nghịch lý là chủ nghĩa khoái lạc, việc theo đuổi thú vui vì lợi ích riêng của nó, dẫn đến anhedonia, tức là không có khả năng tận hưởng bất kỳ loại khoái cảm nào.
		
		{\it``The reason we're all so miserable may be because we're working so hard to avoid being miserable.''}
		
		-- Lý do khiến tất cả chúng ta đau khổ đến vậy có thể là vì chúng ta đang cố gắng quá nhiều để tránh bị đau khổ.
		
		{\it``Dopamine may play a bigger role in the motivation to get a reward than the pleasure of the reward itself. Wanting more than liking.''}
		
		-- Dopamine có thể đóng một vai trò lớn hơn trong việc tạo động lực để nhận được phần thưởng hơn là niềm vui khi nhận được phần thưởng đó. Muốn nhiều hơn là thích.
	\end{quotation}
	{\sf Editorial Reviews.}
	\begin{itemize}
		\item ``{\sc Anna Lembke} deeply understands an experience I hear about often in the therapy room at the nexus between our modern addictions \& our primal brains. Her stories of guiding people to find a healthy balance between pleasure \& pain have the power to transform your life.'' -- {\sc Lori Gottlieb}, ``Dear Therapist'' columnist at {\it The Atlantic, New York Times} bestselling author of {\it Maybe You Should Talk to Someone}
		
		-- {\sc Anna Lembke} hiểu sâu sắc về trải nghiệm mà tôi thường nghe thấy trong phòng trị liệu ở mối liên hệ giữa những cơn nghiện hiện đại \& bộ não nguyên thủy của chúng ta. Những câu chuyện hướng dẫn mọi người tìm ra sự cân bằng lành mạnh giữa niềm vui \& nỗi đau của cô có sức mạnh thay đổi cuộc sống của bạn.
		\item ``Just when you thought you knew all you needed to know about the addiction crisis, along comes Dr. {\sc Anna Lembke} with her 2nd brilliant book on the topic -- this one not about a drug but about the most powerful chemical of all: the dopamine that rules the pain \& pleasure centers of our minds. In an era of overconsumption \& instant gratification, {\it Dopamine Nation} explains the personal \& societal price of being ruled by the next fix -- \& how to manage it. No matter what you might find yourself over-indulging in -- from the Internet to food to work to sex -- you'll find this book riveting, scary, cogent, \& cleverly argued. Lembke weaves patient stories with research, in a voice that's as empathetic as it is clear-eyed.'' -- {\sc Beth Macy}, author of {\it Washington Post} Best Book of the Year, {\it New York Times} Notable Book of 2018 \& bestseller {\it Dopesick: Dealers, Doctors, \& the Drug Company That Addicted America}
		
		-- Ngay khi bạn nghĩ rằng bạn đã biết tất cả những gì cần biết về cuộc khủng hoảng nghiện ngập thì Tiến sĩ {\sc Anna Lembke} xuất hiện với cuốn sách xuất sắc thứ 2 về chủ đề này -- cuốn sách này không phải về ma túy mà về loại hóa chất mạnh nhất: dopamine điều khiển các trung tâm đau đớn \& khoái cảm trong tâm trí chúng ta. Trong thời đại tiêu thụ quá mức \& sự hài lòng ngay lập tức, {\it Dopamine Nation} giải thích cái giá cá nhân \& xã hội của việc bị chi phối bởi giải pháp tiếp theo -- \& cách quản lý nó. Bất kể bạn có thấy mình quá đam mê điều gì - từ Internet, đồ ăn, công việc đến tình dục - bạn sẽ thấy cuốn sách này hấp dẫn, đáng sợ, có sức thuyết phục, \& được lập luận một cách khéo léo. Lembke dệt nên những câu chuyện của bệnh nhân bằng nghiên cứu, bằng một giọng nói vừa đồng cảm vừa trong sáng.
		\item ``We all desire a break from our routines \& those parts of life that upset us. What if, instead of trying to escape these things, we learn to turn toward them, to reach a peaceful harmony with our selves \& the people we share our lives with? Lembke has written a book that radically changes the way we think about mental illness, pleasure, pain, reward, \& stress. Turn about it. You'll be happy you did.'' -- {\sc Daniel Levitin}, {\it New York Times} bestselling author of {\it The Organized Mind} \& {\it Successful Aging}
		
		-- Tất cả chúng ta đều mong muốn được thoát khỏi những thói quen thường ngày của mình \& những phần cuộc sống khiến chúng ta khó chịu. Điều gì sẽ xảy ra nếu thay vì cố gắng trốn tránh những điều này, chúng ta học cách hướng về phía chúng, đạt được sự hòa hợp yên bình với bản thân \& những người mà chúng ta chia sẻ cuộc sống? Lembke đã viết một cuốn sách làm thay đổi hoàn toàn cách chúng ta nghĩ về bệnh tâm thần, niềm vui, nỗi đau, phần thưởng, \& căng thẳng. Xoay quanh nó. Bạn sẽ rất vui vì bạn đã làm.
		\item ``Explore[s] the dichotomy between seeking a readily accessible hit of dopamine -- from our phones, gambling, or a bag of Fritos -- \& maintaining healthy, productive, stable lives.'' -- {\it The New York Times}, Inside the Bestseller List
		
		-- Khám phá [các] sự phân đôi giữa việc tìm kiếm nguồn dopamine dễ tiếp cận -- từ điện thoại, cờ bạc hoặc một túi Fritos -- \& duy trì cuộc sống khỏe mạnh, hiệu quả, ổn định.		
		\item ``[An] eye-opening survey on pleasure-seeking \& addiction $\ldots$ Readers looking for balance will return to Lembke's informative \& fascinating guidance.'' -- {\it Publishers Weekly} (starred review)
		
		-- [Một] cuộc khảo sát mở mang tầm mắt về việc tìm kiếm niềm vui \& chứng nghiện $\ldots$ Những độc giả đang tìm kiếm sự cân bằng sẽ quay lại với hướng dẫn đầy thông tin \& hấp dẫn của Lembke.
		\item ``Fascinating case histories, \& a sensible formula for treatment.'' -- {\it Kirkus Reviews}
		
		-- Lịch sử các trường hợp hấp dẫn, \& một công thức điều trị hợp lý.
	\end{itemize}
	
	\item \cite{Lembke_dopamine_VN}. {\sc Anna Lembke}. {\it Dopamine Nation: Finding Balance in the Age of Indulgence -- Giải Mã Hoóc-môn Dopamine: Sống Cân Bằng Trong Thời Đại Đầy Cám Dỗ}.\hfill{\sf[done]}
	
	\item \cite{Little_personality}. {\sc Brian R. Little}. {\it Who Are You, Really? The Surprising Puzzle of Personality}.\hfill{\sf[done]}
	
	\item \cite{Little_personality_VN}. {\sc Brian R. Little}. {\it Who Are You, Really? The Surprising Puzzle of Personality -- Bạn Thật Sự Là Ai? Khám Phá Đáng Kinh Ngạc Về Tính Cách Con Người}.\hfill{\sf[done]}
	
	\item \cite{Long2021}. Vũ Hoàng Long (chủ biên). {\it Học Trường Chuyên -- Những Góc Nhìn Đa Chiều}.\hfill{\sf[done]}
	
	\item {\sc Henri Poincar\'e}. {\it Reflections Mathematical Creation}.\hfill{\sf[done]}
	
	\item \cite{MNSS_calm}. Ma Nữ Sha Sha. {\it Sức Hút Của Sự Điềm Tĩnh}\footnote{Especially for highly sensitive girls \& women.}.\hfill{\sf[done]}
	
	\item \cite{MacKenzie2015}. Jackson MacKenzie. {\it Psychopath Free: Recovering From Emotionally Abusive Relationships with Narcissists, Sociopaths, \& Other Toxic People}.\hfill{\sf[done]}
	\item \cite{Macmart_depress}. Macmart. {\it Một Cuốn Sách Trầm Cảm}.\hfill{\sf[reading]}
	
	\item \cite{Manson_giving_fuck}. Mark Manson. {\it The Subtle Art of Not Giving A F*ck: A Counterintuitive Approach to Living a Good Life}.\hfill{\sf[reading]}
	
	\item \cite{Manson_giving_fuck_vn}. Mark Manson. {\it The Subtle Art of Not Giving A F*ck: A Counterintuitive Approach to Living a Good Life -- Nghệ Thuật Tinh Tế của Việc ``Đếch'' Quan Tâm: Một Cách Tiếp Cận Khác Thường Để Sống Tốt}.\hfill{\sf[done]}
	
	\item \cite{Mariaskin_OCD}. Amy Mariaskin. {\it Phát Triển Các Mối Quan Hệ Khi Mắc OCD}.\hfill{\sf[done]}
	
	\item \cite{McRaney_not_smart}. {\sc David McRaney}. {\it You Are Not So Smart: Why You Have Too Many Friends on Facebook, Why Your Memory Is Mostly Fiction, an d 46 Other Ways You're Deluding Yourself}.
	
	\item \cite{McRaney_not_smart_VN}. {\sc David McRaney}. {\it You Are Not So Smart: Why You Have Too Many Friends on Facebook, Why Your Memory Is Mostly Fiction, an d 46 Other Ways You're Deluding Yourself -- Bạn Không Thông Minh Lắm Đâu}.\hfill{\sf[done]}
	
	\item \cite{McRaney_less_stupid}. {\sc David McRaney}. {\it You are Now Less Dumb: How to Conquer Mob Mentality, How to Buy Happiness, \& All the Other Ways to Outsmart Yourself}.
	
	\item \cite{McRaney_less_stupid_VN}. {\sc David McRaney}. {\it You are Now Less Dumb: How to Conquer Mob Mentality, How to Buy Happiness, \& All the Other Ways to Outsmart Yourself -- Bạn Đỡ Ngu Ngơ Rồi Đấy}.\hfill{\sf[done]}
	
	\item \cite{Minh2022}. Cao Minh. {\it Thiên Tài Bên Trái, Kẻ Điên Bên Phải}.\hfill{\sf[done]}
	
	\item \cite{Mirza_covert_passive_aggressive_narcissist}. {\sc Debbie Mirza}. {\it The Covert Passive Aggressive Narcissist: Recognizing the Traits \& Finding Healing After Hidden Emotional \& Psychological Abuse (The Narcissism Series)}. {\sf[4420 Amazon ratings][2497 Goodreads ratings]}\hfill{\sf[done]}
	
	{\sf Amazon review.} Do you feel confused \& exhausted by a relationship, \& you can't figure out why? Do you feel like you can't think straight, \& the person in your life seems fine, so you wonder if maybe you are the problem?
	
	Has someone mentioned you might be with a narcissist, or you wonder yourself, \& when you research narcissism, they don't seem completely fit the description, although some of the traits do ring true?
	
	{\it The Covert Passive Aggressive Narcissist} is the most comprehensive \& helpful book on the topic of covert narcissism. Also available in Spanish as El Nacisista Pasivo Agresivo. Find the answers you are looking for. This book delivers:
	\begin{itemize}
		\item A list of traits of the covert narcissist \& how they look like in daily life
		\item The differences between an overt \& a covert narcissist
		\item A checklist to see if you are with a covert narcissist
		\item Real-life stories to illustrate what these traits look like
		\item Explanations of different covert techniques narcissists use to control \& manipulate
		\item A chapter dedicated to what sex looks like with a covert narcissist
		\item Descriptions of covertly narcissist parents Information on what it looks like to have a covertly narcissistic boss or co-worker
		\item A chapter on healing to help give you tools \& hope for a beautiful future, free of toxic relationships.
	\end{itemize}
	You will see that you are not crazy, that your instincts are correct, \& you will learn how to see through covert manipulation \& control.
	
	The most common description a survivor of this type of relationship will use is crazy-making. The emotional abuse \& gaslighting makes you question your own view of reality, \& sometimes your own sanity. You will know after reading this book if the person you are with is a covert narcissist, \& your experience with them will begin to make sense for the 1st time.
	
	When most people think of a narcissist, they think of someone who is grandiose, obviously self-absorbed, sees themselves as superior to others, \& throws fits of rage when they don't get their way. But what if the narcissists is 1 of the nicest people you've ever met? What if they are a great listener, seem to care about others, or are a pillar of the community? What if they are the mother that volunteers at the school, the husband that your friends wish they had, the boss that your co-workers feel so lucky to work for? Parents, spouses, partners, bosses, \& friends who are covert narcissists come across as the nicest people. They can be spiritual leaders, therapists, moms who bring over casseroles to needy people, \& bosses who everyone loves.
	
	A covert narcissist has the same traits of narcissism as the well-known overt type. The difference is when they control \& manipulate, when they demean \& devalue you, it is done in such a subtle way you don't notice it.
	
	This type of narcissism is 1 of the most damaging forms because the abuse is so hidden \& so insidious. You can be in a relationship with a covert narcissist that can last for decades \& not realize you are being psychologically \& emotionally controlled, manipulated, \& abused. There are no visible scars with this form of abuse, \& you are usually the only one that experiences their destructive \& psychologically debilitating behavior.
	
	{\it Living with a covert narcissist drains your spirit \& leaves you questioning your own reality}.
	
	You have been lied to for years, \& it is time to finally see the truth of what you have been through, who you really are, \& how much you deserve love \& happiness.
	\begin{quotation}
		{\it``Narcissists are deeply unhappy people. They get jealous of you when you are experiencing life \& happiness. They do not want you to be happy \& strong as those feelings threaten their ability to control you.''}
		
		Những người ái kỷ là những người vô cùng bất hạnh. Họ ghen tị với bạn khi bạn đang trải nghiệm cuộc sống \& hạnh phúc. Họ không muốn bạn hạnh phúc \& mạnh mẽ vì những cảm xúc đó đe dọa khả năng kiểm soát bạn của họ.
		
		{\it``CNs are very passive. They put the responsibility on you to make sure they are happy \& blame you when they're not.''}
		
		-- CN rất thụ động. Họ đặt trách nhiệm lên bạn để đảm bảo rằng họ hạnh phúc \& đổ lỗi cho bạn khi họ không hạnh phúc.		
		
		{\it``The thing you start noticing when you become aware of the issues with the CN is that most of what they say about you is actually a projection of what is true of them.''}
		
		-- Điều bạn bắt đầu nhận thấy khi nhận thức được các vấn đề của CN là hầu hết những gì họ nói về bạn thực ra chỉ là sự phản ánh những gì đúng về họ.
		
		{\it``When you are with a CN, you can never wi no matter what you do. They will never be fully satisfied with you. You will never be good enough in their eyes. They have to have something they can hold over you in order to control \& manipulate you.''}
		
		-- Khi bạn ở với CN, bạn không bao giờ có thể thắng được dù bạn có làm gì đi nữa. Họ sẽ không bao giờ hoàn toàn hài lòng với bạn. Bạn sẽ không bao giờ đủ tốt trong mắt họ. Họ phải có thứ gì đó có thể khống chế bạn để kiểm soát \& thao túng bạn.
	\end{quotation}
	{\sf Editorial reviews.}
	\begin{itemize}
		\item {\it The Covert Passive-Aggressive Narcissist} belongs on passive-aggressive every survivor's bookshelf. {\sc Debbie Mirza}'s book is a compassionate \& healing resource for anyone seeking relief after narcissistic abuse. She clearly identifies subtle red flags that are often so difficult for survivors to pinpoint, while also encouraging the reader to look inward for solutions. Her warm \& encouraging words are like receiving a written hug when you need it most.'' -- {\sc Jackson MacKenzie}, author of {\it Psychopath Free} \& {\it Whole Again}
		\item ``This is an insight book for therapists working with complex trauma \&{\tt/}or Complex-PTSD. Clinicians can diagnose ``textbook'' narcissism. However, covert passive-aggressive narcissism is difficult to identify \& not widely recognized in the field of mental health therapy. {\sc Debbie} shines a light on interpersonal relationships with a CPAN by naming experiences \& behavioral patterns. As a clinical social worker, I love this book \& use it as a basis of understanding clients who have experienced emotional \& psychological abuse by someone they love. Thank you for having the wisdom to write this book \& validate the invisible scars of survivors because this is where healing begins.'' -- {\sc Denise Malm}, LSWAIC, GMHS
		\item ``The Covert Passive-Aggressive Narcissist brings a massive sigh of relief to people who have been involved with, were raised by, or worked with someone who has made them feel crazy, exhausted, depressed, unworthy, guilty, terrified, \& chronically anxious, while they charmed \& amazed others. {\sc Debbie Mirza} provides insight, answers, \& healing to those who have wondered whether they have been abused by a narcissist but have not found themselves or their answers in the current literature on Narcissist Abuse. As a Clinical Psychologist for over 20 years, I find that {\it The Covert Passive-Aggressive Narcissist} fills in the missing pieces in this field for clinicians \& victims alike. {\sc Debbie} clarifies this phenomenon{\tt/}personality disorder when I have struggled to explain it in my own life \& in the stories of countless patients. You aren't crazy: this book helps you identify \& name the abuse -- so you can be free to truly reclaim your life.'' -- {\sc Robin LW Alchin}, PhD, Clinical Psychologist
		\item ``Passive-aggression. Narcissism. These terms are bandied about in 21st-century America, often without a clear definition, \& without understanding how to respond effectively once they are recognized. Enter {\sc Debbie Mirza}, \& her brilliant, immensely helpful book, {\it The Covert Passive-Aggressive Narcissist}. She realized that passive-aggression \& narcissism coexist, that these behaviors are epidemic in our culture \& that millions of people are guilty of them. She realized it because she herself was once the target of a Covert Passive-Aggressive Narcissist (CN). After finding her way out, she used her experience to write her book, \&, in so doing, she is helping other targets to escape from their base relationships. Chap. 3 uniquely describes those who are likely to be victimized by CNs: empathic, compassionate, nurturing, trusting, dependable, flexible -- all the positive traits of beautiful human beings. Their task is to realize their sense of self-worth in order to extricate themselves from the grip of the Ctextbook \& to find peace. The book offers multiple strategies for doing so, including a checklist of {\it Traits of Real Love}, so that the survivor recognizes it when it is found. It is a book to be read \& re-read by clients \& therapists alike, as it is an important contribution to self-help literature.'' -- {\sc Judy M. Sobczak}, Ph.D., Licensed Psychologist
		\item ``As a psychotherapist, this book has proven to be an outstanding, effective tool to help clients in these types of relationships be able to finally understand \& clearly know what they are dealing with. The author does an excellent job of clearly identifying \& providing a name to all the ``crazy-making'' behaviors the covert passive-aggressive narcissist does in order to make my clients feel like they are never enough. It identifies those gaslighting behaviors, dissects, \& defines them 1 by 1, in a clear, concise way with lots of real-life examples. Before I read this book, I didn't have the words to describe a dysfunctional relationship of this sort. Now I do $\ldots$ \& it has made a world of difference to my clients by not only improving their lives but also eliminating depression \& anxiety symptoms which were a result of the covert passive aggressive narcissist's behaviors.'' -- {\sc Pam Hauke}, MSW, LCSW, SAC
	\end{itemize}
	{\sf From the Author.} ``Dear Reader, Writing this book was a passion project for me. I had you in mind with every word. I wrote this book so you could have the answers you have been looking for, the clarity you deserve. If you have come to this book, I imagine you are probably in a relationship, or coming out of one that has been incredibly confusing \& hurtful. You have most likely thought the issues in the relationship were your fault. You have either been told this by your partner, parent, friend, sibling, or co-worker, or given this message through subtle means. You may be questioning your own reality \& are filled with self-doubt. Your physical health has been affected. The light inside you has dimmed. This person you have experienced is probably well liked, which makes this even more confusing. They may even do things to help others or are in a profession where they are revered. They could even be looked up to as a guru, a therapist, a pastor, a leader of some kind. They are probably successful, charming, \& appear sincere. I know what it is like to be utterly confused by a relationship with someone you love deeply \& honestly. I know what it is like to have the life inside you slowly drain away until you are an exhausted version of yourself with low self-esteem \& little self-confidence. Covert narcissism is the most confusing \& crazy making type of narcissism. These people have the same traits as the well-known overt narcissist, but the way they manifest is very different. You can live with a covert narcissist for years \& not see through their behavior. Years ago, I searched for answers to help with my own confusion. I read many books on narcissism but could not find any on the covert type. After years of piecing together information from various sources, I decided to write the book I had needed \& couldn't find so other survivors, like you, would have the information that would help you heal, all in 1 place. In preparation for this book, I interviewed $> 100$ survivors. I also spoke with therapists in this field. I did in-depth research on the topic because I wanted to make sure this book would be accurate, comprehensive, \& incredibly helpful for you. You deserve that. As I met more \& more people who have experienced this type of relationship, my heart was affected tremendously. Witnessing their pain, their wounded hearts, \& their strength was humbling \& brought out a fierceness in me that made me want to make this the most helpful book I possibly could. In this book, I explain the traits of a covert narcissist. I share lots of stories from people I've interviewed to illustrate the traits. I also spend a lot of time talking about healing. If you have read this far, my hunch is you have probably been through or are going through a tremendously difficult \& crazy-making experience with a covert narcissist. You deserve to find clarity \& ultimately heal the wounds this relationship has caused. I go into further depth of how to heal in another book I wrote called, {\it The Safest Place Possible}, where I share some of my own personal stories \& the process I went through to heal. Being with a covert narcissist can take you far away from the person you really are. My hope is this book will help bring you back to your stunning self. May you find all the answers you are looking for \& come to a place of freedom \& peace. That may not feel possible right now, but trust me, it is. With love, {\sc Debbie Mirza}
	
	{\sf About the Author.} {\sc Debbie Mirza} is an author, restorative coach, \& singer{\tt/}songwriter. She is the author of the best-selling book, The Covert Passive Aggressive Narcissist: Recognizing the Traits \& Finding Healing After Hidden Emotional \& Psychological Abuse, which has helped thousands of people around the world understand the most hidden \& insidious form of narcissism that is currently affecting millions of people without them realizing what they are dealing with (this book is now available in Spanish), \& The Safest Place Possible: A Guide to Healing \& Transformation, a book about the power of self-love to heal after emotional \& psychological trauma. This book includes some of Debbie's personal story of healing, as well as 21 practical, gentle exercises of self-love. As a coach, she works with people who are coming out of relationships with covert narcissists, helping men \& women recognize the truth of what they have experienced, \& come back to their genuine magnificence. She offers online courses as well as guided meditations \& calming music to help with the healing process on her website. Her debut album, Soul Rising can be found on Amazon, iTunes, CDbaby, \& Google Play.  You can also find Debbie on her YouTube channel as well as a private online support group she created on Facebook. Learn more about {\sc Debbie}'s work \& offerings at \url{debbiemirza.com}.
	
	\item {\sc Debbie Mirza}. {\it Worthy of Love: A Gentle \& Restorative Path to Healing After Narcissistic Abuse (The Narcissism Series)}. {\sf[292 Amazon ratings][144 Goodreads ratings]}
	
	\item {\sc Debbie Mirza}. {\it The Safest Place Possible: A Guide to Healing \& Transformation}.
	
	\item {\sc Debbie Mirza}. {\it Rewriting False Messages from Narcissists \& Toxic People: A Guide \& Meditation}.
	
	\item \cite{Murphy_subconscious}. {\sc Joseph Murphy}. {\it The Power of Subconscious Mind}.\hfill{\sf[reading]}
	
	\item \cite{Murphy_subconscious_VN}. {\sc Joseph Murphy}. {\it The Power of Subconscious Mind -- Sức Mạnh Tiềm Thức}.\hfill{\sf[done]}
	
	\item \cite{Ngoc_wolf_sheep}. {\sc Lê Bảo Ngọc}. {\it Không Phải Sói Nhưng Cũng Đừng Là Cừu}.\hfill{\sf[done]}
	
	\item \cite{Peck_road}. {\sc M. Scott Peck}. {\it The Road Less Traveled, Timeless Edition: A New Psychology of Love, Traditional Values \& Spiritual Growth}.
	
	\item \cite{Peck_road_VN}. {\it The Road Less Traveled, Timeless Edition: A New Psychology of Love, Traditional Values \& Spiritual Growth -- Con Đường Chẳng Mấy Ai Đi: Tâm Lý Học Kinh Điển Về Tình Yêu, Phẩm Giá \& Hành Trình Trưởng Thành Tinh Thần}.\hfill{\sf[done]}
	
	\item \cite{Phipps_Gautreys_muu_hen_ke_ban_tap_1}. Mike Phipps, Colin Gautreys. {\it Mưu Hèn Kết Bẩn Nơi Công Sở. Tập 1: Nghệ Thuật Nhận Biết \& Phòng Tránh ``Tiểu Nhân'' Trong Công Việc}.\hfill{\sf[done]}
	
	\item \cite{muu_hen_ke_ban_tap_2}. Alpha Books. {\it Mưu Hèn Kết Bẩn Nơi Công Sở. Tập 2: Nghệ Thuật Thăng Tiến Trong Sự Nghiệp}.\hfill{\sf[done]}
	
	\item \cite{Rutherford2020}. Albert Rutherford. {\it The Art of Thinking Critically: Ask Great Questions, Spot Illogical Reasoning, \& Make Sharp Arguments (The critical Thinker Book 5)}.\hfill{\sf[reading]}
	
	\item \cite{Rutherford2022}. Albert Rutherford. {\it Rèn Luyện Tư Duy Phản Biện}.\hfill{\sf[done]}
	
	\item \cite{Rutherford2023}. Albert Rutherford. {\it The Art of Thinking Critically: Ask Great Questions, Spot Illogical Reasoning, \& Make Sharp Arguments -- Nghệ Thuật Tư Duy Phản Biện}.\hfill{\sf[done]}
	
	\item \cite{Sandberg_Grant2017}. Sheryl Sandberg, Adam Grant. {\it Option B: Facing Adversity, Building Resilience, \& Finding Joy}.\hfill{\sf[reading]}
	
	\item \cite{Sandberg_Grant2019}. Sheryl Sandberg, Adam Grant. {\it Option B: Facing Adversity, Building Resilience, \& Finding Joy -- Phương Án B: Đối Mặt Nghịch Cảnh, Rèn Tính Kiên Cường, \& Tìm Lại Niềm Vui}.\hfill{\sf[done]}
	
	\item \cite{Schwartz2019}. David J. Schwartz. {\it The Magic of Thinking Big -- Dám Nghĩ Lớn}.\hfill{\sf[done]}
	
	\item \cite{Simon_character}. George Simon Jr. {\it Character Disturbance: The Phenomenon of Our Age}.\hfill{\sf[done]}
	
	\item \cite{Simon_sheep}. George Simon Jr. {\it In Sheep's Clothing: Understanding \& Dealing with Manipulative People}.\hfill{\sf[reading]}
	
	\item \cite{Simon_sheep_VN}. George K. Simon. {\it In Sheep's Clothing: Understanding \& Dealing with Manipulative People -- Sói Đội Lốt Cừu: Kẻ Hiếu Chiến Ngầm \& Các Thủ Thuật Thao Túng Tâm Lý}.\hfill{\sf[done]}
	
	\item \cite{Solomon_depression}. {\sc Andrew Solomon}. {\it The Noonday Demon: An Atlas of Depression}.\hfill{\sf[reading]}
	
	{\sf Prize.} National Book Award Winner 2001. Lambda Literary Award Winner 2002.
	
	{\sf Amazon review.} {\it The Noonday Demon} is {\sc Andrew Solomon}'s National Book Award-winning, bestselling, \& transformative masterpiece on depression -- ``the book for a generation, elegantly written, meticulously researched, empathetic, \& enlightening'' ({\it Time}) -- now with a major new chapter covering recently introduced \& novel treatments, suicide \& anti-depressants, pregnancy \& depression, \& much more.
	
	{\it The Noonday Demon} examines depression in personal, cultural, \& scientific terms. Drawing on his own struggles with the illness \& interviews with fellow sufferers, doctors, \& scientists, policy makers \& politicians, drug designers, \& philosophers, {\sc Andrew Solomon} reveals the subtle complexities \& sheer agony of the disease as well as the reasons for hope. He confronts the challenge of defining the illness \& describes the vast range of available medications \& treatments, \& the impact the malady has on various demographic populations -- around the world \& throughout history. He also explores the thorny patch of moral \& ethical questions posed by biological explanations for mental illness. With uncommon humanity, candor, with, \& erudition, award-winning author {\sc Solomon} takes readers on a journey of incomparable range \& resonance into the most pervasive of family secrets. His contribution to our understanding not only of mental illness but also of the human condition is truly stunning.''
	
	{\sf Editorial Reviews.}
	\begin{itemize}
		\item ``All encompassing, brave, \& deeply humane $\ldots$ It is open-minded, critically informed, \& poetic at the same time, \& despite the nature of its subject it is written with far too much \'elan \& elegance ever to become depressing itself.'' -- {\sc Richard Bernstein}, {\it The New York Times}
		\item ``{\it The Noonday Demon} is the ideal \& definitive book on depression. There is nothing falsely consoling about this account, which is the opposite of a bromide\footnote{a chemical which contains bromine, used, especially in the past, to make people feel calm.}, unless to be accompanied by so much intelligence \& understanding is a consolation in itself.'' -- {\sc Edmund White}, author of {\it A Boy's Own Story \& The Flaneur		}
		\item ``An exhaustively researched, provocative, \& often deeply moving survey of depression $\ldots$ original \& vividly recounted, {\sc Solomon} writes engagingly; his style is intimate \& anecdotal $\ldots$ witty \& persuasive. Overall $\ldots$ {\it The Noonday Demon} is a considerable accomplishment. It is likely to provoke discussion \& controversy, \& its generous assortment of voices, from the pathological to the philosophical, makes for rich, variegated reading.'' -- {\sc Joyce Carol Oates}, {\it The New York Times Book Review}
		\item ``The book for a generation $\ldots$ Solomon interweaves\footnote{to twist together 2 or more pieces of thread, wool, etc.} a personal narrative with scientific, philosophical, historical, political, \& cultural insights. $\ldots$ The result is an elegantly written, meticulously researched book that is empathetic \& enlightening, scholarly \& useful $\ldots$ Solomon apologizes that `no book can span the reach of human suffering.' This one comes close.'' -- {\sc Christine Whitehouse}. {\it Time}
		\item ``Both heartrending \& fascinating $\ldots$ the book has a scope \& passionate intelligence that gives it intrigue as well as heft.'' -- {\sc Gail Caldwell}, {\it The Boston Globe}
		\item ``{\it The Noonday Demon} explores the subterranean\footnote{under the ground.} realms of an illness which is on the point of becoming endemic, \& which more than anything else mirrors the present state of our civilization \& its profound discontents. As wide-ranging as it is incisive, this astonishing work is a testimony both to the muted suffering of millions \& to the great courage it must have taken the author to set his mind against it.'' -- {\sc W. G. Emily Nussbaum}, {\it The Village Voice}
		\item ``A wrenching candid, fascinating, \& exhaustive tour of 1 of the darker chambers of the human heart.'' -- {\sc Daniel Goleman}, author of {\it Emotional Intelligence}
		\item ``Everyone will find a piece of himself in Solomon's account, even if he has been spared the experience of watching that kernel blossom into a monstrous \& strangling plant $\ldots$ Solomon shows bravery \& rigor.'' -- {\sc Christopher Caldwell}, {\it Slate magazine}
		\item ``Exhaustive \& eloquent\footnote{1. able to use language \& express your opinions well, especially when you are speaking in public; 2. (of a look or movement) able to express a feeling. có tài hùng biện.}.'' -- {\sc Maria Russo}, Salon.com
		\item ``{\it The Noonday Demon} is an amazingly rich \& absorbing work that deals with depression on many levels of perception. In its flow of insights \& its scope -- encompassing not only the author's own ordeal but also keen inquiries into the biological, social, \& political aspects of the illness -- {\it The Noonday Demon} has achieved a level of authority that should assure its place among the few indispensable works on depression.'' -- {\sc William Styron}, author of {\it Darkness Visible}
		\item ``{\sc Andrew Solomon}'s {\it The Noonday Demon} is immensely readable \& should be universally useful. It is indeed an atlas of depression, sensitively chronicling the illness's characteristics, social \& cultural history, modes of treatment, \& prospects. What makes it remarkable is a highly individual blend of the personal \& the dispassionate, the work of a benign intelligence.'' -- {\sc Harold Bloom}, author of {\it How to Read \& Why \& Shakespeare: The Invention of the Human}
		\item ``Frank $\ldots$ clearheaded [\&] valuable $\ldots$'' -- {\it Entertainment Weekly}
		\item ``Compulsively readable, harrowing, \& helpful, {\it The Noonday Demon} is an act of redemption in an epidemic of sorrow.'' -- {\sc Louise Erdrich}, author of {\it Love Medicine \& The Antelope Wife}
		\item ``{\sc Solomon}'s done his homework $\ldots$ smart, lucid, \& sometimes intensely moving.'' -- {\sc David Gate}, {\it Newsweek}
		\item ``As the great Flaubert discovered, it's hard to write about boring people without being boring yourself. Similarly, it's hard to write at length about depression without depressing the reader. Yet in {\it The Noonday Demon}, {\sc Andrew Solomon}, through his candor, intellectual elegance, \& ultimately his human resilience, manages to write of traumas both deep \& ordinary without leaving the reader traumatized. His book is a large achievement.'' -- {\sc Larry McMurtry}, Pulitzer Prize-winning author of {\it Lonesome Dove}
		\item ``{\sc Solomon}'s highly readable, tag-all-bases new book $\ldots$ gives us nothing less than an evolving portrait of who, collectively, we are $\ldots$ ambitious \& broadly synthesizing $\ldots$ [written with] considerable stylistic grace $\ldots$ {\sc Solomon} is knowledgeable, trenchant, \& an admirable distiller of facts \& perspectives.'' -- {\sc Sven Birkerts}, {\it The New York Observer}
	\end{itemize}
	{\sf About the Author.} {\sc Andrew Solomon} is a professor of psychology at Columbia University, president of PEN American Center, \& a regular contributor to The New Yorker, NPR, \& The New York Times Magazine. A lecturer \& activist, he is the author of Far \& Away: Essays from the Brink of Change: Seven Continents, Twenty-Five Years; the National Book Critics Circle Award-winner Far from the Tree: Parents, Children, \& the Search for Identity, which has won thirty additional national awards; \& The Noonday Demon; An Atlas of Depression, which won the 2001 National Book Award, was a finalist for the Pulitzer Prize, \& has been published in twenty-four languages. He has also written a novel, A Stone Boat, which was a finalist for the Los Angeles Times First Fiction Award \& The Irony Tower: Soviet Artists in a Time of Glasnost. His TED talks have been viewed over ten million times. He lives in New York \& London \& is a dual national. For more information, visit the author's website at \url{AndrewSolomon.com}.
	
	{\sf YouTube.}
	\begin{itemize}
		\item \href{https://www.youtube.com/watch?v=-eBUcBfkVCo}{YouTube{\tt/}{\sc Andrew Solomon}: Trầm cảm, những bí mật được sẻ chia}.\hfill{\sf[done]}
	\end{itemize}
	
	\item \cite{Stout_sociopath}. {\sc Martha Stout}. {\it The Sociopath Next Door}.\hfill{\sf[reading]}
	
	\item \cite{Stout_sociopath_VN}. {\sc Martha Stout}. {\it The Sociopath Next Door -- Kẻ Ác Cạnh Bên}.\hfill{\sf[done]}
		
	\item Martha Stout. {\it The Myth of Sanity: Divided Consciousness \& the Promise of Awareness}.
	
	\item \cite{Thaler_misbehaving_VN}. Richard H. Thaler. {\it Misbehaving: The Making of Behavioral Economics -- Tất Cả Chúng Ta Đều Hành Xử Cảm Tính: Sự Hình Thành Kinh Tế Học Hành Vi}.\hfill{\sf[reading]}
	
	\item \cite{Thomas_psychological_manipulation}. {\sc Shannon Thomas, LCSW}. {\it Healing from Hidden Abuse: A Journey Through the Stages of Recovery from Psychological Abuse}. {\sf[3461 Amazon ratings][2519 Goodreads ratings]}
	
	{\sf Amazon review.} Within every community, toxic people can be found hiding in families, couples, companies, \& places of worship. The cryptic nature of psychological abuse involves repetitious mind games played by 1 individual or a group of people. Psychological abuse leaves no bruises. There are no broken bones. There are no holes in the walls. The bruises, brokeness, \& holes are held tightly within the target of the abuse. {\it Healing from Hidden Abuse} walks the reader through each of the 6 recovery stages researched \& developed by the author. The stages are Despair, Education, Awakening, Boundaries, Restoration, \& Maintenance. A guided Personal Reflections journal is included in the back of the book to help the reader go deeper in their application of the 6 stages of recovery. The journal can be used individually or in a small group setting.
	
	-- Trong mọi cộng đồng, những người độc hại có thể ẩn náu trong gia đình, cặp vợ chồng, công ty, nơi thờ cúng. Bản chất khó hiểu của lạm dụng tâm lý bao gồm các trò chơi trí óc lặp đi lặp lại do một cá nhân hoặc một nhóm người chơi. Bạo hành tinh thần không để lại vết bầm tím. Không có xương gãy. Không có lỗ trên tường. Những vết bầm tím, gãy xương, lỗ thủng được giữ chặt trong mục tiêu bị lạm dụng. {\it Healing from Hidden Abuse} hướng dẫn người đọc qua từng giai đoạn trong số 6 giai đoạn phục hồi do tác giả nghiên cứu \& phát triển. Các giai đoạn là Tuyệt vọng, Giáo dục, Thức tỉnh, Ranh giới, Phục hồi, \& Bảo trì. Một nhật ký Suy ngẫm Cá nhân có hướng dẫn được đính kèm ở cuối cuốn sách để giúp người đọc tìm hiểu sâu hơn trong việc áp dụng 6 giai đoạn phục hồi. Nhật ký có thể được sử dụng riêng lẻ hoặc trong một nhóm nhỏ.
	\begin{itemize}
		\item {\it``Psychological abusers do not take responsibility for their actions, so that must be flung onto someone else.''}
		
		-- Những kẻ bạo hành tâm lý không chịu trách nhiệm về hành động của mình nên việc đó phải đổ lên người khác.
		
		\item {\it``The core inherent faulty thinking of abusers is that everything revolves around them.''}		
		
		-- Suy nghĩ sai lầm cố hữu cốt lõi của những kẻ bạo hành là mọi thứ đều xoay quanh họ.
		
		\item {\it``Abusers like to target people who have something they do not or cannot possess themselves.''}
		
		-- Những kẻ bạo hành thích nhắm vào những người có thứ gì đó mà bản thân họ không có hoặc không thể sở hữu.
	\end{itemize}
	{\sf Editorial reviews.}
	\begin{itemize}
		\item ``Compassionate \& well-researched, a must read for anyone healing from psychological abuse. The warm, conversational writing style \& {\sc Shannon Thomas} professional experience combine to make the perfect recovery resource.'' -- {\sc Jackson MacKenzie}, author of {\it Psychopath Free} \& cofounder of \url{PsychopathFree.com}, an online support community that reaches millions of abuse survivors each month.
		
		-- Từ bi{\tt/}động lòng trắc ẩn \& được nghiên cứu kỹ lưỡng, một cuốn sách phải đọc cho bất kỳ ai đang chữa lành khỏi sự lạm dụng tâm lý. Phong cách viết mang tính trò chuyện, ấm áp của \& {\sc Shannon Thomas} trải nghiệm chuyên môn kết hợp với nhau để tạo nên nguồn tài nguyên phục hồi hoàn hảo.
		
		\item ``{\sc Shannon Thomas} has written an important book about something ugly, hidden, \& difficult to describe. Psychological abuse. How is it possible that 1 person can gain so much power to destroy another person's sense of worth, safety, \& sanity? {\sc Shannon} tells you how, but more importantly, she gives you a roadmap that helps you wake up, break free, heal, \& rebuild your shattered life.'' -- {\sc Leslie Vernick LCSW}, counselor, coach, speaker, \& author of {\it The Emotionally Destructive Marriage} \& {\it The Emotionally Destructive Relationship}.
		
		-- {\sc Shannon Thomas} đã viết một cuốn sách quan trọng về một điều gì đó xấu xí, ẩn giấu, \& khó diễn tả. Lạm dụng tâm lý. Làm sao mà một người có thể có được nhiều quyền lực đến vậy để phá hủy cảm giác về giá trị, sự an toàn \& sự tỉnh táo của người khác? {\sc Shannon} cho bạn biết cách thực hiện, nhưng quan trọng hơn, cô ấy đưa ra lộ trình giúp bạn thức dậy, thoát ra, chữa lành, \& xây dựng lại cuộc đời tan vỡ của mình.
		
		\item ``Few writers are able to connect research, experience, \& intuitive understanding as {\sc Shannon Thomas} does in her groundbreaking new book for survivors of emotional \& psychological trauma. In {\it Healing from Hidden Abuse}, you will find not only evidence of {\sc Shannon}'s expertise as a therapist who has worked with clients suffering from the trauma of covert psychological abuse, but also her powerful mastery of the crucial questions that are needed in order to work through the trauma \& heal.'' -- {\sc Shahida Arabi}, author of {\it Becoming the Narcissist s Nightmare: How to Devalue \& Discard the Narcissist While Supplying Yourself} \& founder of {\it Self-Care Haven}
		
		-- Rất ít nhà văn có thể kết nối nghiên cứu, kinh nghiệm, \& hiểu biết trực quan như {\sc Shannon Thomas} đã làm trong cuốn sách mới mang tính đột phá của mình dành cho những người sống sót sau chấn thương tâm lý \& tâm lý. Trong {\it Healing from Hidden Abuse}, bạn sẽ không chỉ tìm thấy bằng chứng về chuyên môn của {\sc Shannon} với tư cách là một nhà trị liệu đã làm việc với những khách hàng bị tổn thương do lạm dụng tâm lý bí mật, mà còn cả khả năng thông thạo mạnh mẽ của cô ấy về những điều quan trọng những câu hỏi cần thiết để vượt qua tổn thương \& chữa lành.
		
		\item ``In her book, {\it Healing from Hidden Abuse}, {\sc Shannon Thomas} offers words of wisdom \& hope as she shines a spotlight on this necessary topic. Clearly she gets it, \& her explanations of the steps involved in healing are spot on. Not only will you find the body of the book helpful, she goes a step further by offering a detailed guided journal at the end. This resource is a valuable tool for both therapist \& patient.'' Dr. {\sc Les Carter}, author of {\it Enough About You, Let's Talk About Me} \& create of \url{MarriagePath.com}.
		
		-- Trong cuốn sách của mình, {\it Healing from Hidden Abuse}, {\sc Shannon Thomas} đưa ra những lời khôn ngoan \& hy vọng khi cô làm nổi bật chủ đề cần thiết này. Rõ ràng là cô ấy hiểu điều đó, \& những lời giải thích của cô ấy về các bước liên quan đến việc chữa bệnh rất chính xác. Bạn không chỉ thấy nội dung cuốn sách hữu ích mà cô ấy còn tiến một bước xa hơn bằng cách cung cấp một nhật ký có hướng dẫn chi tiết ở cuối. Tài nguyên này là một công cụ có giá trị cho cả nhà trị liệu \& bệnh nhân.
	\end{itemize}
	{\sf About the Author.} {\sc Shannon Thomas, LCSW} is the international bestselling author of {\it Healing from Hidden Abuse: A Journey Through the Stages of Recovery from Psychological Abuse} \& {\it Exposing Financial Abuse: When Money is a Weapon}, \& the owner{\tt/}lead therapist of an award-winning counseling practice in Southlake, TX.
	
	Bridging clinical advice with pop culture language, {\sc Thomas} approaches her counseling work \& writing from the lens of a therapist \& as a fellow survivor of psychological abuse. Her 1st book, {\it Healing from Hidden Abuse}, is an international bestseller, has been published in multiple languages, \& serves as a road map for book studies \& host groups in 11 countries \& 35 states across the United States. {\sc Thomas} also coined the ``6 Stages of Healing from Hidden Abuse'' model, which has been met with favorable reviews \& high applause from readers \& medical professionals across the world.
	
	{\sc Thomas} has been featured in top media outlets including The Oprah Magazine, Associated Press, Business Insider, Reader's Digest, Yahoo!, Yahoo! Finance, Teen Vogue, Elite Daily, \& Bustle.
	
	\item \cite{Thomas_psychological_manipulation_VN}. {\sc Shannon Thomas, LCSW}. {\it Healing from Hidden Abuse: A Journey Through the Stages of Recovery from Psychological Abuse -- Thao Túng Tâm Lý: Nhận Diện, Thức Tỉnh, \& Chữa Lành Những Tổn Thương Tiềm Ẩn}.\hfill{\sf[done]}
	
	\item {\sc Shannon Thomas, LCSW}. {\it Exposing Financial Abuse: When Money is a Weapon}. {\sf[130 Amazon ratings][87 Goodreads ratings]}
	
	{\sf Amazon review.} {\it Exposing Financial Abuse: When Money is a Weapon} is a raw \& shocking expos\'e of economic exploitation \& abuse that occurs within families, the family law courts, among peers, \& places of worship.
	
	Within {\it Exposing Financial Abuse: When Money Is a Weapon}, you will be given the opportunity to pull the curtain back \& see into the lives of those who have been financially harmed by someone close to them. Taking a closer look at this hidden world is a unique gift that cannot be taken lightly or without honor for those who have chosen to allow us to peek into the most personal aspects of their lives.
	
	Test yourself. How would you describe financial abuse? It is quietly happening all around us \& is hidden within our neighborhoods \& communities. You probably know someone who lives within a financially abusive household, \& you don't even know it.
	
	What is financial abuse? Has your spouse or parent taken out lines of credit in your name without your consent? Does your ex-spouse suddenly stop paying child support as a means of furthering their abuse \& control over your life? Has your partner moved money from your joint account to a secret individual account without your prior knowledge or consent? Do your parents use financial gifts as an open door to demand future compliance on your part? Are you blamed for creating financial stress but are not the one who overspends? Did your ex-spouse hide his or her income from being included in the calculations for child \&{\tt/}or spousal support? Have your religious leaders said that you must give to the church 1st, even if that means you cannot provide for your household's basic needs? Do you carry the full burden of making enough money for your household because your partner refuses to maintain steady employment?
	
	\item {\sc Lauren Midgley, Sarah Gilliland, Shannon Thomas, Nicole Smith, Wendy Knutson}. {\it Masterminding Our Way: The Power of 5 Minds}. 5 professional women share their experience within a Mastermind entrepreneur group. They tell how their group started, their individual stories, \& how the Mastermind group has been a benefit to their personal \& business lives. Within the book, the authors provide valuable information by showing the reader how to start their own Mastermind group. {\sf[7 Amazon ratings]}
	
	\item \cite{Thu_overthink}. {\sc Nguyễn Đoàn Minh Thư}. {\it Hành Tinh Của 1  Kẻ Nghĩ Nhiều}.\hfill{\sf[done]}
	
	\item \cite{Wei_Harvard_VN}. {\sc Xiu-Ying Wei}. {\it Harvard Bốn Rưỡi Sáng}.\hfill{\sf[done]}
\end{enumerate}

%------------------------------------------------------------------------------%

\subsection{Philosophy Book}

\begin{enumerate}
	\item \cite{Burg_Mann_giver_VN}. {\sc Bob Burg, John David Mann}. {\it Go-Givers Sell More -- Người Dám Cho Đi Bán Được Nhiều Hơn}.\hfill{\sf[done]}
	
	\item \cite{Chodron_fall_apart}. {\sc Pema Ch\"odr\"on}. {\it When Things Fall Apart: Heart Advice for Difficult Times}.\hfill{\sf[reading]}
	\begin{quotation}
		{\it``Letting there be room for not knowing is the most important thing of all.''}
		
		-- Để dành chỗ cho sự không biết là điều quan trọng nhất trong cả thảy.
		
		{\it``When you have made good friends with yourself, your situation will be more friendly too.''}
		
		-- Khi bạn đã kết bạn tốt với chính mình, hoàn cảnh của bạn cũng sẽ thân thiện hơn.
		
		{\it``Life is a good teacher \& a good friend. Things are always in transition, f we could only realize it.''}
		
		-- Cuộc sống là một người thầy tốt \& một người bạn tốt. Mọi thứ luôn trong quá trình chuyển đổi, chỉ có chúng ta mới nhận ra được điều đó.
	\end{quotation}
	{\sf Amazon review.} {\sc Pema Ch\"odr\"on}'s perennially best-selling classic on overcoming life's difficulties cuts to the heart of spirituality \& personal growth -- now in a newly designed 20th-anniversary edition with a new afterword by {\sc Pema} -- makes for a perfect gift \& addition to one's spiritual library.
	
	How can we live our lives when everything seems to fall apart -- when we are continually overcome by fear, anxiety, \& pain? The answer, {\sc Pema Ch\"odr\"on} suggests, might be just the opposite of what you expect. Here, in her most beloved \& acclaimed work, {\sc Pema} shows that moving {\it toward} painful situations \& becoming intimate with them can open up our hearts in ways we never before imagined. Drawing from traditional Buddhist wisdom, she offers life-changing tools for transforming suffering \& negative patterns into habitual ease \& boundless joy.
	
	{\sf Editorial Reviews.}
	\begin{itemize}
		\item ``Perhaps what makes {\sc Pema}'s message resonate so strongly with people, no matter what their religion or spiritual path, is its universality. Each of us has experienced heartache; how we interact with that feeling, {\sc Pema} says, can create the possibility of a more joyful life.'' -- {\it O, The Oprah Magazine}
		\item ``If you're facing a challenging time in life, this is the book you want. It shows how to develop loving-kindness toward yourself \& then cultivate a fearlessly compassionate attitude toward your own pain \& that of others.'' -- {\it Lion's Roar}
		\item ``{\sc Pema Ch\"odr\"on} is 1 of those spiritual teachers who brings ancient wisdom to bear upon our daily triumphs \& tragedies $\ldots$ Incredibly wise \& poignantly practical.'' -- {\it Spirituality \& Health}
		\item ``{\sc Ch\"odr\"on}'s book is filled with useful advice about how Buddhism helps readers to cope with the grim realities of modern life, including fear, despair, rage, \& the feeling that we are not in control of our lives $\ldots$ {\sc Ch\"odr\"on} demonstrates how effective the Buddhist point of view can be in bringing order into disordered lives.'' -- {\it Publishers Weekly}
		\item ``This is a book that could serve you for a lifetime.'' -- {\it Natural Health}
	\end{itemize}
	{\sf About the Author.} {\sc Pema Ch\"odr\"on} is an American Buddhist nun in the lineage of Ch\"ogyam Trungpa \& resident teacher at Gampo Abbey in Cape Breton, Nova Scotia, the 1st Tibetan Buddhist monastery in North America. She is the author of numerous best-selling books, including The Places That Scare You \& Living Beautifully.
	
	\item \cite{Chodron_fall_apart_VN}. {\sc Pema Ch\"odr\"on}. {\it When Things Fall Apart: Heart Advice for Difficult Times -- Khi Mọi Thứ Sụp Đổ: Lời Khuyên Chân Thành Trong Những Thời Điểm Khó Khăn}.\hfill{\sf[done]}
	
	\item \cite{Chodron_uncertainty}. {\sc Pema Ch\"odr\"on}. {\it Comfortable with Uncertainty: 108 Teachings on Cultivating Fearlessness \& Compassion}. [1370 Amazon ratings][7505 Goodreads ratings]
	
	{\sf Amazon review.} 108 practical teachings for cultivating mindfulness \& compassion in the face of fear \& uncertainty, from the author of {\it When Things Fall Apart}.
	
	{\it Comfortable with Uncertainty} offers short, stand-alone readings designed to help us cultivate compassion \& awareness amid the challenges of daily living. More than a collection of thoughts for the day, it offers a progressive program of spiritual study, leading the reader through essential concepts, themes, \& practices on the Buddhist path.
	
	Readers do not need to have prior knowledge of Buddhist thought or practice, making {\it Comfortable with Uncertainty} a perfect introduction to {\sc Pema Ch\"odr\"on}'s teaching. It features the most essential \& stirring passage from {\sc Pema Ch\"odr\"on}'s previous books, exploring topics such as loving, kindness, meditation, mindfulness, ``nowness'', letting go, \& working with fear \& other painful emotions. Through the course of this book, readers will learn practical methods for heightening awareness \& overcoming habitual patterns that block compassion.

	\begin{quotation}
		{\it``Meditation practice isn't about trying to throw ourselves away \& become something better. It's about befriending who we are already.''}
		
		-- Thực hành thiền không phải là cố gắng vứt bỏ bản thân \& trở thành một điều gì đó tốt đẹp hơn. Đó là về việc kết bạn với con người của chúng ta.
		
		{\it``Do I prefer to grow up \& relate to life directly, or do I choose to live \& die in fear?''}
		
		-- Tôi thích lớn lên \& trực tiếp tiếp xúc với cuộc sống hay tôi chọn sống \& chết trong sợ hãi?
		
		{\it``A warrior accepts that we can never know what will happen to us next. We can try to control the uncontrollable by looking for security \& predictability, always hoping to be comfortable \& safe. But the truth is that we can never avoid uncertainty. This not-knowing is part of the adventure. It's also what makes us afraid.''}
		
		-- Một chiến binh chấp nhận rằng chúng ta không bao giờ có thể biết được điều gì sẽ xảy ra tiếp theo với mình. Chúng ta có thể cố gắng kiểm soát những điều không thể kiểm soát được bằng cách tìm kiếm sự an toàn \& có thể dự đoán được, luôn hy vọng được thoải mái \& an toàn. Nhưng sự thật là chúng ta không bao giờ có thể tránh khỏi sự không chắc chắn. Việc không biết này là một phần của cuộc phiêu lưu. Đó cũng là điều khiến chúng ta sợ hãi.
	\end{quotation}
	{\sf Editorial reviews.}
	\begin{itemize}
		\item ``Gently, conversationally, \& with humor, {\it Comfortable with Uncertainty} offers strategies for seeing \& thinking differently. For many people the approach is nothing less than transformational.'' -- {\it Boston Globe}
		\item ``{\sc Ch\"odr\"on}'s voice is gently humorous, always kind, \& seemingly infinitely wise.'' -- {\it L.A. Times}
	\end{itemize}
		
	\item \cite{Chodron_scare}. {\sc Pema Ch\"odr\"on}. {\it The Places That Scare You: A Guide to Fearlessness in Difficult Times (Deckled Edge)}. [2650 Amazon ratings][20162 Goodreads ratings]
	
	{\sf Amazon review.} Lifelong guidance for learning to change the way we relate to the scary \& difficult moments of our lives, showing us how we can use all of our difficulties \& fears as a way to soften our hearts \& open us to greater kindness.
	
	We always have a choice in how we react to the circumstances of our lives. We can let them harden us \& make us increasingly resentful \& afraid, or we can let them soften us \& allow our inherent human kindness to shine through. Here {\sc Pema Ch\"odr\"on} provides essential tools for dealing with the many difficulties that life throws our way, teaching us how to awaken our basic human goodness \& connect deeply with others -- to accept ourselves \& everything around us complete with faults \& imperfections. She shows the strength that comes from staying in touch with what's happening in our lives right now \& helps us unmask the ways in which our egos cause us to resist life as it is. If we go to the places that scare us, {\sc Pema} suggests, we just might find the boundless life we're always dreamed of.
	
	{\sf Editorial reviews.}
	\begin{itemize}
		\item ``{\sc Pema Ch\"odr\"on} has once again proven herself to be 1 of the very best working in this field.'' -- {\it Library Journal}
		\item ``{\sc Pema Ch\"odr\"on} demonstrates how effective the Buddhist point of view can be in bringing order into disordered lives.'' -- {\it Publishers Weekly}
		\item ``A lively \& accessible take on ancient techniques for transforming terror \& pain into joy \& compassion.'' -- {\it O, The Oprah Magazine}
	\end{itemize}
	
	\item {\sc Pema Ch\"odr\"on}. {\it How to Medicate: A Practical Guide to Making Friends with Your Mind}.
	
	\item {\sc Pema Ch\"odr\"on}. {\it Living Beautifully: with Uncertainty \& Change}.
	
	\item {\sc Pema Ch\"odr\"on}. {\it Welcoming the Unwelcome: Wholehearted Living in a Brokenhearted World}.
	
	\item {\sc Pema Ch\"odr\"on}. {\it Taking the Leap: Freeing Ourselves from Old Habits \& Fears}.
	
	\item {\sc Pema Ch\"odr\"on}. {\it Start Where You Are: A Guide to Compassionate Living}.
	
	\item {\sc Pema Ch\"odr\"on}. {\it The Pocket Pema Chodron (Shambhala Pocket Classics)}.
	
	\item {\sc Pema Ch\"odr\"on}. {\it The Wisdom of No Escape: How to Love Yourself \& Your World}.
	
	\item {\sc Pema Ch\"odr\"on}. {\it Fail, Fail Again, Fail Better: Wise Advice for Learning into the Unknown}.
	
	\item {\sc Pema Ch\"odr\"on}. {\it{\sc Pema Ch\"odr\"on}'s Compassion Cards: Teachings for Awakening the Heart in Everyday Life}.
	
	\item \cite{Chung2022}. Phạm Văn Chung. {\it Friedrich Nietzsche \& Những Suy Niệm Bên Kia Thiện Ác}.\hfill{\sf[done]}
	
	{\sf Comments.} Khó đọc \& hơi tối nghĩa do ảnh hưởng của tác giả người Việt, không phải do {\sc Friedrich Nietzsche} viết tối nghĩa.
	
	\item \cite{Frankl_meaning}. {\sc Viktor Emil Frankl}. {\it Man's Search for Meaning}.\hfill{\sf[reading]}
	
	{\sf Amazon review.} A book for finding purpose \& strength in times of great despair, the international best-seller is still just as relevant today as when it was 1st published.
	
	This seminal book, which has been called ``1 of the outstanding contributions to psychological thought'' by {\sc Carl Rogers} \& ``1 of the great books of our time'' by {\sc Harold Kushner}, has been translated into $> 50$ languages \& sold over $16\cdot10^6$ copies. ``An enduring work of survival literature,'' according to the {\it New York Times}, {\sc Viktor Frankl}'s riveting\footnote{so interesting or exciting that it holds your attention completely. $=$ {\it engrossing}.} account of his time in the Nazi concentration camps, \& his insightful exploration of the human will to find meaning in spite of the worst adversity, has offered solace\footnote{{\it solace somebody} to make somebody feel better or happier when they are sad or disappointed, $=$ {\it comfort}.} \& guidance to generations of readers since it was 1st published in 1946. At the heart of {\sc Frankl}'s theory of logotherapy (from the Greek word for ``meaning'') is a conviction that the primary human drive is not pleasure, as {\sc Freud} maintained, but rather the discovery \& pursuit of what the individual finds meaningful. Today, as new generations face new challenges \& an ever more complex \& uncertain world, {\sc Frankl}'s classic work continues to inspire us all to find significance in the very act of living, in spite of all obstacles.
	
	{\sf Editorial review.}
	\begin{itemize}
		\item ``1 of the 10 most influential books in America.'' -- Library of Congress{\tt/}Book-of-the-Month Club ``Survey of Lifetime Readers''
		\item ``An enduring work of survival literature.'' -- {\it The New York Times}
		\item ``[{\it Man's Search for Meaning}] might well be prescribed for everyone who would understand our time.'' -- {\it Journal of Individual Psychology}
		\item ``An inspiring document of an amazing man who was able to garner some good from an experience so abysmally bad $\ldots$ Highly recommended.'' -- {\t Library Journal}
		\item ``This is a book I try to read every couple of years. It's 1 of the most inspirational books ever written. What is the meaning of life? What do you have when you think you have nothing? Amazing \& heartbreaking stories. This is a book that should be in everyone's library.'' -- {\sc Jimmy Fallon}
		\item ``This is a book I reread a lot $\ldots$ it gives me hope $\ldots$ it gives me a sense of strength.'' -- {\sf Anderson Cooper}, {\it Anderson Cooper 360{\tt/}CNN}
		\item ``1 of the great books of our time.'' -- {\sc Harold S. Kushner}, author of {\it When Bad Things Happen to Good People}
		\item ``1 of the outstanding contributions of psychological thought in the last 50 years.'' -- {\sc Carl R. Rogers} (1959)
	\end{itemize}
	{\sf About the Author.} {\sc Viktor E. Frankl} was professor of neurology \& psychiatry at the University of Vienna Medical School until his death in 1997. His 29 books have been translated into 21 languages. During World War II, he spent 3 years in Auschwitz, Dachau, \& other concentration camps.
	
	{\sc Harold S. Kushner} is rabbi emeritus at Temple Israel in Natick, Massachusetts, \& the author of bestselling books including {\it When Bad Things Happen to Good People, Living a Life That Matters, When All You've Ever Wanted Isn't Enough}.
	
	{\sc William J. Winslade} is a philosopher, lawyer, \& psychoanalyst who teaches psychiatry, medical ethics, \& medical jurisprudence at the University of Texas Medical School in Galveston.
	\begin{quotation}
		{\it``Life ultimately means taking the responsibility to find the right answer to its problems \& to fulfill the tasks which it constantly sets for each individual.''}
		
		-- Cuộc sống suy cho cùng có nghĩa là có trách nhiệm tìm ra câu trả lời đúng đắn cho những vấn đề của mình \& để hoàn thành những nhiệm vụ mà nó không ngừng đặt ra cho mỗi cá nhân.
		
		{\it``Emotion, which is suffering, ceases to be suffering as soon as we form a clear \& precise picture of it.''}
		
		-- Cảm xúc, tức là đau khổ, sẽ ngừng đau khổ ngay khi chúng ta hình thành một bức tranh rõ ràng \& chính xác về nó.
		
		{\it``If there is a meaning in life at all, then there must be a meaning in suffering. Suffering is an ineradicable part of life, even as fate \& death. Without suffering \& death human life cannot be complete.''}
		
		-- Nếu cuộc sống có ý nghĩa gì đó thì đau khổ cũng phải có ý nghĩa. Đau khổ là một phần không thể xóa bỏ được của cuộc sống, kể cả số phận \& cái chết. Không có đau khổ \& cái chết, cuộc sống con người không thể trọn vẹn.
	\end{quotation}
	
	\item \cite{Frankl_meaning_revised}. {\sc Viktor E. Frankl}. {\it Man's Search for Meaning}.\hfill{\sf[reading]}
	
	\item \cite{Frankl_meaning_VN}. {\sc Viktor E. Frankl}. {\it Man's Search for Meaning -- Đi Tìm Lẽ Sống}.\hfill{\sf[done]}
	
	\item {\sc Viktor Emil Frankl}. {\it Yes To Life: In Spite of Everything}.
	
	\item {\sc Viktor Emil Frankl}. {\it The Unheard Cry for Meaning: Psychotherapy \& Humanism}.
	
	\item {\sc Viktor Emil Frankl}. {\it Embracing Hope: On Freedom, Responsibility \& the Meaning of Life}.
	
	\item {\sc Viktor Emil Frankl}. {\it The Will to Meaning: Foundations \& Applications of Logotherapy}.
	
	\item {\sc Viktor Emil Frankl}. {\it The Doctor \& the Soul: From Psychotherapy to Logotherapy}.
	
	{\sc Viktor Emil Frankl}. {\it Man's Search for Ultimate Meaning}.
	
	\item {\sc Viktor Emil Frankl}. {\it Recollections: An Autobiography}.
	
	\item \cite{Hardy1940,Hardy1992,Hardy2022}. {\sc G. H. Hardy}. {\it A Mathematician's Apology}.\hfill{\sf[done]}
	
	\item \cite{Peterson_rule}. {\sc Jordan B. Peterson}. {\it12 Rules for Life: An Antidote to Chaos}.{\sf[81437 Amazon ratings][237111 Goodreads ratings]}\hfill{\sf[reading]}
	
	{\sf Amazon review.} What does everyone in the modern world need to know? Renowned psychologist {\sc Jordan B. Peterson}'s answer to this most difficult of questions uniquely combines the hard-won truths of ancient tradition with the stunning revelations of cutting-edge scientific research.
	
	Humorous, surprising, \& informative, Dr. {\sc Peterson} tells us why skateboarding boys \& girls must be left alone, what terrible fate awaits those who criticize too easily, \& why you should always pet a cat when you meet one on the street.
	
	What does the nervous system of the lowly lobster have to tell us about standing up straight (with our shoulders back) \& about success in life? Why did ancient Egyptians worship the capacity to pay careful attention as the highest of gods? What dreadful paths do people tread when they become resentful, arrogant, \& vengeful?
	
	Dr. {\sc Peterson} journeys broadly, discussing discipline, freedom, adventure, \& responsibility, distilling the world's wisdom into 12 practical \& profound rules for life. {\it12 Rules for Life} shatters the modern commonplaces of science, faith, \& human nature, while transforming \& ennobling the mind \& spirit of its readers.
	
	\begin{quotation}
		{\it``So, attend carefully to your posture. Quit drooping \& hunching around. Speak your mind. Put your desires forward, as if you had a right to them -- at least the same right as others. Walk tall \& gaze forthrightly ahead. Dare to be dangerous. Encourage the serotonin to flow plentifully through the neural pathways desperate for its calming influence.''}
		
		-- Vì vậy, hãy chú ý cẩn thận đến tư thế của bạn. Đừng ủ rũ \& còng lưng nữa. Nói lên suy nghĩ của bạn. Hãy thể hiện những mong muốn của bạn như thể bạn có quyền đối với chúng - ít nhất là quyền như những người khác. Hãy bước cao \& nhìn thẳng về phía trước. Dám tỏ ra nguy hiểm. Khuyến khích serotonin chảy dồi dào qua các con đường thần kinh đang khao khát tác dụng xoa dịu của nó.
		
		{\it``There is very little difference between the capacity for mayhem \& destruction, integrated, \& strength of character. This is 1 of the most difficult lessons of life.''}
		
		-- Có rất ít sự khác biệt giữa khả năng gây hỗn loạn \& hủy diệt, tích hợp, \& sức mạnh của nhân vật. Đây là 1 trong những bài học khó khăn nhất của cuộc đời.
		
		{\it``It is far better to render Beings in your care competent than to protect them.''}
		
		-- Tốt hơn hết là bạn nên cung cấp năng lực cho các Sinh vật mà bạn chăm sóc hơn là bảo vệ chúng.
		
		{\it``Because they really are rules. \& the foremost rule is that you must take responsibility for your own life. Period.''}
		
		-- Bởi vì chúng thực sự là những quy tắc. \& nguyên tắc quan trọng nhất là bạn phải chịu trách nhiệm về cuộc sống của chính mình. Chấm hết.
	\end{quotation}
	{\sf Editorial review.}
	\begin{itemize}
		\item ``{\sc Jordan Peterson}, has become 1 of the best-known Canadians of this generation. In the intellectual category, he's easily the largest international phenomenon since {\sc Marshall McLuhan} $\ldots$ By combining knowledge of the past with a full-hearted optimism \& a generous attitude toward his readers \& listeners, {\sc Peterson} generates an impressive level of intellectual firepower.'' -- {\sc Robert Fulford}, {\it National Post}
		\item ``Like the best intellectual polymaths, {\sc Peterson} invites his readers to embark on their own intellectual, spiritual, \& ideological journeys into the many topics \& disciplines he touches on. It's a counter-intuitive strategy for a population hooked on the instant gratification of ideological comformity \& social media `likes', but if {\sc Peterson} is right, you have nothing to lose but your own misery.'' -- {\it Toronto Star}
		\item ``In a different intellectual league $\ldots$ {\sc Peterson} can make the most difficult ideas \& make them entertaining. This may be why his YouTube videos have had $35\cdot10^6$ views. He is fast becoming the closest that academia has to a rock star.'' -- {\it The Observer}
		\item ``Grow up \& man up is the message from this rock-star psychologist $\ldots$ [A] hardline self-helf manual of self-reliance, good behavior, self-betterment \& individualism that probably reflects his childhood in rural Canada in the 1960s. As with all self-help manuals, there's always a \fbox{kernel of truth}. Formerly a Harvard professor, now at the University of Toronto, {\sc Peterson} retains that whiff of cowboy philosophy -- 1 essay is a homily on doing 1 thing every day to improve yourself. Another, on bringing up little children to behave, is excellent $\ldots$ [{\sc Peterson}] twirls ideas around like a magician.'' -- {\sc Melanie Reid}, {\it The Times}
		\item ``You don't have to agree with [{\sc Peterson}'s politics] to like this book for, once you discard the self-help label, it becomes fascinating. {\sc Peterson} is brilliant on many subjects $\ldots$ So what we have here is a baggy, aggressive, in-your-face, get-real book that, ultimately, is an attempt to lead us back to what {\sc Peterson} sees as the true, the beautiful, \& the good -- i.e., God. In the highest possible sense of the term, I suppose it is a self-help book $\ldots$ Either way, it's a rocky read, but nobody ever said God was easy.'' -- {\sc Bryan Appleyard}, {\it The Times}
		\item ``1 of the most eclectic\footnote{not following one style or set of ideas but choosing from or using a wide variety. chiết trung, chủ nghĩa chiết trung.} \& stimulating public intellectuals at large today, fearless \& impassioned.'' -- {\it The Guardian}
		\item ``Someone with not only humanity \& humor, but serious depth \& substance $\ldots$ {\sc Peterson} has a truly cosmopolitan \& omnivorous intellect, but one that recognizes that things need grounding in a home if they are ever going to be meaningful grasped $\ldots$ As well as being funny, there is a burning sincerity to the man which only the most withered cynic could suspect.'' -- {\it The Spectator}
		\item ``{\sc Peterson} has become a kind of secular prophet who, in an era of lobotomized conformism, thinks out of the box $\ldots$ His message is overwhelmingly vital.'' -- {\sc Melanie Philips}, {\it The Times}
	\end{itemize}	
	
	\item \cite{Peterson_rule_VN}. {\sc Jordan B. Peterson}. {\it12 Rules for Life: An Antidote to Chaos -- 12 Quy Luật Cuộc Đời: Thần Dược Cho Cuộc Sống Hiện Đại}.\hfill{\sf[done]}
	
	\item \cite{Peterson_beyond_order}. {\sc Jordan B. Peterson}. {\it Beyond Order: 12 More Rules for Life}. {\sf[17337 Amazon ratings][26450 Goodreads ratings]}
	
	{\sf Amazon review.} The companion volume to {\it12 Rules for Life} offers further guidance on the perilous path of modern life.
	
	In {\it12 Rules for Life}, clinical psychologist \& celebrated professor at Harvard \& University of Toronto Dr. {\sc Jordan B. Peterson} helped millions of readers impose order on the chaos of their lives. Now, in this hold sequel, {\sc Peterson} delivers 12 more lifesaving principles for resisting the exhausting toll that our desire to order the world inevitably takes.
	
	In a time when the human will increasingly imposes itself over every sphere of life -- from our social structures to our emotional states -- {\sc Peterson} warns that too much security is dangerous. What's more, he offers strategies for overcoming the cultural, scientific, \& psychological fores causing us to tend toward tyranny, \& teaches us how to rely instead on our instinct to find meaning \& purpose, even-- \& especially -- when we find ourselves powerless.
	
	While chaos, in excess, threatens us with instability \& anxiety, unchecked order can petrify us into submission. {\it Beyond Order} provides a call to balance these 2 fundamental principles of reality itself, \& guides us along the straight \& narrow path that divides them.
	\begin{quotation}
		{\it``People remain mentally healthy not merely because of the integrity of their own minds, but because they are constantly being reminded how to think, act, \& speak by those around them.''}
		
		-- Con người giữ được tinh thần khỏe mạnh không chỉ vì sự chính trực trong tâm trí của họ mà còn vì họ thường xuyên được những người xung quanh nhắc nhở về cách suy nghĩ, hành động \& nói năng.
		
		{\it``Humility: It is better to presume ignorance \& invite learning than to assume sufficient knowledge \& risk the consequent blindness.''}
		
		-- Khiêm tốn: Thà cho rằng mình là người thiếu hiểu biết \& mời gọi học tập hơn là cho rằng có đủ kiến thức \& có nguy cơ bị mù quáng.
		
		{\it``{\sc Freud} \& {\sc Jung}, with their intense focus on the autonomous individual psyche, placed too little focus on the role of the community in the maintenance of personal mental health.''}
		
		-- {\sc Freud} \& {\sc Jung}, với sự tập trung cao độ vào tâm lý tự chủ của cá nhân, đã tập trung quá ít vào vai trò của cộng đồng trong việc duy trì sức khỏe tâm thần cá nhân.
	\end{quotation}
	{\sf Editorial review.}
	\begin{itemize}
		\item ``We live in a time when so many young (\& not so young) people feel lost $\ldots$ Mr. {\sc Peterson} talks about the attitudes that will help find the path. It is not a politically correct or officially appro ved path, but it is an intensely practical \& yet heightened one: This life you're living has meaning.'' -- {\sc Peggy Noonan}, {\it Wall Street Journal}
		\item ``{\sc Jordan Peterson} is universally revered -- \& feared -- for his incredible intellect \& emotional insight.'' -- {\sc Dave Rubin}, host of The Rubin Report \& author of {\it Don't Burn This Book}
		\item ``The Peterson way is a harsh way, but it is an idealistic way -- \& for millions of young men, it turns out to be the perfect antidode to the cocktail of coddling \& accusation in which they are raised.'' -- {\sc David Brooks}, {\it New York Times}
		\item ``The worlds needs {\sc Jordan Peterson}.'' -- {\sc Douglas Murray}, author of {\it The Madness of Crowds}
	\end{itemize}
	{\sf About the Author.} Dr. {\sc Jordan B Peterson} is the bestselling author of {\it12 Rules for Life}, which has sold more than $5\cdot10^6$ copies worldwide. After working for decades as a clinical psychologist \& a professor at Harvard \& the University of Toronto, Peterson has become 1 of the world's most influential public intellectuals. His YouTube videos \& podcasts have gathered a worldwide audience of hundreds of millions, \& his global book tour reached more than 250000 people in major cities across the globe. With his students \& colleagues, he has published $> 100$ scientific papers, \& his 1999 book {\it Maps of Meaning} revolutionized the psychology of religion. He lives in Toronto, Ontario with his family.
	
	{\sc Jordan B. Peterson}, raised \& toughened in the frigid wastelands of Northern Alberta, has flown a hammer-head roll in a carbon-fiber stunt-plane, explored an Arizona meteorite crater with astronauts, \& built a Kwagu'l ceremonial bighouse on the upper floor of his Toronto home after being invited into \& named by that Canadian First Nation. He's taught mythology to lawyers, doctors \& business people, consulted for the UN Secretary General, helped his clinical clients manage depression, obsessive-compulsive disorder, anxiety, \& schizophrenia, served as an adviser to senior partners of major Canadian law firms, \& lectured extensively in North America \& Europe. With his students \& colleagues at Harvard \& the University of Toronto, Dr. Peterson has published over a hundred scientific papers, transforming the modern understanding of personality, while his book Maps of Meaning: The Architecture of Belief revolutionized the psychology of religion. The author lives in Toronto, ON. \url{www.jordanbpeterson.com}.
	
	\item \cite{Peterson_beyond_order_VN}. {\sc Jordan B. Peterson}. {\it Beyond Order: 12 More Rules for Life -- Vượt Lên Trật Tự: 12 Quy Tắc cho Cuộc Sống}.\hfill{\sf[done]}
	
	\item \cite{Peterson_map_meaning}. {\sc Jordan B. Peterson}. {\it Maps of Meaning}. {\sf[3372 Amazon ratings][5972 Goodreads ratings]}\hfill{\sf[reading]}
	
	{\sf Amazon review.} Why have people from different cultures \& eras formulated myths \& stories with similar structures? What does this similarity tell us about the mind, morality, \& structure of the world itself? From the author of {\it12 Rules for Life: An Antidote to Chaos} comes a provocative hypothesis that explores the connection between what modern neuropsychology tells us about the brain \& what rituals, myths, \& religious stories have long narrated. A cutting-edge work that brings together neuropsychology, cognitive science, \& Freudian \& Jungian approaches to mythology \& narrative, {\it Maps of Meaning} presents a rich theory that makes the wisdom \& meaning of myth accessible to the critical modern mind.
	
	{\sf Editorial Reviews.}
	\begin{itemize}
		\item ``The book reflects its author's profound moral sense \& vast erudition in areas ranging from clinical psychology to scripture \& a good deal of personal soul-searching \& experience $\ldots$ with patients who include prisoners, alcoholics, \& the mentally ill.'' -- {\it Montreal Gazette}
		\item ``This is not a book to be abstracted \& summarized. Rather it should be read at leisure $\ldots$ \& employed as a stimulus \& reference to expand one's own maps of meaning. I plan to return to {\sc Peterson}'s musings \& mapping many times over the next few years.'' -- {\it Am JPsychiatry}
		\item ``$\ldots$ a brilliant enlargement of our understanding of human motivation $\ldots$ a beautiful work.'' -- {\sc Sheldon H. White}, Harvard University
		\item ``$\ldots$ unique $\ldots$ a brilliant new synthesis of the meaning of mythologies \& our human need to relate in story form the deep structure of our experiences.'' -- {\sc Keith Oatley}, University of Toronto
	\end{itemize}
	{\sf From Inside Flap.} Why would people in different places \& times formulate myths \& stories with similar symbols \& meanings? Are groups of people with different religious or ideological beliefs doomed to eternal conflict? Are the claims of science \& religion truly irreconcilable? What might be done to decrease the individual propensity for group-fostered cruelty? {\it Maps of Meaning} addresses these questions with a provocative new hypothesis that explores the connection between what modern neuropsychology tells us about the brain \& what rituals, myths, \& religious stories have long narrated. {\sc Peterson}'s ambitious interdisciplinary odyssey draws insights from the worlds of religion, cognitive science, \& Jungian approaches to mythology \& narrative. {\it Maps of Meaning} offers a critical guide to the riches of archaic \& modern thought \& invaluable insights into human motivation \& cognition.
	
	{\sf From Back Cover.} ``{\sc Jordan Peterson}'s book is a brilliant enlargement of our understanding of human motivation. He follows a path that has been recommended by many scientist-scholars in the past -- but one that is, in practice, so extraordinarily demanding that it is hardly ever done well. {\sc Peterson} synthesizes research \& scholarly literatures ranging from neuroscience to archaeology. He aligns finds of those literatures with the writings of such authors as {\sc Jung, Nietzsche, Dostoevsky, \& Solzhenitsyn}. There is loving detail in this book -- reflection, thoughtfulness, careful study, a passionate desire to understand. This is a beautiful work.'' -- {\sc Sheldon H. White}, Chair, Department of Psychology, Harvard University
	
	\item {\sc Jordan B. Peterson}. {\it We Who Wrestle with Gods: Perceptions of the Divine}.
	
	\item \cite{Popper_logic_khoa_hoc}. {\sc Karl Popper}. {\it The Logic of Scientific Discovery -- Logic Của Sự Khám Phá Khoa Học}.\hfill{\sf[done]}
	
	\item \cite{Sinek_start_why}. {\sc Simon Sinek}. {\it Start with Why: How Great Leaders Inspire Everyone to Take Action -- Bắt Đầu Với Câu Hỏi Tạo Sao: Nghệ Thuật Truyền Cảm Hứng Trong Kinh Doanh}.\hfill{\sf[done]}
	
	\item \cite{Sinek_Mead_Docker_why}. {\sc Simon Sinek, David Mead, Peter Docker}. {\it Find Your Why: A Practical Guide for Discovering Purpose for You \& Your Team -- Khám Phá Sứ Mệnh Với Câu Hỏi Tại Sao}.\hfill{\sf[done]}
	
	\item \cite{Thoreau_Walden}. {\sc Henry David Thoreau}. {\it Walden}.\hfill{\sf[reading]}
	
	\item \cite{Thoreau_Walden_VN}. {\sc Henry David Thoreau}. {\it Walden -- Một Mình Sống Trong Rừng}.\hfill{\sf[done]}
	
	\item \cite{Trung_dung_viec}. {\sc Giản Tư Trung}. {\it Đúng Việc: Một Góc Nhìn Về Câu Chuyện Khai Minh}.\hfill{\sf[done]}
	
	\item {\sc Giản Tư Trung}. {\it Sư Phạm Khai Phóng}.
	
	\item \cite{Trung_van_hoa}. {\sc Giản Tư Trung}. {\it Quản Trị Bằng Văn Hóa}.\hfill{\sf[done]}
	
	\item \cite{Warren_purpose_driven_life}. {\sc Rick Warren}. {\it The Purpose Driven Life: What on Earth Am I Here For?}. {\sf[16299 Amazon ratings] [279493 Goodreads ratings]}
	
	{\sf Amazon review.} The {\it New York Times \#1} bestselling book by Pastor {\sc Rick Warren} that helps you understand \& live out the purpose of your life.
	
	Before you were born, God already planned your life. God longs for you to discover the life he uniquely created you to live -- here on earth, \& forever in eternity. Let {\it The Purpose Driven Life} show you how. As 1 of the bestselling nonfiction books in history, with $> 35\cdot10^6$ copies sold, {\it The Purpose Driven Life} is far more than just a book; it's the road map for your spiritual journey. A journey that will transform your life.
	
	Designed to be read in 42 days, each chapter provides a daily meditation \& practical steps to help you discover \& live out your purpose, starting with exploring 3 of life's most pressing questions:
	\begin{itemize}
		\item The Question of Existence: Why am I alive?
		\item The Question of Significance: Does my life matter?
		\item The Question of Purpose: What on earth am I here for?
	\end{itemize}
	\begin{quotation}
		{\it``It is only in God that we discover our origin, our identity, our meaning, our purpose, our significance, \& our destiny. Every other path leads to a dead end.''}
		
		-- Chỉ trong Thiên Chúa, chúng ta mới khám phá được nguồn gốc, bản sắc, ý nghĩa, mục đích, ý nghĩa \& vận mệnh của mình. Mọi con đường khác đều dẫn đến ngõ cụt.
		
		{\it``You weren't put on earth to be remembered. You were put here to prepare for eternity.''}
		
		-- Bạn không được sinh ra trên trái đất để được ghi nhớ. Bạn được đặt ở đây để chuẩn bị cho cõi vĩnh hằng.
		
		{\it``Fear is a self-imposed prison that will keep you from becoming what God intends for you to be.''}
		
		-- Sợ hãi là một nhà tù tự áp đặt sẽ ngăn cản bạn trở thành con người mà Chúa dự định cho bạn trở thành.
		
		{\it``We are products of our past, but we don't have to be prisoners of it.''}
		
		-- Chúng ta là sản phẩm của quá khứ nhưng chúng ta không nhất thiết phải trở thành tù nhân của quá khứ.
		
		{\it``Real security can only be found in that which can never be taken from you -- your relationship with God.''}
		
		-- Sự an toàn thực sự chỉ có thể được tìm thấy ở điều không bao giờ có thể bị lấy đi khỏi bạn - mối quan hệ của bạn với Chúa.
	\end{quotation}
	{\sf Editorial review.} ``Movie stars \& political leaders aren't the only ones turning to {\sc Rick Warren} for spiritual guidance. Millions of people -- from NBA \& LPGA players to corporate executives to high school students to prison inmates -- meet regularly to discuss {\it The Purpose Driven Life}.'' -- {\it Time}
	
	{\sf About the Author.} A Time magazine cover article named Rick Warren the most influential spiritual leader in America \& 1 of the 100 most influential people in the world. Tens of millions of copies of Pastor Rick's books have been published in 200 languages. His best-known books, The Purpose Driven Life \& The Purpose Driven Church, were named three times in national surveys of pastors (by Gallup, Barna, \& Lifeway) as the two most helpful books in print. Rick \& his wife, Kay, founded Saddleback Church, the Purpose Driven Network, the PEACE Plan, \& Hope for Mental Health. He is the cofounder of Celebrate Recovery with John Baker. Pastor Rick has spoken in 165 nations. He has spoken at the United Nations, US Congress, numerous parliaments, the World Economic Forum, TED, Aspen Institute, \& lectured at Oxford, Cambridge, Harvard, \& other universities. Rick is executive director of Finishing the Task, a global movement of denominations, organizations, churches, \& individuals working together on the Great Commission goals of ensuring that everyone everywhere has access to a Bible, a believer, \& a local body of Christ.	
\end{enumerate}

%------------------------------------------------------------------------------%

\subsection{Spirituality}

\begin{enumerate}
	\item \cite{Hanh_silence}. {\sc Thích Nhật Hạnh}. {\it Silence: The Power of Quiet in a World Full of Noise}.
	
	\item \cite{Hanh_silence_VN}. {\sc Thích Nhật Hạnh}. {\it Silence: The Power of Quiet in a World Full of Noise -- Tĩnh Lặng: Sức Mạnh Tĩnh Lặng Trong Thế Giới Huyên Náo}.\hfill{\sf[done]}
	
	\item \cite{Nguyen_dont_believe_think}. {\sc Joseph Nguyen}. {\it Don't Believe Everything You Think: Why Your Thinking Is The Beginning \& End Of Suffering}. {\sf[12942 Amazon ratings][15509 Goodreads ratings]}
	
	{\sf Amazon review.} Learn how to overcome anxiety, self-doubt \& self-sabotage without needing to rely on motivation or willpower.
	
	In this book, you'll discover the root cause of all psychological \& emotional suffering \& how to achieve freedom of mind to effortlessly create the life you've always wanted to live. \fbox{Although pain is inevitable, suffering is optimal.} This book offers a completely new paradigm \& understanding of where our human experiences comes from, allowing us to end our own suffering \& create how we want to feel at any moment. In this book, you'll discover:
	\begin{itemize}
		\item The root cause of all psychological \& emotional suffering \& how to end it
		\item How to become unaffected by negative thoughts \& feelings
		\item How to experience unconditional love, peace, \& joy in the present, no matter what our external circumstances look like
		\item How to instantly create a new experience of life if you don't like the one you're in right now
		\item How to break free from a negative thought loop when we inevitably get caught in one
		\item How to let go of anxiety, self-doubt, self-sabotage, \& any self-destructive habits
		\item How to effortlessly create from a state of abundance, flow, \& ease.
		\item How to develop the superpower of being okay with not knowing \& uncertainty
		\item How to access your intuition \& inner wisdom that goes beyond the limitations of thinking
	\end{itemize}
	No matter what has happened to you, where you are from, or what you have done, you can still find total peace, unconditional love, complete fulfillment, \& an abundance of joy in your life.
	
	No person is an exception to this, Darkness only exists because of the light, which means even in our darkest hour, light must exist.
	
	Within the pages of this book, contains timeless wisdom to empower you with the understanding of our mind's infinite potential to create any experience of life that we want no matter the external circumstances.
	
	`Don't Believe Everything You Think' is not about rewiring your brain, rewriting your past, positive thinking or anything of the sort.
	
	We cannot solve our problems with the same level of consciousness that created them. Tactics are temporary. An expansion of consciousness is permanent.
	
	This book was written to help you go beyond your thinking \& discover the truth of what you already intuitively know deep inside your soul.
	\begin{quotation}
		{\it``Thought is not reality; yet it is through thought that our realities are created.''}
		
		-- Suy nghĩ không là thực tế; tuy nhiên chính nhờ tư duy mà thực tế của chúng ta được tạo ra.
		
		{\it``Our feelings do not come from external events, but from our own thinking about the events. Therefore, we can only ever feel what we are thinking.''}
		
		-- Cảm xúc của chúng ta không đến từ các sự kiện bên ngoài mà đến từ suy nghĩ của chính chúng ta về các sự kiện đó. Vì vậy, chúng ta chỉ có thể cảm nhận được những gì chúng ta đang nghĩ.
		
		{\it``Therefore, it's not WHAT we're thinking about that is causing us suffering, but THAT we are thinking.''}
		
		-- Therefore, it's not WHAT we're thinking about that is causing us suffering, but THAT we are thinking.
	\end{quotation}
	{\sf Editorial review.}
	\begin{itemize}
		\item ``{\sc Joseph Nguyen} has represented a simple guide to a much happier life. Although I have read similar works conveying these same insights, Mr. {\sc Nguyen} has the ability to crystallize these concepts in a way that I have not encountered before. Beautifully written, he gets his message across better than any other self-help author I have read \& he also gives very practical advice on implementing this practice in your daily life. Excellent work! I would highly recommend this book to everyone. The audiobook is enjoyable too, \& I enjoy hearing his voice. I would like to read more from this author.'' -- {\sc Laura Spollen}
		\item ``I am so impressed \& in love. What a wonderful contribution to the movement of human consciousness. There is an abundance of rich perspectives packed into this simple tiny little book. If you are wanting to explore powerful ways of integrating mindfulness practices into your daily life, look no further. This sweet little gem will have a permanent place, not only on your bookshelf but in your heart as well. I will definitely be reading this one over \& over \& over again. The insight that I've gained is priceless. There's just nothing better than the continuous unfolding of awareness around our human experience. I'm forever grateful to {\sc Joseph} \& all of those who supported him in writing this book. I can't imagine how many lives will be changed because of the words on these pages. What a beautiful accomplishment. Thank you from the bottom of my heart.'' {\sc Kelly Love}
		\item ``This book is a journey unfolding our minds to have clarity \& understanding, knowing what is \& not what if, as the mind cast an illusion in our daily lives to sieve out thoughts \& thinking \& to know the difference. I enjoyed the explanations given for each chapters \& it's raise your awareness \& expand your consciousness to bring you to a state of non-thinking where you actually feel joy, peace, \& harmony within. It is not easy to stay in the state of non-thinking but the guides given by the author does help to center your thoughts not your thinking to be in the state of non-thinking. This book has help me to understand myself better \& a journey of discovery for me. I think the author for such an amazing work, simple, \& profound. I like the part how the mind thinks \& says it is not possible or the guides given are to simple \& will not work out for me but when I deep dive within, accepting my negative thinking \& acknowledging it, makes a lot of difference in fighting it or resisting it. I recommend anyone who is on the journey of self discovery \& how to deal with your mind, this is the book for you.'' -- {\sc Jay R.}
	\end{itemize}
	{\sf About the Author.} {\sc Joseph Nguyen} is a spiritual thought leader who has a mission of helping others realize their divine purpose, unlock the infinite potential of their own mind, \& live an abundant life free from psychological suffering. He spends most of his time writing, coaching, teaching, speaking, \& sharing timeless wisdom to help people discover their own divinity from within \& how they are the answer they've been looking for their entire lives.
	\item {\sc Joseph Nguyen}. {\it Beyond Thoughts: an exploration of who we are beyond our minds (Beyond Suffering)}. {\sf[498 Amazon ratings][525 Goodreads ratings]}
	
	{\sf Amazon review.} {\it Beyond Thoughts} is a poetry collection that explores the root cause of anxiety, depression, guilt, shame, negative thinking, \& emotional suffering to help you heal.
	
	This book will take you on a beautiful journey of self-discovery, self-love, happiness, hope, \& deep healing to help you find inner peace in a simple, yet profound way.
	
	Here's what you'll discover:
	\begin{itemize}
		\item How to let go of negative thinking, anxiety, guilt, \& shame
		\item How to hold space for yourself \& all emotions so that you are less affected by them
		\item How to heal from the past \& let go of the fear of the future
		\item How to end the vicious cycle of self-judgment
		\item How to love yourself \& others unconditionally
		\item How to find yourself \& discover who you truly are
		\item How to let go of self-limiting beliefs
		\item How to find happiness \& peace in the present moment no matter what you are going through
		\item How to trust yourself \& develop strength, confidence, \& courage in yourself again
		\item How to become conscious of the subconscious, so that it stops controlling you \& you can finally be liberated
		\item How to not only be okay in the unknown, but to thrive in it to create an abundant life filled with love \& joy
	\end{itemize}
	There is something within you that is greater than everything you've ever been through. There is a deep part of you that knows this, which is what drew you here.
	
	Beyond everything you think is your true essence that has been patiently waiting to be discovered.
	
	Welcome home.
	\item {\sc Joseph Nguyen}. {\it Boundaries $=$ Freedom: How To Create Boundaries That Set You Free Without Feeling Guilty (Beyond Suffering)}. {\sf[261 Amazon ratings][213 Goodreads ratings]}
	
	{\it Everything You Need To Know About Setting Boundaries To Find Peace In $< 100$ Pages.} Our emotional suffering does not come from the events in our lives, but from how we respond to them. Each boundary we set helps to create space for a different behavior \& response when we are faced with the same situation that causes our suffering.
	
	The space we create within ourselves is the source of infinite possibilities for us to choose peace \& a new experience of life.
	
	This book was written to help you learn how to create this space so that you will always have room for love \& joy in your life, no matter what happens in it.
	
	Here's what you'll discover:
	\begin{itemize}
		\item How to say `no' without feeling guilty
		\item How to peacefully communicate your boundaries in a way where others will understand
		\item How to become okay with disappointing others, so you be free from being affected by what others think of you
		\item How to create boundaries that prevent you from burning out
		\item How you can find peace no matter what happens in your life
		\item The most important \& life-changing boundary to create stop negative thinking
		\item How to identify the most important boundaries in each area of your life
		\item How to honor your boundaries so that there are fewer relapses into old, destructive patterns
		\begin{quotation}
			{\it``People are not inspired by what we say or do but by how we make them feel.''}
			
			-- Mọi người không được truyền cảm hứng bởi những gì chúng ta nói hoặc làm mà bởi cách chúng ta khiến họ cảm thấy.
			
			{\it``By constantly saying yes to everyone else we are inadvertently saying no to ourselves.''}
			
			-- Bằng cách liên tục nói có với người khác, chúng ta đang vô tình nói không với chính mình.
			
			{\it``The best way to help the world is to help yourself 1st, not by sacrificing yourself.''}
			
			-- Cách tốt nhất để giúp đỡ thế giới là giúp đỡ chính mình trước tiên chứ không phải bằng cách hy sinh bản thân.
		\end{quotation}
	\end{itemize}
	
	\item \cite{Ruiz_4_agreements}. {\it The Four Agreements: A Practical Guide to Personal Freedom (A Toltec Wisdom Book)}.\hfill{\sf[done]}
	
	\item \cite{Ruiz_Mills_4_agreements_VN}. Don Miguel Ruiz, Janet Mills. {\it The Four Agreements: A Practical Guide to Personal Freedom (A Toltec Wisdom Book) -- 4 Thỏa Ước: Bí Quyết Sống Tự Do, Bình An, Hạnh Phúc Giữa Thế Giới Bất Định}.\hfill{\sf[done]}
	
	\item \cite{Ruiz_Ruiz_5th_agreement}. {\it The Fifth Agreements: A Practical Guide to Self-Mastery (A Toltec Wisdom Book)}.\hfill{\sf[done]}
	
	\item \cite{Ruiz_mastery_self}. {\sc don Miguel Ruiz Jr.} {\it The Mastery of Self: A Toltec Guide to Personal Freedom (Toltec Mastery Series)}.
	
	\item \cite{Ruiz_mastery_self_VN}. {\sc don Miguel Ruiz Jr.} {\it The Mastery of Self: A Toltec Guide to Personal Freedom (Toltec Mastery Series) -- Hành Trình Thấu Hiểu Bản Thân \& Tìm Thấy Tự Do}.\hfill{\sf[done]}
	
	\item \cite{Tolle_oneness}. {\sc Eckhart Tolle}. {\it Oneness With All Life}.
	
	\item \cite{Tolle_oneness_VN}. {\sc Eckhart Tolle}. {\it Oneness With All Life -- Hợp Nhất với Vũ Trụ}.\hfill{\sf[done]}
	
	\item \cite{Tolle_now}. {\sc Eckhart Tolle}. {\it The Power of Now: A Guide to Spiritual Enlightenment}.
	
	\item \cite{Tolle_now_VN}. {\sc Eckhart Tolle}. {\it The Power of Now: A Guide to Spiritual Enlightenment -- Sức Mạnh của Hiện Tại}.\hfill{\sf[done]}
	
	\item \cite{Tolle_practice_now}. {\sc Eckhart Tolle}. {\it Practicing The Power of Now: Essential Teachings, Meditations, \& Exercises From The Power of Now}.
	
	\item \cite{Tolle_practice_now_VN}. {\sc Eckhart Tolle}. {\it Practicing The Power of Now: Essential Teachings, Meditations, \& Exercises From The Power of Now -- Trải Nghiệm Sức Mạnh Hiện Tại}.\hfill{\sf[done]}
	
	\item \cite{Tolle_earth}. {\sc Eckhart Tolle}. {\it A New Earth: Awakening to Your Life's Purpose}.
	
	\item \cite{Tolle_earth_VN}. {\sc Eckhart Tolle}. {\it A New Earth: Awakening to Your Life's Purpose -- Thức Tỉnh Mục Đích Sống}.\hfill{\sf[done]}
	
	\item \cite{Tolle_stillness}. {\sc Eckhart Tolle}. {\it Stilless Speaks}.	
	
	\item \cite{Tolle_stillness_VN}. {\sc Eckhart Tolle}. {\it Stillness Speaks -- Sức Mạnh của Tĩnh Lặng}.\hfill{\sf[done]}
\end{enumerate}

%------------------------------------------------------------------------------%

\subsection{Miscellaneous}

\begin{enumerate}
	\item \cite{Anderson_TED}. {\sc Chris Anderson}. {\it TED Talks: The Official TED Guide to Public Speaking: Tips \& Tricks for Giving Unforgettable Speeches \& Presentations}.\hfill{\sf[reading]}
	
	\item \cite{Anderson_TED_VN}. {\sc Chris Anderson}. {\it TED Talks: The Official TED Guide to Public Speaking: Tips \& Tricks for Giving Unforgettable Speeches \& Presentations -- Hùng Biện Kiểu TED: Bí Quyết Diễn Thuyết Trước Đám Đông ``Chuẩn'' TED}.\hfill{\sf[done]}
	
	\item \cite{Aoun_robot-proof}. {\sc Joseph E. Aoun}. {\it Robot-Proof: Higher Education in the Age of Artificial Intelligence}.
	
	\item \cite{Aoun_robot-proof_VN}. {\sc Joseph E. Aoun}. {\it Robot-Proof: Higher Education in the Age of Artificial Intelligence -- Chạy Đua Với Robot: Học Tập Thời Trí Tuệ Nhân Tạo}.\hfill{\sf[done]}
	
	\item \cite{Dixit_Nalebuff2010}. {\sc Avinash K. Dixit, Barry J. Nalebuff}. {\it The Art of Strategy: A Game Theorist's Guide to Success in Business \& Life}.
	
	\item \cite{Dixit_Nalebuff_strategy}. {\sc Avinash K. Dixit, Barry J. Nalebuff}. {\it The Art of Strategy: A Game Theorist's Guide to Success in Business \& Life -- Nghệ Thuật Tư Duy Chiến Lược: Ứng Dụng Của Lý Thuyết Trò Chơi Trong Công Việc \& Cuộc Sống}.\hfill{\sf[done]}
		
	\item \cite{Foer_remember}. {\sc Joshua Foer}. {\it Moonwalking with Einstein: The Art \& Science of Remembering Everything}. {\sf[10340 Amazon ratings][91169 Goodreads ratings]}
	
	{\sf Amazon review.} The blockbuster phenomenon that charts an amazing journey of the mind while revolutionizing our concept of memory. An instant bestseller that has now become a classic, {\it Moonwalking with {\sc Eistein}} recounts {\sc Joshua Foer}'s yearlong quest to improve his memory under the tutelage of top ``mental athletes.'' He draws on cutting-edge research, a surprising cultural history of remembering, \& venerable tricks of the mentalist's trade to transform our understanding of human memory. From the United States Memory Championship to deep within the author's own mind, this is an electrifying work of journalism that reminds us that, in every way that matters, \fbox{we are the sum of our memories}.
	\begin{itemize}
		\item ``Life seems to speed up as we get older because life gets less memorable as we gets older.''
		
		-- Cuộc sống dường như trôi nhanh hơn khi chúng ta già đi vì cuộc sống trở nên ít đáng nhớ hơn khi chúng ta già đi.
		\item ``When you want to get good at something, how you spend your time practicing is far more important than the amount of time you spend.''
		
		-- Khi bạn muốn giỏi một điều gì đó, cách bạn dành thời gian luyện tập quan trọng hơn nhiều so với lượng thời gian bạn dành ra.
		\item ``Chunking is a way to decrease the number of items you have to remember by increasing the size of each item.''
		
		-- Chia nhỏ thông tin là cách giảm số lượng thông tin bạn phải nhớ bằng cách tăng kích thước của mỗi thông tin.		
		\item ``The brain best remembers things that are repeated, rhythmic, rhyming, structured, \& above all easily visualized.''
		
		-- Bộ não ghi nhớ tốt nhất những thứ được lặp đi lặp lại, có nhịp điệu, vần điệu, có cấu trúc và trên hết là dễ hình dung.
	\end{itemize}
	{\sf Editorial reviews.}
	\begin{itemize}
		\item ``Absolutely phenomenal $\ldots$ Part of the beauty of this book is that it makes clear how memory \& understanding are not 2 different things. Building up the ability to reason \& the ability to retain information go hand in hand $\ldots$ The book reminds us that we all start off with pretty much the same tools for the most part, \& we can be intentional about strengthened them, or not.'' -- {\sc Bill Gates}
		\item ``Captivating $\ldots$ His narrative is smart \& funny \&, like the work of Dr. {\sc Oliver Sacks}, it's informed by a humanism that enables its author to place the mysteries of the brain within a larger philosophical \& cultural context.'' -- {\sc Michiko Kakutani}, {\it The New York Times}
		\item ``His passionate \& deeply engrossing book $\ldots$ is a resounding tribute to the muscularity of the mind $\ldots$ In the end, {\it Moonwalking with {\sc Einstein}} reminds us that though brain science is a wild frontier \& the mechanics of memory little understood, our minds are capable of epic achievements.'' -- {\it The Washington Post}
		\item ``{\sc Joshua Foer}'s book $\ldots$ is both fun \& reassuring. All it takes to have a better memory, he contends, are a few tricks \& a good erotic imagination.'' -- {\sc Maureen Dowd}, {\it The New York Times}
		\item ``Highly entertaining.'' -- {\sc Adam Gopnik}, {\it The New Yorker}
		\item ``It's delightful to travel with him on this unlikely journey, \& his entertaining treatment of memory as both sport \& science is spot on $\ldots$ {\it Moonwalking with {\sc Einstein}} proves uplifting: It shows that with motivation, focus, \& a few clever tricks, our minds can do rather extraordinary things.'' -- {\it The Wall Street Journal}
		\item ``It's a terrific book: sometimes weird but mostly smart, funny, \& ultimately a love exploration of the ways that we preserve our lives \& our world in the golden amber of human memory.'' -- {\sc Deborah Blum}, {\it New Scientist}
		\item ``{\sf Foer}'s book is relevant \& entertaining as he shows us ways we can unlock our own talent to remember more.'' -- {\it USA Today}
		\item ``A fascinating scientific analysis of mnemonic mysteries. What we remember, [{\sc Foer}] says, defines who we are.'' -- {\it Entertainment Weekly}
		\item ``Sprightly, entertaining $\ldots$ [{\sc Foer}] has a gift for communicating fairly complex ideas in a manner that is palatable without being patronizing.'' -- {\it Financial Times}
		\item ``[An] inspired \& well-written debut book about not just memorization, but about what it means to be educated \& the best way to become so, about expertise in general, \& about the not-so-hidden `secrets' of acquiring skills.'' -- {\it The Seattle Times}
		\item ``[An] instant bestseller.'' -- {\it San Francisco Chronicle}
		\item ``Funny, curious, erudite, \& full of useful details about ancient techniques of training memory.'' -- {\it The Boston Globe}
		\item ``With originality, high energy, \& an appealing blend of chutzpath \& humility, [{\sf Foer}] writes of his own adventures \& probes the history \& literature of memory, the science of how the brain functions, \& the connections between memory, identity, \& culture $\ldots$ {\it Moonwalking with {\sc Einstein}} $\ldots$ is engaging \& timely.'' -- {\it The Jewish Week}
		\item ``A smart, thoughtful, engaging book.'' -- {\it The Portland Oregonian}
		\item ``Charming $\ldots$ The book is part of a grand tradition, the writer as participating athlete, reminiscent of {\sc George Plimpton} taking up football in {\it Paper Lion}.'' -- {\it O, The Oprah Magazine}
		\item ``[A] wonderful 1st book.'' -- {\it Newcity}
		\item ``Fascinating.'' -- {\it Town \& Country}
		\item ``For 1 year, {\sc Foer} tried to attain total recall, extracting secrets from the top researchers, the real Rain Man, \& the world's memory champs. He triumphed, both in his quest \& in this lively account, which is, no exaggeration, unforgettable.'' -- {\it Parade}
		\item ``In recounting his year in training for the USA Memory Championship, journalist {\sf Foer} delivers a rich history of memory.'' -- {\it Discover Magazine}
		\item ``{\sf Foer}'s history of memory is rich with information about the nature of memory \& how it makes us who we are.'' -- {\it Scientific American}
		\item ``A brief \& pithy recounting of {\sc Foer}'s exploration of the fuzzy borders of his brain -- a marveling at how \& why it's able to do something quite unexpected $\ldots$ {\it Moonwalking with {\sc Einstein}} fits handly inline with the recent tradition of `big idea' books.'' -- {\it The Millions}
		\item ``An original, entertaining exploration about how \& why we remember.'' -- {\it Kirkus Reviews}
		\item ``An engaging, informative, \& for the forgetful, encouraging book.'' -- {\it Booklist}
		\item ``Hard to put down $\ldots$ The mind is a bigger thing than any of us realize, \& {\sc Foer} reminds us to keep exploring it.'' -- {\it Barnes \& Noble Review}
		\item ``He has thought deeply about memory \& his effort yields questions that are well worth reflecting on.'' -- {\it The Daily Beast}
		\item ``Intriguing $\ldots$ {\sc Foer} does an excellent job of tracing the history of the arts of memory.'' -- {\it The Forward}
		\item ``The kind of nonfiction work that gets people talking $\ldots$ A highly enjoyable read.'' -- \url{Thirteen.org}
		\item ``You have to love a writer who employs chick-sexing to help explain human memory. {\sc Foer} is a charmer, a crackling mind, a fresh wind. He approaches a complex topic with so much humanity, humor, \& originality that you don't realize how much you're taking in \& understanding. It's kind of miraculous.'' -- {\sc Mary Roach}, author of {\it Packing for Mars, Bonk, Spook, \& Stiff}
		\item ``{\it Moonwalking with {\sc Einstein}} isn't just a splendid overview of an essential aspect of our humanity -- our memory; it is also a witty \& engaging account of how {\sc Foer} went from being a guy with an average memory to winning the USA Memory Championship.'' -- {\sc Dan Ariely}, professor of behavioral economics at Duke University \& author of {\it The Upside of Irrationality} \& {\it Predictably Irrational}
		\item ``In this marvelous book, {\sc Joshua Foer} invents a new genre of nonfiction. This is a work of science journalism wrapped around an adventure story, a bildungs-roman fused to a vivid investigation of human memory. If you want to understand how we remember, \& how we can all learn to remember better, then read this book.'' -- {\sc Jonah Lehrer}, contributing editor to {\it Wired} \& author of {\it How We Decide} \& {\it Proust Was a Neuroscientist}
		\item ``{\sc Joshua Foer} proves what few of us are willing to get our heads around: there's more room in our brains that we ever imagined. {\it Moonwalking with {\sc Einstein}} isn't a how-to guide to remembering a name or where you put your keys. It's a riveting exploration of humankind's centuries-old obsession with memory, \& 1 man's improbable quest to master his own.'' -- {\sc Stefan Fatsis}, author of {\it A Few Seconds of Panic \& Word Freak}
	\end{itemize}
	{\sf About the Author.} {\sc Joshua Foer} was born in Washington, DC in 1982 \& lives in New Haven, CT with his wife Dinah. His writing has appeared in {\it National Geographic, Esquire, Slate, Outside, New York Times}, \& other publications. He is the co-founder of the Atlas Obscura, an online guide to the world's wonders \& curiosities. He is also the co-founder of the architectural design competition, Sukkah City. {\it Moonwalking with {\sc Einstein}} is his 1st book.
		
	\item \cite{Foer_remember_VN}. {\sc Joshua Foer}. {\it Moonwalking with Einstein: The Art \& Science of Remembering Everything -- Nhảy Moonwalk Cùng Einstein: Nghệ Thuật \& Khoa Học Để Nhớ Được Mọi Thứ}.\hfill{\sf[done]}
	
	\item \cite{Kahn_diginal_pandemic_VN}. Jeffrey P. Kahn. {\it Digital Contact Tracing for Pandemic Response -- Ứng Dụng Công Nghệ Truy Dấu Tiếp Xúc Để Ứng Phó Với Dịch Covid-19}.\hfill{\sf[done]}
	
	\item \cite{Sandel_justice}. Michael Sandel. {\it Justice: What's The Right Thing To Do? -- Phải Trái Đúng Sai}.\hfill{\sf[done]}
	
	\item \cite{Sandel_money}. Michael Sandel. {\it What Money Can't Buy -- Tiền Không Mua Được Gì?}.\hfill{\sf[done]}
	
	\item \cite{Taleb_randomness}. Nassim Nicholas Taleb. {\it Fooled By Randomness: The Hidden Role of Chance in Life \& in the Markets}.\hfill{\sf[reading]}
	
	\item \cite{Taleb_black_swan_VN}. Nassim Nicholas Taleb. {\it The Black Swan: The Impact of the Highly Improbable -- Thiên Nga Đen: Xác Suất Cực Nhỏ, Tác Động Cực Lớn}.\hfill{\sf[reading]}
	
	\item \cite{Taleb_randomness}. Nassim Nicholas Taleb. {\it Fooled By Randomness: The Hidden Role of Chance in Life \& in the Markets}.
	
	\item \cite{Taleb_randomness_VN}. Nassim Nicholas Taleb. {\it Fooled By Randomness: The Hidden Role of Chance in Life \& in the Markets -- Trò Đùa Của Sự Ngẫu Nhiên: Giải Mã Bí Ẩn Quanh Những Điều Tình Cờ}.\hfill{\sf[done]}
	
	\item \cite{Taleb_skin_game}. Nassim Nicholas Taleb. {\it Skin in the Game: Hidden Asymmetries in Daily Life}.
	
	\item \cite{Taleb_skin_game_VN}. Nassim Nicholas Taleb. {\it Skin in the Game: Hidden Asymmetries in Daily Life -- Da Thịt Trong Cuộc Chơi: Những Bất Đối Xứng Ẩn Trong Cuộc Sống Hằng Ngày}.\hfill{\sf[reading]}
	
	\item \cite{Truong_ke_tim_duong}. Phan Văn Trường. {\it Một Đời Như Kẻ Tìm Đường}.\hfill{\sf[done]}
	
	\item \cite{Westover_educated}. {\sc Tara Westover}. {\it Educated: A Memoir.} {\sf[217988 Amazon ratings][1636270 Goodreads ratings]}
	
	National Book Critics Circle Award Winner, 2018.
	
	{\sf Amazon review.} \#1 New York Times, Wall Street Journal, \& Boston Globe bestseller. 1 of the most acclaimed books of our time: un forgettable memoir about a young woman who, kept out of school, leaves her survivalist family \& goes on to earn a PhD from Cambridge University.
	
	``Extraordinary $\ldots$ an act of courage \& self-invention.'' -- {\it The New York Times}
	
	Named 1 of the 10 best books of the year by The New York Times Book Review. 1 of President {\sc Barack Obama}'s favorite books of the year. {\sc Bill Gate}'s holiday reading list. Finalist: National Book Critics Circle's Award In Autobiography \& John Leonard Prize For Best 1st Book. PEN{\tt/}Jean Stein Book Award. Los Angeles Times Book Prize.
	
	Born to survivalists in the mountains of Idaho, {\sc Tara Westover} was 17 the 1st time she set foot in a classroom. Her family was so isolated from mainstream society that there was no one to ensure the children received an education, \& no one to intervene when 1 of {\sc Tara}'s older brothers became violent. When another brother got himself into college, {\sc Tara} decided to try a new kind of life. Her quest for knowledge transformed her, taking her over oceans \& across continents, to Harvard \& to Cambridge University. Only then would she wonder if she'd traveled too far, if there was still a way home.
	
	``Beautiful \& propulsive $\ldots$ Despite the singularity of [{\sc Westover}'s] childhood, the question her book poses are universal: How much of ourselves should we give to those we love? \& how much must we betray them to grow up?'' -- {\it Vogue}
	
	A young woman raised in isolation \& ignorance, yearning for knowledge \& a different life, pursues education at Harvard \& Cambridge University, ultimately questioning the cost of her transformation \& the possibility of returning home.
	\begin{quote}
		\item ``But vindication has no power over guilt. No amount of anger or rage directed at others can subdue it, because guilt is never about them. Guilt is the fear of one's own wretchedness. It has nothing to do with other people.''
		\item ``I had begun to understand that we had lent our voices to a discourse whose sole purpose was to dehumanize \& brutalize others -- because nurturing that discourse was easier, because retaining power always feels like they way forward.''
		\item ``The skill I was learning was a crucial one, the patience to read things I could not yet understand.''
	\end{quote}
	
	{\sf Editorial reviews.}
	\begin{itemize}
		\item ``{\sc Westover} has somehow managed not only to capture her unsurpassably exceptional upbringing, but to make her current situation seem not so exceptional at all, \& resonant for many others.'' -- {\it The New York Times Book Review}
		\item ``{\sc Westover} is a keen \& honest guide to the difficulties of filial love, \& to the enchantment of embracing a life of the mind.'' -- {\it The New Yorke}
		\item ``An amazing story, \& truly inspiring. It's even better than you've heard.'' -- {\sc Bill Gates}
		\item ``Heart-wrenching $\ldots$ a beautiful testament to the power of education to open eyes \& change lives.'' -- {\sc Amy Chua}, {\it The New York Times Book Review}
		\item ``A coming-of-age memoir reminiscent of {\it The Glass Castle}.'' -- {\it O: The Oprah Magazine}
		\item ``{\sc Westover}'s 1-of-a-kind memoir is about the shaping of a mind $\ldots$ In briskly paced prose, she evokes a childhood that completely defined her. Yet it was also, she gradually sensed, deforming her.'' -- {\it The Atlantic}
		\item ``{\sc Tara Westover} is living proof that some people are flat-out, boots-always-laced-up indomitable. Her new book, {\it Educated}, is a heartbreaking, heartwarming, best-in-years memoir about striding beyond the limitations of birth \& environment into a better life. 4 out of 4.'' {\it USA Today}
		\item ``[{\it Educated}] left me speechless with wonder. [{\sc Westover}'s] lyrical prose is mesmerizing, as is her personal story, growing up in a family in which girls were supposed to aspire only to become wives -- \& in which coveting an education was considered sinful. Her journey will surprise \& inspire men \& women alike.'' -- {\it Refinery29}
		\item ``Riveting $\ldots$ {\sc Westover} brings readers deep into this world, a milieu usually hidden from outsiders $\ldots$ Her story is remarkable, as each extreme anecdote described in tidy prose attests.'' -- {\it The Economist}
		\item ``A subtle, nuanced study of how dysfunction of any kind can be normalized even within the most conventional family structure, \& of the damage such containment can do.'' -- {\it Financial Times}
		\item ``Whether narrating scenes of fury \& violence or evoking rural landscapes or tortured self-analysis, {\sc Westover} writes with uncommon intelligence \& grace $\ldots$ 1 of the most improbable \& fascinating journeys I've read in recent years.'' -- {\sc Newsday}
	\end{itemize}
	{\sf Author the Author.} {\sc Tara Westover} s an American historian \& memoirist. Her first book, Educated, debuted at \#1 on the New York Times bestseller list \& remained on the list, in hardcover, for more than two years. The book, a memoir of her upbringing in rural Idaho, was a finalist for a number of national awards, including the Los Angeles Times Book Prize, the PEN{\tt/}Jean Stein Book Award, \& the National Book Critics Circle Award. To date it has been translated into more than 45 languages. The New York Times named Educated one of the 10 Best Books of 2018, \& the American Booksellers Association voted Educated the Nonfiction Book of the Year. For her staggering impact, Time named Westover one of the 100 Most Influential People of 2019. Westover holds a PhD in intellectual history from Trinity College, Cambridge, \& in 2019 she was the Rosenthal Writer in Residence at Harvard University. In 2023, she was awarded the National Humanities Medal by President Biden.	
	
	\item \cite{Westover_educated}. {\sc Tara Westover}. {\it Educated: A Memoir -- Được Học: Tự Truyện.}\hfill{\sf[done]}
\end{enumerate}

%------------------------------------------------------------------------------%

\printbibliography[heading=bibintoc]

\end{document}