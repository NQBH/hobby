\documentclass{article}
\usepackage[backend=biber,natbib=true,style=authoryear]{biblatex}
\addbibresource{/home/nqbh/reference/bib.bib}
\usepackage[utf8]{vietnam}
\usepackage{tocloft}
\renewcommand{\cftsecleader}{\cftdotfill{\cftdotsep}}
\usepackage[colorlinks=true,linkcolor=blue,urlcolor=red,citecolor=magenta]{hyperref}
\usepackage{amsmath,amssymb,amsthm,mathtools,float,graphicx,algpseudocode,algorithm,tcolorbox,tikz,tkz-tab,subcaption}
\DeclareMathOperator{\arccot}{arccot}
\usepackage{enumitem}
\setlist{leftmargin=4mm}
\allowdisplaybreaks
\numberwithin{equation}{section}
\newtheorem{assumption}{Assumption}
\newtheorem{baitoan}{Bài toán}
\newtheorem{cauhoi}{Câu hỏi}
\newtheorem{conjecture}{Conjecture}
\newtheorem{corollary}{Corollary}
\newtheorem{dangtoan}{Dạng toán}
\newtheorem{definition}{Definition}
\newtheorem{dinhly}{Định lý}
\newtheorem{dinhnghia}{Định nghĩa}
\newtheorem{example}{Example}
\newtheorem{ghichu}{Ghi chú}
\newtheorem{hequa}{Hệ quả}
\newtheorem{hypothesis}{Hypothesis}
\newtheorem{lemma}{Lemma}
\newtheorem{luuy}{Lưu ý}
\newtheorem{nhanxet}{Nhận xét}
\newtheorem{notation}{Notation}
\newtheorem{note}{Note}
\newtheorem{principle}{Principle}
\newtheorem{problem}{Problem}
\newtheorem{proposition}{Proposition}
\newtheorem{question}{Question}
\newtheorem{remark}{Remark}
\newtheorem{theorem}{Theorem}
\newtheorem{vidu}{Ví dụ}
\usepackage[left=1cm,right=1cm,top=5mm,bottom=5mm,footskip=4mm]{geometry}
\def\labelitemii{$\circ$}

\title{Elementary {\it\&} Advanced STEM Classes Grades 6--12 {\it\&} Undergraduate\\Các Lớp STEM Sơ Cấp Lớp 6--12 {\it\&} STEM Nâng Cao Bậc Đại Học\\\vspace{2mm}\hrule\vspace{2mm}{\sf Mathematics, Physics, Chemistry, Programming: Pascal, Python, C{\tt/}C++\\Toán Học, Vật Lý, Hóa Học, Lập Trình: Pascal, Python, C{\tt/}C++}\vspace{2mm}\hrule}
\author{Nguyễn Quản Bá Hồng\footnote{Independent Researcher, Ben Tre City, Vietnam\\e-mail: \texttt{nguyenquanbahong@gmail.com}; website: \url{https://nqbh.github.io}.}}
\date{\vspace{-1cm}}

\begin{document}
\maketitle
\begin{abstract}
	Bảng tóm tắt thông tin, e.g., thời lượng, học phí, tài liệu, etc., của các lớp STEM sơ cấp cho các lớp 6--12, luyện thi Tuyển Sinh lớp 10 vào các trường THPT \& Tuyển Sinh Đại Học; \& các lớp STEM cao cấp cho bậc Đại Học, Cao Đẳng, bao gồm các môn: Toán Học, Vật Lý, Hóa Học Sơ Cấp, Cao Cấp, Lập Trình với các ngôn ngữ lập trình: Pascal, Python, C{\tt/}C++.
\end{abstract}
{\sf STEM: Science, Technology, Engineering, {\it\&} Mathematics: Khoa Học, Công Nghệ, Kỹ Thuật, {\it\&} Toán Học.}
\setcounter{secnumdepth}{4}
\setcounter{tocdepth}{3}
%\tableofcontents

%------------------------------------------------------------------------------%

\section{Duration \& Tuition Fee -- Thời Lượng Học \& Học Phí}

\subsection{Elementary STEM -- STEM Sơ Cấp}
Các lớp THCS (cấp 2): Lớp 6--9, THPT (cấp 3): Lớp 10--12:

\begin{luuy}[Tuition fee]
	Học phí đóng đầu tháng. Cam kết sẽ trả lại phần dư nếu thiếu buổi, \& cân nhắc với các trường hợp nghỉ ngang có xin phép. Nếu học với tần suất 2 buổi{\tt/}tuần thì 1 tháng sẽ tương đương $2\cdot4 = 8$ buổi. Học phí sẽ được đóng theo mỗi chu kỳ $8$ buổi (hoặc $10$ buổi) cho tiện để khỏi bù các buổi cả nhóm nghỉ vì mưa, bận đột xuất, etc. Học phí tính theo giờ là công bằng nhất. Rất nhiều bài tập về nhà sẽ được giao để học sinh làm trước ở nhà, để buổi học tiếp theo xung phong lên làm, sẽ được sửa nhanh \& chi tiết nên sẽ tiết kiệm rất nhiều so với học phí, với điều kiện học sinh phải siêng \& tự giác làm bài tập được giao.
\end{luuy}

\begin{luuy}[Language]
	Các bài giảng có thể sử dụng tiếng Việt hoặc tiếng Anh hoặc cả 2 (nếu hiểu \& có nhu cầu học song ngữ).
\end{luuy}

\begin{itemize}\itemsep0em
	\item \textsc{Lớp 6.} Học 3 môn: Toán 6, Hóa Học 6, Vật Lý 6.
	
	{\sf Học phí:} 25000 VND{\tt/}giờ, 1.5--2 giờ{\tt/}buổi. Nếu học 1.5 giờ{\tt/}buổi, thì cứ 8 buổi học (i.e., $\approx1$ tháng) đóng $25000\cdot1.5\cdot8 = 300000$ VND. Nếu học 2 giờ{\tt/}buổi, thì cứ 8 buổi học đóng $25000\cdot2\cdot8 = 400000$ VND.
	
	\begin{luuy}
		Khuyến khích học thử trước 1.5 giờ{\tt/}buổi vì học sinh tiểu học mới lên lớp 6 thường chưa quen chương trình mới cũng như cách học mới cho phù hợp. Sau khi quen hoặc thấy vừa sức có thể nâng lên thành 2 giờ{\tt/}buổi.
	\end{luuy}
	\item \textsc{Lớp 7.} Học 3 môn: Toán 7, Hóa Học 7, Vật Lý 7.
	
	{\sf Học phí:} 25000 VND{\tt/}giờ, tối thiểu 2 giờ{\tt/}buổi. Nếu học 2 giờ{\tt/}buổi, thì cứ 8 buổi học đóng $25000\cdot2\cdot8 = 400000$ VND. Nếu học 2.5 giờ{\tt/}buổi, thì cứ 8 buổi học đóng $25000\cdot2.5\cdot8 = 500000$ VND. 
	\item \textsc{Lớp 8.} Học 4 môn: Toán 8, Hóa Học 8, Vật Lý 8, Tin Học 8: Lập trình với ngôn ngữ Pascal, Python.
	
	{\sf Học phí:} 30000 VND{\tt/}giờ, tối thiểu 2 giờ{\tt/}buổi. Nếu học 2 giờ{\tt/}buổi, thì cứ 8 buổi học đóng $25000\cdot2\cdot8 = 400000$ VND.
	\item \textsc{Lớp 9 $+$ Luyện thi Tuyển sinh THPT.} Học 3 môn: Toán 9, Hóa Học 9, Vật Lý 9.
	
	{\sf Học phí:} 30000 VND{\tt/}giờ, tối thiểu 2 giờ{\tt/}buổi. Nếu học 2 giờ{\tt/}buổi, thì cứ 8 buổi học đóng $30000\cdot2\cdot8 = 480000$ VND, hoặc 10 buổi (i.e., 1 tháng \& 1 tuần) học đóng $30000\cdot2\cdot10 = 600000$ VND cho tiện.
	\item \textsc{Lớp 10.} Học 4 môn: Toán 10, Hóa Học 10, Vật Lý 10, Tin Học 10: Lập trình với ngôn ngữ Python, C{\tt/}C++.
	
	{\sf Học phí:} 35000 VND{\tt/}giờ, tối thiểu giờ{\tt/}buổi. Nếu học 2 giờ{\tt/}buổi, thì cứ 8 buổi học đóng $35000\cdot2\cdot8 = 560000$ VND, hoặc 10 buổi học đóng $35000\cdot2\cdot10 = 700000$ VND cho tiện.
	\item \textsc{Lớp 11.} Học 4 môn: Toán 11, Hóa Học 11, Vật Lý 11, Tin Học 11: Lập trình với ngôn ngữ Python, C{\tt/}C++.
	
	{\sf Học phí:} 40000 VND{\tt/}giờ, tối thiểu 2 giờ{\tt/}buổi. Nếu học 2 giờ{\tt/}buổi, thì cứ 8 buổi học đóng $40000\cdot2\cdot8 = 640000$ VND, hoặc 10 buổi học đóng $40000\cdot2\cdot10 = 800000$ VND cho tiện.
	\item \textsc{Lớp 12.} Học 3 môn: Toán 12, Hóa Học 12, Vật Lý 12, Tin Học 12.
	
	{\sf Học phí:} 40000 VND{\tt/}giờ, tối thiểu 2 giờ{\tt/}buổi. Nếu học 2 giờ{\tt/}buổi, thì cứ 8 buổi học đóng $40000\cdot2\cdot8 = 640000$ VND, hoặc 10 buổi học đóng $40000\cdot2\cdot10 = 800000$ VND cho tiện.
	\item \textsc{Luyện thi Đại học.} Luyện 3 môn: Toán, Lý Hóa. Riêng trường hợp có nguyện vọng thi các trường Cao Đẳng, Đại Học nhóm ngành Công Nghệ Thông Tin có bắt buộc thi môn Lập trình đầu vào vẫn có thể đăng ký.
	
	{\sf Học phí:} 40000--50000 VND{\tt/}giờ, tối thiểu 2 giờ{\tt/}buổi tùy theo trình độ của học sinh \& chuẩn đầu vào của trường Đại Học mà học sinh có nguyện vọng thi tuyển sinh vào.
\end{itemize}

\begin{luuy}
	Tin học các lớp 8, 10, 11 mới có phần Lập trình (Programming) khó nhằn nên mới dạy. Các kỹ năng khác như Microsoft Office, e.g., soạn thảo văn bản với Word: Tin học 6, bảng tính Excel: Tin học 7, Tạo bài giảng thuyết trình với Powerful: Tin học 9, Hệ thống quản trị dữ liệu với Access: Tin học 12 sẽ không có dạy, nhưng có thể hướng dẫn nếu cần khi gặp những chỗ khó.
	
	Tuy nhiên, nếu học sinh bất cứ lớp nào muốn học Lập trình thì vẫn ưu tiên dạy. Đặc biệt các học sinh có nguyện vọng theo nhóm ngành Công Nghệ Thông Tin, hoặc các nhóm ngành Kỹ thuật đòi hỏi Lập trình nhiều thì nên học từ sớm.
\end{luuy}

\subsection{Advanced STEM -- STEM Cao Cấp}
Các lớp cho Cao Đẳng, Đại Học:
\begin{itemize}
	\item \textsc{Advanced{\tt/}Higher Mathematics -- Toán Cao Cấp}. {\sf Học phí:} 50000 VND{\tt/}giờ. Học Online.
	\item \textsc{Advanced Programming -- Lập Trình Nâng Cao}: Các Ngôn Ngữ Python, C{\tt/}C++. {\sf Học phí:} 50000 VND{\tt/}giờ. Học Online.
\end{itemize}

%------------------------------------------------------------------------------%

\section{Rule -- Quy Định}

\subsection{Reward \& Punishment -- Quy Định Thưởng \& Phạt}

\begin{itemize}\itemsep0em
	\item For grades 6--7, each 9--9.4: 30000 VND, 9.5--9.9: 40000 VND, 10: 50000 VND; 5.1--6: $-30000$ VND, 3.1--5: $-40000$ VND, 0--3: $-50000$ VND for each of the following subjects: elementary mathematics, chemistry, physics, programming.
	\item For grades 8--9, each 9--9.4: 40000 VND, 9.5--9.9: 60000 VND, 10: 80000 VND; 5.1--6: $-40000$ VND, 3.1--5: $-60000$ VND, 0--3: $-80000$ VND for each of the following subjects: elementary mathematics, chemistry, physics.
	\item Vòng Huyện Toán, Lý, Hóa, Tin, Giải Toán Trên Máy Tính Cầm Tay (MTCT): Khuyến Khích: 100000 VND, Giải 3: 200000 VND, Giải 2: 300000 VND, Giải 1: 400000 VND.
	\item Vòng Tỉnh Toán, Lý, Hóa, Tin, Giải Toán Trên MTCT: Khuyến Khích: 500000 VND, Giải 3: 600000 VND, Giải 2: 800000 VND, Giải 1: 1000000 VND.
	\item Vòng Quốc Gia Toán, Lý, Hóa, Tin, Giải Toán Trên MTCT: Khuyến Khích: 1200000 VND, Giải 3: 1400000 VND, Giải 2: 1600000 VND, Giải 1: 2000000 VND.
	\item Nếu Cuộc Thi Giải Toán Trên Máy Tính Cầm Tay không còn do Bộ GD \& Đào Tạo phối hợp với BITEX tổ chức 2 vòng Huyện, Tỉnh, Quốc Gia phân biệt nhau mà chỉ có 1 vòng thì tính giải thưởng theo vòng Huyện.
\end{itemize}

\subsection{Common Sense -- Ý Thức}

\begin{itemize}\itemsep0em
	\item Đi học đúng giờ, tốt nhất sớm 5--10 phút so với giờ vào học. Không đi sớm trước giờ học hơn 30 phút vì sẽ làm phiền các nhóm đang học khác.
	\item Giữ gìn vệ sinh. Tuyệt đối không được xả rác. Nếu có phải dọn xong mới được về.
	\item Không làm ồn. Không làm ô nhiễm âm thanh. Không nói chuyện riêng ngoài việc thảo luận bài tập. Nhắc nhở 2 lần sẽ buộc về sớm. Nhắc nhở 3 lần không sửa, không tôn trọng lời nhắc sẽ được nghỉ học luôn.
	\item Học nếu thấy không hiểu điểm nào có thể đợi người dạy nói hết câu rồi hỏi ngay. Đặc biệt, nếu thấy người dạy giảng sai sót chỗ nào, phải chỉnh ngay lập tức, không được vì ngại, sợ nhục mà im im để rồi không hiểu, sẽ mất căn bản lâu dài.
	\item Cấm tuyệt đối bắt nạt bạn học. Cấm không phá hoại đồ dùng của bạn học. Cấm phá hoại kệ sách \& các đồ dùng điện tử, nếu hư hoại sẽ bắt đền. Các trường hợp nghiêm trọng sẽ báo phụ huynh, \& cân nhắc đuổi học vĩnh viễn.
\end{itemize}

%------------------------------------------------------------------------------%

\section{Resource -- Tài Liệu}

\begin{itemize}\itemsep0em
	\item My Curriculum Vitae [CV]: \url{https://github.com/NQBH/miscellaneous/blob/master/CV/NQBH_CV.pdf}.
	\item Các tài liệu được chia sẻ công khai có sẵn trên website: \url{https://nqbh.github.io/} \& \url{https://github.com/NQBH}.	
	\item Tên các học viên (không gồm thông tin cá nhân) \& điểm số của các học sinh ưu tú khóa trước \& khóa hiện tại có thể xem ở link: \url{https://github.com/NQBH/hobby/blob/master/STEM/student/NQBH_elementary_STEM_student.pdf}.
	\item Lịch học có thể xem ở link: \url{https://github.com/NQBH/hobby/blob/master/schedule/NQBH_schedule.pdf}.
	
	Nếu muốn đổi giờ học cho tiện cứ xem lịch trước để chọn giờ phù hợp \& thông báo để đổi lịch 1 cách hợp lý.
	\item Giữ gìn tài liệu cẩn thận, thẳng góc. Tuyệt đối không được xé. Nếu mất, hư thì có thể chủ động nhờ in lại. Nên có bọc nhựa hoặc bìa kẹp tài liệu, etc., để giữ gìn tài liệu cẩn thận.
	\item Tiền in tài liệu dưới 50 trang đã bao gồm trong học phí, không cần phải đóng thêm.
\end{itemize}

%------------------------------------------------------------------------------%

\printbibliography[heading=bibintoc]
	
\end{document}