\documentclass{article}
\usepackage[backend=biber,natbib=true,style=authoryear]{biblatex}
\addbibresource{/home/nqbh/reference/bib.bib}
\usepackage[utf8]{vietnam}
\usepackage{tocloft}
\renewcommand{\cftsecleader}{\cftdotfill{\cftdotsep}}
\usepackage[colorlinks=true,linkcolor=blue,urlcolor=red,citecolor=magenta]{hyperref}
\usepackage{amsmath,amssymb,amsthm,mathtools,float,graphicx,algpseudocode,algorithm,tcolorbox,tikz,tkz-tab,subcaption}
\DeclareMathOperator{\arccot}{arccot}
\usepackage{enumitem}
\setlist{leftmargin=4mm}
\allowdisplaybreaks
\numberwithin{equation}{section}
\newtheorem{assumption}{Assumption}[section]
\newtheorem{baitoan}{Bài toán}
\newtheorem{cauhoi}{Câu hỏi}[section]
\newtheorem{conjecture}{Conjecture}[section]
\newtheorem{corollary}{Corollary}[section]
\newtheorem{dangtoan}{Dạng toán}[section]
\newtheorem{definition}{Definition}[section]
\newtheorem{dinhly}{Định lý}[section]
\newtheorem{dinhnghia}{Định nghĩa}[section]
\newtheorem{example}{Example}[section]
\newtheorem{ghichu}{Ghi chú}[section]
\newtheorem{hequa}{Hệ quả}[section]
\newtheorem{hypothesis}{Hypothesis}[section]
\newtheorem{lemma}{Lemma}[section]
\newtheorem{luuy}{Lưu ý}[section]
\newtheorem{nhanxet}{Nhận xét}[section]
\newtheorem{notation}{Notation}[section]
\newtheorem{note}{Note}[section]
\newtheorem{principle}{Principle}[section]
\newtheorem{problem}{Problem}[section]
\newtheorem{proposition}{Proposition}[section]
\newtheorem{question}{Question}[section]
\newtheorem{remark}{Remark}[section]
\newtheorem{theorem}{Theorem}[section]
\newtheorem{vidu}{Ví dụ}[section]
\usepackage[left=1cm,right=1cm,top=5mm,bottom=5mm,footskip=4mm]{geometry}
\def\labelitemii{$\circ$}

\title{Elementary STEM Classes 6th--12th Grades -- Các Lớp STEM Sơ Cấp Lớp 6--12\\Advanced STEM Classes -- Các Lớp STEM Nâng Cao Bậc Đại Học\\\vspace{5mm}Mathematics, Physics, Chemistry, Pascal, Python, C{\tt/}C++ Programming\\Toán Học, Vật Lý, Hóa Học Sơ Cấp, Lập Trình Pascal, Python, C{\tt/}C++}
\author{Nguyễn Quản Bá Hồng\footnote{Independent Researcher, Ben Tre City, Vietnam\\e-mail: \texttt{nguyenquanbahong@gmail.com}; website: \url{https://nqbh.github.io}.}}
\date{}

\begin{document}
\maketitle
\begin{abstract}
	Các lớp STEM sơ cấp \& STEM cao cấp: Toán Học, Vật Lý, Hóa Học Sơ Cấp, Lập Trình với các ngôn ngữ lập trình: Pascal, Python, C{\tt/}C++.
\end{abstract}
\setcounter{secnumdepth}{4}
\setcounter{tocdepth}{3}
%\tableofcontents

%------------------------------------------------------------------------------%

\section{Duration \& Tuition Fee -- Thời Lượng Học \& Học Phí}

\subsection{Elementary STEM -- Khoa Học, Công Nghệ, Kỹ Thuật, Toán Học Sơ Cấp}
Các lớp THCS (cấp 2): Lớp 6--9, THPT (cấp 3): Lớp 10--12:
\begin{itemize}
	\item \textsc{Lớp 6.} Học 3 môn: Toán 6, Hóa Học 6, Vật Lý 6.
	
	{\sf Học phí:} 25000{\tt/}tiếng, 1.5--2 tiếng{\tt/}buổi. Nếu học 1.5 giờ{\tt/}buổi, thì cứ 8 buổi học (i.e., $\approx1$ tháng) đóng $25000\cdot1.5\cdot8 = 300000$ VND. Nếu học 2 giờ{\tt/}buổi, thì cứ 8 buổi học đóng $25000\cdot2\cdot8 = 400000$ VND.
	
	\begin{luuy}
		Khuyến khích học thử trước 1.5 tiếng{\tt/}buổi vì học sinh tiểu học mới lên lớp 6 thường chưa quen chương trình mới cũng như cách học mới cho phù hợp. Sau khi quen hoặc thấy vừa sức có thể nâng lên thành 2 giờ{\tt/}buổi.
	\end{luuy}
	\item \textsc{Lớp 7.} Học 3 môn: Toán 7, Hóa Học 7, Vật Lý 7.
	
	{\sf Học phí:} 25000{\tt/}tiếng, ít nhất 2 tiếng{\tt/}buổi. Nếu học 2 giờ{\tt/}buổi, thì cứ 8 buổi học đóng $25000\cdot2\cdot8 = 400000$ VND. Nếu học 2.5 giờ{\tt/}buổi, thì cứ 8 buổi học đóng $25000\cdot2.5\cdot8 = 500000$ VND. 
	\item \textsc{Lớp 8.} Học 4 môn: Toán 8, Hóa Học 8, Vật Lý 8, Tin Học 8: Lập trình với ngôn ngữ Pascal, Python.
	
	{\sf Học phí:} 30000{\tt/}tiếng, ít nhất 2 tiếng{\tt/}buổi. Nếu học 2 giờ{\tt/}buổi, thì cứ 8 buổi học đóng $25000\cdot2\cdot8 = 400000$ VND.
	\item \textsc{Lớp 9.} Học 3 môn: Toán 9, Hóa Học 9, Vật Lý 9.
	
	{\sf Học phí:} 30000{\tt/}tiếng $\ge2$ tiếng{\tt/}buổi.
	\item \textsc{Lớp 10.} Học 4 môn: Toán 10, Hóa Học 10, Vật Lý 10, Tin Học 10: Lập trình với ngôn ngữ Python, C{\tt/}C++.
	
	{\sf Học phí:} 35000{\tt/}tiếng $\ge2$ tiếng{\tt/}buổi.
	\item \textsc{Lớp 11.} Học 4 môn: Toán 11, Hóa Học 11, Vật Lý 11, Tin Học 11: Lập trình với ngôn ngữ Python, C{\tt/}C++.
	
	{\sf Học phí:} 40000{\tt/}tiếng $\ge2$ tiếng{\tt/}buổi.
	\item \textsc{Lớp 12.} Học 3 môn: Toán 12, Hóa Học 12, Vật Lý 12, Tin Học 12.
	{\sf Học phí:} 40000{\tt/}tiếng $\ge2$ tiếng{\tt/}buổi.
	\item \textsc{Luyện thi Đại học.} Luyện 3 môn: Toán, Lý Hóa. Riêng trường hợp có nguyện vọng thi các trường Cao Đẳng, Đại Học nhóm ngành Công Nghệ Thông Tin có bắt buộc thi môn Lập trình đầu vào vẫn có thể đăng ký.
	
	{\sf Học phí:} 40000--50000{\tt/}tiếng $\ge2$ tiếng{\tt/}buổi tùy theo trình độ của học sinh \& chuẩn đầu vào của trường Đại Học mà học sinh có nguyện vọng thi tuyển sinh.
\end{itemize}

\begin{luuy}
	Tin học các lớp 8, 10, 11 mới có phần Lập trình (Programming) khó nhằn nên mới dạy. Các kỹ năng khác như Microsoft Office, e.g., soạn thảo văn bản với Word: Tin học 6, bảng tính Excel: Tin học 7, Tạo bài giảng thuyết trình với Powerful: Tin học 9, Hệ thống quản trị dữ liệu với Access: Tin học 12 không có dạy.
	
	Tuy nhiên, nếu học sinh bất cứ lớp nào muốn học Lập trình thì vẫn dạy. Đặc biệt các học sinh có nguyện vọng theo nhóm ngành Công Nghệ Thông Tin, hoặc các nhóm ngành Kỹ thuật đòi hỏi Lập trình nhiều thì nên học từ sớm.
\end{luuy}

\subsection{Advanced STEM -- Khoa Học, Công Nghệ, Kỹ Thuật, Toán Học Cao Cấp}
Các lớp cho Cao Đẳng, Đại Học:
\begin{itemize}
	\item \textsc{Advanced Mathematics -- Toán Cao Cấp}. {\sf Học phí:} 50000 VND{\tt/}giờ. Học Online.
	\item \textsc{Programming -- Lập Trình}: Các Ngôn Ngữ Python, C{\tt/}C++. {\sf Học phí:} 50000 VND{\tt/}giờ. Học Online.
\end{itemize}

%------------------------------------------------------------------------------%

\section{Rule -- Quy Định}

\subsection{Reward \& Punishment -- Quy Định Thưởng \& Phạt}
\begin{itemize}
	\item For grades 6--8, each 9--9.4: 30000 VND, 9.5--9.9: 40000 VND, 10: 50000 VND; 5.1--6: $-30000$ VND, 3.1--5: $-40000$ VND, 0--3: $-50000$ VND for each of the following subjects: elementary mathematics, chemistry, physics, programming.
	\item For grades 9, each 9--9.4: 40000 VND, 9.5--9.9: 60000 VND, 10: 80000 VND; 5.1--6: $-40000$ VND, 3.1--5: $-60000$ VND, 0--3: $-80000$ VND for each of the following subjects: elementary mathematics, chemistry, physics.
	\item Vòng Huyện Toán, Lý, Hóa, Tin, Giải Toán Trên Máy Tính Cầm Tay (MTCT): Khuyến Khích: 100000 VND, Giải 3: 200000 VND, Giải 2: 300000 VND, Giải 1: 400000 VND.
	\item Vòng Tỉnh Toán, Lý, Hóa, Tin, Giải Toán Trên MTCT: Khuyến Khích: 500000 VND, Giải 3: 600000 VND, Giải 2: 800000 VND, Giải 1: 1000000 VND.
	\item Vòng Quốc Gia Toán, Lý, Hóa, Tin, Giải Toán Trên MTCT: Khuyến Khích: 1200000 VND, Giải 3: 1400000 VND, Giải 2: 1600000 VND, Giải 1: 2000000 VND.
	\item Nếu Cuộc Thi Giải Toán Trên Máy Tính Cầm Tay không còn do Bộ GD \& Đào Tạo phối hợp với BITEX tổ chức 2 vòng Huyện, Tỉnh, Quốc Gia phân biệt nhau mà chỉ có 1 vòng thì tính giải thưởng theo vòng Huyện.
\end{itemize}

\subsection{Common Sense Rule}

\begin{itemize}
	\item Đi học đúng giờ, tốt nhất sớm 5--10 phút so với giờ vào học. Không đi sớm trước giờ học hơn 30 phút vì sẽ làm phiền các nhóm đang học khác.
	\item Giữ gìn vệ sinh. Tuyệt đối không được xả rác, nếu có bắt buộc dọn.
	\item Cấm tuyệt đối không được phát kệ sách \& các đồ dùng điện tử. Các trường hợp nghiêm trọng sẽ bị đuổi học vĩnh viễn.
\end{itemize}

%------------------------------------------------------------------------------%

\section{Resource -- Tài Liệu}

\begin{itemize}
	\item Thông tin các học viên \& điểm số của các học sinh khóa trước \& khóa hiện tại có thể xem ở link:\\\url{https://github.com/NQBH/hobby/blob/master/STEM/student/NQBH_elementary_STEM_student.pdf}.
	\item Lịch học có thể xem ở link: \url{https://github.com/NQBH/hobby/blob/master/schedule/NQBH_schedule.pdf}. Nếu muốn đổi giờ học cho tiện cứ xem lịch trước để chọn giờ phù hợp \& thông báo để đổi lịch 1 cách hợp lý.
	\item Các tài liệu được chia sẻ công khai có sẵn trên website các của NQBH: \url{https://nqbh.github.io/}.
	\item Giữ gìn tài liệu cẩn thận, thẳng góc. Tuyệt đối không được xé. Nếu mất, hư có thể nhờ in lại. Nên có bọc nhựa hoặc bìa kẹp để giữ gìn tài liệu.
	\item Tiền in tài liệu dưới 50 trang đã bao gồm trong học phí, không cần phải đóng thêm.
\end{itemize}

%------------------------------------------------------------------------------%

\printbibliography[heading=bibintoc]
	
\end{document}