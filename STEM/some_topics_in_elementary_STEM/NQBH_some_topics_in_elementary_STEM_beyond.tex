\documentclass{article}
\usepackage[backend=biber,natbib=true,style=authoryear]{biblatex}
\addbibresource{/home/hong/1_NQBH/reference/bib.bib}
\usepackage[english,vietnamese]{babel}
\usepackage{tocloft}
\renewcommand{\cftsecleader}{\cftdotfill{\cftdotsep}}
\usepackage[colorlinks=true,linkcolor=blue,urlcolor=red,citecolor=magenta]{hyperref}
\usepackage{algorithm,algpseudocode,amsmath,amssymb,amsthm,float,graphicx,mathtools,multicol,tipa}
\allowdisplaybreaks
\numberwithin{equation}{section}
\newtheorem{assumption}{Assumption}[section]
\newtheorem{conjecture}{Conjecture}[section]
\newtheorem{corollary}{Corollary}[section]
\newtheorem{definition}{Definition}[section]
\newtheorem{example}{Example}[section]
\newtheorem{lemma}{Lemma}[section]
\newtheorem{notation}{Notation}[section]
\newtheorem{principle}{Principle}[section]
\newtheorem{problem}{Problem}[section]
\newtheorem{proposition}{Proposition}[section]
\newtheorem{question}{Question}[section]
\newtheorem{remark}{Remark}[section]
\newtheorem{theorem}{Theorem}[section]
\usepackage[left=0.5in,right=0.5in,top=1.5cm,bottom=1.5cm]{geometry}
\usepackage{fancyhdr}
\pagestyle{fancy}
\fancyhf{}
\lhead{\small \textsc{Sect.} ~\thesection}
\rhead{\small \nouppercase{\leftmark}}
\renewcommand{\sectionmark}[1]{\markboth{#1}{}}
\cfoot{\thepage}
\def\labelitemii{$\circ$}

\title{\underline{\textbf{Series}}\\Some Topics in Elementary STEM}
\author{\selectlanguage{vietnamese} Nguyễn Quản Bá Hồng\footnote{Independent Researcher, Ben Tre City, Vietnam\\e-mail: \texttt{nguyenquanbahong@gmail.com}; website: \url{https://nqbh.github.io}.}}
\date{\today}

\begin{document}
\maketitle
\selectlanguage{vietnamese}
\begin{abstract}
	Bài viết này là 1 bài luận ngắn về STEM \& cũng là phần Giới thiệu cho bộ sách cùng tên.
\end{abstract}

\tableofcontents
\selectlanguage{vietnamese}

%------------------------------------------------------------------------------%

\section{Preface}
\textsc{[Forward vs. Preface\footnote{\textbf{preface} \textipa{/`pref@s/} [n]}]:}

\textit{About the Title.}

\subsection{Dream vs. Reality}

\section{Basic Concepts}

\begin{definition}[STEM]
	
\end{definition}

\section{The Elements of Logic}

\subsection{Counterexamples \& its Role in Elementary Mathematics -- Phản Ví Dụ \& Vai Trò Trong Toán Sơ Cấp}

\begin{definition}
	``A \emph{counterexample}\footnote{\textbf{counterexample} \textipa{/`kaUnt@rIgzA:mpl/, /`kaUnt@rIgz\ae mpl/} [n] \textbf{counterexample (to something)} an example that provides evidence against an idea or theory.} is any exception to a \href{https://en.wikipedia.org/wiki/Generalization}{generalization}. In \href{https://en.wikipedia.org/wiki/Logic}{logic} a counterexample disproves the generalization, \& does so \href{https://en.wikipedia.org/wiki/Rigor}{rigorously} in the fields of \href{https://en.wikipedia.org/wiki/Mathematics}{mathematics} \& \href{https://en.wikipedia.org/wiki/Philosophy}{philosophy}.'' -- \href{https://en.wikipedia.org/wiki/Counterexample}{Wikipedia\emph{\texttt{/}}counterexample}
\end{definition}
``In mathematics, the term ``counterexample'' is also used (by a slight abuse) to refer to examples which illustrate the necessity of the full hypothesis of a theorem. This is most often done by considering a case where a part of the hypothesis is not satisfied \& the conclusion of the theorem does not hold.'' -- \href{https://en.wikipedia.org/wiki/Counterexample}{Wikipedia\texttt{/}counterexample}


%------------------------------------------------------------------------------%

\printbibliography[heading=bibintoc]
	
\end{document}