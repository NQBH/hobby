\documentclass{article}
\usepackage[backend=biber,natbib=true,style=authoryear]{biblatex}
\addbibresource{/home/nqbh/reference/bib.bib}
\usepackage{tocloft}
\renewcommand{\cftsecleader}{\cftdotfill{\cftdotsep}}
\usepackage[colorlinks=true,linkcolor=blue,urlcolor=red,citecolor=magenta]{hyperref}
\usepackage{algorithm,algpseudocode,amsmath,amssymb,amsthm,float,graphicx,mathtools}
\allowdisplaybreaks
\numberwithin{equation}{section}
\newtheorem{assumption}{Assumption}[section]
\newtheorem{conjecture}{Conjecture}[section]
\newtheorem{corollary}{Corollary}[section]
\newtheorem{definition}{Definition}[section]
\newtheorem{example}{Example}[section]
\newtheorem{lemma}{Lemma}[section]
\newtheorem{notation}{Notation}[section]
\newtheorem{principle}{Principle}[section]
\newtheorem{problem}{Problem}[section]
\newtheorem{proposition}{Proposition}[section]
\newtheorem{question}{Question}[section]
\newtheorem{remark}{Remark}[section]
\newtheorem{theorem}{Theorem}[section]
\usepackage[left=0.5in,right=0.5in,top=1.5cm,bottom=1.5cm]{geometry}
\usepackage{fancyhdr}
\pagestyle{fancy}
\fancyhf{}
\lhead{\small Sect.~\thesection}
\rhead{\small\nouppercase{\leftmark}}
\renewcommand{\sectionmark}[1]{\markboth{#1}{}}
\cfoot{\thepage}
\def\labelitemii{$\circ$}

\title{A Mathematician's Apology}
\author{G. H. Hardy}
\date{\today}

\begin{document}
\maketitle
\tableofcontents

%------------------------------------------------------------------------------%

\section*{Foreword}

%------------------------------------------------------------------------------%

\section*{Preface}
``I am indebted for many valuable criticisms to Prof. C. D. Board \& Dr. C. P. Snow, each of whom read my original manuscript. I have incorporated the substance of nearly all of their suggestions in my text, \& have so removed a good many crudities \& obscurities.

In 1 case I have dealt with them differently. My \S28 is based on a short article which I contributed to \textit{Eureka} (the journal of the Cambridge Archimedean Society) early in the year, \& I found it impossible to remodel what I had written so recently \& with so much care. Also, if I had tried to meet such important criticisms seriously, I should have had to expand this section so much as to destroy the whole balance of my essay. I have therefore left it unaltered, but have added a short statement of the chief points made by my critics in a note at the end. G. H. H. Jul 18, 1940'' -- \cite[p. 59]{Hardy1992}

\fbox{\textbf{1}} ``It is a melancholy experience for a professional mathematician to find himself writing about mathematics. The function of a mathematician is to do something, to prove new theorems, to add to mathematics, \& not to talk about what he or other mathematicians have done. Statesmen despise publicists, painters despise art-critics, \& physiologists, physicists, or mathematicians have usually similar feelings; there is no scorn more profound, or on the whole more justifiable, than that of the men who make for the \textit{men} who explain. Exposition, criticism, appreciation, is work for 2nd-rate minds.

I can remember arguing this point once in 1 of the few serious conversations that I ever had with Housman. Housman, in his Leslie Stephen lecture \textit{The Name \& Nature of Poetry}, had denied very emphatically that he was a `critic'; but he had denied it in what seemed to me a singularly perverse way, \& had expressed an admiration for literary criticism which startled \& scandalized me.

He had begun with a quotation from his inaugural lecture, delivered 22 years before--
\begin{quotation}
	Whether the faculty of literary criticism is the best gift that Heaven has in its treasuries, I cannot say; but Heaven seems to think so, for assuredly it is the gift most charily bestowed. Orators \& poets $\ldots$, if rare in comparison with blackberries, are commoner than returns of Halley's comet: literary critics are less common $\ldots$.
\end{quotation}
\& he had continued---
\begin{quotation}
	In these 22 years I have improved in some respects \& deteriorated in others, but I have not so much improved as to become a literary critic, nor so much deteriorated as to fancy that I have become one.
\end{quotation}
It had seemed to me deplorable that a great scholar \& a fine poet should write like this, \&, finding myself next to him in Hall a few weeks later, I plunged in \& said so. Did he really mean what he had said to be taken very seriously? Would the life of the best of critics really have seemed to him comparable with that of a scholar \& a poet? We argued these questions all through dinner, \& I think that finally he agreed with me. I must not seem to claim a dialectical triumph over a man who can no longer contradict me; but `Perhaps not entirely' was, in the end, his reply to the 1st question, \& `Probably no' to the 2nd.

There may have been some doubt about Housman's feelings, \& I do not wish to claim him as on my side; but there is no doubt at all about the feelings of men of science, \& I share them fully. If then I find myself writing, not mathematics but `about' mathematics, it is a confession of weakness, for which I may rightly be scorned or pitied by younger \& more vigorous mathematicians. I write about mathematics because, like any other mathematician who has passed 60, I have no longer the freshness of mind, the energy, or the patience to carry on effectively with my proper job.'' -- \cite[pp. 61--63]{Hardy1992}

\fbox{\textbf{2}} ``I propose to put forward an apology for mathematics; \& I may be told that it needs none, since there are now few studies more generally recognized, for good reasons or bad, as profitable \& praiseworthy. This may be true; indeed it is probable, since the sensational triumphs of Einstein, that stellar astronomy \& atomic physics are the only sciences which stand higher in popular estimation. A mathematician need not now consider himself on the defensive. He does not have to meet the sort of opposition described by Bradley in the admirable defence of metaphysics which forms the introduction to \textit{Appearance \& Reality}.

A metaphysician, says Bradley, will be told that `metaphysical knowledge is wholly impossible', or that `even if possible to a certain degree, it is practically no knowledge worth the name'. `The same problems,' he will hear, `the same disputes, the same sheer failure. Why not abandon it \& come out? Is there nothing else more worth your labor?' There is no one so stupid as to use this sort of language about mathematics. The mass of mathematical truth is obvious \& imposing; its practical applications, the bridges \& steam-engines \& dynamos, obtrude themselves on the dullest imagination. The public does not need to be convinced that there is something in mathematics.

All this is in its way very comforting to mathematicians, but it is hardly possible for a genuine mathematician to be content with it. Any genuine mathematician must feel that it is not on these crude achievements that the real case for mathematics rests, that the popular reputation of mathematics is based largely on ignorance \& confusion, \& that there is room for a more rational defence. At any rate, I am disposed to try to make one. It should be a simpler task than Bradley's difficult  apology.

I shall ask, then, \textit{why is it really worth while to make a serious study of mathematics? What is the proper justification of a mathematician's life?} \& my answers will be, for the most part, such as are to be expected from a mathematician: I think that it is worth while, that there is ample justification. But I should say at once that my defence of mathematics will be a defence of myself, \& that my apology is bound to be to some extent egotistical.  I should not think it worth while to apologize for my subject if I regarded myself as 1 of its failures.

Some egotism of this sort is inevitable, \& I do not feel that it really needs justification. Good work is not done by `humble' men. It is 1 of the 1st duties of a professor, e.g., in any subject, to exaggerate a little both the importance of his subject \& his own importance it it. A man who is always asking `Is what I do worth while?' \& `Am I the right person to do it?' will always be ineffective himself \& a discouragement to others. He must shut his eyes a little \& think a little more of his subject \& himself than they deserve. This is not too difficult: it is harder not to make his subject \& himself ridiculous by shutting his eyes too tightly.'' -- \cite[pp. 63--66]{Hardy1992}

\fbox{\textbf{3}} ``A man who sets out to justify his existence \& his activities has to distinguish 2 different questions. The 1st is whether the work which he does is worth doing; \& the 2nd is why he does it, whatever its value may be. The 1st question is often very difficult, \& the answer very discouraging, but most people will find the 2nd easy enough even then. Their answers, if they are honest, will usually take 1 or other of 2 forms; \& the 2nd form is merely a humbler variation of the 1st, which is the only answer which we need consider seriously.

(1) `I do what I do because it is the one \& only thing that I can do at all well. I am a lawyer, or a stockbroker, or a professional cricketer, because I have some real talent for that particular job. I am a lawyer because I have a fluent tongue, \& am interested in legal subtleties; I am a stockbroker because my judgment of the markets is quick \& sound; I am a professional cricketer because I can bat unusually well. I agree that it might be better to be a poet or a mathematician, but unfortunately I have no talent for such pursuits.'

I am not suggesting that this is a defence by which can be made by most people, since most people can do nothing at all well. But it is impregnable when it can be made without absurdity, as it can by a substantial minority: perhaps 5 or even 10\% of men can do something rather well. It is a tiny minority who can do anything \textit{really} well, \& the number of men who can do 2 things well is negligible. If a man has any genuine talent, he should be ready to make almost any sacrifice in order to cultivate it to the full.

This view was endorsed by Dr Johnson---
\begin{quotation}
	When I told him that I had been to see [his namesake] Johnson ride upon 3 horses, he said `Such a man, sir, should be encouraged, for his performances show the extent of the human powers $\ldots$'---
\end{quotation}
\& similarly he would have an applauded mountain climbers, channel swimmers, \& blindfold chess-players. For my own part, I am entirely in sympathy with all such attempts at remarkable achievement. I feel some sympathy even with conjurors \& ventriloquists; \& when Alekhine \& Bradman set out to beat records, I am quite bitterly disappointed if they fail. \& here both Dr Johnson \& I find ourselves in agreement with the public. As W. J. Turner has said so truly, it is only the `highbrows' (in the unpleasant sense) who do not admire the `real swell'.

We have of course to take account of the differences in value between different activities. I would rather be a novelist or a painter than a statesman of similar rank; \& there are many roads to fame which most of us would reject as actively pernicious. Yet it is seldom that such differences of value will turn the scale in a man's choice of a career, which will almost always be dictated by the limitations of his natural abilities. Poetry is more valuable than cricket, but Bradman would be a fool if he sacrificed his cricket in order to write 2nd-rate minor poetry (\& I suppose that it is unlikely that he could do better). If the cricket were a little less supreme, \& the poetry better, then the choice might be more difficult: I do not know whether I would rather have been Victor Trumper or Rupert Brooke. It is fortunate that such dilemmas occur so seldom.

I may add that they are particularly unlikely to present themselves to a mathematician. It is usual to exaggerate rather grossly the differences between the mental processes of mathematicians \& other people, but it is undeniable that a gift for mathematics is 1 of the most specialized talents, \& that mathematicians as a class are not particularly distinguished for general ability or versatility. If a man is in any sense a real mathematician, then it is 100 to 1 that his mathematics will be far better than anything else he can do, \& that he would be silly if he surrendered any decent opportunity of exercising his 1 talent in order to do undistinguished work in other fields. Such a sacrifice could be justified only by economic necessity or age.'' -- \cite[pp. 66--70]{Hardy1992}

\fbox{\textbf{4}} ``I had better say something here about this question of age, since it is particularly important for mathematicians. No mathematician should ever allow himself to forget that mathematics, more than any other art or science, is a young man's game. To take a simple illustration at a comparatively humble level, the average age of election to the Royal Society is lowest in mathematics.

We can naturally find much more striking illustrations. We may consider, e.g., the career of a man who was certainly 1 of the world's 3 greatest mathematicians. Newton gave up mathematics at 50, \& had lost his enthusiasm long before; he had recognized no doubt by the time that he was 40 that his great creative days were over. His greatest ideas of all, fluxions \& the law of gravitation, came to him about 1666, when he was 24 -- `in those days I was in the prime of my age for invention, \& minded mathematics \& philosophy more than at any time since.' He made big discoveries until he was nearly 40 (the `elliptic orbit' at 37), but after that he did little but polish \& perfect.

Galois died at 21, Abel at 27, Ramanujan at 33, Riemann at 40. There have been men who have done great work a good deal later; Gauss's great memoir on differential geometry was published when he was 50 (though he had had the fundamental ideas 10 years before). I do not know an instance of a major mathematical advance initiated by a man past 50. If a man of mature age loses interest in \& abandons mathematics, the loss is not likely to be very serious either for mathematics or for himself.

On the other hand the gain is no more likely to be substantial; the later records of mathematicians who have left mathematics are not particularly encouraging. Newton made a quite competent Master of the Mint (when he was not quarrelling with anybody). Painlev\'e was a not very successful Premier of France. Laplace's political career was highly discreditable, but he is hardly a fair instance, since he was dishonest rather than incompetent, \& never really `gave up' mathematics. It is very hard to find an instance of a 1st-rate mathematician who has abandoned mathematics \& attained 1st-rate distinction in any other field.\footnote{Pascal seems the best.} There may have been young men who would have been 1st-rate mathematicians if they had stuck to mathematics, but I have never heard of a really plausible example. \& all this is fully borne out by my own very limited experience. Every young mathematician of real talent whom I have known has been faithful to mathematics, \& not from lack of ambition but from abundance of it; they have all recognized that there, if anywhere, lay the road to a life of any distinction.'' -- \cite[pp. 70--73]{Hardy1992}

\fbox{\textbf{5}} ``There is also what I called the `humbler variation' of the standard apology; but I may dismiss this in a very few words.

(2) `There is \textit{nothing} that I can do particularly well. I do what I do because it came my way. I really never had a chance of doing anything else.' \& this apology too I accept as conclusive. It is quite true that most people can do nothing well. if so, it matters very little what career they choose, \& there is really nothing more to say about it. It is a conclusively reply, but hardly one likely to be made by a man with any pride; \& I may assume that none of us would be content with it.'' -- \cite[p. 73]{Hardy1992}

\fbox{\textbf{6}} ``It is time to begin thinking about the 1st question which I put in \S3, \& which is so much more difficult than the 2nd. Is mathematics, what I \& other mathematicians mean by mathematics, worth doing; \& if so, why?

I have been looking again at the 1st pages of the inaugural lecture which I gave at Oxford in 1920, where there is an outline of an apology for mathematics. It is very inadequate (less than a couple of pages), \& it is written in a style (a 1st essay, I suppose, in what I then imagined to be the `Oxford manner') of which I am not now particularly proud; but I still feel that, however much development it may need, it contains the essentials of the matter. I will resume what I said then, as a preface to a fuller discussion.

(1) I began by laying stress on the \textit{harmlessness} of mathematics -- `the study of mathematics is, if an unprofitable, a perfectly harmless \& innocent occupation'. I shall stick to that, but obviously it will need a good deal of expansion \& explanation.

\textit{Is} mathematics `unprofitable'? In some ways, plainly, it is not; e.g., it gives great pleasure to quite a large number of people. I was thinking of `profit', however, in a narrower sense. Is mathematics `useful', \textit{directly} useful, as other sciences such as chemistry \& physiology are? This is not an altogether easy or uncontroversial question, \& I shall ultimately say No, though some mathematicians, \& most outsiders, would no doubt say Yes. \& is mathematics `harmless'? Again the answer is not obvious, \& the question is one which I should have in some ways preferred to avoid, since it raises the whole problem of the effect of science on war. Is mathematics harmless, in the sense in which, e.g., chemistry plainly is not? I shall have to come back to both these questions later.

(2) I went on to say that `the scale of the universe is large \& if we are wasting our time, the waste of the lives of a few university dons is no such overwhelming catastrophe': \& here I may seem to be adopting, or affecting, the pose of exaggerated humility which I repudiated a moment ago. I am sure that that was not what was really in my mind; I was trying to say in a sentence what I have said at much greater length in \S3. I was assuming that we dons really had our little talents, \& that we could hardly be wrong if we did our best to cultivate them fully.

(3) Finally (in what seem to me now some rather painfully rhetorical sentences) I emphasized the permanence of mathematical achievement---
\begin{quotation}
	What we do may be small, but it has a certain character of permanence; \& to have produced anything of the slightest permanent interest, whether it be a copy of verses or a geometrical theorem, is to have done something utterly beyond the powers of the vast majority of men.
\end{quotation}
\&---
\begin{quotation}
	In these days of conflict between ancient \& modern studies, there must surely be something to be said for a study which did not begin with Pythagoras, \& will not end with Einstein, but is the oldest \& the youngest of all.
\end{quotation}
All this is `rhetoric'; but the substance of it seems to me still to ring true, \& I can expand it at once without prejudging any of the other questions which I am leaving open.'' -- \cite[pp. 74--77]{Hardy1992}

\fbox{\textbf{7}} ``I shall assume that I am writing for readers who are full, or have in the past been full, of a proper spirit of ambition. A man's 1st duty, a young man's at any rate, is to be ambitious. Ambition is a noble passion which may legitimately take many forms; there was \textit{something} noble in the ambition of Attila or Napoleon: but the noblest ambition is that of leaving behind one something of permanent value---
\begin{quotation}
	Here, on the level sand,
	
	Between the sea \& land,
	
	What shall I build or write
	
	Against the fall of night?
	
	
	Tell me of runes to grave
	
	That hold the bursting wave
	
	Or bastions to design
	
	For longer date than mine.
\end{quotation}
Ambition has been the driving force behind nearly all the best work of the world. In particular, practically all substantial contributions to human happiness have been made by ambitious men. To take 2 famous examples, were not Lister \& Pasteur ambitious? Or, on a humbler level, King Gillette \& William Willett; \& who in recent times have contributed more to human comfort than they?

Physiology provides particularly good examples just because it is so obviously a `beneficial' study. We must guard against a fallacy common among apologists of science, the fallacy of supposing that the men whose work most benefits humanity are thinking much of that while they do it, that physiologists, e.g., have particularly noble souls. A physiologist may indeed be glad to remember that his work will benefit mankind, but the motives which provide the force \& the inspiration for it are indistinguishable from those of a classical scholar or a mathematician.

There are many highly respectable motives which may lead men to prosecute research, but 3 which are much more important than the rest. The 1st (without which the rest must come to nothing) is intellectual curiosity, desire to know the truth. Then, professional pride, anxiety to be satisfied with one's performance, the shame that overcomes any self-respecting craftsman when his work is unworthy of his talent. Finally, ambition, desire for reputation, \& the position, even the power or the money, which it brings. It may be fine to feel, when you have done your work, that you have added to the happiness or alleviated the sufferings of others, but that will not by why you did it. So if a mathematician, or a chemist, or even a physiologist, were to tell me that the driving force in his work had been the desire to benefit humanity, then I should not believe him (nor should I think the better of him if I did). His dominant motives have been those which I have started, \& in which, surely, there is nothing of which any decent man need be ashamed.'' -- \cite[pp. 77--79]{Hardy1992}

\fbox{\textbf{8}} ``If intellectual curiosity, professional pride, \& ambition are the dominant incentives to research, then assuredly no one has a fairer chance of gratifying them than a mathematician. His subject is the most curious of all -- there is non in which truth plays such odd pranks. It has the most elaborate \& the most fascinating technique, \& gives unrivalled openings for the display of sheer professional skill. Finally, as history proves abundantly, mathematical achievement, whatever its intrinsic worth, is the most enduring of all.

We can see this even in semi-historic civilizations. The Babylonian \& Assyrian civilizations have perished; Hammurabi, Sargon, \& Nebuchadnezzar are empty names; yet Babylonian mathematics is still interesting, \& the Babylonian scale of 60 is still used in astronomy. But of course the crucial case is that of the Greeks.

The Greeks were the 1st mathematicians who are still `real' to us to-day. Oriental mathematics may be an interesting curiosity, but Greek mathematics is the real thing. The Greeks 1st spoke a language which modern mathematicians can understand; as Littlewood said to me once, they are not clever schoolboys or `scholarship candidates', but `Fellows of another college'. So Greek mathematics is `permanent', more permanent even than Greek literature. Archimedes will be remembered when Aeschylus is forgotten, because languages die \& mathematical ideas do not. `Immortality' may be a silly word, but probably a mathematician has the best chance of whatever it may mean.

Nor need he fear very seriously that the future will be unjust to him. Immortality is often ridiculous or cruel: few of us would have chosen to the Og or Ananias or Gallio. Even in mathematics, history sometimes plays strange tricks; Rolle figures in the text-books of elementary calculus as if he had been a mathematician like Newton; Farey is immortal because he failed to understand a theorem which Haros had prove perfectly 14 years before; the names of 5 worthy Norwegians still stand in Abel's \textit{Life}, just fo 1 act of conscientious imbecility, dutifully performed at the expense of their country's greatest man. But on the whole the history of science is far, \& this is particularly true in mathematics. No other subject has such clear-cut or unanimously accepted standards, \& the men who are remembered are almost always the men who merit it. Mathematical fame, if you have the cash to pay for it, is 1 of the soundest \& steadiest of investments.'' -- \cite[pp. 80--82]{Hardy1992}

\fbox{\textbf{9}} ``All this is very comforting for dons, \& especially for professors of mathematics. It is sometimes suggested, by lawyers or politicians or business men, that an academic career is one sought mainly by cautious \& unambitious persons who are primarily for comfort \& security. The reproach is quite misplaced. A don surrenders something, \& in particular the chance of making large sums of money -- it is very hard for a professor to make \pounds2000 a year; \& security of tenure is naturally 1 of the considerations which make this particular surrender easy. That is not why Housman would have refused to be Lord Simon or Lord Beaverbrook. He would have rejected their careers because of his ambition, because he would have scorned to be a man to be forgotten in 20 years.

Yet how painful it is to feel that, with all these advantages, one may fail. I can remember Bertrand Russel telling me of a horrible dream. He was in the top floor of the University Library, about A.D. 2100. A library assistant was going round the shelves carrying an enormous bucket, taking down book after book, glancing at them, restoring them to the shelves or dumping them into the bucket. At last he came to 3 large volumes which Russel could recognize as the last surviving copy of \textit{Principia mathematica}. He took down 1 of the volumes, turned over a few pages, seemed puzzled for a moment by the curious symbolism, closed the volume, balanced it in his hand \& hesitated $\ldots$.'' -- \cite[pp. 82--83]{Hardy1992}

\fbox{\textbf{10}} ``A mathematician, like a painter or a poet, is a maker of patterns. If his patterns are more permanent than theirs, it is because they are made with \textit{ideas}. A painter makes patterns with shapes \& colors, a poet with words. A painting may embody an `idea', but the idea is usually commonplace \& unimportant. In poetry, ideas count for a good deal more; but, as Housman insisted, the importance of ideas in poetry is habitually exaggerated: `I cannot satisfy myself that there are any such things as poetical ideas $\ldots$. Poetry is not the thing said but a way of saying it.'
\begin{quotation}
	Not all the water in the rough rude sea
	
	Can wash the balm from an anointed King.
\end{quotation}
Could lines be better, \& could ideas be at once more trite \& more false? The poverty of the ideas seems hardly to affect the beauty of the verbal pattern. A mathematician, on the other hand, has no material to work with but ideas, \& so his patterns are likely to last longer, since ideas wear less with time than words.

The mathematician's patterns, like the painter's or the poet's, must be \textit{beautiful}; the ideas, like the colors or the words, must fit together in a harmonious way. Beauty is the 1st test: there is no permanent place in the world for ugly mathematics. \& here I must deal with a misconception which is still widespread (though probably much less so now than it was 20 years ago), what Whitehead has called the `literary superstition' that love of \& aesthetic appreciation of mathematics is `a monomania confined to a few eccentrics in each generation'.

It would be difficult now to find an educated man quite insensitive to the aesthetic appeal of mathematics. It may be very hard to \textit{define} mathematical beauty, but that is just as true of beauty of any kind -- we may not know quite what we mean by a beautiful poem, but that does not prevent us from recognizing one when we read it. Even Prof. Hogben, who is out to minimize at all costs the importance of the aesthetic element in mathematics, does not venture to deny its reality. `There are, to be sure, individuals for whom mathematics exercises a coldly impersonal attraction $\ldots$ The aesthetic appeal of mathematics may be very real for a chosen few.' But they are `few', he suggests, \& they feel `coldly' (\& are really rather ridiculous people, who live in silly little university towns sheltered from the fresh breezes of the wide open spaces). In this he is merely echoing Whitehead's `literary superstition'.

The fact is that there are few more `popular' subjects than mathematics. Most people have some appreciation of mathematics, just as most people can enjoy a pleasant tune; \& there are probably more people really interested in mathematics than in music. Appearances may suggest the contrary, but there are easy explanations. Music can be used to stimulate mass emotion, while mathematics cannot; \& musical incapacity is recognized (no doubt rightly) as mildly discreditable, whereas most people are so frightened of the name of mathematics that they are ready, quite unaffectedly, to exaggerate their own mathematical stupidity.

A very little reflection is enough to expose the absurdity of the `literary superstition'. There are masses of chess-players in every civilized country -- in Russia, almost the whole educated population; \& every chess-player can recognize \& appreciate a `beautiful' game or problem. Yet a chess problem is \textit{simply} an exercise in pure mathematics (a game not entirely, since psychology also plays a part), \& everyone who calls a problem `beautiful' is applauding mathematical beauty, even if it is beauty of a comparatively lowly kind. Chess problems are the hymn-tunes of mathematics.

We may learn the same lesson, at a lower level but for a wider public, from bridge, or descending further, from the puzzle columns of the popular newspapers. Nearly all their immense popularity is a tribute to the drawing power of rudimentary mathematics, \& the better makers of puzzles, such as Dudeney or `Caliban', use very little else. They know their business; what the public wants is a little intellectual `kick', \& nothing else has quite the kick of mathematics.

I might add that there is nothing in the world which pleases even famous men (\& men who have used disparaging language about mathematics) quite so much as to discover, or rediscover, a genuine mathematical theorem. Herbert Spencer republished in his autobiography a theorem about circles which he proved when he was 20 (not knowing that it had been proved over 2000 years before by Plato). Prof. Soddy is a more recent \& a more striking example (but \textit{his} theorem really is his own)\footnote{See his letters on the `Hexlet' in \textit{Nature}, vols. 137--9 (1936--7).}'' -- \cite[pp. 84--88]{Hardy1992}

\fbox{\textbf{11}} ``A chess problem is genuine mathematics, but it is in some way `trivial' mathematics. However ingenious \& intricate, however original \& surprising the moves, there is something essential lacking. Chess problems are \textit{unimportant}. The best mathematics is \textit{serious} as well as beautiful -- `important' if you like, but the word is very ambiguous, \& `serious' expresses what I mean much better.

I am not thinking of the `practical' consequences of mathematics. I have to return to that point later: at present I will say only that if a chess problem is, in the crude sense, `useless', then that is equally true of most of the best mathematics; that very little of mathematics is useful practically, \& that that little is comparatively dull. The `seriousness' of a mathematical theorem lies, not in its practical consequences, which are usually negligible, but in the \textit{significance} of the mathematical ideas which it connects. We may say, roughly, that a mathematical idea is `significant' if it can be connected, in a natural \& illuminating way, with a large complex of other mathematical ideas. Thus a serious mathematical theorem, a theorem which connects significant ideas, is likely to lead to important advances in mathematics itself \& even in other sciences. No chess problem has ever affected the general development of scientific thought; Pythagoras, Newton, Einstein have in their times changed its whole direction.

The seriousness of a theorem, of course, does not \textit{lie in} its consequences, which are merely the \textit{evidence} for its seriousness. Shakespeare had an enormous influence on the development of the English language, Otway next to none, but that is not why Shakespeare was the better poet. He was the better poet because he wrote much better poetry. The inferiority of the chess problem, like that of Otway's poetry, lies not in its consequences but in its content.

There is 1 more point which I shall discuss very shortly, not because it is uninteresting but because it is difficult, \& because I have no qualifications for any serious discussion in aesthetics. The beauty of a mathematical theorem \textit{depends} a great deal on its seriousness, as even in poetry the beauty of a line may depend to some extent on the significance of the ideas which it contains. I quoted 2 lines of Shakespeare as an example of the sheer beauty of a verbal pattern; but
\begin{quotation}
	After life's fitful fever he sleeps well
\end{quotation}
seems still more beautiful. The pattern is just as fine, \& in this case the ideas have significance \& the thesis is sound, so that our emotions are stirred much more deeply. The ideas do matter to the pattern, even in poetry, \& much more, naturally, in mathematics; but I must not try to argue the question seriously.'' -- \cite[pp. 88--91]{Hardy1992}

\fbox{\textbf{12}} ``It will be clear by now that, if we are to have any chance of making progress, I must produce examples of `real' mathematical theorems, theorems which every mathematician will admit to be 1st-rate. \& here I am very heavily handicapped by the restrictions under which I am writing. On the 1 hand my examples must be very simple, \& intelligible to a reader who has no specialized mathematical knowledge; no elaborate preliminary explanations must be needed; \& a reader must be able to follow the proofs as well as the enunciations. These conditions exclude, e.g., many of the most beautiful theorems of the theory of numbers, such as Fermat's `2 square' theorem or the law of quadratic reciprocity. \& on the other hand my examples should be drawn from `pukka' mathematics, the mathematics of the working professional mathematician; \& this condition excludes a good deal which it would be comparatively easy to make intelligible but which trespasses on logic \& mathematical philosophy.

I can hardly do better than go back to the Greeks. I will state \& prove 2 of the famous theorems of Greek mathematics. They are `simple' theorems, simple both in idea \& in execution, but there is no doubt at all about their being theorems of the highest class. Each is as fresh \& significant as when it was discovered -- 2000 years have not written a wrinkle on either of them. Finally, both the statements \& the proofs can be mastered in an hour by any intelligent reader, however slender his mathematical equipment.

1. The 1st is Euclid's\footnote{\textit{Elements} IX 20. The real origin of many theorems in the \textit{Elements} is obscure, but there seems to be no particular reason for supposing that this one is not Euclid's own.} proof of the existence of an infinity of prime numbers.

The \textit{prime numbers} or \textit{primes} are the numbers (A) 2, 3, 5, 7, 11, 13, 17, 19, 23, 29, $\ldots$ which cannot be resolved into smaller factors\footnote{There are technical reasons for not counting 1 as a prime.} Thus 37 \& 317 are prime. The primes are the material out of which all numbers are built up by multiplication: thus $666 = 2\cdot3\cdot3\cdot37$. Every number which is no prime itself is divisible by at least 1 prime (usually, of course, by several). We have to prove that there are infinitely many primes, i.e., that the series (A) never comes to an end.

Let us suppose that it does, \& that $2,3,5,\ldots,P$ is the complete series (so that $P$ is the largest prime); \& let us, on this hypothesis, consider the number $Q$ defined by the formula $Q = 2\cdot3\cdot5\cdot\cdots\cdot P + 1$. It is plain that $Q$ is not divisible by any of $2,3,5,\ldots,P$; for it leaves the remainder 1 when divided by any 1 of these numbers. But, if not itself prime, it is divisible by \textit{some} prime, \& therefore there is a prime (which may be $Q$ itself) greater than any of them. This contradicts our hypothesis, that there is no prime greater than $P$; \& therefore this hypothesis is false.

The proof is by \textit{reductio ad absurdum}, \& \textit{reductio ad absurdum}, which Euclid loved so much, is 1 of a mathematician's finest weapons\footnote{The proof can be arranged so as to avoid a \textit{reductio}, \& logicians of some schools would prefer that it should be.} It is a far finer gambit than any chess gambit: a chess player may offer the sacrifice of a pawn or even a piece, but a mathematician offers \textit{the game}.'' -- \cite[pp. 91--94]{Hardy1992}

\fbox{\textbf{13}} ``My 2nd example is Pythagoras's\footnote{The proof traditionally ascribed to Pythagoras, \& certainly a product of his school. The theorem occurs, in a much more general form, in Euclid (\textit{Elements} X 9).} proof of the `irrationality' of $\sqrt{2}$.

A `rational number' is a fraction $\frac{a}{b}$, where $a,b\in\mathbb{Z}$; we may suppose that $a$ \& $b$ have no common factor, since if they had we could remove it. To say that `$\sqrt{2}$ is irrational' is merely another way of saying that $2$ cannot be expressed in the form $\left(\frac{a}{b}\right)^2$; \& this is the same thing as saying that the equation (B) $a^2 = 2b^2$ cannot be satisfied by integral values of $a$ \& $b$ which have no common factor. This is a theorem of pure arithmetic, which does not demand any knowledge of `irrational numbers' or depend on any theory about their nature.

We argue again by \textit{reductio ad absurdum}; we suppose that (B) is true, $a,b$ being integers without any common factor. It follows from (B) that $a^2$ is even (since $2b^2$ is divisible by 2), \& therefore that $a$ is even (since the square of an odd number is odd). If $a$ is even than (C) $a = 2c$ for some integral value of $c$; \& therefore $2b^2 = a^2 = (2c)^2 = 4c^2$ or (D) $b^2 = 2c^2$. Hence $b^2$ is even, \& therefore (for the same reason as before) $b$ is even. I.e., $a$ \& $b$ are both even, \& so have the common factor $2$. This contradicts our hypothesis, \& therefore the hypothesis is false.

It follows from Pythagoras's theorem that the diagonal of a square is incommensurable with the side (that their ratio is not a rational number, that there is no unit of which both are integral multiples). For if we take the side as our unit of length, \& the length of the diagonal is $d$, then, by a very familiar theorem also ascribed to Pythagoras\footnote{Euclid, \textit{Elements} I 47.}, $d^2 = 1^2 + 1^2 = 2$, so that $d$ cannot be a rational number.

I could quote any number of fine theorems from the theory of numbers whose \textit{meaning} anyone can understand. E.g., there is what is called `the fundamental theorem of arithmetic', that any integer can be resolved, \textit{in 1 way only}, into a product of primes. Thus $666 = 2\cdot3\cdot3\cdot37$, \& there is no other decomposition; it is impossible that $666 = 2\cdot11\cdot29$ or that $13\cdot89 = 17\cdot73$ (\& we can see so without working out the products). This theorem is, as its name implies, the foundation of higher arithmetic; but the proof, although not `difficult', requires a certain amount of preface \& might be found tedious by an unmathematical reader.

Another famous \& beautiful theorem is Fermat's `2 square' theorem. The primes may (if we ignore the special prime 2) be arranged in 2 classes; the primes $5,13,17,29,37,41,\ldots$ which leave remainder 1 when divided by 4, \& the primes $3,7,11,19,23,31,\ldots$ which leave remainder 3. All the primes of the 1st class, \& none of the 2nd, can be expressed as the sum of 2 integral squares: thus $5 = 1^2 + 2^2$, $13 = 2^2 + 3^2$, $17 = 1^2 + 4^2$, $29 = 2^2 + 5^2$; but 3, 7, 11, \& 19 are not expressible in this way (as the reader may check by trial). This is Fermat's theorem, which is ranked, very justly, as 1 of the finest of arithmetic. Unfortunately there is no proof within the comprehension of anybody but a fairly expert mathematician.

There are also beautiful theorems in the `theory of aggregates' (\textit{Mengenlehre}), such as Cantor's theorem of the `non-enumerability' of the continuum. Here there is just the opposite difficulty. The proof is easy enough, when once the language has been mastered, but considerable explanation is necessary before the \textit{meaning} of the theorem becomes clear. So I will not try to give more examples. Those which I have given are test cases, \& a reader who cannot appreciate them is unlikely to appreciate anything in mathematics.

I said that a mathematician was a maker of patterns of ideas, \& that beauty \& seriousness were the criteria by which his patterns should be judged. I can hardly believe that anyone who has understood the 2 theorems will dispute that they pass these tests. If we compare them with Dudeney's most ingenious puzzles, or the finest chess problems that masters of that art have composed, their superiority in both respects stands out: there is an unmistakable difference of class. They are much more serious, \& also much more beautiful; can we define, a little more closely, where their superiority lies?'' -- \cite[pp. 94--99]{Hardy1992}

\fbox{\textbf{14}} ``In the 1st place, the superiority of the mathematical theorems in \textit{seriousness} is obvious \& overwhelming. The chess problem is the product of an ingenious but very limited complex of ideas, which do not differ from one another very fundamentally \& have no external repercussions. We should think in the same way if chess had never been invented, whereas the theorems of Euclid \& Pythagoras have influenced thought profoundly, even outside mathematics.

Thus Euclid's theorem is vital for the whole structure of arithmetic. The primes are the raw material out of which we have to build arithmetic, \& Euclid's theorem assures us that we have plenty of material for the task. But the theorem of Pythagoras has wider applications \& provides a better text.

We should observe 1st that Pythagoras's argument is capable of far-reaching extension, \& can be applied, with little change of principle, to very wide classes of `irrationals'. We can prove very similarly (as Theodorus seems to have done) that $\sqrt{3},\sqrt{5},\sqrt{7},\sqrt{11},\sqrt{13},\sqrt{17}$ are irrational, or (going beyond Theodorus) that $\sqrt[3]{2}$ \& $\sqrt[3]{17}$ are irrational\footnote{See Ch. IV of Hardy \& Wright's \textit{Introduction to the Theory of Numbers}, where there are discussions of different generalizations of Pythagoras's argument, \& of a historical puzzle about Theodorus.}.

Euclid's theorem tells us that we have a good supply of material for the construction of a coherent arithmetic of the integers. Pythagoras's theorem \& its extensions tell us that, when we have constructed this arithmetic, it will not prove sufficient for our needs, since there will be many magnitudes which obtrude themselves upon our attention \& which it will be unable to measure; the diagonal of the square is merely the most obvious example. The profound importance of this discovery was recognized at once by the Greek mathematicians. They had begun by assuming (in accordance, I suppose, with the `natural' dictates of `common sense') that all magnitudes of the same kind are commensurable, that any 2 lengths, e.g., are multiples of some common unit, \& they had constructed a theory of proportion based on this assumption. Pythagoras's discovery exposed the unsoundness of this foundation, \& led to the construction of the much more profound theory of Eudoxus which is set out in the 5th book of the \textit{Elements}, \& which is regarded by many modern mathematicians as the finest achievement of Greek mathematics. This theory is astonishingly modern in spirit, \& may be regarded as the beginning of the modern theory of irrational number, which has revolutionized mathematical analysis \& had much influence on recent philosophy.

There is no doubt at all, then, of the `seriousness' of either theorem. It is therefore the better worth remarking that neither theorem has the slightest `practical' importance. In practical applications we are concerned only with comparatively small numbers; only stellar astronomy \& atomic physics deal with `large' numbers, \& they have very little more practical importance, as yet, than the most abstract pure mathematics. I do not know what is the highest degree of accuracy which is ever useful to an engineer -- we shall be very generous if we say 10 significant figures. Then $3.14159265$ (the value of $\pi$ to 8 places of decimals) is the ratio $\frac{314159265}{100000000}$ of 2 numbers of 9 digits. The number of primes $<10^9$ is $50847478$: that is enough for an engineer, \& he can be perfectly happy without the rest. So much for Euclid's theorem; \&, as regards Pythagoras's, it is obvious that irrationals are uninteresting to an engineer, since he is concerned only with approximations, \& all approximations are rational.'' -- \cite[pp. 99--102]{Hardy1992}

\fbox{\textbf{15}} ``A `serious' theorem is a theorem which contains `significant' ideas, \& I suppose that I ought to try to analyze a little more closely the qualities which make a mathematical idea significant. This is very difficult, \& it is unlikely that any analysis which I can give will be very valuable. We can recognize a `significant' idea when we see it, as we can those which occur in my 2 standard theorems; but this power of recognition requires a rather high degree of mathematical sophistication, \& of that familiarity with mathematical ideas which comes only from many years spent in their company. So I must attempt some sort of analysis; \& it should be possible to make one which, however inadequate, is sound \& intelligible so far as it goes. There are 2 things at any rate which seem essential, a certain \textit{generality} \& a certain \textit{depth}; but neither quality is easy to define at all precisely.

A significant mathematical idea, a serious mathematical theorem, should be `general' in some such sense as this. The idea should be one which is a constituent in many mathematical constructs, which is used in the proof of theorems of many different kinds. The theorem should be one which, even if stated originally (like Pythagoras's theorem) in a quite special form, is capable of considerable extension \& is typical of a whole class of theorems of its kind. The relations revealed by the proof should be such as connect many different mathematical ideas. All this is very vague, \& subject to many reservations. But it is easy enough to see that a theorem is unlikely to be serious when it lacks these qualities conspicuously; we have only to take examples from the isolated curiosities in which arithmetic abounds. I take 2, almost at random, from Rouse Ball's \textit{Mathematical Recreations}\footnote{11th edition, 1939 (revised by H. S. M. Coxeter).}

(a) 8712 \& 9801 are the only 4-figure numbers which are integral multiples of their `reversals': $8712 = 4\cdot2178$, $9801 = 9\cdot1089$, \& there are no other numbers below $10000$ which have this property.

(b) There are just 4 numbers (after 1) which are the sums of the cubes of their digits, viz. $153 = 1^3 + 5^3 + 3^3$, $370 = 3^3 + 7^3 + 0^3$, $371 = 3^3 + 7^3 + 1^3$, $407 = 4^3 + 0^3 + 7^3$.

These are odd facts, very suitable for puzzle columns \& likely to amuse amateurs, but there is nothing in them which appeals much to a mathematician. The proofs are neither difficult nor interesting -- merely a little tiresome. The theorems are not serious; \& it is plain that 1 reason (though perhaps not the most important) is the extreme speciality of both the enunciations \& the proofs, which are not capable of any significant generalization.'' -- \cite[pp. 103--105]{Hardy1992}

%------------------------------------------------------------------------------%

\printbibliography[heading=bibintoc]
	
\end{document}