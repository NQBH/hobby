\documentclass{article}
\usepackage[backend=biber,natbib=true,style=authoryear]{biblatex}
\addbibresource{/home/nqbh/reference/bib.bib}
\usepackage{tocloft}
\renewcommand{\cftsecleader}{\cftdotfill{\cftdotsep}}
\usepackage[colorlinks=true,linkcolor=blue,urlcolor=red,citecolor=magenta]{hyperref}
\usepackage{algorithm,algpseudocode,amsmath,amssymb,amsthm,float,graphicx,mathtools}
\allowdisplaybreaks
\numberwithin{equation}{section}
\newtheorem{assumption}{Assumption}[section]
\newtheorem{conjecture}{Conjecture}[section]
\newtheorem{corollary}{Corollary}[section]
\newtheorem{definition}{Definition}[section]
\newtheorem{example}{Example}[section]
\newtheorem{lemma}{Lemma}[section]
\newtheorem{notation}{Notation}[section]
\newtheorem{principle}{Principle}[section]
\newtheorem{problem}{Problem}[section]
\newtheorem{proposition}{Proposition}[section]
\newtheorem{question}{Question}[section]
\newtheorem{remark}{Remark}[section]
\newtheorem{theorem}{Theorem}[section]
\usepackage[left=0.5in,right=0.5in,top=1.5cm,bottom=1.5cm]{geometry}
\usepackage{fancyhdr}
\pagestyle{fancy}
\fancyhf{}
\lhead{\small Sect.~\thesection}
\rhead{\small\nouppercase{\leftmark}}
\renewcommand{\sectionmark}[1]{\markboth{#1}{}}
\cfoot{\thepage}
\def\labelitemii{$\circ$}

\title{A Mathematician's Apology}
\author{G. H. Hardy}
\date{\today}

\begin{document}
\maketitle
\tableofcontents

%------------------------------------------------------------------------------%

\section*{Foreword}

%------------------------------------------------------------------------------%

\section*{Preface}
``I am indebted for many valuable criticisms to Prof. C. D. Board \& Dr. C. P. Snow, each of whom read my original manuscript. I have incorporated the substance of nearly all of their suggestions in my text, \& have so removed a good many crudities \& obscurities.

In 1 case I have dealt with them differently. My \S28 is based on a short article which I contributed to \textit{Eureka} (the journal of the Cambridge Archimedean Society) early in the year, \& I found it impossible to remodel what I had written so recently \& with so much care. Also, if I had tried to meet such important criticisms seriously, I should have had to expand this section so much as to destroy the whole balance of my essay. I have therefore left it unaltered, but have added a short statement of the chief points made by my critics in a note at the end. G. H. H. Jul 18, 1940'' -- \cite[p. 59]{Hardy1992}

\fbox{\textbf{1}} ``It is a melancholy experience for a professional mathematician to find himself writing about mathematics. The function of a mathematician is to do something, to prove new theorems, to add to mathematics, \& not to talk about what he or other mathematicians have done. Statesmen despise publicists, painters despise art-critics, \& physiologists, physicists, or mathematicians have usually similar feelings; there is no scorn more profound, or on the whole more justifiable, than that of the men who make for the \textit{men} who explain. Exposition, criticism, appreciation, is work for 2nd-rate minds.

I can remember arguing this point once in 1 of the few serious conversations that I ever had with Housman. Housman, in his Leslie Stephen lecture \textit{The Name \& Nature of Poetry}, had denied very emphatically that he was a `critic'; but he had denied it in what seemed to me a singularly perverse way, \& had expressed an admiration for literary criticism which startled \& scandalized me.

He had begun with a quotation from his inaugural lecture, delivered 22 years before--
\begin{quotation}
	Whether the faculty of literary criticism is the best gift that Heaven has in its treasuries, I cannot say; but Heaven seems to think so, for assuredly it is the gift most charily bestowed. Orators \& poets $\ldots$, if rare in comparison with blackberries, are commoner than returns of Halley's comet: literary critics are less common $\ldots$.
\end{quotation}
\& he had continued---
\begin{quotation}
	In these 22 years I have improved in some respects \& deteriorated in others, but I have not so much improved as to become a literary critic, nor so much deteriorated as to fancy that I have become one.
\end{quotation}
It had seemed to me deplorable that a great scholar \& a fine poet should write like this, \&, finding myself next to him in Hall a few weeks later, I plunged in \& said so. Did he really mean what he had said to be taken very seriously? Would the life of the best of critics really have seemed to him comparable with that of a scholar \& a poet? We argued these questions all through dinner, \& I think that finally he agreed with me. I must not seem to claim a dialectical triumph over a man who can no longer contradict me; but `Perhaps not entirely' was, in the end, his reply to the 1st question, \& `Probably no' to the 2nd.

There may have been some doubt about Housman's feelings, \& I do not wish to claim him as on my side; but there is no doubt at all about the feelings of men of science, \& I share them fully. If then I find myself writing, not mathematics but `about' mathematics, it is a confession of weakness, for which I may rightly be scorned or pitied by younger \& more vigorous mathematicians. I write about mathematics because, like any other mathematician who has passed 60, I have no longer the freshness of mind, the energy, or the patience to carry on effectively with my proper job.'' -- \cite[pp. 61--63]{Hardy1992}

\fbox{\textbf{2}} ``I propose to put forward an apology for mathematics; \& I may be told that it needs none, since there are now few studies more generally recognized, for good reasons or bad, as profitable \& praiseworthy. This may be true; indeed it is probable, since the sensational triumphs of Einstein, that stellar astronomy \& atomic physics are the only sciences which stand higher in popular estimation. A mathematician need not now consider himself on the defensive. He does not have to meet the sort of opposition described by Bradley in the admirable defence of metaphysics which forms the introduction to \textit{Appearance \& Reality}.

A metaphysician, says Bradley, will be told that `metaphysical knowledge is wholly impossible', or that `even if possible to a certain degree, it is practically no knowledge worth the name'. `The same problems,' he will hear, `the same disputes, the same sheer failure. Why not abandon it \& come out? Is there nothing else more worth your labor?' There is no one so stupid as to use this sort of language about mathematics. The mass of mathematical truth is obvious \& imposing; its practical applications, the bridges \& steam-engines \& dynamos, obtrude themselves on the dullest imagination. The public does not need to be convinced that there is something in mathematics.

All this is in its way very comforting to mathematicians, but it is hardly possible for a genuine mathematician to be content with it. Any genuine mathematician must feel that it is not on these crude achievements that the real case for mathematics rests, that the popular reputation of mathematics is based largely on ignorance \& confusion, \& that there is room for a more rational defence. At any rate, I am disposed to try to make one. It should be a simpler task than Bradley's difficult  apology.

I shall ask, then, \textit{why is it really worth while to make a serious study of mathematics? What is the proper justification of a mathematician's life?} \& my answers will be, for the most part, such as are to be expected from a mathematician: I think that it is worth while, that there is ample justification. But I should say at once that my defence of mathematics will be a defence of myself, \& that my apology is bound to be to some extent egotistical.  I should not think it worth while to apologize for my subject if I regarded myself as 1 of its failures.

Some egotism of this sort is inevitable, \& I do not feel that it really needs justification. Good work is not done by `humble' men. It is 1 of the 1st duties of a professor, e.g., in any subject, to exaggerate a little both the importance of his subject \& his own importance it it. A man who is always asking `Is what I do worth while?' \& `Am I the right person to do it?' will always be ineffective himself \& a discouragement to others. He must shut his eyes a little \& think a little more of his subject \& himself than they deserve. This is not too difficult: it is harder not to make his subject \& himself ridiculous by shutting his eyes too tightly.'' -- \cite[pp. 63--66]{Hardy1992}

\fbox{\textbf{3}} ``A man who sets out to justify his existence \& his activities has to distinguish 2 different questions. The 1st is whether the work which he does is worth doing; \& the 2nd is why he does it, whatever its value may be. The 1st question is often very difficult, \& the answer very discouraging, but most people will find the 2nd easy enough even then. Their answers, if they are honest, will usually take 1 or other of 2 forms; \& the 2nd form is merely a humbler variation of the 1st, which is the only answer which we need consider seriously.

(1) `I do what I do because it is the one \& only thing that I can do at all well. I am a lawyer, or a stockbroker, or a professional cricketer, because I have some real talent for that particular job. I am a lawyer because I have a fluent tongue, \& am interested in legal subtleties; I am a stockbroker because my judgment of the markets is quick \& sound; I am a professional cricketer because I can bat unusually well. I agree that it might be better to be a poet or a mathematician, but unfortunately I have no talent for such pursuits.'

I am not suggesting that this is a defence by which can be made by most people, since most people can do nothing at all well. But it is impregnable when it can be made without absurdity, as it can by a substantial minority: perhaps 5 or even 10\% of men can do something rather well. It is a tiny minority who can do anything \textit{really} well, \& the number of men who can do 2 things well is negligible. If a man has any genuine talent, he should be ready to make almost any sacrifice in order to cultivate it to the full.

This view was endorsed by Dr Johnson---
\begin{quotation}
	When I told him that I had been to see [his namesake] Johnson ride upon 3 horses, he said `Such a man, sir, should be encouraged, for his performances show the extent of the human powers $\ldots$'---
\end{quotation}
\& similarly he would have an applauded mountain climbers, channel swimmers, \& blindfold chess-players. For my own part, I am entirely in sympathy with all such attempts at remarkable achievement. I feel some sympathy even with conjurors \& ventriloquists; \& when Alekhine \& Bradman set out to beat records, I am quite bitterly disappointed if they fail. \& here both Dr Johnson \& I find ourselves in agreement with the public. As W. J. Turner has said so truly, it is only the `highbrows' (in the unpleasant sense) who do not admire the `real swell'.

We have of course to take account of the differences in value between different activities. I would rather be a novelist or a painter than a statesman of similar rank; \& there are many roads to fame which most of us would reject as actively pernicious. Yet it is seldom that such differences of value will turn the scale in a man's choice of a career, which will almost always be dictated by the limitations of his natural abilities. Poetry is more valuable than cricket, but Bradman would be a fool if he sacrificed his cricket in order to write 2nd-rate minor poetry (\& I suppose that it is unlikely that he could do better). If the cricket were a little less supreme, \& the poetry better, then the choice might be more difficult: I do not know whether I would rather have been Victor Trumper or Rupert Brooke. It is fortunate that such dilemmas occur so seldom.

I may add that they are particularly unlikely to present themselves to a mathematician. It is usual to exaggerate rather grossly the differences between the mental processes of mathematicians \& other people, but it is undeniable that a gift for mathematics is 1 of the most specialized talents, \& that mathematicians as a class are not particularly distinguished for general ability or versatility. If a man is in any sense a real mathematician, then it is 100 to 1 that his mathematics will be far better than anything else he can do, \& that he would be silly if he surrendered any decent opportunity of exercising his 1 talent in order to do undistinguished work in other fields. Such a sacrifice could be justified only by economic necessity or age.'' -- \cite[pp. 66--70]{Hardy1992}

\fbox{\textbf{4}} I had better say something here about this question of age, since it is particularly important for mathematicians. No mathematician should ever allow himself to forget that mathematics, more than any other art or science, is a young man's game. To take a simple illustration at a comparatively humble level, the average age of election to the Royal Society is lowest in mathematics.

We can naturally find much more striking illustrations. We may consider, e.g., the career of a man who was certainly 1 of the world's 3 greatest mathematicians. Newton gave up mathematics at 50, \& had lost his enthusiasm long before; he had recognized no doubt by the time that he was 40 that his great creative days were over. His greatest ideas of all, fluxions \& the law of gravitation, came to him about 1666, when he was 24 -- `in those days I was in the prime of my age for invention, \& minded mathematics \& philosophy more than at any time since.' He made big discoveries until he was nearly 40 (the `elliptic orbit' at 37), but after that he did little but polish \& perfect.

Galois died at 21, Abel at 27, Ramanujan at 33, Riemann at 40. There have been men who have done great work a good deal later; Gauss's great memoir on differential geometry was published when he was 50 (though he had had the fundamental ideas 10 years before). I do not know an instance of a major mathematical advance initiated by a man past 50. If a man of mature age loses interest in \& abandons mathematics, the loss is not likely to be very serious either for mathematics or for himself.

On the other hand the gain is no more likely to be substantial; the later records of mathematicians who have left mathematics are not particularly encouraging. Newton made a quite competent Master of the Mint (when he was not quarrelling with anybody). Painlev\'e was a not very successful Premier of France. Laplace's political career was highly discreditable, but he is hardly a fair instance, since he was dishonest rather than incompetent, \& never really `gave up' mathematics. It is very hard to find an instance of a 1st-rate mathematician who has abandoned mathematics \& attained 1st-rate distinction in any other field.\footnote{Pascal seems the best.} There may have been young men who would have been 1st-rate mathematicians if they had stuck to mathematics, but I have never heard of a really plausible example. \& all this is fully borne out by my own very limited experience. Every young mathematician of real talent whom I have known has been faithful to mathematics, \& not from lack of ambition but from abundance of it; they have all recognized that there, if anywhere, lay the road to a life of any distinction.'' -- \cite[pp. 70--73]{Hardy1992}

\fbox{\textbf{5}} There is also what I called the `humbler variation' of the standard apology; but I may dismiss this in a very few words.

(2) `There is \textit{nothing} that I can do particularly well. I do what I do because it came my way. I really never had a chance of doing anything else.' \& this apology too I accept as conclusive. It is quite true that most people can do nothing well. if so, it matters very little what career they choose, \& there is really nothing more to say about it. It is a conclusively reply, but hardly one likely to be made by a man with any pride; \& I may assume that none of us would be content with it.'' -- \cite[p. 73]{Hardy1992}

\fbox{\textbf{6}} It is time to begin thinking about the 1st question which I put in \S3, \& which is so much more difficult than the 2nd. Is mathematics, what I \& other mathematicians mean by mathematics, worth doing; \& if so, why?

I have been looking again at the 1st pages of the inaugural lecture which I gave at Oxford in 1920, where there is an outline of an apology for mathematics. It is very inadequate (less than a couple of pages), \& it is written in a style (a 1st essay, I suppose, in what I then imagined to be the `Oxford manner') of which I am not now particularly proud; but I still feel that, however much development it may need, it contains the essentials of the matter. I will resume what I said then, as a preface to a fuller discussion.

(1) I began by laying stress on the \textit{harmlessness} of mathematics -- `the study of mathematics is, if an unprofitable, a perfectly harmless \& innocent occupation'. I shall stick to that, but obviously it will need a good deal of expansion \& explanation.

'' -- \cite[pp. 74--]{Hardy1992}

%------------------------------------------------------------------------------%

\printbibliography[heading=bibintoc]
	
\end{document}