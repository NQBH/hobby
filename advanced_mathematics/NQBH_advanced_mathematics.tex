\documentclass[oneside]{book}
\usepackage[backend=biber,natbib=true,style=authoryear]{biblatex}
\addbibresource{/home/hong/1_NQBH/reference/bib.bib}
\usepackage[vietnamese,english]{babel}
\usepackage{tocloft}
\renewcommand{\cftsecleader}{\cftdotfill{\cftdotsep}}
\usepackage[colorlinks=true,linkcolor=blue,urlcolor=red,citecolor=magenta]{hyperref}
\usepackage{amsmath,amssymb,amsthm,mathtools,float,graphicx}
\allowdisplaybreaks
\numberwithin{equation}{section}
\newtheorem{assumption}{Assumption}[chapter]
\newtheorem{conjecture}{Conjecture}[chapter]
\newtheorem{corollary}{Corollary}[chapter]
\newtheorem{definition}{Definition}[chapter]
\newtheorem{example}{Example}[chapter]
\newtheorem{lemma}{Lemma}[chapter]
\newtheorem{notation}{Notation}[chapter]
\newtheorem{principle}{Principle}[chapter]
\newtheorem{problem}{Problem}[chapter]
\newtheorem{proposition}{Proposition}[chapter]
\newtheorem{question}{Question}[chapter]
\newtheorem{remark}{Remark}[chapter]
\newtheorem{theorem}{Theorem}[chapter]
\usepackage[left=0.5in,right=0.5in,top=1.5cm,bottom=1.5cm]{geometry}
\usepackage{fancyhdr}
\pagestyle{fancy}
\fancyhf{}
\lhead{\small \textsc{Sect.} ~\thesection}
\rhead{\small \nouppercase{\leftmark}}
\renewcommand{\sectionmark}[1]{\markboth{#1}{}}
\cfoot{\thepage}
\def\labelitemii{$\circ$}

\title{Advanced Mathematics}
\author{\selectlanguage{vietnamese} Nguyễn Quản Bá Hồng\footnote{Independent Researcher, Ben Tre City, Vietnam\\e-mail: \texttt{nguyenquanbahong@gmail.com}}}
\date{\today}

\begin{document}
\maketitle
\setcounter{secnumdepth}{4}
\setcounter{tocdepth}{4}
\tableofcontents

%------------------------------------------------------------------------------%

\chapter{Wikipedia's}

\section{\href{https://en.wikipedia.org/wiki/Symmetrization_methods}{Wikipedia\texttt{/}Symmetrization Methods}}
``In mathematics the \textit{symmetrization methods} are algorithms of transforming a set $A\subset\mathbb{R}^n$ to a ball $B\subset\mathbb{R}^n$ with equal volume $\operatorname{vol}(B) = \operatorname{vol}(A)$ \& centered at the origin. $B$ is called the \textit{symmetrized version} of $A$, usually denoted $A^\star$. These algorithms show up in solving the classical \href{https://en.wikipedia.org/wiki/Isoperimetric_inequality}{isoperimetric inequality} problem, which asks: Given all 2D shapes of a given area, which of them has the minimal \href{https://en.wikipedia.org/wiki/Perimeter}{perimeter}. The conjectured answer was the disk \& \href{https://en.wikipedia.org/wiki/Jakob_Steiner}{Steiner} in 1838 showed this to be true using the Steiner symmetrization method. From this many other isoperimetric problems sprung \& other symmetrization algorithms. E.g., Rayleigh's conjecture is that the 1st \href{https://en.wikipedia.org/wiki/Eigenvalue}{eigenvalue} of the \href{https://en.wikipedia.org/wiki/Dirichlet_problem}{Dirichlet problem} is minimized for the ball (see \href{https://en.wikipedia.org/wiki/Rayleigh%E2%80%93Faber%E2%80%93Krahn_inequality}{Rayleigh--Faber--Krahn inequality} for details). Another problem is that the Newtonian \href{https://en.wikipedia.org/wiki/Capacity_of_a_set}{capacity of a set} $A$ is minimized by $A^\star$ \& this was proved by Polya \& G. Szego (1951) using circular symmetrization.'' -- \href{https://en.wikipedia.org/wiki/Symmetrization_methods}{Wikipedia\texttt{/}symmetrization methods}

\subsection{Symmetrization}
``If $\Omega\subset\mathbb{R}^n$ is measurable, then it is denoted by $\Omega^\star$ the symmetrized version of $\Omega$, i.e., a ball $\Omega^\star\coloneqq B_r(0)\subset\mathbb{R}^n$ s.t. $\operatorname{vol}(\Omega^\star) = \operatorname{vol}(\Omega)$. We denote by $f^\star$ the \href{https://en.wikipedia.org/wiki/Symmetric_decreasing_rearrangement}{symmetric decreasing rearrangement} of nonnegative measurable function $f$ \& define it as $f^\star(x)\coloneqq\int_0^\infty 1_{\{y:f(y) > t\}^\star}(x){\rm d}t$, where $\{y:f(y) > t\}^\star$ is the symmetrized version of preimage set $\{y:f(y) > t\}$. The methods described below have been proved to transform $\Omega$ to $\Omega^\star$, i.e., given a sequence of symmetrization transformations $\{T_k\}$ there is $\lim_{k\to\infty} d_{\rm Ha}(\Omega^\star,T_k(K)) = 0$, where $d_{\rm Ha}$ is the \href{https://en.wikipedia.org/wiki/Hausdorff_distance}{Hausdorff distance} (for discussion \& proofs see [Burchard2009]).'' -- \href{https://en.wikipedia.org/wiki/Symmetrization_methods#Symmetrization}{Wikipedia\texttt{/}symmetrization methods\texttt{/}symmetrization}

\subsection{Steiner symmetrization}
\textsf{Steiner Symmetrization of set $\Omega$.}

``Steiner symmetrization was introduced by Steiner (1838) to solve the isoperimetric theorem stated above. Let $H\subset\mathbb{R}^n$ be a \href{https://en.wikipedia.org/wiki/Hyperplane}{hyperplane} through the origin. Rotate space so that $H$ is the $x_n = 0$ ($x_n$ is $n$th coordinate in $\mathbb{R}^n$) hyperplane. For each ${\bf x}\in H$ let the perpendicular line through ${\bf x}\in H$ be $L_{\bf x} = \{{\bf x} + y{\bf e}_n:y\in\mathbb{R}\}$. Then by replacing each $\Omega\cap L_{\bf x}$ by a line centered at $H$ \& with length $|\Omega\cap L_{\bf x}|$ we obtain the \textit{Steiner symmetrized version}.
\begin{align*}
	\operatorname{St}(\Omega)\coloneqq\left\{{\bf x} + y{\bf e}_n:{\bf x} + z{\bf e}_n\in\Omega\mbox{ for some }{\bf z}\mbox{ \& }|y|\le\frac{1}{2}|\Omega\cap L_x|\right\}.
\end{align*}
It is denoted by $\operatorname{St}(f)$ the \textit{Steiner symmetrization} w.r.t. $x_n = 0$ hyperplane of nonnegative measurable function $f:\mathbb{R}^d\to\mathbb{R}$ \& for fixed $x_1,\ldots,x_{n-1}$ define it as $\operatorname{St}:f(x_1,\ldots,x_{n-1},\cdot)\mapsto(f(x_1,\ldots,x_{n-1}))^\star$.

\subsubsection{Properties}
It preserves convexity: if $\Omega$ is convex, then $\operatorname{St}(\Omega)$ is also convex. It is linear: $\operatorname{St}({\bf x} + \lambda\Omega) = \operatorname{St}({\bf x}) + \lambda\operatorname{St}(\Omega)$. Super-additive: $\operatorname{St}(K) + \operatorname{St}(U)\subset\operatorname{St}(K + U)$.'' -- \href{https://en.wikipedia.org/wiki/Symmetrization_methods#Steiner_symmetrization}{Wikipedia\texttt{/}symmetrization methods\texttt{/}Steiner symmetrization}

\subsection{Circular symmetrization}
\textsf{Fig. Circular symmetrization of set $\Omega$.}

``A popular method for symmetrization in the plane is \textit{Polya's circular symmetrization}. After, its generalization will be described to higher dimensions. Let $\Omega\subset\mathbb{C}$ be a domain; then its circular symmetrization $\operatorname{Circ}(\Omega)$ with regard to the positive real axis is defined as follows: Let $\Omega_t\coloneqq\{\theta\in[0,2\pi]:te^{i\theta}\in\Omega\}$, i.e., contain the arcs of radius $t$ contained in $\Omega$. So it is defined
\begin{itemize}
	\item If $\Omega_t$ is the full circle, then $\operatorname{Circ}(\Omega)\cap\{|z| = t\}\coloneqq\{|z| = t\}$.
	\item If the length is $\operatorname{m}(\Omega_t) = \alpha$, then $\operatorname{Circ}(\Omega)\cap\{|z| = t\}\coloneqq\left\{te^{i\theta}:|\theta| < \frac{\alpha}{2}\right\}$.
	\item $0,\infty\in\operatorname{Circ}(\Omega)$ if $0,\infty\in\Omega$.
\end{itemize}
In higher dimensions $\Omega\subset\mathbb{R}^n$, its spherical symmetrization $\operatorname{Sp}^n(\Omega)$ w.r.t. the positive axis of $x_1$ is defined as follows: Let $\Omega_r\coloneqq\{{\bf x}\in\mathbb{S}^{n-1}:r{\bf x}\in\Omega\}$, i.e., contain the caps of radius $r$ contained in $\Omega$. Also, for the 1st coordinate let $\operatorname{angle}(x_1)\coloneqq\theta$ if $x_1 = r\cos\theta$. So as above
\begin{itemize}
	\item If $\Omega_r$ is the full cap, then $\operatorname{Sp}^n(\Omega)\cap\{|z| = r\}\coloneqq\{|z| = t\}$.
	\item If the surface area is $\operatorname{m}_{\rm s}(\Omega_t) = \alpha$, then $\operatorname{Sp}^n(\Omega)\cap\{|z| = r\}\coloneqq\{x:|x| = r\mbox{ \& }0\le\operatorname{angle}(x_1)\le\theta_\alpha\}=:C(\theta_\alpha)$ where $\theta_\alpha$ is picked so that its surface area is $\operatorname{m}_s(C(\theta_\alpha)) = \alpha$. In words, $C(\theta_\alpha)$ is a cap symmetric around the positive axis $x_1$ with the same area as the intersection $\Omega\cap\{|z| = r\}$.
	\item $0,\infty\in\operatorname{Sp}^n(\Omega)$ iff $0,\infty\in\Omega$.'' -- \href{https://en.wikipedia.org/wiki/Symmetrization_methods#Circular_symmetrization}{Wikipedia\texttt{/}symmetrization methods\texttt{/}circular symmetrization}
\end{itemize}

\subsection{Polarization}
\textsf{Fig: Polarization of set $\Omega$.}

``Let $\Omega\subset\mathbb{R}^n$ be a domain \& $H^{n-1}\subset\mathbb{R}^n$ be a hyperplane through the origin. Denote the reflection across that plane to the positive halfspace $\mathbb{H}^+$ as $\sigma_H$ or just $\sigma$ when it is clear from the context. Also, the reflected $\Omega$ across hyperplane $H$ is defined as $\sigma\Omega$. Then, the polarized $\Omega$ is denoted as $\Omega^\alpha$ \& defined as follows
\begin{itemize}
	\item If ${\bf x}\in\Omega\cap\mathbb{H}^+$, then ${\bf x}\in\Omega^\alpha$.
	\item If ${\bf x}\in\Omega\cap\sigma(\Omega)\cap\mathbb{H}^-$, then ${\bf x}\in\Omega^\sigma$.
	\item If ${\bf x}\in(\Omega\backslash\sigma(\Omega))\cap\mathbb{H}^-$, then $\sigma{\bf x}\in\Omega^\sigma$.
\end{itemize}
In words, $(\Omega\backslash\sigma(\Omega))\cap\mathbb{H}^-$ is simply reflected to the halfspace $\mathbb{H}^+$. It turns out that this transformation can approximate the above ones (in the \href{https://en.wikipedia.org/wiki/Hausdorff_distance}{Hausdorff distance}) (see [Brock \& Solynin2000]).'' -- \href{https://en.wikipedia.org/wiki/Symmetrization_methods#Polarization}{Wikipedia\texttt{/}symmetrization methods\texttt{/}polarization}

%------------------------------------------------------------------------------%

\section{\href{https://en.wikipedia.org/wiki/Interpolation_space}{Wikipedia\texttt{/}Interpolation Space}}
``In the field of \href{https://en.wikipedia.org/wiki/Mathematical_analysis}{mathematical analysis}, an \textit{interpolation space} is a space which lies ``in between'' 2 other \href{https://en.wikipedia.org/wiki/Banach_space}{Banach spaces}. The main applications are in \href{https://en.wikipedia.org/wiki/Sobolev_space}{Sobolev spaces}, where spaces of functions that have a noninteger number of \href{https://en.wikipedia.org/wiki/Derivative}{derivatives} are interpolated from the spaces of functions with integer number of derivatives.'' -- \href{https://en.wikipedia.org/wiki/Interpolation_space}{Wikipedia\texttt{/}interpolation space}

\subsection{History}
``The theory of interpolation of vector spaces began by an observation of \href{https://en.wikipedia.org/wiki/J%C3%B3zef_Marcinkiewicz}{J\'ozef Marcinkiewicz}, later generalized \& now known as the \href{https://en.wikipedia.org/wiki/Riesz-Thorin_theorem}{Riesz--Thorin theorem}. In simple terms, if a linear function is continuous on a certain \href{https://en.wikipedia.org/wiki/Lp_space}{space $L^p$} \& also on a certain space $L^q$, then it is also continuous on the space $L^r$, for any intermediate $r$ between $p$ \& $q$. In other words, $L^r$ is a space which is intermediate between $L^p$ \& $L^q$.

In the development of Sobolev spaces, it became clear that the trace spaces were not any of the usual function spaces (with integer number of derivatives), \& \href{https://en.wikipedia.org/wiki/Jacques-Louis_Lions}{Jacques-Louis Lions} discovered that indeed these trace spaces were constituted of functions that have a noninteger degree of differentiability.

Many methods were designed to generate such spaces of functions, including the \href{https://en.wikipedia.org/wiki/Fourier_transform}{Fourier transform}, complex interpolation, real interpolation, as well as other tools (see e.g. \href{https://en.wikipedia.org/wiki/Fractional_derivative}{fractional derivative}).'' -- \href{https://en.wikipedia.org/wiki/Interpolation_space#History}{Wikipedia\texttt{/}interpolation space\texttt{/}history}

\subsection{The setting of interpolation}
``A \href{https://en.wikipedia.org/wiki/Banach_space}{Banach space} $X$ is said to be \textit{continuously embedded} in a Hausdorff \href{https://en.wikipedia.org/wiki/Topological_vector_space}{topological vector space} $Z$ when $X$ is a linear subspace of $Z$ s.t. the inclusion map from $X$ into $Z$ is continuous. A \textit{compatible couple} $(X_0,X_1)$ of Banach spaces consists of 2 Banach spaces $X_0$ \& $X_1$ that are continuously embedded in the same Hausdorff topological vector space $Z$. The embedding in a linear space $Z$ allows to consider the 2 linear subspaces $X_0\cap X_1$ \& $X_0 + X_1 = \{z\in Z;z = x_0 + x_1,\,x_0\in X_0,\,x_1\in X_1\}$. Interpolation does not depend only upon the isomorphic (nor isometric) equivalence classes of $X_0$ \& $X_1$. It depends in an essential way from the specific \textit{relative position} that $X_0$ \& $X_1$ occupy in a larger space $Z$. One can define norms on $X_0\cap X_1$ \& $X_0 + X_1$ by $\|x\|_{X_0\cap X_1}\coloneqq\max\left(\|x\|_{X_0},\|x\|_{X_1}\right)$, $\|x\|_{X_0 + X_1}\coloneqq\inf\left\{\|x_0\|_{X_0} + \|x_1\|_{X_1};x = x_0 + x_1,\,x_0\in X_0,\,x_1\in X_1\right\}$. Equipped with these norms, the intersection \& the sum are Banach spaces. The following inclusions are all continuous: $X_0\cap X_1\subset X_0$, $X_1\subset X_0 + X_1$. Interpolation studies the family of spaces $X$ that are \textit{intermediate spaces} between $X_0$ \& $X_1$ in the sense that $X_0\cap X_1\subset X\subset X_0 + X_1$, where the 2 inclusions maps are continuous.

An example of this situation is the pair ($L^1(\mathbb{R})$, $L^\infty(\mathbb{R})$), where the 2 Banach spaces are continuously embedded in the space $Z$ of measurable functions on the real line, equipped with the topology of convergence in measure. In this situation, the spaces $L^p(\mathbb{R})$, for $1\le p\le\infty$ are intermediate between $L^1(\mathbb{R})$ \& $L^\infty(\mathbb{R})$. More generally,
\begin{align*}
	L^{p_0}(\mathbb{R})\cap L^{p_1}(\mathbb{R})\subset L^p(\mathbb{R})\subset L^{p_0}(\mathbb{R}) + L^{p_1}(\mathbb{R}),\mbox{ when } 1\le p_0\le p\le p_1\le\infty,
\end{align*}
with continuous injections, so that, under the given condition, $L^p(\mathbb{R})$ is intermediate between $L^{p_0}(\mathbb{R})$ \& $L^{p_1}(\mathbb{R})$.

\begin{definition}[Interpolation pair]
	Given 2 compatible couples $(X_0,X_1)$ \& $(Y_0,Y_1)$, an \emph{interpolation pair} is a couple $(X,Y)$ of Banach spaces with the 2 following properties:
	\begin{itemize}
		\item The space $X$ is intermediate between $X_0$ \& $X_1$, \& $Y$ is intermediate between $Y_0$ \& $Y_1$.
		\item If $L$ is any linear operator from $X_0 + X_1$ to $Y_0 + Y_1$, which maps continuously $X_0$ to $Y_0$ \& $X_1$ to $Y_1$, then it also maps continuously $X$ to $Y$.
	\end{itemize}
\end{definition}
The interpolation pair $(X,Y)$ is said to be of \textit{exponent} $\theta$ (with $0 < \theta < 1$) if there exists a constant $C$ s.t. $\|L\|_{X,Y}\le C\|L\|_{X_0,Y_0}^{1-\theta}\|L\|_{X_1,Y_1}^\theta$ for all operators $L$ as above. The notation $\|L\|_{X,Y}$  is for the norm of $L$ as a map from $X$ to $Y$. If $C = 1$, we say that $(X,Y)$ is an \textit{exact interpolation pair of exponent $\theta$}.'' -- \href{https://en.wikipedia.org/wiki/Interpolation_space#The_setting_of_interpolation}{Wikipedia\texttt{/}interpolation space\texttt{/}the setting of interpolation}

\subsection{Complex interpolation}
``If the scalars are \href{https://en.wikipedia.org/wiki/Complex_number}{complex numbers}, properties of complex \href{https://en.wikipedia.org/wiki/Analytic_function}{analytic functions} are used to define an interpolation space. Given a compatible couple $(X_0,X_1)$ of Banach spaces, the linear space $\mathcal{F}(X_0,X_1)$ consists of all functions $f:\mathbb{C}\to X_0 + X_1$, that are analytic on $S = \{z:0 < \operatorname{Re}(z) < 1\}$, continuous on $\overline{S} = \{z:0\le\operatorname{Re}(z)\le 1\}$, \& for which all the following subsets are bounded: $\{f(z):z\in S\}\subset X_0 + X_1$, $\{f(it):t\in\mathbb{R}\}\subset X_0$, $\{f(1 + it):t\in\mathbb{R}\}\subset X_1$. $\mathcal{F}(X_0,X_1)$ is a Banach space under the norm
\begin{align*}
	\|f\|_{\mathcal{F}(X_0,X_1)}\coloneqq\max\left\{\sup_{t\in\mathbb{R}} \|f(it)\|_{X_0},\sup_{t\in\mathbb{R}} \|f(1 + it)\|_{X_1}\right\}.
\end{align*}

\begin{definition}
	For $0 < \theta < 1$, the \emph{complex interpolation space} $(X_0,X_1)_\theta$ is the linear subspace of $X_0 + X_1$ consisting of all values $f(\theta)$ when $f$ varies in the preceding space of functions, $(X_0,X_1)_\theta = \{x\in X_0 + X_1:x = f(\theta),\,f\in\mathcal{F}(X_0,X_1)\}$. The norm on the complex interpolation space $(X_0,X_1)_\theta$ is defined by $\|x\|_\theta = \inf\left\{\|f\|_{\mathcal{F}(X_0,X_1)}:f(\theta) = x,\,f\in\mathcal{F}(X_0,X_1)\right\}$.
\end{definition}
Equipped with this norm, the complex interpolation space $(X_0,X_1)_\theta$ is a Banach space.

\begin{theorem}
	Given 2 compatible couples of Banach spaces $(X_0,X_1)$ \& $(Y_0,Y_1)$, the pair $((X_0,X_1)_\theta,(Y_0,Y_1)_\theta)$ is an exact interpolation pair of exponent $\theta$, i.e., if $T:X_0 + X_1\to Y_0 + Y_1$, is a linear operator bounded from $X_j$ to $Y_j$, $j = 0,1$, then $T$ is bounded from $(X_0,X_1)_\theta$ to $(Y_0,Y_1)_\theta$ \& $\|T\|_\theta\le\|T\|_0^{1-\theta}\|T\|_1^\theta$.
\end{theorem}
The family of $L^p$ spaces (consisting of complex valued functions) behaves well under complex interpolation. If $(R,\Sigma,\mu)$ is an arbitrary \href{https://en.wikipedia.org/wiki/Measure_(mathematics)}{measure space}, if $1\le p_0,p_1\le\infty$ \& $0 < \theta < 1$, then
\begin{align*}
	(L^{p_0}(R,\Sigma,\mu),L^{p_1}(R,\Sigma,\mu))_\theta = L^p(R,\Sigma,\mu),\ \frac{1}{p} = \frac{1 - \theta}{p_0} + \frac{\theta}{p_1},
\end{align*}
with equality of norms. This fact is closely related to the \href{https://en.wikipedia.org/wiki/Riesz%E2%80%93Thorin_theorem}{Riesz--Thorin theorem}.'' - \href{https://en.wikipedia.org/wiki/Interpolation_space#Complex_interpolation}{Wikipedia\texttt{/}interpolation space\texttt{/}complex interpolation}

\subsection{Real interpolation}
``There are 2 ways for introducing the \textit{real interpolation method}. The 1st \& most commonly used when actually identifying examples of interpolation spaces is the K-method. The 2nd method, the J-method, gives the same interpolation spaces as the K-method when the parameter $\theta$ is in $(0,1)$. That the J- \& K-methods agree is important for the study of duals of interpolation spaces: basically, the dual of an interpolation space constructed by the K-method appears to be a space constructed form the dual couple by the J-method.'' -- \href{https://en.wikipedia.org/wiki/Interpolation_space#Real_interpolation}{Wikipedia\texttt{/}interpolation space\texttt{/}real interpolation}

\subsubsection{K-method}
``The K-method of real interpolation can be used for Banach spaces over the field $\mathbb{R}$ of \href{https://en.wikipedia.org/wiki/Real_number}{real numbers}.

\begin{definition}
	Let $(X_0,X_1)$ be a compatible couple of Banach spaces. For $t > 0$ \& every $x\in X_0 + X_1$, let $K(x,t;X_0,X_1) = \inf\left\{\|x_0\|_{X_0} + t\|x_1\|_{X_1}\:x = x_0 + x_1,\,x_0\in X_0,\,x_1\in X_1\right\}$. Changing the order of the 2 spaces results in: $K(x,t;X_0,X_1) = tK(x,t^{-1};X_1,X_0)$. Let
	\begin{align*}
		\|x\|_{\theta,q;K} &= \left(\int_0^\infty \left(t^{-\theta}K(x,t;X_0,X_1)\right)^q\frac{{\rm d}t}{t}\right)^{\frac{1}{q}},\ 0 < \theta < 1,\,1\le q < \infty,\\
		\|x\|_{\theta,\infty;K} &= \sup_{t > 0} t^{-\theta}K(x,t;X_0,X_1),\ 0\le\theta\le 1.
	\end{align*}
	The K-method of real interpolation consists in taking $K_{\theta,q}(X_0,X_1)$ to be the linear subspace of $X_0 + X_1$ consisting of all $x$ s.t. $\|x\|_{\theta,q;K} < \infty$.
\end{definition}

\begin{example}
	An important example is that of the couple $(L^1(\mathbb{R},\Sigma,\mu),L^\infty(\mathbb{R},\Sigma,\mu))$, where the functional $K(t,f;L^1,L^\infty)$ can be computed explicitly. The measure $\mu$ is supposed \href{https://en.wikipedia.org/wiki/%CE%A3-finite_measure}{$\sigma$-finite}. In this context, the best way of cutting the function $f\in L^1 + L^\infty$ as sum of 2 functions $f_0\in L^1$ \& $f_1\in L^\infty$ is, for some $s > 0$ to be chosen as function of $t$, to let $f_1(x)$ be given for all $x\in\mathbb{R}$ by
	\begin{equation*}
		f_1(x) = \left\{\begin{split}
			&f(x),&&|f(x)| < s,\\
			&\frac{sf(x)}{|f(x)|},&&\mbox{otherwise}.
		\end{split}\right.
	\end{equation*}
	The optimal choice of $s$ leads to the formula
	\begin{align*}
		K(f,t;L^1,L^\infty) = \int_0^1 f^\star(u)\,{\rm d}u,
	\end{align*}
	where $f^\star$ is the \href{https://en.wikipedia.org/wiki/Lorentz_space#Decreasing_rearrangements}{decreasing rearrangement} of $f$.'' -- \href{https://en.wikipedia.org/wiki/Interpolation_space#K-method}{Wikipedia\texttt{/}interpolation space\texttt{/}real interpolation\texttt{/}K-method}
\end{example}

\subsubsection{J-method}
``As with the K-method, the J-method can be used for real Banach spaces.

\begin{definition}
	Let $(X_0,X_1)$ be a compatible couple of Banach spaces. For $t > 0$ \& for every vector $x\in X_0\cap X_1$, let $J(x,t;X_0,X_1) = \max(\|x\|_{X_0},t\|x\|_{X_1})$. A vector $x$ in $X_0 + X_1$ belongs to the interpolation space $J_{\theta,q}(X_0,X_1)$ iff it can be written as $x = \int_0^\infty v(t)\frac{{\rm d}t}{t}$, where $v(t)$ is measurable with values in $X_0\cap X_1$ \& s.t.
	\begin{align*}
		\Phi(v) = \left(\int_0^\infty \left(t^{-\theta}J(v(t),t;X_0,X_1)\right)^q\frac{{\rm d}t}{t}\right)^{\frac{1}{q}} < \infty.
	\end{align*}
	The norm of $x$ in $J_{\theta,q}(X_0,X_1)$ is given by the formula
	\begin{align*}
		\|x\|_{\theta,q;J}\coloneqq\inf_v\left\{\Phi(v):x = \int_0^\infty v(t)\frac{{\rm d}t}{t}\right\}.
	\end{align*}
\end{definition}
\noindent'' -- \href{https://en.wikipedia.org/wiki/Interpolation_space#J-method}{Wikipedia\texttt{/}interpolation space\texttt{/}real interpolation\texttt{/}J-method}

\subsubsection{Relations between the interpolation methods}
``The 2 real interpolation methods are equivalent when $0 < \theta < 1$.

\begin{theorem}
	Let $(X_0,X_1)$ be a compatible couple of Banach spaces. If $0 < \theta < 1$ \& $1\le q\le\infty$, then $J_{\theta,q}(X_0,X_1) = K_{\theta,q}(X_0,X_1)$, with \href{https://en.wikipedia.org/wiki/Norm_(mathematics)#Definition}{equivalence of norms}.
\end{theorem}
The theorem covers degenerate cases that have not been excluded: e.g. if $X_0$ \& $X_1$ form a direct sum, then the intersection \& the $J$-spaces are the null space, \& a simple computation shows that the K-spaces are also null.

When $0 < \theta < 1$, one can speak, up to an equivalent renorming, about \textit{the} Banach space obtained by the real interpolation method with parameters $\theta$ \& $q$. The notation for this real interpolation space is $(X_0,X_1)_{\theta,q}$. One has that
\begin{align*}
	(X_0,X_1)_{\theta,q} = (X_1,X_0)_{1-\theta,q},\ 0 < \theta < 1,\,1\le q\le\infty.
\end{align*}
For a given value of $\theta$, the real interpolation spaces increase with $q$: if $0 < \theta < 1$ \& $1\le q\le r\le\infty$, the following continuous inclusion holds true: $(X_0,X_1)_{\theta,q}\subset(X_0,X_1)_{\theta,r}$.

\begin{theorem}
	Given $0 < \theta < 1$, $1\le q\le\infty$ \& 2 compatible couples $(X_0,X_1)$ \& $(Y_0,Y_1)$, the pair $((X_0,X_1)_{\theta,q},(Y_0,Y_1)_{\theta,q})$ is an exact interpolation pair of exponent $\theta$.
\end{theorem}
A complex interpolation space is usually not isomorphic to 1 of the spaces given by the real interpolation method. However, there is a general relationship.

\begin{theorem}
	Let $(X_0,X_1)$ be a compatible couple of Banach spaces. If $0 < \theta < 1$, then $(X_0,X_1)_{\theta,1}\subset(X_0,X_1)_\theta\subset(X_0,X_1)_{\theta,\infty}$.
\end{theorem}

\begin{example}
	When $X_0 = C([0,1])$ \& $X_1 = C^1([0,1])$, the space of continuously differentiable functions on $[0,1]$, the $(\theta,\infty)$ interpolation method, for $0 < \theta < 1$, gives the \href{https://en.wikipedia.org/wiki/H%C3%B6lder_condition}{H\"older space} $C^{0,\theta}$ of exponent $\theta$. This is because the K-functional $K(f,t;X_0,X_1)$ of this couple is equivalent to
	\begin{align*}
		\sup\left\{|f(u)|,\frac{|f(u) - f(v)|}{1 + t^{-1}|u - v|}:u,v\in[0,1]\right\}.
	\end{align*}
	Only values $0 < t < 1$ are interesting here.
\end{example}
Real interpolation between $L^p$ spaces gives the family of \href{https://en.wikipedia.org/wiki/Lorentz_space}{Lorentz spaces}. Assuming $0 < \theta < 1$ \& $1\le q\le\infty$, one has
\begin{align*}
	(L^1(\mathbb{R},\Sigma,\mu),L^\infty(\mathbb{R},\Sigma,\mu))_{\theta,q} = L^{p,q}(\mathbb{R},\Sigma,\mu),\mbox{ where }\frac{1}{p} = 1 - \theta,
\end{align*}
with equivalent norms. This follows from an \href{https://en.wikipedia.org/wiki/Hardy%27s_inequality}{inequality of Hardy} \& from the value given above of the K-functional for this compatible couple. When $q = p$, the Lorentz space $L^{p,p}$ is equal to $L^p$, up to renorming. When $q = \infty$, the Lorentz space $L^{p,\infty}$ is equal to \href{https://en.wikipedia.org/wiki/Lp_space#Weak_Lp}{weak-$L^p$}.'' -- \href{https://en.wikipedia.org/wiki/Interpolation_space#Relations_between_the_interpolation_methods}{Wikipedia\texttt{/}interpolation space\texttt{/}real interpolation\texttt{/}relations between the interpolation methods}

\subsection{The reiteration theorem}
``An intermediate space $X$ of the compatible couple $(X_0,X_1)$ is said to be of \textit{class $\theta$} if $(X_0,X_1)_{\theta,1}\subset X\subset(X_0,X_1)_{\theta,\infty}$, with continuous injections. Beside all real interpolation spaces $(X_0,X_1)_{\theta,q}$ with parameter $\theta$ \& $1\le q\le\infty$, the complex interpolation space $(X_0,X_1)_\theta$ is an intermediate space of class $\theta$ of the compatible couple $(X_0,X_1)$.

The reiteration theorems says, in essence, that interpolating with a parameter $\theta$ behaves, in some way, like forming a \href{https://en.wikipedia.org/wiki/Convex_combination}{convex combination} $a = (1 - \theta)x_0 + \theta x_1$: taking a further convex combination of 2 convex combinations gives another convex combination.

\begin{theorem}
	Let $A_0,A_1$ be intermediate spaces of the compatible couple $(X_0,X_1)$, of class $\theta_0$ \& $\theta_1$ resp., with $0 < \theta_0\ne\theta_1 < 1$. When $0 < \theta < 1$ \& $1\le q\le\infty$, one has $(A_0,A_1)_{\theta,q} = (X_0,X_1)_{\eta,q}$, $\eta = (1 - \theta)\theta_0 + \theta\theta_1$.
\end{theorem}
It is notable that when interpolating with the real method between $A_0 = (X_0,X_1)_{\theta_0,q_0}$ \& $A_1 = (X_0,X_1)_{\theta_1,q_1}$, only the values of $\theta_0$ \& $\theta_1$ matter. Also, $A_0$ \& $A_1$ can be complex interpolation spaces between $X_0$ \& $X_1$, with parameters $\theta_0$ \& $\theta_1$ resp.

There is also a reiteration theorem for the complex method.

\begin{theorem}
	Let $(X_0,X_1)$ be a compatible couple of complex Banach spaces, \& assume that $X_0\cap X_1$ is dense in $X_0$ \& in $X_1$. Let $A_0 = (X_0,X_1)_{\theta_0}$ \& $A_1 = (X_0,X_1)_{\theta_1}$, where $0\le\theta_0\le\theta_1\le 1$. Assume further that $X_0\cap X_1$ is dense in $A_0\cap A_1$. Then, for every $0\le\theta\le 1$,
	\begin{align*}
		\left((X_0,X_1)_{\theta_0},(X_0,X_1)_{\theta_1}\right)_\theta = (X_0,X_1)_\eta,\ \eta = (1 - \theta)\theta_0 + \theta\theta_1.
	\end{align*}
\end{theorem}
The density condition is always satisfied when $X_0\subset X_1$ or $X_1\subset X_0$.'' -- \href{https://en.wikipedia.org/wiki/Interpolation_space#The_reiteration_theorem}{Wikipedia\texttt{/}interpolation space\texttt{/}the reinteration theorem}

\subsection{Duality}
``Let $(X_0,X_1)$ be a compatible couple, \& assume that $X_0\cap X_1$ is dense in $X_0$ \& in $X_1$. In this case, the restriction map from the (continuous) \href{https://en.wikipedia.org/wiki/Dual_space#Continuous_dual_space}{dual} $X_j'$ of $X_j$, $j = 0,1$, to the dual of $X_0\cap X_1$ is 1-1. It follows that the pair of duals $(X_0',X_1')$ is a compatible couple continuously embedded in the dual $(X_0\cap X_1)'$.

For the complex interpolation method, the following duality result holds:

\begin{theorem}
	Let $(X_0,X_1)$ be a compatible couple of complex Banach spaces, \& assume that $X_0\cap X_1$ is dense in $X_0$ \& in $X_1$. If $X_0$ \& $X_1$ are \href{https://en.wikipedia.org/wiki/Reflexive_space}{reflexive}, then the dual of the complex interpolation space is obtained by interpolating the duals, $((X_0,X_1)_\theta)' = (X_0',X_1')_\theta$, $0 < \theta < 1$.
\end{theorem}
In general, the dual of the space $(X_0,X_1)_\theta$ is equal to $(X_0',X_1')^\theta$, a space defined by a variant of the complex method. The upper-$\theta$ \& lower-$\theta$ methods do not coincide in general, but they do if at least 1 of $X_0,X_1$ is a reflexive space.

For the real interpolation method, the duality holds provided that the parameter $q$ is finite:

\begin{theorem}
	Let $0 < \theta < 1$, $1\le q < \infty$ \& $(X_0,X_1)$ a compatible couple of real Banach spaces. Assume that $X_0\cap X_1$ is dense in $X_0$ \& in $X_1$. Then
	\begin{align*}
		\left((X_0,X_1)_{\theta,q}\right)' = (X_0',X_1')_{\theta,q'},\mbox{ where }\frac{1}{q'} = 1 - \frac{1}{q}.
	\end{align*}
\end{theorem}
\noindent'' -- \href{https://en.wikipedia.org/wiki/Interpolation_space#Duality}{Wikipedia\texttt{/}interpolation space\texttt{/}duality}

\subsection{Discrete definitions}
``Since the function $t\to K(x,t)$ varies regularly (it is increasing, but $\frac{1}{t}K(x,y)$ is decreasing), the definition of the $K_{\theta,q}$-norm of a vector $n$, previously given by an integral, is equivalent to a definition given by a series. This series is obtained by breaking $(0,\infty)$ into pieces $(2^n,2^{n+1})$ of equal mass for the measure $\frac{{\rm d}t}{t}$,
\begin{align*}
	\|x\|_{\theta,q;K}\simeq\left(\sum_{n\in\mathbb{Z}} \left(2^{-\theta n}K(x,2^n;X_0,X_1)\right)^q\right)^{\frac{1}{q}}.
\end{align*}
In the special case where $X_0$ is continuously embedded in $X_1$, one can omit the part of the series with negative indies $n$. In this case, each of the functions $x\to K(x,2^n;X_0,X_1)$ defines an equivalent norm on $X_1$.

The interpolation space $(X_0,X_1)_{\theta,q}$ is a ``diagonal subspace'' of an $l^q$-sum of a sequence of Banach spaces (each one being isomorphic to $X_0 + X_1$). Therefore, when $q$ is finite, the dual of $(X_0,X_1)_{\theta,q}$ is a \href{https://en.wikipedia.org/wiki/Banach_space#General_theory}{quotient} of the $l^p$-sum of the duals, $\frac{1}{p} + \frac{1}{q} = 1$, which leads to the following formula for the discrete $J_{\theta,p}$-norm of a functional $x'$ in the dual of $(X_0,X_1)_{\theta,q}$:
\begin{align*}
	\|x'\|_{\theta,p;J}\simeq\inf\left\{\left(\sum_{n\in\mathbb{Z}} \left(2^{\theta n}\max\left(\|x_n'\|_{X_0'},2^{-n}\|x_n'\|_{X_1'}\right)\right)^p\right)^{\frac{1}{p}}:x' = \sum_{n\in\mathbb{Z}} x_n'\right\}.
\end{align*}
The usual formula for the discrete $J_{\theta,p}$-norm is obtained by changing $n$ to $-n$.

The discrete definition makes several questions easier to study, among which the already mentioned identification of the dual. Other such questions are compactness or weak-compactness of linear operators. Lions \& Peetre have proved that:

\begin{theorem}
	If the linear operator $T$ is \href{https://en.wikipedia.org/wiki/Compact_operator}{compact} from $X_0$ to a Banach space $Y$ \& bounded from $X_1$ to $Y$, then $T$ is compact from $(X_0,X_1)_{\theta,q}$ to $Y$ when $0 < \theta < 1$, $1\le q\le\infty$.
\end{theorem}
Davis, Figiel, Johnson, \& Pe\l czy\'nski have used interpolation in their proof of the following result:

\begin{theorem}
	A bounded linear operator between 2 Banach spaces is \href{https://en.wikipedia.org/wiki/Weak_topology}{weakly compact} iff it factors through a \href{https://en.wikipedia.org/wiki/Reflexive_space}{reflexive space}.
\end{theorem}
\noindent'' -- \href{https://en.wikipedia.org/wiki/Interpolation_space#Discrete_definitions}{Wikipedia\texttt{/}interpolation space\texttt{/}discrete definitions}

\subsubsection{A general interpolation method}
``The space $l^q$ used for the discrete definition can be replaced by an arbitrary \href{https://en.wikipedia.org/wiki/Sequence_space}{sequence space} $Y$ with \href{https://en.wikipedia.org/wiki/Schauder_basis#Unconditionality}{unconditional basis}, \& the weights $a_n = 2^{-\theta n}$, $b_n = 2^{(1 - \theta)n}$, that are used for the $K_{\theta,q}$-norm, can be replaced by general weights $a_n,b_n > 0$, $\sum_{n=1}^\infty \min(a_n,b_n) < \infty$. The interpolation space $K(X_0,X_1,Y,\{a_n\},\{b_n\})$ consists of the vectors $x$ in $X_0 + X_1$ s.t.
\begin{align*}
	\|x\|_{K(X_0,X_1)} = \sup_{m\ge 1} \left\|\sum_{n=1}^m a_nK\left(x,\frac{b_n}{a_n};X_0,X_1\right)y_n\right\|_Y < \infty,
\end{align*}
where $\{y_n\}$ is the unconditional basis of $Y$. This abstract method can be used, e.g., for the proof of the following result:

\begin{theorem}
	A Banach space with unconditional basis is isomorphic to a complemented subspace of a space with \href{https://en.wikipedia.org/wiki/Schauder_basis#Unconditionality}{symmetric basis}.
\end{theorem}
\noindent'' -- \href{https://en.wikipedia.org/wiki/Interpolation_space#A_general_interpolation_method}{Wikipedia\texttt{/}interpolation space\texttt{/}discrete definitions\texttt{/}a general interpolation method}

\subsection{Interpolation of Sobolev \& Besov spaces}
Several interpolation results are available for \href{https://en.wikipedia.org/wiki/Sobolev_space}{Sobolev spaces} \& \href{https://en.wikipedia.org/wiki/Besov_space}{Besov spaces} on $\mathbb{R}^n$, $H_p^s$, $s\in\mathbb{R}$, $1\le p\le\infty$, $B_{p,q}^s$, $s\in\mathbb{R}$, $1\le p,q\le\infty$. These spaces are spaces of \href{https://en.wikipedia.org/wiki/Measurable_function}{measurable functions} on $\mathbb{R}^n$ when $s\ge 0$, \& of \href{https://en.wikipedia.org/wiki/Distribution_(mathematics)}{tempered distributions} on $\mathbb{R}^n$ when $s < 0$. For the rest of the section, the following setting \& notation will be used: $0 < \theta < 1$, $1\le p,p_0,p_1,q,q_0,q_1\le\infty$, $s,s_0,s_1\in\mathbb{R}$, $s_\theta = (1 - \theta)s_0 + \theta s_1$, $\frac{1}{p_\theta} = \frac{1 - \theta}{p_0} + \frac{\theta}{p_1}$, $\frac{1}{q_\theta} = \frac{1 - \theta}{q_0} + \frac{\theta}{q_1}$.

Complex interpolation works well on the class of Sobolev spaces $H_p^s$ (the \href{https://en.wikipedia.org/wiki/Sobolev_space#Bessel_potential_spaces}{Bessel potential spaces}) as well as Besov spaces: $(H_{p_0}^{s_0},H_{p_1}^{s_1})_\theta = H_{p_\theta}^{s_\theta}$, $s_0\ne s_1$, $1 < p_0,p_1 < \infty$. $(B_{p_0,q_0}^{s_0},B_{p_1,q_1}^{s_1})_\theta = B_{p_\theta,q_\theta}^{s_\theta}$, $s_0\ne s_1$.

Real interpolation between Sobolev spaces may give Besov spaces, except when $s_0 = s_1$, $(H_{p_0}^s,H_{p_1}^s)_{\theta,p_\theta} = H_{p_\theta}^s$. When $s_0\ne s_1$ but $p_0 = p_1$, real interpolation between Sobolev spaces gives a Besov space: $(H_p^{s_0},H_p^{s_1})_{\theta,q} = B_{p,q}^{s_\theta}$, $s_0\ne s_1$. Also, $(B_{p,q_0}^{s_0},B_{p,q_1}^{s_1})_{\theta,q} = B_{p,q}^{s_\theta}$, $s_0\ne s_1$. $(B_{p,q_0}^s,B_{p,q_1}^s)_{\theta,q} = B_{p,q_\theta}^s$. $(B_{p_0,q_0}^{s_0},B_{p_1,q_1}^{s_1})_{\theta,q_\theta} = B_{p_\theta,q_\theta}^{s_\theta}$, $s_0\ne s_1$, $p_\theta = q_\theta$.'' -- \href{https://en.wikipedia.org/wiki/Interpolation_space#Interpolation_of_Sobolev_and_Besov_spaces}{Wikipedia\texttt{/}interpolation space\texttt{/}interpolation of Sobolev \& Besov spaces}

%------------------------------------------------------------------------------%

\chapter{Terence Tao's}

\section{\cite{Tao2007}. What Is Good Mathematics?}

\textbf{Abstract.} ``Some personal thoughts \& opinions on what ``good quality mathematics'' is \& whether one should try to define this term rigorously. As a case study, the story of Szemer\'edi's theorem is presented.''

\subsection{The Many Aspects of Mathematical Quality}
``We all agree that mathematicians should strive\footnote{\textbf{strive} [v] [intransitive] to try very hard to achieve something.} to produce good mathematics. \textit{But how does one define ``good mathematics'', \& should one even dare to try at all?} Let us 1st consider the former question. Almost immediately one realizes that there are many different types of mathematics which could be designated\footnote{\textbf{designate} [v] [often passive] \textbf{1.} to say officially that somebody\texttt{/}something has a particular character, name or purpose; to describe somebody\texttt{/}something in a particular way; \textbf{2.} to choose or name somebody\texttt{/}something for a particular job or position; \textbf{3.} (of a symbol) to identify or show something.} ``good''. E.g., ``good mathematics'' could refer (in no particular\footnote{\textbf{particular} [a] [only before noun] \textbf{1.} used to emphasize that you are referring to 1 individual person, thing or type of thing \& not others, \textsc{synonym}: \textbf{specific}; \textbf{2.} greater than usual; special; \textbf{in particular} [idiom] \textbf{1.} especially or particularly; \textbf{2.} special, \textsc{synonym}: \textbf{specific}; \textbf{of particular note} [idiom] especially interesting; [n] \textbf{1.} [countable, usually plural] a fact or detail, especially one that is officially written down; \textbf{2.} (\textbf{particulars}) [plural] written information \& details about a property, business, job, etc.} order) to
\begin{enumerate}
	\item Good mathematical \textit{problem solving} (e.g. a major\footnote{\textbf{major} [a] \textbf{1.} [usually before noun] large, important or serious, \textsc{opposite}: \textbf{minor}; \textbf{2.} [only before noun] greater or more important; main, \textsc{synonym}: \textbf{main}; [n] (\textit{North American English}) \textbf{1.} the main subject or course of a student at college or university; \textbf{2.} a student studying a particular subject as the main part of their course.} breakthrough\footnote{\textbf{breakthrough} [n] an important development or discovery that helps people to achieve or understand something.} on an important mathematical problem);
	\item Good mathematical \textit{technique}\footnote{\textbf{technique} [n] \textbf{1.} [countable] a particular way of doing something that involves using a special skill or process; \textbf{2.} [uncountable, singular] a person's skill or ability in a particular activity.} (e.g. a masterful\footnote{\textbf{masterful} [a] \textbf{1.} (of a person, especially a man) able to control people or situations in a way that shows confidence as a leader; \textbf{2.} (also \textbf{masterly}) showing great skill or understanding.} use of existing\footnote{\textbf{existing} [a] [only before noun] found or used now or at the time being discussed.} methods\footnote{\textbf{method} [n] a particular way of doing something.} or the development\footnote{\textbf{development} [n] \textbf{1.} [uncountable] the process of creating a new method, system, product or theory; \textbf{2.} [countable] a new or advanced method, system, product or theory; \textbf{3.} [uncountable] the process of making a country or area richer \& more successful; \textbf{4.} [uncountable] the way in which a child or other living creature grows before \& after birth.} of new tools\footnote{\textbf{tool} [n] \textbf{1.} a thing that helps somebody to do a job or to achieve something; \textbf{2.} a piece of equipment held in the hand, that is used for making things or repairing things.});
	\item Good mathematical \textit{theory} (e.g. a conceptual\footnote{\textbf{conceptual} [a] connected with or based on ideas.} framework\footnote{\textbf{framework} [n] \textbf{1.} a set of beliefs, ideas or principles that is based as the basis for examining or understanding something; \textbf{2.} a system of rules, laws or agreements that controls the way that something works in business, politics or society.} or choice of notation\footnote{\textbf{notation} [n] [uncountable, countable] \textbf{notation (for something)} a system of signs or symbols used to represent information, especially in mathematics, science \& music.} which systematically\footnote{\textbf{systematically} [adv] \textbf{1.} in a way that follows a system; \textbf{2.} in the same way all through a process or set of results because of the system that is used.} unifies\footnote{\textbf{unify} [v] \textbf{1.} \textbf{unify something} to join people or countries together so that they form a single unit; \textbf{2.} \textbf{unify something (into something)} to put things, especially ideas, together in a good or helpful way.} \& generalizes\footnote{\textbf{generalize} [v] (\textit{British English also} \textbf{generalise}) \textbf{1.} [intransitive] \textbf{generalize (from something)} to use a particular set of facts or ideas in order to form an opinion that is considered valid for a different situation; \textbf{2.} [intransitive] to make a general statement about something \& not look at the details; \textbf{3.} [transitive, often passive] to apply a theory, idea, etc. to a wider group or situation than the original one.} an existing\footnote{\textbf{existing} [a] [only before noun] found or used now or at the time being discussed.} body of results);
	\item Good mathematical \textit{insight}\footnote{\textbf{insight} [n] \textbf{1.} [countable, uncountable] an understanding of a particular situation or thing; \textbf{2.} [uncountable] the ability to see \& understand the truth about people or situations.} (e.g. a major conceptual simplification\footnote{\textbf{simplification} [n] \textbf{1.} [uncountable] \textbf{simplification (of something)} the process of making something less complicated, or easier to do or understand; \textbf{2.} [countable] a change that makes a problem, statement, system, etc. less complicated or easier to understand or do.} or the realization\footnote{\textbf{realization} [n] (\textit{British English also} \textbf{realisation}) \textbf{1.} [uncountable, singular] \textbf{realization (that) $\ldots$} the process of becoming aware of something, \textsc{synonym}: \textbf{awareness}; \textbf{2.} [uncountable] \textbf{realization (of something)} the process of achieving a particular aim, etc., \textsc{synonym}: \textbf{achievement}; \textbf{3.} [uncountable, countable] \textbf{realization (of something)} (\textit{formal}) the act of producing something in an actual or physical form; the thing that is produced.} of a unifying\footnote{\textbf{unify} [v] \textbf{1.} \textbf{unify something} to join people or countries together so that they form a single unit; \textbf{2.} \textbf{unify something (into something)} to put things, especially ideas, together in a good or helpful way.} principle\footnote{\textbf{principle} [n] \textbf{1.} [countable] a law, rule or theory that something is based on; \textbf{2.} [singular] a general or scientific law that explains how something works or why something happens; \textbf{3.} [countable] a belief that is accepted as a reason for acting or thinking in a particular way; \textbf{4.} [countable, usually plural, uncountable] a moral rule or a strong belief that influences your actions; \textbf{in principle} [idiom] \textbf{1.} if something can be done in principle, there is no good reason why it should not be done although it has not yet been done \& there may be some difficulties; \textbf{2.} in general but not in detail.}, analogy\footnote{\textbf{analogy} [n] (plural \textbf{analogies}) [countable, uncountable] a comparison of 1 thing with another thing that has similar features, usually in order to explain it; a feature that is similar.}, or theme\footnote{\textbf{theme} [n] the subject of a talk, piece of writing, exhibition, etc.; an idea that keeps returning in a piece of research or a work of art or literature.});
	\item Good mathematical \textit{discovery}\footnote{\textbf{discovery} [n] (plural \textbf{discoveries}) \textbf{1.} [countable, uncountable] an act or the process of finding somebody\texttt{/}something, or learning about something that was not known about before; \textbf{2.} [countable] a thing, fact or person that is found or learned about for the 1st time.} (e.g. the revelation\footnote{\textbf{revelation} [n] \textbf{1.} [countable] a fact that people are made aware of, especially one that has been secret \& is surprising, \textsc{synonym}: \textbf{disclosure}; \textbf{2.} [uncountable] \textbf{revelation (of something)} the act of making people aware of something that has been secret, \textsc{synonym}: \textbf{disclosure}; \textbf{3.} [countable, uncountable] something that is considered to be a sign or message from God.} of an unexpected\footnote{\textbf{unexpected} [a] surprising; not expected.} \& intriguing\footnote{\textbf{intriguing} [a] very interesting because of being unusual or not having an obvious answer.} new mathematical phenomenon\footnote{\textbf{phenomenon} [n] (plural \textbf{phenomena}) a fact or an event in nature or society, especially one that is not fully understood.}, connection\footnote{\textbf{connection} [n] (\textit{British English also, old-fashioned} \textbf{connexion}) \textbf{1.} [countable] something that connects 2 facts or ideas, \textsc{synonym}: \textbf{link}; \textbf{2.} [countable] a relationship between people or groups of people, often for a particular purpose; \textbf{3.} [uncountable, countable] the action of connecting something to a supply of water, electricity, etc. or to a computer or telephone network; the fact of being connected in this way; \textbf{4.} [countable] a point, especially in an electrical system, where 2 parts connect; \textbf{5.} [countable, usually plural] a means of traveling to another place; \textbf{6.} [countable, usually plural] people that you know, who can help or advise you in your professional or social life; \textbf{in connection with somebody\texttt{/}something} [idiom] for reasons connected with somebody\texttt{/}something; \textbf{in this\texttt{/}that connection} [idiom] for reasons connected with something recently mentioned.}, or counterexample\footnote{\textbf{counterexample} [n] \textbf{counterexample (to something)} an example that provides evidence against an idea or theory.});
	\item Good mathematical \textit{application}\footnote{\textbf{application} [n] \textbf{1.} [uncountable, countable] the use of something such as an idea, method, rule, etc.; a use that something has; \textbf{2.} [countable] a formal (often written) request to an organization or authority for something, such as a job or permission to do something, or to join a group; \textbf{3.} [countable] a program or piece of software designed to do a particular job; \textbf{4.} [countable, uncountable] \textbf{application (of something) (to something)} the use of something to produce a particular physical effect; \textbf{5.} [countable, uncountable] \textbf{application (of something)} the action of putting or spreading something onto a surface or object.} (e.g. to important problems in physics, engineering, computer science, statistics, etc., or from 1 field of mathematics to another);
	\item Good mathematical \textit{exposition}\footnote{\textbf{exposition} [n] [countable, uncountable] (\textit{formal}) a full explanation of a theory, plan, etc.} (e.g. a detailed\footnote{\textbf{detailed} [a] giving many details; paying great attention to details.} \& informative\footnote{\textbf{informative} [a] giving useful information.} survey\footnote{\textbf{survey} [n] \textbf{1.} \textbf{survey (of somebody\texttt{/}something)} an investigation of the opinions, behavior, etc. of a particular group of people, which is usually done by asking them questions; \textbf{2.} an act of examining \& recording the measurements, features, etc. of an area of land in order to make a map or plan of it; \textbf{3.} \textbf{survey (of something)} a general study, view or description of something; [v] \textbf{1.} \textbf{survey somebody\texttt{/}something} to investigate the opinions or behavior of a group of people by asking them a series of questions; \textbf{2.} \textbf{survey something} to study \& give a general description of something; \textbf{3.} \textbf{survey something} to measure \& record the features of an area of land, e.g. in order to make a map or in preparation for building; \textbf{4.} \textbf{survey something} to look carefully at the whole of something, especially in order to get a general impression of it, \textsc{synonym}: \textbf{inspect}.} on a timely\footnote{\textbf{timely} [a] happening at exactly the right time.} mathematical topic or a clear \& well-motivated argument);
	\item Good mathematical \textit{pedagogy}\footnote{\textbf{pedagogy} [n] (plural \textbf{pedagogies}) [uncountable, countable] methods of teaching, especially as a subject of study or as a theory.} (e.g. a lecture\footnote{\textbf{lecture} [n] a talk that is given to a group of people to teach them about a particular subject, often as part of a university or college course; [v] [intransitive] \textbf{lecture (in\texttt{/}on something) (to somebody)} to give a talk or a series of talks to a group of people on a particular subject, especially as a way of teaching in a university or college.} or writing style which enables others to learn \& do mathematics more effectively, or contributions\footnote{\textbf{contribution} [n] \textbf{1.} [usually singular] the part played by a person or thing in achieving, improving or causing something; \textbf{2.} a sum of money that is given to a person or an organization in order to help pay for something, \textsc{synonym}: \textbf{donation}; \textbf{3.} \textbf{contribution (to something)} an item that forms part of a book, magazine, broadcast, discussion, etc.; \textbf{4.} a sum of money that you pay regularly to your employer or the government in order to pay for benefits such as health insurance or a pension.} to mathematical education);
	\item Good mathematical \textit{vision}\footnote{\textbf{vision} [n] \textbf{1.} [uncountable] the ability to see; the area that you can see from a particular position; \textbf{2.} [countable] an idea or a picture in your imagination, especially of what the future will or could be like; \textbf{3.} [uncountable] the ability to think about or plan the future with great imagination \& intelligence.} (e.g. a long-range\footnote{\textbf{long-range} [a] [only before noun] \textbf{1.} traveling a long distance; \textbf{2.} made for a period of time that will last a long way into the future.} \& fruitful program or set of conjectures\footnote{\textbf{conjecture} [n] (\textit{formal}) \textbf{1.} [countable] an opinion or idea that is not based on definite knowledge \& is formed by guessing, \textsc{synonym}: \textbf{guess}; \textbf{2.} [uncountable] the act of forming an opinion or idea that is not based on definite knowledge; [v] [intransitive, transitive] (\textit{formal}) to form an opinion about something even though you do not have much information on it, \textsc{synonym}: \textbf{guess}.});
	\item Good mathematical \textit{taste} (e.g. a research goal which is inherently interesting \& impacts important topics, themes, or questions);
	\item Good mathematical \textit{public relations} (e.g. an effective showcasing of a mathematical achievement to non-mathematicians or from 1 field of mathematics to another);
\end{enumerate}  

%------------------------------------------------------------------------------%

\printbibliography[heading=bibintoc]
	
\end{document}