\documentclass[oneside]{book}
\usepackage[backend=biber,natbib=true,style=authoryear]{biblatex}
\addbibresource{/home/hong/1_NQBH/reference/bib.bib}
\usepackage[vietnamese,english]{babel}
\usepackage{tocloft}
\renewcommand{\cftsecleader}{\cftdotfill{\cftdotsep}}
\usepackage[colorlinks=true,linkcolor=blue,urlcolor=red,citecolor=magenta]{hyperref}
\usepackage{amsmath,amssymb,amsthm,mathtools,float,graphicx}
\allowdisplaybreaks
\numberwithin{equation}{section}
\newtheorem{assumption}{Assumption}[chapter]
\newtheorem{conjecture}{Conjecture}[chapter]
\newtheorem{corollary}{Corollary}[chapter]
\newtheorem{definition}{Definition}[chapter]
\newtheorem{example}{Example}[chapter]
\newtheorem{lemma}{Lemma}[chapter]
\newtheorem{notation}{Notation}[chapter]
\newtheorem{principle}{Principle}[chapter]
\newtheorem{problem}{Problem}[chapter]
\newtheorem{proposition}{Proposition}[chapter]
\newtheorem{question}{Question}[chapter]
\newtheorem{remark}{Remark}[chapter]
\newtheorem{theorem}{Theorem}[chapter]
\usepackage[left=0.5in,right=0.5in,top=1.5cm,bottom=1.5cm]{geometry}
\usepackage{fancyhdr}
\pagestyle{fancy}
\fancyhf{}
\lhead{\small \textsc{Sect.} ~\thesection}
\rhead{\small \nouppercase{\leftmark}}
\renewcommand{\sectionmark}[1]{\markboth{#1}{}}
\cfoot{\thepage}
\def\labelitemii{$\circ$}

\title{Advanced Mathematics}
\author{\selectlanguage{vietnamese} Nguyễn Quản Bá Hồng\footnote{Independent Researcher, Ben Tre City, Vietnam\\e-mail: \texttt{nguyenquanbahong@gmail.com}}}
\date{\today}

\begin{document}
\maketitle
\setcounter{secnumdepth}{4}
\setcounter{tocdepth}{4}
\tableofcontents

%------------------------------------------------------------------------------%

\chapter{Wikipedia's}

\section{\href{https://en.wikipedia.org/wiki/Symmetrization_methods}{Wikipedia\texttt{/}Symmetrization Methods}}
``In mathematics the \textit{symmetrization methods} are algorithms of transforming a set $A\subset\mathbb{R}^n$ to a ball $B\subset\mathbb{R}^n$ with equal volume $\operatorname{vol}(B) = \operatorname{vol}(A)$ \& centered at the origin. $B$ is called the \textit{symmetrized version} of $A$, usually denoted $A^\star$. These algorithms show up in solving the classical \href{https://en.wikipedia.org/wiki/Isoperimetric_inequality}{isoperimetric inequality} problem, which asks: Given all 2D shapes of a given area, which of them has the minimal \href{https://en.wikipedia.org/wiki/Perimeter}{perimeter}. The conjectured answer was the disk \& \href{https://en.wikipedia.org/wiki/Jakob_Steiner}{Steiner} in 1838 showed this to be true using the Steiner symmetrization method. From this many other isoperimetric problems sprung \& other symmetrization algorithms. E.g., Rayleigh's conjecture is that the 1st \href{https://en.wikipedia.org/wiki/Eigenvalue}{eigenvalue} of the \href{https://en.wikipedia.org/wiki/Dirichlet_problem}{Dirichlet problem} is minimized for the ball (see \href{https://en.wikipedia.org/wiki/Rayleigh%E2%80%93Faber%E2%80%93Krahn_inequality}{Rayleigh--Faber--Krahn inequality} for details). Another problem is that the Newtonian \href{https://en.wikipedia.org/wiki/Capacity_of_a_set}{capacity of a set} $A$ is minimized by $A^\star$ \& this was proved by Polya \& G. Szego (1951) using circular symmetrization.'' -- \href{https://en.wikipedia.org/wiki/Symmetrization_methods}{Wikipedia\texttt{/}symmetrization methods}

\subsection{Symmetrization}
``If $\Omega\subset\mathbb{R}^n$ is measurable, then it is denoted by $\Omega^\star$ the symmetrized version of $\Omega$, i.e., a ball $\Omega^\star\coloneqq B_r(0)\subset\mathbb{R}^n$ s.t. $\operatorname{vol}(\Omega^\star) = \operatorname{vol}(\Omega)$. We denote by $f^\star$ the \href{https://en.wikipedia.org/wiki/Symmetric_decreasing_rearrangement}{symmetric decreasing rearrangement} of nonnegative measurable function $f$ \& define it as $f^\star(x)\coloneqq\int_0^\infty 1_{\{y:f(y) > t\}^\star}(x){\rm d}t$, where $\{y:f(y) > t\}^\star$ is the symmetrized version of preimage set $\{y:f(y) > t\}$. The methods described below have been proved to transform $\Omega$ to $\Omega^\star$, i.e., given a sequence of symmetrization transformations $\{T_k\}$ there is $\lim_{k\to\infty} d_{\rm Ha}(\Omega^\star,T_k(K)) = 0$, where $d_{\rm Ha}$ is the \href{https://en.wikipedia.org/wiki/Hausdorff_distance}{Hausdorff distance} (for discussion \& proofs see [Burchard2009]).'' -- \href{https://en.wikipedia.org/wiki/Symmetrization_methods#Symmetrization}{Wikipedia\texttt{/}symmetrization methods\texttt{/}symmetrization}

\subsection{Steiner symmetrization}
\textsf{Steiner Symmetrization of set $\Omega$.}

``Steiner symmetrization was introduced by Steiner (1838) to solve the isoperimetric theorem stated above. Let $H\subset\mathbb{R}^n$ be a \href{https://en.wikipedia.org/wiki/Hyperplane}{hyperplane} through the origin. Rotate space so that $H$ is the $x_n = 0$ ($x_n$ is $n$th coordinate in $\mathbb{R}^n$) hyperplane. For each ${\bf x}\in H$ let the perpendicular line through ${\bf x}\in H$ be $L_{\bf x} = \{{\bf x} + y{\bf e}_n:y\in\mathbb{R}\}$. Then by replacing each $\Omega\cap L_{\bf x}$ by a line centered at $H$ \& with length $|\Omega\cap L_{\bf x}|$ we obtain the \textit{Steiner symmetrized version}.
\begin{align*}
	\operatorname{St}(\Omega)\coloneqq\left\{{\bf x} + y{\bf e}_n:{\bf x} + z{\bf e}_n\in\Omega\mbox{ for some }{\bf z}\mbox{ \& }|y|\le\frac{1}{2}|\Omega\cap L_x|\right\}.
\end{align*}
It is denoted by $\operatorname{St}(f)$ the \textit{Steiner symmetrization} w.r.t. $x_n = 0$ hyperplane of nonnegative measurable function $f:\mathbb{R}^d\to\mathbb{R}$ \& for fixed $x_1,\ldots,x_{n-1}$ define it as $\operatorname{St}:f(x_1,\ldots,x_{n-1},\cdot)\mapsto(f(x_1,\ldots,x_{n-1}))^\star$.

\subsubsection{Properties}
It preserves convexity: if $\Omega$ is convex, then $\operatorname{St}(\Omega)$ is also convex. It is linear: $\operatorname{St}({\bf x} + \lambda\Omega) = \operatorname{St}({\bf x}) + \lambda\operatorname{St}(\Omega)$. Super-additive: $\operatorname{St}(K) + \operatorname{St}(U)\subset\operatorname{St}(K + U)$.'' -- \href{https://en.wikipedia.org/wiki/Symmetrization_methods#Steiner_symmetrization}{Wikipedia\texttt{/}symmetrization methods\texttt{/}Steiner symmetrization}

\subsection{Circular symmetrization}
\textsf{Fig. Circular symmetrization of set $\Omega$.}

``A popular method for symmetrization in the plane is \textit{Polya's circular symmetrization}. After, its generalization will be described to higher dimensions. Let $\Omega\subset\mathbb{C}$ be a domain; then its circular symmetrization $\operatorname{Circ}(\Omega)$ with regard to the positive real axis is defined as follows: Let $\Omega_t\coloneqq\{\theta\in[0,2\pi]:te^{i\theta}\in\Omega\}$, i.e., contain the arcs of radius $t$ contained in $\Omega$. So it is defined
\begin{itemize}
	\item If $\Omega_t$ is the full circle, then $\operatorname{Circ}(\Omega)\cap\{|z| = t\}\coloneqq\{|z| = t\}$.
	\item If the length is $\operatorname{m}(\Omega_t) = \alpha$, then $\operatorname{Circ}(\Omega)\cap\{|z| = t\}\coloneqq\left\{te^{i\theta}:|\theta| < \frac{\alpha}{2}\right\}$.
	\item $0,\infty\in\operatorname{Circ}(\Omega)$ if $0,\infty\in\Omega$.
\end{itemize}
In higher dimensions $\Omega\subset\mathbb{R}^n$, its spherical symmetrization $\operatorname{Sp}^n(\Omega)$ w.r.t. the positive axis of $x_1$ is defined as follows: Let $\Omega_r\coloneqq\{{\bf x}\in\mathbb{S}^{n-1}:r{\bf x}\in\Omega\}$, i.e., contain the caps of radius $r$ contained in $\Omega$. Also, for the 1st coordinate let $\operatorname{angle}(x_1)\coloneqq\theta$ if $x_1 = r\cos\theta$. So as above
\begin{itemize}
	\item If $\Omega_r$ is the full cap, then $\operatorname{Sp}^n(\Omega)\cap\{|z| = r\}\coloneqq\{|z| = t\}$.
	\item If the surface area is $\operatorname{m}_{\rm s}(\Omega_t) = \alpha$, then $\operatorname{Sp}^n(\Omega)\cap\{|z| = r\}\coloneqq\{x:|x| = r\mbox{ \& }0\le\operatorname{angle}(x_1)\le\theta_\alpha\}=:C(\theta_\alpha)$ where $\theta_\alpha$ is picked so that its surface area is $\operatorname{m}_s(C(\theta_\alpha)) = \alpha$. In words, $C(\theta_\alpha)$ is a cap symmetric around the positive axis $x_1$ with the same area as the intersection $\Omega\cap\{|z| = r\}$.
	\item $0,\infty\in\operatorname{Sp}^n(\Omega)$ iff $0,\infty\in\Omega$.'' -- \href{https://en.wikipedia.org/wiki/Symmetrization_methods#Circular_symmetrization}{Wikipedia\texttt{/}symmetrization methods\texttt{/}circular symmetrization}
\end{itemize}

\subsection{Polarization}
\textsf{Fig: Polarization of set $\Omega$.}

``Let $\Omega\subset\mathbb{R}^n$ be a domain \& $H^{n-1}\subset\mathbb{R}^n$ be a hyperplane through the origin. Denote the reflection across that plane to the positive halfspace $\mathbb{H}^+$ as $\sigma_H$ or just $\sigma$ when it is clear from the context. Also, the reflected $\Omega$ across hyperplane $H$ is defined as $\sigma\Omega$. Then, the polarized $\Omega$ is denoted as $\Omega^\alpha$ \& defined as follows
\begin{itemize}
	\item If ${\bf x}\in\Omega\cap\mathbb{H}^+$, then ${\bf x}\in\Omega^\alpha$.
	\item If ${\bf x}\in\Omega\cap\sigma(\Omega)\cap\mathbb{H}^-$, then ${\bf x}\in\Omega^\sigma$.
	\item If ${\bf x}\in(\Omega\backslash\sigma(\Omega))\cap\mathbb{H}^-$, then $\sigma{\bf x}\in\Omega^\sigma$.
\end{itemize}
In words, $(\Omega\backslash\sigma(\Omega))\cap\mathbb{H}^-$ is simply reflected to the halfspace $\mathbb{H}^+$. It turns out that this transformation can approximate the above ones (in the \href{https://en.wikipedia.org/wiki/Hausdorff_distance}{Hausdorff distance}) (see [Brock \& Solynin2000]).'' -- \href{https://en.wikipedia.org/wiki/Symmetrization_methods#Polarization}{Wikipedia\texttt{/}symmetrization methods\texttt{/}polarization}

%------------------------------------------------------------------------------%

\chapter{Terence Tao's}

\section{\cite{Tao2007}. What Is Good Mathematics?}

\textbf{Abstract.} ``Some personal thoughts \& opinions on what ``good quality mathematics'' is \& whether one should try to define this term rigorously. As a case study, the story of Szemer\'edi's theorem is presented.''

\subsection{The Many Aspects of Mathematical Quality}
``We all agree that mathematicians should strive\footnote{\textbf{strive} [v] [intransitive] to try very hard to achieve something.} to produce good mathematics. \textit{But how does one define ``good mathematics'', \& should one even dare to try at all?} Let us 1st consider the former question. Almost immediately one realizes that there are many different types of mathematics which could be designated\footnote{\textbf{designate} [v] [often passive] \textbf{1.} to say officially that somebody\texttt{/}something has a particular character, name or purpose; to describe somebody\texttt{/}something in a particular way; \textbf{2.} to choose or name somebody\texttt{/}something for a particular job or position; \textbf{3.} (of a symbol) to identify or show something.} ``good''. E.g., ``good mathematics'' could refer (in no particular\footnote{\textbf{particular} [a] [only before noun] \textbf{1.} used to emphasize that you are referring to 1 individual person, thing or type of thing \& not others, \textsc{synonym}: \textbf{specific}; \textbf{2.} greater than usual; special; \textbf{in particular} [idiom] \textbf{1.} especially or particularly; \textbf{2.} special, \textsc{synonym}: \textbf{specific}; \textbf{of particular note} [idiom] especially interesting; [n] \textbf{1.} [countable, usually plural] a fact or detail, especially one that is officially written down; \textbf{2.} (\textbf{particulars}) [plural] written information \& details about a property, business, job, etc.} order) to
\begin{enumerate}
	\item Good mathematical \textit{problem solving} (e.g. a major\footnote{\textbf{major} [a] \textbf{1.} [usually before noun] large, important or serious, \textsc{opposite}: \textbf{minor}; \textbf{2.} [only before noun] greater or more important; main, \textsc{synonym}: \textbf{main}; [n] (\textit{North American English}) \textbf{1.} the main subject or course of a student at college or university; \textbf{2.} a student studying a particular subject as the main part of their course.} breakthrough\footnote{\textbf{breakthrough} [n] an important development or discovery that helps people to achieve or understand something.} on an important mathematical problem);
	\item Good mathematical \textit{technique}\footnote{\textbf{technique} [n] \textbf{1.} [countable] a particular way of doing something that involves using a special skill or process; \textbf{2.} [uncountable, singular] a person's skill or ability in a particular activity.} (e.g. a masterful\footnote{\textbf{masterful} [a] \textbf{1.} (of a person, especially a man) able to control people or situations in a way that shows confidence as a leader; \textbf{2.} (also \textbf{masterly}) showing great skill or understanding.} use of existing\footnote{\textbf{existing} [a] [only before noun] found or used now or at the time being discussed.} methods\footnote{\textbf{method} [n] a particular way of doing something.} or the development\footnote{\textbf{development} [n] \textbf{1.} [uncountable] the process of creating a new method, system, product or theory; \textbf{2.} [countable] a new or advanced method, system, product or theory; \textbf{3.} [uncountable] the process of making a country or area richer \& more successful; \textbf{4.} [uncountable] the way in which a child or other living creature grows before \& after birth.} of new tools\footnote{\textbf{tool} [n] \textbf{1.} a thing that helps somebody to do a job or to achieve something; \textbf{2.} a piece of equipment held in the hand, that is used for making things or repairing things.});
	\item Good mathematical \textit{theory} (e.g. a conceptual\footnote{\textbf{conceptual} [a] connected with or based on ideas.} framework\footnote{\textbf{framework} [n] \textbf{1.} a set of beliefs, ideas or principles that is based as the basis for examining or understanding something; \textbf{2.} a system of rules, laws or agreements that controls the way that something works in business, politics or society.} or choice of notation\footnote{\textbf{notation} [n] [uncountable, countable] \textbf{notation (for something)} a system of signs or symbols used to represent information, especially in mathematics, science \& music.} which systematically\footnote{\textbf{systematically} [adv] \textbf{1.} in a way that follows a system; \textbf{2.} in the same way all through a process or set of results because of the system that is used.} unifies\footnote{\textbf{unify} [v] \textbf{1.} \textbf{unify something} to join people or countries together so that they form a single unit; \textbf{2.} \textbf{unify something (into something)} to put things, especially ideas, together in a good or helpful way.} \& generalizes\footnote{\textbf{generalize} [v] (\textit{British English also} \textbf{generalise}) \textbf{1.} [intransitive] \textbf{generalize (from something)} to use a particular set of facts or ideas in order to form an opinion that is considered valid for a different situation; \textbf{2.} [intransitive] to make a general statement about something \& not look at the details; \textbf{3.} [transitive, often passive] to apply a theory, idea, etc. to a wider group or situation than the original one.} an existing\footnote{\textbf{existing} [a] [only before noun] found or used now or at the time being discussed.} body of results);
	\item Good mathematical \textit{insight}\footnote{\textbf{insight} [n] \textbf{1.} [countable, uncountable] an understanding of a particular situation or thing; \textbf{2.} [uncountable] the ability to see \& understand the truth about people or situations.} (e.g. a major conceptual simplification\footnote{\textbf{simplification} [n] \textbf{1.} [uncountable] \textbf{simplification (of something)} the process of making something less complicated, or easier to do or understand; \textbf{2.} [countable] a change that makes a problem, statement, system, etc. less complicated or easier to understand or do.} or the realization\footnote{\textbf{realization} [n] (\textit{British English also} \textbf{realisation}) \textbf{1.} [uncountable, singular] \textbf{realization (that) $\ldots$} the process of becoming aware of something, \textsc{synonym}: \textbf{awareness}; \textbf{2.} [uncountable] \textbf{realization (of something)} the process of achieving a particular aim, etc., \textsc{synonym}: \textbf{achievement}; \textbf{3.} [uncountable, countable] \textbf{realization (of something)} (\textit{formal}) the act of producing something in an actual or physical form; the thing that is produced.} of a unifying\footnote{\textbf{unify} [v] \textbf{1.} \textbf{unify something} to join people or countries together so that they form a single unit; \textbf{2.} \textbf{unify something (into something)} to put things, especially ideas, together in a good or helpful way.} principle\footnote{\textbf{principle} [n] \textbf{1.} [countable] a law, rule or theory that something is based on; \textbf{2.} [singular] a general or scientific law that explains how something works or why something happens; \textbf{3.} [countable] a belief that is accepted as a reason for acting or thinking in a particular way; \textbf{4.} [countable, usually plural, uncountable] a moral rule or a strong belief that influences your actions; \textbf{in principle} [idiom] \textbf{1.} if something can be done in principle, there is no good reason why it should not be done although it has not yet been done \& there may be some difficulties; \textbf{2.} in general but not in detail.}, analogy\footnote{\textbf{analogy} [n] (plural \textbf{analogies}) [countable, uncountable] a comparison of 1 thing with another thing that has similar features, usually in order to explain it; a feature that is similar.}, or theme\footnote{\textbf{theme} [n] the subject of a talk, piece of writing, exhibition, etc.; an idea that keeps returning in a piece of research or a work of art or literature.});
	\item Good mathematical \textit{discovery}\footnote{\textbf{discovery} [n] (plural \textbf{discoveries}) \textbf{1.} [countable, uncountable] an act or the process of finding somebody\texttt{/}something, or learning about something that was not known about before; \textbf{2.} [countable] a thing, fact or person that is found or learned about for the 1st time.} (e.g. the revelation\footnote{\textbf{revelation} [n] \textbf{1.} [countable] a fact that people are made aware of, especially one that has been secret \& is surprising, \textsc{synonym}: \textbf{disclosure}; \textbf{2.} [uncountable] \textbf{revelation (of something)} the act of making people aware of something that has been secret, \textsc{synonym}: \textbf{disclosure}; \textbf{3.} [countable, uncountable] something that is considered to be a sign or message from God.} of an unexpected\footnote{\textbf{unexpected} [a] surprising; not expected.} \& intriguing\footnote{\textbf{intriguing} [a] very interesting because of being unusual or not having an obvious answer.} new mathematical phenomenon\footnote{\textbf{phenomenon} [n] (plural \textbf{phenomena}) a fact or an event in nature or society, especially one that is not fully understood.}, connection\footnote{\textbf{connection} [n] (\textit{British English also, old-fashioned} \textbf{connexion}) \textbf{1.} [countable] something that connects 2 facts or ideas, \textsc{synonym}: \textbf{link}; \textbf{2.} [countable] a relationship between people or groups of people, often for a particular purpose; \textbf{3.} [uncountable, countable] the action of connecting something to a supply of water, electricity, etc. or to a computer or telephone network; the fact of being connected in this way; \textbf{4.} [countable] a point, especially in an electrical system, where 2 parts connect; \textbf{5.} [countable, usually plural] a means of traveling to another place; \textbf{6.} [countable, usually plural] people that you know, who can help or advise you in your professional or social life; \textbf{in connection with somebody\texttt{/}something} [idiom] for reasons connected with somebody\texttt{/}something; \textbf{in this\texttt{/}that connection} [idiom] for reasons connected with something recently mentioned.}, or counterexample\footnote{\textbf{counterexample} [n] \textbf{counterexample (to something)} an example that provides evidence against an idea or theory.});
	\item Good mathematical \textit{application}\footnote{\textbf{application} [n] \textbf{1.} [uncountable, countable] the use of something such as an idea, method, rule, etc.; a use that something has; \textbf{2.} [countable] a formal (often written) request to an organization or authority for something, such as a job or permission to do something, or to join a group; \textbf{3.} [countable] a program or piece of software designed to do a particular job; \textbf{4.} [countable, uncountable] \textbf{application (of something) (to something)} the use of something to produce a particular physical effect; \textbf{5.} [countable, uncountable] \textbf{application (of something)} the action of putting or spreading something onto a surface or object.} (e.g. to important problems in physics, engineering, computer science, statistics, etc., or from 1 field of mathematics to another);
	\item Good mathematical \textit{exposition}\footnote{\textbf{exposition} [n] [countable, uncountable] (\textit{formal}) a full explanation of a theory, plan, etc.} (e.g. a detailed\footnote{\textbf{detailed} [a] giving many details; paying great attention to details.} \& informative\footnote{\textbf{informative} [a] giving useful information.} survey\footnote{\textbf{survey} [n] \textbf{1.} \textbf{survey (of somebody\texttt{/}something)} an investigation of the opinions, behavior, etc. of a particular group of people, which is usually done by asking them questions; \textbf{2.} an act of examining \& recording the measurements, features, etc. of an area of land in order to make a map or plan of it; \textbf{3.} \textbf{survey (of something)} a general study, view or description of something; [v] \textbf{1.} \textbf{survey somebody\texttt{/}something} to investigate the opinions or behavior of a group of people by asking them a series of questions; \textbf{2.} \textbf{survey something} to study \& give a general description of something; \textbf{3.} \textbf{survey something} to measure \& record the features of an area of land, e.g. in order to make a map or in preparation for building; \textbf{4.} \textbf{survey something} to look carefully at the whole of something, especially in order to get a general impression of it, \textsc{synonym}: \textbf{inspect}.} on a timely\footnote{\textbf{timely} [a] happening at exactly the right time.} mathematical topic or a clear \& well-motivated argument);
	\item Good mathematical \textit{pedagogy}\footnote{\textbf{pedagogy} [n] (plural \textbf{pedagogies}) [uncountable, countable] methods of teaching, especially as a subject of study or as a theory.} (e.g. a lecture\footnote{\textbf{lecture} [n] a talk that is given to a group of people to teach them about a particular subject, often as part of a university or college course; [v] [intransitive] \textbf{lecture (in\texttt{/}on something) (to somebody)} to give a talk or a series of talks to a group of people on a particular subject, especially as a way of teaching in a university or college.} or writing style which enables others to learn \& do mathematics more effectively, or contributions\footnote{\textbf{contribution} [n] \textbf{1.} [usually singular] the part played by a person or thing in achieving, improving or causing something; \textbf{2.} a sum of money that is given to a person or an organization in order to help pay for something, \textsc{synonym}: \textbf{donation}; \textbf{3.} \textbf{contribution (to something)} an item that forms part of a book, magazine, broadcast, discussion, etc.; \textbf{4.} a sum of money that you pay regularly to your employer or the government in order to pay for benefits such as health insurance or a pension.} to mathematical education);
	\item Good mathematical \textit{vision}\footnote{\textbf{vision} [n] \textbf{1.} [uncountable] the ability to see; the area that you can see from a particular position; \textbf{2.} [countable] an idea or a picture in your imagination, especially of what the future will or could be like; \textbf{3.} [uncountable] the ability to think about or plan the future with great imagination \& intelligence.} (e.g. a long-range\footnote{\textbf{long-range} [a] [only before noun] \textbf{1.} traveling a long distance; \textbf{2.} made for a period of time that will last a long way into the future.} \& fruitful program or set of conjectures\footnote{\textbf{conjecture} [n] (\textit{formal}) \textbf{1.} [countable] an opinion or idea that is not based on definite knowledge \& is formed by guessing, \textsc{synonym}: \textbf{guess}; \textbf{2.} [uncountable] the act of forming an opinion or idea that is not based on definite knowledge; [v] [intransitive, transitive] (\textit{formal}) to form an opinion about something even though you do not have much information on it, \textsc{synonym}: \textbf{guess}.});
	\item Good mathematical \textit{taste} (e.g. a research goal which is inherently interesting \& impacts important topics, themes, or questions);
	\item Good mathematical \textit{public relations} (e.g. an effective showcasing of a mathematical achievement to non-mathematicians or from 1 field of mathematics to another);
\end{enumerate}  


%------------------------------------------------------------------------------%

%\selectlanguage{english}
%\begin{thebibliography}{99}
%	\bibitem[]{}
%\end{thebibliography}

%------------------------------------------------------------------------------%

\printbibliography[heading=bibintoc]
	
\end{document}