\documentclass[oneside]{book}
\usepackage[backend=biber,natbib=true,style=authoryear]{biblatex}
\addbibresource{/home/hong/1_NQBH/reference/bib.bib}
\usepackage[vietnamese,english]{babel}
\usepackage{tocloft}
\renewcommand{\cftsecleader}{\cftdotfill{\cftdotsep}}
\usepackage[colorlinks=true,linkcolor=blue,urlcolor=red,citecolor=magenta]{hyperref}
\usepackage{amsmath,amssymb,amsthm,mathtools,float,graphicx}
\allowdisplaybreaks
\numberwithin{equation}{section}
\newtheorem{assumption}{Assumption}[chapter]
\newtheorem{conjecture}{Conjecture}[chapter]
\newtheorem{corollary}{Corollary}[chapter]
\newtheorem{definition}{Definition}[chapter]
\newtheorem{example}{Example}[chapter]
\newtheorem{lemma}{Lemma}[chapter]
\newtheorem{notation}{Notation}[chapter]
\newtheorem{principle}{Principle}[chapter]
\newtheorem{problem}{Problem}[chapter]
\newtheorem{proposition}{Proposition}[chapter]
\newtheorem{question}{Question}[chapter]
\newtheorem{remark}{Remark}[chapter]
\newtheorem{theorem}{Theorem}[chapter]
\usepackage[left=0.5in,right=0.5in,top=1.5cm,bottom=1.5cm]{geometry}
\usepackage{fancyhdr}
\pagestyle{fancy}
\fancyhf{}
\lhead{\small \textsc{Sect.} ~\thesection}
\rhead{\small \nouppercase{\leftmark}}
\renewcommand{\sectionmark}[1]{\markboth{#1}{}}
\cfoot{\thepage}
\def\labelitemii{$\circ$}

\title{Graph Theory}
\author{\selectlanguage{vietnamese} Nguyễn Quản Bá Hồng\footnote{Independent Researcher, Ben Tre City, Vietnam\\e-mail: \texttt{nguyenquanbahong@gmail.com}}}
\date{\today}

\begin{document}
\maketitle
\setcounter{secnumdepth}{4}
\setcounter{tocdepth}{4}
\tableofcontents

%------------------------------------------------------------------------------%

\chapter{Wikipedia's}

\section{\href{https://en.wikipedia.org/wiki/Graph_theory}{Wikipedia\texttt{/}Graph Theory}}
\textsf{Fig. A \href{https://en.wikipedia.org/wiki/Graph_drawing}{drawing} of a graph.}

``In mathematics, \textit{graph theory} is the study of \href{https://en.wikipedia.org/wiki/Graph_(discrete_mathematics)}{graphs}, which are mathematical structures used to model pairwise relations between objects. A graph in this context is made up of \href{https://en.wikipedia.org/wiki/Vertex_(graph_theory)}{vertices} (also called \textit{nodes} or \textit{points}) which are connected by \href{https://en.wikipedia.org/wiki/Glossary_of_graph_theory_terms#edge}{edges} (also called \textit{links} or \textit{lines}). A distinction is made between \textit{undirected graphs}, where edges link 2 vertices symmetrically, \& \textit{directed graphs}, where edges link 2 vertices asymmetrically. Graphs are 1 of the principal objects of study in \href{https://en.wikipedia.org/wiki/Discrete_mathematics}{discrete mathematics}.

\subsection{Definitions}

\subsubsection{Graph}

\subsubsection{Directed graph}

\subsection{Applications}

\subsubsection{Computer science}

\subsubsection{Linguistics}

\subsubsection{Physics \& chemistry}

\subsubsection{Social sciences}

\subsubsection{Biology}

\subsubsection{Mathematics}

\subsubsection{Other topics}

\subsection{History}

\subsection{Representation}

\subsubsection{Visual: Graph drawing}

\subsubsection{Tabular: Graph data structures}

\subsection{Problems}

\subsubsection{Enumeration}

\subsubsection{Subgraphs, induced subgraphs, \& minors}

\subsubsection{Graph coloring}

\subsubsection{Subsumption \& unification}

\subsubsection{Route problems}

\subsubsection{Network flow}

\subsubsection{Visibility problems}

\subsubsection{Covering problems}

\subsubsection{Decomposition problems}

\subsubsection{Graph classes}

%------------------------------------------------------------------------------%

%\selectlanguage{english}
%\begin{thebibliography}{99}
%	\bibitem[]{}
%\end{thebibliography}

%------------------------------------------------------------------------------%

\printbibliography[heading=bibintoc]
	
\end{document}