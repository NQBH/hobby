\documentclass[oneside]{book}
\usepackage[backend=biber,natbib=true,style=authoryear]{biblatex}
\addbibresource{/home/hong/1_NQBH/reference/bib.bib}
\usepackage[vietnamese,english]{babel}
\usepackage{tocloft}
\renewcommand{\cftsecleader}{\cftdotfill{\cftdotsep}}
\usepackage[colorlinks=true,linkcolor=blue,urlcolor=red,citecolor=magenta]{hyperref}
\usepackage{amsmath,amssymb,amsthm,mathtools,float,graphicx}
\allowdisplaybreaks
\numberwithin{equation}{section}
\newtheorem{assumption}{Assumption}[chapter]
\newtheorem{conjecture}{Conjecture}[chapter]
\newtheorem{corollary}{Corollary}[chapter]
\newtheorem{definition}{Definition}[chapter]
\newtheorem{example}{Example}[chapter]
\newtheorem{lemma}{Lemma}[chapter]
\newtheorem{notation}{Notation}[chapter]
\newtheorem{principle}{Principle}[chapter]
\newtheorem{problem}{Problem}[chapter]
\newtheorem{proposition}{Proposition}[chapter]
\newtheorem{question}{Question}[chapter]
\newtheorem{remark}{Remark}[chapter]
\newtheorem{theorem}{Theorem}[chapter]
\usepackage[left=0.5in,right=0.5in,top=1.5cm,bottom=1.5cm]{geometry}
\usepackage{fancyhdr}
\pagestyle{fancy}
\fancyhf{}
\lhead{\small \textsc{Sect.} ~\thesection}
\rhead{\small \nouppercase{\leftmark}}
\renewcommand{\sectionmark}[1]{\markboth{#1}{}}
\cfoot{\thepage}
\def\labelitemii{$\circ$}

\title{Inequality}
\author{\selectlanguage{vietnamese} Nguyễn Quản Bá Hồng\footnote{Independent Researcher, Ben Tre City, Vietnam\\e-mail: \texttt{nguyenquanbahong@gmail.com}; website: \url{https://nqbh.github.io}.}}
\date{\today}

\begin{document}
\maketitle
\setcounter{secnumdepth}{4}
\setcounter{tocdepth}{4}
\tableofcontents

%------------------------------------------------------------------------------%

\chapter{Wikipedia's}

\section{\href{https://en.wikipedia.org/wiki/Inequality_(mathematics)}{Wikipedia\texttt{/}Inequality (Mathematics)}}

%------------------------------------------------------------------------------%

\section{\href{https://en.wikipedia.org/wiki/Isoperimetric_inequality}{Wikipedia\texttt{/}Isoperimetric Inequality}}
``In mathematics, the \textit{isoperimetric inequality} is a \href{https://en.wikipedia.org/wiki/Geometry}{geometry} \href{https://en.wikipedia.org/wiki/Inequality_(mathematics)}{inequality} involving the perimeter of a set \& its volume. In $n$-dimensional space $\mathbb{R}^n$ the inequality lower bounds the \href{https://en.wikipedia.org/wiki/Surface_area}{surface area} or \href{https://en.wikipedia.org/wiki/Perimeter}{perimeter} $\operatorname{per}(S)$ of a set $S\subset\mathbb{R}^n$ by its \href{https://en.wikipedia.org/wiki/Volume}{volume} $\operatorname{vol}(S)$,
\begin{align*}
	\operatorname{per}(S)\ge n\operatorname{vol}(S)^{1 - \frac{1}{n}}\operatorname{vol}(B_1)^{\frac{1}{n}},
\end{align*}
where $B_1\subset\mathbb{R}^n$ is a \href{https://en.wikipedia.org/wiki/Unit_sphere}{unit sphere}. The equality holds only when $S$ is a sphere in $\mathbb{R}^n$.

On a plane, i.e., when $n = 2$, the isoperimetric inequality relates the square of the \href{https://en.wikipedia.org/wiki/Circumference}{circumference} of a \href{https://en.wikipedia.org/wiki/Closed_curve}{closed curve} \& the \href{https://en.wikipedia.org/wiki/Area}{area} of a plane region it encloses. \href{https://en.wiktionary.org/wiki/isoperimetric#English}{\textit{Isoperimetric}} literally means ``having the same \href{https://en.wikipedia.org/wiki/Perimeter}{perimeter}''. Specifically in $\mathbb{R}^2$, the isoperimetric inequality states, for the length $L$ of a closed curve \& the area $A$ of the planar region that it encloses, that $L^2\ge 4\pi A$, \& that equality holds iff the curve is a circle.

The \textit{isoperimetric problem} is to determine a \href{https://en.wikipedia.org/wiki/Plane_figure}{plane figure} of the largest possible area whose \href{https://en.wikipedia.org/wiki/Boundary_(topology)}{boundary} has a specified length. The closely related \textit{Dido's problem} asks for a region of the maximal area bounded by a straight line \& a curvilinear \href{https://en.wikipedia.org/wiki/Arc_(geometry)}{arc} whose endpoints belong to that line. It is named after \href{https://en.wikipedia.org/wiki/Dido_(Queen_of_Carthage)}{Dido}, the legendary founder \& 1st queen of \href{https://en.wikipedia.org/wiki/Carthage}{Carthage}. The solution to the isoperimetric problem is given by a \href{https://en.wikipedia.org/wiki/Circle}{circle} \& was known already in \href{https://en.wikipedia.org/wiki/Ancient_Greece}{Ancient Greece}. However, the 1st mathematically rigorous proof of this fact was obtained only in the 19th century. Since then, many other proofs have been found.

The isoperimetric problem has been extended in multiple ways, e.g., to curves on \href{https://en.wikipedia.org/wiki/Differential_geometry_of_surfaces}{surfaces} \& to regions in higher-dimensional spaces. Perhaps the most familiar physical manifestation of the 3D isoperimetric inequality is the shape of a drop of water. Namely, a drop will typically assume a symmetric round shape. Since the amount of water in a drop is fixed, \href{https://en.wikipedia.org/wiki/Surface_tension}{surface tension} forces the drop into a shape which minimizes the surface area of the drop, namely a round sphere.

\subsection{The isoperimetric problem in the plane}

\subsection{On a plane}

\subsection{On a sphere}

\subsection{In $\mathbb{R}^n$}

\subsection{In Hadamard manifolds}

\subsection{In a metric measure space}

\subsection{For graphs}

\subsubsection{Example: Isoperimetric inequalities for hypercubes}

\paragraph{Edge isoperimetric inequality.}

\paragraph{Vertex isoperimetric inequality.}

\subsection{Isoperimetric inequality for triangles}

%------------------------------------------------------------------------------%

\chapter{\cite{Hardy_Littlewood_Polya1952}. Inequality}

\begin{quotation}
	\textit{``Oh! the little more, \& how much it is! \& the little less, \& what worlds away!''} -- Robert Browning
\end{quotation}

\section*{Preface to 1st Edition}
``This book was planned \& begun in 1929. Our original intention was that it should be 1 of the \textit{Cambridge Tracts}, but it soon became plain that a tract\footnote{\textbf{tract} [n] \textbf{1.} (\textit{biology}) a system of connected organs or tissues along which materials or messages pass; \textbf{2.} \textbf{tract (of something)} an area of land, especially a large one; \textbf{3.} a short piece of writing, especially on a religious, moral or political subject, that is intended to influence people's ideas.} would be much too short for our purpose.

Our subjects in writing the book are explained sufficiently in the introductory chapter, but we add a note here about history \& bibliography\footnote{\textbf{bibliography} [n] (plural \textbf{bibliographies}) the list of books, etc. that have been used by somebody writing an article, essay, etc.; a list of books or articles about a particular subject or by a particular author.}. Historical \& bibliographical questions are particularly troublesome\footnote{\textbf{troublesome} [a] causing trouble, pain or difficulties.} in a subject like this, which \fbox{has applications in every part of mathematics but has never been developed systematically}.

It is often really difficult to trace the origin of a familiar inequality. It is quite likely to occur 1st as an auxiliary proposition, often without explicit statement, in a memoir on geometry or astronomy; it may have been rediscovered, many years later, by half a dozen different authors; \& no accessible statement of it may be quite complete. We have almost always found, even with the most famous inequalities, that we have a little new to add.

We have done our best to be accurate \& have given all references we can, but we have never undertaken\footnote{\textbf{undertake} [v] \textbf{1.} \textbf{undertake something} to make yourself responsible for something \& start doing it; \textbf{2.} to agree or promise that you will do something.} systematic bibliographical research. We follow the common practice, when a particular inequality is habitually\footnote{\textbf{habitual} [a] [only before noun] usual or typical of somebody\texttt{/}something.} associated with a particular mathematician's name; we speak of the inequalities of Schwarz, H\"older, \& Jensen, though all these inequalities can be traced further back; \& we do not enumerate\footnote{\textbf{enumerate} [v] (\textit{formal}) \textbf{enumerate something} to name things on a list 1 by 1.} explicitly all the minor additions which are necessary for absolute completeness.

%------------------------------------------------------------------------------%

\section{Introduction}

\subsection{Finite, infinite, \& integral inequalities}

\subsection{Notations}

\subsection{Positive inequalities}

\subsection{Homogeneous inequalities}

\subsection{The axiomatic basis of algebraic inequalities}

\subsection{Comparable functions}

\subsection{Selection of proofs}

\subsection{Selection of subjects}

%------------------------------------------------------------------------------%

\section{Elementary Mean Values}

\subsection{Ordinary means}

\subsection{Weighted means}

\subsection{Limiting cases of $\mathfrak{M}_r(a)$}

\subsection{Cauchy's inequality}

\subsection{The theorem of the arithmetic \& geometric means}

\subsection{Other proofs of the theorem of the means}

\subsection{H\"older's inequality \& its extensions}

\subsection{General properties of the means $\mathfrak{M}_r(a)$}

\subsection{The sums $\mathfrak{G}_r(a)$}

\subsection{Minkowski's inequality}

\subsection{A companion to Minkowski's inequality}

\subsection{Illustrations \& applications of the fundamental inequalities}

\subsection{Inductive proofs of the fundamental inequalities}

\subsection{Elementary inequalities connected with Theorem 37}

\subsection{Elementary proof of Theorem 3}

\subsection{Tchebychef's inequality}

\subsection{Muirhead's theorem}

\subsection{Proof of Muirhead's theorem}

\subsection{An alternative theorem}

\subsection{Further theorems on symmetrical means}

\subsection{The elementary symmetric functions of $n$ positive numbers}

\subsection{A note on definite forms}

\subsection{A theorem concerning strictly positive forms}

\subsection{Miscellaneous theorems \& examples}

%------------------------------------------------------------------------------%

\section{Mean Values with an Arbitrary Function \& the Theory of Convex Functions}

\subsection{Definitions}

\subsection{Equivalent means}

\subsection{A characteristic property of the means $\mathfrak{M}_r$}

\subsection{Comparability}

\subsection{Convex functions}

\subsection{Continuous convex functions}

\subsection{An alternative definition}

\subsection{Equality in the fundamental inequalities}

\subsection{Restatements \& extensions of Theorem 85}

\subsection{Twice differentiable convex functions}

\subsection{Applications of the properties of twice differentiable convex functions}

\subsection{Convex functions of several variables}

\subsection{Generalizations of H\"older's inequality}

\subsection{Some theorems concerning monotonic functions}

\subsection{Sums with an arbitrary function: generalizations of Jensen's inequality}

\subsection{Generalizations of Minkowski's inequality}

\subsection{Comparison of sets}

\subsection{Further general properties of convex functions}

\subsection{Further properties of continuous convex functions}

\subsection{Discontinuous convex functions}

\subsection{Miscellaneous theorems \& examples}

%------------------------------------------------------------------------------%

\section{Various Applications of the Calculus}

\subsection{Introduction}

\subsection{Applications of the mean value theorem}

\subsection{Further applications of elementary differential calculus}

\subsection{Maxima \& minima of functions of 1 variable}

\subsection{Use of Taylor's series}

\subsection{Applications of the theory of maxima \& minima of functions of several variables}

\subsection{Comparison of series \& integrals}

\subsection{An inequality of W. H. Young}

%------------------------------------------------------------------------------%

\section{Infinite Series}

\subsection{Introduction}

\subsection{The means $\mathfrak{M}_r$}

\subsection{The generalization of Theorems 3 \& 9}

\subsection{H\"older's inequality \& its extensions}

\subsection{The means $\mathfrak{M}_r$ (\textit{cont}.)}

\subsection{The sums $\mathfrak{G}_r$}

\subsection{Minkowski's inequality}

\subsection{Tchebychef's inequality}

\subsection{A summary}

\subsection{Miscellaneous theorems \& examples}

%------------------------------------------------------------------------------%

\section{Integrals}

\subsection{Preliminary remarks on Lebesgue integrals}

\subsection{Remarks on null sets \& null functions}

\subsection{Further remarks concerning integration}

\subsection{Remarks on methods of proof}

\subsection{Further remarks on method: the inequality of Schwarz}

\subsection{Definition of the means $\mathfrak{M}_r(f)$ when $r\ne 0$}

\subsection{The geometric mean of a function}

\subsection{Further properties of the geometric mean}

\subsection{H\"older's inequality for integrals}

\subsection{General properties of the means $\mathfrak{M}_r(f)$}

\subsection{Convexity of $\log\mathfrak{M}_r^r$}

\subsection{Minkowski's inequality for integrals}

\subsection{Mean values depending on an arbitrary function}

\subsection{The definition of the Stieltjes integral}

\subsection{Special cases of the Stieltjes integral}

\subsection{Extensions of earlier theorems}

\subsection{The means $\mathfrak{M}_r(f;\phi)$}

\subsection{Distribution functions}

\subsection{Characterization of means values}

\subsection{Remarks on the characteristic properties}

\subsection{Completion of the proof of Theorem 215}

\subsection{Miscellaneous theorems \& examples}

%------------------------------------------------------------------------------%

\section{Some Applications of the Calculus of Variations}

\subsection{Some general remarks}

\subsection{Object of the present chapter}

\subsection{Example of an inequality corresponding to an unattained extremum}

\subsection{1st proof of Theorem 254}

\subsection{2nd proof of Theorem 254}

\subsection{Further examples illustrative of variational methods}

\subsection{Further examples: Wirtinger's inequality}

\subsection{An example involving 2nd derivatives}

\subsection{A simpler theorem}

\subsection{Miscellaneous theorems \& examples}

%------------------------------------------------------------------------------%

\section{Some Theorems Concerning Bilinear \& Multilinear Forms}

\subsection{Introduction}

\subsection{An inequality for multilinear forms with positive variables \& coefficients}

\subsection{A theorem of W. H. Young}

\subsection{Generalizations \& analogues}

\subsection{Applications to Fourier series}

\subsection{The convexity theorem for positive multilinear forms}

\subsection{The convexity theorem for positive multilinear forms}

\subsection{General bilinear forms}

\subsection{Definition of a bounded bilinear form}

\subsection{Some properties of bounded forms in $[p,q]$}

\subsection{The Faltung of 2 forms in $[p,p']$}

\subsection{Some special theorems on forms in $[2,2]$}

\subsection{Application to Hilbert's forms}

\subsection{The convexity theorem for bilinear forms with complex variables \& coefficients}

\subsection{Further properties of a maximal set $(x,y)$}

\subsection{Proof of Theorem 295}

\subsection{Applications of the theorem of M. Riesz}

\subsection{Applications to Fourier series}

\subsection{Miscellaneous theorems \& examples}

%------------------------------------------------------------------------------%

\section{Hilbert's Inequality \& Its Analogues \& Extensions}

\subsection{Hilbert's double series theorem}

\subsection{A general class of bilinear forms}

\subsection{The corresponding theorem for integrals}

\subsection{Extensions of Theorems 318 \& 319}

\subsection{Best possible constants: proof of Theorem 317}

\subsection{Further remarks on Hilbert's theorems}

\subsection{Applications of Hilbert's theorems}

\subsection{Hardy's inequality}

\subsection{Further integral inequalities}

\subsection{Further theorems concerning series}

\subsection{Deduction of theorems on series from theorems on integrals}

\subsection{Carleman's inequality}

\subsection{Theorems with $0 < p < 1$}

\subsection{A theorem with 2 parameters $p$ \& $q$}

\subsection{Miscellaneous theorems \& examples}

%------------------------------------------------------------------------------%

\section{Rearrangements}

\subsection{Rearrangements of finite sets of variables}

\subsection{A theorem concerning the rearrangements of 2 sets}

\subsection{A 2nd proof of Theorem 368}

\subsection{Restatement of Theorem 368}

\subsection{Theorems concerning the rearrangements of 3 sets}

\subsection{Reduction of Theorem 373 to a special case}

\subsection{Completion of the proof}

\subsection{Another proof of Theorem 371}

\subsection{Rearrangements of any number of sets}

\subsection{A further theorem on the rearrangement of any number of sets}

\subsection{Applications}

\subsection{The rearrangement of a function}

\subsection{On the rearrangement of 2 functions}

\subsection{On the rearrangement of 3 functions}

\subsection{Completion of the proof of Theorem 379}

\subsection{An alternative proof}

\subsection{Applications}

\subsection{Another theorem concerning the rearrangement of a function in decreasing order}

\subsection{Proof of Theorem 384}

\subsection{Miscellaneous theorems \& examples}

%------------------------------------------------------------------------------%

\section{Appendices}

\subsection{Appendix I: On strictly positive forms}

\subsection{Appendix II: Thorin's proof \& extension of Theorem 295}

\subsection{Appendix III: On Hilbert's inequality}

%------------------------------------------------------------------------------%

%------------------------------------------------------------------------------%

\printbibliography[heading=bibintoc]
	
\end{document}