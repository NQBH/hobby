\documentclass[oneside]{book}
\usepackage[backend=biber,natbib=true,style=authoryear]{biblatex}
\addbibresource{/home/hong/1_NQBH/reference/bib.bib}
\usepackage[vietnamese,english]{babel}
\usepackage{tocloft}
\renewcommand{\cftsecleader}{\cftdotfill{\cftdotsep}}
\usepackage[colorlinks=true,linkcolor=blue,urlcolor=red,citecolor=magenta]{hyperref}
\usepackage{amsmath,amssymb,amsthm,mathtools,float,graphicx}
\allowdisplaybreaks
\numberwithin{equation}{section}
\newtheorem{assumption}{Assumption}[chapter]
\newtheorem{conjecture}{Conjecture}[chapter]
\newtheorem{corollary}{Corollary}[chapter]
\newtheorem{definition}{Definition}[chapter]
\newtheorem{example}{Example}[chapter]
\newtheorem{lemma}{Lemma}[chapter]
\newtheorem{notation}{Notation}[chapter]
\newtheorem{principle}{Principle}[chapter]
\newtheorem{problem}{Problem}[chapter]
\newtheorem{proposition}{Proposition}[chapter]
\newtheorem{question}{Question}[chapter]
\newtheorem{remark}{Remark}[chapter]
\newtheorem{theorem}{Theorem}[chapter]
\usepackage[left=0.5in,right=0.5in,top=1.5cm,bottom=1.5cm]{geometry}
\usepackage{fancyhdr}
\pagestyle{fancy}
\fancyhf{}
\lhead{\small \textsc{Sect.} ~\thesection}
\rhead{\small \nouppercase{\leftmark}}
\renewcommand{\sectionmark}[1]{\markboth{#1}{}}
\cfoot{\thepage}
\def\labelitemii{$\circ$}

\title{Inequality}
\author{\selectlanguage{vietnamese} Nguyễn Quản Bá Hồng\footnote{Independent Researcher, Ben Tre City, Vietnam\\e-mail: \texttt{nguyenquanbahong@gmail.com}; website: \url{https://nqbh.github.io}.}}
\date{\today}

\begin{document}
\maketitle
\setcounter{secnumdepth}{4}
\setcounter{tocdepth}{4}
\tableofcontents

%------------------------------------------------------------------------------%

\chapter{Wikipedia's}

\section{\href{https://en.wikipedia.org/wiki/Inequality_(mathematics)}{Wikipedia\texttt{/}Inequality (Mathematics)}}

%------------------------------------------------------------------------------%

\section{\href{https://en.wikipedia.org/wiki/Isoperimetric_inequality}{Wikipedia\texttt{/}Isoperimetric Inequality}}
``In mathematics, the \textit{isoperimetric inequality} is a \href{https://en.wikipedia.org/wiki/Geometry}{geometry} \href{https://en.wikipedia.org/wiki/Inequality_(mathematics)}{inequality} involving the perimeter of a set \& its volume. In $n$-dimensional space $\mathbb{R}^n$ the inequality lower bounds the \href{https://en.wikipedia.org/wiki/Surface_area}{surface area} or \href{https://en.wikipedia.org/wiki/Perimeter}{perimeter} $\operatorname{per}(S)$ of a set $S\subset\mathbb{R}^n$ by its \href{https://en.wikipedia.org/wiki/Volume}{volume} $\operatorname{vol}(S)$,
\begin{align*}
	\operatorname{per}(S)\ge n\operatorname{vol}(S)^{1 - \frac{1}{n}}\operatorname{vol}(B_1)^{\frac{1}{n}},
\end{align*}
where $B_1\subset\mathbb{R}^n$ is a \href{https://en.wikipedia.org/wiki/Unit_sphere}{unit sphere}. The equality holds only when $S$ is a sphere in $\mathbb{R}^n$.

On a plane, i.e., when $n = 2$, the isoperimetric inequality relates the square of the \href{https://en.wikipedia.org/wiki/Circumference}{circumference} of a \href{https://en.wikipedia.org/wiki/Closed_curve}{closed curve} \& the \href{https://en.wikipedia.org/wiki/Area}{area} of a plane region it encloses. \href{https://en.wiktionary.org/wiki/isoperimetric#English}{\textit{Isoperimetric}} literally means ``having the same \href{https://en.wikipedia.org/wiki/Perimeter}{perimeter}''. Specifically in $\mathbb{R}^2$, the isoperimetric inequality states, for the length $L$ of a closed curve \& the area $A$ of the planar region that it encloses, that $L^2\ge 4\pi A$, \& that equality holds iff the curve is a circle.

The \textit{isoperimetric problem} is to determine a \href{https://en.wikipedia.org/wiki/Plane_figure}{plane figure} of the largest possible area whose \href{https://en.wikipedia.org/wiki/Boundary_(topology)}{boundary} has a specified length. The closely related \textit{Dido's problem} asks for a region of the maximal area bounded by a straight line \& a curvilinear \href{https://en.wikipedia.org/wiki/Arc_(geometry)}{arc} whose endpoints belong to that line. It is named after \href{https://en.wikipedia.org/wiki/Dido_(Queen_of_Carthage)}{Dido}, the legendary founder \& 1st queen of \href{https://en.wikipedia.org/wiki/Carthage}{Carthage}. The solution to the isoperimetric problem is given by a \href{https://en.wikipedia.org/wiki/Circle}{circle} \& was known already in \href{https://en.wikipedia.org/wiki/Ancient_Greece}{Ancient Greece}. However, the 1st mathematically rigorous proof of this fact was obtained only in the 19th century. Since then, many other proofs have been found.

The isoperimetric problem has been extended in multiple ways, e.g., to curves on \href{https://en.wikipedia.org/wiki/Differential_geometry_of_surfaces}{surfaces} \& to regions in higher-dimensional spaces. Perhaps the most familiar physical manifestation of the 3D isoperimetric inequality is the shape of a drop of water. Namely, a drop will typically assume a symmetric round shape. Since the amount of water in a drop is fixed, \href{https://en.wikipedia.org/wiki/Surface_tension}{surface tension} forces the drop into a shape which minimizes the surface area of the drop, namely a round sphere.'' -- \href{https://en.wikipedia.org/wiki/Isoperimetric_inequality}{Wikipedia\texttt{/}isoperimetric inequality}

\subsection{The isoperimetric problem in the plane}
\textsf{Fig. If a region is not convex, a ``dent'' in its boundary can be ``flipped'' to increase the area of the region while keeping the perimeter unchanged.} \textsf{Fig. An elongated shape can be made more round while keeping its perimeter fixed \& increasing its area.}

``The classical \textit{isoperimetric problem} dates back to antiquity. The problem can be stated as follows: Among all closed \href{https://en.wikipedia.org/wiki/Curve}{curves} in the plane of fixed perimeter, which curve (if any) maximizes the area of its enclosed region? This question can be shown to be equivalent to be following problem: Among all closed curves in the plane enclosing a fixed area, which curve (if any) minimizes the perimeter?

This problem is conceptually related to the \href{https://en.wikipedia.org/wiki/Principle_of_least_action}{principle of least action} in \href{https://en.wikipedia.org/wiki/Physics}{physics}, in that it can be restated: what is the principle of action which encloses the greatest area, with the greatest economy of effort? The 15th-century philosopher \& scientist, Cardinal \href{https://en.wikipedia.org/wiki/Nicholas_of_Cusa}{Nicholas of Cusa}, considered \href{https://en.wikipedia.org/wiki/Rotation}{rotational} action, the process by which a \href{https://en.wikipedia.org/wiki/Circle}{circle} is generated, to be the most direct reflection, in the realm of sensory impressions, of the process by which the universe is created. German astronomer \& astrologer \href{https://en.wikipedia.org/wiki/Johannes_Kepler}{Johannes Kepler} invoked the isoperimetric principle in discussing the morphology of the solar system, in \href{https://en.wikipedia.org/wiki/Mysterium_Cosmographicum}{\textit{Mysterium Cosmographicum}} (\textit{The Sacred Mystery of the Cosmos}, 1596).

Although the circle appears to be an obvious solution to the problem, proving this fact is rather difficult. The 1st progress toward the solution was made by Swiss geometer \href{https://en.wikipedia.org/wiki/Jakob_Steiner}{Jakob Steiner} in 1838, using a geometric method later named \href{https://en.wikipedia.org/wiki/Symmetrization_methods#Steiner_Symmetrization}{\textit{Steiner symmetrization}}. Steiner showed that if a solution existed, then it must be the circle. Steiner's proof was completed later by several other mathematicians.

Steiner begins with some geometric constructions which are easily understood; e.g., it can be shown that any closed curve enclosing a region that is not fully \href{https://en.wikipedia.org/wiki/Convex_set}{convex} can be modified to enclose more area, by ``flipping'' the concave areas so that they become convex. It can further be shown that any closed curve which is not fully symmetrical can be ``tilted'' so that it encloses more area. The 1 shape that is perfectly convex \& symmetrical is the circle, although this, in itself, does not represent a rigorous proof of the isoperimetric theorem.'' -- \href{https://en.wikipedia.org/wiki/Isoperimetric_inequality#The_isoperimetric_problem_in_the_plane}{Wikipedia\texttt{/}isoperimetric inequality\texttt{/}the isoperimetric problem in the plane}

\subsection{On a plane}
``The solution to the isoperimetric problem is usually expressed in the form of an \href{https://en.wikipedia.org/wiki/Inequality_(mathematics)}{inequality} that relates the length $L$ of a closed curve \& the area $A$ of the planar region that it encloses. The \textit{isoperimetric inequality} states that $4\pi A\le L^2$, \& that the equality holds iff the curve is a circle. The \href{https://en.wikipedia.org/wiki/Area_of_a_disk}{area of a disk} of radius $R$ is $\pi R^2$ \& the circumference of the circle is $2\pi R$, so both sides of the inequality $= 4\pi^2R^2$ in this case.

Dozens of proofs of the isoperimetric inequality have been found. In 1902, \href{https://en.wikipedia.org/wiki/Adolf_Hurwitz}{Hurwitz} published a short proof using the \href{https://en.wikipedia.org/wiki/Fourier_series}{Fourier serise} that applies to arbitrary \href{https://en.wikipedia.org/wiki/Rectifiable_curve}{rectifiable curves} (not assumed to be smooth) An elegant direct proof based on comparison of a smooth simple closed curve with an appropriate circle was given by E. Schmidt in 1938. It uses only the \href{https://en.wikipedia.org/wiki/Arc_length}{arc length} formula, expression for the area of a plane region from \href{https://en.wikipedia.org/wiki/Green%27s_theorem}{Green's theorem}, \& the \href{https://en.wikipedia.org/wiki/Cauchy%E2%80%93Schwarz_inequality}{Cauchy--Schwarz inequality}.

For a given closed curve, the \textit{isoperimetric quotient} is defined as the ratio of its area \& that of the circle having the same perimeter. This is equal to $Q = \frac{4\pi A}{L^2}$ \& the isoperimetric inequality says that $Q\le 1$. Equivalently, the \href{https://en.wikipedia.org/wiki/Isoperimetric_ratio}{isoperimetric ratio} $\frac{L^2}{A}$ is at least $4\pi$ for every curve.

The isoperimetric quotient of a regular $n$-gon is $Q_n = \frac{\pi}{n\tan\frac{\pi}{n}}$. Let $C$ be a smooth regular convex closed curve. Then the \textit{improved isoperimetric inequality} states the following $L^2\ge 4\pi A + 8\pi|\widetilde{A}_{0.5}|$, where $L,A,\widetilde{A}_{0.5}$ denote the length of $C$, the area of the region bounded by $C$ \& the oriented area of the Wigner caustic of $C$, respectively, \& the equality holds iff $C$ is a \href{https://en.wikipedia.org/wiki/Curve_of_constant_width}{curve of constant width}.'' -- \href{https://en.wikipedia.org/wiki/Isoperimetric_inequality#On_a_plane}{Wikipedia\texttt{/}isoperimetric inequality\texttt{/}on a plane}

\subsection{On a sphere}
``Let $C$ be a simple closed curve on a \href{https://en.wikipedia.org/wiki/Sphere}{sphere} of radius 1. Denote by $L$ the length of $C$ \& by $A$ the area enclosed by $C$. The \textit{spherical isoperimetric inequality} states that $L^2\ge A(4\pi - A)$, \& that the equality holds iff the curve is a circle. There are, in fact, 2 ways to measure the spherical area enclosed by a simple closed curve, but the inequality is symmetric w.r.t. taking the complement.

This inequality was discovered by \href{https://en.wikipedia.org/wiki/Paul_L%C3%A9vy_(mathematician)}{Paul L\'evy} (1919) who also extended it to higher dimensions \& general surfaces. In the more general case of arbitrary radius $R$, it is known that $L^2\ge 4\pi A - \frac{A^2}{R^2}$.'' -- \href{https://en.wikipedia.org/wiki/Isoperimetric_inequality#On_a_sphere}{Wikipedia\texttt{/}isoperimetric inequality\texttt{/}on a sphere}

\subsection{In $\mathbb{R}^n$}
``The isoperimetric inequality states that a \href{https://en.wikipedia.org/wiki/Sphere}{sphere} has the smallest surface area per given volume. Given a bounded set $S\subset\mathbb{R}^n$ with \href{https://en.wikipedia.org/wiki/Surface_area}{surface area} $\operatorname{per}(S)$ \& \href{https://en.wikipedia.org/wiki/Volume}{volume} $\operatorname{vol}(S)$, the isoperimetric inequality states $\operatorname{per}(S)\ge n\operatorname{vol}(S)^{1 - \frac{1}{n}}\operatorname{vol}(B_1)^{\frac{1}{n}}$, where $B_1\subset\mathbb{R}^n$ is a \href{https://en.wikipedia.org/wiki/Unit_sphere}{unit ball}. The equality holds when $S$ is a ball in $\mathbb{R}^n$. Under additional restrictions on the set (e.g., \href{https://en.wikipedia.org/wiki/Convex_set}{convexity}, \href{https://en.wikipedia.org/wiki/Closed_regular_set}{regularity}, \href{https://en.wikipedia.org/wiki/Smooth_surface}{smooth boundary}), the equality holds for a ball only. But in full generality the situation is more complicated. The relevant result of [Schmidt1949, Sect. 20.7] (for a simpler proof see [Baebler1957]) is clarified in [Hadwiger1957, Sect 5.2.5] as follows. An extremal set consists of a ball \& a ``corona'' that contributes neither to the volume nor to the surface area. I.e., the equality holds for a compact set $S$ iff $S$ contains a closed ball $B$ s.t. $\operatorname{vol}(B) = \operatorname{vol}(S)$ \& $\operatorname{per}(B) = \operatorname{per}(S)$. E.g., the ``corona'' may be a curve.

The proof of the inequality follows directly from \href{https://en.wikipedia.org/wiki/Brunn%E2%80%93Minkowski_theorem}{Brunn--Minkowski inequality} between a set $S$ \& ball with radius $\epsilon$, i.e., $B_\epsilon = \epsilon B_1$. By taking Brunn--Minkowski inequality to the power $n$, subtracting $\operatorname{vol}(S)$ from both sides, dividing them by $\epsilon$, \& taking the limit as $\epsilon\to 0$. ([Osserman1978, Federer1969, \S3.2.43]).

In full generality [Federer1969, \S3.2.43], the isoperimetric inequality states that for any set $S\subset\mathbb{R}^n$ whose \href{https://en.wikipedia.org/wiki/Closure_of_a_set}{closure} has finite \href{https://en.wikipedia.org/wiki/Lebesgue_measure}{Lebesgue measure}
\begin{align*}
	n\omega_n^{\frac{1}{n}}L^n(\overline{S})^{1 - \frac{1}{n}}\le M_\star^{n - 1}(\partial S),
\end{align*}
where $M_\star^{n - 1}$ is the $(n - 1)$-dimensional \href{https://en.wikipedia.org/wiki/Minkowski_content}{Minkowski content}, $L^n$ is the $n$-dimensional Lebesgue measure, \& $\omega_n$ is the volume of the \href{https://en.wikipedia.org/wiki/Unit_ball}{unit ball} in $\mathbb{R}^n$. If the boundary of $S$ is \href{https://en.wikipedia.org/wiki/Rectifiable_curve}{rectifiable}, then the Minkowski content is the $(n - 1)$-dimensional \href{https://en.wikipedia.org/wiki/Hausdorff_measure}{Hausdorff measure}.

The $n$-dimensional isoperimetric inequality is equivalent (for sufficient smooth domains) to the \href{https://en.wikipedia.org/wiki/Sobolev_inequality}{Sobolev inequality} on $\mathbb{R}^n$ with optimal constant:
\begin{align*}
	\left(\int_{\mathbb{R}^n} |u|^{\frac{n}{n - 1}}\right)^{\frac{n - 1}{n}}\le\frac{1}{n}\omega_n^{-\frac{1}{n}}\int_{\mathbb{R}^n} |\nabla u|,\ \forall u\in W^{1,1}(\mathbb{R}^n).
\end{align*}
'' -- \href{https://en.wikipedia.org/wiki/Isoperimetric_inequality#In_Rn}{Wikipedia\texttt{/}isoperimetric inequality\texttt{/}in $\mathbb{R}^n$}

\subsection{In Hadamard manifolds}
``\href{https://en.wikipedia.org/wiki/Hadamard_manifold}{Hadamard manifolds} are complete simply connected manifolds with nonpositive curvature. Thus they generalize the Euclidean space $\mathbb{R}^n$, which is a Hadamard manifold with curvature zero. In 1970's \& early 80's, \href{https://en.wikipedia.org/wiki/Thierry_Aubin}{Thierry Aubin}, \href{https://en.wikipedia.org/wiki/Mikhail_Leonidovich_Gromov}{Misha Gromov}, \href{https://en.wikipedia.org/wiki/Yuri_Burago}{Yuri Burago}, \& \href{https://en.wikipedia.org/wiki/Viktor_Zalgaller}{Viktor Zalgaller} conjectured that the Euclidean isoperimetric inequality $\operatorname{per}(S)\ge n\operatorname{vol}(S)^{1 - \frac{1}{n}}\operatorname{vol}(B_1)^{\frac{1}{n}}$ holds for bounded sets $S$ in Hadamard manifolds, which has become known as the \href{https://en.wikipedia.org/wiki/Cartan%E2%80%93Hadamard_conjecture}{Cartan--Hadamard conjecture}. In dimension 2 this had already been established in 1926 by \href{https://en.wikipedia.org/wiki/Andr%C3%A9_Weil}{Andr\'e Weil}, who was a student of \href{https://en.wikipedia.org/wiki/Jacques_Hadamard}{Hadamard} at the time. In dimensions 3 \& 4 the conjecture was proved by \href{https://en.wikipedia.org/wiki/Bruce_Kleiner}{Bruce Kleiner} in 1992, \& Chris Croke in 1984 respectively.'' -- \href{https://en.wikipedia.org/wiki/Isoperimetric_inequality#In_Hadamard_manifolds}{Wikipedia\texttt{/}isoperimetric inequality\texttt{/}in Hadamard manifolds}

\subsection{In a metric measure space}
``Most of the work on isoperimetric problem has been done in the context of smooth regions in \href{https://en.wikipedia.org/wiki/Euclidean_space}{Euclidean spaces}, or more generally, in \href{https://en.wikipedia.org/wiki/Riemannian_manifold}{Riemannian manifolds}. However, the isoperimetric problem can be formulated in much greater generality, using the notion of \textit{Minkowski content}. Let $(X,\mu,d)$ be a \textit{metric measure space}: $X$ is a \href{https://en.wikipedia.org/wiki/Metric_space}{mertic space} with \href{https://en.wikipedia.org/wiki/Metric_(mathematics)}{metric} $d$, \& $\mu$ is a \href{https://en.wikipedia.org/wiki/Borel_measure}{Borel measure} on $X$. The \textit{boundary measure}, or \href{https://en.wikipedia.org/wiki/Minkowski_content}{Minkowski content}, of a \href{https://en.wikipedia.org/wiki/Measurable}{measurable} subset $A$ of $X$ is defined as the \href{https://en.wikipedia.org/wiki/Lim_inf}{lim inf} $\mu^+(A)\coloneqq\liminf_{\varepsilon\downarrow 0}\frac{\mu(A_\varepsilon) - \mu(A)}{\varepsilon}$, where $A_\varepsilon\coloneqq\{x\in X|d(x,A)\le\varepsilon\}$ is the \textit{$\varepsilon$-extension} of $A$.

The isoperimetric problem in $X$ asks how small can $\mu^+(A)$ be for a given $\mu(A)$. If $X$ is the \href{https://en.wikipedia.org/wiki/Plane_(mathematics)}{Euclidean plane} with the usual distance \& the \href{https://en.wikipedia.org/wiki/Lebesgue_measure}{Lebesgue measure} then this question generalizes the classical isoperimetric problem to planar regions whose boundary is not necessarily smooth, although the answer turns out to be the same.

The function $I(a) = \inf\{\mu^+(A)|\mu(A) = a\}$ is called the \textit{isoperimetric profile} of the metric measure space $(X,\mu,d)$. Isoperimetric profiles have been studied for \href{https://en.wikipedia.org/wiki/Cayley_graph}{Cayley graphs} of \href{https://en.wikipedia.org/wiki/Discrete_group}{discrete groups} \& for special classes of Riemannian manifolds (where usually only regions $A$ with regular boundary are considered).'' -- \href{https://en.wikipedia.org/wiki/Isoperimetric_inequality#In_a_metric_measure_space}{Wikipedia\texttt{/}isoperimetric inequality\texttt{/}isoperimetric inequality}

\subsection{For graphs}
``Main article: \href{https://en.wikipedia.org/wiki/Expander_graph}{Wikipedia\texttt{/}expander graph}. In \href{https://en.wikipedia.org/wiki/Graph_theory}{graph theory}, isoperimetric inequalities are at the heart of the study of \href{https://en.wikipedia.org/wiki/Expander_graphs}{expander graph}, which are \href{https://en.wikipedia.org/wiki/Sparse_graph}{sparse graphs} that have strong connectivity properties. Expander constructions have spawned research in pure \& applied mathematics, with several applications to \href{https://en.wikipedia.org/wiki/Computational_complexity_theory}{complexity theory}, design of robust \href{https://en.wikipedia.org/wiki/Computer_network}{computer networks}, \& the theory of \href{https://en.wikipedia.org/wiki/Error-correcting_code}{error-correcting codes}.

Isoperimetric inequalities for graphs relate the size of vertex subsets to the size of their boundary, which is usually measured by the number of edges leaving the subset (edge expansion) or by the number of neighboring vertices (vertex expansion). For a graph $G$ \& a number $k$, the following are 2 standard isoperimetric parameters for graphs. The edge isoperimetric parameter $\Phi_E(G,k) = \min_{S\subset V} \left\{|E(S,\overline{S})|:|S| = k\right\}$. The vertex isoperimetric parameter: $\Phi_V(G,k) = \min_{S\subset V} \left\{|\Gamma(S)\backslash S|:|S| = k\right\}$. Here $E(S,\overline{S})$ denotes the set of edges leaving $S$ \& $\Gamma(S)$ denotes the set of vertices that have a neighbor in $S$. The isoperimetric problem consists of understanding how the parameters $\Phi_E$ \& $\Phi_V$ behave for natural families of graphs.

\subsubsection{Example: Isoperimetric inequalities for hypercubes}
The $d$-dimensional \href{https://en.wikipedia.org/wiki/Hypercube}{hypercube} $Q_d$ is the graph whose vertices are all Boolean vectors of length $d$, i.e., the set $\{0,1\}^d$. 2 such vectors are connected by an edge in $Q_d$ if they are equal up to a single bit flip, i.e., their \href{https://en.wikipedia.org/wiki/Hamming_distance}{Hamming distance} is exactly 1. The following are the isoperimetric inequalities for the Boolean hypercube.

\paragraph{Edge isoperimetric inequality.} The edge isoperimetric inequality of the hypercube is $\Phi_E(Q_d,k)\ge k(d - \log_2k)$. This bound is tight, as is witnessed by each set $S$ that is the set of vertices of any subcube of $Q_d$.

\paragraph{Vertex isoperimetric inequality.} Harper's theorem says that \textit{Hamming balls} have the smallest vertex boundary among all sets of a given size. Hamming balls are sets that contain all points of \href{https://en.wikipedia.org/wiki/Hamming_weight}{Hamming weight} at most $r$ \& no points of Hamming weight $> r + 1$ for some integer $r$. This theorem implies that any set $S\subset V$ with $|S|\ge\sum_{i=0}^{r} \binom{d}{i}$ satisfies $|S\cup\Gamma(S)|\ge\sum_{i=0}^{r+1} \binom{d}{i}$. As a special case, consider set sizes $k = |S|$ of the form $k = \binom{d}{0} + \binom{d}{1} + \cdots + \binom{d}{r}$ for some integer $r$. Then the above implies that the exact vertex isoperimetric parameter is $\Phi_V(Q_d,k) = \binom{d}{r + 1}$.'' -- \href{https://en.wikipedia.org/wiki/Isoperimetric_inequality#For_graphs}{Wikipedia\texttt{/}isoperimetric inequality\texttt{/}for graphs}

\subsection{Isoperimetric inequality for triangles}
``The isoperimetric inequality for triangles in terms of perimeter $p$ \& area $T$ states that $p^2\ge 12\sqrt{3}T$, with equality for the \href{https://en.wikipedia.org/wiki/Equilateral_triangle}{equilateral triangle}. This is implied, via the \href{https://en.wikipedia.org/wiki/Inequality_of_arithmetic_and_geometric_means}{AM--GM inequality}, by a stronger inequality which has also been called the \textit{isoperimetric inequality for triangles}: $T\le\frac{\sqrt{3}}{4}(abc)^{2/3}$.'' -- \href{https://en.wikipedia.org/wiki/Isoperimetric_inequality#Isoperimetric_inequality_for_triangles}{Wikipedia\texttt{/}isoperimetric inequality\texttt{/}isoperimetric inequality for triangles}

%------------------------------------------------------------------------------%

\chapter{\cite{Hardy_Littlewood_Polya1952}. Inequality}

\begin{quotation}
	\textit{``Oh! the little more, \& how much it is! \& the little less, \& what worlds away!''} -- Robert Browning
\end{quotation}

\section*{Preface to 1st Edition}
``This book was planned \& begun in 1929. Our original intention was that it should be 1 of the \textit{Cambridge Tracts}, but it soon became plain that a tract\footnote{\textbf{tract} [n] \textbf{1.} (\textit{biology}) a system of connected organs or tissues along which materials or messages pass; \textbf{2.} \textbf{tract (of something)} an area of land, especially a large one; \textbf{3.} a short piece of writing, especially on a religious, moral or political subject, that is intended to influence people's ideas.} would be much too short for our purpose.

Our subjects in writing the book are explained sufficiently in the introductory chapter, but we add a note here about history \& bibliography\footnote{\textbf{bibliography} [n] (plural \textbf{bibliographies}) the list of books, etc. that have been used by somebody writing an article, essay, etc.; a list of books or articles about a particular subject or by a particular author.}. Historical \& bibliographical questions are particularly troublesome\footnote{\textbf{troublesome} [a] causing trouble, pain or difficulties.} in a subject like this, which \fbox{has applications in every part of mathematics but has never been developed systematically}.

It is often really difficult to trace the origin of a familiar inequality. It is quite likely to occur 1st as an auxiliary proposition, often without explicit statement, in a memoir on geometry or astronomy; it may have been rediscovered, many years later, by half a dozen different authors; \& no accessible statement of it may be quite complete. We have almost always found, even with the most famous inequalities, that we have a little new to add.

We have done our best to be accurate \& have given all references we can, but we have never undertaken\footnote{\textbf{undertake} [v] \textbf{1.} \textbf{undertake something} to make yourself responsible for something \& start doing it; \textbf{2.} to agree or promise that you will do something.} systematic bibliographical research. We follow the common practice, when a particular inequality is habitually\footnote{\textbf{habitual} [a] [only before noun] usual or typical of somebody\texttt{/}something.} associated with a particular mathematician's name; we speak of the inequalities of Schwarz, H\"older, \& Jensen, though all these inequalities can be traced further back; \& we do not enumerate\footnote{\textbf{enumerate} [v] (\textit{formal}) \textbf{enumerate something} to name things on a list 1 by 1.} explicitly all the minor additions which are necessary for absolute completeness.

%------------------------------------------------------------------------------%

\section{Introduction}

\subsection{Finite, infinite, \& integral inequalities}

\subsection{Notations}

\subsection{Positive inequalities}

\subsection{Homogeneous inequalities}

\subsection{The axiomatic basis of algebraic inequalities}

\subsection{Comparable functions}

\subsection{Selection of proofs}

\subsection{Selection of subjects}

%------------------------------------------------------------------------------%

\section{Elementary Mean Values}

\subsection{Ordinary means}

\subsection{Weighted means}

\subsection{Limiting cases of $\mathfrak{M}_r(a)$}

\subsection{Cauchy's inequality}

\subsection{The theorem of the arithmetic \& geometric means}

\subsection{Other proofs of the theorem of the means}

\subsection{H\"older's inequality \& its extensions}

\subsection{General properties of the means $\mathfrak{M}_r(a)$}

\subsection{The sums $\mathfrak{G}_r(a)$}

\subsection{Minkowski's inequality}

\subsection{A companion to Minkowski's inequality}

\subsection{Illustrations \& applications of the fundamental inequalities}

\subsection{Inductive proofs of the fundamental inequalities}

\subsection{Elementary inequalities connected with Theorem 37}

\subsection{Elementary proof of Theorem 3}

\subsection{Tchebychef's inequality}

\subsection{Muirhead's theorem}

\subsection{Proof of Muirhead's theorem}

\subsection{An alternative theorem}

\subsection{Further theorems on symmetrical means}

\subsection{The elementary symmetric functions of $n$ positive numbers}

\subsection{A note on definite forms}

\subsection{A theorem concerning strictly positive forms}

\subsection{Miscellaneous theorems \& examples}

%------------------------------------------------------------------------------%

\section{Mean Values with an Arbitrary Function \& the Theory of Convex Functions}

\subsection{Definitions}

\subsection{Equivalent means}

\subsection{A characteristic property of the means $\mathfrak{M}_r$}

\subsection{Comparability}

\subsection{Convex functions}

\subsection{Continuous convex functions}

\subsection{An alternative definition}

\subsection{Equality in the fundamental inequalities}

\subsection{Restatements \& extensions of Theorem 85}

\subsection{Twice differentiable convex functions}

\subsection{Applications of the properties of twice differentiable convex functions}

\subsection{Convex functions of several variables}

\subsection{Generalizations of H\"older's inequality}

\subsection{Some theorems concerning monotonic functions}

\subsection{Sums with an arbitrary function: generalizations of Jensen's inequality}

\subsection{Generalizations of Minkowski's inequality}

\subsection{Comparison of sets}

\subsection{Further general properties of convex functions}

\subsection{Further properties of continuous convex functions}

\subsection{Discontinuous convex functions}

\subsection{Miscellaneous theorems \& examples}

%------------------------------------------------------------------------------%

\section{Various Applications of the Calculus}

\subsection{Introduction}

\subsection{Applications of the mean value theorem}

\subsection{Further applications of elementary differential calculus}

\subsection{Maxima \& minima of functions of 1 variable}

\subsection{Use of Taylor's series}

\subsection{Applications of the theory of maxima \& minima of functions of several variables}

\subsection{Comparison of series \& integrals}

\subsection{An inequality of W. H. Young}

%------------------------------------------------------------------------------%

\section{Infinite Series}

\subsection{Introduction}

\subsection{The means $\mathfrak{M}_r$}

\subsection{The generalization of Theorems 3 \& 9}

\subsection{H\"older's inequality \& its extensions}

\subsection{The means $\mathfrak{M}_r$ (\textit{cont}.)}

\subsection{The sums $\mathfrak{G}_r$}

\subsection{Minkowski's inequality}

\subsection{Tchebychef's inequality}

\subsection{A summary}

\subsection{Miscellaneous theorems \& examples}

%------------------------------------------------------------------------------%

\section{Integrals}

\subsection{Preliminary remarks on Lebesgue integrals}

\subsection{Remarks on null sets \& null functions}

\subsection{Further remarks concerning integration}

\subsection{Remarks on methods of proof}

\subsection{Further remarks on method: the inequality of Schwarz}

\subsection{Definition of the means $\mathfrak{M}_r(f)$ when $r\ne 0$}

\subsection{The geometric mean of a function}

\subsection{Further properties of the geometric mean}

\subsection{H\"older's inequality for integrals}

\subsection{General properties of the means $\mathfrak{M}_r(f)$}

\subsection{Convexity of $\log\mathfrak{M}_r^r$}

\subsection{Minkowski's inequality for integrals}

\subsection{Mean values depending on an arbitrary function}

\subsection{The definition of the Stieltjes integral}

\subsection{Special cases of the Stieltjes integral}

\subsection{Extensions of earlier theorems}

\subsection{The means $\mathfrak{M}_r(f;\phi)$}

\subsection{Distribution functions}

\subsection{Characterization of means values}

\subsection{Remarks on the characteristic properties}

\subsection{Completion of the proof of Theorem 215}

\subsection{Miscellaneous theorems \& examples}

%------------------------------------------------------------------------------%

\section{Some Applications of the Calculus of Variations}

\subsection{Some general remarks}

\subsection{Object of the present chapter}

\subsection{Example of an inequality corresponding to an unattained extremum}

\subsection{1st proof of Theorem 254}

\subsection{2nd proof of Theorem 254}

\subsection{Further examples illustrative of variational methods}

\subsection{Further examples: Wirtinger's inequality}

\subsection{An example involving 2nd derivatives}

\subsection{A simpler theorem}

\subsection{Miscellaneous theorems \& examples}

%------------------------------------------------------------------------------%

\section{Some Theorems Concerning Bilinear \& Multilinear Forms}

\subsection{Introduction}

\subsection{An inequality for multilinear forms with positive variables \& coefficients}

\subsection{A theorem of W. H. Young}

\subsection{Generalizations \& analogues}

\subsection{Applications to Fourier series}

\subsection{The convexity theorem for positive multilinear forms}

\subsection{The convexity theorem for positive multilinear forms}

\subsection{General bilinear forms}

\subsection{Definition of a bounded bilinear form}

\subsection{Some properties of bounded forms in $[p,q]$}

\subsection{The Faltung of 2 forms in $[p,p']$}

\subsection{Some special theorems on forms in $[2,2]$}

\subsection{Application to Hilbert's forms}

\subsection{The convexity theorem for bilinear forms with complex variables \& coefficients}

\subsection{Further properties of a maximal set $(x,y)$}

\subsection{Proof of Theorem 295}

\subsection{Applications of the theorem of M. Riesz}

\subsection{Applications to Fourier series}

\subsection{Miscellaneous theorems \& examples}

%------------------------------------------------------------------------------%

\section{Hilbert's Inequality \& Its Analogues \& Extensions}

\subsection{Hilbert's double series theorem}

\subsection{A general class of bilinear forms}

\subsection{The corresponding theorem for integrals}

\subsection{Extensions of Theorems 318 \& 319}

\subsection{Best possible constants: proof of Theorem 317}

\subsection{Further remarks on Hilbert's theorems}

\subsection{Applications of Hilbert's theorems}

\subsection{Hardy's inequality}

\subsection{Further integral inequalities}

\subsection{Further theorems concerning series}

\subsection{Deduction of theorems on series from theorems on integrals}

\subsection{Carleman's inequality}

\subsection{Theorems with $0 < p < 1$}

\subsection{A theorem with 2 parameters $p$ \& $q$}

\subsection{Miscellaneous theorems \& examples}

%------------------------------------------------------------------------------%

\section{Rearrangements}

\subsection{Rearrangements of finite sets of variables}

\subsection{A theorem concerning the rearrangements of 2 sets}

\subsection{A 2nd proof of Theorem 368}

\subsection{Restatement of Theorem 368}

\subsection{Theorems concerning the rearrangements of 3 sets}

\subsection{Reduction of Theorem 373 to a special case}

\subsection{Completion of the proof}

\subsection{Another proof of Theorem 371}

\subsection{Rearrangements of any number of sets}

\subsection{A further theorem on the rearrangement of any number of sets}

\subsection{Applications}

\subsection{The rearrangement of a function}

\subsection{On the rearrangement of 2 functions}

\subsection{On the rearrangement of 3 functions}

\subsection{Completion of the proof of Theorem 379}

\subsection{An alternative proof}

\subsection{Applications}

\subsection{Another theorem concerning the rearrangement of a function in decreasing order}

\subsection{Proof of Theorem 384}

\subsection{Miscellaneous theorems \& examples}

%------------------------------------------------------------------------------%

\section{Appendices}

\subsection{Appendix I: On strictly positive forms}

\subsection{Appendix II: Thorin's proof \& extension of Theorem 295}

\subsection{Appendix III: On Hilbert's inequality}

%------------------------------------------------------------------------------%

%------------------------------------------------------------------------------%

\printbibliography[heading=bibintoc]
	
\end{document}