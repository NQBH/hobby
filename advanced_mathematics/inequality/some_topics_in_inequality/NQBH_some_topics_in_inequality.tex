\documentclass[oneside]{book}
\usepackage[backend=biber,natbib=true,style=authoryear]{biblatex}
\addbibresource{/home/hong/1_NQBH/reference/bib.bib}
\usepackage[vietnamese,english]{babel}
\usepackage{tocloft}
\renewcommand{\cftsecleader}{\cftdotfill{\cftdotsep}}
\usepackage[colorlinks=true,linkcolor=blue,urlcolor=red,citecolor=magenta]{hyperref}
\usepackage{amsmath,amssymb,amsthm,mathtools,float,graphicx}
\allowdisplaybreaks
\numberwithin{equation}{section}
\newtheorem{assumption}{Assumption}[chapter]
\newtheorem{conjecture}{Conjecture}[chapter]
\newtheorem{corollary}{Corollary}[chapter]
\newtheorem{definition}{Definition}[chapter]
\newtheorem{example}{Example}[chapter]
\newtheorem{lemma}{Lemma}[chapter]
\newtheorem{notation}{Notation}[chapter]
\newtheorem{principle}{Principle}[chapter]
\newtheorem{problem}{Problem}[chapter]
\newtheorem{proposition}{Proposition}[chapter]
\newtheorem{question}{Question}[chapter]
\newtheorem{remark}{Remark}[chapter]
\newtheorem{theorem}{Theorem}[chapter]
\usepackage[left=0.5in,right=0.5in,top=1.5cm,bottom=1.5cm]{geometry}
\usepackage{fancyhdr}
\pagestyle{fancy}
\fancyhf{}
\lhead{\small \textsc{Sect.} ~\thesection}
\rhead{\small \nouppercase{\leftmark}}
\renewcommand{\sectionmark}[1]{\markboth{#1}{}}
\cfoot{\thepage}
\def\labelitemii{$\circ$}

\title{Some Topics in Inequality}
\author{\selectlanguage{vietnamese} Nguyễn Quản Bá Hồng\footnote{Independent Researcher, Ben Tre City, Vietnam\\e-mail: \texttt{nguyenquanbahong@gmail.com}; website: \url{https://nqbh.github.io}.}}
\date{\today}

\begin{document}
\maketitle
\setcounter{secnumdepth}{4}
\setcounter{tocdepth}{4}
\tableofcontents

%------------------------------------------------------------------------------%

\chapter*{Preface}

\selectlanguage{vietnamese}
\section*{Notation}
\cite{Anh_Quang2022}.
\begin{itemize}
	\item $\sum_{\rm cyc}$: cyclic sum\texttt{/}tổng hoán vị.
	\item $\sum_{\rm sym}$: symmetric sum\texttt{/}tổng đối xứng.
	\item $\prod_{\rm cyc}$: cyclic product\texttt{/}tích hoán vị.
	\item $\prod_{\rm sym}$: symmetric product\texttt{/}tích đối xứng.
\end{itemize}
MATLAB, Maple, CAS

%------------------------------------------------------------------------------%

\selectlanguage{english}
\part{Inequalities in Elementary Mathematics}

\chapter{Elements of Inequality}

%------------------------------------------------------------------------------%

\chapter{2-Variables Inequality}

%------------------------------------------------------------------------------%

\chapter{3-Variables Inequality}

%------------------------------------------------------------------------------%

\chapter{Multivariate Inequality}

\selectlanguage{vietnamese}
\section{A General Framework}
Cho $m,n\in\mathbb{N}^\star$, xét các đại lượng $f_i(a_1,\ldots,a_n)$, $i = 1,\ldots,m$ sao cho $f_1(a_1,\ldots,a_n)\le f_2(a_1,\ldots,a_n)\le\cdots f_m(a_1,\ldots,a_n)$, hay có thể viết gọn hơn là $f_i(a_1,\ldots,a_n)\le f_j(a_1,\ldots,a_n)\le$, $\forall i,j\in\{1,\ldots,m\}$, $i < j$, hay tương đương $f_i(a_1,\ldots,a_n)\le f_{i+1}(a_1,\ldots,a_n)$, $\forall i = 1,\ldots,m - 1$. Xét các đánh giá có dạng, với $i_0\in\{1,\ldots,m\}$:
\begin{align}
	f_{i_0}(a_1,\ldots,a_n)\le\sum_{i=1,\,i\ne i_0}^m \lambda_if(a_1,\ldots,a_n),
\end{align}
hoặc phân hoạch tập chỉ số $\{1,\ldots,m\}$ thành các tập con $I_1,I_2$ \& xét các đánh giá có dạng:
\begin{align}
	\sum_{i\in I_1} f_{i_0}(a_1,\ldots,a_n)\le\sum_{i\in I_2} \lambda_if(a_1,\ldots,a_n),
\end{align}
Với $f_i$ là các đại lượng trung bình cộng, trung bình nhân, trung bình tổng lũy thừa, i.e., $\sum_{i=1}^n a_i$, $\prod_{i=1}^n a_i$, $\sqrt[k]{\frac{1}{n}\sum_{i=1}^n a_i^k}$.

%------------------------------------------------------------------------------%

\part{Inequalities in Advanced Mathematics}

%------------------------------------------------------------------------------%


\printbibliography[heading=bibintoc]
	
\end{document}