\documentclass[oneside]{book}
\usepackage[backend=biber,natbib=true,style=authoryear]{biblatex}
\addbibresource{/home/hong/1_NQBH/reference/bib.bib}
\usepackage[vietnamese,english]{babel}
\usepackage{tocloft}
\renewcommand{\cftsecleader}{\cftdotfill{\cftdotsep}}
\usepackage[colorlinks=true,linkcolor=blue,urlcolor=red,citecolor=magenta]{hyperref}
\usepackage{amsmath,amssymb,amsthm,mathtools,float,graphicx}
\allowdisplaybreaks
\numberwithin{equation}{section}
\newtheorem{assumption}{Assumption}[chapter]
\newtheorem{conjecture}{Conjecture}[chapter]
\newtheorem{corollary}{Corollary}[chapter]
\newtheorem{definition}{Definition}[chapter]
\newtheorem{example}{Example}[chapter]
\newtheorem{lemma}{Lemma}[chapter]
\newtheorem{notation}{Notation}[chapter]
\newtheorem{principle}{Principle}[chapter]
\newtheorem{problem}{Problem}[chapter]
\newtheorem{proposition}{Proposition}[chapter]
\newtheorem{question}{Question}[chapter]
\newtheorem{remark}{Remark}[chapter]
\newtheorem{theorem}{Theorem}[chapter]
\usepackage[left=0.5in,right=0.5in,top=1.5cm,bottom=1.5cm]{geometry}
\usepackage{fancyhdr}
\pagestyle{fancy}
\fancyhf{}
\lhead{\small \textsc{Sect.} ~\thesection}
\rhead{\small \nouppercase{\leftmark}}
\renewcommand{\sectionmark}[1]{\markboth{#1}{}}
\cfoot{\thepage}
\def\labelitemii{$\circ$}

\title{Mathematical Analysis}
\author{\selectlanguage{vietnamese} Nguyễn Quản Bá Hồng\footnote{Independent Researcher, Ben Tre City, Vietnam\\e-mail: \texttt{nguyenquanbahong@gmail.com}}}
\date{\today}

\begin{document}
\maketitle
\setcounter{secnumdepth}{4}
\setcounter{tocdepth}{4}
\tableofcontents

%------------------------------------------------------------------------------%

%------------------------------------------------------------------------------%

\chapter{Wikipedia's}

\section{\href{https://en.wikipedia.org/wiki/Maxima_and_minima}{Wikipedia\texttt{/}Maxima \& Minima}}
\textsf{Fig. Local \& global maxima \& minima for $\frac{\cos(3\pi x)}{x}$, $0.1\le x\le 1.1$.}

``In \href{https://en.wikipedia.org/wiki/Mathematical_analysis}{mathematical analysis}, the \textit{maxima} \& \textit{minima} (the respective plurals of \textit{maximum} \& \textit{minimum}) of a \href{https://en.wikipedia.org/wiki/Function_(mathematics)}{function}, known collectively as \textit{extrema} (the plural of \textit{extremum}), are the largest \& smallest value of the function, either within a given \href{https://en.wikipedia.org/wiki/Interval_(mathematics)}{range} (the \textit{local} or \textit{relative} extrema), or on the entire \href{https://en.wikipedia.org/wiki/Domain_of_a_function}{domain} (the \textit{global} or \textit{absolute} extrema). \href{https://en.wikipedia.org/wiki/Pierre_de_Fermat}{Pierre de Fermat} was 1 of the 1st mathematicians to propose a general technique, \href{https://en.wikipedia.org/wiki/Adequality}{adequality} for finding the maxima \& minima of functions.

As defined in \href{https://en.wikipedia.org/wiki/Set_theory}{set theory}, the maximum \& minimum of a \href{https://en.wikipedia.org/wiki/Set_(mathematics)}{set} are the \href{https://en.wikipedia.org/wiki/Greatest_and_least_elements}{greatest \& least elements} in the set, respectively. Unbounded \href{https://en.wikipedia.org/wiki/Infinite_set}{infinite sets}, such as the set of \href{https://en.wikipedia.org/wiki/Real_number}{real numbers}, have no minimum or maximum.'' -- \href{https://en.wikipedia.org/wiki/Maxima_and_minima}{Wikipedia\texttt{/}maxima \& minima}

\subsection{Definition}
``A real-valued \href{https://en.wikipedia.org/wiki/Function_(mathematics)}{function} $f$ defined on a \href{https://en.wikipedia.org/wiki/Domain_of_a_function}{domain} $X$ has a \textit{global} (or \textit{absolute}) \textit{maximum point} at $x^\star$, if $f(x^\star)\ge f(x)$ for all $x\in X$. Similarly, the function has a \textit{global} (or \textit{absolute}) \textit{minimum point} at $x^\star$, if $f(x^\star)\le f(x)$ for all $x\in X$. The value of the function at a maximum point is called the \textit{maximum value} of the function, denoted $\max(f(x))$, \& the value of the function at a minimum point is called the \textit{minimum value} of the function. Symbolically, this can be written as follows:
\begin{quotation}
	$x_0\in X$ is a global maximum point of function $f:X\to\mathbb{R}$, if $(\forall x\in X)$, $f(x_0)\ge f(x)$.
\end{quotation}
The definition of global minimum point also proceeds similarly.

If the domain $X$ is a \href{https://en.wikipedia.org/wiki/Metric_space}{metric space}, then $f$ is said to have a \textit{local} (or \textit{relative}) \textit{maximum point} at the point $x^\star$, if there exists some $\varepsilon > 0$ s.t. $f(x^\star)\ge f(x)$ for all $x\in X$ within distance $\varepsilon$ of $x^\star$. Similarly, the function has a \textit{local minimum point} at $x^\star$, if $f(x^\star)\le f(x)$ for all $x\in X$ within distance $\varepsilon$ of $x^\star$. A similar definition can be used when $X$ is a \href{https://en.wikipedia.org/wiki/Topological_space}{topological space}, since the definition just given can be rephrased in terms of neighborhoods. Mathematically, the given definition is written as follows:
\begin{quotation}
	Let $(X,d_X)$ be a metric space \& function $f:X\to\mathbb{R}$. Then $x_0\in X$ is a local maximum point of function $f$ if $(\exists\varepsilon > 0)$ s.t. $(\forall x\in X)$, $d_X(x,x_0) < \varepsilon\Rightarrow f(x_0)\ge f(x)$.
\end{quotation}
The definition of local minimum point can also proceed similarly.

In both the global \& local cases, the concept of a \textit{strict extremum} can be defined. E.g., $x^\star$ is a \textit{strict global maximum point} if for all $x\in X$ with $x\ne x^\star$, we have $f(x^\star) > f(x)$, \& $x^\star$ is a \textit{strict local maximum point} if there exists some $\varepsilon > 0$ s.t., for all $x\in X$ within distance $\varepsilon$ of $x^\star$ with $x\ne x^\star$, we have $f(x^\star) > f(x)$. Note that a point is a strict global maximum point iff it is the unique global maximum point, \& similarly for minimum points.

A \href{https://en.wikipedia.org/wiki/Continuous_function}{continuous} real-valued function with a \href{https://en.wikipedia.org/wiki/Compact_space}{compat} domain always has a maximum point \& a minimum point. An important example is a function whose domain is a closed \& bounded \href{https://en.wikipedia.org/wiki/Interval_(mathematics)}{interval} of \href{https://en.wikipedia.org/wiki/Real_number}{real numbers} (see the graph above).'' -- \href{https://en.wikipedia.org/wiki/Maxima_and_minima#Definition}{Wikipedia\texttt{/}maxima \& minima\texttt{/}definition}

\subsection{Search}
``Finding global maxima \& minima is the goal of \href{https://en.wikipedia.org/wiki/Mathematical_optimization}{mathematical optimization}. If a function is continuous on a closed interval, then by the \href{https://en.wikipedia.org/wiki/Extreme_value_theorem}{extreme value theorem}, global maxima \& minima exist. Furthermore, a global maximum (or minimum) either must be a local maximum (or minimum) in the interior of the domain, or must lie on the boundary of the domain. So a method of finding a global maximum (or minimum) is to look at all the local maxima (or minima) in the interior, \& also look at the maxima (or minima) of the points on the boundary, \& take the largest (or smallest) one.

For \href{https://en.wikipedia.org/wiki/Differentiable_functions}{differentiable functions}, \href{https://en.wikipedia.org/wiki/Fermat%27s_theorem_(stationary_points)}{Fermat's theorem} states that local extrema in the interior of a domain must occur at \href{https://en.wikipedia.org/wiki/Critical_point_(mathematics)}{critical points} (or points where the derivative equals zero). However, not all critical points are extrema. One can distinguish whether a critical point is a local maximum or local minimum by using the \href{https://en.wikipedia.org/wiki/First_derivative_test}{1st derivative test}, \href{https://en.wikipedia.org/wiki/Derivative_test#Second_derivative_test_(single_variable)}{2nd derivative test}, or \href{https://en.wikipedia.org/wiki/Higher-order_derivative_test}{higher-order derivative test}, given sufficient differentiability.

For any function that is defined \href{https://en.wikipedia.org/wiki/Piecewise}{piecewise}, one finds a maximum (or minimum) by finding the maximum (or minimum) of each piece separately, \& then seeing which one is largest (or smallest).'' -- \href{https://en.wikipedia.org/wiki/Maxima_and_minima#Search}{Wikipedia\texttt{/}maxima \& minima\texttt{/}search}

\subsection{Examples}
See \href{https://en.wikipedia.org/wiki/Maxima_and_minima#Examples}{Wikipedia\texttt{/}maxima \& minima\texttt{/}examples} for maxima \& minima for functions $x,x^3,\sqrt[x]{x},x^{-x},\frac{x^3}{3} - x,|x|,\cos x,2\cos x - x,\frac{\cos(3\pi x)}{x},x^3 + 3x^2 - 2x + 1$.

\subsection{Functions of $> 1$ variable}
``Main article: \href{https://en.wikipedia.org/wiki/Second_partial_derivative_test}{Wikipedia\texttt{/}2nd partial derivative test}. For functions of $> 1$ variable, similar conditions apply. E.g., in the (enlargeable) figure on the right, the necessary conditions for a \textit{local} maximum are similar to those of a function with only 1 variable. The 1st \href{https://en.wikipedia.org/wiki/Partial_derivatives}{partial derivatives} as to $z$ (the variable to be maximized) are zero at the maximum (the glowing dot on top in the figure). The 2nd partial derivatives are negative. These are only necessary, not sufficient, conditions for a local maximum, because of the possibility of a \href{https://en.wikipedia.org/wiki/Saddle_point}{saddle point}. For use of these conditions to solve for a maximum, the function $z$ must also be \href{https://en.wikipedia.org/wiki/Differentiable_function}{differentiable} throughout. The \href{https://en.wikipedia.org/wiki/Second_partial_derivative_test}{2nd partial derivative test} can help classify the point as a relative maximum or relative minimum. In contrast, there are substantial differences between functions of 1 variable \& functions of $> 1$ variable in the identification of global extrema. E.g., if a bounded differentiable function $f$ defined on a closed interval in the real line has a single critical point, which is a local minimum, then it is also a global minimum (use the \href{https://en.wikipedia.org/wiki/Intermediate_value_theorem}{intermediate value theorem} \& \href{https://en.wikipedia.org/wiki/Rolle%27s_theorem}{Rolle's theorem} to prove this by \href{https://en.wikipedia.org/wiki/Proof_by_contradiction}{contradiction}). In 2 \& more dimensions, this argument fails. This is illustrated by the function $f(x,y) = x^2 + y^2(1 - x)^3$, $x,y\in\mathbb{R}$, whose only critical point is at $(0,0)$, which is a local minimum with $f(0,0) = 0$. However, it cannot be a global one, because $f(2,3) = -5$.'' -- \href{https://en.wikipedia.org/wiki/Maxima_and_minima#Functions_of_more_than_one_variable}{Wikipedia\texttt{/}maxima \& minima\texttt{/}functions of more than 1 variable}

\textsf{Fig. \href{https://en.wikipedia.org/wiki/Peano_surface}{Peano surface}, a counterexample to some criteria of local maxima of the 19th century.} \textsf{Fig. The global maximum is the point at the top.} \textsf{Fig. Counterexample: The red dot shows a local minimum that is not a global minimum.}

\subsection{Maxima or minima of a functional}
``If the domain of a function for which an extremum is to be found consists itself of functions (i.e., if an extremum is to be found of a \href{https://en.wikipedia.org/wiki/Functional_(mathematics)}{functional}), then the extremum is found using the \href{https://en.wikipedia.org/wiki/Calculus_of_variations}{calculus of variations}.'' -- \href{https://en.wikipedia.org/wiki/Maxima_and_minima#Maxima_or_minima_of_a_functional}{Wikipedia\texttt{/}maxima \& minima\texttt{/}maxima or minima of a functional}

\subsection{In relation to sets}
``Maxima \& minima can also be defined for sets. In general, if an \href{https://en.wikipedia.org/wiki/Ordered_set}{ordered set} $S$ has a \href{https://en.wikipedia.org/wiki/Greatest_element}{greatest element} $m$, then $m$ is a \href{https://en.wikipedia.org/wiki/Maximal_element}{maximal element} of the set, also denoted as $\max(S)$. Furthermore, if $S$ is a subset of an ordered set $T$ \& $m$ is the greatest element of $S$ with (respect to order induced by $T$), then $m$ is a \href{https://en.wikipedia.org/wiki/Supremum}{least upper bound} of $S$ in $T$. Similar results hold for \href{https://en.wikipedia.org/wiki/Least_element}{least element}, \href{https://en.wikipedia.org/wiki/Minimal_element}{minimal element} \& \href{https://en.wikipedia.org/wiki/Infimum}{greatest lower bound}. The maximum \& minimum function for sets are used in \href{https://en.wikipedia.org/wiki/Database}{databases}, \& can be computed rapidly, since the maximum (or minimum) of a set can be computed from the maxima of a partition; formally, they are self-\href{https://en.wikipedia.org/wiki/Decomposable_aggregation_function}{decomposable aggregation functions}.

In the case of a general \href{https://en.wikipedia.org/wiki/Partial_order}{partial order}, the \textit{least element} (i.e., one that is smaller than all others) should not be confused with a \textit{minimal element} (nothing is smaller). Likewise, a \href{https://en.wikipedia.org/wiki/Greatest_element}{greatest element} of a \href{https://en.wikipedia.org/wiki/Partially_ordered_set}{partially ordered set} (poset) is an \href{https://en.wikipedia.org/wiki/Upper_bound}{upper bound} of the set which is contained within the set, whereas a \textit{maximal element} $m$ of a poset $A$ is an element of $A$ s.t. if $m\le b$ (for any $b$ in $A$), then $m = b$. Any least element or greatest element of a poset is unique, but a poset can have several minimal or maximal elements. If a poset has more than 1 maximal element, then these elements will not be mutually comparable.

In a \href{https://en.wikipedia.org/wiki/Total_order}{totally ordered} set, or \textit{chain}, all elements are mutually comparable, so such a set can have at most 1 minimal element \& at most 1 maximal element. Then, due to mutual comparability, the minimal element will also be the least element, \& the maximal element will also be the greatest element. Thus in a totally ordered set, we can simply use the terms \textit{minimum} \& \textit{maximum}.

If a chain is finite, then it will always have a maximum \& a minimum. If a chain is infinite, then it need not have a maximum or a minimum. E.g., the set of \href{https://en.wikipedia.org/wiki/Natural_number}{natural numbers} has no maximum, though it has a minimum. If an infinite chain $S$ is bounded, then the \href{https://en.wikipedia.org/wiki/Topological_closure}{closure} $\operatorname{Cl}(S)$ of the set occasionally has a minimum \& a maximum, in which case they are called the \textit{greatest lower bound} \& the \textit{least upper bound} of the set $S$, respectively.'' -- \href{https://en.wikipedia.org/wiki/Maxima_and_minima#In_relation_to_sets}{Wikipedia\texttt{/}maxima \& minima\texttt{/}in relation to sets}

%------------------------------------------------------------------------------%

\chapter{Walter Rudin. Principles of Mathematical Analysis}

%\selectlanguage{english}
%\begin{thebibliography}{99}
%	\bibitem[]{}
%\end{thebibliography}

%------------------------------------------------------------------------------%

\printbibliography[heading=bibintoc]
	
\end{document}