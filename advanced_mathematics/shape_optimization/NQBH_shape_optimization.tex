\documentclass[oneside]{book}
\usepackage[backend=biber,natbib=true,style=authoryear]{biblatex}
\addbibresource{/home/nqbh/reference/bib.bib}
\usepackage[vietnamese,english]{babel}
\usepackage{tocloft}
\renewcommand{\cftsecleader}{\cftdotfill{\cftdotsep}}
\usepackage[colorlinks=true,linkcolor=blue,urlcolor=red,citecolor=magenta]{hyperref}
\usepackage{amsmath,amssymb,amsthm,mathtools,float,graphicx}
\allowdisplaybreaks
\numberwithin{equation}{section}
\newtheorem{assumption}{Assumption}[chapter]
\newtheorem{conjecture}{Conjecture}[chapter]
\newtheorem{corollary}{Corollary}[chapter]
\newtheorem{definition}{Definition}[chapter]
\newtheorem{example}{Example}[chapter]
\newtheorem{lemma}{Lemma}[chapter]
\newtheorem{notation}{Notation}[chapter]
\newtheorem{principle}{Principle}[chapter]
\newtheorem{problem}{Problem}[chapter]
\newtheorem{proposition}{Proposition}[chapter]
\newtheorem{question}{Question}[chapter]
\newtheorem{remark}{Remark}[chapter]
\newtheorem{theorem}{Theorem}[chapter]
\usepackage[left=0.5in,right=0.5in,top=1.5cm,bottom=1.5cm]{geometry}
\usepackage{fancyhdr}
\pagestyle{fancy}
\fancyhf{}
\lhead{\small Sect.~\thesection}
\rhead{\small\nouppercase{\leftmark}}
\renewcommand{\sectionmark}[1]{\markboth{#1}{}}
\cfoot{\thepage}
\def\labelitemii{$\circ$}

\title{Shape Optimization}
\author{\selectlanguage{vietnamese} Nguyễn Quản Bá Hồng\footnote{Independent Researcher, Ben Tre City, Vietnam\\e-mail: \texttt{nguyenquanbahong@gmail.com}; website: \url{https://nqbh.github.io}.}}
\date{\today}

\begin{document}
\maketitle
\setcounter{secnumdepth}{4}
\setcounter{tocdepth}{4}
\tableofcontents

%------------------------------------------------------------------------------%

\chapter{\cite{Azegami2020}. Hideyuki Azegami. Shape Optimization Problems}
``3D modeling \& numerical analysis utilizing computers are conducted on a day-to-day basis in product design, \& there is a growing interest in the optimization of design using the results of numerical analysis. Optimization problems in which geometrical parameters of a 3D model defined on a computer are taken to be the design variables are called \textit{parametric shape optimization problems}. Software for solving such problems is constructed based on experimental design methods or mathematical programming. However, when increasing the number of design variables in order to increase the degrees of freedom, the solution rapidly becomes more difficult to find.

Conversely, problems in which no geometrical parameters are used \& optimum shapes are obtained from an arbitrary form are called \textit{nonparametric shape optimization problems}. In particular, problems in which the optimum shape is sought by introducing holes are called \textit{topology optimization problems}. Moreover, a problem in which the optimum shape is obtained through domain variations is referred to as a shape optimization problem of domain variation type or a shape optimization problem in a restrictive sense. Numerical methods have been developed to solve these problems \& are being used to seek practical optimum shapes.

\textsf{Fig. 1: Rigidity maximization of a heel counter (provided by ASICS Corporation). (a) External force \& fixed sole. (b) Mises stress initial model. (c) Optimized density (inside). (d) Optimized density (outside).} shows a numerical result for a topology optimization problem in maximizing the rigidity of a heel counter. Chap. 8 explains in detail how to choose the design variables. The external force estimated by the experiment is assumed to be known, \& the work done by it, which is defined as mean compliance in the body of this book, is chosen as the objective function. The condition that mass does not exceed a prescribed value is posed as a constraint condition. \textsf{Fig. 2: Lightening of an aluminum wheel (provided by Quint Corporation). (a) Initial shape. (b) Optimized shape.} shows the result of a numerical analysis w.r.t. a shape optimization problem of domain variation type aiming to decrease the weight of an aluminum wheel. Chap. 9 explains how the design variables should be selected in this case, where volume is selected as the objective function. A constraint is imposed on the Kreisselmeier--Steinhauser function, which expresses the maximum value of Mises stress in an integral form so that it does not exceed its initial value. The analysis also includes a constraint on the shape variations which take into account the symmetry \& manufacturing requirements of the model.

The fundamental principles of nonparametric shape optimization programs used to obtain these results are also based on mathematical programming. However, from the fact that the design variables are functions expressing densities or domain variations, there are issues that cannot be dealt with by finite-dimensional vector spaces, which are the platform for mathematical programming. In this regard, it is possible to drop the nonparametric shape optimization programs into parametric optimization problems in finite-dimensional vector spaces by discretizing continua using a procedure such as FEM. However, when using these methods, one faces a new problem that there is an insufficient smoothness of the function used in updating the density or domain variation. This problem emerges as \fbox{numerical instability phenomena} when conducting numerical analysis. In order to solve this problem, there is a need to think of solutions based on theories capturing shape optimization problems as function optimization problems. The numerical techniques described above are based on such theories. However, there are no books explaining such theories from their foundations.

This book explains the formulation \& solution of shape optimization problems for continua such as elastic bodies \& flow fields in detail from the basics, bearing in mind readers with an engineering background. A continuum refers to a domain over which a BVP of a PDE is defined. W.r.t. the PDE, considering a static elastic body or a steady flow field, elliptic PDEs are assumed. The theories shown in this book, however, can be applied to time-dependent IBVPs related to hyperbolic or parabolic PDEs, as well as to nonlinear problems. These results will be introduced on another occasion. Hence, the shape optimization problems dealt with in this book are described as follows. 1st describe a BVP of an elliptic PDE as a state determination problem \& then define the state determination problem so that when a function of the design variable expressing the density or domain variation is given, it has a unique solution. Using the design variable \& its solution, we define several cost functions by boundary integrals or domain integrals. Among these cost functions, one is set as the objective function \& the remaining as constraint functions in order to construct an optimum design problem. In this way, throughout this book, shape optimization problems will be constructed in the framework of function optimization problems. Their solutions will also be considered as solutions to the function optimization problems.

Based on this sort of conception, this book uses a structure as described below.
\begin{itemize}
	\item Chap. 1 examines a simple optimization design problem of a 1D continuum \& looks at the process until optimum conditions are obtained. In this chapter, the \textit{Lagrange multiplier method} (\textit{adjoint variable method}) is used without proof.
	\item After understanding the usage of the Lagrange multiplier method, optimization theories will be studied from the basics in Chap. 2. It is desired to have an understanding of the principle of the Lagrange multiplier method in this chapter.
	\item After mastering optimization theories, in Chap. 3, algorithms for obtaining the optimal solutions will be considered. The theories \& algorithms shown here are the fundamentals of the academic field referred to as mathematical programming. However, these are applicable to optimization problems defined on finite-dimensional vector spaces. In order to tackle optimization problems defined on functional spaces constructed from sets of functions, which is the aim of this book, functions need to be treated as vectors. A theory systematizing its use is functional analysis.
	\item In Chap. 4, the \textit{variational principle} in mechanics is used to exemplify the basic thinking \& results regarding the functional analysis.
	\item After such preparations, Chap. 5 will look at BVPs of PDEs which are the platform of this book.
	\item Chap. 6 will look at methods for performing numerical analyses of such problems. Numerical analysis is 1 of the academic fields which continues to develop, even now, \& has a variety of perspectives for study. In this book, because theories such as \textit{error estimation} are established, FEMs using the Galerkin method as a guiding principle will be looked at. Here, remarkable results from the functional analysis will be used to show the unique existence of solutions \& in error estimations.
	\item With these results in mind, in Chap. 7, an optimum design problem with a level of abstraction, which can be used in the unification of shape optimization problems considered in this book, is defined, \& its solution \& algorithms are considered. The solutions \& algorithms have the same framework as those shown in Chap. 3. Here, however, vector spaces w.r.t. design variables are replaced by function spaces.
	\item In Chaps. 8--9, the abstract optimum design problem is translated into topology optimization problems of density variation type \& shape optimization problems of domain variation type, respectively. In both chapters, the details of the theory are shown using the Poisson problem, for simplicity. We finally consider shape optimization problems for a linear elastic body \& a Stokes flow field, which are important in engineering, where a method to obtain the derivatives of cost functions will be looked at in detail.
\end{itemize}
Based on the above synopsis, this book will build up theorems using results in mathematics. Therefore, we will be summarizing the key points in mathematical definitions \& theorems. This method of expression is advanced by the fact that the provable facts are clearly shown. If something that we want to investigate in the future is contained in the framework of mathematics, setting up a theory using theorems prepared by great mathematicians is thought to be an extremely effective method. Conversely, mathematics attempts to heighten the level of abstractness in order to understand may things in a unified fashion. This characteristic is also the reason that it can baffle readers with an engineering background. Hence, an attempt has been made to provide explanations using examples from dynamics with the aim of accurately denoting the provable facts using definitions \& theorems. Proofs have been added for the basic theories.''  -- \cite[pp. v--viii]{Azegami2020}

%------------------------------------------------------------------------------%

\section{Basics of Optimal Design}

%------------------------------------------------------------------------------%

\section{Basics of Optimization Theory}

%------------------------------------------------------------------------------%

\section{Basics of Mathematical Programming}

%------------------------------------------------------------------------------%

\section{Basics of Variational Principles \& Fundamental Analysis}

%------------------------------------------------------------------------------%

\section{BVPs of PDEs}

%------------------------------------------------------------------------------%

\section{Fundamentals of Numerical Analysis}

%------------------------------------------------------------------------------%

\section{Abstract Optimum Design Problem}

%------------------------------------------------------------------------------%

\section{Topology Optimization Problems of Density Variation Type}

%------------------------------------------------------------------------------%

\section{Shape Optimization Problems of Domain Variation Type}
``In Chap. 8 we looked at problems for obtaining the optimal topologies of continua with the densities of continua set to be the design variable. In this chapter, we shall look at the type of shape optimization problems in which the boundary of a continuum varies.''

``1st, let us take an abridged look at the history of research relating to a shape optimization problem of domain variation type. This type of shape optimization problem is also referred to as a domain optimization problem \& has been studied since the early 20th century. E.g., among the vast works of Hadamard, there is a description relating to a problem seeking the boundary shape of a thin membrane s.t. the fundamental vibration frequency is maximized. In this description, a notion equivalent to a Fr\'echet derivative of the fundamental frequency when a boundary is moved in the outward normal direction is presented [60], \cite{Sokolowski_Zolesio1992}. Even after that, Fr\'echet derivatives w.r.t. shape variations of domain variation type have been referred to as \textit{shape derivatives}, \& many researchers have announced research results relating to it. To add background to this research, there are works relating to optimal control theory assuming a function as a control variable by mathematicians lead by Lions \cite{Lions1971}.

In this way, theories relating to the calculation methods of shape derivatives have been developed consistently, but research relating to moving the shapes using shape derivatives has not always obtained favorable results. In reality, it is known that if the node coordinates on a boundary of a finite element model are chosen to be the design variable, \& the Fr\'echet derivatives w.r.t. the variation of the design variable are evaluated in order to move the nodes, a numerically unstable phenomenon in which the boundary becomes rippled such as shown in Fig. 9.1a appears [73]. \textsf{Fig. 9.1. Numerical examples w.r.t. the shape optimization problem of a linear elastic body (provided by Quint Corporation). (a) Rippling shape. (b) Optimal shape by $H^1$ gradient method.} Fig. 9.1a shows the result of a numerical analysis w.r.t. a mean compliance minimization problem (Problem 9.12.2) of a 3D linear elastic body. The boundary condition in the state determination problem constrains the displacement on the back edge, while a uniform downward facing nodal force (external force) on the horizontal central line of the front edge was assumed. The boundary condition in the shape variation problem restrains the variation in the normal direction on the front\texttt{/}back \& left\texttt{/}right edges, \& the variation on the horizontal central line on the front\texttt{/}back edge. Numerical analysis of a state determination problem uses the 1st-order finite elements. The calculation method of shape derivatives uses the formula of boundary integration form as shown later.

-- \cite[p. 427]{Azegami2020}

%------------------------------------------------------------------------------%

\section{Appendices}

%------------------------------------------------------------------------------%

\printbibliography[heading=bibintoc]
	
\end{document}