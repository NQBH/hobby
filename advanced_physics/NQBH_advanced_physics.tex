\documentclass[oneside]{book}
\usepackage[backend=biber,natbib=true,style=authoryear]{biblatex}
\addbibresource{/home/hong/1_NQBH/reference/bib.bib}
\usepackage[vietnamese,english]{babel}
\usepackage{tocloft}
\renewcommand{\cftsecleader}{\cftdotfill{\cftdotsep}}
\usepackage[colorlinks=true,linkcolor=blue,urlcolor=red,citecolor=magenta]{hyperref}
\usepackage{amsmath,amssymb,amsthm,mathtools,float,graphicx}
\allowdisplaybreaks
\numberwithin{equation}{section}
\newtheorem{assumption}{Assumption}[chapter]
\newtheorem{conjecture}{Conjecture}[chapter]
\newtheorem{corollary}{Corollary}[chapter]
\newtheorem{definition}{Definition}[chapter]
\newtheorem{example}{Example}[chapter]
\newtheorem{lemma}{Lemma}[chapter]
\newtheorem{notation}{Notation}[chapter]
\newtheorem{principle}{Principle}[chapter]
\newtheorem{problem}{Problem}[chapter]
\newtheorem{proposition}{Proposition}[chapter]
\newtheorem{question}{Question}[chapter]
\newtheorem{remark}{Remark}[chapter]
\newtheorem{theorem}{Theorem}[chapter]
\usepackage[left=0.5in,right=0.5in,top=1.5cm,bottom=1.5cm]{geometry}
\usepackage{fancyhdr}
\pagestyle{fancy}
\fancyhf{}
\lhead{\small \textsc{Sect.} ~\thesection}
\rhead{\small \nouppercase{\leftmark}}
\renewcommand{\sectionmark}[1]{\markboth{#1}{}}
\cfoot{\thepage}
\def\labelitemii{$\circ$}

\title{Advanced Physics}
\author{\selectlanguage{vietnamese} Nguyễn Quản Bá Hồng\footnote{Independent Researcher, Ben Tre City, Vietnam\\e-mail: \texttt{nguyenquanbahong@gmail.com}}}
\date{\today}

\begin{document}
\maketitle
\setcounter{secnumdepth}{4}
\setcounter{tocdepth}{4}
\tableofcontents

%------------------------------------------------------------------------------%

\chapter{Wikipedia's}

\section{\href{https://en.wikipedia.org/wiki/Stationary-action_principle}{Wikipedia\texttt{/}Stationary-Action Principle}}
``The \textit{stationary-action principle} -- also known as the \textit{principle of least action} -- is a \href{https://en.wikipedia.org/wiki/Variational_principle}{variational principle} that, when applied to the \href{https://en.wikipedia.org/wiki/Action_(physics)}{\textit{action}} of a \href{https://en.wikipedia.org/wiki/Mechanics}{mechanical} system, yields the \href{https://en.wikipedia.org/wiki/Equations_of_motion}{equations of motion} for that system. The principle states that the trajectories (i.e., the solutions of the equations of motions) are \href{https://en.wikipedia.org/wiki/Stationary_point}{\textit{stationary points}} of the system's \textit{action functional}. The term ``least action'' is a historical misnomer since the principle has no minimality requirement: the value of the action functional need not be minimal (even locally) on the trajectories. Least action refers to the absolute value of the action functional being minimized.

The principle can be used to derive \href{https://en.wikipedia.org/wiki/Newtonian_mechanics}{Newtonian}, \href{https://en.wikipedia.org/wiki/Lagrangian_mechanics}{Lagrangian}, \& \href{https://en.wikipedia.org/wiki/Hamiltonian_mechanics}{Hamiltonian} \href{https://en.wikipedia.org/wiki/Equations_of_motion}{equations of motion}, \& even \href{https://en.wikipedia.org/wiki/General_relativity}{general relativity} (see \href{https://en.wikipedia.org/wiki/Einstein%E2%80%93Hilbert_action}{Einstein--Hilbert action}). In relativity, a different action must be minimized or maximized.

The classical mechanics \& electromagnetic expressions are a consequence of quantum mechanics. The stationary action method helped in the development of quantum mechanics. In 1933, the physicist \href{https://en.wikipedia.org/wiki/Paul_Dirac}{Paul Dirac} demonstrated how this principle can be used in quantum calculations by discerning the \href{https://en.wikipedia.org/wiki/Path_integral_formulation#Quantum_action_principle}{quantum mechanical underpinning} of the principle in the \href{https://en.wikipedia.org/wiki/Interference_(wave_propagation)#Quantum_interference}{quantum interference} of amplitudes. Subsequently \href{https://en.wikipedia.org/wiki/Julian_Schwinger}{Julian Schwinger} \& \href{https://en.wikipedia.org/wiki/Richard_Feynman}{Richard Feynman} independently applied this principle in quantum electrodynamics.

The principle remains central in \href{https://en.wikipedia.org/wiki/Modern_physics}{modern physics} \& mathematics, being applied in \href{https://en.wikipedia.org/wiki/Thermodynamics}{thermodynamics}, \href{https://en.wikipedia.org/wiki/Fluid_mechanics}{fluid mechanics}, the \href{https://en.wikipedia.org/wiki/Theory_of_relativity}{theory of relativity}, \href{https://en.wikipedia.org/wiki/Quantum_mechanics}{quantum mechanics}, \href{https://en.wikipedia.org/wiki/Particle_physics}{particle physics}, \& \href{https://en.wikipedia.org/wiki/String_theory}{string theory} \& is a focus of modern mathematical investigation in \href{https://en.wikipedia.org/wiki/Morse_theory}{Morse theory}. \href{https://en.wikipedia.org/wiki/Maupertuis%27_principle}{Maupertuis' principle} \& \href{https://en.wikipedia.org/wiki/Hamilton%27s_principle}{Hamilton's principle} exemplify the principle of stationary action.

The action principle is preceded by earlier ideas in \href{https://en.wikipedia.org/wiki/Optics}{optics}. In \href{https://en.wikipedia.org/wiki/Ancient_Greece}{ancient Greece}, \href{https://en.wikipedia.org/wiki/Euclid}{Euclid} wrote in his \textit{Catoptrica} that, for the path of light reflecting from a mirror, the \href{https://en.wikipedia.org/wiki/Angle_of_incidence_(optics)}{angle of incidence} equals the \href{https://en.wikipedia.org/wiki/Angle_of_reflection}{angle of reflection}. \href{https://en.wikipedia.org/wiki/Hero_of_Alexandria}{Hero of Alexandria} later showed that this path was the shortest length \& least time.

Scholars often credit \href{Pierre Louis Maupertuis} for formulating the principle of least action because he wrote about it in 1744 \& 1746. However, \href{https://en.wikipedia.org/wiki/Leonhard_Euler}{Leonhard Euler} discussed the principle in 1744, \& evidence shows that \href{https://en.wikipedia.org/wiki/Gottfried_Leibniz}{Gottfried Leibniz} preceded both by 39 years.'' -- \href{https://en.wikipedia.org/wiki/Stationary-action_principle}{Wikipedia\texttt{/}stationary-action principle}

\subsection{General statement}
\textsf{Fig. As the system evolves, ${\bf q}$ traces a path through \href{https://en.wikipedia.org/wiki/Configuration_space_(physics)}{configuration space} (only some are shown). The path taken by the system (red) has a stationary action ($\delta S = 0$) under small changes in the configuration of the system ($\delta{\bf q}$).}

``The \href{https://en.wikipedia.org/wiki/Action_(physics)}{\textit{action}}, denoted $\mathcal{S}$, of a physical system is defined as the \href{https://en.wikipedia.org/wiki/Integral_(mathematics)}{integral} of the \href{https://en.wikipedia.org/wiki/Lagrangian_mechanics}{Lagrangian} $L$ between 2 instants of \href{https://en.wikipedia.org/wiki/Time_in_physics}{time} $t_1$ \& $t_2$ -- technically a \href{https://en.wikipedia.org/wiki/Functional_(mathematics)}{functional} of the $N$ \href{https://en.wikipedia.org/wiki/Generalized_coordinates}{generalized coordinates} ${\bf q} = (q_1,\ldots,q_n)$ which are functions of time \& define the \href{https://en.wikipedia.org/wiki/Configuration_space_(physics)}{configuration} of the system:
\begin{align*}
	{\bf q}:\mathbb{R}&\to\mathbb{R}^N,\\
	\mathcal{S}[{\bf q},t_1,t_2] &= \int_{t_1}^{t_2} L({\bf q}(t),\dot{\bf q}(t),t)\,{\rm d}t,
\end{align*}
where the dot denotes the \href{https://en.wikipedia.org/wiki/Time_derivative}{time derivative}, \& $t$ is time. Mathematically the principle is $\delta\mathcal{S} = 0$, where $\delta$ (lowercase Greek \href{https://en.wikipedia.org/wiki/Delta_(letter)}{delta}) means a \textit{small} change. In words this reads:
\begin{quotation}
	\textit{The path taken by the system between times $t_1$ \& $t_2$ \& configurations $q_1$ \& $q_2$ is the one for which the \textbf{action} is \textbf{stationary (no change)} to \textbf{1st order}.}
\end{quotation}
Stationary action is not always a minimum, despite the historical name of least action. It is a minimum principle for sufficiently short, finite segments in the path.

In applications the statement \& definition of action are taken together: $\delta\int_{t_1}^{t_2} L({\bf q},\dot{\bf q},t)\,{\rm d}t = 0$. The action \& Lagrangian both contain the dynamics of the system for all times. The term ``path'' is simply refers to a curve traced out by the system in terms of the coordinates in the \href{https://en.wikipedia.org/wiki/Configuration_space_(physics)}{configuration space}, i.e., the curve ${\bf q}(t)$, parametrized by time (see also \href{https://en.wikipedia.org/wiki/Parametric_equation}{parametric equation} for this concept).'' -- \href{https://en.wikipedia.org/wiki/Stationary-action_principle#General_statement}{Wikipedia\texttt{/}stationary-action principle\texttt{/}general statement}

\subsection{Origins, statements, \& controversy}

\subsubsection{Fermat}
``Main article: \href{https://en.wikipedia.org/wiki/Fermat%27s_principle}{Wikipedia\texttt{/}Fermat's principle}. In the 1600s, \href{https://en.wikipedia.org/wiki/Pierre_de_Fermat}{Pierre de Fermat} postulated that \textit{``light travels between 2 given points along the path of shortest time,''} which is known as the \textit{principle of least time} or \href{https://en.wikipedia.org/wiki/Fermat%27s_principle}{Fermat's principle}.

\subsubsection{Maupertuis}
Main article: \href{https://en.wikipedia.org/wiki/Maupertuis_principle}{Maupertuis principle}. Credit for the formulation of the \textit{principle of least action} is commonly given to \href{https://en.wikipedia.org/wiki/Pierre_Louis_Maupertuis}{Pierre Louis Maupertuis}, who felt that ``Nature is thrifty in all its actions'', \& applied the principle broadly:
\begin{quotation}
	``The laws of movement \& of rest deduced from this principle being precisely the same as those observed in nature, we can admire the application of it to all phenomena. The movement of animals, the vegetative growth of plants $\ldots$ are only its consequences; \& the spectacle of the universe becomes so much the grander, so much more beautiful, the worthier of its Author, when one knows that a small number of laws, most wisely established, suffice for all movements.'' -- Pierre Louis Maupertuis
\end{quotation}
This notion of Maupertuis, although somewhat deterministic today, does capture much of the essence of mechanics.

In application to physics, Maupertuis suggested that the quantity to be minimized was the product of the duration (time) of movement within a system by the ``\href{https://en.wikipedia.org/wiki/Vis_viva}{vis viva}'', \fbox{\textbf{Maupertuis' principle}: $\delta\int 2T(t)\,{\rm d}t = 0$}, which is the integral of twice what we now call the \href{https://en.wikipedia.org/wiki/Kinetic_energy}{kinetic eneergy} $T$ of the system.

\subsubsection{Euler}
\href{https://en.wikipedia.org/wiki/Leonhard_Euler}{Leonhard Euler} gave a formulation of the action principle in 1744, in very recognizable terms, in the \textit{Additamentum 2} to his \textit{Methodus Inveniendi Lineas Curvas Maximi Minive Proprietate Gaudentes}. Beginning with the 2nd paragraph:
\begin{quotation}
	Let the mass of the projectile be $M$, \& let its speed be $v$ while being moved over an infinitesimal distance ${\rm d}s$. The body will have a momentum $Mv$ that, when multiplied by the distance ${\rm d}s$, will give $Mv{\rm d}s$, the momentum of the body integrated over the distance ${\rm d}s$. Now I assert that the curve thus described by the body to be the curve (from among all other curves connecting the same endpoints) that minimizes $\int Mv\,{\rm d}s$ or, provided that $M$ is constant along the path, $M\int v\,{\rm d}s$.'' -- Leonhard Euler
\end{quotation}
As Euler states, $\int Mv\,{\rm d}s$ is the integral of the momentum over distance traveled, which, in modern notation, equals the abbreviated or \href{https://en.wikipedia.org/wiki/Reduced_action}{reduced action} \fbox{\textbf{Euler's principle} $\delta\int p\,{\rm d}q = 0$.} Thus, Euler made an equivalent \& (apparently) independent statement of the variational principle in the same year as Maupertuis, albeit slightly later. Curiously, Euler did not claim any priority, as the following episode shows.

\subsubsection{Disputed priority}
Maupertuis' priority was disputed in 1751 by the mathematician \href{https://en.wikipedia.org/wiki/Samuel_K%C3%B6nig}{Samuel K\"onig}, who claimed that it had been invented by \href{https://en.wikipedia.org/wiki/Gottfried_Leibniz}{Gottfried Leibniz} in 1707. Although similar to many of Leibniz's arguments, the principle itself has not been documented in Leibniz's works. K\"onig himself showed a \textit{copy} of a 1707 letter from Leibniz to \href{https://en.wikipedia.org/wiki/Jacob_Hermann_(mathematician)}{Jacob Hermann} with the principle, but the \textit{original} letter has been lost. In contentious proceedings, K\"onig was accused of forgery\footnote{\textbf{forgery} [n] (plural \textbf{forgeries}) \textbf{1.} [uncountable] the crime of copying money, documents, etc. in order to cheat people; \textbf{2.} [countable] something, e.g. a document, piece of paper money, etc., that has been copied in order to cheat people, \textsc{synonym}: \textbf{fake}.}, \& even the \href{https://en.wikipedia.org/wiki/Frederick_the_Great}{King of Prussia} entered the debate, defending Maupertuis (the head of his Academy), while \href{https://en.wikipedia.org/wiki/Voltaire}{Voltaire} defended K\"onig.

Euler, rather than claiming priority, was a staunch defender of Maupertuis, \& Euler himself prosecuted K\"onig for forgery before the Berlin Academy on Apr 13, 1752. The claims of forgery were re-examined 150 years later, \& archival work by C.I. Gerhardt in 1898 \& W. Kabitz in 1913 uncovered other copies of the letter, \& 3 others cited by K\"onig, in the \href{https://en.wikipedia.org/wiki/Bernoulli_family}{Bernoulli} archives.'' -- \href{https://en.wikipedia.org/wiki/Stationary-action_principle#Origins,_statements,_and_controversy}{Wikipedia\texttt{/}stationary-action principle\texttt{/}origins, statements, \& controversy}

\subsection{Further development}
``Euler continued to write on the topic; in his \textit{R\'eflexions sur quelques loix g\'en\'erales de la nature} (1748), he called action ``effort''. His expression corresponds to modern \href{https://en.wikipedia.org/wiki/Potential_energy}{potential energy}, \& his statement of least action says that the total potential energy of a system of bodies at rest is minimized, a principle of modern statics.

\subsubsection{Lagrange \& Hamilton}
Main article: \href{https://en.wikipedia.org/wiki/Hamilton%27s_principle}{Wikipedia\texttt{/}Hamilton's principle}. Much of the calculus of variations was stated by \href{https://en.wikipedia.org/wiki/Joseph-Louis_Lagrange}{Joseph-Louis Lagrange} in 1760 \& he proceeded to apply this to problems in dynamics. In \textit{M\'ecanique analytique} (1788) Lagrange derived the general \href{https://en.wikipedia.org/wiki/Lagrangian_equations_of_motion}{equations of motion} of a mechanical body. \href{https://en.wikipedia.org/wiki/William_Rowan_Hamilton}{William Rowan Hamilton} in 1834 \& 1835 applied the variational principle to the classical \href{https://en.wikipedia.org/wiki/Lagrangian_mechanics}{Lagrangian} \href{https://en.wikipedia.org/wiki/Function_(mathematics)}{function} $L = T - V$ to obtain the \href{https://en.wikipedia.org/wiki/Euler%E2%80%93Lagrange_equations}{Euler--Lagrange equations} in their present form.

\subsubsection{Jacobi, Morse, \& Caratheodory}
In 1842, \href{https://en.wikipedia.org/wiki/Carl_Gustav_Jacobi}{Carl Gustav Jacobi} tackled the problem of whether the variational principle always found minima as opposed to other \href{https://en.wikipedia.org/wiki/Stationary_points}{stationary points} (maxima or stationary \href{https://en.wikipedia.org/wiki/Saddle_points}{saddle points}); most of his work focused on \href{https://en.wikipedia.org/wiki/Geodesics}{geodesics} on 2D surfaces. The 1st clear general statements were given by \href{https://en.wikipedia.org/wiki/Marston_Morse}{Marston Morse} in the 1920s \& 1930s, leading to what is now known as \href{https://en.wikipedia.org/wiki/Morse_theory}{Morse theory}. E.g., Morse showed that the number of \href{https://en.wikipedia.org/wiki/Conjugate_points}{conjugate points} in a trajectory equaled the number of negative eigenvalues in the 2nd variation of the Lagrangian. A particularly elegant derivation of the Euler--Lagrange equation was formulated by \href{https://en.wikipedia.org/wiki/Constantin_Caratheodory}{Constantin Caratheodory} \& published by him in 1935.

\subsubsection{Gauss \& Hertz}
Other extremal principles of \href{https://en.wikipedia.org/wiki/Classical_mechanics}{classical mechanics} have been formulated, e.g. \href{https://en.wikipedia.org/wiki/Gauss%27s_principle_of_least_constraint}{Gauss's principle of least constraint} \& its corollary, \href{https://en.wikipedia.org/wiki/Hertz%27s_principle_of_least_curvature}{Hertz's principle of least curvature}.'' -- \href{https://en.wikipedia.org/wiki/Stationary-action_principle#Further_development}{Wikipedia\texttt{/}stationary-action principle\texttt{/}further development}

\subsection{Disputes about possible teleological aspects}
``The mathematical equivalence of the \href{https://en.wikipedia.org/wiki/Differential_equation}{differential} \href{https://en.wikipedia.org/wiki/Equations_of_motion}{equations of motion} \& their \href{https://en.wikipedia.org/wiki/Integral_equation}{integral} counterpart has important philosophical implications. The differential equations are statements about quantities localized to a single point in space or single moment of time. E.g., \href{https://en.wikipedia.org/wiki/Newton%27s_laws_of_motion}{Newton's 2nd law} ${\bf F} = m{\bf a}$ states that the \textit{instantaneous} force ${\bf F}$ applied to a mass $m$ produces an acceleration ${\bf a}$ at the same \textit{instant}. By contrast, the action principle is not localized to a point; rather, it involves integrals over an interval of time \& (for fields) an extended region of space. Moreover, in the usual formulation of \href{https://en.wikipedia.org/wiki/Classical_physics}{classical} action principles, the initial \& final states of the system are fixed, e.g.,
\begin{quotation}
	\textit{Given that the particle begins at position $x_1$ at time $t_1$ \& ends at position $x_2$ at time $t_2$, the physical trajectory that connects these 2 endpoints is an \href{https://en.wikipedia.org/wiki/Extremum}{extremum} of the action integral}.
\end{quotation}
In particular, the fixing of the \textit{final} state has been interpreted as giving the action principle a \href{https://en.wikipedia.org/wiki/Teleology}{teleological character} which has been controversial historically. However, according to W. Yourgrau \& S. Mandelstam, \textit{the teleological approach $\ldots$ presupposes that the variational principles themselves have mathematical characteristics which they \emph{de facto} do not possess}. In addition, some critics maintain this apparent \href{https://en.wikipedia.org/wiki/Teleology}{teleology} occurs because of the way in which the question was asked. By specifying some but not all aspects of both the initial \& final conditions (the positions but not the velocities) we are making some inferences about the initial conditions from the final conditions, \& it is this ``backward'' inference that can be seen as a teleological explanation. Teleology can also be overcome if we consider the classical description as a limiting case of the \href{https://en.wikipedia.org/wiki/Quantum_mechanics}{quantum} formalism of \href{https://en.wikipedia.org/wiki/Path_integral_formulation}{path integration}, in which stationary paths are obtained as a result of interference of amplitudes along all possible paths.

The short story \href{https://en.wikipedia.org/wiki/Story_of_Your_Life}{Story of Your Life} by the speculative fiction writer \href{https://en.wikipedia.org/wiki/Ted_Chiang}{Ted Chiang} contains visual depictions of \href{https://en.wikipedia.org/wiki/Fermat%27s_Principle}{Fermat's Principle} along with a discussion of its teleological dimension. \href{https://en.wikipedia.org/wiki/Keith_Devlin}{Keith Devlin}'s \textit{The Math Instinct} contains a chapter, ``Elvis the Welsh Corgi Who Can Do Calculus'' that discusses the calculus ``embedded'' in some animals as they solve the ``least time'' problem in actual situations.'' -- \href{https://en.wikipedia.org/wiki/Stationary-action_principle#Disputes_about_possible_teleological_aspects}{Wikipedia\texttt{/}stationary-action principle\texttt{/}disputes about possible teleological aspects}

%------------------------------------------------------------------------------%

\printbibliography[heading=bibintoc]
	
\end{document}