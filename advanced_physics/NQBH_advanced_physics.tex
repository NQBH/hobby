\documentclass[oneside]{book}
\usepackage[backend=biber,natbib=true,style=authoryear]{biblatex}
\addbibresource{/home/hong/1_NQBH/reference/bib.bib}
\usepackage[vietnamese,english]{babel}
\usepackage{tocloft}
\renewcommand{\cftsecleader}{\cftdotfill{\cftdotsep}}
\usepackage[colorlinks=true,linkcolor=blue,urlcolor=red,citecolor=magenta]{hyperref}
\usepackage{amsmath,amssymb,amsthm,mathtools,float,graphicx}
\allowdisplaybreaks
\numberwithin{equation}{section}
\newtheorem{assumption}{Assumption}[chapter]
\newtheorem{conjecture}{Conjecture}[chapter]
\newtheorem{corollary}{Corollary}[chapter]
\newtheorem{definition}{Definition}[chapter]
\newtheorem{example}{Example}[chapter]
\newtheorem{lemma}{Lemma}[chapter]
\newtheorem{notation}{Notation}[chapter]
\newtheorem{principle}{Principle}[chapter]
\newtheorem{problem}{Problem}[chapter]
\newtheorem{proposition}{Proposition}[chapter]
\newtheorem{question}{Question}[chapter]
\newtheorem{remark}{Remark}[chapter]
\newtheorem{theorem}{Theorem}[chapter]
\usepackage[left=0.5in,right=0.5in,top=1.5cm,bottom=1.5cm]{geometry}
\usepackage{fancyhdr}
\pagestyle{fancy}
\fancyhf{}
\lhead{\small \textsc{Sect.} ~\thesection}
\rhead{\small \nouppercase{\leftmark}}
\renewcommand{\sectionmark}[1]{\markboth{#1}{}}
\cfoot{\thepage}
\def\labelitemii{$\circ$}

\title{Advanced Physics}
\author{\selectlanguage{vietnamese} Nguyễn Quản Bá Hồng\footnote{Independent Researcher, Ben Tre City, Vietnam\\e-mail: \texttt{nguyenquanbahong@gmail.com}}}
\date{\today}

\begin{document}
\maketitle
\setcounter{secnumdepth}{4}
\setcounter{tocdepth}{4}
\tableofcontents

%------------------------------------------------------------------------------%

\chapter{Wikipedia's}

\section{\href{https://en.wikipedia.org/wiki/Stationary-action_principle}{Wikipedia\texttt{/}Stationary-Action Principle}}
``The \textit{stationary-action principle} -- also known as the \textit{principle of least action} -- is a \href{https://en.wikipedia.org/wiki/Variational_principle}{variational principle} that, when applied to the \href{https://en.wikipedia.org/wiki/Action_(physics)}{\textit{action}} of a \href{https://en.wikipedia.org/wiki/Mechanics}{mechanical} system, yields the \href{https://en.wikipedia.org/wiki/Equations_of_motion}{equations of motion} for that system. The principle states that the trajectories (i.e., the solutions of the equations of motions) are \href{https://en.wikipedia.org/wiki/Stationary_point}{\textit{stationary points}} of the system's \textit{action functional}. The term ``least action'' is a historical misnomer since the principle has no minimality requirement: the value of the action functional need not be minimal (even locally) on the trajectories. Least action refers to the absolute value of the action functional being minimized.

The principle can be used to derive \href{https://en.wikipedia.org/wiki/Newtonian_mechanics}{Newtonian}, \href{https://en.wikipedia.org/wiki/Lagrangian_mechanics}{Lagrangian}, \& \href{https://en.wikipedia.org/wiki/Hamiltonian_mechanics}{Hamiltonian} \href{https://en.wikipedia.org/wiki/Equations_of_motion}{equations of motion}, \& even \href{https://en.wikipedia.org/wiki/General_relativity}{general relativity} (see \href{https://en.wikipedia.org/wiki/Einstein%E2%80%93Hilbert_action}{Einstein--Hilbert action}). In relativity, a different action must be minimized or maximized.

The classical mechanics \& electromagnetic expressions are a consequence of quantum mechanics. The stationary action method helped in the development of quantum mechanics. In 1933, the physicist \href{https://en.wikipedia.org/wiki/Paul_Dirac}{Paul Dirac} demonstrated how this principle can be used in quantum calculations by discerning the \href{https://en.wikipedia.org/wiki/Path_integral_formulation#Quantum_action_principle}{quantum mechanical underpinning} of the principle in the \href{https://en.wikipedia.org/wiki/Interference_(wave_propagation)#Quantum_interference}{quantum interference} of amplitudes. Subsequently \href{https://en.wikipedia.org/wiki/Julian_Schwinger}{Julian Schwinger} \& \href{https://en.wikipedia.org/wiki/Richard_Feynman}{Richard Feynman} independently applied this principle in quantum electrodynamics.

The principle remains central in \href{https://en.wikipedia.org/wiki/Modern_physics}{modern physics} \& mathematics, being applied in \href{https://en.wikipedia.org/wiki/Thermodynamics}{thermodynamics}, \href{https://en.wikipedia.org/wiki/Fluid_mechanics}{fluid mechanics}, the \href{https://en.wikipedia.org/wiki/Theory_of_relativity}{theory of relativity}, \href{https://en.wikipedia.org/wiki/Quantum_mechanics}{quantum mechanics}, \href{https://en.wikipedia.org/wiki/Particle_physics}{particle physics}, \& \href{https://en.wikipedia.org/wiki/String_theory}{string theory} \& is a focus of modern mathematical investigation in \href{https://en.wikipedia.org/wiki/Morse_theory}{Morse theory}. \href{https://en.wikipedia.org/wiki/Maupertuis%27_principle}{Maupertuis' principle} \& \href{https://en.wikipedia.org/wiki/Hamilton%27s_principle}{Hamilton's principle} exemplify the principle of stationary action.

The action principle is preceded by earlier ideas in \href{https://en.wikipedia.org/wiki/Optics}{optics}. In \href{https://en.wikipedia.org/wiki/Ancient_Greece}{ancient Greece}, \href{https://en.wikipedia.org/wiki/Euclid}{Euclid} wrote in his \textit{Catoptrica} that, for the path of light reflecting from a mirror, the \href{https://en.wikipedia.org/wiki/Angle_of_incidence_(optics)}{angle of incidence} equals the \href{https://en.wikipedia.org/wiki/Angle_of_reflection}{angle of reflection}. \href{https://en.wikipedia.org/wiki/Hero_of_Alexandria}{Hero of Alexandria} later showed that this path was the shortest length \& least time.

Scholars often credit \href{Pierre Louis Maupertuis} for formulating the principle of least action because he wrote about it in 1744 \& 1746. However, \href{https://en.wikipedia.org/wiki/Leonhard_Euler}{Leonhard Euler} discussed the principle in 1744, \& evidence shows that \href{https://en.wikipedia.org/wiki/Gottfried_Leibniz}{Gottfried Leibniz} preceded both by 39 years.'' -- \href{https://en.wikipedia.org/wiki/Stationary-action_principle}{Wikipedia\texttt{/}stationary-action principle}

\subsection{General statement}

\subsection{Origins, statements, \& controversy}

\subsubsection{Fermat}

\subsubsection{Maupertuis}

\subsubsection{Euler}

\subsubsection{Disputed priority}

\subsection{Further development}

\subsubsection{Lagrange \& Hamilton}

\subsubsection{Jacobi, Morse, \& Caratheodory}

\subsubsection{Gauss \& Hertz}

\subsection{Disputes about possible teleological aspects}

%------------------------------------------------------------------------------%

\printbibliography[heading=bibintoc]
	
\end{document}