\documentclass{article}
\usepackage[utf8]{vietnam}
\usepackage{tocloft}
\renewcommand{\cftsecleader}{\cftdotfill{\cftdotsep}}
\usepackage{float}
\usepackage{graphicx}
\usepackage[colorlinks=true,linkcolor=blue,urlcolor=red,citecolor=magenta]{hyperref}
\usepackage{amsmath,amssymb,amsthm,mathtools}
\allowdisplaybreaks
\numberwithin{equation}{section}
\newtheorem{assumption}{Assumption}[section]
\newtheorem{lemma}{Lemma}[section]
\newtheorem{corollary}{Corollary}[section]
\newtheorem{definition}{Definition}[section]
\newtheorem{proposition}{Proposition}[section]
\newtheorem{theorem}{Theorem}[section]
\newtheorem{notation}{Notation}[section]
\newtheorem{remark}{Remark}[section]
\newtheorem{example}{Example}[section]
\newtheorem{ques}{Question}[section]
\newtheorem{problem}{Problem}[section]
\newtheorem{conjecture}{Conjecture}[section]
\usepackage[left=0.5in,right=0.5in,top=1.5cm,bottom=1.5cm]{geometry}
\usepackage{fancyhdr}
\pagestyle{fancy}
\fancyhf{}
\lhead{\small \textsc{Sect.} ~\thesection}
\rhead{\small \nouppercase{\leftmark}}
\renewcommand{\sectionmark}[1]{\markboth{#1}{}}
\cfoot{\thepage}
\def\labelitemii{$\circ$}

\title{Happiness in\texttt{/}vs. Research}
\author{Nguyễn Quản Bá Hồng}
\date{\today}

\begin{document}
\maketitle
\begin{abstract}
	Một vài suy nghĩ cá nhân về nghề nghiên cứu khoa học nói chung, làm toán chuyên nghiệp nói riêng, và cố gắng tìm kiếm\texttt{/}đưa ra vài định nghĩa về sự hạnh phúc, dù còn khá\texttt{/}quá mơ hồ, trong cái nghề đặc biệt này.
\end{abstract}
\tableofcontents

%------------------------------------------------------------------------------%

\section{Research in General}
\begin{ques}
	What makes scientists feel happy?
	
	Điều gì khiến một người làm khoa học, cả chuyên nghiệp lẫn nghiệp dư, cảm thấy hạnh phúc?
\end{ques}

\begin{ques}
	Làm khoa học, đặc biệt là làm toán, phải trở thành một nghề chuyên nghiệp, hay chỉ nên dừng lại như một sở thích cá nhân? Lợi và hại\texttt{/}được và mất của từng lựa chọn?
\end{ques}

\subsection{Why Do We Love Research?}

\subsection{Citations \& Publications}
\begin{quotation}
	``Để an ủi bạn văn sĩ, mình nói rằng là với cái sự xuất bản, khổ nhất vẫn là các nhà toán học. Cũng chẳng  biết là than thở rằng mình khổ hơn bạn có phải là phương pháp an ủi hiệu quả không, nhưng  các nhà toán học khổ như bò thì là điều chắc chắn. Cũng thức khuya dậy sớm như ai, nhưng ơi hơi  viết ra đọc trong nhà ngoài ngõ không ai thèm hiểu. Và tới khi gửi đi, thì cái đau đầu nó mới bắt đầu.''
	
	``Các nhà toán học, để cho cuộc sống của mình (hay của đồng nghiêp đáng kính phòng bên cạnh) thú vị  hơn, thống kê rất chặt số trích dẫn của các bài báo (số lần bài báo được  nhắc hay dùng tới trong một bài báo khác). Công bằng mà nói, trích dẫn nhiều chưa chắc bài báo đã hay. Nhưng mà không có trích dẫn, thì e hèm, có thể chắc chắn là nó tương đối dở.  Cái trò này mới hiểm, vì bụp một cái, một số  cây đa cây đề  tự nhiên tán lá lại bớt sum suê.'' -- \cite{VHV's blog}\texttt{/}\textit{Xuất bản}
\end{quotation}
\subsection{Research Styles}
\begin{example}[\cite{VNE/PHH}]
	
\end{example}

\subsection{Publications Styles: Quantity or Quality?}
Ở đây, mình nghĩ có 2 phong cách xuất bản chính:
\begin{itemize}
	\item Xuất bản đều đều, có thể coi trọng số lượng hơn chất lượng. Nhưng nếu người nghiên cứu đảm bảo cả số lượng và chất lượng trong từng ấn phẩm thì quá perfecto \texttt{[inserted Italian gesture]}!
	\item Xuất bản quá ít vì quá coi trọng chất lượng.
	\begin{itemize}
		\item \textit{Lợi.} Being able to polish your draft towards perfection.
		\item \textit{Hại.} Tăng nguy cơ bị đào thải khỏi ngành xuất bản do không chạy đủ chỉ tiêu.
		
		Remind the common ``idiom'' for scientists\texttt{/}researchers: \textit{Publish or Perish!}
	\end{itemize}
\end{itemize}
Những đồng nghiệp của mình (postDocs) khi mình còn làm ở Berlin thường hay mô tả công việc của họ bằng những cụm từ ``frustrating'', ``a lot of works'', trong những cuộc tán gẫu hàng ngày, thậm chí trong cả những bài seminar talks của họ, với những người hoàn toàn lạ mặt. Câu hỏi đặt ra ở đây, khi họ dùng những tính từ như vậy để mô tả công việc của họ, là: \textit{Họ có cảm thấy hạnh phúc với công việc nghiên cứu của họ hay không?} Hay những tính từ đó chỉ để đánh bật sự siêng năng, chăm chỉ -- thứ phẩm chất mà một nhà khoa học nói chung, hay một nhà toán học nói riêng, bắt buộc phải có; hay đơn thuần chỉ là một lời ca thán ngầm cho việc chịu đựng áp lực của ngành xuất bản?

\subsection{Publish or Perish?}

\section{Làm Toán Để Hạnh Phúc?}
See, e.g., \cite{TN/VHV}.
\begin{itemize}
	\item Theo Vũ Hà Văn, muốn học khoa học cơ bản, người ta cần một chút mơ mộng.
\end{itemize}

\section{Illnesses in Research Careers}
A list of academic syndromes will be suit here and several ways to detect them will be nice to have.

\subsection{Imposter Syndrome -- The Monster in Research Career}
\textit{Mình có đủ khả năng để làm toán chuyên nghiệp (professional mathematics) hay không?}
\begin{quotation}
	``Mục đích của việc làm toán không phải để giành huy chương hay giải thưởng cao nhất, mà để có hiểu biết sâu sắc về toán học và góp sức mình vào sự phát triển và ứng dụng của môn khoa học kỳ diệu này.'' -- \cite{VHV's blog}\texttt{/}\textit{Bạn có cần có những khả năng thật đặc biệt để làm toán?}
\end{quotation}

\subsection{``Big Theorems, Big Theory'' Syndrome in Professional Mathematics}
\begin{quotation}
	``Áp lực của việc cư xử theo phong cách ``thiên tài'' có thể làm cho người trong cuộc bị ám ảnh với hội chứng ``big theorems, big theory'' (chỉ làm việc với những vấn đề tối quan trọng). Một số người khác có thể bị mất sự đánh giá công bằng về công trình của họ hay những công cụ họ đang sử dụng. Một số người khác nữa có thể đánh mất sự dũng cảm để theo đuổi sự nghiệp nghiên cứu. Mặt khác nữa, giải thích sự thành công bằng khả năng thiên phú cá nhân (là một thứ ta không thể control) thay bằng sự cố gắng, phương thức đào tạo và phác định tương lai (là những thứ ta có thể control) sẽ dẫn tới những vấn đề khác nữa.'' -- \cite{VHV's blog}\texttt{/}\textit{Bạn có cần có những khả năng thật đặc biệt để làm toán?}
\end{quotation}
\subsection{Peer Pressure}

\subsection{Some Reasonable Limits of Curiosity}
After a (seminar) talk, an event, or just reading a paper that you just found online:
\begin{quotation}
	- Oh, this guy\texttt{/}man\texttt{/}girl is so good\texttt{/}smart\texttt{/}interesting! I need to look some information about him since I am so curious.
	
	How to look up his\texttt{/}her\texttt{/}their information? Google Scholar \& ResearchGate of course. But I am still curious: How about beyond his\texttt{/}her\texttt{/}their professional career\texttt{/}life: personal\texttt{/}private life?
\end{quotation}
Then you should stop right there. \textit{How deep your curiosity should be?} Always remember: Enough is enough.

\subsection{Stalkers}
Ông thầy phụ người Đức của mình dạy mình một điều rằng không nên đăng những niềm vui làm khoa học lên mạng xã hội, nếu không muốn bị rình rập và tỉa\texttt{/}rỉa. Không phải niềm vui của bạn lúc nào cũng khiến người khác vui lây. Nghĩ đúng thật, kết quả là đến giờ mình vẫn chưa muốn xài lại Facebook và chưa dám gỡ block ổng.

\subsection{Greed \& Jealousy -- Sự Tham Lam \& Đố Kỵ}

\subsection{Định Nghĩa Về Sự Trưởng Thành Trong Khoa Học?}
\begin{ques}
	Sự trưởng thành trong khoa học là như thế nào?
\end{ques}
Có thật nhiều bài báo, có hàng ngàn, thậm chí chục ngàn, trăm ngàn trích dẫn, CV đầy những hội nghị, báo cáo, etc. là trưởng thành khoa học?

Tất cả những thành tích\texttt{/}tựu đó sụp đổ trong chớp mắt nếu không được gây dựng trên nền tảng đạo đức tối thiểu: The real respect is the ultimate divine.

\subsection{Writer's Block}
\begin{ques}
	Làm thế nào để vượt qua writer's blocks?
\end{ques}

\subsection{Quit Research Careers}

\begin{example}[\cite{VNE/NTH}]
	Nguyễn Trung Hà từ bỏ con đường nghiên cứu Toán học để trở thành nhà đầu tư tài chính. Ông cho rằng: ``học toán càng lên cao càng lãng phí''. Triết lý trong công việc của Nguyễn Trung Hà:
	\begin{quotation}
		``Tôi không ép mình phải làm gì, cũng không để công việc gây sức ép. Tôi có thể bỏ qua việc, chứ không thể bỏ qua cái mình thích. Quan trọng nhất là biết tổ chức công việc.''
	\end{quotation}
	
\end{example}
Nếu rời khỏi nghề nghiên cứu, cuộc sống sẽ đưa đẩy những kẻ đã từng hoặc vẫn còn đam mê nghiên cứu như thế nào? Tốt hơn hay tệ hơn? Liệu họ có còn kiểm soát được cuộc sống của họ bên ngoài môi trường phòng lab nữa không?


\paragraph{Quick notes (edit later).}
Mục đích của làm khoa học nói chung, hay toán học nói riêng?

Làm toán có hạnh phúc không?

\textit{Why do I keep trying to fix something which is so wrong at the very beginning?}

Nghe có vẻ nực cười nhưng quá nghiêm túc và chuyên nghiệp trong 1 công việc nào đó lại là nền tảng phát sinh cho việc chơi bẩn và sự phá hoại ngầm, mà thường là từ đồng nghiệp.

Is it because a man needs a purpose to live, some achievements to keep moving forward, or, more importantly, to justify his existence? [Existential crisis]

Mình chán ghét và sợ hãi sự cô đơn khi làm khoa học, đặc biệt là sự im lặng đáng buồn\texttt{/}sợ sau mỗi bài talks. $\Rightarrow$ Để bớt cảm thấy cô đơn và lạc lõng, mình muốn được trích dẫn nhiều.

\textit{Why so serious?} [Joker's quote]

Mình nhận ra là mình thích học rộng (kiểu sưu tầm) hơn là học sâu. Mình thích kết hợp nhiều mảng lạ lạ với nhau hơn là tìm 1 kiến thức mới lạ trong 1 mảng cố định. Mình thích sưu tầm, mình thích lập list. Điều đó làm mình cảm thấy có năng suất, cảm giác mình năng động, và những cảm giác đó giúp mình ham muốn tiến lên \& lao vào làm việc để tiến lên.

A reasonable decomposition into material and spiritual values: learn how to keep balance in life.

Why can't I bond with my supervisors? Is it because of my fucking annoying perfectionism in writing, especially always targeting generalizations instead a concrete simple model?

\section{``Art in Research\texttt{/}Science'' or ``Art or Research\texttt{/}Science?''}
\begin{ques}
	We do research\texttt{/}science to find some arts in order to satisfy ourselves, which is a form of happiness, or we can choose only one of them in reality? Did scientists have both of them in the past?
\end{ques}

\subsection{Usefulness or Pride, Elegance, \& Arrogance}
\begin{ques}
	Mục đích của việc xuất bản ấn phẩm khoa học là gì?
\end{ques}
Khoảng thời gian làm nghiên cứu sinh 1 năm rưỡi, dù ngắn ngủi ở Đức và Áo, giúp mình nhận ra có nhiều kiểu xuất bản hơn là bài báo đăng trên tạp chí kiểu truyền thống, đặc biệt là xuất bản một phần mềm.

Ở đây mình nghĩ có 2 kiểu mục đích chính, hay đúng hơn là 2 sự lựa chọn chính:
\begin{itemize}
	\item Có 1 bài báo được xuất bản trên 1 tạp chí top\texttt{/}tốt.
	\item Có 1 bài báo\texttt{/}ấn phẩm khoa học, thậm chí không cần xuất bản trên một tạp chí (tốt), nhưng có ích với nhiều người vì\texttt{/}và\texttt{/}hoặc có ứng dụng trong nhiều lĩnh vực.
\end{itemize}

\begin{quotation}
	``Đóng góp này còn hơn rất nhiều giáo sư viết ra một đống bài báo không có ai đọc.'' -- \cite{VHV's blog}\texttt{/}\textit{Nhật ký Yale: Học sinh cá biệt}
\end{quotation}
Điều này dẫn đến một câu hỏi mà mình luôn trăn trở tìm câu trả lời:
\begin{ques}
	Làm thế nào để viết một bài báo thu hút nhiều người muốn đọc và trích dẫn?
\end{ques}

\subsection{Wrath \& Peace}
Những lúc cảm giác mơ hồ và mắc kẹt với những vấn đề và những câu hỏi cụ thể, hãy\texttt{/}nên quay lại những vấn đề và câu hỏi mang tính meta nền tảng, để giúp định hướng lại.

Damn! Lost in the parade of words, of my own thoughts again.

\section{Beyond Research Careers}

\subsection{Connection between Mathematics \& Arts}

\section{Miscellaneous}
This text is still growing$\ldots$
\begin{flushright}
	\textit{``The quest for perfection can never end.''}
\end{flushright}

%------------------------------------------------------------------------------%

\begin{thebibliography}{99}
	\bibitem[DTS's blog]{DTS's blog} \href{https://damtson.wordpress.com/}{Đàm Thanh Sơn's blog}.
	
	\bibitem[FB\texttt{/}HHK]{FB/HHK} \href{https://www.facebook.com/hahuy.khoai}{Hà Huy Khoái's Facebook Posts}.
	
	\bibitem[FB\texttt{/}LCKH]{FB/LCKH} \href{https://www.facebook.com/groups/LiemChinhKhoaHoc}{Facebook\texttt{/}Liêm Chính Khoa Học}.
	
	\bibitem[FB\texttt{/}NHVH]{FB/NHVH} \href{https://www.facebook.com/nhvhung}{Nguyễn Hữu Việt Hưng's Facebook Posts}.
	
	\bibitem[FB\texttt{/}NT]{FB/NT} \href{https://www.facebook.com/t.nguyen.2016}{Nguyễn Tuấn's Facebook Posts}.
	
	\bibitem[FB\texttt{/}PHH]{FB/FHH} \href{https://www.facebook.com/hai.phungho.5}{Phùng Hồ Hải's Facebook Posts}.
	
	\bibitem[FB\texttt{/}PTHD]{FB/PTHD} \href{https://www.facebook.com/phan.t.duong.9}{Phan Thị Hà Dương's Facebook Posts}.
	
	\bibitem[HocTheNao.VN]{HocTheNao.VN} \href{http://hocthenao.vn/}{Học Thế Nào}.
	
	\bibitem[MathVN\texttt{/}VHV]{MathVN/VHV} \href{https://www.mathvn.com/2019/03/giao-su-vu-ha-van-hoc-toan-va-lam-toan.html}{Giáo sư Vũ Hà Văn: Học Toán và làm Toán để hạnh phúc}. 2019.
	
	\bibitem[PP]{PP} Prosperous Physicist. \href{https://www.prosperousphysicist.com/dont-become-a-scientist-15-years-later/}{``Don't Become a Scientist!'' 15 Years Later}.
	
	\bibitem[PP2021]{PP2021} Prosperous Physicist. \href{http://www.prosperousphysicist.com/dont-become-a-scientist-20-years-later/}{``Don't Become a Scientist!'' 20 Years Later}.
	
	\bibitem[TN\texttt{/}VHV]{TN/VHV} \href{https://thanhnien.vn/gs-vu-ha-van-lam-toan-de-hanh-phuc-post15849.html}{GS Vũ Hà Văn: Làm toán để hạnh phúc}.
	
	\bibitem[TS]{TS} Tia sáng\texttt{/}Khoa Học Công Nghệ\texttt{/}\href{https://tiasang.com.vn/khoa-hoc-cong-nghe/dung-tro-thanh-mot-nha-khoa-hoc-903}{``Đừng Trở Thành 1 Nhà Khoa Học\ldots''}.
	
	\bibitem[TT's blog]{TT's blog} \href{https://terrytao.wordpress.com/}{Terrence Tao's blog}.
	\begin{itemize}
		\item Terrence Tao. \href{https://terrytao.wordpress.com/advice-on-writing-papers/}{Advice on Writing Papers}.
		\item Terrence Tao. \href{https://terrytao.wordpress.com/career-advice/}{Career Advice}.
		\begin{itemize}
			\item Terence Tao. \href{https://terrytao.wordpress.com/career-advice/take-the-initiative/}{Career advice\texttt{/}Take the initiative}.
			\item Terence Tao. \href{https://terrytao.wordpress.com/career-advice/talk-to-your-advisor/}{Career advice\texttt{/}Talk to your advisor}.
		\end{itemize}
	\end{itemize}
	
	\bibitem[VHV's blog]{VHV's blog} \href{https://vuhavan.wordpress.com/}{Vũ Hà Văn's blog}.
	\begin{itemize}
		\item Vũ Hà Văn. \href{https://vuhavan.wordpress.com/2011/12/24/b%e1%ba%a1n-co-c%e1%ba%a7n-co-nh%e1%bb%afng-kh%e1%ba%a3-nang-th%e1%ba%adt-d%e1%ba%b7c-bi%e1%bb%87t-d%e1%bb%83-lam-toan/}{Bạn có cần có những khả năng thật đặc biệt để làm toán?}. 24.12.2011.
		\item Vũ Hà Văn. \href{https://vuhavan.wordpress.com/2014/02/23/nhat-ky-yale-hoc-sinh-ca-biet/}{Nhật ký Yale: Học sinh cá biệt}. 23.2.2014.
		\item Vũ Hà Văn. \href{https://vuhavan.wordpress.com/2014/09/21/xuat-ban/}{Xuất bản}. 21.9.2014.
		\item Vũ Hà Văn. \href{https://vuhavan.wordpress.com/2014/01/14/vai-suy-nghi-ngan-ve-toan-ung-dung-va-ung-dung-toan/}{Vài suy nghĩ ngắn về toán ứng dụng \& ứng dụng toán}. 14.11.2014.
		\item Vũ Hà Văn. \href{https://vuhavan.wordpress.com/2015/01/20/nhat-ky-yale-hoa-hau-va-giao-su/}{Nhật ký Yale: Hoa hậu và Giáo sư}. 20.1.2015.
		\item Vũ Hà Văn. \href{https://vuhavan.wordpress.com/2018/02/21/ca-nha-deu-beo/}{Dạy \& học toán I: Cả nhà đều béo}. 21.2.2018.
		\item Vũ Hà Văn. \href{https://vuhavan.wordpress.com/2018/02/25/dao-van/}{Đạo văn}. 25.2.2018.
	\end{itemize}
	
	\bibitem[VNE\texttt{/}PHH]{VNE/PHH} VNExpress\texttt{/}\href{https://vnexpress.net/tien-si-nguoi-viet-va-moi-duyen-no-voi-google-4430615.html}{Tiến sĩ người Việt và mối duyên nợ với Google} [Phạm Hy Hiếu].
	
	\bibitem[VNE\texttt{/}NTH]{VNE/NTH} VNExpress\texttt{/}\href{https://vnexpress.net/hoc-toan-cao-cap-nhu-dot-tien-de-suoi-2223794.html}{Học Toán Cao Cấp Như `Đốt Tiền Để Sưởi'} [Nguyễn Trung Hà]. 28.2.2012.
\end{thebibliography}

\end{document}