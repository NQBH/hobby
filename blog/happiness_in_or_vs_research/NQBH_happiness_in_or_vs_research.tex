\documentclass{article}
\usepackage[utf8]{vietnam}
\usepackage{tocloft}
\renewcommand{\cftsecleader}{\cftdotfill{\cftdotsep}}
\usepackage{float}
\usepackage{graphicx}
\usepackage[colorlinks=true,linkcolor=blue,urlcolor=red,citecolor=magenta]{hyperref}
\usepackage{amsmath,amssymb,amsthm,mathtools}
\allowdisplaybreaks
\numberwithin{equation}{section}
\newtheorem{assumption}{Assumption}[section]
\newtheorem{lemma}{Lemma}[section]
\newtheorem{corollary}{Corollary}[section]
\newtheorem{definition}{Definition}[section]
\newtheorem{proposition}{Proposition}[section]
\newtheorem{theorem}{Theorem}[section]
\newtheorem{notation}{Notation}[section]
\newtheorem{remark}{Remark}[section]
\newtheorem{example}{Example}[section]
\newtheorem{ques}{Question}[section]
\newtheorem{problem}{Problem}[section]
\newtheorem{conjecture}{Conjecture}[section]
\usepackage[left=0.5in,right=0.5in,top=1.5cm,bottom=1.5cm]{geometry}
\usepackage{fancyhdr}
\pagestyle{fancy}
\fancyhf{}
\lhead{\small \textsc{Sect.} ~\thesection}
\rhead{\small \nouppercase{\leftmark}}
\renewcommand{\sectionmark}[1]{\markboth{#1}{}}
\cfoot{\thepage}
\def\labelitemii{$\circ$}

\title{Happiness in\texttt{/}vs. Research}
\author{Nguyễn Quản Bá Hồng}
\date{\today}

\begin{document}
\maketitle
\begin{abstract}
	Một vài suy nghĩ cá nhân về nghề nghiên cứu khoa học nói chung, làm toán chuyên nghiệp nói riêng, và cố gắng tìm kiếm\texttt{/}đưa ra vài định nghĩa về sự hạnh phúc, dù còn khá\texttt{/}quá mơ hồ, trong cái nghề đặc biệt này.
\end{abstract}
\tableofcontents

%------------------------------------------------------------------------------%

\section{Research: Citations, \& Publications}
\begin{ques}
	What makes scientists feel happy?
	
	Điều gì khiến một người làm khoa học, cả chuyên nghiệp lẫn nghiệp dư, cảm thấy hạnh phúc?
\end{ques}

\begin{ques}
	Làm khoa học, đặc biệt là làm toán, phải trở thành một nghề chuyên nghiệp, hay chỉ nên dừng lại như một sở thích cá nhân? Lợi và hại\texttt{/}được và mất của từng lựa chọn?
\end{ques}

\subsection{Publications Styles: Quantity or Quality?}
Ở đây, mình nghĩ có 2 phong cách xuất bản chính:
\begin{itemize}
	\item Xuất bản đều đều, có thể coi trọng số lượng hơn chất lượng. Nhưng nếu người nghiên cứu đảm bảo cả số lượng và chất lượng trong từng ấn phẩm thì quá perfecto!
	\item Xuất bản quá ít vì quá coi trọng chất lượng.
	\begin{itemize}
		\item \textbf{Hại.} Tăng nguy cơ bị đào thải khỏi ngành xuất bản do không chạy đủ chỉ tiêu.
		
		Remind the common ``idiom'' for scientists\texttt{/}researchers: \textit{Publish or Perish!}
	\end{itemize}
\end{itemize}
Những đồng nghiệp của mình (postDocs) khi mình còn làm ở Berlin thường hay mô tả công việc của họ bằng những cụm từ ``frustrating'', ``a lot of work'', trong những cuộc tán gẫu hàng ngày, thậm chí trong cả những bài seminar talks của họ, với những người hoàn toàn lạ mặt. Câu hỏi đặt ra ở đây, khi họ dùng những tính từ như vậy để mô tả công việc của họ, là: \textit{Họ có cảm thấy hạnh phúc với công việc nghiên cứu của họ hay không?} Hay những tính từ đó chỉ để đánh bật sự siêng năng, chăm chỉ -- thứ phẩm chất mà một nhà khoa học nói chung, hay một nhà toán học nói riêng, bắt buộc phải có; hay đơn thuần chỉ là một lời ca thán ngầm cho việc chịu đựng áp lực của ngành xuất bản?

\section{Illnesses in Research Careers}

\subsection{Imposter Syndrome -- The Monster in Research Career}

\subsection{Peer Pressure}

\subsection{Quit Research Careers}

\begin{example}[\cite{VNE/NTH}]
	Nguyễn Trung Hà từ bỏ con đường nghiên cứu Toán học để trở thành nhà đầu tư tài chính. Ông cho rằng: ``học toán càng lên cao càng lãng phí''. Triết lý trong công việc của Nguyễn Trung Hà:
	\begin{quotation}
		``Tôi không ép mình phải làm gì, cũng không để công việc gây sức ép. Tôi có thể bỏ qua việc, chứ không thể bỏ qua cái mình thích. Quan trọng nhất là biết tổ chức công việc.''
	\end{quotation}
	
\end{example}
Nếu rời khỏi nghề nghiên cứu, cuộc sống sẽ đưa đẩy nhưng kẻ đã từng hoặc vẫn còn đam mê nghiên cứu như thế nào? Tốt hơn hay tệ hơn? Liệu họ có còn kiểm soát được cuộc sống của họ bên ngoài môi trường phòng lab nữa không?


\paragraph{Quick notes (edit later).}
Mục đích của làm khoa học nói chung, hay toán học nói riêng?

Làm toán có hạnh phúc không?

\textit{Why do I keep trying to fix something which is so wrong at the very beginning?}

Nghe có vẻ nực cười nhưng quá nghiêm túc và chuyên nghiệp trong 1 công việc nào đó lại là nền tảng phát sinh cho việc chơi bẩn và sự phá hoại ngầm, mà thường là từ đồng nghiệp.

Is it because a man needs a purpose to live, some achievements to justify his existence?

\section{``Art in Research\texttt{/}Science'' or ``Art or Research\texttt{/}Science?''}
\begin{ques}
	We do research\texttt{/}science to find some arts in order to satisfy ourselves, which is a form of happiness, or we can choose only one of them in reality? Did scientists have both of them in the past?
\end{ques}

\section{Beyond Research Careers}

%------------------------------------------------------------------------------%

\begin{thebibliography}{99}
	\bibitem[MathVN\texttt{/}VHV]{MathVN/VHV} \href{https://www.mathvn.com/2019/03/giao-su-vu-ha-van-hoc-toan-va-lam-toan.html}{Giáo sư Vũ Hà Văn: Học Toán và làm Toán để hạnh phúc}. 2019.
	
	\bibitem[PP]{PP} Prosperous Physicist. \href{https://www.prosperousphysicist.com/dont-become-a-scientist-15-years-later/}{``Don't Become a Scientist!'' 15 Years Later}.
	
	\bibitem[PP2021]{PP2021} Prosperous Physicist. \href{http://www.prosperousphysicist.com/dont-become-a-scientist-20-years-later/}{``Don't Become a Scientist!'' 20 Years Later}.
	
	\bibitem[TS]{TS} Tia sáng\texttt{/}Khoa Học Công Nghệ\texttt{/}\href{https://tiasang.com.vn/khoa-hoc-cong-nghe/dung-tro-thanh-mot-nha-khoa-hoc-903}{``Đừng Trở Thành 1 Nhà Khoa Học\ldots''}.
	
	\bibitem[VNE\texttt{/}NTH]{VNE/NTH} VNExpress\texttt{/}\href{https://vnexpress.net/hoc-toan-cao-cap-nhu-dot-tien-de-suoi-2223794.html}{Học Toán Cao Cấp Như `Đốt Tiền Để Sưởi'}. 28.2.2012.
\end{thebibliography}

\end{document}