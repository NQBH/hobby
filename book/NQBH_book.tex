\documentclass{article}
\usepackage[backend=biber,natbib=true,style=alphabetic,maxbibnames=50]{biblatex}
\addbibresource{/home/nqbh/reference/bib.bib}
\usepackage[utf8]{vietnam}
\usepackage{tocloft}
\renewcommand{\cftsecleader}{\cftdotfill{\cftdotsep}}
\usepackage[colorlinks=true,linkcolor=blue,urlcolor=red,citecolor=magenta]{hyperref}
\usepackage{amsmath,amssymb,amsthm,float,graphicx,mathtools,soul}
\usepackage{enumitem}
\setlist{leftmargin=4mm}
\allowdisplaybreaks
\newtheorem{assumption}{Assumption}
\newtheorem{baitoan}{Bài toán}
\newtheorem{cauhoi}{Câu hỏi}
\newtheorem{conjecture}{Conjecture}
\newtheorem{corollary}{Corollary}
\newtheorem{dangtoan}{Dạng toán}
\newtheorem{definition}{Definition}
\newtheorem{dinhly}{Định lý}
\newtheorem{dinhnghia}{Định nghĩa}
\newtheorem{example}{Example}
\newtheorem{ghichu}{Ghi chú}
\newtheorem{hequa}{Hệ quả}
\newtheorem{hypothesis}{Hypothesis}
\newtheorem{lemma}{Lemma}
\newtheorem{luuy}{Lưu ý}
\newtheorem{nhanxet}{Nhận xét}
\newtheorem{notation}{Notation}
\newtheorem{note}{Note}
\newtheorem{principle}{Principle}
\newtheorem{problem}{Problem}
\newtheorem{proposition}{Proposition}
\newtheorem{question}{Question}
\newtheorem{remark}{Remark}
\newtheorem{theorem}{Theorem}
\newtheorem{vidu}{Ví dụ}
\usepackage[left=1cm,right=1cm,top=5mm,bottom=5mm,footskip=4mm]{geometry}

\title{Article {\it\&} Book}
\author{Nguyễn Quản Bá Hồng\footnote{Independent Researcher, Ben Tre City, Vietnam\\e-mail: \texttt{nguyenquanbahong@gmail.com}; website: \url{https://nqbh.github.io}.}}
\date{\today}

\begin{document}
\maketitle
\begin{abstract}
	A list of articles \& books planned to buy\texttt{/}download.
\end{abstract}
\tableofcontents

%------------------------------------------------------------------------------%

\section*{Library}
{\sf Websites to download, respectively, books \& scientific articles freely:} \url{https://libgen.is/}, \url{https://sci-hub.se/}.

Có rất nhiều sách liệt kê ở đây nhưng mình không mua. Đơn giản là hứng lên  thì liệt kê vào danh sách những sách \emph{có tiềm năng} để mua nhưng 1 thời gian sau phát hiện hướng viết không cần những sách đó nên thôi. Cứ liệt kê đã, mài dũa sau. Do it 1st, sharpen it later.

\section{Elementary STEM Book}

\subsection{Elementary Mathematics Book}

\subsubsection{Grade 6}

\begin{enumerate}
	\item \cite{Binh_Toan_6_tap_1}. Vũ Hữu Bình. \textit{Nâng Cao \& Phát Triển Toán 6. Tập 1}.\hfill{\sf[finished]}
	\item \cite{Binh_Toan_6_tap_2}. Vũ Hữu Bình. \textit{Nâng Cao \& Phát Triển Toán 6. Tập 2}.\hfill{\sf[finished]}
	\item \cite{Binh_boi_duong_Toan_6_tap_1}. Vũ Hữu Bình, Đặng Văn Quân, Bùi Văn Tuyên. \textit{Bồi Dưỡng Toán 6. Tập 1}.\hfill{\sf[finished]}
	\item \cite{Binh_boi_duong_Toan_6_tap_2}. Vũ Hữu Bình, Nguyễn Thị Quỳnh Anh, Phan Thanh Hồng, Bùi Văn Tuyên, Đặng Văn Tuyến, Nguyễn Thị Thanh Xuân. \textit{Bồi Dưỡng Toán 6. Tập 2}.\hfill{\sf[reading]}
	\item \cite{TLCT_THCS_Toan_6_so_hoc}. Vũ Hữu Bình, Nguyễn Tam Sơn. \textit{Tài Liệu Chuyên Toán THCS Toán 6. Tập 1: Số Học}.\hfill{\sf[finished]}
	\item \cite{TLCT_THCS_Toan_6_hinh_hoc}. Vũ Hữu Bình, Đàm Hiếu Chiến. \textit{Tài Liệu Chuyên Toán THCS Toán 6. Tập 2: Hình Học}.\hfill{\sf[reading]}
	\item \cite{SGK_Toan_6_Canh_Dieu_tap_1}. Đỗ Đức Thái, Đỗ Tiến Đạt, Nguyễn Sơn Hà, Nguyễn Thị Phương Loan, Phạm Sỹ Nam, Phạm Đức Quang. \textit{Toán 6 Tập 1. Cánh Diều}.\hfill{\sf[finished]}
	\item \cite{SGK_Toan_6_Canh_Dieu_tap_2}. Đỗ Đức Thái, Đỗ Tiến Đạt, Nguyễn Sơn Hà, Nguyễn Thị Phương Loan, Phạm Sỹ Nam, Phạm Đức Quang. \textit{Toán 6 Tập 2. Cánh Diều}.\hfill{\sf[finished]}
	\item \cite{SBT_Toan_6_Canh_Dieu_tap_1}. Đỗ Đức Thái. \textit{Bài Tập Toán 6 Tập 1. Cánh Diều}.\hfill{\sf[finished]}
	\item \cite{SBT_Toan_6_Canh_Dieu_tap_2}. Đỗ Đức Thái. \textit{Bài Tập Toán 6 Tập 2. Cánh Diều}.\hfill{\sf[finished]}
	\item \cite{Trong_Toan_6_2021}. Đặng Đức Trọng, Nguyễn Đức Tấn, Phạm Lê Quốc Thắng, Nguyễn Phúc Trường, Cao Hoàng Lợi. \textit{Bồi Dưỡng Năng Lực Tự Học Toán 6}.\hfill{\sf[reading]}
	\item \cite{Tuyen_Toan_6}. Bùi Văn Tuyên. \textit{Bài Tập Nâng Cao \& 1 Số Chuyên Đề Toán 6}.\hfill{\sf[reading]}
\end{enumerate}

\subsubsection{Grade 7}
	
\begin{enumerate}
	\item \cite{Binh_Toan_7_tap_1}. Vũ Hữu Bình. \textit{Nâng Cao \& Phát Triển Toán 7. Tập 1}.\hfill{\sf[finished]}
	\item \cite{Binh_Toan_7_tap_2}. Vũ Hữu Bình. \textit{Nâng Cao \& Phát Triển Toán 7. Tập 2}.\hfill{\sf[finished]}
	\item Vũ Hữu Bình. \textit{Tài Liệu Chuyên Toán THCS Toán 7. Tập 1: Đại Số}.
	\item Vũ Hữu Bình. \textit{Tài Liệu Chuyên Toán THCS Toán 7. Tập 2: Hình Học}.
	\item \cite{Binh_boi_duong_Toan_7_tap_1}. Vũ Hữu Bình, Nguyễn Xuân Bình, Đàm Hiếu Chiến, Phan Thanh Hồng, Nguyễn Thị Thanh Xuân. \textit{Bồi Dưỡng Toán 7 Tập 1}.\hfill{\sf[reading]}
	\item \cite{Binh_boi_duong_Toan_7_tap_2}. Vũ Hữu Bình, Nguyễn Xuân Bình, Đàm Hiếu Chiến. \textit{Bồi Dưỡng Toán 7 Tập 2}.\hfill{\sf[reading]}
	\item \cite{Hung_Mai_Toan_7_hinh_hoc}. Trần Quang Hùng, Đào Thị Hoa Mai. \textit{Tuyển Chọn Các Chuyên Đề Bồi Dưỡng Học Sinh Giỏi Toán 7 Hình Học}.\\\mbox{}\hfill{\sf[reading]}
	\item \cite{SGK_Toan_7_Canh_Dieu_tap_1}. Đỗ Đức Thái, Đỗ Tiến Đạt, Nguyễn Sơn Hà, Nguyễn Thị Phương Loan, Phạm Sỹ Nam, Phạm Đức Quang. \textit{Toán 7 Tập 1. Cánh Diều}.\hfill{\sf[finished]}
	\item \cite{SGK_Toan_7_Canh_Dieu_tap_2}. Đỗ Đức Thái, Đỗ Tiến Đạt, Nguyễn Sơn Hà, Nguyễn Thị Phương Loan, Phạm Sỹ Nam, Phạm Đức Quang. \textit{Toán 7 Tập 2. Cánh Diều}.\hfill{\sf[finished]}
	\item \cite{Trong_Toan_7_2022}. Đặng Đức Trọng, Nguyễn Đức Tấn, Phạm Lê Quốc Thắng, Nguyễn Phúc Trường, Cao Hoàng Lợi, Nguyễn Thị Kiều Anh. \textit{Bồi Dưỡng Năng Lực Tự Học Toán 7}.\hfill{\sf[reading]}
	\item \cite{Tuyen_Toan_7}. Bùi Văn Tuyên. \textit{Bài Tập Nâng Cao \& 1 Số Chuyên Đề Toán 7}.\hfill{\sf[finished]}
\end{enumerate}

\subsubsection{Grade 8}

\begin{enumerate}
	\item \cite{Binh_Toan_8_tap_1}. Vũ Hữu Bình. \textit{Nâng Cao \& Phát Triển Toán 8. Tập 1}.\hfill{\sf[reading]}
	\item \cite{Binh_Toan_8_tap_2}. Vũ Hữu Bình. \textit{Nâng Cao \& Phát Triển Toán 8. Tập 2}.\hfill{\sf[reading]}
	\item Vũ Hữu Bình, Tôn Thân, Đỗ Quang Thiều. \textit{Toán Bồi Dưỡng Học sinh Lớp 8 Đại Số}.
	\item Vũ Hữu Bình, Tôn Thân, Đỗ Quang Thiều. \textit{Toán Bồi Dưỡng Học sinh Lớp 8 Hình Học}.
	\item \cite{Binh_boi_duong_Toan_8_tap_1}. Vũ Hữu Bình, Nguyễn Xuân Bình, Phan Thanh Hồng, Phạm Thị Bạch Ngọc, Nguyễn Thị Thanh Xuân. \textit{Bồi Dưỡng Toán 8 Tập 1}.\hfill{\sf[reading]}
	\item \cite{Binh_boi_duong_Toan_8_tap_2}. Vũ Hữu Bình, Đàm Hiếu Chiến, Nguyễn Bá Đang, Phạm Thị Bạch Ngọc. \textit{Bồi Dưỡng Toán 8 Tập 2}.\hfill{\sf[reading]}
	\item \cite{TLCT_THCS_Toan_8_dai_so}. Vũ Hữu Bình, Trần Hữu Nam, Phạm Thị Bạch Ngọc, Nguyễn Tam Sơn. \textit{Tài Liệu Chuyên Toán THCS Toán 8. Tập 1: Đại Số}.\hfill{\sf[reading]}
	\item \cite{TLCT_THCS_Toan_8_hinh_hoc}. Vũ Hữu Bình, Văn Như Cương, Nguyễn Ngọc Đạm, Nguyễn Bá Đang, Trương Công Thành. \textit{Tài Liệu Chuyên Toán THCS Toán 8. Tập 2: Hình Học}.\hfill{\sf[reading]}
	\item \cite{SGK_Toan_8_tap_1}. Phan Đức Chính, Tôn Thân, Vũ Hữu Bình, Trần Đình Châu, Ngô Hữu Dũng, Phạm Gia Đức, Nguyễn Duy
	Thuận. \textit{Toán 8 Tập 1}.\hfill{\sf[finished]}
	\item \cite{SGK_Toan_8_tap_2}. Phan Đức Chính, Tôn Thân, Nguyễn Huy Đoan, Lê Văn Hồng, Trương Công Thành, Nguyễn Hữu Thảo. \textit{Toán 8 Tập 2}.\hfill{\sf[finished]}
	\item \cite{SGK_Toan_8_Canh_Dieu_tap_1}. Đỗ Đức Thái, Lê Tuấn Anh, Đỗ Tiến Đạt, Nguyễn Sơn Hà, Nguyễn Thị Phương Loan, Phạm Sỹ Nam, Phạm Đức Quang. \textit{Toán 8 Cánh Diều Tập 1}.\hfill{\sf[reading]}
	\item \cite{SGK_Toan_8_Canh_Dieu_tap_2}. Đỗ Đức Thái, Lê Tuấn Anh, Đỗ Tiến Đạt, Nguyễn Sơn Hà, Nguyễn Thị Phương Loan, Phạm Sỹ Nam, Phạm Đức Quang. \textit{Toán 8 Cánh Diều Tập 2}.\hfill{\sf[reading]}
	\item \cite{Tuyen_Toan_8}. Bùi Văn Tuyên. \textit{Bài Tập Nâng Cao \& 1 Số Chuyên Đề Toán 8}.\hfill{\sf[reading]}
	\item \cite{Tuyen_Toan_8_old}. Bùi Văn Tuyên. \textit{Bài Tập Nâng Cao \& 1 Số Chuyên Đề Toán 8}.
\end{enumerate}
	
\subsubsection{Grade 9}

\begin{enumerate}
	\item \cite{Binh_Toan_9_tap_1}. Vũ Hữu Bình. \textit{Nâng Cao \& Phát Triển Toán 9. Tập 1}.\hfill{\sf[reading]}
	\item \cite{Binh_Toan_9_tap_2}. Vũ Hữu Bình. \textit{Nâng Cao \& Phát Triển Toán 9. Tập 2}.\hfill{\sf[reading]}
	\item \cite{Binh_boi_duong_Toan_9_tap_1}. Vũ Hữu Bình, Nguyễn Xuân Bình, Phạm Thị Bạch Ngọc. \textit{Bồi Dưỡng Toán 9. Tập 1.}\hfill{\sf[finished]}
	\item \cite{Binh_boi_duong_Toan_9_tap_2}. Vũ Hữu Bình, Nguyễn Xuân Bình, Phạm Thị Bạch Ngọc. \textit{Bồi Dưỡng Toán 9. Tập 2.}\hfill{\sf[reading]}
	\item \cite{TLCT_THCS_Toan_9_dai_so}. Vũ Hữu Bình, Phạm Thị Bạch Ngọc, Đàm Văn Nhỉ. \textit{Tài Liệu Chuyên Toán THCS Toán 9. Tập 1: Đại Số}.\hfill{\sf[reading]}
	\item \cite{TLCT_THCS_Toan_9_hinh_hoc}. Vũ Hữu Bình, Nguyễn Ngọc Đạm, Nguyễn Bá Đang, Lê Quốc Hán, Hồ Quang Vinh. \textit{Tài Liệu Chuyên Toán THCS Toán 9. Tập 2: Hình Học}.\hfill{\sf[reading]}
	\item \cite{Dung_Can_Anh_BDT_8_9}. Nguyễn Văn Dũng, Võ Quốc Bá Cẩn, Trần Quốc Anh. \textit{Phương Pháp Giải Toán Bất Đẳng Thức \& Cực Trị Dành Cho Học Sinh 8, 9}.\hfill{\sf[reading]}
	\item \cite{Tuyen_Toan_9_old}. Bùi Văn Tuyên. \textit{Bài Tập Nâng Cao \& 1 Số Chuyên Đề Toán 9}.\hfill{\sf[reading]}
	\item Vũ Dương Thụy, Nguyễn Ngọc Đạm. \textit{Toán Nâng Cao \& Các Chuyên Đề Hình Học 9}.
\end{enumerate}

\subsubsection{Secondary School -- Trung Học Cơ Sở [THCS]}

\begin{enumerate}
	\item Vũ Hữu Bình. \textit{9 Chuyên Đề Đại Số THCS}.
	\item Vũ Hữu Bình. \textit{9 Chuyên Đề Số Học THCS}.
	\item Vũ Hữu Bình. \textit{9 Chuyên Đề Hình Học THCS}.
	\item \cite{Dang2018}. Nguyễn Bá Đang. \textit{Phát Triển Kỹ Năng Giải Toán Hình Học Phẳng Dành Cho Bậc THCS}.\hfill{\sf[reading]}
	\item \cite{Dong_23_1001_toan_I}. Nguyễn Đức Đồng. \textit{23 Chuyên Đề Giải 1001 Bài Toán Sơ Cấp. Tập 1}.\hfill{\sf[reading]}
	\item \cite{Dong_23_1001_toan_II}. Nguyễn Đức Đồng. \textit{23 Chuyên Đề Giải 1001 Bài Toán Sơ Cấp. Tập 2}.\hfill{\sf[reading]}
	\item \cite{Hung_Dung_Mai_Qua_Long_Toan_9_hinh_hoc}. Trần Quang Hùng, Nguyễn Tiến Dũng, Đào Thị Hoa Mai, Nguyễn Đăng Quả, Đỗ Xuân Long. \textit{Tuyển Chọn Các Chuyên Đề Bồi Dưỡng Học Sinh Giỏi Toán 9 Hình Học}.\hfill{\sf[reading]}
	\item \cite{Kien_Trung_Khuong_Hanh_Bon}. Nguyễn Trung Kiên, Đặng Thành Trung, Nguyễn Duy Khương, Bùi Hồng Hạnh, Vũ Trung Bồn. \textit{Một Số Chủ Đề Hay \& Khó Trong Kỳ Thi Tuyển Sinh Vào Lớp 10}.\hfill{\sf[reading]}
	\item \cite{Lam_An_Tuan_Toan_9_dai_so}. Nguyễn Tiến Lâm, Trương Quang An, Trịnh Khắc Tuân. \textit{Tuyển Chọn Các Chuyên Đề Bồi Dưỡng Học Sinh Giỏi Toán 9 Đại Số}.\hfill{\sf[reading]}
	\item Phạm Minh Phương, Trần Văn Tấn, Nguyễn Thị Thanh Thủy. \textit{Bồi Dưỡng Học Sinh Giỏi Toán THCS: Số Học}.
	\item \cite{Son_Nghiep_Trung_Can_bdt}. Nguyễn Ngọc Sơn, Chu Đình Nghiệp, Lê Hải Trung, Võ Quốc Bá Cẩn. \textit{Các Chủ Đề Bất Đẳng Thức Ôn Thi Vào Lớp 10}.\hfill{\sf[reading]}
	\item \cite{Son_Tinh_Trung_Cau_rgbt}. Nguyễn Ngọc Sơn, Trần Văn Tình, Lê Hải Trung, Vũ Văn Cầu. \textit{Luyện Thi Vào Lớp 10 Môn Toán Chuyên Đề Rút Gọn Biểu Thức}.\hfill{\sf[reading]}
	\item \cite{Tan_Han_Dung_Duc_Hieu_Phuong_Hau_Nga_Long_Tuan_tap_2}. Nguyễn Đức Tấn, Nguyễn Ngọc Hân, Cao Văn Dũng, Phí Trung Đức, Tạ Minh Hiếu, Thái Nhật Phượng, Hoàng Công Hậu, Trần Thị Phi Nga, Phùng Văn Long, Nguyễn Quang Tuấn. \textit{Ôn Luyện Thi Vào Lớp 10 Chuyên Môn Toán Tập 2}.\\\mbox{}\hfill{\sf[reading]}
\end{enumerate}

\subsubsection{Grade 10}

\begin{enumerate}
	\item \cite{Hai_Hung_Thu_Tung2022_tap_1}. Phạm Việt Hải, Trần Quang Hùng, Ninh Văn Thu, Phạm Đình Tùng. \textit{Nâng Cao \& Phát Triển Toán 10 Tập 1}.\\\mbox{}\hfill{\sf[reading]}
	\item \cite{Hai_Hung_Thu_Tung2022_tap_2}. Phạm Việt Hải, Trần Quang Hùng, Ninh Văn Thu, Phạm Đình Tùng. \textit{Nâng Cao \& Phát Triển Toán 10 Tập 2}.\\\mbox{}\hfill{\sf[reading]}
\end{enumerate}

\subsubsection{Grade 11}

\begin{enumerate}
	\item \cite{Hung_nang_cao_phat_trien_Toan_11_tap_1}. Trần Quang Hùng, Lê Thị Việt Anh, Phạm Việt Hải, Khiếu Thị Hương, Tạ Công Sơn, Nguyễn Xuân Thọ, Ninh Văn Thu, Phạm Đình Tùng. \textit{Nâng Cao \& Phát Triển Toán 11 Tập 1}.\hfill{\sf[reading]}
	\item \cite{Hung_nang_cao_phat_trien_Toan_11_tap_2}. Trần Quang Hùng, Lê Thị Việt Anh, Phạm Việt Hải, Khiếu Thị Hương, Tạ Công Sơn, Nguyễn Xuân Thọ, Ninh Văn Thu, Phạm Đình Tùng. \textit{Nâng Cao \& Phát Triển Toán 11 Tập 2}.\hfill{\sf[reading]}
	\item \cite{Liem_Thang2020}. Nguyễn Xuân Liêm, Đặng Hùng Thắng. \textit{Bài Tập Nâng Cao \& 1 Số Chuyên Đề Đại Số \& Giải Tích 11}.\hfill{\sf[reading]}
	\item \cite{Tan2017}. Trần Văn Tấn. \textit{Bài Tập Nâng Cao \& Một Số Chuyên Đề Hình Học 11}.\hfill{\sf[reading]}
	\item \cite{TL_chuyen_Toan_hinh_hoc_11}. Đoàn Quỳnh, Phạm Khắc Ban, Văn Như Cương, Nguyễn Đăng Phất, Lê Bá Khánh Trình. \textit{Tài Liệu Chuyên Toán Hình Học 11}.\hfill{\sf[reading]}	
\end{enumerate}

\subsubsection{Grade 12}

\begin{enumerate}
	\item \cite{Kiselev_hhkg}. A. P. Kiselev. \textit{Hình Học Không Gian}.\hfill{\sf[reading]}
	\item \cite{TL_chuyen_Toan_giai_tich_12}. Đoàn Quỳnh (CB), Trần Nam Dũng, Hà Huy Khoái, Đặng Hùng Thắng, Nguyễn Trọng Tuấn. \textit{Tài Liệu Chuyên Toán Giải Tích 12}.\hfill{\sf[reading]}
	\item \cite{TL_chuyen_Toan_hinh_hoc_12}. Đoàn Quỳnh (CB), Hạ Vũ Anh, Phạm Khắc Ban, Văn Như Cương, Vũ Đình Hòa. \textit{Tài Liệu Chuyên Toán Hình Học 12}.\hfill{\sf[reading]}
\end{enumerate}

\subsubsection{Miscellaneous}

\begin{enumerate}
	\item \cite{Binh_HHTH}. Vũ Hữu Bình. \textit{Hình Học Tổ Hợp}.\hfill{\sf[reading]}
	\item \cite{Binh_PTNN}. Vũ Hữu Bình. \textit{Phương Trình Nghiệm Nguyên \& Kinh Nghiệm Giải}.\hfill{\sf[reading]}
	\item Võ Quốc Bá Cẩn, Trần Quốc Anh. \textit{Sử Dụng Phương Pháp AM--GM Để Chứng Minh Bất Đẳng Thức}.
	\item Võ Quốc Bá Cẩn, Trần Quốc Anh. \textit{Sử Dụng Phương Pháp Cauchy--Schwarz Để Chứng Minh Bất Đẳng Thức}.
	\item \cite{Dong_23_1001_toan_I}. Nguyễn Đức Đồng. \textit{23 Chuyên Đề Giải 1001 Bài Toán Sơ Cấp I: 12 Chuyên Đề Về Đại Số Sơ Cấp}.\hfill{\sf[reading]}
	\item \cite{Dong_23_1001_toan_II}. Nguyễn Đức Đồng. \textit{23 Chuyên Đề Giải 1001 Bài Toán Sơ Cấp II: 11 Chuyên Đề Về Toán Rời Rạc \& Hình Học Sơ Cấp}.\hfill{\sf[reading]}
	\item \cite{Khai_Huong_bdt}. Phan Huy Khải, Đoàn Thanh Hương. \textit{Các Phương Pháp Hiệu Quả Giải Bài Toán Về Bất Đẳng Thức \& Giá Trị Lớn Nhất Nhỏ Nhất}.\hfill{\sf[reading]}
	\item \cite{Viet2014}. Dương Quốc Việt. \textit{Những Tư Tưởng Cơ Bản Ẩn Chứa Trong Toán Học Phổ Thông}.\hfill{\sf[finished]}
	\item \cite{Dung_cac_phuong_phap_giai_toan_qua_cac_ky_thi_olympic_2022}. Trần Nam Dũng, Nguyễn Văn Huyện, Lê Phúc Lữ, Tống Hữu Nhân, Lương Văn Khải, Bùi Khánh Vĩnh, Nguyễn Công Thành, Nguyễn Nam, Trang Sĩ Trọng, Trần Bình Thuận, Trần Nguyễn Nam Hưng, Trương Tuấn Nghĩa, Đặng Cao Minh, Đào Trọng Toàn. \textit{Các Phương Pháp Giải Toán Qua Các Kỳ Thi Olympic}.\hfill{\sf[reading]}
	\item \cite{Chinh2021_tap_1}. Phan Đức Chính. \textit{Tuyển Tập Những Bài Toán Sơ Cấp Đại Số Tập 1}.\hfill{\sf[reading]}
	\item \cite{Chinh2021_tap_2}. Phan Đức Chính. \textit{Tuyển Tập Những Bài Toán Sơ Cấp Đại Số Tập 2}.\hfill{\sf[reading]}
	\item \cite{Huy2022}. Nguyễn Nhất Huy. \textit{Một Số Chủ Đề Số Học Hướng Tới Kỳ Thi HSG \& Chuyên Toán}.\hfill{\sf[reading]}
	\item \cite{Linh_topic_geometry}. Nguyễn Văn Linh. \textit{1 Số Chủ Đề Hình Học Phẳng}.\hfill{\sf[reading]}
	\item \cite{Linh_108_geometry}. Nguyễn Văn Linh. \textit{108 Bài Toán Hình Học Sơ Cấp}.\hfill{\sf[reading]}
	\item \cite{Nhan_8_geometry_theorem}. Tống Hữu Nhân. \textit{8 Định Lý Chọn Lọc Trong Hình Học Phẳng}.\hfill{\sf[reading]}
	\item \cite{Quy2022}. Bùi Quỹ. \textit{TikZ \& Vẽ Hình \LaTeX\ Vẽ Hình Toán Phổ Thông}.\hfill{\sf[reading]}
	\item Đặng Hùng Thắng, Nguyễn Văn Ngọc, Vũ Kim Thùy. \textit{Bài Giảng Số Học}.
	\item \cite{Son2006}. Đỗ Thanh Sơn. \textit{Chuyên Đề Bồi Dưỡng Học Sinh Giỏi Toán Trung Học Phổ Thông: Phép Biến Hình Trong Mặt Phẳng}.\hfill{\sf[reading]}	
\end{enumerate}

\subsection{Elementary Physics Book}

\subsubsection{Grade 7}

\begin{enumerate}
	\item \cite{Thinh_Lua_ncpt_Vat_Ly_7}. Bùi Gia Thịnh, Lê Thị Lụa, Nguyễn Thị Tâm. \textit{Nâng Cao \& Phát Triển Vật Lý 7}.\hfill{\sf[reading]}
\end{enumerate}

\subsubsection{Grade 8}

\begin{enumerate}
	\item \cite{SGK_Vat_Ly_8}. Vũ Quang, Bùi Gia Thịnh, Dương Tiến Khang, Vũ Trọng Rỹ, Trịnh Thị Hải Yến. \textit{Vật Lý 8}.\hfill{\sf[reading]}
	\item \cite{SBT_Vat_Ly_8}. Bùi Gia Thịnh, Dương Tiến Khang, Vũ Trọng Rỹ, \& Trịnh Thị Hải Yến. \textit{Bài Tập Vật Lý 8}.\hfill{\sf[reading]}
	\item \cite{Thinh_Lua_ncpt_Vat_Ly_8}. Bùi Gia Thịnh, Lê Thị Lụa. \textit{Nâng Cao \& Phát Triển Vật Lý 8}.\hfill{\sf[reading]}
\end{enumerate}

\subsubsection{Grade 9}

\begin{enumerate}
	\item \cite{SGK_Vat_Ly_9}. Vũ Quang, Đoàn Duy Hinh, Nguyễn Văn Hòa, Vũ Quang, Ngô Mai Thanh, Nguyễn Đức Thâm. \textit{Vật Lý 9}.\hfill{\sf[reading]}
	\item \cite{SBT_Vat_Ly_9}. Đoàn Duy Hinh, Nguyễn Văn Hòa, Vũ Quang, Ngô Mai Thanh, Nguyễn Đức Thâm. \textit{Bài Tập Vật Lý 9}.\hfill{\sf[reading]}
	\item \cite{Hoe_Vat_Ly_9}. Nguyễn Cảnh Hòe. \textit{Nâng Cao \& Phát Triển Vật Lý 9}.\hfill{\sf[reading]}
	\item \cite{Hoe_Hoach_Vat_Ly_nang_cao_9}. Nguyễn Cảnh Hòe, Lê Thanh Hoạch. \textit{Vật Lý Nâng Cao 9 Bồi Dưỡng Học Sinh Giỏi Thi Vào Lớp 10}.\hfill{\sf[reading]}
\end{enumerate}

\subsubsection{Secondary School -- Trung Học Cơ Sở [THCS]}

\begin{enumerate}
	\item \cite{Van_500_BT_Vat_Ly_THCS}. Phan Hoàng Văn. \textit{500 Bài Tập Vật Lý Trung Học Cơ Sở}.\hfill{\sf[reading]}
	\item \cite{Van_Quyen_Hanh_Nhu_10_chuyen_Ly}. Nguyễn Văn, Phan Thị Quyên, Bùi Thị Lý Hạnh, Phạm Thị Quỳnh Như. \textit{Giải Thích Chuyên Đề Thi Vào 10 Chuyên Lý}.\hfill{\sf[reading]}
	\item \cite{Vuong_10_chuyen_Ly}. Phạm Hồng Vương. \textit{Giải Thích Bộ Đề Thi Vào 10 Chuyên Lý}.\hfill{\sf[reading]}
\end{enumerate}

\subsubsection{Grade 10}

\begin{enumerate}
	\item \cite{Giang_Hang_Trung2022}. Tô Giang, Trần Thúy Hằng, Lê Minh Trung. \textit{Nâng Cao \& Phát Triển Vật Lý 10}.\hfill{\sf[reading]}
	\item Tô Giang. \textit{Tài liệu chuyên Vật lý. Vật lý 10. Tập 1}.
	\item Phạm Quý Tư, Nguyễn Đình Noãn. \textit{Tài liệu chuyên Vật lý. Vật lý 10. Tập 2}.
\end{enumerate}

\subsubsection{Grade 11}

\begin{enumerate}
	\item Vũ Thanh Khiết, Nguyễn Thế Khôi. \textit{Tài liệu chuyên Vật lý. Vật lý 11. Tập 1}.
	\item Vũ Quang. \textit{Tài liệu chuyên Vật lý. Vật lý 11. Tập 2}.
\end{enumerate}

\subsubsection{Grade 12}

\begin{enumerate}	
	\item Tô Giang, Vũ Thanh Khiết, Nguyễn Thế Khôi. \textit{Tài liệu chuyên Vật lý. Vật lý 12. Tập 1}.
	\item Vũ Quang, Vũ Thanh Khiết. \textit{Tài liệu chuyên Vật lý. Vật lý 12. Tập 2}.
	\item Tô Giang, Đặng Đình Tới, Bùi Trọng Tuân. \textit{Tài liệu chuyên Vật lý -- Bài tập Vật lý 10}.
	\item Lưu Hải An, Nguyễn Hoàng Kim, Vũ Thanh Khiết, Nguyễn Thế Khôi, Lưu Văn Xuân. \textit{Tài liệu chuyên Vật lý -- Bài tập Vật lý 11}.
	\item Tô Giang, Vũ Thanh Khiết, Đặng Đình Tới. \textit{Tài liệu chuyên Vật lý -- Bài tập Vật lý 12}.
	\item Đàm Trung Đồn. \textit{Tài liệu chuyên Vật lý -- Thực hành Vật lý Trung học phổ thông}.
	\item Tô Giang. \textit{Bồi Dưỡng Học Sinh Giỏi Vật Lý THPT: Cơ học 1}.\hfill{\sf[reading]}
	\item Tô Giang. \textit{Bồi Dưỡng Học Sinh Giỏi Vật Lý THPT: Cơ học 2}.\hfill{\sf[reading]}
	\item Tô Giang. \textit{Bồi Dưỡng Học Sinh Giỏi Vật Lý THPT: Cơ học 3}.\hfill{\sf[reading]}
	\item Vũ Thanh Khiết, Nguyễn Thế Khôi. \textit{Bồi Dưỡng Học Sinh Giỏi Vật Lý THPT: Điện học 1}.\hfill{\sf[reading]}
	\item Vũ Thanh Khiết, Tô Giang. \textit{Bồi Dưỡng Học Sinh Giỏi Vật Lý THPT: Điện học 2}.\hfill{\sf[reading]}
	\item Phạm Quý Tư. \textit{Bồi Dưỡng Học Sinh Giỏi Vật Lý THPT: Nhiệt Học \& Vật Lý Phân Tử}.
	\item Ngô Quốc Quỳnh. \textit{Bồi Dưỡng Học Sinh Giỏi Vật Lý THPT: Quang học 1}.\hfill{\sf[reading]}
	\item Vũ Quang. \textit{Bồi Dưỡng Học Sinh Giỏi Vật Lý THPT: Quang học 2}.
	\item \cite{Khiet_Vat_Ly_hien_dai}. Vũ Thanh Khiết. \textit{Bồi Dưỡng Học Sinh Giỏi Vật Lý Trung Học Phổ Thông: Vật Lý Hiện Đại}.\hfill{\sf[reading]}
	\item Phạm Văn Thiều. Đoàn Văn Ro, Nguyễn Văn Phán. \textit{Bồi Dưỡng Học Sinh Giỏi Vật Lý THPT: Phương pháp giải 1 số bài toán điển hình}.
	\item Phạm Văn Thiều. \textit{Bồi Dưỡng Học Sinh Giỏi Vật Lý THPT: Những bài toán tổng hợp: phân tích \& lời giải}.
	\item Bùi Quang Hân, Nguyễn Duy Hiền, Nguyễn Tuyến. \textit{Giải Toán \& Trắc Nghiệm Vật Lý 10. Tập 1: Cơ học}.
	\item Bùi Quang Hân, Nguyễn Duy Hiền, Nguyễn Tuyến. \textit{Giải Toán \& Trắc Nghiệm Vật Lý 10. Tập 2: Nhiệt học}.
	\item Bùi Quang Hân, Nguyễn Duy Hiền, Nguyễn Tuyến. \textit{Giải Toán \& Trắc Nghiệm Vật Lý 11. Tập 1: Tĩnh điện \& Dòng điện không đổi}.
	\item Bùi Quang Hân, Nguyễn Duy Hiền, Nguyễn Tuyến. \textit{Giải Toán \& Trắc Nghiệm Vật Lý 11. Tập 2: Điện từ \& Quang học}.
	\item Bùi Quang Hân, Nguyễn Duy Hiền, Nguyễn Tuyến. \textit{Giải Toán \& Trắc Nghiệm Vật Lý 12. Tập 1: Động lực học vật rắn, Dao động cơ, Sóng cơ}.
	\item Bùi Quang Hân, Nguyễn Duy Hiền, Nguyễn Tuyến. \textit{Giải Toán \& Trắc Nghiệm Vật Lý 12. Tập 2: Dao động \& sóng điện từ, Dòng điện xoay chiều}.
	\item Bùi Quang Hân, Nguyễn Duy Hiền, Nguyễn Tuyến. \textit{Giải Toán \& Trắc Nghiệm Vật Lý 12 -- Tập 3: Sóng ánh sáng, Lượng tử ánh sáng, Thuyết tương đối hẹp, Hạt nhân nguyên tử, Từ vi mô đến vĩ mô}.
	\item Vũ Thanh Khiết, Lưu Hải Ân, Phạm Vũ Kim Hoàng, Nguyễn Đức Hiệp, Nguyễn Hoàng Kim. \textit{Bồi Dưỡng Học Sinh Giỏi Vật Lý THPT: Bài Tập Điện Học -- Quang Học Vật Lý Hiện Đại}.
\end{enumerate}

\subsection{Elementary Chemistry Book}

\subsubsection{Grade 7}

\begin{enumerate}
	\item \cite{SGK_KHTN_7_Canh_Dieu}. Mai Sỹ Tuấn, Đinh Quang Báo, Nguyễn Văn Khánh, Đặng Thị Oanh, Nguyễn Văn Biên, Đào Tuấn Đạt, Phan Thị Thanh Hội, Ngô Văn Hưng, Đỗ Thanh Hữu, Đỗ Thị Quỳnh Mai, Phạm Xuân Quế, Trương Anh Tuấn, Ngô Văn Vụ. \textit{KHTN 7. Cánh Diều}.\hfill{\sf[reading]}
\end{enumerate}

\subsubsection{Grade 8}

\begin{enumerate}
	\item \cite{An_Hoa_Hoc_nang_cao_8_9}. Ngô Ngọc An. \textit{Hóa Học Nâng Cao Bồi Dưỡng Học Sinh Giỏi Các Lớp 8, 9}.\hfill{\sf[finished]}
	\item \cite{SBT_Hoa_Hoc_8}. Nguyễn Cương, Ngô Ngọc An, Đỗ Tất Hiển, Lê Xuân Trọng. \textit{Bài Tập Hóa Học 8}.\hfill{\sf[reading]}
	\item \cite{Giac2021}. Cao Cự Giác. \textit{Bồi Dưỡng Học Sinh Giỏi Hóa Học 8}.\hfill{\sf[reading]}
	\item Nguyễn Xuân Trường, Quách Văn Long, Hoàng Thị Thúy Hương. \textit{Các Chuyên Đề Bồi Dưỡng Học Sinh Giỏi Hóa Học 8}.
	\item \cite{Truong_BTNC_Hoa_Hoc_8_2022}. Nguyễn Xuân Trường. \textit{Bài Tập Nâng Cao Hóa Học 8}.\hfill{\sf[reading]}
	\item \cite{SGK_KHTN_8_Canh_Dieu}. Mai Sỹ Tuấn, Đinh Quang Báo, Nguyễn Văn Khánh, Đặng Thị Oanh, Nguyễn Thị Hồng Hạnh, Đỗ Thị Quỳnh Mai, Lê Thị Phượng, Phạm Xuân Quế, Dương Xuân Quý, Đào Văn Toàn, Trương Anh Tuấn, Lê Thị Tuyết, Ngô Văn Vụ. \textit{KHTN 8. Cánh Diều}.\hfill{\sf[reading]}
	\item \cite{SGK_KHTN_8_KNTTVCS}. Vũ Văn Hùng, Mai Văn Hưng, Lê Kim Long, Vũ Trọng Rỹ, Nguyễn Văn Biên, Nguyễn Hữu Chung, Nguyễn Thu Hà, Lê Trọng Huyền, Nguyễn Thế Hưng, Nguyễn Xuân Thành, Bùi Gia Thịnh, Nguyễn Thị Thuần, Mai Thị Tình, Vũ Thị Minh Tuyến, Nguyễn Văn Vịnh. \textit{KHTN 8. Kết Nối Tri Thức Với Cuộc Sống}.\hfill{\sf[reading]}
	\item \cite{SGK_Hoa_Hoc_8}. Lê Xuân Trọng, Nguyễn Cương, Đỗ Tất Hiển. \textit{Hóa Học 8}.\hfill{\sf[reading]}
\end{enumerate}

\subsubsection{Grade 9}

\begin{enumerate}	
	\item \cite{SGK_Hoa_Hoc_9}. Lê Xuân Trọng, Cao Thị Thặng, Ngô Văn Vụ. \textit{Hóa Học 9}.\hfill{\sf[reading]}
	\item \cite{SBT_Hoa_Hoc_9}. Lê Xuân Trọng, Ngô Ngọc An, Ngô Văn Vụ. \textit{Bài Tập Hóa Học 9}.\hfill{\sf[reading]}
	\item \cite{Truong_BTNC_Hoa_Hoc_9_2021}. Nguyễn Xuân Trường. \textit{Bài Tập Nâng Cao Hóa Học 9}.\hfill{\sf[reading]}
	\item \cite{Vu_Hoa2021}. Ngô Văn Vụ, Phạm Hồng Hoa. \textit{Nâng Cao \& Phát Triển Hóa Học 9}.\hfill{\sf[reading]}
\end{enumerate}

\subsubsection{Secondary School -- Trung Học Cơ Sở [THCS]}

\begin{enumerate}
	\item \cite{Tuan2022}. Vũ Anh Tuấn. \textit{Bồi Dưỡng Hóa Học THCS}.\hfill{\sf[reading]}
	\item \cite{Ninh_Chi_Khu_Lien_Thanh2019}. Trần Trung Ninh, Khiếu Thị Hương Chi, Lê Văn Khu, Trần Thị Kim Liên, Nguyễn Thị Kim Thành. \textit{$500$ Bài Tập Hóa Học Chuyên Trung Học Cơ Sở (Bồi Dưỡng Học Sinh Giỏi)}.\hfill{\sf[reading]}
	\item Nguyễn Đình Hành, Nguyễn Hữu Thọ. \textit{22 Chuyên Đề Hay \& Khó Bồi Dưỡng Học Sinh Giỏi Hóa Học THCS. Tập 1}.
\end{enumerate}

\subsubsection{Grade 10}

\begin{enumerate}
	\item \cite{Ha_Hai_Huyen_Tuan2022}. Nguyễn Thu Hà, Nguyễn Văn Hải, Lê Trọng Huyền, Vũ Anh Tuấn. \textit{Nâng Cao \& Phát Triển Hóa Học 10}.\hfill{\sf[reading]}
	\item \cite{Truong_Long_Huong_bdhsg_Hoa_Hoc_10}. Nguyễn Xuân Trường, Quách Văn Long, Hoàng Thị Thúy Hương. \textit{Bồi Dưỡng Học Sinh Giỏi Hóa Học 10 Theo Chuyên Đề}.\hfill{\sf[reading]}
	\item \cite{An_Hoa_Hoc_co_ban_nang_cao_10}. Ngô Ngọc An. \textit{Hóa Học Cơ Bản \& Nâng Cao 10}.\hfill{\sf[reading]}
	\item Đào Hữu Vinh, Nguyễn Duy Ái. \textit{Tài liệu chuyên Hóa học 10. Tập 2}.\hfill{\sf[reading]}
\end{enumerate}

\subsubsection{Grade 11}

\begin{enumerate}
	\item \cite{An_Hoa_Hoc_nang_cao_11}. Ngô Ngọc An. \textit{Hóa Học Nâng Cao 11}.\hfill{\sf[reading]}
	\item \cite{An_400_BT_Hoa_Hoc_11}. Ngô Ngọc An. \textit{400 Bài Tập Hóa Học 11}.\hfill{\sf[reading]}
\end{enumerate}

\subsubsection{Grade 12}

\begin{enumerate}	
	\item \cite{Son2021}. Trần Quốc Sơn. \textit{Tài Liệu Chuyên Hóa Học 11--12. Tập 1: Hóa Học Hữu Cơ}.\hfill{\sf[reading]}
	\item \cite{Ai2022}. Nguyễn Duy Ái. \textit{Tài Liệu Chuyên Hóa Học 11--12. Tập 2: Hóa Học Vô Cơ}.\hfill{\sf[reading]}
	\item \cite{SGK_Hoa_Hoc_12_co_ban}. Lê Xuân Trọng, Nguyễn Hữu Đĩnh, Từ Vọng Nghi, Đỗ Đình Răng, Cao Thị Thặng. \textit{Hóa Học 12}.\hfill{\sf[reading]}
	\item \cite{SGK_Hoa_Hoc_12_nang_cao}. Nguyễn Xuân Trường, Phạm Văn Hoan, Từ Vọng Nghi, Đỗ Đình Răng, Nguyễn Phú Tuấn. \textit{Hóa Học 12 Nâng Cao}.\\\mbox{}\hfill{\sf[reading]}
	\item \cite{Truong_Long_Huong_bdhsg_Hoa_Hoc_12}. Nguyễn Xuân Trường, Quách Văn Long, Hoàng Thị Thúy Hương. \textit{Bồi Dưỡng Học Sinh Giỏi Hóa Học 12 Theo Chuyên Đề}.\hfill{\sf[reading]}
\end{enumerate}

\subsubsection{High School -- THPT}

\begin{enumerate}
	\item \cite{An_chuoi_PUHH}. Ngô Ngọc An. \textit{Giúp Trí Nhớ Chuỗi Phản Ứng Hóa Học}.\hfill{\sf[reading]}
	\item Trần Quốc Sơn. \textit{Tài Liệu Chuyên Hóa Học THPT: Bài Tập Hữu Cơ. Tập 1}.
	\item Trần Quốc Sơn. \textit{Tài Liệu Chuyên Hóa Học THPT: Bài Tập Hữu Cơ. Tập 2}.
	\item Nguyễn Duy Ái, Nguyễn Tinh Dung, Trần Quốc Sơn, Nguyễn Văn Tòng. \textit{Bồi Dưỡng Học Sinh Giỏi Hóa Học THPT. Tập 3}.
	\item \cite{Lovebook2022}. Gia Đình Lovebook. \textit{Chinh Phục Đỉnh Cao Hóa Học Quốc Gia -- Quốc Tế}.\hfill{\sf[reading]}
\end{enumerate}

\subsection{Elementary Natural Science Book}

\subsubsection{Grade 6}

\begin{enumerate}
	\item \cite{SGK_KHTN_6_Canh_Dieu}. \textit{KHTN 6. Cánh Diều.}\hfill{\sf[finished]}
	\item \cite{ncpt_KHTN_6_tap_1}. Nguyễn Thu Hà, Trần Thúy Hằng, Lê Trọng Huyền, Nguyễn Thị Thu Hương. \textit{Nâng Cao \& Phát Triển KHTN 6 Tập 1}.\hfill{\sf[reading]}
	\item \cite{ncpt_KHTN_6_tap_2}. Hoàng Thị Đào, Trần Thúy Hằng, Vũ Thị Minh Tuyến. \textit{Nâng Cao \& Phát Triển KHTN 6 Tập 2}.\hfill{\sf[reading]}
\end{enumerate}

\subsubsection{Grade 7}

\begin{enumerate}	
	\item \cite{SGK_KHTN_7_Canh_Dieu}. \textit{KHTN 7}. Cánh Diều.\hfill{\sf[reading]}
	\item \cite{ncpt_KHTN_7_tap_1}. Nguyễn Thị Thanh Chi, Trần Thúy Hằng, Vũ Thị Minh Tuyến. \textit{Nâng Cao \& Phát Triển KHTN 7 Tập 1}. (Hóa Học $+$ Vật Lý).\hfill{\sf[reading]}
	\item \cite{ncpt_KHTN_7_tap_2}. Nguyễn Thanh Loan, Trương Thị Nhàn. \textit{Nâng Cao \& Phát Triển KHTN 7 Tập 2} (Sinh Học).\hfill{\sf[reading]}
\end{enumerate}

\subsubsection{Grade 8}

\begin{enumerate}
	\item \cite{SGK_KHTN_8_Canh_Dieu}. \textit{KHTN 8 Cánh Diều}.\hfill{\sf[reading]}
	\item \textit{KHTN 8 Chân Trời Sáng Tạo}.
	\item \cite{SGK_KHTN_8_KNTTVCS}. \textit{KHTN 8 Kết Nối Tri Thức với Cuộc Sống}.\hfill{\sf[reading]}
\end{enumerate}

\subsubsection{Grade 9}

\subsection{Elementary Computer Science}

\begin{enumerate}
	\item \cite{Duc_200_BT_Python}. Nguyễn Tiến Đức. \textit{Tuyển Tập 200 Bài Tập Lập Trình Bằng Ngôn Ngữ Python}.\hfill{\sf[reading]}
	\begin{itemize}
		\item Python source code $+$ input{\tt/}output files:\\\href{https://github.com/NQBH/hobby/tree/master/elementary_computer_science/Python/Duc_200_BTLT_Python}{GitHub{\tt/}NQBH{\tt/}hobby{\tt/}elementary computer science{\tt/}Python{\tt/}NTD 200 BTLT Python}.\footnote{\textsc{url}: \url{https://github.com/NQBH/hobby/tree/master/elementary_computer_science/Python/Duc_200_BTLT_Python}.}
	\end{itemize}
	\item \cite{Olympic30-4_2010_Tin_Hoc}. \textit{Tuyển Tập Đề Thi Olympic 30 Tháng 4, Lần Thứ XVI -- 2010 Tin học}.\hfill{\sf[reading]}
	\item \cite{SGK_Tin_Hoc_11}. Hồ Sĩ Đàm, Hồ Cẩm Hà, Trần Đỗ Hùng, Nguyễn Đức Nghĩa, Nguyễn Thanh Tùng, Ngô Ánh Tuyết. \textit{Tin Học 11}.\\\mbox{}\hfill{\sf[finished]}
	\item \cite{TL_chuyen_Tin_quyen_1}. Hồ Sĩ Đàm, Đỗ Đức Đông, Lê Minh Hoàng, Nguyễn Thanh Hùng. \textit{Tài Liệu Chuyên Tin Học Quyển 1}.\hfill{\sf[reading]}
	\item \cite{TL_chuyen_Tin_quyen_2}. Hồ Sĩ Đàm, Đỗ Đức Đông, Lê Minh Hoàng, Nguyễn Thanh Hùng. \textit{Tài Liệu Chuyên Tin Học Quyển 2}.\hfill{\sf[reading]}
	\item \cite{TL_chuyen_Tin_quyen_3}. Hồ Sĩ Đàm, Đỗ Đức Đông, Lê Minh Hoàng, Nguyễn Thanh Hùng. \textit{Tài Liệu Chuyên Tin Học Quyển 3}.\hfill{\sf[reading]}
	\item \cite{TL_chuyen_Tin_BT_quyen_1}. Hồ Sĩ Đàm, Đỗ Đức Đông, Lê Minh Hoàng, Nguyễn Thanh Hùng. \textit{Tài Liệu Chuyên Tin Học Bài Tập Quyển 1}.\\\mbox{}\hfill{\sf[reading]}
	\item \cite{TL_chuyen_Tin_BT_quyen_2}. Hồ Sĩ Đàm, Đỗ Đức Đông, Lê Minh Hoàng, Nguyễn Thanh Hùng. \textit{Tài Liệu Chuyên Tin Học Bài Tập Quyển 2}.\\\mbox{}\hfill{\sf[reading]}
	\item \cite{TL_chuyen_Tin_BT_quyen_3}. Hồ Sĩ Đàm, Đỗ Đức Đông, Lê Minh Hoàng, Nguyễn Thanh Hùng. \textit{Tài Liệu Chuyên Tin Học Bài Tập Quyển 3}.\\\mbox{}\hfill{\sf[reading]}
	\item \cite{Trung_THCS_Tin}. Vương Thành Trung. \textit{Tuyển Tập Đề Thi Học Sinh Giỏi Cấp Tỉnh Trung Học Cơ Sở \& Đề Thi Vào Lớp 10 Chuyên Tin Môn Tin Học}. {\sc url}: \url{https://github.com/NQBH/elementary_STEM_beyond/tree/main/elementary_computer_science/VTT_THCS}.\hfill{\sf[reading]}
	\item \cite{Trung_THPT_Tin}. Vương Thành Trung. \textit{Tuyển Tập Đề Thi Học Sinh Giỏi Trung Học Phổ Thông Môn Tin Học}. {\sc url}: \url{https://github.com/NQBH/elementary_STEM_beyond/tree/main/elementary_computer_science/VTT_THPT}.\hfill{\sf[reading]}
	\item \cite{Trung_HSG_THPT_Tin}. Vương Thành Trung. \textit{Tuyển Tập Đề Thi Học Sinh Giỏi Cấp Tỉnh Trung Học Phổ Thông Tin Học}.\hfill{\sf[reading]}
	\item \cite{VietSTEM2021}. Học Viện VietSTEM. \textit{Sách Luyện Thi Hội Thi Tin Học Trẻ  với Python Bảng B: Thi Kỹ Năng Lập Trình Cấp Trung Học Cơ Sở}.\hfill{\sf[reading]}
	\item \cite{VietSTEM2022}. Học Viện VietSTEM. \textit{Lập Trình với Python (Hành Trang Cho Tương Lai)}.\hfill{\sf[finished]}
\end{enumerate}

%------------------------------------------------------------------------------%

\section{Advanced STEM Book}

\subsection{Advanced Mathematics Book}

\begin{enumerate}
	\item \cite{Gessen2009}. Masha Gessen. \textit{Perfect Rigor: A Genius \& the Mathematical Breakthrough of the Century}.\hfill{\sf[finished]}
	\item \cite{Gessen2022}. Masha Gessen. \textit{Perfect Rigor: A Genius \& the Mathematical Breakthrough of the Century -- Thiên Tài Kỳ Dị \& Đột Phá Toán Học Của Thế Kỷ}.\hfill{\sf[reading]}
	\item \cite{Giang2019}. Nguyễn Ngọc Giang. \textit{Tích Hợp Toán, Tin, \& Vật Lý}.\hfill{\sf[reading]}
	\item \cite{Launay2022}. Micka\"el Launay. \textit{Toán Học: Một Thiên Tiểu Thuyết -- Lịch Sử Toán Học Kể Từ Thời Tiền Sử Đến Nay}.\hfill{\sf[finished]}
	\item \cite{Viet_Chua2022}. Dương Quốc Việt, Lê Văn Chua. \textit{Cơ Sở Lý Thuyết Galois}.\hfill{\sf[reading]}
	\item \cite{Viet_Ha_Thanh_Dang_Loc2022}. Dương Quốc Việt, Lê Thị Hà, Trương Thị Hồng Thanh, Nguyễn Đạt Đăng, Nguyễn Quang Lộc. \textit{Bài Tập Lý Thuyết Galois}.\hfill{\sf[reading]}
	\item Dương Quốc Việt. \textit{Cở Sở Lý Thuyết Module}.
	\item Nguyễn Xuân Liêm. \textit{Giải Tích Hàm}.
	\item Nguyễn Văn Khuê, Lê Mậu Hải. \textit{Giáo Trình Giải Tích Hàm}.
	\item Lê Mậu Hải, Tăng Văn Long. \textit{Bài Tập Giải Tích Hàm}.
	\item \cite{Viet_Nhi2022}. Dương Quốc Việt, Đàm Văn Nhỉ. \textit{Cơ Sở Lý Thuyết Số \& Đa Thức}.\hfill{\sf[reading]}
	\item \cite{Viet_Dang_Dinh_Ha_Hanh_Minh_Thanh_Thuy2022}. Dương Quốc Việt, Nguyễn Đạt Đăng, Lê Văn Đinh, Lê Thị Hà, Đặng Đình Hanh, Đào Ngọc Minh, Trương Thị Hồng Thanh, Phan Thị Thủy. \textit{Bài Tập Cơ Sở Lý Thuyết Số \& Đa Thức}.\hfill{\sf[reading]}
	\item Nguyễn Doãn Tuấn, Sĩ Đức Quang, Nguyễn Thị Thảo. \textit{Giáo Trình Hình Học Vi Phân}.
	\item Trần Văn Tấn. \textit{Hình Học của Nhóm Biến Đổi}.
	\item Nguyễn Văn Đoành. \textit{Đa Tạp Khả Vi}.
	\item Nguyễn Hữu Việt Hưng. \textit{Đại Số Tuyến Tính}.
	\item Trần Diên Hiền, Nguyến Tiến Tài, Nguyễn Văn Ngọc. \textit{Giáo Trình Lý Thuyết Số}.
	\item \cite{Quy_Liem2012}. Nguyễn Mạnh Quý, Nguyễn Xuân Liêm. \textit{Giáo Trình Phép Tính Vi Phân \& Tích Phân của Hàm 1 Biến Số: Phần Lý Thuyết}.\hfill{\sf[reading]}
	\item Đoàn Quỳnh. \textit{Hình Học Vi Phân}.
	\item Bùi Duy Hiền. \textit{Bài Tập Đại Số Đại Cương}.
	\item \cite{Nhi_Chin_Dung_Dung_Tinh_Dung_Son_Tuan2017}. Đàm Văn Nhỉ, Văn Đức Chín, Trần Thị Hồng Nhung, Lê Xuân Dũng, Trần Trung Tình, Đào Ngọc Dũng, Đặng Xuân Sơn, Nguyễn Anh Tuấn. \textit{Đa Thức -- Chuỗi \& Chuyên Đề Nâng Cao}.
	\item \cite{Hardy1940, Hardy1992, Hardy2022}. G. H. Hardy. \textit{A Mathematician's Apology}. [\href{https://github.com/NQBH/hobby/blob/master/advanced_mathematics/Hardy2017/NQBH_Hardy2017.pdf}{pdf}]. [\href{https://github.com/NQBH/hobby/blob/master/advanced_mathematics/Hardy2017/NQBH_Hardy2017.tex}{\TeX}].\hfill{\sf[finished]}
	\item \cite{Oakley2014}. Barbara Oakley. \textit{A Mind for Numbers: How to Excel at Math \& Science (Even If You Flunked Algebra)}.\hfill{\sf[reading]}
	\item \cite{Oakley2022}. Barbara Oakley. \textit{A Mind for Numbers: How to Excel at Math \& Science (Even If You Flunked Algebra) -- Cách Chinh Phục Toán \& Khoa Học (Ngay Cả Khi Bạn Vừa Trượt Môn Đại Số)}.\hfill{\sf[finished]}
	\item \cite{Oakley_Sejnowski_McConville2018}. Barbara Oakley, Terrence Sejnowski, Alistair McConville. \textit{Learning How to Learn: How to Succeed in School Without Spending All Your Time Studying; A Guide for Kids \& Teens}.\hfill{\sf[reading]}
	\item \cite{Oakley_Sejnowski_McConville2022}. Barbara Oakley, Terrence Sejnowski, Alistair McConville. \textit{Learning How to Learn: How to Succeed in School Without Spending All Your Time Studying; A Guide for Kids \& Teens -- Học Cách Học: Công Cụ Trí Tuệ Mạnh Mẽ Chinh Phục Mọi Môn Học}.\hfill{\sf[finished]}
\end{enumerate}

\subsubsection{Mathematical Analysis}

\begin{enumerate}
	\item \cite{Brezis2011}. Ha\"im Brezis. \textit{Functional Analysis, Sobolev Spaces, \& PDEs}.\hfill{\sf[reading]}
	\item \cite{Evans2010}. Lawrence C. Evans. \textit{Partial Differential Equations}.\hfill{\sf[reading]}
	\item \cite{Rudin1976}. Walter Rudin. \textit{Principles of Mathematical Analysis}.\hfill{\sf[finished]}
\end{enumerate}

\subsubsection{Finite Volume Method FVM}

\begin{enumerate}
	\item \cite{Eymard_Gallouet_Herbin2019}. Robert Eymard, Thierry Gallou\"et, Rapha\`ele Herbin. \textit{Finite Volume Methods}.\hfill{\sf[reading]}
	\item T. Gallou\"et, Rapha\`ele Herbin, J.-C. Latch\'e. \textit{Convergence of the Marker-\&-Cell Scheme for the Incompressible Navier--Stokes Equations on Non-uniform Grids}.\hfill{\sf[reading]}
\end{enumerate}

\subsubsection{Optimal Control}

\begin{enumerate}
	\item \cite{Hintermueller_Kroener2023}. Michael Hinterm\"uller, Axel Kr\"oner. \textit{Differentiability properties for boundary control of fluid-structure interactions of linear elasticity with Navier-Stokes equations with mixed-boundary conditions in a channel}.\hfill{\sf[finished]}
\end{enumerate}

\subsubsection{Shape Optimization}

\begin{enumerate}
	\item \cite{Bandle_Wagner2023}. Catherine Bandle, Alfred Wagner. \textit{Shape Optimization: Variations of Domains \& Applications}.\hfill{\sf[reading]}
	\item \cite{Haubner_Siebenborn_Ulbrich2021}. Johannes Haubner, Martin Siebenborn, Michael Ulbrich. \textit{A continuous perspective on shape optimization via domain transformations}.\hfill{\sf[finished]}
	\item \cite{Haubner_Ulbrich_Ulbrich2020}. Johannes Haubner, Michael Ulbrich, Stefan Ulbrich. \textit{Analysis of shape optimization problems for unsteady fluid-structure interaction}.\hfill{\sf[finished]}
	\item \cite{Hiptmair_Paganini_Sargheini2015}. R. Hiptmair, A. Paganini, S. Sargheini. \textit{Comparison of approximate shape gradients}.\hfill{\sf[finished]}
\end{enumerate}

\subsubsection{Turbulence}

\begin{enumerate}
	\item \cite{Perron_Boivin_Herard2004}. \textit{A FVM to solve the 3{D} NSEs on Unstructured Collocated Meshes}.\hfill{\sf[reading]}
\end{enumerate}

%------------------------------------------------------------------------------%

\subsection{Advanced Physics Book}

\begin{enumerate}
	\item Lương Duyên Bình, Nguyễn Hữu Hồ, Lê Văn Nghĩa, Nguyễn Quang Bình. \textit{Bài Tập Vật Lý Đại Cương. Tập 2: Điện -- Dao Động -- Sóng}.
	\item \cite{Einstein_Infeld_tien_hoa_Vat_Ly}. Albert Einstein, Leopold Infeld. \textit{The Evolution of Physics: From Early Concepts to Relativity \& Quanta -- Sự Tiến Hóa Của Vật Lý: Từ Những Khái Niệm Ban Đầu Đến Thuyết Tương Đối \& Lượng Tử}.\hfill{\sf[finished]}
	\item Vũ Văn Hùng. \textit{Cơ Học Lượng Tử}.
	\item Vũ Văn Hùng. \textit{Bài Tập Cơ Học Lượng Tử}.
	\item Nguyễn Quang Học, Đinh Quang Vinh. \textit{Bài Tập Vật Lý Lý Thuyết 2. Tập 2: Vật Lý Thống Kê}.
	\item Nguyễn Quang Học, Vũ Văn Hùng. \textit{Giáo Trình Vật Lý Thống Kê \& Nhiệt Động Lực Học. Tập 1: Nhiệt Động Lực Học}.
	\item \cite{Hawking_bbc}. Stephen Hawking. \textit{Black Holes: The BBC Reith Lectures -- Lỗ Đen: Các Bài Giảng Trên Đài}.\hfill{\sf[finished]}
	\item \cite{Hawking_lstg}. Stephen Hawking. \textit{A Brief History of Time -- Lược Sử Thời Gian}.\hfill{\sf[finished]}
	\item \cite{Hawking_vttvhd}. Stephen Hawking. \textit{The Universe In A Nutshell -- Vũ Trụ Trong Vỏ Hạt Dẻ}.\hfill{\sf[finished]}
	\item Đào Văn Phúc. \textit{Lịch Sử Vật Lý Học}.
\end{enumerate}

%------------------------------------------------------------------------------%

\subsection{Advanced Chemistry Book}

\begin{enumerate}
	\item Hoàng Nhâm. \textit{Hóa Học Vô Cơ Cơ Bản. Tập 1: Lý Thuyết Đại Cương về Hóa Học}.
	\item Hoàng Nhâm. \textit{Hóa Học Vô Cơ Cơ Bản. Tập 2: Các Nguyên Tố Hóa Học Điển Hình}.
	\item Hoàng Nhâm. \textit{Hóa Học Vô Cơ Cơ Bản. Tập 3: Các Nguyên Tố Chuyển Tiếp}.
	\item Hoàng Nhâm. \textit{Bài Tập Hóa Học Vô Cơ}.
	\item Hoàng Nhâm, Hoàng Nhuận. \textit{Bài Tập Hóa Học Vô Cơ. Quyển I $+$ II: Lý Thuyết Đại Cương về Hóa Học}.
	\item Hoàng Nhâm, Hoàng Nhuận. \textit{Bài Tập Hóa Học Vô Cơ. Quyển III: Hóa Học Các Nguyên Tố}.
	\item Hoàng Nhâm. \textit{Hóa Học Vô Cơ Nâng Cao. Tập 1: Lý Thuyết Đại Cương về Hóa Học}.
	\item Hoàng Nhâm. \textit{Hóa Học Vô Cơ Nâng Cao. Tập 2: Các Nguyên Tố Hóa Học Tiêu Biểu}.
	\item Hoàng Nhâm. \textit{Hóa Học Vô Cơ Nâng Cao. Tập 3: Các Nguyên Tố Chuyển Tiếp}.
	\item Đào Đình Thức. \textit{Cấu Tạo Nguyên Tử \& Liên Kết Hóa Học. Tập 1}.
	\item Đào Đình Thức. \textit{Cấu Tạo Nguyên Tử \& Liên Kết Hóa Học. Tập 2}.
	\item Đỗ Đình Răng, Đặng Đình Bạch, Lệ Thị Anh Đào, Nguyễn Mạnh Hà, Nguyễn Thị Thanh Phong. \textit{Hóa Học Hữu Cơ 3}.
	\item Trần Thành Huế, Nguyễn Ngọc Hà. \textit{Đối Xứng Phân Tử \& Lý Thuyết Nhóm Trong Hóa Học}.
\end{enumerate}

%------------------------------------------------------------------------------%

\subsection{Advanced Computer Science}

\begin{enumerate}
	\item \cite{Chacon_Straub2014}. Scott Chacon, Ben Straub. \textit{Pro Git: Everything You Need to Know About Git}.\hfill{\sf[reading]}
	\item \cite{Durr_Vie2021}. Christoph D\"urr, Jill-J\^enn Vie. \textit{Competitive Programming in Python: 128 Algorithms to Develop Your Coding Skills}.\\\mbox{}\hfill{\sf[reading]}
	\item \cite{Ha_Python_co_ban}. Bùi Việt Hà. \textit{Python Cơ Bản}.\hfill{\sf[finished]}
	\item \cite{Ha_loi_giai_BT_Python_co_ban}. Bùi Việt Hà. \textit{Lời Giải Bài Tập Python Cơ Bản}.\hfill{\sf[reading]}
	\item \cite{Ha_Python_nang_cao}. Bùi Việt Hà. \textit{Python Nâng Cao}.\hfill{\sf[finished]}
	\item \cite{Hien_DevUp}. Nguyễn Hiền. \textit{DevUP}.\hfill{\sf[finished]}
	\item \cite{Hoang_code_dao_ky_su}. Phạm Huy Hoàng. \textit{Code Dạo Ký Sự: Lập Trình Viên Đâu Phải Chỉ Biết Code}.\hfill{\sf[finished]}
	\item \cite{Hoang_toi_di_code_dao}. Phạm Huy Hoàng. \textit{Hello Các bạn, Mình Là Tôi Đi Code Dạo: Chuyện Code, Chuyện Nghề, Chuyện Đời}.\hfill{\sf[finished]}
	\item \cite{Knuth1997}. Donald E. Knuth. \textit{The Art of Computer Programming. Volume 1: Fundamental Algorithms}.\hfill{\sf[reading]}
	\item \cite{Knuth1998}. Donald E. Knuth. \textit{The Art of Computer Programming. Volume 3: Sorting \& Searching}.\hfill{\sf[reading]}
	\item \cite{Laaksonen2020}. Antti Laaksonen. \textit{Guide to Competitive Programming: Learning \& Improving Algorithms Through Contests}.\hfill{\sf[reading]}
	\item \cite{LeCun_Bengio_Hinton2015}. Yann LeCun, Yoshua Bengio, Geoffrey Hinton. \textit{Deep Learning}.\hfill{\sf[reading]}
	\item \cite{Matthes2019}. Eric Matthes. \textit{Python Crash Course: A Hands-on, Project-Based Introduction to Programming}. 2e.\hfill{\sf[reading]}
	\item \cite{Matthes2023}. Eric Matthes. \textit{Python Crash Course: A Hands-on, Project-Based Introduction to Programming}. 3e.\hfill{\sf[reading]}
	\item \cite{Ngoc_Pascal}. Quách Tuấn Ngọc. \textit{Ngôn Ngữ Lập Trình Pascal}.\hfill{\sf[reading]}
	\item \cite{Ngoc_BT_Pascal}. Quách Tuấn Ngọc. \textit{Bài Tập Ngôn Ngữ Lập Trình Pascal}.\hfill{\sf[reading]}
	\item \cite{Ngoc_C}. Quách Tuấn Ngọc. \textit{Ngôn Ngữ Lập Trình C}.\hfill{\sf[reading]}
	\item \cite{Ngoc_C++}. Quách Tuấn Ngọc. \textit{Ngôn Ngữ Lập Trình C++}.\hfill{\sf[finished]}
	\item \cite{Shotts2019}. William Shotts. \textit{The Linux Command Line: A Complete Introduction}.\hfill{\sf[reading]}
	\item \cite{Stroustrup2013}. Bjarne Stroustrup. \textit{The C++ Programming Language, 4th edition}.\hfill{\sf[reading]}
	\item \cite{Stroustrup2018}. Bjarne Stroustrup. \textit{A Tour of C++, 2nd edition}.\hfill{\sf[reading]}
	\item \cite{Thu_Phuong_Tien_Triet_NMLT}. Trần Đan Thư, Nguyễn Thanh Phương, Đinh Bá Tiến, Trần Minh Triết. \textit{Nhập Môn Lập Trình}.\hfill{\sf[reading]}
	\item \cite{Thu_Phuong_Tien_Triet_Phuong_KTLT}. Trần Đan Thư, Nguyễn Thanh Phương, Đinh Bá Tiến, Trần Minh Triết, Đặng Bình Phương. \textit{Kỹ Thuật Lập Trình}.\hfill{\sf[reading]}
	\item \cite{Thu_Tien_Khang_PPLTHDT}. Trần Đan Thư, Đinh Bá Tiến, Nguyễn Tấn Trần Minh Khang. \textit{Phương Pháp Lập Trình Hướng Đối Tượng}.\hfill{\sf[reading]}
\end{enumerate}

%------------------------------------------------------------------------------%

\section{Literary Book}

\begin{enumerate}
	\item \cite{Can_dtnv}. Nguyễn Duy Cần -- Thu Giang. \textit{Để Thành Nhà Văn}.\hfill{\sf[finished]}
	\item \cite{Can_oss}. Nguyễn Duy Cần -- Thu Giang. \textit{Óc Sáng Suốt}.\hfill{\sf[reading]}
	\item \cite{Can_ttt}. Nguyễn Duy Cần -- Thu Giang. \textit{Thuật Tư Tưởng}.\hfill{\sf[reading]}
	\item \cite{Can_tth}. Nguyễn Duy Cần -- Thu Giang. \textit{Tôi Tự Học}.\hfill{\sf[reading]}
	\item \cite{Chip2018}. Huyền Chip. \textit{Giấc Mơ Mỹ -- Đường Đến Stanford}.\hfill{\sf[finished]}
	\item \cite{Coelho2023}. Paul Coelho. \textit{Nhà Giả Kim}.\hfill{\sf[finished]}
	\item \cite{King2000, King2010}. Stephen King. \textit{On Writing: On Writing: A Memoir of the Craft}.\hfill{\sf[finished]}
	\item \cite{Murakami2023}. Haruki Murakami. \textit{Tôi Nói Gì Khi Nói Về Chạy Bộ}\footnote{Haruki Murakami dùng rất rất nhiều cụm ``dù sao''. \textit{Why?}}.\hfill{\sf[finished]}
	\item \cite{Rand_fountainhead}. Ayn Rand. \textit{The Fountainhead -- Suối Nguồn}.\hfill{\sf[finished]}
	\item \cite{Rosie2021a}. Rosie Nguyễn. \textit{Ta Ba Lô Trên Đất Á}.\hfill{\sf[finished]}
	\item \cite{Rosie2021b}. Rosie Nguyễn. \textit{Trên Hành Trình Tự Học}.\hfill{\sf[finished]}
	\item \cite{Rosie2022a}. Rosie Nguyễn. \textit{Mình Nói Gì Khi Nói Về Hạnh Phúc?}.\hfill{\sf[finished]}
	\item \cite{Rosie2022b}. Rosie Nguyễn. \textit{Tuổi Trẻ Đáng Giá Bao Nhiêu?}.\hfill{\sf[finished]}
	\item \cite{Salinger_btdx}. J. D. Salinger. \textit{The Catcher In The Rye -- Bắt Trẻ Đồng Xanh}.\hfill{\sf[finished]}
	\item \cite{Shapiro2014}. Dani Shapiro. \textit{Still Writing: The Perils \& Pleasures of a Creative Life}.\hfill{\sf[reading]}
	\item \cite{Strunk1918}. William Strunk Jr. \textit{The Elements of Style}.\hfill{\sf[finished]}
	\item \cite{Strunk_White2019}. William Strunk Jr, E. B. White. \textit{The Elements of Style}.\hfill{\sf[finished]}
	\item \cite{Van2022}. Nguyễn Phương Văn. \textit{Mặt Trời Trong Suối Lạnh}.\hfill{\sf[finished]}
	\item \cite{VanVu2022}. Vũ Hà Văn. \textit{Giáo Sư Phiêu Lưu Ký: Tản Mạn với 1 Nhà Toán Học}.\hfill{\sf[finished]}
	\item \cite{Vasconcelos_orange_tree}. Jos\'e Mauro de Vasconcelos. \textit{My Sweet Orange Tree -- Cây Cam Ngọt Của Tôi}.\hfill{\sf[finished]}
	\item \cite{Wallace2009}. David Foster Wallace. \textit{This Is Water: Some Thoughts, Delivered on a Significant Occasion, about Living a Compassionate Life}.\hfill{\sf[finished]}
	\item \cite{Wallace2011}. David Foster Wallace. \textit{Infinite Jest}.\hfill{\sf[reading]}
	\item \cite{Zinsser2005}. William Zinsser. \textit{Writing About Your Life: A Journey into the Past}.
	\item \cite{Zinsser2016}. William Zinsser. \textit{On Writing Well: The Classic Guide to Writing Nonfiction}.\hfill{\sf[reading]}
\end{enumerate}

%------------------------------------------------------------------------------%

\section{Psychology Book}
Về các cuốn sách tâm lý, mình có nên liệt kê chúng theo thứ tự hay dần\texttt{/}theo chiều tăng của sự tâm đắc cá nhân, riêng những cuốn đang mua chưa đọc sẽ tạm để ở cuối danh sách, sau khi đọc 1 phần\texttt{/}xong đủ để đánh giá mức độ hay của những cuốn sách đó thì mình sẽ sắp thứ tự sau. Chỉ riêng sách Văn Học, Tâm Lý \& Triết Học mới được áp dụng cách liệt kê này, đặc biệt không áp dụng (được) cho các sách STEM -- đơn giản vì chúng hay theo nhiều lĩnh vực khác nhau, nên không thể nào sắp duy nhất 1 thứ tự trên 1 tập hợp bán thứ tự được (poset -- partial ordering set)?
\begin{enumerate}
	\item \cite{Adler2013}. Alfred Adler. \textit{The Science of Living}.\hfill{\sf[finished]}
	\item \cite{Ariely_reasonably_irrational}. Dan Ariely. \textit{Phi Lý Trí Một Cách Hợp Lý: Câu Trả Lời Hài Hước Cho Những Hiện Tượng Tâm Lý Kỳ Quặc}.\hfill{\sf[finished]}
	\item \cite{Ariely_predictably_irrational}. Dan Ariely. \textit{Predictably Irrational: The Hidden Forces That Shape Our Decisions -- Phi Lý Trí: Khám Phá Những Động Lực Vô Hình Ẩn Sau Những Quyết Định Của Con Người}.\hfill{\sf[finished]}
	\item \cite{Ariely_upside_rationality}. Dan Ariely. \textit{The Upside of Irrationality: The Unexpected Benefits of Defying Logic -- Lẽ Phải Của Phi Lý Trí: Lợi Ích Bất Ngờ Của Việc Phá Bỏ Những Quy Tắc Logic Trong Công Việc \& Cuộc Sống}.\hfill{\sf[finished]}
	\item \cite{Aron2013}. Elaine N. Aron. \textit{The Highly Sensitive Person: How to Thrive When the World Overwhelms You}.\hfill{\sf[finished]}
	\item \cite{Bancroft2003}. Lundy Bancroft. \textit{Why Does He Do That?}.\hfill{\sf[reading]}
	\item \cite{Bancroft2019}. Lundy Bancroft. \textit{Why Does He Do That? -- Tại Sao Anh Ta Làm Thế? Giải Mã Tâm Lý Kẻ Bạo Hành}.\hfill{\sf[finished]}
	\item \cite{Bon2022}. Gustave Le Bon. \textit{Tâm Lý Học Đám Đông}.\hfill{\sf[finished]}
	\item \cite{Cain2013}. Susan Cain. \textit{Quiet: The Power of Introverts in a World That Can't Stop Talking}.\hfill{\sf[reading]}
	\item \cite{Cain2022}. Susan Cain. \textit{Quiet: The Power of Introverts in a World That Can't Stop Talking -- Hướng Nội: Sức Mạnh của Sự Yên Lặng Trong 1 Thế Giới Nói Không Ngừng}.\hfill{\sf[finished]}
	\item \cite{Cain_Mone_Moroz2017}. Susan Cain, Gregory Mone, Erica Moroz. \textit{Quiet Power: The Secret Strengths of Introverted Kids}.\hfill{\sf[reading]}
	\item \cite{Cain_Mone_Moroz2023}. Susan Cain, Gregory Mone, Erica Moroz. \textit{Quiet Power: The Secret Strengths of Introverted Kids -- Trầm Lặng: Sức Mạnh Tiềm Ẩn Của Người Hướng Nội}.\hfill{\sf[finished]}
	\item \cite{Carnegie2021}. Dale Carnegie. \textit{How to Win Friends \& Influence People -- Đắc Nhân Tâm}.\hfill{\sf[finished]}
	\item \cite{Chi2022}. Chi, Nguyễn (The Present Writer). \textit{Một Cuốn Sách về Chủ Nghĩa Tối Giản}. \hfill{\sf[finished]}
	\item \cite{Clear2018}. James Clear. \textit{Atomic Habits; An Easy \& Proven Way to Build Good Habits \& Break Bad Ones}.\hfill{\sf[reading]}
	\item \cite{Clear2022}. James Clear. \textit{Atomic Habits; An Easy \& Proven Way to Build Good Habits \& Break Bad Ones -- Thay Đổi Tí Hon, Hiệu Quả Bất Ngờ: Tạo Thói Quen Tốt, Bỏ Thói Quen Xấu Bằng Phương Pháp Đơn Giản mà Hiệu Quả}.\hfill{\sf[finished]}
	\item Charles Duhigg. \textit{The Power of Habit: Why We Do What We Do in Life \& Business}.
	\item \cite{Csikszentmihalyi2008}. Mihaly Csikszentmihalyi. \textit{Flow: The Psychology of Optimal Experience}.\hfill{\sf[reading]}
	\item \cite{Csikszentmihalyi2013}. Mihaly Csikszentmihalyi. \textit{Creativity: Flow \& the Psychology of Discovery \& Invention}.\hfill{\sf[reading]}
	\item \cite{Csikszentmihalyi2021}. Mihaly Csikszentmihalyi. \textit{Flow: The Psychology of Optimal Experience -- Dòng Chảy: Tâm Lý Học Hiện Đại Trải Nghiệm Tối Ưu}.\hfill{\sf[finished]}
	\item \cite{Dweck_mindset}. Carol S. Dweck. \textit{Mindset: The New Psychology of Success -- Tâm Lý Học Thành Công: Sức Mạnh Của Niềm Tin Phát Huy Tiềm Năng Của Chúng Ta Như Thế Nào}.\hfill\hfill{\sf[finished]}
	\item \cite{DK2018}. DK. \textit{How Psychology Works: The Facts Visually Explained (How Things Work)}.\hfill{\sf[reading]} 
	\item \cite{Eun-Jung2023}. Yoo Eun-Jung. \textit{Không Ai Có Thể Làm Bạn Tổn Thương Trừ Khi Bạn Cho Phép}.\hfill{\sf[finished]} 
	\item \cite{Giang2022a}. Đặng Hoàng Giang. \textit{Điểm Đến Của Cuộc Đời: Đồng Hành Với Người Cận Tử \& Những Bài Học Cho Cuộc Sống}.\\\mbox{}\hfill{\sf[finished]}
	\item \cite{Giang2022b}. Đặng Hoàng Giang. \textit{Bức Xúc Không Làm Ta Vô Can}.\hfill{\sf[finished]}
	\item \cite{Giang2022c}. Đặng Hoàng Giang. \textit{Thiện, Ác \& Smart Phone}.\hfill{\sf[finished]}
	\item \cite{Giang2022d}. Đặng Hoàng Giang. \textit{Tìm Mình Trong Thế Giới Hậu Tuổi Thơ}.\hfill{\sf[finished]}
	\item \cite{Giang2023}. Đặng Hoàng Giang. \textit{Đại Dương Đen: Những Câu Chuyện Từ Thế Giới Của Trầm Cảm}.\hfill{\sf[finished]}
	\item \cite{Gladwell2007}. Malcolm Gladwell. \textit{Blink: The Power of Thinking Without Thinking}.\hfill{\sf[reading]}
	\item \cite{Gladwell_blink}. Malcolm Gladwell. \textit{Blink: The Power of Thinking Without Thinking -- Trong Chớp Mắt: Sức Mạnh Của Việc Nghĩ Mà Không Cần Suy Nghĩ}.\hfill{\sf[finished]}
	\item \cite{Gladwell2008}. Malcolm Gladwell. \textit{Outliers: The Story of Success}.\hfill{\sf[reading]}
	\item \cite{Gladwell_outlier}. Malcolm Gladwell. \textit{Outliers: The Story of Success -- Những Kẻ Xuất Chúng: Cái Nhìn Mới Lạ Về Nguồn Gốc Của Thành Công}.\hfill{\sf[finished]}
	\item \cite{Gladwell2009}. Malcolm Gladwell. \textit{What The Dog Saw: And Other Adventures}.\hfill{\sf[reading]}
	\item \cite{Gladwell_dog}. Malcolm Gladwell. \textit{What The Dog Saw: And Other Adventures -- Chú Chó Nhìn Thấy Gì?: Lật Tẩy Những Góc Khuất Trong Cuộc Sống Xã Hội}.\hfill{\sf[finished]}
	\item \cite{Gladwell2019}. Malcolm Gladwell. \textit{Talking to Strangers: What We Should Know about the People We Don't Know}.\hfill{\sf[reading]}
	\item \cite{Gladwell_stranger}. Malcolm Gladwell. \textit{Talking to Strangers: What We Should Know about the People We Don't Know -- Đọc Vị Người Lạ: Điều Ta Nên Biết Về Những Người Không Quen Biết}.\hfill{\sf[finished]}
	\item \cite{Gladwell2021}. Malcolm Gladwell. \textit{The Bomber Mafia: A Dream, a Temptation, \& the Longest Night of the 2nd World War}.\hfill{\sf[reading]}
	\item \cite{Gladwell_bomber_mafia}. Malcolm Gladwell. \textit{The Bomber Mafia: A Dream, a Temptation, \& the Longest Night of the Second World War -- The Bomber Mafia: Giấc Mơ, Cám Dỗ, \& Đêm Dài Nhất Trong Thế Chiến II}.\hfill{\sf[finished]}
	\item \cite{Gladwell2022}. Malcolm Gladwell. \textit{The Tipping Point: How Little Things Can Make a Big Difference}.\hfill{\sf[reading]}
	\item \cite{Gladwell_tipping_point}. \textit{The Tipping Point: How Little Things Can Make a Big Difference -- Điểm Bùng Phát: Làm Thế Nào Những Điều Nhỏ Bé Tạo Nên Sự Khác Biệt Lớn Lao?}.\hfill{\sf[finished]}
	\item \cite{Grant2013}. Adam Grant. \textit{Give \& Take: A Revolutionary Approach to Success}.\hfill{\sf[reading]}
	\item \cite{Grant2020}. Adam Grant. \textit{Originals: How Non-Conformists Move the World -- Tư Duy Ngược Dịch Chuyển Thế Giới}.\hfill{\sf[finished]}
	\item \cite{Grant2022a}. Adam Grant. \textit{Give \& Take: Why Helping Others Drives Our Success -- Cho \& Nhận: Vì Sao Giúp Người Đưa Ta Đến Thành Công?}.\hfill{\sf[finished]}
	\item \cite{Grant2022b}. Adam Grant. \textit{Think Again: The Power of Knowing What You Don't Know -- Dám Nghĩ Lại: Sức Mạnh của Việc Biết Mình Không Biết}.\hfill{\sf[finished]}
	\item \cite{Greene_laws_power}. Robert Greene. \textit{The 48 Laws of Power -- Nguyên Tắc Chủ Chốt Của Quyền Lực}.\hfill{\sf[finished]}
	\item \cite{Greene_laws_human_nature}. Robert Greene. \textit{The Laws of Human Nature -- Những Quy Luật Của Bản Chất Con Người}.\hfill{\sf[finished]}
	\item \cite{Hare1999}. Robert D. Hare. \textit{Without Conscience: The Disturbing World of the Psychopaths Among Us}.\hfill{\sf[reading]}
	\item \cite{Ichiro_Fumitake2022a}. Kishimi Ichiro, Koga Fumitake. \textit{Dám Bị Ghét}.\hfill{\sf[finished]}
	\item \cite{Ichiro_Fumitake2022b}. Kishimi Ichiro, Koga Fumitake. \textit{Dám Hạnh Phúc}.\hfill{\sf[finished]}
	\item \cite{Jung2022}. Carl Gustav Jung. \textit{Man \& His Symbols -- Con Người \& Biểu Tượng: Sự Thông Đạt Từ Những Biểu Tượng Trong Giấc Mơ}.\hfill{\sf[finished]}
	\item \cite{Kahneman2022}. Daniel Kahneman. \textit{Thinking, Fast \& Slow -- Tư Duy Nhanh \& Chậm: Nên Hay Không Nên Tin Vào Trực Giác?}.\\\mbox{}\hfill{\sf[finished]}
	\item \cite{Kahnweiler2022}. Jennifer B. Kahnweiler. \textit{Quiet Influence -- Sức Mạnh của Sự Trầm Lắng -- The Introvert's Guide to Making a Difference}.\hfill{\sf[finished]}
	\item Harold S. Kushner. \textit{When Bad Things Happen to Good People}.
	\item \cite{Little2017}. Brian R. Little. \textit{Who Are You, Really? The Surprising Puzzle of Personality}.\hfill{\sf[finished]}
	\item \cite{Little2023}. Brian R. Little. \textit{Who Are You, Really? The Surprising Puzzle of Personality -- Bạn Thật Sự Là Ai? Khám Phá Đáng Kinh Ngạc Về Tính Cách Con Người}.\hfill{\sf[finished]}
	\item \cite{Long2021}. Vũ Hoàng Long (chủ biên). \textit{Học Trường Chuyên -- Những Góc Nhìn Đa Chiều}.\hfill{\sf[finished]}
	\item \cite{MacKenzie2015}. Jackson MacKenzie. \textit{Psychopath Free: Recovering From Emotionally Abusive Relationships with Narcissists, Sociopaths, \& Other Toxic People}.\hfill{\sf[finished]}
	\item \cite{Manson_giving_fuck}. Mark Manson. \textit{The Subtle Art of Not Giving A F*ck: A Counterintuitive Approach to Living a Good Life}.\hfill{\sf[reading]}
	\item \cite{Manson_giving_fuck_vn}. Mark Manson. \textit{The Subtle Art of Not Giving A F*ck: A Counterintuitive Approach to Living a Good Life -- Nghệ Thuật Tinh Tế của Việc ``Đếch'' Quan Tâm: Một Cách Tiếp Cận Khác Thường Để Sống Tốt}.\hfill{\sf[finished]}
	\item Amy Mariaskin. \textit{Phát Triển Các Mối Quan Hệ Khi Mắc OCD}.
	\item \cite{McRaney2022a}. David McRaney. \textit{Bạn Không Thông Minh Lắm Đâu}.\hfill{\sf[finished]}
	\item \cite{McRaney2022b}. David McRaney. \textit{Bạn Đỡ Ngu Ngơ Rồi Đấy}.\hfill{\sf[finished]}
	\item \cite{Minh2022}. Cao Minh. \textit{Thiên Tài Bên Trái, Kẻ Điên Bên Phải}.\hfill{\sf[finished]}
	\item \cite{Mirza2017}. Debbie Mirza. \textit{The Covert Passive Aggressive Narcissist: Recognizing the Traits \& Finding Healing After Hidden Emotional \& Psychological Abuse}.\hfill{\sf[finished]}
	\item \cite{Murphy2011}. Joseph Murphy. \textit{The Power of Subconscious Mind}.\hfill{\sf[reading]}
	\item \cite{Murphy2021}. Joseph Murphy. \textit{The Power of Subconscious Mind -- Sức Mạnh Tiềm Thức}.\hfill{\sf[finished]}
	\item \cite{Ngoc2022}. Lê Bảo Ngọc. \textit{Không Phải Sói Nhưng Cũng Đừng Là Cừu}.\hfill{\sf[finished]}
	\item \cite{Phipps_Gautreys_muu_hen_ke_ban_tap_1}. Mike Phipps, Colin Gautreys. \textit{Mưu Hèn Kết Bẩn Nơi Công Sở. Tập 1: Nghệ Thuật Nhận Biết \& Phòng Tránh ``Tiểu Nhân'' Trong Công Việc}.\hfill{\sf[finished]}
	\item \cite{muu_hen_ke_ban_tap_2}. Alpha Books. \textit{Mưu Hèn Kết Bẩn Nơi Công Sở. Tập 2: Nghệ Thuật Thăng Tiến Trong Sự Nghiệp}.\hfill{\sf[finished]}
	\item \cite{Rutherford2020}. Albert Rutherford. \textit{The Art of Thinking Critically: Ask Great Questions, Spot Illogical Reasoning, \& Make Sharp Arguments (The critical Thinker Book 5)}.\hfill{\sf[reading]}
	\item \cite{Rutherford2022}. Albert Rutherford. \textit{Rèn Luyện Tư Duy Phản Biện}.\hfill{\sf[finished]}
	\item \cite{Rutherford2023}. Albert Rutherford. \textit{The Art of Thinking Critically: Ask Great Questions, Spot Illogical Reasoning, \& Make Sharp Arguments -- Nghệ Thuật Tư Duy Phản Biện}.\hfill{\sf[finished]}
	\item \cite{Sandberg_Grant2017}. Sheryl Sandberg, Adam Grant. \textit{Option B: Facing Adversity, Building Resilience, \& Finding Joy}.\hfill{\sf[reading]}
	\item \cite{Sandberg_Grant2019}. Sheryl Sandberg, Adam Grant. \textit{Option B: Facing Adversity, Building Resilience, \& Finding Joy -- Phương Án B: Đối Mặt Nghịch Cảnh, Rèn Tính Kiên Cường, \& Tìm Lại Niềm Vui}.\hfill{\sf[finished]}
	\item \cite{Schwartz2019}. David J. Schwartz. \textit{The Magic of Thinking Big -- Dám Nghĩ Lớn}.\hfill{\sf[finished]}
	\item \cite{Simon2010}. George Simon Jr. \textit{In Sheep's Clothing: Understanding \& Dealing with Manipulative People}.\hfill{\sf[reading]}
	\item \cite{Simon2011}. George Simon Jr. \textit{Character Disturbance: The Phenomenon of Our Age}.\hfill{\sf[finished]}
	\item \cite{Simon2022}. George K. Simon. \textit{In Sheep's Clothing: Understanding \& Dealing with Manipulative People -- Sói Đội Lốt Cừu: Kẻ Hiếu Chiến Ngầm \& Các Thủ Thuật Thao Túng Tâm Lý}.\hfill{\sf[finished]}
	\item \cite{Stout2006}. Martha Stout. \textit{The Sociopath Next Door}.\hfill{\sf[reading]}
	\item \cite{Stout2019}. Martha Stout. \textit{Kẻ Ác Cạnh Bên}.\hfill{\sf[finished]}	
	\item Martha Stout. \textit{The Myth of Sanity: Divided Consciousness \& the Promise of Awareness}.	
	\item \cite{Thaler_misbehaving}. Richard H. Thaler. \textit{Misbehaving: The Making of Behavioral Economics -- Tất Cả Chúng Ta Đều Hành Xử Cảm Tính: Sự Hình Thành Kinh Tế Học Hành Vi}.\hfill{\sf[reading]}
	\item \cite{Thomas2021}. Shannon Thomas, LCSW. \textit{Healing from Hidden Abuse: A Journey Through the Stages of Recovery from Psychological Abuse -- Thao Túng Tâm Lý: Nhận Diện, Thức Tỉnh, \& Chữa Lành Những Tổn Thương Tiềm Ẩn}.\hfill{\sf[finished]}
	\item \cite{Thu2022}. Nguyễn Đoàn Minh Thư. \textit{Hành Tinh Của 1  Kẻ Nghĩ Nhiều}.\hfill{\sf[finished]}
	\item \cite{Wei2022}. Xiu-Ying Wei. \textit{Harvard Bốn Rưỡi Sáng}.\hfill{\sf[finished]}
\end{enumerate}

%------------------------------------------------------------------------------%

\section{Philosophy Book}

\begin{enumerate}
	\item \cite{Chodron2002}. Pema Ch\"odr\"on. \textit{When Things Fall Apart: Heart Advice for Difficult Times}.\hfill{\sf[reading]}
	\item \cite{Chodron2021}. Pema Ch\"odr\"on. \textit{When Things Fall Apart: Heart Advice for Difficult Times -- Khi Mọi Thứ Sụp Đổ: Lời Khuyên Chân Thành Trong Những Thời Điểm Khó Khăn}.\hfill{\sf[finished]}
	\item \cite{Chung2022}. Phạm Văn Chung. \textit{Friedrich Nietzsche \& Những Suy Niệm Bên Kia Thiện Ác}.\hfill{\sf[finished]}
	\item \cite{Frankl2013}. Viktor E. Frankl. \textit{Man's Search for Meaning}.\hfill{\sf[reading]}
	\item \cite{Frankl2017}. Viktor E. Frankl. \textit{Man's Search for Meaning}.\hfill{\sf[reading]}
	\item \cite{Frankl2022}. Viktor E. Frankl. \textit{Man's Search for Meaning -- Đi Tìm Lẽ Sống}.\hfill{\sf[finished]}
	\item \cite{Hardy1940, Hardy1992, Hardy2022}. G. H. Hardy. \textit{A Mathematician's Apology}.\hfill{\sf[finished]}
	\item \cite{Peterson2018}. Jordan B. Peterson. \textit{$12$ Rules for Life: An Antidote to Chaos}.\hfill{\sf[reading]}
	\item \cite{Peterson2022a}. Jordan B. Peterson. \textit{$12$ Quy Luật Cuộc Đời: Thần Dược cho Cuộc Sống Hiện Đại -- 12 Rules for Life: An Antidote to Chaos}. 2e.\hfill{\sf[finished]}
	\item \cite{Peterson2022b}. Jordan B. Peterson. \textit{Vượt Lên Trật Tự: $12$ Quy Tắc cho Cuộc Sống -- Beyond Order}.\hfill{\sf[finished]}
	\item Jordan B. Peterson. \textit{Maps of Meaning}.\hfill{\sf[reading]}
	\item \cite{Thoreau2014}. Henry David Thoreau. \textit{Walden}.\hfill{\sf[reading]}
	\item \cite{Thoreau2023}. Henry David Thoreau. \textit{Walden -- Một Mình Sống Trong Rừng}.\hfill{\sf[finished]}
\end{enumerate}

%------------------------------------------------------------------------------%

\section{Spirituality}

\begin{enumerate}
	\item \cite{Hanh_silence}. Thích Nhật Hạnh. \textit{Silence -- Tĩnh Lặng: Sức Mạnh Tĩnh Lặng Trong Thế Giới Huyên Náo}.\hfill{\sf[finished]}
	\item \cite{Ruiz2011}. \textit{The Four Agreements: A Practical Guide to Personal Freedom (A Toltec Wisdom Book)}.\hfill{\sf[finished]}
	\item \cite{Ruiz_Mills2022}. Don Miguel Ruiz, Janet Mills. \textit{The Four Agreements: A Practical Guide to Personal Freedom (A Toltec Wisdom Book) -- 4 Thỏa Ước: Bí Quyết Sống Tự Do, Bình An, Hạnh Phúc Giữa Thế Giới Bất Định}.\hfill{\sf[finished]}
	\item \cite{Ruiz_Ruiz2011}. \textit{The Fifth Agreements: A Practical Guide to Self-Mastery (A Toltec Wisdom Book)}.\hfill{\sf[finished]}
	\item \cite{Tolle2021d}. Eckhart Tolle. \textit{A New Earth: Awakening to Your Life's Purpose -- Thức Tỉnh Mục Đích Sống}.\hfill{\sf[finished]}
	\item \cite{Tolle2021a}. Eckhart Tolle. \textit{Oneness With All Life -- Hợp Nhất với Vũ Trụ}.\hfill{\sf[finished]}
	\item \cite{Tolle2021b}. Eckhart Tolle. \textit{Practicing The Power of Now -- Trải Nghiệm Sức Mạnh Hiện Tại}.\hfill{\sf[finished]}
	\item \cite{Tolle2021c}. Eckhart Tolle. \textit{The Power of Now -- Sức Mạnh của Hiện Tại}.\hfill{\sf[finished]}
	\item \cite{Tolle2022}. Eckhart Tolle. \textit{Stillness Speaks -- Sức Mạnh của Tĩnh Lặng}.\hfill{\sf[finished]}
\end{enumerate}

%------------------------------------------------------------------------------%

\section{Miscellaneous}

\begin{enumerate}
	\item \cite{Anderson2016}. Chris Anderson. \textit{TED Talks: The Official TED Guide to Public Speaking: Tips \& Tricks for Giving Unforgettable Speeches \& Presentations}.\hfill{\sf[reading]}
	\item \cite{Anderson2022}. Chris Anderson. \textit{TED Talks: The Official TED Guide to Public Speaking: Tips \& Tricks for Giving Unforgettable Speeches \& Presentations -- Hùng Biện Kiểu TED: Bí Quyết Diễn Thuyết Trước Đám Đông ``Chuẩn'' TED}.\hfill{\sf[finished]}
	\item \cite{Aoun2019}. Joseph E. Aoun. \textit{Robot-Proof: Higher Education in the Age of Artificial Intelligence -- Chạy Đua Với Robot: Học Tập Thời Trí Tuệ Nhân Tạo}.\hfill{\sf[finished]}
	\item \cite{Dixit_Nalebuff2010}. Avinash K. Dixit, Barry J. Nalebuff. \textit{The Art of Strategy: A Game Theorist's Guide to Success in Business \& Life}.\\\mbox{}\hfill{\sf[reading]}
	\item \cite{Dixit_Nalebuff_strategy}. Avinash K. Dixit, Barry J. Nalebuff. \textit{The Art of Strategy: A Game Theorist's Guide to Success in Business \& Life -- Nghệ Thuật Tư Duy Chiến Lược: Ứng Dụng Của Lý Thuyết Trò Chơi Trong Công Việc \& Cuộc Sống}.\hfill{\sf[reading]}
	\item \cite{Foer2012}. Joshua Foer. \textit{Moonwalking with Einstein: The Art \& Science of Remembering Everything}.\hfill{\sf[reading]}
	\item \cite{Foer_remember}. Joshua Foer. \textit{Moonwalking with Einstein: The Art \& Science of Remembering Everything -- Nhảy Moonwalk Cùng Einstein: Nghệ Thuật \& Khoa Học Để Nhớ Được Mọi Thứ}.\hfill{\sf[finished]}
	\item \cite{Kahn2020}. Jeffrey P. Kahn. \textit{Digital Contact Tracing for Pandemic Response -- Ứng Dụng Công Nghệ Truy Dấu Tiếp Xúc Để Ứng Phó Với Dịch Covid-19}.\hfill{\sf[finished]}
	\item \cite{Sandel_justice}. Michael Sandel. \textit{Justice: What's The Right Thing To Do? -- Phải Trái Đúng Sai}.\hfill{\sf[finished]}
	\item \cite{Sandel_money}. Michael Sandel. \textit{What Money Can't Buy -- Tiền Không Mua Được Gì?}.\hfill{\sf[finished]}
	\item \cite{Taleb_2008}. Nassim Nicholas Taleb. \textit{Fooled By Randomness: The Hidden Role of Chance in Life \& in the Markets}.\hfill{\sf[reading]}
	\item \cite{Taleb_randomness}. Nassim Nicholas Taleb. \textit{Fooled By Randomness: The Hidden Role of Chance in Life \& in the Markets -- Trò Đùa Của Sự Ngẫu Nhiên: Giải Mã Bí Ẩn Quanh Những Điều Tình Cờ}.\hfill{\sf[finished]}
	\item \cite{Truong2023}. Phan Văn Trường. \textit{Một Đời Như Kẻ Tìm Đường}.\hfill{\sf[finished]}
\end{enumerate}

%------------------------------------------------------------------------------%

\printbibliography[heading=bibintoc]
	
\end{document}