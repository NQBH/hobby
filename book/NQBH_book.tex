\documentclass{article}
\usepackage[backend=biber,natbib=true,style=authoryear]{biblatex}
\addbibresource{/home/nqbh/reference/bib.bib}
\usepackage[english,vietnamese]{babel}
\usepackage{tocloft}
\renewcommand{\cftsecleader}{\cftdotfill{\cftdotsep}}
\usepackage[colorlinks=true,linkcolor=blue,urlcolor=red,citecolor=magenta]{hyperref}
\usepackage{algorithm,algpseudocode,amsmath,amssymb,amsthm,float,graphicx,mathtools,multicol}
\allowdisplaybreaks
\numberwithin{equation}{section}
\newtheorem{assumption}{Assumption}[section]
\newtheorem{conjecture}{Conjecture}[section]
\newtheorem{corollary}{Corollary}[section]
\newtheorem{definition}{Definition}[section]
\newtheorem{example}{Example}[section]
\newtheorem{lemma}{Lemma}[section]
\newtheorem{notation}{Notation}[section]
\newtheorem{principle}{Principle}[section]
\newtheorem{problem}{Problem}[section]
\newtheorem{proposition}{Proposition}[section]
\newtheorem{question}{Question}[section]
\newtheorem{remark}{Remark}[section]
\newtheorem{theorem}{Theorem}[section]
\usepackage[left=0.5in,right=0.5in,top=1.5cm,bottom=1.5cm]{geometry}
\usepackage{fancyhdr}
\pagestyle{fancy}
\fancyhf{}
\lhead{\small Subsect.~\thesubsection}
\rhead{\small \nouppercase{\leftmark}}
\renewcommand{\subsectionmark}[1]{\markboth{#1}{}}
\cfoot{\thepage}
\def\labelitemii{$\circ$}

\title{Book}
\author{\selectlanguage{vietnamese} Nguyễn Quản Bá Hồng\footnote{Independent Researcher, Ben Tre City, Vietnam\\e-mail: \texttt{nguyenquanbahong@gmail.com}; website: \url{https://nqbh.github.io}.}}
\date{\today}

\begin{document}
\maketitle
\selectlanguage{english}
\begin{abstract}
	A list of books planned to buy\texttt{/}download.
\end{abstract}
\selectlanguage{vietnamese}

%------------------------------------------------------------------------------%

\section{Elementary STEM Book}

\subsection{Elementary Mathematics Book}

\begin{enumerate}
	\item \cite{Trong_Toan_6_2021}. Đặng Đức Trọng, Nguyễn Đức Tấn, Phạm Lê Quốc Thắng, Nguyễn Phúc Trường, Cao Hoàng Lợi. \textit{Bồi Dưỡng Năng Lực Tự Học Toán 6}.\hfill\texttt{[bought]}
	\item \cite{Trong_Toan_7_2022}.Đặng Đức Trọng, Nguyễn Đức Tấn, Phạm Lê Quốc Thắng, Nguyễn Phúc Trường, Cao Hoàng Lợi, Nguyễn Thị Kiều Anh. \textit{Bồi Dưỡng Năng Lực Tự Học Toán 7}.\hfill\texttt{[bought]}
	\item \cite{Binh_Toan_6_tap_1, Binh_Toan_6_tap_2}. Vũ Hữu Bình. \textit{Nâng Cao \& Phát Triển Toán 6. Tập 1 $+$ 2}.\hfill\texttt{[bought]}
	\item Vũ Hữu Bình. \textit{Nâng Cao \& Phát Triển Toán 7. Tập 1 $+$ 2}.\hfill\texttt{[buying ...]}
	\item Vũ Hữu Bình. \textit{Nâng Cao \& Phát Triển Toán 8. Tập 1 $+$ 2}.\hfill\texttt{[waiting new version ...]}
	\item Vũ Hữu Bình. \textit{Nâng Cao \& Phát Triển Toán 9. Tập 1 $+$ 2}.\hfill\texttt{[waiting new version ...]}
	\item Vũ Hữu Bình, ***. \textit{Tài Liệu Chuyên Toán THCS Toán 6 -- Tập 1: Số Học}.
	\item Vũ Hữu Bình, ***. \textit{Tài Liệu Chuyên Toán THCS Toán 6 -- Tập 2: Hình Học}.	
	\item Vũ Hữu Bình, ***. \textit{Tài Liệu Chuyên Toán THCS Toán 7 -- Tập 1: Đại Số}.
	\item Vũ Hữu Bình, ***. \textit{Tài Liệu Chuyên Toán THCS Toán 7 -- Tập 2: Hình Học}.
	\item Vũ Hữu Bình, ***. \textit{Tài Liệu Chuyên Toán THCS Toán 8 -- Tập 1: Đại Số}.
	\item Vũ Hữu Bình, ***. \textit{Tài Liệu Chuyên Toán THCS Toán 8 -- Tập 2: Hình Học}.
	\item Vũ Hữu Bình, Phạm Thị Bạch ngọc, Đàm Văn Nhỉ. \textit{Tài Liệu Chuyên Toán THCS Toán 9 -- Tập 1: Đại Số}.\hfill\texttt{[bought]}
	\item Vũ Hữu Bình, Nguyễn Ngọc Đạm, Nguyễn Bá Đang, Lê Quốc Hán, Hồ Quang Vinh. \textit{Tài Liệu Chuyên Toán THCS Toán 9 -- Tập 2: Hình Học}.\hfill\texttt{[bought]}
	\item Vũ Hữu Bình. \textit{Phương Trình Nghiệm Nguyên \& Kinh Nghiệm Giải}.\hfill\texttt{[bought]}
	\item \cite{Tuyen_Toan_7}. Bùi Văn Tuyên. \textit{Bài Tập Nâng Cao \& 1 Số Chuyên Đề Toán 7}.\hfill\texttt{[bought]}
	\item \cite{Tuyen_Toan_8}. Bùi Văn Tuyên. \textit{Bài Tập Nâng Cao \& 1 Số Chuyên Đề Toán 8}.\hfill\texttt{[bought]}
	\item Nguyễn Văn Linh. \textit{1 Số Chủ Đề Hình Học Phẳng}.
	\item Phạm Minh Phương, Trần Văn Tấn, Nguyễn Thị Thanh Thủy. \textit{Bồi Dưỡng Học Sinh Giỏi Toán THCS: Số Học}.
	\item Bùi Văn Tuyên. \textit{Bài Tập Nâng Cao \& 1 Số Chuyên Đề Toán 9}.
	\item \cite{Dung_Can_Anh2020}. Nguyễn Văn Dũng, Võ Quốc Bá Cẩn, Trần Quốc Anh. \textit{Phương Pháp Giải Toán Bất Đẳng Thức \& Cực Trị Dành Cho Học Sinh 8, 9}.\hfill\texttt{[bought]}
	\item \cite{Son_Nghiep_Trung_Can2021}. Nguyễn Ngọc Sơn, Chu Đình Nghiệp, Lê Hải Trung, Võ Quốc Bá Cẩn. \textit{Các Chủ Đề Bất Đẳng Thức Ôn Thi Vào Lớp 10}. 3e.\hfill\texttt{[bought]}
	\item Vũ Hữu Bình. \textit{9 Chuyên Đề Đại Số THCS}.
	\item Vũ Hữu Bình. \textit{9 Chuyên Đề Số Học THCS}.
	\item Vũ Hữu Bình. \textit{9 Chuyên Đề Hình Học THCS}.
	\item Vũ Hữu Bình. \textit{Toán 7 Cơ Bản \& Nâng Cao. Tập 2}.
	\item Bùi Văn Tuyên. \textit{Bài Tập Nâng Cao \& 1 Số Chuyên Đề Toán 8}.
	\item Nguyễn Đức Đồng. \textit{23 Chuyên Đề Giải 1001 Bài Toán Sơ Cấp. Tập 1 $+$ 2}.
	\item Vũ Hữu Bình, Tôn Thân, Đỗ Quang Thiều. \textit{Toán Bồi Dưỡng Học sinh Lớp 8 Đại Số}.
	\item Vũ Hữu Bình, Tôn Thân, Đỗ Quang Thiều. \textit{Toán Bồi Dưỡng Học sinh Lớp 8 Hình Học}.
	\item Vũ Dương Thụy, Nguyễn Ngọc Đạm. \textit{Toán Nâng Cao \& Các Chuyên Đề Hình Học 9}.
	\item Vũ Hữu Bình. \textit{Toán 8 Cơ Bản \& Nâng Cao. Tập 2}.
	\item Vũ Hữu Bình. \textit{Toán 9 Cơ Bản \& Nâng Cao. Tập 2}.
	\item Võ Quốc Bá Cẩn, Trần Quốc Anh. \textit{Sử Dụng Phương Pháp AM--GM Để Chứng Minh Bất Đẳng Thức}.
	\item Võ Quốc Bá Cẩn, Trần Quốc Anh. \textit{Sử Dụng Phương Pháp Cauchy--Schwarz Để Chứng Minh Bất Đẳng Thức}.
	\item \cite{TL_chuyen_Toan_Hinh_Hoc_11}. Đoàn Quỳnh, Phạm Khắc Ban, Văn Như Cương, Nguyễn Đăng Phất, Lê Bá Khánh Trình. \textit{Tài Liệu Chuyên Toán Hình Học 11}.\hfill\texttt{[bought]}
	\item Đặng Hùng Thắng, Nguyễn Văn Ngọc, Vũ Kim Thùy. \textit{Bài Giảng Số Học}.
	\item \cite{Binh_HHTH}. Vũ Hữu Bình. \textit{Hình Học Tổ Hợp}.\hfill\texttt{[bought]}
	\item \cite{Tan2017}. Trần Văn Tấn. \textit{Bài Tập Nâng Cao \& Một Số Chuyên Đề Hình Học 11}.\hfill\texttt{[bought]}
\end{enumerate}

\subsection{Elementary Physics Book}
\begin{enumerate}
	\item \cite{Thinh_Lua2021}. Bùi Gia Thịnh, Lê Thị Lụa. \textit{Nâng Cao \& Phát Triển Vật Lý 8}.\hfill\texttt{[bought]}
	\item Nguyễn Cảnh Hòe. \textit{Nâng Cao \& Phát Triển Vật Lý 9}.
	
	
	\item Tô Giang. \textit{Tài liệu chuyên Vật lý. Vật lý 10 -- Tập 1}.
	\item Phạm Quý Tư, Nguyễn Đình Noãn. \textit{Tài liệu chuyên Vật lý. Vật lý 10 -- Tập 2}.
	\item Vũ Thanh Khiết, Nguyễn Thế Khôi. \textit{Tài liệu chuyên Vật lý. Vật lý 11 -- Tập 1}.
	\item Vũ Quang. \textit{Tài liệu chuyên Vật lý. Vật lý 11 -- Tập 2}.
	\item Tô Giang, Vũ Thanh Khiết, Nguyễn Thế Khôi. \textit{Tài liệu chuyên Vật lý. Vật lý 12 -- Tập 1}.
	\item Vũ Quang, Vũ Thanh Khiết. \textit{Tài liệu chuyên Vật lý. Vật lý 12 -- Tập 2}.
	\item Tô Giang, Đặng Đình Tới, Bùi Trọng Tuân. \textit{Tài liệu chuyên Vật lý -- Bài tập Vật lý 10}.
	\item Lưu Hải An, Nguyễn Hoàng Kim, Vũ Thanh Khiết, Nguyễn Thế Khôi, Lưu Văn Xuân. \textit{Tài liệu chuyên Vật lý -- Bài tập Vật lý 11}.
	\item Tô Giang, Vũ Thanh Khiết, Đặng Đình Tới. \textit{Tài liệu chuyên Vật lý -- Bài tập Vật lý 12}.
	\item Đàm Trung Đồn. \textit{Tài liệu chuyên Vật lý -- Thực hành Vật lý Trung học phổ thông}.
	\item Tô Giang. \textit{Bồi Dưỡng Học Sinh Giỏi Vật Lý THPT: Cơ học 1}.\hfill\texttt{[bought]}
	\item Tô Giang. \textit{Bồi Dưỡng Học Sinh Giỏi Vật Lý THPT: Cơ học 2}.\hfill\texttt{[bought]}
	\item Tô Giang. \textit{Bồi Dưỡng Học Sinh Giỏi Vật Lý THPT: Cơ học 3}.\hfill\texttt{[bought]}
	\item Vũ Thanh Khiết, Nguyễn Thế Khôi. \textit{Bồi Dưỡng Học Sinh Giỏi Vật Lý THPT: Điện học 1}.\hfill\texttt{[bought]}
	\item Vũ Thanh Khiết, Tô Giang. \textit{Bồi Dưỡng Học Sinh Giỏi Vật Lý THPT: Điện học 2}.\hfill\texttt{[bought]}
	\item Phạm Quý Tư. \textit{Bồi Dưỡng Học Sinh Giỏi Vật Lý THPT: Nhiệt Học \& Vật Lý Phân Tử}.
	\item Ngô Quốc Quỳnh. \textit{Bồi Dưỡng Học Sinh Giỏi Vật Lý THPT: Quang học 1}.\hfill\texttt{[bought]}
	\item Vũ Quang. \textit{Bồi Dưỡng Học Sinh Giỏi Vật Lý THPT: Quang học 2}.
	\item Vũ Thanh Khiết. \textit{Bồi Dưỡng Học Sinh Giỏi Vật Lý THPT: Vật Lý Hiện Đại}.
	\item Phạm Văn Thiều. Đoàn Văn Ro, Nguyễn Văn Phán. \textit{Bồi Dưỡng Học Sinh Giỏi Vật Lý THPT: Phương pháp giải 1 số bài toán điển hình}.
	\item Phạm Văn Thiều. \textit{Bồi Dưỡng Học Sinh Giỏi Vật Lý THPT: Những bài toán tổng hợp: phân tích \& lời giải}.
	\item Bùi Quang Hân, Nguyễn Duy Hiền, Nguyễn Tuyến. \textit{Giải Toán \& Trắc Nghiệm Vật Lý 10 -- Tập 1: Cơ học}.
	\item Bùi Quang Hân, Nguyễn Duy Hiền, Nguyễn Tuyến. \textit{Giải Toán \& Trắc Nghiệm Vật Lý 10 -- Tập 2: Nhiệt học}.
	\item Bùi Quang Hân, Nguyễn Duy Hiền, Nguyễn Tuyến. \textit{Giải Toán \& Trắc Nghiệm Vật Lý 11 -- Tập 1: Tĩnh điện \& Dòng điện không đổi}.
	\item Bùi Quang Hân, Nguyễn Duy Hiền, Nguyễn Tuyến. \textit{Giải Toán \& Trắc Nghiệm Vật Lý 11 -- Tập 2: Điện từ \& Quang học}.
	\item Bùi Quang Hân, Nguyễn Duy Hiền, Nguyễn Tuyến. \textit{Giải Toán \& Trắc Nghiệm Vật Lý 12 -- Tập 1: Động lực học vật rắn, Dao động cơ, Sóng cơ}.
	\item Bùi Quang Hân, Nguyễn Duy Hiền, Nguyễn Tuyến. \textit{Giải Toán \& Trắc Nghiệm Vật Lý 12 -- Tập 2: Dao đông \& sóng điện từ, Dòng điện xoay chiều}.
	\item Bùi Quang Hân, Nguyễn Duy Hiền, Nguyễn Tuyến. \textit{Giải Toán \& Trắc Nghiệm Vật Lý 12 -- Tập 3: Sóng ánh sáng, Lượng tử ánh sáng, Thuyết tương đối hẹp, Hạt nhân nguyên tử, Từ vi mô đến vĩ mô}.
	\item Vũ Thanh Khiết, Lưu Hải Ân, Phạm Vũ Kim Hoàng, Nguyễn Đức Hiệp, Nguyễn Hoàng Kim. \textit{Bồi Dưỡng Học Sinh Giỏi Vật Lý THPT: Bài Tập Điện Học -- Quang Học Vật Lý Hiện Đại}.
\end{enumerate}

\subsection{Elementary Chemistry Book}
\begin{enumerate}
	\item Nguyễn Xuân Trường, Quách Văn Long, Hoàng Thị Thúy Hương. \textit{Các Chuyên Đề Bồi Dưỡng Học Sinh Giỏi Hóa Học 8}.
	\item \cite{Giac2021}. Cao Cự Giác. \textit{Bồi Dưỡng Học Sinh Giỏi Hóa Học 8}.\hfill\texttt{[bought]}
	\item \cite{Tuan2022}. Vũ Anh Tuấn. \textit{Bồi Dưỡng Hóa Học THCS}.\hfill\texttt{[bought]}
	\item \cite{Ninh_Chi_Khu_Lien_Thanh2019}. Trần Trung Ninh, Khiếu Thị Hương Chi, Lê Văn Khu, Trần Thị Kim Liên, Nguyễn Thị Kim Thành. \textit{$500$ Bài Tập Hóa Học Chuyên Trung Học Cơ Sở (Bồi Dưỡng Học Sinh Giỏi)}.\hfill\texttt{[bought]}
	\item Nguyễn Đình Hành, Nguyễn Hữu Thọ. \textit{22 Chuyên Đề Hay \& Khó Bồi Dưỡng Học Sinh Giỏi Hóa Học THCS. Tập 1}.
	\item Trần Quốc Sơn. \textit{Tài Liệu Chuyên Hóa Học 11--12. Tập 1: Hóa Học Hữu Cơ}.
	\item Nguyễn Duy Ái. \textit{Tài Liệu Chuyên Hóa Học 11--12. Tập 2: Hóa Học Vô Cơ}.
	\item Đào Hữu Vinh, Nguyễn Duy Ái. \textit{Tài liệu chuyên Hóa học 10 -- Tập 2}.
	\item Trần Quốc Sơn. \textit{Tài Liệu Chuyên Hóa Học THPT: Bài Tập Hữu Cơ. Tập 1}.
	\item Trần Quốc Sơn. \textit{Tài Liệu Chuyên Hóa Học THPT: Bài Tập Hữu Cơ. Tập 2}.
	\item Nguyễn Duy Ái, Nguyễn Tinh Dung, Trần Quốc Sơn, Nguyễn Văn Tòng. \textit{Bồi Dưỡng Hoc Sinh Giỏi Hóa Học THPT. Tập 3}.
\end{enumerate}

%------------------------------------------------------------------------------%

\section{Advanced STEM Book}

\subsection{Advanced Mathematics Book}
\begin{enumerate}
	\item \cite{Viet_Chua2022}. Dương Quốc Việt, Lê Văn Chua. \textit{Cơ Sở Lý Thuyết Galois}.\hfill\texttt{[bought]}
	\item \cite{Viet_Ha_Thanh_Dang_Loc2022}. Dương Quốc Việt, Lê Thị Hà, Trương Thị Hồng Thanh, Nguyễn Đạt Đăng, Nguyễn Quang Lộc. \textit{Bài Tập Lý Thuyết Galois}.\hfill\texttt{[bought]}
	\item Dương Quốc Việt. \textit{Cở Sở Lý Thuyết Module}.
	\item Nguyễn Xuân Liêm. \textit{Giải Tích Hàm}.
	\item Nguyễn Văn Khuê, Lê Mậu Hải. \textit{Giáo Trình Giải Tích Hàm}.
	\item Lê Mậu Hải, Tăng Văn Long. \textit{Bài Tập Giải Tích Hàm}.
	\item \cite{Viet_Nhi2022}. Dương Quốc Việt, Đàm Văn Nhỉ. \textit{Cơ Sở Lý Thuyết Số \& Đa Thức}.\hfill\texttt{[bought]}
	\item \cite{Viet_Dang_Dinh_Ha_Hanh_Minh_Thanh_Thuy2022}. Dương Quốc Việt, Nguyễn Đạt Đăng, Lê Văn Đinh, Lê Thị Hà, Đặng Đình Hanh, Đào Ngọc Minh, Trương Thị Hồng Thanh, Phan Thị Thủy. \textit{Bài Tập Cơ Sở Lý Thuyết Số \& Đa Thức}.\hfill\texttt{[bought]}
	\item Nguyễn Doãn Tuấn, Sĩ Đức Quang, Nguyễn Thị Thảo. \textit{Giáo Trình Hình Học Vi Phân}.
	\item Trần Văn Tấn. \textit{Hình Học của Nhóm Biến Đổi}.
	\item Nguyễn Văn Đoành. \textit{Đa Tạp Khả Vi}.
	\item Nguyễn Hữu Việt Hưng. \textit{Đại Số Tuyến Tính}.
	\item Trần Diên Hiền, Nguyến Tiến Tài, Nguyễn Văn Ngọc. \textit{Giáo Trình Lý Thuyết Số}.
	\item \cite{Quy_Liem2012}. Nguyễn Mạnh Quý, Nguyễn Xuân Liêm. \textit{Giáo Trình Phép Tính Vi Phân \& Tích Phân của Hàm 1 Biến Số: Phần Lý Thuyết}.\hfill\texttt{[bought]}
	\item Đoàn Quỳnh. \textit{Hình Học Vi Phân}.
	\item Bùi Duy Hiền. \textit{Bài Tập Đại Số Đại Cương}.
	\item \cite{Nhi_Chin_Dung_Dung_Tinh_Dung_Son_Tuan2017}. Đàm Văn Nhỉ, Văn Đức Chín, Trần Thị Hồng Nhung, Lê Xuân Dũng, Trần Trung Tình, Đào Ngọc Dũng, Đặng Xuân Sơn, Nguyễn Anh Tuấn. \textit{Đa Thức -- Chuỗi \& Chuyên Đề Nâng Cao}.
\end{enumerate}

%------------------------------------------------------------------------------%

\subsection{Advanced Physics Book}
\begin{enumerate}
	\item Đào Văn Phúc. \textit{Lịch Sử Vật Lý Học}.
	\item Lương Duyên Bình, Nguyễn Hữu Hồ, Lê Văn Nghĩa, Nguyễn Quang Bình. \textit{Bài Tập Vật Lý Đại Cương. Tập 2: Điện -- Dao Động -- Sóng}.
	\item Vũ Văn Hùng. \textit{Cơ Học Lượng Tử}.
	\item Vũ Văn Hùng. \textit{Bài Tập Cơ Học Lượng Tử}.
	\item Nguyễn Quang Học, Đinh Quang Vinh. \textit{Bài Tập Vật Lý Lý Thuyết 2. Tập 2: Vật Lý Thống Kê}.
	\item Nguyễn Quang Học, Vũ Văn Hùng. \textit{Giáo Trình Vật Lý Thống Kê \& Nhiệt Động Lực Học. Tập 1: Nhiệt Động Lực Học}.
\end{enumerate}

%------------------------------------------------------------------------------%

\subsection{Advanced Chemistry Book}
\begin{enumerate}
	\item Hoàng Nhâm. \textit{Hóa Học Vô Cơ Cơ Bản. Tập 1: Lý Thuyết Đại Cương về Hóa Học}.
	\item Hoàng Nhâm. \textit{Hóa Học Vô Cơ Cơ Bản. Tập 2: Các Nguyên Tố Hóa Học Điển Hình}.
	\item Hoàng Nhâm. \textit{Hóa Học Vô Cơ Cơ Bản. Tập 3: Các Nguyên Tố Chuyển Tiếp}.
	\item Hoàng Nhâm. \textit{Bài Tập Hóa Học Vô Cơ}.
	\item Hoàng Nhâm, Hoàng Nhuận. \textit{Bài Tập Hóa Học Vô Cơ. Quyển I $+$ II: Lý Thuyết Đại Cương về Hóa Học}.
	\item Hoàng Nhâm, Hoàng Nhuận. \textit{Bài Tập Hóa Học Vô Cơ. Quyển III: Hóa Học Các Nguyên Tố}.
	\item Hoàng Nhâm. \textit{Hóa Học Vô Cơ Nâng Cao. Tập 1: Lý Thuyết Đại Cương về Hóa Học}.
	\item Hoàng Nhâm. \textit{Hóa Học Vô Cơ Nâng Cao. Tập 2: Các Nguyên Tố Hóa Học Tiêu Biểu}.
	\item Hoàng Nhâm. \textit{Hóa Học Vô Cơ Nâng Cao. Tập 3: Các Nguyên Tố Chuyển Tiếp}.
	\item Đào Đình Thức. \textit{Cấu Tạo Nguyên Tử \& Liên Kết Hóa Học. Tập 1}.
	\item Đào Đình Thức. \textit{Cấu Tạo Nguyên Tử \& Liên Kết Hóa Học. Tập 2}.
	\item Đỗ Đình Răng, Đặng Đình Bạch, Lệ Thị Anh Đào, Nguyễn Mạnh Hà, Nguyễn Thị Thanh Phong. \textit{Hóa Học Hữu Cơ 3}.
	\item Trần Thành Huế, Nguyễn Ngọc Hà. \textit{Đối Xứng Phân Tử \& Lý Thuyết Nhóm Trong Hóa Học}.
\end{enumerate}

%------------------------------------------------------------------------------%

\section{Literary Book}

%------------------------------------------------------------------------------%

\section{Psychology Book}
Về các cuốn sách tâm lý, mình sẽ liệt kê chúng theo thứ tự hay dần\texttt{/}theo chiều tăng của sự tâm đắc cá nhân, riêng những cuốn đang mua chưa đọc sẽ tạm để ở cuối danh sách, sau khi đọc 1 phần\texttt{/}xong đủ để đánh giá mức độ hay của những cuốn sách đó thì mình sẽ sắp thứ tự sau. Chỉ riêng sách Văn Học, Tâm Lý \& Triết Học mới được áp dụng cách liệt kê này, đặc biệt không áp dụng (được) cho các sách STEM -- đơn giản vì chúng hay theo nhiều lĩnh vực khác nhau, nên không thể nào sắp duy nhất 1 thứ tự trên 1 tập hợp bán thứ tự được (poset -- partial ordering set).
\begin{enumerate}
	\item \cite{Nguoi_Ke_Chuyen_2021}. Người Kể Chuyện\texttt{/}Vũ Hoàng Long (chủ biên). \textit{Học Trường Chuyên -- Những Góc Nhìn Đa Chiều}. Facebook Người Kể Chuyện.\hfill\texttt{[bought]}
	\item \cite{Thomas2021}. Shannon Thomas, LCSW. \textit{Thao Túng Tâm Lý: Nhận Thức, Thức Tỉnh \& Chữa Lành Những Tổn Thương Tiềm Ẩn}. Trương Tuấn dịch.
	\item \cite{Kahnweiler2022}. Jennifer B. Kahnweiler. \textit{Quiet Influence -- Sức Mạnh của Sự Trầm Lắng -- The Introvert's Guide to Making a Difference}. Phùng Minh Ngọc dịch.
	\item \cite{Simon2010}. George Simon Jr. \textit{In Sheep's Clothing: Understanding \& Dealing with Manipulative People}.\hfill\texttt{[downloaded]}
	\item \cite{Simon2022}. George K. Simon. \textit{Sói Đội Lốt Cừu: Kẻ Hiếu Chiến Ngầm \& Các Thủ Thuật Thao Túng Tâm Lý}. Nguyễn Hưởng \& Hạo Nhiên dịch.\hfill\texttt{[bought]}
	\item \cite{Grant2013}. Adam Grant. \textit{Give \& Take: A Revolutionary Approach to Success}.\hfill\texttt{[downloaded]}
	\item \cite{Grant2022}. Adam Grant. \textit{Give \& Take: Why Helping Others Drives Our Success -- Cho \& Nhận: Vì Sao Giúp Người Đưa Ta Đến Thành Công?}.\hfill\texttt{[bought]}
	\item Pema Ch\"odr\"on. \textit{When Things Fall Apart. Khi Mọi Thứ Sụp Đổ: Lời Khuyên Chân Thành Trong Những Thời Điểm Khó Khăn}.\hfill\texttt{[buying ...]}
\end{enumerate}

%------------------------------------------------------------------------------%

\section{Philosophy Book}
\begin{enumerate}
	\item \cite{Frankl2013}. Viktor E. Frankl. \textit{Man's Search for Meaning}.\hfill\texttt{[downloaded]}
	\item \cite{Frankl2017}. Viktor E. Frankl. \textit{Man's Search for Meaning}.\hfill\texttt{[downloaded]}
	\item \cite{Frankl2022}. Viktor E. Frankl. \textit{Man's Search for Meaning -- Đi Tìm Lẽ Sống}.\hfill\texttt{[bought]}
	\item \cite{Peterson2018}. Jordan B. Peterson. \textit{$12$ Rules for Life: An Antidote to Chaos}.
	\item \cite{Peterson2022a}. Jordan B. Peterson. \textit{$12$ Quy Luật Cuộc Đời: Thần Dược cho Cuộc Sống Hiện Đại -- 12 Rules for Life: An Antidote to Chaos}. 2e.\hfill\texttt{[bought]}
	\item \cite{Peterson2022b}. Jordan B. Peterson. \textit{Vượt Lên Trật Tự: $12$ Quy Tắc cho Cuộc Sống -- Beyond Order}.\hfill\texttt{[bought]}
	\item Jordan B. Peterson. \textit{Maps of Meaning}.\hfill\texttt{[Downloaded]}
\end{enumerate}

%------------------------------------------------------------------------------%

\section{Miscellaneous}
\begin{enumerate}
	\item \cite{Van2022}. Vũ Hà Văn. \textit{Giáo Sư Phiêu Lưu Ký: Tản Mạn với 1 Nhà Toán Học}.\hfill\texttt{[bought]}
\end{enumerate}

%------------------------------------------------------------------------------%

\printbibliography[heading=bibintoc]
	
\end{document}