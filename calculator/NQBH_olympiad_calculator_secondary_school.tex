\documentclass{article}
\usepackage[backend=biber,natbib=true,style=authoryear,maxbibnames=20]{biblatex}
\addbibresource{/home/nqbh/reference/bib.bib}
\usepackage[utf8]{vietnam}
\usepackage{tocloft}
\renewcommand{\cftsecleader}{\cftdotfill{\cftdotsep}}
\usepackage[colorlinks=true,linkcolor=blue,urlcolor=red,citecolor=magenta]{hyperref}
\usepackage{amsmath,amssymb,amsthm,float,graphicx,mathtools,xfrac}
\allowdisplaybreaks
\newtheorem{assumption}{Assumption}
\newtheorem{baitoan}{Bài toán}
\newtheorem{cauhoi}{Câu hỏi}
\newtheorem{conjecture}{Conjecture}
\newtheorem{corollary}{Corollary}
\newtheorem{dangtoan}{Dạng toán}
\newtheorem{definition}{Definition}
\newtheorem{dinhly}{Định lý}
\newtheorem{dinhnghia}{Định nghĩa}
\newtheorem{example}{Example}
\newtheorem{ghichu}{Ghi chú}
\newtheorem{hequa}{Hệ quả}
\newtheorem{hypothesis}{Hypothesis}
\newtheorem{lemma}{Lemma}
\newtheorem{luuy}{Lưu ý}
\newtheorem{menhde}{Mệnh đề}
\newtheorem{nhanxet}{Nhận xét}
\newtheorem{notation}{Notation}
\newtheorem{note}{Note}
\newtheorem{principle}{Principle}
\newtheorem{problem}{Problem}
\newtheorem{proposition}{Proposition}
\newtheorem{question}{Question}
\newtheorem{remark}{Remark}
\newtheorem{theorem}{Theorem}
\newtheorem{vidu}{Ví dụ}
\usepackage[left=1cm,right=1cm,top=5mm,bottom=5mm,footskip=4mm]{geometry}
\def\labelitemii{$\circ$}
\DeclareRobustCommand{\divby}{%
	\mathrel{\vbox{\baselineskip.95ex\lineskiplimit0pt\hbox{.}\hbox{.}\hbox{.}}}%
}

\title{Olympiad Calculator in Secondary School -- Giải Toán Trên Máy Tính Cầm Tay Cấp Trung Học Cơ Sở}
\author{Nguyễn Quản Bá Hồng\footnote{Independent Researcher, Ben Tre City, Vietnam\\e-mail: \texttt{nguyenquanbahong@gmail.com}; website: \url{https://nqbh.github.io}.}}
\date{\today}

\begin{document}
\maketitle
\begin{abstract}
	\textsc{[en]} This text is a collection of problems, from easy to advanced, about \textit{fraction}. This text is also a supplementary material for my lecture note on Elementary Mathematics grade 9, which is stored \& downloadable at the following link: \href{https://github.com/NQBH/hobby/blob/master/elementary_mathematics/grade_9/NQBH_elementary_mathematics_grade_9.pdf}{GitHub\texttt{/}NQBH\texttt{/}hobby\texttt{/}elementary mathematics\texttt{/}grade 9\texttt{/}lecture}\footnote{\textsc{url}: \url{https://github.com/NQBH/hobby/blob/master/elementary_mathematics/grade_9/NQBH_elementary_mathematics_grade_9.pdf}.}. The latest version of this text has been stored \& downloadable at the following link: \href{https://github.com/NQBH/hobby/blob/master/elementary_mathematics/grade_9/fraction/NQBH_fraction.pdf}{GitHub\texttt{/}NQBH\texttt{/}hobby\texttt{/}elementary mathematics\texttt{/}grade 9\texttt{/}fraction}\footnote{\textsc{url}: \url{https://github.com/NQBH/hobby/blob/master/elementary_mathematics/grade_9/fraction/NQBH_fraction.pdf}.}.
	\vspace{2mm}
	
	\textsc{[vi]} Tài liệu này là 1 bộ sưu tập các bài tập chọn lọc từ cơ bản đến nâng cao về \textit{phân số}. Tài liệu này là phần bài tập bổ sung cho tài liệu chính -- bài giảng \href{https://github.com/NQBH/hobby/blob/master/elementary_mathematics/grade_9/NQBH_elementary_mathematics_grade_9.pdf}{GitHub\texttt{/}NQBH\texttt{/}hobby\texttt{/}elementary mathematics\texttt{/}grade 9\texttt{/}lecture} của tác giả viết cho Toán Sơ Cấp lớp 9. Phiên bản mới nhất của tài liệu này được lưu trữ \& có thể tải xuống ở link sau: \href{https://github.com/NQBH/hobby/blob/master/elementary_mathematics/grade_9/fraction/NQBH_fraction.pdf}{GitHub\texttt{/}NQBH\texttt{/}hobby\texttt{/}elementary mathematics\texttt{/}grade 9\texttt{/}fraction}.
\end{abstract}
\tableofcontents
\newpage

%------------------------------------------------------------------------------%

\section{Tính Giá Trị Biểu Thức}

\begin{baitoan}[Giải Toán Trên MTCT Khu Vực 2003--2004]
	Tính kết quả đúng của tích $M = 2222255555\cdot2222266666$.
\end{baitoan}
\noindent\textsf{Phân tích}: Khi nhập biểu thức vào máy tính cầm tay, máy tính cho kết quả tràn màn hình: $M = 2222255555\cdot2222266666 = 4.938444443\cdot10^{18}$ nên cần tách riêng từng phần ra để tính.

\begin{proof}[1st giải]
	Đặt $A = 22222$, $B = 55555$, $C = 66666$, $M = (10^5A + B)(10^5A + C) = 10^{10}A^2 + 10^5AB + 10^5AC + BC$. Tính trên MTCT: $A^2 = 22222^2 = 493817284$, $AB = 22222\cdot55555 = 1234543210$, $AC = 22222\cdot66666 = 1481451852$, $BC = 3703629630$. Tính trên giấy:
	\begin{table}[H]
		\centering
		\begin{tabular}{|c|c|c|c|c|c|c|c|c|c|c|c|c|c|c|c|c|c|c|c|}
			\hline
			$10^{10}A^2$ & 4 & 9 & 3 & 8 & 1 & 7 & 2 & 8 & 4 & 0 & 0 & 0 & 0 & 0 & 0 & 0 & 0 & 0 & 0 \\
			\hline
			$10^5AB$ &  &  &  &  & 1 & 2 & 3 & 4 & 5 & 4 & 3 & 2 & 1 & 0 & 0 & 0 & 0 & 0 & 0 \\
			\hline
			$10^5AC$ &  &  &  &  & 1 & 4 & 8 & 1 & 4 & 5 & 1 & 8 & 5 & 2 & 0 & 0 & 0 & 0 & 0 \\
			\hline
			$BC$ &  &  &  &  &  &  &  &  &  & 3 & 7 & 0 & 3 & 6 & 2 & 9 & 6 & 3 & 0 \\
			\hline
			$M$ & 4 & 9 & 3 & 8 & 4 & 4 & 4 & 4 & 4 & 3 & 2 & 0 & 9 & 8 & 2 & 9 & 6 & 3 & 0 \\
			\hline
		\end{tabular}
	\end{table}
	\noindent Vậy $M = 4938444443209829630$.
\end{proof}
Trong lời giải thứ nhất, có thể gom $10^5AB + 10^5AC = 10^5A(B + C)$ như sau (2 lời giải gần như giống hệt nhau chỉ khác là lời giải 2 ngắn hơn 1 dòng trong bảng tính trên giấy nháp):

\begin{proof}[2nd giải]
	Đặt $A = 22222$, $B = 55555$, $C = 66666$, $M = (10^5A + B)(10^5A + C) = 10^{10}A^2 + 10^5A(B + C) + BC$. Tính trên MTCT: $A^2 = 22222^2 = 493817284$, $A(B + C) = 22222\cdot(55555 + 66666) = 2715995062$, $BC = 3703629630$. Tính trên giấy:
	\begin{table}[H]
		\centering
		\begin{tabular}{|c|c|c|c|c|c|c|c|c|c|c|c|c|c|c|c|c|c|c|c|}
			\hline
			$10^{10}A^2$ & 4 & 9 & 3 & 8 & 1 & 7 & 2 & 8 & 4 & 0 & 0 & 0 & 0 & 0 & 0 & 0 & 0 & 0 & 0 \\
			\hline
			$10^5(AB + AC)$ &  &  &  &  & 2 & 7 & 1 & 5 & 9 & 9 & 5 & 0 & 6 & 2 & 0 & 0 & 0 & 0 & 0 \\
			\hline
			$BC$ &  &  &  &  &  &  &  &  &  & 3 & 7 & 0 & 3 & 6 & 2 & 9 & 6 & 3 & 0 \\
			\hline
			$M$ & 4 & 9 & 3 & 8 & 4 & 4 & 4 & 4 & 4 & 3 & 2 & 0 & 9 & 8 & 2 & 9 & 6 & 3 & 0 \\
			\hline
		\end{tabular}
	\end{table}
	\noindent Vậy $M = 4938444443209829630$.
\end{proof}

\begin{nhanxet}[Ý tưởng giải quyết với công cụ máy tính cầm tay]
	``Nếu tính trực tiếp trên MTCT thì kết quả vượt quá độ rộng hiển thị của màn hình MTCT, do đó để có kết quả chính xác ta tách các chữ số thành các bộ phận của $k$ chữ số ($k$ tùy thuộc vào từng loại máy tính cụ thể, thông thường thì $k$ bằng $\frac{1}{2}$ độ rộng hiển thị các chữ số trên màn hình MTCT), bắt đầu từ bên phải \& thực hiện tính toán. Sau đó kết hợp tính trên máy tính điện tử \& trên giấy nháp để được kết quả chính xác cuối cùng. Có thể đặt các biến cho các nhóm chữ số để thuận tiện trong biến đổi đại số (như ở 2 cách giải trên).
\end{nhanxet}

\begin{baitoan}[Mở rộng]
	Nêu cách tính kết quả đúng của tích $M = \overline{\underbrace{aa\ldots a}_n\underbrace{bb\ldots b}_n}\cdot\overline{\underbrace{aa\ldots a}_n\underbrace{cc\ldots c}_n}$ với $a,b,c\in\mathbb{N}$, $an\ne0$.
\end{baitoan}

\begin{baitoan}
	Tính chính xác $1023456^3$.
\end{baitoan}

\begin{baitoan}[Giải Toán Trên MTCT Khu Vực 2003--2004]
	Tính kết quả đúng của tích: $M = 20032003\cdot20042004$.
\end{baitoan}

\begin{baitoan}
	Tính kết quả đúng: (a) $M = 3344355664\cdot3333377777$. (b) $N = 123456^3$.
\end{baitoan}

\begin{baitoan}
	Tính kết quả đúng: (a) $A = 2001^3 + 2002^3 + 2004^3 + 2005^3 + 2006^3 + 2007^3 + 2008^3 + 2009^3$. (b) $B = 13032006\cdot13032007$.
\end{baitoan}

\begin{baitoan}
	Tính tổng: $S_n = \sum_{i=1}^n i\cdot i! = 1\cdot1! + 2\cdot2! + \cdots + n\cdot n!$. Áp dụng để tính $S_{16} = \sum_{i=1}^{16} i\cdot i! = 1\cdot1! + 2\cdot2! + \cdots + 16\cdot16!$.
\end{baitoan}

\begin{proof}[Giải]
	Vì $n\cdot n! = (n + 1 - 1)n! = (n + 1)n! - n! = (n + 1)! - n!$ nên ta biến đổi tổng đã cho đồng thời tính toán trên máy \& kết hợp trên giấy nháp như sau: 
\end{proof}



%------------------------------------------------------------------------------%

\section{Tìm Thương \& Số Dư Khi Chia 2 Số Tự Nhiên}

%------------------------------------------------------------------------------%

\section{Tìm ƯCLN \& BCNN}

%------------------------------------------------------------------------------%

\section{Biểu Diễn Số Hữu Tỷ \& Liên Phân Số}

%------------------------------------------------------------------------------%

\section{Biểu Diễn Số Thập Phân Vô Hạn Tuần Hoàn Về Số Hữu Tỷ}

%------------------------------------------------------------------------------%

\section{Tìm Chữ Số Thập Phân Thứ $k$}

%------------------------------------------------------------------------------%

\section{Tính Số Chữ Số của 1 Số Dạng Lũy Thừa}

%------------------------------------------------------------------------------%

\section{Tìm $k$ Chữ Số Tận Cùng của 1 Số Tự Nhiên}

%------------------------------------------------------------------------------%

\section{Các Bài toán Liên Quan về Số Nguyên Tố}

%------------------------------------------------------------------------------%

\section{Tính Tổng Hữu Hạn}

%------------------------------------------------------------------------------%

\section{Các Bài Toán về Hàm Số Bậc Nhất}

%------------------------------------------------------------------------------%

\section{Các Bài Toán về Đa Thức}

%------------------------------------------------------------------------------%

\section{Tìm GTLN \& GTNN của Biểu Thức}

%------------------------------------------------------------------------------%

\section{Phương Trình \& Hệ Phương Trình Nghiệm Nguyên}

%------------------------------------------------------------------------------%

\section{Phương Trình \& Hệ Phương Trình Đại Số}

%------------------------------------------------------------------------------%

\section{Xác Định Các Yếu Tố Liên Quan về Tam Giác, Tứ Giác, \& Hình Tròn}

%------------------------------------------------------------------------------%

\section{Các Bài Toán về Dãy Số}

%------------------------------------------------------------------------------%

\section{Tìm Các Số \& Chữ Số Thỏa Điều Kiện Cho Trước}

%------------------------------------------------------------------------------%

\section{Các Bài Toán Liên Quan về Lượng Giác, Tính Thời Gian, \& Số Đo Góc}

%------------------------------------------------------------------------------%

\section{Các Bài Toán Liên Quan về Lãi Suất, Tiền Lương, Tăng Trưởng}

%------------------------------------------------------------------------------%

\section{Miscellaneous}

\begin{baitoan}[GTN trên MTCT 2023, lớp 8, Bến Tre, 1.]
	Tính giá trị biểu thức:
	\begin{align*}
		M = \frac{2x^3 - 4xy + y^2}{x^2 - 3x + 2y^2} - \frac{y^2 + 4xy}{x(x + y^3)} + (x - y + 5.1)^2\mbox{ với } x = 0.4,\ y = -\frac{1}{3}.
	\end{align*}
\end{baitoan}

\begin{baitoan}[GTN trên MTCT 2023, lớp 8, Bến Tre, 2.]
	Cho $\Delta ABC$ vuông tại $A$, biết $AC = 4.21$ \emph{cm}, $BC = 5.67$ \emph{cm}. Vẽ tia phân giác $BM$ của góc $\widehat{ABC}$, ($M\in AC$). Tính độ dài $BM$.
\end{baitoan}

\begin{baitoan}[GTN trên MTCT 2023, lớp 8, Bến Tre, 3.]
	Tìm $x$:
	\begin{align*}
		\dfrac{4}{\left(2 + \dfrac{2}{1 + \dfrac{9}{10}\cdot0.888\ldots}\right)x - \left(1 + \dfrac{4}{2 + \dfrac{1}{1 + \frac{7}{8}}}\right)} + \dfrac{1}{2 + \dfrac{1}{3 + \frac{1}{4}}} = -1.2727\ldots + \frac{1}{0.0555\ldots}.
	\end{align*}
\end{baitoan}

\begin{baitoan}[GTN trên MTCT 2023, lớp 8, Bến Tre, 4.]
	Cho $a,b\in\mathbb{N}$ thỏa:
	\begin{align*}
		\frac{1945}{2005} = \dfrac{1}{1 + \dfrac{1}{a + \dfrac{1}{2 + \dfrac{1}{b + \frac{1}{2}}}}}.
	\end{align*}
	Tính giá trị của biểu $M = 2022a + 2023b$.
\end{baitoan}

\begin{baitoan}[GTN trên MTCT 2023, lớp 8, Bến Tre, 5.]
	Cho đa thức $P(x) = x^4 - 6x^3 + 27x^2 - 54x + 32$. (a) Phân tích đa thức $P(x)$ thành tích của 3 đa thức. (b) Tìm nghiệm của đa thức $P(x)$.
\end{baitoan}

\begin{baitoan}[GTN trên MTCT 2023, lớp 8, Bến Tre, 6.]
	Cho hình thang vuông $ABCD$ có 2 đáy là $AB,CD$, biết $\widehat{A} = 90^\circ$, $AB = 3.73$ \emph{cm}, $AD = 4.03$ \emph{cm}, $CD = 6.07$ \emph{cm}. 2 đường chéo $AC,BD$ cắt nhau tại $E$. Tính độ dài $BE$.
\end{baitoan}

\begin{baitoan}[GTN trên MTCT 2023, lớp 8, Bến Tre, 7.]
	1 chung cư cao tầng có bóng trên mặt đất có độ dài là \emph{37.8 m}. Cùng thời điểm đó, 1 cây cột điện cao \emph{7.02 m} có bóng trên mặt đất dài \emph{2.6 m}. Hỏi chung cư đó bao nhiêu tầng biết mỗi tầng cao \emph{3.78 m}.
\end{baitoan}

\begin{baitoan}[GTN trên MTCT 2023, lớp 8, Bến Tre, 8.]
	Chị Hiền gửi vào ngân hàng số tiền là \emph{100 000 000 đồng} với lãi suất \emph{0.51\%\texttt{/}tháng}. Hỏi sau $3$ năm chị Hiền nhận được số tiền cả vốn \& lãi là bao nhiêu? (Biết tiền lãi tháng này cộng vào tiền gửi để tính lãi cho tháng sau, làm tròn đến hàng đơn vị; lãi suất không thay đổi trong 3 năm đó.)
\end{baitoan}

\begin{baitoan}[GTN trên MTCT 2023, lớp 8, Bến Tre, 9.]
	(a) Tìm 3 chữ số tận cùng của số $2023^{13}$. (b) Tính giá trị biểu thức
	\begin{align*}
		M = \frac{1}{3\cdot5\cdot7} + \frac{1}{5\cdot7\cdot9} + \cdots + \frac{1}{2017\cdot2019\cdot2021} + \frac{1}{2019\cdot2021\cdot2023}.
	\end{align*}
\end{baitoan}

\begin{baitoan}[GTN trên MTCT 2023, lớp 8, Bến Tre, 10.]
	Để thực hiện công tác phổ cập bơi cho học sinh, trường THCS A đã tiến hành lắp 1 hồ bơi di động bằng bạt có dạng hình hộp chữ nhật. Hồ bơi có chiều dài \emph{19.37 m}, chiều rộng \emph{8.81 m} \& chiều cao \emph{1.2 m}. Trường THCS A cần bơm nước vào hồ bơi sao cho mực nước cao ít nhất là \emph{0.85 m} để phục vụ cho việc dạy học bơi.
	
	Tính số tiền nước ít nhất mà trường THCS A phải sử dụng để bơm nước biết $\rm 1m^3$ nước có giá là \emph{7690 đồng}? (Làm tròn đến hàng đơn vị.)
\end{baitoan}

%------------------------------------------------------------------------------%

\printbibliography[heading=bibintoc]
	
\end{document}