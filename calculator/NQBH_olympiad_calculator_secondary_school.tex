\documentclass{article}
\usepackage[backend=biber,natbib=true,style=authoryear,maxbibnames=20]{biblatex}
\addbibresource{/home/nqbh/reference/bib.bib}
\usepackage[utf8]{vietnam}
\usepackage{tocloft}
\renewcommand{\cftsecleader}{\cftdotfill{\cftdotsep}}
\usepackage[colorlinks=true,linkcolor=blue,urlcolor=red,citecolor=magenta]{hyperref}
\usepackage{amsmath,amssymb,amsthm,float,graphicx,mathtools,xfrac}
\allowdisplaybreaks
\newtheorem{assumption}{Assumption}
\newtheorem{baitoan}{Bài toán}
\newtheorem{cauhoi}{Câu hỏi}
\newtheorem{conjecture}{Conjecture}
\newtheorem{corollary}{Corollary}
\newtheorem{dangtoan}{Dạng toán}
\newtheorem{definition}{Definition}
\newtheorem{dinhly}{Định lý}
\newtheorem{dinhnghia}{Định nghĩa}
\newtheorem{example}{Example}
\newtheorem{ghichu}{Ghi chú}
\newtheorem{hequa}{Hệ quả}
\newtheorem{hypothesis}{Hypothesis}
\newtheorem{lemma}{Lemma}
\newtheorem{luuy}{Lưu ý}
\newtheorem{menhde}{Mệnh đề}
\newtheorem{nhanxet}{Nhận xét}
\newtheorem{notation}{Notation}
\newtheorem{note}{Note}
\newtheorem{principle}{Principle}
\newtheorem{problem}{Problem}
\newtheorem{proposition}{Proposition}
\newtheorem{question}{Question}
\newtheorem{remark}{Remark}
\newtheorem{theorem}{Theorem}
\newtheorem{vidu}{Ví dụ}
\usepackage[left=1cm,right=1cm,top=5mm,bottom=5mm,footskip=4mm]{geometry}
\def\labelitemii{$\circ$}
\DeclareRobustCommand{\divby}{%
	\mathrel{\vbox{\baselineskip.95ex\lineskiplimit0pt\hbox{.}\hbox{.}\hbox{.}}}%
}

\title{Olympiad Calculator in Secondary School -- Giải Toán Trên Máy Tính Cầm Tay Cấp Trung Học Cơ Sở}
\author{Nguyễn Quản Bá Hồng\footnote{Independent Researcher, Ben Tre City, Vietnam\\e-mail: \texttt{nguyenquanbahong@gmail.com}; website: \url{https://nqbh.github.io}.}}
\date{\today}

\begin{document}
\maketitle
\begin{abstract}
	\textsc{[en]} This text is a collection of problems, from easy to advanced, about \textit{fraction}. This text is also a supplementary material for my lecture note on Elementary Mathematics grade 9, which is stored \& downloadable at the following link: \href{https://github.com/NQBH/hobby/blob/master/elementary_mathematics/grade_9/NQBH_elementary_mathematics_grade_9.pdf}{GitHub\texttt{/}NQBH\texttt{/}hobby\texttt{/}elementary mathematics\texttt{/}grade 9\texttt{/}lecture}\footnote{\textsc{url}: \url{https://github.com/NQBH/hobby/blob/master/elementary_mathematics/grade_9/NQBH_elementary_mathematics_grade_9.pdf}.}. The latest version of this text has been stored \& downloadable at the following link: \href{https://github.com/NQBH/hobby/blob/master/elementary_mathematics/grade_9/fraction/NQBH_fraction.pdf}{GitHub\texttt{/}NQBH\texttt{/}hobby\texttt{/}elementary mathematics\texttt{/}grade 9\texttt{/}fraction}\footnote{\textsc{url}: \url{https://github.com/NQBH/hobby/blob/master/elementary_mathematics/grade_9/fraction/NQBH_fraction.pdf}.}.
	\vspace{2mm}
	
	\textsc{[vi]} Tài liệu này là 1 bộ sưu tập các bài tập chọn lọc từ cơ bản đến nâng cao về \textit{phân số}. Tài liệu này là phần bài tập bổ sung cho tài liệu chính -- bài giảng \href{https://github.com/NQBH/hobby/blob/master/elementary_mathematics/grade_9/NQBH_elementary_mathematics_grade_9.pdf}{GitHub\texttt{/}NQBH\texttt{/}hobby\texttt{/}elementary mathematics\texttt{/}grade 9\texttt{/}lecture} của tác giả viết cho Toán Sơ Cấp lớp 9. Phiên bản mới nhất của tài liệu này được lưu trữ \& có thể tải xuống ở link sau: \href{https://github.com/NQBH/hobby/blob/master/elementary_mathematics/grade_9/fraction/NQBH_fraction.pdf}{GitHub\texttt{/}NQBH\texttt{/}hobby\texttt{/}elementary mathematics\texttt{/}grade 9\texttt{/}fraction}.
\end{abstract}
\tableofcontents

%------------------------------------------------------------------------------%

\section{Tính Giá Trị Biểu Thức}

\begin{baitoan}[Giải Toán Trên MTCT Khu Vực 2003--2004]
	Tính kết quả đúng của tích $M = 2222255555\cdot2222266666$.
\end{baitoan}
\noindent\textsf{Phân tích}: Khi nhập biểu thức vào máy tính cầm tay, máy tính cho kết quả tràn màn hình: $M = 2222255555\cdot2222266666 = 4.938444443\cdot10^{18}$ nên cần tách riêng từng phần ra để tính.

\begin{proof}[1st giải]
	Đặt $A = 22222$, $B = 55555$, $C = 66666$, $M = (10^5A + B)(10^5A + C) = 10^{10}A^2 + 10^5AB + 10^5AC + BC$. Tính trên MTCT: $A^2 = 22222^2 = 493817284$, $AB = 22222\cdot55555 = 1234543210$, $AC = 22222\cdot66666 = 1481451852$, $BC = 3703629630$. Tính trên giấy:
	\begin{table}[H]
		\centering
		\begin{tabular}{|c|c|c|c|c|c|c|c|c|c|c|c|c|c|c|c|c|c|c|c|}
			\hline
			$10^{10}A^2$ & 4 & 9 & 3 & 8 & 1 & 7 & 2 & 8 & 4 & 0 & 0 & 0 & 0 & 0 & 0 & 0 & 0 & 0 & 0 \\
			\hline
			$10^5AB$ &  &  &  &  & 1 & 2 & 3 & 4 & 5 & 4 & 3 & 2 & 1 & 0 & 0 & 0 & 0 & 0 & 0 \\
			\hline
			$10^5AC$ &  &  &  &  & 1 & 4 & 8 & 1 & 4 & 5 & 1 & 8 & 5 & 2 & 0 & 0 & 0 & 0 & 0 \\
			\hline
			$BC$ &  &  &  &  &  &  &  &  &  & 3 & 7 & 0 & 3 & 6 & 2 & 9 & 6 & 3 & 0 \\
			\hline
			$M$ & 4 & 9 & 3 & 8 & 4 & 4 & 4 & 4 & 4 & 3 & 2 & 0 & 9 & 8 & 2 & 9 & 6 & 3 & 0 \\
			\hline
		\end{tabular}
	\end{table}
	\noindent Vậy $M = 4938444443209829630$.
\end{proof}
Trong lời giải thứ nhất, có thể gom $10^5AB + 10^5AC = 10^5A(B + C)$ như sau (2 lời giải gần như giống hệt nhau chỉ khác là lời giải 2 ngắn hơn 1 dòng trong bảng tính trên giấy nháp):

\begin{proof}[2nd giải]
	Đặt $A = 22222$, $B = 55555$, $C = 66666$, $M = (10^5A + B)(10^5A + C) = 10^{10}A^2 + 10^5A(B + C) + BC$. Tính trên MTCT: $A^2 = 22222^2 = 493817284$, $A(B + C) = 22222\cdot(55555 + 66666) = 2715995062$, $BC = 3703629630$. Tính trên giấy:
	\begin{table}[H]
		\centering
		\begin{tabular}{|c|c|c|c|c|c|c|c|c|c|c|c|c|c|c|c|c|c|c|c|}
			\hline
			$10^{10}A^2$ & 4 & 9 & 3 & 8 & 1 & 7 & 2 & 8 & 4 & 0 & 0 & 0 & 0 & 0 & 0 & 0 & 0 & 0 & 0 \\
			\hline
			$10^5(AB + AC)$ &  &  &  &  & 2 & 7 & 1 & 5 & 9 & 9 & 5 & 0 & 6 & 2 & 0 & 0 & 0 & 0 & 0 \\
			\hline
			$BC$ &  &  &  &  &  &  &  &  &  & 3 & 7 & 0 & 3 & 6 & 2 & 9 & 6 & 3 & 0 \\
			\hline
			$M$ & 4 & 9 & 3 & 8 & 4 & 4 & 4 & 4 & 4 & 3 & 2 & 0 & 9 & 8 & 2 & 9 & 6 & 3 & 0 \\
			\hline
		\end{tabular}
	\end{table}
	\noindent Vậy $M = 4938444443209829630$.
\end{proof}

\begin{nhanxet}[Ý tưởng giải quyết với công cụ máy tính cầm tay]
	``Nếu tính trực tiếp trên MTCT thì kết quả vượt quá độ rộng hiển thị của màn hình MTCT, do đó để có kết quả chính xác ta tách các chữ số thành các bộ phận của $k$ chữ số ($k$ tùy thuộc vào từng loại máy tính cụ thể, thông thường thì $k$ bằng $\frac{1}{2}$ độ rộng hiển thị các chữ số trên màn hình MTCT), bắt đầu từ bên phải \& thực hiện tính toán. Sau đó kết hợp tính trên máy tính điện tử \& trên giấy nháp để được kết quả chính xác cuối cùng. Có thể đặt các biến cho các nhóm chữ số để thuận tiện trong biến đổi đại số (như ở 2 cách giải trên).
\end{nhanxet}

\begin{baitoan}[Mở rộng]
	Nêu cách tính kết quả đúng của tích $M = \overline{\underbrace{aa\ldots a}_n\underbrace{bb\ldots b}_n}\cdot\overline{\underbrace{aa\ldots a}_n\underbrace{cc\ldots c}_n}$ với $a,b,c\in\mathbb{N}$, $an\ne0$.
\end{baitoan}

\begin{baitoan}
	Tính chính xác $1023456^3$.
\end{baitoan}

\begin{baitoan}[Giải Toán Trên MTCT Khu Vực 2003--2004]
	Tính kết quả đúng của tích: $M = 20032003\cdot20042004$.
\end{baitoan}

\begin{baitoan}
	Tính kết quả đúng: (a) $M = 3344355664\cdot3333377777$. (b) $N = 123456^3$.
\end{baitoan}

\begin{baitoan}
	Tính kết quả đúng: (a) $A = 2001^3 + 2002^3 + 2004^3 + 2005^3 + 2006^3 + 2007^3 + 2008^3 + 2009^3$. (b) $B = 13032006\cdot13032007$.
\end{baitoan}

\begin{baitoan}
	Tính tổng: $S = \sum_{i=1}^n i\cdot i!$. Áp dụng với $n = 16$.
\end{baitoan}



%------------------------------------------------------------------------------%

\section{Tìm Thương \& Số Dư Khi Chia 2 Số Tự Nhiên}

%------------------------------------------------------------------------------%

\section{Tìm ƯCLN \& BCNN}

%------------------------------------------------------------------------------%

\section{Biểu Diễn Số Hữu Tỷ \& Liên Phân Số}

%------------------------------------------------------------------------------%

\section{Biểu Diễn Số Thập Phân Vô Hạn Tuần Hoàn Về Số Hữu Tỷ}

%------------------------------------------------------------------------------%

\section{Tìm Chữ Số Thập Phân Thứ $k$}

%------------------------------------------------------------------------------%

\section{Tính Số Chữ Số của 1 Số Dạng Lũy Thừa}

%------------------------------------------------------------------------------%

\section{Tìm $k$ Chữ Số Tận Cùng của 1 Số Tự Nhiên}

%------------------------------------------------------------------------------%

\section{Các Bài toán Liên Quan về Số Nguyên Tố}

%------------------------------------------------------------------------------%

\section{Tính Tổng Hữu Hạn}

%------------------------------------------------------------------------------%

\section{Các Bài Toán về Hàm Số Bậc Nhất}

%------------------------------------------------------------------------------%

\section{Các Bài Toán về Đa Thức}

%------------------------------------------------------------------------------%

\section{Tìm GTLN \& GTNN của Biểu Thức}

%------------------------------------------------------------------------------%

\section{Phương Trình \& Hệ Phương Trình Nghiệm Nguyên}

%------------------------------------------------------------------------------%

\section{Phương Trình \& Hệ Phương Trình Đại Số}

%------------------------------------------------------------------------------%

\section{Xác Định Các Yếu Tố Liên Quan về Tam Giác, Tứ Giác, \& Hình Tròn}

%------------------------------------------------------------------------------%

\section{Các Bài Toán về Dãy Số}

%------------------------------------------------------------------------------%

\section{Tìm Các Số \& Chữ Số Thỏa Điều Kiện Cho Trước}

%------------------------------------------------------------------------------%

\section{Các Bài Toán Liên Quan về Lượng Giác, Tính Thời Gian, \& Số Đo Góc}

%------------------------------------------------------------------------------%

\section{Các Bài Toán Liên Quan về Lãi Suất, Tiền Lương, Tăng Trưởng}

%------------------------------------------------------------------------------%

\section{Miscellaneous}

%------------------------------------------------------------------------------%

\printbibliography[heading=bibintoc]
	
\end{document}