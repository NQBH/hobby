\documentclass{article}
\usepackage[backend=biber,natbib=true,style=authoryear,maxbibnames=20]{biblatex}
\addbibresource{/home/nqbh/reference/bib.bib}
\usepackage[utf8]{vietnam}
\usepackage{tocloft}
\renewcommand{\cftsecleader}{\cftdotfill{\cftdotsep}}
\usepackage[colorlinks=true,linkcolor=blue,urlcolor=red,citecolor=magenta]{hyperref}
\usepackage{amsmath,amssymb,amsthm,float,graphicx,mathtools,xfrac}
\allowdisplaybreaks
\newtheorem{assumption}{Assumption}
\newtheorem{baitoan}{Bài toán}
\newtheorem{cauhoi}{Câu hỏi}
\newtheorem{conjecture}{Conjecture}
\newtheorem{corollary}{Corollary}
\newtheorem{dangtoan}{Dạng toán}
\newtheorem{definition}{Definition}
\newtheorem{dinhly}{Định lý}
\newtheorem{dinhnghia}{Định nghĩa}
\newtheorem{example}{Example}
\newtheorem{ghichu}{Ghi chú}
\newtheorem{hequa}{Hệ quả}
\newtheorem{hypothesis}{Hypothesis}
\newtheorem{lemma}{Lemma}
\newtheorem{luuy}{Lưu ý}
\newtheorem{menhde}{Mệnh đề}
\newtheorem{nhanxet}{Nhận xét}
\newtheorem{notation}{Notation}
\newtheorem{note}{Note}
\newtheorem{principle}{Principle}
\newtheorem{problem}{Problem}
\newtheorem{proposition}{Proposition}
\newtheorem{question}{Question}
\newtheorem{remark}{Remark}
\newtheorem{theorem}{Theorem}
\newtheorem{vidu}{Ví dụ}
\usepackage[left=1cm,right=1cm,top=5mm,bottom=5mm,footskip=4mm]{geometry}
\def\labelitemii{$\circ$}
\DeclareRobustCommand{\divby}{%
	\mathrel{\vbox{\baselineskip.95ex\lineskiplimit0pt\hbox{.}\hbox{.}\hbox{.}}}%
}

\title{Olympiad Calculator in Secondary School -- Giải Toán Trên Máy Tính Cầm Tay Cấp Trung Học Cơ Sở}
\author{Nguyễn Quản Bá Hồng\footnote{Independent Researcher, Ben Tre City, Vietnam\\e-mail: \texttt{nguyenquanbahong@gmail.com}; website: \url{https://nqbh.github.io}.}}
\date{\today}

\begin{document}
\maketitle
\begin{abstract}
	\textsc{[en]} This text is a collection of problems, from easy to advanced, about \textit{fraction}. This text is also a supplementary material for my lecture note on Elementary Mathematics grade 9, which is stored \& downloadable at the following link: \href{https://github.com/NQBH/hobby/blob/master/elementary_mathematics/grade_9/NQBH_elementary_mathematics_grade_9.pdf}{GitHub\texttt{/}NQBH\texttt{/}hobby\texttt{/}elementary mathematics\texttt{/}grade 9\texttt{/}lecture}\footnote{\textsc{url}: \url{https://github.com/NQBH/hobby/blob/master/elementary_mathematics/grade_9/NQBH_elementary_mathematics_grade_9.pdf}.}. The latest version of this text has been stored \& downloadable at the following link: \href{https://github.com/NQBH/hobby/blob/master/elementary_mathematics/grade_9/fraction/NQBH_fraction.pdf}{GitHub\texttt{/}NQBH\texttt{/}hobby\texttt{/}elementary mathematics\texttt{/}grade 9\texttt{/}fraction}\footnote{\textsc{url}: \url{https://github.com/NQBH/hobby/blob/master/elementary_mathematics/grade_9/fraction/NQBH_fraction.pdf}.}.
	\vspace{2mm}
	
	\textsc{[vi]} Tài liệu này là 1 bộ sưu tập các bài tập chọn lọc từ cơ bản đến nâng cao về \textit{phân số}. Tài liệu này là phần bài tập bổ sung cho tài liệu chính -- bài giảng \href{https://github.com/NQBH/hobby/blob/master/elementary_mathematics/grade_9/NQBH_elementary_mathematics_grade_9.pdf}{GitHub\texttt{/}NQBH\texttt{/}hobby\texttt{/}elementary mathematics\texttt{/}grade 9\texttt{/}lecture} của tác giả viết cho Toán Sơ Cấp lớp 9. Phiên bản mới nhất của tài liệu này được lưu trữ \& có thể tải xuống ở link sau: \href{https://github.com/NQBH/hobby/blob/master/elementary_mathematics/grade_9/fraction/NQBH_fraction.pdf}{GitHub\texttt{/}NQBH\texttt{/}hobby\texttt{/}elementary mathematics\texttt{/}grade 9\texttt{/}fraction}.
\end{abstract}
\tableofcontents

%------------------------------------------------------------------------------%

\section{Tính Giá Trị Biểu Thức}

%------------------------------------------------------------------------------%

\section{Tìm Thương \& Số Dư Khi Chia 2 Số Tự Nhiên}

%------------------------------------------------------------------------------%

\section{Tìm ƯCLN \& BCNN}

%------------------------------------------------------------------------------%

\section{Biểu Diễn Số Hữu Tỷ \& Liên Phân Số}

%------------------------------------------------------------------------------%

\section{Biểu Diễn Số Thập Phân Vô Hạn Tuần Hoàn Về Số Hữu Tỷ}

%------------------------------------------------------------------------------%

\section{Tìm Chữ Số Thập Phân Thứ $k$}

%------------------------------------------------------------------------------%

\section{Tính Số Chữ Số của 1 Số Dạng Lũy Thừa}

%------------------------------------------------------------------------------%

\section{Tìm $k$ Chữ Số Tận Cùng của 1 Số Tự Nhiên}

%------------------------------------------------------------------------------%

\section{Các Bài toán Liên Quan về Số Nguyên Tố}

%------------------------------------------------------------------------------%

\section{Tính Tổng Hữu Hạn}

%------------------------------------------------------------------------------%

\section{Các Bài Toán về Hàm Số Bậc Nhất}

%------------------------------------------------------------------------------%

\section{Các Bài Toán về Đa Thức}

%------------------------------------------------------------------------------%

\section{Tìm GTLN \& GTNN của Biểu Thức}

%------------------------------------------------------------------------------%

\section{Phương Trình \& Hệ Phương Trình Nghiệm Nguyên}

%------------------------------------------------------------------------------%

\section{Phương Trình \& Hệ Phương Trình Đại Số}

%------------------------------------------------------------------------------%

\section{Xác Định Các Yếu Tố Liên Quan về Tam Giác, Tứ Giác, \& Hình Tròn}

%------------------------------------------------------------------------------%

\section{Các Bài Toán về Dãy Số}

%------------------------------------------------------------------------------%

\section{Tìm Các Số \& Chữ Số Thỏa Điều Kiện Cho Trước}

%------------------------------------------------------------------------------%

\section{Các Bài Toán Liên Quan về Lượng Giác, Tính Thời Gian, \& Số Đo Góc}

%------------------------------------------------------------------------------%

\section{Các Bài Toán Liên Quan về Lãi Suất, Tiền Lương, Tăng Trưởng}

%------------------------------------------------------------------------------%

\section{Miscellaneous}

%------------------------------------------------------------------------------%

\printbibliography[heading=bibintoc]
	
\end{document}