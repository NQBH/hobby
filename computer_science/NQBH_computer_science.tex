\documentclass[oneside]{book}
\usepackage[backend=biber,natbib=true,style=authoryear]{biblatex}
\addbibresource{/home/hong/1_NQBH/reference/bib.bib}
\usepackage[vietnamese,english]{babel}
\usepackage{tocloft}
\renewcommand{\cftsecleader}{\cftdotfill{\cftdotsep}}
\usepackage[colorlinks=true,linkcolor=blue,urlcolor=red,citecolor=magenta]{hyperref}
\usepackage{amsmath,amssymb,amsthm,mathtools,float,graphicx}
\allowdisplaybreaks
\numberwithin{equation}{section}
\newtheorem{assumption}{Assumption}[chapter]
\newtheorem{conjecture}{Conjecture}[chapter]
\newtheorem{corollary}{Corollary}[chapter]
\newtheorem{definition}{Definition}[chapter]
\newtheorem{example}{Example}[chapter]
\newtheorem{lemma}{Lemma}[chapter]
\newtheorem{notation}{Notation}[chapter]
\newtheorem{principle}{Principle}[chapter]
\newtheorem{problem}{Problem}[chapter]
\newtheorem{proposition}{Proposition}[chapter]
\newtheorem{question}{Question}[chapter]
\newtheorem{remark}{Remark}[chapter]
\newtheorem{theorem}{Theorem}[chapter]
\usepackage[left=0.5in,right=0.5in,top=1.5cm,bottom=1.5cm]{geometry}
\usepackage{fancyhdr}
\pagestyle{fancy}
\fancyhf{}
\lhead{\small \textsc{Sect.} ~\thesection}
\rhead{\small \nouppercase{\leftmark}}
\renewcommand{\sectionmark}[1]{\markboth{#1}{}}
\cfoot{\thepage}
\def\labelitemii{$\circ$}

\title{Computer Science}
\author{\selectlanguage{vietnamese} Nguyễn Quản Bá Hồng\footnote{Independent Researcher, Ben Tre City, Vietnam\\e-mail: \texttt{nguyenquanbahong@gmail.com}}}
\date{\today}

\begin{document}
\maketitle
\tableofcontents

%------------------------------------------------------------------------------%

\chapter{Wikipedia's}

\section{\href{https://en.wikipedia.org/wiki/Computer_science}{Wikipedia\texttt{/}Computer Science}}
\textsf{\textbf{Fundamental areas of computer science.} \href{https://en.wikipedia.org/wiki/Programming_language_theory}{Programming language theory}, \href{https://en.wikipedia.org/wiki/Computational_complexity_theory}{Computational complexity theory}, \href{https://en.wikipedia.org/wiki/Artificial_intelligence}{Artificial intelligence}, \href{https://en.wikipedia.org/wiki/Computer_architecture}{Computer architecture}.} \href{https://en.wikipedia.org/wiki/History_of_computer_science}{History}, \href{https://en.wikipedia.org/wiki/Outline_of_computer_science}{Outline}, \href{https://en.wikipedia.org/wiki/Glossary_of_computer_science}{Glossary}, \href{https://en.wikipedia.org/wiki/Category:Computer_science}{Category}.

\textit{Computer science} is the study of \href{https://en.wikipedia.org/wiki/Computation}{computation}, \href{https://en.wikipedia.org/wiki/Automation}{automation}, \& \href{https://en.wikipedia.org/wiki/Information}{information}. Computer science spans theoretical disciplines (such as \href{https://en.wikipedia.org/wiki/Algorithm}{algorithms}, \href{https://en.wikipedia.org/wiki/Theory_of_computation}{theory of computation}, \& \href{https://en.wikipedia.org/wiki/Information_theory}{information theory}) to \href{https://en.wikipedia.org/wiki/Applied_science}{practical disciplines} (including the design \& implementation of \href{https://en.wikipedia.org/wiki/Computer_architecture}{hardware} \& \href{https://en.wikipedia.org/wiki/Computer_programming}{software}). Computer science is generally considered an area of \href{https://en.wikipedia.org/wiki/Research}{academic research} \& distinct from \href{https://en.wikipedia.org/wiki/Computer_programming}{computer programming}. 

\href{https://en.wikipedia.org/wiki/Algorithm}{Algorithms} \& \href{https://en.wikipedia.org/wiki/Data_structures}{data structures} are central to computer science. The \href{https://en.wikipedia.org/wiki/Theory_of_computation}{theory of computation} concerns abstract \href{https://en.wikipedia.org/wiki/Models_of_computation}{models of computation} \& general classes of \href{https://en.wikipedia.org/wiki/Computational_problem}{problems} that can be solved using them. The fields of \href{https://en.wikipedia.org/wiki/Cryptography}{cryptography} \& \href{https://en.wikipedia.org/wiki/Computer_security}{computer security} involve studying the means for secure communication \& for preventing \href{https://en.wikipedia.org/wiki/Vulnerability_(computing)}{security vulnerabilities}. \href{https://en.wikipedia.org/wiki/Computer_graphics_(computer_science)}{Computer graphics} \& \href{https://en.wikipedia.org/wiki/Computational_geometry}{computational geometry} address the generation of images. \href{https://en.wikipedia.org/wiki/Programming_language_theory}{Programming language theory} considers approaches to the description of computational processes, \& \href{https://en.wikipedia.org/wiki/Database}{database} theory concerns the management of repositories of data. \href{https://en.wikipedia.org/wiki/Human%E2%80%93computer_interaction}{Human--computer interaction} investigates the interfaces through which humans \& computers interact, \& \href{https://en.wikipedia.org/wiki/Software_engineering}{software engineering} focuses on the design \& principles behind developing software. Areas such as \href{https://en.wikipedia.org/wiki/Operating_system}{operating systems}, \href{https://en.wikipedia.org/wiki/Computer_network}{networks} \& \href{https://en.wikipedia.org/wiki/Embedded_system}{embedded systems} investigate the principles \& design behind \href{https://en.wikipedia.org/wiki/Complex_systems}{complex systems}. \href{https://en.wikipedia.org/wiki/Computer_architecture}{Computer architecture} describes the construction of computer components \& computer-operated equipment. \href{https://en.wikipedia.org/wiki/Artificial_intelligence}{Artificial intelligence} \& \href{https://en.wikipedia.org/wiki/Machine_learning}{machine learning} aim to synthesize goal-orientated processes such as problem-solving, decision-making, environmental adaptation, \href{https://en.wikipedia.org/wiki/Automated_planning_and_scheduling}{planning} \& learning found in humans \& animals. Within artificial intelligence, \href{https://en.wikipedia.org/wiki/Computer_vision}{computer vision} aims to understand \& process image \& video data, while \href{https://en.wikipedia.org/wiki/Natural-language_processing}{natural-language processing} aims to understand \& process textual \& linguistic data.

The fundamental concern of computer science is determining what can \& cannot be automated. The \href{https://en.wikipedia.org/wiki/Turing_Award}{Turning Award} is generally recognized as the highest distinction in computer science.'' -- \href{https://en.wikipedia.org/wiki/Computer_science}{Wikipedia\texttt{/}computer science}

\subsection{History}

\subsection{Etymology}

\subsection{Philosophy}

\subsubsection{Epistemology of computer science}

\subsubsection{Paradigms of computer science}

\subsection{Fields}

\subsubsection{Theoretical computer science}

\paragraph{Theory of computation.}

\paragraph{Information \& coding theory.}

\paragraph{Data structures \& algorithms.}

\paragraph{Programming language theory \& formal methods.}

\subsubsection{Computer systems \& computational processes}

\paragraph{Artificial intelligence.}

\paragraph{Computer architecture \& organization.}

\paragraph{Concurrent, parallel \& distributed computing.}

\paragraph{Computer networks.}

\paragraph{Computer security \& cryptography.}

\paragraph{Databases \& data mining.}

\paragraph{Computer graphics \& visualization.}

\paragraph{Image \& sound processing.}

\subsubsection{Applied computer science}

\paragraph{Computational science, finance \& engineering.}

\paragraph{Social computing \& human--computer interaction.}

\paragraph{Software engineering.}

\subsection{Discoveries}

\subsection{Programming paradigms}

\subsection{Academia}

\subsection{Education}

%------------------------------------------------------------------------------%

\chapter{The Art of Computer Programming}

\begin{thebibliography}{99}
	\bibitem[]{}
\end{thebibliography}

%------------------------------------------------------------------------------%

\printbibliography[heading=bibintoc]
	
\end{document}