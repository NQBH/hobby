\documentclass[oneside]{book}
\usepackage[backend=biber,natbib=true,style=authoryear]{biblatex}
\addbibresource{/home/hong/1_NQBH/reference/bib.bib}
\usepackage[vietnamese,english]{babel}
\usepackage{tocloft}
\renewcommand{\cftsecleader}{\cftdotfill{\cftdotsep}}
\usepackage[colorlinks=true,linkcolor=blue,urlcolor=red,citecolor=magenta]{hyperref}
\usepackage{amsmath,amssymb,amsthm,mathtools,float,graphicx}
\allowdisplaybreaks
\numberwithin{equation}{section}
\newtheorem{assumption}{Assumption}[chapter]
\newtheorem{conjecture}{Conjecture}[chapter]
\newtheorem{corollary}{Corollary}[chapter]
\newtheorem{definition}{Definition}[chapter]
\newtheorem{example}{Example}[chapter]
\newtheorem{lemma}{Lemma}[chapter]
\newtheorem{notation}{Notation}[chapter]
\newtheorem{principle}{Principle}[chapter]
\newtheorem{problem}{Problem}[chapter]
\newtheorem{proposition}{Proposition}[chapter]
\newtheorem{question}{Question}[chapter]
\newtheorem{remark}{Remark}[chapter]
\newtheorem{theorem}{Theorem}[chapter]
\usepackage[left=0.5in,right=0.5in,top=1.5cm,bottom=1.5cm]{geometry}
\usepackage{fancyhdr}
\pagestyle{fancy}
\fancyhf{}
\lhead{\small \textsc{Sect.} ~\thesection}
\rhead{\small \nouppercase{\leftmark}}
\renewcommand{\sectionmark}[1]{\markboth{#1}{}}
\cfoot{\thepage}
\def\labelitemii{$\circ$}

\title{Computer Science}
\author{\selectlanguage{vietnamese} Nguyễn Quản Bá Hồng\footnote{Independent Researcher, Ben Tre City, Vietnam\\e-mail: \texttt{nguyenquanbahong@gmail.com}; website: \url{https://nqbh.github.io}.}}
\date{\today}

\begin{document}
\maketitle
\tableofcontents

%------------------------------------------------------------------------------%

\chapter{Wikipedia's}

\section{\href{https://en.wikipedia.org/wiki/Computer_science}{Wikipedia\texttt{/}Computer Science}}
\textsf{\textbf{Fundamental areas of computer science.} \href{https://en.wikipedia.org/wiki/Programming_language_theory}{Programming language theory}, \href{https://en.wikipedia.org/wiki/Computational_complexity_theory}{Computational complexity theory}, \href{https://en.wikipedia.org/wiki/Artificial_intelligence}{Artificial intelligence}, \href{https://en.wikipedia.org/wiki/Computer_architecture}{Computer architecture}.} \href{https://en.wikipedia.org/wiki/History_of_computer_science}{History}, \href{https://en.wikipedia.org/wiki/Outline_of_computer_science}{Outline}, \href{https://en.wikipedia.org/wiki/Glossary_of_computer_science}{Glossary}, \href{https://en.wikipedia.org/wiki/Category:Computer_science}{Category}.

\textit{Computer science} is the study of \href{https://en.wikipedia.org/wiki/Computation}{computation}, \href{https://en.wikipedia.org/wiki/Automation}{automation}, \& \href{https://en.wikipedia.org/wiki/Information}{information}. Computer science spans theoretical disciplines (such as \href{https://en.wikipedia.org/wiki/Algorithm}{algorithms}, \href{https://en.wikipedia.org/wiki/Theory_of_computation}{theory of computation}, \& \href{https://en.wikipedia.org/wiki/Information_theory}{information theory}) to \href{https://en.wikipedia.org/wiki/Applied_science}{practical disciplines} (including the design \& implementation of \href{https://en.wikipedia.org/wiki/Computer_architecture}{hardware} \& \href{https://en.wikipedia.org/wiki/Computer_programming}{software}). Computer science is generally considered an area of \href{https://en.wikipedia.org/wiki/Research}{academic research} \& distinct from \href{https://en.wikipedia.org/wiki/Computer_programming}{computer programming}. 

\href{https://en.wikipedia.org/wiki/Algorithm}{Algorithms} \& \href{https://en.wikipedia.org/wiki/Data_structures}{data structures} are central to computer science. The \href{https://en.wikipedia.org/wiki/Theory_of_computation}{theory of computation} concerns abstract \href{https://en.wikipedia.org/wiki/Models_of_computation}{models of computation} \& general classes of \href{https://en.wikipedia.org/wiki/Computational_problem}{problems} that can be solved using them. The fields of \href{https://en.wikipedia.org/wiki/Cryptography}{cryptography} \& \href{https://en.wikipedia.org/wiki/Computer_security}{computer security} involve studying the means for secure communication \& for preventing \href{https://en.wikipedia.org/wiki/Vulnerability_(computing)}{security vulnerabilities}. \href{https://en.wikipedia.org/wiki/Computer_graphics_(computer_science)}{Computer graphics} \& \href{https://en.wikipedia.org/wiki/Computational_geometry}{computational geometry} address the generation of images. \href{https://en.wikipedia.org/wiki/Programming_language_theory}{Programming language theory} considers approaches to the description of computational processes, \& \href{https://en.wikipedia.org/wiki/Database}{database} theory concerns the management of repositories of data. \href{https://en.wikipedia.org/wiki/Human%E2%80%93computer_interaction}{Human--computer interaction} investigates the interfaces through which humans \& computers interact, \& \href{https://en.wikipedia.org/wiki/Software_engineering}{software engineering} focuses on the design \& principles behind developing software. Areas such as \href{https://en.wikipedia.org/wiki/Operating_system}{operating systems}, \href{https://en.wikipedia.org/wiki/Computer_network}{networks} \& \href{https://en.wikipedia.org/wiki/Embedded_system}{embedded systems} investigate the principles \& design behind \href{https://en.wikipedia.org/wiki/Complex_systems}{complex systems}. \href{https://en.wikipedia.org/wiki/Computer_architecture}{Computer architecture} describes the construction of computer components \& computer-operated equipment. \href{https://en.wikipedia.org/wiki/Artificial_intelligence}{Artificial intelligence} \& \href{https://en.wikipedia.org/wiki/Machine_learning}{machine learning} aim to synthesize goal-orientated processes such as problem-solving, decision-making, environmental adaptation, \href{https://en.wikipedia.org/wiki/Automated_planning_and_scheduling}{planning} \& learning found in humans \& animals. Within artificial intelligence, \href{https://en.wikipedia.org/wiki/Computer_vision}{computer vision} aims to understand \& process image \& video data, while \href{https://en.wikipedia.org/wiki/Natural-language_processing}{natural-language processing} aims to understand \& process textual \& linguistic data.

The fundamental concern of computer science is determining what can \& cannot be automated. The \href{https://en.wikipedia.org/wiki/Turing_Award}{Turning Award} is generally recognized as the highest distinction in computer science.'' -- \href{https://en.wikipedia.org/wiki/Computer_science}{Wikipedia\texttt{/}computer science}

\subsection{History}

\subsection{Etymology}

\subsection{Philosophy}

\subsubsection{Epistemology of computer science}

\subsubsection{Paradigms of computer science}

\subsection{Fields}

\subsubsection{Theoretical computer science}

\paragraph{Theory of computation.}

\paragraph{Information \& coding theory.}

\paragraph{Data structures \& algorithms.}

\paragraph{Programming language theory \& formal methods.}

\subsubsection{Computer systems \& computational processes}

\paragraph{Artificial intelligence.}

\paragraph{Computer architecture \& organization.}

\paragraph{Concurrent, parallel \& distributed computing.}

\paragraph{Computer networks.}

\paragraph{Computer security \& cryptography.}

\paragraph{Databases \& data mining.}

\paragraph{Computer graphics \& visualization.}

\paragraph{Image \& sound processing.}

\subsubsection{Applied computer science}

\paragraph{Computational science, finance \& engineering.}

\paragraph{Social computing \& human--computer interaction.}

\paragraph{Software engineering.}

\subsection{Discoveries}

\subsection{Programming paradigms}

\subsection{Academia}

\subsection{Education}

%------------------------------------------------------------------------------%

\section{\href{https://en.wikipedia.org/wiki/How_to_Solve_it_by_Computer}{Wikipedia\texttt{/}How to Solve it by Computer}}
``\textit{How to Solve it by Computer} is a \href{https://en.wikipedia.org/wiki/Computer_science}{computer science} book by R. G. Dromey, 1st published by \href{https://en.wikipedia.org/wiki/Prentice-Hall}{Prentice-Hall} in 182. It is occasionally used as a \href{https://en.wikipedia.org/wiki/Textbook}{textbook}, especially in India.

It is an introduction to the \textit{whys} of \href{https://en.wikipedia.org/wiki/Algorithms}{algorithms} \& \href{https://en.wikipedia.org/wiki/Data_structures}{data structures}. Features of the book:
\begin{enumerate}
	\item The design factors associated with problems
	\item The creative process behind coming up with innovative solutions for algorithms \& data structures
	\item The line of reasoning behind the constraints, factors \& the design choices made.
\end{enumerate}
The very fundamental algorithms portrayed by this book are mostly presented in \href{https://en.wikipedia.org/wiki/Pseudocode}{pseudocode} \&\texttt{/}or \href{https://en.wikipedia.org/wiki/Pascal_(programming_language)}{Pascal} notation.'' -- \href{https://en.wikipedia.org/wiki/How_to_Solve_it_by_Computer}{Wikipedia\texttt{/}How to Solve it by Computer}

%------------------------------------------------------------------------------%

\part{The Art of Computer Programming}

\section*{\href{https://www-cs-faculty.stanford.edu/~knuth/taocp.html}{The Art of Computer Programming (TAOCP)}}
``At the end of 1999, these books were named among the best 12 physical-science monographs of the century by \href{http://web.mnstate.edu/schwartz/centurylist2.html}{American Scientists}, along with: Dirac on quantum mechanics, Einstein on relativity, Mandelbrot on fractals, Pauling on the chemical bond, Russell \& Whitehead on foundations of mathematics, von Neumann \& Morgensstern on game theory, Wiener on cybernetics, Woodward \& Hoffmann on orbital symmetry, Feynmann on quantum electrodynamics, Smith on search for structure, \& Einstein's collected papers. Wow'' \href{https://www-cs-faculty.stanford.edu/~knuth/taocp.html}{``historic'' publisher's brochure from the 1st edition of Vol. 1 (1968)}. A complimentary \textit{downloadable PDF containing the collected indexes} is \href{https://www.informit.com/store/art-of-computer-programming-volumes-1-4a-boxed-set-9780321751041}{available from the publisher} to registered owners of the 4-volume boxed set. This PDF also includes the complete indexes of Vols. 1, 2, 3, \& 4A, as well as to Vol. 1 Fascicle 1 \& to Vol. 4 Fascicles 5 \& 6.''

\subsection*{eBook versions}
``These volumes are now available also in portable electronic form, using PDF format prepared by the experts at \href{https://msp.org/}{Mathematical Sciences Publishers}. Special care has been taken to make the search feature work well. Thousands of useful ``clickable'' cross-references are also provided -- from exercises to their answers \& back, from the index to the text, from the text to important tables \& figures, etc.

\textbf{Warning.} Unfortunately, however, non-PDF versions have also appeared, against my recommendations, \& those versions are frankly quite awful. A great deal of expertise \& care is necessary to do the job right. If you have been misled into purchasing 1 of these inferior versions (e.g.,a Kindle version), the publishers have told me that they will replace your copy with the PDF edition that I have personally approved. \textbf{Do not purchase eTAOCP in Kindle format if you expect the mathematics to make sense}. (The ePUB format may be just as bad; I really don't want to know, \& I am really sorry that it was released). Please do not tell me about errors that you find in a non-PDF eBook; such mistakes should be reported directly to the publisher. Some non-PDF versions also masquerade as PDF. You can tell an authorized version because its copyright page (with the exception of Vol. 4 Fascicle 5) will say `Electronic version by Mathematical Sciences Publishers (MSP)'.

The authorized PDF versions can be purchased at \url{www.informit.com/taocp}. If you have purchased a different version of the eBook, \& can provide proof of purchase of that eBook, you can obtain a gratis PDF verson by sending email \& proof of purchase to \url{taocp@pearson.com}.''

\subsection*{Volume 1}
\begin{itemize}
	\item \textit{Fundamental Algorithms}, 3rd Edition (Reading, Massachusetts: Addison-Wesley, 1997), xx+650pp.
	\item \textit{Volume 1 Fascicle 1}, MMIX: A RISC Computer for the New Millennium (2005), v+134pp.
\end{itemize}

\subsubsection{Brochure}

\begin{quotation}
	``I am overwhelmed by the wealth of exciting \& fresh material you have managed to pack into the book, especially in view of the fact that it is only the 1st of 7 volumes! ``Monumental'' is the only word for it $\ldots$ Moreover, it is written with a grace \& humor that is, as you know, exceedingly rare in books on mathematics. I greatly enjoyed your dedication, your flow-chart for reading the series, your notes on the exercises; above all, your choice of illustrative material throughout \& the clarity \& brevity with which you explain everything.'' -- Martin Gardner, \textit{Mathematical Games, Scientific American}
\end{quotation}
``This combined reference \& text, \textit{Fundamental Algorithms}, is the 1st volume of a planned 7-volume series. The series will provide a unified, readable, \& theoretically sound summary of the present knowledge of computer programming techniques, plus a study of their historical development.

The point of view adopted by the author differs from many contemporary books about compute programming. The author does not try to teach the reader how to use somebody else's subroutines, but is concerned rather with teaching the reader how to write better subroutines himself.

A reader who is interested primarily in programming rather than in the associated mathematics may stop reading each section as soon as the mathematics become recognizably difficult. On the other hand, a mathematically oriented reader will find a wealth of interesting material.

As a reference the series provides valuable information for system programmers, analyst programmers, \& others in the computer \& related software industries. All 7 volumes of this series may also be used in senior or graduate courses such as: Information Structures, Computer Science, Combinatorial Mathematics, Computer-Oriented Finite Mathematics, or Fundamentals of Symbolic Machine Language Programming.

Among the areas covered in Vol. 1 are the representation of information inside a computer; the structural interrelations between data elements \& how to deal with them efficiently; plus applications to simulation, numerical methods, software design, \& other factors. Also included is an introduction to fundamental topics in discrete mathematics, of special importance in the study of computer programming techniques.

There are over 850 exercises, graded according to the level of difficulty from extremely simple questions to unsolved research problems. Answers are supplied for over 90\% of the exercises. This enhances the value of the book for self-study, classroom use, \& for reference. \& it helps make it possible to organize the book so that it can be read by both mathematicians \& non-mathematicians.'' 634 pages, 71 figures, (1968), \$19.50

\textsf{Fig. 28. Representation of polynomials using 4-directional links. Shaded areas of nodes indicate information irrelevant in the context considered.}

%------------------------------------------------------------------------------%

\chapter{The Art of Computer Programming. Vol. 1: Fundamental Algorithms}

\begin{quotation}
	\textit{This series of books is affectionately\footnote{\textbf{affectionately} [adv] in a way that shows caring feelings \& love for somebody.} dedicated to the Type 650 computer once installed at Case Institute of Technology, in remembrance\footnote{\textbf{remembrance} [n] [uncountable] the act or process of remembering an event in the past or a person who is dead.} of many pleasant\footnote{\textbf{pleasant} [a] (\textbf{pleasanter, pleasantest}) (\textbf{more pleasant} \& \textbf{most pleasant} are more common) enjoyable, pleasing or attractive, \textsc{opposite}: \textbf{unpleasant}.} evenings.}
\end{quotation}

\section*{Preface}
\begin{quotation}
	\textit{``Here is your book, the one your thousands of letters have asked us to publish. It has taken us years to do, checking \& rechecking countless\footnote{\textbf{countless} [a] [usually before noun] very many; too many to be counted or mentioned.} recipes\footnote{\textbf{recipe} [n] \textbf{1.} \textbf{recipe (for something)} a set of instructions that tells you how to cook something \& the ingredients you need for it; \textbf{2.} \textbf{recipe for something} a method or an idea that seems likely to have a particular result, \textsc{synonym}: \textbf{formula}.} to bring you only the best, only the interesting, only the perfect. Now we can say, without a shadow of a doubt, that every single 1 of them, if you follow the directions to the letter, will work for you exactly as well as it did for us, even if you have never cooked before.''} -- McCall's Cookbook (1963)
\end{quotation}
``\textsc{The process} of preparing programs for a digital\footnote{\textbf{digital} [a] \textbf{1.} using a system of receiving \& sending information as a series of the numbers 1 \& 0, showing that an electronic signal is there or is not there; connected with computer technology; \textbf{2.} (of clocks, watches, etc.) displaying only the appropriate numbers, rather than pointing to numbers from a larger set of numbers; other information displayed in this way; \textbf{3.} connected with a finger or the fingers of the hand.} computer is especially attractive\footnote{\textbf{attractive} [a] \textbf{1.} (of a person or an animal) pleasant to look at, especially in a sexual way; making an animal interested in a sexual way, \textsc{opposite}: \textbf{unattractive}; \textbf{2.} (of a thing or a place) pleasant to look at or be in, \textsc{opposite}: \textbf{unattractive}; \textbf{3.} having features or qualities that make something seem interesting \& worth having, \textsc{synonym}: \textbf{appealing}, \textsc{opposite}: \textbf{unattractive}; \textbf{4.} (\textit{physics}) involving the force thta pulls things towards each other, \textsc{opposite}: \textbf{repulsive}.}, not only because it can be economically\footnote{\textbf{economically} [adv] \textbf{1.} in a way that is connected with the trade, industry \& development of wealth of a country, an area or a society; \textbf{2.} in a way that provides good service or value in relation to the amount of time or money spent.} \& scientifically\footnote{\textbf{scientifically} [adv] \textbf{1.} in a way that is connected with science; \textbf{2.} in a careful \& organized way.} rewarding\footnote{\textbf{rewarding} [a] \textbf{1.} (of an activity) worth doing; that makes you happy because you think it is useful or important; \textbf{2.} producing a lot of money, \textsc{synonym}: \textbf{profitable}.}, but also because it can be an aesthetic\footnote{\textbf{aesthetic} [a] (\textit{North American English also} \textbf{esthetic}) \textbf{1.} concerned with beauty \& art \& the understanding of beautiful things; \textbf{2.} beautiful to look at; [n] (\textit{North American English also} \textbf{esthetic}) \textbf{1.} [countable] \textbf{aesthetic (of something)} a set of principles that express the aesthetic qualities \& ideas of a particular artist or a particular group of artists, writers, etc.; \textbf{2.} (\textbf{aesthetics}) [uncountable] the branch of philosophy that studies the principles of beauty, especially in art.} experience much like composing\footnote{\textbf{compose} [v] \textbf{1.} (\textbf{be composed of something}) to be made or formed from several substances, parts of people; \textbf{2.} (not used in the progressive tenses) \textbf{compose something} to combine together to form a whole, \textsc{synonym}: \textbf{make something up}; \textbf{3.} to write a piece of music; \textbf{4.} to write something, especially a poem.} poetry\footnote{\textbf{poetry} [n] [uncountable] a collection of poems; poems in general, \textsc{synonym}: \textbf{verse}.} or music. This book is the 1st volume of a multi-volume set of books that has been designed to train the reader in various skills that go into a programmer's craft\footnote{\textbf{craft} [n] \textbf{1.} [countable, uncountable] an activity involving a special skill at making things with your hands; \textbf{2.} [singular] all the skills needed for a particular activity; \textbf{3.} (plural \textbf{craft}) [countable] a boat or ship; [v] [usually passive] \textbf{craft something} to make something using a special skill, \textsc{synonym}: \textbf{fashion}.}.

The following chapters are \textit{not} meant to serve as an introduction to computer programming; the reader is supposed to have had some previous experience. The prerequisites\footnote{\textbf{prerequisite} [n] [usually singular] something that must exist or happen before something else can happen or be done, \textsc{synonym}: \textbf{precondition}.} are actually very simple, but a beginner requires time \& practice in order to understand the concept of a digital computer. The reader should possess:
\begin{itemize}
	\item[a)] Some idea of how a stored-program digital computer works; not necessarily the electronics\footnote{\textbf{electronics} [n] \textbf{1.} [uncountable] the branch of science \& technology that studies electric currents in electronic equipment; \textbf{2.} [uncountable] the use of electronic technology; the making of electronic products; \textbf{3.} [plural] the electronic circuits \& components used in electronic equipment.}, rather the manner\footnote{\textbf{manner} [n] \textbf{1.} [singular] the way that something is done or happens; \textbf{2.} [singular] the way that somebody behaves towards other people; \textbf{3.} (\textbf{manners}) [plural] behavior that is considered to be polite in a particular society or culture; \textbf{4.} (\textbf{manners (of somebody\texttt{/}something)}) [plural] the habits \& customs of a particular group of people; \textbf{all manner of somebody\texttt{/}something} [idiom] many different types of people or things; \textbf{in the manner of somebody\texttt{/}something} [idiom] in a style that is typical of somebody\texttt{/}something.} in which instructions\footnote{\textbf{instruction} [n] \textbf{1.} (\textbf{instructions}) [plural] detailed information on how to do or use something, \textsc{synonym}: \textbf{direction}; \textbf{2.} [countable, usually plural] something that somebody tells you to do, \textsc{synonym}: \textbf{order}; \textbf{3.} [countable] (\textit{computing}) a code in a program that tells a computer to perform a particular operation; \textbf{4.} [uncountable] the act of teaching something to somebody.} can be kept in the machine\footnote{\textbf{machine} [n] \textbf{1.} (often in compounds) a piece of equipment with moving parts that is designed to do a particular job. The power used to work a machine may be electricity, steam, gas, etc. or human power; \textbf{2.} a group of people who operate in an efficient way within an organization.}'s memory\footnote{\textbf{memory} [n] (plural \textbf{memories}) \textbf{1.} [countable, uncountable] your ability to remember things; the part of your mind in which you store things that you remember; \textbf{2.} [uncountable] \textbf{in\texttt{/}within $\ldots$ memory} the period of time that somebody is able to remember events; \textbf{3.} [countable] a thought of something that you remember from the past; \textbf{4.} [uncountable, countable] the part of a computer where data are stored; the amount of space in a computer for storing data; \textbf{5.} [uncountable] \textbf{memory (of somebody)} what is remembered about somebody after they have died; \textbf{from memory} [idiom] without reading or looking at notes; \textbf{in memory of somebody, to the memory of somebody} [idiom] intended to show respect \& remind people of somebody who has died; \textbf{within\texttt{/}in living memory} [idiom] at a time, or during the time, that is remembered by people still alive.} \& successively\footnote{\textbf{successive} [a] [only before noun] following immediately one after the other, \textsc{synonym}: \textbf{consecutive}.} executed\footnote{\textbf{execute} [v] \textbf{1.} [usually passive] to kill somebody, especially as a legal punishment; \textbf{2.} \textbf{execute something} to do a piece of work, perform a duty, put a plan into action, etc.; \textbf{3.} \textbf{execute something} (\textit{computing}) carry out an instruction or program; \textbf{4.} \textbf{execute something} (\textit{law}) to follow the instructions in a legal document; to make a document legally valid.}.
	\item[b)] An ability to put the solutions to problems into such explicit\footnote{\textbf{explicit} [a] \textbf{1.} saying something clearly \& exactly; \textbf{2.} showing or referring to sex in a very obvious or detailed way.} terms that a computer can ``understand'' them. (These machines have no common sense\footnote{\textbf{common sense} [n] [uncountable] the ability to think about things in a practical way \& make sensible decisions.}; they do exactly\footnote{\textbf{exactly} [adv] used to emphasize that something is correct in every way or in every detail, \textsc{synonym}: \textbf{precisely}.} as they are told, no more \& no less. This fact is the hardest concept to grasp\footnote{\textbf{grasp} [v] \textbf{1.} to understand something completely; \textbf{2.} \textbf{grasp an opportunity} to take an opportunity without hesitating \& use it; \textbf{3.} \textbf{grasp somebody\texttt{/}something} to take a firm hold of somebody\texttt{/}something, \textsc{synonym}: \textbf{grip}; [n] [usually singular] \textbf{1.} a person's understanding of a subject; \textbf{2.} a firm hold of somebody\texttt{/}something or control over somebody\texttt{/}something; \textbf{3.} the ability to get or achieve something.} when one 1st tries to use a computer.)
	\item[c)] Some knowledge of the most elementary\footnote{\textbf{elementary} [a] \textbf{1.} connected with the 1st stages of a course of study, or the 1st years at school; \textbf{2.} of the most basic kind; \textbf{3.} very simple \& easy.} computer techniques, such as looping\footnote{\textbf{loop} [n] \textbf{1.} a shape like a curve or circle made by a line curving right around; \textbf{2.} a piece of rope, wire, etc. in the shape of a curve or circle; \textbf{3.} a long, narrow piece of film or tape on which the pictures \& sound are repeated continuously; \textbf{4.} (\textit{computing}) a set of instructions that is repeated again \& again until a particular condition is satisfied; \textbf{5.} a complete circuit for electrical current; \textbf{6.} (\textit{British English}) a railway line or road that leaves the main track or road \& then joins it again; [v] \textbf{1.} [transitive] \textbf{loop something $+$ adv.\texttt{/}prep.} to form or bend something into a loop; \textbf{2.} [intransitive] \textbf{$+$ adv.\texttt{/}prep.} to move in a way that makes the shape of a loop; \textbf{loop the loop} [idiom] to fly or make a plane fly in a circle going up \& down.} (performing\footnote{\textbf{perform} [v] \textbf{1.} [transitive] \textbf{perform something} to do something, such as a piece of work, task or duty, \textsc{synonym}: \textbf{carry something out}; \textbf{2.} [intransitive] \textbf{$+$ adv.\texttt{/}prep.} to work or function well or badly. In this meaning, where there is no adverb or preposition, \textbf{perform} means `perform well'.; \textbf{3.} [transitive, intransitive] \textbf{perform (something)} to entertain an audience by playing a piece of music, acting in a play, etc.} a set of instructions\footnote{\textbf{instruction} [n] \textbf{1.} (\textbf{instructions}) [plural] detailed information on how to do or use something, \textsc{synonym}: \textbf{direction}; \textbf{2.} [countable, usually plural] something that somebody tells you to do, \textsc{synonym}: \textbf{order}; \textbf{3.} [countable] (\textit{computing}) a code in a program that tells a computer to perform a particular operation; \textbf{4.} [uncountable] the act of teaching something to somebody.} repeatedly\footnote{\textbf{repeatedly} [adv] many times; again \& again.}), the use of subroutines\footnote{\textbf{subroutine} [n] (also \textbf{subprogram}) (\textit{computing}) a set of instructions which perform a task within a program.}, \& the use of indexed\footnote{\textbf{index} [n] \textbf{1.} (plural \textbf{indexes}) \textbf{index (to something)} (in a book or set of books) an alphabetical list of names, subjects, etc. with the numbers of the pages on which they are mentioned; \textbf{2.} (\textbf{indexes, indices}) a number in a system or scale that represents the average value of particular prices, shares, etc. compared with a previous or standard value; \textbf{3.} (\textbf{indices}) \textbf{index of something} a sign or measure that something else can be judged by; \textbf{4.} (in compounds) a number that gives the value of a physical quality in terms of a standard formula; \textbf{5.} (usually \textbf{indices} [plural] \textit{mathematics}) a small number written above another number to show how many times the other number must be multiplied by itself. In the equation $4^2 = 16$, the number 2 is an index.; \textbf{6.} (\textbf{indexes}) (\textit{computing}) a list of items, each of which identifies a particular record in a computer file or database \& contains information about its address; [v] \textbf{1.} \textbf{index something} to record names, subjects, etc. in an index; \textbf{2.} \textbf{index something} to provide an index to something; \textbf{3.} [usually passive] \textbf{index something (to something)} to link salaries, prices, etc. to the level of prices of food \& other goods so that they both increase at the same rate.} variables\footnote{\textbf{variable} [n] \textbf{1.} an element or a feature that is likely to vary or change, \textsc{opposite}: \textbf{constant}; \textbf{2.} a property that is measured or observed in an experiment or a study; a property that is adjusted in an experiment, \textsc{opposite}: \textbf{constant}; \textbf{3.} (\textit{mathematics}) a quantity in a calculation that can take any of a set of different numerical values, represented by a symbol such as $x$, \textsc{opposite}: \textbf{constant}; [a] \textbf{1.} often changing; likely to change, \textsc{synonym}: \textbf{fluctuating}, \textsc{opposite}: \textbf{constant}; \textbf{2.} not the same in all parts or cases; not having a fixed pattern, \textsc{synonym}: \textbf{diverse}. When \textbf{variable} is used to describe the quality of something, the tone is slightly disapproving, meaning that some parts of it are good \& some are bad, \textsc{synonym}: \textbf{inconsistent, mixed}, \textsc{opposite}: \textbf{consistent, uniform}; \textbf{3.} that can be changed to meet different needs or suit different conditions, \textsc{opposite}: \textbf{fixed}; \textbf{4.} (\textit{mathematics}) (of a quantity) that can take any of a set of different numerical values, represented by a symbol such as $x$, \textsc{opposite}: \textbf{constant}.}.
	\item[d)] A little knowledge of common computer jargon\footnote{\textbf{jargon} [n] [uncountable] (\textit{often disapproving}) words or expressions that are used by a particular profession or group of people, \& are difficult for others to understand.} -- ``memory,'' ``registers\footnote{\textbf{register} [v] \textbf{1.} [transitive, intransitive] to record the name of somebody\texttt{/}something on an official list; \textbf{2.} [transitive] \textbf{register something} to make your opinion known officially or publicly; \textbf{3.} [intransitive] \textbf{$+$ noun} (of a measuring instrument) to show or record an amount; \textbf{4.} [transitive] \textbf{register something} to achieve a particular score or result; \textbf{5.} [transitive] \textbf{register something} t notice something \& remember it; [n] \textbf{1.} [countable] \textbf{register (of something)} an official list or record of names or items; a book that contains such a list; \textbf{2.} [countable, uncountable] (\textit{linguistics}) the level \& style of a piece of writing or speech, that is usually appropriate to the situation that it is used in; \textbf{3.} [countable] (\textit{computing}) (in electronic devices) a location in a store of data, used for a particular purpose \& with quick access time.},'' ``bits\footnote{\textbf{bit} [n] \textbf{1.} [countable] the smallest unit of information used by a computer; \textbf{2.} (\textbf{a bit}) [singular] (used as an adverb) (\textit{especially British English, rather informal}) rather; \textbf{3.} (\textbf{a bit}) [singular] (used as an adverb) a small amount; a little; \textbf{4.} [countable] \textbf{bit of something} (\textit{especially British English, rather informal}) a small piece or part of something; \textbf{5.} [countable] \textbf{bit of something} (\textit{especially British English, rather informal}) a small amount of something; \textbf{bit by bit} [idiom] (\textit{rather informal}) a piece or part at a time; gradually; \textbf{every bit as good, bad, etc. (as somebody\texttt{/}something)} [idiom] (\textit{rather informal}) just as good, bad, etc.; equally good, bad, etc.},'' ``floating\footnote{\textbf{float} [v] \textbf{1.} [intransitive] \textbf{$+$ adv.\texttt{/}prep.} to move slowly on or in water or in the air; \textbf{2.} [intransitive] to stay on or near the surface of a liquid \& not sink; \textbf{3.} [transitive] \textbf{float something ($+$ adv.\texttt{/}prep.)} to make something move on or near the surface of a liquid; \textbf{4.} [transitive] \textbf{float something} to suggest an idea or a plan for other people to consider; \textbf{5.} [transitive] \textbf{float something} to sell shares in a company or business to the public for the 1st time; \textbf{6.} [transitive, intransitive] (\textit{economics}) if a government floats its country's money or allows it to float, it allows its value to change freely according to the value of the money of other countries.} point,'' ``overflow\footnote{\textbf{overflow} [v] \textbf{1.} [intransitive, transitive] to be so full that the contents go over the sides; \textbf{2.} [intransitive] \textbf{overflow (with something)} (of a place) to have too many people in it; \textbf{3.} [intransitive, transitive] \textbf{overflow (into something) $|$ overflow (something)} to spread beyond the limits of a place or container that is too full; [n] \textbf{1.} [uncountable, singular] a number of people or things that do not fit into the space available; \textbf{2.} [uncountable, singular] the action of liquid flowing out of a container, etc. that is already full; the liquid that flows out; \textbf{3.} (also \textbf{overflow pipe}) [countable] a pipe that allows extra liquid to flow away safely when a container is full; \textbf{4.} [countable, usually singular] (\textit{computing}) a fault that happens because a number or data item is too large for the computer to represent it exactly.},'' ``software\footnote{\textbf{software} [n] [uncountable] the programs used by a computer for doing particular jobs.}.'' Most words not defined in the text are given brief definitions\footnote{\textbf{definition} [n] \textbf{1.} [countable] an exact statement or description of the nature, extent or meaning of something; \textbf{2.} [countable] a statement of the exact meaning of a word or phrase, especially in a dictionary; \textbf{3.} [uncountable] the action or process of stating the exact meaning of a word or phrase; \textbf{by definition} [idiom] as a result of what something is.} in the index at the close of each volume.
\end{itemize}
These 4 prerequisites can perhaps be summed up into the single requirement that the reader should have already written \& tested at least, say, 4 programs for at least 1 computer.

I have tried to write this set of books in such a way that it will fill several needs. In the 1st place, these books are reference works that summarize\footnote{\textbf{summarize} [v] (\textit{British English also} \textbf{summarise}) [transitive, intransitive] \textbf{summarize (something)} to give a summary of something.} the knowledge that has been acquired\footnote{\textbf{acquire} [v] \textbf{1.} \textbf{acquire something} to learn or develop a skill, habit or quality; \textbf{2.} \textbf{acquire something} to obtain something by buying or being given it; \textbf{3.} \textbf{acquire something} to come to have a particular reputation.} in several important fields. In the 2nd place, they can be used as textbooks\footnote{\textbf{textbook} [n] (\textit{North American English also} \textbf{text}) a book that teaches a particular subject \& that is used especially in schools \& colleges.} for self-study\footnote{\textbf{self-study} [n] [uncountable] the activity of learning about something without a teacher to help you; [a] designed to help students to learn about something without a teacher to help them.} or for college\footnote{\textbf{college} [n] \textbf{1.} [countable, uncountable] (often in names) (in the US) a university where students can study for a degree after they have left school; \textbf{2.} [countable, uncountable] (often in names) (in Britain) a place where students go to study or to receive training after they have left school; \textbf{3.} [countable, uncountable] (often in names) 1 of the separate institutions that come British universities, such as Oxford \& Cambridge, are divided into; \textbf{4.} [countable, uncountable] (often in names) (in the US) 1 of the main divisions of some large universities; \textbf{5.} [countable $+$ singular or plural verb] (usually in names) an organized group of professional people with special interests, duties or powers.} courses\footnote{\textbf{course} [n] \textbf{1.} [countable] a series of classes or lectures on a particular subject; \textbf{2.} [countable] (\textit{especially British English}) a period of study at a college or university that leads to an exam or a qualification; \textbf{3.} [singular] the way that something develops or should develop; \textbf{4.} (also \textbf{course of action}) [countable] a way of acting in or dealing with a particular situation; \textbf{5.} [countable, usually singular] the general direction in which somebody's ideas or actions are moving; \textbf{6.} [uncountable, countable, usually singular] a direction or route follows by a ship or an aircraft, or by another moving object; \textbf{7.} [countable] \textbf{course (of something)} a series of medical treatments.} in the computer \& information\footnote{\textbf{information} [n] [uncountable] \textbf{1.} facts or details about somebody\texttt{/}something that are provided or learned; \textbf{2.} data that are stored, analyzed or passed on by a computer; \textbf{3.} what is shown by a particular arrangement of things.} sciences\footnote{\textbf{information scienec} [n] (also \textbf{informatics}) [uncountable] (\textit{computing}) the study of processes for storing \& obtaining information.}. To meet both of these objectives, I have incorporated\footnote{\textbf{incorporate} [v] \textbf{1.} to include something so that it forms a part of something; \textbf{2.} [usually passive] (\textit{business}) to create a legally recognized company.} a large number of exercises into the text \& have furnished\footnote{\textbf{furnish} [v] \textbf{1.} \textbf{furnish something (with something)} to put furniture in a house, room, etc.; \textbf{2.} (\textit{formal}) to supply something; to supply or provide somebody\texttt{/}something with something.} answers for most of them. I have also made an effort to fill the page with facts rather than with vague\footnote{\textbf{vague} [a] (\textbf{vaguer, vaguest}) not having or giving enough information or details about something.}, general commentary\footnote{\textbf{commentary} [n] (plural \textbf{commentaries}) \textbf{1.} [countable] \textbf{commentary (on something)} a written explanation or discussion of something such as a theory or book; \textbf{2.} [countable, uncountable] \textbf{commentary (on something)} a criticism or discussion of something.}.

This set of books is intended\footnote{\textbf{intend} [v] \textbf{1.} to have a plan, result or purpose in your mind when you do something; \textbf{2.} \textbf{intend something as something $|$ intend something to be something} to plan that something should have a particular meaning or use.}\,\footnote{\textbf{intended} [a] \textbf{1.} meant or designed to be something or to be used by somebody; \textbf{2.} [only before noun] that you are trying to achieve or reach.} for people who will be more than just casually\footnote{\textbf{casual} [a] \textbf{1.} [usually before noun] without paying attention to detail; \textbf{2.} [usually before noun] not showing much care or thought; \textbf{3.} [usually before noun] (of a relationship) lasting only a short time \& without deep affection; \textbf{4.} [usually before noun] (\textit{British English}) (of work) not permanent; not regular; \textbf{5.} not formal; \textbf{6.} [only before noun] happening by chance; doing something by chance.} interested\footnote{\textbf{interested} [a] \textbf{1.} [not usually before noun] giving your attention to something because you enjoy finding out about it or doing it; showing interest in something \& finding it exciting; \textbf{2.} [usually before noun] in a position to gain from a situation or be affected by it.} in computers, yet it is by no means only for the computer specialist\footnote{\textbf{specialist} [n] \textbf{1.} a doctor who has specialized in a particular area of medicine; \textbf{2.} \textbf{specialist (in something)} a person who is an expert in a particular area of work or study; [a] [only before noun] \textbf{1.} connected with a doctor who has specialized in a particular area of medicine; \textbf{2.} having or involving detailed knowledge of a particular topic or area of study.}. Indeed, 1 of my main goals has been to make these programming\footnote{\textbf{programming} [n] [uncountable] \textbf{1.} the process of writing \& testing programs for computers; \textbf{2.} \textbf{programming (of something)} the activity of planning which television or radio programmes to broadcast; the programmes that are broadcast; \textbf{3.} factors, ranging from genetic to social, that instruct a person or animal to behave in a certain way.} techniques more accessible\footnote{\textbf{accessible} [a] \textbf{1.} that can be reached, entered, used or obtained; \textbf{2.} easy to understand.} to the many people working in other fields who can make fruitful use of computers, yet who cannot afford\footnote{\textbf{afford} [v] \textbf{1.} [no passive] (usually used with \textit{can, could} or \textit{be able to}, especially in negative sentences or questions) to have enough money or time to be able to buy or to do something; \textbf{2.} [no passive] \textbf{afford to do something} (usually used with \textit{can} or \textit{could}, especially in negative sentences \& questions) if you say that you cannot afford to do something, you mean that you should not do it because it will cause problems for you if you do; \textbf{3.} (\textit{formal}) to provide somebody with something.} the time to locate\footnote{\textbf{locate} [v] \textbf{1.} [transitive] \textbf{locate somebody\texttt{/}something} to find the exact position of somebody\texttt{/}something; \textbf{2.} [transitive] \textbf{locate something $+$ adv.\texttt{/}prep.} to put or build something in a particular place; \textbf{3.} [intransitive] \textbf{$+$ adv.\texttt{/}prep.} (\textit{especially North American English}) to start a business in a particular place.} all of the necessary information that is buried\footnote{\textbf{bury} [v] \textbf{1.} \textbf{bury somebody\texttt{/}something} to place something in the ground, especially a dead body in a grave; \textbf{2.} [often passive] \textbf{bury somebody\texttt{/}something} to cover something with soil, rocks, leaves, etc.; \textbf{3.} \textbf{buy something} to ignore or hide a feeling, a mistake, etc.} in technical\footnote{\textbf{technical} [a] \textbf{1.} [usually before noun] connected with the use of science or technology; involving the use of machines; \textbf{2.} [usually before noun] connected with a particular type of activity, or the skills \& processes needed for it; \textbf{3.} [usually before noun] (of language, writing or ideas) requiring knowledge \& understanding of a particular subject; \textbf{4.} connected with the details of a law or set of rules.} journals\footnote{\textbf{journal} [n] \textbf{1.} a newspaper or magazine that deals with a particular subject or profession; \textbf{2.} a written record of the things you do or see every day.}.

We might call the subject of these books ``nonnumerical analysis.'' Computers have traditionally\footnote{\textbf{traditionally} [adv] \textbf{1.} according to what has always or usually happened in the past; \textbf{2.} according to the beliefs, customs or way of life that have existed for a long time among a particular group of people; according to what is believed.} been associated\footnote{\textbf{associate} [v] \textbf{1.} [transitive] to make a connection between people or things in your mind, \textsc{synonym}: \textbf{connect, relate, link}, \textsc{opposite}: \textbf{dissociate}; \textbf{2.} [intransitive] \textbf{associate with somebody} to spend time with somebody, especially somebody that others do not approve of; \textbf{3.} [transitive] \textbf{associate yourself with something} to show that your support or agree with something, \textsc{opposite}: \textbf{dissociate}; [n] \textbf{1.} a person that you work with, do business with or spend a lot of time with; \textbf{2.} (\textbf{Association}) \textbf{associate (of something)} an associate member of an organization; [a] [only before noun] \textbf{1.} (often in titles) of a lower rank; having fewer rights in a particular profession or organization; \textbf{2.} joined or connected with a profession or an organization.}\,\footnote{\textbf{associated} [a] \textbf{1.} if 1 thing is associated with another, the 2 things are connected because they happen together or 1 thing causes the other, \textsc{synonym}: \textbf{connected}; \textbf{2.} if a person is associated with a person, organization or idea, they support it; \textbf{3.} (of a company) connected or joined with another company or companies.} with the solution\footnote{\textbf{solution} [n] \textbf{1.} [countable] a way of solving a problem or dealing with a difficult situation, \textsc{synonym}: \textbf{answer}; \textbf{2.} [countable] \textbf{solution (to\texttt{/}for\texttt{/}of something)} an answer to a problem in mathematics; \textbf{3.} [countable, uncountable] a liquid in which a substance has been dissolved, so that the substance has become part of the liquid; the state of being dissolved in a liquid.} of numerical\footnote{\textbf{numerical} [a] (also \textit{less frequent} \textbf{numeric}) [usually before noun] connected with numbers; expressed in numbers.} problems such as the calculation\footnote{\textbf{calculation} [n] [countable, uncountable] \textbf{1.} the act or process of using numbers to find out an amount; \textbf{2.} the process of using your judgment to decide what the results would be of doing something.} of the roots\footnote{\textbf{root} [n] \textbf{1.} [countable] the part of a plant that grows under the ground \& absorbs water \& minerals that it sends to the rest of the plant; \textbf{2.} [countable, usually singular] \textbf{root of something} the main cause of something, such as a problem or difficult situation; \textbf{3.} [countable, usually plural] the basis of something; \textbf{4.} (\textbf{roots}) [plural] the feelings or connections that you have with a place because you have lived there or your family came from there; \textbf{5.} [countable] (\textit{linguistics}) the part of a word that has the main meaning \& that its other forms are based on; a word that other words are formed from; \textbf{6.} [countable] \textbf{root (of something)} (\textit{mathematics}) a quantity which, when multiplied by itself a particular number of times, produces another quantity; \textbf{root \& branch} [idiom] thorough \& complete; \textbf{take root} [idiom] \textbf{1.} (of a plant) to develop roots; \textbf{2.} (of an idea) to become widely accepted; [v] [intransitive] (of a plant) to grow roots; \textbf{root something\texttt{/}somebody out} [phrasal verb] to find a person or thing that is causing a problem \& remove or get rid of them.} of an equation\footnote{\textbf{equation} [n] \textbf{1.} [countable] (\textit{mathematics}) a statement showing that 2 amounts or values are equal; \textbf{2.} [countable] (\textit{chemistry}) a statement using symbols to show the changes that happen in a chemical reaction; \textbf{3.} [uncountable, singular] the act of making something equal or considering something as equal; \textbf{4.} [countable, usually singular] a situation in which several factors must be considered \& dealt with.}, numerical interpolation\footnote{\textbf{interpolation} [n] [uncountable, countable] \textbf{1.} (\textit{formal}) a remark that interrupts a conversation; the act of making a remark that interrupts a conversation; \textbf{2.} (\textit{formal}) a thing that is added to a piece of writing; the act of adding something to a piece of writing, \textsc{synonym}: \textbf{insertion}; \textbf{3.} (\textit{mathematics}) the act of adding a value into a series by calculating it from surrounding known values.} \& integration\footnote{\textbf{integration} [n] \textbf{1.} [uncountable, countable] the act or process of combining 2 or more things so that they work together; \textbf{2.} [uncountable] the act or process of mixing people who have previously been separated, usually because of color, race or religion; \textbf{3.} [uncountable, countable] \textbf{integration (of something)} (\textit{mathematics}) the process of finding an integral or integrals.}, etc., but such topics are not treated here except in passing\footnote{\textbf{passing} [n] [uncountable] \textbf{1.} \textbf{the passing of time\texttt{/}the years} the process of time going by; \textbf{2.} \textbf{passing (of somebody\texttt{/}something)} the fact of something ending or of somebody dying; \textbf{3.} \textbf{the passing of something} the act of making something become a law; \textbf{in passing} [idiom] done or said while you are giving your attention to something else.}. Numerical computer programming is an extremely\footnote{\textbf{extremely} [adv] (usually with adjectives \& adverbs) to a very high degree.} interesting \& rapidly\footnote{\textbf{rapidly} [adv] in a short period of time or at a fast rate.} expanding\footnote{\textbf{expand} [v] \textbf{1.} [intransitive, transitive] to become greater in size, number or importance; to make something greater in size, number or importance; \textbf{2.} [transitive] \textbf{expand something} to write something such as a scientific formula in a longer form; \textbf{expand on\texttt{/}upon something} [phrasal verb] to add more details \& give more information about something.} field, \& many books have been written about it. Since the early 1960s, however, computers have been used even more often for problems in which numbers occur only by coincidence\footnote{\textbf{coincidence} [n] \textbf{1.} [countable, uncountable] the fact of 2 things happening at the same time by chance, often in a surprising way; \textbf{2.} [singular, uncountable] \textbf{coincidence of A with\texttt{/}\& B} the fact of things being present in the same place at the same time; \textbf{3.} [singular, uncountable] \textbf{coincidence of something} the fact of 2 or more opinions, etc. being the same.}; the computer's decision-making\footnote{\textbf{decision-making} [n] (also \textbf{decision making}) [uncountable] the process of deciding about something important, especially in a group of people or in an organization.} capabilities\footnote{\textbf{capability} [n] (plural \textbf{capabilities}) \textbf{1.} [countable, uncountable] the ability or qualities necessary to do something; \textbf{2.} [countable] the power or weapons that a country has for war or for military action.} are being used, rather than its ability\footnote{\textbf{ability} [n] (plural \textbf{abilities}) \textbf{1.} [singular] the fact that somebody\texttt{/}something is able to do something, \textsc{opposite}: \textbf{inability}; \textbf{2.} [uncountable, countable] a level of skill or intelligence.} to do arithmetic\footnote{\textbf{arithmetic} [n] [uncountable] \textbf{1.} the type of mathematics that deals with the use of numbers in counting \& calculation; \textbf{2.} the use of numbers in counting \& calculation; [a] (\textit{mathematics}) \textbf{1.} \textbf{arithmetic progression\texttt{/}series} a series in which the interval between each term \& the next remains constant; \textbf{2.} \textbf{arithmetic mean} $=$ \textbf{mean}; \textbf{3.} $=$ \textbf{arithmetical}.}\,\footnote{\textbf{arithmetical} [a] (also \textbf{arithmetic}) connected with arithmetic.}. We have some use for addition\footnote{\textbf{addition} [n] \textbf{1.} [uncountable, countable] the act of adding something to something else, \textsc{opposite}: \textbf{removal}; \textbf{2.} [countable] \textbf{addition (to something)} a thing that is added to something else; \textbf{3.} [uncountable, countable] the process of adding 2 or more numbers together to find their total; \textbf{in addition (to somebody\texttt{/}something)} [idiom] used to introduce a new fact or argument.} \& subtraction\footnote{\textbf{subtraction} [n] [uncountable, countable] the process of taking a number or amount away from another number or amount.} in nonnumerical problems, but we rarely feel any need for multiplication\footnote{\textbf{multiplication} [n] [uncountable] \textbf{1.} the act or process of multiplying 1 number by another, \textsc{opposite}: \textbf{division}; \textbf{2.} the process of increasing very much in number or amount.} \& division\footnote{\textbf{division} [n] \textbf{1.} [uncountable, countable, usually singular] the process or result of dividing into separate parts; the process or result of dividing something or sharing it out; \textbf{2.} [uncountable] the process of providing 1 number by another; \textbf{3.} [countable, usually plural, uncountable] a disagreement or difference in people's opinions of ways of life, especially between members of a society or an organization; \textbf{4.} [countable] \textbf{division (of something)} a part of something into which it is divided; \textbf{5.} [countable $+$ singular or plural verb] (abbr., \textbf{Div.}) a large \& important unit or section of an organization.}. Of course, even a person who is primarily\footnote{\textbf{primarily} [adv] mainly, \textsc{synonym}: \textbf{chiefly}.} concerned\footnote{\textbf{concerned} [a] \textbf{1.} worried \& feeling concern about something; \textbf{2.} interested in something; \textbf{as\texttt{/}so far as somebody\texttt{/}something is concerned, as\texttt{/}so far as somebody\texttt{/}something goes} [idiom] usd to give facts or an opinion about a particular aspect of something.} with numerical computer programming will benefit\footnote{\textbf{benefit} [n] \textbf{1.} [countable, uncountable] a helpful \& useful effect that something has; an advantage that something provides; \textbf{2.} [uncountable, countable] (\textit{British English}) money provided by the government to people who need financial help because they are unemployed, sick, etc.; \textbf{give somebody the benefit of the doubt} [idiom] to accept that somebody has told the truth or has not done something wrong because you cannot prove that they have not told the truth\texttt{/}have done something wrong; [v] \textbf{1.} [intransitive] to be in a better position because of something; \textbf{2.} [transitive] \textbf{benefit somebody\texttt{/}something} to be useful or provide an advantage to somebody\texttt{/}something.} from a study of the nonnumerical techniques, for they are present in the background of numerical programs as well.

The results of research in nonnumerical analysis are scattered\footnote{\textbf{scatter} [v] \textbf{1.} [intransitive, transitive] (\textit{physics}) to change direction or spread in many directions; to make something change direction or spread in many directions; \textbf{2.} [transitive, often passive] \textbf{scatter something (on\texttt{/}over\texttt{/}around something)} to throw or drop things in different directions so that they cover an area; \textbf{3.} [transitive, often passive] to be found spread over an area rather than all together; \textbf{4.} [intransitive, transitive] (of people or animals) to move very quickly in different directions; to make people or animals do this, \textsc{synonym}: \textbf{disperse}; [n] (also \textbf{scattering}) [singular] \textbf{scatter (of something)} a small amount of something or a small number of people or things spread over an area.}\,\footnote{\textbf{scattered} [a] \textbf{scattered (throughout\texttt{/}across\texttt{/}around something)} spread far apart over a wide area, \textsc{synonym}: \textbf{dispersed}.} throughout numerous\footnote{\textbf{numerous} [a] [usually before noun] existing in large numbers, \textsc{synonym}: \textbf{many}.} technical journals. My approach has been to try to distill\footnote{\textbf{distil} [v] (\textit{North American English also} \textbf{distill}) \textbf{1.} \textbf{distil something (from something)} to make a liquid pure by heating it until it becomes a gas, then cooling it \& collecting the drops of liquid that form; \textbf{2.} \textbf{distil  something} to make something such as a strong alcoholic drink in this way; \textbf{3.} \textbf{distil something (from\texttt{/}into something)} to get the essential meaning or ideas from thoughts, information or experiences.} this vast literature by studying the techniques that are most basic, in the sense that they can be applied to many types of programming situations\footnote{\textbf{situation} [n] \textbf{1.} all the circumstances \& things that are happening at a particular time \& in a particular place; \textbf{2.} the area or place where something is located.}. I have attempted\footnote{\textbf{attempt} [n] \textbf{1.} [countable, uncountable] an act of trying to do something difficult, often with no success; \textbf{2.} [countable] an act of trying to kill somebody; [v] to try to do or provide something, especially something difficult.}\,\footnote{\textbf{attempted} [a] [only before noun] (of a crime, etc.) that somebody has tried to do but without success.} to coordinate\footnote{\textbf{coordinate} [v] (\textit{British English also} \textbf{co-ordinate}) to organize the different parts of an activity \& the people involved in it so that it works well; \textbf{coordinate with somebody} [phrasal verb] to reach an agreement with other people about how to work together effectively; [n] (\textit{British English also} \textbf{co-ordinate}) 1 of the numbers or letters used to fix the position of a point on a map or graph.} the ideas into more or less of a ``theory,'' as well as to show how the theory applies to a wide variety\footnote{\textbf{variety} [n] (plural \textbf{varieties}) \textbf{1.} [singular] \textbf{variety (of something)} a number or range of different things of the same general type; \textbf{2.} [uncountable] the quality of not being the same in all parts or not doing the same thing all the time, \textsc{synonym}: \textbf{diversity}; \textbf{3.} [countable] a type of a thing, e.g. a plant or language, that is different from others in the same general group. In biology, a \textbf{variety} is a category below a \textbf{species} \& \textbf{subspecies}, used especially to describe plants.} of practical problems.

Of course, ``nonnumerical analysis'' is a terribly\footnote{\textbf{terribly} [adv] \textbf{1.} (\textit{especially British English}) very; \textbf{2.} very much; very badly.} negative name for this field of study; it is much better to have a positive, descriptive\footnote{\textbf{descriptive} [a] \textbf{1.} describing what something is like, rather than saying what it should be like or what category it belongs to; \textbf{2.} saying or showing clearly what something is like; giving a clear account of something.} term that characterizes\footnote{\textbf{characterize} [v] (\textit{British English also} \textbf{characterise}) \textbf{1.} [usually passive] \textbf{characterize something} to be the most typical or most obvious quality or feature of something, \textsc{synonym}: \textbf{typify}; \textbf{2.} [usually passive] \textbf{characterize something} to be the feature or quality that makes something different from similar things, \textsc{synonym}: \textbf{distinguish}; \textbf{3.} [often passive] \textbf{characterize somebody\texttt{/}something (as something)} to describe something\texttt{/}somebody in a particular way.} the subject. ``Information processing\footnote{\textbf{process} [n] \textbf{1.} a series of actions that are taken in order to achieve a particular result; \textbf{2.} a series of things that happen, especially ones that result in natural changes; \textbf{3.} a method of doing or making something, especially one that is used in industry; \textbf{in process} [idiom] being done; continuing; \textbf{in the process} [idiom] while doing something else, especially as a result that is not intended; \textbf{in the process of (doing) something} [idiom] in the middle of doing something that takes some time to do; [v] \textbf{1.} to treat raw material, food, etc. in order to change it, preserve it, etc.; \textbf{2.} \textbf{process something} to deal officially with something such as a document, application or request; \textbf{3.} \textbf{process something} to deal with information by performing a series of operations on it, especially using a computer; \textbf{processing} [n] [uncountable].}'' is too broad\footnote{\textbf{broad} [a] (\textbf{broader, broadest}) \textbf{1.} wide, \textsc{opposite}: \textbf{narrow}; \textbf{2.} including a great variety of things, \textsc{opposite}: \textbf{narrow}; \textbf{3.} [only before noun] general; not detailed; \textbf{4.} with most people agreeing about something in a general way; \textbf{5.} covering a wide area.} a designation\footnote{\textbf{designation} [n] \textbf{1.} [uncountable, countable] the fact of describing somebody\texttt{/}something as having a particular character, status or purpose; \textbf{2.} [countable] a way of naming or describing somebody\texttt{/}something; a name or label.} for the material\footnote{\textbf{material} [n] \textbf{1.} [countable, uncountable] a substance from which a thing is or can be made; a substance with a particular quality; \textbf{2.} [uncountable] information or ideas used in books or other work; \textbf{3.} [countable, usually plural, uncountable] things that are needed in order to do a particular activity, \textsc{synonym}: \textbf{resource}; \textbf{4.} [uncountable, countable] cloth used for making clothes, etc., \textsc{synonym}: \textbf{cloth, fabric}; [a] \textbf{1.} [only before noun] connected with money \& possessions rather than with the needs of the mind or spirit, \textsc{opposite}: \textbf{spiritual}; \textbf{2.} [only before noun] connected with the physical world rather than with the mind or spirit, \textsc{opposite}: \textbf{spiritual}; \textbf{3.} important \& needing to be considered. In law, \textbf{material} is used to describe evidence or facts that are important, especially when these facts might have an effect on the result of a case.} I am considering, \& ``programming techniques'' is too narrow\footnote{\textbf{narrow} [a] (\textbf{narrower, narrowest}) \textbf{1.} measuring a short distance from 1 side to the other, especially in relation to length, \textsc{opposite}: \textbf{broad, wide}; \textbf{2.} limited in extent, amount or variety, \textsc{synonym}: \textbf{restricted}, \textsc{opposite}: \textbf{wide}; \textbf{3.} limited in range \& unwilling or unable to accept opinions that are different from yours, \textsc{opposite}: \textbf{broad}; \textbf{4.} strict in meaning, \textsc{opposite}: \textbf{broad}; \textbf{5.} [usually before noun] only just achieved; [v] [intransitive, transitive] \textbf{1.} to become or make something more limited in extent, amount or variety; \textbf{2.} to become or make something less wide; \textbf{narrow something down (to something)} [phrasal verb] to reduce the number of possibilities or choices.}. Therefore I wish to propose\footnote{\textbf{propose} [v] \textbf{1.} to suggest a plan or an idea for people to consider \& decide on; \textbf{2.} to suggest an explanation of something for people to consider.} \textit{analysis\footnote{\textbf{analysis} [n] (plural \textbf{analyses}) \textbf{1.} [uncountable, countable] the detailed study or examination of something in order to understand more about it; the result of the study; \textbf{2.} [uncountable, countable] a careful examination of a substance in order to find out what it consists of; \textbf{3.} [uncountable] $=$ \textbf{psychoanalysis}; \textbf{in the final\texttt{/}last analysis} [idiom] used to say what is most important after everything has been discussed or considered.} of algorithms\footnote{\textbf{algorithm} [n] a process or set of rules to be followed when solving a particular problem, especially by a computer.}} as an appropriate\footnote{\textbf{appropriate} [a] suitable, acceptable or correct for the particular circumstances; [v] \textbf{1.} \textbf{appropriate something} to start to use something that belongs to a different time, place or culture; \textbf{2.} \textbf{appropriate something} to take somebody's idea, property or money for your own use, especially without permission; \textbf{3.} \textbf{appropriate something} to take something from somebody, legally \& often by force, \textsc{synonym}: \textbf{seize}; \textbf{4.} \textbf{appropriate something} to take or give something, especially money, for a particular purpose.} name for the subject matter covered in these books. This name is meant to imply\footnote{\textbf{imply} [v] \textbf{1.} to make it seem likely that something is true or exists, \textsc{synonym}: \textbf{suggest}; \textbf{2.} to suggest that something is true or that you feel or think something, without saying so directly; \textbf{3.} (of a fact or event) to suggest something as a likely or necessary result.} ``the theory of the properties of particular computer algorithms.''

The complete set of books, entitled \textit{The Art of Computer Programming}, has the following general outline:
\begin{itemize}
	\item \textit{Vol. 1: Fundamental\footnote{\textbf{fundamental} [a] \textbf{1.} serious \& very important; affecting the most central \& important parts of something, \textsc{synonym}: \textbf{basic}; \textbf{2.} forming the necessary basis of something, \textsc{synonym}: \textbf{essential}.} Algorithms}
	\begin{itemize}
		\item Chap. 1. Basic Concepts
		\item Chap. 2. Information Structures\footnote{\textbf{structure} [n] \textbf{1.} [uncountable, countable] the way in which the parts of something are connected together, arranged or organized; a particular arrangement of parts; \textbf{2.} [countable] a thing that is made of several parts arranged in a particular way, e.g. a building; \textbf{3.} [uncountable, countable] the state of being well organized or planned with all the parts linked together; a careful plan; [v] [often passive] to arrange or organize something into a system or pattern.}
	\end{itemize}
	\item \textit{Vol. 2. Seminumerical Algorithms}
	\begin{itemize}
		\item Chap. 3. Random\footnote{\textbf{random} [a] \textbf{1.} [usually before noun] done, chosen, etc. so that all possible choices have an equal chance of being considered; \textbf{2.} without any regular pattern; [n] \textbf{at random} [idiom] without deciding in advance what is going to happen; without any regular pattern.} Numbers
		\item Chap. 4. Arithmetic
	\end{itemize}
	\item \textit{Vol. 3. Sorting\footnote{\textbf{sort} [n] a group or type of people or things that have similar characteristics; a particular variety or type, \textsc{synonym}: \textbf{kind}; \textbf{of sorts, of a sort, a sort of something} [idiom] used when you are saying that something is the thing mentioned, but only to a certain degree, or that something is not a good example of something; [v] [often passive] to arrange things in groups or in a particular order according to their type or to what they contain; \textbf{sort something out} [phrasal verb] (\textit{rather informal}) to deal with something successfully; to stop something from being a problem. \textbf{Sort something out} is much more common in general English; in academic English, a more formal word or expression such as \textbf{resolve} or \textbf{deal with} is more common.; \textbf{sort something out (from something)} [phrasal verb] to separate something from something else; to recognize something as different from something else; \textbf{sort through something} [phrasal verb] to look at or consider different things in order to find the most useful or important ones.} \& Searching\footnote{\textbf{search} [n] \textbf{1.} an attempt to find somebody\texttt{/}something, especially by looking carefully for them\texttt{/}it; \textbf{2.} an act of looking for information in a computer database, on the Internet, etc.; an act of looking for information in official documents; [v] \textbf{1.} [intransitive, transitive] to look carefully for something\texttt{/}somebody; to examine a particular place when looking for something\texttt{/}somebody; \textbf{2.} [transitive] \textbf{search somebody (for something)} (especially of the police) to examine somebody's clothes, etc. in order to find something that they may be hiding; \textbf{3.} [intransitive, transitive] to look in a computer database, on the Internet, etc. in order to find information; to look at official documents in order to find information; \textbf{4.} [intransitive] \textbf{search for something} to think carefully, especially in order to find the answer to a problem; \textbf{search something\texttt{/}somebody out} [phrasal verb] to look for something\texttt{/}somebody until you find them.}}
	\begin{itemize}
		\item Chap. 5. Sorting
		\item Chap. 6. Searching
	\end{itemize}
	\item \textit{Vol. 4. Combinatorial Algorithms}
	\begin{itemize}
		\item Chap. 7. Combinatorial Searching
		\item Chap. 8. Recursion\footnote{\textbf{recursion} [n] [uncountable] (\textit{mathematics}) the process of repeating a function, each time applying it to the result of the previous stage.}
	\end{itemize}
	\item \textit{Vol. 5. Syntactical\footnote{\textbf{syntactic} [a] (\textit{linguistics}) connected with syntax}\,\footnote{\textbf{syntactically} [adv] (\textit{linguistics}) in a way that is connected with syntax.} Algorithms}
	\begin{itemize}
		\item Chap. 9. Lexical\footnote{\textbf{lexical} [a] [usually before noun] (\textit{linguistics}) connected with the words of a language.} Scanning\footnote{\textbf{scan} [v] \textbf{1.} [transitive, intransitive] to look at every part of something carefully, especially because you are looking for a particular thing or person; \textbf{2.} [transitive] \textbf{scan something (for something)} to look quickly at something in order to find relevant information; \textbf{3.} [transitive, usually passive] to use a machine to get an image of the inside of somebody's body \& show it on a computer screen; \textbf{4.} [transitive, usually passive] to use a machine that uses light or another means to read data or to produce an image of an object or document in digital form; [n] \textbf{1.} a medical test in which a machine produces an image of the inside of a person's body \& shows it on a computer screen; \textbf{2.} \textbf{scan (of something)} the use of a machine to read data or produce an image of something; \textbf{3.} \textbf{scan (of something)} the act of looking quickly through something written or printed, usually in order to find something or to get a general idea about what is in it.}
		\item Chap. 10. Parsing\footnote{\textbf{parse} [v] \textbf{parse something} to divide a sentence into parts \& describe the grammar of each word or part.}
	\end{itemize}
\end{itemize}
Vol. 4 deals with such a large topic, it actually represents\footnote{\textbf{represent} [v] \textbf{1.} \textbf{$+$ noun} \textit{linking verb} (not used in the progressive tenses) to be something; to be equal to something, \textsc{synonym}: \textbf{constitute}; \textbf{2.} (not used in the progressive tenses) \textbf{represent something} to be a symbol or sign of something, \textsc{synonym}: \textbf{symbolize}; \textbf{3.} [no passive] \textbf{represent something} to be typical of something; \textbf{4.} to show or describe somebody\texttt{/}something in a particular way, \textsc{synonym}: \textbf{present}; \textbf{5.} (not used in the progressive tenses) \textbf{represent somebody\texttt{/}something} to include a particular type or number of people or things; \textbf{6.} \textbf{be represented $+$ adv.\texttt{/}prep.} to be present to a particular degree; \textbf{7.} to act or speak for somebody\texttt{/}something; to attend or take part in an event on behalf of somebody; \textbf{8.} \textbf{represent somebody\texttt{/}something} to show somebody\texttt{/}something, especially in a picture or diagram, \textsc{synonym}: \textbf{depict}; \textbf{9.} \textbf{represent something} to say or suggest something that you want somebody to believe or pay attention to.} several\footnote{\textbf{several} [determiner, pronoun] more than 2 but not very many.} separate\footnote{\textbf{separate} [a] \textbf{1.} forming a unity by itself; not joined to, touching or close to something\texttt{/}somebody else; \textbf{2.} [usually before noun] different; not connected; \textbf{go your (own) separate ways} [idiom] to end a relationship with somebody; [v] \textbf{1.} [intransitive, transitive] to divide into different parts or groups; to divide things into different parts or groups; \textbf{2.} [intransitive, transitive] to move apart; to make people or things move apart; \textbf{3.} [transitive] to be between 2 people, areas, countries, etc. so that they are not touching or connected; \textbf{4.} [intransitive] to stop living together as a couple with your husband, wife or partner; \textbf{5.} [transitive] to make somebody\texttt{/}something different in some way from somebody\texttt{/}something else, \textsc{synonym}: \textbf{divide}; \textbf{separate out $|$ separate something out} [phrasal verb] to divide into different parts; to divide something into different parts.} books (Vols. 4A, 4B, \& so on). 2 additional\footnote{\textbf{additional} [a] more than was 1st mentioned or is already present or is usual, \textsc{synonym}: \textbf{extra}.} volumes on more specialized\footnote{\textbf{specialized} [a] (\textit{British English also} \textbf{specialised}) \textbf{1.} connected with a particular area of work or study; \textbf{2.} requiring or involving detailed \& particular knowledge or training; \textbf{3.} designed or developed for a particular purpose or area of knowledge.} topics\footnote{\textbf{topic} [n] a particular subject that is studied, written about or discussed.} are also planned: Vol. 6, \textit{The Theory of Languages} (Chap. 11); Vol. 7, \textit{Compilers}\footnote{\textbf{compiler} [n] \textbf{1.} a person who complies something; \textbf{2.} (\textit{computing}) a program that translates instructions from 1 computer language into another for a computer to understand.} (Chap. 12).

I started out in 1962 to write a single book with this sequence\footnote{\textbf{sequence} [n] \textbf{1.} [countable] \textbf{sequence (of something)} a set of connected events, actions, numbers, etc. that have a particular order; \textbf{2.} [countable, uncountable] the order that connected events, actions, etc. happen in or should happen it; \textbf{3.} [countable] (\textit{biology}) the order in which a set of genes or parts of molecules are arranged; \textbf{4.} [countable] a part of a film that deals with 1 subject or topic or consists of 1 scene; \textbf{5.} [countable] (\textit{mathematics}) an ordered list of numbers; [v] \textbf{1.} \textbf{sequence something} (\textit{biology}) to identify the order in which a set of genes or parts of molecules are arranged; \textbf{2.} \textbf{sequence something} to arrange things in a sequence.} of chapters, but I soon found that it was more important to treat\footnote{\textbf{treat} [v] \textbf{1.} to behave in a particular way when dealing with somebody\texttt{/}something; \textbf{2.} to consider something in a particular way; \textbf{3.} [often passive] to deal with or discuss a subject in a piece of writing or speech; \textbf{4.} [often passive] to give medical care or attention to a person, an illness or an injury; \textbf{5.} to use a chemical substance or process to clean, protect, preserve, etc. something.} the subjects in depth\footnote{\textbf{depth} [n] \textbf{1.} [countable, uncountable] the distance from the top or surface to the bottom of something; how deep something is; \textbf{2.} [uncountable] \textbf{depth (of something)} the fact of having or providing a lot of information or knowledge; \textbf{3.} [uncountable] \textbf{depth (of something)} the fact of being very important or serious; \textbf{4.} [uncountable] the quality in an image that makes it appear not to be flat; \textbf{the depths of something} [idiom] \textbf{1.} the deepest part of something; \textbf{2.} the most serious or extreme part of something; \textbf{in depth} [idiom] in a detailed \& thorough way.} rather than to skim\footnote{\textbf{skim} [v] \textbf{1.} [intransitive, transitive] to read something quickly in order to find a particular point or the main points; \textbf{2.} [transitive] \textbf{skim something} to remove a substance such as fat from the surface of a liquid; \textbf{3.} [intransitive, transitive, no passive] to move quickly \& lightly over a surface, not touching it or only touching it occasionally.} over them lightly\footnote{\textbf{lightly} [adv] \textbf{1.} gently; with very little force or effort; \textbf{2.} to a small degree; not much; \textbf{3.} without being seriously considered.}. The resulting length of the text has meant that each chapter by itself contains more than enough material for a 1-semester\footnote{\textbf{semester} [n] 1 of the 2 periods that the school or college year is divided into in some countries, e.g. the US.} college course; so it has become sensible\footnote{\textbf{sensible} [a] \textbf{1.} (of actions, plans, decisions, etc.) done or chosen with good judgment based on reason \& experience rather than emotion; practical; \textbf{2.} (of people) able to make good judgments based on reason \& experience rather than emotion.} to publish\footnote{\textbf{publish} [v] \textbf{1.} [transitive] \textbf{publish something} to produce a book, magazine, etc. \& sell it to the public; \textbf{2.} [transitive] to print a letter, an article, etc. in a newspaper or magazine; \textbf{3.} [transitive, intransitive] (of an author) to have your work printed in a newspaper, magazine, etc., or printed \& sold to the public; \textbf{4.} [transitive] to make information available to the public, \textsc{synonym}: \textbf{release}.} the series\footnote{\textbf{series} [n] (plural \textbf{series}) \textbf{1.} [countable, usually singular] \textbf{series of something} several events or things of a similar kind that come or happen one after the other; \textbf{2.} [countable] a set of radio or television programmes that deal with the same subject or that have the same characters; \textbf{3.} [uncountable, countable] (\textit{specialist}) an electrical circuit in which the current passes through all the parts in the correct order; \textbf{4.} [countable] (\textit{mathematics}) the sum of the terms in a sequence.} in separate volumes. I know that it is strange\footnote{\textbf{strange} [a] (\textbf{stranger, strangest}) \textbf{1.} unusual or surprising, especially in a way that is difficult to understand or explain; \textbf{2.} not familiar because you have not visited, seen or experienced it before.} to have only 1 or 2 chapters in an entire\footnote{\textbf{entire} [a] [only before noun] (used when you are emphasizing that the whole of something is involved) including everything, everyone or every part, \textsc{synonym}: \textbf{whole}.} book, but I have decided to retain\footnote{\textbf{retain} [v] \textbf{1.} \textbf{retain somebody\texttt{/}something} to keep somebody\texttt{/}something; to continue to have something \& not lose it or get rid of it; \textbf{2.} \textbf{retain something} to take in a substance \& keep holding it; \textbf{3.} \textbf{retain something} to remember or continue to hold something; \textbf{4.} \textbf{retain somebody\texttt{/}something} (\textit{law}) to employ a professional person such as a lawyer or doctor; to make regular payments to such a person in order to keep their services.} the original\footnote{\textbf{original} [a] \textbf{1.} [only before noun] present or existing from the beginning; 1st or earliest; \textbf{2.} new \& interesting in a way that is different from anything that has existed before; able to produce new \& interesting ideas; \textbf{3.} [usually before noun] painted, written, etc. by the artist rather than copied; [n] \textbf{1.} the earliest form of something, from which copies are later made; \textbf{2.} a book, text or play in the language in which it was 1st written; \textbf{in the original} [idiom] in the language in which a book, etc. was 1st written, before being translated.} chapter numbering in order to facilitate\footnote{\textbf{facilitate} [v] \textbf{facilitate something} to make an action or a process possible or easier.} cross references\footnote{\textbf{cross references} [n] \textbf{cross reference (to something)} a note that tells a reader to look in another part of a book or file for further information; [v] \textbf{cross-reference something} to give cross references to another text or part of a text.}. A shorter version of Vols. 1--5 is planned, intended specifically\footnote{\textbf{specifically} [adv] \textbf{1.} in a way that is connected with or intended for 1 particular person, thing or group only; \textbf{2.} in a detailed, exact \& clear way; \textbf{3.} used when you want to add more detailed \& exact information.} to serve\footnote{\textbf{serve} [v] \textbf{1.} [intransitive, transitive] to have a particular effect, use or result; \textbf{2.} [transitive] to be useful to somebody in achieving something; \textbf{3.} [transitive] to provide an area or a group of people with a product or service; \textbf{4.} [intransitive, transitive] to work or perform duties for a person, an organization, a country, etc.; \textbf{5.} [transitive] \textbf{serve something} to spend a period of time in prison; \textbf{6.} [transitive] to give somebody food or drink, e.g. at a restaurant or during a meal; \textbf{7.} [transitive] (\textit{law}) to give or send somebody an official document, especially one that orders them to appear in court; \textbf{serve something up} [phrasal verb] to give, offer or provide something.} as a more general reference \&\texttt{/}or text for undergraduate computer courses; its contents\footnote{\textbf{content} [n] \textbf{1.} (\textbf{contents}) [plural] \textbf{content (of something)} the things that are contained in something; \textbf{2.} (\textbf{content}) [plural] the different sections that are contained in a book, magazine, journal or website; a list of these sections; \textbf{3.} [singular] the subject matter of a book, speech, programme, etc.; \textbf{4.} [singular] (following a noun or an adjective) the amount of a substance that is contained in something else; \textbf{5.} [uncountable] the information or other material contained on a website, CD-ROM, etc.; [a] [not before noun] satisfied \& happy with what you have; willing to do or accept something; [v] \textbf{content yourself with something} to accept \& be satisfied with something \& not try to have or do something better.} will be a subset\footnote{\textbf{subset} [n] (\textit{specialist}) a smaller group of things or people formed from the members of a larger group.} of the material in these books, with the more specialized information omitted\footnote{\textbf{omit} [v] \textbf{1.} to not include something\texttt{/}somebody, either deliberately or because you have forgotten it\texttt{/}them; \textbf{2.} \textbf{omit to do something} to not do or fail to do something.}. The same chapter numbering will be used in the abridged\footnote{\textbf{abridged} [a] (of a book, play, etc.) made shorter by leaving parts out, \textsc{opposite}: \textbf{unabridged}.} edition\footnote{\textbf{edition} [n] \textbf{1.} (abbr., \textbf{ed.}) \textbf{edition (of something)} the total number of copies of a book, newspaper or magazine published at 1 time; \textbf{2.} the form in which a book is published.} as in the complete work.

The present volume may be considered as the ``intersection'' of the entire set, in the sense that it contains basic material that is used in all the other books. Vols. 2--5, on the other hand, may be read independently\footnote{\textbf{independently} [adv] \textbf{1.} without being controlled or influenced by somebody\texttt{/}something else; \textbf{2.} without help from other people; \textbf{3.} in a way that is not connected with somebody\texttt{/}something else, \textsc{synonym}: \textbf{separately}.} of each other. Vol. 1 is not only a reference book to be used in connection with the remaining\footnote{\textbf{remaining} [a] [only before noun] \textbf{1.} not yet used, dealt with or resolved; \textbf{2.} still existing, present or in use.} volumes; it may also be used in college courses or for self-study as a text on the subject of \textit{data structures} (emphasizing the material of Chap. 2), or as a text on the subject of \textit{discrete mathematics} (emphasizing the material of Sects. 1.1, 1.2, 1.3.3, \& 2.3.4), or as a text on the subject of \textit{machine-language programming} (emphasizing the material of Sects. 1.3 \& 1.4).

The point of view I have adopted\footnote{\textbf{adopt} [v] \textbf{1.} [transitive] \textbf{adopt something} to start to use a particular method or to show a particular attitude towards somebody\texttt{/}something; \textbf{2.} [transitive] \textbf{adopt something} to formally accept a suggestion or policy by voting; \textbf{3.} [transitive, intransitive] \textbf{adopt (somebody)} to take somebody else's child into your family \& become its legal parent(s); \textbf{4.} \textbf{adopt something} to choose a new name or custom \& begin to use it as your own; to choose \& move to a country as your permanent home; \textbf{5.} [transitive] \textbf{adopt something} to use a particular manner or way of speaking.}\,\footnote{\textbf{adopted} [a] \textbf{1.} an adopted child has legally become part of a family that is not the one in which they were born; \textbf{2.} an adopted country is one in which somebody chooses to live although it is not the one they were born in.} while writing these chapters differs from that taken in most contemporary\footnote{\textbf{contemporary} [a] \textbf{1.} belonging to the present time, \textsc{synonym}: \textbf{modern}; \textbf{2.} (especially of people \& society) belonging to the same time as somebody\texttt{/}something else; [n] (plural \textbf{contemporaries}) a person or thing living or existing at the same time as somebody\texttt{/}something else, especially somebody who is about the same age as somebody else.} books about computer programming in that I am not trying to teach the reader how to use somebody else's software. I am concerned rather with teaching people how to write better software themselves.

My original goal was to bring readers to the frontiers\footnote{\textbf{frontier} [n] \textbf{1.} [countable] a line that separates 2 countries, etc.; the land near this line, \textsc{synonym}: \textbf{border}; \textbf{2.} (\textbf{the frontier}) [singular] the edge of land where people live \& have built towns, beyond which the country is wild \& unknown, especially in the western US in the 19th century; \textbf{3.} [countable, usually plural] \textbf{(at the) frontier of something} the limit of something, especially the limit of what is known about a particular subject or activity.} of knowledge in every subject that was treated. But it is extremely\footnote{\textbf{extremely} [adv] (usually with adjectives \& adverbs) to a very high degree.} difficult to keep up with a field that is economically profitable\footnote{\textbf{profitable} [a] \textbf{1.} that makes or is likely to make money; \textbf{2.} that gives somebody an advantage or a useful result.}, \& the rapid\footnote{\textbf{rapid} [a] [usually before noun] happening in a short period of time or at a fast rate.} rise of computer science has made such a dream impossible. The subject has become a vast tapestry\footnote{\textbf{tapestry} [n] [countable, uncountable] (plural \textbf{tapestries}) a picture or pattern that is made by weaving colored wool onto heavy cloth; the art of doing this.} with tens of thousands of subtle\footnote{\textbf{subtle} [a] (\textbf{subtler, subtlest}) (\textbf{more subtle} is also common) \textbf{1.} (\textit{often approving}) (especially of a change or difference) not very obvious; not easy to notice; \textbf{2.} (of a person or their behavior) behaving in a clever way \& using indirect methods in order to achieve something; \textbf{3.} showing a good understanding of things that are not obvious to other people.} results contributed by tens of thousands of talented\footnote{\textbf{talented} [a] having a natural ability to do something well.} people all over the world. Therefore my new goal has been to concentrate\footnote{\textbf{concentrate} [v] \textbf{1.} [transitive, often passive] \textbf{concentrate something $+$ adv.\texttt{/}prep.} to bring something together in 1 place; \textbf{2.} [intransitive, transitive] to give all your attention to something \& not think about anything else; \textbf{3.} [transitive] \textbf{concentrate something} to increase the strength of a substance by reducing its volume, e.g. by boiling it; \textbf{concentrate on something} [phrasal verb] to spend more time doing 1 particular thing than others; [n] [countable, uncountable] \textbf{concentrate (of something)} a substance that is made stronger because water or other substances have been removed.} on ``classic'' techniques that are likely to remain\footnote{\textbf{remain} [v] (not usually used in the progressive tenses) \textbf{1.} \textit{linking verb} to continue to be something; to be still in the same state or condition; \textbf{2.} [intransitive] \textbf{remain (of something)} to still be present after the other parts have been removed or used; to continue to exist; \textbf{3.} [intransitive] to still need to be done, said or dealt with; \textbf{4.} [intransitive] \textbf{$+$ adv.\texttt{/}prep.} to stay in the same place; to not leave; \textbf{it remains to be seen (whether\texttt{/}what, etc.), something remains to be seen} [idiom] used to say that you cannot yet know something.} important for many more decades, \& to describe\footnote{\textbf{describe} [v] \textbf{1.} [often passive] to give an account of something in words. \textbf{Described} is often used after a noun phrase, without \textit{that is}\texttt{/}\textit{was}, etc.; \textbf{2.} [often passive] to say what somebody\texttt{/}something is like; to say what somebody\texttt{/}something is; \textbf{3.} \textbf{describe something} to make a movement which has a particular shape; to form a particular shape; \textbf{4.} \textbf{describe something} (\textit{specialist}) (of a diagram or calculation) to represent something.} them as well as I can. In particular, I have tried to trace\footnote{\textbf{trace} [v] \textbf{1.} [often passive] \textbf{trace something (back) (to something)} to discover or describe when or how something began; \textbf{2.} \textbf{trace something (from something) (to something)} to describe a process or the development of something; \textbf{3.} [often passive] to find or discover somebody\texttt{/}something by looking carefully for them\texttt{/}it; \textbf{4.} \textbf{trace something (out)} to draw a line or lines on a surface; \textbf{5.} \textbf{trace something} to take a particular path or route; \textbf{6.} \textbf{trace something} to follow the shape or outline of something; [n] \textbf{1.} [countable, uncountable] a mark, object or sign that shows that somebody\texttt{/}something existed or was present. A \textbf{trace fossil} is the evidence of animal activity preserved in a rock.; \textbf{2.} [countable] a very small amount of something; \textbf{3.} [countable] a line or pattern displayed by a machine to show information about something that is being recorded or measured; \textbf{4.} [countable] \textbf{trace (of something)} a line following the path of something.} the history\footnote{\textbf{history} [n] (plural \textbf{histories}) \textbf{1.} [uncountable] all the events that happened in the past; \textbf{2.} [uncountable, singular] \textbf{history (of something)} the past events concerned in the development of a particular place, subject, etc.; \textbf{3.} [singular] \textbf{history (of something)} a record of something happening frequently in the past life of a person, family or place; the set of faces that are known about somebody's past life; \textbf{4.} [countable] \textbf{history (of something)} a written or spoken account of past events; \textbf{5.} [uncountable] the study of past events as a subject at school or university; \textbf{make history} [idiom] to do something so important that it will be recorded in history.} of each subject, \& to provide a solid foundation\footnote{\textbf{foundation} [n] \textbf{1.} [countable] a principle, an idea or a face that something is based on \& that it grows from; \textbf{2.} [uncountable] \textbf{foundation (of something)} the act of starting a new organization or institution, \textsc{synonym}: \textbf{establishment}; \textbf{3.} [uncountable] \textbf{foundation (of something)} the act of being the 1st to start living in a town or country; \textbf{4.} [countable] an organization that is established to provide money for a particular purpose, e.g. for scientific research or charity; \textbf{5.} [countable, usually plural] \textbf{foundation (of something)} a layer of bricks, etc. that forms the solid underground based of a building.} for future progress. I have attempted to choose terminology\footnote{\textbf{terminology} [n] (plural \textbf{terminologies}) \textbf{1.} [uncountable, countable] the set of technical words or expressions used in a particular subject; \textbf{2.} [uncountable] words used with particular meanings.} that is concise\footnote{\textbf{concise} [a] giving only the information that is necessary \& important, using few words.} \& consistent\footnote{\textbf{consistent} [a] \textbf{1.} \textbf{consistent with something} in agreement with something; not contradicting something, \textsc{opposite}: \textbf{inconsistent}; \textbf{2.} happening in the same way \& continuing for a period of time, \textsc{opposite}: \textbf{inconsistent}; \textbf{3.} always behaving in the same way, or having the same opinions or standards, \textsc{opposite}: \textbf{inconsistent}; \textbf{4.} (of an argument or a set of ideas) having different parts that all agree with each other, \textsc{opposite}: \textbf{inconsistent}.} with current\footnote{\textbf{current} [a] \textbf{1.} [only before noun] existing, happening or being used now; of the present time; \textbf{2.} being used by or accepted by many people at the present time; [n] \textbf{1.} an area of water or air moving in a definite direction, especially through a surrounding area of water or air in which there is less movement; \textbf{2.} the rate of flow of electric charge through a wire, etc., measured in units of amperes; \textbf{3.} the fact of particular ideas, opinions or feelings being present in a group of people.} usage\footnote{\textbf{usage} [n] \textbf{1.} [uncountable] the fact of something being used; how much something is used; \textbf{2.} [uncountable, countable] the way in which words are used in a language; \textbf{3.} [countable] a custom, practice or habit that people have.}. I have tried to include all of the known ideas about sequential\footnote{\textbf{sequential} [a] forming or following in a logical order.} computer programming that are both beautiful \& easy to state.

A few words are in order about the mathematical content of this set of books. The material has been organized so that persons with no more than a knowledge of high-school algebra may read it, skimming briefly\footnote{\textbf{briefly} [adv] \textbf{1.} in few words; \textbf{2.} for a short time.} over the more mathematical portions\footnote{\textbf{portion} [n] \textbf{1.} \textbf{portion (of something)} 1 part of something larger; \textbf{2.} \textbf{portion (of something)} an amount of food that is large enough for 1 person; \textbf{3.} [usually singular] \textbf{portion (of something)} a part of something that is shared with other people, \textsc{synonym}: \textbf{share}.}; yet a reader who is mathematically inclined\footnote{\textbf{incline} [v] \textbf{1.} [intransitive, transitive] to tend to think or behave in a particular way; to make somebody do this; \textbf{2.} [intransitive, transitive] to lean or slope in a particular direction; to make something lean or slope.}\,\footnote{\textbf{inclined} [a] \textbf{1.} \textbf{inclined to do something} tending to do something; likely to do something; \textbf{2.} [not before noun] \textbf{inclined (to do something)} wanting to do something; \textbf{3.} \textbf{inclined to agree, believe, think, etc.} used when you are expressing an opinion but do not want to express it very strongly; \textbf{4.} (used with particular adverbs) having a natural ability for something; preferring to do something; \textbf{5.} sloping; at an angle.} will learn about many interesting mathematical techniques related to discrete\footnote{\textbf{discrete} [a] independent of other things of the same type, \textsc{synonym}: \textbf{separate}.} mathematics. This dual\footnote{\textbf{dual} [a] [only before noun] having 2 parts or aspects.} level of presentation has been achieved in part by assigning ratings to each of the exercises so that the primarily mathematical ones are marked specifically as such, \& also by arranging mos sections so that the main mathematical results are stated \textit{before} their proofs. The proofs are either left as exercises (with answers to be found in a separate section) or they are given at the end of a section.

viii

'' -- \cite[Preface, pp. v--]{Knuth1997}

%------------------------------------------------------------------------------%

\chapter{\cite{Matthes2019}. Python Crash Course: A Hands-on, Project-based Introduction to Programming}

\section*{Preface to the 2nd Edition}
``\textit{Python Crash Course} is being used in middle schools \& high schools, \& also in college classes. Students who are assigned more advanced textbooks are using \textit{Python Crash Course} as a companion text for their classes \& finding it a worthwhile supplement. People are using it to enhance their skills on the job \& to start working on their own side projects. In short, people are using the book for the full range of purposes I had hoped they would.

The opportunity to write a 2nd edition of \textit{Python Crash Course} has been thoroughly enjoyable. Although Python is a mature language, it continues to evolve as every language does. My goal in revising the book was to make it \fbox{leaner \& simpler}. There is no longer any reason to learn Python 2, so this edition focuses on Python 3 only. Many Python packages have become easier to install, so setup \& installation instructions are easier. I've added a few topics that I've realized readers would benefit from, \& I've updated some sections to reflect new, simpler ways of doing things in Python. I've also clarified some sections where certain details of the languages were not presented as accurately as they could have been. All the projects have been completely updated using popular, well-maintained libraries that you can confidently use to build your own projects.

The following is a summary of specific changes that have been made in the 2nd edition:
\begin{itemize}
	\item In Chap. 1, the instructions for installing Python have been simplified for users of all major operating systems. I now recommend the text editor Sublime Text, which is popular among beginner \& professional programmers \& works well on all operating systems.
	\item In Chap. 2 includes a more accurate description of how variables are implemented in Python. Variables are described as \textit{labels} for values, which leads to a better understanding of how variables behave in Python. The book now users f-strings, introduced in Python 3.6. This is a much simpler way to use variable values in strings. The use of underscores to represent large numbers, such as \verb|1_000_000|, was also introduced in Python 3.6 \& is included in this edition. Multiple assignment of variables was previously introduced in 1 of the projects, \& that description has been generalized \& moved to Chap. 2 for the benefit of all readers. Finally, a clear convention for representing constant values in Python is included in this chapter.
	\item In Chap. 6, I introduce the \texttt{get()} method for retrieving values from a dictionary, which can return a default value if a key does not exist.
	\item The Alien Invasion project (Chaps. 12--14) is now entirely class-based. The game itself is a class, rather than a series of functions. This greatly simplifies the overall structure of the game, vastly reducing the number of function calls \& parameters required. Readers familiar with the 1st edition will appreciate the simplicity this new class-based approach provides. Pygame can now be installed in 1 line on all systems, \& readers are given the option of running the game in fullscreen mode or in a windowed mode.
	\item In the data visualization projects, the installation instructions for Matplotlib are simpler for all operating systems. The visualizations featuring Matplotlib use the \texttt{subplots()} function, which will be easier to build upon as you learn to create more complex visualizations. The Rolling Dice project in Chap. 15 uses Plotly, a well-maintained visualization library that features a clean syntax \& beautiful, fully customizable output.
	\item In Chap. 16, the weather project is based on data from NOAA, which should be more stable over the next few years than the site used in the 1st edition. The mapping project focuses on global earthquake activity; by the end of this project you'll have a stunning visualization showing Earth's tectonic plate boundaries through a focus on the locations of all earthquakes over a given time period. You'll learn to plot any data set involving geographic points.
	\item Chap. 17 uses Plotly to visualize Python-related activity in open source projects on GitHub.
	\item The Learning Log project (Chaps. 18--20) is built using the latest version of Django \& styled using the latest version of Bootstrap. The process of deploying the project to Heroku has been simplified using the \texttt{django-heroku} package, \& uses environment variables rather than modifying the \texttt{settings.py} files. This is a simpler approach \& is more consistent with how professional programmers deploy modern Django projects.
	\item Appendix A has been fully updated to recommend current best practices in installing Python. Appendix B includes detailed instructions for setting up Sublime Text \& brief descriptions of most of the major text editors \& IDEs in current use. Appendix C directs readers to newer, more popular online resources for getting help, \& Appendix D continues to offer a mini crash course in using Git for version control.
	\item The index has been thoroughly updated to allow you to use \textit{Python Crash Course} as a reference for all of your future Python projects.'' -- \cite[Preface to the 2nd Edition, pp. xxvii--xxix]{Matthes2019}
\end{itemize}

%------------------------------------------------------------------------------%

\section*{Introduction}

\subsection*{Who is This Book For?}
``The goal of this book is to bring you up to speed with Python as quickly as possible so you can build programs that work -- games, data visualizations, \& web applications -- while developing a foundation in programming that will serve you well for the rest of your life. \textit{Python Crash Course} is writing for people of any age who have never before programmed in Python or have never programmed at all. This book is for those who want to learn the basics of programming quickly so they can focus on interesting projects, \& those who like to test their understanding of new concepts by solving meaningful problems. \textit{Python Crash Course} is also perfect for middle school \& high school teachers who want to offer their students a project-based introduction to programming. If you're taking a college class \& want a friendlier introduction to Python than the text you've been assigned, this book could make your class easier as well.'' -- \cite[Introduction, p. xxxiv]{Matthes2019}

\subsection*{What Can You Expect to Learn?}
``The purpose of this book is to make you a good programmer in general \& a good Python programmer in particular. You'll learn efficiently \& adopt good habits as I provide you with a solid foundation in general programming concepts. After working your way through \textit{Python Crash Course}, you should be ready to move on to more advanced Python techniques, \& your next programming language will be even easier to grasp.

In the 1st part of this book, you'll learn basic programming concepts you need to know to write Python programs. These concepts are the same as those you'd learn when starting out in almost any programming language. You'll learn about different kinds of data \& the ways you can store data in lists \& dictionaries within your programs. You'll learn to build collections of data \& work through those collections in efficient ways. You'll learn to use \texttt{while} loops \& \texttt{if} statements to test for certain conditions so you can run specific sections of code while those conditions are true \& run other sections when they're not -- a technique that greatly helps you automate processes.

You'll learn to accept input from users to make your programs interactive \& to keep your programs running as long as the user is active. You'll explore how to write functions to make parts of your program reusable, so you only have to write blocks of code that perform certain actions once \& then use that code as many times as you like. You'll then extend this concept to more complicated behavior with classes, making fairly simple programs respond to a variety of situations. You'll learn to write programs that handle common errors gracefully. After working through each of these basic concepts, you'll write a few short programs that solve some well-defined problems. Finally, you'll take your 1st step toward intermediate programming by learning how to write tests for your code so you can develop your programs further without worrying about introducing bugs. All the information in Part I will prepare you for taking on larger, more complex projects.

In Part II, you'll apply what you learned in Part I to 3 projects. You can do any or all of these projects in whichever order works best for you. In the 1st project (Chaps. 12--14), you'll create a \textit{Space Invaders}--style shooting game called \textit{Alien Invasion}, which consists of levels of increasing difficulty. After you've completed this project, you should be well on your way to being able to develop your own 2D games.

The 2nd project (Chaps. 15--17) introduces you to data visualization. Data scientists aim to make sense of the vast amount of information available to them through a variety of visualization techniques. You'll work with data sets that you generate through code, data sets that you download from online sources, \& data sets your programs download automatically. After you've completed this project, you'll be able to write programs that sift through large data sets \& make visual representations of that stored information.

In the 3rd project (Chaps. 18--20), you'll build a small web application called Learning Log. This project allows you to keep a journal of ideas \& concepts you've learned about a specific topic. You'll be able to keep separate logs for different topics \& allow others to create an account \& start their own journals. You'll also learn how to deploy your project so anyone can access it online from anywhere.'' -- \cite[Introduction, pp. xxxiv--xxxv]{Matthes2019}

\subsection*{Online Resources}
``You can find all the supplementary resources for the book online at \url{https://nostarch.com/pythoncrashcourse2e/} or \url{http://ehmatthes.github.io/pcc_2e/}. These resources include:
\begin{itemize}
	\item \textbf{Setup instructions.} These instructions are identical to what's in the book but include active links you can click for all the difference pieces. If you're having any setup issues, refer to this resource.
	\item \textbf{Updates.} Python, like all languages, is constantly evolving. I maintain a thorough set of updates, so if anything isn't working, check here to see whether instructions have changed.
	\item \textbf{Solutions to exercises.} You should spend significant time on your own attempting the exercises in the ``Try It Yourself'' sections. But if you're stuck \& can't make any progress, solutions to most of the exercises are online.
	\item \textbf{Cheat sheets.} A full set of downloadable cheat sheets for a quick reference to major concepts is also online.'' -- \cite[Introduction, p. xxxv]{Matthes2019}
\end{itemize}

\subsection*{Why Python?}
``Every year I consider whether to continue using Python or whether to move on to a different language -- perhaps one that's newer to the programming world. But I continue to focus on Python for many reasons. \fbox{Python is an incredibly efficient language}: your programs will do more in fewer lines of code than many other languages would require. Python's syntax will also help you write ``clean'' code. Your code will be easy to read, easy to debug, \& easy to extend \& build upon compared to other languages.

People use Python for many purposes: to make games, build web applications, solve business problems, \& develop internal tools at all kinds of interesting companies. Python is also used heavily in scientific fields for academic research \& applied work.

1 of the most important reasons I continue to use Python is because of the Python community, which includes an incredibly diverse \& welcoming group of people. \fbox{Community is essential to programmers because programming isn't a solitary pursuit.} Most of us, even the most experienced programmers, need to ask advice from others who have already solved similar problems. Having a well-connected \& supportive community is fully supportive of people like you who are learning Python as your 1st programming language. Python is a great language to learn $\ldots$'' -- \cite[Introduction, p. xxxvi]{Matthes2019}

%------------------------------------------------------------------------------%

\begin{center}
	\huge Part I: Basics
\end{center}
``Part I of this book teaches you the basic concepts you'll need to write Python programs. Many of these concepts are common to all programming languages, so they'll be useful throughout your life as a programmer.
\begin{itemize}
	\item In Chap. 1 you'll install Python on your computer \& run your 1st program, which prints the message \textit{Hello world!} to the screen.
	\item In Chap. 2 you'll learn to store information in variables \& work with text \& numerical values.
	\item Chaps. 3--4 introduce lists. Lists can store as much information as you want in 1 variable, allowing you to work with that data efficiently. You'll be able to work with hundreds, thousands, \& even millions of values in just a few lines of code.
	\item In Chap. 5 you'll use \texttt{if} statements to write code that responds 1 way if certain conditions are true, \& responds in a different way if those conditions are not true.
	\item Chap. 6 shows you how to use Python's dictionaries, which let you make connections between different pieces of information. Like lists, dictionaries can contain as much information as you need to store.
	\item In Chap. 7 you'll learn how to accept input from users to make your programs interactive. You'll also learn about \texttt{while} loops, which run blocks of code repeatedly as long as certain conditions remain true.
	\item In Chap. 8 you'll write functions, which are named blocks of code that perform a specific task \& can be run whether you need them.
	\item Chap. 9 introduces \textit{classes}, which allow you to model real-world objects, such as dogs, cats, people, cars, rockets, \& much more, so your code can represent anything real or abstract.
	\item Chap. 10 shows you how to work with files \& handle errors so your programs won't crash unexpectedly. You'll store data before your program closes, \& read the data back in when the program runs again. You'll learn about Python's exceptions, which allow you to anticipate errors, \& make your programs handle those errors gracefully.
	\item In Chap. 11 you'll learn to write tests fr your code to check that your programs work the way you intend them to. As a result, you'll be able to expand your programs without worrying about introducing new bugs. Testing your code is 1 of the 1st skills that will help you transition from beginner to intermediate programmer.'' -- \cite[pp. 1--2]{Matthes2019}
\end{itemize}

\section{Getting Started}

\subsection{Setting Up Your Programming Environment}
``Python differs slightly on different operating systems, so you'll need to keep a few considerations in mind. In the following sections, we'll make sure Python is set up correctly on your system.'' -- \cite[p. 3]{Matthes2019}

\subsubsection{Python Versions}
``Every programming language evolves as new ideas \& technologies emerge, \& the developers of Python have continually made the language more versatile \& powerful.'' ``In this section, we'll find out if Python is already installed on your system \& whether you need to install a newer version. Appendix A contains a comprehensive guide to installing the latest version of Python on each major operating system as well. Some old Python projects still use Python 2, but you should use Python 3. If Python 2 is installed on your system, it's probably there to support some older programs that your system needs.'' -- \cite[p. 4]{Matthes2019}

\subsubsection{Running Snippets of Python Code}
``You can run Python's interpreter in a terminal window, allowing you to try bits of Python code without having to save \& run an entire program.'' ``The \texttt{>>>} prompt indicates that you should be using the terminal window, \& the bold text is the code you should type in \& then execute by pressing \texttt{Enter}. Most of the examples in the book are small, self-contained programs that you'll run from your text editor rather than the terminal, because you'll write most of your code in the text editor. But sometimes basic concepts will be shown in a series of snippets run through a Python terminal session to demonstrate particular concepts more efficiently. When you see 3 angle brackets in a code listing, you're looking at code \& output from a terminal session.'' -- \cite[p. 4]{Matthes2019}

\subsection{About the Sublime Text Editor}
``Sublime Text is a simple text editor that can be installed on all modern operating systems. Sublime Text lets you run almost all of your programs directly from the editor instead of through a terminal. Your code runs in a terminal session embedded in the Sublime Text window, which makes it easy to see the output.

Sublime Text is a beginner-friendly editor, but many professional programmers use it as well. If you become comfortable using it while learning Python, you can continue using it as you progress to larger \& more complicated projects. Sublime Text has a very liberal licensing policy: you can use the editor free of charge as long as you want, but the developers request that you purchase a license if you like it \& want to keep using it.

Appendix B provides information on other text editors. If you're curious about the other options, you might want to skim that appendix at this point. If you want to begin programming quickly, you can use Sublime Text to start \& consider other editors once you've gained some experience as a programmer.'' -- \cite[pp. 4--5]{Matthes2019}

\subsection{Python on Different Operating Systems}
``Python is a cross-platform programming language, which means it runs on all the major operating systems. Any Python program you write should run on any modern computer that has Python installed. However, the methods for setting up Python on different operating systems vary slightly.'' -- \cite[p. 5]{Matthes2019}

\subsubsection{Python on Windows}
``Window doesn't always come with Python, so you'll probably need to install it, \& then install Sublime Text.'' -- \cite[p. 5]{Matthes2019}

\paragraph{Installing Python.} ``1st, check whether Python is installed on your system. Open a command window by entering \texttt{command} into the Start menu or by holding down the \texttt{Shift} key while right-clicking on your desktop \& selecting \texttt{Open command window here} from the menu. In the terminal window, enter \texttt{python} in lowercase. If you get a Python prompt (\texttt{>>>}) in response, Python is installed on your system. If you see an error message telling you that \texttt{python} is not a recognized command, Python isn't installed.

In that case, or if you see a version of Python earlier than Python 3.6, you need to download a Python installer for Windows. Go to \url{https://python.org/} \& hover over the \textbf{Downloads} link. You should see a button for downloading the latest version of Python. Click the button, which should automatically start downloading the correct installer for your system. After you've downloaded the file, run the installer. Make sure you select the option \texttt{Add Python to PATH}, which will make it easier to configure your system correctly.'' -- \cite[pp. 5--6]{Matthes2019}

\paragraph{Running Python in a Terminal Session.} ``Open a command window \& enter \texttt{python} in lowercase. You should see a Python prompt (\texttt{>>>}), which means Windows has found the version of Python you just installed.''

\begin{verbatim}
	C:\> python
	Python 3.7.2 (v3.7.2:9a3ffc0492, Dec 23 2018, 23:09:28) [MSC v.1916 64 bit
	(AMD64)] on win32
	Type "help", "copyright", "credits" or "license" for more information.
	>>>
\end{verbatim}
``Any time you want to run a snippet of Python code, open a command window \& start a Python terminal session. To close the terminal session, press \texttt{Ctrl}-\texttt{Z} \& then press \texttt{Enter}, or enter the command \texttt{exit()}.'' -- \cite[p. 6]{Matthes2019}

\paragraph{Installing Sublime Text.} ``You can download an installer for Sublime text at \url{https://sublimetext.com/}. Click the download link \& look for a Windows installer. After downloading the installer, run the installer \& accept all of its defaults.'' -- \cite[p. 7]{Matthes2019}

\subsubsection{Python on macOS}
``Python is already installed on most macOS systems, but it's most likely an outdated version that you won't want to learn on.'' -- \cite[p. 7]{Matthes2019}

\paragraph{Checking Whether Python 3 Is Installed.} ``Open a terminal window by going to \texttt{Applications $\triangleright$ Utilities $\triangleright$ Terminal}. You can also press \texttt{cmd}-spacebar, type \texttt{terminal}, \& then press \texttt{Enter}. To see which version of Python is installed, enter \texttt{python} with a lowercase \texttt{p} -- this also starts the Python interpreter within the terminal, allowing you to enter Python commands. You should see output telling you which Python version is installed on your system \& a \texttt{>>>} prompt where you can start entering Python commands, like this:
\begin{verbatim}
	$ python
	Python 2.7.15 (default, Aug 17 2018, 22:39:05)
	[GCC 4.2.1 Compatible Apple LLVM 9.1.0 (clang-902.0.39.2)] on darwin
	Type "help", "copyright", "credits", or "license" for more information.
	>>>
\end{verbatim}
This output indicates that Python 2.7.15 is currently the default version installed on this computer. Once you've seen this output, press \texttt{Ctrl}-\texttt{D} or enter \texttt{exit()} to leave the Python prompt \& return to a terminal prompt.

To check whether you have Python 3 installed, enter the command \texttt{python3}. You'll probably get an error message, meaning you don't have any versions of Python 3 installed. If the output shows you have Python 3.6 or a later version installed, you can skip ahead to ``Running Python in a Terminal Session'' on p. 8. If Python 3 isn't installed by default, you'll need to install it manually. Note that whenever you see the \texttt{python} command in this book, you need to use the \texttt{python3} command instead to make sure you're using Python 3, not Python 2; they different significantly enough that you'll run into trouble trying to run the code in this book using Python 2.

If you see any version earlier than Python 3.6, follow the instructions in the next section to install the latest version.'' -- \cite[p. 7]{Matthes2019}

\paragraph{Installing the Latest Version of Python.} ``You can find a Python installer for your system at \url{https://python.org/}. Hover over the \textbf{Download} link, \& you should see a button for downloading the latest Python version. Click the button, which should automatically start downloading the correct installer for your system. After the file downloads, run the installer. When you're finished, enter the following at a terminal prompt:
\begin{verbatim}
	$ python3 --version
	Python 3.7.2
\end{verbatim}
You should see output similar to this, in which case, you're ready to try out Python. Whenever you see the command \texttt{python}, make sure you see \texttt{python3}.'' -- \cite[pp. 7--8]{Matthes2019}

\paragraph{Running Python in a Terminal Session.} ``You can now try running snippets of Python code by opening a terminal \& typing \texttt{python3}.'' ``Remember that you can close the Python interpreter by pressing \texttt{Ctrl}-\texttt{D} or by entering the command \texttt{exit()}.'' -- \cite[p. 8]{Matthes2019}

\paragraph{Installing Sublime Text.} ``To install the Sublime Text editor, you need to download the installer at \url{https://sublimetext.com/}. Click the \textbf{Download} link \& look for an installer for macOS. After the installer downloads, open it \& then drag the Sublime Text icon into your \texttt{Applications} folder.'' -- \cite[p. 8]{Matthes2019}

\subsubsection{Python on Linux}
``\fbox{Linux systems are designed for programming, so Python is already installed on most Linux computers}. The people who write \& maintain Linux expect you to do your own programming at some point \& encourage you to do so. For this reason, there's very little to install \& only a few settings to change to start programming.'' -- \cite[p. 8]{Matthes2019}

\paragraph{Checking Your Version of Python.} ``Open a terminal window by running the Terminal application on your system (in Ubuntu, you can press \texttt{Ctrl}-\texttt{Alt}-\texttt{T}). To find out which version of Python is installed, this command starts the Python interpreter. You should see output indicating which version of Python is installed \& a \texttt{>>>} prompt where you can start entering Python commands, like this:
\begin{verbatim}
	$ python3
	Python 3.7.2 (default, Dec 27 2018, 04:01:51)
	[GCC 7.3.0] on linux
	Type "help", "copyright", "credits" or "license" for more information.
	>>>
\end{verbatim}
This output indicates that Python 3.7.2 is currently the default version of Python installed on this computer. When you've seen this output, press \texttt{Ctrl}-\texttt{D} or enter \texttt{exit()} to leave the Python prompt \& return to a terminal prompt. Whenever you see the \texttt{python} command in this book, enter \texttt{python3} instead.'' -- \cite[pp. 8--9]{Matthes2019}

\paragraph{Running Python in a Terminal Session.} ``You can try running snippets of Python code by opening a terminal \& entering \texttt{python3}, as you did when checking your version. Do this again, \& when you have Python running, enter the following line in the terminal session:
\begin{verbatim}
	>>> print("Hello Python interpreter!")
	Hello Python interpreter!
	>>>
\end{verbatim}
The message should print directly in the current terminal window. Remember that you can close the Python interpreter by pressing \texttt{Ctrl}-\texttt{D} or by entering the command \texttt{exit()}.'' -- \cite[p. 9]{Matthes2019}

\paragraph{Installing Sublime Text.} ``On Linux, you can install Sublime Text from the Ubuntu Software Center. Click the Ubuntu Software icon in your menu, \& search for \textbf{Sublime Text}. Click to install it, \& then launch it.'' -- \cite[p. 9]{Matthes2019}

\subsection{Running a Hello World Program}
``With a recent version of Python \& Sublime Text installed, you're almost ready to run your 1st Python program written in a text editor. But before doing so, you need to make sure Sublime Text is set up to use the correct version of Python on your system.'' -- \cite[p. 9]{Matthes2019}

\subsubsection{Configuring Sublime Text to Use the Correct Python Version}
``If the \texttt{python} command on your system runs Python 3, you won't need to configure anything \& can skip to the next section. If you use the \texttt{python3} command, you'll need to configure Sublime Text to use the correct Python version when it runs your programs.

Click the Sublime Text icon to launch it, or search for Sublime Text in your system's search bar \& then launch it. Go to \texttt{Tools $\triangleright$ Build System $\triangleright$ New Build System}, which will open a new configuration file for you. Delete what you see \& enter the following: \texttt{Python3.sublime-build}
\begin{verbatim}
	{
	    "cmd": ["python3", "-u", "$file"],
	}
\end{verbatim}
This code tells Sublime Text to use your system's \texttt{python3} command when running your Python program files. Save the file as \texttt{Python3.sublime-build} in the default directory that Sublime Text opens when you choose Save.'' -- \cite[pp. 9--10]{Matthes2019}

\subsubsection{Running \texttt{hello\_world.py}}
``Before you write your 1st program, make a folder called \verb|python_work| somewhere on your system for your projects. It's best to use lowercase letters \& underscores for spaces in file \& folder names, because Python uses these naming conventions.

Open Sublime Text, \& save an empty Python (\texttt{File $\triangleright$ Save As}) called \verb|hello_world.py| in your \verb|python_work| folder. The extension \texttt{.py} tells Sublime Text that the code in your file is written in Python, which tells it how to run the program \& highlight the text in a helpful way. After you've saved your file, enter the following line in the text editor: \verb|hello_world.py|
\begin{verbatim}
	print("Hello Python world!")
\end{verbatim}
If the \texttt{python} command works on your system, you can run your program by selecting \texttt{Tools $\triangleright$ Build} in the menu or by pressing \texttt{Ctrl}-\texttt{B} (\texttt{cmd}-\texttt{B} on macOS). If you had to configure Sublime Text in the previous section, select \texttt{Tools $\triangleright$ Build System} \& then select \texttt{Python 3}. From now on you'll be able to select \texttt{Tools $\triangleright$ Build} or just press \texttt{Ctrl}-\texttt{B} (or \texttt{cmd}-\texttt{B}) to run your programs. A terminal screen should appear at the bottom of the Sublime Text window, showing the following output:
\begin{verbatim}
	Hello Python world!
	[Finished in 0.1s]
\end{verbatim}
If you don't see this output, something might have gone wrong in the program. Check every character on the line you entered. Did you accidentally capitalize \texttt{print}? Did you forget 1 or both of the quotation marks or parentheses? Programming languages expect very specific syntax, \& if you don't provide that, you'll get errors.'' -- \cite[p. 10]{Matthes2019}

\subsection{Troubleshooting}
``If you can't get \verb|hello_world.py| to run, here are a few remedies you can try that are also good general solutions for any programming problem:
\begin{itemize}
	\item When a program contains a significant error, Python displays a \textit{traceback}, which is an error report. Python looks through the file \& tries to identify the problem. Check the traceback; it might give you a clue as to what issue is preventing the program from running.
	\item Step away from your computer, take a short break, \& then try again. Remember that syntax is very important in programming so even a missing colon, a mismatched quotation mark, or mismatched parentheses can prevent a program from running properly. Reread the relevant parts of this chapter, look over your code, \& try to find the mistake.
	\item Start over again. You probably don't need to uninstall any software, but it might make sense to delete your \verb|hello_world.py| file \& re-create it from scratch.
	\item Ask someone else to follow the steps in this chapter, on your computer or a different one, \& what they do carefully. You might have missed 1 small step that someone else happens to catch.
\end{itemize}

%------------------------------------------------------------------------------%

\section{Variables \& Simple Data Types}

%------------------------------------------------------------------------------%

\section{Introducing Lists}

%------------------------------------------------------------------------------%

\section{Working with Lists}

%------------------------------------------------------------------------------%

\section{\texttt{if} Statements}

%------------------------------------------------------------------------------%

\section{Dictionaries}

%------------------------------------------------------------------------------%

\section{User Input \& \texttt{while} Loops}

%------------------------------------------------------------------------------%

\section{Functions}

%------------------------------------------------------------------------------%

\section{Classes}

%------------------------------------------------------------------------------%

\section{Files \& Exceptions}

%------------------------------------------------------------------------------%

\section{Testing Your Code}

%------------------------------------------------------------------------------%

\begin{center}
	\huge Part II: Projects
\end{center}

\begin{center}
	\Large Project 1: Alien Invasion
\end{center}

\section{A Ship that Fires Bullets}

%------------------------------------------------------------------------------%

\section{Aliens!}

%------------------------------------------------------------------------------%

\section{Scoring}

%------------------------------------------------------------------------------%

\begin{center}
	\Large Project 2: Data Visualization
\end{center}

\section{Generating Data}

%------------------------------------------------------------------------------%

\section{Downloading Data}

%------------------------------------------------------------------------------%

\section{Working with APIs}

%------------------------------------------------------------------------------%

\begin{center}
	\Large Project 3: Web Applications
\end{center}

\section{Getting Started with Django}

%------------------------------------------------------------------------------%

\section{User Accounts}

%------------------------------------------------------------------------------%

\section{Styling \& Deploying an App}

%------------------------------------------------------------------------------%

\section*{Afterword}

%------------------------------------------------------------------------------%

\section{Appendix A: Installation \& Troubleshooting}

%------------------------------------------------------------------------------%

\section{Appendix B: Text Editors \& IDEs}

%------------------------------------------------------------------------------%

\section{Appendix C: Getting Help}

%------------------------------------------------------------------------------%

\section{Appendix D: Using Git for Version Control}


%------------------------------------------------------------------------------%

\printbibliography[heading=bibintoc]
	
\end{document}