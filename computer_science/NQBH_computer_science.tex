\documentclass[oneside]{book}
\usepackage[backend=biber,natbib=true,style=authoryear]{biblatex}
\addbibresource{/home/hong/1_NQBH/reference/bib.bib}
\usepackage[vietnamese,english]{babel}
\usepackage{tocloft}
\renewcommand{\cftsecleader}{\cftdotfill{\cftdotsep}}
\usepackage[colorlinks=true,linkcolor=blue,urlcolor=red,citecolor=magenta]{hyperref}
\usepackage{amsmath,amssymb,amsthm,mathtools,float,graphicx}
\allowdisplaybreaks
\numberwithin{equation}{section}
\newtheorem{assumption}{Assumption}[chapter]
\newtheorem{conjecture}{Conjecture}[chapter]
\newtheorem{corollary}{Corollary}[chapter]
\newtheorem{definition}{Definition}[chapter]
\newtheorem{example}{Example}[chapter]
\newtheorem{lemma}{Lemma}[chapter]
\newtheorem{notation}{Notation}[chapter]
\newtheorem{principle}{Principle}[chapter]
\newtheorem{problem}{Problem}[chapter]
\newtheorem{proposition}{Proposition}[chapter]
\newtheorem{question}{Question}[chapter]
\newtheorem{remark}{Remark}[chapter]
\newtheorem{theorem}{Theorem}[chapter]
\usepackage[left=0.5in,right=0.5in,top=1.5cm,bottom=1.5cm]{geometry}
\usepackage{fancyhdr}
\pagestyle{fancy}
\fancyhf{}
\lhead{\small \textsc{Sect.} ~\thesection}
\rhead{\small \nouppercase{\leftmark}}
\renewcommand{\sectionmark}[1]{\markboth{#1}{}}
\cfoot{\thepage}
\def\labelitemii{$\circ$}

\title{Computer Science}
\author{\selectlanguage{vietnamese} Nguyễn Quản Bá Hồng\footnote{Independent Researcher, Ben Tre City, Vietnam\\e-mail: \texttt{nguyenquanbahong@gmail.com}}}
\date{\today}

\begin{document}
\maketitle
\tableofcontents

%------------------------------------------------------------------------------%

\chapter{Wikipedia's}

\section{\href{https://en.wikipedia.org/wiki/Computer_science}{Wikipedia\texttt{/}Computer Science}}
\textsf{\textbf{Fundamental areas of computer science.} \href{https://en.wikipedia.org/wiki/Programming_language_theory}{Programming language theory}, \href{https://en.wikipedia.org/wiki/Computational_complexity_theory}{Computational complexity theory}, \href{https://en.wikipedia.org/wiki/Artificial_intelligence}{Artificial intelligence}, \href{https://en.wikipedia.org/wiki/Computer_architecture}{Computer architecture}.} \href{https://en.wikipedia.org/wiki/History_of_computer_science}{History}, \href{https://en.wikipedia.org/wiki/Outline_of_computer_science}{Outline}, \href{https://en.wikipedia.org/wiki/Glossary_of_computer_science}{Glossary}, \href{https://en.wikipedia.org/wiki/Category:Computer_science}{Category}.

\textit{Computer science} is the study of \href{https://en.wikipedia.org/wiki/Computation}{computation}, \href{https://en.wikipedia.org/wiki/Automation}{automation}, \& \href{https://en.wikipedia.org/wiki/Information}{information}. Computer science spans theoretical disciplines (such as \href{https://en.wikipedia.org/wiki/Algorithm}{algorithms}, \href{https://en.wikipedia.org/wiki/Theory_of_computation}{theory of computation}, \& \href{https://en.wikipedia.org/wiki/Information_theory}{information theory}) to \href{https://en.wikipedia.org/wiki/Applied_science}{practical disciplines} (including the design \& implementation of \href{https://en.wikipedia.org/wiki/Computer_architecture}{hardware} \& \href{https://en.wikipedia.org/wiki/Computer_programming}{software}). Computer science is generally considered an area of \href{https://en.wikipedia.org/wiki/Research}{academic research} \& distinct from \href{https://en.wikipedia.org/wiki/Computer_programming}{computer programming}. 

\href{https://en.wikipedia.org/wiki/Algorithm}{Algorithms} \& \href{https://en.wikipedia.org/wiki/Data_structures}{data structures} are central to computer science. The \href{https://en.wikipedia.org/wiki/Theory_of_computation}{theory of computation} concerns abstract \href{https://en.wikipedia.org/wiki/Models_of_computation}{models of computation} \& general classes of \href{https://en.wikipedia.org/wiki/Computational_problem}{problems} that can be solved using them. The fields of \href{https://en.wikipedia.org/wiki/Cryptography}{cryptography} \& \href{https://en.wikipedia.org/wiki/Computer_security}{computer security} involve studying the means for secure communication \& for preventing \href{https://en.wikipedia.org/wiki/Vulnerability_(computing)}{security vulnerabilities}. \href{https://en.wikipedia.org/wiki/Computer_graphics_(computer_science)}{Computer graphics} \& \href{https://en.wikipedia.org/wiki/Computational_geometry}{computational geometry} address the generation of images. \href{https://en.wikipedia.org/wiki/Programming_language_theory}{Programming language theory} considers approaches to the description of computational processes, \& \href{https://en.wikipedia.org/wiki/Database}{database} theory concerns the management of repositories of data. \href{https://en.wikipedia.org/wiki/Human%E2%80%93computer_interaction}{Human--computer interaction} investigates the interfaces through which humans \& computers interact, \& \href{https://en.wikipedia.org/wiki/Software_engineering}{software engineering} focuses on the design \& principles behind developing software. Areas such as \href{https://en.wikipedia.org/wiki/Operating_system}{operating systems}, \href{https://en.wikipedia.org/wiki/Computer_network}{networks} \& \href{https://en.wikipedia.org/wiki/Embedded_system}{embedded systems} investigate the principles \& design behind \href{https://en.wikipedia.org/wiki/Complex_systems}{complex systems}. \href{https://en.wikipedia.org/wiki/Computer_architecture}{Computer architecture} describes the construction of computer components \& computer-operated equipment. \href{https://en.wikipedia.org/wiki/Artificial_intelligence}{Artificial intelligence} \& \href{https://en.wikipedia.org/wiki/Machine_learning}{machine learning} aim to synthesize goal-orientated processes such as problem-solving, decision-making, environmental adaptation, \href{https://en.wikipedia.org/wiki/Automated_planning_and_scheduling}{planning} \& learning found in humans \& animals. Within artificial intelligence, \href{https://en.wikipedia.org/wiki/Computer_vision}{computer vision} aims to understand \& process image \& video data, while \href{https://en.wikipedia.org/wiki/Natural-language_processing}{natural-language processing} aims to understand \& process textual \& linguistic data.

The fundamental concern of computer science is determining what can \& cannot be automated. The \href{https://en.wikipedia.org/wiki/Turing_Award}{Turning Award} is generally recognized as the highest distinction in computer science.'' -- \href{https://en.wikipedia.org/wiki/Computer_science}{Wikipedia\texttt{/}computer science}

\subsection{History}

\subsection{Etymology}

\subsection{Philosophy}

\subsubsection{Epistemology of computer science}

\subsubsection{Paradigms of computer science}

\subsection{Fields}

\subsubsection{Theoretical computer science}

\paragraph{Theory of computation.}

\paragraph{Information \& coding theory.}

\paragraph{Data structures \& algorithms.}

\paragraph{Programming language theory \& formal methods.}

\subsubsection{Computer systems \& computational processes}

\paragraph{Artificial intelligence.}

\paragraph{Computer architecture \& organization.}

\paragraph{Concurrent, parallel \& distributed computing.}

\paragraph{Computer networks.}

\paragraph{Computer security \& cryptography.}

\paragraph{Databases \& data mining.}

\paragraph{Computer graphics \& visualization.}

\paragraph{Image \& sound processing.}

\subsubsection{Applied computer science}

\paragraph{Computational science, finance \& engineering.}

\paragraph{Social computing \& human--computer interaction.}

\paragraph{Software engineering.}

\subsection{Discoveries}

\subsection{Programming paradigms}

\subsection{Academia}

\subsection{Education}

%------------------------------------------------------------------------------%

\chapter{The Art of Computer Programming}

\section{\href{https://www-cs-faculty.stanford.edu/~knuth/taocp.html}{The Art of Computer Programming (TAOCP)}}
``At the end of 1999, these books were named among the best 12 physical-science monographs of the century by \href{http://web.mnstate.edu/schwartz/centurylist2.html}{American Scientists}, along with: Dirac on quantum mechanics, Einstein on relativity, Mandelbrot on fractals, Pauling on the chemical bond, Russell \& Whitehead on foundations of mathematics, von Neumann \& Morgensstern on game theory, Wiener on cybernetics, Woodward \& Hoffmann on orbital symmetry, Feynmann on quantum electrodynamics, Smith on search for structure, \& Einstein's collected papers. Wow'' \href{https://www-cs-faculty.stanford.edu/~knuth/taocp.html}{``historic'' publisher's brochure from the 1st edition of Vol. 1 (1968)}. A complimentary \textit{downloadable PDF containing the collected indexes} is \href{https://www.informit.com/store/art-of-computer-programming-volumes-1-4a-boxed-set-9780321751041}{available from the publisher} to registered owners of the 4-volume boxed set. This PDF also includes the complete indexes of Vols. 1, 2, 3, \& 4A, as well as to Vol. 1 Fascicle 1 \& to Vol. 4 Fascicles 5 \& 6.''

\subsection{eBook versions}
``These volumes are now available also in portable electronic form, using PDF format prepared by the experts at \href{https://msp.org/}{Mathematical Sciences Publishers}. Special care has been taken to make the search feature work well. Thousands of useful ``clickable'' cross-references are also provided -- from exercises to their answers \& back, from the index to the text, from the text to important tables \& figures, etc.

\textbf{Warning.} Unfortunately, however, non-PDF versions have also appeared, against my recommendations, \& those versions are frankly quite awful. A great deal of expertise \& care is necessary to do the job right. If you have been misled into purchasing 1 of these inferior versions (e.g.,a Kindle version), the publishers have told me that they will replace your copy with the PDF edition that I have personally approved. \textbf{Do not purchase eTAOCP in Kindle format if you expect the mathematics to make sense}. (The ePUB format may be just as bad; I really don't want to know, \& I am really sorry that it was released). Please do not tell me about errors that you find in a non-PDF eBook; such mistakes should be reported directly to the publisher. Some non-PDF versions also masquerade as PDF. You can tell an authorized version because its copyright page (with the exception of Vol. 4 Fascicle 5) will say `Electronic version by Mathematical Sciences Publishers (MSP)'.

The authorized PDF versions can be purchased at \url{www.informit.com/taocp}. If you have purchased a different version of the eBook, \& can provide proof of purchase of that eBook, you can obtain a gratis PDF verson by sending email \& proof of purchase to \url{taocp@pearson.com}.''

\subsection{Volume 1}
\begin{itemize}
	\item \textit{Fundamental Algorithms}, 3rd Edition (Reading, Massachusetts: Addison-Wesley, 1997), xx+650pp.
	\item \textit{Volume 1 Fascicle 1}, MMIX: A RISC Computer for the New Millennium (2005), v+134pp.
\end{itemize}

\subsubsection{Brochure}

\begin{quotation}
	``I am overwhelmed by the wealth of exciting \& fresh material you have managed to pack into the book, especially in view of the fact that it is only the 1st of 7 volumes! ``Monumental'' is the only word for it $\ldots$ Moreover, it is written with a grace \& humor that is, as you know, exceedingly rare in books on mathematics. I greatly enjoyed your dedication, your flow-chart for reading the series, your notes on the exercises; above all, your choice of illustrative material throughout \& the clarity \& brevity with which you explain everything.'' -- Martin Gardner, \textit{Mathematical Games, Scientific American}
\end{quotation}
``This combined reference \& text, \textit{Fundamental Algorithms}, is the 1st volume of a planned 7-volume series. The series will provide a unified, readable, \& theoretically sound summary of the present knowledge of computer programming techniques, plus a study of their historical development.

The point of view adopted by the author differs from many contemporary books about compute programming. The author does not try to teach the reader how to use somebody else's subroutines, but is concerned rather with teaching the reader how to write better subroutines himself.

A reader who is interested primarily in programming rather than in the associated mathematics may stop reading each section as soon as the mathematics become recognizably difficult. On the other hand, a mathematically oriented reader will find a wealth of interesting material.

As a reference the series provides valuable information for system programmers, analyst programmers, \& others in the computer \& related software industries. All 7 volumes of this series may also be used in senior or graduate courses such as: Information Structures, Computer Science, Combinatorial Mathematics, Computer-Oriented Finite Mathematics, or Fundamentals of Symbolic Machine Language Programming.

Among the areas covered in Vol. 1 are the representation of information inside a computer; the structural interrelations between data elements \& how to deal with them efficiently; plus applications to simulation, numerical methods, software design, \& other factors. Also included is an introduction to fundamental topics in discrete mathematics, of special importance in the study of computer programming techniques.

There are over 850 exercises, graded according to the level of difficulty from extremely simple questions to unsolved research problems. Answers are supplied for over 90\% of the exercises. This enhances the value of the book for self-study, classroom use, \& for reference. \& it helps make it possible to organize the book so that it can be read by both mathematicians \& non-mathematicians.'' 634 pages, 71 figures, (1968), \$19.50

\textsf{Fig. 28. Representation of polynomials using 4-directional links. Shaded areas of nodes indicate information irrelevant in the context considered.}
%------------------------------------------------------------------------------%

\begin{thebibliography}{99}
	\bibitem[]{}
\end{thebibliography}

%------------------------------------------------------------------------------%

\printbibliography[heading=bibintoc]
	
\end{document}