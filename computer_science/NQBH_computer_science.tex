\documentclass[oneside]{book}
\usepackage[backend=biber,natbib=true,style=authoryear]{biblatex}
\addbibresource{/home/hong/1_NQBH/reference/bib.bib}
\usepackage[vietnamese,english]{babel}
\usepackage{tocloft}
\renewcommand{\cftsecleader}{\cftdotfill{\cftdotsep}}
\usepackage[colorlinks=true,linkcolor=blue,urlcolor=red,citecolor=magenta]{hyperref}
\usepackage{amsmath,amssymb,amsthm,mathtools,float,graphicx}
\allowdisplaybreaks
\numberwithin{equation}{section}
\newtheorem{assumption}{Assumption}[chapter]
\newtheorem{conjecture}{Conjecture}[chapter]
\newtheorem{corollary}{Corollary}[chapter]
\newtheorem{definition}{Definition}[chapter]
\newtheorem{example}{Example}[chapter]
\newtheorem{lemma}{Lemma}[chapter]
\newtheorem{notation}{Notation}[chapter]
\newtheorem{principle}{Principle}[chapter]
\newtheorem{problem}{Problem}[chapter]
\newtheorem{proposition}{Proposition}[chapter]
\newtheorem{question}{Question}[chapter]
\newtheorem{remark}{Remark}[chapter]
\newtheorem{theorem}{Theorem}[chapter]
\usepackage[left=0.5in,right=0.5in,top=1.5cm,bottom=1.5cm]{geometry}
\usepackage{fancyhdr}
\pagestyle{fancy}
\fancyhf{}
\lhead{\small \textsc{Sect.} ~\thesection}
\rhead{\small \nouppercase{\leftmark}}
\renewcommand{\sectionmark}[1]{\markboth{#1}{}}
\cfoot{\thepage}
\def\labelitemii{$\circ$}

\title{Computer Science}
\author{\selectlanguage{vietnamese} Nguyễn Quản Bá Hồng\footnote{Independent Researcher, Ben Tre City, Vietnam\\e-mail: \texttt{nguyenquanbahong@gmail.com}}}
\date{\today}

\begin{document}
\maketitle
\tableofcontents

%------------------------------------------------------------------------------%

\chapter{Wikipedia's}

\section{\href{https://en.wikipedia.org/wiki/Computer_science}{Wikipedia\texttt{/}Computer Science}}
\textsf{\textbf{Fundamental areas of computer science.} \href{https://en.wikipedia.org/wiki/Programming_language_theory}{Programming language theory}, \href{https://en.wikipedia.org/wiki/Computational_complexity_theory}{Computational complexity theory}, \href{https://en.wikipedia.org/wiki/Artificial_intelligence}{Artificial intelligence}, \href{https://en.wikipedia.org/wiki/Computer_architecture}{Computer architecture}.} \href{https://en.wikipedia.org/wiki/History_of_computer_science}{History}, \href{https://en.wikipedia.org/wiki/Outline_of_computer_science}{Outline}, \href{https://en.wikipedia.org/wiki/Glossary_of_computer_science}{Glossary}, \href{https://en.wikipedia.org/wiki/Category:Computer_science}{Category}.

\textit{Computer science} is the study of \href{https://en.wikipedia.org/wiki/Computation}{computation}, \href{https://en.wikipedia.org/wiki/Automation}{automation}, \& \href{https://en.wikipedia.org/wiki/Information}{information}. Computer science spans theoretical disciplines (such as \href{https://en.wikipedia.org/wiki/Algorithm}{algorithms}, \href{https://en.wikipedia.org/wiki/Theory_of_computation}{theory of computation}, \& \href{https://en.wikipedia.org/wiki/Information_theory}{information theory}) to \href{https://en.wikipedia.org/wiki/Applied_science}{practical disciplines} (including the design \& implementation of \href{https://en.wikipedia.org/wiki/Computer_architecture}{hardware} \& \href{https://en.wikipedia.org/wiki/Computer_programming}{software}). Computer science is generally considered an area of \href{https://en.wikipedia.org/wiki/Research}{academic research} \& distinct from \href{https://en.wikipedia.org/wiki/Computer_programming}{computer programming}. 

\href{https://en.wikipedia.org/wiki/Algorithm}{Algorithms} \& \href{https://en.wikipedia.org/wiki/Data_structures}{data structures} are central to computer science. The \href{https://en.wikipedia.org/wiki/Theory_of_computation}{theory of computation} concerns abstract \href{https://en.wikipedia.org/wiki/Models_of_computation}{models of computation} \& general classes of \href{https://en.wikipedia.org/wiki/Computational_problem}{problems} that can be solved using them. The fields of \href{https://en.wikipedia.org/wiki/Cryptography}{cryptography} \& \href{https://en.wikipedia.org/wiki/Computer_security}{computer security} involve studying the means for secure communication \& for preventing \href{https://en.wikipedia.org/wiki/Vulnerability_(computing)}{security vulnerabilities}. \href{https://en.wikipedia.org/wiki/Computer_graphics_(computer_science)}{Computer graphics} \& \href{https://en.wikipedia.org/wiki/Computational_geometry}{computational geometry} address the generation of images. \href{https://en.wikipedia.org/wiki/Programming_language_theory}{Programming language theory} considers approaches to the description of computational processes, \& \href{https://en.wikipedia.org/wiki/Database}{database} theory concerns the management of repositories of data. \href{https://en.wikipedia.org/wiki/Human%E2%80%93computer_interaction}{Human--computer interaction} investigates the interfaces through which humans \& computers interact, \& \href{https://en.wikipedia.org/wiki/Software_engineering}{software engineering} focuses on the design \& principles behind developing software. Areas such as \href{https://en.wikipedia.org/wiki/Operating_system}{operating systems}, \href{https://en.wikipedia.org/wiki/Computer_network}{networks} \& \href{https://en.wikipedia.org/wiki/Embedded_system}{embedded systems} investigate the principles \& design behind \href{https://en.wikipedia.org/wiki/Complex_systems}{complex systems}. \href{https://en.wikipedia.org/wiki/Computer_architecture}{Computer architecture} describes the construction of computer components \& computer-operated equipment. \href{https://en.wikipedia.org/wiki/Artificial_intelligence}{Artificial intelligence} \& \href{https://en.wikipedia.org/wiki/Machine_learning}{machine learning} aim to synthesize goal-orientated processes such as problem-solving, decision-making, environmental adaptation, \href{https://en.wikipedia.org/wiki/Automated_planning_and_scheduling}{planning} \& learning found in humans \& animals. Within artificial intelligence, \href{https://en.wikipedia.org/wiki/Computer_vision}{computer vision} aims to understand \& process image \& video data, while \href{https://en.wikipedia.org/wiki/Natural-language_processing}{natural-language processing} aims to understand \& process textual \& linguistic data.

The fundamental concern of computer science is determining what can \& cannot be automated. The \href{https://en.wikipedia.org/wiki/Turing_Award}{Turning Award} is generally recognized as the highest distinction in computer science.'' -- \href{https://en.wikipedia.org/wiki/Computer_science}{Wikipedia\texttt{/}computer science}

\subsection{History}

\subsection{Etymology}

\subsection{Philosophy}

\subsubsection{Epistemology of computer science}

\subsubsection{Paradigms of computer science}

\subsection{Fields}

\subsubsection{Theoretical computer science}

\paragraph{Theory of computation.}

\paragraph{Information \& coding theory.}

\paragraph{Data structures \& algorithms.}

\paragraph{Programming language theory \& formal methods.}

\subsubsection{Computer systems \& computational processes}

\paragraph{Artificial intelligence.}

\paragraph{Computer architecture \& organization.}

\paragraph{Concurrent, parallel \& distributed computing.}

\paragraph{Computer networks.}

\paragraph{Computer security \& cryptography.}

\paragraph{Databases \& data mining.}

\paragraph{Computer graphics \& visualization.}

\paragraph{Image \& sound processing.}

\subsubsection{Applied computer science}

\paragraph{Computational science, finance \& engineering.}

\paragraph{Social computing \& human--computer interaction.}

\paragraph{Software engineering.}

\subsection{Discoveries}

\subsection{Programming paradigms}

\subsection{Academia}

\subsection{Education}

%------------------------------------------------------------------------------%

\part{The Art of Computer Programming}

\section*{\href{https://www-cs-faculty.stanford.edu/~knuth/taocp.html}{The Art of Computer Programming (TAOCP)}}
``At the end of 1999, these books were named among the best 12 physical-science monographs of the century by \href{http://web.mnstate.edu/schwartz/centurylist2.html}{American Scientists}, along with: Dirac on quantum mechanics, Einstein on relativity, Mandelbrot on fractals, Pauling on the chemical bond, Russell \& Whitehead on foundations of mathematics, von Neumann \& Morgensstern on game theory, Wiener on cybernetics, Woodward \& Hoffmann on orbital symmetry, Feynmann on quantum electrodynamics, Smith on search for structure, \& Einstein's collected papers. Wow'' \href{https://www-cs-faculty.stanford.edu/~knuth/taocp.html}{``historic'' publisher's brochure from the 1st edition of Vol. 1 (1968)}. A complimentary \textit{downloadable PDF containing the collected indexes} is \href{https://www.informit.com/store/art-of-computer-programming-volumes-1-4a-boxed-set-9780321751041}{available from the publisher} to registered owners of the 4-volume boxed set. This PDF also includes the complete indexes of Vols. 1, 2, 3, \& 4A, as well as to Vol. 1 Fascicle 1 \& to Vol. 4 Fascicles 5 \& 6.''

\subsection*{eBook versions}
``These volumes are now available also in portable electronic form, using PDF format prepared by the experts at \href{https://msp.org/}{Mathematical Sciences Publishers}. Special care has been taken to make the search feature work well. Thousands of useful ``clickable'' cross-references are also provided -- from exercises to their answers \& back, from the index to the text, from the text to important tables \& figures, etc.

\textbf{Warning.} Unfortunately, however, non-PDF versions have also appeared, against my recommendations, \& those versions are frankly quite awful. A great deal of expertise \& care is necessary to do the job right. If you have been misled into purchasing 1 of these inferior versions (e.g.,a Kindle version), the publishers have told me that they will replace your copy with the PDF edition that I have personally approved. \textbf{Do not purchase eTAOCP in Kindle format if you expect the mathematics to make sense}. (The ePUB format may be just as bad; I really don't want to know, \& I am really sorry that it was released). Please do not tell me about errors that you find in a non-PDF eBook; such mistakes should be reported directly to the publisher. Some non-PDF versions also masquerade as PDF. You can tell an authorized version because its copyright page (with the exception of Vol. 4 Fascicle 5) will say `Electronic version by Mathematical Sciences Publishers (MSP)'.

The authorized PDF versions can be purchased at \url{www.informit.com/taocp}. If you have purchased a different version of the eBook, \& can provide proof of purchase of that eBook, you can obtain a gratis PDF verson by sending email \& proof of purchase to \url{taocp@pearson.com}.''

\subsection*{Volume 1}
\begin{itemize}
	\item \textit{Fundamental Algorithms}, 3rd Edition (Reading, Massachusetts: Addison-Wesley, 1997), xx+650pp.
	\item \textit{Volume 1 Fascicle 1}, MMIX: A RISC Computer for the New Millennium (2005), v+134pp.
\end{itemize}

\subsubsection{Brochure}

\begin{quotation}
	``I am overwhelmed by the wealth of exciting \& fresh material you have managed to pack into the book, especially in view of the fact that it is only the 1st of 7 volumes! ``Monumental'' is the only word for it $\ldots$ Moreover, it is written with a grace \& humor that is, as you know, exceedingly rare in books on mathematics. I greatly enjoyed your dedication, your flow-chart for reading the series, your notes on the exercises; above all, your choice of illustrative material throughout \& the clarity \& brevity with which you explain everything.'' -- Martin Gardner, \textit{Mathematical Games, Scientific American}
\end{quotation}
``This combined reference \& text, \textit{Fundamental Algorithms}, is the 1st volume of a planned 7-volume series. The series will provide a unified, readable, \& theoretically sound summary of the present knowledge of computer programming techniques, plus a study of their historical development.

The point of view adopted by the author differs from many contemporary books about compute programming. The author does not try to teach the reader how to use somebody else's subroutines, but is concerned rather with teaching the reader how to write better subroutines himself.

A reader who is interested primarily in programming rather than in the associated mathematics may stop reading each section as soon as the mathematics become recognizably difficult. On the other hand, a mathematically oriented reader will find a wealth of interesting material.

As a reference the series provides valuable information for system programmers, analyst programmers, \& others in the computer \& related software industries. All 7 volumes of this series may also be used in senior or graduate courses such as: Information Structures, Computer Science, Combinatorial Mathematics, Computer-Oriented Finite Mathematics, or Fundamentals of Symbolic Machine Language Programming.

Among the areas covered in Vol. 1 are the representation of information inside a computer; the structural interrelations between data elements \& how to deal with them efficiently; plus applications to simulation, numerical methods, software design, \& other factors. Also included is an introduction to fundamental topics in discrete mathematics, of special importance in the study of computer programming techniques.

There are over 850 exercises, graded according to the level of difficulty from extremely simple questions to unsolved research problems. Answers are supplied for over 90\% of the exercises. This enhances the value of the book for self-study, classroom use, \& for reference. \& it helps make it possible to organize the book so that it can be read by both mathematicians \& non-mathematicians.'' 634 pages, 71 figures, (1968), \$19.50

\textsf{Fig. 28. Representation of polynomials using 4-directional links. Shaded areas of nodes indicate information irrelevant in the context considered.}

%------------------------------------------------------------------------------%

\chapter{The Art of Computer Programming. Vol. 1: Fundamental Algorithms}

\begin{quotation}
	\textit{This series of books is affectionately\footnote{\textbf{affectionately} [adv] in a way that shows caring feelings \& love for somebody.} dedicated to the Type 650 computer once installed at Case Institute of Technology, in remembrance\footnote{\textbf{remembrance} [n] [uncountable] the act or process of remembering an event in the past or a person who is dead.} of many pleasant\footnote{\textbf{pleasant} [a] (\textbf{pleasanter, pleasantest}) (\textbf{more pleasant} \& \textbf{most pleasant} are more common) enjoyable, pleasing or attractive, \textsc{opposite}: \textbf{unpleasant}.} evenings.}
\end{quotation}

\section*{Preface}
\begin{quotation}
	\textit{``Here is your book, the one your thousands of letters have asked us to publish. It has taken us years to do, checking \& rechecking countless\footnote{\textbf{countless} [a] [usually before noun] very many; too many to be counted or mentioned.} recipes\footnote{\textbf{recipe} [n] \textbf{1.} \textbf{recipe (for something)} a set of instructions that tells you how to cook something \& the ingredients you need for it; \textbf{2.} \textbf{recipe for something} a method or an idea that seems likely to have a particular result, \textsc{synonym}: \textbf{formula}.} to bring you only the best, only the interesting, only the perfect. Now we can say, without a shadow of a doubt, that every single 1 of them, if you follow the directions to the letter, will work for you exactly as well as it did for us, even if you have never cooked before.''} -- McCall's Cookbook (1963)
\end{quotation}
``\textsc{The process} of preparing programs for a digital\footnote{\textbf{digital} [a] \textbf{1.} using a system of receiving \& sending information as a series of the numbers 1 \& 0, showing that an electronic signal is there or is not there; connected with computer technology; \textbf{2.} (of clocks, watches, etc.) displaying only the appropriate numbers, rather than pointing to numbers from a larger set of numbers; other information displayed in this way; \textbf{3.} connected with a finger or the fingers of the hand.} computer is especially attractive\footnote{\textbf{attractive} [a] \textbf{1.} (of a person or an animal) pleasant to look at, especially in a sexual way; making an animal interested in a sexual way, \textsc{opposite}: \textbf{unattractive}; \textbf{2.} (of a thing or a place) pleasant to look at or be in, \textsc{opposite}: \textbf{unattractive}; \textbf{3.} having features or qualities that make something seem interesting \& worth having, \textsc{synonym}: \textbf{appealing}, \textsc{opposite}: \textbf{unattractive}; \textbf{4.} (\textit{physics}) involving the force thta pulls things towards each other, \textsc{opposite}: \textbf{repulsive}.}, not only because it can be economically\footnote{\textbf{economically} [adv] \textbf{1.} in a way that is connected with the trade, industry \& development of wealth of a country, an area or a society; \textbf{2.} in a way that provides good service or value in relation to the amount of time or money spent.} \& scientifically\footnote{\textbf{scientifically} [adv] \textbf{1.} in a way that is connected with science; \textbf{2.} in a careful \& organized way.} rewarding\footnote{\textbf{rewarding} [a] \textbf{1.} (of an activity) worth doing; that makes you happy because you think it is useful or important; \textbf{2.} producing a lot of money, \textsc{synonym}: \textbf{profitable}.}, but also because it can be an aesthetic\footnote{\textbf{aesthetic} [a] (\textit{North American English also} \textbf{esthetic}) \textbf{1.} concerned with beauty \& art \& the understanding of beautiful things; \textbf{2.} beautiful to look at; [n] (\textit{North American English also} \textbf{esthetic}) \textbf{1.} [countable] \textbf{aesthetic (of something)} a set of principles that express the aesthetic qualities \& ideas of a particular artist or a particular group of artists, writers, etc.; \textbf{2.} (\textbf{aesthetics}) [uncountable] the branch of philosophy that studies the principles of beauty, especially in art.} experience much like composing\footnote{\textbf{compose} [v] \textbf{1.} (\textbf{be composed of something}) to be made or formed from several substances, parts of people; \textbf{2.} (not used in the progressive tenses) \textbf{compose something} to combine together to form a whole, \textsc{synonym}: \textbf{make something up}; \textbf{3.} to write a piece of music; \textbf{4.} to write something, especially a poem.} poetry\footnote{\textbf{poetry} [n] [uncountable] a collection of poems; poems in general, \textsc{synonym}: \textbf{verse}.} or music. This book is the 1st volume of a multi-volume set of books that has been designed to train the reader in various skills that go into a programmer's craft\footnote{\textbf{craft} [n] \textbf{1.} [countable, uncountable] an activity involving a special skill at making things with your hands; \textbf{2.} [singular] all the skills needed for a particular activity; \textbf{3.} (plural \textbf{craft}) [countable] a boat or ship; [v] [usually passive] \textbf{craft something} to make something using a special skill, \textsc{synonym}: \textbf{fashion}.}.

The following chapters are \textit{not} meant to serve as an introduction to computer programming; the reader is supposed to have had some previous experience. The prerequisites\footnote{\textbf{prerequisite} [n] [usually singular] something that must exist or happen before something else can happen or be done, \textsc{synonym}: \textbf{precondition}.} are actually very simple, but a beginner requires time \& practice in order to understand the concept of a digital computer. The reader should possess:
\begin{itemize}
	\item[a)] Some idea of how a stored-program digital computer works; not necessarily the electronics\footnote{\textbf{electronics} [n] \textbf{1.} [uncountable] the branch of science \& technology that studies electric currents in electronic equipment; \textbf{2.} [uncountable] the use of electronic technology; the making of electronic products; \textbf{3.} [plural] the electronic circuits \& components used in electronic equipment.}, rather the manner\footnote{\textbf{manner} [n] \textbf{1.} [singular] the way that something is done or happens; \textbf{2.} [singular] the way that somebody behaves towards other people; \textbf{3.} (\textbf{manners}) [plural] behavior that is considered to be polite in a particular society or culture; \textbf{4.} (\textbf{manners (of somebody\texttt{/}something)}) [plural] the habits \& customs of a particular group of people; \textbf{all manner of somebody\texttt{/}something} [idiom] many different types of people or things; \textbf{in the manner of somebody\texttt{/}something} [idiom] in a style that is typical of somebody\texttt{/}something.} in which instructions\footnote{\textbf{instruction} [n] \textbf{1.} (\textbf{instructions}) [plural] detailed information on how to do or use something, \textsc{synonym}: \textbf{direction}; \textbf{2.} [countable, usually plural] something that somebody tells you to do, \textsc{synonym}: \textbf{order}; \textbf{3.} [countable] (\textit{computing}) a code in a program that tells a computer to perform a particular operation; \textbf{4.} [uncountable] the act of teaching something to somebody.} can be kept in the machine\footnote{\textbf{machine} [n] \textbf{1.} (often in compounds) a piece of equipment with moving parts that is designed to do a particular job. The power used to work a machine may be electricity, steam, gas, etc. or human power; \textbf{2.} a group of people who operate in an efficient way within an organization.}'s memory\footnote{\textbf{memory} [n] (plural \textbf{memories}) \textbf{1.} [countable, uncountable] your ability to remember things; the part of your mind in which you store things that you remember; \textbf{2.} [uncountable] \textbf{in\texttt{/}within $\ldots$ memory} the period of time that somebody is able to remember events; \textbf{3.} [countable] a thought of something that you remember from the past; \textbf{4.} [uncountable, countable] the part of a computer where data are stored; the amount of space in a computer for storing data; \textbf{5.} [uncountable] \textbf{memory (of somebody)} what is remembered about somebody after they have died; \textbf{from memory} [idiom] without reading or looking at notes; \textbf{in memory of somebody, to the memory of somebody} [idiom] intended to show respect \& remind people of somebody who has died; \textbf{within\texttt{/}in living memory} [idiom] at a time, or during the time, that is remembered by people still alive.} \& successively\footnote{\textbf{successive} [a] [only before noun] following immediately one after the other, \textsc{synonym}: \textbf{consecutive}.} executed\footnote{\textbf{execute} [v] \textbf{1.} [usually passive] to kill somebody, especially as a legal punishment; \textbf{2.} \textbf{execute something} to do a piece of work, perform a duty, put a plan into action, etc.; \textbf{3.} \textbf{execute something} (\textit{computing}) carry out an instruction or program; \textbf{4.} \textbf{execute something} (\textit{law}) to follow the instructions in a legal document; to make a document legally valid.}.
	\item[b)] An ability to put the solutions to problems into such explicit\footnote{\textbf{explicit} [a] \textbf{1.} saying something clearly \& exactly; \textbf{2.} showing or referring to sex in a very obvious or detailed way.} terms that a computer can ``understand'' them. (These machines have no common sense\footnote{\textbf{common sense} [n] [uncountable] the ability to think about things in a practical way \& make sensible decisions.}; they do exactly\footnote{\textbf{exactly} [adv] used to emphasize that something is correct in every way or in every detail, \textsc{synonym}: \textbf{precisely}.} as they are told, no more \& no less. This fact is the hardest concept to grasp\footnote{\textbf{grasp} [v] \textbf{1.} to understand something completely; \textbf{2.} \textbf{grasp an opportunity} to take an opportunity without hesitating \& use it; \textbf{3.} \textbf{grasp somebody\texttt{/}something} to take a firm hold of somebody\texttt{/}something, \textsc{synonym}: \textbf{grip}; [n] [usually singular] \textbf{1.} a person's understanding of a subject; \textbf{2.} a firm hold of somebody\texttt{/}something or control over somebody\texttt{/}something; \textbf{3.} the ability to get or achieve something.} when one 1st tries to use a computer.)
	\item[c)] Some knowledge of the most elementary\footnote{\textbf{elementary} [a] \textbf{1.} connected with the 1st stages of a course of study, or the 1st years at school; \textbf{2.} of the most basic kind; \textbf{3.} very simple \& easy.} computer techniques, such as looping\footnote{\textbf{loop} [n] \textbf{1.} a shape like a curve or circle made by a line curving right around; \textbf{2.} a piece of rope, wire, etc. in the shape of a curve or circle; \textbf{3.} a long, narrow piece of film or tape on which the pictures \& sound are repeated continuously; \textbf{4.} (\textit{computing}) a set of instructions that is repeated again \& again until a particular condition is satisfied; \textbf{5.} a complete circuit for electrical current; \textbf{6.} (\textit{British English}) a railway line or road that leaves the main track or road \& then joins it again; [v] \textbf{1.} [transitive] \textbf{loop something $+$ adv.\texttt{/}prep.} to form or bend something into a loop; \textbf{2.} [intransitive] \textbf{$+$ adv.\texttt{/}prep.} to move in a way that makes the shape of a loop; \textbf{loop the loop} [idiom] to fly or make a plane fly in a circle going up \& down.} (performing\footnote{\textbf{perform} [v] \textbf{1.} [transitive] \textbf{perform something} to do something, such as a piece of work, task or duty, \textsc{synonym}: \textbf{carry something out}; \textbf{2.} [intransitive] \textbf{$+$ adv.\texttt{/}prep.} to work or function well or badly. In this meaning, where there is no adverb or preposition, \textbf{perform} means `perform well'.; \textbf{3.} [transitive, intransitive] \textbf{perform (something)} to entertain an audience by playing a piece of music, acting in a play, etc.} a set of instructions\footnote{\textbf{instruction} [n] \textbf{1.} (\textbf{instructions}) [plural] detailed information on how to do or use something, \textsc{synonym}: \textbf{direction}; \textbf{2.} [countable, usually plural] something that somebody tells you to do, \textsc{synonym}: \textbf{order}; \textbf{3.} [countable] (\textit{computing}) a code in a program that tells a computer to perform a particular operation; \textbf{4.} [uncountable] the act of teaching something to somebody.} repeatedly\footnote{\textbf{repeatedly} [adv] many times; again \& again.}), the use of subroutines\footnote{\textbf{subroutine} [n] (also \textbf{subprogram}) (\textit{computing}) a set of instructions which perform a task within a program.}, \& the use of indexed variables.
	\item[d)] A little knowledge of common computer jargon -- ``memory,'' ``registers,'' ``bits,'' ``floating point,'' ``overflow,'' ``software.'' Most words not defined in the text are given brief definitions in the index at the close of each volume.
\end{itemize}
'' -- \cite[Preface, pp. v--]{Knuth1997}

%------------------------------------------------------------------------------%

%\begin{thebibliography}{99}
%	\bibitem[]{}
%\end{thebibliography}

%------------------------------------------------------------------------------%

\printbibliography[heading=bibintoc]
	
\end{document}