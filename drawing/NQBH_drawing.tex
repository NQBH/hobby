\documentclass[oneside]{book}
\usepackage[backend=biber,natbib=true,style=authoryear]{biblatex}
\addbibresource{/home/hong/1_NQBH/reference/bib.bib}
\usepackage[vietnamese,english]{babel}
\usepackage{tocloft}
\renewcommand{\cftsecleader}{\cftdotfill{\cftdotsep}}
\usepackage[colorlinks=true,linkcolor=blue,urlcolor=red,citecolor=magenta]{hyperref}
\usepackage{amsmath,amssymb,amsthm,mathtools,float,graphicx}
\allowdisplaybreaks
\numberwithin{equation}{section}
\newtheorem{assumption}{Assumption}[chapter]
\newtheorem{conjecture}{Conjecture}[chapter]
\newtheorem{corollary}{Corollary}[chapter]
\newtheorem{definition}{Definition}[chapter]
\newtheorem{example}{Example}[chapter]
\newtheorem{lemma}{Lemma}[chapter]
\newtheorem{notation}{Notation}[chapter]
\newtheorem{principle}{Principle}[chapter]
\newtheorem{problem}{Problem}[chapter]
\newtheorem{proposition}{Proposition}[chapter]
\newtheorem{question}{Question}[chapter]
\newtheorem{remark}{Remark}[chapter]
\newtheorem{theorem}{Theorem}[chapter]
\usepackage[left=0.5in,right=0.5in,top=1.5cm,bottom=1.5cm]{geometry}
\usepackage{fancyhdr}
\pagestyle{fancy}
\fancyhf{}
\lhead{\small \textsc{Sect.} ~\thesection}
\rhead{\small \nouppercase{\leftmark}}
\renewcommand{\sectionmark}[1]{\markboth{#1}{}}
\cfoot{\thepage}
\def\labelitemii{$\circ$}

\title{Drawing}
\author{\selectlanguage{vietnamese} Nguyễn Quản Bá Hồng\footnote{Independent Researcher, Ben Tre City, Vietnam\\e-mail: \texttt{nguyenquanbahong@gmail.com}}}
\date{\today}

\begin{document}
\maketitle
\setcounter{secnumdepth}{4}
\setcounter{tocdepth}{4}
\tableofcontents

%------------------------------------------------------------------------------%

\chapter{Wikipedia's}

\section{\href{https://en.wikipedia.org/wiki/Graphics_tablet}{Wikipedia\texttt{/}Graphics Tablet}}
\textsf{Fig. A graphic tablet.}

``A \textit{graphics tablet} (also known as a \textit{digitizer, digital graphic tablet, pen tablet, drawing tablet}, or \textit{digital art board}) is a computer \href{https://en.wikipedia.org/wiki/Input_device}{input device} that enables a user to hand-draw images, animations \& graphics, with a special pen-like \href{https://en.wikipedia.org/wiki/Stylus_(computing)}{stylus}, similar to the way a person draws images with a pencil \& paper. These tablets may also be used to capture data or handwritten signatures. It can also be used to trace an image from a piece of paper that is taped or otherwise secured to the tablet surface. Capturing data in this way, by tracing or entering the corners of linear \href{https://en.wikipedia.org/wiki/Polygonal_chain}{polylines} or shapes, is called \href{https://en.wikipedia.org/wiki/Digitizing}{digitizing}.

The device consists of a rough surface upon which the user may ``draw'' or trace an image using the attached \href{https://en.wikipedia.org/wiki/Stylus_(computing)}{stylus}, a pen-like drawing apparatus. The image is shown on the computer \href{https://en.wikipedia.org/wiki/Computer_display}{minor}, though some graphic tablets now also incorporate an \href{https://en.wikipedia.org/wiki/LCD_screen}{LCD screen} for more realistic or natural experience \& usability.

Some tablets are intended as a replacement for the computer mouse as the primary pointing \& navigation device for desktop computers.'' -- \href{https://en.wikipedia.org/wiki/Graphics_tablet}{Wikipedia\texttt{/}graphics tablet}

\subsection{History}

\subsection{Characteristics}

\subsection{Types}

\subsection{Pucks}

\subsection{Embedded LCD tablets}

\subsection{Uses}

\subsection{Similar devices}

%------------------------------------------------------------------------------%

\selectlanguage{english}
\begin{thebibliography}{99}
	\bibitem[]{}
\end{thebibliography}

%------------------------------------------------------------------------------%

\printbibliography[heading=bibintoc]
	
\end{document}