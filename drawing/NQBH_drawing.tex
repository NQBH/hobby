\documentclass[oneside]{book}
\usepackage[backend=biber,natbib=true,style=authoryear]{biblatex}
\addbibresource{/home/hong/1_NQBH/reference/bib.bib}
\usepackage[vietnamese,english]{babel}
\usepackage{tocloft}
\renewcommand{\cftsecleader}{\cftdotfill{\cftdotsep}}
\usepackage[colorlinks=true,linkcolor=blue,urlcolor=red,citecolor=magenta]{hyperref}
\usepackage{amsmath,amssymb,amsthm,mathtools,float,graphicx}
\allowdisplaybreaks
\numberwithin{equation}{section}
\newtheorem{assumption}{Assumption}[chapter]
\newtheorem{conjecture}{Conjecture}[chapter]
\newtheorem{corollary}{Corollary}[chapter]
\newtheorem{definition}{Definition}[chapter]
\newtheorem{example}{Example}[chapter]
\newtheorem{lemma}{Lemma}[chapter]
\newtheorem{notation}{Notation}[chapter]
\newtheorem{principle}{Principle}[chapter]
\newtheorem{problem}{Problem}[chapter]
\newtheorem{proposition}{Proposition}[chapter]
\newtheorem{question}{Question}[chapter]
\newtheorem{remark}{Remark}[chapter]
\newtheorem{theorem}{Theorem}[chapter]
\usepackage[left=0.5in,right=0.5in,top=1.5cm,bottom=1.5cm]{geometry}
\usepackage{fancyhdr}
\pagestyle{fancy}
\fancyhf{}
\lhead{\small \textsc{Sect.} ~\thesection}
\rhead{\small \nouppercase{\leftmark}}
\renewcommand{\sectionmark}[1]{\markboth{#1}{}}
\cfoot{\thepage}
\def\labelitemii{$\circ$}

\title{Drawing}
\author{\selectlanguage{vietnamese} Nguyễn Quản Bá Hồng\footnote{Independent Researcher, Ben Tre City, Vietnam\\e-mail: \texttt{nguyenquanbahong@gmail.com}}}
\date{\today}

\begin{document}
\maketitle
\setcounter{secnumdepth}{4}
\setcounter{tocdepth}{4}
\tableofcontents

%------------------------------------------------------------------------------%

\chapter{Wikipedia's}

\section{\href{https://en.wikipedia.org/wiki/Graphics_tablet}{Wikipedia\texttt{/}Graphics Tablet}}
\textsf{Fig. A graphic tablet.}

``A \textit{graphics tablet} (also known as a \textit{digitizer, digital graphic tablet, pen tablet, drawing tablet}, or \textit{digital art board}) is a computer \href{https://en.wikipedia.org/wiki/Input_device}{input device} that enables a user to hand-draw images, animations \& graphics, with a special pen-like \href{https://en.wikipedia.org/wiki/Stylus_(computing)}{stylus}, similar to the way a person draws images with a pencil \& paper. These tablets may also be used to capture data or handwritten signatures. It can also be used to trace an image from a piece of paper that is taped or otherwise secured to the tablet surface. Capturing data in this way, by tracing or entering the corners of linear \href{https://en.wikipedia.org/wiki/Polygonal_chain}{polylines} or shapes, is called \href{https://en.wikipedia.org/wiki/Digitizing}{digitizing}.

The device consists of a rough surface upon which the user may ``draw'' or trace an image using the attached \href{https://en.wikipedia.org/wiki/Stylus_(computing)}{stylus}, a pen-like drawing apparatus. The image is shown on the computer \href{https://en.wikipedia.org/wiki/Computer_display}{minor}, though some graphic tablets now also incorporate an \href{https://en.wikipedia.org/wiki/LCD_screen}{LCD screen} for more realistic or natural experience \& usability.

Some tablets are intended as a replacement for the computer mouse as the primary pointing \& navigation device for desktop computers.'' -- \href{https://en.wikipedia.org/wiki/Graphics_tablet}{Wikipedia\texttt{/}graphics tablet}

\subsection{History}
``The 1st electronic handwriting device was the \href{https://en.wikipedia.org/wiki/Telautograph}{Telautograph}, patented by \href{https://en.wikipedia.org/wiki/Elisha_Gray}{Elisha Gray} in 1888.

The 1st graphic tablet resembling contemporary tablets \& used for handwriting recognition by a computer was the \textit{Stylator} in 1957. Better known (\& often misstated as the 1st digitizer tablet) is the \href{https://en.wikipedia.org/wiki/RAND_Tablet}{RAND Tablet} also known as the \textit{Grafacon} (for Graphic Converter), introduced in 1964. The RAND Tablet employed a grid of wires under the surface of the pad that encoded horizontal \& vertical \href{https://en.wikipedia.org/wiki/Coordinates}{coordinates} in a small \href{https://en.wikipedia.org/wiki/Electrostatic}{electrostatic} signal. The stylus received the signal by capacitive coupling, which could then be decoded back as coordinate information.

The \href{https://en.wikipedia.org/wiki/Acoustic_tablet}{acoustic tablet}, or \textit{spark tablet}, used a stylus that generated clicks with a \href{https://en.wikipedia.org/wiki/Spark_plug}{spark plug}. The clicks were then triangulated by a series of microphones to locate the pen in space. The system was fairly complex \& expensive, \& the sensors were susceptible to interference by external noise.

Digitizers were popularized in the mid-1970s \& early 1980s by the commercial success of the ID (Intelligent Digitizer) \& BitPad manufactured by the Summagraphics Corp. The Summagraphics digitizers were sold under the company's name but were also private labeled for \href{https://en.wikipedia.org/wiki/Hewlett-Packard}{HP}, \href{https://en.wikipedia.org/wiki/Tektronix}{Tektronix}, \href{https://en.wikipedia.org/wiki/Apple_Inc.}{Apple}, \href{https://en.wikipedia.org/wiki/Evans_%26_Sutherland}{Evans \& Sutherland} \& several other graphic system manufacturers. The ID model was the 1st graphics tablet to make use of what was at the time, the new \href{https://en.wikipedia.org/wiki/Microprocessor}{Intel microprofessor} technology. This embedded processing power allowed the ID models to have twice the accuracy of previous models while still making use of the same foundation technology. Key to this accuracy improvement were 2 US Patents issued to Stephen Domyan, Robert Davis, \& Edward Snyder. The Bit Pad model was the 1st attempt at a low cost graphics tablet with an initial selling price of \$555 when other graphics tablets were selling in the \$2,000--\$3,000 price range. This lower cost opened up the opportunities for would be entrepreneurs to be able to write graphics software for a multitude of new applications. These digitizers were used as the input device for many high-end \href{https://en.wikipedia.org/wiki/Computer-aided_design}{CAD} (Computer Aided Design) systems as well as bundled with PCs \& PC-based CAD software like \href{https://en.wikipedia.org/wiki/AutoCAD}{AutoCAD}. These tablets used a \href{https://en.wikipedia.org/wiki/Magnetostriction}{magnetostriction} technology which used wires made of a special \href{https://en.wikipedia.org/wiki/Alloy}{alloy} stretched over a solid substrate to accurately locate the tip of a stylus or the center of a digitizer cursor on the surface of the tablet. This technology also allowed Proximity or ``Z'' axis measurement.

In 1981, musician \href{https://en.wikipedia.org/wiki/Todd_Rundgren}{Todd Rundgren} created the 1st color graphic tablet software for personal computers, which was licensed to Apple as the Utopia Graphic Tablet System.

In 1981 also was released an \href{https://en.wikipedia.org/wiki/Quantel_Paintbox}{Quantel Paintbox} color graphic workstation. This model was equipped with a 1st pressure sensitive tablet.

The 1st \href{https://en.wikipedia.org/wiki/Home_computer}{home computer} graphic tablet was the \href{https://en.wikipedia.org/wiki/KoalaPad}{KoalaPad}, released in 1983. Though originally designed for the \href{https://en.wikipedia.org/wiki/Apple_II}{Apple II}, the Koala eventually broadened its applicability to practically all home computers with graphic support, examples of which include the \href{https://en.wikipedia.org/wiki/TRS-80_Color_Computer}{TRS-80 Color Computer}, \href{https://en.wikipedia.org/wiki/Commodore_64}{Commodore 64}, \& \href{https://en.wikipedia.org/wiki/Atari_8-bit_family}{Atari 8-bit family}. Competing tablets were eventually produced; the tablets produced by \href{https://en.wikipedia.org/wiki/Atari}{Atari} were generally considered to be of high quality.

In the 1980s, several vendors of graphic tablets began to include additional functions, such as \href{https://en.wikipedia.org/wiki/Handwriting_recognition}{handwriting recognition} \& on-tablet menus.'' -- \href{https://en.wikipedia.org/wiki/Graphics_tablet#History}{Wikipedia\texttt{/}graphics tablet\texttt{/}history}

\subsection{Characteristics}
``Typically tablets are characterized by size of the device, drawing area, its resolution size ($\ll$active area$\gg$, which is measured in \href{https://en.wikipedia.org/wiki/Lines_per_inch}{lpi}), pressure sensitivity (level of varying the size of strokes with pressure), number of buttons \& types \& number of interfaces: \href{https://en.wikipedia.org/wiki/Bluetooth}{Bluetooth}, \href{https://en.wikipedia.org/wiki/USB}{USB}; etc. The actual drawing accuracy is restricted to pen's nib size.'' -- \href{https://en.wikipedia.org/wiki/Graphics_tablet#Characteristics}{Wikipedia\texttt{/}graphics tablet\texttt{/}characteristics}

\subsection{Types}
``There have been many attempts to categorize the technologies that have been used for graphic tablets:
\begin{itemize}
	\item \textbf{Passive tablets.} Passive tablets make use of \href{https://en.wikipedia.org/wiki/Electromagnetic_induction}{electromagnetic induction} technology, where the horizontal \& vertical wires of the tablet operate as both transmitting \& receiving coils (as opposed to the wires of the RAND Tablet which only transmit). The tablet generates an electromagnetic signal, which is received by the \href{https://en.wikipedia.org/wiki/LC_circuit}{LC circuit} in the stylus. The wires in the tablet then change to a receiving mode \& read the signal generated by the stylus. Modern arrangements also provide \href{https://en.wikipedia.org/wiki/Pressure}{pressure} sensitivity \& 1 or more buttons, with the electronics for this information present in the stylus. On older tablets, changing the pressure on the stylus nib or pressing a button changed the properties of the LC circuit, affecting the signal generated by the pen, which modern ones often encode into the signal as a digital data stream. By using electromagnetic signals, the tablet is able to sense the stylus position without the stylus having to even touch the surface, \& powering the pen with this signal means that devices used with the tablet never need batteries. Activslate 50, the model used with \href{https://en.wikipedia.org/wiki/Promethean_Ltd}{Promethean} white boards, also uses a hybrid of this technology.
	\item \textbf{Active tablets.} Active tablets differ in that the stylus used contains self-powered electronics that generate \& transmit a signal to the tablet. These styluses rely on an internal battery rather than the tablet for their power, resulting in a bulkier stylus. Eliminating the need to power the pen means that such tablets may listen for pen signals constantly, as they do not have to alternate between transmit \& receive modes, which can result in less jitter.
	\item \textbf{Optical tablets.} Optical tablets operate by a very small digital camera in the stylus \& then doing pattern matching on the image of the paper. The most successful example is the technology developed by \href{https://en.wikipedia.org/wiki/Anoto}{Anoto}.
	\item \textbf{Acoustic tablets.} Early models were described as spark tablets -- a small sound generator was mounted in the stylus, \& the acoustic signal picked up by 2 microphones placed near the writing surface. Some modern designs are able to read positions in 3 dimensions.
	\item \textbf{Capacitive tablets.} These tablets have also been designed to use an \href{https://en.wikipedia.org/wiki/Electrostatics}{electrostatic} or capacitive signal. Scriptel's designs are 1 example of a high-performance tablet detecting an electrostatic signal. Unlike the type of capacitive design used for \href{https://en.wikipedia.org/wiki/Touchscreen}{touchscreens}, the Scriptel design is able to detect the position of the pen while it is in proximity to or hovering above the tablet. Many multi-touch tablets use capacitive sensing.
\end{itemize}
For all these technologies, the tablet can use the received signal to also determine the distance of the stylus from the surface of the tablet, the tilt (angle from vertical) of the stylus, \& other information in addition to the horizontal \& vertical positions, such as clicking buttons of the stylus or the rotation of the stylus.

Compared to touchscreens, a graphic tablet generally offers much higher precision, the ability to track an object which is not touching the tablet, \& can gather much more information about the stylus, but is typically more expensive, \& can only be used with the special stylus or other accessories.

Some tablets, especially inexpensive ones aimed at young children, come with a corded stylus, using technology similar to older \href{https://en.wikipedia.org/wiki/RAND_Tablet}{RAND tablets}.'' -- \href{https://en.wikipedia.org/wiki/Graphics_tablet#Types}{Wikipedia\texttt{/}graphics tablet\texttt{/}types}

\subsection{Pucks}
\textsf{Fig. A large-format graphic tablet by manufacturer Summagraphics (OEM'd to Gerber): The puck's external copper coil can be clearly seen.}

``After styluses, pucks are the most commonly used tablet accessory. A puck is a mouse-like device that can detect its absolute position \& rotation. This is opposed to a \href{https://en.wikipedia.org/wiki/Mouse_(computing)}{mouse}, which can only sense its relative velocity on a surface (most tablet drivers are capable of allowing a puck to emulate a mouse in operation, \& many pucks are marketed as a ``mouse''). Pucks range in size \& shape; some are externally indistinguishable from a mouse, while others are a fairly large device with dozens of buttons \& controls. Professional pucks often have a \href{https://en.wikipedia.org/wiki/Reticle}{reticle} or \href{https://en.wikipedia.org/wiki/Loupe}{loupe} which allows the user to see the exact point on the tablet's surface targeted by the puck, for detailed tracing \& \href{https://en.wikipedia.org/wiki/Computer_aided_design}{computer aided design} (CAD) work.

Pucks are still used on the \href{https://en.wikipedia.org/wiki/Microsoft_Surface}{Microsoft Surface} range \& were recently used on the \href{https://en.wikipedia.org/wiki/Dell}{Dell} Canvas. However, they have been largely discontinued by most manufacturers in favor of physical \href{https://en.wikipedia.org/wiki/Keyboard_shortcut}{hotkeys} \& dials.'' -- \href{https://en.wikipedia.org/wiki/Graphics_tablet#Pucks}{Wikipedia\texttt{/}graphics tablet\texttt{/}pucks}

\subsection{Embedded LCD tablets}
``Some graphics tablets incorporate an \href{https://en.wikipedia.org/wiki/LCD}{LCD} into the tablet itself, allowing the user to draw directly on the display surface.

Graphic tablet\texttt{/}screen hybrids offer advantages over both standard PC \href{https://en.wikipedia.org/wiki/Touchscreen}{touchscreens} \& ordinary graphic tablets. Unlike touchscreens, they offer pressure sensitivity, \& their input resolution is generally higher. While their pressure sensitivity \& resolution are typically no better than those of ordinary tablets, they offer the additional advantage of directly seeing the location of the physical pen device relatively to the image on the screen. This often allows for increased accuracy \& a more tactile, ``real'' feeling to the use of the device.

The graphical tablet manufacturer \href{https://en.wikipedia.org/wiki/Wacom_(company)}{Wacom} holds many \href{https://en.wikipedia.org/wiki/Patents}{patents} on key technologies for graphic tablets, which forces competitors to use other technologies or license Wacom's patents. The displays are often sold for thousands of dollars. E.g., the \href{https://en.wikipedia.org/wiki/Wacom_Cintiq}{Wacom Cintiq} series ranges from just below US\$1,000 to over US\$2,000.

Some commercially available graphic tablet-screen hybrids include: \href{https://en.wikipedia.org/wiki/Monoprice}{Monoprice} 19-Inch Interactive Display, \href{https://en.wikipedia.org/wiki/Cintiq}{Cintiq} from \href{https://en.wikipedia.org/wiki/Wacom_(company)}{Wacom}, Kamvas (e.g. Kamvas Studio 22) from \href{https://en.wikipedia.org/wiki/Huion}{Huion}, \href{https://en.wikipedia.org/wiki/XP-PEN}{XP-PEN}, Parblo, Ugee, VEIKK. There have also been \href{https://en.wikipedia.org/wiki/Do-it-yourself}{do-it-yourself} projects where conventional used \href{https://en.wikipedia.org/wiki/LCD_monitor}{LCD monitors} \& graphics tablets have been converted to a graphics tablet-screen hybrid.'' -- \href{https://en.wikipedia.org/wiki/Graphics_tablet#Embedded_LCD_tablets}{Wikipedia\texttt{/}graphics tablet\texttt{/}embedded LCD tablets}

\subsection{Uses}
``Graphic tablets, because of their stylus-based interface \& ability to detect some or all of pressure, tilt, \& other attributes of the stylus \& its interaction with the tablet, are widely considered to offer a very natural way to create \href{https://en.wikipedia.org/wiki/Computer_graphics}{computer graphics}, especially 2D computer graphics. Indeed, many graphic packages can make use of the pressure (\&, sometimes, stylus tilt or rotation) information generated by a tablet, by modifying the brush size, shape, \href{https://en.wikipedia.org/wiki/Opacity_(optics)}{opacity}, \href{https://en.wikipedia.org/wiki/Color}{color}, or other attributes based on data received from the graphic tablet.

In \href{https://en.wikipedia.org/wiki/East_Asia}{East Asia}, graphic tablets, known as ``pen tablets'', are widely used in conjunction with input-method editor software (\href{https://en.wikipedia.org/wiki/Input_Method_Editor}{IMEs}) to write \href{https://en.wikipedia.org/wiki/Chinese_language}{Chinese}, \href{https://en.wikipedia.org/wiki/Japanese_language}{Japanese}, \& \href{https://en.wikipedia.org/wiki/Korean_language}{Korean} characters (\href{https://en.wikipedia.org/wiki/CJK}{CJK}). The technology is popular \& inexpensive \& offers a method for interacting with the computer in a more natural way than typing on the keyboard, with the pen tablet supplanting the role of the computer mouse. Uptake of \href{https://en.wikipedia.org/wiki/Handwriting_recognition}{handwriting recognition} among users who use alphabetic scripts has been slower.

Graphic tablets are commonly used in the \fbox{artistic world}. Using a pen-like stylus on a graphic tablet combined with a graphics-editing program, such as \href{https://en.wikipedia.org/wiki/Adobe_Illustrator}{Illustrator}, \href{https://en.wikipedia.org/wiki/Adobe_Photoshop}{Photoshop} by \href{https://en.wikipedia.org/wiki/Adobe_Systems}{Adobe Systems}, \href{https://en.wikipedia.org/wiki/Corel_Painter}{Corelpainter}, or \href{https://en.wikipedia.org/wiki/Krita}{Krita} gives artists a lot of precision when creating digital drawings or artwork. Photographers can also find working with a graphic tablet during their \href{https://en.wikipedia.org/wiki/Image_editing}{post processing} can really speed up tasks like creating a detailed layer mask or dodging \& burning.

\href{https://en.wikipedia.org/wiki/Educator}{Educators} make use of tablets in classrooms to project handwritten notes or lessons \& to allow students to do the same, as well as providing feedback on \href{https://en.wikipedia.org/wiki/Homework}{student work} submitted electronically. Online teachers may also use a tablet for marking student work, or for live \href{https://en.wikipedia.org/wiki/Tutorial}{tutorials} or lessons, especially where complex visual information or \href{https://en.wikipedia.org/wiki/Mathematical_equation}{mathematical equations} are required. \href{https://en.wikipedia.org/wiki/Student}{Students} are also increasingly using them as \href{https://en.wikipedia.org/wiki/Note-taking}{note-taking} devices, especially during university \href{https://en.wikipedia.org/wiki/Lecture}{lectures} while following along with the \href{https://en.wikipedia.org/wiki/Lecturer}{lecturer}. They facilitate smooth online teaching process \& are popularly used along with face-cam to mimic classroom experience.

Tablets are also popular for \href{https://en.wikipedia.org/wiki/Technical_drawing}{technilca drawings} \& \href{https://en.wikipedia.org/wiki/Computer-aided_design}{CAD}, as one can typically put a piece of paper on them without interfering with their function.

Finally, tablets are gaining popularity as a replacement for the \href{https://en.wikipedia.org/wiki/Computer_mouse}{computer mouse} as a pointing device. They can feel more intuitive to some users than a mouse, as the position of a pen on a tablet typically corresponds to the location of the pointer on the \href{https://en.wikipedia.org/wiki/GUI}{GUI} shown on the computer screen. Those artists using a pen for graphic work may, as a matter of convenience, use a tablet \& pen for standard computer operations rather than put down the pen \& find a mouse. Popular \href{https://en.wikipedia.org/wiki/Rhythm_game}{rhythm game} \href{https://en.wikipedia.org/wiki/Osu!}{osu!} allows utilizing a tablet as a way of playing.

Graphic tablets are available in various sizes \& price ranges; \href{https://en.wikipedia.org/wiki/ISO_216}{A6}-sized tablets being relatively inexpensive \& \href{https://en.wikipedia.org/wiki/ISO_216}{A3}-sized tablets far more expensive. Modern tablets usually connect the the computer via a \href{https://en.wikipedia.org/wiki/Universal_Serial_Bus}{USB} or \href{https://en.wikipedia.org/wiki/HDMI}{HDMI} interface.'' -- \href{https://en.wikipedia.org/wiki/Graphics_tablet#Uses}{Wikipedia\texttt{/}graphics tablet\texttt{/}uses}

\subsection{Similar devices}
``\href{https://en.wikipedia.org/wiki/Interactive_whiteboard}{Interactive whiteboards} offer high-resolution wall size graphic tablets up to 95'' (241,3 cm) along with options for pressure \& multiple input. These are becoming commonplace in schools \& meeting rooms around the world.

Earlier \href{https://en.wikipedia.org/wiki/Resistive_touch_screen}{resistive touch screen} devices (like \href{https://en.wikipedia.org/wiki/Personal_digital_assistant}{PDAs}, early \href{https://en.wikipedia.org/wiki/Smartphone}{smartphones}, \& \href{https://en.wikipedia.org/wiki/Tablet_PC}{tablet PCs}) were typically equipped with styluses, but accuracy of stylus input was very limited.

The more modern \href{https://en.wikipedia.org/wiki/Capacitive_touch_screen}{capacitive touch screens} such as those found on some \href{https://en.wikipedia.org/wiki/Table_computer}{table computers}, \href{https://en.wikipedia.org/wiki/Tablet_computer}{tablet computers}, \href{https://en.wikipedia.org/wiki/Laptop}{laptops} \& the \href{https://en.wikipedia.org/wiki/Nintendo_DS}{Nitendo DS} operate in similar ways, but they usually use either optical grids or a pressure-sensitive film instead so do not need a special pointing device. Some of the latest models with capacitive input can be equipped with specialized styluses, \& then these input devices can be used similar to full-function graphics tablet.

A graphic tablet is also used for Audio-\href{https://en.wikipedia.org/wiki/Haptic_technology}{Haptic} products where blind or visually impaired people touch swelled graphics on a graphic tablet \& get audio feedback from that. The product that is using this technology is called \href{https://en.wikipedia.org/wiki/Tactile_Talking_Tablet}{Tactile Talking Tablet} or T3.'' -- \href{https://en.wikipedia.org/wiki/Graphics_tablet#Similar_devices}{Wikipedia\texttt{/}graphics tablet\texttt{/}similar devices}

%------------------------------------------------------------------------------%

\chapter{\href{https://www.creativebloq.com/}{Creative Blog}: Art \& Design Inspiration}

\href{https://www.creativebloq.com/features/best-drawing-tablet}{Creative Blog\texttt{/}the best drawing tablets in 2022: our pick of the best graphics tablets}.

\section{\href{https://www.creativebloq.com/reviews/xencelabs-pen-tablet-medium-bundle}{Creative Blog\texttt{/}Xencelabs Pen Tablet Medium review}}
``Xencelabs' pen tablet takes on Wacom with its expert product design and competitive pricing.''

\textbf{Verdict.} ``This is an ideal pen tablet for illustrators, digital painters \& photographers who want a reliable, solidly built piece of equipment. It's easily portable, wireless \& pretty much flawless in performance.'' ``Xencelabs are the new kids on the block in the graphics tablet world. The team includes ex-Wacom employees, \& they've put all their design \& industry know-how into the Pen Display Medium -- it's as user- \& artist-friendly as you'd hope. We tried the Pen Display Medium Bundle, which comes with the Xencelabs Pen Tablet, 2 pens (the 3 button \& Thin Pen) \& a case, a Quick Key remote, \& other bits like a drawing glove \& a very nice soft tablet carrying case, all for just \$359.99\texttt{/}\pounds319.99.''

\subsection{Xencelab Pen Tablet: Design \& Build}
``Xencelabs Pen Display Medium sits just under a 13-inch MacBook in size, neither too big nor too small, \& at only 8mm thick it looks \& feels like a very nice piece of design. It has an ergonomic curved front edge, which feels like the user's comfort has been taken into account. This is especially welcome considering those long hours spent drawing.

The tablet seems super sleek \& well made -- from the texture, or `tooth', of the active drawing zone, to the metal alloy underside with 6 rubber non-slip pads, it all adds to a build quality you might expect from a more premium piece of kit by someone like Wacom -- an Intuos Pro medium would be a fair comparison.

The active area is delineated by 4 corner LEDs you can customize in 8 different colors. It's a nice addition. You can set different colors for different programmes, e.g., when they light up blue, you know you set for Photoshop, pink for Affinity, yellow for Corel, etc., or whatever you like.

3 small buttons sit at the top of tablet, fully customizable of course. They seem perfect for accessing the tablet or pen settings, e.g., or for switching programmes. You may well ask -- where are the rest of the shortcut buttons, so common on graphics tablets? \& that's where the Quick Key Remote comes in $\ldots$''

\subsection{Xencelabs Pen Tablet: Quick Key Remote}
``Because Xencelabs seems to be all about the refinement \& attention to detail on a single product (meaning the bundle as a whole), those quick key shortcuts have been lifted off the tablet, save the 3 primary buttons, \& configured, if you will, into this very cool, separate,  OLED display \& 9-button remote with physical wheel dial.

The remote itself is as nice a build as the tablet \& is customizable with up to 40 shortcuts. The best thing about it is its ability to configure by color too, just like the tablet itself. A bit of time spent in the settings means a relatively easy configuration of dial sequences, which is perfect for scrolling, zooming, rotating, \& changing brush sizes. Just assign a color to each job.

You can configure the 8 buttons into different sets for different art jobs, so to speak -- Set A for sketching, B for editing, C for coloring etc. The OLED display makes it easier to remember which button is set to what, when in each configuration. If that all sounds a bit like too much trouble -- it really isn't. Compared to other drawing tablets, this, as well as the initial set up \& driver installation, is pretty easy \& stress free. Xencelabs has got the user's experience from beginning to end in mind, \& it shows.''

\subsection{Xencelabs Pen Tablet: Stylus\texttt{/}Pen}
`Opening up \& unboxing the Xencelabs pen tablet medium bundle is pretty satisfying, even surprising, especially considering the cost. It's when you uncover the pen case that it really gets a bit `ooh la la'. The pens also have a nice weight -- reminiscent of an old fountain pen.

Wacom sells its slim pen separately, but Xencelabs includes one with the Pen Tablet medium. So, you get a regular 3-button pen \& a slim 2-button pen, both with `erasers' on the opposite ends (or whatever you want to configure them as). I spent most of my time using the slim pen \& not even bothering with the regular, but it is nice to have a choice.

Being able to customize both pens to different settings is a nice touch -- you can grab one for shading or light pencil work, \& use the other for inking or painting in, e.g. It saves a great deal of time having to go back into the settings \& reconfigure each time you want something different. Both pens have the standard (\& very high) levels of pressure sensitivity, \& 60 degree tilt functions, as standard. The drawing experience with the pens is really very good -- no lag, smooth lines, \& no marks missing when sketching at speed.

10 extra nibs are included in the case, 4 of them felt (as in the material) for extra `tooth', plus a nib extracting ring.''

\subsection{Xencelabs Pen Tablet: Power}
``The pen tablet \& Quick Key Remote both link up with to your computer via USB (USB-A to USB-C connector included) \& once charged, are both connected to the tablet wirelessly via bluetooth -- with a dongle included in pen case. The battery of both the tablet \& remote lasted a fair few hours. The pen takes less than an hour to fully recharge (whilst using it), \& the remote about half an hour. The pens are of course completely battery free.''

\subsection{Xencelabs Tablet: Price}
``The cost of the bundle is very reasonable: \$359.99\texttt{/}\pounds319.99. This includes the Xencelabs Pen Tablet, 2 pens (the 3 button \& Thin Pne) \& case, the Quick Key remote, a drawing glove \& a very nice soft tablet carrying case.

The Tablet alone is \$279.99\texttt{/}\pounds259.99, the remote \$89.99\texttt{/}\pounds79.99, \& the regular \& thin pen \$49.99\texttt{/}\pounds35.99 \& \$46.99\texttt{/}\pounds33.99 respectively. The bundle is clearly a bargain. It is significantly more than an \href{https://www.creativebloq.com/reviews/xp-pen-deco-pro}{XP-Pen Deco Pro} at \$129.99, but worth the hike in price due to the design build, quality \& expertise that has gone into it.

It sits around the same price as a \href{https://www.creativebloq.com/reviews/wacom-intuos-pro-review}{Wacom Intuos Pro Medium} \$379.95\texttt{/}\pounds329. Both the XP-PEN Deco Pro \& the Wacom Intuos Pro Medium are comparable to the Xencelabs Pen Tablet Medium in terms of their technical prowess.''

\subsection{Xencelabs Tablet: Should You Buy It?}
``Between the Deco Pro \& the Intuos Medium \& this tablet, Xencelabs comes out on top. The Xencelabs Pen Display Medium bundle is a clear signal that the Xencelabs team are serious competitors in this market, offering a capable tablet \& accessories set that is understated yet very cool, for a sensible price. Whether you're a professional or are just starting out, we recommend Xencelabs Pen Display Medium Bundle.'' -- \href{https://www.creativebloq.com/author/ben-brady}{Ben Brady}, Jul 23, 2021

%------------------------------------------------------------------------------%

\chapter{\href{https://www.dpreview.com/}{Digital Photography Review} [DPReview]}

\section{\href{https://www.dpreview.com/reviews/xencelabs-pen-tablet-small-review-the-more-affordable-rival-to-wacom-intuos-pro}{DPReview\texttt{/}Xencelabs Pen Tablet Small Review: The More Affordable Rival to Wacom's Intuos Pro}}
``\href{https://www.xencelabs.com/product/xencelabs-pen-tablet-small/}{Xencelabs\texttt{/}Xencelabs Pen Tablet Small} \$199.99.''

\textbf{Key Takeaways.}
\begin{itemize}
	\item ``Slightly smaller, lighter \& more affordable than the \href{https://www.wacom.com/en-us/products/pen-tablets/wacom-intuos-pro#Use}{Wacom Intuos Pro S}
	\item Works tethered or wirelessly with a wide range of Windows, MacOS, Linux \& Android apps from a built-in battery
	\item Battery-free pens that match the Intuos Pro's resolution, pressure sensitivity \& tilt detection
	\item Lacks Intuos Pro's ability to double as a huge multi-touch capable touchpad
	\item Fewer controls on the tablet than on Wacom rivals
	\item Includes 2 digital pens in different sizes\texttt{/}button counts, whereas Wacom only gives you 1 \& charges a premium for extras
	\item Includes nicely-made travel cases, handy if you edit on the road''
\end{itemize}
``Since the pens draw their power from an electromagnetic field created by the tablet itself, there's also no need for batteries in each pen. That saves you from replacing them at regular intervals \& makes the pens lighter \& less tiring to use.'' -- \href{https://www.dpreview.com/members/6245232230}{Mike Tomkins}, published Dec 13, 2021

%------------------------------------------------------------------------------%

\chapter{\href{https://justcreative.com/}{Just Creative}}

\section{\href{https://justcreative.com/xencelabs-pen-tablet-medium-bundle-se-review/}{Just Creative\texttt{/}Xencelabs Pen Tablet -- Medium Bundle SE Review}}
``

'' -- \href{https://justcreative.com/author/jacob-cass/}{Jacob Cass}, May 23, 2022


%------------------------------------------------------------------------------%


\chapter{\href{https://www.xencelabs.com}{Xencelabs}}



%------------------------------------------------------------------------------%

\selectlanguage{english}
\begin{thebibliography}{99}
	\bibitem[]{}
\end{thebibliography}

%------------------------------------------------------------------------------%

\printbibliography[heading=bibintoc]
	
\end{document}