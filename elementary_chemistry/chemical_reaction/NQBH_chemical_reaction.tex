\documentclass{article}
\usepackage[backend=biber,natbib=true,style=authoryear]{biblatex}
\addbibresource{/home/nqbh/reference/bib.bib}
\usepackage[utf8]{vietnam}
\usepackage{tocloft}
\renewcommand{\cftsecleader}{\cftdotfill{\cftdotsep}}
\usepackage[colorlinks=true,linkcolor=blue,urlcolor=red,citecolor=magenta]{hyperref}
\usepackage{amsmath,amssymb,amsthm,mathtools,float,graphicx,algpseudocode,algorithm,tcolorbox,tikz,tkz-tab,subcaption}
\DeclareMathOperator{\arccot}{arccot}
\usepackage[inline]{enumitem}
\usepackage[version=4]{mhchem}
\allowdisplaybreaks
\numberwithin{equation}{section}
\newtheorem{assumption}{Assumption}[section]
\newtheorem{nhanxet}{Nhận xét}[section]
\newtheorem{conjecture}{Conjecture}[section]
\newtheorem{corollary}{Corollary}[section]
\newtheorem{hequa}{Hệ quả}[section]
\newtheorem{definition}{Definition}[section]
\newtheorem{dinhnghia}{Định nghĩa}[section]
\newtheorem{example}{Example}[section]
\newtheorem{vidu}{Ví dụ}[section]
\newtheorem{lemma}{Lemma}[section]
\newtheorem{notation}{Notation}[section]
\newtheorem{principle}{Principle}[section]
\newtheorem{problem}{Problem}[section]
\newtheorem{baitoan}{Bài toán}[section]
\newtheorem{proposition}{Proposition}[section]
\newtheorem{menhde}{Mệnh đề}[section]
\newtheorem{question}{Question}[section]
\newtheorem{cauhoi}{Câu hỏi}[section]
\newtheorem{quytac}{Quy tắc}
\newtheorem{remark}{Remark}[section]
\newtheorem{luuy}{Lưu ý}[section]
\newtheorem{theorem}{Theorem}[section]
\newtheorem{tiende}{Tiên đề}[section]
\newtheorem{dinhly}{Định lý}[section]
\usepackage[left=0.5in,right=0.5in,top=1.5cm,bottom=1.5cm]{geometry}
\usepackage{fancyhdr}
\pagestyle{fancy}
\fancyhf{}
\lhead{\small Subsect.~\thesubsection}
\rhead{\small\nouppercase{\leftmark}}
\renewcommand{\subsectionmark}[1]{\markboth{#1}{}}
\cfoot{\thepage}
\def\labelitemii{$\circ$}

\title{Chemical Reaction -- Phản Ứng Hóa Học}
\author{Nguyễn Quản Bá Hồng\footnote{Independent Researcher, Ben Tre City, Vietnam\\e-mail: \texttt{nguyenquanbahong@gmail.com}; website: \url{https://nqbh.github.io}.}}
\date{\today}

\begin{document}
\maketitle
\begin{abstract}
	\textsc{[en]} This text is a collection of problems, from easy to advanced, about chemical reaction. This text is also a supplementary material for my lecture note on Elementary Chemistry grade 7--8, which is stored \& downloadable at the following link: \href{https://github.com/NQBH/hobby/blob/master/elementary_chemistry/grade_8/NQBH_elementary_chemistry_grade_8.pdf}{GitHub\texttt{/}NQBH\texttt{/}hobby\texttt{/}elementary chemistry\texttt{/}grade 8\texttt{/}lecture}\footnote{\textsc{url}: \url{https://github.com/NQBH/hobby/blob/master/elementary_chemistry/grade_8/NQBH_elementary_chemistry_grade_8.pdf}.}. The latest version of this text has been stored \& downloadable at the following link: \href{https://github.com/NQBH/hobby/blob/master/elementary_chemistry/grade_8/real/NQBH_real.pdf}{GitHub\texttt{/}NQBH\texttt{/}hobby\texttt{/}elementary chemistry\texttt{/}grade 8\texttt{/}chemical reaction}\footnote{\textsc{url}: \url{https://github.com/NQBH/hobby/blob/master/elementary_chemistry/grade_8/real/NQBH_real.pdf}.}.
	\vspace{2mm}
	
	\textsc{[vi]} Tài liệu này là 1 bộ sưu tập các bài tập chọn lọc từ cơ bản đến nâng cao về phản ứng hóa học. Tài liệu này là phần bài tập bổ sung cho tài liệu chính -- bài giảng \href{https://github.com/NQBH/hobby/blob/master/elementary_chemistry/grade_8/NQBH_elementary_chemistry_grade_8.pdf}{GitHub\texttt{/}NQBH\texttt{/}hobby\texttt{/}elementary chemistry\texttt{/}grade 8\texttt{/}lecture} của tác giả viết cho Hóa Học Sơ Cấp lớp 8. Phiên bản mới nhất của tài liệu này được lưu trữ \& có thể tải xuống ở link sau: \href{https://github.com/NQBH/hobby/blob/master/elementary_chemistry/grade_8/real/NQBH_real.pdf}{GitHub\texttt{/}NQBH\texttt{/}hobby\texttt{/}elementary chemistry\texttt{/}grade 8\texttt{/}chemical reaction}.
\end{abstract}
\setcounter{secnumdepth}{4}
\setcounter{tocdepth}{3}
\tableofcontents

%------------------------------------------------------------------------------%

\section{Sự Biến Đổi Chất}

%------------------------------------------------------------------------------%

\section{Phản Ứng Hóa Học}
``\begin{enumerate*}
	\item[\textbf{1.}] \textbf{Phân biệt hiện tượng vật lý \& hiện tượng hóa học}:
	\begin{enumerate*}
		\item[$\bullet$] \textit{Hiện tượng vật lý}: Khi chất đổi về thể hay hình dạng. Không có chất mới nào sinh ra.
		\item[$\bullet$] \textit{Hiện tượng hóa học}: Khi có sự biến đổi từ chất này thành chất khác. Hiện tượng trong đó có sinh ra chất mới.
	\end{enumerate*}
	\item[\textbf{2.}] \textit{Phản ứng hóa học} là quá trình làm biến đổi chất này thành chất khác. Trong phản ứng hóa học, liên kết giữa các nguyên tử thay đổi. Các phản ứng hóa học có thể xảy ra: $A + B\to C + D$, $A + B\to C$, $A\to C + D$.
	\item[\textbf{3.}] \textbf{$2$ định luật hóa học cơ bản.}
	\begin{enumerate*}
		\item[$\bullet$] \textit{Định luật thành phần không đổi}: 1 hợp chất, dù điều chế bằng bất kỳ cách nào, cũng luôn có thành phần không đổi về khối lượng. Ứng dụng: Dựa vào tỷ lệ khối lượng giữa các nguyên tố cấu tạo nên 1 chất là không đổi $\to$ tỷ số nguyên tử không đổi $\to$ lập công thức hóa học của chất đó.
		\item[$\bullet$] \textit{Định luật bảo toàn khối lượng}: Các chất tham gia $\to$ Các chất tạo thành. Tổng khối lượng chất tham gia $=$ Tổng khối lượng chất tạo thành. Ứng dụng: Tính khối lượng của các chất tham gia phản ứng hay chất tạo thành sau phản ứng.
	\end{enumerate*}
	\item[\textbf{4.}] Phương trình hóa học cho biết công thức hóa học các chất phản ứng \& chất mới sinh ra trong phản ứng hóa học, cho biết tỷ lệ số phân tử chất phản ứng \& chất mới sinh ra trong phản ứng hóa học. Lưu ý khi lập phương trình hóa học:
	\begin{enumerate*}
		\item[$\bullet$] Viết đúng công thức hóa học của các chất phản ứng \& chất mới sinh ra.
		\item[$\bullet$] Chọn hệ số phân tử sao cho số nguyên tử của mỗi nguyên tố ở 2 vế đều bằng nhau. Cách làm:
		\begin{enumerate*}
			\item[$\circ$] Nên bắt đầu từ những nguyên tố mà số nguyên tử có nhiều \& không bằng nhau.
			\item[$\circ$] Trường hợp số nguyên tử của 1 nguyên tố ở vế này là số chẵn \& ở vế kia là số lẻ thì trước hết phải đặt hệ số $2$ cho chất mà số nguyên tử là số lẻ, rồi tiếp tục đặt hệ số cho phân tử chứa số nguyên tử chẵn ở vế còn lại sao cho số nguyên tử của nguyên tố này ở 2 vế bằng nhau.
		\end{enumerate*}
	\end{enumerate*}
	Trong quá trình cân bằng không được thay đổi các chỉ số nguyên tử trong các công thức hóa học.
	\item[\textbf{5.}] \textit{Tính hiệu suất phản ứng}: Thực tế do 1 số nguyên nhân chất tham gia phản ứng không tác dụng hết, i.e., hiệu suất $< 100$\%, Người ta có thể tính hiệu suất phản ứng như sau:
	\begin{enumerate*}
		\item[(a)] Dựa vào 1 trong các chất tham gia phản ứng: Công thức tính: $H\% = \frac{\mbox{lượng thực tế đã phản ứng}}{\mbox{lượng tổng số đã lấy}}\cdot100\%$.
		\item[(b)] Dựa vào 1 trong các chất tạo thành: Công thức tính: $H\% = \frac{\mbox{lượng thực tế thu được}}{\mbox{lượng thu theo lý thuyết (theo phương trình phản ứng)}}\cdot100\%$.
		\item[(c)] Bài toán hiệu suất còn mở rộng ra: Cho hiệu suất phản ứng rồi tính lượng chất tham gia hoặc tạo thành.'' -- \cite[Chap. 2, pp. 44--46]{An_400_BT_Hoa_Hoc_8_2020}
	\end{enumerate*}
\end{enumerate*}

\begin{baitoan}[\cite{An_400_BT_Hoa_Hoc_8_2020}, p. 45]
	Lập phương trình hóa học của phản ứng có sơ đồ sau: \emph{\ce{Al + O2 -> Al2O3}}.\hfill\textsf{Ans:} $4,3,2$.
\end{baitoan}

\begin{baitoan}[\cite{An_400_BT_Hoa_Hoc_8_2020}, p. 45]
	Lập phương trình hóa học của phản ứng có sơ đồ sau: \emph{\ce{Al + H2SO4 -> Al2(SO4)3 + H2}}.\\\mbox{}\hfill\textsf{Ans:} $2,3,1,3$.
\end{baitoan}







%------------------------------------------------------------------------------%

\section{Định Luật Bảo Toàn Khối Lượng}

%------------------------------------------------------------------------------%

\section{Phương Trình Hóa Học}

%------------------------------------------------------------------------------%

\printbibliography[heading=bibintoc]
	
\end{document}