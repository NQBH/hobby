\documentclass{article}
\usepackage[backend=biber,natbib=true,style=authoryear]{biblatex}
\addbibresource{/home/nqbh/reference/bib.bib}
\usepackage[utf8]{vietnam}
\usepackage{tocloft}
\renewcommand{\cftsecleader}{\cftdotfill{\cftdotsep}}
\usepackage[colorlinks=true,linkcolor=blue,urlcolor=red,citecolor=magenta]{hyperref}
\usepackage{amsmath,amssymb,amsthm,mathtools,float,graphicx,algpseudocode,algorithm,tcolorbox,tikz,tkz-tab,subcaption}
\DeclareMathOperator{\arccot}{arccot}
\usepackage[inline]{enumitem}
\usepackage[version=4]{mhchem}
\allowdisplaybreaks
\numberwithin{equation}{section}
\newtheorem{assumption}{Assumption}[section]
\newtheorem{nhanxet}{Nhận xét}[section]
\newtheorem{conjecture}{Conjecture}[section]
\newtheorem{corollary}{Corollary}[section]
\newtheorem{hequa}{Hệ quả}[section]
\newtheorem{definition}{Definition}[section]
\newtheorem{dinhnghia}{Định nghĩa}[section]
\newtheorem{example}{Example}[section]
\newtheorem{vidu}{Ví dụ}[section]
\newtheorem{lemma}{Lemma}[section]
\newtheorem{notation}{Notation}[section]
\newtheorem{principle}{Principle}[section]
\newtheorem{problem}{Problem}[section]
\newtheorem{baitoan}{Bài toán}
\newtheorem{proposition}{Proposition}[section]
\newtheorem{menhde}{Mệnh đề}[section]
\newtheorem{question}{Question}[section]
\newtheorem{cauhoi}{Câu hỏi}[section]
\newtheorem{quytac}{Quy tắc}
\newtheorem{remark}{Remark}[section]
\newtheorem{luuy}{Lưu ý}[section]
\newtheorem{theorem}{Theorem}[section]
\newtheorem{tiende}{Tiên đề}[section]
\newtheorem{dinhly}{Định lý}[section]
\usepackage[left=0.5in,right=0.5in,top=1.5cm,bottom=1.5cm]{geometry}
\usepackage{fancyhdr}
\pagestyle{fancy}
\fancyhf{}
\lhead{\small Sect.~\thesection}
\rhead{\small\nouppercase{\leftmark}}
\renewcommand{\subsectionmark}[1]{\markboth{#1}{}}
\cfoot{\thepage}
\def\labelitemii{$\circ$}

\title{Chemical Reaction -- Phản Ứng Hóa Học}
\author{Nguyễn Quản Bá Hồng\footnote{Independent Researcher, Ben Tre City, Vietnam\\e-mail: \texttt{nguyenquanbahong@gmail.com}; website: \url{https://nqbh.github.io}.}}
\date{\today}

\begin{document}
\maketitle
\begin{abstract}
	\textsc{[en]} This text is a collection of problems, from easy to advanced, about chemical reaction. This text is also a supplementary material for my lecture note on Elementary Chemistry grade 7--8, which is stored \& downloadable at the following link: \href{https://github.com/NQBH/hobby/blob/master/elementary_chemistry/grade_8/NQBH_elementary_chemistry_grade_8.pdf}{GitHub\texttt{/}NQBH\texttt{/}hobby\texttt{/}elementary chemistry\texttt{/}grade 8\texttt{/}lecture}\footnote{\textsc{url}: \url{https://github.com/NQBH/hobby/blob/master/elementary_chemistry/grade_8/NQBH_elementary_chemistry_grade_8.pdf}.}. The latest version of this text has been stored \& downloadable at the following link: \href{https://github.com/NQBH/hobby/blob/master/elementary_chemistry/chemical_reaction/NQBH_chemical_reaction.pdf}{GitHub\texttt{/}NQBH\texttt{/}hobby\texttt{/}elementary chemistry\texttt{/}grade 8\texttt{/}chemical reaction}\footnote{\textsc{url}: \url{https://github.com/NQBH/hobby/blob/master/elementary_chemistry/chemical_reaction/NQBH_chemical_reaction.pdf}.}.
	\vspace{2mm}
	
	\textsc{[vi]} Tài liệu này là 1 bộ sưu tập các bài tập chọn lọc từ cơ bản đến nâng cao về phản ứng hóa học. Tài liệu này là phần bài tập bổ sung cho tài liệu chính -- bài giảng \href{https://github.com/NQBH/hobby/blob/master/elementary_chemistry/grade_8/NQBH_elementary_chemistry_grade_8.pdf}{GitHub\texttt{/}NQBH\texttt{/}hobby\texttt{/}elementary chemistry\texttt{/}grade 8\texttt{/}lecture} của tác giả viết cho Hóa Học Sơ Cấp lớp 8. Phiên bản mới nhất của tài liệu này được lưu trữ \& có thể tải xuống ở link sau: \href{https://github.com/NQBH/hobby/blob/master/elementary_chemistry/grade_8/real/NQBH_real.pdf}{GitHub\texttt{/}NQBH\texttt{/}hobby\texttt{/}elementary chemistry\texttt{/}grade 8\texttt{/}chemical reaction}.
\end{abstract}
\setcounter{secnumdepth}{4}
\setcounter{tocdepth}{3}
\tableofcontents

%------------------------------------------------------------------------------%

\section{Sự Biến Đổi Chất}
``\begin{enumerate*}
	\item[\textbf{1.}] \textit{Sự biến đổi vật lý} (còn gọi là \textit{hiện tượng vật lý}) là sự biến đổi về hình dạng hay về trạng thái của chất (chất không thay đổi). E.g.: Nghiền đường kính thành bột mịn (sự biến đổi về hình dạng). Đun nước, nước lỏng chuyển thành hơi nước. Làm lạnh, hơi nước lại ngưng tụ thành nước lỏng, làm lạnh tiếp đến $0^\circ$C, nước lỏng lại chuyển thành nước rắn, i.e., nước đá (sự biến đổi về trạng thái).
	\item[\textbf{2.}] \textit{Sự biến đổi hóa học} (còn gọi là \textit{hiện tượng hóa học}) là sự biến đổi chất này thành chất khác. E.g.: Rượu (mùi thơm, vị cay) lên men thành giấm (mùi giấm, vị chua). Đốt cháy tờ giấy, giấy biến thành tro \& khí carbon dioxide \ce{CO2} (còn gọi là khí carbonic).'' -- \cite[Chap. 2, p. 32]{Truong2021}
\end{enumerate*}

\begin{baitoan}[\cite{Truong2021}, \textbf{II.1.}, p. 33]
	Quan sát hiện tương:
	\begin{enumerate*}
		\item[(a)] Lưu huỳnh cháy tạo thành khí sunfurơ \emph{\ce{SO2}}.
		\item[(b)] Nước đá tan thành nước lỏng.
		\item[(c)] Sắt bị gỉ chuyển thành 1 chất màu đỏ.
		\item[(d)] Thủy tinh nóng chảy.
	\end{enumerate*}
	Cho biết đâu là hiện tượng vật lý, đâu là hiện tượng hóa học.
\end{baitoan}

\begin{baitoan}[\cite{Truong2021}, \textbf{II.2.}, p. 33]
	Xét các hiện tượng sau đây \& chỉ rõ đâu là hiện tượng vật lý, đâu là hiện tượng hóa học.
	\begin{enumerate*}
		\item[(a)] Cồn để trong lọ không kín bị bay hơi.
		\item[(b)] Khi đốt đèn cồn, cồn cháy biến đổi thành khí carbonic \& hơi nước.
		\item[(c)] Dây tóc trong bóng đèn điện nóng \& sáng lên khi dòng điện chạy qua.
		\item[(d)] Nhựa đường được đung nóng, chảy lỏng.
	\end{enumerate*}
\end{baitoan}

\begin{baitoan}[\cite{Truong2021}, \textbf{II.3.}, p. 33]
	Những việc làm nào sau đây là sự biến đổi vật lý, sự biến đổi hóa học?
	\begin{enumerate*}
		\item[(a)] Giũa 1 đinh sắt thành mạt sắt. Ngâm mạt sắt trong ống nghiệm đựng acid hydrochloric, thu được sắt clorua \& khí hydro.
		\item[(b)] Cho 1 ít đường vào ống nghiệm đựng nước, khuấy cho đường tan hết ta được nước đường. Đun sôi nước đường trên ngọn lửa đèn cồn, nước bay hơi hết, tiếp tục đung ta được chất rắn màu đen \& có chất khí thoát ra, khí này làm đục nước vôi trong.
	\end{enumerate*}
\end{baitoan}

\begin{baitoan}[\cite{Truong2021}, \textbf{II.4.}, p. 33]
	Xét các thí nghiệm sau với chất rắn natri hiđrocacbonat \emph{\ce{NaHCO3}} (còn gọi là \emph{thuốc muối}) như sau:
	\begin{enumerate*}
		\item[(a)] Hòa tan 1 ít bột \emph{\ce{NaHCO3}} vào nước được dung dịch trong suốt.
		\item[(b)] Hòa tan 1 ít bột \emph{\ce{NaHCO3}} vào nước chanh hoặc giấm thấy sủi bọt.
		\item[(c)] Đun nóng 1 ít bột \emph{\ce{NaHCO3}} trong ống nghiệm, màu trắng không đổi nhưng thoát ra 1 chất khí có thể làm đục nước vôi trong.
	\end{enumerate*}
	Cho biết trong những thí nghiệm trên, đâu là sự biến đổi vật lý, đâu là sự biến đổi hóa học? Giải thích.
\end{baitoan}

\begin{baitoan}[\cite{Truong2021}, \textbf{II.5.}, p. 34]
	Nến được làm bằng parafin. Khi đốt nến, lúc đầu paràin chảy lỏng \& thấm vào bấc, sau đó chuyển thành hơi parafin, hơi cháy biến thành khí carbon dioxide \& hơi nước. Cho biết sự biến đổi vật lý \& sự biến đổi hóa học trong việc đốt nến.
\end{baitoan}

\begin{baitoan}[\cite{Truong2021}, \textbf{II.6.}, p. 34]
	Đập nhỏ đá vôi rồi xếp vào lò nung ở nhiệt độ khoảng $1000^\circ$ ta được vôi sống \& có khí carbon dioxide thoát ra từ miệng lò. Cho vôi sống vào nước ta được vôi tôi. Đâu là sự biến đổi vật lý? Sự biến đổi hóa học?
\end{baitoan}

\begin{baitoan}[\cite{Truong2021}, \textbf{II.7.}, p. 34]
	Khi quan sát 1 hiện tượng, dựa vào đâu ta có thể dự đoán được đó là hiện tượng hóa học?
\end{baitoan}

\begin{baitoan}[\cite{An_400_BT_Hoa_Hoc_8_2020}, \textbf{75.}, p. 46]
	Xét các hiện tượng sau đây, hiện tượng nào là hiện tượng vật lý, hiện tượng hóa học?
	\begin{enumerate*}
		\item[(a)] Cho vôi sống (\ce{CaO}) hòa tan vào nước.
		\item[(b)] Đinh sắt để trong không khí bị gỉ.
		\item[(c)] Cồn để trong lọ không kín bị bay hơi.
		\item[(d)] Dây tóc trong bóng đèn điện nóng \& sáng lên khi dòng điện chạy qua.
	\end{enumerate*}
\end{baitoan}

\begin{baitoan}[\cite{An_400_BT_Hoa_Hoc_8_2020}, \textbf{76.}, p. 46]
	Khi chiên mỡ có sự biến đổi như sau: trước hết 1 phần mỡ bị chảy lỏng \& nếu tiếp tục đun quá lửa mỡ sẽ khét. Trong 2 giai đoạn trên, giai đoạn nào có sự biến đổi hóa học? Giải thích.
\end{baitoan}

\begin{baitoan}[\cite{An_400_BT_Hoa_Hoc_8_2020}, \textbf{77.}, p. 46]
	Trong phòng thí nghiệm có 1 em học sinh làm $2$ thí nghiệm sau:
	\begin{enumerate*}
		\item[(a)] Đốt cháy 1 băng magie cháy thành ngọn lửa sáng.
		\item[(b)] Đun đường trong 1 ống thử, mới đầu đường nóng chảy, sau đó ngả màu nâu, rồi đen đi.
	\end{enumerate*}
	Giải thích xem thí nghiệm trên có sự biến đổi hóa học không? Vì sao?
\end{baitoan}

\begin{baitoan}[\cite{An_400_BT_Hoa_Hoc_8_2020}, \textbf{78.}, p. 47]
	\begin{enumerate*}
		\item[(a)] Về mùa hè thức ăn thường bị thiu, ôi. Đó có phải là sự biến đổi hóa học không?
		\item[(b)] Trong các hiện tượng sau, hiện tượng nào là hiện tượng hóa học: trứng bị thối, mực hòa tan vào nước, tẩy màu vải xanh thành trắng?
	\end{enumerate*}
\end{baitoan}

\begin{baitoan}[\cite{An_400_BT_Hoa_Hoc_8_2020}, \textbf{79.}, p. 47]
\begin{enumerate*}
	\item[(a)] Khi đánh diêm có lửa bắt cháy. Hiện tượng đó là hiện tượng gì?
	\item[(b)] Rượu để hở lâu ngày trong không khí thường bị chua. Có thể xem hiện tượng trên là sự biến đổi hóa học không? Vì sao?
	\end{enumerate*}
\end{baitoan}

\begin{baitoan}[\cite{An_400_BT_Hoa_Hoc_8_2020}, \textbf{80.}, p. 47]
	Các hiện tượng sau đây thuộc về hiện tượng vật lý hay hóa học?
	\begin{enumerate*}
		\item[(a)] Sự tạo thành 1 lớp mỏng màu xanh trên mâm đồng.
		\item[(b)] Sự tạo thành chất bột màu xám khi nung nóng bột sắt với lưu huỳnh.
		\item[(c)] 1 lá đồng bị nung nóng, trên mặt đồng có phủ 1 lớp màu đen.
	\end{enumerate*}
\end{baitoan}

\begin{baitoan}[\cite{An_400_BT_Hoa_Hoc_8_2020}, \textbf{81.}, p. 47]
	\begin{enumerate*}
		\item[(a)] Khi quan sát 1 hiện tượng, dựa vào đâu có thể dự đoán được nó là hiện tượng hóa học, trong đó có phản ứng hóa học xảy r?
		\item[(b)] 1 học sinh làm 3 thí nghiệm với chất rắn bicacbonat natri \emph{\ce{NaHCO3}} (thuốc muối trị đầy hơi màu trắng).
		\begin{enumerate*}
			\item[$\bullet$] 1st thí nghiệm: Hòa tan 1 ít thuốc muối rắn trên vào nước được dung dịch trong suốt.
			\item[$\bullet$] 2nd thí nghiệm: Hòa tan 1 ít thuốc muối rắn trên vào nước chanh hoặc giấm thấy sủi bọt mạnh.
			\item[$\bullet$] 3rd thí nghiệm: Đun nóng 1 ít chất rắn trên trong ống nghiệm, màu trắng không đổi nhưng thoát ra 1 chất khí làm đục nước vôi trong.
		\end{enumerate*}
		Trong những thí nghiệm trên, thí nghiệm nào là sự biến đổi hóa học? Giải thích.
	\end{enumerate*}
\end{baitoan}

%------------------------------------------------------------------------------%

\section{Phản Ứng Hóa Học}
``\begin{enumerate*}
	\item[\textbf{1.}] \textit{Phản ứng hóa học} là quá trình làm biến đổi chất này (\textit{chất tham gia} hay \textit{chất phản ứng}) thành chất khác (\textit{sản phẩm} hay \textit{chất tạo thành}).
	\item[\textbf{2.}] Trong phản ứng hóa học chỉ có liên kết giữa các nguyên tử thay đổi làm cho phân tử của chất này biến đổi thành phân tử của chất khác.
	\item[\textbf{3.}] \textit{Điều kiện xảy ra phản ứng}: Các chất tham gia phản ứng phải tiếp xúc với nhau. Phần lớn các trường hợp cần đun nóng. 1 số trường hợp cần chất xúc tác.
	\item[\textbf{4.}] \textit{Dấu hiệu nhận biết có phản ứng xảy ra}: Có ít nhất 1 trong các dấu hiệu sau: Có chất kết tủa (chất không tan). Có chất khí thoát ra (sủi bọt). Có sự thay đổi màu sắc. Có sự tỏa nhiệt hoặc phát sáng.
	\item[\textbf{5.}] \textit{Tốc độ của phản ứng hóa học}: Phản ứng hóa học của những chất khác nhau xảy ra với tốc độ khác nhau. E.g., sự gỉ của sắt trong không khí ẩm là phản ứng hóa học của sắt với oxi \& hơi nước xảy ra rất chậm. Sự nổ của hỗn hợp khí hydro \& oxi là phản ứng hóa học của hydro với oxi tạo ra nước, xảy ra rất nhanh (tức thời).
	\item[\textbf{6.}] \textit{Những yếu tố ảnh hưởng đến tốc độ của phản ứng}:
	\begin{enumerate*}
		\item[(a)] \textit{Nhiệt độ}: Tốc độ của phản ứng hóa học tăng khi tăng nhiệt độ \& giảm khi giảm nhiệt độ. Đối với nhiều phản ứng hóa học, khi nhiệt độ tăng thêm $10^\circ$C thì tốc độ phản ứng tăng khoảng $2$ lần.
		\item[(b)] \textit{Kích thước hạt}: Kích thước của các hạt chất rắn càng nhỏ (i.e., diện tích tiếp xúc càng lớn) thì tốc độ phản ứng hóa học càng tăng. Ngược lại, kích thước của các hạt chất rắn càng lớn (diện tích tiếp xúc càng nhỏ) thì tốc độ phản ứng càng giảm.
		\item[(c)] \textit{Độ đậm đặc của dung dịch các chất tham gia phản ứng}: Dung dịch các chất phản ứng càng đậm đặc, tốc độ phản ứng càng tăng \& ngược lại, dung dịch càng loãng thì tốc độ phản ứng càng giảm.'' -- \cite[p. 34]{Truong2021}
	\end{enumerate*}	
\end{enumerate*}

``\begin{enumerate*}
	\item[\textbf{1.}] \textbf{Phân biệt hiện tượng vật lý \& hiện tượng hóa học}:
	\begin{enumerate*}
		\item[$\bullet$] \textit{Hiện tượng vật lý}: Khi chất đổi về thể hay hình dạng. Không có chất mới nào sinh ra.
		\item[$\bullet$] \textit{Hiện tượng hóa học}: Khi có sự biến đổi từ chất này thành chất khác. Hiện tượng trong đó có sinh ra chất mới.
	\end{enumerate*}
	\item[\textbf{2.}] \textit{Phản ứng hóa học} là quá trình làm biến đổi chất này thành chất khác. Trong phản ứng hóa học, liên kết giữa các nguyên tử thay đổi. Các phản ứng hóa học có thể xảy ra: $A + B\to C + D$, $A + B\to C$, $A\to C + D$.
	\item[\textbf{3.}] \textbf{$2$ định luật hóa học cơ bản.}
	\begin{enumerate*}
		\item[$\bullet$] \textit{Định luật thành phần không đổi}: 1 hợp chất, dù điều chế bằng bất kỳ cách nào, cũng luôn có thành phần không đổi về khối lượng. Ứng dụng: Dựa vào tỷ lệ khối lượng giữa các nguyên tố cấu tạo nên 1 chất là không đổi $\to$ tỷ số nguyên tử không đổi $\to$ lập công thức hóa học của chất đó.
		\item[$\bullet$] \textit{Định luật bảo toàn khối lượng}: Các chất tham gia $\to$ Các chất tạo thành. Tổng khối lượng chất tham gia $=$ Tổng khối lượng chất tạo thành. Ứng dụng: Tính khối lượng của các chất tham gia phản ứng hay chất tạo thành sau phản ứng.
	\end{enumerate*}
	\item[\textbf{4.}] Phương trình hóa học cho biết công thức hóa học các chất phản ứng \& chất mới sinh ra trong phản ứng hóa học, cho biết tỷ lệ số phân tử chất phản ứng \& chất mới sinh ra trong phản ứng hóa học. Lưu ý khi lập phương trình hóa học:
	\begin{enumerate*}
		\item[$\bullet$] Viết đúng công thức hóa học của các chất phản ứng \& chất mới sinh ra.
		\item[$\bullet$] Chọn hệ số phân tử sao cho số nguyên tử của mỗi nguyên tố ở 2 vế đều bằng nhau. Cách làm:
		\begin{enumerate*}
			\item[$\circ$] Nên bắt đầu từ những nguyên tố mà số nguyên tử có nhiều \& không bằng nhau.
			\item[$\circ$] Trường hợp số nguyên tử của 1 nguyên tố ở vế này là số chẵn \& ở vế kia là số lẻ thì trước hết phải đặt hệ số $2$ cho chất mà số nguyên tử là số lẻ, rồi tiếp tục đặt hệ số cho phân tử chứa số nguyên tử chẵn ở vế còn lại sao cho số nguyên tử của nguyên tố này ở 2 vế bằng nhau.
		\end{enumerate*}
	\end{enumerate*}
	Trong quá trình cân bằng không được thay đổi các chỉ số nguyên tử trong các công thức hóa học.
	\item[\textbf{5.}] \textit{Tính hiệu suất phản ứng}: Thực tế do 1 số nguyên nhân chất tham gia phản ứng không tác dụng hết, i.e., hiệu suất $< 100$\%, Người ta có thể tính hiệu suất phản ứng như sau:
	\begin{enumerate*}
		\item[(a)] Dựa vào 1 trong các chất tham gia phản ứng: Công thức tính: $H\% = \frac{\mbox{lượng thực tế đã phản ứng}}{\mbox{lượng tổng số đã lấy}}\cdot100\%$.
		\item[(b)] Dựa vào 1 trong các chất tạo thành: Công thức tính: $H\% = \frac{\mbox{lượng thực tế thu được}}{\mbox{lượng thu theo lý thuyết (theo phương trình phản ứng)}}\cdot100\%$.
		\item[(c)] Bài toán hiệu suất còn mở rộng ra: Cho hiệu suất phản ứng rồi tính lượng chất tham gia hoặc tạo thành.'' -- \cite[Chap. 2, pp. 44--46]{An_400_BT_Hoa_Hoc_8_2020}
	\end{enumerate*}
\end{enumerate*}

\begin{baitoan}[\cite{Truong2021}, \textbf{II.8.}, p. 35]
	Ghi lại phương trình chữ của phản ứng hóa học trong các hiện tượng mô tả dưới đây:
	\begin{enumerate*}
		\item[(a)] Đốt lưu huỳnh ngoài không khí, lưu huỳnh hóa hợp với khí oxi tạo ra khí sunfurơ \emph{\ce{SO2}} có mùi hắc.
		\item[(b)] Ở nhiệt độ cao, nước bị phân hủy sinh ra khí hydro \& khí oxi.
		\item[(c)] Khi nung, đá vôi \emph{\ce{CaCO3}} bị phân hủy sinh ra vôi sống \emph{\ce{CaO}} \& khí carbonic \emph{\ce{CO2}}.
		\item[(d)] Vôi tôi \emph{\ce{Ca(OH)2}} tác dụng với khí \emph{\ce{CO2}} tạo ra \emph{\ce{CaCO3}} \& \emph{\ce{H2O}}.
	\end{enumerate*}
\end{baitoan}

\begin{baitoan}[\cite{Truong2021}, \textbf{II.9.}, pp. 35--36]
	Trong phản ứng hóa học, cho biết:
	\begin{enumerate*}
		\item[(a)] Hạt vi mô nào được bảo toàn, hạt nào còn có thể bị chia nhỏ ra?
		\item[(b)] Vì sao có sự biến đổi phân tử này thành phân tử khác?
		\item[(c)] Nguyên tử có thể bị chia nhỏ hay không?
	\end{enumerate*}
\end{baitoan}

\begin{baitoan}[\cite{Truong2021}, \textbf{II.10.}, p. 36]
	Trong phản ứng hóa học, phân tử \emph{\ce{HgO}} có thể bị chia thành những nguyên tử gì?
\end{baitoan}

\begin{baitoan}[\cite{Truong2021}, \textbf{II.11.}, p. 36]
	Giải thích vì sao có sự biến đổi chất này thành chất khác trong phản ứng hóa học.
\end{baitoan}

\begin{baitoan}[\cite{Truong2021}, \textbf{II.12.}, p. 36]
	1 trong những điều kiện để phản ứng hóa học xảy ra là các chất tham gia phản ứng phải được tiếp xúc với nhau, sự tiếp xúc càng nhiều thì phản ứng càng dễ. Giải thích vì sao khi đưa than vào lò đốt, người ta phải đập nhỏ than.
\end{baitoan}

\begin{baitoan}[\cite{Truong2021}, \textbf{II.13.}, p. 36]
	Sắt để trong không khí ẩm dễ bị gỉ. Giải thích vì sao người ta có thể phòng chống giả bằng cách bôi dầu, mỡ trên bề mặt các đồ dùng bằng sắt.
\end{baitoan}

\begin{baitoan}[\cite{Truong2021}, \textbf{II.14.}, p. 36]
	Kim loại nhôm tác dụng với acid hydrochloric sinh ra khí hydro. Chọn phương án A hay B trong mỗi trường hợp sau để thu được $V{\rm cm}^3$ khí hydro 1 cách nhanh nhất.
	\begin{enumerate*}
		\item[(a)] A: $1$\emph{g} nhôm tác dụng với acid hydrochloric lạnh. B: $1$\emph{g} nhôm tác dụng với acid hydrochloric nóng.
		\item[(b)] A: $1$\emph{g} nhôm tác dụng với acid hydrochloric lạnh. B: $1$\emph{g} bột nhôm tác dụng với acid hydrochloric lạnh.
		\item[(c)] A: $1$\emph{g} nhôm tác dụng với acid hydrochloric đặc. B: $1$\emph{g} nhôm tác dụng với acid hydrochloric loãng.
		\item[(d)] A: $2$\emph{g} nhôm tác dụng với acid hydrochloric loãng. B: $1$\emph{g} nhôm tác dụng với acid hydrochloric loãng.
		\item[(e)] A: $2$\emph{g} nhôm tác dụng với acid hydrochloric nóng. B: $1$\emph{g} nhôm tác dụng với acid hydrochloric lạnh.
	\end{enumerate*}
\end{baitoan}

\begin{baitoan}[\cite{Truong2021}, \textbf{II.15.}, pp. 36--37]
	Cho 1 lá sắt nhỏ tác dụng với dung dịch acid hydrochloric, nhận thấy nhiệt độ của quá trình phản ứng tăng dần. Thể tích khí hydro thu được tương ứng với thời gian đo được như sau:
	
	\begin{table}[H]
		\centering
		\begin{tabular}{|c|c|c|c|c|c|c|c|c|}
			\hline
			Thể tích ($\rm cm^3$) & $3$ & $10$ & $50$ & $78$ & $85$ & $89$ & $90$ & $90$ \\
			\hline
			Thời gian (phút) & $1$ & $2$ & $3$ & $4$ & $5$ & $6$ & $7$ & $8$ \\
			\hline
		\end{tabular}
	\end{table}
	\begin{enumerate*}
		\item[(a)] Thể tích khí hydro thu được trong quá trình thí nghiệm thay đổi như thế nào?
		\item[(b)] Vẽ đồ thị biểu diễn thể tích khí hydro thu được theo thời gian thí nghiệm (thể tích khí trên trục tung, thời gian trên trục hoành). Chú thích rõ trên mỗi trục.
		\item[(c)] Chúng ta biết tốc độ của phản ứng hóa học xảy ra chậm dần theo thời gian. Nhưng ở đây, phản ứng hóa học lại xảy ra nhanh từ phút thứ $2$ đến phút thứ $3$. Giải thích sự tăng tốc độ này của phản ứng hóa học.
		\item[(d)] Độ dốc của đồ thị xảy ra như thế nào kể từ phút thứ $7$ trở đi?
		\item[(e)] Phản ứng hóa học kết thúc sau thời gian bao lâu?
	\end{enumerate*}
\end{baitoan}

\begin{baitoan}[\cite{An_400_BT_Hoa_Hoc_8_2020}, p. 45]
	Lập phương trình hóa học của phản ứng có sơ đồ sau: \emph{\ce{Al + O2 -> Al2O3}}.\hfill\textsf{Ans:} $4,3,2$.
\end{baitoan}

\begin{baitoan}[\cite{An_400_BT_Hoa_Hoc_8_2020}, p. 45]
	Lập phương trình hóa học của phản ứng có sơ đồ sau: \emph{\ce{Al + H2SO4 -> Al2(SO4)3 + H2}}.\\\mbox{}\hfill\textsf{Ans:} $2,3,1,3$.
\end{baitoan}

%------------------------------------------------------------------------------%

\section{Định Luật Bảo Toàn Khối Lượng}

%------------------------------------------------------------------------------%

\section{Định Luật Bảo Toàn Khối Lượng}

%------------------------------------------------------------------------------%

\section{Phương Trình Hóa Học}

%------------------------------------------------------------------------------%

\printbibliography[heading=bibintoc]
	
\end{document}