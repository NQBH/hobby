\documentclass[oneside]{book}
\usepackage[backend=biber,natbib=true,style=authoryear]{biblatex}
\addbibresource{/home/hong/1_NQBH/reference/bib.bib}
\usepackage[utf8]{vietnam}
\usepackage{tocloft}
\renewcommand{\cftsecleader}{\cftdotfill{\cftdotsep}}
\usepackage[colorlinks=true,linkcolor=blue,urlcolor=red,citecolor=magenta]{hyperref}
\usepackage{amsmath,amssymb,amsthm,mathtools,float,graphicx,algpseudocode,algorithm,tcolorbox,tikz,tkz-tab,diagbox}
\DeclareMathOperator{\arccot}{arccot}
\usepackage[inline]{enumitem}
\allowdisplaybreaks
\numberwithin{equation}{section}
\newtheorem{assumption}{Assumption}[section]
\newtheorem{nhanxet}{Nhận xét}[section]
\newtheorem{conjecture}{Conjecture}[section]
\newtheorem{corollary}{Corollary}[section]
\newtheorem{hequa}{Hệ quả}[section]
\newtheorem{definition}{Definition}[section]
\newtheorem{dinhnghia}{Định nghĩa}[section]
\newtheorem{example}{Example}[section]
\newtheorem{vidu}{Ví dụ}[section]
\newtheorem{lemma}{Lemma}[section]
\newtheorem{notation}{Notation}[section]
\newtheorem{principle}{Principle}[section]
\newtheorem{problem}{Problem}[section]
\newtheorem{baitoan}{Bài toán}[section]
\newtheorem{proposition}{Proposition}[section]
\newtheorem{menhde}{Mệnh đề}[section]
\newtheorem{question}{Question}[section]
\newtheorem{cauhoi}{Câu hỏi}[section]
\newtheorem{remark}{Remark}[section]
\newtheorem{luuy}{Lưu ý}[section]
\newtheorem{theorem}{Theorem}[section]
\newtheorem{dinhly}{Định lý}[section]
\usepackage[left=0.5in,right=0.5in,top=1.5cm,bottom=1.5cm]{geometry}
\usepackage{fancyhdr}
\pagestyle{fancy}
\fancyhf{}
\lhead{\small \textsc{Sect.} ~\thesection}
\rhead{\small \nouppercase{\leftmark}}
\renewcommand{\sectionmark}[1]{\markboth{#1}{}}
\cfoot{\thepage}
\def\labelitemii{$\circ$}

\title{Some Topics in Elementary Chemistry\texttt{/}Grade 10}
\author{Nguyễn Quản Bá Hồng\footnote{Independent Researcher, Ben Tre City, Vietnam\\e-mail: \texttt{nguyenquanbahong@gmail.com}; website: \url{https://nqbh.github.io}.}}
\date{\today}

\begin{document}
\frontmatter
\maketitle
\setcounter{secnumdepth}{4}
\setcounter{tocdepth}{3}
\tableofcontents
\newpage

%------------------------------------------------------------------------------%

\mainmatter

\chapter*{Preface}

Tóm tắt kiến thức Hóa học lớp 10 theo chương trình giáo dục của Việt Nam \& một số chủ đề nâng cao.

\section*{Notation, Abbreviation, Convention}
\begin{itemize}
	\item askt: ánh sáng khuếch tán.
	\item asmt: ánh sáng mặt trời.
	\item đpnc: điện phân nóng chảy.
	\item $\uparrow,\downarrow$: sản phẩm khí, sản phẩm rắn (kết tủa).
	\item (s): solid -- chất rắn.
	\item (l): liquid -- chất lỏng.
	\item (g): gas, chất khí (hơi).
	\item (aq): aqueous\footnote{\textbf{aqueous} [a] [usually before noun] (\textit{specialist}) containing or involving water.} -- chất tan trong nước (dung dịch).
	\item $E_{\rm a}$: activation\footnote{\textbf{activation} [n] [uncountable] \textbf{activation (of something)} the fact or process of making something such as a device or chemical process start working.} energy\footnote{\textbf{energy} [n] \textbf{1.} [uncountable, countable] the ability of matter or radiation to perform work because of its mass, movement, electrical charge, etc.; \textbf{2.} [uncountable] a source of power that can be used by somebody\texttt{/}something, e.g. to provide light \& heat, or to work machines; \textbf{3.} [uncountable] the effort needed to do work or other physical or mental activities; \textbf{4.} (\textbf{energies}) [plural] the physical \& mental effort that you use to do something.} -- năng lượng hoạt hóa.
	\item $E_{\rm b}$: bond\footnote{\textbf{bond} [n] \textbf{1.} something that forms a connection between people or groups, such as a feeling of friendship or shared ideas \& experiences; \textbf{2.} the way in which atoms are held together in a chemical compound; \textbf{3.} an agreement by a government or a company to pay somebody interest on the money they have lent after a particular period of time; a document containing this agreement; \textbf{4.} the way in which 2 surfaces are joined together, often using glue; [v] \textbf{1.} [transitive, intransitive] to join 2 things firmly together; to join firmly to something else; \textbf{2.} [transitive] to join atoms together by a chemical bond; \textbf{3.} [intransitive] \textbf{bond (with somebody)} to develop or create a relationship of trust with somebody.} energy -- năng lượng liên kết.
	\item SATP: standard ambient\footnote{\textbf{ambient} [a] [only before noun] in the surrounding area; on all sides.} temperature \& pressure -- điều kiện chuẩn về nhiệt độ \& áp suất.
	\item $\Delta H$: enthalpy change -- biến thiên enthalpy.
	\item $\Delta_{\rm f}H_{298}^0$: standard enthalpy of formation\footnote{\textbf{formation} [n] \textbf{1.} [uncountable] the action of forming something; the process of being formed; \textbf{2.} [countable] a thing that has been formed, especially in a particular place or in a particular way; \textbf{3.} [countable, uncountable] a particular arrangement or pattern of people or things.} at $298$ K -- enthalpy tạo thành chuẩn ở $298$ K.
	\item $\Delta_{\rm r}H_{298}^0$: standard enthalpy change of reaction\footnote{\textbf{reaction} [n] \textbf{1.} [countable, uncountable] what you do, say or think as a result of something that has happened; \textbf{2.} [countable] (\textit{chemistry}) a chemical change produced by 2 or more substances acting on each other; \textbf{3.} [countable, uncountable] (\textit{medical}) a response by the body, usually a bad one, to something such as a drug or a chemical substance; \textbf{4.} [uncountable, countable] (\textit{physics}) a force shown by something in response to another force, which is of equal strength \& acts in the opposite direction; \textbf{5.} [countable, usually singular] \textbf{reaction (against something)} a change in people's attitudes or behavior caused by strong disapproval of other very different attitudes; \textbf{6.} [uncountable] opposition to social or political progress or change; \textbf{7.} (\textbf{reactions}) [plural] the ability to move quickly in response to something, especially if in danger.} at $298$ K -- biến thiên enthalpy chuẩn của phản ứng ở $298$ K.
\end{itemize}
``Từ lâu, hóa học được mệnh danh là ``trung tâm của các ngành khoa học'' vì nhiều ngành khoa học như vật lý, sinh học, y học, khoa học Trái Đất, $\ldots$ đều lấy hóa học làm nền tảng cho sự phát triển. Hóa học cũng là cơ sở phát triển cho nhiều ngành công nghiệp khác như vật liệu, luyện kim, điện tử, dược phẩm, dầu khí, $\ldots$ Trong cuộc sống hằng ngày, hóa học hiện diễn ở khắp mọi nơi. Từ lương thực -- thực phẩm, đồ dùng thiết yếu trong gia đình, dụng cụ học tập, thuốc chữa bệnh, nguyên liệu sản xuất, $\ldots$ đến hương thơm quyến rũ của nước hoa, mỹ phẩm, $\ldots$ đều là những sản phẩm của hóa học.'' -- \cite[p. 3]{SGK_Hoa_Hoc_10_Chan_Troi_Sang_Tao}

%------------------------------------------------------------------------------%

\section{Nhập Môn Hóa Học}
\textbf{Nội dung.} \textit{Đối tượng nghiên cứu của hóa học, vai trò của hóa học đối với đời sống, sản xuất, $\ldots$, phương pháp học tập \& nghiên cứu hóa học}.

\subsection{Đối tượng nghiên cứu của hóa học}

\begin{dinhnghia}[Hóa học]
	``\emph{Hóa học} là ngành khoa học thuộc lĩnh vực khoa học tự nhiên, nghiên cứu về thành phần, cấu trúc, tính chất, \& sự biến đổi của chất cũng như ứng dụng của chúng.'' -- \cite[p. 7]{SGK_Hoa_Hoc_10_Chan_Troi_Sang_Tao}
\end{dinhnghia}
``Khi đốt nến (được làm bằng paraffin), nếu chảy ra ở dạng lỏng, thấm vào bấc, cháy trong không khí, sinh ra khí carbon dioxide \& hơi nước.'' -- \cite[p. 7]{SGK_Hoa_Hoc_10_Chan_Troi_Sang_Tao}

\subsubsection{Vai trò của hóa học trong đời sống \& sản xuất}
``Hóa học có vai trò quan trọng trong đời sống, sản xuất \& nghiên cứu khoa học.'' -- \cite[p. 8]{SGK_Hoa_Hoc_10_Chan_Troi_Sang_Tao}

\subsubsection{Phương pháp học tập Hóa học}
``Phương pháp học tập hóa học nhằm phát triển năng lực hóa học, bao gồm:
\begin{enumerate*}
	\item[\textbf{1.}] Phương pháp tìm hiểu lý thuyết;
	\item[\textbf{2.}] Phương pháp học tập thông qua thực hành thí nghiệm;
	\item[\textbf{3.}] Phương pháp luyện tập, ôn tập;
	\item[\textbf{4.}] Phương pháp học tập trải nghiệm.'' -- \cite[p. 9]{SGK_Hoa_Hoc_10_Chan_Troi_Sang_Tao}
\end{enumerate*}

\subsubsection{Phương pháp nghiên cứu Hóa học}
``Khi nghiên cứu 1 vấn đề hóa học, chúng ta cần có phương pháp nghiên cứu. Không có phương pháp nào là chung cho mọi nghiên cứu. Tùy vào mục đích \& đối tượng nghiên cứu mà chúng ta lựa chọn phương pháp cho phù hợp.
\begin{enumerate}
	\item \textbf{Phương pháp nghiên cứu lý thuyết} là sử dụng những định luật, nguyên lý, quy tắc, cơ chế, mô hình, $\ldots$ cũng như các kết quả nghiên cứu đã có để tiếp tục làm rõ những vấn đề của lý thuyết hóa học.
	\item \textbf{Phương pháp nghiên cứu thực nghiệm} là nghiên cứu những vấn đề dựa trên kết quả thí nghiệm, khảo sát, thu thập số liệu, phân tích, định lượng, $\ldots$
	\item \textbf{Phương pháp nghiên cứu ứng dụng} nhằm mục đích giải quyết các vấn đề hóa học được ứng dụng trong các lĩnh vực khác nhau.'' -- \cite[p. 10]{SGK_Hoa_Hoc_10_Chan_Troi_Sang_Tao}
\end{enumerate}

\begin{vidu}
	``Để nghiên cứu thành phần hóa học \& bước đầu ứng dụng tinh dầu tràm trà (\emph{Melaleuca alternifolia}) trong sản xuất nước súc miệng, các nhà nghiên cứu đã thực hiện theo các bước được mô tả trong  \cite[Hình 1.12: \textsf{Các bước thực hiện trong đề tài nghiên cứu thành phần hóa học \& bước đầu ứng dụng tinh dầu tràm trà trong sản xuất nước súc miệng}, p. 10]{SGK_Hoa_Hoc_10_Chan_Troi_Sang_Tao}:
	\begin{enumerate*}
		\item[\textbf{1.}] Nghiên cứu thành phần hóa học \& ứng dụng của tinh dầu tràm trà làm nước súc miệng qua các công trình khoa học trên các tạp chí đã được xuất bản.
		\item[\textbf{2.}] Đặt giả thuyết: tinh dầu tràm trà có khả năng kháng khuẩn.
		\item[\textbf{3.}] Thí nghiệm chiết xuất tinh dầu bằng phương pháp chưng cất lôi cuốn hơi nước.
		\item[\textbf{4.}] Khảo sát hoạt tính kháng khuẩn của sản phẩm nước súc miệng từ tinh dầu tràm trà.'' -- \cite[p. 10]{SGK_Hoa_Hoc_10_Chan_Troi_Sang_Tao}
	\end{enumerate*}
\end{vidu}
``\textit{Phương pháp nghiên cứu hóa học} bao gồm: nghiên cứu lý thuyết, nghiên cứu thực nghiệm \& nghiên cứu ứng dụng. Phương pháp nghiên cứu hóa học thường bao gồm 1 số bước:
\begin{enumerate*}
	\item[\textbf{1.}] Xác định vấn đề nghiên cứu;
	\item[\textbf{2.}] Nêu giả thuyết khoa học;
	\item[\textbf{3.}] Thực hiện nghiên cứu (lý thuyết, thực nghiệm, ứng dụng);
	\item[\textbf{4.}] Viết báo cáo: thảo luận kết quả \& kết luận vấn đề.'' -- \cite[p. 11]{SGK_Hoa_Hoc_10_Chan_Troi_Sang_Tao}
\end{enumerate*}
``\textit{Mưa acid} là 1 thuật ngữ chung chỉ sự tích lũy của các chất gây ô nhiễm, có khả năng chuyển hóa trong nước mưa tạo nên môi trường acid. Các chất gây ô nhiễm chủ yếu là khí $\rm SO_2$ \& $\rm NO_x$ thải ra từ các quá trình sản xuất trong đời sống, đặc biệt là quá trình đối cháy than đá, dầu mỏ, \& các nhiên liệu tự nhiên khác. Hiện tượng này gây ảnh hưởng trực tiếp đến đời sống con người, động -- thực vật \& có thể làm thay đổi thành phần của nước các sông, hồ, giết chết các loài cá \& những sinh vật khác, đồng thời hủy hoại các công trình kiến trúc.'' -- \cite[p. 11]{SGK_Hoa_Hoc_10_Chan_Troi_Sang_Tao}

``Hóa học là 1 ngành khoa học thuộc lĩnh vực khoa học tự nhiên, kết hợp chặt chẽ giữa lý thuyết \& thực nghiệm. Hóa học còn được gọi là ``khoa học trung tâm'' vì nó là cầu nối giữa các ngành khoa học tự nhiên khác như vật lý, địa chất, \& sinh học, $\ldots$ Theo truyền thống, hóa học được chia thành 5 chuyên ngành chính, bao gồm: hóa lý thuyết \& hóa lý, hóa vô cơ, hóa hữu cơ, hóa phân tích, hóa sinh.'' -- \cite[p. 11]{SGK_Hoa_Hoc_10_Chan_Troi_Sang_Tao}

%------------------------------------------------------------------------------%

\chapter{Cấu Tạo Nguyên Tử}

\section{Thành Phần của Nguyên Tử}
\textbf{Nội dung.} \textit{Thành phần của nguyên tử, so sánh khối lượng của electron với proton \& neutron, kích thước của hạt nhân với kích thước nguyên tử}.

\subsection{Thành phần cấu tạo nguyên tử}
``Từ thời cổ Hy Lạp, nhà triết học Democritous (460--370 BC) cho rằng mọi vật chất được tạo thành từ các phần tử rất nhỏ được gọi là ``atomos'', i.e., không thể phá hủy \& không thể chia nhỏ hơn được nữa. Đến giữa thế kỷ XIX, các nhà khoa học cho rằng: các chất đều được cấu tạo nên từ những hạt rất nhỏ, không thể phân chia được nữa, gọi là \textit{nguyên tử}. Vào cuối thế kỷ XIX, đầu thế kỷ XX, bằng những nghiên cứu thực nghiệm, các nhà khoa học đã chứng minh sự tồn tại của nguyên tử \& nguyên tử có cấu tạo phức tạp.'' -- \cite[p. 13]{SGK_Hoa_Hoc_10_Chan_Troi_Sang_Tao}

\begin{figure}[H]
	\centering
	\includegraphics[scale=0.15]{mo_hinh_nguyen_tu}
	\caption{Mô hình nguyên tử, \cite[Hình 2.1, p. 13]{SGK_Hoa_Hoc_10_Chan_Troi_Sang_Tao}.}
\end{figure}
``Nguyên tử gồm hạt nhân chứa proton, neutron \& vỏ nguyên tử chứa electron.'' -- \cite[p. 14]{SGK_Hoa_Hoc_10_Chan_Troi_Sang_Tao}

\subsection{Sự tìm ra Electron}
``Năm 1897, nhà vật lý người Anh Joseph John Thomson (1856--1940) thực hiện thí nghiệm phóng điện trong 1 ống thủy tinh gần như chân không (gọi là \textit{ống tia âm cực}). Ông quan sát thấy màn huỳnh quang trong ống phát sáng do những tia phat phát ra từ cực âm (gọi là \textit{tia cực âm}) \& những tia này bị hút về cực dương của trường điện  (\cite[Hình 2.2: \textsf{Thí nghiệm của Thomson}, p. 13]{SGK_Hoa_Hoc_10_Chan_Troi_Sang_Tao}), chứng tỏ chúng tích điện âm. Đó chính là chùm các hạt \textit{electron}.'' ``Trong nguyên tử tồn tại 1 loại hạt có khối lượng \& mang điện tích âm, được gọi là \textit{electron} (ký hiệu là $e$). Hạt electron có: Điện tích: $q_{\rm e} = -1.602\cdot 10^{-19}$ C (Coulomb)\footnote{See, e.g., \cite{SGK_Vat_Ly_11_co_ban, SGK_Vat_Ly_11_nang_cao}.}. Khối lượng: $m_{\rm e} = 9.11\cdot 10^{-28}$ g. Người ta chưa phát hiện được điện tích nào nhỏ hơn $1.602\cdot 10^{-19}$ C nên nó được dùng làm điện tích đơn vị, điện tích của electron được quy ước là $-1$.'' -- \cite[p. 14]{SGK_Hoa_Hoc_10_Chan_Troi_Sang_Tao}

\subsubsection{Thí nghiệm giọt dầu của Millikan}
``Năm 1909, nhà vật lý thực nghiệm người Mỹ là R. A. Millikan đã tiến hành phun các giọt dầu vào 1 hộp trong suốt. Bên trong hộp chứa 2 tấm kim loại được nối vào nguồn điện 1 chiều với 1 đầu tích điện âm ($-$) \& 1 đầu tích điện dương ($+$). Trong hộp còn có thiết bị phát ra 1 chùm tia R\"ontgen (tia X) để ion hóa các giọt dầu (cấp cho nó 1 điện tích). Tia X có khả năng đánh bật các electron khỏi không khí giữa các tấm kim loại \& các electron sẽ bám vào các giọt dầu, làm chúng tích điện âm. Bằng cách thay đổi cường độ trường điện, Millikan có thể kiểm soát tốc độ rơi của các giọt dầu. Chuyển động của các giọt dầu trong thiết bị phụ thuộc vào điện tích của mỗi giọt \& vào trường điện. Millikan đã quan sát các giọt dầu bằng kính thiên văn. Millikan có thể làm cho các giọt dầu rơi chậm hơn, nhanh hơn, hoặc khiến chúng dừng lại khi thay đổi cường độ của trường điện. Từ những quan sát của mình, ông đã tính được điện tích \& khối lượng của electron.'' -- \cite[p. 15]{SGK_Hoa_Hoc_10_Chan_Troi_Sang_Tao} (xem mô hình \textsf{Thí nghiệm giọt dầu của Millikan}).

\subsection{Sự khám phá hạt nhân nguyên tử}
``Năm 1911, nhà vật lý người New Zealand là Ernest Rutherford (1871--1937) đã tiến hành bắn phá 1 chùm hạt alpha (ký hiệu là $\alpha$, hạt nhân của nguyên tử helium, mang điện tích $+2$, có khối lượng gấp khoảng $7500$ lần khối lượng electron) lên 1 lá vàng siêu mỏng (\cite[Hình 2.3: \textsf{Thí nghiệm khám phá hạt nhân nguyên tử của Rutherford}, p. 16]{SGK_Hoa_Hoc_10_Chan_Troi_Sang_Tao}) \& quan sát đường đi của chúng sau khi bắn phá bằng màn huỳnh quang (zinc sulfide, ZnS).'' -- \cite[p. 15]{SGK_Hoa_Hoc_10_Chan_Troi_Sang_Tao}

``Nguyên tử có cấu tạo rỗng, gồm hạt nhân ở trung tâm \& lớp vỏ là các electron chuyển động xung quanh hạt nhân. Nguyên tử trung hòa về điện: \textit{số đơn vị điện tích dương của hạt nhân bằng số đơn vị điện tích âm của các electron trong nguyên tử}.'' -- \cite[p. 16]{SGK_Hoa_Hoc_10_Chan_Troi_Sang_Tao}

\subsection{Cấu tạo hạt nhân nguyên tử}
``Vào năm 1918, khi bắn phá hạt nhân nguyên tử nitrogen bằng các hạt $\alpha$ (thực hiện trong máy gia tốc hạt), Rutherford đã nhận thấy sự xuất hiện hạt nhân nguyên tử oxygen \& 1 loại hạt mang 1 đơn vị điện tích dương (${\rm e}_0$ hay $+1$), đó là \textit{proton} (ký hiệu là p). Năm 1932, khi dùng các hạt $\alpha$ để bắn phá hạt nhân nguyên tử beryllium, J. Chadwick nhận thấy sự xuất hiện của 1 loại hạt có khối lượng xấp xỉ hạt proton, nhưng không mang điện, ông gọi chúng là \textit{neutron} (ký hiệu là n).'' -- \cite[pp. 16--17]{SGK_Hoa_Hoc_10_Chan_Troi_Sang_Tao}. ``Hạt nhân nguyên tử gồm 2 loại hạt là proton \& neutron. Proton mang điện tích dương ($+1$) \& neutron không mang điện. Proton \& neutron có khối lượng gần bằng nhau.'' -- \cite[p. 17]{SGK_Hoa_Hoc_10_Chan_Troi_Sang_Tao}

\subsection{Kích thước \& khối lượng nguyên tử}
``Nếu hình dung hạt nhân là 1 khối cầu có kích thước như viên bi thì nguyên tử sẽ là 1 khối cầu có kích thước bằng 1 sân bóng đá.'' -- \cite[p. 17]{SGK_Hoa_Hoc_10_Chan_Troi_Sang_Tao}. ``Đơn vị nanomet (nm) hay angstrom ($\mathring{\rm A}$) thường được sử dụng để biểu thị kích thước nguyên tử. $\rm 1\ nm = 10^{-9}\ m$, $\rm 1\ \mathring{\rm A} = 10^{-10}\ m$, $\rm 1\ nm = 10\ \mathring{\rm A}$.'' ``Nếu xem nguyên tử như 1 quả cầu, trong đó các electron chuyển động rất nhanh xung quanh hạt nhân thì nguyên tử đó có đường kính khoảng $10^{-10}$ m \& đường kính hạt nhân khoảng $10^{-14}$ m. Như vậy, đường kính của nguyên tử lớn hơn đường kính của hạt nhân khoảng $10^4$ lần.'' -- \cite[p. 18]{SGK_Hoa_Hoc_10_Chan_Troi_Sang_Tao}

``Những hiểu biết của nhân loại về vũ trụ \& thế giới xung quanh ngày càng phát triển. Người Hy Lạp cổ đại lần đầu tiên đoán được sự tồn tại của các hạt gọi là \textit{nguyên tử}. Khoảng 1500 năm sau, người ta đã chứng minh được sự tồn tại của nguyên tử \& xem chúng là những hạt nhỏ nhất, tạo nên vật chất. Sau đó không lâu, người ta phát hiện ra nguyên tử được tạo thành từ 3 loại hạt cơ bản là proton, neutron, \& electron. Tuy nhiên, các hạt này vẫn chưa phải những hạt nhỏ nhất trong vũ trụ. Ngày nay các công trình nghiên cứu cho thấy proton \& neutron được tạo thành bởi các hạt nhỏ hơn, gọi là \textit{hạt quark}.'' -- \cite[p. 18]{SGK_Hoa_Hoc_10_Chan_Troi_Sang_Tao}

\begin{table}[H]
	\centering
	\begin{tabular}{|c|c|c|c|}
		\hline
		\textbf{Hạt} & \textbf{Điện tích tương đối} & \textbf{Khối lượng (amu)} & \textbf{Khối lượng (g)} \\
		\hline
		\textbf{p} & $+1$ & $\approx 1$ & $1.673\cdot 10^{-24}$ \\
		\hline
		\textbf{n} & $0$ & $\approx 1$ & $1.675\cdot 10^{-24}$ \\
		\hline 
		\textbf{e} & $-1$ & $\frac{1}{1840}\approx 0.00055$ & $9.11\cdot 10^{-28}$ \\
		\hline
	\end{tabular}
	\caption{1 số tính chất của các loại hạt cơ bản trong nguyên tử, \cite[Bảng 2.1, p. 18]{SGK_Hoa_Hoc_10_Chan_Troi_Sang_Tao}.}
\end{table}
``Để biểu thị khối lượng của nguyên tử, các hạt proton, neutron, \& electron, người ta dùng đơn vị \textit{khối lượng nguyên tử}, ký hiệu là amu. 1 amu bằng $\frac{1}{12}$ khối lượng nguyên tử của carbon $-12$. $\rm 1\ amu = 1.66\cdot 10^{-24}\ g$.'' -- \cite[p. 18]{SGK_Hoa_Hoc_10_Chan_Troi_Sang_Tao}

``Khối lượng của nguyên tử gần bằng khối lượng hạt nhân do khối lượng của các electron không đáng kể so với khối lượng của proton \& neutron.'' -- \cite[p. 19]{SGK_Hoa_Hoc_10_Chan_Troi_Sang_Tao}

%------------------------------------------------------------------------------%

\section{Nguyên Tố Hóa Học}
\textbf{Nội dung.} \textit{Nguyên tố hóa học, số hiệu nguyên tử \& ký hiệu nguyên tử, đồng vị, nguyên tử khối, nguyên tử khối trung bình (theo amu) dựa vào khối lượng nguyên tử \& \% số nguyên tử của các đồng vị theo phổ khối lượng được cung cấp}.

\subsection{Hạt nhân nguyên tử}
``Số đơn vị điện tích hạt nhân (Z) $=$ số proton (P) $=$ số electron (E). Điện tích hạt nhân $=$ $+$Z.'' -- \cite[p. 20]{SGK_Hoa_Hoc_10_Chan_Troi_Sang_Tao}

``Số khối (A) $=$ số proton (P) + số neutron (N).'' -- \cite[p. 21]{SGK_Hoa_Hoc_10_Chan_Troi_Sang_Tao}

\subsection{Nguyên tố hóa học}
``\textit{Số hiệu nguyên tử} của 1 nguyên tố được quy ước bằng số đơn vị điện tích hạt nhân nguyên tử của nguyên tố đó. Số hiệu nguyên tử (ký hiệu là Z) cho biết: số proton trong hạt nhân nguyên tử, số electron trong nguyên tử.  

\begin{dinhnghia}[Số hiệu nguyên tử]
	Số đơn vị điện tích hạt nhân nguyên tử của 1 nguyên tố được gọi là \emph{số hiệu nguyên tử (Z)} của nguyên tố đó. Mỗi nguyên tố hóa học có 1 số hiệu nguyên tử.
\end{dinhnghia}
Năm 1913, nhà vật lý người Anh là H. Moseley đã thực hiện thí nghiệm khảo sát bản chất tự nhiên của tia X. Moseley sử dụng 1 chùm tia electron có năng lượng cao để bắn vào các tấm kim loại khác nhau làm anode \& thu được tia X. Moseley phát hiện ra rằng, bước sóng của tia X luôn không đổi đối với 1 kim loại nhất định \& thay đổi khi thay anode bằng những kim loại khác. Từ đó, ông cho rằng bước sóng này phụ thuộc vào số proton trong nguyên tử của mỗi số nguyên tố kim loại được dùng làm anode.'' -- \cite[p. 21]{SGK_Hoa_Hoc_10_Chan_Troi_Sang_Tao} (xem \textsf{Mô hình thí nghiệm khảo sát bản chất tự nhiên của tia X của Henry Moseley}).

``Protium, deuterium, \& tritium là các loại nguyên tử của nguyên tố hydrogen.

\begin{dinhnghia}[Nguyên tố hóa học]
	\emph{Nguyên tố hóa học} là tập hợp những nguyên tử có cùng điện tích hạt nhân.
\end{dinhnghia}
\textit{Số đơn vị điện tích hạt nhân nguyên tử} (còn được gọi là \textit{số hiệu nguyên tử}) của 1 nguyên tố hóa học \& số khối được xem là những đặc trưng cơ bản của nguyên tử. Để ký hiệu nguyên tử, người ta thường ghi các chỉ số đặc trưng ở bên trái ký hiệu nguyên tố với số khối A ở trên, số hiệu Z ở phía dưới. Ký hiệu nguyên tử được sử dụng để biểu thị nguyên tử của 1 nguyên tố hóa học. $\rm{}_Z^AX$ (A: số khối, Z: số hiệu nguyên tử, X: ký hiệu nguyên tố hóa học)'' -- \cite[p. 22]{SGK_Hoa_Hoc_10_Chan_Troi_Sang_Tao}

\subsection{Đồng vị}
``Các nguyên tử của cùng 1 nguyên tố hóa học có thể có số khối khác nhau. Sở dĩ như vậy vì hạt nhân của các nguyên tử đó có cùng số proton, nhưng có thể khác số neutron. Những nguyên tử này được gọi là \textit{đồng vị} của 1 nguyên tố hóa học. Trong tự nhiên, hầu hết các nguyên tố được tìm thấy dưới dạng hỗn hợp của các đồng vị. 1 nguyên tố hóa học dù ở dạng đơn chất hay hợp chất thì tỷ lệ giữa các đồng vị của nguyên tố này là không đổi. E.g., các quả chuối đều chứa nguyên tố potassium (K) trong thành phần dinh dưỡng của chúng. Chúng có thể khác nhau về kích thước, hình dáng, mùi vị cũng như được thu hoạch ở những vị trí địa lý khác nhau nhưng đều chứa $93.26$\% số nguyên tử $\rm{}_{19}^{39}K$, $6.73$\% số nguyên tử $\rm{}_{19}^{41}K$, \& $0.01$\% số nguyên tử $\rm{}_{19}^{40}K$ trong tổng số nguyên tử potassium có trong chúng. Ngoài những đồng vị bền, các nguyên tố hóa học còn có 1 số đồng bị không bền, gọi là \textit{đồng vị phóng xạ}, được sử dụng nhiều trong đời sống, y học, nghiên cứu khoa học, $\ldots$'' -- \cite[pp. 22--23]{SGK_Hoa_Hoc_10_Chan_Troi_Sang_Tao}

``Kim cương là 1 trong những dạng tồn tại của nguyên tố carbon trong tự nhiên. Nguyên tố này có 2 đồng bị bền với số khối lần lượt là 12 \& 13.'' ``Các đồng vị của 1 nguyên tố hóa học là những nguyên tử có cùng số proton (P), cùng số hiệu nguyên tử (Z), nhưng khác nhau về số neutron (N). Do đó, số khối (A) của chúng khác nhau.'' -- \cite[p. 23]{SGK_Hoa_Hoc_10_Chan_Troi_Sang_Tao}

\subsection{Nguyên tử khối \& nguyên tử khối trung bình}

\begin{dinhnghia}[Nguyên tử khối]
	\emph{Nguyên tử khối} là khối lượng tương đối của nguyên tử.
\end{dinhnghia}
``Khối lượng của 1 nguyên tử bằng tổng khối lượng của proton, neutron, \& electron trong nguyên tử đó. Proton \& neutron đều có khối lượng gần bằng 1 amu, electron có khối lượng nhỏ hơn rất nhiều (khoảng $0.00055$ amu). Do đó, có thể coi \textit{nguyên tử khối có giá trị bằng số khối}. Nguyên tử khối của 1 nguyên tử cho biết khối lượng của nguyên tử đó nặng gấp bao nhiêu lần đơn vị khối lượng nguyên tử (1 amu).'' ``Mỗi nguyên tố thường có nhiều đồng vị, do đó trong thực tế, người ta thường sử dụng giá trị \textit{nguyên tử khối trung bình}. Muốn xác định giá trị nguyên tử khối trung bình của 1 nguyên tố, ta cần phải biết được phần trăm số nguyên tử các đồng vị của nguyên tố đó trong tự nhiên. Người ta thường sử dụng \textit{phương pháp phổ khối lượng} (MassSpectrometry -- MS) để xác định \% số nguyên tử các đồng vị trong tự nhiên của các nguyên tố. Đây cũng là 1 phương pháp quan trọng trong việc phân tích thành phần \& cấu trúc các chất.'' ``Trong tự nhiên, nguyên tố copper có 2 đồng vị với \%  số nguyên tử tương ứn là $\rm{}_{29}^{63}Cu$ ($69.15$\%) \& $\rm{}_{29}^{65}Cu$ ($30.85$\%).'' -- \cite[p. 23]{SGK_Hoa_Hoc_10_Chan_Troi_Sang_Tao}

``Trong tự nhiên, chlorine có 2 đồng vị là $\rm{}_{17}^{35}Cl$ \& ${}_{17}^{37}Cl$ có tỷ lệ \% số nguyên tử tương ứng là $75.76\%$ \& $24.24$\%. Các xác định nguyên tử khối trung bình của chlorine:
\begin{align*}
	\rm\overline{A}_{Cl} = \frac{(A_{35_{Cl}}\cdot\%^{35}Cl) + (A_{37_{Cl}}\cdot\%^{37}Cl)}{100} = \frac{35\cdot 75.76 + 37\cdot 24.24}{100} = 35.48.
\end{align*}

\begin{menhde}
	Công thức tính nguyên tử khối trung bình của nguyên tố X:
	\begin{align*}
		\overline{A}_X = \frac{1}{100}\sum_{i=1}^k a_iA_i,
	\end{align*}
	$\overline{A}_X$ là nguyên tử khối trung bình của $X$, $A_i$ là nguyên tử khối đồng vị thứ $i$, $a_i$ là tỷ lệ \% số nguyên tử đồng vị thứ $i$, $i = 1,\ldots,k$.
\end{menhde}
Trong thể dục thể thao, có 1 số vận động viên sử dụng các loại chất kích thích trong thi đấu, gọi là \textit{dopping}, dẫn đến thành tích đạt được của họ không thật so với năng lực vốn có. 1 trong các loại dopping thường gặp nhất là testosterone tổng hợp. Tỷ lệ giữa 2 đồng vị $\rm{}_6^{12}C$ ($98.98$\%) \& $\rm{}_6^{13}C$ ($1.11$\%) là không đổi đối với testosterone tự nhiên trong cơ thể. Trong khi testosterone tổng hợp (i.e., dopping) có \% số nguyên tử đồng vị $\rm{}_6^{13}C$ ít hơn testosterone tự nhiên. Đây chính là mấu chốt của xét nghiệm CIR (Carbon Isotope Ratio -- Tỷ lệ đồng vị carbon) -- 1 xét nghiệm với mục đích xác định xem vận động viên có sử dụng dopping hay không. Giả sử, thực hiện phân tích CIR đối với 1 vận động viên thu được kết quả \% số nguyên tử đồng vị $\rm{}_6^{12}C$ là $x$ \& $\rm{}_6^{13}C$ là $y$. Từ tỷ lệ đó, người ta tính được nguyên tử khối trung bình của carbon trong mẫu phân tích có giá trị là $12.0098$.'' -- \cite[p. 24]{SGK_Hoa_Hoc_10_Chan_Troi_Sang_Tao}

``\textit{Phổ khối} hay \textit{phổ khối lượng} chủ yếu được sử dụng để xác định nguyên tử khối, phân tử khối của các chất \& hàm lượng các đồng vị bền của 1 nguyên tố. Ngày nay, phương pháp này được sử dụng rộng rãi trong nhiều lĩnh vực khác nhau với các ứng dụng chính như: xác định khối lượng tương đối; nhận dạng, định danh, \& xác định cấu trúc các chuỗi peptide, protein; nghiên cứu đồng vị; định tính, định lượng trong các mẫu sinh học, thực phẩm, nông thủy sản, môi trường; $\ldots$ Trong phổ khối lượng của mẫu chất chứa chlorine sẽ xuất hiện 2 tín hiệu có giá trị m\texttt{/}z\footnote{m là \textit{khối lượng}, z là số đơn vị điện tích của ion. Đối với phổ khối lượng của chlorine ($\rm z = 1$), do đó m\texttt{/}z có giá trị bằng khối lượng nguyên tử hay nguyên tử khối.} bằng $35$ \& $37$ ứng với $\rm{}^{35}Cl$ \& $\rm{}^{37}$ có cường độ tương ứng với tỷ lệ xấp xỉ là $3:1$. Do vậy, đồng vị $\rm{}_{17}^{35}Cl$ chiếm khoảng $75.76$\% \& đồng vị $\rm{}_{17}^{37}C$ chiếm khoảng $24.24$\% về số nguyên tử trong tự nhiên. Từ đó người ta tính được nguyên tử khối trung bình của chlorine.'' -- \cite[p. 24]{SGK_Hoa_Hoc_10_Chan_Troi_Sang_Tao}

``Silicon là nguyên tố được sử dụng để chế tạo vật lieej bán dẫn, có vai trò quan trọng trong sản xuất công nghiệp. Trong tự nhiên, nguyên tố này có 3 đồng vị với số khối lần lượt là $28,29,30$.'' ``Trong tự nhiên, magnesium có 3 đồng vị bền là $\rm{}^{24}Mg,{}^{25}Mg,{}^{26}Mg$. Phương pháp phổ khối lượng xác nhận đồng vị $\rm{}^{26}Mg$ chiếm tỷ lệ \% số nguyên tử là $11\%$.'' -- \cite[p. 25]{SGK_Hoa_Hoc_10_Chan_Troi_Sang_Tao}

%------------------------------------------------------------------------------%

\section{Cấu Trúc Lớp Vỏ Electron của Nguyên Tử}
\textbf{Nội dung.} \textit{So sánh mô hình của Rutherford--Bohr với mô hình hiện đại mô tả sự chuyển động của electron trong nguyên tử; orbital nguyên tử (AO), mô tả được hình dạng của AO (s, p), số lượng electron trong 1 AO; lớp, phân lớp electron \& mối quan hệ về số lượng phân lớp trong 1 lớp, số lượng AO trong 1 phân lớp; cấu hình electron nguyên tử theo lớp, phân lớp electron \& theo ô orbital khi biết số hiệu nguyên tử $\rm Z$, tính chất hóa học cơ bản (kim loại hay phi kim) của nguyên tố tương ứng}.

\subsection{Sự chuyển động của electron trong nguyên tử}

%------------------------------------------------------------------------------%

\chapter{Bảng Tuần Hoàn Các Nguyên Tố Hóa Học}

\section{Cấu Tạo Bảng Tuần Hoàn Các Nguyên Tố Hóa Học}

%------------------------------------------------------------------------------%

\section{Xu Hướng Biến Đổi 1 Số Tính Chất của Nguyên Tử Các Nguyên Tố, Thành Phần \& 1 Số Tính Chất của Hợp Chất Trong 1 Chu Kỳ \& Nhóm}

%------------------------------------------------------------------------------%

\section{Định Luật Tuần Hoàn -- Ý Nghĩa của Bảng Tuần Hoàn Các Nguyên Tố Hóa Học}

%------------------------------------------------------------------------------%

\chapter{Liên Kết Hóa Học}

\section{Quy Tắc Octet}

%------------------------------------------------------------------------------%

\section{Liên Kết Ion}

%------------------------------------------------------------------------------%

\section{Liên Kết Cộng Hóa Trị}

%------------------------------------------------------------------------------%

\section{Liên Kết Hyđrogen \& Tương Tác van der Waals}

%------------------------------------------------------------------------------%

\chapter{Phản Ứng Oxi Hóa -- Khử}

\section{Phản Ứng Oxi Hóa -- Khử \& Ứng Dụng Trong Cuộc Sống}

%------------------------------------------------------------------------------%

\chapter{Năng Lượng Hóa Học}

\section{Enthalpy Tạo Thành \& Biến Thiên Enthalpy của Phản Ứng Hóa Học}

%------------------------------------------------------------------------------%

\section{Tính Biến Thiên Enthalpy của Phản Ứng Hóa Học}

%------------------------------------------------------------------------------%

\chapter{Tốc Độ Phản Ứng Hóa Học}

\section{Phương Trình Tốc Độ Phản Ứng \& Hằng Số Tốc Độ Phản Ứng}

%------------------------------------------------------------------------------%

\section{Các Yếu Tố Ảnh Hưởng Đến Tốc Độ Phản Ứng Hóa Học}

%------------------------------------------------------------------------------%

\chapter{Nguyên Tố Nhóm VIIA -- Halogen}

\section{Tính Chất Vật Lý \& Hóa Học Các Đơn Chất Nhóm VIIA}

%------------------------------------------------------------------------------%

\section{Hyđrogen Halide \& 1 Số Phản Ứng của Ion Halide} 

%------------------------------------------------------------------------------%

\printbibliography[heading=bibintoc]
	
\end{document}