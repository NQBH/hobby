\documentclass[oneside]{book}
\usepackage[backend=biber,natbib=true,style=authoryear]{biblatex}
\addbibresource{/home/hong/1_NQBH/reference/bib.bib}
\usepackage[utf8]{vietnam}
\usepackage{tocloft}
\renewcommand{\cftsecleader}{\cftdotfill{\cftdotsep}}
\usepackage[colorlinks=true,linkcolor=blue,urlcolor=red,citecolor=magenta]{hyperref}
\usepackage{amsmath,amssymb,amsthm,mathtools,float,graphicx,algpseudocode,algorithm,tcolorbox,tikz,tkz-tab,diagbox}
\DeclareMathOperator{\arccot}{arccot}
\usepackage[inline]{enumitem}
\allowdisplaybreaks
\numberwithin{equation}{section}
\newtheorem{assumption}{Assumption}[section]
\newtheorem{nhanxet}{Nhận xét}[section]
\newtheorem{conjecture}{Conjecture}[section]
\newtheorem{corollary}{Corollary}[section]
\newtheorem{hequa}{Hệ quả}[section]
\newtheorem{definition}{Definition}[section]
\newtheorem{dinhnghia}{Định nghĩa}[section]
\newtheorem{example}{Example}[section]
\newtheorem{vidu}{Ví dụ}[section]
\newtheorem{lemma}{Lemma}[section]
\newtheorem{notation}{Notation}[section]
\newtheorem{principle}{Principle}[section]
\newtheorem{problem}{Problem}[section]
\newtheorem{baitoan}{Bài toán}[section]
\newtheorem{proposition}{Proposition}[section]
\newtheorem{menhde}{Mệnh đề}[section]
\newtheorem{nguyenly}{Nguyên lý}[section]
\newtheorem{question}{Question}[section]
\newtheorem{quytac}{Quy tắc}[section]
\newtheorem{cauhoi}{Câu hỏi}[section]
\newtheorem{remark}{Remark}[section]
\newtheorem{luuy}{Lưu ý}[section]
\newtheorem{dinhluat}{Định luật}[section]
\newtheorem{theorem}{Theorem}[section]
\newtheorem{dinhly}{Định lý}[section]
\usepackage[left=0.5in,right=0.5in,top=1.5cm,bottom=1.5cm]{geometry}
\usepackage{fancyhdr}
\pagestyle{fancy}
\fancyhf{}
\lhead{\small \textsc{Sect.} ~\thesection}
\rhead{\small \nouppercase{\leftmark}}
\renewcommand{\sectionmark}[1]{\markboth{#1}{}}
\cfoot{\thepage}
\def\labelitemii{$\circ$}

\title{Some Topics in Elementary Chemistry\texttt{/}Grade 10}
\author{Nguyễn Quản Bá Hồng\footnote{Independent Researcher, Ben Tre City, Vietnam\\e-mail: \texttt{nguyenquanbahong@gmail.com}; website: \url{https://nqbh.github.io}.}}
\date{\today}

\begin{document}
\frontmatter
\maketitle
\setcounter{secnumdepth}{4}
\setcounter{tocdepth}{3}
\tableofcontents
\newpage

%------------------------------------------------------------------------------%

\mainmatter

\chapter*{Preface}

Tóm tắt kiến thức Hóa học lớp 10 theo chương trình giáo dục của Việt Nam \& một số chủ đề nâng cao.

\section*{Notation, Abbreviation, Convention}
\begin{itemize}
	\item askt: ánh sáng khuếch tán.
	\item asmt: ánh sáng mặt trời.
	\item đpnc: điện phân nóng chảy.
	\item $\uparrow,\downarrow$: sản phẩm khí, sản phẩm rắn (kết tủa).
	\item (s): solid -- chất rắn.
	\item (l): liquid -- chất lỏng.
	\item (g): gas, chất khí (hơi).
	\item (aq): aqueous\footnote{\textbf{aqueous} [a] [usually before noun] (\textit{specialist}) containing or involving water.} -- chất tan trong nước (dung dịch).
	\item $E_{\rm a}$: activation\footnote{\textbf{activation} [n] [uncountable] \textbf{activation (of something)} the fact or process of making something such as a device or chemical process start working.} energy\footnote{\textbf{energy} [n] \textbf{1.} [uncountable, countable] the ability of matter or radiation to perform work because of its mass, movement, electrical charge, etc.; \textbf{2.} [uncountable] a source of power that can be used by somebody\texttt{/}something, e.g. to provide light \& heat, or to work machines; \textbf{3.} [uncountable] the effort needed to do work or other physical or mental activities; \textbf{4.} (\textbf{energies}) [plural] the physical \& mental effort that you use to do something.} -- năng lượng hoạt hóa.
	\item $E_{\rm b}$: bond\footnote{\textbf{bond} [n] \textbf{1.} something that forms a connection between people or groups, such as a feeling of friendship or shared ideas \& experiences; \textbf{2.} the way in which atoms are held together in a chemical compound; \textbf{3.} an agreement by a government or a company to pay somebody interest on the money they have lent after a particular period of time; a document containing this agreement; \textbf{4.} the way in which 2 surfaces are joined together, often using glue; [v] \textbf{1.} [transitive, intransitive] to join 2 things firmly together; to join firmly to something else; \textbf{2.} [transitive] to join atoms together by a chemical bond; \textbf{3.} [intransitive] \textbf{bond (with somebody)} to develop or create a relationship of trust with somebody.} energy -- năng lượng liên kết.
	\item SATP: standard ambient\footnote{\textbf{ambient} [a] [only before noun] in the surrounding area; on all sides.} temperature \& pressure -- điều kiện chuẩn về nhiệt độ \& áp suất.
	\item $\Delta H$: enthalpy change -- biến thiên enthalpy.
	\item $\Delta_{\rm f}H_{298}^0$: standard enthalpy of formation\footnote{\textbf{formation} [n] \textbf{1.} [uncountable] the action of forming something; the process of being formed; \textbf{2.} [countable] a thing that has been formed, especially in a particular place or in a particular way; \textbf{3.} [countable, uncountable] a particular arrangement or pattern of people or things.} at $298$ K -- enthalpy tạo thành chuẩn ở $298$ K.
	\item $\Delta_{\rm r}H_{298}^0$: standard enthalpy change of reaction\footnote{\textbf{reaction} [n] \textbf{1.} [countable, uncountable] what you do, say or think as a result of something that has happened; \textbf{2.} [countable] (\textit{chemistry}) a chemical change produced by 2 or more substances acting on each other; \textbf{3.} [countable, uncountable] (\textit{medical}) a response by the body, usually a bad one, to something such as a drug or a chemical substance; \textbf{4.} [uncountable, countable] (\textit{physics}) a force shown by something in response to another force, which is of equal strength \& acts in the opposite direction; \textbf{5.} [countable, usually singular] \textbf{reaction (against something)} a change in people's attitudes or behavior caused by strong disapproval of other very different attitudes; \textbf{6.} [uncountable] opposition to social or political progress or change; \textbf{7.} (\textbf{reactions}) [plural] the ability to move quickly in response to something, especially if in danger.} at $298$ K -- biến thiên enthalpy chuẩn của phản ứng ở $298$ K.
\end{itemize}
``Từ lâu, hóa học được mệnh danh là ``trung tâm của các ngành khoa học'' vì nhiều ngành khoa học như vật lý, sinh học, y học, khoa học Trái Đất, $\ldots$ đều lấy hóa học làm nền tảng cho sự phát triển. Hóa học cũng là cơ sở phát triển cho nhiều ngành công nghiệp khác như vật liệu, luyện kim, điện tử, dược phẩm, dầu khí, $\ldots$ Trong cuộc sống hằng ngày, hóa học hiện diễn ở khắp mọi nơi. Từ lương thực -- thực phẩm, đồ dùng thiết yếu trong gia đình, dụng cụ học tập, thuốc chữa bệnh, nguyên liệu sản xuất, $\ldots$ đến hương thơm quyến rũ của nước hoa, mỹ phẩm, $\ldots$ đều là những sản phẩm của hóa học.'' -- \cite[p. 3]{SGK_Hoa_Hoc_10_Chan_Troi_Sang_Tao}

%------------------------------------------------------------------------------%

\section{Nhập Môn Hóa Học}
\textbf{Nội dung.} \textit{Đối tượng nghiên cứu của hóa học, vai trò của hóa học đối với đời sống, sản xuất, $\ldots$, phương pháp học tập \& nghiên cứu hóa học}.

\subsection{Đối tượng nghiên cứu của hóa học}

\begin{dinhnghia}[Hóa học]
	``\emph{Hóa học} là ngành khoa học thuộc lĩnh vực khoa học tự nhiên, nghiên cứu về thành phần, cấu trúc, tính chất, \& sự biến đổi của chất cũng như ứng dụng của chúng.'' -- \cite[p. 7]{SGK_Hoa_Hoc_10_Chan_Troi_Sang_Tao}
\end{dinhnghia}
``Khi đốt nến (được làm bằng paraffin), nếu chảy ra ở dạng lỏng, thấm vào bấc, cháy trong không khí, sinh ra khí carbon dioxide \& hơi nước.'' -- \cite[p. 7]{SGK_Hoa_Hoc_10_Chan_Troi_Sang_Tao}

\subsubsection{Vai trò của hóa học trong đời sống \& sản xuất}
``Hóa học có vai trò quan trọng trong đời sống, sản xuất \& nghiên cứu khoa học.'' -- \cite[p. 8]{SGK_Hoa_Hoc_10_Chan_Troi_Sang_Tao}

\subsubsection{Phương pháp học tập Hóa học}
``Phương pháp học tập hóa học nhằm phát triển năng lực hóa học, bao gồm:
\begin{enumerate*}
	\item[\textbf{1.}] Phương pháp tìm hiểu lý thuyết;
	\item[\textbf{2.}] Phương pháp học tập thông qua thực hành thí nghiệm;
	\item[\textbf{3.}] Phương pháp luyện tập, ôn tập;
	\item[\textbf{4.}] Phương pháp học tập trải nghiệm.'' -- \cite[p. 9]{SGK_Hoa_Hoc_10_Chan_Troi_Sang_Tao}
\end{enumerate*}

\subsubsection{Phương pháp nghiên cứu Hóa học}
``Khi nghiên cứu 1 vấn đề hóa học, chúng ta cần có phương pháp nghiên cứu. Không có phương pháp nào là chung cho mọi nghiên cứu. Tùy vào mục đích \& đối tượng nghiên cứu mà chúng ta lựa chọn phương pháp cho phù hợp.
\begin{enumerate}
	\item \textbf{Phương pháp nghiên cứu lý thuyết} là sử dụng những định luật, nguyên lý, quy tắc, cơ chế, mô hình, $\ldots$ cũng như các kết quả nghiên cứu đã có để tiếp tục làm rõ những vấn đề của lý thuyết hóa học.
	\item \textbf{Phương pháp nghiên cứu thực nghiệm} là nghiên cứu những vấn đề dựa trên kết quả thí nghiệm, khảo sát, thu thập số liệu, phân tích, định lượng, $\ldots$
	\item \textbf{Phương pháp nghiên cứu ứng dụng} nhằm mục đích giải quyết các vấn đề hóa học được ứng dụng trong các lĩnh vực khác nhau.'' -- \cite[p. 10]{SGK_Hoa_Hoc_10_Chan_Troi_Sang_Tao}
\end{enumerate}

\begin{vidu}
	``Để nghiên cứu thành phần hóa học \& bước đầu ứng dụng tinh dầu tràm trà (\emph{Melaleuca alternifolia}) trong sản xuất nước súc miệng, các nhà nghiên cứu đã thực hiện theo các bước được mô tả trong  \cite[Hình 1.12: \textsf{Các bước thực hiện trong đề tài nghiên cứu thành phần hóa học \& bước đầu ứng dụng tinh dầu tràm trà trong sản xuất nước súc miệng}, p. 10]{SGK_Hoa_Hoc_10_Chan_Troi_Sang_Tao}:
	\begin{enumerate*}
		\item[\textbf{1.}] Nghiên cứu thành phần hóa học \& ứng dụng của tinh dầu tràm trà làm nước súc miệng qua các công trình khoa học trên các tạp chí đã được xuất bản.
		\item[\textbf{2.}] Đặt giả thuyết: tinh dầu tràm trà có khả năng kháng khuẩn.
		\item[\textbf{3.}] Thí nghiệm chiết xuất tinh dầu bằng phương pháp chưng cất lôi cuốn hơi nước.
		\item[\textbf{4.}] Khảo sát hoạt tính kháng khuẩn của sản phẩm nước súc miệng từ tinh dầu tràm trà.'' -- \cite[p. 10]{SGK_Hoa_Hoc_10_Chan_Troi_Sang_Tao}
	\end{enumerate*}
\end{vidu}
``\textit{Phương pháp nghiên cứu hóa học} bao gồm: nghiên cứu lý thuyết, nghiên cứu thực nghiệm \& nghiên cứu ứng dụng. Phương pháp nghiên cứu hóa học thường bao gồm 1 số bước:
\begin{enumerate*}
	\item[\textbf{1.}] Xác định vấn đề nghiên cứu;
	\item[\textbf{2.}] Nêu giả thuyết khoa học;
	\item[\textbf{3.}] Thực hiện nghiên cứu (lý thuyết, thực nghiệm, ứng dụng);
	\item[\textbf{4.}] Viết báo cáo: thảo luận kết quả \& kết luận vấn đề.'' -- \cite[p. 11]{SGK_Hoa_Hoc_10_Chan_Troi_Sang_Tao}
\end{enumerate*}
``\textit{Mưa acid} là 1 thuật ngữ chung chỉ sự tích lũy của các chất gây ô nhiễm, có khả năng chuyển hóa trong nước mưa tạo nên môi trường acid. Các chất gây ô nhiễm chủ yếu là khí $\rm SO_2$ \& $\rm NO_x$ thải ra từ các quá trình sản xuất trong đời sống, đặc biệt là quá trình đối cháy than đá, dầu mỏ, \& các nhiên liệu tự nhiên khác. Hiện tượng này gây ảnh hưởng trực tiếp đến đời sống con người, động -- thực vật \& có thể làm thay đổi thành phần của nước các sông, hồ, giết chết các loài cá \& những sinh vật khác, đồng thời hủy hoại các công trình kiến trúc.'' -- \cite[p. 11]{SGK_Hoa_Hoc_10_Chan_Troi_Sang_Tao}

``Hóa học là 1 ngành khoa học thuộc lĩnh vực khoa học tự nhiên, kết hợp chặt chẽ giữa lý thuyết \& thực nghiệm. Hóa học còn được gọi là ``khoa học trung tâm'' vì nó là cầu nối giữa các ngành khoa học tự nhiên khác như vật lý, địa chất, \& sinh học, $\ldots$ Theo truyền thống, hóa học được chia thành 5 chuyên ngành chính, bao gồm: hóa lý thuyết \& hóa lý, hóa vô cơ, hóa hữu cơ, hóa phân tích, hóa sinh.'' -- \cite[p. 11]{SGK_Hoa_Hoc_10_Chan_Troi_Sang_Tao}

%------------------------------------------------------------------------------%

\chapter{Cấu Tạo Nguyên Tử}

\section{Thành Phần của Nguyên Tử}
\textbf{Nội dung.} \textit{Thành phần của nguyên tử, so sánh khối lượng của electron với proton \& neutron, kích thước của hạt nhân với kích thước nguyên tử}.

\subsection{Thành phần cấu tạo nguyên tử}
``Từ thời cổ Hy Lạp, nhà triết học Democritous (460--370 BC) cho rằng mọi vật chất được tạo thành từ các phần tử rất nhỏ được gọi là ``atomos'', i.e., không thể phá hủy \& không thể chia nhỏ hơn được nữa. Đến giữa thế kỷ XIX, các nhà khoa học cho rằng: các chất đều được cấu tạo nên từ những hạt rất nhỏ, không thể phân chia được nữa, gọi là \textit{nguyên tử}. Vào cuối thế kỷ XIX, đầu thế kỷ XX, bằng những nghiên cứu thực nghiệm, các nhà khoa học đã chứng minh sự tồn tại của nguyên tử \& nguyên tử có cấu tạo phức tạp.'' -- \cite[p. 13]{SGK_Hoa_Hoc_10_Chan_Troi_Sang_Tao}

\begin{figure}[H]
	\centering
	\includegraphics[scale=0.15]{mo_hinh_nguyen_tu}
	\caption{Mô hình nguyên tử, \cite[Hình 2.1, p. 13]{SGK_Hoa_Hoc_10_Chan_Troi_Sang_Tao}.}
\end{figure}
``Nguyên tử gồm hạt nhân chứa proton, neutron \& vỏ nguyên tử chứa electron.'' -- \cite[p. 14]{SGK_Hoa_Hoc_10_Chan_Troi_Sang_Tao}

\subsection{Sự tìm ra Electron}
``Năm 1897, nhà vật lý người Anh Joseph John Thomson (1856--1940) thực hiện thí nghiệm phóng điện trong 1 ống thủy tinh gần như chân không (gọi là \textit{ống tia âm cực}). Ông quan sát thấy màn huỳnh quang trong ống phát sáng do những tia phat phát ra từ cực âm (gọi là \textit{tia cực âm}) \& những tia này bị hút về cực dương của trường điện  (\cite[Hình 2.2: \textsf{Thí nghiệm của Thomson}, p. 13]{SGK_Hoa_Hoc_10_Chan_Troi_Sang_Tao}), chứng tỏ chúng tích điện âm. Đó chính là chùm các hạt \textit{electron}.'' ``Trong nguyên tử tồn tại 1 loại hạt có khối lượng \& mang điện tích âm, được gọi là \textit{electron} (ký hiệu là $e$). Hạt electron có: Điện tích: $q_{\rm e} = -1.602\cdot 10^{-19}$ C (Coulomb)\footnote{See, e.g., \cite{SGK_Vat_Ly_11_co_ban, SGK_Vat_Ly_11_nang_cao}.}. Khối lượng: $m_{\rm e} = 9.11\cdot 10^{-28}$ g. Người ta chưa phát hiện được điện tích nào nhỏ hơn $1.602\cdot 10^{-19}$ C nên nó được dùng làm điện tích đơn vị, điện tích của electron được quy ước là $-1$.'' -- \cite[p. 14]{SGK_Hoa_Hoc_10_Chan_Troi_Sang_Tao}

\subsubsection{Thí nghiệm giọt dầu của Millikan}
``Năm 1909, nhà vật lý thực nghiệm người Mỹ là R. A. Millikan đã tiến hành phun các giọt dầu vào 1 hộp trong suốt. Bên trong hộp chứa 2 tấm kim loại được nối vào nguồn điện 1 chiều với 1 đầu tích điện âm ($-$) \& 1 đầu tích điện dương ($+$). Trong hộp còn có thiết bị phát ra 1 chùm tia R\"ontgen (tia X) để ion hóa các giọt dầu (cấp cho nó 1 điện tích). Tia X có khả năng đánh bật các electron khỏi không khí giữa các tấm kim loại \& các electron sẽ bám vào các giọt dầu, làm chúng tích điện âm. Bằng cách thay đổi cường độ trường điện, Millikan có thể kiểm soát tốc độ rơi của các giọt dầu. Chuyển động của các giọt dầu trong thiết bị phụ thuộc vào điện tích của mỗi giọt \& vào trường điện. Millikan đã quan sát các giọt dầu bằng kính thiên văn. Millikan có thể làm cho các giọt dầu rơi chậm hơn, nhanh hơn, hoặc khiến chúng dừng lại khi thay đổi cường độ của trường điện. Từ những quan sát của mình, ông đã tính được điện tích \& khối lượng của electron.'' -- \cite[p. 15]{SGK_Hoa_Hoc_10_Chan_Troi_Sang_Tao} (xem mô hình \textsf{Thí nghiệm giọt dầu của Millikan}).

\subsection{Sự khám phá hạt nhân nguyên tử}
``Năm 1911, nhà vật lý người New Zealand là Ernest Rutherford (1871--1937) đã tiến hành bắn phá 1 chùm hạt alpha (ký hiệu là $\alpha$, hạt nhân của nguyên tử helium, mang điện tích $+2$, có khối lượng gấp khoảng $7500$ lần khối lượng electron) lên 1 lá vàng siêu mỏng (\cite[Hình 2.3: \textsf{Thí nghiệm khám phá hạt nhân nguyên tử của Rutherford}, p. 16]{SGK_Hoa_Hoc_10_Chan_Troi_Sang_Tao}) \& quan sát đường đi của chúng sau khi bắn phá bằng màn huỳnh quang (zinc sulfide, ZnS).'' -- \cite[p. 15]{SGK_Hoa_Hoc_10_Chan_Troi_Sang_Tao}

``Nguyên tử có cấu tạo rỗng, gồm hạt nhân ở trung tâm \& lớp vỏ là các electron chuyển động xung quanh hạt nhân. Nguyên tử trung hòa về điện: \textit{số đơn vị điện tích dương của hạt nhân bằng số đơn vị điện tích âm của các electron trong nguyên tử}.'' -- \cite[p. 16]{SGK_Hoa_Hoc_10_Chan_Troi_Sang_Tao}

\subsection{Cấu tạo hạt nhân nguyên tử}
``Vào năm 1918, khi bắn phá hạt nhân nguyên tử nitrogen bằng các hạt $\alpha$ (thực hiện trong máy gia tốc hạt), Rutherford đã nhận thấy sự xuất hiện hạt nhân nguyên tử oxygen \& 1 loại hạt mang 1 đơn vị điện tích dương (${\rm e}_0$ hay $+1$), đó là \textit{proton} (ký hiệu là p). Năm 1932, khi dùng các hạt $\alpha$ để bắn phá hạt nhân nguyên tử beryllium, J. Chadwick nhận thấy sự xuất hiện của 1 loại hạt có khối lượng xấp xỉ hạt proton, nhưng không mang điện, ông gọi chúng là \textit{neutron} (ký hiệu là n).'' -- \cite[pp. 16--17]{SGK_Hoa_Hoc_10_Chan_Troi_Sang_Tao}. ``Hạt nhân nguyên tử gồm 2 loại hạt là proton \& neutron. Proton mang điện tích dương ($+1$) \& neutron không mang điện. Proton \& neutron có khối lượng gần bằng nhau.'' -- \cite[p. 17]{SGK_Hoa_Hoc_10_Chan_Troi_Sang_Tao}

\subsection{Kích thước \& khối lượng nguyên tử}
``Nếu hình dung hạt nhân là 1 khối cầu có kích thước như viên bi thì nguyên tử sẽ là 1 khối cầu có kích thước bằng 1 sân bóng đá.'' -- \cite[p. 17]{SGK_Hoa_Hoc_10_Chan_Troi_Sang_Tao}. ``Đơn vị nanomet (nm) hay angstrom ($\mathring{\rm A}$) thường được sử dụng để biểu thị kích thước nguyên tử. $\rm 1\ nm = 10^{-9}\ m$, $\rm 1\ \mathring{\rm A} = 10^{-10}\ m$, $\rm 1\ nm = 10\ \mathring{\rm A}$.'' ``Nếu xem nguyên tử như 1 quả cầu, trong đó các electron chuyển động rất nhanh xung quanh hạt nhân thì nguyên tử đó có đường kính khoảng $10^{-10}$ m \& đường kính hạt nhân khoảng $10^{-14}$ m. Như vậy, đường kính của nguyên tử lớn hơn đường kính của hạt nhân khoảng $10^4$ lần.'' -- \cite[p. 18]{SGK_Hoa_Hoc_10_Chan_Troi_Sang_Tao}

``Những hiểu biết của nhân loại về vũ trụ \& thế giới xung quanh ngày càng phát triển. Người Hy Lạp cổ đại lần đầu tiên đoán được sự tồn tại của các hạt gọi là \textit{nguyên tử}. Khoảng 1500 năm sau, người ta đã chứng minh được sự tồn tại của nguyên tử \& xem chúng là những hạt nhỏ nhất, tạo nên vật chất. Sau đó không lâu, người ta phát hiện ra nguyên tử được tạo thành từ 3 loại hạt cơ bản là proton, neutron, \& electron. Tuy nhiên, các hạt này vẫn chưa phải những hạt nhỏ nhất trong vũ trụ. Ngày nay các công trình nghiên cứu cho thấy proton \& neutron được tạo thành bởi các hạt nhỏ hơn, gọi là \textit{hạt quark}.'' -- \cite[p. 18]{SGK_Hoa_Hoc_10_Chan_Troi_Sang_Tao}

\begin{table}[H]
	\centering
	\begin{tabular}{|c|c|c|c|}
		\hline
		\textbf{Hạt} & \textbf{Điện tích tương đối} & \textbf{Khối lượng (amu)} & \textbf{Khối lượng (g)} \\
		\hline
		\textbf{p} & $+1$ & $\approx 1$ & $1.673\cdot 10^{-24}$ \\
		\hline
		\textbf{n} & $0$ & $\approx 1$ & $1.675\cdot 10^{-24}$ \\
		\hline 
		\textbf{e} & $-1$ & $\frac{1}{1840}\approx 0.00055$ & $9.11\cdot 10^{-28}$ \\
		\hline
	\end{tabular}
	\caption{1 số tính chất của các loại hạt cơ bản trong nguyên tử, \cite[Bảng 2.1, p. 18]{SGK_Hoa_Hoc_10_Chan_Troi_Sang_Tao}.}
\end{table}
``Để biểu thị khối lượng của nguyên tử, các hạt proton, neutron, \& electron, người ta dùng đơn vị \textit{khối lượng nguyên tử}, ký hiệu là amu. 1 amu bằng $\frac{1}{12}$ khối lượng nguyên tử của carbon $-12$. $\rm 1\ amu = 1.66\cdot 10^{-24}\ g$.'' -- \cite[p. 18]{SGK_Hoa_Hoc_10_Chan_Troi_Sang_Tao}

``Khối lượng của nguyên tử gần bằng khối lượng hạt nhân do khối lượng của các electron không đáng kể so với khối lượng của proton \& neutron.'' -- \cite[p. 19]{SGK_Hoa_Hoc_10_Chan_Troi_Sang_Tao}

%------------------------------------------------------------------------------%

\section{Nguyên Tố Hóa Học}
\textbf{Nội dung.} \textit{Nguyên tố hóa học, số hiệu nguyên tử \& ký hiệu nguyên tử, đồng vị, nguyên tử khối, nguyên tử khối trung bình (theo amu) dựa vào khối lượng nguyên tử \& \% số nguyên tử của các đồng vị theo phổ khối lượng được cung cấp}.

\subsection{Hạt nhân nguyên tử}
``Số đơn vị điện tích hạt nhân (Z) $=$ số proton (P) $=$ số electron (E). Điện tích hạt nhân $=$ $+$Z.'' -- \cite[p. 20]{SGK_Hoa_Hoc_10_Chan_Troi_Sang_Tao}

``Số khối (A) $=$ số proton (P) + số neutron (N).'' -- \cite[p. 21]{SGK_Hoa_Hoc_10_Chan_Troi_Sang_Tao}

\subsection{Nguyên tố hóa học}
``\textit{Số hiệu nguyên tử} của 1 nguyên tố được quy ước bằng số đơn vị điện tích hạt nhân nguyên tử của nguyên tố đó. Số hiệu nguyên tử (ký hiệu là Z) cho biết: số proton trong hạt nhân nguyên tử, số electron trong nguyên tử.  

\begin{dinhnghia}[Số hiệu nguyên tử]
	Số đơn vị điện tích hạt nhân nguyên tử của 1 nguyên tố được gọi là \emph{số hiệu nguyên tử (Z)} của nguyên tố đó. Mỗi nguyên tố hóa học có 1 số hiệu nguyên tử.
\end{dinhnghia}
Năm 1913, nhà vật lý người Anh là H. Moseley đã thực hiện thí nghiệm khảo sát bản chất tự nhiên của tia X. Moseley sử dụng 1 chùm tia electron có năng lượng cao để bắn vào các tấm kim loại khác nhau làm anode \& thu được tia X. Moseley phát hiện ra rằng, bước sóng của tia X luôn không đổi đối với 1 kim loại nhất định \& thay đổi khi thay anode bằng những kim loại khác. Từ đó, ông cho rằng bước sóng này phụ thuộc vào số proton trong nguyên tử của mỗi số nguyên tố kim loại được dùng làm anode.'' -- \cite[p. 21]{SGK_Hoa_Hoc_10_Chan_Troi_Sang_Tao} (xem \textsf{Mô hình thí nghiệm khảo sát bản chất tự nhiên của tia X của Henry Moseley}).

``Protium, deuterium, \& tritium là các loại nguyên tử của nguyên tố hydrogen.

\begin{dinhnghia}[Nguyên tố hóa học]
	\emph{Nguyên tố hóa học} là tập hợp những nguyên tử có cùng điện tích hạt nhân.
\end{dinhnghia}
\textit{Số đơn vị điện tích hạt nhân nguyên tử} (còn được gọi là \textit{số hiệu nguyên tử}) của 1 nguyên tố hóa học \& số khối được xem là những đặc trưng cơ bản của nguyên tử. Để ký hiệu nguyên tử, người ta thường ghi các chỉ số đặc trưng ở bên trái ký hiệu nguyên tố với số khối A ở trên, số hiệu Z ở phía dưới. Ký hiệu nguyên tử được sử dụng để biểu thị nguyên tử của 1 nguyên tố hóa học. $\rm{}_Z^AX$ (A: số khối, Z: số hiệu nguyên tử, X: ký hiệu nguyên tố hóa học)'' -- \cite[p. 22]{SGK_Hoa_Hoc_10_Chan_Troi_Sang_Tao}

\subsection{Đồng vị}
``Các nguyên tử của cùng 1 nguyên tố hóa học có thể có số khối khác nhau. Sở dĩ như vậy vì hạt nhân của các nguyên tử đó có cùng số proton, nhưng có thể khác số neutron. Những nguyên tử này được gọi là \textit{đồng vị} của 1 nguyên tố hóa học. Trong tự nhiên, hầu hết các nguyên tố được tìm thấy dưới dạng hỗn hợp của các đồng vị. 1 nguyên tố hóa học dù ở dạng đơn chất hay hợp chất thì tỷ lệ giữa các đồng vị của nguyên tố này là không đổi. E.g., các quả chuối đều chứa nguyên tố potassium (K) trong thành phần dinh dưỡng của chúng. Chúng có thể khác nhau về kích thước, hình dáng, mùi vị cũng như được thu hoạch ở những vị trí địa lý khác nhau nhưng đều chứa $93.26$\% số nguyên tử $\rm{}_{19}^{39}K$, $6.73$\% số nguyên tử $\rm{}_{19}^{41}K$, \& $0.01$\% số nguyên tử $\rm{}_{19}^{40}K$ trong tổng số nguyên tử potassium có trong chúng. Ngoài những đồng vị bền, các nguyên tố hóa học còn có 1 số đồng bị không bền, gọi là \textit{đồng vị phóng xạ}, được sử dụng nhiều trong đời sống, y học, nghiên cứu khoa học, $\ldots$'' -- \cite[pp. 22--23]{SGK_Hoa_Hoc_10_Chan_Troi_Sang_Tao}

``Kim cương là 1 trong những dạng tồn tại của nguyên tố carbon trong tự nhiên. Nguyên tố này có 2 đồng bị bền với số khối lần lượt là 12 \& 13.'' ``Các đồng vị của 1 nguyên tố hóa học là những nguyên tử có cùng số proton (P), cùng số hiệu nguyên tử (Z), nhưng khác nhau về số neutron (N). Do đó, số khối (A) của chúng khác nhau.'' -- \cite[p. 23]{SGK_Hoa_Hoc_10_Chan_Troi_Sang_Tao}

\subsection{Nguyên tử khối \& nguyên tử khối trung bình}

\begin{dinhnghia}[Nguyên tử khối]
	\emph{Nguyên tử khối} là khối lượng tương đối của nguyên tử.
\end{dinhnghia}
``Khối lượng của 1 nguyên tử bằng tổng khối lượng của proton, neutron, \& electron trong nguyên tử đó. Proton \& neutron đều có khối lượng gần bằng 1 amu, electron có khối lượng nhỏ hơn rất nhiều (khoảng $0.00055$ amu). Do đó, có thể coi \textit{nguyên tử khối có giá trị bằng số khối}. Nguyên tử khối của 1 nguyên tử cho biết khối lượng của nguyên tử đó nặng gấp bao nhiêu lần đơn vị khối lượng nguyên tử (1 amu).'' ``Mỗi nguyên tố thường có nhiều đồng vị, do đó trong thực tế, người ta thường sử dụng giá trị \textit{nguyên tử khối trung bình}. Muốn xác định giá trị nguyên tử khối trung bình của 1 nguyên tố, ta cần phải biết được phần trăm số nguyên tử các đồng vị của nguyên tố đó trong tự nhiên. Người ta thường sử dụng \textit{phương pháp phổ khối lượng} (MassSpectrometry -- MS) để xác định \% số nguyên tử các đồng vị trong tự nhiên của các nguyên tố. Đây cũng là 1 phương pháp quan trọng trong việc phân tích thành phần \& cấu trúc các chất.'' ``Trong tự nhiên, nguyên tố copper có 2 đồng vị với \%  số nguyên tử tương ứn là $\rm{}_{29}^{63}Cu$ ($69.15$\%) \& $\rm{}_{29}^{65}Cu$ ($30.85$\%).'' -- \cite[p. 23]{SGK_Hoa_Hoc_10_Chan_Troi_Sang_Tao}

``Trong tự nhiên, chlorine có 2 đồng vị là $\rm{}_{17}^{35}Cl$ \& ${}_{17}^{37}Cl$ có tỷ lệ \% số nguyên tử tương ứng là $75.76\%$ \& $24.24$\%. Các xác định nguyên tử khối trung bình của chlorine:
\begin{align*}
	\rm\overline{A}_{Cl} = \frac{(A_{35_{Cl}}\cdot\%^{35}Cl) + (A_{37_{Cl}}\cdot\%^{37}Cl)}{100} = \frac{35\cdot 75.76 + 37\cdot 24.24}{100} = 35.48.
\end{align*}

\begin{menhde}
	Công thức tính nguyên tử khối trung bình của nguyên tố X:
	\begin{align*}
		\overline{A}_X = \frac{1}{100}\sum_{i=1}^k a_iA_i,
	\end{align*}
	$\overline{A}_X$ là nguyên tử khối trung bình của $X$, $A_i$ là nguyên tử khối đồng vị thứ $i$, $a_i$ là tỷ lệ \% số nguyên tử đồng vị thứ $i$, $i = 1,\ldots,k$.
\end{menhde}
Trong thể dục thể thao, có 1 số vận động viên sử dụng các loại chất kích thích trong thi đấu, gọi là \textit{dopping}, dẫn đến thành tích đạt được của họ không thật so với năng lực vốn có. 1 trong các loại dopping thường gặp nhất là testosterone tổng hợp. Tỷ lệ giữa 2 đồng vị $\rm{}_6^{12}C$ ($98.98$\%) \& $\rm{}_6^{13}C$ ($1.11$\%) là không đổi đối với testosterone tự nhiên trong cơ thể. Trong khi testosterone tổng hợp (i.e., dopping) có \% số nguyên tử đồng vị $\rm{}_6^{13}C$ ít hơn testosterone tự nhiên. Đây chính là mấu chốt của xét nghiệm CIR (Carbon Isotope Ratio -- Tỷ lệ đồng vị carbon) -- 1 xét nghiệm với mục đích xác định xem vận động viên có sử dụng dopping hay không. Giả sử, thực hiện phân tích CIR đối với 1 vận động viên thu được kết quả \% số nguyên tử đồng vị $\rm{}_6^{12}C$ là $x$ \& $\rm{}_6^{13}C$ là $y$. Từ tỷ lệ đó, người ta tính được nguyên tử khối trung bình của carbon trong mẫu phân tích có giá trị là $12.0098$.'' -- \cite[p. 24]{SGK_Hoa_Hoc_10_Chan_Troi_Sang_Tao}

``\textit{Phổ khối} hay \textit{phổ khối lượng} chủ yếu được sử dụng để xác định nguyên tử khối, phân tử khối của các chất \& hàm lượng các đồng vị bền của 1 nguyên tố. Ngày nay, phương pháp này được sử dụng rộng rãi trong nhiều lĩnh vực khác nhau với các ứng dụng chính như: xác định khối lượng tương đối; nhận dạng, định danh, \& xác định cấu trúc các chuỗi peptide, protein; nghiên cứu đồng vị; định tính, định lượng trong các mẫu sinh học, thực phẩm, nông thủy sản, môi trường; $\ldots$ Trong phổ khối lượng của mẫu chất chứa chlorine sẽ xuất hiện 2 tín hiệu có giá trị m\texttt{/}z\footnote{m là \textit{khối lượng}, z là số đơn vị điện tích của ion. Đối với phổ khối lượng của chlorine ($\rm z = 1$), do đó m\texttt{/}z có giá trị bằng khối lượng nguyên tử hay nguyên tử khối.} bằng $35$ \& $37$ ứng với $\rm{}^{35}Cl$ \& $\rm{}^{37}$ có cường độ tương ứng với tỷ lệ xấp xỉ là $3:1$. Do vậy, đồng vị $\rm{}_{17}^{35}Cl$ chiếm khoảng $75.76$\% \& đồng vị $\rm{}_{17}^{37}C$ chiếm khoảng $24.24$\% về số nguyên tử trong tự nhiên. Từ đó người ta tính được nguyên tử khối trung bình của chlorine.'' -- \cite[p. 24]{SGK_Hoa_Hoc_10_Chan_Troi_Sang_Tao}

``Silicon là nguyên tố được sử dụng để chế tạo vật lieej bán dẫn, có vai trò quan trọng trong sản xuất công nghiệp. Trong tự nhiên, nguyên tố này có 3 đồng vị với số khối lần lượt là $28,29,30$.'' ``Trong tự nhiên, magnesium có 3 đồng vị bền là $\rm{}^{24}Mg,{}^{25}Mg,{}^{26}Mg$. Phương pháp phổ khối lượng xác nhận đồng vị $\rm{}^{26}Mg$ chiếm tỷ lệ \% số nguyên tử là $11\%$.'' -- \cite[p. 25]{SGK_Hoa_Hoc_10_Chan_Troi_Sang_Tao}

%------------------------------------------------------------------------------%

\section{Cấu Trúc Lớp Vỏ Electron của Nguyên Tử}
\textbf{Nội dung.} \textit{So sánh mô hình của Rutherford--Bohr với mô hình hiện đại mô tả sự chuyển động của electron trong nguyên tử; orbital nguyên tử (AO), mô tả được hình dạng của AO (s, p), số lượng electron trong 1 AO; lớp, phân lớp electron \& mối quan hệ về số lượng phân lớp trong 1 lớp, số lượng AO trong 1 phân lớp; cấu hình electron nguyên tử theo lớp, phân lớp electron \& theo ô orbital khi biết số hiệu nguyên tử $\rm Z$, tính chất hóa học cơ bản (kim loại hay phi kim) của nguyên tố tương ứng}.

\subsection{Sự chuyển động của electron trong nguyên tử}
``Theo mô hình nguyên tử của Rutherford--Bohr, các electron chuyển động trên những quỹ đạo hình tròn hay bầu dục xác định xung quanh hạt nhân. Theo mô hình hiện đại, trong nguyên tử, các electron chuyển động rất nhanh xung quanh hạt nhân không theo 1 quỹ đạo xác định, tạo thành đám mây electron.'' -- \cite[p. 27]{SGK_Hoa_Hoc_10_Chan_Troi_Sang_Tao}

``Các electron chuyển động rất nhanh xung quanh hạt nhân với xác suất tìm thấy không giống nhau, tạo thành đám mây electron. Khu vực không gian xung quanh hạt nhân mà tại đó xác suất có mặt (xác suất tìm thấy) electron $\approx90\%$ gọi là \textit{orbital\footnote{\textbf{orbital} [a] [only before noun] connected with the movement in a regular pattern of 1 object around another; [n] (\textit{chemistry, physics}) a region around an atom or molecule in which an electron of a given energy \& momentum can be found.} nguyên tử}.'' ``Dựa trên sự khác nhau về hình dạng \& sự định hướng trong không gian của các orbital, người ta phân loại thành orbital s, orbital p, orbital d, \& orbital f.'' -- \cite[p. 27]{SGK_Hoa_Hoc_10_Chan_Troi_Sang_Tao}

\begin{dinhnghia}
	``\emph{Orbital nguyên tử} (\emph{Atomic Orbital}, abbr., \emph{AO}) là khu vực không gian xung quanh hạt nhân nguyên tử mà tại đó xác suất tìm thấy electron là lớn nhất ($\approx90\%$).''
\end{dinhnghia}
``1 số AO thường gặp: s, p, d, f. Các AO có hình dạng khác nhau: AO s có dạng hình cầu, AO p có dạng hình số 8 nổi, AO d \& f có hình dạng phức tạp.'' -- \cite[p. 28]{SGK_Hoa_Hoc_10_Chan_Troi_Sang_Tao}

\subsection{Lớp \& phân lớp electron}
``Trong nguyên tử, các electron được sắp xếp thành từng lớp \& phân lớp theo năng lượng từ thấp đến cao.'' ``Trong nguyên tử, các electron được sắp xếp thành từng lớp (ký hiệu K, L, M, N, O, P, Q) từ gần đến xa hạt nhân, theo thứ tự từ lớp $n = 1$ đến $n = 7$. Các electron trên cùng 1 lớp có năng lượng gần bằng nhau.'' -- \cite[p. 28]{SGK_Hoa_Hoc_10_Chan_Troi_Sang_Tao}

``Trong 1 phân lớp, các orbital có cùng mức năng lượng, chỉ khác nhau về sự định hướng trong không gian. Số lượng \& hình dạng orbital phụ thuộc vào đặc điểm của mỗi phân lớp electron.'' ``Mỗi lớp electron phân chia thành các phân lớp, được ký hiệu bằng các chữ cái viết thường: s, p, d, f. Các electron thuộc các phân lớp s, p, d, \& f lần lượt có các số AO tương ứng 1, 3, 5, \& 7. Các electron trên cùng 1 phân lớp có năng lượng bằng nhau.'' -- \cite[p. 28]{SGK_Hoa_Hoc_10_Chan_Troi_Sang_Tao}

``Trong 1 phân lớp, các orbital có cùng mức năng lượng, chỉ khác nhau về sự định hướng trong không gian. Số lượng \& hình dạng orbital phụ thuộc vào đặc điểm của mỗi phân lớp electron.'' ``Mỗi lớp electron phân chia thành các phaa lớp, được ký hiệu bằng các chữ cái viết thường: s, p, d, f. Các electron thuộc các phân lớp s, p, d, \& f được gọi tương ứng là các electron s, p, d, \& f. Các phân lớp s, p, d, \& f lần lượt có các số AO tương ứng 1, 3, 5, \& 7. Các electron trên cùng 1 phân lớp có năng lượng bằng nhau. Với 4 lớp đầu (1, 2, 3, 4) số phân lớp trong mỗi lớp bằng số thứ tự của lớp đó.'' -- \cite[p. 28]{SGK_Hoa_Hoc_10_Chan_Troi_Sang_Tao}

\subsection{Cấu hình electron nguyên tử}
``Trong nguyên tử, các electron trên mỗi orbital có 1 mức năng lượng xác định. Người ta gọi mức năng lượng này là \textit{mức năng lượng orbital nguyên tử} (\textit{mức năng lượng AO}). Các electron trên các orbital khác nhau của cùng 1 phân lớp có năng lượng như nhau. E.g., phân lớp 2p có 3 orbital $\rm 2p_x,2p_y,2p_z$; các electron của các orbital p trong phân lớp này tuy có sự định hướng trong không gian khác nhau nhưng chúng có cùng mức năng lượng AO.'' -- \cite[p. 29]{SGK_Hoa_Hoc_10_Chan_Troi_Sang_Tao}

\begin{nguyenly}[Nguyên lý vững bền]
	``Ở trạng thái cơ bản, các electron trong nguyên tử chiếm lần lượt những orbital có mức năng lượng từ thấp đến cao: 1s 2s 2p 3s 3p 4s 3d 4p 5s 4d 5p $\ldots$'' -- \cite[p. 30]{SGK_Hoa_Hoc_10_Chan_Troi_Sang_Tao}
\end{nguyenly}
``Để biểu diễn orbital nguyên tử, người ta sử dụng các ô vuông, gọi là \textit{ô lượng tử} (\cite[Hình 4.8: \textsf{(a) Chiều chuyển động tự quay của electron quanh trục của nó; (b) Cách biểu diễn 2 electron trong 1 orbital}, p. 30]{SGK_Hoa_Hoc_10_Chan_Troi_Sang_Tao}). Mỗi ô lượng tử ứng với 1 AO. Mỗi AO chứa tối đa 2 electron. Nếu trong AO chỉ chứa 1 electron thì electron đó gọi là \textit{electron độc thân} (ký hiệu bởi 1 mũi tên hướng lên $\uparrow$). Ngược lại, nếu AO chứa đủ 2 electron thì các electron đó gọi là \textit{electron ghép đôi} (ký hiệu bởi 2 mũi tên ngược chiều nhau $\uparrow\downarrow$).'' -- \cite[p. 30]{SGK_Hoa_Hoc_10_Chan_Troi_Sang_Tao}. \cite[Hình 4.9: \textsf{(a) Electron ghép đôi \& electron độc thân; (b) Sự sắp xếp electron trên các orbital của nguyên tử oxygen.}, p. 30]{SGK_Hoa_Hoc_10_Chan_Troi_Sang_Tao}

\begin{nguyenly}[Nguyên lý Pauli]
	``Mỗi orbital chỉ chứa tối đa 2 electron \& có chiều tự quay ngược nhau.'' -- \cite[p. 30]{SGK_Hoa_Hoc_10_Chan_Troi_Sang_Tao}
\end{nguyenly}
``Dựa vào nguyên lý Pauli, ta dễ dàng xác định được số AO \& số electron tối đa trong mỗi phân lớp \& trong mỗi lớp theo Bảng \ref{tab:so AO & so electron toi da cua cac lop n = 1 den n = 4}.'' -- \cite[p. 30]{SGK_Hoa_Hoc_10_Chan_Troi_Sang_Tao}

\begin{table}[H]
	\centering
	\begin{tabular}{|c|c|l|l|p{4cm}|p{3.5cm}|}
		\hline
		$n$ & \textbf{Tên lớp} & \textbf{Tên phân lớp} & \textbf{Số AO trong mỗi phân lớp} & \textbf{Số electron tối đa trong mỗi phân lớp} & \textbf{Số electron tối đa trong mỗi lớp} \\
		\hline
		1 & K & s & 1 & 2 & 2 \\
		\hline
		2 & L & s, p & 1, 3 & 2, 6 & 8 \\
		\hline
		3 & M & s, p, d & 1, 3, 5 & 2, 6, 10 & 18 \\
		\hline
		4 & N & s, p, d, f & 1, 3, 5, 7 & 2, 4, 10, 14 & 32 \\
		\hline
	\end{tabular}
	\caption{Số AO \& số electron tối đa của các lớp $n = 1$ đến $n = 4$, \cite[Bảng 4.1, p. 31]{SGK_Hoa_Hoc_10_Chan_Troi_Sang_Tao}.}
	\label{tab:so AO & so electron toi da cua cac lop n = 1 den n = 4}
\end{table}
``Số electron tối đa trong lớp $n$ là $2n^2$ ($n\le 4$).'' -- \cite[p. 31]{SGK_Hoa_Hoc_10_Chan_Troi_Sang_Tao}

\begin{proof}[Chứng minh]
	Với $n\in\{1,2,3,4\}$, số phân lớp trong mỗi lớp bằng số thứ tự của lớp đó, i.e., lớp $n$ có $n$ phân lớp. Các phân lớp s, p, d, \& f lần lượt có các số AO tương ứng 1, 3, 5, \& 7, tương ứng với số electron tối đa trong mỗi phân lớp là 2, 6, 10, 14. Kết hợp 2 điều này, số electron tối đa trong lớp $n$ sẽ là $2 + 4 + \cdots + 2(2n - 1) = \sum_{i=1}^n 2(2i - 1) = 2\sum_{i=1}^n (2i - 1) = 2n^2$, trong đó đẳng thức cuối cùng sử dụng công thức tính tổng của $n$ số lẻ đầu tiên\footnote{Công thức này có thể dễ dàng chứng minh bằng quy nạp hoặc gom các số hạng có tổng bằng $2n$ lại với nhau, xem tài liệu của tác giả cho chương trình Toán lớp 11 \href{https://github.com/NQBH/hobby/blob/master/elementary_mathematics/grade_11/NQBH_elementary_mathematics_grade_11.pdf}{GitHub\texttt{/}NQBH\texttt{/}hobby\texttt{/}elementary mathematics\texttt{/}grade 11\texttt{/}lecture} để tham khảo cách chứng minh quy nạp \& chương trình Toán lớp 6 \href{https://github.com/NQBH/hobby/blob/master/elementary_mathematics/grade_6/NQBH_elementary_mathematics_grade_6.pdf}{GitHub\texttt{/}NQBH\texttt{/}hobby\texttt{/}elementary mathematics\texttt{/}grade 6\texttt{/}lecture} cho cách gom các cặp hạng cách đều đầu--cuối 1 cách thích hợp.}: $\sum_{i=1}^n (2i - 1) = 1 + 3 + \cdots + (2n - 1) = n^2$, $\forall n\in\mathbb{N}^\star$. Vậy số electron tối đa trong lớp $n\in\{1,2,3,4\}$ là $2n^2$.
\end{proof}
``Các phân lớp: $\rm s^2,p^6,d^{10},f^{14}$ chứa đủ số electron tối đa gọi là \textit{phân lớp bão hòa}. Các phân lớp $\rm s^1,p^3,d^5,f^7$ chứa 1 nửa số electron tối đa gọi là \textit{phân lớp nửa bão hòa}. Các phân lớp chưa đủ số electron tối đa gọi là \textit{phân lớp chưa bão hòa}.'' -- \cite[p. 31]{SGK_Hoa_Hoc_10_Chan_Troi_Sang_Tao}

\begin{figure}[H]
	\centering
	\includegraphics[scale=0.15]{su_phan_bo_electron_vao_cac_AO_trong_phan_lop_p}
	\caption{Sự phân bố electron vào các AO trong phân lớp p, \cite[Hình 4.10, p. 31]{SGK_Hoa_Hoc_10_Chan_Troi_Sang_Tao}.}
\end{figure}

\begin{quytac}[Quy tắc Hund]
	``Trong cùng 1 phân lớp chưa bão hòa, các electron sẽ phân bố vào các orbital sao cho số electron độc thân là tối đa.'' -- \cite[p. 31]{SGK_Hoa_Hoc_10_Chan_Troi_Sang_Tao}
\end{quytac}
``\textit{Cấu hình electron nguyên tử} biểu diễn sự phân bố electron trong vỏ nguyên tử trên các phân lớp thuộc các lớp khác nhau. Quy ước cách biểu diễn sự phân bố electron trên các phân lớp  thuộc các lớp như sau:

\begin{figure}[H]
	\centering
	\includegraphics[scale=0.15]{quy_uoc_cach_bieu_dien_su_phan_bo_electron_tren_cac_phan_lop_thuoc_cac_lop}
	\caption{Quy ước cách biểu diễn sự phân bố electron trên các phân lớp thuộc các lớp, \cite[p. 32]{SGK_Hoa_Hoc_10_Chan_Troi_Sang_Tao}.}
\end{figure}
Cách viết cấu hình electron:
\begin{enumerate}
	\item Xác định số electron của nguyên tử.
	\item Các electron được phân bố theo thứ tự các AO có mức năng lượng tăng dần, theo các nguyên tắc \& quy tắc phân bố electron trong nguyên tử.
	\item Viết cấu hình electron theo thứ tự các phân lớp trong 1 lớp \& theo thứ tự của các lớp electron.''
\end{enumerate}

\begin{vidu}
	Ca ($\rm Z = 20$): Thứ tự mức năng lượng orbital: $\rm 1s^22s^22p^63s^23p^64s^2$. cấu hình electron: $\rm 1s^22s^22p^63s^23p^64s^2$ hoặc viết gọn là: $\rm[Ar]4s^2$, $\rm[Ar]$ là ký hiệu cấu hình electron nguyên tử của nguyên tố argon, là khí hiếm gần nhất đứng trước Ca. Cấu hình electron theo ô orbital: $\boxed{\uparrow\downarrow}\ \boxed{\uparrow\downarrow}\ \boxed{\uparrow\downarrow}\boxed{\uparrow\downarrow}\boxed{\uparrow\downarrow}\ \boxed{\uparrow\downarrow}\ \boxed{\uparrow\downarrow}\boxed{\uparrow\downarrow}\boxed{\uparrow\downarrow}\boxed{\uparrow\downarrow}$.'' -- \cite[p. 32]{SGK_Hoa_Hoc_10_Chan_Troi_Sang_Tao}
\end{vidu}
``Cấu hình electron nguyên tử phải được viết theo thứ tự các lớp electrron \& phân lớp trong mỗi lớp. Trong đó:
\begin{enumerate*}
	\item[$\bullet$] Số thứ tự lớp electron được viết bằng các số tự nhiên ($n = 1,2,3,\ldots$).
	\item[$\bullet$] Phân lớp được ký hiệu bằng các chữ cái thường s, p, d, f.
	\item[$\bullet$] Số electron của từng phân lớp được ghi bằng chỉ số ở phía trên, bên phải ký hiệu của phân lớp.'' -- \cite[p. 32]{SGK_Hoa_Hoc_10_Chan_Troi_Sang_Tao}
\end{enumerate*}

\begin{vidu}
	``Đối với Fe ($\rm Z = 26$): Thứ tự mức năng lượng orbital: $\rm 1s^22s^22p^63s^23p^64s^23d^6$. Sắp xếp lại vị trí các phân lớp theo thứ tự lớp (Bước 3), thu được cấu hình electron nguyên tử. Cấu hình electron: $\rm 1s^22s^22p^63s^23p^63d^64s^2$ hoặc viết gọn là: $\rm[Ar]3d^64s^2$, $\rm[Ar]$ là ký hiệu cấu hình electron nguyên tử của nguyên tố argon, là khí hiềm gần nhất đứng trước $\rm Fe$. Cấu hình electron theo ô orbital: $\boxed{\uparrow\downarrow}\ \boxed{\uparrow\downarrow}\ \boxed{\uparrow\downarrow}\boxed{\uparrow\downarrow}\boxed{\uparrow\downarrow}\ \boxed{\uparrow\downarrow}\ \boxed{\uparrow\downarrow}\boxed{\uparrow\downarrow}\boxed{\uparrow\downarrow}\ \boxed{\uparrow\downarrow}\boxed{\uparrow}\boxed{\uparrow}\boxed{\uparrow}\boxed{\uparrow}\ \boxed{\uparrow\downarrow}$.'' -- \cite[p. 32]{SGK_Hoa_Hoc_10_Chan_Troi_Sang_Tao}
\end{vidu}
``Dựa vào các nguyên lý \& quy tắc nêu ở trên, ta có thể viết cấu hình electron nguyên tử của các nguyên tố. Khi tham gia các phản ứng hóa học, thông thương electron lớp ngoài cùng của các nguyên tử sẽ thay đổi, chúng có vai trò quyết định đến tính chất hóa học đặc trưng của nguyên tố (tính kim loại, tính phi kim, $\ldots$). Các nguyên tử có $1,2,3$ electron ở lớp ngoài cùng thường là các nguyên tử của nguyên tố kim loại; các nguyên tử có $5,6,7$ electron ở lớp ngoài cùng thường là nguyên tử của các nguyên tố phi kim; các nguyên tử có 4 electron ở lớp ngoài cùng có thể là nguyên tử của nguyên tố kim loại hoặc phi kim; các nguyên tử có 8 electron ở lớp ngoài cùng là nguyên tử của nguyên tố khí hiếm (trừ He có 2 electron ở lớp ngoài cùng).'' -- \cite[p. 33]{SGK_Hoa_Hoc_10_Chan_Troi_Sang_Tao}

Xem \cite[Bảng 4.2: \textsf{Cấu hình electron nguyên tử của 1 số nguyên tố}, p. 33]{SGK_Hoa_Hoc_10_Chan_Troi_Sang_Tao}

\begin{vidu}
	``Lithium là 1 nguyên tố có nhiều công dụng, được sử dụng trong chế tạo máy bay \& trong 1 số loại pin nhất định. Pin Lithium--Ion (pin Li-Ion) đang ngày càng phổ biến, nó cung cấp năng lượng cho cuộc sống của hàng triệu người mỗi ngày thông qua các thiết  bị như máy tính xách tay, điện thoại di động, xe Hybrid, xe điện, $\ldots$ nhờ trọng lượng nhẹ, cung cấp năng lượng cao \& khả năng sạc lại.'' -- \cite[p. 33]{SGK_Hoa_Hoc_10_Chan_Troi_Sang_Tao}
\end{vidu}
``Dựa vào số lượng electron lớp ngoài cùng của nguyên tử nguyên tố, có thể dự đoán 1 nguyên tố là kim loại, phi kim hay khí hiếm.'' -- \cite[p. 33]{SGK_Hoa_Hoc_10_Chan_Troi_Sang_Tao}

%------------------------------------------------------------------------------%

\chapter{Bảng Tuần Hoàn Các Nguyên Tố Hóa Học}

\section{Cấu Tạo Bảng Tuần Hoàn Các Nguyên Tố Hóa Học}
\textbf{Nội dung.} \textit{Lịch sử phát minh định luật tuần hoàn \& bảng tuần hoàn các nguyên tố hóa học, cấu tạo của bảng tuần hoàn các nguyên tố hóa học \& nêu được các khái niệm liên quan (ô, chu kỳ, nhóm), nguyên tắc sắp xếp của bảng tuần hoàn các nguyên tố hóa học (dựa theo cấu hình electron), phân loại nguyên tố (dựa theo cấu hình electron: nguyên tố s, p, d, f; dựa theo tính chất hóa học: kim loại, phi kim, khí hiếm).}

``Cách đây hàng nghìn năm, người ta chỉ biết đến 1 số nguyên tố như đồng (copper), bạc (silver), \& vàng (gold). Mãi đến năm 1700, cũng chỉ mới có 13 nguyên tố được xác định. Khi đó, các nhà hóa học nghi ngờ rằng vẫn còn nhiều nguyên tố bí ẩn khác chưa được khám phá. Bằng việc sử dụng các phương pháp khoa học hiện đại, chỉ trong 1 thập kỷ (1765--1775) đã có thêm 5 nguyên tố hóa học được xác định. Trong đó, có 3 khí không màu là hydrogen, nitrogen, \& oxygen. Tính đến năm 2016, tổng cộng đã có 118 nguyên tố hóa học được xác định trên bảng tuần hoàn các nguyên tố hóa học.'' -- \cite[p. 33]{SGK_Hoa_Hoc_10_Chan_Troi_Sang_Tao}

\subsection{Lịch sử phát minh định luật tuần hoàn \& bảng tuần hoàn các nguyên tố hóa học}
``Năm 1869, nhà hóa học \& giáo viên người Nga, Dmitri Ivanovich Mendeleev đã công bố 1 \textit{Bảng tuần hoàn các nguyên tố hóa học}. Trong năm này, 1 nhà hóa học người Đức, L. Meyer cũng đã công bố 1 Bảng tuần hoàn tương tự. Tuy nhiên, Mendeleev là người công bố trước \& giải thích tốt hơn về sự hữu dụng của bảng tuần hoàn do ông đề nghị. 2 nhà khoa học đều sắp xếp các nguyên tố vào các hàng \& các cột theo chiều tăng dần khối lượng nguyên tử, bắt đầu ở hàng mới (Bảng của Mendeleev) hoặc cột mới (Bảng của Mayer) khi tính chất của nguyên tố lặp lại.'' -- \cite[pp. 33--34]{SGK_Hoa_Hoc_10_Chan_Troi_Sang_Tao}. Xem \cite[Hình 5.1: \textsf{Bảng tuần hoàn các nguyên tố hóa học của Mendeleev (1869)}, p. 34]{SGK_Hoa_Hoc_10_Chan_Troi_Sang_Tao}.

``Năm 1871, Mendeleev đã phát biểu định luật tuần hoàn như sau: \textit{Tính chất của các nguyên tố, cũng như tính chất của các đơn chất \& hợp chất tạo nên từ các nguyên tố đó biến đổi tuần hoàn theo trọng lượng nguyên tử của chúng (trọng lượng nguyên tử được hiểu là khối lượng nguyên tử)}. Mendeleev đã dự đoán về các nguyên tố mới, gồm 10 nguyên tố, trong đó có 3 nguyên tố (sau này chính là các nguyên tố Sc, Ga, \& Ge) được ông miêu tả khá tỉ mỉ về tính chất vật lý của đơn chất \& 1 số hợp chất của chúng, 7 nguyên tố còn lại do vị trí của chúng trong bảng tuần hoàn không thuận lợi cho việc tiên đoán, nên ông chỉ mới ước lượng được khối lượng nguyên tử của chúng.

Bảng tuần hoàn các nguyên tố hóa học hiện nay được xây dựng trên cơ sở tính chất của các nguyên tố \& đơn chất, cũng như thành phần \& tính chất của các hợp chất tạo nên từ các nguyên tố đó biến đổi tuần hoàn khi chúng được sắp xếp theo chiều tăng của số hiệu nguyên tử của nguyên tố. Cách xây dựng này không những giúp so sánh, dự đoán tính chất của đơn chất \& hợp chất, mà còn cung cấp nhiều thông tin về mỗi nguyên tố hóa học, cũng như định hướng cho việc tiếp tục nghiên cứu các nguyên tố mới.'' -- \cite[p. 36]{SGK_Hoa_Hoc_10_Chan_Troi_Sang_Tao}

``Năm 1869, nhà hóa học Mendeleev đã công bố bảng tuần hoàn các nguyên tố hóa học, trong đó, các nguyên tố đã được sắp xếp theo thứ tự tăng dần khối lượng nguyên tử. Bảng tuần hoàn hiện đại ngày nay được xây dựng trên cơ sở mối liên hệ giữa số hiệu nguyên tử \& tính chất của nguyên tố, các nguyên tố được sắp xếp theo thứ tự tăng dần số hiệu nguyên tử.'' -- \cite[p. 38]{SGK_Hoa_Hoc_10_Chan_Troi_Sang_Tao}

\subsection{Bảng tuần hoàn các nguyên tố hóa học}

\begin{figure}[h]
	\centering
	\includegraphics[scale=0.15]{o_nguyen_to_aluminium}
	\caption{Ô nguyên tố aluminium, \cite[p. 38]{SGK_Hoa_Hoc_10_Chan_Troi_Sang_Tao}.}
\end{figure}
``Mỗi nguyên tố hóa học được xếp vào 1 ô trong bảng tuần hoàn các nguyên tố hóa học, gọi là \textit{ô nguyên tố}. Số thự tự của 1 ô nguyên tố bằng số hiệu nguyên tử của nguyên tố hóa học trong ô đó.'' -- \cite[p. 38]{SGK_Hoa_Hoc_10_Chan_Troi_Sang_Tao}

``Các nguyên tố có cùng số lớp electron trong nguyên tử được xếp thành 1 hàng, gọi là \textit{chu kỳ}. Số thứ tự của chu kỳ bằng số lớp electron của nguyên tử các nguyên tố trong chu kỳ. Bảng tuần hoàn gồm 7 chu kỳ: 
\begin{enumerate*}
	\item[$\bullet$] Các chu kỳ 1, 2, \& 3 là \textit{các chu kỳ nhỏ}.
	\item[$\bullet$] Các chu kỳ 4, 5, 6, \& 7 là \textit{các chu kỳ lớn}.'' -- \cite[p. 38]{SGK_Hoa_Hoc_10_Chan_Troi_Sang_Tao}
\end{enumerate*}

``Bảng tuần hoàn hiện nay có 18 cột, chia thành 8 nhóm A (IA--VIIIA) \& 8 nhóm B (IB--VIIIB). Mỗi cột tương ứng với 1 nhóm, riêng nhóm VIIIB có 3 cột (\cite[Hình 5.2: \textsf{Bảng tuần hoàn các nguyên tố hóa học}, p. 38]{SGK_Hoa_Hoc_10_Chan_Troi_Sang_Tao}).'' -- \cite[p. 38]{SGK_Hoa_Hoc_10_Chan_Troi_Sang_Tao}

``\textit{Electron hóa trị} là những electron có khả năng tham gia hình thành liên kết hóa học. Chúng thường nằm ở \textit{lớp ngoài cùng} hoặc ở cả \textit{phân lớp sát lớp ngoài cùng} nếu phân lớp đó chưa bão hòa. Những nguyên tố có cùng số electron hóa trị thường có tính chất hóa học tương tự nhau.'' -- \cite[p. 39]{SGK_Hoa_Hoc_10_Chan_Troi_Sang_Tao}

``\textit{Nhóm} là tập hợp các nguyên tố mà nguyên tử có cấu hình electron tương tự nhau (trừ nhóm VIIIB), do đó có tính chất hóa học gần giống nhau \& được xếp theo cột. Số thứ tự của nhóm A bằng số electron ở lớp ngoài cùng của nguyên tử các nguyên tố trong nhóm.'' -- \cite[p. 39]{SGK_Hoa_Hoc_10_Chan_Troi_Sang_Tao}

``Các nguyên tố hóa học cũng có thể được chia thành các khối như sau:
\begin{enumerate*}
	\item[$\bullet$] \textit{Khối các nguyên tố $\rm s$} gồm các nguyên tố thuộc nhóm IA \& nhóm IIA, có cấu hình electron: [Khí hiếm] $n{\rm s}^{1\div 2}$.
	\item[$\bullet$] \textit{Khối các nguyên tố $\rm p$} gồm các nguyên tố thuộc nhóm IIIA--nhóm VIIIA (trừ nguyên tố He), có cấu hình electron: [Khí hiếm] $n{\rm s}^2n{\rm p}^{1\div 6}$.
	\item[$\bullet$] \textit{Khối các nguyên tố $\rm d$} gồm các nguyên tố thuộc nhóm B, có cấu hình electron: [Khí hiếm] $(n - 1){\rm d}^{1\div 10}n{\rm s}^{1\div 2}$.
	\item[$\bullet$] \textit{Khối các nguyên tố $\rm f$} gồm các nguyên tố xếp thành 2 hàng ở cuối bảng tuần hoàn, có cấu hình electron: [Khí hiếm] $(n - 2){\rm f}^{0\div 14}(n - 1){\rm d}^{0\div 2}n{\rm s}^2$ (trong đó $n = 6$  \& $n = 7$). Chúng gồm 14 nguyên tố họ Lanthanide (từ Ce đến Lu) \& 14 nguyên tố họ Actinide (từ Th đến Lr).'' -- \cite[pp. 39--40]{SGK_Hoa_Hoc_10_Chan_Troi_Sang_Tao}
\end{enumerate*}

\begin{vidu}[Nitrogen]
	``Nitrogen là thành phần dinh dưỡng cần thiết cho sự sinh trưởng, phát triển, \& sinh sản của thực vật. Nitrogen có số hiệu nguyên tử là 7.'' -- \cite[p. 40]{SGK_Hoa_Hoc_10_Chan_Troi_Sang_Tao}
\end{vidu}
``Dựa vào cấu hình electron, người ta phân loại các nguyên tố thành nguyên tố s, nguyên tố p, nguyên tố d, \& nguyên tố f. Dựa vào tính chất hóa học, người ta phân loại các nguyên tố thành nguyên tố kim loại, nguyên tố phi kim \& nguyên tố khí hiếm.'' -- \cite[p. 40]{SGK_Hoa_Hoc_10_Chan_Troi_Sang_Tao}

``Các nguyên tố hóa học được xếp vào 1 bảng theo những nguyên tắc nhất định, gọi là \textit{bảng tuần hoàn}. Bảng tuần hoàn hiện nay gồm 118 nguyên tố hóa học. Vị trí của mỗi nguyên tố hóa học trong bảng tuần hoàn được xác định qua số thứ tự ô nguyên tố, chu kỳ, \& nhóm. Khi sắp xếp như vậy, sự tuần hoàn tính chất của các đơn chất \& hợp chất được thể hiện qua chu kỳ \& nhóm.'' -- \cite[p. 40]{SGK_Hoa_Hoc_10_Chan_Troi_Sang_Tao}

``\textbf{Nguyên tắc sắp xếp các nguyên tố trong bảng tuần hoàn.}
\begin{enumerate*}
	\item[$\bullet$] Các nguyên tố được xếp theo chiều tăng dần của điện tích hạt nhân nguyên tử.
	\item[$\bullet$] Các nguyên tố có cùng số lớp electron trong nguyên tử được xếp cùn 1 chu kỳ.
	\item[$\bullet$] Các nguyên tố có cùng số electron hóa trị trong nguyên tử được xếp cùng 1 nhóm, trừ nhóm VIIIB.'' -- \cite[p. 40]{SGK_Hoa_Hoc_10_Chan_Troi_Sang_Tao}
\end{enumerate*}

\begin{vidu}[Silicon]
	``Silicon là 1 nguyên tố phổ biến \& có nhiều ứng dụng trong cuộc sống. Silicon siêu tinh khiết là chất bán dẫn, được dùng trong kỹ thuật vô tuyến \& điện tử. Ngoài ra, nguyên tố này còn được sử dụng để chế tạo pin mặt trời nhằm mục đích chuyển đổi năng lượng ánh sáng thành năng lượng điện để cung cấp cho các thiết bị trên tàu vũ trụ.'' -- \cite[p. 40]{SGK_Hoa_Hoc_10_Chan_Troi_Sang_Tao}
\end{vidu}
``Năm 1789, nhà khoa học A. Lavoisier người Pháp đã xếp 33 nguyên tố hóa học thành nhóm các chất khí, kim loại, phi kim, \& ``đất''. Năm 1829, nhà hóa học người Đức J. W. D\"obereiner đã nghiên cứu 1 hệ thống phân loại các nguyên tố hóa học.

\begin{figure}[H]
	\centering
	\includegraphics[scale=0.15]{chlorine_bromine_iodine_lithium_sodium_potassium}
	\caption{Chlorine, Bromine, Iodine, Lithium, Sodium, Potassium, \cite[p. 41]{SGK_Hoa_Hoc_10_Chan_Troi_Sang_Tao}.}
\end{figure}
\textbf{Thông tin về 2 bộ 3 nguyên tố theo nguyên tắc phân loại các nguyên tố hóa học theo D\"obereiner.} Năm 1864, nhà hóa học người Anh, J. Newlands đề xuất 1 sơ đồ sắp xếp các nguyên tố. Ông nhận thấy rằng khi các nguyên tố được sắp xếp theo khối lượng nguyên tử tăng dần, tính chất của chúng lặp lại có quy luật tương tự như 1 quãng 8 trong âm nhạc, trong đó nguyên tố thứ 8 lặp lại tính chất của nguyên tố đầu tiên. Sau đó, ông đặt tên cho quy luật này là ``Quy luật quãng 8''.'' -- \cite[p. 41]{SGK_Hoa_Hoc_10_Chan_Troi_Sang_Tao}. Xem \textsf{Cách sắp xếp các nguyên tố hóa học của John Newlands} (\cite{Buthelezi_Dingrando_Hainen_Wistrom_Zike2013}), \cite[p. 41]{SGK_Hoa_Hoc_10_Chan_Troi_Sang_Tao}.

\begin{vidu}[Neon]
	``Neon tạo ra ánh sáng màu đỏ khi sử dụng trong các ống phóng điện chân không, được sử dụng rộng rãi trong các biển quảng cáo.'' ``$\rm Ne$ có số hiệu nguyên tử là $10$.'' -- \cite[p. 42]{SGK_Hoa_Hoc_10_Chan_Troi_Sang_Tao}
\end{vidu}

\begin{vidu}[Magnesium]
	``\emph{Magnesium} được sử dụng để làm cho hợp kim bền nhẹ, đặc biệt được ứng dụng cho ngành công nghiệp hàng không.'' ``$\rm Mg$ có số hiệu nguyên tử là $12$.'' -- \cite[p. 42]{SGK_Hoa_Hoc_10_Chan_Troi_Sang_Tao}
\end{vidu}

%------------------------------------------------------------------------------%

\section{Xu Hướng Biến Đổi 1 Số Tính Chất của Nguyên Tử Các Nguyên Tố, Thành Phần \& 1 Số Tính Chất của Hợp Chất Trong 1 Chu Kỳ \& Nhóm}
\textbf{Nội dung.} \textit{Xu hướng biến đổi bán kính nguyên tử trong 1 chu kỳ, trong 1 nhóm (nhóm A); xu hướng biến đổi độ âm điện \& tính kim loại, phi kim của nguyên tử các nguyên tố trong 1 chu kỳ, trong 1 nhóm (nhóm A); xu hướng biến đổi thành phần \& tính chất acid}\texttt{/}\textit{base của các oxide \& các hydroxide theo chu kỳ; phương trình hóa học minh họa}.

``\textit{Kim loại kiềm} là các kim loại thuộc nhóm IA, bao gồm: lithium (Li), sodium (Na), potassium (K), rubidium (Rb), caesium (Cs), francium (Fr). Chúng phản ứng được với nước \& giải phóng khí hydrogen.'' -- \cite[p. 43]{SGK_Hoa_Hoc_10_Chan_Troi_Sang_Tao}

\subsection{Bán kính nguyên tử}
Xem \cite[Hình 6.1: \textsf{Bán kính nguyên tử của 1 số nguyên tố được biểu diễn bằng pm ($\rm 1\ pm = 10^{-12}\ m$)}]{SGK_Hoa_Hoc_10_Chan_Troi_Sang_Tao} (\cite{Buthelezi_Dingrando_Hainen_Wistrom_Zike2013}).

``\textbf{Xu hướng biến đổi bán kính nguyên tử.} Bán kính nguyên tử của các nguyên tố nhóm A có xu hướng biến đổi tuần hoàn theo chiều tăng của điện tích hạt nhân:
\begin{enumerate*}
	\item[$\bullet$] \textit{Trong 1 chu kỳ}, nguyên tử của các nguyên tố có cùng số lớp electron. Từ trái sang phải, điện tích hạt nhân nguyên tử tăng dần nên electron lớp ngoài cùng sẽ bị hạt nhân hút mạnh hơn, vì vậy bán kính nguyên tử của các nguyên tố có xu hướng \textit{giảm dần}.
	\item[$\bullet$] \textit{Trong 1 nhóm}, theo chiều từ trên xuống dưới, số lớp electron tăng dần nên bán kính nguyên tử có xu hướng \textit{tăng}.'' -- \cite[p. 44]{SGK_Hoa_Hoc_10_Chan_Troi_Sang_Tao}
\end{enumerate*}

\subsection{Độ âm điện}
``\textit{Độ âm điện} của 1 nguyên tử đặc trưng cho khả năng hút electron của nguyên tử đó khi tạo thành liên kết hóa học. Trong hóa học, có nhiều thang đo độ âm điện khác nhau do các nhà khoa học tính toán dựa trên những cơ sở khác nhau. Dưới đây giới thiệu bảng giá trị (Bảng \ref{tab:gia tri do am dien cua nguyen tu 1 so nguyen to nhom A theo Pauling}) độ âm điện của nhà hóa học L. C. Pauling (\cite{Buthelezi_Dingrando_Hainen_Wistrom_Zike2013}) đề xuất năm 1932.'' -- \cite[p. 44]{SGK_Hoa_Hoc_10_Chan_Troi_Sang_Tao}

\begin{table}[H]
	\centering
	\begin{tabular}{|c|c|c|c|c|c|c|c|c|}
		\hline
		\diagbox{\textbf{Chu kỳ}}{\textbf{Nhóm}}& IA & IIA & IIIA & IVA & VA & VIA & VIIA & VIIIA \\
		\hline
		1 & H 2.20 &  &  &  &  &  &  & He \\
		\hline
		2 & Li 0.98 & Be 1.57 & B 2.04 & C 2.55 & N 3.04 & O 3.44 & F 3.98 & Ne \\
		\hline
		3 & Na 0.93 & Mg 1.31 & Al 1.61 & Si 1.90 & P 2.19 & S 2.58 & Cl 3.16 & Ar \\
		\hline
		4 & K 0.82 & Ca 1.00 & Ga 1.81 & Ge 2.01 & As 2.18 & Se 2.55 & Br 2.96 & Kr \\
		\hline
		5 & Rb 0.82 & Sr 0.95 & In 1.78 & Sn 1.96 & Sb 2.05 & Te 2.10 & I 2.66 & Xe \\
		\hline
		6 & Cs 0.79 & Ba 0.89 & Tl 1.80 & Pb 1.80 & Bi 1.90 & Po 2.00 & At 2.20 & Rn \\
		\hline
	\end{tabular}
	\caption{Giá trị độ âm điện của nguyên tử 1 số nguyên tố nhóm A theo Pauling, \cite[Bảng 6.1, p. 44]{SGK_Hoa_Hoc_10_Chan_Troi_Sang_Tao}.}
	\label{tab:gia tri do am dien cua nguyen tu 1 so nguyen to nhom A theo Pauling}
\end{table}
``\textbf{Xu hướng biến đổi độ âm điện.} Độ âm điện của nguyên tử các nguyên tố nhóm A có xu hướng biến đổi tuần hoàn theo chiều tăng của điện tích hạt nhân:
\begin{enumerate*}
	\item[$\bullet$] \textit{Trong 1 chu kỳ}, theo chiều tăng dần của điện tích hạt nhân, lực hút giữa hạt nhân với các electron lớp ngoài cùng cũng tăng. Do đó, độ âm điện của nguyên tử các nguyên tố có xu hướng \textit{tăng dần}.
	\item[$\bullet$] \textit{Trong 1 nhóm}, theo chiều tăng dần của điện tích hạt nhân, bán kính nguyên tử tăng nhanh, lực hút giữa hạt nhân với các electron lớp ngoài cùng giảm. Do đó, độ âm điện của nguyên tử các nguyên tố có xu hướng \textit{giảm dần}.'' -- \cite[p. 45]{SGK_Hoa_Hoc_10_Chan_Troi_Sang_Tao}
\end{enumerate*}

\subsection{Tính kim loại, tính phi kim}

\begin{dinhnghia}[Tính kim loại, tính phi kim]
	``\emph{Tính kim loại} là tính chất của 1 nguyên tố mà nguyên tử dễ nhường electron. \emph{Tính phi kim} là tính chất của 1 nguyên tố mà nguyên tử dễ nhận electron.'' -- \cite[p. 45]{SGK_Hoa_Hoc_10_Chan_Troi_Sang_Tao}
\end{dinhnghia}
``\textbf{Xu hướng biến đổi tính kim loại, tính phi kim.} Tính kim loại, tính phi kim của các nguyên tố nhóm A có xu hướng biến đổi tuần hoàn theo chiều tăng của điện tích hạt nhân:
\begin{enumerate*}
	\item[$\bullet$] \textit{Trong 1 chu kỳ}, theo chiều tăng dần của điện tích hạt nhân, lực hút giữa hạt nhân với các electron lớp ngoài cùng tăng. Do đó, \textit{tính kim loại của các nguyên tố giảm dần, tính phi kim tăng dần}.
	\item[$\bullet$] \textit{Trong 1 nhóm}, theo chiều tăng dần của điện tích hạt nhân, lực hút giữa hạt nhân với các electron lớp ngoài cùng giảm. Do đó, \textit{tính kim loại của các nguyên tố tăng dần, tính phi kim giảm dần}.'' -- \cite[p. 46]{SGK_Hoa_Hoc_10_Chan_Troi_Sang_Tao}
\end{enumerate*}

\subsection{Tính acid--base của oxide \& hydroxide}

\begin{table}[H]
	\centering
	\resizebox{\columnwidth}{!}{%
	\begin{tabular}{|p{0.105\textwidth}|p{0.17\textwidth}|p{0.17\textwidth}|p{0.12\textwidth}|p{0.115\textwidth}|p{0.115\textwidth}|p{0.115\textwidth}|}
		\hline
		\textbf{IA} & \textbf{IIA} & \textbf{IIIA} & \textbf{IVA} & \textbf{VA} & \textbf{VIA} & \textbf{VIIA} \\
		\hline
		$\rm Li_2O$\newline(basic oxide) & $\rm BeO$\newline(Oxide lưỡng tính) & $\rm B_2O_3$\newline(Acidic oxide) & $\rm CO_2$\newline(Acidic oxide) & $\rm N_2O_5$\newline(Acidic oxide) &  &  \\
		\hline
		$\rm LiOH$\newline(Base mạnh) & $\rm Be(OH)_2$ (Hydroxide lưỡng tính) & $\rm H_3BO_3$\newline(Acid yếu) & $\rm H_2CO_3$\newline(Acid yếu) & $\rm HNO_3$\newline(Acid mạnh) &  &  \\
		\hline
		$\rm Na_2O$\newline(Basic oxide) & $\rm MgO$\newline(Basic oxide) & $\rm Al_2O_3$\newline(Oxide lưỡng tính) & $\rm SiO_2$\newline(Acidic oxide) & $\rm P_2O_5$\newline(Acidic oxide) & $\rm SO_3$\newline(Acidic oxide) & $\rm Cl_2O_7$\newline(Acidic oxide) \\
		\hline
		$\rm NaOH$\newline(Base mạnh) & $\rm Mg(OH)_2$\newline(Base yếu) & $\rm Al(OH)_3$ (Hydroxide lưỡng tính) & $\rm H_2SiO_3$\newline(Acid yếu) & $\rm H_3PO_4$ (Acid trung bình) & $\rm H_2SO_4$\newline(Acide mạnh) & $\rm HClO_4$ (Acid rất mạnh) \\
		\hline
	\end{tabular}%
	}
	\caption{Tính acid -- base của oxide \& hydroxide tương ứng của các nguyên tố thuộc chu kỳ 2 \& 3 (ứng với hóa trị cao nhất của các nguyên tố), \cite[Bảng 6.2, p. 47]{SGK_Hoa_Hoc_10_Chan_Troi_Sang_Tao}.}
\end{table}
``Trong 1 chu kỳ, theo chiều tăng dần của điện tích hạt nhân, tính base của oxide \& hydroxide tương ứng giảm dần, tính acid của chúng tăng dần.'' -- \cite[p. 47]{SGK_Hoa_Hoc_10_Chan_Troi_Sang_Tao}

\begin{vidu}[Aspartame]
	``Aspartame là 1 chất làm ngọt nhân tạo, được sử dụng trong 1 số loại soda dành cho người ăn kiêng.'' ``Công thức cấu tạo của Aspartame: $\rm C_{14}H_{18}N_2O_5$.'' -- \cite[p. 47]{SGK_Hoa_Hoc_10_Chan_Troi_Sang_Tao}
\end{vidu}

%------------------------------------------------------------------------------%

\section{Định Luật Tuần Hoàn -- Ý Nghĩa của Bảng Tuần Hoàn Các Nguyên Tố Hóa Học}
\textbf{Nội dung.} \textit{Định luật tuần hoàn; ý nghĩa của bảng tuần hoàn các nguyên tố hóa học: mối liên hệ giữa vị trí (trong bảng tuần hoàn các nguyên tố hóa học) với tính chất \& ngược lại}.

\begin{vidu}
	``Fluorine được sử dụng làm chất oxi hóa cho nhiên liệu lỏng dùng trong tên lửa. Fluorine (F) là 1 nguyên tố hóa học có số hiệu nguyên tử bằng 9, thuộc chu kỳ 2, nhóm VIIA.'' -- \cite[p. 49]{SGK_Hoa_Hoc_10_Chan_Troi_Sang_Tao}
\end{vidu}

\subsection{Định luật tuần hoàn}
Xem \cite[Bảng 7.1: \textsf{Cấu hình electron lớp ngoài cùng của nguyên tử các nguyên tố nhóm A}, p. 49]{SGK_Hoa_Hoc_10_Chan_Troi_Sang_Tao}.

``Sự biến đổi tuần hoàn về cấu hình electron lớp ngoài cùng của nguyên tử các nguyên tố khi điện tích hạt nhân tăng dần chính là nguyên nhân của sự biến đổi tuần hoàn về tính chất của các nguyên tố, cũng như hợp chất của chúng.'' -- \cite[p. 50]{SGK_Hoa_Hoc_10_Chan_Troi_Sang_Tao}

\begin{dinhluat}[Định luật tuần hoàn]
	``Tính chất của các nguyên tố \& đơn chất, cũng như thành phần \& tính chất của các hợp chất tạo nên từ các nguyên tố đó biến đổi tuần hoàn theo chiều tăng của điện tích hạt nhân nguyên tử.'' -- \cite[p. 50]{SGK_Hoa_Hoc_10_Chan_Troi_Sang_Tao}
\end{dinhluat}

\subsection{Ý nghĩa của bảng tuần hoàn các nguyên tố hóa học}

\begin{vidu}[Natri]
	\begin{enumerate*}
		\item[(a)] \emph{Cấu hình electron Na:} $\rm 1s^22s^22p^63s^1$. Số lớp electron: $3$. Số electron lớp ngoài cùng: $1$.
		\item[(b)] \emph{Vị trí:} Chu kỳ: $3$. Nhóm: IA.
		\item[(c)] \emph{Tính chất:} Độ âm điện nhỏ, bán kính nguyên tử lớn, $\ldots$. $\rm Na$: kim loại mạnh. $\rm NaOH$: base mạnh. $\rm Na_2O$: basic oxide.
	\end{enumerate*}
	Xem \cite[Hình 7.1: \textsf{Mối quan hệ vị trí, cấu hình electron \& tính chất của sodium}, p. 50]{SGK_Hoa_Hoc_10_Chan_Troi_Sang_Tao}
\end{vidu}
``Khi biết vị trí của 1 nguyên tố trong bảng tuần hoàn, có thể suy ra cấu tạo nguyên tử của nguyên tố đó \& ngược lại. Từ đó, có thể suy ra những tính chất hóa học cơ bản của nó.'' -- \cite[p. 51]{SGK_Hoa_Hoc_10_Chan_Troi_Sang_Tao}

\begin{vidu}[Potassium hydroxide (KOH)]
	``\emph{Potassium hydroxide (KOH)} là 1 trong những hóa chất quan trọng của ngành công nghiệp. Chất này được sử dụng để sản xuất chất tẩy rửa gia dụng, thuốc nhuộm vải, phân bón, $\ldots$.'' -- \cite[p. 51]{SGK_Hoa_Hoc_10_Chan_Troi_Sang_Tao}
\end{vidu}

%------------------------------------------------------------------------------%

\chapter{Liên Kết Hóa Học}

\begin{quotation}
	\textbf{Nội dung.} \textit{Quy tắc octet trong quá trình hình thành liên kết hóa học co các nguyên tố nhóm A}.
\end{quotation}

\section{Quy Tắc Octet}
``Khi liên kết với nhau, nguyên tử của các nguyên tố dường như đã cố gắng ``bắt chước'' cấu hình electron nguyên tử của các nguyên tố khí hiếm để bền vững hơn. Điều này đã được nhà hóa học người Mỹ Lewis (1875--1946) đề nghị khi nghiên cứu về sự hình thành phân tử từ các nguyên tử.'' -- \cite[p. 52]{SGK_Hoa_Hoc_10_Chan_Troi_Sang_Tao}

\subsection{Liên kết hóa học}
``Phân tử được tạo nên từ các nguyên tử bằng các \textit{liên kết hóa học}.'' -- \cite[p. 53]{SGK_Hoa_Hoc_10_Chan_Troi_Sang_Tao}

\subsection{Quy tắc Octet}
``Để đạt cấu hình electron bền vững của các khí hiếm gần nhất, nguyên tử của các nguyên tố có xu hướng nhường, hoặc nhận thêm, hoặc góp chung các electron hóa trị với các nguyên tử khác khi tham gia liên kết hóa học. E.g., liên kết giữa 2 nguyên tử nitrogen (N) trong phân tử nitrogen ($\rm N_2$) được tạo thành do mỗi nguyên tử nitrogen đã góp chung 3 electron hóa trị, tạo nên 3 cặp electron chung như \cite[Hình 8.2: \textsf{Sự hình thành liên kết trong phân tử nitrogen}, p. 53]{SGK_Hoa_Hoc_10_Chan_Troi_Sang_Tao}

``Nguyên tử sodium có 1 electron ở lớp ngoài cùng. Nếu mất đi 1 electron này, nguyên tử sodium sẽ đạt được cấu hình electron bền vững sau:

\begin{figure}[H]
	\centering
	\includegraphics[scale=0.15]{su_hinh_thanh_ion_Na}
	\caption{Sự hình thành ion $\rm Na^+$, \cite[Hình 8.3, p. 53]{SGK_Hoa_Hoc_10_Chan_Troi_Sang_Tao}.}
\end{figure}
Phần tử thu được mang điện tích dương, gọi là \textit{ion sodium}, ký hiệu là $\rm Na^+$. Tương tự, nguyên tử fluorine có 7 electron ở lớp ngoài cùng. Khi nhận vào 1 electron, nguyên tử fluorine sẽ đạt được cấu hình electron bền vững sau:

\begin{figure}[H]
	\centering
	\includegraphics[scale=0.15]{su_hinh_thanh_ion_F}
	\caption{Sự hình thành ion $\rm F^-$, \cite[Hình 8.3, p. 54]{SGK_Hoa_Hoc_10_Chan_Troi_Sang_Tao}.}
\end{figure}
Phần tử thu được mang điện tích âm, gọi là \textit{ion fluoride}, ký hiệu $\rm F^-$.

\begin{quytac}[Quy tắc octet (bát tử)]
	Trong quá trình hình thành liên kết hóa học, nguyên tử của các nguyên tố nhóm A có xu hướng tạo thành lớp vỏ ngoài cùng có 8 electron tương ứng với khí hiếm gần nhất (hoặc 2 electron với khí hiếm helium).
\end{quytac}
Không phải trong mọi trường hợp, nguyên tử của các nguyên tố khi tham gia liên kết đều tuân theo quy tắc octet. Người ta nhận thấy 1 số phân tử có thể không tuân theo quy tắc octet. E.g., $\rm NO,BH_3,SF_6,\ldots$. Với nguyên tử của các nguyên tố nhóm B, người ta áp dụng 1 quy tắc khác, tương ứng với quy tắc octet, là quy tắc 18 electron để giải thích xu hướng khi tham gia liên kết hóa học của chúng.'' -- \cite[pp. 53--54]{SGK_Hoa_Hoc_10_Chan_Troi_Sang_Tao}

%------------------------------------------------------------------------------%

\section{Liên Kết Ion}
\textbf{Nội dung.} \textit{Sự hình thành liên kết ion (1 số ví dụ điển hình tuân theo quy tắc octet), cấu tạo tinh thể $\rm NaCl$, nguyên nhân các hợp chất ion thường ở trạng thái rắn trong điều kiện thường (dạng tinh thể ion), mô hình tinh thể $\rm NaCl$}.

``Hơn 50\% dược phẩm sử dụng trong y tế được sản xuất dưới dạng muối với mục đích thúc đẩy sự hấp thu các dược chất vào máu, tăng cường hiệu quả điều trị. Trong đó, thường gặp nhất là các muối hydrochloride, sodium hoặc sulfate. Muối thường là các hợp chất chứa liên kết ion.'' -- \cite[p. 55]{SGK_Hoa_Hoc_10_Chan_Troi_Sang_Tao}

\subsection{Ion \& sự hình thành liên kết ion}
``Khi cho electron, nguyên tử trở thành \textit{ion dương} (\textit{cation}). Khi nhận electron, nguyên tử trở thành \textit{ion âm} (\textit{anion}). Giá trị điện tích trên cation hoặc anion bằng số electron mà nguyên tử đã nhường hoặc nhận.'' -- \cite[p. 55]{SGK_Hoa_Hoc_10_Chan_Troi_Sang_Tao}

``Khi cho sodium tác dụng với chlorine, ta thu được sodium chlorine (NaCl). Phản ứng giữa sodium \& chlorine có thể được minh họa bởi sơ đồ:

\begin{figure}[H]
	\centering
	\includegraphics[scale=0.15]{su_hinh_thanh_lien_ket_ion_trong_NaCl}
	\caption{Minh họa sự hình thành liên kết ion trong phân tử NaCl, \cite[Hình 9.2, p. 56]{SGK_Hoa_Hoc_10_Chan_Troi_Sang_Tao}.}
\end{figure}
Phương trình hóa học: $\rm 2Na + Cl_2\to 2NaCl$.

\begin{dinhnghia}[Liên kết ion]
	\emph{Liên kết ion} là liên kết được hình thành bởi lực hút tĩnh điện giữa các ion mang điện tích trái dấu.
\end{dinhnghia}
Liên kết ion thường được hình thành khi kim loại điển hình tác dụng với phi kim điển hình.'' -- \cite[p. 56]{SGK_Hoa_Hoc_10_Chan_Troi_Sang_Tao}

\subsection{Tinh thể ion}
``NaCl là hợp chất ion phổ biến \& quen thuộc trong đời sống. Trong điều kiện thường, hợp chất này tồn tại dưới dạng tinh thể rắn, cứng, dễ tan trong nước \& có nhiệt độ nóng chảy khá cao ($801^\circ$C).

\begin{figure}[H]
	\centering
	\includegraphics[scale=0.15]{tinh_the_NaCl}
	\caption{Tinh thể NaCl thực tế \& mô hình ô mạng tinh thể NaCl, \cite[Hình 9.3, p. 56]{SGK_Hoa_Hoc_10_Chan_Troi_Sang_Tao}.}
\end{figure}
Ô mạng tinh thể là đơn vị nhỏ nhất của mạng tinh thể, hiển thị cấu trúc không gian 3 chiều của toàn bộ tinh thể. Tinh thể của 1 chất có thể xem là 1 ô mạng lặp đi lặp lại trong không gian 3 chiều.'' -- \cite[p. 56]{SGK_Hoa_Hoc_10_Chan_Troi_Sang_Tao}

``Do các hợp chất ion có cấu trúc tinh thể \& lực hút tĩnh điện mạnh nên chúng thường tồn tại ở trạng thái rắn trong điều kiện thường. Trong điều kiện thường, các hợp chất ion thường tồn tại ở trạng thái rắn, khó nóng chảy, khó bay hơi \& không dẫn điện ở trạng thái rắn. Hợp chất ion thường dễ tan trong nước, tạo thành dung dịch có khả năng dẫn điện.'' -- \cite[p. 57]{SGK_Hoa_Hoc_10_Chan_Troi_Sang_Tao}

\begin{vidu}[Ion $\rm Na^+$]
	``Ion $\rm Na^+$ đóng vai trò rất quan trọng trong việc điều hòa huyết áp của cơ thể. Tuy nhiên, nếu cơ thể hấp thụ 1 lượng lớn ion này sẽ dẫn đến các vấn đề về tim mạch \& thận. Các nhà khoa học khuyến cáo lượng ion $\rm Na^+$ nạp vào cơ thể nên $\in[500;2300)$ mg mỗi ngày đối với 1 người lớn để đảm bảo sức khỏe.'' -- \cite[p. 57]{SGK_Hoa_Hoc_10_Chan_Troi_Sang_Tao}
\end{vidu}

\begin{vidu}[Sodium oxide $\rm Na_2O$]
	``\emph{Sodium oxide ($\rm Na_2O$)} có trong thành phần thủy tinh \& các sản phẩm gốm sứ.'' -- \cite[p. 58]{SGK_Hoa_Hoc_10_Chan_Troi_Sang_Tao}
\end{vidu}

%------------------------------------------------------------------------------%

\section{Liên Kết Cộng Hóa Trị}

%------------------------------------------------------------------------------%

\section{Liên Kết Hyđrogen \& Tương Tác van der Waals}

%------------------------------------------------------------------------------%

\chapter{Phản Ứng Oxi Hóa -- Khử}

\section{Phản Ứng Oxi Hóa -- Khử \& Ứng Dụng Trong Cuộc Sống}

%------------------------------------------------------------------------------%

\chapter{Năng Lượng Hóa Học}

\section{Enthalpy Tạo Thành \& Biến Thiên Enthalpy của Phản Ứng Hóa Học}

%------------------------------------------------------------------------------%

\section{Tính Biến Thiên Enthalpy của Phản Ứng Hóa Học}

%------------------------------------------------------------------------------%

\chapter{Tốc Độ Phản Ứng Hóa Học}

\section{Phương Trình Tốc Độ Phản Ứng \& Hằng Số Tốc Độ Phản Ứng}

%------------------------------------------------------------------------------%

\section{Các Yếu Tố Ảnh Hưởng Đến Tốc Độ Phản Ứng Hóa Học}

%------------------------------------------------------------------------------%

\chapter{Nguyên Tố Nhóm VIIA -- Halogen}

\section{Tính Chất Vật Lý \& Hóa Học Các Đơn Chất Nhóm VIIA}

%------------------------------------------------------------------------------%

\section{Hyđrogen Halide \& 1 Số Phản Ứng của Ion Halide} 

%------------------------------------------------------------------------------%

\printbibliography[heading=bibintoc]
	
\end{document}