\documentclass{article}
\usepackage[backend=biber,natbib=true,style=authoryear]{biblatex}
\addbibresource{/home/hong/1_NQBH/reference/bib.bib}
\usepackage[utf8]{vietnam}
\usepackage{tocloft}
\renewcommand{\cftsecleader}{\cftdotfill{\cftdotsep}}
\usepackage[colorlinks=true,linkcolor=blue,urlcolor=red,citecolor=magenta]{hyperref}
\usepackage{amsmath,amssymb,amsthm,mathtools,float,graphicx,algpseudocode,algorithm,tcolorbox,,diagbox}
\usepackage[inline]{enumitem}
\allowdisplaybreaks
\numberwithin{equation}{section}
\newtheorem{assumption}{Assumption}[section]
\newtheorem{conjecture}{Conjecture}[section]
\newtheorem{corollary}{Corollary}[section]
\newtheorem{hequa}{Hệ quả}[section]
\newtheorem{definition}{Definition}[section]
\newtheorem{dinhnghia}{Định nghĩa}[section]
\newtheorem{example}{Example}[section]
\newtheorem{vidu}{Ví dụ}[section]
\newtheorem{lemma}{Lemma}[section]
\newtheorem{notation}{Notation}[section]
\newtheorem{principle}{Principle}[section]
\newtheorem{problem}{Problem}[section]
\newtheorem{baitoan}{Bài toán}[section]
\newtheorem{proposition}{Proposition}[section]
\newtheorem{question}{Question}[section]
\newtheorem{cauhoi}{Câu hỏi}[section]
\newtheorem{remark}{Remark}[section]
\newtheorem{luuy}{Lưu ý}[section]
\newtheorem{theorem}{Theorem}[section]
\newtheorem{dinhly}{Định lý}[section]
\usepackage[left=0.5in,right=0.5in,top=1.5cm,bottom=1.5cm]{geometry}
\usepackage{fancyhdr}
\pagestyle{fancy}
\fancyhf{}
\lhead{\small Sect.~\thesection}
\rhead{\small \nouppercase{\leftmark}}
\renewcommand{\sectionmark}[1]{\markboth{#1}{}}
\cfoot{\thepage}
\def\labelitemii{$\circ$}

\title{Cheatsheet in Elementary Chemistry\texttt{/}Grade 10}
\author{Nguyễn Quản Bá Hồng\footnote{Independent Researcher, Ben Tre City, Vietnam\\e-mail: \texttt{nguyenquanbahong@gmail.com}; website: \url{https://nqbh.github.io}.}}
\date{\today}

\begin{document}
\maketitle
\begin{abstract}
	\textsc{[en]} This text is a cheatsheet of formulas in Elementary Chemistry Grade 10. This text is also a supplementary material for my lecture note on Elementary Chemistry grade 10, which is stored \& downloadable at the following link: \href{https://github.com/NQBH/hobby/blob/master/elementary_chemistry/grade_10/NQBH_elementary_chemistry_grade_10.pdf}{GitHub\texttt{/}NQBH\texttt{/}hobby\texttt{/}elementary chemistry\texttt{/}grade 10\texttt{/}lecture}\footnote{\textsc{url}: \url{https://github.com/NQBH/hobby/blob/master/elementary_chemistry/grade_10/NQBH_elementary_chemistry_grade_10.pdf}.}. The latest version of this text has been stored \& downloadable at the following link: \href{https://github.com/NQBH/hobby/blob/master/elementary_chemistry/grade_10/cheatsheet/NQBH_elementary_chemistry_grade_10_cheatsheet.pdf}{GitHub\texttt{/}NQBH\texttt{/}hobby\texttt{/}elementary chemistry\texttt{/}grade 10\texttt{/}cheatsheet}\footnote{\textsc{url}: \url{https://github.com/NQBH/hobby/blob/master/elementary_chemistry/grade_10/cheatsheet/NQBH_elementary_chemistry_grade_10_cheatsheet.pdf}.}.
	\vspace{2mm}
	
	\textsc{[vi]} Tài liệu này là 1 bảng tóm tắt kiến thức \& công thức của Hóa Học Sơ Cấp lớp 10. Tài liệu này là phần bài tập bổ sung cho tài liệu chính -- bài giảng \href{https://github.com/NQBH/hobby/blob/master/elementary_chemistry/grade_10/NQBH_elementary_chemistry_grade_10.pdf}{GitHub\texttt{/}NQBH\texttt{/}hobby\texttt{/}elementary chemistry\texttt{/}grade 10\texttt{/}lecture} của tác giả viết cho Hóa Học Sơ Cấp lớp 10. Phiên bản mới nhất của tài liệu này được lưu trữ \& có thể tải xuống ở link sau: \href{https://github.com/NQBH/hobby/blob/master/elementary_chemistry/grade_10/cheatsheet/NQBH_elementary_chemistry_grade_10_cheatsheet.pdf}{GitHub\texttt{/}NQBH\texttt{/}hobby\texttt{/}elementary chemistry\texttt{/}grade 10\texttt{/}cheatsheet}.
\end{abstract}
\tableofcontents
\newpage

%------------------------------------------------------------------------------%

\section{Cấu Tạo Nguyên Tử}
\textbf{\S2. Thành Phần của Nguyên Tử.} \textit{}Nguyên tử gồm hạt nhân chứa proton, neutron \& vỏ nguyên tử chứa electron. Hạt electron (e) có: Điện tích: $q_{\rm e} = -1.602\cdot 10^{-19}$ C (Coulomb). Khối lượng: $m_{\rm e} = 9.11\cdot 10^{-28}$ g. Điện tích của electron: điện tích đơn vị $-1$. Nguyên tử có cấu tạo rỗng, gồm hạt nhân ở trung tâm \& lớp vỏ là các electron chuyển động xung quanh hạt nhân. Nguyên tử trung hòa về điện: \textit{số đơn vị điện tích dương của hạt nhân bằng số đơn vị điện tích âm của các electron trong nguyên tử}. Hạt nhân nguyên tử gồm 2 loại hạt là proton \& neutron. Proton mang điện tích dương ($+1$) \& neutron không mang điện. Proton \& neutron có khối lượng gần bằng nhau. Đơn vị nanomet (nm) hay angstrom ($\mathring{\rm A}$) thường được sử dụng để biểu thị kích thước nguyên tử. $\rm 1\ nm = 10^{-9}\ m$, $\rm 1\ \mathring{\rm A} = 10^{-10}\ m$, $\rm 1\ nm = 10\ \mathring{\rm A}$.
\begin{table}[h]
	\centering
	\begin{tabular}{|c|c|c|c|}
		\hline
		\textbf{Hạt} & \textbf{Điện tích tương đối} & \textbf{Khối lượng (amu)} & \textbf{Khối lượng (g)} \\
		\hline
		\textbf{p} & $+1$ & $\approx 1$ & $1.673\cdot 10^{-24}$ \\
		\hline
		\textbf{n} & $0$ & $\approx 1$ & $1.675\cdot 10^{-24}$ \\
		\hline 
		\textbf{e} & $-1$ & $\frac{1}{1840}\approx 0.00055$ & $9.11\cdot 10^{-28}$ \\
		\hline
	\end{tabular}
	\caption{1 số tính chất của các loại hạt cơ bản trong nguyên tử.}
\end{table}
Để biểu thị khối lượng của nguyên tử, các hạt proton, neutron, \& electron, người ta dùng đơn vị \textit{khối lượng nguyên tử}, ký hiệu là amu. 1 amu bằng $\frac{1}{12}$ khối lượng nguyên tử của carbon $-12$. $\rm 1\ amu = 1.66\cdot 10^{-24}\ g$. Khối lượng của nguyên tử gần bằng khối lượng hạt nhân do khối lượng của các electron không đáng kể so với khối lượng của proton \& neutron. \textbf{\S3. Nguyên tố hóa học.} Số đơn vị điện tích hạt nhân\texttt{/}số hiệu nguyên tử (Z) $=$ số proton (P) $=$ số electron (E). Điện tích hạt nhân $= +$Z. Số khối (A) $=$ số proton (P) + số neutron (N). $\rm{}_Z^AX$ (A: số khối, Z: số hiệu nguyên tử, X: ký hiệu nguyên tố hóa học). Công thức tính nguyên tử khối trung bình của nguyên tố X: $\overline{A}_X = \frac{1}{100}\sum_{i=1}^k a_iA_i$, $\overline{A}_X$ là nguyên tử khối trung bình của $X$, $A_i$ là nguyên tử khối đồng vị thứ $i$, $a_i$ là tỷ lệ \% số nguyên tử đồng vị thứ $i$, $i = 1,\ldots,k$. \textbf{\S4. Cấu trúc lớp vỏ electron của nguyên tử.} Các electron chuyển động rất nhanh xung quanh hạt nhân với xác suất tìm thấy không giống nhau, tạo thành đám mây electron. Khu vực không gian xung quanh hạt nhân mà tại đó xác suất có mặt (xác suất tìm thấy) electron $\approx90\%$ gọi là \textit{orbital nguyên tử}. \emph{Orbital nguyên tử} (\emph{Atomic Orbital}, abbr., \emph{AO}) là khu vực không gian xung quanh hạt nhân nguyên tử mà tại đó xác suất tìm thấy electron là lớn nhất ($\approx90\%$). 1 số AO thường gặp: s, p, d, f. Các AO có hình dạng khác nhau: AO s có dạng hình cầu, AO p có dạng hình số 8 nổi, AO d \& f có hình dạng phức tạp. Trong nguyên tử, các electron được sắp xếp thành từng lớp \& phân lớp theo năng lượng từ thấp đến cao. Trong nguyên tử, các electron được sắp xếp thành từng lớp (ký hiệu K, L, M, N, O, P, Q) từ gần đến xa hạt nhân, theo thứ tự từ lớp $n = 1$ đến $n = 7$. Các electron trên cùng 1 lớp có năng lượng gần bằng nhau. Trong 1 phân lớp, các orbital có cùng mức năng lượng, chỉ khác nhau về sự định hướng trong không gian. Số lượng \& hình dạng orbital phụ thuộc vào đặc điểm của mỗi phân lớp electron. Mỗi lớp electron phân chia thành các phân lớp, được ký hiệu bằng các chữ cái viết thường: s, p, d, f. Các electron thuộc các phân lớp s, p, d, \& f lần lượt có các số AO tương ứng 1, 3, 5, \& 7. Các electron trên cùng 1 phân lớp có năng lượng bằng nhau. Trong 1 phân lớp, các orbital có cùng mức năng lượng, chỉ khác nhau về sự định hướng trong không gian. Số lượng \& hình dạng orbital phụ thuộc vào đặc điểm của mỗi phân lớp electron. Mỗi lớp electron phân chia thành các phân lớp, được ký hiệu bằng các chữ cái viết thường: s, p, d, f. Các electron thuộc các phân lớp s, p, d, \& f được gọi tương ứng là các electron s, p, d, \& f. Các phân lớp s, p, d, \& f lần lượt có các số AO tương ứng 1, 3, 5, \& 7. Các electron trên cùng 1 phân lớp có năng lượng bằng nhau. Với 4 lớp đầu (1, 2, 3, 4) số phân lớp trong mỗi lớp bằng số thứ tự của lớp đó. Trong nguyên tử, các electron trên mỗi orbital có 1 mức năng lượng xác định. Người ta gọi mức năng lượng này là \textit{mức năng lượng orbital nguyên tử} (\textit{mức năng lượng AO}). Các electron trên các orbital khác nhau của cùng 1 phân lớp có năng lượng như nhau. E.g., phân lớp 2p có 3 orbital $\rm 2p_x,2p_y,2p_z$; các electron của các orbital p trong phân lớp này tuy có sự định hướng trong không gian khác nhau nhưng chúng có cùng mức năng lượng AO. \textit{Nguyên lý vững bền}: Ở trạng thái cơ bản, các electron trong nguyên tử chiếm lần lượt những orbital có mức năng lượng từ thấp đến cao: 1s 2s 2p 3s 3p 4s 3d 4p 5s 4d 5p $\ldots$. Để biểu diễn orbital nguyên tử, người ta sử dụng các ô vuông, gọi là \textit{ô lượng tử}. Mỗi ô lượng tử ứng với 1 AO. Mỗi AO chứa tối đa 2 electron. Nếu trong AO chỉ chứa 1 electron thì electron đó gọi là \textit{electron độc thân} (ký hiệu bởi 1 mũi tên hướng lên $\uparrow$). Ngược lại, nếu AO chứa đủ 2 electron thì các electron đó gọi là \textit{electron ghép đôi} (ký hiệu bởi 2 mũi tên ngược chiều nhau $\uparrow\downarrow$). \textit{Nguyên lý Pauli}: Mỗi orbital chỉ chứa tối đa 2 electron \& có chiều tự quay ngược nhau. Số electron tối đa trong lớp $n$ là $2n^2$ ($n\le 4$).
\begin{table}[h]
	\centering
	\begin{tabular}{|c|c|l|p{2.5cm}|p{4cm}|p{3cm}|}
		\hline
		$n$ & \textbf{Tên lớp} & \textbf{Tên phân lớp} & \textbf{Số AO trong mỗi phân lớp} & \textbf{Số electron tối đa trong mỗi phân lớp} & \textbf{Số electron tối đa trong mỗi lớp} \\
		\hline
		1 & K & s & 1 & 2 & 2 \\
		\hline
		2 & L & s, p & 1, 3 & 2, 6 & 8 \\
		\hline
		3 & M & s, p, d & 1, 3, 5 & 2, 6, 10 & 18 \\
		\hline
		4 & N & s, p, d, f & 1, 3, 5, 7 & 2, 4, 10, 14 & 32 \\
		\hline
	\end{tabular}
	\caption{Số AO \& số electron tối đa của các lớp $n = 1$ đến $n = 4$.}
\end{table}
Các phân lớp: $\rm s^2,p^6,d^{10},f^{14}$ chứa đủ số electron tối đa gọi là \textit{phân lớp bão hòa}. Các phân lớp $\rm s^1,p^3,d^5,f^7$ chứa 1 nửa số electron tối đa gọi là \textit{phân lớp nửa bão hòa}. Các phân lớp chưa đủ số electron tối đa gọi là \textit{phân lớp chưa bão hòa}. \textit{Quy tắc Hund}: Trong cùng 1 phân lớp chưa bão hòa, các electron sẽ phân bố vào các orbital sao cho số electron độc thân là tối đa. \textit{Cách viết cấu hình electron}:
\begin{enumerate*}
	\item[\textbf{1.}] Xác định số electron của nguyên tử.
	\item[\textbf{2.}] Các electron được phân bố theo thứ tự các AO có mức năng lượng tăng dần, theo các nguyên tắc \& quy tắc phân bố electron trong nguyên tử.
	\item[\textbf{3.}] Viết cấu hình electron theo thứ tự các phân lớp trong 1 lớp \& theo thứ tự của các lớp electron.
\end{enumerate*}
Cấu hình electron nguyên tử phải được viết theo thứ tự các lớp electrron \& phân lớp trong mỗi lớp. Trong đó:
\begin{enumerate*}
	\item[$\bullet$] Số thứ tự lớp electron được viết bằng các số tự nhiên ($n = 1,2,3,\ldots$).
	\item[$\bullet$] Phân lớp được ký hiệu bằng các chữ cái thường s, p, d, f.
	\item[$\bullet$] Số electron của từng phân lớp được ghi bằng chỉ số ở phía trên, bên phải ký hiệu của phân lớp.'' -- \cite[p. 32]{SGK_Hoa_Hoc_10_Chan_Troi_Sang_Tao}
\end{enumerate*}
Dựa vào các nguyên lý \& quy tắc nêu ở trên, ta có thể viết cấu hình electron nguyên tử của các nguyên tố. Khi tham gia các phản ứng hóa học, thông thương electron lớp ngoài cùng của các nguyên tử sẽ thay đổi, chúng có vai trò quyết định đến tính chất hóa học đặc trưng của nguyên tố (tính kim loại, tính phi kim, $\ldots$). Các nguyên tử có $1,2,3$ electron ở lớp ngoài cùng thường là các nguyên tử của nguyên tố kim loại; các nguyên tử có $5,6,7$ electron ở lớp ngoài cùng thường là nguyên tử của các nguyên tố phi kim; các nguyên tử có 4 electron ở lớp ngoài cùng có thể là nguyên tử của nguyên tố kim loại hoặc phi kim; các nguyên tử có 8 electron ở lớp ngoài cùng là nguyên tử của nguyên tố khí hiếm (trừ He có 2 electron ở lớp ngoài cùng). Dựa vào số lượng electron lớp ngoài cùng của nguyên tử nguyên tố, có thể dự đoán 1 nguyên tố là kim loại, phi kim hay khí hiếm.

%------------------------------------------------------------------------------%

\section{Bảng Tuần Hoàn Các Nguyên Tố Hóa Học}
\textbf{\S5. Cấu tạo bảng tuần hoàn các nguyên tố hóa học.} Mỗi nguyên tố hóa học được xếp vào 1 ô trong bảng tuần hoàn các nguyên tố hóa học, gọi là \textit{ô nguyên tố}. Số thự tự của 1 ô nguyên tố bằng số hiệu nguyên tử của nguyên tố hóa học trong ô đó. Các nguyên tố có cùng số lớp electron trong nguyên tử được xếp thành 1 hàng, gọi là \textit{chu kỳ}. Số thứ tự của chu kỳ bằng số lớp electron của nguyên tử các nguyên tố trong chu kỳ. Bảng tuần hoàn gồm 7 chu kỳ: 
\begin{enumerate*}
	\item[$\bullet$] Các chu kỳ 1, 2, \& 3 là \textit{các chu kỳ nhỏ}.
	\item[$\bullet$] Các chu kỳ 4, 5, 6, \& 7 là \textit{các chu kỳ lớn}.
\end{enumerate*}
Bảng tuần hoàn hiện nay có 18 cột, chia thành 8 nhóm A (IA--VIIIA) \& 8 nhóm B (IB--VIIIB). Mỗi cột tương ứng với 1 nhóm, riêng nhóm VIIIB có 3 cột. \textit{Electron hóa trị} là những electron có khả năng tham gia hình thành liên kết hóa học. Chúng thường nằm ở \textit{lớp ngoài cùng} hoặc ở cả \textit{phân lớp sát lớp ngoài cùng} nếu phân lớp đó chưa bão hòa. Những nguyên tố có cùng số electron hóa trị thường có tính chất hóa học tương tự nhau. \textit{Nhóm} là tập hợp các nguyên tố mà nguyên tử có cấu hình electron tương tự nhau (trừ nhóm VIIIB), do đó có tính chất hóa học gần giống nhau \& được xếp theo cột. Số thứ tự của nhóm A bằng số electron ở lớp ngoài cùng của nguyên tử các nguyên tố trong nhóm. Các nguyên tố hóa học cũng có thể được chia thành các khối như sau:
\begin{enumerate*}
	\item[$\bullet$] \textit{Khối các nguyên tố $\rm s$} gồm các nguyên tố thuộc nhóm IA \& nhóm IIA, có cấu hình electron: [Khí hiếm] $n{\rm s}^{1\div 2}$.
	\item[$\bullet$] \textit{Khối các nguyên tố $\rm p$} gồm các nguyên tố thuộc nhóm IIIA--nhóm VIIIA (trừ nguyên tố He), có cấu hình electron: [Khí hiếm] $n{\rm s}^2n{\rm p}^{1\div 6}$.
	\item[$\bullet$] \textit{Khối các nguyên tố $\rm d$} gồm các nguyên tố thuộc nhóm B, có cấu hình electron: [Khí hiếm] $(n - 1){\rm d}^{1\div 10}n{\rm s}^{1\div 2}$.
	\item[$\bullet$] \textit{Khối các nguyên tố $\rm f$} gồm các nguyên tố xếp thành 2 hàng ở cuối bảng tuần hoàn, có cấu hình electron: [Khí hiếm] $(n - 2){\rm f}^{0\div 14}(n - 1){\rm d}^{0\div 2}n{\rm s}^2$ (trong đó $n = 6$  \& $n = 7$). Chúng gồm 14 nguyên tố họ Lanthanide (từ Ce đến Lu) \& 14 nguyên tố họ Actinide (từ Th đến Lr).
\end{enumerate*}
Dựa vào cấu hình electron, người ta phân loại các nguyên tố thành nguyên tố s, nguyên tố p, nguyên tố d, \& nguyên tố f. Dựa vào tính chất hóa học, người ta phân loại các nguyên tố thành nguyên tố kim loại, nguyên tố phi kim \& nguyên tố khí hiếm. Các nguyên tố hóa học được xếp vào 1 bảng theo những nguyên tắc nhất định, gọi là \textit{bảng tuần hoàn}. Bảng tuần hoàn hiện nay gồm 118 nguyên tố hóa học. Vị trí của mỗi nguyên tố hóa học trong bảng tuần hoàn được xác định qua số thứ tự ô nguyên tố, chu kỳ, \& nhóm. Khi sắp xếp như vậy, sự tuần hoàn tính chất của các đơn chất \& hợp chất được thể hiện qua chu kỳ \& nhóm. \textit{Nguyên tắc sắp xếp các nguyên tố trong bảng tuần hoàn}:
\begin{enumerate*}
	\item[$\bullet$] Các nguyên tố được xếp theo chiều tăng dần của điện tích hạt nhân nguyên tử.
	\item[$\bullet$] Các nguyên tố có cùng số lớp electron trong nguyên tử được xếp cùng 1 chu kỳ.
	\item[$\bullet$] Các nguyên tố có cùng số electron hóa trị trong nguyên tử được xếp cùng 1 nhóm, trừ nhóm VIIIB.
\end{enumerate*}
\textbf{\S6. Xu hướng biến đổi 1 số tính chất của nguyên tử các nguyên tố, thành phần, \& 1 số tính chất của hợp chất trong 1 chu kỳ \& nhóm.} \textit{Kim loại kiềm} là các kim loại thuộc nhóm IA, bao gồm: lithium (Li), sodium (Na), potassium (K), rubidium (Rb), caesium (Cs), francium (Fr). Chúng phản ứng được với nước \& giải phóng khí hydrogen. \textit{Xu hướng biến đổi bán kính nguyên tử}: Bán kính nguyên tử của các nguyên tố nhóm A có xu hướng biến đổi tuần hoàn theo chiều tăng của điện tích hạt nhân:
\begin{enumerate*}
	\item[$\bullet$] \textit{Trong 1 chu kỳ}, nguyên tử của các nguyên tố có cùng số lớp electron. Từ trái sang phải, điện tích hạt nhân nguyên tử tăng dần nên electron lớp ngoài cùng sẽ bị hạt nhân hút mạnh hơn, vì vậy bán kính nguyên tử của các nguyên tố có xu hướng \textit{giảm dần}.
	\item[$\bullet$] \textit{Trong 1 nhóm}, theo chiều từ trên xuống dưới, số lớp electron tăng dần nên bán kính nguyên tử có xu hướng \textit{tăng}.
\end{enumerate*}
\textit{Độ âm điện} của 1 nguyên tử đặc trưng cho khả năng hút electron của nguyên tử đó khi tạo thành liên kết hóa học. Trong hóa học, có nhiều thang đo độ âm điện khác nhau do các nhà khoa học tính toán dựa trên những cơ sở khác nhau. \begin{table}[h]
	\centering
	\begin{tabular}{|c|c|c|c|c|c|c|c|c|}
		\hline
		\diagbox{\textbf{Chu kỳ}}{\textbf{Nhóm}}& IA & IIA & IIIA & IVA & VA & VIA & VIIA & VIIIA \\
		\hline
		1 & H 2.20 &  &  &  &  &  &  & He \\
		\hline
		2 & Li 0.98 & Be 1.57 & B 2.04 & C 2.55 & N 3.04 & O 3.44 & F 3.98 & Ne \\
		\hline
		3 & Na 0.93 & Mg 1.31 & Al 1.61 & Si 1.90 & P 2.19 & S 2.58 & Cl 3.16 & Ar \\
		\hline
		4 & K 0.82 & Ca 1.00 & Ga 1.81 & Ge 2.01 & As 2.18 & Se 2.55 & Br 2.96 & Kr \\
		\hline
		5 & Rb 0.82 & Sr 0.95 & In 1.78 & Sn 1.96 & Sb 2.05 & Te 2.10 & I 2.66 & Xe \\
		\hline
		6 & Cs 0.79 & Ba 0.89 & Tl 1.80 & Pb 1.80 & Bi 1.90 & Po 2.00 & At 2.20 & Rn \\
		\hline
	\end{tabular}
	\caption{Giá trị độ âm điện của nguyên tử 1 số nguyên tố nhóm A theo Pauling.}
\end{table}
\textit{Xu hướng biến đổi độ âm điện}: Độ âm điện của nguyên tử các nguyên tố nhóm A có xu hướng biến đổi tuần hoàn theo chiều tăng của điện tích hạt nhân:
\begin{enumerate*}
	\item[$\bullet$] \textit{Trong 1 chu kỳ}, theo chiều tăng dần của điện tích hạt nhân, lực hút giữa hạt nhân với các electron lớp ngoài cùng cũng tăng. Do đó, độ âm điện của nguyên tử các nguyên tố có xu hướng \textit{tăng dần}.
	\item[$\bullet$] \textit{Trong 1 nhóm}, theo chiều tăng dần của điện tích hạt nhân, bán kính nguyên tử tăng nhanh, lực hút giữa hạt nhân với các electron lớp ngoài cùng giảm. Do đó, độ âm điện của nguyên tử các nguyên tố có xu hướng \textit{giảm dần}.
\end{enumerate*}
\emph{Tính kim loại} là tính chất của 1 nguyên tố mà nguyên tử dễ nhường electron. \emph{Tính phi kim} là tính chất của 1 nguyên tố mà nguyên tử dễ nhận electron. \textit{Xu hướng biến đổi tính kim loại, tính phi kim}: Tính kim loại, tính phi kim của các nguyên tố nhóm A có xu hướng biến đổi tuần hoàn theo chiều tăng của điện tích hạt nhân:
\begin{enumerate*}
	\item[$\bullet$] \textit{Trong 1 chu kỳ}, theo chiều tăng dần của điện tích hạt nhân, lực hút giữa hạt nhân với các electron lớp ngoài cùng tăng. Do đó, \textit{tính kim loại của các nguyên tố giảm dần, tính phi kim tăng dần}.
	\item[$\bullet$] \textit{Trong 1 nhóm}, theo chiều tăng dần của điện tích hạt nhân, lực hút giữa hạt nhân với các electron lớp ngoài cùng giảm. Do đó, \textit{tính kim loại của các nguyên tố tăng dần, tính phi kim giảm dần}.
\end{enumerate*}
\textit{Tính acid--base của oxide \& hydroxide}:
\begin{table}[h]
	\centering
	\resizebox{\columnwidth}{!}{%
		\begin{tabular}{|p{0.105\textwidth}|p{0.17\textwidth}|p{0.17\textwidth}|p{0.12\textwidth}|p{0.115\textwidth}|p{0.115\textwidth}|p{0.115\textwidth}|}
			\hline
			\textbf{IA} & \textbf{IIA} & \textbf{IIIA} & \textbf{IVA} & \textbf{VA} & \textbf{VIA} & \textbf{VIIA} \\
			\hline
			$\rm Li_2O$\newline(basic oxide) & $\rm BeO$\newline(Oxide lưỡng tính) & $\rm B_2O_3$\newline(Acidic oxide) & $\rm CO_2$\newline(Acidic oxide) & $\rm N_2O_5$\newline(Acidic oxide) &  &  \\
			\hline
			$\rm LiOH$\newline(Base mạnh) & $\rm Be(OH)_2$ (Hydroxide lưỡng tính) & $\rm H_3BO_3$\newline(Acid yếu) & $\rm H_2CO_3$\newline(Acid yếu) & $\rm HNO_3$\newline(Acid mạnh) &  &  \\
			\hline
			$\rm Na_2O$\newline(Basic oxide) & $\rm MgO$\newline(Basic oxide) & $\rm Al_2O_3$\newline(Oxide lưỡng tính) & $\rm SiO_2$\newline(Acidic oxide) & $\rm P_2O_5$\newline(Acidic oxide) & $\rm SO_3$\newline(Acidic oxide) & $\rm Cl_2O_7$\newline(Acidic oxide) \\
			\hline
			$\rm NaOH$\newline(Base mạnh) & $\rm Mg(OH)_2$\newline(Base yếu) & $\rm Al(OH)_3$ (Hydroxide lưỡng tính) & $\rm H_2SiO_3$\newline(Acid yếu) & $\rm H_3PO_4$ (Acid trung bình) & $\rm H_2SO_4$\newline(Acide mạnh) & $\rm HClO_4$ (Acid rất mạnh) \\
			\hline
		\end{tabular}%
	}
	\caption{Tính acid -- base của oxide \& hydroxide tương ứng của các nguyên tố thuộc chu kỳ 2 \& 3 (ứng với hóa trị cao nhất của các nguyên tố).}
\end{table}
Trong 1 chu kỳ, theo chiều tăng dần của điện tích hạt nhân, tính base của oxide \& hydroxide tương ứng giảm dần, tính acid của chúng tăng dần. \textbf{\S7. Định luật tuần hoàn -- ý nghĩa của bảng tuần hoàn các nguyên tố hóa học.} Sự biến đổi tuần hoàn về cấu hình electron lớp ngoài cùng của nguyên tử các nguyên tố khi điện tích hạt nhân tăng dần chính là nguyên nhân của sự biến đổi tuần hoàn về tính chất của các nguyên tố, cũng như hợp chất của chúng. \textit{Định luật tuần hoàn}: Tính chất của các nguyên tố \& đơn chất, cũng như thành phần \& tính chất của các hợp chất tạo nên từ các nguyên tố đó biến đổi tuần hoàn theo chiều tăng của điện tích hạt nhân nguyên tử. \textit{Ý nghĩa của bảng tuần hoàn các nguyên tố hóa học}: Khi biết vị trí của 1 nguyên tố trong bảng tuần hoàn, có thể suy ra cấu tạo nguyên tử của nguyên tố đó \& ngược lại. Từ đó, có thể suy ra những tính chất hóa học cơ bản của nó. \textit{Quy tắc octet}\texttt{/}\textit{bát tử}:  Trong quá trình hình thành liên kết hóa học, nguyên tử của các nguyên tố nhóm A có xu hướng tạo thành lớp vỏ ngoài cùng có 8 electron tương ứng với khí hiếm gần nhất (hoặc 2 electron với khí hiếm helium). Không phải trong mọi trường hợp, nguyên tử của các nguyên tố khi tham gia liên kết đều tuân theo quy tắc octet. Người ta nhận thấy 1 số phân tử có thể không tuân theo quy tắc octet, e.g., $\rm NO,BH_3,SF_6,\ldots$. Với nguyên tử của các nguyên tố nhóm B, người ta áp dụng 1 quy tắc khác, tương ứng với quy tắc octet, là quy tắc 18 electron để giải thích xu hướng khi tham gia liên kết hóa học của chúng.

%------------------------------------------------------------------------------%

\section{Liên Kết Hóa Học}
\textbf{\S8. Quy tắc octet.} Khi liên kết với nhau, nguyên tử của các nguyên tố dường như đã cố gắng ``bắt chước'' cấu hình electron nguyên tử của các nguyên tố khí hiếm (i.e., 8 electron ở lớp ngoài cùng) để bền vững hơn. Phân tử được tạo nên từ các nguyên tử bằng các \textit{liên kết hóa học}. Để đạt cấu hình electron bền vững của các khí hiếm gần nhất, nguyên tử của các nguyên tố có xu hướng nhường, hoặc nhận thêm, hoặc góp chung các electron hóa trị với các nguyên tử khác khi tham gia liên kết hóa học. \textbf{\S9. Liên kết ion.} Khi cho electron, nguyên tử trở thành \textit{ion dương} (\textit{cation}). Khi nhận electron, nguyên tử trở thành \textit{ion âm} (\textit{anion}). Giá trị điện tích trên cation hoặc anion bằng số electron mà nguyên tử đã nhường hoặc nhận. \emph{Liên kết ion} là liên kết được hình thành bởi lực hút tĩnh điện giữa các ion mang điện tích trái dấu. Liên kết ion thường được hình thành khi kim loại điển hình tác dụng với phi kim điển hình. Ô mạng tinh thể là đơn vị nhỏ nhất của mạng tinh thể, hiển thị cấu trúc không gian 3 chiều của toàn bộ tinh thể. Tinh thể của 1 chất có thể xem là 1 ô mạng lặp đi lặp lại trong không gian 3 chiều. Do các hợp chất ion có cấu trúc tinh thể \& lực hút tĩnh điện mạnh nên chúng thường tồn tại ở trạng thái rắn trong điều kiện thường. Trong điều kiện thường, các hợp chất ion thường tồn tại ở trạng thái rắn, khó nóng chảy, khó bay hơi \& không dẫn điện ở trạng thái rắn. Hợp chất ion thường dễ tan trong nước, tạo thành dung dịch có khả năng dẫn điện.\newpage\textbf{\S10. Liên kết cộng hóa trị.}

%------------------------------------------------------------------------------%

\section{Phản Ứng Oxi Hóa -- Khử}

%------------------------------------------------------------------------------%

\section{Năng Lượng Hóa Học}

%------------------------------------------------------------------------------%

\printbibliography[heading=bibintoc]	
	
\end{document}