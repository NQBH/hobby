\documentclass{article}
\usepackage[backend=biber,natbib=true,style=authoryear]{biblatex}
\addbibresource{/home/hong/1_NQBH/reference/bib.bib}
\usepackage[utf8]{vietnam}
\usepackage{tocloft}
\renewcommand{\cftsecleader}{\cftdotfill{\cftdotsep}}
\usepackage[colorlinks=true,linkcolor=blue,urlcolor=red,citecolor=magenta]{hyperref}
\usepackage{amsmath,amssymb,amsthm,mathtools,float,graphicx,algpseudocode,algorithm,tcolorbox}
\usepackage[inline]{enumitem}
\allowdisplaybreaks
\numberwithin{equation}{section}
\newtheorem{assumption}{Assumption}[section]
\newtheorem{conjecture}{Conjecture}[section]
\newtheorem{corollary}{Corollary}[section]
\newtheorem{hequa}{Hệ quả}[section]
\newtheorem{definition}{Definition}[section]
\newtheorem{dinhnghia}{Định nghĩa}[section]
\newtheorem{example}{Example}[section]
\newtheorem{vidu}{Ví dụ}[section]
\newtheorem{lemma}{Lemma}[section]
\newtheorem{notation}{Notation}[section]
\newtheorem{principle}{Principle}[section]
\newtheorem{problem}{Problem}[section]
\newtheorem{baitoan}{Bài toán}[section]
\newtheorem{proposition}{Proposition}[section]
\newtheorem{question}{Question}[section]
\newtheorem{cauhoi}{Câu hỏi}[section]
\newtheorem{remark}{Remark}[section]
\newtheorem{luuy}{Lưu ý}[section]
\newtheorem{theorem}{Theorem}[section]
\newtheorem{dinhly}{Định lý}[section]
\usepackage[left=0.5in,right=0.5in,top=1.5cm,bottom=1.5cm]{geometry}
\usepackage{fancyhdr}
\pagestyle{fancy}
\fancyhf{}
\lhead{\small \textsc{Sect.} ~\thesection}
\rhead{\small \nouppercase{\leftmark}}
\renewcommand{\sectionmark}[1]{\markboth{#1}{}}
\cfoot{\thepage}
\def\labelitemii{$\circ$}

\title{Cheatsheet in Elementary Chemistry\texttt{/}Grade 10}
\author{Nguyễn Quản Bá Hồng\footnote{Independent Researcher, Ben Tre City, Vietnam\\e-mail: \texttt{nguyenquanbahong@gmail.com}; website: \url{https://nqbh.github.io}.}}
\date{\today}

\begin{document}
\maketitle
\tableofcontents

\section{Cấu Tạo Nguyên Tử}
\textbf{2.} \textsc{Thành Phần của Nguyên Tử.} Nguyên tử gồm hạt nhân chứa proton, neutron \& vỏ nguyên tử chứa electron. Hạt electron ($\rm e$) có: Điện tích: $q_{\rm e} = -1.602\cdot 10^{-19}$ C (Coulomb). Khối lượng: $m_{\rm e} = 9.11\cdot 10^{-28}$ g. Điện tích của electron: điện tích đơn vị $-1$. Nguyên tử có cấu tạo rỗng, gồm hạt nhân ở trung tâm \& lớp vỏ là các electron chuyển động xung quanh hạt nhân. Nguyên tử trung hòa về điện: \textit{số đơn vị điện tích dương của hạt nhân bằng số đơn vị điện tích âm của các electron trong nguyên tử}.




%------------------------------------------------------------------------------%

\section{Bảng Tuần Hoàn Các Nguyên Tố Hóa Học}

%------------------------------------------------------------------------------%

\section{Liên Kết Hóa Học}

%------------------------------------------------------------------------------%

\section{Phản Ứng Oxi Hóa -- Khử}

%------------------------------------------------------------------------------%

\section{Năng Lượng Hóa Học}

%------------------------------------------------------------------------------%

\printbibliography[heading=bibintoc]	
	
\end{document}