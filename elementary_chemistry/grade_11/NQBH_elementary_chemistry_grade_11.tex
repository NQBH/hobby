\documentclass[oneside]{book}
\usepackage[backend=biber,natbib=true,style=authoryear]{biblatex}
\addbibresource{/home/hong/1_NQBH/reference/bib.bib}
\usepackage[utf8]{vietnam}
\usepackage{tocloft}
\renewcommand{\cftsecleader}{\cftdotfill{\cftdotsep}}
\usepackage[colorlinks=true,linkcolor=blue,urlcolor=red,citecolor=magenta]{hyperref}
\usepackage{amsmath,amssymb,amsthm,mathtools,float,graphicx,algpseudocode,algorithm,tcolorbox,tikz,tkz-tab,diagbox}
\DeclareMathOperator{\arccot}{arccot}
\usepackage[inline]{enumitem}
\allowdisplaybreaks
\numberwithin{equation}{section}
\newtheorem{assumption}{Assumption}[section]
\newtheorem{nhanxet}{Nhận xét}[section]
\newtheorem{conjecture}{Conjecture}[section]
\newtheorem{corollary}{Corollary}[section]
\newtheorem{hequa}{Hệ quả}[section]
\newtheorem{definition}{Definition}[section]
\newtheorem{dinhnghia}{Định nghĩa}[section]
\newtheorem{example}{Example}[section]
\newtheorem{vidu}{Ví dụ}[section]
\newtheorem{lemma}{Lemma}[section]
\newtheorem{notation}{Notation}[section]
\newtheorem{principle}{Principle}[section]
\newtheorem{problem}{Problem}[section]
\newtheorem{baitoan}{Bài toán}[section]
\newtheorem{proposition}{Proposition}[section]
\newtheorem{menhde}{Mệnh đề}[section]
\newtheorem{question}{Question}[section]
\newtheorem{cauhoi}{Câu hỏi}[section]
\newtheorem{remark}{Remark}[section]
\newtheorem{luuy}{Lưu ý}[section]
\newtheorem{theorem}{Theorem}[section]
\newtheorem{dinhly}{Định lý}[section]
\usepackage[left=0.5in,right=0.5in,top=1.5cm,bottom=1.5cm]{geometry}
\usepackage{fancyhdr}
\pagestyle{fancy}
\fancyhf{}
\lhead{\small \textsc{Sect.} ~\thesection}
\rhead{\small \nouppercase{\leftmark}}
\renewcommand{\sectionmark}[1]{\markboth{#1}{}}
\cfoot{\thepage}
\def\labelitemii{$\circ$}

\title{Some Topics in Elementary Chemistry\texttt{/}Grade 11}
\author{Nguyễn Quản Bá Hồng\footnote{Independent Researcher, Ben Tre City, Vietnam\\e-mail: \texttt{nguyenquanbahong@gmail.com}; website: \url{https://nqbh.github.io}.}}
\date{\today}

\begin{document}
\frontmatter
\maketitle
\setcounter{secnumdepth}{4}
\setcounter{tocdepth}{3}
\tableofcontents
\newpage

%------------------------------------------------------------------------------%

\chapter*{Preface}

Tóm tắt kiến thức Hóa học lớp 11 theo chương trình giáo dục của Việt Nam \& một số chủ đề nâng cao.

%------------------------------------------------------------------------------%

\mainmatter

\chapter{Sự Điện Ly}

``S. Arrhenius (1859--1927), người Thụy Điển, được giải Nobel về Hóa học năm 1903.'' -- \cite[p. 3]{SGK_Hoa_Hoc_11_nang_cao}

\section{Sự Điện Ly}
\textbf{Nội dung.} \textit{Các khái niệm về sự điện ly \& chất điện ly, nguyên nhân tính dẫn điện của dung dịch chất điện ly \& cơ chế của quá trình điện ly}.

\subsection{Hiện tượng điện ly}

\subsubsection{Thí nghiệm}
``Chuẩn bị 3 cốc: cốc a đựng nước cất, cốc b đựng dung dịch saccarozơ ($\rm C_{12}H_{22}O_{11}$), cốc c đựng dung dịch natri clorua (NaCl) rồi lắp vào bộ dụng cụ như \cite[Hình 1.1: \textsf{Bộ dụng cụ chứng minh tính dẫn điện của dung dịch}, p. 4]{SGK_Hoa_Hoc_11_nang_cao}. Khi nối các đầu dây dẫn điện với cùng 1 nguồn điện, ta chỉ thấy bóng đèn ở cốc đựng dung dịch NaCl bật sáng. Vậy dung dịch NaCl dẫn điện, còn nước cất \& dung dịch saccarozơ không dẫn điện. Nếu làm các thí nghiệm tương tự, người ta thấy NaCl rắn, khan; NaOH rắn, khan; các dung dịch ancol etylic ($\rm C_2H_5OH$); glixerol ($\rm HOCH_2CH(OH)CH_2OH$) không dẫn điện. Ngược lại các dung dịch axit, bazơ, \& muối đều dẫn điện.'' -- \cite[p. 4]{SGK_Hoa_Hoc_11_nang_cao}

\subsubsection{Nguyên nhân tính dẫn điện của các dung dịch axit, bazơ, \& muối trong nước}
``Ngay từ năm 1887, Arrhenius đã giả thiết \& sau này thực nghiệm đã xác nhận rằng, tính dẫn điện của các dung dịch axit, bazơ, \& muối là do trong dung dịch của chúng có các tiểu phân mang điện tích chuyển động tự do được gọi là các \textit{ion}. Như vậy các axit, bazơ, \& muối khi hòa tan trong nước phân ly ra các ion, nên dung dịch của chúng dẫn điện.

\begin{dinhnghia}[Sự điện ly, chất điện ly]
	Quá trình phân lý các chất trong nước ra ion là \emph{sự điện ly}. Những chất tan trong nước phân ly ra ion được gọi là \emph{những chất điện ly}\footnote{Nhiều chất khi nóng chảy cũng phân ly ra ion, nên ở trạng thái nóng chảy các chất này dẫn diện được.}.
\end{dinhnghia}
Vậy axit, bazơ, \& muối là những chất điện ly.'' -- \cite[p. 5]{SGK_Hoa_Hoc_11_nang_cao}

\subsection{Cơ chế của quá trình điện ly}

\subsubsection{Cấu tạo của phân tử $\rm H_2O$}
``Phân tử $\rm H_2O$ có cấu tạo như \cite[Hình 1.2: \textsf{Cấu tạo của phân tử nước. Mô hình đặc của phân tử nước}, p. 5]{SGK_Hoa_Hoc_11_nang_cao}. Liên kết $\rm O$ -- $\rm H$ là liên kết cộng hóa trị phân cực, cặp electron dùng chung lệch về phía oxi, nên ở oxi có dư điện tích âm, còn ở hiđro có dư điện tích dương. Vì vậy, phân tử $\rm H_2O$ là phân tử có cực.'' -- \cite[p. 5]{SGK_Hoa_Hoc_11_nang_cao}

\subsubsection{Quá trình điện ly của NaCl trong nước}
``NaCl là \textit{hợp chất ion}, i.e., gồm những cation $\rm Na^+$ \& anion $\rm Cl^-$  liên kết với nhau bằng lực tĩnh điện. Khi cho NaCl tinh thể vào nước, những ion $\rm Na^+$ \& $\rm Cl^-$ trên bề mặt tinh thể hút về chúng các phân tử $\rm H_2O$ (cation hút đầu âm \& anion hút đầu dương). Quá trình tương tác giữa các phân tử nước có cực \& các ion của muối kết hợp với sự chuyển động hỗn loạn không ngừng của các phân tử nước làm cho các ion $\rm Na^+$ \& $\rm Cl^-$ của muối tách dần khỏi tinh thể \& hòa tan trong nước (\cite[p. 1.3: \textsf{Sơ đồ quá trình phân ly ra ion của tinh thể NaCl trong nước}, p. 6]{SGK_Hoa_Hoc_11_nang_cao}). Từ sơ đồ trên ta thấy sự điện ly của NaCl trong nước có thể được biểu diễn bằng \textit{phương trình điện ly} như sau: $\rm NaCl\ (dd)\to Na^+\ (dd) + Cl^-\ (dd)$. Tuy nhiên, để đơn giản người ta thường viết: $\rm NaCl\to Na^+ + Cl^-$.'' -- \cite[p. 6]{SGK_Hoa_Hoc_11_nang_cao}

\texttt{Jujutsu Kaisen, Chap. 189.}

\subsubsection{Quá trình điện ly của HCl trong nước}
``Phân tử hiđro clorua (HCl) cũng là phân tử có cực tương tự phân tử nước. Cực dương ở phía hiddro, cực âm ở phía clo. Khi tan trong nước, các phần tử HCl hút về chúng những cực ngược dấu của các phân tử nước. Do sự tương tác giữa các phân tử nước \& phân tử HCl, kết hợp với sự chuyển động không ngừng của các phân tử nước dẫn đến sự điện ly phân tử HCl ra các ion $\rm H^+$ \& $Cl^-$ (\cite[Hình 1.4: \textsf{Sơ đồ quá  trình phân ly ra ion của phân tử HCl trong nước (Thực tế trong dung dịch $\rm H^+$ luôn tồn tại dưới dạng $\rm H_3O^+$)}, p. 6]{SGK_Hoa_Hoc_11_nang_cao}). Phương trình điện ly của HCl trong nước như sau: $\rm HCl\to H^+ + Cl^-$. Trong các phân tử ancol etylic, saccarozơ, glixerol, có liên kết phân cực nhưng rất yếu, nên dưới tác dụng của các phân tử nước chúng không thể phân ly ra ion được, chúng là các \textit{chất không điện ly}.'' -- \cite[pp. 6--7]{SGK_Hoa_Hoc_11_nang_cao}

%------------------------------------------------------------------------------%

\section{Phân Loại Các Chất Điện Ly}
\textbf{Nội dung.} \textit{Độ điện ly, cân bằng điện ly, chất điện ly mạnh \& chất điện ly yếu}.

\subsection{Độ điện ly}

\subsubsection{Thí nghiệm}
``Chuẩn bị 2 cốc: 1 cốc dung dịch HCl $0.10$ M, cốc kia đựng dung dịch $\rm CH_3COOH$ $0.10$ M rồi lắp vào bộ dụng cụ như \cite[ Hình 1.1, p. 4]{SGK_Hoa_Hoc_11_nang_cao}. Khi nối các đầu dây dẫn điện với cùng 1 nguồn điện, ta thấy bóng đèn ở cốc đựng dung dịch HCl sáng hơn so với bóng đèn ở cốc đựng dung dịch $\rm CH_3COOH$. Điều đó chứng tỏ rằng: nồng độ các ion trong dung dịch HCl lớn hơn nồng độ các ion trong dung dịch $\rm CH_3COOH$, i.e., số phần tử HCl phân ly ra ion nhiều hơn so với số phân tử $\rm CH_3COOH$ phân ly ra ion.'' -- \cite[p. 8]{SGK_Hoa_Hoc_11_nang_cao}

\subsubsection{Độ điện ly}
``Để đánh giá mức độ phân ly ra ion của chất điện ly trong dung dịch, người ta dùng khái niệm độ điện ly.

\begin{dinhnghia}[Độ điện ly]
	\emph{Độ điện ly} $\alpha$ của chất điện ly là tỷ số giữa số phân tử phân ly ra ion ($n$) \& tổng số phân tử hòa tan ($n_0$). $\alpha = \frac{n}{n_0}$.
\end{dinhnghia}
Độ điện ly của các chất điện ly khác nhau nằm trong khoảng $0 < \alpha\le 1$. Khi 1 chất có $\alpha = 0$, quá trình điện ly không xảy ra, đó là \textit{chất không điện ly}. Độ điện ly thường được biểu diễn dưới dạng phần trăm.'' -- \cite[p. 8]{SGK_Hoa_Hoc_11_nang_cao}

\subsection{Chất điện ly mạnh \& chất điện ly yếu}

\subsubsection{Chất điện ly mạnh}

\begin{dinhnghia}[Chất điện ly mạnh]
	``\emph{Chất điện ly mạnh} là chất khi tan trong nước\footnote{Tất cả các chất đều ít nhiều tan trong nước. E.g., ở $25^\circ$C nồng độ bão hòa của $\rm BaSO_4$ là $1.0\cdot 10^{-5}$ mol\texttt{/}l, của AgCl là $1.2\cdot 10^{-5}$ mol\texttt{/}, của $\rm CaCO_3$ là $6.9\cdot 10^{-5}$ mol\texttt{/}l, của $\rm Fe(OH)_2$ là $5.8\cdot 10^{-6}$ mol\texttt{/}l.}, các phân tử hòa tan đều phân ly ra ion.'' -- \cite[p. 9]{SGK_Hoa_Hoc_11_nang_cao}
\end{dinhnghia}
``Vậy chất điện ly mạnh có $\alpha = 1$. Đó là các axit mạnh, e.g., HCl, $\rm HNO_3$, $\rm HClO_4$, $\rm H_2SO_4$, $\ldots$; các bazơ mạnh, e.g., NaOH, KOH, $\rm Ba(OH)_2$, $\ldots$ \& hầu hết các muối. Trong phương trình điện ly của chất điện ly mạnh, người ta dùng 1 mũi tên chỉ chiều của quá trình điện ly. E.g., $\rm Na_2SO_4\to 2Na^+ + SO_4^{2-}$. Vì sự điện ly của $\rm Na_2SO_4$ là hoàn toàn, nên ta dễ dàng tính được nồng độ các ion do $\rm NaSO_4$ phân ly ra.'' -- \cite[p. 8]{SGK_Hoa_Hoc_11_nang_cao}. E.g., trong dung dịch $\rm NaSO_4$ $a$ M, nồng độ ion $\rm Na^+$ là $2a$ M \& nồng độ ion $\rm SO_4^{2-}$ là $a$ M.'' -- \cite[p. 9]{SGK_Hoa_Hoc_11_nang_cao}

\subsubsection{Chất điện ly yếu}

\begin{dinhnghia}[Chất điện ly yếu]
	``\emph{Chất điện ly yếu} là chất khi tan trong nước chỉ có 1 phần số phân tử hòa tan phân ly ra ion, phần còn lại vẫn tồn tại dưới dạng phân tử trong dung dịch.'' -- \cite[p. 9]{SGK_Hoa_Hoc_11_nang_cao}
\end{dinhnghia}
``Vậy độ điện ly của chất điện ly yếu nằm trong khoảng $\alpha\in(0,1)$. Những chất điện ly yếu là các axit yếu, e.g., $\rm CH_3COOH$, HClO, $\rm H_2S$, HF, $\rm H_2SO_3$, $\rm H_2CO_3$, $\ldots$; các bazơ yếu, như $\rm Bi(OH)_3$, $\rm Mg(OH)_2$, $\ldots$. Trong phương trình điện ly của chất điện ly yếu, người ta dùng 2 mũi tên ngược chiều nhau. E.g., $\rm CH_3COOH\rightleftarrows H^+ + CH_3COO^-$.

\paragraph{Cân bằng điện ly.} Sự điện ly của chất điện ly yếu là quá trình thuận nghịch, khi nào tốc độ phân ly \& tốc độ kết hợp cá ion tạo lại phân tử bằng nhua, cân bằng của quá trình điện ly được thiết lập. \textit{Cân bằng điện li} là \textit{cân bằng động}. Giống như mọi cân bằng hóa học khác, cân bằng điện ly cũng có hằng số cân bằng K \& tuần theo nguyên lý chuyển dịch cân bằng Lơ Sa-tơ-li-ê.

\paragraph{Ảnh hưởng của sự pha loãng đến độ điện ly.} \textit{Khi pha loãng dung dịch, độ điệ ly của các chất điện ly đều tăng}. E.g., ở $25^\circ$C độ điện ly của $\rm CH_3COOH$ trong dung dịch $0.10$ M là $1.32$\%, trong dung dịch $0.043$ M là 2\% \& trong dung dịch $0.010$ M là $4.11$\%. Có thể giải thích hiện tượng này như sau. Khi pha loãng dung dịch, các ion dương \& âm của chất điện ly dời xa nhau hơn, ít có điều kiện va chạm vào nhau để tạo lại phân tử, trong khi đó sự pha loãng không cản trở đến sự điện ly của các phân tử.'' -- \cite[pp. 9--10]{SGK_Hoa_Hoc_11_nang_cao}

%------------------------------------------------------------------------------%

\section{Axit, Bazơ, \& Muối}
\textbf{Nội dung.} \textit{Axit, bazơ theo thuyết Arrenius \& thuyết Br\o nsted; phương trình điện ly của các axit, bazơ, \& muối trong nước; hằng số phân ly axit, hằng số phân ly bazơ}.

\subsection{Axit \& bazơ theo thuyết Arrenius}

\subsubsection{Định nghĩa}

\begin{dinhnghia}[Axit]
	\emph{Axit} là chất khi tan trong nước phân ly ra cation $\rm H^+$.
\end{dinhnghia}
``E.g., $\rm HCl\to H^+ + Cl^-$, $\rm CH_3COOH\rightleftarrows H^+ + CH_3COO^-$. Các dung dịch axit đều có 1 số tính chất chung, đó là tính chất của các cation $\rm H^+$ trong dung dịch.'' -- \cite[p. 11]{SGK_Hoa_Hoc_11_nang_cao}

\begin{dinhnghia}[Bazơ]
	\emph{Bazơ} là chất khi tan trong nước phân ly ra anion $\rm OH^-$.
\end{dinhnghia}
``E.g., $\rm NaOH\to Na^+ + OH^-$. Các dung dịch bazơ đều có 1 số tính chất chung, đó là tính chất của các anion $\rm OH^-$ trong dung dịch.'' -- \cite[p. 11]{SGK_Hoa_Hoc_11_nang_cao}

\subsubsection{Axit nhiều nấc, bazơ nhiều nấc}

\paragraph{Axit nhiều nấc.} ``Phân tử HCl cũng như phân tử $\rm CH_3COOH$ trong dung dịch nước chỉ phân ly 1 nấc ra ion $\rm H^+$. Đó là các \textit{axit 1 nấc}. Đối với axit $\rm H_3PO_4$ thì: $\rm H_3PO_4\rightleftarrows H^+ + H_2PO_4^-$, $K_1 = 7.6\cdot 10^{-3}$, $\rm H_2PO_4^-\rightleftarrows H^+ + HPO_4^{2-}$, $K_2 = 6.2\cdot 10^{-8}$, $\rm HPO_4^{2-}\rightleftarrows H^+ + PO_4^{3-}$, $K_3 = 4.4\cdot 10^{-13}$. Phân tử $\rm H_3PO_4$ phân ly 3 nấc ra ion $\rm H^+$, $\rm H_3PO_4$ là \textit{axit 3 nấc}.

\begin{dinhnghia}[Axit nhiều nấc]
	Những axit khi tan trong nước mà phân tử phân ly nhiều nấc ra ion $H^+$ là các \emph{axit nhiều nấc}.'' -- \cite[p. 11]{SGK_Hoa_Hoc_11_nang_cao}
\end{dinhnghia}

\paragraph{Bazơ nhiều nấc.} ``Phân tử NaOH khi tan trong nước chỉ phân ly 1 nấc ra ion $\rm OH^-$, NaOH là \textit{bazơ 1 nấc}. Đối với $\rm Mg(OH)_2$ thì: $\rm Mg(OH)_2\rightleftarrows Mg(OH)^+ + OH^-$, $\rm Mg(OH)^+\rightleftarrows Mg^{2+} + OH^-$. Phân tử $\rm Mg(OH)_2$ phân ly 2 nấc ra ion $\rm OH^-$, $Mg(OH)_2$ là \textit{bazơ 2 nấc}.

\begin{dinhnghia}[Bazơ nhiều nấc]
	Những bazơ khi tan trong nước mà phân tử phân ly nhiều nấc ra ion $OH^-$ là các \emph{bazơ nhiều nấc}.'' -- \cite[p. 12]{SGK_Hoa_Hoc_11_nang_cao}
\end{dinhnghia}

\subsubsection{Hiđroxit lưỡng tính}

\begin{dinhnghia}[Hiđroxit lưỡng tính]
	``\emph{Hiđroxit lưỡng tính} là hiđroxit khi tan trong nước vừa có thể phân ly như axit, vừa có thể phân ly như bazơ.
\end{dinhnghia}
E.g., $\rm Zn(OH)_2$ là hiđroxit lưỡng tính: $\rm Zn(OH)_2\rightleftarrows Zn^{2+} + 2OH^-$: phân ly theo kiểu bazơ, $\rm Zn(OH)_2\rightleftarrows 2H^+ + ZnO_2^{2-}$\footnote{Thực tế trong dung dịch tồn tại ion $[\rm Zn(OH)_4]^{2-}$: $\rm Zn(OH)_2 + 2H_2O\rightleftarrows[Zn(OH)_4]^{2-} + 2H^+$.}: phân ly theo kiểu axit. Để thể hiện tính axit của $\rm Zn(OH)_2$ người ta thường viết nó dưới dạng $H_2ZnO_2$. 1 số hiđroxit lưỡng tính thường gặp là $\rm Al(OH)_3$, $\rm Zn(OH)_2$, $\rm Pb(OH)_2$, $\rm Sn(OH)_2$, $\rm Cr(OH)_3$, $\rm Cu(OH)_2$. Chúng đều tan ít trong nước \& lực axit, lực\footnote{Lực axit hay lực bazơ được đánh giá bằng hằng số phân ly K của axit hay bazơ đó.} bazơ đều yếu.'' -- \cite[p. 12]{SGK_Hoa_Hoc_11_nang_cao}

\subsection{Khái niệm về axit \& bazơ theo thuyết Br\o nsted}

\subsubsection{Định nghĩa}

\begin{dinhnghia}[Axit, bazơ]
	``\emph{Axit} là chất nhường proton $(\rm H^+)$. \emph{Bazơ} là chất nhận proton. Axit $\rightleftarrows$ Bazơ $+\rm H^+$.
\end{dinhnghia}

\begin{vidu}
	$\rm CH_3COOH + H_2O\rightleftarrows H_3O^+ + CH_3COO^-$. Trong phản ứng này, $\rm CH_3COOH$ nhường $\rm H^+$ cho $\rm H_2O$, $\rm CH_3COOH$ là axit; $\rm H_2O$ nhận $\rm H^+$, $\rm H_2O$ là bazơ. Theo phản ứng nghịch $\rm CH_3COO^-$ nhận $\rm H^+$, $\rm CH_3COO^-$ là bazơ, còn $\rm H_3O^+$ (ion oxoni) nhường $\rm H^+$, $\rm H_3O^+$ là axit.
\end{vidu}

\begin{vidu}
	$\rm NH_3 + H_2O\rightleftarrows NH_4^+ + OH^-$, $\rm NH_3$ là bazơ, $\rm H_2O$ là axit. Theo phản ứng nghịch $\rm NH_4^+$ là axit \& $\rm OH^-$ là bazơ.
\end{vidu}

\begin{vidu}
	$\rm HCO_3^- + H_2O\rightleftarrows H_3O^+ + CO_3^{2-}$. $\rm HCO_3^-$ \& $H_3O^+$ là axit, $\rm H_2O$ \& $\rm CO_3^{2-}$ là bazơ. $\rm HCO_3^- + H_2O\rightleftarrows H_2CO_3 + OH^-$. $\rm HCO_3^-$ \& $\rm OH^-$ là bazơ, $\rm H_2O$ \& $\rm H_2CO_3$ là axit. Ion $\rm HCO_3^-$ vừa có thể nhường proton vừa có thể nhận proton, vậy $HCO_3^-$ là chất lưỡng tính.
\end{vidu}

\begin{nhanxet}
	\begin{enumerate*}
		\item[$\bullet$] Phân tử $\rm H_2O$ có thể đóng vai trò axit hay bazơ. Vậy $\rm H_2O$ là chất lưỡng tính.
		\item[$\bullet$] Theo thuyết Br\o nsted, axit \& bazơ có thể là phân tử hoặc ion.'' -- \cite[pp. 12--13]{SGK_Hoa_Hoc_11_nang_cao}
	\end{enumerate*}
\end{nhanxet}

\subsubsection{Ưu điểm của thuyết Br\o nsted}
``Theo thuyết Arrenius, trong phân tử axit phải có hiđro \& trong nước phân ly ra ion $\rm H^+$, trong phân tử bazơ phải có nhóm $\rm OH$ \& trong nước phân ly ra ion $\rm OH^-$. Vậy thuyết Arrenius chỉ đúng cho trường hợp dung môi là nước. Ngoài ra, có những chất không chứa nhóm $\rm OH$, nhưng là bazơ, e.g., $\rm NH_3$, các amin thì thuyết Arrenius không giải thích được. Thuyết Br\o nsted tổng quát hơn, nó áp dụng đúng cho bất kỳ dung môi nào có khả năng nhường \& nhận proton, cả khi vắng mặt dung môi. Tuy nhiên, ở đây chúng ta chỉ nghiên cứu tính chất axit -- bazơ trong dung môi nước, nên cả 2 thuyết đều cho kết quả giống nhau.'' -- \cite[p. 13]{SGK_Hoa_Hoc_11_nang_cao}

\subsection{Hằng số phân ly axit \& bazơ}

\subsubsection{Hằng số phân ly axit}
``Sự điện ly của axit yếu trong nước là quá trình thuận nghịch, ở trạng thái cân bằng có thể áp dụng biểu thức hằng số cân bằng cho nó. E.g.,
\begin{align}
	\label{SGK Hoa 11 (1) p. 13}
	\rm CH_3COOH\rightleftarrows H^+ + CH_3COO^-,\ K_a = \frac{[H^+][CH_3COO^-]}{[CH_3COOH]}.
\end{align}
Trong đó: $\rm [H^+],[CH_3COO^-]$, \& $\rm [CH_3COOH]$ là nông độ mol của $\rm H^+$, $\rm CH_3COO^-$, \& $\rm CH_3COOH$ lúc cân bằng.

Cân bằng trong dung dịch $\rm CH_3COOH$ có thể viết:
\begin{align}
	\label{SGK Hoa 11 (2) p. 14}
	\rm CH_3COOH + H_2O\rightleftarrows H_3O^+ + CH_3COO^-,\ K_a = \frac{[H_3O^+][CH_3COO^-]}{[CH_3COOH]}.
\end{align}
$\rm H_2O$ trong cân bằng \eqref{SGK Hoa 11 (2) p. 14} là dung môi, trong dung dịch loãng nồng độ của $\rm H_2O$ được coi là hằng số, nên không có mặt trong biểu thức tính K. Phương trình \eqref{SGK Hoa 11 (1) p. 13} được viết theo thuyết Arrenius, phương trình \eqref{SGK Hoa 11 (2) p. 14} được viết theo thuyết Br\o nsted. 2 cách viết này cho kết quả giống nhau, i.e., giá trị $\rm K_a$ như nhau, vì nồng độ $\rm H^+$ hay nồng độ $\rm H_3O^+$ trong dung dịch chỉ là 1.

\begin{menhde}
	$\rm K_a$ là \emph{hằng số phân ly axit}. Giá trị $\rm K_a$ chỉ phụ thuộc vào bản chất axit \& nhiệt độ. Giá trị $\rm K_a$ của axit càng nhỏ, lực axit của nó càng yếu.
\end{menhde}

\begin{vidu}
	Ở $25^\circ$, $\rm K_a$ của $\rm CH_3COOH$ là $1.75\cdot 10^{-5}$ \& của $\rm HClO$ là $5.0\cdot 10^{-8}$. Vậy lực axit của $\rm HClO$ yếu hơn của $\rm CH_3COOH$, i.e., nếu 2 axit có cùng nồng độ mol \& ở cùng nhiệt độ thì nồng độ mol của $\rm H^+$ trong dung dịch $\rm HClO$ nhỏ hơn.'' -- \cite[pp. 13--14]{SGK_Hoa_Hoc_11_nang_cao}
\end{vidu}

\subsubsection{Hằng số phân ly bazơ}

\begin{vidu}
	``$\rm NH_3$ \& $\rm CH_3COO^-$ ở trong nước đều là các bazơ yếu:
	\begin{align*}
		\rm NH_3 + H_2O&\rightleftarrows\rm NH_4^+ + OH^-,\ K_b = \frac{[NH_4^+][OH^-]}{[NH_3]},\\
		\rm CH_3COO^- + H_2O&\rightleftarrows\rm CH_3COOH + OH^-,\ K_b = \frac{[CH_3COOH][OH^-]}{[CH_3COO^-]}.
	\end{align*}
	trong đó $\rm[NH_4^+],[OH^-],[NH_3],[CH_3COOH]$, \& $\rm[CH_3COO^-]$ là nồng độ mol của $\rm NH_4^+,OH^-,NH_3,CH_3COOH$, \& $\rm CH_3COO^-$ lúc cân bằng.
\end{vidu}

\begin{menhde}
	$\rm K_b$ là \emph{hằng số phân ly bazơ}. Giá trị $\rm K_b$ chỉ phụ thuộc vào bản chất bazơ \& nhiệt độ. Giá trị $\rm K_b$ của bazơ càng nhỏ, lực bazơ của nó càng yếu.'' -- \cite[p. 14]{SGK_Hoa_Hoc_11_nang_cao}
\end{menhde}

\subsection{Muối}

\subsubsection{Định nghĩa}

\begin{dinhnghia}[Muối]
	``\emph{Muối} là hợp chất, khi tan trong nước phân ly ra cation kim loại (hoặc cation $\rm NH_4^+$) \& anion gốc axit.
\end{dinhnghia}
E.g., $\rm (NH_4)_2SO_4\to 2NH_4^+ + SO_4^{2-}$, $\rm NaHCO_3\to Na^+ + HCO_3^-$. Muối mà anion gốc axit không còn hiđro có khả năng phân ly ra ion $\rm H^+$ (hiđro có tính axit)\footnote{Trong gốc axit của 1 số muối (e.g., $\rm Na_2HPO_3, NaH_2PO_2$) vẫn còn hiđro, nhưng là muối trung hòa vì các hiđro đó không có khả năng phân ly ra ion $\rm H^+$.} được gọi là \textit{muối trung hòa}. E.g., $\rm NaCl,(NH_4)_2SO_4,Na_2CO_3$. Nếu anion gốc axit của muối vẫn còn hiđro có khả năng phân ly ra ion $\rm H^+$, thì muối đó được gọi là \textit{muối axit}. E.g., $\rm NaHCO_3,NaH_2PO_4,NaHSO_4$. Ngoài ra có 1 số muối phức tạp thường gặp như muối kép: $\rm NaCl.KCl;KCl.MgCl_2.6H_2O;\ldots$ phức chất: $\rm[Ag(NH_3)_2]Cl;[Cu(NH_3)_4]SO_4;\ldots$'' -- \cite[pp. 14--15]{SGK_Hoa_Hoc_11_nang_cao}

\subsubsection{Sự điện ly của muối trong nước}
``Hầu hết các muối (kể cả muối kép) khi tan trong nước phân ly hoàn toàn ra cation kim loại (hoặc cation $\rm NH_4^+$) \& anion gốc axit (trừ 1 số muối như $\rm HgCl_2,Hg(CN)_2,\ldots$ là các chất điện ly yếu). E.g., $\rm K_2SO_4\to 2K^+ + SO_4^{2-}$, $\rm NaCl.KCL\to Na^+ + K^+ + 2Cl^-$, $\rm NaHSO_3\to Na^+ + HSO_3^-$. Nếu anion gốc axit còn chứa hiđro có tính axit, thì gốc này tiếp tục phân ly yếu ra ion $\rm H^+$. E.g., $\rm HSO_3^-\rightleftarrows H^+ + SO_3^{2-}$. Phức chất khi tan trong nước phân ly hoàn toàn ra ion phức (được ghi trong dấu móc vuông), sau đó ion phức phân ly yếu ra các cấu tử thành phần. E.g., $\rm[Ag(NH_3)_2]Cl\to[Ag(NH_3)_2]^+ + Cl^-$, $\rm[Ag(NH_3)_2]^+\rightleftarrows Ag^+ + 2NH_3$.'' -- \cite[p. 15]{SGK_Hoa_Hoc_11_nang_cao}

%------------------------------------------------------------------------------%

\section{Sự Điện Ly của Nước. pH. Chất Chỉ Thị Axit--Bazơ}
\textbf{Nội dung.} \textit{Tích số ion của nước, cách đánh giá độ axit \& độ kiềm của các dung dịch theo nồng độ ion $\rm H^+$ \& $\rm pH$, màu của 1 số chất chỉ thị trong dung dịch ở các khoảng pH khác nhau}.

\subsection{Nước là chất điện ly rất yếu}

\subsubsection{Sự điện ly của nước}
``Bằng dụng cụ đo nhạy, người ta thấy nước cũng dẫn điện nhưng cực kỳ yếu. Nước là chất điện ly rất yếu $\rm H_2O\rightleftarrows H^+ + OH^-$.'' -- \cite[p. 17]{SGK_Hoa_Hoc_11_nang_cao}

\subsubsection{Tích số ion của nước}
``Từ phương trình trên ta có thể viết được biểu thức hằng số cân bằng $K$ của phản ứng $\rm K = \frac{[H^+][OH^-]}{[H_2O]}$. Thực nghiệm đã xác định được rằng, ở nhiệt độ thường cứ $555$ triệu phân tử nước chỉ có 1 phân tử phân ly ra ion, nên $\rm[H_2O]$ được coi là hằng số. Từ đó, đặt $\rm K_{H_2O} = K[H_2O] = [H^+][OH^-]$. $\rm K_{H_2O}$ được gọi là \textit{tích số ion của nước, tích số này là hằng số ở nhiệt độ xác định}. Ở $25^\circ$C: $\rm K_{H_2O} = [H^+][OH^-] = 1.0\cdot 10^{-14}$, tuy nhiên giá trị này còn được dùng ở nhiệt độ không khác nhiều với $25^\circ$C. \textit{1 cách gần đúng, có thể coi giá trị tích số ion của nước là hằng số cả trong dung dịch loãng của các chất khác nhau}. Vì 1 phân tử $\rm H_2O$ phân ly ra 1 ion $\rm H^+$ \& 1 ion $\rm OH^-$, nên trong nước: $\rm[H^+] = [OH^-] = \sqrt{1.0\cdot 10^{-14}} = 1.0\cdot 10^{-7}$ M. Nước có môi trường trung tính, nên có thể định nghĩa:

\begin{dinhnghia}[Môi trường trung tính]
	\emph{Môi trường trung tính} là môi trường trong đó $\rm[H^+] = [OH^-] = 1.0\cdot 10^{-7}$ M.'' -- \cite[p. 17]{SGK_Hoa_Hoc_11_nang_cao}
\end{dinhnghia}

\subsubsection{Ý nghĩa tích số ion của nước}

\paragraph{Môi trường axit.} ``Khi hòa tan axit vào nước, nồng độ $\rm H^+$ tăng, nên nồng độ $\rm OH^-$ phải giảm sao cho tích số ion của nước không đổi. E.g., hòa tan axit vào nước để nồng độ $\rm H^+$ bằng $1.0\cdot 10^{-3}$ M thì nồng độ $\rm OH^-$ là: $\rm[OH^-] = \frac{1.0\cdot 10^{-14}}{[H^+]} = \frac{1.0\cdot 10^{-14}}{1.0\cdot 10^{-3}} = 1.0\cdot 10^{-11}$ N. Vậy \textit{môi trường axit là môi trường trong đó $\rm[H^+] > [OH^-]$ hay $\rm[H^+] > 1.0\cdot 10^{-7}$ M}.'' -- \cite[p. 18]{SGK_Hoa_Hoc_11_nang_cao}

\paragraph{Môi trường kiềm.} ``Khi hòa tan bazơ vào nước, nồng độ $\rm OH^-$ tăng, nên nồng độ $H^+$ phải giảm sao cho tích số ion của nước không đổi. E.g., hòa tan bazơ vào nước để nồng độ $\rm OH^-$ bằng $1.0\cdot 10^{-5}$ M thì nồng độ $\rm H^+$ là: $\rm[H^+] = \frac{1.0\cdot 10^{-14}}{[OH^-]} = \frac{1.0\cdot 10^{-14}}{1.0\cdot 10^{-5}} = 1.0\cdot 10^{-9}$ M.

\begin{dinhnghia}
	\emph{Môi trường kiềm} là môi trường trong đó: $\rm[H^+] < [OH^-]$ hay $\rm[H^+] < 1.0\cdot 10^{-7}$ M.
\end{dinhnghia}
Những thí dụ trên cho thấy, nếu biết nồng độ $\rm H^+$ trong dung dịch nước, thì nồng độ $\rm OH^-$ cũng được xác định \& ngược lại. Vì vậy, độ axit \& độ kiềm của dung dịch có thể được đánh giá chỉ bằng nồng độ $\rm H^+$. Môi trường trung tính: $\rm[H^+] = 1.0\cdot 10^{-7}$ M. Môi trường axit: $\rm[H^+] > 1.0\cdot 1-^{-7}$ M. Môi trường kiềm: $\rm[H^+] < 1.0\cdot 10^{-7}$ M.'' -- \cite[p. 18]{SGK_Hoa_Hoc_11_nang_cao}

\subsection{Khái niệm về pH. Chất chỉ thị Axit -- Bazơ}

\subsubsection{Khái niệm về pH}
``Dựa vào nồng độ $\rm H^+$ trong dung dịch nước có thể đánh giá được độ axit \& độ kiềm của dung dịch. Nhưng dung dịch thường dùng có nồng độ $\rm H^+$ nhỏ, để tránh ghi nồng độ $\rm H^+$ với số mũ âm, người ta dùng pH với quy ước như sau: $\rm[H^+] = 1.0\cdot 10^{-pH}$ M.\footnote{Về mặt toán học $\rm pH = -\lg[H^+]$.} Nếu $[{\rm H}^+] = 1.0\cdot 10^{-a}$ M thì ${\rm pH} = a$. E.g., $\rm[H^+] = 1.0\cdot 10^{-1}$ M $\Rightarrow\rm pH = 1.00$: môi trường axit. $\rm[H^+] = 1.0\cdot 10^{-7}$ M $\Rightarrow\rm pH = 7.00$: môi trường trung tính. $\rm[H^+] = 1.0\cdot 10^{-11}$ M $\Rightarrow\rm pH = 11.00$: môi trường kiềm. Thang pH thường dùng có giá trị từ 1--14. Giá trị pH có ý nghĩa to lớn trong thực tế. E.g., pH của máu người \& động vật có giá trị gần như không đổi. Thực vật có thể sinh trưởng bình thường chỉ khi giá trị pH của dung dịch trong đất ở trong khoảng xác định đặc trưng cho mỗi loại cây. Tốc độ ăn mòn kim loại trong nước tự nhiên phụ thuộc rất nhiều vào pH của nước mà kim loại tiếp xúc.'' -- \cite[pp. 18--19]{SGK_Hoa_Hoc_11_nang_cao}

\subsubsection{Chất chỉ thị axit -- bazơ}
``Chất chỉ thị axit -- bazơ là chất có màu biến đổi phụ thuộc vào giá trị pH của dung dịch. E.g., màu của 2 chất chỉ thị axit -- bazơ là quỳ \& phenolphtalein trong các khoảng pH khác nhau được đưa ra trong \cite[Bảng 1.1, p. 19]{SGK_Hoa_Hoc_11_nang_cao}. \textit{Quỳ}: đỏ $\rm pH\le 6$, tím $\rm pH = 7.0$, xanh $\rm pH\ge 8$. \textit{Phenolphtalein}: không màu $\rm pH < 8.3$, hồng $\rm pH\ge 8.3$.\footnote{Trong dung dịch xút đặc màu hồng bị mất.} Trộn lẫn 1 số chất chỉ thị có màu biến đổi kế tiếp nhau theo giá trị pH, ta được hỗn hợp chất \textit{chỉ thị vạn năng}. Dùng băng giấy tẩm dung dịch hỗn hợp này có thể xác định được gần đúng gía trị pH của dung dịch (\cite[Hình 1.5: \textsf{Màu của chất chỉ thị vạn năng (thuốc thử MERCK của Đức) ở các giá trị pH khác nhau}, p. 19]{SGK_Hoa_Hoc_11_nang_cao}). Để xác định tương đối chính xác giá trị pH của dung dịch người ta dùng máy đo pH.'' -- \cite[p. 19]{SGK_Hoa_Hoc_11_nang_cao}

\subsection{pH \& sự sâu răng}
``Răng được bảo vệ bởi lớp men cứng, dày khoảng 2 mm. Lớp men này là hợp chất $\rm Ca(PO_4)_3OH$ \& được tạo thành bằng phản ứng:
\begin{align}
	\label{SGK Hoa 11 (1) p. 21}
	\rm 5Ca^{2+} + 3PO_4^{3-} + OH^-\rightleftarrows Ca_5(PO_4)_3OH.
\end{align}
Quá trình tạo men này là sự bảo vệ tự nhiên của con người chống lại bệnh sâu răng. Sau bữa ăn, vi khuẩn trong miệng tấn công các thức ăn còn lưu lại trên răng tạo thành các axit hữu cơ như axit axetic, axit lactic. Thức ăn với hàm lượng đường cao tạo điều kiện tốt cho việc sản sinh ra các axit đó. Lượng axit trong miệng tăng, pH giảm, làm cho phản ứng sau xảy ra: $\rm H^+ + OH^-\to H_2O$. Khi nồng độ $\rm OH^-$ giảm, theo nguyên lý Sa-tơ-li-ê, cân bằng \eqref{SGK Hoa 11 (1) p. 21} chuyển dịch theo chiều nghịch \& men răng bị mòn, tạo điều kiện cho sâu răng phát triển. Biện pháp tốt nhất phòng sâu răng là ăn thức ăn ít chua, ít đường, đánh răng sau khi ăn. Người ta thường trộn vào thuốc đánh răng Nà hay $\rm SnF_2$, vì ion $\rm F^-$ tạo điều kiện cho phản ứng sau xảy ra: $\rm 5Ca^{2+} + 3PO_4^{3-} + F^-\to Ca_5(PO_4)_3F$. Hợp chất $Ca_5(PO_4)_3F$ là men răng thay thế 1 phần $\rm Ca_5(PO_4)_3OH$. Ở nước ta, 1 số người có thói quen ăn trầu, việc này rất tốt cho việc tạo men răng theo phản ứng \eqref{SGK Hoa 11 (1) p. 21}, vì trong trầu có vôi tôi ()$\rm Ca(OH)_2$), chứa các ion $\rm Ca^{2+}$ \& $\rm OH^-$ làm cho cân bằng \eqref{SGK Hoa 11 (1) p. 21} chuyển dịch theo chiều thuận.'' -- \cite[p. 21]{SGK_Hoa_Hoc_11_nang_cao}

%------------------------------------------------------------------------------%

\section{Luyện Tập: Axit, Bazơ, \& Muối}
\textbf{Nội dung.} \textit{Củng cố kiến thức về axit, bazơ, \& muối, kỹ năng tính pH của các dung dịch axit 1 nấc \& bazơ 1 nấc}.

\subsection{Kiến thức cần nắm vững}
``\begin{enumerate*}
	\item[\textbf{1.}] Axit khi tan trong nước phân ly ra cation $\rm H^+$ (theo thuyết Arrenius) hoặc axit là chất nhường proton $\rm H^+$ (theo thuyết Br\o nsted). Bazơ khi tan trong nước phân ly ra anion $\rm OH^-$ (theo thuyết Arrenius) hoặc bazơ là chất nhận proton $\rm H^+$ (theo thuyết Br\o nsted).
	\item[\textbf{2.}] Chất lưỡng tính vừa có thể thể hiện tính axit, vừa có thể thể hiện tính bazơ.
	\item[\textbf{3.}] Hầu hết các muối khi tan trong nước phân ly hoàn toàn ra cation kim loại (hoặc cation $\rm NH_4^+$) \& anion gốc axit. Nếu gốc axit còn chứa hiđro có tính axit, thì gốc đó tiếp tục phân ly yếu ra cation $\rm H^+$ \& anion gốc axit.
	\item[\textbf{4.}] Hằng số phân ly axit $\rm K_a$ \& hằng số phân ly bazơ $\rm K_b$ là các đại lượng đặc trưng cho lực axit \& lực bazơ của axit yếu \& bazơ yếu trong nước. 
	\item[\textbf{5.}] Tích số ion của nước là $\rm K_{H_2O} = [H^+][OH^-] = 1.0\cdot 10^{-14}$. 1 cách gần đúng có thể coi giá trị của tích số này là hằng số cả trong dung dịch loãng của các chất khác nhau.
	\item[\textbf{6.}] Giá trị $\rm[H^+]$ \& $\rm pH$ đặc trưng cho các môi trường: Môi trường trung tính: $\rm[H^+] = 1.0\cdot 10^{-7}$ M hay $\rm pH = 7.00$. Môi trường axit: $\rm[H^+] > 1.0\cdot 10^{-7}$ M hay $\rm pH < 7.00$. Môi trường kiềm: $\rm[H^+] < 1.0\cdot 10^{-7}$ M hay $\rm pH > 7.00$.
	\item[\textbf{7.}] Màu của quỳ, phenolphtalein \& chất chỉ thị vạn năng trong dung dịch ở các giá trị pH khác nhau.'' -- \cite[p. 22]{SGK_Hoa_Hoc_11_nang_cao}
\end{enumerate*}

%------------------------------------------------------------------------------%

\section{Phản Ứng Trao Đổi Ion Trong Dung Dịch Các Chất Điện Ly}
\textbf{Nội dung.} \textit{Bản chất \& điều kiện xảy ra phản ứng trao đổi ion trong dung dịch các chất điện ly, phương trình ion rút gọn của phản ứng trong dung dịch các chất điện ly}.

\subsection{Điều kiện xảy ra phản ứng trao đổi ion trong dung dịch các chất điện ly}

\subsubsection{Phản ứng tạo thành chất kết tủa}
``\textit{Thí nghiệm}: Nhỏ dung dịch natri sunfat ($\rm Na_2SO_4$) vào ống nghiệm đựng dung dịch bari clorua ($\rm BaCl_2$) thấy kết tủa trắng của $\rm BaSO_4$ xuất hiện:
\begin{align}
	\label{SGK Hoa 11 (1) p. 24}
	\rm NaSO_4 + BaCl_2\to 2NaCl + BaSO_4\downarrow.
\end{align}
\textit{Giải thích}: $\rm Na_2SO_4$ \& $\rm BaCl_2$ đều dễ tan \& phân ly mạnh trong nước: $\rm Na_2SO_4\to 2Na^+ + SO_4^{2-}$, $\rm BaCl_2\to Ba^{2+} + 2Cl^-$. Trong số 4 ion được phân ly ra chỉ có các ion $\rm Ba^{2+}$ \& $\rm SO_4^{2-}$ kết hợp được với nhau tạo thành chất kết tủa là $\rm BaSO_4$ (\cite[Hình 1.6: \textsf{Chất kết tủa $\rm BaSO_4$}, p. 24]{SGK_Hoa_Hoc_11_nang_cao}), nên thực chất phản ứng trong dung dịch là: $\rm Ba^{2+} + SO_4^{2-}\to BaSO_4\downarrow$. Phương trình này được gọi là \textit{phương trình ion rút gọn} của phản ứng \eqref{SGK Hoa 11 (1) p. 24}. \textit{Phương trình ion rút gọn cho biết bản chất của phản ứng trong dung dịch các chất điện ly}. Cách chuyển phương trình hóa học dưới dạng phân tử thành phương trình ion rút gọn như sau:
\begin{itemize}
	\item Chuyển tất cả các chất vừa dễ tan vừa điện ly mạnh thành ion, các chất khí, kết tủa, điện ly yếu để nguyên dạng phân tử. Phương trình thu được là phương trình ion đầy đủ, e.g., đối với phản ứng \eqref{SGK Hoa 11 (1) p. 24} ta có: $\rm 2Na^+ + SO_4^{2-} + Ba_{2+} + 2Cl^-\to BaSO_4\downarrow + 2Na^+ + 2Cl^-$.
	\item Lược bỏ những ion không tham gia phản ứng, ta được phương trình ion rút gọn: $\rm Ba^{2+} + SO_4^{2-}\to BaSO_4\downarrow$. Từ phương trình này ta thấy rằng, muốn điều chế $\rm BaSO_4$ cần trộn 2 dung dịch, 1 dung dịch chứa ion $\rm Ba^{2+}$ \& dung dịch kia chứa ion $\rm SO_4^{2-}$.'' -- \cite[pp. 24--25]{SGK_Hoa_Hoc_11_nang_cao}
\end{itemize}

\subsubsection{Phản ứng tạo thành chất điện ly yếu}

\paragraph{Phản ưng tạo thành nước.} ``\textit{Thí nghiệm}: Chuẩn bị 1 cốc đựng 25 ml dung dịch NaOH $0.10$ M. Nhỏ vào đó vài giọt dung dịch phenolphtalein. Dung dịch có màu hồng (\cite[Hình 1.7: \textsf{Màu của phenolphtalein trong môi trường kiềm}, p. 25]{SGK_Hoa_Hoc_11_nang_cao}). Rót từ từ dung dịch HCl $0.10$ M vào cốc trên, vừa rót vừa khuấy cho đến khi mất màu. Phản ứng như sau: $\rm HCl + NaOH\to NaCl + H_2O$. \textit{Giải thích}: NaOH \& HCl đều dễ tan \& phân ly mạnh trong nước: $\rm NaOH\to Na^+ + OH^-$, $HCl\to H^+ + Cl^-$. Các ion $\rm OH^-$ trong dung dịch NaOH làm cho phenolphtalein chuyển sang màu hồng. Khi cho dung dịch HCl vào dung dịch NaOH, chỉ có các ion $\rm H^+$ của HCl phản ứng với các ion $\rm OH^-$ của NaOH tạo thành chất điện ly rất yếu là $\rm H_2O$. Phương trình ion rút gọn là: $\rm H^+ + OH^-\to H_2O$. Khi màu của dung dịch trong cốc mất, đó là lúc các ion $\rm H^+$ của HCl đã phản ứng hết với các ion $\rm OH^-$ của NaOH. Phản ứng giữa dung dịch axit \& hiđroxit có tính bazơ rất dễ xảy ra vì tạo thành chất điện ly rất yếu là $\rm H_2O$. E.g., $\rm Mg(OH)_2$ ít tan trong nước, những dễ dàng tan trong dung dịch axit mạnh: $\rm Mg(OH)_2\ (r) + 2H^+\to Mg^{2+} + 2H_2O$.'' -- \cite[p. 25]{SGK_Hoa_Hoc_11_nang_cao}

\paragraph{Phản ứng tạo thành axit yếu.} ``\textit{Thí nghiệm}: Nhỏ dung dịch HCl vào ống nghiệm đựng dung dịch $\rm CH_3COONa$, axit yếu $\rm CH_3COOH$ sẽ tạo thành. $\rm HCl + CH_3COONa\to CH_3COOH + NaCl$. \textit{Giải thích}: HCl \& $CH_3COONa$ là các chất dễ tan \& phân ly mạnh: $\rm HCl\to H^+ + Cl^-$, $\rm CH_3COONa\to Na^+ + CH_3COO^-$. Trong dung dịch, các ion $\rm H^+$ sẽ kết hợp với các ion $\rm CH_3COO^-$ tạo thành chất điện ly yếu là $\rm CH_3COOH$ (mùi giấm). Phương trình ion rút gọn: $\rm CH_3COO^- + H^+\to CH_3COOH$.'' -- \cite[pp. 25--26]{SGK_Hoa_Hoc_11_nang_cao}

\subsubsection{Phản ứng tạo thành chất khí}
``\textit{Thí nghiệm}: Rót dung dịch HCl vào cốc đựng dung dịch $\rm Na_2CO_3$ ta thấy có bọt khí thoát ra: $\rm 2HCl + Na_2CO_3\to 2NaCl + CO_2\uparrow + H_2O$. \textit{Giải thích}: HCl \& $\rm Na_2CO_3$ đều dễ tan \& phân ly mạnh: $\rm HCl\to H^+ + Cl^-$, $\rm Na_2CO_3\to 2Na^+ + CO_3^{2-}$. Các ion $\rm H^+$ \& $\rm CO_3^{2-}$ trong dung dịch kết hợp với nhau tạo thành axit yếu là $\rm H_2CO_3$. Axit này không bền bị phân hủy ra $\rm CO_2$ \& $\rm H_2O$: $\rm H^+ + CO_3^{2-}\to HCO_3^-$, $\rm H^+ + HCO_3^-\to H_2CO_3$, $\rm H_2CO_3\to CO_2\uparrow + H_2O$. Phương trình rút gọn: $2H^+ + CO_3^{2-}\to CO_2\uparrow + H_2O$. Phản ứng giữa muối cacbonat \& dung dịch axit rất dễ xảy ra, vì vừa tạo thành chất điện ly rất yếu là $\rm H_2O$, vừa tạo ra chất khí $\rm CO_2$ tách khỏi môi trường phản ứng. E.g., các muối cacbonat ít tan trong nước, nhưng dễ tan trong các dung dịch axit. E.g., đá vôi ($\rm CaCO_3$) rất dễ tan trong dung dịch HCl (\cite[Hình 1.8: \textsf{Phản ứng tạo thành chất khí $\rm CO_2$}, p. 26]{SGK_Hoa_Hoc_11_nang_cao}). $\rm CaCO_3\ (r) + 2HCl\to CO_2\uparrow + H_2O + CaCl_2$. Phương trình ion rút gọn: $\rm CaCO_3\ (r) + 2H^+\to Ca^{2+} + CO_2\uparrow + H_2O$.

\noindent\textbf{Kết luận.}
\begin{enumerate*}
	\item[(a)] Phản ứng xảy ra trong dung dịch các chất điện ly là phản ứng giữa các ion.
	\item[(b)] Phản ứng trao đổi ion trong dung dịch các chất điện ly chỉ xảy ra khi các ion kết hợp được với nhau tạo thành ít nhất 1 trong các chất sau:
	\begin{enumerate*}
		\item[$\bullet$] Chất kết tủa.
		\item[$\bullet$] Chất điện ly yếu.
		\item[$\bullet$] Chất khí.'' -- \cite[pp. 26--27]{SGK_Hoa_Hoc_11_nang_cao}
	\end{enumerate*}
\end{enumerate*}

\subsection{Phản ứng thủy phân của muối}

\subsubsection{Khái niệm sự thủy phân của muối}
``Nước nguyên chất có $\rm pH = 7.0$ nhưng nhiều muối khi tan trong nước làm cho pH biến đổi, điều đó chứng tỏ muối đã tham gia phản ứng trao đổi ion với nước làm cho nồng độ $\rm H^+$ trong nước biến đổi. \textit{Phản ứng trao đổi ion giữa muối \& nước là phản ứng thủy phân của muối}.'' -- \cite[p. 27]{SGK_Hoa_Hoc_11_nang_cao}

\subsubsection{Phản ứng thủy phân của muối}

\begin{vidu}
	``Khi xác định pH của dung dịch $\rm CH_3COONa$ trong nước, ta thấy $\rm pH > 7.0$. Điều này được giải thích như sau: $\rm CH_3COONa$ hòa tan trong nước phân ly ra ion theo phương trình $\rm CH_3COONa\to Na^+ + CH_3COO^-$. Anion $\rm CH_3COO^-$ phản ứng với $\rm H_2O$ tạo ra chất điện ly yếu $\rm CH_3COOH$. Phương trình ion rút gọn: $\rm CH_3COO^- + HOH\rightleftarrows CH_3COOH + OH^-$. Các anion $\rm OH^-$ được giải phóng, nên môi trường có $\rm pH > 7.0$. Sản phẩm phản ứng là axit ($\rm CH_3COOH$) \& bazơ ($\rm OH^-$), do đó có phản ứng ngược lại. Cation $\rm Na^+$ trong muối $\rm CH_3COONa$ là cation của bazơ mạnh ($\rm NaOH$), nên không phản ứng với nước.'' -- \cite[p. 27]{SGK_Hoa_Hoc_11_nang_cao}
\end{vidu}

\begin{vidu}
	``pH của dung dịch $\rm Fe(NO_3)_3$ nhỏ hơn $7.0$ vì cation $\rm Fe^{3+}$ được tạo ra do sự điện ly của $\rm Fe(NO_3)_3$ tác dụng với $\rm H_2O$ tạo thành chất điện ly yếu là $\rm Fe(OH)^{2+}$ \& giải phóng các ion $\rm H^+$: $\rm Fe^{3+} + HOH\rightleftarrows Fe(OH)^{2+} + H^+$. Nồng độ $\rm H^+$ tăng lên, nên dung dịch có $\rm pH < 7.0$. Phản ứng là thuận nghịch vì $\rm Fe(OH)^{2+}$ là bazơ, còn $\rm H^+$ là axit, nên có phản ứng ngược lại. Ion $\rm NO_3^-$ là gốc của axit mạnh ($\rm HNO_3$), nên không phản ứng với nước.'' -- \cite[p. 27]{SGK_Hoa_Hoc_11_nang_cao}
\end{vidu}

\begin{vidu}
	``Khi hòa tan $\rm(CH_3COO)_2Pb$ trong nước, cả 2 ion $\rm Pb^{2+}$ \& $\rm CH_3COO^-$ đều bị thủy phân. Môi trường là axit hay kiềm phụ thuộc vào độ thủy phân của 2 ion.'' -- \cite[p. 28]{SGK_Hoa_Hoc_11_nang_cao}
\end{vidu}

\begin{vidu}
	``Những muối axit như $\rm NaHCO_3$, $\rm KH_2PO_4$, $\rm K_2HPO_4$ khi hòa tan trong nước phân ly ra các anion $\rm HCO_3^-$, $\rm H_2PO_4^-$, $\rm HPO_4^{2-}$. Các ion này là lưỡng tính. Chúng cũng phản ứng với $\rm H_2O$, môi trường của dung dịch tùy thuộc vào bản chất của anion.'' -- \cite[p. 28]{SGK_Hoa_Hoc_11_nang_cao}
\end{vidu}
\noindent\textbf{Kết luận.}
\begin{enumerate*}
	\item[(a)] ``Khi muối trung hòa tạo bởi cation của bazơ mạnh \& anion gốc axit yếu tan trong nước thì gốc axit yếu bị thủy phân, môi trường của dung dịch là kiềm ($\rm pH > 7.0$). E.g., $\rm CH_3COONa,K_2S,Na_2CO_3$.
	\item[(b)] Khi muối trung hòa tạo bởi cation của bazơ yếu \& anion gốc axit mạnh, tan trong nước thì cation của bazơ yếu bị thủy phân làm cho dung dịch có tính axit ($\rm pH < 7.0$). E.g., $\rm Fe(NO_3)_3,NH_4Cl,ZnBr_2$.
	\item[(c)] Khi muối trung hòa tạo bởi cation của bazơ mạnh \& anion gốc axit mạnh tan trong nước các ion không bị thủy phân, môi trường của dung dịch vẫn trung tính ($\rm pH = 7.0$). E.g., $\rm NaCl,KNO_3,KI$.
	\item[(d)] Khi muối trung hòa tạo bởi cation của bazơ yếu \& anion gốc axit yếu tan trong nước cả cation \& anion đều bị thủy phân. Môi trường của dung dịch phụ thuộc vào độ thủy phân của 2 ion.'' -- \cite[p. 28]{SGK_Hoa_Hoc_11_nang_cao}
\end{enumerate*}

%------------------------------------------------------------------------------%

\section{Luyện Tập: Phản Ứng Trao Đổi Ion Trong Dung Dịch Các Chất Điện Ly}
\textbf{Nội dung.} \textit{Điều kiện xảy ra phản ứng trao đổi ion trong dung dịch các chất điện ly, phương trình ion rút gọn của các phản ứng}.

\subsection{Kiến thức cần nắm vững}
``\begin{enumerate*}
	\item[\textbf{1.}] Phản ứng trao đổi ion trong dung dịch các chất điện ly chỉ xảy ra khi các ion kết hợp được với nhau tạo thành ít nhất 1 trong các chất sau:
	\begin{enumerate*}
		\item[(a)] Chất kết tủa.
		\item[(b)] Chất điện ly yếu.
		\item[(c)] Chất khí.
	\end{enumerate*}
	\item[\textbf{2.}] Phản ứng thủy phân của muối là phản ứng trao đổi ion giữa muối \& nước. Chỉ những muối chứa gốc axit yếu \&\texttt{/}hoặc cation của bazơ yếu mới bị thủy phân.
	\item[\textbf{3.}] Phương trình ion rút gọn cho biết bản chất của phản ứng trong dung dịch các chất điện ly. Trong phương trình ion rút gọn của phản ứng, người ta lược bỏ những ion không tham gia phản ứng, còn những chất kết tủa, điện ly yếu, chất khí được giữ nguyên dưới dạng phân tử.'' -- \cite[p. 30]{SGK_Hoa_Hoc_11_nang_cao}
\end{enumerate*}

%------------------------------------------------------------------------------%

\chapter{Nhóm Nitơ}

\begin{quotation}
	\textbf{Nội dung.} \textit{Các nguyên tố của nhóm nitơ; vị trí, cấu tạo nguyên tử \& phân tử của chúng trong bảng tuần hoàn; các tính chất cơ bản của các đơn chất \& hợp chất của nitơ, photpho; điều chế nitơ, photpho \& 1 số họp chất quan trọng của chúng}.
\end{quotation}

\section{Khái Quát về Nhóm Nitơ}
\textbf{Nội dung.} \textit{Cấu tạo phân tử, các tính chất vật lý \& hóa học của nitơ, phương pháp điều chế nitơ trong phòng thí nghiệm, trong công nghiệp \& ứng dụng của nitơ}.

\subsection{Cấu tạo phân tử}
``Nguyên tử nitơ có cấu hình electron $\rm 1s^22s^22p^3$, phân lớp ngoài cùng có 3 electron độc thân. 2 nguyên tử nitơ liên kết với nhau bằng 3 liên kết cộng hóa trị không có cực, tạo thành phân tử $\rm N_2$.'' ``Công thức electron $:N\ \vdots\ \vdots\ N:$. Công thức cấu tạo: $N\equiv N$.'' -- \cite[p. 37]{SGK_Hoa_Hoc_11_nang_cao}

\subsection{Tính chất vật lý}
``Ở điều kiện thường, nitơ là chất khí không màu, không mùi, không vị, hơi nhẹ hơn không khí, hóa lỏng ở $-196^\circ$ C, hóa rắn ở $-210^\circ$ C. Khí nitơ tan rất ít trong nước (ở điều kiện thường, 1 lít nước hòa tan được $0.015$ lít khí nitơ). Nitơ không duy trì sự cháy \& sự hô hấp.'' -- \cite[p. 37]{SGK_Hoa_Hoc_11_nang_cao}

\subsection{Tính chất hóa học}
``Vì có liên kết 3 với năng lượng liên kết lớn ($\rm E_{N=N} = 946$ kJ\texttt{/}mol) nên \textit{phân tử nitơ rất bền}. Ở nhiệt độ thường, nitơ khá trơ về mặt hóa học nhưng ở nhiệt độ cao nitơ trở nên hoạt động hơn \& có thể tác dụng với nhiều chất. Nguyên tử nitơ có độ âm điện lớn chỉ nhỏ hơn độ âm điện của  flo, clo, \& oxi. Tùy thuộc vào chất phản ứng mà nitơ thể hiện \textit{tính oxi hóa} hay \textit{tính khử}. Tuy nhiên, tính oxi hóa vẫn trội hơn tính khử.'' -- \cite[p. 37]{SGK_Hoa_Hoc_11_nang_cao}

\subsubsection{Tính oxi hóa}

\paragraph{Tác dụng với hiđro.} ``Ở nhiệt độ cao (trên $400^\circ$ C), áp suất cao \& có chất xúc tác, nitơ tác dụng trực tiếp với hiđro tạo ra khí amoniac. Đây là phản ứng thuận nghịch \& tỏa nhiệt.
\begin{align*}
	\rm\overset{0}{N}_2 + 3H_2***
\end{align*}
\texttt{Stop to learn LaTeX package chemarrow ...}

%------------------------------------------------------------------------------%

\section{Nitơ}

%------------------------------------------------------------------------------%

\section{Amoniac \& Muối Amoni}

%------------------------------------------------------------------------------%

\section{Axit Nitric \& Muối Nitrat}

%------------------------------------------------------------------------------%

\section{Luyện Tập: Tính Chất của Nitơ \& Hợp Chất của Nitơ}

%------------------------------------------------------------------------------%

\section{Photpho}

%------------------------------------------------------------------------------%

\section{Axit Photphoric \& Muối Photphat}

%------------------------------------------------------------------------------%

\section{Phân Bón Hóa Học}

%------------------------------------------------------------------------------%

\section{Luyện Tập: Tính Chất của Photpho \& Các Hợp Chất của Photpho}

%------------------------------------------------------------------------------%

\section{Thực Hành: Tính Chất của 1 Số Hợp Chất Nitơ, Photpho. Phân Biệt 1 Số Loại Phân Bón Hóa Học}

%------------------------------------------------------------------------------%

\chapter{Nhóm Cacbon}

\section{Khái Quát về Nhóm Cacbon}

%------------------------------------------------------------------------------%

\section{Cacbon}

%------------------------------------------------------------------------------%

\section{Hợp Chất của Cacbon}

%------------------------------------------------------------------------------%

\section{Silic \& Hợp Chất của Silic}

%------------------------------------------------------------------------------%

\section{Công Nghiệp Silicat}

%------------------------------------------------------------------------------%

\section{Luyện Tập: Tính Chất của Cacbon, Silic, \& Hợp Chất của Chúng}

%------------------------------------------------------------------------------%

\chapter{Đại Cương về Hóa Học Hữu Cơ}

\section{Hóa Học Hữu Cơ \& Hợp Chất Hữu Cơ}

%------------------------------------------------------------------------------%

\section{Phân Loại \& Gọi Tên Hợp Chất Hữu Cơ}

%------------------------------------------------------------------------------%

\section{Phân Tích Nguyên Tố}

%------------------------------------------------------------------------------%

\section{Công Thức Phân Tử Hợp Chất Hữu Cơ}

%------------------------------------------------------------------------------%

\section{Luyện Tập: Chất Hữu Cơ, Công Thức Phân Tử}

%------------------------------------------------------------------------------%

\section{Cấu Trúc Phân Tử Hợp Chất Hữu Cơ}

%------------------------------------------------------------------------------%

\section{Phản Ứng Hữu Cơ}

%------------------------------------------------------------------------------%

\section{Luyện Tập: Cấu Trúc Phân Tử Hợp Chất Hữu Cơ}

%------------------------------------------------------------------------------%

\chapter{Hidrocacbon No}

\section{Ankan: Đồng Đẳng, Đồng Phân \& Danh Pháp}

%------------------------------------------------------------------------------%

\section{Ankan: Cấu Trúc Phân Tử \& Tính Chất Vật Lý}

%------------------------------------------------------------------------------%

\section{Ankan: Tính Chất Hóa Học, Điều Chế \& Ứng Dụng}

%------------------------------------------------------------------------------%

\section{Xicloankan}

%------------------------------------------------------------------------------%

\section{Luyện Tập: Ankan \& Xicloankan}

%------------------------------------------------------------------------------%

\section{Thực Hành: Phân Tích Định Tính: Điều Chế \& Tính Chất của Metan}

%------------------------------------------------------------------------------%

\chapter{Hidrocacbon Không No}

\section{Anken: Danh Pháp, Cấu Trúc, \& Đồng Phân}

%------------------------------------------------------------------------------%

\section{Anken: Tính Chất, Điều Chế, \& Ứng Dụng}

%------------------------------------------------------------------------------%

\section{Ankađien}

%------------------------------------------------------------------------------%

\section{Khái Niệm về Tecpen}

%------------------------------------------------------------------------------%

\section{Ankin}

%------------------------------------------------------------------------------%

\section{Luyện Tập: Hidrocacbon Không No}

%------------------------------------------------------------------------------%

\section{Thực Hành: Tính Chất của Hidrocacbon Không No}

%------------------------------------------------------------------------------%

\chapter{Hidrocacbon Thơm -- Nguồn Hidrocacbon Thiên Nhiên}

\section{Benzen \& Ankylbenzen}

%------------------------------------------------------------------------------%

\section{Stiren \& Naphtalen}

%------------------------------------------------------------------------------%

\section{Nguồn Hidrocacbon Thiên Nhiên}

%------------------------------------------------------------------------------%

\section{Luyện Tập: So Sánh Đặc Điểm Cấu Trúc \& Tính Chất của Hidrocacbon Thơm với Hidrocacbon No \& Không No}

%------------------------------------------------------------------------------%

\section{Thực Hành: Tính Chất của 1 Số Hidrocacbon Thơm}

%------------------------------------------------------------------------------%

\chapter{Dẫn Xuất Halogen. Ancol -- Phenol}

\section{Dẫn Xuất Halogen của Hidrocacbon}

%------------------------------------------------------------------------------%

\section{Luyện Tập: Dẫn Xuất Halogen}

%------------------------------------------------------------------------------%

\section{Ancol: Cấu Tạo, Danh Pháp, Tính Chất Vật Lý}

%------------------------------------------------------------------------------%

\section{Ancol: Tính Chất Hóa Học, Điều Chế, \& Ứng Dụng}

%------------------------------------------------------------------------------%

\section{Phenol}

%------------------------------------------------------------------------------%

\section{Luyện Tập: Ancol, Phenol}

%------------------------------------------------------------------------------%

\section{Thực Hành: Tính Chất của 1 Vài Dẫn Xuất Halogen, Ancol, \& Phenol}

%------------------------------------------------------------------------------%

\chapter{Anđehit -- Xeton -- Axit Cacboxylic}

\section{Anđehit \& Xeton}

%------------------------------------------------------------------------------%

\section{Luyện Tập: Anđehit \& Xeton}

%------------------------------------------------------------------------------%

\section{Axit Cacboxylic: Cấu Trúc, Danh Pháp, \& Tính Chất Vật Lý}

%------------------------------------------------------------------------------%

\section{Axit Cacboxylic: Tính Chất Hóa Học, Điều Chế, \& Ứng Dụng}

%------------------------------------------------------------------------------%

\section{Luyện Tập: Axit Cacboxylic}

%------------------------------------------------------------------------------%

\section{Thực Hành: Tính Chất của Anđehit \& Axit Cacboxylic}

%------------------------------------------------------------------------------%

\printbibliography[heading=bibintoc]
	
\end{document}