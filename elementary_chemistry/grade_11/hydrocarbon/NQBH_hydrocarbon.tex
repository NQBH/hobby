\documentclass{article}
\usepackage[backend=biber,natbib=true,style=authoryear,maxbibnames=50]{biblatex}
\addbibresource{/home/nqbh/reference/bib.bib}
\usepackage[utf8]{vietnam}
\usepackage{tocloft}
\renewcommand{\cftsecleader}{\cftdotfill{\cftdotsep}}
\usepackage[colorlinks=true,linkcolor=blue,urlcolor=red,citecolor=magenta]{hyperref}
\usepackage{amsmath,amssymb,amsthm,mathtools,float,graphicx,algpseudocode,algorithm,tcolorbox,booktabs,chemfig}
\usepackage[version=4]{mhchem}
\usepackage[inline]{enumitem}
\allowdisplaybreaks
\numberwithin{equation}{section}
\newtheorem{assumption}{Assumption}[section]
\newtheorem{conjecture}{Conjecture}[section]
\newtheorem{corollary}{Corollary}[section]
\newtheorem{dangtoan}{Dạng toán}[section]
\newtheorem{hequa}{Hệ quả}[section]
\newtheorem{definition}{Definition}[section]
\newtheorem{dinhnghia}{Định nghĩa}[section]
\newtheorem{example}{Example}[section]
\newtheorem{vidu}{Ví dụ}[section]
\newtheorem{lemma}{Lemma}[section]
\newtheorem{notation}{Notation}[section]
\newtheorem{principle}{Principle}[section]
\newtheorem{problem}{Problem}[section]
\newtheorem{baitoan}{Bài toán}[section]
\newtheorem{proposition}{Proposition}[section]
\newtheorem{question}{Question}[section]
\newtheorem{cauhoi}{Câu hỏi}[section]
\newtheorem{remark}{Remark}[section]
\newtheorem{luuy}{Lưu ý}[section]
\newtheorem{theorem}{Theorem}[section]
\newtheorem{dinhly}{Định lý}[section]
\usepackage[left=0.5in,right=0.5in,top=1.5cm,bottom=1.5cm]{geometry}
\usepackage{fancyhdr}
\pagestyle{fancy}
\fancyhf{}
\lhead{\small Sect.~\thesection}
\rhead{\small \nouppercase{\leftmark}}
\renewcommand{\sectionmark}[1]{\markboth{#1}{}}
\cfoot{\thepage}
\def\labelitemii{$\circ$}

\title{Hydrocarbon}
\author{Nguyễn Quản Bá Hồng\footnote{Independent Researcher, Ben Tre City, Vietnam\\e-mail: \texttt{nguyenquanbahong@gmail.com}; website: \url{https://nqbh.github.io}.}}
\date{\today}

\begin{document}
\maketitle
\begin{abstract}
	1 bộ sưu tập các bài tập chọn lọc từ cơ bản đến nâng cao cho Hóa học sơ cấp lớp 11. Tài liệu này là phần bài tập bổ sung cho tài liệu chính \href{https://github.com/NQBH/hobby/blob/master/elementary_chemistry/grade_11/NQBH_elementary_chemistry_grade_11.pdf}{GitHub\texttt{/}NQBH\texttt{/}hobby\texttt{/}elementary chemistry\texttt{/}grade 11\texttt{/}lecture}\footnote{\textsc{url}: \url{https://github.com/NQBH/hobby/blob/master/elementary_chemistry/grade_11/NQBH_elementary_chemistry_grade_11.pdf}.} của tác giả viết cho Toán lớp 11. Phiên bản mới nhất của tài liệu này được lưu trữ ở link sau: \href{https://github.com/NQBH/hobby/blob/master/elementary_chemistry/grade_11/problem/NQBH_elementary_chemistry_grade_11_problem.pdf}{GitHub\texttt{/}NQBH\texttt{/}hobby\texttt{/}elementary chemistry\texttt{/}grade 11\texttt{/}problem}\footnote{\textsc{url}: \url{https://github.com/NQBH/hobby/blob/master/elementary_chemistry/grade_11/problem/NQBH_elementary_chemistry_grade_11_problem.pdf}.}.
\end{abstract}
\tableofcontents
\newpage

%------------------------------------------------------------------------------%

\section{Hydrocarbon No}

\subsection{Ankan}

\begin{baitoan}[\cite{SGK_Hoa_Hoc_11_co_ban}, \textbf{1.}, p. 115]
	Thế nào là hydrocarbon no, ankan, xicloankan?
\end{baitoan}

\begin{baitoan}[\cite{SGK_Hoa_Hoc_11_co_ban}, \textbf{2.}, p. 115]
	Viết CTPT của các hydrocarbon tương ứng với các gốc ankyl sau: \emph{\ce{-CH3,-C3H7,-C6H_{13}}}.
\end{baitoan}

\begin{baitoan}[\cite{SGK_Hoa_Hoc_11_co_ban}, \textbf{3.}, p. 115]
	Viết PTHH của các phản ứng sau: (a) Propan tác dụng với clo (theo tỷ lệ mol $1:1$) khi chiếu sáng. (b) Tách 1 phân tử hydro từ phân tử propan. (c) Đốt cháy hexan.
\end{baitoan}

\begin{baitoan}[\cite{SGK_Hoa_Hoc_11_co_ban}, \textbf{4.}, p. 116]
	Các hydrocarbon no được dùng làm nhiên liệu là do nguyên nhân nào sau đây? {\sf A.} Hydrocarbon no có phản ứng thế. {\sf B.} Hydrocarbon no có nhiều trong tự nhiên. {\sf C.} Hydrocarbon no là chất nhẹ hơn nước. {\sf D.} Hydrocarbon no cháy tỏa nhiều nhiệt \& có nhiều trong tự nhiên. 
\end{baitoan}

\begin{baitoan}[\cite{SGK_Hoa_Hoc_11_co_ban}, \textbf{5.}, p. 116]
	Giải thích: (a) Tại sao không được để các bình chứa xăng, dầu (gồm các ankan) gần lửa, trong khi đó người ta có thể nấu chảy nhựa đường (trong thành phần cũng có các ankan) để làm đường giao thông. (b) Không dùng nước để dập các đám cháy xăng, dầu mà phải dùng cát hoặc bình chứa khí carbonic.
\end{baitoan}

\begin{baitoan}[\cite{SGK_Hoa_Hoc_11_co_ban}, \textbf{6.}, p. 116]
	CTCT sau ứng với tên gọi nào?
	\begin{center}
		\chemfig[atom sep=2em]{CH_3-CH(=[6]CH_3)-CH_2-CH_2-CH_3}
	\end{center}
	{\sf A.} neopentan. {\sf B.} $2$-metylpentan. {\sf C.} isobutan. {\sf D.} $1,1$-đimetylbutan.
\end{baitoan}

\begin{baitoan}[\cite{SGK_Hoa_Hoc_11_co_ban}, \textbf{7.}, p. 116]
	Khi đốt cháy hoàn toàn $3.6$\emph{g} ankan X thu được $5.6$\emph{l} khí \emph{\ce{CO2}} (đktc). CTPT của X? {\sf A.} \emph{\ce{C3H8}}. {\sf B.} \emph{\ce{C5H_{10}}}. {\sf C.} \emph{\ce{C5H_{12}}}. {\sf D.} \emph{\ce{C4H_{10}}}. 
\end{baitoan}

\begin{baitoan}[\cite{SBT_Hoa_Hoc_11_co_ban}, \textbf{5.1}, p. 35]
	Điền vào chỗ khuyết những từ thích hợp trong các từ \& cụm từ: \emph{ankan, xicloankan, hydrocarbon no, hydrocarbon không no, phản ứng thế}. Hydrocarbon mà phân tử chỉ có liên kết đơn được gọi là $\ldots$. Hydrocarbon no có mạch không vòng được gọi là $\ldots$. Hydrocarbon no có 1 mạch vòng được gọi là $\ldots$. Tính chất hóa học đặc trưng của hydrocarbon no là $\ldots$.
\end{baitoan}

\begin{proof}[Giải]
	Hydrocarbon mà phân tử chỉ có liên kết đơn được gọi là \textit{hydrocarbon no}. Hydrocarbon no có mạch không vòng được gọi là \textit{ankan}. Hydrocarbon no có 1 mạch vòng được gọi là \textit{xicloankan}. Tính chất hóa học đặc trưng của hydrocarbon no là \textit{phản ứng thế}.
\end{proof}

\begin{baitoan}[\cite{SBT_Hoa_Hoc_11_co_ban}, \textbf{5.2}, p. 35]
	Nhận xét nào sai? {\sf A.} Tất cả các ankan đều có công thức phân tử \emph{\ce{C_nH_{2n+2}}}. {\sf B.} Tất cả các chất có công thức phân tử \emph{\ce{C_nH_{2n+2}}} đều là ankan. {\sf C.} Tất cả các ankan đều chỉ có liên kết đơn trong phân tử. {\sf D.} Tất cả các chất chỉ có liên kết đơn trong phân tử đều là ankan.
\end{baitoan}

\begin{proof}[Giải]
	{\sf D.} sai. Vì có nhiều chất chỉ có liên kết đơn trong phân tử nhưng không là ankan, e.g., \ce{H2,CO2}.
\end{proof}

\begin{baitoan}[\cite{SBT_Hoa_Hoc_11_co_ban}, \textbf{5.3}, p. 35]
	Chất sau có tên là gì?
	\begin{center}
		\chemfig[atom sep=2em]{CH_3-CH_2-CH(-[6]C(-[4]H)(-[6]CH_3)-CH3)-CH_2-CH_3}
	\end{center}
	{\sf A.} $3$-isopropylpentan. {\sf B.} $2$-metyl-$3$-etylpentan. {\sf C.} $3$-etyl-$2$-metylpentan. {\sf D.} $3$-etyl$-4$-metylpentan.\hfill{\sf Ans: C.}
\end{baitoan}

\begin{baitoan}[\cite{SBT_Hoa_Hoc_11_co_ban}, \textbf{5.4}, p. 36]
	Tên nào đúng với công thức sau? {\sf A.} $3$-isopropyl-$5,5$-đimetylhexan. {\sf B.} $2,2$-đimetyl-$4$-isopropylhexan. {\sf C.} $3$-etyl-$2,5,5$-trimetylhexan. {\sf D.} $4$-etyl-$2,2,5$-trimetylhexan.\hfill{\sf Ans: D.}
	\begin{center}
		\chemfig[atom sep=2em]{CH_3-CH_2-CH(-[6]CH(-[4]CH_3)(-[6]CH_3))-CH_2-C(-[2]CH_3)(-[6]CH_3)-CH3}.
	\end{center}
\end{baitoan}

\begin{baitoan}[\cite{SBT_Hoa_Hoc_11_co_ban}, \textbf{5.5}, p. 36]
	Tổng số liên kết cộng hóa trị trong 1 phân tử \emph{\ce{C3H8}}? {\sf A.} $11$. {\sf B.} $10$. {\sf C.} $3$ {\sf D.} $8$.\hfill{\sf Ans: B.}
\end{baitoan}

\begin{baitoan}[Mở rộng \cite{SBT_Hoa_Hoc_11_co_ban}, \textbf{5.5}, p. 36]
	Tổng số liên kết cộng hóa trị trong 1 phân tử \emph{\ce{C_nH_{2n+2}}} là bao nhiêu?
\end{baitoan}

\begin{proof}[Giải]
	Mỗi H cho 1 liên kết, nên $2n + 2$ H cho $2n + 2$ liên kết. Sắp $n$ nguyên tử C thành 1 hàng, có $n - 1$ liên kết đơn giữa chúng. Nên có tất cả $2n + 2 + n - 1 = 3n + 1$ liên kết cộng hóa trị.
\end{proof}

\begin{baitoan}[\cite{SBT_Hoa_Hoc_11_co_ban}, \textbf{5.6}, p. 36]
	2 chất $2$-metylpropan \& butan khác nhau về: {\sf A.} công thức cấu tạo. {\sf B.} công thức phân tử. {\sf C.} số nguyên tử carbon. {\sf D.} số liên kết cộng hóa trị.\hfill{\sf Ans: A.}
\end{baitoan}

\begin{baitoan}[\cite{SBT_Hoa_Hoc_11_co_ban}, \textbf{5.7}, p. 36]
	Tất cả các ankan có cùng công thức gì? {\sf A.} Công thức đơn giản nhất. {\sf B.} Công thức chung. {\sf C.} Công thức cấu tạo. {\sf D.} Công thức phân tử.\hfill{\sf Ans: B.}
\end{baitoan}

\begin{baitoan}[\cite{SBT_Hoa_Hoc_11_co_ban}, \textbf{5.8}, p. 36]
	Trong các chất dưới đây, chất nào có nhiệt độ sôi thấp nhất? {\sf A.} Butan. {\sf B.} Etan. {\sf C.} Metan. {\sf D.} Propan.\hfill{\sf Ans: C.}
\end{baitoan}

\begin{baitoan}[\cite{SBT_Hoa_Hoc_11_co_ban}, \textbf{5.9}, p. 36]
	Gọi tên IUPAC của các ankan: (a) \emph{\ce{(CH3)2CH-CH2-C(CH3)3}} (tên thông dụng là \emph{isooctan}). (b) \emph{\ce{CH3-CH2-CH(CH3)-CH(CH3)-[CH2]4-CH(CH3)2}}.
\end{baitoan}

\begin{proof}[Giải]
	(a) $2,2,4$-trimetylpentan. (b) $3,4,9$-trimetylđecan.
\end{proof}

\begin{baitoan}[\cite{SBT_Hoa_Hoc_11_co_ban}, \textbf{5.10}, p. 36]
	Viết công thức cấu tạo thu gọn của: (a) $4$-etyl-$2,3,3$-trimetylheptan. (b) $3,5$-đietyl-$2,2,3$-trimetyloctan.
\end{baitoan}
\noindent\textit{Giải.}
\begin{center}
	\chemfig[atom sep=2em]{CH_3-CH(-[6]CH_3)-C(-[2]CH_3)(-[6]CH_3)-CH(-[6]CH_2(-[6]CH_3))-CH_2-CH_2-CH_3},\hspace{1cm}\chemfig[atom sep=2em]{CH_3-C(-[2]CH_3)(-[6]CH_3)-C(-[2]CH_3)(-[6]CH_2(-[6]CH_3))-CH_2-CH(-[6]CH_2(-[6]CH_3))-CH_2-CH_2-CH_3}
\end{center}

\begin{baitoan}[\cite{SBT_Hoa_Hoc_11_co_ban}, \textbf{5.11}, p. 37]
	Cho A là 1 ankan thể khí. Để đốt cháy hoàn toàn $1.2$\emph{l} A cần dùng vừa hết $6$\emph{l} oxi lấy ở cùng điều kiện. (a) Xác định CTPT chất A. (b) Cho chất A tác dụng với khí clo ở $25^\circ$ \& có ánh sáng. Hỏi có thể thu được mấy dẫn xuất monoclo của A? Cho biết tên của mỗi dẫn xuất đó. Dẫn xuất nào thu được nhiều hơn?
\end{baitoan}

\begin{proof}[Giải]
	Gọi $A$: \ce{C_nH_{2n+2}}. \ce{C_nH_{2n+2} + $\frac{3n+1}{2}$O2 -> $n$CO2 + $(n+1)$H2O}. Vì $A$ là chất khí, $\frac{n_{\ce{O2}}}{n_A} = \frac{V_{\ce{O2}}}{V_A}\Leftrightarrow\frac{3n + 1}{2} = \frac{6}{1.2} = 5\Rightarrow n = 3$. CTPT A: \ce{C3H8}. \ce{CH3-CH2-CH3 + Cl2 ->[as][$25^\circ$C] X or Y + HCl} trong đó X: 1-clopropan (43\%) \& Y: 2-clopropan (57\%)
	\begin{center}
		X: \ce{CH3-CH2-CH2-Cl},\hspace{1cm}Y: \chemfig[atom sep=2em]{CH_3-CH(-[6]Cl)-CH_3}
	\end{center}
\end{proof}

\begin{luuy}
	Vì đề bài chỉ cho ``lấy ở cùng điều kiện'' nên không thể tính số mol của A \& oxi bằng công thức $n = \frac{V}{22.4}$ (đktc) hoặc $n = \frac{V}{24}$ (điều kiện thường: $20^\circ{\rm C}$ \& $1$\emph{atm}) được.
\end{luuy}

\begin{baitoan}[\cite{SBT_Hoa_Hoc_11_co_ban}, \textbf{5.12}, p. 37]
	Để đốt cháy hoàn toàn $1.45$\emph{g} 1 ankan phải dùng vừa hết $3.64$\emph{l} \emph{\ce{O2}} (đktc). (a) Xác định CTPT của ankan đó. (b) Viết CTCT các đồng phân ứng với CTPT đó. Ghi tên tương ứng.
\end{baitoan}

\begin{baitoan}[\cite{SBT_Hoa_Hoc_11_co_ban}, \textbf{5.13}, p. 37]
	Khi đốt cháy hoàn toàn $1.8$\emph{g} 1 ankan, người ta thấy trong sản phẩm tạo thành khối lượng \emph{\ce{CO2}} nhiều hơn khối lượng \emph{\ce{H2O}} là $2.8$\emph{g}. (a) Xác định CTPT của ankan mang đốt. (b) Viết CTCT \& tên tất cả các đồng phân ứng với CTPT đó.
\end{baitoan}

\begin{baitoan}[\cite{SBT_Hoa_Hoc_11_co_ban}, \textbf{5.14}, p. 37]
	Đốt cháy hoàn toàn $2.86$\emph{g} hỗn hợp gồm hexan \& octan người ta thu được $4.48$\emph{l} khí \emph{\ce{CO2}} (đktc). (a) Xác định \% về khối lượng của từng chất trong hỗn hợp ankan mang đốt.
\end{baitoan}

\begin{baitoan}[\cite{SBT_Hoa_Hoc_11_co_ban}, \textbf{5.15}, p. 37]
	1 loại xăng là hỗn hợp của các ankan \& có CTPT là \emph{\ce{C7H_{16}}} \& \emph{\ce{C8H_{18}}}. Để đốt cháy hoàn toàn $6.950$\emph{g} xăng đó phải dùng vừa hết $17.08$\emph{l} khí \emph{\ce{O2}} (đktc). Xác định \% về khối lượng của từng chất trong loại xăng đó. 
\end{baitoan}

\begin{baitoan}[\cite{SBT_Hoa_Hoc_11_co_ban}, \textbf{5.16}, p. 37]
	Hỗn hợp M chứa 2 ankan kế tiếp nhau trong dãy đồng đẳng. Để đốt cháy hoàn toàn $22.2$\emph{g} M cần dùng vừa hết $54.88$\emph{l} \emph{\ce{O2}} (đktc). Xác định CTPT \& \% về khối lượng của từng chất trong hỗn hợp M.
\end{baitoan}

\begin{baitoan}[\cite{SBT_Hoa_Hoc_11_co_ban}, \textbf{5.17}, p. 38]
	Hỗn hợp X chứa ancol etylic \emph{\ce{C2H5OH}} \& 2 ankan kế tiếp nhau trong dãy đồng đẳng. Khi đốt cháy hoàn toàn $18.9$\emph{g} X, thu được $26.1$\emph{g \ce{H2O}} \& $26.88$\emph{l \ce{CO2}} (đktc). Xác định CTPT \& \% về khối lượng của từng ankan trong hỗn hợp X.
\end{baitoan}

\begin{baitoan}[\cite{SBT_Hoa_Hoc_11_co_ban}, \textbf{5.25}, p. 39]
	Tìm nhận xét đúng: {\sf A.} Tất cả ankan \& tất cả xicloankan đều không tham gia phản ứng cộng. {\sf B.} Tất cả ankan \& tất cả xicloankan đều có thể tham gia phản ứng cộng. {\sf C.} Tất cả ankan không tham gia phản ứng cộng nhưng 1 số xicloankan lại có thể tham gia phản ứng cộng. {\sf D.} 1 số ankan có thể tham gia phản ứng cộng \& tất cả xicloankan không thể tham gia phản ứng cộng.
\end{baitoan}

\begin{baitoan}[\cite{SBT_Hoa_Hoc_11_co_ban}, \textbf{5.26}, p. 40]
	Các ankann không tham gia loại phản ứng nào? {\sf A.} Phản ứng thế. {\sf B.} Phản ứng cộng. {\sf C.} Phản ứng tách. {\sf D.} Phản ứng cháy.
\end{baitoan}

\begin{baitoan}[\cite{SBT_Hoa_Hoc_11_co_ban}, \textbf{5.27}, p. 40]
	Cho clo tác dụng với butan, thu được 2 dẫn xuất monoclo \emph{\ce{C4H8Cl}}. (a) Dùng CTCT viết PTHH, ghi tên các sản phẩm. (b) Tính \% của mỗi sản phẩm, biết nguyên tử hydro liên kết với carbon bậc 2 có khả năng bị thế cao hơn $3$ lần so với nguyên tử hydro liên kết với carbon bậc 1.
\end{baitoan}

\begin{baitoan}[\cite{SBT_Hoa_Hoc_11_co_ban}, \textbf{5.28}, p. 40]
	Hỗn hợp M ở thể lỏng, chứa 2 ankan. Để đốt cháy hoàn toàn hỗn hợp M cần dùng vừa hết $63.28$\emph{l} không khí (đktc). Hấp thụ hết sản phẩm cháy vào dung dịch \emph{\ce{Ca(OH)2}} lấy dư, thu được $36$\emph{g} chất kết tủa. (a) Tính khối lượng hỗn hợp M biết oxi chiếm $20$\% thể tích không khí. (b) Xác định CTPT \& \% khối lượng của từng chất trong hỗn hợp M nếu biết thêm 2 ankan khác nhau 2 nguyên tử carbon.
\end{baitoan}

\begin{baitoan}[\cite{SBT_Hoa_Hoc_11_co_ban}, \textbf{5.29}${}^\star$, p. 40]
	1 bình kín dung tích $11.2$\emph{l} có chứa $6.4$\emph{g} \emph{\ce{O2}} \& $1.36$\emph{g} hỗn hợp khí A gồm 2 ankan. Nhiệt độ trong bình là $0^\circ{\rm C}$ \& áp suất là $p_1$\emph{atm}. Bật tia lửa điện trong bình đó thì hỗn hợp A cháy hoàn toàn. Sau phản ứng, nhiệt độ trong bình là $136.5^\circ{\rm C}$ \& áp suất là $p_2$\emph{atm}. Nếu dẫn các chất trong bình sau phản ứng vào dung dịch \emph{\ce{Ca(OH)2}} lấy dư thì có $9$\emph{g} kết tủa tạo thành. (a) Tính $p_1,p_2$ biết thể tích bình không đổi. (b) Xác định CTPT \& \% thể tích từng chất trong hỗn hợp A, biết số mol của ankan có phân tử khối nhỏ nhiều gấp $1.5$ lần số mol của ankan có phân tử khối lớn.
\end{baitoan}

\begin{baitoan}[\cite{SBT_Hoa_Hoc_11_co_ban}, \textbf{5.30}${}^\star$, p. 40]
	Chất A có CTPT \emph{\ce{C6H_{14}}}. Khi A tác dụng với clo, có thể tạo ra tối đa $3$ dẫn xuất monoclo \emph{\ce{C6H_{13}Cl}} \& $7$ dẫn xuất điclo \emph{\ce{C6H_{12}Cl2}}. Viết CTCT của A \& của các dẫn xuất monoclo, điclo của A.
\end{baitoan}

\begin{baitoan}[\cite{SGK_Hoa_Hoc_11_co_ban}, \textbf{1.}, p. 123]
	Viết CTCT của các ankan sau: pentan, $2$-metylbutan, isobutan. Các chất trên còn có tên gọi nào khác không?
\end{baitoan}

\begin{baitoan}[\cite{SGK_Hoa_Hoc_11_co_ban}, \textbf{2.}, p. 123]
	Ankan Y mạch không nhánh có công thức đơn giản nhất là \emph{\ce{C2H5}}. (a) Tìm CTPT, viết CTCT \& gọi tên chất Y. (b) Viết PTHH phản ứng của Y với clo khi chiếu sáng, chỉ rõ sản phẩm chính của phản ứng.
\end{baitoan}

\begin{baitoan}[\cite{SGK_Hoa_Hoc_11_co_ban}, \textbf{3.}, p. 123]
	Đốt cháy hoàn toàn $3.36$\emph{l} hỗn hợp khí A gồm metan \& etan thu được $4.48$\emph{l} khí carbonic. Các thể tích khí được đo ở đktc. Tính thành phần \% về thể tích của mỗi khí trong hỗn hợp A.
\end{baitoan}

\begin{baitoan}[\cite{SGK_Hoa_Hoc_11_co_ban}, \textbf{4.}, p. 123]
	Khi $1$\emph{g} metan cháy tỏa ra $55.6$\emph{kJ}. Cần đốt bao nhiêu lít khí metan (đktc) để lượng nhiệt sinh ra đủ đung $1$\emph{l} nước ($D = 1$\emph{g\texttt{/}$\rm cm^3$}) từ $25^\circ{\rm C}$ lên $100^\circ{\rm C}$. Biết muốn nâng $1$\emph{g} nước lên $1^\circ{\rm C}$ cần tiêu tốn $4.18$\emph{J} \& giả sử nhiệt sinh ra chỉ dùng để làm tăng nhiệt độ của nước.
\end{baitoan}

%------------------------------------------------------------------------------%

\subsection{Xicloankan}

\begin{baitoan}[\cite{SGK_Hoa_Hoc_11_co_ban}, \textbf{1.}, p. 120]
	\emph{Đ\texttt{/}S}? {\bf A.} Xicloankan chỉ có khả năng tham gia phản ứng cộng mở vòng. {\bf B.} Xicloankan chỉ có khả năng tham gia phản ứng thế. {\bf C.} Mọi xicloankan đều có khả năng tham gia phản ứng thế \& phản ứng cộng. {\bf D.} 1 số xicloankan có khả năng tham gia phản ứng cộng mở vòng.
\end{baitoan}

\begin{baitoan}[\cite{SGK_Hoa_Hoc_11_co_ban}, \textbf{2.}, p. 120]
	Khi sục khí xiclopropan vào dung dịch brom sẽ quan sát thấy hiện tượng nào sau đây? {\bf A.} Màu dung dịch không đổi. {\bf B.} Màu dung dịch đậm lên. {\bf C.} Màu dung dịch bị nhạt dần. {\bf D.} Màu dung dịch từ không màu chuyển thành nâu đỏ.
\end{baitoan}

\begin{baitoan}[\cite{SGK_Hoa_Hoc_11_co_ban}, \textbf{3.}, p. 121]
	Viết PTHH của phản ứng xảy ra khi: (a) Sục khí xiclopropan vào dung dịch brom. (b) Dẫn hỗn hợp xiclopropan, xiclopentan \& hydro đi vào trong ống có bột niken, nung nóng. (c) Đun nóng xiclohexan với brom theo tỷ lệ mol $1:1$.
\end{baitoan}

\begin{baitoan}[\cite{SGK_Hoa_Hoc_11_co_ban}, \textbf{4.}, p. 121]
	Trình bày phương pháp hóa học phân biệt 2 khí không màu propan \& xiclopropan đựng trong các bình riêng biệt.
\end{baitoan}

\begin{baitoan}[\cite{SGK_Hoa_Hoc_11_co_ban}, \textbf{5.}, p. 121]
	Xicloankan đơn vòng X có tỷ khối so với nitơ bằng $2$. Lập CTPT của X. Viết PTHH (ở dạng CTCT) minh họa tính chất hóa học của X, biết X tác dụng với \emph{\ce{H2}} (xúc tác \emph{Ni}) chỉ tạo ra 1 sản phẩm.
\end{baitoan}

\begin{baitoan}[\cite{SBT_Hoa_Hoc_11_co_ban}, \textbf{5.18}, p. 38]
	\emph{Đ\texttt{/}S?} (a) Các monoxicloankan đều có công thức phân tử \emph{\ce{C_nH_{2n}}}. (b) Các chất có công thức phân tử \emph{\ce{C_nH_{2n}}} đều là monoxicloankan. (c) Các xicloankan đều chỉ có liên kết đơn. (d) Các chất chỉ có liên kết đơn đều là xicloankan.
\end{baitoan}

\begin{baitoan}[\cite{SBT_Hoa_Hoc_11_co_ban}, \textbf{5.19}, p. 38]
	Hợp chất dưới đây có tên là gì?
	\begin{center}
		\chemfig[atom sep=2em]{CH_3-*6(-(-CH_3)---(-C_2H_5)--)}
	\end{center}
	{\sf A.} $1$-etyl-$4,5$-đimetylxiclohexan. {\sf B.} $1$-etyl-$3,4$-đimetylxiclohexan. {\sf C.} $1,2$-đimetyl-$4$-etylxiclohexan. {\sf D.} $4$-etyl-$1,2$-đimetylxiclohexan.	
\end{baitoan}

\begin{baitoan}[\cite{SBT_Hoa_Hoc_11_co_ban}, \textbf{5.20}, p. 38]
	Tìm nhận xét đúng: {\sf A.} Xiclohexan vừa có phản ứng thế, vừa có phản ứng cộng. {\sf B.} Xiclohexan không có phản ứng thế, không có phản ứng cộng. {\sf C.} Xiclohexan có phản ứng thế, không có phản ứng cộng. {\sf D.} Xiclohexan không có phản ứng thế, có phản ứng cộng.
\end{baitoan}

\begin{baitoan}[\cite{SBT_Hoa_Hoc_11_co_ban}, \textbf{5.21}, p. 39]
	Viết CTCT của: (a) $1,1$-đimetylxiclopropan. (b) $1$-etyl-$1$-metylxiclohexan. (c) $1$-metyl-$4$-isopropylxiclohexan.
\end{baitoan}

\begin{baitoan}[\cite{SBT_Hoa_Hoc_11_co_ban}, \textbf{5.22}, p. 39]
	1 monoxicloankan có tỷ khối hơi so với nitơ bằng $3$. (a) Xác định CTPT của xicloankan đó. (b) Viết CTCT \& tên tất cả các xicloankan ứng với CTPT tìm được.
\end{baitoan}

\begin{baitoan}[\cite{SBT_Hoa_Hoc_11_co_ban}, \textbf{5.23}, p. 39]
	Hỗn hợp khí A chứa 1 ankan \& 1 monoxicloankan. Tỷ khối của A đối với hydro là $25.8$. Đốt cháy hoàn toàn $2.58$\emph{g} A rồi hấp thụ hết sản phẩm cháy vào dung dịch \emph{\ce{Ba(OH)2}} dư, thu được $35.46$\emph{g} kết tủa. Xác định CTPT \& \% thể tích của từng chất trong hỗn hợp khí A.
\end{baitoan}

\begin{baitoan}[\cite{SBT_Hoa_Hoc_11_co_ban}, \textbf{5.24}, p. 39]
	Chất khí A là 1 xicloankan. Khi đốt cháy $672$\emph{ml} A (đktc), thì thấy khối lượng \emph{\ce{CO2}} tạo thành nhiều hơn khối lượng nước tạo thành $3.12$\emph{g}. (a) Xác định CTPT chất A. (b) Viết CTCT \& tên các xicloankan ứng với CTPT tìm được. (c) Cho chất A qua dung dịch brom, màu của dung dịch mất đi. Xác định CTCT đúng của chất A.
\end{baitoan}

\begin{baitoan}[\cite{SGK_Hoa_Hoc_11_co_ban}, \textbf{5.}, p. 123]
	Khi cho iso pentan tác dụng với brom theo tỷ lệ mol $1:1$, sản phẩm chính thu được là: {\sf A.} $2$-brompentan. {\sf B.} $1$-brompentan. {\sf C.} $1,3$-đibrompentan. {\sf D.} $2,3$-đibrompentan.
\end{baitoan}

\begin{baitoan}[\cite{SGK_Hoa_Hoc_11_co_ban}, \textbf{6.}, p. 123]
	\emph{Đ\texttt{/}S?} (a) Ankan là hydrocarbon no, mạch hở. (b) Ankan có thể bị tách hydro thành anken. (c) Crăckinh ankan thu được hỗn hợp các ankan. (d) Phản ứng của clo với ankan tạo thành ankyl clorua thuộc loại phản ứng thế. (e) Ankan có nhiều trong dầu mỏ.
\end{baitoan}

%------------------------------------------------------------------------------%

\section{Hydrocarbon Không No}

\subsection{Anken}

\begin{baitoan}[\cite{SGK_Hoa_Hoc_11_co_ban}, \textbf{1.}, p. 132]
	So sánh anken với ankan về đặc điểm cấu tạo \& tính chất hóa học. Cho ví dụ minh họa.
\end{baitoan}

\begin{baitoan}[\cite{SGK_Hoa_Hoc_11_co_ban}, \textbf{2.}, p. 132]
	Ứng với CTPT \emph{\ce{C5H_{10}}} có bao nhiêu anken đồng phân cấu tạo? {\sf A.} $4$. {\sf B.} $5$. {\sf C.} $3$. {\sf D.} $7$.
\end{baitoan}

\begin{baitoan}[\cite{SGK_Hoa_Hoc_11_co_ban}, \textbf{3.}, p. 132]
	Viết PTHH của phản ứng xảy ra khi: (a) Propilen tác dụng với hydro, đun nóng (xúc tác \emph{Ni}). (b) But-$2$-en tác dụng với hydro clorua. (c) Metylpropen tác dụng với nước có xúc tác acid. (d) Trùng hợp but-$1$-en.
\end{baitoan}

\begin{baitoan}[\cite{SGK_Hoa_Hoc_11_co_ban}, \textbf{4.}, p. 132]
	Trình bày phương pháp hóa học để: (a) Phân biệt metan \& etilen. (b) Tách lấy khí metan từ hỗn hợp với etilen. (c) Phân biệt 2 bình không dán nhãn đựng hexan \& hex-$1$-en. Viết PTHH của các phản ứng đã dùng.
\end{baitoan}

\begin{baitoan}[\cite{SGK_Hoa_Hoc_11_co_ban}, \textbf{5.}, p. 132]
	Chất nào sau đây làm mất màu dung dịch brom? {\sf A.} butan. {\sf B.} but-$1$-en. {\sf C.} carbon dioxide. {\sf D.} metylpropan.
\end{baitoan}

\begin{baitoan}[\cite{SGK_Hoa_Hoc_11_co_ban}, \textbf{6.}, p. 132]
	Dẫn từ từ $3.36$\emph{l} hỗn hợp gồm etilen \& propilen (đktc) vào dung dịch brom thấy dung dịch bị nhạt màu \& không còn khí thoát ra. Khối lượng dung dịch sau phản ứng tăng $4.9$\emph{g}. (a) Viết các PTHH \& giải thích các hiện tượng ở thí nghiệm trên. (b) Tính thành phần \% về thể tích của mỗi khí trong hỗn hợp ban đầu.
\end{baitoan}

\begin{baitoan}[\cite{SBT_Hoa_Hoc_11_co_ban}, \textbf{6.1.}, p. 41]
	Gọi tên hợp chất:
	\begin{center}
		\chemfig[atom sep=2em]{CH_3-C(-[2]CH_3)(-[6]CH_3)-CH_2-CH=CH_2}
	\end{center}
	{\sf A.} $2$-đimetylpent-$4$-en. {\sf B.} $2,2$-đimetylpent-$4$-en. {\sf C.} $4$-đimetylpent-$1$-en. {\sf D.} $4,4$-đimetylpent-$1$-en.
\end{baitoan}

\begin{baitoan}[\cite{SBT_Hoa_Hoc_11_co_ban}, \textbf{6.2.}, p. 41]
	Gọi tên hợp chất:
	\begin{center}
		\chemfig[atom sep=2em]{CH_3-CH_2-C(=[6]CH_2)-CH_2-CH_3}
	\end{center}
	{\sf A.} $3$-metylenpentan. {\sf B.} $1,1$-đietyleten. {\sf C.} $2$-etylbut-$1$-en. {\sf D.} $3$-etylbut-$3$-en.
\end{baitoan}

\begin{baitoan}[\cite{SBT_Hoa_Hoc_11_co_ban}, \textbf{6.3.}, p. 41]
	\emph{Đ\texttt{/}S?} (a) Tất cả các anken đều có công thức là \emph{\ce{C_nH_{2n}}}. (b) Tất cả các chất có công thức chung \emph{\ce{C_nH_{2n}}} đều là anken. (c) Tất cả các anken đều làm mất màu dung dịch brom. (d) Các chất làm mất màu dung dịch brom đều là anken.
\end{baitoan}

\begin{baitoan}[\cite{SBT_Hoa_Hoc_11_co_ban}, \textbf{6.4.}, p. 42]
	Gọi tên \& cho biết hợp chất nào là $2,4$-đimetylhex-$1$-en:
	\begin{center}
		\chemfig[atom sep=2em]{CH_3-CH(-[6]CH_3)-CH_2-CH(-[6]CH_3)-CH=CH_2},\hspace{1cm}\chemfig[atom sep=2em]{CH_3-CH(-[6]CH_2-CH_3)-CH_2-C(=[6]CH_2)-CH_3},\\\vspace{5mm}\chemfig[atom sep=2em]{CH_2=C(-[6]CH_3)-CH_2-CH_2-CH(-[6]CH_3)-CH_3},\hspace{1cm}\chemfig[atom sep=2em]{CH_2=C(-[6]CH_3)-CH_2-CH(-[6]CH_3)-CH_2-CH_2-CH_3}.
	\end{center}
\end{baitoan}

\begin{baitoan}[\cite{SBT_Hoa_Hoc_11_co_ban}, \textbf{6.5.}, p. 42]
	Để phân biệt etan \& eten, dùng phản ứng nào là thuận tiện nhất? {\sf A.} Phản ứng đốt cháy. {\sf B.} Phản ứng cộng với hydro. {\sf C.} Phản ứng với nước brom. {\sf D.} Phản ứng trùng hợp.
\end{baitoan}

\begin{baitoan}[\cite{SBT_Hoa_Hoc_11_co_ban}, \textbf{6.6.}, p. 42]
	Trình bày phương pháp hóa học để phân biệt 3 khí: etan, etilen, \& carbon dioxide.
\end{baitoan}

\begin{baitoan}[\cite{SBT_Hoa_Hoc_11_co_ban}, \textbf{6.7.}, p. 42]
	Hỗn hợp khí A chứa 1 ankan \& 1 anken. Khối lượng hỗn hợp A là $9$\emph{g} \& thể tích là $8.96$\emph{l}. Đốt cháy hoàn toàn A, thu được $13.44$\emph{l} \emph{\ce{CO2}}. Các thể tích được đo ở đktc. Xác định CTPT \& \% thể tích từng chất trong A.
\end{baitoan}

\begin{baitoan}[\cite{SBT_Hoa_Hoc_11_co_ban}, \textbf{6.8.}, p. 42]
	$0.7$\emph{g} 1 anken có thể làm mất màu $16$\emph{g} dung dịch brom (trong \emph{\ce{CCl4}} có nồng độ $12.5$\%. (a) Xác định CTPT chất A. (b) Viết CTCT của tất cả các đồng phân cấu tạo ứng với CTPT tìm được. Ghi tên từng đồng phân.
\end{baitoan}

\begin{baitoan}[\cite{SBT_Hoa_Hoc_11_co_ban}, \textbf{6.9.}, p. 42]
	Hỗn hợp khí A chứa eten \& hydro. Tỷ khối của A đối với hydro là $7.5$. Dẫn A đi qua chất xúc tác \emph{Ni} nung nóng thì A biến thành hỗn hợp khí B có tỷ khối đối với hydro là $9$. Tính hiệu suất phản ứng cộng hydro của eten.
\end{baitoan}

\begin{baitoan}[\cite{SBT_Hoa_Hoc_11_co_ban}, \textbf{6.10.}, p. 43]
	Hỗn hợp khí A chứa hydro \& 1 anken. Tỷ khối của A đối với hydro là $6$. Đun nóng nhẹ hỗn hợp A có mặt chất xúc tác \emph{Ni} thì A biến thành hỗn hợp khí B không làm mất màu nước brom \& có tỷ khối đối với hydro là $9$. Xác định CTPT \& \% thể tích của từng chất trong hỗn hợp A \& hỗn hợp B.
\end{baitoan}

\begin{baitoan}[\cite{SBT_Hoa_Hoc_11_co_ban}, \textbf{6.11.}, p. 43]
	Hỗn hợp khí A chứa hydro \& 2 anken kế tiếp nhau trong dãy đồng đẳng. Tỷ khối của A đối với hydro là $8.26$. Đun nóng nhẹ hỗn hợp A có mặt chất xúc tác \emph{Ni} thì A biến thành hỗn hợp khí B không làm mất màu nước brom \& có tỷ khối đối với hydro là $11.8$. Xác định CTPT \& \% thể tích của từng chất trong hỗn hợp A \& hỗn hợp B.
\end{baitoan}

\begin{baitoan}[\cite{SBT_Hoa_Hoc_11_co_ban}, \textbf{6.12.}, p. 43]
	Hỗn hợp khí A chứa hydro, 1 ankan \& 1 anken. Dẫn $13.44$\emph{l} A đi qua chất xúc tác \emph{Ni} nung nóng thì thu được $10.08$\emph{l} hỗn hợp khí B. Dẫn B đi qua bình đựng nước brom thì màu của dung dịch nhạt đi, khối lượng của bình tăng thêm $3.15$\emph{g}. Sau thí nghiệm, còn lại $8.4$\emph{l} hỗn hợp khí C có tỷ khối đối với hydro là $17.80$. Biết các thể tích được đo ở đktc \& các phản ứng đều xảy ra hoàn toàn. Xác định CTPT \& \% thể tích của từng chất trong mỗi hỗn hợp A, B, \& C.
\end{baitoan}

\begin{baitoan}[\cite{SBT_Hoa_Hoc_11_co_ban}, \textbf{6.13.}, p. 43]
	Hỗn hợp khí A chứa hydro, 1 ankan, \& 1 anken. Đốt cháy hoàn toàn $100$\emph{ml} A, thu được $210$\emph{ml} khí \emph{\ce{CO2}}. Nếu đun nóng nhẹ $100$\emph{ml} A có mặt chất xúc tác \emph{Ni} thì còn lại $70$\emph{ml} 1 chất khí duy nhất. Các thể  tích khí đều đo ở cùng 1 điều kiện. (a) Xác định CTPT \& \% thể tích của từng chất trong hỗn hợp A. (b) Tính thể tích oxi vừa đủ để đốt cháy hoàn toàn $100$\emph{ml} A.
\end{baitoan}

%------------------------------------------------------------------------------%

\subsection{Ankađien}

\begin{baitoan}[\cite{SBT_Hoa_Hoc_11_co_ban}, \textbf{6.14.}, p. 44]
	Cho isopren ($2$-metylbuta-$1,3$-đien) phản ứng cộng với brom theo tỷ lệ $1:1$ về số mol. Hỏi có thể thu được tối đa mấy đồng phân cấu tạo có cùng công thức phân tử \emph{\ce{C5H8Br2}}? {\sf A.} $1$. {\sf B.} $2$. {\sf C.} $3$. {\sf D.} $4$.
\end{baitoan}

\begin{baitoan}[\cite{SBT_Hoa_Hoc_11_co_ban}, \textbf{6.15.}, p. 44]
	Trong các chất dưới đây, chất nào được gọi tên là \emph{đivinyl}? {\sf A.} \emph{\ce{CH2=C=CH-CH3}}. {\sf B.} \emph{\ce{CH2=CH-CH=CH2}}. {\sf C.} \emph{\ce{CH2=CH-CH2-CH=CH2}}. {\sf D.} \emph{\ce{CH2=CH-CH=CH-CH3}}.
\end{baitoan}

\begin{baitoan}[\cite{SBT_Hoa_Hoc_11_co_ban}, \textbf{6.16.}, p. 44]
	\emph{Đ\texttt{/}S?} (a) Các chất có công thức \emph{\ce{C_nH_{2n-2}}} đều là ankađien. (b) Các ankađien đều có công thức \emph{\ce{C_nH_{2n-2}}}. (c) Các ankađien đều có $2$ liên kết đôi. (d) Các chất có $2$ liên kết đôi đều là ankađien.
\end{baitoan}

\begin{baitoan}[\cite{SBT_Hoa_Hoc_11_co_ban}, \textbf{6.17.}, p. 44]
	Viết CTCT của: (a) $2,3$-đimetylbuta-$1,3$-đien. (b) $3$-metylpenta-$1,4$-đien.
\end{baitoan}

\begin{baitoan}[\cite{SBT_Hoa_Hoc_11_co_ban}, \textbf{6.18.}, p. 44]
	Chất A là 1 ankađien liên hợp có mạch carbon phân nhánh. Để đốt cháy hoàn toàn $3.4$\emph{g} A cần dùng vừa hết $7.84$\emph{l} \emph{\ce{O2}} lấy ở đktc. Xác định CTPT, CTCT \& tên của chất A.
\end{baitoan}

\begin{baitoan}[\cite{SBT_Hoa_Hoc_11_co_ban}, \textbf{6.19.}, p. 44]
	Hỗn hợp khí A chứa 1 ankan \& 1 ankađien. Để đốt cháy hoàn toàn $6.72$\emph{l} A phải dùng vừa hết $28$\emph{l} \emph{\ce{O2}} (các thể tích lấy ở đktc). Dẫn sản phẩm cháy qua bình thứ nhất đựng \emph{\ce{H2SO4}} đặc, sau đó qua bình thứ 2 đựng dung dịch \emph{\ce{NaOH}} (lấy dư) thì khối lượng bình thứ nhất tăng $p$\emph{g} \& bình thứ 2 tăng $35.2$\emph{g}. (a) Xác định CTPT \& \% theo thể tích của từng chất trong hỗn hợp A. (b) Tính $p$.
\end{baitoan}

\begin{baitoan}[\cite{SBT_Hoa_Hoc_11_co_ban}, \textbf{6.20.}, p. 45]
	Ghép tên chất: $4$-etyl-$2$-metylhexan, $1,1$-etylmetylxiclopropan, $3,3$-đimetylbut-$1$-en, đivinyl, isopropylxiclopropan, isopren, $2,2,4,4$-tetrametylpentan, $2,3$-đimetylbut-$2$-en với CTCT:\\\emph{\ce{(CH3)3CCH2C(CH3)3},  \ce{(CH3)2CHCH2CH(CH2CH3)2}, \ce{(CH3)2C=C(CH3)2}, \ce{CH2=CHC(CH3)3}, \ce{CH2=CHC(CH3)=CH2}},\\\emph{\ce{CH2=CHCH=CH2}}.
\end{baitoan}

\begin{baitoan}[\cite{SBT_Hoa_Hoc_11_co_ban}, \textbf{6.21.}, p. 45]
	Gọi tên hợp chất:
	\begin{center}
		\chemfig[atom sep=2em]{CH_3-CH(-[6]CH_2(-[6]CH_3))-CH(-[6]CH_3)-CH=CH_2}
	\end{center}
	{\sf A.} $2$-etyl-$3$-metylpent-$4$-en. {\sf B.} $4$-etyl-$3$-metylpent-$1$-en. {\sf C.} $3,4$-đimetyl-hex-$5$-en. {\sf D.} $3,4$-đimetyl-hex-$1$-en.
\end{baitoan}

\begin{baitoan}[\cite{SBT_Hoa_Hoc_11_co_ban}, \textbf{6.22.}, p. 46]
	Gọi tên hợp chất:
	\begin{center}
		\chemfig[atom sep=2em]{CH_2=CH-CH(-[6]CH_3)-CH=CH-CH_3}
	\end{center}
	{\sf A.} $3$-metylhexa-$1,2$-đien. {\sf B.} $4$-metylhexa-$1,5$-đien. {\sf C.} $3$-metylhexa-$1,4$-đien. {\sf D.} $3$-metylhexa-$1,3$-đien.
\end{baitoan}

\begin{baitoan}[\cite{SBT_Hoa_Hoc_11_co_ban}, \textbf{6.23.}, p. 46]
	Trong các chất sau, chất nào là ankađien liên hợp?
	\begin{center}
		\chemfig[atom sep=2em]{CH_2=CH-CH_2-CH=CH_2},\hspace{5mm}\chemfig[atom sep=2em]{CH_2=C(-[6]CH_3)-C(-[6]CH_3)=CH_2},\hspace{5mm}\emph{\ce{CH2=CH-CH2-CH=CH-CH3}},\hspace{5mm}\emph{\ce{CH2=C=CH2}}.
	\end{center}
\end{baitoan}

\begin{baitoan}[\cite{SBT_Hoa_Hoc_11_co_ban}, \textbf{6.24.}, p. 46]
	Hỗn hợp khí A chứa nitơ \& 2 hydrocarbon kế tiếp nhau trong 1 dãy đồng đẳng. Khối lượng hỗn hợp A là $18.3$\emph{g} \& thể tích của nó là $11.2$\emph{l}. Trộn A với 1 lượng dư oxi rồi đốt cháy, thu được $11.7$\emph{g} \emph{\ce{H2O}} \& $21.28$\emph{l} khí \emph{\ce{CO2}}. Các thể tích đo ở đktc. Xác định CTPT \& \% về khối lượng của từng hydrocarbon trong hỗn hợp A.
\end{baitoan}

%------------------------------------------------------------------------------%

\subsection{Ankin}

\begin{baitoan}[\cite{SBT_Hoa_Hoc_11_co_ban}, \textbf{6.25.}, pp. 46--47]
	Gọi tên chất:
	\begin{center}
		\chemfig[atom sep=2em]{CH_3-C(-[2]CH_3)(-[6]CH_3)-C~CH}.
	\end{center}
	{\sf A.} $2,2$-đimetylbut-$1$-in. {\sf B.} $2,2$-đimetylbut-$3$-in. {\sf C.} $3,3$-đimetylbut-$1$-in. {\sf D.} $3,3$-đimetylbut-$2$-in.
\end{baitoan}

\begin{baitoan}[\cite{SBT_Hoa_Hoc_11_co_ban}, \textbf{6.26.}, p. 47]
	Có $4$ chất: metan, etilen, but-$1$-in \& but$-2$-in. Trong $4$ chất đó, có mấy chất tác dụng được với dung dịch \emph{\ce{AgNO3}} trong amoniac tạo thành kết tủa? {\sf A.} $4$ chất. {\sf B.} $3$ chất. {\sf C.} $2$ chất. {\sf D.} $1$ chất.
\end{baitoan}

\begin{baitoan}[\cite{SBT_Hoa_Hoc_11_co_ban}, \textbf{6.27.}, p. 47]
	\emph{Đ\texttt{/}S?} (a) Tất cả các ankin đều cháy khi được đốt trong oxi. (b) Tất cả các ankin đều làm mất màu dung dịch \emph{\ce{KMnO4}}. (c) Tất cả các ankin đều làm mất màu dung dịch brom. (d) Tất cả các ankin đều tác dụng với dung dịch \emph{\ce{AgNO3}} trong amoniac. (e) Tất cả ankin đều tác dụng được với hydro ở nhiệt độ cao \& có chất xúc tác \emph{Ni}.
\end{baitoan}
Bài tập sơ đồ chuỗi phản ứng hóa học: \cite[\textbf{6.28.}, p. 47]{SBT_Hoa_Hoc_11_co_ban}.

\begin{baitoan}[\cite{SBT_Hoa_Hoc_11_co_ban}, \textbf{6.29.}, p. 47]
	Hỗn hợp khí A chứa hydro \& 1 ankin. Tỷ khối của A đối với hydro là $4.8$. Đun nóng hỗn hợp A có mặt chất xúc tác \emph{Ni} thì phản ứng xảy ra với hiệu suất được coi là $100$\%, tạo ra hỗn hợp khí B không làm mất màu nước brom \& có tỷ khối đối với hydro là $8$. Xác định CTPT \& \% về thể tích của từng chất trong hỗn hợp A \& hỗn hợp B.
\end{baitoan}

\begin{baitoan}[\cite{SBT_Hoa_Hoc_11_co_ban}, \textbf{6.30.}, p. 47]
	Hỗn hợp khí A chứa \emph{\ce{C2H2,H2}}. Tỷ khối của A đối với hydro là $5$. Dẫn $20.16$\emph{l} A đi nhanh qua chất xúc tác \emph{Ni} nung nóng thì nó biến thành $10.08$\emph{l} hỗn hợp khí B. Dẫn hỗn hợp B đi từ từ qua bình đựng nước brom (có dư) cho phản ứng xảy ra hoàn toàn thì còn lại $7.39$\emph{l} hỗn hợp khí C. Các thể tích được đo ở đktc. (a) Tính \% thể tích từng chất trong mỗi hỗn hợp A, B, \& C. (b) Khối lượng bình đựng nước brom tăng thêm bao nhiêu \emph{g}?
\end{baitoan}

\begin{baitoan}[\cite{SBT_Hoa_Hoc_11_co_ban}, \textbf{6.31.}, p. 48]
	Hỗn hợp khí A chứa hydro, 1 anken, \& 1 ankin. Đốt cháy hoàn toàn $90$\emph{ml} A thu được $120$\emph{ml} \emph{\ce{CO2}}. Đun nóng $90$\emph{ml} A có mặt chất xúc tác \emph{Ni} thì sau phản ứng chỉ còn lại $40$\emph{ml} 1 ankan duy nhất. Các thể tích đo ở cùng 1 điều kiện. (a) Xác định CTPT \& \% thể tích từng chất trong hỗn hợp A. (b) Tính thể tích \emph{\ce{O2}} vừa đủ để đốt cháy hoàn toàn $90$\emph{ml} A.
\end{baitoan}

\begin{baitoan}[\cite{SBT_Hoa_Hoc_11_co_ban}, \textbf{6.32.}, p. 48]
	(a) CTPT nào phù hợp với penten? {\sf A.} \emph{\ce{C5H8}}. {\sf B.} \emph{\ce{C5H10}}. {\sf C.} \emph{\ce{C5H12}}. {\sf D.} \emph{\ce{C3H6}}. (b) Hợp chất nào là ankin? {\sf A.} \emph{\ce{C2H2}}. {\sf B.} \emph{\ce{C8H8}}. {\sf C.} \emph{\ce{C4H4}}. {\sf D.} \emph{\ce{C6H6}}. (c) Gốc nào là ankyl? {\sf A.} \emph{\ce{-C3H5}}. {\sf B.} \emph{\ce{-C6H5}}. {\sf C.} \emph{\ce{-C2H3}}. {\sf D.} \emph{\ce{-C2H5}}.
\end{baitoan}

\begin{baitoan}[\cite{SBT_Hoa_Hoc_11_co_ban}, \textbf{6.33.}, p. 48]
	(a) Chất nào có nhiệt độ sôi cao nhất? {\sf A.} Eten. {\sf B.} Propen. {\sf C.} But-$1$-en. {\sf D.} Pent-$1$-en. (b) Chất nào không tác dụng với dung dịch \emph{\ce{AgNO3}} trong amoniac? {\sf A.} But-$1$-in. {\sf B.} But-$2$-in. {\sf C.} Propin. {\sf D.} Etin. (c) Chất nào không tác dụng với \emph{\ce{Br2}} (tan trong \emph{\ce{CCl4}})? {\sf A.} But-$1$-in. {\sf B.} But-$1$-en. {\sf C.} Xiclobutan. {\sf D.} Xiclopropan.
\end{baitoan}

\begin{baitoan}[\cite{SBT_Hoa_Hoc_11_co_ban}, \textbf{6.34.}, p. 48]
	Viết PTHH của các phản ứng xảy ra trong quá trình điều chế PVC xuất phát từ các chất vô cơ: \emph{\ce{CaO,HCl,H2O,C}}.
\end{baitoan}

\begin{baitoan}[\cite{SBT_Hoa_Hoc_11_co_ban}, \textbf{6.35.}, p. 49]
	Hỗn hợp khí A chứa metan, axetilen, \& propen. Đốt cháy hoàn toàn $11$\emph{g} hỗn hợp A, thu được $12.6$\emph{g \ce{H2O}}. Mặc khác, nếu lấy $11.2$\emph{l} A (đktc) đem dẫn qua nước brom (lấy dư) thì khối lượng brom nguyên chất dự phản ứng tối đa là $100$\emph{g}. Xác định thành phần \% theo khối lượng \& theo thế tích của từng chất trong hỗn hợp A.
\end{baitoan}

\begin{baitoan}[\cite{SBT_Hoa_Hoc_11_co_ban}, \textbf{6.36.}, p. 49]
	1 bình kín dung tích $8.4$\emph{l} có chứa $4.96$\emph{g \ce{O2}} \& $1.3$\emph{g} hỗn hợp khí A gồm $2$ hydrocarbon. Nhiệt độ trung bình $t_1 = 0^\circ{\rm C}$ \& áp suất trong bình $p_1 = 0.5$\emph{atm}. Bật tia lửa điện trong bình kín đó thì hỗn hợp A cháy hoàn toàn. Sau phản ứng, nhiệt độ trong bình là $t_2 = 136.5^\circ{\rm C}$ \& áp suất là $p_2$\emph{atm}. Dẫn các chất trong bình sau phản ứng đi qua bình thứ nhất đựng \emph{\ce{H2SO4}} đặc, sau đó qua bình 2 đựng dung dịch \emph{NaOH} (có dư) thì khối lượng bình thứ 2 tăng $4.18$\emph{g}. (a) Tính $p_2$ biết thể tích bình không đổi. (b) Xác định CTPT \& \% theo thể tích của từng chất trong hỗn hợp A nếu biết thêm trong hỗn hợp đó có 1 chất là anken \& 1 chất là ankin.
\end{baitoan}

\begin{baitoan}[\cite{SBT_Hoa_Hoc_11_co_ban}, \textbf{6.37.}, p. 49]
	Trình bày phương pháp hóa học để phân biệt các hydrocarbon sau: (a) axetilen \& metan. (b) axetilen \& etilen. (c) axetilen, etilen, \& metan. (d) but-$1$-in \& but-$2$-in.
\end{baitoan}

\begin{baitoan}[\cite{SBT_Hoa_Hoc_11_co_ban}, \textbf{6.38.}, p. 49]
	Cho biết phương pháp làm sạch chất khí: (a) metan lẫn tạp chất là axetilen \& etilen. (b) etilen lẫn tạp chất là axetilen.
\end{baitoan}

%------------------------------------------------------------------------------%

\section{Hydrocarbon Thơm. Nguồn Hydrocarbon Thiên Nhiên. Hệ Thống Hóa về Hydrocarbon}

\subsection{Benzen \& Đồng Đẳng. 1 Số Hydrocarbon Thơm Khác}

%------------------------------------------------------------------------------%

\subsection{Hydrocarbon Thơm}

%------------------------------------------------------------------------------%

\subsection{Nguồn Hydrocarbon Thiên Nhiên}

%------------------------------------------------------------------------------%

\subsection{Hệ Thống Hóa về Hydrocarbon}

%------------------------------------------------------------------------------%

\printbibliography[heading=bibintoc]
	
\end{document}