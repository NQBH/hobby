\documentclass{article}
\usepackage[backend=biber,natbib=true,style=authoryear,maxbibnames=50]{biblatex}
\addbibresource{/home/nqbh/reference/bib.bib}
\usepackage[utf8]{vietnam}
\usepackage{tocloft}
\renewcommand{\cftsecleader}{\cftdotfill{\cftdotsep}}
\usepackage[colorlinks=true,linkcolor=blue,urlcolor=red,citecolor=magenta]{hyperref}
\usepackage{amsmath,amssymb,amsthm,mathtools,float,graphicx,algpseudocode,algorithm,tcolorbox,booktabs}
\usepackage[version=4]{mhchem}
\usepackage[inline]{enumitem}
\allowdisplaybreaks
\numberwithin{equation}{section}
\newtheorem{assumption}{Assumption}[section]
\newtheorem{conjecture}{Conjecture}[section]
\newtheorem{corollary}{Corollary}[section]
\newtheorem{dangtoan}{Dạng toán}[section]
\newtheorem{hequa}{Hệ quả}[section]
\newtheorem{definition}{Definition}[section]
\newtheorem{dinhnghia}{Định nghĩa}[section]
\newtheorem{example}{Example}[section]
\newtheorem{vidu}{Ví dụ}[section]
\newtheorem{lemma}{Lemma}[section]
\newtheorem{notation}{Notation}[section]
\newtheorem{principle}{Principle}[section]
\newtheorem{problem}{Problem}[section]
\newtheorem{baitoan}{Bài toán}[section]
\newtheorem{proposition}{Proposition}[section]
\newtheorem{question}{Question}[section]
\newtheorem{cauhoi}{Câu hỏi}[section]
\newtheorem{remark}{Remark}[section]
\newtheorem{luuy}{Lưu ý}[section]
\newtheorem{theorem}{Theorem}[section]
\newtheorem{dinhly}{Định lý}[section]
\usepackage[left=0.5in,right=0.5in,top=1.5cm,bottom=1.5cm]{geometry}
\usepackage{fancyhdr}
\pagestyle{fancy}
\fancyhf{}
\lhead{\small Sect.~\thesection}
\rhead{\small \nouppercase{\leftmark}}
\renewcommand{\sectionmark}[1]{\markboth{#1}{}}
\cfoot{\thepage}
\def\labelitemii{$\circ$}

\title{Hydrocarbon}
\author{Nguyễn Quản Bá Hồng\footnote{Independent Researcher, Ben Tre City, Vietnam\\e-mail: \texttt{nguyenquanbahong@gmail.com}; website: \url{https://nqbh.github.io}.}}
\date{\today}

\begin{document}
\maketitle
\begin{abstract}
	1 bộ sưu tập các bài tập chọn lọc từ cơ bản đến nâng cao cho Hóa học sơ cấp lớp 11. Tài liệu này là phần bài tập bổ sung cho tài liệu chính \href{https://github.com/NQBH/hobby/blob/master/elementary_chemistry/grade_11/NQBH_elementary_chemistry_grade_11.pdf}{GitHub\texttt{/}NQBH\texttt{/}hobby\texttt{/}elementary chemistry\texttt{/}grade 6\texttt{/}lecture}\footnote{\textsc{url}: \url{https://github.com/NQBH/hobby/blob/master/elementary_chemistry/grade_11/NQBH_elementary_chemistry_grade_11.pdf}.} của tác giả viết cho Toán lớp 6. Phiên bản mới nhất của tài liệu này được lưu trữ ở link sau: \href{https://github.com/NQBH/hobby/blob/master/elementary_chemistry/grade_11/problem/NQBH_elementary_chemistry_grade_11_problem.pdf}{GitHub\texttt{/}NQBH\texttt{/}hobby\texttt{/}elementary chemistry\texttt{/}grade 6\texttt{/}problem}\footnote{\textsc{url}: \url{https://github.com/NQBH/hobby/blob/master/elementary_chemistry/grade_11/problem/NQBH_elementary_chemistry_grade_11_problem.pdf}.}.
\end{abstract}
\tableofcontents
\newpage

%------------------------------------------------------------------------------%

\section{Hydrocarbon No}

\subsection{Ankan}

\begin{baitoan}[\cite{SBT_Hoa_Hoc_11_co_ban}, \textbf{5.1}, p. 35]
	Điền vào chỗ khuyết những từ thích hợp trong các từ \& cụm từ: \emph{ankan, xicloankan, hydrocarbon no, hydrocarbon không no, phản ứng thế}. Hydrocarbon mà phân tử chỉ có liên kết đơn được gọi là $\ldots$. Hydrocarbon no có mạch không vòng được gọi là $\ldots$. Hydrocarbon no có 1 mạch vòng được gọi là $\ldots$. Tính chất hóa học đặc trưng của hydrocarbon no là $\ldots$.
\end{baitoan}

\begin{baitoan}[\cite{SBT_Hoa_Hoc_11_co_ban}, \textbf{5.2}, p. 35]
	Nhận xét nào sai? {\sf A.} Tất cả các ankan đều có công thức phân tử \emph{\ce{C_nH_{2n+2}}}. {\sf B.} Tất cả các chất có công thức phân tử \emph{\ce{C_nH_{2n+2}}} đều là ankan. {\sf C.} Tất cả các ankan đều chỉ có liên kết đơn trong phân tử. {\sf D.} Tất cả các chất chỉ có liên kết đơn trong phân tử đều là ankan.
\end{baitoan}
\texttt{Insert \cite[\textbf{5.3.}, \textbf{5.4.}, pp. 35--36]{SBT_Hoa_Hoc_11_co_ban} by using chemfig ...}

\begin{baitoan}[\cite{SBT_Hoa_Hoc_11_co_ban}, \textbf{5.5}, p. 36]
	Tổng số liên kết cộng hóa trị trong 1 phân tử \emph{\ce{C3H8}}? {\sf A.} $11$. {\sf B.} $10$. {\sf C.} $3$ {\sf D.} $8$.
\end{baitoan}

\begin{baitoan}[Mở rộng \cite{SBT_Hoa_Hoc_11_co_ban}, \textbf{5.5}, p. 36]
	Tổng số liên kết cộng hóa trị trong 1 phân tử \emph{\ce{C_nH_{2n+2}}} là bao nhiêu?
\end{baitoan}

\begin{baitoan}[\cite{SBT_Hoa_Hoc_11_co_ban}, \textbf{5.6}, p. 36]
	2 chất $2$-metylpropan \& butan khác nhau về: {\sf A.} công thức cấu tạo. {\sf B.} công thức phân tử. {\sf C.} số nguyên tử carbon. {\sf D.} số liên kết cộng hóa trị.
\end{baitoan}

\begin{baitoan}[\cite{SBT_Hoa_Hoc_11_co_ban}, \textbf{5.7}, p. 36]
	Tất cả các ankan có cùng công thức gì? {\sf A.} Công thức đơn giản nhất. {\sf B.} Công thức chung. {\sf C.} Công thức cấu tạo. {\sf D.} Công thức phân tử.
\end{baitoan}

\begin{baitoan}[\cite{SBT_Hoa_Hoc_11_co_ban}, \textbf{5.8}, p. 36]
	Trong các chất dưới đây, chất nào có nhiệt độ sôi thấp nhất? {\sf A.} Butan. {\sf B.} Etan. {\sf C.} Metan. {\sf D.} Propan.
\end{baitoan}

\begin{baitoan}[\cite{SBT_Hoa_Hoc_11_co_ban}, \textbf{5.9}, p. 36]
	Gọi tên IUPAC của các ankan có công thức: (a) \emph{\ce{(CH3)2CH-CH2-C(CH3)3}} (tên thông dụng là \emph{isooctan}). (b) \emph{\ce{CH3-CH2-CH(CH3)-CH(CH3)-[CH2]4-CH(CH3)2}}.
\end{baitoan}

\begin{baitoan}[\cite{SBT_Hoa_Hoc_11_co_ban}, \textbf{5.10}, p. 36]
	Viết công thức cấu tạo thu gọn của: (a) $4$-etyl-$2,3,3$-trimetylheptan. (b) $3,5$-đietyl-$2,2,3$-trimetyloctan.
\end{baitoan}

\begin{baitoan}[\cite{SBT_Hoa_Hoc_11_co_ban}, \textbf{5.11}, p. 37]
	Cho A là 1 ankan thể khí. Để đốt cháy hoàn toàn $1.2$\emph{l} A cần dùng vừa hết $6$\emph{l} oxi lấy ở cùng điều kiện. (a) Xác định CTPT chất A. (b) Cho chất A tác dụng với khí clo ở $25^\circ$ \& có ánh sáng. Hỏi có thể thu được mấy dẫn xuất monoclo của A? Cho biết tên của mỗi dẫn xuất đó. Dẫn xuất nào thu được nhiều hơn?
\end{baitoan}

\begin{baitoan}[\cite{SBT_Hoa_Hoc_11_co_ban}, \textbf{5.12}, p. 37]
	Để đốt cháy hoàn toàn $1.45$\emph{g} 1 ankan phải dùng vừa hết $3.64$\emph{l} \emph{\ce{O2}} (đktc). (a) Xác định CTPT của ankan đó. (b) Viết CTCT các đồng phân ứng với CTPT đó. Ghi tên tương ứng.
\end{baitoan}

\begin{baitoan}[\cite{SBT_Hoa_Hoc_11_co_ban}, \textbf{5.13}, p. 37]
	Khi đốt cháy hoàn toàn $1.8$\emph{g} 1 ankan, người ta thấy trong sản phẩm tạo thành khối lượng \emph{\ce{CO2}} nhiều hơn khối lượng \emph{\ce{H2O}} là $2.8$\emph{g}. (a) Xác định CTPT của ankan mang đốt. (b) Viết CTCT \& tên tất cả các đồng phân ứng với CTPT đó.
\end{baitoan}

\begin{baitoan}[\cite{SBT_Hoa_Hoc_11_co_ban}, \textbf{5.14}, p. 37]
	Đốt cháy hoàn toàn $2.86$\emph{g} hỗn hợp gồm hexan \& octan người ta thu được $4.48$\emph{l} khí \emph{\ce{CO2}} (đktc). (a) Xác định \% về khối lượng của từng chất trong hỗn hợp ankan mang đốt.
\end{baitoan}

\begin{baitoan}[\cite{SBT_Hoa_Hoc_11_co_ban}, \textbf{5.15}, p. 37]
	1 loại xăng là hỗn hợp của các ankan \& có CTPT là \emph{\ce{C7H_{16}}} \& \emph{\ce{C8H_{18}}}. Để đốt cháy hoàn toàn $6.950$\emph{g} xăng đó phải dùng vừa hết $17.08$\emph{l} khí \emph{\ce{O2}} (đktc). Xác định \% về khối lượng của từng chất trong loại xăng đó. 
\end{baitoan}

\begin{baitoan}[\cite{SBT_Hoa_Hoc_11_co_ban}, \textbf{5.16}, p. 37]
	Hỗn hợp M chứa 2 ankan kế tiếp nhau trong dãy đồng đẳng. Để đốt cháy hoàn toàn $22.2$\emph{g} M cần dùng vừa hết $54.88$\emph{l} \emph{\ce{O2}} (đktc). Xác định CTPT \& \% về khối lượng của từng chất trong hỗn hợp M.
\end{baitoan}

\begin{baitoan}[\cite{SBT_Hoa_Hoc_11_co_ban}, \textbf{5.17}, p. 38]
	Hỗn hợp X chứa ancol etylic \emph{\ce{C2H5OH}} \& 2 ankan kế tiếp nhau trong dãy đồng đẳng. Khi đốt cháy hoàn toàn $18.9$\emph{g} X, thu được $26.1$\emph{g \ce{H2O}} \& $26.88$\emph{l \ce{CO2}} (đktc). Xác định CTPT \& \% về khối lượng của từng ankan trong hỗn hợp X.
\end{baitoan}

\begin{baitoan}[\cite{SBT_Hoa_Hoc_11_co_ban}, \textbf{5.25}, p. 39]
	Tìm nhận xét đúng: {\bf A.} Tất cả ankan \& tất cả xicloankan đều không tham gia phản ứng cộng. {\bf B.} Tất cả ankan \& tất cả xicloankan đều có thể tham gia phản ứng cộng. {\bf C.} Tất cả ankan không tham gia phản ứng cộng nhưng 1 số xicloankan lại có thể tham gia phản ứng cộng. {\bf D.} 1 số ankan có thể tham gia phản ứng cộng \& tất cả xicloankan không thể tham gia phản ứng cộng.
\end{baitoan}

\begin{baitoan}[\cite{SBT_Hoa_Hoc_11_co_ban}, \textbf{5.26}, p. 40]
	Các ankann không tham gia loại phản ứng nào? {\bf A.} Phản ứng thế. {\bf B.} Phản ứng cộng. {\bf C.} Phản ứng tách. {\bf D.} Phản ứng cháy.
\end{baitoan}

\begin{baitoan}[\cite{SBT_Hoa_Hoc_11_co_ban}, \textbf{5.27}, p. 40]
	Cho clo tác dụng với butan, thu được 2 dẫn xuất monoclo \emph{\ce{C4H8Cl}}. (a) Dùng CTCT viết PTHH, ghi tên các sản phẩm. (b) Tính \% của mỗi sản phẩm, biết nguyên tử hydro liên kết với carbon bậc 2 có khả năng bị thế cao hơn $3$ lần so với nguyên tử hydro liên kết với carbon bậc 1.
\end{baitoan}

\begin{baitoan}[\cite{SBT_Hoa_Hoc_11_co_ban}, \textbf{5.28}, p. 40]
	Hỗn hợp M ở thể lỏng, chứa 2 ankan. Để đốt cháy hoàn toàn hỗn hợp M cần dùng vừa hết $63.28$\emph{l} không khí (đktc). Hấp thụ hết sản phẩm cháy vào dung dịch \emph{\ce{Ca(OH)2}} lấy dư, thu được $36$\emph{g} chất kết tủa. (a) Tính khối lượng hỗn hợp M biết oxi chiếm $20$\% thể tích không khí. (b) Xác định CTPT \& \% khối lượng của từng chất trong hỗn hợp M nếu biết thêm 2 ankan khác nhau 2 nguyên tử carbon.
\end{baitoan}

\begin{baitoan}[\cite{SBT_Hoa_Hoc_11_co_ban}, \textbf{5.29}${}^\star$, p. 40]
	1 bình kín dung tích $11.2$\emph{l} có chứa $6.4$\emph{g} \emph{\ce{O2}} \& $1.36$\emph{g} hỗn hợp khí A gồm 2 ankan. Nhiệt độ trong bình là $0^\circ$C \& áp suất là $p_1$\emph{atm}. Bật tia lửa điện trong bình đó thì hỗn hợp A cháy hoàn toàn. Sau phản ứng, nhiệt độ trong bình là $136.5^\circ$C \& áp suất là $p_2$\emph{atm}. Nếu dẫn các chất trong bình sau phản ứng vào dung dịch \emph{\ce{Ca(OH)2}} lấy dư thì có $9$\emph{g} kết tủa tạo thành. (a) Tính $p_1,p_2$ biết thể tích bình không đổi. (b) Xác định CTPT \& \% thể tích từng chất trong hỗn hợp A, biết số mol của ankan có phân tử khối nhỏ nhiều gấp $1.5$ lần số mol của ankan có phân tử khối lớn.
\end{baitoan}

\begin{baitoan}[\cite{SBT_Hoa_Hoc_11_co_ban}, \textbf{5.30}${}^\star$, p. 40]
	Chất A có CTPT \emph{\ce{C6H_{14}}}. Khi A tác dụng với clo, có thể tạo ra tối đa $3$ dẫn xuất monoclo \emph{\ce{C6H_{13}Cl}} \& $7$ dẫn xuất điclo \emph{\ce{C6H_{12}Cl2}}. Viết CTCT của A \& của các dẫn xuất monoclo, điclo của A.
\end{baitoan}

%------------------------------------------------------------------------------%

\subsection{Xicloankan}

\begin{baitoan}[\cite{SBT_Hoa_Hoc_11_co_ban}, \textbf{5.18}, p. 38]
	\emph{Đ\texttt{/}S?} (a) Các monoxicloankan đều có công thức phân tử \emph{\ce{C_nH_{2n}}}. (b) Các chất có công thức phân tử \emph{\ce{C_nH_{2n}}} đều là monoxicloankan. (c) Các xicloankan đều chỉ có liên kết đơn. (d) Các chất chỉ có liên kết đơn đều là xicloankan.
\end{baitoan}
\texttt{Insert \cite{SBT_Hoa_Hoc_11_co_ban}, \textbf{5.18}, p. 38 ...}

\begin{baitoan}[\cite{SBT_Hoa_Hoc_11_co_ban}, \textbf{5.20}, p. 38]
	Tìm nhận xét đúng: {\bf A.} Xiclohexan vừa có phản ứng thế, vừa có phản ứng cộng. {\bf B.} Xiclohexan không có phản ứng thế, không có phản ứng cộng. {\bf C.} Xiclohexan có phản ứng thế, không có phản ứng cộng. {\bf D.} Xiclohexan không có phản ứng thế, có phản ứng cộng.
\end{baitoan}

\begin{baitoan}[\cite{SBT_Hoa_Hoc_11_co_ban}, \textbf{5.21}, p. 39]
	Viết CTCT của: (a) $1,1$-đimetylxiclopropan; (b) $1$-etyl-$1$-metylxiclohexan; (c) $1$-metyl-$4$-isopropylxiclohexan.
\end{baitoan}

\begin{baitoan}[\cite{SBT_Hoa_Hoc_11_co_ban}, \textbf{5.22}, p. 39]
	1 monoxicloankan có tỷ khối hơi so với nitơ bằng $3$. (a) Xác định CTPT của xicloankan đó. (b) Viết CTCT \& tên tất cả các xicloankan ứng với CTPT tìm được.
\end{baitoan}

\begin{baitoan}[\cite{SBT_Hoa_Hoc_11_co_ban}, \textbf{5.23}, p. 39]
	Hỗn hợp khí A chứa 1 ankan \& 1 monoxicloankan. Tỷ khối của A đối với hydro là $25.8$. Đốt cháy hoàn toàn $2.58$\emph{g} A rồi hấp thụ hết sản phẩm cháy vào dung dịch \emph{\ce{Ba(OH)2}} dư, thu được $35.46$\emph{g} kết tủa. Xác định CTPT \& \% thể tích của từng chất trong hỗn hợp khí A.
\end{baitoan}

\begin{baitoan}[\cite{SBT_Hoa_Hoc_11_co_ban}, \textbf{5.24}, p. 39]
	Chất khí A là 1 xicloankan. Khi đốt cháy $672$\emph{ml} A (đktc), thì thấy khối lượng \emph{\ce{CO2}} tạo thành nhiều hơn khối lượng nước tạo thành $3.12$\emph{g}. (a) Xác định CTPT chất A. (b) Viết CTCT \& tên các xicloankan ứng với CTPT tìm được. (c) Cho chất A qua dung dịch brom, màu của dung dịch mất đi. Xác định CTCT đúng của chất A.
\end{baitoan}

%------------------------------------------------------------------------------%

\section{Hydrocarbon Không No}

\subsection{Anken}

%------------------------------------------------------------------------------%

\subsection{Ankađien}

%------------------------------------------------------------------------------%

\subsection{Ankin}

%------------------------------------------------------------------------------%

\section{Hydrocarbon Thơm. Nguồn Hydrocarbon Thiên Nhiên. Hệ Thống Hóa về Hydrocarbon}

\subsection{Benzen \& Đồng Đẳng. 1 Số Hydrocarbon Thơm Khác}

%------------------------------------------------------------------------------%

\subsection{Hydrocarbon Thơm}

%------------------------------------------------------------------------------%

\subsection{Nguồn Hydrocarbon Thiên Nhiên}

%------------------------------------------------------------------------------%

\subsection{Hệ Thống Hóa về Hydrocarbon}

%------------------------------------------------------------------------------%

\printbibliography[heading=bibintoc]
	
\end{document}