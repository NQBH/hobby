\documentclass{article}
\usepackage[backend=biber,natbib=true,style=authoryear]{biblatex}
\addbibresource{/home/nqbh/reference/bib.bib}
\usepackage[utf8]{vietnam}
\usepackage{tocloft}
\renewcommand{\cftsecleader}{\cftdotfill{\cftdotsep}}
\usepackage[colorlinks=true,linkcolor=blue,urlcolor=red,citecolor=magenta]{hyperref}
\usepackage{amsmath,amssymb,amsthm,mathtools,float,graphicx,algpseudocode,algorithm,tcolorbox,tikz,tkz-tab,subcaption}
\DeclareMathOperator{\arccot}{arccot}
\usepackage[inline]{enumitem}
\usepackage[version=4]{mhchem}
\allowdisplaybreaks
\numberwithin{equation}{section}
\newtheorem{assumption}{Assumption}[section]
\newtheorem{baitoan}{Bài toán}
\newtheorem{cauhoi}{Câu hỏi}[section]
\newtheorem{conjecture}{Conjecture}[section]
\newtheorem{corollary}{Corollary}[section]
\newtheorem{definition}{Definition}[section]
\newtheorem{dinhly}{Định lý}[section]
\newtheorem{dinhnghia}{Định nghĩa}[section]
\newtheorem{example}{Example}[section]
\newtheorem{hequa}{Hệ quả}[section]
\newtheorem{lemma}{Lemma}[section]
\newtheorem{luuy}{Lưu ý}[section]
\newtheorem{notation}{Notation}[section]
\newtheorem{principle}{Principle}[section]
\newtheorem{problem}{Problem}[section]
\newtheorem{proposition}{Proposition}[section]
\newtheorem{question}{Question}[section]
\newtheorem{remark}{Remark}[section]
\newtheorem{theorem}{Theorem}[section]
\newtheorem{vidu}{Ví dụ}[section]
\usepackage[left=0.5in,right=0.5in,top=1.5cm,bottom=1.5cm]{geometry}
\usepackage{fancyhdr}
\pagestyle{fancy}
\fancyhf{}
\lhead{\small Sect.~\thesection}
\rhead{\small\nouppercase{\leftmark}}
\renewcommand{\subsectionmark}[1]{\markboth{#1}{}}
\cfoot{\thepage}
\def\labelitemii{$\circ$}

\title{Nitrogen -- Nitơ}
\author{Nguyễn Quản Bá Hồng\footnote{Independent Researcher, Ben Tre City, Vietnam\\e-mail: \texttt{nguyenquanbahong@gmail.com}; website: \url{https://nqbh.github.io}.}}
\date{\today}

\begin{document}
\maketitle
\begin{abstract}
	
\end{abstract}
\setcounter{secnumdepth}{4}
\setcounter{tocdepth}{3}
\tableofcontents

%------------------------------------------------------------------------------%

\section{Theory -- Lý Thuyết}
\cite[pp. 38--]{An_Hoa_Hoc_nang_cao_11_2020}

\section{Problem}

\begin{baitoan}[\cite{An_400_BT_Hoa_Hoc_11_2021}, \textbf{40.}, p. 19]
	(a) Cân bằng các phương trình hóa học của phản ứng giữa kim loại $M$ hóa trị $n$ với dung dịch \emph{\ce{HNO3}} thu được sản phẩm là muối nitrat, nước, \& 1 trong các chất \emph{\ce{NO,N2O,NH4NO3}}. (b) Cho 1 miếng \emph{Al} vào dung dịch chứa \emph{\ce{NaOH,NaNO3}}. Viết các phương trình hóa học dưới dạng phân tử \& dạng ion.
\end{baitoan}

\begin{proof}[Giải]
	(a) \ce{3M + $4n$HNO3 -> 3M(NO3)_n + $n$NO + $2n$H2O}, \ce{8M + $10n$HNO3 -> 8M(NO3)_n + $n$N2O + $5n$H2O}, \ce{8M + $10n$HNO3 -> 8M(NO3)_n + $n$NH4NO3 + $3n$H2O}. (b) \ce{8Al + 5NaOH + 3NaNO3 + 2H2O -> 8NaAlO2 + 3NH3 ^}, \ce{8Al + 5OH- + 3NO3- + 2H2O -> 8AlO2- + 3NH3 ^}, \ce{2Al + 2NaOH} (nếu dư) \ce{+ 2H2O -> 2NaAlO2- + 3H2 ^}. \ce{2Al + 2OH- + 2H2O -> 2AlO2- + 3H2 ^}.
\end{proof}

\begin{luuy} 
	\begin{itemize}
		\item (a) Gốc \emph{\ce{NO3-}} trong môi trường acid có khả năng oxi hóa như \emph{\ce{HNO3}}. Gốc \emph{\ce{NO3^-}} trong môi trường trung tính không có khả năng oxi hóa. Gốc \emph{\ce{NO3-}} trong môi trường kiềm có thể bị \emph{Zn, Al} khử đến \emph{\ce{NH3}}.
		\item[(b)] Cho \emph{Zn} vào  dung dịch chứa \emph{\ce{NaOH,NaNO3}}: \emph{\ce{4Zn + NaNO3 + 7NaOH + 6H2O -> 4Na2[(Zn(OH)4] + NH3 ^}, \ce{Zn + 2NaOH} (nếu dư) \ce{-> Na2ZnO2 + H2 ^}}.
	\end{itemize}	 
\end{luuy}

\begin{baitoan}[\cite{An_400_BT_Hoa_Hoc_11_2021}, \textbf{58.}, p. 21, TS ĐHQG TPHCM 1999]
	Cho hỗn hợp A gồm 3 kim loại X, Y, Z có hóa trị lần lượt là III, II, I \& tỷ lệ mol lần lượt là $1:2:3$, trong đó số mol của X bằng $x$ mol. Hòa tan hoàn toàn A bằng dung dịch có chứa $y$ gam \emph{\ce{HNO3}} (lấy dư $25$\%). Sau phản ứng thu được dung dịch B không chứa \emph{\ce{NH4NO3}} \& $V$\emph{l} hỗn hợp khí G (đktc) gồm \emph{\ce{NO2,NO}}. Lập biểu thức tính $y$ theo $x$ \& $V$.
\end{baitoan}

\begin{proof}[Giải]
	$n_{\rm X}:n_{\rm Y}:n_{\rm Z} = 1:2:3$, $n_{\rm X} = x$ mol, suy ra $n_{\rm Y} = 2x$ mol, $n_{\rm Z} = 3x$ mol. \ce{X + 6HNO3 -> X(NO3)3 + 3NO2 ^ + 3H2O}, \ce{Y + 4HNO3 -> Y(NO3)2 + 2NO2 ^ + 2H2O}, \ce{Z + 2HNO3 -> ZNO3 + NO2 ^ + H2O}, \ce{X + 4HNO3 -> X(NO3)2 + 2NO ^ + 4H2O}, \ce{3Y + 8HNO3 -> 3Y(NO3)2 + 2NO ^ + 4H2O}, \ce{3Z + 4HNO3 -> 3ZNO3 + NO ^ + 2H2O}.
	
	Có $\sum n_{\scriptsize\ce{HNO3}\mbox{pư}} = \sum n_{\scriptsize\rm N\mbox{ in muối}} + \sum n_{\scriptsize\ce{N}\mbox{ in }G} = 3n_{\rm X} + 2n_{\rm Y} + n_{\rm Z} + n_{\scriptsize\ce{N}\mbox{ in }\ce{NO2}} + n_{\scriptsize\ce{N}\mbox{ in }\ce{NO}} = 3x + 2\cdot2x + 3x + \frac{V}{22.4}\Rightarrow y = M_{\ce{HNO3}}\sum n_{\scriptsize\ce{HNO3}\mbox{pư}}\cdot125\% = 1.25\cdot63\left(10x + \frac{V}{22.4}\right) = 787.5x + \frac{225}{64}V$.
\end{proof}

\begin{baitoan}[\cite{An_400_BT_Hoa_Hoc_11_2021}, \textbf{59.}, p. 21]
	Trộn \emph{\ce{CuO}} với oxit kim loại M hóa trị II theo tỷ lệ số mol tương ứng là $1:2$ được hỗn hợp B. Cho $4.8$\emph{g} hỗn hợp B vào 1 ống sứ, nung nóng rồi cho 1 dòng khí \emph{CO} đi qua đến khi phản ứng xảy ra hoàn toàn thu được chất rắn D. Hỗn hợp D tác dụng vừa đủ với $160$\emph{ml} dung dịch \emph{\ce{HNO3}} $1.25$\emph{M} thu được $V$\emph{l} khí \emph{NO} (đktc). Tính $V$.
\end{baitoan}

\begin{proof}[Giải]
	Đặt $x\coloneqq n_{\ce{CuO}}$, thì $n_{\ce{MO}} = 2x$. $80x + 2x(M + 16) = 4.8\Rightarrow x(M + 56) = 2.4$. $n_{\ce{HNO3}} = 0.16\cdot1.25 = 0.2$ mol. \ce{CuO + CO ->[$t^\circ$] Cu + CO2 ^}, \ce{MO + CO ->[$t^\circ$] M + CO2 ^}.
\end{proof}

%------------------------------------------------------------------------------%

\subsection{Dựa vào cấu hình electron xác định số oxi hóa \& tính chất hóa học của nitơ, photpho \& các hợp chất của chúng}

\begin{baitoan}[\cite{An_Hoa_Hoc_nang_cao_11_2020}, \textbf{1.}, p. 46, TS ĐHCĐ khối B 2002]
	\begin{enumerate*}
		\item[(a)] Phân nhóm chính nhóm V của hệ thống tuần hoàn gồm những nguyên tố nào? Viết cấu hình electron của 2 nguyên tố đầu tiên là \emph{N,P}. Từ đó giải thích tại sao \emph{N} chỉ cho hợp chất có hóa trị 3 trong khi \emph{P} có thể có hóa trị 3 \& 5.
		\item[(b)] Chỉ dùng 1 hóa chất, cho biết cách phân biệt \emph{\ce{Fe2O3,Fe3O4}}. Viết các phương trình phản ứng.
	\end{enumerate*}
\end{baitoan}

\begin{baitoan}[\cite{An_Hoa_Hoc_nang_cao_11_2020}, \textbf{2.}, p. 47]
	Giải thích vì sao độ âm điện của \emph{N} \& \emph{Cl} đều bằng $3$, nhưng ở nhiệt độ thường \emph{\ce{N2}} có tính oxi hóa kém hơn \emph{\ce{Cl2}}? Khi nào \emph{\ce{N2}} trở nên hoạt động hơn?
\end{baitoan}

\begin{baitoan}[\cite{An_Hoa_Hoc_nang_cao_11_2020}, \textbf{3.}, p. 47]
	2 nguyên tố A \& B ở 2 phân nhóm chính liên tiếp trong bảng tuần hoàn. B thuộc nhóm V. Ở trạng thái đơn chất A \& B không phản ứng với nhau. Tổng số proton trong hạt nhân nguyên tử của A \& B là $23$.
	\begin{enumerate*}
		\item[(a)] Viết cấu hình electron của A \& B.
		\item[(b)] Từ các đơn chất A, B \& các hóa chất cần thiết, viết các phương trình phản ứng điều chế 2 acid trong đó A \& B có số oxi hóa dương cao nhất.
	\end{enumerate*}
\end{baitoan}

\begin{baitoan}[\cite{An_Hoa_Hoc_nang_cao_11_2020}, \textbf{4.}, p. 48]
	\begin{enumerate*}
		\item[(a)] Tổng số hạt proton, neutron, electron của nguyên tử 1 nguyên tố là $21$.
		\begin{enumerate*}
			\item[(1)] Xác định tên nguyên tố đó.
			\item[(2)] Viết cấu hình electron nguyên tử của nguyên tố đó.
			\item[(3)] Tính tổng số obital trong nguyên tử của nguyên tố đó. (TS ĐHYD 1998)
		\end{enumerate*}
		\item[(b)] Vì sao nitơ là khí tương đối trơ ở nhiệt độ thường.
	\end{enumerate*}
\end{baitoan}

%------------------------------------------------------------------------------%

\subsection{Hoàn thành các phương trình phản ứng, viết các phương trình phản ứng dạng phân tử \& ion rút gọn}

%------------------------------------------------------------------------------%

\subsection{Vận dụng nguyên lý chuyển dịch cân bằng Le Chatelier trong phản ứng thuận nghịch}

%------------------------------------------------------------------------------%

\subsection{Biện luận xác định công thức phân tử \& ion có trong dung dịch}

%------------------------------------------------------------------------------%

\subsection{Nhận biết, điều chế, \& tinh chế các chất}

%------------------------------------------------------------------------------%

\subsection{Thành phần hỗn hợp khí \& áp suất}

%------------------------------------------------------------------------------%

\subsection{Tính hiệu suất phản ứng, nồng độ dung dịch các chất, hằng số cân bằng}

%------------------------------------------------------------------------------%

\subsection{Xác định công thức phân tử \& tên nguyên tố}

%------------------------------------------------------------------------------%

\subsection{Tính khối lượng chất tham gia phản ứng}

%------------------------------------------------------------------------------%

\printbibliography[heading=bibintoc]
	
\end{document}