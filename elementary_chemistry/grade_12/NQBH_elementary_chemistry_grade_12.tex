\documentclass{article}
\usepackage[backend=biber,natbib=true,style=authoryear]{biblatex}
\addbibresource{/home/hong/1_NQBH/reference/bib.bib}
\usepackage[utf8]{vietnam}
\usepackage{tocloft}
\renewcommand{\cftsecleader}{\cftdotfill{\cftdotsep}}
\usepackage[colorlinks=true,linkcolor=blue,urlcolor=red,citecolor=magenta]{hyperref}
\usepackage{amsmath,amssymb,amsthm,mathtools,float,graphicx,algpseudocode,algorithm,tcolorbox,tikz,tkz-tab,subcaption,chemfig}
\DeclareMathOperator{\arccot}{arccot}
\usepackage[inline]{enumitem}
\usepackage[version=4]{mhchem}
\allowdisplaybreaks
\numberwithin{equation}{section}
\newtheorem{assumption}{Assumption}[section]
\newtheorem{nhanxet}{Nhận xét}[section]
\newtheorem{conjecture}{Conjecture}[section]
\newtheorem{corollary}{Corollary}[section]
\newtheorem{hequa}{Hệ quả}[section]
\newtheorem{definition}{Definition}[section]
\newtheorem{dinhnghia}{Định nghĩa}[section]
\newtheorem{example}{Example}[section]
\newtheorem{vidu}{Ví dụ}[section]
\newtheorem{lemma}{Lemma}[section]
\newtheorem{notation}{Notation}[section]
\newtheorem{principle}{Principle}[section]
\newtheorem{problem}{Problem}[section]
\newtheorem{baitoan}{Bài toán}[section]
\newtheorem{proposition}{Proposition}[section]
\newtheorem{menhde}{Mệnh đề}[section]
\newtheorem{question}{Question}[section]
\newtheorem{cauhoi}{Câu hỏi}[section]
\newtheorem{quytac}{Quy tắc}
\newtheorem{remark}{Remark}[section]
\newtheorem{luuy}{Lưu ý}[section]
\newtheorem{theorem}{Theorem}[section]
\newtheorem{tiende}{Tiên đề}[section]
\newtheorem{dinhly}{Định lý}[section]
\usepackage[left=0.5in,right=0.5in,top=1.5cm,bottom=1.5cm]{geometry}
\usepackage{fancyhdr}
\pagestyle{fancy}
\fancyhf{}
\lhead{\small Subsect.~\thesubsection}
\rhead{\small \nouppercase{\leftmark}}
\renewcommand{\sectionmark}[1]{\markboth{#1}{}}
\cfoot{\thepage}
\def\labelitemii{$\circ$}

\title{Some Topics in Elementary Chemistry\texttt{/}Grade 12}
\author{Nguyễn Quản Bá Hồng\footnote{Independent Researcher, Ben Tre City, Vietnam\\e-mail: \texttt{nguyenquanbahong@gmail.com}; website: \url{https://nqbh.github.io}.}}
\date{\today}

\begin{document}
\maketitle
\begin{abstract}
	
\end{abstract}
\setcounter{secnumdepth}{4}
\setcounter{tocdepth}{3}
\tableofcontents
\newpage

%------------------------------------------------------------------------------%

\section{Este -- Lipit}
\textsf{\textbf{Nội dung.} Cấu tạo, tính chất của este \& lipit; phản ứng xà phòng hóa; xà phòng \& các chất giặt rửa tổng hợp.}

\subsection{Este}
\textsf{\textbf{Nội dung.} Công thức cấu tạo của este \& 1 vài dẫn xuất của axit cacboxylic; tính chất vật lý\texttt{/}hóa học \& ứng dụng của este.}

\subsubsection{Khái niệm về este \& dẫn xuất khác của axit cacboxylic}

\paragraph{Cấu tạo phân tử.} ``Khi thay nhóm OH ở nhóm cacboxyl của axit cacboxylic bằng nhóm OR thì được este. Este đơn giản có công thức cấu tạo: \chemfig[atom sep=2em]{R-C(=[6]O)-O-R'} với R, R' là gốc hydrocarbon no, không no hoặc thơm (trừ trường hợp este của axit fomic có R là H). Este là dẫn xuất của axit cacboxylic. 1 vài dẫn xuất khác của axit cacboxylic có công thức cấu tạo như sau: anhiđrit axit, halogenua axit, amit.'' -- \cite[p. 4]{SGK_Hoa_Hoc_12_nang_cao}
\begin{center}
	\chemfig[atom sep=2em]{R-C(=[6]O)-O-C(=[6]O)-R'},\ \chemfig[atom sep=2em]{R-C(=[6]O)-X},\ \chemfig[atom sep=2em]{R-C(=[6]O)-NR_2'}
\end{center}

\paragraph{Cách gọi tên este.} ``Tên este gồm: tên gốc hydrocarbon R' $+$ tên anion gốc axit (đuôi ``at''). etyl fomat, vinyl axetat, metyl benzoat, benzyl axetat.'' -- \cite[p. 4]{SGK_Hoa_Hoc_12_nang_cao}
\begin{center}
	\chemfig[atom sep=2em]{H-C(=[6]O)-O-C_2H_5},\ \chemfig[atom sep=2em]{CH_3-C(=[6]O)-O-CH=CH_2},\ \chemfig[atom sep=2em]{C_6H_5-C(=[6]O)-O-CH_3},\ \chemfig[atom sep=2em]{CH_3-C(=[6]O)-O-CH_2C_6H_5}
\end{center}

\paragraph{Tính chất vật lý của este.} ``Giữa các phân tử este không có liên kết hydro vì thế este có nhiệt độ sôi thấp hơn so với axit \& ancol có cùng số nguyên tử C. Các este thường là những chất lỏng, nhẹ hơn nước, rất ít tan trong nước, có khả năng hòa tan được nhiều chất hữu cơ khác nhau. Những este có khối lượng phân tử rất lớn có thể ở trạng thái rắn (như mỡ động vật, sáp ong, $\ldots$). Các este thường có mùi thơm dễ chịu, e.g., isoamyl axetat có mùi chuối chín, etyl butirat có mùi dứa, etyl isovalerat có mùi áo, $\ldots$'' -- \cite[pp. 4--5]{SGK_Hoa_Hoc_12_nang_cao}

\subsubsection{Tính chất hóa học của este}

\paragraph{Phản ứng ở nhóm chức.}

\subparagraph{Phản ứng thủy phân.} ``Este bị thủy phân cả trong môi trường axit \& môi trường kiềm. Thủy phân este trong môi trường axit là phản ứng nghịch với phản ứng este hóa: \ce{R-COO-{R'} + H-OH <-->[H2SO4,$t^\circ$] R-COOH + {R'}-OH}. Thủy phân este trong môi trường kiềm là phản ứng 1 chiều \& còn được gọi là \textit{phản ứng xà phòng hóa}: \ce{R-COO-{R'} + NaOH ->[H2O,$t^\circ$] R-COONa + {R'}-OH}.'' -- \cite[pp. 5]{SGK_Hoa_Hoc_12_nang_cao}

\subparagraph{Phản ứng khử.} ``Este bị khử bởi liti nhôm hiđrua (\ce{LiAlH4}), khi đó nhóm \chemfig[atom sep=2em]{R-C(=[6]O)-} (gọi là \textit{nhóm axyl}) trở thành ancol bậc I: \ce{R-COO-{R'} ->[LiAlH4,$t^\circ$] R-CH2-OH + {R'}-OH}.'' -- \cite[pp. 5]{SGK_Hoa_Hoc_12_nang_cao}

\paragraph{Phản ứng ở gốc hydrocarbon.} ``Este có thể tham gia phản ứng thế, cộng, tách, trùng hợp, $\ldots$ Sau đây chỉ xét phản ứng cộng \& phản ứng trùng hợp.'' -- \cite[pp. 5]{SGK_Hoa_Hoc_12_nang_cao}

\subparagraph{Phản ứng cộng vào gốc không no.} ``Gốc hydrocarbon không no ở este có phản ứng cộng với \ce{H2,Br2,Cl2}, $\ldots$ giống như hydrocarbon không no. E.g., \ce{CH3[CH2]7CH=CH[CH2]7COOCH3 (metyl oleat) + H2 ->[Ni,$t^\circ$] CH3[CH2]16COOCH3 (metyl stearat)}.'' -- \cite[pp. 5]{SGK_Hoa_Hoc_12_nang_cao}

\subparagraph{Phản ứng trùng hợp.} ``1 số este đơn giản có liên kết \ce{C=C} tham gia phản ứng trùng hợp giống như anken. E.g.,
\setchemfig{scheme debug=false}
\schemestart\chemfig[atom sep=2em]{nCH_2=CH-C(=[6]O)-O-CH_3} (metyl acrylat)\ce{->[xt,$t^\circ$]}(\chemfig[atom sep=2em]{-CH(-[6]COOCH3)-CH2-})$_n$ (poly(metyl acrylat))\schemestop

%------------------------------------------------------------------------------%

\subsection{Lipit}

%------------------------------------------------------------------------------%

\subsection{Chất Giặt Rửa}

%------------------------------------------------------------------------------%

\subsection{Mối Liên Hệ Giữa Hydrocarbon \& 1 Số Dẫn Xuất của Hydrocarbon}

%------------------------------------------------------------------------------%

\section{Carbohydrate}

\subsection{Glucose}

%------------------------------------------------------------------------------%

\subsection{Saccarose}

%------------------------------------------------------------------------------%

\subsection{Tinh Bột}

%------------------------------------------------------------------------------%

\subsection{Xenlulozơ}

%------------------------------------------------------------------------------%

\subsection{Cấu Trúc \& Tính Chất của 1 Số Carbohydrate Tiêu Biểu}

%------------------------------------------------------------------------------%

\subsection{Điều Chế Este \& Tính Chất của 1 Số Carbohydrate}

%------------------------------------------------------------------------------%

\section{Amin -- Amino Axit -- Protein}

\subsection{Amin}

%------------------------------------------------------------------------------%

\subsection{Amino Axit}

%------------------------------------------------------------------------------%

\subsection{Peptit \& Protein}

%------------------------------------------------------------------------------%

\subsection{Cấu Tạo \& Tính Chất của Amin, Amino Axit, Protein}

%------------------------------------------------------------------------------%

\subsection{1 Số Tính Chất của Amin, Amino Axit, \& Protein}

%------------------------------------------------------------------------------%

\section{Polyme \& Vật Liệu Polyme}

\subsection{Đại Cương về Polyme}

%------------------------------------------------------------------------------%

\subsection{Vật Liệu Polyme}

%------------------------------------------------------------------------------%

\subsection{Polyme \& Vật Liệu Polyme}

%------------------------------------------------------------------------------%

\section{Đại Cương về Kim Loại}

\subsection{Kim Loại \& Hợp Kim}

%------------------------------------------------------------------------------%

\subsection{Dãy Điện Hóa của Kim Loại}

%------------------------------------------------------------------------------%

\subsection{Tính Chất của Kim Loại}

%------------------------------------------------------------------------------%

\subsection{Sự Điện Phân -- Sự Ăn Mòn Kim Loại -- Điều Chế Kim Loại}

%------------------------------------------------------------------------------%

\subsection{Dãy Điện Hóa của Kim Loại. Điều Chế Kim Loại}

%------------------------------------------------------------------------------%

\subsection{Ăn Mòn Kim Loại. Chống Ăn Mòn Kim Loại}

%------------------------------------------------------------------------------%

\section{Kim Loại Kiềm -- Kim Loại Kiềm Thổ -- Nhôm}

\subsection{Kim Loại Kiềm}

%------------------------------------------------------------------------------%

\subsection{1 Số Hợp Chất Quan Trọng của Kim Loại Kiềm}

%------------------------------------------------------------------------------%

\subsection{Kim Loại Kiềm Thổ}

%------------------------------------------------------------------------------%

\subsection{1 Số Hợp Chất Quan Trọng của Kim Loại Kiềm Thổ}

%------------------------------------------------------------------------------%

\subsection{Tính Chất của Kim Loại Kiềm, Kim Loại Kiềm Thổ}

%------------------------------------------------------------------------------%

\subsection{Nhôm}

%------------------------------------------------------------------------------%

\subsection{1 Số Hợp Chất Quan Trọng của Nhôm}

%------------------------------------------------------------------------------%

\subsection{Tính Chất của Nhôm \& Hợp Chất của Nhôm}

%------------------------------------------------------------------------------%

\subsection{Tính Chất của Kim Loại Kiềm, Kim Loại Kiềm Thổ \& Hợp Chất của Chúng}

%------------------------------------------------------------------------------%

\subsection{Tính Chất của Nhôm \& Hợp Chất của Nhôm}

%------------------------------------------------------------------------------%

\section{Crom -- Sắt -- Đồng}

\subsection{Crom}

%------------------------------------------------------------------------------%

\subsection{1 Số Hợp Chất của Crom}

%------------------------------------------------------------------------------%

\subsection{Sắt}

%------------------------------------------------------------------------------%

\subsection{1 Số Hợp Chất của Sắt}

%------------------------------------------------------------------------------%

\subsection{Hợp Kim của Sắt}

%------------------------------------------------------------------------------%

\subsection{Đồng \& 1 Số Hợp Chất của Đồng}

%------------------------------------------------------------------------------%

\subsection{Sơ Lược về 1 Số Kim Loại Khác}

%------------------------------------------------------------------------------%

\subsection{Tính Chất của Crom, Sắt, \& Những Hợp Chất của Chúng}

%------------------------------------------------------------------------------%

\subsection{Tính Chất của Đồng \& Hợp Chất của Đồng. Sơ Lược về Các Kim Loại Ag, Au, Ni, Zn, Sn, Pb}

%------------------------------------------------------------------------------%

\subsection{Tính Chất Hóa Học của Crom, Sắt, Đồng, \& Những Hợp Chất của Chúng}

%------------------------------------------------------------------------------%

\section{Phân Biệt 1 Số Chất Vô Cơ. Chuẩn Độ Dung Dịch}

\subsection{Nhận Biết 1 Số Cation trong Dung Dịch}

%------------------------------------------------------------------------------%

\subsection{Nhận Biết 1 Số Anion trong Dung Dịch}

%------------------------------------------------------------------------------%

\subsection{Nhận Biết 1 Số Chất Khí}

%------------------------------------------------------------------------------%

\subsection{Chuẩn Độ Acid--Base}

%------------------------------------------------------------------------------%

\subsection{Chuẩn Độ Oxi Hóa--Khử bằng Phương Pháp Pemanganat}

%------------------------------------------------------------------------------%

\subsection{Nhận Biết 1 Số Chất Vô Cơ}

%------------------------------------------------------------------------------%

\subsection{Nhận Biết 1 Số Ion trong Dung Dịch}

%------------------------------------------------------------------------------%

\subsection{Chuẩn Độ Dung Dịch}

%------------------------------------------------------------------------------%

\section{Hóa Học \& Vấn Đề Phát Triển Kinh Tế, Xã Hội, Môi Trường}

\subsection{Hóa Học \& Vấn Đề Phát Triển Kinh Tế}

%------------------------------------------------------------------------------%

\subsection{Hóa Học \& Vấn Đề Xã Hội}

%------------------------------------------------------------------------------%

\subsection{Hóa Học \& Vấn Đề Môi Trường}

%------------------------------------------------------------------------------%

\printbibliography[heading=bibintoc]
	
\end{document}