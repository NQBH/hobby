\documentclass{article}
\usepackage[backend=biber,natbib=true,style=authoryear]{biblatex}
\addbibresource{/home/hong/1_NQBH/reference/bib.bib}
\usepackage[utf8]{vietnam}
\usepackage{tocloft}
\renewcommand{\cftsecleader}{\cftdotfill{\cftdotsep}}
\usepackage[colorlinks=true,linkcolor=blue,urlcolor=red,citecolor=magenta]{hyperref}
\usepackage{amsmath,amssymb,amsthm,mathtools,float,graphicx,algpseudocode,algorithm,tcolorbox}
\usepackage[inline]{enumitem}
\usepackage[version=4]{mhchem}
\allowdisplaybreaks
\numberwithin{equation}{section}
\newtheorem{assumption}{Assumption}[section]
\newtheorem{conjecture}{Conjecture}[section]
\newtheorem{corollary}{Corollary}[section]
\newtheorem{hequa}{Hệ quả}[section]
\newtheorem{definition}{Definition}[section]
\newtheorem{dinhnghia}{Định nghĩa}[section]
\newtheorem{example}{Example}[section]
\newtheorem{vidu}{Ví dụ}[section]
\newtheorem{lemma}{Lemma}[section]
\newtheorem{notation}{Notation}[section]
\newtheorem{principle}{Principle}[section]
\newtheorem{problem}{Problem}[section]
\newtheorem{baitoan}{Bài toán}[section]
\newtheorem{proposition}{Proposition}[section]
\newtheorem{question}{Question}[section]
\newtheorem{cauhoi}{Câu hỏi}[section]
\newtheorem{remark}{Remark}[section]
\newtheorem{luuy}{Lưu ý}[section]
\newtheorem{theorem}{Theorem}[section]
\newtheorem{dinhly}{Định lý}[section]
\usepackage[left=0.5in,right=0.5in,top=1.5cm,bottom=1.5cm]{geometry}
\usepackage{fancyhdr}
\pagestyle{fancy}
\fancyhf{}
\lhead{\small Sect.~\thesection}
\rhead{\small \nouppercase{\leftmark}}
\renewcommand{\sectionmark}[1]{\markboth{#1}{}}
\cfoot{\thepage}
\def\labelitemii{$\circ$}

\title{Problems in Elementary Chemistry\texttt{/}Grade 12}
\author{Nguyễn Quản Bá Hồng\footnote{Independent Researcher, Ben Tre City, Vietnam\\e-mail: \texttt{nguyenquanbahong@gmail.com}; website: \url{https://nqbh.github.io}.}}
\date{\today}

\begin{document}
\maketitle
\begin{abstract}
	1 bộ sưu tập các bài toán chọn lọc từ cơ bản đến nâng cao cho Hóa học sơ cấp lớp 12. Tài liệu này là phần bài tập bổ sung cho tài liệu chính \href{https://github.com/NQBH/hobby/blob/master/elementary_chemistry/grade_12/NQBH_elementary_chemistry_grade_12.pdf}{GitHub\texttt{/}NQBH\texttt{/}hobby\texttt{/}elementary chemistry\texttt{/}grade 12\texttt{/}lecture}\footnote{\textsc{url}: \url{https://github.com/NQBH/hobby/blob/master/elementary_chemistry/grade_12/NQBH_elementary_chemistry_grade_12.pdf}.} của tác giả viết cho Toán lớp 6. Phiên bản mới nhất của tài liệu này được lưu trữ ở link sau: \href{https://github.com/NQBH/hobby/blob/master/elementary_chemistry/grade_12/problem/NQBH_elementary_chemistry_grade_12_problem.pdf}{GitHub\texttt{/}NQBH\texttt{/}hobby\texttt{/}elementary chemistry\texttt{/}grade 12\texttt{/}problem}\footnote{\textsc{url}: \url{https://github.com/NQBH/hobby/blob/master/elementary_chemistry/grade_12/problem/NQBH_elementary_chemistry_grade_12_problem.pdf}.}.
\end{abstract}
\tableofcontents
\newpage

%------------------------------------------------------------------------------%

\section{Este -- Lipit}

\subsection{Este}

\subsubsection{Dựa vào tính chất hóa học để suy luận công thức este}

\begin{baitoan}[\cite{An2008}, \textbf{1.}, p. 5]
	Thủy phân chất X có công thức \ce{C8H14O5} thu được ancol etylic \& chất hữu cơ Y. Cho biết số mol X bằng số mol ancol etylic bằng $\frac{1}{2}$ số mol Y. Y được điều chế trực tiếp từ glucose bằng phản ứng lên men. Trùng ngưng Y thu được 1 polyme. Công thức cấu tạo của X, Y?
\end{baitoan}

%------------------------------------------------------------------------------%

\subsection{Lipit}

%------------------------------------------------------------------------------%

\subsection{Chất Giặt Rửa}

%------------------------------------------------------------------------------%

\subsection{Mối Liên Hệ Giữa Hydrocarbon \& 1 Số Dẫn Xuất của Hydrocarbon}

%------------------------------------------------------------------------------%

\section{Carbohydrate}

\subsection{Glucose}

%------------------------------------------------------------------------------%

\subsection{Saccarose}

%------------------------------------------------------------------------------%

\subsection{Tinh Bột}

%------------------------------------------------------------------------------%

\subsection{Xenlulozơ}

%------------------------------------------------------------------------------%

\subsection{Cấu Trúc \& Tính Chất của 1 Số Carbohydrate Tiêu Biểu}

%------------------------------------------------------------------------------%

\subsection{Điều Chế Este \& Tính Chất của 1 Số Carbohydrate}

%------------------------------------------------------------------------------%

\section{Amin -- Amino Axit -- Protein}

\subsection{Amin}

%------------------------------------------------------------------------------%

\subsection{Amino Axit}

%------------------------------------------------------------------------------%

\subsection{Peptit \& Protein}

%------------------------------------------------------------------------------%

\subsection{Cấu Tạo \& Tính Chất của Amin, Amino Axit, Protein}

%------------------------------------------------------------------------------%

\subsection{1 Số Tính Chất của Amin, Amino Axit, \& Protein}

%------------------------------------------------------------------------------%

\section{Polyme \& Vật Liệu Polyme}

\subsection{Đại Cương về Polyme}

%------------------------------------------------------------------------------%

\subsection{Vật Liệu Polyme}

%------------------------------------------------------------------------------%

\subsection{Polyme \& Vật Liệu Polyme}

%------------------------------------------------------------------------------%

\section{Đại Cương về Kim Loại}

\subsection{Kim Loại \& Hợp Kim}

%------------------------------------------------------------------------------%

\subsection{Dãy Điện Hóa của Kim Loại}

%------------------------------------------------------------------------------%

\subsection{Tính Chất của Kim Loại}

%------------------------------------------------------------------------------%

\subsection{Sự Điện Phân -- Sự Ăn Mòn Kim Loại -- Điều Chế Kim Loại}

%------------------------------------------------------------------------------%

\subsection{Dãy Điện Hóa của Kim Loại. Điều Chế Kim Loại}

%------------------------------------------------------------------------------%

\subsection{Ăn Mòn Kim Loại. Chống Ăn Mòn Kim Loại}

%------------------------------------------------------------------------------%

\section{Kim Loại Kiềm -- Kim Loại Kiềm Thổ -- Nhôm}

\subsection{Kim Loại Kiềm}

%------------------------------------------------------------------------------%

\subsection{1 Số Hợp Chất Quan Trọng của Kim Loại Kiềm}

%------------------------------------------------------------------------------%

\subsection{Kim Loại Kiềm Thổ}

%------------------------------------------------------------------------------%

\subsection{1 Số Hợp Chất Quan Trọng của Kim Loại Kiềm Thổ}

%------------------------------------------------------------------------------%

\subsection{Tính Chất của Kim Loại Kiềm, Kim Loại Kiềm Thổ}

%------------------------------------------------------------------------------%

\subsection{Nhôm}

%------------------------------------------------------------------------------%

\subsection{1 Số Hợp Chất Quan Trọng của Nhôm}

%------------------------------------------------------------------------------%

\subsection{Tính Chất của Nhôm \& Hợp Chất của Nhôm}

%------------------------------------------------------------------------------%

\subsection{Tính Chất của Kim Loại Kiềm, Kim Loại Kiềm Thổ \& Hợp Chất của Chúng}

%------------------------------------------------------------------------------%

\subsection{Tính Chất của Nhôm \& Hợp Chất của Nhôm}

%------------------------------------------------------------------------------%

\section{Crom -- Sắt -- Đồng}

\subsection{Crom}

%------------------------------------------------------------------------------%

\subsection{1 Số Hợp Chất của Crom}

%------------------------------------------------------------------------------%

\subsection{Sắt}

%------------------------------------------------------------------------------%

\subsection{1 Số Hợp Chất của Sắt}

%------------------------------------------------------------------------------%

\subsection{Hợp Kim của Sắt}

%------------------------------------------------------------------------------%

\subsection{Đồng \& 1 Số Hợp Chất của Đồng}

%------------------------------------------------------------------------------%

\subsection{Sơ Lược về 1 Số Kim Loại Khác}

%------------------------------------------------------------------------------%

\subsection{Tính Chất của Crom, Sắt, \& Những Hợp Chất của Chúng}

%------------------------------------------------------------------------------%

\subsection{Tính Chất của Đồng \& Hợp Chất của Đồng. Sơ Lược về Các Kim Loại Ag, Au, Ni, Zn, Sn, Pb}

%------------------------------------------------------------------------------%

\subsection{Tính Chất Hóa Học của Crom, Sắt, Đồng, \& Những Hợp Chất của Chúng}

%------------------------------------------------------------------------------%

\section{Phân Biệt 1 Số Chất Vô Cơ. Chuẩn Độ Dung Dịch}

\subsection{Nhận Biết 1 Số Cation trong Dung Dịch}

%------------------------------------------------------------------------------%

\subsection{Nhận Biết 1 Số Anion trong Dung Dịch}

%------------------------------------------------------------------------------%

\subsection{Nhận Biết 1 Số Chất Khí}

%------------------------------------------------------------------------------%

\subsection{Chuẩn Độ Acid--Base}

%------------------------------------------------------------------------------%

\subsection{Chuẩn Độ Oxi Hóa--Khử bằng Phương Pháp Pemanganat}

%------------------------------------------------------------------------------%

\subsection{Nhận Biết 1 Số Chất Vô Cơ}

%------------------------------------------------------------------------------%

\subsection{Nhận Biết 1 Số Ion trong Dung Dịch}

%------------------------------------------------------------------------------%

\subsection{Chuẩn Độ Dung Dịch}

%------------------------------------------------------------------------------%

\section{Hóa Học \& Vấn Đề Phát Triển Kinh Tế, Xã Hội, Môi Trường}

\subsection{Hóa Học \& Vấn Đề Phát Triển Kinh Tế}

%------------------------------------------------------------------------------%

\subsection{Hóa Học \& Vấn Đề Xã Hội}

%------------------------------------------------------------------------------%

\subsection{Hóa Học \& Vấn Đề Môi Trường}

%------------------------------------------------------------------------------%

\newpage
\printbibliography[heading=bibintoc]
	
\end{document}