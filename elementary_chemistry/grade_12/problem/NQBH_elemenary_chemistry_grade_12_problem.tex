\documentclass{article}
\usepackage[backend=biber,natbib=true,style=authoryear]{biblatex}
\addbibresource{/home/hong/1_NQBH/reference/bib.bib}
\usepackage[utf8]{vietnam}
\usepackage{tocloft}
\renewcommand{\cftsecleader}{\cftdotfill{\cftdotsep}}
\usepackage[colorlinks=true,linkcolor=blue,urlcolor=red,citecolor=magenta]{hyperref}
\usepackage{amsmath,amssymb,amsthm,mathtools,float,graphicx,algpseudocode,algorithm,tcolorbox,chemfig}
\usepackage[inline]{enumitem}
\usepackage[version=4]{mhchem}
\allowdisplaybreaks
\numberwithin{equation}{section}
\newtheorem{assumption}{Assumption}[section]
\newtheorem{conjecture}{Conjecture}[section]
\newtheorem{corollary}{Corollary}[section]
\newtheorem{hequa}{Hệ quả}[section]
\newtheorem{definition}{Definition}[section]
\newtheorem{dinhnghia}{Định nghĩa}[section]
\newtheorem{example}{Example}[section]
\newtheorem{vidu}{Ví dụ}[section]
\newtheorem{lemma}{Lemma}[section]
\newtheorem{notation}{Notation}[section]
\newtheorem{principle}{Principle}[section]
\newtheorem{problem}{Problem}[section]
\newtheorem{baitoan}{Bài toán}[section]
\newtheorem{proposition}{Proposition}[section]
\newtheorem{question}{Question}[section]
\newtheorem{cauhoi}{Câu hỏi}[section]
\newtheorem{remark}{Remark}[section]
\newtheorem{luuy}{Lưu ý}[section]
\newtheorem{theorem}{Theorem}[section]
\newtheorem{dinhly}{Định lý}[section]
\usepackage[left=0.5in,right=0.5in,top=1.5cm,bottom=1.5cm]{geometry}
\usepackage{fancyhdr}
\pagestyle{fancy}
\fancyhf{}
\lhead{\small Subsect.~\thesubsection}
\rhead{\small\nouppercase{\leftmark}}
\renewcommand{\subsectionmark}[1]{\markboth{#1}{}}
\cfoot{\thepage}
\def\labelitemii{$\circ$}

\title{Problems in Elementary Chemistry\texttt{/}Grade 12}
\author{Nguyễn Quản Bá Hồng\footnote{Independent Researcher, Ben Tre City, Vietnam\\e-mail: \texttt{nguyenquanbahong@gmail.com}; website: \url{https://nqbh.github.io}.}}
\date{\today}

\begin{document}
\maketitle
\begin{abstract}
	1 bộ sưu tập các bài toán chọn lọc từ cơ bản đến nâng cao cho Hóa học sơ cấp lớp 12. Tài liệu này là phần bài tập bổ sung cho tài liệu chính \href{https://github.com/NQBH/hobby/blob/master/elementary_chemistry/grade_12/NQBH_elementary_chemistry_grade_12.pdf}{GitHub\texttt{/}NQBH\texttt{/}hobby\texttt{/}elementary chemistry\texttt{/}grade 12\texttt{/}lecture}\footnote{\textsc{url}: \url{https://github.com/NQBH/hobby/blob/master/elementary_chemistry/grade_12/NQBH_elementary_chemistry_grade_12.pdf}.} của tác giả viết cho Toán lớp 6. Phiên bản mới nhất của tài liệu này được lưu trữ ở link sau: \href{https://github.com/NQBH/hobby/blob/master/elementary_chemistry/grade_12/problem/NQBH_elementary_chemistry_grade_12_problem.pdf}{GitHub\texttt{/}NQBH\texttt{/}hobby\texttt{/}elementary chemistry\texttt{/}grade 12\texttt{/}problem}\footnote{\textsc{url}: \url{https://github.com/NQBH/hobby/blob/master/elementary_chemistry/grade_12/problem/NQBH_elementary_chemistry_grade_12_problem.pdf}.}.
\end{abstract}
\tableofcontents
\newpage

%------------------------------------------------------------------------------%

``Để giải 1 bài toán hóa cần phải làm các bước sau:
\begin{enumerate*}
	\item[\textbf{1.}] Viết phương trình hóa học theo dữ kiện đầu bài.
	\item[\textbf{2.}] Đặt ẩn số cho mỗi phương trình hóa học đã viết.
	\item[\textbf{3.}] Tìm mối liên quan giữa các ẩn số với dữ kiện đầu bài để thiết lập phương trình toán, sau đó giải phương trình hoặc hệ phương trình toán.'' -- \cite[p. 17]{An2008}
\end{enumerate*}

\section{Este -- Lipit}

\subsection{Este}

\begin{vidu}[\cite{An2008}, Ví dụ 1, p. 17]
	``Đốt cháy $1$ este đơn chức, phương trình hóa học sẽ viết như sau:
	
	\emph{\ce{C_nH_{2n+1-2k}COOC_mH_{2m+1-2k'} + $\frac{3m + 3n + 1 - k - k'}{2}$O2 -> $(m + n + 1)$CO2 + $(m + n + 1 - k - k')$H2O}}. Sở dĩ phải dùng các chỉ số $m,n,k,k'$ vì không biết các acid, ancol tạo este là no hay không no. Các dữ liệu rất đa dạng có thể là phản ứng cháy, phản ứng este hóa, phản ứng xà phòng hóa, etc. $\ldots$ Phần này tạm phân loại có $7$ dạng dữ kiện khác nhau.
\end{vidu}
Bước 2: Đặt ẩn số cho mỗi phương trình hóa học, thông thường nên đặt ẩn số là số mol. E.g., $a$ mol, $b$ mol, etc. $\ldots$ cũng có trường hợp đặt ẩn số là số gam.

\begin{vidu}[\cite{An2008}, Ví dụ 2, p. 17]
	Đốt $8.6$g 1 este đơn chức thu được $8.96$l khí \emph{\ce{CO2}} (đktc) \& $5.4$g \emph{\ce{H2O}}. Thể tích \emph{\ce{O2}} cần $10.08$l (đktc). Tìm công thức este trên.
\end{vidu}

\begin{proof}[Giải]
	Bước 1: \ce{C_nH_{2n+1-2k}COOC_mH_{2m+1-2k'} + $\frac{3m + 3n + 1 - k - k'}{2}$O2 -> $(m + n + 1)$CO2 + $(m + n + 1 - k - k')$H2O}. Bước 2: Gọi số mol este tham gia phản ứng là $a$ mol. Suy ra: $n_{\ce{O2}} = \frac{3m + 3n + 1 - k - k'}{2}a = \frac{V_{\ce{O2}}}{22.4} = \frac{10.08}{22.4} = 0.4$ mol, $n_{\ce{CO2}} = (m + n + 1)a = \frac{V_{\ce{CO2}}}{22.4} = \frac{8.96}{22.4} = 0.4$ mol, $n_{\ce{H2O}} = (m + n + 1 - k - k')a = \frac{m_{\ce{H2O}}}{M_{\ce{H2O}}} = \frac{5.4}{18} = 0.3$ mol. Các ẩn số là cặp $m + n$ \& $k + k'$. Giải hệ phương trình
	\begin{equation*}
		\left\{\begin{split}
			(3m + 3n + 1 - k - k')a &= 0.9,\\
			(m + n + 1)a &= 0.4,\\
			(m + n + 1 - k - k')a &= 0.3,
		\end{split}\right.
	\end{equation*}
	suy ra $m + n = 3$, $k + k' = 1$ tương ứng sẽ có các cặp nghiệm: $(m,n) = (3,0)$ hoặc $(m,n) = (2,1)$, \& $k = 0$, $k' = 1$. Este là \ce{HCOO-C3H5}, \ce{C2H3COOCH3}, \ce{CH3COOC2H3}.
\end{proof}

\subsubsection{Dựa vào tính chất hóa học để suy luận công thức este}

\begin{baitoan}[\cite{An2008}, \textbf{1.}, p. 5]
	Thủy phân chất X có công thức \ce{C8H14O5} thu được ancol etylic \& chất hữu cơ Y. Cho biết số mol X bằng số mol ancol etylic bằng $\frac{1}{2}$ số mol Y. Y được điều chế trực tiếp từ glucose bằng phản ứng lên men. Trùng ngưng Y thu được 1 polyme. Công thức cấu tạo của X, Y?
\end{baitoan}

\begin{baitoan}[\cite{An2008}, \textbf{2.}, pp. 5--6]
	Thủy phân este A bằng dung dịch \emph{\ce{NaOH}} thu được muối B \& chất D. Biết:
	\begin{enumerate*}
		\item[$\bullet$] B tác dụng với dung dịch \emph{\ce{AgNO3}} trong \emph{\ce{NH3}} thu được \emph{\ce{Ag v}} \& dung dịch X. Cho dung dịch X tác dụng với dung dịch \emph{\ce{H2SO4}} loãng thu được khí \emph{\ce{CO2}}.
		\item[$\bullet$] D có công thức \emph{\ce{(CH2O)_n}}: \emph{\ce{D ->[$+$H2][Ni,$t^\circ$] E ->[+HCl] F}}.
		\item[$\bullet$] F có công thức \emph{\ce{(CH2Cl)_n}}.
	\end{enumerate*}
	Công thức cấu tạo của A, B, D, E, F là:
	\begin{itemize}
		\item[{\rm\sf A.}] A: \emph{\ce{HCOOCH=CH2}}, B: \emph{\ce{HCOONa}}, D: \emph{\ce{CH3CHO}}, E: \emph{\ce{C2H5OH}}, F: \emph{\ce{C2H5Cl}}.
		\item[{\rm\sf B.}] A: \emph{\ce{HCOOCH2CHO}}, B: \emph{\ce{HCOONa}}, D: \emph{\ce{HOCH2CHO}}, E: \emph{\ce{HO(CH2)2OH}}, F: \emph{\ce{Cl(CH2)2Cl}}.
		\item[{\rm\sf C.}] A: \emph{\ce{CH3COOCH=CH2}}, B: \emph{\ce{CH3COONa}}, D: \emph{\ce{CH3CHO}}, E: \emph{\ce{C2H5OH}}, F: \emph{\ce{C2H5Cl}}.
		\item[{\rm\sf D.}] A: \emph{\ce{HCOOCH2CH3}}, B: \emph{\ce{HCOONa}}, D: \emph{\ce{CH3CH2OH}}, E: \emph{\ce{CH3CHO}}, F: \emph{\ce{C2H5Cl}}.
	\end{itemize}
\end{baitoan}

\begin{baitoan}[\cite{An2008}, \textbf{3.}, p. 6]
	Hợp chất hữu cơ X có công thức \emph{\ce{C4H7O2Cl}} khi thủy phân trong môi trường kiềm được các sản phẩm trong đó có $2$ chất có khả năng tráng bạc. Công thức cấu tạo đúng của X là:
	\begin{enumerate*}
		\item[{\rm\sf A.}] \emph{\ce{HCOO-CH2-CHCl-CH3}};
		\item[{\rm\sf B.}] \emph{\ce{CH3COO-CH2Cl}};
		\item[{\rm\sf C.}] \emph{\ce{C2H5COO-CH2-CH3}};
		\item[{\rm\sf D.}] \emph{\ce{HCOOCHCl-CH2-CH3}}.
	\end{enumerate*}
\end{baitoan}

\begin{baitoan}[\cite{An2008}, \textbf{4.}, p. 6]
	Thủy phân este có công thức phân tử \emph{\ce{C4H8O2}} với xúc tác acid vô cơ loãng, thu được $2$ sản phẩm hữu cơ X, Y (chỉ chứa các nguyên tử \emph{\ce{C,H,O}}). Từ X có thể điều chế trực tiếp ra Y bằng 1 phản ứng duy nhất. Chất X là:
	\begin{enumerate*}
		\item[{\rm\sf A.}] Acid acetic;
		\item[{\rm\sf B.}] Ancol etylic;
		\item[{\rm\sf C.}] Etyl axetat;
		\item[{\rm\sf D.}] Axit fomic.
	\end{enumerate*}
\end{baitoan}

\begin{baitoan}[\cite{An2008}, \textbf{5.}, p. 7]
	Cho các chất metanol (A), nước (B), etanol (C), acid acetic (D), phenol (E). Độ linh động của nguyên tử \emph{\ce{H}} trong nhóm \emph{\ce{(-OH)}} của phân tử mỗi chất tăng dần theo thứ tự sau:
	\begin{enumerate*}
		\item[{\rm\sf A.}] A, B, C, D, E;
		\item[{\rm\sf B.}] E, B, A, C, D;
		\item[{\rm\sf C.}] B, A, C, D, E;
		\item[{\rm\sf D.}] C, A, B, E, D.
	\end{enumerate*}
\end{baitoan}

\begin{baitoan}[\cite{An2008}, \textbf{6.}, p. 7]
	Có 2 hợp chất hữu cơ X, Y chứa các nguyên tố \emph{\ce{C,H,O}}, phân tử khối đều bằng $74$. Biết X tác dụng được với \emph{\ce{Na}}, cả X, Y đều tác dụng được với dung dịch \emph{\ce{NaOH}} \& dung dịch \emph{\ce{AgNO3}} tan trong \emph{\ce{NH3}}. Vậy X, Y có thể là:
	\begin{enumerate*}
		\item[{\rm\sf A.}] \emph{\ce{C4H9OH,HCOOC2H5}};
		\item[{\rm\sf B.}] \emph{\ce{CH3COOCH3,HOC2H4CHO}};
		\item[{\rm\sf C.}] \emph{\ce{OHC-COOH,C2H5COOH}};
		\item[{\rm\sf D.}] \emph{\ce{OHC-COOH,HCOOC2H5}}.
	\end{enumerate*}
\end{baitoan}

\paragraph{1 số vấn đề cần lưu ý.}
\begin{enumerate}
	\item ``Khi đầu bài cho este là este no đơn chức thì công thức tổng quát \ce{C_nH_{2n}O2}, $n\ge 2$: Khi đầu bài cho dữ kiện đốt cháy 1 este tìm thấy: $n_{\ce{CO2}} = n_{\ce{H2O}}$ thì este đó là este no đơn chức. $n_{\ce{CO2}} > n_{\ce{H2O}}$ thì este đó là este không no.
	\item Khi đầu bài cho 2 chất hữu cơ đơn chức mạch hở tác dụng với \ce{NaOH} cho:
	\begin{itemize}
		\item 2 muối \& 1 ancol, có những khả năng 2 chất hữu cơ đó là: (\ce{RCOO{R'}} \& \ce{R_1COO{R'}}) hoặc (\ce{RCOO{R'}} \& \ce{R_1COOH}).
		\item 1 muối \& 1 ancol, có những khả năng 2 chất hữu cơ đó là:
		\begin{itemize}
			\item 1 este \& 1 ancol có gốc hydrocarbon giống ancol trong este.
			\item 1 este \& 1 acid có gốc hydrocarbon giống acid trong este.
			\item 1 acid \& 1 ancol.
		\end{itemize}
		\item 1 muối \& 2 ancol, có những khả năng 2 chất hữu cơ đó là:(\ce{RCOO{R'}} \& \ce{RCOOR_1}) hoặc (\ce{RCOO{R'}} \& \ce{R_1OH}).
		\item Khi cho muối của acid hữu cơ đung nóng với vôi tôi xút:
		
	\end{itemize}
\end{enumerate}





%------------------------------------------------------------------------------%

\subsection{Lipit}

%------------------------------------------------------------------------------%

\subsection{Chất Giặt Rửa}

%------------------------------------------------------------------------------%

\subsection{Mối Liên Hệ Giữa Hydrocarbon \& 1 Số Dẫn Xuất của Hydrocarbon}

%------------------------------------------------------------------------------%

\newpage
\section{Carbohydrate}

\subsection{Công Thức Cấu Tạo \& Tính Chất của Carbonhydrate}

\begin{baitoan}[\cite{An2008}, \textbf{1.}, p. 52]
	Trong những phát biểu sau đây, phát biểu nào đúng (Đ), phát biểu nào sai (S)?
	\begin{enumerate*}
		\item[{\rm\sf A.}] Glucose \& fructose là đồng phân cấu tạo của nhau.
		\item[{\rm\sf B.}] Có thể phân biệt glucose \& fructose bằng phản ứng tráng bạc.
		\item[{\rm\sf C.}] Trong dung dịch nước, glucose ưu tiên dạng mạch vòng hơn dạng mạch hở.
		\item[{\rm\sf D.}] Trong dung dịch nước, metyl $\alpha$-glucozit có thể chuyển sang dạng mạch hở.
	\end{enumerate*}
\end{baitoan}

\begin{baitoan}[\cite{An2008}, \textbf{2.}, p. 52]
	Maltose có tính khử do:
	\begin{enumerate*}
		\item[{\rm\sf A.}] trong dung dịch \emph{\ce{AgNO3}} nó bị thủy phân.
		\item[{\rm\sf B.}] phân tử còn nhóm \emph{\ce{-OH}} hemiaxetal chuyển thành nhóm \emph{\ce{-CHO}}.
		\item[{\rm\sf C.}] nó là 1 đisascearit.
		\item[{\rm\sf D.}] nó là đồng phân của saccarose.
	\end{enumerate*}
\end{baitoan}

\begin{baitoan}[\cite{An2008}, \textbf{3.}, p. 52]
	Saccarose \& maltose được gọi là đisaccarit vì:
	\begin{enumerate*}
		\item[{\rm\sf A.}] có phân tử khối bằng $2$ lần glucose.
		\item[{\rm\sf B.}] thủy phân sinh ra $2$ đơn vị monosaccarit.
		\item[{\rm\sf C.}] có tính chất hóa học tương tự monosaccarit.
		\item[{\rm\sf D.}] phân tử có số nguyên tử carbon gấp $2$ lần glucose.
	\end{enumerate*}
\end{baitoan}

\begin{baitoan}[\cite{An2008}, \textbf{4.}, p. 52]
	Giữa tinh bột, saccarose, glucose có đặc điểm chung là:
	\begin{enumerate*}
		\item[{\rm\sf A.}] chúng thuộc loại carbonhydrat.
		\item[{\rm\sf B.}] đều tác dụng với \emph{\ce{Cu(OH)2}} cho dung dịch xanh lam.
		\item[{\rm\sf C.}] đều bị thủy phân bởi dung dịch acid.
		\item[{\rm\sf D.}] đều có phản ứng tráng bạc.
	\end{enumerate*}
\end{baitoan}

\begin{baitoan}[\cite{An2008}, \textbf{5.}, p. 52]
	Cellulose không phản ứng với tác nhân nào dưới đây?
	 \begin{enumerate*}
	 	\item[{\rm\sf A.}] \emph{\ce{HNO3}đ\texttt{/}\ce{H2SO4}đ\texttt{/}$t^\circ$}.
	 	\item[{\rm\sf B.}] \emph{\ce{H2}\texttt{/}\ce{Ni}}.
	 	\item[{\rm\sf C.}] \emph{\ce{Cu(OH)2 + NH3}}.
	 	\item[{\rm\sf D.}] \emph{(\ce{CS2 + NaOH})}.
	 \end{enumerate*}
\end{baitoan}

\begin{baitoan}[\cite{An2008}, \textbf{6.}, p. 53]
	Ghi Đ (đúng) hoặc S (sai) vào ô vuông mỗi nội dung sau:
	\begin{enumerate*}
		\item[{\rm\sf A.}] Có thể phân biệt glucose \& fructose bằng vị giác.
		\item[{\rm\sf B.}] Dung dịch maltose có tính khử vì đã bị thủy phân thành glucose.
		\item[{\rm\sf C.}] Tinh bột \& cellulose không thể hiện tính khử vì trong phân tử hầu như không có nhóm \emph{\ce{-OH}} hemiaxetal tự do.
		\item[{\rm\sf D.}] Tinh bột có phản ứng màu với iot vì có cấu trúc vòng xoắn.
	\end{enumerate*}
\end{baitoan}

%------------------------------------------------------------------------------%

\subsection{Nhận Biết Các Carbohydrate}

\begin{baitoan}[\cite{An2008}, \textbf{7.}, p. 53]
	Cho các dung dịch: glucose, glixerol, axit axetic, etanol. Dùng thuốc thử nào sau đây để nhận biết các dung dịch trên?
	\begin{enumerate*}
		\item[{\rm\sf A.}] \emph{\ce{Cu(OH)2}} trong môi trường kiềm.
		\item[{\rm\sf B.}] \emph{\ce{[Ag(NH3)2]OH}}.
		\item[{\rm\sf C.}] \emph{\ce{Na}} kim loại.
		\item[{\rm\sf D.}] Nước brom.
	\end{enumerate*}
\end{baitoan}

\begin{baitoan}[\cite{An2008}, \textbf{8.}, p. 53]
	Có 3 dung dịch gồm: glucose, glixerol, etanol đựng trong 3 lọ bị mất nhãn. Để phân biệt các dung dịch trên, ta dùng thuốc thử là:
	\begin{enumerate*}
		\item[{\rm\sf A.}] \emph{\ce{[Ag(NH3)2]OH}}.
		\item[{\rm\sf B.}] \emph{\ce{Cu(OH)2}}, $t^\circ$.
		\item[{\rm\sf C.}] Nước brom.
		\item[{\rm\sf D.}] \emph{\ce{CH3OH}\texttt{/}\ce{HCl}}.
	\end{enumerate*}
\end{baitoan}

\begin{baitoan}[\cite{An2008}, \textbf{9.}, p. 53]
	Có 3 dung dịch gồm: glucose, fomanđehit, etanol đựng trong 3 lọ bị mất nhãn. Để nhận biết các dung dịch trên, ta dùng thuốc thử là:
	\begin{enumerate*}
		\item[{\rm\sf A.}] \emph{\ce{[Ag(NH3)2]OH}}.
		\item[{\rm\sf B.}] \emph{\ce{Cu(OH)2}}, $t^\circ$.
		\item[{\rm\sf C.}] Nước brom.
		\item[{\rm\sf D.}] Natri kim loại.
	\end{enumerate*}
\end{baitoan}

\begin{baitoan}[\cite{An2008}, \textbf{10.}, p. 53]
	Có 3 dung dịch gồm: saccarose, maltose, anđehit axetic đựng trong 3 lọ bị mất nhãn. Để nhận biết các dung dịch trên, ta dùng thuốc thử là:
	\begin{enumerate*}
		\item[{\rm\sf A.}] \emph{\ce{[Ag(NH3)2]OH}}.
		\item[{\rm\sf B.}] \emph{\ce{[Ag(NH3)2]OH}} \& \emph{\ce{Cu(OH)2}}, $t^\circ$.
		\item[{\rm\sf C.}] \emph{\ce{Cu(OH)2}}.
		\item[{\rm\sf D.}] Nước brom.
	\end{enumerate*}
\end{baitoan}

\subsection{Xác Định Khối Lượng Carbohydrate}

\begin{baitoan}[\cite{An2008}, \textbf{11.}, p. 54]
	Cho lên men $\rm1m^3$ nước rỉ đường, sau đó chưng cất thu được $60$ lít cồn $96^\circ$. $D_{\rm ancol\ etylic} = 0.789$\emph{g\texttt{/}ml} ở $\rm20^\circ C$. Khối lượng glucose có trong $\rm1m^3$ rỉ đường glucose trên là (biết hiệu suất quá trình lên men đạt $80$\%):
	\begin{enumerate*}
		\item[{\rm\sf A.}] $222.292$\emph{kg}.
		\item[{\rm\sf B.}] $232.39$\emph{kg}.
		\item[{\rm\sf C.}] $245.6$\emph{kg}.
		\item[{\rm\sf D.}] $111.145$\emph{kg}.
	\end{enumerate*}
\end{baitoan}

\begin{baitoan}[\cite{An2008}, \textbf{12.}, p. 54]
	Người ta dùng mùn cưa chứa $50$\% cellulose làm nguyên liệu để sản xuất ancol etylic. Để sản xuất $1$ tấn ancol etylic thì khối lượng mùn cưa cần dùng là (biết hiệu suất cả quá trình là $70$\%):
	\begin{enumerate*}
		\item[{\rm\sf A.}] $\approx5031$\emph{kg}.
		\item[{\rm\sf B.}] $\approx2515.53$\emph{kg}.
		\item[{\rm\sf C.}] $\approx4561$\emph{kg}.
		\item[{\rm\sf D.}] $6052$\emph{kg}.
	\end{enumerate*}
\end{baitoan}

\begin{baitoan}[\cite{An2008}, \textbf{13.}, p. 55]
	Để sản xuất $0.5$ tấn cellulose trinitrate thì khối lượng cellulose cần dùng là:
	\begin{enumerate*}
		\item[{\rm\sf A.}] $390.9$\emph{kg}.
		\item[{\rm\sf B.}] $619.8$\emph{kg}.
		\item[{\rm\sf C.}] $309.9$\emph{kg}.
		\item[{\rm\sf D.}] $408$\emph{kg}.
	\end{enumerate*}
\end{baitoan}

\begin{baitoan}[\cite{An2008}, \textbf{14.}, p. 55]
	Người ta dùng khoai lang chứa $20$\% tinh bột để sản xuất ancol etylic. Khối lượng khoai lang trên để sản xuất $0.5$ tấn ancol etylic là (biết sự hao hụt trong sản xuất là $15$\%):
	\begin{enumerate*}
		\item[{\rm\sf A.}] $5179$\emph{kg}.
		\item[{\rm\sf B.}] $10358$\emph{kg}.
		\item[{\rm\sf C.}] $2071.6$\emph{kg}.
		\item[{\rm\sf D.}] $6279.8$\emph{kg}.
	\end{enumerate*}
\end{baitoan}

\subsection{Xác Định Khối Lượng Sản Phẩm của Carbohydrate}

\begin{baitoan}[\cite{An2008}, \textbf{15.}, p. 56]
	Cho $2.5$\emph{kg} glucose chứa $20$\% tạp chất lên men thành ancol etylic. Thể tích ancol etylic thu được là (biết $D$ của ancol etylic là $0.8$\emph{g\texttt{/}ml} \& trong quá trình chế biến hao hụt $10$\%):
	\begin{enumerate*}
		\item[{\rm\sf A.}] $1150$\emph{ml}.
		\item[{\rm\sf B.}] $1510$\emph{ml}.
		\item[{\rm\sf C.}] $1260$\emph{ml}.
		\item[{\rm\sf D.}] $750$\emph{ml}.
	\end{enumerate*}
\end{baitoan}

\begin{baitoan}[\cite{An2008}, \textbf{16.}, p. 56]
	Người ta dùng tinh bột chứa $70$\% nguyên chất để sản xuất ancol etylic. Nếu dùng $1$ tấn tinh bột này thì khối lượng ancol etylic thu được là (biết hao hụt trong toàn bộ quá trình sản xuất là $15$\%):
	\begin{enumerate*}
		\item[{\rm\sf A.}] $397.3$\emph{kg}.
		\item[{\rm\sf B.}] $337.9$\emph{kg}.
		\item[{\rm\sf C.}] $373.7$\emph{kg}.
		\item[{\rm\sf D.}] $400.5$\emph{kg}.
	\end{enumerate*}
\end{baitoan}

\begin{baitoan}[\cite{An2008}, \textbf{17.}, p. 56]
	Đun nóng dung dịch chứa $18$\emph{g} glucose với \emph{\ce{AgNO3}} đủ phản ứng trong dung dịch \emph{\ce{NH3}} thấy \emph{\ce{Ag}} tách ra. Khối lượng \emph{\ce{AgNO3}} cần dùng là:
	\begin{enumerate*}
		\item[{\rm\sf A.}] $34$\emph{g}.
		\item[{\rm\sf B.}] $35$\emph{g}.
		\item[{\rm\sf C.}] $43$\emph{g}.
		\item[{\rm\sf D.}] $68$\emph{g}.
	\end{enumerate*}
\end{baitoan}

\begin{baitoan}[\cite{An2008}, \textbf{18.}, p. 57]
	Người ta dùng gạo nếp chứa $80$\% tinh bột để sản xuất còn, nếu dùng $10$\emph{kg} gạo nếp này thì thể tích cồn $96^\circ$ thu được là (hiệu suất của quá trình lên men đạt $80$\% \& $D_{\scriptsize\mbox{cồn }96^\circ} = 0.807$\emph{g\texttt{/}ml}):
	\begin{enumerate*}
		\item[{\rm\sf A.}] $\approx 4.9$\emph{l}.
		\item[{\rm\sf B.}] $\approx 4.8$\emph{l}.
		\item[{\rm\sf C.}] $\approx 4.7$\emph{l}.
		\item[{\rm\sf D.}] $\approx 7.4$\emph{l}.
	\end{enumerate*}
\end{baitoan}

\begin{baitoan}[\cite{An2008}, \textbf{19.}, p. 57]
	Để phân biệt các dung dịch hóa chất riêng biệt là saccarose, maltose, etanol, \& fomanđehit, có thể dùng 1 trong các hóa chất nào sau đây:
	\begin{enumerate*}
		\item[{\rm\sf A.}] \emph{\ce{Cu(OH)2}\texttt{/}\ce{OH-}}.
		\item[{\rm\sf B.}] \emph{\ce{AgNO3}\texttt{/}\ce{NH3}}.
		\item[{\rm\sf C.}] \emph{\ce{H2}\texttt{/}\ce{Ni},$t^\circ$}.
		\item[{\rm\sf D.}] Vôi sữa.
	\end{enumerate*}
\end{baitoan}

\begin{baitoan}[\cite{An2008}, \textbf{20.}, p. 57]
	Có 3 dung dịch gồm: saccarose, glucose, glixerol đựng trong 3 lọ bị mất nhãn. Để phân biệt các dung dịch trên ta dùng thuốc thử là:
	\begin{enumerate*}
		\item[{\rm\sf A.}] \emph{\ce{[Ag(NH3)2]OH}}.
		\item[{\rm\sf B.}] \emph{\ce{[Ag(NH3)2]OH,H2SO4}} đun nhẹ.
		\item[{\rm\sf C.}] \emph{\ce{Cu(OH)2}}.
		\item[{\rm\sf D.}] \emph{\ce{CH3OH}\texttt{/}\ce{HCl}}.
	\end{enumerate*}
\end{baitoan}

\begin{baitoan}[\cite{An2008}, \textbf{21.}, pp. 57--58]
	Có 4 dung dịch gồm: saccarose, maltose, glixerol, anđehit axetic. Để nhận biết các dung dịch trên, ta dùng thuốc thử là:
	\begin{enumerate*}
		\item[{\rm\sf A.}] \emph{\ce{AgNO3}\texttt{/}\ce{NH3}}.
		\item[{\rm\sf B.}] \emph{\ce{Cu(OH)2}} \& \emph{\ce{[Ag(NH3)2]OH}}.
		\item[{\rm\sf C.}] \emph{\ce{[Ag(NH3)2]OH,H2SO4}} đun nhẹ, \emph{\ce{Cu(OH)2}}.
		\item[{\rm\sf D.}] Nước brom.
	\end{enumerate*}
\end{baitoan}

\begin{baitoan}[\cite{An2008}, \textbf{22.}, p. 58]
	Người ta dùng khoai chứa $20$\% tinh bột để sản xuất glucose. Nếu thủy phân 1 tấn khoai nói trên thì lượng glucose thu được là (biết hiệu suất phản ứng là $70$\%):
	\begin{enumerate*}
		\item[{\rm\sf A.}] $155.54$\emph{kg}.
		\item[{\rm\sf B.}] $151.9$\emph{kg}.
		\item[{\rm\sf C.}] $165.9$\emph{kg}.
		\item[{\rm\sf D.}] $251.8$\emph{kg}.
	\end{enumerate*}
\end{baitoan}	

\begin{baitoan}[\cite{An2008}, \textbf{23.}, p. 58]
	Cho $a$\emph{g} glucose lên men, khí \emph{\ce{CO2}} sinh ra cho vào nước vôi trong dư thu được $10$\emph{g} chất kết tủa. Giá trị của $a$ là (hiệu suất phản ứng là $80$\%):
	\begin{enumerate*}
		\item[{\rm\sf A.}] $22.5$\emph{g}.
		\item[{\rm\sf B.}] $11.25$\emph{g}.
		\item[{\rm\sf C.}] $11.75$\emph{g}.
		\item[{\rm\sf D.}] $12.15$\emph{g}.
	\end{enumerate*}
\end{baitoan}

\begin{baitoan}[\cite{An2008}, \textbf{24.}, p. 58]
	Cho $250$\emph{ml} dung dịch glucose chưa rõ nồng độ tác dụng với 1 lượng dư \emph{\ce{AgNO3}} trong dung dịch \emph{\ce{NH3}} thu được $2.16$\emph{g} bạc kết tủa. Nồng độ mol của dung dịch glucose đã dùng là:
	\begin{enumerate*}
		\item[{\rm\sf A.}] $0.3$\emph{M}.
		\item[{\rm\sf B.}] $0.4$\emph{M}.
		\item[{\rm\sf C.}] $0.2$\emph{M}.
		\item[{\rm\sf D.}] $0.1$\emph{M}.
	\end{enumerate*}
\end{baitoan}

\begin{baitoan}[\cite{An2008}, \textbf{25.}, p. 58]
	Câu đúng trong các câu sau là:
	\begin{enumerate*}
		\item[{\rm\sf A.}] Tinh bột là cellulose đều tham gia phản ứng tráng bạc.
		\item[{\rm\sf B.}] Tinh bột, saccarose, \& cellulose có công thức chung là \emph{\ce{C_n(H2O)_n}}.
		\item[{\rm\sf C.}] Tinh bột, saccarose, \& cellulose có công thức chung là \emph{\ce{C_n(H2O)_m}}.
		\item[{\rm\sf D.}] Tinh bột, saccarose, \& cellulose là monosaccarit.
	\end{enumerate*}
\end{baitoan}

\begin{baitoan}[\cite{An2008}, \textbf{26.}, p. 58]
	Saccarose có thể tác dụng được với chất nào sau đây: (1) \emph{\ce{H2}\texttt{/}\ce{Ni}}, $t^\circ$; (2) \emph{\ce{Cu(OH)2}}; (3) \emph{\ce{[Ag(NH3)2]OH}}; (4) \emph{\ce{CH3COOH}} (\emph{\ce{H2SO4}} đặc).
	\begin{enumerate*}
		\item[{\rm\sf A.}] (1), (2).
		\item[{\rm\sf B.}] (2), (4).
		\item[{\rm\sf C.}] (2), (3).
		\item[{\rm\sf D.}] (1), (4).
	\end{enumerate*}
\end{baitoan}

\begin{baitoan}[\cite{An2008}, \textbf{27.}, p. 58]
	Từ 1 tấn nước mía chứa $13$\% saccarose, cho biết hiệu suất thu hồi saccarose đạt $80$\% thì khối lượng saccarose thu được là:
	\begin{enumerate*}
		\item[{\rm\sf A.}] $104$\emph{kg}.
		\item[{\rm\sf B.}] $140$\emph{kg}.
		\item[{\rm\sf C.}] $108$\emph{kg}.
		\item[{\rm\sf D.}] $204$\emph{kg}.
	\end{enumerate*}
\end{baitoan}

\begin{baitoan}[\cite{An2008}, \textbf{28.}, p. 58]
	Thủy phân hoàn toàn $62.5$\emph{g} dung dịch saccarose $17.1$\% trong môi trường acid (vừa đủ) thu được dung dịch X. Cho \emph{\ce{AgNO3}} trong dung dịch \emph{\ce{NH3}} vào dung dịch X \& đun nhẹ thì khối lượng bạc thu được là:
	\begin{enumerate*}
		\item[{\rm\sf A.}] $6.57$\emph{g}.
		\item[{\rm\sf B.}] $7.65$\emph{g}.
		\item[{\rm\sf C.}] $6.75$\emph{g}.
		\item[{\rm\sf D.}] $8.5$\emph{g}.
	\end{enumerate*}
\end{baitoan}

%------------------------------------------------------------------------------%

\section{Amin -- Amino Axit -- Protein}

\subsection{Amin}

%------------------------------------------------------------------------------%

\subsection{Amino Axit}

%------------------------------------------------------------------------------%

\subsection{Peptit \& Protein}

%------------------------------------------------------------------------------%

\subsection{Cấu Tạo \& Tính Chất của Amin, Amino Axit, Protein}

%------------------------------------------------------------------------------%

\subsection{1 Số Tính Chất của Amin, Amino Axit, \& Protein}

%------------------------------------------------------------------------------%

\section{Polyme \& Vật Liệu Polyme}

\subsection{Đại Cương về Polyme}

%------------------------------------------------------------------------------%

\subsection{Vật Liệu Polyme}

%------------------------------------------------------------------------------%

\subsection{Polyme \& Vật Liệu Polyme}

%------------------------------------------------------------------------------%

\section{Đại Cương về Kim Loại}

\subsection{Kim Loại \& Hợp Kim}

%------------------------------------------------------------------------------%

\subsection{Dãy Điện Hóa của Kim Loại}

%------------------------------------------------------------------------------%

\subsection{Tính Chất của Kim Loại}

%------------------------------------------------------------------------------%

\subsection{Sự Điện Phân -- Sự Ăn Mòn Kim Loại -- Điều Chế Kim Loại}

%------------------------------------------------------------------------------%

\subsection{Dãy Điện Hóa của Kim Loại. Điều Chế Kim Loại}

%------------------------------------------------------------------------------%

\subsection{Ăn Mòn Kim Loại. Chống Ăn Mòn Kim Loại}

%------------------------------------------------------------------------------%

\section{Kim Loại Kiềm -- Kim Loại Kiềm Thổ -- Nhôm}

\subsection{Kim Loại Kiềm}

%------------------------------------------------------------------------------%

\subsection{1 Số Hợp Chất Quan Trọng của Kim Loại Kiềm}

%------------------------------------------------------------------------------%

\subsection{Kim Loại Kiềm Thổ}

%------------------------------------------------------------------------------%

\subsection{1 Số Hợp Chất Quan Trọng của Kim Loại Kiềm Thổ}

%------------------------------------------------------------------------------%

\subsection{Tính Chất của Kim Loại Kiềm, Kim Loại Kiềm Thổ}

%------------------------------------------------------------------------------%

\subsection{Nhôm}

%------------------------------------------------------------------------------%

\subsection{1 Số Hợp Chất Quan Trọng của Nhôm}

%------------------------------------------------------------------------------%

\subsection{Tính Chất của Nhôm \& Hợp Chất của Nhôm}

%------------------------------------------------------------------------------%

\subsection{Tính Chất của Kim Loại Kiềm, Kim Loại Kiềm Thổ \& Hợp Chất của Chúng}

%------------------------------------------------------------------------------%

\subsection{Tính Chất của Nhôm \& Hợp Chất của Nhôm}

%------------------------------------------------------------------------------%

\section{Crom -- Sắt -- Đồng}

\subsection{Crom}

%------------------------------------------------------------------------------%

\subsection{1 Số Hợp Chất của Crom}

%------------------------------------------------------------------------------%

\subsection{Sắt}

%------------------------------------------------------------------------------%

\subsection{1 Số Hợp Chất của Sắt}

%------------------------------------------------------------------------------%

\subsection{Hợp Kim của Sắt}

%------------------------------------------------------------------------------%

\subsection{Đồng \& 1 Số Hợp Chất của Đồng}

%------------------------------------------------------------------------------%

\subsection{Sơ Lược về 1 Số Kim Loại Khác}

%------------------------------------------------------------------------------%

\subsection{Tính Chất của Crom, Sắt, \& Những Hợp Chất của Chúng}

%------------------------------------------------------------------------------%

\subsection{Tính Chất của Đồng \& Hợp Chất của Đồng. Sơ Lược về Các Kim Loại Ag, Au, Ni, Zn, Sn, Pb}

%------------------------------------------------------------------------------%

\subsection{Tính Chất Hóa Học của Crom, Sắt, Đồng, \& Những Hợp Chất của Chúng}

%------------------------------------------------------------------------------%

\section{Phân Biệt 1 Số Chất Vô Cơ. Chuẩn Độ Dung Dịch}

\subsection{Nhận Biết 1 Số Cation trong Dung Dịch}

%------------------------------------------------------------------------------%

\subsection{Nhận Biết 1 Số Anion trong Dung Dịch}

%------------------------------------------------------------------------------%

\subsection{Nhận Biết 1 Số Chất Khí}

%------------------------------------------------------------------------------%

\subsection{Chuẩn Độ Acid--Base}

%------------------------------------------------------------------------------%

\subsection{Chuẩn Độ Oxi Hóa--Khử bằng Phương Pháp Pemanganat}

%------------------------------------------------------------------------------%

\subsection{Nhận Biết 1 Số Chất Vô Cơ}

%------------------------------------------------------------------------------%

\subsection{Nhận Biết 1 Số Ion trong Dung Dịch}

%------------------------------------------------------------------------------%

\subsection{Chuẩn Độ Dung Dịch}

%------------------------------------------------------------------------------%

\section{Hóa Học \& Vấn Đề Phát Triển Kinh Tế, Xã Hội, Môi Trường}

\subsection{Hóa Học \& Vấn Đề Phát Triển Kinh Tế}

%------------------------------------------------------------------------------%

\subsection{Hóa Học \& Vấn Đề Xã Hội}

%------------------------------------------------------------------------------%

\subsection{Hóa Học \& Vấn Đề Môi Trường}

%------------------------------------------------------------------------------%

\newpage
\printbibliography[heading=bibintoc]
	
\end{document}