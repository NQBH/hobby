\documentclass{article}
\usepackage[backend=biber,natbib=true,style=authoryear]{biblatex}
\addbibresource{/home/hong/1_NQBH/reference/bib.bib}
\usepackage[utf8]{vietnam}
\usepackage{tocloft}
\renewcommand{\cftsecleader}{\cftdotfill{\cftdotsep}}
\usepackage[colorlinks=true,linkcolor=blue,urlcolor=red,citecolor=magenta]{hyperref}
\usepackage{amsmath,amssymb,amsthm,mathtools,float,graphicx,algpseudocode,algorithm,tcolorbox,chemfig}
\usepackage[inline]{enumitem}
\usepackage[version=4]{mhchem}
\allowdisplaybreaks
\numberwithin{equation}{section}
\newtheorem{assumption}{Assumption}[section]
\newtheorem{conjecture}{Conjecture}[section]
\newtheorem{corollary}{Corollary}[section]
\newtheorem{hequa}{Hệ quả}[section]
\newtheorem{definition}{Definition}[section]
\newtheorem{dinhnghia}{Định nghĩa}[section]
\newtheorem{example}{Example}[section]
\newtheorem{vidu}{Ví dụ}[section]
\newtheorem{lemma}{Lemma}[section]
\newtheorem{notation}{Notation}[section]
\newtheorem{principle}{Principle}[section]
\newtheorem{problem}{Problem}[section]
\newtheorem{baitoan}{Bài toán}[section]
\newtheorem{proposition}{Proposition}[section]
\newtheorem{question}{Question}[section]
\newtheorem{cauhoi}{Câu hỏi}[section]
\newtheorem{remark}{Remark}[section]
\newtheorem{luuy}{Lưu ý}[section]
\newtheorem{theorem}{Theorem}[section]
\newtheorem{dinhly}{Định lý}[section]
\usepackage[left=0.5in,right=0.5in,top=1.5cm,bottom=1.5cm]{geometry}
\usepackage{fancyhdr}
\pagestyle{fancy}
\fancyhf{}
\lhead{\small Subsect.~\thesubsection}
\rhead{\small\nouppercase{\leftmark}}
\renewcommand{\subsectionmark}[1]{\markboth{#1}{}}
\cfoot{\thepage}
\def\labelitemii{$\circ$}

\title{Problems in Elementary Chemistry\texttt{/}Grade 12}
\author{Nguyễn Quản Bá Hồng\footnote{Independent Researcher, Ben Tre City, Vietnam\\e-mail: \texttt{nguyenquanbahong@gmail.com}; website: \url{https://nqbh.github.io}.}}
\date{\today}

\begin{document}
\maketitle
\begin{abstract}
	1 bộ sưu tập các bài toán chọn lọc từ cơ bản đến nâng cao cho Hóa học sơ cấp lớp 12. Tài liệu này là phần bài tập bổ sung cho tài liệu chính \href{https://github.com/NQBH/hobby/blob/master/elementary_chemistry/grade_12/NQBH_elementary_chemistry_grade_12.pdf}{GitHub\texttt{/}NQBH\texttt{/}hobby\texttt{/}elementary chemistry\texttt{/}grade 12\texttt{/}lecture}\footnote{\textsc{url}: \url{https://github.com/NQBH/hobby/blob/master/elementary_chemistry/grade_12/NQBH_elementary_chemistry_grade_12.pdf}.} của tác giả viết cho Toán lớp 6. Phiên bản mới nhất của tài liệu này được lưu trữ ở link sau: \href{https://github.com/NQBH/hobby/blob/master/elementary_chemistry/grade_12/problem/NQBH_elementary_chemistry_grade_12_problem.pdf}{GitHub\texttt{/}NQBH\texttt{/}hobby\texttt{/}elementary chemistry\texttt{/}grade 12\texttt{/}problem}\footnote{\textsc{url}: \url{https://github.com/NQBH/hobby/blob/master/elementary_chemistry/grade_12/problem/NQBH_elementary_chemistry_grade_12_problem.pdf}.}.
\end{abstract}
\tableofcontents
\newpage

%------------------------------------------------------------------------------%

``Để giải 1 bài toán hóa cần phải làm các bước sau:
\begin{enumerate*}
	\item[\textbf{1.}] Viết phương trình hóa học theo dữ kiện đầu bài.
	\item[\textbf{2.}] Đặt ẩn số cho mỗi phương trình hóa học đã viết.
	\item[\textbf{3.}] Tìm mối liên quan giữa các ẩn số với dữ kiện đầu bài để thiết lập phương trình toán, sau đó giải phương trình hoặc hệ phương trình toán.'' -- \cite[p. 17]{An2008}
\end{enumerate*}

\section{Este -- Lipit}

\subsection{Este}

\begin{vidu}[\cite{An2008}, Ví dụ 1, p. 17]
	``Đốt cháy $1$ este đơn chức, phương trình hóa học sẽ viết như sau:
	
	\emph{\ce{C_nH_{2n+1-2k}COOC_mH_{2m+1-2k'} + $\frac{3m + 3n + 1 - k - k'}{2}$O2 -> $(m + n + 1)$CO2 + $(m + n + 1 - k - k')$H2O}}. Sở dĩ phải dùng các chỉ số $m,n,k,k'$ vì không biết các acid, ancol tạo este là no hay không no. Các dữ liệu rất đa dạng có thể là phản ứng cháy, phản ứng este hóa, phản ứng xà phòng hóa, etc. $\ldots$ Phần này tạm phân loại có $7$ dạng dữ kiện khác nhau.
\end{vidu}
Bước 2: Đặt ẩn số cho mỗi phương trình hóa học, thông thường nên đặt ẩn số là số mol. E.g., $a$ mol, $b$ mol, etc. $\ldots$ cũng có trường hợp đặt ẩn số là số gam.

\begin{vidu}[\cite{An2008}, Ví dụ 2, p. 17]
	Đốt $8.6$g 1 este đơn chức thu được $8.96$l khí \emph{\ce{CO2}} (đktc) \& $5.4$g \emph{\ce{H2O}}. Thể tích \emph{\ce{O2}} cần $10.08$l (đktc). Tìm công thức este trên.
\end{vidu}

\begin{proof}[Giải]
	Bước 1: \ce{C_nH_{2n+1-2k}COOC_mH_{2m+1-2k'} + $\frac{3m + 3n + 1 - k - k'}{2}$O2 -> $(m + n + 1)$CO2 + $(m + n + 1 - k - k')$H2O}. Bước 2: Gọi số mol este tham gia phản ứng là $a$ mol. Suy ra: $n_{\ce{O2}} = \frac{3m + 3n + 1 - k - k'}{2}a = \frac{V_{\ce{O2}}}{22.4} = \frac{10.08}{22.4} = 0.4$ mol, $n_{\ce{CO2}} = (m + n + 1)a = \frac{V_{\ce{CO2}}}{22.4} = \frac{8.96}{22.4} = 0.4$ mol, $n_{\ce{H2O}} = (m + n + 1 - k - k')a = \frac{m_{\ce{H2O}}}{M_{\ce{H2O}}} = \frac{5.4}{18} = 0.3$ mol. Các ẩn số là cặp $m + n$ \& $k + k'$. Giải hệ phương trình
	\begin{equation*}
		\left\{\begin{split}
			(3m + 3n + 1 - k - k')a &= 0.9,\\
			(m + n + 1)a &= 0.4,\\
			(m + n + 1 - k - k')a &= 0.3,
		\end{split}\right.
	\end{equation*}
	suy ra $m + n = 3$, $k + k' = 1$ tương ứng sẽ có các cặp nghiệm: $(m,n) = (3,0)$ hoặc $(m,n) = (2,1)$, \& $k = 0$, $k' = 1$. Este là \ce{HCOO-C3H5}, \ce{C2H3COOCH3}, \ce{CH3COOC2H3}.
\end{proof}

\subsubsection{Dựa vào tính chất hóa học để suy luận công thức este}

\begin{baitoan}[\cite{An2008}, \textbf{1.}, p. 5]
	Thủy phân chất X có công thức \ce{C8H14O5} thu được ancol etylic \& chất hữu cơ Y. Cho biết số mol X bằng số mol ancol etylic bằng $\frac{1}{2}$ số mol Y. Y được điều chế trực tiếp từ glucose bằng phản ứng lên men. Trùng ngưng Y thu được 1 polyme. Công thức cấu tạo của X, Y?
\end{baitoan}

\begin{baitoan}[\cite{An2008}, \textbf{2.}, pp. 5--6]
	Thủy phân este A bằng dung dịch \emph{\ce{NaOH}} thu được muối B \& chất D. Biết:
	\begin{enumerate*}
		\item[$\bullet$] B tác dụng với dung dịch \emph{\ce{AgNO3}} trong \emph{\ce{NH3}} thu được \emph{\ce{Ag v}} \& dung dịch X. Cho dung dịch X tác dụng với dung dịch \emph{\ce{H2SO4}} loãng thu được khí \emph{\ce{CO2}}.
		\item[$\bullet$] D có công thức \emph{\ce{(CH2O)_n}}: \emph{\ce{D ->[$+$H2][Ni,$t^\circ$] E ->[+HCl] F}}.
		\item[$\bullet$] F có công thức \emph{\ce{(CH2Cl)_n}}.
	\end{enumerate*}
	Công thức cấu tạo của A, B, D, E, F là:
	\begin{itemize}
		\item[{\rm\sf A.}] A: \emph{\ce{HCOOCH=CH2}}, B: \emph{\ce{HCOONa}}, D: \emph{\ce{CH3CHO}}, E: \emph{\ce{C2H5OH}}, F: \emph{\ce{C2H5Cl}}.
		\item[{\rm\sf B.}] A: \emph{\ce{HCOOCH2CHO}}, B: \emph{\ce{HCOONa}}, D: \emph{\ce{HOCH2CHO}}, E: \emph{\ce{HO(CH2)2OH}}, F: \emph{\ce{Cl(CH2)2Cl}}.
		\item[{\rm\sf C.}] A: \emph{\ce{CH3COOCH=CH2}}, B: \emph{\ce{CH3COONa}}, D: \emph{\ce{CH3CHO}}, E: \emph{\ce{C2H5OH}}, F: \emph{\ce{C2H5Cl}}.
		\item[{\rm\sf D.}] A: \emph{\ce{HCOOCH2CH3}}, B: \emph{\ce{HCOONa}}, D: \emph{\ce{CH3CH2OH}}, E: \emph{\ce{CH3CHO}}, F: \emph{\ce{C2H5Cl}}.
	\end{itemize}
\end{baitoan}

\begin{baitoan}[\cite{An2008}, \textbf{3.}, p. 6]
	Hợp chất hữu cơ X có công thức \emph{\ce{C4H7O2Cl}} khi thủy phân trong môi trường kiềm được các sản phẩm trong đó có $2$ chất có khả năng tráng bạc. Công thức cấu tạo đúng của X là:
	\begin{enumerate*}
		\item[{\rm\sf A.}] \emph{\ce{HCOO-CH2-CHCl-CH3}};
		\item[{\rm\sf B.}] \emph{\ce{CH3COO-CH2Cl}};
		\item[{\rm\sf C.}] \emph{\ce{C2H5COO-CH2-CH3}};
		\item[{\rm\sf D.}] \emph{\ce{HCOOCHCl-CH2-CH3}}.
	\end{enumerate*}
\end{baitoan}

\begin{baitoan}[\cite{An2008}, \textbf{4.}, p. 6]
	Thủy phân este có công thức phân tử \emph{\ce{C4H8O2}} với xúc tác acid vô cơ loãng, thu được $2$ sản phẩm hữu cơ X, Y (chỉ chứa các nguyên tử \emph{\ce{C,H,O}}). Từ X có thể điều chế trực tiếp ra Y bằng 1 phản ứng duy nhất. Chất X là:
	\begin{enumerate*}
		\item[{\rm\sf A.}] Acid acetic;
		\item[{\rm\sf B.}] Ancol etylic;
		\item[{\rm\sf C.}] Etyl axetat;
		\item[{\rm\sf D.}] Axit fomic.
	\end{enumerate*}
\end{baitoan}

\begin{baitoan}[\cite{An2008}, \textbf{5.}, p. 7]
	Cho các chất metanol (A), nước (B), etanol (C), acid acetic (D), phenol (E). Độ linh động của nguyên tử \emph{\ce{H}} trong nhóm \emph{\ce{(-OH)}} của phân tử mỗi chất tăng dần theo thứ tự sau:
	\begin{enumerate*}
		\item[{\rm\sf A.}] A, B, C, D, E;
		\item[{\rm\sf B.}] E, B, A, C, D;
		\item[{\rm\sf C.}] B, A, C, D, E;
		\item[{\rm\sf D.}] C, A, B, E, D.
	\end{enumerate*}
\end{baitoan}

\begin{baitoan}[\cite{An2008}, \textbf{6.}, p. 7]
	Có 2 hợp chất hữu cơ X, Y chứa các nguyên tố \emph{\ce{C,H,O}}, phân tử khối đều bằng $74$. Biết X tác dụng được với \emph{\ce{Na}}, cả X, Y đều tác dụng được với dung dịch \emph{\ce{NaOH}} \& dung dịch \emph{\ce{AgNO3}} tan trong \emph{\ce{NH3}}. Vậy X, Y có thể là:
	\begin{enumerate*}
		\item[{\rm\sf A.}] \emph{\ce{C4H9OH,HCOOC2H5}};
		\item[{\rm\sf B.}] \emph{\ce{CH3COOCH3,HOC2H4CHO}};
		\item[{\rm\sf C.}] \emph{\ce{OHC-COOH,C2H5COOH}};
		\item[{\rm\sf D.}] \emph{\ce{OHC-COOH,HCOOC2H5}}.
	\end{enumerate*}
\end{baitoan}

\paragraph{1 số vấn đề cần lưu ý.}
\begin{enumerate}
	\item ``Khi đầu bài cho este là este no đơn chức thì công thức tổng quát \ce{C_nH_{2n}O2}, $n\ge 2$: Khi đầu bài cho dữ kiện đốt cháy 1 este tìm thấy: $n_{\ce{CO2}} = n_{\ce{H2O}}$ thì este đó là este no đơn chức. $n_{\ce{CO2}} > n_{\ce{H2O}}$ thì este đó là este không no.
	\item Khi đầu bài cho 2 chất hữu cơ đơn chức mạch hở tác dụng với \ce{NaOH} cho:
	\begin{itemize}
		\item 2 muối \& 1 ancol, có những khả năng 2 chất hữu cơ đó là: (\ce{RCOO{R'}} \& \ce{R_1COO{R'}}) hoặc (\ce{RCOO{R'}} \& \ce{R_1COOH}).
		\item 1 muối \& 1 ancol, có những khả năng 2 chất hữu cơ đó là:
		\begin{itemize}
			\item 1 este \& 1 ancol có gốc hydrocarbon giống ancol trong este.
			\item 1 este \& 1 acid có gốc hydrocarbon giống acid trong este.
			\item 1 acid \& 1 ancol.
		\end{itemize}
		\item 1 muối \& 2 ancol, có những khả năng 2 chất hữu cơ đó là:(\ce{RCOO{R'}} \& \ce{RCOOR_1}) hoặc (\ce{RCOO{R'}} \& \ce{R_1OH}).
		\item Khi cho muối của acid hữu cơ đung nóng với vôi tôi xút:
		
	\end{itemize}
\end{enumerate}





%------------------------------------------------------------------------------%

\subsection{Lipit}

%------------------------------------------------------------------------------%

\subsection{Chất Giặt Rửa}

%------------------------------------------------------------------------------%

\subsection{Mối Liên Hệ Giữa Hydrocarbon \& 1 Số Dẫn Xuất của Hydrocarbon}

%------------------------------------------------------------------------------%

\section{Carbohydrate}

\subsection{Glucose}

%------------------------------------------------------------------------------%

\subsection{Saccarose}

%------------------------------------------------------------------------------%

\subsection{Tinh Bột}

%------------------------------------------------------------------------------%

\subsection{Xenlulozơ}

%------------------------------------------------------------------------------%

\subsection{Cấu Trúc \& Tính Chất của 1 Số Carbohydrate Tiêu Biểu}

%------------------------------------------------------------------------------%

\subsection{Điều Chế Este \& Tính Chất của 1 Số Carbohydrate}

%------------------------------------------------------------------------------%

\section{Amin -- Amino Axit -- Protein}

\subsection{Amin}

%------------------------------------------------------------------------------%

\subsection{Amino Axit}

%------------------------------------------------------------------------------%

\subsection{Peptit \& Protein}

%------------------------------------------------------------------------------%

\subsection{Cấu Tạo \& Tính Chất của Amin, Amino Axit, Protein}

%------------------------------------------------------------------------------%

\subsection{1 Số Tính Chất của Amin, Amino Axit, \& Protein}

%------------------------------------------------------------------------------%

\section{Polyme \& Vật Liệu Polyme}

\subsection{Đại Cương về Polyme}

%------------------------------------------------------------------------------%

\subsection{Vật Liệu Polyme}

%------------------------------------------------------------------------------%

\subsection{Polyme \& Vật Liệu Polyme}

%------------------------------------------------------------------------------%

\section{Đại Cương về Kim Loại}

\subsection{Kim Loại \& Hợp Kim}

%------------------------------------------------------------------------------%

\subsection{Dãy Điện Hóa của Kim Loại}

%------------------------------------------------------------------------------%

\subsection{Tính Chất của Kim Loại}

%------------------------------------------------------------------------------%

\subsection{Sự Điện Phân -- Sự Ăn Mòn Kim Loại -- Điều Chế Kim Loại}

%------------------------------------------------------------------------------%

\subsection{Dãy Điện Hóa của Kim Loại. Điều Chế Kim Loại}

%------------------------------------------------------------------------------%

\subsection{Ăn Mòn Kim Loại. Chống Ăn Mòn Kim Loại}

%------------------------------------------------------------------------------%

\section{Kim Loại Kiềm -- Kim Loại Kiềm Thổ -- Nhôm}

\subsection{Kim Loại Kiềm}

%------------------------------------------------------------------------------%

\subsection{1 Số Hợp Chất Quan Trọng của Kim Loại Kiềm}

%------------------------------------------------------------------------------%

\subsection{Kim Loại Kiềm Thổ}

%------------------------------------------------------------------------------%

\subsection{1 Số Hợp Chất Quan Trọng của Kim Loại Kiềm Thổ}

%------------------------------------------------------------------------------%

\subsection{Tính Chất của Kim Loại Kiềm, Kim Loại Kiềm Thổ}

%------------------------------------------------------------------------------%

\subsection{Nhôm}

%------------------------------------------------------------------------------%

\subsection{1 Số Hợp Chất Quan Trọng của Nhôm}

%------------------------------------------------------------------------------%

\subsection{Tính Chất của Nhôm \& Hợp Chất của Nhôm}

%------------------------------------------------------------------------------%

\subsection{Tính Chất của Kim Loại Kiềm, Kim Loại Kiềm Thổ \& Hợp Chất của Chúng}

%------------------------------------------------------------------------------%

\subsection{Tính Chất của Nhôm \& Hợp Chất của Nhôm}

%------------------------------------------------------------------------------%

\section{Crom -- Sắt -- Đồng}

\subsection{Crom}

%------------------------------------------------------------------------------%

\subsection{1 Số Hợp Chất của Crom}

%------------------------------------------------------------------------------%

\subsection{Sắt}

%------------------------------------------------------------------------------%

\subsection{1 Số Hợp Chất của Sắt}

%------------------------------------------------------------------------------%

\subsection{Hợp Kim của Sắt}

%------------------------------------------------------------------------------%

\subsection{Đồng \& 1 Số Hợp Chất của Đồng}

%------------------------------------------------------------------------------%

\subsection{Sơ Lược về 1 Số Kim Loại Khác}

%------------------------------------------------------------------------------%

\subsection{Tính Chất của Crom, Sắt, \& Những Hợp Chất của Chúng}

%------------------------------------------------------------------------------%

\subsection{Tính Chất của Đồng \& Hợp Chất của Đồng. Sơ Lược về Các Kim Loại Ag, Au, Ni, Zn, Sn, Pb}

%------------------------------------------------------------------------------%

\subsection{Tính Chất Hóa Học của Crom, Sắt, Đồng, \& Những Hợp Chất của Chúng}

%------------------------------------------------------------------------------%

\section{Phân Biệt 1 Số Chất Vô Cơ. Chuẩn Độ Dung Dịch}

\subsection{Nhận Biết 1 Số Cation trong Dung Dịch}

%------------------------------------------------------------------------------%

\subsection{Nhận Biết 1 Số Anion trong Dung Dịch}

%------------------------------------------------------------------------------%

\subsection{Nhận Biết 1 Số Chất Khí}

%------------------------------------------------------------------------------%

\subsection{Chuẩn Độ Acid--Base}

%------------------------------------------------------------------------------%

\subsection{Chuẩn Độ Oxi Hóa--Khử bằng Phương Pháp Pemanganat}

%------------------------------------------------------------------------------%

\subsection{Nhận Biết 1 Số Chất Vô Cơ}

%------------------------------------------------------------------------------%

\subsection{Nhận Biết 1 Số Ion trong Dung Dịch}

%------------------------------------------------------------------------------%

\subsection{Chuẩn Độ Dung Dịch}

%------------------------------------------------------------------------------%

\section{Hóa Học \& Vấn Đề Phát Triển Kinh Tế, Xã Hội, Môi Trường}

\subsection{Hóa Học \& Vấn Đề Phát Triển Kinh Tế}

%------------------------------------------------------------------------------%

\subsection{Hóa Học \& Vấn Đề Xã Hội}

%------------------------------------------------------------------------------%

\subsection{Hóa Học \& Vấn Đề Môi Trường}

%------------------------------------------------------------------------------%

\newpage
\printbibliography[heading=bibintoc]
	
\end{document}