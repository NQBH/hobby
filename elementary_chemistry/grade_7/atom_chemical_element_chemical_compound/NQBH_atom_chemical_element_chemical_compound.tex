\documentclass{article}
\usepackage[backend=biber,natbib=true,style=authoryear,maxbibnames=10]{biblatex}
\addbibresource{/home/nqbh/reference/bib.bib}
\usepackage[utf8]{vietnam}
\usepackage{tocloft}
\renewcommand{\cftsecleader}{\cftdotfill{\cftdotsep}}
\usepackage[colorlinks=true,linkcolor=blue,urlcolor=red,citecolor=magenta]{hyperref}
\usepackage{amsmath,amssymb,amsthm,float,graphicx,mathtools,tikz,tipa}
\usepackage[version=4]{mhchem}
\allowdisplaybreaks
\newtheorem{assumption}{Assumption}
\newtheorem{baitoan}{Bài toán}
\newtheorem{cauhoi}{Câu hỏi}
\newtheorem{conjecture}{Conjecture}
\newtheorem{corollary}{Corollary}
\newtheorem{dangtoan}{Dạng toán}
\newtheorem{definition}{Definition}
\newtheorem{dinhly}{Định lý}
\newtheorem{dinhnghia}{Định nghĩa}
\newtheorem{example}{Example}
\newtheorem{ghichu}{Ghi chú}
\newtheorem{hequa}{Hệ quả}
\newtheorem{hypothesis}{Hypothesis}
\newtheorem{lemma}{Lemma}
\newtheorem{luuy}{Lưu ý}
\newtheorem{nhanxet}{Nhận xét}
\newtheorem{notation}{Notation}
\newtheorem{note}{Note}
\newtheorem{principle}{Principle}
\newtheorem{problem}{Problem}
\newtheorem{proposition}{Proposition}
\newtheorem{question}{Question}
\newtheorem{remark}{Remark}
\newtheorem{theorem}{Theorem}
\newtheorem{vidu}{Ví dụ}
\usepackage[left=1cm,right=1cm,top=5mm,bottom=5mm,footskip=4mm]{geometry}
\def\labelitemii{$\circ$}

\title{Atom, Chemical Element, \& Chemical Compound\\Nguyên Tử, Nguyên Tố Hóa Học, \& Hợp Chất Hóa Học}
\author{Nguyễn Quản Bá Hồng\footnote{Independent Researcher, Ben Tre City, Vietnam\\e-mail: \texttt{nguyenquanbahong@gmail.com}; website: \url{https://nqbh.github.io}.}}
\date{\today}

\begin{document}
\maketitle
\begin{abstract}
	\textsc{[en]} This text is a collection of problems, from easy to advanced, about atom, chemical element, \& chemical compound. This text is also a supplementary material for my lecture note on Elementary Chemistry, which is stored \& downloadable at the following link: \href{https://github.com/NQBH/hobby/blob/master/elementary_chemistry/grade_8/NQBH_elementary_chemistry_grade_8.pdf}{GitHub\texttt{/}NQBH\texttt{/}hobby\texttt{/}elementary chemistry\texttt{/}grade 8\texttt{/}lecture}\footnote{\textsc{url}: \url{https://github.com/NQBH/hobby/blob/master/elementary_chemistry/grade_8/NQBH_elementary_chemistry_grade_8.pdf}.}. The latest version of this text has been stored \& downloadable at the following link: \href{https://github.com/NQBH/hobby/blob/master/elementary_chemistry/chemical_reaction/NQBH_chemical_reaction.pdf}{GitHub\texttt{/}NQBH\texttt{/}hobby\texttt{/}elementary chemistry\texttt{/}grade 8\texttt{/}atom}\footnote{\textsc{url}: \url{https://github.com/NQBH/hobby/blob/master/elementary_chemistry/atom/NQBH_atom.pdf}.}.
	\vspace{2mm}
	
	\textsc{[vi]} Tài liệu này là 1 bộ sưu tập các bài tập chọn lọc từ cơ bản đến nâng cao về nguyên tử, nguyên tố hóa học, \& hợp chất hóa học. Tài liệu này là phần bài tập bổ sung cho tài liệu chính -- bài giảng \href{https://github.com/NQBH/hobby/blob/master/elementary_chemistry/grade_8/NQBH_elementary_chemistry_grade_8.pdf}{GitHub\texttt{/}NQBH\texttt{/}hobby\texttt{/}elementary chemistry\texttt{/}grade 8\texttt{/}lecture} của tác giả viết cho Hóa Học Sơ Cấp. Phiên bản mới nhất của tài liệu này được lưu trữ \& có thể tải xuống ở link sau: \href{https://github.com/NQBH/hobby/blob/master/elementary_chemistry/grade_8/real/NQBH_real.pdf}{GitHub\texttt{/}NQBH\texttt{/}hobby\texttt{/}elementary chemistry\texttt{/}grade 8\texttt{/}atom}.
\end{abstract}
\setcounter{secnumdepth}{4}
\setcounter{tocdepth}{3}
\tableofcontents
\newpage

%------------------------------------------------------------------------------%

\section*{Abbreviation, Convention, \& Notation -- Viết Tắt, Quy Ước, \& Ký Hiệu}

\subsection*{Notation -- Ký Hiệu}

\begin{itemize}
	\item $\%m_{A|A_xB_y}$: \% khối lượng của nguyên tố $A$ trong hợp chất $A_xB_y$, \& được tính bởi công thức $\%m_{A|A_xB_y}\coloneqq\frac{xM_A}{xM_A + yM_B}$.
	\item $m_{A|A_xB_y}$: khối lượng của nguyên tố $A$ trong hợp chất $A_xB_y$, \& được tính bởi công thức $m_{A|A_xB_y}\coloneqq m_{A_xB_y}\cdot\%m_{A|A_xB_y} = m_{A_xB_y}\frac{xM_A}{xM_A + yM_B}$.
\end{itemize}

%------------------------------------------------------------------------------%

\section{Atom -- Nguyên Tử}
\textsf{\textbf{Nội dung.} Mô hình nguyên tử của Rutherford--Bohr -- mô hình sắp xếp electron trong lớp vỏ nguyên tử, khối lượng của 1 nguyên tử theo đơn vị quốc tế amu (đơn vị khối lượng nguyên tử).}
\begin{quotation}
	\textbf{atom} [n] \texttt{/}\textipa{'\ae t@m}\texttt{/}: the smallest particle of a chemical element that can exist.
	
	E.g., the splitting of the atom; 2 atoms of hydrogen with 1 atom of oxygen to form a molecule of water; The scientist Ernest Rutherford was the first person to split the atom; positively charged atoms.
\end{quotation}
Khoảng năm 440 BC, nhà triết học Hy Lạp, Democritus cho rằng nếu chia nhỏ nhiều lần 1 đồng tiền vàng cho đến khi ``không thể phân chia được nữa'', thì sẽ được 1 hạt gọi là \textit{nguyên tử}. (``Nguyên tử'' trong tiếng Hy Lạp là \textit{atomos}, nghĩa là ``không chia nhỏ hơn được nữa'').

\textbf{Kích thước nguyên tử.} Có thể coi nguyên tử như những quả cầu cực nhỏ. Đường kính của nguyên tử nhỏ hơn đường kính của sợi tóc $\approx100000$--$500000$ lần, mà đường kính của sợi tóc là $0.1$mm. Vì thế, không thể quan sát nguyên tử bằng mắt hoặc các kính hiển vi thông thường.

\subsection{Khái niệm nguyên tử}
Các nhà khoa học hiện nay đã tìm thấy hàng chục triệu chất khác nhau. Tuy nhiên, khi phân tích các chất đó, người ta thấy mọi chất đều được cấu tạo từ những \textit{hạt cực kỳ nhỏ bé, không mang điện}. Những hạt đó được gọi là \textit{nguyên tử}.

\begin{vidu}[\cite{SGK_KHTN_7_Canh_Dieu}, p. 10]
	Đồng tiền vàng được cấu tạo từ các nguyên tử \emph{vàng (gold)}. Khí oxygen \emph{\ce{O2}} được cấu tạo từ các\footnote{Khí oxygen gồm rất nhiều phân tử oxygen \ce{O2}, \& mỗi phân tử  oxygen \ce{O2}được cấu tạo từ 2 nguyên tử oxygen O.} nguyên tử oxygen. Kim cương, than chì đều được cấu tạo từ các nguyên tử carbon \emph{C}. Nước được tạo nên từ các nguyên tử hydrogen \emph{H} \& oxygen \emph{O} (phân tử nước có công thức hóa học là \emph{\ce{H2O}}). Đường ăn, có công thức phân tử là \emph{\ce{C12H22O11}} được tạo nên tử các nguyên tử carbon \emph{C}, oxygen \emph{O}, \& hydrogen \emph{H}.
\end{vidu}

\begin{baitoan}[\cite{SGK_KHTN_7_Canh_Dieu}, 1, p. 10]
	Kể tên vài chất có chứa nguyên tử oxygen.
\end{baitoan}

\begin{proof}[Giải]
	Khí oxygen \ce{O2}, khí carbonic \ce{CO2}, nước \ce{H2O}, đường \ce{C12H22O11}, oxide kim loại \ce{M_xO_y} với M là kim loại, e.g., \ce{FeO,Fe2O3}, \ce{Fe3O4,Cu2O,CuO,MgO}, $\ldots$.
\end{proof}

\subsection{Cấu tạo nguyên tử}
Nguyên tử được coi như 1 quả cầu, gồm vỏ nguyên tử \& hạt nhân nguyên tử.
\begin{enumerate}
	\item \textbf{Vỏ nguyên tử.} Vỏ nguyên tử được tạo bởi 1 hay nhiều electron chuyển động xung quanh hạt nhân. Electron ký hiệu là e, mang điện tích âm \& có giá trị bằng 1 điện tích nguyên tố\footnote{1 điện tích nguyên tố $= 1.605\cdot10^{-19}$C, với C là viết tắt của Coulomb.}, được viết đơn giản là $-1$.
	\begin{quotation}
		\textbf{electron} [n] \texttt{/}\textipa{I'lektr6n}\texttt{/}, \texttt{/}\textipa{I'lektrA:n}\texttt{/} (\textit{physics}): a very small piece of matter ($=$ a substance) with a negative electric charge, found in all atoms.
	\end{quotation}
	\item \textbf{Hạt nhân nguyên tử.} Hạt nhân nằm ở tâm \& có kích thước rất nhỏ so với kích thước của nguyên tử. Hạt nhân nguyên tử được tạo bởi các proton \& neutron.
	\begin{enumerate}
		\item Proton ký hiệu là p, mang điện tích dương \& có giá trị bằng 1 điện tích nguyên tố, được viết là $+1$. Điện tích của proton bằng điện tích của electron về độ lớn nhưng khác dấu.
		\item Neutron ký hiệu là n, không mang điện.
	\end{enumerate}
	\begin{quotation}
		\textbf{proton} [n] \texttt{/}\textipa{'pr@Ut6n}\texttt{/}, \texttt{/}\textipa{'pr@UA:n}\texttt{/} (\textit{physics}): a very small piece of matter ($=$ a substance) with a positive electric charge that forms part of the nucleus ($=$ central part) of an atom.
		
		\textbf{neutron} [n] \texttt{/}\textipa{'nju:tr6n}\texttt{/}, \texttt{/}\textipa{'nu:trA:n}\texttt{/} (\textit{physics}): a very small piece of matter ($=$ a substance) that carries no electric charge \& that forms part of the nucleus ($=$ central part) of an atom.
	\end{quotation}
\end{enumerate}
Kích thước của hạt nhân rất nhỏ so với kích thước của nguyên tử. Nếu coi hạt nhân là quả bóng có đường kính là $10$cm thì nguyên tử sẽ là quả cầu khổng lồ với đường kính là 1 km (lớn gấp 10000 lần kích thước của hạt nhân nguyên tử).

Điện tích của hạt nhân nguyên tử bằng tổng điện tích của các proton. Số đơn vị điện tích hạt nhân bằng số proton. Trong nguyên tử, số electron bằng số proton.

\begin{vidu}[\cite{SGK_KHTN_7_Canh_Dieu}, p. 11]
	(a) Nguyên tử nitrogen (nitơ) \emph{N} có $7$ proton nên nitrogen có $7$ electron, có điện tích hạt nhân là $+7$, số đơn vị điện tích hạt nhân là $7$. (b) Nguyên tử helium gồm hạt nhân có $2$ proton, $2$ neutron, \& vỏ nguyên tử có $2$ electron.
	\begin{figure}[H]
		\centering
		\includegraphics[scale=0.4]{helium}
		\caption{Mô hình cấu tạo nguyên tử helium.}
	\end{figure}
\end{vidu}

\begin{baitoan}[\cite{SGK_KHTN_7_Canh_Dieu}, 3, p. 11]
	Trong các hạt cấu tạo nên nguyên tử: (a) Hạt nào mang điện tích âm? (b) Hạt nào mang điện tích dương? (c) Hạt nào không mang điện?
\end{baitoan}

\begin{baitoan}[\cite{SGK_KHTN_7_Canh_Dieu}, 1, p. 11]
	Quan sát mô hình cấu tạo nguyên tử lithium \& hoàn thành thông tin chú thích các thành phần trong cấu tạo nguyên tử lithium.
	\begin{figure}[H]
		\centering
		\includegraphics[scale=0.4]{lithium}
		\caption{Mô hình cấu tạo nguyên tử lithium.}
	\end{figure}
\end{baitoan}
	
\begin{baitoan}[\cite{SGK_KHTN_7_Canh_Dieu}, 2, p. 11]
	Hoàn thành thông tin:
	\begin{table}[H]
		\centering
		\begin{tabular}{|c|c|c|c|c|}
			\hline
			\textbf{Nguyên tử} & \textbf{Số proton} & \textbf{Số neutron} & \textbf{Số electron} & \textbf{Điện tích hạt nhân} \\
			\hline
			Hydrogen & 1 & 0 &  &  \\
			\hline
			Carbon &  & 6 & 6 &  \\
			\hline
			Phosphorus & 15 & 16 &  &  \\
			\hline
		\end{tabular}
	\end{table}
\end{baitoan}

\begin{baitoan}[\cite{SGK_KHTN_7_Canh_Dieu}, 3, p. 12]
	Aluminium \emph{Al} là kim loại có nhiều ứng dụng trong thực tiễn, được dùng làm dây dẫn điện, chế tạo các thiết bị, máy móc trong công nghiệp \& nhiều đồ dùng sinh hoạt. Cho biết tổng số hạt trong hạt nhân nguyên tử aluminium là $27$, số đơn vị điện tích hạt nhân là $13$. Nêu cách tính số hạt mỗi loại trong nguyên tử aluminium \& cho biết điện tích hạt nhân của aluminium.
\end{baitoan}

\begin{vidu}[Điện tích của nguyên tử helium]
	Nguyên tử helium \emph{He} có $2$ proton, mỗi proton có điện tích $+1$, tổng số điện tích (dương): $+2$; có $2$ electron, mỗi electron có điện tích $-1$, tổng số điện tích (âm): $-2$. Tổng điện tích trong nguyên tử helium bằng $(+2) + (-2) = 0$. Ta nói nguyên tử helium \emph{He} \emph{không mang điện} hay \emph{trung hòa về điện}.
\end{vidu}

\begin{baitoan}[\cite{SGK_KHTN_7_Canh_Dieu}, p. 12]
	Cho biết nguyên tử sulfur (lưu huỳnh) có $16$ electron. Hỏi nguyên tử sulfur có bao nhiêu proton? Chứng minh nguyên tử sulfur trung hòa về điện.
\end{baitoan}

\subsection{Sự chuyển động của electron trong nguyên tử}
Theo mô hình của Rutherford--Bohr, trong nguyên tử, các electron chuyển động trên những quỹ đạo xác định xung quanh hạt nhân, như các hành tinh quay quanh Mặt Trời.

Trong nguyên tử, các electron được xếp thành từng lớp. Các electron được sắp xếp lần lượt vào các lớp theo chiều từ gần hạt nhân ra ngoài. Mỗi lớp có số electron tối đa xác định, như lớp thứ nhất có tối đa 2 electron, lớp thứ 2 có tối đa 8 electron, $\ldots$

\begin{vidu}[\cite{SGK_KHTN_7_Canh_Dieu}, p. 12]
	Nguyên tử oxygen \emph{O} có $8$ electron, được phân bố thành 2 lớp electron, lớp thứ nhất có $2$ electron, lớp thứ 2 có $6$ electron. Ta nói nguyên tử oxygen có $6$ electron ở lớp ngoài cùng.
\end{vidu}

\begin{baitoan}[\cite{SGK_KHTN_7_Canh_Dieu}, 4, p. 12]
	Hình sau mô tả thành phần cấu tạo của nguyên tử sodium (natri), ở giữa là hạt nhân, mỗi vòng tròn lớn tiếp theo là 1 lớp electron, mỗi chấm chỉ 1 electron:
	\begin{figure}[H]
		\centering
		\includegraphics[scale=0.4]{sodium}
		\caption{Mô hình cấu tạo nguyên tử sodium.}
	\end{figure}
	\noindent Cho biết nguyên tử sodium có bao nhiêu lớp electron. Mỗi lớp có bao nhiêu electron?
\end{baitoan}
Ernest Rutherford (1871--1937), nhà vật lý người New Zealand, đã đưa ra mô hình hành tinh nguyên tử để giải thích cấu tạo nguyên tử. Năm 1911, ông đã khám phá ra hầu hết các nguyên tử có cấu tạo rỗng, gồm hạt nhân ở giữa tích điện dương \& vỏ nguyên tử gồm các electron tích điện âm. Mô hình hành tinh nguyên tử của Rutherford chưa mô tả được sự phân bố electron trong vỏ nguyên tử. Sau đó, nhà vật lý người Đan Mạch, Niels Bohr đã đề xuất 1 mô hình mới chỉ rõ các electron được sắp xếp trên các lớp khác nhau.

\begin{baitoan}[\cite{SGK_KHTN_7_Canh_Dieu}, 4, p. 13]
	Nguyên tử nitrogen \& silicon có số electron lần lượt là $7$ \& $14$. Cho biết mỗi nguyên tử nitrogen \& silicon có bao nhiêu lớp electron \& có bao nhiêu electron ở lớp ngoài cùng.
\end{baitoan}

\begin{baitoan}[\cite{SGK_KHTN_7_Canh_Dieu}, 5, p. 13]
	Quan sát hình vẽ mô tả cấu tạo nguyên tử carbon \& aluminium:
	\begin{figure}[H]
		\centering
		\includegraphics[scale=0.4]{carbon_aluminium}
		\caption{Mô hình cấu tạo nguyên tử carbon \& nguyên tử aluminium.}
	\end{figure}
	\noindent Cho biết mỗi nguyên tử đó có bao nhiêu lớp electron \& số electron trên mỗi lớp electron đó.
\end{baitoan}
Trong số các nguyên tử đã biết hiện nay, nguyên tử có kích thước lớn nhất là francium, có chứa 7 lớp electron. Nguyên tử helium có kích thước nhỏ nhất với 1 lớp electron.

\subsection{Khối lượng nguyên tử}
Nguyên tử có khối lượng rất nhỏ. 1 gam của bất kỳ chất nào cũng chứa tới hàng tỷ tỷ nguyên tử. Do vậy, để biểu thị khối lượng của nguyên tử, người ta dùng đơn vị khối lượng nguyên tử, ký hiệu là amu (atomic mass unit). 1 amu $= 1.6605\cdot10^{-24}$ g. Khối lượng của 1 nguyên tử bằng tổng khối lượng của proton, neutron, \& electron trong nguyên tử đó.

Proton \& neutron đều có khối lượng xấp xỉ bằng 1 amu. Khối lượng của electron là $0.00055$ amu, nhỏ hơn nhiều lần so với khối lượng của proton \& neutron nên có thể coi khối lượng nguyên tử bằng khối lượng hạt nhân.

\begin{vidu}[\cite{SGK_KHTN_7_Canh_Dieu}, p. 13]
	(a) Nguyên tử hydrogen \emph{H} chỉ có $1$ proton, nên khối lượng nguyên tử hydrogen là $1$ amu. (b) Nguyên tử oxygen có $8$ proton \& $8$ neutron, nên khối lượng nguyên tử oxygen là: $8\cdot1 + 8\cdot1 = 16$ amu.
\end{vidu}

\begin{baitoan}[\cite{SGK_KHTN_7_Canh_Dieu}, 5, p. 13]
	Trong 3 loại hạt tạo nên nguyên tử, hạt nào có khối lượng nhỏ nhất?
\end{baitoan}

\begin{baitoan}[\cite{SGK_KHTN_7_Canh_Dieu}, 6, p. 13]
	Khối lượng của nguyên tử được tính bằng đơn vị nào?
\end{baitoan}

\begin{baitoan}[\cite{SGK_KHTN_7_Canh_Dieu}, 6, p. 13]
	Cho biết: (a) Số proton, neutron, electron trong mỗi nguyên tử carbon \& aluminium. (b) Khối lượng nguyên tử của carbon \& aluminium.
\end{baitoan}

\begin{baitoan}[\cite{SGK_KHTN_7_Canh_Dieu}, 7, p. 14]
	Hoàn thành thông tin còn thiếu trong bảng sau:
	\begin{table}[H]
		\centering
		\begin{tabular}{|c|c|c|c|}
			\hline
			\textbf{Hạt trong nguyên tử} & \textbf{Khối lượng (amu)} & \textbf{Điện tích} & \textbf{Vị trí trong nguyên tử} \\
			\hline
			Proton &  & $+1$ &  \\
			\hline
			Neutron &  &  & Hạt nhân \\
			\hline
			Electron & $0.00055$ &  &  \\
			\hline
		\end{tabular}
	\end{table}
\end{baitoan}

\begin{baitoan}[\cite{SGK_KHTN_7_Canh_Dieu}, p. 14]
	Ruột của bút chì thường được làm từ than chì \& đất sét. Than chì được cấu tạo từ các nguyên tử carbon. (a) Ghi chú thích tên các hạt tương ứng trong mô hình cấu tạo nguyên tử carbon. (b) Tìm hiểu ý nghĩa của các ký hiệu HB, 2B, \& 6B được ghi trên 1 số loại bút chì.
\end{baitoan}
\noindent\textsf{\textbf{Tóm tắt kiến thức.}}
\begin{itemize}
	\item \textit{Nguyên tử} là những hạt cực kỳ nhỏ bé, không mang điện, cấu tạo nên chất. Cấu tạo nguyên tử gồm vỏ nguyên tử \& hạt nhân nguyên tử.
	\item \textit{Hạt nhân của nguyên tử} mang điện tích dương, được tạo bởi các proton \& neutron. Vỏ nguyên tử gồm 1 hay nhiều electron mang điện tích âm.
	\item Theo \textit{mô hình Rutherford--Bohr}, trong nguyên tử, electron phân bố trên các lớp electron \& chuyển động quanh hạt nhân nguyên tử trên những quỹ đạo xác định.
	\item \textit{Khối lượng nguyên tử} được coi bằng tổng khối lượng của proton \& neutron có trong nguyên tử, được tính bằng đơn vị amu.
\end{itemize}

%------------------------------------------------------------------------------%

\section{Chemical Element -- Nguyên Tố Hóa Học}
\textsf{\textbf{Nội dung.} Nguyên tố hóa học, ký hiệu nguyên tố hóa học.}

%------------------------------------------------------------------------------%

\section{Chemical Periodic Table -- Sơ Lược về Bảng Tuần Hoàn Các Nguyên Tố Hóa Học}

%------------------------------------------------------------------------------%

\section{Phân Tử, Đơn Chất, Hợp Chất}

%------------------------------------------------------------------------------%

\section{Giới Thiệu về Liên Kết Hóa Học}

%------------------------------------------------------------------------------%

\section{Hóa Trị, Công Thức Hóa Học}

\begin{dangtoan}
	Từ lượng chất tính lượng nguyên tố.
\end{dangtoan}

\begin{baitoan}[\cite{Tuan2022}, p. 70]
	Tính khối lượng \emph{Fe} \& khối lượng oxi có trong $20$\emph{g} \emph{\ce{Fe2(SO4)3}}.
\end{baitoan}

\begin{proof}[Giải]
	$M_{\ce{Fe2(SO4)3}} = 2\cdot56 + 3(32 + 4\cdot16) = 400$ g\texttt{/}mol$\Rightarrow m_{\ce{Fe|Fe2(SO4)3}} = \%m_{\ce{Fe|Fe2(SO4)3}}\cdot m_{\ce{Fe2(SO4)3}} = \frac{2\cdot56}{2\cdot56 + 3(32 + 4\cdot16)}\cdot20 = 5.6$g$\Rightarrow m_{\ce{O|Fe2(SO4)3}} = m_{\ce{Fe2(SO4)3}}\cdot\%m_{\ce{O|Fe2(SO4)3}} = 20\cdot\frac{12\cdot16}{2\cdot56 + 3(32 + 4\cdot16)} = 9.6$g.
\end{proof}
Dễ dàng tính được khối lượng S trong 20g \ce{Fe2(SO4)3} theo 2 cách: \textit{Cách 1.} Tính theo tỷ lệ \% khối lượng của S trong \ce{Fe2(SO4)3} tương tự lời giải trên: $m_{\ce{S|Fe2(SO4)3}} = m_{\ce{Fe2(SO4)3}}\cdot\%m_{\ce{S|Fe2(SO4)3}} = 20\cdot\frac{3\cdot32}{2\cdot56 + 3(32 + 4\cdot16)} = 4.8$g. \textit{Cách 2.} Sử dụng khối lượng của hợp chất bằng tổng khối lượng của các thành phần: $m_{\ce{S|Fe2(SO4)3}} = m_{\ce{Fe2(SO4)3}} - m_{\ce{Fe|Fe2(SO4)3}} - m_{\ce{O|Fe2(SO4)3}} = 20 - 5.6 - 9.6 = 4.8$g. Dễ thấy Cách 2 tiện hơn sau khi đã biết khối lượng của Fe \& O trong \ce{Fe2(SO4)3}.

\begin{dangtoan}
	Từ lượng nguyên tố tính lượng chất.
\end{dangtoan}

\begin{baitoan}[\cite{Tuan2022}, p. 71]
	Cần bao nhiêu \emph{kg} ure \emph{\ce{(NH2)2CO}} để có $5.6$\emph{kg} đạm (nitơ)?
\end{baitoan}

\begin{proof}[Giải]
	$m_{\ce{(NH2)2CO}} = \frac{m_{\ce{N|(NH2)2CO}}}{\%m_{\ce{N|(NH2)2CO}}} = \frac{5.6\cdot(2(14 + 2) + 12 + 16)}{2\cdot14} = 12$kg.
\end{proof}

\begin{dangtoan}
	Từ lượng nguyên tố này tính lượng nguyên tố kia
\end{dangtoan}

\begin{baitoan}[\cite{Tuan2022}, p. 71]
	Trong supephotphat kép thường có bao nhiêu kg canxi ứng với $49.6$\emph{kg} photpho?
\end{baitoan}

\begin{dangtoan}
	Tính \% khối lượng các nguyên tố trong hợp chất.
\end{dangtoan}

\begin{baitoan}[\cite{Tuan2022}, p. 71]
	Tính \% khối lượng các nguyên tố trong hợp chất sắt(III) sunfat.
\end{baitoan}

\begin{proof}[Giải]
	CTHH của sắt(III) sunfat: \ce{Fe2(SO4)3}$\Rightarrow\%m_{\ce{Fe}}:\%m_{\ce{S}}:\%m_{\ce{O}} = (2\cdot56):(3\cdot32):(12\cdot16) = 112:96:192 = 7:6:12 = 28\%:24\%:48\%$.
\end{proof}

\begin{dangtoan}
	Tìm nguyên tố.
\end{dangtoan}

\begin{baitoan}[\cite{Tuan2022}, p. 71]
	Nguyên tố X trong bảng tuần hoàn có oxit cao nhất dạng \emph{\ce{X2O5}}. Hợp chất khí với hydro của X chứa $8.82$\% khối lượng hydro. X là nguyên tố nào?
\end{baitoan}

\begin{proof}[Giải]
	Nếu oxit cao nhất là \ce{X2O5} thì hợp chất kí với hydro là \ce{XH3}. $M_X = \frac{3}{8.82}\cdot91.18 = 31\Rightarrow$ X: P.
\end{proof}


%------------------------------------------------------------------------------%

\printbibliography[heading=bibintoc]
	
\end{document}