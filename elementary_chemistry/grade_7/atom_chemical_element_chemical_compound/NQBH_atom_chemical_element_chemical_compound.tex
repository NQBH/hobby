\documentclass{article}
\usepackage[backend=biber,natbib=true,style=authoryear,maxbibnames=10]{biblatex}
\addbibresource{/home/nqbh/reference/bib.bib}
\usepackage[utf8]{vietnam}
\usepackage{tocloft}
\renewcommand{\cftsecleader}{\cftdotfill{\cftdotsep}}
\usepackage[colorlinks=true,linkcolor=blue,urlcolor=red,citecolor=magenta]{hyperref}
\usepackage{amsmath,amssymb,amsthm,float,graphicx,mathtools,tikz,tipa}
\usepackage[version=4]{mhchem}
\allowdisplaybreaks
\newtheorem{assumption}{Assumption}
\newtheorem{baitoan}{Bài toán}
\newtheorem{cauhoi}{Câu hỏi}
\newtheorem{conjecture}{Conjecture}
\newtheorem{corollary}{Corollary}
\newtheorem{dangtoan}{Dạng toán}
\newtheorem{definition}{Definition}
\newtheorem{dinhly}{Định lý}
\newtheorem{dinhnghia}{Định nghĩa}
\newtheorem{example}{Example}
\newtheorem{ghichu}{Ghi chú}
\newtheorem{hequa}{Hệ quả}
\newtheorem{hypothesis}{Hypothesis}
\newtheorem{lemma}{Lemma}
\newtheorem{luuy}{Lưu ý}
\newtheorem{nhanxet}{Nhận xét}
\newtheorem{notation}{Notation}
\newtheorem{note}{Note}
\newtheorem{principle}{Principle}
\newtheorem{problem}{Problem}
\newtheorem{proposition}{Proposition}
\newtheorem{question}{Question}
\newtheorem{remark}{Remark}
\newtheorem{theorem}{Theorem}
\newtheorem{vidu}{Ví dụ}
\usepackage[left=1cm,right=1cm,top=5mm,bottom=5mm,footskip=4mm]{geometry}
\def\labelitemii{$\circ$}

\title{Atom, Chemical Element, \& Chemical Compound\\Nguyên Tử, Nguyên Tố Hóa Học, \& Hợp Chất Hóa Học}
\author{Nguyễn Quản Bá Hồng\footnote{Independent Researcher, Ben Tre City, Vietnam\\e-mail: \texttt{nguyenquanbahong@gmail.com}; website: \url{https://nqbh.github.io}.}}
\date{\today}

\begin{document}
\maketitle
\begin{abstract}
	\textsc{[en]} This text is a collection of problems, from easy to advanced, about atom, chemical element, \& chemical compound. This text is also a supplementary material for my lecture note on Elementary Chemistry, which is stored \& downloadable at the following link: \href{https://github.com/NQBH/hobby/blob/master/elementary_chemistry/grade_8/NQBH_elementary_chemistry_grade_8.pdf}{GitHub\texttt{/}NQBH\texttt{/}hobby\texttt{/}elementary chemistry\texttt{/}grade 8\texttt{/}lecture}\footnote{\textsc{url}: \url{https://github.com/NQBH/hobby/blob/master/elementary_chemistry/grade_8/NQBH_elementary_chemistry_grade_8.pdf}.}. The latest version of this text has been stored \& downloadable at the following link: \href{https://github.com/NQBH/hobby/blob/master/elementary_chemistry/chemical_reaction/NQBH_chemical_reaction.pdf}{GitHub\texttt{/}NQBH\texttt{/}hobby\texttt{/}elementary chemistry\texttt{/}grade 8\texttt{/}atom}\footnote{\textsc{url}: \url{https://github.com/NQBH/hobby/blob/master/elementary_chemistry/atom/NQBH_atom.pdf}.}.
	\vspace{2mm}
	
	\textsc{[vi]} Tài liệu này là 1 bộ sưu tập các bài tập chọn lọc từ cơ bản đến nâng cao về nguyên tử, nguyên tố hóa học, \& hợp chất hóa học. Tài liệu này là phần bài tập bổ sung cho tài liệu chính -- bài giảng \href{https://github.com/NQBH/hobby/blob/master/elementary_chemistry/grade_8/NQBH_elementary_chemistry_grade_8.pdf}{GitHub\texttt{/}NQBH\texttt{/}hobby\texttt{/}elementary chemistry\texttt{/}grade 8\texttt{/}lecture} của tác giả viết cho Hóa Học Sơ Cấp. Phiên bản mới nhất của tài liệu này được lưu trữ \& có thể tải xuống ở link sau: \href{https://github.com/NQBH/hobby/blob/master/elementary_chemistry/grade_8/real/NQBH_real.pdf}{GitHub\texttt{/}NQBH\texttt{/}hobby\texttt{/}elementary chemistry\texttt{/}grade 8\texttt{/}atom}.
\end{abstract}
\setcounter{secnumdepth}{4}
\setcounter{tocdepth}{3}
\tableofcontents
\newpage

%------------------------------------------------------------------------------%

\section*{Abbreviation, Convention, \& Notation -- Viết Tắt, Quy Ước, \& Ký Hiệu}

\subsection*{Notation -- Ký Hiệu}

\begin{itemize}
	\item $\%m_{A|A_xB_y}$: \% khối lượng của nguyên tố $A$ trong hợp chất $A_xB_y$, \& được tính bởi công thức $\%m_{A|A_xB_y}\coloneqq\frac{xM_A}{xM_A + yM_B}$.
	\item $m_{A|A_xB_y}$: khối lượng của nguyên tố $A$ trong hợp chất $A_xB_y$, \& được tính bởi công thức $m_{A|A_xB_y}\coloneqq m_{A_xB_y}\cdot\%m_{A|A_xB_y} = m_{A_xB_y}\frac{xM_A}{xM_A + yM_B}$.
\end{itemize}

%------------------------------------------------------------------------------%

\section{Atom -- Nguyên Tử}
\textsf{\textbf{Nội dung.} Mô hình nguyên tử của Rutherford--Bohr -- mô hình sắp xếp electron trong lớp vỏ nguyên tử, khối lượng của 1 nguyên tử theo đơn vị quốc tế amu (đơn vị khối lượng nguyên tử).}
\begin{quotation}
	\textbf{atom} [n] \texttt{/}\textipa{'\ae t@m}\texttt{/}: the smallest particle of a chemical element that can exist.
	
	E.g., the splitting of the atom; 2 atoms of hydrogen with 1 atom of oxygen to form a molecule of water; The scientist Ernest Rutherford was the first person to split the atom; positively charged atoms.
\end{quotation}
Khoảng năm 440 BC, nhà triết học Hy Lạp, Democritus cho rằng nếu chia nhỏ nhiều lần 1 đồng tiền vàng cho đến khi ``không thể phân chia được nữa'', thì sẽ được 1 hạt gọi là \textit{nguyên tử}. (``Nguyên tử'' trong tiếng Hy Lạp là \textit{atomos}, nghĩa là ``không chia nhỏ hơn được nữa'').

\textbf{Kích thước nguyên tử.} Có thể coi nguyên tử như những quả cầu cực nhỏ. Đường kính của nguyên tử nhỏ hơn đường kính của sợi tóc $\approx100000$--$500000$ lần, mà đường kính của sợi tóc là $0.1$mm. Vì thế, không thể quan sát nguyên tử bằng mắt hoặc các kính hiển vi thông thường.

\subsection{Khái niệm nguyên tử}
Các nhà khoa học hiện nay đã tìm thấy hàng chục triệu chất khác nhau. Tuy nhiên, khi phân tích các chất đó, người ta thấy mọi chất đều được cấu tạo từ những \textit{hạt cực kỳ nhỏ bé, không mang điện}. Những hạt đó được gọi là \textit{nguyên tử}.

\begin{vidu}[\cite{SGK_KHTN_7_Canh_Dieu}, p. 10]
	Đồng tiền vàng được cấu tạo từ các nguyên tử \emph{vàng (gold)}. Khí oxygen \emph{\ce{O2}} được cấu tạo từ các\footnote{Khí oxygen gồm rất nhiều phân tử oxygen \ce{O2}, \& mỗi phân tử  oxygen \ce{O2}được cấu tạo từ 2 nguyên tử oxygen O.} nguyên tử oxygen. Kim cương, than chì đều được cấu tạo từ các nguyên tử carbon \emph{C}. Nước được tạo nên từ các nguyên tử hydrogen \emph{H} \& oxygen \emph{O} (phân tử nước có công thức hóa học là \emph{\ce{H2O}}). Đường ăn, có công thức phân tử là \emph{\ce{C12H22O11}} được tạo nên tử các nguyên tử carbon \emph{C}, oxygen \emph{O}, \& hydrogen \emph{H}.
\end{vidu}

\begin{baitoan}[\cite{SGK_KHTN_7_Canh_Dieu}, 1, p. 10]
	Kể tên vài chất có chứa nguyên tử oxygen.
\end{baitoan}

\begin{proof}[Giải]
	Khí oxygen \ce{O2}, khí carbonic \ce{CO2}, nước \ce{H2O}, đường \ce{C12H22O11}, oxide kim loại \ce{M_xO_y} với M là kim loại, e.g., \ce{FeO,Fe2O3}, \ce{Fe3O4,Cu2O,CuO,MgO}, $\ldots$.
\end{proof}

\subsection{Cấu tạo nguyên tử}
Nguyên tử được coi như 1 quả cầu, gồm vỏ nguyên tử \& hạt nhân nguyên tử.
\begin{enumerate}
	\item \textbf{Vỏ nguyên tử.} Vỏ nguyên tử được tạo bởi 1 hay nhiều electron chuyển động xung quanh hạt nhân. Electron ký hiệu là e, mang điện tích âm \& có giá trị bằng 1 điện tích nguyên tố\footnote{1 điện tích nguyên tố $= 1.605\cdot10^{-19}$C, với C là viết tắt của Coulomb.}, được viết đơn giản là $-1$.
	\begin{quotation}
		\textbf{electron} [n] \texttt{/}\textipa{I'lektr6n}\texttt{/}, \texttt{/}\textipa{I'lektrA:n}\texttt{/} (\textit{physics}): a very small piece of matter ($=$ a substance) with a negative electric charge, found in all atoms.
	\end{quotation}
	\item \textbf{Hạt nhân nguyên tử.} Hạt nhân nằm ở tâm \& có kích thước rất nhỏ so với kích thước của nguyên tử. Hạt nhân nguyên tử được tạo bởi các proton \& neutron.
	\begin{enumerate}
		\item Proton ký hiệu là p, mang điện tích dương \& có giá trị bằng 1 điện tích nguyên tố, được viết là $+1$. Điện tích của proton bằng điện tích của electron về độ lớn nhưng khác dấu.
		\item Neutron ký hiệu là n, không mang điện.
	\end{enumerate}
	\begin{quotation}
		\textbf{proton} [n] \texttt{/}\textipa{'pr@Ut6n}\texttt{/}, \texttt{/}\textipa{'pr@UA:n}\texttt{/} (\textit{physics}): a very small piece of matter ($=$ a substance) with a positive electric charge that forms part of the nucleus ($=$ central part) of an atom.
		
		\textbf{neutron} [n] \texttt{/}\textipa{'nju:tr6n}\texttt{/}, \texttt{/}\textipa{'nu:trA:n}\texttt{/} (\textit{physics}): a very small piece of matter ($=$ a substance) that carries no electric charge \& that forms part of the nucleus ($=$ central part) of an atom.
	\end{quotation}
\end{enumerate}
Kích thước của hạt nhân rất nhỏ so với kích thước của nguyên tử. Nếu coi hạt nhân là quả bóng có đường kính là $10$cm thì nguyên tử sẽ là quả cầu khổng lồ với đường kính là 1 km (lớn gấp 10000 lần kích thước của hạt nhân nguyên tử).

Điện tích của hạt nhân nguyên tử bằng tổng điện tích của các proton. Số đơn vị điện tích hạt nhân bằng số proton. Trong nguyên tử, số electron bằng số proton.

\begin{vidu}[\cite{SGK_KHTN_7_Canh_Dieu}, p. 11]
	(a) Nguyên tử nitrogen (nitơ) \emph{N} có $7$ proton nên nitrogen có $7$ electron, có điện tích hạt nhân là $+7$, số đơn vị điện tích hạt nhân là $7$. (b) Nguyên tử helium gồm hạt nhân có $2$ proton, $2$ neutron, \& vỏ nguyên tử có $2$ electron.
	\begin{figure}[H]
		\centering
		\includegraphics[scale=0.4]{helium}
		\caption{Mô hình cấu tạo nguyên tử helium.}
	\end{figure}
\end{vidu}

\begin{baitoan}[\cite{SGK_KHTN_7_Canh_Dieu}, 3, p. 11]
	Trong các hạt cấu tạo nên nguyên tử: (a) Hạt nào mang điện tích âm? (b) Hạt nào mang điện tích dương? (c) Hạt nào không mang điện?
\end{baitoan}

\begin{baitoan}[\cite{SGK_KHTN_7_Canh_Dieu}, 1, p. 11]
	Quan sát mô hình cấu tạo nguyên tử lithium \& hoàn thành thông tin chú thích các thành phần trong cấu tạo nguyên tử lithium.
	\begin{figure}[H]
		\centering
		\includegraphics[scale=0.4]{lithium}
		\caption{Mô hình cấu tạo nguyên tử lithium.}
	\end{figure}
\end{baitoan}
	
\begin{baitoan}[\cite{SGK_KHTN_7_Canh_Dieu}, 2, p. 11]
	Hoàn thành thông tin:
	\begin{table}[H]
		\centering
		\begin{tabular}{|c|c|c|c|c|}
			\hline
			Nguyên tử & Số proton & Số neutron & Số electron & Điện tích hạt nhân \\
			\hline
			Hydrogen & 1 & 0 &  &  \\
			\hline
			Carbon &  & 6 & 6 &  \\
			\hline
			Phosphorus & 15 & 16 &  &  \\
			\hline
		\end{tabular}
	\end{table}
\end{baitoan}

\begin{baitoan}[\cite{SGK_KHTN_7_Canh_Dieu}, 3, p. 12]
	Aluminium \emph{Al} là kim loại có nhiều ứng dụng trong thực tiễn, được dùng làm dây dẫn điện, chế tạo các thiết bị, máy móc trong công nghiệp \& nhiều đồ dùng sinh hoạt. Cho biết tổng số hạt trong hạt nhân nguyên tử aluminium là $27$, số đơn vị điện tích hạt nhân là $13$. Nêu cách tính số hạt mỗi loại trong nguyên tử aluminium \& cho biết điện tích hạt nhân của aluminium.
\end{baitoan}

\begin{vidu}[Điện tích của nguyên tử helium]
	Nguyên tử helium \emph{He} có $2$ proton, mỗi proton có điện tích $+1$, tổng số điện tích (dương): $+2$; có $2$ electron, mỗi electron có điện tích $-1$, tổng số điện tích (âm): $-2$. Tổng điện tích trong nguyên tử helium bằng $(+2) + (-2) = 0$. Ta nói nguyên tử helium \emph{He} \emph{không mang điện} hay \emph{trung hòa về điện}.
\end{vidu}

\begin{baitoan}[\cite{SGK_KHTN_7_Canh_Dieu}, p. 12]
	Cho biết nguyên tử sulfur (lưu huỳnh) có $16$ electron. Hỏi nguyên tử sulfur có bao nhiêu proton? Chứng minh nguyên tử sulfur trung hòa về điện.
\end{baitoan}

\subsection{Sự chuyển động của electron trong nguyên tử}
Theo mô hình của Rutherford--Bohr, trong nguyên tử, các electron chuyển động trên những quỹ đạo xác định xung quanh hạt nhân, như các hành tinh quay quanh Mặt Trời.

Trong nguyên tử, các electron được xếp thành từng lớp. Các electron được sắp xếp lần lượt vào các lớp theo chiều từ gần hạt nhân ra ngoài. Mỗi lớp có số electron tối đa xác định, như lớp thứ nhất có tối đa 2 electron, lớp thứ 2 có tối đa 8 electron, $\ldots$

\begin{vidu}[\cite{SGK_KHTN_7_Canh_Dieu}, p. 12]
	Nguyên tử oxygen \emph{O} có $8$ electron, được phân bố thành 2 lớp electron, lớp thứ nhất có $2$ electron, lớp thứ 2 có $6$ electron. Ta nói nguyên tử oxygen có $6$ electron ở lớp ngoài cùng.
\end{vidu}

\begin{baitoan}[\cite{SGK_KHTN_7_Canh_Dieu}, 4, p. 12]
	Hình sau mô tả thành phần cấu tạo của nguyên tử sodium (natri), ở giữa là hạt nhân, mỗi vòng tròn lớn tiếp theo là 1 lớp electron, mỗi chấm chỉ 1 electron:
	\begin{figure}[H]
		\centering
		\includegraphics[scale=0.4]{sodium}
		\caption{Mô hình cấu tạo nguyên tử sodium.}
	\end{figure}
	\noindent Cho biết nguyên tử sodium có bao nhiêu lớp electron. Mỗi lớp có bao nhiêu electron?
\end{baitoan}
Ernest Rutherford (1871--1937), nhà vật lý người New Zealand, đã đưa ra mô hình hành tinh nguyên tử để giải thích cấu tạo nguyên tử. Năm 1911, ông đã khám phá ra hầu hết các nguyên tử có cấu tạo rỗng, gồm hạt nhân ở giữa tích điện dương \& vỏ nguyên tử gồm các electron tích điện âm. Mô hình hành tinh nguyên tử của Rutherford chưa mô tả được sự phân bố electron trong vỏ nguyên tử. Sau đó, nhà vật lý người Đan Mạch, Niels Bohr đã đề xuất 1 mô hình mới chỉ rõ các electron được sắp xếp trên các lớp khác nhau.

\begin{baitoan}[\cite{SGK_KHTN_7_Canh_Dieu}, 4, p. 13]
	Nguyên tử nitrogen \& silicon có số electron lần lượt là $7$ \& $14$. Cho biết mỗi nguyên tử nitrogen \& silicon có bao nhiêu lớp electron \& có bao nhiêu electron ở lớp ngoài cùng.
\end{baitoan}

\begin{baitoan}[\cite{SGK_KHTN_7_Canh_Dieu}, 5, p. 13]
	Quan sát hình vẽ mô tả cấu tạo nguyên tử carbon \& aluminium:
	\begin{figure}[H]
		\centering
		\includegraphics[scale=0.4]{carbon_aluminium}
		\caption{Mô hình cấu tạo nguyên tử carbon \& nguyên tử aluminium.}
	\end{figure}
	\noindent Cho biết mỗi nguyên tử đó có bao nhiêu lớp electron \& số electron trên mỗi lớp electron đó.
\end{baitoan}
Trong số các nguyên tử đã biết hiện nay, nguyên tử có kích thước lớn nhất là francium, có chứa 7 lớp electron. Nguyên tử helium có kích thước nhỏ nhất với 1 lớp electron.

\subsection{Khối lượng nguyên tử}
Nguyên tử có khối lượng rất nhỏ. 1 gam của bất kỳ chất nào cũng chứa tới hàng tỷ tỷ nguyên tử. Do vậy, để biểu thị khối lượng của nguyên tử, người ta dùng đơn vị khối lượng nguyên tử, ký hiệu là amu (atomic mass unit). 1 amu $= 1.6605\cdot10^{-24}$ g. Khối lượng của 1 nguyên tử bằng tổng khối lượng của proton, neutron, \& electron trong nguyên tử đó.

Proton \& neutron đều có khối lượng xấp xỉ bằng 1 amu. Khối lượng của electron là $0.00055$ amu, nhỏ hơn nhiều lần so với khối lượng của proton \& neutron nên có thể coi khối lượng nguyên tử bằng khối lượng hạt nhân.

\begin{vidu}[\cite{SGK_KHTN_7_Canh_Dieu}, p. 13]
	(a) Nguyên tử hydrogen \emph{H} chỉ có $1$ proton, nên khối lượng nguyên tử hydrogen là $1$ amu. (b) Nguyên tử oxygen có $8$ proton \& $8$ neutron, nên khối lượng nguyên tử oxygen là: $8\cdot1 + 8\cdot1 = 16$ amu.
\end{vidu}

\begin{baitoan}[\cite{SGK_KHTN_7_Canh_Dieu}, 5, p. 13]
	Trong 3 loại hạt tạo nên nguyên tử, hạt nào có khối lượng nhỏ nhất?
\end{baitoan}

\begin{baitoan}[\cite{SGK_KHTN_7_Canh_Dieu}, 6, p. 13]
	Khối lượng của nguyên tử được tính bằng đơn vị nào?
\end{baitoan}

\begin{baitoan}[\cite{SGK_KHTN_7_Canh_Dieu}, 6, p. 13]
	Cho biết: (a) Số proton, neutron, electron trong mỗi nguyên tử carbon \& aluminium. (b) Khối lượng nguyên tử của carbon \& aluminium.
\end{baitoan}

\begin{baitoan}[\cite{SGK_KHTN_7_Canh_Dieu}, 7, p. 14]
	Hoàn thành thông tin còn thiếu trong bảng sau:
	\begin{table}[H]
		\centering
		\begin{tabular}{|c|c|c|c|}
			\hline
			Hạt trong nguyên tử & Khối lượng (amu) & Điện tích & Vị trí trong nguyên tử \\
			\hline
			Proton &  & $+1$ &  \\
			\hline
			Neutron &  &  & Hạt nhân \\
			\hline
			Electron & $0.00055$ &  &  \\
			\hline
		\end{tabular}
	\end{table}
\end{baitoan}

\begin{baitoan}[\cite{SGK_KHTN_7_Canh_Dieu}, p. 14]
	Ruột của bút chì thường được làm từ than chì \& đất sét. Than chì được cấu tạo từ các nguyên tử carbon. (a) Ghi chú thích tên các hạt tương ứng trong mô hình cấu tạo nguyên tử carbon. (b) Tìm hiểu ý nghĩa của các ký hiệu HB, 2B, \& 6B được ghi trên 1 số loại bút chì.
\end{baitoan}
\noindent\fbox{%
	\parbox{\textwidth}{%
		\noindent\textsf{\textbf{Tóm tắt kiến thức.}} \fbox{\bf 1} \textit{Nguyên tử} là những hạt cực kỳ nhỏ bé, không mang điện, cấu tạo nên chất. Cấu tạo nguyên tử gồm vỏ nguyên tử \& hạt nhân nguyên tử. \fbox{\bf 2} \textit{Hạt nhân của nguyên tử} mang điện tích dương, được tạo bởi các proton \& neutron. Vỏ nguyên tử gồm 1 hay nhiều electron mang điện tích âm. \fbox{\bf 3} Theo \textit{mô hình Rutherford--Bohr}, trong nguyên tử, electron phân bố trên các lớp electron \& chuyển động quanh hạt nhân nguyên tử trên những quỹ đạo xác định. \fbox{\bf 4} \textit{Khối lượng nguyên tử} được coi bằng tổng khối lượng của proton \& neutron có trong nguyên tử, được tính bằng đơn vị amu.
	}%
}


%------------------------------------------------------------------------------%

\section{Chemical Element -- Nguyên Tố Hóa Học}
\textsf{\textbf{Nội dung.} Nguyên tố hóa học, ký hiệu nguyên tố hóa học.}

\subsection{Khái niệm nguyên tố hóa học}

\begin{dinhnghia}
	\emph{Nguyên tố hóa học} là tập hợp những nguyên tử có cùng số proton trong hạt nhân.
\end{dinhnghia}

\begin{vidu}[Đồng vị của carbon]
	Hình vẽ sau mô tả những nguyên tử khác nhau nhưng cùng có $6$ proton trong hạt nhân nên thuộc cùng nguyên tố carbon.
	\begin{figure}[H]
		\centering
		\includegraphics[scale=0.4]{carbon}
		\caption{Mô hình cấu tạo các nguyên tử khác nhau thuộc cùng nguyên tố carbon.}
	\end{figure}
\end{vidu}

\begin{baitoan}[\cite{SGK_KHTN_7_Canh_Dieu}, 1, p. 15]
	Các nguyên tử của cùng nguyên tố hóa học có đặc điểm gì giống nhau?
\end{baitoan}
1 nguyên tố hóa học được đặc trưng bởi số proton trong nguyên tử. Các nguyên tử của cùng nguyên tố hóa học đều có tính chất hóa học giống nhau.

Cho đến nay, Liên minh Quốc tế về Hóa học thuần túy \& Hóa học ứng dụng (International Union of Pure \& Applied Chemistry, abbr., IUPAC) đã công bố tìm thấy 118 nguyên tố hóa học, trong đó trên 90 nguyên tố có trong tự nhiên, số còn lại do con người tổng hợp được, gọi là các \textit{nguyên tố nhân tạo}. Hiện nay, các nhà khoa học vẫn đang tiếp tục nghiên cứu để tìm ra những nguyên tố hóa học mới.

\textbf{Các nguyên tố hóa học trong cơ thể con người.} Các chất trong cơ thể chúng ta được tạo thành từ khoảng 25 nguyên tố hóa học, nhưng chủ yếu là các nguyên tố: oxygen, carbon, hydrogen, phosphorus, calcium, nitrogen. Trong đó, nguyên tố calcium có nhiều trong xương \& men răng. Nguyên tố iron (sắt) là thành phần quan trọng của hồng cầu trong máu.

\begin{baitoan}[\cite{SGK_KHTN_7_Canh_Dieu}, 1, p. 16]
	Số lượng mỗi loại hạt của 1 số nguyên tử được nêu trong bảng dưới đây. Cho biết những nguyên tử nào trong bảng thuộc cùng 1 nguyên tố hóa học:
	\begin{table}[H]
		\centering
		\begin{tabular}{|c|c|c|c|}
			\hline
			Nguyên tử & Số proton & Số neutron & Số electron \\
			\hline
			X1 & 8 & 9 & 8 \\
			\hline
			X2 & 7 & 8 & 7 \\
			\hline
			X3 & 8 & 8 & 8 \\
			\hline
			X4 & 6 & 6 & 6 \\
			\hline
			X5 & 7 & 7 & 7 \\
			\hline
			X6 & 11 & 12 & 11 \\
			\hline
			X7 & 8 & 10 & 8 \\
			\hline
			X8 & 6 & 8 & 6 \\
			\hline
		\end{tabular}
	\end{table}
\end{baitoan}

\subsection{Tên nguyên tố hóa học}
Mỗi nguyên tố hóa học đều có tên gọi riêng. Việc đặt tên nguyên tố hóa học dựa vào nhiều cách khác nhau như: liên quan đến tính chất \& ứng dụng của nguyên tố; theo tên các nhà khoa học hoặc theo tên các địa danh.

\begin{vidu}[\cite{SGK_KHTN_7_Canh_Dieu}, p. 16]
	(a) Tên nguyên tố carbon (thành phần chính của than) bắt nguồn từ tiếng Latin, ``carbo'' nghĩa là than. (b) Tên nguyên tố hydrogen bắt nguồn từ tiếng Hy Lạp, nghĩa là tạo ra nước. (c) Tên nguyên tố mendelevium bắt nguồn từ tên nhfa hóa học người Nga D. I. Mendeleev. (d) Tên nguyên tố polonium bắt nguồn từ tên đất nước Ba Lan (Poland).
\end{vidu}
Có 13 nguyên tố hóa học đã quen dùng trong đời sống của người Việt Nam: vàng (gold), bạc (silver), đồng (copper), chì (lead), sắt (iron), nhôm (aluminium), kẽm (zinc), lưu huỳnh (sulfur), thiếc (tin), nitơ (nitrogen), natri (sodium), kali (potassium), \& thủy ngân (mercury). Trong thực tế, các nguyên tố này được dùng cả tên tiếng Việt \& tên tiếng Anh để tiện tra cứu.

\textbf{Bảng: Tên gọi \& ký hiệu của 1 số nguyên tố hóa học.}

\subsection{Ký hiệu hóa học}

\begin{dinhnghia}
	Mỗi nguyên tố hóa học được biểu diễn bằng 1 ký hiệu riêng, được gọi là \emph{ký hiệu hóa học của nguyên tố}.
\end{dinhnghia}
Ký hiệu hóa học của nguyên tố được biểu diễn bằng 1 hoặc 2 chữ cái trong tên nguyên tố. Chữ cái đầu tiên được viết ở dạng chữ in hoa, chữ cái thứ 2 (nếu có) ở dạng chữ thường.

\begin{vidu}
	Ký hiệu hóa học của nguyên tố hydrogen là \emph{H}, của nguyên tố oxygen là \emph{O}, của nguyên tố carbon là \emph{C}, của nguyên tố chlorine là \emph{Cl}, $\ldots$
\end{vidu}

\begin{baitoan}[\cite{SGK_KHTN_7_Canh_Dieu}, 1, p. 17]
	Kể tên \& viết ký hiệu của $3$ nguyên tố hóa học chiếm khối lượng lớn nhất trong vỏ Trái Đất.
\end{baitoan}

\begin{baitoan}[\cite{SGK_KHTN_7_Canh_Dieu}, 2, p. 17]
	Nguyên tố hóa học nào có nhiều nhất trong vũ trụ?
\end{baitoan}

\begin{vidu}
	1 số nguyên tố tố hóa học \& ký hiệu: Iodine \emph{I}, Fluorine \emph{Fl}, Phosphorus \emph{P}, Neon \emph{Ne}, Silicon \emph{Si}, Aluminium \emph{Al}.
\end{vidu}

\begin{baitoan}[\cite{SGK_KHTN_7_Canh_Dieu}, 3, p. 17]
	Đọc \& viết tên các nguyên tố hóa học có ký hiệu: \emph{C, O, Mg, S}.
\end{baitoan}
Trong 1 số trường hợp, ký hiệu hóa học của nguyên tố không tương ứng với tên theo IUPAC.

\begin{vidu}
	(a) Ký hiệu nguyên tố potassium là \emph{K}, bắt nguồn từ tên Latin: kalium. (b) Ký hiệu nguyên tố copper là \emph{Cu}, bắt nguồn từ tên Latin: cuprum.
\end{vidu}

\begin{baitoan}[\cite{SGK_KHTN_7_Canh_Dieu}, 4--5, p. 18]
	Hoàn thành thông tin về tên hoặc ký hiệu hóa học của nguyên tố: (a) \emph{Li}. (b) Helium. (c) \emph{Na}. (d) \emph{Al}. (e) Neon. (f) Phosphorus. (g) \emph{Cl}. (h) \emph{F}.
\end{baitoan}

\begin{baitoan}[\cite{SGK_KHTN_7_Canh_Dieu}, p. 18]
	Calcium là 1 nguyên tố hóa học có nhiều trong xương \& răng, giúp cho xương \& răng chắc khỏe. Ngoài ra, calcium còn cần cho quá trình hoạt động của thần kinh, cơ, tim, chuyển hóa của tế bào \& quá trình đông máu. Thực phẩm \& thuốc bổ chứa nguyên tố calcium giúp phòng ngừa bệnh loãng xương ở tuổi già \& hỗ trợ quá trình phát triển chiều cao của trẻ em. (a) Viết ký hiệu hóa học của nguyên tố calcium \& đọc tên. (b) Kể tên 3 thực phẩm có chứa nhiều calcium.
\end{baitoan}
\noindent\fbox{%
	\parbox{\textwidth}{%
		\noindent\textsf{\textbf{Tóm tắt kiến thức.}} \fbox{\bf 1} \textit{Nguyên tố hóa học} là tập hợp những nguyên tử có cùng số proton trong hạt nhân. \fbox{\bf 2} Mỗi nguyên tố hóa học có tên gọi \& ký hiệu hóa học riêng. \fbox{\bf 3} Ký hiệu hóa học của nguyên tố được biểu diễn bằng 1 hoặc 2 chữ cái trong tên nguyên tố; trong đó, chữ cái đầu tiên được viết ở dạng chữ in hoa, chữ cái thứ 2 (nếu có) được viết ở dạng chữ thường.
	}%
}

%------------------------------------------------------------------------------%

\section{Chemical Periodic Table -- Sơ Lược về Bảng Tuần Hoàn Các Nguyên Tố Hóa Học}
\textsf{\textbf{Nội dung.} Nguyên tắc xây dựng bảng tuần hoàn các nguyên tố hóa học, cấu trúc bảng tuần hoàn: ô, nhóm, chu kỳ, sử dụng bảng tuần hoàn để chỉ ra các nhóm nguyên tố\texttt{/}nguyên tố kim loại, các nhóm nguyên tố\texttt{/}nguyên tố phi kim, nhóm nguyên tố khí hiếm trong bảng tuần hoàn.}

\subsection{Nguyên tắc sắp xếp các nguyên tố hóa học trong bảng tuần hoàn}
Các nguyên tố hóa học được xếp theo quy luật trong 1 bảng, gọi là \textit{bảng tuần hoàn các nguyên tố hóa học} (gọi tắt là \textit{bảng tuần hoàn}). Bảng tuần hoàn hiện nay có 118 nguyên tố hóa học \& được sắp xếp theo nguyên tắc sau:
\begin{itemize}
	\item Các nguyên tố hóa học được xếp theo chiều tăng dần của điện tích hạt nhân nguyên tử.
	\item Các nguyên tố được xếp trong cùng 1 hàng có cùng số lớp electron trong nguyên tử.
	\item Các nguyên tố trong cùng 1 cột có tính chất hóa học tương tự nhau.
\end{itemize}
Năm 1869, nhà bác học Nga D.I. Mendeleev (1834--1907), đã tiến hành nghiên cứu việc phân loại các nguyên tố hóa học. Ông đã phát hiện ra sự thay đổi tuần hoàn tính chất các nguyên tố theo khối lượng nguyên tử của chúng \& sắp xếp 63 nguyên tố hóa học đã biết vào bảng theo chiều tăng dần của khối lượng nguyên tử.

\begin{baitoan}[\cite{SGK_KHTN_7_Canh_Dieu}, 1, p. 20]
	Cho biết số đơn vị điện tích hạt nhân của mỗi nguyên tử \emph{C, Si, O, P, N, S} lần lượt là $6,14,8,15,7,16$. Sắp xếp các nguyên tố trên chiều điện tích hạt nhân tăng dần từ trái sang phải \& từ trên xuống dưới.
\end{baitoan}

\begin{baitoan}[\cite{SGK_KHTN_7_Canh_Dieu}, 1, p. 20]
	Việc tìm ra bảng tuần hoàn là 1 trong những phát hiện xuất sắc nhất trong ngành hóa học. Tìm hiểu lịch sử phát minh ra bảng tuần hoàn các nguyên tố hóa học.
\end{baitoan}

\subsection{Cấu tạo bảng tuần hoàn}
Bảng tuần hoàn gồm các ô được sắp xếp thành các hàng \& các cột.

\subsubsection{Ô nguyên tố}
Mỗi nguyên tố hóa học được xếp vào 1 ô của bảng tuần hoàn, gọi là \textit{ô nguyên tố}. Ô nguyên tố cho biết: số hiệu nguyên tử, ký hiệu hóa học, tên nguyên tố, \& khối lượng nguyên tử của nguyên tố đó.

\begin{baitoan}[\cite{SGK_KHTN_7_Canh_Dieu}, 1, p. 20]
	Hình sau cho biết các thông tin gì về nguyên tố carbon?
	\begin{figure}[H]
		\centering
		\includegraphics[scale=0.3]{carbon_element}
		\caption{Ô nguyên tố carbon.}
	\end{figure}
\end{baitoan}
Số hiệu nguyên tử (ký hiệu là Z) bằng số đơn vị điện tích hạt nhân (bằng số proton \& bằng số electron trong nguyên tử của nguyên tố) \& cũng là số thứ tự của nguyên tố trong bảng tuần hoàn.

\begin{baitoan}[\cite{SGK_KHTN_7_Canh_Dieu}, 1, p. 20]
	Tìm nguyên tố hóa học có số thứ tự lần lượt là $16$ \& $20$ trong bảng tuần hoàn. Đọc tên 2 nguyên tố. Cho biết số hiệu nguyên tử, ký hiệu hóa học, \& khối lượng nguyên tử của 2 nguyên tố đó.
\end{baitoan}

\subsubsection{Chu kỳ}
\textit{Chu kỳ} gồm các nguyên tố mà nguyên tử của chúng có cùng số lớp electron \& được xếp thành hàng theo chiều tăng dần của điện tích hạt nhân. \textit{Số thứ tự của chu kỳ} bằng số lớp electron trong nguyên tử của các nguyên tố trong chu kỳ đó. Bảng tuần hoàn hiện nay gồm 7 chu kỳ, được đánh số từ 1 đến 7.
\begin{itemize}
	\item \textit{Chu kỳ 1} gồm 2 nguyên tố là H \& He. Nguyên tử của các nguyên tố này có 1 lớp electron. Điện tích hạt nhân tăng từ H là $+1$ đến He là $+2$.
	\begin{figure}[H]
		\centering
		\includegraphics[scale=0.3]{hydrogen_helium}
		\caption{Mô hình cấu tạo nguyên tử hydrogen \& helium.}
	\end{figure}
	\item \textit{Chu kỳ 2} gồm 8 nguyên tố từ Li đến Ne. Nguyên tử của các nguyên tố này có 2 lớp electron. Điện tích hạt nhân tăng dần từ Li là $+3$ đến Ne là $+10$.
	\item \textit{Chu kỳ 3} gồm 8 nguyên tố từ Na đến Ar. Nguyên tử của các nguyên tố này có 3 lớp electron. Điện tích hạt nhân tăng dần từ Na là $+11$ đến Ar là $+18$.
	\begin{figure}[H]
		\centering
		\includegraphics[scale=0.3]{sodium_argon}
		\caption{Mô hình cấu tạo nguyên tử sodium \& argon.}
	\end{figure}
\end{itemize}

\begin{baitoan}[\cite{SGK_KHTN_7_Canh_Dieu}, 3, p. 21]
	Quan sát bảng tuần hoàn, cho biết số hiệu nguyên tử lần lượt của nguyên tử carbon \emph{C} \& aluminium \emph{Al}. 2 nguyên tố đó nằm ở chu kỳ nào trong bảng tuần hoàn? Từ đó cho biết số lớp electron của \emph{C} \& \emph{Al}.
\end{baitoan}

\begin{baitoan}[\cite{SGK_KHTN_7_Canh_Dieu}, 2, p. 21]
	Nguyên tố X có số thứ tự $18$ trong bảng tuần hoàn. Cho biết nguyên tố đó ở chu kỳ nào \& có mấy lớp electron.
\end{baitoan}

\begin{baitoan}[\cite{SGK_KHTN_7_Canh_Dieu}, 3, p. 21]
	Dựa vào mô hình cấu tạo nguyên tử sodium \& argon, cho biết 1 số thông tin về nguyên tố sodium \& argon (số hiệu nguyên tử, điện tích hạt nhân, số lớp electron, chu kỳ, số electron ở lớp ngoài cùng).
\end{baitoan}
Trong 1 chu kỳ, khi đi từ trái sang phải theo chiều tăng dần của điện tích hạt nhân: mở đầu chu kỳ là 1 kim loại điển hình (trừ chu kỳ 1), cuối chu kỳ là 1 phi kim điển hình (trừ chu kỳ 7), \& kết thúc chu kỳ là 1 khí hiếm.

\begin{vidu}[\cite{SGK_KHTN_7_Canh_Dieu}, p. 22]
	 Trong chu kỳ 3, mở đầu chu kỳ là nguyên tố sodium \emph{Na}, là 1 kim loại điển hình; cuối chu kỳ là nguyên tố chlorine \emph{Cl}, là 1 phi kim điển hình \& kết thúc chu kỳ là nguyên tố khí hiếm argon \emph{Ar}.
\end{vidu}

\begin{baitoan}[\cite{SGK_KHTN_7_Canh_Dieu}, 3, p. 22]
	Nguyên tố X tạo nên chất khí duy trì sự hô hấp của con người, động vật, thực vật, \& có nhiều trong không khí. Cho biết tên của nguyên tố X. Nguyên tố X nằm ở ô nào \& chu kỳ nào trong bảng tuần hoàn?
\end{baitoan}

\subsubsection{Nhóm}
\textit{Nhóm} gồm các nguyên tố có tính chất hóa học tương tự nhau, được xếp thành cột theo chiều tăng dần của điện tích hạt nhân. Bảng tuần hoàn gồm 18 cột, trong đó có 8 cột là nhóm A \& 10 cột là nhóm B (còn gọi là nhóm các nguyên tố kim loại chuyển tiếp). Nhóm A được đánh số thứ tự bằng số La Mã lần lượt từ nhóm IA đến VIIIA.

Số thứ tự của nhóm A bằng số electron lớp ngoài cùng trong nguyên tử của nguyên tố thuộc nhóm đó.

\begin{baitoan}[\cite{SGK_KHTN_7_Canh_Dieu}, 3, p. 22]
	Quan sát hình sau \& bảng tuần hoàn, cho biết số electron lớp ngoài cùng của nguyên tử \emph{Li} (lithium) \& \emph{Cl} (chlorine).
	\begin{figure}[H]
		\centering
		\includegraphics[scale=0.3]{lithium_chlorine}
		\caption{Mô hình cấu tạo nguyên tử lithium \& chlorine.}
	\end{figure}
	\noindent 2 nguyên tố đó nằm ở nhóm nào trong bảng tuần hoàn?
\end{baitoan}
Nhóm IA gồm các nguyên tố kim loại hoạt động mạnh (kim loại điển hình), trừ hydrogen H. Nguyên tử của chúng đều có 1 electron ở lớp ngoài cùng. Điện tích hạt nhân của các nguyên tử kim loại trong nhóm IA tăng dần từ Li ($+3$) đến Fr ($+87$).

Nhóm VIIA gồm các nguyên tố phi kim hoạt động mạnh (phi kim điển hình), trừ tennessine Ts. Nguyên tử của chúng đều có 7 electron ở lớp ngoài cùng. Điện tích hạt nhân của các nguyên tử phi kim trong nhóm VIIA tăng dần từ F ($+9$) đến At ($+85$).

Nhóm VIIIA gồm các nguyên tố khí hiếm. Nguyên tử của chúng đều có 8 electron ở lớp ngoài cùng (trừ helium). Điện tích hạt nhân tăng dần từ He ($+2$) đến Og ($+118$).

\begin{baitoan}[\cite{SGK_KHTN_7_Canh_Dieu}, 3, p. 23]
	Cho các nguyên tố có số thứ tự lần lượt là $9,18,19$. Số electron lớp ngoài cùng của mỗi nguyên tố trên là bao nhiêu? Cho biết mỗi nguyên tố nằm ở nhóm nào \& đó là kim loại, phi kim hay khí hiếm.
\end{baitoan}

\begin{baitoan}[\cite{SGK_KHTN_7_Canh_Dieu}, p. 23]
	Ngoài $8$ nhóm A, bảng tuần hoàn còn có nhóm B. Tìm hiểu về các nhóm B.
\end{baitoan}

\subsection{Vị trí của các nguyên tố kim loại, phi kim \& khí hiểm trong bảng tuần hoàn}
Các nguyên tố hóa học được chia thành 3 loại: kim loại, phi kim, \& khí hiếm.
\begin{itemize}
	\item \textbf{Các nguyên tố kim loại.} Hơn 80\% các nguyên tố trong bảng tuần hoàn là \textit{kim loại}. Chúng nằm ở phía bên trái \& góc dưới bên phải của bảng tuần hoàn. Các nguyên tố nhóm IA (trừ hydrogen) đều là kim loại điển hình.
	\item \textbf{Các nguyên tố phi kim.} Các nguyên tố nằm ở phía trên, bên phải của bảng tuần hoàn là các \textit{nguyên tố phi kim}. Trong đó, các phi kim hoạt động mạnh nằm ở phía trên. Các nguyên tố nhóm VIIA hầu hết là những phi kim điển hình, fluorine ở đầu nhóm là phi kim hoạt động mạnh nhất.
	\item \textbf{Các nguyên tố khí hiếm.} Tất cả các nguyên tố nằm trong nhóm VIIIA được gọi là \textit{nguyên tố khí hiếm}.
\end{itemize}

\begin{baitoan}[\cite{SGK_KHTN_7_Canh_Dieu}, p. 23]
	Quan sát bảng tuần hoàn các nguyên tố hóa học, cho biết vị trí của các nguyên tố kim loại, phi kim, \& khí hiếm.
\end{baitoan}
Nguyên tố lithium \ce{_7^3Li} nằm ở ô số 3 trong bảng tuần hoàn. Ở điều kiện thường, lithium là kim loại nhẹ nhất. Lithium có nhiều ứng dụng trong cuộc sống: được sử dụng trong chế tạo máy bay, trong y học, đặc biệt được sử dụng chế tạo pin lithium. Pin lithium là 1 loại pin sạc được dùng trong điện thoại, máy tính, máy chụp hình, $\ldots$ Nó được kỳ vọng sẽ thay thế cho acquy chì trong ô tô, xe máy \& các loại xe điện, $\ldots$ góp phần bảo vệ môi trường.

\subsection{Ý nghĩa của bảng tuần hoàn}
Sử dụng bảng tuần hoàn để biết các thông tin của 1 nguyên tố hóa học: tên nguyên tố, số hiệu nguyên tử, ký hiệu hóa học, khối lượng nguyên tử.

Sử dụng bảng tuần hoàn để biết vị trí của nguyên tố hóa học (ô, chu kỳ, nhóm). Từ đó nhận ra được nguyên tố kim loại, phi kim hay khí hiếm.
\begin{itemize}
	\item Các nguyên tố ở nhóm IA, IIA, IIIA là kim loại (trừ hydrogen \& boron).
	\item Hầu hết các nguyên tố ở nhóm VA, VIA, VIIA là phi kim.
	\item Các nguyên tố ở nhóm VIIA là khí hiếm.
\end{itemize}

\begin{vidu}[\cite{SGK_KHTN_7_Canh_Dieu}, p. 24]
	Sử dụng bảng tuần hoàn biết được nguyên tố sulfur \emph{S} ở ô số $16$, chu kỳ $3$, nhóm VIA \& đó là nguyên tố phi kim.
\end{vidu}

\begin{baitoan}[\cite{SGK_KHTN_7_Canh_Dieu}, p. 24]
	Nguyên tố X nằm ở chu kỳ 2, nhóm VA trong bảng tuần hoàn. Cho biết 1 số thông tin của nguyên tố X (tên nguyên tố, ký hiệu hóa học, khối lượng nguyên tử), vị trí ô của nguyên tố trong bảng tuần hoàn. Nguyên tố đó là kim loại, phi kim hay khí hiếm?
\end{baitoan}

\noindent\fbox{%
	\parbox{\textwidth}{%
		\noindent\textsf{\textbf{Tóm tắt kiến thức.}} \fbox{\bf 1} Các nguyên tố hóa học trong bảng tuần hoàn được sắp xếp theo chiều tăng dần của điện tích hạt nhân nguyên tử. Các nguyên tố cùng chu kỳ có cùng số lớp electron. Các nguyên tố cùng nhóm có tính chất hóa học tương tự nhau. \fbox{\bf 2} Bảng tuần hoàn gồm các nguyên tố hóa học mà vị trí được đặc trưng bởi ô nguyên tố, chu kỳ, \& nhóm. \fbox{\bf 3} Bảng tuần hoàn cho biết: các thông tin của 1 nguyên tố; vị trí của các nguyên tố; nguyên tố đó là kim loại, phi kim hay khí hiếm.
	}%
}

\begin{baitoan}[\cite{SGK_KHTN_7_Canh_Dieu}, 1., p. 26]
	Những phát biểu sau nói về đặc điểm của các hạt cấu tạo nên nguyên tử. Cho biết tên hạt ứng với mỗi phát biểu. (a) Hạt mang điện tích dương. (b) Hạt được tìm thấy cùng với proton trong hạt nhân. (c) Hạt có thể xuất hiện với số lượng khác nhau trong các nguyên tử của cùng 1 nguyên tố. (d) Hạt có trong lớp vỏ xung quanh hạt nhân. (e) Hạt mang điện tích âm. (f) Hạt có khối lượng rất nhỏ, có thể bỏ qua khi tính khối lượng nguyên tử. (g) Hạt không mang điện tích.
\end{baitoan}

\begin{baitoan}[\cite{SGK_KHTN_7_Canh_Dieu}, 2., p. 26]
	Điền thông tin thích hợp vào chỗ trống: (a) Hạt nhân của nguyên tử được cấu tạo bởi các hạt $\ldots$ (b) 1 nguyên tử có $17$ proton trong hạt nhân, số electron chuyển động quanh hạt nhân là $\ldots$ (c) 1 nguyên tử có $10$ electron, số proton trong hạt nhân của nguyên tử đó là $\ldots$ (d) Khối lượng nguyên tử X bằng $19$ amu, số electron của nguyên tử đó là $9$. Số neutron của nguyên tử X là $\ldots$ (e) 1 nguyên tử có $3$ proton, $4$ neutron, \& $3$ electron. Khối lượng của nguyên tử đó là $\ldots$
\end{baitoan}

\begin{baitoan}[\cite{SGK_KHTN_7_Canh_Dieu}, 3., p. 26]
	Viết ký hiệu hóa học của các nguyên tố sau: hydrogen, helium, carbon, nitrogen, oxygen, sodium.
\end{baitoan}

\begin{baitoan}[\cite{SGK_KHTN_7_Canh_Dieu}, 4., p. 26]
	Mô hình sắp xếp electron trong nguyên tử của nguyên tố X như sau:
	\begin{figure}[H]
		\centering
		\includegraphics[scale=0.3]{SGK_KHTN_7_CD_p_26}
	\end{figure}
	\noindent(a) Trong nguyên tử X có bao nhiêu electron \& các electron được sắp xếp thành mấy lớp? (b) Cho biết tên nguyên tố X. (c) Gọi tên 1 nguyên tố khác mà nguyên tử của nó có cùng số lớp electron với nguyên tử nguyên tố X.
\end{baitoan}

\begin{baitoan}[\cite{SGK_KHTN_7_Canh_Dieu}, 5., p. 27]
	Hoàn thành những thông tin còn thiếu trong bảng sau:
	\begin{table}[H]
		\centering
		\begin{tabular}{|c|c|c|c|c|c|}
			\hline
			 Tên nguyên tố & Ký hiệu hóa học & Số proton & Số neutron & Số electron & Khối lượng nguyên tử (amu) \\
			\hline
			&  &  & 10 & 9 &  \\
			\hline
			Sulfur &  &  &  & 16 & 32 \\
			\hline
			&  & 12 &  &  & 24 \\
			\hline
			&  & 1 &  &  & 2 \\
			\hline
			&  &  &  & 11 & 23 \\
			\hline
		\end{tabular}
	\end{table}
\end{baitoan}

\begin{baitoan}[\cite{SGK_KHTN_7_Canh_Dieu}, 6., p. 27]
	Số proton \& số neutron của 2 nguyên tử X, Y được cho trong bảng sau:
	\begin{table}[H]
		\centering
		\begin{tabular}{|c|c|c|}
			\hline
			Nguyên tử & X & Y \\
			\hline
			Số proton & 6 & 6 \\
			\hline
			Số neutron & 6 & 8 \\
			\hline
		\end{tabular}
	\end{table}
	\noindent(a) Tính khối lượng của nguyên tử X \& nguyên tử Y. (b) Nguyên tử X \& nguyên tử Y có thuộc cùng 1 nguyên tố hóa học không? Vì sao?
\end{baitoan}

\begin{baitoan}[\cite{SGK_KHTN_7_Canh_Dieu}, 7., p. 27]
	Cho các nguyên tố sau: \emph{Ca, S, Na, Mg, F, Ne}. Sử dụng bảng tuần hoàn các nguyên tố hóa học: (a) Sắp xếp các nguyên tố trên theo chiều tăng dần điện tích hạt nhân. (b) Cho biết mỗi nguyên tố trong dãy trên là kim loại, phi kim hay khí hiếm.
\end{baitoan}

\begin{baitoan}[\cite{SGK_KHTN_7_Canh_Dieu}, 8., p. 27]
	Dựa vào bảng tuần hoàn, cho biết 1 số thông tin của các nguyên tố có số hiệu nguyên tử lần lượt là $12,15,18$. Điền các thông tin theo bảng sau:
	\begin{table}[H]
		\centering
		\begin{tabular}{|c|c|c|c|c|c|p{3.5cm}|}
			\hline
			Số hiệu nguyên tử & Tên nguyên tố & Ký hiệu hóa học & Khối lượng nguyên tử & Chu kỳ & Nhóm & Kim loại, phi kim hay khí hiếm? \\
			\hline
			12 &  &  &  &  &  &  \\
			\hline
			15 &  &  &  &  &  &  \\
			\hline
			18 &  &  &  &  &  &  \\
			\hline
		\end{tabular}
	\end{table}
\end{baitoan}

\begin{baitoan}[\cite{SGK_KHTN_7_Canh_Dieu}, 9., p. 27]
	Biết nguyên tử của nguyên tố M có $3$ lớp electron \& có $2$ electron ở lớp ngoài cùng. Xác định vị trí của M trong bảng tuần hoàn (ô, chu kỳ, nhóm) \& cho biết M là kim loại, phi kim hay khí hiếm.
\end{baitoan}

%------------------------------------------------------------------------------%

\section{Molecule, Compound -- Phân Tử, Đơn Chất, Hợp Chất}
\textsf{\textbf{Nội dung.} Phân tử, đơn chất, hợp chất, tính khối lượng phân tử theo đơn vị amu.}

Ta cảm nhận được mùi thơm của nhiều loại hoa, quả chín là do 1 số chất có trong hoa, quả chín tách ra những hạt rất nhỏ, lan tỏa vào không khí, tác động lên khứu giác của con người. Những hạt như vậy được gọi là \textit{phân tử}.

\subsection{Phân tử}

\subsubsection{Khái niệm phân tử}

\begin{vidu}[Sự lan tỏa của iodine]
	Lấy 1 lượng nhỏ iodine cho vào bình tam giác không màu, đậy kín lại, sau đó đặt vào cốc nước ấm \& quan sát. Ta thấy xuất hiện màu tím ở trong bình. Hiện tượng này là do iodine đã tách ra thành những hạt màu tím vô cùng nhỏ lan tỏa trong bình. Những hạt đó được gọi là phân tử. Với iodine, mỗi phân tử gồm 2 nguyên tử gắn kết với nhau bằng liên kết hóa học.
\end{vidu}

\begin{vidu}[Đường tan trong nước]
	Nếu cho 1 lượng nhỏ đường ăn \emph{\ce{C12H22O11}} vào cốc đựng nước rồi khuấy, sau 1 thời gian sẽ không nhìn thấy đường trong cốc \& dung dịch trong cốc có vị ngọt. Sở dĩ như vậy là do các hạt đường ban đầu đã tách ra thành các phân tử đường \& lan tỏa vào trong nước. Mỗi phân tử đường gồm nhiều nguyên tử \emph{C, H}, \& \emph{O} liên kết với nhau.
\end{vidu}

\begin{vidu}
	Khi để cốc nước \emph{\ce{H2O}} trong không khí, nước sẽ cạn dần. Đó là do các phân tử nước tách ra, tỏa vào không khí. Mỗi phân tử nước gồm 2 nguyên tử \emph{H} \& 1 nguyên tử \emph{O}.
\end{vidu}
Trong 3 ví dụ trên, iodine, đường, \& nước đều do các phân tử hợp thành. Các phân tử của 1 chất giống nhau về thành phần \& hình dạng. E.g., nước được hợp thành từ các phân tử có 2 nguyên tử H, 1 nguyên tử O, \& có dạng gấp khúc.

\begin{baitoan}[\cite{SGK_KHTN_7_Canh_Dieu}, 1, p. 29]
	Giải thích 1 số hiện tượng sau: (a) Khi mở lọ nước hoa hoặc mở lọ đựng 1 số loại tinh dầu sẽ ngửi thấy có mùi thơm. (b) Quần áo sau khi giặt xong, phơi trong không khí 1 thời gian sẽ khô.
\end{baitoan}

\begin{baitoan}[\cite{SGK_KHTN_7_Canh_Dieu}, 1, p. 29]
	Khi nói về nước, có 2 ý kiến sau: (a) Phân tử nước trong nước đá, nước lỏng, \& hơi nước là giống nhau. (b) Phân tử nước trong nước đá, nước lỏng, \& hơi nước là khác nhau. Ý kiến nào là đúng? Vì sao?
\end{baitoan}
Tính chất hóa học của chất chính là tính chất hóa học của phân tử tạo thành chất đó.

\begin{dinhnghia}[Phân tử]
	\emph{Phân tử} là hạt đại diện cho chất, gồm 1 số nguyên tử gắn kết với nhau bằng liên kết hóa học \& thể hiện đầy đủ tính chất hóa học của chất.
\end{dinhnghia}

\begin{baitoan}[\cite{SGK_KHTN_7_Canh_Dieu}, 1, p. 29]
	\emph{Đ\texttt{/}S?} (a) Trong 1 phân tử, các nguyên tử luôn giống nhau. (b) Trong 1 phân tử, các nguyên tử luôn khác nhau. (c) Trong 1 phân tử, các nguyên tử có thể giống nhau hoặc khác nhau.
\end{baitoan}

\begin{baitoan}[\cite{SGK_KHTN_7_Canh_Dieu}, 1, p. 29]
	1 số nhiên liệu như xăng, dầu, $\ldots$ dễ tách ra các phân tử \& lan tỏa trong không khí. Cần bảo quản các nhiên liệu trên như thế nào để bảo đảm an toàn?
\end{baitoan}

\subsubsection{Khối lượng phân tử}
Khối lượng phân tử bằng tổng khối lượng các nguyên tử có trong phân tử. Đơn vị của khối lượng phân tử là amu.

\begin{vidu}[Cách tính khối lượng phân tử nước]
	Xác định số nguyên tử của mỗi nguyên tố: Phân tử nước gồm $2$ nguyên tử \emph{H} \& $1$ nguyên tử \emph{O}. Khối lượng phân tử nước: $2\cdot1 +1\cdot16 = 18$ amu.
\end{vidu}

\begin{baitoan}[\cite{SGK_KHTN_7_Canh_Dieu}, 2, p. 30]
	Tính khối lượng phân tử của fluorine \emph{\ce{F2}} \& methane \emph{\ce{CH4}}.
\end{baitoan}

\subsection{Đơn chất}
1 số chất khí có mô hình phân tử như sau: (a) Hydrogen \ce{H2}. (b) Nitrogen \ce{N2}. (c) Chlorine \ce{Cl2}.

\subsection{Hợp chất}

%------------------------------------------------------------------------------%

\section{Giới Thiệu về Liên Kết Hóa Học}

%------------------------------------------------------------------------------%

\section{Hóa Trị, Công Thức Hóa Học}

\begin{dangtoan}
	Từ lượng chất tính lượng nguyên tố.
\end{dangtoan}

\begin{baitoan}[\cite{Tuan2022}, p. 70]
	Tính khối lượng \emph{Fe} \& khối lượng oxi có trong $20$\emph{g} \emph{\ce{Fe2(SO4)3}}.
\end{baitoan}

\begin{proof}[Giải]
	$M_{\ce{Fe2(SO4)3}} = 2\cdot56 + 3(32 + 4\cdot16) = 400$ g\texttt{/}mol$\Rightarrow m_{\ce{Fe|Fe2(SO4)3}} = \%m_{\ce{Fe|Fe2(SO4)3}}\cdot m_{\ce{Fe2(SO4)3}} = \frac{2\cdot56}{2\cdot56 + 3(32 + 4\cdot16)}\cdot20 = 5.6$g$\Rightarrow m_{\ce{O|Fe2(SO4)3}} = m_{\ce{Fe2(SO4)3}}\cdot\%m_{\ce{O|Fe2(SO4)3}} = 20\cdot\frac{12\cdot16}{2\cdot56 + 3(32 + 4\cdot16)} = 9.6$g.
\end{proof}
Dễ dàng tính được khối lượng S trong 20g \ce{Fe2(SO4)3} theo 2 cách: \textit{Cách 1.} Tính theo tỷ lệ \% khối lượng của S trong \ce{Fe2(SO4)3} tương tự lời giải trên: $m_{\ce{S|Fe2(SO4)3}} = m_{\ce{Fe2(SO4)3}}\cdot\%m_{\ce{S|Fe2(SO4)3}} = 20\cdot\frac{3\cdot32}{2\cdot56 + 3(32 + 4\cdot16)} = 4.8$g. \textit{Cách 2.} Sử dụng khối lượng của hợp chất bằng tổng khối lượng của các thành phần: $m_{\ce{S|Fe2(SO4)3}} = m_{\ce{Fe2(SO4)3}} - m_{\ce{Fe|Fe2(SO4)3}} - m_{\ce{O|Fe2(SO4)3}} = 20 - 5.6 - 9.6 = 4.8$g. Dễ thấy Cách 2 tiện hơn sau khi đã biết khối lượng của Fe \& O trong \ce{Fe2(SO4)3}.

\begin{dangtoan}
	Từ lượng nguyên tố tính lượng chất.
\end{dangtoan}

\begin{baitoan}[\cite{Tuan2022}, p. 71]
	Cần bao nhiêu \emph{kg} ure \emph{\ce{(NH2)2CO}} để có $5.6$\emph{kg} đạm (nitơ)?
\end{baitoan}

\begin{proof}[Giải]
	$m_{\ce{(NH2)2CO}} = \frac{m_{\ce{N|(NH2)2CO}}}{\%m_{\ce{N|(NH2)2CO}}} = \frac{5.6\cdot(2(14 + 2) + 12 + 16)}{2\cdot14} = 12$kg.
\end{proof}

\begin{dangtoan}
	Từ lượng nguyên tố này tính lượng nguyên tố kia
\end{dangtoan}

\begin{baitoan}[\cite{Tuan2022}, p. 71]
	Trong supephotphat kép thường có bao nhiêu kg canxi ứng với $49.6$\emph{kg} photpho?
\end{baitoan}

\begin{dangtoan}
	Tính \% khối lượng các nguyên tố trong hợp chất.
\end{dangtoan}

\begin{baitoan}[\cite{Tuan2022}, p. 71]
	Tính \% khối lượng các nguyên tố trong hợp chất sắt(III) sunfat.
\end{baitoan}

\begin{proof}[Giải]
	CTHH của sắt(III) sunfat: \ce{Fe2(SO4)3}$\Rightarrow\%m_{\ce{Fe}}:\%m_{\ce{S}}:\%m_{\ce{O}} = (2\cdot56):(3\cdot32):(12\cdot16) = 112:96:192 = 7:6:12 = 28\%:24\%:48\%$.
\end{proof}

\begin{dangtoan}
	Tìm nguyên tố.
\end{dangtoan}

\begin{baitoan}[\cite{Tuan2022}, p. 71]
	Nguyên tố X trong bảng tuần hoàn có oxit cao nhất dạng \emph{\ce{X2O5}}. Hợp chất khí với hydro của X chứa $8.82$\% khối lượng hydro. X là nguyên tố nào?
\end{baitoan}

\begin{proof}[Giải]
	Nếu oxit cao nhất là \ce{X2O5} thì hợp chất kí với hydro là \ce{XH3}. $M_X = \frac{3}{8.82}\cdot91.18 = 31\Rightarrow$ X: P.
\end{proof}


%------------------------------------------------------------------------------%

\printbibliography[heading=bibintoc]
	
\end{document}