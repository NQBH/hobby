\documentclass{article}
\usepackage[backend=biber,natbib=true,style=alphabetic,maxbibnames=50]{biblatex}
\addbibresource{/home/nqbh/reference/bib.bib}
\usepackage[utf8]{vietnam}
\usepackage{tocloft}
\renewcommand{\cftsecleader}{\cftdotfill{\cftdotsep}}
\usepackage[colorlinks=true,linkcolor=blue,urlcolor=red,citecolor=magenta]{hyperref}
\usepackage{amsmath,amssymb,amsthm,float,graphicx,mathtools,tikz,tipa}
\usepackage[version=4]{mhchem}
\allowdisplaybreaks
\newtheorem{assumption}{Assumption}
\newtheorem{baitoan}{Bài toán}
\newtheorem{cauhoi}{Câu hỏi}
\newtheorem{conjecture}{Conjecture}
\newtheorem{corollary}{Corollary}
\newtheorem{dangtoan}{Dạng toán}
\newtheorem{definition}{Definition}
\newtheorem{dinhly}{Định lý}
\newtheorem{dinhnghia}{Định nghĩa}
\newtheorem{example}{Example}
\newtheorem{ghichu}{Ghi chú}
\newtheorem{hequa}{Hệ quả}
\newtheorem{hypothesis}{Hypothesis}
\newtheorem{lemma}{Lemma}
\newtheorem{luuy}{Lưu ý}
\newtheorem{nhanxet}{Nhận xét}
\newtheorem{notation}{Notation}
\newtheorem{note}{Note}
\newtheorem{principle}{Principle}
\newtheorem{problem}{Problem}
\newtheorem{proposition}{Proposition}
\newtheorem{question}{Question}
\newtheorem{remark}{Remark}
\newtheorem{theorem}{Theorem}
\newtheorem{vidu}{Ví dụ}
\usepackage[left=1cm,right=1cm,top=5mm,bottom=5mm,footskip=4mm]{geometry}

\title{Problem {\it\&} Solution: Atom, Chemical Element, \textit{\&} Chemical Compound\\Bài Tập {\it\&} Lời Giải: Nguyên Tử, Nguyên Tố Hóa Học, \textit{\&} Hợp Chất Hóa Học}
\author{Nguyễn Quản Bá Hồng\footnote{Independent Researcher, Ben Tre City, Vietnam\\e-mail: \texttt{nguyenquanbahong@gmail.com}; website: \url{https://nqbh.github.io}.}}
\date{\today}

\begin{document}
\maketitle
\begin{abstract}
	\textsc{[en]} This text is a collection of problems, from easy to advanced, about atom, chemical element, \& chemical compound. This text is also a supplementary material for my lecture note on Elementary Chemistry, which is stored \& downloadable at the following link: \href{https://github.com/NQBH/hobby/blob/master/elementary_chemistry/grade_7/NQBH_elementary_chemistry_grade_7.pdf}{GitHub\texttt{/}NQBH\texttt{/}hobby\texttt{/}elementary chemistry\texttt{/}grade 7\texttt{/}lecture}\footnote{\textsc{url}: \url{https://github.com/NQBH/hobby/blob/master/elementary_chemistry/grade_7/NQBH_elementary_chemistry_grade_7.pdf}.}. The latest version of this text has been stored \& downloadable at the following link: \href{https://github.com/NQBH/hobby/blob/master/elementary_chemistry/chemical_reaction/NQBH_chemical_reaction.pdf}{GitHub\texttt{/}NQBH\texttt{/}hobby\texttt{/}elementary chemistry\texttt{/}grade 7\texttt{/}atom}\footnote{\textsc{url}: \url{https://github.com/NQBH/hobby/blob/master/elementary_chemistry/atom/NQBH_atom.pdf}.}.
	\vspace{2mm}
	
	\textsc{[vi]} Tài liệu này là 1 bộ sưu tập các bài tập chọn lọc từ cơ bản đến nâng cao về nguyên tử, nguyên tố hóa học, \& hợp chất hóa học. Tài liệu này là phần bài tập bổ sung cho tài liệu chính -- bài giảng \href{https://github.com/NQBH/hobby/blob/master/elementary_chemistry/grade_7/NQBH_elementary_chemistry_grade_7.pdf}{GitHub\texttt{/}NQBH\texttt{/}hobby\texttt{/}elementary chemistry\texttt{/}grade 7\texttt{/}lecture} của tác giả viết cho Hóa Học Sơ Cấp. Phiên bản mới nhất của tài liệu này được lưu trữ \& có thể tải xuống ở link sau: \href{https://github.com/NQBH/hobby/blob/master/elementary_chemistry/grade_7/real/NQBH_real.pdf}{GitHub\texttt{/}NQBH\texttt{/}hobby\texttt{/}elementary chemistry\texttt{/}grade 7\texttt{/}atom}.
\end{abstract}
\setcounter{secnumdepth}{4}
\setcounter{tocdepth}{3}
\tableofcontents
\newpage

%------------------------------------------------------------------------------%

\section{Atom -- Nguyên Tử}

\begin{baitoan}[\cite{SGK_KHTN_7_Canh_Dieu}, p. 10]
	Khoảng năm 440 B.C., nhà triết học Hy Lạp Democritus cho rằng nếu chia nhỏ nhiều lần 1 đồng tiền vàng cho đến khi ``không thể phân chia được nữa'', thì sẽ được 1 hạt gọi là {\rm nguyên tử}. (``Nguyên tử'' trong tiếng Hy Lạp là atomos, i.e., ``không chia nhỏ hơn được nữa''). Vậy nguyên tử có phải là hạt nhỏ nhất không?
\end{baitoan}

\begin{proof}[1st giải]
	Nguyên tử không phải là hạt nhỏ nhất. Trong nguyên tử còn có các hạt: electron, proton, neutron.
\end{proof}
1 lời giải khác, chi tiết hơn sau khi tra Google để biết thêm:
\begin{proof}[2nd giải]
	``Nguyên tử không phải là hạt nhỏ nhất, nó được cấu tạo bởi 1 hạt nhân trung tâm \& các electron (điện tử) chuyển động xung quanh trên các quỹ đạo có năng lượng xác định (mẫu nguyên tử của Borth). Hạt nhân nguyên tử được cấu tạo bởi các hạt baryon gồm 2 loại là proton \& neutron. Trong 1 thời gian dài, 3 loại hạt nêu trên (neutron, proton, \& electron) được coi là thành phần cơ bản của vật chất. Nhưng sau đó thì các tương tác cơ bản được cho rằng đều được truyền bởi các loại hạt truyền gọi chung là các \textit{boson} \& dần dần các loại hạt này cũng lần lượt được xác minh bằng các thực nghiệm. Hiện nay, người ta cũng biết rằng, proton \& neutron được cấu tạo từ các hạt nhỏ hơn, mỗi proton hoặc neutron được tạo thành bởi 3 hạt quark. \& tất nhiên, cho tới ngày nay, việc có hạt nào nhỏ hơn quark hay không thì không hoàn toàn chắc chắn.'' [$\ldots$] ``Thế giới hạt cơ bản hiện nay đã được xác nhận chia ra làm 2 nhóm chính là \textit{fermion} (các hạt tạo nên vật chất trong vũ trụ) \& \textit{boson} (các hạt truyền tương tác).'' -- \href{https://giaoducthoidai.vn/hon-2000-nam-truy-tim-hat-co-ban-post444110.html}{Giáo dục \& Thời Đại{\tt/}Hơn 2000 năm truy tìm hạt cơ bản}.
\end{proof}
``Trong khoa học vật lý, các \textit{hạt hạ nguyên tử} (tiếng Anh: \textit{subatomic particle}) là các hạt nhỏ hơn nhiều lần so với các nguyên tử, là 1 khái niệm để chỉ các hạt cấu thành nên nguyên tử, cùng các hạt được giải phóng trong các phản ứng hạt nhân hay phản ứng phân rã. E.g., electron, proton, neutron là các hạt hạ nguyên tử thường được nhắc đến. Có 2 loại hạt hạ nguyên tử: \textit{hạt sơ cấp}, không được cấu tạo từ các hạt khác, \& \textit{hạt tổ hợp}. \textit{Vật lý hạt} \& \textit{vật lý hạt nhân} nghiên cứu các hạt này \& các chúng tương tác với nhau.'' -- \href{https://vi.wikipedia.org/wiki/H%E1%BA%A1t_h%E1%BA%A1_nguy%C3%AAn_t%E1%BB%AD}{Wikipedia{\tt/}hạt hạ nguyên tử}

\begin{baitoan}[\cite{SGK_KHTN_7_Canh_Dieu}, 1, p. 10]
	Nguyên tử là gì? Cho ví dụ.
\end{baitoan}

\begin{proof}[Giải]
	Nguyên tử là những hạt cực kì nhỏ, không mang điện, cấu tạo nên mọi chất. E.g.: Khối vàng nguyên chất được cấu tạo từ nguyên tử vàng. Kim cương, than chì đều được cấu tạo từ nguyên tử carbon. Nước được tạo nên tử các nguyên tử hydrogen H \& oxygen O.
\end{proof}

\begin{baitoan}[\cite{SGK_KHTN_7_Canh_Dieu}, 2, p. 10]
	Kể tên vài chất có chứa nguyên tử oxygen.
\end{baitoan}

\begin{proof}[Giải]
	Khí oxygen \ce{O2}, khí ozon \ce{O3}, carbonic \ce{CO2}, nước \ce{H2O}, đường ăn \ce{C12H22O11}, oxide kim loại \ce{M_xO_y} với M là kim loại, e.g., \ce{FeO,Fe2O3}, \ce{Fe3O4,Cu2O,CuO,MgO}, $\ldots$; oxide phi kim, e.g., CO, \ce{CO2,SO3,NO,NO2,N2O5,P2O5}; các base \ce{M(OH)n} với M là kim loại hóa trị $n\in\mathbb{N}^\star$ luôn có chứa gốc hydroxide OH \& gốc hydroxide này chứa nguyên tử oxygen O.
\end{proof}

\begin{baitoan}[\cite{SGK_KHTN_7_Canh_Dieu}, 3, p. 11]
	Trong các hạt cấu tạo nên nguyên tử: (a) Hạt nào mang điện tích âm? (b) Hạt nào mang điện tích dương? (c) Hạt nào không mang điện?
\end{baitoan}

\begin{proof}[Giải]
	(a) Hạt electron e mang điện tích âm. (b) Hạt proton p mang điện tích dương. (c) Hạt neutron n không mang điện.
\end{proof}

\begin{baitoan}[\cite{SGK_KHTN_7_Canh_Dieu}, 1, p. 11]
	Quan sát mô hình cấu tạo nguyên tử lithium \& hoàn thành thông tin chú thích các thành phần trong cấu tạo nguyên tử lithium.
	\begin{figure}[H]
		\centering
		\includegraphics[scale=0.4]{lithium}
		\caption{Mô hình cấu tạo nguyên tử lithium.}
	\end{figure}
\end{baitoan}

\begin{proof}[Giải]
	(1) Electron. (2) Hạt nhân. (3) Proton. (4) Neutron.
\end{proof}

\begin{baitoan}[\cite{SGK_KHTN_7_Canh_Dieu}, 2, p. 11]
	Hoàn thành thông tin:
	\begin{table}[H]
		\centering
		\begin{tabular}{|c|c|c|c|c|}
			\hline
			Nguyên tử & Số proton & Số neutron & Số electron & Điện tích hạt nhân \\
			\hline
			Hydrogen & 1 & 0 &  &  \\
			\hline
			Carbon &  & 6 & 6 &  \\
			\hline
			Phosphorus & 15 & 16 &  &  \\
			\hline
		\end{tabular}
	\end{table}
	\noindent Trong bảng trên, nếu không cho trước số neutron, liệu có xác định được số neutron bằng các số liệu khác không? Vì sao?
\end{baitoan}

\begin{proof}[Giải]
	Trong nguyên tử, số electron bằng số proton \& điện tích hạt nhân nguyên tử bằng tổng điện tích các proton. Hydrogen có số proton $= 1\Rightarrow$ số electron $= 1\Rightarrow$ điện tích hạt nhân $= +1$. Carbon có số electron $= 6\Rightarrow$ số proton $= 6\Rightarrow$ điện tích hạt nhân $= +6$. Phosphorus có số proton $= 15\Rightarrow$ số electron $= 15\Rightarrow$ điện tích hạt nhân $= +15$.
	\begin{table}[H]
		\centering
		\begin{tabular}{|c|c|c|c|c|}
			\hline
			Nguyên tử & Số proton & Số neutron & Số electron & Điện tích hạt nhân \\
			\hline
			Hydrogen & 1 & 0 & 1 & $+1$ \\
			\hline
			Carbon & $6$ & 6 & 6 & $+6$ \\
			\hline
			Phosphorus & 15 & 16 & 15 & $+15$ \\
			\hline
		\end{tabular}
	\end{table}
	Nếu không cho trước số neutron, không thể xác định chính xác số neutron chỉ dựa vào 3 số liệu: số proton, số electron, điện tích hạt nhân được vì neutron không mang điện nên không có đẳng thức nào ràng buộc số neutron vào số proton, số electron, điện tích hạt nhân, nên không thể để xác định được số neutron. Hơn, nữa số neutron có thể thay đổi tùy theo số đồng vị của 1 nguyên tố hóa học.
\end{proof}

\begin{luuy}
	Nếu ký hiệu $E_{\rm X},Z_{\rm X},N_{\rm X},A_{\rm X}$ lần lượt là số electron, số proton, số neutron, điện tích hạt nhân, số khối của hạt nhân của 1 nguyên tố hóa học X thì $E_{\rm X} = Z_{\rm X}$, $A_{\rm X} = Z_{\rm X} + N_{\rm X} = E_{\rm X} + N_{\rm X}$, $Z_{\rm X}\le N_{\rm X}\le1.5Z_{\rm X}$, \& điện tích hạt nhân của nguyên tử nguyên tố X bằng $+Z_{\rm X}$.
\end{luuy}

\begin{baitoan}[\cite{SGK_KHTN_7_Canh_Dieu}, 3, p. 12]
	Aluminium {\rm Al} là kim loại có nhiều ứng dụng trong thực tiễn, được dùng làm dây dẫn điện, chế tạo các thiết bị, máy móc trong công nghiệp \& nhiều đồ dùng sinh hoạt. Tổng số hạt trong hạt nhân nguyên tử aluminium là $27$, số đơn vị điện tích hạt nhân là $13$. Nêu cách tính số hạt mỗi loại trong nguyên tử aluminium \& cho biết điện tích hạt nhân của aluminium.
\end{baitoan}

\begin{proof}[Giải]
	...
\end{proof}

\begin{baitoan}[\cite{SGK_KHTN_7_Canh_Dieu}, p. 12]
	Nguyên tử sulfur (lưu huỳnh) có $16$ electron. Hỏi nguyên tử sulfur có bao nhiêu proton? Chứng minh nguyên tử sulfur trung hòa về điện.
\end{baitoan}

\begin{baitoan}[\cite{SGK_KHTN_7_Canh_Dieu}, 4, p. 12]
	Hình sau mô tả thành phần cấu tạo của nguyên tử sodium (natri), ở giữa là hạt nhân, mỗi vòng tròn lớn tiếp theo là 1 lớp electron, mỗi chấm chỉ 1 electron:
	\begin{figure}[H]
		\centering
		\includegraphics[scale=0.4]{sodium}
		\caption{Mô hình cấu tạo nguyên tử sodium.}
	\end{figure}
	\noindent Nguyên tử sodium có bao nhiêu lớp electron. Mỗi lớp có bao nhiêu electron?
\end{baitoan}

\begin{baitoan}[\cite{SGK_KHTN_7_Canh_Dieu}, 4, p. 13]
	Nguyên tử nitrogen \& silicon có số electron lần lượt là $7$ \& $14$. Mỗi nguyên tử nitrogen \& silicon có bao nhiêu lớp electron \& có bao nhiêu electron ở lớp ngoài cùng.
\end{baitoan}

\begin{baitoan}[\cite{SGK_KHTN_7_Canh_Dieu}, 5, p. 13]
	Quan sát hình vẽ mô tả cấu tạo nguyên tử carbon \& aluminium:
	\begin{figure}[H]
		\centering
		\includegraphics[scale=0.4]{carbon_aluminium}
		\caption{Mô hình cấu tạo nguyên tử carbon \& nguyên tử aluminium.}
	\end{figure}
	\noindent Mỗi nguyên tử đó có bao nhiêu lớp electron \& số electron trên mỗi lớp electron đó.
\end{baitoan}

\begin{baitoan}[\cite{SGK_KHTN_7_Canh_Dieu}, 5, p. 13]
	Trong 3 loại hạt tạo nên nguyên tử, hạt nào có khối lượng nhỏ nhất?
\end{baitoan}

\begin{baitoan}[\cite{SGK_KHTN_7_Canh_Dieu}, 6, p. 13]
	Khối lượng của nguyên tử được tính bằng đơn vị nào?
\end{baitoan}

\begin{baitoan}[\cite{SGK_KHTN_7_Canh_Dieu}, 6, p. 13]
	Cho biết: (a) Số proton, neutron, electron trong mỗi nguyên tử carbon \& aluminium. (b) Khối lượng nguyên tử của carbon \& aluminium.
\end{baitoan}

\begin{baitoan}[\cite{SGK_KHTN_7_Canh_Dieu}, 7, p. 14]
	Hoàn thành thông tin còn thiếu trong bảng sau:
	\begin{table}[H]
		\centering
		\begin{tabular}{|c|c|c|c|}
			\hline
			Hạt trong nguyên tử & Khối lượng (amu) & Điện tích & Vị trí trong nguyên tử \\
			\hline
			Proton &  & $+1$ &  \\
			\hline
			Neutron &  &  & Hạt nhân \\
			\hline
			Electron & $0.00055$ &  &  \\
			\hline
		\end{tabular}
	\end{table}
\end{baitoan}

\begin{baitoan}[\cite{SGK_KHTN_7_Canh_Dieu}, p. 14]
	Ruột của bút chì thường được làm từ than chì \& đất sét. Than chì được cấu tạo từ các nguyên tử carbon. (a) Ghi chú thích tên các hạt tương ứng trong mô hình cấu tạo nguyên tử carbon. (b) Tìm hiểu ý nghĩa của các ký hiệu HB, 2B, \& 6B được ghi trên 1 số loại bút chì.
\end{baitoan}

%------------------------------------------------------------------------------%

\section{Chemical Element -- Nguyên Tố Hóa Học}

\begin{baitoan}[\cite{SGK_KHTN_7_Canh_Dieu}, 1, p. 15]
	Các nguyên tử của cùng nguyên tố hóa học có đặc điểm gì giống nhau?
\end{baitoan}

\begin{baitoan}[\cite{SGK_KHTN_7_Canh_Dieu}, 1, p. 16]
	Số lượng mỗi loại hạt của 1 số nguyên tử được nêu trong bảng dưới đây. Những nguyên tử nào trong bảng thuộc cùng 1 nguyên tố hóa học?
	\begin{table}[H]
		\centering
		\begin{tabular}{|c|c|c|c|}
			\hline
			Nguyên tử & Số proton & Số neutron & Số electron \\
			\hline
			X1 & 8 & 9 & 8 \\
			\hline
			X2 & 7 & 8 & 7 \\
			\hline
			X3 & 8 & 8 & 8 \\
			\hline
			X4 & 6 & 6 & 6 \\
			\hline
			X5 & 7 & 7 & 7 \\
			\hline
			X6 & 11 & 12 & 11 \\
			\hline
			X7 & 8 & 10 & 8 \\
			\hline
			X8 & 6 & 8 & 6 \\
			\hline
		\end{tabular}
	\end{table}
\end{baitoan}

\begin{baitoan}[\cite{SGK_KHTN_7_Canh_Dieu}, 1, p. 17]
	Kể tên \& viết ký hiệu của $3$ nguyên tố hóa học chiếm khối lượng lớn nhất trong vỏ Trái Đất.
\end{baitoan}

\begin{baitoan}[\cite{SGK_KHTN_7_Canh_Dieu}, 2, p. 17]
	Nguyên tố hóa học nào có nhiều nhất trong vũ trụ?
\end{baitoan}

\begin{baitoan}[\cite{SGK_KHTN_7_Canh_Dieu}, 3, p. 17]
	Đọc \& viết tên các nguyên tố hóa học có ký hiệu: {\rm C, O, Mg, S}.
\end{baitoan}

\begin{baitoan}[\cite{SGK_KHTN_7_Canh_Dieu}, 4--5, p. 18]
	Hoàn thành thông tin về tên hoặc ký hiệu hóa học của nguyên tố: {\rm(a) Li. (b) Helium. (c) Na. (d) Al. (e) Neon. (f) Phosphorus. (g) Cl. (h) F}.
\end{baitoan}

\begin{baitoan}[\cite{SGK_KHTN_7_Canh_Dieu}, p. 18]
	Calcium là 1 nguyên tố hóa học có nhiều trong xương \& răng, giúp cho xương \& răng chắc khỏe. Ngoài ra, calcium còn cần cho quá trình hoạt động của thần kinh, cơ, tim, chuyển hóa của tế bào \& quá trình đông máu. Thực phẩm \& thuốc bổ chứa nguyên tố calcium giúp phòng ngừa bệnh loãng xương ở tuổi già \& hỗ trợ quá trình phát triển chiều cao của trẻ em. (a) Viết ký hiệu hóa học của nguyên tố calcium \& đọc tên. (b) Kể tên 3 thực phẩm có chứa nhiều calcium.
\end{baitoan}

%------------------------------------------------------------------------------%

\section{Chemical Periodic Table -- Sơ Lược về Bảng Tuần Hoàn Các Nguyên Tố Hóa Học}

\begin{baitoan}[\cite{SGK_KHTN_7_Canh_Dieu}, 1, p. 20]
	Số đơn vị điện tích hạt nhân của mỗi nguyên tử {\rm C, Si, O, P, N, S} lần lượt là $6,14,8,15,7,16$. Sắp xếp các nguyên tố trên chiều điện tích hạt nhân tăng dần từ trái sang phải \& từ trên xuống dưới.
\end{baitoan}

\begin{baitoan}[\cite{SGK_KHTN_7_Canh_Dieu}, 1, p. 20]
	Việc tìm ra bảng tuần hoàn là 1 trong những phát hiện xuất sắc nhất trong ngành hóa học. Tìm hiểu lịch sử phát minh ra bảng tuần hoàn các nguyên tố hóa học.
\end{baitoan}

\begin{baitoan}[\cite{SGK_KHTN_7_Canh_Dieu}, 1, p. 20]
	Hình sau cho biết các thông tin gì về nguyên tố carbon?
	\begin{figure}[H]
		\centering
		\includegraphics[scale=0.3]{carbon_element}
		\caption{Ô nguyên tố carbon.}
	\end{figure}
\end{baitoan}

\begin{baitoan}[\cite{SGK_KHTN_7_Canh_Dieu}, 1, p. 20]
	Tìm nguyên tố hóa học có số thứ tự lần lượt là $16$ \& $20$ trong bảng tuần hoàn. Đọc tên 2 nguyên tố. Số hiệu nguyên tử, ký hiệu hóa học, \& khối lượng nguyên tử của 2 nguyên tố đó?
\end{baitoan}

\begin{baitoan}[\cite{SGK_KHTN_7_Canh_Dieu}, 3, p. 21]
	Quan sát bảng tuần hoàn, cho biết số hiệu nguyên tử lần lượt của nguyên tử carbon {\rm C} \& aluminium {\rm Al}. 2 nguyên tố đó nằm ở chu kỳ nào trong bảng tuần hoàn? Từ đó cho biết số lớp electron của {\rm C} \& {\rm Al}.
\end{baitoan}

\begin{baitoan}[\cite{SGK_KHTN_7_Canh_Dieu}, 2, p. 21]
	Nguyên tố X có số thứ tự $18$ trong bảng tuần hoàn. Nguyên tố đó ở chu kỳ nào \& có mấy lớp electron?
\end{baitoan}

\begin{baitoan}[\cite{SGK_KHTN_7_Canh_Dieu}, 3, p. 21]
	Dựa vào mô hình cấu tạo nguyên tử sodium \& argon, cho biết 1 số thông tin về nguyên tố sodium \& argon (số hiệu nguyên tử, điện tích hạt nhân, số lớp electron, chu kỳ, số electron ở lớp ngoài cùng).
\end{baitoan}

\begin{baitoan}[\cite{SGK_KHTN_7_Canh_Dieu}, 3, p. 22]
	Nguyên tố X tạo nên chất khí duy trì sự hô hấp của con người, động vật, thực vật, \& có nhiều trong không khí. Tên của nguyên tố X? Nguyên tố X nằm ở ô nào \& chu kỳ nào trong bảng tuần hoàn?
\end{baitoan}

\begin{baitoan}[\cite{SGK_KHTN_7_Canh_Dieu}, 3, p. 22]
	Quan sát hình sau \& bảng tuần hoàn, cho biết số electron lớp ngoài cùng của nguyên tử {\rm Li} (lithium) \& {\rm Cl} (chlorine).
	\begin{figure}[H]
		\centering
		\includegraphics[scale=0.3]{lithium_chlorine}
		\caption{Mô hình cấu tạo nguyên tử lithium \& chlorine.}
	\end{figure}
	\noindent 2 nguyên tố đó nằm ở nhóm nào trong bảng tuần hoàn?
\end{baitoan}

\begin{baitoan}[\cite{SGK_KHTN_7_Canh_Dieu}, 3, p. 23]
	Cho các nguyên tố có số thứ tự lần lượt là $9,18,19$. Số electron lớp ngoài cùng của mỗi nguyên tố trên là bao nhiêu? Mỗi nguyên tố nằm ở nhóm nào \& đó là kim loại, phi kim hay khí hiếm.
\end{baitoan}

\begin{baitoan}[\cite{SGK_KHTN_7_Canh_Dieu}, p. 23]
	Ngoài $8$ nhóm A, bảng tuần hoàn còn có nhóm B. Tìm hiểu về các nhóm B.
\end{baitoan}

\begin{baitoan}[\cite{SGK_KHTN_7_Canh_Dieu}, p. 23]
	Quan sát bảng tuần hoàn các nguyên tố hóa học, cho biết vị trí của các nguyên tố kim loại, phi kim, \& khí hiếm.
\end{baitoan}

\begin{baitoan}[\cite{SGK_KHTN_7_Canh_Dieu}, p. 24]
	Nguyên tố X nằm ở chu kỳ 2, nhóm VA trong bảng tuần hoàn. 1 số thông tin của nguyên tố X (tên nguyên tố, ký hiệu hóa học, khối lượng nguyên tử), vị trí ô của nguyên tố trong bảng tuần hoàn.? Nguyên tố đó là kim loại, phi kim hay khí hiếm?
\end{baitoan}

\begin{baitoan}[\cite{SGK_KHTN_7_Canh_Dieu}, 1., p. 26]
	Những phát biểu sau nói về đặc điểm của các hạt cấu tạo nên nguyên tử. Tên hạt ứng với mỗi phát biểu? (a) Hạt mang điện tích dương. (b) Hạt được tìm thấy cùng với proton trong hạt nhân. (c) Hạt có thể xuất hiện với số lượng khác nhau trong các nguyên tử của cùng 1 nguyên tố. (d) Hạt có trong lớp vỏ xung quanh hạt nhân. (e) Hạt mang điện tích âm. (f) Hạt có khối lượng rất nhỏ, có thể bỏ qua khi tính khối lượng nguyên tử. (g) Hạt không mang điện tích.
\end{baitoan}

\begin{baitoan}[\cite{SGK_KHTN_7_Canh_Dieu}, 2., p. 26]
	Điền thông tin thích hợp vào chỗ trống: (a) Hạt nhân của nguyên tử được cấu tạo bởi các hạt $\ldots$ (b) 1 nguyên tử có $17$ proton trong hạt nhân, số electron chuyển động quanh hạt nhân là $\ldots$ (c) 1 nguyên tử có $10$ electron, số proton trong hạt nhân của nguyên tử đó là $\ldots$ (d) Khối lượng nguyên tử X bằng $19$ amu, số electron của nguyên tử đó là $9$. Số neutron của nguyên tử X là $\ldots$ (e) 1 nguyên tử có $3$ proton, $4$ neutron, \& $3$ electron. Khối lượng của nguyên tử đó là $\ldots$
\end{baitoan}

\begin{baitoan}[\cite{SGK_KHTN_7_Canh_Dieu}, 3., p. 26]
	Viết ký hiệu hóa học của các nguyên tố sau: hydrogen, helium, carbon, nitrogen, oxygen, sodium.
\end{baitoan}

\begin{baitoan}[\cite{SGK_KHTN_7_Canh_Dieu}, 4., p. 26]
	Mô hình sắp xếp electron trong nguyên tử của nguyên tố X như sau:
	\begin{figure}[H]
		\centering
		\includegraphics[scale=0.3]{SGK_KHTN_7_CD_p_26}
	\end{figure}
	\noindent(a) Trong nguyên tử X có bao nhiêu electron \& các electron được sắp xếp thành mấy lớp? (b) Tên nguyên tố X? (c) Gọi tên 1 nguyên tố khác mà nguyên tử của nó có cùng số lớp electron với nguyên tử nguyên tố X.
\end{baitoan}

\begin{baitoan}[\cite{SGK_KHTN_7_Canh_Dieu}, 5., p. 27]
	Hoàn thành những thông tin còn thiếu trong bảng sau:
	\begin{table}[H]
		\centering
		\begin{tabular}{|c|c|c|c|c|c|}
			\hline
			Tên nguyên tố & Ký hiệu hóa học & Số proton & Số neutron & Số electron & Khối lượng nguyên tử (amu) \\
			\hline
			&  &  & 10 & 9 &  \\
			\hline
			Sulfur &  &  &  & 16 & 32 \\
			\hline
			&  & 12 &  &  & 24 \\
			\hline
			&  & 1 &  &  & 2 \\
			\hline
			&  &  &  & 11 & 23 \\
			\hline
		\end{tabular}
	\end{table}
\end{baitoan}

\begin{baitoan}[\cite{SGK_KHTN_7_Canh_Dieu}, 6., p. 27]
	Số proton \& số neutron của 2 nguyên tử X, Y được cho trong bảng sau:
	\begin{table}[H]
		\centering
		\begin{tabular}{|c|c|c|}
			\hline
			Nguyên tử & X & Y \\
			\hline
			Số proton & 6 & 6 \\
			\hline
			Số neutron & 6 & 8 \\
			\hline
		\end{tabular}
	\end{table}
	\noindent(a) Tính khối lượng của nguyên tử X \& nguyên tử Y. (b) Nguyên tử X \& nguyên tử Y có thuộc cùng 1 nguyên tố hóa học không? Vì sao?
\end{baitoan}

\begin{baitoan}[\cite{SGK_KHTN_7_Canh_Dieu}, 7., p. 27]
	Cho các nguyên tố sau: {\rm Ca, S, Na, Mg, F, Ne}. Sử dụng bảng tuần hoàn các nguyên tố hóa học: (a) Sắp xếp các nguyên tố trên theo chiều tăng dần điện tích hạt nhân. (b) Mỗi nguyên tố trong dãy trên là kim loại, phi kim hay khí hiếm?
\end{baitoan}

\begin{baitoan}[\cite{SGK_KHTN_7_Canh_Dieu}, 8., p. 27]
	Dựa vào bảng tuần hoàn, cho biết 1 số thông tin của các nguyên tố có số hiệu nguyên tử lần lượt là $12,15,18$. Điền các thông tin theo bảng sau:
	\begin{table}[H]
		\centering
		\begin{tabular}{|c|c|c|c|c|c|p{3.5cm}|}
			\hline
			Số hiệu nguyên tử & Tên nguyên tố & Ký hiệu hóa học & Khối lượng nguyên tử & Chu kỳ & Nhóm & Kim loại, phi kim hay khí hiếm? \\
			\hline
			12 &  &  &  &  &  &  \\
			\hline
			15 &  &  &  &  &  &  \\
			\hline
			18 &  &  &  &  &  &  \\
			\hline
		\end{tabular}
	\end{table}
\end{baitoan}

\begin{baitoan}[\cite{SGK_KHTN_7_Canh_Dieu}, 9., p. 27]
	Biết nguyên tử của nguyên tố M có $3$ lớp electron \& có $2$ electron ở lớp ngoài cùng. Xác định vị trí của M trong bảng tuần hoàn (ô, chu kỳ, nhóm) \& cho biết M là kim loại, phi kim hay khí hiếm.
\end{baitoan}

%------------------------------------------------------------------------------%

\section{Molecule, Compound -- Phân Tử, Đơn Chất, Hợp Chất}

\begin{baitoan}[\cite{SGK_KHTN_7_Canh_Dieu}, 1, p. 29]
	Giải thích 1 số hiện tượng sau: (a) Khi mở lọ nước hoa hoặc mở lọ đựng 1 số loại tinh dầu sẽ ngửi thấy có mùi thơm. (b) Quần áo sau khi giặt xong, phơi trong không khí 1 thời gian sẽ khô.
\end{baitoan}

\begin{baitoan}[\cite{SGK_KHTN_7_Canh_Dieu}, 1, p. 29]
	Khi nói về nước, có 2 ý kiến sau: (a) Phân tử nước trong nước đá, nước lỏng, \& hơi nước là giống nhau. (b) Phân tử nước trong nước đá, nước lỏng, \& hơi nước là khác nhau. Ý kiến nào là đúng? Vì sao?
\end{baitoan}

\begin{baitoan}[\cite{SGK_KHTN_7_Canh_Dieu}, 1, p. 29]
	{\rm Đ\texttt{/}S?} (a) Trong 1 phân tử, các nguyên tử luôn giống nhau. (b) Trong 1 phân tử, các nguyên tử luôn khác nhau. (c) Trong 1 phân tử, các nguyên tử có thể giống nhau hoặc khác nhau.
\end{baitoan}

\begin{baitoan}[\cite{SGK_KHTN_7_Canh_Dieu}, 1, p. 29]
	1 số nhiên liệu như xăng, dầu, $\ldots$ dễ tách ra các phân tử \& lan tỏa trong không khí. Cần bảo quản các nhiên liệu trên như thế nào để bảo đảm an toàn?
\end{baitoan}

\begin{baitoan}[\cite{SGK_KHTN_7_Canh_Dieu}, 2, p. 30]
	Tính khối lượng phân tử của fluorine {\rm\ce{F2}} \& methane {\rm\ce{CH4}}.
\end{baitoan}

\begin{baitoan}[\cite{SGK_KHTN_7_Canh_Dieu}, 3, p. 30]
	Những chất nào là dơn chất trong các chất sau? (a) Kim loại sodium được tạo thành từ nguyên tố {\rm Na}. (b) Lactic acid có trong sữa chua, được tạo thành từ các nguyên tố {\rm C, H, O}. (c) Kim cương được tạo thành từ nguyên tố {\rm C}. (d) Muối ăn {\rm NaCl} được tạo thành từ các nguyên tố {\rm Na} \& {\rm Cl}.
\end{baitoan}

\begin{baitoan}[\cite{SGK_KHTN_7_Canh_Dieu}, 2, p. 31]
	Nêu 2 đơn chất kim loại thường được sử dụng để làm dây dẫn điện.
\end{baitoan}

\begin{baitoan}[\cite{SGK_KHTN_7_Canh_Dieu}, 3, p. 31]
	Đơn chất nào được tạo ra trong quá trình quang hợp của cây xanh \& có vai trò quan trọng đối với sự sống của con người?
\end{baitoan}

\begin{baitoan}[\cite{SGK_KHTN_7_Canh_Dieu}, 4, p. 31]
	Trong các chất sau, chất nào là đơn chất, chất nào là hợp chất? (a) Đường ăn. (b) Nước. (c) Khí hydrogen (được tạo thành từ nguyên tố {\rm H}). (d) Vitamin C (được tạo thành từ các nguyên tố {\rm C, H, O}). (e) Sulfur (được tạo thành từ nguyên tố {\rm S}).
\end{baitoan}

\begin{baitoan}[\cite{SGK_KHTN_7_Canh_Dieu}, 5, p. 32]
	Acetic acid có trong giấm ăn \& là chất được sử dụng nhiều trong công nghiệp; oxygen chiếm khoảng $21$\% thể tích không khí, có vai trò quan trọng đối với sự sống; hydrogen peroxide có nhiều ứng dụng trong công nghiệp \& là chất sát khuẩn mạnh. Chất nào là đơn chất, chất nào là hợp chất?
\end{baitoan}

%------------------------------------------------------------------------------%

\section{Giới Thiệu về Liên Kết Hóa Học}

\begin{baitoan}[\cite{SGK_KHTN_7_Canh_Dieu}, 1, p. 33]
	Quan sát hình sau \& cho biết số electron ở lớp ngoài cùng của vỏ nguyên tử khí hiếm:
	\begin{figure}[H]
		\centering
		\includegraphics[scale=0.3]{helium_neon_argon}
		\caption{Mô hình cấu tạo nguyên tử của 1 số nguyên tố khí hiếm: (a) Helium He. (b) Neon Ne. (c) Argon Ar.}
	\end{figure}
\end{baitoan}

\begin{baitoan}[\cite{SGK_KHTN_7_Canh_Dieu}, p. 34]
	Tìm hiểu 1 số ứng dụng của helium trong thực tiễn.
\end{baitoan}

\begin{baitoan}[\cite{SGK_KHTN_7_Canh_Dieu}, 2, p. 34]
	Lớp vỏ của các ion {\rm\ce{Na+,Cl-}} tương tự vỏ nguyên tử của nguyên tố khí hiếm nào?
\end{baitoan}

\begin{baitoan}[\cite{SGK_KHTN_7_Canh_Dieu}, 3, p. 34]
	So sánh về số electron, số lớp electron giữa nguyên tử {\rm Na} \& ion {\rm\ce{Na+}}.
\end{baitoan}

\begin{baitoan}
	Tại sao bán kính nguyên tử của {\rm Na} khi biến thành ion {\rm\ce{Na+}} thì nhỏ lại trong khi bán kính nguyên tử của {\rm Cl} khi biến thành ion {\rm\ce{Cl-}} thì lại tăng lên?
\end{baitoan}

\begin{baitoan}[\cite{SGK_KHTN_7_Canh_Dieu}, 1, p. 35]
	Số electron ở lớp ngoài cùng của nguyên tử {\rm K, F} lần lượt là $1,7$. Khi {\rm K} kết hợp với {\rm F} để tạo thành phân tử potassium fluoride, nguyên tử {\rm K} cho hay nhận bao nhiêu electron? Vẽ sơ đồ tạo thành liên kết trong phân tử potassium fluoride.
\end{baitoan}

\begin{baitoan}[\cite{SGK_KHTN_7_Canh_Dieu}, 4, p. 35]
	Các ion {\rm\ce{Mg^2+, O^2-}} có lớp vỏ tương tự khí hiếm nào?
\end{baitoan}

\begin{baitoan}[\cite{SGK_KHTN_7_Canh_Dieu}, 5, p. 35]
	So sánh về số electron, số lớp electron giữa nguyên tử {\rm Mg} \& ion {\rm\ce{Mg^2+}}.
\end{baitoan}

\begin{baitoan}[\cite{SGK_KHTN_7_Canh_Dieu}, 2, p. 35]
	Nguyên tử {\rm Ca} có $2$ electron ở lớp ngoài cùng. Vẽ sơ đồ tạo thành liên kết khi nguyên tử {\rm Ca} kết hợp với nguyên tử {\rm O} tạo ra phân tử calcium oxide.
\end{baitoan}

\begin{baitoan}[\cite{SGK_KHTN_7_Canh_Dieu}, 3, p. 36]
	Nguyên tử {\rm K} kết hợp với nguyên tử {\rm Cl} tạo thành phân tử potassium chloride. Ở điều kiện thường, potassium chloride là chất rắn, chất lỏng hay chất khí? Vì sao?
\end{baitoan}

\begin{baitoan}[\cite{SGK_KHTN_7_Canh_Dieu}, 3, p. 36]
	Nguyên tử {\rm H} trong phân tử hydrogen {\rm\ce{H2}} có lớp vỏ tương tự khí hiếm nào?
\end{baitoan}

\begin{baitoan}[\cite{SGK_KHTN_7_Canh_Dieu}, 4, p. 36]
	2 nguyên tử {\rm Cl} liên kết với nhau tạo thành phân tử chlorine. (a) Mỗi nguyên tử {\rm Cl} cần thêm bao nhiêu electron vào lớp ngoài cùng để có lớp vỏ tương tự khí hiếm? (b) Vẽ sơ đồ tạo thành liên kết trong phân tử chlorine {\rm\ce{Cl2}}.
\end{baitoan}

\begin{baitoan}[\cite{SGK_KHTN_7_Canh_Dieu}, 7, p. 37]
	Trong phân tử nước, mỗi nguyên tử {\rm H, O} có bao nhiêu electron ở lớp ngoài cùng?
\end{baitoan}

\begin{baitoan}[\cite{SGK_KHTN_7_Canh_Dieu}, 5, p. 37]
	Mỗi nguyên tử {\rm H} kết hợp với $1$ nguyên tử {\rm Cl} tạo thành phân tử hydrogen chloride {\rm HCl}. Vẽ sơ đồ tạo thành phân tử hydrogen chloride từ nguyên tử {\rm H} \& nguyên tử {\rm Cl}.
\end{baitoan}

\begin{baitoan}[\cite{SGK_KHTN_7_Canh_Dieu}, 6, p. 37]
	Mỗi nguyên tử {\rm N} kết hợp với $3$ nguyên tử {\rm H} tạo thành phân tử ammonia {\rm\ce{NH3}}. Vẽ sơ đồ tạo thành phân tử ammonia.
\end{baitoan}

\begin{baitoan}[\cite{SGK_KHTN_7_Canh_Dieu}, 7, p. 37]
	$2$ nguyên tử {\rm N} kết hợp với nhau tạo thành phân tử nitrogen. Vẽ sơ đồ tạo thành liên kết trong phân tử nitrogen {\rm\ce{N2}}.
\end{baitoan}

\begin{baitoan}[\cite{SGK_KHTN_7_Canh_Dieu}, p. 38]
	Giải thích các hiện tượng sau: (a) Nước tinh khiết hầu như không dẫn điện, nhưng nước biển lại dẫn được điện. (b) Khi cho đường ăn vào chảo rồi đun nóng sẽ thấy đường ăn nhanh chóng chuyển từ thể rắn sang thể lỏng, làm như vậy với muối ăn thấy muối ăn vẫn ở thể rắn.
\end{baitoan}

\begin{baitoan}[\cite{SGK_KHTN_7_Canh_Dieu}, 9, p. 38]
	So sánh 1 số tính chất chung của chất cộng hóa trị với chất ion.
\end{baitoan}

%------------------------------------------------------------------------------%

\section{Hóa Trị, Công Thức Hóa Học}

\begin{baitoan}[\cite{SGK_KHTN_7_Canh_Dieu}, 1, p. 39]
	So sánh hóa trị của nguyên tố \& số electron mà nguyên tử của nguyên tố đã góp chung để tạo ra liên kết.
\end{baitoan}

\begin{baitoan}[\cite{SGK_KHTN_7_Canh_Dieu}, 1, p. 40]
	Xác định hóa trị của {\rm C, O} trong carbon dioxide {\rm\ce{CO2}}.
	\begin{figure}[H]
		\centering
		\includegraphics[scale=0.3]{CO2}
		\caption{Sơ đồ liên kết cộng hóa trị giữa C \& O trong phân tử carbon dioxide \ce{CO2}.}
	\end{figure}
\end{baitoan}

\begin{baitoan}[\cite{SGK_KHTN_7_Canh_Dieu}, 2, p. 40]
	Vẽ sơ đồ hình thành liên kết giữa nguyên tử {\rm N} \& $3$ nguyên tử {\rm H}. Cho biết liên kết đó thuộc loại liên kết nào. Hóa trị của mỗi nguyên tố trong hợp chất tạo thành là bao nhiêu?
\end{baitoan}

\begin{baitoan}[\cite{SGK_KHTN_7_Canh_Dieu}, 2, p. 41]
	Cát (sand) được sử dụng nhiều trong xây dựng \& là nguyên liệu chính để sản xuất thủy tinh. Silicon oxide là thành phần chính của cát. Phân tử silicon oxide gồm $1$ nguyên tử Si liên kết với $2$ nguyên tử O. Dựa vào hóa trị của các nguyên tố trong bảng \ref{tab: hoa tri nguyen tu}, tính tích hóa trị \& số nguyên tử của mỗi nguyên tố trong phân tử silicon oxide. Nhận xét về tích đó.
\end{baitoan}

\begin{baitoan}[\cite{SGK_KHTN_7_Canh_Dieu}, 3, p. 41]
	Dựa vào hóa trị của các nguyên tố trong bảng \ref{tab: hoa tri nguyen tu} \& quy tắc hóa trị, mỗi nguyên tử {\rm Mg} có thể kết hợp được với bao nhiêu nguyên tử {\rm Cl}?
\end{baitoan}

\begin{baitoan}[\cite{SGK_KHTN_7_Canh_Dieu}, 4, p. 41]
	Nguyên tố A có hóa trị III, nguyên tố B có hóa trị II. Tính tỷ lệ nguyên tử của A \& B trong hợp chất tạo thành từ 2 nguyên tố đó.
\end{baitoan}

\begin{baitoan}[\cite{SGK_KHTN_7_Canh_Dieu}, 3, p. 41]
	Cho CTHH của 1 số chất như sau: (a) {\rm\ce{N2}} (nitrogen). (b) {\rm NaCl} (sodium chloride). (c) {\rm\ce{MgSO4}} (magnesium sulfate). Xác định nguyên tố tạo thành mỗi chất \& số nguyên tử của mỗi nguyên tố có trong phân tử.
\end{baitoan}

\begin{baitoan}[\cite{SGK_KHTN_7_Canh_Dieu}, 5, p. 42]
	Viết CTHH của các chất: (a) Sodium sulfide biết trong phân tử có 2 nguyên tử {\rm Na} \& 1 nguyên tử {\rm S}. (b) Phosphoric acid biết trong phân tử có $3$ nguyên tử {\rm H}, $1$ nguyên tử {\rm P}, \& $4$ nguyên tử {\rm O}.
\end{baitoan}

\begin{baitoan}[\cite{SGK_KHTN_7_Canh_Dieu}, 7, p. 42]
	Đường glucose là nguồn cung cấp năng lượng quan trọng cho hoạt động sống của con người. Đường glucose có CTHH là {\rm\ce{C6H12O6}}. (a) Glucose được tạo thành từ những nguyên tố nào? (b) Khối lượng mỗi nguyên tố trong 1 phân tử glucose bằng bao nhiêu? (c) Khối lượng phân tử glucose là bao nhiêu?
\end{baitoan}

\begin{baitoan}[\%$m$ of MgO]
	Tính \% khối lượng của {\rm Mg, O} trong hợp chất {\rm MgO}.
\end{baitoan}

\begin{proof}[1st giải]
	Khối lượng của nguyên tố O trong MgO là: $1\cdot16 = 16$ amu. Khối lượng của nguyên tố Mg trong MgO là: $1\cdot24 = 24$ amu. Suy ra khối lượng phân tử MgO là: $16 + 24 = 40$ amu. Phần trăm về khối lượng của Mg trong hợp chất MgO là: $\frac{24}{40}\cdot100\% = 60$\%. Phần trăm về khối lượng của O trong hợp chất MgO là: $\frac{16}{40}\cdot100\% = 40$\%.
\end{proof}

\begin{proof}[2nd giải]
	$M_{\rm MgO} = M_{\rm Mg} + M_{\rm O} = 24 + 16 = 40$. \%$m_{\rm Mg|MgO} = \frac{M_{\rm Mg}}{M_{\rm MgO}}\cdot100\% = \frac{24}{40}\cdot\cdot100\% = 60$\%. \%$m_{\rm O|MgO} = \frac{M_{\rm O}}{M_{\rm MgO}}\cdot100\% = \frac{16}{40}\cdot\cdot100\% = 60$\% (hoặc \%$m_{\rm O|MgO} = 100\% - \%m_{\rm Mg|MgO} = 100\% - 60\% = 40\%$).
\end{proof}

\begin{baitoan}[\cite{SGK_KHTN_7_Canh_Dieu}, 4, p. 43]
	Có ý kiến cho rằng: Trong nước, số nguyên tử {\rm H} gấp $2$ lần số nguyên tử {\rm O} nên \% khối lượng của {\rm H} trong nước gấp 2 lần \% khối lượng {\rm O}. Ý kiến trên có đúng không? Tính \% khối lượng của {\rm H, O} trong nước để chứng minh.
\end{baitoan}

\begin{baitoan}[\cite{SGK_KHTN_7_Canh_Dieu}, 8, p. 43]
	Calcium carbonate là thành phần chính của đá vôi, có CTHH là {\rm\ce{CaCO3}}. Tính \% khối lượng của mỗi nguyên tố trong hợp chất trên.
\end{baitoan}

\begin{baitoan}[\cite{SGK_KHTN_7_Canh_Dieu}, 9, p. 43]
	Citric acid là hợp chất được sử dụng nhiều trong công nghiệp thực phẩm, dược phẩm. Trong tự nhiên, citric acid có trong quả chanh \& 1 số loại quả như bưởi, cam, $\ldots$  Citric acid có CTHH là {\rm\ce{C6H8O7}}. Tính \% khối lượng của mỗi nguyên tố trong citric acid.
\end{baitoan}

\begin{baitoan}[\cite{SGK_KHTN_7_Canh_Dieu}, p. 43]
	Potassium (kali) rất cần thiết cho cây trồng, đặc biệt trong giai đoạn cây trưởng thành, ra hoa, kết trái. Để cung cấp {\rm K} cho cây, có thể sử dụng phân potassium chloride \& potassium sulfate có CTHH lần lượt là {\rm KCl, \ce{K2SO4}}. Người trồng cây muốn sử dụng loại phân bón có hàm lượng {\rm K} cao hơn thì nên chọn loại phân bón nào?
\end{baitoan}

\begin{baitoan}[\cite{SGK_KHTN_7_Canh_Dieu}, Ví dụ 2, p. 44]
	Xác định hóa trị của {\rm Fe} trong hợp chất có CTHH là {\rm\ce{Fe2O3}}.
\end{baitoan}

\begin{proof}[Giải]
	Gọi hóa trị của {\rm Fe} trong hợp chất là $a$. Vì O có hóa trị II nên khi áp dụng quy tắc hóa trị, ta có: $a\cdot2 =$ II$\cdot3\Rightarrow a =$ III. Vậy Fe có hóa trị III trong hợp chất \ce{Fe2O3}.
\end{proof}

\begin{baitoan}
	Xác định hóa trị của {\rm Fe} trong hợp chất có CTHH là {\rm\ce{FeO, Fe3O4}}.
\end{baitoan}

\begin{luuy}[oxide]
	Sắt {\rm Fe} có 3 dạng oxide: {\rm\ce{FeO, Fe2O3, Fe3O4}}. Copper (đồng) {\rm Cu} có 2 dạng oxide: {\rm\ce{Cu2O, CuO}}.
\end{luuy}

\begin{baitoan}[\cite{SGK_KHTN_7_Canh_Dieu}, 10, p. 44]
	Xác định hóa trị của mỗi nguyên tố trong các hợp chất sau: {\rm HBr, BaO}.
\end{baitoan}

\begin{baitoan}[\cite{SGK_KHTN_7_Canh_Dieu}, Ví dụ 3, p. 44]
	Lập CTHH của hợp chất tạo bởi {\rm S} hóa trị VI \& {\rm O}.
\end{baitoan}

\begin{proof}[Giải]
	Đặt CTHH của hợp chất là \ce{S_xO_y}. Theo quy tắc hóa trị, ta có: VI$\cdot x =$ II$\cdot y$. Ta có tỷ lệ: $\frac{x}{y} = \frac{II}{VI} = \frac{1}{3}$. Chọn $x = 1$ \& $y = 3$. CTHH của hợp chất là \ce{SO3}.
\end{proof}

\begin{baitoan}[\cite{SGK_KHTN_7_Canh_Dieu}, Ví dụ 4, p. 45]
	R là hợp chất của {\rm S, O}, khối lượng phân tử của R là $64$ amu. Biết \% khối lượng của oxygen trong R là $50$\%. Xác định CTHH của R.
\end{baitoan}

\begin{proof}[Giải]
	Đặt CTHH của R là \ce{S_xO_y}. Khối lượng của nguyên tố O trong 1 phân tử R là: $\frac{64\cdot50}{100} = 32$ amu. Khối lượng của nguyên tố S trong 1 phân tử R là: $64 - 32 = 32$ amu. Ta có: $16y = 32$ amu $\Rightarrow y = 2$, $32x = 32$ amu $\Rightarrow x = 1$. Vậy CTHH của R là \ce{SO2}.
\end{proof}

\begin{baitoan}[\cite{SGK_KHTN_7_Canh_Dieu}, 11, p. 45]
	Hợp chất X được tạo thành bởi {\rm Fe, O} có khối lượng phân tử là $160$ amu. Biết \% khối lượng của {\rm Fe} trong X là $70$\%. Xác định CTHH của X.
\end{baitoan}

\begin{baitoan}[\cite{SGK_KHTN_7_Canh_Dieu}, 1., p. 46]
	(a) Nêu ý nghĩa của CTHH. (b) Mỗi CTHH sau đây cho biết những thông tin gì? {\rm\ce{Na2CO3, O2, H2SO4, KNO3}}.
\end{baitoan}

\begin{baitoan}[\cite{SGK_KHTN_7_Canh_Dieu}, 2., p. 46]
	Viết CTHH \& tính khối lượng phân tử của các hợp chất sau: (a) Calcium oxide (vôi sống), biết trong phân tử có {\rm 1 Ca \& 1 O}. (b) Hydrogen sulfide, biết trong phân tử có {\rm 2 H \& 1 S}. (c) Sodium sulfate, biết trong phân tử có {\rm 2 Na, 1 S, \& 4 O}.
\end{baitoan}

\begin{baitoan}[\cite{SGK_KHTN_7_Canh_Dieu}, 3., p. 46]
	Cho CTHH của 1 số chất như sau: {\rm\ce{F2, LiCl, Cl2, MgO, HCl}}. Trong các CTHH trên, công thức nào là của đơn chất, công thức nào là của hợp chất?
\end{baitoan}

\begin{baitoan}[\cite{SGK_KHTN_7_Canh_Dieu}, 4., p. 46]
	1 số chất có CTHH như sau; {\rm\ce{BaSO4, Cu(OH)2, Zn3(PO4)2}}. Dựa vào bảng \ref{tab: hoa tri nhom nguyen tu}, tính hóa trị của các nguyên tố {\rm Ba, Cu, Zn} trong các hợp chất trên.
\end{baitoan}

\begin{baitoan}[\cite{SGK_KHTN_7_Canh_Dieu}, 5., p. 46]
	Lập CTHH của những chất tạo thành từ các nguyên tố: (a) {\rm C, S}. (b) {\rm Mg, S}. (c) {\rm Al, Br}. Biết hóa trị của các nguyên tố trong các hợp chất tạo thành như sau: {\rm C: IV, S: II, Mg: II, Al: III, Br: I}.
\end{baitoan}

\begin{baitoan}[\cite{SGK_KHTN_7_Canh_Dieu}, 6., p. 46]
	Các hợp chất của calcium có nhiều ứng dụng trong đời sống: {\rm\ce{CaSO4}} là thành phần chính của thạch cao. Thạch cao được dùng để đúc tượng, sản xuất các vật liệu xây dựng, $\ldots$ {\rm\ce{CaCO3}} là thành phần chính của đá vôi. Đá vôi được dùng nhiều trong công nghiệp sản xuất xi măng. {\rm\ce{CaCl2}} được dùng để hút ẩm, chống đóng băng tuyết trên mặt đường ở xứ lạnh. Tính \% khối lượng của calcium trong các hợp chất trên.
\end{baitoan}

\begin{baitoan}[\cite{SGK_KHTN_7_Canh_Dieu}, 7., p. 46]
	Copper(II) sulfate có trong thành phần của 1 số thuốc diệt nấm, trừ sâu, \& diệt cỏ cho cây trồng. Copper(II) sulfate được tạo thành từ các nguyên tố {\rm Cu, S, O}, \& có khối lượng phân tử là $160$ amu. \% khối lượng của các nguyên tố {\rm Cu, S, O} trong copper(II) sulfate lần lượt là: $40$\%, $20$\%, \& $40$\%. Xác định CTHH của copper(II) sulfate. 
\end{baitoan}

%------------------------------------------------------------------------------%

\section{Miscellaneous}

\subsection*{Notation -- Ký Hiệu}

\begin{itemize}
	\item $\%m_{A|A_xB_y}$: \% khối lượng của nguyên tố $A$ trong hợp chất $A_xB_y$, \& được tính bởi công thức $\%m_{A|A_xB_y}\coloneqq\frac{xM_A}{xM_A + yM_B}$.
	\item $m_{A|A_xB_y}$: khối lượng của nguyên tố $A$ trong hợp chất $A_xB_y$, \& được tính bởi công thức $m_{A|A_xB_y}\coloneqq m_{A_xB_y}\cdot\%m_{A|A_xB_y} = m_{A_xB_y}\frac{xM_A}{xM_A + yM_B}$.
\end{itemize}

\begin{dangtoan}
	Từ lượng chất tính lượng nguyên tố.
\end{dangtoan}

\begin{baitoan}[\cite{Tuan2022}, p. 70]
	Tính khối lượng {\rm Fe} \& khối lượng oxi có trong {\rm20g} {\rm\ce{Fe2(SO4)3}}.
\end{baitoan}

\begin{proof}[Giải]
	$M_{\ce{Fe2(SO4)3}} = 2\cdot56 + 3(32 + 4\cdot16) = 400$ g\texttt{/}mol$\Rightarrow m_{\ce{Fe|Fe2(SO4)3}} = \%m_{\ce{Fe|Fe2(SO4)3}}\cdot m_{\ce{Fe2(SO4)3}} = \frac{2\cdot56}{2\cdot56 + 3(32 + 4\cdot16)}\cdot20 = 5.6$g$\Rightarrow m_{\ce{O|Fe2(SO4)3}} = m_{\ce{Fe2(SO4)3}}\cdot\%m_{\ce{O|Fe2(SO4)3}} = 20\cdot\frac{12\cdot16}{2\cdot56 + 3(32 + 4\cdot16)} = 9.6$g.
\end{proof}
Dễ dàng tính được khối lượng S trong 20g \ce{Fe2(SO4)3} theo 2 cách: \textit{Cách 1.} Tính theo tỷ lệ \% khối lượng của S trong \ce{Fe2(SO4)3} tương tự lời giải trên: $m_{\ce{S|Fe2(SO4)3}} = m_{\ce{Fe2(SO4)3}}\cdot\%m_{\ce{S|Fe2(SO4)3}} = 20\cdot\frac{3\cdot32}{2\cdot56 + 3(32 + 4\cdot16)} = 4.8$g. \textit{Cách 2.} Sử dụng khối lượng của hợp chất bằng tổng khối lượng của các thành phần: $m_{\ce{S|Fe2(SO4)3}} = m_{\ce{Fe2(SO4)3}} - m_{\ce{Fe|Fe2(SO4)3}} - m_{\ce{O|Fe2(SO4)3}} = 20 - 5.6 - 9.6 = 4.8$g. Dễ thấy Cách 2 tiện hơn sau khi đã biết khối lượng của Fe \& O trong \ce{Fe2(SO4)3}.

\begin{dangtoan}
	Từ lượng nguyên tố tính lượng chất.
\end{dangtoan}

\begin{baitoan}[\cite{Tuan2022}, p. 71]
	Cần bao nhiêu {\rm kg} ure {\rm\ce{(NH2)2CO}} để có {\rm5.6 kg} đạm (nitơ)?
\end{baitoan}

\begin{proof}[Giải]
	$m_{\ce{(NH2)2CO}} = \frac{m_{\ce{N|(NH2)2CO}}}{\%m_{\ce{N|(NH2)2CO}}} = \frac{5.6\cdot(2(14 + 2) + 12 + 16)}{2\cdot14} = 12$kg.
\end{proof}

\begin{dangtoan}
	Từ lượng nguyên tố này tính lượng nguyên tố kia
\end{dangtoan}

\begin{baitoan}[\cite{Tuan2022}, p. 71]
	Trong supephotphat kép thường có bao nhiêu kg canxi ứng với {\rm49.6 kg} photpho?
\end{baitoan}

\begin{dangtoan}
	Tính \% khối lượng các nguyên tố trong hợp chất.
\end{dangtoan}

\begin{baitoan}[\cite{Tuan2022}, p. 71]
	Tính \% khối lượng các nguyên tố trong hợp chất sắt(III) sunfat.
\end{baitoan}

\begin{proof}[Giải]
	CTHH của sắt(III) sunfat: \ce{Fe2(SO4)3}$\Rightarrow\%m_{\ce{Fe}}:\%m_{\ce{S}}:\%m_{\ce{O}} = (2\cdot56):(3\cdot32):(12\cdot16) = 112:96:192 = 7:6:12 = 28\%:24\%:48\%$.
\end{proof}

\begin{dangtoan}
	Tìm nguyên tố.
\end{dangtoan}

\begin{baitoan}[\cite{Tuan2022}, p. 71]
	Nguyên tố X trong bảng tuần hoàn có oxit cao nhất dạng {\rm\ce{X2O5}}. Hợp chất khí với hydro của X chứa $8.82$\% khối lượng hydro. X là nguyên tố nào?
\end{baitoan}

\begin{proof}[Giải]
	Nếu oxit cao nhất là \ce{X2O5} thì hợp chất kí với hydro là \ce{XH3}. $M_X = \frac{3}{8.82}\cdot91.18 = 31\Rightarrow$ X: P.
\end{proof}

%------------------------------------------------------------------------------%

\printbibliography[heading=bibintoc]

\end{document}