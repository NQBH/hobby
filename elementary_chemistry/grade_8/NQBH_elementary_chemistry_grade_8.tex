\documentclass{article}
\usepackage[backend=biber,natbib=true,style=authoryear]{biblatex}
\addbibresource{/home/hong/1_NQBH/reference/bib.bib}
\usepackage[utf8]{vietnam}
\usepackage{tocloft}
\renewcommand{\cftsecleader}{\cftdotfill{\cftdotsep}}
\usepackage[colorlinks=true,linkcolor=blue,urlcolor=red,citecolor=magenta]{hyperref}
\usepackage{amsmath,amssymb,amsthm,mathtools,float,graphicx,algpseudocode,algorithm,tcolorbox,tikz,tkz-tab,subcaption}
\DeclareMathOperator{\arccot}{arccot}
\usepackage[inline]{enumitem}
\allowdisplaybreaks
\numberwithin{equation}{section}
\newtheorem{assumption}{Assumption}[section]
\newtheorem{nhanxet}{Nhận xét}[section]
\newtheorem{conjecture}{Conjecture}[section]
\newtheorem{corollary}{Corollary}[section]
\newtheorem{hequa}{Hệ quả}[section]
\newtheorem{definition}{Definition}[section]
\newtheorem{dinhnghia}{Định nghĩa}[section]
\newtheorem{example}{Example}[section]
\newtheorem{vidu}{Ví dụ}[section]
\newtheorem{lemma}{Lemma}[section]
\newtheorem{notation}{Notation}[section]
\newtheorem{principle}{Principle}[section]
\newtheorem{problem}{Problem}[section]
\newtheorem{baitoan}{Bài toán}[section]
\newtheorem{proposition}{Proposition}[section]
\newtheorem{menhde}{Mệnh đề}[section]
\newtheorem{question}{Question}[section]
\newtheorem{cauhoi}{Câu hỏi}[section]
\newtheorem{quytac}{Quy tắc}
\newtheorem{remark}{Remark}[section]
\newtheorem{luuy}{Lưu ý}[section]
\newtheorem{theorem}{Theorem}[section]
\newtheorem{tiende}{Tiên đề}[section]
\newtheorem{dinhly}{Định lý}[section]
\usepackage[left=0.5in,right=0.5in,top=1.5cm,bottom=1.5cm]{geometry}
\usepackage{fancyhdr}
\pagestyle{fancy}
\fancyhf{}
\lhead{\small \textsc{Sect.} ~\thesection}
\rhead{\small \nouppercase{\leftmark}}
\renewcommand{\sectionmark}[1]{\markboth{#1}{}}
\cfoot{\thepage}
\def\labelitemii{$\circ$}

\title{Some Topics in Elementary Chemistry\texttt{/}Grade 8}
\author{Nguyễn Quản Bá Hồng\footnote{Independent Researcher, Ben Tre City, Vietnam\\e-mail: \texttt{nguyenquanbahong@gmail.com}; website: \url{https://nqbh.github.io}.}}
\date{\today}

\begin{document}
\maketitle
\begin{abstract}
	Tóm tắt kiến thức Hóa học lớp 8 theo chương trình giáo dục của Việt Nam \& một số chủ đề nâng cao.
\end{abstract}
\setcounter{secnumdepth}{4}
\setcounter{tocdepth}{3}
\tableofcontents
\newpage

%------------------------------------------------------------------------------%

``Hóa học là khoa học nghiên cứu các chất, sự biến đổi chất, \& ứng dụng của chúng. Hóa học có vai trò rất quan trọng trong cuộc sống của chúng ta. Khi học tập môn Hóa học, cần thực hiện các hoạt động sau: Tự thu thập tìm kiếm kiến thức, xử lý thông tin, vận dụng, \& ghi nhớ. Học tốt môn Hóa học là nắm vững \& có khả năng vận dụng kiến thức đã học.'' ``Để học tốt môn Hóa học cần phải:
\begin{enumerate*}
	\item[$\bullet$] Biết làm thí nghiệm hóa học, biết quan sát hiện tượng trong thí nghiệm, trong thiên nhiên cũng như trong cuộc sống.
	\item[$\bullet$] \textit{Có hứng thú say mê, chủ động, chú ý rèn luyện phương pháp tư duy, óc suy luận sáng tạo}.
	\item[$\bullet$] \textit{Cũng phải nhớ nhưng nhớ 1 cách chọn lọc thông minh}.
	\item[$\bullet$] \textit{Phải đọc thêm sách, rèn luyện lòng ham thích đọc sách \& cách đọc sách}.'' -- \cite[p. 5]{SGK_Hoa_Hoc_8}
\end{enumerate*}

\section{Chất -- Nguyên Tử -- Phân Tử}
\textsf{\textbf{Nội dung.} Chất, hỗn hợp, nguyên tử \& thành phần cấu tạo của nguyên tử, nguyên tố hóa học, nguyên tử khối, phân tử, phân tử khối, đơn chất, hợp chất, công thức hóa học dùng biểu diễn chất, hóa trị.}

\subsection{Chất}

\subsubsection{Chất có ở đâu?}

%------------------------------------------------------------------------------%

\subsection{Nguyên Tử}

%------------------------------------------------------------------------------%

\subsection{Nguyên Tố Hóa Học}

%------------------------------------------------------------------------------%

\subsection{Đơn Chất \& Hợp Chất -- Phân Tử}

%------------------------------------------------------------------------------%

\subsection{Công Thức Hóa Học}

%------------------------------------------------------------------------------%

\subsection{Hóa Trị}

%------------------------------------------------------------------------------%

\section{Phản Ứng Hóa Học}

\subsection{Sự Biến Đổi Chất}

%------------------------------------------------------------------------------%

\subsection{Phản Ứng Hóa Học}

%------------------------------------------------------------------------------%

\subsection{Định Luật Bảo Toàn Khối Lượng}

%------------------------------------------------------------------------------%

\subsection{Phương Trình Hóa Học}

%------------------------------------------------------------------------------%

\section{Mol \& Tính Toán Hóa Học}
\textsf{\textbf{Nội dung.} Mol, khối lượng mol, thể tích mol, chuyển đổi giữa khối lượng, thể tích, \& lượng chất, tỷ khối của 2 khí, sử dụng công thức hóa học \& phương trình hóa học trong tính toán hóa học.}

\subsection{Mol}
``$\ldots$ kích thước \& khối lượng của nguyên tử, phân tử là vô cùng nhỏ bé, không thể cân, đo, đếm chúng được. Nhưng trong Hóa học lại cần biết có bao nhiêu nguyên tử hoặc phân tử \& khối lượng, thể tích của chúng tham gia \& tạo thành trong 1 phản ứng hóa học. Để đáp ứng được yêu cầu này, các nhà khoa học đã đề xuất 1 khái niệm dành cho các hạt vi mô (i.e., hạt vô cùng nhỏ), đó là MOL.'' -- \cite[p. 63]{SGK_Hoa_Hoc_8}

\subsubsection{Mol}

\begin{dinhnghia}[Mol, số Avogadro]
	``\emph{Mol} là lượng chất có chứa $6\cdot 10^{23}$ nguyên tử hoặc phân tử của chất đó. Con số $6\cdot 10^{23}$ được gọi là \emph{số Avogadro} \& được ký hiệu là $N$.'' -- \cite[p. 63]{SGK_Hoa_Hoc_8}
\end{dinhnghia}

\subsubsection{Khối lượng mol}

\begin{dinhnghia}[Khối lượng mol]
	``\emph{Khối lượng mol} (ký hiệu là $M$) của 1 chất là khối lượng tính bằng gam của $N$ nguyên tử hoặc phân tử chất đó.
\end{dinhnghia}
Khối lượng mol nguyên tử hay phân tử của 1 chất có \textit{cùng số trị} với nguyên tử khối hay phân tử khối của chất đó.'' -- \cite[p. 63]{SGK_Hoa_Hoc_8}

\subsubsection{Thể tích mol của chất khí}

\begin{dinhnghia}[Thể tích mol của chất khí]
	``\emph{Thể tích mol của chất khí} là thể tích chiếm bởi $N$ phân tử của chất khí đó.
\end{dinhnghia}
Người ta đã xác định được rằng: \textit{1 mol của bất kỳ chất khí nào, trong cùng điều kiện về nhiệt độ \& áp suất, đều chiếm những thể tích bằng nhau. Nếu ở nhiệt độ $0^\circ$C \& áp suất 1 atm (được gọi là \emph{điều kiện tiêu chuẩn}, abbr., \emph{đktc})}, thì thể tích đó là $22.4$ l. Như vậy, những chất khí khác nhau thường có khối lượng mol không như nhau, nhưng thể tích mol của chúng (đo ở cùng nhiệt độ \& áp suất) là bằng nhau.'' ``Ở điều kiện bình thường ($20^\circ$C \& 1 atm), 1 mol chất khí có thể tích là 24 lít.'' -- \cite[pp. 63--64]{SGK_Hoa_Hoc_8}

%------------------------------------------------------------------------------%

\subsection{Chuyển Đổi Giữa Khối Lượng, Thể Tích, \& Lượng Chất}

%------------------------------------------------------------------------------%

\subsection{Tỷ Khối của Chất Khí}

%------------------------------------------------------------------------------%

\subsection{Tính Theo Công Thức Hóa Học}

%------------------------------------------------------------------------------%

\subsection{Tính Theo Phương Trình Hóa Học}

%------------------------------------------------------------------------------%

\section{Oxi -- Không Khí}

\subsection{Tính Chất của Oxi}

%------------------------------------------------------------------------------%

\subsection{Sự Oxi Hóa -- Phản Ứng Hóa Hợp -- Ứng Dụng của Oxi}

%------------------------------------------------------------------------------%

\subsection{Oxit}

%------------------------------------------------------------------------------%

\subsection{Điều Chế Khí Oxi -- Phản Ứng Phân Hủy}

%------------------------------------------------------------------------------%

\subsection{Không Khí -- Sự Cháy}

%------------------------------------------------------------------------------%

\section{Hydro -- Nước}

\subsection{Tính Chất -- Ứng Dụng của Hydro}

%------------------------------------------------------------------------------%

\subsection{Phản Ứng Oxi Hóa -- Khử}

%------------------------------------------------------------------------------%

\subsection{Điều Chế Khí Hydro -- Phản Ứng Thế}

%------------------------------------------------------------------------------%

\subsection{Nước}

%------------------------------------------------------------------------------%

\subsection{Acid -- Base -- Muối}

%------------------------------------------------------------------------------%

\section{Dung Dịch}

\subsection{Dung Dịch}

%------------------------------------------------------------------------------%

\subsection{Độ Tan của 1 Chất Trong Nước}

%------------------------------------------------------------------------------%

\subsection{Nồng Độ Dung Dịch}

%------------------------------------------------------------------------------%

\subsection{Pha Chế Dung Dịch}

%------------------------------------------------------------------------------%

\printbibliography[heading=bibintoc]
	
\end{document}