\documentclass{article}
\usepackage[backend=biber,natbib=true,style=authoryear]{biblatex}
\addbibresource{/home/hong/1_NQBH/reference/bib.bib}
\usepackage[utf8]{vietnam}
\usepackage{tocloft}
\renewcommand{\cftsecleader}{\cftdotfill{\cftdotsep}}
\usepackage[colorlinks=true,linkcolor=blue,urlcolor=red,citecolor=magenta]{hyperref}
\usepackage{amsmath,amssymb,amsthm,mathtools,float,graphicx,algpseudocode,algorithm,tcolorbox,tikz,tkz-tab,subcaption}
\DeclareMathOperator{\arccot}{arccot}
\usepackage[inline]{enumitem}
\allowdisplaybreaks
\numberwithin{equation}{section}
\newtheorem{assumption}{Assumption}[section]
\newtheorem{nhanxet}{Nhận xét}[section]
\newtheorem{conjecture}{Conjecture}[section]
\newtheorem{corollary}{Corollary}[section]
\newtheorem{hequa}{Hệ quả}[section]
\newtheorem{definition}{Definition}[section]
\newtheorem{dinhnghia}{Định nghĩa}[section]
\newtheorem{example}{Example}[section]
\newtheorem{vidu}{Ví dụ}[section]
\newtheorem{lemma}{Lemma}[section]
\newtheorem{notation}{Notation}[section]
\newtheorem{principle}{Principle}[section]
\newtheorem{problem}{Problem}[section]
\newtheorem{baitoan}{Bài toán}[section]
\newtheorem{proposition}{Proposition}[section]
\newtheorem{menhde}{Mệnh đề}[section]
\newtheorem{question}{Question}[section]
\newtheorem{cauhoi}{Câu hỏi}[section]
\newtheorem{quytac}{Quy tắc}
\newtheorem{remark}{Remark}[section]
\newtheorem{luuy}{Lưu ý}[section]
\newtheorem{theorem}{Theorem}[section]
\newtheorem{tiende}{Tiên đề}[section]
\newtheorem{dinhly}{Định lý}[section]
\usepackage[left=0.5in,right=0.5in,top=1.5cm,bottom=1.5cm]{geometry}
\usepackage{fancyhdr}
\pagestyle{fancy}
\fancyhf{}
\lhead{\small Subsect.~\thesubsection}
\rhead{\small \nouppercase{\leftmark}}
\renewcommand{\subsectionmark}[1]{\markboth{#1}{}}
\cfoot{\thepage}
\def\labelitemii{$\circ$}

\title{Some Topics in Elementary Chemistry\texttt{/}Grade 8}
\author{Nguyễn Quản Bá Hồng\footnote{Independent Researcher, Ben Tre City, Vietnam\\e-mail: \texttt{nguyenquanbahong@gmail.com}; website: \url{https://nqbh.github.io}.}}
\date{\today}

\begin{document}
\maketitle
\begin{abstract}
	Tóm tắt kiến thức Hóa học lớp 8 theo chương trình giáo dục của Việt Nam \& một số chủ đề nâng cao.
\end{abstract}
\setcounter{secnumdepth}{4}
\setcounter{tocdepth}{3}
\tableofcontents
\newpage

%------------------------------------------------------------------------------%

``Hóa học là khoa học nghiên cứu các chất, sự biến đổi chất, \& ứng dụng của chúng. Hóa học có vai trò rất quan trọng trong cuộc sống của chúng ta. Khi học tập môn Hóa học, cần thực hiện các hoạt động sau: Tự thu thập tìm kiếm kiến thức, xử lý thông tin, vận dụng, \& ghi nhớ. Học tốt môn Hóa học là nắm vững \& có khả năng vận dụng kiến thức đã học.'' ``Để học tốt môn Hóa học cần phải:
\begin{enumerate*}
	\item[$\bullet$] Biết làm thí nghiệm hóa học, biết quan sát hiện tượng trong thí nghiệm, trong thiên nhiên cũng như trong cuộc sống.
	\item[$\bullet$] \textit{Có hứng thú say mê, chủ động, chú ý rèn luyện phương pháp tư duy, óc suy luận sáng tạo}.
	\item[$\bullet$] \textit{Cũng phải nhớ nhưng nhớ 1 cách chọn lọc thông minh}.
	\item[$\bullet$] \textit{Phải đọc thêm sách, rèn luyện lòng ham thích đọc sách \& cách đọc sách}.'' -- \cite[p. 5]{SGK_Hoa_Hoc_8}
\end{enumerate*}

\section{Chất -- Nguyên Tử -- Phân Tử}
\textsf{\textbf{Nội dung.} Chất, hỗn hợp, nguyên tử \& thành phần cấu tạo của nguyên tử, nguyên tố hóa học, nguyên tử khối, phân tử, phân tử khối, đơn chất, hợp chất, công thức hóa học dùng biểu diễn chất, hóa trị.}

\subsection{Chất}

\subsubsection{Chất có ở đâu?}
``Tất cả những gì thấy được, kể cả bản thân cơ thể mỗi chúng ta, $\ldots$ đều là những vật thể. Có những vật thể tự nhiên như người, động vật, cây cỏ, sông suối, đất đá, $\ldots$ Nhà ở, đồ dùng, quần áo, sách vở, phương tiện vận chuyển, công cụ sản xuất, $\ldots$ là những \textit{vật thể nhân tạo}. Các vật thể tự nhiên gồm có 1 số chất khác nhau. E.g.: Thân cây mía gồm có các chất: đường (tên hóa học là saccarose), nước, cellulose, $\ldots$; khí quyển gồm có các chất: khí nitơ, khí oxi, $\ldots$; trong nước biển có chất muối ăn (tên hóa học là natri clorua), $\ldots$; đá vôi có thành phần chính là chất canxi carbonat. Còn các \textit{vật thể nhân tạo} được làm bằng vật liệu. Mọi vật liệu đều là chất hay hỗn hợp 1 số chất. E.g.: nhôm, chất dẻo\footnote{Tên gọi chung 1 loại chất mà thông thường được gọi là \textit{nhựa} (e.g., dép nhựa chính là dép làm bằng 1 loại chất dẻo, $\ldots$). Có nhiều loại chất dẻo, tên hóa học khác nhau.}, thủy tinh, $\ldots$ là chất; gỗ gồm có cellulose là chính; thép gồm có sắt \& 1 số chất khác, $\ldots$'' ``Ngày nay, khoa học đã biết hàng chục triệu chất khác nhau. Có những chất sẵn có trong tự nhiên. Nhiều chất do con người điều chế được, e.g.: chất dẻo, cao su, tơ sợi tổng hợp, dược phẩm, thuốc nổ, $\ldots$'' -- \cite[p. 7]{SGK_Hoa_Hoc_8}

\subsubsection{Tính chất của chất}

\paragraph{Mỗi chất có những tính chất nhất định.} ``Trạng thái hay thể (rắn, lỏng, khí), màu, mùi, vị, tính tan hay không tan trong nước (hay trong 1 chất lỏng khác), nhiệt độ nóng chảy, nhiệt độ sôi, khối lượng riêng, tính dẫn điện, dẫn nhiệt, $\ldots$ là những \textit{tính chất vật lý}. Còn khả năng biến đổi thành chất khác, e.g., khả năng bị phân hủy, tính cháy được (khi 1 chất cháy không phải là nó mất đi, mà là biến đổi thành chất khác) là những \textit{tính chất hóa học}.'' -- \cite[p. 8]{SGK_Hoa_Hoc_8}

\textbf{Cách để biết được tính chất của chất.}
\begin{enumerate*}
	\item[(a)] \textit{Quan sát}: Quan sát kỹ 1 chất ta có thể nhận ra 1 số tính chất bề ngoài của nó. E.g., ta biết được lưu huỳnh \& photpho đỏ đều là chất rắn nhưng lưu huỳnh màu vàng tươi; đồng \& nhôm đều có ánh kim, đồng là kim loại màu đỏ, còn nhôm thì màu trắng.
	\item[(b)] \textit{Dùng dụng cụ đo}: Muốn biết được 1 chất nóng chảy hay sôi ở nhiệt độ nào, có khối lượng riêng bằng bao nhiêu phải dùng dụng cụ đo. E.g., theo kết quả đo ta biết được nhiệt độ nóng chảy của lưu huỳnh $t_{\rm nc}^\circ = 113^\circ$C.
	\item[(c)] \textit{Làm thí nghiệm}: Những tính chất như có tan trong nước, có dẫn điện \& dẫn nhiệt hay không thì phải thử, i.e., làm thí nghiệm. Làm thí nghiệm thử tính tan khi pha nước đường hay nước muối. Để thử tính dẫn điện, ta cắm 2 chốt của chui cắm điện cho tiếp xúc với chất (e.g., lưu huỳnh, miếng nhôm, $\ldots$). Bóng đèn sáng hay không là biết chất có dẫn điện hay không. Nhôm \& đồng dẫn được điện, còn lưu huỳnh \& photpho đỏ thì không. Về tính chất hóa học thì đều phải làm thí nghiệm mới biết được.'' -- \cite[p. 8]{SGK_Hoa_Hoc_8}
\end{enumerate*}

\paragraph{Lợi ích của việc hiểu biết tính chất của chất.}
``\begin{enumerate*}
	\item[(a)] \textit{Giúp phân biệt chất này với chất khác, i.e., nhận biết được chất}: Những chất khác nhau có thể có 1 số tính chất giống nhau. Song mỗi chất có 1 số tính chất riêng khác biệt với chất khác. E.g., nước \& cồn (tên hóa học là rượu etylic) đều là chất lỏng trong suốt, không màu, song cồn cháy được, còn nước thì không. Do đó, ta có thể phân biệt được 2 chất.
	\item[(b)] \textit{Biết cách sử dụng chất}: E.g., biết axit sunfuric đặc là chất làm bỏng, cháy da thịt, vải, ta cần phải tránh không để axit này dây vào người, áo quần.
	\item[(c)] \textit{Biết ứng dụng chất thích hợp trong đời sống \& sản xuất}: E.g., cao su là chất không thấm nước lại có tính chất đàn hồi, chịu mài mòn nên được dùng chế tạo lốp xe.'' -- \cite[p. 8]{SGK_Hoa_Hoc_8}
\end{enumerate*}

\subsubsection{Chất tinh khiết}

\paragraph{Hỗn hợp.} ``Quan sát chai nước khoáng \& ống nước cất. Nước bên trong đều trong suốt, không màu. Tất nhiên, cả 2 đều uống được, nhưng nước cất được dùng để pha chế thuốc tiêm \& sử dụng trong phòng thí nghiệm, còn nước khoáng thì không. Nước cất là chất tinh khiết (không có lẫn chất khác), còn nước khoáng có lẫn 1 số chất tan\footnote{Đó là những chất có tên chung là \textit{chất khoáng}. Trên nhãn chai nước khoáng thường ghi hàm lượng các chất khoáng hòa tan.}. Cũng như nước khoáng, nước biển, nước sông suối, nước hồ ao, nước giếng, $\ldots$ kể cả nước máy đều có lẫn 1 số chất khác.

\begin{dinhnghia}[Hỗn hợp]
	2 hay nhiều chất trộn lẫn vào nhau được gọi là \textit{hỗn hợp}.
\end{dinhnghia}
Vậy, nước tự nhiên là 1 hỗn hợp.'' -- \cite[p. 9]{SGK_Hoa_Hoc_8}

\paragraph{Chất tinh khiết.} ``Chưng cất bất kỳ thứ nước tự nhiên nào đều thu được nước cất. \textit{Làm thế nào để khẳng định được nước cất là chất tinh khiết?} Tiến hành đo nhiệt độ nóng chảy, nhiệt độ sôi, khối lượng riêng của nước cất. Chỉ nước tinh khiết mới có: $t_{\rm nc}^0 = 0^\circ$C, $t_{\rm s}^\circ = 100^\circ$C, $D = 1{\rm g\texttt{/}cm^3},\ldots$ Với nước tự nhiên, các giá trị này đều sai khác nhiều ít tùy theo các chất khác có lẫn nhiều hay ít.'' -- \cite[p. 10]{SGK_Hoa_Hoc_8}

\paragraph{Tách chất ra khỏi hỗn hợp.} ``\textit{Thí nghiệm}:
\begin{enumerate*}
	\item[$\bullet$] Bỏ muối ăn vào nước, khuấy cho tan được hỗn hợp nước \& muối trong suốt (được gọi là \textit{dung dịch muối ăn}).
	\item[$\bullet$] Đung nóng, nước sôi, \& bay hơi.
	\item[$\bullet$] Muối ăn kết tinh vì có nhiệt độ sôi cao ($t_{\rm s}^\circ = 1450^\circ$C).
\end{enumerate*}
Tương tự, trong nước tự nhiên có hòa tan 1 số chất rắn \& cả chất khí. Khi đun nóng các chất khí thoát đi, những chất rắn lắng xuống, hơi nước bay lên \& ngưng tụ lại thành nước cất. Vậy, dựa vào nhiệt độ sôi khác nhau ta có thể tách riêng được 1 số chất ra khỏi hỗn hợp bằng cách chưng cất. Ngoài ra, có thể dựa vào sự khác nhau về các tính chất khác như khối lượng riêng, tính tan, $\ldots$ \& bằng cách thích hợp ta đều có thể tách riêng được chất. I.e., dựa vào tính chất vật lý khác nhau ta có thể tách riêng 1 số chất ra khỏi hỗn hợp.'' -- \cite[p. 10]{SGK_Hoa_Hoc_8}
\vspace{2mm}

\noindent\textbf{Tóm tắt kiến thức.}
``\begin{enumerate*}
	\item[\textbf{1.}] Chất có khắp nơi, ở đâu có vật thể là ở đó có chất. Mỗi chất (tinh khiết) có những tính chất vật lý \& hóa học nhất định.
	\item[\textbf{2.}] Nước tự nhiên gồm nhiều chất trộn lẫn là 1 hỗn hợp. Nước cất là chất tinh khiết.
	\item[\textbf{3.}] Dựa vào sự khác nhau về tính chất vật lý có thể tách 1 chất ra khỏi hỗn hợp.'' -- \cite[p. 11]{SGK_Hoa_Hoc_8}
\end{enumerate*}

%------------------------------------------------------------------------------%

\subsection{Thực Hành: Tính Chất Nóng Chảy của Chất Tách Chất Từ Hỗn Hợp}
\textsf{\textbf{Nội dung.} Theo dõi sự nóng chảy của 1 số chất, qua đó thấy được sự khác nhau về tính chất này giữa các chất; biết cách tách riêng chất từ hỗn hợp 2 chất.}

\subsubsection{Theo dõi sự nóng chảy của các chất parafin \& lưu huỳnh}
``Lấy 1 ít mỗi chất vào 2 ống nghiệm. Đặt đứng 2 ống nghiệm \& nhiệt kế vào 1 cốc nước. Đun nóng cốc nước bằng đèn cồn. Theo dõi nhiệt độ ghi trên nhiệt kế, đồng thời quan sát chất nào nóng chảy. Khi nước sôi thì ngừng đun.'' -- \cite[p. 12]{SGK_Hoa_Hoc_8}

\subsubsection{Tách riêng chất từ hỗn hợp muối ăn \& cát}
``Bỏ hỗn hợp muối ăn \& cát vào cốc nước, khuấy đều. Đổ nước từ từ theo đũa thủy tinh qua phễu có giấy lọc, thu lấy phần nước lọc vào cốc. Đổ phần nước lọc vào ống nghiệm. Dùng kẹp gỗ cặp ống nghiệm rồi đun nóng cho đến khi nước bay hơi hết. Khi đun nóng, để ống nghiệm hơi nghiêng, lúc đầu hơ dọc ống nghiệm trên ngọn lửa cho nóng đều, sau mới đun phần đáy ống. Hướng miệng ống nghiệm về phía không có người. Quan sát chất còn lại trong ống nghiệm \& trên giấy lọc.'' -- \cite[p. 13]{SGK_Hoa_Hoc_8}

%------------------------------------------------------------------------------%

\subsection{Nguyên Tử}

\subsubsection{Khái niệm nguyên tử}
``Các chất đều được tạo ra từ những hạt vô cùng nhỏ, trung hòa về điện được gọi là \textit{nguyên tử}. Có hàng chục triệu chất khác nhau, nhưng chỉ có trên 1 trăm loại nguyên tử. Hình dung nguyên tử như 1 quả cầu cực kỳ bé, đường kính vào cỡ $10^{-8} = 0.00000001$cm. Nguyên tử gồm hạt nhân mang điện tích dương \& vỏ tạo bởi 1 hay nhiều electron mang điện tích âm. Electron, ký hiệu là e, có điện tích âm nhỏ nhất \& quy ước ghi bằng dấu âm ($-$).'' -- \cite[p. 14]{SGK_Hoa_Hoc_8}

\subsubsection{Hạt nhân nguyên tử}
``Hạt nhân nguyên tử tạo bởi proton \& neutron. Proton ký hiệu là p, có điện tích như electron nhưng trái dấu, ghi bằng dấu dương ($+$). Neutron không mang điện, ký hiệu là $n$. Các nguyên tử cùng loại đều có cùng số proton trong hạt nhân. \& trong 1 nguyên tử có bao nhiêu proton thì cũng có bấy nhiêu electron, i.e.: số p $=$ số e. Proton \& neutron có cùng khối lượng, còn electron có khối lượng rất bé (chỉ $\approx0.0005$ lần khối lượng của proton), không đáng kể. Vì vậy, khối lượng của hạt nhân được coi là khối lượng của nguyên tử.'' -- \cite[p. 14]{SGK_Hoa_Hoc_8}

\subsubsection{Lớp electron}
``Trong nguyên tử, electron luôn chuyển động rất nhanh quanh hạt nhân \& sắp xếp thành từng lớp, mỗi lớp có 1 số electron nhất định.'' -- \cite[p. 14]{SGK_Hoa_Hoc_8}. ``Nguyên tử có thể liên kết được với nhau. Chính nhờ electron mà nguyên tử có khả năng này.'' -- \cite[p. 15]{SGK_Hoa_Hoc_8}
\vspace{2mm}

\noindent\textbf{Tóm tắt kiến thức.}
``\begin{enumerate*}
	\item[\textbf{1.}] Nguyên tử là hạt vô cùng nhỏ \& trung hòa về điện. Nguyên tử gồm hạt nhân mang điện tích dương \& vỏ tạo bởi 1 hay nhiều electron mang điện tích âm.
	\item[\textbf{2.}] Hạt nhân tạo bởi proton \& neutron.
	\item[\textbf{3.}] Trong mỗi nguyên tử, số proton (p,$+$) bằng số electron (e,$-$).
	\item[\textbf{4.}] Electron luôn chuyển động quanh hạt nhân \& sắp xếp thành từng lớp.'' -- \cite[p. 15]{SGK_Hoa_Hoc_8}
\end{enumerate*}

``\begin{enumerate*}
	\item[\textbf{1.}] Nếu xếp hàng liền nhau thì với độ dài $1$mm thôi cũng đã có từ vài triệu đến hơn chục triệu nguyên tử. E.g., phải $4$ triệu nguyên tử sắt mới dài được thế. Nhỏ bé như vậy nhưng nguyên tử đã được con người nghĩ đến từ thế kỷ V trước công nguyên. Cho đến đầu thế kỷ XIX mới có những quan niệm đúng về nguyên tử. Nhưng đó cũng chỉ là những giả thuyết khoa học. Sang thế kỷ XX mới có những bằng chứng về sự tồn tại của nguyên tử. Khoảng giữa thế kỷ XX thì chụp được ảnh nguyên tử trên đầu nhọn rất mảnh của 1 sợi kim loại vonfam (kim loại làm dây tóc bóng đèn điện). \& đến năm 1999, nhờ thiết bị coi như 1 camera nhanh nhất hiện nay trên thế giới, người ta đã quan sát được nguyên tử đang chuyển động trong 1 phản ứng hóa học. Điều này mở đường cho Hóa học sẽ phát triển mạnh mẽ ở thế kỷ XXI.
	\item[\textbf{2.}] Nguyên tử hydro bé nhất. Về tầm vóc thì hydro chỉ đáng là em út. Nhưng về tuổi tác, nguyên tử hydro có thể coi là anh cả. Trong Vũ Trụ thời nguyên thủy, nguyên tử hydro được tạo thành trước từ 1 proton \& 1 electron. Mãi sau mới đến các nguyên tử khác như heli,$\ldots$, carbon, oxi, $\ldots$, sắt, $\ldots$, được tạo thành theo cách tăng dần số proton (đồng thời cả số neutron) trong hạt nhân. Cho đến nay, nguyên tử hydro vẫn có nhiều nhất, chiếm $75$\% khối lượng toàn Vũ Trụ. Trong tự nhiên, nguyên tử hydro có 1 người anh em sinh đôi là đơteri, với tỷ lệ rất ít, $\approx0.016$\%. Nguyên tử đơteri còn có tên là ``hydro nặng'', chỉ khác là có thêm 1 neutron trong hạt nhân.'' -- \cite[p. 16]{SGK_Hoa_Hoc_8}
\end{enumerate*}

%------------------------------------------------------------------------------%

\subsection{Nguyên Tố Hóa Học}

%------------------------------------------------------------------------------%

\subsection{Đơn Chất \& Hợp Chất -- Phân Tử}

%------------------------------------------------------------------------------%

\subsection{Công Thức Hóa Học}

%------------------------------------------------------------------------------%

\subsection{Hóa Trị}

%------------------------------------------------------------------------------%

\section{Phản Ứng Hóa Học}

\subsection{Sự Biến Đổi Chất}

%------------------------------------------------------------------------------%

\subsection{Phản Ứng Hóa Học}

%------------------------------------------------------------------------------%

\subsection{Định Luật Bảo Toàn Khối Lượng}

%------------------------------------------------------------------------------%

\subsection{Phương Trình Hóa Học}

%------------------------------------------------------------------------------%

\section{Mol \& Tính Toán Hóa Học}
\textsf{\textbf{Nội dung.} Mol, khối lượng mol, thể tích mol, chuyển đổi giữa khối lượng, thể tích, \& lượng chất, tỷ khối của 2 khí, sử dụng công thức hóa học \& phương trình hóa học trong tính toán hóa học.}

\subsection{Mol}
``$\ldots$ kích thước \& khối lượng của nguyên tử, phân tử là vô cùng nhỏ bé, không thể cân, đo, đếm chúng được. Nhưng trong Hóa học lại cần biết có bao nhiêu nguyên tử hoặc phân tử \& khối lượng, thể tích của chúng tham gia \& tạo thành trong 1 phản ứng hóa học. Để đáp ứng được yêu cầu này, các nhà khoa học đã đề xuất 1 khái niệm dành cho các hạt vi mô (i.e., hạt vô cùng nhỏ), đó là MOL.'' -- \cite[p. 63]{SGK_Hoa_Hoc_8}

\subsubsection{Mol}

\begin{dinhnghia}[Mol, số Avogadro]
	``\emph{Mol} là lượng chất có chứa $6\cdot 10^{23}$ nguyên tử hoặc phân tử của chất đó. Con số $6\cdot 10^{23}$ được gọi là \emph{số Avogadro} \& được ký hiệu là $N$.'' -- \cite[p. 63]{SGK_Hoa_Hoc_8}
\end{dinhnghia}

\subsubsection{Khối lượng mol}

\begin{dinhnghia}[Khối lượng mol]
	``\emph{Khối lượng mol} (ký hiệu là $M$) của 1 chất là khối lượng tính bằng gam của $N$ nguyên tử hoặc phân tử chất đó.
\end{dinhnghia}
Khối lượng mol nguyên tử hay phân tử của 1 chất có \textit{cùng số trị} với nguyên tử khối hay phân tử khối của chất đó.'' -- \cite[p. 63]{SGK_Hoa_Hoc_8}

\subsubsection{Thể tích mol của chất khí}

\begin{dinhnghia}[Thể tích mol của chất khí]
	``\emph{Thể tích mol của chất khí} là thể tích chiếm bởi $N$ phân tử của chất khí đó.
\end{dinhnghia}
Người ta đã xác định được rằng: \textit{1 mol của bất kỳ chất khí nào, trong cùng điều kiện về nhiệt độ \& áp suất, đều chiếm những thể tích bằng nhau. Nếu ở nhiệt độ $0^\circ$C \& áp suất 1 atm (được gọi là \emph{điều kiện tiêu chuẩn}, abbr., \emph{đktc})}, thì thể tích đó là $22.4$ l. Như vậy, những chất khí khác nhau thường có khối lượng mol không như nhau, nhưng thể tích mol của chúng (đo ở cùng nhiệt độ \& áp suất) là bằng nhau.'' ``Ở điều kiện bình thường ($20^\circ$C \& 1 atm), 1 mol chất khí có thể tích là 24 lít.'' -- \cite[pp. 63--64]{SGK_Hoa_Hoc_8}
\vspace{2mm}

\noindent\textbf{Tóm tắt kiến thức.}
``\begin{enumerate*}
	\item[\textbf{1.}] Mol là lượng chất có chứa $N = 6\cdot 10^{23}$ nguyên tử hoặc phân tử chất đó.
	\item[\textbf{2.}] Khối lượng mol của 1 chất là khối lượng của $N$ nguyên tử hoặc phân tử chất đó, tính bằng gam, có số trị bằng nguyên tử khối hoặc phân tử khối.
	\item[\textbf{3.}] Thể tích mol của chất khí là thể tích chiếm bởi $N$ phân tử chất đó. Ở đktc, thể tích mol của các chất khí đều bằng $22.4$ lít.'' -- \cite[p. 64]{SGK_Hoa_Hoc_8}
\end{enumerate*}

%------------------------------------------------------------------------------%

\subsection{Chuyển Đổi Giữa Khối Lượng, Thể Tích, \& Lượng Chất}
``Trong tính toán hóa học, chúng ta thường phải chuyển đổi giữa khối lượng, thể tích của chất khí thành số mol \& ngược lại.'' -- \cite[p. 66]{SGK_Hoa_Hoc_8}

\subsubsection{Cách chuyển đổi giữa lượng chất \& khối lượng chất}
``Nếu đặt $n$ là số mol chất, $M$ là khối lượng mol chất \& $m$ là khối lượng chất, ta có công thức chuyển đổi sau: $m = nM$ (g), rút ra $n = \frac{m}{M}$ (mol), $M = \frac{m}{n}$ (g\texttt{/}mol).'' -- \cite[p. 66]{SGK_Hoa_Hoc_8}

\subsubsection{Cách chuyển đổi giữa lượng chất \& thể tích chất khí}
``Nếu đặt $n$ là số mol chất khí, $V$ là thể tích chất khí (đktc), ta có công thức chuyển đổi: $V = 22.4n$ (l), rút ra $n = \frac{V}{22.4}$ (mol).'' -- \cite[p. 66]{SGK_Hoa_Hoc_8}
\vspace{2mm}

\noindent\textbf{Tóm tắt kiến thức.}
``\begin{enumerate*}
	\item[\textbf{1.}] Công thức chuyển đổi giữa lượng chất $n$ \& khối lượng chất $m$: $n = \frac{n}{M}$ (mol), ($M$: khối lượng mol của chất).
	\item[\textbf{2.}] Công thức chuyển đổi giữa lượng chất $n$ \& thể tích của chất khí $V$ ở điều kiện tiêu chuẩn: $n = \frac{V}{22.4}$ (mol).'' -- \cite[p. 67]{SGK_Hoa_Hoc_8}
\end{enumerate*}

%------------------------------------------------------------------------------%

\subsection{Tỷ Khối của Chất Khí}
``Khi nghiên cứu về tính chất của 1 chất khí nào đó, 1 câu hỏi được đặt ra là chất khí này nặng hay nhẹ hơn chất khí đã biết là bao nhiêu, hoặc nặng hay nhẹ hơn không khí bao nhiêu lần?'' -- \cite[p. 68]{SGK_Hoa_Hoc_8}

\subsubsection{Cách nhận biết khí A nặng\texttt{/}nhẹ hơn khí B}
``Để biết khí A nặng hay nhẹ hơn khí B bằng bao nhiêu lần, ta so sánh khối lượng mol của khí A ($M_{\rm A}$) với khối lượng mol của khí B ($M_{\rm B}$): $d_{A\texttt{/}B} = \frac{M_A}{M_B}$, $d_{A\texttt{/}B}$ là \textit{tỷ khối} của khí A đối với khí B.'' -- \cite[p. 68]{SGK_Hoa_Hoc_8}

\subsubsection{Cách nhận biết khí A nặng hay nhẹ hơn không khí}
``Để biết khí A nặng hay nhẹ hơn không khí bằng bao nhiêu lần, ta so sánh khối lượng mol của khí A ($M_{\rm A}$) với khối lượng ``mol không khí'' là 29 g\texttt{/}mol. $d_{\rm A\texttt{/}kk} = \frac{M_{\rm A}}{29}$, $d_{\rm A\texttt{/}kk}$ là \textit{tỷ khối} của khí A đối với không khí. Khói lượng ``mol không khí'' là khối lượng của $0.8$ mol khí nitơ ($\rm N_2$) $+$ khối lượng của $0.2$ mol khí oxi $(\rm O_2$): $M_{\rm kk} = 28\cdot 0.8 + 32\cdot 0.2 = 28.8\approx 29$ (g\texttt{/}mol).'' -- \cite[p. 68]{SGK_Hoa_Hoc_8}
\vspace{2mm}

\noindent\textbf{Tóm tắt kiến thức.} ``Công thức tính tỷ khối của: Khí A đối với khí B: $d_{\rm A\texttt{/}B} = \frac{M_{\rm A}}{M_{\rm B}}$. Khí A đối với không khí: $d_{\rm A\texttt{/}kk} = \frac{M_{\rm A}}{29}$.''

``Trong lòng đất luôn luôn xảy ra sự phân hủy 1 số hợp chất vô cơ \& hợp chất hữu cơ, sinh ra khí Carbon dioxide $\rm CO_2$. Khí cacbon dioxit không có màu, không có mùi, không duy trì sự cháy \& sự sống của con người \& động vật. Mặt khác, khí Carbon dioxide lại nặng hơn không khí $1.52$ lần. Vì vậy, khí carbon dioxide thường tích tụ trong đáy giếng khơi, trên nền hang sâu. Người \& động vật xuống những nơi này sẽ bị chết ngạt nếu không mang theo bình dưỡng khí hoặc thông khí trước khi xuống.'' -- \cite[p. 69]{SGK_Hoa_Hoc_8}

%------------------------------------------------------------------------------%

\subsection{Tính Theo Công Thức Hóa Học}

%------------------------------------------------------------------------------%

\subsection{Tính Theo Phương Trình Hóa Học}

%------------------------------------------------------------------------------%

\section{Oxi -- Không Khí}

\subsection{Tính Chất của Oxi}

%------------------------------------------------------------------------------%

\subsection{Sự Oxi Hóa -- Phản Ứng Hóa Hợp -- Ứng Dụng của Oxi}

%------------------------------------------------------------------------------%

\subsection{Oxit}

%------------------------------------------------------------------------------%

\subsection{Điều Chế Khí Oxi -- Phản Ứng Phân Hủy}

%------------------------------------------------------------------------------%

\subsection{Không Khí -- Sự Cháy}

%------------------------------------------------------------------------------%

\section{Hydro -- Nước}

\subsection{Tính Chất -- Ứng Dụng của Hydro}

%------------------------------------------------------------------------------%

\subsection{Phản Ứng Oxi Hóa -- Khử}

%------------------------------------------------------------------------------%

\subsection{Điều Chế Khí Hydro -- Phản Ứng Thế}

%------------------------------------------------------------------------------%

\subsection{Nước}

%------------------------------------------------------------------------------%

\subsection{Acid -- Base -- Muối}

%------------------------------------------------------------------------------%

\section{Dung Dịch}

\subsection{Dung Dịch}

%------------------------------------------------------------------------------%

\subsection{Độ Tan của 1 Chất Trong Nước}

%------------------------------------------------------------------------------%

\subsection{Nồng Độ Dung Dịch}

%------------------------------------------------------------------------------%

\subsection{Pha Chế Dung Dịch}

%------------------------------------------------------------------------------%

\printbibliography[heading=bibintoc]
	
\end{document}