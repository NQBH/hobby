\documentclass{article}
\usepackage[backend=biber,natbib=true,style=authoryear,maxbibnames=10]{biblatex}
\addbibresource{/home/nqbh/reference/bib.bib}
\usepackage[utf8]{vietnam}
\usepackage{tocloft}
\renewcommand{\cftsecleader}{\cftdotfill{\cftdotsep}}
\usepackage[colorlinks=true,linkcolor=blue,urlcolor=red,citecolor=magenta]{hyperref}
\usepackage{amsmath,amssymb,amsthm,float,graphicx,mathtools,tikz,tipa}
\usepackage[version=4]{mhchem}
\allowdisplaybreaks
\newtheorem{assumption}{Assumption}
\newtheorem{baitoan}{Bài toán}
\newtheorem{cauhoi}{Câu hỏi}
\newtheorem{conjecture}{Conjecture}
\newtheorem{corollary}{Corollary}
\newtheorem{dangtoan}{Dạng toán}
\newtheorem{definition}{Definition}
\newtheorem{dinhly}{Định lý}
\newtheorem{dinhnghia}{Định nghĩa}
\newtheorem{example}{Example}
\newtheorem{ghichu}{Ghi chú}
\newtheorem{hequa}{Hệ quả}
\newtheorem{hypothesis}{Hypothesis}
\newtheorem{lemma}{Lemma}
\newtheorem{luuy}{Lưu ý}
\newtheorem{nhanxet}{Nhận xét}
\newtheorem{notation}{Notation}
\newtheorem{note}{Note}
\newtheorem{principle}{Principle}
\newtheorem{problem}{Problem}
\newtheorem{proposition}{Proposition}
\newtheorem{question}{Question}
\newtheorem{remark}{Remark}
\newtheorem{theorem}{Theorem}
\newtheorem{vidu}{Ví dụ}
\usepackage[left=1cm,right=1cm,top=5mm,bottom=5mm,footskip=4mm]{geometry}

\title{Acid, Base, pH, Oxide, Salt -- Muối}
\author{Nguyễn Quản Bá Hồng\footnote{Independent Researcher, Ben Tre City, Vietnam\\e-mail: \texttt{nguyenquanbahong@gmail.com}; website: \url{https://nqbh.github.io}.}}
\date{\today}

\begin{document}
\maketitle
\begin{abstract}
	\textsc{[en]} This text is a collection of problems, from easy to advanced, about \textit{acid base pH oxide salt}. This text is also a supplementary material for my lecture note on Elementary Chemistry, which is stored \& downloadable at the following link: \href{https://github.com/NQBH/hobby/blob/master/elementary_chemistry/grade_8/NQBH_elementary_chemistry_grade_8.pdf}{GitHub\texttt{/}NQBH\texttt{/}hobby\texttt{/}elementary chemistry\texttt{/}grade 8\texttt{/}lecture}\footnote{\textsc{url}: \url{https://github.com/NQBH/hobby/blob/master/elementary_chemistry/grade_8/NQBH_elementary_chemistry_grade_8.pdf}.}. The latest version of this text has been stored \& downloadable at the following link: \href{https://github.com/NQBH/hobby/blob/master/elementary_chemistry/acid_base_pH_oxide_salt/NQBH_acid_base_pH_oxide_salt.pdf}{GitHub\texttt{/}NQBH\texttt{/}hobby\texttt{/}elementary chemistry\texttt{/}grade 8\texttt{/}acid base pH oxide salt}\footnote{\textsc{url}: \url{https://github.com/NQBH/hobby/blob/master/elementary_chemistry/acid_base_pH_oxide_salt/NQBH_acid_base_pH_oxide_salt.pdf}.}.
	\vspace{2mm}
	
	\textsc{[vi]} Tài liệu này là 1 bộ sưu tập các bài tập chọn lọc từ cơ bản đến nâng cao về \textit{phản ứng hóa học}. Tài liệu này là phần bài tập bổ sung cho tài liệu chính -- bài giảng \href{https://github.com/NQBH/hobby/blob/master/elementary_chemistry/grade_8/NQBH_elementary_chemistry_grade_8.pdf}{GitHub\texttt{/}NQBH\texttt{/}hobby\texttt{/}elementary chemistry\texttt{/}grade 8\texttt{/}lecture} của tác giả viết cho Hóa Học Sơ Cấp. Phiên bản mới nhất của tài liệu này được lưu trữ \& có thể tải xuống ở link sau: \href{https://github.com/NQBH/hobby/blob/master/elementary_chemistry/grade_8/real/NQBH_real.pdf}{GitHub\texttt{/}NQBH\texttt{/}hobby\texttt{/}elementary chemistry\texttt{/}grade 8\texttt{/}acid base pH oxide salt}.
\end{abstract}
\setcounter{secnumdepth}{4}
\setcounter{tocdepth}{3}
\tableofcontents
\newpage

%------------------------------------------------------------------------------%

\section{Acid}

\noindent\fbox{%
	\parbox{\textwidth}{%
		\noindent\textsf{\textbf{Kiến thức cốt lõi.}} \fbox{\bf 1} \textit{Acid} là những hợp chất trong phân tử có nguyên tử hydrogen liên kết với góc acid. Khi tan trong nước, acid tạo ra ion \ce{H+}. \fbox{\bf 2} \textit{Dung dịch acid} có vị chua, làm quỳ tím chuyển sang màu đỏ, tác dụng với nhiều kim loại tạo ra khí hydrogen. \fbox{\bf 3} Hydrochloric acid, sulfuric acid, \& acetic acid là những acid có nhiều ứng dụng trong đời sống \& trong công nghiệp. 
	}%
}


%------------------------------------------------------------------------------%

\section{Base}

\noindent\fbox{%
	\parbox{\textwidth}{%
		\noindent\textsf{\textbf{Kiến thức cốt lõi.}} \fbox{\bf 1} \textit{Base} là những hợp chất trong phân tử có nguyên tử kim loại liên kết với nhóm hydroxide. Khi tan trong nước, base tạo ra ion \ce{OH-}. \fbox{\bf 2} Base tan trong nước được gọi là \textit{kiềm}. \fbox{\bf 3} \textit{Dung dịch base} làm quỳ tím chuyển sang màu xanh, phenolphthalein không màu chuyển sang màu hồng. \fbox{\bf 4} Base tác dụng với dung dịch acid tạo thành muối \& nước.
	}%
}

%------------------------------------------------------------------------------%

\section{Thang pH}

\noindent\fbox{%
	\parbox{\textwidth}{%
		\noindent\textsf{\textbf{Kiến thức cốt lõi.}} \fbox{\bf 1} Để biểu thị độ acid hoặc base của dung dịch, ta dùng giá trị pH. pH $= 7$: dung dịch có môi trường trung tính. pH $> 7$: dung dịch có môi trường base. pH $< 7$: dung dịch có môi trường acid. \fbox{\bf 2} pH của môi trường có ảnh hưởng mạnh đến đời sống của động vật \& thực vật. \fbox{\bf 3} Để xác định giá trị pH gần đúng của dung dịch, có thể dùng giấy chỉ thị màu.
	}%
}

%------------------------------------------------------------------------------%

\section{Oxide}

\noindent\fbox{%
	\parbox{\textwidth}{%
		\noindent\textsf{\textbf{Kiến thức cốt lõi.}} \fbox{\bf 1} \textit{Oxide} là hợp chất của oxygen với 1 nguyên tố khác. \fbox{\bf 2} Oxide được phân thành 4 loại: oxide base, oxide acid, oxide lưỡng tính, \& oxide trung tính. \fbox{\bf 3} Oxide base tác dụng với dung dịch acid tạo ra muối \& nước. \fbox{\bf 4} Oxide acid tác dụng với dung dịch base tạo ra muối \& nước.
	}%
}

%------------------------------------------------------------------------------%

\section{Salt -- Muối}

\noindent\fbox{%
	\parbox{\textwidth}{%
		\noindent\textsf{\textbf{Kiến thức cốt lõi.}} \fbox{\bf 1} \textit{Muối} là những hợp chất được tạo ra khi thay thế ion \ce{H+} trong acid bằng ion kim loại hoặc ion ammonium \ce{NH4+}. \fbox{\bf 2} Muối tác dụng với kim loại, dung dịch acid, dung dịch base, dung dịch muối. \fbox{\bf 3} Muối có thể được tạo ra bằng cách cho dung dịch acid tác dụng với: base, oxide base, muối hoặc cho 2 dung dịch muối tác dụng với nhau, $\ldots$. \fbox{\bf 4} Acid, base, \& oxide có các tính chất hóa học sau: Dung dịch acid: làm quỳ tím chuyển sang màu đỏ, tác dụng với kim loại, base, oxide base, muối. Dung dịch base: làm quỳ tím chuyển sang màu xanh, tác dụng với dung dịch acid, oxide acid \& với dung dịch muối. Oxide base tác dụng với dung dịch acid, oxide acid tác dụng với dung dịch base.
	}%
}

%------------------------------------------------------------------------------%

\section{Phân Bón Hóa Học}

\noindent\fbox{%
	\parbox{\textwidth}{%
		\noindent\textsf{\textbf{Kiến thức cốt lõi.}} \fbox{\bf 1} \textit{Phân bón hóa học} là những hóa chất có chứa các nguyên tố dinh dưỡng dùng để bón cho cây trồng \& được chia thành 3 loại: đa lượng, trung lượng, \& vi lượng. \fbox{\bf 2} \textit{Phân đa lượng} gồm: phân đạm cung cấp nguyên tố nitrogen, phân lân cung cấp nguyên tố phosphorus, phân kali cung cấp nguyên tố potassium, phân hỗn hợp cung cấp cho cây 2 hoặc 3 nguyên tố trên. \fbox{\bf 3} Để phát huy tối đa hiệu quả của phân bón, tránh gây tác hại đến môi trường cần phải sử dụng phân bón hóa học đúng loại, đúng lúc, đúng liều lượng, \& đúng cách.
	}%
}

%------------------------------------------------------------------------------%

\printbibliography[heading=bibintoc]
	
\end{document}