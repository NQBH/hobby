\documentclass{article}
\usepackage[backend=biber,natbib=true,style=alphabetic,maxbibnames=50]{biblatex}
\addbibresource{/home/nqbh/reference/bib.bib}
\usepackage[utf8]{vietnam}
\usepackage{tocloft}
\renewcommand{\cftsecleader}{\cftdotfill{\cftdotsep}}
\usepackage[colorlinks=true,linkcolor=blue,urlcolor=red,citecolor=magenta]{hyperref}
\usepackage{amsmath,amssymb,amsthm,float,graphicx,mathtools,tikz,tipa}
\usepackage[version=4]{mhchem}
\allowdisplaybreaks
\newtheorem{assumption}{Assumption}
\newtheorem{baitoan}{Bài toán}
\newtheorem{cauhoi}{Câu hỏi}
\newtheorem{conjecture}{Conjecture}
\newtheorem{corollary}{Corollary}
\newtheorem{dangtoan}{Dạng toán}
\newtheorem{definition}{Definition}
\newtheorem{dinhly}{Định lý}
\newtheorem{dinhnghia}{Định nghĩa}
\newtheorem{example}{Example}
\newtheorem{ghichu}{Ghi chú}
\newtheorem{hequa}{Hệ quả}
\newtheorem{hypothesis}{Hypothesis}
\newtheorem{lemma}{Lemma}
\newtheorem{luuy}{Lưu ý}
\newtheorem{nhanxet}{Nhận xét}
\newtheorem{notation}{Notation}
\newtheorem{note}{Note}
\newtheorem{principle}{Principle}
\newtheorem{problem}{Problem}
\newtheorem{proposition}{Proposition}
\newtheorem{question}{Question}
\newtheorem{remark}{Remark}
\newtheorem{theorem}{Theorem}
\newtheorem{thinghiem}{Thí nghiệm}
\newtheorem{vidu}{Ví dụ}
\usepackage[left=1cm,right=1cm,top=5mm,bottom=5mm,footskip=4mm]{geometry}

\title{Acid, Base, pH, Oxide, Salt -- Muối}
\author{Nguyễn Quản Bá Hồng\footnote{Independent Researcher, Ben Tre City, Vietnam\\e-mail: \texttt{nguyenquanbahong@gmail.com}; website: \url{https://nqbh.github.io}.}}
\date{\today}

\begin{document}
\maketitle
\begin{abstract}
	\textsc{[en]} This text is a collection of problems, from easy to advanced, about \textit{acid base pH oxide salt}. This text is also a supplementary material for my lecture note on Elementary Chemistry, which is stored \& downloadable at the following link: \href{https://github.com/NQBH/hobby/blob/master/elementary_chemistry/grade_8/NQBH_elementary_chemistry_grade_8.pdf}{GitHub\texttt{/}NQBH\texttt{/}hobby\texttt{/}elementary chemistry\texttt{/}grade 8\texttt{/}lecture}\footnote{\textsc{url}: \url{https://github.com/NQBH/hobby/blob/master/elementary_chemistry/grade_8/NQBH_elementary_chemistry_grade_8.pdf}.}. The latest version of this text has been stored \& downloadable at the following link: \href{https://github.com/NQBH/hobby/blob/master/elementary_chemistry/acid_base_pH_oxide_salt/NQBH_acid_base_pH_oxide_salt.pdf}{GitHub\texttt{/}NQBH\texttt{/}hobby\texttt{/}elementary chemistry\texttt{/}grade 8\texttt{/}acid base pH oxide salt}\footnote{\textsc{url}: \url{https://github.com/NQBH/hobby/blob/master/elementary_chemistry/acid_base_pH_oxide_salt/NQBH_acid_base_pH_oxide_salt.pdf}.}.
	\vspace{2mm}
	
	\textsc{[vi]} Tài liệu này là 1 bộ sưu tập các bài tập chọn lọc từ cơ bản đến nâng cao về \textit{phản ứng hóa học}. Tài liệu này là phần bài tập bổ sung cho tài liệu chính -- bài giảng \href{https://github.com/NQBH/hobby/blob/master/elementary_chemistry/grade_8/NQBH_elementary_chemistry_grade_8.pdf}{GitHub\texttt{/}NQBH\texttt{/}hobby\texttt{/}elementary chemistry\texttt{/}grade 8\texttt{/}lecture} của tác giả viết cho Hóa Học Sơ Cấp. Phiên bản mới nhất của tài liệu này được lưu trữ \& có thể tải xuống ở link sau: \href{https://github.com/NQBH/hobby/blob/master/elementary_chemistry/grade_8/real/NQBH_real.pdf}{GitHub\texttt{/}NQBH\texttt{/}hobby\texttt{/}elementary chemistry\texttt{/}grade 8\texttt{/}acid base pH oxide salt}.
\end{abstract}
\setcounter{secnumdepth}{4}
\setcounter{tocdepth}{3}
\tableofcontents
\newpage

%------------------------------------------------------------------------------%

\section{Acid}
\textsf{\textbf{Nội dung.} Khái niệm acid (tạo ra ion \ce{H+}, thí nghiệm của hydrochloric acid (làm đổi màu chất chỉ thị, phản ứng với kim loại), giải thích hiện tượng xảy ra trong thí nghiệm (viết PTHH) \& nhận xét về tính chất của acid, 1 số ứng dụng của 1 số acid thông dụng.}

Các quả có vị chua, e.g., quả sấu, quả me, quả chanh, quả cam, $\ldots$ chứa 1 số loại acid.

\subsection{Khái niệm acid}

\begin{vidu}[\cite{SGK_KHTN_8_Canh_Dieu}, p. 47]
	Giấm ăn hoặc chanh thường được cho vào nước chấm để tạo ra vị chua; sấu, me, hoặc cà chua cũng tạo ra vị chua cho 1 số món ăn.
\end{vidu}
Vị chua của giấm ăn \& các loại quả ở trên được tạo ra bởi 1 loại hợp chất gọi là acid. Khi tan trong nước, acid tạo ra ion \ce{H+} làm cho dung dịch có vị chua.

\begin{dinhnghia}[Acid]
	\emph{Acid} là những hợp chất trong phân tử có nguyên tử hydrogen liên kết với gốc acid. Khi tan trong nước, acid tạo ra ion \emph{\ce{H+}}.
\end{dinhnghia}
Acid tạo ra ion \ce{H+} theo sơ đồ sau:
\begin{align}
	\boxed{\mbox{acid} \to\mbox{ion \ce{H+}} + \mbox{ion âm gốc acid}.}
\end{align}

\begin{vidu}[\cite{SGK_KHTN_8_Canh_Dieu}, p. 47]
	(a) Hydrochloric acid $\to$ Ion hydrogen $+$ Ion chloride: \emph{\ce{HCl -> H+ + Cl-}}. (b) Sulfuric acid $\to$ Ion hydrogen $+$ Ion sulfate: \emph{\ce{H2SO4 -> 2H+ + SO4^2-}}.
\end{vidu}

\begin{baitoan}[\cite{SGK_KHTN_8_Canh_Dieu}, 1, p. 47]
	Nêu đặc điểm chung về thành phần phân tử của các acid.
\end{baitoan}

\begin{baitoan}[\cite{SGK_KHTN_8_Canh_Dieu}, 1, p. 47]
	Viết sơ đồ tạo thành ion \emph{\ce{H+}} từ nitric acid \emph{\ce{HNO3}}.
\end{baitoan}

\subsection{Tính chất hóa học của acid}

\subsubsection{Làm đổi màu chất chỉ thị}

\begin{thinghiem}[\cite{SGK_KHTN_8_Canh_Dieu}, Thí nghiệm 1, p. 48]
	\emph{Chuẩn bị:} Dụng cụ: Mặt kính đồng hồ, ống hút nhỏ giọt. Hóa chất: Dung dịch \emph{HCl} loãng, giấy quỳ tím. \emph{Tiến hành:} Đặt mẩu giấy quỳ tím lên mặt kính đồng hồ, lấy dung dịch \emph{HCl} loãng \& nhỏ 1 giọt lên mẩu giấy quỳ tím. Mô tả các hiện tượng xảy ra.
\end{thinghiem}
Các dung dịch sulfuric acid loãng, acetic acid, $\ldots$ cũng làm giấy quỳ tím chuyển màu tương tự như với dung dịch hydrochloric acid. \textit{Dung dịch acid làm quỳ tím chuyển sang màu đỏ}. Quỳ tím được dùng làm chất chỉ thị màu để nhận ra dung dịch acid. 

\begin{baitoan}[\cite{SGK_KHTN_8_Canh_Dieu}, 2, p. 48]
	Khi thảo luận về tác dụng của dung dịch acid với quỳ tím có 2 ý kiến sau: (a) Nước làm quỳ tím đổi màu. (b) Dung dịch acid làm quỳ tím đổi màu. Đề xuất 1 thí nghiệm để xác định ý kiến đúng trong 2 ý kiến trên.
\end{baitoan}

\begin{baitoan}[\cite{SGK_KHTN_8_Canh_Dieu}, 3, p. 48]
	Lần lượt nhỏ lên 3 mẩu giấy quỳ tím mỗi dung dịch sau: (a) Nước đường. (b) Nước chanh. (c) Nước muối (dung dịch \emph{NaCl}). Trường hợp nào quỳ tím sẽ chuyển sang màu đỏ?
\end{baitoan}

\subsubsection{Tác dụng với kim loại}

\begin{thinghiem}[\cite{SGK_KHTN_8_Canh_Dieu}, Thí nghiệm 2, p. 48]
	\emph{Chuẩn bị:} Dụng cụ: Giá để ống nghiệm, ống nghiệm, ống hút nhỏ giọt. Hóa chất: Dung dịch \emph{HCl} loãng, \emph{Zn} viên. \emph{Tiến hành:} Cho 1 viên \emph{Zn} vào ống nghiệm, sau đó cho thêm vào ống nghiệm $\approx 2$ \emph{mL} dung dịch \emph{HCl} loãng. Mô tả các hiện tượng xảy ra. Những dấu hiệu nào chứng tỏ có các phản ứng hóa học giữa dung dịch \emph{HCl} \& \emph{Zn}?
\end{thinghiem}

\begin{proof}[Giải]
	Dung dịch HCl đã phản ứng với Zn tạo ra chất khí. PTHH của phản ứng trên như sau: zinc $+$ hydrochloric acid $\to$ zinc chloride $+$ hydrogen: \ce{Zn + $2$HCl -> ZnCl2 + H2 ^}.
\end{proof}
Dung dịch các acid khác như sulfuric acid loãng, acetic acid, $\ldots$ cũng có phản ứng hóa học với nhiều kim loại tạo ra muối \& khí hydrogen. \textit{Dung dịch acid tác dụng được với nhiều kim loại tạo ra muối \& khí hydrogen}.\footnote{Riêng \ce{HNO3,H2SO4} \textit{đặc} tác dụng với kim loại sẽ được học sau.}
\begin{align}
	\label{acid + metal}
	\boxed{\mbox{acid} + \mbox{metal}\to\mbox{salt} + \mbox{hydrogen}.}
\end{align}
Cụ thể, với kim loại M hóa trị I \& acid \ce{H_xX} với gốc acid \ce{X^{x-}} có hóa trị $x\in\mathbb{N}^\star$, phương trình \eqref{acid + metal} trở thành:
\begin{align}
	\boxed{\ce{$x$M + H_xX -> M_xX + $\frac{x}{2}$H2 ^},\ \forall x\in\mathbb{N}^\star.}
\end{align}
Với kim loại M hóa trị II \& acid \ce{H_xX} với gốc acid \ce{X^{x-}} có hóa trị $x\in\mathbb{N}^\star$, phương trình \eqref{acid + metal} trở thành:
\begin{equation}
	\boxed{\left\{\begin{split}
			\ce{$x$M + $2$H_xX &-> M_xX2 + $x$H2 ^},&&\forall x\in\mathbb{N}^\star,\,x\ne2,\\
			\ce{M + H2X &-> MX + H2}&&\mbox{if } x = 2\ (\mbox{II}).
		\end{split}\right.}
\end{equation}
Tổng quát, với kim loại M hóa trị $m\in\mathbb{N}^\star$ \& acid \ce{H_xX} với gốc acid \ce{X^{x-}} có hóa trị $x\in\mathbb{N}^\star$, phương trình \eqref{acid + metal} trở thành:
\begin{equation}
	\boxed{\left\{\begin{split}
			\ce{$x$M + $m$H_xX &-> M_xX_m + $\frac{mx}{2}$H2 ^},&&\forall m,x\in\mathbb{N}^\star,\,m\ne x,\\
			\ce{M + H_xX &-> MX + $\frac{x}{2}$H2 ^},&&\forall x\in\mathbb{N}^\star,\mbox{ if } m = x.
		\end{split}\right.}
\end{equation}

\begin{baitoan}[\cite{SGK_KHTN_8_Canh_Dieu}, 1, p. 49]
	Người ta thường tránh muối dưa, cà trong các dụng cụ làm bằng nhôm. Cho biết lý do của việc làm trên.
\end{baitoan}

\begin{baitoan}[\cite{SGK_KHTN_8_Canh_Dieu}, 4, p. 49]
	Viết PTHH xảy ra trong các trường hợp sau: (a) Dung dịch \emph{\ce{H2SO4}} loãng tác dụng với \emph{Zn}. (b) Dung dịch \emph{HCl} loãng tác dụng với \emph{Mg}.
\end{baitoan}

\subsection{Ứng dụng của 1 số acid}

\subsubsection{Hydrochloric acid HCl}
\textit{Hydrochloric acid} có trong dạ dày của người \& động vật giúp tiêu hóa thức ăn. Hydrochloric acid được sử dụng nhiều trong công nghiệp. 1 số ứng dụng quan trọng của hydrochloric acid: tẩy rửa kim loại, sản xuất chất dẻo, điều chiếu glucose \ce{C6H12O6}.

\subsubsection{Sulfuric acid \ce{H2SO4}}
\textit{Sulfuric acid} là 1 hóa chất quan trọng được sử dụng nhiều trong công nghiệp. 1 số ứng dụng quan trọng của sulfuric acid: sản xuất giấy, tơ sợi, sản xuất ắc quy, sản xuất chất dẻo, sản xuất phân bón, sản xuất sơn.

\subsubsection{Acetic acid \ce{CH3COOH}}
\textit{Acetic acid} là 1 acid hữu cơ có trong giấm ăn với nồng độ $\approx4$\%. 1 số ứng dụng của acetic acid: sản xuất tơ nhân tạo, sản xuất thuốc diệt côn trùng, sản xuất phẩm nhuộm, sản xuất dược phẩm, sản xuất chất dẻo.

\begin{baitoan}[\cite{SGK_KHTN_8_Canh_Dieu}, 2, p. 50]
	Nêu tên 1 số món ăn có sử dụng giấm ăn trong quá trình chế biến.
\end{baitoan}
\noindent\fbox{%
	\parbox{\textwidth}{%
		\noindent\textsf{\textbf{Kiến thức cốt lõi.}} \fbox{\bf 1} \textit{Acid} là những hợp chất trong phân tử có nguyên tử hydrogen liên kết với góc acid. Khi tan trong nước, acid tạo ra ion \ce{H+}. \fbox{\bf 2} \textit{Dung dịch acid} có vị chua, làm quỳ tím chuyển sang màu đỏ, tác dụng với nhiều kim loại tạo ra khí hydrogen. \fbox{\bf 3} Hydrochloric acid, sulfuric acid, \& acetic acid là những acid có nhiều ứng dụng trong đời sống \& trong công nghiệp. 
	}%
}

%------------------------------------------------------------------------------%

\section{Base}
\textsf{\textbf{Nội dung.} Khái niệm base (tạo ra ion \ce{OH-}, kiềm là các hydroxide tan tốt trong nước, thí nghiệm base làm đổi màu chất chỉ thị, phản ứng với acid tạo muối, giải thích hiện tượng xảy ra trong thí nghiệm (viết PTHH) \& nhận xét tính chất của base, tra bảng tính tan để biết 1 hydroxide cụ thể thuộc loại kiềm hoặc base không tan.}

\begin{baitoan}[\cite{SGK_KHTN_8_Canh_Dieu}, p. 51]
	Để tránh nguyên liệu bị nát vụn khi chế biến, trong quá trình làm mứt người ta thường ngâm nguyên liệu vào nước vôi trong. Trong quá trình đó, độ chua của 1 số loại quả sẽ giảm đi. Vì sao?
\end{baitoan}

\subsection{Khái niệm base}

\begin{dinhnghia}[Base]
	\emph{Base} là những hợp chất trong phân tử có nguyên tử kim loại liên kết với nhóm hydroxide. Khi tan trong nước, base tạo ra ion \emph{\ce{OH-}}.
\end{dinhnghia}

\begin{vidu}
	(a) Sodium hydroxide $+$ ion sodium $\to$ ion hydroxide: \emph{\ce{NaOH -> Na+ + OH-}}. (b) Calcium hydroxide $\to$ ion calcium $+$ ion hydroxide: \emph{\ce{Ca(OH)2 -> Ca^2+ + 2OH-}}.
\end{vidu}

\begin{baitoan}[\cite{SGK_KHTN_8_Canh_Dieu}, p. 51]
	Trong các chất \emph{\ce{Cu(OH)2,MgSO4,NaCl,Ba(OH)2}}, những chất nào là base?
\end{baitoan}
Tên gọi \& CTHH của 1 số base thông dụng: KOH: potassium hydroxide, \ce{Mg(OH)2}: magnesium hydroxide, \ce{Cu(OH)2}: copper(II) hydroxide.

\subsection{Phân loại base}
Base được chia thành 2 loại chính: base tan \& base không tan trong nước. \textit{Base tan trong nước} còn được gọi là \textit{kiềm}, e.g., NaOH, KOH, \ce{Ba(OH)2}, $\ldots$ Tính tan của các base trong nước được trình bày trong bảng tính tan.

\begin{baitoan}[\cite{SGK_KHTN_8_Canh_Dieu}, 1, p. 52]
	Dựa vào bảng tính tan, cho biết những base nào sau đây là kiềm: \emph{KOH, \ce{Fe(OH)2,Ba(OH)2,Cu(OH)2}}.
\end{baitoan}

\subsection{Tính chất hóa học}

\subsubsection{Làm đổi màu chất chỉ thị}

\begin{thinghiem}[\cite{SGK_KHTN_8_Canh_Dieu}, Thí nghiệm 1, p. 52]
	\emph{Chuẩn bị:} Dụng cụ: Giá để ống nghiệm, ống nghiệm, ống hút nhỏ giọt, mặt kính đồng hồ. Hóa chất: Dung dịch \emph{NaOH} loãng, giấy quỳ tím, dung dịch phenolphthalein. \emph{Tiến hành:} Đặt giấy quỳ tím lên mặt kính đồng hồ, lấy $\approx1$ \emph{mL} dung dịch \emph{NaOH} cho vào ống nghiệm. Nhỏ 1 giọt dung dịch \emph{NaOH} lên mẩu giấy quỳ tím, nhỏ 1 giọt dung dịch phenolphthalein vào ống nghiệm có dung dịch \emph{NaOH}. Mô tả các hiện tượng xảy ra.
\end{thinghiem}
Các dung dịch base khác cũng làm đổi màu quỳ tím \& phenolphthalein tương tự NaOH. \textit{Dung dịch base làm quỳ tím chuyển sang màu xanh, phenolphthalein không màu chuyển sang màu hồng}. Quỳ tím \& phenolphthalein được dùng làm chất chỉ thị màu để nhận biết dung dịch base.

\begin{baitoan}[\cite{SGK_KHTN_8_Canh_Dieu}, 2, p. 52]
	Có 2 dung dịch giấm ăn \& nước vôi trong. Nêu cách phân biệt 2 dung dịch trên bằng: (a) Quỳ tím. (b) Phenolphthalein.
\end{baitoan}

\subsubsection{Tác dụng với acid}

\begin{thinghiem}[\cite{SGK_KHTN_8_Canh_Dieu}, Thí nghiệm 2, p. 53]
	\emph{Chuẩn bị:} Dụng cụ: Giá để ống nghiệm, ống nghiệm, ống hút nhỏ giọt. Hóa chất: Dung dịch \emph{NaOH} loãng, dung dịch \emph{HCl} loãng, dung dịch phenolphthalein. \emph{Tiến hành:} Cho $\approx1$ \emph{mL} dung dịch \emph{NaOH} vào ống nghiệm, thêm tiếp 1 giọt dung dịch phenolphthalein \& lắc nhẹ. Nhỏ từ từ dung dịch \emph{HCl} loãng vào ống nghiệm đến khi dung dịch trong ống nghiệm mất màu thì dừng lại. Mô tả các hiện tượng xảy ra. Giải thích sự thay đổi màu của dung dịch trong ống nghiệm trong quá trình thí nghiệm.
\end{thinghiem}

\begin{proof}[Giải]
	Sodium hydroxide tác dụng với hydrochloric acid tạo ra sodium chloride \& nước theo PTHH: \ce{NaOH + HCl -> NaCl + H2O} (sodium hydroxide $\to$ sodium chloride).
\end{proof}

\begin{thinghiem}[\cite{SGK_KHTN_8_Canh_Dieu}, Thí nghiệm 3, p. 53]
	\emph{Chuẩn bị:} Dụng cụ: Giá để ống nghiệm, ống nghiệm, ống hút nhỏ giọt, thìa thủy tinh. Hóa chất: \emph{\ce{Mg(OH)2}} (được điều chế sẵn), dung dịch \emph{HCl}, nước cất. \emph{Tiến hành:} Lấy 1 lượng nhỏ \emph{\ce{Mg(OH)2}} cho vào ống nghiệm, thêm vào $\approx1$ \emph{mL} nước cất, lắc nhẹ. Tiếp tục nhỏ từ từ dung dịch \emph{HCl} vào ống nghiệm đến khi không nhìn thấy chất rắn trong ống nghiệm thì dừng lại. Mô tả các hiện tượng xảy ra. Giải thích sự thay đổi màu của dung dịch trong ống nghiệm trong quá trình thí nghiệm.
\end{thinghiem}

\begin{proof}[Giải]
	Magnesium hydroxide tác dụng với hydrochloric acid tạo ra magnesium chloride \& nước theo PTHH: \ce{Mg(OH)2 + $2$HCl -> MgCl2 + $2$H2O} (magnesium hydroxide $\to$ magnesium chloride).
\end{proof}
Các base khác, e.g., KOH, \ce{Cu(OH)2}, $\ldots$ cũng tác dụng với acid tạo ra muối \& nước. \textit{Base tác dụng với dung dịch acid tạo ra muối \& nước}.

\begin{baitoan}[\cite{SGK_KHTN_8_Canh_Dieu}, 3, p. 54]
	Viết PTHH xảy ra khi cho các base: \emph{KOH, \ce{Cu(OH)2, Mg(OH)2}} lần lượt tác dụng với: (a) Dung dịch acid \emph{HCl}. (b) Dung dịch acid \emph{\ce{H2SO4}}.
\end{baitoan}

\begin{baitoan}[\cite{SGK_KHTN_8_Canh_Dieu}, 4, p. 54]
	Hoàn thành PTHH: (a) \emph{KOH $\to$ \ce{K2SO4}}. (b) \emph{\ce{Mg(OH)2} $\to$ \ce{MgSO4}}. (c) \emph{\ce{Al(OH)3 + H2SO4}}.
\end{baitoan}

\begin{baitoan}[\cite{SGK_KHTN_8_Canh_Dieu}, p. 54]
	1 loại thuốc dành cho bệnh nhân đau dạ dày có chứa \emph{\ce{Al(OH)3,Mg(OH)2}}. Viết PTHH xảy ra giữa acid \emph{HCl} có trong dạ dày với các chất trên.
\end{baitoan}

\begin{vidu}[\cite{SGK_KHTN_8_Canh_Dieu}, p. 54, NaOH]
	Sodium hydroxide \emph{NaOH} là 1 trong những hóa chất được sử dụng phổ biến nhất trong phòng thí nghiệm \& trong công nghiệp. Phần lớn lượng sodium hydroxide sản xuất ra được sử dụng trong công nghiệp để sản xuất giấy, nhôm, chất tẩy rửa, các muối sodium, $\ldots$ Sodium hydroxide hút ẩm mạnh \& khi tiếp xúc với không khí sẽ phản ứng với khí carbon dioxide trong không khí tạo thành sodium carbonate. Vì vậy, cần phải chú ý trong việc bảo quản sodium hydroxide. Sodium hydroxide có thể ăn mòn da, làm rụng tóc, gây hại nghiêm trọng cho mắt \& hệ hô hấp. Vì vậy, cần thận trọng khi tiếp xúc với sodium hydroxide.
\end{vidu}
\noindent\fbox{%
	\parbox{\textwidth}{%
		\noindent\textsf{\textbf{Kiến thức cốt lõi.}} \fbox{\bf 1} \textit{Base} là những hợp chất trong phân tử có nguyên tử kim loại liên kết với nhóm hydroxide. Khi tan trong nước, base tạo ra ion \ce{OH-}. \fbox{\bf 2} Base tan trong nước được gọi là \textit{kiềm}. \fbox{\bf 3} \textit{Dung dịch base} làm quỳ tím chuyển sang màu xanh, phenolphthalein không màu chuyển sang màu hồng. \fbox{\bf 4} Base tác dụng với dung dịch acid tạo thành muối \& nước.
	}%
}

%------------------------------------------------------------------------------%

\section{Thang pH}

\noindent\fbox{%
	\parbox{\textwidth}{%
		\noindent\textsf{\textbf{Kiến thức cốt lõi.}} \fbox{\bf 1} Để biểu thị độ acid hoặc base của dung dịch, ta dùng giá trị pH. pH $= 7$: dung dịch có môi trường trung tính. pH $> 7$: dung dịch có môi trường base. pH $< 7$: dung dịch có môi trường acid. \fbox{\bf 2} pH của môi trường có ảnh hưởng mạnh đến đời sống của động vật \& thực vật. \fbox{\bf 3} Để xác định giá trị pH gần đúng của dung dịch, có thể dùng giấy chỉ thị màu.
	}%
}

%------------------------------------------------------------------------------%

\section{Oxide}
\textsf{\textbf{Nội dung.} Oxide là hợp chất của oxygen với 1 nguyên tố khác, PTHH tạo oxide từ kim loại\texttt{/}phi kim với oxygen, phân loại các oxide theo khả năng phản ứng với acid\texttt{/}base (oxide acid, oxide base, oxide lưỡng tính, oxide trung tính, thí nghiệm oxide kim loại phản ứng với acid, oxide phi kim phản ứng với base: nêu \& giải thích được hiện tượng xảy ra trong thí nghiệm (viết PTHH), tính chất hóa học của oxide.}

Thạch anh \ce{SiO2}, đá khô \ce{CO2}, hồng ngọc \ce{Al2O3} đều do các oxide tạo nên.

\subsection{Khái niệm oxide}
Kim loại hoặc phi kim khi tác dụng với oxygen tạo ra oxide.

\begin{vidu}[\cite{SGK_KHTN_8_Canh_Dieu}, p. 59]
	(a) \emph{\ce{$4$Al + $3$O2 -> $2$Al2O3}}: Aluminium $\to$ Alumininum oxide.\\(b) \emph{\ce{C + O2 -> CO2 ^}}: Carbon $\to$ Carbon dioxide.
\end{vidu}

\begin{dinhnghia}[Oxide]
	\emph{Oxide} là hợp chất của oxygen với 1 nguyên tố khác.
\end{dinhnghia}

\begin{vidu}[\cite{SGK_KHTN_8_Canh_Dieu}, p. 59]
	1 số oxide có nhiều trong tự nhiên như: Silicon dioxide \emph{\ce{SiO2}} -- thành phần chính của cát. Aluminium oxide \emph{\ce{Al2O3}} -- thành phần chính của quặng bauxite (boxit). Carbon dioxide \emph{\ce{CO2}} có trong không khí.
\end{vidu}

\begin{baitoan}[\cite{SGK_KHTN_8_Canh_Dieu}, 1, p. 59]
	Trong các chất \emph{\ce{Na2SO4, P2O5, CaCO3, SO2}}, chất nào là oxide?
\end{baitoan}

\begin{baitoan}[\cite{SGK_KHTN_8_Canh_Dieu}, 1, p. 59]
	Viết các PTHH xảy ra giữa oxygen \& các đơn chất để tạo ra các oxide sau: \emph{\ce{SO2,CuO,CO2,Na2O}}.
\end{baitoan}

\subsection{Phân loại oxide}
Dựa vào khả năng phản ứng với acid \& base, oxide được phân thành 4 loại như sau:
\begin{itemize}
	\item \textit{Oxide base} là những oxide tác dụng được với dung dịch acid tạo thành muối \& nước. Đa số các oxide kim loại là oxide base, e.g., CuO, CaO, MgO, $\ldots$
	\item \textit{Oxide acid} là những oxide tác dụng được với dung dịch base tạo thành muối \& nước. Các oxide acid thường là oxide của các phi kim, e.g., \ce{CO2,SO2,SO3,P2O5}, $\ldots$
	\item \textit{Oxide lưỡng tính} là những oxide tác dụng với dung dịch acid \& tác dụng với dung dịch base tạo thành muối \& nước. 1 số oxide lưỡng tính thường gặp, e.g., \ce{Al2O3,ZnO}, $\ldots$
	\item \textit{Oxide trung tính} là những oxide không tác dụng với dung dịch acid, dung dịch base. 1 số oxide trung tính, e.g., CO, NO, \ce{N2O}, $\ldots$
\end{itemize}

\begin{baitoan}[\cite{SGK_KHTN_8_Canh_Dieu}, 2, p. 60]
	Các oxide sau đây thuộc những loại oxide nào (oxide base, oxide acid, oxide lưỡng tính, oxide trung tính): \emph{\ce{Na2O,Al2O3,SO3,N2O}}.
\end{baitoan}

\subsection{Tính chất hóa học của oxide}

\subsubsection{Oxide base tác dụng với dung dịch acid}

\begin{thinghiem}
	\emph{Chuẩn bị:} Dụng cụ: Ống nghiệm, giá để ống nghiệm, thìa thủy tinh, ống hút nhỏ giọt. Hóa chất: \emph{CuO}, dung dịch \emph{HCl} loãng. \emph{Tiến hành:} Lấy 1 lượng nhỏ \emph{CuO} cho vào ống nghiệm, cho tiếp vào ống nghiệm $\approx1$\emph{--2 mL} dung dịch \emph{HCl}, lắc nhẹ. Mô tả các hiện tượng xảy ra. Dấu hiệu nào chứng tỏ có xảy ra phản ứng hóa học giữa \emph{CuO} \& dung dịch \emph{HCl}?
\end{thinghiem}

\begin{proof}[Giải]
	CuO đã phản ứng với dung dịch HCl tạo ra \ce{CuCl2} theo PTHH: \ce{CuO + 2HCl -> CuCl2 + H2O} (copper(II) oxide $\to$ copper(II) chloride). Dấu hiệu chứng tỏ có xảy ra phản ứng hóa học giữa \emph{CuO} \& dung dịch \emph{HCl} là dung dịch HCl không màu chuyển sang màu lục lam của dung dịch \ce{CuCl2}.
\end{proof}

\begin{luuy}[\ce{CuCl2}]
	Copper(II) chloride \emph{\ce{CuCl2}} là 1 chất rắn màu nâu, từ từ hấp thụ hơi nước để tạo thành hợp chất ngậm 2 nước màu lục lam. Copper(II) chloride là 1 trong những hợp chất copper(II) phổ biến nhất, chỉ sau hợp chất copper(II) sulfate \emph{\ce{CuSO4}}. Xem thêm \href{https://vi.wikipedia.org/wiki/%C4%90%E1%BB%93ng(II)_chloride}{Wikipedia\emph{\texttt{/}}Đồng(II) chloride}.
\end{luuy}

\begin{baitoan}[\cite{SGK_KHTN_8_Canh_Dieu}, 2, p. 60]
	Viết PTHH giữa các cặp chất sau: (a) \emph{\ce{H2SO4}, MgO}. (b) \emph{\ce{H2SO4}, CuO}. (c) \emph{HCl, \ce{Fe2O3}}.
\end{baitoan}
Nhiều oxide của các kim loại khác như MgO, CaO, \ce{Fe2O3}, $\ldots$ cũng tác dụng với dung dịch acid tạo ra muối \& nước tương tự như CuO. Oxide base tác dụng với dung dịch acid tạo ra muối \& nước:
\begin{align}
	\label{oxide base + acid}
	\boxed{\mbox{oxide base} + \mbox{acid}\to\mbox{salt} + \ce{H2O}.}
\end{align}
Cụ thể, với kim loại M hóa trị I \& acid \ce{H_xX} với gốc acid \ce{X^{x-}} có hóa trị $x\in\mathbb{N}^\star$, phương trình \eqref{oxide base + acid} trở thành:
\begin{align}
	\boxed{\ce{$x$M2O + $2$H_xX -> $2$M_xX + $x$H2O},\ \forall x\in\mathbb{N}^\star.}
\end{align}
Với kim loại M hóa trị II \& acid \ce{H_xX} với gốc acid \ce{X^{x-}} có hóa trị $x\in\mathbb{N}^\star$, phương trình \eqref{oxide base + acid} trở thành:
\begin{equation}
	\boxed{\left\{\begin{split}
			\ce{$x$MO + $2$H_xX &-> M_xX2 + $x$H2O},&&\forall x\in\mathbb{N}^\star,\,x\ne2,\\
			\ce{MO + H2X &-> MX + H2O}&&\mbox{ if } x = 2.
		\end{split}\right.}
\end{equation}
Tổng quát, với kim loại M hóa trị $m\in\mathbb{N}^\star$ \& acid \ce{H_xX} với gốc acid \ce{X^{x-}} có hóa trị $x\in\mathbb{N}^\star$, phương trình \eqref{oxide base + acid} trở thành:
\begin{equation}
	\boxed{\left\{\begin{split}
			\ce{$x$M2O_m + $2m$H_xX &-> $2$M_xX_m + $mx$H2O},&&\forall m,x\in\mathbb{N}^\star,\,m\ne2,\\
			\ce{$x$MO + $2$H_xX &-> M_xX2 + $x$H2O},&&\forall x\in\mathbb{N}^\star,\mbox{ if } m = 2.
		\end{split}\right.}
\end{equation}

\subsubsection{Oxide acid tác dụng với dung dịch base}

\begin{thinghiem}
	\emph{Chuẩn bị:} Dụng cụ: Bình tam giác (loại \emph{100 mL}), ống thủy tinh, ống nối cao su. Hóa chất: Dung dịch nước vôi trong, \emph{\ce{CO2}} (được điều chế từ bình tạo khí \emph{\ce{CO2}}). \emph{Tiến hành:} Cho vào bình tam giác $\approx$ \emph{30 mL} nước vôi trong, dẫn khí \emph{\ce{CO2}} từ từ vào dung dịch, khi dung dịch vẫn đục thì dừng lai. Mô tả các hiện tượng xảy ra. Giải thích.
\end{thinghiem}

\begin{proof}[Giải]
	\ce{CO2} đã phản ứng với dung dịch \ce{Ca(OH)2} tạo ra \ce{CaCO3} không tan theo PTHH: \ce{CO2 + Ca(OH)2 -> CaCO3 v + H2O} (calcium hydroxide $\to$ calcium carbonate).
\end{proof}
Nhiều oxide của phi kim (nonmetal), e.g., \ce{SO2,SO3,P2O5}, $\ldots$ cũng tác dụng với dung dịch base tạo thành muối \& nước tương tự \ce{CO2}. Oxide acid tác dụng được với dung dịch base tạo ra muối \& nước:
\begin{align}
	\label{oxide acid + base}
	\boxed{\mbox{oxide acid} + \mbox{base}\to\mbox{salt} + \ce{H2O}.}
\end{align}
Tổng quát,
\begin{align}
	\ce{$?$\overline{\rm M}2O_{\overline{m}} + $?$M(OH)_m -> $?$M_a(\overline{\rm M}_bO_c)_d + $?$H2O.}
\end{align}

\begin{baitoan}[\cite{SGK_KHTN_8_Canh_Dieu}, 3, p. 61]
	Viết các PTHH xảy ra khi cho dung dịch \emph{KOH} phản ứng với các chất sau: \emph{\ce{SO2,CO2,SO3}}.
\end{baitoan}

\begin{vidu}[\cite{SGK_KHTN_8_Canh_Dieu}, p. 61, Ứng dụng của \ce{SO2}]
	Sulfur dioxide \emph{\ce{SO2}} được sử dụng phần lớn để sản xuất \emph{\ce{H2SO4}}. Ngoài ra, \emph{\ce{SO2}} còn được dùng để tẩy trắng bột gỗ trong công nghiệp giấy, làm chất diệt nấm mốc, $\ldots$
	
	Trong sản xuất rượu vang, \emph{\ce{SO2}} được dùng làm chất chống oxi hóa, ức chế 1 số loại vi khuẩn, do đó có thể lưu trữ rượu được lâu hơn. Tuy nhiên, lượng \emph{\ce{SO2}} có trong rượu luôn được kiểm soát 1 cách nghiêm ngặt để không làm ảnh hưởng đến sức khỏe người sử dụng.
\end{vidu}

\noindent\fbox{%
	\parbox{\textwidth}{%
		\noindent\textsf{\textbf{Kiến thức cốt lõi.}} \fbox{\bf 1} \textit{Oxide} là hợp chất của oxygen với 1 nguyên tố khác. \fbox{\bf 2} Oxide được phân thành 4 loại: oxide base, oxide acid, oxide lưỡng tính, \& oxide trung tính. \fbox{\bf 3} Oxide base tác dụng với dung dịch acid tạo ra muối \& nước. \fbox{\bf 4} Oxide acid tác dụng với dung dịch base tạo ra muối \& nước.
	}%
}

%------------------------------------------------------------------------------%

\section{Salt -- Muối}

\noindent\fbox{%
	\parbox{\textwidth}{%
		\noindent\textsf{\textbf{Kiến thức cốt lõi.}} \fbox{\bf 1} \textit{Muối} là những hợp chất được tạo ra khi thay thế ion \ce{H+} trong acid bằng ion kim loại hoặc ion ammonium \ce{NH4+}. \fbox{\bf 2} Muối tác dụng với kim loại, dung dịch acid, dung dịch base, dung dịch muối. \fbox{\bf 3} Muối có thể được tạo ra bằng cách cho dung dịch acid tác dụng với: base, oxide base, muối hoặc cho 2 dung dịch muối tác dụng với nhau, $\ldots$. \fbox{\bf 4} Acid, base, \& oxide có các tính chất hóa học sau: Dung dịch acid: làm quỳ tím chuyển sang màu đỏ, tác dụng với kim loại, base, oxide base, muối. Dung dịch base: làm quỳ tím chuyển sang màu xanh, tác dụng với dung dịch acid, oxide acid \& với dung dịch muối. Oxide base tác dụng với dung dịch acid, oxide acid tác dụng với dung dịch base.
	}%
}

%------------------------------------------------------------------------------%

\section{Phân Bón Hóa Học}

\noindent\fbox{%
	\parbox{\textwidth}{%
		\noindent\textsf{\textbf{Kiến thức cốt lõi.}} \fbox{\bf 1} \textit{Phân bón hóa học} là những hóa chất có chứa các nguyên tố dinh dưỡng dùng để bón cho cây trồng \& được chia thành 3 loại: đa lượng, trung lượng, \& vi lượng. \fbox{\bf 2} \textit{Phân đa lượng} gồm: phân đạm cung cấp nguyên tố nitrogen, phân lân cung cấp nguyên tố phosphorus, phân kali cung cấp nguyên tố potassium, phân hỗn hợp cung cấp cho cây 2 hoặc 3 nguyên tố trên. \fbox{\bf 3} Để phát huy tối đa hiệu quả của phân bón, tránh gây tác hại đến môi trường cần phải sử dụng phân bón hóa học đúng loại, đúng lúc, đúng liều lượng, \& đúng cách.
	}%
}

%------------------------------------------------------------------------------%

\printbibliography[heading=bibintoc]
	
\end{document}