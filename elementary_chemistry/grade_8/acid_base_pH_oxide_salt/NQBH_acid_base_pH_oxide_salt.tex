\documentclass{article}
\usepackage[backend=biber,natbib=true,style=alphabetic,maxbibnames=50]{biblatex}
\addbibresource{/home/nqbh/reference/bib.bib}
\usepackage[utf8]{vietnam}
\usepackage{tocloft}
\renewcommand{\cftsecleader}{\cftdotfill{\cftdotsep}}
\usepackage[colorlinks=true,linkcolor=blue,urlcolor=red,citecolor=magenta]{hyperref}
\usepackage{amsmath,amssymb,amsthm,float,graphicx,mathtools,tikz,tipa}
\usepackage[version=4]{mhchem}
\allowdisplaybreaks
\newtheorem{assumption}{Assumption}
\newtheorem{baitoan}{Bài toán}
\newtheorem{cauhoi}{Câu hỏi}
\newtheorem{conjecture}{Conjecture}
\newtheorem{corollary}{Corollary}
\newtheorem{dangtoan}{Dạng toán}
\newtheorem{definition}{Definition}
\newtheorem{dinhly}{Định lý}
\newtheorem{dinhnghia}{Định nghĩa}
\newtheorem{example}{Example}
\newtheorem{ghichu}{Ghi chú}
\newtheorem{hequa}{Hệ quả}
\newtheorem{hypothesis}{Hypothesis}
\newtheorem{lemma}{Lemma}
\newtheorem{luuy}{Lưu ý}
\newtheorem{nhanxet}{Nhận xét}
\newtheorem{notation}{Notation}
\newtheorem{note}{Note}
\newtheorem{principle}{Principle}
\newtheorem{problem}{Problem}
\newtheorem{proposition}{Proposition}
\newtheorem{question}{Question}
\newtheorem{remark}{Remark}
\newtheorem{theorem}{Theorem}
\newtheorem{thinghiem}{Thí nghiệm}
\newtheorem{vidu}{Ví dụ}
\usepackage[left=1cm,right=1cm,top=5mm,bottom=5mm,footskip=4mm]{geometry}

\title{Acid, Base, pH, Oxide, Salt -- Muối}
\author{Nguyễn Quản Bá Hồng\footnote{Independent Researcher, Ben Tre City, Vietnam\\e-mail: \texttt{nguyenquanbahong@gmail.com}; website: \url{https://nqbh.github.io}.}}
\date{\today}

\begin{document}
\maketitle
\begin{abstract}
	\textsc{[en]} This text is a collection of problems, from easy to advanced, about \textit{acid base pH oxide salt}. This text is also a supplementary material for my lecture note on Elementary Chemistry, which is stored \& downloadable at the following link: \href{https://github.com/NQBH/hobby/blob/master/elementary_chemistry/grade_8/NQBH_elementary_chemistry_grade_8.pdf}{GitHub\texttt{/}NQBH\texttt{/}hobby\texttt{/}elementary chemistry\texttt{/}grade 8\texttt{/}lecture}\footnote{\textsc{url}: \url{https://github.com/NQBH/hobby/blob/master/elementary_chemistry/grade_8/NQBH_elementary_chemistry_grade_8.pdf}.}. The latest version of this text has been stored \& downloadable at the following link: \href{https://github.com/NQBH/hobby/blob/master/elementary_chemistry/acid_base_pH_oxide_salt/NQBH_acid_base_pH_oxide_salt.pdf}{GitHub\texttt{/}NQBH\texttt{/}hobby\texttt{/}elementary chemistry\texttt{/}grade 8\texttt{/}acid base pH oxide salt}\footnote{\textsc{url}: \url{https://github.com/NQBH/hobby/blob/master/elementary_chemistry/acid_base_pH_oxide_salt/NQBH_acid_base_pH_oxide_salt.pdf}.}.
	\vspace{2mm}
	
	\textsc{[vi]} Tài liệu này là 1 bộ sưu tập các bài tập chọn lọc từ cơ bản đến nâng cao về \textit{phản ứng hóa học}. Tài liệu này là phần bài tập bổ sung cho tài liệu chính -- bài giảng \href{https://github.com/NQBH/hobby/blob/master/elementary_chemistry/grade_8/NQBH_elementary_chemistry_grade_8.pdf}{GitHub\texttt{/}NQBH\texttt{/}hobby\texttt{/}elementary chemistry\texttt{/}grade 8\texttt{/}lecture} của tác giả viết cho Hóa Học Sơ Cấp. Phiên bản mới nhất của tài liệu này được lưu trữ \& có thể tải xuống ở link sau: \href{https://github.com/NQBH/hobby/blob/master/elementary_chemistry/grade_8/real/NQBH_real.pdf}{GitHub\texttt{/}NQBH\texttt{/}hobby\texttt{/}elementary chemistry\texttt{/}grade 8\texttt{/}acid base pH oxide salt}.
\end{abstract}
\setcounter{secnumdepth}{4}
\setcounter{tocdepth}{3}
\tableofcontents
\newpage

%------------------------------------------------------------------------------%

\section{Acid}
\textsf{\textbf{Nội dung.} Khái niệm acid (tạo ra ion \ce{H+}, thí nghiệm của hydrochloric acid (làm đổi màu chất chỉ thị, phản ứng với kim loại), giải thích hiện tượng xảy ra trong thí nghiệm (viết PTHH) \& nhận xét về tính chất của acid, 1 số ứng dụng của 1 số acid thông dụng.}

Các quả có vị chua, e.g., quả sấu, quả me, quả chanh, quả cam, $\ldots$ chứa 1 số loại acid.

\subsection{Khái niệm acid}

\begin{vidu}[\cite{SGK_KHTN_8_Canh_Dieu}, p. 47]
	Giấm ăn hoặc chanh thường được cho vào nước chấm để tạo ra vị chua; sấu, me, hoặc cà chua cũng tạo ra vị chua cho 1 số món ăn.
\end{vidu}
Vị chua của giấm ăn \& các loại quả ở trên được tạo ra bởi 1 loại hợp chất gọi là acid. Khi tan trong nước, acid tạo ra ion \ce{H+} làm cho dung dịch có vị chua.

\begin{dinhnghia}[Acid]
	\emph{Acid} là các hợp chất trong phân tử có nguyên tử hydrogen liên kết với gốc acid. Khi tan trong nước, acid tạo ra ion \emph{\ce{H+}}.
\end{dinhnghia}
Acid tạo ra ion \ce{H+} theo sơ đồ sau:
\begin{align}
	\boxed{\mbox{acid} \to\mbox{ion \ce{H+}} + \mbox{ion âm gốc acid}.}
\end{align}

\begin{vidu}[\cite{SGK_KHTN_8_Canh_Dieu}, p. 47]
	(a) Hydrochloric acid $\to$ Ion hydrogen $+$ Ion chloride: \emph{\ce{HCl -> H+ + Cl-}}. (b) Sulfuric acid $\to$ Ion hydrogen $+$ Ion sulfate: \emph{\ce{H2SO4 -> 2H+ + SO4^2-}}.
\end{vidu}

\subsection{Tính chất hóa học của acid}

\subsubsection{Làm đổi màu chất chỉ thị}

\begin{thinghiem}[\cite{SGK_KHTN_8_Canh_Dieu}, Thí nghiệm 1, p. 48]
	\emph{Chuẩn bị:} Dụng cụ: Mặt kính đồng hồ, ống hút nhỏ giọt. Hóa chất: Dung dịch \emph{HCl} loãng, giấy quỳ tím. \emph{Tiến hành:} Đặt mẩu giấy quỳ tím lên mặt kính đồng hồ, lấy dung dịch \emph{HCl} loãng \& nhỏ 1 giọt lên mẩu giấy quỳ tím. Mô tả các hiện tượng xảy ra.
\end{thinghiem}
Các dung dịch sulfuric acid loãng, acetic acid, $\ldots$ cũng làm giấy quỳ tím chuyển màu tương tự như với dung dịch hydrochloric acid. \textit{Dung dịch acid làm quỳ tím chuyển sang màu đỏ}. Quỳ tím được dùng làm chất chỉ thị màu để nhận ra dung dịch acid. 

\subsubsection{Tác dụng với kim loại}

\begin{thinghiem}[\cite{SGK_KHTN_8_Canh_Dieu}, Thí nghiệm 2, p. 48]
	\emph{Chuẩn bị:} Dụng cụ: Giá để ống nghiệm, ống nghiệm, ống hút nhỏ giọt. Hóa chất: Dung dịch \emph{HCl} loãng, \emph{Zn} viên. \emph{Tiến hành:} Cho 1 viên \emph{Zn} vào ống nghiệm, sau đó cho thêm vào ống nghiệm $\approx 2$ \emph{mL} dung dịch \emph{HCl} loãng. Mô tả các hiện tượng xảy ra. Những dấu hiệu nào chứng tỏ có các phản ứng hóa học giữa dung dịch \emph{HCl} \& \emph{Zn}?
\end{thinghiem}

\begin{proof}[Giải]
	Dung dịch HCl đã phản ứng với Zn tạo ra chất khí. PTHH của phản ứng trên như sau: zinc $+$ hydrochloric acid $\to$ zinc chloride $+$ hydrogen: \ce{Zn + $2$HCl -> ZnCl2 + H2 ^}.
\end{proof}
Dung dịch các acid khác như sulfuric acid loãng, acetic acid, $\ldots$ cũng có phản ứng hóa học với nhiều kim loại tạo ra muối \& khí hydrogen. \textit{Dung dịch acid tác dụng được với nhiều kim loại tạo ra muối \& khí hydrogen}.\footnote{Riêng \ce{HNO3,H2SO4} \textit{đặc} tác dụng với kim loại sẽ được học sau.}
\begin{align}
	\label{acid + metal}
	\boxed{\mbox{acid} + \mbox{metal}\to\mbox{salt} + \mbox{hydrogen}.}
\end{align}
Cụ thể, với kim loại M hóa trị I \& acid \ce{H_xX} với gốc acid \ce{X^{x-}} có hóa trị $x\in\mathbb{N}^\star$, phương trình \eqref{acid + metal} trở thành:
\begin{align}
	\boxed{\ce{$x$M + H_xX -> M_xX + $\frac{x}{2}$H2 ^},\ \forall x\in\mathbb{N}^\star.}
\end{align}
Với kim loại M hóa trị II \& acid \ce{H_xX} với gốc acid \ce{X^{x-}} có hóa trị $x\in\mathbb{N}^\star$, phương trình \eqref{acid + metal} trở thành:
\begin{equation}
	\boxed{\left\{\begin{split}
			\ce{$x$M + $2$H_xX &-> M_xX2 + $x$H2 ^},&&\forall x\in\mathbb{N}^\star,\,x\ne2,\\
			\ce{M + H2X &-> MX + H2}&&\mbox{if } x = 2\ (\mbox{II}).
		\end{split}\right.}
\end{equation}
Tổng quát, với kim loại M hóa trị $m\in\mathbb{N}^\star$ \& acid \ce{H_xX} với gốc acid \ce{X^{x-}} có hóa trị $x\in\mathbb{N}^\star$, phương trình \eqref{acid + metal} trở thành:
\begin{equation}
	\boxed{\left\{\begin{split}
			\ce{$x$M + $m$H_xX &-> M_xX_m + $\frac{mx}{2}$H2 ^},&&\forall m,x\in\mathbb{N}^\star,\,m\ne x,\\
			\ce{M + H_xX &-> MX + $\frac{x}{2}$H2 ^},&&\forall x\in\mathbb{N}^\star,\mbox{ if } m = x.
		\end{split}\right.}
\end{equation}

\subsection{Ứng dụng của 1 số acid}

\subsubsection{Hydrochloric acid HCl}
\textit{Hydrochloric acid} có trong dạ dày của người \& động vật giúp tiêu hóa thức ăn. Hydrochloric acid được sử dụng nhiều trong công nghiệp. 1 số ứng dụng quan trọng của hydrochloric acid: tẩy rửa kim loại, sản xuất chất dẻo, điều chiếu glucose \ce{C6H12O6}.

\subsubsection{Sulfuric acid \ce{H2SO4}}
\textit{Sulfuric acid} là 1 hóa chất quan trọng được sử dụng nhiều trong công nghiệp. 1 số ứng dụng quan trọng của sulfuric acid: sản xuất giấy, tơ sợi, sản xuất ắc quy, sản xuất chất dẻo, sản xuất phân bón, sản xuất sơn.

\subsubsection{Acetic acid \ce{CH3COOH}}
\textit{Acetic acid} là 1 acid hữu cơ có trong giấm ăn với nồng độ $\approx4$\%. 1 số ứng dụng của acetic acid: sản xuất tơ nhân tạo, sản xuất thuốc diệt côn trùng, sản xuất phẩm nhuộm, sản xuất dược phẩm, sản xuất chất dẻo.

\noindent\fbox{%
	\parbox{\textwidth}{%
		\noindent\textsf{\textbf{Kiến thức cốt lõi.}} \fbox{\bf 1} \textit{Acid} là các hợp chất trong phân tử có nguyên tử hydrogen liên kết với góc acid. Khi tan trong nước, acid tạo ra ion \ce{H+}. \fbox{\bf 2} \textit{Dung dịch acid} có vị chua, làm quỳ tím chuyển sang màu đỏ, tác dụng với nhiều kim loại tạo ra khí hydrogen. \fbox{\bf 3} Hydrochloric acid, sulfuric acid, \& acetic acid là các acid có nhiều ứng dụng trong đời sống \& trong công nghiệp. 
	}%
}

%------------------------------------------------------------------------------%

\section{Base}
\textsf{\textbf{Nội dung.} Khái niệm base (tạo ra ion \ce{OH-}, kiềm là các hydroxide tan tốt trong nước, thí nghiệm base làm đổi màu chất chỉ thị, phản ứng với acid tạo muối, giải thích hiện tượng xảy ra trong thí nghiệm (viết PTHH) \& nhận xét tính chất của base, tra bảng tính tan để biết 1 hydroxide cụ thể thuộc loại kiềm hoặc base không tan.}

\subsection{Khái niệm base}

\begin{dinhnghia}[Base]
	\emph{Base} là các hợp chất trong phân tử có nguyên tử kim loại liên kết với nhóm hydroxide. Khi tan trong nước, base tạo ra ion \emph{\ce{OH-}}.
\end{dinhnghia}

\begin{vidu}
	(a) Sodium hydroxide $+$ ion sodium $\to$ ion hydroxide: \emph{\ce{NaOH -> Na+ + OH-}}. (b) Calcium hydroxide $\to$ ion calcium $+$ ion hydroxide: \emph{\ce{Ca(OH)2 -> Ca^2+ + 2OH-}}.
\end{vidu}
Tên gọi \& CTHH của 1 số base thông dụng: KOH: potassium hydroxide, \ce{Mg(OH)2}: magnesium hydroxide, \ce{Cu(OH)2}: copper(II) hydroxide.

\subsection{Phân loại base}
Base được chia thành 2 loại chính: base tan \& base không tan trong nước. \textit{Base tan trong nước} còn được gọi là \textit{kiềm}, e.g., NaOH, KOH, \ce{Ba(OH)2}, $\ldots$ Tính tan của các base trong nước được trình bày trong bảng tính tan.

\subsection{Tính chất hóa học}

\subsubsection{Làm đổi màu chất chỉ thị}

\begin{thinghiem}[\cite{SGK_KHTN_8_Canh_Dieu}, Thí nghiệm 1, p. 52]
	\emph{Chuẩn bị:} Dụng cụ: Giá để ống nghiệm, ống nghiệm, ống hút nhỏ giọt, mặt kính đồng hồ. Hóa chất: Dung dịch \emph{NaOH} loãng, giấy quỳ tím, dung dịch phenolphthalein. \emph{Tiến hành:} Đặt giấy quỳ tím lên mặt kính đồng hồ, lấy $\approx1$ \emph{mL} dung dịch \emph{NaOH} cho vào ống nghiệm. Nhỏ 1 giọt dung dịch \emph{NaOH} lên mẩu giấy quỳ tím, nhỏ 1 giọt dung dịch phenolphthalein vào ống nghiệm có dung dịch \emph{NaOH}. Mô tả các hiện tượng xảy ra.
\end{thinghiem}
Các dung dịch base khác cũng làm đổi màu quỳ tím \& phenolphthalein tương tự NaOH. \textit{Dung dịch base làm quỳ tím chuyển sang màu xanh, phenolphthalein không màu chuyển sang màu hồng}. Quỳ tím \& phenolphthalein được dùng làm chất chỉ thị màu để nhận biết dung dịch base.

\subsubsection{Tác dụng với acid}

\begin{thinghiem}[\cite{SGK_KHTN_8_Canh_Dieu}, Thí nghiệm 2, p. 53]
	\emph{Chuẩn bị:} Dụng cụ: Giá để ống nghiệm, ống nghiệm, ống hút nhỏ giọt. Hóa chất: Dung dịch \emph{NaOH} loãng, dung dịch \emph{HCl} loãng, dung dịch phenolphthalein. \emph{Tiến hành:} Cho $\approx1$ \emph{mL} dung dịch \emph{NaOH} vào ống nghiệm, thêm tiếp 1 giọt dung dịch phenolphthalein \& lắc nhẹ. Nhỏ từ từ dung dịch \emph{HCl} loãng vào ống nghiệm đến khi dung dịch trong ống nghiệm mất màu thì dừng lại. Mô tả các hiện tượng xảy ra. Giải thích sự thay đổi màu của dung dịch trong ống nghiệm trong quá trình thí nghiệm.
\end{thinghiem}

\begin{proof}[Giải]
	Sodium hydroxide tác dụng với hydrochloric acid tạo ra sodium chloride \& nước theo PTHH: \ce{NaOH + HCl -> NaCl + H2O} (sodium hydroxide $\to$ sodium chloride).
\end{proof}

\begin{thinghiem}[\cite{SGK_KHTN_8_Canh_Dieu}, Thí nghiệm 3, p. 53]
	\emph{Chuẩn bị:} Dụng cụ: Giá để ống nghiệm, ống nghiệm, ống hút nhỏ giọt, thìa thủy tinh. Hóa chất: \emph{\ce{Mg(OH)2}} (được điều chế sẵn), dung dịch \emph{HCl}, nước cất. \emph{Tiến hành:} Lấy 1 lượng nhỏ \emph{\ce{Mg(OH)2}} cho vào ống nghiệm, thêm vào $\approx1$ \emph{mL} nước cất, lắc nhẹ. Tiếp tục nhỏ từ từ dung dịch \emph{HCl} vào ống nghiệm đến khi không nhìn thấy chất rắn trong ống nghiệm thì dừng lại. Mô tả các hiện tượng xảy ra. Giải thích sự thay đổi màu của dung dịch trong ống nghiệm trong quá trình thí nghiệm.
\end{thinghiem}

\begin{proof}[Giải]
	Magnesium hydroxide tác dụng với hydrochloric acid tạo ra magnesium chloride \& nước theo PTHH: \ce{Mg(OH)2 + $2$HCl -> MgCl2 + $2$H2O} (magnesium hydroxide $\to$ magnesium chloride).
\end{proof}
Các base khác, e.g., KOH, \ce{Cu(OH)2}, $\ldots$ cũng tác dụng với acid tạo ra muối \& nước. \textit{Base tác dụng với dung dịch acid tạo ra muối \& nước}.

\begin{vidu}[\cite{SGK_KHTN_8_Canh_Dieu}, p. 54, NaOH]
	Sodium hydroxide \emph{NaOH} là 1 trong các hóa chất được sử dụng phổ biến nhất trong phòng thí nghiệm \& trong công nghiệp. Phần lớn lượng sodium hydroxide sản xuất ra được sử dụng trong công nghiệp để sản xuất giấy, nhôm, chất tẩy rửa, các muối sodium, $\ldots$ Sodium hydroxide hút ẩm mạnh \& khi tiếp xúc với không khí sẽ phản ứng với khí carbon dioxide trong không khí tạo thành sodium carbonate. Vì vậy, cần phải chú ý trong việc bảo quản sodium hydroxide. Sodium hydroxide có thể ăn mòn da, làm rụng tóc, gây hại nghiêm trọng cho mắt \& hệ hô hấp. Vì vậy, cần thận trọng khi tiếp xúc với sodium hydroxide.
\end{vidu}
\noindent\fbox{%
	\parbox{\textwidth}{%
		\noindent\textsf{\textbf{Kiến thức cốt lõi.}} \fbox{\bf 1} \textit{Base} là các hợp chất trong phân tử có nguyên tử kim loại liên kết với nhóm hydroxide. Khi tan trong nước, base tạo ra ion \ce{OH-}. \fbox{\bf 2} Base tan trong nước được gọi là \textit{kiềm}. \fbox{\bf 3} \textit{Dung dịch base} làm quỳ tím chuyển sang màu xanh, phenolphthalein không màu chuyển sang màu hồng. \fbox{\bf 4} Base tác dụng với dung dịch acid tạo thành muối \& nước.
	}%
}

%------------------------------------------------------------------------------%

\section{Thang pH}
\textsf{\textbf{Nội dung.} Thang pH, sử dụng pH để đánh giá độ acid--base của dung dịch, 1 số thí nghiệm đo pH (bằng giấy chỉ thị) 1 số loại thực phẩm (đồ uống, hoa quả, $\ldots$), liên hệ pH trong dạ dày, trong máu, trong nước mưa, đất.}

pH là 1 trong các tiêu chí quan trọng để xác định chất lượng của nước sinh hoạt, lựa chọn đất cho cây trồng. Khi kiểm tra sức khỏe, người ta cũng thường xem xét đến pH của máu \& nước tiểu.

\begin{vidu}[\cite{SGK_KHTN_8_Canh_Dieu}, p. 55]
	(a) Nước sinh hoạt có pH $\approx6$--$8.5$. (b) Cây chè thích hợp với đất có pH $\approx5$--$6$.
\end{vidu}

\subsection{Thang pH}

\begin{vidu}[\cite{SGK_KHTN_8_Canh_Dieu}, p. 55]
	Nước ép từ các loại quả chanh, bưởi, \& cam đều có vị chua, song độ chua của chúng khác nhau. Người ta nói các loại nước ép trên có độ acid khác nhau hay có pH khác nhau.
\end{vidu}

\begin{vidu}[\cite{SGK_KHTN_8_Canh_Dieu}, p. 55]
	Khi nhúng giấy quỳ tím vào nước xà phòng hoặc nước vôi trong sẽ thấy giấy quỳ có màu xanh đậm, nhạt khác nhau. Người ta nói các dung dịch trên có độ base khác nhau hay pH khác nhau.
\end{vidu}
Thang pH được dùng để biểu thị độ acid, base của dung dịch. Thang pH thường dùng có các giá trị từ 1--14.
\begin{figure}[H]
	\centering
	\includegraphics[scale=0.25]{thang_pH}
	\caption{Thang pH.}
\end{figure}
\begin{itemize}
	\item Nếu pH $= 7$ thì dung dịch có \textit{môi trường trung tính} (không có tính acid \& không có tính base). Nước tinh khiết (nước cất) có pH $= 7$.
	\item Nếu pH $> 7$ thì dung dịch có môi trường base, pH càng lớn thì độ base của dung dịch càng lớn.
	\item Nếu pH $< 7$ thì dung dịch có môi trường acid, pH càng nhỏ thì độ acid của dung dịch càng lớn.
\end{itemize}
Như vậy, khi biết giá trị pH của dung dịch dựa vào thang pH, ta không chỉ biết được dung dịch đó có tính acid, base, hay trung tính mà còn biết được mức độ acid hoặc mức độ base của dung dịch.

Khi sử dụng giấy chỉ thị màu để xác định pH của dung dịch cần phải đối chiếu với thang màu pH tương ứng.

\subsection{Ý nghĩa của pH}
pH có ý nghĩa to lớn trong thực tiễn.

\begin{vidu}[\cite{SGK_KHTN_8_Canh_Dieu}, p. 56]
	(a) Tôm, cá sống ở môi trường nước có pH $\approx7$--$8.5$ \& rất nhạy cảm với sự thay đổi pH của môi trường. (b) Trong cơ thể người, pH của máu luôn được duy trì ổn định trong phạm vi rất hẹp $\approx7.35$--$7.45$. (c) Thực vật chỉ phát triển được bình thường khi giá trị pH của dung dịch trong đất ở trong khoảng xác định, đặc trưng cho mỗi loại cây.
\end{vidu}
pH của môi trường có ảnh hưởng nhiều đến đời sống của động vật \& thực vật, do vậy cần phải quan tâm đến pH của môi trường nước, môi trường đất để có các biện pháp can thiệp kịp thời nhằm duy trì được pH tối ưu với đời sống của người, động vật, \& thực vật.

Ở 1 số khu vực, không khí bị ô nhiễm bởi các chất khí như \ce{SO2,NO2}, $\ldots$ sinh ra trong sản xuất công nghiệp \& đốt cháy nhiên liệu. Các khí này có thể hòa tan vào nước mưa \& làm pH của nước mưa giảm đi. Khi pH của nước mưa nhỏ hơn $5.6$ gọi là \textit{hiện tượng mưa acid}. Mưa acid có thể làm thay đổi pH của môi trường nước trong tự nhiên \& ảnh hưởng nghiêm trọng đến sự phát triển của động, thực vật.

\subsection{Xác định pH dung dịch bằng giấy chỉ thị màu}

\begin{thinghiem}[\cite{SGK_KHTN_8_Canh_Dieu}, p. 57, Xác định pH của các dung dịch giấm ăn, nước xà phòng, nước vôi trong]
	\emph{Chuẩn bị:} Dụng cụ: Mặt kính đồng hồ, ống hút nhỏ giọt. Hóa chất: Giấy chỉ thị màu, các dung dịch giấm ăn, nước xà phòng, nước vôi trong. \emph{Tiến hành:} Đặt giấy chỉ thị lên mặt kính đồng hồ, nhỏ 1 giọt dung dịch giấm ăn lên giấy. So màu của giấy chỉ thị sau khi nhỏ giấm ăn với thang màu pH tương ứng \& ghi lại giá trị pH. Làm tương tự đối với dung dịch nước xà phòng \& nước vôi trong. Kết quả xác định pH cho biết điều gì? Báo cáo kết quả xác định pH của các dung dịch.
\end{thinghiem}
Dùng giấy chỉ thị màu để xác định pH của dung dịch sẽ cho kết quả với độ chính xác không cao. Khi cần xác định pH của dung dịch với độ chính xác cao, người ta dùng các thiết bị đo pH như máy đo pH để bàn, máy đo pH cầm tay, bút đo pH.

\noindent\fbox{%
	\parbox{\textwidth}{%
		\noindent\textsf{\textbf{Kiến thức cốt lõi.}} \fbox{\bf 1} Để biểu thị độ acid hoặc base của dung dịch, ta dùng giá trị pH. pH $= 7$: dung dịch có môi trường trung tính. pH $> 7$: dung dịch có môi trường base. pH $< 7$: dung dịch có môi trường acid. \fbox{\bf 2} pH của môi trường có ảnh hưởng mạnh đến đời sống của động vật \& thực vật. \fbox{\bf 3} Để xác định giá trị pH gần đúng của dung dịch, có thể dùng giấy chỉ thị màu.
	}%
}

%------------------------------------------------------------------------------%

\section{Oxide}
\textsf{\textbf{Nội dung.} Oxide là hợp chất của oxygen với 1 nguyên tố khác, PTHH tạo oxide từ kim loại\texttt{/}phi kim với oxygen, phân loại các oxide theo khả năng phản ứng với acid\texttt{/}base (oxide acid, oxide base, oxide lưỡng tính, oxide trung tính, thí nghiệm oxide kim loại phản ứng với acid, oxide phi kim phản ứng với base: nêu \& giải thích được hiện tượng xảy ra trong thí nghiệm (viết PTHH), tính chất hóa học của oxide.}

\begin{vidu}
	Thạch anh \emph{\ce{SiO2}}, đá khô \emph{\ce{CO2}}, hồng ngọc \emph{\ce{Al2O3}} đều do các oxide tạo nên.
\end{vidu}

\subsection{Khái niệm oxide}
Kim loại hoặc phi kim khi tác dụng với oxygen tạo ra \emph{oxide}.

\begin{vidu}[\cite{SGK_KHTN_8_Canh_Dieu}, p. 59]
	(a) \emph{\ce{$4$Al + $3$O2 -> $2$Al2O3}}: Aluminium $\to$ Alumininum oxide. (b) \emph{\ce{C + O2 -> CO2 ^}}: Carbon $\to$ Carbon dioxide.
\end{vidu}

\begin{dinhnghia}[Oxide]
	\emph{Oxide} là hợp chất của oxygen với 1 nguyên tố khác.
\end{dinhnghia}

\begin{vidu}[\cite{SGK_KHTN_8_Canh_Dieu}, p. 59]
	1 số oxide có nhiều trong tự nhiên như: Silicon dioxide \emph{\ce{SiO2}} -- thành phần chính của cát. Aluminium oxide \emph{\ce{Al2O3}} -- thành phần chính của quặng bauxite (boxit). Carbon dioxide \emph{\ce{CO2}} có trong không khí.
\end{vidu}

\subsection{Phân loại oxide}
Dựa vào khả năng phản ứng với acid \& base, oxide được phân thành 4 loại như sau:
\begin{itemize}
	\item \textit{Oxide base} là các oxide tác dụng được với dung dịch acid tạo thành muối \& nước. Đa số các oxide kim loại là oxide base, e.g., CuO, CaO, MgO, $\ldots$
	\item \textit{Oxide acid} là các oxide tác dụng được với dung dịch base (kiềm) tạo thành muối \& nước. Các oxide acid thường là oxide của các phi kim, e.g., \ce{CO2,SO2,SO3,P2O5}, $\ldots$
	\item \textit{Oxide lưỡng tính} là các oxide tác dụng với dung dịch acid \& tác dụng với dung dịch base tạo thành muối \& nước. 1 số oxide lưỡng tính thường gặp, e.g., \ce{Al2O3,ZnO}, $\ldots$
	\item \textit{Oxide trung tính} (còn được gọi là \textit{oxide không tạo muối}) là các oxide không tác dụng với dung dịch acid, dung dịch base. 1 số oxide trung tính, e.g., CO, NO, \ce{N2O}, $\ldots$
\end{itemize}

\subsection{Tính chất hóa học của oxide}

\subsubsection{Oxide base tác dụng với nước}

\begin{vidu}[\cite{SGK_Hoa_Hoc_9}, p. 4]
	\emph{BaO} phản ứng với nước tạo thành dung dịch \emph{barium hydroxide \ce{Ba(OH)2}}, thuộc loại base: \emph{\ce{BaO (r) + H2O (l) -> Ba(OH)2 (dd)}}.
\end{vidu}
1 số oxide base tác dụng với nước tạo thành dung dịch base (kiềm).
\begin{align}
	\label{oxide base + H2O}
	\boxed{\mbox{oxide base} + \ce{H2O ->} \mbox{base}.}
\end{align}
Cụ thể, với kim loại M hóa trị I, \eqref{oxide base + H2O} trở thành:
\begin{align}
	\boxed{\ce{M2O + H2O -> $2$MOH}.}
\end{align}
Với kim loại M hóa trị II, \eqref{oxide base + H2O} trở thành:
\begin{align*}
	\boxed{\ce{MO + H2O -> M(OH)2}.}
\end{align*}
Tổng quát, với kim loại M hóa trị $m\in\mathbb{N}^\star$, \eqref{oxide base + H2O} trở thành:
\begin{equation*}
	\boxed{\left\{\begin{split}
		\ce{M2O_m + $m$H2O &-> $2$M(OH)m},&&\forall m\in\mathbb{N}^\star,\,m\ne2,\\
		\ce{MO + H2O &-> M(OH)2},&&\mbox{if } m = 2.
	\end{split}\right.}
\end{equation*}

\subsubsection{Oxide base tác dụng với dung dịch acid}

\begin{thinghiem}[\cite{SGK_KHTN_8_Canh_Dieu}, Thí nghiệm 1, p. 60]
	\emph{Chuẩn bị:} Dụng cụ: Ống nghiệm, giá để ống nghiệm, thìa thủy tinh, ống hút nhỏ giọt. Hóa chất: \emph{CuO}, dung dịch \emph{HCl} loãng. \emph{Tiến hành:} Lấy 1 lượng nhỏ \emph{CuO} cho vào ống nghiệm, cho tiếp vào ống nghiệm $\approx1$\emph{--2 mL} dung dịch \emph{HCl}, lắc nhẹ. Mô tả các hiện tượng xảy ra. Dấu hiệu nào chứng tỏ có xảy ra phản ứng hóa học giữa \emph{CuO} \& dung dịch \emph{HCl}?
\end{thinghiem}

\begin{proof}[Giải]
	\textit{Hiện tượng}: Bột CuO màu đen bị hòa tan, tạo thành dung dịch màu xanh lam. \textit{Nhận xét}: Màu xanh làm là màu của dung dịch đồng(II) clorua. CuO đã phản ứng với dung dịch HCl tạo ra \ce{CuCl2} theo PTHH: \ce{CuO + 2HCl -> CuCl2 + H2O} (copper(II) oxide $\to$ copper(II) chloride). Dấu hiệu chứng tỏ có xảy ra phản ứng hóa học giữa \emph{CuO} \& dung dịch \emph{HCl} là dung dịch HCl không màu chuyển sang màu lục lam của dung dịch \ce{CuCl2}.
\end{proof}

\begin{luuy}[\ce{CuCl2}]
	Copper(II) chloride \emph{\ce{CuCl2}} là 1 chất rắn màu nâu, từ từ hấp thụ hơi nước để tạo thành hợp chất ngậm 2 nước màu lục lam. Copper(II) chloride là 1 trong các hợp chất copper(II) phổ biến nhất, chỉ sau hợp chất copper(II) sulfate \emph{\ce{CuSO4}}. Xem thêm \href{https://vi.wikipedia.org/wiki/%C4%90%E1%BB%93ng(II)_chloride}{Wikipedia\emph{\texttt{/}}đồng(II) chloride}.
\end{luuy}
Nhiều oxide của các kim loại khác như MgO, CaO, \ce{Fe2O3}, $\ldots$ cũng tác dụng với dung dịch acid tạo ra muối \& nước tương tự như CuO. Oxide base tác dụng với dung dịch acid tạo ra muối \& nước:
\begin{align}
	\label{oxide base + acid}
	\boxed{\mbox{oxide base} + \mbox{acid}\to\mbox{salt} + \ce{H2O}.}
\end{align}
Cụ thể, với kim loại M hóa trị I \& acid \ce{H_xX} với gốc acid \ce{X^{x-}} có hóa trị $x\in\mathbb{N}^\star$, phương trình \eqref{oxide base + acid} trở thành:
\begin{align}
	\boxed{\ce{$x$M2O + $2$H_xX -> $2$M_xX + $x$H2O},\ \forall x\in\mathbb{N}^\star.}
\end{align}
Với kim loại M hóa trị II \& acid \ce{H_xX} với gốc acid \ce{X^{x-}} có hóa trị $x\in\mathbb{N}^\star$, phương trình \eqref{oxide base + acid} trở thành:
\begin{equation}
	\boxed{\left\{\begin{split}
			\ce{$x$MO + $2$H_xX &-> M_xX2 + $x$H2O},&&\forall x\in\mathbb{N}^\star,\,x\ne2,\\
			\ce{MO + H2X &-> MX + H2O}&&\mbox{if } x = 2.
		\end{split}\right.}
\end{equation}
Tổng quát, với kim loại M hóa trị $m\in\mathbb{N}^\star$ \& acid \ce{H_xX} với gốc acid \ce{X^{x-}} có hóa trị $x\in\mathbb{N}^\star$, phương trình \eqref{oxide base + acid} trở thành:
\begin{equation}
	\boxed{\left\{\begin{split}
			\ce{$x$M2O_m + $2m$H_xX &-> $2$M_xX_m + $mx$H2O},&&\forall m,x\in\mathbb{N}^\star,\,m\ne2,\\
			\ce{$x$MO + $2$H_xX &-> M_xX2 + $x$H2O},&&\forall x\in\mathbb{N}^\star,\mbox{ if } m = 2.
		\end{split}\right.}
\end{equation}

\subsubsection{Oxide base tác dụng với oxide acid}
Bằng thực nghiệm, người ta đã chứng minh được: 1 số oxide base như Cao, BaO, \ce{Na2O}, $\ldots$ tác dụng được với oxide acid tạo thành muối, e.g., \ce{BaO (r) + CO2 (k) -> BaCO3 (r)}. \textit{1 số oxide base tác dụng với oxide acid tạo thành muối.}

\subsubsection{Oxide acid tác dụng với nước}

\begin{vidu}[\cite{SGK_Hoa_Hoc_9}, p. 5]
	Diphosphor pentoxide \emph{\ce{P2O5}} tác dụng với \emph{\ce{H2O}} tạo thành dung dịch acid phosphoric \emph{\ce{H3PO4}: \ce{P2O5 (r) + $3$H2O (l) -> $2$H3PO4 (dd)}}.
\end{vidu}
Thí nghiệm với nhiều oxide acid khác như \ce{SO2,SO3,N2O5}, $\ldots$ ta cũng thu được những dung dịch acid tương ứng. \textit{Nhiều oxide acid tác dụng với nước tạo thành dung dịch acid.}

\subsubsection{Oxide acid tác dụng với dung dịch base}

\begin{thinghiem}[\cite{SGK_KHTN_8_Canh_Dieu}, Thí nghiệm 2, p. 61]
	\emph{Chuẩn bị:} Dụng cụ: Bình tam giác (loại \emph{100 mL}), ống thủy tinh, ống nối cao su. Hóa chất: Dung dịch nước vôi trong, \emph{\ce{CO2}} (được điều chế từ bình tạo khí \emph{\ce{CO2}}). \emph{Tiến hành:} Cho vào bình tam giác $\approx$ \emph{30 mL} nước vôi trong, dẫn khí \emph{\ce{CO2}} từ từ vào dung dịch, khi dung dịch vẫn đục thì dừng lai. Mô tả các hiện tượng xảy ra. Giải thích.
\end{thinghiem}

\begin{proof}[Giải]
	\ce{CO2} đã phản ứng với dung dịch \ce{Ca(OH)2} tạo ra muối \ce{CaCO3} không tan theo PTHH: \ce{CO2 + Ca(OH)2 -> CaCO3 v + H2O} (calcium hydroxide $\to$ calcium carbonate).
\end{proof}
Nhiều oxide của phi kim (nonmetal), e.g., \ce{SO2,SO3,P2O5}, $\ldots$ cũng tác dụng với dung dịch base tạo thành muối \& nước tương tự \ce{CO2}. \textit{Oxide acid tác dụng được với dung dịch base tạo ra muối \& nước}:
\begin{align}
	\label{oxide acid + base}
	\boxed{\mbox{oxide acid} + \mbox{base}\to\mbox{salt} + \ce{H2O}.}
\end{align}
Tổng quát,
\begin{align}
	\ce{$?$\overline{\rm M}2O_{\overline{m}} + $?$M(OH)_m -> $?$M_a(\overline{\rm M}_bO_c)_d + $?$H2O.}
\end{align}

\begin{vidu}[\cite{SGK_KHTN_8_Canh_Dieu}, p. 61, Ứng dụng của \ce{SO2}]
	Sulfur dioxide \emph{\ce{SO2}} được sử dụng phần lớn để sản xuất \emph{\ce{H2SO4}}. Ngoài ra, \emph{\ce{SO2}} còn được dùng để tẩy trắng bột gỗ trong công nghiệp giấy, làm chất diệt nấm mốc, $\ldots$
	
	Trong sản xuất rượu vang, \emph{\ce{SO2}} được dùng làm chất chống oxi hóa, ức chế 1 số loại vi khuẩn, do đó có thể lưu trữ rượu được lâu hơn. Tuy nhiên, lượng \emph{\ce{SO2}} có trong rượu luôn được kiểm soát 1 cách nghiêm ngặt để không làm ảnh hưởng đến sức khỏe người sử dụng.
\end{vidu}

\noindent\fbox{%
	\parbox{\textwidth}{%
		\noindent\textsf{\textbf{Kiến thức cốt lõi.}} \fbox{\bf 1} \textit{Oxide} là hợp chất của oxygen với 1 nguyên tố khác. \fbox{\bf 2} Oxide được phân thành 4 loại: oxide base, oxide acid, oxide lưỡng tính, \& oxide trung tính. \fbox{\bf 3} Oxide base tác dụng với dung dịch acid tạo ra muối \& nước. \fbox{\bf 4} Oxide acid tác dụng với dung dịch base tạo ra muối \& nước.
	}%
}

%------------------------------------------------------------------------------%

\section{Salt -- Muối}
\textsf{\textbf{Nội dung.} Khái niệm về muối (các muối thông thường là hợp chất được hình thành từ sự thay thế ion \ce{H+} của acid bởi ion kim loại hoặc ion \ce{NH4+}), 1 số muối tan \& muối không tan từ bảng tính tan, 1 số phương pháp điều chế muối, tên 1 số loại muối thông dụng, thí nghiệm muối phản ứng với kim loại, với acid, với base, với muối, giải thích hiện tượng xảy ra trong thí nghiệm (viết PTHH), tính chất hóa học của muối, mối quan hệ giữa acid, base, oxide, \& muối, tính chất hóa học của acid, base, oxide.}

\begin{vidu}[\cite{SGK_KHTN_8_Canh_Dieu}, p. 62]
	Muối là loại hợp chất có nhiều trong tự nhiên, trong nước biển, trong đất, trong các mỏ. (a) Muối ăn \emph{NaCl} có nhiều trong nước biển. (b) \emph{\ce{CaCO3}} có nhiều trong các mỏ đá vôi.
\end{vidu}

\subsection{Khái niệm muối}
Khi dung dịch acid tác dụng với kim loại, base, oxide base sẽ tạo ra muối.

\begin{vidu}[\cite{SGK_KHTN_8_Canh_Dieu}, p. 62]
	\emph{\ce{HCl + NaOH -> NaCl + H2O}} (sodium chloride). Trong phản ứng này ion \emph{\ce{H+}} của hydrochloric acid đã được thay thế bởi ion \emph{\ce{Na+}}.
\end{vidu}
Khi tác dụng với oxide base hoặc kim loại, ion \ce{H+} của acid cũng được thay thế bởi ion kim loại.

\begin{vidu}[\cite{SGK_KHTN_8_Canh_Dieu}, p. 63]
	\emph{\ce{H2SO4 + CuO -> CuSO4 + H2O}} (copper(II)) sulfate). Muối ammonium được tạo ra khi thay thế ion \emph{\ce{H+}} của acid bằng ion ammonium \emph{\ce{NH4+}}.
\end{vidu}

\begin{vidu}[\cite{SGK_KHTN_8_Canh_Dieu}, p. 63]
	\emph{\ce{NH4NO3}}: ammonium nitrate, \emph{\ce{(NH4)2SO4}}: ammonium sulfate.
\end{vidu}

\begin{dinhnghia}
	\emph{Muối} là các hợp chất được tạo ra khi thay thế ion \emph{\ce{H+}} trong acid bằng ion kim loại hoặc ion ammonium \emph{\ce{NH4+}}.
\end{dinhnghia}

\subsection{Tên gọi của muối}
Tên gọi muối của 1 số acid:
\begin{itemize}
	\item Hydrochloric acid HCl $\to$ muối chloride, e.g., sodium chloride NaCl, $\ldots$
	\item Sulfuric acid \ce{H2SO4} $\to$ muối sulfate, e.g., copper(II) sulfate \ce{CuSO4}, $\ldots$
	\item Phosphoric acid \ce{H3PO4} $\to$ muối phosphate, e.g., potassium phosphate \ce{K3PO4}, $\ldots$
	\item Carbonic acid \ce{H2CO3} $\to$ muối carbonate, e.g., calcium carbonate \ce{CaCO3}, $\ldots$
	\item Nitric acid \ce{HNO3} $\to$ muối nitrate, e.g., magnesium nitrate \ce{Mg(NO3)2}.
\end{itemize}

\subsection{Tính tan của muối}
\begin{itemize}
	\item Có muối tan tốt trong nước, e.g., NaCl, \ce{CuSO4,Ca(NO3)2}, $\ldots$
	\item Có muối ít tan trong nước, e.g., \ce{CaSO4,PbCl2}, $\ldots$
	\item Có muối không tan trong nước, e.g., \ce{CaCO3,BaSO4,AgCl}, $\ldots$
\end{itemize}
Tính tan của 1 số muối được trình bày trong bảng tính tan của các chất.

\subsection{Tính chất hóa học của muối}

\subsubsection{Tác dụng với kim loại}

\begin{thinghiem}[\cite{SGK_KHTN_8_Canh_Dieu}, Thí nghiệm 1, p. 64]
	\emph{Chuẩn bị:} Dụng cụ: Giá để ống nghiệm, ống nghiệm, ống hút nhỏ giọt, miếng bìa màu trắng. Hóa chất: Mẩu dây đồng, dung dịch \emph{\ce{AgNO3}}. \emph{Tiến hành:} Cho mẩu dây đồng (dài $\approx2$ \emph{cm}) vào ống nghiệm, thêm vào ống nghiệm $\approx2$ \emph{mL} dung dịch \emph{\ce{AgNO3}}. Đặt miếng bìa trắng sau ống nghiệm. Mô tả các hiện tượng xảy ra. Bề mặt sợi dây đồng \& màu dung dịch trong ống nghiệm thay đổi như thế nào? Giải thích.
\end{thinghiem}
Phản ứng cũng xảy ra tương tự khi cho Mg, Zn, $\ldots$ vào các dung dịch \ce{CuSO4,AgNO3}, $\ldots$ \textit{Dung dịch muối có thể tác dụng với kim loại tạo thành muối mới \& kim loại mới}.

\subsubsection{Tác dụng với acid}

\begin{thinghiem}[\cite{SGK_KHTN_8_Canh_Dieu}, Thí nghiệm 2, p. 64]
	\emph{Chuẩn bị:} Dụng cụ: Giá để ống nghiệm, ống nghiệm, ống hút nhỏ giọt. Hóa chất: Dung dịch \emph{\ce{BaCl2}}, dung dịch \emph{\ce{H2SO4}} loãng. \emph{Tiến hành:} Lấy $\approx2$ \emph{mL} dung dịch \emph{\ce{BaCl2}} cho vào ống nghiệm, sau đó nhỏ từ từ từng giọt dung dịch \emph{\ce{H2SO4}} vào ống nghiệm ($\approx5$ giọt). Mô tả các hiện tượng xảy ra. Giải thích.
\end{thinghiem}

\begin{proof}[Giải]
	Dung dịch \ce{BaCl2} phản ứng với dung dịch \ce{H2SO4} tạo ra \ce{BaSO4} không tan, màu trắng theo PTHH: \ce{BaCl2 + H2SO4 -> BaSO4 v + $2$HCl} (barium chloride $\to$ barium sulfate).
\end{proof}
Nhiều muối khác cũng tác dụng được với dung dịch acid tạo thành muối mới \& acid mới. \textit{Muối có thể tác dụng với dung dịch acid tạo thành muối mới \& acid mới}.

\subsubsection{Tác dụng với base}

\begin{thinghiem}[\cite{SGK_KHTN_8_Canh_Dieu}, Thí nghiệm 3, p. 65]
	\emph{Chuẩn bị:} Dụng cụ: Giá để ống nghiệm, ống nghiệm, ống hút nhỏ giọt. Hóa chất: Dung dịch \emph{\ce{CuSO4}}, dung dịch \emph{\ce{NaOH}}. \emph{Tiến hành:} Lấy $\approx2$ \emph{mL} dung dịch \emph{\ce{CuSO4}} cho vào ống nghiệm, sau đó nhỏ từ từ từng giọt dung dịch \emph{\ce{NaOH}} vào ống nghiệm. Mô tả các hiện tượng xảy ra. Giải thích.
\end{thinghiem}

\begin{proof}[Giải]
	Dung dịch \ce{CuSO4} phản ứng với dung dịch NaOH tạo ra chất không tan \ce{Cu(OH)2} theo PTHH: \ce{CuSO4 + $2$NaOH -> Cu(OH)2 v + Na2SO4} (copper(II) sulfate $\to$ sodium sulfate).
\end{proof}
\textit{Muối có thể tác dụng với dung dịch base tạo thành muối mới \& base mới}.

\subsubsection{Tác dụng với muối}

\begin{thinghiem}[\cite{SGK_KHTN_8_Canh_Dieu}, Thí nghiệm 4, p. 66]
	\emph{Chuẩn bị:} Dụng cụ: Giá để ống nghiệm, ống nghiệm, ống hút nhỏ giọt. Hóa chất: Dung dịch \emph{\ce{Na2CO3}}, dung dịch \emph{\ce{CaCl2}}. \emph{Tiến hành:} Lấy $\approx2$ \emph{mL} dung dịch \emph{\ce{Na2CO3}} cho vào ống nghiệm, sau đó nhỏ từ từ từng giọt dung dịch \emph{\ce{CaCl2}} vào ống nghiệm. Mô tả các hiện tượng xảy ra. Giải thích.
\end{thinghiem}

\begin{proof}[Giải]
	Dung dịch \ce{Na2CO3} phản ứng với dung dịch \ce{CaCl2} tạo ra \ce{CaCO3} không tan theo PTHH: \ce{CaCl2 + Na2CO3 -> CaCO3 v + $2$NaCl} (calcium chloride $+$ sodium carbonate $\to$ calcium carbonate $+$ sodium chloride).
\end{proof}
\textit{2 dung dịch muối có thể tác dụng với nhau tạo thành 2 muối mới}.

\subsection{Mối quan hệ giữa acid, base, oxide, \& muối}
Mối quan hệ giữa acid, base, oxide, \& muối được tóm tắt trong sơ đồ sau:
\begin{figure}[H]
	\centering
	\includegraphics[scale=0.3]{acid_base_oxide_salt}
	\caption{Sơ đồ mối quan hệ giữa acid, base, oxide, \& muối.}
	\label{fig:acid_base_oxide_salt}
\end{figure}

\subsection{1 số phương pháp điều chế muối}
Theo sơ đồ \ref{fig:acid_base_oxide_salt}, muối có thể được tạo ra bằng các phương pháp sau:
\begin{itemize}
	\item Cho dung dịch acid tác dụng với base, e.g., \ce{H2SO4 + Cu(OH)2 -> CuSO4 + H2O}.
	\item Cho dung dịch acid tác dụng với oxide base, e.g., \ce{$3$H2SO4 + Al2O3 -> Al2(SO4)3 + $3$H2O}.
	\item Cho dung dịch acid tác dụng với muối, e.g., \ce{$2$HCl + CaCO3 -> CaCl2 + CO2 ^ + H2O}.\footnote{Acid \ce{H2CO3} mới tạo ra trong dung dịch bị phân hủy thành \ce{CO2} \& \ce{H2O}.}
	\item Cho dung dịch base tác dụng với oxide acid, e.g., \ce{$2$NaOH + CO2 -> Na2CO3 + H2O}.
	\item Cho dung dịch 2 muối tác dụng với nhau, e.g., \ce{CaCl2 + Na2CO3 -> CaCO3 v + $2$NaCl}.
\end{itemize}

\begin{vidu}[\cite{SGK_KHTN_8_Canh_Dieu},  p. 67, Ứng dụng của sodium carbonate (soda)]
	Soda là hóa chất thông dụng. Ngoài các ứng dụng trong công nghiệp, soda còn có các ứng dụng trong đời sống. Soda được coi là chất tẩy rửa đa năng, có thể làm sạch dầu mỡ \& khử trùng bề mặt. Để làm sạch các vết bẩn khó giặt như dầu mỡ, trà, cà phê bám trên quần áo cần ngâm quần áo vào nước ấm có hòa tan soda (theo tỷ lệ \emph{8 g\texttt{/}L} $\approx8$ phút hoặc lâu hơn, sau đó tiến hành giặt như bình thường.
\end{vidu}
\noindent\fbox{%
	\parbox{\textwidth}{%
		\noindent\textsf{\textbf{Kiến thức cốt lõi.}} \fbox{\bf 1} \textit{Muối} là các hợp chất được tạo ra khi thay thế ion \ce{H+} trong acid bằng ion kim loại hoặc ion ammonium \ce{NH4+}. \fbox{\bf 2} Muối tác dụng với kim loại, dung dịch acid, dung dịch base, dung dịch muối. \fbox{\bf 3} Muối có thể được tạo ra bằng cách cho dung dịch acid tác dụng với: base, oxide base, muối hoặc cho 2 dung dịch muối tác dụng với nhau, $\ldots$. \fbox{\bf 4} Acid, base, \& oxide có các tính chất hóa học sau: Dung dịch acid: làm quỳ tím chuyển sang màu đỏ, tác dụng với kim loại, base, oxide base, muối. Dung dịch base: làm quỳ tím chuyển sang màu xanh, tác dụng với dung dịch acid, oxide acid \& với dung dịch muối. Oxide base tác dụng với dung dịch acid, oxide acid tác dụng với dung dịch base.
	}%
}

%------------------------------------------------------------------------------%

\section{Phân Bón Hóa Học}
\textsf{\textbf{Nội dung.} Vai trò của phân bón (1 trong các nguồn bổ sung 1 số nguyên tố: đa lượng, trung lượng, vi lượng dưới dạng vô cơ \& hữu cơ) đối với cây trồng, thành phần \& tác dụng cơ bản của 1 số loại phân bón hóa học đối với cây trồng (phân đạm, phân lân, phân kali, phân N--P--K), ảnh hưởng của việc sử dụng phân bón hóa học (không đúng cách, không đúng liều lượng) đến môi trường của đất, nước, \& sức khỏe của con người, biện pháp giảm thiểu ô nhiễm của phân bón.}

\subsection{Khái niệm về phân bón hóa học}
Ngoài các nguyên tố C, H, \& O được hấp thụ từ nước \& không khí, cây xanh cần nhiều nguyên tố hóa học khác như: N, P, K, Ca, Mg, S, Si, B, Zn, Fe, Cu, $\ldots$ Các nguyên tố dinh dưỡng này được cây hấp thụ chủ yếu từ đất ở dạng hợp chất. Để bổ sung các nguyên tố dinh dưỡng cho cây trong quá trình canh tác, người ta sử dụng phân bón hóa học.

\begin{dinhnghia}[Phân bón hóa học]
	\emph{Phân bón hóa học} là các hóa chất có chứa các nguyên tố dinh dưỡng dùng để bón cho cây nhằm nâng cao năng suất của cây trồng.
\end{dinhnghia}
Phân bón hóa học được chia thành 3 loại:
\begin{itemize}
	\item \textit{Phân bón đa lượng} cung cấp cho cây các nguyên tố dinh dưỡng: N, P, K.
	\item \textit{Phân bón trung lượng} cung cấp cho cây các nguyên tố dinh dưỡng: Ca, Mg, S.
	\item \textit{Phân bón vi lượng} cung cấp 1 lượng rất nhỏ các nguyên tố dinh dưỡng: Si, B, Zn, Fe, Cu, $\ldots$
\end{itemize}

\subsection{1 số loại phân bón đa lượng}

\subsubsection{Phân đạm}

\begin{dinhnghia}[Phân đạm]
	\emph{Phân đạm} là các hợp chất cung cấp nguyên tố dinh dưỡng nitrogen \emph{N} cho cây trồng.
\end{dinhnghia}
Phân đạm kích thích quá trình sinh trưởng giúp cây trồng phát triển nhanh, cho nhiều hạt, củ hoặc quả \& làm tăng tỷ lệ protein thực vật. Các 3 loại phân đạm phổ biến:
\begin{itemize}
	\item \textit{Urea} \ce{(NH2)2CO} là chất rắn màu trắng, tan tốt trong nước, dùng để bón lót hoặc bón thúc, phù hợp với nhiều loại cây, nhiều loại đất.
	\item \textit{Ammonium nitrate} \ce{NH4NO3} là chất rắn màu trắng, tan tốt trong nước, thường dùng để bón thúc, phù hợp với nhiều loại đất.
	\item \textit{Ammonium sulfate} \ce{(NH4)2SO4} là chất rắn màu trắng, tan tốt trong nước, dùng để bón thúc. Ammonium sulfate làm tăng độ chua của đất vì vậy không phù hợp với đất chua, mặn.
\end{itemize}

\subsubsection{Phân lân}

\begin{dinhnghia}[Phân lân]
	\emph{Phân lân} là các hợp chất cung cấp cho cây trồng nguyên tố dinh dưỡng phosphorus \emph{P} dưới dạng các muối phosphate.
\end{dinhnghia}
Phân lân kích thích sự phát triển của rễ cây, quá trình đẻ nhánh \& nảy chồ; thúc đẩy cây ra hoa, quả sớm; tăng khả năng chống chịu của cây. Có 2 loại phân lân phổ biến:
\begin{itemize}
	\item \textit{Phân lân nung chảy} chứa các muối phosphate của calcium \& magnesium. Phân lân nung chảy có tính kiềm, ít tan trong nước; dùng để bón lót; phù hợp cho đất chua, phèn, đất đồi núi dốc; thích hợp cho lúa, ngô, \& cây lâu năm.
	\item \textit{Superphosphate} \ce{Ca(H2PO4)2} dễ tan trong nước, làm chua đất, dùng để bón lót hoặc bón thúc; thích hợp với cây ngắn ngày, với đất chua cần khử acid trước khi bón.
\end{itemize}

\subsubsection{Phân kali}

\begin{dinhnghia}[Phân kali]
	\emph{Phân kali} là các hợp chất cung cấp cho cây trồng nguyên tố dinh dưỡng potassium \emph{K} dưới dạng các muối.
\end{dinhnghia}
Phân kali làm tăng hàm lượng tinh bột, protein, vitamin, đường, $\ldots$ trong quả, củ, thân; tăng khả năng chống chịu của cây trồng đối với hạn hán, rét hại, sâu bệnh. Có 2 loại phân kali phổ biến:
\begin{itemize}
	\item \textit{Potassium chloride} KCl dễ tan trong nước; dùng để bón lót, bón thúc; thích hợp cho cây lấy tinh bột, lấy củ, lấy dầu; không thích hợp với đất nhiễm mặn.
	\item \textit{Potassium sulfate} \ce{K2SO4} dễ tan trong nước; dùng để bón lót, bón thúc; thích hợp cho cây lấy tinh bột, củ, lấy dầu, rất thích hợp cho cây không ưa nguyên tố chlorine nhưng cần nguyên tố sulfur; rất phù hợp với đất bazan \& đất xám.
\end{itemize}

\subsubsection{Phân hỗn hợp}

\begin{dinhnghia}[Phân hỗn hợp]
	\emph{Phân hỗn hợp} là loại phân chứa nhiều nguyên tố dinh dưỡng, thường gặp nhất là phân hỗn hợp chứa cả 3 nguyên tố \emph{N, P, K} \& được gọi là \emph{phân NPK}.
\end{dinhnghia}
Loại phân này được tạo ra khi trộn các loại phân đơn theo tỷ lệ N:P:K nhất định.

Độ dinh dưỡng của mỗi loại phân N, P, K được tính theo \% khối lượng N, \ce{P2O5,K2O} \& được ghi trên bao bì chứa chúng.

Phân hỗn hợp đảm bảo cho cây trồng phát triển ở tất cả các giai đoạn của quá trình sinh trưởng.

\subsection{Tác dụng của phân bón hóa học đến môi trường}
Việc sử dụng phân bón hóa học sẽ giúp tăng năng suất, chất lượng cây trồng \& góp phần cải tạo đất. Tuy nhiên, nếu sử dụng không hợp lý, phân bón hóa học có thể gây nên 1 số ảnh hưởng tiêu cực đến môi trường.

Phân bón hóa học dư thừa có thể theo nguồn nước ngấm sâu vào đất dẫn đến ô nhiễm đất, ô nhiễm nguồn nước ngầm. Phân bón bị rửa trôi cũng làm ô nhiễm nguồn nước mặt.

\subsection{1 số biện pháp để giảm thiểu ô nhiễm của phân bón hóa học}
Trước khi sử dụng, cần phải biết được nguồn gốc, chất lượng của loại phân bón; đọc kỹ hướng dẫn trên bao bì để nắm rõ loại phân, liều lượng, cách thức, \& hiệu quả sử dụng.

Để giảm thiểu ô nhiễm môi trường trong quá trình sử dụng cần tuân thủ các nguyên tắc sau:
\begin{itemize}
	\item \textbf{Bón đúng loại phân.} Cần căn cứ vào nhu cầu dinh dưỡng của cây trồng trong từng giai đoạn sinh trưởng, từng loại đất để lựa chọn loại phân phù hợp. E.g., đất chua cần hạn chế bón phân có tính acid, đất kiềm cần hạn chế bón phân có tính kiềm.
	\item \textbf{Bón đúng lúc.} Cần chia ra nhiều lần bón \& đúng thời điểm cây đang có nhu cầu được cung cấp dinh dưỡng.
	\item \textbf{Bón đúng liều lượng.} Cần bón đúng liều lượng, không bón thiếu, không bón thừa; thường xuyên theo dõi quá trình phát triển của cây trồng, đất đai, biến đổi thời tiết để có thể điều chỉnh lượng phân bón cho phù hợp.
	\item \textbf{Bón đúng cách.} Cần lựa chọn đúng cách bón cho từng loại cây trồng, từng vụ sản xuất, từng loại phân \& từng loại đất, để hạn chế phân bị rửa trôi, phân hủy hoặc làm cây bị tổn thương. E.g., đối với phân bón lót thì cần tưới đủ nước, vùi phân xuống đất ở vị trí \& độ sâu thích hợp, $\ldots$
\end{itemize}
\noindent\fbox{%
	\parbox{\textwidth}{%
		\noindent\textsf{\textbf{Kiến thức cốt lõi.}} \fbox{\bf 1} \textit{Phân bón hóa học} là các hóa chất có chứa các nguyên tố dinh dưỡng dùng để bón cho cây trồng \& được chia thành 3 loại: đa lượng, trung lượng, \& vi lượng. \fbox{\bf 2} \textit{Phân đa lượng} gồm: phân đạm cung cấp nguyên tố nitrogen, phân lân cung cấp nguyên tố phosphorus, phân kali cung cấp nguyên tố potassium, phân hỗn hợp cung cấp cho cây 2 hoặc 3 nguyên tố trên. \fbox{\bf 3} Để phát huy tối đa hiệu quả của phân bón, tránh gây tác hại đến môi trường cần phải sử dụng phân bón hóa học đúng loại, đúng lúc, đúng liều lượng, \& đúng cách.
	}%
}

%------------------------------------------------------------------------------%

\printbibliography[heading=bibintoc]
	
\end{document}