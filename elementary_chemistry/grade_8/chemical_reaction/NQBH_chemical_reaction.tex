\documentclass{article}
\usepackage[backend=biber,natbib=true,style=authoryear,maxbibnames=10]{biblatex}
\addbibresource{/home/nqbh/reference/bib.bib}
\usepackage[utf8]{vietnam}
\usepackage{tocloft}
\renewcommand{\cftsecleader}{\cftdotfill{\cftdotsep}}
\usepackage[colorlinks=true,linkcolor=blue,urlcolor=red,citecolor=magenta]{hyperref}
\usepackage{amsmath,amssymb,amsthm,float,graphicx,mathtools,tikz,tipa}
\usepackage[version=4]{mhchem}
\allowdisplaybreaks
\newtheorem{assumption}{Assumption}
\newtheorem{baitoan}{Bài toán}
\newtheorem{cauhoi}{Câu hỏi}
\newtheorem{conjecture}{Conjecture}
\newtheorem{corollary}{Corollary}
\newtheorem{dangtoan}{Dạng toán}
\newtheorem{definition}{Definition}
\newtheorem{dinhly}{Định lý}
\newtheorem{dinhnghia}{Định nghĩa}
\newtheorem{example}{Example}
\newtheorem{ghichu}{Ghi chú}
\newtheorem{hequa}{Hệ quả}
\newtheorem{hypothesis}{Hypothesis}
\newtheorem{lemma}{Lemma}
\newtheorem{luuy}{Lưu ý}
\newtheorem{nhanxet}{Nhận xét}
\newtheorem{notation}{Notation}
\newtheorem{note}{Note}
\newtheorem{principle}{Principle}
\newtheorem{problem}{Problem}
\newtheorem{proposition}{Proposition}
\newtheorem{question}{Question}
\newtheorem{remark}{Remark}
\newtheorem{theorem}{Theorem}
\newtheorem{vidu}{Ví dụ}
\usepackage[left=1cm,right=1cm,top=5mm,bottom=5mm,footskip=4mm]{geometry}

\title{Chemical Reaction -- Phản Ứng Hóa Học}
\author{Nguyễn Quản Bá Hồng\footnote{Independent Researcher, Ben Tre City, Vietnam\\e-mail: \texttt{nguyenquanbahong@gmail.com}; website: \url{https://nqbh.github.io}.}}
\date{\today}

\begin{document}
\maketitle
\begin{abstract}
	\textsc{[en]} This text is a collection of problems, from easy to advanced, about \textit{chemical reaction}. This text is also a supplementary material for my lecture note on Elementary Chemistry, which is stored \& downloadable at the following link: \href{https://github.com/NQBH/hobby/blob/master/elementary_chemistry/grade_8/NQBH_elementary_chemistry_grade_8.pdf}{GitHub\texttt{/}NQBH\texttt{/}hobby\texttt{/}elementary chemistry\texttt{/}grade 8\texttt{/}lecture}\footnote{\textsc{url}: \url{https://github.com/NQBH/hobby/blob/master/elementary_chemistry/grade_8/NQBH_elementary_chemistry_grade_8.pdf}.}. The latest version of this text has been stored \& downloadable at the following link: \href{https://github.com/NQBH/hobby/blob/master/elementary_chemistry/chemical_reaction/NQBH_chemical_reaction.pdf}{GitHub\texttt{/}NQBH\texttt{/}hobby\texttt{/}elementary chemistry\texttt{/}grade 8\texttt{/}chemical reaction}\footnote{\textsc{url}: \url{https://github.com/NQBH/hobby/blob/master/elementary_chemistry/chemical_reaction/NQBH_chemical_reaction.pdf}.}.
	\vspace{2mm}
	
	\textsc{[vi]} Tài liệu này là 1 bộ sưu tập các bài tập chọn lọc từ cơ bản đến nâng cao về \textit{phản ứng hóa học}. Tài liệu này là phần bài tập bổ sung cho tài liệu chính -- bài giảng \href{https://github.com/NQBH/hobby/blob/master/elementary_chemistry/grade_8/NQBH_elementary_chemistry_grade_8.pdf}{GitHub\texttt{/}NQBH\texttt{/}hobby\texttt{/}elementary chemistry\texttt{/}grade 8\texttt{/}lecture} của tác giả viết cho Hóa Học Sơ Cấp. Phiên bản mới nhất của tài liệu này được lưu trữ \& có thể tải xuống ở link sau: \href{https://github.com/NQBH/hobby/blob/master/elementary_chemistry/grade_8/real/NQBH_real.pdf}{GitHub\texttt{/}NQBH\texttt{/}hobby\texttt{/}elementary chemistry\texttt{/}grade 8\texttt{/}chemical reaction}.
\end{abstract}
\setcounter{secnumdepth}{4}
\setcounter{tocdepth}{3}
\tableofcontents
\newpage

%------------------------------------------------------------------------------%

\section{Physical- \& Chemical Transformations -- Biến Đổi Vật Lý \& Biến Đổi Hóa Học}
\textsf{\textbf{Nội dung.} Biến đổi vật lý, biến đổi hóa học.}

\begin{baitoan}[\cite{SGK_KHTN_8_Canh_Dieu}, p. 12]
	Các hiện tượng sau mô tả hiện tượng chất bị biến đổi thành chất khác hay chỉ mô tả sự thay đổi về tính chất vật lý (trạng thái, kích thước, hình dạng, $\ldots$)? (a) Xẻ mẩu giấy vụn. (b) Hòa tan đường vào nước. (c) Đinh sắt bị uốn cong. (d) Đốt mẩu giấy vụn. (e) Đun đường. (f) Đinh sắt bị gỉ.
\end{baitoan}

\subsection{Sự biến đổi chất}

\subsubsection{Sự biến đổi vật lý}

\begin{dinhnghia}[Biến đổi vật lý]
	\emph{Biến đổi vật lý} là hiện tượng chất có sự biến đổi về trạng thái, kích thước, $\ldots$ nhưng vẫn giữ nguyên là chất ban đầu.
\end{dinhnghia}

\begin{vidu}
	Nước hoa khuếch tán trong không khí, hòa tan đường vào nước, làm đá trong tủ lạnh, $\ldots$ là các biến đổi vật lý.
\end{vidu}

\begin{baitoan}[\cite{SGK_KHTN_8_Canh_Dieu}, 1, p. 13]
	Kể thêm vài hiện tượng xảy ra trong thực tế có sự biến đổi vật lý.
\end{baitoan}

\subsubsection{Sự biến đổi hóa học}

\begin{dinhnghia}[Biến đổi hóa học]
	\emph{Biến đổi hóa học} là hiện tượng chất có sự biến đổi tạo ra chất khác.
\end{dinhnghia}

\begin{vidu}
	Quá trình tiêu hóa thức ăn, trứng để lâu ngày bị thối, nung đá vôi tạo thành vôi sống, $\ldots$ là các biến đổi hóa học.
\end{vidu}

\begin{baitoan}[\cite{SGK_KHTN_8_Canh_Dieu}, 2, p. 14]
	Kể thêm vài hiện tượng xảy ra trong thực tế có sự biến đổi hóa học.
\end{baitoan}

\subsection{Phân biệt sự biến đổi vật lý \& sự biến đổi hóa học}

\begin{baitoan}[\cite{SGK_KHTN_8_Canh_Dieu}, p. 14]
	Gắn cây nến (có thành phần chính là paraffin) trên đĩa sứ, đốt nến cháy trong khoảng $1$ phút. Mô tả các hiện tượng xảy ra trong quá trình nến cháy, chỉ ra giai đoạn diễn ra sự biến đổi vật lý, giai đoạn diễn ra sự biến đổi hóa học. Biết nến cháy trong không khí chủ yếu tạo ra khí carbon dioxide \& hơi nước.
\end{baitoan}

\begin{baitoan}[\cite{SGK_KHTN_8_Canh_Dieu}, 3, p. 14]
	Quá trình nào diễn ra sự biến đổi vật lý, quá trình nào diễn ra sự biến đổi hóa học? (a) Quả táo khi vẫn còn tươi $\to$ Quả táo để lâu ngày bị hỏng. (b) Vỏ lon nước ngọt $\to$ Vỏ lon nước ngọt bị bóp méo. (c) Bánh mì trước khi nướng $\to$ Bánh mì bị nướng cháy. (d) Hạt gạo $\to$ Bột gạo.
\end{baitoan}

\begin{baitoan}[\cite{SGK_KHTN_8_Canh_Dieu}, 4, p. 14]
	Nêu những điểm khác nhau giữa sự biến đổi vật lý \& sự biến đổi hóa học.
\end{baitoan}

\begin{baitoan}[\cite{SGK_KHTN_8_Canh_Dieu}, 3, p. 15]
	Trường hợp nào diễn ra sự biến đổi vật lý, trường hợp nào diễn ra sự biến đổi hóa học? (a) Khi có dòng điện đi qua, dây tóc bóng đèn (làm bằng kim loại tungsten) nóng \& sáng lên. (b) Hiện tượng băng tan. (c) Thức ăn bị ôi thiu. (d) Đốt cháy khí methane \emph{\ce{CH4}} thu được khí carbon dioxide \emph{\ce{CO2}} \& hơi nước \emph{\ce{H2O}}.
\end{baitoan}
Động Phong Nha (Động nước) là động tiêu biểu nhất của hệ thống hang động thuộc quần thể danh thắng Phong Nha--Kẻ Bàng. Đặc trưng của nơi đây là có nhiều thạch nhũ với các hình dáng đẹp, độc đáo. Hiện tượng thạch nhũ được tạo thành chủ yếu là do sự biến đổi hóa học. Ở các vùng núi đá vôi (thành phần chủ yếu là \ce{CaCO3}), khi trời mưa, nước mưa kết hợp với \ce{CO2} trong không khí tạo thành môi trường acid, làm tan được đá vôi (\ce{CaCO3} chuyển hóa thành \ce{Ca(HCO3)2}). Khi nước có chứa \ce{Ca(HCO3)2} chảy qua các khe đá vào trong các hang động (ở đây có sự thay đổi về nhiệt độ \& áp suất), \ce{Ca(HCO3)2} chuyển thành \ce{CaCO3} rắn, không tan. Lớp \ce{CaCO3} dần dần tích lại ngày càng nhiều, qua hàng triệu triệu năm tạo thành thạch nhũ với những hình thù đa dạng, đẹp mắt.

\noindent\fbox{%
	\parbox{\textwidth}{%
		\noindent\textsf{\textbf{Kiến thức cốt lõi.}} \fbox{\bf 1} \textit{Biến đổi vật lý} là hiện tượng chất có sự biến đổi về trạng thái, kích thước, $\ldots$ nhưng vẫn giữ nguyên là chất ban đầu. \fbox{\bf 2} \textit{Biến đổi hóa học} là hiện tượng chất có sự biến đổi tạo ra chất khác.
	}%
}

%------------------------------------------------------------------------------%

\section{Chemical Reactions \& Its Energy -- Phản Ứng Hóa Học \& Năng Lượng của Phản Ứng Hóa Học}
\textsf{\textbf{Nội dung.} Phản ứng hóa học, chất đầu \& sản phẩm, sự sắp xếp khác nhau của các nguyên tử trong phân tử chất đầu \& sản phẩm, 1 số dấu hiệu chứng tỏ có phản ứng hóa học xảy ra, phản ứng thu\texttt{/}tỏa nhiệt, các ứng dụng phổ biến của phản ứng tỏa nhiệt (đốt cháy than, xăng, dầu).}

\subsection{Phản ứng hóa học}

\begin{dinhnghia}[Phản ứng hóa học, chất tham gia phản ứng, sản phẩm]
	Quá trình biến đổi từ chất này thành chất khác gọi là \emph{phản ứng hóa học}. Chất ban đầu bị biến đổi trong phản ứng được gọi là \emph{chất tham gia phản ứng}, chất tạo thành sau phản ứng được gọi là \emph{chất sản phẩm}.
\end{dinhnghia}

\begin{vidu}[Tạo \ce{H2O}]
	Đốt cháy khí hydrogen trong không khí tạo ra ngọn lửa màu xanh, sau đó đưa ngọn lửa của khí hydrogen đang cháy vào trong bình đựng khí oxygen thì thấy khí hydrogen cháy mạnh hơn, sáng hơn, \& trên thành bình xuất hiện những giọt nước nhỏ. Ở đây đã diễn ra sự biến đổi hóa học, trong đó xảy ra quá trình biến đổi hydrogen \& oxygen tạo thành nước. Quá trình này đã xảy ra phản ứng hóa học.
	\begin{figure}[H]
		\centering
		\includegraphics[scale=0.3]{H2O}
		\caption{Thí nghiệm điều chế \& đốt cháy khí hydrogen trong khí oxygen.}
		\label{fig: H2O}
	\end{figure}
	Trong thí nghiệm này, chất tham gia phản ứng là hydrogen \emph{\ce{H2, O2}} \& chất sản phẩm là nước \emph{\ce{H2O}}.
\end{vidu}

\begin{baitoan}[\cite{SGK_KHTN_8_Canh_Dieu}, 1, p. 16]
	Quan sát hình \ref{fig: H2O}, có những quá trình biến đổi hóa học nào xảy ra?
\end{baitoan}

\begin{vidu}[\cite{SGK_KHTN_8_Canh_Dieu}, Ví dụ 1, p. 17]
	Khi đung nóng hỗn hợp bột sắt \& bột lưu huỳnh ta được hợp chất iron(II) sulfide \emph{FeS}. Chất tham gia phản ứng là sắt \& lưu huỳnh. Chất sản phẩm là iron(II) sulfide.
\end{vidu}

\begin{vidu}[\cite{SGK_KHTN_8_Canh_Dieu}, Ví dụ 2, p. 17]
	Nến cháy trong không khí tạo thành khí carbon dioxide \& hơi nước. Chất tham gia phản ứng là paraffin \& oxygen. Chất sản phẩm là carbon dioxide \& nước.
\end{vidu}

\begin{baitoan}[\cite{SGK_KHTN_8_Canh_Dieu}, 2, p. 17]
	Xác định chất tham gia phản ứng \& chất sản phẩm trong 2 trường hợp sau: (a) Đốt cháy methane tạo thành khí carbon dioxide \& nước. (b) Carbon (thành phần chính của than) cháy trong khí oxygen tạo thành khí carbon dioxide.
\end{baitoan}

\subsection{Diễn biến của phản ứng hóa học}
Phản ứng hóa học xảy ra trong thí nghiệm khí hydrogen cháy trong oxygen tạo thành nước, quá trình đó được mô tả theo sơ đồ sau:
\begin{figure}[H]
	\centering
	\includegraphics[scale=0.3]{PUHH_H2O}
	\caption{Sơ đồ mô tả phản ứng hóa học giữa khí hydrogen \& khí oxygen tạo thành nước.}
	\label{fig: PUHH H2O}
\end{figure}
\noindent Ứng với phương trình hóa học: \ce{$2$H2 + O2 ->[$t^\circ$] $2$H2O}. Trong sơ đồ \ref{fig: PUHH H2O}, các liên kết trong phân tử \ce{H2, O2} bị phá vỡ \& hình thành liên kết mới giữa 1 nguyên tử O \& 2 nguyên tử H.

Các biến đổi hóa học xảy ra khi có sự phá vỡ liên kết trong các chất tham gia phản ứng \& sự hình thành các liên kết mới để tạo ra các chất sản phẩm. Trong phản ứng hóa học, chỉ có liên kết giữa các nguyên tử thay đổi làm cho phân tử này biến đổi thành phân tử khác, kết quả là chất này biến đổi thành chất khác. Số nguyên tử của mỗi nguyên tố trước \& sau phản ứng không thay đổi.

\begin{baitoan}[\cite{SGK_KHTN_8_Canh_Dieu}, 3, p. 17]
	Quan sát sơ đồ \ref{fig: PUHH H2O}: (a) Trước phản ứng, những nguyên tử nào liên kết với nhau? (b) Sau phản ứng, những nguyên tử nào liên kết với nhau? (c) So sánh số nguyên tử \emph{H} \& số nguyên tử \emph{O} trước \& sau phản ứng.
\end{baitoan}

\begin{baitoan}[\cite{SGK_KHTN_8_Canh_Dieu}, 1, p. 18]
	Đốt cháy khí methane \emph{\ce{CH4}} trong không khí thu được carbon dioxide \emph{\ce{CO2}} \& nước \emph{\ce{H2O}} theo sơ đồ sau:
	\begin{figure}[H]
		\centering
		\includegraphics[scale=0.3]{CH4}
		\caption{Sơ đồ mô tả phản ứng đốt cháy khí methane trong không khí.}
		\label{fig: CH4}
	\end{figure}
	Quan sát sơ đồ \ref{fig: CH4}: (a) Trước phản ứng có các chất nào, những nguyên tử nào liên kết với nhau? (b) Sau phản ứng, có các chất nào được tạo thành, những nguyên tử nào liên kết với nhau? (c) So sánh số nguyên tử \emph{C, H, O} trước \& sau phản ứng.
\end{baitoan}

\subsection{Dấu hiệu có phản ứng hóa học xảy ra}
Để nhận biết có phản ứng hóa học xảy ra có thể dựa vào các dấu hiệu sau:

\fbox{\bf 1} \textit{Có sự thay đổi màu sắc, mùi, $\ldots$ của các chất; tạo ra chất khí hoặc chất không tan (kết tủa), $\ldots$}

\begin{vidu}
	 (a) Trong phản ứng giữa khí hydrogen với khí oxygen, nước tạo ra không còn tính chất của hydrogn \& oxygen nữa (nước ở thể lỏng, không cháy được, $\ldots$). (b) Trong phản ứng của sắt tác dụng với hydrochloric acid, quan sát thấy có bọt khí bay lên.
\end{vidu}

\begin{baitoan}[\cite{SGK_KHTN_8_Canh_Dieu}, 4, p. 18]
	Chỉ ra sự khác biệt về tính chất của nước với hydrogen \& oxygen.
\end{baitoan}
Dấu hiệu có phản ứng hóa học xảy ra trong phản ứng phân hủy đường:

\begin{baitoan}[\cite{SGK_KHTN_8_Canh_Dieu}, p. 18]
	 Cho khoảng 1 thìa cafe đường ăn vào ống nghiệm, sau đó đun trên ngọn lửa đèn cồn. Mô tả trạng thái (thể, màu sắc, $\ldots$) của đường trước \& sau khi đun. Nêu dấu hiệu chứng tỏ có phản ứng hóa học xảy ra.
\end{baitoan}

\begin{baitoan}[\cite{SGK_KHTN_8_Canh_Dieu}, p. 19]
	Nước đường để trong không khí 1 thời gian có vị chua. Trong trường hợp này, dấu hiệu nào chứng tỏ có phản ứng hóa học xảy ra?
\end{baitoan}
\fbox{\bf 2} \textit{Có sự tỏa nhiệt \& phát sáng}: Sự tỏa nhiệt \& phát sáng cũng có thể là dấu hiệu của phản ứng hóa học xảy ra.

\begin{vidu}[\cite{SGK_KHTN_8_Canh_Dieu}, p. 19]
	Khi đốt nến, nến cháy có sự tỏa nhiệt \& phát sáng.
\end{vidu}

\begin{baitoan}[\cite{SGK_KHTN_8_Canh_Dieu}, 1, p. 19]
	Những dấu hiệu nào thường dùng để nhận biết có phản ứng hóa học xảy ra?
\end{baitoan}

\subsection{Phản ứng thu\texttt{/}tỏa nhiệt}

\noindent\fbox{%
	\parbox{\textwidth}{%
		\noindent\textsf{\textbf{Kiến thức cốt lõi.}} \fbox{\bf 1} \textit{Phản ứng hóa học} là quá trình biến đổi từ chất này thành chất khác. \fbox{\bf 2} Trong phản ứng hóa học, chỉ có liên kết giữa các nguyên tử thay đổi làm cho phân tử này biến đổi thành phân tử khác, kết quả là chất này biến đổi thành chất khác. \fbox{\bf 3} Dấu hiệu thường dùng để nhận biết có phản ứng hóa học xảy ra: có sự thay đổi màu sắc, mùi, $\ldots$ của các chất; tạo ra chất khí hoặc chất không tan (kết tủa); có sự tỏa nhiệt \& phát sáng; $\ldots$ \fbox{\bf 4} \textit{Phản ứng tỏa nhiệt} là phản ứng tỏa ra năng lượng dưới dạng nhiệt. \fbox{\bf 5} \textit{Phản ứng thu nhiệt} là phản ứng thu vào năng lượng dưới dạng nhiệt.
	}%
}

%------------------------------------------------------------------------------%

\section{Định Luật Bảo Toàn Khối Lượng. Phương Trình Hóa Học}

\noindent\fbox{%
	\parbox{\textwidth}{%
		\noindent\textsf{\textbf{Kiến thức cốt lõi.}} \fbox{\bf 1} \textit{Định luật bảo toàn khối lượng}: Trong 1 phản ứng hóa học, tổng khối lượng của các chất sản phẩm bằng tổng khối lượng của các chất tham gia phản ứng. \fbox{\bf 2} Trong 1 phản ứng có $n$ chất ($n\in\mathbb{N}$, $n\ge2$) (bao gồm cả chất tham gia phản ứng \& chất sản phẩm), nếu biết khối lượng của $(n - 1)$ chất thì có thể tính được khối lượng của chất còn lại. \fbox{\bf 3} \textit{Phương trình hóa học} (PTHH) biểu diễn ngắn gọn phản ứng hóa học bằng ký hiệu \& CTHH. \fbox{\bf 4} Các bước lập PTHH: \textit{Bước 1}: Viết sơ đồ phản ứng. \textit{Bước 2}: So sánh số nguyên tử của mỗi nguyên tố có trong phân tử của các chất tham gia phản ứng \& các chất sản phẩm. \textit{Bước 3}: Cân bằng số nguyên tử của mỗi nguyên tố. \textit{Bước 4}: Kiểm tra \& viết PTHH. \fbox{\bf 5} Phương trình hóa học cho biết chất tham gia phản ứng, chất sản phẩm \& tỷ lệ về số nguyên tử hoặc số phân tử giữa các chất cũng như từng cặp chất trong phản ứng.
	}%
}

%------------------------------------------------------------------------------%

\section{Mol \& Tỷ Khối của Chất Khí}

\noindent\fbox{%
	\parbox{\textwidth}{%
		\noindent\textsf{\textbf{Kiến thức cốt lõi.}} \fbox{\bf 1} \textit{Mol} là lượng chất có chứa $6.022\cdot10^{23}$ nguyên tử hoặc phân tử của chất đó. \fbox{\bf 2} \textit{Khối lượng mol}. (ký hiệu là $M$) của 1 chất là khối lượng tính bằng gam của $N$ nguyên tử hoặc phân tử chất đó. \fbox{\bf 3} \textit{Thể tích mol} của chất khí là thể tích chiếm bởi $N$ phân tử của chất khí đó. Ở điều kiện chuẩn (áp suất 1 bar, nhiệt độ $25^\circ$C), thể tích mol của các chất khí đều bằng $24.79$ lít. \fbox{\bf 4} Công thức chuyển đổi giữa số mol $n$ \& khối lượng $m$ chất: $n = \frac{m}{M}$ mol. \fbox{\bf 5} Công thức chuyển đổi giữa số mol $n$ \& thể tích $V$ của chất khí ở điều kiện chuẩn: $n = \frac{V}{24.79}$ mol. \fbox{\bf 6} Công thức tính tỷ khối của khí A đối với khí B: $d_{\rm A\texttt{/}B} = \frac{M_{\rm A}}{M_{\rm B}}$.
	}%
}

%------------------------------------------------------------------------------%

\section{Tính Theo Phương Trình Hóa Học}

\noindent\fbox{%
	\parbox{\textwidth}{%
		\noindent\textsf{\textbf{Kiến thức cốt lõi.}} \fbox{\bf 1} Các bước tính khối lượng \& số mol của chất tham gia, chất sản phẩm trong phản ứng hóa học: \textit{Bước 1}: Viết PTHH của phản ứng. \textit{Bước 2}: Tính số mol chất đã biết dựa vào khối lượng hoặc thể tích. \textit{Bước 3}: Dựa vào PTHH để tìm số mol chất tham gia hoặc chất sản phẩm. \textit{Bước 4}: Tính khối lượng hoặc thể tích của chất cần tìm. \fbox{\bf 2} \textit{Hiệu suất phản ứng} là tỷ lệ giữa lượng sản phẩm thu được theo thực tế \& lượng sản phẩm thu được theo lý thuyết.
	}%
}

%------------------------------------------------------------------------------%

\section{Nồng Độ Dung Dịch}

\noindent\fbox{%
	\parbox{\textwidth} (ký hiệu là $C$\%) của 1 dung dịch là số gam chất tan có trong $100$ gam dung dịch: $C\% = \frac{m_{\rm ct}\cdot100\%}{m_{\rm dd}}$. \fbox{\bf 4} \textit{Nồng độ mol} (ký hiệu là $C_M$) của 1 dung dịch là số mol chất tan có trong 1 lít dung dịch. $C_M = \frac{n}{V}$ mol\texttt{/}L.
	}%
}

%------------------------------------------------------------------------------%

\section{Tốc Độ Phản Ứng \& Chất Xúc Tác}

\noindent\fbox{%
	\parbox{\textwidth}{%
		\noindent\textsf{\textbf{Kiến thức cốt lõi.}} \fbox{\bf 1} \textit{Tốc độ phản ứng} là đại lượng chỉ mức độ nhanh hay chậm của 1 phản ứng hóa học. \fbox{\bf 2} Các yếu tố ảnh hưởng đến tốc độ phản ứng: \textit{Diện tích bề mặt tiếp xúc}: Diện tích bề mặt tiếp xúc càng lớn, tốc độ phản ứng càng nhanh. \textit{Nhiệt độ}: Khi tăng nhiệt độ, phản ứng diễn ra với tốc độ nhanh hơn. \textit{Nồng độ}: Nồng độ các chất phản ứng càng cao, tốc độ phản ứng càng nhanh. \textit{Chất xúc tác} làm tăng tốc độ phản ứng nhưng không bị thay đổi cả về lượng \& chất sau phản ứng. \textit{Chất ức chế} làm giảm tốc độ phản ứng.
	}%
}

%------------------------------------------------------------------------------%

\printbibliography[heading=bibintoc]
	
\end{document}