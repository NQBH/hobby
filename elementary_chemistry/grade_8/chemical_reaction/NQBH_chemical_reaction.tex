\documentclass{article}
\usepackage[backend=biber,natbib=true,style=authoryear,maxbibnames=10]{biblatex}
\addbibresource{/home/nqbh/reference/bib.bib}
\usepackage[utf8]{vietnam}
\usepackage{tocloft}
\renewcommand{\cftsecleader}{\cftdotfill{\cftdotsep}}
\usepackage[colorlinks=true,linkcolor=blue,urlcolor=red,citecolor=magenta]{hyperref}
\usepackage{amsmath,amssymb,amsthm,float,graphicx,mathtools,tikz,tipa}
\usepackage[version=4]{mhchem}
\allowdisplaybreaks
\newtheorem{assumption}{Assumption}
\newtheorem{baitoan}{Bài toán}
\newtheorem{cauhoi}{Câu hỏi}
\newtheorem{conjecture}{Conjecture}
\newtheorem{corollary}{Corollary}
\newtheorem{dangtoan}{Dạng toán}
\newtheorem{definition}{Definition}
\newtheorem{dinhly}{Định lý}
\newtheorem{dinhnghia}{Định nghĩa}
\newtheorem{example}{Example}
\newtheorem{ghichu}{Ghi chú}
\newtheorem{hequa}{Hệ quả}
\newtheorem{hypothesis}{Hypothesis}
\newtheorem{lemma}{Lemma}
\newtheorem{luuy}{Lưu ý}
\newtheorem{nhanxet}{Nhận xét}
\newtheorem{notation}{Notation}
\newtheorem{note}{Note}
\newtheorem{principle}{Principle}
\newtheorem{problem}{Problem}
\newtheorem{proposition}{Proposition}
\newtheorem{question}{Question}
\newtheorem{remark}{Remark}
\newtheorem{theorem}{Theorem}
\newtheorem{vidu}{Ví dụ}
\usepackage[left=1cm,right=1cm,top=5mm,bottom=5mm,footskip=4mm]{geometry}

\title{Chemical Reaction -- Phản Ứng Hóa Học}
\author{Nguyễn Quản Bá Hồng\footnote{Independent Researcher, Ben Tre City, Vietnam\\e-mail: \texttt{nguyenquanbahong@gmail.com}; website: \url{https://nqbh.github.io}.}}
\date{\today}

\begin{document}
\maketitle
\begin{abstract}
	\textsc{[en]} This text is a collection of problems, from easy to advanced, about \textit{chemical reaction}. This text is also a supplementary material for my lecture note on Elementary Chemistry, which is stored \& downloadable at the following link: \href{https://github.com/NQBH/hobby/blob/master/elementary_chemistry/grade_8/NQBH_elementary_chemistry_grade_8.pdf}{GitHub\texttt{/}NQBH\texttt{/}hobby\texttt{/}elementary chemistry\texttt{/}grade 8\texttt{/}lecture}\footnote{\textsc{url}: \url{https://github.com/NQBH/hobby/blob/master/elementary_chemistry/grade_8/NQBH_elementary_chemistry_grade_8.pdf}.}. The latest version of this text has been stored \& downloadable at the following link: \href{https://github.com/NQBH/hobby/blob/master/elementary_chemistry/chemical_reaction/NQBH_chemical_reaction.pdf}{GitHub\texttt{/}NQBH\texttt{/}hobby\texttt{/}elementary chemistry\texttt{/}grade 8\texttt{/}chemical reaction}\footnote{\textsc{url}: \url{https://github.com/NQBH/hobby/blob/master/elementary_chemistry/chemical_reaction/NQBH_chemical_reaction.pdf}.}.
	\vspace{2mm}
	
	\textsc{[vi]} Tài liệu này là 1 bộ sưu tập các bài tập chọn lọc từ cơ bản đến nâng cao về \textit{phản ứng hóa học}. Tài liệu này là phần bài tập bổ sung cho tài liệu chính -- bài giảng \href{https://github.com/NQBH/hobby/blob/master/elementary_chemistry/grade_8/NQBH_elementary_chemistry_grade_8.pdf}{GitHub\texttt{/}NQBH\texttt{/}hobby\texttt{/}elementary chemistry\texttt{/}grade 8\texttt{/}lecture} của tác giả viết cho Hóa Học Sơ Cấp. Phiên bản mới nhất của tài liệu này được lưu trữ \& có thể tải xuống ở link sau: \href{https://github.com/NQBH/hobby/blob/master/elementary_chemistry/grade_8/real/NQBH_real.pdf}{GitHub\texttt{/}NQBH\texttt{/}hobby\texttt{/}elementary chemistry\texttt{/}grade 8\texttt{/}chemical reaction}.
\end{abstract}
\setcounter{secnumdepth}{4}
\setcounter{tocdepth}{3}
\tableofcontents
\newpage

%------------------------------------------------------------------------------%

\section{Biến Đổi Vật Lý \& Biến Đổi Hóa Học}
\textsf{\textbf{Nội dung.} Biến đổi vật lý, biến đổi hóa học.}

\begin{baitoan}[\cite{SGK_KHTN_8_Canh_Dieu}, p. 12]
	Các hiện tượng sau mô tả hiện tượng chất bị biến đổi thành chất khác hay chỉ mô tả sự thay đổi về tính chất vật lý (trạng thái, kích thước, hình dạng, $\ldots$)? (a) Xẻ mẩu giấy vụn. (b) Hòa tan đường vào nước. (c) Đinh sắt bị uốn cong. (d) Đốt mẩu giấy vụn. (e) Đun đường. (f) Đinh sắt bị gỉ.
\end{baitoan}

\subsection{Sự biến đổi chất}

\subsubsection{Sự biến đổi vật lý}

\begin{dinhnghia}[Biến đổi vật lý]
	\emph{Biến đổi vật lý} là hiện tượng chất có sự biến đổi về trạng thái, kích thước, $\ldots$ nhưng vẫn giữ nguyên là chất ban đầu.
\end{dinhnghia}

\begin{vidu}
	Nước hoa khuếch tán trong không khí, hòa tan đường vào nước, làm đá trong tủ lạnh, $\ldots$ là các biến đổi vật lý.
\end{vidu}

\begin{baitoan}[\cite{SGK_KHTN_8_Canh_Dieu}, 1, p. 13]
	Kể thêm vài hiện tượng xảy ra trong thực tế có sự biến đổi vật lý.
\end{baitoan}

\subsubsection{Sự biến đổi hóa học}

\begin{dinhnghia}[Biến đổi hóa học]
	\emph{Biến đổi hóa học} là hiện tượng chất có sự biến đổi tạo ra chất khác.
\end{dinhnghia}

\begin{vidu}
	Quá trình tiêu hóa thức ăn, trứng để lâu ngày bị thối, nung đá vôi tạo thành vôi sống, $\ldots$ là các biến đổi hóa học.
\end{vidu}

\begin{baitoan}[\cite{SGK_KHTN_8_Canh_Dieu}, 2, p. 14]
	Kể thêm vài hiện tượng xảy ra trong thực tế có sự biến đổi hóa học.
\end{baitoan}

\subsection{Phân biệt sự biến đổi vật lý \& sự biến đổi hóa học}

\begin{baitoan}[\cite{SGK_KHTN_8_Canh_Dieu}, p. 14]
	Gắn cây nến (có thành phần chính là paraffin) trên đĩa sứ, đốt nến cháy trong khoảng $1$ phút. Mô tả các hiện tượng xảy ra trong quá trình nến cháy, chỉ ra giai đoạn diễn ra sự biến đổi vật lý, giai đoạn diễn ra sự biến đổi hóa học. Biết nến cháy trong không khí chủ yếu tạo ra khí carbon dioxide \& hơi nước.
\end{baitoan}

\begin{baitoan}[\cite{SGK_KHTN_8_Canh_Dieu}, 3, p. 14]
	Quá trình nào diễn ra sự biến đổi vật lý, quá trình nào diễn ra sự biến đổi hóa học? (a) Quả táo khi vẫn còn tươi $\to$ Quả táo để lâu ngày bị hỏng. (b) Vỏ lon nước ngọt $\to$ Vỏ lon nước ngọt bị bóp méo. (c) Bánh mì trước khi nướng $\to$ Bánh mì bị nướng cháy. (d) Hạt gạo $\to$ Bột gạo.
\end{baitoan}

\begin{baitoan}[\cite{SGK_KHTN_8_Canh_Dieu}, 4, p. 14]
	Nêu những điểm khác nhau giữa sự biến đổi vật lý \& sự biến đổi hóa học.
\end{baitoan}

\begin{baitoan}[\cite{SGK_KHTN_8_Canh_Dieu}, 3, p. 15]
	Trường hợp nào diễn ra sự biến đổi vật lý, trường hợp nào diễn ra sự biến đổi hóa học? (a) Khi có dòng điện đi qua, dây tóc bóng đèn (làm bằng kim loại tungsten) nóng \& sáng lên. (b) Hiện tượng băng tan. (c) Thức ăn bị ôi thiu. (d) Đốt cháy khí methane \emph{\ce{CH4}} thu được khí carbon dioxide \emph{\ce{CO2}} \& hơi nước \emph{\ce{H2O}}.
\end{baitoan}
Động Phong Nha (Động nước) là động tiêu biểu nhất của hệ thống hang động thuộc quần thể danh thắng Phong Nha--Kẻ Bàng. Đặc trưng của nơi đây là có nhiều thạch nhũ với các hình dáng đẹp, độc đáo. Hiện tượng thạch nhũ được tạo thành chủ yếu là do sự biến đổi hóa học. Ở các vùng núi đá vôi (thành phần chủ yếu là \ce{CaCO3}), khi trời mưa, nước mưa kết hợp với \ce{CO2} trong không khí tạo thành môi trường acid, làm tan được đá vôi (\ce{CaCO3} chuyển hóa thành \ce{Ca(HCO3)2}). Khi nước có chứa \ce{Ca(HCO3)2} chảy qua các khe đá vào trong các hang động (ở đây có sự thay đổi về nhiệt độ \& áp suất), \ce{Ca(HCO3)2} chuyển thành \ce{CaCO3} rắn, không tan. Lớp \ce{CaCO3} dần dần tích lại ngày càng nhiều, qua hàng triệu triệu năm tạo thành thạch nhũ với những hình thù đa dạng, đẹp mắt.

\noindent\fbox{%
	\parbox{\textwidth}{%
		\noindent\textsf{\textbf{Kiến thức cốt lõi.}} \fbox{\bf 1} \textit{Biến đổi vật lý} là hiện tượng chất có sự biến đổi về trạng thái, kích thước, $\ldots$ nhưng vẫn giữ nguyên là chất ban đầu. \fbox{\bf 2} \textit{Biến đổi hóa học} là hiện tượng chất có sự biến đổi tạo ra chất khác.
	}%
}

%------------------------------------------------------------------------------%

\section{Phản Ứng Hóa Học \& Năng Lượng của Phản Ứng Hóa Học}

%------------------------------------------------------------------------------%

\section{Định Luật Bảo Toàn Khối Lượng. Phương Trình Hóa Học}

%------------------------------------------------------------------------------%

\section{Mol \& Tỷ Khối của Chất Khí}

%------------------------------------------------------------------------------%

\section{Tính Theo Phương Trình Hóa Học}

%------------------------------------------------------------------------------%

\section{Nồng Độ Dung Dịch}

%------------------------------------------------------------------------------%

\section{Tốc Độ Phản Ứng \& Chất Xúc Tác}

%------------------------------------------------------------------------------%

\printbibliography[heading=bibintoc]
	
\end{document}