\documentclass{article}
\usepackage[backend=biber,natbib=true,style=authoryear,maxbibnames=20]{biblatex}
\addbibresource{/home/nqbh/reference/bib.bib}
\usepackage[utf8]{vietnam}
\usepackage{tocloft}
\renewcommand{\cftsecleader}{\cftdotfill{\cftdotsep}}
\usepackage[colorlinks=true,linkcolor=blue,urlcolor=red,citecolor=magenta]{hyperref}
\usepackage{amsmath,amssymb,amsthm,float,graphicx,mathtools,diagbox,tikz,tipa}
\usepackage[version=4]{mhchem}
\allowdisplaybreaks
\newtheorem{assumption}{Assumption}
\newtheorem{baitoan}{Bài toán}
\newtheorem{cauhoi}{Câu hỏi}
\newtheorem{conjecture}{Conjecture}
\newtheorem{corollary}{Corollary}
\newtheorem{dangtoan}{Dạng toán}
\newtheorem{definition}{Definition}
\newtheorem{dinhly}{Định lý}
\newtheorem{dinhnghia}{Định nghĩa}
\newtheorem{example}{Example}
\newtheorem{ghichu}{Ghi chú}
\newtheorem{hequa}{Hệ quả}
\newtheorem{hypothesis}{Hypothesis}
\newtheorem{lemma}{Lemma}
\newtheorem{luuy}{Lưu ý}
\newtheorem{nhanxet}{Nhận xét}
\newtheorem{notation}{Notation}
\newtheorem{note}{Note}
\newtheorem{principle}{Principle}
\newtheorem{problem}{Problem}
\newtheorem{proposition}{Proposition}
\newtheorem{question}{Question}
\newtheorem{remark}{Remark}
\newtheorem{theorem}{Theorem}
\newtheorem{thinghiem}{Thí nghiệm}
\newtheorem{vidu}{Ví dụ}
\usepackage[left=1cm,right=1cm,top=5mm,bottom=5mm,footskip=4mm]{geometry}

\title{Chemical Reaction -- Phản Ứng Hóa Học}
\author{Nguyễn Quản Bá Hồng\footnote{Independent Researcher, Ben Tre City, Vietnam\\e-mail: \texttt{nguyenquanbahong@gmail.com}; website: \url{https://nqbh.github.io}.}}
\date{\today}

\begin{document}
\maketitle
\begin{abstract}
	\textsc{[en]} This text is a collection of problems, from easy to advanced, about \textit{chemical reaction}. This text is also a supplementary material for my lecture note on Elementary Chemistry, which is stored \& downloadable at the following link: \href{https://github.com/NQBH/hobby/blob/master/elementary_chemistry/grade_8/NQBH_elementary_chemistry_grade_8.pdf}{GitHub\texttt{/}NQBH\texttt{/}hobby\texttt{/}elementary chemistry\texttt{/}grade 8\texttt{/}lecture}\footnote{\textsc{url}: \url{https://github.com/NQBH/hobby/blob/master/elementary_chemistry/grade_8/NQBH_elementary_chemistry_grade_8.pdf}.}. The latest version of this text has been stored \& downloadable at the following link: \href{https://github.com/NQBH/hobby/blob/master/elementary_chemistry/chemical_reaction/NQBH_chemical_reaction.pdf}{GitHub\texttt{/}NQBH\texttt{/}hobby\texttt{/}elementary chemistry\texttt{/}grade 8\texttt{/}chemical reaction}\footnote{\textsc{url}: \url{https://github.com/NQBH/hobby/blob/master/elementary_chemistry/chemical_reaction/NQBH_chemical_reaction.pdf}.}.
	\vspace{2mm}
	
	\textsc{[vi]} Tài liệu này là 1 bộ sưu tập các bài tập chọn lọc từ cơ bản đến nâng cao về \textit{phản ứng hóa học}. Tài liệu này là phần bài tập bổ sung cho tài liệu chính -- bài giảng \href{https://github.com/NQBH/hobby/blob/master/elementary_chemistry/grade_8/NQBH_elementary_chemistry_grade_8.pdf}{GitHub\texttt{/}NQBH\texttt{/}hobby\texttt{/}elementary chemistry\texttt{/}grade 8\texttt{/}lecture} của tác giả viết cho Hóa Học Sơ Cấp. Phiên bản mới nhất của tài liệu này được lưu trữ \& có thể tải xuống ở link sau: \href{https://github.com/NQBH/hobby/blob/master/elementary_chemistry/grade_8/real/NQBH_real.pdf}{GitHub\texttt{/}NQBH\texttt{/}hobby\texttt{/}elementary chemistry\texttt{/}grade 8\texttt{/}chemical reaction}.
\end{abstract}
\setcounter{secnumdepth}{4}
\setcounter{tocdepth}{3}
\tableofcontents
\newpage

%------------------------------------------------------------------------------%

\section{Physical- \& Chemical Transformations -- Biến Đổi Vật Lý \& Biến Đổi Hóa Học}
\textsf{\textbf{Nội dung.} Biến đổi vật lý, biến đổi hóa học.}

\begin{baitoan}[\cite{SGK_KHTN_8_Canh_Dieu}, p. 12]
	Các hiện tượng sau mô tả hiện tượng chất bị biến đổi thành chất khác hay chỉ mô tả sự thay đổi về tính chất vật lý (trạng thái, kích thước, hình dạng, $\ldots$)? (a) Xẻ mẩu giấy vụn. (b) Hòa tan đường vào nước. (c) Đinh sắt bị uốn cong. (d) Đốt mẩu giấy vụn. (e) Đun đường. (f) Đinh sắt bị gỉ.
\end{baitoan}

\subsection{Sự biến đổi chất}

\subsubsection{Sự biến đổi vật lý}

\begin{dinhnghia}[Biến đổi vật lý]
	\emph{Biến đổi vật lý} là hiện tượng chất có sự biến đổi về trạng thái, kích thước, $\ldots$ nhưng vẫn giữ nguyên là chất ban đầu.
\end{dinhnghia}

\begin{vidu}
	Nước hoa khuếch tán trong không khí, hòa tan đường vào nước, làm đá trong tủ lạnh, $\ldots$ là các biến đổi vật lý.
\end{vidu}

\begin{baitoan}[\cite{SGK_KHTN_8_Canh_Dieu}, 1, p. 13]
	Kể thêm vài hiện tượng xảy ra trong thực tế có sự biến đổi vật lý.
\end{baitoan}

\subsubsection{Sự biến đổi hóa học}

\begin{dinhnghia}[Biến đổi hóa học]
	\emph{Biến đổi hóa học} là hiện tượng chất có sự biến đổi tạo ra chất khác.
\end{dinhnghia}

\begin{vidu}
	Quá trình tiêu hóa thức ăn, trứng để lâu ngày bị thối, nung đá vôi tạo thành vôi sống, $\ldots$ là các biến đổi hóa học.
\end{vidu}

\begin{baitoan}[\cite{SGK_KHTN_8_Canh_Dieu}, 2, p. 14]
	Kể thêm vài hiện tượng xảy ra trong thực tế có sự biến đổi hóa học.
\end{baitoan}

\subsection{Phân biệt sự biến đổi vật lý \& sự biến đổi hóa học}

\begin{baitoan}[\cite{SGK_KHTN_8_Canh_Dieu}, p. 14]
	Gắn cây nến (có thành phần chính là paraffin) trên đĩa sứ, đốt nến cháy trong khoảng $1$ phút. Mô tả các hiện tượng xảy ra trong quá trình nến cháy, chỉ ra giai đoạn diễn ra sự biến đổi vật lý, giai đoạn diễn ra sự biến đổi hóa học. Biết nến cháy trong không khí chủ yếu tạo ra khí carbon dioxide \& hơi nước.
\end{baitoan}

\begin{baitoan}[\cite{SGK_KHTN_8_Canh_Dieu}, 3, p. 14]
	Quá trình nào diễn ra sự biến đổi vật lý, quá trình nào diễn ra sự biến đổi hóa học? (a) Quả táo khi vẫn còn tươi $\to$ Quả táo để lâu ngày bị hỏng. (b) Vỏ lon nước ngọt $\to$ Vỏ lon nước ngọt bị bóp méo. (c) Bánh mì trước khi nướng $\to$ Bánh mì bị nướng cháy. (d) Hạt gạo $\to$ Bột gạo.
\end{baitoan}

\begin{baitoan}[\cite{SGK_KHTN_8_Canh_Dieu}, 4, p. 14]
	Nêu những điểm khác nhau giữa sự biến đổi vật lý \& sự biến đổi hóa học.
\end{baitoan}

\begin{baitoan}[\cite{SGK_KHTN_8_Canh_Dieu}, 3, p. 15]
	Trường hợp nào diễn ra sự biến đổi vật lý, trường hợp nào diễn ra sự biến đổi hóa học? (a) Khi có dòng điện đi qua, dây tóc bóng đèn (làm bằng kim loại tungsten) nóng \& sáng lên. (b) Hiện tượng băng tan. (c) Thức ăn bị ôi thiu. (d) Đốt cháy khí methane \emph{\ce{CH4}} thu được khí carbon dioxide \emph{\ce{CO2}} \& hơi nước \emph{\ce{H2O}}.
\end{baitoan}
Động Phong Nha (Động nước) là động tiêu biểu nhất của hệ thống hang động thuộc quần thể danh thắng Phong Nha--Kẻ Bàng. Đặc trưng của nơi đây là có nhiều thạch nhũ với các hình dáng đẹp, độc đáo. Hiện tượng thạch nhũ được tạo thành chủ yếu là do sự biến đổi hóa học. Ở các vùng núi đá vôi (thành phần chủ yếu là \ce{CaCO3}), khi trời mưa, nước mưa kết hợp với \ce{CO2} trong không khí tạo thành môi trường acid, làm tan được đá vôi (\ce{CaCO3} chuyển hóa thành \ce{Ca(HCO3)2}). Khi nước có chứa \ce{Ca(HCO3)2} chảy qua các khe đá vào trong các hang động (ở đây có sự thay đổi về nhiệt độ \& áp suất), \ce{Ca(HCO3)2} chuyển thành \ce{CaCO3} rắn, không tan. Lớp \ce{CaCO3} dần dần tích lại ngày càng nhiều, qua hàng triệu triệu năm tạo thành thạch nhũ với những hình thù đa dạng, đẹp mắt.

\noindent\fbox{%
	\parbox{\textwidth}{%
		\noindent\textsf{\textbf{Kiến thức cốt lõi.}} \fbox{\bf 1} \textit{Biến đổi vật lý} là hiện tượng chất có sự biến đổi về trạng thái, kích thước, $\ldots$ nhưng vẫn giữ nguyên là chất ban đầu. \fbox{\bf 2} \textit{Biến đổi hóa học} là hiện tượng chất có sự biến đổi tạo ra chất khác.
	}%
}

%------------------------------------------------------------------------------%

\section{Chemical Reactions \& Its Energy -- Phản Ứng Hóa Học \& Năng Lượng của Phản Ứng Hóa Học}
\textsf{\textbf{Nội dung.} Phản ứng hóa học, chất đầu \& sản phẩm, sự sắp xếp khác nhau của các nguyên tử trong phân tử chất đầu \& sản phẩm, 1 số dấu hiệu chứng tỏ có phản ứng hóa học xảy ra, phản ứng thu\texttt{/}tỏa nhiệt, các ứng dụng phổ biến của phản ứng tỏa nhiệt (đốt cháy than, xăng, dầu).}

\subsection{Phản ứng hóa học}

\begin{dinhnghia}[Phản ứng hóa học, chất tham gia phản ứng, sản phẩm]
	Quá trình biến đổi từ chất này thành chất khác gọi là \emph{phản ứng hóa học}. Chất ban đầu bị biến đổi trong phản ứng được gọi là \emph{chất tham gia phản ứng}, chất tạo thành sau phản ứng được gọi là \emph{chất sản phẩm}.
\end{dinhnghia}

\begin{vidu}[Tạo \ce{H2O}]
	Đốt cháy khí hydrogen trong không khí tạo ra ngọn lửa màu xanh, sau đó đưa ngọn lửa của khí hydrogen đang cháy vào trong bình đựng khí oxygen thì thấy khí hydrogen cháy mạnh hơn, sáng hơn, \& trên thành bình xuất hiện những giọt nước nhỏ. Ở đây đã diễn ra sự biến đổi hóa học, trong đó xảy ra quá trình biến đổi hydrogen \& oxygen tạo thành nước. Quá trình này đã xảy ra phản ứng hóa học.
	\begin{figure}[H]
		\centering
		\includegraphics[scale=0.3]{H2O}
		\caption{Thí nghiệm điều chế \& đốt cháy khí hydrogen trong khí oxygen.}
		\label{fig: H2O}
	\end{figure}
	Trong thí nghiệm này, chất tham gia phản ứng là hydrogen \emph{\ce{H2, O2}} \& chất sản phẩm là nước \emph{\ce{H2O}}.
\end{vidu}

\begin{baitoan}[\cite{SGK_KHTN_8_Canh_Dieu}, 1, p. 16]
	Quan sát hình \ref{fig: H2O}, có những quá trình biến đổi hóa học nào xảy ra?
\end{baitoan}

\begin{vidu}[\cite{SGK_KHTN_8_Canh_Dieu}, Ví dụ 1, p. 17]
	Khi đung nóng hỗn hợp bột sắt \& bột lưu huỳnh ta được hợp chất iron(II) sulfide \emph{FeS}. Chất tham gia phản ứng là sắt \& lưu huỳnh. Chất sản phẩm là iron(II) sulfide.
\end{vidu}

\begin{vidu}[\cite{SGK_KHTN_8_Canh_Dieu}, Ví dụ 2, p. 17]
	Nến cháy trong không khí tạo thành khí carbon dioxide \& hơi nước. Chất tham gia phản ứng là paraffin \& oxygen. Chất sản phẩm là carbon dioxide \& nước.
\end{vidu}

\begin{baitoan}[\cite{SGK_KHTN_8_Canh_Dieu}, 2, p. 17]
	Xác định chất tham gia phản ứng \& chất sản phẩm trong 2 trường hợp sau: (a) Đốt cháy methane tạo thành khí carbon dioxide \& nước. (b) Carbon (thành phần chính của than) cháy trong khí oxygen tạo thành khí carbon dioxide.
\end{baitoan}

\subsection{Diễn biến của phản ứng hóa học}
Phản ứng hóa học xảy ra trong thí nghiệm khí hydrogen cháy trong oxygen tạo thành nước, quá trình đó được mô tả theo sơ đồ sau:
\begin{figure}[H]
	\centering
	\includegraphics[scale=0.3]{PUHH_H2O}
	\caption{Sơ đồ mô tả phản ứng hóa học giữa khí hydrogen \& khí oxygen tạo thành nước.}
	\label{fig: PUHH H2O}
\end{figure}
\noindent Ứng với phương trình hóa học: \ce{$2$H2 + O2 ->[$t^\circ$] $2$H2O}. Trong sơ đồ \ref{fig: PUHH H2O}, các liên kết trong phân tử \ce{H2, O2} bị phá vỡ \& hình thành liên kết mới giữa 1 nguyên tử O \& 2 nguyên tử H.

Các biến đổi hóa học xảy ra khi có sự phá vỡ liên kết trong các chất tham gia phản ứng \& sự hình thành các liên kết mới để tạo ra các chất sản phẩm. Trong phản ứng hóa học, chỉ có liên kết giữa các nguyên tử thay đổi làm cho phân tử này biến đổi thành phân tử khác, kết quả là chất này biến đổi thành chất khác. Số nguyên tử của mỗi nguyên tố trước \& sau phản ứng không thay đổi.

\begin{baitoan}[\cite{SGK_KHTN_8_Canh_Dieu}, 3, p. 17]
	Quan sát sơ đồ \ref{fig: PUHH H2O}: (a) Trước phản ứng, những nguyên tử nào liên kết với nhau? (b) Sau phản ứng, những nguyên tử nào liên kết với nhau? (c) So sánh số nguyên tử \emph{H} \& số nguyên tử \emph{O} trước \& sau phản ứng.
\end{baitoan}

\begin{baitoan}[\cite{SGK_KHTN_8_Canh_Dieu}, 1, p. 18]
	Đốt cháy khí methane \emph{\ce{CH4}} trong không khí thu được carbon dioxide \emph{\ce{CO2}} \& nước \emph{\ce{H2O}} theo sơ đồ sau:
	\begin{figure}[H]
		\centering
		\includegraphics[scale=0.3]{CH4}
		\caption{Sơ đồ mô tả phản ứng đốt cháy khí methane trong không khí.}
		\label{fig: CH4}
	\end{figure}
	Quan sát sơ đồ \ref{fig: CH4}: (a) Trước phản ứng có các chất nào, những nguyên tử nào liên kết với nhau? (b) Sau phản ứng, có các chất nào được tạo thành, những nguyên tử nào liên kết với nhau? (c) So sánh số nguyên tử \emph{C, H, O} trước \& sau phản ứng.
\end{baitoan}

\subsection{Dấu hiệu có phản ứng hóa học xảy ra}
Để nhận biết có phản ứng hóa học xảy ra có thể dựa vào các dấu hiệu sau:

\fbox{\bf 1} \textit{Có sự thay đổi màu sắc, mùi, $\ldots$ của các chất; tạo ra chất khí hoặc chất không tan (kết tủa), $\ldots$}

\begin{vidu}
	 (a) Trong phản ứng giữa khí hydrogen với khí oxygen, nước tạo ra không còn tính chất của hydrogn \& oxygen nữa (nước ở thể lỏng, không cháy được, $\ldots$). (b) Trong phản ứng của sắt tác dụng với hydrochloric acid, quan sát thấy có bọt khí bay lên.
\end{vidu}

\begin{baitoan}[\cite{SGK_KHTN_8_Canh_Dieu}, 4, p. 18]
	Chỉ ra sự khác biệt về tính chất của nước với hydrogen \& oxygen.
\end{baitoan}
Dấu hiệu có phản ứng hóa học xảy ra trong phản ứng phân hủy đường:

\begin{baitoan}[\cite{SGK_KHTN_8_Canh_Dieu}, p. 18]
	 Cho khoảng 1 thìa cafe đường ăn vào ống nghiệm, sau đó đun trên ngọn lửa đèn cồn. Mô tả trạng thái (thể, màu sắc, $\ldots$) của đường trước \& sau khi đun. Nêu dấu hiệu chứng tỏ có phản ứng hóa học xảy ra.
\end{baitoan}

\begin{baitoan}[\cite{SGK_KHTN_8_Canh_Dieu}, p. 19]
	Nước đường để trong không khí 1 thời gian có vị chua. Trong trường hợp này, dấu hiệu nào chứng tỏ có phản ứng hóa học xảy ra?
\end{baitoan}
\fbox{\bf 2} \textit{Có sự tỏa nhiệt \& phát sáng}: Sự tỏa nhiệt \& phát sáng cũng có thể là dấu hiệu của phản ứng hóa học xảy ra.

\begin{vidu}[\cite{SGK_KHTN_8_Canh_Dieu}, p. 19]
	Khi đốt nến, nến cháy có sự tỏa nhiệt \& phát sáng.
\end{vidu}

\begin{baitoan}[\cite{SGK_KHTN_8_Canh_Dieu}, 1, p. 19]
	Những dấu hiệu nào thường dùng để nhận biết có phản ứng hóa học xảy ra?
\end{baitoan}

\subsection{Phản ứng thu\texttt{/}tỏa nhiệt}

\noindent\fbox{%
	\parbox{\textwidth}{%
		\noindent\textsf{\textbf{Kiến thức cốt lõi.}} \fbox{\bf 1} \textit{Phản ứng hóa học} là quá trình biến đổi từ chất này thành chất khác. \fbox{\bf 2} Trong phản ứng hóa học, chỉ có liên kết giữa các nguyên tử thay đổi làm cho phân tử này biến đổi thành phân tử khác, kết quả là chất này biến đổi thành chất khác. \fbox{\bf 3} Dấu hiệu thường dùng để nhận biết có phản ứng hóa học xảy ra: có sự thay đổi màu sắc, mùi, $\ldots$ của các chất; tạo ra chất khí hoặc chất không tan (kết tủa); có sự tỏa nhiệt \& phát sáng; $\ldots$ \fbox{\bf 4} \textit{Phản ứng tỏa nhiệt} là phản ứng tỏa ra năng lượng dưới dạng nhiệt. \fbox{\bf 5} \textit{Phản ứng thu nhiệt} là phản ứng thu vào năng lượng dưới dạng nhiệt.
	}%
}

%------------------------------------------------------------------------------%

\section{Định Luật Bảo Toàn Khối Lượng. Phương Trình Hóa Học}

\noindent\fbox{%
	\parbox{\textwidth}{%
		\noindent\textsf{\textbf{Kiến thức cốt lõi.}} \fbox{\bf 1} \textit{Định luật bảo toàn khối lượng}: Trong 1 phản ứng hóa học, tổng khối lượng của các chất sản phẩm bằng tổng khối lượng của các chất tham gia phản ứng. \fbox{\bf 2} Trong 1 phản ứng có $n$ chất ($n\in\mathbb{N}$, $n\ge2$) (bao gồm cả chất tham gia phản ứng \& chất sản phẩm), nếu biết khối lượng của $(n - 1)$ chất thì có thể tính được khối lượng của chất còn lại. \fbox{\bf 3} \textit{Phương trình hóa học} (PTHH) biểu diễn ngắn gọn phản ứng hóa học bằng ký hiệu \& CTHH. \fbox{\bf 4} Các bước lập PTHH: \textit{Bước 1}: Viết sơ đồ phản ứng. \textit{Bước 2}: So sánh số nguyên tử của mỗi nguyên tố có trong phân tử của các chất tham gia phản ứng \& các chất sản phẩm. \textit{Bước 3}: Cân bằng số nguyên tử của mỗi nguyên tố. \textit{Bước 4}: Kiểm tra \& viết PTHH. \fbox{\bf 5} Phương trình hóa học cho biết chất tham gia phản ứng, chất sản phẩm \& tỷ lệ về số nguyên tử hoặc số phân tử giữa các chất cũng như từng cặp chất trong phản ứng.
	}%
}

%------------------------------------------------------------------------------%

\section{Mol \& Tỷ Khối của Chất Khí}

\noindent\fbox{%
	\parbox{\textwidth}{%
		\noindent\textsf{\textbf{Kiến thức cốt lõi.}} \fbox{\bf 1} \textit{Mol} là lượng chất có chứa $6.022\cdot10^{23}$ nguyên tử hoặc phân tử của chất đó. \fbox{\bf 2} \textit{Khối lượng mol}. (ký hiệu là $M$) của 1 chất là khối lượng tính bằng gam của $N$ nguyên tử hoặc phân tử chất đó. \fbox{\bf 3} \textit{Thể tích mol} của chất khí là thể tích chiếm bởi $N$ phân tử của chất khí đó. Ở điều kiện chuẩn (áp suất 1 bar, nhiệt độ $25^\circ$C), thể tích mol của các chất khí đều bằng $24.79$ lít. \fbox{\bf 4} Công thức chuyển đổi giữa số mol $n$ \& khối lượng $m$ chất: $n = \frac{m}{M}$ mol. \fbox{\bf 5} Công thức chuyển đổi giữa số mol $n$ \& thể tích $V$ của chất khí ở điều kiện chuẩn: $n = \frac{V}{24.79}$ mol. \fbox{\bf 6} Công thức tính tỷ khối của khí A đối với khí B: $d_{\rm A\texttt{/}B} = \frac{M_{\rm A}}{M_{\rm B}}$.
	}%
}

%------------------------------------------------------------------------------%

\section{Tính Theo Phương Trình Hóa Học}

\noindent\fbox{%
	\parbox{\textwidth}{%
		\noindent\textsf{\textbf{Kiến thức cốt lõi.}} \fbox{\bf 1} Các bước tính khối lượng \& số mol của chất tham gia, chất sản phẩm trong phản ứng hóa học: \textit{Bước 1}: Viết PTHH của phản ứng. \textit{Bước 2}: Tính số mol chất đã biết dựa vào khối lượng hoặc thể tích. \textit{Bước 3}: Dựa vào PTHH để tìm số mol chất tham gia hoặc chất sản phẩm. \textit{Bước 4}: Tính khối lượng hoặc thể tích của chất cần tìm. \fbox{\bf 2} \textit{Hiệu suất phản ứng} là tỷ lệ giữa lượng sản phẩm thu được theo thực tế \& lượng sản phẩm thu được theo lý thuyết.
	}%
}

%------------------------------------------------------------------------------%

\section{Nồng Độ Dung Dịch}
\textsf{\textbf{Nội dung.} Dung dịch là hỗn hợp lỏng đồng nhất của các chất đã tan trong nhau, độ tan của 1 chất trong nước, nồng độ \%, nồng độ mol, thí nghiệm pha 1 dung dịch theo 1 nồng độ cho trước.}

Các dung dịch thường có ghi kèm theo nồng độ xác định như nước muối sinh lý 0.9\%, sulfuric acid 1 mol\texttt{/}L, $\ldots$

Khi hòa chất rắn vào nước, có chất tan nhiều, có chất tan ít, có chất không tan trong nước. Làm thế nào để so sánh khả năng hòa tan trong nước của các chất \& xác định khối lượng chất tan có trong 1 dung dịch?

\subsection{Dung dịch, chất tan, \& dung môi}

\begin{dinhnghia}[Chất tan, dung dịch, dung môi]
	\emph{Dung môi} là chất có khả năng hòa tan chất khác để tạo thành dung dịch. \emph{Chất tan} là chất bị hòa tan trong dung môi. \emph{Dung dịch} là hỗn hợp (lỏng) đồng nhất của chất tan \& dung môi.
\end{dinhnghia}

\begin{thinghiem}[\cite{SGK_Hoa_Hoc_8}, Thí nghiệm 1, p. 135]
	Cho 1 thìa nhỏ đường vào cốc nước, khuấy nhẹ. Đường tan trong nước tạo thành nước đường. Nước đường là chất lỏng đồng nhất, không phân biệt được đâu là đường, đâu là nước. Ta nói: Đường là \emph{chất tan}, nước là \emph{dung môi} của đường, nước đường là \emph{dung dịch}.
\end{thinghiem}

\begin{thinghiem}[\cite{SGK_Hoa_Hoc_8}, Thí nghiệm 2, p. 135]
	Cho 1 thìa nhỏ dầu ăn hoặc mỡ ăn vào cốc thứ nhất đựng xăng hoặc dầu hỏa, vào cốc thứ 2 đựng nước, khuấy nhẹ. Xăng hòa tan được dầu ăn, tạo thành dung dịch. Nước không hòa tan được dầu ăn. Ta nói: Xăng là \emph{dung môi} của dầu ăn, nước không là dung môi của dầu ăn.
\end{thinghiem}

\begin{thinghiem}[\cite{SGK_KHTN_8_Canh_Dieu}, p. 36]
	Khi cho 1 thìa muối ăn vào cốc nước \& khuấy đều, ta được dung dịch muối ăn, trong đó các hạt muối ăn bị tan ra \& phân bố đều trong nước tạo thành hỗn hợp đồng nhất. Trong quá trình này, muối ăn là \emph{chất tan}, nước là \emph{dung môi}, \& nước muối là \emph{dung dịch}.
\end{thinghiem}

\subsubsection{Dung dịch chưa bão hòa. Dung dịch bão hòa}

\begin{thinghiem}[\cite{SGK_Hoa_Hoc_8}, Thí nghiệm, p. 136]
	Cho dần dần \& liên tục đường vào cốc nước, khuấy nhẹ. Ở giai đoạn đầu, ta được dung dịch đường, dung dịch này vẫn có thể hòa tan thêm đường. Ta có \emph{dung dịch đường chưa bão hòa}. Ở giai đoạn sau ta được 1 dung dịch đường không thể hòa tan thêm đường. Ta có \emph{dung dịch đường bão hòa}.
\end{thinghiem}

\begin{thinghiem}[\cite{SGK_KHTN_8_Canh_Dieu}, p. 36]
	Cho dần muối ăn vào cốc chứa \emph{200 mL} nước, khuấy đều cho đến khi muối ăn không thể hòa tan thêm được nữa, tách bỏ chất rắn không tan, ta thu được \emph{dung dịch bão hòa}.
\end{thinghiem}
Trong thực tế, dung môi thường là nước ở thể lỏng, chất tan có thể ở thể rắn, lỏng hoặc khí.

\begin{dinhnghia}[Dung dịch chưa bão hòa, dung dịch bão hòa]
	Ở nhiệt độ, áp suất nhất định, dung dịch có thể hòa tan thêm chất tan được gọi là \emph{dung dịch chưa bão hòa}, dung dịch không thể hòa tan thêm chất tan được gọi là \emph{dung dịch bão hòa}.
\end{dinhnghia}

\begin{nhanxet}
	Về định nghĩa dung dịch chưa bão hòa \& dung dịch bão hòa, chương trình Hóa Học 8 cũ (\cite{SGK_Hoa_Hoc_8}, Hóa Học 8) chỉ xét ở 1 nhiệt độ xác định, trong khi chương trình Hóa Học 8 cải cách: Khoa Học Tự nhiên 8 Cánh Diều (\cite{SGK_KHTN_8_Canh_Dieu}) \& Khoa Học Tự Nhiên 8 Kết Nối Tri Thức với Cuộc Sống (\cite{SGK_KHTN_8_KNTTVCS}) xét thêm yếu tố áp suất.
\end{nhanxet}

\begin{baitoan}[\cite{SGK_KHTN_8_KNTTVCS}, p. 20, Nhận biết dung dịch, chất tan, \& dung môi]
	\emph{Chuẩn bị:} nước, muối ăn, sữa bột (hoặc bột sắn, bột gạo, $\ldots$), copper(II) sulfate; cốc thủy tinh, đũa khuấy.
	
	\emph{Tiến hành:} Cho khoảng \emph{20 mL} nước vào 4 cốc thủy tinh, đánh số (1), (2), (3), \& (4). Cho vào cốc (1) 1 thìa (khoảng \emph{3 g}) muối ăn hạt, cốc (2) 1 thìa copper(II) sulfate, cốc (3) 1 thìa sữa bột, cốc (4) 4 thìa muối ăn. Khuấy đều $\approx2$ phút, sau đó để yên. (a) Trong các cốc (1), (2), (3), cốc nào chứa dung dịch? Dựa vào dấu hiệu nào để nhận biết? Chỉ ra chất tan, dung môi trong dung dịch thu được. (b) Phần dung dịch ở cốc (4) có phải là dung dịch bão hòa ở nhiệt độ phòng không? Giải thích.
\end{baitoan}

\begin{baitoan}[\cite{SGK_KHTN_8_KNTTVCS}, p. 20]
	Nêu cách pha dung dịch bão hòa của sodium carbonate \emph{\ce{Na2CO3}} trong nước.
\end{baitoan}

\subsection{Độ tan của 1 chất trong nước}
Trong cùng điều kiện về nhiệt độ \& áp suất, khả năng tan trong cùng 1 dung môi của các chất là khác nhau. Với cùng 1 lượng dung môi xác định, những chất tan tốt cần lượng lớn chất tan để tạo dung dịch bão hòa, còn những chất tan kém chỉ cần 1 lượng nhỏ chất tan đã thu được dung dịch bão hòa. Để đặc trưng cho khả năng tan của mỗi chất, người ta dùng khái niệm \textit{độ tan}.

\subsubsection{Chất tan \& chất không tan}

\begin{thinghiem}[\cite{SGK_Hoa_Hoc_8}, Thí nghiệm 1, p. 139]
	Lấy 1 lượng nhỏ calcium carbonate sạch \emph{\ce{CaCO3}} cho vào nước cất, lắc mạnh. Lọc lấy nước lọc. Nhỏ vài giọt nước lọc trên tấm kính sạch. Làm bay hơi nước từ từ cho đến hết. \emph{Quan sát:} Sau khi bay hơi nước, trên tấm kính không để lại dấu vết. \emph{Kết luận:} Calcium carbonate không tan trong nước.
\end{thinghiem}

\begin{thinghiem}[\cite{SGK_Hoa_Hoc_8}, Thí nghiệm 2, p. 139]
	Thay muối calcium carbonate bằng muối ăn sodium chloride \emph{NaCl} rồi làm thí nghiệm như trên. \emph{Quan sát:} Sau khi bay hết hơi nước, trên tấm kính có vết mờ. \emph{Kết luận:} Sodium chloride tan được trong nước.
\end{thinghiem}
Có \textit{chất không tan} \& có \textit{chất tan} trong nước. Có \textit{chất tan nhiều} \& có \textit{chất tan ít} trong nước.

\subsubsection{Tính tan trong nước của 1 số acid, base, muối}
\begin{itemize}
	\item \textit{Acid}: Hầu hết acid tan được trong nước, trừ acid silicic \ce{H2SiO3}.
	\item \textit{Base}: Phần lớn các base không tan trong nước, trừ 1 số như: KOH, NaOH, \ce{Ba(OH)2}, còn \ce{Ca(OH)2} ít tan.
	\item \textit{Muối}: (a) Những muối sodium \ce{Na_xX}, potassium \ce{K_xX} (với $x$ là hóa trị của phi kim X) đều tan. (b) Những muối nitrate \ce{X(NO3)_x} đều tan (với $x$ là hóa trị của kim loại X). (c) Phần lớn các muối chloride, sulfate tan được. Nhưng phần lớn muối carbonate không tan. 
\end{itemize}
Xem thêm bảng tính tan của acid, base, muối.

\subsubsection{Khái niệm độ tan của 1 chất trong nước}

\begin{dinhnghia}[Độ tan]
	\emph{Độ tan} (ký hiệu là $S$) của 1 chất trong nước là số gam chất đó hòa tan trong \emph{100 g} nước để tạo thành dung dịch bão hòa ở 1 nhiệt độ, áp suất xác định.
\end{dinhnghia}
Hiển nhiên, độ tan của chất không tan trong nước là $S = \frac{\rm0\ g}{\rm100\ g} = 0$. Các chất khác nhau có độ tan khác nhau.

\begin{vidu}[\cite{SGK_Hoa_Hoc_8}, p. 140]
	Ở $25^\circ$\emph{C}, độ tan của đường trong nước là \emph{204 g\texttt{/}100 g \ce{H2O}}, độ tan của \emph{NaCl} trong nước là \emph{36 g\texttt{/}100 g \ce{H2O}}, \& độ tan của \emph{\ce{AgNO3}} là \emph{222 g\texttt{/}100 g \ce{H2O}}.
\end{vidu}

\begin{vidu}[\cite{SGK_KHTN_8_Canh_Dieu}, p. 36]
	Lượng muối ăn hòa tan tối đa trong \emph{100 g} nước tạo thành dung dịch bão hòa ở $20^\circ$\emph{C} là \emph{35.9 g}. Người ta nói độ tan của muối ăn là \emph{35.9 g} trong \emph{100 g nước} ở $20^\circ$\emph{C}.
\end{vidu}

\begin{baitoan}[\cite{SGK_KHTN_8_Canh_Dieu}, 1., p. 36]
	Dung dịch bão hòa là gì?
\end{baitoan}

\begin{baitoan}[\cite{SGK_KHTN_8_Canh_Dieu}, 2., p. 36]
	Tính khối lượng sodium chloride cần hòa tan trong \emph{200 g} nước ở $20^\circ$\emph{C} để thu được dung dịch muối ăn bão hòa.
\end{baitoan}

\subsubsection{Những yếu tố ảnh hưởng đến độ tan}
(a) \textit{Độ tan của chất rắn} trong nước phụ thuộc vào \textit{nhiệt độ} $T$ (${}^\circ$). Trong nhiều trường hợp, khi tăng nhiệt độ thì độ tan của chất rắn cũng tăng theo. Số ít trường hợp, khi tăng nhiệt độ thì độ tan lại giảm.
\begin{figure}[H]
	\centering
	\includegraphics[scale=0.3]{nhiet_do_do_tan_chat_ran}
	\caption{Ảnh hưởng của nhiệt độ đến độ tan của chất rắn.}
\end{figure}
\noindent(b) \textit{Độ tan của chất khí} trong nước phụ thuộc vào \textit{nhiệt độ \& áp suất}. Độ tan của chất khí trong nước sẽ tăng, nếu ta giảm nhiệt độ $T$ \& tăng áp suất $p$.
\begin{figure}[H]
	\centering
	\includegraphics[scale=0.3]{nhiet_do_do_tan_chat_khi}
	\caption{Ảnh hưởng của nhiệt độ đến độ tan của chất khí.}
\end{figure}

\subsubsection{Cách tính độ tan của 1 chất trong nước}

\begin{baitoan}[\cite{SGK_KHTN_8_Canh_Dieu}, p. 37]
	Tính độ tan của muối potassium chloride \emph{KCl} ở $20^\circ$\emph{C}, biết $50$ gam nước hòa tan tối đa $17$ gam muối.
\end{baitoan}

\begin{proof}[Giải]
	Ở $20^\circ$\emph{C}, 50 g nước hòa tan tối đa 17 g KCl. Ở $20^\circ$\emph{C}, 100 g nước hòa tan tối đa S g KCl. $\Rightarrow S = \frac{17\cdot100}{50} = 17\cdot2 = 34$ g\texttt{/}100 g \ce{H2O}. Vậy độ tan của potassium chloride trong nước ở $20^\circ$\emph{C} là 34 g\texttt{/}100 g \ce{H2O}.
\end{proof}

\begin{baitoan}[\cite{SGK_KHTN_8_Canh_Dieu}, 1, p. 37]
	Tính độ tan của muối sodium nitrate \emph{\ce{NaNO3}} ở $0^\circ$\emph{C}, biết để tạo ra dung dịch \emph{\ce{NaNO3}} bão hòa người ta cần hòa tan $14.2$ g muối trong $20$ g nước.
\end{baitoan}
Công thức tính độ tan của 1 chất ở nhiệt độ xác định là \fbox{$S = \frac{m_{\rm ct}\cdot100}{m_{\tiny\mbox{nước}}}$ (g\texttt{/}100 g \ce{H2O})}, trong đó: $S$ là \textit{độ tan}, đơn vị g\texttt{/}100 g nước; $m_{\rm ct}$ là \textit{khối lượng của chất tan} được hòa tan trong nước để tạo thành dung dịch bão hòa, có đơn vị là gam; $m_{\tiny\mbox{nước}}$ là \textit{khối lượng của nước}, có đơn vị là gam.

\subsection{Làm thế nào để quá trình hòa tan chất rắn trong nước xảy ra nhanh hơn?}
Muốn quá trình hòa tan xảy ra nhanh hơn, ta thực hiện các biện pháp sau:
\begin{enumerate}
	\item \textit{Khuấy dung dịch}: Sự khuấy làm cho chất rắn bị hòa tan nhanh hơn, vì nó luôn luôn tạo ra sự tiếp xúc mới giữa chất rắn \& các phân tử nước.
	\item \textit{Đun nóng dung dịch}: Đun nóng dung dịch làm cho chất rắn bị hòa tan nhanh hơn. Vì ở nhiệt độ càng cao, các phân tử nước chuyển động càng nhanh, làm tăng dần số lần va chạm giữa các phân tử nước với bề mặt chất rắn.
	\item \textit{Nghiền nhỏ chất rắn}: Kích thước của chất rắn càng nhỏ thì chất rắn bị hòa tan càng nhanh, vì gia tăng diện tích tiếp xúc giữa chất rắn với các phân tử nước.
\end{enumerate}

\subsubsection{Ảnh hưởng của nhiệt độ đến độ tan của chất rắn trong nước}
Khi tăng nhiệt độ, độ tan của hầu hết các chất rắn như đường, muối ăn, $\ldots$ đều tăng. Có 1 số chất khi tăng nhiệt độ, độ tan lại giảm.

\begin{vidu}[Độ tan của \ce{C12H22O11}]
	Độ tan của đường ăn trong nước ở $30^\circ$\emph{C} là \emph{216.7 g} trong khi ở $60^\circ$\emph{C} là \emph{288.8 g}.
\end{vidu}

\begin{baitoan}[\cite{SGK_KHTN_8_Canh_Dieu}, 2, p. 37]
	(a) Có thể hòa tan tối đa bao nhiêu gam đường ăn trong \emph{250 g} nước ở $30^\circ$\emph{C}? (b) Có thể hòa tan tối đa bao nhiêu gam đường ăn trong \emph{250 g} nước ở $60^\circ$\emph{C}?
\end{baitoan}

\begin{baitoan}[\cite{SGK_KHTN_8_KNTTVCS}, 1, p. 21]
	Ở nhiệt độ $25^\circ$\emph{C}, khi cho \emph{12 g} muối X vào \emph{20 g} nước, khuấy kỹ thì còn lại \emph{5 g} muối không tan. Tính độ tan của muối X.
\end{baitoan}

\begin{baitoan}[\cite{SGK_KHTN_8_KNTTVCS}, 2, p. 21]
	Ở nhiệt độ $18^\circ$\emph{C}, khi hòa tan hết \emph{53 g} \emph{\ce{Na2CO3}} trong \emph{250 g} nước thì được dung dịch bão hòa. Tính độ tan của \emph{\ce{Na2CO3}} trong nước ở nhiệt độ trên.
\end{baitoan}

\begin{vidu}[\cite{SGK_KHTN_8_KNTTVCS}, p. 21]
	(a) Ngày nóng, cá thường ngoi lên phía mặt nước để hô hấp vì độ tan của oxygen trong nước đã bị giảm đi khi nhiệt độ tăng. (b) Trong sản xuất nước ngọt có gas, người ta nén khí carbon dioxide ở áp suất cao để tăng độ tan của khí này trong nước.
\end{vidu}
Nói chung, độ tan của hầu hết chất khí giảm khi nhiệt độ tăng hoặc áp suất giảm.

\subsection{Nồng độ dung dịch}
Để biểu thị lượng chất tan có trong 1 lượng dung môi hoặc lượng dung dịch cụ thể người ta dùng khái niệm \textit{nồng độ dung dịch} (hay: để định lượng 1 dung dịch đặc hay loãng, người ta dùng đại lượng \textit{nồng độ}). Có 2 loại nồng độ dung dịch thường được sử dụng là \textit{nồng độ \%} \& \textit{nồng độ mol}.

\subsubsection{Nồng độ \% của dung dịch}

\begin{dinhnghia}[Nồng độ \%]
	\emph{Nồng độ phần trăm} (ký hiệu là $C$\%) của 1 dung dịch là số gam chất tan có trong \emph{100 g} dung dịch.
\end{dinhnghia}
Công thức tính nồng độ \% của dung dịch là: \fbox{$C\% = \frac{m_{\rm ct}}{m_{\rm dd}}\cdot100$\%}, trong đó: $C$\% là \textit{nồng độ \%} của dung dịch, đơn vị \%; $m_{\rm ct}$ là \textit{khối lượng chất tan}, có đơn vị là gam; $m_{\rm dd}$ là \textit{khối lượng dung dịch}, có đơn vị là gam. Khối lượng dung dịch bằng tổng khối lượng chất tan \& khối lượng dung môi: \fbox{$m_{\rm dd} = m_{\rm ct} + m_{\rm dm}$} với $m_{\rm dm}$ là \textit{khối lượng dung môi}.
\begin{align*}
	\mbox{Khối lượng dung dịch} = \mbox{khối lượng chất tan} + \mbox{khối lượng dung môi}.
\end{align*}

\begin{baitoan}[\cite{SGK_Hoa_Hoc_8}, Thí dụ 1, p. 143]
	Hòa tan \emph{15 g NaCl} (natri clorua\emph{\texttt{/}}sodium chloride) vào \emph{45 g} nước. Tính nồng độ \% của dung dịch.
\end{baitoan}

\begin{proof}[1st giải]
	Khối lượng của dung dịch NaCl: $m_{\rm dd} = m_{\rm NaCl} + m_{\ce{H2O}} = 15 + 45 = 60$ g. Nồng độ \% của dung dịch NaCl: $C\% = \frac{m_{\rm NaCl}}{m_{\rm dd}}\cdot100\% = \frac{15}{60}\cdot100\% = 25\%$.
\end{proof}

\begin{proof}[2nd giải]
	Nồng độ \% của dung dịch NaCl: $C\% = \frac{m_{\rm NaCl}}{m_{\rm dd}}\cdot100\% = = \frac{m_{\rm NaCl}}{m_{\rm NaCl} + m_{\ce{H2O}}}\cdot100\% = \frac{15}{15 + 45}\cdot100\% = \frac{15}{60}\cdot100\% = 25\%$.
\end{proof}

\begin{baitoan}[\cite{SGK_KHTN_8_Canh_Dieu}, Ví dụ 1, p. 38]
	Hòa tan \emph{20 g} đường ăn trong \emph{60 g} nước thu được dung dịch đường. Tính $C$\emph{\%} của dung dịch đường đó.
\end{baitoan}

\begin{proof}[1st giải]
	Khối lượng dung dịch đường: $m_{\rm dd} = m_{\tiny\mbox{đường}} + m_{\tiny\mbox{nước}} = 20 + 60 = 80$ g. Nồng độ \% của dung dịch: $C\% = \frac{m_{\tiny\mbox{đường}}}{m_{\rm dd}}\cdot100\% = \frac{20}{80}\cdot100\% = 25\%$.
\end{proof}

\begin{proof}[2nd giải]
	Nồng độ \% của dung dịch: $C\% = \frac{m_{\tiny\mbox{đường}}}{m_{\rm dd}}\cdot100\% = \frac{m_{\tiny\mbox{đường}}}{m_{\tiny\mbox{đường}} + m_{\tiny\mbox{nước}}}\cdot100\% = \frac{20}{20 + 60} \cdot100\% = \frac{20}{80}\cdot100\% = 25\%$.
\end{proof}
Nếu biết được nồng độ \% của dung dịch thì ta có thể xác định được khối lượng chất tan \& khối lượng dung dịch theo các biểu thức sau:
\begin{align*}
	\boxed{m_{\rm ct} = m_{\rm dd}C\%,\ m_{\rm dd} = \frac{m_{\rm ct}}{C\%} = \frac{m_{\rm ct}}{C\%},\ m_{\rm dm} = m_{\rm dd} - m_{\rm ct} = \frac{m_{\rm ct}}{C\%} - m_{\rm dd}C\%.}
\end{align*}

\begin{baitoan}[\cite{SGK_Hoa_Hoc_8}, Thí dụ 2, p. 143]
	1 dung dịch \emph{\ce{H2SO4}} có nồng độ \emph{14\%}. Tính khối lượng \emph{\ce{H2SO4}} có trong \emph{150 g} dung dịch.
\end{baitoan}

\begin{proof}[Giải]
	Khối lượng \ce{H2SO4} có trong 150 g dung dịch 14\%: $m_{\ce{H2SO4}} = m_{\rm dd}C\% = 150\cdot14\% = \frac{150\cdot14}{100} = 21$ g.
\end{proof}

\begin{baitoan}[\cite{SGK_Hoa_Hoc_8}, Thí dụ 3, p. 144]
	Hòa tan \emph{50 g} đường vào nước, được dung dịch đường có nồng độ \emph{25\%}. Tính: (a) Khối lượng dung dịch đường pha chế được. (b) Khối lượng nước cần dùng cho sự pha chế.
\end{baitoan}

\begin{proof}[Giải]
	(a) Khối lượng dung dịch đường pha chế được: $m_{\rm dd} = m_{\rm dd} = \frac{m_{\tiny\mbox{đường}}}{C\%} = = \frac{50}{25\%} = \frac{50\cdot100}{25} = 200$ g. (b) Khối lượng nước cần dùng cho sự pha chế: $m_{\ce{H2O}} = m_{\rm dd} - m_{\tiny\mbox{đường}} = 200 - 50 = 150$ g.
\end{proof}

\begin{baitoan}[\cite{SGK_KHTN_8_Canh_Dieu}, Ví dụ 2, p. 38]
	Muốn pha \emph{300 g} dung dịch muối \emph{\ce{CuSO4} 10\%} cần dùng bao nhiêu \emph{g} muối \& bao nhiêu \emph{g} nước?
\end{baitoan}

\begin{proof}[Giải]
	Khối lượng chất tan cần dùng là: $m_{\tiny\mbox{muối}} = m_{\rm dd}\cdot C\% = 300\cdot10\% = 30$ g. Khối lượng nước cần dùng là: $m_{\tiny\mbox{nước}} = m_{\rm dd} - m_{\tiny\mbox{muối}} = 300 - 30 = 270$ g.
\end{proof}

\begin{baitoan}[\cite{SGK_KHTN_8_Canh_Dieu}, 1, p. 38]
	Dung dịch D-glucose $5$\% được sử dụng trong y tế làm dịch truyền, nhằm cung cấp nước \& năng lượng cho bệnh nhân bị suy nhược cơ thể hoặc sau phẫu thuật. Biết trong 1 chai dịch truyền có chứa \emph{25 g} đường D-glucose. Tính lượng dung dịch \& lượng nước có trong chai dịch truyền đó.
\end{baitoan}

\begin{baitoan}[\cite{SGK_KHTN_8_Canh_Dieu}, 2, p. 38]
	Từ muối ăn, nước, \& những dụng cụ cần thiết, nêu cách pha \emph{500 g} dung dịch nước muối $0.9$\%.
\end{baitoan}

\begin{thinghiem}[Pha chế 100 g dung dịch đường ăn (saccharose) \ce{C12H22O11} 15\%]
	 \emph{Chuẩn bị:} Dụng cụ: Cân điện tử, cốc thủy tinh (loại \emph{250 mL}), đũa thủy tinh. Hóa chất: Đường ăn, nước cất.
	 
	 \emph{Tiến hành:} Bước 1: Cân chính xác \emph{15 g} đường ăn cho vào cốc dung tích \emph{250 mL}. Bước 2: Cân lấy \emph{85 g} nước cất, rồi cho dần vào cốc \& khuấy nhẹ cho tới khi đường tan hết, thu được \emph{100 g} dung dịch đường nồng độ $15$\%.
\end{thinghiem}

\begin{baitoan}[\cite{SGK_KHTN_8_KNTTVCS}, p. 21]
	Dung dịch nước oxy già chứa chất tan hydrogen peroxide \emph{\ce{H2O2}}. Tính khối lượng hydrogen peroxide có trong \emph{20 g} dung dịch nước oxy già $3$\%.
\end{baitoan}

\begin{proof}[Giải]
	Khối lượng hydrogen peroxide có trong 20 g dung dịch nước oxy già 3\%: $m_{\ce{H2O2}} = m_{\rm dd}\cdot C\% = 20\cdot3\% = \frac{20\cdot3}{100} = 0.6$ g.
\end{proof}

\subsubsection{Nồng độ mol của dung dịch}

\begin{dinhnghia}[Nồng độ mol]
	\emph{Nồng độ mol} (ký hiệu là $C_{\rm M}$) của 1 dung dịch là số mol chất tan có trong \emph{1 L} dung dịch. Đơn vị của nồng độ mol là \emph{mol\texttt{/}L} \& thường được ký hiệu là $M$.
\end{dinhnghia}
Công thức tính nồng độ mol của dung dịch: \fbox{$C_{\rm M} = \frac{n}{V}$}, trong đó: $C_{\rm M}$ là \textit{nồng độ mol} của dung dịch, có đơn vị là mol\texttt{/}L \& thường được biễu diễn là M; $n$ là \textit{số mol chất tan}, có đơn vị là mol, $V$ là \textit{thể tích dung dịch}, có đơn vị là lít (L).

\begin{baitoan}[\cite{SGK_Hoa_Hoc_8}, Thí dụ 1, p. 144]
	Trong \emph{200 mL} dung dịch có hòa tan \emph{16 g \ce{CuSO4}}. Tính nồng độ mol của dung dịch.
\end{baitoan}

\begin{proof}[Giải]
	Số mol \ce{CuSO4} có trong dung dịch: $n_{\ce{CuSO4}} = \frac{m_{\ce{CuSO4}}}{M_{\ce{CuSO4}}} = \frac{16}{160} = 0.1$ mol. Nồng độ mol của dung dịch \ce{CuSO4}: $C_{\rm M} = \frac{n_{\ce{CuSO4}}}{V} = \frac{0.1}{0.2} = 0.5$ mol\texttt{/}L $= 0.5$M.
\end{proof}

\begin{baitoan}[\cite{SGK_Hoa_Hoc_8}, Thí dụ 2, p. 144]
	Trộn \emph{2 L} dung dịch đường \emph{0.5M} với \emph{3 L} dung dịch đường \emph{1M}. Tính nồng độ mol của dung dịch đường sau khi trộn.
\end{baitoan}

\begin{proof}[Giải]
	Số mol đường có trong dung dịch 1: $n_1 = C_{\rm M1}V_1 = 0.5\cdot2 = 1$ mol. Số mol đường có trong dung dịch 2: $n_2 = C_{\rm M2}V_2 = 1\cdot3 = 3$ mol. Thể tích của dung dịch đường sau khi trộn: $V = V_1 + V_2 = 2 + 3 = 5$ L. Nồng độ mol của dung dịch đường sau khi trộn: $C_{\rm M} = \frac{n}{V} = \frac{n_1 + n_2}{V} = \frac{3 + 1}{5} = \frac{4}{5} = 0.8$M.
\end{proof}

\begin{baitoan}[\cite{SGK_KHTN_8_Canh_Dieu}, Ví dụ 3, p. 39]
	Hòa tan hoàn toàn \emph{4.2 g} sodium hydrogencarbonate \emph{\ce{NaHCO3}} trong nước thu được \emph{500 mL} dung dịch. Tính nồng độ của dung dịch này.
\end{baitoan}

\begin{proof}[Giải]
	Số mol của \ce{NaHCO3} có trong dung dịch là: $n_{\rm NaHCO_3} = \frac{4.2}{84} = 0.05$ mol. Nồng độ mol của dung dịch \ce{NaHCO3} là: $C_{\rm M} = \frac{0.05}{0.5} = 0.1$M.
\end{proof}
Nếu biết được nồng độ mol của dung dịch ta có thể xác định được số mol chất tan \& thể tích dung dịch theo các biểu thức sau:
\begin{align*}
	\boxed{n =C_{\rm M}V,\ V = \frac{n}{C_{\rm M}}.}
\end{align*}

\begin{baitoan}[\cite{SGK_KHTN_8_Canh_Dieu}, 3, p. 39]
	Tính số \emph{g} chất tan cần pha để pha chế \emph{100 mL} dung dịch \emph{\ce{CuSO4} 0.1 M}.
\end{baitoan}
Có nhiều cách khác nhau để biểu thị nồng độ dung dịch. Để thuận tiện cho việc nghiên cứu, ngoài việc sử dụng nồng độ \% \& nồng độ mol, các nhà khoa học còn sử dụng thêm các loại nồng độ khác như \textit{nồng độ đương lượng} \& \textit{nồng độ molan}.

\begin{thinghiem}[Pha chế dung dịch sodium bicarbonate 0.2 M]
	Sodium bicarbonate (hay còn gọi là sodium hydrogencarbonate, \ce{NaHCO3}) là thành phần chính của thuốc muối được sử dụng nhiều trong chế biến thực phẩm, y tế, vệ sinh vật dụng trong gia đình, $\ldots$ Để pha chế 100 mL dung dịch sodium bicarbonate 0.2 M có thể thực hiện theo thí nghiệm sau:
	
	\emph{Chuẩn bị:} Dụng cụ: Cân điện tử, phễu thủy tinh, ống đong, bình tam giác (loại \emph{250 mL}). Hóa chất: \emph{\ce{NaHCO3}}, nước cất.
	
	\emph{Tiến hành:} Bước 1: Cân chính xác \emph{1.68 g} muối \emph{\ce{NaHCO3}} cho vào bình tam giác. Bước 2: Thêm \emph{100 mL} nước cất vào bình tam giác, khuấy đều cho muối tan hết, thu được dung dịch \emph{\ce{NaHCO3} 0.2 M} (1 cách gần đúng có thể coi thể tích dung dịch muối \ce{NaHCO3} là \emph{100 mL}).
\end{thinghiem}

\begin{baitoan}[\cite{SGK_KHTN_8_Canh_Dieu}, p. 40]
	Glucose được tạo ra từ các quá trình chuyển hóa thực phẩm \& là 1 trong các nguồn cung cấp năng lượng chính cho cơ thể chúng ta. Với người bình thường, nồng độ glucose trong máu luôn được duy trì ổn định. Tìm hiểu \& cho biết chỉ số nồng độ glucose trong máu của người bình thường nằm trong khoảng nào. Nếu chỉ số nồng độ glucose trong máu của 1 người lớn hơn mức bình thường thì người đó có nguy cơ mắc bệnh gì?
\end{baitoan}

\begin{baitoan}[\cite{SGK_KHTN_8_KNTTVCS}, p. 22]
	Hòa tan hoàn toàn \emph{1.35 g} copper(II) chloride vào nước, thu được \emph{50 mL} dung dịch. Tính nồng độ mol của dung dịch copper(II) chloride thu được.
\end{baitoan}

\begin{proof}[Giải]
	Số mol chất tan: $n_{\ce{CuCl2}} = \frac{m_{\ce{CuCl2}}}{M_{\ce{CuCl2}}} = \frac{1.35}{135} = 0.01$ mol. Đổi đơn vị: 50 mL $= 0.05$ L. Nồng độ mol dung dịch copper(II) chloride là: $C_{\rm M(\ce{CuCl2})} = \frac{n_{\ce{CuCl2}}}{V} = \frac{0.01}{0.05} = 0.2$ mol\texttt{/}L.
\end{proof}

\begin{baitoan}[\cite{SGK_KHTN_8_KNTTVCS}, 1, p. 22]
	Tính khối lượng \emph{\ce{H2SO4}} có trong \emph{20 g} dung dịch \emph{\ce{H2SO4} 98\%}.
\end{baitoan}

\begin{baitoan}[\cite{SGK_KHTN_8_KNTTVCS}, 2, p. 22]
	Trộn lẫn \emph{2 L} dung dịch urea \emph{0.02 M} (dung dịch A) với \emph{3 L} dung dịch urea \emph{0.1 M} (dung dịch B), thu được \emph{5 L} dung dịch C. (a) Tính số mol urea trong dung dịch A, B, \& C. (b) Tính nồng độ mol của dung dịch C. Nhận xét về giá trị nồng độ mol của dung dịch C so với nồng độ mol của dung dịch A, B.
\end{baitoan}

\subsection{Thực hành pha chế dung dịch theo 1 nồng độ cho trước}

\begin{thinghiem}[\cite{SGK_KHTN_8_KNTTVCS}, p. 23, Pha 100 g dung dịch muối ăn nồng độ 0.9\%]
	\emph{Chuẩn bị:} muối ăn khan, nước cất; cốc thủy tinh, cân, ống đong.
	
	\emph{Tiến hành:} Xác định khối lượng muối ăn $m_1$ \& nước $m_2$ dựa vào công thức: $C\% = \frac{m_{\rm ct}}{m_{\rm dd}}\cdot100\%$. Cân $m_1$ g muối ăn rồi cho vào cốc thủy tinh. Cân $m_2$ g nước cất, rót vào cốc, lắc đều cho muối tan hết.
	
	(a) Tại sao phải dùng muối ăn khan để pha dung dịch? (b) Dung dịch muối ăn nồng độ \emph{0.9\%} có thể được dùng để làm gì?
\end{thinghiem}

\begin{vidu}[Oresol]
	\emph{Oresol} (abbr., Oral Rehydration Solution) là 1 loại dung dịch có tác dụng bù nước \& điện giải. Trong Oresol có 1 số thành phần chính là: sodium chloride, sodium bicarbonate, potassium chloride, glucose. Nồng độ các chất trong 1 loại dung dịch Oresol được WHO \& UNICEF khuyên dùng là \emph{ion sodium: 0.075 mol\texttt{/}L, ion chloride: 0.065 mol\texttt{/}L, ion potassium: 0.020 mol\texttt{/}L, ion citrate: 0.010 mol\texttt{/}L, glucose: 0.075 mol\texttt{/}L, $\ldots$}
\end{vidu}
Có thể pha chế: (a) 1 dung dịch có nồng độ xác định để làm thí nghiệm. (b)  Pha chế dung dịch nước muối 0.9\% (có thể dùng thay nước muối sinh lý trong 1 số trường hợp).

\subsubsection{Cách pha chế 1 dung dịch theo nồng độ cho trước}

\begin{baitoan}[\cite{SGK_Hoa_Hoc_8}, Bài tập 1, p. 147]
	Từ muối \emph{\ce{CuSO4}}, nước cất, \& những dụng cụ cần thiết, tính toán \& giới thiệu cách pha chế: (a) \emph{50 g} dung dịch \emph{\ce{CuSO4}} có nồng độ \emph{10\%}. (b) \emph{50 mL} dung dịch \emph{\ce{CuSO4}} có nồng độ \emph{1M}.
\end{baitoan}

\begin{proof}[Giải]
	(a) \textit{Tính toán}: Khối lượng chất tan: $m_{\ce{CuSO4}} = m_{\rm dd}C\% = 50\cdot10\% = \frac{50\cdot10}{100} = 5$ g. Khối lượng dung môi (nước): $m_{\rm dm} = m_{\rm dd} - m_{\ce{CuSO4}} = 50 - 5 = 45$ g. \textit{Cách pha chế}: Cân lấy 5 g \ce{CuSO4} khan (màu trắng) cho vào cốc có dung tích 100 mL. Cân lấy 45 g (hoặc đong lấy 45 mL\footnote{Vì khối lượng riêng của nước cất: $D_{\ce{H2O}} = \frac{m_{\ce{H2O}}}{V_{\ce{H2O}}}\approx997$kg\texttt{/}$\rm m^3$.}) nước cất, rồi đổ dần dần vào cốc \& khuấy nhẹ. Được 50 g dung dịch \ce{CuSO4} 10\%. (b) Số mol chất tan: $n_{\ce{CuSO4}} = C_{\rm M}V = \frac{1\cdot50}{1000} = 0.05$ mol. Khối lượng của 0.05 mol \ce{CuSO4}: $m_{\ce{CuSO4}} = n_{\ce{CuSO4}}M_{\ce{CuSO4}} = 0.05\cdot160 = 8$ g. \textit{Cách pha chế}: Cân lấy 8 g \ce{CuSO4} cho vào cốc thủy tinh có dung tích 100 mL. Đổ dần dần nước cất vào cốc \& khuấy nhẹ cho đủ 50 mL dung dịch. Ta được 50 mL dung dịch \ce{CuSO4} 1M.
\end{proof}

\subsubsection{Cách pha loãng 1 dung dịch theo nồng độ cho trước}

\begin{baitoan}[\cite{SGK_Hoa_Hoc_8}, Bài tập 2, p. 148]
	Có nước cất \& những dụng cụ cần thiết, tính toán \& giới thiệu các cách pha chế: (a) \emph{100 mL} dung dịch \emph{\ce{MgSO4} 0.4M} từ dung dịch \emph{\ce{MgSO4} 2M}. (b) \emph{150 g} dung dịch \emph{NaCl 2.5\%} từ dung dịch \emph{NaCl 10\%}.
\end{baitoan}

\begin{proof}[Giải]
	(a) \textit{Tính toán}: Số mol chất tan có trong 100 mL dung dịch \ce{MgSO4} 0.4M: $n_{\ce{MgSO4}} = C_{\rm M}V = \frac{0.4\cdot100}{1000} = 0.04$ mol.  Thể tích dung dịch \ce{MgSO4} 2M trong đó có chứa 0.04 mol \ce{MgSO4}: $V = \frac{n_{\ce{MgSO4}}}{C_{\rm M}} = \frac{0.04}{2} = 0.02$ L $= 20$ mL. \textit{Cách pha chế}: Đong lấy 20 mL dung dịch \ce{MgSO4} 2M cho vào cốc chia độ có dung tích 200 mL. Thêm từ từ nước cất vào cốc đến vạch 100 mL \& khuấy đều, ta được 100 mL dung dịch \ce{MgSO4} 0.4M. (b) \textit{Tính toán}: Khối lượng NaCl có trong 150 g dung dịch NaCl 2.5\%: $m_{\rm NaCl} = m_{\rm dd}C\% = 150\cdot2.5\% = \frac{150\cdot2.5}{100} = 3.75$ g. Khối lượng dung dịch NaCl ban đầu có chứa 3.75 g NaCl: $m_{\rm dd1} = \frac{m_{\rm NaCl}}{C\%} = \frac{3.75\cdot100}{10} = 37.5$ g. Khối lượng nước cần dùng để pha chế: $m_{\ce{H2O}} = m_{\rm dd2} - m_{\rm dd1} = 150 - 37.5 = 112.5$ g. \textit{Cách pha chế}: Cân lấy 37.5 g dung dịch NaCl 10\% ban đầu, sau đó đổ vào cốc hoặc bình tam giác có dung tích vào khoảng 200 mL. Cân lấy 112.5 g nước cất hoặc đong 112.5 mL nước cất, sau đó đổ vào cốc đựng dung dịch NaCl nói trên. Khuấy đều, ta được 150 g dung dịch NaCl 2.5\%.
\end{proof}

\noindent\fbox{%
	\parbox{\textwidth}{%
		\noindent\textsf{\textbf{Kiến thức cốt lõi.}} \fbox{\bf 1} \textit{Dung dịch} là hỗn hợp lỏng đồng nhất của chất tan \& dung môi. \fbox{\bf 2} Ở nhiệt độ \& áp suất nhất định: \textit{Dung dịch chưa bão hòa} là dung dịch có thể hòa tan thêm chất tan. \textit{Dung dịch bão hòa} là dung dịch không thể hòa tan thêm chất tan. \fbox{\bf 3} \textit{Độ tan} (ký hiệu là $S$) của 1 chất trong nước là số gam chất đó hòa tan trong $100$ gam nước để tạo thành dung dịch bão hòa ở 1 nhiệt độ, áp suất xác định: $S = \frac{m_{\rm ct}}{m_{\tiny\mbox{nước}}}\cdot100$ ($S$: độ tan, $m_{\rm ct}$: khối lượng chất tan, $m_{\tiny\mbox{nước}}$: khối lượng nước). Nói chung, độ tan của chất rắn sẽ tăng nếu tăng nhiệt độ. Độ tan của chất khí sẽ tăng nếu giảm nhiệt độ \& tăng áp suất. \fbox{\bf 4} \textit{Nồng độ phần \%} (ký hiệu là $C$\%) của 1 dung dịch là số gam chất tan có trong $100$ gam dung dịch: $C\% = \frac{m_{\rm ct}\cdot100\%}{m_{\rm dd}}$ ($C$\%: nồng độ \%, $m_{\rm ct}$: khối lượng chất tan, $m_{\rm dd}$: khối lượng dung dịch). \fbox{\bf 5} \textit{Nồng độ mol} (ký hiệu là $C_{\rm M}$) của 1 dung dịch là số mol chất tan có trong 1 lít dung dịch. $C_{\rm M} = \frac{n_{\rm ct}}{V_{\rm dd}}$ mol\texttt{/}L ($C_{\rm M}$: nồng độ mol, $n_{\rm ct}$: số mol chất tan, $V_{\rm dd}$: thể tích dung dịch). \fbox{\bf 6} Muốn chất rắn tan nhanh trong nước, ta thực hiện 1, 2, hoặc cả 3 biện pháp: Khuấy dung dịch, đun nóng dung dịch, nghiền nhỏ chất rắn. 
	}%
}

\newpage
\subsection{Problems}

\subsubsection{Dung dịch}

\begin{baitoan}[\cite{SGK_Hoa_Hoc_8}, 1., p. 138]
	Thế nào là dung dịch, dung dịch chưa bão hòa, dung dịch bão hòa? Cho ví dụ.
\end{baitoan}

\begin{baitoan}[\cite{SGK_Hoa_Hoc_8}, 2., p. 138]
	Mô tả những thí nghiệm chứng minh rằng muốn hòa tan nhanh 1 chất rắn trong nước ta có thể chọn những biện pháp: nghiền nhỏ chất rắn, đun nóng, khuấy dung dịch.
\end{baitoan}	

\begin{baitoan}[\cite{SGK_Hoa_Hoc_8}, 3., p. 138]
	Mô tả cách tiến hành những thí nghiệm sau: (a) Chuyển đổi từ 1 dung dịch \emph{NaCl} bão hòa thành 1 dung dịch chưa bão hòa (ở nhiệt độ phòng). (b) Chuyển đổi từ 1 dung dịch \emph{NaCl} chưa bão hòa thành 1 dung dịch bão hòa (ở nhiệt độ phòng).
\end{baitoan}

\begin{baitoan}[\cite{SGK_Hoa_Hoc_8}, 4., p. 138]
	Cho biết ở nhiệt độ phòng thí nghiệm ($\approx20^\circ$ C), \emph{10 g} nước có thể hòa tan tối đa \emph{20 g} đường; \emph{3.6 g} muối ăn. (a) Dẫn ra những ví dụ về khối lượng của đường, muối ăn để tạo ra những dung dịch chưa bão hòa với \emph{10 g} nước. (b) Có nhận xét gì nếu người ta khuấy \emph{25 g} đường vào \emph{10 g} nước; \emph{3.5 g} muối ăn vào \emph{10 g} nước (nhiệt độ phòng thí nghiệm)?
\end{baitoan}

\begin{baitoan}[\cite{SGK_Hoa_Hoc_8}, 5., p. 138]
	Trộn \emph{1 ml} rượu etylic (cồn) với \emph{10 ml} nước cất. Câu nào diễn đạt đúng? {\sf A.} Chất tan là rượu etylic, dung môi là nước. {\sf B.} Chất tan là nước, dung môi là rượu etylic. {\sf C.} Nước hoặc rượu etylic có thể là chất tan hoặc là dung môi. {\sf D.} Cả 2 chất nước \& rượu etylic vừa là chất tan, vừa là dung môi.
\end{baitoan}

\begin{baitoan}[\cite{SGK_Hoa_Hoc_8}, 6., p. 138]
	\emph{Đ\texttt{/}S?} Dung dịch là hỗn hợp: {\sf A.} Của chất rắn trong chất lỏng. {\sf B.} Của chất khí trong chất lỏng. {\sf C.} Đồng nhất của chất rắn \& dung môi. {\sf D.} Đồng nhất của dung môi \& chất tan.
\end{baitoan}

\begin{baitoan}[\cite{SBT_Hoa_Hoc_8}, 40.1, p. 56]
	Trong phòng thí nghiệm có sẵn 1 dung dịch \emph{NaCl}. Bằng phương pháp thực nghiệm, xác định dung dịch \emph{NaCl} này là bão hòa hay chưa bão hòa. Trình bày cách làm.
\end{baitoan}

\begin{proof}[Giải]
	Lấy khoảng 50 mL dung dịch NaCl cho vào bình tam giác. Cân 1 lượng muối tinh khiết NaCl (e.g., 1 g NaCl) cho vào bình đựng dung dịch NaCl, lắc kỹ 1 thời gian. Nếu: (a) Có hiện tượng 1 phần hoặc toàn lượng NaCl bị hòa tan, ta kết luận dung dịch NaCl ban đầu là \textit{chưa bão hòa} ở nhiệt độ phòng. Không xảy ra hiện tượng gì (lượng NaCl thêm vào bình không bị hòa tan), ta kết luận dung dịch NaCl ban đầu là \textit{bão hòa} ở nhiệt độ phòng.
\end{proof}

\subsubsection{Độ tan của 1 chất trong nước}

\begin{baitoan}[\cite{SGK_Hoa_Hoc_8}, 1., p. 142]
	\emph{Đ\texttt{/}S?} Độ tan của 1 chất trong nước ở nhiệt độ xác định là: {\sf A.} Số gam chất đó có thể tan trong \emph{100 g} dung dịch. {\sf B.} Số gam chất đó có thể tan trong \emph{100 g} nước. {\sf C.} Số gam chất đó có thể tan trong \emph{100 g} dung môi để tạo thành dung dịch bão hòa. {\sf D.} Số gam chất đó có thể tan trong \emph{100 g} nước để tạo thành dung dịch bão hòa.
\end{baitoan}

\begin{baitoan}[\cite{SGK_Hoa_Hoc_8}, 2., p. 142]
	Khi tăng nhiệt độ thì độ tan của các chất rắn trong nước: {\sf A.} Đều tăng. {\sf B.} Đều giảm. {\sf C.} Phần lớn là tăng. {\sf D.} Phần lớn là giảm. {\sf E.} Không tăng \& cũng không giảm.
\end{baitoan}

\begin{baitoan}[\cite{SGK_Hoa_Hoc_8}, 3., p. 142]
	Khi giảm nhiệt độ \& tăng áp suất thì độ tan của chất khí trong nước: {\sf A.} Đều tăng. {\sf B.} Đều giảm. {\sf C.} Có thể tăng \& có thể giảm. {\sf D.} Không tăng \& cũng không giảm.
\end{baitoan}

\begin{baitoan}[\cite{SGK_Hoa_Hoc_8}, 4., p. 142]
	Dựa vào đồ thị về độ tan của các chất rắn trong nước \ref{fig:nhiet_do_do_tan_chat_ran}, cho biết độ tan của các muối \emph{\ce{NaNO3, KBr, KNO3, NH4Cl, NaCl, Na2SO4}} ở nhiệt độ $10^\circ$\emph{C} \& $60^\circ$\emph{C}.
\end{baitoan}

\begin{baitoan}[\cite{SGK_Hoa_Hoc_8}, 5., p. 142]
	Xác định độ tan của muối \emph{\ce{Na2CO3}} trong nước ở $18^\circ$\emph{C}. Biết ở nhiệt độ này khi hòa tan hết \emph{53 g \ce{Na2CO3}} trong \emph{250 g} nước thì được dung dịch bão hòa.
\end{baitoan}

\begin{baitoan}[\cite{SBT_Hoa_Hoc_8}, 41.1, p. 56]
	Căn cứ vào đồ thị sau về độ tan của chất rắn trong nước, ước tính độ tan của các muối \emph{\ce{NaNO3, KBr, KNO3, NH4Cl, NaCl, Na2SO4}} ở nhiệt độ: (a) $20^\circ$\emph{C}. (b) $40^\circ$\emph{C}.
	\begin{figure}[H]
		\centering
		\includegraphics[scale=0.3]{nhiet_do_do_tan_chat_ran}
		\caption{Ảnh hưởng của nhiệt độ đến độ tan của chất rắn.}
		\label{fig:nhiet_do_do_tan_chat_ran}
	\end{figure}
\end{baitoan}

\begin{baitoan}[\cite{SBT_Hoa_Hoc_8}, 41.2, p. 56]
	Căn cứ vào đồ thị về độ tan của chất khí trong nước, ước lượng độ tan của các khí \emph{NO, \ce{O2, N2}} ở $20^\circ$\emph{C}. Cho biết có bao nhiêu milliliter những khí trên tan trong \emph{1 L} nước. Biết ở $20^\circ$\emph{C} \& \emph{1 atm, 1 mol} chất khí có thể tích là \emph{24 L} \& khối lượng riêng của nước là \emph{1 g\texttt{/}mL}.	
	\begin{figure}[H]
		\centering
		\includegraphics[scale=0.3]{nhiet_do_do_tan_chat_khi}
		\caption{Ảnh hưởng của nhiệt độ đến độ tan của chất khí.}
		\label{fig:nhiet_do_do_tan_chat_khi}
	\end{figure}
\end{baitoan}

\begin{baitoan}[\cite{SBT_Hoa_Hoc_8}, 41.3, p. 56]
	Tính khối lượng muối natri clorua \emph{NaCl} có thể tan trong \emph{750 g} nước ở $25^\circ$\emph{C}. Biết ở nhiệt độ này độ tan của \emph{NaCl} là \emph{36.2 g}.
\end{baitoan}

\begin{baitoan}[\cite{SBT_Hoa_Hoc_8}, 41.4, p. 56]
	Tính khối lượng muối \emph{\ce{AgNO3}} có thể tan trong \emph{250 g} nước ở $25^\circ$\emph{C}. Biết độ tan của \emph{\ce{AgNO3}} ở $25^\circ$\emph{C} là \emph{222 g}.
\end{baitoan}

\begin{baitoan}[\cite{SBT_Hoa_Hoc_8}, $41.5^\star$, p. 56]
	Biết độ tan của muối \emph{KCl} ở 	$20^\circ$\emph{C} là \emph{34 g}. 1 dung dịch \emph{KCl} nóng có chứa \emph{50 g KCl} trong \emph{130 g \ce{H2O}} được làm lạnh về nhiệt độ $20^\circ$\emph{C}. Cho biết: (a) Có bao nhiêu \emph{g KCl} tan trong dung dịch. (b) Có bao nhiêu \emph{g KCl} tách ra khỏi dung dịch.
\end{baitoan}

\begin{baitoan}[\cite{SBT_Hoa_Hoc_8}, 41.6, p. 57]
	1 dung dịch có chứa \emph{26.5 g NaCl} trong \emph{75 g \ce{H2O}} ở $25^\circ$\emph{C}. Xác định dung dịch \emph{NaCl} nói trên là chưa bão hòa hay bão hòa. Biết độ tan của \emph{NaCl} trong nước ở $25^\circ$\emph{C} là \emph{36 g}.
\end{baitoan}

\begin{baitoan}[\cite{SBT_Hoa_Hoc_8}, 41.7, p. 57]
	Có bao nhiêu \emph{g \ce{NaNO3}} sẽ tách ra khỏi \emph{200 g} dung dịch bão hòa \emph{\ce{NaNO3}} ở $50^\circ$\emph{C}, nếu dung dịch này được làm lạnh đến $20^\circ$\emph{C}? Biết \emph{$S_{\ce{NaNO3}(50^\circ C)} = 114$ g, $S_{\ce{NaNO3}(20^\circ C)} = 88$ g}.
\end{baitoan}

\begin{baitoan}[\cite{An_350_BT_Hoa_Hoc_8}, 322., p. 109]
	(a) Khi hòa tan rượu vào nước \& khi hòa tan sữa vào nước, trường hợp nào tạo ra dung dịch? Căn cứ vào đặc tính quan trọng nào để nhận ra dung dịch? (b) Ở $20^\circ$\emph{C 500 mL \ce{H2O}} hòa tan tối đa \emph{0.02 g \ce{O2}}. Ở $20^\circ$\emph{C 250 mL \ce{H2O}} hòa tan tối đa \emph{0.0045 g \ce{N2}}. Hỏi độ tan của \emph{\ce{O2, N2}} trong nước ở $20^\circ$\emph{C}? Biết $D_{\ce{H2O}} = 1$ \emph{g\texttt{/}mL}.
\end{baitoan}

\begin{baitoan}[\cite{An_350_BT_Hoa_Hoc_8}, 323., p. 109]
	(a) Ở $25^\circ$\emph{C} độ tan của \emph{NaCl} là \emph{36 g}. Hỏi phải hòa tan bao nhiêu \emph{g NaCl} vào \emph{150 mL} nước để được dung dịch bão hòa ở nhiệt độ đó. (b) Ở nhiệt độ $20^\circ$\emph{C} độ tan của \emph{KCl} là \emph{34 g}. Muốn có \emph{330 g} dung dịch \emph{KCl} bão hòa ở nhiệt độ $20^\circ$\emph{C} thì cần bao nhiêu \emph{g KCl}? Cần bao nhiêu \emph{g} nước?
\end{baitoan}

\begin{baitoan}[\cite{An_350_BT_Hoa_Hoc_8}, 323., p. 109]
	
\end{baitoan}

\subsubsection{Nồng độ dung dịch}

\begin{baitoan}[\cite{SGK_Hoa_Hoc_8}, 1., p. 145]
	Bằng cách nào có được \emph{200 g} dung dịch \emph{\ce{BaCl2} 5\%}? {\sf A.} Hòa tan \emph{190 g \ce{BaCl2}} trong \emph{10 g} nước. {\sf B.} Hòa tan \emph{10 g \ce{BaCl2}} trong \emph{190 g} nước. {\sf C.} Hòa tan \emph{100 g \ce{BaCl2}} trong \emph{100 g} nước. {\sf D.} Hòa tan \emph{200 g \ce{BaCl2}} trong \emph{10 g} nước. {\sf E.} Hòa tan \emph{10 g \ce{BaCl2}} trong \emph{200 g} nước.
\end{baitoan}

\begin{baitoan}[\cite{SGK_Hoa_Hoc_8}, 2., p. 145]
	Tính nồng độ mol của \emph{850 mL} dung dịch có hòa tan \emph{20 g \ce{KNO3}}. {\sf A.} \emph{0.233M}. {\sf B.} \emph{23.3M}. {\sf C.} \emph{2.33M}. {\sf D.} \emph{233M}.
\end{baitoan}

\begin{baitoan}[\cite{SGK_Hoa_Hoc_8}, 3., p. 146]
	Tính nồng độ mol của mỗi dung dịch sau: (a) \emph{1 mol KCl} trong \emph{750 ml} dung dịch. (b) \emph{0.5 mol \ce{MgCl2}} trong \emph{1.5 L} dung dịch. (c) \emph{400 g \ce{CuSO4}} trong \emph{4 L} dung dịch. (d) \emph{0.06 mol \ce{Na2CO3}} trong \emph{1500 mL} dung dịch.
\end{baitoan}

\begin{baitoan}[\cite{SGK_Hoa_Hoc_8}, 4., p. 146]
	Tính số mol \& số gam chất tan trong mỗi dung dịch sau: (a) \emph{1 L} dung dịch \emph{NaCl 0.5M}. (b) \emph{500 mL} dung dịch \emph{\ce{KNO3} 2M}. (c) \emph{250 mL} dung dịch \emph{\ce{CaCl2} 0.1M}. (d) \emph{2 L} dung dịch \emph{\ce{Na2SO4} 0.3M}.
\end{baitoan}

\begin{baitoan}[\cite{SGK_Hoa_Hoc_8}, 5., p. 146]
	Tính nồng độ \% của những dung dịch sau: (a) \emph{20 g KCl} trong \emph{600 g} dung dịch. (b) \emph{32 g \ce{NaNO3}} trong \emph{2 kg} dung dịch. (c) \emph{75 g \ce{K2SO4}} trong \emph{1500 g} dung dịch.
\end{baitoan}

\begin{baitoan}[\cite{SGK_Hoa_Hoc_8}, 6., p. 146]
	Tính số gam chất tan cần dùng để pha chế mỗi dung dịch sau: (a) \emph{2.5 L} dung dịch \emph{NaCl 0.9M}. (b) \emph{50 g} dung dịch \emph{MgCl2 4\%}. (c) \emph{250 mL} dung dịch \emph{\ce{MgSO4} 0.1M}.
\end{baitoan}

\begin{baitoan}[\cite{SGK_Hoa_Hoc_8}, 7., p. 146]
	Ở nhiệt độ $25^\circ$\emph{C}, độ tan của muối ăn là \emph{36 g}, của đường là \emph{204 g}. Tính nồng độ \% của các dung dịch bão hòa muối ăn \& đường ở nhiệt độ trên.
\end{baitoan}

\begin{baitoan}[\cite{SBT_Hoa_Hoc_8}, 42.1, p. 57]
	Chọn câu trả lời đúng nhất \& chỉ ra chỗ sai của câu trả lời không đúng sau: (a) Nồng độ \% của dung dịch cho biết: {\sf 1.} Số gam chất tan trong \emph{100 g} dung môi. {\sf 2.} Số gam chất tan trong \emph{100 g} dung dịch. {\sf 3.} Số gam chất tan trong \emph{1 L} dung dịch. {\sf 4.} Số gam chất tan trong \emph{1 L} dung môi. {\sf 5.} Số gam chất tan trong 1 lượng dung dịch xác định. (b) Nồng độ mol của dung dịch cho biết: {\sf 1.} Số gam chất tan trong \emph{1 L} dung dịch. {\sf 2.} Số mol chất tan trong \emph{1 L} dung dịch. {\sf 3.} Số mol chất tan trong \emph{1 L} dung môi. {\sf 4.} Số gam chất tan trong \emph{1 L} dung môi. {\sf 5.} Số mol chất tan trong 1 thể tích xác định dung dịch.
\end{baitoan}

\begin{baitoan}[\cite{SBT_Hoa_Hoc_8}, 42.2, p. 57]
	Trong phòng thí nghiệm có các lọ đựng dung dịch \emph{NaCl, \ce{H2SO4}, NaOH} có cùng nồng độ là \emph{0.5M}. (a) Lấy 1 ít mỗi dung dịch trên vào ống nghiệm riêng biệt. Hỏi phải lấy như thế nào để có số mol chất tan có trong mỗi ống nghiệm là bằng nhau? (b) Nếu thể tích dung dịch có trong mỗi ống nghiệm là \emph{5 ml}, tính số gam chất tan có trong mỗi ống nghiệm.
\end{baitoan}

\begin{baitoan}[\cite{SBT_Hoa_Hoc_8}, 42.3, p. 58]
	Để xác định độ tan của 1 muối trong nước bằng phương pháp thực nghiệm, người ta dựa vào những kết quả như sau: Nhiệt độ của dung dịch muối bão hòa đo được là $19^\circ$\emph{C}. Chén nung rỗng có khối lượng là \emph{47.1 g}. Chén nung đựng dung dịch muối bão hòa có khối lượng là \emph{69.6 g}. Chén nung \& muối kết tinh thu được sau khi làm bay hết hơi nước, có khối lượng là \emph{49.6 g}. Cho biết: (a) Khối lượng muối kết tinh thu được là bao nhiêu. (b) Độ tan của muối ở nhiệt độ $19^\circ$\emph{C}. (c) Nồng độ \% của dung dịch muối bão hòa ở nhiệt độ $19^\circ$\emph{C}.
\end{baitoan}

\begin{baitoan}[\cite{SBT_Hoa_Hoc_8}, 42.4, p. 58]
	Làm bay hơi \emph{300 g} nước ra khỏi \emph{700 g} dung dịch muối $12$\%, nhận thấy có \emph{5 g} muối tách khỏi dung dịch bão hòa. Xác định nồng độ \% của dung dịch muối bão hòa trong điều kiện thí nghiệm trên.
\end{baitoan}

\begin{baitoan}[\cite{SBT_Hoa_Hoc_8}, 42.5, p. 58]
	1 dung dịch \emph{\ce{CuSO4}} có khối lượng riêng là \emph{1.206 g\texttt{/}mL}. Khi cô cạn \emph{165.84 mL} dung dịch này người ta thu được \emph{36 g \ce{CuSO4}}. Xác định nồng độ \% của dung dịch \emph{\ce{CuSO4}} đã dùng.
\end{baitoan}

\begin{baitoan}[\cite{SBT_Hoa_Hoc_8}, 42.6, p. 58]
	Điền vào ô trống của bảng các số liệu thích hợp của mỗi dung dịch glucose \emph{\ce{C6H12O6}} trong nước:
	\begin{table}[H]
		\centering
		\begin{tabular}{|c|c|c|c|c|}
			\hline
			Các dung dịch & Khối lượng \ce{C6H12O6} & Số mol \ce{C6H12O6} & Thể tích dung dịch & Nồng độ mol $C_{\rm M}$ \\
			\hline
			Dung dịch 1 & 12.6 g &  & 219 mL &  \\
			\hline
			Dung dịch 2 &  & 1.08 &  & 0.519M \\
			\hline
			Dung dịch 3 &  &  & 1.62 L & 1.08M \\
			\hline
		\end{tabular}
	\end{table}
\end{baitoan}

\begin{baitoan}[\cite{SBT_Hoa_Hoc_8}, 42.7, p. 58]
	Trình bày phương pháp thực nghiệm để xác định nồng độ \% \& nồng độ mol của 1 mẫu dung dịch \emph{\ce{CuSO4}} có sẵn trong phòng thí nghiệm.
\end{baitoan}

\subsubsection{Pha chế dung dịch}

\begin{baitoan}[\cite{SGK_Hoa_Hoc_8}, 1., p. 149]
	Làm bay hơi \emph{60 g} nước từ dung dịch có nồng độ \emph{15\%}, được dung dịch mới có nồng độ \emph{18\%}. Xác định khối lượng của dung dịch ban đầu.
\end{baitoan}

\begin{baitoan}[\cite{SGK_Hoa_Hoc_8}, 2., p. 149]
	Đun nhẹ \emph{20 g} dung dịch \emph{\ce{CuSO4}} cho đến khi nước bay hơi hết, người ta thu được chất rắn màu trắng là \emph{\ce{CuSO4}} khan. Chấy này có khối lượng là \emph{3.6 g}. Xác định nồng độ \% của dung dịch \emph{\ce{CuSO4}}.
\end{baitoan}

\begin{baitoan}[\cite{SGK_Hoa_Hoc_8}, 3., p. 149]
	Cân lấy \emph{10.6 g \ce{Na2CO3}} cho vào cốc chia độ có dung tích \emph{500 mL}. Rót từ từ nước cất vào cốc cho đến vạch \emph{200 mL}. Khuấy nhẹ cho \emph{\ce{Na2CO3}} tan hết, ta được dung dịch \emph{\ce{Na2CO3}}. Biết \emph{1 mL} dung dịch này cho khối lượng là \emph{1.05 g}. Xác định nồng độ \% \& nồng độ mol của dung dịch vừa pha chế được.
\end{baitoan}

\begin{baitoan}[\cite{SGK_Hoa_Hoc_8}, $4^\star$., p. 149]
	Điền các giá trị chưa biết vào những ô để trống trong bảng, bằng cách thực hiện các tính toán theo mỗi cột:
	\begin{table}[H]
		\centering
		\begin{tabular}{|c|c|c|c|c|c|}
			\hline
			\backslashbox{Đại lượng}{Dung dịch}& NaCl & \ce{Ca(OH)2} & \ce{BaCl2} & KOH & \ce{CuSO4} \\
			\hline
			$m_{\rm ct}$ & 30 g & 0.148 g &  &  & 3 g \\
			\hline
			$m_{\ce{H2O}}$ & 170 g &  &  &  &  \\
			\hline
			$m_{\rm dd}$ &  &  & 150 g &  &  \\
			\hline
			$V_{\rm dd}$ &  & 200 mL &  & 300 mL &  \\
			\hline
			$D_{\rm dd}$ (g\texttt{/}mL) & 1.1 & 1 & 1.2 & 1.04 & 1.15 \\
			\hline
			$C$\% &  &  & 20\% &  & 15\% \\
			\hline
			$C_{\rm M}$ &  &  &  & 2.5M &  \\
			\hline
		\end{tabular}
	\end{table}
\end{baitoan}

\begin{baitoan}[\cite{SGK_Hoa_Hoc_8}, $5^\star$., p. 149]
	Tìm độ tan của 1 muối trong nước bằng phương pháp thực nghiệm, người ta có được những kết quả sau: (a) Nhiệt độ của dung dịch muối bão hòa là $20^\circ$\emph{C}. (b) Chén sứ nung có khối lượng \emph{60.26 g}. (c) Chén sứ đựng dung dịch muối có khối lượng \emph{86.26 g}. (d) Khối lượng chén nung \& muối kết tinh sau khi làm bay hết hơi nước là \emph{66.26 g}. Xác định độ tan của muối ở nhiệt độ $20^\circ$\emph{C}.
\end{baitoan}

\subsubsection{Miscellaneous}

\begin{baitoan}[\cite{SGK_Hoa_Hoc_8}, 1., p. 151]
	Các ký hiệu sau cho ta biết những điều gì? (a) $S_{\ce{KNO3}(\rm20^\circ C)} = 31.6$ \emph{g}, $S_{\ce{KNO3}(\rm100^\circ C)} = 246$ \emph{g}, $S_{\ce{CuSO4}(\rm20^\circ C)} = 20.7$ \emph{g}, $S_{\ce{CuSO4}(\rm100^\circ C)} = 75.4$ \emph{g}. (b) $S_{\ce{CO2}(\rm20^\circ C,1 atm)} = 1.73$ \emph{g}, $S_{\ce{CO2}(\rm60^\circ C,1 atm)} = 0.07$ \emph{g}.
\end{baitoan}

\begin{baitoan}[\cite{SGK_Hoa_Hoc_8}, 2., p. 151]
	1 người đã pha loãng acid bằng cách rót từ từ \emph{20 g} dung dịch \emph{\ce{H2SO4} 50\%} vào nước \& sau đó thu được \emph{50 g} dung dịch \emph{\ce{H2SO4}}. (a) Tính nồng độ \% của dung dịch \emph{\ce{H2SO4}} sau khi pha loãng. (b) Tính nồng độ mol của dung dịch \emph{\ce{H2SO4}} sau khi pha loãng, biết dung dịch này có khối lượng riêng là \emph{1.1 g\texttt{/}$\rm cm^3$}.
\end{baitoan}

\begin{baitoan}[\cite{SGK_Hoa_Hoc_8}, 3., p. 151]
	Biết $S_{\ce{K2SO4}(\rm20^\circ C)} = 11.1$ \emph{g}. Tính nồng độ \% của dung dịch \emph{\ce{K2SO4}} bão hòa ở nhiệt độ này.
\end{baitoan}

\begin{baitoan}[\cite{SGK_Hoa_Hoc_8}, $4^\star$., p. 151]
	Trong \emph{800 mL} của 1 dung dịch có chứa \emph{8 g NaOH}. (a) Tính nồng độ mol của dung dịch này. (b) Phải thêm bao nhiêu milliliter nước vào \emph{200 mL} dung dịch này để được dung dịch \emph{NaOH 0.1M}?
\end{baitoan}

\begin{baitoan}[\cite{SGK_Hoa_Hoc_8}, 5., p. 151]
	Trình bày cách pha chế: (a) \emph{400 g} dung dịch \emph{\ce{CuSO4} 4\%}. (b) \emph{300 mL} dung dịch \emph{NaCl 3M}.
\end{baitoan}

\begin{baitoan}[\cite{SGK_Hoa_Hoc_8}, 6., p. 151]
	Trình bày cách pha chế: (a) \emph{150 g} dung dịch \emph{\ce{CuSO4} 2\%} từ dung dịch \emph{\ce{CuSO4} 20\%}. (b) \emph{250 mL} dung dịch \emph{NaOH 0.5M} từ dung dịch \emph{NaOH 2M}.
\end{baitoan}

\begin{baitoan}[\cite{SGK_Hoa_Hoc_8}, p. 152]
	Tính toán \& pha chế các dung dịch sau: (a) \emph{50 g} dung dịch đường có nồng độ \emph{15\%}. (b) \emph{100 mL} dung dịch sodium chloride có nồng độ \emph{0.2M}. (c) \emph{50 g} dung dịch đường 5\% từ dung dịch đường có nồng độ \emph{15\%} ở trên. (d) \emph{50 mL} dung dịch sodium chloride có nồng độ \emph{0.1M} từ dung dịch sodium chloride có nồng độ \emph{0.2M} ở trên.
\end{baitoan}

\begin{baitoan}[\cite{SBT_Hoa_Hoc_8}, 43.1, p. 59]
	Từ dung dịch \emph{\ce{MgSO4} 2M} làm thế nào pha chế được \emph{100 mL} dung dịch \emph{\ce{MgSO4} 0.4M}?
\end{baitoan}

\begin{baitoan}[\cite{SBT_Hoa_Hoc_8}, 43.2, p. 59]
	Từ dung dịch \emph{NaCl 1M}, trình bày cách pha chế \emph{250 mL} dung dịch \emph{NaCl 0.2M}.
\end{baitoan}

\begin{baitoan}[\cite{SBT_Hoa_Hoc_8}, 43.3, p. 59]
	Trình bày cách pha chế \emph{150 mL} dung dịch \emph{\ce{HNO3} 0.25M} bằng cách pha loãng dung dịch \emph{\ce{HNO3} 5M} có sẵn.
\end{baitoan}

\begin{baitoan}[\cite{SBT_Hoa_Hoc_8}, 43.4, p. 59]
	Từ glucose \emph{\ce{C6H12O6}} \& nước cất, trình bày cách pha chế \emph{200 g} dung dịch glucose $2$\%.
\end{baitoan}

\begin{baitoan}[\cite{SBT_Hoa_Hoc_8}, 43.5, p. 59]
	Trình bày cách pha chế các dung dịch theo những yêu cầu sau: (a) \emph{250 mL} dung dịch có nồng độ \emph{0.1M} của những chất sau: \emph{NaCl, \ce{KNO3, CuSO4}}. (b) \emph{200 g} dung dịch có nồng độ \emph{10\%} của mỗi chất nói trên.
\end{baitoan}

\begin{baitoan}[\cite{SBT_Hoa_Hoc_8}, 43.6, p. 59]
	Có những dung dịch ban đầu như sau: \emph{NaCl 2M, \ce{MgSO4} 0.5M, \ce{KNO3} 4M}. Làm thế nào có thể pha chế được những dung dịch theo những yêu cầu sau: (a) \emph{500 mL} dung dịch \emph{NaCl 0.5M}. (b) \emph{2 L} dung dịch \emph{\ce{MgSO4} 0.2M}. (c) \emph{50 mL} dung dịch \emph{\ce{KNO3} 0.2M}.
\end{baitoan}

\begin{baitoan}[\cite{SBT_Hoa_Hoc_8}, 43.7, p. 59]
	Từ các muối \& nước cất, trình bày cách pha chế các dung dịch sau: (a) \emph{2.5 kg} dung dịch \emph{NaCl 0.9\%}. (b) \emph{50 g} dung dịch \emph{\ce{MgCl2} 4\%}. (c) \emph{250 g} dung dịch \emph{\ce{MgSO4} 0.1\%}.
\end{baitoan}

\begin{baitoan}[\cite{SBT_Hoa_Hoc_8}, $43.8^\star$, p. 60]
	Có $2$ lọ đựng dung dịch \emph{\ce{H2SO4}}. Lọ thứ nhất có nồng độ \emph{1M}, lọ thứ 2 có nồng độ \emph{3M}. Tính toán \& trình bày cách pha chế \emph{50 mL} dung dịch \emph{\ce{H2SO4}} có nồng độ \emph{1.5M} từ 2 dung dịch acid đã cho.
\end{baitoan}

\begin{baitoan}[\cite{SBT_Hoa_Hoc_8}, $43.9^\star$, p. 60]
	Cần dùng bao nhiêu milliliter dung dịch \emph{NaOH 3\%} có khối lượng riêng là \emph{1.05 g\texttt{/}mL} \& bao nhiêu milliliter dung dịch \emph{NaOH 10\%} có khối lượng riêng là \emph{1.12 g\texttt{/}mL} để pha chế được \emph{2 L} dung dịch \emph{NaOH 8\%} có khối lượng riêng là \emph{1.10 g\texttt{/}mL?}
\end{baitoan}

\begin{baitoan}[\cite{SBT_Hoa_Hoc_8}, 44.1, p. 60]
	Cân \emph{10.6 g} muối \emph{\ce{Na2CO3}} cho vào cốc chia độ. Rót vào cốc khoảng vài chục milliliter nước cất, khuấy cho muối tan hết. Sau đó rót thêm nước vào cốc cho đủ \emph{200 mL}. Ta được dung dịch \emph{\ce{Na2CO3}} có khối lượng riêng là \emph{1.05 g\texttt{/}mL}. Tính nồng độ \% \& nồng độ mol của dung dịch vừa pha chế.
\end{baitoan}

\begin{baitoan}[\cite{SBT_Hoa_Hoc_8}, 44.2, p. 60]
	Có: \emph{\ce{CuSO4}} \& nước cất. Tính toán \& trình bày cách pha chế để có được những sản phẩm sau: (a) \emph{50 mL} dung dịch \emph{\ce{CuSO4}} có nồng độ \emph{1M}. (b) \emph{50 g} dung dịch \emph{\ce{CuSO4}} có nồng độ \emph{10\%}.
\end{baitoan}

\begin{baitoan}[\cite{SBT_Hoa_Hoc_8}, 44.3, p. 60]
	Bảng dưới đây cho biết độ tan của 1 muối trong nước thay đổi theo nhiệt độ:
	\begin{table}[H]
		\centering
		\begin{tabular}{|c|c|c|c|c|c|}
			\hline
			Nhiệt độ (${}^\circ$C) & 20 & 30 & 40 & 50 & 60 \\
			\hline
			Độ tan (g\texttt{/}100 g nước) & 5 & 11 & 18 & 28 & 40 \\
			\hline
		\end{tabular}
	\end{table}
	\noindent(a) Vẽ đồ thị biểu diễn độ tan của muối trong nước (trục tung biểu thị khối lượng chất tan, trục hoành biểu thị nhiệt độ). (b) Căn cứ vào đồ thị, ước lượng độ tan của muối ở $25^\circ$\emph{C} \& $55^\circ$\emph{C}. (c) Tính số gam muối tan trong: \emph{200 g} nước để có dung dịch bão hòa ở nhiệt độ $20^\circ$\emph{C}; \emph{2 kg} nước để có dung dịch bão hòa ở nhiệt độ $50^\circ$\emph{C}.
\end{baitoan}

\begin{baitoan}[\cite{SBT_Hoa_Hoc_8}, 44.4, pp. 60--61]
	Người ta pha chế 1 dung dịch \emph{NaCl} ở $20^\circ$\emph{C} bằng cách hòa tan \emph{23.5 g NaCl} trong \emph{75 g} nước. Căn cứ vào độ tan của \emph{NaCl} trong nước $S_{\rm NaCl(20^\circ C)} = 32$ \emph{g}, cho biết dung dịch \emph{NaCl} đã pha chế là bão hòa hay chưa bão hòa. Nếu dung dịch \emph{NaCl} là chưa bão hòa, làm thế nào để có được dung dịch \emph{NaCl} bão hòa ở $20^\circ$\emph{C}?
\end{baitoan}

\begin{baitoan}[\cite{SBT_Hoa_Hoc_8}, 44.5, p. 61]
	Tính toán \& trình bày cách pha chế \emph{0.5 L} dung dịch \emph{\ce{H2SO4}} có nồng độ \emph{1 M} từ \emph{\ce{H2SO4}} có nồng độ \emph{98\%}, khối lượng riêng là \emph{1.84 g\texttt{/}mL}.
\end{baitoan}

\begin{baitoan}[\cite{SBT_Hoa_Hoc_8}, $44.6^\star$, p. 61]
	A là dung dịch \emph{\ce{H2SO4}} có nồng độ \emph{0.2M}. B là dung dịch \emph{\ce{H2SO4}} có nồng độ \emph{0.5M}. (a) Nếu trộn A \& B theo tỷ lệ thể tích $V_A:V_B = 2:3$ được dung dịch C. Xác định nồng độ mol của dung dịch C. (b) Phải trộn A \& B theo tỷ lệ nào về thể tích để được dung dịch \emph{\ce{H2SO4}} có nồng độ \emph{0.3M}?
\end{baitoan}

\begin{baitoan}[\cite{SBT_Hoa_Hoc_8}, $44.7^\star$, p. 61]
	Có \emph{200 g} dung dịch \emph{NaOH 5\%} (dung dịch A). (a) Cần phải trộn thêm vào dung dịch A bao nhiêu gam dung dịch \emph{NaOH 10\%} để được dung dịch \emph{NaOH 8\%}? (b) Cần hòa tan bao nhiêu gam \emph{NaOH} vào dung dịch A để có dung dịch \emph{NaOH 8\%}? (c) Làm bay hơi nước dung dịch A, người ta cũng thu được dung dịch \emph{NaOH 8\%}. Tính khối lượng nước đã bay hơi.
\end{baitoan}

%------------------------------------------------------------------------------%

\section{Tốc Độ Phản Ứng \& Chất Xúc Tác}

\noindent\fbox{%
	\parbox{\textwidth}{%
		\noindent\textsf{\textbf{Kiến thức cốt lõi.}} \fbox{\bf 1} \textit{Tốc độ phản ứng} là đại lượng chỉ mức độ nhanh hay chậm của 1 phản ứng hóa học. \fbox{\bf 2} Các yếu tố ảnh hưởng đến tốc độ phản ứng: \textit{Diện tích bề mặt tiếp xúc}: Diện tích bề mặt tiếp xúc càng lớn, tốc độ phản ứng càng nhanh. \textit{Nhiệt độ}: Khi tăng nhiệt độ, phản ứng diễn ra với tốc độ nhanh hơn. \textit{Nồng độ}: Nồng độ các chất phản ứng càng cao, tốc độ phản ứng càng nhanh. \textit{Chất xúc tác} làm tăng tốc độ phản ứng nhưng không bị thay đổi cả về lượng \& chất sau phản ứng. \textit{Chất ức chế} làm giảm tốc độ phản ứng.
	}%
}

%------------------------------------------------------------------------------%

\printbibliography[heading=bibintoc]
	
\end{document}