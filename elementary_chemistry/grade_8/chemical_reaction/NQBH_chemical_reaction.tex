\documentclass{article}
\usepackage[backend=biber,natbib=true,style=authoryear,maxbibnames=20]{biblatex}
\addbibresource{/home/nqbh/reference/bib.bib}
\usepackage[utf8]{vietnam}
\usepackage{tocloft}
\renewcommand{\cftsecleader}{\cftdotfill{\cftdotsep}}
\usepackage[colorlinks=true,linkcolor=blue,urlcolor=red,citecolor=magenta]{hyperref}
\usepackage{amsmath,amssymb,amsthm,float,graphicx,mathtools,tikz,tipa}
\usepackage[version=4]{mhchem}
\allowdisplaybreaks
\newtheorem{assumption}{Assumption}
\newtheorem{baitoan}{Bài toán}
\newtheorem{cauhoi}{Câu hỏi}
\newtheorem{conjecture}{Conjecture}
\newtheorem{corollary}{Corollary}
\newtheorem{dangtoan}{Dạng toán}
\newtheorem{definition}{Definition}
\newtheorem{dinhly}{Định lý}
\newtheorem{dinhnghia}{Định nghĩa}
\newtheorem{example}{Example}
\newtheorem{ghichu}{Ghi chú}
\newtheorem{hequa}{Hệ quả}
\newtheorem{hypothesis}{Hypothesis}
\newtheorem{lemma}{Lemma}
\newtheorem{luuy}{Lưu ý}
\newtheorem{nhanxet}{Nhận xét}
\newtheorem{notation}{Notation}
\newtheorem{note}{Note}
\newtheorem{principle}{Principle}
\newtheorem{problem}{Problem}
\newtheorem{proposition}{Proposition}
\newtheorem{question}{Question}
\newtheorem{remark}{Remark}
\newtheorem{theorem}{Theorem}
\newtheorem{thinghiem}{Thí nghiệm}
\newtheorem{vidu}{Ví dụ}
\usepackage[left=1cm,right=1cm,top=5mm,bottom=5mm,footskip=4mm]{geometry}

\title{Chemical Reaction -- Phản Ứng Hóa Học}
\author{Nguyễn Quản Bá Hồng\footnote{Independent Researcher, Ben Tre City, Vietnam\\e-mail: \texttt{nguyenquanbahong@gmail.com}; website: \url{https://nqbh.github.io}.}}
\date{\today}

\begin{document}
\maketitle
\begin{abstract}
	\textsc{[en]} This text is a collection of problems, from easy to advanced, about \textit{chemical reaction}. This text is also a supplementary material for my lecture note on Elementary Chemistry, which is stored \& downloadable at the following link: \href{https://github.com/NQBH/hobby/blob/master/elementary_chemistry/grade_8/NQBH_elementary_chemistry_grade_8.pdf}{GitHub\texttt{/}NQBH\texttt{/}hobby\texttt{/}elementary chemistry\texttt{/}grade 8\texttt{/}lecture}\footnote{\textsc{url}: \url{https://github.com/NQBH/hobby/blob/master/elementary_chemistry/grade_8/NQBH_elementary_chemistry_grade_8.pdf}.}. The latest version of this text has been stored \& downloadable at the following link: \href{https://github.com/NQBH/hobby/blob/master/elementary_chemistry/chemical_reaction/NQBH_chemical_reaction.pdf}{GitHub\texttt{/}NQBH\texttt{/}hobby\texttt{/}elementary chemistry\texttt{/}grade 8\texttt{/}chemical reaction}\footnote{\textsc{url}: \url{https://github.com/NQBH/hobby/blob/master/elementary_chemistry/chemical_reaction/NQBH_chemical_reaction.pdf}.}.
	\vspace{2mm}
	
	\textsc{[vi]} Tài liệu này là 1 bộ sưu tập các bài tập chọn lọc từ cơ bản đến nâng cao về \textit{phản ứng hóa học}. Tài liệu này là phần bài tập bổ sung cho tài liệu chính -- bài giảng \href{https://github.com/NQBH/hobby/blob/master/elementary_chemistry/grade_8/NQBH_elementary_chemistry_grade_8.pdf}{GitHub\texttt{/}NQBH\texttt{/}hobby\texttt{/}elementary chemistry\texttt{/}grade 8\texttt{/}lecture} của tác giả viết cho Hóa Học Sơ Cấp. Phiên bản mới nhất của tài liệu này được lưu trữ \& có thể tải xuống ở link sau: \href{https://github.com/NQBH/hobby/blob/master/elementary_chemistry/grade_8/real/NQBH_real.pdf}{GitHub\texttt{/}NQBH\texttt{/}hobby\texttt{/}elementary chemistry\texttt{/}grade 8\texttt{/}chemical reaction}.
\end{abstract}
\setcounter{secnumdepth}{4}
\setcounter{tocdepth}{3}
\tableofcontents
\newpage

%------------------------------------------------------------------------------%

\section{Physical- \& Chemical Transformations -- Biến Đổi Vật Lý \& Biến Đổi Hóa Học}
\textsf{\textbf{Nội dung.} Biến đổi vật lý, biến đổi hóa học.}

\begin{baitoan}[\cite{SGK_KHTN_8_Canh_Dieu}, p. 12]
	Các hiện tượng sau mô tả hiện tượng chất bị biến đổi thành chất khác hay chỉ mô tả sự thay đổi về tính chất vật lý (trạng thái, kích thước, hình dạng, $\ldots$)? (a) Xẻ mẩu giấy vụn. (b) Hòa tan đường vào nước. (c) Đinh sắt bị uốn cong. (d) Đốt mẩu giấy vụn. (e) Đun đường. (f) Đinh sắt bị gỉ.
\end{baitoan}

\subsection{Sự biến đổi chất}

\subsubsection{Sự biến đổi vật lý}

\begin{dinhnghia}[Biến đổi vật lý]
	\emph{Biến đổi vật lý} là hiện tượng chất có sự biến đổi về trạng thái, kích thước, $\ldots$ nhưng vẫn giữ nguyên là chất ban đầu.
\end{dinhnghia}

\begin{vidu}
	Nước hoa khuếch tán trong không khí, hòa tan đường vào nước, làm đá trong tủ lạnh, $\ldots$ là các biến đổi vật lý.
\end{vidu}

\begin{baitoan}[\cite{SGK_KHTN_8_Canh_Dieu}, 1, p. 13]
	Kể thêm vài hiện tượng xảy ra trong thực tế có sự biến đổi vật lý.
\end{baitoan}

\subsubsection{Sự biến đổi hóa học}

\begin{dinhnghia}[Biến đổi hóa học]
	\emph{Biến đổi hóa học} là hiện tượng chất có sự biến đổi tạo ra chất khác.
\end{dinhnghia}

\begin{vidu}
	Quá trình tiêu hóa thức ăn, trứng để lâu ngày bị thối, nung đá vôi tạo thành vôi sống, $\ldots$ là các biến đổi hóa học.
\end{vidu}

\begin{baitoan}[\cite{SGK_KHTN_8_Canh_Dieu}, 2, p. 14]
	Kể thêm vài hiện tượng xảy ra trong thực tế có sự biến đổi hóa học.
\end{baitoan}

\subsection{Phân biệt sự biến đổi vật lý \& sự biến đổi hóa học}

\begin{baitoan}[\cite{SGK_KHTN_8_Canh_Dieu}, p. 14]
	Gắn cây nến (có thành phần chính là paraffin) trên đĩa sứ, đốt nến cháy trong khoảng $1$ phút. Mô tả các hiện tượng xảy ra trong quá trình nến cháy, chỉ ra giai đoạn diễn ra sự biến đổi vật lý, giai đoạn diễn ra sự biến đổi hóa học. Biết nến cháy trong không khí chủ yếu tạo ra khí carbon dioxide \& hơi nước.
\end{baitoan}

\begin{baitoan}[\cite{SGK_KHTN_8_Canh_Dieu}, 3, p. 14]
	Quá trình nào diễn ra sự biến đổi vật lý, quá trình nào diễn ra sự biến đổi hóa học? (a) Quả táo khi vẫn còn tươi $\to$ Quả táo để lâu ngày bị hỏng. (b) Vỏ lon nước ngọt $\to$ Vỏ lon nước ngọt bị bóp méo. (c) Bánh mì trước khi nướng $\to$ Bánh mì bị nướng cháy. (d) Hạt gạo $\to$ Bột gạo.
\end{baitoan}

\begin{baitoan}[\cite{SGK_KHTN_8_Canh_Dieu}, 4, p. 14]
	Nêu những điểm khác nhau giữa sự biến đổi vật lý \& sự biến đổi hóa học.
\end{baitoan}

\begin{baitoan}[\cite{SGK_KHTN_8_Canh_Dieu}, 3, p. 15]
	Trường hợp nào diễn ra sự biến đổi vật lý, trường hợp nào diễn ra sự biến đổi hóa học? (a) Khi có dòng điện đi qua, dây tóc bóng đèn (làm bằng kim loại tungsten) nóng \& sáng lên. (b) Hiện tượng băng tan. (c) Thức ăn bị ôi thiu. (d) Đốt cháy khí methane \emph{\ce{CH4}} thu được khí carbon dioxide \emph{\ce{CO2}} \& hơi nước \emph{\ce{H2O}}.
\end{baitoan}
Động Phong Nha (Động nước) là động tiêu biểu nhất của hệ thống hang động thuộc quần thể danh thắng Phong Nha--Kẻ Bàng. Đặc trưng của nơi đây là có nhiều thạch nhũ với các hình dáng đẹp, độc đáo. Hiện tượng thạch nhũ được tạo thành chủ yếu là do sự biến đổi hóa học. Ở các vùng núi đá vôi (thành phần chủ yếu là \ce{CaCO3}), khi trời mưa, nước mưa kết hợp với \ce{CO2} trong không khí tạo thành môi trường acid, làm tan được đá vôi (\ce{CaCO3} chuyển hóa thành \ce{Ca(HCO3)2}). Khi nước có chứa \ce{Ca(HCO3)2} chảy qua các khe đá vào trong các hang động (ở đây có sự thay đổi về nhiệt độ \& áp suất), \ce{Ca(HCO3)2} chuyển thành \ce{CaCO3} rắn, không tan. Lớp \ce{CaCO3} dần dần tích lại ngày càng nhiều, qua hàng triệu triệu năm tạo thành thạch nhũ với những hình thù đa dạng, đẹp mắt.

\noindent\fbox{%
	\parbox{\textwidth}{%
		\noindent\textsf{\textbf{Kiến thức cốt lõi.}} \fbox{\bf 1} \textit{Biến đổi vật lý} là hiện tượng chất có sự biến đổi về trạng thái, kích thước, $\ldots$ nhưng vẫn giữ nguyên là chất ban đầu. \fbox{\bf 2} \textit{Biến đổi hóa học} là hiện tượng chất có sự biến đổi tạo ra chất khác.
	}%
}

%------------------------------------------------------------------------------%

\section{Chemical Reactions \& Its Energy -- Phản Ứng Hóa Học \& Năng Lượng của Phản Ứng Hóa Học}
\textsf{\textbf{Nội dung.} Phản ứng hóa học, chất đầu \& sản phẩm, sự sắp xếp khác nhau của các nguyên tử trong phân tử chất đầu \& sản phẩm, 1 số dấu hiệu chứng tỏ có phản ứng hóa học xảy ra, phản ứng thu\texttt{/}tỏa nhiệt, các ứng dụng phổ biến của phản ứng tỏa nhiệt (đốt cháy than, xăng, dầu).}

\subsection{Phản ứng hóa học}

\begin{dinhnghia}[Phản ứng hóa học, chất tham gia phản ứng, sản phẩm]
	Quá trình biến đổi từ chất này thành chất khác gọi là \emph{phản ứng hóa học}. Chất ban đầu bị biến đổi trong phản ứng được gọi là \emph{chất tham gia phản ứng}, chất tạo thành sau phản ứng được gọi là \emph{chất sản phẩm}.
\end{dinhnghia}

\begin{vidu}[Tạo \ce{H2O}]
	Đốt cháy khí hydrogen trong không khí tạo ra ngọn lửa màu xanh, sau đó đưa ngọn lửa của khí hydrogen đang cháy vào trong bình đựng khí oxygen thì thấy khí hydrogen cháy mạnh hơn, sáng hơn, \& trên thành bình xuất hiện những giọt nước nhỏ. Ở đây đã diễn ra sự biến đổi hóa học, trong đó xảy ra quá trình biến đổi hydrogen \& oxygen tạo thành nước. Quá trình này đã xảy ra phản ứng hóa học.
	\begin{figure}[H]
		\centering
		\includegraphics[scale=0.3]{H2O}
		\caption{Thí nghiệm điều chế \& đốt cháy khí hydrogen trong khí oxygen.}
		\label{fig: H2O}
	\end{figure}
	Trong thí nghiệm này, chất tham gia phản ứng là hydrogen \emph{\ce{H2, O2}} \& chất sản phẩm là nước \emph{\ce{H2O}}.
\end{vidu}

\begin{baitoan}[\cite{SGK_KHTN_8_Canh_Dieu}, 1, p. 16]
	Quan sát hình \ref{fig: H2O}, có những quá trình biến đổi hóa học nào xảy ra?
\end{baitoan}

\begin{vidu}[\cite{SGK_KHTN_8_Canh_Dieu}, Ví dụ 1, p. 17]
	Khi đung nóng hỗn hợp bột sắt \& bột lưu huỳnh ta được hợp chất iron(II) sulfide \emph{FeS}. Chất tham gia phản ứng là sắt \& lưu huỳnh. Chất sản phẩm là iron(II) sulfide.
\end{vidu}

\begin{vidu}[\cite{SGK_KHTN_8_Canh_Dieu}, Ví dụ 2, p. 17]
	Nến cháy trong không khí tạo thành khí carbon dioxide \& hơi nước. Chất tham gia phản ứng là paraffin \& oxygen. Chất sản phẩm là carbon dioxide \& nước.
\end{vidu}

\begin{baitoan}[\cite{SGK_KHTN_8_Canh_Dieu}, 2, p. 17]
	Xác định chất tham gia phản ứng \& chất sản phẩm trong 2 trường hợp sau: (a) Đốt cháy methane tạo thành khí carbon dioxide \& nước. (b) Carbon (thành phần chính của than) cháy trong khí oxygen tạo thành khí carbon dioxide.
\end{baitoan}

\subsection{Diễn biến của phản ứng hóa học}
Phản ứng hóa học xảy ra trong thí nghiệm khí hydrogen cháy trong oxygen tạo thành nước, quá trình đó được mô tả theo sơ đồ sau:
\begin{figure}[H]
	\centering
	\includegraphics[scale=0.3]{PUHH_H2O}
	\caption{Sơ đồ mô tả phản ứng hóa học giữa khí hydrogen \& khí oxygen tạo thành nước.}
	\label{fig: PUHH H2O}
\end{figure}
\noindent Ứng với phương trình hóa học: \ce{$2$H2 + O2 ->[$t^\circ$] $2$H2O}. Trong sơ đồ \ref{fig: PUHH H2O}, các liên kết trong phân tử \ce{H2, O2} bị phá vỡ \& hình thành liên kết mới giữa 1 nguyên tử O \& 2 nguyên tử H.

Các biến đổi hóa học xảy ra khi có sự phá vỡ liên kết trong các chất tham gia phản ứng \& sự hình thành các liên kết mới để tạo ra các chất sản phẩm. Trong phản ứng hóa học, chỉ có liên kết giữa các nguyên tử thay đổi làm cho phân tử này biến đổi thành phân tử khác, kết quả là chất này biến đổi thành chất khác. Số nguyên tử của mỗi nguyên tố trước \& sau phản ứng không thay đổi.

\begin{baitoan}[\cite{SGK_KHTN_8_Canh_Dieu}, 3, p. 17]
	Quan sát sơ đồ \ref{fig: PUHH H2O}: (a) Trước phản ứng, những nguyên tử nào liên kết với nhau? (b) Sau phản ứng, những nguyên tử nào liên kết với nhau? (c) So sánh số nguyên tử \emph{H} \& số nguyên tử \emph{O} trước \& sau phản ứng.
\end{baitoan}

\begin{baitoan}[\cite{SGK_KHTN_8_Canh_Dieu}, 1, p. 18]
	Đốt cháy khí methane \emph{\ce{CH4}} trong không khí thu được carbon dioxide \emph{\ce{CO2}} \& nước \emph{\ce{H2O}} theo sơ đồ sau:
	\begin{figure}[H]
		\centering
		\includegraphics[scale=0.3]{CH4}
		\caption{Sơ đồ mô tả phản ứng đốt cháy khí methane trong không khí.}
		\label{fig: CH4}
	\end{figure}
	Quan sát sơ đồ \ref{fig: CH4}: (a) Trước phản ứng có các chất nào, những nguyên tử nào liên kết với nhau? (b) Sau phản ứng, có các chất nào được tạo thành, những nguyên tử nào liên kết với nhau? (c) So sánh số nguyên tử \emph{C, H, O} trước \& sau phản ứng.
\end{baitoan}

\subsection{Dấu hiệu có phản ứng hóa học xảy ra}
Để nhận biết có phản ứng hóa học xảy ra có thể dựa vào các dấu hiệu sau:

\fbox{\bf 1} \textit{Có sự thay đổi màu sắc, mùi, $\ldots$ của các chất; tạo ra chất khí hoặc chất không tan (kết tủa), $\ldots$}

\begin{vidu}
	 (a) Trong phản ứng giữa khí hydrogen với khí oxygen, nước tạo ra không còn tính chất của hydrogn \& oxygen nữa (nước ở thể lỏng, không cháy được, $\ldots$). (b) Trong phản ứng của sắt tác dụng với hydrochloric acid, quan sát thấy có bọt khí bay lên.
\end{vidu}

\begin{baitoan}[\cite{SGK_KHTN_8_Canh_Dieu}, 4, p. 18]
	Chỉ ra sự khác biệt về tính chất của nước với hydrogen \& oxygen.
\end{baitoan}
Dấu hiệu có phản ứng hóa học xảy ra trong phản ứng phân hủy đường:

\begin{baitoan}[\cite{SGK_KHTN_8_Canh_Dieu}, p. 18]
	 Cho khoảng 1 thìa cafe đường ăn vào ống nghiệm, sau đó đun trên ngọn lửa đèn cồn. Mô tả trạng thái (thể, màu sắc, $\ldots$) của đường trước \& sau khi đun. Nêu dấu hiệu chứng tỏ có phản ứng hóa học xảy ra.
\end{baitoan}

\begin{baitoan}[\cite{SGK_KHTN_8_Canh_Dieu}, p. 19]
	Nước đường để trong không khí 1 thời gian có vị chua. Trong trường hợp này, dấu hiệu nào chứng tỏ có phản ứng hóa học xảy ra?
\end{baitoan}
\fbox{\bf 2} \textit{Có sự tỏa nhiệt \& phát sáng}: Sự tỏa nhiệt \& phát sáng cũng có thể là dấu hiệu của phản ứng hóa học xảy ra.

\begin{vidu}[\cite{SGK_KHTN_8_Canh_Dieu}, p. 19]
	Khi đốt nến, nến cháy có sự tỏa nhiệt \& phát sáng.
\end{vidu}

\begin{baitoan}[\cite{SGK_KHTN_8_Canh_Dieu}, 1, p. 19]
	Những dấu hiệu nào thường dùng để nhận biết có phản ứng hóa học xảy ra?
\end{baitoan}

\subsection{Phản ứng thu\texttt{/}tỏa nhiệt}

\noindent\fbox{%
	\parbox{\textwidth}{%
		\noindent\textsf{\textbf{Kiến thức cốt lõi.}} \fbox{\bf 1} \textit{Phản ứng hóa học} là quá trình biến đổi từ chất này thành chất khác. \fbox{\bf 2} Trong phản ứng hóa học, chỉ có liên kết giữa các nguyên tử thay đổi làm cho phân tử này biến đổi thành phân tử khác, kết quả là chất này biến đổi thành chất khác. \fbox{\bf 3} Dấu hiệu thường dùng để nhận biết có phản ứng hóa học xảy ra: có sự thay đổi màu sắc, mùi, $\ldots$ của các chất; tạo ra chất khí hoặc chất không tan (kết tủa); có sự tỏa nhiệt \& phát sáng; $\ldots$ \fbox{\bf 4} \textit{Phản ứng tỏa nhiệt} là phản ứng tỏa ra năng lượng dưới dạng nhiệt. \fbox{\bf 5} \textit{Phản ứng thu nhiệt} là phản ứng thu vào năng lượng dưới dạng nhiệt.
	}%
}

%------------------------------------------------------------------------------%

\section{Định Luật Bảo Toàn Khối Lượng. Phương Trình Hóa Học}

\noindent\fbox{%
	\parbox{\textwidth}{%
		\noindent\textsf{\textbf{Kiến thức cốt lõi.}} \fbox{\bf 1} \textit{Định luật bảo toàn khối lượng}: Trong 1 phản ứng hóa học, tổng khối lượng của các chất sản phẩm bằng tổng khối lượng của các chất tham gia phản ứng. \fbox{\bf 2} Trong 1 phản ứng có $n$ chất ($n\in\mathbb{N}$, $n\ge2$) (bao gồm cả chất tham gia phản ứng \& chất sản phẩm), nếu biết khối lượng của $(n - 1)$ chất thì có thể tính được khối lượng của chất còn lại. \fbox{\bf 3} \textit{Phương trình hóa học} (PTHH) biểu diễn ngắn gọn phản ứng hóa học bằng ký hiệu \& CTHH. \fbox{\bf 4} Các bước lập PTHH: \textit{Bước 1}: Viết sơ đồ phản ứng. \textit{Bước 2}: So sánh số nguyên tử của mỗi nguyên tố có trong phân tử của các chất tham gia phản ứng \& các chất sản phẩm. \textit{Bước 3}: Cân bằng số nguyên tử của mỗi nguyên tố. \textit{Bước 4}: Kiểm tra \& viết PTHH. \fbox{\bf 5} Phương trình hóa học cho biết chất tham gia phản ứng, chất sản phẩm \& tỷ lệ về số nguyên tử hoặc số phân tử giữa các chất cũng như từng cặp chất trong phản ứng.
	}%
}

%------------------------------------------------------------------------------%

\section{Mol \& Tỷ Khối của Chất Khí}

\noindent\fbox{%
	\parbox{\textwidth}{%
		\noindent\textsf{\textbf{Kiến thức cốt lõi.}} \fbox{\bf 1} \textit{Mol} là lượng chất có chứa $6.022\cdot10^{23}$ nguyên tử hoặc phân tử của chất đó. \fbox{\bf 2} \textit{Khối lượng mol}. (ký hiệu là $M$) của 1 chất là khối lượng tính bằng gam của $N$ nguyên tử hoặc phân tử chất đó. \fbox{\bf 3} \textit{Thể tích mol} của chất khí là thể tích chiếm bởi $N$ phân tử của chất khí đó. Ở điều kiện chuẩn (áp suất 1 bar, nhiệt độ $25^\circ$C), thể tích mol của các chất khí đều bằng $24.79$ lít. \fbox{\bf 4} Công thức chuyển đổi giữa số mol $n$ \& khối lượng $m$ chất: $n = \frac{m}{M}$ mol. \fbox{\bf 5} Công thức chuyển đổi giữa số mol $n$ \& thể tích $V$ của chất khí ở điều kiện chuẩn: $n = \frac{V}{24.79}$ mol. \fbox{\bf 6} Công thức tính tỷ khối của khí A đối với khí B: $d_{\rm A\texttt{/}B} = \frac{M_{\rm A}}{M_{\rm B}}$.
	}%
}

%------------------------------------------------------------------------------%

\section{Tính Theo Phương Trình Hóa Học}

\noindent\fbox{%
	\parbox{\textwidth}{%
		\noindent\textsf{\textbf{Kiến thức cốt lõi.}} \fbox{\bf 1} Các bước tính khối lượng \& số mol của chất tham gia, chất sản phẩm trong phản ứng hóa học: \textit{Bước 1}: Viết PTHH của phản ứng. \textit{Bước 2}: Tính số mol chất đã biết dựa vào khối lượng hoặc thể tích. \textit{Bước 3}: Dựa vào PTHH để tìm số mol chất tham gia hoặc chất sản phẩm. \textit{Bước 4}: Tính khối lượng hoặc thể tích của chất cần tìm. \fbox{\bf 2} \textit{Hiệu suất phản ứng} là tỷ lệ giữa lượng sản phẩm thu được theo thực tế \& lượng sản phẩm thu được theo lý thuyết.
	}%
}

%------------------------------------------------------------------------------%

\section{Nồng Độ Dung Dịch}
\textsf{\textbf{Nội dung.} Dung dịch là hỗn hợp lỏng đồng nhất của các chất đã tan trong nhau, độ tan của 1 chất trong nước, nồng độ \%, nồng độ mol.}

Các dung dịch thường có ghi kèm theo nồng độ xác định như nước muối sinh lý 0.9\%, sulfuric acid 1 mol\texttt{/}L, $\ldots$

Khi hòa chất rắn vào nước, có chất tan nhiều, có chất tan ít, có chất không tan trong nước. Làm thế nào để so sánh khả năng hòa tan trong nước của các chất \& xác định khối lượng chất tan có trong 1 dung dịch?

\subsection{Dung dịch, chất tan, \& dung môi}

\begin{dinhnghia}[Dung dịch]
	\emph{Dung dịch} là hỗn hợp lỏng đồng nhất của chất tan \& dung môi.
\end{dinhnghia}
Trong thực tế, dung môi thường là nước ở thể lỏng, chất tan có thể ở thể rắn, lỏng hoặc khí. Ở nhiệt độ, áp suất nhất định, dung dịch có thể hòa tan thêm chất tan đó được gọi là \textit{dung dịch chưa bão hòa}, dung dịch không thể hòa tan thêm chất tan đó được gọi là \textit{dung dịch bão hòa}.

\begin{vidu}
	Khi cho 1 thìa muối ăn vào cốc nước \& khuấy đều, ta được dung dịch muối ăn, trong đó các hạt muối ăn bị tan ra \& phân bố đều trong nước tạo thành hỗn hợp đồng nhất. Trong quá trình này, muối ăn là \emph{chất tan}, nước là \emph{dung môi}, \& nước muối là \emph{dung dịch}.
\end{vidu}

\begin{baitoan}[\cite{SGK_KHTN_8_KNTTVCS}, p. 20, Nhận biết dung dịch, chất tan, \& dung môi]
	\emph{Chuẩn bị:} nước, muối ăn, sữa bột (hoặc bột sắn, bột gạo, $\ldots$), copper(II) sulfate; cốc thủy tinh, đũa khuấy.
	
	\emph{Tiến hành:} Cho khoảng \emph{20 mL} nước vào 4 cốc thủy tinh, đánh số (1), (2), (3), \& (4). Cho vào cốc (1) 1 thìa (khoảng \emph{3 g}) muối ăn hạt, cốc (2) 1 thìa copper(II) sulfate, cốc (3) 1 thìa sữa bột, cốc (4) 4 thìa muối ăn. Khuấy đều $\approx2$ phút, sau đó để yên. (a) Trong các cốc (1), (2), (3), cốc nào chứa dung dịch? Dựa vào dấu hiệu nào để nhận biết? Chỉ ra chất tan, dung môi trong dung dịch thu được. (b) Phần dung dịch ở cốc (4) có phải là dung dịch bão hòa ở nhiệt độ phòng không? Giải thích.
\end{baitoan}

\begin{baitoan}[\cite{SGK_KHTN_8_KNTTVCS}, p. 20]
	Nêu cách pha dung dịch bão hòa của sodium carbonate \emph{\ce{Na2CO3}} trong nước.
\end{baitoan}

\subsection{Độ tan của 1 chất trong nước}
Trong cùng điều kiện về nhiệt độ \& áp suất, khả năng tan trong cùng 1 dung môi của các chất là khác nhau. Với cùng 1 lượng dung môi xác định, những chất tan tốt cần lượng lớn chất tan để tạo dung dịch bão hòa, còn những chất tan kém chỉ cần 1 lượng nhỏ chất tan đã thu được dung dịch bão hòa. Để đặc trưng cho khả năng tan của mỗi chất, người ta dùng khái niệm \textit{độ tan}.

\begin{dinhnghia}[Độ tan]
	\emph{Độ tan} (ký hiệu là $S$) của 1 chất trong nước là số gam chất đó hòa tan trong \emph{100 g} nước để tạo thành dung dịch bão hòa ở 1 nhiệt độ, áp suất xác định.
\end{dinhnghia}
Các chất khác nhau có độ tan khác nhau.

\begin{vidu}[Độ tan của muối ăn]
	Cho dần muối ăn vào cốc chứa 200 mL nước, khuấy đều cho đến khi muối ăn không thể hòa tan thêm được nữa, tách bỏ chất rắn không tan, ta thu được dung dịch bão hòa. Lượng muối ăn hòa tan trong $100$ gam nước tạo thành dung dịch bão hòa ở $20^\circ$C là $35.9$ gam. Người ta nói \emph{độ tan của muối ăn} là $35.9$ gam trong $100$ gam nước ở $20^\circ$C.
\end{vidu}

\begin{vidu}[Độ tan của NaCl]
	Độ tan của \emph{NaCl} trong nước ở $25^\circ$C là \emph{36 g\texttt{/}100 g \ce{H2O}}.
\end{vidu}

\subsubsection{Cách tính độ tan của 1 chất trong nước}

\begin{baitoan}[\cite{SGK_KHTN_8_Canh_Dieu}, p. 37]
	Tính độ tan của muối potassium chloride \emph{KCl} ở $20^\circ$C, biết $50$ gam nước hòa tan tối đa $17$ gam muối.
\end{baitoan}

\begin{proof}[Giải]
	Ở $20^\circ$C, 50 g nước hòa tan tối đa 17 g KCl. Ở $20^\circ$C, 100 g nước hòa tan tối đa S g KCl. $\Rightarrow S = \frac{17\cdot100}{50} = 17\cdot2 = 34$ g\texttt{/}100 g \ce{H2O}. Vậy độ tan của potassium chloride trong nước ở $20^\circ$C là 34 g\texttt{/}100 g \ce{H2O}.
\end{proof}

\begin{baitoan}[\cite{SGK_KHTN_8_Canh_Dieu}, 1, p. 37]
	Tính độ tan của muối sodium nitrate \emph{\ce{NaNO3}} ở $0^\circ$C, biết để tạo ra dung dịch \emph{\ce{NaNO3}} bão hòa người ta cần hòa tan $14.2$ g muối trong $20$ g nước.
\end{baitoan}
Công thức tính độ tan của 1 chất ở nhiệt độ xác định là \fbox{$S = \frac{m_{\rm ct}\cdot100}{m_{\footnotesize\mbox{nước}}}$ (g\texttt{/}100 g \ce{H2O})}, trong đó: $S$ là \textit{độ tan}, đơn vị g\texttt{/}100 g nước; $m_{\rm ct}$ là \textit{khối lượng của chất tan} được hòa tan trong nước để tạo thành dung dịch bão hòa, có đơn vị là gam; $m_{\footnotesize\mbox{nước}}$ là \textit{khối lượng của nước}, có đơn vị là gam.

\subsubsection{Ảnh hưởng của nhiệt độ đến độ tan của chất rắn trong nước}
Khi tăng nhiệt độ, độ tan của hầu hết các chất rắn như đường, muối ăn, $\ldots$ đều tăng. Có 1 số chất khi tăng nhiệt độ, độ tan lại giảm.

\begin{vidu}[Độ tan của \ce{C12H22O11}]
	Độ tan của đường ăn trong nước ở $30^\circ$ là \emph{216.7 g} trong khi ở $60^\circ$ là \emph{288.8 g}.
\end{vidu}

\begin{baitoan}[\cite{SGK_KHTN_8_Canh_Dieu}, 2, p. 37]
	(a) Có thể hòa tan tối đa bao nhiêu gam đường ăn trong \emph{250 g} nước ở $30^\circ$C? (b) Có thể hòa tan tối đa bao nhiêu gam đường ăn trong \emph{250 g} nước ở $60^\circ$C?
\end{baitoan}

\begin{baitoan}[\cite{SGK_KHTN_8_KNTTVCS}, 1, p. 21]
	Ở nhiệt độ $25^\circ$C, khi cho \emph{12 g} muối X vào \emph{20 g} nước, khuấy kỹ thì còn lại \emph{5 g} muối không tan. Tính độ tan của muối X.
\end{baitoan}

\begin{baitoan}[\cite{SGK_KHTN_8_KNTTVCS}, 2, p. 21]
	Ở nhiệt độ $18^\circ$C, khi hòa tan hết \emph{53 g} \emph{\ce{Na2CO3}} trong \emph{250 g} nước thì được dung dịch bão hòa. Tính độ tan của \emph{\ce{Na2CO3}} trong nước ở nhiệt độ trên.
\end{baitoan}

\begin{vidu}[\cite{SGK_KHTN_8_KNTTVCS}, p. 21]
	(a) Ngày nóng, cá thường ngoi lên phía mặt nước để hô hấp vì độ tan của oxygen trong nước đã bị giảm đi khi nhiệt độ tăng. (b) Trong sản xuất nước ngọt có gas, người ta nén khí carbon dioxide ở áp suất cao để tăng độ tan của khí này trong nước.
\end{vidu}
Nói chung, độ tan của hầu hết chất khí giảm khi nhiệt độ tăng hoặc áp suất giảm.

\subsection{Nồng độ dung dịch}
Để biểu thị lượng chất tan có trong 1 lượng dung môi hoặc lượng dung dịch cụ thể người ta dùng khái niệm \textit{nồng độ dung dịch} (hay: để định lượng 1 dung dịch đặc hay loãng, người ta dùng đại lượng \textit{nồng độ}). Có 2 loại nồng độ dung dịch thường được sử dụng là \textit{nồng độ \%} \& \textit{nồng độ mol}.

\subsubsection{Nồng độ \% của dung dịch}

\begin{dinhnghia}[Nồng độ \%]
	\emph{Nồng độ phần trăm} (ký hiệu là $C$\%) của 1 dung dịch là số gam chất tan có trong \emph{100 g} dung dịch.
\end{dinhnghia}
Công thức tính nồng độ \% của dung dịch là: \fbox{$C\% = \frac{m_{\rm ct}\cdot100}{m_{\rm dd}}$\%}, trong đó: $C$\% là \textit{nồng độ \%} của dung dịch, đơn vị \%; $m_{\rm ct}$ là \textit{khối lượng chất tan}, có đơn vị là gam; $m_{\rm dd}$ là \textit{khối lượng dung dịch}, có đơn vị là gam. Khối lượng dung dịch bằng tổng khối lượng chất tan \& khối lượng dung môi: \fbox{$m_{\rm dd} = m_{\rm ct} + m_{\rm dm}$} với $m_{\rm dm}$ là \textit{khối lượng dung môi}.
\begin{align*}
	\mbox{Khối lượng dung dịch} = \mbox{khối lượng chất tan} + \mbox{khối lượng dung môi}.
\end{align*}

\begin{baitoan}[\cite{SGK_KHTN_8_Canh_Dieu}, Ví dụ 1, p. 38]
	Hòa tan \emph{20 g} đường ăn trong \emph{60 g} nước thu được dung dịch đường. Tính $C$\% của dung dịch đường đó.
\end{baitoan}

\begin{proof}[Giải]
	Khối lượng dung dịch đường: $m_{\rm dd} = m_{\footnotesize\mbox{đường}} + m_{\footnotesize\mbox{nước}} = 20 + 60 = 80$ g. Nồng độ \% của dung dịch: $C\% = \frac{20\cdot100}{80} = 25$\%.
\end{proof}
Nếu biết được nồng độ \% của dung dịch thì ta có thể xác định được khối lượng chất tan \& khối lượng dung dịch theo các biểu thức sau:
\begin{align*}
	\boxed{m_{\rm ct} = \frac{m_{\rm dd}\cdot C\%}{100},\ m_{\rm dd} = \frac{m_{\rm ct}\cdot100}{C\%}.}
\end{align*}

\begin{baitoan}[\cite{SGK_KHTN_8_Canh_Dieu}, Ví dụ 2, p. 38]
	Muốn pha \emph{300 g} dung dịch muối \emph{\ce{CuSO4} 10\%} cần dùng bao nhiêu \emph{g} muối \& bao nhiêu \emph{g} nước?
\end{baitoan}

\begin{proof}[Giải]
	Khối lượng chất tan cần dùng là: $m_{\footnotesize\mbox{muối}} = \frac{m_{\rm dd}\cdot C\%}{100} = \frac{300\cdot10}{100} = 30$ g. Khối lượng nước cần dùng là: $m_{\footnotesize\mbox{nước}} = m_{\rm dd} - m_{\footnotesize\mbox{muối}} = 300 - 30 = 270$ g.
\end{proof}

\begin{baitoan}[\cite{SGK_KHTN_8_Canh_Dieu}, 1, p. 38]
	Dung dịch D-glucose $5$\% được sử dụng trong y tế làm dịch truyền, nhằm cung cấp nước \& năng lượng cho bệnh nhân bị suy nhược cơ thể hoặc sau phẫu thuật. Biết trong 1 chai dịch truyền có chứa \emph{25 g} đường D-glucose. Tính lượng dung dịch \& lượng nước có trong chai dịch truyền đó.
\end{baitoan}

\begin{baitoan}[\cite{SGK_KHTN_8_Canh_Dieu}, 2, p. 38]
	Từ muối ăn, nước, \& những dụng cụ cần thiết, nêu cách pha \emph{500 g} dung dịch nước muối $0.9$\%.
\end{baitoan}

\begin{thinghiem}[Pha chế 100 g dung dịch đường ăn (saccharose) \ce{C12H22O11} 15\%]
	 \emph{Chuẩn bị:} Dụng cụ: Cân điện tử, cốc thủy tinh (loại \emph{250 mL}), đũa thủy tinh. Hóa chất: Đường ăn, nước cất.
	 
	 \emph{Tiến hành:} Bước 1: Cân chính xác \emph{15 g} đường ăn cho vào cốc dung tích \emph{250 mL}. Bước 2: Cân lấy \emph{85 g} nước cất, rồi cho dần vào cốc \& khuấy nhẹ cho tới khi đường tan hết, thu được \emph{100 g} dung dịch đường nồng độ $15$\%.
\end{thinghiem}

\begin{baitoan}[\cite{SGK_KHTN_8_KNTTVCS}, p. 21]
	Dung dịch nước oxy già chứa chất tan hydrogen peroxide \emph{\ce{H2O2}}. Tính khối lượng hydrogen peroxide có trong \emph{20 g} dung dịch nước oxy già $3$\%.
\end{baitoan}

\begin{proof}[Giải]
	Khối lượng hydrogen peroxide có trong 20 g dung dịch nước oxy già 3\%: $m_{\ce{H2O2}} = \frac{m_{\rm dd}\cdot C\%}{100\%} = \frac{20\cdot3}{100} = 0.6$ g.
\end{proof}

\subsubsection{Nồng độ mol của dung dịch}

\begin{dinhnghia}[Nồng độ mol]
	\emph{Nồng độ mol} (ký hiệu là $C_M$) của 1 dung dịch là số mol chất tan có trong \emph{1 L} dung dịch. Đơn vị của nồng độ mol là \emph{mol\texttt{/}L} \& thường được ký hiệu là $M$.
\end{dinhnghia}
Công thức tính nồng độ mol của dung dịch: \fbox{$C_M = \frac{n}{V}$}, trong đó: $C_M$ là \textit{nồng độ mol} của dung dịch, có đơn vị là mol\texttt{/}L \& thường được biễu diễn là M; $n$ là \textit{số mol chất tan}, có đơn vị là mol, $V$ là \textit{thể tích dung dịch}, có đơn vị là lít (L).

\begin{baitoan}[\cite{SGK_KHTN_8_Canh_Dieu}, Ví dụ 3, p. 39]
	Hòa tan hoàn toàn \emph{4.2 g} sodium hydrogencarbonate \emph{\ce{NaHCO3}} trong nước thu được \emph{500 mL} dung dịch. Tính nồng độ của dung dịch này.
\end{baitoan}

\begin{proof}[Giải]
	Số mol của \ce{NaHCO3} có trong dung dịch là: $n_{\rm NaHCO_3} = \frac{4.2}{84} = 0.05$ mol. Nồng độ mol của dung dịch \ce{NaHCO3} là: $C_M = \frac{0.05}{0.5} = 0.1$ M.
\end{proof}
Nếu biết được nồng độ mol của dung dịch ta có thể xác định được số mol chất tan \& thể tích dung dịch theo các biểu thức sau:
\begin{align*}
	\boxed{n = C_MV,\ V = \frac{n}{C_M}.}
\end{align*}

\begin{baitoan}[\cite{SGK_KHTN_8_Canh_Dieu}, 3, p. 39]
	Tính số \emph{g} chất tan cần pha để pha chế \emph{100 mL} dung dịch \emph{\ce{CuSO4} 0.1 M}.
\end{baitoan}
Có nhiều cách khác nhau để biểu thị nồng độ dung dịch. Để thuận tiện cho việc nghiên cứu, ngoài việc sử dụng nồng độ \% \& nồng độ mol, các nhà khoa học còn sử dụng thêm các loại nồng độ khác như \textit{nồng độ đương lượng} \& \textit{nồng độ molan}.

\begin{thinghiem}[Pha chế dung dịch sodium bicarbonate 0.2 M]
	Sodium bicarbonate (hay còn gọi là sodium hydrogencarbonate, \ce{NaHCO3}) là thành phần chính của thuốc muối được sử dụng nhiều trong chế biến thực phẩm, y tế, vệ sinh vật dụng trong gia đình, $\ldots$ Để pha chế 100 mL dung dịch sodium bicarbonate 0.2 M có thể thực hiện theo thí nghiệm sau:
	
	\emph{Chuẩn bị:} Dụng cụ: Cân điện tử, phễu thủy tinh, ống đong, bình tam giác (loại \emph{250 mL}). Hóa chất: \emph{\ce{NaHCO3}}, nước cất.
	
	\emph{Tiến hành:} Bước 1: Cân chính xác \emph{1.68 g} muối \emph{\ce{NaHCO3}} cho vào bình tam giác. Bước 2: Thêm \emph{100 mL} nước cất vào bình tam giác, khuấy đều cho muối tan hết, thu được dung dịch \emph{\ce{NaHCO3} 0.2 M} (1 cách gần đúng có thể coi thể tích dung dịch muối \ce{NaHCO3} là \emph{100 mL}).
\end{thinghiem}

\begin{baitoan}[\cite{SGK_KHTN_8_Canh_Dieu}, p. 40]
	Glucose được tạo ra từ các quá trình chuyển hóa thực phẩm \& là 1 trong các nguồn cung cấp năng lượng chính cho cơ thể chúng ta. Với người bình thường, nồng độ glucose trong máu luôn được duy trì ổn định. Tìm hiểu \& cho biết chỉ số nồng độ glucose trong máu của người bình thường nằm trong khoảng nào. Nếu chỉ số nồng độ glucose trong máu của 1 người lớn hơn mức bình thường thì người đó có nguy cơ mắc bệnh gì?
\end{baitoan}

\begin{baitoan}[\cite{SGK_KHTN_8_KNTTVCS}, p. 22]
	Hòa tan hoàn toàn \emph{1.35 g} copper(II) chloride vào nước, thu được \emph{50 mL} dung dịch. Tính nồng độ mol của dung dịch copper(II) chloride thu được.
\end{baitoan}

\begin{proof}[Giải]
	Số mol chất tan: $n_{\ce{CuCl2}} = \frac{m_{\ce{CuCl2}}}{M_{\ce{CuCl2}}} = \frac{1.35}{135} = 0.01$ mol. Đổi đơn vị: 50 mL $= 0.05$ L. Nồng độ mol dung dịch copper(II) chloride là: $C_{M(\ce{CuCl2})} = \frac{n_{\ce{CuCl2}}}{V} = \frac{0.01}{0.05} = 0.2$ mol\texttt{/}L.
\end{proof}

\begin{baitoan}[\cite{SGK_KHTN_8_KNTTVCS}, 1, p. 22]
	Tính khối lượng \emph{\ce{H2SO4}} có trong \emph{20 g} dung dịch \emph{\ce{H2SO4} 98\%}.
\end{baitoan}

\begin{baitoan}[\cite{SGK_KHTN_8_KNTTVCS}, 2, p. 22]
	Trộn lẫn \emph{2 L} dung dịch urea \emph{0.02 M} (dung dịch A) với \emph{3 L} dung dịch urea \emph{0.1 M} (dung dịch B), thu được \emph{5 L} dung dịch C. (a) Tính số mol urea trong dung dịch A, B, \& C. (b) Tính nồng độ mol của dung dịch C. Nhận xét về giá trị nồng độ mol của dung dịch C so với nồng độ mol của dung dịch A, B.
\end{baitoan}

\subsection{Thực hành pha chế dung dịch theo 1 nồng độ cho trước}

\begin{thinghiem}[\cite{SGK_KHTN_8_KNTTVCS}, p. 23, Pha 100 g dung dịch muối ăn nồng độ 0.9\%]
	\emph{Chuẩn bị:} muối ăn khan, nước cất; cốc thủy tinh, cân, ống đong.
	
	\emph{Tiến hành:} Xác định khối lượng muối ăn $m_1$ \& nước $m_2$ dựa vào công thức: $C\% = \frac{m_{\rm ct}}{m_{\rm dd}}\cdot100\%$. Cân $m_1$ g muối ăn rồi cho vào cốc thủy tinh. Cân $m_2$ g nước cất, rót vào cốc, lắc đều cho muối tan hết.
	
	(a) Tại sao phải dùng muối ăn khan để pha dung dịch? (b) Dung dịch muối ăn nồng độ $0.9$\% có thể được dùng để làm gì?
\end{thinghiem}

\begin{vidu}[Oresol]
	\emph{Oresol} (abbr., Oral Rehydration Solution) là 1 loại dung dịch có tác dụng bù nước \& điện giải. Trong Oresol có 1 số thành phần chính là: sodium chloride, sodium bicarbonate, potassium chloride, glucose. Nồng độ các chất trong 1 loại dung dịch Oresol được WHO \& UNICEF khuyên dùng là \emph{ion sodium: 0.075 mol\texttt{/}L, ion chloride: 0.065 mol\texttt{/}L, ion potassium: 0.020 mol\texttt{/}L, ion citrate: 0.010 mol\texttt{/}L, glucose: 0.075 mol\texttt{/}L, $\ldots$}
\end{vidu}
Có thể pha chế: (a) 1 dung dịch có nồng độ xác định để làm thí nghiệm. (b)  Pha chế dung dịch nước muối 0.9\% (có thể dùng thay nước muối sinh lý trong 1 số trường hợp).

\noindent\fbox{%
	\parbox{\textwidth} (ký hiệu là $C$\%) của 1 dung dịch là số gam chất tan có trong $100$ gam dung dịch: $C\% = \frac{m_{\rm ct}\cdot100\%}{m_{\rm dd}}$ ($C$\%: nồng độ \%, $m_{\rm ct}$: khối lượng chất tan, $m_{\rm dd}$: khối lượng dung dịch). \fbox{\bf 4} \textit{Nồng độ mol} (ký hiệu là $C_M$) của 1 dung dịch là số mol chất tan có trong 1 lít dung dịch. $C_M = \frac{n_{\rm ct}}{V_{\rm dd}}$ mol\texttt{/}L ($C_M$: nồng độ mol, $n_{\rm ct}$: số mol chất tan, $V_{\rm dd}$: thể tích dung dịch).
	}%
}

%------------------------------------------------------------------------------%

\section{Tốc Độ Phản Ứng \& Chất Xúc Tác}

\noindent\fbox{%
	\parbox{\textwidth}{%
		\noindent\textsf{\textbf{Kiến thức cốt lõi.}} \fbox{\bf 1} \textit{Tốc độ phản ứng} là đại lượng chỉ mức độ nhanh hay chậm của 1 phản ứng hóa học. \fbox{\bf 2} Các yếu tố ảnh hưởng đến tốc độ phản ứng: \textit{Diện tích bề mặt tiếp xúc}: Diện tích bề mặt tiếp xúc càng lớn, tốc độ phản ứng càng nhanh. \textit{Nhiệt độ}: Khi tăng nhiệt độ, phản ứng diễn ra với tốc độ nhanh hơn. \textit{Nồng độ}: Nồng độ các chất phản ứng càng cao, tốc độ phản ứng càng nhanh. \textit{Chất xúc tác} làm tăng tốc độ phản ứng nhưng không bị thay đổi cả về lượng \& chất sau phản ứng. \textit{Chất ức chế} làm giảm tốc độ phản ứng.
	}%
}

%------------------------------------------------------------------------------%

\printbibliography[heading=bibintoc]
	
\end{document}