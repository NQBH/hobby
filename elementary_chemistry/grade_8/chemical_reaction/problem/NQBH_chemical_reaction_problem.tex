\documentclass{article}
\usepackage[backend=biber,natbib=true,style=alphabetic,maxbibnames=50]{biblatex}
\addbibresource{/home/nqbh/reference/bib.bib}
\usepackage[utf8]{vietnam}
\usepackage{tocloft}
\renewcommand{\cftsecleader}{\cftdotfill{\cftdotsep}}
\usepackage[colorlinks=true,linkcolor=blue,urlcolor=red,citecolor=magenta]{hyperref}
\usepackage{amsmath,amssymb,amsthm,float,graphicx,mathtools,tikz,tipa}
\usepackage[version=4]{mhchem}
\allowdisplaybreaks
\newtheorem{assumption}{Assumption}
\newtheorem{baitoan}{Bài toán}
\newtheorem{cauhoi}{Câu hỏi}
\newtheorem{conjecture}{Conjecture}
\newtheorem{corollary}{Corollary}
\newtheorem{dangtoan}{Dạng toán}
\newtheorem{definition}{Definition}
\newtheorem{dinhly}{Định lý}
\newtheorem{dinhnghia}{Định nghĩa}
\newtheorem{example}{Example}
\newtheorem{ghichu}{Ghi chú}
\newtheorem{hequa}{Hệ quả}
\newtheorem{hypothesis}{Hypothesis}
\newtheorem{lemma}{Lemma}
\newtheorem{luuy}{Lưu ý}
\newtheorem{nhanxet}{Nhận xét}
\newtheorem{notation}{Notation}
\newtheorem{note}{Note}
\newtheorem{principle}{Principle}
\newtheorem{problem}{Problem}
\newtheorem{proposition}{Proposition}
\newtheorem{question}{Question}
\newtheorem{remark}{Remark}
\newtheorem{theorem}{Theorem}
\newtheorem{vidu}{Ví dụ}
\usepackage[left=1cm,right=1cm,top=5mm,bottom=5mm,footskip=4mm]{geometry}

\title{Problem: Atom, Chemical Element, \textit{\&} Chemical Compound\\Bài Tập: Nguyên Tử, Nguyên Tố Hóa Học, \textit{\&} Hợp Chất Hóa Học}
\author{Nguyễn Quản Bá Hồng\footnote{Independent Researcher, Ben Tre City, Vietnam\\e-mail: \texttt{nguyenquanbahong@gmail.com}; website: \url{https://nqbh.github.io}.}}
\date{\today}

\begin{document}
\maketitle

%------------------------------------------------------------------------------%

\section{Mol, Khối Lượng Mol, Thể Tích Mol của Chất Khí}

\begin{baitoan}[\cite{An_Hoa_Hoc_nang_cao_8_9}, Ví dụ 1, p. 34]
	(a) {2.5 mol} gồm bao nhiêu nguyên{\tt/}phân tử?  (b) {\rm 0.5 mol NaCl} (sodium chloride) gồm bao nhiêu phân tử {\rm NaCl}? 
\end{baitoan}

\begin{baitoan}[\cite{An_Hoa_Hoc_nang_cao_8_9}, Ví dụ 2, p. 34]
	(a) Tính khối lượng của {\rm0.5 mol Na}. (b) Tính khối lượng của {\rm0.2 mol NaOH}.
\end{baitoan}

\begin{baitoan}[\cite{An_Hoa_Hoc_nang_cao_8_9}, Ví dụ 3, p. 34]
	(a) Trong {\rm8.4 g} iron có bao nhiêu mol iron? (b) Tính thể tích của {\rm8 g} khí oxygen. (c) Tính khối lượng của {\rm67.2 L} khí nitrogen.
\end{baitoan}

\begin{baitoan}[\cite{An_Hoa_Hoc_nang_cao_8_9}, Ví dụ 4, p. 35]
	Trong {\rm4.05 g} aluminium {\rm Al}. Tính: (a) Số mol aluminium. (b) Số nguyên tử aluminium.
\end{baitoan}

\begin{baitoan}[\cite{An_Hoa_Hoc_nang_cao_8_9}, 1., p. 35]
	Cho biết tỷ số khối lượng của các nguyên tố {\rm C, S} trong hợp chất carbon disulfide là $\frac{m_{\rm C}}{m_{\rm S}} = \frac{3}{16}$. Tính tỷ lệ số nguyên tử {\rm C,S} trong carbon disulfide, tỷ lệ này có phù hợp với {\rm CTHH} của hợp chất {\rm\ce{CS2}} không?
\end{baitoan}

\begin{baitoan}[\cite{An_Hoa_Hoc_nang_cao_8_9}, 2., p. 35]
	1 oxide của nitrogen có phân tử khối là $108$ \& $\frac{m_{\rm N}}{m_{\rm O}} = \frac{7}{20}$. {\rm CTHH} của oxide? {\sf A.} {\rm\ce{N2O5}}. {\sf B.} {\rm NO}. {\rm C.} {\rm\ce{NO2}}. {\sf D.} {\rm\ce{N2O}}.
\end{baitoan}

\begin{baitoan}[\cite{An_Hoa_Hoc_nang_cao_8_9}, 3., p. 36]
	1 hợp chất tạo bỏi 2 nguyên tố {\rm P, O}, trong đó oxygen chiếm {\rm43.64\%} về khối lượng, biết phân tử khối là $110$. {\rm CTHH} của hợp chất? {\sf A.} {\rm\ce{P2O5}}. {\sf B.} {\rm\ce{P2O3}}. {\rm C.} {\rm PO}. {\sf D.} {\rm\ce{P2O}}.
\end{baitoan}

\begin{baitoan}[\cite{An_Hoa_Hoc_nang_cao_8_9}, 4., p. 36]
	Tính khối lượng của {\rm0.5 mol iron Fe}.
\end{baitoan}

\begin{baitoan}[\cite{An_Hoa_Hoc_nang_cao_8_9}, 5., p. 36]
	(a) Trong {\rm112 g} calcium có bao nhiêu mol calcium? (b) Tính khối lượng của {\rm0.5 mol} acid hydrochloric {\rm HCl}. (c) Trong {\rm49 g} acid sulfuric có bao nhiêu mol {\rm\ce{H2SO4}}?
\end{baitoan}

\begin{baitoan}[\cite{An_Hoa_Hoc_nang_cao_8_9}, 6., p. 37]
	Cho biết {\rm16 g} khí oxygen: (a) Có bao nhiêu mol khí oxygen? (b) Có bao nhiêu phân tử oxygen? (c) Có thể tích bao nhiêu {\rm L} (đktc)?
\end{baitoan}

\begin{baitoan}[\cite{An_Hoa_Hoc_nang_cao_8_9}, 7., p. 37]
	Tính thể tích khí oxygen \& thể tích không khí (đktc) cần thiết để đốt cháy: (a) {\rm1 mol carbon}. (b) {\rm1 mol} phosphor. (c) {\rm1 mol} sulfur (lưu huỳnh). Biết oxygen chiếm {\rm20\%} thể tích không khí.
\end{baitoan}

\begin{baitoan}[\cite{An_Hoa_Hoc_nang_cao_8_9}, 8., p. 38]
	Tính thể tích hỗn hợp gồm {\rm14 g} nitrogen \& {\rm4 g} khí {\rm NO}.
\end{baitoan}

\begin{baitoan}[\cite{An_Hoa_Hoc_nang_cao_8_9}, 9., p. 38]
	Tính số mol nước {\rm\ce{H2O}} có trong {\rm0.8 L} nước. Biết $D = 1$ {\rm g{\tt/}$\rm cm^3$}.
\end{baitoan}

\begin{baitoan}[\cite{An_Hoa_Hoc_nang_cao_8_9}, 10., p. 39]
	Tính số mol, số phân tử sodium hydroxide {\rm NaOH} có trong $0.05\ dm^3$ {\rm NaOH} biết $D = 1.2$ {\rm g{\tt/}$\rm cm^3$}.
\end{baitoan}

\begin{baitoan}[\cite{An_Hoa_Hoc_nang_cao_8_9}, 11., p. 39]
	Tính thể tích của: (a) {\rm14 g} khí nitrogen. (b) Hỗn hợp gồm {\rm2 g} khí hydrogen \& {\rm34 g} khí amoniac {\rm\ce{NH3}}.
\end{baitoan}

\begin{baitoan}[\cite{An_Hoa_Hoc_nang_cao_8_9}, 12., p. 39]
	Tính thể tích \& khối lượng của: (a) {\rm5 mol} nhôm, biết $D_{\rm Al} = 2.7$ {\rm g{\tt/}$\rm cm^3$}.
\end{baitoan}

\begin{baitoan}[\cite{An_Hoa_Hoc_nang_cao_8_9}, 13., p. 39]
	(a) Tính khối lượng của hỗn hợp gồm {\rm5.6 L} khí chlorine \& {\rm11.2 L} khí oxygen. (b) Phân tử đường gồm $12$ nguyên tử {\rm C}, $22$ nguyên tử {\rm H}, \& $11$ nguyên tử {\rm O}. Tính khối lượng mol phân tử \& thành phần {\rm\%} các nguyên tố của đường.
\end{baitoan}

\begin{baitoan}[\cite{An_Hoa_Hoc_nang_cao_8_9}, 14., p. 39]
	Tính số phân tử, khối lượng, \& thể tích của các lượng chất: (a) {\rm0.2 mol} khí {\rm\ce{CO2}}. (b) {\rm2 mol Fe} biết $D_{\rm Fe} = 7.8$ {\rm g{\tt/}$\rm cm^3$}. (c) {\rm0.5 mol} khí hydrocarbon {\rm HCl}. (d) {\rm0.2 mol} rượu ethylic. Biết $D = 0.8$ {\rm g{\tt/}$\rm cm^3$}.
\end{baitoan}

\begin{baitoan}[\cite{An_Hoa_Hoc_nang_cao_8_9}, 15., p. 39]
	Trong phân tử zinc oxide {\rm ZnO} cứ $16$ phân tử khối lượng của oxygen thì có $65.38$ phần khối lượng zinc. Tìm nguyên tử khối của zinc.
\end{baitoan}

\begin{baitoan}[\cite{An_Hoa_Hoc_nang_cao_8_9}, 16., p. 39]
	Trong vỏ Trái Đất hydrogen chiếm {\rm1\%} về khối lượng \& silicon chiếm {\rm26\%}. Hỏi số nguyên tử của nguyên tố nào có nhiều hơn trong vỏ Trái Đất.
\end{baitoan}

\begin{baitoan}[\cite{An_Hoa_Hoc_nang_cao_8_9}, 17., p. 39]
	Tìm khối lượng mol phân tử của 1 chất khí biết $\rm400\ cm^3$ chất khí đó có khối lượng {\rm1.143 g}.
\end{baitoan}

%------------------------------------------------------------------------------%

\section{1 Số Định Luật Hóa Học Cơ Bản. Các Loại Phản Ứng Hóa Học. Phương Trình Hóa Học}

%------------------------------------------------------------------------------%

\printbibliography[heading=bibintoc]

\end{document}