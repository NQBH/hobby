\documentclass{article}
\usepackage[backend=biber,natbib=true,style=alphabetic,maxbibnames=50]{biblatex}
\addbibresource{/home/nqbh/reference/bib.bib}
\usepackage[utf8]{vietnam}
\usepackage{tocloft}
\renewcommand{\cftsecleader}{\cftdotfill{\cftdotsep}}
\usepackage[colorlinks=true,linkcolor=blue,urlcolor=red,citecolor=magenta]{hyperref}
\usepackage{amsmath,amssymb,amsthm,float,graphicx,mathtools,tikz,tipa}
\usepackage[version=4]{mhchem}
\allowdisplaybreaks
\newtheorem{assumption}{Assumption}
\newtheorem{baitoan}{Bài toán}
\newtheorem{cauhoi}{Câu hỏi}
\newtheorem{conjecture}{Conjecture}
\newtheorem{corollary}{Corollary}
\newtheorem{dangtoan}{Dạng toán}
\newtheorem{definition}{Definition}
\newtheorem{dinhly}{Định lý}
\newtheorem{dinhnghia}{Định nghĩa}
\newtheorem{example}{Example}
\newtheorem{ghichu}{Ghi chú}
\newtheorem{hequa}{Hệ quả}
\newtheorem{hypothesis}{Hypothesis}
\newtheorem{lemma}{Lemma}
\newtheorem{luuy}{Lưu ý}
\newtheorem{nhanxet}{Nhận xét}
\newtheorem{notation}{Notation}
\newtheorem{note}{Note}
\newtheorem{principle}{Principle}
\newtheorem{problem}{Problem}
\newtheorem{proposition}{Proposition}
\newtheorem{question}{Question}
\newtheorem{remark}{Remark}
\newtheorem{theorem}{Theorem}
\newtheorem{vidu}{Ví dụ}
\usepackage[left=1cm,right=1cm,top=5mm,bottom=5mm,footskip=4mm]{geometry}

\title{Problem: Atom, Chemical Element, \textit{\&} Chemical Compound\\Bài Tập: Nguyên Tử, Nguyên Tố Hóa Học, \textit{\&} Hợp Chất Hóa Học}
\author{Nguyễn Quản Bá Hồng\footnote{Independent Researcher, Ben Tre City, Vietnam\\e-mail: \texttt{nguyenquanbahong@gmail.com}; website: \url{https://nqbh.github.io}.}}
\date{\today}

\begin{document}
\maketitle

%------------------------------------------------------------------------------%

\section{Mol, Khối Lượng Mol, Thể Tích Mol của Chất Khí}

\begin{baitoan}[\cite{An_Hoa_Hoc_nang_cao_8_9}, Ví dụ 1, p. 34]
	(a) {2.5 mol} gồm bao nhiêu nguyên{\tt/}phân tử?  (b) {\rm 0.5 mol NaCl} (sodium chloride) gồm bao nhiêu phân tử {\rm NaCl}? 
\end{baitoan}

\begin{baitoan}[\cite{An_Hoa_Hoc_nang_cao_8_9}, Ví dụ 2, p. 34]
	(a) Tính khối lượng của {\rm0.5 mol Na}. (b) Tính khối lượng của {\rm0.2 mol NaOH}.
\end{baitoan}

\begin{baitoan}[\cite{An_Hoa_Hoc_nang_cao_8_9}, Ví dụ 3, p. 34]
	(a) Trong {\rm8.4 g} iron có bao nhiêu mol iron? (b) Tính thể tích của {\rm8 g} khí oxygen. (c) Tính khối lượng của {\rm67.2 L} khí nitrogen.
\end{baitoan}

\begin{baitoan}[\cite{An_Hoa_Hoc_nang_cao_8_9}, Ví dụ 4, p. 35]
	Trong {\rm4.05 g} aluminium {\rm Al}. Tính: (a) Số mol aluminium. (b) Số nguyên tử aluminium.
\end{baitoan}

\begin{baitoan}[\cite{An_Hoa_Hoc_nang_cao_8_9}, 1., p. 35]
	Cho biết tỷ số khối lượng của các nguyên tố {\rm C, S} trong hợp chất carbon disulfide là $\frac{m_{\rm C}}{m_{\rm S}} = \frac{3}{16}$. Tính tỷ lệ số nguyên tử {\rm C,S} trong carbon disulfide, tỷ lệ này có phù hợp với {\rm CTHH} của hợp chất {\rm\ce{CS2}} không?
\end{baitoan}

\begin{baitoan}[\cite{An_Hoa_Hoc_nang_cao_8_9}, 2., p. 35]
	1 oxide của nitrogen có phân tử khối là $108$ \& $\frac{m_{\rm N}}{m_{\rm O}} = \frac{7}{20}$. {\rm CTHH} của oxide? {\sf A.} {\rm\ce{N2O5}}. {\sf B.} {\rm NO}. {\rm C.} {\rm\ce{NO2}}. {\sf D.} {\rm\ce{N2O}}.
\end{baitoan}

\begin{baitoan}[\cite{An_Hoa_Hoc_nang_cao_8_9}, 3., p. 36]
	1 hợp chất tạo bỏi 2 nguyên tố {\rm P, O}, trong đó oxygen chiếm {\rm43.64\%} về khối lượng, biết phân tử khối là $110$. {\rm CTHH} của hợp chất? {\sf A.} {\rm\ce{P2O5}}. {\sf B.} {\rm\ce{P2O3}}. {\rm C.} {\rm PO}. {\sf D.} {\rm\ce{P2O}}.
\end{baitoan}

\begin{baitoan}[\cite{An_Hoa_Hoc_nang_cao_8_9}, 4., p. 36]
	Tính khối lượng của {\rm0.5 mol iron Fe}.
\end{baitoan}

\begin{baitoan}[\cite{An_Hoa_Hoc_nang_cao_8_9}, 5., p. 36]
	(a) Trong {\rm112 g} calcium có bao nhiêu mol calcium? (b) Tính khối lượng của {\rm0.5 mol} acid hydrochloric {\rm HCl}. (c) Trong {\rm49 g} acid sulfuric có bao nhiêu mol {\rm\ce{H2SO4}}?
\end{baitoan}

\begin{baitoan}[\cite{An_Hoa_Hoc_nang_cao_8_9}, 6., p. 37]
	Cho biết {\rm16 g} khí oxygen: (a) Có bao nhiêu mol khí oxygen? (b) Có bao nhiêu phân tử oxygen? (c) Có thể tích bao nhiêu {\rm L} (đktc)?
\end{baitoan}

\begin{baitoan}[\cite{An_Hoa_Hoc_nang_cao_8_9}, 7., p. 37]
	Tính thể tích khí oxygen \& thể tích không khí (đktc) cần thiết để đốt cháy: (a) {\rm1 mol carbon}. (b) {\rm1 mol} phosphor. (c) {\rm1 mol} sulfur (lưu huỳnh). Biết oxygen chiếm {\rm20\%} thể tích không khí.
\end{baitoan}

\begin{baitoan}[\cite{An_Hoa_Hoc_nang_cao_8_9}, 8., p. 38]
	Tính thể tích hỗn hợp gồm {\rm14 g} nitrogen \& {\rm4 g} khí {\rm NO}.
\end{baitoan}

\begin{baitoan}[\cite{An_Hoa_Hoc_nang_cao_8_9}, 9., p. 38]
	Tính số mol nước {\rm\ce{H2O}} có trong {\rm0.8 L} nước. Biết $D = 1$ {\rm g{\tt/}$\rm cm^3$}.
\end{baitoan}

\begin{baitoan}[\cite{An_Hoa_Hoc_nang_cao_8_9}, 10., p. 39]
	Tính số mol, số phân tử sodium hydroxide {\rm NaOH} có trong $0.05\ dm^3$ {\rm NaOH} biết $D = 1.2$ {\rm g{\tt/}$\rm cm^3$}.
\end{baitoan}

\begin{baitoan}[\cite{An_Hoa_Hoc_nang_cao_8_9}, 11., p. 39]
	Tính thể tích của: (a) {\rm14 g} khí nitrogen. (b) Hỗn hợp gồm {\rm2 g} khí hydrogen \& {\rm34 g} khí amoniac {\rm\ce{NH3}}.
\end{baitoan}

\begin{baitoan}[\cite{An_Hoa_Hoc_nang_cao_8_9}, 12., p. 39]
	Tính thể tích \& khối lượng của: (a) {\rm5 mol} nhôm, biết $D_{\rm Al} = 2.7$ {\rm g{\tt/}$\rm cm^3$}.
\end{baitoan}

\begin{baitoan}[\cite{An_Hoa_Hoc_nang_cao_8_9}, 13., p. 39]
	(a) Tính khối lượng của hỗn hợp gồm {\rm5.6 L} khí chlorine \& {\rm11.2 L} khí oxygen. (b) Phân tử đường gồm $12$ nguyên tử {\rm C}, $22$ nguyên tử {\rm H}, \& $11$ nguyên tử {\rm O}. Tính khối lượng mol phân tử \& thành phần {\rm\%} các nguyên tố của đường.
\end{baitoan}

\begin{baitoan}[\cite{An_Hoa_Hoc_nang_cao_8_9}, 14., p. 39]
	Tính số phân tử, khối lượng, \& thể tích của các lượng chất: (a) {\rm0.2 mol} khí {\rm\ce{CO2}}. (b) {\rm2 mol Fe} biết $D_{\rm Fe} = 7.8$ {\rm g{\tt/}$\rm cm^3$}. (c) {\rm0.5 mol} khí hydrocarbon {\rm HCl}. (d) {\rm0.2 mol} rượu ethylic. Biết $D = 0.8$ {\rm g{\tt/}$\rm cm^3$}.
\end{baitoan}

\begin{baitoan}[\cite{An_Hoa_Hoc_nang_cao_8_9}, 15., p. 39]
	Trong phân tử zinc oxide {\rm ZnO} cứ $16$ phân tử khối lượng của oxygen thì có $65.38$ phần khối lượng zinc. Tìm nguyên tử khối của zinc.
\end{baitoan}

\begin{baitoan}[\cite{An_Hoa_Hoc_nang_cao_8_9}, 16., p. 39]
	Trong vỏ Trái Đất hydrogen chiếm {\rm1\%} về khối lượng \& silicon chiếm {\rm26\%}. Hỏi số nguyên tử của nguyên tố nào có nhiều hơn trong vỏ Trái Đất.
\end{baitoan}

\begin{baitoan}[\cite{An_Hoa_Hoc_nang_cao_8_9}, 17., p. 39]
	Tìm khối lượng mol phân tử của 1 chất khí biết $\rm400\ cm^3$ chất khí đó có khối lượng {\rm1.143 g}.
\end{baitoan}

%------------------------------------------------------------------------------%

\section{1 Số Định Luật Hóa Học Cơ Bản. Các Loại Phản Ứng Hóa Học. Phương Trình Hóa Học}

\begin{baitoan}[\cite{An_Hoa_Hoc_nang_cao_8_9}, Ví dụ, p. 43]
	Cho {\rm50 g NaOH} tác dụng với {\rm36.5 g HCl}. Tính khối lượng muối tạo thành sau phản ứng.
\end{baitoan}

\begin{baitoan}[\cite{An_Hoa_Hoc_nang_cao_8_9}, 1., p. 44]
	1 em học sinh làm 3 thí nghiệm với chất rắn bicarbonate (thuốc muối trị đầy hơi màu trắng).
	\begin{itemize}
		\item Thí nghiệm 1: Hòa tan 1 ít thuốc muối rắn trên vào nước được dung dịch trong suốt.
		\item Thí nghiệm 2: Hòa tan 1 ít thuốc muối rắng trên vào nước chanh hoặc giấm thấy sủi bọt mạnh.
		\item Thí nghiệm 3: Đun nóng 1 ít chất rắn trên trong ống nghiệm, màu trắng không đổi nhưng thoát ra 1 chất khí làm đục nước vôi trong.
	\end{itemize}
	Thí nghiệm nào có sự biến đổi hóa học? Giải thích.
\end{baitoan}

\begin{baitoan}[\cite{An_Hoa_Hoc_nang_cao_8_9}, 2., p. 44]
	Đốt bột aluminium cháy theo phản ứng: aluminium $+$ khí oxygen $\to$ aluminium oxide {\rm\ce{Al2O3}}. Cho biết khối lượng aluminium đã phản ứng là {\rm54 g} \& khối lượng aluminium oxide sinh ra là {\rm102 g}. Tính khối lượng oxygen đã dùng.
\end{baitoan}

\begin{baitoan}[\cite{An_Hoa_Hoc_nang_cao_8_9}, 3., p. 45]
	Đốt {\rm58 g} khí butan {\rm\ce{C4H10}} cần dùng {\rm208 g} khí oxygen \& tạo ra {\rm90 g} hơi nước \& khí carbonic {\rm\ce{CO2}}, khối lượng {\rm\ce{CO2}} sinh ra là: {\sf A.} {\rm98 g}. {\sf B.} {\rm176 g}. {\sf C.} {\rm200 g}. {\sf D.} {\rm264 g}.
\end{baitoan}

\begin{baitoan}[\cite{An_Hoa_Hoc_nang_cao_8_9}, 4., p. 45]
	Nung hỗn hợp gồm 2 muối {\rm\ce{CaCO3,MgCO3}} thu được {\rm76 g} 2 oxide \& {\rm66 g \ce{CO2}}. Tính khối lượng hỗn hợp 2 muối ban đầu.
\end{baitoan}

\begin{baitoan}[\cite{An_Hoa_Hoc_nang_cao_8_9}, 5., p. 45]
	Lấy cùng 1 lượng {\rm\ce{KClO3,KMnO4}} để điều chế khí oxygen. Chất nào cho nhiều khí oxygen hơn? Viết {\rm PTHH} \& giải thích.
\end{baitoan}

\begin{baitoan}[\cite{An_Hoa_Hoc_nang_cao_8_9}, 6., p. 46]
	Trên đĩa cân, ở vị trí cân bằng, có đặt 1 cốc có dung tích là {\rm0.5 L}. Sau đó, dùng khí carbonic {\rm\ce{CO2}} để đẩy không khí khỏi cốc đó. Phải đặt thêm vào đĩa cân bên kia quả cân bao nhiêu để cân thăng bằng trở lại? Biết khí {\rm\ce{CO2}} nặng gấp $1.5$ lần không khí, thể tích khí {\rm\ce{CO2}} (đktc).
\end{baitoan}

\begin{baitoan}[\cite{An_Hoa_Hoc_nang_cao_8_9}, 7., p. 46]
	Viết {\rm PTHH} các chất sau với than: (a) Iron (III) oxide {\rm\ce{Fe2O3}}. (b) Zinc oxide {\rm ZnO}.
\end{baitoan}

\begin{baitoan}[\cite{An_Hoa_Hoc_nang_cao_8_9}, 8., p. 46]
	Viết {\rm PTHH} điều chế: {\rm Sn} từ {\rm\ce{SnO2}, Fe} từ {\rm\ce{Fe3O4}}, chỉ từ {\rm\ce{PbO2}} khi dùng carbon oxide làm chất khử.
\end{baitoan}

\begin{baitoan}[\cite{An_Hoa_Hoc_nang_cao_8_9}, 9., p. 46]
	Cần bao nhiêu carbon dioxide tham gia phản ứng với $160$ tấn {\rm\ce{Fe2O3}}? Biết áu phản ứng có iron \& khí carbonic tạo thành.
\end{baitoan}

\begin{baitoan}[\cite{An_Hoa_Hoc_nang_cao_8_9}, 10., p. 46]
	Điều chế vôi sống bằng cách nung đá vôi {\rm\ce{CaCO3}}. Tính lượng vôi sống thu được từ $1$ tấn đá vôi có chứa {\rm10\%} tạp chất.
\end{baitoan}

\begin{baitoan}[\cite{An_Hoa_Hoc_nang_cao_8_9}, 11., p. 48]
	{\rm7 g} Lithium đẩy được {\rm1 g} hydrogen ra khỏi nước. Xác định hóa trị của lithium trong hợp chất tạo thành sau phản ứng.
\end{baitoan}

\begin{baitoan}[\cite{An_Hoa_Hoc_nang_cao_8_9}, 12., p. 48]
	(a) Cho biết sơ đồ của phản ứng phân hủy mercury (thủy ngân) oxide: {\rm\ce{HgO -> Hg + O2}}. Tính khối lượng khí oxygen sinh ra khi có {\rm8 mol HgO} tham gia phản ứng. (b) Tính khối lượng mercury sinh ra khí có {\rm434 g HgO} tham gia phản ứng. (c) Tính khối lượng mercury oxide đã được phân hủy khi có {\rm150.75 g Hg} sinh ra.
\end{baitoan}

\begin{baitoan}[\cite{An_Hoa_Hoc_nang_cao_8_9}, 13., p. 49]
	Khi phân hủy {\rm100 g} mẫu quặng zinc, thu được {\rm32.5 g} zinc. Tính thành phần {\rm\%} của {\rm ZnS} trong quặng đó. Biết {\rm Zn} trong quặng chỉ ở dạng sulfide {\rm ZnS}.
\end{baitoan}

\begin{baitoan}[\cite{An_Hoa_Hoc_nang_cao_8_9}, 14., p. 49]
	Phản ứng phân hủy \& phản ứng hóa hợp khác nhau thế nào? Đối với mỗi loại phản ứng, dẫn ra 1 ví dụ để minh họa.
\end{baitoan}

\begin{baitoan}[\cite{An_Hoa_Hoc_nang_cao_8_9}, 15., p. 50]
	Phản ứng của copper oxide bị khử bởi khí hydrogen thuộc loại phản ứng gì? Tính số {\rm g CuO} bị khử hết bởi {\rm4 g} hydrogen?
\end{baitoan}

\begin{baitoan}[\cite{An_Hoa_Hoc_nang_cao_8_9}, 16., p. 50]
	Trong phòng thí nghiệm, điều chế iron oxide từ {\rm\ce{Fe3O4}} bằng cách oxy hóa iron ở nhiệt độ cao. (a) Tính số {\rm g} iron \& oxygen cần dùng để có thể điều chế được {\rm2.32 g} oxide sắt từ. (b) Tính số {\rm g} potassium manganate (VII) (hay kali pemanganat) {\rm\ce{KMnO4}} cần dùng để có được lượng oxygen dùng cho phản ứng trên, biết khi đun nóng {\rm2 mol \ce{KMnO4}} thoát ra {\rm1 mol \ce{O2}}.
\end{baitoan}

%------------------------------------------------------------------------------%

\printbibliography[heading=bibintoc]

\end{document}