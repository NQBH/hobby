\documentclass{article}
\usepackage[backend=biber,natbib=true,style=alphabetic,maxbibnames=50]{biblatex}
\addbibresource{/home/nqbh/reference/bib.bib}
\usepackage[utf8]{vietnam}
\usepackage{tocloft}
\renewcommand{\cftsecleader}{\cftdotfill{\cftdotsep}}
\usepackage[colorlinks=true,linkcolor=blue,urlcolor=red,citecolor=magenta]{hyperref}
\usepackage{amsmath,amssymb,amsthm,float,graphicx,mathtools,diagbox,tikz,tipa}
\usepackage[version=4]{mhchem}
\allowdisplaybreaks
\newtheorem{assumption}{Assumption}
\newtheorem{baitoan}{Bài toán}
\newtheorem{cauhoi}{Câu hỏi}
\newtheorem{conjecture}{Conjecture}
\newtheorem{corollary}{Corollary}
\newtheorem{dangtoan}{Dạng toán}
\newtheorem{definition}{Definition}
\newtheorem{dinhly}{Định lý}
\newtheorem{dinhnghia}{Định nghĩa}
\newtheorem{example}{Example}
\newtheorem{ghichu}{Ghi chú}
\newtheorem{hequa}{Hệ quả}
\newtheorem{hypothesis}{Hypothesis}
\newtheorem{lemma}{Lemma}
\newtheorem{luuy}{Lưu ý}
\newtheorem{nhanxet}{Nhận xét}
\newtheorem{notation}{Notation}
\newtheorem{note}{Note}
\newtheorem{principle}{Principle}
\newtheorem{problem}{Problem}
\newtheorem{proposition}{Proposition}
\newtheorem{question}{Question}
\newtheorem{remark}{Remark}
\newtheorem{theorem}{Theorem}
\newtheorem{thinghiem}{Thí nghiệm}
\newtheorem{vidu}{Ví dụ}
\usepackage[left=1cm,right=1cm,top=5mm,bottom=5mm,footskip=4mm]{geometry}

\title{Problem: Concentration -- Bài Tập Nồng Độ Dung Dịch}
\author{Nguyễn Quản Bá Hồng\footnote{Independent Researcher, Ben Tre City, Vietnam\\e-mail: \texttt{nguyenquanbahong@gmail.com}; website: \url{https://nqbh.github.io}.}}
\date{\today}

\begin{document}
\maketitle
\begin{abstract}
	
\end{abstract}
\setcounter{secnumdepth}{4}
\setcounter{tocdepth}{3}
\tableofcontents

%------------------------------------------------------------------------------%

\begin{baitoan}[\cite{An_Hoa_Hoc_nang_cao_8_9}, 1., p. 103]
	(a) Chuyển sang nồng độ {\rm\%} dung dịch {\rm NaOH 2M} có khối lượng riêng $D = 1.08$ {\rm g{\tt/}mL}. (b) Cần bao nhiêu {\rm g NaOH} để pha chế được {\rm3 L} dung dịch {\rm NaOH 10\%} biết khối lượng riêng của dung dịch là {\rm1.115 g{\tt/}mL}.
\end{baitoan}

\begin{baitoan}[\cite{An_Hoa_Hoc_nang_cao_8_9}, 2., p. 103]
	Phải thêm bao nhiêu {\rm L} nước vào {2 L} dung dịch {\rm NaOH 1M} để thu được dung dịch có nồng độ {\rm0.1M}?
\end{baitoan}

\begin{baitoan}[\cite{An_Hoa_Hoc_nang_cao_8_9}, 3., p. 104]
	Hòa tan {\rm5.72 g \ce{Na2CO3.$10$H2O}} (soda tinh thể) vào {\rm44.28 mL} nước. Xác định nồng độ {\rm\%} của dung dịch.
\end{baitoan}

\begin{baitoan}[\cite{An_Hoa_Hoc_nang_cao_8_9}, 4., p. 104]
	Cho thêm nước vào {\rm150 g} dung dịch acid {\rm HCl} nồng độ {\rm2.65\%} để tạo {\rm2 L} dung dịch. Tính nồng độ M của dung dịch thu được
\end{baitoan}

\begin{baitoan}[\cite{An_Hoa_Hoc_nang_cao_8_9}, 5., p. 104]
	Cho sản phẩm thu được khi oxy hóa hoàn toàn {\rm8 L} khí sulfur trioxide (đktc) vào trong {\rm57.2 mL} dung dịch {\rm\ce{H2SO4} 60\%} có $D = 1.5$ {\rm g{\tt/}mL}. Tính nồng độ \% của dung dịch acid thu được.
\end{baitoan}

\begin{baitoan}[\cite{An_Hoa_Hoc_nang_cao_8_9}, 6., p. 105]
	Xác định khối lượng {\rm NaCl} kết tinh trở lại khi làm lạnh {\rm548 g} dung dịch {\rm NaCl} bão hòa ở $50^\circ${\rm C} xuống còn $0^\circ${\rm C}. Biết độ tan của {\rm NaCl} ở $50^\circ${\rm C} là {\rm37 g} \& $0^\circ${\rm C} là {\rm35 g}.
\end{baitoan}

\begin{baitoan}[\cite{An_Hoa_Hoc_nang_cao_8_9}, 7., p. 106]
	Cho $V_1$ {\rm L} dung dịch chứa {\rm7.3 g HCl} (dung dịch A) \& $V_2$ {\rm L} dung dịch chứa {\rm58.4 g HCl} (dung dịch B). Trộn dung dịch A với dung dịch B, ta được dung dịch mới (dung dịch C). Thể tích dung dịch C bằng $V_1 + V_2 = 3$ {\rm L}. (a) Tính nồng độ mol của dung dịch C. (b) Tính nồng độ mol của dung dịch A \& dung dịch B. Biết hiệu số nồng độ $C_{\rm M,A} - C_{\rm M,B} = 0.6${\rm M}.
\end{baitoan}

\begin{baitoan}[\cite{An_Hoa_Hoc_nang_cao_8_9}, 8., p. 106]
	Cho {\rm6.72 L} khí {\rm\ce{SO2}} vào {\rm200 mL} dung dịch {\rm\ce{Ca(OH)2} 1M}. Tính khối lượng muối tạo thành.
\end{baitoan}

\begin{baitoan}[\cite{An_Hoa_Hoc_nang_cao_8_9}, 9., p. 106]
	Ở $25^\circ${\rm C} có {\rm175 g} dung dịch {\rm\ce{CuSO4}} bão hòa. Đun nóng dung dịch lên $90^\circ${\rm C}, phải thêm bao nhiêu {\rm g \ce{CuSO4}} để được dung dịch bão hòa ở nhiệt độ này? Biết $S_{\ce{CuSO4}}$ ở $25^\circ${\rm C} là {\rm40 g}, $S_{\ce{CuSO4}}$ ở $90^\circ${\rm C} là {\rm80 g}.
\end{baitoan}

\begin{baitoan}[\cite{An_Hoa_Hoc_nang_cao_8_9}, 10., p. 106]
	Cho {\rm365 g} dung dịch {\rm HCl} tác dụng vừa đủ với {\rm307 g} dung dịch {\rm\ce{Na2CO3}}. Sau phản ứng thu được dung dịch muối có nồng độ {\rm9\%}. Xác định nồng độ {\rm\%} của dung dịch {\rm HCl} \& dung dịch {\rm\ce{Na2CO3}}.
\end{baitoan}

\begin{baitoan}[\cite{An_Hoa_Hoc_nang_cao_8_9}, 11., p. 106]
	Tính khối lượng tinh thể {\rm\ce{CuSO4.$5$H2O}} tách ra khi làm lạnh {\rm1877 g} dung dịch {\rm\ce{CuSO4.$5$H2O}} ở $85^\circ${\rm C} xuống còn $12^\circ${\rm C}. Biết $S_{\ce{CuSO4}}$ ở $85^\circ${\rm C} là {\rm87.7 g}, $S_{\ce{CuSO4}}$ ở $12^\circ${\rm C} là {\rm35.5 g}.
\end{baitoan}

\begin{baitoan}[\cite{An_Hoa_Hoc_nang_cao_8_9}, 12., p. 107]
	Có {\rm16 mL} dung dịch {\rm HCl} nồng độ {\rm1.25M} (dung dịch A). (a) Cần phải thêm bao nhiêu {\rm mL} nước vào dung dịch A để được dung dịch {\rm HCl} có nồng độ {\rm0.25M}? (b) Nếu trộn dung dịch A với {\rm80 mL} dung dịch {\rm HCl} nồng độ $a$ {\rm mol{\tt/}L} thì cũng được dung dịch có nồng độ {\rm0.25M}. Xác định $a$.
\end{baitoan}

\begin{baitoan}[\cite{An_Hoa_Hoc_nang_cao_8_9}, 13., p. 107]
	Cho dung dịch acid acetic nồng độ $x${\rm\%} tác dụng với dung dịch {\rm NaOH} nồng độ {\rm10\%}, thu được dung dịch muối nồng độ {\rm10.25\%}. Tính nồng độ $x${\rm\%}.
\end{baitoan}

\begin{baitoan}[\cite{An_Hoa_Hoc_nang_cao_8_9}, 14., p. 107]
	Dung dịch A là dung dịch {\rm\ce{H2SO4}}, dung dịch B là dung dịch {\rm NaOH}. Trộn A \& B theo tỷ lệ thể tích $V_A:V_B = 3:2$ thì dung dịch C có chứa A dư. Trung hòa {\rm1 L} dung dịch C cần {\rm40 g} dung dịch {\rm KOH 28\%}. Trộn A \& B theo tỷ lệ thể tích $V_A:V_B = 2:3$ thì được dung dịch D có chứa B dư. Trung hòa {\rm1 L} dung dịch D cần {\rm29.2 g} dung dịch {\rm HCl 25\%}. Tính nồng độ mol của A \& B.
\end{baitoan}

\begin{baitoan}[\cite{An_Hoa_Hoc_nang_cao_8_9}, 15., p. 107]
	Hòa tan 1 lượng muối carbonate của 1 kim loại hóa trị II bằng dung dịch {\rm\ce{H2SO4} 14.7\%}. Sau khi phản ứng xảy ra hoàn toàn, thì được dung dịch chứa {\rm17\%} muối sulfate tan. Xác định {\rm CTPT} muối carbonate.
\end{baitoan}

\begin{baitoan}[\cite{An_Hoa_Hoc_nang_cao_8_9}, 16., p. 107]
	Hòa tan hoàn toàn {\rm14.2 g} hỗn hợp A gồm 2 muối là {\rm\ce{MgCO3}} \& muối carbonate của kim loại R vào dung dịch acid {\rm HCl 7.3\%} vừa đủ thu được dung dịch B \& {3.36 L} khí {\rm\ce{CO2}} (đktc). Nồng độ {\rm\ce{MgCl2}} trong dung dịch B bằng {\rm6.028\%}. Xác định kim loại R. Biết hóa trị của kim loại từ I$\to$III.
\end{baitoan}

\begin{baitoan}[\cite{An_Hoa_Hoc_nang_cao_8_9}, 17., p. 107]
	Khi làm lạnh {\rm600 g} dung dịch {\rm NaCl} bão hòa ở $90^\circ${\rm C} tới $0^\circ${\rm C} thì lượng dung dịch thu được là bao nhiêu? Biết $S_{\rm NaCl}$ ở $90^\circ${\rm C} là {\rm50 g}, $S_{\rm NaCl}$ ở $0^\circ${\rm C} là {\rm35 g}.
\end{baitoan}

%------------------------------------------------------------------------------%

\printbibliography[heading=bibintoc]

\end{document}