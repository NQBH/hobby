\documentclass{article}
\usepackage[backend=biber,natbib=true,style=authoryear,maxbibnames=10]{biblatex}
\addbibresource{/home/nqbh/reference/bib.bib}
\usepackage[utf8]{vietnam}
\usepackage{tocloft}
\renewcommand{\cftsecleader}{\cftdotfill{\cftdotsep}}
\usepackage[colorlinks=true,linkcolor=blue,urlcolor=red,citecolor=magenta]{hyperref}
\usepackage{amsmath,amssymb,amsthm,float,graphicx,mathtools,tikz}
\usepackage[version=4]{mhchem}
\allowdisplaybreaks
\newtheorem{assumption}{Assumption}
\newtheorem{baitoan}{Bài toán}
\newtheorem{cauhoi}{Câu hỏi}
\newtheorem{conjecture}{Conjecture}
\newtheorem{corollary}{Corollary}
\newtheorem{dangtoan}{Dạng toán}
\newtheorem{definition}{Definition}
\newtheorem{dinhly}{Định lý}
\newtheorem{dinhnghia}{Định nghĩa}
\newtheorem{example}{Example}
\newtheorem{ghichu}{Ghi chú}
\newtheorem{hequa}{Hệ quả}
\newtheorem{hypothesis}{Hypothesis}
\newtheorem{lemma}{Lemma}
\newtheorem{luuy}{Lưu ý}
\newtheorem{nhanxet}{Nhận xét}
\newtheorem{notation}{Notation}
\newtheorem{note}{Note}
\newtheorem{principle}{Principle}
\newtheorem{problem}{Problem}
\newtheorem{proposition}{Proposition}
\newtheorem{question}{Question}
\newtheorem{remark}{Remark}
\newtheorem{theorem}{Theorem}
\newtheorem{vidu}{Ví dụ}
\usepackage[left=1cm,right=1cm,top=5mm,bottom=5mm,footskip=4mm]{geometry}
\def\labelitemii{$\circ$}

\title{Hydrogen, Water -- Hiđro, Nước}
\author{Nguyễn Quản Bá Hồng\footnote{Independent Researcher, Ben Tre City, Vietnam\\e-mail: \texttt{nguyenquanbahong@gmail.com}; website: \url{https://nqbh.github.io}.}}
\date{\today}

\begin{document}
\maketitle
\begin{abstract}
	\textsc{[en]} This text is a collection of problems, from easy to advanced, about \textit{hydrogen \& air}. This text is also a supplementary material for my lecture note on Elementary Chemistry grade 8, which is stored \& downloadable at the following link: \href{https://github.com/NQBH/hobby/blob/master/elementary_chemistry/grade_8/NQBH_elementary_chemistry_grade_8.pdf}{GitHub\texttt{/}NQBH\texttt{/}hobby\texttt{/}elementary chemistry\texttt{/}grade 8\texttt{/}lecture}\footnote{\textsc{url}: \url{https://github.com/NQBH/hobby/blob/master/elementary_chemistry/grade_8/NQBH_elementary_chemistry_grade_8.pdf}.}. The latest version of this text has been stored \& downloadable at the following link: \href{https://github.com/NQBH/hobby/blob/master/elementary_chemistry/grade_8/hydrogen/NQBH_hydrogen.pdf}{GitHub\texttt{/}NQBH\texttt{/}hobby\texttt{/}elementary chemistry\texttt{/}grade 8\texttt{/}hydrogen}\footnote{\textsc{url}: \url{https://github.com/NQBH/hobby/blob/master/elementary_chemistry/grade_8/hydrogen/NQBH_hydrogen.pdf}.}.
	\vspace{2mm}
	
	\textsc{[vi]} Tài liệu này là 1 bộ sưu tập các bài tập chọn lọc từ cơ bản đến nâng cao về \textit{oxi \& không khí}. Tài liệu này là phần bài tập bổ sung cho tài liệu chính -- bài giảng \href{https://github.com/NQBH/hobby/blob/master/elementary_chemistry/grade_8/NQBH_elementary_chemistry_grade_8.pdf}{GitHub\texttt{/}NQBH\texttt{/}hobby\texttt{/}elementary chemistry\texttt{/}grade 8\texttt{/}lecture} của tác giả viết cho Hóa Sơ Cấp lớp 8. Phiên bản mới nhất của tài liệu này được lưu trữ \& có thể tải xuống ở link sau: \href{https://github.com/NQBH/hobby/blob/master/elementary_chemistry/grade_8/hydrogen/NQBH_hydrogen.pdf}{GitHub\texttt{/}NQBH\texttt{/}hobby\texttt{/}elementary chemistry\texttt{/}grade 8\texttt{/}hydrogen}.
\end{abstract}
\tableofcontents
\newpage

%------------------------------------------------------------------------------%

\section{Wikipedia's}

\subsection{\href{https://en.wikipedia.org/wiki/Hydrogen}{Wikipedia\texttt{/}Hydrogen}}
``\textit{Hydrogen} is the \href{https://en.wikipedia.org/wiki/Chemical_element}{chemical element} with the \href{https://en.wikipedia.org/wiki/Symbol_(chemistry)}{symbol} H \& \href{https://en.wikipedia.org/wiki/Atomic_number}{atomic number} 1. Hydrogen is the lightest element. At \href{https://en.wikipedia.org/wiki/Standard_temperature_and_pressure}{standard conditions} hydrogen is a \href{https://en.wikipedia.org/wiki/Gas}{gas} of \href{https://en.wikipedia.org/wiki/Diatomic_molecule}{diatomic moleculse} having the \href{https://en.wikipedia.org/wiki/Chemical_formula}{formula} \ce{H2}. It is \href{https://en.wikipedia.org/wiki/Transparency_(optics)}{colorless}, \href{https://en.wikipedia.org/wiki/Sense_of_smell}{odorless}, \href{https://en.wikipedia.org/wiki/Taste}{tasteless}, non-toxic, \& highly \href{https://en.wikipedia.org/wiki/Combustible}{combustible}. Hydrogen is the \href{https://en.wikipedia.org/wiki/Abundance_of_the_chemical_elements}{most abundant} chemical substance in the \href{https://en.wikipedia.org/wiki/Universe}{universe}, constituting roughly $75$\% of all \href{https://en.wikipedia.org/wiki/Baryon}{normal} \href{https://en.wikipedia.org/wiki/Matter}{matter}. \href{https://en.wikipedia.org/wiki/Star}{Stars} such as the \href{https://en.wikipedia.org/wiki/Sun}{Sun} are mainly composed of hydrogen in the \href{https://en.wikipedia.org/wiki/Plasma_state}{plasma state}. Most of the hydrogen on Earth exists in \href{https://en.wikipedia.org/wiki/Molecular_geometry}{molecular forms} such as \href{https://en.wikipedia.org/wiki/Water}{water} \& \href{https://en.wikipedia.org/wiki/Organic_compound}{organic compounds}. For the most common \href{https://en.wikipedia.org/wiki/Isotope}{isotope} of hydrogen (symbol \ce{^1H}) each \href{https://en.wikipedia.org/wiki/Atom}{atom} has 1 \href{https://en.wikipedia.org/wiki/Proton}{proton}, 1 \href{https://en.wikipedia.org/wiki/Electron}{electron}, \& no \href{https://en.wikipedia.org/wiki/Neutron}{neutrons}.

In the early \href{https://en.wikipedia.org/wiki/Universe}{universe}, the formation of protons, the nuclei of hydrogen, occurred during the 1st second after the \href{https://en.wikipedia.org/wiki/Big_Bang}{Big Bang}. The emergence of neutral hydrogen atoms throughout the universe occurred about 370000 years later during the \href{https://en.wikipedia.org/wiki/Recombination_(cosmology)}{recombination epoch}, when the \href{https://en.wikipedia.org/wiki/Plasma_(physics)}{plasma} had cooled enough for \href{https://en.wikipedia.org/wiki/Electrons}{electrons} to remain bound to protons.

Hydrogen is \href{https://en.wikipedia.org/wiki/Nonmetallic}{nonmetallic} (except it becomes \href{https://en.wikipedia.org/wiki/Metallic_hydrogen}{metallic} at extremely high pressures) \& readily forms a single \href{https://en.wikipedia.org/wiki/Covalent_bond}{covalent bond} with most nonmetallic elements, forming compounds such as water \& nearly all \href{https://en.wikipedia.org/wiki/Organic_compound}{organic compounds}. Hydrogen plays a particularly important role in \href{https://en.wikipedia.org/wiki/Acid%E2%80%93base_reaction}{acid--base reactions} because these reactions usually involve the exchange of protons between soluble molecules. In \href{https://en.wikipedia.org/wiki/Ionic_compound}{ionic compounds}, hydrogen can take the form of a negative charge (i.e., \href{https://en.wikipedia.org/wiki/Anion}{anion}) where it is known as a \href{https://en.wikipedia.org/wiki/Hydride}{hydride}, or as a positively charged (i.e., \href{https://en.wikipedia.org/wiki/Cation}{cation}) \href{https://en.wikipedia.org/wiki/Chemical_species}{species} denoted by the symbol \ce{H^+}. The \ce{H^+} cation is simply a \href{https://en.wikipedia.org/wiki/Proton}{proton} (symbol p) but its behavior in \href{https://en.wikipedia.org/wiki/Aqueous_solution}{aqueous solutions} \& in \href{https://en.wikipedia.org/wiki/Ionic_compound}{ionic compounds} involves \href{https://en.wikipedia.org/wiki/Electric-field_screening}{screeing} of its \href{https://en.wikipedia.org/wiki/Electric_charge}{electric charge} by nearby \href{https://en.wikipedia.org/wiki/Chemical_polarity}{polar} molecules or anions. Because hydrogen is the only neutral atom for which the \href{https://en.wikipedia.org/wiki/Schr%C3%B6dinger_equation}{Schr\"odinger equation} can be solved analytically, the study of its energetics \& chemical bonding has played a key role in the development of \href{https://en.wikipedia.org/wiki/Quantum_mechanics}{quantum mechanics}.

Hydrogen gas was 1st artificially produced in the early 16th century by the reaction of acids on metals. In 1766--1781, \href{https://en.wikipedia.org/wiki/Henry_Cavendish}{Henry Cavendish} was the 1st to recognize that hydrogen gas was a discrete substance, \& that it produces water when burned, the property for which it was later named: in Greek, hydrogen means ``water-former''.

\href{https://en.wikipedia.org/wiki/Hydrogen_production}{Industrial production} is mainly from \href{https://en.wikipedia.org/wiki/Steam_reforming}{steam reforming} of \href{https://en.wikipedia.org/wiki/Natural_gas}{natural gas}, oil reforming, or \href{https://en.wikipedia.org/wiki/Coal_gasification}{coal gasification}. A small percentage is also produced using more energy-intensive methods such as the \href{https://en.wikipedia.org/wiki/Electrolysis_of_water}{electrolysis of water}. Most hydrogen is used near the site of its production, the 2 largest uses being \href{https://en.wikipedia.org/wiki/Fossil_fuel}{fossil fuel} processing (e.g., \href{https://en.wikipedia.org/wiki/Hydrocracking}{hydrocracking}) \& \href{https://en.wikipedia.org/wiki/Ammonia}{ammonia} production, mostly for the fertilizer market. It can be burned to produce heat or combined with oxygen in \href{https://en.wikipedia.org/wiki/Fuel_cells}{fuel cells} to generate electricity directly, with water being the only emissions at the point of usage. Hydrogen atoms (but not gaseous molecules) are problematic in \href{https://en.wikipedia.org/wiki/Metallurgy}{metallurgy} because they can \href{https://en.wikipedia.org/wiki/Hydrogen_embrittlement}{embrittle} many metals.'' -- \href{https://en.wikipedia.org/wiki/Hydrogen}{Wikipedia\texttt{/}hydrogen}

\subsubsection{Properties}

\subsubsection{History}

\subsubsection{Cosmic Prevalence \& Distribution}

\subsubsection{Production}

\subsubsection{Applications}

\subsubsection{Biological Reactions}

\subsubsection{Safety \& Precautions}

%------------------------------------------------------------------------------%

\subsection{\href{https://en.wikipedia.org/wiki/Water}{Wikipedia\texttt{/}Water}}

%------------------------------------------------------------------------------%

\section{Tính Chất của Hydro. Phản Ứng Oxi Hóa--Khử}

\begin{baitoan}[\cite{An_400_BT_Hoa_Hoc_8_2020}, 279., p. 143]
	Viết PTHH của hydro với các oxide kim loại sau: (a) sắt (II, III) oxide; (b) bạc (I) oxide; (c) sắt (III) oxide. Trong những phản ứng trên, chất nào là chất khử? Chất nào là chất oxi hóa?
\end{baitoan}

\begin{baitoan}[\cite{An_400_BT_Hoa_Hoc_8_2020}, 280., p. 143]
	Khử $33.45$\emph{g} chì (II) oxide bằng khí hydro. (a) Tính số gam chì kim loại thu được. (b) Tính thể tích khí hydro (đktc) cần dùng.
\end{baitoan}

\begin{baitoan}[\cite{An_400_BT_Hoa_Hoc_8_2020}, 281., p. 143]
	Cho $8.4$\emph{g} sắt tác dụng với 1 lượng dung dịch \emph{HCl} vừa đủ. Dẫn toàn bộ lượng khí sinh ra qua $16$\emph{g} đồng (II) oxide nóng. (a) Tính thể tích khí hydro sinh ra (đktc). (b) Tính lượng kim loại đồng thu được sau phản ứng.
\end{baitoan}

\begin{baitoan}[\cite{An_400_BT_Hoa_Hoc_8_2020}, 282., p. 143]
	Khử oxide sắt từ bằng khí hydro ở nhiệt độ cao, thu được $30.24$\emph{g} sắt. Tính khối lượng oxide sắt từ cần dùng.
\end{baitoan}

\begin{baitoan}[\cite{An_400_BT_Hoa_Hoc_8_2020}, 283., p. 143]
	Cho các sơ đồ phản ứng oxi hóa--khử sau. Cân bằng PTHH, xác định chất oxi hóa, chất khử. (a) \emph{\ce{Fe2O3 + H2 -> Fe + H2O}}; (b) \emph{\ce{Al + C -> Al4C3}}; (c) \emph{\ce{CuO + Al -> Al2O3 + Cu}}; (d) \emph{\ce{Fe3O4 + CO -> FeO + CO2}}.
\end{baitoan}

\begin{baitoan}[\cite{An_400_BT_Hoa_Hoc_8_2020}, 284., p. 143]
	Cho $m$\emph{g} sắt (III) oxide tác dụng với hydro thu được $8.4$\emph{g} sắt. (a) Viết PTHH, xác định chất oxi hóa, chất khử, sự oxi hóa, sự khử. (b) Tính số \emph{g} sắt (III) oxide đã tham gia phản ứng.
\end{baitoan}

\begin{baitoan}[\cite{An_400_BT_Hoa_Hoc_8_2020}, 285., pp. 143--144]
	Lập các PTHH theo sơ đồ phản ứng sau: (a) sắt (III) oxide $+$ nhôm $\to$ nhôm oxide $+$ sắt; (b) nhôm oxide $+$ carbon $\to$ nhôm cacbua $+$ khí cacbon monooxide; (c) hydro sunfua $+$ oxi $\to$ khí sunfurơ $+$ nước; (d) đồng (II) hydroxide $\to$ đồng (II) oxide $+$ nước; (e) kali oxide $+$ carbon dioxide $\to$ kali cacbonat. Trong các phản ứng trên, phản ứng nào là phản ứng oxi hóa--khử? Xác định chất oxi hóa, chất khử, sự oxi hóa, sự khử.
\end{baitoan}

\begin{baitoan}[\cite{An_400_BT_Hoa_Hoc_8_2020}, 286., p. 144]
	Hoàn thành PTHH của những phản ứng giữa các chất sau: (a) \emph{\ce{Al + O2 ->}} ?; (b) \emph{\ce{P + O2 ->}} ?; (c) \emph{\ce{Fe + Cl2 ->}} ?; (d) \emph{\ce{KClO3 ->}} ? + ?; (e) \emph{\ce{H2 + Fe3O4 ->}} ? + ?.
\end{baitoan}

\begin{baitoan}[\cite{An_400_BT_Hoa_Hoc_8_2020}, 287., p. 144]
	Muốn điều chế $42$\emph{g} sắt phải dùng khí nào để khử sắt (III) oxide \& cho biết thể tích khí cần phải dùng.
\end{baitoan}

\begin{baitoan}[\cite{An_400_BT_Hoa_Hoc_8_2020}, 288., p. 144]
	Dùng hydro để khử đồng (II) oxide. (a) Nếu khử $m$\emph{g} đồng (II) oxit thì thu được bao nhiêu \emph{g} đồng? (b) Cho $m = 20$\emph{g}. Tính kết quả bằng số.
\end{baitoan}

\begin{baitoan}[\cite{An_400_BT_Hoa_Hoc_8_2020}, 289., p. 144]
	Xác định CTPT của \emph{\ce{Cu_xO_y}} biết tỷ lệ khối lượng giữa \emph{Cu} \& \emph{O} trong oxide là $4:1$. Viết phương trình phản ứng điều chế \emph{\ce{Cu,CuSO4}} từ \emph{\ce{Cu_xO_y}} (các chất phản ứng khác tự chọn).
\end{baitoan}

\begin{baitoan}[\cite{An_400_BT_Hoa_Hoc_8_2020}, 290., p. 144]
	Cho sơ đồ phản ứng oxi hóa--khử sau. Cân bằng phương trình phản ứng. Xác định chất oxi hóa, chất khử. (a) \emph{\ce{SO2 + Mg -> MgO + S}}; (b) \emph{\ce{SO2 + O2 -> SO3}}; (c) \emph{\ce{H2 + SO2 -> H2O + S}}; (d) \emph{\ce{S + KClO3 -> SO2 + KCl}}; (e) \emph{\ce{CuS + O2 -> CuO + SO2}}.
\end{baitoan}

\begin{baitoan}[\cite{An_400_BT_Hoa_Hoc_8_2020}, 291., p. 144]
	Cân bằng các PTHH sau \& xác định chất oxi hóa, chất khử. (a) \emph{\ce{N_xO_y + Cu -> CuO + N2}}; (b) \emph{\ce{Fe + Cl2 -> FeCl3}}; (c) \emph{\ce{Fe_xO_y + H2 -> Fe + H2O}}; (d) \emph{\ce{NO2 + C -> N2 + CO2}}.
\end{baitoan}

\begin{baitoan}[\cite{An_400_BT_Hoa_Hoc_8_2020}, 292., p. 144]
	Có 4 ống đựng riêng biệt các khí sau: không khí, khí oxi, khí hydro, khí carbonic. Bằng cách nào có thể phân biệt được các chất khí trong mỗi ống?
\end{baitoan}

\begin{baitoan}[\cite{An_400_BT_Hoa_Hoc_8_2020}, 293., p. 145]
	(a) 1 oxide base có thành phần \% khối lượng của oxi là $7.17$\%. Tìm CTPT của oxide biết kim loại hóa trị II. (b) Muốn điều chế $31.05$\emph{g} kim loại trên cần bao nhiêu \emph{l} khí \emph{\ce{H2}} (đktc)?
\end{baitoan}

\begin{baitoan}[\cite{An_400_BT_Hoa_Hoc_8_2020}, 294., p. 145]
	Dùng \emph{\ce{H2}} để khử $a$\emph{g} \emph{CuO} thu được $b$\emph{g} \emph{Cu}. Cho lượng đồng này tác dụng với \emph{\ce{Cl2}} thu được $33.75$\emph{g} \emph{\ce{CuCl2}}. Tính $a,b$.
\end{baitoan}

\begin{baitoan}[\cite{An_400_BT_Hoa_Hoc_8_2020}, 295., p. 145]
	Cho hỗn hợp \emph{\ce{CuO,Fe2O3}} tác dụng với \emph{\ce{H2}} ở nhiệt độ thích hợp. Hỏi nếu thu được $26.4$\emph{g} hỗn hợp \emph{Cu,Fe}, trong đó khối lượng \emph{Cu} gấp $1.2$ lần khối lượng \emph{Fe} thì cần dùng tất cả bao nhiêu \emph{l} khí hydro?
\end{baitoan}

\begin{baitoan}[\cite{An_400_BT_Hoa_Hoc_8_2020}, 296., p. 145]
	Dùng \emph{\ce{H2}} khử $31.2$\emph{g} hỗn hợp \emph{\ce{CuO,Fe3O4}}, trong hỗn hợp khối lượng \emph{\ce{Fe3O4}} hơn khối lượng \emph{CuO} là $15.2$\emph{g}. Tính khối lượng \emph{Cu,Fe} thu được.
\end{baitoan}

\begin{baitoan}[\cite{An_400_BT_Hoa_Hoc_8_2020}, 297., p. 145]
	Cho \emph{\ce{H2}} khử $16$\emph{g} hỗn hợp \emph{\ce{Fe2O3,CuO}}, trong đó khối lượng \emph{CuO} chiếm $25$\%. (a) Tính khối lượng \emph{Fe,Cu} thu được sau phản ứng. (b) Tính tổng thể tích \emph{\ce{H2}} đã tham gia phản ứng.
\end{baitoan}

\begin{baitoan}[\cite{An_400_BT_Hoa_Hoc_8_2020}, 298., p. 145]
	Cho hỗn hợp \emph{\ce{PbO,Fe2O3}} tác dụng với \emph{\ce{H2}} ở nhiệt độ cao. Hỏi nếu thu được $52.6$\emph{g} hỗn hợp \emph{\ce{Pb,Fe}}, trong đó khối lượng \emph{Pb} gấp $3.696$ lần khối lượng \emph{Fe} thì cần dùng tất cả bao nhiêu \emph{l} \emph{\ce{H2}} (đktc)?
\end{baitoan}

\begin{baitoan}[\cite{An_400_BT_Hoa_Hoc_8_2020}, 299., p. 145]
	Cho $8.4$\emph{l} khí hydro tác dụng với $2.8$\emph{l} khí oxi. Tính số \emph{g} nước tạo thành, biết các khí đo ở đktc.
\end{baitoan}

\begin{baitoan}[\cite{An_400_BT_Hoa_Hoc_8_2020}, 300., p. 145]
	Có 1 hỗn hợp gồm $60$\% \emph{\ce{Fe2O3}} \& $40$\% \emph{CuO}. Dùng \emph{\ce{H2}} (dư) để khử $20$\emph{g} hỗn hợp đó. (a) Tính khối lượng \emph{Fe,Cu} thu được sau phản ứng. (b) Tính số mol \emph{\ce{H2}} đã tham gia phản ứng.
\end{baitoan}

\begin{baitoan}[\cite{An_400_BT_Hoa_Hoc_8_2020}, 301., p. 145]
	Dùng khí hydro hoặc khí carbon oxide để khử sắt (III) oxide thành sắt. Để điều chế $35$\emph{g} sắt, tính thể tích khí hydro \& thể tích khí carbon oxide lần lượt là (các khí đo ở đktc): {\sf A.} $42$\emph{l}, $21$\emph{l}. {\sf B.} $42$\emph{l}, $42$\emph{l}. {\sf C.} $10.5$\emph{l}, $21$\emph{l}. {\sf D.} $21$\emph{l}, $21$\emph{l}.
\end{baitoan}

\begin{baitoan}[\cite{An_400_BT_Hoa_Hoc_8_2020}, 302., p. 145]
	Trường hợp nào sau đây chứa 1 khối lượng hydro ít nhất? {\sf A.} $6\cdot10^{23}$ phân tử \emph{\ce{H2}}. {\sf B.} $3\cdot10^{23}$ phân tử \emph{\ce{H2O}}. {\sf C.} $0.6$\emph{g} \emph{\ce{CH4}}. {\sf D.} $1.5$\emph{g} \emph{\ce{NH4Cl}}.
\end{baitoan}

%------------------------------------------------------------------------------%

\section{Điều Chế Hydro. Phản Ứng Thế}

\begin{baitoan}[\cite{An_400_BT_Hoa_Hoc_8_2020}, 303., p. 146]
	Lập PTHH \& xác định loại phản ứng. (a) sắt $+$ acid hydrochloric $\to$ ?; (b) kali clorat \ce{->[$t^\circ$]} ?; (c) sắt $+$ đồng sunfat $\to$ ?; (d) nhôm $+$ oxi $\to$ ?; (e) nước \ce{->[{điện phân}][acid sulfuric]} ?; (f) khí carbonic $+$ magie $\to$ ?.
\end{baitoan}

\begin{baitoan}[\cite{An_400_BT_Hoa_Hoc_8_2020}, 304., p. 146]
	Điện phân 1 lượng nước thu được khí hydro \& oxi. Nếu dùng lượng khí \emph{\ce{H2}} thu được để khử sắt (III) oxide thu được $16.8$\emph{g}. Hỏi phải điện phân bao nhiêu \emph{l} nước biết $D_{\ce{H2O}} = 1$\emph{g\texttt{/}ml}?
\end{baitoan}

\begin{baitoan}[\cite{An_400_BT_Hoa_Hoc_8_2020}, 305., p. 146]
	Cho $11.2$\emph{g} sắt tác dụng với dung dịch \emph{\ce{H2SO4}} loãng có chứa $12.25$\emph{g \ce{H2SO4}}. (a) Chất nào còn dư sau phản ứng \& dư bao nhiêu \emph{g}? (b) Tính thể tích khí hydro thu được ở đktc.
\end{baitoan}

\begin{baitoan}[\cite{An_400_BT_Hoa_Hoc_8_2020}, 306., p. 146]
	Cho các kim loại \emph{K,Ca,Al} lần lượt tác dụng với dung dịch \emph{HCl}. (a) Nếu cho cùng số mmol của 1 trong các kim loại trên tác dụng với acid \emph{HCl} thì kim loại nào cho nhiều \emph{\ce{H2}} hơn? (b) Nếu thu được cùng số mol khí \emph{\ce{H2}} thì khối lượng kim loại nào ít hơn?
\end{baitoan}

\begin{baitoan}[\cite{An_400_BT_Hoa_Hoc_8_2020}, 307., p. 146]
	Cho $5.4$\emph{g Al} vào dung dịch \emph{\ce{H2SO4}} loãng có chứa $39.2$\emph{g \ce{H2SO4}}. (a) Chất nào còn dư sau phản ứng \& dư bao nhiêu \emph{g}? (b) Tính thể tích khí hydro thu được ở đktc.
\end{baitoan}

\begin{baitoan}[\cite{An_400_BT_Hoa_Hoc_8_2020}, 308., p. 146]
	Cho $5.1$\emph{g} hỗn hợp \emph{Al,Mg} vào dung dịch \emph{\ce{H2SO4}} loãng, dư thu được $5.6$\emph{l} khí \emph{\ce{H2}} (đktc). Tính khối lượng mỗi kim loại ban đầu. Biết phản ứng xảy ra hoàn toàn.
\end{baitoan}

\begin{baitoan}[\cite{An_400_BT_Hoa_Hoc_8_2020}, 309., p. 147]
	Cho kẽm hoặc sắt tác dụng với dung dịch acid hydrochloride \emph{HCl} để điều chế khí hydro. Nếu muốn điều chế $2.24$\emph{l} khí hydro (đktc) thì phải dùng số \emph{g} kẽm hoặc sắt lần lượt là: {\sf A.} $6.5$\emph{g}, $5.6$\emph{g}. {\sf B.} $16$\emph{g}, $8$\emph{g}. {\sf C.} $13$\emph{g}, $11.2$\emph{g}. {\sf D.} $9.75$\emph{g}, $8.4$\emph{g}.
\end{baitoan}

\begin{baitoan}[\cite{An_400_BT_Hoa_Hoc_8_2020}, 310., p. 147]
	Điện phân hoàn toàn $2$\emph{l} nước ở trạng thái lỏng (biết khối lượng riêng $D$ của nước là $1$\emph{kg\texttt{/}l}). Tính thể tích khí hydro \& thể tích khí oxi thu được.
\end{baitoan}

\begin{baitoan}[\cite{An_400_BT_Hoa_Hoc_8_2020}, 311., p. 147]
	So sánh thể tích khí hydro (đktc) thu được trong mỗi trường hợp sau: (a) $0.1$\emph{mol Zn} tác dụng với dung dịch \emph{\ce{H2SO4}} loãng dư. $0.1$\emph{mol Al} tác dụng với dung dịch \emph{\ce{H2SO4}} loãng dư. (b) $0.2$\emph{mol Zn} tác dụng với dung dịch \emph{HCl} dư. $0.2$\emph{mol Al} tác dụng với dung dịch \emph{HCl} dư.
\end{baitoan}

\begin{baitoan}[\cite{An_400_BT_Hoa_Hoc_8_2020}, 312., p. 147]
	Dùng hydro để khử hoàn toàn $a$\emph{g} \emph{\ce{Fe2O3}} \& thu được $b$\emph{g Fe}. Cho lượng sắt này tác dụng với dung dịch \emph{\ce{H2SO4}} loãng dư thì thu được $5.6$\emph{l} khí \emph{\ce{H2}} (ở đktc). Tính $a,b$.
\end{baitoan}

\begin{baitoan}[\cite{An_400_BT_Hoa_Hoc_8_2020}, 313., p. 147]
	Cho lá sắt có khối lượng $50$\emph{g} vào 1 dung dịch đồng sunfat. Sau 1 thời gian, nhấc lá sắt ra thì khối lượng lá sắt là $51$\emph{g}. Tính số \emph{mol} muối sắt tạo thành sau phản ứng biết tất cả đồng sinh ra bám trên bề mặt lá sắt.
\end{baitoan}

\begin{baitoan}[\cite{An_400_BT_Hoa_Hoc_8_2020}, 314., p. 147]
	Nhúng 1 lá nhôm vào dung dịch \emph{\ce{CuSO4}}. Sau phản ứng lấy lá nhôm ra thấy khối lượng dung dịch nhẹ đi $1.38$\emph{g}. Tính khối lượng nhôm đã phản ứng.
\end{baitoan}

%------------------------------------------------------------------------------%

\section{Nước, Acid, Base, Muối}

%------------------------------------------------------------------------------%

\section{Miscellaneous}

%------------------------------------------------------------------------------%

\printbibliography[heading=bibintoc]
	
\end{document}