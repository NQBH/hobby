\documentclass{article}
\usepackage[backend=biber,natbib=true,style=authoryear,maxbibnames=10]{biblatex}
\addbibresource{/home/nqbh/reference/bib.bib}
\usepackage[utf8]{vietnam}
\usepackage{tocloft}
\renewcommand{\cftsecleader}{\cftdotfill{\cftdotsep}}
\usepackage[colorlinks=true,linkcolor=blue,urlcolor=red,citecolor=magenta]{hyperref}
\usepackage{amsmath,amssymb,amsthm,float,graphicx,mathtools,tikz}
\usepackage[version=4]{mhchem}
\allowdisplaybreaks
\newtheorem{assumption}{Assumption}
\newtheorem{baitoan}{Bài toán}
\newtheorem{cauhoi}{Câu hỏi}
\newtheorem{conjecture}{Conjecture}
\newtheorem{corollary}{Corollary}
\newtheorem{dangtoan}{Dạng toán}
\newtheorem{definition}{Definition}
\newtheorem{dinhly}{Định lý}
\newtheorem{dinhnghia}{Định nghĩa}
\newtheorem{example}{Example}
\newtheorem{ghichu}{Ghi chú}
\newtheorem{hequa}{Hệ quả}
\newtheorem{hypothesis}{Hypothesis}
\newtheorem{lemma}{Lemma}
\newtheorem{luuy}{Lưu ý}
\newtheorem{nhanxet}{Nhận xét}
\newtheorem{notation}{Notation}
\newtheorem{note}{Note}
\newtheorem{principle}{Principle}
\newtheorem{problem}{Problem}
\newtheorem{proposition}{Proposition}
\newtheorem{question}{Question}
\newtheorem{remark}{Remark}
\newtheorem{theorem}{Theorem}
\newtheorem{vidu}{Ví dụ}
\usepackage[left=1cm,right=1cm,top=5mm,bottom=5mm,footskip=4mm]{geometry}
\def\labelitemii{$\circ$}

\title{Hydrogen, Water -- Hiđro, Nước}
\author{Nguyễn Quản Bá Hồng\footnote{Independent Researcher, Ben Tre City, Vietnam\\e-mail: \texttt{nguyenquanbahong@gmail.com}; website: \url{https://nqbh.github.io}.}}
\date{\today}

\begin{document}
\maketitle
\begin{abstract}
	\textsc{[en]} This text is a collection of problems, from easy to advanced, about \textit{hydrogen \& air}. This text is also a supplementary material for my lecture note on Elementary Chemistry grade 8, which is stored \& downloadable at the following link: \href{https://github.com/NQBH/hobby/blob/master/elementary_chemistry/grade_8/NQBH_elementary_chemistry_grade_8.pdf}{GitHub\texttt{/}NQBH\texttt{/}hobby\texttt{/}elementary chemistry\texttt{/}grade 8\texttt{/}lecture}\footnote{\textsc{url}: \url{https://github.com/NQBH/hobby/blob/master/elementary_chemistry/grade_8/NQBH_elementary_chemistry_grade_8.pdf}.}. The latest version of this text has been stored \& downloadable at the following link: \href{https://github.com/NQBH/hobby/blob/master/elementary_chemistry/grade_8/hydrogen/NQBH_hydrogen.pdf}{GitHub\texttt{/}NQBH\texttt{/}hobby\texttt{/}elementary chemistry\texttt{/}grade 8\texttt{/}hydrogen}\footnote{\textsc{url}: \url{https://github.com/NQBH/hobby/blob/master/elementary_chemistry/grade_8/hydrogen/NQBH_hydrogen.pdf}.}.
	\vspace{2mm}
	
	\textsc{[vi]} Tài liệu này là 1 bộ sưu tập các bài tập chọn lọc từ cơ bản đến nâng cao về \textit{oxy \& không khí}. Tài liệu này là phần bài tập bổ sung cho tài liệu chính -- bài giảng \href{https://github.com/NQBH/hobby/blob/master/elementary_chemistry/grade_8/NQBH_elementary_chemistry_grade_8.pdf}{GitHub\texttt{/}NQBH\texttt{/}hobby\texttt{/}elementary chemistry\texttt{/}grade 8\texttt{/}lecture} của tác giả viết cho Hóa Sơ Cấp lớp 8. Phiên bản mới nhất của tài liệu này được lưu trữ \& có thể tải xuống ở link sau: \href{https://github.com/NQBH/hobby/blob/master/elementary_chemistry/grade_8/hydrogen/NQBH_hydrogen.pdf}{GitHub\texttt{/}NQBH\texttt{/}hobby\texttt{/}elementary chemistry\texttt{/}grade 8\texttt{/}hydrogen}.
\end{abstract}
\tableofcontents
\newpage

%------------------------------------------------------------------------------%

\section{Wikipedia's}

\subsection{\href{https://en.wikipedia.org/wiki/Hydrogen}{Wikipedia\texttt{/}Hydrogen}}
``\textit{Hydrogen} is the \href{https://en.wikipedia.org/wiki/Chemical_element}{chemical element} with the \href{https://en.wikipedia.org/wiki/Symbol_(chemistry)}{symbol} H \& \href{https://en.wikipedia.org/wiki/Atomic_number}{atomic number} 1. Hydrogen is the lightest element. At \href{https://en.wikipedia.org/wiki/Standard_temperature_and_pressure}{standard conditions} hydrogen is a \href{https://en.wikipedia.org/wiki/Gas}{gas} of \href{https://en.wikipedia.org/wiki/Diatomic_molecule}{diatomic moleculse} having the \href{https://en.wikipedia.org/wiki/Chemical_formula}{formula} \ce{H2}. It is \href{https://en.wikipedia.org/wiki/Transparency_(optics)}{colorless}, \href{https://en.wikipedia.org/wiki/Sense_of_smell}{odorless}, \href{https://en.wikipedia.org/wiki/Taste}{tasteless}, non-toxyc, \& highly \href{https://en.wikipedia.org/wiki/Combustible}{combustible}. Hydrogen is the \href{https://en.wikipedia.org/wiki/Abundance_of_the_chemical_elements}{most abundant} chemical substance in the \href{https://en.wikipedia.org/wiki/Universe}{universe}, constituting roughly $75$\% of all \href{https://en.wikipedia.org/wiki/Baryon}{normal} \href{https://en.wikipedia.org/wiki/Matter}{matter}. \href{https://en.wikipedia.org/wiki/Star}{Stars} such as the \href{https://en.wikipedia.org/wiki/Sun}{Sun} are mainly composed of hydrogen in the \href{https://en.wikipedia.org/wiki/Plasma_state}{plasma state}. Most of the hydrogen on Earth exists in \href{https://en.wikipedia.org/wiki/Molecular_geometry}{molecular forms} such as \href{https://en.wikipedia.org/wiki/Water}{water} \& \href{https://en.wikipedia.org/wiki/Organic_compound}{organic compounds}. For the most common \href{https://en.wikipedia.org/wiki/Isotope}{isotope} of hydrogen (symbol \ce{^1H}) each \href{https://en.wikipedia.org/wiki/Atom}{atom} has 1 \href{https://en.wikipedia.org/wiki/Proton}{proton}, 1 \href{https://en.wikipedia.org/wiki/Electron}{electron}, \& no \href{https://en.wikipedia.org/wiki/Neutron}{neutrons}.

In the early \href{https://en.wikipedia.org/wiki/Universe}{universe}, the formation of protons, the nuclei of hydrogen, occurred during the 1st second after the \href{https://en.wikipedia.org/wiki/Big_Bang}{Big Bang}. The emergence of neutral hydrogen atoms throughout the universe occurred about 370000 years later during the \href{https://en.wikipedia.org/wiki/Recombination_(cosmology)}{recombination epoch}, when the \href{https://en.wikipedia.org/wiki/Plasma_(physics)}{plasma} had cooled enough for \href{https://en.wikipedia.org/wiki/Electrons}{electrons} to remain bound to protons.

Hydrogen is \href{https://en.wikipedia.org/wiki/Nonmetallic}{nonmetallic} (except it becomes \href{https://en.wikipedia.org/wiki/Metallic_hydrogen}{metallic} at extremely high pressures) \& readily forms a single \href{https://en.wikipedia.org/wiki/Covalent_bond}{covalent bond} with most nonmetallic elements, forming compounds such as water \& nearly all \href{https://en.wikipedia.org/wiki/Organic_compound}{organic compounds}. Hydrogen plays a particularly important role in \href{https://en.wikipedia.org/wiki/Acid%E2%80%93base_reaction}{acid--base reactions} because these reactions usually involve the exchange of protons between soluble molecules. In \href{https://en.wikipedia.org/wiki/Ionic_compound}{ionic compounds}, hydrogen can take the form of a negative charge (i.e., \href{https://en.wikipedia.org/wiki/Anion}{anion}) where it is known as a \href{https://en.wikipedia.org/wiki/Hydride}{hydride}, or as a positively charged (i.e., \href{https://en.wikipedia.org/wiki/Cation}{cation}) \href{https://en.wikipedia.org/wiki/Chemical_species}{species} denoted by the symbol \ce{H^+}. The \ce{H^+} cation is simply a \href{https://en.wikipedia.org/wiki/Proton}{proton} (symbol p) but its behavior in \href{https://en.wikipedia.org/wiki/Aqueous_solution}{aqueous solutions} \& in \href{https://en.wikipedia.org/wiki/Ionic_compound}{ionic compounds} involves \href{https://en.wikipedia.org/wiki/Electric-field_screening}{screeing} of its \href{https://en.wikipedia.org/wiki/Electric_charge}{electric charge} by nearby \href{https://en.wikipedia.org/wiki/Chemical_polarity}{polar} molecules or anions. Because hydrogen is the only neutral atom for which the \href{https://en.wikipedia.org/wiki/Schr%C3%B6dinger_equation}{Schr\"odinger equation} can be solved analytically, the study of its energetics \& chemical bonding has played a key role in the development of \href{https://en.wikipedia.org/wiki/Quantum_mechanics}{quantum mechanics}.

Hydrogen gas was 1st artificially produced in the early 16th century by the reaction of acids on metals. In 1766--1781, \href{https://en.wikipedia.org/wiki/Henry_Cavendish}{Henry Cavendish} was the 1st to recognize that hydrogen gas was a discrete substance, \& that it produces water when burned, the property for which it was later named: in Greek, hydrogen means ``water-former''.

\href{https://en.wikipedia.org/wiki/Hydrogen_production}{Industrial production} is mainly from \href{https://en.wikipedia.org/wiki/Steam_reforming}{steam reforming} of \href{https://en.wikipedia.org/wiki/Natural_gas}{natural gas}, oil reforming, or \href{https://en.wikipedia.org/wiki/Coal_gasification}{coal gasification}. A small percentage is also produced using more energy-intensive methods such as the \href{https://en.wikipedia.org/wiki/Electrolysis_of_water}{electrolysis of water}. Most hydrogen is used near the site of its production, the 2 largest uses being \href{https://en.wikipedia.org/wiki/Fossil_fuel}{fossil fuel} processing (e.g., \href{https://en.wikipedia.org/wiki/Hydrocracking}{hydrocracking}) \& \href{https://en.wikipedia.org/wiki/Ammonia}{ammonia} production, mostly for the fertilizer market. It can be burned to produce heat or combined with oxygen in \href{https://en.wikipedia.org/wiki/Fuel_cells}{fuel cells} to generate electricity directly, with water being the only emissions at the point of usage. Hydrogen atoms (but not gaseous molecules) are problematic in \href{https://en.wikipedia.org/wiki/Metallurgy}{metallurgy} because they can \href{https://en.wikipedia.org/wiki/Hydrogen_embrittlement}{embrittle} many metals.'' -- \href{https://en.wikipedia.org/wiki/Hydrogen}{Wikipedia\texttt{/}hydrogen}

\subsubsection{Properties}

\subsubsection{History}

\subsubsection{Cosmic Prevalence \& Distribution}

\subsubsection{Production}

\subsubsection{Applications}

\subsubsection{Biological Reactions}

\subsubsection{Safety \& Precautions}

%------------------------------------------------------------------------------%

\subsection{\href{https://en.wikipedia.org/wiki/Water}{Wikipedia\texttt{/}Water}}

%------------------------------------------------------------------------------%

\section{Tính Chất của Hydro}
``\fbox{\bf 1} \textbf{Tính chất của Hydro.} \textit{Tính chất vật lý}: Hydro là chất khí không màu, không mùi, không vị, là khí nhẹ nhất trong các chất khí, tan rất ít trong nước. \textit{Tính chất hóa học}: Khí hydro có tính khử. \textit{Tác dụng với đơn chất}: Khí hydro tác dụng với 1 số đơn chất, e.g., \ce{H2} tác dụng với \ce{O2} tạo thành \ce{H2O}. \ce{$2$H2 + O2 ->[$t^\circ$] $2$H2O}. \textit{Tác dụng với đơn chất}: Khí hydro tác dụng được với 1 số oxyde kim loại khi đun nóng, tạo thành nước \& giải phóng kim loại, e.g., \ce{H2 + CuO ->[$t^\circ$] Cu + H2O}, ở phản ứng này, hydro đã chiếm nguyên tố oxy của CuO để tạo ra \ce{H2O} \& giải phóng đồng. \fbox{\bf 2} \textbf{Ứng dụng của Hydro.} Bơm khí cầu (do rất nhẹ; đèn xì oxy-hydro (do cháy tỏa rất nhiều nhiệt); điều chế 1 số kim loại từ oxyde của chúng (do có tính khử ở nhiệt độ cao); sản xuất acid hydrochloric, amoniac, phân đạm,$\ldots$, \& sản xuất nhiên liệu.'' -- \cite[Chap. 5, \S1, pp. 78--79]{Truong_BTNC_Hoa_Hoc_8_2022}

\begin{baitoan}[\cite{Truong_BTNC_Hoa_Hoc_8_2022}, V.1, p. 79]
	Cho biết heli \emph{He} là khí trơ, nó không tác dụng với những chất khác, phân tử khí heli chỉ có 1 nguyên tử \& có phân tử khối $M_{\rm He} = 4$đvC. Cho biết: (a) Khí \emph{He} nặng hơn khí \emph{\ce{H2}} bao nhiêu lần? (b) Ưu điểm \& nhược điểm khi dùng khí \emph{\ce{H2}} \& khí \emph{He} để bơm vào khinh khí cầu. Nên dùng khí \emph{\ce{H2}} hay khí \emph{He}? Giải thích. (c) Cần bơm vào bóng thám không để bóng lên cao hơn trong khí quyển thì dùng khí nào?
\end{baitoan}

\begin{baitoan}[\cite{Truong_BTNC_Hoa_Hoc_8_2022}, V.2, p. 79]
	Khí hydro \& khí metan có 1 số điểm giống nhau như: Đều cho ngọn lửa màu xanh, không khói khi cháy trong không khí. Đều tạo ra hỗn hợp nổ khi trộn lẫn với không khí. Đều tạo ra nước khi cháy. Làm thế nào để phân biệt được 2 khí này.
\end{baitoan}

\begin{baitoan}[\cite{Truong_BTNC_Hoa_Hoc_8_2022}, V.3, p. 79]
	Có 5 lọ đựng riêng biệt các chất khí sau: không khí, khí carbonic, oxy, hydro, nitơ. Bằng thí nghiệm nào có thể nhận biết chất khí trong mỗi lọ. Giải thích \& viết PTHH.
\end{baitoan}

\begin{baitoan}[\cite{Truong_BTNC_Hoa_Hoc_8_2022}, V.4, p. 79]
	Bằng thí nghiệm hóa học, chứng minh trong thành phần của acid hydrochloric có nguyên tố hydro.
\end{baitoan}

\begin{baitoan}[\cite{Truong_BTNC_Hoa_Hoc_8_2022}, V.5, p. 79]
	Cho $48$\emph{g CuO} tác dụng với khí hydro khi đun nóng. (a) Tính số \emph{g} đồng điều chế được. (b) Tính thể tích khí \emph{\ce{H2}} (đktc) cần dùng cho phản ứng trên.
\end{baitoan}

\begin{baitoan}[\cite{Truong_BTNC_Hoa_Hoc_8_2022}, V.6, p. 80]
	Trong phòng thí nghiệm, điều chế sắt bằng cách cho khí \emph{\ce{H2}} đi qua ống sứ đựng \emph{\ce{Fe2O3}} đun nóng \& thu được $11.2$\emph{g} sắt. (a) Viết PTHH của phản ứng đã xảy ra. (b) Tính số \emph{g \ce{Fe2O3}} đã tham gia phản ứng. (c) Tính số \emph{l} khí \emph{\ce{H2}} đã dùng ở đktc.
\end{baitoan}

\begin{baitoan}[\cite{Truong_BTNC_Hoa_Hoc_8_2022}, V.7, p. 80]
	Trong phòng thí nghiệm, dùng \emph{CO} để khử \emph{\ce{Fe3O4}} \& dùng \emph{\ce{H2}} để khử \emph{\ce{Fe2O3}} ở nhiệt độ cao. Cho biết trong mỗi phản ứng trên đều có $0.1$\emph{mol} mỗi loại oxyde sắt tham gia. (a) Viết PTHH của các phản ứng xảy ra. (b) Tính thể tích khí \emph{CO,\ce{H2}} ở đktc cần dùng cho mỗi phản ứng trên. (c) Tính số \emph{g} sắt thu được trong mỗi phản ứng. 
\end{baitoan}

\begin{baitoan}[\cite{Truong_BTNC_Hoa_Hoc_8_2022}, V.8, p. 80]
	Có 1 hỗn hợp gồm $75$\% \emph{\ce{Fe2O3}} \& $25$\% \emph{CuO}. Dùng \emph{\ce{H2}} (dư) để khử $16$\emph{g} hỗn hợp đó. (a) Tính khối lượng \emph{Fe,Cu} thu được sau phản ứng. (b) Tính số mol \emph{\ce{H2}} đã tham gia phản ưng.
\end{baitoan}

\begin{baitoan}[\cite{Truong_BTNC_Hoa_Hoc_8_2022}, V.9, p. 80]
	Dùng \emph{\ce{H2}} (dư) để khử $m$\emph{g} \emph{\ce{Fe2O3}}  \& thu được $n$\emph{g Fe}. Cho lượng \emph{Fe} này tác dụng với dung dịch \emph{\ce{H2SO4}} (dư) thì được $2.8$\emph{l \ce{H2}} (đktc). Tính $m,n$.
\end{baitoan}

%------------------------------------------------------------------------------%

\section{Phản Ứng Oxi Hóa--Khử}
``\fbox{\bf 1} \textbf{Chất khử \& chất oxy hóa.} Chất chiếm oxy của chất khác là \textit{chất khử}. Khí oxy hoặc chất nhường oxy cho chất khác là \textit{chất oxy hóa}. \fbox{\bf 2} \textbf{Sự khử \& sự oxy hóa.} Quy trình tách nguyên tử oxy khỏi hợp chất là \textit{sự khử}. Quá trình hóa hợp của nguyên tử oxy với chất khác là \textit{sự oxy hóa}. \fbox{\bf 3} \textbf{Phản ứng oxy hóa--khử.} \textit{Phản ứng oxy hóa--khử} là phản ứng hóa học trong đó xảy ra đồng thời sự oxy hóa \& sự khử.'' -- \cite[Chap. 5, \S2, pp. 80--81]{Truong_BTNC_Hoa_Hoc_8_2022}

\begin{vidu}
	Phản ứng \emph{\ce{Fe2O3 + 3CO ->[$t^\circ$] 2Fe + 3CO2}} là \emph{phản ứng oxy hóa--khử} vì xảy ra đồng thời sự oxy hóa \emph{CO} thành \emph{\ce{CO2}} \& sự khử \emph{\ce{Fe2O3}} thành \emph{Fe}, cụ thể: \emph{\ce{Fe2O3}} là \emph{chất oxy hóa}, \emph{CO} là \emph{chất khử}, quá trình \emph{\ce{CO -> CO2}} là \emph{sự oxy hóa CO}, quá trình \emph{\ce{Fe2O3 -> Fe}} là \emph{sự khử \ce{Fe2O3}}.
\end{vidu}

\begin{baitoan}[\cite{Truong_BTNC_Hoa_Hoc_8_2022}, V.10, p. 81]
	Trong những phản ứng oxy hóa--khử sau: \emph{\ce{$2$Mg + O2 -> $2$MgO, $2$H2 + O2 -> $2$H2O, Fe2O3 + $2$Al -> Al2O3 + Fe, Fe3O4 + $4$CO -> $3$Fe + $4$CO2, $2$Mg + CO2 -> $2$MgO + C}}. (a) Chất nào là chất khử? Chất nào là chất oxy hóa? (b) Quá trình nào được gọi là sự khử? Quá trình nào được gọi là sự oxy hóa? (c) Vì sao những phản ứng hóa học trên được gọi là phản ứng oxy hóa--khử? 
\end{baitoan}

\begin{baitoan}[\cite{Truong_BTNC_Hoa_Hoc_8_2022}, V.11, pp. 81--82]
	Tính thể tích (đktc) chất khử cần dùng \& khối lượng kim loại thu được trong các thí nghiệm sau: (a) Khử hỗn hợp gồm $10$\emph{g CuO} \& $55.75$\emph{g PbO} ở nhiệt độ cao bằng khí \emph{\ce{H2}}. (b) Khử hỗn hợp gồm $0.1$\emph{mol \ce{Fe2O3}} \& $0.05$\emph{mol \ce{Fe3O4}} ở nhiệt độ coa bằng khí \emph{CO}.
\end{baitoan}

\begin{baitoan}[\cite{Truong_BTNC_Hoa_Hoc_8_2022}, V.12, p. 82]
	Khử 1 hỗn hợp gồm có $3.2$\emph{g \ce{Fe2O3}}; $8$\emph{g CuO} \& $2.23$\emph{g PbO} ở nhiệt độ cao bằng khí \emph{\ce{H2}}. (a) Viết các PTHH. (b) Tính khối lượng \& thể tích (đktc) chất khử cần dùng. (c) Tính khối lượng của mỗi kim loại thu được.
\end{baitoan}

\begin{baitoan}[\cite{An_400_BT_Hoa_Hoc_8_2020}, 279., p. 143]
	Viết PTHH của hydro với các oxyde kim loại sau: (a) sắt (II, III) oxyde; (b) bạc (I) oxyde; (c) sắt (III) oxyde. Trong những phản ứng trên, chất nào là chất khử? Chất nào là chất oxy hóa?
\end{baitoan}

\begin{baitoan}[\cite{An_400_BT_Hoa_Hoc_8_2020}, 280., p. 143]
	Khử $33.45$\emph{g} chì (II) oxyde bằng khí hydro. (a) Tính số gam chì kim loại thu được. (b) Tính thể tích khí hydro (đktc) cần dùng.
\end{baitoan}

\begin{baitoan}[\cite{An_400_BT_Hoa_Hoc_8_2020}, 281., p. 143]
	Cho $8.4$\emph{g} sắt tác dụng với 1 lượng dung dịch \emph{HCl} vừa đủ. Dẫn toàn bộ lượng khí sinh ra qua $16$\emph{g} đồng (II) oxyde nóng. (a) Tính thể tích khí hydro sinh ra (đktc). (b) Tính lượng kim loại đồng thu được sau phản ứng.
\end{baitoan}

\begin{baitoan}[\cite{An_400_BT_Hoa_Hoc_8_2020}, 282., p. 143]
	Khử oxyde sắt từ bằng khí hydro ở nhiệt độ cao, thu được $30.24$\emph{g} sắt. Tính khối lượng oxyde sắt từ cần dùng.
\end{baitoan}

\begin{baitoan}[\cite{An_400_BT_Hoa_Hoc_8_2020}, 283., p. 143]
	Cho các sơ đồ phản ứng oxy hóa--khử sau. Cân bằng PTHH, xác định chất oxy hóa, chất khử. (a) \emph{\ce{Fe2O3 + H2 -> Fe + H2O}}; (b) \emph{\ce{Al + C -> Al4C3}}; (c) \emph{\ce{CuO + Al -> Al2O3 + Cu}}; (d) \emph{\ce{Fe3O4 + CO -> FeO + CO2}}.
\end{baitoan}

\begin{baitoan}[\cite{An_400_BT_Hoa_Hoc_8_2020}, 284., p. 143]
	Cho $m$\emph{g} sắt (III) oxyde tác dụng với hydro thu được $8.4$\emph{g} sắt. (a) Viết PTHH, xác định chất oxy hóa, chất khử, sự oxy hóa, sự khử. (b) Tính số \emph{g} sắt (III) oxyde đã tham gia phản ứng.
\end{baitoan}

\begin{baitoan}[\cite{An_400_BT_Hoa_Hoc_8_2020}, 285., pp. 143--144]
	Lập các PTHH theo sơ đồ phản ứng sau: (a) sắt (III) oxyde $+$ nhôm $\to$ nhôm oxyde $+$ sắt; (b) nhôm oxyde $+$ carbon $\to$ nhôm cacbua $+$ khí cacbon monooxyde; (c) hydro sunfua $+$ oxy $\to$ khí sunfurơ $+$ nước; (d) đồng (II) hydroxyde $\to$ đồng (II) oxyde $+$ nước; (e) kali oxyde $+$ carbon dioxyde $\to$ kali cacbonat. Trong các phản ứng trên, phản ứng nào là phản ứng oxy hóa--khử? Xác định chất oxy hóa, chất khử, sự oxy hóa, sự khử.
\end{baitoan}

\begin{baitoan}[\cite{An_400_BT_Hoa_Hoc_8_2020}, 286., p. 144]
	Hoàn thành PTHH của những phản ứng giữa các chất sau: (a) \emph{\ce{Al + O2 ->}} ?; (b) \emph{\ce{P + O2 ->}} ?; (c) \emph{\ce{Fe + Cl2 ->}} ?; (d) \emph{\ce{KClO3 ->}} ? + ?; (e) \emph{\ce{H2 + Fe3O4 ->}} ? + ?.
\end{baitoan}

\begin{baitoan}[\cite{An_400_BT_Hoa_Hoc_8_2020}, 287., p. 144]
	Muốn điều chế $42$\emph{g} sắt phải dùng khí nào để khử sắt (III) oxyde \& cho biết thể tích khí cần phải dùng.
\end{baitoan}

\begin{baitoan}[\cite{An_400_BT_Hoa_Hoc_8_2020}, 288., p. 144]
	Dùng hydro để khử đồng (II) oxyde. (a) Nếu khử $m$\emph{g} đồng (II) oxyt thì thu được bao nhiêu \emph{g} đồng? (b) Cho $m = 20$\emph{g}. Tính kết quả bằng số.
\end{baitoan}

\begin{baitoan}[\cite{An_400_BT_Hoa_Hoc_8_2020}, 289., p. 144]
	Xác định CTPT của \emph{\ce{Cu_xO_y}} biết tỷ lệ khối lượng giữa \emph{Cu} \& \emph{O} trong oxyde là $4:1$. Viết phương trình phản ứng điều chế \emph{\ce{Cu,CuSO4}} từ \emph{\ce{Cu_xO_y}} (các chất phản ứng khác tự chọn).
\end{baitoan}

\begin{baitoan}[\cite{An_400_BT_Hoa_Hoc_8_2020}, 290., p. 144]
	Cho sơ đồ phản ứng oxy hóa--khử sau. Cân bằng phương trình phản ứng. Xác định chất oxy hóa, chất khử. (a) \emph{\ce{SO2 + Mg -> MgO + S}}; (b) \emph{\ce{SO2 + O2 -> SO3}}; (c) \emph{\ce{H2 + SO2 -> H2O + S}}; (d) \emph{\ce{S + KClO3 -> SO2 + KCl}}; (e) \emph{\ce{CuS + O2 -> CuO + SO2}}.
\end{baitoan}

\begin{baitoan}[\cite{An_400_BT_Hoa_Hoc_8_2020}, 291., p. 144]
	Cân bằng các PTHH sau \& xác định chất oxy hóa, chất khử. (a) \emph{\ce{N_xO_y + Cu -> CuO + N2}}; (b) \emph{\ce{Fe + Cl2 -> FeCl3}}; (c) \emph{\ce{Fe_xO_y + H2 -> Fe + H2O}}; (d) \emph{\ce{NO2 + C -> N2 + CO2}}.
\end{baitoan}

\begin{baitoan}[\cite{An_400_BT_Hoa_Hoc_8_2020}, 292., p. 144]
	Có 4 ống đựng riêng biệt các khí sau: không khí, khí oxy, khí hydro, khí carbonic. Bằng cách nào có thể phân biệt được các chất khí trong mỗi ống?
\end{baitoan}

\begin{baitoan}[\cite{An_400_BT_Hoa_Hoc_8_2020}, 293., p. 145]
	(a) 1 oxyde base có thành phần \% khối lượng của oxy là $7.17$\%. Tìm CTPT của oxyde biết kim loại hóa trị II. (b) Muốn điều chế $31.05$\emph{g} kim loại trên cần bao nhiêu \emph{l} khí \emph{\ce{H2}} (đktc)?
\end{baitoan}

\begin{baitoan}[\cite{An_400_BT_Hoa_Hoc_8_2020}, 294., p. 145]
	Dùng \emph{\ce{H2}} để khử $a$\emph{g} \emph{CuO} thu được $b$\emph{g} \emph{Cu}. Cho lượng đồng này tác dụng với \emph{\ce{Cl2}} thu được $33.75$\emph{g} \emph{\ce{CuCl2}}. Tính $a,b$.
\end{baitoan}

\begin{baitoan}[\cite{An_400_BT_Hoa_Hoc_8_2020}, 295., p. 145]
	Cho hỗn hợp \emph{\ce{CuO,Fe2O3}} tác dụng với \emph{\ce{H2}} ở nhiệt độ thích hợp. Hỏi nếu thu được $26.4$\emph{g} hỗn hợp \emph{Cu,Fe}, trong đó khối lượng \emph{Cu} gấp $1.2$ lần khối lượng \emph{Fe} thì cần dùng tất cả bao nhiêu \emph{l} khí hydro?
\end{baitoan}

\begin{baitoan}[\cite{An_400_BT_Hoa_Hoc_8_2020}, 296., p. 145]
	Dùng \emph{\ce{H2}} khử $31.2$\emph{g} hỗn hợp \emph{\ce{CuO,Fe3O4}}, trong hỗn hợp khối lượng \emph{\ce{Fe3O4}} hơn khối lượng \emph{CuO} là $15.2$\emph{g}. Tính khối lượng \emph{Cu,Fe} thu được.
\end{baitoan}

\begin{baitoan}[\cite{An_400_BT_Hoa_Hoc_8_2020}, 297., p. 145]
	Cho \emph{\ce{H2}} khử $16$\emph{g} hỗn hợp \emph{\ce{Fe2O3,CuO}}, trong đó khối lượng \emph{CuO} chiếm $25$\%. (a) Tính khối lượng \emph{Fe,Cu} thu được sau phản ứng. (b) Tính tổng thể tích \emph{\ce{H2}} đã tham gia phản ứng.
\end{baitoan}

\begin{baitoan}[\cite{An_400_BT_Hoa_Hoc_8_2020}, 298., p. 145]
	Cho hỗn hợp \emph{\ce{PbO,Fe2O3}} tác dụng với \emph{\ce{H2}} ở nhiệt độ cao. Hỏi nếu thu được $52.6$\emph{g} hỗn hợp \emph{\ce{Pb,Fe}}, trong đó khối lượng \emph{Pb} gấp $3.696$ lần khối lượng \emph{Fe} thì cần dùng tất cả bao nhiêu \emph{l} \emph{\ce{H2}} (đktc)?
\end{baitoan}

\begin{baitoan}[\cite{An_400_BT_Hoa_Hoc_8_2020}, 299., p. 145]
	Cho $8.4$\emph{l} khí hydro tác dụng với $2.8$\emph{l} khí oxy. Tính số \emph{g} nước tạo thành, biết các khí đo ở đktc.
\end{baitoan}

\begin{baitoan}[\cite{An_400_BT_Hoa_Hoc_8_2020}, 300., p. 145]
	Có 1 hỗn hợp gồm $60$\% \emph{\ce{Fe2O3}} \& $40$\% \emph{CuO}. Dùng \emph{\ce{H2}} (dư) để khử $20$\emph{g} hỗn hợp đó. (a) Tính khối lượng \emph{Fe,Cu} thu được sau phản ứng. (b) Tính số mol \emph{\ce{H2}} đã tham gia phản ứng.
\end{baitoan}

\begin{baitoan}[\cite{An_400_BT_Hoa_Hoc_8_2020}, 301., p. 145]
	Dùng khí hydro hoặc khí carbon oxyde để khử sắt (III) oxyde thành sắt. Để điều chế $35$\emph{g} sắt, tính thể tích khí hydro \& thể tích khí carbon oxyde lần lượt là (các khí đo ở đktc): {\sf A.} $42$\emph{l}, $21$\emph{l}. {\sf B.} $42$\emph{l}, $42$\emph{l}. {\sf C.} $10.5$\emph{l}, $21$\emph{l}. {\sf D.} $21$\emph{l}, $21$\emph{l}.
\end{baitoan}

\begin{baitoan}[\cite{An_400_BT_Hoa_Hoc_8_2020}, 302., p. 145]
	Trường hợp nào sau đây chứa 1 khối lượng hydro ít nhất? {\sf A.} $6\cdot10^{23}$ phân tử \emph{\ce{H2}}. {\sf B.} $3\cdot10^{23}$ phân tử \emph{\ce{H2O}}. {\sf C.} $0.6$\emph{g} \emph{\ce{CH4}}. {\sf D.} $1.5$\emph{g} \emph{\ce{NH4Cl}}.
\end{baitoan}

%------------------------------------------------------------------------------%

\section{Điều Chế Hydro. Phản Ứng Thế}
``\fbox{\bf 1} \textbf{Điều chế hydro.} (a) \textit{Trong phòng thí nghiệm}: Cho các kim loại hoạt động như kẽm, nhôm, sắt, $\ldots$ tác dụng với dung dịch acid hydrochloric hay dung dịch acid sulfuric loãng, e.g., \ce{Zn + $2$HCl -> ZnCl2 + H2 ^ , $2$Al +  $3$H2SO4 -> Al2(SO4)3 + $3$H2 ^}. Thu \ce{H2} vào ống nghiệm (hoặc lọ) bằng cách đẩy không khí hay đẩy nước. (b) \textit{Trong công nghiệp}: Điện phân nước: \ce{$2$H2O ->[\mbox{điện phân}] $2$H2 ^ + O2 ^}. Khử oxy của \ce{H2O} trong lò khí than: \ce{H2O \mbox{(hơi)} + C \mbox{(nóng đỏ)} ->[$t^\circ$] CO ^ + H2 ^}. Phân hủy khí metan ở nhiệt độ cao: \ce{CH4 ->[$t^\circ$] C + $2$H2 ^}. ``\fbox{\bf 2} \textbf{Phản ứng thế.} \textit{Phản ứng thế} là phản ứng hóa học trong đó nguyên tử của đơn chất thay thế nguyên tử của 1 nguyên tố khác trong hợp chất.'' -- \cite[Chap. 5, \S3, pp. 82--83]{Truong_BTNC_Hoa_Hoc_8_2022}

\begin{vidu}
	Ngâm đinh sắt trong dung dịch \emph{\ce{CuSO4}} màu xanh, sau 1 thời gian thấy dung dịch nhạt dần màu xanh \& có đồng màu đỏ bám lên đinh sắt. \emph{\ce{Fe + CuSO4 -> FeSO4 + Cu v}}.
\end{vidu}

\begin{baitoan}[\cite{Truong_BTNC_Hoa_Hoc_8_2022}, V.13, p. 83]
	Cho biết thế nào là: (a) phản ứng hóa hợp? (b) phản ứng phân hủy? (c) phản ứng thế? Đối với mỗi loại phản ứng, cho 2 ví dụ minh họa.
\end{baitoan}

\begin{baitoan}[\cite{Truong_BTNC_Hoa_Hoc_8_2022}, V.14, p. 83]
	Có 3 lọ, mỗi lọ đựng 1 chất lỏng không màu sau: nước, nước vôi trong, dung dịch acid sulfuric loãng. Nêu phương pháp hóa học nhận biết mỗi chất \& viết PTHH (nếu có phản ứng xảy ra).
\end{baitoan}

\begin{baitoan}[\cite{Truong_BTNC_Hoa_Hoc_8_2022}, V.15, p. 83]
	Có những chất sau: \emph{Zn, Cu, Al, \ce{H2O, C12H22O11, KMnO4, KClO3}}, dung dịch \emph{HCl}, dung dịch \emph{\ce{H2SO4}} loãng. (a) Những chất nào có thể dùng để điều chế khí hydro? (b) Những chất nào có thể dùng để điều chế khí oxy? Viết PTHH của các phản ứng xảy ra \& nói cách thu khí \emph{\ce{H2,O2}}.
\end{baitoan}

\begin{baitoan}[\cite{Truong_BTNC_Hoa_Hoc_8_2022}, V.16, p. 83]
	Trong phòng thí nghiệm, cho kẽm hoặc sắt tác dụng với dung dịch acid hydrochloric để điều chế hydro. Nếu muốn điều chế $5.6$\emph{l \ce{H2}} (đktc) thì phải dùng: (a) bao nhiêu \emph{g} kẽm? (b) bao nhiêu \emph{g} sắt?
\end{baitoan}

\begin{baitoan}[\cite{Truong_BTNC_Hoa_Hoc_8_2022}, V.17, p. 84]
	Cho $13$\emph{g} kẽm vào 1 dung dịch chứa $0.5$\emph{mol} acid hydrochloric. (a) Tính thể tích \emph{\ce{H2}} thu được ở đktc. (b) Sau phản ứng, chất nào còn dư \& dư bao nhiêu \emph{g}?
\end{baitoan}

\begin{baitoan}[\cite{Truong_BTNC_Hoa_Hoc_8_2022}, V.18, p. 84]
	Cho phoi bào sắt vào 1 dung dịch chứa $0.4$\emph{mol \ce{H2SO4}}. Sau 1 thời gian sắt tan hoàn toàn \& thu được $3.36$\emph{l \ce{H2}} (đktc). (a) Tính khối lượng sắt đã phản ứng. (b) Sau phản ứng còn \emph{\ce{H2SO4}} không \& nếu dư thì dư bao nhiêu \emph{g}?
\end{baitoan}

\begin{baitoan}[\cite{Truong_BTNC_Hoa_Hoc_8_2022}, V.19, p. 84]
	Tính lượng kẽm cần dùng để điều chế đủ hydro (đktc) bơm vào 1 quả bóng thám không có dung tích $\rm4.48m^3$ khi cho kẽm tác dụng với acid hydrochloric.
\end{baitoan}

\begin{baitoan}[\cite{Truong_BTNC_Hoa_Hoc_8_2022}, V.20, p. 84]
	Trong bình đốt khí, dùng tia lửa điện để đốt 1 hỗn hợp gồm $\rm56cm^3$ hydro \& $\rm40cm^3$ oxy. (a) Tính khối lượng nước tạo thành sau phản ứng. (b) Sau phản ứng có thừa khí nào hay không? Bao nhiêu $\rm cm^3$? (Các thể tích được đo ở đktc).
\end{baitoan}

\begin{baitoan}[\cite{Truong_BTNC_Hoa_Hoc_8_2022}, V.21, p. 84]
	Điện phân $1$\emph{l} nước (ở $4^\circ$C) thì được bao nhiêu \emph{l} khí \emph{\ce{H2}} \& bao nhiêu \emph{l} khí \emph{\ce{O2}} (đktc)? Biết khối lượng riêng của nước ở $4^\circ$C là $D = 1$\emph{g\texttt{/}ml}.
\end{baitoan}

\begin{baitoan}[\cite{Truong_BTNC_Hoa_Hoc_8_2022}, V.22, p. 84]
	Phân hủy $45$\emph{g} nước bằng dòng điện. (a) Tính khối lượng hydro, khối lượng oxy thu được. Tính tỷ số: $\frac{\mbox{khối lượng hydro}}{\mbox{khối lượng oxy}}$. (b) Tính thể tích khí hydro, thể tích khí oxy thu được (đktc). Tính tỷ số: $\frac{\mbox{thể tích hydro}}{\mbox{thể tích oxy}}$.
\end{baitoan}

\begin{baitoan}[\cite{Truong_BTNC_Hoa_Hoc_8_2022}, V.23, pp. 84--85]
	Cho các sơ đồ phản ứng sau: (a) \emph{\ce{P + O2 -> P2O5}}. (b) \emph{\ce{HgO -> Hg + O2}}. (c) \emph{\ce{Al + HCl -> AlCl3 + H2}}. (d) \emph{\ce{Fe + CuCl2 -> FeCl2 + Cu}}. Lập PTHH các phản ứng trên \& cho biết chúng thuộc loại phản ứng nào.
\end{baitoan}

\begin{baitoan}[\cite{Truong_BTNC_Hoa_Hoc_8_2022}, V.24, p. 85]
	Cho biết những phản ứng hóa học nào dưới đây có thể được dùng điều chế khí hydro: (1) Trong phòng thí nghiệm. (2) Trong công nghiệp. Vì sao? (a) \emph{\ce{$2$H2O ->[\mbox{điện phân}] $2$H2 ^ + O2 ^}}. (b) \emph{\ce{Fe + $2$HCl -> FeCl2 + H2 ^}}. (c) \emph{\ce{Zn + H2SO4 -> ZnSO4 + H2 ^}}.
\end{baitoan}

\begin{baitoan}[\cite{Truong_BTNC_Hoa_Hoc_8_2022}, V.25, p. 85]
	Dẫn ra 1 PTHH đối với mỗi phản ứng sau \& cho biết phản ứng thuộc loại nào. (a) Oxy hóa đơn chất bằng khí oxy. (b) Khử oxide kim loại bằng khí hydro. (c) Đẩy hydro trong acid bằng kim loại. (d) Phản ứng giữa oxide kim loại với nước. (e) Phản ứng giữa oxide phi kim với nước.
\end{baitoan}

\begin{baitoan}[\cite{Truong_BTNC_Hoa_Hoc_8_2022}, V.26, pp. 85--86]
	Có những sơ đồ phản ứng hóa học sau: (a) \emph{\ce{Mg + HCl -> MgCl2 + H2}}. (b) \emph{\ce{H2 + O2 -> H2O}}. (c) \emph{\ce{PbO + H2 -> Pb + H2O}}. (d) \emph{\ce{KClO3 -> KCl + O2}}. (e) \emph{\ce{CaCO3 + CO2 + H2O -> Ca(HCO3)2}}. (f) \emph{\ce{Fe + O2 -> Fe3O4}}. (g) \emph{\ce{CuO + CO -> Cu + CO2}}. (h) \emph{\ce{Fe + CuSO4 -> FeSO4 + Cu}}. (i) \emph{\ce{Al + H2SO4 -> Al2(SO4)3 + H2}}. (j) \emph{\ce{Zn + Cl2 -> ZnCl2}}. Lập PTHH của các phản ứng trên \& cho biết phản ứng thuộc loại nào.
\end{baitoan}

\begin{baitoan}[\cite{Truong_BTNC_Hoa_Hoc_8_2022}, V.27, p. 86]
	Từ những chất: \emph{Mg,Al,S}, dung dịch acid hydrochloric \emph{HCl, \ce{KClO3}, PbO}, viết PTHH để điều chế các chất \emph{Pb,\ce{SO2,MgO,Al2O3}}.
\end{baitoan}

\begin{baitoan}[\cite{Truong_BTNC_Hoa_Hoc_8_2022}, V.28, p. 86]
	Khử $0.15$\emph{mol \ce{Fe2O3}} ở nhiệt độ cao bằng những chất khác nhau: khí \emph{CO}, khí \emph{\ce{H2}}, bột \emph{Al}. (a) Viết PTHH các phản ứng xảy ra. (b) Các phản ứng hóa học trên thuộc loại phản ứng nào? Cho biết vai trò của mỗi chất tham gia ở các phản ứng trên. (c) Tính thể tích (đktc) của chất khử thể khí \& khối lượng của chất khử thể rắn đã dùng. (d) Khối lượng sắt thu được sau các phản ứng trên có khác nhau không? Giải thích. Khối lượng là bao nhiêu?
\end{baitoan}

\begin{baitoan}[\cite{Truong_BTNC_Hoa_Hoc_8_2022}, V.29, p. 86]
	Cho hỗn hợp khí \emph{CO,\ce{CO2}} đi qua dung dịch \emph{\ce{Ca(OH)2}} (còn gọi là \emph{nước vôi trong}) dư, thu được $1$\emph{g} chất kết tủa trắng. Nếu cho hỗn hợp khí này đi qua bột \emph{CuO} nóng, dư thì thu được $0.64$\emph{g} đồng. (a) Viết PTHH các phản ứng xảy ra. (b) Tính thể tích của hỗn hợp khí ở đktc \& thể tích của mỗi khí có trong hỗn hợp. (c) Bằng phương pháp hóa học nào có thể tác riêng mỗi khí ra khỏi hỗn hợp? Viết PTHH các phản ứng xảy ra.
\end{baitoan}

\begin{baitoan}[\cite{Truong_BTNC_Hoa_Hoc_8_2022}, V.30, pp. 86--87]
	Viết PTHH thực hiện những biến đổi sau: (a) Từ các chất: \emph{\ce{KMnO4},Fe,Cu,HCl} điều chế các chất cần thiết để thực hiện biến đổi: \emph{Cu $\to$ CuO $\to$ Cu}. (b) Từ các chất: \emph{\ce{KClO3,Zn,Fe,H2SO4}} loãng, điều chế các chất cần thiết để thực hiện biến đổi: \emph{Fe $\to$ \ce{Fe3O4} $\to$ Fe}.
\end{baitoan}

\begin{baitoan}[\cite{Truong_BTNC_Hoa_Hoc_8_2022}, V.31, p. 87]
	Cho $3.25$\emph{g Zn} tác dụng với 1 lượng dung dịch \emph{HCl} vừa đủ. Dẫn toàn bộ lượng khí sinh ra cho đi qua $6$\emph{g CuO} đun nóng. (a) Viết PTHH các phản ứng xảy ra. (b) Tính khối lượng \emph{Cu} thu được sau phản ứng \& cho biết chất nào là chất khử? Chất oxi hóa? (c) Chất nào còn dư sau phản ứng hydro khử \emph{CuO}? Khối lượng của nó là bao nhiêu?
\end{baitoan}

\begin{baitoan}[\cite{Truong_BTNC_Hoa_Hoc_8_2022}, V.32, p. 87]
	Khử hoàn toàn $5.43$\emph{g} 1 hỗn hợp gồm có \emph{CuO,PbO} bằng khí \emph{\ce{H2}}, thu được $0.9$\emph{g \ce{H2O}}. (a) Viết PTHH các phản ứng đã xảy ra. (b) Tính thành phần \% theo khối lượng của các oxide có trong hỗn hợp ban đầu. (c) Tính thành phần \% theo khối lượng của hỗn hợp chất rắn thu được sau phản ứng.
\end{baitoan}

\begin{baitoan}[\cite{Truong_BTNC_Hoa_Hoc_8_2022}, V.33, p. 87]
	(a) Khử hoàn toàn $5.575$\emph{g} 1 oxide chì bằng khí \emph{\ce{H2}}, thu được $5.175$\emph{g} chì. Tìm CTHH của oxide chì. (b) Khử hoàn toàn $4$\emph{g} 1 oxide đồng bằng khí \emph{\ce{H2}}, thu được $3.2$\emph{g} đồng. Tìm CTHH của oxide đồng.
\end{baitoan}

\begin{baitoan}[\cite{An_400_BT_Hoa_Hoc_8_2020}, 303., p. 146]
	Lập PTHH \& xác định loại phản ứng. (a) sắt $+$ acid hydrochloric $\to$ ?; (b) kali clorat \ce{->[$t^\circ$]} ?; (c) sắt $+$ đồng sunfat $\to$ ?; (d) nhôm $+$ oxy $\to$ ?; (e) nước \ce{->[{điện phân}][acid sulfuric]} ?; (f) khí carbonic $+$ magie $\to$ ?.
\end{baitoan}

\begin{baitoan}[\cite{An_400_BT_Hoa_Hoc_8_2020}, 304., p. 146]
	Điện phân 1 lượng nước thu được khí hydro \& oxy. Nếu dùng lượng khí \emph{\ce{H2}} thu được để khử sắt (III) oxyde thu được $16.8$\emph{g}. Hỏi phải điện phân bao nhiêu \emph{l} nước biết $D_{\ce{H2O}} = 1$\emph{g\texttt{/}ml}?
\end{baitoan}

\begin{baitoan}[\cite{An_400_BT_Hoa_Hoc_8_2020}, 305., p. 146]
	Cho $11.2$\emph{g} sắt tác dụng với dung dịch \emph{\ce{H2SO4}} loãng có chứa $12.25$\emph{g \ce{H2SO4}}. (a) Chất nào còn dư sau phản ứng \& dư bao nhiêu \emph{g}? (b) Tính thể tích khí hydro thu được ở đktc.
\end{baitoan}

\begin{baitoan}[\cite{An_400_BT_Hoa_Hoc_8_2020}, 306., p. 146]
	Cho các kim loại \emph{K,Ca,Al} lần lượt tác dụng với dung dịch \emph{HCl}. (a) Nếu cho cùng số mmol của 1 trong các kim loại trên tác dụng với acid \emph{HCl} thì kim loại nào cho nhiều \emph{\ce{H2}} hơn? (b) Nếu thu được cùng số mol khí \emph{\ce{H2}} thì khối lượng kim loại nào ít hơn?
\end{baitoan}

\begin{baitoan}[\cite{An_400_BT_Hoa_Hoc_8_2020}, 307., p. 146]
	Cho $5.4$\emph{g Al} vào dung dịch \emph{\ce{H2SO4}} loãng có chứa $39.2$\emph{g \ce{H2SO4}}. (a) Chất nào còn dư sau phản ứng \& dư bao nhiêu \emph{g}? (b) Tính thể tích khí hydro thu được ở đktc.
\end{baitoan}

\begin{baitoan}[\cite{An_400_BT_Hoa_Hoc_8_2020}, 308., p. 146]
	Cho $5.1$\emph{g} hỗn hợp \emph{Al,Mg} vào dung dịch \emph{\ce{H2SO4}} loãng, dư thu được $5.6$\emph{l} khí \emph{\ce{H2}} (đktc). Tính khối lượng mỗi kim loại ban đầu. Biết phản ứng xảy ra hoàn toàn.
\end{baitoan}

\begin{baitoan}[\cite{An_400_BT_Hoa_Hoc_8_2020}, 309., p. 147]
	Cho kẽm hoặc sắt tác dụng với dung dịch acid hydrochloride \emph{HCl} để điều chế khí hydro. Nếu muốn điều chế $2.24$\emph{l} khí hydro (đktc) thì phải dùng số \emph{g} kẽm hoặc sắt lần lượt là: {\sf A.} $6.5$\emph{g}, $5.6$\emph{g}. {\sf B.} $16$\emph{g}, $8$\emph{g}. {\sf C.} $13$\emph{g}, $11.2$\emph{g}. {\sf D.} $9.75$\emph{g}, $8.4$\emph{g}.
\end{baitoan}

\begin{baitoan}[\cite{An_400_BT_Hoa_Hoc_8_2020}, 310., p. 147]
	Điện phân hoàn toàn $2$\emph{l} nước ở trạng thái lỏng (biết khối lượng riêng $D$ của nước là $1$\emph{kg\texttt{/}l}). Tính thể tích khí hydro \& thể tích khí oxy thu được.
\end{baitoan}

\begin{baitoan}[\cite{An_400_BT_Hoa_Hoc_8_2020}, 311., p. 147]
	So sánh thể tích khí hydro (đktc) thu được trong mỗi trường hợp sau: (a) $0.1$\emph{mol Zn} tác dụng với dung dịch \emph{\ce{H2SO4}} loãng dư. $0.1$\emph{mol Al} tác dụng với dung dịch \emph{\ce{H2SO4}} loãng dư. (b) $0.2$\emph{mol Zn} tác dụng với dung dịch \emph{HCl} dư. $0.2$\emph{mol Al} tác dụng với dung dịch \emph{HCl} dư.
\end{baitoan}

\begin{baitoan}[\cite{An_400_BT_Hoa_Hoc_8_2020}, 312., p. 147]
	Dùng hydro để khử hoàn toàn $a$\emph{g} \emph{\ce{Fe2O3}} \& thu được $b$\emph{g Fe}. Cho lượng sắt này tác dụng với dung dịch \emph{\ce{H2SO4}} loãng dư thì thu được $5.6$\emph{l} khí \emph{\ce{H2}} (ở đktc). Tính $a,b$.
\end{baitoan}

\begin{baitoan}[\cite{An_400_BT_Hoa_Hoc_8_2020}, 313., p. 147]
	Cho lá sắt có khối lượng $50$\emph{g} vào 1 dung dịch đồng sunfat. Sau 1 thời gian, nhấc lá sắt ra thì khối lượng lá sắt là $51$\emph{g}. Tính số \emph{mol} muối sắt tạo thành sau phản ứng biết tất cả đồng sinh ra bám trên bề mặt lá sắt.
\end{baitoan}

\begin{baitoan}[\cite{An_400_BT_Hoa_Hoc_8_2020}, 314., p. 147]
	Nhúng 1 lá nhôm vào dung dịch \emph{\ce{CuSO4}}. Sau phản ứng lấy lá nhôm ra thấy khối lượng dung dịch nhẹ đi $1.38$\emph{g}. Tính khối lượng nhôm đã phản ứng.
\end{baitoan}

%------------------------------------------------------------------------------%

\section{Nước}
``\fbox{\bf 1} \textbf{Thành phần hóa học của nước.} \textit{Nước} là hợp chất tạo bởi 2 nguyên tố là hydro \& oxy, chúng đã hóa hợp với nhau theo 1 tỷ lệ nhất định là: Tỷ lệ về thể tích: 2 phần khí hydro \& 1 phần khí oxi. Tỷ lệ về khối lượng: 11 phần hydro \& 89 phần oxi. \fbox{\bf 2} \textbf{Tính chất của nước.} (a) \textit{Tính chất vật lý}: Nước là chất lỏng không màu, không mùi, không vị. Dưới áp suất của khí quyển, nước sôi ở $100^\circ$C \& đông đặc (hóa rắn) ở $0^\circ$C. Ở $4^\circ$C, nước có khối lượng riêng $D = 1$\emph{g\texttt{/}ml}. Nước hòa toàn được nhiều chất rắn, chất lỏng, \& chất khí. (b) \textit{Tính chất hóa học}: Tác dụng với kim loại: Nước tác dụng với 1 số kim loại ở nhiệt độ thường như K, Na, Ca, $\ldots$ \& 1 số kim loại ở nhiệt độ cao như Zn, Fe, Al, $\ldots$, e.g., \ce{$2$K + $2$H2O -> $2$KOH + H2 ^ , Ca + $2$H2O -> Ca(OH)2 +  H2 ^ , Fe + H2O ->[$t^\circ$] FeO + H2 ^}. Tác dụng với oxide: Nước tác dụng với 1 số oxide kim loại tạo ra base, e.g., \ce{K2O + H2O -> $2$KOH, CaO + H2O -> Ca(OH)2}. Nước tác dụng với nhiều oxide phi kim tạo ra acid, e.g., \ce{CO2 + H2O -> H2CO3, SO3 + H2O -> H2SO4}. \fbox{\bf 3} \textbf{Nhận biết dung dịch acid \& dung dịch base.} Dùng quỳ tím: Dung dịch acid làm cho quỳ tím chuyển thành màu đỏ. Dung dịch base làm cho quỳ tím chuyển thành màu xanh.'' -- \cite[Chap. 5, \S4, pp. 87--88]{Truong_BTNC_Hoa_Hoc_8_2022}

\begin{baitoan}[\cite{Truong_BTNC_Hoa_Hoc_8_2022}, V.34, p. 88]
	Có 4 lọ đựng riêng biệt: nước cất, dung dịch \emph{NaOH}, dung dịch \emph{\ce{H2SO4}}, dung dịch \emph{NaCl}. Bằng cách nào có thể nhận biết được từng chất trong mỗi lọ?
\end{baitoan}

\begin{baitoan}[\cite{Truong_BTNC_Hoa_Hoc_8_2022}, V.35, p. 88]
	Khối lượng nước trên hành tinh của chúng ta có chừng $1.4\cdot10^{18}$ tấn. Tính khối lượng nguyên tố hydro \& oxy có trong lượng nước này.
\end{baitoan}

\begin{baitoan}[\cite{Truong_BTNC_Hoa_Hoc_8_2022}, V.36, p. 89]
	Trong ống đựng khí có chứa hỗn hợp gồm $10$\emph{ml} hydro \& $10$\emph{ml} oxy. Bật tia lửa điện để đốt hỗn hợp khí. Viết phương trình của phản ứng hóa học đã xảy ra \& cho biết khí nào còn dư sau phản ứng (sau khi đã làm lạnh ống) \& dư bao nhiêu? Các thể tích khí đều đo ở đktc.
\end{baitoan}

\begin{baitoan}[\cite{Truong_BTNC_Hoa_Hoc_8_2022}, V.37, p. 89]
	Bằng những phương pháp nào có thể chứng minh được thành phần định tính \& định lượng của nước? Viết PTHH của các phản ứng xảy ra.
\end{baitoan}

\begin{baitoan}[\cite{Truong_BTNC_Hoa_Hoc_8_2022}, V.38, p. 89]
	Thể tích nước ở trạng thái lỏng sẽ thu được bao nhiêu khi đốt $112$\emph{l} khí \emph{\ce{H2}} (đktc) với khí \emph{\ce{O2}} dư?
\end{baitoan}

\begin{baitoan}[\cite{Truong_BTNC_Hoa_Hoc_8_2022}, V.39, p. 89]
	Cho các sơ đồ phản ứng sau: \emph{\ce{Na2O + H2O -> NaOH, BaO + H2O -> Ba(OH)2, SO2 + H2O -> H2SO3, P2O5 + H2O -> H3PO4}}. (a) Lập PTHH của các phản ứng đó \& cho biết chúng thuộc loại phản ứng nào. (b) Các sản phẩm tạo thành, chất nào là base? Chất nào là acid? Cách nhận biết dung dịch base? Dung dịch acid?
\end{baitoan}

\begin{baitoan}[\cite{Truong_BTNC_Hoa_Hoc_8_2022}, V.40, p. 89]
	Viết PTHH biểu diễn các biến hóa sau: (a) \emph{\ce{Na \to Na2O \to NaOH}}. (b) \emph{\ce{Ca \to CaO \to Ca(OH)2}}. (c) \emph{\ce{C \to CO2 \to H2CO3}}. (d) \emph{\ce{P \to P2O5 \to H3PO4}}. (e) \emph{\ce{S \to SO2 \to SO3 \to H2SO4}}. Cho biết mỗi phản ứng đó thuộc loại phản ứng nào?
\end{baitoan}

\begin{baitoan}[\cite{Truong_BTNC_Hoa_Hoc_8_2022}, V.41, pp. 89--90]
	Đốt $\rm10cm^3$ khí hydro trong $\rm10cm^3$ khí oxy. Thể tích chất khí hoặc hơi còn lại sau phản ứng ở $100^\circ$C \& áp suất của khí quyển là: {\sf A.} $\rm5cm^3$ hydro \& $\rm10cm^3$ hơi nước. {\sf B.} $\rm10cm^3$ hydro \& $\rm10cm^3$ hơi nước. {\sf C.} Chỉ có \& $\rm10cm^3$ hơi nước. {\sf D.} $\rm5cm^3$ oxi \& $\rm10cm^3$ hơi nước. 
\end{baitoan}

\begin{baitoan}[\cite{Truong_BTNC_Hoa_Hoc_8_2022}, V.42, p. 90]
	Có 5 lọ đựng riêng biệt các chất lỏng sau: nước, rượt etylic, dung dịch \emph{HCl}, dung dịch \emph{NaOH}, dung dịch \emph{\ce{Ca(OH)2}}. Bằng phương pháp hóa học nào có thể nhận biết được mỗi chất?
\end{baitoan}

\begin{baitoan}[\cite{Truong_BTNC_Hoa_Hoc_8_2022}, V.43, p. 90]
	Viết các PTHH \& dùng quỳ tím để chứng minh: (a) \emph{\ce{CO2,SO2,SO3,N2O5,P2O5}} là các oxide acid. (b) \emph{\ce{Na2O,K2O,CaO,BaO}} là các oxide base.
\end{baitoan}

\begin{baitoan}[\cite{Truong_BTNC_Hoa_Hoc_8_2022}, V.44, p. 90]
	Đối với mỗi loại hợp chất là base \& acid, viết 3 PTHH của nước với những oxide tương ứng.
\end{baitoan}

\begin{baitoan}[\cite{Truong_BTNC_Hoa_Hoc_8_2022}, V.45, p. 90]
	Cho những chất sau: \emph{\ce{P2O5,Ag,H2O,KClO3,Cu,CO2,Zn,Na2O,SFe2O3,CaCO3,HCl}}. Chọn dùng trong số những chất trên để điều chế những chất dưới đây bằng cách viết các PTHH của các phản ứng (ghi điều kiện, nếu có): \emph{\ce{NaOH,Ca(OH)2,H2SO3,H2CO3,Fe,H2,O2}}.
\end{baitoan}

\begin{baitoan}[\cite{Truong_BTNC_Hoa_Hoc_8_2022}, V.46, p. 90]
	Cho $17.2$\emph{g} hỗn hợp \emph{Ca,CaO} tác dụng với lượng nước dư thì được $3.36$\emph{l \ce{H2}} ở đktc. (a) Viết PTHH của các phản ứng xảy ra \& tính khối lượng mỗi chất có trong hỗn hợp. (b) Tính khối lượng của chất tan trong dung dịch sau phản ứng.
\end{baitoan}

%------------------------------------------------------------------------------%

\section{Acid, Base, Muối}

\begin{baitoan}[\cite{An_400_BT_Hoa_Hoc_8_2020}, 315., p. 147]
	(a) Viết công thức các acid \& base tương ứng với các oxyde sau: \emph{\ce{MgO,Al2O3,SO2,SiO2,SO3}, \ce{CO2,P2O5,N2O5,Fe2O3}}. (b) Cho các CTHH: \emph{\ce{CaCO3,Na2SO3,Cu2O,Na2O,HCl,ZnSO4,Fe(OH)3,H3PO4,Ca(OH)2,Al(OH)3}, \ce{Cu(OH)2,CO,CO2,NO,KHSO4,N2O5,Fe2O3,SO3,P2O5,HNO3,H2O,Fe(NO3)3,Fe2(SO4)3,Na3PO4,CaO,CuO,NaHCO3,FeO}}. Gọi tên từng chất \& cho biết mỗi chất thuộc loại nào.
\end{baitoan}

\begin{baitoan}[\cite{An_400_BT_Hoa_Hoc_8_2020}, 316., p. 148]
	Cho 1 hỗn hợp chứa $4.6$\emph{g} natri \& $3.9$\emph{g} kali tác dụng với nước. (a) Viết PTHH. (b) Tính thể tích khí hydro thu được (đktc). (c) Dung dịch sau phản ứng làm biến đổi màu giấy quỳ tím như thế nào?
\end{baitoan}

\begin{baitoan}[\cite{An_400_BT_Hoa_Hoc_8_2020}, 317., p. 148]
	Cho các nguyên tố hóa học: natri, đồng, photpho, magie, nhôm, carbon, lưu huỳnh. (a) Viết công thức các oxyde của những nguyên tố này theo hóa trị cao nhất của chúng. (b) Viết PTHH của các oxyde trên (nếu có) với nước. (c) Dung dịch nào phản ứng làm biến đổi màu giấy quỳ tím?
\end{baitoan}

\begin{baitoan}[\cite{An_400_BT_Hoa_Hoc_8_2020}, 318., p. 148]
	Nếu cho $210$\emph{kg} vôi sống \emph{CaO} tác dụng với nước. Tính lượng \emph{\ce{Ca(OH)2}} thu được theo lý thuyết. Biết vôi sống có $10$\% tạp chát không tác dụng với nước.
\end{baitoan}

\begin{baitoan}[\cite{An_400_BT_Hoa_Hoc_8_2020}, 319., p. 148]
	Cho các CTHH: \emph{\ce{CaCl2,Cu2O,NaO2,KSO4,Al(SO4)3,Na2PO4,AlO3,Zn(OH)2,CuOH}, \ce{MgNO3,NaCO3,CaCO3,Fe2(SO4)3,FeCO3}}. Sửa các CTHH sai.
\end{baitoan}

\begin{baitoan}[\cite{An_400_BT_Hoa_Hoc_8_2020}, 320., p. 148]
	Viết các phương trình biểu diễn chuyển hóa sau: (a) \emph{Na $\to$ \ce{Na2O} $\to$ NaOH}; (b) \emph{Ca $\to$ CaO $\to$ \ce{Ca(OH)2} $\to$ \ce{CaCO3}}; (c) \emph{\ce{H2} $\to$ \ce{H2O} $\to$ NaOH}; (d) \emph{CuO $\to$ \ce{H2O} $\to$ \ce{H2SO4} $\to$ \ce{H2}}; (e) \emph{Cu $\to$ CuO $\to$ Cu}.
\end{baitoan}

\begin{baitoan}[\cite{An_400_BT_Hoa_Hoc_8_2020}, 321., p. 148]
	Cho biết gốc acid \& tính hóa trị của gốc acid trong các acid sau: \emph{\ce{H2S,HNO3,H2SO4,H2SiO3}, \ce{H3PO4,HClO4,H2Cr2O7,CH3COOH}}.
\end{baitoan}

\begin{baitoan}[\cite{An_400_BT_Hoa_Hoc_8_2020}, 322., p. 148]
	Viết công thức của các hydroxyde ứng với các kim loại sau: natri, canxi, crom, bari, kali, đồng, kẽm, sắt, cho biết hóa trị của crom là III, của đồng là II, \& của sắt là III.
\end{baitoan}

\begin{baitoan}[\cite{An_400_BT_Hoa_Hoc_8_2020}, 323., pp. 148--149]
	(a) Lập các PTHH theo sơ đồ sau: kali oxyde $+$ nước $\to$ kali hydroxyde, kẽm $+$ acid sulfuric $\to$ kẽm sunfat $+$ hydro, magie oxyde $+$ acid nitric $\to$ magie nitrat $+$ nước, canxi $+$ acid phosphoric $\to$ canxi photphat $+$ hydro, oxy sắt từ \emph{\ce{FeO.Fe2O3}} $+$ acid hydrochloric $\to$ sắt (II) clorua $+$ sắt (III) clorua $+$ nước. (b) Cho $8.6$\emph{g} hỗn hợp \emph{Ca,CaO} tác dụng với nước dư, thu được $1.68$\emph{l} khí hydro (đktc). Tính khối lượng mỗi chất có trong hỗn hợp. Làm thế nào biết được dung dịch sau phản ứng là acid hay base?
\end{baitoan}

%------------------------------------------------------------------------------%

\section{Miscellaneous}

\begin{baitoan}[\cite{An_400_BT_Hoa_Hoc_8_2020}, 324., p. 149]
	Trong bình đốt khí, dùng tia lửa điện để đốt 1 hỗn hợp gồm $\rm28cm^3$ hydro \& $\rm20cm^3$ oxy. (a) Sau phản ứng có thừa khí nào không? Thừa bao nhiêu $\rm cm^3$? (b) Tính khối lượng nước tạo thành. Biết các thể tích khí đo ở đktc.
\end{baitoan}

\begin{baitoan}[\cite{An_400_BT_Hoa_Hoc_8_2020}, 325., p. 149]
	Cho lá kẽm có khối lượng $50$\emph{g} vào dung dịch đồng sunfat. Sau thời gian phản ứng kết thúc thì khối lượng lá kẽm là $49.82$\emph{g}. Tính: (a) Khối lượng kẽm đã tác dụng. (b) Khối lượng đồng sunfat có trong dung dịch.
\end{baitoan}

\begin{baitoan}[\cite{An_400_BT_Hoa_Hoc_8_2020}, 326., p. 149]
	Có 4 chất rắn ở dạng bột là \emph{Al,Cu,\ce{Fe2O3,CUO}}. Nếu chỉ dùng thuốc thử là dung dịch \emph{HCl} thì có thể phân biệt 4 chất trên được không? Nếu có thì viết các PTHH.
\end{baitoan}

\begin{baitoan}[\cite{An_400_BT_Hoa_Hoc_8_2020}, 327., p. 149]
	Có 4 lọ mất nhãn đựng riêng biệt: nước cất, dung dịch acid \emph{HCl}, dung dịch \emph{KOH}, dung dịch \emph{KCl}. Nêu cách phân biệt các chất trên.
\end{baitoan}

\begin{baitoan}[\cite{An_400_BT_Hoa_Hoc_8_2020}, 328., p. 149]
	Hoàn thành các PTHH: (a) \emph{\ce{Mg + HCl ->}} ?; (b) \emph{\ce{Al + H2SO4 ->}} ?; (c) \emph{\ce{MgO + HCl ->}} ?; (d) \emph{\ce{CaO + H3PO4 ->}} ?; (e) \emph{\ce{CaO + HNO4 ->}} ?.
\end{baitoan}

\begin{baitoan}[\cite{An_400_BT_Hoa_Hoc_8_2020}, 329., p. 150]
	(a) Viết công thức của các muối sau: kali clorua, canxi nitrat, đồng sunfat, natri sunfit, natri nitrat, canxi photphat, đồng carbonat. (b) Cho biết các chất dưới đây thuộc loại hợp chất nào, viết công thức của các chất đó: natri hydroxyde, khí carbonic, khí sunfurơ, sắt (III) oxyde, muối ăn, acid hydrochloric, acid phosphoric.
\end{baitoan}

\begin{baitoan}[\cite{An_400_BT_Hoa_Hoc_8_2020}, 330., p. 150]
	(a) Từ những hóa chất cho sẵn: \emph{\ce{KMnO4,Fe,dd CuSO4, dd H2SO4}} loãng, viết các PTHH để điều chế các chất theo sơ đồ chuyển hóa: \emph{Cu $\to$ CuO $\to$ Cu}. (b) Khi điện phân nước thu được $2$ thể tích \emph{\ce{H2}} \& $1$ thể tích khí \emph{\ce{O2}} (cùng điều kiện nhiệt độ, áp suất). Từ kết quả này, chứng minh CTHH của nước.
\end{baitoan}

\begin{baitoan}[\cite{An_400_BT_Hoa_Hoc_8_2020}, 331., p. 150]
	Khử $50$\emph{g} hỗn hợp đồng (II) oxyde \& sắt (II) oxyde bằng khí hydro. Tính thể tích khí hydro cần dùng, biết trong hỗn hợp, đồng (II) oxyde chiếm $20$\% về khối lượng. Các phản ứng đó thuộc loại phản ứng gì?
\end{baitoan}

\begin{baitoan}[\cite{An_400_BT_Hoa_Hoc_8_2020}, 332., p. 150]
	Dùng khí \emph{\ce{H2}} để khử $50$\emph{g} hỗn hợp A gồm đồng (II) oxyde \& sắt (III) oxyde. Biết trong hỗn hợp sắt (III) oxyt chiếm $80$\% khối lượng. Tính thể tích khí \emph{\ce{H2}} cần dùng.
\end{baitoan}

\begin{baitoan}[\cite{An_400_BT_Hoa_Hoc_8_2020}, 333., p. 150]
	Cho các chất: nhôm, oxy, nước, đồng sunfat, sắt, acid hydrochloric. Điều chế đồng, đồng (II) oxyde, nhôm clorua (bằng $2$ phương pháp) \& sắt (II) clorua. Viết các phương trình phản ứng.
\end{baitoan}

\begin{baitoan}[\cite{An_400_BT_Hoa_Hoc_8_2020}, 334., p. 150]
	Cho $60.5$\emph{g} hỗn hợp gồm 2 kim loại \emph{Zn,Fe} tác dụng với dung dịch acid hydrochloric. Thành phần \% về khối lượng của \emph{Fe} trong hỗn hợp là $46.289$\%. Tính: (a) Khối lượng mỗi chất trong hỗn hợp. (b) Thể tích khí \emph{\ce{H2}} (đktc) sinh ra khi cho hỗn hợp 2 kim loại trên tác dụng với dung dịch acid hydrochloric. (c) Khối lượng các muối tạo thành.
\end{baitoan}

\begin{baitoan}[\cite{An_400_BT_Hoa_Hoc_8_2020}, 335., p. 150]
	Cho $22.4$\emph{g} sắt tác dụng với dung dịch loãng có chứa $24.5$\emph{g} acid \emph{\ce{H2SO4}}. (a) Tính thể tích khí \emph{\ce{H2}} thu được ở đktc. (b) Chất nào thừa sau phản ứng \& thừa bao nhiêu \emph{g}?	
\end{baitoan}

\begin{baitoan}[\cite{An_400_BT_Hoa_Hoc_8_2020}, 336., p. 151]
	(a) Để đốt cháy $68$\emph{g} hỗn hợp khí hydro \& khí \emph{CO} cần $89.6$\emph{l} khí oxy (đktc). Xác định thành phần \% của hỗn hợp ban đầu. Nêu các phương pháp giải bài toán. (b) Khi khử $1.20$\emph{g} oxyde của 1 kim loại, trong đó kim loại có hóa trị cao nhất, cần dùng $\rm335cm^3$ khí hydro (ở đktc). Xác định kim loại đó.
\end{baitoan}

%------------------------------------------------------------------------------%

\printbibliography[heading=bibintoc]
	
\end{document}