\documentclass{article}
\usepackage[backend=biber,natbib=true,style=authoryear]{biblatex}
\addbibresource{/home/nqbh/reference/bib.bib}
\usepackage[utf8]{vietnam}
\usepackage{tocloft}
\renewcommand{\cftsecleader}{\cftdotfill{\cftdotsep}}
\usepackage[colorlinks=true,linkcolor=blue,urlcolor=red,citecolor=magenta]{hyperref}
\usepackage{amsmath,amssymb,amsthm,mathtools,float,graphicx,algpseudocode,algorithm,tcolorbox}
\usepackage[version=4]{mhchem}
\usepackage[inline]{enumitem}
\allowdisplaybreaks
\numberwithin{equation}{section}
\newtheorem{assumption}{Assumption}[section]
\newtheorem{baitoan}{Bài toán}
\newtheorem{cauhoi}{Câu hỏi}[section]
\newtheorem{conjecture}{Conjecture}[section]
\newtheorem{corollary}{Corollary}[section]
\newtheorem{dangtoan}{Dạng toán}[section]
\newtheorem{definition}{Definition}[section]
\newtheorem{dinhly}{Định lý}[section]
\newtheorem{dinhnghia}{Định nghĩa}[section]
\newtheorem{example}{Example}[section]
\newtheorem{ghichu}{Ghi chú}[section]
\newtheorem{hequa}{Hệ quả}[section]
\newtheorem{hypothesis}{Hypothesis}[section]
\newtheorem{lemma}{Lemma}[section]
\newtheorem{luuy}{Lưu ý}[section]
\newtheorem{nhanxet}{Nhận xét}[section]
\newtheorem{notation}{Notation}[section]
\newtheorem{note}{Note}[section]
\newtheorem{principle}{Principle}[section]
\newtheorem{problem}{Problem}[section]
\newtheorem{proposition}{Proposition}[section]
\newtheorem{question}{Question}[section]
\newtheorem{remark}{Remark}[section]
\newtheorem{theorem}{Theorem}[section]
\newtheorem{vidu}{Ví dụ}[section]
\usepackage[left=0.5in,right=0.5in,top=1.5cm,bottom=1.5cm]{geometry}
\usepackage{fancyhdr}
\pagestyle{fancy}
\fancyhf{}
\lhead{\small Sect.~\thesection}
\rhead{\small\nouppercase{\leftmark}}
\renewcommand{\subsectionmark}[1]{\markboth{#1}{}}
\cfoot{\thepage}
\def\labelitemii{$\circ$}

\title{Oxygen, Air -- Oxi, Không Khí}
\author{Nguyễn Quản Bá Hồng\footnote{Independent Researcher, Ben Tre City, Vietnam\\e-mail: \texttt{nguyenquanbahong@gmail.com}; website: \url{https://nqbh.github.io}.}}
\date{\today}

\begin{document}
\maketitle
\begin{abstract}
	\textsc{[en]} This text is a collection of problems, from easy to advanced, about oxygen. This text is also a supplementary material for my lecture note on Elementary Mathematics grade 8, which is stored \& downloadable at the following link: \href{https://github.com/NQBH/hobby/blob/master/elementary_chemistry/grade_8/NQBH_elementary_chemistry_grade_8.pdf}{GitHub\texttt{/}NQBH\texttt{/}hobby\texttt{/}elementary chemistry\texttt{/}grade 8\texttt{/}lecture}\footnote{\textsc{url}: \url{https://github.com/NQBH/hobby/blob/master/elementary_chemistry/grade_8/NQBH_elementary_chemistry_grade_8.pdf}.}. The latest version of this text has been stored \& downloadable at the following link: \href{https://github.com/NQBH/hobby/blob/master/elementary_chemistry/grade_8/oxygen/NQBH_oxygen.pdf}{GitHub\texttt{/}NQBH\texttt{/}hobby\texttt{/}elementary chemistry\texttt{/}grade 8\texttt{/}oxygen}\footnote{\textsc{url}: \url{https://github.com/NQBH/hobby/blob/master/elementary_chemistry/grade_8/oxygen/NQBH_oxygen.pdf}.}.
	\vspace{2mm}
	
	\textsc{[vi]} Tài liệu này là 1 bộ sưu tập các bài tập chọn lọc từ cơ bản đến nâng cao về biểu thức đại số. Tài liệu này là phần bài tập bổ sung cho tài liệu chính -- bài giảng \href{https://github.com/NQBH/hobby/blob/master/elementary_chemistry/grade_8/NQBH_elementary_chemistry_grade_8.pdf}{GitHub\texttt{/}NQBH\texttt{/}hobby\texttt{/}elementary chemistry\texttt{/}grade 8\texttt{/}lecture} của tác giả viết cho Toán Sơ Cấp lớp 8. Phiên bản mới nhất của tài liệu này được lưu trữ \& có thể tải xuống ở link sau: \href{https://github.com/NQBH/hobby/blob/master/elementary_chemistry/grade_8/oxygen/NQBH_oxygen.pdf}{GitHub\texttt{/}NQBH\texttt{/}hobby\texttt{/}elementary chemistry\texttt{/}grade 8\texttt{/}oxygen}.
\end{abstract}
\tableofcontents
\newpage

%------------------------------------------------------------------------------%

\section{Oxi -- Phản Ứng Hóa Hợp}

\begin{baitoan}[\cite{An_400_BT_Hoa_Hoc_8_2020}, \textbf{193.}, p. 102]
	Nêu tính chất hóa học quan trọng của oxi. Cho ví dụ minh họa.
\end{baitoan}

\begin{baitoan}[\cite{An_400_BT_Hoa_Hoc_8_2020}, \textbf{194.}, p. 102]
	Đốt cháy $6.2$\emph{g} photpho trong bình chứa $6.16$\emph{l} khí oxi (đktc) tạo thành điphotpho pentaoxit \emph{\ce{P2O5}}.
\end{baitoan}

\begin{baitoan}[\cite{An_400_BT_Hoa_Hoc_8_2020}, \textbf{195.}, p. 102]
	Đốt cháy $2.24$\emph{l} khí metan trong $28$\emph{l} không khí tạo ra khí carbonic \& hơi nước. Sau phản ứng chất nào còn thừa \& số mol thừa là bao nhiêu?
\end{baitoan}

\begin{baitoan}[\cite{An_400_BT_Hoa_Hoc_8_2020}, \textbf{196.}, p. 102]
	Đốt cháy $36$\emph{kg} than đá có chứa $0.5$\% tạp chất lưu huỳnh \& $1.5$\% tạp chất khác không cháy được. Tính thể tích khí \emph{\ce{CO2,SO2}} tạo thành (ở đktc).
\end{baitoan}

\begin{baitoan}[\cite{An_400_BT_Hoa_Hoc_8_2020}, \textbf{197.}, p. 102]
	Xác định khối lượng của những hỗn hợp các chất sau: (a) $4.5\cdot10^{23}$ phân tử oxi; $7.5\cdot10^{23}$ phân tử khí carbonic; $0.12\cdot10^{23}$ phân tử ozon. (b) $0.45\cdot10^{23}$ phân tử \emph{NaCl} \& $0.75\cdot10^{22}$ phân tử \emph{\ce{CH3COOH}} axit axetic.
\end{baitoan}

\begin{baitoan}[\cite{An_400_BT_Hoa_Hoc_8_2020}, \textbf{198.}, p. 103]
	Đốt cháy hoàn toàn $2.8$\emph{g} hỗn hợp carbon \& lưu huỳnh cần $3.36$\emph{l} \emph{\ce{O2}} (đktc). Tính khối lượng mỗi chất có trong hỗn hợp.
\end{baitoan}

\begin{baitoan}[\cite{An_400_BT_Hoa_Hoc_8_2020}, \textbf{199.}, p. 103]
	Người ta dùng đèn xì oxi-axetilen để hàn \& cắt kim loại. Phản ứng cháy của axetilen \emph{\ce{C2H2}} trong oxi tạo thành khí carbonic \& hơi nước. Tính thể tích khí oxi (đktc) cần thiết để đốt cháy $1$ \emph{mol} khí axetilen.
\end{baitoan}

\begin{baitoan}[\cite{An_400_BT_Hoa_Hoc_8_2020}, \textbf{200.}, p. 103]
	Cho biết $1.5\cdot10^{24}$ phân tử oxi: (a) Là bao nhiêu mol phân tử oxi? (b) Có khối lượng là bao nhiêu gam? (c) Có thể tích là bao nhiêu lít (đktc)?
\end{baitoan}

\begin{baitoan}[\cite{An_400_BT_Hoa_Hoc_8_2020}, \textbf{201.}, p. 103]
	(a) Trong $16$\emph{g} khí oxi có bao nhiêu mol nguyên tử oxi \& bao nhiêu mol phân tử oxi. (b) Tính tỷ khối của oxi với nitơ, với không khí.
\end{baitoan}

\begin{baitoan}[\cite{An_400_BT_Hoa_Hoc_8_2020}, \textbf{202.}, p. 103]
	Lập PTHH biểu diễn phản ứng hóa hợp của lưu huỳnh với các kim loại sau: (a) nhôm; (b) sắt; (c) chì; (d) natri. Biết các hợp chất điều chế được có CTHH là \emph{\ce{Al2S3,FeS,PbS,Na2S}}.
\end{baitoan}

\begin{baitoan}[\cite{An_400_BT_Hoa_Hoc_8_2020}, \textbf{203.}, p. 103]
	Viết PTHH của các phản ứng tạo ra các oxit \emph{\ce{SO2,Fe3O4,Al2O3,K2O}} từ các đơn chất \& cho biết của trạng thái chúng trong điều kiện bình thường.
\end{baitoan}

\begin{baitoan}[\cite{An_400_BT_Hoa_Hoc_8_2020}, \textbf{204.}, p. 103]
	Muốn dập tắt ngọn lửa do xăng, dầu cháy, người ta thường trùm vải dày hoặc phủ cát trên ngọn lửa mà không dùng nước. Giải thích.
\end{baitoan}

\begin{baitoan}[\cite{An_400_BT_Hoa_Hoc_8_2020}, \textbf{205.}, p. 103]
	Cho biết $6.72$\emph{l} khí oxi (đktc): (a) Có bao nhiêu mol oxi? (b) Có khối lượng là bao nhiêu gam? (c) Có bao nhiêu phân tử oxi?
\end{baitoan}

\begin{baitoan}[\cite{An_400_BT_Hoa_Hoc_8_2020}, \textbf{206.}, p. 103]
	Cho biết $4.5\cdot10^{23}$ phân tử oxi: (a) Có thể tích bao nhiêu lít (đktc)? (b) Có khối lượng bao nhiêu gam?
\end{baitoan}

\begin{baitoan}[\cite{An_400_BT_Hoa_Hoc_8_2020}, \textbf{207.}, p. 104]
	Tính thể tích khí oxi cần thiết để đốt cháy hoàn toàn khí metan \emph{\ce{CH4}} có trong $0.5{\rm m}^3$ khí chứa $2$\% khí không cháy. Các thể tích khí đo ở đktc.
\end{baitoan}

\begin{baitoan}[\cite{An_400_BT_Hoa_Hoc_8_2020}, \textbf{208.}, p. 104]
	Đốt cháy hoàn toàn 1 hỗn hợp khí gồm có \emph{CO} \& \emph{\ce{H2}} cần dùng $6.72$\emph{l} khí \emph{\ce{O2}}. Khí sinh ra có $4.48$\emph{l} khí \emph{\ce{CO2}}. Tính thành phần \% của hỗn hợp khí ban đầu theo thể tích hỗn hợp.
\end{baitoan}

\begin{baitoan}[\cite{An_400_BT_Hoa_Hoc_8_2020}, \textbf{209.}, p. 104]
	Giải thích vì sao $1$ \emph{mol} các chất ở trạng thái rắn, lỏng, khí, tuy có số phân tử như nhau nhưng lại có thể tích không bằng nhau?
\end{baitoan}

\begin{baitoan}[\cite{An_400_BT_Hoa_Hoc_8_2020}, \textbf{210.}, p. 104]
	Viết phương trình phản ứng của oxi lần lượt tác dụng với: (a) 3 kim loại hóa trị I, II, III; (b) 3 phi kim; (c) 3 hợp chất.
\end{baitoan}

\begin{baitoan}[\cite{An_400_BT_Hoa_Hoc_8_2020}, \textbf{211.}, p. 104]
	Trong các phản ứng hóa học sau, phản ứng nào là phản ứng hóa hợp? (cân bằng phương trình phản ứng) (a) \emph{\ce{Fe + O2 -> Fe3O4}}; (b) \emph{\ce{MgCO3 -> MgO + CO2}}; (c) \emph{\ce{CuO + H2 -> H2O + Cu}}; (d) \emph{\ce{CaO + H2O -> Ca(OH)2}}; (e) \emph{\ce{SO2 + O2 -> SO3}}; (f) \emph{\ce{Na2O + H2O -> NaOH}}.
\end{baitoan}

\begin{baitoan}[\cite{An_400_BT_Hoa_Hoc_8_2020}, \textbf{212.}, p. 104]
	Tính thể tích khí oxi (đktc) cần thiết để đốt cháy $1$\emph{kg} than biết than chứa $96$\%\emph{C} \& $4$\% tạp chất trơ. Tính khối lượng khí \emph{\ce{CO2}} sinh ra. Nêu cách nhận biết khí \emph{\ce{CO2}}.
\end{baitoan}

\begin{baitoan}[\cite{An_400_BT_Hoa_Hoc_8_2020}, \textbf{213.}, p. 104]
	Viết PTHH của các phản ứng hóa hợp của từng cặp chất sau: (a) Sắt \& clo (tạo thành sắt(III) clorua). (b) Kali \& lưu huỳnh (tạo thành kali sunfua). (c) Crom \& clo (tạo thành crom(III) clorua). (d) Đồng \& oxi (tạo thành đồng(II) oxi). (e) Nhôm \& oxi (tạo thành nhôm oxi).
\end{baitoan}

\begin{baitoan}[\cite{An_400_BT_Hoa_Hoc_8_2020}, \textbf{214.}, p. 104]
	Đốt cháy $6.4$\emph{g} lưu huỳnh trong 1 bình chứa $2.24$\emph{l} khí oxi (đktc). Tính khối lượng khí sunfurơ \emph{\ce{SO2}} thu được.
\end{baitoan}

\begin{baitoan}[\cite{An_400_BT_Hoa_Hoc_8_2020}, \textbf{215.}, p. 105]
	Đốt cháy quặng pirit sắt \emph{\ce{FeS2}} trong khí oxi thì tạo ra sắt(III) oxit \& khí sunfurơ. Viết PTHH của phản ứng.
\end{baitoan}

\begin{baitoan}[\cite{An_400_BT_Hoa_Hoc_8_2020}, \textbf{216.}, p. 105]
	Trong quá trình quang hợp, cây cối trên mỗi hecta đất trong 1 ngày hấp thụ khoảng $100$\emph{kg} carbonic \& sau khi đồng hóa cây cối nhả ra khí oxi. Tính khối lượng oxi mỗi ngày cây nhả ra. Biết số mol khí oxi do cây nhả ra bằng số mol khí carbonic được hấp thụ.
\end{baitoan}

\begin{baitoan}[\cite{An_400_BT_Hoa_Hoc_8_2020}, \textbf{217.}, p. 105]
	Viết phương trình phản ứng đốt cháy khí metan \emph{\ce{CH4}}, khí axetilen \emph{\ce{C2H2}}, rượu etylic (cồn) \emph{\ce{C2H5OH}}. Biết khi đốt cháy các chất trên cho khí carbonic \& hơi nước.
\end{baitoan}

%------------------------------------------------------------------------------%

\section{Oxi -- Phản Ứng Phân Hủy}

\begin{baitoan}[\cite{An_400_BT_Hoa_Hoc_8_2020}, \textbf{218.}, p. 105]
	Viết tên \& CTHH của 4 oxit bazơ \& 4 oxit axit. Chỉ ra các axit \& các bazơ tương ứng của mỗi oxit được nêu ra.
\end{baitoan}

\begin{baitoan}[\cite{An_400_BT_Hoa_Hoc_8_2020}, \textbf{219.}, p. 105]
	Có 1 số CTHH được viết như sau: \emph{\ce{FeOH,NaO,Ca2S,CaO,Cu2O,NaCl2,FeCl2,CuO,Al2O3}}. Chỉ ra những CTHH viết sai \& viết lại cho đúng.
\end{baitoan}

\begin{baitoan}[\cite{An_400_BT_Hoa_Hoc_8_2020}, \textbf{220.}, p. 105]
	Để sản xuất vôi sống \emph{CaO} dùng trong xây dựng \& khử độ chua của đất, người ta thường nung đá vôi. (a) Viết PTHH của phản ứng, biết khi nung đá vôi cho vôi sống \emph{CaO} \& khí \emph{\ce{CO2}}. (b) Phản ứng nung đá vôi thuộc loại phản ứng nào? Vì sao? (c) Tính khối lượng đá vôi cần dùng để điều chế $56$ tấn vôi sống.
\end{baitoan}

\begin{baitoan}[\cite{An_400_BT_Hoa_Hoc_8_2020}, \textbf{221.}, p. 105]
	Có những chất sau: sắt, cacbon, hydro, khí gas (butan \emph{\ce{C4H_{10}}}). Cho biết sự oxi hóa chất nào sẽ tạo ra: (a) Oxit thể rắn. (b) Oxit ở thể khí. (c) Oxit ở thể lỏng. (d) Oxit ở thể khí \& oxit ở thể lỏng.
\end{baitoan}

\begin{baitoan}[\cite{An_400_BT_Hoa_Hoc_8_2020}, \textbf{222.}, p. 105]
	Lập công thức các bazơ ứng với các oxit sau đây: \emph{\ce{BaO,K2O,FeO,CaO,Cr2O3}}.
\end{baitoan}

\begin{baitoan}[\cite{An_400_BT_Hoa_Hoc_8_2020}, \textbf{223.}, p. 106]
	Cho \emph{\ce{NO,CaO,P2O5,SO3,CO,ZnO,Mn2O7,N2O5,ZnO,Cu2O}}. Những chất nào là oxi axit, oxit bazơ?
\end{baitoan}

\begin{baitoan}[\cite{An_400_BT_Hoa_Hoc_8_2020}, \textbf{224.}, p. 106]
	Xác định CTHH của nhôm oxit biết tỷ lệ khối lượng của 2 nguyên tố nhôm \& oxi bằng $4.5:4$.
\end{baitoan}

\begin{baitoan}[\cite{An_400_BT_Hoa_Hoc_8_2020}, \textbf{225.}, p. 106]
	Tính khối lượng kali clorat cần thiết để điều chế được: (a) $24$\emph{g} khí oxi. (b) $33.6$\emph{l} khí oxi.
\end{baitoan}

\begin{baitoan}[\cite{An_400_BT_Hoa_Hoc_8_2020}, \textbf{226.}, p. 106]
	Khi nung nóng kali pemanganat \emph{\ce{KMnO4}}, chất này bị phân hủy cho \emph{\ce{K2MnO4,MnO2}} \& khí \emph{\ce{O2}}. Tính khối lượng \emph{\ce{KMnO4}} cần thiết để điều chế $16.8$\emph{l} khí oxi (đktc). 
\end{baitoan}

\begin{baitoan}[\cite{An_400_BT_Hoa_Hoc_8_2020}, \textbf{227.}, p. 106]
	(a) Tính số gam sắt \& oxi cần dùng để điều chế $4.64$\emph{g} oxit sắt từ \emph{\ce{Fe3O4}}. (b) Tính số gam kali clorat \emph{\ce{KClO3}} cần dùng để có được lượng oxi dùng cho phản ứng trên.
\end{baitoan}

\begin{baitoan}[\cite{An_400_BT_Hoa_Hoc_8_2020}, \textbf{228.}, p. 106]
	Nung nóng thủy ngân (II) oxit \emph{HgO} thì được thủy ngân \& oxi. Tính thể tích khí oxi thu được khi nung $54.25$\emph{g} \emph{HgO}.
\end{baitoan}

\begin{baitoan}[\cite{An_400_BT_Hoa_Hoc_8_2020}, \textbf{229.}, p. 106]
	Cần điều chế $2.24$\emph{l} khí oxi (đktc) trong phòng thí nghiệm bằng cách nhiệt phân 1 số chất. Chọn dùng 1 chất trong các chất sau đây có khối lượng nhỏ nhất \& khối lượng đó là bao nhiêu gam? (a) \emph{\ce{KClO3}}; (b) \emph{\ce{KMnO4}}; (c) \emph{HgO}.
\end{baitoan}

\begin{baitoan}[\cite{An_400_BT_Hoa_Hoc_8_2020}, \textbf{230.}, p. 106]
	Đốt cháy hoàn toàn $0.5$\emph{kg} than chứa $90$\% \emph{C} \& $10$\% tạp chất không cháy. Tính thể tích không khí cần dùng, biết $V_{\rm kk} = 5V_{\rm O_2}$.
\end{baitoan}

\begin{baitoan}[\cite{An_400_BT_Hoa_Hoc_8_2020}, \textbf{231.}, p. 106]
	Để điều chế oxi, người ta điện phân nước. Tính khối lượng nước cần dùng để điều chế $224{\rm m}^3$ \emph{\ce{O2}} (đktc).
\end{baitoan}

\begin{baitoan}[\cite{An_400_BT_Hoa_Hoc_8_2020}, \textbf{232.}, p. 106]
	Cho các oxit sau: \emph{\ce{CO2,SO2,P2O5,Al2O3,Fe3O4}}. (a) Chúng được tạo thành từ các đơn chất nào? (b) Viết phương trình phản ứng \& nêu điều kiện phản ứng (nếu có) điều chế các oxit trên.
\end{baitoan}

\begin{baitoan}[\cite{An_400_BT_Hoa_Hoc_8_2020}, \textbf{233.}, p. 106]
	Hỗn hợp \emph{\ce{C2H2}} \& \emph{\ce{O2}} với tỷ lệ nào về thể tích thì phản ứng cháy sẽ tạo ra nhiệt độ cao nhất? Ứng dụng phản ứng này để làm gì?
\end{baitoan}

\begin{baitoan}[\cite{An_400_BT_Hoa_Hoc_8_2020}, \textbf{234.}, p. 106]
	Oxit của 1 nguyên tố có hóa trị (II) chứa $20$\% oxi (về khối lượng). Xác định CTHH của oxit.
\end{baitoan}

\begin{baitoan}[\cite{An_400_BT_Hoa_Hoc_8_2020}, \textbf{235.}, p. 107]
	1 oxit của lưu huỳnh trong đó oxi chiếm $60$\% về khối lượng. Tìm CTPT của oxit đó.
\end{baitoan}

\begin{baitoan}[\cite{An_400_BT_Hoa_Hoc_8_2020}, \textbf{236.}, p. 107]
	Có 1 quặng sắt hàm lượng $50$\%. Khi phân tích 1 mẫu quặng này, người ta nhận thấy có $2.8$\emph{g} sắt. Trong mẫu quặng trên, tinhs khối lượng sắt(III) oxit \emph{\ce{Fe2O3}} ứng với hàm lượng sắt nói trên.
\end{baitoan}

\begin{baitoan}[\cite{An_400_BT_Hoa_Hoc_8_2020}, \textbf{237.}, p. 107]
	Tỷ lệ khối lượng của nitơ \& oxi trong 1 oxit là $7:20$. Xác định công thức oxit của nitơ.
\end{baitoan}

\begin{baitoan}[\cite{An_400_BT_Hoa_Hoc_8_2020}, \textbf{238.}, p. 107]
	Cho $28.4$\emph{g} điphotpho pentaoxit \emph{\ce{P2O5}} vào cốc chứa $90$\emph{g} \emph{\ce{H2O}} để tạo thành axit photphoric \emph{\ce{H3PO4}}. Tính khối lượng axit \emph{\ce{H3PO4}} tạo thành.
\end{baitoan}

\begin{baitoan}[\cite{An_400_BT_Hoa_Hoc_8_2020}, \textbf{239.}, p. 107]
	1 oxit của photpho có thành phần: \emph{P} chiếm $43.66$\%; \emph{O} chiếm $56.34$\%. Biết phân tử khối của oxit bằng $142$. Xác định công thức của oxit.
\end{baitoan}

\begin{baitoan}[\cite{An_400_BT_Hoa_Hoc_8_2020}, \textbf{240.}, p. 107]
	Trong giờ thực hành thí nghiệm, 1 em học sinh đốt cháy $3.2$\emph{g} bột lưu huỳnh trong $1.12$\emph{l} oxi (đktc). Vậy lưu huỳnh cháy hết hay còn dư?
\end{baitoan}

\begin{baitoan}[\cite{An_400_BT_Hoa_Hoc_8_2020}, \textbf{241.}, p. 107]
	Tính thể tích khí oxi \& thể tích không khí (đktc) cần thiết để đốt cháy: (a) $1$ \emph{mol} carbon; (b) $1.5$ \emph{mol} photpho.
\end{baitoan}

\begin{baitoan}[\cite{An_400_BT_Hoa_Hoc_8_2020}, \textbf{242.}, p. 107]
	Tính khối lượng của $\frac{N}{2}$ nguyên tử oxi, của $\frac{N}{4}$ phân tử oxi \& so sánh 2 kết quả này.
\end{baitoan}

\begin{baitoan}[\cite{An_400_BT_Hoa_Hoc_8_2020}, \textbf{243.}, p. 107]
	Cho 1 luồng không khí khô đi qua bột đồng (dư) nung nóng. Khí thu được sau phản ứng là khí gì?
\end{baitoan}

\begin{baitoan}[\cite{An_400_BT_Hoa_Hoc_8_2020}, \textbf{244.}, p. 107]
	Đốt cháy $1$\emph{kg} than trong khí oxi, biết trong than có $5$\% tạp chất không cháy. (a) Tính thể tích oxi (đktc) cần thiết đốt cháy $1$\emph{kg} than trên. (b) Tính thể tích khí carbonic (đktc) sinh ra trong phản ứng.
\end{baitoan}

\begin{baitoan}[\cite{An_400_BT_Hoa_Hoc_8_2020}, \textbf{245.}, p. 107]
	Đốt cháy lưu huỳnh trong bình chứa $4.8$\emph{l \ce{O2}}. Sau phản ứng người ta thu được $2.4$\emph{l} khí \emph{\ce{SO2}}. (a) Tính khối lượng lưu huỳnh đã cháy. Biết các khí ở điều kiện $20^\circ{\rm C}$, $1$\emph{atm}. (b) Tính khối lượng khí \emph{\ce{O2}} còn dư sau phản ứng. Cho biết $1$ \emph{mol} khí bất kỳ ở điều kiện bình thường ($20^\circ{\rm C}$, $1$\emph{atm}) chiếm thể tích là $24$\emph{l}.
\end{baitoan}

\begin{baitoan}[\cite{An_400_BT_Hoa_Hoc_8_2020}, \textbf{246.}, p. 108]
	Đốt cháy $6.2$\emph{g} photpho trong bình chứa $7.84$\emph{l} oxi (đktc). Cho biết sau khi cháy: (a) Photpho hay oxi, chất nào còn thừa \& khối lượng là bao nhiêu? (b) Chất nào được tạo thành \& khối lượng là bao nhiêu?
\end{baitoan}

\begin{baitoan}[\cite{An_400_BT_Hoa_Hoc_8_2020}, \textbf{247.}, p. 108]
	Xác định CTHH của 1 oxit của lưu huỳnh có khối lượng mol là $64$\emph{g} \& biết thành phần \% về khối lượng của nguyên tố lưu huỳnh trong oxit là $50$\%. CTHH của oxit là gì? {\sf A.} \emph{\ce{SO2}}. {\sf B.} \emph{\ce{SO3}}. {\sf C.} \emph{\ce{SO}}. {\sf D.} \emph{\ce{SO4}}.
\end{baitoan}

\begin{baitoan}[\cite{An_400_BT_Hoa_Hoc_8_2020}, \textbf{248.}, p. 108]
	1 oxit của photpho có thành phần \% của \emph{P} bằng $43.66$\%. Biết phân tử khối của oxit bằng $142$ \emph{đvC}. CTHH của oxit là: {\sf A.} \emph{\ce{P2O3}}. {\sf B.} \emph{\ce{P2O5}}. {\sf C.} \emph{\ce{PO2}}. {\sf D.} \emph{\ce{P2O4}}. {\sf A.} $0.84$\emph{g} \& $0.32$\emph{g}. {\sf B.} $2.52$\emph{g} \& $0.96$\emph{g}. {\sf C.} $1.68$\emph{g} \& $0.64$\emph{g}. {\sf D.} $0.95$\emph{g} \& $0.74$\emph{g}. (b) Số gam kali pemanganat \emph{\ce{KMnO4}} cần dùng để điều chế lượng khí oxi dùng cho phản ứng trên là: {\sf A.} $3.16$\emph{g}. {\sf B.} $9.48$\emph{g}. {\sf C.} $5.24$\emph{g}. {\sf D.} $6.32$\emph{g}.
\end{baitoan}

\begin{baitoan}[\cite{An_400_BT_Hoa_Hoc_8_2020}, \textbf{249.}, p. 108]
	Khi đốt cháy sắt trong oxi được oxit sắt từ \emph{\ce{Fe3O4}} ở nhiệt độ cao. (a) Số gam sắt \& khí oxi cần dùng để điều chế $2.32$\emph{g} oxit sắt từ lần lượt là: 
\end{baitoan}

\begin{baitoan}[\cite{An_400_BT_Hoa_Hoc_8_2020}, \textbf{250.}, p. 108]
	1 oxit được tạo bởi 2 nguyên tố sắt \& oxi trong đó tỷ lệ khối lượng giữa sắt \& oxi là $\frac{7}{3}$. Xác định CTHH của oxit sắt.
\end{baitoan}

\begin{baitoan}[\cite{An_400_BT_Hoa_Hoc_8_2020}, \textbf{251.}, p. 108]
	Tính khối lượng khí carbonic sinh ra trong mỗi trường hợp sau: (a) Khi đốt $0.3$ \emph{mol} carbon trong bình chứa $4.48$\emph{l} khí oxi (đktc). (b) Khi đốt $6$\emph{g} carbon trong bình chứa $13.44$\emph{l} khí oxi.
\end{baitoan}

\begin{baitoan}[\cite{An_400_BT_Hoa_Hoc_8_2020}, \textbf{252.}, p. 109]
	(a) Nêu những CTHH oxit phi kim không phải là oxit axit. Tại sao? (b) Nêu những kim loại ở trạng thái hóa trị cao cũng tạo ra oxit axit.
\end{baitoan}

\begin{baitoan}[\cite{An_400_BT_Hoa_Hoc_8_2020}, \textbf{253.}, p. 109]
	Nung $a$\emph{g} \emph{\ce{KClO3}} \& $b$\emph{g} \emph{\ce{KMnO4}} thu được cùng 1 lượng \emph{\ce{O2}}. Tính tỷ lệ $\frac{a}{b}$ là: {\sf A.} $\frac{7}{27}$. {\sf B.} $\frac{7}{26.5}$. {\sf C.} $\frac{7}{27.08}$. {\sf D.} $\frac{8}{28}$.
\end{baitoan}

%------------------------------------------------------------------------------%

\section{Miscellaneous}

\begin{baitoan}[\cite{An_400_BT_Hoa_Hoc_8_2020}, \textbf{254.}, p. 109]
	Để điều chế 1 lượng lớn oxi trong công nghiệp, người ta dùng những phương pháp nào \& bằng những nguyên liệu gì?
\end{baitoan}

\begin{baitoan}[\cite{An_400_BT_Hoa_Hoc_8_2020}, \textbf{255.}, p. 109]
	(a) Lấy cùng 1 lượng \emph{\ce{KClO3}} \& \emph{\ce{KMnO4}} để điều chế khí \emph{\ce{O2}}. Chất nào cho nhiều khí \emph{\ce{O2}} hơn? (b) Nếu điều chế cùng 1 thể tích khí oxi thì dùng chất nào kinh tế hơn? Biết giá \emph{\ce{KMnO4}} là $30000$ \emph{VND\texttt{/}kg} \& \emph{\ce{KClO3}} là $96000$ \emph{VND\texttt{/}kg}. Viết phương trình phản ứng \& giải thích.
\end{baitoan}

\begin{baitoan}[\cite{An_400_BT_Hoa_Hoc_8_2020}, \textbf{256.}, p. 109]
	1 bình nén chứa $3.2$\emph{kg} oxi. Nếu dùng cho đèn xì bằng khí axetilen thì có thể đốt cháy bao nhiêu $\rm m^3$ khí axetilen (đktc)?
\end{baitoan}

\begin{baitoan}[\cite{An_400_BT_Hoa_Hoc_8_2020}, \textbf{257.}, p. 109]
	Oxi hóa hoàn toàn $5.4$\emph{g Al}. (a) Tính thể tích oxi cần dùng. (b) Tính số gam \emph{\ce{KMnO4}} cần dùng để điều chế lượng oxi trên.
\end{baitoan}

\begin{baitoan}[\cite{An_400_BT_Hoa_Hoc_8_2020}, \textbf{258.}, p. 109]
	Người ta điều chế vôi sống \emph{CaO} bằng cách nung đá vôi \emph{\ce{CaCO3}}. Tính lượng vôi sống thu được từ $1$ tấn đá vôi có chứa $10$\% tạp chất. {\sf A.} $0.252$ tấn. {\sf B.} $0.378$ tấn. {\sf C.} $0.504$ tấn. {\sf D.} $0.606$ tấn.
\end{baitoan}

\begin{baitoan}[\cite{An_400_BT_Hoa_Hoc_8_2020}, \textbf{259.}, p. 109]
	Đốt quặng kèm sunfua \emph{ZnS} \& đốt quặng pirit sắt \emph{\ce{FeS2}} đều sinh ra sunfurơ \emph{\ce{SO2}} theo sơ đồ phản ứng: \emph{\ce{ZnS + O2 -> ZnO + SO2}}, \emph{\ce{FeS2 + O2 -> Fe2O3 + SO2}}. (a) Cân bằng phương trình. (b) Hỏi muốn điều chế $44.8\rm m^3$ khí \emph{\ce{SO2}} (đktc) cần bao nhiêu \emph{kg ZnS} hoặc cần bao nhiêu \emph{kg \ce{FeS2}}?
\end{baitoan}

\begin{baitoan}[\cite{An_400_BT_Hoa_Hoc_8_2020}, \textbf{260.}, p. 110]
	Tính thể tích hỗn hợp khí thu được khi đốt $28$\emph{g} hỗn hợp gồm carbon \& lưu huỳnh. Biết carbon chiếm $42.86$\% khối lượng hỗn hợp.
\end{baitoan}

\begin{baitoan}[\cite{An_400_BT_Hoa_Hoc_8_2020}, \textbf{261.}, p. 110]
	Đốt cháy hỗn hợp bột \emph{Mg} \& bột \emph{Al} cần $33.6$\emph{l} khí \emph{\ce{O2}} (đktc). Biết khối lượng \emph{Al} là $2.7$\emph{g}. Thành phần \% của 2 kim loại \emph{Al,Mg} trong hỗn hợp lần lượt là: {\sf A.} $3.8$\% \& $96.2$\%. {\sf B.} $7$\% \& $93$\%. {\sf C.} $6.5$\% \& $93.5$\%. {\sf D.} $60$\% \& $40$\%.
\end{baitoan}

\begin{baitoan}[\cite{An_400_BT_Hoa_Hoc_8_2020}, \textbf{262.}, p. 110]
	Đốt cháy photpho trong bình đựng $6.72$\emph{l \ce{O2}} (đktc) thu được $14.2$\emph{g} điphotpho pentaoxit. Khối lượng photpho cháy là: {\sf A.} $6$\emph{g}. {\sf B.} $6.1$\emph{g}. {\sf C.} $6.2$\emph{g}. {\sf D.} $7.5$\emph{g}.
\end{baitoan}

\begin{baitoan}[\cite{An_400_BT_Hoa_Hoc_8_2020}, \textbf{263.}, p. 110]
	Trong phòng thí nghiệm cần điều chế $5.6$\emph{l} khí \emph{\ce{O2}} (đktc). Hỏi phải dùng bao nhiêu gam \emph{\ce{KClO3}}? (Biết khí oxi thu được sau phản ứng bị hao hụt $10$\%). {\sf A.} $\approx22.46$\emph{g}. {\sf B.} $22$\emph{g}. {\sf C.} $\approx22.3$\emph{g}. {\sf D.} $30$\emph{g}.
\end{baitoan}

\begin{baitoan}[\cite{An_400_BT_Hoa_Hoc_8_2020}, \textbf{264.}, p. 110]
	Đốt cháy $125$\emph{g} quặng pirit sắt chứa $4$\% tạp chất trong oxi thì được sắt(III) oxit \& khí sunfurơ. Thể tích khí sunfurơ thu được là: {\sf A.} $44$\emph{l}. {\sf B.} $44.5$\emph{l}. {\sf C.} $44.8$\emph{l}. {\sf D.} $55.8$\emph{l}.
\end{baitoan}

\begin{baitoan}[\cite{An_400_BT_Hoa_Hoc_8_2020}, \textbf{265.}, p. 110]
	Cho biết những phản ứng sau đây thuộc loại phản ứng hóa hợp hay phản ứng phân hủy? (a) \emph{\ce{MgO + CO2 -> MgCO3}}. (b) \emph{\ce{Cu(NO3)2 ->[$t^\circ$] CuO + 2No2 + \frac{1}{2}O2}}. (c) \emph{\ce{2Al(OH)3 ->[$t^\circ$] Al2O3 + 3H2O}}. (d) \emph{\ce{4HNO3 ->[$t^\circ$] 4NO2 ^ + O2 ^ + 2H2O}}. (e) \emph{\ce{2Au + 3Cl2 -> 2AuCl3}}. (f) \emph{\ce{2NO + O2 -> 2NO2}}.
\end{baitoan}

\begin{baitoan}[\cite{An_400_BT_Hoa_Hoc_8_2020}, \textbf{266.}, pp. 110--111]
	Chỉ ra những phản ứng hóa học có xảy ra sự oxi hóa trong các phản ứng cho dưới đây \& chỉ ra những chất oxi hóa: (a) \emph{\ce{4NH3 + 5O2 ->[$t^\circ$][xt] 4NO + 6H2O}}. (b) \emph{\ce{4Cr + 3O2 ->[$t^\circ$] 2Cr2O3}}. (c) \emph{\ce{FeO + H2 ->[$t^\circ$] Fe + H2O}}. (d) \emph{\ce{Fe3O4 + CO ->[$t^\circ$] 3FeO + CO2}}. (e) \emph{\ce{Na2O + H2O -> 2NaOH}}.
\end{baitoan}

\begin{baitoan}[\cite{An_400_BT_Hoa_Hoc_8_2020}, \textbf{267.}, p. 111]
	1 bình kín có dung tích $1.4$\emph{l} chứa đầy không khí (đktc). Nếu đốt cháy $2.5$\emph{g} photpho \emph{P} trong bình, thì photpho có cháy hết không? Biết thể tích oxi chiếm $\frac{1}{5}$ thể tích không khí.
\end{baitoan}

\begin{baitoan}[\cite{An_400_BT_Hoa_Hoc_8_2020}, \textbf{268.}, p. 111]
	Muốn điều chế $5.04$\emph{l} khí oxi (đktc) cần phải dùng bao nhiêu gam kali clorat \emph{\ce{KClO3}}? {\sf A.} $18$\emph{g}. {\sf B.} $18.4$\emph{g}. {\sf C.} $18375$\emph{g}. {\sf D.} $20.3$\emph{g}.
\end{baitoan}

\begin{baitoan}[\cite{An_400_BT_Hoa_Hoc_8_2020}, \textbf{269.}, p. 111]
	Viết các phương trình phản ứng mà sản phẩm là: (a) Oxit kim loại. (b) Oxit phi kim. (c) Oxit kim loại \& oxit phi kim.
\end{baitoan}

\begin{baitoan}[\cite{An_400_BT_Hoa_Hoc_8_2020}, \textbf{270.}, p. 111]
	Nung $150$\emph{kg} đá vôi có lẫn $20$\% tạp chất ta được vôi sống \emph{CaO} \& khí carbonic. Khối lượng vôi sống thu được là: {\sf A.} $67$\emph{kg}. {\sf B.} $67.5$\emph{kg}. {\sf C.} $87.2$\emph{kg}. {\sf D.} $67.2$\emph{kg}.
\end{baitoan}

\begin{baitoan}[\cite{An_400_BT_Hoa_Hoc_8_2020}, \textbf{271.}, p. 111]
	Đốt cháy $1$ tạ than chứa $96$\% \emph{C}, còn lại là tạp chất không cháy. Hỏi cần bao nhiêu $\rm m^3$ không khí (đktc) để đốt cháy hết lượng than trên? (Biết $V_{\rm O_2} = \frac{1}{5}V_{\rm kk}$.) {\sf A.} $890\rm m^3$. {\sf B.} $896\rm m^3$. {\sf c.} $895\rm m^3$. {\sf D.} $900\rm m^3$. 
\end{baitoan}

\begin{baitoan}[\cite{An_400_BT_Hoa_Hoc_8_2020}, \textbf{272.}, p. 111]
	1 mẫu quặng chứa $82$\% \emph{\ce{Fe2O3}}. Phần trăm khối lượng của sắt trong quặng là: {\sf A.} $57.4$\%. {\sf B.} $57$\%. {\sf C.} $54.7$\%. {\sf D.} $56.4$\%.
\end{baitoan}

\begin{baitoan}[\cite{An_400_BT_Hoa_Hoc_8_2020}, \textbf{273.}, p. 111]
	Thành phần \% khối lượng của oxi trong oxit của 1 nguyên tố hóa trị III bằng $47$\%. Nguyên tố tạo oxit là: {\sf A.} Photpho. {\sf B.} Canxi. {\sf C.} Nhôm. {\sf D.} Crom.
\end{baitoan}

\begin{baitoan}[\cite{An_400_BT_Hoa_Hoc_8_2020}, \textbf{274.}, p. 111]
	1 quặng sắt chứa $90$\% \emph{\ce{Fe3O4}} còn lại là tạp chất. (a) Nếu dùng khí hydro để khử $0.5$ tấn quặng thì khối lượng sắt thu được bao nhiêu? {\sf A.} $0.32586$ tấn. {\sf B.} $0.32$ tấn. {\sf C.} $0.22$ tấn. {\sf D.} $0.45$ tấn. (b) Khối lượng khí hydro cần dùng là:  {\sf A.} $0.016$ tấn. {\sf B.} $0.0155$ tấn. {\sf C.} $0.0165$ tấn. {\sf D.} $0.0255$ tấn.
\end{baitoan}

\begin{baitoan}[\cite{An_400_BT_Hoa_Hoc_8_2020}, \textbf{275.}, p. 111]
	Kẽm oxit được điều chế bằng cách đốt bột \emph{Zn} ngoài không khí. (a) Muốn điều chế $20.25$\emph{g} kẽm oxit thì bột \emph{Zn} cần dùng là: {\sf A.} $16.2$\emph{g}. {\sf B.} $16.25$\emph{g}. {\sf C.} $17$\emph{g}. {\sf D.} $16.3$\emph{g}. (b) Nếu hiệu suất phản ứng là $96$\% thì bột \emph{Zn} phải dùng là bao nhiêu? {\sf A.} $16.925$\emph{g}. {\sf B.} $16.9$\emph{g}. {\sf C.} $17$\emph{g}. {\sf D.} $20.9$\emph{g}.
\end{baitoan}

\begin{baitoan}[\cite{An_400_BT_Hoa_Hoc_8_2020}, \textbf{276.}, p. 111]
	Đốt cháy $100$\emph{g} hỗn hợp bột lưu huỳnh \emph{S} \& sắt \emph{Fe} dùng hết $33.6$\emph{l} khí oxi (đktc). Tính khối lượng mỗi chất có trong hỗn hợp. Biết \emph{Fe} tác dụng với oxi ở nhiệt độ cao cho \emph{\ce{Fe3O4}}.
\end{baitoan}

\begin{baitoan}[\cite{An_400_BT_Hoa_Hoc_8_2020}, \textbf{277.}, p. 111]
	Để có $5.6$\emph{l} khí \emph{\ce{O2}} (đktc) để làm thí nghiệm. Cần phải lấy khối lượng kali pemanganat \emph{\ce{KMnO4}} bao nhiêu? Biết hiệu suất phản ứng là $96$\%. {\sf A.} $80$\emph{g}. {\sf B.} $80.5$\emph{g}. {\sf C.} $80.6$\emph{g}. {\sf D.} $90$\emph{g}.
\end{baitoan}

\begin{baitoan}[\cite{An_400_BT_Hoa_Hoc_8_2020}, \textbf{278.}, p. 111]
	Đốt cháy $9.84$\emph{g} hỗn hợp gồm carbon \& lưu huỳnh trong đó carbon chiếm $2.44$\% về khối lượng. Tính thành phần \% về thể tích các khí sinh ra.
\end{baitoan}

%------------------------------------------------------------------------------%

\printbibliography[heading=bibintoc]
	
\end{document}