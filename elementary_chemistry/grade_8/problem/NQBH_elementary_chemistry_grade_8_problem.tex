\documentclass{article}
\usepackage[backend=biber,natbib=true,style=authoryear]{biblatex}
\addbibresource{/home/hong/1_NQBH/reference/bib.bib}
\usepackage[utf8]{vietnam}
\usepackage{tocloft}
\renewcommand{\cftsecleader}{\cftdotfill{\cftdotsep}}
\usepackage[colorlinks=true,linkcolor=blue,urlcolor=red,citecolor=magenta]{hyperref}
\usepackage{amsmath,amssymb,amsthm,mathtools,float,graphicx,algpseudocode,algorithm,tcolorbox}
\usepackage[inline]{enumitem}
\usepackage[version=4]{mhchem}
\allowdisplaybreaks
\numberwithin{equation}{section}
\newtheorem{assumption}{Assumption}[section]
\newtheorem{conjecture}{Conjecture}[section]
\newtheorem{corollary}{Corollary}[section]
\newtheorem{hequa}{Hệ quả}[section]
\newtheorem{definition}{Definition}[section]
\newtheorem{dinhnghia}{Định nghĩa}[section]
\newtheorem{example}{Example}[section]
\newtheorem{vidu}{Ví dụ}[section]
\newtheorem{lemma}{Lemma}[section]
\newtheorem{notation}{Notation}[section]
\newtheorem{principle}{Principle}[section]
\newtheorem{problem}{Problem}[section]
\newtheorem{baitoan}{Bài toán}[section]
\newtheorem{proposition}{Proposition}[section]
\newtheorem{question}{Question}[section]
\newtheorem{cauhoi}{Câu hỏi}[section]
\newtheorem{remark}{Remark}[section]
\newtheorem{luuy}{Lưu ý}[section]
\newtheorem{theorem}{Theorem}[section]
\newtheorem{dinhly}{Định lý}[section]
\usepackage[left=0.5in,right=0.5in,top=1.5cm,bottom=1.5cm]{geometry}
\usepackage{fancyhdr}
\pagestyle{fancy}
\fancyhf{}
\lhead{\small Sect.~\thesection}
\rhead{\small \nouppercase{\leftmark}}
\renewcommand{\sectionmark}[1]{\markboth{#1}{}}
\cfoot{\thepage}
\def\labelitemii{$\circ$}

\title{Problems in Elementary Chemistry\texttt{/}Grade 8}
\author{Nguyễn Quản Bá Hồng\footnote{Independent Researcher, Ben Tre City, Vietnam\\e-mail: \texttt{nguyenquanbahong@gmail.com}; website: \url{https://nqbh.github.io}.}}
\date{\today}

\begin{document}
\maketitle
\begin{abstract}
	1 bộ sưu tập các bài toán chọn lọc từ cơ bản đến nâng cao cho Hóa sơ cấp lớp 8. Tài liệu này là phần bài tập bổ sung cho tài liệu chính \href{https://github.com/NQBH/hobby/blob/master/elementary_chemistry/grade_8/NQBH_elementary_chemistry_grade_8.pdf}{GitHub\texttt{/}NQBH\texttt{/}hobby\texttt{/}elementary chemistry\texttt{/}grade 8\texttt{/}lecture}\footnote{\textsc{url}: \url{https://github.com/NQBH/hobby/blob/master/elementary_chemistry/grade_8/NQBH_elementary_chemistry_grade_8.pdf}.} của tác giả viết cho Hóa Học lớp 8. Phiên bản mới nhất của tài liệu này được lưu trữ ở link sau: \href{https://github.com/NQBH/hobby/blob/master/elementary_chemistry/grade_8/problem/NQBH_elementary_chemistry_grade_8_problem.pdf}{GitHub\texttt{/}NQBH\texttt{/}hobby\texttt{/}elementary chemistry\texttt{/}grade 8\texttt{/}problem}\footnote{\textsc{url}: \url{https://github.com/NQBH/hobby/blob/master/elementary_chemistry/grade_8/problem/NQBH_elementary_chemistry_grade_8_problem.pdf}.}.
\end{abstract}
\tableofcontents
\newpage

%------------------------------------------------------------------------------%

\section{Chất -- Nguyên Tử -- Phân Tử}

\subsection{Nguyên Tử -- Nguyên Tố Hóa Học}

\begin{baitoan}[\cite{An2011}, \textbf{1.}, p. 5]
	Nguyên tử oxi có $8$ proton trong hạt nhân. Cho biết thành phần hạt nhân của $3$ nguyên tử X, Y, Z theo bảng sau: hạt nhân nguyên tử X có $8$ proton, $8$ neutron; hạt nhân nguyên tử Y có $8$ proton, $9$ neutron; nguyên tử Z có $8$ proton, $10$ neutron. Những nguyên tử này thuộc cùng nguyên tố nào? Vì sao?
\end{baitoan}

\begin{baitoan}[\cite{An2011}, \textbf{2.}, p. 5]
	Hạt nhân nguyên tử \emph{Na} có $11$ proton, natri có nguyên tử khối là $23$. Sắt có nguyên tử khối bằng $56$, trong hạt nhân có $30$ neutron. Xác định tổng số hạt (proton, neutron, \& electron) tạo thành nguyên tử natri \& sắt.
\end{baitoan}

\begin{baitoan}[\cite{An2011}, \textbf{3.}, p. 5]
	Nguyên tử nhôm có số proton là $13$. Trong nguyên tử nhôm, số hạt mang điện nhiều hơn số hạt không mang điện là $12$ hạt. Xác định nguyên tử khối của nhôm.
\end{baitoan}

\begin{baitoan}[\cite{An2011}, \textbf{4.}, p. 6]
	1 nguyên tử R có tổng số hạt mang điện \& không mang điện là $34$. Trong số đó hạt mang điện gấp $1.8(3)$ lần số hạt không mang điện. Xác định nguyên tử khối của R.
\end{baitoan}

\begin{baitoan}[\cite{An2011}, \textbf{5.}, p. 6]
	Tổng số hạt proton, neutron, electron trong $2$ nguyên tử A \& B là $142$, trong đó số hạt mang điện nhiều hơn số hạt không mang điện là $42$. Số hạt mang điện của nguyên tử B nhiều hơn của nguyên tử A là $12$. Xác định số proton của A \& B.
\end{baitoan}

\begin{baitoan}[\cite{An2011}, \textbf{6.}, p. 6]
	1 nguyên tố gồm 2 đồng vị có số nguyên tử tỷ lệ với nhau là $27:23$. Hạt nhân của đồng vị thứ nhất chứa $35$ proton \& $44$ neutron. Hạt nhân của đồng vị thứ 2 chứa nhiều hơn $2$ neutron. Xác định nguyên tử khối trung bình của nguyên tố trên.
\end{baitoan}

%------------------------------------------------------------------------------%

\subsection{Đơn Chất \& Hợp Chất -- Phân Tử}

\begin{baitoan}[\cite{An2011}, \textbf{7.}, p. 7]
	Kết quả phân tích cho thấy trong phân tử khí \ce{CO2} có $27.3$\% \& $72.7$\% \ce{O} theo khối lượng. Biết nguyên tử khối của \ce{C} là $12.011$. Xác định nguyên tử khối của oxi.
\end{baitoan}

\begin{baitoan}[\cite{An2011}, \textbf{8.}, p. 8]
	Trong nguyên tử \ce{X} có số hạt mang điện nhiều hơn số hạt không mang điện là $14$. Hợp chất \ce{XY_n} có đặc điểm: \ce{X} chiếm $15.0468$\% về khối lượng, tổng số proton là $100$, tổng số neutron là $106$. Xác định số proton \& số neutron của \ce{X}.
\end{baitoan}

\begin{baitoan}[\cite{An2011}, \textbf{9.}, p. 8]
	Oxi có $3$ đồng vị: \ce{_8^16O,_8^17O,_8^18O}, còn carbon có $2$ đồng vị bền là \ce{_6^12C,_6^13C}. Hỏi có bao nhiêu loại phân tử khí carbonic.
\end{baitoan}

\begin{baitoan}[\cite{An2011}, \textbf{10.}, p. 8]
	Hợp chất \ce{Ba(NO3)y} có phân tử khối là $261$, \ce{Ba} có nguyên tử khối là $137$ \& hóa trị II. Xác định hóa trị của nhóm \ce{NO3}.
\end{baitoan}

\begin{baitoan}[\cite{An2011}, \textbf{11.}, p. 9]
	1 hợp chất, phân tử gồm $2$ nguyên tử của nguyên tố Y liên kết với 1 nguyên tử oxi \& nặng hơn phân tử hydro $31$ lần. Xác định nguyên tử khối của Y.
\end{baitoan}

\begin{baitoan}[\cite{An2011}, \textbf{12.}, p. 9]
	Cho biết công thức hóa học của nguyên tố X với nhóm \ce{(SO4)} hóa trị II \& hợp chất của nhóm nguyên tử Y với \ce{H} như sau: \ce{X2(SO4)3,H2Y}. Xác định công thức hóa học hợp chất của X \& Y.
\end{baitoan}

\begin{baitoan}[\cite{An2011}, \textbf{13.}, p. 9]
	2 nguyên tử X kết hợp với 1 nguyên tử O tạo ra phân tử oxi. Trong phân tử, nguyên tố oxi chiếm $25.8$\% về khối lượng. Xác định nguyên tử khối của X.
\end{baitoan}

%------------------------------------------------------------------------------%

\subsection{Công Thức Hóa Học -- Hóa Trị}

\begin{baitoan}[\cite{An2011}, \textbf{14.}, p. 9]
	Nguyên tử khối của sắt là $56$, của oxi là $16$. Phân tử khối của oxit sắt này là $160$. Xác định công thức phân tử oxit sắt.
\end{baitoan}

\begin{baitoan}[\cite{An2011}, \textbf{15.}, p. 10]
	Có hợp chất \ce{MX2} với đặc điểm như sau:
	\begin{enumerate*}
		\item[$\bullet$] Tổng số hạt proton, neutron, electron là $140$ trong đó số hạt không mang điện kém hơn số hạt mang điện là $44$.
		\item[$\bullet$] Nguyên tử khối của M nhỏ hơn nguyên tử khối của X là $11$.
		\item[$\bullet$] Tổng số hạt trong ion \ce{X-} nhiều hơn trong \ce{M^2+} là $19$.
	\end{enumerate*}
	Xác định công thức phân tử \ce{MX2}.
\end{baitoan}

\begin{baitoan}[\cite{An2011}, \textbf{16.}, p. 10]
	Muối crom sunfat có phân tử khối là $392$ \& có công thức \ce{Cr2(SO4)x}. Tìm hóa trị của crom. Cho biết hóa trị của nhóm \ce{SO4} là II.
\end{baitoan}

\begin{baitoan}[\cite{An2011}, \textbf{17.}, p. 10]
	1 hợp chất X gồm $2$ nguyên tố \ce{C} \& \ce{S}, có phân tử khối nặng hơn khí oxi $2.375$ lần. Xác định công thức phân tử của \ce{X}.
\end{baitoan}

\begin{baitoan}[\cite{An2011}, \textbf{18.}, p. 11]
	1 oxit sắt có thành phần gồm $10.5$ phần khối lượng sắt \& $4$ phần khối lượng oxi. Xác định công thức hóa học của oxit sắt.
\end{baitoan}

\begin{baitoan}[\cite{An2011}, \textbf{19.}, p. 11]
	Lưu huỳnh có nguyên tử khối bằng $32$. Trong nguyên tử lưu huỳnh số hạt mang điện gấp đôi số hạt không mang điện. Tính tổng số hạt (proton, neutron, \& electron) trong nguyên tử lưu huỳnh.
\end{baitoan}

\begin{baitoan}[\cite{An2011}, \textbf{20.}, p. 11]
	Biết rằng khối lượng 1 nguyên tử oxi nặng gấp $15.842$ lần \& khối lượng của nguyên tử carbon nặng gấp $11.906$ lần khối lượng nguyên tử hydro. Hỏi nếu chọn $\frac{1}{12}$ khối lượng nguyên tử carbon làm đơn vị thì \ce{H,O} có nguyên tử khối là bao nhiêu?
\end{baitoan}

\begin{baitoan}[\cite{An2011}, \textbf{21.}, p. 11]
	Nguyên tử X có tổng số hạt là $95$. Biết số hạt không mang điện bằng $0.5833$ số hạt mang điện. Xác định số proton của nguyên tử X.
\end{baitoan}

\begin{baitoan}[\cite{An2011}, \textbf{22.}, p. 11]
	Cho biết tổng số hạt trong $1$ nguyên tử của nguyên tố X là $58$. Số hạt trong nhân lớn hơn số hạt ở vỏ là $20$ hạt. Xác định nguyên tử khối của X.
\end{baitoan}

\begin{baitoan}[\cite{An2011}, \textbf{23.}, p. 11]
	Tổng số proton, neutron, electron trong nguyên tử của 1 nguyên tố là $34$. Xác định số proton của nguyên tử đó.
\end{baitoan}

\begin{baitoan}[\cite{An2011}, \textbf{24.}, p. 11]
	Nguyên tử của nguyên tố hóa học X có tổng số hạt proton, neutron, electron bằng $180$, trong đó tổng số các hạt mang điện gấp $1.432$ lần số hạt neutron. Xác định số proton của nguyên tử X.
\end{baitoan}

\begin{baitoan}[\cite{An2011}, \textbf{25.}, p. 11]
	Tính thành phần \% các đồng vị của carbon. Biết rằng carbon ở trạng thái tự nhiên có $2$ đồng vị \ce{_6^12C} \& \ce{_6^13C} có nguyên tử khối là $12.011$.
\end{baitoan}

\begin{baitoan}[\cite{An2011}, \textbf{26.}, p. 11]
	Đồng có 2 đồng vị \ce{_29^65Cu,_29^63Cu}. Nguyên tử khối trung bình của đồng là $63.54$. Tính thành phần \% của mỗi đồng vị.
\end{baitoan}

\begin{baitoan}[\cite{An2011}, \textbf{27.}, p. 11]
	1 hợp chất có công thức phân tử là \ce{Na_x(SO4)y} \& có phân tử khối là $142$. Xác định công thức phân tử của hợp chất.
\end{baitoan}

\begin{baitoan}[\cite{An2011}, \textbf{28.}, p. 12]
	Oxit của 1 nguyên tố M hóa trị V trong đó nguyên tố X chiếm $43.67$\% về khối lượng. Xác định công thức phân tử của oxit.
\end{baitoan}

\begin{baitoan}[\cite{An2011}, \textbf{29.}, p. 12]
	Các hạt cấu tạo nên hạt nhân của hầu hết các nguyên tử là:
	\begin{enumerate*}
		\item[{\rm\sf A.}] proton \& electron;
		\item[{\rm\sf B.}] neutron \& electron;
		\item[{\rm\sf C.}] proton \& neutron;
		\item[{\rm\sf D.}] neutron, proton, \& electron.
	\end{enumerate*}
\end{baitoan}

\begin{baitoan}[\cite{An2011}, \textbf{30.}, p. 12]
	Các hạt cấu tọa nên hầu hết các nguyên tử là:
	\begin{enumerate*}
		\item[{\rm\sf A.}] neutron \& electron;
		\item[{\rm\sf B.}] neutron \& proton;
		\item[{\rm\sf C.}] proton \& electron;
		\item[{\rm\sf D.}] neutron, proton, \& electron.
	\end{enumerate*}
\end{baitoan}

\begin{baitoan}[\cite{An2011}, \textbf{31.}, p. 12]
	Nguyên tố hóa học là những nguyên tử có cùng:
	\begin{enumerate*}
		\item[{\rm\sf A.}] số proton \& neutron;
		\item[{\rm\sf B.}] số neutron;
		\item[{\rm\sf C.}] số proton;
		\item[{\rm\sf D.}] số electron.
	\end{enumerate*}
\end{baitoan}

\begin{baitoan}[\cite{An2011}, \textbf{32.}, p. 12]
	Công thức hóa học của nguyên tố R với hydro là \ce{H2R} \& M với oxit là \ce{M2O3}. Nếu R \& M kết hợp với nhau thì có công thức hóa học là:
	\begin{enumerate*}
		\item[{\rm\sf A.}] \ce{M2R};
		\item[{\rm\sf B.}] \ce{M3R2};
		\item[{\rm\sf C.}] \ce{M2R3};
		\item[{\rm\sf D.}] \ce{MR}.
	\end{enumerate*}
\end{baitoan}

\begin{baitoan}[\cite{An2011}, \textbf{33.}, p. 12]
	1 nguyên tử có $18$ electron. Số lớp electron của nguyên tử đó là:
	\begin{enumerate*}
		\item[{\rm\sf A.}] $3$;
		\item[{\rm\sf B.}] $4$;
		\item[{\rm\sf C.}] $2$;
		\item[{\rm\sf D.}] $5$.
	\end{enumerate*}
\end{baitoan}

\begin{baitoan}[\cite{An2011}, \textbf{34.}, p. 12]
	Hạt nhân nguyên tử Y có $7$ proton. Số electron lớp ngoài cùng của nguyên tử Y là:
	\begin{enumerate*}
		\item[{\rm\sf A.}] $4$;
		\item[{\rm\sf B.}] $5$;
		\item[{\rm\sf C.}] $3$;
		\item[{\rm\sf D.}] $7$.
	\end{enumerate*}
\end{baitoan}

\begin{baitoan}[\cite{An2011}, \textbf{35.}, p. 12]
	Phát biểu đúng là:
	\begin{enumerate*}
		\item[{\rm\sf A.}] Nguyên tố hóa học tồn tại ở dạng hóa hợp.
		\item[{\rm\sf B.}] Nguyên tố hóa học tồn tại ở dạng tự do.
		\item[{\rm\sf C.}] Nguyên tố hóa học có thể tồn tại ở dạng tự do \& phần lớn ở dạng hóa hợp.
		\item[{\rm\sf D.}] Số nguyên tố hóa học có nhiều hơn chất.
	\end{enumerate*}
\end{baitoan}

\begin{baitoan}[\cite{An2011}, \textbf{36.}, p. 12]
	Biết nguyên tố X có nguyên tử khối bằng $\frac{5}{2}$ nguyên tử khối của oxi. X có nguyên tử khối là:
	\begin{enumerate*}
		\item[{\rm\sf A.}] $20$;
		\item[{\rm\sf B.}] $40$;
		\item[{\rm\sf C.}] $30$;
		\item[{\rm\sf D.}] $50$.
	\end{enumerate*}
\end{baitoan}

\begin{baitoan}[\cite{An2011}, \textbf{37.}, p. 12]
	Nguyên tử \ce{C} có khối lượng bằng $1.996\cdot 10^{-23}$g. Khối lượng tính bằng gam của nguyên tử \ce{Na} là:
	\begin{enumerate*}
		\item[{\rm\sf A.}] $\approx3.82\cdot10^{-23}$g;
		\item[{\rm\sf B.}] $\approx3.28\cdot10^{-23}$g;
		\item[{\rm\sf C.}] $1.91\cdot10^{-23}$g;
		\item[{\rm\sf D.}] $4.15\cdot10^{-23}$g.
	\end{enumerate*}
\end{baitoan}

\begin{baitoan}[\cite{An2011}, \textbf{38.}, p. 12]
	2 nguyên tử X kết hợp với $3$ nguyên tử oxi tạo ra phân tử oxit. Trong phân tử, oxi chiếm $30$\% về khối lượng. Nguyên tử khối của X là:
	\begin{enumerate*}
		\item[{\rm\sf A.}] $23$;
		\item[{\rm\sf B.}] $56$;
		\item[{\rm\sf C.}] $52$;
		\item[{\rm\sf D.}] $55$.
	\end{enumerate*}
\end{baitoan}

\begin{baitoan}[\cite{An2011}, \textbf{39.}, p. 13]
	Hợp chất X gồm 2 nguyên tố C, H \& có phân tử khối nặng hơn khí \ce{H2} $15$ lần. Công thức hóa học của X là:
	\begin{enumerate*}
		\item[{\rm\sf A.}] \ce{CH4};
		\item[{\rm\sf B.}] \ce{C2H4};
		\item[{\rm\sf C.}] \ce{C2H6};
		\item[{\rm\sf D.}] \ce{C2H2}.
	\end{enumerate*}
\end{baitoan}

\begin{baitoan}[\cite{An2011}, \textbf{40.}, p. 13]
	Hạt nhân của 1 nguyên tử có số proton bằng số neutron \& có nguyên tử khối bằng $12$. Số electron lớp ngoài cùng của nguyên tử đó là:
	\begin{enumerate*}
		\item[{\rm\sf A.}] $4$;
		\item[{\rm\sf B.}] $3$;
		\item[{\rm\sf C.}] $2$;
		\item[{\rm\sf D.}] $5$.
	\end{enumerate*}
\end{baitoan}

\begin{baitoan}[\cite{An2011}, \textbf{41.}, p. 13]
	Biết \ce{S} hóa trị IV, chọn công thức hóa học nào phù hợp với quy tắc hóa trị trong số các công thức sau đây:
	\begin{enumerate*}
		\item[{\rm\sf A.}] \ce{SO3};
		\item[{\rm\sf B.}] \ce{SO2};
		\item[{\rm\sf C.}] \ce{S2O3};
		\item[{\rm\sf D.}] \ce{S2O}.
	\end{enumerate*}
\end{baitoan}

\begin{baitoan}[\cite{An2011}, \textbf{42.}, p. 13]
	1 hợp chất phân tử gồm 1 nguyên tử X liên kết với $3$ nguyên tử \ce{O}. Nguyên tố oxi chiếm $60$\% về khối lượng của hợp chất.
	\begin{enumerate*}
		\item[(a)] Tìm nguyên tử khối của X.
		\item[(b)] Phân tử nặng bằng oxit của kim loại nào?
	\end{enumerate*}
\end{baitoan}

\begin{baitoan}[\cite{An2011}, \textbf{43.}, p. 13]
	Giải thích vì sao các nguyên tử liên kết được với nhau? Khả năng liên kết của nguyên tử phụ thuộc vào yếu tố nào?
\end{baitoan}

\begin{baitoan}[\cite{An2011}, \textbf{44.}, p. 13]
	Trong phản ứng hóa học, nguyên tử hay phân tử được bảo toàn? Tại sao có sự biến đổi phân tử này thành phân tử khác?
\end{baitoan}

\begin{baitoan}[\cite{An2011}, \textbf{45.}, p. 13]
	Giải thích vì sao các nguyên tử liên kết được với nhau? Khả năng liên kết của nguyên tử phụ thuộc vào yếu tố nào?
\end{baitoan}

\begin{baitoan}[\cite{An2011}, \textbf{46.}, p. 13]
	1 oxit có công thức phân tử \ce{Mn2Ox}, có phân tử khối là $222$. Xác định hóa trị của \ce{Mn}.
\end{baitoan}

\begin{baitoan}[\cite{An2011}, \textbf{47.}, p. 13]
	\begin{enumerate*}
		\item[(a)] Tính số nguyên tử nitơ có trong $14$g nitơ.
		\item[(b)] Trong $16$g oxi có số nguyên tử oxi bằng hay lớn hơn số nguyên tử nitơ trên.
	\end{enumerate*}
\end{baitoan}

\begin{baitoan}[\cite{An2011}, \textbf{48.}, p. 13]
	Nếu phần trăm của kim loại X trong muối carbonat là $40$\% thì phần trăm khối lượng của kim loại X trong muối photphat là bao nhiêu?
\end{baitoan}

\begin{baitoan}[\cite{An2011}, \textbf{49.}, p. 13]
	Trong vỏ Trái Đất có $2.5$\% kali \& $3.4$\% canxi (về khối lượng). Hỏi vai trò nguyên tố kali \& canxi, nguyên tố nào có nhiều nguyên tử hơn trong vỏ Trái Đất?
\end{baitoan}

\begin{baitoan}[\cite{An2011}, \textbf{50.}, p. 13]
	Khi phân tích thủy ngân oxit người ta thấy cứ $108$ phần khối lượng oxit thì có $100$ phần khối lượng thủy ngân. Tính hóa trị của thủy ngân trong hợp chất này.
\end{baitoan}

\begin{baitoan}[\cite{An2011}, \textbf{51.}, p. 13]
	Thành phần \% về khối lượng của kali \& natri trong vỏ Trái Đất gần bằng nhau. Cho biết nguyên tố nào chứa số nguyên tử nhiều hơn \& nhiều hơn bao nhiêu lần?
\end{baitoan}

\begin{baitoan}[\cite{An2011}, \textbf{52.}, p. 14]
	1 hợp chất phân tử gồm 1 nguyên tử của nguyên tố X liên kết với 2 nguyên tử oxi. Nguyên tố oxi chiếm $50$\% về khối lượng trong hợp chất. Xác định nguyên tố X.
\end{baitoan}

\begin{baitoan}[\cite{An2011}, \textbf{53.}, p. 14]
	Xác định hóa trị của nguyên tố clo trong các hợp chất sau: \ce{HCl,Cl2O3,KClO3,HClO3}, \ce{Cl2O7,Cl2O}.
\end{baitoan}

\begin{baitoan}[\cite{An2011}, \textbf{54.}, p. 14]
	Cho các công thức \ce{CaO,NaO,Ca(HCO3)2,Na(HCO3)2,Al2O3,Fe3O4,FeO,Mg2O}, \ce{Cu2O,CuO,Hg2O,Ag2O,AgOH,Zn(OH)2,Fe2O2,Cr2O,NaCl2}. Xác định công thức hóa học viết đúng.
\end{baitoan}

\begin{baitoan}[\cite{An2011}, \textbf{55.}, p. 14]
	Xác định hóa trị của các nguyên tố trong các hợp chất sau:
	\begin{enumerate*}
		\item[(a)] \ce{NH3,NO,N2O,NO2,N2O5};
		\item[(b)] \ce{H2S,SO2,SO3,Al2S3};
		\item[(c)] \ce{CO,CO2};
		\item[(d)] \ce{P2O5,PH3,P2O3,PCl3,Ca3P2,Zn3P2} (thuốc chuột).
	\end{enumerate*}
\end{baitoan}

\begin{baitoan}[\cite{An2011}, \textbf{56.}, p. 14]
	Có 1 can nhựa đựng dầu hỏa có lẫn nước, làm cách nào để lấy được dầu hỏa?
\end{baitoan}

\begin{baitoan}[\cite{An2011}, \textbf{57.}, p. 14]
	So sánh xem nguyên tử lưu huỳnh nặng hay nhẹ bao nhiêu lần so với nguyên tử oxi, nguyên tử hydro, \& nguyên tử carbon.
\end{baitoan}

\begin{baitoan}[\cite{An2011}, \textbf{58.}, p. 14]
	Có 1 hỗn hợp gồm $2$ khí là khí oxi \& khí \ce{CO2}, bằng cách nào có  thể tách được khí oxi?
\end{baitoan}

\begin{baitoan}[\cite{An2011}, \textbf{59.}, p. 14]
	Để tách chất có những phương pháp phổ biến sau: bay hơi, chưng cất, lọc. Chọn phương pháp phù hợp để:
	\begin{enumerate*}
		\item[(a)] Tách bụi có trong không khí;
		\item[(b)] Tách rượu nguyên chất từ rượu loãng;
		\item[(c)] Tách nước cất từ nước thường.
	\end{enumerate*}
\end{baitoan}

\begin{baitoan}[\cite{An2011}, \textbf{60.}, p. 14]
	Oxit của 1 nguyên tố hóa trị V chứa $43.67$\% nguyên tố đó. Xác định nguyên tử khối của nguyên tố đó.
\end{baitoan}

\begin{baitoan}[\cite{An2011}, \textbf{61.}, p. 14]
	Tỷ lệ khối lượng của O \& H trong phân tử nước là $\frac{8}{1}$. Trong phân tử nước có $2$ nguyên tử hydro. Xác định số nguyên tử oxi.
\end{baitoan}

\begin{baitoan}[\cite{An2011}, \textbf{62.}, p. 14]
	Oxit của 1 nguyên tố hóa trị III chứa $17.29$\% oxi. Tìm nguyên tử khối của nguyên tố đó.
\end{baitoan}

\begin{baitoan}[\cite{An2011}, \textbf{63.}, p. 14]
	Dựa vào hóa trị của nhóm \ce{(PO4)} trong acid photphonic \ce{H3PO4}, xác định hóa trị của \ce{Al} trong \ce{AlPO4}; của \ce{Fe} trong \ce{Fe3(PO4)2}.
\end{baitoan}

\begin{baitoan}[\cite{An2011}, \textbf{64.}, p. 14]
	2 nguyên tử X kết hợp với 1 nguyên tử \ce{O} tạo ra phân tử oxit. Trong phân tử oxit, nguyên tử oxi chiếm $25.8$\% về khối lượng. Xác định nguyên tử khối của X.
\end{baitoan}

\begin{baitoan}[\cite{An2011}, \textbf{65.}, p. 15]
	Oxit của kim loại ở mức hóa trị thấp chứa $22.56$\% oxi còn oxit của kim loại đó ở mức hóa trị cao chứa $50.48$\%. Xác định nguyên tử khối của kim loại.
\end{baitoan}

\begin{baitoan}[\cite{An2011}, \textbf{66.}, p. 15]
	1 nguyên tử M kết hợp với $3$ nguyên tử \ce{H} tạo thành hợp chất với hydro. Trong phân tử, khối lượng \ce{H} chiếm $17.65$\%. Xác định nguyên tố M.
\end{baitoan}

\begin{baitoan}[\cite{An2011}, \textbf{67.}, p. 15]
	1 hợp chất có thành phần gồm $2$ nguyên tố là \ce{C} \& \ce{O}. Thành phần của hợp chất (theo khối lượng) có $42.6$\% là nguyên tố carbon còn lại là nguyên tố oxi. Xác định tỷ lệ số nguyên tử carbon \& số nguyên tử oxi trong hợp chất.
\end{baitoan}

\begin{baitoan}[\cite{An2011}, \textbf{68.}, p. 15]
	1 hợp chất có phân tử khối bằng $62$ đvC. Trong phân tử hợp chất nguyên tố oxi chiếm $25.8$\% theo khối lượng, còn lại là nguyên tố X. Xác định nguyên tố X, biết rằng trong hợp chất có $2$ nguyên tử X.
\end{baitoan}

\begin{baitoan}[\cite{An2011}, \textbf{69.}, p. 15]
	1 hợp chất khí X có thành phần gồm $2$ nguyên tố carbon \& oxi. Biết tỷ lệ về khối lượng của carbon đối với oxi là $m_{\rm C}:m_{\rm O} = 3:8$. Chất khí trên là 1 trong những chất khí chủ yếu làm Trái Đất nóng dần lên (hiệu ứng nhà kính). Xác định công thức phân tử khí này \& giải thích hiệu ứng nhà kính.
\end{baitoan}

\begin{baitoan}[\cite{An2011}, \textbf{70.}, p. 15]
	1 oxit sắt, trong đó nguyên tố sắt chiếm $70$\% theo khối lượng. Xác định công thức phân tử oxit sắt.
\end{baitoan}

\begin{baitoan}[\cite{An2011}, \textbf{71.}, p. 15]
	1 nguyên tử kim loại kết hợp với 1 nguyên tử \ce{O} tạo ra phân tử oxit, trong đó kim loại chiếm $80$\% về khối lượng. Xác định tên kim loại.
\end{baitoan}

\begin{baitoan}[\cite{An2011}, \textbf{72.}, p. 15]
	Tính khối lượng của $4\cdot 10^{23}$ nguyên tử đồng.
\end{baitoan}

\begin{baitoan}[\cite{An2011}, \textbf{73.}, p. 15]
	Điền từ hoặc cụm từ thích hợp vào chỗ trống trong các câu sau:
	\begin{enumerate*}
		\item[(a)] Nước tự nhiên gồm $\ldots$ là $\ldots$ Nước cất là $\ldots$.
		\item[(b)] Dựa vào sự khác nhau về $\ldots$ có thể tách $\ldots$ ra khỏi hỗn hợp.
		\item[(c)] Nguyên tử là $\ldots$ Nguyên tử gồm $\ldots$ \& $\ldots$.
		\item[(d)] Không khí là $\ldots$, trong đó có các $\ldots$ như $\ldots$ \& các $\ldots$, như $\ldots$ \& $\ldots$ nước ở trạng thái $\ldots$.
	\end{enumerate*}
\end{baitoan}

\begin{baitoan}[\cite{An2011}, \textbf{74.}, p. 15]
	Điền từ hoặc cụm từ thích hợp vào chỗ trống trong các câu sau:
	\begin{enumerate*}
		\item[(a)] Nguyên tố hóa học là $\ldots$ có cùng $\ldots$.
		\item[(b)] $\ldots$ biểu diễn nguyên tố \& chỉ 1 nguyên tử của $\ldots$.
		\item[(c)] Phân tử là $\ldots$, gồm $\ldots$ \& $\ldots$ của chất.
		\item[(d)] $\ldots$ chỉ có liên kết giữa các nguyên tử thay đổi làm cho phân tử hay $\ldots$.
	\end{enumerate*}
\end{baitoan}

\begin{baitoan}[\cite{An2011}, \textbf{75.}, p. 16]
	Trong những phát biểu sau, phát biểu nào không đúng?
	\begin{enumerate*}
		\item[{\rm\sf A.}] Hạt nhân nguyên tử tạo bởi proton \& neutron.
		\item[{\rm\sf B.}] Số electron trong nguyên tử bằng số neutron.
		\item[{\rm\sf C.}] Số proton trong hạt nhân bằng số electron ở lớp vỏ nguyên tử.
		\item[{\rm\sf D.}] Trong nguyên tử, electron luôn chuyển động rất nhanh quanh nhân \& sắp xếp thành từng lớp.
	\end{enumerate*}
\end{baitoan}

\begin{baitoan}[\cite{An2011}, \textbf{76.}, p. 16]
	Chọn dãy gồm các công thức hóa học đều viết đúng:
	\begin{enumerate*}
		\item[{\rm\sf A.}] \ce{K2O2,NaOH,AlSO4,H2SO4};
		\item[{\rm\sf B.}] \ce{K2O,NaOH,Al2(SO4)3,H2SO4};
		\item[{\rm\sf C.}] \ce{K2O,K2OH,Al2(SO4)3,H3SO4};
		\item[{\rm\sf D.}] \ce{KO,Na(OH)2,Al2(SO4)3,H2SO4}.
	\end{enumerate*}
\end{baitoan}

\begin{baitoan}[\cite{An2011}, \textbf{77.}, p. 16]
	Mệnh đề nào sau đây không đúng?
	\begin{enumerate*}
		\item[{\rm\sf A.}] Chỉ có hạt nhân nguyên tử canxi mới có $20$ proton.
		\item[{\rm\sf B.}] Chỉ có hạt nhân nguyên tử canxi mới có $20$ neutron.
		\item[{\rm\sf C.}] Chỉ có trong nguyên tử canxi mới có $20$ electron.
		\item[{\rm\sf D.}] Nguyên tử khối của canxi gồm $20$ proton \& $20$ neutron.
	\end{enumerate*}
\end{baitoan}

\begin{baitoan}[\cite{An2011}, \textbf{78.}, p. 16]
	Chọn dãy gồm các công thức hóa học đều viết đúng:
	\begin{enumerate*}
		\item[{\rm\sf A.}] \ce{CaO,CaOH,Al2PO4,HPO4};
		\item[{\rm\sf B.}] \ce{Ca2O,Ca(OH)2,Al(PO4)3,H2PO4};
		\item[{\rm\sf C.}] \ce{CaO,Ca(OH)2,AlPO4,H3PO4};
		\item[{\rm\sf D.}] \ce{CaO2,Ca(OH)2,Al3PO4,H(PO4)2}.
	\end{enumerate*}
\end{baitoan}

\begin{baitoan}[\cite{An2011}, \textbf{79.}, p. 16]
	Trong công thức hóa học của metan \ce{CH4} \& kali oxit \ce{K2O}, xác định được:
	\begin{enumerate*}
		\item[{\rm\sf A.}] Carbon hóa trị IV, kali hóa trị I;
		\item[{\rm\sf B.}] Carbon hóa trị IV, kali hóa trị II;
		\item[{\rm\sf C.}] Carbon hóa trị I, kali hóa trị II;
		\item[{\rm\sf D.}] Carbon hóa trị II, kali hóa trị I.
	\end{enumerate*}
\end{baitoan}

\begin{baitoan}[\cite{An2011}, \textbf{80.}, p. 16]
	Dãy nguyên tố hóa học nào dưới đây đều là kim loại:
	\begin{enumerate*}
		\item[{\rm\sf A.}] \ce{Cu,Ca,Al,P};
		\item[{\rm\sf B.}] \ce{Cu,Fe,S,C};
		\item[{\rm\sf C.}] \ce{Cu,Fe,Al,Na};
		\item[{\rm\sf D.}] \ce{Cu,Fe,Al,H}.
	\end{enumerate*}
\end{baitoan}

\begin{baitoan}[\cite{An2011}, \textbf{81.}, p. 16]
	Dãy nguyên tố hóa học nào dưới đây đều là phi kim:
	\begin{enumerate*}
		\item[{\rm\sf A.}] \ce{S,N,Ca,C};
		\item[{\rm\sf B.}] \ce{S,P,Ag,C};
		\item[{\rm\sf C.}] \ce{S,Hg,C,N};
		\item[{\rm\sf D.}] \ce{S,C,P,N}.
	\end{enumerate*}
\end{baitoan}

%------------------------------------------------------------------------------%

\section{Phản Ứng Hóa Học}

\subsection{Định Luật Bảo Toàn Khối Lượng Các Chất}

\begin{baitoan}[\cite{An2011}, \textbf{82.}, p. 25]
	Cho hỗn hợp 2 muối \ce{A2SO4} \& \ce{BSO4} có khối lượng $44.2$g tác dụng vừa đủ với dung dịch \ce{BaCl2} thì cho $69.9$g kết tủa \ce{BaSO4}. Tính khối lượng 2 muối tan.
\end{baitoan}

\begin{baitoan}[\cite{An2011}, \textbf{83.}, p. 25]
	Đốt cháy $1.5$g kim loại \ce{Mg} trong không khí thu được $2.5$g hợp chất magie oxit \ce{MgO}. Xác định khối lượng oxi đã phản ứng.
\end{baitoan}

\begin{baitoan}[\cite{An2011}, \textbf{84.}, p. 25]
	Cho $m$g kim loại natri vào $50$g nước thấy thoát ra $0.05$g khí hydro \& thu được $51.1$g dung dịch natri hydroxide.
	\begin{enumerate*}
		\item[(a)] Viết phương trình hóa học của phản ứng.
		\item[(b)] Tính giá trị của $m$.
	\end{enumerate*}
\end{baitoan}

\begin{baitoan}[\cite{An2011}, \textbf{85.}, p. 26]
	Cho $5.6$g kim loại \ce{Fe} hòa tan hoàn  toàn vào $18.4$g dung dịch acid \ce{Hcl}. Sau phản ứng thu được dung dịch muối \ce{FeCl2} \& giải phóng $0.2$g khí hydro.
	\begin{enumerate*}
		\item[(a)] Viết phương trình hóa học của phản ứng.
		\item[(b)] Xác định khối lượng dung dịch muối \ce{FeCl2} thu được.
	\end{enumerate*}
\end{baitoan}

\begin{baitoan}[\cite{An2011}, \textbf{86.}, p. 26]
	Xác định công thức phân tử hợp chất A, biết rằng khi đốt cháy $1$ mol chất A cần $6.5$ mol \ce{O2} thu được $4$ mol \ce{CO2} \& $5$ mol \ce{H2O}.
\end{baitoan}

\begin{baitoan}[\cite{An2011}, \textbf{87.}, p. 26]
	Đốt nóng hỗn hợp gồm $1.4$g \ce{Fe} \& $1.6$g \ce{S} trong bình kín không có không khí thu được sắt (II) sunfua \ce{FeS}. Tính khối lượng \ce{FeS} thu được sau phản ứng, biết lượng \ce{S} dùng dư $0.8$g.
\end{baitoan}

%------------------------------------------------------------------------------%

\subsection{Lập Phương Trình Hóa Học}

\begin{baitoan}[\cite{An2011}, \textbf{88.}, p. 27]
	Cân bằng các phản ứng hóa học sau: \emph{\ce{Al + O2 ->[$t^\circ$] Al2O3}, \ce{CO + O2 ->[$t^\circ$] CO2}, \ce{Fe + HCl -> FeCl2 + H2}, \ce{Al(OH)3 ->[$t^\circ$] Al2O3 + H2O}}. 
\end{baitoan}

\begin{baitoan}[\cite{An2011}, \textbf{89.}, p. 27]
	Cho sơ đồ các phản ứng sau:
	\begin{enumerate*}
		\item[(a)] \emph{\ce{N2O5 + H2O -> HNO3}};
		\item[(b)] \emph{\ce{CaO + HCL -> CaCl2 + H2O}}.
	\end{enumerate*}
	Lập phương trình hóa học \& cho biết tỷ lệ số phân tử của các chất trong mỗi phản ứng.
\end{baitoan}

\begin{baitoan}[\cite{An2011}, \textbf{90.}, p. 27]
	Cân bằng các phản ứng hóa học sau: \emph{\ce{Fe + O2 ->[$t^\circ$] Fe3O4}, \ce{CaCO3 + HCl -> CaCl2 + CO2 + H2O}, \ce{Na2CO3 + BaCl2 -> BaSO3 v + NaCl}, \ce{Fe3O4 + HCl -> FeCl2 + FeCl3 + H2O}}.
\end{baitoan}

\begin{baitoan}[\cite{An2011}, \textbf{91.}, p. 28]
	Cân bằng các phản ứng hóa học sau: \emph{\ce{Na + H2O -> NaOH + H2}, \ce{Al + H2SO4 -> Al2(SO4)3 + H2}, \ce{Fe + AgNO3 -> Fe(NO3)3 + Ag}, \ce{Al + Fe2O3 ->[$t^\circ$] Al2O3 + Fe}}.
\end{baitoan}

\begin{baitoan}[\cite{An2011}, \textbf{92.}, p. 28]
	Cho sơ đồ các phản ứng sau:
	\begin{enumerate*}
		\item[(a)] \emph{\ce{K + S -> K2S}};
		\item[(b)] \emph{\ce{Fe + Cl2 ->[$t^\circ$] FeCl3}};
		\item[(c)] \emph{\ce{Ca(OH)2 + CO2 -> CaCO3 + H2O}};
		\item[(d)] \emph{\ce{FeS2 + O2 ->[$t^\circ$] Fe2O3 + SO2}}.
	\end{enumerate*}
	Lập phương trình hóa học của các phản ứng trên.
\end{baitoan}

\begin{baitoan}[\cite{An2011}, \textbf{93.}, p. 29]
	Hoàn thành các phương trình phản ứng sau:
	\begin{enumerate*}
		\item[(a)] \emph{\ce{Fe2O3 +} ? \ce{-> Fe + CO2}};
		\item[(b)] \emph{\ce{NaOH +} ? \ce{-> Fe(OH)2 + NaCl}};
		\item[(c)] \emph{\ce{CH4 +} ? \ce{-> CO2 + H2O}};
		\item[(d)] \emph{? \ce{+ O2 -> MgO}};
		\item[(e)] \emph{? \ce{+ CuCl2 -> FeCl2 + Cu}};
		\item[(f)] \emph{\ce{Fe +} ? \ce{-> FeSO4 + H2 ^}}.
	\end{enumerate*}
\end{baitoan}

\begin{baitoan}[\cite{An2011}, \textbf{94.}, p. 29]
	Cân bằng các phản ứng hóa học sau: \emph{\ce{CaO + HNO3 -> Ca(NO3)2 + H2O}, \ce{P + O2 -> P2O5}, \ce{KMnO4 ->[$t^\circ$] K2MnO4 + MnO2 + O2 ^}, \ce{KClO3 ->[$t^\circ$] KCl + O2 ^}, \ce{N_xO_y + Cu -> CuO + N2}, \ce{Fe_xO_y + HCl -> FeCl_{2y/x} + H2O}}.
\end{baitoan}

\begin{baitoan}[\cite{An2011}, \textbf{95.}, p. 30]
	Hoàn thành các phương trình phản ứng sau: \emph{\ce{Al + Cl2 ->} ?, \ce{K + H2O ->} ? \ce{+ H2 ^}, \ce{Zn +} ? \ce{-> ZnCl2 + } ?, \ce{Ca(OH)2 +} ? \ce{-> Ca3(PO4)2 + H2O}, \ce{AgNO3 +} ? \ce{-> AgCl v + Cu(NO3)2}}.
\end{baitoan}

\begin{baitoan}[\cite{An2011}, \textbf{96.}, p. 30]
	Trong các mệnh đề sau, mệnh đề nào phản ánh bản chất của định luật bảo toàn khối lượng:
	\begin{enumerate*}
		\item[\textbf{1.}] Trong các phản ứng hóa học nguyên tử được bảo toàn, không tự nhiên sinh ra \& cũng không tự nhiên mất đi.
		\item[\textbf{2.}] Tổng khối lượng các sản phẩm bằng tổng khối lượng của các chất phản ứng.
		\item[\textbf{3.}] Trong phản ứng hóa học, nguyên tử không bị phân chia.
		\item[\textbf{4.}] Số phần tử các sản phẩm bằng số phần các chất phản ứng.
	\end{enumerate*}

	\begin{enumerate*}
		\item[{\rm\sf A.}] 1 \& 4;
		\item[{\rm\sf B.}] 1 \& 3;
		\item[{\rm\sf C.}] 3 \& 4;
		\item[{\rm\sf D.}] 1.
	\end{enumerate*}
\end{baitoan}

\begin{baitoan}[\cite{An2011}, \textbf{97.}, pp. 30--31]
	Trong các cách phát biểu về định luật bảo toàn khối lượng như sau. Cách phát biểu nào đúng:
	\begin{enumerate*}
		\item[{\rm\sf A.}] Tổng sản phẩm các chất bằng tổng chất tham gia.
		\item[{\rm\sf B.}] Trong 1 phản ứng, tổng số phân tử chất tham gia bằng tổng số phân tử chất tạo thành.
		\item[{\rm\sf C.}] Trong 1 phản ứng hóa học, tổng khối lượng của các sản phẩm bằng tổng khối lượng của các chất phản ứng.
		\item[{\rm\sf D.}] Trong phản ứng hóa học, khối lượng các nguyên tử không đổi.
	\end{enumerate*}
\end{baitoan}

\begin{baitoan}[\cite{An2011}, \textbf{98.}, p. 31]
	Cho $m$g kim loại nhôm tan hoàn toàn trong $3.65$g acid hydrochloric \ce{HCl}, sau phản ứng thu được $4.45$g muối nhôm clorua (\ce{AlCl3}) \& giải phóng $0.1$g khí \ce{H2}. Khối lượng kim loại nhôm ($m$) đã phản ứng là:
	\begin{enumerate*}
		\item[{\rm\sf A.}] $1.8$g;
		\item[{\rm\sf B.}] $0.9$g;
		\item[{\rm\sf C.}] $1.2$g;
		\item[{\rm\sf D.}] $0.45$g.
	\end{enumerate*}
\end{baitoan}

\begin{baitoan}[\cite{An2011}, \textbf{99.}, p. 31]
	Nung hỗn hợp gồm $3$g \ce{C} \& $10$g \ce{CuO} trong bình kín, sau phản ứng thu được $a$g \ce{Cu} \& giải phóng $2.75$g khí \ce{CO2}. Giá trị của $a$ là:
	\begin{enumerate*}
		\item[{\rm\sf A.}] $10.25$g;
		\item[{\rm\sf B.}] $10.5$g;
		\item[{\rm\sf C.}] $5.75$g;
		\item[{\rm\sf D.}] $9.75$g.
	\end{enumerate*}
\end{baitoan}

\begin{baitoan}[\cite{An2011}, \textbf{100.}, p. 31]
	Hòa tan hoàn toàn $6.2$g \ce{Na2O} vào nước thu được $8$g \ce{NaOH}. Khối lượng nước tham gia phản ứng là:
	\begin{enumerate*}
		\item[{\rm\sf A.}] $0.9$g;
		\item[{\rm\sf B.}] $1.8$g;
		\item[{\rm\sf C.}] $2$g;
		\item[{\rm\sf D.}] $1.6$g.
	\end{enumerate*}
\end{baitoan}

\begin{baitoan}[\cite{An2011}, \textbf{101.}, p. 31]
	Đốt cháy hoàn toàn $2.1$g khí \ce{C3H6} trong $a$g oxi, sau phản ứng thu được $9.3$g khí \ce{CO2} \& \ce{H2O}. Giá trị của $a$ là:
	\begin{enumerate*}
		\item[{\rm\sf A.}] $7.2$;
		\item[{\rm\sf B.}] $3.6$;
		\item[{\rm\sf C.}] $2.7$;
		\item[{\rm\sf D.}] $7.6$.
	\end{enumerate*}
\end{baitoan}

\begin{baitoan}[\cite{An2011}, \textbf{102.}, p. 31]
	Hòa tan hoàn toàn $20$g hỗn hợp $2$ muối \ce{A2CO3} \& \ce{BCO3} vào $14.6$g \ce{HCl} thu được dung dịch X \& $12.4$g \ce{CO2} \& \ce{H2O}. Tổng khối lượng muối tạo thành trong dung dịch X là:
	\begin{enumerate*}
		\item[{\rm\sf A.}] $11.2$g;
		\item[{\rm\sf B.}] $20.2$g;
		\item[{\rm\sf C.}] $22.2$g;
		\item[{\rm\sf D.}] $25.3$g.
	\end{enumerate*}
\end{baitoan}

\begin{baitoan}[\cite{An2011}, \textbf{103.}, p. 31]
	Khi cho $80$kg đất đèn có thành phần chính là canxi cacbua\emph{\texttt{/}}calci carbide hóa hợp $36$kg nước thu được $74$g calci hydroxide \& $26$g khí axetilen được  thể hiện ở phản ứng sau: calci carbide $+$ nước $\to$ calci hydroxide $+$ khí axetilen. Tỷ lệ \% calci carbide nguyên chất có trong đất đèn là:
	\begin{enumerate*}
		\item[{\rm\sf A.}] $80$\%;
		\item[{\rm\sf B.}] $85$\%;
		\item[{\rm\sf C.}] $75$\%;
		\item[{\rm\sf D.}] $90$\%.
	\end{enumerate*}
\end{baitoan}

\begin{baitoan}[\cite{An2011}, \textbf{104.}, p. 31]
	Khi nung đá vôi $90$\% khối lượng calci carbonat \ce{CaCO3} thu được $5.6$ tấn calci oxide \ce{CaO} \& $4.4$ tấn khí carbonic. Khối lượng đá vôi đem nung là:
	\begin{enumerate*}
		\item[{\rm\sf A.}] $12.111$ tấn;
		\item[{\rm\sf B.}] $11.111$ tấn;
		\item[{\rm\sf C.}] $10.55$ tấn;
		\item[{\rm\sf D.}] $13.112$ tấn.
	\end{enumerate*}
\end{baitoan}

\begin{baitoan}[\cite{An2011}, \textbf{105.}, p. 31]
	Khi nung miếng kim loại sắt trong không khí thấy khối lượng:
	\begin{enumerate*}
		\item[{\rm\sf A.}] giảm ít;
		\item[{\rm\sf B.}] tăng lên;
		\item[{\rm\sf C.}] không tăng, không giảm;
		\item[{\rm\sf D.}] giảm nhiều.
	\end{enumerate*}
\end{baitoan}

\begin{baitoan}[\cite{An2011}, \textbf{106.}, p. 32]
	Cho $4.8$g magiê tác dụng vừa đủ với $200$g dung dịch acid hydrochloric thu được dung dịch magiee clorua \& thoát ra $0.4$g khí hydro. Khối lượng dung dịch magiê clorua thu được là:
	\begin{enumerate*}
		\item[{\rm\sf A.}] $200.4$g;
		\item[{\rm\sf B.}] $210$g;
		\item[{\rm\sf C.}] $204.4$g;
		\item[{\rm\sf D.}] $240.4$g.
	\end{enumerate*}
\end{baitoan}

\begin{baitoan}[\cite{An2011}, \textbf{107.}, p. 32]
	Cho $5.4$g nhôm tác dụng với $4.8$g khí oxi tạo thành nhôm oxit (\ce{Al2O3}). Khối lượng nhôm oxit thu được là:
	\begin{enumerate*}
		\item[{\rm\sf A.}] $9.8$g;
		\item[{\rm\sf B.}] $11.5$g;
		\item[{\rm\sf C.}] $10.2$g;
		\item[{\rm\sf D.}] $20.4$g.
	\end{enumerate*}
\end{baitoan}

\begin{baitoan}[\cite{An2011}, \textbf{108.}, p. 32]
	Cho $8.4$g bột sắt cháy hết trong $3.2$g oxi tạo ra oxit sắt từ \ce{Fe3O4}. Khối lượng oxit sắt từ thu được là:
	\begin{enumerate*}
		\item[{\rm\sf A.}] $11.6$g;
		\item[{\rm\sf B.}] $10.6$g;
		\item[{\rm\sf C.}] $16.1$g;
		\item[{\rm\sf D.}] $12.4$g.
	\end{enumerate*}
\end{baitoan}

\begin{baitoan}[\cite{An2011}, \textbf{109.}, p. 32]
	Biết đồng oxit \ce{CuO} bị khử là $400$g, khối lượng khí hydro đã dùng là $10$g khối lượng nước tạo ra là $90$g. Khối lượng đồng sinh ra là:
	\begin{enumerate*}
		\item[{\rm\sf A.}] $230$g;
		\item[{\rm\sf B.}] $320$g;
		\item[{\rm\sf C.}] $390$g;
		\item[{\rm\sf D.}] $310$g.
	\end{enumerate*}
\end{baitoan}

\begin{baitoan}[\cite{An2011}, \textbf{110.}, p. 32]
	Cho biết khối lượng khí hydro đã dùng là $5$g, khối lượng \ce{Cu} sinh ra là $160$g, khối lượng nước tạo ra là $45$g. Khối lượng đồng oxit bị khử là:
	\begin{enumerate*}
		\item[{\rm\sf A.}] $200$g;
		\item[{\rm\sf B.}] $195$g;
		\item[{\rm\sf C.}] $165$g;
		\item[{\rm\sf D.}] $205$g.
	\end{enumerate*}
\end{baitoan}

\begin{baitoan}[\cite{An2011}, \textbf{111.}, p. 32]
	Cho $2.7$g kim loại nhôm phản ứng với dung dịch acid sulfuric \ce{H2SO4} vừa đủ thì thu được $17.1$g muối nhôm sunfat \ce{Al2(SO4)3}, \& $0.3$g khí \ce{H2}. Khối lượng \ce{H2SO4} đã dùng là:
	\begin{enumerate*}
		\item[{\rm\sf A.}] $17.4$g;
		\item[{\rm\sf B.}] $14.7$g;
		\item[{\rm\sf C.}] $16.9$g;
		\item[{\rm\sf D.}] $34.8$g.
	\end{enumerate*}
\end{baitoan}

\begin{baitoan}[\cite{An2011}, \textbf{112.}, p. 32]
	Đốt cháy $a$g chất X cần dùng $6.4$g \ce{O2} \& thu được $4.4$g \ce{CO2} \& $3.6$g \ce{H2O}. Giá trị của $a$ là:
	\begin{enumerate*}
		\item[{\rm\sf A.}] $0.8$;
		\item[{\rm\sf B.}] $3.2$;
		\item[{\rm\sf C.}] $1.6$;
		\item[{\rm\sf D.}] $1.8$.
	\end{enumerate*}
\end{baitoan}

\begin{baitoan}[\cite{An2011}, \textbf{113.}, p. 32]
	Cho $19.1$g hỗn hợp X gồm $2$ muối \ce{Na2SO4} \& \ce{MgSO4} tác dụng vừa đủ với $31.2$g \ce{BaCl2} thu được $34.95$g kết tủa \ce{BaSO4} \& $2$ muối tan (\ce{NaCl,MgCl2}). Khối lượng 2 muối tan sau phản ứng là:
	\begin{enumerate*}
		\item[{\rm\sf A.}] $15.35$g;
		\item[{\rm\sf B.}] $13.53$g;
		\item[{\rm\sf C.}] $15.57$g;
		\item[{\rm\sf D.}] $17.75$g.
	\end{enumerate*}
\end{baitoan}

\begin{baitoan}[\cite{An2011}, \textbf{114.}, p. 32]
	Chọn phương trình hóa học đã cân bằng đúng:
	\begin{enumerate*}
		\item[{\rm\sf A.}] \emph{\ce{NH4NO3 ->[$t^\circ$] N2 + O2 + 2H2O}};
		\item[{\rm\sf B.}] \emph{\ce{2NH4NO3 ->[$t^\circ$] 2N2 + O2 + 4H2O}};
		\item[{\rm\sf C.}] \emph{\ce{2NH4NO3 ->[$t^\circ$] 2N2 + 2O2 + 4H2O}};
		\item[{\rm\sf D.}] \emph{\ce{2NH4NO3 ->[$t^\circ$] N2 + O2 + 4H2O}}.
	\end{enumerate*}
\end{baitoan}

\begin{baitoan}[\cite{An2011}, \textbf{115.}, pp. 32--33]
	Chọn phương trình hóa học đã cân bằng đúng:\\
	\begin{enumerate*}
		\item[{\rm\sf A.}] \emph{\ce{(NH4)2Cr2O7 ->[$t^\circ$] 2N2 + Cr2O3 + 4H2O}};
		\item[{\rm\sf B.}] \emph{\ce{2(NH4)2Cr2O7 ->[$t^\circ$] 2N2 + 2Cr2O3 + 2H2O}};
		\item[{\rm\sf C.}] \emph{\ce{(NH4)2Cr2O7 ->[$t^\circ$] N2 + Cr2O3 + 4H2O}};
		\item[{\rm\sf D.}] \emph{\ce{(NH4)2Cr2O7 ->[$t^\circ$] N2 + Cr2O3 + 2H2O}}.
	\end{enumerate*}
\end{baitoan}

\begin{baitoan}[\cite{An2011}, \textbf{116.}, p. 33]
	Chọn phương trình hóa học đã cân bằng đúng:
	\begin{enumerate*}
		\item[{\rm\sf A.}] \emph{\ce{NaHCO3 ->[$t^\circ$] Na2CO3 + CO2 + H2O}};
		\item[{\rm\sf B.}] \emph{\ce{2NaHCO3 ->[$t^\circ$] Na2CO3 + CO2 + H2O}};
		\item[{\rm\sf C.}] \emph{\ce{2NaHCO3 ->[$t^\circ$] Na2CO3 + 2CO2 + H2O}};
		\item[{\rm\sf D.}] \emph{\ce{2NaHCO3 ->[$t^\circ$] Na2CO3 + CO2 + 2H2O}}.
	\end{enumerate*}
\end{baitoan}

\begin{baitoan}[\cite{An2011}, \textbf{117.}, p. 33]
	Chọn phương trình hóa học đã cân bằng đúng:
	\begin{enumerate*}
		\item[{\rm\sf A.}] \emph{\ce{(NH4)2CO3 ->[$t^\circ$] NH3 + CO2 + H2O}};
		\item[{\rm\sf B.}] \emph{\ce{(NH4)2CO3 ->[$t^\circ$] 2NH3 + CO2 + 2H2O}};
		\item[{\rm\sf C.}] \emph{\ce{(NH4)2CO3 ->[$t^\circ$] 2NH3 + 2CO2 + H2O}};
		\item[{\rm\sf D.}] \emph{\ce{(NH4)2CO3 ->[$t^\circ$] 2NH3 + CO2 + H2O}}.
	\end{enumerate*}
\end{baitoan}

\begin{baitoan}[\cite{An2011}, \textbf{118.}, p. 33]
	Chọn phương trình hóa học đã cân bằng đúng:
	\begin{enumerate*}
		\item[{\rm\sf A.}] \emph{\ce{Cu(NO3)2 ->[$t^\circ$] CuO + NO2 + O2}};
		\item[{\rm\sf B.}] \emph{\ce{2Cu(NO3)2 ->[$t^\circ$] 2CuO + 4NO2 + O2}};
		\item[{\rm\sf C.}] \emph{\ce{2Cu(NO3)2 ->[$t^\circ$] 2CuO + NO2 + O2}};
		\item[{\rm\sf D.}] \emph{\ce{2Cu(NO3)2 ->[$t^\circ$] 2CuO + 2NO2 + O2}}.
	\end{enumerate*}
\end{baitoan}

\begin{baitoan}[\cite{An2011}, \textbf{119.}, p. 33]
	Cho sơ đồ phản ứng sau: \emph{\ce{Fe(OH)y + H2SO4 -> Fex(SO4)y + H2O}}. Chọn $x,y$ bằng các chỉ số thích hợp để lập được phương trình hóa học trên ($x\ne y$).
	\begin{enumerate*}
		\item[{\rm\sf A.}] $x = 1$, $y = 2$;
		\item[{\rm\sf B.}] $x = 2$, $y = 3$;
		\item[{\rm\sf C.}] $x = 3$, $y = 1$;
		\item[{\rm\sf D.}] $x = 2$, $y = 4$.
	\end{enumerate*}
\end{baitoan}

\begin{baitoan}[\cite{An2011}, \textbf{120.}, p. 33]
	Cho sơ đồ phản ứng sau: \emph{\ce{Al(OH)y + H2SO4 -> Alx(SO4)y + H2O}}. Chọn $x,y$ bằng các chỉ số thích hợp để lập được phương trình hóa học trên ($x\ne y$).
	\begin{enumerate*}
		\item[{\rm\sf A.}] $x = 2$, $y = 1$;
		\item[{\rm\sf B.}] $x = 3$, $y = 4$;
		\item[{\rm\sf C.}] $x = 2$, $y = 3$;
		\item[{\rm\sf D.}] $x = 4$, $y = 3$.
	\end{enumerate*}
\end{baitoan}

%------------------------------------------------------------------------------%

\section{Công Thức Hóa Học -- Phương Trình Hóa Học}

\subsection{Tính Theo Công Thức Hóa Học}

%------------------------------------------------------------------------------%

\subsection{Tính Theo Phương Trình Hóa Học}

%------------------------------------------------------------------------------%

\section{Oxi -- Không Khí}

\subsection{Sự Oxi Hóa -- Oxit}

%------------------------------------------------------------------------------%

\subsection{Phản Ứng Hóa Hợp, Phản Ứng Phân Hủy}

%------------------------------------------------------------------------------%

\section{Hydro -- Nước}

\subsection{Phản Ứng Oxi Hóa -- Khử}

%------------------------------------------------------------------------------%

\subsection{Axit -- Bazơ -- Muối}

%------------------------------------------------------------------------------%

\section{Dung Dịch}

\subsection{Dung Dịch \& Độ Tan của 1 Chất Trong Nước}

%------------------------------------------------------------------------------%

\subsection{Pha Trộn Dung Dịch}

%------------------------------------------------------------------------------%

\subsection{Chuyển Đổi Nồng Độ Dung Dịch}

%------------------------------------------------------------------------------%

\subsection{Bài Hóa Liên Quan đến Nồng Độ Dung Dịch}

%------------------------------------------------------------------------------%

\printbibliography[heading=bibintoc]
	
\end{document}