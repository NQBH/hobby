\documentclass{article}
\usepackage[backend=biber,natbib=true,style=authoryear]{biblatex}
\addbibresource{/home/hong/1_NQBH/reference/bib.bib}
\usepackage[utf8]{vietnam}
\usepackage{tocloft}
\renewcommand{\cftsecleader}{\cftdotfill{\cftdotsep}}
\usepackage[colorlinks=true,linkcolor=blue,urlcolor=red,citecolor=magenta]{hyperref}
\usepackage{amsmath,amssymb,amsthm,mathtools,float,graphicx,algpseudocode,algorithm,tcolorbox,tikz,tkz-tab,subcaption}
\DeclareMathOperator{\arccot}{arccot}
\usepackage[inline]{enumitem}
\allowdisplaybreaks
\numberwithin{equation}{section}
\newtheorem{assumption}{Assumption}[section]
\newtheorem{nhanxet}{Nhận xét}[section]
\newtheorem{conjecture}{Conjecture}[section]
\newtheorem{corollary}{Corollary}[section]
\newtheorem{hequa}{Hệ quả}[section]
\newtheorem{definition}{Definition}[section]
\newtheorem{dinhnghia}{Định nghĩa}[section]
\newtheorem{example}{Example}[section]
\newtheorem{vidu}{Ví dụ}[section]
\newtheorem{lemma}{Lemma}[section]
\newtheorem{notation}{Notation}[section]
\newtheorem{principle}{Principle}[section]
\newtheorem{problem}{Problem}[section]
\newtheorem{baitoan}{Bài toán}[section]
\newtheorem{proposition}{Proposition}[section]
\newtheorem{menhde}{Mệnh đề}[section]
\newtheorem{question}{Question}[section]
\newtheorem{cauhoi}{Câu hỏi}[section]
\newtheorem{quytac}{Quy tắc}
\newtheorem{remark}{Remark}[section]
\newtheorem{luuy}{Lưu ý}[section]
\newtheorem{theorem}{Theorem}[section]
\newtheorem{tiende}{Tiên đề}[section]
\newtheorem{dinhly}{Định lý}[section]
\usepackage[left=0.5in,right=0.5in,top=1.5cm,bottom=1.5cm]{geometry}
\usepackage{fancyhdr}
\pagestyle{fancy}
\fancyhf{}
\lhead{\small \textsc{Sect.} ~\thesection}
\rhead{\small \nouppercase{\leftmark}}
\renewcommand{\sectionmark}[1]{\markboth{#1}{}}
\cfoot{\thepage}
\def\labelitemii{$\circ$}

\title{Some Topics in Elementary Chemistry\texttt{/}Grade 9}
\author{Nguyễn Quản Bá Hồng\footnote{Independent Researcher, Ben Tre City, Vietnam\\e-mail: \texttt{nguyenquanbahong@gmail.com}; website: \url{https://nqbh.github.io}.}}
\date{\today}

\begin{document}
\maketitle
\begin{abstract}
	Tóm tắt kiến thức Hóa học lớp 9 theo chương trình giáo dục của Việt Nam \& một số chủ đề nâng cao.
\end{abstract}
\setcounter{secnumdepth}{4}
\setcounter{tocdepth}{3}
\tableofcontents
\newpage

%------------------------------------------------------------------------------%

\section{Các Loại Hợp Chất Vô Cơ}

\subsection{Tính Chất Hóa Học của Oxit. Khái Quát về Sự Phân Loại Oxit}

%------------------------------------------------------------------------------%

\subsection{1 Số Oxit Quan Trọng}

%------------------------------------------------------------------------------%

\subsection{Tính Chất Hóa Học của Acid}

%------------------------------------------------------------------------------%

\subsection{1 Số Acid Quan Trọng}

%------------------------------------------------------------------------------%

\subsection{Tính Chất Hóa Học của Base}

%------------------------------------------------------------------------------%

\subsection{1 Số Base Quan Trọng}

%------------------------------------------------------------------------------%

\subsection{Tính Chất Hóa Học của Muối}

%------------------------------------------------------------------------------%

\subsection{1 Số Muối Quan Trọng}

%------------------------------------------------------------------------------%

\subsection{Phân Bón Hóa Học}

%------------------------------------------------------------------------------%

\subsection{Mối Quan Hệ Giữa Các Loại Hợp Chất Vô Cơ}

%------------------------------------------------------------------------------%

\section{Kim Loại}

\subsection{Tính Chất Vật Lý của Kim Loại}

%------------------------------------------------------------------------------%

\subsection{Tính Chất Hóa Học của Kim Loại}

%------------------------------------------------------------------------------%

\subsection{Dãy Hoạt Động Hóa Học của Kim Loại}

%------------------------------------------------------------------------------%

\subsection{Nhôm}

%------------------------------------------------------------------------------%

\subsection{Sắt}

%------------------------------------------------------------------------------%

\subsection{Hợp Kim Sắt: Gang, Thép}

%------------------------------------------------------------------------------%

\subsection{Sự Ăn Mòn Kim Loại \& Bảo Vệ Kim Loại Không Bị Ăn Mòn}

%------------------------------------------------------------------------------%

\section{Phi Kim. Sơ Lược về Bảng Tuần Hoàn Các Nguyên Tố Hóa Học}

\subsection{Tính Chất của Phi Kim}

%------------------------------------------------------------------------------%

\subsection{Clo}

%------------------------------------------------------------------------------%

\subsection{Carbon}

%------------------------------------------------------------------------------%

\subsection{Các Oxit của Carbon}

%------------------------------------------------------------------------------%

\subsection{Acid Carbonic \& Muối Carbonat}

%------------------------------------------------------------------------------%

\subsection{Silic. Công Nghiệp Silicat}

%------------------------------------------------------------------------------%

\subsection{Sơ Lược về Bảng Tuần Hoàn Các Nguyên Tố Hóa Học}

%------------------------------------------------------------------------------%

\section{Hydrocarbon. Nhiên Liệu}

\subsection{Khái Niệm về Hợp Chất Hữu Cơ \& Hóa Học Hữu Cơ}

%------------------------------------------------------------------------------%

\subsection{Cấu Tạo Phân Tử Hợp Chất Hữu Cơ}

%------------------------------------------------------------------------------%

\subsection{Metan}

%------------------------------------------------------------------------------%

\subsection{Etilen}

%------------------------------------------------------------------------------%

\subsection{Axetilen}

%------------------------------------------------------------------------------%

\subsection{Benzen}

%------------------------------------------------------------------------------%

\subsection{Dầu Mỏ \& Khí Thiên Nhiên}

%------------------------------------------------------------------------------%

\subsection{Nhiên Liệu}

%------------------------------------------------------------------------------%

\section{Dẫn Xuất của Hydrocarbon. Polyme}

\subsection{Rượu Etylic}

%------------------------------------------------------------------------------%

\subsection{Acid Axetic}

%------------------------------------------------------------------------------%

\subsection{Mối Liên Hệ giữa Etilen, Rượu Etylic \& Acid Axetic}

%------------------------------------------------------------------------------%

\subsection{Chất Béo}

%------------------------------------------------------------------------------%

\subsection{Glucozơ}

%------------------------------------------------------------------------------%

\subsection{Saccarozơ}

%------------------------------------------------------------------------------%

\subsection{Tinh Bột \& Xenlulozơ}

%------------------------------------------------------------------------------%

\subsection{Protein}

%------------------------------------------------------------------------------%

\subsection{Polime}

%------------------------------------------------------------------------------%

\printbibliography[heading=bibintoc]
	
\end{document}