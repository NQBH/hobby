\documentclass{article}
\usepackage[backend=biber,natbib=true,style=alphabetic,maxbibnames=50]{biblatex}
\addbibresource{/home/nqbh/reference/bib.bib}
\usepackage[utf8]{vietnam}
\usepackage{tocloft}
\renewcommand{\cftsecleader}{\cftdotfill{\cftdotsep}}
\usepackage[colorlinks=true,linkcolor=blue,urlcolor=red,citecolor=magenta]{hyperref}
\usepackage{amsmath,amssymb,amsthm,float,graphicx,mathtools,diagbox,tikz,tipa}
\usepackage[version=4]{mhchem}
\allowdisplaybreaks
\newtheorem{assumption}{Assumption}
\newtheorem{baitoan}{Bài toán}
\newtheorem{cauhoi}{Câu hỏi}
\newtheorem{conjecture}{Conjecture}
\newtheorem{corollary}{Corollary}
\newtheorem{dangtoan}{Dạng toán}
\newtheorem{definition}{Definition}
\newtheorem{dinhly}{Định lý}
\newtheorem{dinhnghia}{Định nghĩa}
\newtheorem{example}{Example}
\newtheorem{ghichu}{Ghi chú}
\newtheorem{hequa}{Hệ quả}
\newtheorem{hypothesis}{Hypothesis}
\newtheorem{lemma}{Lemma}
\newtheorem{luuy}{Lưu ý}
\newtheorem{nhanxet}{Nhận xét}
\newtheorem{notation}{Notation}
\newtheorem{note}{Note}
\newtheorem{principle}{Principle}
\newtheorem{problem}{Problem}
\newtheorem{proposition}{Proposition}
\newtheorem{question}{Question}
\newtheorem{remark}{Remark}
\newtheorem{theorem}{Theorem}
\newtheorem{thinghiem}{Thí nghiệm}
\newtheorem{vidu}{Ví dụ}
\usepackage[left=1cm,right=1cm,top=5mm,bottom=5mm,footskip=4mm]{geometry}

\title{Problem: Hydrocarbon {\it\&} Polyme -- Bài Tập: Hydrocarbon {\it\&} Polyme}
\author{Nguyễn Quản Bá Hồng\footnote{Independent Researcher, Ben Tre City, Vietnam\\e-mail: \texttt{nguyenquanbahong@gmail.com}; website: \url{https://nqbh.github.io}.}}
\date{\today}

\begin{document}
\maketitle
\begin{abstract}
	
\end{abstract}
\setcounter{secnumdepth}{4}
\setcounter{tocdepth}{3}
\tableofcontents

%------------------------------------------------------------------------------%

\begin{baitoan}[\cite{An_Hoa_Hoc_nang_cao_8_9}, 1., p. 126]
	Đốt cháy hoàn toàn 1 hợp chất A chỉ chứa 2 nguyên tố, thu được {\rm22 g} khí {\rm\ce{CO2}} \& {\rm9g \ce{H2O}}. Biết $\rm1\ dm^3$ chất đó trong các điều kiện tiêu chuẩn nặng {\rm1.25 g}. Hỏi: (a) A là chất hữu cơ hay vô cơ? Giải thích. (b) Tính tỷ lệ nguyên tử 2 nguyên tố trong phân tử chất A. (c) Tìm {\rm CTPT} của A, viết {\rm CTCT} của A.
\end{baitoan}

\begin{baitoan}[\cite{An_Hoa_Hoc_nang_cao_8_9}, 2., p. 127]
	Khi cho {\rm2.8 L} hỗn hợp ethylen \& methane đi qua bình đựng nước bromine, thấy {\rm4 g} bromine đã tham gia phản ứng. Tính thành phần {\rm\%} về thể tích các khí trong hỗn hợp biết các phản ứng xảy ra hoàn toàn, thể tích các khí đo ở đktc.
\end{baitoan}

\begin{baitoan}[\cite{An_Hoa_Hoc_nang_cao_8_9}, 3., p. 127]
	Làm thế nào để nhận biết 3 lọ dung dịch đựng 3 hóa chất benzen, rượu etilic, acid acetic.
\end{baitoan}

\begin{baitoan}[\cite{An_Hoa_Hoc_nang_cao_8_9}, 4., p. 128]
	Cho {\rm10 mL} rượu $96^\circ$ tác dụng với sodium lấy dư. (a) Viết {\rm PTHH}. (b) Tìm khối lượng \& thể tích rượu nguyên chất đã tham gia phản ứng biết khối lượng riêng của rượu là {\rm0.8 g{\tt/}mL}. (c) Tính thể tích hydrogen thu được ở đktc biết khối lượng riêng của nước là {\rm1 g{\tt/}mL}.
\end{baitoan}

\begin{baitoan}[\cite{An_Hoa_Hoc_nang_cao_8_9}, 5., p. 129]
	Đốt cháy hoàn toàn {\rm30 mL} rượu ethylic chưa biết rõ độ rượu, cho toàn bộ sản phẩm sinh ra đi qua nước vôi trong dư, lọc lấy kết tủa, sấy khô cân được {\rm100 g}. (a) Viết {\rm PTHH}. Thể tích không khí (chứa {\rm20\%} thể tích oxygen) để đốt lượng rượu đó, giả sử phản ứng xảy ra hoàn toàn. (b) Xác định độ rượu biết khối lượng riêng của rượu ethylic nguyên chất là {\rm0.8 g{\tt/}mL}.
\end{baitoan}

\begin{baitoan}[\cite{An_Hoa_Hoc_nang_cao_8_9}, 6., p. 130]
	Đun {\rm8.9 kg} {\rm\ce{(C17H35COO)3C3H5}} với 1 lượng dư {\rm NaOH}. (a) Viết {\rm PTHH}. (b) Tính lượng glyxerol sinh ra. (c) Tính lượng xà phòng bánh thu được, nếu phản ứng xảy ra hoàn toàn \& xà phòng chứa {\rm60\%} (theo khối lượng) {\rm\ce{C17H35COONa}}.
\end{baitoan}

\begin{baitoan}[\cite{An_Hoa_Hoc_nang_cao_8_9}, 7., p. 130]
	Cho {\rm220 mL} rượu ethylic lên men giấm. Dung dịch thu được cho trung hòa vừa đủ bằng dung dịch {\rm NaOH} \& thu được {\rm208 g} muối khan. Tính hiệu suất phản ứng rượu lên men thành giấm. Biết khối lượng riêng của rượt là {\rm0.8 g{\tt/}mL}.
\end{baitoan}

\begin{baitoan}[\cite{An_Hoa_Hoc_nang_cao_8_9}, 8., p. 131]
	Cho {\rm27.2 g} hỗn hợp rượu ethylic \& acid acetic tác dụng với {\rm Na} dư giải phóng ra {\rm5.6 L} khí hydrogen (đktc). Nếu hỗn hợp đó cho tham gia phản ứng este hóa ta thu được bao nhiêu {\rm g} este (giả sử phản ứng xảy ra hoàn toàn). Hỗn hợp sau khi tham gia phản ứng este hóa đem đổ vào {\rm20 mL} nước. TÍnh nồng độ các chất trong dung dịch đó.
\end{baitoan}

\begin{baitoan}[\cite{An_Hoa_Hoc_nang_cao_8_9}, 9., p. 132]
	Cho {\rm10 L} hỗn hợp khí {\rm\ce{CH4,C2H2}} tác dụng với {\rm10 L \ce{H2}}. Sau khi phản ứng thu được {\rm16 L} hỗn hợp khí. Tính thành phần {\rm\%} của mỗi khí theo thể tích hỗn hợp trước \& sau phản ứng (giả sử phản ứng xảy ra hoàn toàn).
\end{baitoan}

\begin{baitoan}[\cite{An_Hoa_Hoc_nang_cao_8_9}, 10., p. 133]
	1 chất hữu cơ có phân tử khối là $26$. Xác định {\rm CTPT} của hợp chất biết sản phẩm của sự đốt cháy hợp chất đó là khí carbonic \& hơi nước.
\end{baitoan}

\begin{baitoan}[\cite{An_Hoa_Hoc_nang_cao_8_9}, 11., p. 133]
	Cho hydrocarbon A \& B. Đốt cháy hết {\rm1 L} A, cần {\rm6 L} khí oxygen, thu được {\rm4 L} khí {\rm\ce{CO2}}, {\rm4 L} hơi nước. Đốt cháy hết {\rm1 L} B cần {\rm5.5 L} oxygen, thu được {\rm4 L \ce{CO2}, 3 L} hơi nước. Các thể tích khí đo ở cùng điều kiện. Xác định công thức phân tử của A \& B.
\end{baitoan}

\begin{baitoan}[\cite{An_Hoa_Hoc_nang_cao_8_9}, 12., p. 133]
	Đốt cháy 1 hợp chất hữu cơ thu được khí {\rm\ce{CO2}} \& hơi nước với tỷ lệ thể tích là $V_{\ce{CO2}}:V_{\ce{H2O}} = 3:2$. Tỷ khối hơi của hợp chất hữu cơ đối với hydrogen là $36$. Các thể tích khí đo ở cùng điều kiện. Xác định {\rm CTPT} hợp chất hữu cơ.
\end{baitoan}

\begin{baitoan}[\cite{An_Hoa_Hoc_nang_cao_8_9}, 13., p. 133]
	Đốt cháy hoàn toàn {\rm3 g} chất A, thu được {\rm2.24 L \ce{CO2}} (đktc) \& {\rm 1.8 g} nước. Tỷ khối hơi của A với methane là $3.75$. Tìm {\rm CTCT} của chất A biết chất A tác dụng với dung dịch {\rm NaOH}.
\end{baitoan}

\begin{baitoan}[\cite{An_Hoa_Hoc_nang_cao_8_9}, 14., p. 133]
	Trộn hydrocarbon A với lượng khí {\rm\ce{H2}} được hỗn hợp khí B. Đốt cháy hết {\rm4.8 g} B tạo ra {\rm13.2 g} khí {\rm\ce{CO2}} \& mặt khác {\rm4.8 g} hỗn hợp đó làm mất màu dung dịch chứa {\rm32 g} bromine. Xác định {\rm CTPT} của A.
\end{baitoan}

\begin{baitoan}[\cite{An_Hoa_Hoc_nang_cao_8_9}, 15., p. 133]
	Dẫn hỗn hợp gồm {\rm6.72 L} (đktc) khí methane \& ethylen qua bình đựng dung dịch brom có khối lượng {\rm56 g} thì khối lượng bình tăng thêm {\rm10\%}. Nếu đốt cháy hoàn toàn hỗn hợp trên rồi dẫn toàn bộ sản phẩm cháy vào {\rm500 mL} dung dịch {\rm NaOH 1.2M}. Tính khối lượng muối tạo thành.
\end{baitoan}

\begin{baitoan}[\cite{An_Hoa_Hoc_nang_cao_8_9}, 16., p. 134]
	Xác định độ rượu của các dung dịch sau: (a) Cho {\rm25 g} rượu ethylic A tác dụng vừa đủ với {\rm80 g} dung dịch acid acetic {\rm15\%}. (b) Cho {\rm20 g} dung dịch rượu ethylic B tác dụng với sodium thu được {\rm5.6 L} khí hydrogen (đktc). Biết khối lượng của rượu là {\rm0.8 g{\tt/}mL}, của nước là {\rm1 g{\tt/}mL}.
\end{baitoan}

\begin{baitoan}[\cite{An_Hoa_Hoc_nang_cao_8_9}, 17., p. 134]
	Đốt cháy hoàn toàn {\rm8.04 g} chất A thu được {\rm6.36 g \ce{Na2CO3}} \& {\rm2.64 g \ce{CO2}}. Phân tử khối của A là {\rm134 đvC}. (a) Xác định {\rm CTPT} của A. (b) A là hợp chất vô cơ hay hữu cơ. Viết {\rm CTCT} của A.
\end{baitoan}

\begin{baitoan}[\cite{An_Hoa_Hoc_nang_cao_8_9}, 18., p. 134, TS THPT chuyên ĐHKHTN Hà Nội 1998]
	Chia {\rm39.6 g} hỗn hợp rượu ethylic \& rượu X có công thức {\rm\ce{C_nH_{2n}(OH)2}} thành 2 phần bằng nhau. Lấy phần thứ 1 cho tác dụng hết với {\rm Na} thu được {\rm5.6 L \ce{H2}} (đktc); đốt cháy hoàn toàn phần thứ 2 thu được {\rm17.92 L \ce{CO2}} (đktc). Tìm {\rm CTPT} của rượu X biết mỗi nguyên tử carbon chỉ liên kết với 1 nhóm {\rm OH}.
\end{baitoan}

\begin{baitoan}[\cite{An_Hoa_Hoc_nang_cao_8_9}, 19., p. 134, TS PTNK ĐHKHTN Tp. HCM 1998]
	1 hỗn hợp A gồm {\rm\ce{H2}} \& 1 olefin thể tích bằng nhau. Nung nóng hỗn hợp này có xúc tác {\rm Ni} thu được hỗn hợp X. Hiệu suất của phản ứng đạt {\rm75\%}. Tỷ khối của hỗn hợp X so với {\rm\ce{H2}} bằng {\rm23.2}. Tìm {\rm CTPT} olefin.
\end{baitoan}

\begin{baitoan}[\cite{An_Hoa_Hoc_nang_cao_8_9}, 23., p. 135, TS THPT chuyên Lê Hồng Phong Tp. HCM 2003--2004]
	Đốt cháy hoàn toàn 1 hỗn hợp các lượng bằng nhau về số mol của 2 hydrocarbon, thu được {\rm1.76 g \ce{CO2}} \& {\rm0.9 g \ce{H2O}}. Xác định {\rm CTCT} của 2 hydrocarbon biết chúng có cùng số nguyên tử carbon trong phân tử.
\end{baitoan}

\begin{baitoan}[\cite{An_Hoa_Hoc_nang_cao_8_9}, 24., p. 135, TS THPT Trần Đại Nghĩa 2003--2004]
	Hỗn hợp A gồm 3 hydrocarbon khí {\rm\ce{C_nH_{2n+2},C_pH_{2p}}, \ce{C_mH_{2m-2}}}. Đốt cháy hoàn toàn {\rm2.688 L} (đktc) hỗn hợp A, sau phản ứng dẫn hỗn hợp sản phẩm lần lượt qua bình 1 đựng {\rm\ce{H2SO4}} đặc \& bình 2 đựng dung dịch {\rm KOH} đặc, thấy khối lượng bình 1 tăng {\rm5.04 g} \& bình 2 tăng {\rm14.08 g}. (a) Biết trong hỗn hợp A, thể tích hydrocarbon {\rm\ce{C_mH_{2m-2}}} gấp $3$ lần thể tích hydrocarbon {\rm\ce{C_nH_{2n+2}}}. Tính thành phần {\rm\%} theo thể tích của hỗn hợp A. (b) Xác định {\rm CTPT} 3 hydrocarbon nếu biết thêm trong hỗn hợp A có 2 hydrocarbon có số nguyên tử carbon bằng nhau \& bằng $\frac{1}{2}$ số nguyên tử carbon của hydrocarbon còn lại.
\end{baitoan}

%------------------------------------------------------------------------------%

\printbibliography[heading=bibintoc]

\end{document}