\documentclass{article}
\usepackage[backend=biber,natbib=true,style=alphabetic,maxbibnames=50]{biblatex}
\addbibresource{/home/nqbh/reference/bib.bib}
\usepackage[utf8]{vietnam}
\usepackage{tocloft}
\renewcommand{\cftsecleader}{\cftdotfill{\cftdotsep}}
\usepackage[colorlinks=true,linkcolor=blue,urlcolor=red,citecolor=magenta]{hyperref}
\usepackage{amsmath,amssymb,amsthm,float,graphicx,mathtools,diagbox,tikz,tipa}
\usepackage[version=4]{mhchem}
\allowdisplaybreaks
\newtheorem{assumption}{Assumption}
\newtheorem{baitoan}{}
\newtheorem{cauhoi}{Câu hỏi}
\newtheorem{conjecture}{Conjecture}
\newtheorem{corollary}{Corollary}
\newtheorem{dangtoan}{Dạng toán}
\newtheorem{definition}{Definition}
\newtheorem{dinhly}{Định lý}
\newtheorem{dinhnghia}{Định nghĩa}
\newtheorem{example}{Example}
\newtheorem{ghichu}{Ghi chú}
\newtheorem{hequa}{Hệ quả}
\newtheorem{hypothesis}{Hypothesis}
\newtheorem{lemma}{Lemma}
\newtheorem{luuy}{Lưu ý}
\newtheorem{nhanxet}{Nhận xét}
\newtheorem{notation}{Notation}
\newtheorem{note}{Note}
\newtheorem{principle}{Principle}
\newtheorem{problem}{Problem}
\newtheorem{proposition}{Proposition}
\newtheorem{question}{Question}
\newtheorem{remark}{Remark}
\newtheorem{theorem}{Theorem}
\newtheorem{thinghiem}{Thí nghiệm}
\newtheorem{vidu}{Ví dụ}
\usepackage[left=1cm,right=1cm,top=5mm,bottom=5mm,footskip=4mm]{geometry}

\title{Problem: Oxide Acid $+$ Base Reaction\\Bài Tập: Oxide Acid Tác Dụng với Dung Dịch Kiềm}
\author{Nguyễn Quản Bá Hồng\footnote{Independent Researcher, Ben Tre City, Vietnam\\e-mail: \texttt{nguyenquanbahong@gmail.com}; website: \url{https://nqbh.github.io}.}}
\date{\today}

\begin{document}
\maketitle
\tableofcontents

%------------------------------------------------------------------------------%

\section{\ce{CO2} $+$ NaOH{\tt/}KOH}

\begin{luuy}
	Các từ ``hoặc'' hay ký hiệu ``{\tt/}'' sau đây có nghĩa là ``chỉ 1 trong 2'' không phải ``logical or'' (trường hợp sau có thể xảy ra 1 trong 2 hoặc cả 2).
\end{luuy}

\noindent\textbf{\textsf{Trường hợp \ce{CO2} tác dụng với dung dịch NaOH hoặc KOH.}} Thứ tự phản ứng: \ce{CO2 + $2$NaOH -> Na2CO3 + H2O} hoặc \ce{CO2 + $2$KOH -> K2CO3 + H2O} (1). Nếu NaOH{\tt/}KOH hết mà vẫn sục khí \ce{CO2} vào: \ce{CO2 + Na2CO3 + H2O -> $2$NaHCO3} hoặc \ce{CO2 + K2CO3 + H2O -> $2$KHCO3} (2). (1) $+$ (2): \ce{CO2 + NaOH -> NaHCO3} hoặc \ce{CO2 + KOH -> KHCO3} (3). Đặt $a = \frac{n_{\rm NaOH}}{n_{\ce{CO2}}}$ hoặc $a = \frac{n_{\rm KOH}}{n_{\ce{CO2}}}$ là tỷ số mol của NaOH{\tt/}KOH \& \ce{CO2}:
\begin{itemize}
	\item Nếu $0 < a\le 1$: Chỉ tạo muối acid \ce{NaHCO3}{\tt/}\ce{KHCO3}, viết phản ứng (3).
	\item Nếu $1 < a < 2$: Tạo cả muối acid \ce{NaHCO3}{\tt/}\ce{KHCO3} \& muối trung hòa \ce{Na2CO3}{\tt/}\ce{K2CO3}, viết 2 phản ứng (1) \& (3).
	\item Nếu $a\ge 2$: Chỉ tạo muối trung hòa \ce{Na2CO3}{\tt/}\ce{K2CO3}, viết phản ứng (1).
\end{itemize}
\textbf{\textsf{Trường hợp \ce{CO2} tác dụng với dung dịch NaOH \& KOH.}} Nếu bài toán cho \ce{CO2} phản ứng với dung dịch chứa $x$ mol NaOH \& $y$ mol KOH thì có thể thay 2 base này bởi 1 base tương đương (hay ``base trung bình''{\tt/}``averaged base'') MOH. Khi đó, xét tỷ số $a = \frac{n_{\rm MOH}}{n_{\ce{CO2}}}$ là tỷ số mol của MOH \& \ce{CO2}:
\begin{equation}
	\label{MOH}
	\tag{MOH}
	\left\{\begin{split}
		n_{\rm MOH} &= n_{\rm NaOH} + n_{\rm KOH} = x + y,\\
		m_{\rm MOH} &= m_{\rm NaOH} + m_{\rm KOH} = 23x + 39y,
	\end{split}\right.\Rightarrow M_{\rm MOH} = \frac{m_{\rm MOH}}{n_{\rm MOH}} = \frac{23x + 39y}{x + y}.
\end{equation}

\section{\ce{SO2} $+$ NaOH{\tt/}KOH}
Trường hợp \ce{SO2} phản ứng với NaOH, KOH, hoặc dung dịch chứa cả NaOH \& KOH hoàn toàn tương tự, chỉ cần thay nguyên tố C bởi nguyên tố S trong các phương trình \& công thức trên:
\vspace{2mm}

\noindent\textbf{\textsf{Trường hợp \ce{SO2} tác dụng với dung dịch NaOH hoặc KOH.}} Thứ tự phản ứng: \ce{SO2 + $2$NaOH -> Na2SO3 + H2O} (1). Nếu NaOH hết mà vẫn sục khí \ce{SO2} vào: \ce{SO2 + Na2SO3 + H2O -> $2$NaHSO3} (2). (1) $+$ (2): \ce{SO2 + NaOH -> NaHSO3} (3). Đặt $a = \frac{n_{\rm NaOH}}{n_{\ce{SO2}}}$ là tỷ số mol của NaOH \& \ce{SO2}.
\begin{itemize}
	\item Nếu $0 < a\le 1$: Chỉ tạo muối acid \ce{NaHSO3}{\tt/}\ce{KHSO3}, viết phản ứng (3).
	\item Nếu $1 < a < 2$: Tạo cả muối acid \ce{NaHSO3}{\tt/}\ce{KHSO3} \& muối trung hòa \ce{Na2SO3}{\tt/}\ce{K2SO3}, viết 2 phản ứng (1) \& (3).
	\item Nếu $a\ge 2$: Chỉ tạo muối trung hòa \ce{Na2SO3}{\tt/}\ce{K2SO3}, viết phản ứng (1).
\end{itemize}
\textbf{\textsf{Trường hợp \ce{SO2} tác dụng với dung dịch NaOH \& KOH.}} Nếu bài toán cho \ce{SO2} phản ứng với dung dịch chứa $x$ mol NaOH \& $y$ mol KOH thì có thể thay 2 base này bởi 1 base tương đương (hay ``base trung bình''{\tt/}``averaged base'') MOH. Khi đó, xét tỷ số $a = \frac{n_{\rm MOH}}{n_{\ce{SO2}}}$ là tỷ số mol của MOH cho bởi \eqref{MOH} \& \ce{SO2}.

%------------------------------------------------------------------------------%

\section{\ce{CO2} $+$ \ce{Ba(OH)2}{\tt/}\ce{Ca(OH)2}}
\textbf{\textsf{Trường hợp \ce{CO2} tác dụng với dung dịch \ce{Ba(OH)2} hoặc \ce{Ca(OH)2}.}} Thứ tự phản ứng: \ce{CO2 + Ba(OH)2 -> BaCO3 v + H2O} hoặc \ce{CO2 + Ca(OH)2 -> CaCO3 v + H2O} (1). Nếu \ce{Ba(OH)2}{\tt/}\ce{Ca(OH)2} hết mà vẫn sục khí \ce{CO2} vào: \ce{CO2 + BaCO3 + H2O -> Ba(HCO3)2} hoặc \ce{CO2 + CaCO3 + H2O -> Ca(HCO3)2} (2). (1) $+$ (2): \ce{$2$CO2 + Ba(OH)2 -> Ba(HCO3)2} hoặc \ce{$2$CO2 + Ca(OH)2 -> Ca(HCO3)2} (3). Đặt $a = \frac{n_{\ce{Ba(OH)2}}}{n_{\ce{CO2}}}$ hoặc $a = \frac{n_{\ce{Ca(OH)2}}}{n_{\ce{CO2}}}$ là tỷ số mol của \ce{Ba(OH)2}{\tt/}\ce{Ca(OH)2} \& \ce{CO2}:
\begin{itemize}
	\item Nếu $0 < a\le\frac{1}{2}$: Chỉ tạo muối acid \ce{Ba(HCO3)2}{\tt/}\ce{Ca(HCO3)2}, viết phản ứng (3).
	\item Nếu $\frac{1}{2} < a < 1$: Tạo cả 2 muối kết tủa \ce{BaCO3}{\tt/}\ce{CaCO3} \& muối acid \ce{Ba(HCO3)2}{\tt/}\ce{Ca(HCO3)2}, viết 2 phản ứng (1) \& (3).
	\item Nếu $a\ge1$: Chỉ tạo muối kết tủa \ce{BaCO3}{\tt/}\ce{CaCO3}, viết phản ứng (1).
\end{itemize}
\textbf{\textsf{Trường hợp \ce{CO2} tác dụng với dung dịch \ce{Ba(OH)2} \& \ce{Ca(OH)2}.}} Nếu bài toán cho \ce{CO2} phản ứng với dung dịch chứa $x$ mol \ce{Ba(OH)2} \& $y$ mol \ce{Ca(OH)2} thì có thể thay 2 base này bởi 1 base tương đương (hay ``base trung bình''{\tt/}``averaged base'') \ce{M(OH)2}. Khi đó, xét tỷ số $a = \frac{n_{\rm M(OH)2}}{n_{\ce{CO2}}}$ là tỷ số mol của \ce{M(OH)2} \& \ce{CO2}:
\begin{equation}
	\label{M(OH)2}
	\tag{$\rm M(OH)_2$}
	\left\{\begin{split}
		n_{\ce{M(OH)2}} &= n_{\ce{Ba(OH)2}} + n_{\ce{Ca(OH)2}} = x + y,\\
		m_{\ce{M(OH)2}} &= m_{\ce{Ba(OH)2}} + m_{\ce{Ca(OH)2}} = 171x + 74y,
	\end{split}\right.\Rightarrow M_{\ce{M(OH)2}} = \frac{m_{\ce{M(OH)2}}}{n_{\ce{M(OH)2}}} = \frac{171x + 74y}{x + y}.
\end{equation}

\section{\ce{SO2} $+$ \ce{Ba(OH)2}{\tt/}\ce{Ca(OH)2}}
Trường hợp \ce{SO2} phản ứng với \ce{Ba(OH)2,Ca(OH)2}, hoặc dung dịch chứa cả \ce{Ba(OH)} \& \ce{Ca(OH)2} hoàn toàn tương tự, chỉ cần thay nguyên tố C bởi nguyên tố S trong các phương trình \& công thức trên:
\vspace{2mm}

\noindent\textbf{\textsf{Trường hợp \ce{SO2} tác dụng với dung dịch \ce{Ba(OH)2} hoặc \ce{Ca(OH)2}.}} Thứ tự phản ứng: \ce{SO2 + Ba(OH)2 -> BaSO3 v + H2O} hoặc \ce{SO2 + Ca(OH)2 -> CaSO3 v + H2O} (1). Nếu \ce{Ba(OH)2}{\tt/}\ce{Ca(OH)2} hết mà vẫn sục khí \ce{SO2} vào: \ce{SO2 + BaSO3 + H2O -> Ba(HSO3)2} hoặc \ce{SO2 + CaSO3 + H2O -> Ca(HSO3)2} (2). (1) $+$ (2): \ce{$2$SO2 + Ba(OH)2 -> Ba(HSO3)2} hoặc \ce{$2$SO2 + Ca(OH)2 -> Ca(HSO3)2} (3). Đặt $a = \frac{n_{\ce{Ba(OH)2}}}{n_{\ce{SO2}}}$ hoặc $a = \frac{n_{\ce{Ca(OH)2}}}{n_{\ce{SO2}}}$ là tỷ số mol của \ce{Ba(OH)2}{\tt/}\ce{Ca(OH)2} \& \ce{SO2}:
\begin{itemize}
	\item Nếu $0 < a\le\frac{1}{2}$: Chỉ tạo muối acid \ce{Ba(HSO3)2}{\tt/}\ce{Ca(HSO3)2}, viết phản ứng (3).
	\item Nếu $\frac{1}{2} < a < 1$: Tạo cả 2 muối kết tủa \ce{BaSO3}{\tt/}\ce{CaSO3} \& muối acid \ce{Ba(HSO3)2}{\tt/}\ce{Ca(HSO3)2}, viết 2 phản ứng (1) \& (3).
	\item Nếu $a\ge1$: Chỉ tạo muối kết tủa \ce{BaSO3}{\tt/}\ce{CaSO3}, viết phản ứng (1).
\end{itemize}
\textbf{\textsf{Trường hợp \ce{SO2} tác dụng với dung dịch \ce{Ba(OH)2} \& \ce{Ca(OH)2}.}} Nếu bài toán cho \ce{SO2} phản ứng với dung dịch chứa $x$ mol \ce{Ba(OH)2} \& $y$ mol \ce{Ca(OH)2} thì có thể thay 2 base này bởi 1 base tương đương (hay ``base trung bình''{\tt/}``averaged base'') \ce{M(OH)2}. Khi đó, xét tỷ số $a = \frac{n_{\rm M(OH)2}}{n_{\ce{SO2}}}$ là tỷ số mol của \ce{M(OH)2} cho bởi \eqref{M(OH)2} \& \ce{SO2}.

\section{\ce{CO2} $+$ NaOH{\tt/}KOH \& \ce{Ba(OH)2}{\tt/}\ce{Ca(OH)2}}
\textbf{\textsf{Trường hợp \ce{CO2} tác dụng với dung dịch chứa $x$ mol NaOH{\tt/}KOH \& $y$ mol \ce{Ba(OH)2}{\tt/}\ce{Ca(OH)2}.}} Có thể coi các phản ứng xảy ra theo thứ tự: \ce{CO2 + Ba(OH)2 -> BaCO3 v + H2O} hoặc \ce{CO2 + Ca(OH)2 -> CaCO3 v + H2O} (1). Nếu \ce{Ca(OH)2} mà vẫn sục khí \ce{CO2} vào thì \ce{CO2 + $2$NaOH -> Na2CO3 + H2O} hoặc \ce{CO2 + $2$KOH -> K2CO3 + H2O}K (2). Nếu NaOH{\tt/}KOH hết mà vẫn sục khí \ce{CO2} vào thì \ce{CO2 + Na2CO3 + H2O -> $2$NaHCO3} hoặc \ce{CO2 + K2CO3 + H2O -> $2$KHCO3} (3). (2) $+$ (3): \ce{CO2 + NaOH -> NaHCO3} hoặc \ce{CO2 + KOH -> KHCO3} (4). Nếu \ce{Na2CO3}{\tt/}\ce{K2CO3} hết mà \ce{CO2} còn thì kết tủa tan dần: \ce{CO2 + BaCO3 + H2O -> Ba(HCO3)2} hoặc \ce{CO2 + CaCO3 + H2O -> Ca(HCO3)2} (5). (1) $+$ (5): \ce{$2$CO2 + Ca(OH)2 -> Ca(HCO3)2} (6). Tổng hợp lại:
\begin{itemize}
	\item Nếu $n_{\ce{CO2}}\le y$: Chỉ tạo muối \ce{BaCO3}{\tt/}\ce{CaCO3}, viết phản ứng (1).
	\item Nếu $y < n_{\ce{CO2}}\le\frac{1}{2}x + y$: Tạo muối \ce{BaCO3}{\tt/}\ce{CaCO3} \& \ce{Na2CO3}{\tt/}\ce{K2CO3}, viết 2 phản ứng (1) \& (2).
	\item Nếu $\frac{1}{2}x + y < n_{\ce{CO2}} < x + y$: Tạo muối \ce{BaCO3}{\tt/}\ce{CaCO3}, \ce{Na2CO3}{\tt/}\ce{K2CO3}, \& \ce{NaHCO3}{\tt/}\ce{KHCO3}, viết 3 phản ứng (1), (2), \& (4).
	\item Nếu $x + y = n_{\ce{CO2}}$: Tạo muối \ce{BaCO3}{\tt/}\ce{CaCO3} \& \ce{NaHCO3}{\tt/}\ce{KHCO3}, viết 2 phản ứng (1) \& (4).
	\item Nếu $x + y < n_{\ce{CO2}} < x + 2y$: Tạo muối \ce{BaCO3}{\tt/}\ce{CaCO3}, \ce{Ba(HCO3)2}{\tt/}\ce{Ca(HCO3)2}, \& \ce{NaHCO3}{\tt/}\ce{KHCO3}, viết 3 phản ứng (1), (4), \& (6).
	\item Nếu $x + 2y\le n_{\ce{CO2}}$: Tạo muối \ce{Ba(HCO3)2}{\tt/}\ce{Ca(HCO3)2} \& \ce{NaHCO3}{\tt/}\ce{KHCO3}, viết 2 phản ứng (4) \& (6).
\end{itemize}

\section{\ce{SO2} $+$ NaOH{\tt/}KOH \& \ce{Ba(OH)2}{\tt/}\ce{Ca(OH)2}}
\textbf{\textsf{Trường hợp \ce{SO2} tác dụng với dung dịch chứa $x$ mol NaOH{\tt/}KOH \& $y$ mol \ce{Ba(OH)2}{\tt/}\ce{Ca(OH)2}.}} Có thể coi các phản ứng xảy ra theo thứ tự: \ce{SO2 + Ba(OH)2 -> BaSO3 v + H2O} hoặc \ce{SO2 + Ca(OH)2 -> CaSO3 v + H2O} (1). Nếu \ce{Ca(OH)2} mà vẫn sục khí \ce{SO2} vào thì \ce{SO2 + $2$NaOH -> Na2SO3 + H2O} hoặc \ce{SO2 + $2$KOH -> K2SO3 + H2O}K (2). Nếu NaOH{\tt/}KOH hết mà vẫn sục khí \ce{SO2} vào thì \ce{SO2 + Na2SO3 + H2O -> $2$NaHSO3} hoặc \ce{SO2 + K2SO3 + H2O -> $2$KHSO3} (3). (2) $+$ (3): \ce{SO2 + NaOH -> NaHSO3} hoặc \ce{SO2 + KOH -> KHSO3} (4). Nếu \ce{Na2SO3}{\tt/}\ce{K2SO3} hết mà \ce{SO2} còn thì kết tủa tan dần: \ce{SO2 + BaSO3 + H2O -> Ba(HSO3)2} hoặc \ce{SO2 + CaSO3 + H2O -> Ca(HSO3)2} (5). (1) $+$ (5): \ce{$2$SO2 + Ca(OH)2 -> Ca(HSO3)2} (6). Tổng hợp lại:
\begin{itemize}
	\item Nếu $n_{\ce{SO2}}\le y$: Chỉ tạo muối \ce{BaSO3}{\tt/}\ce{CaSO3}, viết phản ứng (1).
	\item Nếu $y < n_{\ce{SO2}}\le\frac{1}{2}x + y$: Tạo muối \ce{BaSO3}{\tt/}\ce{CaSO3} \& \ce{Na2SO3}{\tt/}\ce{K2SO3}, viết 2 phản ứng (1) \& (2).
	\item Nếu $\frac{1}{2}x + y < n_{\ce{SO2}} < x + y$: Tạo muối \ce{BaSO3}{\tt/}\ce{CaSO3}, \ce{Na2SO3}{\tt/}\ce{K2SO3}, \& \ce{NaHSO3}{\tt/}\ce{KHSO3}, viết 3 phản ứng (1), (2), \& (4).
	\item Nếu $x + y = n_{\ce{SO2}}$: Tạo muối \ce{BaSO3}{\tt/}\ce{CaSO3} \& \ce{NaHSO3}{\tt/}\ce{KHSO3}, viết 2 phản ứng (1) \& (4).
	\item Nếu $x + y < n_{\ce{SO2}} < x + 2y$: Tạo muối \ce{BaSO3}{\tt/}\ce{CaSO3}, \ce{Ba(HSO3)2}{\tt/}\ce{Ca(HSO3)2}, \& \ce{NaHSO3}{\tt/}\ce{KHSO3}, viết 3 phản ứng (1), (4), \& (6).
	\item Nếu $x + 2y\le n_{\ce{SO2}}$: Tạo muối \ce{Ba(HSO3)2}{\tt/}\ce{Ca(HSO3)2} \& \ce{NaHSO3}{\tt/}\ce{KHSO3}, viết 2 phản ứng (4) \& (6).
\end{itemize}

%------------------------------------------------------------------------------%

\section{\ce{P2O5} $+$ NaOH{\tt/}KOH}
\noindent\textbf{\textsf{Trường hợp \ce{P2O5} tác dụng với dung dịch NaOH hoặc KOH.}} Khi anhiđrit photphoric (hay phosphor pentoxide hoặc diphosphor pentoxide) \ce{P2O5} tác dụng với kiềm, tùy theo tỷ lệ mol giữa \ce{P2O5} \& dung dịch NaOH{\tt/}KOH cho 3 loại muối: Thứ tự các phản ứng: \ce{P2O5 + $3$H2O -> $2$H3PO4}, \ce{H3PO4 + NaOH -> NaH2PO4 + H2O} hoặc \ce{H3PO4 + KOH -> KH2PO4 + H2O}. Nếu NaOH{\tt/}KOH còn, \ce{H3PO4} hết: \ce{NaOH + NaH2PO4 -> Na2HPO4 + H2O} hoặc \ce{KOH + KH2PO4 -> K2HPO4 + H2O}. Nếu NaOH{\tt/}KOH còn, \ce{Na2HPO4} hết: \ce{NaOH + Na2HPO4 -> Na3PO4 + H2O}. Đặt $a = \frac{n_{\rm NaOH}}{n_{\ce{P2O5}}}$ hoặc $a = \frac{n_{\rm KOH}}{n_{\ce{P2O5}}}$ là tỷ số mol của NaOH{\tt/}KOH \& \ce{P2O5}:
\begin{itemize}
	\item Nếu $0 < a\le2$: Chỉ tạo muối \ce{NaH2PO4}{\tt/}\ce{KH2PO4}.
	\item Nếu $2 < a < 4$: Tạo 2 muối acid \ce{NaH2PO4}{\tt/}\ce{KH2PO4} \& \ce{Na2HPO4}{\tt/}\ce{K2HPO4}.
	\item Nếu $a = 4$: Chỉ tạo muối acid \ce{Na2HPO4}{\tt/}\ce{K2HPO4}.
	\item Nếu $4 < a < 6$: Tạo muối acid \ce{Na2HPO4}{\tt/}\ce{K2HPO4} \& muối trung hòa \ce{Na3PO4}{\tt/}\ce{K3PO4}.
	\item Nếu $a\ge6$: Chỉ tạo muối trung hòa \ce{Na3PO4}{\tt/}\ce{K3PO4}.
\end{itemize}
Có thể ghi gom lại các {\rm PTHH} giữa \ce{H3PO4,NaOH} thành trực tiếp giữa \ce{P2O5,NaOH} như sau (see, e.g., \cite[pp. 204--205]{An_chuoi_PUHH}): \ce{P2O5 + H2O + $2$NaOH -> $2$NaH2PO4} hoặc \ce{P2O5 + H2O + $2$KOH -> $2$KH2PO4}, \ce{P2O5 + $4$NaOH -> $2$Na2HPO4 + H2O} hoặc \ce{P2O5 + $4$KOH -> $2$K2HPO4 + H2O}, \ce{P2O5 + $6$NaOH -> $2$Na3PO4 + $3$H2O} hoặc \ce{P2O5 + $6$KOH -> $2$K3PO4 + $3$H2O}.

\textbf{Muối phosphate.} Tính chất: Acid phosphoric \ce{H3PO4} cho 3 loại muối: 1 muối trung hòa \& 2 muối acid (hiđrophotphat \& đihiđrophotphat). Các muối trung hòa \& muối acid của các kim loại Na, K \& ion \ce{NH4+} là tan. Các kim loại khác chỉ có muối đihiđrophotphat là tan được, ngoài ra đều không tan hoặc ít tan trong nước (see, e.g., \cite[p. 206]{An_chuoi_PUHH}).
\vspace{2mm}

\noindent\textbf{\textsf{Trường hợp \ce{P2O5} tác dụng với dung dịch NaOH \& KOH.}} Nếu bài toán cho \ce{P2O5} phản ứng với dung dịch chứa $x$ mol NaOH \& $y$ mol KOH thì có thể thay 2 base này bởi 1 base tương đương (hay ``base trung bình''{\tt/}``averaged base'') MOH. Khi đó, xét tỷ số $a = \frac{n_{\rm MOH}}{n_{\ce{P2O5}}}$ là tỷ số mol của MOH cho bởi \eqref{MOH} \& \ce{P2O5}.

%------------------------------------------------------------------------------%

\section{\ce{P2O5} $+$ \ce{Ba(OH)2}{\tt/}\ce{Ca(OH)2}}
\noindent\textbf{\textsf{Trường hợp \ce{P2O5} tác dụng với dung dịch \ce{Ba(OH)2} hoặc \ce{Ca(OH)2}}} Khi anhiđrit photphoric (hay phosphor pentoxide hoặc diphosphor pentoxide) \ce{P2O5} tác dụng với kiềm, tùy theo tỷ lệ mol giữa \ce{P2O5} \& dung dịch \ce{Ba(OH)2}{\tt/}\ce{Ca(OH)2} cho 3 loại muối: Thứ tự các phản ứng: \ce{P2O5 + $3$H2O -> $2$H3PO4}, \ce{$2$H3PO4 + Ba(OH)2 -> Ba(H2PO4)2 + $2$H2O} hoặc \ce{$2$H3PO4 + Ca(OH)2 -> Ca(H2PO4)2 + $2$H2O}. Nếu \ce{Ba(OH)2}{\tt/}\ce{Ca(OH)2} còn, \ce{H3PO4} hết: \ce{Ba(OH)2 + Ba(H2PO4)2 -> $2$BaHPO4 v + $2$H2O} hoặc \ce{Ca(OH)2 + Ca(H2PO4)2 -> $2$CaHPO4 v + $2$H2O}. Nếu \ce{Ba(OH)2}{\tt/}\ce{Ca(OH)2} còn, \ce{BaHPO4} hết: \ce{Ba(OH)2 + $2$BaHPO4 -> Ba3(PO4)2 v + $2$H2O} hoặc \ce{Ca(OH)2 + $2$CaHPO4 -> Ca3(PO4)2 v + $2$H2O}. Đặt $a = \frac{n_{\ce{Ba(OH)2}}}{n_{\ce{P2O5}}}$ hoặc $a = \frac{n_{\ce{Ca(OH)2}}}{n_{\ce{P2O5}}}$ là tỷ số mol của \ce{Ba(OH)2}{\tt/}\ce{Ca(OH)2} \& \ce{P2O5}:
\begin{itemize}
	\item Nếu $0 < a\le1$: Chỉ tạo muối \ce{Ba(H2PO4)2}{\tt/}\ce{Ca(H2PO4)2}.
	\item Nếu $1 < a < 2$: Tạo 2 muối acid \ce{Ba(H2PO4)2}{\tt/}\ce{Ca(H2PO4)2} \& \ce{BaHPO4}{\tt/}\ce{CaHPO4}.
	\item Nếu $a = 2$: Chỉ tạo muối acid \ce{BaHPO4}{\tt/}\ce{CaHPO4}.
	\item Nếu $2 < a < 3$: Tạo muối acid \ce{BaHPO4}{\tt/}\ce{CaHPO4} \& muối trung hòa \ce{Ba3(PO4)2}{\tt/}\ce{Ca3(PO4)2}.
	\item Nếu $a\ge6$: Chỉ tạo muối trung hòa \ce{Ba3(PO4)2}{\tt/}\ce{Ca3(PO4)2}.
\end{itemize}
Có thể ghi gom lại các {\rm PTHH} giữa \ce{H3PO4,NaOH} thành trực tiếp giữa \ce{P2O5,NaOH} như sau (see, e.g., \cite[pp. 204--205]{An_chuoi_PUHH}): \ce{P2O5 + H2O + $2$NaOH -> $2$NaH2PO4} hoặc \ce{P2O5 + H2O + $2$KOH -> $2$KH2PO4}, \ce{P2O5 + $4$NaOH -> $2$Na2HPO4 + H2O} hoặc \ce{P2O5 + $4$KOH -> $2$K2HPO4 + H2O}, \ce{P2O5 + $6$NaOH -> $2$Na3PO4 + $3$H2O} hoặc \ce{P2O5 + $6$KOH -> $2$K3PO4 + $3$H2O}.
\begin{luuy}
	Các muối {\rm\ce{Ba3(PO4)2,BaHPO4,Ca3(PO4)2,CaHPO4}} không tan.
\end{luuy}
\noindent\textbf{\textsf{Trường hợp \ce{P2O5} tác dụng với dung dịch \ce{Ba(OH)2} \& \ce{Ca(OH)2}.}} Nếu bài toán cho \ce{P2O5} phản ứng với dung dịch chứa $x$ mol \ce{Ba(OH)2} \& $y$ mol \ce{Ca(OH)2} thì có thể thay 2 base này bởi 1 base tương đương (hay ``base trung bình''{\tt/}``averaged base'') \ce{M(OH)2}. Khi đó, xét tỷ số $a = \frac{n_{\rm M(OH)2}}{n_{\ce{P2O5}}}$ là tỷ số mol của \ce{M(OH)2} cho bởi \eqref{M(OH)2} \& \ce{P2O5}.

%------------------------------------------------------------------------------%

\section{Problem}

\begin{baitoan}[\cite{An_400_BT_Hoa_Hoc_9}, 9.b, p. 13]
	Cho khí {\rm\ce{CO2}} (đktc) phản ứng với {\rm80 g} dung dịch {\rm NaOH 25\%} để tạo thành hỗn hợp muối acid \& muối trung hòa theo tỷ lệ số mol là $2:3$. Tính thể tích {\rm\ce{CO2}} cần dùng.\hfill{\sf Ans: $7$ L.}
\end{baitoan}

\begin{baitoan}[\cite{Nguyen_Buu_Can_500_BT_Hoa_Hoc_THCS}, 239., p. 102]
	Dẫn khí {\rm\ce{CO2}} vào {\rm1.2 L} dung dịch {\rm\ce{Ca(OH)2} 0.1M} thấy tạo ra {\rm5 g} 1 muối không tan cùng với 1 muối tan. (a) Tính thể tích khí {\rm\ce{CO2}} đã dùng (đktc). (b) Tính khối lượng \& nồng độ mol của muối tan. (c) Tính thể tích {\rm\ce{CO2}} (đktc) trong trường hợp chỉ tạo ra muối không tan. Tính khối lượng muối không tan đó.\hfill{\sf Ans: $7$ L.}
\end{baitoan}

\begin{baitoan}[\cite{Nguyen_Buu_Can_500_BT_Hoa_Hoc_THCS}, 260., p. 105]
	Cho {\rm5.6 L} khí {\rm\ce{CO2}} lội qua dung dịch {\rm NaOH 20\%}, $D = 1.22$ {\rm g{\tt/}mol}. (a) Tính khối lượng muối tạo thành. (b) Tính nồng độ $\%$ các chất có trong dung dịch sau phản ứng.\hfill{\sf Ans: (a) $4.256$ L. (b) $0.058$M, $11.34$ g. (c) $12$ g.}
\end{baitoan}

\begin{baitoan}[\cite{An_350_BT_Hoa_Hoc_9}, 6.a, p. 7]
	Cho \emph{2.24 L \ce{CO2}} (đktc) tác dụng hoàn toàn với \emph{25 g} dung dịch \emph{NaOH 20\%}. Tính khối lượng muối tạo thành.\hfill{\sf Ans: $6.3$ g \ce{NaHCO3}, $2.65$ g \ce{Na2CO3}.}
\end{baitoan}

\begin{baitoan}[\cite{An_350_BT_Hoa_Hoc_9}, 9.a, p. 11]
	Dẫn $V$ \emph{L} khí \emph{\ce{CO2}} (đktc) qua \emph{250 mL} dung dịch \emph{\ce{Ca(OH)2} 1M} thấy có \emph{12.5 g} kết tủa. Tính $V$.\hfill{\sf Ans: $V\in\{2.8,8.4\}$.}
\end{baitoan}

\begin{baitoan}[\cite{An_350_BT_Hoa_Hoc_9}, 12.a, p. 15]
	Cho \emph{7.84 g CaO} tan hoàn toàn vào nước được dung dịch A. Dẫn \emph{2.24 L} khí \emph{\ce{CO2}} (đktc) vào dung dịch A. Tính khối lượng các chất sau phản ứng.\hfill{\sf Ans: $10$ g \ce{CaCO3}, $2.96$ g \ce{Ca(OH)2} dư.}
\end{baitoan}

\begin{baitoan}[\cite{Truong_BTNC_Hoa_Hoc_9_2021}, 5.6., p. 15]
	Dùng 1 dung dịch chứa {\rm20 g NaOH} để hấp thụ hoàn toàn {\rm22 g \ce{CO2}}. Muối nào được tạo thành \& với khối lượng bao nhiêu?\hfill{\sf Ans: $42$ g \ce{NaHCO3}.}
\end{baitoan}

\begin{baitoan}[\cite{Truong_BTNC_Hoa_Hoc_9_2021}, 5.7., p. 15]
	Cho {\rm4.48 L \ce{CO2}} (đktc) tác dụng hoàn toàn với {\rm50 g} dung dịch {\rm NaOH 20\%}. Tính khối lượng muối tạo ra trong dung dịch.\hfill{\sf Ans: $12.6$ g \ce{NaHCO3}, $5.3$ g \ce{Na2CO3}.}
\end{baitoan}

\begin{baitoan}[\cite{Truong_BTNC_Hoa_Hoc_9_2021}, 5.8., p. 15]
	Cho dung dịch {\rm NaOH 25\%} có khối lượng riêng $D = 1.28$ {\rm g{\tt/}mL}. Hỏi {\rm150 mL} dung dịch kiềm đó có khả năng hấp thụ được tối đa bao nhiêu {\rm L \ce{CO2}} ở đktc?\hfill{\sf Ans: $26.88$ L.}
\end{baitoan}

\begin{baitoan}[\cite{Truong_BTNC_Hoa_Hoc_9_2021}, 5.9., p. 15]
	Cho {\rm0.1 mol \ce{CO2}} hấp thụ vào {\rm400 mL} dung dịch {\rm NaOH $a$\%}, $D = 1.18$ {\rm g{\tt/}mL}, sau đó thêm lượng dư {\rm\ce{BaCl2}} vào thấy tạo thành {\rm18.715 g} kết tủa. Tính $a$.\hfill{\sf Ans: $1.6525$.}
\end{baitoan}

\begin{baitoan}[\cite{Truong_BTNC_Hoa_Hoc_9_2021}, 5.10., p. 15]
	Cho {\rm1.12 L \ce{CO2}} (đktc) tác dụng vừa đủ với {\rm100 mL} dung dịch {\rm NaOH} tạo ra muối trung hòa. (a) Viết {\rm PTHH}. (b) Tính nồng độ mol của dung dịch {\rm NaOH}. (c) Tính nồng độ $\%$ của dung dịch muối sau phản ứng. Biết dung dịch sau phản ứng có khối lượng là {\rm105 g}.\hfill{\sf Ans: (b) $1$M. (c) $5.04\%$.}
\end{baitoan}

\begin{baitoan}[\cite{Truong_BTNC_Hoa_Hoc_9_2021}, 5.11., p. 15]
	Biết {\rm2.24 L \ce{CO2}} (đktc) tác dụng vừa đủ với {\rm200 mL} dung dịch {\rm\ce{Ba(OH)2}} sinh ra chất kết tủa màu trắng. (a) Viết {\rm PTHH}. (b) Tính nồng độ mol của dung dịch {\rm\ce{Ba(OH)2}}. (c) Tính khối lượng chất kết tủa thu được.\hfill{\sf Ans: (b) $0.5$M. (c) $19.7$ g.}
\end{baitoan}

\begin{baitoan}[\cite{Truong_BTNC_Hoa_Hoc_9_2021}, 8.4., p. 19]
	Cho {\rm6.2 g \ce{Na2O}} tan vào nước. Tính thể tích khí {\rm\ce{SO2}} (đktc) cần thiết sục vào dung dịch trên để thu được: (a) Muối trung hòa. (b) Muối acid. (c) Hỗn hợp muối acid \& muối trung hòa có tỷ lệ số mol là $2:1$. (d) Hỗn hợp muối acid \& muối trung hòa có tỷ lệ số mol là $a:b$ với $a,b\in\mathbb{R}$, $a,b > 0$.\hfill{\sf Ans: (a) $2.24$ L. (b) $4.48$ L. (c) $3.36$ L.}
\end{baitoan}

\begin{baitoan}[\cite{Truong_BTNC_Hoa_Hoc_9_2021}, 8.6., p. 19]
	Cho {\rm1.568 L \ce{CO2}} (đktc) lội chậm qua dung dịch chứa {\rm3.2 g NaOH}. Xác định thành phần định tính \& định lượng chất được sinh ra sau phản ứng.\hfill{\sf Ans: $5.04$ g \ce{NaHCO3}, $1.06$ g \ce{Na2CO3}.}
\end{baitoan}

\begin{baitoan}[\cite{Truong_BTNC_Hoa_Hoc_9_2021}, 8.7., p. 19]
	Dẫn khí {\rm\ce{CO2}} điều chế được bằng cách cho {\rm100 g \ce{CaCO3}} tác dụng với dung dịch {\rm HCl} dư, đi qua dung dịch có chứa {\rm60 g NaOH}. Tính khối lượng muối sodium điều chế được.\hfill{\sf Ans: $42$ g \ce{NaHCO3}, $53$ g \ce{Na2CO3}.}
\end{baitoan}

\begin{baitoan}[\cite{Truong_Long_Huong_bdhsg_Hoa_Hoc_9}, Ví dụ 1, p. 44]
	Hấp thụ hoàn toàn {\rm7.84 L} (đktc) khí {\rm\ce{CO2}} vào {\rm200 mL} dung dịch {\rm KOH 1.5M} \& {\rm\ce{K2CO3} 1M}. Sau khi các phản ứng xảy ra hoàn toàn thu được dung dịch X. Tính khối lượng mỗi muối có trong dung dịch X.\\\mbox{}\hfill{\sf Ans: 20.7 g \ce{K2CO3} \& 40 g \ce{KHCO3}.}
\end{baitoan}

\begin{baitoan}[\cite{Truong_Long_Huong_bdhsg_Hoa_Hoc_9}, Ví dụ 2, p. 44]
	Hấp thụ hoàn toàn {\rm0.4 mol} khí {\rm\ce{CO2}} vào dung dịch chứa {\rm0.15 mol \ce{Ca(OH)2}} \& {\rm0.2 mol KOH}. Sau khi các phản ứng xảy ra hoàn toàn, thu được $m$ {\rm g} kết tủa. Tính $m$.\hfill{\sf Ans: $m = 10$}
\end{baitoan}

\begin{baitoan}[\cite{Truong_Long_Huong_bdhsg_Hoa_Hoc_9}, Ví dụ 3, p. 45]
	Hấp thụ hoàn toàn {\rm4.48 L \ce{CO2}} (đktc) vào {\rm200 mL} dung dịch X gồm {\rm\ce{Na2CO3} 0.3M \& NaOH $x$M}, sau khi các phản ứng xảy ra hoàn toàn thu được dung dịch Y. Cho toàn bộ Y tác dụng với dung dịch {\rm\ce{CaCl2}} (dư), thu được {\rm10 g} kết tủa. Tính $x$.\hfill{\sf Ans: $x = 1.2$}
\end{baitoan}

\begin{baitoan}[\cite{Truong_Long_Huong_bdhsg_Hoa_Hoc_9}, Ví dụ 4, p. 45]
	Hấp thụ hết {\rm6.72 L \ce{CO2}} (đktc) vào {\rm200 mL} dung dịch chứa {\rm KOH 1M} \& {\rm NaOH $x$M}. Sau khi các phản ứng xảy ra hoàn toàn, làm khô dung dịch thu được {\rm32.8 g} chất rắn khan. Giả sử trong quá trình làm khô dung dịch không xảy ra các {\rm PƯHH}. Tính $x$.\hfill{\sf Ans: $x = 1.5$}
\end{baitoan}

\begin{baitoan}[\cite{Truong_Long_Huong_bdhsg_Hoa_Hoc_9}, Ví dụ 5, p. 46]
	Cho {\rm28.4 g \ce{P2O5}} vào {\rm750 mL} dung dịch {\rm NaOH 1.5M}. Sau khi các phản ứng xảy ra hoàn toàn, thu được dung dịch chứa $m$ {\rm g} muối. Tìm $m$.\hfill{\sf Ans: $m = 63.95$}
\end{baitoan}

\begin{baitoan}[\cite{Truong_Long_Huong_bdhsg_Hoa_Hoc_9}, Ví dụ 6, p. 47, TS THPT Chuyên Phan Bội Châu, Nghệ An]
	Cho $m$ {\rm g \ce{P2O5}} vào {\rm19.6 g} dung dịch {\rm\ce{H3PO4} 5\%} thu được dung dịch X. Cho dung dịch X phản ứng hết với {\rm100 mL} dung dịch {\rm KOH 1M} thu được dung dịch Y. Cô cạn dung dịch Y thu được {\rm6.48 g} chất rắn khan. (a) Viết {\rm PTHH}. (b) Tính khối lượng các chất có trong {\rm6.48 g} chất rắn \& giá trị $m$.\hfill{\sf Ans: $m = 0.71$}
\end{baitoan}

\begin{notation}
	Denote by $\mathbb{R}_{\ge0}\coloneqq\{x\in\mathbb{R}|x\ge0\}$ the set of all nonnegative real numbers.
\end{notation}

\begin{baitoan}
	Cho $V$ {\rm L \ce{CO2}} (đktc) phản ứng với $m$ {\rm g} dung dịch chứa {\rm NaOH $a\%$, KOH $b\%$, \ce{Ba(OH)2} $c\%$, \& \ce{Ca(OH)2} $d\%$}. Biện luận theo 4 tham số thực không âm $a,b,c,d\in\mathbb{R}_{\ge0}$ (các tham số nào bằng $0$ tức là vắng mặt các chất tương ứng) để viết {\rm PTHH} \& tính khối lượng từng ion \& từng muối tạo thành trong tất cả các trường hợp có thể. Tính nồng độ $\%$ của từng ion \& từng dung dịch muối đó sau khi loại bỏ tất cả các kết tủa nếu có.
\end{baitoan}

\begin{baitoan}
	Cho $V_1$ {\rm L \ce{CO2}} phản ứng với $V_2$ {\rm L} dung dịch chứa {\rm NaOH $a$M, KOH $b$M, \ce{Ba(OH)2} $c$M, \& \ce{Ca(OH)2} $d$M}. Biện luận theo 4 tham số thực không âm $a,b,c,d\in\mathbb{R}_{\ge0}$ (các tham số nào bằng $0$ tức là vắng mặt các chất tương ứng) để viết {\rm PTHH} \& tính khối lượng từng ion \& từng muối tạo thành trong tất cả các trường hợp có thể. Tính nồng độ mol của từng ion \& từng dung dịch muối đó sau khi loại bỏ tất cả các kết tủa nếu có biết khối lượng riêng của nước $D_{\ce{H2O}} = 999.9720$ {\rm kg{\tt/}$\rm m^3$}.
\end{baitoan}

\begin{baitoan}
	Cho $V$ {\rm L \ce{SO2}} (đktc) phản ứng với $m$ {\rm g} dung dịch chứa {\rm NaOH $a\%$, KOH $b\%$, \ce{Ba(OH)2} $c\%$, \& \ce{Ca(OH)2} $d\%$}. Biện luận theo 4 tham số thực không âm $a,b,c,d\in\mathbb{R}_{\ge0}$ (các tham số nào bằng $0$ tức là vắng mặt các chất tương ứng) để viết {\rm PTHH} \& tính khối lượng từng ion \& từng muối tạo thành trong tất cả các trường hợp có thể. Tính nồng độ $\%$ của từng ion \& từng dung dịch muối đó sau khi loại bỏ tất cả các kết tủa nếu có.
\end{baitoan}

\begin{baitoan}
	Cho $V_1$ {\rm L \ce{SO2}} phản ứng với $V_2$ {\rm L} dung dịch chứa {\rm NaOH $a$M, KOH $b$M, \ce{Ba(OH)2} $c$M, \& \ce{Ca(OH)2} $d$M}. Biện luận theo 4 tham số thực không âm $a,b,c,d\in\mathbb{R}_{\ge0}$ (các tham số nào bằng $0$ tức là vắng mặt các chất tương ứng) để viết {\rm PTHH} \& tính khối lượng từng ion \& từng muối tạo thành trong tất cả các trường hợp có thể. Tính nồng độ mol của từng ion \& từng dung dịch muối đó sau khi loại bỏ tất cả các kết tủa nếu có biết khối lượng riêng của nước $D_{\ce{H2O}} = 999.9720$ {\rm kg{\tt/}$\rm m^3$}.
\end{baitoan}

\begin{baitoan}
	Cho $V$ {\rm L \ce{P2O5}} (đktc) phản ứng với $m$ {\rm g} dung dịch chứa {\rm NaOH $a\%$, KOH $b\%$, \ce{Ba(OH)2} $c\%$, \& \ce{Ca(OH)2} $d\%$}. Biện luận theo 4 tham số thực không âm $a,b,c,d\in\mathbb{R}_{\ge0}$ (các tham số nào bằng $0$ tức là vắng mặt các chất tương ứng) để viết {\rm PTHH} \& tính khối lượng từng ion \& từng muối tạo thành trong tất cả các trường hợp có thể. Tính nồng độ $\%$ của từng ion \& từng dung dịch muối đó sau khi loại bỏ tất cả các kết tủa nếu có.
\end{baitoan}

\begin{baitoan}
	Cho $V_1$ {\rm L \ce{P2O5}} phản ứng với $V_2$ {\rm L} dung dịch chứa {\rm NaOH $a$M, KOH $b$M, \ce{Ba(OH)2} $c$M, \& \ce{Ca(OH)2} $d$M}. Biện luận theo 4 tham số thực không âm $a,b,c,d\in\mathbb{R}_{\ge0}$ (các tham số nào bằng $0$ tức là vắng mặt các chất tương ứng) để viết {\rm PTHH} \& tính khối lượng từng ion \& từng muối tạo thành trong tất cả các trường hợp có thể. Tính nồng độ mol của từng ion \& từng dung dịch muối đó sau khi loại bỏ tất cả các kết tủa nếu có biết khối lượng riêng của nước $D_{\ce{H2O}} = 999.9720$ {\rm kg{\tt/}$\rm m^3$}.
\end{baitoan}

\begin{baitoan}
	Viết chương trình {\sf Pascal, Python, C{\tt/}C++} để giải bài toán oxide acid phản ứng với dung dịch base:
	\begin{itemize}
		\item {\sf Input.} Line 1: Số test $t\in\mathbb{N}^\star$. Line 2: Tên của oxide acid \& (các) base. Line 3: Số mol hoặc thể tích oxide acid (đktc) hoặc khối lượng oxide acid đó, khối lượng hoặc thể tích dung dịch dung dịch, 4 số thực không âm $a,b,c,d\in\mathbb{R}_{\ge0}$ tương ứng 4  nồng độ $\%$ nếu khối lượng của dung dịch vừa được cho hoặc 4 nồng độ mol nếu thể tích của dung dịch vừa được cho.
		\item {\sf Output.} Các {\rm PTHH}, khối lượng từng chất trong dung dịch sau phản ứng, khối lượng từng kết tủa, nồng độ $\%$ từng chất tan sau khi đã loại bỏ kết tủa nếu Input nhập vào khối lượng dung dịch, nồng độ mol từng chất tan sau khi đã loại bỏ kết tủa nếu Input nhập vào thể tích dung dịch.
		\item {\sf Sample.}
		\begin{table}[H]
			\centering
			\begin{tabular}{|l|l|}
				\hline
				\verb|oxide_acid_base.inp| & \verb|oxide_acid_base.out| \\
				\hline
				4 &  \\
				{\tt CO2 NaOH} &  \\
				{\tt n = 0.1mol V = 200mL 1.5} &  \\
				{\tt CO2 NaOH KOH} &  \\
				{\tt n = 0.1mol V = 200mL 1.5 2.2} &  \\
				{\tt SO2 KOH Ba(OH)2 Ca(OH)2} &  \\
				{\tt m = 4.4g V = 200mL 1.5 2.2 3.75} &  \\
				{\tt P2O5 NaOH KOH Ba(OH)2 Ca(OH)2}&  \\
				{\tt V = 336mL m = 200g 1.25 2.5 3.75 4}&  \\
				\hline
			\end{tabular}
		\end{table}		
	\end{itemize}
\end{baitoan}

%------------------------------------------------------------------------------%

\printbibliography[heading=bibintoc]

\end{document}