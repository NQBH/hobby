\documentclass{article}
\usepackage[backend=biber,natbib=true,style=alphabetic,maxbibnames=50]{biblatex}
\addbibresource{/home/nqbh/reference/bib.bib}
\usepackage[utf8]{vietnam}
\usepackage{tocloft}
\renewcommand{\cftsecleader}{\cftdotfill{\cftdotsep}}
\usepackage[colorlinks=true,linkcolor=blue,urlcolor=red,citecolor=magenta]{hyperref}
\usepackage{amsmath,amssymb,amsthm,float,graphicx,mathtools,diagbox,tikz,tipa}
\usepackage[version=4]{mhchem}
\allowdisplaybreaks
\newtheorem{assumption}{Assumption}
\newtheorem{baitoan}{}
\newtheorem{cauhoi}{Câu hỏi}
\newtheorem{conjecture}{Conjecture}
\newtheorem{corollary}{Corollary}
\newtheorem{dangtoan}{Dạng toán}
\newtheorem{definition}{Definition}
\newtheorem{dinhly}{Định lý}
\newtheorem{dinhnghia}{Định nghĩa}
\newtheorem{example}{Example}
\newtheorem{ghichu}{Ghi chú}
\newtheorem{hequa}{Hệ quả}
\newtheorem{hypothesis}{Hypothesis}
\newtheorem{lemma}{Lemma}
\newtheorem{luuy}{Lưu ý}
\newtheorem{nhanxet}{Nhận xét}
\newtheorem{notation}{Notation}
\newtheorem{note}{Note}
\newtheorem{principle}{Principle}
\newtheorem{problem}{Problem}
\newtheorem{proposition}{Proposition}
\newtheorem{question}{Question}
\newtheorem{remark}{Remark}
\newtheorem{theorem}{Theorem}
\newtheorem{thinghiem}{Thí nghiệm}
\newtheorem{vidu}{Ví dụ}
\usepackage[left=1cm,right=1cm,top=5mm,bottom=5mm,footskip=4mm]{geometry}

\title{Problem: Oxide Acid $+$ Base Reaction\\Bài Tập: Oxide Acid Tác Dụng với Dung Dịch Kiềm}
\date{}

\begin{document}
\maketitle
\vspace{-2cm}

%------------------------------------------------------------------------------%

\section{\ce{CO2,SO2} $+$ NaOH, KOH}
\textit{Trường hợp \ce{CO2} tác dụng với dung dịch NaOH hoặc KOH.} Thứ tự phản ứng: \ce{CO2 + $2$NaOH -> Na2CO3 + H2O} (1). Nếu NaOH hết mà vẫn sục khí \ce{CO2} vào: \ce{CO2 + Na2CO3 + H2O -> $2$NaHCO3} (2). (1) $+$ (2): \ce{CO2 + NaOH -> NaHCO3} (3). Đặt $a = \frac{n_{\rm NaOH}}{n_{\ce{CO2}}}$ là tỷ số mol của NaOH \& \ce{CO2}.
\begin{itemize}
	\item Nếu $0 < a\le 1$: Chỉ tạo muối acid \ce{NaHCO3}, viết phản ứng (3).
	\item Nếu $1 < a < 2$: Tạo cả muối acid \ce{NaHCO3} \& muối trung hòa \ce{Na2CO3}, viết 2 phản ứng (1) \& (3).
	\item Nếu $a\ge 2$: Chỉ tạo muối trung hòa \ce{Na2CO3}, viết phản ứng (1).
\end{itemize}
Nếu bài toán cho \ce{CO2} phản ứng với dung dịch chứa $x$ mol NaOH \& $y$ mol KOH thì có thể thay 2 base này bởi 1 base tương đương (hay ``base trung bình''{\tt/}``averaged base'') MOH. Khi đó, xét tỷ số $a = \frac{n_{\rm MOH}}{n_{\ce{CO2}}}$ là tỷ số mol của MOH \& \ce{CO2}:
\begin{equation}
	\label{MOH}
	\tag{MOH}
	\left\{\begin{split}
		n_{\rm MOH} &= n_{\rm NaOH} + n_{\rm KOH} = x + y,\\
		m_{\rm MOH} &= m_{\rm NaOH} + m_{\rm KOH} = 23x + 39y,
	\end{split}\right.\Rightarrow M = \frac{m_{\rm KOH}}{n_{\rm KOH}} = \frac{23x + 39y}{x + y}.
\end{equation}
Trường hợp \ce{SO2} phản ứng với NaOH, KOH, hoặc dung dịch chứa cả NaOH \& KOH hoàn toàn tương tự (chỉ cần thay nguyên tố C bởi nguyên tố S trong các phương trình \& công thức trên):

\noindent\textit{Trường hợp \ce{SO2} tác dụng với dung dịch NaOH hoặc KOH.} Thứ tự phản ứng: \ce{SO2 + $2$NaOH -> Na2SO3 + H2O} (1). Nếu NaOH hết mà vẫn sục khí \ce{SO2} vào: \ce{SO2 + Na2SO3 + H2O -> $2$NaHSO3} (2). (1) $+$ (2): \ce{SO2 + NaOH -> NaHSO3} (3). Đặt $a = \frac{n_{\rm NaOH}}{n_{\ce{SO2}}}$ là tỷ số mol của NaOH \& \ce{SO2}.
\begin{itemize}
	\item Nếu $0 < a\le 1$: Chỉ tạo muối acid \ce{NaHSO3}, viết phản ứng (3).
	\item Nếu $1 < a < 2$: Tạo cả muối acid \ce{NaHSO3} \& muối trung hòa \ce{Na2SO3}, viết 2 phản ứng (1) \& (3).
	\item Nếu $a\ge 2$: Chỉ tạo muối trung hòa \ce{Na2SO3}, viết phản ứng (1).
\end{itemize}
Nếu bài toán cho \ce{SO2} phản ứng với dung dịch chứa $x$ mol NaOH \& $y$ mol KOH thì có thể thay 2 base này bởi 1 base tương đương (hay ``base trung bình''{\tt/}``averaged base'') MOH. Khi đó, xét tỷ số $a = \frac{n_{\rm MOH}}{n_{\ce{CO2}}}$ là tỷ số mol của MOH cho bởi \eqref{MOH} \& \ce{CO2}.

\begin{baitoan}[\cite{Truong_Long_Huong_bdhsg_Hoa_Hoc_9}, Ví dụ 1, p. 44]
	Hấp thụ hoàn toàn {\rm7.84 L} (đktc) khí {\rm\ce{CO2}} vào {\rm200 mL} dung dịch {\rm KOH 1.5M} \& {\rm\ce{K2CO3} 1M}. Sau khi các phản ứng xảy ra hoàn toàn thu được dung dịch X. Tính khối lượng mỗi muối có trong dung dịch X.
\end{baitoan}

\begin{baitoan}[\cite{Truong_Long_Huong_bdhsg_Hoa_Hoc_9}, Ví dụ 2, p. 44]
	Hấp thụ hoàn toàn {\rm0.4 mol} khí {\rm\ce{CO2}} vào dung dịch chứa {\rm0.15 mol \ce{Ca(OH)2}} \& {\rm0.2 mol KOH}. Sau khi các phản ứng xảy ra hoàn toàn, thu được $m$ {\rm g} kết tủa. Tính $m$.
\end{baitoan}

\begin{baitoan}[\cite{Truong_Long_Huong_bdhsg_Hoa_Hoc_9}, Ví dụ 3, p. 45]
	Hấp thụ hoàn toàn {\rm4.48 L \ce{CO2}} (đktc) vào {\rm200 mL} dung dịch X gồm {\rm\ce{Na2CO3} 0.3M \& NaOH $x$M}, sau khi các phản ứng xảy ra hoàn toàn thu được dung dịch Y. Cho toàn bộ Y tác dụng với dung dịch {\rm\ce{CaCl2}} (dư), thu được {\rm10 g} kết tủa. Tính $x$.
\end{baitoan}

\begin{baitoan}[\cite{Truong_Long_Huong_bdhsg_Hoa_Hoc_9}, Ví dụ 4, p. 45]
	Hấp thụ hết {\rm6.72 L \ce{CO2}} (đktc) vào {\rm200 mL} dung dịch chứa {\rm KOH 1M} \& {\rm NaOH $x$M}. Sau khi các phản ứng xảy ra hoàn toàn, làm khô dung dịch thu được {\rm32.8 g} chất rắn khan. Giả sử trong quá trình làm khô dung dịch không xảy ra các {\rm PƯHH}. Tính $x$.
\end{baitoan}

\begin{baitoan}[\cite{Truong_Long_Huong_bdhsg_Hoa_Hoc_9}, Ví dụ 5, p. 46]
	Cho {\rm28.4 g \ce{P2O5}} vào {\rm750 mL} dung dịch {\rm NaOH 1.5M}. Sau khi các phản ứng xảy ra hoàn toàn, thu được dung dịch chứa $m$ {\rm g} muối. Tìm $m$.
\end{baitoan}

\begin{baitoan}[\cite{Truong_Long_Huong_bdhsg_Hoa_Hoc_9}, Ví dụ 6, p. 47, TS THPT Chuyên Phan Bội Châu, Nghệ An]
	Cho $m$ {\rm g \ce{P2O5}} vào {\rm19.6 g} dung dịch {\rm\ce{H3PO4} 5\%} thu được dung dịch X. Cho dung dịch X phản ứng hết với {\rm100 mL} dung dịch {\rm KOH 1M} thu được dung dịch Y. Cô cạn dung dịch Y thu được {\rm6.48 g} chất rắn khan. (a) Viết {\rm PTHH}. (b) Tính khối lượng các chất có trong {\rm6.48 g} chất rắn \& giá trị $m$.
\end{baitoan}

%------------------------------------------------------------------------------%

\section{\ce{SO2,CO2} $+$ \ce{Ba(OH)2,Ca(OH)2}}
\textit{Trường hợp \ce{CO2} tác dụng với dung dịch \ce{Ca(OH)2} hoặc \ce{Ba(OH)2}.}

%------------------------------------------------------------------------------%

\section{\ce{SO2,CO2} $+$ \ce{NaOH,KOH,Ba(OH)2,Ca(OH)2}}

%------------------------------------------------------------------------------%

\section{\ce{P2O5} $+$ NaOH, KOH}

%------------------------------------------------------------------------------%

\printbibliography[heading=bibintoc]

\end{document}