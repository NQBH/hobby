\documentclass{article}
\usepackage[backend=biber,natbib=true,style=alphabetic,maxbibnames=50]{biblatex}
\addbibresource{/home/nqbh/reference/bib.bib}
\usepackage[utf8]{vietnam}
\usepackage{tocloft}
\renewcommand{\cftsecleader}{\cftdotfill{\cftdotsep}}
\usepackage[colorlinks=true,linkcolor=blue,urlcolor=red,citecolor=magenta]{hyperref}
\usepackage{amsmath,amssymb,amsthm,float,graphicx,mathtools,diagbox,tikz,tipa}
\usepackage[version=4]{mhchem}
\allowdisplaybreaks
\newtheorem{assumption}{Assumption}
\newtheorem{baitoan}{Bài toán}
\newtheorem{cauhoi}{Câu hỏi}
\newtheorem{conjecture}{Conjecture}
\newtheorem{corollary}{Corollary}
\newtheorem{dangtoan}{Dạng toán}
\newtheorem{definition}{Definition}
\newtheorem{dinhly}{Định lý}
\newtheorem{dinhnghia}{Định nghĩa}
\newtheorem{example}{Example}
\newtheorem{ghichu}{Ghi chú}
\newtheorem{hequa}{Hệ quả}
\newtheorem{hypothesis}{Hypothesis}
\newtheorem{lemma}{Lemma}
\newtheorem{luuy}{Lưu ý}
\newtheorem{nhanxet}{Nhận xét}
\newtheorem{notation}{Notation}
\newtheorem{note}{Note}
\newtheorem{principle}{Principle}
\newtheorem{problem}{Problem}
\newtheorem{proposition}{Proposition}
\newtheorem{question}{Question}
\newtheorem{remark}{Remark}
\newtheorem{theorem}{Theorem}
\newtheorem{thinghiem}{Thí nghiệm}
\newtheorem{vidu}{Ví dụ}
\usepackage[left=1cm,right=1cm,top=5mm,bottom=5mm,footskip=4mm]{geometry}

\title{Problem: Inorganic Compound -- Bài Tập Hợp Chất Vô Cơ}
\author{Nguyễn Quản Bá Hồng\footnote{Independent Researcher, Ben Tre City, Vietnam\\e-mail: \texttt{nguyenquanbahong@gmail.com}; website: \url{https://nqbh.github.io}.}}
\date{\today}

\begin{document}
\maketitle
\begin{abstract}
	\textsf{[en]} This text is a collection of problems, from easy to advanced, about \textit{inorganic compound}, which is also a supplementary material for my lecture note on Elementary Chemistry, which is stored \& downloadable at the following link: \href{https://github.com/NQBH/hobby/blob/master/elementary_chemistry/grade_9/NQBH_elementary_chemistry_grade_9.pdf}{GitHub\texttt{/}NQBH\texttt{/}hobby\texttt{/}elementary chemistry\texttt{/}grade 9\texttt{/}lecture}\footnote{\textsc{url}: \url{https://github.com/NQBH/hobby/blob/master/elementary_chemistry/grade_9/NQBH_elementary_chemistry_grade_9.pdf}.}. The latest version of this text has been stored \& downloadable at the following link: \href{https://github.com/NQBH/hobby/blob/master/elementary_chemistry/inorganic_compound/NQBH_inorganic_compound.pdf}{GitHub\texttt{/}NQBH\texttt{/}hobby\texttt{/}elementary chemistry\texttt{/}grade 9\texttt{/}inorganic compound}\footnote{\textsc{url}: \url{https://github.com/NQBH/hobby/blob/master/elementary_chemistry/inorganic_compound/NQBH_inorganic_compound.pdf}.}.
	
	\textsf{\textbf{Keyword.} Inorganic compound.}
	\vspace{2mm}
	
	\textsf{[vi]} Tài liệu này là 1 bộ sưu tập các bài tập chọn lọc từ cơ bản đến nâng cao về \textit{phản ứng hóa học}, cũng là phần bài tập bổ sung cho tài liệu chính -- bài giảng \href{https://github.com/NQBH/hobby/blob/master/elementary_chemistry/grade_9/NQBH_elementary_chemistry_grade_9.pdf}{GitHub\texttt{/}NQBH\texttt{/}hobby\texttt{/}elementary chemistry\texttt{/}grade 9\texttt{/}lecture} của tác giả viết cho Hóa Học Sơ Cấp. Phiên bản mới nhất của tài liệu này được lưu trữ \& có thể tải xuống ở link sau: \href{https://github.com/NQBH/hobby/blob/master/elementary_chemistry/grade_9/real/NQBH_real.pdf}{GitHub\texttt{/}NQBH\texttt{/}hobby\texttt{/}elementary chemistry\texttt{/}grade 9\texttt{/}inorganic compound}.
	
	\textsf{\textbf{Từ khóa.} Hợp chất vô cơ.}
\end{abstract}
\setcounter{secnumdepth}{4}
\setcounter{tocdepth}{3}
\tableofcontents
\newpage

%------------------------------------------------------------------------------%

\section{Oxide}

\subsection{Qualitative Problem -- Bài tập định tính}

\begin{baitoan}[\cite{SGK_Hoa_Hoc_9}, 1., p. 6]
	Có các oxide: \emph{Cao, \ce{Fe2O3,SO3}}. Oxide nào có thể tác dụng được với: (a) nước? (b) hydrochloric acid? (c) sodium hydroxide? Viết các PTHH.
\end{baitoan}

\begin{baitoan}[\cite{SGK_Hoa_Hoc_9}, 2., p. 6]
	Có các chất: \emph{\ce{H2O,KOH,K2O,CO2}}. Cho biết các cặp chất có thể tác dụng với nhau.
\end{baitoan}

\begin{baitoan}[\cite{SGK_Hoa_Hoc_9}, 3., p. 6]
	Từ các chất: calcium oxide, lưu huỳnh dioxide, carbon dioxide, lưu huỳnh trioxide, zinc oxide, chọn chất thích hợp điền vào các sơ đồ phản ứng: (a) sulfuric acid $+$ $\ldots\to$ zinc sulfate $+$ nước. (b) sodium hydroxide $+$ $\ldots\to$ sodium sulfate $+$ nước. (c) nước $+$ $\ldots\to$ acid sulfurous. (d) nước $+$ $\ldots\to$ calcium hydroxide. (e) calcium oxide $+$ $\ldots\to$ calcium carbonate. Dùng các CTHH để viết tất cả các PTHH của các sơ đồ phản ứng trên.
\end{baitoan}

\begin{baitoan}[\cite{SGK_Hoa_Hoc_9}, 4., p. 6]
	Cho các oxide: \emph{\ce{CO2,SO2,Na2O,CaO,CuO}}. Chọn các chất tác dụng được với: (a) nước, tạo thành dung dịch acid. (b) nước, tạo thành dung dịch base. (c) dung dịch acid, tạo thành muối \& nước. (d) dung dịch base, tạo thành muối \& nước. Viết các PTHH.
\end{baitoan}

\begin{baitoan}[\cite{SGK_Hoa_Hoc_9}, 5., p. 6]
	Có hỗn hợp khí \emph{\ce{CO2,O2}}. Làm thế nào để có thể thu được khí \emph{\ce{O2}} từ hỗn hợp trên? Trình bày cách làm \& viết PTHH.
\end{baitoan}

\begin{baitoan}[\cite{SGK_Hoa_Hoc_9}, 1., p. 9]
	Bằng phương pháp hóa học nào có thể nhận biết được từng chất trong mỗi dãy chất sau? (a) 2 chất rắn màu trắng \emph{CaO, \ce{Na2O}}. (b) 2 chất khí không màu \emph{\ce{CO2,O2}}. Viết các PTHH.
\end{baitoan}

\begin{baitoan}[\cite{SGK_Hoa_Hoc_9}, 2., p. 9]
	Nhận biết từng chất trong mỗi nhóm chất sau bằng phương pháp hóa học. (a) \emph{CaO, \ce{CaCO3}}. (b) \emph{CaO, MgO}. Viết các PTHH.
\end{baitoan}

\begin{baitoan}[\cite{SGK_Hoa_Hoc_9}, 1., p. 11]
	Viết PTHH cho mỗi chuyển đổi: (a) \emph{S $\to$ \ce{SO2} $\to$ \ce{CaSO3}}. (b) \emph{\ce{SO2} $\to$ \ce{Na2SO3}}. (c) \emph{\ce{SO2} $\to$ \ce{H2SO3} $\to$ \ce{Na2SO3} $\to$ \ce{SO2}}.
\end{baitoan}

\begin{baitoan}[\cite{SGK_Hoa_Hoc_9}, 2., p. 11]
	Nhận biết từng chất trong mỗi nhóm chất sau bằng phương pháp hóa học. (a) 2 chất rắn màu trắng \emph{CaO, \ce{P2O5}}. (b) 2 chất khí không màu \emph{\ce{SO2,O2}}. Viết các PTHH.
\end{baitoan}

\begin{baitoan}[\cite{SGK_Hoa_Hoc_9}, 3., p. 11]
	Có các khí ẩm (khí có lẫn hơi nước): carbon dioxide, hydrogen, oxygen, lưu huỳnh dioxide. Khí nào có thể được làm khô bằng calcium oxide? Giải thích.
\end{baitoan}

\begin{baitoan}[\cite{SGK_Hoa_Hoc_9}, 4., p. 11]
	Có những chất khí sau: \emph{\ce{CO2,H2,O2,SO2,N2}}. Cho biết chất nào có tính chất sau: (a) nặng hơn không khí. (b) nhẹ hơn không khí. (c) cháy được trong không khí. (d) tác dụng với nước tạo thành dung dịch acid. (e) làm đục nước vôi trong. (f) đổi màu giấy quỳ tím ẩm thành đỏ.
\end{baitoan}

\begin{baitoan}[\cite{SGK_Hoa_Hoc_9}, 5., p. 11]
	Khí lưu huỳnh dioxide được tạo thành từ cặp chất nào sau đây? (a) \emph{\ce{K2SO3,H2SO4}}. (b) \emph{\ce{K2SO4}, HCl}. (c) \emph{\ce{Na2SO3}, NaOH}. (d) \emph{\ce{Na2SO4,CuCl2}}. (e) \emph{\ce{Na2SO3}, NaCl}. Viết PTHH.
\end{baitoan}

\begin{baitoan}[\cite{An_350_BT_Hoa_Hoc_9}, 1., p. 5]
	Nêu các base \& acid tương ứng của các oxide: \emph{\ce{SO2,SO3,N2O5,CaO,K2O,CuO,Mn2O7}}.
\end{baitoan}

\begin{baitoan}[\cite{An_350_BT_Hoa_Hoc_9}, 2., p. 5]
	Trong các oxide: \emph{CaO, \ce{Al2O3,NO,N2O5,CO2,SO2,MgO,CO,Fe2O3}}, oxide nào là oxide tạo muối.
\end{baitoan}

\begin{baitoan}[\cite{An_350_BT_Hoa_Hoc_9}, 3., p. 5]
	Cho các oxide: \emph{\ce{Na2O,Fe2O3,Fe3O4,SO3,CaO}}. Viết phương trình phản ứng (nếu có) khi cho các oxide này lần lượt tác dụng với nước, dung dịch \emph{NaOH}, dung dịch \emph{HCl}.
\end{baitoan}

\begin{baitoan}[\cite{An_350_BT_Hoa_Hoc_9}, 4.a, p. 6]
	Cho các chất sau: \emph{\ce{CaCl2} (khan), \ce{P2O5,H2SO4} (đặc), \ce{Ba(OH)2} (rắn)}, chất nào được dùng để làm khô khí \emph{\ce{CO2}}? Giải thích bằng PTHH.
\end{baitoan}

\begin{baitoan}[\cite{An_350_BT_Hoa_Hoc_9}, 4.b, p. 6]
	Có 4 oxide riêng biệt: \emph{\ce{Na2O,Al2O3,Fe2O3,MgO}}. Làm thế nào để có thể nhận biết được mỗi oxide bằng phương pháp hóa học với điều kiện chỉ được dùng thêm $2$ chất?
\end{baitoan}

\begin{baitoan}[\cite{An_350_BT_Hoa_Hoc_9}, 6.b, p. 7]
	Làm thế nào để nhận ra sự có mặt của mỗi khí trong hỗn hợp gồm \emph{\ce{CO,CO2,SO3}} bằng phương pháp hóa học. Viết các PTHH (nếu có).
\end{baitoan}

\subsection{Quantitative Problem -- Bài tập định lượng}

\begin{baitoan}[\cite{SGK_Hoa_Hoc_9}, 6., p. 6]
	Cho \emph{1.6 g} copper(II) oxide tác dụng với \emph{100 g} dung dịch acid sulfuric có nồng độ \emph{20\%}. (a) Viết PTHH. (b) Tính nồng độ \% của các chất có trong dung dịch sau khi phản ứng kết thúc.
\end{baitoan}

\begin{baitoan}[\cite{SGK_Hoa_Hoc_9}, 3., p. 9]
	\emph{200 mL} dung dịch \emph{HCl} có nồng độ \emph{3.5M} hòa tan vừa hết \emph{20 g} hỗn hợp 2 oxide \emph{CuO, \ce{Fe2O3}}. (a) Viết các PTHH. (b) Tính khối lượng của mỗi oxide có trong mỗi hỗn hợp ban đầu.
\end{baitoan}

\begin{baitoan}[\cite{SGK_Hoa_Hoc_9}, 4., p. 9]
	Biết \emph{2.24 L} khí \emph{\ce{CO2}} (đktc) tác dụng vừa hết với \emph{200 mL} dung dịch \emph{\ce{Ba(OH)2}}, sản phẩm là \emph{\ce{BaCO3,H2O}}. (a) Viết PTHH. (b) Tính nồng độ mol của dung dịch \emph{\ce{Ba(OH)2}} đã dùng. (c) Tính khối lượng chất kết tủa thu được.
\end{baitoan}

\begin{baitoan}[\cite{SGK_Hoa_Hoc_9}, 6., p. 11]
	Dẫn \emph{112 mL} khí \emph{\ce{SO2}} (đktc) đi qua \emph{700 mL} dung dịch \emph{\ce{Ca(OH)2}} có nồng độ \emph{0.01M}, sản phẩm là muối calcium sulfite. (a) Viết PTHH. (b) Tính khối lượng các chất sau phản ứng.
\end{baitoan}

\begin{baitoan}[\cite{An_350_BT_Hoa_Hoc_9}, 5.a, p. 6]
	Cho $a$ \emph{g Na} tác dụng với $p$ \emph{g} nước thu được dung dịch \emph{NaOH} nồng độ $x$\%. Cho $b$ \emph{g \ce{Na2O}} tác dụng với $p$ \emph{g} nước cũng thu được dung dịch \emph{NaOH} nồng độ $x$\%. Lập biểu thức tính $p$ theo $a,b$.
\end{baitoan}

\begin{baitoan}[\cite{An_350_BT_Hoa_Hoc_9}, 5.b, p. 6]
	Khử hoàn toàn \emph{3.2 g} hỗn hợp \emph{CuO, \ce{Fe2O3}} bằng \emph{\ce{H2}} tạo ra \emph{0.9 g \ce{H2O}}. Tính khối lượng hỗn hợp kim loại thu được.
\end{baitoan}

\begin{baitoan}[\cite{An_350_BT_Hoa_Hoc_9}, 6.a, p. 7]
	Cho \emph{2.24 L \ce{CO2}} (đktc) tác dụng hoàn toàn với \emph{25 g} dung dịch \emph{NaOH 20\%}. Tính khối lượng muối tạo thành.
\end{baitoan}

\begin{baitoan}[\cite{An_350_BT_Hoa_Hoc_9}, 7.a, p. 8]
	Nung $m$ \emph{g} hỗn hợp chất rắn A gồm \emph{\ce{Fe2O3}} \& \emph{FeO} với lượng thiếu khí \emph{CO} thu được hỗn hợp chất rắn B có khối lượng \emph{47.84 g} \& \emph{5.6 L \ce{CO2}}. Tính $m$.
\end{baitoan}

\begin{baitoan}[\cite{An_350_BT_Hoa_Hoc_9}, 7.b, p. 9]
	Cho \emph{11.6 g} hỗn hợp \emph{\ce{Fe2O3}} \& \emph{FeO} có tỷ lệ số mol là $1:1$ vào \emph{300 mL} dung dịch \emph{HCl 2M} được dung dịch A. Tính nồng độ mol của các chất trong dung dịch sau phản ứng (thể tích dung dịch thay đổi không đáng kể).
\end{baitoan}

\begin{baitoan}[\cite{An_350_BT_Hoa_Hoc_9}, 8.a, p. 9]
	Nung nóng kim loại M trong không khí đến khối lượng không đổi thu được chất rắn N. Khối lượng của M bằng $\frac{7}{10}$ khối lượng của N. Tìm CTPT của N.
\end{baitoan}

\begin{baitoan}[\cite{An_350_BT_Hoa_Hoc_9}, 8.b, p. 9]
	Cho 1 oxide base tác dụng với dung dịch \emph{\ce{H2SO4} 24.5\%} thu được dung dịch 1 muối có nồng độ \emph{32.2\%}. Tìm CTPT của oxide base.
\end{baitoan}

\begin{baitoan}[\cite{An_350_BT_Hoa_Hoc_9}, 9.a, p. 11]
	Dẫn $V$ \emph{L} khí \emph{\ce{CO2}} (đktc) qua \emph{250 mL} dung dịch \emph{\ce{Ca(OH)2} 1M} thấy có \emph{12.5 g} kết tủa. Tính $V$.
\end{baitoan}

\begin{baitoan}[\cite{An_350_BT_Hoa_Hoc_9}, 9.b, p. 11]
	Dùng khí \emph{\ce{H2}} để khử $a$ \emph{g} oxide sắt. Sản phẩm hơi tạo ra cho qua $100$ \emph{g} acid \emph{\ce{H2SO4} 98\%} thì nồng độ acid giảm đi \emph{3.405\%}. Chất rắn thu được sau phản ứng trên cho tác dụng hết với dung dịch \emph{HCl} thấy thoát ra \emph{3.36 L} \emph{\ce{H2}} (đktc). Xác định CTPT oxide sắt.
\end{baitoan}

\begin{baitoan}[\cite{An_350_BT_Hoa_Hoc_9}, 10.a, p. 13]
	Để xác định CTPT oxide sắt người ta làm thí nghiệm như sau: Hòa tan $a$ \emph{g} oxide sắt thì cần \emph{300 mL} dung dịch \emph{HCl 3M}. Cho toàn bộ $a$ \emph{g} oxide sắt nung nóng tác dụng với \emph{CO} dư thu được \emph{16.8 g} sắt. Xác định CTPT oxide sắt.
\end{baitoan}

\begin{baitoan}[\cite{An_350_BT_Hoa_Hoc_9}, 10.b, p. 13]
	1 loại đá vôi chứa \emph{80\% \ce{CaCO3}} \& \emph{20\%} tạp chất không bị phân hủy bởi nhiệt. Khi nung $a$ \emph{g} đá vôi trên thu được chất rắn có khối lượng bằng \emph{75\%} khối lượng đá trước khi nung. (a) Tính hiệu suất phản ứng phân hủy \emph{\ce{CaCO3}}. (b) Tính thành phần \% khối lượng \emph{CaO} trong chất rắn sau khi nung.
\end{baitoan}

\begin{baitoan}[\cite{An_350_BT_Hoa_Hoc_9}, 11.a, p. 14]
	Khử hoàn toàn \emph{5.8 g} 1 oxide sắt bằng \emph{CO} ở nhiệt độ cao. Sản phẩm sau phản ứng cho qua dung dịch nước vôi trong dư tạo \emph{10 g} kết tủa. Xác định CTPT oxide sắt.
\end{baitoan}

\begin{baitoan}[\cite{An_350_BT_Hoa_Hoc_9}, 11.b, p. 14]
	Nung $1.5$ tấn đá vôi chứa \emph{85\% \ce{CaCO3}} thì có thể thu được bao nhiêu \emph{kg} vôi sống? Biết hiệu suất phản ứng là \emph{90\%}.
\end{baitoan}

\begin{baitoan}[\cite{An_350_BT_Hoa_Hoc_9}, 12.a, p. 15]
	Cho \emph{7.84 g CaO} tan hoàn toàn vào nước được dung dịch A. Dẫn \emph{2.24 L} khí \emph{\ce{CO2}} (đktc) vào dung dịch A. Tính khối lượng các chất sau phản ứng.
\end{baitoan}

\begin{baitoan}[\cite{An_350_BT_Hoa_Hoc_9}, 12.b, p. 15]
	Nung $1$ tấn đá vôi thì thu được \emph{428.4 kg} vôi sống \emph{CaO}. Hiệu suất quá trình nung vôi là \emph{85\%}, tính tỷ lệ \emph{\%} khối lượng tạp chất có trong đá vôi.
\end{baitoan}

%------------------------------------------------------------------------------%

\section{Acid}

\subsection{Qualitative Problem -- Bài tập định tính}

\begin{baitoan}[\cite{An_350_BT_Hoa_Hoc_9}, 24.a, p. 24]
	Bằng phương pháp hóa học, phân biệt 3 dung dịch: \emph{HCl, NaOH, \ce{Ba(OH)2}}.
\end{baitoan}

\subsection{Quantitative Problem -- Bài tập định lượng}

\begin{baitoan}[\cite{An_350_BT_Hoa_Hoc_9}, 13.a, p. 16]
	Lấy \emph{4.2 g} bột sắt cho tác dụng với \emph{50 mL} dung dịch \emph{\ce{H2SO4} 1M} đến khi kết thúc phản ứng thu được $V$ \emph{L} khí \emph{\ce{H2}} bay ra ở đktc: (a) Cho biết chát nào còn dư sau phản ứng? (b) Tính $V$.
\end{baitoan}

\begin{baitoan}[\cite{An_350_BT_Hoa_Hoc_9}, 13.b, p. 16]
	Cho \emph{29.4 g} dung dịch \emph{\ce{H2SO4} 20\%} vào \emph{100 g} dung dịch \emph{\ce{BaCl2} 5.2\%}. (a) Viết PTHH xảy ra \& tính khối lượng kết tủa tạo thành. (b) Tính nồng độ \% của những chất có trong dung dịch.
\end{baitoan}

\begin{baitoan}[\cite{An_350_BT_Hoa_Hoc_9}, 14.a, p. 17]
	Hòa tan 1 lượng \emph{CuO} cần \emph{100 mL} dung dịch \emph{HCl 1M}. (a) Tính khối lượng \emph{CuO} đã tham gia phản ứng. (b) Tính nồng độ mol của dung dịch sau phản ứng. Biết thể tích dung dịch thay đổi không đáng kể.
\end{baitoan}

\begin{baitoan}[\cite{An_350_BT_Hoa_Hoc_9}, 14.b, p. 17]
	Trộn $c$ \emph{g} bột \emph{Fe} \& $b$ \emph{g} bột \emph{S} rồi nung nóng ở nhiệt độ cao (không có không khí). Hòa tan hỗn hợp sau phản ứng bằng dung dịch \emph{HCl} dư thu được chất rắn X nặng \emph{0.4 g} \& khí Y có tỷ khối so với \emph{\ce{H2}} bằng $9$. Khí Y sục từ từ qua dung dịch \emph{\ce{Pb(NO3)2}} thấy tạo thành \emph{11.95 g} kết tủa. (a) Tính $b,c$. (b) Tính hiệu suất phản ứng nung nóng bột \emph{Fe} \& bột \emph{S}.
\end{baitoan}

\begin{baitoan}[\cite{An_350_BT_Hoa_Hoc_9}, 15., p. 18]
	Hỗn hợp X gồm 2 kim loại \emph{Mg, Fe}. Dung dịch Y là dung dịch \emph{HCl $a$ M}. Thí nghiệm 1: Cho \emph{10.8 g} hỗn hợp X vào \emph{2 L} dung dịch Y có \emph{4.48 L \ce{H2}} (đktc) bay ra. Thí nghiệm 2: Cho \emph{10.8 g} hỗn hợp X vào \emph{3 L} dung dịch Y có \emph{5.6 L \ce{H2}} (đktc) bay ra. Tính $a$ \& tính khối lượng mỗi kim loại trong hỗn hợp X.
\end{baitoan}

\begin{baitoan}[\cite{An_350_BT_Hoa_Hoc_9}, 16., p. 19]
	Hòa tan hoàn toàn \emph{4 g} hỗn hợp gồm \emph{Fe} \& 1 kim loại hóa trị II vào dung dịch \emph{HCl} thì thu được \emph{2.24 L \ce{H2}} (đktc). Nếu chỉ dùng \emph{2.4 g} kim loại hóa trị II cho vào dung dịch \emph{HCl} thì dùng không hết \emph{500 mL} dung dịch \emph{HCl 1M}. Tìm tên kim loại hóa trị II.
\end{baitoan}

\begin{baitoan}[\cite{An_350_BT_Hoa_Hoc_9}, 17., p. 17]
	Trộn \emph{CuO} với 1 oxide kim loại hóa trị II không đổi theo tỷ lệ số mol $1:2$ được hỗn hợp A, cho luồng khí \emph{\ce{H2}} dư qua \emph{2.4 g} hỗn hợp A nung nóng đến phản ứng hoàn toàn được chất rắn B. Để hòa tan hết B cần \emph{100 mL} dung dịch \emph{\ce{HNO3} 1M} chỉ thoát ra khí \emph{NO} duy nhất. Phản ứng xảy ra theo phương trình: \emph{\ce{$3$Cu + $8$HNO3 -> $3$Cu(NO3)2 + $2$NO + $4$H2O, $3$M + $8$HNO3 -> $3$M(NO3)2 + $2$NO + $4$H2O}}. Xác định tên kim loại hóa trị II.
\end{baitoan}

\begin{baitoan}[\cite{An_350_BT_Hoa_Hoc_9}, 18., p. 21]
	1 hỗn hợp X gồm \emph{Al, Mg, Cu} có khối lượng là \emph{5 g} khi hòa tan trong dung dịch \emph{HCl} dư thấy thoát ra \emph{4.48 $\rm dm^3$} khí (đktc) \& thu được dung dịch Y cùng chất rắn Z. Lọc \& nung chất rắn Z trong không khí đến khối lượng không đổi cân nặng \emph{1.375 g}. Tính khối lượng mỗi kim loại.
\end{baitoan}

%------------------------------------------------------------------------------%

\section{Base}

\subsection{Qualitative Problem -- Bài tập định tính}

\subsection{Quantitative Problem -- Bài tập định lượng}

\begin{baitoan}[\cite{An_350_BT_Hoa_Hoc_9}, 19., p. 21]
	Cho \emph{150 mL} dung dịch \emph{NaOH 0.5M} vào \emph{150 mL} dung dịch \emph{HCl 1M}. (a) Viết PTHH. (b) Nếu cho giấy quỳ tím vào dung dịch sau phản ứng, thì màu của giấy quỳ thay đổi như thế nào? Vì sao? (c) Tính khối lượng muối tạo thành sau phản ứng.
\end{baitoan}

\begin{baitoan}[\cite{An_350_BT_Hoa_Hoc_9}, 20., p. 22]
	Cho $m$ \emph{g NaOH} nguyên chất tác dụng với dung dịch \emph{\ce{Cu(NO3)2}} có dư, thu được \emph{29.4 g} kết tủa \emph{\ce{Cu(OH)2}}. (a) Viết PTHH. (b) Tính $m$.
\end{baitoan}

\begin{baitoan}[\cite{An_350_BT_Hoa_Hoc_9}, 21.a, p. 22]
	Nếu có \emph{20 g} dung dịch sodium hydroxide \emph{20\%} phải dùng hết bao nhiêu \emph{g} dung dịch hydrochloric acid \emph{25\%} để trung hòa.
\end{baitoan}

\begin{baitoan}[\cite{An_350_BT_Hoa_Hoc_9}, 21.b, p. 22]
	Hòa tan \emph{12.4 g \ce{Na2O}} vào \emph{1 L} nước ta được dung dịch X. Lấy \emph{0.5 L} dung dịch X cho tác dụng với $V$ \emph{mL} dung dịch \emph{\ce{Fe2(SO4)3} 0.5M} (vừa đủ) tạo thành 1 kết tủa \& dung dịch Y. Tính $V$.
\end{baitoan}

\begin{baitoan}[\cite{An_350_BT_Hoa_Hoc_9}, 22., p. 23]
	Dung dịch X chứa \emph{2.7 g \ce{CuCl2}} cho tác dụng với dung dịch Y chứa \emph{NaOH} (lấy dư). Sau khi phản ứng kết thúc thu được kết tủa Z lọc lấy kết tủa Z đem nung đến khối lượng không đổi, thu được chất rắn T. (a) Viết PTHH. (b) Tính khối lượng kết tủa Z \& chất rắn T.
\end{baitoan}

\begin{baitoan}[\cite{An_350_BT_Hoa_Hoc_9}, 23., p. 23]
	Cho \emph{200 mL} dung dịch \emph{HCl 0.2M}. (a) Tính thể tích dung dịch \emph{NaOH 0.2M} cần để trung hòa dung dịch acid trên. Tính nồng độ mol của dung dịch muối tạo thành. (b) Nếu cho dung dịch acid trên tác dụng với \emph{\ce{CaCO3}}. Tính khối lượng \emph{\ce{CaCO3}} để phản ứng xảy ra vừa đủ \& thể tích khí bay lên.
\end{baitoan}

\begin{baitoan}[\cite{An_350_BT_Hoa_Hoc_9}, 24.b, p. 24]
	Để trung hòa \emph{25 mL} dung dịch X cần dùng \emph{30 mL} dung dịch \emph{HCl 1M}. Khi cho \emph{25 mL} dung dịch X tác dụng với 1 lượng dư \emph{\ce{Na2CO3}} thấy tạo thành \emph{1.97 g} kết tủa. Tính nồng độ mol của \emph{NaOH, \ce{Ba(OH)2}} trong dung dịch X.
\end{baitoan}

\begin{baitoan}[\cite{An_350_BT_Hoa_Hoc_9}, 25., p. 25]
	Cho \emph{0.594 g} hỗn hợp \emph{Na, Ba} hòa tan hoàn toàn vào nước thu được dung dịch A \& khí B. Trung hòa dung dịch A cần \emph{100 mL HCl}. Cô cạn dung dịch sau phản ứng thu được \emph{0.949 g} muối. (a) Tính thể tích khí B (đktc), nồng độ mol của dung dịch \emph{HCl}. (b) Tính khối lượng mỗi kim loại.
\end{baitoan}

%------------------------------------------------------------------------------%

\section{Salt -- Muối}

\subsection{Qualitative Problem -- Bài tập định tính}

\begin{baitoan}[\cite{An_350_BT_Hoa_Hoc_9}, 44., p. 37]
	Viết PTHH để thực hiện chuỗi chuyển hóa sau: (a) \emph{\ce{FeS2} $\to$ \ce{SO2} $\to$ \ce{SO3} $\to$ \ce{H2SO4} $\to$ \ce{CuSO4}}. (b) \emph{\ce{AlCl3} $\to$ \ce{Al(OH)3} $\to$ \ce{Al2O3} $\to$ \ce{Al2(SO4)3} $\to$ \ce{AlCl3}}. (c) \emph{Na $\to$ \ce{Na2O} $\to$ NaOH $\to$ \ce{Na2CO3} $\to$ \ce{NaHCO3}}. (d) Cho các chất: \emph{\ce{SO2,Fe2O3,Ba(OH)2,HCl,KHCO3}}. Chất nào tác dụng được với dung dịch \emph{\ce{H2SO4}}? Chất nào tác dụng được với dung dịch \emph{KOH}? Viết PTHH.
\end{baitoan}

\subsection{Quantitative Problem -- Bài tập định lượng}

\subsubsection{Tính khối lượng muối \& thể tích khí \ce{CO2}}

\begin{baitoan}[\cite{An_350_BT_Hoa_Hoc_9}, 26., p. 27]
	Cho \emph{8.25 g} hỗn hợp bột kim loại \emph{Mg, Fe} tác dụng hết với dung dịch \emph{HCl} thấy thoát ra \emph{5.6 L \ce{H2}} (đktc). Tính khối lượng muối tạo thành.
\end{baitoan}

\begin{baitoan}[\cite{An_350_BT_Hoa_Hoc_9}, 27., p. 27]
	Cho \emph{1.84 g} carbonate của 2 kim loại hóa trị II, tác dụng hết với dung dịch \emph{HCl} thu được \emph{0.672 L \ce{CO2}} \& dung dịch X. Tính khối lượng muối trong dung dịch X.
\end{baitoan}

\begin{baitoan}[\cite{An_350_BT_Hoa_Hoc_9}, 28., p. 28]
	Cho \emph{19.7 g} muối carbonate của kim loại hóa trị II bằng dung dịch \emph{\ce{H2SO4}} loãng dư thu được \emph{23.3 g} muối sulfate. Tính thể tích \emph{\ce{CO2}} \& xác định CTPT của muối.
\end{baitoan}

\begin{baitoan}[\cite{An_350_BT_Hoa_Hoc_9}, 29., p. 28]
	Hòa tan \emph{21.5 g} hỗn hợp \emph{\ce{BaCl2,CaCl2}} vào \emph{250 mL \ce{H2O}} để được dung dịch X. Thêm vào dung dịch X \emph{200 mL} dung dịch \emph{\ce{Na2CO3} 1M} thấy tách ra \emph{19.85 g} kết tủa \& còn nhận được \emph{400 mL} dung dịch Y. Tính nồng độ mol các chất trong dung dịch Y.
\end{baitoan}

\begin{baitoan}[\cite{An_350_BT_Hoa_Hoc_9}, 30., p. 29]
	Trong \emph{1 L} dung dịch hỗn hợp X gồm \emph{0.2 mol \ce{Na2CO3}} \& \emph{0.5 mol \ce{(NH4)2CO3}}. Cho \emph{86 g} hỗn hợp \emph{\ce{BaCl2,CaCl2}} vào dung dịch X. Sau khi phản ứng kết thúc, ta thu được \emph{79.4 g} kết tủa Y. Tính khối lượng các chất trong kết tủa Y.
\end{baitoan}

\begin{baitoan}[\cite{An_350_BT_Hoa_Hoc_9}, 31., p. 30]
	Cho \emph{5.8 g} muối carbonate \emph{\ce{MCO3}} của kim loại M tan hoàn toàn trong dung dịch \emph{\ce{H2SO4}} loãng vừa đủ, thu được 1 chất khí \& dung dịch X. Cô cạn dung dịch X thu được \emph{7.6 g} muối sulfate trung hòa, khan. Xác định CTHH của muối carbonate.
\end{baitoan}

\begin{baitoan}[\cite{An_350_BT_Hoa_Hoc_9}, 32., p. 30]
	Hòa tan hoàn toàn \emph{14.2 g} hỗn hợp A gồm \emph{\ce{MgCO3}} \& muối carbonate của kim loại R vào acid \emph{HCl 7.3\%} vừa đủ, thu được dung dịch B \& \emph{3.36 L} khí \emph{\ce{CO2}} (đktc). Nồng độ \emph{\ce{MgCl2}} trong dung dịch B bằng \emph{6.028\%}. Xác định kim loại R.
\end{baitoan}

\begin{baitoan}[\cite{An_350_BT_Hoa_Hoc_9}, 33.a, p. 31]
	Có hỗn hợp gồm 2 muối \emph{NaCl, NaBr}. Khi cho dung dịch \emph{\ce{AgNO3}} vừa đủ vào hỗn hợp trên người ta thu được lượng kết tủa bằng khối lượng \emph{\ce{AgNO3}} tham gia phản ứng. Tính \% khối lượng mỗi chất trong hỗn hợp.
\end{baitoan}

\begin{baitoan}[\cite{An_350_BT_Hoa_Hoc_9}, 33.b, p. 31]
	Cho 2 cốc đựng dung dịch \emph{HCl} đặt trên 2 đĩa cân A \& B: cân ở trạng thái thăng bằng. Cho $a$ \emph{g \ce{CaCO3}} vào cốc A \& $b$ \emph{g \ce{M2CO3}} (M: kim loại kiềm) vào cốc B. Sau khi 2 muối đã tan hoàn toàn, cân trở lại vị trí thăng bằng. Thiết lập biểu thức tính nguyên tử khối của M theo $a,b$. Áp dụng cho $a = 5$ \emph{g}, $b = 4.8$ \emph{g}. Xác định kim loại M.
\end{baitoan}

\begin{baitoan}[\cite{An_350_BT_Hoa_Hoc_9}, 34., p. 32]
	Cho từ từ dung dịch chứa $a$ \emph{mol HCl} vào dung dịch chứa $b$ \emph{mol \ce{Na2CO3}} đồng thời khuấy đều, thu được $V$ \emph{L} khí (ở đktc) \& dung dịch X. Khi co dư nước vôi trong vào dung dịch X thấy có xuất hiện kết tủa. Tính biểu thức liên hệ giữa $V$ với $a,b$.
\end{baitoan}

\begin{baitoan}[\cite{An_350_BT_Hoa_Hoc_9}, 35., p. 32]
	Cho \emph{1.9 g} hỗn hợp muối carbonate \& hydrocarbonate (i.e., bicarbonate) của kim loại kiềm M tác dụng hết với dung dịch \emph{HCl} (dư), sinh ra \emph{0.448 L} khí (đktc). Xác định kim loại M.
\end{baitoan}

\begin{baitoan}[\cite{An_350_BT_Hoa_Hoc_9}, 36., p. 33]
	Khi hòa tan hydroxide kim loại \emph{\ce{M(OH)2}} bằng 1 lượng vừa đủ dung dịch \emph{\ce{H2SO4} 20\%} thu được dung dịch muối trung hòa có nồng độ \emph{27.21\%}. Xác định kim loại M.
\end{baitoan}

\subsubsection{Kim loại mạnh đẩy kim loại yếu ra khỏi dung dịch muối}

\begin{baitoan}[\cite{An_350_BT_Hoa_Hoc_9}, 37., p. 33]
	Nhúng 1 lá nhôm vào dung dịch \emph{\ce{CuSO4}}. Sau phản ứng lấy lá nhôm ra thì thấy khối lượng dung dịch nhẹ đi \emph{1.38 g}. Tính khối lượng \emph{Al} đã phản ứng.
\end{baitoan}

\begin{baitoan}[\cite{An_350_BT_Hoa_Hoc_9}, 38., p. 34]
	Nhúng 1 thanh graphite phủ kim loại A hóa trị II vào dung dịch \emph{\ce{CuSO4}} dư. Sau phản ứng thanh graphite giảm \emph{0.04 g}. Tiếp tục nhúng thanh graphite này vào dung dịch \emph{\ce{AgNO3}} dư, khi phản ứng kết thúc khối lượng thanh graphite tăng \emph{6.08 g} (so với khối lượng thanh graphite sau khi nhúng vào \emph{\ce{CuSO4}}). Tìm tên kim loại A \& khối lượng kim loại A đã phủ lên thanh graphite lúc đầu. Coi như toàn bộ kim loại tạo thành đều bám vào thanh graphite.
\end{baitoan}

\begin{baitoan}[\cite{An_350_BT_Hoa_Hoc_9}, 39., p. 35]
	Nhúng thanh kim loại \emph{Zn} vào 1 dung dịch chứa hỗn hợp \emph{3.2 g \ce{CuSO4}} \& \emph{6.24 g \ce{CdSO4}}. Hỏi sau khi \emph{Cu, Cd} bị đẩy hoàn toàn khỏi dung dịch thì khối lượng thanh \emph{Zn} tăng hay giảm bao nhiêu?	
\end{baitoan}

\begin{baitoan}[\cite{An_350_BT_Hoa_Hoc_9}, 40., p. 35]
	Cho 1 lá đồng có khối lượng \emph{5 g} vào \emph{125 g} dung dịch \emph{\ce{AgNO3} 4\%}. Sau 1 thời gian, khi lấy lá đồng ra thì khối lượng \emph{\ce{AgNO3}} trong dung dịch giảm \emph{17\%}. Xác định khối lượng kim loại \emph{Cu} sau phản ứng.
\end{baitoan}

\begin{baitoan}[\cite{An_350_BT_Hoa_Hoc_9}, 41., p. 36]
	Cho $m$ \emph{g} hỗn hợp \emph{Zn, Fe} vào lượng dư dung dịch \emph{\ce{CuSO4}}. Sau khi kết thúc các phản ứng, lọc bỏ phần dung dịch thu được $m$ \emph{g} chất rắn. Tính thành phần \% theo khối lượng của \emph{Zn} trong hỗn hợp ban đầu.
\end{baitoan}

\begin{baitoan}[\cite{An_350_BT_Hoa_Hoc_9}, 42., p. 36]
	Cho 1 lượng bột \emph{Zn} vào dung dịch X gồm \emph{\ce{FeCl2,CuCl2}}. Khối lượng chất rắn sau khi các phản ứng xảy ra hoàn toàn nhỏ hơn khối lượng bột \emph{Zn} ban đầu là \emph{0.5 g}. Cô cạn phần dung dịch sau phản ứng thu được \emph{13.6 g} muối khan. Tính tổng khối lượng các muối trong X.
\end{baitoan}

\begin{baitoan}[\cite{An_350_BT_Hoa_Hoc_9}, 43., p. 36]
	Hòa tan hoàn toàn \emph{13.8 g} muối carbonate 1 kim loại kiềm \emph{\ce{R2CO3}} trong \emph{110 mL} dung dịch \emph{HCl 2M}. Sau khi phản ứng xảy ra hoàn toàn, ta thấy còn dư acid trong dung dịch thu được \& thể tích khí thoát ra $V_1$ vượt quá \emph{2016 mL} (đktc). Xác định CTHH muối carbonate.
\end{baitoan}

\subsubsection{Dạng bài toán chứng minh acid còn dư hay hỗn hợp các chất còn dư}

\begin{baitoan}[\cite{An_350_BT_Hoa_Hoc_9}, 37., p. 33]
	
\end{baitoan}

\begin{baitoan}[\cite{An_350_BT_Hoa_Hoc_9}, 37., p. 33]
	
\end{baitoan}

\begin{baitoan}[\cite{An_350_BT_Hoa_Hoc_9}, 37., p. 33]
	
\end{baitoan}


%------------------------------------------------------------------------------%

\printbibliography[heading=bibintoc]
	
\end{document}