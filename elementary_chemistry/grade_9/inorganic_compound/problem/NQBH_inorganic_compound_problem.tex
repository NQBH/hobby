\documentclass{article}
\usepackage[backend=biber,natbib=true,style=authoryear,maxbibnames=20]{biblatex}
\addbibresource{/home/nqbh/reference/bib.bib}
\usepackage[utf8]{vietnam}
\usepackage{tocloft}
\renewcommand{\cftsecleader}{\cftdotfill{\cftdotsep}}
\usepackage[colorlinks=true,linkcolor=blue,urlcolor=red,citecolor=magenta]{hyperref}
\usepackage{amsmath,amssymb,amsthm,float,graphicx,mathtools,diagbox,tikz,tipa}
\usepackage[version=4]{mhchem}
\allowdisplaybreaks
\newtheorem{assumption}{Assumption}
\newtheorem{baitoan}{Bài toán}
\newtheorem{cauhoi}{Câu hỏi}
\newtheorem{conjecture}{Conjecture}
\newtheorem{corollary}{Corollary}
\newtheorem{dangtoan}{Dạng toán}
\newtheorem{definition}{Definition}
\newtheorem{dinhly}{Định lý}
\newtheorem{dinhnghia}{Định nghĩa}
\newtheorem{example}{Example}
\newtheorem{ghichu}{Ghi chú}
\newtheorem{hequa}{Hệ quả}
\newtheorem{hypothesis}{Hypothesis}
\newtheorem{lemma}{Lemma}
\newtheorem{luuy}{Lưu ý}
\newtheorem{nhanxet}{Nhận xét}
\newtheorem{notation}{Notation}
\newtheorem{note}{Note}
\newtheorem{principle}{Principle}
\newtheorem{problem}{Problem}
\newtheorem{proposition}{Proposition}
\newtheorem{question}{Question}
\newtheorem{remark}{Remark}
\newtheorem{theorem}{Theorem}
\newtheorem{thinghiem}{Thí nghiệm}
\newtheorem{vidu}{Ví dụ}
\usepackage[left=1cm,right=1cm,top=5mm,bottom=5mm,footskip=4mm]{geometry}

\title{Problem: Inorganic Compound -- Bài Tập Hợp Chất Vô Cơ}
\author{Nguyễn Quản Bá Hồng\footnote{Independent Researcher, Ben Tre City, Vietnam\\e-mail: \texttt{nguyenquanbahong@gmail.com}; website: \url{https://nqbh.github.io}.}}
\date{\today}

\begin{document}
\maketitle
\begin{abstract}
	\textsf{[en]} This text is a collection of problems, from easy to advanced, about \textit{inorganic compound}, which is also a supplementary material for my lecture note on Elementary Chemistry, which is stored \& downloadable at the following link: \href{https://github.com/NQBH/hobby/blob/master/elementary_chemistry/grade_9/NQBH_elementary_chemistry_grade_9.pdf}{GitHub\texttt{/}NQBH\texttt{/}hobby\texttt{/}elementary chemistry\texttt{/}grade 9\texttt{/}lecture}\footnote{\textsc{url}: \url{https://github.com/NQBH/hobby/blob/master/elementary_chemistry/grade_9/NQBH_elementary_chemistry_grade_9.pdf}.}. The latest version of this text has been stored \& downloadable at the following link: \href{https://github.com/NQBH/hobby/blob/master/elementary_chemistry/inorganic_compound/NQBH_inorganic_compound.pdf}{GitHub\texttt{/}NQBH\texttt{/}hobby\texttt{/}elementary chemistry\texttt{/}grade 9\texttt{/}inorganic compound}\footnote{\textsc{url}: \url{https://github.com/NQBH/hobby/blob/master/elementary_chemistry/inorganic_compound/NQBH_inorganic_compound.pdf}.}.
	
	\textsf{\textbf{Keyword.} Inorganic compound.}
	\vspace{2mm}
	
	\textsf{[vi]} Tài liệu này là 1 bộ sưu tập các bài tập chọn lọc từ cơ bản đến nâng cao về \textit{phản ứng hóa học}, cũng là phần bài tập bổ sung cho tài liệu chính -- bài giảng \href{https://github.com/NQBH/hobby/blob/master/elementary_chemistry/grade_9/NQBH_elementary_chemistry_grade_9.pdf}{GitHub\texttt{/}NQBH\texttt{/}hobby\texttt{/}elementary chemistry\texttt{/}grade 9\texttt{/}lecture} của tác giả viết cho Hóa Học Sơ Cấp. Phiên bản mới nhất của tài liệu này được lưu trữ \& có thể tải xuống ở link sau: \href{https://github.com/NQBH/hobby/blob/master/elementary_chemistry/grade_9/real/NQBH_real.pdf}{GitHub\texttt{/}NQBH\texttt{/}hobby\texttt{/}elementary chemistry\texttt{/}grade 9\texttt{/}inorganic compound}.
	
	\textsf{\textbf{Từ khóa.} Hợp chất vô cơ.}
\end{abstract}
\setcounter{secnumdepth}{4}
\setcounter{tocdepth}{3}
\tableofcontents
\newpage

%------------------------------------------------------------------------------%

\section{\href{https://en.wikipedia.org/wiki/Inorganic_compound}{Wikipedia\texttt{/}Inorganic Compound}}
``In chemistry, an \textit{inorganic compound} is typically a \href{https://en.wikipedia.org/wiki/Chemical_compound}{chemical compound} that lacks \href{https://en.wikipedia.org/wiki/Carbon%E2%90%93hydrogen_bond}{carbon--hydrogen bonds}, i.e., a compound that is not an \href{https://en.wikipedia.org/wiki/Organic_compound}{organic compound}. The study of inorganic compounds is a subfield of chemistry known as \href{https://en.wikipedia.org/wiki/Inorganic_chemistry}{\textit{inorganic chemistry}}.

Inorganic compounds comprise most of the \href{https://en.wikipedia.org/wiki/Earth%27s_crust}{Earth's crust}, although the compositions of the deep \href{https://en.wikipedia.org/wiki/Mantle_(geology)}{mantle} remain active areas of investigation.

Some simple \href{https://en.wikipedia.org/wiki/Carbon}{carbon} compounds are often considered inorganic. Examples include the allotropes of carbon (\href{https://en.wikipedia.org/wiki/Graphite}{graphite}, \href{https://en.wikipedia.org/wiki/Diamond}{diamond}, \href{https://en.wikipedia.org/wiki/Buckminsterfullerene}{buckminsterfullerene}, etc.), \href{https://en.wikipedia.org/wiki/Carbon_monoxide}{carbon monoxide}, \href{https://en.wikipedia.org/wiki/Carbon_dioxide}{carbon dioxide}, \href{https://en.wikipedia.org/wiki/Carbide}{carbides}, \& the following \href{https://en.wikipedia.org/wiki/Salt_(chemistry)}{salts} of inorganic \href{https://en.wikipedia.org/wiki/Anion}{annions}: \href{https://en.wikipedia.org/wiki/Carbonate}{carbonates}, \href{https://en.wikipedia.org/wiki/Cyanide}{cyanides}, \href{https://en.wikipedia.org/wiki/Cyanate}{cyanates}, \& \href{https://en.wikipedia.org/wiki/Thiocyanate}{thiocyanates}. Many of these are normal parts of mostly organic systems, including \href{https://en.wikipedia.org/wiki/Organism}{organisms}; describing a chemical as inorganic does not necessarily mean that it does not occur within \href{https://en.wikipedia.org/wiki/Life}{living} things.'' -- \href{https://en.wikipedia.org/wiki/Inorganic_compound}{Wikipedia\texttt{/}inorganic compound}

%------------------------------------------------------------------------------%

\section{Oxide}

\subsection{Phân loại \& tính chất hóa học của oxide}

\begin{baitoan}[\cite{An_350_BT_Hoa_Hoc_9}, 1., p. 5]
	Nêu các base \& acid tương ứng của các oxide: \emph{\ce{SO2,SO3,N2O5,CaO,K2O,CuO,Mn2O7}}.
\end{baitoan}

\begin{baitoan}[\cite{An_350_BT_Hoa_Hoc_9}, 2., p. 5]
	Trong các oxide: \emph{CaO, \ce{Al2O3,NO,N2O5,CO2,SO2,MgO,CO,Fe2O3}}, oxide nào là oxide tạo muối.
\end{baitoan}

\begin{baitoan}[\cite{An_350_BT_Hoa_Hoc_9}, 3., p. 5]
	Cho các oxide: \emph{\ce{Na2O,Fe2O3,Fe3O4,SO3,CaO}}. Viết phương trình phản ứng (nếu có) khi cho các oxide này lần lượt tác dụng với nước, dung dịch \emph{NaOH}, dung dịch \emph{HCl}.
\end{baitoan}

\begin{baitoan}[\cite{An_350_BT_Hoa_Hoc_9}, 4., p. 6]
	(a) Cho các chất sau: \emph{\ce{CaCl2} (khan), \ce{P2O5,H2SO4} (đặc), \ce{Ba(OH)2} (rắn)}, chất nào được dùng để làm khô khí \emph{\ce{CO2}}? Giải thích bằng PTHH. (b) Có 4 oxide riêng biệt: \emph{\ce{Na2O,Al2O3,Fe2O3,MgO}}. Làm thế nào để có thể nhận biết được mỗi oxide bằng phương pháp hóa học với điều kiện chỉ được dùng thêm $2$ chất? 
\end{baitoan}

\begin{baitoan}[\cite{An_350_BT_Hoa_Hoc_9}, 5., p. 6]
	(a) Cho $a$ \emph{g Na} tác dụng với $p$ \emph{g} nước thu được dung dịch \emph{NaOH} nồng độ $x$\%. Cho $b$ \emph{g \ce{Na2O}} tác dụng với $p$ \emph{g} nước cũng thu được dung dịch \emph{NaOH} nồng độ $x$\%. Lập biểu thức tính $p$ theo $a,b$. (b) Khử hoàn toàn \emph{3.2 g} hỗn hợp \emph{CuO, \ce{Fe2O3}} bằng \emph{\ce{H2}} tạo ra \emph{0.9 g \ce{H2O}}. Tính khối lượng hỗn hợp kim loại thu được.
\end{baitoan}

\begin{baitoan}[\cite{An_350_BT_Hoa_Hoc_9}, 6., p. 7]
	(a) Cho \emph{2.24 L \ce{CO2}} (đktc) tác dụng hoàn toàn với \emph{25 g} dung dịch \emph{NaOH 20\%}. Tính khối lượng muối tạo thành. (b) Làm thế nào để nhận ra sự có mặt của mỗi khí trong hỗn hợp gồm \emph{\ce{CO,CO2,SO3}} bằng phương pháp hóa học. Viết các PTHH (nếu có).
\end{baitoan}

\begin{baitoan}[\cite{An_350_BT_Hoa_Hoc_9}, 7., pp. 8--9]
	(a) Nung $m$ \emph{g} hỗn hợp chất rắn A gồm \emph{\ce{Fe2O3}} \& \emph{FeO} với lượng thiếu khí \emph{CO} thu được hỗn hợp chất rắn B có khối lượng \emph{47.84 g} \& \emph{5.6 L \ce{CO2}}. Tính $m$. (b) Cho \emph{11.6 g} hỗn hợp \emph{\ce{Fe2O3}} \& \emph{FeO} có tỷ lệ số mol là $1:1$ vào \emph{300 mL} dung dịch \emph{HCl 2M} được dung dịch A. Tính nồng độ mol của các chất trong dung dịch sau phản ứng (thể tích dung dịch thay đổi không đáng kể).
\end{baitoan}

\begin{baitoan}[\cite{An_350_BT_Hoa_Hoc_9}, 8., p. 9]
	(a) Nung nóng kim loại M trong không khí đến khối lượng không đổi thu được chất rắn N. Khối lượng của M bằng $\frac{7}{10}$ khối lượng của N. Tìm CTPT của N. (b) Cho 1 oxide base tác dụng với dung dịch \emph{\ce{H2SO4} 24.5\%} thu được dung dịch 1 muối có nồng độ \emph{32.2\%}. Tìm CTPT của oxide base.
\end{baitoan}

\begin{baitoan}[\cite{An_350_BT_Hoa_Hoc_9}, 9., p. 11]
	(a) Dẫn $V$ \emph{L} khí \emph{\ce{CO2}} (đktc) qua \emph{250 mL} dung dịch \emph{\ce{Ca(OH)2} 1M} thấy có \emph{12.5 g} kết tủa. Tính $V$. (b) Dùng khí \emph{\ce{H2}} để khử $a$ \emph{g} oxide sắt. Sản phẩm hơi tạo ra cho qua $100$ \emph{g} acid \emph{\ce{H2SO4} 98\%} thì nồng độ acid giảm đi \emph{3.405\%}. Chất rắn thu được sau phản ứng trên cho tác dụng hết với dung dịch \emph{HCl} thấy thoát ra \emph{3.36 L} \emph{\ce{H2}} (đktc). Xác định CTPT oxide sắt.
\end{baitoan}

%------------------------------------------------------------------------------%

\section{Acid}

%------------------------------------------------------------------------------%

\section{Base}

%------------------------------------------------------------------------------%

\section{Salt -- Muối}

%------------------------------------------------------------------------------%

\printbibliography[heading=bibintoc]
	
\end{document}