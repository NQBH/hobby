\documentclass{article}
\usepackage[backend=biber,natbib=true,style=alphabetic,maxbibnames=50]{biblatex}
\addbibresource{/home/nqbh/reference/bib.bib}
\usepackage[utf8]{vietnam}
\usepackage{tocloft}
\renewcommand{\cftsecleader}{\cftdotfill{\cftdotsep}}
\usepackage[colorlinks=true,linkcolor=blue,urlcolor=red,citecolor=magenta]{hyperref}
\usepackage{amsmath,amssymb,amsthm,float,graphicx,mathtools,diagbox,tikz,tipa}
\usepackage[version=4]{mhchem}
\allowdisplaybreaks
\newtheorem{assumption}{Assumption}
\newtheorem{baitoan}{Bài toán}
\newtheorem{cauhoi}{Câu hỏi}
\newtheorem{conjecture}{Conjecture}
\newtheorem{corollary}{Corollary}
\newtheorem{dangtoan}{Dạng toán}
\newtheorem{definition}{Definition}
\newtheorem{dinhly}{Định lý}
\newtheorem{dinhnghia}{Định nghĩa}
\newtheorem{example}{Example}
\newtheorem{ghichu}{Ghi chú}
\newtheorem{hequa}{Hệ quả}
\newtheorem{hypothesis}{Hypothesis}
\newtheorem{lemma}{Lemma}
\newtheorem{luuy}{Lưu ý}
\newtheorem{nhanxet}{Nhận xét}
\newtheorem{notation}{Notation}
\newtheorem{note}{Note}
\newtheorem{principle}{Principle}
\newtheorem{problem}{Problem}
\newtheorem{proposition}{Proposition}
\newtheorem{question}{Question}
\newtheorem{remark}{Remark}
\newtheorem{theorem}{Theorem}
\newtheorem{thinghiem}{Thí nghiệm}
\newtheorem{vidu}{Ví dụ}
\usepackage[left=1cm,right=1cm,top=5mm,bottom=5mm,footskip=4mm]{geometry}

\title{Problem {\it\&} Solution: Inorganic Compound\\Bài Tập Hợp Chất Vô Cơ {\it\&} Lời Giải}
\author{Nguyễn Quản Bá Hồng\footnote{Independent Researcher, Ben Tre City, Vietnam\\e-mail: \texttt{nguyenquanbahong@gmail.com}; website: \url{https://nqbh.github.io}.}}
\date{\today}

\begin{document}
\maketitle
\begin{abstract}
	\textsf{[en]} This text is a collection of problems, from easy to advanced, about \textit{inorganic compound}, which is also a supplementary material for my lecture note on Elementary Chemistry, which is stored \& downloadable at the following link: \href{https://github.com/NQBH/hobby/blob/master/elementary_chemistry/grade_9/NQBH_elementary_chemistry_grade_9.pdf}{GitHub\texttt{/}NQBH\texttt{/}hobby\texttt{/}elementary chemistry\texttt{/}grade 9\texttt{/}lecture}\footnote{\textsc{url}: \url{https://github.com/NQBH/hobby/blob/master/elementary_chemistry/grade_9/NQBH_elementary_chemistry_grade_9.pdf}.}. The latest version of this text has been stored \& downloadable at the following link: \href{https://github.com/NQBH/hobby/blob/master/elementary_chemistry/inorganic_compound/NQBH_inorganic_compound.pdf}{GitHub\texttt{/}NQBH\texttt{/}hobby\texttt{/}elementary chemistry\texttt{/}grade 9\texttt{/}inorganic compound}\footnote{\textsc{url}: \url{https://github.com/NQBH/hobby/blob/master/elementary_chemistry/inorganic_compound/NQBH_inorganic_compound.pdf}.}.
	
	\textsf{\textbf{Keyword.} Inorganic compound.}
	\vspace{2mm}
	
	\textsf{[vi]} Tài liệu này là 1 bộ sưu tập các bài tập chọn lọc từ cơ bản đến nâng cao về \textit{phản ứng hóa học}, cũng là phần bài tập bổ sung cho tài liệu chính -- bài giảng \href{https://github.com/NQBH/hobby/blob/master/elementary_chemistry/grade_9/NQBH_elementary_chemistry_grade_9.pdf}{GitHub\texttt{/}NQBH\texttt{/}hobby\texttt{/}elementary chemistry\texttt{/}grade 9\texttt{/}lecture} của tác giả viết cho Hóa Học Sơ Cấp. Phiên bản mới nhất của tài liệu này được lưu trữ \& có thể tải xuống ở link sau: \href{https://github.com/NQBH/hobby/blob/master/elementary_chemistry/grade_9/real/NQBH_real.pdf}{GitHub\texttt{/}NQBH\texttt{/}hobby\texttt{/}elementary chemistry\texttt{/}grade 9\texttt{/}inorganic compound}.
	
	\textsf{\textbf{Từ khóa.} Hợp chất vô cơ.}
\end{abstract}
\setcounter{secnumdepth}{4}
\setcounter{tocdepth}{3}
\tableofcontents
\newpage

%------------------------------------------------------------------------------%

\section{Oxide}

\subsection{Qualitative problem -- Bài tập định tính}

\begin{baitoan}[\cite{SGK_KHTN_8_Canh_Dieu}, 1, p. 59]
	Trong các chất \emph{\ce{Na2SO4,P2O5,CaCO3,SO2}}, chất nào là oxide?\hfill{\sf Ans: \ce{P2O5,SO2}.}
\end{baitoan}

\begin{baitoan}[\cite{SGK_KHTN_8_Canh_Dieu}, 1, p. 59]
	Viết các PTHH xảy ra giữa oxygen \& các đơn chất để tạo ra các oxide sau: \emph{\ce{SO2,CuO,CO2,Na2O}}.
\end{baitoan}

\begin{proof}[Giải]
	\ce{S + O2 -> SO2, $2$Cu + O2 -> $2$CuO, C + O2 -> CO2, 4Na + O2 -> 2Na2O}.
\end{proof}

\begin{baitoan}[\cite{SGK_KHTN_8_Canh_Dieu}, 2, p. 60]
	Các oxide sau đây thuộc các loại oxide nào (oxide base, oxide acid, oxide lưỡng tính, oxide trung tính): \emph{\ce{Na2O,Al2O3,SO3,N2O}}.
\end{baitoan}

\begin{proof}[Giải]
	
\end{proof}

\begin{baitoan}[\cite{SGK_KHTN_8_Canh_Dieu}, 2, p. 60]
	Viết PTHH giữa các cặp chất sau: (a) \emph{\ce{H2SO4}, MgO}. (b) \emph{\ce{H2SO4}, CuO}. (c) \emph{HCl, \ce{Fe2O3}}.
\end{baitoan}

\begin{baitoan}[\cite{SGK_KHTN_8_Canh_Dieu}, 3, p. 61]
	Viết các PTHH xảy ra khi cho dung dịch \emph{KOH} phản ứng với các chất sau: \emph{\ce{SO2,CO2,SO3}}.
\end{baitoan}

\begin{baitoan}[\cite{SGK_Hoa_Hoc_9}, 1., p. 6]
	Có các oxide: \emph{CaO, \ce{Fe2O3,SO3}}. Oxide nào có thể tác dụng được với: (a) nước? (b) hydrochloric acid? (c) sodium hydroxide? Viết các PTHH.
\end{baitoan}

\begin{proof}[Giải]
	(a) Các oxide tác dụng với nước: CaO, \ce{SO3}. \ce{CaO + H2O -> Ca(OH)2, SO3 + H2O -> H2SO4}. (b) Các oxide tác dụng với hydrochloric acid: CaO, \ce{Fe2O3}: \ce{CaO + $2$HCl -> CaCl2 + H2O, Fe2O3 + $6$HCl -> $2$FeCl3 + $3$H2O}. (c) Các oxide tác dụng với sodium hydroxide: \ce{SO3}. \ce{SO3 + NaOH -> NaHSO4, SO3 + $2$NaOH -> Na2SO4 + H2O}.
\end{proof}

\begin{baitoan}[\cite{SGK_Hoa_Hoc_9}, 2., p. 6]
	Có các chất: \emph{\ce{H2O,KOH,K2O,CO2}}. Cho biết các cặp chất có thể tác dụng với nhau.
\end{baitoan}

\begin{proof}[Giải]
	Các cặp chất có thể tác dụng với nhau: \ce{H2O} \& \ce{CO2}, \ce{H2O} \& \ce{K2O}, \ce{CO2} \& \ce{K2O}, \ce{CO2} \& KOH. PTHH: \ce{CO2 + H2O -> H2CO3, K2O + H2O -> $2$KOH, K2O + CO2 -> K2CO3, CO2 + KOH -> KHCO3, CO2 + $2$KOH -> K2CO3 + H2O}.
\end{proof}

\begin{baitoan}[\cite{SGK_Hoa_Hoc_9}, 3., p. 6]
	Từ các chất: calcium oxide, lưu huỳnh dioxide, carbon dioxide, lưu huỳnh trioxide, zinc oxide, chọn chất thích hợp điền vào các sơ đồ phản ứng: (a) sulfuric acid $+$ $\ldots\to$ zinc sulfate $+$ nước. (b) sodium hydroxide $+$ $\ldots\to$ sodium sulfate $+$ nước. (c) nước $+$ $\ldots\to$ acid sulfurous. (d) nước $+$ $\ldots\to$ calcium hydroxide. (e) calcium oxide $+$ $\ldots\to$ calcium carbonate. Dùng các CTHH để viết tất cả các PTHH của các sơ đồ phản ứng trên.
\end{baitoan}

\begin{proof}[Giải]
	(a) sulfuric acid $+$ zinc oxide $\to$ zinc sulfate $+$ nước: \ce{H2SO4 + ZnO -> ZnSO4 + H2O}. (b) sodium hydroxide $+$ lưu huỳnh trioxide $\to$ sodium sulfate $+$ nước: \ce{$2$NaOH + SO3 -> Na2SO4 + H2O}. (c) nước $+$ lưu huỳnh dioxide $\to$ acid sulfurous: \ce{H2O + SO2 -> H2SO3}. (d) nước $+$ calcium oxide $\to$ calcium hydroxide: \ce{H2O + CaO -> Ca(OH)2}. (e) calcium oxide $+$ carbon dioxide $\to$ calcium carbonate: \ce{CaO + CO2 -> CaCO3 v}.
\end{proof}

\begin{baitoan}[\cite{SGK_Hoa_Hoc_9}, 4., p. 6]
	Cho các oxide: \emph{\ce{CO2,SO2,Na2O,CaO,CuO}}. Chọn các chất tác dụng được với: (a) nước, tạo thành dung dịch acid. (b) nước, tạo thành dung dịch base. (c) dung dịch acid, tạo thành muối \& nước. (d) dung dịch base, tạo thành muối \& nước. Viết các PTHH.
\end{baitoan}

\begin{proof}[Giải]
	(a) \ce{CO2,SO2} tác dụng với nước tạo thành dung dịch acid: \ce{CO2 + H2O -> H2CO3, SO2 + H2O -> H2SO3}. (b) \ce{Na2O}, CaO tác dụng với nước tạo thành dung dịch base: \ce{Na2O + H2O -> $2$NaOH, CaO + H2O -> Ca(OH)2}. (c) \ce{Na2O}, CaO, CuO tác dụng với dung dịch acid tạo thành muối \& nước: \ce{Na2O + $2$HCl -> $2$HCl + H2O, CaO + $2$HNO3 -> Ca(NO3)2 + H2O, CuO + H2SO4 -> CuSO4 + H2O}. (d) \ce{CO2,SO2} tác dụng với dung dịch base tạo thành muối \& nước: \ce{CO2 + Ca(OH)2 -> CaCO3 v + H2O, SO2 + Ca(OH)2 -> CaSO3 v + H2O}.
\end{proof}

\begin{baitoan}[\cite{SGK_Hoa_Hoc_9}, 5., p. 6]
	Có hỗn hợp khí \emph{\ce{CO2,O2}}. Làm thế nào để có thể thu được khí \emph{\ce{O2}} từ hỗn hợp trên? Trình bày cách làm \& viết PTHH.
\end{baitoan}

\begin{proof}[Giải]
	Dẫn hỗn hợp khí \ce{CO2,O2} đi qua bình đựng dung dịch kiềm (lấy dư), e.g., \ce{Ca(OH)2,NaOH}, $\ldots$, khí \ce{CO2} bị hấp thụ hết do có phản ứng với kiềm: \ce{CO2 + Ca(OH)2 -> CaCO2 v + H2O} hoặc \ce{CO2 + $2$NaOH -> Na2CO3 + H2O}. Khí thoát ra khỏi bình chỉ có \ce{O2} nên sẽ thu được khí \ce{O2}.
\end{proof}

\begin{baitoan}[\cite{SGK_Hoa_Hoc_9}, 1., p. 9]
	Bằng phương pháp hóa học nào có thể nhận biết được từng chất trong mỗi dãy chất sau? (a) 2 chất rắn màu trắng \emph{CaO, \ce{Na2O}}. (b) 2 chất khí không màu \emph{\ce{CO2,O2}}. Viết các PTHH.
\end{baitoan}

\begin{proof}[Giải]
	(a) Lấy mỗi chất cho vào mỗi cốc đựng nước, khuấy cho đến khi chất cho vào không tan nữa. Lọc để thu lấy 2 dung dịch. Dẫn khí \ce{CO2} vào mỗi dung dịch. Dung dịch nào xuất hiện kết tủa thì đó là dung dịch \ce{Ca(OH)2}, tương ứng với cốc lúc đầu là CaO. Dung dịch nào không thấy kết tủa thì tương ứng với cốc lúc đầu là \ce{Na2O}. PTHH: \ce{Na2O + H2O -> $2$NaOH, CaO + H2O -> Ca(OH)2, CO2 + $2$NaOH -> Na2CO3 + H2O, CO2 + Ca(OH)2 -> CaCO3 v + H2O}. (b) \textit{Cách 1.} Cho tàn đóm đỏ vào từng khí. Khí nào làm tàn đóm bùng cháy trở lại là khí \ce{O2}, còn lại là \ce{CO2}. \textit{Cách 2.} Sục 2 chất khí không màu vào 2 ống nghiệm chứa nước vôi \ce{Ca(OH)2} trong. Ống nghiệm nào bị vẩn đục, thì khí ban đầu là \ce{CO2}: \ce{CO2 + Ca(OH)2 -> CaCO3 v + H2O}, khí còn lại là \ce{O2}.
\end{proof}

\begin{baitoan}[\cite{SGK_Hoa_Hoc_9}, 2., p. 9]
	Nhận biết từng chất trong mỗi nhóm chất sau bằng phương pháp hóa học. (a) \emph{CaO, \ce{CaCO3}}. (b) \emph{CaO, MgO}. Viết các PTHH.
\end{baitoan}

\begin{proof}[Giải]
	(a) Lấy mỗi chất cho vào ống nghiệm hoặc cốc chứa sẵn nước. Ở ống nào thấy chất rắn tan \& nóng lên, chất cho vào là CaO. Ở ống nghiệm nào thấy chất rắn không tan \& không nóng lên, chất cho vào là \ce{CaCO3}. PTHH: \ce{CaO + H2O -> Ca(OH)2}. (b) Lấy mỗi chất cho vào ống nghiệm hoặc cốc chứa sẵn nước. Ở ống nào thấy chất rắn tan \& nóng lên, chất cho vào là CaO. Ở ống nghiệm nào thấy chất rắn không tan \& không nóng lên, chất cho vào là MgO. PTHH: \ce{CaO + H2O -> Ca(OH)2}.
\end{proof}

\begin{baitoan}[\cite{SGK_Hoa_Hoc_9}, 1., p. 11]
	Viết PTHH cho mỗi chuyển đổi: (a) \emph{S $\to$ \ce{SO2} $\to$ \ce{CaSO3}}. (b) \emph{\ce{SO2} $\to$ \ce{Na2SO3}}. (c) \emph{\ce{SO2} $\to$ \ce{H2SO3} $\to$ \ce{Na2SO3} $\to$ \ce{SO2}}.
\end{baitoan}

\begin{proof}[Giải]
	(a) \ce{S + O2 ->[$t^\circ$] SO2, SO2 + CaO -> CaSO3} hoặc \ce{SO2 + Ca(OH)2 -> CaSO3 + H2O}. (b) \ce{SO2 + H2O -> H2SO3, Na2O + H2SO3 -> Na2SO3 + H2O, Na2SO3 + H2SO4 -> Na2SO4 + SO2 ^ + H2O}. (c) \ce{SO2 + $2$NaOH -> Na2SO3 + H2O} hoặc \ce{Na2O + SO2 -> Na2SO3}.
\end{proof}

\begin{baitoan}[\cite{SGK_Hoa_Hoc_9}, 2., p. 11]
	Nhận biết từng chất trong mỗi nhóm chất sau bằng phương pháp hóa học. (a) 2 chất rắn màu trắng \emph{CaO, \ce{P2O5}}. (b) 2 chất khí không màu \emph{\ce{SO2,O2}}. Viết các PTHH.
\end{baitoan}

\begin{proof}[1st giải]
	(a) Cho nước vào 2 ống nghiệm có chứa CaO \& \ce{P2O5}. Sau đó cho quỳ tím vào mỗi dung dịch. Dung dịch nào làm đổi màu quỳ tím thành xanh là dung dịch base, tương ứng với chất ban đầu là CaO. Dung dịch nào làm đổi màu quỳ tím thành đỏ là dung dịch acid, chất ban đầu là \ce{P2O5}. PTHH: \ce{CaO + H2O -> Ca(OH)2, P2O5 + $3$H2O -> $2$H3PO4}. (b) Lấy mẫu thử từng khí. Lấy quỳ tím ẩm cho vào từng mẫu thử. Mẫu nào làm quỳ tím hóa đỏ là \ce{SO2}, còn lại là \ce{O2}. PTHH: \ce{SO2 + H2O -> H2SO3}.
\end{proof}

\begin{proof}[2nd giải]
	(a) Cho nước vào 2 ống nghiệm có chứa CaO \& \ce{P2O5}. Sau đó cho phenolphthalein vào mỗi dung dịch. Dung dịch nào hóa hồng là dung dịch base, tương ứng với chất ban đầu là CaO. Dung dịch nào không đổi màu là dung dịch acid, chất ban đầu là \ce{P2O5}. (b) Dẫn lần lượt từng khí vào dung dịch nước vôi trong, nếu có kết tủa xuất hiện thì khí dẫn vào là \ce{SO2}: \ce{SO2 + Ca(OH)2 -> CaSO3 v + H2O}. Nếu không có hiện tượng gì thì khí dẫn vào là khí \ce{O2}. Hoặc có thể đưa que đóm con than hồng vào 2 khí, que đóm sẽ bùng cháy trong khí \ce{O2}.
\end{proof}

\begin{baitoan}[\cite{SGK_Hoa_Hoc_9}, 3., p. 11]
	Có các khí ẩm (khí có lẫn hơi nước): carbon dioxide, hydrogen, oxygen, lưu huỳnh dioxide. Khí nào có thể được làm khô bằng calcium oxide? Giải thích.
\end{baitoan}

\begin{baitoan}[\cite{SGK_Hoa_Hoc_9}, 4., p. 11]
	Có những chất khí sau: \emph{\ce{CO2,H2,O2,SO2,N2}}. Cho biết chất nào có tính chất sau: (a) nặng hơn không khí. (b) nhẹ hơn không khí. (c) cháy được trong không khí. (d) tác dụng với nước tạo thành dung dịch acid. (e) làm đục nước vôi trong. (f) đổi màu giấy quỳ tím ẩm thành đỏ.
\end{baitoan}

\begin{baitoan}[\cite{SGK_Hoa_Hoc_9}, 5., p. 11]
	Khí lưu huỳnh dioxide được tạo thành từ cặp chất nào sau đây? (a) \emph{\ce{K2SO3,H2SO4}}. (b) \emph{\ce{K2SO4}, HCl}. (c) \emph{\ce{Na2SO3}, NaOH}. (d) \emph{\ce{Na2SO4,CuCl2}}. (e) \emph{\ce{Na2SO3}, NaCl}. Viết PTHH.
\end{baitoan}

\begin{baitoan}[\cite{SBT_Hoa_Hoc_9}, 1.1., p. 3]
	Có các oxide: \emph{\ce{H2O,SO2,CuO,CO2}, CaO, MgO}. Cho biết các chất nào có thể điều chế bằng: (a) phản ứng hóa hợp? Viết PTHH. (b) phản ứng phân hủy? Viết PTHH.
\end{baitoan}

\begin{baitoan}[\cite{SBT_Hoa_Hoc_9}, 1.2., p. 3]
	Viết CTHH \& tên gọi của: (a) $5$ oxide base. (b) $5$ oxide acid.
\end{baitoan}

\begin{baitoan}[\cite{SBT_Hoa_Hoc_9}, 1.3., p. 3]
	Khí carbon monooxide \emph{CO} có lẫn các tạp chất là khí carbon dioxide \emph{\ce{CO2}} \& lưu huỳnh dioxide \emph{\ce{SO2}}. Làm thế nào tách được các tạp chất ra khỏi \emph{CO}? Viết các PTHH.
\end{baitoan}

\begin{baitoan}[\cite{SBT_Hoa_Hoc_9}, 1.4., p. 3]
	Tìm CTHH của các oxide có thành phần khối lượng: (a) \emph{S: 50\%}. (b) \emph{C: 42.8\%}. (c) \emph{Mn: 49.6\%}. (d) \emph{Pb: 86.6\%}.
\end{baitoan}

\begin{baitoan}[\cite{SBT_Hoa_Hoc_9}, 2.1., p. 4]
	Kim loại M tác dụng với dung dịch \emph{HCl} sinh ra khí hydrogen. Dẫn khí hydrogen đi qua oxide của kim loại N nung nóng. Oxide này bị khử cho kim loại N. M \& N là: {\sf A.} copper \& chì. {\sf B.} zinc \& copper. {\sf C.} chì \& zinc. {\sf D.} copper \& silver.
\end{baitoan}

\begin{baitoan}[\cite{SBT_Hoa_Hoc_9}, 2.2., p. 4]
	Calcium oxide tiếp xúc lâu ngày với không khí sẽ bị giảm chất lượng. Giải thích hiện tượng này \& minh họa bằng PTHH.
\end{baitoan}

\begin{baitoan}[\cite{SBT_Hoa_Hoc_9}, 2.3., p. 4]
	Viết các PTHH thực hiện các chuyển đổi hóa học theo sơ đồ: (a) \emph{CaO $\to$ \ce{Ca(OH)2} $\to$ \ce{CaCO3} $\to$ CaO $\to$ \ce{CaCl2}}. (b) \emph{CaO $\to$ \ce{CaCO3}}.
\end{baitoan}

\begin{baitoan}[\cite{SBT_Hoa_Hoc_9}, 2.9., p. 5]
	Điền các chất: \emph{CuO, CO, \ce{H2,SO3,P2O5,H2O}} thích hợp vào các sơ đồ phản ứng: (a) $\ldots$ \emph{\ce{+ H2O -> H2SO4}}. (b) \emph{\ce{H2O + $\ldots$ -> H3PO4}}. (c) $\ldots$ \emph{\ce{+ HCl -> CuCl2 + H2O}}. (d) $\ldots$ \emph{\ce{+ H2SO4 -> CuSO4 +} $\ldots$}. (e) \emph{\ce{CuO + $\ldots$ ->[$t^\circ$] Cu + H2O}}.
\end{baitoan}

\begin{baitoan}[\cite{SBT_Hoa_Hoc_9}, 2.4., p. 4]
	\emph{CaO} là oxide base, \emph{\ce{P2O5}} là oxide acid. Chúng đều là các chất rắn, màu trắng. Bằng các phương pháp hóa học nào có thể giúp ta nhận biết được mỗi chất trên?
\end{baitoan}

\begin{baitoan}[\cite{An_350_BT_Hoa_Hoc_9}, 1., p. 5]
	Nêu các base \& acid tương ứng của các oxide: \emph{\ce{SO2,SO3,N2O5,CaO,K2O,CuO,Mn2O7}}.
\end{baitoan}

\begin{baitoan}[\cite{An_350_BT_Hoa_Hoc_9}, 2., p. 5]
	Trong các oxide: \emph{CaO, \ce{Al2O3,NO,N2O5,CO2,SO2,MgO,CO,Fe2O3}}, oxide nào là oxide tạo muối.
\end{baitoan}

\begin{baitoan}[\cite{An_350_BT_Hoa_Hoc_9}, 3., p. 5]
	Cho các oxide: \emph{\ce{Na2O,Fe2O3,Fe3O4,SO3,CaO}}. Viết phương trình phản ứng (nếu có) khi cho các oxide này lần lượt tác dụng với nước, dung dịch \emph{NaOH}, dung dịch \emph{HCl}.
\end{baitoan}

\begin{baitoan}[\cite{An_350_BT_Hoa_Hoc_9}, 4.a, p. 6]
	Cho các chất sau: \emph{\ce{CaCl2} (khan), \ce{P2O5,H2SO4} (đặc), \ce{Ba(OH)2} (rắn)}, chất nào được dùng để làm khô khí \emph{\ce{CO2}}? Giải thích bằng PTHH.
\end{baitoan}

\begin{baitoan}[\cite{An_350_BT_Hoa_Hoc_9}, 4.b, p. 6]
	Có 4 oxide riêng biệt: \emph{\ce{Na2O,Al2O3,Fe2O3,MgO}}. Làm thế nào để có thể nhận biết được mỗi oxide bằng phương pháp hóa học với điều kiện chỉ được dùng thêm $2$ chất?
\end{baitoan}

\begin{baitoan}[\cite{An_350_BT_Hoa_Hoc_9}, 6.b, p. 7]
	Làm thế nào để nhận ra sự có mặt của mỗi khí trong hỗn hợp gồm \emph{\ce{CO,CO2,SO3}} bằng phương pháp hóa học. Viết các PTHH (nếu có).
\end{baitoan}

\subsection{Quantitative problem -- Bài tập định lượng}

\begin{baitoan}[\cite{SGK_Hoa_Hoc_9}, 6., p. 6]
	Cho \emph{1.6 g} copper(II) oxide tác dụng với \emph{100 g} dung dịch acid sulfuric có nồng độ \emph{20\%}. (a) Viết PTHH. (b) Tính nồng độ \% của các chất có trong dung dịch sau khi phản ứng kết thúc.
\end{baitoan}

\begin{proof}[Giải]
	$m_{\ce{H2SO4}} = m_{\rm dd\ce{H2SO4}}C\% = 100\cdot20\% = 20$ g. $n_{\rm CuO} = \dfrac{m_{\rm CuO}}{M_{\rm CuO}} = \dfrac{1.6}{80} = 0.02$ mol, $n_{\ce{H2SO4}} = \dfrac{m_{\ce{H2SO4}}}{M_{\ce{H2SO4}}} = \dfrac{20}{98} = \dfrac{10}{49}$ mol. (a) PTHH: \ce{CuO + H2SO4 -> CuSO4 + H2O}. Vì $\dfrac{n_{\rm CuO}}{1} < \dfrac{n_{\ce{H2SO4}}}{1}$ ($0.02 < \frac{10}{49}$) nên CuO phản ứng hết, \ce{H2SO4} dư, suy ra khối lượng \ce{CuSO4} tạo thành \& \ce{H2SO4} phản ứng tính theo số mol CuO. (b) Dung dịch sau phản ứng có 2 chất tan: \ce{CuSO4} \& \ce{H2SO4} còn dư. $C\%_{\ce{CuSO4}} = \dfrac{m_{\ce{CuSO4}}}{m_{\rm dd}}\cdot100\% = \dfrac{0.02\cdot160}{1.6 + 100}\cdot100\% = \dfrac{400}{127}\%\approx3.1496063\%$. $C\%_{\ce{H2SO4}} = \dfrac{m_{\ce{H2SO4}\footnotesize\mbox{dư}}}{m_{\rm dd}}\cdot100\% = \dfrac{20 - 0.02\cdot98}{1.6 + 100}\cdot100\% = \dfrac{45100}{2540}\%\approx17.756\%$. Vậy $C\%_{\ce{CuSO4}}\approx3.1496063\%$, $C\%_{\ce{H2SO4}}\approx17.756\%$.
\end{proof}

\begin{baitoan}[Mở rộng \cite{SGK_Hoa_Hoc_9}, 6., p. 6]
	Cho $m_1$ \emph{g} copper(II) oxide tác dụng với $m_2$ \emph{g} dung dịch acid sulfuric có nồng độ $C\%$. Tính nồng độ \% của các chất có trong dung dịch sau khi phản ứng kết thúc theo $m_1,m_2,C\%$ biết sẽ lọc ra \emph{CuO} khỏi dung dịch nếu \emph{CuO} dư.
\end{baitoan}

\begin{proof}[Giải]
	$m_{\ce{H2SO4}} = m_{\rm dd\ce{H2SO4}}C\% = m_2C\%$ g, $n_{\rm CuO} = \dfrac{m_{\rm CuO}}{M_{\rm CuO}} = \dfrac{m_1}{80}$ mol, $n_{\ce{H2SO4}} = \dfrac{m_{\ce{H2SO4}}}{M_{\ce{H2SO4}}} = \dfrac{m_2C\%}{98}$ mol. PTHH: \ce{CuO + H2SO4 -> CuSO4 + H2O}. Theo định luật bảo toàn khối lượng, $m_{\rm dd} = m_{\rm CuO\footnotesize\mbox{pư}} + m_{\rm dd\ce{H2SO4}} = m_{\rm CuO} + m_{\rm dd\ce{H2SO4}} - m_{\rm CuO\footnotesize\mbox{dư}} = m_1 + m_2 - m_{\rm CuO\footnotesize\mbox{dư}}$ g. Xét 2 trường hợp:
	\begin{itemize}
		\item[(a)] Nếu $n_{\rm CuO} < n_{\ce{H2SO4}}$, i.e., nếu $m_1,m_2,C\%$ thỏa $\frac{m_1}{80} < \frac{m_2C\%}{98}$ thì CuO phản ứng hết, \ce{H2SO4} dư, suy ra $n_{\rm CuO} = n_{\ce{H2SO4}\footnotesize\mbox{pư}} = n_{\ce{CuSO4}} = \dfrac{m_1}{80}$ mol, $m_{\ce{H2SO4}\footnotesize\mbox{dư}} = m_{\ce{H2SO4}} - m_{\ce{H2SO4}\footnotesize\mbox{pư}} = m_2C\% - 98\dfrac{m_1}{80}$. Dung dịch sau phản ứng có 2 chất tan: \ce{CuSO4} \& \ce{H2SO4} còn dư.
		\begin{align*}
			C\%_{\ce{CuSO4}} &= \frac{m_{\ce{CuSO4}}}{m_{\rm dd}}\cdot100\% = \frac{n_{\ce{CuSO4}}M_{\ce{CuSO4}}}{m_{\rm dd}}\cdot100\% = \frac{\dfrac{m_1}{80}\cdot160}{m_1 + m_2}\cdot100\% = \frac{200m_1}{m_1 + m_2}\%,\\
			C\%_{\ce{H2SO4}} &= \frac{m_{\ce{H2SO4}\footnotesize\mbox{dư}}}{m_{\rm dd}}\cdot100\% = \frac{100\left(m_2C\% - \dfrac{98m_1}{80}\right)}{m_1 + m_2}\% = \frac{100m_2C\% - 122.5m_1}{m_1 + m_2}\%.
		\end{align*}
		\item[(b)] Nếu $n_{\rm CuO} = n_{\ce{H2SO4}}$, i.e., nếu $m_1,m_2,C\%$ thỏa $\frac{m_1}{80} = \frac{m_2C\%}{98}$ thì cả CuO \& \ce{H2SO4} đều phản ứng hết. Dung dịch sau phản ứng có duy nhất 1 chất tan \ce{CuSO4} \& $n_{\ce{CuSO4}} = n_{\rm CuO} = n_{\ce{H2SO4}} = \dfrac{m_1}{80}$:
		\begin{align*}
			C\%_{\ce{CuSO4}} &= \frac{m_{\ce{CuSO4}}}{m_{\rm dd}}\cdot100\% = \frac{n_{\ce{CuSO4}}M_{\ce{CuSO4}}}{m_{\rm dd}}\cdot100\% = \frac{\frac{m_1}{80}\cdot160}{m_1 + m_2}\cdot100\% = \frac{200m_1}{m_1 + m_2}\%.
		\end{align*}
		\item[(c)] Nếu $n_{\rm CuO} > n_{\ce{H2SO4}}$, i.e., $\frac{m_1}{80} > \frac{m_2C\%}{98}$ thì \ce{H2SO4} phản ứng hết, CuO dư, suy ra $n_{\rm CuO\footnotesize\mbox{pư}} = n_{\ce{H2SO4}} = n_{\ce{CuSO4}} = \frac{m_2C\%}{98}$. Dung dịch sau phản ứng chỉ có duy nhất 1 chất tan \ce{CuSO4} \&
		\begin{align*}
			C\%_{\ce{CuSO4}} &= \frac{m_{\ce{CuSO4}}}{m_{\rm dd}}\cdot100\% = \frac{n_{\ce{CuSO4}}M_{\ce{CuSO4}}}{m_{\rm dd}}\cdot100\% = \frac{160\cdot\dfrac{m_2C\%}{98}}{\dfrac{m_2C\%}{98}\cdot80 + m_2} = \frac{80C\%}{40C\% + 49},
		\end{align*}
		không phụ thuộc vào $m_2$.
	\end{itemize}
	Vậy nồng độ \% của các chất có trong dung dịch sau khi phản ứng kết thúc:
	\begin{equation*}
		C\%_{\ce{CuSO4}} = \left\{\begin{split}
			&\frac{200m_1}{m_1 + m_2}\%,&&\mbox{nếu }\frac{m_1}{80}\le\frac{m_2C\%}{98},\\
			&\frac{80C\%}{40C\% + 49},&&\mbox{nếu }\frac{m_1}{80} > \frac{m_2C\%}{98},
		\end{split}\right.
	\end{equation*}
	\begin{equation*}
		C\%_{\ce{H2SO4}} = \left\{\begin{split}
			&\frac{100m_2C\% - 122.5m_1}{m_1 + m_2}\%,&&\mbox{nếu }\frac{m_1}{80} < \frac{m_2C\%}{98},\\
			&0,&&\mbox{nếu }\frac{m_1}{80}\ge\frac{m_2C\%}{98},
		\end{split}\right. = \frac{100\max\left\{m_2C\% - \frac{49}{40}m_1,0\right\}}{m_1 + m_2}\%.
	\end{equation*}
\end{proof}

\begin{baitoan}[\cite{SGK_Hoa_Hoc_9}, 3., p. 9]
	\emph{200 mL} dung dịch \emph{HCl} có nồng độ \emph{3.5M} hòa tan vừa hết \emph{20 g} hỗn hợp 2 oxide \emph{CuO, \ce{Fe2O3}}. (a) Viết các PTHH. (b) Tính khối lượng của mỗi oxide có trong mỗi hỗn hợp ban đầu.
\end{baitoan}

\begin{proof}[Giải]
	$n_{\rm HCl} = C_{\rm M,HCl}V_{\rm ddHCl} = 3.5\cdot0.2 = 0.7$ mol. Đặt $x = n_{\rm CuO}$, $y = n_{\ce{Fe2O3}}$. (a) PTHH: \ce{CuO + $2$HCl -> CuCl2 + H2O, Fe2O3 + $6$HCl -> $2$FeCl3 + $3$H2O}. (b) Có $n_{\rm HCl} = 2x + 6y = 0.7$ mol, $m_{\rm hh} = 80x + 160y = 20$ g, nên ta có hệ phương trình:
	\begin{equation*}
		\left\{\begin{split}
			2x + 6y &= 0.7,\\
			80x + 160y &= 20,
		\end{split}\right.\Leftrightarrow\left\{\begin{split}
			x &= 0.05,\\
			y &= 0.1.
		\end{split}\right.
	\end{equation*}
	$n_{\rm CuO} = 0.05$ mol $\Rightarrow m_{\rm CuO} = n_{\rm CuO}M_{\rm CuO} = 0.05\cdot80 = 4$ g, $n_{\ce{Fe2O3}} = 0.1$ mol $\Rightarrow m_{\ce{Fe2O3}} = n_{\ce{Fe2O3}}M_{\ce{Fe2O3}} = 0.1\cdot160 = 16$ g (hoặc $m_{\ce{Fe2O3}} = m_{\rm hh} - m_{\rm CuO} = 20 - 4 = 16$ g).
\end{proof}

\begin{baitoan}[Mở rộng \cite{SGK_Hoa_Hoc_9}, 3., p. 9]
	$V$ \emph{L} dung dịch \emph{HCl} có nồng độ $C_{\rm M}$\emph{M} hòa tan vừa hết $m$ \emph{g} hỗn hợp 2 oxide \emph{CuO, \ce{Fe2O3}}. Tính khối lượng của mỗi oxide có trong mỗi hỗn hợp ban đầu.
\end{baitoan}

\begin{proof}[Giải]
	$n_{\rm HCl} = C_{\rm M,HCl}V_{\rm ddHCl} = C_{\rm M}V$ mol. Đặt $x = n_{\rm CuO}$, $y = n_{\ce{Fe2O3}}$. PTHH: \ce{CuO + $2$HCl -> CuCl2 + H2O, Fe2O3 + $6$HCl -> $2$FeCl3 + $3$H2O}. Có $n_{\rm HCl} = 2x + 6y = C_{\rm M}V$ mol, $m_{\rm hh} = 80x + 160y = m$ g, nên ta có hệ phương trình:
	\begin{equation*}
		\left\{\begin{split}
			2x + 6y &= C_{\rm M}V,\\
			80x + 160y &= m,
		\end{split}\right.\Leftrightarrow\left\{\begin{split}
			x + 3y &= \frac{C_{\rm M}V}{2},\\
			x + 2y &= \frac{m}{80}.
		\end{split}\right.\Leftrightarrow\left\{\begin{split}
			x &= \frac{3m}{80} - C_{\rm M}V,\\
			y &= \frac{C_{\rm M}V}{2} - \frac{m}{80}.
		\end{split}\right.
	\end{equation*}
	$n_{\rm CuO} = \dfrac{3m}{80} - C_{\rm M}V$ mol $\Rightarrow m_{\rm CuO} = n_{\rm CuO}M_{\rm CuO} = 80\left(\dfrac{3m}{80} - C_{\rm M}V\right) = 3m - 80C_{\rm M}V$ g, $n_{\ce{Fe2O3}} = \dfrac{C_{\rm M}V}{2} - \dfrac{m}{80}$ mol $\Rightarrow m_{\ce{Fe2O3}} = n_{\ce{Fe2O3}}M_{\ce{Fe2O3}} = 160\left(\dfrac{C_{\rm M}V}{2} - \dfrac{m}{80}\right) = 80C_{\rm M}V - 2m$ g (hoặc $m_{\ce{Fe2O3}} = m_{\rm hh} - m_{\rm CuO} = m - (3m - 80C_{\rm M}V) = 80C_{\rm M}V - 2m$ g). Vậy $m_{\rm CuO} = 3m - 80C_{\rm M}V$ g, $m_{\ce{Fe2O3}} = 80C_{\rm M}V - 2m$ g.
\end{proof}

\begin{baitoan}[\cite{SGK_Hoa_Hoc_9}, 4., p. 9]
	Biết \emph{2.24 L} khí \emph{\ce{CO2}} (đktc) tác dụng vừa hết với \emph{200 mL} dung dịch \emph{\ce{Ba(OH)2}}, sản phẩm là \emph{\ce{BaCO3,H2O}}. (a) Viết PTHH. (b) Tính nồng độ mol của dung dịch \emph{\ce{Ba(OH)2}} đã dùng. (c) Tính khối lượng chất kết tủa thu được.
\end{baitoan}

\begin{proof}[Giải]
	$n_{\ce{CO2}} = \dfrac{V_{\ce{CO2}}}{22.4} = \dfrac{2.24}{22.4} = 0.1$ mol. (a) \ce{CO2 + Ba(OH)2 -> BaCO3 v + H2O}. (b) Vì \ce{CO2} tác dụng vừa hết nên $n_{\ce{Ba(OH)2}} = n_{\ce{CO2}} = 0.1$ mol. $C_{\rm M,\ce{Ba(OH)2}} = \dfrac{n_{\ce{Ba(OH)2}}}{V_{\rm dd\ce{Ba(OH)2}}} = \dfrac{0.1}{0.2} = 0.5$M. (c) Chất kết tủa sau phản ứng là \ce{BaCO3} \& $n_{\ce{BaCO3}} = n_{\ce{CO2}} = 0.1$ mol $\Rightarrow m_{\ce{BaCO3}} = n_{\ce{BaCO3}}M_{\ce{BaCO3}} = 0.1\cdot197 = 19.7$ g.
\end{proof}

\begin{baitoan}[Mở rộng \cite{SGK_Hoa_Hoc_9}, 4., p. 9]
	Cho $V_1$ \emph{L} khí \emph{\ce{CO2}} (đktc) tác dụng với $V_2$ \emph{L} dung dịch \emph{\ce{Ba(OH)2}} nồng độ $C_{\rm M}$\emph{M}. (a) Viết PTHH. (b) Tính nồng độ mol của dung dịch \emph{\ce{Ba(OH)2}} đã dùng \& khối lượng chất kết tủa thu được theo $V_1,V_2,C_{\rm M}$.
\end{baitoan}

\begin{baitoan}[\cite{SGK_Hoa_Hoc_9}, 6., p. 11]
	Dẫn \emph{112 mL} khí \emph{\ce{SO2}} (đktc) đi qua \emph{700 mL} dung dịch \emph{\ce{Ca(OH)2}} có nồng độ \emph{0.01M}, sản phẩm là muối calcium sulfite. (a) Viết PTHH. (b) Tính khối lượng các chất sau phản ứng.
\end{baitoan}

\begin{baitoan}[\cite{SBT_Hoa_Hoc_9}, 1.5., p. 3]
	Biết \emph{1.12 L} khí carbon dioxide (đktc) tác dụng vừa đủ với \emph{100 mL} dung dịch \emph{NaOH} tạo ra muối trung hòa. (a) Viết PTHH. (b) Tính nồng độ mol của dung dịch \emph{NaOH} đã dùng.	
\end{baitoan}

\begin{baitoan}[\cite{SBT_Hoa_Hoc_9}, 1.6., p. 3]
	Cho \emph{15.3 g} oxide của kim loại hóa trị $2$ vào nước thu được \emph{200 g} dung dịch base với nồng độ \emph{8.55\%}. Xác định công thức của oxide trên.
\end{baitoan}

\begin{baitoan}[\cite{SBT_Hoa_Hoc_9}, 1.7., p. 3]
	Cho \emph{38.4 g} 1 oxide acid của phi kim X có hóa trị $4$ tác dụng vừa đủ với dung dịch \emph{NaOH} thu được \emph{400 g} dung dịch muối nồng độ \emph{18.9\%}. Xác định công thức của oxide.
\end{baitoan}

\begin{baitoan}[\cite{SBT_Hoa_Hoc_9}, 2.5., p. 4]
	1 loại đá vôi chứa \emph{80\% \ce{CaCO3}}. Nung $1$ tấn đá vôi loại này có thể thu được bao nhiêu \emph{kg} vôi sống \emph{CaO}, nếu hiệu suất là \emph{85\%}?
\end{baitoan}

\begin{baitoan}[\cite{SBT_Hoa_Hoc_9}, 2.6., p. 4]
	Để tôi vôi, người ta đã dùng 1 khối lượng nước bằng \emph{70\%} khối lượng vôi sống. Cho biết khối lượng nước đã dùng lớn hơn bao nhiêu lần so với khối lượng nước tính theo PTHH?
\end{baitoan}

\begin{baitoan}[\cite{SBT_Hoa_Hoc_9}, 2.7., p. 4]
	Cho \emph{8 g} lưu huỳnh trioxide \emph{\ce{SO3}} tác dụng với \emph{\ce{H2O}}, thu được \emph{250 mL} dung dịch acid sulfuric \emph{\ce{H2SO4}}. (a) Viết PTHH. (b) Xác định nồng độ mol của dung dịch acid thu được.
\end{baitoan}

\begin{baitoan}[\cite{SBT_Hoa_Hoc_9}, 2.8., p. 4]
	Dẫn \emph{1.12 L} khí lưu huỳnh dioxide (đktc) đi qua \emph{700 mL} dung dịch \emph{\ce{Ca(OH)2} 0.1M}. (a) Viết PTHH. (b) Tính khối lượng các chất sau phản ứng.
\end{baitoan}

\begin{baitoan}[\cite{SBT_Hoa_Hoc_9}, 2.10., p. 4]
	Nung nóng \emph{13.1 g} 1 hỗn hợp gồm \emph{Mg, Zn, Al} trong không khí đến phản ứng hoàn toàn thu được \emph{20.3 g} hỗn hợp gồm \emph{MgO, ZnO, \ce{Al2O3}}. Hòa tan \emph{20.3 g} hỗn hợp oxide này cần dùng $V$ \emph{L} dung dịch \emph{HCl 0.4M}. (a) Tính $V$. (b) Tính khối lượng muối clorua tạo ra.
\end{baitoan}

\begin{baitoan}[\cite{An_400_BT_Hoa_Hoc_9}, 1., p. 12]
	(a) Cho rất từ từ dung dịch A chứa $a$ {\rm mol HCl} vào dung dịch B chứa $b$ {\rm mol \ce{Na2CO3} ($a < 2b$)} thì thu được dung dịch C \& $V$ {\rm L} khí. Tính $V$. (b) Nếu cho dung dịch B vào dung dịch A thì được dung dịch D \& $V_1$ {\rm L} khí. Biết các phản ứng xảy ra hoàn toàn, các thể tích khí đo ở đktc. Lập biểu thức nêu mối quan hệ giữa $V_1$ với $a,b$.
\end{baitoan}

\begin{proof}[Giải]
	(a) Khi cho rất từ từ dung dịch HCl vào dung dịch \ce{Na2CO3}: \ce{HCl + Na2CO3 -> NaHCO3 + NaCl} với $n_{\rm HCl,\mbox{\small pư}} = n_{\ce{Na2CO3},\mbox{\small pư}} = n_{\ce{NaHCO3}} = b$ mol. Vì có khí bay ra nên $a > b$, hay $b < a < 2b$, suy ra $a - b < b$, i.e., $n_{\rm HCl,dư} = a - b < n_{\ce{NaHCO3}} = b$: \ce{HCl + NaHCO3 -> NaCl + H2O + CO2 ^} với $n_{\ce{CO2}} = n_{\rm HCl} = a - b$ mol, suy ra $V_{\ce{CO2}} = 22.4n_{\ce{CO2}} = 22.4(a - b)$ L. (b) Khi cho \ce{Na2CO3} vào dung dịch HCl: \ce{Na2CO3 + $2$HCl -> NaCl + H2O + CO2 ^}. Vì $a < 2b$ nên tính theo số mol HCl: $V_1 = 22.4n_{\ce{CO2}} = 22.4\frac{n_{\rm HCl}}{2} = 11.2a$ L.
\end{proof}

\begin{baitoan}[\cite{An_400_BT_Hoa_Hoc_9}, 2., p. 12]
	Cho {\rm31.8 g} hỗn hợp X gồm 2 muối {\rm\ce{MgCO3,CaCO3}} vào {\rm0.8 L} dung dịch {\rm HCl 1M} thu được dung dịch Z. (a) Hỏi dung dịch Z có dư acid không? (b) Lượng {\rm\ce{CO2}} có thể thu được bao nhiêu? (c) Cho vào dung dịch Z 1 lượng dung dịch {\rm\ce{NaHCO3}} dư thì thể tích khí {\rm\ce{CO2}} thu được là {\rm2.24 L} (đktc). Tính khối lượng mỗi muối trong hỗn hợp X.
\end{baitoan}

\begin{baitoan}[\cite{An_400_BT_Hoa_Hoc_9}, 3., p. 12]
	Có 3 bình đựng lần lượt các dung dịch {\rm KOH 1M, 2M, 3M}, mỗi bình chứa {\rm1 L} dung dịch. Trộn lẫn các dung dịch này sao cho dung dịch {\rm KOH 1.8M} thu được có thể tích lớn nhất.
\end{baitoan}

\begin{baitoan}[Mở rộng \cite{An_400_BT_Hoa_Hoc_9}, 3., p. 12]
	Cho $a,b,c,d\in\mathbb{R}$, $a,b,c > 0$. Có 3 bình đựng lần lượt các dung dịch {\rm KOH $a$M, $b$M, $c$M}, mỗi bình chứa {\rm1 L} dung dịch. Biện luận theo $a,b,c,d$ để trộn lẫn các dung dịch này sao cho dung dịch {\rm KOH $d$M} thu được có thể tích lớn nhất.
\end{baitoan}

\begin{baitoan}[\cite{An_400_BT_Hoa_Hoc_9}, 4., p. 12]
	Cho {\rm19.7 g} muối carbonate của kim loại hóa trị {\rm II} tác dụng hết với dung dịch {\rm\ce{H2SO4}} loãng, dư thu được {\rm23.3 g} muối sulfate. Công thức muối carbonate của kim loại hóa trị {\rm II}?
\end{baitoan}

\begin{baitoan}[\cite{An_400_BT_Hoa_Hoc_9}, 5., p. 12]
	Chọn các chất thích hợp \& cân bằng {\rm PTHH}: {\rm(a) \ce{X1 + X2 -> Br2 + MnBr2 + H2O}, (b) \ce{X3 + X4 + X5 -> HCl + H2SO4}, (c) \ce{A_1 + A_2 -> SO2 + H2O}, (d) \ce{B1 + B2 -> NH3 + Ca(NO3)2 + H2O}, (e) \ce{D1 + D2 + D3 -> Cl2 + MnSO4 + K2SO4 + Na2SO4 + H2O}}.
\end{baitoan}

\begin{baitoan}[\cite{An_400_BT_Hoa_Hoc_9}, 6., p. 12]
	Hợp chất A bị phân hủy ở nhiệt độ cao theo {\rm PTPƯ: 2A $\to$ B + 2D + 4E}. Sản phẩm tạo thành đều ở thể khí, khối lượng mol trung bình của hỗn hợp khí sau phản ứng là {\rm22.86 g{\tt/}mol}. Tính khối lượng mol của A.
\end{baitoan}

\begin{baitoan}[\cite{An_400_BT_Hoa_Hoc_9}, 7., p. 13]
	Cho {\rm39.6 g} hỗn hợp gồm {\rm\ce{KHSO3,K2CO3}} vào {\rm400 g} dung dịch {\rm HCl 7.3\%}, khi xong phản ứng thu được hỗn hợp khí X có tỷ khối so với khí hydrogen bằng $25.33$ \& 1 dung dịch A. (a) Chứng minh acid còn dư. (b) Tính $C\%$ các chất trong dung dịch A.
\end{baitoan}

\begin{baitoan}[\cite{An_400_BT_Hoa_Hoc_9}, 8., p. 13]
	Hòa tan {\rm21.5 g} hỗn hợp {\rm\ce{BaCl2,CaCl2}} vào {\rm178.5 mL} nước để được dung dịch A. Thêm vào dung dịch A {\rm175 mL} dung dịch {\rm\ce{Na2CO3} 1M} thấy tách ra {\rm19.85 g} kết tủa \& còn nhận được {\rm400 mL} dung dịch B. Tính nồng độ $\%$ của dung dịch {\rm\ce{BaCl2,CaCl2}}.
\end{baitoan}

\begin{baitoan}[\cite{An_400_BT_Hoa_Hoc_9}, 9.a, p. 13]
	Chỉ được dùng thêm quỳ tím \& các ống nghiệm, chỉ rõ phương pháp nhận ra các dung dịch bị mất nhãn: {\rm\ce{NaHSO4,Na2CO3,Na2SO3,BaCl2,Na2S}}.
\end{baitoan}

\begin{baitoan}[\cite{An_400_BT_Hoa_Hoc_9}, 9.b, p. 13]
	Cho khí {\rm\ce{CO2}} (đktc) phản ứng với {\rm80 g} dung dịch {\rm NaOH 25\%} để tạo thành hỗn hợp muối acid \& muối trung hòa theo tỷ lệ số mol là $2:3$. Tính thể tích {\rm\ce{CO2}} cần dùng.
\end{baitoan}

\begin{baitoan}[\cite{An_400_BT_Hoa_Hoc_9}, 10., p. 13]
	Cho {\rm0.2 mol CuO} tan hết trong dung dịch {\rm\ce{H2SO4} 20\%} đun nóng (lượng vừa đủ). Sau đó làm nguội dung dịch đến $10^\circ${\rm C}. Tính khối lượng tinh thể {\rm\ce{CuSO4.$5$H2O}} đã tách khỏi dung dịch, biết độ tan của {\rm\ce{CuSO4}} ở $10^\circ${\rm C} là {\rm17.4g}.
\end{baitoan}

\begin{baitoan}[\cite{An_400_BT_Hoa_Hoc_9}, 11., p. 13]
	Để có được {\rm200 mL} dung dịch {\rm NaCl 0.1M}. Có thể làm theo cách nào? {\sf A.} Lấy {\rm5.85 g NaCl} hòa tan trong {\rm200 mL} nước cất. {\sf B.} Lấy {\rm5.85 g NaCl} hòa tan trong {\rm194.15 g} nước cất. {\sf C.} Hòa tan {\rm1.17 g NaCl} trong {\rm100 mL} nước cất sau đó bổ sung thêm nước cho đến {\rm200 mL}. {\sf D.} Lấy 1 cốc chia độ, cho nước vào rồi cho {\rm1.17 g NaCl} cho đến lúc đạt thể tích {\rm250 mL}.
\end{baitoan}

\begin{baitoan}[\cite{An_400_BT_Hoa_Hoc_9}, 12., pp. 13--14]
	Đốt cháy hoàn toàn 1 chất vô cơ A trong không khí thì chỉ thu được {\rm1.6 g} iron ({\rm III}) oxide \& {\rm0.896 L} khí sunfurơ (đktc). (a) Xác định {\rm CTPT} của A. (b) Viết {\rm PTHH} để thực hiện chuỗi chuyển hóa: A $\to$ {\rm \ce{SO2}} $\to$ muối $A_1$ $\to$ $A_3$; A $\to$ kết tủa $A_2$.
\end{baitoan}

\begin{baitoan}[\cite{An_400_BT_Hoa_Hoc_9}, 13., p. 14]
	Hòa tan 1 ít {\rm NaCl} vào nước được $V$ {\rm mL} dung dịch A có khối lượng riêng $d$, thêm $V_1$ {\rm mL} nước vào dung dịch A được $(V + V_1)$ mL dung dịch B có khối lượng riêng $d_1$. Chứng minh $d > d_1$. Biết khối lượng riêng của nước là {\rm1 g{\tt/}mL}.
\end{baitoan}

\begin{baitoan}[\cite{An_400_BT_Hoa_Hoc_9}, 14., p. 14]
	Trộn $V_1$ {\rm L} dung dịch {\rm HCl 0.6M} với $V_2$ {\rm L} dung dịch {\rm NaOH 0.4M} thu được {\rm0.6 L} dung dịch A. Tính $V_1,V_2$ biết {\rm0.6 L} dung dịch A có thể hòa tan hết {\rm1.02 g \ce{Al2O3}} (coi sự pha trộn làm thay đổi thể tích không đáng kể).
\end{baitoan}

\begin{baitoan}[\cite{An_400_BT_Hoa_Hoc_9}, 15., p. 14]
	Có 5 dung dịch các chất: {\rm\ce{H2SO4,HCl,NaOH,KCl,BaCl2}}. Trình bày phương pháp phân biệt các dung dịch này mà chỉ dùng quỳ tím làm thuốc thử.
\end{baitoan}

\begin{baitoan}[\cite{An_400_BT_Hoa_Hoc_9}, 16., p. 14]
	Có 2 cốc, cốc A đựng {\rm200 mL} dung dịch chứa {\rm\ce{Na2CO3} 1M} \& {\rm\ce{NaHCO3} 1.5M}. Cốc B đựng {\rm173mL} dung dịch {\rm HCl 7.7\%}, $D = 1.37$ {\rm g{\tt/}mL}. Tiến hành 2 thí nghiệm:
	\begin{itemize}
		\item Thí nghiệm 1: Đổ rất từ từ cốc B vào cốc A.
		\item Thí nghiệm 2: Đổ rất từ từ cốc A vào cốc B.
	\end{itemize}
	Tính thể tích khí (đktc) thoát ra trong mỗi trường hợp sau khi đổ hết cốc này vào cốc kia.
\end{baitoan}

\begin{baitoan}[\cite{An_400_BT_Hoa_Hoc_9}, 17., p. 14]
	Cho $x$ {\rm g} dung dịch {\rm\ce{H2SO4}} loãng nồng độ $C\%$ tác dụng hoàn toàn với hỗn hợp 2 kim loại potassium \& iron (dùng dư), sau phản ứng khối lượng chung đã giảm $0.0469x$ {\rm g}. Tính $C\%$.
\end{baitoan}

\begin{baitoan}[\cite{An_400_BT_Hoa_Hoc_9}, 18., p. 14]
	Hòa tan {\rm450 g} potassium nitrate vào {\rm500 g} nước cất ở $25^\circ${\rm C} (dung dịch X). Biết độ tan của {\rm\ce{KNO3}} ở $20^\circ${\rm C} là {\rm32 g}. Xác định khối lượng potassium nitrate tách ra khỏi dung dịch khi làm lạnh dung dịch X đến $20^\circ${\rm C}.
\end{baitoan}

\begin{baitoan}[\cite{An_400_BT_Hoa_Hoc_9}, 19., pp. 14--15, HSG lớp 8 Tp. HCM 2000--2001]
	Khi cho $a$ {\rm g Fe} vào trong {\rm400 mL} dung dịch {\rm HCl}, sau khi phản ứng kết thúc đem cô cạn dung dịch thu được {\rm6.2 g} chất rắn X. Nếu cho hỗn hợp gồm $a$ {\rm g Fe} \& $b$ {\rm g Mg} vào trong {\rm400 mL} dung dịch {\rm HCl} thì sau khi phản ứng kết thúc, thu được {\rm896 mL \ce{H2}} (đktc) \& cô cạn dung dịch thì thu được {\rm6.68 g} chất rắn Y. Tính $a,b$, nồng độ mol của dung dịch {\rm HCl} \& thành phần khối lượng các chất trong X, Y. (Giả sử {\rm Mg} không phản ứng với nước \& khi phản ứng với acid, {\rm Mg} phản ứng trước, hết {\rm Mg} mới đến {\rm Fe}. Cho biết các phản ứng đều xảy ra hoàn toàn).
\end{baitoan}

\begin{baitoan}[\cite{An_400_BT_Hoa_Hoc_9}, 20., p. 15]
	Khử $a$ {\rm g} 1 iron oxide bằng {\rm CO} nóng, dư đến hoàn toàn thu được {\rm Fe} \& khí A. Hòa tan lượng sắt trên trong dung dịch {\rm\ce{H2SO4}} loãng dư thoát ra {\rm1.68 L \ce{H2}} (đktc). Hấp thụ toàn bộ khí A bằng {\rm\ce{Ca(OH)2}} dư thu được kết tủa. Tìm công thức iron oxide.
\end{baitoan}

\begin{baitoan}[\cite{An_400_BT_Hoa_Hoc_9}, 21., p. 15]
	Nung $m$ {\rm g} hỗn hợp chất rắn A gồm {\rm\ce{Fe2O3}, FeO} với lượng thiếu {\rm CO} thu được hỗn hợp chất rắn B có khối lượng {\rm47.84 g} \& {\rm5.6 L \ce{CO2}} (đktc). Tính $m$.
\end{baitoan}

\begin{baitoan}[\cite{An_400_BT_Hoa_Hoc_9}, 22., p. 15]
	Dung dịch X là dung dịch {\rm\ce{H2SO4}}, dung dịch Y là dung dịch {\rm NaOH}. Nếu trộn X \& Y theo tỷ lệ thể tích là $V_X:V_Y = 3:2$ thì được dung dịch A có chứa X dư. Trung hòa {\rm1 L} A cần {\rm40 g KOH 20\%}. Nếu trộn X \& Y theo tỷ lệ thể tích $V_X:V_Y = 2:3$ thì được dung dịch B có chứa Y dư. Trung hòa {\rm1 L} B cần {\rm29.2 g} dung dịch {\rm HCl 25\%}. Tính nồng độ mol của X \& Y.
\end{baitoan}

\begin{baitoan}[\cite{An_400_BT_Hoa_Hoc_9}, 23., p. 15]
	(a) Bằng phương pháp hóa học, phân biệt 4 muối sau: {\rm\ce{Na2CO3,MgCO3,BaCO3,CaCl2}}. (b) Chọn 2 dung dịch muối thích hợp để phân biệt 4 dung dịch các chất: {\rm\ce{BaCl2,HCl,K2SO4,Na3PO4}}.
\end{baitoan}

\begin{baitoan}[\cite{An_400_BT_Hoa_Hoc_9}, 24., p. 15]
	Đốt cháy hoàn toàn {\rm6.8 g} 1 hợp chất vô cơ A chỉ thu được {\rm4.48 L} khí {\rm\ce{SO2}} (đktc) \& {\rm3.6 g} nước. Tính thể tích khí {\rm\ce{O2}} đã dùng \& xác định {\rm CTPT} của A.
\end{baitoan}

\begin{baitoan}[\cite{An_400_BT_Hoa_Hoc_9}, 25., p. 15]
	Làm thế nào để nhận ra sự có mặt của mỗi khí trong hỗn hợp gồm {\rm CO, \ce{CO2,SO3}} bằng phương pháp hóa học, viết {\rm PTHH}.
\end{baitoan}

\begin{baitoan}[\cite{An_400_BT_Hoa_Hoc_9}, 26., p. 15]
	Hòa tan {\rm NaOH} rắn vào nước để tạo thành 2 dung dịch A \& B với nồng độ $\%$ của dung dịch A gấp $3$ lần nồng độ $\%$ của dung dịch B. Nếu đem trộn 2 dung dịch A \& B theo tỷ lệ khối lượng $m_A:m_B = 5:2$ thì thu được dung dịch C có nồng độ $\%$ là $20\%$. Xác định nồng độ $\%$ của 2 dung dịch A \& B.
\end{baitoan}

\begin{baitoan}[\cite{An_400_BT_Hoa_Hoc_9}, 27., pp. 15--16]
	Hỏi có bao nhiêu {\rm g NaCl} kết tinh khi làm lạnh {\rm600 g} dung dịch {\rm NaCl} bão hòa ở $90^\circ${\rm C}. Biết độ tan của {\rm NaCl} ở $90^\circ${\rm C} là {\rm50 g} \& ở $0^\circ${\rm C} là {\rm35 g}.
\end{baitoan}

\begin{baitoan}[\cite{An_400_BT_Hoa_Hoc_9}, 28., p. 16]
	Nêu phương pháp tách hỗn hợp gồm 3 khí {\rm\ce{Cl2,H2,CO2}} thành các chất nguyên chất.
\end{baitoan}

\begin{baitoan}[\cite{An_400_BT_Hoa_Hoc_9}, 29., p. 16]
	Tinh chế các chất khí: (a) {\rm\ce{O2}} có lẫn {\rm\ce{Cl2,CO2,SO2}}. (b) {\rm\ce{Cl2}} có lẫn {\rm\ce{O2,CO2,SO2}}. (c) {\rm\ce{CO2}} có lẫn khí {\rm HCl} \& hơi nước.
\end{baitoan}

\begin{baitoan}[\cite{An_400_BT_Hoa_Hoc_9}, 30., p. 16]
	Oxide của 1 kim loại hóa trị {\rm III} có khối lượng {\rm32 g} tan hết trong {\rm294 g} dung dịch {\rm\ce{H2SO4} 20\%}. Tìm {\rm CTPT} của oxide kim loại đó.
\end{baitoan}

\begin{baitoan}[\cite{An_400_BT_Hoa_Hoc_9}, 31., p. 16]
	Cho {\rm19.6 g} acid phosphoric tác dụng với {\rm200 g} dung dịch potassium hydroxide có nồng độ $8.4\%$. Thu được các muối nào sau phản ứng? Tính khối lượng của mỗi muối.
\end{baitoan}

\begin{baitoan}[\cite{An_400_BT_Hoa_Hoc_9}, 32., p. 16]
	Phân bón A có chứa $80\%$ ammonium nitrate. Phân bón B có chứa $82\%$ calcium nitrate. Nếu cần {\rm56 kg} nitrogen để bón ruộng thì nên mua loại phân nào? Vì sao?
\end{baitoan}

\begin{baitoan}[\cite{An_400_BT_Hoa_Hoc_9}, 33., p. 16]
	Nêu phương pháp tách hỗn hợp đá vôi, vôi sống, thạch cao, \& muối ăn thành từng chất nguyên chất.
\end{baitoan}

\begin{baitoan}[\cite{An_400_BT_Hoa_Hoc_9}, 34., p. 16]
	Nêu phương pháp tách hỗn hợp đá vôi, silicon dioxide, \& iron ({\rm II}) chloride thành từng chất nguyên chất.
\end{baitoan}

\begin{baitoan}[\cite{An_400_BT_Hoa_Hoc_9}, 35., p. 16]
	Nêu phương pháp tách hỗn hợp 3 khí {\rm\ce{O2,H2,SO2}} thành các chất nguyên chất.
\end{baitoan}

\begin{baitoan}[\cite{An_400_BT_Hoa_Hoc_9}, 36., p. 16]
	Nêu phương pháp tinh chế {\rm Cu} trong quặng {\rm Cu} có lẫn {\rm Fe, S}, \& {\rm Ag}.
\end{baitoan}

\begin{baitoan}[\cite{An_400_BT_Hoa_Hoc_9}, 37., p. 16]
	Cần thêm bao nhiêu {\rm g \ce{SO3}} vào dung dịch {\rm\ce{H2SO4} 10\%} để được {\rm100 g} dung dịch {\rm\ce{H2SO4} 20\%}?
\end{baitoan}

\begin{baitoan}[Mở rộng \cite{An_400_BT_Hoa_Hoc_9}, 37., p. 16]
	Cần thêm bao nhiêu {\rm g \ce{SO3}} vào dung dịch {\rm\ce{H2SO4} $a\%$} để được {\rm100 g} dung dịch {\rm\ce{H2SO4} $b\%$}, với $a,b\in\mathbb{R}$, $a,b > 0$?
\end{baitoan}

\begin{baitoan}[\cite{An_400_BT_Hoa_Hoc_9}, 38., p. 16]
	Phải hòa tan thêm bao nhiêu {\rm g} potassium hydroxide nguyên chất vào {\rm1200 g} dung dịch {\rm KOH 12\%} để có dung dịch {\rm KOH 20\%}?
\end{baitoan}

\begin{baitoan}[\cite{An_400_BT_Hoa_Hoc_9}, 39., p. 16]
	Cần phải dùng bao nhiêu {\rm L \ce{H2SO4}} có tỷ khối $d = 1.84$ \& bao nhiêu {\rm L} nước cất để pha thành {\rm10 L} dung dịch {\rm\ce{H2SO4}} có $d = 1.28$?
\end{baitoan}

\begin{baitoan}[\cite{An_400_BT_Hoa_Hoc_9}, 40.a, p. 16]
	(a) Trộn {\rm2 L} dung dịch {\rm HCl 4M} vào {\rm 1 L} dung dịch {\rm HCl 0.5 M}. Tính nồng độ mol của dung dịch mới.
\end{baitoan}

\begin{baitoan}[Mở rộng \cite{An_400_BT_Hoa_Hoc_9}, 40., p. 16]
	(a) Trộn $V_1$ {\rm L} dung dịch {\rm HCl $a$M} vào $V_2$ {\rm L} dung dịch {\rm HCl $b$M}. Tính nồng độ mol của dung dịch mới.
\end{baitoan}

\begin{baitoan}[\cite{An_400_BT_Hoa_Hoc_9}, 40.b, p. 16]
	Trộn {\rm150 g} dung dịch {\rm NaOH 10\%} vào {\rm460 g} dung dịch {\rm NaOH $x$\%} để tạo thành dung dịch $6\%$. Tính $x$.
\end{baitoan}

\begin{baitoan}[\cite{An_400_BT_Hoa_Hoc_9}, 41., p. 16]
	Cần lấy bao nhiêu {\rm mL} dung dịch {\rm HCl} có nồng độ $36\%$, $d = 1.19$, để pha thành {\rm5 L} dung dịch acid {\rm HCl} có nồng độ {\rm0.5M}.
\end{baitoan}

\begin{baitoan}[\cite{An_400_BT_Hoa_Hoc_9}, 42., p. 17]
	Cho {\rm100 g} dung dịch {\rm\ce{H2SO4} 19.6\%} vào {\rm400 g} dung dịch {\rm\ce{BaCl2} 13\%}. (a) Tính khối lượng kết tủa. (b) Tính nồng độ $\%$ các chất có trong dung dịch sau phản ứng.
\end{baitoan}

\begin{baitoan}[\cite{An_400_BT_Hoa_Hoc_9}, 43., p. 17]
	Hòa tan {\rm8.96 L} khí {\rm HCl} (đktc) vào {\rm185.4 g} nước được dung dịch M. Lấy {\rm50 g} dung dịch M cho tác dụng với {\rm85 g} dung dịch {\rm\ce{AgNO3} 16\%} thì thu được dung dịch N \& 1 chất kết tủa.
\end{baitoan}

\begin{baitoan}[\cite{An_400_BT_Hoa_Hoc_9}, 44., p. 17]
	Cho {\rm11.6 g} hỗn hợp {\rm\ce{Fe2O3}, FeO} có tỷ lệ số mol là $1:1$ vào {\rm300 mL} dung dịch {\rm HCl 2M} được dung dịch A. (a) Tính nồng độ mol của các chất trong dung dịch sau phản ứng (thể tích dung dịch thay đổi không đáng kể). (b) Tính thể tích dung dịch {\rm NaOH 1.5M} đủ để tác dụng hết với dung dịch A.
\end{baitoan}

\begin{baitoan}[\cite{An_400_BT_Hoa_Hoc_9}, 45., p. 17]
	Cho sản phẩm thu được khi oxy hóa hoàn toàn {\rm5.6 L} khí sunfurơ (đktc) vào trong {\rm57.2 mL} dung dịch {\rm\ce{H2SO4} 60\%} có $D = 1.5$ {\rm g{\tt/}mL}. Tính nồng độ $\%$ của dung dịch acid thu được.
\end{baitoan}

\begin{baitoan}[\cite{An_400_BT_Hoa_Hoc_9}, 46., p. 17]
	Cho {\rm200 g} dung dịch {\rm\ce{BaCl2} 5.2\%} tác dụng với {\rm58.8 g} dung dịch {\rm\ce{H2SO4} 20\%}. Tính nồng độ $\%$ của các chất có trong dung dịch.
\end{baitoan}

\begin{baitoan}[\cite{An_400_BT_Hoa_Hoc_9}, 47.a, p. 17]
	Tính tỷ lệ thể tích của 2 dung dịch {\rm HCl 0.2M \& 1M} để trộn thành dung dịch {\rm HCl 0.4M}.
\end{baitoan}

\begin{baitoan}[\cite{An_400_BT_Hoa_Hoc_9}, 47.b, p. 17]
	Tính khối lượng {\rm\ce{Na2O}} \& khối lượng nước cần để có được {\rm200 g} dung dịch {\rm NaOH 10\%}.
\end{baitoan}

\begin{baitoan}[\cite{An_400_BT_Hoa_Hoc_9}, 48., p. 17]
	1 loại đá chứa {\rm80\% \ce{CaCO3}}, phần còn lại là tạp chất trơ. Nung đá vôi trên tới phản ứng hoàn toàn. Hỏi khối lượng của chất rắn thu được sau khi nung bằng bao nhiêu $\%$ khối lượng đá trước khi nung \& tính {\rm\% CaO} trong chất rắn sau khi nung.
\end{baitoan}

\begin{baitoan}[\cite{An_400_BT_Hoa_Hoc_9}, 49., p. 17]
	Khi nung hỗn hợp {\rm\ce{CaCO3,MgCO3}} thì khối lượng chất rắn thu được sau phản ứng chỉ bằng $\frac{1}{2}$ khối lượng ban đầu. Xác định thành phần $\%$ khối lượng các chất trong hỗn hợp ban đầu.
\end{baitoan}

\begin{baitoan}[\cite{An_400_BT_Hoa_Hoc_9}, 50., p. 17]
	Trong quặng bôxit trung bình có $50\%$ aluminium oxide. Kim loại luyện được từ oxide đó còn chứa $1.5\%$ tạp chất. Tính lượng aluminium nguyên chất điều chế được  từ $0.5$ tấn quặng boxit.
\end{baitoan}

\begin{baitoan}[\cite{An_400_BT_Hoa_Hoc_9}, 51., pp. 17--18]
	Đốt cháy hỗn hợp {\rm CuO, FeO} với {\rm C} có dư thì được chất rắn A \& khí B. Cho B tác dụng với nước vôi trong có dư thu được {\rm8 g} kết tủa. Chất rắn A cho tác dụng với dung dịch {\rm HCl} có nồng độ $10\%$ thì cần dùng 1 lượng acid là {\rm73 g} sẽ vừa đủ. (a) Viết {\rm PTHH}. (b) Tính khối lượng {\rm CuO, FeO} trong hỗn hợp ban đầu \& thể tích khí B (đktc).
\end{baitoan}

\begin{baitoan}[\cite{An_400_BT_Hoa_Hoc_9}, 52., p. 18]
	Khi phân hủy bằng nhiệt {\rm14.2 g} hỗn hợp {\rm\ce{CaCO3,MgCO3}}, thu được {\rm6.6 g \ce{CO2}} (đktc). Tính thành phần $\%$ các chất trong hỗn hợp.
\end{baitoan}

\begin{baitoan}[\cite{An_400_BT_Hoa_Hoc_9}, 53., p. 18]
	Cho {\rm38.2 g} hỗn hợp {\rm\ce{Na2CO3,K2CO3}} vào dung dịch {\rm HCl}. Dẫn lượng khí sinh ra qua nước vôi trong có dư thu được {\rm30 g} kết tủa. Tính khối lượng mỗi muối trong hỗn hợp ban đầu.
\end{baitoan}

\begin{baitoan}[\cite{An_400_BT_Hoa_Hoc_9}, 54., p. 18]
	Cho {\rm0.325 g} hỗn hợp gồm {\rm NaCl, KCl} được hòa tan vào nước. Sau đó cho dung dịch {\rm\ce{AgNO3}} vào dung dịch trên, ta được 1 kết tủa; sấy kết tủa đến khối lượng không đổi thấy cân nặng {\rm0.717 g}. Tính thành phần $\%$ các chất trong hỗn hợp.
\end{baitoan}

\begin{baitoan}[\cite{An_400_BT_Hoa_Hoc_9}, 55., p. 18]
	{\rm\ce{Al4C3,CaC2}} tác dụng với nước theo {\rm PTHH: \ce{Al4C3 + $12$H2O -> $4$Al(OH)3 + $3$CH4, CaC2 + $2$H2O -> Ca(OH)2 + C2H2}}. Cho hỗn hợp 2 chất trên tác dụng với nước dư thu được {\rm2.016 L} hỗn hợp khí. Lấy hỗn hợp này đốt cháy hoàn toàn thu được {\rm2.688 L \ce{CO2}}. Các thể tích đều đo ở đktc. Tính lượng {\rm\ce{Al4C3,CaC2}} trong hỗn hợp.
\end{baitoan}

\begin{baitoan}[\cite{An_400_BT_Hoa_Hoc_9}, 56., p. 18]
	Dùng thuốc thử thích hợp, nhận biết các dung dịch sau đã mất nhãn: {\rm(a) \ce{NaCl,NaBr,KI,HCl,H2SO4,KOH}. (b) \ce{Na2SO4,H2SO4,NaOH,KCl,NaNO3}}.
\end{baitoan}

\begin{baitoan}[\cite{An_400_BT_Hoa_Hoc_9}, 57., p. 18]
	Dùng thuốc thử thích hợp để nhận biết các dung dịch: {\rm(a) KI, HCl, NaCl, \ce{H2SO4}. (b) HCl, HBr, NaCl, NaOH. (c) NaF, \ce{CaCl2}, KBr, \ce{MgI2}}.
\end{baitoan}

\begin{baitoan}[\cite{An_400_BT_Hoa_Hoc_9}, 58., p. 18]
	Chỉ dùng thêm 1 hóa chất, phân biệt các lọ mất nhãn: {\rm(a) \ce{MgCl2}, KBr, NaI, \ce{AgNO3,NH4HCO3}. (b) NaBr, \ce{ZnSO4, Na2CO3, AgNO3, BaCl2}}.
\end{baitoan}

\begin{baitoan}[\cite{An_400_BT_Hoa_Hoc_9}, 59., p. 18]
	Chỉ dùng thêm 1 hóa chất, phân biệt các dung dịch: {\rm(a) \ce{K2SO4,K2CO3,K2SiO3,K2S,K2SO3}. (b) \ce{MgCl2}, NaBr, \ce{Ca(NO3)2}}.
\end{baitoan}

\begin{baitoan}[\cite{An_400_BT_Hoa_Hoc_9}, 60., p. 18]
	Có 5 lọ, mỗi lọ đựng 1 trong các hóa chất: {\rm FeO, CuO, \ce{Fe3O4,Ag2O,MnO2}}. Dùng phương pháp hóa học để nhận biết từng hóa chất trong mỗi lọ.
\end{baitoan}

\begin{baitoan}[\cite{An_400_BT_Hoa_Hoc_9}, 61.a, p. 19]
	Chỉ có nước \& khí carbonic có thể phân biệt được 5 chất bột trắng sau hay không? Nếu được, trình bày cách phân biệt: {\rm NaCl, \ce{Na2SO4,BaCO3,Na2CO3,BaSO4}}? 
\end{baitoan}

\begin{baitoan}[\cite{An_400_BT_Hoa_Hoc_9}, 61.b, p. 19]
	Trình bày các nguyên tắc tiến hành phân biệt 4 chất: {\rm\ce{BaSO4,BaCO3,NaCl,Na2CO3}} với điều kiện chỉ dùng thêm {\rm HCl} loãng.
\end{baitoan}

\begin{baitoan}[\cite{An_400_BT_Hoa_Hoc_9}, 62.a, p. 19]
	Nêu cách nhận biết {\rm CaO, \ce{Na2O}, MgO, \ce{P2O5}} đều là chất bột trắng.
\end{baitoan}

\begin{baitoan}[\cite{An_400_BT_Hoa_Hoc_9}, 62.b, p. 19]
	Bằng phương pháp hóa học, nhận biết 4 kim loại có màu trắng bạc {\rm Al, Ag, Fe, Mg}.
\end{baitoan}

\begin{baitoan}[\cite{An_400_BT_Hoa_Hoc_9}, 63.a, p. 19]
	Từ các nguyên liệu chính là {\rm\ce{CO2,NaCl,NH4Cl}}, viết các phương trình phản ứng điều chế {\rm\ce{NH4HCO3}} tinh khiết.
\end{baitoan}

\begin{baitoan}[\cite{An_400_BT_Hoa_Hoc_9}, 63.b, p. 19]
	Điều chế 3 oxide, 2 acid, \& 2 muối từ các hóa chất: {\rm Mg, \ce{H2O}}, không khí, \& {\rm S}. Viết {\rm PTHH}.
\end{baitoan}

\begin{baitoan}[\cite{An_400_BT_Hoa_Hoc_9}, 64., p. 19]
	Chỉ từ {\rm Cu, NaCl, \ce{H2O}}, nêu cách điều chế để thu được {\rm\ce{Cu(OH)2}}. Viết {\rm PTHH}.
\end{baitoan}

\begin{baitoan}[\cite{An_400_BT_Hoa_Hoc_9}, 65.a, p. 19]
	Cho các chất: Aluminium, oxygen, nước, copper sulfate, iron, acid hydrochloric. Điều chế copper, copper oxide, aluminium chloride (bằng 2 phương pháp) \& iron ({\rm II}) chloride. Viết {\rm PTHH}.
\end{baitoan}

\begin{baitoan}[\cite{An_400_BT_Hoa_Hoc_9}, 65.b, p. 19]
	Bằng cách nào từ iron ta có thể điều chế iron ({\rm II}) hydroxide, iron ({\rm III}) hydroxide? Viết {\rm PTHH}.
\end{baitoan}

\begin{baitoan}[\cite{An_400_BT_Hoa_Hoc_9}, 66., p. 19]
	Chỉ từ quặng pirit {\rm\ce{FeS2,O2,H2O}}, có chất xúc tác thích hợp. Viết {\rm PTPƯ} điều chế muối iron ({\rm III}) sulfate.
\end{baitoan}

\begin{baitoan}[\cite{An_400_BT_Hoa_Hoc_9}, 67., p. 19]
	Viết các {\rm PTPƯ} phản ứng điều chế trực tiếp: (a) {\rm Cu $\to$ \ce{CuCl2} bằng 3 phương pháp}. (b) {\rm\ce{CuCl2} $\to$ Cu} bằng 2 phương pháp. (c) {\rm Fe $\to$ \ce{FeCl3}} bằng 2 phương pháp.
\end{baitoan}

\begin{baitoan}[\cite{An_400_BT_Hoa_Hoc_9}, 68.a, p. 19]
	Chỉ từ các chất {\rm\ce{KMnO4,BaCl2,H2SO4}, Fe} có thể điều chế được các khí gì?
\end{baitoan}

\begin{baitoan}[\cite{An_400_BT_Hoa_Hoc_9}, 68.b, p. 19]
	Muốn điều chế 3 chất rắn: {\rm NaOH, \ce{NaHCO3,Na2CO3}}. (a) Trình bày 3 phương pháp điều chế mỗi chất. (b) Chỉ dùng 1 thuốc thử, nhận biết từng dung dịch các chất trên.
\end{baitoan}

\begin{baitoan}[\cite{An_400_BT_Hoa_Hoc_9}, 69.a, pp. 19--20]
	Khí nitrogen bị lẫn các tạp chất {\rm CO, \ce{CO2,H2}}, \& hơi nước. Làm thế nào để thu được {\rm\ce{N2}} tinh khiết.
\end{baitoan}

\begin{baitoan}[\cite{An_400_BT_Hoa_Hoc_9}, 69.b, p. 20]
	Khi đốt cháy than, thu được hỗn hợp khí {\rm CO, \ce{CO2}}. Trình bày phương pháp hóa học để thu được từng khí.
\end{baitoan}

\begin{baitoan}[\cite{An_400_BT_Hoa_Hoc_9}, 70., p. 20]
	Nêu phương pháp hóa học để làm sạch các khí: (a) Methane có lẫn khí acetylen. (b) Ethylen có lẫn khí carbonic.
\end{baitoan}

\begin{baitoan}[\cite{An_400_BT_Hoa_Hoc_9}, 72., p. 20]
	Nêu phương pháp tách các hỗn hợp sau thành các chất nguyên chất: (a) Hỗn hợp khí gồm: {\rm\ce{Cl2,H2,CO2}}. (b) Hỗn hợp khí gồm: {\rm\ce{SO2,CO2}, CO}. (c) Hỗn hợp khí gồm: {\rm\ce{SO2,O2}, HCl}.
\end{baitoan}

\begin{baitoan}[\cite{An_400_BT_Hoa_Hoc_9}, 73., p. 20]
	Tinh chế: (a) {\rm\ce{CaSO3}} có lẫn {\rm\ce{CaCO3,Na2CO3}}. (b) Muối ăn có lẫn {\rm\ce{CaCl2,CaSO4,Na2SO3}}. (c) {\rm Cu} có lẫn {\rm Fe, Ag, S}.
\end{baitoan}

\begin{baitoan}[\cite{An_400_BT_Hoa_Hoc_9}, 74.a, p. 20]
	Trình bày phương pháp hóa học để lấy từng oxide từ hỗn hợp: {\rm\ce{SiO2,Al2O3,Fe2O3}, CuO}.
\end{baitoan}

\begin{baitoan}[\cite{An_400_BT_Hoa_Hoc_9}, 74.b, p. 20]
	Trình bày phương pháp lấy từng kim loại {\rm Cu, Fe} từ hỗn hợp các oxide: {\rm\ce{SiO2,Al2O3}, CuO, FeO}.
\end{baitoan}

\begin{baitoan}[\cite{An_400_BT_Hoa_Hoc_9}, 74.c, p. 20]
	Bằng phương pháp hóa học, tách từng kim loại ra khỏi hỗn hợp gồm {\rm Al, Fe, Ag, Cu}.
\end{baitoan}

\begin{baitoan}[\cite{An_400_BT_Hoa_Hoc_9}, 75., p. 20]
	Nêu cách tách các chất ra khỏi hỗn hợp: (a) {\rm\ce{Cl2} có lẫn \ce{N2,H2}}. (b) {\rm\ce{Cl2} có lẫn \ce{CO2}}.
\end{baitoan}

\begin{baitoan}[\cite{An_400_BT_Hoa_Hoc_9}, 76., p. 20]
	Nêu cách tinh chế: (a) Muối ăn có lẫn {\rm\ce{MgCl2}} \& {\rm NaBr}. (b) Acid hydrochloric có lẫn acid {\rm\ce{H2SO4}}.
\end{baitoan}

\begin{baitoan}[\cite{An_400_BT_Hoa_Hoc_9}, 77., p. 20]
	1 loại muối ăn có lẫn tạp chất {\rm\ce{CaCl2,MgCl2,Na2SO4,MgSO4,CaSO4}}. Trình bày cách loại các tạp chất để thu được muối ăn tinh khiết.
\end{baitoan}

\begin{baitoan}[\cite{An_400_BT_Hoa_Hoc_9}, 78., p. 20]
	Tìm cách tách lấy từng muối trong hỗn hợp rắn gồm: ammonium chloride, barium chloride, magnesium chloride. Viết {\rm PTHH}.
\end{baitoan}

\begin{baitoan}[\cite{Nguyen_Buu_Can_500_BT_Hoa_Hoc_THCS}, 201., p. 97]
	Định nghĩa \& phân loại oxide.
\end{baitoan}

\begin{proof}[Giải]
	(a) Oxide là hợp chất của oxygen với 1 nguyên tố khác, e.g., iron (III) oxide \ce{Fe2O3}, sulfur dioxide \ce{SO2}, nước \ce{H2O} cũng là 1 oxide. (b) Dựa vào tính chất hóa học, oxide được chia làm 4 loại chính: oxide acid, oxide base, oxide trung tính, oxide lưỡng tính.
\end{proof}

\begin{baitoan}[\cite{Nguyen_Buu_Can_500_BT_Hoa_Hoc_THCS}, 202., p. 97]
	Phân loại oxide acid, oxide base, oxide lưỡng tính: {\rm FeO, ZnO, \ce{Al2O3,CaO,Mn2O7,P2O5,N2O5,SiO2}}.
\end{baitoan}

\begin{proof}[Giải]
	Oxide acid: \ce{Mn2O7,P2O5,N2O5,SiO2}. oxide base: CaO, FeO. Oxide lưỡng tính: ZnO, \ce{Al2O3}.
\end{proof}

\begin{baitoan}[\cite{Nguyen_Buu_Can_500_BT_Hoa_Hoc_THCS}, 203., p. 97]
	Cho các oxide: {\rm CaO, \ce{SiO2,Fe2O3,Fe3O4,P2O5}}. Chất nào tan trong nước, chất nào tan trong dung dịch kiềm? Chất nào dùng để hút ẩm? Viết {\rm PTHH}.
\end{baitoan}

\begin{proof}[Giải]
	CaO, \ce{P2O5} tan trong nước, \& dùng để hút ẩm. \ce{CaO + H2O -> Ca(OH)2, P2O5 + $3$H2O -> $2$H3PO4}. \ce{SiO2} tan chậm trong kiềm: \ce{SiO2 + $2$NaOH -> Na2SiO3 + H2O}. CaO, \ce{Fe2O3,Fe3O4} tan trong acid: \ce{CaO + $2$HCl -> CaCl2 + H2O, Fe2O3 + $6$HCl -> $2$FeCl3 + $3$H2O, Fe3O4 + $8$HCl -> FeCl2 + $2$FeCl3 + $4$H2O}.
\end{proof}

\begin{baitoan}[\cite{Nguyen_Buu_Can_500_BT_Hoa_Hoc_THCS}, 204., p. 97]
	Trình bày tính chất của calcium oxide.
\end{baitoan}
Xem, e.g., \href{https://vi.wikipedia.org/wiki/Calci_oxide}{Wikipedia{\tt/}calci oxide}, \href{https://en.wikipedia.org/wiki/Calcium_oxide}{Wikipedia{\tt/}calcium oxide}.

\begin{proof}[Giải]
	(a) Tính chất vật lý: Calcium oxide (CTHH: CaO, các tên gọi thông thường khác: vôi sống, vôi nung) là 1 oxide của calcium, được sử dụng rộng rãi. CaO là chất rắn màu trắng, có khối lượng mol bằng $26.1$g{\tt/}mol, \href{https://vi.wikipedia.org/wiki/H%E1%BB%87_s%E1%BB%91_gi%C3%A3n_n%E1%BB%9F_nhi%E1%BB%87t}{hệ số giãn nở nhiệt} $0.148$, nhiệt độ nóng chảy $2572^\circ$C. (b) Tính chất hóa học: CaO là oxide base nên tác dụng được với nước, acid, \& oxide acid. CaO tác dụng với nước: \ce{CaO + H2O -> Ca(OH)2}. CaO tác dụng với acid, e.g., \ce{CaO + $2$HCl -> CaCl2 + H2O}. CaO tác dụng với oxide acid: \ce{CaO + CO2 -> CaCO3}.
\end{proof}

\begin{baitoan}[\cite{Nguyen_Buu_Can_500_BT_Hoa_Hoc_THCS}, 205., p. 97]
	Để calcium oxide (vôi sống) lâu ngày trong không khí sẽ bị kém phẩm chất. Giải thích hiện tượng \& viết {\rm PTHH}.
\end{baitoan}

\begin{proof}[Giải]
	Vôi sống CaO bị biến thành calcium carbonate \ce{CaCO3} đồng theo theo 2 biến hóa: \ce{CaO + CO2 -> CaCO3} hoặc \ce{CaO + H2O -> Ca(OH)2, CO2 + Ca(OH)2 -> CaCO3 + H2O}.
\end{proof}

\begin{baitoan}[\cite{Nguyen_Buu_Can_500_BT_Hoa_Hoc_THCS}, 206., p. 97]
	Có 3 lọ đựng chất bột màu trắng: {\rm MgO, \ce{Na2O,P2O5}}. Nêu phương pháp thực nghiệm để nhận biết 3 chất \& viết {\rm PTHH}.
\end{baitoan}

\begin{proof}[Giải]
	Lấy ở mỗi lọ 1 ít hóa chất cho vào từng ống nghiệm hòa tan vào nước. Chất không tan: MgO. Chất tan được: \ce{Na2O,P2O5}, \ce{Na2O + H2O -> $2$NaOH, P2O5 + $3$H2O -> $2$H3PO4}. Sau đó nhúng quỳ tím vào 2 dung dịch thu được. Dung dịch nào làm quỳ tím hóa xanh là dung dịch NaOH, chất hòa tan là \ce{Na2O}. Dung dịch làm quỳ tím hóa đỏ là \ce{H3PO4} \& chất hòa tan là \ce{P2O5}.
\end{proof}

\begin{baitoan}[\cite{Nguyen_Buu_Can_500_BT_Hoa_Hoc_THCS}, 207., p. 97]
	Có hỗn hợp 2 chất rắn là {\rm CaO, \ce{Fe2O3}}. Bằng phương pháp hóa học nào có thể tách riêng được {\rm\ce{Fe2O3}}? Viết {\rm PTHH}.
\end{baitoan}

\begin{proof}[Giải]
	Ngâm hỗn hợp CaO, \ce{Fe2O3} trong lượng nước dư. Chỉ có CaO tác dụng với \ce{H2O} tạo thành hợp chất tan được trong \ce{H2O}, lọc, tách riêng được \ce{Fe2O3}: \ce{CaO + H2O -> Ca(OH)2}. 
\end{proof}

\begin{baitoan}[\cite{Nguyen_Buu_Can_500_BT_Hoa_Hoc_THCS}, 208., p. 98]
	Viết {\rm PTHH} thực hiện các biến hóa hóa học: {\rm(a) CaO $\to$ \ce{Ca(OH)2} $\to$ \ce{CaCO3} $\to$ CaO. (b) CaO $\to$ \ce{CaCO3}. (c) CaO $\to$ \ce{Ca(NO3)2}}.
\end{baitoan}

\begin{proof}[Giải]
	(a) \ce{CaO + H2O -> Ca(OH)2, Ca(OH)2 + CO2 -> CaCO3 v + H2O, CaO ->[$t^\circ$] CaO + CO2 ^}. (b) \ce{CaO + CO2 -> CaCO3}. (c) \ce{CaO + $2$HNO3 -> Ca(NO3)2 + H2}.
\end{proof}

\begin{baitoan}[\cite{Nguyen_Buu_Can_500_BT_Hoa_Hoc_THCS}, 209., p. 98]
	Hoàn thành các chuỗi biến hóa: {\rm(a) Cu $\to$ CuO $\to$ \ce{CuCl2} $\to$ \ce{Cu(OH)2} $\to$ CuO $\to$ Cu. (b) P $\to$ \ce{P2O5} $\to$ \ce{H3PO4} $\to$ \ce{NaH2PO4} $\to$ \ce{Na2HPO4} $\to$ \ce{Na3PO4}}.
\end{baitoan}

\begin{proof}[Giải]
	(a) \ce{$2$Cu + O2 ->[$t^\circ$] $2$CuO, CuO + $2$HCl -> CuCl2 + H2O, CuCl2 + $2$NaOH -> $2$NaCl + Cu(OH)2, Cu(OH)2 ->[$t^\circ$] CuO + H2O, CuO + H2 ->[$t^\circ$] Cu + H2O} hoặc \ce{CuO + CO ->[$t^\circ$] Cu + CO2}. (b) \ce{$4$P + $5$O2 ->[$t^\circ$] $2$P2O5, P2O5 + $3$H2O -> $2$H3PO4, H3PO4 + NaOH -> NaH2PO4 + H2O, NaH2PO4 + NaOH -> Na2HPO4 + H2O, Na2HPO4 + NaOH -> Na3PO4 + H2O}.
\end{proof}

\begin{baitoan}[\cite{Nguyen_Buu_Can_500_BT_Hoa_Hoc_THCS}, 210., p. 98]
	Hoàn thành chuỗi biến hóa: Carbon $\to$ carbon ({\rm IV}) oxide $\to$ calcium carbonate $\to$ calcium bicarbonate $\to$ đá vôi $\to$ vôi sống $\to$ vôi tôi.
\end{baitoan}

\begin{proof}[Giải]
	\ce{C + O2 ->[$t^\circ$] CO2, CO2 + CaO -> CaCO3} hoặc \ce{Ca(OH)2} (dư) \ce{+ CO2 -> CaCO3 v + H2O, CaCO3 + CO2 + H2O -> Ca(HCO3)2, Ca(HCO3)2 ->[$t^\circ$] CaCO3 v + H2O + CO2 ^, CaCO3 ->[$t^\circ$] CaO + CO2 ^, CaO + H2O -> Ca(OH)2}.
\end{proof}

\begin{baitoan}[\cite{Nguyen_Buu_Can_500_BT_Hoa_Hoc_THCS}, 211., p. 98]
	Có hỗn hợp khí gồm {\rm\ce{CO2,O2}}. Làm thế nào có thể thu được khí {\rm\ce{O2}} tinh khiết từ hỗn hợp trên? Trình bày cách làm \& viết {\rm PTHH}.
\end{baitoan}

\begin{proof}[Giải]
	Cho hỗn hợp 2 khí \ce{CO2,O2} lội chậm qua dung dịch kiềm dư, khí \ce{CO2} bị giữ lại trong dung dịch kiềm, khí đi qua dung dịch kiềm là \ce{O2}.
\end{proof}

\begin{baitoan}[\cite{Nguyen_Buu_Can_500_BT_Hoa_Hoc_THCS}, 212., p. 98]
	Có 4 gói bột oxide màu đen tương tự nhau: {\rm CuO, AgO, FeO, \ce{MnO2}}. Chỉ dùng dung dịch {\rm HCl} có thể nhận biết được các oxide nào?
\end{baitoan}

\begin{proof}[Giải]
	Dung dịch HCl có thể nhận biết được các oxide: \ce{CuO + $2$HCl -> CuCl2 + H2O, CuCl2} màu xanh, \ce{MnO2 + $4$HCl -> MnCl2 + Cl2 ^ + $2$H2O, Cl2} mùi hắc, màu vàng lục. \ce{Ag2O + $2$HCl -> $2$AgCl2 v + H2O, AgCl} màu trắng. \ce{FeO + $2$HCl -> FeCl2 + H2O, FeCL2} màu lục nhạt.
\end{proof}

\begin{baitoan}[\cite{Nguyen_Buu_Can_500_BT_Hoa_Hoc_THCS}, 213., p. 98]
	Có 3 chất: {\rm Mg, Al, \ce{Al2O3}}. Chỉ được dùng 1 hóa chất làm thuốc thử phân biệt 3 chất trên. Viết {\rm PTHH}.
\end{baitoan}

\begin{proof}[Giải]
	Có thể dùng thuôc thử NaOH: \ce{$2$Al + $2$NaOH + $2$H2O -> $2$NaAlO2 + $3$H2 ^, Al2O3 + $2$NaOH -> $2$NaAlO2 + H2O}, còn lại Mg không tác dụng với NaOH.
\end{proof}

\begin{baitoan}[\cite{Nguyen_Buu_Can_500_BT_Hoa_Hoc_THCS}, 215., p. 98]
	Định nghĩa \& phân loại acid? (a) Nêu phương pháp chính để điều chế acid. Cho các ví dụ minh họa. (b) Viết $4$ phản ứng thông thường tạo thành các acid {\rm HCl, \ce{H2SO4,H3PO4,HNO3}}.
\end{baitoan}

\begin{proof}[Giải]
	(a) Acid là hợp chất mà phân tử gồm 1 hay nhiều nguyên tử hydrogen liên kết với gốc acid. CTPT của acid có dạng \ce{H_xA} với A: gốc acid, $x$: hóa trị của A. (b) Dựa vào thành phần, acid được chia thành 2 loại: Acid không có oxygen (hidra acid), e.g., HBr, HCl, \ce{H2S}, $\ldots$ Acid có oxygen, e.g., \ce{HNO3,H3PO4,H2SO4}, $\ldots$
\end{proof}

\begin{baitoan}[\cite{Nguyen_Buu_Can_500_BT_Hoa_Hoc_THCS}, 216., p. 99]
	Trình bày tính chất hóa học của acid sulfuric.
\end{baitoan}

\begin{proof}[Giải]
	Tính chất hóa học của acid sulfuric: (a) Dung dịch \ce{H2SO4} loãng có đầy đủ t ính chất của acid: Làm quỳ tím chuyển sang màu đỏ. Tác dụng với base tạo thành muối \& nước, e.g., \ce{H2SO4 + $2$KOH -> K2SO4 + H2O}. Tác dụng với oxide base tạo thành muối \& nước: \ce{H2SO4 + BaO -> BaSO4 + H2O}. Tác dụng với kim loại tạo thành muối \& giải phóng \ce{H2}: \ce{Zn + H2SO4 -> ZnSO4 + H2 ^}. (b) Acid sulfuric đặc, nóng tác dụng hầu hết các kim loại để tạo thành muối, nhưng không giải phóng khí hydrogen, e.g., \ce{$2$Fe + $6$H2SO4 \mbox{đ} -> Fe2(SO4)3 + $6$SO2 ^ + $6$H2O}.
\end{proof}

\begin{baitoan}[\cite{Nguyen_Buu_Can_500_BT_Hoa_Hoc_THCS}, 217., p. 99]
	Khi cho khí carbonic vào nước có nhuộm quỳ tím thì nước chuyển sang màu đỏ, khi đun nóng thì màu nước lại chuyển thành màu tím. Giải thích hiện tượng.
\end{baitoan}

\begin{proof}[Giải]
	\ce{CO2} là oxide acid, khi hòa tan trong \ce{H2O} tạo thành acid \ce{H2CO3} nên làm quỳ tím hóa đỏ. Khi đung nóng \ce{H2CO3} phân hủy cho \ce{CO2} \& \ce{H2O}, khí \ce{CO2} bay lên nên màu nước lại chuyển thành tím.
\end{proof}

\begin{baitoan}[\cite{Nguyen_Buu_Can_500_BT_Hoa_Hoc_THCS}, 218., p. 99]
	Base là gì? Kiềm là gì? Kể tên các base là kiềm. Nêu cách gọi tên base. Các base: {\rm NaOH}, dung dịch {\rm\ce{Ca(OH)2}, KOH} có tên riêng gì?
\end{baitoan}

\begin{baitoan}[\cite{Nguyen_Buu_Can_500_BT_Hoa_Hoc_THCS}, 219., p. 99]
	Cho biết aluminium hydroxide là hợp chất lưỡng t ính, viết các phương trình phản ứng của aluminium hydroxide với các dung dịch {\rm HCl, NaOH}.
\end{baitoan}

\begin{baitoan}[\cite{Nguyen_Buu_Can_500_BT_Hoa_Hoc_THCS}, 220., p. 99]
	(a) Phản ứng nào đặc trưng cho oxide base, phản ứng nào chỉ đặc trưng cho oxide base kiềm? (b) Phản ứng nào đặc trưng cho mọi base? Phản ứng nào đặc trưng cho kiềm?
\end{baitoan}

\begin{baitoan}[\cite{Nguyen_Buu_Can_500_BT_Hoa_Hoc_THCS}, 221., p. 99]
	Trình bày tính chất hóa học của sodium hydroxide.
\end{baitoan}

\begin{baitoan}[\cite{Nguyen_Buu_Can_500_BT_Hoa_Hoc_THCS}, 222., p. 99]
	Làm thế nào để điều chế được calcium hydroxide từ calcium oxide? Phương pháp này có thể áp dụng để điều chế copper ({\rm II}) hydroxide từ copper ({\rm II}) oxide được không? Vì sao?
\end{baitoan}

\begin{baitoan}[\cite{Nguyen_Buu_Can_500_BT_Hoa_Hoc_THCS}, 223., p. 99]
	Định nghĩa \& phân loại muối.
\end{baitoan}

\begin{baitoan}[\cite{Nguyen_Buu_Can_500_BT_Hoa_Hoc_THCS}, 224., p. 99]
	Muối X vừa tác dụng được với dung dịch {\rm HCl}, vừa tác dụng được với dung dịch {\rm NaOH}. Hỏi muối X thuộc loại muối trung hòa hay muối acid? Cho ví dụ minh họa.
\end{baitoan}

\begin{baitoan}[\cite{Nguyen_Buu_Can_500_BT_Hoa_Hoc_THCS}, 225., p. 99]
	Định nghĩa phản ứng trao đổi. Điều kiện để phản ứng trao đổi xảy ra, cho ví dụ minh họa. Phản ứng trung hòa có phải là phản ứng trao đổi không?
\end{baitoan}

\begin{baitoan}[\cite{Nguyen_Buu_Can_500_BT_Hoa_Hoc_THCS}, 226., p. 99]
	Khí {\rm\ce{CO2}} được điều chế bằng phản ứng giữa acid {\rm HCl \& \ce{CaCO3}} có lẫn hơi nước \& khí hydro chloride {\rm HCl}. Làm thế nào để thu được {\rm\ce{CO2}} tinh khiết?
\end{baitoan}

\begin{baitoan}[\cite{Nguyen_Buu_Can_500_BT_Hoa_Hoc_THCS}, 227., p. 100]
	Hoàn thành {\rm PTHH}: {\rm(a) \ce{H2SO4 + Ba(NO3)2}. (b) \ce{HCl + AgNO3}. (c) \ce{HNO3 + CaCO3}. (d) \ce{CuCl2 + KOH}. (e) \ce{FeSO4 + NaOH}. (f) \ce{Ba(NO3)2 + Na2SO4}. (g) \ce{MgSO4 + BaCl2}. (h) \ce{FeCl3 + NaOH}}. Giải thích tại sao phản ứng lại xảy ra.
\end{baitoan}

\begin{baitoan}[\cite{Nguyen_Buu_Can_500_BT_Hoa_Hoc_THCS}, 228., p. 100]
	Cho biết trong dung dịch có thể đồng thời tồn tại các chất sau được không? {\rm(a) NaOH, HBr. (b) \ce{H2SO4,BaCl2}. (c) KCl, \ce{NaNO3}. (d) \ce{Ca(OH)2,H2SO4}. (e) NaCl, KOH}.
\end{baitoan}

\begin{baitoan}[\cite{Nguyen_Buu_Can_500_BT_Hoa_Hoc_THCS}, 229., p. 100]
	Bổ túc \& cân bằng {\rm PTHH}: {\rm(a) NaCl $\to$ \ce{PbCl2 v}. (b) \ce{Fe(SO4)3 -> Fe(OH)3}. (c) HCl $\to$ \ce{CO2 ^}. (d) \ce{CO2 -> CaCO3 v}. (e) \ce{Ba(OH)2 -> BaSO4 v}. (f) \ce{Cu(NO3)2 -> Cu(OH)2 v}. (g) \ce{H2SO4 -> SO2 ^}}.
\end{baitoan}

\begin{baitoan}[\cite{Nguyen_Buu_Can_500_BT_Hoa_Hoc_THCS}, 230., p. 100]
	Cho biết trong dung dịch đồng thời có thể tồn tại các chất sau được không? {\rm(a) KCl, \ce{NaNO3}. (b) KOH, HCl. (c) \ce{Na3PO4,CaCl2}. (d) HBr, \ce{AgNO3}}.
\end{baitoan}

\begin{baitoan}[\cite{Nguyen_Buu_Can_500_BT_Hoa_Hoc_THCS}, 231., p. 100]
	Có 4 chất rắn: đá vôi, soda, muối ăn, potassium sulfate. Làm cách nào để phân biệt chúng khi chỉ được dùng nước \& 1 hóa chất? Viết {\rm PTHH}.
\end{baitoan}

\begin{baitoan}[\cite{Nguyen_Buu_Can_500_BT_Hoa_Hoc_THCS}, 232., p. 101]
	Có 3 ống nghiệm đựng 3 chất lỏng trong suốt, không màu là 3 dung dịch {\rm NaCl, HCl, \ce{Na2CO3}}. Không dùng thêm 1 chất nào khác kể cả quỳ tím, làm thế nào nhận ra từng chất.
\end{baitoan}

\begin{baitoan}[\cite{Nguyen_Buu_Can_500_BT_Hoa_Hoc_THCS}, 233., p. 101]
	Hòa tan {\rm15.5 g \ce{Na2O}} vào nước tạo thành {\rm0.5 L} dung dịch. (a) Tính nồng độ mol của dung dịch thu được. (b) Tính thể tích dung dịch {\rm\ce{H2SO4} 20\%}, $d = 1.14$ {\rm g{\tt/}mL}, cần để trung hòa dung dịch trên. (c) Tính nồng độ mol của dung dịch sau phản ứng.
\end{baitoan}

\begin{baitoan}[\cite{Nguyen_Buu_Can_500_BT_Hoa_Hoc_THCS}, 234., p. 101]
	(a) Tìm công thức của iron oxide trong đó iron chiếm $70\%$ khối lượng. (b) Khử hoàn toàn {\rm2.4 g} hỗn hợp {\rm CuO, \ce{Fe_xO_y}} cùng số mol như nhau bằng hydrogen thu được {\rm1.76 g} kim loại. Hòa tan kim loại đó bằng dung dịch {\rm HCl} dư thấy thoát ra {\rm0.448 L \ce{H2}} (đktc). Xác định công thức của iron oxide.
\end{baitoan}

\begin{baitoan}[\cite{Nguyen_Buu_Can_500_BT_Hoa_Hoc_THCS}, 235., p. 101]
	Cho {\rm9.4 kg \ce{K2O}} vào nước. Tính khối lượng {\rm\ce{SO2}} cần thiết phản ứng với dung dịch trên để tạo thành: (a) Muối trung hòa. (b) Muối acid. (c) Hỗn hợp muối acid \& muối trung hòa theo tỷ số mol là $1:2$. (d) Hỗn hợp muối acid \& muối trung hòa theo tỷ số mol là $a:b$ với $a,b\in\mathbb{R}$, $a,b > 0$ cho trước.
\end{baitoan}

\begin{baitoan}[\cite{Nguyen_Buu_Can_500_BT_Hoa_Hoc_THCS}, 236.a, p. 101]
	Hòa tan hoàn toàn {\rm1.44 g} kim loại hóa trị {\rm II} bằng {\rm250 mL} dung dịch {\rm\ce{H2SO4} 0.3M}. Để trung hòa lượng acid dư cần dùng {\rm60 mL} dung dịch {\rm NaOH 0.5M}. Xác định kim loại đó.
\end{baitoan}

\begin{baitoan}[\cite{Nguyen_Buu_Can_500_BT_Hoa_Hoc_THCS}, 236.b, p. 101]
	Trộn {\rm300 mL} dung dịch {\rm HCl 0.5M} với {\rm200 mL} dung dịch {\rm\ce{Ba(OH)2}} nồng độ $a${\rm M} thu được {\rm500 mL} dung dịch trong đó nồng độ {\rm HCl} là {\rm0.02M}. Tính $a$.
\end{baitoan}

\begin{baitoan}[\cite{Nguyen_Buu_Can_500_BT_Hoa_Hoc_THCS}, 237., p. 102]
	Cần thêm bao nhiêu {\rm g \ce{SO3}} vào {\rm100 g} dung dịch {\rm\ce{H2SO4} 10\%} để được dung dịch {\rm\ce{H2SO4} 20\%}.
\end{baitoan}

\begin{baitoan}[\cite{Nguyen_Buu_Can_500_BT_Hoa_Hoc_THCS}, 238., p. 102]
	Để hòa tan hoàn toàn {\rm5.1 g} oxide kim loại hóa trị {\rm III}, phải dùng {\rm43.8 g} dung dịch {\rm HCl 25\%}. Tìm oxide kim loại đó.
\end{baitoan}

\begin{baitoan}[\cite{Nguyen_Buu_Can_500_BT_Hoa_Hoc_THCS}, 239., p. 102]
	Dẫn khí {\rm\ce{CO2}} vào {\rm1.2 L} dung dịch {\rm\ce{Ca(OH)2} 0.1M} thấy tạo ra {\rm5 g} 1 muối không tan cùng với 1 muối tan. (a) Tính thể tích khí {\rm\ce{CO2}} đã dùng (đktc). (b) Tính khối lượng \& nồng độ mol của muối tan. (c) Tính thể tích {\rm\ce{CO2}} (đktc) trong trường hợp chỉ tạo ra muối không tan. Tính khối lượng muối không tan đó.
\end{baitoan}

\begin{baitoan}[\cite{Nguyen_Buu_Can_500_BT_Hoa_Hoc_THCS}, 240., p. 102]
	Dung dịch X chứa hỗn hợp {\rm HCl, \ce{H2SO4}}. Lấy {\rm50 mL} dung dịch X cho tác dụng với {\rm\ce{AgNO3}} dư thấy tạo thành {\rm2.87 g} kết tủa. Lấy {\rm50 mL} dung dịch X cho tác dụng với {\rm\ce{BaCl2}} dư thấy tạo thành {\rm4.66 g} kết tủa. (a) Tính nồng độ mol của mỗi acid trong dung dịch X. (b) Cần dùng bao nhiêu {\rm mL} dung dịch {\rm NaOH 0.2M} trung hòa {\rm50 mL} dung dịch X.
\end{baitoan}

\begin{baitoan}[\cite{Nguyen_Buu_Can_500_BT_Hoa_Hoc_THCS}, 241., p. 102]
	Sau khi nung {\rm8 g} 1 hỗn hợp zinc carbonate \& zinc oxide, thu được {\rm6.24 g ZnO}. (a) Tính \% khối lượng hỗn hợp ban đầu. (b) Khí sinh ra được cho vào 1 dung dịch calcium hydroxide. Tính khối lượng calcium hydroxide để phản ứng chỉ tạo thành muối không tan.
\end{baitoan}

\begin{baitoan}[\cite{Nguyen_Buu_Can_500_BT_Hoa_Hoc_THCS}, 242., p. 102]
	Để trung hòa 1 dung dịch chứa {\rm189 g \ce{HNO3}}, đầu tiên dùng dung dịch có chứa {\rm112 g KOH}. Sau đó lại đổ thêm dung dịch {\rm\ce{Ba(OH)2} 25\%} cho trung hòa hết acid. (a) Viết {\rm PTHH}. (b) Tính khối lượng dung dịch {\rm\ce{Ba(OH)2}} đã dùng.
\end{baitoan}

\begin{baitoan}[\cite{Nguyen_Buu_Can_500_BT_Hoa_Hoc_THCS}, 243., p. 103]
	Viết {\rm PTHH} thực hiện các biến hóa: {\rm\ce{FeS2} $\to$ \ce{SO2} $\to$ \ce{SO3} $\to$ \ce{H2SO4}}. Tính lượng acid sulfuric thu được từ {\rm60 kg} quặng pirit nếu hiệu suất phản ứng là {\rm85\%} so với lý thuyết.
\end{baitoan}

\begin{baitoan}[\cite{Nguyen_Buu_Can_500_BT_Hoa_Hoc_THCS}, 244., p. 103]
	Hòa tan {\rm3.1 g \ce{Na2O}} vào nước để được {\rm2 L} dung dịch. (a) Tính nồng độ mol của dung dịch thu được. (b) Muốn làm trung hòa dung dịch trên phải cần bao nhiêu {\rm g} dung dịch {\rm\ce{H2SO4} 20\%}. (c) Tính nồng độ phân tử {\rm g} của muối tạo thành sau phản ứng. Biết dung dịch {\rm\ce{H2SO4} 20\%} có khối lượng riêng {\rm1.14 g{\tt/}mL}.
\end{baitoan}

\begin{baitoan}[\cite{Nguyen_Buu_Can_500_BT_Hoa_Hoc_THCS}, 245., p. 103]
	Cho {\rm11 g} dung dịch {\rm\ce{H2SO4} 20\%} vào {\rm400 g} dung dịch {\rm\ce{BaCl2} 5.2\%}. (a) Viết {\rm PTHH} \& tính khối lượng kết tủa tạo thành. (b) Tính nồng độ $\%$ của các chất có trong dung dịch sau khi tách bỏ kết tủa.
\end{baitoan}

\begin{baitoan}[\cite{Nguyen_Buu_Can_500_BT_Hoa_Hoc_THCS}, 246., p. 103]
	(a) Có thể điều chế khí anhidrit sunfurơ bằng cách cho {\rm\ce{H2SO4}} đặc tác dụng với sulfur (lưu huỳnh) ở nhiệt độ cao, hay với copper kim loại khi đun nóng. Viết {\rm PTHH}. (b) Oxy hóa hoàn toàn {\rm8 L} khí anhidrit sunfurơ {\rm\ce{SO2}} (đktc). Sản phẩm thu được cho tan trong {\rm57.2 mL} dung dịch {\rm\ce{H2SO4} 60\%} khối lượng riêng {\rm1.5 g{\tt/}mL}. Tính nồng độ $\%$ của dung dịch acid thu được.
\end{baitoan}

\begin{baitoan}[\cite{Nguyen_Buu_Can_500_BT_Hoa_Hoc_THCS}, 247., p. 103]
	Trộn {\rm30 mL} dung dịch có chứa {\rm2.22 g} calcium chloride với {\rm70 mL} dung dịch chứa {\rm1.7 g} silver nitrate. (a) Viết {\rm PTHH}. (b) Tính lượng kết tủa thu được. (c) Tính nồng độ mol của chất còn lại trong dung dịch. Cho thể tích dung dịch sau phản ứng thay đổi không đáng kể.
\end{baitoan}

\begin{baitoan}[\cite{Nguyen_Buu_Can_500_BT_Hoa_Hoc_THCS}, 248., p. 104]
	Cho {\rm38.2 g} hỗn hợp {\rm\ce{Na2CO3,K2CO3}} vào dung dịch {\rm HCl}. Dẫn lượng khí sinh ra qua nước vôi trong có dư thu được {\rm30 g} kết tủa. Tính khối lượng mỗi muối trong hỗn hợp.
\end{baitoan}

\begin{baitoan}[\cite{Nguyen_Buu_Can_500_BT_Hoa_Hoc_THCS}, 249., p. 104]
	Từ $80$ tấn quặng pirit chứa {\rm40\%} sulfur, sản xuất được $92$ tấn acid sulfuric. (a) Tính hiệu suất của quá trình sản xuất. (b) Từ lượng acid sulfuric này, có thể pha chế được bao nhiêu tấn dung dịch {\rm\ce{H2SO4} 23\%}.
\end{baitoan}

\begin{baitoan}[\cite{Nguyen_Buu_Can_500_BT_Hoa_Hoc_THCS}, 250., p. 104]
	Hòa tan {\rm13.3 g} hỗn hợp gồm {\rm NaCl, KCl} vào nước được {\rm500 g} dung dịch A. Lấy $\frac{1}{10}$ dung dịch A cho phản ứng với {\rm\ce{AgNO3}} dư được {\rm2.87 g} kết tủa. (a) Tính số {\rm g} mỗi muối ban đầu dùng. (b) Tính nồng độ $\%$ các muối trong dung dịch A.
\end{baitoan}

\begin{baitoan}[\cite{Nguyen_Buu_Can_500_BT_Hoa_Hoc_THCS}, 251., p. 104]
	Để hòa tan {\rm2.4 g} oxide 1 kim loại hóa trị {\rm II} cần dùng {\rm2.19 g} acid {\rm HCl}. Xác định oxide kim loại đó.
\end{baitoan}

\begin{baitoan}[\cite{Nguyen_Buu_Can_500_BT_Hoa_Hoc_THCS}, 252., p. 104]
	Cho {\rm1.568 L} khí carbonic (đktc) lội chậm qua dung dịch có hòa tan {\rm3.2 g NaOH}. Xác định thành phần định tính \& định lượng chất được sinh ra sau phản ứng.
\end{baitoan}

\begin{baitoan}[\cite{Nguyen_Buu_Can_500_BT_Hoa_Hoc_THCS}, 253.a, p. 104]
	Viết {\rm PTHH} để thực hiện các biến hóa theo sơ đồ: {\rm(a) Cu $\to$ CuO $\to$ \ce{CuSO4} $\to$ \ce{Cu(OH)2} $\to$ CuO. (b) CaO $\to$ \ce{Ca(OH)2} $\to$ \ce{CaCO3} $\to$ CaO}.
\end{baitoan}

\begin{baitoan}[\cite{Nguyen_Buu_Can_500_BT_Hoa_Hoc_THCS}, 253.b, p. 104]
	Trộn 1 dung dịch chứa {\rm5.1 g} sodium chloride với 1 dung dịch chứa {\rm5.1 g} silver nitrate. Tính lượng kết tủa được tạo thành sau phản ứng.
\end{baitoan}

\begin{baitoan}[\cite{Nguyen_Buu_Can_500_BT_Hoa_Hoc_THCS}, 254., p. 105]
	Cần dùng bao nhiêu {\rm L} dung dịch {\rm NaOH 0.5M} để trung hòa {\rm250 mL} dung dịch X chứa acid {\rm HCl 2M \& \ce{H2SO4 1.5M}}?
\end{baitoan}

\begin{baitoan}[\cite{Nguyen_Buu_Can_500_BT_Hoa_Hoc_THCS}, 255., p. 105]
	Trộn {\rm50 mL} dung dịch {\rm\ce{Na2CO3} 0.2M} với {\rm100 mL} dung dịch {\rm\ce{CaCl2} 0.15M} thì thu được 1 lượng kết tủa đúng bằng lượng kết tủa thu được khi trộn {\rm50 mL \ce{Na2CO3}} cho trên với {\rm100 mL} dung dịch {\rm\ce{BaCl2}} nồng độ $a${\rm M}? Tính $a$.
\end{baitoan}

\begin{baitoan}[\cite{Nguyen_Buu_Can_500_BT_Hoa_Hoc_THCS}, 256., p. 105]
	Cho {\rm1 g} hợp chất chloride chưa biết hóa trị vào 1 dung dịch silver nitrate lấy dư. Ta thu được 1 chất kết tủa màu trắng, đem sấy khô \& cân nặng {\rm2.65 g}. Xác định công thức của iron chloride.
\end{baitoan}

\begin{baitoan}[\cite{Nguyen_Buu_Can_500_BT_Hoa_Hoc_THCS}, 257.a, p. 105]
	Có 3 gói phân hóa học {\rm KCl, \ce{NH4NO3}} \& superphosphat (supe lân). Dựa vào phản ứng đặc trưng nào để phân biệt chúng.
\end{baitoan}

\begin{baitoan}[\cite{Nguyen_Buu_Can_500_BT_Hoa_Hoc_THCS}, 257.b, p. 105]
	Điều chế phân đạm urê bằng cách cho khí carbonic tác dụng với amoniac {\rm\ce{NH3}} ở nhiệt độ áp suất cao \& có xúc tác theo {\rm PTHH}: {\rm\ce{CO2 + $2$NH3 ->[xt] CO(NH2)2 + H2O}}. Tính thể tích khí {\rm\ce{CO2,NH3}} (đktc) cần lấy để sản xuất $10$ tấn urê, hiệu suất của quá trình là $80\%$.
\end{baitoan}

\begin{baitoan}[\cite{Nguyen_Buu_Can_500_BT_Hoa_Hoc_THCS}, 258., p. 105]
	Hòa tan hoàn toàn {\rm55 g} hỗn hợp {\rm\ce{Na2CO3,Na2SO3}} trong {\rm250 g} dung dịch {\rm HCl 14.6\%}. Biết phản ứng chỉ tạo ra muối trung hòa. (a) Tính thể tích khí thu được sau phản ứng (đktc). (b) Tính nồng độ $\%$ của muối có trong dung dịch sau phản ứng.
\end{baitoan}

\begin{baitoan}[\cite{Nguyen_Buu_Can_500_BT_Hoa_Hoc_THCS}, 259., p. 105]
	Để hòa tan hoàn toàn {\rm3.6 g} magnesium phải dùng bao nhiêu {\rm mL} dung dịch hỗn hợp {\rm HCl 1M} \& {\rm\ce{H2SO4} 0.75M}.
\end{baitoan}

\begin{baitoan}[\cite{Nguyen_Buu_Can_500_BT_Hoa_Hoc_THCS}, 260., p. 105]
	Cho {\rm5.6 L} khí {\rm\ce{CO2}} lội qua dung dịch {\rm NaOH 20\%}, $D = 1.22$ {\rm g{\tt/}mol}. (a) Tính khối lượng muối tạo thành. (b) Tính nồng độ $\%$ các chất có trong dung dịch sau phản ứng.
\end{baitoan}

\begin{baitoan}[\cite{Nguyen_Buu_Can_500_BT_Hoa_Hoc_THCS}, 261., p. 106]
	Cho dung dịch {\rm\ce{H2SO4}} vào dung dịch {\rm NaOH} thu được {\rm3.6 g} muối sulfate acid \& {\rm2/84 g} muối trung tính. Tính lượng dung dịch {\rm\ce{H2SO4} 49\%} \& dung dịch $20\%$ đã dùng.
\end{baitoan}

\begin{baitoan}[\cite{Nguyen_Buu_Can_500_BT_Hoa_Hoc_THCS}, 262., p. 106]
	Hòa tan {\rm49.6 g} hỗn hợp gồm 1 muối sulfate \& 1 muối carbonate của cùng 1 kim loại hóa trị {\rm I} vào nước thu được dung dịch A. Chia dung dịch A làm 2 phần bằng nhau.
	\begin{itemize}
		\item Phần 1: Cho phản ứng với lượng dư dung dịch acid sulfuric thu được {\rm2.24 L} khí (đktc).
		\item Phần 2: Cho phản ứng với lượng dư dung dịch {\rm\ce{BaCl2}} thu được {\rm43 g} kết tủa trắng.
	\end{itemize}
	(a) Tìm công thức 2 muối ban đầu. (b) Tính $\%$ khối lượng các muối trên có trong hỗn hợp.
\end{baitoan}

\begin{baitoan}[\cite{Nguyen_Buu_Can_500_BT_Hoa_Hoc_THCS}, 263., p. 106]
	Tính thể tích dung dịch {\rm KOH 5.6\%}, $D = 1.045$, cần dùng để làm trung hòa hết {\rm350 mL} dung dịch {\rm\ce{H2SO4} 0.5M}.
\end{baitoan}

\begin{baitoan}[\cite{Nguyen_Buu_Can_500_BT_Hoa_Hoc_THCS}, 264., p. 106]
	Cho acid hydrochloric phản ứng với {\rm6 g} hỗn hợp dạng bột gồm {\rm Mg, MgO}. (a) Tính thành phần $\%$ khối lượng của {\rm MgO} có trong hỗn hợp nếu phản ứng tạo ra {\rm2.24 L} khí {\rm\ce{H2}} (đktc). (b) Tính thể tích dung dịch {\rm HCl 20\%}, $D = 1.1$ {\rm g{\tt/}mL}, vừa đủ để phản ứng với hỗn hợp đó.
\end{baitoan}

\begin{baitoan}[\cite{Nguyen_Buu_Can_500_BT_Hoa_Hoc_THCS}, 265., p. 106]
	Dung dịch A chứa hỗn hợp {\rm NaOH, \ce{Ba(OH)2}}. Để trung hòa {\rm50 mL} dung dịch A cần dùng {\rm60 mL} dung dịch {\rm HCl 0.1M}. Khi cho {\rm50 mL} dung dịch A tác dụng với 1 lượng dư {\rm\ce{Na2CO3}} thấy tạo thành {\rm0.197 g} kết tủa. Tính nồng độ mol của {\rm NaOH, \ce{Ba(OH)2}} trong dung dịch A.
\end{baitoan}

\begin{baitoan}[\cite{Nguyen_Buu_Can_500_BT_Hoa_Hoc_THCS}, 266., p. 106]
	Hòa tan hoàn toàn {\rm27.4 g} hỗn hợp gồm {\rm\ce{M2CO3,MHCO3}} ({\rm M} là kim loại kiềm) bằng {\rm500 mL} dung dịch {\rm HCl 1M} thấy thoát ra {\rm6.72 L \ce{CO2}} (đktc). Để trung hòa lượng acid còn dư phải dùng {\rm50 mL} dung dịch {\rm NaOH 2M}. (a) Xác định 2 muối ban đầu. (b) Tính $\%$ khối lượng các muối trên.
\end{baitoan}

\begin{baitoan}[\cite{Nguyen_Buu_Can_500_BT_Hoa_Hoc_THCS}, 267., p. 107]
	Thả {\rm12 g} hỗn hợp aluminium \& silver vào dung dịch {\rm\ce{H2SO4} 7.35\%}. Sau khi phản ứng kết thúc, thu được {\rm13.44 L} khí hydrogen (đktc). (a) Tính $\%$ khối lượng mỗi kim loại có trong hỗn hợp. (b) Tính thể tích dung dịch {\rm\ce{H2SO4}} cần dùng biết khối lượng riêng $d = 1.025$ {\rm g{\tt/}mL}.
\end{baitoan}

\begin{baitoan}[\cite{Nguyen_Buu_Can_500_BT_Hoa_Hoc_THCS}, 268., p. 107]
	Cho {\rm100 g} hỗn hợp 2 muối chloride của cùng 1 kim loại A có hóa trị {\rm II \& III} tác dụng hoàn toàn với 1 dung dịch {\rm NaOH} lấy dư. Biết khối lượng của hydroxide kim loại hóa trị {\rm II} là {\rm19.8 g} \& khối lượng chloride kim loại hóa trị {\rm II} bằng $\frac{1}{2}$ khối lượng mol của A. (a) Xác định kim loại A. (b) Tính $\%$ khối lượng của 2 muối trong hỗn hợp.
\end{baitoan}

\begin{baitoan}[\cite{Nguyen_Buu_Can_500_BT_Hoa_Hoc_THCS}, 269., p. 107]
	Các oxide: {\rm\ce{SO2,CO2}, CO, CaO, MgO, \ce{Na2O,Al2O3,N2O5,K2O}}. Các oxide vừa tác dụng được với nước, vừa tác dụng được với kiềm. {\rm(1) \ce{SO2,CO2,Na2O}, CO, CaO. (2) \ce{SO2,CO2,N2O5}. (3) \ce{Na2O,Al2O3}, CaO, MgO, CuO. (4) CaO, \ce{Na2O,K2O}. (5) \ce{Al2O3,K2O}, CuO, MgO, CO. {\sf A.} (2), (4). {\sf B.} (1), (2), (3). {\sf C.} (2), (3), (4). {\sf D.} (3), (5).}
\end{baitoan}

\begin{baitoan}[\cite{Nguyen_Buu_Can_500_BT_Hoa_Hoc_THCS}, 270., p. 107]
	Hợp chất nào sau đây là base? {\sf A.} Copper ({\rm II}) nitrate. {\sf B.} Potassium chloride. {\sf C.} sulfur dioxide. {\sf D.} calcium hydroxide.
\end{baitoan}

\begin{baitoan}[\cite{Nguyen_Buu_Can_500_BT_Hoa_Hoc_THCS}, 271., p. 108]
	1 trong các thuốc thử sau có thể dùng để phân biệt dung dịch sodium sulfate \& dung dịch sodium carbonate: {\sf A.} Dung dịch barium chloride. {\sf B.} Dung dịch acid hydrochloric. {\sf C.} Dung dịch lead (chì) ({\rm II}) nitrate. {\sf D.} Dung dịch sodium hydroxide.
\end{baitoan}

Cho các oxide: \ce{Al2O3,CaO,CO,Mn2O7,P2O5,N2O5,NO,SiO2,ZnO,Fe2O3}. Giải 4 bài toán tiếp theo:

\begin{baitoan}[\cite{Nguyen_Buu_Can_500_BT_Hoa_Hoc_THCS}, 272., p. 108]
	Oxide acid? {\rm{\sf A.} \ce{Al2O3,CO,P2O5,SiO2,NO}. {\sf B.} \ce{P2O5,N2O5,ZnO,Mn2O7}. {\sf C.} \ce{N2O5,P2O5,SiO2,Mn2O7}. {\sf D.} \ce{Al2O3,SiO2,NO}.}
\end{baitoan}

\begin{baitoan}[\cite{Nguyen_Buu_Can_500_BT_Hoa_Hoc_THCS}, 273., p. 108]
	Oxide base? {\rm{\sf A.} \ce{Al2O3,CaO,Fe2O3,SiO2}. {\sf B.} \ce{CaO,Fe2O3}. {\sf C.} \ce{Mn2O7,Fe2O3,ZnO,Al2O3}. {\sf D.} CaO, NO, CO, \ce{SiO2,Al2O3}.}
\end{baitoan}

\begin{baitoan}[\cite{Nguyen_Buu_Can_500_BT_Hoa_Hoc_THCS}, 274., p. 108]
	Oxide lưỡng tính? {\rm{\sf A.} \ce{Al2O3}, ZnO. {\sf B.} \ce{Mn2O7,SiO2}, NO, ZnO. {\sf C.} \ce{Fe2O3,CO,Al2O3,P2O5}. {\sf D.} ZnO, CO,\ce{Fe2O3,P2O5}.}
\end{baitoan}

\begin{baitoan}[\cite{Nguyen_Buu_Can_500_BT_Hoa_Hoc_THCS}, 275., p. 108]
	Oxide không tạo muối? {\rm{\sf A.} CaO, CO, \ce{SiO2}. {\sf B.} CO, NO. {\sf C.} NO, ZnO, \ce{Mn2O7}. {\sf D.} CaO, NO, \ce{Mn2O7,SiO2}.}
\end{baitoan}

\begin{baitoan}[\cite{Nguyen_Buu_Can_500_BT_Hoa_Hoc_THCS}, 276., p. 108]
	Để 1 mẫu sodium hydroxide trên miếng kính trong không khí, sau vài ngày thấy có chất rắn màu trắng phủ ngoài. Nếu nhỏ vài giọt dung dịch {\rm HCl} vào chất rắn màu trắng thấy có khí không màu, không mùi thoát ra. Chất rắn màu trắng này là sản phẩm của phản ứng sodium hydroxide với: {\sf A.} Oxygen trong không khí. {\sf B.} Hơi nước trong không khí. {\sf C.} Carbon dioxide \& oxygen trong không khí. {\sf D.} Carbon dioxide trong không khí.
\end{baitoan}

\begin{baitoan}[\cite{Nguyen_Buu_Can_500_BT_Hoa_Hoc_THCS}, 277., p. 109]
	Có 3 oxide màu trắng: {\rm MgO, \ce{Al2O3,Na2O}}. Có thể nhận biết được các chất đó bằng thuốc thử nào? {\sf A.} Chỉ dùng nước. {\sf B.} Chỉ dùng acid. {\sf C.} Chỉ dùng kiềm. {\sf D.} Dùng nước \& kiềm.
\end{baitoan}

\begin{baitoan}[\cite{Nguyen_Buu_Can_500_BT_Hoa_Hoc_THCS}, 278., p. 109]
	Các thí nghiệm nào sau đây sẽ tạo ra chất kết tủa khi trộn? (1) Dung dịch sodium chloride \& dung dịch lead nitrate. (2) Dung dịch sodium carbonate \& dung dịch zinc sulfate. (3) Dung dịch sodium sulfate \& dung dịch aluminium chloride. (4) Dung dịch zinc sulfate \& dung dịch copper ({\rm II}) chloride. (5) Dung dịch barium chloride \& dung dịch nitrate. {\sf A.} (1), (2), (5). {\sf B.} (1), (2), (3). {\sf C.} (2), (4), (5). {\sf D.} (3), (4), (5).
\end{baitoan}

\begin{baitoan}[\cite{Nguyen_Buu_Can_500_BT_Hoa_Hoc_THCS}, 279., p. 109]
	{\rm(1) \ce{H2 + $\ldots$ -> Cu + H2O}. (2) \ce{$\ldots$ + O2 -> $2$H2O}. (3) \ce{C + H2O -> CO + $\ldots$}. (4) \ce{Mg + H2O -> $\ldots$ + H2 ^}. (5) \ce{Mg + $2$HCl -> $\ldots$ + H2 ^}.} Các chất được điền vào chỗ trống lần lượt là: {\rm{\sf A.} Mg, \ce{H2,Cl,O2,H2}. {\sf B.} CuO, \ce{H2,H2,MgO,MgCl2}. {\sf C.} \ce{H2,Cu,Mg,O2,H2O}. {\sf D.} \ce{H2,CuO,MgO,O2,H2}.}
\end{baitoan}

\begin{baitoan}[\cite{Nguyen_Buu_Can_500_BT_Hoa_Hoc_THCS}, 280., pp. 109--110]
	Có các chất: copper, copper ({\rm II}) oxide, magnesium carbonate, magnesium, magnesium oxide. Chất nào tác dụng với dung dịch acid hydrochloric hoặc acid sulfuric loãng sinh ra: (a) Chất khí cháy được trong không khí? (b) Chất khí làm đục nước vôi trong? (c) Dung dịch có màu xanh? (d) Dung dịch không màu \& nước?
\end{baitoan}

\begin{baitoan}[\cite{Nguyen_Buu_Can_500_BT_Hoa_Hoc_THCS}, 281., p. 110]
	Có 4 oxide: {\rm I. \ce{SO3}. II. CaO. III. \ce{CrO3}. IV. MgO}. Tập hợp nào sau đây chỉ gồm oxide acid? {\rm{\sf A.} I, II. {\sf B.} II, III. {\sf C.} I, III. {\sf D.} III, IV}.
\end{baitoan}

\begin{baitoan}[\cite{Nguyen_Buu_Can_500_BT_Hoa_Hoc_THCS}, 282., p. 110]
	Cho phương trình phản ứng: {\rm\ce{$2$NaOH + X -> $2$Y + H2O}}. X, Y? {\rm{\sf A.} \ce{H2SO4, Na2SO4}. {\sf B.} \ce{N2O5,NaNO3}. {\sf C.} HCl, NaCl. {\sf D.} A, B đều đúng.}
\end{baitoan}

\begin{baitoan}[\cite{Nguyen_Buu_Can_500_BT_Hoa_Hoc_THCS}, 283., p. 110]
	Cho sơ đồ chuyển hóa: {\rm X $\to$ \ce{SO2} $\to$ Y $\to$ \ce{H2SO4}} với X là chất rắn. X, Y? {\rm{\sf A.} X: \ce{FeS2}, Y: \ce{SO3}. {\sf B.} X: \ce{FeS2} hoặc S, Y: \ce{SO3}. {\rm C.} X: \ce{O2}, Y: \ce{SO3}.} {\sf D.} Tất cả đều đúng.
\end{baitoan}

\begin{baitoan}[\cite{Nguyen_Buu_Can_500_BT_Hoa_Hoc_THCS}, 284., p. 110]
	Có 5 ống nghiệm chứa các dung dịch sau: {\rm\ce{Ba(NO3)2,H2SO4,NaOH,Na2CO3}}. Chỉ dùng 1 hóa chất duy nhất để nhận biết các hóa chất ở trong ống nghiệm: {\sf A.} Dùng phenolphtalein không màu. {\sf B.} Dùng giấy quỳ tím. {\sf C.} Dùng dung dịch acid {\rm HCl}. {\sf D.} Dùng dung dịch {\rm\ce{BaCl2}}.
\end{baitoan}

\begin{baitoan}[\cite{Nguyen_Buu_Can_500_BT_Hoa_Hoc_THCS}, 285., p. 110]
	Có các chất rắn: {\rm MgO, \ce{P2O5,Ba(OH)2,Na2SO4}}. Dùng các thuốc thử nào có thể phân biệt được các chất này? {\sf A.} Dùng {\rm\ce{H2O}}, giấy quỳ tím. {\sf B.} Dùng acid {\rm\ce{H2SO4}}, phenolphtalein không màu. {\sf C.} Dùng dung dịch {\rm NaOH}, quỳ tím. {\sf D.} Tất cả đều sai.
\end{baitoan}

\begin{baitoan}[\cite{Nguyen_Buu_Can_500_BT_Hoa_Hoc_THCS}, 286., p. 111]
	Có 5 dung dịch: {\rm\ce{Na2CO3,BaCl2,CH3COONa,Ba(HCO3)2}, NaCl}. Chỉ dùng dung dịch {\rm\ce{H2SO4}} có thể nhận biết được mấy chất? {\sf A.} $1$. {\sf B.} $2$. {\sf C.} $3$. {\sf D.} $5$.
\end{baitoan}

\begin{baitoan}[\cite{Nguyen_Buu_Can_500_BT_Hoa_Hoc_THCS}, 287., p. 111]
	Để loại bỏ khí {\rm\ce{CO2}} có lẫn trong hỗn hợp {\rm\ce{O2,CO2}}, cho hỗn hợp đi qua dung dịch chứa: {\rm{\sf A.} HCl. {\sf B.} \ce{Na2SO4}. {\sf C.} NaCl. {\sf D.} \ce{Ca(OH)2}}.
\end{baitoan}

\begin{baitoan}[\cite{Nguyen_Buu_Can_500_BT_Hoa_Hoc_THCS}, 288., p. 111]
	Nhỏ 1 giọt quỳ tím vào dung dịch {\rm NaOH}, dung dịch có màu xanh, nhỏ từ từ dung dịch {\rm HCl} cho tới dư vào dung dịch có màu xanh trên thì: {\sf A.} Màu xanh vẫn không thay đổi. {\sf B.} Màu xanh nhạt dần rồi mất hẳn. {\sf C.} Màu xanh nhạt dần, mất hẳn rồi chuyển sang màu đỏ. {\sf D.} Màu xanh đậm thêm dần.
\end{baitoan}

\begin{baitoan}[\cite{Nguyen_Buu_Can_500_BT_Hoa_Hoc_THCS}, 289., p. 111]
	Các cặp chất nào sau đây cùng tồn tại trong 1 dung dịch? {\rm{\sf A.} KCl, \ce{NaNO3}. {\sf B.} KOH, HCl. {\sf C.} HCl, \ce{AgNO3}. {\sf D.} \ce{NaHCO3}, NaOH}.
\end{baitoan}

\begin{baitoan}[\cite{Nguyen_Buu_Can_500_BT_Hoa_Hoc_THCS}, 290., p. 111]
	Để hòa tan hết {\rm5.1 g \ce{M2O3}} phải dùng {\rm43.8 g} dung dịch {\rm HCl 25\%}. Công thức của {\rm\ce{M2O3}}? {\rm{\sf A.} \ce{Fe2O3}. {\sf B.} \ce{Al2O3}. {\sf C.} \ce{Cr2O3}.} {\sf D.} Tất cả đều sai.
\end{baitoan}

\begin{baitoan}[\cite{An_350_BT_Hoa_Hoc_9}, 5.a, p. 6]
	Cho $a$ \emph{g Na} tác dụng với $p$ \emph{g} nước thu được dung dịch \emph{NaOH} nồng độ $x$\%. Cho $b$ \emph{g \ce{Na2O}} tác dụng với $p$ \emph{g} nước cũng thu được dung dịch \emph{NaOH} nồng độ $x$\%. Lập biểu thức tính $p$ theo $a,b$.
\end{baitoan}

\begin{proof}[Giải]
	
\end{proof}

\begin{baitoan}[\cite{An_350_BT_Hoa_Hoc_9}, 5.b, p. 6]
	Khử hoàn toàn \emph{3.2 g} hỗn hợp \emph{CuO, \ce{Fe2O3}} bằng \emph{\ce{H2}} tạo ra \emph{0.9 g \ce{H2O}}. Tính khối lượng hỗn hợp kim loại thu được.
\end{baitoan}

\begin{baitoan}[\cite{An_350_BT_Hoa_Hoc_9}, 6.a, p. 7]
	Cho \emph{2.24 L \ce{CO2}} (đktc) tác dụng hoàn toàn với \emph{25 g} dung dịch \emph{NaOH 20\%}. Tính khối lượng muối tạo thành.
\end{baitoan}

\begin{baitoan}[\cite{An_350_BT_Hoa_Hoc_9}, 7.a, p. 8]
	Nung $m$ \emph{g} hỗn hợp chất rắn A gồm \emph{\ce{Fe2O3}} \& \emph{FeO} với lượng thiếu khí \emph{CO} thu được hỗn hợp chất rắn B có khối lượng \emph{47.84 g} \& \emph{5.6 L \ce{CO2}}. Tính $m$.
\end{baitoan}

\begin{baitoan}[\cite{An_350_BT_Hoa_Hoc_9}, 7.b, p. 9]
	Cho \emph{11.6 g} hỗn hợp \emph{\ce{Fe2O3}} \& \emph{FeO} có tỷ lệ số mol là $1:1$ vào \emph{300 mL} dung dịch \emph{HCl 2M} được dung dịch A. Tính nồng độ mol của các chất trong dung dịch sau phản ứng (thể tích dung dịch thay đổi không đáng kể).
\end{baitoan}

\begin{baitoan}[\cite{An_350_BT_Hoa_Hoc_9}, 8.a, p. 9]
	Nung nóng kim loại M trong không khí đến khối lượng không đổi thu được chất rắn N. Khối lượng của M bằng $\frac{7}{10}$ khối lượng của N. Tìm CTPT của N.
\end{baitoan}

\begin{baitoan}[\cite{An_350_BT_Hoa_Hoc_9}, 8.b, p. 9]
	Cho 1 oxide base tác dụng với dung dịch \emph{\ce{H2SO4} 24.5\%} thu được dung dịch 1 muối có nồng độ \emph{32.2\%}. Tìm CTPT của oxide base.
\end{baitoan}

\begin{baitoan}[\cite{An_350_BT_Hoa_Hoc_9}, 9.a, p. 11]
	Dẫn $V$ \emph{L} khí \emph{\ce{CO2}} (đktc) qua \emph{250 mL} dung dịch \emph{\ce{Ca(OH)2} 1M} thấy có \emph{12.5 g} kết tủa. Tính $V$.
\end{baitoan}

\begin{baitoan}[\cite{An_350_BT_Hoa_Hoc_9}, 9.b, p. 11]
	Dùng khí \emph{\ce{H2}} để khử $a$ \emph{g} oxide sắt. Sản phẩm hơi tạo ra cho qua $100$ \emph{g} acid \emph{\ce{H2SO4} 98\%} thì nồng độ acid giảm đi \emph{3.405\%}. Chất rắn thu được sau phản ứng trên cho tác dụng hết với dung dịch \emph{HCl} thấy thoát ra \emph{3.36 L} \emph{\ce{H2}} (đktc). Xác định CTPT oxide sắt.
\end{baitoan}

\begin{baitoan}[\cite{An_350_BT_Hoa_Hoc_9}, 10.a, p. 13]
	Để xác định CTPT oxide sắt người ta làm thí nghiệm như sau: Hòa tan $a$ \emph{g} oxide sắt thì cần \emph{300 mL} dung dịch \emph{HCl 3M}. Cho toàn bộ $a$ \emph{g} oxide sắt nung nóng tác dụng với \emph{CO} dư thu được \emph{16.8 g} sắt. Xác định CTPT oxide sắt.
\end{baitoan}

\begin{baitoan}[\cite{An_350_BT_Hoa_Hoc_9}, 10.b, p. 13]
	1 loại đá vôi chứa \emph{80\% \ce{CaCO3}} \& \emph{20\%} tạp chất không bị phân hủy bởi nhiệt. Khi nung $a$ \emph{g} đá vôi trên thu được chất rắn có khối lượng bằng \emph{75\%} khối lượng đá trước khi nung. (a) Tính hiệu suất phản ứng phân hủy \emph{\ce{CaCO3}}. (b) Tính thành phần \% khối lượng \emph{CaO} trong chất rắn sau khi nung.
\end{baitoan}

\begin{baitoan}[\cite{An_350_BT_Hoa_Hoc_9}, 11.a, p. 14]
	Khử hoàn toàn \emph{5.8 g} 1 oxide sắt bằng \emph{CO} ở nhiệt độ cao. Sản phẩm sau phản ứng cho qua dung dịch nước vôi trong dư tạo \emph{10 g} kết tủa. Xác định CTPT oxide sắt.
\end{baitoan}

\begin{baitoan}[\cite{An_350_BT_Hoa_Hoc_9}, 11.b, p. 14]
	Nung $1.5$ tấn đá vôi chứa \emph{85\% \ce{CaCO3}} thì có thể thu được bao nhiêu \emph{kg} vôi sống? Biết hiệu suất phản ứng là \emph{90\%}.
\end{baitoan}

\begin{baitoan}[\cite{An_350_BT_Hoa_Hoc_9}, 12.a, p. 15]
	Cho \emph{7.84 g CaO} tan hoàn toàn vào nước được dung dịch A. Dẫn \emph{2.24 L} khí \emph{\ce{CO2}} (đktc) vào dung dịch A. Tính khối lượng các chất sau phản ứng.
\end{baitoan}

\begin{baitoan}[\cite{An_350_BT_Hoa_Hoc_9}, 12.b, p. 15]
	Nung $1$ tấn đá vôi thì thu được \emph{428.4 kg} vôi sống \emph{CaO}. Hiệu suất quá trình nung vôi là \emph{85\%}, tính tỷ lệ \emph{\%} khối lượng tạp chất có trong đá vôi.
\end{baitoan}

%------------------------------------------------------------------------------%

\section{Acid}

\subsection{Qualitative problem -- Bài tập định tính}

\begin{baitoan}[\cite{SGK_KHTN_8_Canh_Dieu}, 1, p. 47]
	Nêu đặc điểm chung về thành phần phân tử của các acid.
\end{baitoan}

\begin{baitoan}[\cite{SGK_KHTN_8_Canh_Dieu}, 1, p. 47]
	Viết sơ đồ tạo thành ion \emph{\ce{H+}} từ nitric acid \emph{\ce{HNO3}}.
\end{baitoan}

\begin{baitoan}[\cite{SGK_KHTN_8_Canh_Dieu}, 2, p. 48]
	Khi thảo luận về tác dụng của dung dịch acid với quỳ tím có 2 ý kiến sau: (a) Nước làm quỳ tím đổi màu. (b) Dung dịch acid làm quỳ tím đổi màu. Đề xuất 1 thí nghiệm để xác định ý kiến đúng trong 2 ý kiến trên.
\end{baitoan}

\begin{baitoan}[\cite{SGK_KHTN_8_Canh_Dieu}, 3, p. 48]
	Lần lượt nhỏ lên 3 mẩu giấy quỳ tím mỗi dung dịch sau: (a) Nước đường. (b) Nước chanh. (c) Nước muối (dung dịch \emph{NaCl}). Trường hợp nào quỳ tím sẽ chuyển sang màu đỏ?
\end{baitoan}

\begin{baitoan}[\cite{SGK_KHTN_8_Canh_Dieu}, 1, p. 49]
	Người ta thường tránh muối dưa, cà trong các dụng cụ làm bằng nhôm. Cho biết lý do của việc làm trên.
\end{baitoan}

\begin{baitoan}[\cite{SGK_KHTN_8_Canh_Dieu}, 4, p. 49]
	Viết PTHH xảy ra trong các trường hợp sau: (a) Dung dịch \emph{\ce{H2SO4}} loãng tác dụng với \emph{Zn}. (b) Dung dịch \emph{HCl} loãng tác dụng với \emph{Mg}.
\end{baitoan}

\begin{baitoan}[\cite{SGK_KHTN_8_Canh_Dieu}, 2, p. 50]
	Nêu tên 1 số món ăn có sử dụng giấm ăn trong quá trình chế biến.
\end{baitoan}

\begin{baitoan}[\cite{SGK_Hoa_Hoc_9}, 1., p. 14]
	Từ \emph{Mg, MgO, \ce{Mg(OH)2}} \& dung dịch acid sulfuric loãng, viết các PTHH của phản ứng điều chế magnesium sulfate.
\end{baitoan}

\begin{baitoan}[\cite{SGK_Hoa_Hoc_9}, 2., p. 14]
	Có các chất sau: \emph{CuO, Mg, \ce{Al2O3,Fe(OH)3,Fe2O3}}. Chọn 1 trong các chất đã cho tác dụng với dung dịch \emph{HCl} sinh ra: (a) khí nhẹ hơn không khí \& cháy được trong không khí. (b) dung dịch có màu xanh lam. (c) dung dịch có màu vàng nâu. (d) dung dịch không có màu. Viết các PTHH.
\end{baitoan}

\begin{baitoan}[\cite{SGK_Hoa_Hoc_9}, 3., p. 14]
	Viết các PTHH: (a) magnesium oxide \& acid nitric. (b) copper(II) oxide \& hydrochloric acid. (c) aluminium oxide \& sulfuric acid. (d) iron \& hydrochloric acid. (e) zinc \& sulfuric acid loãng.
\end{baitoan}

\begin{baitoan}[\cite{SGK_Hoa_Hoc_9}, 1., p. 19]
	Có các chất: \emph{CuO, \ce{BaCl2}, Zn, ZnO}. Chất nào tác dụng với dung dịch \emph{HCl}, dung dịch \emph{\ce{H2SO4}} loãng sinh ra: (a) chất khí cháy được trong không khí? (b) dung dịch có màu xanh lam? (c) chất kết tủa màu trắng không tan trong nước \& acid? (d) dung dịch không màu \& nước? Viết tất cả các PTHH.
\end{baitoan}

\begin{baitoan}[\cite{SGK_Hoa_Hoc_9}, 2., p. 19]
	Sản xuất acid sulfuric trong công nghiệp cần phải có các nguyên liệu chủ yếu nào? Cho biết mục đích của mỗi công đoạn sản xuất acid sulfuric \& dẫn ra các phản ứng hóa học.
\end{baitoan}

\begin{baitoan}[\cite{SGK_Hoa_Hoc_9}, 3., p. 19]
	Bằng cách nào có thể nhận biết được từng chất trong mỗi cặp chất sau theo phương pháp hóa học? (a) Dung dịch \emph{HCl} \& dung dịch \emph{H2SO4}. (b) Dung dịch \emph{NaCl} \& dung dịch \emph{\ce{Na2SO4}}. (c) Dung dịch \emph{\ce{Na2SO4}} \& dung dịch \emph{\ce{H2SO4}}. Viết các PTHH.
\end{baitoan}

\begin{baitoan}[\cite{SGK_Hoa_Hoc_9}, 5., p. 19]
	Sử dụng các chất có sẵn: \emph{Cu, Fe, CuO, KOH, \ce{C6H12O6}} (glucose), dung dịch \emph{\ce{H2SO4}} loãng, \emph{\ce{H2SO4}} đặc \& các dụng cụ thí nghiệm cần thiết để làm các thí nghiệm chứng minh: (a) Dung dịch \emph{\ce{H2SO4}} loãng có các tính chất hóa học của acid. (b) \emph{\ce{H2SO4}} đặc có các tính chất hóa học riêng. Viết PTHH cho mỗi thí nghiệm.
\end{baitoan}

\begin{baitoan}[\cite{SGK_Hoa_Hoc_9}, 1., p. 21]
	Có các oxide: \emph{\ce{SO2, CuO, Na2O, CO2}}. Cho biết các oxide nào tác dụng được với: (a) nước. (b) hydrochloric acid. (c) sodium hydroxide. Viết các PTHH.
\end{baitoan}

\begin{baitoan}[\cite{SGK_Hoa_Hoc_9}, 2., p. 21]
	Các oxide nào sau: \emph{\ce{H2O,CuO,Na2O,CO2,P2O5}} có thể điều chế bằng: (a) phản ứng hóa hợp? Viết PTHH. (b) phản ứng hóa hợp \& phản ứng phân hủy? Viết PTHH.
\end{baitoan}

\begin{baitoan}[\cite{SGK_Hoa_Hoc_9}, 3., p. 21]
	Khí \emph{CO} được dùng làm chất đốt trong công nghiệp, có lẫn tạp chất là các khí \emph{\ce{CO2,SO2}}. Làm thế nào có thể loại bỏ được các tạp chất ra khỏi \emph{CO} bằng hóa chất rẻ tiền nhất? Viết các PTHH.
\end{baitoan}

\begin{baitoan}[\cite{SGK_Hoa_Hoc_9}, 4., p. 21]
	Cần phải điều chế 1 lượng muối copper(II) sulfate. Phương pháp nào sau đây tiết kiệm được acid sulfuric? (a) Acid sulfuric tác dụng với copper(II) oxide. (b) Acid sulfuric đặc tác dụng với kim loại đồng. Vì sao?
\end{baitoan}

\begin{baitoan}[\cite{SGK_Hoa_Hoc_9}, 5., p. 21]
	Thực hiện các chuyển đổi hóa học sau bằng cách viết các PTHH (ghi điều kiện của phản ứng, nếu có): (a) \emph{S $\to$ \ce{SO2} $\to$ \ce{SO3} $\to$ \ce{H2SO4}}. (b) \emph{\ce{SO2} $\to$ \ce{Na2SO3}}. (c) \emph{\ce{H2SO4} $\to$ \ce{SO2} $\to$ \ce{H2SO3} $\to$ \ce{Na2SO3} $\to$ \ce{SO2}}. (d) \emph{\ce{H2SO4} $\to$ \ce{Na2SO4} $\to$ \ce{BaSO4}}.
\end{baitoan}

\begin{baitoan}[\cite{SBT_Hoa_Hoc_9}, 3.1., p. 5]
	Dung dịch \emph{HCl} đều tác dụng với các chất trong dãy nào sau đây? {\sf A.} \emph{Mg, \ce{Fe2O3,Cu(OH)2}, Ag}. {\sf B.} \emph{Fe, MgO, \ce{Zn(OH)2,Na2SO4}}. {\sf C.} \emph{CuO, Al, \ce{Fe(OH)3,CaCO3}}. {\sf D.} \emph{Zn, BaO, \ce{Mg(OH)2,SO2}}.
\end{baitoan}

\begin{baitoan}[\cite{SBT_Hoa_Hoc_9}, 3.2., p. 5]
	Có các dung dịch \emph{KOH, HCl, \ce{H2SO4}} (loãng), các chất rắn \emph{\ce{Fe(OH)3}, Cu} \& các chất khí \emph{\ce{CO2}, NO}. Các chất nào có thể tác dụng với nhau từng đôi một? Viết các PTHH. (Biết \emph{\ce{H2SO4}} loãng không tác dụng với \emph{Cu}.)
\end{baitoan}

\begin{baitoan}[\cite{SBT_Hoa_Hoc_9}, 3.3., p. 6]
	Có các oxide: \emph{\ce{Fe2O3,SO2,CuO,MgO,CO2}}. (a) Các oxide nào tác dụng được với dung dịch \emph{\ce{H2SO4}}? (b) Các oxide nào tác dụng được với dung dịch \emph{NaOH}? (c) Các oxide nào tác dụng được với \emph{\ce{H2O}}? Viết các PTHH.
\end{baitoan}

\begin{baitoan}[\cite{SBT_Hoa_Hoc_9}, 3.4., p. 6]
	Có hỗn hợp gồm bột kim loại đồng \& sắt. Chọn phương pháp hóa học để tách riêng bột đồng ra khỏi hỗn hợp. Viết các PTHH.
\end{baitoan}

\begin{baitoan}[\cite{SBT_Hoa_Hoc_9}, 4.1., p. 6]
	Dung dịch \emph{\ce{H2SO4}} tác dụng được với các chất trong dãy: {\sf A.} \emph{CuO, \ce{BaCl2,NaCl,FeCO3}}. {\sf B.} \emph{Cu, \ce{Cu(OH)2,Na2CO3}, KCl}. {\sf C.} \emph{Fe, ZnO, \ce{MgCl2}, NaOH}. {\sf D.} \emph{Mg, \ce{BaCl2,K2CO3,Al2O3}}.
\end{baitoan}

\begin{baitoan}[\cite{SBT_Hoa_Hoc_9}, 4.2., pp. 6--7]
	Cần phải điều chế 1 lượng muối đồng sulfate. Phương pháp nào sau đây tiết kiệm được acid sulfuric? (a) Acid sulfuric tác dụng với copper(II) oxide. (b) Acid sulfuric đặc tác dụng với copper kim loại. Viết các PTHH \& giải thích.
\end{baitoan}

\begin{baitoan}[\cite{SBT_Hoa_Hoc_9}, 4.3., p. 7]
	Cho các chất sau: đồng, các hợp chất của đồng \& acid sulfuric. Viết các PTHH điều chế đồng(II) sulfate từ các chất đã cho, cần ghi rõ các điều kiện của phản ứng.
\end{baitoan}

\begin{baitoan}[\cite{SBT_Hoa_Hoc_9}, 4.4., p. 7]
	Có 3 lọ không nhãn, mỗi lọ đựng 1 trong các chất rắn: \emph{CuO, \ce{BaCl2,Na2CO3}}. Chọn 1 thuốc thử để có thể nhận biết được cả 3 chất trên. Giải thích \& viết PTHH.
\end{baitoan}

\begin{baitoan}[\cite{SBT_Hoa_Hoc_9}, 4.5., p. 7]
	Có 4 lọ không nhãn, mỗi lọ đựng 1 dung dịch không màu: \emph{HCl, NaCl, \ce{H2SO4, Na2SO4}}. Nhận biết dung dịch đựng trong mỗi lọ bằng phương pháp hóa học. Viết các PTHH.
\end{baitoan}

\begin{baitoan}[\cite{SBT_Hoa_Hoc_9}, 5.1., p. 7]
	Có các chất sau: \emph{Cu, Zn, MgO, NaOH, \ce{Na2CO3}}. Dẫn ra các phản ứng hóa học của dung dịch \emph{HCl} \& dung dịch \emph{\ce{H2SO4}} loãng với các chất đã cho để chứng minh 2 acid này có tính chất hóa học giống nhau.
\end{baitoan}

\begin{baitoan}[\cite{SBT_Hoa_Hoc_9}, 5.2., p. 8]
	Để phân biệt được 2 dung dịch \emph{\ce{Na2SO4,Na2CO3}}, người ta dùng: {\sf A.} \emph{\ce{BaCl2}}. {\sf B.} \emph{HCl}. {\sf C.} \emph{\ce{Pb(NO3)2}}. {\sf D.} \emph{NaOH}.
\end{baitoan}

\begin{baitoan}[\cite{SBT_Hoa_Hoc_9}, 5.3., p. 8]
	Điền các chất: \emph{CuO, MgO, \ce{H2O,SO2,CO2}} thích hợp vào các PTHH \& cân bằng chúng: (a) \emph{\ce{HCl + $\ldots$ -> CuCl2 + $\ldots$}}. (b) \emph{\ce{H2SO4 + Na2SO3 -> Na2SO4 + $\ldots$ + $\ldots$}}. (c) \emph{\ce{HCl + CaCO3 -> CaCl2 + $\ldots$ + $\ldots$}}. (d) \emph{\ce{H2SO4 + $\ldots$ -> MgSO4 + $\ldots$}}. (e) $\ldots$ $+$ $\ldots$ \emph{\ce{<=> H2SO3}}.
\end{baitoan}

\begin{baitoan}[\cite{SBT_Hoa_Hoc_9}, 5.4., p. 8]
	Cho các chất: \emph{Cu, \ce{Na2SO3, H2SO4}}. (a) Viết các PTHH của phản ứng điều chế \emph{\ce{SO2}} từ các chất này. (b) Cần điều chế $n$ \emph{mol \ce{SO2}}, chọn chất nào để tiết kiệm được \emph{\ce{H2SO4}}. Giải thích cho sự lựa chọn.
\end{baitoan}

\begin{baitoan}[\cite{An_350_BT_Hoa_Hoc_9}, 24.a, p. 24]
	Bằng phương pháp hóa học, phân biệt 3 dung dịch: \emph{HCl, NaOH, \ce{Ba(OH)2}}.
\end{baitoan}

\subsection{Quantitative problem -- Bài tập định lượng}

\begin{baitoan}[\cite{SGK_Hoa_Hoc_9}, 4., p. 14]
	Có \emph{10 g} hỗn hợp bột 2 kim loại đồng \& sắt. Giới thiệu phương pháp xác định thành phần \% (theo khối lượng) của mỗi kim loại trong hỗn hợp theo: (a) Phương pháp hóa học. Viết PTHH. (b) Phương pháp vật lý. (Biết copper không tác dụng với acid \emph{HCl} \& acid \emph{\ce{H2SO4}} loãng).
\end{baitoan}

\begin{baitoan}[\cite{SGK_Hoa_Hoc_9}, 4., p. 19]
	Bảng sau cho biết kết quả của $6$ thí nghiệm xảy ra giữa \emph{Fe} \& dung dịch \emph{\ce{H2SO4}} loãng. Trong mỗi thí nghiệm người ta dùng \emph{0.2 g Fe} tác dụng với thể tích bằng nhau của acid, nhưng có nồng độ khác nhau.
	\begin{table}[H]
		\centering
		\begin{tabular}{|c|c|c|c|c|}
			\hline
			Thí nghiệm & Nồng độ acid & Nhiệt độ (${}^\circ$C) & Sắt ở dạng & Thời gian phản ứng xong (s) \\
			\hline
			1 & 1M & 25 & Lá & 190 \\
			\hline
			2 & 2M & 25 & Bột & 85 \\
			\hline
			3 & 2M & 35 & Lá & 62 \\
			\hline
			4 & 2M & 50 & Bột & 15 \\
			\hline
			5 & 2M & 35 & Bột & 45 \\
			\hline
			6 & 3M & 50 & Bột & 11 \\
			\hline
		\end{tabular}
	\end{table}
	\noindent Các thí nghiệm nào chứng tỏ: (a) Phản ứng xảy ra nhanh hơn khi tăng nhiệt độ? (b) Phản ứng xảy ra nhanh hơn khi tăng diện tích tiếp xúc? (c) Phản ứng xảy ra nhanh hơn khi tăng nồng độ acid?
\end{baitoan}

\begin{baitoan}[\cite{SGK_Hoa_Hoc_9}, 6., p. 19]
	Cho 1 lượng mạt sắt dư vào \emph{50 mL} dung dịch \emph{HCl}. Phản ứng xong, thu được \emph{3.36 L} khí (đktc). (a) Viết PTHH. (b) Tính khối lượng mạt sắt đã tham gia phản ứng. (c) Tìm nồng độ mol của dung dịch \emph{HCl} đã dùng.
\end{baitoan}

\begin{baitoan}[\cite{SGK_Hoa_Hoc_9}, 7., p. 19]
	Hòa tan hoàn toàn \emph{12.1 g} hỗn hợp bột \emph{CuO, ZnO} cần \emph{100 mL} dung dịch \emph{HCl 3M}. (a) Viết các PTHH. (b) Tính \% theo khối lượng của mỗi oxide trong hỗn hợp ban đầu. (c) Tính khối lượng dung dịch \emph{\ce{H2SO4}} nồng độ \emph{20\%} để hòa tan hoàn toàn hỗn hợp các oxide trên.
\end{baitoan}

\begin{baitoan}[\cite{SBT_Hoa_Hoc_9}, 3.5., p. 6]
	Tìm CTHH của các acid có thành phần khối lượng sau: (a) \emph{H: 2.1\%, N: 29.8\%, O: 68.1\%}. (b) \emph{H: 2.4\%, S: 39.1\%, O: 58.5\%}. (c) \emph{H: 3.7\%, P: 37.8\%, O: 58.5\%}.
\end{baitoan}

\begin{baitoan}[\cite{SBT_Hoa_Hoc_9}, 3.6., p. 6]
	(a) Trên 2 đĩa cân ở vị trí thăng bằng có 2 cốc, mỗi cốc đựng 1 dung dịch có hòa tan \emph{0.2 mol \ce{HNO3}}. Thêm vào cốc thứ nhất \emph{20 g \ce{CaCO3}}, thêm vào cốc thứ 2 \emph{20 g \ce{MgCO3}}. Sau khi phản ứng kết thúc, 2 đĩa cân còn giữ vị trí thăng bằng không? Giải thích. (b) Nếu dung dịch trong mỗi cốc có hòa tan \emph{0.5 mol \ce{HNO3}} \& cũng làm thí nghiệm như trên. Phản ứng kết thúc, 2 đĩa cân còn giữ vị trí thăng bằng không? Giải thích.
\end{baitoan}

\begin{baitoan}[\cite{SBT_Hoa_Hoc_9}, 4.6., p. 7]
	Cho 1 lượng bột sắt dư vào \emph{50 mL} dung dịch acid sulfuric. Phản ứng xong, thu được \emph{3.36 L} khí hydrogen (đktc). (a) Viết PTHH. (b) Tính khối lượng sắt đã tham gia phản ứng. (c) Tính nồng độ mol của dung dịch acid sulfuric đã dùng.
\end{baitoan}

\begin{baitoan}[\cite{SBT_Hoa_Hoc_9}, 4.7., p. 7]
	Trung hòa \emph{20 mL} dung dịch \emph{\ce{H2SO4 1M}} bằng dung dịch \emph{NaOH 20\%}. (a) Viết PTHH. (b) Tính khối lượng dung dịch \emph{NaOH} cần dùng. (c) Nếu trung hòa dung dịch acid sulfuric trên bằng dung dịch \emph{KOH 5.6\%}, có khối lượng riêng là \emph{1.045 g\texttt{/}mL}, thì cần bao nhiêu \emph{mL} dung dịch \emph{KOH}?
\end{baitoan}

\begin{baitoan}[\cite{SBT_Hoa_Hoc_9}, 4.8., p. 7]
	Cho dung dịch \emph{HCl 0.5M} tác dụng vừa đủ với \emph{21.6 g} hỗn hợp A gồm \emph{Fe, FeO, \ce{FeCO3}}. Thấy thoát ra 1 hỗn hợp khí có tỷ khối đối với \emph{\ce{H2}} là $15$ \& tạo ra \emph{31.75 g} muối clorua. (a) Tính thể tích dung dịch \emph{HCl} đã dùng. (b) Tính \% khối lượng của mỗi chất trong hỗn hợp A.
\end{baitoan}

\begin{baitoan}[\cite{SBT_Hoa_Hoc_9}, 5.5., p. 8]
	(a) Viết các PTHH của phản ứng điều chế khí hydrogen từ các chất: \emph{Zn}, dung dịch \emph{HCl}, dung dịch \emph{\ce{H2SO4}}. (b) So sánh thể tích khí hydrogen (cùng điều kiện $t^\circ$ \& $p$) thu được của từng cặp phản ứng trong các thí nghiệm: Thí nghiệm 1: \emph{0.1 mol Zn} tác dụng với dung dịch \emph{HCl} dư; \emph{0.1 mol Zn} tác dụng với dung dịch \emph{\ce{H2SO4}} dư.	Thí nghiệm 2: \emph{0.1 mol \ce{H2SO4}} tác dụng với \emph{Zn} dư; \emph{0.1 mol HCl} tác dụng với \emph{Zn} dư.
\end{baitoan}

\begin{baitoan}[\cite{SBT_Hoa_Hoc_9}, 5.6., p. 8]
	Để tác dụng vừa đủ với \emph{44.8 g} hỗn hợp gồm \emph{FeO, \ce{Fe2O3,Fe3O4}} cần phải dùng \emph{400 mL} dung dịch \emph{\ce{H2SO4} 2M}. Sau phản ứng thấy tạo ra $a$ \emph{g} hỗn hợp muối sulfate. Tính $a$.
\end{baitoan}

\begin{baitoan}[\cite{SBT_Hoa_Hoc_9}, 5.7., p. 8]
	Từ $80$ tấn quặng pirit chứa \emph{40\%} lưu huỳnh, người ta sản xuất được $73.5$ tấn acid sulfuric. (a) Tính hiệu suất của quá trình sản xuất acid sulfuric. (b) Tính khối lượng dung dịch \emph{\ce{H2SO4} 50\%} thu được từ $73.5$ tấn \emph{\ce{H2SO4}} đã được sản xuất ở trên.
\end{baitoan}

\begin{baitoan}[\cite{An_350_BT_Hoa_Hoc_9}, 13.a, p. 16]
	Lấy \emph{4.2 g} bột sắt cho tác dụng với \emph{50 mL} dung dịch \emph{\ce{H2SO4} 1M} đến khi kết thúc phản ứng thu được $V$ \emph{L} khí \emph{\ce{H2}} bay ra ở đktc: (a) Cho biết chất nào còn dư sau phản ứng? (b) Tính $V$.
\end{baitoan}

\begin{baitoan}[\cite{An_350_BT_Hoa_Hoc_9}, 13.b, p. 16]
	Cho \emph{29.4 g} dung dịch \emph{\ce{H2SO4} 20\%} vào \emph{100 g} dung dịch \emph{\ce{BaCl2} 5.2\%}. (a) Viết PTHH xảy ra \& tính khối lượng kết tủa tạo thành. (b) Tính nồng độ \% của những chất có trong dung dịch.
\end{baitoan}

\begin{baitoan}[\cite{An_350_BT_Hoa_Hoc_9}, 14.a, p. 17]
	Hòa tan 1 lượng \emph{CuO} cần \emph{100 mL} dung dịch \emph{HCl 1M}. (a) Tính khối lượng \emph{CuO} đã tham gia phản ứng. (b) Tính nồng độ mol của dung dịch sau phản ứng. Biết thể tích dung dịch thay đổi không đáng kể.
\end{baitoan}

\begin{baitoan}[\cite{An_350_BT_Hoa_Hoc_9}, 14.b, p. 17]
	Trộn $c$ \emph{g} bột \emph{Fe} \& $b$ \emph{g} bột \emph{S} rồi nung nóng ở nhiệt độ cao (không có không khí). Hòa tan hỗn hợp sau phản ứng bằng dung dịch \emph{HCl} dư thu được chất rắn X nặng \emph{0.4 g} \& khí Y có tỷ khối so với \emph{\ce{H2}} bằng $9$. Khí Y sục từ từ qua dung dịch \emph{\ce{Pb(NO3)2}} thấy tạo thành \emph{11.95 g} kết tủa. (a) Tính $b,c$. (b) Tính hiệu suất phản ứng nung nóng bột \emph{Fe} \& bột \emph{S}.
\end{baitoan}

\begin{baitoan}[\cite{An_350_BT_Hoa_Hoc_9}, 15., p. 18]
	Hỗn hợp X gồm 2 kim loại \emph{Mg, Fe}. Dung dịch Y là dung dịch \emph{HCl $a$ M}. Thí nghiệm 1: Cho \emph{10.8 g} hỗn hợp X vào \emph{2 L} dung dịch Y có \emph{4.48 L \ce{H2}} (đktc) bay ra. Thí nghiệm 2: Cho \emph{10.8 g} hỗn hợp X vào \emph{3 L} dung dịch Y có \emph{5.6 L \ce{H2}} (đktc) bay ra. Tính $a$ \& tính khối lượng mỗi kim loại trong hỗn hợp X.
\end{baitoan}

\begin{baitoan}[\cite{An_350_BT_Hoa_Hoc_9}, 16., p. 19]
	Hòa tan hoàn toàn \emph{4 g} hỗn hợp gồm \emph{Fe} \& 1 kim loại hóa trị II vào dung dịch \emph{HCl} thì thu được \emph{2.24 L \ce{H2}} (đktc). Nếu chỉ dùng \emph{2.4 g} kim loại hóa trị II cho vào dung dịch \emph{HCl} thì dùng không hết \emph{500 mL} dung dịch \emph{HCl 1M}. Tìm tên kim loại hóa trị II.
\end{baitoan}

\begin{baitoan}[\cite{An_350_BT_Hoa_Hoc_9}, 17., p. 17]
	Trộn \emph{CuO} với 1 oxide kim loại hóa trị II không đổi theo tỷ lệ số mol $1:2$ được hỗn hợp A, cho luồng khí \emph{\ce{H2}} dư qua \emph{2.4 g} hỗn hợp A nung nóng đến phản ứng hoàn toàn được chất rắn B. Để hòa tan hết B cần \emph{100 mL} dung dịch \emph{\ce{HNO3} 1M} chỉ thoát ra khí \emph{NO} duy nhất. Phản ứng xảy ra theo phương trình: \emph{\ce{$3$Cu + $8$HNO3 -> $3$Cu(NO3)2 + $2$NO + $4$H2O, $3$M + $8$HNO3 -> $3$M(NO3)2 + $2$NO + $4$H2O}}. Xác định tên kim loại hóa trị II.
\end{baitoan}

\begin{baitoan}[\cite{An_350_BT_Hoa_Hoc_9}, 18., p. 21]
	1 hỗn hợp X gồm \emph{Al, Mg, Cu} có khối lượng là \emph{5 g} khi hòa tan trong dung dịch \emph{HCl} dư thấy thoát ra \emph{4.48 $\rm dm^3$} khí (đktc) \& thu được dung dịch Y cùng chất rắn Z. Lọc \& nung chất rắn Z trong không khí đến khối lượng không đổi cân nặng \emph{1.375 g}. Tính khối lượng mỗi kim loại.
\end{baitoan}

%------------------------------------------------------------------------------%

\section{Base}

\subsection{Qualitative problem -- Bài tập định tính}

\begin{baitoan}[\cite{SGK_KHTN_8_Canh_Dieu}, p. 51]
	Để tránh nguyên liệu bị nát vụn khi chế biến, trong quá trình làm mứt người ta thường ngâm nguyên liệu vào nước vôi trong. Trong quá trình đó, độ chua của 1 số loại quả sẽ giảm đi. Vì sao?
\end{baitoan}

\begin{baitoan}[\cite{SGK_KHTN_8_Canh_Dieu}, p. 51]
	Trong các chất \emph{\ce{Cu(OH)2,MgSO4,NaCl,Ba(OH)2}}, các chất nào là base?
\end{baitoan}

\begin{baitoan}[\cite{SGK_KHTN_8_Canh_Dieu}, 1, p. 52]
	Dựa vào bảng tính tan, cho biết các base nào sau đây là kiềm: \emph{KOH, \ce{Fe(OH)2,Ba(OH)2,Cu(OH)2}}.
\end{baitoan}

\begin{baitoan}[\cite{SGK_KHTN_8_Canh_Dieu}, 2, p. 52]
	Có 2 dung dịch giấm ăn \& nước vôi trong. Nêu cách phân biệt 2 dung dịch trên bằng: (a) Quỳ tím. (b) Phenolphthalein.
\end{baitoan}

\begin{baitoan}[\cite{SGK_KHTN_8_Canh_Dieu}, 3, p. 54]
	Viết PTHH xảy ra khi cho các base: \emph{KOH, \ce{Cu(OH)2, Mg(OH)2}} lần lượt tác dụng với: (a) Dung dịch acid \emph{HCl}. (b) Dung dịch acid \emph{\ce{H2SO4}}.
\end{baitoan}

\begin{baitoan}[\cite{SGK_KHTN_8_Canh_Dieu}, 4, p. 54]
	Hoàn thành PTHH: (a) \emph{KOH $\to$ \ce{K2SO4}}. (b) \emph{\ce{Mg(OH)2} $\to$ \ce{MgSO4}}. (c) \emph{\ce{Al(OH)3 + H2SO4}}.
\end{baitoan}

\begin{baitoan}[\cite{SGK_KHTN_8_Canh_Dieu}, p. 54]
	1 loại thuốc dành cho bệnh nhân đau dạ dày có chứa \emph{\ce{Al(OH)3,Mg(OH)2}}. Viết PTHH xảy ra giữa acid \emph{HCl} có trong dạ dày với các chất trên.
\end{baitoan}

\begin{baitoan}[\cite{SGK_Hoa_Hoc_9}, 1., p. 25]
	Có phải tất cả các chất kiềm đều là base không? Dẫn ra CTHH của 3 chất kiềm để minh họa. Có phải tất cả các base đều là chất kiềm không? Dẫn ra CTHH của các base để minh họa.
\end{baitoan}

\begin{baitoan}[\cite{SGK_Hoa_Hoc_9}, 2., p. 25]
	Có các base sau: \emph{\ce{Cu(OH)2,NaOH,Ba(OH)2}}. Cho biết những base nào: (a) tác dụng được với dung dịch \emph{HCl}. (b) bị nhiệt phân hủy. (c) tác dụng được với \emph{\ce{CO2}}. (d) đổi màu quỳ tím thành xanh. Viết các PTHH.
\end{baitoan}

\begin{baitoan}[\cite{SGK_Hoa_Hoc_9}, 3., p. 25]
	Từ các chất có sẵn: \emph{\ce{Na2O,CaO,H2O}}. Viết các PTHH điều chế các dung dịch base.
\end{baitoan}

\begin{baitoan}[\cite{SGK_Hoa_Hoc_9}, 4., p. 25]
	Có 4 lọ không nhãn, mỗi lọ đựng 1 dung dịch không màu sau: \emph{NaCl, \ce{Ba(OH)2}, NaOH, \ce{Na2SO4}}. Chỉ được dùng quỳ tím, làm thế nào nhận biết dung dịch đựng trong mỗi lọ bằng phương pháp hóa học? Viết các PTHH.
\end{baitoan}

\begin{baitoan}[\cite{SGK_Hoa_Hoc_9}, 1., p. 27]
	Có 3 lọ không nhãn, mỗi lọ đựng 1 chất rắn sau: \emph{NaOH, NaCl, \ce{Ba(OH)2}}. Trình bày cách nhận biết chất đựng trong mỗi lọ bằng phương pháp hóa học. Viết các PTHH (nếu có).
\end{baitoan}

\begin{baitoan}[\cite{SGK_Hoa_Hoc_9}, 2., p. 27]
	Có các chất: \emph{Zn, \ce{Zn(OH)2,NaOH,Fe(OH)3,CuSO4}, NaCl, HCl}. Chọn chất thích hợp điền vào mỗi sơ đồ phản ứng sau \& lập PTHH: (a) \emph{$\ldots$ \ce{->[$t^\circ$] Fe2O3 + H2O}}. (b) \emph{\ce{H2SO4 + $\ldots$ -> Na2SO4 + H2O}}. (c) \emph{\ce{H2SO4 + $\ldots$ -> ZnSO4 + H2O}}. (d) \emph{\ce{NaOH + $\ldots$ -> NaCl + H2O}}. (e) \emph{$\ldots$ \ce{+ CO2 -> Na2CO3 + H2O}}.
\end{baitoan}

\begin{baitoan}[\cite{SGK_Hoa_Hoc_9}, 1., p. 30]
	Viết các PTHH thực hiện các chuyển đổi hóa học: (a) \emph{\ce{CaCO3} $\to$ CaO $\to$ \ce{Ca(OH)2} $\to$ \ce{CaCO3}}. (b) \emph{CaO $\to$ \ce{CaCl2}}. (c) \emph{\ce{Ca(OH)2} $\to$ \ce{Ca(NO3)2}}.
\end{baitoan}

\begin{baitoan}[\cite{SGK_Hoa_Hoc_9}, 2., p. 30]
	Có 3 lọ không nhãn, mỗi lọ đựng 1 trong 3 chất rắn màu trắng: \emph{\ce{CaCO3,Ca(OH)2}, CaO}. Nhận biết chất đựng trong mỗi lọ bằng phương pháp hóa học. Viết các PTHH.
\end{baitoan}

\begin{baitoan}[\cite{SGK_Hoa_Hoc_9}, 3., p. 30]
	Viết các PTHH của phản ứng khi dung dịch \emph{NaOH} tác dụng với dung dịch \emph{\ce{H2SO4}} tạo ra: (a) muối sodium hydrosunfate. (b) muối sodium sulfate.
\end{baitoan}

\begin{baitoan}[\cite{SGK_Hoa_Hoc_9}, 4., p. 30]
	1 dung dịch bão hòa khí \emph{\ce{CO2}} trong nước có $\rm pH = 4$. Giải thích \& viết PTHH của \emph{\ce{CO2}} với nước.
\end{baitoan}

\begin{baitoan}[\cite{SGK_Hoa_Hoc_9}, 7.1., p. 9]
	Nêu các tính chất hóa học giống \& khác nhau của base tan (kiềm) \& base không tan. Dẫn ra ví dụ, viết PTHH.
\end{baitoan}

\begin{baitoan}[\cite{SGK_Hoa_Hoc_9}, 7.2., p. 9]
	Các base khi bị nung nóng tạo ra oxide là: {\sf A.} \emph{\ce{Mg(OH)2,Cu(OH2),Zn(OH)2,Fe(OH)3}}. {\sf B.} \emph{\ce{Ca(OH)2,Al(OH)3}, KOH, NaOH}. {\sf C.} \emph{\ce{Zn(OH)2,Mg(OH)2,Fe(OH)3}, KOH}. {\sf D.} \emph{\ce{Fe(OH)3,Al(OH)3,Zn(OH)2}, NaOH}.
\end{baitoan}

\begin{baitoan}[\cite{SGK_Hoa_Hoc_9}, 7.3., p. 9]
	Dung dịch \emph{HCl}, khí \emph{\ce{CO2}} đều tác dụng với: {\sf A.} \emph{\ce{Ca(OH)2,Ba(OH)2}, NaOH, KOH}. {\sf B.} \emph{\ce{Ca(OH)2,Al(OH)3}, KOH, NaOH}. {\sf C.} \emph{NaOH, KOH, \ce{Fe(OH)3, Ba(OH)3}}. {\sf D.} \emph{\ce{Ca(OH)2,Cr(OH)3}, KOH}.
\end{baitoan}

\begin{baitoan}[\cite{SGK_Hoa_Hoc_9}, 7.4., p. 9]
	Viết CTHH của các: (a) base ứng với các oxide: \emph{\ce{Na2O,Al2O3,Fe2O3}, BaO}. (b) oxide ứng với các base: \emph{KOH, \ce{Ca(OH)2,Zn(OH)2,Cu(OH)2}}.
\end{baitoan}

\begin{baitoan}[\cite{SGK_Hoa_Hoc_9}, 7.5., p. 9]
	Có 3 lọ không nhãn, mỗi lọ đựng 1 trong các chất rắn: \emph{\ce{Cu(OH)2,Ba(OH)2,Na2CO3}}. Chọn 1 thuốc thử để có thể nhận biết được cả 3 chất này. Viết các PTHH.
\end{baitoan}

\begin{baitoan}[\cite{SGK_Hoa_Hoc_9}, 8.1., p. 9]
	Bằng phương pháp hóa học nào có thể phân biệt được 2 dung dịch base: \emph{NaOH, \ce{Ca(OH)2}}? Viết PTHH.
\end{baitoan}

\begin{baitoan}[\cite{SGK_Hoa_Hoc_9}, 8.2., p. 9]
	Có 4 lọ không nhãn, mỗi lọ đựng 1 trong các dung dịch sau: \emph{NaOH, \ce{Na2SO4,H2SO4}, HCl}. Nhận biết dung dịch trong mỗi lọ bằng phương pháp hóa học. Viết các PTHH.
\end{baitoan}

\begin{baitoan}[\cite{SGK_Hoa_Hoc_9}, 8.3., p. 10]
	Cho các chất: \emph{\ce{Na2CO3,Ca(OH)2}, NaCl}. (a) Từ các chất đã cho, viết các PTHH điều chế \emph{NaOH}. (b) Nếu các chất đã cho có khối lượng bằng nhau, ta dùng phản ứng nào để có thể điều chế được khối lượng \emph{NaOH} nhiều hơn?
\end{baitoan}

\begin{baitoan}[\cite{SGK_Hoa_Hoc_9}, 8.4., p. 10]
	Bảng sau cho biết giá trị pH của dung dịch 1 số chất:
	\begin{table}[H]
		\centering
		\begin{tabular}{|c|c|c|c|c|c|}
			\hline
			Dung dịch & A & B & C & D & E \\
			\hline
			pH & 13 & 3 & 1 & 7 & 8 \\
			\hline
		\end{tabular}
	\end{table}
	\noindent(a) Dự đoán trong các dung dịch trên: (1) Dung dịch nào có thể là acid, e.g., \emph{HCl, \ce{H2SO4}}? (2) Dung dịch nào có thể là base, e.g., \emph{NaOH, \ce{Ca(OH)2}}? (3) Dung dịch nào có thể là đường, muối \emph{NaCl}, nước cất? (4) Dung dịch nào có thể là acid acetic (có trong giấm ăn)? (5) Dung dịch nào có tính base yếu, e.g., \emph{\ce{NaHCO3}}? (b) Cho biết: (1) Dung dịch nào có phản ứng với \emph{Mg}, với \emph{NaOH}? (2) Dung dịch nào có phản ứng với dung dịch \emph{HCl}? (3) Các dung dịch nào trộn với nhau từng đôi một sẽ xảy ra phản ứng hóa học?	
\end{baitoan}

\subsection{Quantitative problem -- Bài tập định lượng}

\begin{baitoan}[\cite{SGK_Hoa_Hoc_9}, 4., p. 25]
	Cho \emph{15.5 g} sodium oxide \emph{\ce{Na2O}} tác dụng với nước, thu được \emph{0.5 L} dung dịch base. (a) Viết PTHH \& tính nồng độ mol của dung dịch base thu được. (b) Tính thể tích dung dịch \emph{\ce{H2SO4} 20\%}, có khối lượng riêng \emph{1.14 g\texttt{/}mL} cần dùng để trung hòa dung dịch base nói trên.
\end{baitoan}

\begin{baitoan}[\cite{SGK_Hoa_Hoc_9}, 3., p. 27]
	Dẫn từ từ \emph{1.568 L} khí \emph{\ce{CO2}} (đktc) vào 1 dung dịch có hòa tan \emph{6.4 g NaOH}, sản phẩm là muối \emph{\ce{Na2CO3}}. (a) Chất nào đã lấy dư \& dư là bao nhiêu (\emph{L} hoặc \emph{g})? (b) Tính khối lượng muối thu được sau phản ứng.
\end{baitoan}

\begin{baitoan}[\cite{SGK_Hoa_Hoc_9}, 8.5., p. 10]
	\emph{3.04 g} hỗn hợp \emph{NaOH, KOH} tác dụng vừa đủ với dung dịch \emph{HCl}, thu được \emph{4.15 g} các muối clorua. (a) Viết các PTHH. (b) Tính khối lượng của mỗi hydroxide trong hỗn hợp ban đầu.
\end{baitoan}

\begin{baitoan}[\cite{SGK_Hoa_Hoc_9}, 8.6., p. 10]
	Cho \emph{10 g \ce{CaCO3}} tác dụng với dung dịch \emph{HCl} dư. (a) Tính thể tích khí \emph{\ce{CO2}} thu được ở đktc. (b) Dẫn khí \emph{\ce{CO2}} thu được ở trên vào lọ đựng \emph{50 g} dung dịch \emph{NaOH 40\%}. Tính khối lượng muối carbonate thu được.
\end{baitoan}

\begin{baitoan}[\cite{SGK_Hoa_Hoc_9}, 8.7., p. 10]
	Cho $m$ \emph{g} hỗn hợp gồm \emph{\ce{Mg(OH)2,Cu(OH)2}, NaOH} tác dụng vừa đủ với \emph{400 mL} dung dịch \emph{HCl 1M} \& tạo thành \emph{24.1 g} muối clorua. Tính $m$.
\end{baitoan}

\begin{baitoan}[\cite{An_350_BT_Hoa_Hoc_9}, 19., p. 21]
	Cho \emph{150 mL} dung dịch \emph{NaOH 0.5M} vào \emph{150 mL} dung dịch \emph{HCl 1M}. (a) Viết PTHH. (b) Nếu cho giấy quỳ tím vào dung dịch sau phản ứng, thì màu của giấy quỳ thay đổi như thế nào? Vì sao? (c) Tính khối lượng muối tạo thành sau phản ứng.
\end{baitoan}

\begin{baitoan}[\cite{An_350_BT_Hoa_Hoc_9}, 20., p. 22]
	Cho $m$ \emph{g NaOH} nguyên chất tác dụng với dung dịch \emph{\ce{Cu(NO3)2}} có dư, thu được \emph{29.4 g} kết tủa \emph{\ce{Cu(OH)2}}. (a) Viết PTHH. (b) Tính $m$.
\end{baitoan}

\begin{baitoan}[\cite{An_350_BT_Hoa_Hoc_9}, 21.a, p. 22]
	Nếu có \emph{20 g} dung dịch sodium hydroxide \emph{20\%} phải dùng hết bao nhiêu \emph{g} dung dịch hydrochloric acid \emph{25\%} để trung hòa.
\end{baitoan}

\begin{baitoan}[\cite{An_350_BT_Hoa_Hoc_9}, 21.b, p. 22]
	Hòa tan \emph{12.4 g \ce{Na2O}} vào \emph{1 L} nước ta được dung dịch X. Lấy \emph{0.5 L} dung dịch X cho tác dụng với $V$ \emph{mL} dung dịch \emph{\ce{Fe2(SO4)3} 0.5M} (vừa đủ) tạo thành 1 kết tủa \& dung dịch Y. Tính $V$.
\end{baitoan}

\begin{baitoan}[\cite{An_350_BT_Hoa_Hoc_9}, 22., p. 23]
	Dung dịch X chứa \emph{2.7 g \ce{CuCl2}} cho tác dụng với dung dịch Y chứa \emph{NaOH} (lấy dư). Sau khi phản ứng kết thúc thu được kết tủa Z lọc lấy kết tủa Z đem nung đến khối lượng không đổi, thu được chất rắn T. (a) Viết PTHH. (b) Tính khối lượng kết tủa Z \& chất rắn T.
\end{baitoan}

\begin{baitoan}[\cite{An_350_BT_Hoa_Hoc_9}, 23., p. 23]
	Cho \emph{200 mL} dung dịch \emph{HCl 0.2M}. (a) Tính thể tích dung dịch \emph{NaOH 0.2M} cần để trung hòa dung dịch acid trên. Tính nồng độ mol của dung dịch muối tạo thành. (b) Nếu cho dung dịch acid trên tác dụng với \emph{\ce{CaCO3}}. Tính khối lượng \emph{\ce{CaCO3}} để phản ứng xảy ra vừa đủ \& thể tích khí bay lên.
\end{baitoan}

\begin{baitoan}[\cite{An_350_BT_Hoa_Hoc_9}, 24.b, p. 24]
	Để trung hòa \emph{25 mL} dung dịch X cần dùng \emph{30 mL} dung dịch \emph{HCl 1M}. Khi cho \emph{25 mL} dung dịch X tác dụng với 1 lượng dư \emph{\ce{Na2CO3}} thấy tạo thành \emph{1.97 g} kết tủa. Tính nồng độ mol của \emph{NaOH, \ce{Ba(OH)2}} trong dung dịch X.
\end{baitoan}

\begin{baitoan}[\cite{An_350_BT_Hoa_Hoc_9}, 25., p. 25]
	Cho \emph{0.594 g} hỗn hợp \emph{Na, Ba} hòa tan hoàn toàn vào nước thu được dung dịch A \& khí B. Trung hòa dung dịch A cần \emph{100 mL HCl}. Cô cạn dung dịch sau phản ứng thu được \emph{0.949 g} muối. (a) Tính thể tích khí B (đktc), nồng độ mol của dung dịch \emph{HCl}. (b) Tính khối lượng mỗi kim loại.
\end{baitoan}

%------------------------------------------------------------------------------%

\section{pH}

\begin{baitoan}[\cite{SGK_KHTN_8_Canh_Dieu}, p. 55]
	Dung dịch X làm quỳ tím chuyển sang màu đỏ. Kết luận nào sau đây là đúng? Giải thích. (a) Dung dịch X có pH $< 7$. (b) Dung dịch X có pH $> 7$.
\end{baitoan}

\begin{baitoan}[\cite{SGK_KHTN_8_Canh_Dieu}, 1, p. 57]
	Trong sản xuất nông nghiệp, người ta thường bón vôi cho các ruộng bị chua. Sau khi bón vôi vào ruộng, pH của môi trường sẽ tăng lên hay giảm đi? Giải thích.
\end{baitoan}

\begin{baitoan}[\cite{SGK_KHTN_8_Canh_Dieu}, 2, p. 58]
	Xác định pH của 1 số loại nước ép trái cây: chanh, cam, táo, dưa hấu.
\end{baitoan}

\begin{baitoan}[\cite{SGK_KHTN_8_Canh_Dieu}, 3, p. 58]
	Xác định pH của 1 số đồ uống: bia, nước uống có gas, sữa tươi.
\end{baitoan}

\begin{baitoan}[\cite{SGK_KHTN_8_Canh_Dieu}, 3, p. 58, Tìm hiểu sự đổi màu của nước bắp cải tím khi tác dụng với các dung dịch acid \& base.]
	Xay bắp cải tím với nước, lọc bã qua rây để giữ lại nước lọc. Cho nước lọc thu được vào 4 cốc thủy tinh không màu có đánh số từ 1--4, sau đó thêm vào các cốc: Cốc 1: nước vắt từ quả chanh. Cốc 2: dung dịch nước rửa chén. Cốc 3: nước xà phòng. Cốc 4: giấm ăn. Quan sát hiện tượng xảy ra \& nhận xét.
\end{baitoan}

%------------------------------------------------------------------------------%

\section{Salt -- Muối}

\subsection{Qualitative problem -- Bài tập định tính}

\begin{baitoan}[\cite{SGK_KHTN_8_Canh_Dieu}, 3, p. 63]
	Cho biết các muối: \emph{\ce{Na3PO4,MgCl2,CaCO3,CuSO4,KNO3}} tương ứng với acid nào trong số các acid sau: \emph{HCl, \ce{H2SO4,H3PO4,HNO3,H2CO3}}.
\end{baitoan}

\begin{baitoan}[\cite{SGK_KHTN_8_Canh_Dieu}, 1, p. 63]
	Gọi tên các muối: \emph{KCl, \ce{ZnSO4,MgCO3,Ca3(PO4)2,Cu(NO3)2,Al2(SO4)3}}.
\end{baitoan}

\begin{baitoan}[\cite{SGK_KHTN_8_Canh_Dieu}, 2, p. 63]
	Sử dụng bảng tính tan, cho biết muối nào sau đây tan được trong nước: \emph{\ce{K2SO4,Na2CO3,AgNO3}, KCl, \ce{CaCl2,BaCO3,MgSO4}}.
\end{baitoan}

\begin{baitoan}[\cite{SGK_KHTN_8_Canh_Dieu}, 3, p. 64]
	Dung dịch \emph{\ce{CuSO4}} có màu xanh lam, dung dịch \emph{\ce{ZnSO4}} không màu. Viết PTHH xảy ra khi ngâm \emph{Zn} trong dung dịch \emph{\ce{CuSO4}}, dự đoán sự thay đổi về màu của dung dịch trong quá trình trên.
\end{baitoan}

\begin{baitoan}[\cite{SGK_KHTN_8_Canh_Dieu}, 4, p. 64]
	Viết PTHH của phản ứng xảy ra trong các trường hơp sau: (a) Cho \emph{Fe} vào dung dịch \emph{\ce{CuSO4}}. (b) Cho \emph{Zn} vào dung dịch \emph{\ce{AgNO3}}.
\end{baitoan}

\begin{proof}[Giải]
	Kết quả thí nghiệm trên cho thấy có phản ứng hóa học giữa dung dịch \ce{AgNO3} \& Cu. PTHH: \ce{2AgNO3 + Cu -> Cu(NO3)2 + 2Ag v} (silver nitrate $\to$ copper(II) nitrate).
\end{proof}

\begin{baitoan}[\cite{SGK_KHTN_8_Canh_Dieu}, 5, p. 65]
	Dự đoán các hiện tượng xảy ra trong các thí nghiệm sau: (a) Nhỏ dung dịch \emph{\ce{H2SO4}} loãng vào dung dịch \emph{\ce{Na2CO3}}. (b) Nhỏ dung dịch \emph{HCl} loãng vào dung dịch \emph{\ce{AgNO3}}. Giải thích \& viết PTHH xảy ra (nếu có).
\end{baitoan}

\begin{baitoan}[\cite{SGK_KHTN_8_Canh_Dieu}, 6, p. 65]
	Viết PTHH xảy ra trong các trường hợp sau: (a) Dung dịch \emph{\ce{FeCl3}} tác dụng với dung dịch \emph{NaOH}. (b) Dung dịch \emph{\ce{CuCl2}} tác dụng với dung dịch \emph{KOH}.
\end{baitoan}

\begin{baitoan}[\cite{SGK_KHTN_8_Canh_Dieu}, 6, p. 65]
	Hoàn thành các PTHH theo các sơ đồ: (a) \emph{MgO $\to$ \ce{MgSO4}}. (b) \emph{KOH $\to$ \ce{Cu(OH)2 v}}.
\end{baitoan}

\begin{baitoan}[\cite{SGK_KHTN_8_Canh_Dieu}, 8, p. 66]
	Viết PTHH xảy ra giữa các dung dịch sau: (a) Dung dịch \emph{NaCl} với dung dịch \emph{\ce{AgNO3}}. (b) Dung dịch \emph{\ce{Na2SO4}} với dung dịch \emph{\ce{BaCl2}}. (c) Dung dịch \emph{\ce{K2CO3}} với dung dịch \emph{\ce{Ca(NO3)2}}.
\end{baitoan}

\begin{baitoan}[\cite{SGK_KHTN_8_Canh_Dieu}, 9, p. 66]
	Viết các PTHH theo sơ đồ chuyển hóa sau: \emph{CuO $\to$ \ce{CuSO4} $\to$ \ce{CuCl2} $\to$ \ce{Cu(OH)2}}.
\end{baitoan}

\begin{baitoan}[\cite{SGK_KHTN_8_Canh_Dieu}, p. 67]
	Muối \emph{\ce{Al2(SO4)3}} được dùng trong công nghiệp để nhuộm vải, thuộc da, làm trong nước, $\ldots$ Tính khối lượng \emph{\ce{Al2(SO4)3}} tạo thành khi cho \emph{51 kg \ce{Al2O3}} tác dụng hết với dung dịch \emph{\ce{H2SO4}}.
\end{baitoan}

\begin{baitoan}[\cite{SGK_KHTN_8_Canh_Dieu}, 10, p. 67]
	Viết $3$ PTHH khác nhau để tạo ra \emph{\ce{Na2SO4}} từ \emph{NaOH}.
\end{baitoan}

\begin{baitoan}[\cite{SGK_KHTN_8_Canh_Dieu}, 11, p. 67]
	Viết $3$ PTHH khác nhau để điều chế \emph{\ce{CuCl2}}.
\end{baitoan}

\begin{baitoan}[\cite{SGK_Hoa_Hoc_9}, 1., p. 33]
	Dẫn ra 1 dung dịch muối khi tác dụng với 1 dung dịch chất khác thì tạo ra: (a) chất khí. (b) chất kết tủa. Viết các PTHH.
\end{baitoan}

\begin{baitoan}[\cite{SGK_Hoa_Hoc_9}, 2., p. 33]
	Có 3 lọ không nhãn, mỗi lọ đựng 1 dung dịch muối sau: \emph{\ce{CuSO4,AgNO3}, NaCl}. Dùng những dung dịch có sẵn trong phòng thí nghiệm để nhận biết chất đựng trong mỗi lọ. Viết các PTHH.
\end{baitoan}

\begin{baitoan}[\cite{SGK_Hoa_Hoc_9}, 3., p. 33]
	Có các dung dịch muối: \emph{\ce{Mg(NO3)2,CuCl2}}. Cho biết muối nào có thể tác dụng với: (a) Dung dịch \emph{NaOH}. (b) Dung dịch \emph{HCl}. (c) Dung dịch \emph{\ce{AgNO3}}. Nếu có phản ứng, viết các PTHH.
\end{baitoan}

\begin{baitoan}[\cite{SGK_Hoa_Hoc_9}, 4., p. 33]
	Cho các dung dịch muối sau phản ứng với nhau từng đôi một, viết dấu $\cdot$ nếu có phản ứng \& viết PTHH, dấu $\circ$ nếu không.
\end{baitoan}

\begin{baitoan}[\cite{SGK_Hoa_Hoc_9}, 5., p. 33]
	Ngâm 1 đinh sắt sạch trong dung dịch copper(II) sulfate. Câu trả lời nào sau đây là đúng nhất cho hiện tượng quan sát được? {\sf A.} không có hiện tượng nào xảy ra. {\sf B.} Kim loại đồng màu đỏ bám ngoài đinh sắt, đinh sắt không có sự thay đổi. {\sf C.} 1 phần đinh sắt bị hòa tan, kim loại đồng bám ngoài đinh sắt \& màu xanh lam của dung dịch ban đầu nhạt dần. {\sf D.} Không có chất mới nào được sinh ra, chỉ có 1 phần đinh sắt bị hòa tan. Giải thích cho sự lựa chọn \& viết PTHH, nếu có.
\end{baitoan}

\begin{baitoan}[\cite{SGK_Hoa_Hoc_9}, 1., p. 36]
	Cho các muối: \emph{\ce{CaCO3,CaSO4,Pb(NO3)2}, NaCl}. Muối nào nói trên: (a) không được phép có trong nước ăn vì tính độc hại của nó? (b) không độc nhưng cũng không nên có trong nước ăn vì vị mặn của nó? (c) không tan trong nước, nhưng bị phân hủy ở nhiệt độ cao? (d) rất ít tan trong nước \& khó bị phân hủy ở nhiệt độ cao?
\end{baitoan}

\begin{baitoan}[\cite{SGK_Hoa_Hoc_9}, 2., p. 36]
	2 dung dịch tác dụng với nhau, sản phẩm thu được có \emph{NaCl}. Cho biết 2 dung dịch chất ban đầu có thể là các chất nào. Minh họa bằng các PTHH.
\end{baitoan}

\begin{baitoan}[\cite{SGK_Hoa_Hoc_9}, 3., p. 36]
	(a) Viết phương trình điện phân dung dịch muối ăn (có màng ngăn). (b) Các sản phẩm của sự điện phân dung dịch \emph{NaCl} có nhiều ứng dụng quan trọng: Khí clo dùng để: $\ldots$ Khí hydrogen dùng để: $\ldots$. Sodium hydroxide dùng để: $\ldots$ Điền các ứng dựng sau vào các chỗ trống cho phù hợp: tẩy trắng vải, giấy; nấu xà phòng; sản xuất hydrochloric acid; chế tạo hóa chất trừ sâu, diệt cỏ dại; hàn cắt kim loại; sát trùng, diệt khuẩn nước ăn; nhiên liệu cho động cơ tên lửa; bơm khí cầu, bóng thám không; sản xuất nhôm, sản xuất chất dẻo PVC; chế biến dầu mỏ.
\end{baitoan}

\begin{baitoan}[\cite{SGK_Hoa_Hoc_9}, 4., p. 36]
	Dung dịch \emph{NaOH} có thể dùng để phân biệt 2 muối có trong mỗi cặp chất sau được không? (a) Dung dịch \emph{\ce{K2SO4}} \& dung dịch \emph{\ce{Fe2(SO4)3}}. (b) Dung dịch \emph{\ce{Na2SO4}} \& dung dịch \emph{\ce{CuSO4}}. (c) Dung dịch \emph{NaCl} \& dung dịch \emph{\ce{BaCl2}}. Viết các PTHH, nếu có.
\end{baitoan}

\begin{baitoan}[\cite{SBT_Hoa_Hoc_9}, 9.1., p. 11]
	Thuốc thử dùng để phân biệt 2 dung dịch sodium sulfate \& sodium sunfite là: {\sf A.} dung dịch barium chloride. {\sf B.} dung dịch hydrochloric acid. {\sf C.} dung dịch chì nitrate. {\sf D.} dung dịch sodium hydroxide.
\end{baitoan}

\begin{baitoan}[\cite{SBT_Hoa_Hoc_9}, 9.2., p. 11]
	
\end{baitoan}

\begin{baitoan}[\cite{SBT_Hoa_Hoc_9}, 9.3., p. 11]
	
\end{baitoan}

\begin{baitoan}[\cite{SBT_Hoa_Hoc_9}, 9.4., p. 11]
	
\end{baitoan}

\begin{baitoan}[\cite{SBT_Hoa_Hoc_9}, 9.5., p. 11]
	
\end{baitoan}

\begin{baitoan}[\cite{SBT_Hoa_Hoc_9}, 9.6., p. 11]
	
\end{baitoan}

\begin{baitoan}[\cite{An_350_BT_Hoa_Hoc_9}, 44., p. 37]
	Viết PTHH để thực hiện chuỗi chuyển hóa sau: (a) \emph{\ce{FeS2} $\to$ \ce{SO2} $\to$ \ce{SO3} $\to$ \ce{H2SO4} $\to$ \ce{CuSO4}}. (b) \emph{\ce{AlCl3} $\to$ \ce{Al(OH)3} $\to$ \ce{Al2O3} $\to$ \ce{Al2(SO4)3} $\to$ \ce{AlCl3}}. (c) \emph{Na $\to$ \ce{Na2O} $\to$ NaOH $\to$ \ce{Na2CO3} $\to$ \ce{NaHCO3}}. (d) Cho các chất: \emph{\ce{SO2,Fe2O3,Ba(OH)2,HCl,KHCO3}}. Chất nào tác dụng được với dung dịch \emph{\ce{H2SO4}}? Chất nào tác dụng được với dung dịch \emph{KOH}? Viết PTHH.
\end{baitoan}

\subsection{Quantitative problem -- Bài tập định lượng}

\begin{baitoan}[\cite{SGK_Hoa_Hoc_9}, 6., p. 33]
	Trộn \emph{30 mL} dung dịch có chứa \emph{2.22 g \ce{CaCl2}} với \emph{70 mL} dung dịch có chứa \emph{1.7 g \ce{AgNO3}}. (a) Cho biết hiện tượng quan sát được \& viết PTHH. (b) Tính khối lượng chất rắn sinh ra. (c) Tính nồng độ mol của chất còn lại trong dung dịch sau phản ứng. Cho thể tích của dung dịch thay đổi không đáng kể.
\end{baitoan}

\begin{baitoan}[\cite{SGK_Hoa_Hoc_9}, 5., p. 36]
	Trong phòng thí nghiệm có thể dùng các muối \emph{\ce{KClO3}} hoặc \emph{\ce{KNO3}} để điều chế khí oxygen bằng phản ứng phân hủy. (a) Viết các PTHH. (b) Nếu dùng \emph{0.1 mol} mỗi chất thì thể tích khí oxygen thu được có khác nhau không? Tính thể tích khí oxygen thu được. (c) Cần điều chế \emph{1.12 L} khí oxygen, tính khối lượng mỗi chất cần dùng. Các thể tích khí được đo ở đktc.
\end{baitoan}

\begin{baitoan}[\cite{SBT_Hoa_Hoc_9}, 9.7., p. 11]
	
\end{baitoan}

\begin{baitoan}[\cite{SBT_Hoa_Hoc_9}, 9.8., p. 11]
	
\end{baitoan}

\subsubsection{Tính khối lượng muối \& thể tích khí \ce{CO2}}

\begin{baitoan}[\cite{An_350_BT_Hoa_Hoc_9}, 26., p. 27]
	Cho \emph{8.25 g} hỗn hợp bột kim loại \emph{Mg, Fe} tác dụng hết với dung dịch \emph{HCl} thấy thoát ra \emph{5.6 L \ce{H2}} (đktc). Tính khối lượng muối tạo thành.
\end{baitoan}

\begin{baitoan}[\cite{An_350_BT_Hoa_Hoc_9}, 27., p. 27]
	Cho \emph{1.84 g} carbonate của 2 kim loại hóa trị II, tác dụng hết với dung dịch \emph{HCl} thu được \emph{0.672 L \ce{CO2}} \& dung dịch X. Tính khối lượng muối trong dung dịch X.
\end{baitoan}

\begin{baitoan}[\cite{An_350_BT_Hoa_Hoc_9}, 28., p. 28]
	Cho \emph{19.7 g} muối carbonate của kim loại hóa trị II bằng dung dịch \emph{\ce{H2SO4}} loãng dư thu được \emph{23.3 g} muối sulfate. Tính thể tích \emph{\ce{CO2}} \& xác định CTPT của muối.
\end{baitoan}

\begin{baitoan}[\cite{An_350_BT_Hoa_Hoc_9}, 29., p. 28]
	Hòa tan \emph{21.5 g} hỗn hợp \emph{\ce{BaCl2,CaCl2}} vào \emph{250 mL \ce{H2O}} để được dung dịch X. Thêm vào dung dịch X \emph{200 mL} dung dịch \emph{\ce{Na2CO3} 1M} thấy tách ra \emph{19.85 g} kết tủa \& còn nhận được \emph{400 mL} dung dịch Y. Tính nồng độ mol các chất trong dung dịch Y.
\end{baitoan}

\begin{baitoan}[\cite{An_350_BT_Hoa_Hoc_9}, 30., p. 29]
	Trong \emph{1 L} dung dịch hỗn hợp X gồm \emph{0.2 mol \ce{Na2CO3}} \& \emph{0.5 mol \ce{(NH4)2CO3}}. Cho \emph{86 g} hỗn hợp \emph{\ce{BaCl2,CaCl2}} vào dung dịch X. Sau khi phản ứng kết thúc, ta thu được \emph{79.4 g} kết tủa Y. Tính khối lượng các chất trong kết tủa Y.
\end{baitoan}

\begin{baitoan}[\cite{An_350_BT_Hoa_Hoc_9}, 31., p. 30]
	Cho \emph{5.8 g} muối carbonate \emph{\ce{MCO3}} của kim loại M tan hoàn toàn trong dung dịch \emph{\ce{H2SO4}} loãng vừa đủ, thu được 1 chất khí \& dung dịch X. Cô cạn dung dịch X thu được \emph{7.6 g} muối sulfate trung hòa, khan. Xác định CTHH của muối carbonate.
\end{baitoan}

\begin{baitoan}[\cite{An_350_BT_Hoa_Hoc_9}, 32., p. 30]
	Hòa tan hoàn toàn \emph{14.2 g} hỗn hợp A gồm \emph{\ce{MgCO3}} \& muối carbonate của kim loại R vào acid \emph{HCl 7.3\%} vừa đủ, thu được dung dịch B \& \emph{3.36 L} khí \emph{\ce{CO2}} (đktc). Nồng độ \emph{\ce{MgCl2}} trong dung dịch B bằng \emph{6.028\%}. Xác định kim loại R.
\end{baitoan}

\begin{baitoan}[\cite{An_350_BT_Hoa_Hoc_9}, 33.a, p. 31]
	Có hỗn hợp gồm 2 muối \emph{NaCl, NaBr}. Khi cho dung dịch \emph{\ce{AgNO3}} vừa đủ vào hỗn hợp trên người ta thu được lượng kết tủa bằng khối lượng \emph{\ce{AgNO3}} tham gia phản ứng. Tính \% khối lượng mỗi chất trong hỗn hợp.
\end{baitoan}

\begin{baitoan}[\cite{An_350_BT_Hoa_Hoc_9}, 33.b, p. 31]
	Cho 2 cốc đựng dung dịch \emph{HCl} đặt trên 2 đĩa cân A \& B: cân ở trạng thái thăng bằng. Cho $a$ \emph{g \ce{CaCO3}} vào cốc A \& $b$ \emph{g \ce{M2CO3}} (M: kim loại kiềm) vào cốc B. Sau khi 2 muối đã tan hoàn toàn, cân trở lại vị trí thăng bằng. Thiết lập biểu thức tính nguyên tử khối của M theo $a,b$. Áp dụng cho $a = 5$ \emph{g}, $b = 4.8$ \emph{g}. Xác định kim loại M.
\end{baitoan}

\begin{baitoan}[\cite{An_350_BT_Hoa_Hoc_9}, 34., p. 32]
	Cho từ từ dung dịch chứa $a$ \emph{mol HCl} vào dung dịch chứa $b$ \emph{mol \ce{Na2CO3}} đồng thời khuấy đều, thu được $V$ \emph{L} khí (ở đktc) \& dung dịch X. Khi co dư nước vôi trong vào dung dịch X thấy có xuất hiện kết tủa. Tính biểu thức liên hệ giữa $V$ với $a,b$.
\end{baitoan}

\begin{baitoan}[\cite{An_350_BT_Hoa_Hoc_9}, 35., p. 32]
	Cho \emph{1.9 g} hỗn hợp muối carbonate \& hydrocarbonate (i.e., bicarbonate) của kim loại kiềm M tác dụng hết với dung dịch \emph{HCl} (dư), sinh ra \emph{0.448 L} khí (đktc). Xác định kim loại M.
\end{baitoan}

\begin{baitoan}[\cite{An_350_BT_Hoa_Hoc_9}, 36., p. 33]
	Khi hòa tan hydroxide kim loại \emph{\ce{M(OH)2}} bằng 1 lượng vừa đủ dung dịch \emph{\ce{H2SO4} 20\%} thu được dung dịch muối trung hòa có nồng độ \emph{27.21\%}. Xác định kim loại M.
\end{baitoan}

\subsubsection{Kim loại mạnh đẩy kim loại yếu ra khỏi dung dịch muối}

\begin{baitoan}[\cite{An_350_BT_Hoa_Hoc_9}, 37., p. 33]
	Nhúng 1 lá nhôm vào dung dịch \emph{\ce{CuSO4}}. Sau phản ứng lấy lá nhôm ra thì thấy khối lượng dung dịch nhẹ đi \emph{1.38 g}. Tính khối lượng \emph{Al} đã phản ứng.
\end{baitoan}

\begin{baitoan}[\cite{An_350_BT_Hoa_Hoc_9}, 38., p. 34]
	Nhúng 1 thanh graphite phủ kim loại A hóa trị II vào dung dịch \emph{\ce{CuSO4}} dư. Sau phản ứng thanh graphite giảm \emph{0.04 g}. Tiếp tục nhúng thanh graphite này vào dung dịch \emph{\ce{AgNO3}} dư, khi phản ứng kết thúc khối lượng thanh graphite tăng \emph{6.08 g} (so với khối lượng thanh graphite sau khi nhúng vào \emph{\ce{CuSO4}}). Tìm tên kim loại A \& khối lượng kim loại A đã phủ lên thanh graphite lúc đầu. Coi như toàn bộ kim loại tạo thành đều bám vào thanh graphite.
\end{baitoan}

\begin{baitoan}[\cite{An_350_BT_Hoa_Hoc_9}, 39., p. 35]
	Nhúng thanh kim loại \emph{Zn} vào 1 dung dịch chứa hỗn hợp \emph{3.2 g \ce{CuSO4}} \& \emph{6.24 g \ce{CdSO4}}. Hỏi sau khi \emph{Cu, Cd} bị đẩy hoàn toàn khỏi dung dịch thì khối lượng thanh \emph{Zn} tăng hay giảm bao nhiêu?	
\end{baitoan}

\begin{baitoan}[\cite{An_350_BT_Hoa_Hoc_9}, 40., p. 35]
	Cho 1 lá đồng có khối lượng \emph{5 g} vào \emph{125 g} dung dịch \emph{\ce{AgNO3} 4\%}. Sau 1 thời gian, khi lấy lá đồng ra thì khối lượng \emph{\ce{AgNO3}} trong dung dịch giảm \emph{17\%}. Xác định khối lượng kim loại \emph{Cu} sau phản ứng.
\end{baitoan}

\begin{baitoan}[\cite{An_350_BT_Hoa_Hoc_9}, 41., p. 36]
	Cho $m$ \emph{g} hỗn hợp \emph{Zn, Fe} vào lượng dư dung dịch \emph{\ce{CuSO4}}. Sau khi kết thúc các phản ứng, lọc bỏ phần dung dịch thu được $m$ \emph{g} chất rắn. Tính thành phần \% theo khối lượng của \emph{Zn} trong hỗn hợp ban đầu.
\end{baitoan}

\begin{baitoan}[\cite{An_350_BT_Hoa_Hoc_9}, 42., p. 36]
	Cho 1 lượng bột \emph{Zn} vào dung dịch X gồm \emph{\ce{FeCl2,CuCl2}}. Khối lượng chất rắn sau khi các phản ứng xảy ra hoàn toàn nhỏ hơn khối lượng bột \emph{Zn} ban đầu là \emph{0.5 g}. Cô cạn phần dung dịch sau phản ứng thu được \emph{13.6 g} muối khan. Tính tổng khối lượng các muối trong X.
\end{baitoan}

\begin{baitoan}[\cite{An_350_BT_Hoa_Hoc_9}, 43., p. 36]
	Hòa tan hoàn toàn \emph{13.8 g} muối carbonate 1 kim loại kiềm \emph{\ce{R2CO3}} trong \emph{110 mL} dung dịch \emph{HCl 2M}. Sau khi phản ứng xảy ra hoàn toàn, ta thấy còn dư acid trong dung dịch thu được \& thể tích khí thoát ra $V_1$ vượt quá \emph{2016 mL} (đktc). Xác định CTHH muối carbonate.
\end{baitoan}

\subsubsection{Dạng bài toán chứng minh acid còn dư hay hỗn hợp các chất còn dư}

\begin{baitoan}[\cite{An_350_BT_Hoa_Hoc_9}, 37., p. 33]
	
\end{baitoan}

\begin{baitoan}[\cite{An_350_BT_Hoa_Hoc_9}, 37., p. 33]
	
\end{baitoan}

\begin{baitoan}[\cite{An_350_BT_Hoa_Hoc_9}, 37., p. 33]
	
\end{baitoan}

%------------------------------------------------------------------------------%

\section{Phân Bón Hóa Học}

\subsection{Qualitative problem -- Bài tập định tính}

\begin{baitoan}[\cite{SGK_KHTN_8_Canh_Dieu}, 1, p. 68]
	Phân bón hóa học là gì? Theo nhu cầu của cây trồng, phân bón được chia thành các loại nào?
\end{baitoan}

\begin{baitoan}[\cite{SGK_KHTN_8_Canh_Dieu}, 2, p. 69]
	Các loại phân đạm đều chứa nguyên tố hóa học nào? Nêu tác dụng chính của phân đạm đối với cây trồng.
\end{baitoan}

\begin{baitoan}[\cite{SGK_KHTN_8_Canh_Dieu}, 3, p. 69]
	Phân lân cung cấp nguyên tố dinh dưỡng nào cho cây trồng? Nêu tác dụng chính của phân lân đối với cây trồng.
\end{baitoan}

\begin{baitoan}[\cite{SGK_KHTN_8_Canh_Dieu}, 4, p. 70]
	Phân lân hóa học có ảnh hưởng thế nào đến môi trường?
\end{baitoan}

\begin{baitoan}[\cite{SGK_KHTN_8_Canh_Dieu}, 5, p. 71]
	Khi sử dụng phân bón hóa học cần tuân thủ những nguyên tắc nào?
\end{baitoan}

\begin{baitoan}[\cite{SGK_KHTN_8_Canh_Dieu}, p. 71]
	Lúa là cây lương thực chủ yếu ở nước ta, tìm hiểu \& cho biết: Quá trình sinh trưởng của cây lúa được chia thành mấy giai đoạn, với mỗi giai đoạn đó cần bón cho lúa loại phân nào.
\end{baitoan}

\begin{baitoan}[\cite{SGK_Hoa_Hoc_9}, 1., p. 39]
	Có các loại phân bón hóa học: \emph{KCl, \ce{NH4NO3, NH4Cl, (NH4)2SO4, Ca3(PO4)2, Ca(H2PO4)2}, \ce{(NH4)2HPO4, KNO3}}. (a) Cho biết tên hóa học của các phân bón này. (b) Sắp xếp các phân bón này thành 2 nhóm phân bón đơn \& phân bón kép. (c) Trộn các phân bón nào với nhau ta được phân bón kép NPK?
\end{baitoan}

\begin{baitoan}[\cite{SGK_Hoa_Hoc_9}, 2., p. 39]
	Có 3 mẫu phân bón hóa học không ghi nhãn: phân kali \emph{KCl}, phân đạm \emph{\ce{NH4NO3}} \& phân supephotphat (phân lân) \emph{\ce{Ca(H2PO4)2}}. Nhận biết mỗi mẫu phân bón trên băng phương pháp hóa học.
\end{baitoan}

\subsection{Quantitative problem -- Bài tập định lượng}

\begin{baitoan}[\cite{SGK_Hoa_Hoc_9}, 3., p. 39]
	1 người làm vườn đã dùng \emph{500 g \ce{(NH4)2SO4}} để bón rau. (a) Nguyên tố dinh dưỡng nào có trong loại phân bón này? (b) Tính thành phần \% của nguyên tố dinh dưỡng trong phân bón. (c) Tính khối lượng của nguyên tố dinh dưỡng bón cho ruộng rau.
\end{baitoan}

%------------------------------------------------------------------------------%

\section{Miscellaneous}

\subsection{Qualitative problem -- Bài tập định tính}

\begin{baitoan}[\cite{SGK_KHTN_8_Canh_Dieu}, 1., p. 72]
	Trong các chất: \emph{HCl, CuO, KOH, \ce{CaCO3,H2SO4,Fe(OH)2}}, chất nào là acid, base, kiềm?
\end{baitoan}

\begin{baitoan}[\cite{SGK_KHTN_8_Canh_Dieu}, 2., p. 72]
	Trong các chất: \emph{\ce{CuSO4,SO2,MgCl2,CaO,Na2CO3}}, chất nào là muối, oxide base, oxide acid. Viết tên gọi các muối.
\end{baitoan}

\begin{baitoan}[\cite{SGK_KHTN_8_Canh_Dieu}, 3., p. 72]
	Chất nào trong dãy chất sau: \emph{CuO, \ce{Mg(OH)2}, Fe, \ce{SO2}, HCl, \ce{CuSO4}} tác dụng được với: (a) dung dịch \emph{NaOH}. (b) dung dịch \emph{\ce{H2SO4}} loãng. Viết các PTHH của các phản ứng (nếu có).
\end{baitoan}

\begin{baitoan}[\cite{SGK_KHTN_8_Canh_Dieu}, 4., p. 72]
	Viết các PTHH theo các sơ đồ: (a) \emph{\ce{HCl + $?$ -> NaCl + H2O}}. (b) \emph{\ce{NaOH + $?$ -> Cu(OH)2 v + $?$}}. (c) \emph{\ce{KOH + $?$ -> K2SO4 + $?$}}. (d) \emph{\ce{Ba(NO3)2 + $?$ -> BaSO4 v + $?$}}.
\end{baitoan}

\begin{baitoan}[\cite{SGK_KHTN_8_Canh_Dieu}, 5., p. 72]
	Viết các PTHH theo các sơ đồ chuyển hóa sau: (a) \emph{CuO $\to$ \ce{CuSO4} $\to$ \ce{Cu(OH)2}}. (b) \emph{Mg $\to$ \ce{MgCl2} $\to$ \ce{Mg(OH)2}}. (c) \emph{NaOH $\to$ \ce{Na2SO4} $\to$ NaCl}. (d) \emph{\ce{K2CO3} $\to$ \ce{CaCO3} $\to$ \ce{CaCl2}}.
\end{baitoan}

\begin{baitoan}[\cite{SGK_Hoa_Hoc_9}, 1., p. 41]
	Chất nào trong các thuốc thử sau có thể dùng để phân biệt dung dịch sodium sulfate \& dung dịch sodium carbonate? (a) Dung dịch barium chloride. (b) Dung dịch hydrochloric acid. (c) Dung dịch chì nitrate. (d) Dung dịch bạc nitrate. (e) Dung dịch sodium hydroxide. Giải thích \& viết các PTHH.
\end{baitoan}

\begin{baitoan}[\cite{SGK_Hoa_Hoc_9}, 2., p. 41]
	Cho các dung dịch sau lần lượt phản ứng với nhau từng đôi một, ghi $1$ nếu có phản ứng, $0$ nếu không có phản ứng. Viết các PTHH nếu có.
	\begin{table}[H]
		\centering
		\begin{tabular}{|c|c|c|c|}
			\hline
			& NaOH & HCl & \ce{H2SO4} \\
			\hline
			\ce{CuSO4} &  &  &  \\
			\hline
			HCl &  &  &  \\
			\hline
			\ce{Ba(OH)2} &  &  &  \\
			\hline
		\end{tabular}
	\end{table}
\end{baitoan}

\begin{baitoan}[\cite{SGK_Hoa_Hoc_9}, 4., p. 41]
	Có các chất: \emph{\ce{Na2O}, Na, NaOH, \ce{Na2SO4,Na2CO3}, NaCl}. (a) Dựa vào mối quan hệ giữa các chất, sắp xếp các chất trên thành 1 dãy chuyển đổi hóa học. (b) Viết các PTHH cho dãy chuyển đổi hóa học ở (a).
\end{baitoan}

\begin{baitoan}[\cite{SGK_Hoa_Hoc_9}, 2., p. 43]
	Để 1 mẩu sodium hydroxide trên tấm kính trong không khí, sau vài ngày thấy có chất rắn màu trắng phủ ngoài. Nếu nhỏ vài giọt dung dịch \emph{HCl} vào chất rắn trắng thấy có khí thoát ra, khí này làm đục nước vôi trong. Chất rắn màu trắng là sản phẩm phản ứng của sodium hydroxide với chất nào sau đây? Giải thích \& viết PTHH minh họa. (a) Oxygen trong không khí. (b) Hơi nước trong không khí. (c) Carbon dioxide \& oxygen trong không khí. (d) Carbon dioxide \& hơi nước trong không khí. (e) Carbon dioxide trong không khí.
\end{baitoan}

\subsection{Quantitative problem -- Bài tập định lượng}

\begin{baitoan}[\cite{SGK_KHTN_8_Canh_Dieu}, 6., p. 72]
	Cho \emph{100 mL} dung dịch \emph{\ce{Na2SO4} 0.5 M} tác dụng vừa đủ với dung dịch \emph{\ce{BaCl2}} thì thu được $m$ \emph{g} kết tủa. (a) Viết PTHH của phản ứng xảy ra. (b) Tính $m$. (c) Tính nồng độ mol của dung dịch \emph{\ce{BaCl2}}, biết thể tích dung dịch \emph{\ce{BaCl2}} đã dùng là \emph{50 mL}.
\end{baitoan}

\begin{baitoan}[\cite{SGK_KHTN_8_Canh_Dieu}, 7., p. 72]
	Viết các PTHH điều chế \emph{\ce{MgCl2}} trực tiếp từ \emph{MgO, \ce{Mg(OH)2,MgSO4}}.
\end{baitoan}

\begin{baitoan}[\cite{SGK_KHTN_8_Canh_Dieu}, 8., p. 72]
	Biết dung dịch \emph{NaCl} có \emph{pH $= 7$}. Chỉ dùng quỳ tím, nêu cách nhận biết các dung dịch không màu, đựng trong 3 ống nghiệm riêng rẽ: \emph{NaOH, HCl, NaCl}.
\end{baitoan}

\begin{baitoan}[\cite{SGK_KHTN_8_Canh_Dieu}, 9., p. 72]
	Việc bón phân NPK cho cây cà phê sau khi trồng 4 năm được chia thành 4 thời kỳ như sau:
	\begin{table}[H]
		\centering
		\begin{tabular}{|l|l|}
			\hline
			Thời kỳ & Lượng phân bón \\
			\hline
			Bón thúc ra hoa & 0.5 kg phân NPK 10-12-5\texttt{/}cây \\
			\hline
			Bón đậu quả, ra quả & 0.7 kg phân NPK 12-8-2\texttt{/}cây \\
			\hline
			Bón quả lớn, hạn chế rụng quả & 0.7 kg phân NPK 12-8-12\texttt{/}cây \\
			\hline
			Bón thúc quả lớn, tăng dưỡng chất cho quả & 0.6 kg phân NPK 16-16-16\texttt{/}cây \\
			\hline
		\end{tabular}
	\end{table}
	\noindent(a) Tính lượng \emph{N} đã cung cấp cho cây trong cả 4 thời kỳ. (b) Nguyên tố dinh dưỡng potassium được bổ sung cho cây nhiều nhất ở thời kỳ nào?
\end{baitoan}

\begin{baitoan}[\cite{SGK_Hoa_Hoc_9}, 3., p. 43]
	Trộn 1 dung dịch có hòa tan \emph{0.2 mol \ce{CuCl2}} với 1 dung dịch có hòa tan \emph{20 g NaOH}. Lọc hỗn hợp các chất sau phản ứng, được kết tủa \& nước lọc. Nung kết tủa đến khi khối lượng không đổi. (a) Viết các PTHH. (b) Tính khối lượng chất rắn thu được sau khi nung. (c) Tính khối lượng các chất tan có trong nước lọc.
\end{baitoan}

\begin{baitoan}[\cite{An_Hoa_Hoc_nang_cao_8_9}, 1., p. 61]
	Viết {\rm CTHH} của các muối: calcium chloride, potassium nitrate, potassium phosphate, aluminium sulfate, iron ({\rm III}) nitrate.
\end{baitoan}

\begin{proof}[Giải]
	CTHH các muối đã cho: canxi clorua{\tt/}calcium chloride: \ce{CaCl2}, kali nitrat{\tt/}potassium nitrate \ce{KNO3}, kali photphat{\tt/}potassium phosphate \ce{K3PO4}, nhôm sunfat{\tt/}aluminium sulfate \ce{Al2(SO4)3}, sắt (III) nitrat{\tt/}iron (III) nitrate: \ce{Fe(NO3)3}.
\end{proof}

\begin{baitoan}[\cite{An_Hoa_Hoc_nang_cao_8_9}, 2., p. 62]
	Phân loại: {\rm\ce{KOH,CuCl2,Al2O3,ZnSO4,CuO,Zn(OH)2,H3PO4,HNO3}}.
\end{baitoan}

\begin{proof}[Giải]
	Oxide base: \ce{Al2O3}, CuO. Acid: \ce{H3PO4,HNO3}. Base: KOH, \ce{Zn(OH)2}. Muối: \ce{ZnSO4,CuCl2,CuSO4}.
\end{proof}

\begin{baitoan}[\cite{An_Hoa_Hoc_nang_cao_8_9}, 3., p. 62]
	Cho biết gốc acid \& tính hóa trị của gốc acid trong các {\rm CTHH}: {\rm\ce{H2S,HNO3,H2SO4,H2SiO3,H3PO4}, \ce{HClO4,H2Cr2O7,CH3COOH}}.
\end{baitoan}

\begin{baitoan}[\cite{An_Hoa_Hoc_nang_cao_8_9}, 4., p. 62]
	Viết công thức của các hydroxide ứng với các kim loại: sodium, calcium, chromium, barium, potassium, copper, zinc, iron.
\end{baitoan}

\begin{baitoan}[\cite{An_Hoa_Hoc_nang_cao_8_9}, 5., p. 62]
	Viết {\rm PTHH} biểu diễn các biến hóa: (a) {\rm Ca $\to$ CaO $\to$ \ce{Ca(OH)2}}. (b) {\rm Ca $\to$ \ce{Ca(OH)2}}.
\end{baitoan}

\begin{baitoan}[\cite{An_Hoa_Hoc_nang_cao_8_9}, 6., p. 63]
	Tính khối lượng sodium hydroxide thu được khi cho sodium tác dụng với nước: (a) {\rm46 g} sodium. (b) {0.3 mol} sodium.\hfill{\sf Ans: (a) 80 g. (b) 12 g.}
\end{baitoan}

\begin{baitoan}[\cite{An_Hoa_Hoc_nang_cao_8_9}, 7., p. 63]
	Tìm hiểu về copper ({\rm II}) oxide: (a) Cách điều chế. (b) Chất này thuộc loại hợp chất nào? (c) Tính chất vật lý. (d) Tính chất hóa học. Viết {\rm PTHH} \& phân loại các phản ứng đó.
\end{baitoan}

\begin{baitoan}[\cite{An_Hoa_Hoc_nang_cao_8_9}, 8., p. 64]
	Trong các oxide: {\rm\ce{SO3,CO,CuO,Na2O,CaO,CO2,Al2O3}}, oxide nào hòa tan trong nước? Viết {\rm PTHH} \& gọi tên các sản phẩm tạo thành.
\end{baitoan}

\begin{baitoan}[\cite{An_Hoa_Hoc_nang_cao_8_9}, 9., p. 64]
	Phân loại: {\rm\ce{CaO,H2SO4,Fe(OH)2,FeSO4,CaSO4,LiOH,MnO2,CuCl2,Mn(OH)2,SO2}}.
\end{baitoan}

\begin{baitoan}[\cite{An_Hoa_Hoc_nang_cao_8_9}, 10., p. 65]
	Viết {\rm PTHH} biểu diễn các biến hóa: (a) {\rm S $\to$ \ce{SO2} $\to$ \ce{H2SO3}}. (b) {\rm Cu $\to$ CuO $\to$ \ce{CuSO4}}. (c) {\rm Ca $\to$ CaO $\to$ \ce{Ca(OH)2}}. (d) {\rm P $\to$ \ce{P2O5} $\to$ \ce{H3PO4}}.
\end{baitoan}

\begin{baitoan}[\cite{An_Hoa_Hoc_nang_cao_8_9}, 11., p. 65]
	Cho các chất: {\rm\ce{Na2O,P2O5}}, dung dịch acid {\rm\ce{H2SO4}}, dung dịch {\rm KOH}. Bằng phương pháp hóa học, nêu cách nhận biết các hợp chất trên.
\end{baitoan}

\begin{baitoan}[\cite{An_Hoa_Hoc_nang_cao_8_9}, 12., p. 66]
	Hoàn thành {\rm PTHH}: (a) {\rm Mg + HCl}. (b) {\rm Al + \ce{H2SO4}}. (c) {\rm MgO + HCl}. (d) {\rm CaO + \ce{H3PO4}}. (e) {\rm CaO + \ce{HNO3}}.
\end{baitoan}

\begin{baitoan}[\cite{An_Hoa_Hoc_nang_cao_8_9}, 13., p. 66]
	Khi cho kẽm tác dụng với acid hydrochloric, thu được {\rm10 g} khí hydro. Tính số mol acid hydrochloric tham gia phản ứng.\hfill{\sf Ans: 10 mol.}
\end{baitoan}

\begin{baitoan}[\cite{An_Hoa_Hoc_nang_cao_8_9}, 14., p. 66]
	Tìm {\rm CTHH} của các chất có thành phần theo khối lượng: {\rm(a) H: 2.04\%, S: 32.65\%, O: 65.31\%. (b) Cu: 40\%, S: 20\%, O: 40\%}.
\end{baitoan}

\begin{baitoan}[\cite{An_Hoa_Hoc_nang_cao_8_9}, 15., p. 67]
	Lập {\rm CTHH} của các base ứng với các oxide: {\rm CaO, FeO, \ce{Li2O}, BaO}.
\end{baitoan}

\begin{baitoan}[\cite{An_Hoa_Hoc_nang_cao_8_9}, 16., p. 67]
	Khi cho barium tác dụng với nước \& cho barium oxide tác dụng với nước đều cho ta barium hydroxide. Viết {\rm PTHH}.
\end{baitoan}

\begin{baitoan}[\cite{An_Hoa_Hoc_nang_cao_8_9}, 17., p. 68]
	Diphosphor pentoxide là 1 chất rắn trắng khi để ra ngoài không khí thì bị chảy rữa. Tại sao? Viết {\rm PTHH}.
\end{baitoan}
\begin{baitoan}[\cite{An_Hoa_Hoc_nang_cao_8_9}, 18., p. 68]
	Từ $100$ tấn quặng chứa $40\%$ lưu huỳnh có thể điều chế được bao nhiêu tấn acid sulfuric?\hfill{\sf Ans: 122.5 tấn.}
\end{baitoan}

\begin{baitoan}[\cite{An_Hoa_Hoc_nang_cao_8_9}, 19., p. 68]
	Viết {\rm CTHH} của các muối: potassium chloride, calcium nitrate, copper sulfate, sodium sulfite, sodium nitrate, calcium phosphate, copper carbonate.
\end{baitoan}

\begin{baitoan}[\cite{An_Hoa_Hoc_nang_cao_8_9}, 20., p. 69]
	Tính khối lượng vôi tôi {\rm\ce{Ca(OH)2}} có thể thu được khi cho {\rm 140 kg} vôi sống {\rm CaO} tác dụng với nước. Biết trong vôi sống có chứa $10\%$ tạp chất.\hfill{\sf Ans: 166.5 kg.}
\end{baitoan}

\begin{baitoan}[\cite{An_Hoa_Hoc_nang_cao_8_9}, 21., p. 69]
	Có bao nhiêu {\rm g} copper có thể bị {\rm0.5 mol} zinc đẩy ra khỏi dung dịch muối copper sulfate?\hfill{\sf Ans: 32 g.}
\end{baitoan}

\begin{baitoan}[\cite{An_Hoa_Hoc_nang_cao_8_9}, 22., p. 69]
	Có thể điều chế được các chất mới nào khi cho các chất: calcium oxide, nước, acid sulfuric, zinc. Viết {\rm PTHH}.
\end{baitoan}

\begin{baitoan}[\cite{An_Hoa_Hoc_nang_cao_8_9}, 23., p. 70]
	Tìm phương pháp xác định xem trong 3 lọ, lọ nào đựng dung dịch acid, muối ăn, \& dung dịch kiềm (base).
\end{baitoan}

\begin{baitoan}[\cite{An_Hoa_Hoc_nang_cao_8_9}, 24., p. 70]
	Cho các chất: aluminium, oxygen, nước, copper sulfate, iron, acid hydrochloric. Điều chế copper, copper oxide, aluminium chloride (bằng 2 phương pháp), \& iron chloride. Viết {\rm PTHH}.
\end{baitoan}

\begin{baitoan}[\cite{An_Hoa_Hoc_nang_cao_8_9}, 25., p. 70]
	Muốn điều chế calcium sulfate từ sulfur \& calcium cần thêm ít nhất các hóa chất gì? Viết {\rm PTHH}.
\end{baitoan}

\begin{baitoan}[\cite{An_Hoa_Hoc_nang_cao_8_9}, 26., p. 71]
	Đổ vào dung dịch chứa {\rm27 g} copper chloride, {\rm12 g} mạt sắt. Tính lượng {\rm Cu} thu được sau phản ứng.\hfill{\sf Ans: 12.8 g.}
\end{baitoan}

\begin{baitoan}[\cite{An_Hoa_Hoc_nang_cao_8_9}, 27., p. 71]
	Trong 1 ống nghiệm, hòa tan {\rm5g} copper sulfate ngậm nước {\rm\ce{CuSO4.$5$H2O}}, rồi thả vào đó 1 miếng kẽm. Có bao nhiêu {\rm g} đồng nguyên chất thoát ra sau phản ứng, biết đã lấy thừa kẽm.\hfill{\sf Ans: 1.28 g.}
\end{baitoan}

\begin{baitoan}[\cite{An_Hoa_Hoc_nang_cao_8_9}, 28., p. 72]
	Viết {\rm PTHH}: (a) {\rm\ce{CuSO4} $\to$ Cu $\to$ CuO $\to$ \ce{Cu(NO3)2}}. (b) {\rm Ca $\to$ \ce{CaCl2} $\to$ \ce{Ca(OH)2}}.
\end{baitoan}

\begin{baitoan}[\cite{An_Hoa_Hoc_nang_cao_8_9}, 29., p. 72]
	Viết {\rm PTHH} biểu diễn các biến hóa: 
	\begin{equation*}
		{\rm CuO}\ \left[\begin{split}
			&\to{\rm Cu}\\
			&\to\ce{CuSO4}\to{\rm Cu}.\\
			&\to\ce{CuCl2}
		\end{split}\right.
	\end{equation*}
\end{baitoan}

\begin{baitoan}[\cite{An_Hoa_Hoc_nang_cao_8_9}, 30., p. 73]
	Hoàn thành {\rm PTHH}: (a) {\rm Zn + \ce{H2SO4}}. (b) {\rm Mg + \ce{H2SO4}}. (c) {\rm ZnO + \ce{HNO3}}. (d) {\rm CaO + HCl}. (e) {\rm MgO + \ce{H2SO4}}. (f) {\rm\ce{Al2O3} + HCl}. (g) {\rm\ce{Na2O + H2SO4}}.
\end{baitoan}

\begin{baitoan}[\cite{An_Hoa_Hoc_nang_cao_8_9}, 31., p. 73]
	Có thể thu được bao nhiêu {\rm g \ce{H2}} khi cho {\rm13 g} zinc tác dụng với acid hydrochloric lấy dư? Có bao nhiêu {\rm g} muối được tạo thành trong phản ứng này?\hfill{\sf Ans: 27.2 g.}
\end{baitoan}

\begin{baitoan}[\cite{An_Hoa_Hoc_nang_cao_8_9}, 32., p. 73]
	Tính thể tích khí hydrogen thu được (đktc) khi cho {\rm2.4g} magnesium tác dụng hoàn toàn với dung dịch acid sulfuric.\hfill{\sf Ans: 2.24 L.}
\end{baitoan}

\begin{baitoan}[\cite{An_Hoa_Hoc_nang_cao_8_9}, 33., p. 74]
	Cho {\rm7 g} calcium oxide tác dụng với dung dịch chứa {\rm35 g} acid nitric. Tính lượng muối tạo thành.\hfill{\sf Ans: 20.5 g.}
\end{baitoan}

\begin{baitoan}[\cite{An_Hoa_Hoc_nang_cao_8_9}, 34., p. 74]
	Hòa tan {\rm1.6 g} copper oxide trong {\rm100 g} dung dịch {\rm\ce{H2SO4} 20\%}. (a) Viết {\rm PTHH}. (b) Bao nhiêu {\rm g} acid đã tham gia phản ứng. (c) Bao nhiêu {\rm g} muối đồng được tạo thành. (d) Tính nồng độ $\%$ của acid trong dung dịch thu được sau phản ứng.\hfill{\sf Ans: (b) 1.96 g. (c) 3.2 g. (d) 17.8.}
\end{baitoan}

\begin{baitoan}[\cite{An_Hoa_Hoc_nang_cao_8_9}, 35., p. 75]
	Cho các oxide: {\rm\ce{CO2,SiO2,Na2O,Fe2O3,P2O5}}. Chất nào tan trong nước, chất nào tan trong dung dịch kiềm, chất nào tan trong dung dịch {\rm HCl}. Viết {\rm PTHH}.
\end{baitoan}

\begin{baitoan}[\cite{An_Hoa_Hoc_nang_cao_8_9}, 36., p. 76]
	(a) Từ {\rm60 kg} quặng pirit. Tính lượng {\rm\ce{H2SO4} 96\%} thu được từ quặng này nếu hiệu suất là $85\%$ so với lý thuyết. (b) Từ {\rm80} tấn quặng pirit chứa {\rm40\% S} sản xuất được {\rm92} tấn {\rm\ce{H2SO4}}. Tính hiệu suất.\hfill{\sf Ans: (a) 86.77 kg. (b) 93.88\%.}
\end{baitoan}

\begin{baitoan}[\cite{An_Hoa_Hoc_nang_cao_8_9}, 37., p. 77]
	Cho {\rm114 g} dung dịch {\rm\ce{H2SO4} 20\%} vào {\rm400 g} dung dịch {\rm\ce{BaCl2} 5.2\%}. (a) Viết {\rm PTHH} \& tính khối lượng kết tủa tạo thành. (b) Tính nồng độ $\%$ của các chất có trong dung dịch sau khi tách bỏ kết tủa.\hfill{\sf Ans: (a) 23.3 g. (b) 1.48\%, 2.65\%.}
\end{baitoan}

\begin{baitoan}[\cite{An_Hoa_Hoc_nang_cao_8_9}, 38., p. 78]
	Khi cho $a$ {\rm g} dung dịch {\rm \ce{H2SO4}} nồng độ $A\%$ tác dụng với 1 lượng hỗn hợp 2 kim loại {\rm Na, Zn} (dùng dư) thì khối lượng {\rm \ce{H2}} tạo thành là $0.05a$ {\rm g}. Xác định nồng độ $A\%$.\hfill{\sf Ans: $15.8\%$.}
\end{baitoan}

\begin{baitoan}[\cite{An_Hoa_Hoc_nang_cao_8_9}, 39., p. 78]
	Trộn lẫn {\rm100 mL} dung dịch {\rm\ce{NaHSO4} 1M} với {\rm100 mL} dung dịch {\rm NaOH 2M} được dung dịch A. Cô cạn dung dịch A thì thu được hỗn hợp các chất nào?\hfill{\sf Ans: 14.2 g, 4 g.}
\end{baitoan}

\begin{baitoan}[\cite{An_Hoa_Hoc_nang_cao_8_9}, 40., p. 78]
	Cho {\rm15.9 g} hỗn hợp 2 muối {\rm\ce{MgCO3,CaCO3}} vào {\rm0.4 L} dung dịch {\rm HCl 1M} thu được dung dịch X. Hỏi dung dịch X có dư acid không?\hfill{\sf Ans: Acid dư.}
\end{baitoan}

\begin{baitoan}[\cite{An_Hoa_Hoc_nang_cao_8_9}, 41., p. 78]
	Cho {\rm6.2 g \ce{Na2O}} vào nước. Tính thể tích khí {\rm\ce{SO2}} (đktc) cần thiết với dung dịch trên để tạo 2 muối.\\\mbox{}\hfill{\sf Ans: 2.24 L $< V <$ 4.48 L.}
\end{baitoan}

\begin{baitoan}[\cite{An_Hoa_Hoc_nang_cao_8_9}, 42., p. 78]
	Tìm các ký hiệu bằng chữ cái trong sơ đồ sau \& hoàn thành sơ đồ bằng {\rm PTHH}: (a) {\rm A $\to$ CaO $\to$ \ce{Ca(OH)2} $\to$ A $\to$ \ce{Ca(HCO3)2} $\to$ \ce{CaCl2} $\to$ A}. (b) {\rm\ce{FeS2} $\to$ M $\to$ N $\to$ D $\to$ \ce{CaSO4}}. (c) {\rm\ce{CuSO4} $\to$ B $\to$ C $\to$ D $\to$ Cu}.
\end{baitoan}

\begin{baitoan}[\cite{An_Hoa_Hoc_nang_cao_8_9}, 43., p. 78]
	Làm thế nào để nhận biết được $3$ acid {\rm HCl, \ce{HNO3,H2SO4}} cùng tồn tại trong dung dịch loãng.	
\end{baitoan}

\begin{baitoan}[\cite{An_Hoa_Hoc_nang_cao_8_9}, 44.a, p. 78]
	Viết {\rm PTHH} thực hiện chuyển hóa: {\rm\ce{Cl2} $\to$ A $\to$ B $\to$ C $\to$ A $\to$ \ce{Cl2}}, trong đó $A$ là chất khí, B \& C là hợp chất chứa chlorine.
\end{baitoan}

\begin{baitoan}[\cite{An_Hoa_Hoc_nang_cao_8_9}, 45., p. 78]
	Hòa tan hoàn toàn $a$ {\rm g \ce{R2O3}} cần $b$ {\rm g} dung dịch {\rm\ce{H2SO4} 12.25\%} thì vừa đủ. Sau phản ứng thu được dung dịch muối có nồng độ $15.36\%$. Xác định kim loại R.\hfill{\sf Ans: Cr.}
\end{baitoan}

\begin{baitoan}[\cite{An_Hoa_Hoc_nang_cao_8_9}, 46., p. 79]
	Hòa tan {\rm13.2 g} hỗn hợp X gồm 2 kim loại có cùng hóa trị vào {\rm400 mL} dung dịch {\rm HCl 1.5M}. Cô cạn dung dịch sau phản ứng thu được {\rm32.7 g} hỗn hợp muối khan. Hỗn hợp X có tan hết trong dung dịch {\rm HCl} không?\hfill{\sf Ans: Không tan hết.}
\end{baitoan}

\begin{baitoan}[\cite{An_Hoa_Hoc_nang_cao_8_9}, 47., p. 79]
	Trộn $V_1$ {\rm L} dung dịch {\rm HCl 0.6M} với $V_2$ {\rm L} dung dịch {\rm NaOH 0.4M} thu được {\rm0.6 L} dung dịch A. Tính $V_1,V_2$ biết {\rm0.6 L} dung dịch A có thể hòa tan hết {\rm1.02 g \ce{Al2O3}}. Biết sự pha trộn không làm thay đổi thể tích 1 cách đáng kể.\footnote{Đã học ở Vật lý 8 về sự đan xen của các nguyên tử, phân tử của 2 hay nhiều dung dịch khi trộn vào nhau, xem \cite[\S19, pp. 68--70]{SGK_Vat_Ly_8}.}\hfill{\sf Ans: $(V_1,V_2)\in\{(0.3,0.3),(0.22,0.38)\}$.}
\end{baitoan}

\begin{baitoan}[\cite{An_Hoa_Hoc_nang_cao_8_9}, 48., p. 79]
	Cho {\rm39.6 g} hỗn hợp gồm {\rm\ce{KHSO3, K2CO3}} vào {\rm400 g} dung dịch {\rm HCl 7.3\%}. Sau phản ứng thu được hỗn hợp khí X có tỷ khối hơi so với {\rm\ce{H2}} bằng $25.33$ \& 1 dung dịch Y. (a) Chứng minh acid còn dư. (b) Tính $C\%$ các chất trong dung dịch Y.\hfill{\sf Ans: $8.78\%$, $2.58\%$.}
\end{baitoan}

\begin{baitoan}[\cite{An_Hoa_Hoc_nang_cao_8_9}, 20., p. 135]
	Để khử {\rm6.4 g} 1 oxide kim loại cần {\rm2.688 L} khí {\rm\ce{H2}}. Nếu lấy lượng kim loại đó cho tác dụng với dung dịch {\rm HCl} dư thì giải phóng {\rm1.792 L} khí {\rm\ce{H2}}. Tìm tên kim loại biết thể tích các khí đo ở đktc.
\end{baitoan}

\begin{baitoan}[\cite{An_Hoa_Hoc_nang_cao_8_9}, 21., p. 135]
	Có 4 oxide riêng biệt: {\rm\ce{Na2O,Al2O3,Fe2O3}, MgO}. Làm thế nào để nhận biết mỗi oxide bằng phương pháp hóa học với điều kiện chỉ được dùng thêm 2 chất là {\rm\ce{H2O}} \& dung dịch {\rm HCl}.
\end{baitoan}

\begin{baitoan}[\cite{An_Hoa_Hoc_nang_cao_8_9}, 22., p. 135]
	Cho $a$ {\rm a Fe} hòa tan trong dung dịch {\rm HCl} (thí nghiệm 1). sau khi cô cạn dung dịch thu được {\rm3.1 g} chất rắn. Nếu cho $a$ {\rm g Fe} \& $b$ {\rm g Mg} (thí nghiệm 2) vào dung dịch {\rm HCl} loãng (cùng lượng như trên) thu được {\rm4.48 mL \ce{H2}} \& sau khi cô cạn dung dịch thu được {\rm3.34 g} chất rắn. Tính $a,b$.
\end{baitoan}

\begin{baitoan}[\cite{An_Hoa_Hoc_nang_cao_8_9}, 25., p. 135]
	Cho {\rm31.8 g} hỗn hợp 2 muối {\rm\ce{MgCO3,CaCO3}} vào {\rm0.8 L} dung dịch {\rm HCl 1M} thu được dung dịch Z. (a) Dung dịch Z có dư acid không? (b) Tính $V$ {\rm L \ce{CO2}} sinh ra là bao nhiêu?
\end{baitoan}

\begin{baitoan}[\cite{An_Hoa_Hoc_nang_cao_8_9}, 38., p. 78]
	Khi cho $a$ \emph{g} dung dịch \emph{\ce{H2SO4}} nồng độ $A$\emph{\%} tác dụng với 1 lượng hỗn hợp 2 kim loại \emph{Na,Zn} (dùng dư) thì khối lượng \emph{\ce{H2}} tạo thành là $0.05a$ \emph{g}. Xác định nồng độ $A$\emph{\%}.
\end{baitoan}

\begin{baitoan}[\cite{An_Hoa_Hoc_nang_cao_8_9}, 39., p. 78]
	Trộn lẫn \emph{100 mL} dung dịch \emph{\ce{NaHSO4} 1M} với \emph{100 mL} dung dịch \emph{NaOH 2M} được dung dịch A. Cô cạn dung dịch A thì thu được hỗn hợp các chất nào?
\end{baitoan}

\begin{baitoan}[\cite{An_Hoa_Hoc_nang_cao_8_9}, 40., p. 78]
	Cho \emph{15.9 g} hỗn hợp 2 muối \emph{\ce{MgCO3,CaCO3}} vào \emph{0.4 L} dung dịch \emph{HCl 1M} thu được dung dịch X. Hỏi dung dịch X có dư acid không?
\end{baitoan}

\begin{baitoan}[\cite{An_Hoa_Hoc_nang_cao_8_9}, 41., p. 78]
	Cho \emph{6.2 g \ce{Na2O}} vào nước. Tính thể tích khí \emph{\ce{SO2}} (đktc) cần thiết với dung dịch trên để tạo 2 muối.
\end{baitoan}

\begin{baitoan}[\cite{An_Hoa_Hoc_nang_cao_8_9}, 45., p. 78]
	Hòa tan hoàn toàn $a$ \emph{g \ce{R2O3}} cần $b$ \emph{g} dung dịch \emph{\ce{H2SO4} 12.25\%} thì vừa đủ. Sau phản ứng thu được dung dịch muối có nồng độ \emph{15.36\%}. Xác định kim loại R.
\end{baitoan}

\begin{baitoan}[\cite{An_Hoa_Hoc_nang_cao_8_9}, 46., p. 79]
	Hòa tan \emph{13.2 g} hỗn hợp X gồm 2 kim loại có cùng hóa trị vào \emph{400 mL} dung dịch \emph{HCl 1.5M}. Cô cạn dung dịch sau phản ứng thu được \emph{32.7 g} hỗn hợp muối khan. Hỗn hợp X có tan hết trong dung dịch \emph{HCl} không?
\end{baitoan}

\begin{baitoan}[\cite{An_Hoa_Hoc_nang_cao_8_9}, 47., p. 79]
	Trộn $V_1$ \emph{L} dung dịch \emph{HCl 0.6M} với $V_2$ \emph{L} dung dịch \emph{NaOH 0.4M} thu được \emph{0.6 L} dung dịch A. Tính $V_1,V_2$ biết \emph{0.6 L} dung dịch A có thể hòa tan hết \emph{1.02 g \ce{Al2O3}}. Biết sự pha trộn không làm thay đổi thể tích 1 cách đáng kể.\footnote{Đã học ở Vật lý 8 về sự đan xen của các nguyên tử, phân tử của 2 hay nhiều dung dịch khi trộn vào nhau.}
\end{baitoan}

\begin{baitoan}[\cite{An_Hoa_Hoc_nang_cao_8_9}, 48., p. 79]
	Cho \emph{39.6 g} hỗn hợp gồm \emph{\ce{KHSO3, K2CO3}} vào \emph{400 g} dung dịch \emph{HCl 7.3\%}. Sau phản ứng thu được hỗn hợp khí X có tỷ khối hơi so với \emph{\ce{H2}} bằng $25.33$ \& 1 dung dịch Y. (a) Chứng minh acid còn dư. (b) Tính C\% các chất trong dung dịch Y.
\end{baitoan}

%------------------------------------------------------------------------------%

\printbibliography[heading=bibintoc]
	
\end{document}