\documentclass{article}
\usepackage[backend=biber,natbib=true,style=alphabetic,maxbibnames=50]{biblatex}
\addbibresource{/home/nqbh/reference/bib.bib}
\usepackage[utf8]{vietnam}
\usepackage{tocloft}
\renewcommand{\cftsecleader}{\cftdotfill{\cftdotsep}}
\usepackage[colorlinks=true,linkcolor=blue,urlcolor=red,citecolor=magenta]{hyperref}
\usepackage{amsmath,amssymb,amsthm,float,graphicx,mathtools,diagbox,tikz,tipa}
\usepackage[version=4]{mhchem}
\allowdisplaybreaks
\newtheorem{assumption}{Assumption}
\newtheorem{baitoan}{}
\newtheorem{cauhoi}{Câu hỏi}
\newtheorem{conjecture}{Conjecture}
\newtheorem{corollary}{Corollary}
\newtheorem{dangtoan}{Dạng toán}
\newtheorem{definition}{Definition}
\newtheorem{dinhly}{Định lý}
\newtheorem{dinhnghia}{Định nghĩa}
\newtheorem{example}{Example}
\newtheorem{ghichu}{Ghi chú}
\newtheorem{hequa}{Hệ quả}
\newtheorem{hypothesis}{Hypothesis}
\newtheorem{lemma}{Lemma}
\newtheorem{luuy}{Lưu ý}
\newtheorem{nhanxet}{Nhận xét}
\newtheorem{notation}{Notation}
\newtheorem{note}{Note}
\newtheorem{principle}{Principle}
\newtheorem{problem}{Problem}
\newtheorem{proposition}{Proposition}
\newtheorem{question}{Question}
\newtheorem{remark}{Remark}
\newtheorem{theorem}{Theorem}
\newtheorem{thinghiem}{Thí nghiệm}
\newtheorem{vidu}{Ví dụ}
\usepackage[left=1cm,right=1cm,top=5mm,bottom=5mm,footskip=4mm]{geometry}

\begin{document}

\begin{baitoan}[\cite{An_Hoa_Hoc_nang_cao_8_9}, 1., p. 61]
	Viết {\rm CTHH} của các muối: calci chloride, potassium nitrate, potassium phosphate, aluminium sulfate, iron ({\rm III}) nitrate.
\end{baitoan}

\begin{baitoan}[\cite{An_Hoa_Hoc_nang_cao_8_9}, 2., p. 62]
	Phân loại: {\rm\ce{KOH,CuCl2,Al2O3,ZnSO4,CuO,Zn(OH)2,H3PO4,HNO3}}.
\end{baitoan}

\begin{baitoan}[\cite{An_Hoa_Hoc_nang_cao_8_9}, 3., p. 62]
	Cho biết gốc acid \& tính hóa trị của gốc acid trong các {\rm CTHH}: {\rm\ce{H2S,HNO3,H2SO4,H2SiO3,H3PO4}, \ce{HClO4,H2Cr2O7,CH3COOH}}.
\end{baitoan}

\begin{baitoan}[\cite{An_Hoa_Hoc_nang_cao_8_9}, 4., p. 62]
	Viết công thức của các hydroxide ứng với các kim loại: sodium, calcium, chromium, barium, potassium, copper, zinc, iron.
\end{baitoan}

\begin{baitoan}[\cite{An_Hoa_Hoc_nang_cao_8_9}, 5., p. 62]
	Viết {\rm PTHH} biểu diễn các biến hóa: (a) {\rm Ca $\to$ CaO $\to$ \ce{Ca(OH)2}}. (b) {\rm Ca $\to$ \ce{Ca(OH)2}}.
\end{baitoan}

\begin{baitoan}[\cite{An_Hoa_Hoc_nang_cao_8_9}, 6., p. 63]
	Tính khối lượng sodium hydroxide thu được khi cho sodium tác dụng với nước: (a) {\rm46 g} sodium. (b) {0.3 mol} sodium.\hfill{\sf Ans: (a) 80 g. (b) 12 g.}
\end{baitoan}

\begin{baitoan}[\cite{An_Hoa_Hoc_nang_cao_8_9}, 7., p. 63]
	Tìm hiểu về copper ({\rm II}) oxide: (a) Cách điều chế. (b) Chất này thuộc loại hợp chất nào? (c) Tính chất vật lý. (d) Tính chất hóa học. Viết {\rm PTHH} \& phân loại các phản ứng đó.
\end{baitoan}

\begin{baitoan}[\cite{An_Hoa_Hoc_nang_cao_8_9}, 8., p. 64]
	Trong các oxide: {\rm\ce{SO3,CO,CuO,Na2O,CaO,CO2,Al2O3}}, oxide nào hòa tan trong nước? Viết {\rm PTHH} \& gọi tên các sản phẩm tạo thành.
\end{baitoan}

\begin{baitoan}[\cite{An_Hoa_Hoc_nang_cao_8_9}, 9., p. 64]
	Phân loại: {\rm\ce{CaO,H2SO4,Fe(OH)2,FeSO4,CaSO4,LiOH,MnO2,CuCl2,Mn(OH)2,SO2}}.
\end{baitoan}

\begin{baitoan}[\cite{An_Hoa_Hoc_nang_cao_8_9}, 10., p. 65]
	Viết {\rm PTHH} biểu diễn các biến hóa: (a) {\rm S $\to$ \ce{SO2} $\to$ \ce{H2SO3}}. (b) {\rm Cu $\to$ CuO $\to$ \ce{CuSO4}}. (c) {\rm Ca $\to$ CaO $\to$ \ce{Ca(OH)2}}. (d) {\rm P $\to$ \ce{P2O5} $\to$ \ce{H3PO4}}.
\end{baitoan}

\begin{baitoan}[\cite{An_Hoa_Hoc_nang_cao_8_9}, 11., p. 65]
	Cho các chất: {\rm\ce{Na2O,P2O5}}, dung dịch acid {\rm\ce{H2SO4}}, dung dịch {\rm KOH}. Bằng phương pháp hóa học, nêu cách nhận biết các hợp chất trên.
\end{baitoan}

\begin{baitoan}[\cite{An_Hoa_Hoc_nang_cao_8_9}, 12., p. 66]
	Hoàn thành {\rm PTHH}: (a) {\rm Mg + HCl}. (b) {\rm Al + \ce{H2SO4}}. (c) {\rm MgO + HCl}. (d) {\rm CaO + \ce{H3PO4}}. (e) {\rm CaO + \ce{HNO3}}.
\end{baitoan}

\begin{baitoan}[\cite{An_Hoa_Hoc_nang_cao_8_9}, 13., p. 66]
	Khi cho kẽm tác dụng với acid hydrochloric, thu được {\rm10 g} khí hydro. Tính số mol acid hydrochloric tham gia phản ứng.\hfill{\sf Ans: 10 mol.}
\end{baitoan}

\begin{baitoan}[\cite{An_Hoa_Hoc_nang_cao_8_9}, 14., p. 66]
	Tìm {\rm CTHH} của các chất có thành phần theo khối lượng: {\rm(a) H: 2.04\%, S: 32.65\%, O: 65.31\%. (b) Cu: 40\%, S: 20\%, O: 40\%}.
\end{baitoan}

\begin{baitoan}[\cite{An_Hoa_Hoc_nang_cao_8_9}, 15., p. 67]
	Lập {\rm CTHH} của các base ứng với các oxide: {\rm CaO, FeO, \ce{Li2O}, BaO}.
\end{baitoan}

\begin{baitoan}[\cite{An_Hoa_Hoc_nang_cao_8_9}, 16., p. 67]
	Khi cho barium tác dụng với nước \& cho barium oxide tác dụng với nước đều cho ta barium hydroxide. Viết {\rm PTHH}.
\end{baitoan}

\begin{baitoan}[\cite{An_Hoa_Hoc_nang_cao_8_9}, 17., p. 68]
	Diphosphor pentoxide là 1 chất rắn trắng khi để ra ngoài không khí thì bị chảy rữa. Tại sao? Viết {\rm PTHH}.
\end{baitoan}
\begin{baitoan}[\cite{An_Hoa_Hoc_nang_cao_8_9}, 18., p. 68]
	Từ $100$ tấn quặng chứa $40\%$ lưu huỳnh có thể điều chế được bao nhiêu tấn acid sulfuric?\hfill{\sf Ans: 122.5 tấn.}
\end{baitoan}

\begin{baitoan}[\cite{An_Hoa_Hoc_nang_cao_8_9}, 19., p. 68]
	Viết {\rm CTHH} ủa các muối: potassium chloride, calcium nitrate, copper sulfate, sodium sulfite, sodium nitrate, calcium phosphate, copper carbonate.
\end{baitoan}

\begin{baitoan}[\cite{An_Hoa_Hoc_nang_cao_8_9}, 20., p. 69]
	Tính khối lượng vôi tôi {\rm\ce{Ca(OH)2}} có thể thu được khi cho {\rm 140 kg} vôi sống {\rm CaO} tác dụng với nước. Biết trong vôi sống có chứa $10\%$ tạp chất.\hfill{\sf Ans: 166.5 kg.}
\end{baitoan}

\begin{baitoan}[\cite{An_Hoa_Hoc_nang_cao_8_9}, 21., p. 69]
	Có bao nhiêu {\rm g} copper có thể bị {\rm0.5 mol} zinc đẩy ra khỏi dung dịch muối copper sulfate?\hfill{\sf Ans: 32 g.}
\end{baitoan}

\begin{baitoan}[\cite{An_Hoa_Hoc_nang_cao_8_9}, 22., p. 69]
	Có thể điều chế được các chất mới nào khi cho các chất: calcium oxide, nước, acid sulfuric, zinc. Viết {\rm PTHH}.
\end{baitoan}

\begin{baitoan}[\cite{An_Hoa_Hoc_nang_cao_8_9}, 23., p. 70]
	Tìm phương pháp xác định xem trong 3 lọ, lọ nào đựng dung dịch acid, muối ăn, \& dung dịch kiềm (base).
\end{baitoan}

\begin{baitoan}[\cite{An_Hoa_Hoc_nang_cao_8_9}, 24., p. 70]
	Cho các chất: aluminium, oxygen, nước, copper sulfate, iron, acid hydrochloric. Điều chế copper, copper oxide, aluminium chloride (bằng 2 phương pháp), \& iron chloride. Viết {\rm PTHH}.
\end{baitoan}

\begin{baitoan}[\cite{An_Hoa_Hoc_nang_cao_8_9}, 25., p. 70]
	Muốn điều chế calcium sulfate từ sulfur \& calcium cần thêm ít nhất các hóa chất gì? Viết {\rm PTHH}.
\end{baitoan}

\begin{baitoan}[\cite{An_Hoa_Hoc_nang_cao_8_9}, 26., p. 71]
	Đổ vào dung dịch chứa {\rm27 g} copper chloride, {\rm12 g} mạt sắt. Tính lượng {\rm Cu} thu được sau phản ứng.\hfill{\sf Ans: 12.8 g.}
\end{baitoan}

\begin{baitoan}[\cite{An_Hoa_Hoc_nang_cao_8_9}, 27., p. 71]
	Trong 1 ống nghiệm, hòa tan {\rm5g} copper sulfate ngậm nước {\rm\ce{CuSO4.$5$H2O}}, rồi thả vào đó 1 miếng kẽm. Có bao nhiêu {\rm g} đồng nguyên chất thoát ra sau phản ứng, biết đã lấy thừa kẽm.\hfill{\sf Ans: 1.28 g.}
\end{baitoan}

\begin{baitoan}[\cite{An_Hoa_Hoc_nang_cao_8_9}, 28., p. 72]
	Viết {\rm PTHH}: (a) {\rm\ce{CuSO4} $\to$ Cu $\to$ CuO $\to$ \ce{Cu(NO3)2}}. (b) {\rm Ca $\to$ \ce{CaCl2} $\to$ \ce{Ca(OH)2}}.
\end{baitoan}

\begin{baitoan}[\cite{An_Hoa_Hoc_nang_cao_8_9}, 29., p. 72]
	Viết {\rm PTHH} biểu diễn các biến hóa: 
	\begin{equation*}
		{\rm CuO}\ \left[\begin{split}
			&\to{\rm Cu}\\
			&\to\ce{CuSO4}\to{\rm Cu}.\\
			&\to\ce{CuCl2}
		\end{split}\right.
	\end{equation*}
\end{baitoan}

\begin{baitoan}[\cite{An_Hoa_Hoc_nang_cao_8_9}, 30., p. 73]
	Hoàn thành {\rm PTHH}: (a) {\rm Zn + \ce{H2SO4}}. (b) {\rm Mg + \ce{H2SO4}}. (c) {\rm ZnO + \ce{HNO3}}. (d) {\rm CaO + HCl}. (e) {\rm MgO + \ce{H2SO4}}. (f) {\rm\ce{Al2O3} + HCl}. (g) {\rm\ce{Na2O + H2SO4}}.
\end{baitoan}

\begin{baitoan}[\cite{An_Hoa_Hoc_nang_cao_8_9}, 31., p. 73]
	Có thể thu được bao nhiêu {\rm g \ce{H2}} khi cho {\rm13 g} zinc tác dụng với acid hydrochloric lấy dư? Có bao nhiêu {\rm g} muối được tạo thành trong phản ứng này?\hfill{\sf Ans: 27.2 g.}
\end{baitoan}

\begin{baitoan}[\cite{An_Hoa_Hoc_nang_cao_8_9}, 32., p. 73]
	Tính thể tích khí hydrogen thu được (đktc) khi cho {\rm2.4g} magnesium tác dụng hoàn toàn với dung dịch acid sulfuric.\hfill{\sf Ans: 2.24 L.}
\end{baitoan}

\begin{baitoan}[\cite{An_Hoa_Hoc_nang_cao_8_9}, 33., p. 74]
	Cho {\rm7 g} calcium oxide tác dụng với dung dịch chứa {\rm35 g} acid nitric. Tính lượng muối tạo thành.\hfill{\sf Ans: 20.5 g.}
\end{baitoan}

\begin{baitoan}[\cite{An_Hoa_Hoc_nang_cao_8_9}, 34., p. 74]
	Hòa tan {\rm1.6 g} copper oxide trong {\rm100 g} dung dịch {\rm\ce{H2SO4} 20\%}. (a) Viết {\rm PTHH}. (b) Bao nhiêu {\rm g} acid đã tham gia phản ứng. (c) Bao nhiêu {\rm g} muối đồng được tạo thành. (d) Tính nồng độ $\%$ của acid trong dung dịch thu được sau phản ứng.\hfill{\sf Ans: (b) 1.96 g. (c) 3.2 g. (d) 17.8.}
\end{baitoan}

\begin{baitoan}[\cite{An_Hoa_Hoc_nang_cao_8_9}, 35., p. 75]
	Cho các oxide: {\rm\ce{CO2,SiO2,Na2O,Fe2O3,P2O5}}. Chất nào tan trong nước, chất nào tan trong dung dịch kiềm, chất nào tan trong dung dịch {\rm HCl}. Viết {\rm PTHH}.
\end{baitoan}

\begin{baitoan}[\cite{An_Hoa_Hoc_nang_cao_8_9}, 36., p. 76]
	(a) Từ {\rm60 kg} quặng pirit. Tính lượng {\rm\ce{H2SO4} 96\%} thu được từ quặng này nếu hiệu suất là $85\%$ so với lý thuyết. (b) Từ {\rm80} tấn quặng pirit chứa {\rm40\% S} sản xuất được {\rm92} tấn {\rm\ce{H2SO4}}. Tính hiệu suất.\hfill{\sf Ans: (a) 86.77 kg. (b) 93.88\%.}
\end{baitoan}

\begin{baitoan}[\cite{An_Hoa_Hoc_nang_cao_8_9}, 37., p. 77]
	Cho {\rm114 g} dung dịch {\rm\ce{H2SO4} 20\%} vào {\rm400 g} dung dịch {\rm\ce{BaCl2} 5.2\%}. (a) Viết {\rm PTHH} \& tính khối lượng kết tủa tạo thành. (b) Tính nồng độ $\%$ của các chất có trong dung dịch sau khi tách bỏ kết tủa.\hfill{\sf Ans: (a) 23.3 g. (b) 1.48\%, 2.65\%.}
\end{baitoan}

\begin{baitoan}[\cite{An_Hoa_Hoc_nang_cao_8_9}, 38., p. 78]
	Khi cho $a$ {\rm g} dung dịch {\rm \ce{H2SO4}} nồng độ $A\%$ tác dụng với 1 lượng hỗn hợp 2 kim loại {\rm Na, Zn} (dùng dư) thì khối lượng {\rm \ce{H2}} tạo thành là $0.05a$ {\rm g}. Xác định nồng độ $A\%$.\hfill{\sf Ans: $15.8\%$.}
\end{baitoan}

\begin{baitoan}[\cite{An_Hoa_Hoc_nang_cao_8_9}, 39., p. 78]
	Trộn lẫn {\rm100 mL} dung dịch {\rm\ce{NaHSO4} 1M} với {\rm100 mL} dung dịch {\rm NaOH 2M} được dung dịch A. Cô cạn dung dịch A thì thu được hỗn hợp các chất nào?\hfill{\sf Ans: 14.2 g, 4 g.}
\end{baitoan}

\begin{baitoan}[\cite{An_Hoa_Hoc_nang_cao_8_9}, 40., p. 78]
	Cho {\rm15.9 g} hỗn hợp 2 muối {\rm\ce{MgCO3,CaCO3}} vào {\rm0.4 L} dung dịch {\rm HCl 1M} thu được dung dịch X. Hỏi dung dịch X có dư acid không?\hfill{\sf Ans: Acid dư.}
\end{baitoan}

\begin{baitoan}[\cite{An_Hoa_Hoc_nang_cao_8_9}, 41., p. 78]
	Cho {\rm6.2 g \ce{Na2O}} vào nước. Tính thể tích khí {\rm\ce{SO2}} (đktc) cần thiết với dung dịch trên để tạo 2 muối.\\\mbox{}\hfill{\sf Ans: 2.24 L $< V <$ 4.48 L.}
\end{baitoan}

\begin{baitoan}[\cite{An_Hoa_Hoc_nang_cao_8_9}, 42., p. 78]
	Tìm các ký hiệu bằng chữ cái trong sơ đồ sau \& hoàn thành sơ đồ bằng {\rm PTHH}: (a) {\rm A $\to$ CaO $\to$ \ce{Ca(OH)2} $\to$ A $\to$ \ce{Ca(HCO3)2} $\to$ \ce{CaCl2} $\to$ A}. (b) {\rm\ce{FeS2} $\to$ M $\to$ N $\to$ D $\to$ \ce{CaSO4}}. (c) {\rm\ce{CuSO4} $\to$ B $\to$ C $\to$ D $\to$ Cu}.
\end{baitoan}

\begin{baitoan}[\cite{An_Hoa_Hoc_nang_cao_8_9}, 43., p. 78]
	Làm thế nào để nhận biết được $3$ acid {\rm HCl, \ce{HNO3,H2SO4}} cùng tồn tại trong dung dịch loãng.	
\end{baitoan}

\begin{baitoan}[\cite{An_Hoa_Hoc_nang_cao_8_9}, 44.a, p. 78]
	Viết {\rm PTHH} thực hiện chuyển hóa: {\rm\ce{Cl2} $\to$ A $\to$ B $\to$ C $\to$ A $\to$ \ce{Cl2}}, trong đó $A$ là chất khí, B \& C là hợp chất chứa chlorine.
\end{baitoan}

\begin{baitoan}[\cite{An_Hoa_Hoc_nang_cao_8_9}, 45., p. 78]
	Hòa tan hoàn toàn $a$ {\rm g \ce{R2O3}} cần $b$ {\rm g} dung dịch {\rm\ce{H2SO4} 12.25\%} thì vừa đủ. Sau phản ứng thu được dung dịch muối có nồng độ $15.36\%$. Xác định kim loại R.\hfill{\sf Ans: Cr.}
\end{baitoan}

\begin{baitoan}[\cite{An_Hoa_Hoc_nang_cao_8_9}, 46., p. 79]
	Hòa tan {\rm13.2 g} hỗn hợp X gồm 2 kim loại có cùng hóa trị vào {\rm400 mL} dung dịch {\rm HCl 1.5M}. Cô cạn dung dịch sau phản ứng thu được {\rm32.7 g} hỗn hợp muối khan. Hỗn hợp X có tan hết trong dung dịch {\rm HCl} không?\hfill{\sf Ans: Không tan hết.}
\end{baitoan}

\begin{baitoan}[\cite{An_Hoa_Hoc_nang_cao_8_9}, 47., p. 79]
	Trộn $V_1$ {\rm L} dung dịch {\rm HCl 0.6M} với $V_2$ {\rm L} dung dịch {\rm NaOH 0.4M} thu được {\rm0.6 L} dung dịch A. Tính $V_1,V_2$ biết {\rm0.6 L} dung dịch A có thể hòa tan hết {\rm1.02 g \ce{Al2O3}}. Biết sự pha trộn không làm thay đổi thể tích 1 cách đáng kể.\footnote{Đã học ở Vật lý 8 về sự đan xen của các nguyên tử, phân tử của 2 hay nhiều dung dịch khi trộn vào nhau, xem \cite[\S19, pp. 68--70]{SGK_Vat_Ly_8}.}\hfill{\sf Ans: $(V_1,V_2)\in\{(0.3,0.3),(0.22,0.38)\}$.}
\end{baitoan}

\begin{baitoan}[\cite{An_Hoa_Hoc_nang_cao_8_9}, 48., p. 79]
	Cho {\rm39.6 g} hỗn hợp gồm {\rm\ce{KHSO3, K2CO3}} vào {\rm400 g} dung dịch {\rm HCl 7.3\%}. Sau phản ứng thu được hỗn hợp khí X có tỷ khối hơi so với {\rm\ce{H2}} bằng $25.33$ \& 1 dung dịch Y. (a) Chứng minh acid còn dư. (b) Tính $C\%$ các chất trong dung dịch Y.\hfill{\sf Ans: $8.78\%$, $2.58\%$.}
\end{baitoan}

\begin{baitoan}[\cite{An_Hoa_Hoc_nang_cao_8_9}, 20., p. 135]
	Để khử {\rm6.4 g} 1 oxide kim loại cần {\rm2.688 L} khí {\rm\ce{H2}}. Nếu lấy lượng kim loại đó cho tác dụng với dung dịch {\rm HCl} dư thì giải phóng {\rm1.792 L} khí {\rm\ce{H2}}. Tìm tên kim loại biết thể tích các khí đo ở đktc.
\end{baitoan}

\begin{baitoan}[\cite{An_Hoa_Hoc_nang_cao_8_9}, 21., p. 135]
	Có 4 oxide riêng biệt: {\rm\ce{Na2O,Al2O3,Fe2O3}, MgO}. Làm thế nào để nhận biết mỗi oxide bằng phương pháp hóa học với điều kiện chỉ được dùng thêm 2 chất là {\rm\ce{H2O}} \& dung dịch {\rm HCl}.
\end{baitoan}

\begin{baitoan}[\cite{An_Hoa_Hoc_nang_cao_8_9}, 22., p. 135]
	Cho $a$ {\rm a Fe} hòa tan trong dung dịch {\rm HCl} (thí nghiệm 1). sau khi cô cạn dung dịch thu được {\rm3.1 g} chất rắn. Nếu cho $a$ {\rm g Fe} \& $b$ {\rm g Mg} (thí nghiệm 2) vào dung dịch {\rm HCl} loãng (cùng lượng như trên) thu được {\rm4.48 mL \ce{H2}} \& sau khi cô cạn dung dịch thu được {\rm3.34 g} chất rắn. Tính $a,b$.
\end{baitoan}

\begin{baitoan}[\cite{An_Hoa_Hoc_nang_cao_8_9}, 25., p. 135]
	Cho {\rm31.8 g} hỗn hợp 2 muối {\rm\ce{MgCO3,CaCO3}} vào {\rm0.8 L} dung dịch {\rm HCl 1M} thu được dung dịch Z. (a) Dung dịch Z có dư acid không? (b) Tính $V$ {\rm L \ce{CO2}} sinh ra là bao nhiêu?
\end{baitoan}

%------------------------------------------------------------------------------%

\begin{baitoan}[\cite{An_400_BT_Hoa_Hoc_9}, 1., p. 12]
	(a) Cho rất từ từ dung dịch A chứa $a$ {\rm mol HCl} vào dung dịch B chứa $b$ {\rm mol \ce{Na2CO3}} ($a < 2b$) thì thu được dung dịch C \& $V$ {\rm L} khí. Tính $V$. (b) Nếu cho dung dịch B vào dung dịch A thì được dung dịch D \& $V_1$ {\rm L} khí. Biết các phản ứng xảy ra hoàn toàn, các thể tích khí đo ở đktc. Lập biểu thức nêu mối quan hệ giữa $V_1$ với $a,b$.
\end{baitoan}

\begin{baitoan}[\cite{An_400_BT_Hoa_Hoc_9}, 2., p. 12]
	Cho {\rm31.8 g} hỗn hợp X gồm 2 muối {\rm\ce{MgCO3,CaCO3}} vào {\rm0.8 L} dung dịch {\rm HCl 1M} thu được dung dịch Z. (a) Hỏi dung dịch Z có dư acid không? (b) Lượng {\rm\ce{CO2}} có thể thu được bao nhiêu? (c) Cho vào dung dịch Z 1 lượng dung dịch {\rm\ce{NaHCO3}} dư thì thể tích khí {\rm\ce{CO2}} thu được là {\rm2.24 L} (đktc). Tính khối lượng mỗi muối trong hỗn hợp X.
\end{baitoan}

\begin{baitoan}[\cite{An_400_BT_Hoa_Hoc_9}, 3., p. 12]
	Có 3 bình đựng lần lượt các dung dịch {\rm KOH 1M, 2M, 3M}, mỗi bình chứa {\rm1 L} dung dịch. Trộn lẫn các dung dịch này sao cho dung dịch {\rm KOH 1.8M} thu được có thể tích lớn nhất.
\end{baitoan}

\begin{baitoan}[Mở rộng \cite{An_400_BT_Hoa_Hoc_9}, 3., p. 12]
	Cho $a,b,c,d\in\mathbb{R}$, $a,b,c > 0$. Có 3 bình đựng lần lượt các dung dịch {\rm KOH $a$M, $b$M, $c$M}, mỗi bình chứa {\rm1 L} dung dịch. Biện luận theo $a,b,c,d$ để trộn lẫn các dung dịch này sao cho dung dịch {\rm KOH $d$M} thu được có thể tích lớn nhất.
\end{baitoan}

\begin{baitoan}[\cite{An_400_BT_Hoa_Hoc_9}, 4., p. 12]
	Cho {\rm19.7 g} muối carbonate của kim loại hóa trị {\rm II} tác dụng hết với dung dịch {\rm\ce{H2SO4}} loãng, dư thu được {\rm23.3 g} muối sulfate. Công thức muối carbonate của kim loại hóa trị {\rm II}?
\end{baitoan}

\begin{baitoan}[\cite{An_400_BT_Hoa_Hoc_9}, 5., p. 12]
	Chọn các chất thích hợp \& cân bằng {\rm PTHH}: {\rm(a) \ce{X1 + X2 -> Br2 + MnBr2 + H2O}, (b) \ce{X3 + X4 + X5 -> HCl + H2SO4}, (c) \ce{A_1 + A_2 -> SO2 + H2O}, (d) \ce{B1 + B2 -> NH3 + Ca(NO3)2 + H2O}, (e) \ce{D1 + D2 + D3 -> Cl2 + MnSO4 + K2SO4 + Na2SO4 + H2O}}.
\end{baitoan}

\begin{baitoan}[\cite{An_400_BT_Hoa_Hoc_9}, 6., p. 12]
	Hợp chất A bị phân hủy ở nhiệt độ cao theo {\rm PTPƯ: 2A $\to$ B + 2D + 4E}. Sản phẩm tạo thành đều ở thể khí, khối lượng mol trung bình của hỗn hợp khí sau phản ứng là {\rm22.86 g{\tt/}mol}. Tính khối lượng mol của A.
\end{baitoan}

\begin{baitoan}[\cite{An_400_BT_Hoa_Hoc_9}, 7., p. 13]
	Cho {\rm39.6 g} hỗn hợp gồm {\rm\ce{KHSO3,K2CO3}} vào {\rm400 g} dung dịch {\rm HCl 7.3\%}, khi xong phản ứng thu được hỗn hợp khí X có tỷ khối so với khí hydrogen bằng $25.33$ \& 1 dung dịch A. (a) Chứng minh acid còn dư. (b) Tính $C\%$ các chất trong dung dịch A.
\end{baitoan}

\begin{baitoan}[\cite{An_400_BT_Hoa_Hoc_9}, 8., p. 13]
	Hòa tan {\rm21.5 g} hỗn hợp {\rm\ce{BaCl2,CaCl2}} vào {\rm178.5 mL} nước để được dung dịch A. Thêm vào dung dịch A {\rm175 mL} dung dịch {\rm\ce{Na2CO3} 1M} thấy tách ra {\rm19.85 g} kết tủa \& còn nhận được {\rm400 mL} dung dịch B. Tính nồng độ $\%$ của dung dịch {\rm\ce{BaCl2,CaCl2}}.
\end{baitoan}

\begin{baitoan}[\cite{An_400_BT_Hoa_Hoc_9}, 9.a, p. 13]
	Chỉ được dùng thêm quỳ tím \& các ống nghiệm, chỉ rõ phương pháp nhận ra các dung dịch bị mất nhãn: {\rm\ce{NaHSO4,Na2CO3,Na2SO3,BaCl2,Na2S}}.
\end{baitoan}

\begin{baitoan}[\cite{An_400_BT_Hoa_Hoc_9}, 9.b, p. 13]
	Cho khí {\rm\ce{CO2}} (đktc) phản ứng với {\rm80 g} dung dịch {\rm NaOH 25\%} để tạo thành hỗn hợp muối acid \& muối trung hòa theo tỷ lệ số mol là $2:3$. Tính thể tích {\rm\ce{CO2}} cần dùng.
\end{baitoan}

\begin{baitoan}[\cite{An_400_BT_Hoa_Hoc_9}, 10., p. 13]
	Cho {\rm0.2 mol CuO} tan hết trong dung dịch {\rm\ce{H2SO4} 20\%} đun nóng (lượng vừa đủ). Sau đó làm nguội dung dịch đến $10^\circ${\rm C}. Tính khối lượng tinh thể {\rm\ce{CuSO4.$5$H2O}} đã tách khỏi dung dịch, biết độ tan của {\rm\ce{CuSO4}} ở $10^\circ${\rm C} là {\rm17.4g}.
\end{baitoan}

\begin{baitoan}[\cite{An_400_BT_Hoa_Hoc_9}, 11., p. 13]
	Để có được {\rm200 mL} dung dịch {\rm NaCl 0.1M}. Có thể làm theo cách nào? {\sf A.} Lấy {\rm5.85 g NaCl} hòa tan trong {\rm200 mL} nước cất. {\sf B.} Lấy {\rm5.85 g NaCl} hòa tan trong {\rm194.15 g} nước cất. {\sf C.} Hòa tan {\rm1.17 g NaCl} trong {\rm100 mL} nước cất sau đó bổ sung thêm nước cho đến {\rm200 mL}. {\sf D.} Lấy 1 cốc chia độ, cho nước vào rồi cho {\rm1.17 g NaCl} cho đến lúc đạt thể tích {\rm250 mL}.
\end{baitoan}

\begin{baitoan}[\cite{An_400_BT_Hoa_Hoc_9}, 12., pp. 13--14]
	Đốt cháy hoàn toàn 1 chất vô cơ A trong không khí thì chỉ thu được {\rm1.6 g} iron ({\rm III}) oxide \& {\rm0.896 L} khí sunfurơ (đktc). (a) Xác định {\rm CTPT} của A. (b) Viết {\rm PTHH} để thực hiện chuỗi chuyển hóa: A $\to$ {\rm \ce{SO2}} $\to$ muối $A_1$ $\to$ $A_3$; A $\to$ kết tủa $A_2$.
\end{baitoan}

\begin{baitoan}[\cite{An_400_BT_Hoa_Hoc_9}, 13., p. 14]
	Hòa tan 1 ít {\rm NaCl} vào nước được $V$ {\rm mL} dung dịch A có khối lượng riêng $d$, thêm $V_1$ {\rm mL} nước vào dung dịch A được $(V + V_1)$ mL dung dịch B có khối lượng riêng $d_1$. Chứng minh $d > d_1$. Biết khối lượng riêng của nước là {\rm1 g{\tt/}mL}.
\end{baitoan}

\begin{baitoan}[\cite{An_400_BT_Hoa_Hoc_9}, 14., p. 14]
	Trộn $V_1$ {\rm L} dung dịch {\rm HCl 0.6M} với $V_2$ {\rm L} dung dịch {\rm NaOH 0.4M} thu được {\rm0.6 L} dung dịch A. Tính $V_1,V_2$ biết {\rm0.6 L} dung dịch A có thể hòa tan hết {\rm1.02 g \ce{Al2O3}} (coi sự pha trộn làm thay đổi thể tích không đáng kể).
\end{baitoan}

\begin{baitoan}[\cite{An_400_BT_Hoa_Hoc_9}, 15., p. 14]
	Có 5 dung dịch các chất: {\rm\ce{H2SO4,HCl,NaOH,KCl,BaCl2}}. Trình bày phương pháp phân biệt các dung dịch này mà chỉ dùng quỳ tím làm thuốc thử.
\end{baitoan}

\begin{baitoan}[\cite{An_400_BT_Hoa_Hoc_9}, 16., p. 14]
	Có 2 cốc, cốc A đựng {\rm200 mL} dung dịch chứa {\rm\ce{Na2CO3} 1M} \& {\rm\ce{NaHCO3} 1.5M}. Cốc B đựng {\rm173mL} dung dịch {\rm HCl 7.7\%}, $D = 1.37$ {\rm g{\tt/}mL}. Tiến hành 2 thí nghiệm:
	\begin{itemize}
		\item Thí nghiệm 1: Đổ rất từ từ cốc B vào cốc A.
		\item Thí nghiệm 2: Đổ rất từ từ cốc A vào cốc B.
	\end{itemize}
	Tính thể tích khí (đktc) thoát ra trong mỗi trường hợp sau khi đổ hết cốc này vào cốc kia.
\end{baitoan}

\begin{baitoan}[\cite{An_400_BT_Hoa_Hoc_9}, 17., p. 14]
	Cho $x$ {\rm g} dung dịch {\rm\ce{H2SO4}} loãng nồng độ $C\%$ tác dụng hoàn toàn với hỗn hợp 2 kim loại potassium \& iron (dùng dư), sau phản ứng khối lượng chung đã giảm $0.0469x$ {\rm g}. Tính $C\%$.
\end{baitoan}

\begin{baitoan}[\cite{An_400_BT_Hoa_Hoc_9}, 18., p. 14]
	Hòa tan {\rm450 g} potassium nitrate vào {\rm500 g} nước cất ở $25^\circ${\rm C} (dung dịch X). Biết độ tan của {\rm\ce{KNO3}} ở $20^\circ${\rm C} là {\rm32 g}. Xác định khối lượng potassium nitrate tách ra khỏi dung dịch khi làm lạnh dung dịch X đến $20^\circ${\rm C}.
\end{baitoan}

\begin{baitoan}[\cite{An_400_BT_Hoa_Hoc_9}, 19., pp. 14--15, HSG lớp 8 Tp. HCM 2000--2001]
	Khi cho $a$ {\rm g Fe} vào trong {\rm400 mL} dung dịch {\rm HCl}, sau khi phản ứng kết thúc đem cô cạn dung dịch thu được {\rm6.2 g} chất rắn X. Nếu cho hỗn hợp gồm $a$ {\rm g Fe} \& $b$ {\rm g Mg} vào trong {\rm400 mL} dung dịch {\rm HCl} thì sau khi phản ứng kết thúc, thu được {\rm896 mL \ce{H2}} (đktc) \& cô cạn dung dịch thì thu được {\rm6.68 g} chất rắn Y. Tính $a,b$, nồng độ mol của dung dịch {\rm HCl} \& thành phần khối lượng các chất trong X, Y. (Giả sử {\rm Mg} không phản ứng với nước \& khi phản ứng với acid, {\rm Mg} phản ứng trước, hết {\rm Mg} mới đến {\rm Fe}. Cho biết các phản ứng đều xảy ra hoàn toàn).
\end{baitoan}

\begin{baitoan}[\cite{An_400_BT_Hoa_Hoc_9}, 20., p. 15]
	Khử $a$ {\rm g} 1 iron oxide bằng {\rm CO} nóng, dư đến hoàn toàn thu được {\rm Fe} \& khí A. Hòa tan lượng sắt trên trong dung dịch {\rm\ce{H2SO4}} loãng dư thoát ra {\rm1.68 L \ce{H2}} (đktc). Hấp thụ toàn bộ khí A bằng {\rm\ce{Ca(OH)2}} dư thu được kết tủa. Tìm công thức iron oxide.
\end{baitoan}

\begin{baitoan}[\cite{An_400_BT_Hoa_Hoc_9}, 21., p. 15]
	Nung $m$ {\rm g} hỗn hợp chất rắn A gồm {\rm\ce{Fe2O3}, FeO} với lượng thiếu {\rm CO} thu được hỗn hợp chất rắn B có khối lượng {\rm47.84 g} \& {\rm5.6 L \ce{CO2}} (đktc). Tính $m$.
\end{baitoan}

\begin{baitoan}[\cite{An_400_BT_Hoa_Hoc_9}, 22., p. 15]
	Dung dịch X là dung dịch {\rm\ce{H2SO4}}, dung dịch Y là dung dịch {\rm NaOH}. Nếu trộn X \& Y theo tỷ lệ thể tích là $V_X:V_Y = 3:2$ thì được dung dịch A có chứa X dư. Trung hòa {\rm1 L} A cần {\rm40 g KOH 20\%}. Nếu trộn X \& Y theo tỷ lệ thể tích $V_X:V_Y = 2:3$ thì được dung dịch B có chứa Y dư. Trung hòa {\rm1 L} B cần {\rm29.2 g} dung dịch {\rm HCl 25\%}. Tính nồng độ mol của X \& Y.
\end{baitoan}

\begin{baitoan}[\cite{An_400_BT_Hoa_Hoc_9}, 23., p. 15]
	(a) Bằng phương pháp hóa học, phân biệt 4 muối sau: {\rm\ce{Na2CO3,MgCO3,BaCO3,CaCl2}}. (b) Chọn 2 dung dịch muối thích hợp để phân biệt 4 dung dịch các chất: {\rm\ce{BaCl2,HCl,K2SO4,Na3PO4}}.
\end{baitoan}

\begin{baitoan}[\cite{An_400_BT_Hoa_Hoc_9}, 24., p. 15]
	Đốt cháy hoàn toàn {\rm6.8 g} 1 hợp chất vô cơ A chỉ thu được {\rm4.48 L} khí {\rm\ce{SO2}} (đktc) \& {\rm3.6 g} nước. Tính thể tích khí {\rm\ce{O2}} đã dùng \& xác định {\rm CTPT} của A.
\end{baitoan}

\begin{baitoan}[\cite{An_400_BT_Hoa_Hoc_9}, 25., p. 15]
	Làm thế nào để nhận ra sự có mặt của mỗi khí trong hỗn hợp gồm {\rm CO, \ce{CO2,SO3}} bằng phương pháp hóa học, viết {\rm PTHH}.
\end{baitoan}

\begin{baitoan}[\cite{An_400_BT_Hoa_Hoc_9}, 26., p. 15]
	Hòa tan {\rm NaOH} rắn vào nước để tạo thành 2 dung dịch A \& B với nồng độ $\%$ của dung dịch A gấp $3$ lần nồng độ $\%$ của dung dịch B. Nếu đem trộn 2 dung dịch A \& B theo tỷ lệ khối lượng $m_A:m_B = 5:2$ thì thu được dung dịch C có nồng độ $\%$ là $20\%$. Xác định nồng độ $\%$ của 2 dung dịch A \& B.
\end{baitoan}

\begin{baitoan}[\cite{An_400_BT_Hoa_Hoc_9}, 27., pp. 15--16]
	Hỏi có bao nhiêu {\rm g NaCl} kết tinh khi làm lạnh {\rm600 g} dung dịch {\rm NaCl} bão hòa ở $90^\circ${\rm C}. Biết độ tan của {\rm NaCl} ở $90^\circ${\rm C} là {\rm50 g} \& ở $0^\circ${\rm C} là {\rm35 g}.
\end{baitoan}

\begin{baitoan}[\cite{An_400_BT_Hoa_Hoc_9}, 28., p. 16]
	Nêu phương pháp tách hỗn hợp gồm 3 khí {\rm\ce{Cl2,H2,CO2}} thành các chất nguyên chất.
\end{baitoan}

\begin{baitoan}[\cite{An_400_BT_Hoa_Hoc_9}, 29., p. 16]
	Tinh chế các chất khí: (a) {\rm\ce{O2}} có lẫn {\rm\ce{Cl2,CO2,SO2}}. (b) {\rm\ce{Cl2}} có lẫn {\rm\ce{O2,CO2,SO2}}. (c) {\rm\ce{CO2}} có lẫn khí {\rm HCl} \& hơi nước.
\end{baitoan}

\begin{baitoan}[\cite{An_400_BT_Hoa_Hoc_9}, 30., p. 16]
	Oxide của 1 kim loại hóa trị {\rm III} có khối lượng {\rm32 g} tan hết trong {\rm294 g} dung dịch {\rm\ce{H2SO4} 20\%}. Tìm {\rm CTPT} của oxide kim loại đó.
\end{baitoan}

\begin{baitoan}[\cite{An_400_BT_Hoa_Hoc_9}, 31., p. 16]
	Cho {\rm19.6 g} acid phosphoric tác dụng với {\rm200 g} dung dịch potassium hydroxide có nồng độ $8.4\%$. Thu được các muối nào sau phản ứng? Tính khối lượng của mỗi muối.
\end{baitoan}

\begin{baitoan}[\cite{An_400_BT_Hoa_Hoc_9}, 32., p. 16]
	Phân bón A có chứa $80\%$ ammonium nitrate. Phân bón B có chứa $82\%$ calcium nitrate. Nếu cần {\rm56 kg} nitrogen để bón ruộng thì nên mua loại phân nào? Vì sao?
\end{baitoan}

\begin{baitoan}[\cite{An_400_BT_Hoa_Hoc_9}, 33., p. 16]
	Nêu phương pháp tách hỗn hợp đá vôi, vôi sống, thạch cao, \& muối ăn thành từng chất nguyên chất.
\end{baitoan}

\begin{baitoan}[\cite{An_400_BT_Hoa_Hoc_9}, 34., p. 16]
	Nêu phương pháp tách hỗn hợp đá vôi, silicon dioxide, \& iron ({\rm II}) chloride thành từng chất nguyên chất.
\end{baitoan}

\begin{baitoan}[\cite{An_400_BT_Hoa_Hoc_9}, 35., p. 16]
	Nêu phương pháp tách hỗn hợp 3 khí {\rm\ce{O2,H2,SO2}} thành các chất nguyên chất.
\end{baitoan}

\begin{baitoan}[\cite{An_400_BT_Hoa_Hoc_9}, 36., p. 16]
	Nêu phương pháp tinh chế {\rm Cu} trong quặng {\rm Cu} có lẫn {\rm Fe, S}, \& {\rm Ag}.
\end{baitoan}

\begin{baitoan}[\cite{An_400_BT_Hoa_Hoc_9}, 37., p. 16]
	Cần thêm bao nhiêu {\rm g \ce{SO3}} vào dung dịch {\rm\ce{H2SO4} 10\%} để được {\rm100 g} dung dịch {\rm\ce{H2SO4} 20\%}?
\end{baitoan}

\begin{baitoan}[Mở rộng \cite{An_400_BT_Hoa_Hoc_9}, 37., p. 16]
	Cần thêm bao nhiêu {\rm g \ce{SO3}} vào dung dịch {\rm\ce{H2SO4} $a\%$} để được {\rm100 g} dung dịch {\rm\ce{H2SO4} $b\%$}, với $a,b\in\mathbb{R}$, $a,b > 0$?
\end{baitoan}

\begin{baitoan}[\cite{An_400_BT_Hoa_Hoc_9}, 38., p. 16]
	Phải hòa tan thêm bao nhiêu {\rm g} potassium hydroxide nguyên chất vào {\rm1200 g} dung dịch {\rm KOH 12\%} để có dung dịch {\rm KOH 20\%}?
\end{baitoan}

\begin{baitoan}[\cite{An_400_BT_Hoa_Hoc_9}, 39., p. 16]
	Cần phải dùng bao nhiêu {\rm L \ce{H2SO4}} có tỷ khối $d = 1.84$ \& bao nhiêu {\rm L} nước cất để pha thành {\rm10 L} dung dịch {\rm\ce{H2SO4}} có $d = 1.28$?
\end{baitoan}

\begin{baitoan}[\cite{An_400_BT_Hoa_Hoc_9}, 40.a, p. 16]
	(a) Trộn {\rm2 L} dung dịch {\rm HCl 4M} vào {\rm 1 L} dung dịch {\rm HCl 0.5 M}. Tính nồng độ mol của dung dịch mới.
\end{baitoan}

\begin{baitoan}[Mở rộng \cite{An_400_BT_Hoa_Hoc_9}, 40., p. 16]
	(a) Trộn $V_1$ {\rm L} dung dịch {\rm HCl $a$M} vào $V_2$ {\rm L} dung dịch {\rm HCl $b$M}. Tính nồng độ mol của dung dịch mới.
\end{baitoan}

\begin{baitoan}[\cite{An_400_BT_Hoa_Hoc_9}, 40.b, p. 16]
	Trộn {\rm150 g} dung dịch {\rm NaOH 10\%} vào {\rm460 g} dung dịch {\rm NaOH $x$\%} để tạo thành dung dịch $6\%$. Tính $x$.
\end{baitoan}

\begin{baitoan}[\cite{An_400_BT_Hoa_Hoc_9}, 41., p. 16]
	Cần lấy bao nhiêu {\rm mL} dung dịch {\rm HCl} có nồng độ $36\%$, $d = 1.19$, để pha thành {\rm5 L} dung dịch acid {\rm HCl} có nồng độ {\rm0.5M}.
\end{baitoan}

\begin{baitoan}[\cite{An_400_BT_Hoa_Hoc_9}, 42., p. 17]
	Cho {\rm100 g} dung dịch {\rm\ce{H2SO4} 19.6\%} vào {\rm400 g} dung dịch {\rm\ce{BaCl2} 13\%}. (a) Tính khối lượng kết tủa. (b) Tính nồng độ $\%$ các chất có trong dung dịch sau phản ứng.
\end{baitoan}

\begin{baitoan}[\cite{An_400_BT_Hoa_Hoc_9}, 43., p. 17]
	Hòa tan {\rm8.96 L} khí {\rm HCl} (đktc) vào {\rm185.4 g} nước được dung dịch M. Lấy {\rm50 g} dung dịch M cho tác dụng với {\rm85 g} dung dịch {\rm\ce{AgNO3} 16\%} thì thu được dung dịch N \& 1 chất kết tủa.
\end{baitoan}

\begin{baitoan}[\cite{An_400_BT_Hoa_Hoc_9}, 44., p. 17]
	Cho {\rm11.6 g} hỗn hợp {\rm\ce{Fe2O3}, FeO} có tỷ lệ số mol là $1:1$ vào {\rm300 mL} dung dịch {\rm HCl 2M} được dung dịch A. (a) Tính nồng độ mol của các chất trong dung dịch sau phản ứng (thể tích dung dịch thay đổi không đáng kể). (b) Tính thể tích dung dịch {\rm NaOH 1.5M} đủ để tác dụng hết với dung dịch A.
\end{baitoan}

\begin{baitoan}[\cite{An_400_BT_Hoa_Hoc_9}, 45., p. 17]
	Cho sản phẩm thu được khi oxy hóa hoàn toàn {\rm5.6 L} khí sunfurơ (đktc) vào trong {\rm57.2 mL} dung dịch {\rm\ce{H2SO4} 60\%} có $D = 1.5$ {\rm g{\tt/}mL}. Tính nồng độ $\%$ của dung dịch acid thu được.
\end{baitoan}

\begin{baitoan}[\cite{An_400_BT_Hoa_Hoc_9}, 46., p. 17]
	Cho {\rm200 g} dung dịch {\rm\ce{BaCl2} 5.2\%} tác dụng với {\rm58.8 g} dung dịch {\rm\ce{H2SO4} 20\%}. Tính nồng độ $\%$ của các chất có trong dung dịch.
\end{baitoan}

\begin{baitoan}[\cite{An_400_BT_Hoa_Hoc_9}, 47.a, p. 17]
	Tính tỷ lệ thể tích của 2 dung dịch {\rm HCl 0.2M \& 1M} để trộn thành dung dịch {\rm HCl 0.4M}.
\end{baitoan}

\begin{baitoan}[\cite{An_400_BT_Hoa_Hoc_9}, 47.b, p. 17]
	Tính khối lượng {\rm\ce{Na2O}} \& khối lượng nước cần để có được {\rm200 g} dung dịch {\rm NaOH 10\%}.
\end{baitoan}

\begin{baitoan}[\cite{An_400_BT_Hoa_Hoc_9}, 48., p. 17]
	1 loại đá chứa {\rm80\% \ce{CaCO3}}, phần còn lại là tạp chất trơ. Nung đá vôi trên tới phản ứng hoàn toàn. Hỏi khối lượng của chất rắn thu được sau khi nung bằng bao nhiêu $\%$ khối lượng đá trước khi nung \& tính {\rm\% CaO} trong chất rắn sau khi nung.
\end{baitoan}

\begin{baitoan}[\cite{An_400_BT_Hoa_Hoc_9}, 49., p. 17]
	Khi nung hỗn hợp {\rm\ce{CaCO3,MgCO3}} thì khối lượng chất rắn thu được sau phản ứng chỉ bằng $\frac{1}{2}$ khối lượng ban đầu. Xác định thành phần $\%$ khối lượng các chất trong hỗn hợp ban đầu.
\end{baitoan}

\begin{baitoan}[\cite{An_400_BT_Hoa_Hoc_9}, 50., p. 17]
	Trong quặng bôxit trung bình có $50\%$ aluminium oxide. Kim loại luyện được từ oxide đó còn chứa $1.5\%$ tạp chất. Tính lượng aluminium nguyên chất điều chế được  từ $0.5$ tấn quặng boxit.
\end{baitoan}

\begin{baitoan}[\cite{An_400_BT_Hoa_Hoc_9}, 51., pp. 17--18]
	Đốt cháy hỗn hợp {\rm CuO, FeO} với {\rm C} có dư thì được chất rắn A \& khí B. Cho B tác dụng với nước vôi trong có dư thu được {\rm8 g} kết tủa. Chất rắn A cho tác dụng với dung dịch {\rm HCl} có nồng độ $10\%$ thì cần dùng 1 lượng acid là {\rm73 g} sẽ vừa đủ. (a) Viết {\rm PTHH}. (b) Tính khối lượng {\rm CuO, FeO} trong hỗn hợp ban đầu \& thể tích khí B (đktc).
\end{baitoan}

\begin{baitoan}[\cite{An_400_BT_Hoa_Hoc_9}, 52., p. 18]
	Khi phân hủy bằng nhiệt {\rm14.2 g} hỗn hợp {\rm\ce{CaCO3,MgCO3}}, thu được {\rm6.6 g \ce{CO2}} (đktc). Tính thành phần $\%$ các chất trong hỗn hợp.
\end{baitoan}

\begin{baitoan}[\cite{An_400_BT_Hoa_Hoc_9}, 53., p. 18]
	Cho {\rm38.2 g} hỗn hợp {\rm\ce{Na2CO3,K2CO3}} vào dung dịch {\rm HCl}. Dẫn lượng khí sinh ra qua nước vôi trong có dư thu được {\rm30 g} kết tủa. Tính khối lượng mỗi muối trong hỗn hợp ban đầu.
\end{baitoan}

\begin{baitoan}[\cite{An_400_BT_Hoa_Hoc_9}, 54., p. 18]
	Cho {\rm0.325 g} hỗn hợp gồm {\rm NaCl, KCl} được hòa tan vào nước. Sau đó cho dung dịch {\rm\ce{AgNO3}} vào dung dịch trên, ta được 1 kết tủa; sấy kết tủa đến khối lượng không đổi thấy cân nặng {\rm0.717 g}. Tính thành phần $\%$ các chất trong hỗn hợp.
\end{baitoan}

\begin{baitoan}[\cite{An_400_BT_Hoa_Hoc_9}, 55., p. 18]
	{\rm\ce{Al4C3,CaC2}} tác dụng với nước theo {\rm PTHH: \ce{Al4C3 + $12$H2O -> $4$Al(OH)3 + $3$CH4, CaC2 + $2$H2O -> Ca(OH)2 + C2H2}}. Cho hỗn hợp 2 chất trên tác dụng với nước dư thu được {\rm2.016 L} hỗn hợp khí. Lấy hỗn hợp này đốt cháy hoàn toàn thu được {\rm2.688 L \ce{CO2}}. Các thể tích đều đo ở đktc. Tính lượng {\rm\ce{Al4C3,CaC2}} trong hỗn hợp.
\end{baitoan}

\begin{baitoan}[\cite{An_400_BT_Hoa_Hoc_9}, 56., p. 18]
	Dùng thuốc thử thích hợp, nhận biết các dung dịch sau đã mất nhãn: {\rm(a) \ce{NaCl,NaBr,KI,HCl,H2SO4,KOH}. (b) \ce{Na2SO4,H2SO4,NaOH,KCl,NaNO3}}.
\end{baitoan}

\begin{baitoan}[\cite{An_400_BT_Hoa_Hoc_9}, 57., p. 18]
	Dùng thuốc thử thích hợp để nhận biết các dung dịch: {\rm(a) KI, HCl, NaCl, \ce{H2SO4}. (b) HCl, HBr, NaCl, NaOH. (c) NaF, \ce{CaCl2}, KBr, \ce{MgI2}}.
\end{baitoan}

\begin{baitoan}[\cite{An_400_BT_Hoa_Hoc_9}, 58., p. 18]
	Chỉ dùng thêm 1 hóa chất, phân biệt các lọ mất nhãn: {\rm(a) \ce{MgCl2}, KBr, NaI, \ce{AgNO3,NH4HCO3}. (b) NaBr, \ce{ZnSO4, Na2CO3, AgNO3, BaCl2}}.
\end{baitoan}

\begin{baitoan}[\cite{An_400_BT_Hoa_Hoc_9}, 59., p. 18]
	Chỉ dùng thêm 1 hóa chất, phân biệt các dung dịch: {\rm(a) \ce{K2SO4,K2CO3,K2SiO3,K2S,K2SO3}. (b) \ce{MgCl2}, NaBr, \ce{Ca(NO3)2}}.
\end{baitoan}

\begin{baitoan}[\cite{An_400_BT_Hoa_Hoc_9}, 60., p. 18]
	Có 5 lọ, mỗi lọ đựng 1 trong các hóa chất: {\rm FeO, CuO, \ce{Fe3O4,Ag2O,MnO2}}. Dùng phương pháp hóa học để nhận biết từng hóa chất trong mỗi lọ.
\end{baitoan}

\begin{baitoan}[\cite{An_400_BT_Hoa_Hoc_9}, 61.a, p. 19]
	Chỉ có nước \& khí carbonic có thể phân biệt được 5 chất bột trắng sau hay không? Nếu được, trình bày cách phân biệt: {\rm NaCl, \ce{Na2SO4,BaCO3,Na2CO3,BaSO4}}? 
\end{baitoan}

\begin{baitoan}[\cite{An_400_BT_Hoa_Hoc_9}, 61.b, p. 19]
	Trình bày các nguyên tắc tiến hành phân biệt 4 chất: {\rm\ce{BaSO4,BaCO3,NaCl,Na2CO3}} với điều kiện chỉ dùng thêm {\rm HCl} loãng.
\end{baitoan}

\begin{baitoan}[\cite{An_400_BT_Hoa_Hoc_9}, 62.a, p. 19]
	Nêu cách nhận biết {\rm CaO, \ce{Na2O}, MgO, \ce{P2O5}} đều là chất bột trắng.
\end{baitoan}

\begin{baitoan}[\cite{An_400_BT_Hoa_Hoc_9}, 62.b, p. 19]
	Bằng phương pháp hóa học, nhận biết 4 kim loại có màu trắng bạc {\rm Al, Ag, Fe, Mg}.
\end{baitoan}

\begin{baitoan}[\cite{An_400_BT_Hoa_Hoc_9}, 63.a, p. 19]
	Từ các nguyên liệu chính là {\rm\ce{CO2,NaCl,NH4Cl}}, viết các phương trình phản ứng điều chế {\rm\ce{NH4HCO3}} tinh khiết.
\end{baitoan}

\begin{baitoan}[\cite{An_400_BT_Hoa_Hoc_9}, 63.b, p. 19]
	Điều chế 3 oxide, 2 acid, \& 2 muối từ các hóa chất: {\rm Mg, \ce{H2O}}, không khí, \& {\rm S}. Viết {\rm PTHH}.
\end{baitoan}

\begin{baitoan}[\cite{An_400_BT_Hoa_Hoc_9}, 64., p. 19]
	Chỉ từ {\rm Cu, NaCl, \ce{H2O}}, nêu cách điều chế để thu được {\rm\ce{Cu(OH)2}}. Viết {\rm PTHH}.
\end{baitoan}

\begin{baitoan}[\cite{An_400_BT_Hoa_Hoc_9}, 65.a, p. 19]
	Cho các chất: Aluminium, oxygen, nước, copper sulfate, iron, acid hydrochloric. Điều chế copper, copper oxide, aluminium chloride (bằng 2 phương pháp) \& iron ({\rm II}) chloride. Viết {\rm PTHH}.
\end{baitoan}

\begin{baitoan}[\cite{An_400_BT_Hoa_Hoc_9}, 65.b, p. 19]
	Bằng cách nào từ iron ta có thể điều chế iron ({\rm II}) hydroxide, iron ({\rm III}) hydroxide? Viết {\rm PTHH}.
\end{baitoan}

\begin{baitoan}[\cite{An_400_BT_Hoa_Hoc_9}, 66., p. 19]
	Chỉ từ quặng pirit {\rm\ce{FeS2,O2,H2O}}, có chất xúc tác thích hợp. Viết {\rm PTPƯ} điều chế muối iron ({\rm III}) sulfate.
\end{baitoan}

\begin{baitoan}[\cite{An_400_BT_Hoa_Hoc_9}, 67., p. 19]
	Viết các {\rm PTPƯ} phản ứng điều chế trực tiếp: (a) {\rm Cu $\to$ \ce{CuCl2} bằng 3 phương pháp}. (b) {\rm\ce{CuCl2} $\to$ Cu} bằng 2 phương pháp. (c) {\rm Fe $\to$ \ce{FeCl3}} bằng 2 phương pháp.
\end{baitoan}

\begin{baitoan}[\cite{An_400_BT_Hoa_Hoc_9}, 68.a, p. 19]
	Chỉ từ các chất {\rm\ce{KMnO4,BaCl2,H2SO4}, Fe} có thể điều chế được các khí gì?
\end{baitoan}

\begin{baitoan}[\cite{An_400_BT_Hoa_Hoc_9}, 68.b, p. 19]
	Muốn điều chế 3 chất rắn: {\rm NaOH, \ce{NaHCO3,Na2CO3}}. (a) Trình bày 3 phương pháp điều chế mỗi chất. (b) Chỉ dùng 1 thuốc thử, nhận biết từng dung dịch các chất trên.
\end{baitoan}

\begin{baitoan}[\cite{An_400_BT_Hoa_Hoc_9}, 69.a, pp. 19--20]
	Khí nitrogen bị lẫn các tạp chất {\rm CO, \ce{CO2,H2}}, \& hơi nước. Làm thế nào để thu được {\rm\ce{N2}} tinh khiết.
\end{baitoan}

\begin{baitoan}[\cite{An_400_BT_Hoa_Hoc_9}, 69.b, p. 20]
	Khi đốt cháy than, thu được hỗn hợp khí {\rm CO, \ce{CO2}}. Trình bày phương pháp hóa học để thu được từng khí.
\end{baitoan}

\begin{baitoan}[\cite{An_400_BT_Hoa_Hoc_9}, 70., p. 20]
	Nêu phương pháp hóa học để làm sạch các khí: (a) Methane có lẫn khí acetylen. (b) Ethylen có lẫn khí carbonic.
\end{baitoan}

\begin{baitoan}[\cite{An_400_BT_Hoa_Hoc_9}, 72., p. 20]
	Nêu phương pháp tách các hỗn hợp sau thành các chất nguyên chất: (a) Hỗn hợp khí gồm: {\rm\ce{Cl2,H2,CO2}}. (b) Hỗn hợp khí gồm: {\rm\ce{SO2,CO2}, CO}. (c) Hỗn hợp khí gồm: {\rm\ce{SO2,O2}, HCl}.
\end{baitoan}

\begin{baitoan}[\cite{An_400_BT_Hoa_Hoc_9}, 73., p. 20]
	Tinh chế: (a) {\rm\ce{CaSO3}} có lẫn {\rm\ce{CaCO3,Na2CO3}}. (b) Muối ăn có lẫn {\rm\ce{CaCl2,CaSO4,Na2SO3}}. (c) {\rm Cu} có lẫn {\rm Fe, Ag, S}.
\end{baitoan}

\begin{baitoan}[\cite{An_400_BT_Hoa_Hoc_9}, 74.a, p. 20]
	Trình bày phương pháp hóa học để lấy từng oxide từ hỗn hợp: {\rm\ce{SiO2,Al2O3,Fe2O3}, CuO}.
\end{baitoan}

\begin{baitoan}[\cite{An_400_BT_Hoa_Hoc_9}, 74.b, p. 20]
	Trình bày phương pháp lấy từng kim loại {\rm Cu, Fe} từ hỗn hợp các oxide: {\rm\ce{SiO2,Al2O3}, CuO, FeO}.
\end{baitoan}

\begin{baitoan}[\cite{An_400_BT_Hoa_Hoc_9}, 74.c, p. 20]
	Bằng phương pháp hóa học, tách từng kim loại ra khỏi hỗn hợp gồm {\rm Al, Fe, Ag, Cu}.
\end{baitoan}

\begin{baitoan}[\cite{An_400_BT_Hoa_Hoc_9}, 75., p. 20]
	Nêu cách tách các chất ra khỏi hỗn hợp: (a) {\rm\ce{Cl2} có lẫn \ce{N2,H2}}. (b) {\rm\ce{Cl2} có lẫn \ce{CO2}}.
\end{baitoan}

\begin{baitoan}[\cite{An_400_BT_Hoa_Hoc_9}, 76., p. 20]
	Nêu cách tinh chế: (a) Muối ăn có lẫn {\rm\ce{MgCl2}} \& {\rm NaBr}. (b) Acid hydrochloric có lẫn acid {\rm\ce{H2SO4}}.
\end{baitoan}

\begin{baitoan}[\cite{An_400_BT_Hoa_Hoc_9}, 77., p. 20]
	1 loại muối ăn có lẫn tạp chất {\rm\ce{CaCl2,MgCl2,Na2SO4,MgSO4,CaSO4}}. Trình bày cách loại các tạp chất để thu được muối ăn tinh khiết.
\end{baitoan}

\begin{baitoan}[\cite{An_400_BT_Hoa_Hoc_9}, 78., p. 20]
	Tìm cách tách lấy từng muối trong hỗn hợp rắn gồm: ammonium chloride, barium chloride, magnesium chloride. Viết {\rm PTHH}.
\end{baitoan}

\begin{baitoan}[\cite{Nguyen_Buu_Can_500_BT_Hoa_Hoc_THCS}, 201., p. 97]
	Định nghĩa \& phân loại oxide.
\end{baitoan}

\begin{baitoan}[\cite{Nguyen_Buu_Can_500_BT_Hoa_Hoc_THCS}, 202., p. 97]
	Phân loại oxide acid, oxide base, oxide lưỡng tính: {\rm FeO, ZnO, \ce{Al2O3,CaO,Mn2O7,P2O5,N2O5,SiO2}}.
\end{baitoan}

\begin{baitoan}[\cite{Nguyen_Buu_Can_500_BT_Hoa_Hoc_THCS}, 203., p. 97]
	Cho các oxide: {\rm CaO, \ce{SiO2,Fe2O3,Fe3O4,P2O5}}. Chất nào tan trong nước, chất nào tan trong dung dịch kiềm? Chất nào dùng để hút ẩm? Viết {\rm PTHH}.
\end{baitoan}

\begin{baitoan}[\cite{Nguyen_Buu_Can_500_BT_Hoa_Hoc_THCS}, 204., p. 97]
	Trình bày tính chất của calcium oxide.
\end{baitoan}

\begin{baitoan}[\cite{Nguyen_Buu_Can_500_BT_Hoa_Hoc_THCS}, 205., p. 97]
	Để calcium oxide (vôi sống) lâu ngày trong không khí sẽ bị kém phẩm chất. Giải thích hiện tượng \& viết {\rm PTHH}.
\end{baitoan}

\begin{baitoan}[\cite{Nguyen_Buu_Can_500_BT_Hoa_Hoc_THCS}, 206., p. 97]
	Có 3 lọ đựng chất bột màu trắng: {\rm MgO, \ce{Na2O,P2O5}}. Nêu phương pháp thực nghiệm để nhận biết 3 chất \& viết {\rm PTHH}.
\end{baitoan}

\begin{baitoan}[\cite{Nguyen_Buu_Can_500_BT_Hoa_Hoc_THCS}, 207., p. 97]
	Có hỗn hợp 2 chất rắn là {\rm CaO, \ce{Fe2O3}}. Bằng phương pháp hóa học nào có thể tách riêng được {\rm\ce{Fe2O3}}? Viết {\rm PTHH}.
\end{baitoan}

\begin{baitoan}[\cite{Nguyen_Buu_Can_500_BT_Hoa_Hoc_THCS}, 208., p. 98]
	Viết {\rm PTHH} thực hiện các biến hóa hóa học: {\rm(a) CaO $\to$ \ce{Ca(OH)2} $\to$ \ce{CaCO3} $\to$ CaO. (b) CaO $\to$ \ce{CaCO3}. (c) CaO $\to$ \ce{Ca(NO3)2}}.
\end{baitoan}

\begin{baitoan}[\cite{Nguyen_Buu_Can_500_BT_Hoa_Hoc_THCS}, 209., p. 98]
	Hoàn thành các chuỗi biến hóa: {\rm(a) Cu $\to$ CuO $\to$ \ce{CuCl2} $\to$ \ce{Cu(OH)2} $\to$ CuO $\to$ Cu. (b) P $\to$ \ce{P2O5} $\to$ \ce{H3PO4} $\to$ \ce{NaH2PO4} $\to$ \ce{Na2HPO4} $\to$ \ce{Na3PO4}}.
\end{baitoan}

\begin{baitoan}[\cite{Nguyen_Buu_Can_500_BT_Hoa_Hoc_THCS}, 210., p. 98]
	Hoàn thành chuỗi biến hóa: Carbon $\to$ carbon ({\rm IV}) oxide $\to$ calcium carbonate $\to$ calcium bicarbonate $\to$ đá vôi $\to$ vôi sống $\to$ vôi tôi.
\end{baitoan}

\begin{baitoan}[\cite{Nguyen_Buu_Can_500_BT_Hoa_Hoc_THCS}, 211., p. 98]
	Có hỗn hợp khí gồm {\rm\ce{CO2,O2}}. Làm thế nào có thể thu được khí {\rm\ce{O2}} tinh khiết từ hỗn hợp trên? Trình bày cách làm \& viết {\rm PTHH}.
\end{baitoan}

\begin{baitoan}[\cite{Nguyen_Buu_Can_500_BT_Hoa_Hoc_THCS}, 212., p. 98]
	Có 4 gói bột oxide màu đen tương tự nhau: {\rm CuO, AgO, FeO, \ce{MnO2}}. Chỉ dùng dung dịch {\rm HCl} có thể nhận biết được các oxide nào?
\end{baitoan}

\begin{baitoan}[\cite{Nguyen_Buu_Can_500_BT_Hoa_Hoc_THCS}, 213., p. 98]
	Có 3 chất: {\rm Mg, Al, \ce{Al2O3}}. Chỉ được dùng 1 hóa chất làm thuốc thử phân biệt 3 chất trên. Viết {\rm PTHH}.
\end{baitoan}

\begin{baitoan}[\cite{Nguyen_Buu_Can_500_BT_Hoa_Hoc_THCS}, 214., p. 98]
	Có 3 lọ đựng chất bột màu trắng: {\rm MgO, \ce{Na2O,P2O5}}. Nêu phương pháp thực nghiệm để nhận biết 3 chất \& viết {\rm PTHH}.
\end{baitoan}

\begin{baitoan}[\cite{Nguyen_Buu_Can_500_BT_Hoa_Hoc_THCS}, 215., p. 98]
	Định nghĩa \& phân loại acid? (a) Nêu phương pháp chính để điều chế acid. Cho các ví dụ minh họa. (b) Viết $4$ phản ứng thông thường tạo thành các acid {\rm HCl, \ce{H2SO4,H3PO4,HNO3}}.
\end{baitoan}

\begin{baitoan}[\cite{Nguyen_Buu_Can_500_BT_Hoa_Hoc_THCS}, 216., p. 99]
	Trình bày tính chất hóa học của acid sulfuric.
\end{baitoan}

\begin{baitoan}[\cite{Nguyen_Buu_Can_500_BT_Hoa_Hoc_THCS}, 217., p. 99]
	Khi cho khí carbonic vào nước có nhuộm quỳ tím thì nước chuyển sang màu đỏ, khi đun nóng thì màu nước lại chuyển thành màu tím. Giải thích hiện tượng.
\end{baitoan}

\begin{baitoan}[\cite{Nguyen_Buu_Can_500_BT_Hoa_Hoc_THCS}, 218., p. 99]
	Base là gì? Kiềm là gì? Kể tên các base là kiềm. Nêu cách gọi tên base. Các base: {\rm NaOH}, dung dịch {\rm\ce{Ca(OH)2}, KOH} có tên riêng gì?
\end{baitoan}

\begin{baitoan}[\cite{Nguyen_Buu_Can_500_BT_Hoa_Hoc_THCS}, 219., p. 99]
	Cho biết aluminium hydroxide là hợp chất lưỡng t ính, viết các phương trình phản ứng của aluminium hydroxide với các dung dịch {\rm HCl, NaOH}.
\end{baitoan}

\begin{baitoan}[\cite{Nguyen_Buu_Can_500_BT_Hoa_Hoc_THCS}, 220., p. 99]
	(a) Phản ứng nào đặc trưng cho oxide base, phản ứng nào chỉ đặc trưng cho oxide base kiềm? (b) Phản ứng nào đặc trưng cho mọi base? Phản ứng nào đặc trưng cho kiềm?
\end{baitoan}

\begin{baitoan}[\cite{Nguyen_Buu_Can_500_BT_Hoa_Hoc_THCS}, 221., p. 99]
	Trình bày tính chất hóa học của sodium hydroxide.
\end{baitoan}

\begin{baitoan}[\cite{Nguyen_Buu_Can_500_BT_Hoa_Hoc_THCS}, 222., p. 99]
	Làm thế nào để điều chế được calcium hydroxide từ calcium oxide? Phương pháp này có thể áp dụng để điều chế copper ({\rm II}) hydroxide từ copper ({\rm II}) oxide được không? Vì sao?
\end{baitoan}

\begin{baitoan}[\cite{Nguyen_Buu_Can_500_BT_Hoa_Hoc_THCS}, 223., p. 99]
	Định nghĩa \& phân loại muối.
\end{baitoan}

\begin{baitoan}[\cite{Nguyen_Buu_Can_500_BT_Hoa_Hoc_THCS}, 224., p. 99]
	Muối X vừa tác dụng được với dung dịch {\rm HCl}, vừa tác dụng được với dung dịch {\rm NaOH}. Hỏi muối X thuộc loại muối trung hòa hay muối acid? Cho ví dụ minh họa.
\end{baitoan}

\begin{baitoan}[\cite{Nguyen_Buu_Can_500_BT_Hoa_Hoc_THCS}, 225., p. 99]
	Định nghĩa phản ứng trao đổi. Điều kiện để phản ứng trao đổi xảy ra, cho ví dụ minh họa. Phản ứng trung hòa có phải là phản ứng trao đổi không?
\end{baitoan}

\begin{baitoan}[\cite{Nguyen_Buu_Can_500_BT_Hoa_Hoc_THCS}, 226., p. 99]
	Khí {\rm\ce{CO2}} được điều chế bằng phản ứng giữa acid {\rm HCl \& \ce{CaCO3}} có lẫn hơi nước \& khí hydro chloride {\rm HCl}. Làm thế nào để thu được {\rm\ce{CO2}} tinh khiết?
\end{baitoan}

\begin{baitoan}[\cite{Nguyen_Buu_Can_500_BT_Hoa_Hoc_THCS}, 227., p. 100]
	Hoàn thành {\rm PTHH}: {\rm(a) \ce{H2SO4 + Ba(NO3)2}. (b) \ce{HCl + AgNO3}. (c) \ce{HNO3 + CaCO3}. (d) \ce{CuCl2 + KOH}. (e) \ce{FeSO4 + NaOH}. (f) \ce{Ba(NO3)2 + Na2SO4}. (g) \ce{MgSO4 + BaCl2}. (h) \ce{FeCl3 + NaOH}}. Giải thích tại sao phản ứng lại xảy ra.
\end{baitoan}

\begin{baitoan}[\cite{Nguyen_Buu_Can_500_BT_Hoa_Hoc_THCS}, 228., p. 100]
	Cho biết trong dung dịch có thể đồng thời tồn tại các chất sau được không? {\rm(a) NaOH, HBr. (b) \ce{H2SO4,BaCl2}. (c) KCl, \ce{NaNO3}. (d) \ce{Ca(OH)2,H2SO4}. (e) NaCl, KOH}.
\end{baitoan}

\begin{baitoan}[\cite{Nguyen_Buu_Can_500_BT_Hoa_Hoc_THCS}, 229., p. 100]
	Bổ túc \& cân bằng {\rm PTHH}: {\rm(a) NaCl $\to$ \ce{PbCl2 v}. (b) \ce{Fe(SO4)3 -> Fe(OH)3}. (c) HCl $\to$ \ce{CO2 ^}. (d) \ce{CO2 -> CaCO3 v}. (e) \ce{Ba(OH)2 -> BaSO4 v}. (f) \ce{Cu(NO3)2 -> Cu(OH)2 v}. (g) \ce{H2SO4 -> SO2 ^}}.
\end{baitoan}

\begin{baitoan}[\cite{Nguyen_Buu_Can_500_BT_Hoa_Hoc_THCS}, 230., p. 100]
	Cho biết trong dung dịch đồng thời có thể tồn tại các chất sau được không? {\rm(a) KCl, \ce{NaNO3}. (b) KOH, HCl. (c) \ce{Na3PO4,CaCl2}. (d) HBr, \ce{AgNO3}}.
\end{baitoan}

\begin{baitoan}[\cite{Nguyen_Buu_Can_500_BT_Hoa_Hoc_THCS}, 231., p. 100]
	Có 4 chất rắn: đá vôi, soda, muối ăn, potassium sulfate. Làm cách nào để phân biệt chúng khi chỉ được dùng nước \& 1 hóa chất? Viết {\rm PTHH}.
\end{baitoan}

\begin{baitoan}[\cite{Nguyen_Buu_Can_500_BT_Hoa_Hoc_THCS}, 232., p. 101]
	Có 3 ống nghiệm đựng 3 chất lỏng trong suốt, không màu là 3 dung dịch {\rm NaCl, HCl, \ce{Na2CO3}}. Không dùng thêm 1 chất nào khác kể cả quỳ tím, làm thế nào nhận ra từng chất.
\end{baitoan}

\begin{baitoan}[\cite{Nguyen_Buu_Can_500_BT_Hoa_Hoc_THCS}, 233., p. 101]
	Hòa tan {\rm15.5 g \ce{Na2O}} vào nước tạo thành {\rm0.5 L} dung dịch. (a) Tính nồng độ mol của dung dịch thu được. (b) Tính thể tích dung dịch {\rm\ce{H2SO4} 20\%}, $d = 1.14$ {\rm g{\tt/}mL}, cần để trung hòa dung dịch trên. (c) Tính nồng độ mol của dung dịch sau phản ứng.
\end{baitoan}

\begin{baitoan}[\cite{Nguyen_Buu_Can_500_BT_Hoa_Hoc_THCS}, 234., p. 101]
	(a) Tìm công thức của iron oxide trong đó iron chiếm $70\%$ khối lượng. (b) Khử hoàn toàn {\rm2.4 g} hỗn hợp {\rm CuO, \ce{Fe_xO_y}} cùng số mol như nhau bằng hydrogen thu được {\rm1.76 g} kim loại. Hòa tan kim loại đó bằng dung dịch {\rm HCl} dư thấy thoát ra {\rm0.448 L \ce{H2}} (đktc). Xác định công thức của iron oxide.
\end{baitoan}

\begin{baitoan}[\cite{Nguyen_Buu_Can_500_BT_Hoa_Hoc_THCS}, 235., p. 101]
	Cho {\rm9.4 kg \ce{K2O}} vào nước. Tính khối lượng {\rm\ce{SO2}} cần thiết phản ứng với dung dịch trên để tạo thành: (a) Muối trung hòa. (b) Muối acid. (c) Hỗn hợp muối acid \& muối trung hòa theo tỷ số mol là $1:2$. (d) Hỗn hợp muối acid \& muối trung hòa theo tỷ số mol là $a:b$ với $a,b\in\mathbb{R}$, $a,b > 0$ cho trước.
\end{baitoan}

\begin{baitoan}[\cite{Nguyen_Buu_Can_500_BT_Hoa_Hoc_THCS}, 236.a, p. 101]
	Hòa tan hoàn toàn {\rm1.44 g} kim loại hóa trị {\rm II} bằng {\rm250 mL} dung dịch {\rm\ce{H2SO4} 0.3M}. Để trung hòa lượng acid dư cần dùng {\rm60 mL} dung dịch {\rm NaOH 0.5M}. Xác định kim loại đó.
\end{baitoan}

\begin{baitoan}[\cite{Nguyen_Buu_Can_500_BT_Hoa_Hoc_THCS}, 236.b, p. 101]
	Trộn {\rm300 mL} dung dịch {\rm HCl 0.5M} với {\rm200 mL} dung dịch {\rm\ce{Ba(OH)2}} nồng độ $a${\rm M} thu được {\rm500 mL} dung dịch trong đó nồng độ {\rm HCl} là {\rm0.02M}. Tính $a$.
\end{baitoan}

\begin{baitoan}[\cite{Nguyen_Buu_Can_500_BT_Hoa_Hoc_THCS}, 237., p. 102]
	Cần thêm bao nhiêu {\rm g \ce{SO3}} vào {\rm100 g} dung dịch {\rm\ce{H2SO4} 10\%} để được dung dịch {\rm\ce{H2SO4} 20\%}.
\end{baitoan}

\begin{baitoan}[\cite{Nguyen_Buu_Can_500_BT_Hoa_Hoc_THCS}, 238., p. 102]
	Để hòa tan hoàn toàn {\rm5.1 g} oxide kim loại hóa trị {\rm III}, phải dùng {\rm43.8 g} dung dịch {\rm HCl 25\%}. Tìm oxide kim loại đó.
\end{baitoan}

\begin{baitoan}[\cite{Nguyen_Buu_Can_500_BT_Hoa_Hoc_THCS}, 239., p. 102]
	Dẫn khí {\rm\ce{CO2}} vào {\rm1.2 L} dung dịch {\rm\ce{Ca(OH)2} 0.1M} thấy tạo ra {\rm5 g} 1 muối không tan cùng với 1 muối tan. (a) Tính thể tích khí {\rm\ce{CO2}} đã dùng (đktc). (b) Tính khối lượng \& nồng độ mol của muối tan. (c) Tính thể tích {\rm\ce{CO2}} (đktc) trong trường hợp chỉ tạo ra muối không tan. Tính khối lượng muối không tan đó.
\end{baitoan}

\begin{baitoan}[\cite{Nguyen_Buu_Can_500_BT_Hoa_Hoc_THCS}, 240., p. 102]
	Dung dịch X chứa hỗn hợp {\rm HCl, \ce{H2SO4}}. Lấy {\rm50 mL} dung dịch X cho tác dụng với {\rm\ce{AgNO3}} dư thấy tạo thành {\rm2.87 g} kết tủa. Lấy {\rm50 mL} dung dịch X cho tác dụng với {\rm\ce{BaCl2}} dư thấy tạo thành {\rm4.66 g} kết tủa. (a) Tính nồng độ mol của mỗi acid trong dung dịch X. (b) Cần dùng bao nhiêu {\rm mL} dung dịch {\rm NaOH 0.2M} trung hòa {\rm50 mL} dung dịch X.
\end{baitoan}

\begin{baitoan}[\cite{Nguyen_Buu_Can_500_BT_Hoa_Hoc_THCS}, 241., p. 102]
	Sau khi nung {\rm8 g} 1 hỗn hợp zinc carbonate \& zinc oxide, thu được {\rm6.24 g ZnO}. (a) Tính \% khối lượng hỗn hợp ban đầu. (b) Khí sinh ra được cho vào 1 dung dịch calcium hydroxide. Tính khối lượng calcium hydroxide để phản ứng chỉ tạo thành muối không tan.
\end{baitoan}

\begin{baitoan}[\cite{Nguyen_Buu_Can_500_BT_Hoa_Hoc_THCS}, 242., p. 102]
	Để trung hòa 1 dung dịch chứa {\rm189 g \ce{HNO3}}, đầu tiên dùng dung dịch có chứa {\rm112 g KOH}. Sau đó lại đổ thêm dung dịch {\rm\ce{Ba(OH)2} 25\%} cho trung hòa hết acid. (a) Viết {\rm PTHH}. (b) Tính khối lượng dung dịch {\rm\ce{Ba(OH)2}} đã dùng.
\end{baitoan}

\begin{baitoan}[\cite{Nguyen_Buu_Can_500_BT_Hoa_Hoc_THCS}, 243., p. 103]
	Viết {\rm PTHH} thực hiện các biến hóa: {\rm\ce{FeS2} $\to$ \ce{SO2} $\to$ \ce{SO3} $\to$ \ce{H2SO4}}. Tính lượng acid sulfuric thu được từ {\rm60 kg} quặng pirit nếu hiệu suất phản ứng là {\rm85\%} so với lý thuyết.
\end{baitoan}

\begin{baitoan}[\cite{Nguyen_Buu_Can_500_BT_Hoa_Hoc_THCS}, 244., p. 103]
	Hòa tan {\rm3.1 g \ce{Na2O}} vào nước để được {\rm2 L} dung dịch. (a) Tính nồng độ mol của dung dịch thu được. (b) Muốn làm trung hòa dung dịch trên phải cần bao nhiêu {\rm g} dung dịch {\rm\ce{H2SO4} 20\%}. (c) Tính nồng độ phân tử {\rm g} của muối tạo thành sau phản ứng. Biết dung dịch {\rm\ce{H2SO4} 20\%} có khối lượng riêng {\rm1.14 g{\tt/}mL}.
\end{baitoan}

\begin{baitoan}[\cite{Nguyen_Buu_Can_500_BT_Hoa_Hoc_THCS}, 245., p. 103]
	Cho {\rm11 g} dung dịch {\rm\ce{H2SO4} 20\%} vào {\rm400 g} dung dịch {\rm\ce{BaCl2} 5.2\%}. (a) Viết {\rm PTHH} \& tính khối lượng kết tủa tạo thành. (b) Tính nồng độ $\%$ của các chất có trong dung dịch sau khi tách bỏ kết tủa.
\end{baitoan}

\begin{baitoan}[\cite{Nguyen_Buu_Can_500_BT_Hoa_Hoc_THCS}, 246., p. 103]
	(a) Có thể điều chế khí anhidrit sunfurơ bằng cách cho {\rm\ce{H2SO4}} đặc tác dụng với sulfur (lưu huỳnh) ở nhiệt độ cao, hay với copper kim loại khi đun nóng. Viết {\rm PTHH}. (b) Oxy hóa hoàn toàn {\rm8 L} khí anhidrit sunfurơ {\rm\ce{SO2}} (đktc). Sản phẩm thu được cho tan trong {\rm57.2 mL} dung dịch {\rm\ce{H2SO4} 60\%} khối lượng riêng {\rm1.5 g{\tt/}mL}. Tính nồng độ $\%$ của dung dịch acid thu được.
\end{baitoan}

\begin{baitoan}[\cite{Nguyen_Buu_Can_500_BT_Hoa_Hoc_THCS}, 247., p. 103]
	Trộn {\rm30 mL} dung dịch có chứa {\rm2.22 g} calcium chloride với {\rm70 mL} dung dịch chứa {\rm1.7 g} silver nitrate. (a) Viết {\rm PTHH}. (b) Tính lượng kết tủa thu được. (c) Tính nồng độ mol của chất còn lại trong dung dịch. Cho thể tích dung dịch sau phản ứng thay đổi không đáng kể.
\end{baitoan}

\begin{baitoan}[\cite{Nguyen_Buu_Can_500_BT_Hoa_Hoc_THCS}, 248., p. 104]
	Cho {\rm38.2 g} hỗn hợp {\rm\ce{Na2CO3,K2CO3}} vào dung dịch {\rm HCl}. Dẫn lượng khí sinh ra qua nước vôi trong có dư thu được {\rm30 g} kết tủa. Tính khối lượng mỗi muối trong hỗn hợp.
\end{baitoan}

\begin{baitoan}[\cite{Nguyen_Buu_Can_500_BT_Hoa_Hoc_THCS}, 249., p. 104]
	Từ $80$ tấn quặng pirit chứa {\rm40\%} sulfur, sản xuất được $92$ tấn acid sulfuric. (a) Tính hiệu suất của quá trình sản xuất. (b) Từ lượng acid sulfuric này, có thể pha chế được bao nhiêu tấn dung dịch {\rm\ce{H2SO4} 23\%}.
\end{baitoan}

\begin{baitoan}[\cite{Nguyen_Buu_Can_500_BT_Hoa_Hoc_THCS}, 250., p. 104]
	Hòa tan {\rm13.3 g} hỗn hợp gồm {\rm NaCl, KCl} vào nước được {\rm500 g} dung dịch A. Lấy $\frac{1}{10}$ dung dịch A cho phản ứng với {\rm\ce{AgNO3}} dư được {\rm2.87 g} kết tủa. (a) Tính số {\rm g} mỗi muối ban đầu dùng. (b) Tính nồng độ $\%$ các muối trong dung dịch A.
\end{baitoan}

\begin{baitoan}[\cite{Nguyen_Buu_Can_500_BT_Hoa_Hoc_THCS}, 251., p. 104]
	Để hòa tan {\rm2.4 g} oxide 1 kim loại hóa trị {\rm II} cần dùng {\rm2.19 g} acid {\rm HCl}. Xác định oxide kim loại đó.
\end{baitoan}

\begin{baitoan}[\cite{Nguyen_Buu_Can_500_BT_Hoa_Hoc_THCS}, 252., p. 104]
	Cho {\rm1.568 L} khí carbonic (đktc) lội chậm qua dung dịch có hòa tan {\rm3.2 g NaOH}. Xác định thành phần định tính \& định lượng chất được sinh ra sau phản ứng.
\end{baitoan}

\begin{baitoan}[\cite{Nguyen_Buu_Can_500_BT_Hoa_Hoc_THCS}, 253.a, p. 104]
	Viết {\rm PTHH} để thực hiện các biến hóa theo sơ đồ: {\rm(a) Cu $\to$ CuO $\to$ \ce{CuSO4} $\to$ \ce{Cu(OH)2} $\to$ CuO. (b) CaO $\to$ \ce{Ca(OH)2} $\to$ \ce{CaCO3} $\to$ CaO}.
\end{baitoan}

\begin{baitoan}[\cite{Nguyen_Buu_Can_500_BT_Hoa_Hoc_THCS}, 253.b, p. 104]
	Trộn 1 dung dịch chứa {\rm5.1 g} sodium chloride với 1 dung dịch chứa {\rm5.1 g} silver nitrate. Tính lượng kết tủa được tạo thành sau phản ứng.
\end{baitoan}

\begin{baitoan}[\cite{Nguyen_Buu_Can_500_BT_Hoa_Hoc_THCS}, 254., p. 105]
	Cần dùng bao nhiêu {\rm L} dung dịch {\rm NaOH 0.5M} để trung hòa {\rm250 mL} dung dịch X chứa acid {\rm HCl 2M \& \ce{H2SO4 1.5M}}?
\end{baitoan}

\begin{baitoan}[\cite{Nguyen_Buu_Can_500_BT_Hoa_Hoc_THCS}, 255., p. 105]
	Trộn {\rm50 mL} dung dịch {\rm\ce{Na2CO3} 0.2M} với {\rm100 mL} dung dịch {\rm\ce{CaCl2} 0.15M} thì thu được 1 lượng kết tủa đúng bằng lượng kết tủa thu được khi trộn {\rm50 mL \ce{Na2CO3}} cho trên với {\rm100 mL} dung dịch {\rm\ce{BaCl2}} nồng độ $a${\rm M}? Tính $a$.
\end{baitoan}

\begin{baitoan}[\cite{Nguyen_Buu_Can_500_BT_Hoa_Hoc_THCS}, 256., p. 105]
	Cho {\rm1 g} hợp chất chloride chưa biết hóa trị vào 1 dung dịch silver nitrate lấy dư. Ta thu được 1 chất kết tủa màu trắng, đem sấy khô \& cân nặng {\rm2.65 g}. Xác định công thức của iron chloride.
\end{baitoan}

\begin{baitoan}[\cite{Nguyen_Buu_Can_500_BT_Hoa_Hoc_THCS}, 257.a, p. 105]
	Có 3 gói phân hóa học {\rm KCl, \ce{NH4NO3}} \& superphosphat (supe lân). Dựa vào phản ứng đặc trưng nào để phân biệt chúng.
\end{baitoan}

\begin{baitoan}[\cite{Nguyen_Buu_Can_500_BT_Hoa_Hoc_THCS}, 257.b, p. 105]
	Điều chế phân đạm urê bằng cách cho khí carbonic tác dụng với amoniac {\rm\ce{NH3}} ở nhiệt độ áp suất cao \& có xúc tác theo {\rm PTHH}: {\rm\ce{CO2 + $2$NH3 ->[xt] CO(NH2)2 + H2O}}. Tính thể tích khí {\rm\ce{CO2,NH3}} (đktc) cần lấy để sản xuất $10$ tấn urê, hiệu suất của quá trình là $80\%$.
\end{baitoan}

\begin{baitoan}[\cite{Nguyen_Buu_Can_500_BT_Hoa_Hoc_THCS}, 258., p. 105]
	Hòa tan hoàn toàn {\rm55 g} hỗn hợp {\rm\ce{Na2CO3,Na2SO3}} trong {\rm250 g} dung dịch {\rm HCl 14.6\%}. Biết phản ứng chỉ tạo ra muối trung hòa. (a) Tính thể tích khí thu được sau phản ứng (đktc). (b) Tính nồng độ $\%$ của muối có trong dung dịch sau phản ứng.
\end{baitoan}

\begin{baitoan}[\cite{Nguyen_Buu_Can_500_BT_Hoa_Hoc_THCS}, 259., p. 105]
	Để hòa tan hoàn toàn {\rm3.6 g} magnesium phải dùng bao nhiêu {\rm mL} dung dịch hỗn hợp {\rm HCl 1M} \& {\rm\ce{H2SO4} 0.75M}.
\end{baitoan}

\begin{baitoan}[\cite{Nguyen_Buu_Can_500_BT_Hoa_Hoc_THCS}, 260., p. 105]
	Cho {\rm5.6 L} khí {\rm\ce{CO2}} lội qua dung dịch {\rm NaOH 20\%}, $D = 1.22$ {\rm g{\tt/}mol}. (a) Tính khối lượng muối tạo thành. (b) Tính nồng độ $\%$ các chất có trong dung dịch sau phản ứng.
\end{baitoan}

\begin{baitoan}[\cite{Nguyen_Buu_Can_500_BT_Hoa_Hoc_THCS}, 261., p. 106]
	Cho dung dịch {\rm\ce{H2SO4}} vào dung dịch {\rm NaOH} thu được {\rm3.6 g} muối sulfate acid \& {\rm2/84 g} muối trung tính. Tính lượng dung dịch {\rm\ce{H2SO4} 49\%} \& dung dịch $20\%$ đã dùng.
\end{baitoan}

\begin{baitoan}[\cite{Nguyen_Buu_Can_500_BT_Hoa_Hoc_THCS}, 262., p. 106]
	Hòa tan {\rm49.6 g} hỗn hợp gồm 1 muối sulfate \& 1 muối carbonate của cùng 1 kim loại hóa trị {\rm I} vào nước thu được dung dịch A. Chia dung dịch A làm 2 phần bằng nhau.
	\begin{itemize}
		\item Phần 1: Cho phản ứng với lượng dư dung dịch acid sulfuric thu được {\rm2.24 L} khí (đktc).
		\item Phần 2: Cho phản ứng với lượng dư dung dịch {\rm\ce{BaCl2}} thu được {\rm43 g} kết tủa trắng.
	\end{itemize}
	(a) Tìm công thức 2 muối ban đầu. (b) Tính $\%$ khối lượng các muối trên có trong hỗn hợp.
\end{baitoan}

\begin{baitoan}[\cite{Nguyen_Buu_Can_500_BT_Hoa_Hoc_THCS}, 263., p. 106]
	Tính thể tích dung dịch {\rm KOH 5.6\%}, $D = 1.045$, cần dùng để làm trung hòa hết {\rm350 mL} dung dịch {\rm\ce{H2SO4} 0.5M}.
\end{baitoan}

\begin{baitoan}[\cite{Nguyen_Buu_Can_500_BT_Hoa_Hoc_THCS}, 264., p. 106]
	Cho acid hydrochloric phản ứng với {\rm6 g} hỗn hợp dạng bột gồm {\rm Mg, MgO}. (a) Tính thành phần $\%$ khối lượng của {\rm MgO} có trong hỗn hợp nếu phản ứng tạo ra {\rm2.24 L} khí {\rm\ce{H2}} (đktc). (b) Tính thể tích dung dịch {\rm HCl 20\%}, $D = 1.1$ {\rm g{\tt/}mL}, vừa đủ để phản ứng với hỗn hợp đó.
\end{baitoan}

\begin{baitoan}[\cite{Nguyen_Buu_Can_500_BT_Hoa_Hoc_THCS}, 265., p. 106]
	Dung dịch A chứa hỗn hợp {\rm NaOH, \ce{Ba(OH)2}}. Để trung hòa {\rm50 mL} dung dịch A cần dùng {\rm60 mL} dung dịch {\rm HCl 0.1M}. Khi cho {\rm50 mL} dung dịch A tác dụng với 1 lượng dư {\rm\ce{Na2CO3}} thấy tạo thành {\rm0.197 g} kết tủa. Tính nồng độ mol của {\rm NaOH, \ce{Ba(OH)2}} trong dung dịch A.
\end{baitoan}

\begin{baitoan}[\cite{Nguyen_Buu_Can_500_BT_Hoa_Hoc_THCS}, 266., p. 106]
	Hòa tan hoàn toàn {\rm27.4 g} hỗn hợp gồm {\rm\ce{M2CO3,MHCO3}} ({\rm M} là kim loại kiềm) bằng {\rm500 mL} dung dịch {\rm HCl 1M} thấy thoát ra {\rm6.72 L \ce{CO2}} (đktc). Để trung hòa lượng acid còn dư phải dùng {\rm50 mL} dung dịch {\rm NaOH 2M}. (a) Xác định 2 muối ban đầu. (b) Tính $\%$ khối lượng các muối trên.
\end{baitoan}

\begin{baitoan}[\cite{Nguyen_Buu_Can_500_BT_Hoa_Hoc_THCS}, 267., p. 107]
	Thả {\rm12 g} hỗn hợp aluminium \& silver vào dung dịch {\rm\ce{H2SO4} 7.35\%}. Sau khi phản ứng kết thúc, thu được {\rm13.44 L} khí hydrogen (đktc). (a) Tính $\%$ khối lượng mỗi kim loại có trong hỗn hợp. (b) Tính thể tích dung dịch {\rm\ce{H2SO4}} cần dùng biết khối lượng riêng $d = 1.025$ {\rm g{\tt/}mL}.
\end{baitoan}

\begin{baitoan}[\cite{Nguyen_Buu_Can_500_BT_Hoa_Hoc_THCS}, 268., p. 107]
	Cho {\rm100 g} hỗn hợp 2 muối chloride của cùng 1 kim loại A có hóa trị {\rm II \& III} tác dụng hoàn toàn với 1 dung dịch {\rm NaOH} lấy dư. Biết khối lượng của hydroxide kim loại hóa trị {\rm II} là {\rm19.8 g} \& khối lượng chloride kim loại hóa trị {\rm II} bằng $\frac{1}{2}$ khối lượng mol của A. (a) Xác định kim loại A. (b) Tính $\%$ khối lượng của 2 muối trong hỗn hợp.
\end{baitoan}

\begin{baitoan}[\cite{Nguyen_Buu_Can_500_BT_Hoa_Hoc_THCS}, 269., p. 107]
	Các oxide: {\rm\ce{SO2,CO2}, CO, CaO, MgO, \ce{Na2O,Al2O3,N2O5,K2O}}. Các oxide vừa tác dụng được với nước, vừa tác dụng được với kiềm. {\rm(1) \ce{SO2,CO2,Na2O}, CO, CaO. (2) \ce{SO2,CO2,N2O5}. (3) \ce{Na2O,Al2O3}, CaO, MgO, CuO. (4) CaO, \ce{Na2O,K2O}. (5) \ce{Al2O3,K2O}, CuO, MgO, CO. {\sf A.} (2), (4). {\sf B.} (1), (2), (3). {\sf C.} (2), (3), (4). {\sf D.} (3), (5).}
\end{baitoan}

\begin{baitoan}[\cite{Nguyen_Buu_Can_500_BT_Hoa_Hoc_THCS}, 270., p. 107]
	 Hợp chất nào sau đây là base? {\sf A.} Copper ({\rm II}) nitrate. {\sf B.} Potassium chloride. {\sf C.} sulfur dioxide. {\sf D.} calcium hydroxide.
\end{baitoan}

\begin{baitoan}[\cite{Nguyen_Buu_Can_500_BT_Hoa_Hoc_THCS}, 271., p. 108]
	1 trong các thuốc thử sau có thể dùng để phân biệt dung dịch sodium sulfate \& dung dịch sodium carbonate: {\sf A.} Dung dịch barium chloride. {\sf B.} Dung dịch acid hydrochloric. {\sf C.} Dung dịch lead (chì) ({\rm II}) nitrate. {\sf D.} Dung dịch sodium hydroxide.
\end{baitoan}

Cho các oxide: \ce{Al2O3,CaO,CO,Mn2O7,P2O5,N2O5,NO,SiO2,ZnO,Fe2O3}. Giải 4 bài toán tiếp theo:

\begin{baitoan}[\cite{Nguyen_Buu_Can_500_BT_Hoa_Hoc_THCS}, 272., p. 108]
	Oxide acid? {\rm{\sf A.} \ce{Al2O3,CO,P2O5,SiO2,NO}. {\sf B.} \ce{P2O5,N2O5,ZnO,Mn2O7}. {\sf C.} \ce{N2O5,P2O5,SiO2,Mn2O7}. {\sf D.} \ce{Al2O3,SiO2,NO}.}
\end{baitoan}

\begin{baitoan}[\cite{Nguyen_Buu_Can_500_BT_Hoa_Hoc_THCS}, 273., p. 108]
	Oxide base? {\rm{\sf A.} \ce{Al2O3,CaO,Fe2O3,SiO2}. {\sf B.} \ce{CaO,Fe2O3}. {\sf C.} \ce{Mn2O7,Fe2O3,ZnO,Al2O3}. {\sf D.} CaO, NO, CO, \ce{SiO2,Al2O3}.}
\end{baitoan}

\begin{baitoan}[\cite{Nguyen_Buu_Can_500_BT_Hoa_Hoc_THCS}, 274., p. 108]
	Oxide lưỡng tính? {\rm{\sf A.} \ce{Al2O3}, ZnO. {\sf B.} \ce{Mn2O7,SiO2}, NO, ZnO. {\sf C.} \ce{Fe2O3,CO,Al2O3,P2O5}. {\sf D.} ZnO, CO,\ce{Fe2O3,P2O5}.}
\end{baitoan}

\begin{baitoan}[\cite{Nguyen_Buu_Can_500_BT_Hoa_Hoc_THCS}, 275., p. 108]
	Oxide không tạo muối? {\rm{\sf A.} CaO, CO, \ce{SiO2}. {\sf B.} CO, NO. {\sf C.} NO, ZnO, \ce{Mn2O7}. {\sf D.} CaO, NO, \ce{Mn2O7,SiO2}.}
\end{baitoan}

\begin{baitoan}[\cite{Nguyen_Buu_Can_500_BT_Hoa_Hoc_THCS}, 276., p. 108]
	Để 1 mẫu sodium hydroxide trên miếng kính trong không khí, sau vài ngày thấy có chất rắn màu trắng phủ ngoài. Nếu nhỏ vài giọt dung dịch {\rm HCl} vào chất rắn màu trắng thấy có khí không màu, không mùi thoát ra. Chất rắn màu trắng này là sản phẩm của phản ứng sodium hydroxide với: {\sf A.} Oxygen trong không khí. {\sf B.} Hơi nước trong không khí. {\sf C.} Carbon dioxide \& oxygen trong không khí. {\sf D.} Carbon dioxide trong không khí.
\end{baitoan}

\begin{baitoan}[\cite{Nguyen_Buu_Can_500_BT_Hoa_Hoc_THCS}, 277., p. 109]
	Có 3 oxide màu trắng: {\rm MgO, \ce{Al2O3,Na2O}}. Có thể nhận biết được các chất đó bằng thuốc thử nào? {\sf A.} Chỉ dùng nước. {\sf B.} Chỉ dùng acid. {\sf C.} Chỉ dùng kiềm. {\sf D.} Dùng nước \& kiềm.
\end{baitoan}

\begin{baitoan}[\cite{Nguyen_Buu_Can_500_BT_Hoa_Hoc_THCS}, 278., p. 109]
	Các thí nghiệm nào sau đây sẽ tạo ra chất kết tủa khi trộn? (1) Dung dịch sodium chloride \& dung dịch lead nitrate. (2) Dung dịch sodium carbonate \& dung dịch zinc sulfate. (3) Dung dịch sodium sulfate \& dung dịch aluminium chloride. (4) Dung dịch zinc sulfate \& dung dịch copper ({\rm II}) chloride. (5) Dung dịch barium chloride \& dung dịch nitrate. {\sf A.} (1), (2), (5). {\sf B.} (1), (2), (3). {\sf C.} (2), (4), (5). {\sf D.} (3), (4), (5).
\end{baitoan}

\begin{baitoan}[\cite{Nguyen_Buu_Can_500_BT_Hoa_Hoc_THCS}, 279., p. 109]
	{\rm(1) \ce{H2 + $\ldots$ -> Cu + H2O}. (2) \ce{$\ldots$ + O2 -> $2$H2O}. (3) \ce{C + H2O -> CO + $\ldots$}. (4) \ce{Mg + H2O -> $\ldots$ + H2 ^}. (5) \ce{Mg + $2$HCl -> $\ldots$ + H2 ^}.} Các chất được điền vào chỗ trống lần lượt là: {\rm{\sf A.} Mg, \ce{H2,Cl,O2,H2}. {\sf B.} CuO, \ce{H2,H2,MgO,MgCl2}. {\sf C.} \ce{H2,Cu,Mg,O2,H2O}. {\sf D.} \ce{H2,CuO,MgO,O2,H2}.}
\end{baitoan}

\begin{baitoan}[\cite{Nguyen_Buu_Can_500_BT_Hoa_Hoc_THCS}, 280., pp. 109--110]
	Có các chất: copper, copper ({\rm II}) oxide, magnesium carbonate, magnesium, magnesium oxide. Chất nào tác dụng với dung dịch acid hydrochloric hoặc acid sulfuric loãng sinh ra: (a) Chất khí cháy được trong không khí? (b) Chất khí làm đục nước vôi trong? (c) Dung dịch có màu xanh? (d) Dung dịch không màu \& nước?
\end{baitoan}

\begin{baitoan}[\cite{Nguyen_Buu_Can_500_BT_Hoa_Hoc_THCS}, 281., p. 110]
	Có 4 oxide: {\rm I. \ce{SO3}. II. CaO. III. \ce{CrO3}. IV. MgO}. Tập hợp nào sau đây chỉ gồm oxide acid? {\rm{\sf A.} I, II. {\sf B.} II, III. {\sf C.} I, III. {\sf D.} III, IV}.
\end{baitoan}

\begin{baitoan}[\cite{Nguyen_Buu_Can_500_BT_Hoa_Hoc_THCS}, 282., p. 110]
	Cho phương trình phản ứng: {\rm\ce{$2$NaOH + X -> $2$Y + H2O}}. X, Y? {\rm{\sf A.} \ce{H2SO4, Na2SO4}. {\sf B.} \ce{N2O5,NaNO3}. {\sf C.} HCl, NaCl. {\sf D.} A, B đều đúng.}
\end{baitoan}

\begin{baitoan}[\cite{Nguyen_Buu_Can_500_BT_Hoa_Hoc_THCS}, 283., p. 110]
	Cho sơ đồ chuyển hóa: {\rm X $\to$ \ce{SO2} $\to$ Y $\to$ \ce{H2SO4}} với X là chất rắn. X, Y? {\rm{\sf A.} X: \ce{FeS2}, Y: \ce{SO3}. {\sf B.} X: \ce{FeS2} hoặc S, Y: \ce{SO3}. {\rm C.} X: \ce{O2}, Y: \ce{SO3}.} {\sf D.} Tất cả đều đúng.
\end{baitoan}

\begin{baitoan}[\cite{Nguyen_Buu_Can_500_BT_Hoa_Hoc_THCS}, 284., p. 110]
	Có 5 ống nghiệm chứa các dung dịch sau: {\rm\ce{Ba(NO3)2,H2SO4,NaOH,Na2CO3}}. Chỉ dùng 1 hóa chất duy nhất để nhận biết các hóa chất ở trong ống nghiệm: {\sf A.} Dùng phenolphtalein không màu. {\sf B.} Dùng giấy quỳ tím. {\sf C.} Dùng dung dịch acid {\rm HCl}. {\sf D.} Dùng dung dịch {\rm\ce{BaCl2}}.
\end{baitoan}

\begin{baitoan}[\cite{Nguyen_Buu_Can_500_BT_Hoa_Hoc_THCS}, 285., p. 110]
	Có các chất rắn: {\rm MgO, \ce{P2O5,Ba(OH)2,Na2SO4}}. Dùng các thuốc thử nào có thể phân biệt được các chất này? {\sf A.} Dùng {\rm\ce{H2O}}, giấy quỳ tím. {\sf B.} Dùng acid {\rm\ce{H2SO4}}, phenolphtalein không màu. {\sf C.} Dùng dung dịch {\rm NaOH}, quỳ tím. {\sf D.} Tất cả đều sai.
\end{baitoan}

\begin{baitoan}[\cite{Nguyen_Buu_Can_500_BT_Hoa_Hoc_THCS}, 286., p. 111]
	Có 5 dung dịch: {\rm\ce{Na2CO3,BaCl2,CH3COONa,Ba(HCO3)2}, NaCl}. Chỉ dùng dung dịch {\rm\ce{H2SO4}} có thể nhận biết được mấy chất? {\sf A.} $1$. {\sf B.} $2$. {\sf C.} $3$. {\sf D.} $5$.
\end{baitoan}

\begin{baitoan}[\cite{Nguyen_Buu_Can_500_BT_Hoa_Hoc_THCS}, 287., p. 111]
	Để loại bỏ khí {\rm\ce{CO2}} có lẫn trong hỗn hợp {\rm\ce{O2,CO2}}, cho hỗn hợp đi qua dung dịch chứa: {\rm{\sf A.} HCl. {\sf B.} \ce{Na2SO4}. {\sf C.} NaCl. {\sf D.} \ce{Ca(OH)2}}.
\end{baitoan}

\begin{baitoan}[\cite{Nguyen_Buu_Can_500_BT_Hoa_Hoc_THCS}, 288., p. 111]
	Nhỏ 1 giọt quỳ tím vào dung dịch {\rm NaOH}, dung dịch có màu xanh, nhỏ từ từ dung dịch {\rm HCl} cho tới dư vào dung dịch có màu xanh trên thì: {\sf A.} Màu xanh vẫn không thay đổi. {\sf B.} Màu xanh nhạt dần rồi mất hẳn. {\sf C.} Màu xanh nhạt dần, mất hẳn rồi chuyển sang màu đỏ. {\sf D.} Màu xanh đậm thêm dần.
\end{baitoan}

\begin{baitoan}[\cite{Nguyen_Buu_Can_500_BT_Hoa_Hoc_THCS}, 289., p. 111]
	Các cặp chất nào sau đây cùng tồn tại trong 1 dung dịch? {\rm{\sf A.} KCl, \ce{NaNO3}. {\sf B.} KOH, HCl. {\sf C.} HCl, \ce{AgNO3}. {\sf D.} \ce{NaHCO3}, NaOH}.
\end{baitoan}

\begin{baitoan}[\cite{Nguyen_Buu_Can_500_BT_Hoa_Hoc_THCS}, 290., p. 111]
	Để hòa tan hết {\rm5.1 g \ce{M2O3}} phải dùng {\rm43.8 g} dung dịch {\rm HCl 25\%}. Công thức của {\rm\ce{M2O3}}? {\rm{\sf A.} \ce{Fe2O3}. {\sf B.} \ce{Al2O3}. {\sf C.} \ce{Cr2O3}.} {\sf D.} Tất cả đều sai.
\end{baitoan}

%------------------------------------------------------------------------------%

\printbibliography[heading=bibintoc]

\end{document}