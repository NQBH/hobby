\documentclass{article}
\usepackage[backend=biber,natbib=true,style=alphabetic,maxbibnames=50]{biblatex}
\addbibresource{/home/nqbh/reference/bib.bib}
\usepackage[utf8]{vietnam}
\usepackage{tocloft}
\renewcommand{\cftsecleader}{\cftdotfill{\cftdotsep}}
\usepackage[colorlinks=true,linkcolor=blue,urlcolor=red,citecolor=magenta]{hyperref}
\usepackage{amsmath,amssymb,amsthm,float,graphicx,mathtools,diagbox,tikz,tipa}
\usepackage[version=4]{mhchem}
\allowdisplaybreaks
\newtheorem{assumption}{Assumption}
\newtheorem{baitoan}{}
\newtheorem{cauhoi}{Câu hỏi}
\newtheorem{conjecture}{Conjecture}
\newtheorem{corollary}{Corollary}
\newtheorem{dangtoan}{Dạng toán}
\newtheorem{definition}{Definition}
\newtheorem{dinhly}{Định lý}
\newtheorem{dinhnghia}{Định nghĩa}
\newtheorem{example}{Example}
\newtheorem{ghichu}{Ghi chú}
\newtheorem{hequa}{Hệ quả}
\newtheorem{hypothesis}{Hypothesis}
\newtheorem{lemma}{Lemma}
\newtheorem{luuy}{Lưu ý}
\newtheorem{nhanxet}{Nhận xét}
\newtheorem{notation}{Notation}
\newtheorem{note}{Note}
\newtheorem{principle}{Principle}
\newtheorem{problem}{Problem}
\newtheorem{proposition}{Proposition}
\newtheorem{question}{Question}
\newtheorem{remark}{Remark}
\newtheorem{theorem}{Theorem}
\newtheorem{thinghiem}{Thí nghiệm}
\newtheorem{vidu}{Ví dụ}
\usepackage[left=1cm,right=1cm,top=5mm,bottom=5mm,footskip=4mm]{geometry}

\begin{document}

\begin{baitoan}[\cite{An_Hoa_Hoc_nang_cao_8_9}, 1., p. 61]
	Viết {\rm CTHH} của các muối: calci chloride, potassium nitrate, potassium phosphate, aluminium sulfate, iron (III) nitrate.
\end{baitoan}

\begin{baitoan}[\cite{An_Hoa_Hoc_nang_cao_8_9}, 2., p. 62]
	Phân loại: {\rm\ce{KOH,CuCl2,Al2O3,ZnSO4,CuO,Zn(OH)2,H3PO4,HNO3}}.
\end{baitoan}

\begin{baitoan}[\cite{An_Hoa_Hoc_nang_cao_8_9}, 3., p. 62]
	Cho biết gốc acid \& tính hóa trị của gốc acid trong các {\rm CTHH}: {\rm\ce{H2S,HNO3,H2SO4,H2SiO3,H3PO4}, \ce{HClO4,H2Cr2O7,CH3COOH}}.
\end{baitoan}

\begin{baitoan}[\cite{An_Hoa_Hoc_nang_cao_8_9}, 4., p. 62]
	Viết công thức của các hydroxide ứng với các kim loại: sodium, calcium, chromium, barium, potassium, copper, zinc, iron.
\end{baitoan}

\begin{baitoan}[\cite{An_Hoa_Hoc_nang_cao_8_9}, 5., p. 62]
	Viết {\rm PTHH} biểu diễn các biến hóa: (a) {\rm Ca $\to$ CaO $\to$ \ce{Ca(OH)2}}. (b) {\rm Ca $\to$ \ce{Ca(OH)2}}.
\end{baitoan}

\begin{baitoan}[\cite{An_Hoa_Hoc_nang_cao_8_9}, 6., p. 63]
	Tính khối lượng sodium hydroxide thu được khi cho sodium tác dụng với nước: (a) {\rm46 g} sodium. (b) {0.3 mol} sodium.\hfill{\sf Ans: (a) 80 g. (b) 12 g.}
\end{baitoan}

\begin{baitoan}[\cite{An_Hoa_Hoc_nang_cao_8_9}, 7., p. 63]
	Tìm hiểu về copper (II) oxide: (a) Cách điều chế. (b) Chất này thuộc loại hợp chất nào? (c) Tính chất vật lý. (d) Tính chất hóa học. Viết {\rm PTHH} \& phân loại các phản ứng đó.
\end{baitoan}

\begin{baitoan}[\cite{An_Hoa_Hoc_nang_cao_8_9}, 8., p. 64]
	Trong các oxide: {\rm\ce{SO3,CO,CuO,Na2O,CaO,CO2,Al2O3}}, oxide nào hòa tan trong nước? Viết {\rm PTHH} \& gọi tên các sản phẩm tạo thành.
\end{baitoan}

\begin{baitoan}[\cite{An_Hoa_Hoc_nang_cao_8_9}, 9., p. 64]
	Phân loại: {\rm\ce{CaO,H2SO4,Fe(OH)2,FeSO4,CaSO4,LiOH,MnO2,CuCl2,Mn(OH)2,SO2}}.
\end{baitoan}

\begin{baitoan}[\cite{An_Hoa_Hoc_nang_cao_8_9}, 10., p. 65]
	Viết {\rm PTHH} biểu diễn các biến hóa: (a) {\rm S $\to$ \ce{SO2} $\to$ \ce{H2SO3}}. (b) {\rm Cu $\to$ CuO $\to$ \ce{CuSO4}}. (c) {\rm Ca $\to$ CaO $\to$ \ce{Ca(OH)2}}. (d) {\rm P $\to$ \ce{P2O5} $\to$ \ce{H3PO4}}.
\end{baitoan}

\begin{baitoan}[\cite{An_Hoa_Hoc_nang_cao_8_9}, 11., p. 65]
	Cho các chất: {\rm\ce{Na2O,P2O5}}, dung dịch acid {\rm\ce{H2SO4}}, dung dịch {\rm KOH}. Bằng phương pháp hóa học, nêu cách nhận biết các hợp chất trên.
\end{baitoan}

\begin{baitoan}[\cite{An_Hoa_Hoc_nang_cao_8_9}, 12., p. 66]
	Hoàn thành {\rm PTHH}: (a) {\rm Mg + HCl}. (b) {\rm Al + \ce{H2SO4}}. (c) {\rm MgO + HCl}. (d) {\rm CaO + \ce{H3PO4}}. (e) {\rm CaO + \ce{HNO3}}.
\end{baitoan}

\begin{baitoan}[\cite{An_Hoa_Hoc_nang_cao_8_9}, 13., p. 66]
	Khi cho kẽm tác dụng với acid hydrochloric, thu được {\rm10 g} khí hydro. Tính số mol acid hydrochloric tham gia phản ứng.\hfill{\sf Ans: 10 mol.}
\end{baitoan}

\begin{baitoan}[\cite{An_Hoa_Hoc_nang_cao_8_9}, 14., p. 66]
	Tìm {\rm CTHH} của các chất có thành phần theo khối lượng: {\rm(a) H: 2.04\%, S: 32.65\%, O: 65.31\%. (b) Cu: 40\%, S: 20\%, O: 40\%}.
\end{baitoan}

\begin{baitoan}[\cite{An_Hoa_Hoc_nang_cao_8_9}, 15., p. 67]
	Lập {\rm CTHH} của các base ứng với các oxide: {\rm CaO, FeO, \ce{Li2O}, BaO}.
\end{baitoan}

\begin{baitoan}[\cite{An_Hoa_Hoc_nang_cao_8_9}, 16., p. 67]
	Khi cho barium tác dụng với nước \& cho barium oxide tác dụng với nước đều cho ta barium hydroxide. Viết {\rm PTHH}.
\end{baitoan}

\begin{baitoan}[\cite{An_Hoa_Hoc_nang_cao_8_9}, 17., p. 68]
	Diphosphor pentoxide là 1 chất rắn trắng khi để ra ngoài không khí thì bị chảy rữa. Tại sao? Viết {\rm PTHH}.
\end{baitoan}
\begin{baitoan}[\cite{An_Hoa_Hoc_nang_cao_8_9}, 18., p. 68]
	Từ $100$ tấn quặng chứa $40\%$ lưu huỳnh có thể điều chế được bao nhiêu tấn acid sulfuric?\hfill{\sf Ans: 122.5 tấn.}
\end{baitoan}

\begin{baitoan}[\cite{An_Hoa_Hoc_nang_cao_8_9}, 19., p. 68]
	Viết {\rm CTHH} ủa các muối: potassium chloride, calcium nitrate, copper sulfate, sodium sulfite, sodium nitrate, calcium phosphate, copper carbonate.
\end{baitoan}

\begin{baitoan}[\cite{An_Hoa_Hoc_nang_cao_8_9}, 20., p. 69]
	Tính khối lượng vôi tôi {\rm\ce{Ca(OH)2}} có thể thu được khi cho {\rm 140 kg} vôi sống {\rm CaO} tác dụng với nước. Biết trong vôi sống có chứa $10\%$ tạp chất.\hfill{\sf Ans: 166.5 kg.}
\end{baitoan}

\begin{baitoan}[\cite{An_Hoa_Hoc_nang_cao_8_9}, 21., p. 69]
	Có bao nhiêu {\rm g} copper có thể bị {\rm0.5 mol} zinc đẩy ra khỏi dung dịch muối copper sulfate?\hfill{\sf Ans: 32 g.}
\end{baitoan}

\begin{baitoan}[\cite{An_Hoa_Hoc_nang_cao_8_9}, 22., p. 69]
	Có thể điều chế được các chất mới nào khi cho các chất: calcium oxide, nước, acid sulfuric, zinc. Viết {\rm PTHH}.
\end{baitoan}

\begin{baitoan}[\cite{An_Hoa_Hoc_nang_cao_8_9}, 23., p. 70]
	Tìm phương pháp xác định xem trong 3 lọ, lọ nào đựng dung dịch acid, muối ăn, \& dung dịch kiềm (base).
\end{baitoan}

\begin{baitoan}[\cite{An_Hoa_Hoc_nang_cao_8_9}, 24., p. 70]
	Cho các chất: aluminium, oxygen, nước, copper sulfate, iron, acid hydrochloric. Điều chế copper, copper oxide, aluminium chloride (bằng 2 phương pháp), \& iron chloride. Viết {\rm PTHH}.
\end{baitoan}

\begin{baitoan}[\cite{An_Hoa_Hoc_nang_cao_8_9}, 25., p. 70]
	Muốn điều chế calcium sulfate từ sulfur \& calcium cần thêm ít nhất các hóa chất gì? Viết {\rm PTHH}.
\end{baitoan}

\begin{baitoan}[\cite{An_Hoa_Hoc_nang_cao_8_9}, 26., p. 71]
	Đổ vào dung dịch chứa {\rm27 g} copper chloride, {\rm12 g} mạt sắt. Tính lượng {\rm Cu} thu được sau phản ứng.\hfill{\sf Ans: 12.8 g.}
\end{baitoan}

\begin{baitoan}[\cite{An_Hoa_Hoc_nang_cao_8_9}, 27., p. 71]
	Trong 1 ống nghiệm, hòa tan {\rm5g} copper sulfate ngậm nước {\rm\ce{CuSO4.$5$H2O}}, rồi thả vào đó 1 miếng kẽm. Có bao nhiêu {\rm g} đồng nguyên chất thoát ra sau phản ứng, biết đã lấy thừa kẽm.\hfill{\sf Ans: 1.28 g.}
\end{baitoan}

\begin{baitoan}[\cite{An_Hoa_Hoc_nang_cao_8_9}, 28., p. 72]
	Viết {\rm PTHH}: (a) {\rm\ce{CuSO4} $\to$ Cu $\to$ CuO $\to$ \ce{Cu(NO3)2}}. (b) {\rm Ca $\to$ \ce{CaCl2} $\to$ \ce{Ca(OH)2}}.
\end{baitoan}

\begin{baitoan}[\cite{An_Hoa_Hoc_nang_cao_8_9}, 29., p. 72]
	Viết {\rm PTHH} biểu diễn các biến hóa: 
	\begin{equation*}
		{\rm CuO}\ \left[\begin{split}
			&\to{\rm Cu}\\
			&\to\ce{CuSO4}\to{\rm Cu}.\\
			&\to\ce{CuCl2}
		\end{split}\right.
	\end{equation*}
\end{baitoan}

\begin{baitoan}[\cite{An_Hoa_Hoc_nang_cao_8_9}, 30., p. 73]
	Hoàn thành {\rm PTHH}: (a) {\rm Zn + \ce{H2SO4}}. (b) {\rm Mg + \ce{H2SO4}}. (c) {\rm ZnO + \ce{HNO3}}. (d) {\rm CaO + HCl}. (e) {\rm MgO + \ce{H2SO4}}. (f) {\rm\ce{Al2O3} + HCl}. (g) {\rm\ce{Na2O + H2SO4}}.
\end{baitoan}

\begin{baitoan}[\cite{An_Hoa_Hoc_nang_cao_8_9}, 31., p. 73]
	Có thể thu được bao nhiêu {\rm g \ce{H2}} khi cho {\rm13 g} zinc tác dụng với acid hydrochloric lấy dư? Có bao nhiêu {\rm g} muối được tạo thành trong phản ứng này?\hfill{\sf Ans: 27.2 g.}
\end{baitoan}

\begin{baitoan}[\cite{An_Hoa_Hoc_nang_cao_8_9}, 32., p. 73]
	Tính thể tích khí hydrogen thu được (đktc) khi cho {\rm2.4g} magnesium tác dụng hoàn toàn với dung dịch acid sulfuric.\hfill{\sf Ans: 2.24 L.}
\end{baitoan}

\begin{baitoan}[\cite{An_Hoa_Hoc_nang_cao_8_9}, 33., p. 74]
	Cho {\rm7 g} calcium oxide tác dụng với dung dịch chứa {\rm35 g} acid nitric. Tính lượng muối tạo thành.\hfill{\sf Ans: 20.5 g.}
\end{baitoan}

\begin{baitoan}[\cite{An_Hoa_Hoc_nang_cao_8_9}, 34., p. 74]
	Hòa tan {\rm1.6 g} copper oxide trong {\rm100 g} dung dịch {\rm\ce{H2SO4} 20\%}. (a) Viết {\rm PTHH}. (b) Bao nhiêu {\rm g} acid đã tham gia phản ứng. (c) Bao nhiêu {\rm g} muối đồng được tạo thành. (d) Tính nồng độ $\%$ của acid trong dung dịch thu được sau phản ứng.\hfill{\sf Ans: (b) 1.96 g. (c) 3.2 g. (d) 17.8.}
\end{baitoan}

\begin{baitoan}[\cite{An_Hoa_Hoc_nang_cao_8_9}, 35., p. 75]
	Cho các oxide: {\rm\ce{CO2,SiO2,Na2O,Fe2O3,P2O5}}. Chất nào tan trong nước, chất nào tan trong dung dịch kiềm, chất nào tan trong dung dịch {\rm HCl}. Viết {\rm PTHH}.
\end{baitoan}

\begin{baitoan}[\cite{An_Hoa_Hoc_nang_cao_8_9}, 36., p. 76]
	(a) Từ {\rm60 kg} quặng pirit. Tính lượng {\rm\ce{H2SO4} 96\%} thu được từ quặng này nếu hiệu suất là $85\%$ so với lý thuyết. (b) Từ {\rm80} tấn quặng pirit chứa {\rm40\% S} sản xuất được {\rm92} tấn {\rm\ce{H2SO4}}. Tính hiệu suất.\hfill{\sf Ans: (a) 86.77 kg. (b) 93.88\%.}
\end{baitoan}

\begin{baitoan}[\cite{An_Hoa_Hoc_nang_cao_8_9}, 37., p. 77]
	Cho {\rm114 g} dung dịch {\rm\ce{H2SO4} 20\%} vào {\rm400 g} dung dịch {\rm\ce{BaCl2} 5.2\%}. (a) Viết {\rm PTHH} \& tính khối lượng kết tủa tạo thành. (b) Tính nồng độ $\%$ của các chất có trong dung dịch sau khi tách bỏ kết tủa.\hfill{\sf Ans: (a) 23.3 g. (b) 1.48\%, 2.65\%.}
\end{baitoan}

\begin{baitoan}[\cite{An_Hoa_Hoc_nang_cao_8_9}, 38., p. 78]
	Khi cho $a$ {\rm g} dung dịch {\rm \ce{H2SO4}} nồng độ $A\%$ tác dụng với 1 lượng hỗn hợp 2 kim loại {\rm Na, Zn} (dùng dư) thì khối lượng {\rm \ce{H2}} tạo thành là $0.05a$ {\rm g}. Xác định nồng độ $A\%$.\hfill{\sf Ans: $15.8\%$.}
\end{baitoan}

\begin{baitoan}[\cite{An_Hoa_Hoc_nang_cao_8_9}, 39., p. 78]
	Trộn lẫn {\rm100 mL} dung dịch {\rm\ce{NaHSO4} 1M} với {\rm100 mL} dung dịch {\rm NaOH 2M} được dung dịch A. Cô cạn dung dịch A thì thu được hỗn hợp các chất nào?\hfill{\sf Ans: 14.2 g, 4 g.}
\end{baitoan}

\begin{baitoan}[\cite{An_Hoa_Hoc_nang_cao_8_9}, 40., p. 78]
	Cho {\rm15.9 g} hỗn hợp 2 muối {\rm\ce{MgCO3,CaCO3}} vào {\rm0.4 L} dung dịch {\rm HCl 1M} thu được dung dịch X. Hỏi dung dịch X có dư acid không?\hfill{\sf Ans: Acid dư.}
\end{baitoan}

\begin{baitoan}[\cite{An_Hoa_Hoc_nang_cao_8_9}, 41., p. 78]
	Cho {\rm6.2 g \ce{Na2O}} vào nước. Tính thể tích khí {\rm\ce{SO2}} (đktc) cần thiết với dung dịch trên để tạo 2 muối.\\\mbox{}\hfill{\sf Ans: 2.24 L $< V <$ 4.48 L.}
\end{baitoan}

\begin{baitoan}[\cite{An_Hoa_Hoc_nang_cao_8_9}, 42., p. 78]
	Tìm các ký hiệu bằng chữ cái trong sơ đồ sau \& hoàn thành sơ đồ bằng {\rm PTHH}: (a) {\rm A $\to$ CaO $\to$ \ce{Ca(OH)2} $\to$ A $\to$ \ce{Ca(HCO3)2} $\to$ \ce{CaCl2} $\to$ A}. (b) {\rm\ce{FeS2} $\to$ M $\to$ N $\to$ D $\to$ \ce{CaSO4}}. (c) {\rm\ce{CuSO4} $\to$ B $\to$ C $\to$ D $\to$ Cu}.
\end{baitoan}

\begin{baitoan}[\cite{An_Hoa_Hoc_nang_cao_8_9}, 43., p. 78]
	Làm thế nào để nhận biết được $3$ acid {\rm HCl, \ce{HNO3,H2SO4}} cùng tồn tại trong dung dịch loãng.	
\end{baitoan}

\begin{baitoan}[\cite{An_Hoa_Hoc_nang_cao_8_9}, 44.a, p. 78]
	Viết {\rm PTHH} thực hiện chuyển hóa: {\rm\ce{Cl2} $\to$ A $\to$ B $\to$ C $\to$ A $\to$ \ce{Cl2}}, trong đó $A$ là chất khí, B \& C là hợp chất chứa chlorine.
\end{baitoan}

\begin{baitoan}[\cite{An_Hoa_Hoc_nang_cao_8_9}, 45., p. 78]
	Hòa tan hoàn toàn $a$ {\rm g \ce{R2O3}} cần $b$ {\rm g} dung dịch {\rm\ce{H2SO4} 12.25\%} thì vừa đủ. Sau phản ứng thu được dung dịch muối có nồng độ $15.36\%$. Xác định kim loại R.\hfill{\sf Ans: Cr.}
\end{baitoan}

\begin{baitoan}[\cite{An_Hoa_Hoc_nang_cao_8_9}, 46., p. 79]
	Hòa tan {\rm13.2 g} hỗn hợp X gồm 2 kim loại có cùng hóa trị vào {\rm400 mL} dung dịch {\rm HCl 1.5M}. Cô cạn dung dịch sau phản ứng thu được {\rm32.7 g} hỗn hợp muối khan. Hỗn hợp X có tan hết trong dung dịch {\rm HCl} không?\hfill{\sf Ans: Không tan hết.}
\end{baitoan}

\begin{baitoan}[\cite{An_Hoa_Hoc_nang_cao_8_9}, 47., p. 79]
	Trộn $V_1$ {\rm L} dung dịch {\rm HCl 0.6M} với $V_2$ {\rm L} dung dịch {\rm NaOH 0.4M} thu được {\rm0.6 L} dung dịch A. Tính $V_1,V_2$ biết {\rm0.6 L} dung dịch A có thể hòa tan hết {\rm1.02 g \ce{Al2O3}}. Biết sự pha trộn không làm thay đổi thể tích 1 cách đáng kể.\footnote{Đã học ở Vật lý 8 về sự đan xen của các nguyên tử, phân tử của 2 hay nhiều dung dịch khi trộn vào nhau, xem \cite[\S19, pp. 68--70]{SGK_Vat_Ly_8}.}\hfill{\sf Ans: $(V_1,V_2)\in\{(0.3,0.3),(0.22,0.38)\}$.}
\end{baitoan}

\begin{baitoan}[\cite{An_Hoa_Hoc_nang_cao_8_9}, 48., p. 79]
	Cho {\rm39.6 g} hỗn hợp gồm {\rm\ce{KHSO3, K2CO3}} vào {\rm400 g} dung dịch {\rm HCl 7.3\%}. Sau phản ứng thu được hỗn hợp khí X có tỷ khối hơi so với {\rm\ce{H2}} bằng $25.33$ \& 1 dung dịch Y. (a) Chứng minh acid còn dư. (b) Tính $C\%$ các chất trong dung dịch Y.\hfill{\sf Ans: $8.78\%$, $2.58\%$.}
\end{baitoan}

\begin{baitoan}[\cite{An_Hoa_Hoc_nang_cao_8_9}, 20., p. 135]
	Để khử {\rm6.4 g} 1 oxide kim loại cần {\rm2.688 L} khí {\rm\ce{H2}}. Nếu lấy lượng kim loại đó cho tác dụng với dung dịch {\rm HCl} dư thì giải phóng {\rm1.792 L} khí {\rm\ce{H2}}. Tìm tên kim loại biết thể tích các khí đo ở đktc.
\end{baitoan}

\begin{baitoan}[\cite{An_Hoa_Hoc_nang_cao_8_9}, 21., p. 135]
	Có 4 oxide riêng biệt: {\rm\ce{Na2O,Al2O3,Fe2O3}, MgO}. Làm thế nào để nhận biết mỗi oxide bằng phương pháp hóa học với điều kiện chỉ được dùng thêm 2 chất là {\rm\ce{H2O}} \& dung dịch {\rm HCl}.
\end{baitoan}

\begin{baitoan}[\cite{An_Hoa_Hoc_nang_cao_8_9}, 22., p. 135]
	Cho $a$ {\rm a Fe} hòa tan trong dung dịch {\rm HCl} (thí nghiệm 1). sau khi cô cạn dung dịch thu được {\rm3.1 g} chất rắn. Nếu cho $a$ {\rm g Fe} \& $b$ {\rm g Mg} (thí nghiệm 2) vào dung dịch {\rm HCl} loãng (cùng lượng như trên) thu được {\rm4.48 mL \ce{H2}} \& sau khi cô cạn dung dịch thu được {\rm3.34 g} chất rắn. Tính $a,b$.
\end{baitoan}

\begin{baitoan}[\cite{An_Hoa_Hoc_nang_cao_8_9}, 25., p. 135]
	Cho {\rm31.8 g} hỗn hợp 2 muối {\rm\ce{MgCO3,CaCO3}} vào {\rm0.8 L} dung dịch {\rm HCl 1M} thu được dung dịch Z. (a) Dung dịch Z có dư acid không? (b) Tính $V$ {\rm L \ce{CO2}} sinh ra là bao nhiêu?
\end{baitoan}

%------------------------------------------------------------------------------%

\begin{baitoan}[\cite{An_400_BT_Hoa_Hoc_9}, 1., p. 12]
	(a) Cho rất từ từ dung dịch A chứa $a$ {\rm mol HCl} vào dung dịch B chứa $b$ {\rm mol \ce{Na2CO3}} ($a < 2b$) thì thu được dung dịch C \& $V$ {\rm L} khí. Tính $V$. (b) Nếu cho dung dịch B vào dung dịch A thì được dung dịch D \& $V_1$ {\rm L} khí. Biết các phản ứng xảy ra hoàn toàn, các thể tích khí đo ở đktc. Lập biểu thức nêu mối quan hệ giữa $V_1$ với $a,b$.
\end{baitoan}

\begin{baitoan}[\cite{An_400_BT_Hoa_Hoc_9}, 2., p. 12]
	Cho {\rm31.8 g} hỗn hợp X gồm 2 muối {\rm\ce{MgCO3,CaCO3}} vào {\rm0.8 L} dung dịch {\rm HCl 1M} thu được dung dịch Z. (a) Hỏi dung dịch Z có dư acid không? (b) Lượng {\rm\ce{CO2}} có thể thu được bao nhiêu? (c) Cho vào dung dịch Z 1 lượng dung dịch {\rm\ce{NaHCO3}} dư thì thể tích khí {\rm\ce{CO2}} thu được là {\rm2.24 L} (đktc). Tính khối lượng mỗi muối trong hỗn hợp X.
\end{baitoan}

\begin{baitoan}[\cite{An_400_BT_Hoa_Hoc_9}, 3., p. 12]
	Có 3 bình đựng lần lượt các dung dịch {\rm KOH 1M, 2M, 3M}, mỗi bình chứa {\rm1 L} dung dịch. Trộn lẫn các dung dịch này sao cho dung dịch {\rm KOH 1.8M} thu được có thể tích lớn nhất.
\end{baitoan}

\begin{baitoan}[Mở rộng \cite{An_400_BT_Hoa_Hoc_9}, 3., p. 12]
	Cho $a,b,c,d\in\mathbb{R}$, $a,b,c > 0$. Có 3 bình đựng lần lượt các dung dịch {\rm KOH $a$M, $b$M, $c$M}, mỗi bình chứa {\rm1 L} dung dịch. Biện luận theo $a,b,c,d$ để trộn lẫn các dung dịch này sao cho dung dịch {\rm KOH $d$M} thu được có thể tích lớn nhất.
\end{baitoan}

\begin{baitoan}[\cite{An_400_BT_Hoa_Hoc_9}, 4., p. 12]
	Cho {\rm19.7 g} muối carbonate của kim loại hóa trị II tác dụng hết với dung dịch {\rm\ce{H2SO4}} loãng, dư thu được {\rm23.3 g} muối sulfate. Công thức muối carbonate của kim loại hóa trị II?
\end{baitoan}

\begin{baitoan}[\cite{An_400_BT_Hoa_Hoc_9}, 5., p. 12]
	Chọn các chất thích hợp \& cân bằng {\rm PTHH}: {\rm(a) \ce{X1 + X2 -> Br2 + MnBr2 + H2O}, (b) \ce{X3 + X4 + X5 -> HCl + H2SO4}, (c) \ce{A_1 + A_2 -> SO2 + H2O}, (d) \ce{B1 + B2 -> NH3 + Ca(NO3)2 + H2O}, (e) \ce{D1 + D2 + D3 -> Cl2 + MnSO4 + K2SO4 + Na2SO4 + H2O}}.
\end{baitoan}

\begin{baitoan}[\cite{An_400_BT_Hoa_Hoc_9}, 6., p. 12]
	Hợp chất A bị phân hủy ở nhiệt độ cao theo {\rm PTPƯ: 2A $\to$ B + 2D + 4E}. Sản phẩm tạo thành đều ở thể khí, khối lượng mol trung bình của hỗn hợp khí sau phản ứng là {\rm22.86 g{\tt/}mol}. Tính khối lượng mol của A.
\end{baitoan}

\begin{baitoan}[\cite{An_400_BT_Hoa_Hoc_9}, 7., p. 13]
	Cho {\rm39.6 g} hỗn hợp gồm {\rm\ce{KHSO3,K2CO3}} vào {\rm400 g} dung dịch {\rm HCl 7.3\%}, khi xong phản ứng thu được hỗn hợp khí X có tỷ khối so với khí hydrogen bằng $25.33$ \& 1 dung dịch A. (a) Chứng minh acid còn dư. (b) Tính $C\%$ các chất trong dung dịch A.
\end{baitoan}

\begin{baitoan}[\cite{An_400_BT_Hoa_Hoc_9}, 8., p. 13]
	Hòa tan {\rm21.5 g} hỗn hợp {\rm\ce{BaCl2,CaCl2}} vào {\rm178.5 mL} nước để được dung dịch A. Thêm vào dung dịch A {\rm175 mL} dung dịch {\rm\ce{Na2CO3} 1M} thấy tách ra {\rm19.85 g} kết tủa \& còn nhận được {\rm400 mL} dung dịch B. Tính nồng độ $\%$ của dung dịch {\rm\ce{BaCl2,CaCl2}}.
\end{baitoan}

\begin{baitoan}[\cite{An_400_BT_Hoa_Hoc_9}, 9.a, p. 13]
	Chỉ được dùng thêm quỳ tím \& các ống nghiệm, chỉ rõ phương pháp nhận ra các dung dịch bị mất nhãn: {\rm\ce{NaHSO4,Na2CO3,Na2SO3,BaCl2,Na2S}}.
\end{baitoan}

\begin{baitoan}[\cite{An_400_BT_Hoa_Hoc_9}, 9.b, p. 13]
	Cho khí {\rm\ce{CO2}} (đktc) phản ứng với {\rm80 g} dung dịch {\rm NaOH 25\%} để tạo thành hỗn hợp muối acid \& muối trung hòa theo tỷ lệ số mol là $2:3$. Tính thể tích {\rm\ce{CO2}} cần dùng.
\end{baitoan}

\begin{baitoan}[\cite{An_400_BT_Hoa_Hoc_9}, 10., p. 13]
	Cho {\rm0.2 mol CuO} tan hết trong dung dịch {\rm\ce{H2SO4} 20\%} đun nóng (lượng vừa đủ). Sau đó làm nguội dung dịch đến $10^\circ${\rm C}. Tính khối lượng tinh thể {\rm\ce{CuSO4.$5$H2O}} đã tách khỏi dung dịch, biết độ tan của {\rm\ce{CuSO4}} ở $10^\circ${\rm C} là {\rm17.4g}.
\end{baitoan}

\begin{baitoan}[\cite{An_400_BT_Hoa_Hoc_9}, 11., p. 13]
	Để có được {\rm200 mL} dung dịch {\rm NaCl 0.1M}. Có thể làm theo cách nào? {\sf A.} Lấy {\rm5.85 g NaCl} hòa tan trong {\rm200 mL} nước cất. {\sf B.} Lấy {\rm5.85 g NaCl} hòa tan trong {\rm194.15 g} nước cất. {\sf C.} Hòa tan {\rm1.17 g NaCl} trong {\rm100 mL} nước cất sau đó bổ sung thêm nước cho đến {\rm200 mL}. {\sf D.} Lấy 1 cốc chia độ, cho nước vào rồi cho {\rm1.17 g NaCl} cho đến lúc đạt thể tích {\rm250 mL}.
\end{baitoan}

\begin{baitoan}[\cite{An_400_BT_Hoa_Hoc_9}, 12., pp. 13--14]
	Đốt cháy hoàn toàn 1 chất vô cơ A trong không khí thì chỉ thu được {\rm1.6 g} iron (III) oxide \& {\rm0.896 L} khí sunfurơ (đktc). (a) Xác định {\rm CTPT} của A. (b) Viết {\rm PTHH} để thực hiện chuỗi chuyển hóa: A $\to$ {\rm \ce{SO2}} $\to$ muối $A_1$ $\to$ $A_3$; A $\to$ kết tủa $A_2$.
\end{baitoan}

\begin{baitoan}[\cite{An_400_BT_Hoa_Hoc_9}, 13., p. 14]
	Hòa tan 1 ít {\rm NaCl} vào nước được $V$ {\rm mL} dung dịch A có khối lượng riêng $d$, thêm $V_1$ {\rm mL} nước vào dung dịch A được $(V + V_1)$ mL dung dịch B có khối lượng riêng $d_1$. Chứng minh $d > d_1$. Biết khối lượng riêng của nước là {\rm1 g{\tt/}mL}.
\end{baitoan}

\begin{baitoan}[\cite{An_400_BT_Hoa_Hoc_9}, 14., p. 14]
	Trộn $V_1$ {\rm L} dung dịch {\rm HCl 0.6M} với $V_2$ {\rm L} dung dịch {\rm NaOH 0.4M} thu được {\rm0.6 L} dung dịch A. Tính $V_1,V_2$ biết {\rm0.6 L} dung dịch A có thể hòa tan hết {\rm1.02 g \ce{Al2O3}} (coi sự pha trộn làm thay đổi thể tích không đáng kể).
\end{baitoan}

\begin{baitoan}[\cite{An_400_BT_Hoa_Hoc_9}, 15., p. 14]
	Có 5 dung dịch các chất: {\rm\ce{H2SO4,HCl,NaOH,KCl,BaCl2}}. Trình bày phương pháp phân biệt các dung dịch này mà chỉ dùng quỳ tím làm thuốc thử.
\end{baitoan}

\begin{baitoan}[\cite{An_400_BT_Hoa_Hoc_9}, 16., p. 14]
	Có 2 cốc, cốc A đựng {\rm200 mL} dung dịch chứa {\rm\ce{Na2CO3} 1M} \& {\rm\ce{NaHCO3} 1.5M}. Cốc B đựng {\rm173mL} dung dịch {\rm HCl 7.7\%}, $D = 1.37$ {\rm g{\tt/}mL}. Tiến hành 2 thí nghiệm:
	\begin{itemize}
		\item Thí nghiệm 1: Đổ rất từ từ cốc B vào cốc A.
		\item Thí nghiệm 2: Đổ rất từ từ cốc A vào cốc B.
	\end{itemize}
	Tính thể tích khí (đktc) thoát ra trong mỗi trường hợp sau khi đổ hết cốc này vào cốc kia.
\end{baitoan}

\begin{baitoan}[\cite{An_400_BT_Hoa_Hoc_9}, 17., p. 14]
	Cho $x$ {\rm g} dung dịch {\rm\ce{H2SO4}} loãng nồng độ $C\%$ tác dụng hoàn toàn với hỗn hợp 2 kim loại potassium \& iron (dùng dư), sau phản ứng khối lượng chung đã giảm $0.0469x$ {\rm g}. Tính $C\%$.
\end{baitoan}

\begin{baitoan}[\cite{An_400_BT_Hoa_Hoc_9}, 18., p. 14]
	Hòa tan {\rm450 g} potassium nitrate vào {\rm500 g} nước cất ở $25^\circ${\rm C} (dung dịch X). Biết độ tan của {\rm\ce{KNO3}} ở $20^\circ${\rm C} là {\rm32 g}. Xác định khối lượng potassium nitrate tách ra khỏi dung dịch khi làm lạnh dung dịch X đến $20^\circ${\rm C}.
\end{baitoan}

\begin{baitoan}[\cite{An_400_BT_Hoa_Hoc_9}, 19., pp. 14--15, HSG lớp 8 Tp. HCM 2000--2001]
	Khi cho $a$ {\rm g Fe} vào trong {\rm400 mL} dung dịch {\rm HCl}, sau khi phản ứng kết thúc đem cô cạn dung dịch thu được {\rm6.2 g} chất rắn X. Nếu cho hỗn hợp gồm $a$ {\rm g Fe} \& $b$ {\rm g Mg} vào trong {\rm400 mL} dung dịch {\rm HCl} thì sau khi phản ứng kết thúc, thu được {\rm896 mL \ce{H2}} (đktc) \& cô cạn dung dịch thì thu được {\rm6.68 g} chất rắn Y. Tính $a,b$, nồng độ mol của dung dịch {\rm HCl} \& thành phần khối lượng các chất trong X, Y. (Giả sử {\rm Mg} không phản ứng với nước \& khi phản ứng với acid, {\rm Mg} phản ứng trước, hết {\rm Mg} mới đến {\rm Fe}. Cho biết các phản ứng đều xảy ra hoàn toàn).
\end{baitoan}

\begin{baitoan}[\cite{An_400_BT_Hoa_Hoc_9}, 20., p. 15]
	Khử $a$ {\rm g} 1 iron oxide bằng {\rm CO} nóng, dư đến hoàn toàn thu được {\rm Fe} \& khí A. Hòa tan lượng sắt trên trong dung dịch {\rm\ce{H2SO4}} loãng dư thoát ra {\rm1.68 L \ce{H2}} (đktc). Hấp thụ toàn bộ khí A bằng {\rm\ce{Ca(OH)2}} dư thu được kết tủa. Tìm công thức iron oxide.
\end{baitoan}

\begin{baitoan}[\cite{An_400_BT_Hoa_Hoc_9}, 21., p. 15]
	Nung $m$ {\rm g} hỗn hợp chất rắn A gồm {\rm\ce{Fe2O3}, FeO} với lượng thiếu {\rm CO} thu được hỗn hợp chất rắn B có khối lượng {\rm47.84 g} \& {\rm5.6 L \ce{CO2}} (đktc). Tính $m$.
\end{baitoan}

\begin{baitoan}[\cite{An_400_BT_Hoa_Hoc_9}, 22., p. 15]
	Dung dịch X là dung dịch {\rm\ce{H2SO4}}, dung dịch Y là dung dịch {\rm NaOH}. Nếu trộn X \& Y theo tỷ lệ thể tích là $V_X:V_Y = 3:2$ thì được dung dịch A có chứa X dư. Trung hòa {\rm1 L} A cần {\rm40 g KOH 20\%}. Nếu trộn X \& Y theo tỷ lệ thể tích $V_X:V_Y = 2:3$ thì được dung dịch B có chứa Y dư. Trung hòa {\rm1 L} B cần {\rm29.2 g} dung dịch {\rm HCl 25\%}. Tính nồng độ mol của X \& Y.
\end{baitoan}

\begin{baitoan}[\cite{An_400_BT_Hoa_Hoc_9}, 23., p. 15]
	(a) Bằng phương pháp hóa học, phân biệt 4 muối sau: {\rm\ce{Na2CO3,MgCO3,BaCO3,CaCl2}}. (b) Chọn 2 dung dịch muối thích hợp để phân biệt 4 dung dịch các chất: {\rm\ce{BaCl2,HCl,K2SO4,Na3PO4}}.
\end{baitoan}

\begin{baitoan}[\cite{An_400_BT_Hoa_Hoc_9}, 24., p. 15]
	Đốt cháy hoàn toàn {\rm6.8 g} 1 hợp chất vô cơ A chỉ thu được {\rm4.48 L} khí {\rm\ce{SO2}} (đktc) \& {\rm3.6 g} nước. Tính thể tích khí {\rm\ce{O2}} đã dùng \& xác định {\rm CTPT} của A.
\end{baitoan}

\begin{baitoan}[\cite{An_400_BT_Hoa_Hoc_9}, 25., p. 15]
	Làm thế nào để nhận ra sự có mặt của mỗi khí trong hỗn hợp gồm {\rm CO, \ce{CO2,SO3}} bằng phương pháp hóa học, viết {\rm PTHH}.
\end{baitoan}

\begin{baitoan}[\cite{An_400_BT_Hoa_Hoc_9}, 26., p. 15]
	Hòa tan {\rm NaOH} rắn vào nước để tạo thành 2 dung dịch A \& B với nồng độ $\%$ của dung dịch A gấp $3$ lần nồng độ $\%$ của dung dịch B. Nếu đem trộn 2 dung dịch A \& B theo tỷ lệ khối lượng $m_A:m_B = 5:2$ thì thu được dung dịch C có nồng độ $\%$ là $20\%$. Xác định nồng độ $\%$ của 2 dung dịch A \& B.
\end{baitoan}

\begin{baitoan}[\cite{An_400_BT_Hoa_Hoc_9}, 27., pp. 15--16]
	Hỏi có bao nhiêu {\rm g NaCl} kết tinh khi làm lạnh {\rm600 g} dung dịch {\rm NaCl} bão hòa ở $90^\circ${\rm C}. Biết độ tan của {\rm NaCl} ở $90^\circ${\rm C} là {\rm50 g} \& ở $0^\circ${\rm C} là {\rm35 g}.
\end{baitoan}

\begin{baitoan}[\cite{An_400_BT_Hoa_Hoc_9}, 28., p. 16]
	Nêu phương pháp tách hỗn hợp gồm 3 khí {\rm\ce{Cl2,H2,CO2}} thành các chất nguyên chất.
\end{baitoan}

\begin{baitoan}[\cite{An_400_BT_Hoa_Hoc_9}, 29., p. 16]
	Tinh chế các chất khí: (a) {\rm\ce{O2}} có lẫn {\rm\ce{Cl2,CO2,SO2}}. (b) {\rm\ce{Cl2}} có lẫn {\rm\ce{O2,CO2,SO2}}. (c) {\rm\ce{CO2}} có lẫn khí {\rm HCl} \& hơi nước.
\end{baitoan}

\begin{baitoan}[\cite{An_400_BT_Hoa_Hoc_9}, 30., p. 16]
	Oxide của 1 kim loại hóa trị III có khối lượng {\rm32 g} tan hết trong {\rm294 g} dung dịch {\rm\ce{H2SO4} 20\%}. Tìm {\rm CTPT} của oxide kim loại đó.
\end{baitoan}

\begin{baitoan}[\cite{An_400_BT_Hoa_Hoc_9}, 31., p. 16]
	Cho {\rm19.6 g} acid phosphoric tác dụng với {\rm200 g} dung dịch potassium hydroxide có nồng độ $8.4\%$. Thu được các muối nào sau phản ứng? Tính khối lượng của mỗi muối.
\end{baitoan}

\begin{baitoan}[\cite{An_400_BT_Hoa_Hoc_9}, 32., p. 16]
	Phân bón A có chứa $80\%$ ammonium nitrate. Phân bón B có chứa $82\%$ calcium nitrate. Nếu cần {\rm56 kg} nitrogen để bón ruộng thì nên mua loại phân nào? Vì sao?
\end{baitoan}

\begin{baitoan}[\cite{An_400_BT_Hoa_Hoc_9}, 33., p. 16]
	Nêu phương pháp tách hỗn hợp đá vôi, vôi sống, thạch cao, \& muối ăn thành từng chất nguyên chất.
\end{baitoan}

\begin{baitoan}[\cite{An_400_BT_Hoa_Hoc_9}, 34., p. 16]
	Nêu phương pháp tách hỗn hợp đá vôi, silicon dioxide, \& iron (II) chloride thành từng chất nguyên chất.
\end{baitoan}

\begin{baitoan}[\cite{An_400_BT_Hoa_Hoc_9}, 35., p. 16]
	Nêu phương pháp tách hỗn hợp 3 khí {\rm\ce{O2,H2,SO2}} thành các chất nguyên chất.
\end{baitoan}

\begin{baitoan}[\cite{An_400_BT_Hoa_Hoc_9}, 36., p. 16]
	Nêu phương pháp tinh chế {\rm Cu} trong quặng {\rm Cu} có lẫn {\rm Fe, S}, \& {\rm Ag}.
\end{baitoan}

\begin{baitoan}[\cite{An_400_BT_Hoa_Hoc_9}, 37., p. 16]
	Cần thêm bao nhiêu {\rm g \ce{SO3}} vào dung dịch {\rm\ce{H2SO4} 10\%} để được {\rm100 g} dung dịch {\rm\ce{H2SO4} 20\%}?
\end{baitoan}

\begin{baitoan}[Mở rộng \cite{An_400_BT_Hoa_Hoc_9}, 37., p. 16]
	Cần thêm bao nhiêu {\rm g \ce{SO3}} vào dung dịch {\rm\ce{H2SO4} $a\%$} để được {\rm100 g} dung dịch {\rm\ce{H2SO4} $b\%$}, với $a,b\in\mathbb{R}$, $a,b > 0$?
\end{baitoan}

\begin{baitoan}[\cite{An_400_BT_Hoa_Hoc_9}, 38., p. 16]
	Phải hòa tan thêm bao nhiêu {\rm g} potassium hydroxide nguyên chất vào {\rm1200 g} dung dịch {\rm KOH 12\%} để có dung dịch {\rm KOH 20\%}?
\end{baitoan}

\begin{baitoan}[\cite{An_400_BT_Hoa_Hoc_9}, 39., p. 16]
	Cần phải dùng bao nhiêu {\rm L \ce{H2SO4}} có tỷ khối $d = 1.84$ \& bao nhiêu {\rm L} nước cất để pha thành {\rm10 L} dung dịch {\rm\ce{H2SO4}} có $d = 1.28$?
\end{baitoan}

\begin{baitoan}[\cite{An_400_BT_Hoa_Hoc_9}, 40.a, p. 16]
	(a) Trộn {\rm2 L} dung dịch {\rm HCl 4M} vào {\rm 1 L} dung dịch {\rm HCl 0.5 M}. Tính nồng độ mol của dung dịch mới.
\end{baitoan}

\begin{baitoan}[Mở rộng \cite{An_400_BT_Hoa_Hoc_9}, 40., p. 16]
	(a) Trộn $V_1$ {\rm L} dung dịch {\rm HCl $a$M} vào $V_2$ {\rm L} dung dịch {\rm HCl $b$M}. Tính nồng độ mol của dung dịch mới.
\end{baitoan}

\begin{baitoan}[\cite{An_400_BT_Hoa_Hoc_9}, 40.b, p. 16]
	Trộn {\rm150 g} dung dịch {\rm NaOH 10\%} vào {\rm460 g} dung dịch {\rm NaOH $x$\%} để tạo thành dung dịch $6\%$. Tính $x$.
\end{baitoan}

\begin{baitoan}[\cite{An_400_BT_Hoa_Hoc_9}, 41., p. 16]
	Cần lấy bao nhiêu {\rm mL} dung dịch {\rm HCl} có nồng độ $36\%$, $d = 1.19$, để pha thành {\rm5 L} dung dịch acid {\rm HCl} có nồng độ {\rm0.5M}.
\end{baitoan}

\begin{baitoan}[\cite{An_400_BT_Hoa_Hoc_9}, 42., p. 17]
	Cho {\rm100 g} dung dịch {\rm\ce{H2SO4} 19.6\%} vào {\rm400 g} dung dịch {\rm\ce{BaCl2} 13\%}. (a) Tính khối lượng kết tủa. (b) Tính nồng độ $\%$ các chất có trong dung dịch sau phản ứng.
\end{baitoan}

\begin{baitoan}[\cite{An_400_BT_Hoa_Hoc_9}, 43., p. 17]
	Hòa tan {\rm8.96 L} khí {\rm HCl} (đktc) vào {\rm185.4 g} nước được dung dịch M. Lấy {\rm50 g} dung dịch M cho tác dụng với {\rm85 g} dung dịch {\rm\ce{AgNO3} 16\%} thì thu được dung dịch N \& 1 chất kết tủa.
\end{baitoan}

\begin{baitoan}[\cite{An_400_BT_Hoa_Hoc_9}, 44., p. 17]
	Cho {\rm11.6 g} hỗn hợp {\rm\ce{Fe2O3}, FeO} có tỷ lệ số mol là $1:1$ vào {\rm300 mL} dung dịch {\rm HCl 2M} được dung dịch A. (a) Tính nồng độ mol của các chất trong dung dịch sau phản ứng (thể tích dung dịch thay đổi không đáng kể). (b) Tính thể tích dung dịch {\rm NaOH 1.5M} đủ để tác dụng hết với dung dịch A.
\end{baitoan}

\begin{baitoan}[\cite{An_400_BT_Hoa_Hoc_9}, 45., p. 17]
	Cho sản phẩm thu được khi oxy hóa hoàn toàn {\rm5.6 L} khí sunfurơ (đktc) vào trong {\rm57.2 mL} dung dịch {\rm\ce{H2SO4} 60\%} có $D = 1.5$ {\rm g{\tt/}mL}. Tính nồng độ $\%$ của dung dịch acid thu được.
\end{baitoan}

\begin{baitoan}[\cite{An_400_BT_Hoa_Hoc_9}, 46., p. 17]
	Cho {\rm200 g} dung dịch {\rm\ce{BaCl2} 5.2\%} tác dụng với {\rm58.8 g} dung dịch {\rm\ce{H2SO4} 20\%}. Tính nồng độ $\%$ của các chất có trong dung dịch.
\end{baitoan}

\begin{baitoan}[\cite{An_400_BT_Hoa_Hoc_9}, 47.a, p. 17]
	Tính tỷ lệ thể tích của 2 dung dịch {\rm HCl 0.2M \& 1M} để trộn thành dung dịch {\rm HCl 0.4M}.
\end{baitoan}

\begin{baitoan}[\cite{An_400_BT_Hoa_Hoc_9}, 47.b, p. 17]
	Tính khối lượng {\rm\ce{Na2O}} \& khối lượng nước cần để có được {\rm200 g} dung dịch {\rm NaOH 10\%}.
\end{baitoan}

\begin{baitoan}[\cite{An_400_BT_Hoa_Hoc_9}, 48., p. 17]
	1 loại đá chứa {\rm80\% \ce{CaCO3}}, phần còn lại là tạp chất trơ. Nung đá vôi trên tới phản ứng hoàn toàn. Hỏi khối lượng của chất rắn thu được sau khi nung bằng bao nhiêu $\%$ khối lượng đá trước khi nung \& tính {\rm\% CaO} trong chất rắn sau khi nung.
\end{baitoan}

\begin{baitoan}[\cite{An_400_BT_Hoa_Hoc_9}, 49., p. 17]
	Khi nung hỗn hợp {\rm\ce{CaCO3,MgCO3}} thì khối lượng chất rắn thu được sau phản ứng chỉ bằng $\frac{1}{2}$ khối lượng ban đầu. Xác định thành phần $\%$ khối lượng các chất trong hỗn hợp ban đầu.
\end{baitoan}

\begin{baitoan}[\cite{An_400_BT_Hoa_Hoc_9}, 50., p. 17]
	Trong quặng bôxit trung bình có $50\%$ aluminium oxide. Kim loại luyện được từ oxide đó còn chứa $1.5\%$ tạp chất. Tính lượng aluminium nguyên chất điều chế được  từ $0.5$ tấn quặng boxit.
\end{baitoan}

\begin{baitoan}[\cite{An_400_BT_Hoa_Hoc_9}, 51., pp. 17--18]
	Đốt cháy hỗn hợp {\rm CuO, FeO} với {\rm C} có dư thì được chất rắn A \& khí B. Cho B tác dụng với nước vôi trong có dư thu được {\rm8 g} kết tủa. Chất rắn A cho tác dụng với dung dịch {\rm HCl} có nồng độ $10\%$ thì cần dùng 1 lượng acid là {\rm73 g} sẽ vừa đủ. (a) Viết {\rm PTHH}. (b) Tính khối lượng {\rm CuO, FeO} trong hỗn hợp ban đầu \& thể tích khí B (đktc).
\end{baitoan}

\begin{baitoan}[\cite{An_400_BT_Hoa_Hoc_9}, 52., p. 18]
	Khi phân hủy bằng nhiệt {\rm14.2 g} hỗn hợp {\rm\ce{CaCO3,MgCO3}}, thu được {\rm6.6 g \ce{CO2}} (đktc). Tính thành phần $\%$ các chất trong hỗn hợp.
\end{baitoan}

\begin{baitoan}[\cite{An_400_BT_Hoa_Hoc_9}, 53., p. 18]
	Cho {\rm38.2 g} hỗn hợp {\rm\ce{Na2CO3,K2CO3}} vào dung dịch {\rm HCl}. Dẫn lượng khí sinh ra qua nước vôi trong có dư thu được {\rm30 g} kết tủa. Tính khối lượng mỗi muối trong hỗn hợp ban đầu.
\end{baitoan}

\begin{baitoan}[\cite{An_400_BT_Hoa_Hoc_9}, 54., p. 18]
	Cho {\rm0.325 g} hỗn hợp gồm {\rm NaCl, KCl} được hòa tan vào nước. Sau đó cho dung dịch {\rm\ce{AgNO3}} vào dung dịch trên, ta được 1 kết tủa; sấy kết tủa đến khối lượng không đổi thấy cân nặng {\rm0.717 g}. Tính thành phần $\%$ các chất trong hỗn hợp.
\end{baitoan}

\begin{baitoan}[\cite{An_400_BT_Hoa_Hoc_9}, 55., p. 18]
	{\rm\ce{Al4C3,CaC2}} tác dụng với nước theo {\rm PTHH: \ce{Al4C3 + $12$H2O -> $4$Al(OH)3 + $3$CH4, CaC2 + $2$H2O -> Ca(OH)2 + C2H2}}. Cho hỗn hợp 2 chất trên tác dụng với nước dư thu được {\rm2.016 L} hỗn hợp khí. Lấy hỗn hợp này đốt cháy hoàn toàn thu được {\rm2.688 L \ce{CO2}}. Các thể tích đều đo ở đktc. Tính lượng {\rm\ce{Al4C3,CaC2}} trong hỗn hợp.
\end{baitoan}

\begin{baitoan}[\cite{An_400_BT_Hoa_Hoc_9}, 56., p. 18]
	Dùng thuốc thử thích hợp, nhận biết các dung dịch sau đã mất nhãn: {\rm(a) \ce{NaCl,NaBr,KI,HCl,H2SO4,KOH}. (b) \ce{Na2SO4,H2SO4,NaOH,KCl,NaNO3}}.
\end{baitoan}

\begin{baitoan}[\cite{An_400_BT_Hoa_Hoc_9}, 57., p. 18]
	Dùng thuốc thử thích hợp để nhận biết các dung dịch: {\rm(a) KI, HCl, NaCl, \ce{H2SO4}. (b) HCl, HBr, NaCl, NaOH. (c) NaF, \ce{CaCl2}, KBr, \ce{MgI2}}.
\end{baitoan}

\begin{baitoan}[\cite{An_400_BT_Hoa_Hoc_9}, 58., p. 18]
	Chỉ dùng thêm 1 hóa chất, phân biệt các lọ mất nhãn: {\rm(a) \ce{MgCl2}, KBr, NaI, \ce{AgNO3,NH4HCO3}. (b) NaBr, \ce{ZnSO4, Na2CO3, AgNO3, BaCl2}}.
\end{baitoan}

\begin{baitoan}[\cite{An_400_BT_Hoa_Hoc_9}, 59., p. 18]
	Chỉ dùng thêm 1 hóa chất, phân biệt các dung dịch: {\rm(a) \ce{K2SO4,K2CO3,K2SiO3,K2S,K2SO3}. (b) \ce{MgCl2}, NaBr, \ce{Ca(NO3)2}}.
\end{baitoan}

\begin{baitoan}[\cite{An_400_BT_Hoa_Hoc_9}, 60., p. 18]
	Có 5 lọ, mỗi lọ đựng 1 trong các hóa chất: {\rm FeO, CuO, \ce{Fe3O4,Ag2O,MnO2}}. Dùng phương pháp hóa học để nhận biết từng hóa chất trong mỗi lọ.
\end{baitoan}

\begin{baitoan}[\cite{An_400_BT_Hoa_Hoc_9}, 61.a, p. 19]
	Chỉ có nước \& khí carbonic có thể phân biệt được 5 chất bột trắng sau hay không? {\rm NaCl, \ce{Na2SO4,BaCO3,Na2CO3,BaSO4}}? Nếu được, trình bày cách phân biệt.
\end{baitoan}

\begin{baitoan}[\cite{An_400_BT_Hoa_Hoc_9}, 61.b, p. 19]
	Trình bày các nguyên tắc tiến hành phân biệt 4 chất: {\rm\ce{BaSO4,BaCO3,NaCl,Na2CO3}} với điều kiện chỉ dùng thêm {\rm HCl} loãng.
\end{baitoan}

\begin{baitoan}[\cite{An_400_BT_Hoa_Hoc_9}, 62.a, p. 19]
	Nêu cách nhận biết {\rm CaO, \ce{Na2O}, MgO, \ce{P2O5}} đều là chất bột trắng.
\end{baitoan}

\begin{baitoan}[\cite{An_400_BT_Hoa_Hoc_9}, 62.b, p. 19]
	Bằng phương pháp hóa học, nhận biết 4 kim loại có màu trắng bạc {\rm Al, Ag, Fe, Mg}.
\end{baitoan}

\begin{baitoan}[\cite{An_400_BT_Hoa_Hoc_9}, 63.a, p. 19]
	Từ các nguyên liệu chính là {\rm\ce{CO2,NaCl,NH4Cl}}, viết các phương trình phản ứng điều chế {\rm\ce{NH4HCO3}} tinh khiết.
\end{baitoan}

\begin{baitoan}[\cite{An_400_BT_Hoa_Hoc_9}, 63.b, p. 19]
	Điều chế 3 oxide, 2 acid, \& 2 muối từ các hóa chất: {\rm Mg, \ce{H2O}}, không khí, \& {\rm S}. Viết {\rm PTHH}.
\end{baitoan}

\begin{baitoan}[\cite{An_400_BT_Hoa_Hoc_9}, 64., p. 19]
	Chỉ từ {\rm Cu, NaCl, \ce{H2O}}, nêu cách điều chế để thu được {\rm\ce{Cu(OH)2}}. Viết {\rm PTHH}.
\end{baitoan}

\begin{baitoan}[\cite{An_400_BT_Hoa_Hoc_9}, 65.a, p. 19]
	Cho các chất: Aluminium, oxygen, nước, copper sulfate, iron, acid hydrochloric. Điều chế copper, copper oxide, aluminium chloride (bằng 2 phương pháp) \& iron (II) chloride. Viết {\rm PTHH}.
\end{baitoan}

\begin{baitoan}[\cite{An_400_BT_Hoa_Hoc_9}, 65.b, p. 19]
	Bằng cách nào từ iron ta có thể điều chế iron (II) hydroxide, iron (III) hydroxide? Viết {\rm PTHH}.
\end{baitoan}

\begin{baitoan}[\cite{An_400_BT_Hoa_Hoc_9}, 66., p. 19]
	Chỉ từ quặng pirit {\rm\ce{FeS2,O2,H2O}}, có chất xúc tác thích hợp. Viết {\rm PTPƯ} điều chế muối iron (III) sulfate.
\end{baitoan}

\begin{baitoan}[\cite{An_400_BT_Hoa_Hoc_9}, 67., p. 19]
	Viết các {\rm PTPƯ} phản ứng điều chế trực tiếp: (a) {\rm Cu $\to$ \ce{CuCl2} bằng 3 phương pháp}. (b) {\rm\ce{CuCl2} $\to$ Cu} bằng 2 phương pháp. (c) {\rm Fe $\to$ \ce{FeCl3}} bằng 2 phương pháp.
\end{baitoan}

\begin{baitoan}[\cite{An_400_BT_Hoa_Hoc_9}, 68.a, p. 19]
	Chỉ từ các chất {\rm\ce{KMnO4,BaCl2,H2SO4}, Fe} có thể điều chế được các khí gì?
\end{baitoan}

\begin{baitoan}[\cite{An_400_BT_Hoa_Hoc_9}, 68.b, p. 19]
	Muốn điều chế 3 chất rắn: {\rm NaOH, \ce{NaHCO3,Na2CO3}}. (a) Trình bày 3 phương pháp điều chế mỗi chất. (b) Chỉ dùng 1 thuốc thử, nhận biết từng dung dịch các chất trên.
\end{baitoan}

\begin{baitoan}[\cite{An_400_BT_Hoa_Hoc_9}, 69.a, pp. 19--20]
	Khí nitrogen bị lẫn các tạp chất {\rm CO, \ce{CO2,H2}}, \& hơi nước. Làm thế nào để thu được {\rm\ce{N2}} tinh khiết.
\end{baitoan}

\begin{baitoan}[\cite{An_400_BT_Hoa_Hoc_9}, 69.b, p. 20]
	Khi đốt cháy than, thu được hỗn hợp khí {\rm CO, \ce{CO2}}. Trình bày phương pháp hóa học để thu được từng khí.
\end{baitoan}

\begin{baitoan}[\cite{An_400_BT_Hoa_Hoc_9}, 70., p. 20]
	Nêu phương pháp hóa học để làm sạch các khí: (a) Methane có lẫn khí acetylen. (b) Ethylen có lẫn khí carbonic.
\end{baitoan}

\begin{baitoan}[\cite{An_400_BT_Hoa_Hoc_9}, 72., p. 20]
	Nêu phương pháp tách các hỗn hợp sau thành các chất nguyên chất: (a) Hỗn hợp khí gồm: {\rm\ce{Cl2,H2,CO2}}. (b) Hỗn hợp khí gồm: {\rm\ce{SO2,CO2}, CO}. (c) Hỗn hợp khí gồm: {\rm\ce{SO2,O2}, HCl}.
\end{baitoan}

\begin{baitoan}[\cite{An_400_BT_Hoa_Hoc_9}, 73., p. 20]
	Tinh chế: (a) {\rm\ce{CaSO3}} có lẫn {\rm\ce{CaCO3,Na2CO3}}. (b) Muối ăn có lẫn {\rm\ce{CaCl2,CaSO4,Na2SO3}}. (c) {\rm Cu} có lẫn {\rm Fe, Ag, S}.
\end{baitoan}

\begin{baitoan}[\cite{An_400_BT_Hoa_Hoc_9}, 74.a, p. 20]
	Trình bày phương pháp hóa học để lấy từng oxide từ hỗn hợp: {\rm\ce{SiO2,Al2O3,Fe2O3}, CuO}.
\end{baitoan}

\begin{baitoan}[\cite{An_400_BT_Hoa_Hoc_9}, 74.b, p. 20]
	Trình bày phương pháp lấy từng kim loại {\rm Cu, Fe} từ hỗn hợp các oxide: {\rm\ce{SiO2,Al2O3}, CuO, FeO}.
\end{baitoan}

\begin{baitoan}[\cite{An_400_BT_Hoa_Hoc_9}, 74.c, p. 20]
	Bằng phương pháp hóa học, tách từng kim loại ra khỏi hỗn hợp gồm {\rm Al, Fe, Ag, Cu}.
\end{baitoan}

\begin{baitoan}[\cite{An_400_BT_Hoa_Hoc_9}, 75., p. 20]
	Nêu cách tách các chất ra khỏi hỗn hợp: (a) {\rm\ce{Cl2} có lẫn \ce{N2,H2}}. (b) {\rm\ce{Cl2} có lẫn \ce{CO2}}.
\end{baitoan}

\begin{baitoan}[\cite{An_400_BT_Hoa_Hoc_9}, 76., p. 20]
	Nêu cách tinh chế: (a) Muối ăn có lẫn {\rm\ce{MgCl2}} \& {\rm NaBr}. (b) Acid hydrochloric có lẫn acid {\rm\ce{H2SO4}}.
\end{baitoan}

\begin{baitoan}[\cite{An_400_BT_Hoa_Hoc_9}, 77., p. 20]
	1 loại muối ăn có lẫn tạp chất {\rm\ce{CaCl2,MgCl2,Na2SO4,MgSO4,CaSO4}}. Trình bày cách loại các tạp chất để thu được muối ăn tinh khiết.
\end{baitoan}

\begin{baitoan}[\cite{An_400_BT_Hoa_Hoc_9}, 78., p. 20]
	Tìm cách tách lấy từng muối trong hỗn hợp rắn gồm: ammonium chloride, barium chloride, magnesium chloride. Viết {\rm PTHH}.
\end{baitoan}

%------------------------------------------------------------------------------%

\printbibliography[heading=bibintoc]

\end{document}