\documentclass{article}
\usepackage[backend=biber,natbib=true,style=alphabetic,maxbibnames=50]{biblatex}
\addbibresource{/home/nqbh/reference/bib.bib}
\usepackage[utf8]{vietnam}
\usepackage{tocloft}
\renewcommand{\cftsecleader}{\cftdotfill{\cftdotsep}}
\usepackage[colorlinks=true,linkcolor=blue,urlcolor=red,citecolor=magenta]{hyperref}
\usepackage{amsmath,amssymb,amsthm,float,graphicx,mathtools,diagbox,tikz,tipa}
\usepackage[version=4]{mhchem}
\allowdisplaybreaks
\newtheorem{assumption}{Assumption}
\newtheorem{baitoan}{}
\newtheorem{cauhoi}{Câu hỏi}
\newtheorem{conjecture}{Conjecture}
\newtheorem{corollary}{Corollary}
\newtheorem{dangtoan}{Dạng toán}
\newtheorem{definition}{Definition}
\newtheorem{dinhly}{Định lý}
\newtheorem{dinhnghia}{Định nghĩa}
\newtheorem{example}{Example}
\newtheorem{ghichu}{Ghi chú}
\newtheorem{hequa}{Hệ quả}
\newtheorem{hypothesis}{Hypothesis}
\newtheorem{lemma}{Lemma}
\newtheorem{luuy}{Lưu ý}
\newtheorem{nhanxet}{Nhận xét}
\newtheorem{notation}{Notation}
\newtheorem{note}{Note}
\newtheorem{principle}{Principle}
\newtheorem{problem}{Problem}
\newtheorem{proposition}{Proposition}
\newtheorem{question}{Question}
\newtheorem{remark}{Remark}
\newtheorem{theorem}{Theorem}
\newtheorem{thinghiem}{Thí nghiệm}
\newtheorem{vidu}{Ví dụ}
\usepackage[left=1cm,right=1cm,top=5mm,bottom=5mm,footskip=4mm]{geometry}

\begin{document}

\begin{baitoan}[\cite{An_Hoa_Hoc_nang_cao_8_9}, 1., p. 61]
	Viết {\rm CTHH} của các muối: calci chloride, potassium nitrate, potassium phosphate, aluminium sulfate, iron (III) nitrate.
\end{baitoan}

\begin{baitoan}[\cite{An_Hoa_Hoc_nang_cao_8_9}, 2., p. 62]
	Phân loại: {\rm\ce{KOH,CuCl2,Al2O3,ZnSO4,CuO,Zn(OH)2,H3PO4,HNO3}}.
\end{baitoan}

\begin{baitoan}[\cite{An_Hoa_Hoc_nang_cao_8_9}, 3., p. 62]
	Cho biết gốc acid \& tính hóa trị của gốc acid trong các {\rm CTHH}: {\rm\ce{H2S,HNO3,H2SO4,H2SiO3,H3PO4}, \ce{HClO4,H2Cr2O7,CH3COOH}}.
\end{baitoan}

\begin{baitoan}[\cite{An_Hoa_Hoc_nang_cao_8_9}, 4., p. 62]
	Viết công thức của các hydroxide ứng với các kim loại: sodium, calcium, chromium, barium, potassium, copper, zinc, iron.
\end{baitoan}

\begin{baitoan}[\cite{An_Hoa_Hoc_nang_cao_8_9}, 5., p. 62]
	Viết {\rm PTHH} biểu diễn các biến hóa: (a) {\rm Ca $\to$ CaO $\to$ \ce{Ca(OH)2}}. (b) {\rm Ca $\to$ \ce{Ca(OH)2}}.
\end{baitoan}

\begin{baitoan}[\cite{An_Hoa_Hoc_nang_cao_8_9}, 6., p. 63]
	Tính khối lượng sodium hydroxide thu được khi cho sodium tác dụng với nước: (a) {\rm46 g} sodium. (b) {0.3 mol} sodium.\hfill{\sf Ans: (a) 80 g. (b) 12 g.}
\end{baitoan}

\begin{baitoan}[\cite{An_Hoa_Hoc_nang_cao_8_9}, 7., p. 63]
	Tìm hiểu về copper (II) oxide: (a) Cách điều chế. (b) Chất này thuộc loại hợp chất nào? (c) Tính chất vật lý. (d) Tính chất hóa học. Viết {\rm PTHH} \& phân loại các phản ứng đó.
\end{baitoan}

\begin{baitoan}[\cite{An_Hoa_Hoc_nang_cao_8_9}, 8., p. 64]
	Trong các oxide: {\rm\ce{SO3,CO,CuO,Na2O,CaO,CO2,Al2O3}}, oxide nào hòa tan trong nước? Viết {\rm PTHH} \& gọi tên các sản phẩm tạo thành.
\end{baitoan}

\begin{baitoan}[\cite{An_Hoa_Hoc_nang_cao_8_9}, 9., p. 64]
	Phân loại: {\rm\ce{CaO,H2SO4,Fe(OH)2,FeSO4,CaSO4,LiOH,MnO2,CuCl2,Mn(OH)2,SO2}}.
\end{baitoan}

\begin{baitoan}[\cite{An_Hoa_Hoc_nang_cao_8_9}, 10., p. 65]
	Viết {\rm PTHH} biểu diễn các biến hóa: (a) {\rm S $\to$ \ce{SO2} $\to$ \ce{H2SO3}}. (b) {\rm Cu $\to$ CuO $\to$ \ce{CuSO4}}. (c) {\rm Ca $\to$ CaO $\to$ \ce{Ca(OH)2}}. (d) {\rm P $\to$ \ce{P2O5} $\to$ \ce{H2PO4}}.
\end{baitoan}

\begin{baitoan}[\cite{An_Hoa_Hoc_nang_cao_8_9}, 11., p. 65]
	Cho các chất: {\rm\ce{Na2O,P2O5}}, dung dịch acid {\rm\ce{H2SO4}}, dung dịch {\rm KOH}. Bằng phương pháp hóa học, nêu cách nhận biết các hợp chất trên.
\end{baitoan}

\begin{baitoan}[\cite{An_Hoa_Hoc_nang_cao_8_9}, 12., p. 66]
	Hoàn thành {\rm PTHH}: (a) {\rm Mg + HCl}. (b) {\rm Al + \ce{H2SO4}}. (c) {\rm MgO + HCl}. (d) {\rm CaO + \ce{H3PO4}}. (e) {\rm CaO + \ce{HNO3}}.
\end{baitoan}

\begin{baitoan}[\cite{An_Hoa_Hoc_nang_cao_8_9}, 13., p. 66]
	Khi cho kẽm tác dụng với acid hydrochloric, thu được {\rm10 g} khí hydro. Tính số mol acid hydrochloric tham gia phản ứng.\hfill{\sf Ans: 10 mol.}
\end{baitoan}

\begin{baitoan}[\cite{An_Hoa_Hoc_nang_cao_8_9}, 14., p. 66]
	Tìm {\rm CTHH} của các chất có thành phần theo khối lượng: (a) {\rm H: 2.04\%, S: 32.65\%, O: 65.31\%}. (b) {\rm Cu: 40\%, S: 20\%, O: 40\%}.
\end{baitoan}

\begin{baitoan}[\cite{An_Hoa_Hoc_nang_cao_8_9}, 15., p. 67]
	Lập {\rm CTHH} của các base ứng với các oxide: {\rm CaO, FeO, \ce{Li2O}, BaO}.
\end{baitoan}

\begin{baitoan}[\cite{An_Hoa_Hoc_nang_cao_8_9}, 16., p. 67]
	Khi cho barium tác dụng với nước \& cho barium oxide tác dụng với nước đều cho ta barium hydroxide. Viết {\rm PTHH}.
\end{baitoan}

\begin{baitoan}[\cite{An_Hoa_Hoc_nang_cao_8_9}, 17., p. 68]
	Diphosphor pentoxide là 1 chất rắn trắng khi để ra ngoài không khí thì bị chảy rữa. Tại sao? Viết {\rm PTHH}.
\end{baitoan}
\begin{baitoan}[\cite{An_Hoa_Hoc_nang_cao_8_9}, 18., p. 68]
	Từ $100$ tấn quặng chứa {\rm40\%} lưu huỳnh có thể điều chế được bao nhiêu tấn acid sulfuric?\hfill{\sf Ans: 122.5 tấn.}
\end{baitoan}

\begin{baitoan}[\cite{An_Hoa_Hoc_nang_cao_8_9}, 19., p. 68]
	Viết {\rm CTHH} ủa các muối: potassium chloride, calcium nitrate, copper sulfate, sodium sulfite, sodium nitrate, calcium phosphate, copper carbonate.
\end{baitoan}

\begin{baitoan}[\cite{An_Hoa_Hoc_nang_cao_8_9}, 20., p. 69]
	Tính khối lượng vôi tôi {\rm\ce{Ca(OH)2}} có thể thu được khi cho {\rm 140 kg} vôi sống {\rm CaO} tác dụng với nước. Biết trong vôi sống có chứa {\rm10\%} tạp chất.\hfill{\sf Ans: 166.5 kg.}
\end{baitoan}

\begin{baitoan}[\cite{An_Hoa_Hoc_nang_cao_8_9}, 21., p. 69]
	Có bao nhiêu {\rm g} copper có thể bị {\rm0.5 mol} zinc đẩy ra khỏi dung dịch muối copper sulfate?\hfill{\sf Ans: 32 g.}
\end{baitoan}

\begin{baitoan}[\cite{An_Hoa_Hoc_nang_cao_8_9}, 22., p. 69]
	Có thể điều chế được các chất mới nào khi cho các chất: calcium oxide, nước, acid sulfuric, zinc. Viết {\rm PTHH}.
\end{baitoan}

\begin{baitoan}[\cite{An_Hoa_Hoc_nang_cao_8_9}, 23., p. 70]
	Tìm phương pháp xác định xem trong 3 lọ, lọ nào đựng dung dịch acid, muối ăn, \& dung dịch kiềm (base).
\end{baitoan}

\begin{baitoan}[\cite{An_Hoa_Hoc_nang_cao_8_9}, 24., p. 70]
	Cho các chất: aluminium, oxygen, nước, copper sulfate, iron, acid hydrochloric. Điều chế copper, copper oxide, aluminium chloride (bằng 2 phương pháp), \& iron chloride. Viết {\rm PTHH}.
\end{baitoan}

\begin{baitoan}[\cite{An_Hoa_Hoc_nang_cao_8_9}, 25., p. 70]
	Muốn điều chế calcium sulfate từ sulfur \& calcium cần thêm ít nhất các hóa chất gì? Viết {\rm PTHH}.
\end{baitoan}

\begin{baitoan}[\cite{An_Hoa_Hoc_nang_cao_8_9}, 26., p. 71]
	Đổ vào dung dịch chứa {\rm27 g} copper chloride, {\rm12 g} mạt sắt. Tính lượng copper thu được sau phản ứng.\hfill{\sf Ans: 12.8 g.}
\end{baitoan}

\begin{baitoan}[\cite{An_Hoa_Hoc_nang_cao_8_9}, 27., p. 71]
	Trong 1 ống nghiệm, hòa tan {\rm5g} copper sulfate ngậm nước {\rm\ce{CuSO4.$5$H2O}}, rồi thả vào đó 1 miếng kẽm. Có bao nhiêu {\rm g} đồng nguyên chất thoát ra sau phản ứng, biết đã lấy thừa kẽm.\hfill{\sf Ans: 1.28 g.}
\end{baitoan}

\begin{baitoan}[\cite{An_Hoa_Hoc_nang_cao_8_9}, 28., p. 72]
	Viết {\rm PTHH}: (a) {\rm{\ce CuSO4} $\to$ Cu $\to$ CuO $\to$ \ce{Cu(NO3)2}}. (b) {\rm Ca $\to$ CaCl2 $\to$ \ce{Ca(OH)2}}.
\end{baitoan}

\begin{baitoan}[\cite{An_Hoa_Hoc_nang_cao_8_9}, 29., p. 72]
	Viết {\rm PTHH} biểu diễn các biến hóa: 
	\begin{equation*}
		{\rm CuO}\ \left[\begin{split}
			&\to{\rm Cu}\\
			&\to\ce{CuSO4}\to{\rm Cu}.\\
			&\to\ce{CuCl2}
		\end{split}\right.
	\end{equation*}
\end{baitoan}

\begin{baitoan}[\cite{An_Hoa_Hoc_nang_cao_8_9}, 30., p. 73]
	Hoàn thành {\rm PTHH}: (a) {\rm Zn + \ce{H2SO4}}. (b) {\rm Mg + \ce{H2SO4}}. (c) {\rm ZnO + \ce{HNO3}}. (d) {\rm CaO + HCl}. (e) {\rm MgO + \ce{H2SO4}}. (f) {\rm\ce{Al2O3} + HCl}. (g) {\rm\ce{Na2O + H2SO4}}.
\end{baitoan}

\begin{baitoan}[\cite{An_Hoa_Hoc_nang_cao_8_9}, 31., p. 73]
	Có thể thu được bao nhiêu {\rm g \ce{H2}} khi cho {\rm13 g} zinc tác dụng với acid hydrochloric lấy dư? Có bao nhiêu {\rm g} muối được tạo thành trong phản ứng này?\hfill{\sf Ans: 27.2 g.}
\end{baitoan}

\begin{baitoan}[\cite{An_Hoa_Hoc_nang_cao_8_9}, 32., p. 73]
	Tính thể tích khí hydrogen thu được (đktc) khi cho {\rm2.4g} magnesium tác dụng hoàn toàn với dung dịch acid sulfuric.\hfill{\sf Ans: 2.24 L.}
\end{baitoan}

\begin{baitoan}[\cite{An_Hoa_Hoc_nang_cao_8_9}, 33., p. 74]
	Cho {\rm7 g} calcium oxide tác dụng với dung dịch chứa {\rm35 g} acid nitric. Tính lượng muối tạo thành.\hfill{\sf Ans: 20.5 g.}
\end{baitoan}

\begin{baitoan}[\cite{An_Hoa_Hoc_nang_cao_8_9}, 34., p. 74]
	Hòa tan {\rm1.6 g} copper oxide trong {\rm100 g} dung dịch {\rm\ce{H2SO4} 20\%}. (a) Viết {\rm PTHH}. (b) Bao nhiêu {\rm g} acid đã tham gia phản ứng. (c) Bao nhiêu {\rm g} muối đồng được tạo thành. (d) Tính nồng độ {\rm\%} của acid trong dung dịch thu được sau phản ứng.\hfill{\sf Ans: (b) 1.96 g. (c) 3.2 g. (d) 17.8.}
\end{baitoan}

\begin{baitoan}[\cite{An_Hoa_Hoc_nang_cao_8_9}, 35., p. 75]
	Cho các oxide: {\rm\ce{CO2,SiO2,Na2O,Fe2O3,P2O5}}. Chất nào tan trong nước, chất nào tan trong dung dịch kiềm, chất nào tan trong dung dịch {\rm HCl}. Viết {\rm PTHH}.
\end{baitoan}

\begin{baitoan}[\cite{An_Hoa_Hoc_nang_cao_8_9}, 36., p. 76]
	(a) Từ {\rm60 kg} quặng pirit. Tính lượng {\rm\ce{H2SO4} 96\%} thu được từ quặng này nếu hiệu suất là {\rm85\%} so với lý thuyết. (b) Từ {\rm80} tấn quặng pirit chứa {\rm40\% S} sản xuất được {\rm92} tấn {\rm\ce{H2SO4}}. Tính hiệu suất của quá trình.\hfill{\sf Ans: (a) 86.77 kg. (b) 93.88\%.}
\end{baitoan}

\begin{baitoan}[\cite{An_Hoa_Hoc_nang_cao_8_9}, 37., p. 77]
	Cho {\rm114 g} dung dịch {\rm\ce{H2SO4} 20\%} vào {\rm400 g} dung dịch {\rm\ce{BaCl2} 5.2\%}. (a) Viết {\rm PTHH} \& tính khối lượng kết tủa tạo thành. (b) Tính nồng độ {\rm\%} của các chất có trong dung dịch sau khi tách bỏ kết tủa.\hfill{\sf Ans: (a) 23.3 g. (b) 1.48\%, 2.65\%.}
\end{baitoan}

\begin{baitoan}[\cite{An_Hoa_Hoc_nang_cao_8_9}, 38., p. 78]
	Khi cho $a$ {\rm g} dung dịch {\rm \ce{H2SO4}} nồng độ $A${\rm\%} tác dụng với 1 lượng hỗn hợp 2 kim loại {\rm Na,Zn} (dùng dư) thì khối lượng {\rm \ce{H2}} tạo thành là $0.05a$ {\rm g}. Xác định nồng độ $A${\rm\%}.\hfill{\sf Ans:15.8\%.}
\end{baitoan}

\begin{baitoan}[\cite{An_Hoa_Hoc_nang_cao_8_9}, 39., p. 78]
	Trộn lẫn {\rm100 mL} dung dịch {\rm\ce{NaHSO4} 1M} với {\rm100 mL} dung dịch {\rm NaOH 2M} được dung dịch A. Cô cạn dung dịch A thì thu được hỗn hợp các chất nào?\hfill{\sf Ans: 14.2 g, 4 g.}
\end{baitoan}

\begin{baitoan}[\cite{An_Hoa_Hoc_nang_cao_8_9}, 40., p. 78]
	Cho {\rm15.9 g} hỗn hợp 2 muối {\rm\ce{MgCO3,CaCO3}} vào {\rm0.4 L} dung dịch {\rm HCl 1M} thu được dung dịch X. Hỏi dung dịch X có dư acid không?\hfill{\sf Ans: Acid dư.}
\end{baitoan}

\begin{baitoan}[\cite{An_Hoa_Hoc_nang_cao_8_9}, 41., p. 78]
	Cho {\rm6.2 g \ce{Na2O}} vào nước. Tính thể tích khí {\rm\ce{SO2}} (đktc) cần thiết với dung dịch trên để tạo 2 muối.\hfill{\sf Ans: $V\in(2.24\ {\rm L},4.48\ {\rm L})$.}
\end{baitoan}

\begin{baitoan}[\cite{An_Hoa_Hoc_nang_cao_8_9}, 42., p. 78]
	Tìm các ký hiệu bằng chữ cái trong sơ đồ sau \& hoàn thành sơ đồ bằng {\rm PTHH}: (a) {\rm A $\to$ CaO $\to$ \ce{Ca(OH)2} $\to$ A $\to$ \ce{Ca(HCO3)2} $\to$ \ce{CaCl2} $\to$ A}. (b) {\rm\ce{FeS2} $\to$ M $\to$ N $\to$ D $\to$ \ce{CaSO4}}. (c) {\rm\ce{CuSO4} $\to$ B $\to$ C $\to$ D $\to$ Cu}.
\end{baitoan}

\begin{baitoan}[\cite{An_Hoa_Hoc_nang_cao_8_9}, 43., p. 78]
	Làm thế nào để nhận biết được $3$ acid {\rm HCl, \ce{HNO3,H2SO4}} cùng tồn tại trong dung dịch loãng.	
\end{baitoan}

\begin{baitoan}[\cite{An_Hoa_Hoc_nang_cao_8_9}, 44.a, p. 78]
	Viết {\rm PTHH} thực hiện chuyển hóa: {\rm\ce{Cl2} $\to$ A $\to$ B $\to$ C $\to$ A $\to$ \ce{Cl2}}, trong đó $A$ là chất khí, B \& C là hợp chất chứa chlorine.
\end{baitoan}

\begin{baitoan}[\cite{An_Hoa_Hoc_nang_cao_8_9}, 45., p. 78]
	Hòa tan hoàn toàn $a$ {\rm g \ce{R2O3}} cần $b$ {\rm g} dung dịch {\rm\ce{H2SO4} 12.25\%} thì vừa đủ. Sau phản ứng thu được dung dịch muối có nồng độ {\rm15.36\%}. Xác định kim loại R.\hfill{\sf Ans: Cr.}
\end{baitoan}

\begin{baitoan}[\cite{An_Hoa_Hoc_nang_cao_8_9}, 46., p. 79]
	Hòa tan {\rm13.2 g} hỗn hợp X gồm 2 kim loại có cùng hóa trị vào {\rm400 mL} dung dịch {\rm HCl 1.5M}. Cô cạn dung dịch sau phản ứng thu được {\rm32.7 g} hỗn hợp muối khan. Hỗn hợp X có tan hết trong dung dịch {\rm HCl} không?\hfill{\sf Ans: Không tan hết.}
\end{baitoan}

\begin{baitoan}[\cite{An_Hoa_Hoc_nang_cao_8_9}, 47., p. 79]
	Trộn $V_1$ {\rm L} dung dịch {\rm HCl 0.6M} với $V_2$ {\rm L} dung dịch {\rm NaOH 0.4M} thu được {\rm0.6 L} dung dịch A. Tính $V_1,V_2$ biết {\rm0.6 L} dung dịch A có thể hòa tan hết {\rm1.02 g \ce{Al2O3}}. Biết sự pha trộn không làm thay đổi thể tích 1 cách đáng kể.\footnote{Đã học ở Vật lý 8 về sự đan xen của các nguyên tử, phân tử của 2 hay nhiều dung dịch khi trộn vào nhau, xem \cite[\S19, pp. 68--70]{SGK_Vat_Ly_8}.}\hfill{\sf Ans: $(V_1,V_2)\in\{(0.3,0.3),(0.22,0.38)\}$.}
\end{baitoan}

\begin{baitoan}[\cite{An_Hoa_Hoc_nang_cao_8_9}, 48., p. 79]
	Cho {\rm39.6 g} hỗn hợp gồm {\rm\ce{KHSO3, K2CO3}} vào {\rm400 g} dung dịch {\rm HCl 7.3\%}. Sau phản ứng thu được hỗn hợp khí X có tỷ khối hơi so với {\rm\ce{H2}} bằng $25.33$ \& 1 dung dịch Y. (a) Chứng minh acid còn dư. (b) Tính C\% các chất trong dung dịch Y.\hfill{\sf Ans: 8.78\%, 2.58\%.}
\end{baitoan}

%------------------------------------------------------------------------------%

\printbibliography[heading=bibintoc]

\end{document}