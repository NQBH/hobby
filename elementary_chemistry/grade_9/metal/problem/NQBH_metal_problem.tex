\documentclass{article}
\usepackage[backend=biber,natbib=true,style=alphabetic,maxbibnames=50]{biblatex}
\addbibresource{/home/nqbh/reference/bib.bib}
\usepackage[utf8]{vietnam}
\usepackage{tocloft}
\renewcommand{\cftsecleader}{\cftdotfill{\cftdotsep}}
\usepackage[colorlinks=true,linkcolor=blue,urlcolor=red,citecolor=magenta]{hyperref}
\usepackage{amsmath,amssymb,amsthm,float,graphicx,mathtools,diagbox,tikz,tipa}
\usepackage[version=4]{mhchem}
\allowdisplaybreaks
\newtheorem{assumption}{Assumption}
\newtheorem{baitoan}{Bài toán}
\newtheorem{cauhoi}{Câu hỏi}
\newtheorem{conjecture}{Conjecture}
\newtheorem{corollary}{Corollary}
\newtheorem{dangtoan}{Dạng toán}
\newtheorem{definition}{Definition}
\newtheorem{dinhly}{Định lý}
\newtheorem{dinhnghia}{Định nghĩa}
\newtheorem{example}{Example}
\newtheorem{ghichu}{Ghi chú}
\newtheorem{hequa}{Hệ quả}
\newtheorem{hypothesis}{Hypothesis}
\newtheorem{lemma}{Lemma}
\newtheorem{luuy}{Lưu ý}
\newtheorem{nhanxet}{Nhận xét}
\newtheorem{notation}{Notation}
\newtheorem{note}{Note}
\newtheorem{principle}{Principle}
\newtheorem{problem}{Problem}
\newtheorem{proposition}{Proposition}
\newtheorem{question}{Question}
\newtheorem{remark}{Remark}
\newtheorem{theorem}{Theorem}
\newtheorem{thinghiem}{Thí nghiệm}
\newtheorem{vidu}{Ví dụ}
\usepackage[left=1cm,right=1cm,top=5mm,bottom=5mm,footskip=4mm]{geometry}

\title{Problem: Metal -- Bài Tập: Kim Loại}
\author{Nguyễn Quản Bá Hồng\footnote{Independent Researcher, Ben Tre City, Vietnam\\e-mail: \texttt{nguyenquanbahong@gmail.com}; website: \url{https://nqbh.github.io}.}}
\date{\today}

\begin{document}
\maketitle
\begin{abstract}
	
\end{abstract}
\setcounter{secnumdepth}{4}
\setcounter{tocdepth}{3}
\tableofcontents

%------------------------------------------------------------------------------%

\begin{baitoan}[\cite{An_Hoa_Hoc_nang_cao_8_9}, 1., p. 89]
	(a) Sắt là nguyên tố có nhiều hóa trị, phổ biến là (II) \& (III). Viết các {\rm PTHH} minh họa. (b) Cho các kim loại {\rm Cu, Al, Fe, Ag}. Các kim loại nào tác dụng với acid hydrochloric? Các kim loại nào tác dụng được với dung dịch {\rm\ce{CuSO4}}? Dung dịch {\rm\ce{AgNO3}}? Viết các {\rm PTHH} tương ứng.
\end{baitoan}

\begin{baitoan}[\cite{An_Hoa_Hoc_nang_cao_8_9}, 2., p. 90]
	Có thể điều chế bao nhiêu {\rm kg} aluminium từ $1$ tấn quặng nhôm chứa {\rm95\%} aluminium oxide biết hiệu suất phản ứng là $98\%$.
\end{baitoan}

\begin{baitoan}[\cite{An_Hoa_Hoc_nang_cao_8_9}, 3., p. 90]
	(a) Tại sao không nên dùng chậu nhôm đựng nước vôi. (b) Viết PTHH giữa {\rm\ce{Fe3O4}} với {\rm\ce{H2SO4}}.
\end{baitoan}

\begin{baitoan}[\cite{An_Hoa_Hoc_nang_cao_8_9}, 4., p. 91]
	Cho {\rm1.38 g} 1 kim loại hóa trị (I) tác dụng hết với nước cho {\rm0.2 g} hydrogen. Xác định kim loại đó.
\end{baitoan}

\begin{baitoan}[\cite{An_Hoa_Hoc_nang_cao_8_9}, 5., p. 91]
	Trong quặng boxit trung bình có $50\%$ aluminium oxide. Kim loại luyện được từ oxide đó còn chứa $1.5\%$ tạp chất. Tính lượng nhôm nguyên chất điều chế được từ $0.5$ tấn quặng boxit.
\end{baitoan}

\begin{baitoan}[\cite{An_Hoa_Hoc_nang_cao_8_9}, 6., p. 92]
	Cho bản kẽm có khối lượng {\rm50 g} vào dung dịch đồng sulfate. Sau 1 thời gian phản ứng kết thúc thì khối lượng bản kẽm là {\rm49.82 g}. Tính: (a) Khối lượng kẽm đã tác dụng. (b) Khối lượng đồng sulfate có trong dung dịch.
\end{baitoan}

\begin{baitoan}[\cite{An_Hoa_Hoc_nang_cao_8_9}, 7., p. 92]
	Để thu được $1000$ tấn gang chứa $95\%$ sắt, $5\%$ carbon (các nguyên tố khác chiếm 1 lượng không đáng kể) thì theo lý thuyết phải cần bao nhiêu tấn {\rm\ce{Fe2O3}} \& bao nhiêu tấn than cốc. 
\end{baitoan}

\begin{baitoan}[\cite{An_Hoa_Hoc_nang_cao_8_9}, 8., p. 93]
	Cho {\rm5.4 g} 1 kim loại tác dụng với chlorine có dư thu được {\rm26.7  g} muối. Xác định kim loại đem phản ứng, biết kim loại có hóa trị từ I $\to$ III.
\end{baitoan}

\begin{baitoan}[\cite{An_Hoa_Hoc_nang_cao_8_9}, 9., p. 94]
	1 nguyên tố R có oxide cao nhất chiếm $60\%$ oxi theo khối lượng. Hợp chất khí của R với hydrogen có tỷ khối hơi so với không khí là $1.172$. Xác định công thức oxide của R.
\end{baitoan}

\begin{baitoan}[\cite{An_Hoa_Hoc_nang_cao_8_9}, 10., p. 94, TS PTNK ĐH KHTN Tp. HCM 1998]
	1 hỗn hợp X gồm kim loại {\rm M} ({\rm M} có hóa trị II \& III) \& oxide $\rm M_xO_y$ của kim loại ấy. Khối lượng hỗn hợp X là {\rm27.2 g}. Khi cho X tác dụng với {\rm0.8 L HCl 2M} thì hỗn hợp X tan hết cho dung dịch A cần {\rm0.6 L} dung dịch {\rm NaOH 1M}. Xác định {\rm M}, $\rm M_xO_y$, \& {\rm\%M, \%$\rm M_xO_y$} (theo khối lượng) trong hỗn hợp X. Biết trong 2 chất này có 1 chất có số mol bằng $2$ lần số mol chất kia.
\end{baitoan}

\begin{baitoan}[\cite{An_Hoa_Hoc_nang_cao_8_9}, 11., p. 94]
	A là kim loại hóa trị II. Lấy 2 thanh A cùng khối lượng. Thanh thứ nhất nhúng vào dung dịch {\rm\ce{CuSO4}}, sau 1 thời gian khối lượng giảm $3.6\%$. Thanh thứ 2 nhúng vào dung dịch {\rm\ce{HgSO4}}, sau 1 thời gian khối lượng tăng $6.675\%$. Nồng độ mol của 2 dung dịch {\rm\ce{CuSO4,HgSO4}} giảm cùng 1 số mol như nhau. Xác định tên kim loại A.
\end{baitoan}

\begin{baitoan}[\cite{An_Hoa_Hoc_nang_cao_8_9}, 12., p. 94]
	Khử {\rm3.48 g} 1 oxide của kim loại M cần dùng {\rm1.344 L} khí {\rm\ce{H2}} (ở đktc). Tìm {\rm CTPT} của oxide kim loại.
\end{baitoan}

\begin{baitoan}[\cite{An_Hoa_Hoc_nang_cao_8_9}, 13., p. 94]
	Cho hỗn hợp {\rm Al, Fe} tác dụng với hỗn hợp dung dịch chứa {\rm\ce{AgNO3,Cu(NO3)2}} thu được dung dịch B \& chất rắn D gồm 3 kim loại. Cho D tác dụng với dung dịch {\rm HCl} dư có khí bay ra. Xác định thành phần chất rắn D.
\end{baitoan}

\begin{baitoan}[\cite{An_Hoa_Hoc_nang_cao_8_9}, 14., p. 94]
	Cho {\rm2 g} hỗn hợp {\rm Fe} \& kim loại hóa trị II vào dung dịch {\rm HCl} có dư thì thu được {\rm1.12 L \ce{H2}} (đktc). Mặt khác, nếu hòa tan {\rm4.8 g} kim loại hóa trị II đó cần chưa đến {\rm500 mL} dung dịch {\rm HCl}. Xác định kim loại hóa trị II.
\end{baitoan}

\begin{baitoan}[\cite{An_Hoa_Hoc_nang_cao_8_9}, 15., pp. 94--95]
	X là hỗn hợp 2 kim loại {\rm Mg, Zn}. Y là dung dịch {\rm\ce{H2SO4}} chưa rõ nồng độ.
	\begin{itemize}
		\item Thí nghiệm 1: Cho {\rm24.3 g} X vào {\rm2 L} Y, sinh ra {\rm8.96 L \ce{H2}}.
		\item Thí nghiệm 2: Cho {\rm24.3 g} X vào {\rm3 L} Y, sinh ra {\rm11.2 L \ce{H2}}.
	\end{itemize}
	Lập luận chứng tỏ trong thí nghiệm 1 thì X chưa tan hết, trong thí nghiệm 2 thì X tan hết.
\end{baitoan}

\begin{baitoan}[\cite{An_Hoa_Hoc_nang_cao_8_9}, 16., p. 95]
	Cho {\rm8 g \ce{Fe_xO_y}} tác dụng với $V$ {\rm mL} dung dịch {\rm HCl 2M} lấy dư $25\%$ với lượng cần thiết. Đun nóng khan dung dịch sau phản ứng thu được {\rm16.25 g} muối khan. (a) Xác định {\rm CTPT} {\rm\ce{Fe_xO_y}}. (b) Tính $V$.
\end{baitoan}

\begin{baitoan}[\cite{An_Hoa_Hoc_nang_cao_8_9}, 17., p. 95]
	Nung nóng kim loại X trong không khí đến khối lượng không đổi được chất rắn Y. Khối lượng của X bằng $\frac{7}{10}$ khối lượng Y. Tìm {\rm CTPT} của chất rắn Y.
\end{baitoan}

\begin{baitoan}[\cite{An_Hoa_Hoc_nang_cao_8_9}, 18., p. 95]
	Cho {\rm3.87 g} hỗn hợp A gồm {\rm Mg, Al} vào {\rm250 mL} dung dịch X chứa {\rm HCl 1M} \& {\rm\ce{H2SO4} 0.5M} được dung dịch B \& {\rm4.368 L \ce{H2}} (ở đktc). Biện luận xem hỗn hợp A còn dư hay đã phản ứng hết.
\end{baitoan}

\begin{baitoan}[\cite{An_Hoa_Hoc_nang_cao_8_9}, 19., p. 95]
	Nguyên tố R tạo thành hợp chất khí với hydrogen có {\rm CTHH} là {\rm\ce{RH4}}. Trong hợp chất cao nhất với oxide chứa $72.73\%$ là oxygen. (a) Xác định tên nguyên tố R. (b) Cho biết vị trí của R trong bảng tuần hoàn.
\end{baitoan}

\begin{baitoan}[\cite{An_Hoa_Hoc_nang_cao_8_9}, 20., p. 95]
	{\rm Đ{\tt/}S?} {\sf A.} Trong cùng 1 chu kỳ, khi điện tích hạt nhân tăng dần, tính phi kim tăng dần \& bán kính nguyên tử giảm dần. {\sf B.} Trong chu kỳ, theo chiều tăng điện tích hạt nhân, tính acid của các oxide \& hydroxide giảm dần. {\sf C.} Trong cùng 1 nhóm, khi điện tích hạt nhân tăng dần thì tính base của các oxide \& hydrogen tăng dần. {\sf D. B} sai.
\end{baitoan}

\begin{baitoan}[\cite{An_400_BT_Hoa_Hoc_9}, 79., p. 32]
	Hòa tan {\rm13.2 g} hỗn hợp A gồm 2 kim loại có cùng hóa trị vào {\rm400 mL} dung dịch {\rm HCl 1.5M}. Cô cạn dung dịch sau phản ứng thu được {\rm32.7 g} hỗn hợp muối khan. (a) Chứng minh hỗn hợp A không tan hết. (b) Tính thể tích hydrogen sinh ra.
\end{baitoan}

\begin{baitoan}[\cite{An_400_BT_Hoa_Hoc_9}, 80., p. 32]
	Hỗn hợp A gồm 2 kim loại {\rm Mg, Zn}. B là dung dịch {\rm\ce{H2SO4}} có nồng độ $x$ {\rm mol{\tt/}L}. Trường hợp 1: Cho {\rm24.3 g} A vào {\rm 2 L} B sinh ra {\rm8.96 L} khí {\rm\ce{H2}}. Trường hợp 2: Cho {\rm24.3 g} A vào {\rm 3 L} B sinh ra {\rm11.2 L} khí {\rm\ce{H2}}. (a) Chứng minh trong trường hợp 1 thì hỗn hợp kim loại chưa tan hết, trong trường hợp 2 acid còn dư. (b) Tính nồng độ $x$ {\rm mol{\tt/}L} của dung dịch B \& $\%$ khối lượng mỗi kim loại trong A (cho biết khí {\rm\ce{H2}} sinh ra ở đktc).
\end{baitoan}

\begin{baitoan}[\cite{An_400_BT_Hoa_Hoc_9}, 81., p. 32]
	Khi cho {\rm0.6 g} 1 kim loại thuộc nhóm IIA tác dụng với nước thì có {\rm0.336 L} hydrogen thoát ra (đktc). Xác định kim loại đó.
\end{baitoan}

\begin{baitoan}[\cite{An_400_BT_Hoa_Hoc_9}, 82., p. 32]
	Hòa tan hỗn hợp {\rm6.4 g CuO} \& {\rm16 g \ce{Fe2O3}} trong {\rm320 mL} dung dịch {\rm HCl 2M}. Sau phản ứng có $m$ {\rm g} chất rắn không tan. Tính $m$.
\end{baitoan}

\begin{baitoan}[\cite{An_400_BT_Hoa_Hoc_9}, 83., p. 32]
	Oxide cao nhất của 1 nguyên tố R ứng với công thức $\rm R_2O_5$. Hợp chất của nó với hydrogen là 1 chất có thành phần khối lượng là $82.35\%$ R \& $17.65\%$ hydrogen. Xác định nguyên tố R.
\end{baitoan}

\begin{baitoan}[\cite{An_400_BT_Hoa_Hoc_9}, 84., p. 32]
	A, B là 2 nguyên tố nằm trong 2 phân nhóm chính kế tiếp nhau trong bảng tuần hoàn. Biết A thuộc nhóm VI \& tổng số proton trong 2 hạt nhân của A, B là $25$, đơn chất A \& B tác dụng được với nhau. Xác định 2 nguyên tố A, B.
\end{baitoan}

\begin{baitoan}[\cite{An_400_BT_Hoa_Hoc_9}, 85., p. 32]
	1 nguyên tố R mà oxide cao nhất của nó chứa $60\%$ oxygen theo khối lượng. Hợp chất khí của R với hydrogen có tỷ khối hơi so với khí hydrogen bằng $17$. (a) Xác định R, công thức oxide của R \& công thức hợp chất khí của R với hydrogen. (b) Viết 1 {\rm PTPƯ} minh họa tính chất hóa học đặc trưng của loại oxide này.
\end{baitoan}

\begin{baitoan}[\cite{An_400_BT_Hoa_Hoc_9}, 86., pp. 32--33]
	Hòa tan hết {\rm11.2 g} hỗn hợp gồm 2 kim loại M (hóa trị $x$) \& M' (hóa trị $y$) trong dung dịch {\rm HCl} rồi sau đó cô cạn dung dịch thu được {\rm39.6 g} muối khan. Tính thể tích khí hydrogen sinh ra?
\end{baitoan}

\begin{baitoan}[\cite{An_400_BT_Hoa_Hoc_9}, 87., p. 33]
	Hòa tan $x$ {\rm g} 1 kim loại M trong {\rm200 g} dung dịch {\rm HCl 7.3\%} (lượng acid vừa đủ) thu được dung dịch A trong đó nồng độ của muối M tạo thành là $11.96\%$ (theo khối lượng). Xác định kim loại M.
\end{baitoan}

\begin{baitoan}[\cite{An_400_BT_Hoa_Hoc_9}, 88., p. 33]
	2 thanh kim loại giống nhau (đều cùng nguyên tố R hóa trị II) \& có cùng khối lượng. Cho thanh thứ nhất vào dung dịch {\rm\ce{Cu(NO3)2}} \& thanh thứ 2 vào dung dịch {\rm\ce{Pb(NO3)2}}. Sau 1 thời gian, khi số mol 2 muối bằng nhau, lấy 2 thanh kim loại đó ra khỏi dung dịch thấy khối lượng thanh thứ nhất giảm đi $0.2\%$ còn khối lượng thanh thứ 2 tăng $28.4\%$. Xác định nguyên tố R.
\end{baitoan}

\begin{baitoan}[\cite{An_400_BT_Hoa_Hoc_9}, 89., p. 33]
	Hòa tan $m$ {\rm g} 1 iron oxide cần {\rm150 mL} dung dịch {\rm HCl 3M}, nếu khử toàn bộ $m$ {\rm g} iron oxide trên bằng {\rm CO} nóng, dư thì thu được {\rm8.4 g} iron. Xác định công thức iron oxide.
\end{baitoan}

\begin{baitoan}[\cite{An_400_BT_Hoa_Hoc_9}, 90., p. 33]
	Cho dòng khí {\rm CO} đi qua {\rm11.6 g} oxide sắt nung nóng, đến phản ứng hoàn toàn nhận được sắt nguyên chất \& lượng khí được hấp thụ bởi dung dịch {\rm\ce{Ca(OH)2}} dư, thu được {\rm20 g} kết tủa. Xác định {\rm CTPT} iron oxide.
\end{baitoan}

\begin{baitoan}[\cite{An_400_BT_Hoa_Hoc_9}, 91., p. 33]
	Cho hỗn hợp A gồm 3 oxide: {\rm\ce{Al2O3,CuO,K2O}}. Tiến hành 3 thí nghiệm:
	\begin{itemize}
		\item Thí nghiệm 1: Nếu cho hỗn hợp A vào nước dư, khuấy kỹ thấy còn {\rm15 g} chất rắn không tan.
		\item Thí nghiệm 2: Nếu cho thêm vào hỗn hợp A 1 lượng {\rm\ce{Al2O3}} bằng $50\%$ lượng {\rm\ce{Al2O3}} trong A ban đầu rồi lại hòa tan vào nước dư. Sau thí nghiệm còn lại {\rm21 g} chất rắn không tan.
		\item Thí nghiệm 3: Nếu cho vào hỗn hợp A 1 lượng {\rm\ce{Al2O3}} bằng $75\%$ lượng {\rm\ce{Al2O3}} trong A, rồi lại hòa tan vào nước dư, thấy còn lại {\rm25 g} chất rắn không tan.
	\end{itemize}
	Tính khối lượng mỗi oxide trong hỗn hợp A.
\end{baitoan}

\begin{baitoan}[\cite{An_400_BT_Hoa_Hoc_9}, 92., p. 33]
	Hợp chất khí với hydrogen của 1 nguyên tố ứng với công t hức $\rm RH_4$. Oxide cao nhất của nó chứa $53.3\%$ oxygen. Xác định nguyên tố R.
\end{baitoan}

\begin{baitoan}[\cite{An_400_BT_Hoa_Hoc_9}, 93., pp. 33--34]
	Khử {\rm3.48 g} 1 oxide của kim loại M cần dùng {\rm1.344 L \ce{H2}} (đktc). Toàn bộ lượng kim loại M thu được cho tác dụng với dung dịch {\rm HCl} dư cho {\rm1.008 L \ce{H2}} (đktc). Tìm {\rm CTPT} của M.
\end{baitoan}

\begin{baitoan}[\cite{An_400_BT_Hoa_Hoc_9}, 94., p. 34]
	Hòa tan hoàn  toàn {\rm5.94 g Al} vào dung dịch {\rm NaOH} dư được khí thứ nhất. Cho {\rm1.896 g \ce{KMnO4}} tác dụng hết với acid {\rm HCl} đặc, dư được khí thứ 2. Nhiệt phân hoàn toàn {\rm12.25\% \ce{KClO3}} có xúc tác thu được khí thứ 3. Cho toàn bộ lượng các khí điều chế ở trên vào 1 bình kín rồi đốt cháy để các phản ứng xảy ra hoàn toàn. Sau đó làm lạnh bình để cho hơi nước ngưng tụ hết \& giả thiết các chất tan hết vào nước thu được dung dịch E. Viết {\rm PTHH} \& tính nồng độ $C\%$ của dung dịch E.
\end{baitoan}

\begin{baitoan}[\cite{An_400_BT_Hoa_Hoc_9}, 95., p. 34]
	Chỉ dùng thêm nước, nhận biết 4 chất rắn: {\rm\ce{Na2O,Al2O3,Fe3O4}, Al} chứa trong các lọ riêng biệt. Viết {\rm PTHH}.
\end{baitoan}

\begin{baitoan}[\cite{An_400_BT_Hoa_Hoc_9}, 96., p. 34]
	Cho {\rm416 g} dung dịch {\rm\ce{BaCl2} 12\%} tác dụng vừa đủ với dung dịch chứa {\rm27.36 g} muối sulfate kim loại A. Sau khi lọc bỏ kết tủa thu được {\rm800 mL} dung dịch {\rm0.2M} của chloride kim loại A. Tìm {\rm CTPT} của muối sulfate kim loại A.
\end{baitoan}

\begin{baitoan}[\cite{An_400_BT_Hoa_Hoc_9}, 97., p. 34]
	Hòa tan {\rm6.75 g} 1 kim loại M chưa rõ hóa trị vào dung dịch acid thì cần {\rm500 mL} dung dịch {\rm HCl 1.5 M}. Xác định kim loại M.
\end{baitoan}

\begin{baitoan}[\cite{An_400_BT_Hoa_Hoc_9}, 98., p. 34]
	Cho {\rm4.15 g} hỗn hợp bột {\rm Fe, Al} tác dụng với {\rm200 mL} dung dịch {\rm\ce{CuSO4} 0.525M}. Khuấy kỹ hỗn hợp để phản ứng xảy ra hoàn toàn. Đem lọc kết tủa A gồm 2 kim loại có khối lượng {\rm7.48 g} \& dung dịch nước lọc. Tìm số mol các kim loại trong hỗn hợp ban đầu \& trong hỗn hợp A.
\end{baitoan}

\begin{baitoan}[\cite{An_400_BT_Hoa_Hoc_9}, 99.a, p. 34]
	Có 3 lọ đựng 3 hỗn hợp dạng bột: {\rm(Al $+$ \ce{Al2O3}), (Fe $+$ \ce{Fe2O3}), (FeO $+$ \ce{Fe2O3})}. Dùng phương pháp hóa học để nhận biết chúng. Viết {\rm PTHH}.
\end{baitoan}

\begin{baitoan}[\cite{An_400_BT_Hoa_Hoc_9}, 99.b, p. 34]
	Trình bày phương pháp tách {\rm\ce{Fe2O3}} ra khỏi hỗn hợp {\rm\ce{Fe2O3,Al2O3,SiO2}} ở dạng bột. Chỉ dùng duy nhất 1 hóa chất.
\end{baitoan}

\begin{baitoan}[\cite{An_400_BT_Hoa_Hoc_9}, 100., p. 34]
	1 quả cầu làm bằng nhôm có khối lượng {\rm1.404 kg} \& thể tích $\rm0.62\ dm^3$. Quả cầu này có chỗ rỗng bên trong không? Nếu có, tính thể tích chỗ rỗng đó, biết $D_{\rm Al} = 2.7$ {\rm kg{\tt/}$\rm dm^3$}.
\end{baitoan}

\begin{baitoan}[\cite{An_400_BT_Hoa_Hoc_9}, 101., pp. 34--35]
	2 cốc đựng dung dịch {\rm HCl} đặt trên 2 đĩa cân A \& B: cân ở trạng thái cân bằng. Cho {\rm5 g \ce{CaCO3}} vào cốc A \& {\rm4.8 g \ce{M2CO3}} (M là kim loại) vào cốc B. Sau khi 2 muối đã tan hoàn toàn, cân trở lại vị trí cân bằng. Xác định kim loại M.
\end{baitoan}

\begin{baitoan}[\cite{An_400_BT_Hoa_Hoc_9}, 102., p. 35]
	Oxide cao nhất của kim loại R chứa $52.94\%$ khối lượng R. Tìm {\rm CTPT} của oxide.
\end{baitoan}

\begin{baitoan}[\cite{An_400_BT_Hoa_Hoc_9}, 103., p. 35]
	Cho 1 thanh {\rm Fe} vào {\rm100 mL} dung dịch chứa 2 muối {\rm\ce{Cu(NO3)2} 0.5M, \ce{AgNO3} 2M}. Sau phản ứng lấy thanh {\rm Fe} ra khỏi dung dịch, rửa sạch, \& làm khô thì khối lượng thanh {\rm Fe} tăng hay giảm? Giải thích.
\end{baitoan}

\begin{baitoan}[\cite{An_400_BT_Hoa_Hoc_9}, 104., p. 35]
	Cho 1 thanh kim loại tác dụng vừa đủ với dung dịch muối nitrate của kim loại hóa trị II, sau 1 thời gian khi khối lượng thanh {\rm Pb} không đổi nữa thì lấy ra khỏi dung dịch thấy khối lượng của nó giảm đi {\rm14.3 g}. Cho thanh sắt có khối lượng {\rm50 g} vào dung dịch sau phản ứng trên, khối lượng thanh sắt không đổi nữa thì lấy ra khỏi dung dịch, rửa sạch, sấy khô cân nặng {\rm65.1 g}. Tìm tên kim loại hóa trị II.
\end{baitoan}

\begin{baitoan}[\cite{An_400_BT_Hoa_Hoc_9}, 105., p. 35]
	Hòa tan hoàn toàn {\rm2 g} hỗn hợp gồm 1 kim loại hóa trị II \& 1 kim loại hóa trị III cần dùng {\rm31.025 g} dung dịch {\rm HCl 20\%}. (a) Tính $V_{\rm H_2}$ thoát ra ở đktc. (b) Tính khối lượng muối khô được tạo thành.
\end{baitoan}

\begin{baitoan}[\cite{An_400_BT_Hoa_Hoc_9}, 106., p. 35]
	Hỗn hợp M gồm oxide của 1 kim loại hóa trị II \& muối carbonate của kim loại đó được hòa tan hết bằng acid {\rm\ce{H2SO4}} loãng vừa đủ tạo ra khí N \& dung dịch L. Đem cô cạn dung dịch L thu được 1 lượng muối khan bằng $168\%$ khối lượng M. Xác định kim loại hóa trị II, biết khí N nặng bằng $44\%$ khối lượng của M.
\end{baitoan}

\begin{baitoan}[\cite{An_400_BT_Hoa_Hoc_9}, 107., p. 35]
	Nhúng 1 lá {\rm Al} vào dung dịch {\rm\ce{CuSO4}}, sau 1 thời gian lấy lá nhôm ra khỏi dung dịch thì thấy khối lượng dung dịch giảm {\rm1.38 g}. Khối lượng {\rm Al} đã phản ứng là bao nhiêu?
\end{baitoan}

\begin{baitoan}[\cite{An_400_BT_Hoa_Hoc_9}, 108., p. 35]
	Cho dòng {\rm\ce{H2}} dư qua {\rm2.36 g} hỗn hợp {\rm Fe, FeO, \ce{Fe2O3}} được đốt nóng. Sau phản ứng trong ống còn lại {\rm1.96 g Fe}. Nếu cho {\rm2.36 g} hỗn hợp đầu tác dụng với dung dịch {\rm\ce{CuSO4}} đến phản ứng hoàn toàn, lọc lấy chất rắn làm khô cân nặng {\rm2.48 g}. Tính khối lượng từng chất trong hỗn hợp.
\end{baitoan}

\begin{baitoan}[\cite{An_400_BT_Hoa_Hoc_9}, 109., p. 35]
	Cho {\rm16 g} hỗn hợp {\rm MgO, \ce{Fe2O3}} tan hết trong {\rm245 g} dung dịch {\rm\ce{H2SO4} 20\%}. Sau phản ứng trung hòa acid còn dư bằng {\rm50 g} dung dịch {\rm NaOH 24\%}. Tính khối lượng mỗi oxide.
\end{baitoan}

\begin{baitoan}[\cite{An_400_BT_Hoa_Hoc_9}, 110., p. 36]
	Cho {\rm31.4 g} hỗn hợp {\rm Mg, Al, Fe} phản ứng với dung dịch {\rm HCl} dư thoát ra {\rm17.04 L \ce{H2}} (đktc) \& dung dịch A. Tính khối lượng mỗi kim loại biết thể tích {\rm\ce{H2}} thoát ra do {\rm Al} gấp $2$ lần thể tích {\rm\ce{H2}} thoát ra do {\rm Mg}.
\end{baitoan}

\begin{baitoan}[\cite{An_400_BT_Hoa_Hoc_9}, 111., p. 36]
	Cho {\rm3.06 g} oxide {\rm\ce{M_xO_y}} của kim loại M có hóa trị không đổi (hóa trị từ I đến III) tan trong {\rm\ce{HNO3}} dư thu được {\rm5.22 g} muối. Xác định {\rm CTPT} của oxide {\rm\ce{M_xO_y}}.
\end{baitoan}

\begin{baitoan}[\cite{An_400_BT_Hoa_Hoc_9}, 112., p. 36]
	Cho {\rm2 g} hỗn hợp {\rm Fe} \& kim loại hóa trị II vào dung dịch {\rm HCl} có dư thì thu được {\rm1.12 L \ce{H2}} (đktc). Mặt khác, nếu hòa tan {\rm4.8 g} kim loại hóa trị II đó cần chưa đến {\rm500 mL} dung dịch {\rm HCl}. Xác định kim loại hóa trị II.
\end{baitoan}

\begin{baitoan}[\cite{An_400_BT_Hoa_Hoc_9}, 113., p. 36]
	Hòa tan 1 lượng muối carbonate của 1 kim loại hóa trị II bằng dung dịch {\rm\ce{H2SO4} 16\%}. Sau khi khí không thoát ra nữa thì thu được dung dịch chứa {\rm22.2\%} muối sulfate. Tìm {\rm CTPT} của muối carbonate.
\end{baitoan}

\begin{baitoan}[\cite{An_400_BT_Hoa_Hoc_9}, 114., p. 36]
	Nung nóng {\rm11.6 g} 1 iron oxide bằng khí {\rm CO} nóng, dư đến hoàn toàn thu được iron nguyên chất \& lượng khí được hấp thụ bởi dung dịch {\rm\ce{Ca(OH)2}} dư tách ra {\rm20 g} kết tủa. Tìm {\rm CTPT} của iron oxide.
\end{baitoan}

\begin{baitoan}[\cite{An_400_BT_Hoa_Hoc_9}, 115., p. 36]
	Khử $m$ {\rm g} iron oxide bằng khí hydrogen nóng, dư. Cho hơi nước tạo ra được hấp thụ bằng {\rm100 g} acid {\rm\ce{H2SO4} 98\%} thì nồng độ acid giảm đi $3.405\%$. Chất rắn thu được sau phản ứng khử được hòa tan bằng {\rm\ce{H2SO4}} loãng thoát ra {\rm3.36 L \ce{H2}} (đktc). (a) Viết {\rm PTHH}. (b) Tìm {\rm CTPT} của iron oxide.
\end{baitoan}

\begin{baitoan}[\cite{An_400_BT_Hoa_Hoc_9}, 116., p. 36]
	Hòa tan 1 lượng muối carbonate của 1 kim loại hóa trị III bằng dung dịch {\rm\ce{H2SO4} 16\%}. Sau khi khí không thoát ra nữa, được dung dịch chứa $20\%$ muối sulfate tan. Xác định tên kim loại hóa trị III.
\end{baitoan}

\begin{baitoan}[\cite{An_400_BT_Hoa_Hoc_9}, 117., p. 36]
	1 dung dịch A có chứa {\rm NaOH} \& {\rm0.3 mol \ce{NaAlO2}}. Cho {\rm1 mol HCl} vào A thu được {\rm15.6 g} kết tủa. Tính khối lượng {\rm NaOH} trong dung dịch A.
\end{baitoan}

\begin{baitoan}[\cite{An_400_BT_Hoa_Hoc_9}, 118., p. 36]
	Bằng phương pháp hóa học, nhận biết các hỗn hợp: {\rm(Fe $+$ \ce{Fe2O3}), (Fe $+$ FeO), (FeO $+$ \ce{Fe2O3}}).
\end{baitoan}

\begin{baitoan}[\cite{An_400_BT_Hoa_Hoc_9}, 119., pp. 36--37]
	Cho {\rm100 g} hỗn hợp 2 muối chloride của cùng 1 kim loại R (có hóa trị II \& III) tác dụng với {\rm KOH} dư. Kết tủa hydroxide hóa trị II bằng {\rm19.8 g} còn khối lượng chloride kim loại R hóa trị II bằng $\frac{1}{2}$ khối lượng mol của R. Tìm kim loại R.
\end{baitoan}

\begin{baitoan}[\cite{An_400_BT_Hoa_Hoc_9}, 120., p. 37]
	Hòa tan {\rm1.6 g} oxide kim loại hóa trị III bằng {\rm100 g} dung dịch {\rm\ce{H2SO4}} loãng. Khi thêm vào hỗn hợp sau phản ứng 1 lượng {\rm\ce{MgCO3}} vừa đủ còn thấy thoát ra $\rm0.224\ dm^3$ {\rm\ce{CO2}} (đktc). Sau đó cô cạn dung dịch thu được {\rm9.36 g} muối sulfate khô. Tìm nồng độ $\%$ {\rm\ce{H2SO4}} \& tên kim loại hóa trị III.
\end{baitoan}

\begin{baitoan}[\cite{An_400_BT_Hoa_Hoc_9}, 121., p. 37]
	Khử hoàn toàn {\rm4.06 g} 1 oxide kim loại bằng {\rm CO} ở nhiệt độ cao thành kim loại. Dẫn toàn bộ khí sinh ra vào bình đựng {\rm\ce{Ca(OH)2}} dư, thấy tạo thành {\rm7 g} kết tủa. Nếu lấy lượng kim loại sinh ra hòa tan hết vào dung dịch {\rm HCl} dư thì thu được {\rm1.176 L \ce{H2}} (đktc). (a) Xác định {\rm CTPT} oxide kim loại. (b) Cho {\rm4.06 g} oxide kim loại trên tác dụng hoàn toàn với {\rm500 mL} dung dịch {\rm\ce{H2SO4}} đặc, nóng (dư) thu được dung dịch X \& khí {\rm\ce{SO2}} bay ra. Xác định nồng độ {\rm mol{\tt/}L} của muối trong dung dịch X (coi thể tích dung dịch thay đổi không đáng kể trong quá trình phản ứng).
\end{baitoan}

\begin{baitoan}[\cite{An_400_BT_Hoa_Hoc_9}, 122., p. 37]
	Sắp xếp các kim loại sau theo tính hoạt động hóa học tăng dần: {\rm Na, Al, Zn, Pb, Fe, Sn, Ag, Cu}.
\end{baitoan}

\begin{baitoan}[\cite{An_400_BT_Hoa_Hoc_9}, 123., p. 37]
	Bổ túc phản ứng: {\rm\ce{Fe_xO_y + H2 -> A + B}}.
\end{baitoan}

\begin{baitoan}[\cite{An_400_BT_Hoa_Hoc_9}, 124., p. 37]
	Cho hỗn hợp {\rm Al, Fe} tác dụng với hỗn hợp dung dịch chứa {\rm\ce{AgNO3,Cu(NO3)2}} thu được dung dịch B \& chất rắn D gồm 3 kim loại. Cho D tác dụng với dung dịch {\rm HCl} dư có khí bay lên. Tìm thành phần chất rắn D.
\end{baitoan}

\begin{baitoan}[\cite{An_400_BT_Hoa_Hoc_9}, 125., pp. 37--38]
	{\rm Đ{\tt/}S?} (a) Sự phá hủy kim loại hay hợp kim dưới tác dụng hóa học của môi trường xung quanh gọi là {\rm sự ăn mòn kim loại}. (b) Ăn mòn kim loại là sự phá hủy kim loại bởi chất khí hay hơi nước ở nhiệt độ cao. (c) Ăn mòn kim loại là sự phá hủy kim loại do kim loại tiếp xúc với dung dịch acid tạo dòng điện.
\end{baitoan}

\begin{baitoan}[\cite{An_400_BT_Hoa_Hoc_9}, 126., p. 38]
	Hợp chất nào sau đây phản ứng được với chlorine? Viết {\rm PTHH} nếu có: {\rm NaCl, NaOH, \ce{CaCO3,H2O}}.
\end{baitoan}

\begin{baitoan}[\cite{An_400_BT_Hoa_Hoc_9}, 127., p. 38]
	Quá trình nào không sinh ra khí carbonic? {\sf A.} Đối cháy khí đốt tự nhiên. {\sf B.} Sản xuất vôi sống. {\sf C.} Sản xuất vôi tôi. {\sf D.} Quang hợp của cây xanh.
\end{baitoan}

\begin{baitoan}[\cite{An_400_BT_Hoa_Hoc_9}, 128., p. 38]
	Hàm lượng khí {\rm\ce{CO2}} trong khí quyển của Trái Đất gần như không đổi là vì: {\sf A.} {\rm\ce{CO2}} không có khả năng tác dụng với chất khí khác trong không khí. {\sf B.} Trong quá trình quang hợp cây xanh hấp thụ khí {\rm\ce{CO2}}, mặt khác lượng {\rm\ce{CO2}} được sinh ra do đốt cháy nhiên liệu, sự hô hấp của người \& động vật. {\sf C.} {\rm\ce{CO2}} hòa tan trong nước mưa. {\sf D.} {\rm\ce{CO2}} bị phân hủy bởi nhiệt.
\end{baitoan}

\begin{baitoan}[\cite{An_400_BT_Hoa_Hoc_9}, 129., p. 38]
	Khí carbon monoxide {\rm CO} có tính chất độc là do khả năng kết hợp với hemoglobin trong máu làm mất khả năng vận chuyển oxygen của máu. Trong trường hợp nào sau đây gây tử vong do ngộ độc {\rm CO}? {\sf A.} Dùng bình gas để nấu nướng ở ngoài trời. {\sf B.} Đốt bếp, lò trong nhà không được thông gió tốt. {\sf C.} Nổ (chạy) máy ôtô trong nhà xe đóng kín.
\end{baitoan}

\begin{baitoan}[\cite{An_400_BT_Hoa_Hoc_9}, 130., p. 38]
	Khí {\rm CO, \ce{CO2}} bị coi là chất làm ô nhiễm môi trường vì: {\sf A.} Nồng độ $\%V$ {\rm CO} cho phép trong không khí là $10$--$20$ phần triệu, nếu đến $50$ phần triệu sẽ có hại cho não. {\sf B.} {\rm\ce{CO2}} tuy không độc nhưng gây hiệu ứng nhà kính làm Trái Đất nóng lên. {\sf C.} {\rm\ce{CO2}}  cần cho cây xanh quang hợp nên không gây ô nhiễm. {\sf D. A, B} đúng.
\end{baitoan}

\begin{baitoan}[\cite{An_400_BT_Hoa_Hoc_9}, 131., p. 39]
	Dung dịch chất nào sau đây không thể chứa trong bình thủy tinh \& giải thích? {\sf A.} {\rm\ce{HNO3}}. {\sf B.} {\rm\ce{H2SO4}}. {\sf C.} {\rm HCl}. {\sf D.} {\rm HF}.
\end{baitoan}

\begin{baitoan}[\cite{An_400_BT_Hoa_Hoc_9}, 132., p. 39]
	Hòa tan 1 lượng sodium kim loại vào nước, thu được dung dịch X \& $a$ mol khí bay ra. Cho $b$ mol khí {\rm\ce{CO2}} hấp thụ hoàn toàn vào dung dịch X, được dung dịch Y. Cho biết các chất tan trong Y theo mối quan hệ giữa $a,b$.
\end{baitoan}

\begin{baitoan}[\cite{An_400_BT_Hoa_Hoc_9}, 133., p. 39]
	Để khử {\rm6.4 g} 1 oxide kim loại cần {\rm2.688 L} khí {\rm\ce{H2}} (đktc). Nếu lấy lượng kim loại đó cho tác dụng với dung dịch {\rm HCl} dư thì giải phóng {\rm1.792 L} khí {\rm\ce{H2}} (đktc). Tìm tên kim loại.
\end{baitoan}

\begin{baitoan}[\cite{An_400_BT_Hoa_Hoc_9}, 134., p. 39]
	Cho oxide {\rm\ce{M_xO_y}} của kim loại M có hóa trị không đổi. Xác định công thức của oxide này biết {\rm3.96 g} oxide nguyên chất tan trong dung dịch {\rm\ce{HNO3}} dư thu được {\rm5.22 g} muối.
\end{baitoan}

\begin{baitoan}[\cite{An_400_BT_Hoa_Hoc_9}, 135., p. 39]
	Cho {\rm16 g} iron oxide có công thức {\rm\ce{Fe_xO_y}} tác dụng với {\rm120 mL} dung dịch {\rm HCl} thì thu được {\rm32.5 g} muối khan. Tìm {\rm CTPT} iron oxide \& nồng độ {\rm mol{\tt/}L} của dung dịch {\rm HCl}.
\end{baitoan}

\begin{baitoan}[\cite{An_400_BT_Hoa_Hoc_9}, 136., p. 39]
	Oxide của kim loại R ở mức hóa trị thấp nhất chứa $22.56\%$ oxygen, cũng oxide của kim loại đó ở mức hóa trị cao nhất chứa $50.48\%$ oxygen. Tìm nguyên tử khối của kim loại đó.
\end{baitoan}

\begin{baitoan}[\cite{An_400_BT_Hoa_Hoc_9}, 137., p. 39]
	Nguyên tố X có thể tạo ra 2 loại oxide mà trong mỗi oxide hàm lượng $\%$ của X là $40\%$ \& $50\%$. Xác định nguyên tố X.
\end{baitoan}

\begin{baitoan}[\cite{An_400_BT_Hoa_Hoc_9}, 138., p. 39]
	Khi oxy hóa {\rm2 g} 1 nguyên tố hóa học có hóa trị IV bằng oxygen, thu được {\rm2.54 g} oxide. Xác định công thức oxide.
\end{baitoan}

\begin{baitoan}[\cite{An_400_BT_Hoa_Hoc_9}, 139.a, p. 39]
	Kim loại iron tạo được 3 oxide: {\rm FeO, \ce{Fe2O3,Fe3O4}}. Nếu hàm lượng {\rm Fe} trong oxide là $70\%$ (theo khối lượng) thì trong các oxide trên, oxide nào phù hợp?
\end{baitoan}

\begin{baitoan}[\cite{An_400_BT_Hoa_Hoc_9}, 139.b, p. 39]
	Hợp kim của copper \& aluminium được kết cấu theo tỷ lệ $12.3\%$ aluminium. Tìm {\rm CTPT}.
\end{baitoan}

\begin{baitoan}[\cite{An_400_BT_Hoa_Hoc_9}, 140.a, p. 39]
	Cho {\rm5.4 g} 1 kim loại hóa trị III tác dụng với chlorine có dư thu được {\rm26.7 g} muối. Xác định kim loại đem phản ứng.
\end{baitoan}

\begin{baitoan}[\cite{An_400_BT_Hoa_Hoc_9}, 140.b, p. 39]
	Cho {\rm5.6 g} 1 oxide kim loại tác dụng vừa đủ với acid {\rm HCl} cho {\rm11.1 g} muối chlorine của kim loại đó. Cho biết tên kim loại.
\end{baitoan}

\begin{baitoan}[\cite{An_400_BT_Hoa_Hoc_9}, 141., p. 39]
	Thêm từ từ dung dịch {\rm\ce{H2SO4} 10\%} vào ly đựng 1 muối carbonate của kim loại hóa trị I, cho tới khi vừa thoát hết khí {\rm\ce{CO2}} thì thu được dung dịch muối sulfate có nồng độ $13.63\%$. Xác định {\rm CTPT} muối carbonate.
\end{baitoan}

\begin{baitoan}[\cite{An_400_BT_Hoa_Hoc_9}, 142., p. 40]
	Cho {\rm1 g} iron chloride chưa rõ hóa trị của iron vào 1 dung dịch {\rm\ce{AgNO3}} dư, được 1 chất kết tủa trắng, sau khi sấy khô có khối lượng {\rm2.65 g}. Xác định hóa trị của iron \& viết {\rm PTHH}.
\end{baitoan}

\begin{baitoan}[\cite{An_400_BT_Hoa_Hoc_9}, 143., p. 40]
	Để hòa tan hoàn toàn {\rm8 g} 1 oxide kim loại cần dùng {\rm300 mL} dung dịch {\rm HCl 1M}. Xác định {\rm CTPT} oxide kim loại.
\end{baitoan}

\begin{baitoan}[\cite{An_400_BT_Hoa_Hoc_9}, 144., p. 40]
	
\end{baitoan}

\begin{baitoan}[\cite{An_400_BT_Hoa_Hoc_9}, 145., p. 40]
	
\end{baitoan}

\begin{baitoan}[\cite{An_400_BT_Hoa_Hoc_9}, 146., p. 40]
	
\end{baitoan}

\begin{baitoan}[\cite{An_400_BT_Hoa_Hoc_9}, 147., p. 40]
	
\end{baitoan}

\begin{baitoan}[\cite{An_400_BT_Hoa_Hoc_9}, 148., p. 40]
	
\end{baitoan}

\begin{baitoan}[\cite{An_400_BT_Hoa_Hoc_9}, 149., p. 40]
	
\end{baitoan}

\begin{baitoan}[\cite{An_400_BT_Hoa_Hoc_9}, 150., pp. 40--41]
	
\end{baitoan}

\begin{baitoan}[\cite{An_400_BT_Hoa_Hoc_9}, 151., p. 41]
	
\end{baitoan}

%------------------------------------------------------------------------------%

\printbibliography[heading=bibintoc]

\end{document}