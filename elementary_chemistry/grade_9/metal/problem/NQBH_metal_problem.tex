\documentclass{article}
\usepackage[backend=biber,natbib=true,style=alphabetic,maxbibnames=50]{biblatex}
\addbibresource{/home/nqbh/reference/bib.bib}
\usepackage[utf8]{vietnam}
\usepackage{tocloft}
\renewcommand{\cftsecleader}{\cftdotfill{\cftdotsep}}
\usepackage[colorlinks=true,linkcolor=blue,urlcolor=red,citecolor=magenta]{hyperref}
\usepackage{amsmath,amssymb,amsthm,float,graphicx,mathtools,diagbox,tikz,tipa}
\usepackage[version=4]{mhchem}
\allowdisplaybreaks
\newtheorem{assumption}{Assumption}
\newtheorem{baitoan}{Bài toán}
\newtheorem{cauhoi}{Câu hỏi}
\newtheorem{conjecture}{Conjecture}
\newtheorem{corollary}{Corollary}
\newtheorem{dangtoan}{Dạng toán}
\newtheorem{definition}{Definition}
\newtheorem{dinhly}{Định lý}
\newtheorem{dinhnghia}{Định nghĩa}
\newtheorem{example}{Example}
\newtheorem{ghichu}{Ghi chú}
\newtheorem{hequa}{Hệ quả}
\newtheorem{hypothesis}{Hypothesis}
\newtheorem{lemma}{Lemma}
\newtheorem{luuy}{Lưu ý}
\newtheorem{nhanxet}{Nhận xét}
\newtheorem{notation}{Notation}
\newtheorem{note}{Note}
\newtheorem{principle}{Principle}
\newtheorem{problem}{Problem}
\newtheorem{proposition}{Proposition}
\newtheorem{question}{Question}
\newtheorem{remark}{Remark}
\newtheorem{theorem}{Theorem}
\newtheorem{thinghiem}{Thí nghiệm}
\newtheorem{vidu}{Ví dụ}
\usepackage[left=1cm,right=1cm,top=5mm,bottom=5mm,footskip=4mm]{geometry}

\title{Problem: Metal -- Bài Tập: Kim Loại}
\author{Nguyễn Quản Bá Hồng\footnote{Independent Researcher, Ben Tre City, Vietnam\\e-mail: \texttt{nguyenquanbahong@gmail.com}; website: \url{https://nqbh.github.io}.}}
\date{\today}

\begin{document}
\maketitle
\begin{abstract}
	
\end{abstract}
\setcounter{secnumdepth}{4}
\setcounter{tocdepth}{3}
\tableofcontents

%------------------------------------------------------------------------------%

\begin{baitoan}[\cite{An_Hoa_Hoc_nang_cao_8_9}, 1., p. 89]
	(a) Sắt là nguyên tố có nhiều hóa trị, phổ biến là (II) \& (III). Viết các {\rm PTHH} minh họa. (b) Cho các kim loại {\rm Cu, Al, Fe, Ag}. Các kim loại nào tác dụng với acid hydrochloric? Các kim loại nào tác dụng được với dung dịch {\rm\ce{CuSO4}}? Dung dịch {\rm\ce{AgNO3}}? Viết các {\rm PTHH} tương ứng.
\end{baitoan}

\begin{baitoan}[\cite{An_Hoa_Hoc_nang_cao_8_9}, 2., p. 90]
	Có thể điều chế bao nhiêu {\rm kg} aluminium từ $1$ tấn quặng nhôm chứa {\rm95\%} aluminium oxide biết hiệu suất phản ứng là {\rm98\%}.
\end{baitoan}

\begin{baitoan}[\cite{An_Hoa_Hoc_nang_cao_8_9}, 3., p. 90]
	(a) Tại sao không nên dùng chậu nhôm đựng nước vôi. (b) Viết PTHH giữa {\rm\ce{Fe3O4}} với {\rm\ce{H2SO4}}.
\end{baitoan}

\begin{baitoan}[\cite{An_Hoa_Hoc_nang_cao_8_9}, 4., p. 91]
	Cho {\rm1.38 g} 1 kim loại hóa trị (I) tác dụng hết với nước cho {\rm0.2 g} hydrogen. Xác định kim loại đó.
\end{baitoan}

\begin{baitoan}[\cite{An_Hoa_Hoc_nang_cao_8_9}, 5., p. 91]
	Trong quặng boxit trung bình có {\rm50\%} aluminium oxide. Kim loại luyện được từ oxide đó còn chứa {\rm1.5\%} tạp chất. Tính lượng nhôm nguyên chất điều chế được từ $0.5$ tấn quặng boxit.
\end{baitoan}

\begin{baitoan}[\cite{An_Hoa_Hoc_nang_cao_8_9}, 6., p. 92]
	Cho bản kẽm có khối lượng {\rm50 g} vào dung dịch đồng sulfate. Sau 1 thời gian phản ứng kết thúc thì khối lượng bản kẽm là {\rm49.82 g}. Tính: (a) Khối lượng kẽm đã tác dụng. (b) Khối lượng đồng sulfate có trong dung dịch.
\end{baitoan}

\begin{baitoan}[\cite{An_Hoa_Hoc_nang_cao_8_9}, 7., p. 92]
	Để thu được $1000$ tấn gang chứa {\rm95\%} sắt, {\rm5\%} carbon (các nguyên tố khác chiếm 1 lượng không đáng kể) thì theo lý thuyết phải cần bao nhiêu tấn {\rm\ce{Fe2O3}} \& bao nhiêu tấn than cốc. 
\end{baitoan}

\begin{baitoan}[\cite{An_Hoa_Hoc_nang_cao_8_9}, 8., p. 93]
	Cho {\rm5.4 g} 1 kim loại tác dụng với chlorine có dư thu được {\rm26.7  g} muối. Xác định kim loại đem phản ứng, biết kim loại có hóa trị từ I $\to$ III.
\end{baitoan}

\begin{baitoan}[\cite{An_Hoa_Hoc_nang_cao_8_9}, 9., p. 94]
	1 nguyên tố R có oxide cao nhất chiếm {\rm60\%} oxi theo khối lượng. Hợp chất khí của R với hydrogen có tỷ khối hơi so với không khí là $1.172$. Xác định công thức oxide của R.
\end{baitoan}

\begin{baitoan}[\cite{An_Hoa_Hoc_nang_cao_8_9}, 10., p. 94, TS PTNK ĐH KHTN Tp. HCM 1998]
	1 hỗn hợp X gồm kim loại {\rm M} ({\rm M} có hóa trị II \& III) \& oxide $\rm M_xO_y$ của kim loại ấy. Khối lượng hỗn hợp X là {\rm27.2 g}. Khi cho X tác dụng với {\rm0.8 L HCl 2M} thì hỗn hợp X tan hết cho dung dịch A cần {\rm0.6 L} dung dịch {\rm NaOH 1M}. Xác định {\rm M}, $\rm M_xO_y$, \& {\rm\%M, \%$\rm M_xO_y$} (theo khối lượng) trong hỗn hợp X. Biết trong 2 chất này có 1 chất có số mol bằng $2$ lần số mol chất kia.
\end{baitoan}

\begin{baitoan}[\cite{An_Hoa_Hoc_nang_cao_8_9}, 11., p. 94]
	A là kim loại hóa trị II. Lấy 2 thanh A cùng khối lượng. Thanh thứ nhất nhúng vào dung dịch {\rm\ce{CuSO4}}, sau 1 thời gian khối lượng giảm {\rm3.6\%}. Thanh thứ 2 nhúng vào dung dịch {\rm\ce{HgSO4}}, sau 1 thời gian khối lượng tăng {\rm6.675\%}. Nồng độ mol của 2 dung dịch {\rm\ce{CuSO4,HgSO4}} giảm cùng 1 số mol như nhau. Xác định tên kim loại A.
\end{baitoan}

\begin{baitoan}[\cite{An_Hoa_Hoc_nang_cao_8_9}, 12., p. 94]
	Khử {\rm3.48 g} 1 oxide của kim loại M cần dùng {\rm1.344 L} khí {\rm\ce{H2}} (ở đktc). Tìm {\rm CTPT} của oxide kim loại.
\end{baitoan}

\begin{baitoan}[\cite{An_Hoa_Hoc_nang_cao_8_9}, 13., p. 94]
	Cho hỗn hợp {\rm Al, Fe} tác dụng với hỗn hợp dung dịch chứa {\rm\ce{AgNO3,Cu(NO3)2}} thu được dung dịch B \& chất rắn D gồm 3 kim loại. Cho D tác dụng với dung dịch {\rm HCl} dư có khí bay ra. Xác định thành phần chất rắn D.
\end{baitoan}

\begin{baitoan}[\cite{An_Hoa_Hoc_nang_cao_8_9}, 14., p. 94]
	Cho {\rm2 g} hỗn hợp {\rm Fe} \& kim loại hóa trị II vào dung dịch {\rm HCl} có dư thì thu được {\rm1.12 L \ce{H2}} (đktc). Mặt khác, nếu hòa tan {\rm4.8 g} kim loại hóa trị II đó cần chưa đến {\rm500 mL} dung dịch {\rm HCl}. Xác định kim loại hóa trị II.
\end{baitoan}

\begin{baitoan}[\cite{An_Hoa_Hoc_nang_cao_8_9}, 15., pp. 94--95]
	X là hỗn hợp 2 kim loại {\rm Mg, Zn}. Y là dung dịch {\rm\ce{H2SO4}} chưa rõ nồng độ.
	\begin{itemize}
		\item Thí nghiệm 1: Cho {\rm24.3 g} X vào {\rm2 L} Y, sinh ra {\rm8.96 L \ce{H2}}.
		\item Thí nghiệm 2: Cho {\rm24.3 g} X vào {\rm3 L} Y, sinh ra {\rm11.2 L \ce{H2}}.
	\end{itemize}
	Lập luận chứng tỏ trong thí nghiệm 1 thì X chưa tan hết, trong thí nghiệm 2 thì X tan hết.
\end{baitoan}

\begin{baitoan}[\cite{An_Hoa_Hoc_nang_cao_8_9}, 16., p. 95]
	Cho {\rm8 g \ce{Fe_xO_y}} tác dụng với $V$ {\rm mL} dung dịch {\rm HCl 2M} lấy dư {\rm25\%} với lượng cần thiết. Đun nóng khan dung dịch sau phản ứng thu được {\rm16.25 g} muối khan. (a) Xác định {\rm CTPT} {\rm\ce{Fe_xO_y}}. (b) Tính $V$.
\end{baitoan}

\begin{baitoan}[\cite{An_Hoa_Hoc_nang_cao_8_9}, 17., p. 95]
	Nung nóng kim loại X trong không khí đến khối lượng không đổi được chất rắn Y. Khối lượng của X bằng $\frac{7}{10}$ khối lượng Y. Tìm {\rm CTPT} của chất rắn Y.
\end{baitoan}

\begin{baitoan}[\cite{An_Hoa_Hoc_nang_cao_8_9}, 18., p. 95]
	Cho {\rm3.87 g} hỗn hợp A gồm {\rm Mg, Al} vào {\rm250 mL} dung dịch X chứa {\rm HCl 1M} \& {\rm\ce{H2SO4} 0.5M} được dung dịch B \& {\rm4.368 L \ce{H2}} (ở đktc). Biện luận xem hỗn hợp A còn dư hay đã phản ứng hết.
\end{baitoan}

\begin{baitoan}[\cite{An_Hoa_Hoc_nang_cao_8_9}, 19., p. 95]
	Nguyên tố R tạo thành hợp chất khí với hydrogen có {\rm CTHH} là {\rm\ce{RH4}}. Trong hợp chất cao nhất với oxide chứa {\rm72.73\%} là oxygen. (a) Xác định tên nguyên tố R. (b) Cho biết vị trí của R trong bảng tuần hoàn.
\end{baitoan}

\begin{baitoan}[\cite{An_Hoa_Hoc_nang_cao_8_9}, 20., p. 95]
	{\rm Đ{\tt/}S?} {\sf A.} Trong cùng 1 chu kỳ, khi điện tích hạt nhân tăng dần, tính phi kim tăng dần \& bán kính nguyên tử giảm dần. {\sf B.} Trong chu kỳ, theo chiều tăng điện tích hạt nhân, tính acid của các oxide \& hydroxide giảm dần. {\sf C.} Trong cùng 1 nhóm, khi điện tích hạt nhân tăng dần thì tính base của các oxide \& hydrogen tăng dần. {\sf D. B} sai.
\end{baitoan}

\begin{baitoan}[\cite{An_400_BT_Hoa_Hoc_9}, 79., p. 32]
	Hòa tan {\rm13.2 g} hỗn hợp A gồm 2 kim loại có cùng hóa trị vào {\rm400 mL} dung dịch {\rm HCl 1.5M}. Cô cạn dung dịch sau phản ứng thu được {\rm32.7 g} hỗn hợp muối khan. (a) Chứng minh hỗn hợp A không tan hết. (b) Tính thể tích hydrogen sinh ra.
\end{baitoan}

\begin{baitoan}[\cite{An_400_BT_Hoa_Hoc_9}, 80., p. 32]
	Hỗn hợp A gồm 2 kim loại {\rm Mg, Zn}. B là dung dịch {\rm\ce{H2SO4}} có nồng độ $x$ {\rm mol{\tt/}L}. Trường hợp 1: Cho {\rm24.3 g} A vào {\rm 2 L} B sinh ra {\rm8.96 L} khí {\rm\ce{H2}}. Trường hợp 2: Cho {\rm24.3 g} A vào {\rm 3 L} B sinh ra {\rm11.2 L} khí {\rm\ce{H2}}. (a) Chứng minh trong trường hợp 1 thì hỗn hợp kim loại chưa tan hết, trong trường hợp 2 acid còn dư. (b) Tính nồng độ $x$ {\rm mol{\tt/}L} của dung dịch B \& {\rm\%} khối lượng mỗi kim loại trong A (cho biết khí {\rm\ce{H2}} sinh ra ở đktc).
\end{baitoan}

\begin{baitoan}[\cite{An_400_BT_Hoa_Hoc_9}, 81., p. 32]
	Khi cho {\rm0.6 g} 1 kim loại thuộc nhóm IIA tác dụng với nước thì có {\rm0.336 L} hydrogen thoát ra (đktc). Xác định kim loại đó.
\end{baitoan}

\begin{baitoan}[\cite{An_400_BT_Hoa_Hoc_9}, 82., p. 32]
	Hòa tan hỗn hợp {\rm6.4 g CuO} \& {\rm16 g \ce{Fe2O3}} trong {\rm320 mL} dung dịch {\rm HCl 2M}. Sau phản ứng có $m$ {\rm g} chất rắn không tan. Tính $m$.
\end{baitoan}

\begin{baitoan}[\cite{An_400_BT_Hoa_Hoc_9}, 83., p. 32]
	Oxide cao nhất của 1 nguyên tố R ứng với công thức $\rm R_2O_5$. Hợp chất của nó với hydrogen là 1 chất có thành phần khối lượng là {\rm82.35\%} R \& {\rm17.65\%} hydrogen. Xác định nguyên tố R.
\end{baitoan}

\begin{baitoan}[\cite{An_400_BT_Hoa_Hoc_9}, 84., p. 32]
	A, B là 2 nguyên tố nằm trong 2 phân nhóm chính kế tiếp nhau trong bảng tuần hoàn. Biết A thuộc nhóm VI \& tổng số proton trong 2 hạt nhân của A, B là $25$, đơn chất A \& B tác dụng được với nhau. Xác định 2 nguyên tố A, B.
\end{baitoan}

\begin{baitoan}[\cite{An_400_BT_Hoa_Hoc_9}, 85., p. 32]
	1 nguyên tố R mà oxide cao nhất của nó chứa {\rm60\%} oxygen theo khối lượng. Hợp chất khí của R với hydrogen có tỷ khối hơi so với khí hydrogen bằng $17$. (a) Xác định R, công thức oxide của R \& công thức hợp chất khí của R với hydrogen. (b) Viết 1 {\rm PTPƯ} minh họa tính chất hóa học đặc trưng của loại oxide này.
\end{baitoan}

\begin{baitoan}[\cite{An_400_BT_Hoa_Hoc_9}, 86., pp. 32--33]
	Hòa tan hết {\rm11.2 g} hỗn hợp gồm 2 kim loại M (hóa trị $x$) \& M' (hóa trị $y$) trong dung dịch {\rm HCl} rồi sau đó cô cạn dung dịch thu được {\rm39.6 g} muối khan. Tính thể tích khí hydrogen sinh ra?
\end{baitoan}

\begin{baitoan}[\cite{An_400_BT_Hoa_Hoc_9}, 87., p. 33]
	Hòa tan $x$ {\rm g} 1 kim loại M trong {\rm200 g} dung dịch {\rm HCl 7.3\%} (lượng acid vừa đủ) thu được dung dịch A trong đó nồng độ của muối M tạo thành là {\rm11.96\%} (theo khối lượng). Xác định kim loại M.
\end{baitoan}

\begin{baitoan}[\cite{An_400_BT_Hoa_Hoc_9}, 88., p. 33]
	2 thanh kim loại giống nhau (đều cùng nguyên tố R hóa trị II) \& có cùng khối lượng. Cho thanh thứ nhất vào dung dịch {\rm\ce{Cu(NO3)2}} \& thanh thứ 2 vào dung dịch {\rm\ce{Pb(NO3)2}}. Sau 1 thời gian, khi số mol 2 muối bằng nhau, lấy 2 thanh kim loại đó ra khỏi dung dịch thấy khối lượng thanh thứ nhất giảm đi {\rm0.2\%} còn khối lượng thanh thứ 2 tăng {\rm28.4\%}. Xác định nguyên tố R.
\end{baitoan}

\begin{baitoan}[\cite{An_400_BT_Hoa_Hoc_9}, 89., p. 33]
	Hòa tan $m$ {\rm g} 1 iron oxide cần {\rm150 mL} dung dịch {\rm HCl 3M}, nếu khử toàn bộ $m$ {\rm g} iron oxide trên bằng {\rm CO} nóng, dư thì thu được {\rm8.4 g} iron. Xác định công thức iron oxide.
\end{baitoan}

\begin{baitoan}[\cite{An_400_BT_Hoa_Hoc_9}, 90., p. 33]
	Cho dòng khí {\rm CO} đi qua {\rm11.6 g} oxide sắt nung nóng, đến phản ứng hoàn toàn nhận được sắt nguyên chất \& lượng khí được hấp thụ bởi dung dịch {\rm\ce{Ca(OH)2}} dư, thu được {\rm20 g} kết tủa. Xác định {\rm CTPT} iron oxide.
\end{baitoan}

\begin{baitoan}[\cite{An_400_BT_Hoa_Hoc_9}, 91., p. 33]
	Cho hỗn hợp A gồm 3 oxide: {\rm\ce{Al2O3,CuO,K2O}}. Tiến hành 3 thí nghiệm:
	\begin{itemize}
		\item Thí nghiệm 1: Nếu cho hỗn hợp A vào nước dư, khuấy kỹ thấy còn {\rm15 g} chất rắn không tan.
		\item Thí nghiệm 2: Nếu cho thêm vào hỗn hợp A 1 lượng {\rm\ce{Al2O3}} bằng {\rm50\%} lượng {\rm\ce{Al2O3}} trong A ban đầu rồi lại hòa tan vào nước dư. Sau thí nghiệm còn lại {\rm21 g} chất rắn không tan.
		\item Thí nghiệm 3: Nếu cho vào hỗn hợp A 1 lượng {\rm\ce{Al2O3}} bằng {\rm75\%} lượng {\rm\ce{Al2O3}} trong A, rồi lại hòa tan vào nước dư, thấy còn lại {\rm25 g} chất rắn không tan.
	\end{itemize}
	Tính khối lượng mỗi oxide trong hỗn hợp A.
\end{baitoan}

\begin{baitoan}[\cite{An_400_BT_Hoa_Hoc_9}, 92., p. 33]
	Hợp chất khí với hydrogen của 1 nguyên tố ứng với công t hức $\rm RH_4$. Oxide cao nhất của nó chứa {\rm53.3\%} oxygen. Xác định nguyên tố R.
\end{baitoan}

\begin{baitoan}[\cite{An_400_BT_Hoa_Hoc_9}, 93., pp. 33--34]
	Khử {\rm3.48 g} 1 oxide của kim loại M cần dùng {\rm1.344 L \ce{H2}} (đktc). Toàn bộ lượng kim loại M thu được cho tác dụng với dung dịch {\rm HCl} dư cho {\rm1.008 L \ce{H2}} (đktc). Tìm {\rm CTPT} của M.
\end{baitoan}

\begin{baitoan}[\cite{An_400_BT_Hoa_Hoc_9}, 94., p. 34]
	Hòa tan hoàn  toàn {\rm5.94 g Al} vào dung dịch {\rm NaOH} dư được khí thứ nhất. Cho {\rm1.896 g \ce{KMnO4}} tác dụng hết với acid {\rm HCl} đặc, dư được khí thứ 2. Nhiệt phân hoàn toàn {\rm12.25\% \ce{KClO3}} có xúc tác thu được khí thứ 3. Cho toàn bộ lượng các khí điều chế ở trên vào 1 bình kín rồi đốt cháy để các phản ứng xảy ra hoàn toàn. Sau đó làm lạnh bình để cho hơi nước ngưng tụ hết \& giả thiết các chất tan hết vào nước thu được dung dịch E. Viết {\rm PTHH} \& tính nồng độ $C\%$ của dung dịch E.
\end{baitoan}

\begin{baitoan}[\cite{An_400_BT_Hoa_Hoc_9}, 95., p. 34]
	Chỉ dùng thêm nước, nhận biết 4 chất rắn: {\rm\ce{Na2O,Al2O3,Fe3O4}, Al} chứa trong các lọ riêng biệt. Viết {\rm PTHH}.
\end{baitoan}

\begin{baitoan}[\cite{An_400_BT_Hoa_Hoc_9}, 96., p. 34]
	Cho {\rm416 g} dung dịch {\rm\ce{BaCl2} 12\%} tác dụng vừa đủ với dung dịch chứa {\rm27.36 g} muối sulfate kim loại A. Sau khi lọc bỏ kết tủa thu được {\rm800 mL} dung dịch {\rm0.2M} của chloride kim loại A. Tìm {\rm CTPT} của muối sulfate kim loại A.
\end{baitoan}

\begin{baitoan}[\cite{An_400_BT_Hoa_Hoc_9}, 97., p. 34]
	Hòa tan {\rm6.75 g} 1 kim loại M chưa rõ hóa trị vào dung dịch acid thì cần {\rm500 mL} dung dịch {\rm HCl 1.5 M}. Xác định kim loại M.
\end{baitoan}

\begin{baitoan}[\cite{An_400_BT_Hoa_Hoc_9}, 98., p. 34]
	Cho {\rm4.15 g} hỗn hợp bột {\rm Fe, Al} tác dụng với {\rm200 mL} dung dịch {\rm\ce{CuSO4} 0.525M}. Khuấy kỹ hỗn hợp để phản ứng xảy ra hoàn toàn. Đem lọc kết tủa A gồm 2 kim loại có khối lượng {\rm7.48 g} \& dung dịch nước lọc. Tìm số mol các kim loại trong hỗn hợp ban đầu \& trong hỗn hợp A.
\end{baitoan}

\begin{baitoan}[\cite{An_400_BT_Hoa_Hoc_9}, 99.a, p. 34]
	Có 3 lọ đựng 3 hỗn hợp dạng bột: {\rm(Al $+$ \ce{Al2O3}), (Fe $+$ \ce{Fe2O3}), (FeO $+$ \ce{Fe2O3})}. Dùng phương pháp hóa học để nhận biết chúng. Viết {\rm PTHH}.
\end{baitoan}

\begin{baitoan}[\cite{An_400_BT_Hoa_Hoc_9}, 99.b, p. 34]
	Trình bày phương pháp tách {\rm\ce{Fe2O3}} ra khỏi hỗn hợp {\rm\ce{Fe2O3,Al2O3,SiO2}} ở dạng bột. Chỉ dùng duy nhất 1 hóa chất.
\end{baitoan}

\begin{baitoan}[\cite{An_400_BT_Hoa_Hoc_9}, 100., p. 34]
	1 quả cầu làm bằng nhôm có khối lượng {\rm1.404 kg} \& thể tích $\rm0.62\ dm^3$. Quả cầu này có chỗ rỗng bên trong không? Nếu có, tính thể tích chỗ rỗng đó, biết $D_{\rm Al} = 2.7$ {\rm kg{\tt/}$\rm dm^3$}.
\end{baitoan}

%------------------------------------------------------------------------------%

\printbibliography[heading=bibintoc]

\end{document}