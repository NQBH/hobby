\documentclass{article}
\usepackage[backend=biber,natbib=true,style=alphabetic,maxbibnames=50]{biblatex}
\addbibresource{/home/nqbh/reference/bib.bib}
\usepackage[utf8]{vietnam}
\usepackage{tocloft}
\renewcommand{\cftsecleader}{\cftdotfill{\cftdotsep}}
\usepackage[colorlinks=true,linkcolor=blue,urlcolor=red,citecolor=magenta]{hyperref}
\usepackage{amsmath,amssymb,amsthm,float,graphicx,mathtools,diagbox,tikz,tipa}
\usepackage[version=4]{mhchem}
\allowdisplaybreaks
\newtheorem{assumption}{Assumption}
\newtheorem{baitoan}{Bài toán}
\newtheorem{cauhoi}{Câu hỏi}
\newtheorem{conjecture}{Conjecture}
\newtheorem{corollary}{Corollary}
\newtheorem{dangtoan}{Dạng toán}
\newtheorem{definition}{Definition}
\newtheorem{dinhly}{Định lý}
\newtheorem{dinhnghia}{Định nghĩa}
\newtheorem{example}{Example}
\newtheorem{ghichu}{Ghi chú}
\newtheorem{hequa}{Hệ quả}
\newtheorem{hypothesis}{Hypothesis}
\newtheorem{lemma}{Lemma}
\newtheorem{luuy}{Lưu ý}
\newtheorem{nhanxet}{Nhận xét}
\newtheorem{notation}{Notation}
\newtheorem{note}{Note}
\newtheorem{principle}{Principle}
\newtheorem{problem}{Problem}
\newtheorem{proposition}{Proposition}
\newtheorem{question}{Question}
\newtheorem{remark}{Remark}
\newtheorem{theorem}{Theorem}
\newtheorem{thinghiem}{Thí nghiệm}
\newtheorem{vidu}{Ví dụ}
\usepackage[left=1cm,right=1cm,top=5mm,bottom=5mm,footskip=4mm]{geometry}

\title{Problem: Metal -- Bài Tập Kim Loại}
\author{Nguyễn Quản Bá Hồng\footnote{Independent Researcher, Ben Tre City, Vietnam\\e-mail: \texttt{nguyenquanbahong@gmail.com}; website: \url{https://nqbh.github.io}.}}
\date{\today}

\begin{document}
\maketitle
\begin{abstract}
	
\end{abstract}
\setcounter{secnumdepth}{4}
\setcounter{tocdepth}{3}
\tableofcontents

%------------------------------------------------------------------------------%

\begin{baitoan}[\cite{An_Hoa_Hoc_nang_cao_8_9}, 1., p. 89]
	(a) Sắt là nguyên tố có nhiều hóa trị, phổ biến là (II) \& (III). Viết các {\rm PTHH} minh họa. (b) Cho các kim loại {\rm Cu, Al, Fe, Ag}. Các kim loại nào tác dụng với acid hydrochloric? Các kim loại nào tác dụng được với dung dịch {\rm\ce{CuSO4}}? Dung dịch {\rm\ce{AgNO3}}? Viết các {\rm PTHH} tương ứng.
\end{baitoan}

\begin{baitoan}[\cite{An_Hoa_Hoc_nang_cao_8_9}, 2., p. 90]
	Có thể điều chế bao nhiêu {\rm kg} aluminium từ $1$ tấn quặng nhôm chứa {\rm95\%} aluminium oxide biết hiệu suất phản ứng là {\rm98\%}.
\end{baitoan}

\begin{baitoan}[\cite{An_Hoa_Hoc_nang_cao_8_9}, 3., p. 90]
	(a) Tại sao không nên dùng chậu nhôm đựng nước vôi. (b) Viết PTHH giữa {\rm\ce{Fe3O4}} với {\rm\ce{H2SO4}}.
\end{baitoan}

\begin{baitoan}[\cite{An_Hoa_Hoc_nang_cao_8_9}, 4., p. 91]
	Cho {\rm1.38 g} 1 kim loại hóa trị (I) tác dụng hết với nước cho {\rm0.2 g} hydrogen. Xác định kim loại đó.
\end{baitoan}

\begin{baitoan}[\cite{An_Hoa_Hoc_nang_cao_8_9}, 5., p. 91]
	Trong quặng boxit trung bình có {\rm50\%} aluminium oxide. Kim loại luyện được từ oxide đó còn chứa {\rm1.5\%} tạp chất. Tính lượng nhôm nguyên chất điều chế được từ $0.5$ tấn quặng boxit.
\end{baitoan}

\begin{baitoan}[\cite{An_Hoa_Hoc_nang_cao_8_9}, 6., p. 92]
	Cho bản kẽm có khối lượng {\rm50 g} vào dung dịch đồng sulfate. Sau 1 thời gian phản ứng kết thúc thì khối lượng bản kẽm là {\rm49.82 g}. Tính: (a) Khối lượng kẽm đã tác dụng. (b) Khối lượng đồng sulfate có trong dung dịch.
\end{baitoan}

\begin{baitoan}[\cite{An_Hoa_Hoc_nang_cao_8_9}, 7., p. 92]
	Để thu được $1000$ tấn gang chứa {\rm95\%} sắt, {\rm5\%} carbon (các nguyên tố khác chiếm 1 lượng không đáng kể) thì theo lý thuyết phải cần bao nhiêu tấn {\rm\ce{Fe2O3}} \& bao nhiêu tấn than cốc. 
\end{baitoan}

\begin{baitoan}[\cite{An_Hoa_Hoc_nang_cao_8_9}, 8., p. 93]
	Cho {\rm5.4 g} 1 kim loại tác dụng với chlorine có dư thu được {\rm26.7  g} muối. Xác định kim loại đem phản ứng, biết kim loại có hóa trị từ I $\to$ III.
\end{baitoan}

\begin{baitoan}[\cite{An_Hoa_Hoc_nang_cao_8_9}, 9., p. 94]
	1 nguyên tố R có oxide cao nhất chiếm {\rm60\%} oxi theo khối lượng. Hợp chất khí của R với hydrogen có tỷ khối hơi so với không khí là $1.172$. Xác định công thức oxide của R.
\end{baitoan}

\begin{baitoan}[\cite{An_Hoa_Hoc_nang_cao_8_9}, 10., p. 94, TS PTNK ĐH KHTN Tp. HCM 1998]
	1 hỗn hợp X gồm kim loại {\rm M} ({\rm M} có hóa trị II \& III) \& oxide $\rm M_xO_y$ của kim loại ấy. Khối lượng hỗn hợp X là {\rm27.2 g}. Khi cho X tác dụng với {\rm0.8 L HCl 2M} thì hỗn hợp X tan hết cho dung dịch A cần {\rm0.6 L} dung dịch {\rm NaOH 1M}. Xác định {\rm M}, $\rm M_xO_y$, \& {\rm\%M, \%$\rm M_xO_y$} (theo khối lượng) trong hỗn hợp X. Biết trong 2 chất này có 1 chất có số mol bằng $2$ lần số mol chất kia.
\end{baitoan}

\begin{baitoan}[\cite{An_Hoa_Hoc_nang_cao_8_9}, 11., p. 94]
	A là kim loại hóa trị II. Lấy 2 thanh A cùng khối lượng. Thanh thứ nhất nhúng vào dung dịch {\rm\ce{CuSO4}}, sau 1 thời gian khối lượng giảm {\rm3.6\%}. Thanh thứ 2 nhúng vào dung dịch {\rm\ce{HgSO4}}, sau 1 thời gian khối lượng tăng {\rm6.675\%}. Nồng độ mol của 2 dung dịch {\rm\ce{CuSO4,HgSO4}} giảm cùng 1 số mol như nhau. Xác định tên kim loại A.
\end{baitoan}

\begin{baitoan}[\cite{An_Hoa_Hoc_nang_cao_8_9}, 12., p. 94]
	Khử {\rm3.48 g} 1 oxide của kim loại M cần dùng {\rm1.344 L} khí {\rm\ce{H2}} (ở đktc). Tìm {\rm CTPT} của oxide kim loại.
\end{baitoan}

\begin{baitoan}[\cite{An_Hoa_Hoc_nang_cao_8_9}, 13., p. 94]
	Cho hỗn hợp {\rm Al, Fe} tác dụng với hỗn hợp dung dịch chứa {\rm\ce{AgNO3,Cu(NO3)2}} thu được dung dịch B \& chất rắn D gồm 3 kim loại. Cho D tác dụng với dung dịch {\rm HCl} dư có khí bay ra. Xác định thành phần chất rắn D.
\end{baitoan}

\begin{baitoan}[\cite{An_Hoa_Hoc_nang_cao_8_9}, 14., p. 94]
	Cho {\rm2 g} hỗn hợp {\rm Fe} \& kim loại hóa trị II vào dung dịch {\rm HCl} có dư thì thu được {\rm1.12 L \ce{H2}} (đktc). Mặt khác, nếu hòa tan {\rm4.8 g} kim loại hóa trị II đó cần chưa đến {\rm500 mL} dung dịch {\rm HCl}. Xác định kim loại hóa trị II.
\end{baitoan}

\begin{baitoan}[\cite{An_Hoa_Hoc_nang_cao_8_9}, 15., pp. 94--95]
	X là hỗn hợp 2 kim loại {\rm Mg, Zn}. Y là dung dịch {\rm\ce{H2SO4}} chưa rõ nồng độ. Thí nghiệm 1: Cho {\rm24.3 g} X vào {\rm2 L} Y, sinh ra {\rm8.96 L \ce{H2}}. Thí nghiệm 2: Cho {\rm24.3 g} X vào {\rm3 L} Y, sinh ra {\rm11.2 L \ce{H2}}. Lập luận chứng tỏ trong thí nghiệm 1 thì X chưa tan hết, trong thí nghiệm 2 thì X tan hết.
\end{baitoan}

\begin{baitoan}[\cite{An_Hoa_Hoc_nang_cao_8_9}, 16., p. 95]
	Cho {\rm8 g \ce{Fe_xO_y}} tác dụng với $V$ {\rm mL} dung dịch {\rm HCl 2M} lấy dư {\rm25\%} với lượng cần thiết. Đun nóng khan dung dịch sau phản ứng thu được {\rm16.25 g} muối khan. (a) Xác định {\rm CTPT} {\rm\ce{Fe_xO_y}}. (b) Tính $V$.
\end{baitoan}

\begin{baitoan}[\cite{An_Hoa_Hoc_nang_cao_8_9}, 17., p. 95]
	Nung nóng kim loại X trong không khí đến khối lượng không đổi được chất rắn Y. Khối lượng của X bằng $\frac{7}{10}$ khối lượng Y. Tìm {\rm CTPT} của chất rắn Y.
\end{baitoan}

\begin{baitoan}[\cite{An_Hoa_Hoc_nang_cao_8_9}, 18., p. 95]
	Cho {\rm3.87 g} hỗn hợp A gồm {\rm Mg, Al} vào {\rm250 mL} dung dịch X chứa {\rm HCl 1M} \& {\rm\ce{H2SO4} 0.5M} được dung dịch B \& {\rm4.368 L \ce{H2}} (ở đktc). Biện luận xem hỗn hợp A còn dư hay đã phản ứng hết.
\end{baitoan}

\begin{baitoan}[\cite{An_Hoa_Hoc_nang_cao_8_9}, 19., p. 95]
	Nguyên tố R tạo thành hợp chất khí với hydrogen có {\rm CTHH} là {\rm\ce{RH4}}. Trong hợp chất cao nhất với oxide chứa {\rm72.73\%} là oxygen. (a) Xác định tên nguyên tố R. (b) Cho biết vị trí của R trong bảng tuần hoàn.
\end{baitoan}

\begin{baitoan}[\cite{An_Hoa_Hoc_nang_cao_8_9}, 20., p. 95]
	{\rm Đ{\tt/}S?} {\sf A.} Trong cùng 1 chu kỳ, khi điện tích hạt nhân tăng dần, tính phi kim tăng dần \& bán kính nguyên tử giảm dần. {\sf B.} Trong chu kỳ, theo chiều tăng điện tích hạt nhân, tính acid của các oxide \& hydroxide giảm dần. {\sf C.} Trong cùng 1 nhóm, khi điện tích hạt nhân tăng dần thì tính base của các oxide \& hydrogen tăng dần. {\sf D. B} sai.
\end{baitoan}

%------------------------------------------------------------------------------%

\printbibliography[heading=bibintoc]

\end{document}