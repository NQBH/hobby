\documentclass{article}
\usepackage[backend=biber,natbib=true,style=authoryear]{biblatex}
\addbibresource{/home/nqbh/reference/bib.bib}
\usepackage[utf8]{vietnam}
\usepackage{tocloft}
\renewcommand{\cftsecleader}{\cftdotfill{\cftdotsep}}
\usepackage[colorlinks=true,linkcolor=blue,urlcolor=red,citecolor=magenta]{hyperref}
\usepackage{amsmath,amssymb,amsthm,mathtools,float,graphicx,algpseudocode,algorithm,tcolorbox,tikz,tkz-tab,subcaption}
\DeclareMathOperator{\arccot}{arccot}
\usepackage[inline]{enumitem}
\usepackage[version=4]{mhchem}
\allowdisplaybreaks
\numberwithin{equation}{section}
\newtheorem{assumption}{Assumption}[section]
\newtheorem{nhanxet}{Nhận xét}[section]
\newtheorem{conjecture}{Conjecture}[section]
\newtheorem{corollary}{Corollary}[section]
\newtheorem{hequa}{Hệ quả}[section]
\newtheorem{definition}{Definition}[section]
\newtheorem{dinhnghia}{Định nghĩa}[section]
\newtheorem{example}{Example}[section]
\newtheorem{vidu}{Ví dụ}[section]
\newtheorem{lemma}{Lemma}[section]
\newtheorem{notation}{Notation}[section]
\newtheorem{principle}{Principle}[section]
\newtheorem{problem}{Problem}[section]
\newtheorem{baitoan}{Bài toán}[section]
\newtheorem{proposition}{Proposition}[section]
\newtheorem{menhde}{Mệnh đề}[section]
\newtheorem{question}{Question}[section]
\newtheorem{cauhoi}{Câu hỏi}[section]
\newtheorem{quytac}{Quy tắc}
\newtheorem{remark}{Remark}[section]
\newtheorem{luuy}{Lưu ý}[section]
\newtheorem{theorem}{Theorem}[section]
\newtheorem{tiende}{Tiên đề}[section]
\newtheorem{dinhly}{Định lý}[section]
\usepackage[left=0.5in,right=0.5in,top=1.5cm,bottom=1.5cm]{geometry}
\usepackage{fancyhdr}
\pagestyle{fancy}
\fancyhf{}
\lhead{\small Subsect.~\thesubsection}
\rhead{\small\nouppercase{\leftmark}}
\renewcommand{\subsectionmark}[1]{\markboth{#1}{}}
\cfoot{\thepage}
\def\labelitemii{$\circ$}

\title{Nitrogen -- Nitơ}
\author{Nguyễn Quản Bá Hồng\footnote{Independent Researcher, Ben Tre City, Vietnam\\e-mail: \texttt{nguyenquanbahong@gmail.com}; website: \url{https://nqbh.github.io}.}}
\date{\today}

\begin{document}
\maketitle
\begin{abstract}
	
\end{abstract}
\setcounter{secnumdepth}{4}
\setcounter{tocdepth}{3}
\tableofcontents

%------------------------------------------------------------------------------%

\section{Theory -- Lý Thuyết}
\cite[pp. 38--]{An_Hoa_Hoc_nang_cao_11_2020}

\section{Problem}

\subsection{Dựa vào cấu hình electron xác định số oxi hóa \& tính chất hóa học của nitơ, photpho \& các hợp chất của chúng}

\begin{baitoan}[\cite{An_Hoa_Hoc_nang_cao_11_2020}, \textbf{1.}, p. 46, TS ĐHCĐ khối B 2002]
	\begin{enumerate*}
		\item[(a)] Phân nhóm chính nhóm V của hệ thống tuần hoàn gồm những nguyên tố nào? Viết cấu hình electron của 2 nguyên tố đầu tiên là \emph{N,P}. Từ đó giải thích tại sao \emph{N} chỉ cho hợp chất có hóa trị 3 trong khi \emph{P} có thể có hóa trị 3 \& 5.
		\item[(b)] Chỉ dùng 1 hóa chất, cho biết cách phân biệt \emph{\ce{Fe2O3,Fe3O4}}. Viết các phương trình phản ứng.
	\end{enumerate*}
\end{baitoan}

\begin{baitoan}[\cite{An_Hoa_Hoc_nang_cao_11_2020}, \textbf{2.}, p. 47]
	Giải thích vì sao độ âm điện của \emph{N} \& \emph{Cl} đều bằng $3$, nhưng ở nhiệt độ thường \emph{\ce{N2}} có tính oxi hóa kém hơn \emph{\ce{Cl2}}? Khi nào \emph{\ce{N2}} trở nên hoạt động hơn?
\end{baitoan}

\begin{baitoan}[\cite{An_Hoa_Hoc_nang_cao_11_2020}, \textbf{3.}, p. 47]
	2 nguyên tố A \& B ở 2 phân nhóm chính liên tiếp trong bảng tuần hoàn. B thuộc nhóm V. Ở trạng thái đơn chất A \& B không phản ứng với nhau. Tổng số proton trong hạt nhân nguyên tử của A \& B là $23$.
	\begin{enumerate*}
		\item[(a)] Viết cấu hình electron của A \& B.
		\item[(b)] Từ các đơn chất A, B \& các hóa chất cần thiết, viết các phương trình phản ứng điều chế 2 acid trong đó A \& B có số oxi hóa dương cao nhất.
	\end{enumerate*}
\end{baitoan}

\begin{baitoan}[\cite{An_Hoa_Hoc_nang_cao_11_2020}, \textbf{4.}, p. 48]
	\begin{enumerate*}
		\item[(a)] Tổng số hạt proton, neutron, electron của nguyên tử 1 nguyên tố là $21$.
		\begin{enumerate*}
			\item[(1)] Xác định tên nguyên tố đó.
			\item[(2)] Viết cấu hình electron nguyên tử của nguyên tố đó.
			\item[(3)] Tính tổng số obital trong nguyên tử của nguyên tố đó. (TS ĐHYD 1998)
		\end{enumerate*}
		\item[(b)] Vì sao nitơ là khí tương đối trơ ở nhiệt độ thường.
	\end{enumerate*}
\end{baitoan}

%------------------------------------------------------------------------------%

\subsection{Hoàn thành các phương trình phản ứng, viết các phương trình phản ứng dạng phân tử \& ion rút gọn}

%------------------------------------------------------------------------------%

\subsection{Vận dụng nguyên lý chuyển dịch cân bằng Le Chatelier trong phản ứng thuận nghịch}

%------------------------------------------------------------------------------%

\subsection{Biện luận xác định công thức phân tử \& ion có trong dung dịch}

%------------------------------------------------------------------------------%

\subsection{Nhận biết, điều chế, \& tinh chế các chất}

%------------------------------------------------------------------------------%

\subsection{Thành phần hỗn hợp khí \& áp suất}

%------------------------------------------------------------------------------%

\subsection{Tính hiệu suất phản ứng, nồng độ dung dịch các chất, hằng số cân bằng}

%------------------------------------------------------------------------------%

\subsection{Xác định công thức phân tử \& tên nguyên tố}

%------------------------------------------------------------------------------%

\subsection{Tính khối lượng chất tham gia phản ứng}

%------------------------------------------------------------------------------%

\printbibliography[heading=bibintoc]
	
\end{document}