\documentclass[oneside]{book}
\usepackage[backend=biber,natbib=true,style=authoryear]{biblatex}
\addbibresource{/home/hong/1_NQBH/reference/bib.bib}
\usepackage[utf8]{vietnam}
\usepackage{tocloft}
\renewcommand{\cftsecleader}{\cftdotfill{\cftdotsep}}
\usepackage[colorlinks=true,linkcolor=blue,urlcolor=red,citecolor=magenta]{hyperref}
\usepackage{amsmath,amssymb,amsthm,mathtools,float,graphicx,algpseudocode,algorithm,tcolorbox,tikz,tkz-tab,diagbox}
\DeclareMathOperator{\arccot}{arccot}
\usepackage[inline]{enumitem}
\allowdisplaybreaks
\numberwithin{equation}{section}
\newtheorem{assumption}{Assumption}[section]
\newtheorem{nhanxet}{Nhận xét}[section]
\newtheorem{conjecture}{Conjecture}[section]
\newtheorem{corollary}{Corollary}[section]
\newtheorem{hequa}{Hệ quả}[section]
\newtheorem{definition}{Definition}[section]
\newtheorem{dinhnghia}{Định nghĩa}[section]
\newtheorem{example}{Example}[section]
\newtheorem{vidu}{Ví dụ}[section]
\newtheorem{lemma}{Lemma}[section]
\newtheorem{notation}{Notation}[section]
\newtheorem{principle}{Principle}[section]
\newtheorem{problem}{Problem}[section]
\newtheorem{baitoan}{Bài toán}[section]
\newtheorem{proposition}{Proposition}[section]
\newtheorem{menhde}{Mệnh đề}[section]
\newtheorem{question}{Question}[section]
\newtheorem{cauhoi}{Câu hỏi}[section]
\newtheorem{remark}{Remark}[section]
\newtheorem{luuy}{Lưu ý}[section]
\newtheorem{theorem}{Theorem}[section]
\newtheorem{dinhly}{Định lý}[section]
\usepackage[left=0.5in,right=0.5in,top=1.5cm,bottom=1.5cm]{geometry}
\usepackage{fancyhdr}
\pagestyle{fancy}
\fancyhf{}
\lhead{\small \textsc{Sect.} ~\thesection}
\rhead{\small \nouppercase{\leftmark}}
\renewcommand{\sectionmark}[1]{\markboth{#1}{}}
\cfoot{\thepage}
\def\labelitemii{$\circ$}

\title{Some Topics in Elementary Computer Science\texttt{/}Grade 11}
\author{Nguyễn Quản Bá Hồng\footnote{Independent Researcher, Ben Tre City, Vietnam\\e-mail: \texttt{nguyenquanbahong@gmail.com}; website: \url{https://nqbh.github.io}.}}
\date{\today}

\begin{document}
\frontmatter
\maketitle
\setcounter{secnumdepth}{4}
\setcounter{tocdepth}{3}
\tableofcontents
\newpage

%------------------------------------------------------------------------------%

\mainmatter

\chapter*{Preface}

Tóm tắt kiến thức Tin học lớp 11 theo chương trình giáo dục của Việt Nam \& một số chủ đề nâng cao.

%------------------------------------------------------------------------------%

\chapter{1 Số Khái Niệm về Lập Trình \& Ngôn Ngữ Lập Trình}

\begin{quotation}
	\textbf{Nội dung.} \textit{Khái niệm cơ sở về lập trình, khái niệm \& các thành phần của ngôn ngữ lập trình, vai trò \& phân loại chương trình dịch}.
\end{quotation}

\section{Khái Niệm Lập Trình \& Ngôn Ngữ Lập Trình}

\section{Bạn Biết Gì Về Các Ngôn Ngữ Lập Trình?}

\section{Các Thành Phần Của Ngôn Ngữ Lập Trình}

\section{Ngôn Ngữ Pascal}

%------------------------------------------------------------------------------%

\chapter{Chương Trình Đơn Giản}

\section{Cấu Trúc Chương Trình}

\section{1 Số Kiểu Dữ Liệu Chuẩn}

\section{Khai Báo Biến}

\section{Phép Toán, Biểu Thức, Câu Lệnh Gán}

\section{Các Thủ Tục Chuẩn Vào\texttt{/}Ra Đơn Giản}

\section{Soạn Thảo, Dịch, Thực Hiện \& Hiệu Chỉnh Chương Trình}

%------------------------------------------------------------------------------%

\chapter{Cấu Trúc Rẽ Nhánh \& Lặp}

\section{Cấu Trúc Rẽ Nhánh}

\section{Cấu Trúc Lặp}

%------------------------------------------------------------------------------%

\chapter{Kiểu Dữ Liệu Có Cấu Trúc}

\section{Kiểu Mảng}

\section{Kiểu Xâu}

\section{Kiểu Bản Ghi}

%------------------------------------------------------------------------------%

\chapter{Tệp \& Thao Tác với Tệp}

\section{Kiểu Dữ Liệu Tệp}

\section{Thao Tác với Tệp}

\section{Ví Dụ Làm Việc với Tệp}

%------------------------------------------------------------------------------%

\chapter{Chương Trình Con \& Lập Trình Có Cấu Trúc}

\section{Chương Trình Con \& Phân Loại}

\section{Ví Dụ về Cách Viết \& Sử Dụng Chương Trình Con}

\section{Ai Là Lập Trình Viên Đầu Tiên?}

\section{Thư Viện Chương Trình Con Chuẩn}

\section{Âm Thanh}

%------------------------------------------------------------------------------%

\appendix

\chapter{Miscellaneous}

\section{1 Số Phép Toán Thường Dùng}

\section{Giá Trị Phép Toán Logic}

\section{Môi Trường Turbo Pascal}

\section{1 Số Tên Dành Riêng}

\section{1 Số Kiểu Dữ Liệu Chuẩn}

\section{1 Số Thủ Tục \& Hàm Chuẩn}

\section{Câu Lệnh Rẽ Nhánh \& Lặp}

\section{Câu Lệnh \texttt{with}}

\section{1 Số Thông Báo Lỗi}

\section{Câu Lệnh Rẽ Nhánh \& Lặp Trong C++}

%------------------------------------------------------------------------------%

\printbibliography[heading=bibintoc]
	
\end{document}