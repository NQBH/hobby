\documentclass{article}
\usepackage[backend=biber,natbib=true,style=authoryear]{biblatex}
\addbibresource{/home/hong/1_NQBH/reference/bib.bib}
\usepackage[utf8]{vietnam}
\usepackage{tocloft}
\renewcommand{\cftsecleader}{\cftdotfill{\cftdotsep}}
\usepackage[colorlinks=true,linkcolor=blue,urlcolor=red,citecolor=magenta]{hyperref}
\usepackage{amsmath,amssymb,amsthm,mathtools,float,graphicx,algpseudocode,algorithm,tcolorbox}
\usepackage[inline]{enumitem}
\allowdisplaybreaks
\numberwithin{equation}{section}
\newtheorem{assumption}{Assumption}[section]
\newtheorem{conjecture}{Conjecture}[section]
\newtheorem{corollary}{Corollary}[section]
\newtheorem{hequa}{Hệ quả}[section]
\newtheorem{definition}{Definition}[section]
\newtheorem{dinhnghia}{Định nghĩa}[section]
\newtheorem{example}{Example}[section]
\newtheorem{vidu}{Ví dụ}[section]
\newtheorem{lemma}{Lemma}[section]
\newtheorem{notation}{Notation}[section]
\newtheorem{principle}{Principle}[section]
\newtheorem{problem}{Problem}[section]
\newtheorem{baitoan}{Bài toán}[section]
\newtheorem{proposition}{Proposition}[section]
\newtheorem{question}{Question}[section]
\newtheorem{cauhoi}{Câu hỏi}[section]
\newtheorem{remark}{Remark}[section]
\newtheorem{luuy}{Lưu ý}[section]
\newtheorem{theorem}{Theorem}[section]
\newtheorem{dinhly}{Định lý}[section]
\usepackage[left=0.5in,right=0.5in,top=1.5cm,bottom=1.5cm]{geometry}
\usepackage{fancyhdr}
\pagestyle{fancy}
\fancyhf{}
\lhead{\small \textsc{Sect.} ~\thesection}
\rhead{\small \nouppercase{\leftmark}}
\renewcommand{\sectionmark}[1]{\markboth{#1}{}}
\cfoot{\thepage}
\def\labelitemii{$\circ$}

\title{Problems in Elementary Computer Science\texttt{/}Grade 11}
\author{Nguyễn Quản Bá Hồng\footnote{Independent Researcher, Ben Tre City, Vietnam\\e-mail: \texttt{nguyenquanbahong@gmail.com}; website: \url{https://nqbh.github.io}.}}
\date{\today}

\begin{document}
\maketitle
\begin{abstract}
	1 bộ sưu tập các bài tập Tin học chọn lọc từ cơ bản đến nâng cao cho Tin học sơ cấp lớp 11. Tài liệu này là phần bài tập bổ sung cho tài liệu chính \href{https://github.com/NQBH/hobby/blob/master/elementary_computer_science/grade_11/NQBH_elementary_computer_science_grade_11.pdf}{GitHub\texttt{/}NQBH\texttt{/}hobby\texttt{/}elementary computer science\texttt{/}grade 11\texttt{/}lecture}\footnote{Explicitly, \url{https://github.com/NQBH/hobby/blob/master/elementary_computer_science/grade_11/NQBH_elementary_computer_science_grade_11.pdf}.} của tác giả viết cho chương trình Tin học lớp 11. Phiên bản mới nhất của tài liệu này được lưu trữ ở link sau: \href{https://github.com/NQBH/hobby/blob/master/elementary_computer_science/grade_11/problem/NQBH_elementary_computer_science_grade_11_problem.pdf}{GitHub\texttt{/}NQBH\texttt{/}hobby\texttt{/}elementary computer science\texttt{/}grade 11\texttt{/}problem}\footnote{Explicitly, \url{https://github.com/NQBH/hobby/blob/master/elementary_computer_science/grade_11/problem/NQBH_elementary_computer_science_grade_11_problem.pdf}.}.
\end{abstract}
\tableofcontents
\newpage

%------------------------------------------------------------------------------%

\section{Miscellaneous}

\begin{baitoan}
	Viết thuật toán \& các chương trình bằng các ngôn ngữ lập trình \textsc{Pascal, Python, C\texttt{/}C++} để giải phương trình bậc nhất, bậc 2, bậc 3, \& bậc 4 với các hệ số thực được nhập từ bàn phím.
\end{baitoan}

\begin{baitoan}
	Viết thuật toán \& các chương trình bằng các ngôn ngữ lập trình \textsc{Pascal, Python, C\texttt{/}C++} để xuất ra màn hình, với $n\in\mathbb{N}$ được nhập từ bàn phím.:
	\begin{enumerate*}
		\item[(a)] Số Fibonacci thứ $n$ 
		\item[(b)] $n$ số Fibonacci đầu tiên.
	\end{enumerate*}
\end{baitoan}

%------------------------------------------------------------------------------%

\section{Solutions}

%------------------------------------------------------------------------------%


\printbibliography[heading=bibintoc]
	
\end{document}