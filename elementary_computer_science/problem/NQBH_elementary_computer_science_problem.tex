\documentclass{article}
\usepackage[backend=biber,natbib=true,style=alphabetic,maxbibnames=10]{biblatex}
\addbibresource{/home/nqbh/reference/bib.bib}
\usepackage[utf8]{vietnam}
\usepackage{tocloft}
\renewcommand{\cftsecleader}{\cftdotfill{\cftdotsep}}
\usepackage[colorlinks=true,linkcolor=blue,urlcolor=red,citecolor=magenta]{hyperref}
\usepackage{amsmath,amssymb,amsthm,float,graphicx,mathtools}
\allowdisplaybreaks
\newtheorem{assumption}{Assumption}
\newtheorem{baitoan}{Bài toán}
\newtheorem{cauhoi}{Câu hỏi}
\newtheorem{conjecture}{Conjecture}
\newtheorem{corollary}{Corollary}
\newtheorem{dangtoan}{Dạng toán}
\newtheorem{definition}{Definition}
\newtheorem{dinhly}{Định lý}
\newtheorem{dinhnghia}{Định nghĩa}
\newtheorem{example}{Example}
\newtheorem{ghichu}{Ghi chú}
\newtheorem{hequa}{Hệ quả}
\newtheorem{hypothesis}{Hypothesis}
\newtheorem{lemma}{Lemma}
\newtheorem{luuy}{Lưu ý}
\newtheorem{nhanxet}{Nhận xét}
\newtheorem{notation}{Notation}
\newtheorem{note}{Note}
\newtheorem{principle}{Principle}
\newtheorem{problem}{Problem}
\newtheorem{proposition}{Proposition}
\newtheorem{question}{Question}
\newtheorem{remark}{Remark}
\newtheorem{theorem}{Theorem}
\newtheorem{vidu}{Ví dụ}
\usepackage[left=1cm,right=1cm,top=5mm,bottom=5mm,footskip=4mm]{geometry}
\def\labelitemii{$\circ$}
\DeclareRobustCommand{\divby}{%
	\mathrel{\vbox{\baselineskip.65ex\lineskiplimit0pt\hbox{.}\hbox{.}\hbox{.}}}%
}

\title{Problems in Elementary Computer Science}
\author{Nguyễn Quản Bá Hồng\footnote{Independent Researcher, Ben Tre City, Vietnam\\e-mail: \texttt{nguyenquanbahong@gmail.com}; website: \url{https://nqbh.github.io}.}}
\date{\today}

\begin{document}
\maketitle
\begin{abstract}
	1 bộ sưu tập các bài toán chọn lọc từ cơ bản đến nâng cao cho Tin học sơ cấp. Phiên bản mới nhất của tài liệu này được lưu trữ ở link sau: \href{https://github.com/NQBH/hobby/blob/master/elementary_computer_science/problem/NQBH_elementary_computer_science_problem.pdf}{GitHub\texttt{/}NQBH\texttt{/}hobby\texttt{/}elementary computer science\texttt{/}problem}\footnote{\textsc{url}: \url{https://github.com/NQBH/hobby/blob/master/elementary_computer_science/problem/NQBH_elementary_computer_science_problem.pdf}.}.
\end{abstract}
\tableofcontents

%------------------------------------------------------------------------------%

\section{Notes on Python Commands}

\begin{enumerate}
	\item Để sử dụng các hàm toán học trong Python, cần import thư viện \texttt{math} vào chương trình: \texttt{from math import *}
	\item Để mở file dữ liệu vào \texttt{prob.INP} chỉ để đọc dữ liệu \& mở file dữ liệu ra \texttt{prob.OUT} để thay đổi dữ liệu trong file: \texttt{file = open("prob.inp")} \& \texttt{file2 = open("prob.out", "w")}.
\end{enumerate}

\section{Problems in Elementary Mathematics}

\begin{baitoan}[Even vs. odd]
	Viết thuật toán \& các chương trình bằng các ngôn ngữ lập trình \textsc{Pascal, Python, C\texttt{/}C++} để xét tính chẵn lẻ của $n\in\mathbb{Z}$ được nhập từ bàn phím.
\end{baitoan}

\begin{itemize}
	\item Pascal script: \href{https://github.com/NQBH/hobby/blob/master/elementary_computer_science/Pascal/even_odd.pas}{GitHub\texttt{/}NQBH\texttt{/}hobby\texttt{/}elementary computer science\texttt{/}Pascal\texttt{/}even vs. odd}.
	\item Python script: \href{https://github.com/NQBH/hobby/blob/master/elementary_computer_science/Python/even_odd.py}{GitHub\texttt{/}NQBH\texttt{/}hobby\texttt{/}elementary computer science\texttt{/}Python\texttt{/}even vs. odd}.
\end{itemize}

\begin{baitoan}[Divisible by]
	Viết thuật toán \& các chương trình bằng các ngôn ngữ lập trình \textsc{Pascal, Python, C\texttt{/}C++} để kiểm tra liệu $a\divby b$ hay không, với $a,b\in\mathbb{Z}$ được nhập từ bàn phím.
\end{baitoan}

\begin{itemize}
	\item Pascal script: \href{https://github.com/NQBH/hobby/blob/master/elementary_computer_science/Pascal/divisible_by.pas}{GitHub\texttt{/}NQBH\texttt{/}hobby\texttt{/}elementary computer science\texttt{/}Pascal\texttt{/}divisible by}.
	\item Python script: \href{https://github.com/NQBH/hobby/blob/master/elementary_computer_science/Python/divisible_by.py}{GitHub\texttt{/}NQBH\texttt{/}hobby\texttt{/}elementary computer science\texttt{/}Python\texttt{/}divisible by}.
\end{itemize}

\begin{baitoan}[Triangle]
	Viết thuật toán \& các chương trình bằng các ngôn ngữ lập trình \textsc{Pascal, Python, C\texttt{/}C++} để liệu $a,b,c$ có phải là độ dài của: (a) 1 tam giác. (b) 1 tam giác nhọn. (c) 1 tam giác vuông. (d) 1 tam giác tù.
\end{baitoan}

\begin{itemize}
	\item Python script: \href{https://github.com/NQBH/hobby/blob/master/elementary_computer_science/Python/triangle.py}{GitHub\texttt{/}NQBH\texttt{/}hobby\texttt{/}elementary computer science\texttt{/}Python\texttt{/}triangle}.
\end{itemize}

\begin{baitoan}[Polynomial equation]
	Viết thuật toán \& các chương trình bằng các ngôn ngữ lập trình \textsc{Pascal, Python, C\texttt{/}C++} để giải phương trình bậc nhất, bậc 2, bậc 3, \& bậc 4 với các hệ số thực được nhập từ bàn phím.
\end{baitoan}

\begin{baitoan}[Fibonacci sequence]
	Viết thuật toán \& các chương trình bằng các ngôn ngữ lập trình \textsc{Pascal, Python, C\texttt{/}C++} để xuất ra màn hình, với $n\in\mathbb{N}$ được nhập từ bàn phím: (a) Số Fibonacci thứ $n$. (b) $n$ số Fibonacci đầu tiên.
\end{baitoan}

\begin{baitoan}[1st $n$ square roots]
	Viết chương trình \textsc{Pascal, C\texttt{/}C++, Python} xuất ra căn bậc 2 của $n$ số tự nhiên đầu tiên với $n\in\mathbb{N}^\star$ được nhập từ bàn phím.
\end{baitoan}

\begin{baitoan}[Số chính phương -- Square number]
	Viết chương trình \textsc{Pascal, C\texttt{/}C++, Python} để kiểm tra 1 số $n\in\mathbb{N}^\star$ được nhập từ bàn phím có phải là số chính phương hay không.
\end{baitoan}

\begin{baitoan}[1st $n$ cube roots]
	Viết chương trình \textsc{Pascal, C\texttt{/}C++, Python} xuất ra căn bậc 3 của $n$ số tự nhiên đầu tiên với $n\in\mathbb{N}^\star$ được nhập từ bàn phím.
\end{baitoan}

\begin{baitoan}
	Viết chương trình \textsc{Pascal, C\texttt{/}C++, Python} để kiểm tra 1 số $n\in\mathbb{N}^\star$ được nhập từ bàn phím có phải là lập phương của 1 số tự nhiên hay không.
\end{baitoan}

\begin{baitoan}[1st $n$ $n$th roots]
	Viết chương trình \textsc{Pascal, C\texttt{/}C++, Python} xuất ra căn bậc $n$ của $m$ số tự nhiên đầu tiên với $m,n\in\mathbb{N}^\star$ được nhập từ bàn phím.
\end{baitoan}

\begin{baitoan}
	Viết chương trình \textsc{Pascal, C\texttt{/}C++, Python} để kiểm tra 1 số $m$ được nhập từ bàn phím có phải là lũy thừa bậc $n$ của 1 số tự nhiên hay không với $m,n\in\mathbb{N}^\star$ được nhập từ bàn phím.
\end{baitoan}

%------------------------------------------------------------------------------%

\section{Algebraic Expression -- Biểu Thức Đại Số}

\begin{baitoan}[\cite{VietSTEM2021}, 1., p. 15, Vũng Tàu 2020]
	Cho $a,b,c\in\mathbb{N}^\star$. {\sf Yêu cầu:} Tính giá trị của biểu thức $S = \dfrac{a^2 + b^2 + c^2}{abc} + \sqrt{abc}$.
	\begin{itemize}
		\item {\sf Dữ liệu vào:} File \verb|root.inp| chứa 3 số nguyên dương $a,b,c$. Mỗi số trên 1 dòng.
		\item {\sf Kết quả:} Ghi vào File \verb|root.out| kết quả $S$ tính được (làm tròn lấy 2 chữ số sau phần thập phân). E.g.,
		\begin{table}[H]
			\centering
			\begin{tabular}{|l|l|}
				\hline
				\texttt{root.inp} & \texttt{root.out} \\
				\hline
				2 & 4.25 \\
				1 &  \\
				2 &  \\
				\hline
			\end{tabular}
		\end{table}
	\end{itemize}
\end{baitoan}
Python script: \href{https://github.com/NQBH/hobby/blob/master/elementary_computer_science/Python/root.py}{GitHub\texttt{/}NQBH\texttt{/}hobby\texttt{/}elementary computer science\texttt{/}Python\texttt{/}root.py}\footnote{\textsc{url}: \url{https://github.com/NQBH/hobby/blob/master/elementary_computer_science/Python/root.py}.}. Input: \href{https://github.com/NQBH/hobby/blob/master/elementary_computer_science/Python/root.inp}{root.inp}. Output: \href{https://github.com/NQBH/hobby/blob/master/elementary_computer_science/Python/root.out}{root.out}.
\begin{verbatim}
	from math import *
	file_in = open("root.inp")
	file_out = open("root.out", "w")
	a = file_in.readline()
	b = file_in.readline()
	c = file_in.readline()
	a = int(a)
	b = int(b)
	c = int(c)
	S = (a**a + b**b + c**c)/(a*b*c) + sqrt(a*b*c)
	S = str(round(S,2))
	file_out.write(S)
	file_in.close()
	file_out.close()
\end{verbatim}

\begin{luuy}
	Tương tự, ta có thể tính hầu như bất kỳ hàm số $f(a,b,c)$ 3 biến $a,b,c$ với $f$ là 1 hàm số có thể viết được nhờ thư viện \texttt{math} của Python. Tổng quát hơn, ta có thể tính bất kỳ hàm số nhiều biến $f(x_1,x_2,\ldots,x_n)$ với $x_i$, $i = 1,2,\ldots,n$, $n\in\mathbb{N}^\star$ là các biến, với $f$ là 1 hàm số có thể viết được nhờ thư viện \texttt{math} của Python.
\end{luuy}

\begin{baitoan}[\cite{VietSTEM2021}, 2., p. 19, Bắc Giang 2020]
	Nhà An có 1 trang trại rộng lớn. Do sở thích của An nên bố An chỉ nuôi gà \& chó. 1 hôm bố An đố con gái nhà mình nuôi bao nhiêu gà, bao nhiêu chó? Bố An cho biết nhà có tổng số gà \& chó là $x$ con. Do số lượng nhiều \& khó đếm từng loại nên An chỉ đếm được tổng số chân của gà \& chó là $y$ chân. Giúp An trả lời câu đố.
	\begin{itemize}
		\item {\sf Dữ liệu vào:} Đọc từ file văn bản \verb|toanco.inp| gồm 2 số nguyên dương $x,y$ trên 1 dòng. 2 số cách nhau 1 khoảng trống ($x\le10^5$, $y\le4\cdot10^5$).
		\item {\sf Kết quả:} Ghi ra file văn bản \verb|toanco.out| gồm 2 số tương ứng là số gà \& số chó tìm được. 2 số cách nhau 1 khoảng trống. Giả sử bài toán luôn có nghiệm.
		\begin{table}[H]
			\centering
			\begin{tabular}{|l|l|}
				\hline
				\texttt{toanco.inp} & \texttt{toanco.out} \\
				\hline
				36 100 & 22 14 \\
				\hline
			\end{tabular}
		\end{table}
	\end{itemize}
\end{baitoan}
Python script: \href{https://github.com/NQBH/hobby/blob/master/elementary_computer_science/Python/toanco.py}{GitHub\texttt{/}NQBH\texttt{/}hobby\texttt{/}elementary computer science\texttt{/}Python\texttt{/}toanco.py}\footnote{\textsc{url}: \url{https://github.com/NQBH/hobby/blob/master/elementary_computer_science/Python/toanco.py}.}. Input: \href{https://github.com/NQBH/hobby/blob/master/elementary_computer_science/Python/toanco.inp}{toanco.inp}. Output: \href{https://github.com/NQBH/hobby/blob/master/elementary_computer_science/Python/toanco.out}{toanco.out}.
\begin{verbatim}
	file_in = open("toanco.inp")
	file_out = open("toanco.out", "w")
	s = file_in.readline()
	s = s.split()
	x = int(s[0])
	y = int(s[1])
	a = int(2*x - y/2)
	b = int(y/2 - x)
	file_out.write(str(a) + " " + str(b))
	file_in.close()
	file_out.close()
\end{verbatim}

\begin{baitoan}[\cite{VietSTEM2021}, 4., p. 26, Quãng Ngãi 2020, Lãi suất-- Interest rate]
	1 người gửi tiền vào ngân hàng có kỳ hạn là $c$ tháng với lãi suất mỗi tháng là $k$\emph{\%}, số tiền gửi ban đầu là $A$ (đơn vị triệu đồng). 
	\begin{itemize}
		\item {\sf Yêu cầu:} Tính số tiền người đó nhận được sau $t$ tháng. Biết tiền lãi mỗi tháng được cộng dồn vào tiền gốc, nếu nhận tiền trước kỳ hạn thì số tiền được tính với lãi suất không kỳ hạn là $h$\emph{\%} của số tiền ban đầu $A$ nhân với số tháng đã gửi. Trong trường hợp rút tiền sau kỳ hạn thì số tháng sau kỳ hạn sẽ được tính với lãi suất không kỳ hạn là $h$\emph{\%} so với số tiền thu được đã qua kỳ hạn.
		\item {\sf Dữ liệu vào:} Tệp văn bản \verb|bl2.inp| ghi $5$ số kỳ hạn $c$ (nếu $c = 0$ là gửi không kỳ hạn), thời gian gửi $t$, số tiền ban đầu $A$, lãi suất có kỳ hạn $k$, lãi suất không kỳ hạn $h$, các số cách nhau 1 ký tự trắng.
		\item {\sf Dữ liệu ra:} Tệp văn bản \verb|bl2.out| ghi $1$ số là số tiền nhận được (làm tròn đến $1$ số lẻ sau dấu chấm thập phân). E.g.,
		\begin{table}[H]
			\centering
			\begin{tabular}{|l|l|}
				\hline
				\texttt{bl2.inp} & \texttt{bl2.out} \\
				\hline
				12 13 100 1.0 0.2 & 112.9 \\
				\hline
				0 10 100 1.0 0.2 & 102.0 \\
				\hline
			\end{tabular}
		\end{table}
	\end{itemize}
\end{baitoan}

%------------------------------------------------------------------------------%

\section{Number Theory -- Số Học}

\begin{baitoan}[\cite{VietSTEM2021}, 3., p. 20, Yên Bái 2020, Tổng nguyên tố]
	Viết chương trình nhập vào $2$ số nguyên $a,b\in\mathbb{Z}$, $0 < a < b$. (a) Tìm \& tính tổng các số nguyên tố của dãy số từ $a$ đến $b$. (b) Xuất ra màn hình các số chia hết cho $5$ của dãy số từ $a$ đến $b$. (c) \emph{(Bội của $n\in\mathbb{N}^\star$)} Xuất ra màn hình các số chia hết cho $n$ của dãy số từ $a$ đến $b$ với $n\in\mathbb{N}^\star$ được nhập từ bàn phím. E.g., nhập $a = 6$, $b = 22$. Kết quả tổng các số nguyên tố trong dãy số từ $6$ đến $22$: $7 + 11 + 13 + 17 + 19 = 67$. Các số chia hết cho $5$ của dãy số từ $6$ đến $22$: $10,15,20$.
\end{baitoan}

\begin{baitoan}[\cite{VietSTEM2021}, 4., p. 22, Hải Dương 2020, Số mạnh mẽ]
	\emph{Số mạnh mẽ} là số khi nó chia hết cho 1 số nguyên tố thì cũng chia hết cho cả bình phương của số nguyên tố đó, i.e., $a\in\mathbb{N}^\star$ là số mạnh mẽ $\Leftrightarrow$ ($a\divby p\Rightarrow a\divby p^2$, $\forall p$: prime). E.g., $25$ là số mạnh mẽ, vì nó chia hết cho số nguyên tố $5$ \& chia hết cho cả $5^2 = 25$. Viết chương trình liệt kê các số mạnh mẽ không vượt quá $1000$.
\end{baitoan}
See, e.g., \href{https://en.wikipedia.org/wiki/Powerful_number}{Wikipedia\texttt{/}powerful number}, \href{https://mathworld.wolfram.com/PowerfulNumber.html}{MathWorld\texttt{/}powerful number}.

\begin{baitoan}[\cite{VietSTEM2021}, 5., p. 23, Việt Nam 2020, Bội chính phương]
	Cho 1 dãy số $A$ có $n$ phần tử. Tìm số nguyên dương $P$ nhỏ nhất thỏa mãn: $a$ là số chính phương \& $a$ chia hết cho tất cả các phần tử của dãy số $A$.
	\begin{itemize}
		\item {\sf Yêu cầu:} In ra phần dư của phép chia khi chia $a$ cho $10^9 + 7$.
		\item {\sf Dữ liệu vào:} Vào từ thiết bị theo khuôn dạng sau: Dòng đầu tiên chứa số nguyên dương $n$ là số lượng phần tử của dãy số. Dòng tiếp theo chứa $n$ số nguyên dương là các phần tử của dãy $A$. Các số trên 1 dòng được ghi cách nhau bởi dấu cách.
		\item {\sf Kết quả:} Ghi ra thiết bị ra gồm 1 số nguyên duy nhất là kết quả của bài toán. E.g.,
		\begin{table}[H]
			\centering
			\begin{tabular}{|l|l|}
				\hline
				Dữ liệu vào & Dữ liệu ra \\
				\hline
				3 & 36 \\
				2 1 3 &  \\
				\hline
			\end{tabular}
		\end{table}
	\end{itemize}
\end{baitoan}

\begin{baitoan}[\cite{VietSTEM2021}, 1., p. 25, Hải Dương 2020, Số hạnh phúc \& số buồn bã-- Happy- \& sad numbers]
	Với 1 số nguyên dương bất kỳ, thay thế số đó bằng tổng bình phương các chữ số của nó \& cứ lặp lại quá trình đó sẽ có các trường hợp sau xảy ra: Kết thúc bằng $1$ -- ta gọi số đó là \emph{số hạnh phúc\texttt{/}happy number}. Kết thúc bằng $0$ -- ta gọi số đó là \emph{số buồn bã\texttt{/}sad number}. Lặp lại vô hạn lần -- số đó không hạnh phúc cũng không buồn bã. E.g., số $44$: lần 1: $4^2 + 4^2 = 32$, lần 2: $3^2 + 2^2 = 13$, lần 3: $1^2 + 3^2 = 10$, lần 4: $1^2 + 0^2 = 1$, nên $44$ là số hạnh phúc. Viết chương trình để kiểm tra xem ngày sinh của 1 người bất kỳ có phải là số hạnh phúc không?
\end{baitoan}

\begin{baitoan}[\cite{VietSTEM2021}, 2., p. 25, Gia Lai 2019, Phân số tối giản -- Irreducible fraction]
	1 chuỗi được gọi là có dạng phân số nếu nó có dạng \verb|`tử_số/mẫu_số'|. Viết chương trình nhập vào chuỗi có dạng phân số, sau đó xuất ra dạng tối giản của phân số đó. E.g., Chuỗi \verb|`12/15'| biểu diễn cho phân số. Dạng tối giản của phân số đó là \verb|`3/5'|.
\end{baitoan}

\begin{baitoan}[Tổng tất cả, tổng phần tử chẵn, lẻ, bình phương, lập phương, lũy thừa bậc $n$, căn bậc 2, 3, \& căn bậc $n$, nghịch đảo, nghịch đảo bình phương, nghịch đảo lập phương, nghịch đảo lũy thừa bậc $n$, nghịch đảo căn bậc 2, 3, \& nghịch đảo căn bậc $n$ -- Sums of all, odds, evens, squares, cubes, $n$th powers, square roots, cube roots, $n$th roots, reciprocals of square, of cubes, of $n$th powers, of square roots, of cube roots, of $n$th roots]
	Cho 1 dãy gồm $n$ số nguyên: $(a_i)_{i=1}^m = a_1,a_2,\ldots,a_m$, $m\in\mathbb{N}^\star$, $a_i\in\mathbb{Z}$, $\forall i = 1,2,\ldots,m$, mỗi số có giá trị không vượt quá $10^9$.
	\begin{itemize}
		\item {\sf Yêu cầu:} Tính tổng $S$ tất cả các phần tử, tổng $S_{\rm even}$ các số chẵn, tổng $S_{\rm odd}$ các số lẻ, tổng $S_{\rm sqr}$ bình phương, tổng $S_{\rm sqr,even}$ bình phương các số chẵn, tổng $S_{\rm sqr,odd}$ bình phương các số lẻ, tổng $S_{\rm cb}$ lập phương, tổng $S_{\rm cb,even}$ lập phương các số chẵn, tổng $S_{\rm cb,odd}$ lập phương các số lẻ, tổng $S_{{\rm pwr},n}$ lũy thừa bậc $n$, tổng $S_{{\rm pwr,even},n}$ lũy thừa bậc $n$ các số chẵn, tổng $S_{{\rm pwr,odd},n}$ lũy thừa bậc $n$ các số lẻ, tổng $S_{\rm sqrt}$ căn bậc 2, tổng $S_{\rm sqrt,even}$ căn bậc 2 các số chẵn, tổng $S_{\rm sqrt,odd}$ căn bậc 2 các số lẻ, tổng $S_{\rm cbrt}$ căn bậc 3, tổng $S_{\rm cbrt,even}$ căn bậc 3 các số chẵn, tổng $S_{\rm cbrt,odd}$ căn bậc 3 các số lẻ, tổng $S_{{\rm rt},n}$ căn bậc $n$ của các số, tổng $S_{{\rm rt,even},n}$ căn bậc $n$ của các số chẵn, tổng $S_{{\rm rt,odd},n}$ căn bậc $n$ của các số lẻ trong dãy $(a_i)_{i=1}^m$.
		\item {\sf Dữ liệu:} Dòng đầu tiên chứa $m\in\mathbb{N}^\star$, $1\le m\le10^9$. Dòng thứ 2 chứa $n\in\mathbb{N}^\star$. $m$ dòng tiếp theo, dòng thứ $i + 2$ chứa $a_i$, $\forall i = 1,2,\ldots,m - 1$.
	\end{itemize}
\end{baitoan}

\begin{proof}[Giải]
	Công thức toán học tính các tổng:
	\begin{align*}
		S&\coloneqq\sum_{i=1}^m a_i = a_1 + a_2 + \cdots + a_m,\ S_{\rm even}\coloneqq\sum_{i=1,\,2\mid a_i}^m a_i,\ S_{\rm odd}\coloneqq\sum_{i=1,\,2\nmid a_i}^m a_i,\\
		S_{\rm sqr}&\coloneqq\sum_{i=1}^m a_i^2 = a_1^2 + a_2^2 + \cdots + a_m^2,\ S_{\rm sqr,even}\coloneqq\sum_{i=1,\,2\mid a_i}^m a_i^2,\ S_{\rm sqr,odd}\coloneqq\sum_{i=1,\,2\nmid a_i}^m a_i^2,\\
		S_{\rm cb}&\coloneqq\sum_{i=1}^m a_i^3 = a_1^3 + a_2^3 + \cdots + a_m^3,\ S_{\rm cb, even}\coloneqq\sum_{i=1,\,2\mid a_i}^m a_i^3,\ S_{\rm odd}\coloneqq\sum_{i=1,\,2\nmid a_i}^m a_i^3,\\
		S_{{\rm pwr},n}&\coloneqq\sum_{i=1}^m a_i^n = a_1^n + a_2^n + \cdots + a_m^n,\ S_{{\rm pwr,even},n}\coloneqq\sum_{i=1,\,2\mid a_i}^m a_i^n,\ S_{{\rm pwr,odd},n}\coloneqq\sum_{i=1,\,2\nmid a_i}^m a_i^n,\ \forall n\in\mathbb{N}^\star,\\
		S_{\rm sqrt}&\coloneqq\sum_{i=1}^m \sqrt{a_i} = \sqrt{a_1} + \sqrt{a_2} + \cdots + \sqrt{a_m},\ S_{\rm sqrt,even}\coloneqq\sum_{i=1,\,2\mid a_i}^m \sqrt{a_i},\ S_{\rm sqrt,odd}\coloneqq\sum_{i=1,\,2\nmid a_i}^m \sqrt{a_i},\\
		S_{\rm cbrt}&\coloneqq\sum_{i=1}^m \sqrt[3]{a_i} = \sqrt[3]{a_1} + \sqrt[3]{a_2} + \cdots + \sqrt[3]{a_m},\ S_{\rm cbrt,even}\coloneqq\sum_{i=1,\,2\mid a_i}^m \sqrt[3]{a_i},\ S_{\rm cbrt,odd}\coloneqq\sum_{i=1,\,2\nmid a_i}^m \sqrt[3]{a_i},\\
		S_{{\rm rt},n}&\coloneqq\sum_{i=1}^m \sqrt[n]{a_i} = \sqrt[n]{a_1} + \sqrt[n]{a_2} + \cdots + \sqrt[n]{a_m},\ S_{{\rm rt,even},n}\coloneqq\sum_{i=1,\,2\mid a_i}^m \sqrt[n]{a_i},\ S_{{\rm rt,odd},n}\coloneqq\sum_{i=1,\,2\nmid a_i}^m \sqrt[n]{a_i},\ \forall n\in\mathbb{N}^\star,\\
		S_{\rm rcpc}&\coloneqq\sum_{i=1}^m \frac{1}{a_i} = \frac{1}{a_1} + \frac{1}{a_2} + \cdots + \frac{1}{a_m},\ S_{\rm rcpc,even}\coloneqq\sum_{i=1,\,2\mid a_i,\,a_i\ne0}^m \frac{1}{a_i},\ S_{\rm rcpc,odd}\coloneqq\sum_{i=1,\,2\nmid a_i}^m \frac{1}{a_i},\\
		S_{\rm rcpc,sqr}&\coloneqq\sum_{i=1}^m \frac{1}{a_i^2} = \frac{1}{a_1^2} + \frac{1}{a_2^2} + \cdots + \frac{1}{a_m^2},\ S_{\rm rcpc,sqr,even}\coloneqq\sum_{i=1,\,2\mid a_i,\,a_i\ne0}^m \frac{1}{a_i^2},\ S_{\rm rcpc,sqr,odd}\coloneqq\sum_{i=1,\,2\nmid a_i}^m \frac{1}{a_i^2},\\
		S_{\rm rcpc,cb}&\coloneqq\sum_{i=1}^m \frac{1}{a_i^3} = \frac{1}{a_1^3} + \frac{1}{a_2^3} + \cdots + \frac{1}{a_m^3},\ S_{\rm rcpc,cb,even}\coloneqq\sum_{i=1,\,2\mid a_i,\,a_i\ne0}^m \frac{1}{a_i^3},\ S_{\rm rcpc,cb,odd}\coloneqq\sum_{i=1,\,2\nmid a_i}^m \frac{1}{a_i^3},\\
		S_{{\rm rcpc,pwr},n}&\coloneqq\sum_{i=1}^m \frac{1}{a_i^n} = \frac{1}{a_1^n} + \frac{1}{a_2^n} + \cdots + \frac{1}{a_m^n},\ S_{{\rm rcpc,even,pwr},n}\coloneqq\sum_{i=1,\,2\mid a_i,\,a_i\ne0}^m \frac{1}{a_i^n},\ S_{{\rm rcpc,odd,pwr},n}\coloneqq\sum_{i=1,\,2\nmid a_i}^m \frac{1}{a_i^n},\ \forall n\in\mathbb{N}^\star,\\
		S_{\rm rcpc,sqrt}&\coloneqq\sum_{i=1}^m \frac{1}{\sqrt{a_i}} = \frac{1}{\sqrt{a_1}} + \frac{1}{\sqrt{a_2}} + \cdots + \frac{1}{\sqrt{a_m}},\ S_{\rm rcpc,sqrt,even}\coloneqq\sum_{i=1,\,2\mid a_i,\,a_i\ne0}^m \frac{1}{\sqrt{a_i}},\ S_{\rm rcpc,sqrt,odd}\coloneqq\sum_{i=1,\,2\nmid a_i}^m \frac{1}{\sqrt{a_i}},\\
		S_{\rm rcpc,cbrt}&\coloneqq\sum_{i=1}^m \frac{1}{\sqrt[3]{a_i}} = \frac{1}{\sqrt[3]{a_1}} + \frac{1}{\sqrt[3]{a_2}} + \cdots + \frac{1}{\sqrt[3]{a_m}},\ S_{\rm rcpc,cbrt,even}\coloneqq\sum_{i=1,\,2\mid a_i,\,a_i\ne0}^m \frac{1}{\sqrt[3]{a_i}},\ S_{\rm rcpc,cbrt,odd}\coloneqq\sum_{i=1,\,2\nmid a_i}^m \frac{1}{\sqrt[3]{a_i}},\\
		S_{{\rm rcpc,rt},n}&\coloneqq\sum_{i=1}^m \frac{1}{\sqrt[n]{a_i}} = \frac{1}{\sqrt[n]{a_1}} + \frac{1}{\sqrt[n]{a_2}} + \cdots + \frac{1}{\sqrt[n]{a_m}},\ S_{{\rm rcpc,even,rt},n}\coloneqq\sum_{i=1,\,2\mid a_i,\,a_i\ne0}^m \frac{1}{\sqrt[n]{a_i}},\ S_{{\rm rcpc,odd,rt},n}\coloneqq\sum_{i=1,\,2\nmid a_i}^m \frac{1}{\sqrt[n]{a_i}},\ \forall n\in\mathbb{N}^\star.
	\end{align*}
	Dựa vào các công thức này, sử dụng vòng lặp \texttt{for} hoặc \texttt{while} để tính các tổng này.
\end{proof}

\begin{nhanxet}
	Nếu chỉ tính tổng $S_{\rm odd}$ các số lẻ của dãy $(a_i)_{i=1}^n\subset\mathbb{Z}$ thì ta có bài toán \emph{\cite[3., p. 25, Tây Ninh 2019]{VietSTEM2021}}.
\end{nhanxet}

\begin{nhanxet}[Mở rộng $\mathbb{Z}$ ra $\mathbb{R},\mathbb{C}$]
	Các tổng $S,S_{\rm sqr},S_{\rm cb},S_{{\rm pwr},n},S_{\rm sqrt},S_{\rm cbrt},S_{{\rm rt},n},S_{\rm rcpc}$ (i.e., các tổng không có liên quan đến tính chẵn lẻ) vẫn có thể áp dụng cho các dãy số thực thay vì chỉ cho dãy số nguyên, i.e., áp dụng cho $(a_i)_{i=1}^n\subset\mathbb{R}$, $a_i\in\mathbb{R}$, $\forall i = 1,2,\ldots,n$, thay vì chỉ cho $(a_i)_{i=1}^n\subset\mathbb{R}$, $a_i\in\mathbb{Z}$, $\forall i = 1,2,\ldots,n$, thậm chí có thể áp dụng cho các dãy số phức $(a_i)_{i=1}^n\subset\mathbb{C}$, $a_i\in\mathbb{C}$, $\forall i = 1,2,\ldots,n$.
\end{nhanxet}

\begin{nhanxet}[Mở rộng từ dãy hữu hạn sang chuỗi]
	Bài toán trên có thể mở rộng ra cho chuỗi (series) số nguyên $(a_n)_{i=1}^\infty\subset\mathbb{Z}$, chuỗi số thực $(a_n)_{i=1}^\infty\subset\mathbb{R}$, \& chuỗi số phức $(a_n)_{i=1}^\infty\subset\mathbb{C}$. Đương nhiên, 1 chương trình máy tính chỉ có thể lặp (e.g., \texttt{for, while}) hữu hạn lần chứ không thể lặp vô hạn lần (infinite loop error) nên ta chỉ có thể tính tổng riêng $S_m$ của chuỗi $S$, e.g.,
	\begin{align*}
		S_m\coloneqq\sum_{i=1}^m a_i\to S\coloneqq\sum_{i=1}^\infty a_i\mbox{ as } n\to\infty,\mbox{ i.e. } \lim_{m\to\infty} S_m = S\mbox{ if } S\in\overline{\mathbb{R}},
	\end{align*}
	trong đó $\overline{\mathbb{R}}\coloneqq\mathbb{R}\cup\{\pm\infty\}$ ký hiệu tập số thực mở rộng bao gồm tập số thực, âm- \& dương vô cực.
\end{nhanxet}

\begin{baitoan}[\cite{VietSTEM2021}, 5., p. 26, Nghệ An 2019, Nguyên tố -- Prime, Goldbach conjecture]
	Minh đố An: Cho 1 số chẵn $k\in\mathbb{N}$, $2\le k\le1000$, tìm 2 số nguyên tố sao cho tổng của chúng bằng số chẵn $k$ đã cho.
	\begin{itemize}
		\item {\sf Yêu cầu:} Viết chương trình \textsc{Pascal, Python, C\texttt{/}C++} giúp An trả lời câu hỏi của Minh.
		\item {\sf Dữ liệu vào:} Tệp văn bản \verb|prime.inp|: Dòng đầu tiên chứa $n\in\mathbb{N}^\star$ tương ứng số test. $n$ dòng tiếp theo, mỗi dòng chứa 1 số $k$, i.e., $k_i$, $i = 1,2,\ldots,n$.
		\item {\sf Dữ liệu ra:} Tệp văn bản \verb|prime.out| gồm $n$ dòng tương ứng $n$ kết quả. Mỗi kết quả hiển thị tổng 2 số nguyên tố bằng số $k$ nhập vào. E.g.,
		\begin{table}[H]
			\centering
			\begin{tabular}{|l|l|}
				\hline
				\texttt{prime.inp} & \texttt{prime.out} \\
				\hline
				2 & $8 = 5 + 3$ \\
				8 & $24 = 19 + 5$ \\
				24 & \\
				\hline
			\end{tabular}
		\end{table}
	\end{itemize}
\end{baitoan}

\begin{baitoan}[\cite{VietSTEM2021}, 6., p. 27, Tây Ninh 2019, Số hoàn hảo -- Perfect number]
	\emph{Số hoàn hảo} là 1 số tự nhiên mà tổng tất cả các ước tự nhiên thực sự của nó bằng chính nó. Trong đó ước thực sự của 1 số là các ước dương không bằng số đó. Lập trình nhập vào 1 số tự nhiên có $2$ chữ số bất kỳ. In ra màn hình thông báo số vừa nhập có phải là số hoàn hảo hay không? Nếu là số hoàn hảo thì in tất cả các ước của số đó.
\end{baitoan}

\begin{baitoan}[\cite{VietSTEM2021}, 7., p. 27, Đồng Nai 2020, Số may mắn -- Lucky number]
	Để động viên thành tích học tập xuất sắc của các em học sinh lớp 6-3 trong năm học 2019--2020, thầy giáo chủ nhiệm đã chuẩn bị các món quà được đánh số từ $1$ đến $n$. Sau đó thầy giáo sẽ cho các em lên bốc thăm để nhận món quà may mắn của mình. Đầu tiên thầy giáo sẽ ghi tất cả số nguyên lẻ từ $1$ đến $n$, sau đó sẽ ghi tất cả các số nguyên chẵn từ $2$ đến $n$ (theo thứ tự tăng dần) để tạo thành 1 dãy số phần thưởng. Mỗi bạn sẽ bốc thăm 1 số $k$ ứng với con số của món quà mình đạt được.
	\begin{itemize}
		\item {\sf Yêu cầu:} In số của món quà học sinh đạt được.
		\item {\sf Dữ liệu vào:} Dòng duy nhất ghi số nguyên $n$ \& $k$, $1\le k\le n\le1000$.
		\item {\sf Dữ liệu ra:} In số của món quà học sinh đạt được:
		\begin{table}[H]
			\centering
			\begin{tabular}{|l|l|}
				\hline
				\verb|lucky_number.inp| & \verb|lucky_number.out| \\
				\hline
				10 6 & 2 \\
				\hline
			\end{tabular}
		\end{table}
	\end{itemize}
\end{baitoan}

\begin{baitoan}[\cite{VietSTEM2021}, 8., p. 27, Ninh Bình 2019, Ước chung lớn nhất ƯCLN -- greatest common divisor gcd]
	Nhập vào $3$ số từ bàn phím, kiểm soát dữ liệu nhập vào là số nguyên dương. Lập trình tìm \emph{ƯCLN} của $3$ số này. E.g., nhập vào $3$ số: $4,6,12$ thì kết quả \emph{ƯCLN} là $2$.
\end{baitoan}

\begin{baitoan}[Bội chung nhỏ nhất BCNN -- least common multiplier lcd]
	Nhập vào $n\in\mathbb{N}^\star$ số từ bàn phím, kiểm soát dữ liệu nhập vào là số nguyên dương. Lập trình tìm \emph{ƯCLN} \& \emph{BCNN} của $n$ số này.
\end{baitoan}



%------------------------------------------------------------------------------%

\section{Problems in Elementary Physics}

%------------------------------------------------------------------------------%

\section{Problems in Elementary Chemistry}

%------------------------------------------------------------------------------%

\section{Resources}
\cite{VietSTEM2021, VietSTEM2022, TLGK_chuyen_Tin_quyen_1, TLGK_chuyen_Tin_quyen_2, TLGK_chuyen_Tin_quyen_3}.

%------------------------------------------------------------------------------%

\section{Miscellaneous}

\begin{baitoan}[\cite{Olympic30-4_2010_Tin_Hoc}, 1., p. 5, Connect]
	Cho $n$ số nguyên dương $a_1,a_2,\ldots,a_n$, $n\in\mathbb{N}$, $1 < n\le100$, $0 < a_i\le10^9$, $\forall i = 1,2,\ldots,n$. Từ các số nguyên này người ta tạo ra 1 số nguyên mới bằng cách kết nối tất cả các số đã cho viết liên tiếp nhau. E.g., với $n = 4$ \& các số $12,34,567,890$ ta có thể tạo ra các số mới như sau: $1234567890$, $3456789012$, $8905673412$, $\ldots$ Dễ thấy có $4! = 24$ cách tạo mới như vậy. Trong trường hợp này, số lớn nhất có thể tạo thành là $8905673412$.
	\begin{itemize}
		\item {\sf Yêu cầu:} Cho $n$ \& các số $a_1,a_2,\ldots,a_n$. Xác định số lớn nhất có thể kết nối được theo quy tắc trên.
		\item {\sf Dữ liệu vào:} Cho trong file văn bản \verb|connect.inp| gồm $n + 1$ dòng. Dòng đầu tiên ghi số nguyên $n$. Trong các dòng còn lại, dòng thứ $i + 1$ ghi số $a_i$.
		\item \emph{Dữ liệu ra:} Ghi vào file văn bản \verb|connect.out| số lớn nhất được kết nối thành từ các số ban đầu. E.g.,
		\begin{table}[H]
			\centering
			\begin{tabular}{|l|l|}
				\hline
				\texttt{connect.inp} & \texttt{connect.out} \\
				\hline
				4 & 8905673412 \\
				12 &  \\
				34 &  \\
				567 &  \\
				890 &  \\
				\hline
			\end{tabular}
		\end{table}
	\end{itemize}
\end{baitoan}


%------------------------------------------------------------------------------%

\printbibliography[heading=bibintoc]
	
\end{document}