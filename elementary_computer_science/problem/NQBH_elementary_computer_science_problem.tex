\documentclass{article}
\usepackage[backend=biber,natbib=true,style=alphabetic,maxbibnames=10]{biblatex}
\addbibresource{/home/nqbh/reference/bib.bib}
\usepackage[utf8]{vietnam}
\usepackage{tocloft}
\renewcommand{\cftsecleader}{\cftdotfill{\cftdotsep}}
\usepackage[colorlinks=true,linkcolor=blue,urlcolor=red,citecolor=magenta]{hyperref}
\usepackage{amsmath,amssymb,amsthm,float,graphicx,mathtools}
\allowdisplaybreaks
\newtheorem{assumption}{Assumption}
\newtheorem{baitoan}{Bài toán}
\newtheorem{cauhoi}{Câu hỏi}
\newtheorem{conjecture}{Conjecture}
\newtheorem{corollary}{Corollary}
\newtheorem{dangtoan}{Dạng toán}
\newtheorem{definition}{Definition}
\newtheorem{dinhly}{Định lý}
\newtheorem{dinhnghia}{Định nghĩa}
\newtheorem{example}{Example}
\newtheorem{ghichu}{Ghi chú}
\newtheorem{hequa}{Hệ quả}
\newtheorem{hypothesis}{Hypothesis}
\newtheorem{lemma}{Lemma}
\newtheorem{luuy}{Lưu ý}
\newtheorem{nhanxet}{Nhận xét}
\newtheorem{notation}{Notation}
\newtheorem{note}{Note}
\newtheorem{principle}{Principle}
\newtheorem{problem}{Problem}
\newtheorem{proposition}{Proposition}
\newtheorem{question}{Question}
\newtheorem{remark}{Remark}
\newtheorem{theorem}{Theorem}
\newtheorem{vidu}{Ví dụ}
\usepackage[left=1cm,right=1cm,top=5mm,bottom=5mm,footskip=4mm]{geometry}
\def\labelitemii{$\circ$}
\DeclareRobustCommand{\divby}{%
	\mathrel{\vbox{\baselineskip.65ex\lineskiplimit0pt\hbox{.}\hbox{.}\hbox{.}}}%
}

\title{Problems in Elementary Computer Science}
\author{Nguyễn Quản Bá Hồng\footnote{Independent Researcher, Ben Tre City, Vietnam\\e-mail: \texttt{nguyenquanbahong@gmail.com}; website: \url{https://nqbh.github.io}.}}
\date{\today}

\begin{document}
\maketitle
\begin{abstract}
	1 bộ sưu tập các bài toán chọn lọc từ cơ bản đến nâng cao cho Tin học sơ cấp. Phiên bản mới nhất của tài liệu này được lưu trữ ở link sau: \href{https://github.com/NQBH/hobby/blob/master/elementary_computer_science/problem/NQBH_elementary_computer_science_problem.pdf}{GitHub\texttt{/}NQBH\texttt{/}hobby\texttt{/}elementary computer science\texttt{/}problem}\footnote{\textsc{url}: \url{https://github.com/NQBH/hobby/blob/master/elementary_computer_science/problem/NQBH_elementary_computer_science_problem.pdf}.}.
\end{abstract}
\tableofcontents

%------------------------------------------------------------------------------%

\section{Basic Problems}

\begin{baitoan}[Even vs. odd]
	Viết thuật toán \& các chương trình bằng các ngôn ngữ lập trình \textsc{Pascal, Python, C\texttt{/}C++} để xét tính chẵn lẻ của $n\in\mathbb{Z}$ được nhập từ bàn phím.
\end{baitoan}

\begin{itemize}
	\item Pascal script: \href{https://github.com/NQBH/hobby/blob/master/elementary_computer_science/Pascal/even_odd.pas}{GitHub\texttt{/}NQBH\texttt{/}hobby\texttt{/}elementary computer science\texttt{/}Pascal\texttt{/}even vs. odd}.
	\item Python script: \href{https://github.com/NQBH/hobby/blob/master/elementary_computer_science/Python/even_odd.py}{GitHub\texttt{/}NQBH\texttt{/}hobby\texttt{/}elementary computer science\texttt{/}Python\texttt{/}even vs. odd}.
\end{itemize}

\begin{baitoan}[Divisible by]
	Viết thuật toán \& các chương trình bằng các ngôn ngữ lập trình \textsc{Pascal, Python, C\texttt{/}C++} để kiểm tra liệu $a\divby b$ hay không, với $a,b\in\mathbb{Z}$ được nhập từ bàn phím.
\end{baitoan}

\begin{itemize}
	\item Pascal script: \href{https://github.com/NQBH/hobby/blob/master/elementary_computer_science/Pascal/divisible_by.pas}{GitHub\texttt{/}NQBH\texttt{/}hobby\texttt{/}elementary computer science\texttt{/}Pascal\texttt{/}divisible by}.
	\item Python script: \href{https://github.com/NQBH/hobby/blob/master/elementary_computer_science/Python/divisible_by.py}{GitHub\texttt{/}NQBH\texttt{/}hobby\texttt{/}elementary computer science\texttt{/}Python\texttt{/}divisible by}.
\end{itemize}

\begin{baitoan}[Triangle]
	Viết thuật toán \& các chương trình bằng các ngôn ngữ lập trình \textsc{Pascal, Python, C\texttt{/}C++} để liệu $a,b,c$ có phải là độ dài của: (a) 1 tam giác. (b) 1 tam giác nhọn. (c) 1 tam giác vuông. (d) 1 tam giác tù.
\end{baitoan}

\begin{itemize}
	\item Python script: \href{https://github.com/NQBH/hobby/blob/master/elementary_computer_science/Python/triangle.py}{GitHub\texttt{/}NQBH\texttt{/}hobby\texttt{/}elementary computer science\texttt{/}Python\texttt{/}triangle}.
\end{itemize}

\begin{baitoan}[Polynomial equation]
	Viết thuật toán \& các chương trình bằng các ngôn ngữ lập trình \textsc{Pascal, Python, C\texttt{/}C++} để giải phương trình bậc nhất, bậc 2, bậc 3, \& bậc 4 với các hệ số thực được nhập từ bàn phím.
\end{baitoan}

\begin{baitoan}[Fibonacci sequence]
	Viết thuật toán \& các chương trình bằng các ngôn ngữ lập trình \textsc{Pascal, Python, C\texttt{/}C++} để xuất ra màn hình, với $n\in\mathbb{N}$ được nhập từ bàn phím: (a) Số Fibonacci thứ $n$. (b) $n$ số Fibonacci đầu tiên.
\end{baitoan}

\begin{baitoan}[Program to print out 1st $n$ square roots]
	Viết chương trình \textsc{Pascal, C-C++, Python} xuất ra căn bậc 2 của $n$ số tự nhiên đầu tiên với $n\in\mathbb{N}^\star$ được nhập từ bàn phím.
\end{baitoan}

\begin{baitoan}[Số chính phương]
	Viết chương trình \textsc{Pascal, C-C++, Python} để kiểm tra 1 số $n\in\mathbb{N}^\star$ được nhập từ bàn phím có phải là số chính phương hay không.
\end{baitoan}

\begin{baitoan}[Program to print out 1st $n$ cube roots]
	Viết chương trình \textsc{Pascal, C-C++, Python} xuất ra căn bậc 3 của $n$ số tự nhiên đầu tiên với $n\in\mathbb{N}^\star$ được nhập từ bàn phím.
\end{baitoan}

\begin{baitoan}
	Viết chương trình \textsc{Pascal, C-C++, Python} để kiểm tra 1 số $n\in\mathbb{N}^\star$ được nhập từ bàn phím có phải là lập phương của 1 số tự nhiên hay không.
\end{baitoan}

\begin{baitoan}[Program to print out 1st $n$ $n$th roots]
	Viết chương trình \textsc{Pascal, C-C++, Python} xuất ra căn bậc $n$ của $m$ số tự nhiên đầu tiên với $m,n\in\mathbb{N}^\star$ được nhập từ bàn phím.
\end{baitoan}

\begin{baitoan}
	Viết chương trình \textsc{Pascal, C-C++, Python} để kiểm tra 1 số $m$ được nhập từ bàn phím có phải là lũy thừa bậc $n$ của 1 số tự nhiên hay không với $m,n\in\mathbb{N}^\star$ được nhập từ bàn phím.
\end{baitoan}

%------------------------------------------------------------------------------%

\printbibliography[heading=bibintoc]
	
\end{document}