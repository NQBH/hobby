\documentclass{article}
\usepackage[backend=biber,natbib=true,style=alphabetic,maxbibnames=10]{biblatex}
\addbibresource{/home/nqbh/reference/bib.bib}
\usepackage[utf8]{vietnam}
\usepackage{tocloft}
\renewcommand{\cftsecleader}{\cftdotfill{\cftdotsep}}
\usepackage[colorlinks=true,linkcolor=blue,urlcolor=red,citecolor=magenta]{hyperref}
\usepackage{amsmath,amssymb,amsthm,float,graphicx,mathtools}
\allowdisplaybreaks
\newtheorem{assumption}{Assumption}
\newtheorem{baitoan}{Bài toán}
\newtheorem{cauhoi}{Câu hỏi}
\newtheorem{conjecture}{Conjecture}
\newtheorem{corollary}{Corollary}
\newtheorem{dangtoan}{Dạng toán}
\newtheorem{definition}{Definition}
\newtheorem{dinhly}{Định lý}
\newtheorem{dinhnghia}{Định nghĩa}
\newtheorem{example}{Example}
\newtheorem{ghichu}{Ghi chú}
\newtheorem{hequa}{Hệ quả}
\newtheorem{hypothesis}{Hypothesis}
\newtheorem{lemma}{Lemma}
\newtheorem{luuy}{Lưu ý}
\newtheorem{nhanxet}{Nhận xét}
\newtheorem{notation}{Notation}
\newtheorem{note}{Note}
\newtheorem{principle}{Principle}
\newtheorem{problem}{Problem}
\newtheorem{proposition}{Proposition}
\newtheorem{question}{Question}
\newtheorem{remark}{Remark}
\newtheorem{theorem}{Theorem}
\newtheorem{vidu}{Ví dụ}
\usepackage[left=1cm,right=1cm,top=5mm,bottom=5mm,footskip=4mm]{geometry}
\def\labelitemii{$\circ$}
\DeclareRobustCommand{\divby}{%
	\mathrel{\vbox{\baselineskip.65ex\lineskiplimit0pt\hbox{.}\hbox{.}\hbox{.}}}%
}

\title{Problems in Elementary Computer Science}
\author{Nguyễn Quản Bá Hồng\footnote{Independent Researcher, Ben Tre City, Vietnam\\e-mail: \texttt{nguyenquanbahong@gmail.com}; website: \url{https://nqbh.github.io}.}}
\date{\today}

\begin{document}
\maketitle
\begin{abstract}
	1 bộ sưu tập các bài toán chọn lọc từ cơ bản đến nâng cao cho Tin học sơ cấp. Phiên bản mới nhất của tài liệu này được lưu trữ ở link sau: \href{https://github.com/NQBH/hobby/blob/master/elementary_computer_science/problem/NQBH_elementary_computer_science_problem.pdf}{GitHub\texttt{/}NQBH\texttt{/}hobby\texttt{/}elementary computer science\texttt{/}problem}\footnote{\textsc{url}: \url{https://github.com/NQBH/hobby/blob/master/elementary_computer_science/problem/NQBH_elementary_computer_science_problem.pdf}.}.
\end{abstract}
\tableofcontents

%------------------------------------------------------------------------------%

\section{Basic Problems}

\begin{baitoan}[Even vs. odd]
	Viết thuật toán \& các chương trình bằng các ngôn ngữ lập trình \textsc{Pascal, Python, C\texttt{/}C++} để xét tính chẵn lẻ của $n\in\mathbb{Z}$ được nhập từ bàn phím.
\end{baitoan}

\begin{itemize}
	\item Pascal script: \href{https://github.com/NQBH/hobby/blob/master/elementary_computer_science/Pascal/even_odd.pas}{GitHub\texttt{/}NQBH\texttt{/}hobby\texttt{/}elementary computer science\texttt{/}Pascal\texttt{/}even vs. odd}.
	\item Python script: \href{https://github.com/NQBH/hobby/blob/master/elementary_computer_science/Python/even_odd.py}{GitHub\texttt{/}NQBH\texttt{/}hobby\texttt{/}elementary computer science\texttt{/}Python\texttt{/}even vs. odd}.
\end{itemize}

\begin{baitoan}[Divisible by]
	Viết thuật toán \& các chương trình bằng các ngôn ngữ lập trình \textsc{Pascal, Python, C\texttt{/}C++} để kiểm tra liệu $a\divby b$ hay không, với $a,b\in\mathbb{Z}$ được nhập từ bàn phím.
\end{baitoan}

\begin{itemize}
	\item Pascal script: \href{https://github.com/NQBH/hobby/blob/master/elementary_computer_science/Pascal/divisible_by.pas}{GitHub\texttt{/}NQBH\texttt{/}hobby\texttt{/}elementary computer science\texttt{/}Pascal\texttt{/}divisible by}.
	\item Python script: \href{https://github.com/NQBH/hobby/blob/master/elementary_computer_science/Python/divisible_by.py}{GitHub\texttt{/}NQBH\texttt{/}hobby\texttt{/}elementary computer science\texttt{/}Python\texttt{/}divisible by}.
\end{itemize}

\begin{baitoan}[Triangle]
	Viết thuật toán \& các chương trình bằng các ngôn ngữ lập trình \textsc{Pascal, Python, C\texttt{/}C++} để liệu $a,b,c$ có phải là độ dài của: (a) 1 tam giác. (b) 1 tam giác nhọn. (c) 1 tam giác vuông. (d) 1 tam giác tù.
\end{baitoan}

\begin{itemize}
	\item Python script: \href{https://github.com/NQBH/hobby/blob/master/elementary_computer_science/Python/triangle.py}{GitHub\texttt{/}NQBH\texttt{/}hobby\texttt{/}elementary computer science\texttt{/}Python\texttt{/}triangle}.
\end{itemize}

\begin{baitoan}[Polynomial equation]
	Viết thuật toán \& các chương trình bằng các ngôn ngữ lập trình \textsc{Pascal, Python, C\texttt{/}C++} để giải phương trình bậc nhất, bậc 2, bậc 3, \& bậc 4 với các hệ số thực được nhập từ bàn phím.
\end{baitoan}

\begin{baitoan}[Fibonacci sequence]
	Viết thuật toán \& các chương trình bằng các ngôn ngữ lập trình \textsc{Pascal, Python, C\texttt{/}C++} để xuất ra màn hình, với $n\in\mathbb{N}$ được nhập từ bàn phím: (a) Số Fibonacci thứ $n$. (b) $n$ số Fibonacci đầu tiên.
\end{baitoan}

\begin{baitoan}[Program to print out 1st $n$ square roots]
	Viết chương trình \textsc{Pascal, C\texttt{/}C++, Python} xuất ra căn bậc 2 của $n$ số tự nhiên đầu tiên với $n\in\mathbb{N}^\star$ được nhập từ bàn phím.
\end{baitoan}

\begin{baitoan}[Số chính phương]
	Viết chương trình \textsc{Pascal, C\texttt{/}C++, Python} để kiểm tra 1 số $n\in\mathbb{N}^\star$ được nhập từ bàn phím có phải là số chính phương hay không.
\end{baitoan}

\begin{baitoan}[Program to print out 1st $n$ cube roots]
	Viết chương trình \textsc{Pascal, C\texttt{/}C++, Python} xuất ra căn bậc 3 của $n$ số tự nhiên đầu tiên với $n\in\mathbb{N}^\star$ được nhập từ bàn phím.
\end{baitoan}

\begin{baitoan}
	Viết chương trình \textsc{Pascal, C\texttt{/}C++, Python} để kiểm tra 1 số $n\in\mathbb{N}^\star$ được nhập từ bàn phím có phải là lập phương của 1 số tự nhiên hay không.
\end{baitoan}

\begin{baitoan}[Program to print out 1st $n$ $n$th roots]
	Viết chương trình \textsc{Pascal, C\texttt{/}C++, Python} xuất ra căn bậc $n$ của $m$ số tự nhiên đầu tiên với $m,n\in\mathbb{N}^\star$ được nhập từ bàn phím.
\end{baitoan}

\begin{baitoan}
	Viết chương trình \textsc{Pascal, C\texttt{/}C++, Python} để kiểm tra 1 số $m$ được nhập từ bàn phím có phải là lũy thừa bậc $n$ của 1 số tự nhiên hay không với $m,n\in\mathbb{N}^\star$ được nhập từ bàn phím.
\end{baitoan}

%------------------------------------------------------------------------------%

\section{Number Theory -- Số Học}

%------------------------------------------------------------------------------%

\section{Algebraic Expression -- Biểu Thức Đại Số}

\begin{baitoan}[\cite{VietSTEM2021}, 1., p. 15, Vũng Tàu 2020]
	Cho $a,b,c\in\mathbb{N}^\star$. \emph{Yêu cầu:} Tính giá trị của biểu thức $S = \dfrac{a^2 + b^2 + c^2}{abc} + \sqrt{abc}$.
	\begin{itemize}
		\item \emph{Dữ liệu vào:} File \verb|root.inp| chứa 3 số nguyên dương $a,b,c$. Mỗi số trên 1 dòng.
		\item \emph{Kết quả:} Ghi vào File \verb|root.out| kết quả $S$ tính được (làm tròn lấy 2 chữ số sau phần thập phân). E.g.,
		\begin{table}[H]
			\centering
			\begin{tabular}{|c|c|}
				\hline
				\texttt{root.inp} & \texttt{root.out} \\
				\hline
				2 & 4.25 \\
				1 &  \\
				2 &  \\
				\hline
			\end{tabular}
		\end{table}
	\end{itemize}
\end{baitoan}
Python script: \href{https://github.com/NQBH/hobby/blob/master/elementary_computer_science/Python/root.py}{GitHub\texttt{/}NQBH\texttt{/}hobby\texttt{/}elementary computer science\texttt{/}Python\texttt{/}root.py}\footnote{\textsc{url}: \url{https://github.com/NQBH/hobby/blob/master/elementary_computer_science/Python/root.py}.}. Input: \href{https://github.com/NQBH/hobby/blob/master/elementary_computer_science/Python/root.inp}{root.inp}. Output: \href{https://github.com/NQBH/hobby/blob/master/elementary_computer_science/Python/root.out}{root.out}.
\begin{verbatim}
	from math import *
	file_in = open("root.inp")
	file_out = open("root.out", "w")
	a = file_in.readline()
	b = file_in.readline()
	c = file_in.readline()
	a = int(a)
	b = int(b)
	c = int(c)
	S = (a**a + b**b + c**c)/(a*b*c) + sqrt(a*b*c)
	S = str(round(S,2))
	file_out.write(S)
	file_in.close()
	file_out.close()
\end{verbatim}

\begin{luuy}
	Tương tự, ta có thể tính hầu như bất kỳ hàm số $f(a,b,c)$ 3 biến $a,b,c$ với $f$ là 1 hàm số có thể viết được nhờ thư viện \texttt{math} của Python. Tổng quát hơn, ta có thể tính bất kỳ hàm số nhiều biến $f(x_1,x_2,\ldots,x_n)$ với $x_i$, $i = 1,2,\ldots,n$, $n\in\mathbb{N}^\star$ là các biến, với $f$ là 1 hàm số có thể viết được nhờ thư viện \texttt{math} của Python.
\end{luuy}

\begin{baitoan}[\cite{VietSTEM2021}, 2., p. 19, Bắc Giang 2020]
	Nhà An có 1 trang trại rộng lớn. Do sở thích của An nên bố An chỉ nuôi gà \& chó. 1 hôm bố An đố con gái nhà mình nuôi bao nhiêu gà, bao nhiêu chó? Bố An cho biết nhà có tổng số gà \& chó là $x$ con. Do số lượng nhiều \& khó đếm từng loại nên An chỉ đếm được tổng số chân của gà \& chó là $y$ chân. Giúp An trả lời câu đố.
	\begin{itemize}
		\item \emph{Dữ liệu vào:} đọc từ file văn bản \verb|toanco.inp| gồm 2 số nguyên dương $x,y$ trên 1 dòng. 2 số cách nhau 1 khoảng trống ($x\le10^5$, $y\le4\cdot10^5$).
		\item \emph{Kết quả:} ghi ra file văn bản \verb|toanco.out| gồm 2 số tương ứng là số gà \& số chó tìm được. 2 số cách nhau 1 khoảng trống. Giả sử bài toán luôn có nghiệm.
		\begin{table}[H]
			\centering
			\begin{tabular}{|c|c|}
				\hline
				\texttt{toanco.inp} & \texttt{toanco.out} \\
				\hline
				36 100 & 22 14 \\
				\hline
			\end{tabular}
		\end{table}
	\end{itemize}
\end{baitoan}
Python script: \href{https://github.com/NQBH/hobby/blob/master/elementary_computer_science/Python/toanco.py}{GitHub\texttt{/}NQBH\texttt{/}hobby\texttt{/}elementary computer science\texttt{/}Python\texttt{/}toanco.py}\footnote{\textsc{url}: \url{https://github.com/NQBH/hobby/blob/master/elementary_computer_science/Python/toanco.py}.}. Input: \href{https://github.com/NQBH/hobby/blob/master/elementary_computer_science/Python/toanco.inp}{toanco.inp}. Output: \href{https://github.com/NQBH/hobby/blob/master/elementary_computer_science/Python/toanco.out}{toanco.out}.
\begin{verbatim}
	file_in = open("toanco.inp")
	file_out = open("toanco.out", "w")
	s = file_in.readline()
	s = s.split()
	x = int(s[0])
	y = int(s[1])
	a = int(2*x - y/2)
	b = int(y/2 - x)
	file_out.write(str(a) + " " + str(b))
	file_in.close()
	file_out.close()
\end{verbatim}

\begin{baitoan}[\cite{VietSTEM2021}, 3., p. 20, Yên Bái 2020]
	Viết chương trình nhập vào $2$ số nguyên $a,b\in\mathbb{Z}$, $0 < a < b$. (a) Tìm \& tính tổng các số nguyên tố của dãy số từ $a$ đến $b$. (b) Xuất ra màn hình các số chia hết cho $5$ của dãy số từ $a$ đến $b$. (c) \emph{(Bội của $n\in\mathbb{N}^\star$)} Xuất ra màn hình các số chia hết cho $n$ của dãy số từ $a$ đến $b$ với $n\in\mathbb{N}^\star$ được nhập từ bàn phím. E.g., nhập $a = 6$, $b = 22$. Kết quả tổng các số nguyên tố trong dãy số từ $6$ đến $22$: $7 + 11 + 13 + 17 + 19 = 67$. Các số chia hết cho $5$ của dãy số từ $6$ đến $22$: $10,15,20$.
\end{baitoan}

\begin{baitoan}[\cite{VietSTEM2021}, 4., p. 22, Hải Dương 2020]
	\emph{Số mạnh mẽ} là số khi nó chia hết cho 1 số nguyên tố thì cũng chia hết cho cả bình phương của số nguyên tố đó, i.e., $a\in\mathbb{N}^\star$ là số mạnh mẽ $\Leftrightarrow$ ($a\divby p\Rightarrow a\divby p^2$, $\forall p$: prime). E.g., $25$ là số mạnh mẽ, vì nó chia hết cho số nguyên tố $5$ \& chia hết cho cả $5^2 = 25$. Viết chương trình liệt kê các số mạnh mẽ không vượt quá $1000$.
\end{baitoan}

%------------------------------------------------------------------------------%

\section{Resources}
\cite{VietSTEM2021, VietSTEM2022, TLGK_chuyen_Tin_quyen_1, TLGK_chuyen_Tin_quyen_2, TLGK_chuyen_Tin_quyen_3}.

\section{Notes on Python Commands}

\begin{enumerate}
	\item Để sử dụng các hàm toán học trong Python, cần import thư viện \texttt{math} vào chương trình: \texttt{from math import *}
	\item Để mở file dữ liệu vào \texttt{prob.INP} chỉ để đọc dữ liệu \& mở file dữ liệu ra \texttt{prob.OUT} để thay đổi dữ liệu trong file: \texttt{file = open("prob.INP")} \& \texttt{file2 = open("prob.OUT", "w")}.
\end{enumerate}

%------------------------------------------------------------------------------%

\section{Miscellaneous}

\begin{baitoan}[\cite{Olympic30-4_2010_Tin_Hoc}, 1., p. 5, Connect]
	Cho $n$ số nguyên dương $a_1,a_2,\ldots,a_n$, $n\in\mathbb{N}$, $1 < n\le100$, $0 < a_i\le10^9$, $\forall i = 1,2,\ldots,n$. Từ các số nguyên này người ta tạo ra 1 số nguyên mới bằng cách kết nối tất cả các số đã cho viết liên tiếp nhau. E.g., với $n = 4$ \& các số $12,34,567,890$ ta có thể tạo ra các số mới như sau: $1234567890$, $3456789012$, $8905673412$, $\ldots$ Dễ thấy có $4! = 24$ cách tạo mới như vậy. Trong trường hợp này, số lớn nhất có thể tạo thành là $8905673412$.
	\begin{itemize}
		\item \emph{Yêu cầu:} Cho $n$ \& các số $a_1,a_2,\ldots,a_n$. Xác định số lớn nhất có thể kết nối được theo quy tắc trên.
		\item \emph{Dữ liệu vào:} Cho trong file văn bản \verb|connect.inp| gồm $n + 1$ dòng. Dòng đầu tiên ghi số nguyên $n$. Trong các dòng còn lại, dòng thứ $i + 1$ ghi số $a_i$.
		\item \emph{Dữ liệu ra:} Ghi vào file văn bản \verb|connect.out| số lớn nhất được kết nối thành từ các số ban đầu. E.g.,
		\begin{table}[H]
			\centering
			\begin{tabular}{|c|c|}
				\hline
				\texttt{connect.inp} & \texttt{connect.out} \\
				\hline
				4 & 8905673412 \\
				12 &  \\
				34 &  \\
				567 &  \\
				890 &  \\
				\hline
			\end{tabular}
		\end{table}
	\end{itemize}
\end{baitoan}


%------------------------------------------------------------------------------%

\printbibliography[heading=bibintoc]
	
\end{document}