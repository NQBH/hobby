\documentclass{article}
\usepackage[backend=biber,natbib=true,style=authoryear]{biblatex}
\addbibresource{/home/nqbh/reference/bib.bib}
\usepackage[utf8]{vietnam}
\usepackage{tocloft}
\renewcommand{\cftsecleader}{\cftdotfill{\cftdotsep}}
\usepackage[colorlinks=true,linkcolor=blue,urlcolor=red,citecolor=magenta]{hyperref}
\usepackage{amsmath,amssymb,amsthm,mathtools,float,graphicx,algpseudocode,algorithm,tcolorbox}
\usepackage[inline]{enumitem}
\allowdisplaybreaks
\numberwithin{equation}{section}
\newtheorem{assumption}{Assumption}[section]
\newtheorem{conjecture}{Conjecture}[section]
\newtheorem{corollary}{Corollary}[section]
\newtheorem{hequa}{Hệ quả}[section]
\newtheorem{definition}{Definition}[section]
\newtheorem{dinhnghia}{Định nghĩa}[section]
\newtheorem{example}{Example}[section]
\newtheorem{vidu}{Ví dụ}[section]
\newtheorem{lemma}{Lemma}[section]
\newtheorem{notation}{Notation}[section]
\newtheorem{principle}{Principle}[section]
\newtheorem{problem}{Problem}[section]
\newtheorem{baitoan}{Bài toán}[section]
\newtheorem{proposition}{Proposition}[section]
\newtheorem{question}{Question}[section]
\newtheorem{cauhoi}{Câu hỏi}[section]
\newtheorem{remark}{Remark}[section]
\newtheorem{luuy}{Lưu ý}[section]
\newtheorem{theorem}{Theorem}[section]
\newtheorem{dinhly}{Định lý}[section]
\usepackage[left=0.5in,right=0.5in,top=1.5cm,bottom=1.5cm]{geometry}
\usepackage{fancyhdr}
\pagestyle{fancy}
\fancyhf{}
\lhead{\small Subsect.~\thesubsection}
\rhead{\small\nouppercase{\leftmark}}
\renewcommand{\subsectionmark}[1]{\markboth{#1}{}}
\cfoot{\thepage}
\def\labelitemii{$\circ$}
\DeclareRobustCommand{\divby}{%
	\mathrel{\vbox{\baselineskip.65ex\lineskiplimit0pt\hbox{.}\hbox{.}\hbox{.}}}%
}

\title{Problems in Elementary Computer Science}
\author{Nguyễn Quản Bá Hồng\footnote{Independent Researcher, Ben Tre City, Vietnam\\e-mail: \texttt{nguyenquanbahong@gmail.com}; website: \url{https://nqbh.github.io}.}}
\date{\today}

\begin{document}
\maketitle
\begin{abstract}
	1 bộ sưu tập các bài toán chọn lọc từ cơ bản đến nâng cao cho Tin học sơ cấp. Phiên bản mới nhất của tài liệu này được lưu trữ ở link sau: \href{https://github.com/NQBH/hobby/blob/master/elementary_computer_science/problem/NQBH_elementary_computer_science_problem.pdf}{GitHub\texttt{/}NQBH\texttt{/}hobby\texttt{/}elementary computer science\texttt{/}problem}\footnote{\textsc{url}: \url{https://github.com/NQBH/hobby/blob/master/elementary_computer_science/problem/NQBH_elementary_computer_science_problem.pdf}.}.
\end{abstract}
\tableofcontents
\newpage

%------------------------------------------------------------------------------%

\section{Basic Problems}

\begin{baitoan}[Even vs. odd]
	Viết thuật toán \& các chương trình bằng các ngôn ngữ lập trình \textsc{Pascal, Python, C\texttt{/}C++} để xét tính chẵn lẻ của $n\in\mathbb{Z}$ được nhập từ bàn phím.
\end{baitoan}

\begin{itemize}
	\item Pascal script: \href{https://github.com/NQBH/hobby/blob/master/elementary_computer_science/Pascal/even_odd.pas}{GitHub\texttt{/}NQBH\texttt{/}hobby\texttt{/}elementary computer science\texttt{/}Pascal\texttt{/}even vs. odd}.
	\item Python script: \href{https://github.com/NQBH/hobby/blob/master/elementary_computer_science/Python/even_odd.py}{GitHub\texttt{/}NQBH\texttt{/}hobby\texttt{/}elementary computer science\texttt{/}Python\texttt{/}even vs. odd}.
\end{itemize}

\begin{baitoan}[Divisible by]
	Viết thuật toán \& các chương trình bằng các ngôn ngữ lập trình \textsc{Pascal, Python, C\texttt{/}C++} để kiểm tra liệu $a\divby b$ hay không, với $a,b\in\mathbb{Z}$ được nhập từ bàn phím.
\end{baitoan}

\begin{itemize}
	\item Pascal script: \href{https://github.com/NQBH/hobby/blob/master/elementary_computer_science/Pascal/divisible_by.pas}{GitHub\texttt{/}NQBH\texttt{/}hobby\texttt{/}elementary computer science\texttt{/}Pascal\texttt{/}divisible by}.
	\item Python script: \href{https://github.com/NQBH/hobby/blob/master/elementary_computer_science/Python/divisible_by.py}{GitHub\texttt{/}NQBH\texttt{/}hobby\texttt{/}elementary computer science\texttt{/}Python\texttt{/}divisible by}.
\end{itemize}

\begin{baitoan}[Triangle]
	Viết thuật toán \& các chương trình bằng các ngôn ngữ lập trình \textsc{Pascal, Python, C\texttt{/}C++} để liệu $a,b,c$ có phải là độ dài của:
	\begin{enumerate*}
		\item[(a)] 1 tam giác;
		\item[(b)] 1 tam giác nhọn;
		\item[(c)] 1 tam giác vuông;
		\item[(d)] 1 tam giác tù.
	\end{enumerate*}
\end{baitoan}

\begin{itemize}
	\item Python script: \href{https://github.com/NQBH/hobby/blob/master/elementary_computer_science/Python/triangle.py}{GitHub\texttt{/}NQBH\texttt{/}hobby\texttt{/}elementary computer science\texttt{/}Python\texttt{/}triangle}.
\end{itemize}

\begin{baitoan}[Polynomial equation]
	Viết thuật toán \& các chương trình bằng các ngôn ngữ lập trình \textsc{Pascal, Python, C\texttt{/}C++} để giải phương trình bậc nhất, bậc 2, bậc 3, \& bậc 4 với các hệ số thực được nhập từ bàn phím.
\end{baitoan}

\begin{baitoan}[Fibonacci sequence]
	Viết thuật toán \& các chương trình bằng các ngôn ngữ lập trình \textsc{Pascal, Python, C\texttt{/}C++} để xuất ra màn hình, với $n\in\mathbb{N}$ được nhập từ bàn phím.:
	\begin{enumerate*}
		\item[(a)] Số Fibonacci thứ $n$ 
		\item[(b)] $n$ số Fibonacci đầu tiên.
	\end{enumerate*}
\end{baitoan}

%------------------------------------------------------------------------------%

\printbibliography[heading=bibintoc]
	
\end{document}