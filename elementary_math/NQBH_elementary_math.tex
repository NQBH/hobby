\documentclass{article}
\usepackage[utf8]{vietnam}
\usepackage{tocloft}
\renewcommand{\cftsecleader}{\cftdotfill{\cftdotsep}}
\usepackage{float}
\usepackage{graphicx}
\usepackage[colorlinks=true,linkcolor=blue,urlcolor=red,citecolor=magenta]{hyperref}
\usepackage{amsmath,amssymb,amsthm,mathtools}
\allowdisplaybreaks
\numberwithin{equation}{section}
\newtheorem{assumption}{Assumption}[section]
\newtheorem{lemma}{Lemma}[section]
\newtheorem{corollary}{Corollary}[section]
\newtheorem{definition}{Definition}[section]
\newtheorem{proposition}{Proposition}[section]
\newtheorem{theorem}{Theorem}[section]
\newtheorem{notation}{Notation}[section]
\newtheorem{remark}{Remark}[section]
\newtheorem{example}{Example}[section]
\newtheorem{ques}{Question}[section]
\newtheorem{problem}{Problem}[section]
\newtheorem{conjecture}{Conjecture}[section]
\usepackage[left=0.5in,right=0.5in,top=1.5cm,bottom=1.5cm]{geometry}
\usepackage{fancyhdr}
\pagestyle{fancy}
\fancyhf{}
\lhead{\small \textsc{Sect.} ~\thesection}
\rhead{\small \nouppercase{\leftmark}}
\renewcommand{\sectionmark}[1]{\markboth{#1}{}}
\cfoot{\thepage}
\def\labelitemii{$\circ$}

\title{Some Topics in Elementary Mathematics}
\author{Nguyễn Quản Bá Hồng}
\date{\today}

\begin{document}
\maketitle
\begin{abstract}
	Một vài chủ đề trong Toán Sơ Cấp và ứng dụng (nếu có) trong Khoa học nói chung và Toán Cao Cấp nói riêng.
\end{abstract}
\tableofcontents

%------------------------------------------------------------------------------%

\section*{General Rules for the Author}
\begin{enumerate}
	\item Always try to find and add physical interpretations and real world applications for the considered mathematical objects or terminologies.
	\item Always consider general problems first and then their particular or special cases, and then (optional) generalizations.
	\item Read terminologies in \href{https://www.wikipedia.org/}{Wikipedia} and check \href{https://math.stackexchange.com/}{Mathematics Stack Exchange} for interpretations and further information.
	\item (Optional) Bridges\texttt{/}connections between elementary and advanced mathematics.
	\item (Optional) Some codes (\textsc{Matlab}, C++, Python, etc.) will be nice for further practice and illustrations.
\end{enumerate}

\section{Combinatorics\texttt{/}Tổ Hợp}

\section{Elementary Algebra\texttt{/}Đại Số Sơ Cấp}

\subsection{Equation \& System of Equations\texttt{/}Phương Trình \& Hệ Phương Trình}

\subsection{Inequality\texttt{/}Bất Đẳng Thức}

\section{Elementary Calculus \& Elementary Analysis\texttt{/}Giải Tích Sơ Cấp}

\section{Elementary Geometry\texttt{/}Hình Học Sơ Cấp}

\subsection{2D Geometry\texttt{/}Hình Học Phẳng}

\subsection{3D Geometry\texttt{/}Hình Học Không Gian}

\section{Probability\texttt{/}Xác Suất}

\section{Statistics\texttt{/}Thống Kê}

\section{Miscellaneous}

\subsection{Discrete Mathematics\texttt{/}Toán Rời Rạc}

%------------------------------------------------------------------------------%
	
\end{document}