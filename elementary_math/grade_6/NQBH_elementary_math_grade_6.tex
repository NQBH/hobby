\documentclass{article}
\usepackage[backend=biber,natbib=true,style=authoryear]{biblatex}
\addbibresource{/home/hong/1_NQBH/reference/bib.bib}
\usepackage[utf8]{vietnam}
\usepackage{tocloft}
\renewcommand{\cftsecleader}{\cftdotfill{\cftdotsep}}
\usepackage[colorlinks=true,linkcolor=blue,urlcolor=red,citecolor=magenta]{hyperref}
\usepackage{amsmath,amssymb,amsthm,mathtools,float,graphicx}
\allowdisplaybreaks
\numberwithin{equation}{section}
\newtheorem{assumption}{Assumption}[section]
\newtheorem{lemma}{Lemma}[section]
\newtheorem{corollary}{Corollary}[section]
\newtheorem{definition}{Definition}[section]
\newtheorem{proposition}{Proposition}[section]
\newtheorem{theorem}{Theorem}[section]
\newtheorem{notation}{Notation}[section]
\newtheorem{remark}{Remark}[section]
\newtheorem{example}{Example}[section]
\newtheorem{ques}{Question}[section]
\newtheorem{problem}{Problem}[section]
\newtheorem{conjecture}{Conjecture}[section]
\usepackage[left=0.5in,right=0.5in,top=1.5cm,bottom=1.5cm]{geometry}
\usepackage{fancyhdr}
\pagestyle{fancy}
\fancyhf{}
\lhead{\small \textsc{Sect.} ~\thesection}
\rhead{\small \nouppercase{\leftmark}}
\renewcommand{\sectionmark}[1]{\markboth{#1}{}}
\cfoot{\thepage}
\def\labelitemii{$\circ$}

\title{Elementary Mathematics\texttt{/}Grade 6}
\author{Nguyễn Quản Bá Hồng}
\date{\today}

\begin{document}
\maketitle
\begin{abstract}
	
\end{abstract}
\tableofcontents

%------------------------------------------------------------------------------%

\section{Phân Số \& Số Thập Phân}

\subsection{Phân Số với Tử \& Mẫu là Số Nguyên}
\begin{definition}[Phân số\texttt{/}Fractionals]
	1 \emph{phân số} có tử và mẫu số là số nguyên là biểu thức có dạng $\frac{a}{b}$, $a,b\in\mathbb{Z}$, $b\ne 0$. $a$: tử số (numerator), $b$: mẫu số (denominator).
	
	Phân số $\frac{a}{b}$, $a\in\mathbb{Z}$, $b\in\mathbb{Z}^\star$, được gọi là \emph{phân số tối giản} nếu $\operatorname{gcd}(a,b) = 1$, ở đây $\operatorname{gcd}$ ký hiệu \emph{ước chung lớn nhất} (greatest common divisor).\footnote{Hoặc ký hiệu Việt Nam là: $\operatorname{UCLN}(a,b)$.}
\end{definition}

\subsubsection{Khái niệm 2 phân số bằng nhau.} 2 phân số được gọi là \textit{bằng nhau} nếu chúng cùng biểu diễn một giá trị, i.e. (tức\texttt{/}nghĩa là),
\begin{align}
	\boxed{\frac{a}{b} = \frac{c}{d}\Leftrightarrow b\ne 0, d\ne 0,\ ad = bc.}
\end{align}
Vế sau có nghĩa là \textit{nhân chéo chia ngang}, hay được gọi là \textit{quy tắc bằng nhau của 2 phân số}.

\textbf{Chú ý.} luôn nhớ điều kiện mẫu số 2 phân số phải khác 0.

\begin{example}
	Trong Sách Giáo Khoa Toán 6, Cánh Diều, của Đỗ Đức Thái chủ biên, có viết:
	\begin{quotation}
		``Xét 2 phân số $\frac{a}{b}$ và $\frac{c}{d}$. Nếu $\frac{a}{b} = \frac{c}{d}$ thì $ad = bc$\footnote{Phép nhân: $a\times b = a\cdot b = ab$.}. Ngược lại, nếu $ad = bc$ thì $\frac{a}{b} = \frac{c}{d}$.''
	\end{quotation}
	Phản ví dụ: $a = 0$, $b = 0$ thì $ad = bc = 0$, nhưng $\frac{0}{0}\ne\frac{c}{d}$ và phân số $\frac{0}{0}$ không có nghĩa.
\end{example}
\textbf{Mẹo nhanh.} Xét dấu (sign) của tử số và mẫu số khi so sánh 2 phân số $\frac{a}{b}$ và $\frac{c}{d}$. Nếu trong 4 số $a,b,c,d$, có 1 hoặc 3 số âm, còn lại dương, thì 2 phân số không bằng nhau.

\subsubsection{Tính Chất Cơ Bản của Phân Số}
\begin{align}
	\boxed{\frac{a}{b} = \frac{ac}{bc},\ \frac{a}{b} = \frac{a:c}{b:c},\ a,b,c\in\mathbb{Z},\ b\ne 0,\ c\ne 0.}
\end{align}
trong đó đẳng thức thứ 2 yêu cầu $c\in\operatorname{UC}(a,b)$ để phân số đều có tử và mẫu nguyên.

\paragraph{Rút gọn về phân số tối giản.}
Để rút gọn phân số với tử và mẫu là số nguyên về phân số tối giản:
\begin{itemize}
	\item Tìm $\operatorname{UCLN}$ của tử và mẫu sau khi đã bỏ dấu $-$ (nếu có).
	\item Chia cả tử và mẫu cho $\operatorname{UCLN}$ vừa tìm được.
\end{itemize}

\paragraph{Quy đồng mẫu nhiều phân số.}
\begin{ques}
	Tại sao phải quy đồng mẫu nhiều phân số?
\end{ques}

\begin{proof}[Answer]
	\begin{itemize}
		\item Để tiện so sánh 2 phân số.
		\item Để tiện cho việc giải phương trình.
	\end{itemize}
\end{proof}

\begin{ques}
	Cách để quy đồng mẫu nhiều phân số?
\end{ques}
Để quy đồng mẫu nhiều phân số:
\begin{enumerate}
	\item Viết các phân số đã cho về phân số có mẫu dương. Tìm $\operatorname{BCNN}$ của các mẫu dương đó để làm mẫu chung.
	
	\textbf{Note.} Nếu các mẫu số nguyên tố cùng nhau, thì $\operatorname{BCNN}$ của chúng chính là tích của chúng.
	\item Tìm thừa số phụ của mỗi mẫu (bằng cách chia mẫu chung cho từng mẫu).
	\item Nhân tử và mẫu của mỗi phân số ở Bước 1 với thừa số phụ tương ứng.
\end{enumerate}

\subsection{So Sánh Các Phân Số. Hỗn Số Dương}



%------------------------------------------------------------------------------%

\begin{thebibliography}{99}
	\bibitem[]{}
\end{thebibliography}

\printbibliography[heading=bibintoc]

	
\end{document}