\documentclass{article}
\usepackage[backend=biber,natbib=true,style=authoryear]{biblatex}
\addbibresource{/home/hong/1_NQBH/reference/bib.bib}
\usepackage[utf8]{vietnam}
\usepackage{tocloft}
\renewcommand{\cftsecleader}{\cftdotfill{\cftdotsep}}
\usepackage[colorlinks=true,linkcolor=blue,urlcolor=red,citecolor=magenta]{hyperref}
\usepackage{amsmath,amssymb,amsthm,mathtools,float,graphicx}
\allowdisplaybreaks
\numberwithin{equation}{section}
\newtheorem{assumption}{Assumption}[section]
\newtheorem{lemma}{Lemma}[section]
\newtheorem{corollary}{Corollary}[section]
\newtheorem{definition}{Định nghĩa}[section]
\newtheorem{proposition}{Proposition}[section]
\newtheorem{theorem}{Định lý}[section]
\newtheorem{notation}{Notation}[section]
\newtheorem{remark}{Lưu ý}[section]
\newtheorem{example}{Ví dụ}[section]
\newtheorem{ques}{Question}[section]
\newtheorem{problem}{Bài toán}[section]
\newtheorem{conjecture}{Conjecture}[section]
\usepackage[left=0.5in,right=0.5in,top=1.5cm,bottom=1.5cm]{geometry}
\usepackage{fancyhdr}
\pagestyle{fancy}
\fancyhf{}
\lhead{\small \textsc{Sect.} ~\thesection}
\rhead{\small \nouppercase{\leftmark}}
\renewcommand{\sectionmark}[1]{\markboth{#1}{}}
\cfoot{\thepage}
\def\labelitemii{$\circ$}

\title{Elementary Mathematics\texttt{/}Grade 6}
\author{Nguyễn Quản Bá Hồng}
\date{\today}

\begin{document}
\maketitle
\begin{abstract}
	
\end{abstract}
\tableofcontents

%------------------------------------------------------------------------------%

\section*{Notation\texttt{/}Ký Hiệu}
\begin{itemize}
	\item $x\in[a,b]$: $x\ge a$ và $x\le b$.
\end{itemize}

\section{Một Số Yếu Tố Thống kê \& Xác Suất}

\subsection{Xác Suất Thực Nghiệm Trong 1 Số Trò Chơi \& Thí Nghiệm Đơn Giản}

\subsubsection{Xác Suất Thực Nghiệm Trong Trò Chơi Tung Đồng Xu}
\begin{definition}[Xác suất thực nghiệm]
	\emph{Xác suất thực nghiệm xuất hiện mặt $N$} khi tung đồng xu nhiều lần bằng:
	\begin{align*}
		\frac{\mbox{Số lần mặt $N$ xuất hiện}}{\mbox{Tổng số lần tung đồng xu}} = \frac{\mbox{Số lần mặt $N$ xuất hiện}}{\mbox{Số lần mặt $N$ xuất hiện} + \mbox{Số lần mặt $S$ xuất hiện}}\in[0,1].
	\end{align*}
	\emph{Xác suất thực nghiệm xuất hiện mặt $S$} khi tung đồng xu nhiều lần bằng:
	\begin{align*}
		\frac{\mbox{Số lần mặt $S$ xuất hiện}}{\mbox{Tổng số lần tung đồng xu}} = \frac{\mbox{Số lần mặt $S$ xuất hiện}}{\mbox{Số lần mặt $N$ xuất hiện} + \mbox{Số lần mặt $S$ xuất hiện}}\in[0,1].
	\end{align*}
\end{definition}
Từ định nghĩa, xác suất thực nghiệm xuất hiện mặt $N$ (hoặc mặt $S$) phản ảnh số lần xuất hiện mặt đó so với tổng số lần tiền hành thực nghiệm.

\textbf{Nhận xét.}
\begin{itemize}
	\item Xác suất thực nghiệm xuất hiện mặt $N$ bằng 0 khi và chỉ khi không có mặt $N$ nào trong tất cả lần tung đồng xu.
	\item Xác suất thực nghiệm xuất hiện mặt $N$ bằng 1 khi và chỉ khi không có mặt $S$ nào trong tất cả lần tung đồng xu.
	\item Xác suất thực nghiệm xuất hiện mặt $s$ bằng 0 khi và chỉ khi không có mặt $S$ nào trong tất cả lần tung đồng xu.
	\item Xác suất thực nghiệm xuất hiện mặt $S$ bằng 1 khi và chỉ khi không có mặt $N$ nào trong tất cả lần tung đồng xu.
\end{itemize}

\begin{problem}
	Tung 2 đồng xu cân đối \& đồng chất $T$ lần ($T$ viết tắt của ``tổng số''), trong đó:
	\begin{itemize}
		\item 2 đồng xu sấp xuất hiện $SS$ lần.
		\item 1 đồng xu sấp, 1 đồng xu ngửa xuất hiện $SN$ lần.
		\item 2 đồng xu ngửa xuất hiện $NN$ lần.
	\end{itemize}
	Hiển nhiên: $T = SS + SN + NN$. Khi đó:
	\begin{itemize}
		\item Xác suất thực nghiệm để có 1 đồng xu sấp, 1 đồng xu ngửa $= \frac{SN}{T} = \frac{SN}{SS + SN + NN}\in[0,1]$.
		\item Xác suất thực nghiệm để có 2 đồng xu đều ngửa $= \frac{NN}{T} = \frac{NN}{SS + SN + NN}\in[0,1]$.
		\item Xác suất thực nghiệm để có 2 đồng xu đều sấp $= \frac{SS}{T} = \frac{SS}{SS + SN + NN}\in[0,1]$.
		\item Xác suất thực nghiệm để có ít nhất 1 đồng xu sấp $= \frac{SS + SN}{T} = \frac{SS + SN}{SS + SN + NN}\in[0,1]$.
		\item Xác suất thực nghiệm để có ít nhất 1 đồng xu ngửa $= \frac{SN + NN}{T} = \frac{SN + NN}{SS + SN + NN}\in[0,1]$.
	\end{itemize}
\end{problem}

\section{Phân Số \& Số Thập Phân}

\subsection{Phân Số với Tử \& Mẫu là Số Nguyên}
\begin{definition}[Phân số\texttt{/}Fractionals]
	1 \emph{phân số} có tử và mẫu số là số nguyên là biểu thức có dạng $\frac{a}{b}$, $a,b\in\mathbb{Z}$, $b\ne 0$. $a$: tử số (numerator), $b$: mẫu số (denominator).
	
	Phân số $\frac{a}{b}$, $a\in\mathbb{Z}$, $b\in\mathbb{Z}^\star$, được gọi là \emph{phân số tối giản} nếu $\operatorname{gcd}(a,b) = 1$, ở đây $\operatorname{gcd}$ ký hiệu \emph{ước chung lớn nhất} (greatest common divisor).\footnote{Hoặc ký hiệu Việt Nam là: $\operatorname{UCLN}(a,b)$.}
\end{definition}

\subsubsection{Khái niệm 2 phân số bằng nhau.} 2 phân số được gọi là \textit{bằng nhau} nếu chúng cùng biểu diễn một giá trị, i.e. (tức\texttt{/}nghĩa là),
\begin{align}
	\boxed{\frac{a}{b} = \frac{c}{d}\Leftrightarrow b\ne 0, d\ne 0,\ ad = bc.}
\end{align}
Vế sau có nghĩa là \textit{nhân chéo chia ngang}, hay được gọi là \textit{quy tắc bằng nhau của 2 phân số}.

\textbf{Chú ý.} luôn nhớ điều kiện mẫu số 2 phân số phải khác 0.

\begin{example}
	Trong Sách Giáo Khoa Toán 6, Cánh Diều, của Đỗ Đức Thái chủ biên, có viết:
	\begin{quotation}
		``Xét 2 phân số $\frac{a}{b}$ và $\frac{c}{d}$. Nếu $\frac{a}{b} = \frac{c}{d}$ thì $ad = bc$\footnote{Phép nhân: $a\times b = a\cdot b = ab$.}. Ngược lại, nếu $ad = bc$ thì $\frac{a}{b} = \frac{c}{d}$.''
	\end{quotation}
	Phản ví dụ: $a = 0$, $b = 0$ thì $ad = bc = 0$, nhưng $\frac{0}{0}\ne\frac{c}{d}$ và phân số $\frac{0}{0}$ không có nghĩa.
\end{example}
\textbf{Mẹo nhanh.} Xét dấu (sign) của tử số và mẫu số khi so sánh 2 phân số $\frac{a}{b}$ và $\frac{c}{d}$. Nếu trong 4 số $a,b,c,d$, có 1 hoặc 3 số âm, còn lại dương, thì 2 phân số không bằng nhau.

\subsubsection{Tính Chất Cơ Bản của Phân Số}
\begin{align}
	\boxed{\frac{a}{b} = \frac{ac}{bc},\ \frac{a}{b} = \frac{a:c}{b:c},\ a,b,c\in\mathbb{Z},\ b\ne 0,\ c\ne 0.}
\end{align}
trong đó đẳng thức thứ 2 yêu cầu $c\in\operatorname{UC}(a,b)$ để phân số đều có tử và mẫu nguyên.

\paragraph{Rút gọn về phân số tối giản.}
Để rút gọn phân số với tử và mẫu là số nguyên về phân số tối giản:
\begin{enumerate}
	\item Tìm $\operatorname{UCLN}$ của tử và mẫu sau khi đã bỏ dấu $-$ (nếu có).
	\item Chia cả tử và mẫu cho $\operatorname{UCLN}$ vừa tìm được.
\end{enumerate}

\paragraph{Quy đồng mẫu nhiều phân số.}
\begin{ques}
	Tại sao phải quy đồng mẫu nhiều phân số?
\end{ques}

\begin{proof}[Answer]
	\begin{itemize}
		\item Để tiện so sánh 2 phân số.
		\item Để tiện cho việc giải phương trình.
	\end{itemize}
\end{proof}

\begin{ques}
	Cách để quy đồng mẫu nhiều phân số?
\end{ques}
Để quy đồng mẫu nhiều phân số:
\begin{enumerate}
	\item Viết các phân số đã cho về phân số có mẫu dương. Tìm $\operatorname{BCNN}$ của các mẫu dương đó để làm mẫu chung.
	
	\textbf{Note.} Nếu các mẫu số nguyên tố cùng nhau, thì $\operatorname{BCNN}$ của chúng chính là tích của chúng.
	\item Tìm thừa số phụ của mỗi mẫu (bằng cách chia mẫu chung cho từng mẫu).
	\item Nhân tử và mẫu của mỗi phân số ở Bước 1 với thừa số phụ tương ứng.
\end{enumerate}

\subsection{So Sánh Các Phân Số. Hỗn Số Dương}

\section{Hình Học Phẳng}

\subsection{Điểm. Đường Thẳng}
\textbf{Quy ước.} Khi nói 2 điểm mà không nói gì thêm, ta hiểu đó là 2 điểm phân biệt.

\noindent\textbf{Chú ý.} Mỗi hình là tập hợp các điểm. Hình có thể chỉ gồm 1 điểm.

\begin{remark}[Phân biệt đường thẳng vs. đoạn thẳng]
	Đường thẳng không bị giới hạn về 2 phía, trong khi đoạn thẳng bị giới hạn về 2 phía bởi 2 đầu mút của nó.
\end{remark}

\begin{definition}
	Điểm $A$ \emph{thuộc\texttt{/}nằm trên} đường thẳng $d$ (hay đường thẳng $d$ \emph{đi qua} điểm $A$) \& được ký hiệu là $A\in d$. Điểm $B$ \emph{không thuộc\texttt{/}không nằm trên} đường thẳng $d$ (hay đường thẳng $d$ \emph{không đi qua} điểm $B$) \& được ký hiệu là $B\notin d$.
\end{definition}

\begin{remark}
	Có vô số điểm thuộc 1 đoạn\texttt{/}đường thẳng.
\end{remark}
Thật vậy, đoạn thẳng $AB$ có vô số điểm bởi vì: lấy $M_1$ là trung điểm của $AB$, lấy $M_2$ là trung điểm của đoạn $AM_1$, lấy $M_3$ là trung điểm của đoạn $AM_2$, tương tự như vậy, thì có vô số lần lấy trung điểm, tương ứng vô hạn điểm.

\begin{theorem}
	Có 1 \& chỉ 1 đường thẳng đi qua 2 điểm $A$ \& $B$ (phân biệt).
\end{theorem}
Đường thẳng đi qua 2 điểm $A$, $B$ còn được gọi là \emph{đường thẳng $AB$}, hay \emph{đường thẳng $BA$}.

\begin{definition}[3 điểm thẳng hàng, không thẳng hàng]
	Khi 3 điểm cùng thuộc 1 đường thẳng, chúng được gọi là \emph{thẳng hàng}. Khi 3 điểm không cùng thuộc bất kỳ đường thẳng nào, chúng được gọi là \emph{không thẳng hàng}.
\end{definition}

\begin{theorem}
	Trong 3 điểm thẳng hàng, có 1 \& chỉ 1 điểm nằm giữa 2 điểm còn lại.
\end{theorem}

\subsection{2 Đường Thẳng Cắt Nhau. 2 Đường Thẳng Song Song}
\begin{definition}[2 đường thẳng cắt nhau]
	2 đường thẳng chỉ có 1 điểm chung gọi là \emph{2 đường thẳng cắt nhau} \& điểm chung được gọi là \emph{giao điểm} của 2 đường đó.
\end{definition}

\begin{definition}[2 đường thẳng song song]
	2 đường thẳng $a$ \& $b$ không có điểm chung nào được gọi là \emph{song song với nhau}. Viết $a//b$ hoặc $b//a$.
\end{definition}
\textbf{Chú ý.} 2 đường thẳng \emph{trùng nhau} thì không thuộc vào 2 định nghĩa trên. 





%------------------------------------------------------------------------------%

\begin{thebibliography}{99}
	\bibitem[Toán 6]{Toan6} Đỗ Đức Thái, Lê Tuấn Anh, Đỗ Tiến Đạt, Nguyễn Sơn Hà, Nguyễn Thị Phương Loan, Phạm Sỹ Nam, Phạm Đức Quang. \textit{Toán 6, Tập 1, 2}. NXB ĐHSP.
	
	\bibitem[VHB]{VHB} Vũ Hữu Bình. \textit{Nâng Cao \& Phát Triển Toán 6, Tập 1, 2}. NXB GDVN.
\end{thebibliography}

\printbibliography[heading=bibintoc]
	
\end{document}