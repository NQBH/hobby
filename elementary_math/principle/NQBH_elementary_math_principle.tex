\documentclass{article}
\usepackage[backend=biber,natbib=true,style=authoryear]{biblatex}
\addbibresource{/home/hong/1_NQBH/reference/bib.bib}
\usepackage[utf8]{vietnam}
\usepackage{tocloft}
\renewcommand{\cftsecleader}{\cftdotfill{\cftdotsep}}
\usepackage[colorlinks=true,linkcolor=blue,urlcolor=red,citecolor=magenta]{hyperref}
\usepackage{amsmath,amssymb,amsthm,mathtools,float,graphicx}
\allowdisplaybreaks
\numberwithin{equation}{section}
\newtheorem{assumption}{Assumption}[section]
\newtheorem{lemma}{Lemma}[section]
\newtheorem{corollary}{Corollary}[section]
\newtheorem{definition}{Định nghĩa}[section]
\newtheorem{proposition}{Proposition}[section]
\newtheorem{theorem}{Định lý}[section]
\newtheorem{notation}{Notation}[section]
\newtheorem{remark}{Lưu ý}[section]
\newtheorem{example}{Ví dụ}[section]
\newtheorem{question}{Question}[section]
\newtheorem{problem}{Bài toán}[section]
\newtheorem{conjecture}{Conjecture}[section]
\usepackage[left=0.5in,right=0.5in,top=1.5cm,bottom=1.5cm]{geometry}
\usepackage{fancyhdr}
\pagestyle{fancy}
\fancyhf{}
\lhead{\small \textsc{Sect.} ~\thesection}
\rhead{\small \nouppercase{\leftmark}}
\renewcommand{\sectionmark}[1]{\markboth{#1}{}}
\cfoot{\thepage}
\def\labelitemii{$\circ$}

\title{Elementary Mathematics\texttt{/}Principles}
\author{Nguyễn Quản Bá Hồng}
\date{\today}

\begin{document}
\maketitle
\begin{abstract}
	Một vài nguyên tắc cá nhân trong việc giảng dạy Toán Sơ Cấp.
\end{abstract}
\tableofcontents
\vspace{5mm}
\noindent\textbf{principle} [n] \textbf{1.} [countable, usually plural, uncountable] a moral rule or a strong belief that influences your actions; \textbf{2.} [countable] a law, a rule or a theory that something is based on; \textbf{3.} [countable] a belief that is accepted as a reason for acting or thinking in a particular way; \textbf{4.} [countable, uncountable] a general or scientific law that explains how something works or why something happens.

%------------------------------------------------------------------------------%

\section{Moral Principles\texttt{/}Nguyên Tắc Đạo Đức}
\begin{enumerate}
	\item Học sinh nên\texttt{/}phải dừng ngay người giảng, hoặc ít nhất khi người giảng nói xong câu, nếu phát hiện bất cứ sai xót trong tính toán hoặc nghiêm trọng hơn là lỗi logic (logic là yếu tố quan trọng nhất của Khoa học cơ bản nói chung và Toán học nói riêng).
	\item Chý trọng tâm lý học sinh.
	\item Học sinh đừng\texttt{/}không nên ngại hỏi câu hỏi ngu ngốc\texttt{/}ngớ ngẩn.
	\item Đặc biệt chú ý sức khỏe, cả thể chất lẫn tinh thần, đặc biệt là phòng chống những bệnh tâm lý.
\end{enumerate}

\section{Miscellaneous\texttt{/}Vài Thứ Linh Tinh Khác}
\begin{enumerate}
	\item Nên\texttt{/}cố gắng tập thể dục mỗi ngày để đầu óc minh mẫn. Không nên làm việc quá sức mà bỏ tập thể dục.
	\item Tôi không thích, đúng hơn là cực ghét, việc dịch \& viết phiên âm tiếng Việt của các nhà Khoa học nói chung \& các nhà Toán học nói riêng trong Bộ Sách Giáo Khoa. Tôi nghĩ nên viết tên đúng gốc hoặc viết phiên âm tiếng Anh để thể hiện sự tôn trọng \& nhất quán.
\end{enumerate}

%------------------------------------------------------------------------------%

\printbibliography[heading=bibintoc]
	
\end{document}