\documentclass{article}
\usepackage[backend=biber,natbib=true,style=authoryear]{biblatex}
\addbibresource{/home/nqbh/reference/bib.bib}
\usepackage[english,vietnamese]{babel}
\usepackage{tocloft}
\renewcommand{\cftsecleader}{\cftdotfill{\cftdotsep}}
\usepackage[colorlinks=true,linkcolor=blue,urlcolor=red,citecolor=magenta]{hyperref}
\usepackage{algorithm,algpseudocode,amsmath,amssymb,amsthm,float,graphicx,mathtools,multicol}
\allowdisplaybreaks
\numberwithin{equation}{section}
\newtheorem{assumption}{Assumption}[section]
\newtheorem{nhanxet}{Nhận xét}[section]
\newtheorem{conjecture}{Conjecture}[section]
\newtheorem{corollary}{Corollary}[section]
\newtheorem{hequa}{Hệ quả}[section]
\newtheorem{definition}{Definition}[section]
\newtheorem{dinhnghia}{Định nghĩa}[section]
\newtheorem{example}{Example}[section]
\newtheorem{vidu}{Ví dụ}[section]
\newtheorem{lemma}{Lemma}[section]
\newtheorem{notation}{Notation}[section]
\newtheorem{principle}{Principle}[section]
\newtheorem{problem}{Problem}[section]
\newtheorem{baitoan}{Bài toán}[section]
\newtheorem{proposition}{Proposition}[section]
\newtheorem{menhde}{Mệnh đề}[section]
\newtheorem{question}{Question}[section]
\newtheorem{cauhoi}{Câu hỏi}[section]
\newtheorem{quytac}{Quy tắc}
\newtheorem{remark}{Remark}[section]
\newtheorem{luuy}{Lưu ý}[section]
\newtheorem{theorem}{Theorem}[section]
\newtheorem{tiende}{Tiên đề}[section]
\newtheorem{dinhly}{Định lý}[section]
\usepackage[left=0.5in,right=0.5in,top=1.5cm,bottom=1.5cm]{geometry}
\usepackage{fancyhdr}
\pagestyle{fancy}
\fancyhf{}
\lhead{\small Sect.~\thesection}
\rhead{\small\nouppercase{\leftmark}}
\renewcommand{\sectionmark}[1]{\markboth{#1}{}}
\cfoot{\thepage}
\def\labelitemii{$\circ$}

\title{Diophantine Equation}
\author{\selectlanguage{vietnamese} Nguyễn Quản Bá Hồng\footnote{Independent Researcher, Ben Tre City, Vietnam\\e-mail: \texttt{nguyenquanbahong@gmail.com}; website: \url{https://nqbh.github.io}.}}
\date{\today}

\begin{document}
\maketitle
\selectlanguage{english}
\begin{abstract}
	A set of problems of Diophantine equations.
\end{abstract}

\tableofcontents
\selectlanguage{vietnamese}

%------------------------------------------------------------------------------%

\section{\href{https://en.wikipedia.org/wiki/Diophantine_equation}{Wikipedia\texttt{/}Diophantine Equation}}
\textsf{Finding all \href{https://en.wikipedia.org/wiki/Pythagorean_triple}{right triangles with integer side-lengths} is equivalent to solving the Diophantine equation $a^2 + b^2 = c^2$.}

``In mathematics, a \textit{Diophantine equation} is a \href{https://en.wikipedia.org/wiki/Polynomial_equation}{polynomial equation}, usually involving 2 or more \href{https://en.wikipedia.org/wiki/Unknown_(mathematics)}{unknowns}, s.t. the only \href{https://en.wikipedia.org/wiki/Equation_solving}{solutions} of interest are the \href{https://en.wikipedia.org/wiki/Integer}{integer} ones. A \textit{linear Diophantine equation} equates to a constant the sum of 2 or more \href{https://en.wikipedia.org/wiki/Monomials}{monomials}, each of \href{https://en.wikipedia.org/wiki/Degree_of_a_polynomial}{degree} 1. An \textit{exponential Diophantine equation} is one in which unknowns can appear in \href{https://en.wikipedia.org/wiki/Exponent}{exponents}.

\textit{Diophantine problems} have fewer equations than unknowns \& involve finding integers that solve simultaneously all equations. As such \href{https://en.wikipedia.org/wiki/Systems_of_equations}{systems of equations} define \href{https://en.wikipedia.org/wiki/Algebraic_curve}{algebraic curves}, \href{https://en.wikipedia.org/wiki/Algebraic_surface}{algebraic surfaces}, or, more generally, \href{https://en.wikipedia.org/wiki/Algebraic_set}{algebraic sets}, their study is a part of \href{https://en.wikipedia.org/wiki/Algebraic_geometry}{algebraic geometry} that is called \href{https://en.wikipedia.org/wiki/Diophantine_geometry}{\textit{Diophantine geometry}}.

The word \textit{Diophantine} refers to the \href{https://en.wikipedia.org/wiki/Greek_mathematics#Hellenistic}{Hellenistic mathematician} of the 3rd century, \href{https://en.wikipedia.org/wiki/Diophantus}{Diophantus} of \href{https://en.wikipedia.org/wiki/Alexandria}{Alexandria}, who made a study of such equations \& was 1 of the 1st mathematicians to introduce \href{https://en.wikipedia.org/wiki/Mathematical_symbol}{symbolism} into \href{https://en.wikipedia.org/wiki/Algebra}{algebra}. The mathematical study of Diophantine problems that Diophantus initiated is now called \textit{Diophantine analysis}.

While individual equations present a kind of puzzle \& have been considered throughout history, the formulation of general theories of Diophantine equations (beyond the case of linear \& \href{https://en.wikipedia.org/wiki/Quadratic_equation}{quadratic} equations) was an achievement of the 20th century.'' -- \href{https://en.wikipedia.org/wiki/Diophantine_equation}{Wikipedia\texttt{/}Diophantine equation}

\subsection{Examples of Diophantine Equation}

\subsection{Linear Diophantine Equations}

\subsection{Homogeneous Equations}

\subsection{Diophantine Analysis}

\subsection{Exponential Diophantine Equations}

%------------------------------------------------------------------------------%

\section{Phương Pháp Xét Tính Chia Hết}

\begin{baitoan}[\cite{Binh_PTNN}, Thí dụ 1, p. 6]
	Giải phương trình nghiệm nguyên $3x + 17y = 159$.
\end{baitoan}

\begin{baitoan}[\cite{Binh_PTNN}, Thí dụ 2, p. 6]
	Tìm nghiệm nguyên của phương trình $xy - x - y = 2$.
\end{baitoan}

\begin{baitoan}[\cite{Binh_PTNN}, Thí dụ 3, p. 7]
	Tìm nghiệm nguyên của phương trình $2xy - x + y = 3$.
\end{baitoan}

%------------------------------------------------------------------------------%

\printbibliography[heading=bibintoc]
	
\end{document}