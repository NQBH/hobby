\documentclass[oneside]{book}
\usepackage[backend=biber,natbib=true,style=authoryear]{biblatex}
\addbibresource{/home/hong/1_NQBH/reference/bib.bib}
\usepackage[vietnamese,english]{babel}
\usepackage{tocloft}
\renewcommand{\cftsecleader}{\cftdotfill{\cftdotsep}}
\usepackage[colorlinks=true,linkcolor=blue,urlcolor=red,citecolor=magenta]{hyperref}
\usepackage{algorithm,algpseudocode,amsmath,amssymb,amsthm,float,graphicx,mathtools,tcolorbox}
\allowdisplaybreaks
\numberwithin{equation}{section}
\newtheorem{assumption}{Assumption}[section]
\newtheorem{conjecture}{Conjecture}[section]
\newtheorem{corollary}{Corollary}[section]
\newtheorem{definition}{Definition}[section]
\newtheorem{example}{Example}[section]
\newtheorem{lemma}{Lemma}[section]
\newtheorem{notation}{Notation}[section]
\newtheorem{principle}{Principle}[section]
\newtheorem{problem}{Problem}[section]
\newtheorem{proposition}{Proposition}[section]
\newtheorem{question}{Question}[section]
\newtheorem{remark}{Remark}[section]
\newtheorem{theorem}{Theorem}[section]
\usepackage[left=0.5in,right=0.5in,top=1.5cm,bottom=1.5cm]{geometry}
\usepackage{fancyhdr}
\pagestyle{fancy}
\fancyhf{}
\lhead{\small \textsc{Chap.} ~\thechapter}
\rhead{\small \nouppercase{\leftmark}}
\renewcommand{\chaptermark}[1]{\markboth{#1}{}}
\cfoot{\thepage}
\def\labelitemii{$\circ$}

\title{Some Topics in Elementary Mathematics:\\Problems, Theories, Applications, \& Bridges to Advanced Mathematics}
\author{\selectlanguage{vietnamese} Nguyễn Quản Bá Hồng}
\date{\today}

\begin{document}
\selectlanguage{english}
\maketitle
\setcounter{secnumdepth}{3}
\setcounter{tocdepth}{3}
\tableofcontents

%------------------------------------------------------------------------------%

\selectlanguage{vietnamese}
\section*{Foreword}
Một vài chủ đề trong Toán Sơ Cấp và ứng dụng (nếu có) trong Khoa học nói chung và Toán Cao Cấp nói riêng.

\section*{Acknowledgment}
Tài liệu này như là một lời tri ân của cá nhân tôi đến thầy cô Tổ Toán Trường Trung học Phổ Thông Chuyên Bến Tre, đặc biệt thầy Quí, thầy Liêm, thầy Huynh, cô An, cô Hạnh, etc.

%------------------------------------------------------------------------------%

\paragraph*{Disclaimer.}
\begin{itemize}
	\item I do not and will not apologize when having written this text in 2 languages [English, Vietnamese] randomly.
	\item Instead of paraphrasing multiple resources to act like an expert, I will quote directly sentences and paragraphs in various references. The reasons are simple: I am lazy to paraphrase, and I love their original crafts.
\end{itemize}

\section*{General Rules for the Author}
\begin{enumerate}
	\item Always try to find and add physical interpretations and real world applications for the considered mathematical objects or terminologies.
	\item Always consider general problems first and then their particular or special cases, and then (optional) generalizations.
	\item Read terminologies in \href{https://www.wikipedia.org/}{Wikipedia} and check \href{https://math.stackexchange.com/}{Mathematics Stack Exchange} for interpretations and further information.
	\item Add mathematical histories and mathematicians for motivations.
	\item (Optional) Bridges\texttt{/}connections between elementary and advanced mathematics.
	\item (Optional) Some codes (\textsc{Matlab}, C++, Python, etc.) will be nice for further practice and illustrations.
\end{enumerate}

\section*{Conventions}
\begin{itemize}
	\item `e.g.' $\coloneqq$ `for example', or `for instance'.
	\item `i.e.' $\coloneqq$ `which means', `that means', or `in other words'.
	\item `iff' $\coloneqq$ `if and only if' $=$ `equivalent to', in mathematical notation: $\Leftrightarrow$.
	\item IMO $\coloneqq$ International Mathematical Olympiad.
	\item `lhs', or `LHS' $\coloneqq$ left-hand side
	\item `rhs', or `RHS' $\coloneqq$ right-hand side.
	\item `s.t.' $\coloneqq$ `such that'.
	\item VMO $\coloneqq$ Vietnamese Mathematical Olympiad.
	\item `w.l.o.g.' $\coloneqq$ `without loss of generality'.
\end{itemize}

%------------------------------------------------------------------------------%

%\selectlanguage{english}

\part{Algebra}

\chapter{Elementary Algebra}

\begin{quotation}
	``One cannot escape the feeling $\ldots$ that these mathematical formulae have an independent existence and an intelligence of their own $\ldots$ that they are wiser than we are, wiser even than their discoverers $\ldots$ that we get more out of them that was originally put into them.'' -- Heinrich Hertz, quoted by F.J. Dyson
\end{quotation}
``Algebra is what most people associate with mathematics. In a sense, this is justified. Mathematics is the study of abstract objects, numerical, logical, or geometrical, that follow a set of several carefully chosen axioms. And basic algebra is about the simplest meaningful thing that can satisfy the above definition of mathematics. There are only a dozen or so postulates, but that is enough to make the system beautifully symmetric.'' [$\ldots$]

``There is more than 1 algebra, though. \textit{Algebra} is the study of numbers with the operations of addition, subtraction, multiplication, and division. \textit{Matrix algebra}, for example, does much the same but with groups of numbers instead of using just one. Other algebras use all kinds of operations and all kinds of `numbers' but they, sometimes surprisingly, tend to have much of the same properties as normal algebra.'' [$\ldots$]

``Algebra is the basic foundation of a large part of applied mathematics. Problems of mechanics, economics, chemistry, electronics, optimization, and so on are answered by algebra and differential calculus, which is an advanced form of algebra. In fact, algebra is so important that most of its secrets have been discovered -- so it can be safely put into a high-school curriculum. However, a few gems can still be found here and there.'' -- \cite[Chap. 3, p. 35]{Tao2006}

``It is now time to split mathematics into branches. 1st, algebra. A section on algebraic identities hones computational skills. It is followed naturally by inequalities. In general, any inequality can be reduced to the question of finding the minimum of a function. But this is a highly nontrivial matter, and that makes the subject exciting. We discuss the fact that squares are nonnegative, the Cauchy--Schwarz inequality, the triangle inequality, the arithmetic mean-geometric mean inequality, and also Sturm's method for proving inequalities.

Our treatment of algebra continues with polynomials. We focus on quadratic polynomials, the relations between zeros and coefficients, the properties of the derivative of a polynomial, problems about the location of the zeros in the complex plane or on the real axis, and methods for proving irreducibility of polynomials (e.g., \textit{Eisenstein criterion}). From all special polynomials we present the most important, the \textit{Chebyshev polynomials}.'' [$\ldots$] -- \cite[Chap. 2, p. 25]{Gelca_Andreescu2017}

\paragraph{Note.} I really want to start with a section titled ``Identities \& Inequalities'', same as \cite[Sect. 2.1]{Gelca_Andreescu2017}, or ``Equations, Identities, \& Inequalities'' (or something similar), to emphasize their relationship, both supportive and opposite\texttt{/}contrary. However, the topic of inequalities will take several pages because it has been my favorite since I was in high school\footnote{I still like inequalities, and actually have to like it until now even I moved to advanced\texttt{/}professional mathematics instead of elementary mathematics as the old days. Of course, the topic of elementary inequality has to give its seat for more advanced forms of inequality, especially integral and functional inequalities in mathematical analysis.}. So I will separate this couple here and there is no apologize given, obviously.

\section{Algebraic Identity}
``The scope of this section is to train algebraic skills. Our idea is to hide behind each problem an important algebraic identity.'' -- \cite[Subsect. 2.1.1, p. 25]{Gelca_Andreescu2017}

\begin{problem}
	Prove that $\sum_{i=1}^n i^3 = \left(\sum_{i=1}^n i\right)^2$ for all $n\in\mathbb{N}^\star$.\footnote{I.e., the sum of the 1st few cubes will always be a square.}
\end{problem}

\begin{problem}[\cite{Gelca_Andreescu2017}, Prob. 12, p. 6] Prove that
	\begin{align*}
		\sum_{i=n+1}^{2n} \frac{1}{i} = \sum_{i=1}^{2n} (-1)^{i+1}\frac{1}{i},\ \forall n\in\mathbb{N}^\star,
	\end{align*}
	or more explicitly,
	\begin{align*}
		\frac{1}{n + 1} + \frac{1}{n + 2} + \cdots + \frac{1}{2n} = 1 - \frac{1}{2} + \frac{1}{3} - \cdots + \frac{1}{2n - 1} - \frac{1}{2n}.
	\end{align*}
\end{problem}

\begin{problem}[\cite{Gelca_Andreescu2017}, p. 26, Soviet Union college entrance exam]
	given 2 segments of lengths $a$ and $b$, construct with a straightedge and a compass a segment of length $\sqrt[4]{a^4 + b^4}$.
\end{problem}

\begin{proof}[Solution]
	See, e.g., \cite[p. 26]{Gelca_Andreescu2017}.
\end{proof}

\begin{lemma}[Sophie Germain identity]
	$a^4 + b^4 = (a^2 + \sqrt{2}ab + b^2)(a^2 - \sqrt{2}ab + b^2)$, for all $a,b\in\mathbb{R}$.
\end{lemma}

\begin{problem}
	Factorize $a^3 + b^3 + c^3 - 3abc$. \emph{($\Downarrow$)}
\end{problem}

\begin{lemma}
	For all $a,b,c\in\mathbb{R}$,
	\begin{align*}
		a^3 + b^3 + c^3 - 3abc = (a + b + c)(a^2 + b^2 + c^2 - ab - bc - ca) = \frac{1}{2}(a + b + c)\left[(a - b)^2 + (b - c)^2 + (c - a)^2\right].
	\end{align*}
\end{lemma}

\begin{proof}[Proof]
	``This identity arises from computing the \textit{circulant determinant}:
	\begin{align*}
		D = \begin{vmatrix}
			a & b & c\\c & a & b\\ b & c &a
		\end{vmatrix}
	\end{align*}
	in 2 ways: 1st by expanding with \textit{Sarrus' rule}, and 2nd by adding up all columns to the 1st, factoring $(a + b + c)$, and then expanding the remaining determinant.'' -- \cite[p. 27]{Gelca_Andreescu2017}
\end{proof}

\begin{problem}[\cite{Gelca_Andreescu2017}, p. 26, Titu Andreescu's]
	Let $x,y,z$ be distinct real numbers. Prove that $\sqrt[3]{x - y} + \sqrt[3]{y - z} + \sqrt[3]{z - x}\ne 0$.
\end{problem}

\begin{proof}[Hint]
	Set $a\coloneqq\sqrt[3]{y - z}$, $b\coloneqq\sqrt[3]{z - x}$, $c\coloneqq\sqrt[3]{x - y}$. If $a + b + c = 0$, then $a^3 + b^3 + c^3 = 3abc$: obtain absurdity right here. See \cite[p. 27]{Gelca_Andreescu2017}.
\end{proof}

\begin{problem}
	Find all $a,b,c,d\in\mathbb{Z}$ s.t. $a + b + c + d = 0$ and $a^3 + b^3 + c^3 + d^3 = 24$.
\end{problem}

\begin{proof}[Hint]
	Substitute the 1st equation into the 2nd and then factorize.
\end{proof}

\section{Equation \& System of Equations}

\begin{problem}[\cite{Gelca_Andreescu2017}, p. 26, Titu Andreescu's]
	Solve in $\mathbb{R}$ the system of equations
	\begin{equation*}
		\left\{\begin{split}
			(3x + y)(x + 3y)\sqrt{xy} &= 14,\\
			(x + y)(x^2 + 14xy + y^2) &= 36.
		\end{split}\right.
	\end{equation*}
\end{problem}

\begin{proof}[Hint]
	Set $u\coloneqq\sqrt{x}$, $v\coloneqq\sqrt{y}$. Use the \textit{binomial expansion} with exponent $= 6$ to obtain $u + v = 2$ and $u - v = \pm\sqrt{2}$. Solutions: $(x,y) = \left(\frac{3}{2}\pm\sqrt{2},\frac{3}{2}\mp\sqrt{2}\right)$. See \cite[p. 26]{Gelca_Andreescu2017}.
\end{proof}

\begin{problem}[\cite{Gelca_Andreescu2017}, Prob. 96, p. 28]
	Solve in $\mathbb{R}$ the equation $\sqrt[3]{x - 1} + \sqrt[3]{x} + \sqrt[3]{x + 1} = 0$.
\end{problem}

\begin{problem}[\cite{Gelca_Andreescu2017}, Prob. 103, p. 31]
	Find all $n\in\mathbb{N}^\star$ for which the equation $nx^4 + 4x + 3 = 0$ has a real root.
\end{problem}

\begin{problem}[\cite{Gelca_Andreescu2017}, Prob. 104, p. 31]
	Find all triples $(x,y,z)$ of real numbers that are solutions to the system of equations
	\begin{equation*}
		\left\{\begin{split}
			\frac{4x^2}{4x^2 + 1} &= y,\\
			\frac{4y^2}{4y^2 + 1} &= z,\\
			\frac{4z^2}{4z^2 + 1} &= x.
		\end{split}\right.
	\end{equation*}
\end{problem}

\begin{problem}[\cite{Gelca_Andreescu2017}, Prob. 108, p. 31]
	Find all pairs $(x,y)\in\mathbb{R}^2$ solving the system
	\begin{equation*}
		\left\{\begin{split}
			x^4 + 2x^3 - y &= -\frac{1}{4} + \sqrt{3},\\
			y^4 + 2y^3 - x &= -\frac{1}{4} - \sqrt{3}.
		\end{split}\right.
	\end{equation*}
\end{problem}

\section{Polynomials}
``Many algebra questions concern polynomials of one or more variables, $\ldots$'' -- \cite[p. 41]{Tao2006}

\begin{definition}[Polynomial]
	A \emph{polynomial of 1 variable} is a function, say $f(x)$, of the form
	\begin{align*}
		f(x) = \sum_{i=0}^n a_ix^i = a_nx^n + a_{n-1}x^{n-1} + a_{n-2}x^{n-2} + \cdots + a_1x + a_0.
	\end{align*}
	The $a_i$s are constants, and $a_n\ne 0$ is assumed. We call $n$ the \emph{degree} of $f$.
\end{definition}
Polynomials in many variables do not have as nice a form as the one-dimensional (1D) polynomials, but are quite useful nevertheless. Anyway, $f(x,y,z)$ is a \emph{polynomial in 3 variables} if it takes the form
\begin{align*}
	f(x,y,z) = \sum_{k,l,m} a_{k,l,m}x^ky^lz^m,
\end{align*}
where the $a_{k,l,m}$ are (real) constants, and the summation runs over all nonnegative $k$, $l$, and $m$ s.t. $k + l + m\le n$, and it is assumed that at least 1 of the nonzero $a_{k,l,m}$ satisfy $k + l + m = n$. We again call $n$ the \emph{degree} of $f$; polynomials of degree 2 are \textit{quadratic}, degree 3 are \emph{cubic}, and so forth. If the degree is 0, then the polynomial is said to be \emph{trivial} or \emph{constant}. If all nonzero $a_{k,l,m}$ satisfy $n = k + l + m$, then $f$ is said to be \emph{homogeneous}. Homogeneous polynomials have the property that
\begin{align*}
	f(tx_1,\ldots,tx_m) = t^mf(x_1,\ldots,x_m),\ \forall x_1,\ldots,x_m,t.
\end{align*}
A polynomial $f$ of $m$ variables is said to be \emph{factored} into 2 polynomials $p$ and $q$ if $f(x_1,\ldots,x_m) = p(x_1,\ldots,x_m)q(x_1,\ldots,x_m)$ for all $x_1,\ldots,x_m$; $p$ and $q$ are then said to be \emph{factors} of $f$. The degree of a polynomial is equal to the sum of the degrees of its factors. A polynomial is \emph{irreducible} if it cannot be factored into nontrivial factors.

The \emph{roots} of a polynomial $f(x_1,\ldots,x_m)$ are the values of $(x_1,\ldots,x_m)$ which return a zero value, so that $f(x_1,\ldots,x_m) = 0$. Polynomials of 1 variable can have as many roots as their degree; in fact, if multiplicities and complex roots are counted, polynomials of 1 variable always have exactly as many roots as their degree.

\begin{example}[Roots of quadratic polynomials]
	The roots of a quadratic polynomial $f(x) = ax^2 + bx + c$ is given by the well-known \emph{quadratic formula}:
	\begin{align*}
		x = \frac{-b\pm\sqrt{b^2 - 4ac}}{2a}.
	\end{align*}
\end{example}
``\textit{Cubic} and \textit{quartic} polynomials also have formulae for their roots, but they are much messier and not very useful in practice. Once one gets to quintic and higher polynomials, there is no elementary formula at all! And polynomials of 2 or more variables are even worse; typically one has an infinite number of roots.

The roots of a factor are a subset of the roots of the original polynomial; this can be a useful piece of information in deciding whether one polynomial divides another. In particular, $x - a$ divides $f(x)$ iff $f(a) = 0$, because $a$ is a root of $x - a$. In particular, for any polynomial $f(x)$ of 1 variable and any real number $t$, $x - t$ divides $f(x) - f(t)$.'' -- \cite[p. 42]{Tao2006}

\begin{theorem}[Fundamental theorem of algebra]
	
\end{theorem}

\begin{problem}[Australian Mathematics Competition 1987, p. 13]
	Let $a,b,c\in\mathbb{R}$ s.t.
	\begin{align}
		\label{Tao2006 Eqn. (12)}
		\frac{1}{a} + \frac{1}{b} + \frac{1}{c} = \frac{1}{a + b + c},
	\end{align}
	with all denominators nonzero. Prove that
	\begin{align*}
		\frac{1}{a^5} + \frac{1}{b^5} + \frac{1}{c^5} = \frac{1}{(a + b + c)^5}.
	\end{align*}
\end{problem}
``At 1st this question looks simple. There is really only one piece of information given, so there should be a straightforward sequence of logical steps leading to the result we want. Well, an initial attempt to deduce the 2nd equation from the 1st may be to raise both sides of \eqref{Tao2006 Eqn. (12)} to the 5th power, which gets something similar to the desired result, but with a whole lot of messy terms on the LHS. There seems to be no other obvious manipulation. So much for the direct approach.''

\begin{proof}[Hint] Combine the 3 reciprocals on the LHS of \eqref{Tao2006 Eqn. (12)} to get
	\begin{align}
		\label{Tao2006 Eqn. (14)}
		ab^2 + a^2b + a^2c + ac^2 + b^2c + bc^2 + 3abc = abc,
	\end{align}
	where the latter is algebraically simple since it contains no reciprocals. By factorization, \eqref{Tao2006 Eqn. (14)} $\Leftrightarrow(a + b)(b + c)(c + a) = 0\Leftrightarrow(a = -b)\lor(b = -c)\lor(c = -a)$, which implies, in particular,
	\begin{align*}
		\frac{1}{a^{2n+1}} + \frac{1}{b^{2n+1}} + \frac{1}{c^{2n+1}} = \frac{1}{(a + b + c)^{2n + 1}},\ \forall n\in\mathbb{N},
	\end{align*}
	where $n = 2$ yields the desired result.
\end{proof}
``Substitutions do not seem appropriate: the equations \eqref{Tao2006 Eqn. (12)} or \eqref{Tao2006 Eqn. (14)} are simple enough as they are, and substitutions would not make them much simpler. So we will try to guess and prove an intermediate result. The best kind of intermediate result is a parametrization, as this can be substituted directly into the desired equation.'' ``The best way to deal with roots of polynomials is to factorize the polynomial (and vice versa). \textit{What are the factors?}'' [$\ldots$] ``$\ldots$ and the only workable form of a polynomial is a breakup into factors. But to find out what they are, we have to experiment. The polynomial is homogeneous, so its factors should be too. The polynomial is symmetric, so the factors should be symmetries of each other. The polynomial is cubic, so there should be a linear factor.'' -- \cite[p. 44]{Tao2006}

``The factorization of polynomials, or impossibility thereof, is a fascinating piece of mathematics. The following question is instructive because it uses just about every trick in the book to find a solution.'' -- \cite[p. 45]{Tao2006}

A polynomial having degree at most $n$ has at most $n$ roots.
\begin{problem}[$\star\star$]
	Prove that any polynomial of the form $f(x) = \prod_{i=1}^n (x - a_i)^2 + 1$ where $a_1,\ldots,a_n\in\mathbb{Z}$,\footnote{NQBH: Typos in \cite[Problem. 3.4, p. 45.]{Tao2006}: there is no index $i = 0$ in both its statement and its proof (wew,  2nd edition after 15 years from the 1st one though).} cannot be factorized into 2 nontrivial polynomials, each with integer coefficients.
\end{problem}

\begin{proof}[Proof]
	Suppose that $f(x)$ is factorizable into 2 nontrivial integer polynomials, $p(x)$ and $q(x)$: $f(x) = p(x)q(x)$ for all $x$. In particular, $p(a_i)q(a_i) = f(a_i) = 1$, hence $p(a_i) = q(a_i) = \pm1$ for all $i = 1,\ldots,n$. Note that $\deg p + \deg q = \deg f = 2n$, hence 1 of them has a degree of at most $n$, w.l.o.g., assume $\deg p\le n$. Since $f$ has no real roots and $f(x)\ge 1$ for all $x\in\mathbb{R}$, $p$ has no roots and never changes sign, w.l.o.g., assume $p(x) > 0$ for all $x\in\mathbb{R}$. Then $p(a_i) = q(a_i) = 1$ for all $i = 1,\ldots,n$, i.e., $p(x) - 1$ and $q(x) - 1$ have at least $n$ roots, hence $\deg p\ge n$, $\deg q\ge n$. Combine this with the assumption $\deg p\le n$ before to obtain $\deg p = \deg q = n$. Since all the roots of $p(x) - 1$ and $q(x) - 1$ are the $a_i$s,
	\begin{align*}
		p(x) - 1 = r\prod_{i=1}^n (x - a_i),\ q(x) - 1 = s\prod_{i=1}^n (x - a_i)
	\end{align*}
	for some $r,s\in\mathbb{Z}^\star$. Apply these formulas into $f(x) = p(x)q(x)$ to get
	\begin{align}
		\label{Tao2006/Prob. 3.4/1}
		\prod_{i=1}^n (x - a_i)^2 + 1 = \left(1 + r\prod_{i=1}^n (x - a_i)\right)\left(1 + s\prod_{i=1}^n (x - a_i)\right).
	\end{align}
	Comparing the $x^n$ coefficients yields $rs = 1$, hence $r = s = \pm 1$. For $r = s = 1$, \eqref{Tao2006/Prob. 3.4/1} becomes
	\begin{align}
		\label{Tao2006/Prob. 3.4/2}
		\prod_{i=1}^n (x - a_i)^2 + 1 = \left(1 + \prod_{i=1}^n (x - a_i)\right)\left(1 + \prod_{i=1}^n (x - a_i)\right),
	\end{align}
	which is equivalent to $2\prod_{i=1}^n (x - a_i) = 0$ for all $x\in\mathbb{R}$, which is absurd\texttt{/}ridiculous. The case $r = s = -1$ is similar. 
\end{proof}

\begin{problem}
	Prove that the polynomial $f(x) = \prod_{i=1}^n (x - a_i) + 1$ cannot be factorized into 2 smaller integer polynomials, where the $a_i$s are integers.
\end{problem}

\begin{proof}[Hint]
	If $f(x) = p(x)q(x)$, look at $p(x) - q(x)$.
\end{proof}

\begin{problem}
	Let $f(x)$ be a polynomial with integer coefficients, and let $a,b\in\mathbb{Z}$. Show that $f(a) - f(b)$ can only equal 1 when $a$, $b$ are consecutive.
\end{problem}

\begin{proof}[Hint]
	$(a - b)|(f(a) - f(b)) = 1$, hence $a - b = \pm 1$.
\end{proof}

\begin{problem}[\cite{Gelca_Andreescu2017}, Prob. 88, p. 27]
	Prove that any polynomial with real coefficients that takes only nonnegative values can be written as the sum of the squares of 2 polynomials.
\end{problem}

\begin{problem}[\cite{Gelca_Andreescu2017}, Prob. 122, p. 37]
	Let $P(z)$ be a polynomial with real coefficients whose roots can be covered by a disk of radius $R$. Prove that for any $k\in\mathbb{R}$, the roots of the polynomial $nP(z) - kP'(z)$ can be covered by a disk of radius $R + |k|$, where $n\coloneqq\deg P$, and $P'(z)$ is the derivative.
\end{problem}

\subsection{Rationals vs. Irrationals}
\begin{definition}[Rational, irrational]
	
\end{definition}
\begin{problem}[\cite{Gelca_Andreescu2017}, Prob. 1, p. 3]
	Prove that $\sqrt{2} + \sqrt{3} + \sqrt{5}$ is an irrational number.
\end{problem}

\section{Inequality}
Here we consider various types of inequalities according to their arguments, i.e., inequalities for integers, inequalities for reals, and those for complex numbers. Of course, the former involves number theory, while the latter two belong to elementary analysis and calculus.

\subsection{Inequalities in $\mathbb{Z}$}

\begin{problem}[\cite{Gelca_Andreescu2017}, Prob. 15, p. 6]
	Prove that $3^n\ge n^3$ for all $n\in\mathbb{N}^\star$.
\end{problem}

\begin{problem}[\cite{Gelca_Andreescu2017}, Prob. 16, p. 6]
	Prove that
	\begin{align*}
		\left(\frac{n}{3}\right)^n < n! < \left(\frac{n}{2}\right)^n,\ \forall n\in\mathbb{N}^\star,\,n\ge 6.
	\end{align*}
\end{problem}

\begin{problem}[\cite{Gelca_Andreescu2017}, Prob. 17, p. 6]
	Prove that
	\begin{align*}
		\sum_{i=1}^n \frac{1}{i^3} < \frac{3}{2},\ \forall n\in\mathbb{N}^\star.
	\end{align*}
\end{problem}

\begin{problem}[\cite{Gelca_Andreescu2017}, p. 8]
	Let $(x_n)_{n\ge 0}$ be a sequence of nonnegative integers s.t. for every index $k$, the number of the terms of the sequence that are $\le k$ is finite. We denote this number by $y_k$. Prove that for any $m,n\in\mathbb{N}^\star$, the following inequality holds
	\begin{align*}
		\sum_{i=0}^n x_i + \sum_{i=0}^m y_i\ge(m + 1)(n + 1).
	\end{align*}
\end{problem}

\begin{proof}[Proof]
	See, e.g., \cite[pp. 8--9]{Gelca_Andreescu2017}.
\end{proof}

\begin{problem}[\cite{Gelca_Andreescu2017}, Prob. 95, p. 28]
	Prove that
	\begin{align}
		\sum_{k=1}^{31} \frac{1}{(k - 1)^{4/5} - k^{4/5} + (k - 1)^{4/5}} < \frac{3}{2} + \sum_{k=1}^{31} (k - 1)^{1/5}.
	\end{align}
\end{problem}

\subsection{Inequalities in $\mathbb{R}$}

\subsubsection{Simplest Inequality $x^2\ge 0$}
``The simplest inequality in algebra says that the square of any real number is nonnegative, and it is equal to zero iff the number is zero.'' -- \cite[Subsect. 2.1.2, p. 28]{Gelca_Andreescu2017}

\begin{problem}[\cite{Gelca_Andreescu2017}, p. 28, Titu Andreescu's]
	Find the minimum of the function $f:(0,\infty)^3\to\mathbb{R}$, $f(x,y,z) = x^z + y^z - (xy)^{z/4}$.
\end{problem}

\begin{proof}[Solution]
	Rewrite $f(x,y,z) = \left(x^{z/2} - y^{z/2}\right)^2 + 2\left[(xy)^{z/4} - \frac{1}{4}\right]^2 - \frac{1}{8}$, thus
	\begin{align*}
		\min_{(\mathbb{R}_+^\star)^3} f = -\frac{1}{8},\mbox{ and }\operatorname{argmin}_{(\mathbb{R}_+^\star)^3} f = \left\{\left(a,a,\log_a\frac{1}{16}\right);a\in(0,1)\cup(1,\infty)\right\}.
	\end{align*}
	See \cite[p. 29]{Gelca_Andreescu2017}.
\end{proof}

\begin{problem}[\cite{Gelca_Andreescu2017}, p. 29, 2001 USA team selection test, Titu Andreescu's]
	Let $(a_n)_{n\ge 0}$ be a sequence of real numbers s.t. $a_{n+1}\ge a_n^2 + \frac{1}{5}$, for all $n\ge 0$. Prove that $\sqrt{a_{n+5}}\ge a_{n-5}^2$, for all $n\ge 5$.
\end{problem}

\begin{proof}[Proof]
	See, e.g., \cite[p. 29]{Gelca_Andreescu2017}.
\end{proof}

\begin{problem}[64th W.L. Putnam Mathematics Competition]
	Let $f$ be a continuous function on the unit square. Prove that
	\begin{align*}
		\int_0^1 \left(\int_0^1 f(x,y)\,{\rm d}x\right)^2\,{\rm d}y + \int_0^1 \left(\int_0^1 f(x,y)\,{\rm d}y\right)^2\,{\rm d}x\le\left(\int_0^1\int_0^1 f(x,y)\,{\rm d}x\,{\rm d}y\right)^2 + \int_0^1\int_0^1 f(x,y)^2\,{\rm d}x\,{\rm d}y.
	\end{align*}
\end{problem}

\begin{proof}[Proof]
	Prove for a Riemann sum, and then pass to the limit. See, e.g., \cite[pp. 30--31]{Gelca_Andreescu2017}.
\end{proof}

\begin{problem}[\cite{Gelca_Andreescu2017}, Prob. 101, p. 31]
	Find $\min_{a,b\in\mathbb{R}}\max(a^2 + b,b^2 + a)$.
\end{problem}

\begin{problem}[\cite{Gelca_Andreescu2017}, Prob. 102, p. 31]
	Prove that for all $x\in\mathbb{R}$, $2^x + 3^x - 4^x + 6^x - 9^x\le 1$.
\end{problem}

\begin{problem}[\cite{Gelca_Andreescu2017}, Prob. 105, p. 31]
	Find the minimum of $\log_{x_1}\left(x_2 - \frac{1}{4}\right) + \log_{x_2}\left(x_3 - \frac{1}{4}\right) + \cdots + \log_{x_n}\left(x_1 - \frac{1}{4}\right)$, over all $x_1,\ldots,x_n\in\left(\frac{1}{4},1\right)$.
\end{problem}

\begin{problem}[\cite{Gelca_Andreescu2017}, Prob. 106, p. 31]
	Let $a,b\in\mathbb{R}$ s.t. $9a^2 + 8ab + 7b^2\le 6$. Prove that $7a + 5b + 12ab\le 9$.
\end{problem}

\begin{problem}[\cite{Gelca_Andreescu2017}, Prob. 107, p. 31]
	Let $(a_i)_{i=1}^n$ be real numbers s.t. $\sum_{i=1}^n a_i\ge n^2$ and $\sum_{i=1}^n a_i^2\le n^3 + 1$. Prove that $n - 1\le a_i\le n + 1$ for all $i = 1,\ldots,n$.
\end{problem}

\begin{problem}[\cite{Gelca_Andreescu2017}, Prob. 109, p. 32]
	Let $n\in\mathbb{N}^\star$ be even. Prove that for any $x\in\mathbb{R}$ there are at least $2^{n/2}$ choices of the sign $\pm$ s.t. $\pm x^n\pm x^{n-1}\pm\cdots\pm x < \frac{1}{2}$.
\end{problem}

\subsection{Cauchy--Schwartz Inequality}
I followed \cite[Subsect. 2.1.3, p. 32]{Gelca_Andreescu2017}.

\begin{theorem}[Cauchy--Schwarz (or Cauchy--Bunyakovski--Schwarz) inequality]
	\begin{align}
		\boxed{\sum_{i=1}^n a_i^2\sum_{i=1}^n b_i^2\ge\left(\sum_{i=1}^n a_ib_i\right)^2,}
	\end{align}
	where the equality holds iff the $a_i$'s and the $b_i$'s are proportional.
\end{theorem}
The expression
\begin{align*}
	\sum_{i=1}^n a_i^2\sum_{i=1}^n b_i^2 - \left(\sum_{i=1}^n a_ib_i\right)^2
\end{align*}
is a quadratic function in the $a_i$'s and $b_i$'s. For it to have only nonnegative values, it should be a sum of squares\footnote{NQBH: there is the so-called Sum of Squares (SOS) method to prove 3-variable inequalities.}, which is true by the \textit{Lagrange identity}:
\begin{align*}
	\sum_{i=1}^n a_i^2\sum_{i=1}^n b_i^2 - \left(\sum_{i=1}^n a_ib_i\right)^2 = \sum_{i < j} (a_ib_j - a_jb_i)^2.
\end{align*}
This proof works only in the finite-dimensional case, while the Cauchy--Schwarz inequality is true in far more generality, e.g. for square integrable functions. Its correct framework is that of a real or complex vector space, which could be finite or infinite dimensional, endowed with an \textit{inner product} $\langle\cdot,\cdot\rangle$.

\begin{definition}[Inner product, norm]
	The quantity $\|{\bf x}\|\coloneqq\sqrt{\langle{\bf x},{\bf x}\rangle}$ is called the \emph{norm} of ${\bf x}$.
\end{definition}

\begin{example}[Inner product spaces]
	\begin{itemize}
		\item Euclidean spaces $\mathbb{R}^n$, $n\in\mathbb{N}^\star$, with the usual dot product,
		\begin{align*}
			\langle(a_1,\ldots,a_n),(b_1,\ldots,b_n)\rangle\coloneqq\sum_{i=1}^n a_ib_i,\ \forall a_i,b_i\in\mathbb{R},\,i = 1,\ldots,n.
		\end{align*}
		\item Multidimensional complex spaces $\mathbb{C}^n$, $n\in\mathbb{N}^\star$, with the inner product
		\begin{align*}
			\langle(z_1,\ldots,z_n),(w_1,\ldots,w_n)\rangle\coloneqq\sum_{i=1}^n z_i\overline{w_i},\ \forall z_i,w_i\in\mathbb{C},\,i = 1,\ldots,n.
		\end{align*}
		\item The space of square integrable functions on an interval $[a,b]$ with the inner product
		\begin{align*}
			\langle f,g\rangle\coloneqq\int_a^b f(t)\overline{g(t)}\,{\rm d}t,\ \forall f,g\in L^2([a,b]).
		\end{align*}
	\end{itemize}
\end{example}

\begin{theorem}[Cauchy--Schwarz inequality]
	Let ${\bf x},{\bf y}$ be 2 vectors. Then $\|{\bf x}\|\|{\bf y}\|\ge|\langle{\bf x},{\bf y}\rangle|$, with equality iff the vectors ${\bf x}$ and ${\bf y}$ are parallel and point in the same direction.
\end{theorem}

\begin{proof}[Proof]
	See any standard text in linear algebra, or, e.g., \cite[p. 33]{Gelca_Andreescu2017}.
\end{proof}

\begin{example}
	If $f,g\in C([a,b],\mathbb{C})$, or more generally $f,g\in L^2([a,b],\mathbb{C})$, then
	\begin{align*}
		\int_a^b |f(t)|^2\,{\rm d}t\int_a^b |g(t)|^2\,{\rm d}t\ge\left|\int_a^b f(t)\overline{g(t)}\,{\rm d}t\right|^2.
	\end{align*}
\end{example}

\begin{problem}[\cite{Gelca_Andreescu2017}, p. 33]
	Find the maximum of the function $f(x,y,z) = 5x - 6y + 7z$ on the ellipsoid $2x^2 + 3y^2 + 4z^2 = 1$.
\end{problem}

\begin{proof}[Answer]
	Answer: $\max_{\{2x^2 + 3y^2 + 4z^2 = 1\}} f(x,y,z) = \frac{\sqrt{147}}{2}$ attained at the points
	\begin{align*}
		\left\{(x,y,z)\in\mathbb{R}^3;2x^2 + 3y^2 + 4z^2 = 1,\,x,z > 0,\, y < 0,\ x:y:z = \frac{5}{\sqrt{2}}:-\frac{6}{\sqrt{3}}:\frac{7}{2}\right\}.
	\end{align*}
	See, e.g., \cite[p. 33]{Gelca_Andreescu2017}.
\end{proof}

\begin{problem}[IMO 1993 Shortlist, Titu Andreescu's]
	Prove that
	\begin{align*}
		\frac{a}{b + 2c + 3d} + \frac{b}{c + 2d + 3a} + \frac{c}{d + 2a + 3b} + \frac{d}{a + 2b + 3c}\ge\frac{2}{3},\ \forall a,b,c,d > 0.
	\end{align*}
\end{problem}

\begin{proof}[Proof]
	See, e.g., \cite[p. 34]{Gelca_Andreescu2017}.
\end{proof}

\begin{problem}[\cite{Gelca_Andreescu2017}, Prob. 110, p. 34]
	If $a,b,c > 0$, prove that $9a^2b^2c^2\le(a^b + b^2c + c^a)(ab^2 + bc^2 + ca^2)$.
\end{problem}

\begin{problem}[\cite{Gelca_Andreescu2017}, Prob. 111, p. 34]
	If $\sum_{i=1}^n a_i = n$, prove that $\sum_{i=1}^n a_i^4\ge n$.
\end{problem}

\begin{problem}[\cite{Gelca_Andreescu2017}, Prob. 112, p. 34]
	Let $(a_i)_{i=1}^n\subset\mathbb{R}$ be distinct. Find the maximum of $\sum_{i=1}^n a_ia_{\sigma(i)}$ over all permutations of the set $\{1,\ldots,n\}$.
\end{problem}

\begin{problem}[\cite{Gelca_Andreescu2017}, Prob. 113, p. 34]
	Let $(a_i)_{i=1}^n\subset(0,\infty)$. Prove that for any $(x_i)_{i=1}^n\subset\mathbb{R}$, the quantity $\sum_{i=1}^n a_ix_i^2 - \frac{\left(\sum_{i=1}^n a_ix_i\right)^2}{\sum_{i=1}^n a_i}$ is nonnegative.
\end{problem}

\begin{problem}[\cite{Gelca_Andreescu2017}, Prob. 114, p. 34]
	Find all $n,k_1,\ldots,k_n\in\mathbb{N}^\star$ s.t. $\sum_{i=1}^n = 5n - 4$ and $\sum_{i=1}^n \frac{1}{k_i} = 1$.
\end{problem}

\begin{problem}[\cite{Gelca_Andreescu2017}, Prob. 115, p. 35]
	Prove that the finite sequence $(a_i)_{i=0}^n\subset(0,\infty)$ is a geometric progression iff $\left(\sum_{i=0}^{n-1} a_ia_{i+1}\right)^2 = \sum_{i=0}^{n-1} a_i^2\sum_{i=1}^n a_i^2$.
\end{problem}

\begin{problem}[\cite{Gelca_Andreescu2017}, Prob. 116, p. 35]
	Let $P(x)$ be a polynomial with positive real coefficients. Prove that $\sqrt{P(a)P(b)}\ge P(\sqrt{ab})$, for all $a,b\in(0,\infty)$.
\end{problem}

\begin{problem}[\cite{Gelca_Andreescu2017}, Prob. 117, p. 35]
	Consider the real numbers $x_0 > x_1 > \cdots > x_n$. Prove that $x_0 + \frac{1}{x_0 - x_1} = \sum_{i=1}^{n-1} \frac{1}{x_i - x_{i+1}}\ge x_n + 2n$. When does equality hold?
\end{problem}

\begin{problem}[\cite{Gelca_Andreescu2017}, Prob. 119, p. 35]
	Prove that
	\begin{align*}
		\frac{1}{a + b} + \frac{1}{b + c} + \frac{1}{c + a} + \frac{1}{2\sqrt[3]{abc}}\ge\frac{\left(a + b + c + \sqrt[3]{abc}\right)^2}{(a + b)(b + c)(c + a)},\ \forall a,b,c > 0.
	\end{align*}
\end{problem}

\subsection{Triangle Inequality}
``In its most general form, the \textit{triangle inequality} states that in a metric space $X$ the \textit{distance function} $\delta$ satisfies $\delta(x,y)\le\delta(x,z) + \delta(y,z)$, for any $x,y,z\in X$. An equivalent form is $|\delta(x,y) - \delta(y,z)|\le\delta(x,z)$.

Here are some familiar examples of distance functions: the distance between 2 real or complex numbers as the absolute value of their difference, the distance between 2 vectors in $n$-dimensional Euclidean space as the length of their difference $\|{\bf v} - {\bf w}\|$, the distance between 2 matrices as the norm of their difference, the distance between 2 continuous functions on the same interval as the supremum of the absolute value of their difference. In all these cases the triangle inequality holds.'' -- \cite[Subsect. 2.1.4, p. 35]{Gelca_Andreescu2017}

\begin{problem}[T.B. Soulami's book \textit{Les olympiades de math\'ematiques: Réflexes et strat\'egies} (Ellipses, 1999), \cite{Gelca_Andreescu2017}, p. 36]
	For $a,b,c > 0$, prove the inequality $\sqrt{a^2 - ab + b^2} + \sqrt{b^2 - bc + c^2}\ge\sqrt{a^2 + ac + c^2}$.
\end{problem}

\begin{proof}[Proof]
	See, e.g., \cite[p. 36]{Gelca_Andreescu2017}.
\end{proof}

\begin{problem}[\cite{Gelca_Andreescu2017}, p. 36]
	Let $P(x)$ be a polynomial whose coefficients lie in the interval $[1,2]$, and let $Q(x)$ and $R(x)$ be 2 nonconstant polynomials s.t. $P(x) = Q(x)R(x)$, with $Q(x)$ having the dominant coefficient equal to 1. Prove that $|Q(3)| > 1$.
\end{problem}

\begin{proof}[Proof]
	See, e.g., \cite[pp. 36--37]{Gelca_Andreescu2017}.
\end{proof}

\begin{problem}[\cite{Gelca_Andreescu2017}, Prob. 123, p. 37]
	Prove that $a,b,c > 0$ are the side lengths of a triangle iff $a^b + b^2 + c^2 < 2\sqrt{a^2b^2 + b^2c^2 + c^2a^2}$.
\end{problem}

\begin{problem}[\cite{Gelca_Andreescu2017}, Prob. 125, p. 37]
	Let $(V_i)_{i=1}^m$ and $(W_i)_{i=1}^m$ be isometries of $\mathbb{R}^n$ ($m,n\in\mathbb{N}^\star$). Assume that for all ${\bf x}\in\mathbb{R}^n$ with $\|{\bf x}\|\le 1$, $\|V_i{\bf x} - W_i{\bf x}\|\le 1$, $i = 1,\ldots,m$. Prove that
	\begin{align*}
		\left\|\left(\prod_{i=1}^m V_i\right){\bf x} - \left(\prod_{i=1}^m W_i\right){\bf x}\right\|\le m,\ \forall{\bf x}\in\mathbb{R}^n,\,\|{\bf x}\|\le 1.
	\end{align*}
\end{problem}

\subsection{Arithmetic Mean--Geometric Mean Inequality}
``\textit{Jensen's inequality}, which will be discussed in the section about convex functions, states that if $f$ is a real-valued concave function, then'' -- \cite[Subsect. 2.1.5, p. 38]{Gelca_Andreescu2017}
\begin{align*}
	f\left(\sum_{i=1}^n \lambda_ix_i\right)\ge\sum_{i=1}^n \lambda_if(x_i),\ \forall(x_i)_{i=1}^n\subset\operatorname{Dom}(f),\,\forall(\lambda_i)_{i=1}^n\subset(0,\infty),\mbox{ s.t. }\sum_{i=1}^n \lambda_i = 1.
\end{align*}
i.e., this inequality holds for any $x_1,\ldots,x_n$ in the \textit{domain} of $f$, denoted by $\operatorname{Dom}(f)$, and for any positive \textit{weights} $\lambda_1,\ldots,\lambda_n$ with $\sum_{i=1}^n \lambda_i = 1$. Moreover, if the function $f$ is nowhere linear (i.e., if it is strictly concave) and the numbers $\lambda_i$'s are nonzero, then equality holds iff $x_1 = \cdots = x_n$.

Apply this to the concave function $f(x) = \log x$, the positive numbers $(x_i)_{i=1}^n$, and the weights $\lambda_i = \frac{1}{n}$, $i = 1,\ldots,n$, to obtain
\begin{align*}
	\log\frac{1}{n}\sum_{i=1}^n x_i\ge\frac{1}{n}\sum_{i=1}^n \log x_i.
\end{align*}
Exponentiation yields the following inequality.

\begin{theorem}[Arithmetic mean--geometric mean inequality (AM--GM)]
	Let $(x_i)_{i=1}^n\in[0,\infty)$. Then
	\begin{align*}
		\frac{1}{n}\sum_{i=1}^n\ge\sqrt[n]{\prod_{i=1}^n x_i},\mbox{ or explicitly, }\frac{x_1 + \cdots + x_n}{n}\ge\sqrt[n]{x_1\cdots x_n},
	\end{align*}
	with equality iff all numbers are equal.
\end{theorem}

\begin{proof}[Proof]
	See, e.g., \cite[pp. 38--39]{Gelca_Andreescu2017}.
\end{proof}

\subsection{Selected Problems in Inequalities}
``Now we demonstrate a less frequently encountered form of induction that can be traced back to Cauchy's work, where it was used to prove the \textit{arithmetic mean-geometric mean inequality}.'' -- \cite[p. 9]{Gelca_Andreescu2017}

\begin{problem}[\cite{Gelca_Andreescu2017}, p. 9, D. Buşneag, I. Maftei, \textit{Themes for Mathematics Circles \& Contests} (Scrisul Românesc, Craiova, 1983)]
	Let $(a_i)_{i=1}^n\subset\mathbb{R}$ s.t. $a_i > 1$ for $i = 1,\ldots,n$. Prove that
	\begin{align*}
		\sum_{i=1}^n \frac{1}{1 + a_i}\ge\frac{n}{1 + \sqrt[n]{\prod_{i=1}^n a_i}}.
	\end{align*}
\end{problem}

\begin{proof}[Proof]
	See, e.g., \cite[pp. 9--10]{Gelca_Andreescu2017}.
\end{proof}

\begin{problem}[\cite{Gelca_Andreescu2017}, Prob. 35. p. 8]
	Show that if $(a_i)_{i=1}^n\subset[0,\infty)$, then
	\begin{align*}
		\prod_{i=1}^n (1 + a_i)\ge\left(1 + \sqrt[n]{\prod_{i=1}^n a_i}\right)^n.
	\end{align*}
\end{problem}

\subsection{Inequalities in $\mathbb{C}$}

%------------------------------------------------------------------------------%

\part{Analysis}

\chapter{Elementary Calculus \& Elementary Analysis}
``Analysis is also a heavily explored subject, and it is just as general as algebra: essentially, analysis is the study of functions and their properties. The more complicated the properties, the `higher' the analysis. The lowest form of analysis is studying functions satisfying simple algebraic properties$\ldots$'' -- \cite[Chap. 3, p. 36]{Tao2006}

\begin{problem}[\cite{Gelca_Andreescu2017}, Prob. 22, p. 7]
	Prove that any function defined on the entire real axis can be written as the sum of 2 functions whose graphs admit centers of symmetry.
\end{problem}

%------------------------------------------------------------------------------%

\chapter{Functional Equation}

\section{Basic Terminologies}

\section{Selected Problems in Functional Equation}
\begin{problem}
	Let $f:\mathbb{R}\to\mathbb{R}$ s.t.\footnote{``These problems are a good way to learn how to think mathematically, because there is only 1 or 2 pieces of data that can be used, so there should be a clear direction in which to go. It is sort of a `pocket mathematics'’, where instead of the 3 dozen axioms and countless thousands of theorems, one only has a handful of `axioms' (i.e. data) to use. And yet, it still has its surprises.'' -- \cite[Chap. 3, p. 36]{Tao2006}} $f$ is continuous, $f(0) = 1$, and
	\begin{align*}
		f(m + n + 1) = f(m) + f(n),\ \forall m,n\in\mathbb{R}.
	\end{align*}
	Show that $f(x) = 1 + x$ for all real numbers $x$.
\end{problem}

\begin{proof}[Hint]
	1st prove for all integers, then for all rationals, and then for all reals.
\end{proof}

\begin{problem}[Greitzer 1978, p. 19]
	Suppose $f:\mathbb{N}^\star\to\mathbb{N}^\star$ s.t. $f$ satisfies $f(n + 1) > f(f(n))$ for all positive integers $n$. Show that $f(n) = n$ for all positive integers $n$.
\end{problem}
``This equation looks insufficient to prove what we want. After all, h\textit{ow can an inequality prove an equality?}'' Functional equations are easier to handle because one can apply various substitutions and the like and gradually manipulate our original data into a manageable form.

\begin{proof}[Hint]
	1st make the inequality `stronger': $f(n + 1)\ge f(f(n)) + 1$, for all $n\in\mathbb{N}^\star$. Prove $f(m)\ge n$ for all $m\ge n$ by induction, and thus, in particular, when $m = n$, $f(n)\ge n$ for all $n\in\mathbb{N}^\star$. Hence, $f(n + 1)\ge f(f(n)) + 1\ge f(n) + 1 > f(n)$, i.e., $f$ is an increasing function: $f(m) > f(n)\Leftrightarrow m > n$. Then $f(n + 1) > f(f(n))$ implies $n + 1 > f(n)$. See \cite[pp. 36--38]{Tao2006} for a full proof.
\end{proof}
``Always try to use tactics that get you closer to the objective, unless all available direct approaches have been exhausted. Only then you should think about going sideways, or -- occasionally -- backwards.'' -- \cite[p. 37]{Tao2006}

\begin{problem}[Australian Mathematics Competition 1984, p. 7]
	\label{prob: Australian Mathematics Competition 1984, p. 7}
	Suppose $f:\mathbb{N}^\star\to\mathbb{Z}$ with the following properties: (a) $f(2) = 2$; (b) $f(mn) = f(m)f(n)$ for all $m,n\in\mathbb{N}^\star$; (c) $f(m) > f(n)$ if $m > n$. Find $f(1983)$.
\end{problem}
``Now we have to find out a particular value of $f$. The best way is to try to evaluate all of $f$, not just $f(1983)$. (1983 is just the year of the question anyway.) This is, of course, assuming there is only 1 solution of $f$. But implicit in the question is the fact that there is only 1 possible value of $f(1983)$ (otherwise there would be more than 1 answer), and because of the ordinariness of 1983 we might reasonably conjecture that there is only 1 solution to $f$.'' -- \cite[p. 39]{Tao2006}

\begin{proof}[Hint]
	By induction, $f(2^n) = 2^n$ for all $n\in\mathbb{N}_0$ (including $f(1) = 1$ by substituting $m = 1$, $n = 2$ in (b)). Prove $f(n) = n$ for all $n\in\mathbb{N}^\star$ by strong induction (use the even-odd argument in the induction step). 
\end{proof}
``Because we seem to be relying on past results to attain the new ones, the general proof smells heavily on induction. And because we are not just using one previous result, but several previous results, we probably need \textit{strong} induction.'' -- \cite[p. 40]{Tao2006}

\begin{problem}
	Show that Problem \ref{prob: Australian Mathematics Competition 1984, p. 7} can still be solved if we replace (a) with the weaker condition (a') $f(n) = n$ for at least 1 integer $n\ge 2$.
\end{problem}

\begin{problem}
	Show that Problem \ref{prob: Australian Mathematics Competition 1984, p. 7} can still be solved if we allow $f(n)$ to be a real number, rather than just an integer, i.e., $f:\mathbb{N}^\star\to\mathbb{R}$. For an additional challenge, solve Problem \ref{prob: Australian Mathematics Competition 1984, p. 7} with this assumption and with (a) replaced by (a').
\end{problem}

\begin{proof}[Hint]
	1st prove $f(3) = 3$, by comparing $f(2^n)$ with $f(3^n)$ for various integers $m,n\in\mathbb{N}^\star$.
\end{proof}

\begin{problem}[1986 International Mathematical Olympiad, Q5]
	Find all (if any) functions $f:[0,\infty)\to[0,\infty)$, s.t. (a) $f(xf(y))f(y) = f(x + y)$ for all $x,y\in[0,\infty)$; (b) $f(2) = 0$; (c) $f(x)\ne 0$ for every $0\le x < 2$.
\end{problem}

\begin{proof}[Hint]
	(a) involves products of values of $f$, and (b) and (c) involve a function having a value of zero or nonzero. What can one say when a product equals 0?
\end{proof}

\begin{problem}[\cite{Gelca_Andreescu2017}, Prob. 7, p. 3]
	Show that there does not exist a function $f:\mathbb{Z}\to\{1,2,3\}$ satisfying $f(x)\ne f(y)$ for all $x,y\in\mathbb{Z}$ s.t. $|x - y|\in\{2,3,5\}$.
\end{problem}

\begin{problem}[\cite{Gelca_Andreescu2017}, Prob. 8, p. 3]
	Show that there does not exist a strictly increasing function $f:\mathbb{N}\to\mathbb{N}$ satisfying $f(2) = 3$ and $f(mn) = f(m)f(n)$ for all $m,n\in\mathbb{N}$.
\end{problem}

\begin{problem}[\cite{Gelca_Andreescu2017}, Prob. 9, p. 3]
	Determine all functions $f:\mathbb{N}\to\mathbb{N}$ satisfying
	\begin{align*}
		xf(y) + yf(x) = (x + y)f(x^2 + y^2),\ \forall x,y\in\mathbb{N}^\star.
	\end{align*}
\end{problem}

\begin{problem}[\cite{Gelca_Andreescu2017}, p. 7]
	Let $f:\mathbb{N}\to\mathbb{N}$ be a strictly increasing function s.t. $f(2) = 2$ and $f(mn) = f(m)f(n)$ for every relatively prime pair of $m,n\in\mathbb{N}^\star$. Prove that $f(n) = n$ for every $n\in\mathbb{N}^\star$.
\end{problem}

\begin{proof}[Proof]
	See, e.g., \cite[p. 8]{Gelca_Andreescu2017}.
\end{proof}

\begin{definition}[Multiplicative function]
	A function $f:\mathbb{N}\to\mathbb{C}$ with the property that $f(1) = 1$ and $f(mn) = f(m)f(n)$ whenever $m$ and $n$ are coprime is called a \emph{multiplicative function}.
\end{definition}

\begin{example}
	The \emph{Euler totient function} and the \emph{M\"obius function} are multiplicative functions.
\end{example}

\begin{theorem}[Paul Erd\"os']
	Any increasing multiplicative function that is not constant is of the form $f(n) = n^\alpha$ for some $\alpha > 0$.
\end{theorem}

\begin{problem}[\cite{Gelca_Andreescu2017}, Prob. 34, pp. 10--11]
	Let $f:\mathbb{R}\to\mathbb{R}$ be a function satisfying
	\begin{align*}
		f\left(\frac{x_1 + x_2}{2}\right) = \frac{f(x_1) + f(x_2)}{2},\ \forall x_1,x_2\in\mathbb{R}.
	\end{align*}
	Prove that
	\begin{align*}
		f\left(\frac{1}{n}\sum_{i=1}^n x_i\right) = \frac{1}{n}\sum_{i=1}^n f(x_i),\ \forall x_i\in\mathbb{R},\,i = 1,\ldots,n.
	\end{align*}
\end{problem}

\begin{problem}[\cite{Andreescu_Mortici_Tetiva2017}, Prob. 2, p. 2]
	Determine all monotone functions $f:\mathbb{N}^\star\to\mathbb{R}$ s.t. $f(xy) = f(x) + f(y)$ for all $x,y\in\mathbb{N}^\star$.
\end{problem}

\begin{proof}[Solution]
	See, e.g., \cite[pp. 2--3]{Andreescu_Mortici_Tetiva2017}.
\end{proof}

%------------------------------------------------------------------------------%

\part{Calculus}

%------------------------------------------------------------------------------%

\part{Combinatorics}

\chapter{Combinatorics}

\begin{problem}[\cite{Gelca_Andreescu2017}, Prob. 19, p. 6]
	Prove that for any $n\in\mathbb{N}^\star$, a $2^n\times 2^n$ checkerboard with a $1\times 1$ corner square removed can be tiled by pieces of the form described in \cite[Fig. 2, p. 6]{Gelca_Andreescu2017} (3-cell $L$-shape)
\end{problem}

\section{Combinatorial Geometry}

\begin{problem}[\cite{Gelca_Andreescu2017}, Prob. 4, p. 3]
	Let $\mathcal{F} = \{E_1,\ldots,E_m\}$ be a family of subsets with $n - 2$ elements of a set $S$ with $n$ elements, $n\ge 3$. Show that if the union of any 3 subsets from $\mathcal{F}$ is not equal to $S$, then the union of all subsets from $\mathcal{F}$ is different from $S$.
\end{problem}

\begin{problem}[\cite{Gelca_Andreescu2017}, Prob. 10, p. 3]
	Show that the interval $[0,1]$ cannot be partitioned into 2 disjoint sets $A$ and $B$ s.t. $B = A + a$ for some real number $a$.
\end{problem}

\begin{problem}[\cite{Gelca_Andreescu2017}, p. 4]
	Finite many lines divide the plane into regions. Show that these regions can be colored by 2 colors in such a way that neighboring regions have different colors.
\end{problem}

\begin{proof}[Solution]
	See, e.g., \cite[p. 4]{Gelca_Andreescu2017}.
\end{proof}

\begin{problem}[\cite{Gelca_Andreescu2017}, Prob. 25, p. 7]
	It is given a finite set $A$ of lines in a plane. It is known that, for some $k\in\mathbb{N}^\star$, $k\ge 3$, for every subset $B$ of $A$ consisting of $k^2 + 1$ lines there are $k$ points in the plane s.t. each line in $B$ passes through at least 1 of them. Prove that there are $k$ points in the plane s.t. every line in $A$ passes through at least 1 of them.
\end{problem}

\section{Combinatorial Set Theory}

\begin{problem}[\cite{Gelca_Andreescu2017}, p. 12, 67th W.L. Putnam Mathematical Competition, 2006]
	Prove that for every set $X = \{x_1,\ldots,x_n\}$ of $n$ real numbers, there exists a nonempty subset $S$ of $X$ and an integer $m$ s.t.
	\begin{align*}
		\left|m + \sum_{s\in S} s\right|\le\frac{1}{n + 1}.
	\end{align*}
\end{problem}

\begin{proof}[Proof]
	See, e.g., \cite[p. 12]{Gelca_Andreescu2017}.
\end{proof}

%------------------------------------------------------------------------------%

\part{Geometry}

\chapter{Elementary\texttt{/}Euclidean Geometry}

\begin{quotation}
	``Archimedes will be remembered when Aeschylus is forgotten, because languages die and mathematical ideas do not.'' -- G. H. Hardy, `\textit{A Mathematical Apology}'
\end{quotation}
``Euclidean geometry was the 1st branch of mathematics to be treated in anything like the modern fashion (with postulates, definitions, theorems, and so forth); and even now geometry is conducted in a very logical, tightly knit fashion. There are several basic results which can be used to systematically attack and resolve questions about geometrical objects and ideas. This idea can be taken to extremes with coordinate geometry, which transforms points, lines, triangles, and circles into a quadratic mess of coordinates, crudely converting geometry into algebra. But the true beauty of geometry is in how a non-obvious looking fact can be shown to be undeniably true by the repeated application of obvious facts.'' -- \cite[Chap. 4, p. 49]{Tao2006}

\section{2D Geometry}
\begin{theorem}[Thales' theorem]
	The angle subtended by a diameter is a right angle.
\end{theorem}
``Geometry is full of things like this: results you can check by drawing a picture and measuring angles and lengths, but are not immediately obvious, like the theorem that the midpoints of the 4 sides of a quadrilateral always make up a parallelogram. These facts -- they have a certain something about them.'' -- \cite[Chap. 4, p. 50]{Tao2006}

\begin{problem}[Australian Mathematics Competition 1987, p. 12]
	$ABC$ is a triangle inscribed in a circle. The angle bisectors of $A,B,C$ meet the circle at $D,E,F$, respectively. Show that $AD\bot EF$.
\end{problem}

\subsection{Triangle}

\paragraph{Basic.}
\begin{enumerate}
	\item \textit{Sum of angles in a triangle.} $\alpha + \beta + \gamma = 180^\circ$.
	\item \textit{Sine rule.}
	\item \textit{Cosine rule.}
	\item \textit{Area formula.}
	\item \textit{Heron's formula.}
	\item \textit{Triangle inequality.}
\end{enumerate}

\begin{problem}
	Show that the perpendicular bisectors of a triangle are concurrent.
\end{problem}

\begin{proof}[Proof]
	See \cite[p. ix]{Tao2006}.
\end{proof}

\begin{problem}[\cite{Tao2006}, Prob. 1.1, p. 1]
	A triangle has its lengths in an arithmetic progression, with difference $d$. The area of the triangle is $t$. Find the lengths and angles of the triangle.
\end{problem}
\textit{Comments.} An `evaluate $\ldots$' problem. ``The equalities are likely to be more useful than the inequalities, since our objective and data come in the form of equalities.''

\begin{problem}[\cite{Gelca_Andreescu2017}, Prob. 29, p. 9]
	Show that an isosceles triangle with one angle of $120^\circ$ can be dissected into $n\ge 4$ triangles similar to it.
\end{problem}

\begin{problem}[\cite{Gelca_Andreescu2017}, Prob. 120, p. 37]
	Let $a,b,c$ be the side lengths of a triangle with the property that for any $n\in\mathbb{N}^\star$, the numbers $a^n,b^n,c^n$ can also be the side lengths of a triangle. Prove that the triangle is necessarily isosceles.
\end{problem}

\begin{problem}[\cite{Gelca_Andreescu2017}, Prob. 121, p. 37]
	Given the vectors $\vec{a},\vec{b},\vec{c}$ in the plane, show that $\|\vec{a}\| + \|\vec{b}\| + \|\vec{c}\| + \|\vec{a} + \vec{b} + \vec{c}\|\ge\|\vec{a} + \vec{b}\| + \|\vec{b} + \vec{c}\| + \|\vec{c} + \vec{a}\|$.
\end{problem}

\begin{problem}[\cite{Gelca_Andreescu2017}, Prob. 126, p. 38]
	Given an equilateral triangle $ABC$ and a point $P$ that does not lie on the circumcircle of $ABC$, show  that one can construct a triangle with sides the segments $PA$, $PB$, and $PC$. If $P$ lies on the circumcircle, show that 1 of these segments is equal to the sum of the other 2.
\end{problem}

\begin{problem}[\cite{Gelca_Andreescu2017}, Prob. 127, p. 38]
	Let $M$ be a point in the plane of the triangle $ABC$ whose centroid is $G$. Prove that
	\begin{align*}
		MA^3\cdot BC + MB^3\cdot AC + MC^3\cdot AB\ge 3MG\cdot AB\cdot BC\cdot CA.
	\end{align*}
\end{problem}

\subsection{Quadrilateral}

\begin{problem}[\cite{Gelca_Andreescu2017}, Prob. 124, p. 37]
	Let $ABCD$ be a convex cyclic quadrilateral. Prove that $|AB - CD| + |AD - BC|\ge 2|AC - BD|$.
\end{problem}

\subsection{Polygon}

\begin{definition}[$n$-gon]
	An \emph{$n$-gon} is a polygon with $n$ sides.
\end{definition}

\begin{problem}[\cite{Gelca_Andreescu2017}, Prob. 30, p. 9]
	Show that for all $n > 3$ there exists an $n$-gon whose sides are not all equal and s.t. the sum of the distances from any interior point to each of the sides is constant.
\end{problem}

\begin{problem}[\cite{Gelca_Andreescu2017}, Prob. 31, p. 9]
	The vertices of a convex polygon are colored by at least 3 colors s.t. no 2 consecutive vertices have the same color. Prove that one can dissect the polygon into triangles by diagonals that do not cross and whose endpoints have different colors.
\end{problem}

\begin{problem}[\cite{Gelca_Andreescu2017}, Prob. 32, p. 9]
	Prove that any polygon (convex or not) can be dissected into triangles by interior diagonals.
\end{problem}

\begin{problem}[\cite{Andreescu_Mortici_Tetiva2017}, Prob. 1, p. 1]
	The midpoints of the bases of a trapezoid, the point at which its lateral sides meet, \& the point of intersection of its diagonals are 4 collinear points.
\end{problem}

\begin{proof}[Proof]
	Hint: Use Ceva's \& Thales' interception theorems. See, e.g., \cite[pp. 1--2]{Andreescu_Mortici_Tetiva2017}.
\end{proof}

\begin{theorem}[Ceva's theorem]
	***
\end{theorem}

\begin{theorem}[Thales's interception theorem]
	***
\end{theorem}

\begin{problem}[\cite{Andreescu_Mortici_Tetiva2017}, p. 2]
	If the midpoints of 2 opposite sides of a quadrilateral \& the intersection point of its diagonals are 3 collinear points, then the quadrilateral is actually a trapezoid.
\end{problem}
``As we said, this is a well-known theorem in elementary Euclidean geometry, so why bother to mention it here? Well, this is because we find in it a very good example of a problem that needs a \textit{(mathematical) bridge}. Namely, you noticed that the problem statement is very easy to understand even for a person who only has a very humble background in geometry -- but that person wouldn't be able to \textit{solve} the problem. You could be familiar with basic notions as collinearity \& parallelism, you could know such things as properties of angles determined by 2 parallel lines \& a transversal, but any attempt to solve the problem with such tools will fail. One needs much more in order to achieve such a goal, namely, one needs a \textit{new theory} -- we are talking about the theory of similarity. In other words, if you want to solve this problem, you have to raise your knowledge to new facts that are not mentioned in its statement. You need to throw a \textit{bridge} from the narrow realm where you are stuck to a larger extent.''


\section{3D Geometry}

\begin{problem}[\cite{Gelca_Andreescu2017}, Prob. 5, p. 3]
	Every point of the 3D space is colored red, green, or blue. Prove that 1 of the colors attains all distances, meaning that any positive real number represents the distance between 2 points of this color.
\end{problem}

\begin{problem}	[\cite{Gelca_Andreescu2017}, Prob. 6, p. 3]
	The union of 9 planar surfaces, each of area equal to 1, has a total area equal to 5. Prove that the overlap of some 2 of these surfaces has an area $\ge\frac{1}{9}$.
\end{problem}

%------------------------------------------------------------------------------%

\part{Number Theory}

%------------------------------------------------------------------------------%

\part{Probability}

%------------------------------------------------------------------------------%

\part{Statistics}

%------------------------------------------------------------------------------%

\part{Trigonometry}

\chapter{Trigonometry}

\section{Trigonometrical Inequality}

\begin{problem}[\cite{Gelca_Andreescu2017}, Prob. 13, p. 6]
	Prove thta $|\sin nx|\le n|\sin x|$, for all $x\in\mathbb{R}$, $n\in\mathbb{N}^\star$.
\end{problem}

\begin{problem}[\cite{Gelca_Andreescu2017}, Prob. 14, p. 6]
	Prove that for any $x_1,\ldots,x_n\in\mathbb{R}$, $n\ge 1$,
	\begin{align*}
		\sum_{i=1}^n |\sin x_i| + \left|\cos\left(\sum_{i=1}^n x_i\right)\right|\ge 1.
	\end{align*}
\end{problem}

\begin{problem}[\cite{Gelca_Andreescu2017}, Prob. 118, p. 35]
	Prove that
	\begin{align*}
		\frac{\sin^3 a}{\sin b} + \frac{\cos^3 a}{\cos b}\ge\sec(a - b),\ \forall a,b\in\left(0,\frac{\pi}{2}\right).
	\end{align*}
\end{problem}

%------------------------------------------------------------------------------%



%------------------------------------------------------------------------------%



%------------------------------------------------------------------------------%



%------------------------------------------------------------------------------%



%------------------------------------------------------------------------------%



%------------------------------------------------------------------------------%

\chapter{Linear Algebra}

\begin{problem}[\cite{Gelca_Andreescu2017}, Prob. 87, p. 27]
	Let $A$ and $B$ be 2 $n\times n$ matrices that commute and s.t. for some positive integers $p$ and $q$, $A^p = \mathcal{I}_n$ and $B^q = \mathcal{O}_n$. Prove that $A + B$ is invertible, and find its inverse.
\end{problem}

%------------------------------------------------------------------------------%

\chapter{Number Theory}

\paragraph{Notations.} The \textit{set of nonnegative integers}\texttt{/}\textit{natural numbers with zero}\texttt{/}\textit{naturals with zero} is denoted by
\begin{align*}
	\mathbb{N}_0 = \mathbb{N}^0 = \mathbb{N}^\star\cup\{0\}\coloneqq\{0,1,2,\ldots\} = \{x\in\mathbb{Z};x\ge 0\} = \mathbb{Z}_0^+ = \mathbb{Z}_{\ge 0}.
\end{align*}
The \textit{set of positive integers}\texttt{/}\textit{natural numbers without zero}\texttt{/}\textit{naturals without zero} is denoted by
\begin{align*}
	\mathbb{N}^\star = \mathbb{N}^+ = \mathbb{N}_0\backslash\{0\} = \mathbb{N}_1 = \{1,2,\ldots\} = \{x\in\mathbb{Z};x > 0\} = \mathbb{Z}^+ = \mathbb{Z}_{> 0} = \mathbb{Z}_{\ge 1}.
\end{align*}
See, e.g., \href{https://en.wikipedia.org/wiki/Natural_number}{Wikipedia\texttt{/}natural number}. The existence of such a set is established in \textit{set theory}, see, e.g., \cite{Halmos1960, Halmos1974, Kaplansky1972, Kaplansky1977}. Note that these notations are usually different in each text. E.g., in \cite[Chap. 2, p. 10]{Tao2006}: ``A \textit{natural number} is a positive integer (we will not consider 0 a natural number). The set of natural numbers will be denoted as ${\bf N}$.''

\begin{quotation}
	``Number theory may not necessarily be divine, but it still has an aura of mystique about it. Unlike algebra, which has as its backbone the laws of manipulating equations, number theory seems to derive its results from a source unknown.'' -- \cite[Chap. 2, p. 9]{Tao2006}
	
	``Basic number theory is a pleasant backwater of mathematics. But the applications that stem from the basic concepts of integers and divisibility are amazingly diverse and powerful. The concept of divisibility leads naturally to that of \textit{primes}, which moves into the detailed nature of factorization and then to one of the jewels of mathematics in the last part of the previous century: the prime number theorem, which can predict the number of primes less than a given number to a good degree of accuracy. Meanwhile, the concept of integer operations lends itself to modular arithmetic, which can be generalized from a subset of the integers to the algebra of finite groups, rings, and fields, and leads to algebraic number theory, when the concept of `number' is expanded into irrational surds, elements of splitting 	fields, and complex numbers. Number theory is a fundamental cornerstone which supports a sizeable chunk of mathematics. And, of course, it is fun too.'' -- \cite[Chap. 2, p. 10]{Tao2006}
\end{quotation}
The following theorem is 1st conjectured by Fermat.

\begin{theorem}[Lagrange's theorem]
	Every positive integer is a sum of 4 perfect squares.
\end{theorem}
\textit{Comment.} ``Algebraically, we are talking about an extremely simple equation: but because we are restricted to the integers, the rules of algebra fail. The result is infuriatingly innocent-looking and experimentation shows that it does seem to work, but offers no explanation why. Indeed, Lagrange's theorem cannot be easily proved by the elementary means covered in this book: an excursion into \textit{Gaussian integers} or something similar is needed.'' -- \cite[Chap. 2, p. 9]{Tao2006}

\section{Modular Arithmetic}
\begin{definition}[Prime number]
	A \emph{prime number} is a natural number with exactly 2 factors: itself and 1; we do not consider 1 to be prime. 2 natural numbers $m$ and $n$ are \emph{coprime} if their only common factor is 1.
\end{definition}

\begin{problem}[\cite{Gelca_Andreescu2017}, Prob. 2, p. 3]
	Show that no set of 9 consecutive integers can be partitioned into 2 sets with the product of the elements of the 1st set equal to the product of the elements of the 2nd set.
\end{problem}

\begin{problem}[\cite{Gelca_Andreescu2017}, Prob. 3, p. 3]
	Find the least positive integer $n$ s.t. any set of $n$ pairwise relatively prime integers $> 1$ and $< 2005$ contains at least 1 prime number.
\end{problem}

\begin{problem}[\cite{Gelca_Andreescu2017}, Prob. 11, p. 3]
	Let $n > 1$ be an arbitrary real number and let $k$ be the number of positive prime numbers $\le n$. Select $k + 1$ positive integers s.t. none of them divides the product of all the others. Prove that there exists a number among the chosen $k + 1$ that is bigger than $n$.
\end{problem}

\begin{theorem}[Fermat's little theorem]
	Let $p$ be a prime number, and $n$ a positive integer. Then $n^p - n$ is divisible by $p$.
\end{theorem}

\begin{proof}[Proof]
	See, e.g., \cite[p. 4]{Gelca_Andreescu2017}.
\end{proof}

\begin{definition}[Modular arithmetic]
	The notation `$x = y\ (\operatorname{mod} n)$', which we read as `$x$ equals to $y$ module $n$', means that $x$ and $y$ differ by a multiple of $n$. The notation `$(\operatorname{mod} n)$' signifies that we are working in a \emph{modular arithmetic} where the \emph{modulus $n$} has been identified with 0.
\end{definition}
Modular arithmetic also differs from standard arithmetic in that inequalities do not exist, and that all numbers are integers.

\begin{example}
	$7/2\ne 3.5\ (\operatorname{mod} 5)$, $7/2 = 12/2 = 6\ (\operatorname{mod} 5)$ because $7 = 12\ (\operatorname{mod} 5)$.\footnote{``It may seem strange to divide in this round-about way, but in fact one can find that there is no real contradiction, although some divisions are illegal, just as division-by-zero is illegal within the traditional field of real numbers. As a general rule, division is OK if the denominator is coprime with the modulus $n$.'' -- \cite[p. 10]{Tao2006}}
\end{example}
The following statements can be proved by elementary number theory; all revolve around the basic idea of \textit{modular arithmetic}, which provides the power of algebra but limited to a finite number of integers.
\begin{problem}
	A natural number $n$ is always has the same last digit as its 5th power $n^5$.
\end{problem}

\begin{problem}
	$n$ is a multiple of 9 iff the sum of its digits is a multiple of 9.
\end{problem}

\begin{theorem}[Wilson's theorem]
	For $n\in\mathbb{N}^\star$, $(n - 1)! + 1$ is a multiple of $n$ iff $n$ is a prime number.
\end{theorem}

\begin{problem}
	If $k$ is a positive odd number, then $\sum_{i=1}^n i^k = 1^k + 2^k + \cdots + n^k$ is divisible by $n + 1$.
\end{problem}

\begin{problem}
	Prove that there are exactly 4 numbers that are $n$ digits long (allowing for padding by zeroes) and which are exactly the same last digits as their square. e.g., the 4 3-digit numbers with this property are $000$, $001$, $625$, and $876$.
\end{problem}
This problem can eventually lead to the notion of \textit{p-adics}, being sort of an infinite-dimensional form of modular arithmetic.

\begin{problem}[\cite{Gelca_Andreescu2017}, Prob. 18, p. 6]
	Prove that for any $n\in\mathbb{N}^\star$, there exists an $n$-digit number
	\begin{itemize}
		\item[(a)] divisible by $2^n$ and containing only the digits 2 and 3;
		\item[(b)] divisible by $5^n$ and containing only the digits 5, 6, 7, 8, 9.
	\end{itemize}
\end{problem}

\subsection{Digits}
``One can learn something about a number (in particular, whether it is divisible by\footnote{also 3.} 9) by summing all its digits. In higher mathematics, it turns out that this operation is not particularly important (it has proven far more effective to study numbers directly, rather than
via their digit expansion), but it is quite popular in recreational mathematics and has even been given mystical connotations by some! Certainly, digit summing appears fairly often in mathematics competition problems$\ldots$''

\begin{problem}
	Show that among any 18 consecutive 3-digit numbers there is at least one which is divisible by the sum of its digits.
\end{problem}
\texttt{Skip Chap. 2 in \cite{Tao2006}, come back later to Number Theory.}

\section{Selected Problems in Number Theory}
A typical technique of proof in number theory: prove by the \textit{principle of mathematical induction} (chứng minh bằng \textit{phương pháp}\texttt{/}\textit{nguyên lý quy nạp toán học}).

\begin{problem}[5th W.L. Putnam Mathematical Competition, \cite{Gelca_Andreescu2017}, p. 5]
	For $m\in\mathbb{N}^\star$, $n\in\mathbb{N}^\star$, $n\ge 2$, define $f_1(n)\coloneqq n$, $f_2(n)\coloneqq n^{f_1(n)}$, $\ldots$, $f_{i+1}(n)\coloneqq n^{f_i(n)}$, $\ldots$. Prove that
	\begin{align*}
		f_m(n) < n\underbrace{!!\cdots!}_{m\ {\rm times}} < f_{m+1}(n).
	\end{align*}
\end{problem}

\begin{proof}[Proof]
	Its solution combines several proofs by induction, see, e.g., \cite[pp. 5--6]{Gelca_Andreescu2017}.
\end{proof}

\begin{problem}[\cite{Gelca_Andreescu2017}, Prob. 20, p. 7]
	Given a sequence of integers $x_1,\ldots,x_n$ whose sum is 1, prove that exactly 1 of the cyclic shifts $x_1,x_2,\ldots,x_n$; $x_2,\ldots,x_n,x_1$; $\ldots$; $x_n,x_1,\ldots,x_{n-1}$ has all of its partial sums positive. (By a \emph{partial sum} we mean the sum of the 1st $k$ terms, $k\le n$.)
\end{problem}

\begin{problem}[\cite{Gelca_Andreescu2017}, Prob. 21, p. 7]
	Let $x_1,\ldots,x_n,y_1,\ldots,y_m$ be positive integers, $m,n > 1$. Assume that $\sum_{i=1}^n x_i = \sum_{i=1}^m y_i < mn$. Prove that in the equality $\sum_{i=1}^n x_i = \sum_{i=1}^m y_i$ one can suppress some (but not all) terms in such a way that the equality is still satisfied.
\end{problem}

\begin{problem}[\cite{Gelca_Andreescu2017}, Prob. 23, p. 7]
	Prove that for any $n\in\mathbb{N}^\star$, $n\ge 2$ there is $m\in\mathbb{N}^\star$ that can be written simultaneously as a sum of $2,3,\ldots,n$ squares of nonzero integers.
\end{problem}

\begin{problem}[\cite{Gelca_Andreescu2017}, Prob. 24, p. 7]
	Let $n\in\mathbb{N}^\star$, $n\ge 2$, and let $a_1,\ldots,a_{2n+1}$ be positive real numbers s.t. $a_1 < \cdots < a_{2n+1}$. Prove that
	\begin{align*}
		\sum_{i=1}^{2n+1} (-1)^{i+1}\sqrt[n]{a_i} < \sqrt[n]{\sum_{i=1}^{2n+1} (-1)^{i+1}a_i}.
	\end{align*}
\end{problem}

\begin{definition}[Fibonacci sequence]
	The \emph{Fibonacci sequence} $(F_n)_{n\ge 0}$ is defined by $F_0\coloneqq 0$, $F_1\coloneqq 1$, and $F_{n+1}\coloneqq F_n + F_{n-1}$, for all $n\ge 1$.
\end{definition}

\begin{problem}[\cite{Gelca_Andreescu2017}, Prob. 26, p. 9]
	Show that every positive integer can be written as a sum of distinct terms of the Fibonacci sequence.
\end{problem}

\begin{theorem}[Some basic properties of Fibonacci sequence]
	The following properties hold:
	\begin{itemize}
		\item $F_{2n+1} = F_{n+1}^2 + F_n^2$, for all $n\in\mathbb{N}$.\footnote{\cite[Prob. 27, p. 9]{Gelca_Andreescu2017}.}
		\item $F_{3n} = F_{n+1}^3 + F_n^3 - F_{n-1}^3$, for all $n\in\mathbb{N}$.\footnote{\cite[Prob. 28, p. 9]{Gelca_Andreescu2017}.}
	\end{itemize}
\end{theorem}

\begin{problem}[\cite{Gelca_Andreescu2017}, Prob. 33, p. 9]
	Prove any any positive integer can be represented as $\pm1^2\pm2^2\pm\cdots\pm n^2$ for some $n\in\mathbb{N}^\star$ and some choice of the signs.
\end{problem}

\begin{problem}[\cite{Gelca_Andreescu2017}, p. 11, IMO 1972, proposed by Russia]
	Prove that every set of 10 2-digit integer numbers has 2 disjoint subsets with the same sum of elements.
\end{problem}

\begin{proof}[Proof]
	See, e.g., \cite[p. 11]{Gelca_Andreescu2017}.
\end{proof}

\begin{problem}[\cite{Gelca_Andreescu2017}, p. 11, IMO 1985, proposed by Mongolia]
	Given a set $M$ of 1985 distinct positive integers, none of which has a prime divisor greater than 26, prove that $M$ contains at least 1 subset of 4 distinct elements whose product is the 4th power of an integer.
\end{problem}

\begin{proof}[Proof]
	See, e.g., \cite[pp. 11--12]{Gelca_Andreescu2017}.
\end{proof}

\begin{problem}[\cite{Gelca_Andreescu2017}, Prob. 86, p. 27]
	Show that for no positive integer $n$ can both $n + 3$ and $n^2 + 3n + 3$ be perfect cubes.
\end{problem}

\begin{problem}[\cite{Gelca_Andreescu2017}, Prob. 89, p. 27]
	Prove that for any $n\in\mathbb{N}$, the number $5^{5^{n+1}} + 5^{5^n} + 1$ is not prime.
\end{problem}

\begin{problem}[\cite{Gelca_Andreescu2017}, Prob. 90, p. 27]
	Show that for an odd integer $n\ge 5$,
	\begin{align}
		\binom{n}{0}5^{n-1} - \binom{n}{1}5^{n-2} + \binom{n}{2}5^{n-3} - \cdots + \binom{n}{n - 1}
	\end{align}
	is not a prime number.
\end{problem}

\begin{problem}[\cite{Gelca_Andreescu2017}, Prob. 91, p. 27]
	Factor $5^{1985} - 1$ into a product of 3 integers, each of which is $> 5^{100}$.
\end{problem}

\begin{problem}[\cite{Gelca_Andreescu2017}, Prob. 92, p. 28]
	Prove that the number $\frac{5^{125} - 1}{5^{25} - 1}$ is not prime.
\end{problem}

\begin{problem}[\cite{Gelca_Andreescu2017}, Prob. 93, p. 28]
	Let $a$ and $b$ be coprime integers greater than 1. Prove that for $n\ge 0$, $a^{2n} + b^{2n}$ is divisible by $a + b$.
\end{problem}

\begin{problem}[\cite{Gelca_Andreescu2017}, Prob. 94, p. 28]
	Prove that any integer can be written as the sum of five perfect cubes.
\end{problem}

\begin{problem}[\cite{Gelca_Andreescu2017}, Prob. 97, p. 28]
	Find all triples $(x,y,z)\in(\mathbb{N}^\star)^3$ s.t. $x^3 + y^3 + z^3 - 3xyz = p$, where $p$ is a prime number $> 3$.
\end{problem}

\begin{problem}[\cite{Gelca_Andreescu2017}, Prob. 98, p. 28]
	Let $a,b,c$ be distinct positive integers s.t. $ab + bc + ca\ge 3k^2 - 1$, where $k\in\mathbb{N}^\star$. Prove that $a^3 + b^3 + c^3\ge 3(abc + 3k)$.
\end{problem}

\begin{problem}[\cite{Gelca_Andreescu2017}, Prob. 99, p. 28]
	Show that the expression $(x^2 - yz)^3 + (y^2 - zx)^3 + (x^2 - yz)^3 - 3(x^2 - yz)(y^2 - zx)(z^2 - xy)$ is a perfect square.
\end{problem}

\begin{problem}[\cite{Gelca_Andreescu2017}, Prob. 100, p. 28]
	Find all triples $(m,n,p)$ of positive integers s.t. $m + n + p = 2002$ and the system of equations $\frac{x}{y} + \frac{y}{x} = m$, $\frac{y}{z} + \frac{z}{y} = n$, $\frac{z}{x} + \frac{x}{z} = p$ has at least 1 solution in nonzero real numbers $\mathbb{R}^\star$.
\end{problem}	

%------------------------------------------------------------------------------%

\chapter{Probability}

%------------------------------------------------------------------------------%

\chapter{Statistics}

%------------------------------------------------------------------------------%



%------------------------------------------------------------------------------%

\chapter{Miscellaneous}

\section{Discrete Mathematics}

\section{Mathematical Bridges}
``Many breakthroughs\footnote{\textbf{breakthrough} [n] an important development or discovery that helps people to achieve or understand something.} in research \&, more generally, solutions to problems come as the result of someone making \textit{connections}. These connections are sometimes quite subtle\footnote{\textbf{subtle} [a] \textbf{1.} (\textit{often approving}) (especially of a change or difference) not very obvious; not easy to notice; \textbf{2.} (of a person or their behavior) behaving in a clever way \& using indirect methods in order to achieve something; \textbf{3.} showing a good understanding of things that are not obvious to other people.}, \& at 1st blush\footnote{\textbf{blush} [v] \textbf{1.} [intransitive] to become red in the face because you are embarrassed or ashamed, \textsc{synonym}: \textbf{go red}; \textbf{2.} [transitive] \textbf{blush to do something} to be ashamed or embarrassed about something; [n] \textbf{1.} [countable] the red color that spreads over your face when you are embarrassed or ashamed; \textbf{2.} (\textit{North American English}) [uncountable, countable] a colored cream or powder that some people put on their cheeks ($=$ on their faces below the eyes) to give them more color.}, they may not appear to be plausible\footnote{\textbf{plausible} [a] (of an excuse or explanation) reasonable \& likely to be true; \textsc{opposite}: \textbf{implausible}.} candidates\footnote{\textbf{candidate} [n] \textbf{1.} a person who is trying to be elected or is applying for a job; \textbf{2.} \textbf{candidate (for something)} a person or thing that is considered suitable for something or that is likely to be something or get something; in biochemistry, \textbf{candidate gene} is a gene that is suspected of being involved in the expression of a trait e.g. a disease; \textbf{3.} (\textit{British English}) a person taking an exam.} for part of the solution to a difficult problem. In this book, we  think of these connections as \textit{bridges}. A bridge enables the possibility of a solution to a problem that may have a very elementary statement but whose solution may involve more complicated realms\footnote{\textbf{realm} [n] \textbf{1.} an area of activity, interest or knowledge; \textbf{2.} (\textit{formal}) a country ruled by a king or queen, \textsc{synonym}: \textbf{kingdom}.} that may not be directly indicated by the problem statement. Bridges extend \& build on existing ideas \& provide new knowledge \& strategies for the solver.'' [$\ldots$]

\cite[Chap. 1]{Andreescu_Mortici_Tetiva2017} ``explores the metaphor\footnote{\textbf{metaphor} [n] [countable, uncountable] \textbf{1.} \textbf{metaphor (for something)} something that represents another situations or idea; \textbf{2.} a word or phrase used to describe somebody\texttt{/}something else, in a way that is different from its normal use, in order to show that the 2 things have the same qualities \& to make the description more powerful; the use of such words \& phrases.} of bridges by presenting a myriad\footnote{\textbf{myriad} [a] (\textit{literary}) extremely large in number.} of problems that span a diverse\footnote{\textbf{diverse} [a] very different from each other; containing people or things of various kinds.} set of mathematical fields. In subsequent chapters, it is left to the \textit{reader} to decide what constitutes\footnote{\textbf{constitute} [v] \textbf{1.} \textit{linking verb} (not used in the progressive tenses) \textbf{$+$ noun} to be considered to be something; \textbf{2.} \textit{linking verb} (not used in the progressive tenses) \textbf{$+$ noun} to be the parts that together form something, \textsc{synonym}: \textbf{make something up}; \textbf{3.} [transitive, usually passive] \textbf{(be) $\ldots$ constituted} to form a group legally or officially, \textsc{synonym}: \textbf{establish, set something up}.} a bridge. Indeed, different people may well have different opinions of whether something is a (useful) bridge or not.'' [$\ldots$] ``Bridges can be found everywhere -- \& not just in mathematics.'' -- \cite[Preface, p. v]{Andreescu_Mortici_Tetiva2017}

\section{Strategies in Problem Solving}
\begin{quotation}
	``Like and unlike the proverb above, the solution to a problem begins (and continues, and ends) with simple, logical steps. But as long as one steps in a firm, clear direction, with long strides and sharp vision, one would need far, far less than the millions of steps needed to journey a thousand miles. And mathematics, being abstract, has no physical constraints; one can always restart from scratch, try new avenues of attack, or backtrack at an instant's notice. One does not always have these luxuries in other forms of problem-solving (e.g. trying to go home if you are lost).
	
	Of course, this does not necessarily make it easy; if it was easy, then this book would be substantially shorter. But it makes it possible.
	
	There are several general strategies and perspectives to solve a problem correctly; \cite{Polya2014} is a classic reference for many of these.'' -- \cite[Chap. 1, p. 1]{Tao2006}
\end{quotation}
Here the strategies in \cite[Chap. 1, pp. 1--7]{Tao2006} are recalled briefly, with or without quotation marks:
\begin{enumerate}
	\item \textbf{Understand the problem.} \textit{What kind of problem is it?} There are 3 main types of problems:
	\begin{enumerate}
		\item \textit{`Show that \ldots' or `Evaluate $\ldots$' questions}\texttt{/}\textit{problems}, in which a certain statement has to be proved true, or a certain expression has to be worked out. These problems start with given data and the objective is to deduce some statement or find the value of an expression. This type of problem is generally easier than the other 2 types because there is a clearly visible objective, one that can be deliberately approached.
		\item \textit{`Find a $\ldots$' or `Find all $\ldots$' questions}\texttt{/}\textit{problems}, which requires one to find something (or everything) that satisfies certain requirements. These problems are more hit-and-miss; generally one has to guess 1 answer that nearly works, and then tweak it a bit to make it more correct; or alternatively one can alter the requirements that the object-to-find must satisfy, so that they are easier to satisfy.
		
		A typical strategy for ``find a\texttt{/}all' problems: List all, or as many as possible, available options\texttt{/}possibilities and then use pure eliminations.
		\item \textit{`Is there a \ldots' questions}\texttt{/}\textit{problems}, which either require you to prove a statement or provide a counterexample (and thus is 1 of the previous 2 types of problems). These problems are typically the hardest, because one must 1st make a decision on whether an object exists or not, and provide a proof on one hand, or a counterexample on the other.
	\end{enumerate}
	\textit{Why is categorizing a problem, or recognizing the type of a problem, important?} Because: ``The type of problem is important because it determines the basic method of approach.''
	\begin{align*}
		\boxed{\mbox{Type of problem}\Rightarrow\mbox{Basic method of approach}.}
	\end{align*}
	``Of course, not all questions fall into these neat categories; but the general format of any question will still indicate the basic strategy to pursue when solving a problem.''
	\item \textbf{Understand the data.} ``\textit{What is given in the problem?} Usually, a question talks about a number of objects satisfying some special requirements. To understand the data, one needs to see how the objects and requirements react to each other. This is important in focusing attention on the proper techniques and notation to handle the problem.''
	\item \textbf{Understand the objective.} ``\textit{What do we want?} One may need to find an object, prove a statement, determine the existence of an object with special properties, or whatever. Like the flip side of this strategy, `understand the data', knowing the objective helps focus attention on the best weapons to use. Knowing the objective also helps in creating tactical goals which we know will bring us closer to solving the question.''
	\item \textbf{Select good notation.} ``Now that we have our data and objective, we must represent it in an efficient way, so that the data and objective are both represented as simply as possible. This usually involves the thoughts of the past 2 strategies.''
	\item \textbf{Write down what you know in the notation selected; draw a diagram.} ``Putting everything down on paper helps in 3 ways:
	\begin{enumerate}
		\item you have an easy reference later on;
		\item the paper is a good thing to stare at when you are stuck;
		\item the physical act of writing down of what you know can trigger new inspirations and connections.
	\end{enumerate}
	Be careful, though, of writing superfluous material, and do not overload your paper with minutiae; 1 compromise is to highlight those facts which you think will be most useful, and put more questionable, redundant, or crazy ideas in another part of your scratch paper.'' ``Many of these facts may prove to be useless or distracting. But we can use some judgments to separate the valuable facts from the unhelpful ones.''
	\item \textbf{Modify the problem slightly.} ``There are many ways to vary a problem into one which may be easier to deal with:
	\begin{enumerate}
		\item Consider a special case of the problem, e.g., extreme or degenerate cases.
		\item Solve a simplified version of the problem.
		\item Formulate a conjecture which would imply the problem, and try to prove that first.
		\item Derive some consequence of the problem, and try to prove that first.
		\item Reformulate the problem (e.g., take the contrapositive, prove by contradiction, or try some substitution).
		\item Examine solutions of similar problems.
		\item Generalize the problem.
	\end{enumerate}
	This is useful when you cannot even get started on a problem, because solving for a simpler related problem sometimes reveals the way to go on the main problem. Similarly, considering extreme cases and solving the problem with additional assumptions can also shed light on the general solution. But be warned that special cases are, by their nature, special, and some elegant technique could conceivably apply to them and yet have absolutely no utility in solving the general case. This tends to happen when the special case is \textit{too} special. Start with modest assumptions 1st, because then you are sticking as closely as possible to the spirit of the problem.''
	\item \textbf{Modify the problem significantly.} ``In this more aggressive type of strategy, we perform major modifications to a problem such as removing data, swapping the data with the objective, or negating the objective (e.g., trying to disprove a statement rather than prove it). Basically, we try to push the problem until it breaks, and then try to identify where the breakdown occurred; this identifies what the key components of the data are, as well as where the main difficulty will lie. These exercises can also help in getting an instinctive feel of what strategies are likely to work, and which ones are likely to fail.'' ``We could omit some objectives $\ldots$'' ``We can also omit some data $\ldots$''. ``(Sometimes one can \textit{partially} omit data $\ldots$ but this is getting complicated. Stick with the simple options 1st.)'' ``Reversal of the problem (swapping data with objective) leads to some interesting ideas though.'' ``Do not forget, though, that a question can be solved in more than 1 way, and no particular way can really be judged the absolute best.''
	\item \textbf{Prove results about our question.} ``Data is there to be used, so one should pick up the data and play with it. Can it produce more meaningful data? Also, proving small results could be beneficial later on, when trying to prove the main result or to find the answer. However small the result, do not forget it -- it could have bearing later on. Besides, it gives you something to do if you are stuck.''
	\item \textbf{Simplify, exploit data, \& reach tactical goals.} ``Now we have set up notation and have a few equations, we should seriously look at attaining our tactical goals that we have established. In simple problems, there are usually standard ways of doing this. (E.g., algebraic simplification is usually discussed thoroughly in high-school level textbooks.) Generally, this part is the longest and most difficult part of the problem: however, once can avoid getting lost if one remembers the relevant theorems, the data and how they can be used, and most importantly the objective. It is also a good idea to not apply any given technique or method blindly, but to think ahead and see where one could hope such a technique to take one; this can allow one to save enormous amounts of time by eliminating unprofitable directions of inquiry before sinking lots of effort into them, and conversely to give the most promising directions priority.''
\end{enumerate}
``Don't get discouraged; put effort and imagination into each problem; and only if all else fails, look at the solution from the back of the book. But even if you are successful, you should read the solution, since many times it gives a new insight and, more important, opens the door toward more advanced mathematics.'' [$\ldots$] ``Every once in a while, for a problem that you have solved, write down the solution in detail, then compare it to the one given at the end of the book. It is very important that your solutions be correct, structured, convincing, and easy to follow.'' -- \cite[A Study Guide, p. xvii]{Gelca_Andreescu2017}

``Listen and you will forget, learn and you will remember, do it yourself and you will understand.'' -- \cite[p. 6]{Gelca_Andreescu2017}

\section{Mathematical Olympiads}

\subsection{PUTNAM}
``A problem book at the college level. A study guide for the Putnam competition. A bridge between high school problem solving and mathematical research. A friendly introduction to fundamental concepts and results. All these desires gave life to the pages that follow.

The William Lowell Putnam Mathematical Competition is the most prestigious mathematics competition at the undergraduate level in the world. Historically, this annual event began in 1938, following a suggestion of William Lowell Putnam, who realized the merits of an intellectual intercollegiate competition. Nowadays, over 2500 students from more than 300 colleges and universities in the USA and Canada take part in it. The name Putnam has become synonymous with excellence in undergraduate mathematics.

Using the Putnam competition as a symbol, we lay the foundations of higher mathematics from a unitary, problem-based perspective. As such, \textit{Putnam \& Beyond} is a journey through the world of college mathematics, providing a link between the stimulating problems of the high school years and the demanding problems of scientific investigation. It gives motivated students a chance to learn concepts and acquire strategies, hone their skills and test their knowledge, seek connections, and discover real world applications. Its ultimate goal is to build the appropriate background for graduate studies, whether in mathematics or applied sciences.

Our point of view is that in mathematics it is \fbox{more important to understand \textit{why} than to know \textit{how}}. Because of this we insist on proofs and reasoning. After all, mathematics means, as the Romanian mathematician Grigore Moisil once said, ``correct reasoning''. The ways of mathematical thinking are universal in today's science.

\textit{Putnam \& Beyond} targets primarily Putnam training sessions, problem-solving seminars, and math clubs at the college level, filling a gap in the undergraduate curriculum. But it does more than that. Written in the structured manner of a textbook, but with strong emphasis on problems and individual work, it covers what we think are the most important topics and techniques in undergraduate mathematics, brought together within the confines of a single book in order to strengthen one's belief in the \fbox{unitary nature of mathematics}.'' [$\ldots$] ``When organizing the material, we were inspired by Georgia O'Keeffe's words: ``Details are confusing. It is only by selection, by elimination, by emphasis that we get at the real meaning of things.'''' [$\ldots$]

``As sources of problems and ideas we used the Putnam exam itself, the \textit{International Competition in Mathematics for University Students}, the \textit{International Mathematical Olympiad}, national contests from the USA, Romania, Russia, China, India, Bulgaria, mathematics journals such as the \textit{American Mathematical Monthly, Mathematics Magazine, Revista Matematică din Timişsoara (Timişsoara Mathematics Gazette), Gazeta Matematică (Mathematics Gazette, Bucharest), Kvant (Quantum), Középiskolai Matematikai Lapok (Mathematical Magazine for High Schools (Budapest)}), and a very rich collection of Romanian publications.''

``Considerable care has been taken in selecting the most elegant solutions and writing them so as to stir imagination and stimulate research. We always ``judged mathematical proofs'', as Andrew Wiles once said, ``by their beauty''.'' -- \cite[Preface to the 1st Edition, p. ix]{Gelca_Andreescu2017}.

\section{Method of Proof}
This section is devoted to explain some \textit{methods of mathematical proof}, including \textit{argument by contradiction,  principle of mathematical induction, pigeonhole principle, use of an ordering on a set}, and \textit{principle of invariance}.

``The basic nature of these methods and their universal use throughout mathematics makes this separate treatment necessary.'' [$\ldots$] ``And since these are fundamental methods in mathematics, you should try to understand them in depth, for ``it is better to understand many things than to know many
things'' (Gustave Le Bon).'' -- \cite[Chap. 1, p. 1]{Gelca_Andreescu2017}

\subsection{Argument by Contradiction}
``The method of argument by contradiction proves a statement in the following way:
\begin{tcolorbox}
	\textit{1st, the statement is assumed to be false. Then, a sequence of logical deductions yields a conclusion that contradicts either the hypothesis (indirect method), or a fact known to be true (reductio ad absurdum). This contradiction implies that the original statement must be true.}
\end{tcolorbox}
This is a method that Euclid loved, and you can find it applied in some of the most beautiful proofs from his Elements. Euclid's most famous proof is that of the \textit{infinitude of prime numbers}.'' -- \cite[Sect. 1.1, p. 1]{Gelca_Andreescu2017}

\begin{theorem}[Euclid's theorem]
	There are infinitely many prime numbers.
\end{theorem}

\begin{proof}[Proof]
	See, e.g., \cite[pp. 1--2]{Gelca_Andreescu2017}.
\end{proof}
 
\begin{problem}[Euler's]
	Prove that there is no polynomial $P(x) = \sum_{i=0}^n a_ix^i$ with integer coefficients and of degree at least 1 with the property that $P(0),P(1),P(2),\ldots$ are all prime numbers.
\end{problem}

\begin{proof}[Proof]
	See, e.g., \cite[p. 2]{Gelca_Andreescu2017}.
\end{proof}

\begin{problem}[I. Tomescu's book \textit{Problems in Combinatorics} (Wiley, 1985)]
	Let $F = \{E_1,\ldots,E_s\}$ be a family of subsets with $r$ elements of some set $X$. Show that if the intersection of any $r + 1$ (not necessarily distinct) sets in $F$ is nonempty, then the intersection of all sets in $F$ is nonempty.
\end{problem}

\begin{proof}[Proof]
	See \cite[p. 2]{Gelca_Andreescu2017}.
\end{proof}

\subsection{Mathematical Induction}
``The principle of \textit{mathematical induction}, which lies at the very heart of Peano's axiomatic construction of the set of positive integers, is stated as follows.

\begin{theorem}[Induction principle]
	Given $P(n)$, a property depending on a positive integer $n$,
	\begin{itemize}
		\item[(i)] if $P(n_0)$ is true for some positive integer $n_0$, and
		\item[(ii)] if for every $k\ge n_0$, $P(k)$ true implies $P(k + 1)$ true, then $P(n)$ is true for all $n\ge n_0$.
	\end{itemize}
\end{theorem}
This means that when proving a statement by mathematical induction you should
\begin{itemize}
	\item[(i)] check the base case and
	\item[(ii)] verify the inductive step by showing how to pass from an arbitrary integer to the next.'' -- \cite[Sect. 1.2, pp. 3--4]{Gelca_Andreescu2017}
\end{itemize}
``Even more powerful is strong induction.

\begin{theorem}[Induction principle (strong form)]
	Given $P(n)$ a property that depends on an integer $n$,
	\begin{itemize}
		\item[(i)] If $P(n_0),P(n_0 + 1),\ldots,P(n_0 + m)$ are true for some $n_0\in\mathbb{N}^\star$ and $m\in\mathbb{N}$, and
		\item[(ii)] if for every $k > n_0 + m$, $P(j)$ true for all $n_0\le j < k$ implies $P(k)$ true, then $P(n)$ is true for all $n\ge n_0$.'' -- \cite[p. 7]{Gelca_Andreescu2017}
	\end{itemize}
\end{theorem}

\subsection{Pigeonhole Principle}
``The \textit{pigeonhole principle} (or \textit{Dirichlet's box principle}) is usually applied to problems in combinatorial set theory, combinatorial geometry, and number theory. In its intuitive form, it can be stated as follows.

\begin{theorem}[Pigeonhole principle]
	If $kn + 1$ objects ($k\ge 1$ not necessarily finite) are distributed among $n$ boxes, 1 of the boxes will contain at least $k + 1$ objects.
\end{theorem}
This is merely an observation, and it was Dirichlet who 1st used it to prove nontrivial mathematical results. The name comes from the intuitive image of several pigeons entering randomly in some holes. If there are more pigeons than holes, then we know for sure that 1 hole has more than one pigeon.'' -- \cite[Sect. 1.3, p. 11]{Gelca_Andreescu2017}

\section{Why Mathematics? (Meta)}
\begin{quotation}
	``But I just like mathematics because it is fun. Mathematical problems, or puzzles, are important to real mathematics (like solving real-life problems), just as fables, stories, and anecdotes are important to the young in understanding real life.'' -- \cite[Preface, p. viii]{Tao2006}
\end{quotation}
The prefaces of, as the whole book, \cite{Tao2006} are also very pleasant to read.

%------------------------------------------------------------------------------%

\selectlanguage{vietnamese}
\begin{thebibliography}{99}
	\bibitem[AoPS]{AoPS} \href{https://artofproblemsolving.com/}{Art of Problem Solving} (AoPS).
	
	\bibitem[MSE]{MSE} \href{https://math.stackexchange.com/}{StackExchange\texttt{/}Mathematics}.
	
	\bibitem[NQBH]{NQBH} \href{https://github.com/NQBH/hobby/tree/master/elementary_math}{Nguyễn Quản Bá Hồng's Elementary Mathematics}.
	\begin{itemize}
		\item Nguyễn Quản Bá Hồng. \href{https://github.com/NQBH/hobby/blob/master/elementary_math/grade_6/NQBH_elementary_math_grade_6.pdf}{\textit{Elementary Mathematics}\texttt{/}\textit{Grade 6}}. Mar 2022--now.
	\end{itemize}
	
	\bibitem[TerryTao]{TerryTao} \href{https://terrytao.wordpress.com}{Terence Tao's blog}.
	\begin{itemize}
		\item Terence Tao. \href{https://terrytao.wordpress.com/books/solving-mathematical-problems-a-personal-perspective/}{\textit{Solving Mathematical Problems: A Personal Perspective}}.
	\end{itemize}
\end{thebibliography}

\selectlanguage{english}

\printbibliography[heading=bibintoc]

\end{document}