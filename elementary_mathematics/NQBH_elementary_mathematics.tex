\documentclass[oneside]{book}
\usepackage[backend=biber,natbib=true,style=authoryear]{biblatex}
\addbibresource{/home/hong/1_NQBH/reference/bib.bib}
\usepackage[vietnamese,english]{babel}
\usepackage{tocloft}
\renewcommand{\cftsecleader}{\cftdotfill{\cftdotsep}}
\usepackage[colorlinks=true,linkcolor=blue,urlcolor=red,citecolor=magenta]{hyperref}
\usepackage{amsmath,amssymb,amsthm,mathtools,float,graphicx}
\allowdisplaybreaks
\numberwithin{equation}{section}
\newtheorem{assumption}{Assumption}[chapter]
\newtheorem{conjecture}{Conjecture}[chapter]
\newtheorem{corollary}{Corollary}[chapter]
\newtheorem{definition}{Definition}[chapter]
\newtheorem{example}{Example}[chapter]
\newtheorem{lemma}{Lemma}[chapter]
\newtheorem{notation}{Notation}[chapter]
\newtheorem{principle}{Principle}[chapter]
\newtheorem{problem}{Problem}[chapter]
\newtheorem{proposition}{Proposition}[chapter]
\newtheorem{question}{Question}[chapter]
\newtheorem{remark}{Remark}[chapter]
\newtheorem{theorem}{Theorem}[chapter]
\usepackage[left=0.5in,right=0.5in,top=1.5cm,bottom=1.5cm]{geometry}
\usepackage{fancyhdr}
\pagestyle{fancy}
\fancyhf{}
\lhead{\small \textsc{Sect.} ~\thesection}
\rhead{\small \nouppercase{\leftmark}}
\renewcommand{\sectionmark}[1]{\markboth{#1}{}}
\cfoot{\thepage}
\def\labelitemii{$\circ$}

\title{Elementary Mathematics}
\author{\selectlanguage{vietnamese} Nguyễn Quản Bá Hồng\footnote{Independent Researcher, Ben Tre City, Vietnam\\e-mail: \texttt{nguyenquanbahong@gmail.com}}}
\date{\today}

\begin{document}
\maketitle
\setcounter{secnumdepth}{4}
\setcounter{tocdepth}{4}
\tableofcontents

%------------------------------------------------------------------------------%

\chapter{Wikipedia's}

\section{\href{https://en.wikipedia.org/wiki/How_to_Solve_It}{Wikipedia\texttt{/}How to Solve It}}
``\textit{How to Solve It} (1945) is a small volume by mathematician \href{https://en.wikipedia.org/wiki/George_P%C3%B3lya}{George P\'olya} describing methods of \href{https://en.wikipedia.org/wiki/Problem_solving}{problem solving}.'' -- \href{https://en.wikipedia.org/wiki/How_to_Solve_It}{Wikipedia\texttt{/}how to solve it}

\subsection{4 principles}
``\textit{How to Solve It} suggests the following steps when solving a \href{https://en.wikipedia.org/wiki/Mathematical_problem}{mathematical problem}:
\begin{enumerate}
	\item 1st, you have to \textit{understand the problem}.
	\item After understanding, \textit{make a plan}.
	\item \textit{Carry out the plan}.
	\item \textit{Look back} on your work. How could it be better?
\end{enumerate}
If this technique fails, P\'olya advises: ``If you can't solve a problem, then there is an easier problem you can solve: find it.'' Or: ``If you cannot solve the proposed problem, try to solve 1st some related problem. Could you imagine a more accessible related problem?'''' -- \href{https://en.wikipedia.org/wiki/How_to_Solve_It#Four_principles}{Wikipedia\texttt{/}how to solve it\texttt{/}4 principles}

\subsubsection{1st principle: Understand the problem}
``Understanding the problem'' is often neglected as being obvious \& is not even mentioned in many mathematics classes. Yet students are often stymied in their efforts to solve it, simply because they don't understand it fully, or even in part. In order to remedy this oversight, P\'olya taught teachers how to prompt each student with appropriate questions, depending on the situation, such as:
\begin{itemize}
	\item What are you asked to find or show?
	\item Can you restate the problem in your own words?
	\item Can you think of a picture of a diagram that might help you understand the problem?
	\item Is there enough information to enable you to find a solution?
	\item Do you understand all the words used in stating the problem?
	\item Do you need to ask a question to get the answer?
\end{itemize}
The teacher is to select the question with the appropriate level of difficulty for each student to ascertain if each student understands at their own level, moving up or down the list to prompt each student, until each one can respond with something constructive.'' -- \href{https://en.wikipedia.org/wiki/How_to_Solve_It#First_principle:_Understand_the_problem}{Wikipedia\texttt{/}how to solve it\texttt{/}4 principles\texttt{/}1st principle: understand the problem}

\subsubsection{2nd principle: Devise a plan}
``P\'olya mentions that there are many reasonable ways to solve problems. The skill at choosing an appropriate strategy is best learned by solving many problems. You will find choosing a strategy increasingly easy. A partial list of strategies is included:
\begin{itemize}
	\item Guess \& check
	\item Make an orderly list
	\item Eliminate possibilities
	\item Use symmetry
	\item Consider special cases
	\item Use direct reasoning
	\item Solve an equation
\end{itemize}
Also suggested:
\begin{itemize}
	\item Look for a pattern
	\item Draw a picture
	\item Solve a simpler problem
	\item Use a model
	\item Work backward
	\item Use a formula
	\item Be creative
	\item Applying these rules to devise a plan takes your own skill \& judgment.
\end{itemize}
P\'olya lays a big emphasis on the teachers' behavior. A teacher should support students with devising their own plan with a question method that goes from the most general questions to more particular questions, with the goal that the last step to having a plan is made by the student. He maintains that just showing students a plan, no matter how good it is, does not help them.'' -- \href{https://en.wikipedia.org/wiki/How_to_Solve_It#Second_principle:_Devise_a_plan}{Wikipedia\texttt{/}how to solve it\texttt{/}4 principles\texttt{/}2nd principle: devise a plan}

\subsubsection{3rd principle: Carry out the plan}
``This step is usually easier than devising the plan. In general, all you need is care \& patience, given that you have the necessary skills. Persist with the plan that you have chosen. If it continues not to work, discard it \& choose another. Don't be misled; this is how mathematics is done, even by professionals.'' -- \href{https://en.wikipedia.org/wiki/How_to_Solve_It#Third_principle:_Carry_out_the_plan}{Wikipedia\texttt{/}how to solve it\texttt{/}4 principles\texttt{/}3rd principle: carry out the plan}

\subsubsection{4th principle: Review\texttt{/}extend}
``P\'olya mentions that much can be gained by taking the time to reflect \& look back at what you have done, what worked \& what did not, \& with thinking about other problems where this could be useful. Doing this will enable you to predict what strategy to use to solve future problems, if these relate to the original problem.'' -- \href{https://en.wikipedia.org/wiki/How_to_Solve_It#Fourth_principle:_Review/extend}{Wikipedia\texttt{/}how to solve it\texttt{/}4 principles\texttt{/}4th principle: review\texttt{/}extend}

\subsection{Heuristics}
``The book contains a dictionary-style set of \href{https://en.wikipedia.org/wiki/Heuristics}{heuristics}, many of which have to do with generating a more accessible problem. E.g.:

\textbf{Heuristic $|$ Informal Description $|$ Formal analogue}
\begin{itemize}
	\item \href{https://en.wikipedia.org/wiki/Analogy}{Analogy} $|$ Can you find a problem analogous to your problem \& solve that? $|$ \href{https://en.wikipedia.org/wiki/Map_(mathematics)}{map}
	\item Auxiliary Elements $|$ Can you add some new element to your problem to get closer to a solution? $|$ \href{https://en.wikipedia.org/wiki/Extension_(predicate_logic)}{Extension}
	\item \href{https://en.wikipedia.org/wiki/Generalization}{Generalization} $|$ Can you find a problem more general than your problem? $|$ \href{https://en.wikipedia.org/wiki/Generalization}{Generalization}
	\item \href{https://en.wikipedia.org/wiki/Induction_(philosophy)}{Induction} $|$ Can you solve your problem by deriving a generalization from some examples? $|$ \href{https://en.wikipedia.org/wiki/Induction_(philosophy)}{Induction}
	\item Variation of the Problem $|$ Can you vary or change your problem to create a new problem (or set of problems) whose solution(s) will help you solve your original problem? $|$ \href{https://en.wikipedia.org/wiki/Search_algorithm}{Search}
	\item Auxiliary Problem $|$ Can you find a subproblem or side problem whose solution will help you solve your problem? $|$ \href{https://en.wikipedia.org/wiki/Subgoal}{Subgoal}
	\item Here is a problem related to yours \& solved before $|$ Can you find a problem related to yours that has already been solved \& use that to solve your problem? $|$ \href{https://en.wikipedia.org/wiki/Pattern_recognition}{Pattern recognization}, \href{https://en.wikipedia.org/wiki/Pattern_matching}{Pattern matching}, \href{https://en.wikipedia.org/wiki/Reduction_(complexity)}{Reduction}
	\item \href{https://en.wikipedia.org/wiki/Special_case}{Specialization} $|$ Can you find a problem more specialized? $|$ \href{https://en.wikipedia.org/wiki/Special_case}{Specialization}
	\item \href{https://en.wikipedia.org/wiki/Decomposition_(computer_science)}{Decomposing} \& Recombining $|$ Can you decompose the problem \& ``recombine its elements in some new manner''? $|$ \href{https://en.wikipedia.org/wiki/Divide_and_conquer_algorithm}{Divide \& conquer}
	\item \href{https://en.wikipedia.org/wiki/Working_backward_from_the_goal}{Working backward} $|$ Can you start with the goal \& work backwards to something you already know? $|$ \href{https://en.wikipedia.org/wiki/Backward_chaining}{Backward chaining}
	\item Draw a Figure $|$ Can you draw a picture of the problem? $|$ \href{https://en.wikipedia.org/wiki/Diagrammatic_Reasoning}{Diagrammatic Reasoning}
\end{itemize}
'' -- \href{https://en.wikipedia.org/wiki/How_to_Solve_It#Heuristics}{Wikipedia\texttt{/}how to solve it\texttt{/}heuristics}

\subsection{Influence}
\begin{itemize}
	\item ``The book has been translated into several languages \& has sold over a million copies, \& has been continuously in print since its 1st publication.
	\item \href{https://en.wikipedia.org/wiki/Marvin_Minsky}{Marvin Minsky} said in his paper \textit{Steps Toward Artificial Intelligence} that ``everyone should know the work of George P\'olya on how to solve problems.''
	\item P\'olya's book has had a large influence on mathematics textbooks as evidenced by the bibliographies for \href{https://en.wikipedia.org/wiki/Mathematics_education}{mathematics education}.
	\item Russian inventor \href{https://en.wikipedia.org/wiki/Genrich_Altshuller}{Genrich Altshuller} developed an elaborate set of methods for problem solving known as \href{https://en.wikipedia.org/wiki/TRIZ}{TRIZ}, which in many aspects reproduces or parallels P\'olya's work.
	\item \href{https://en.wikipedia.org/wiki/How_to_Solve_it_by_Computer}{How to Solve it by Computer} is a computer science book by R. G. Dromey. It was inspired by P\'olya's work.'' -- \href{https://en.wikipedia.org/wiki/How_to_Solve_It#Influence}{Wikipedia\texttt{/}how to solve it\texttt{/}influence}
\end{itemize}

%------------------------------------------------------------------------------%

\chapter{\cite{Polya2014}. How to Solve It: A New Aspect of Mathematical Methods}

\section*{From the Preface to the 1st Printing}
``A great discovery solves a great problem but there is a grain\footnote{\textbf{grain} [n] \textbf{1.} [uncountable, countable] the small hard seeds of food plants such as wheat, rice, etc.; a single seed of such a plant; \textbf{2.} [countable] \textbf{grain (of something)} a small piece of a particular substance; usually a hard substance; \textbf{3.} [countable, usually singular] \textbf{grain of something} a very small amount; \textbf{4.} [countable] an individual particle or crystal in metal, rock, etc., usually explained with a lens or microscope.} of discovery in the solution of any problem. Your problem may be modest\footnote{\textbf{modest} [a] \textbf{1.} fairly limited or small in amount; \textbf{2.} not expensive, rich or impressive; \textbf{3.} (of people, especially women, or their clothes) not showing too much of the body; not intended to attract attention, especially in a sexual way; \textbf{4.} (\textit{approving}) not talking much about your own abilities or possessions.}; but it challenges your curiosity\footnote{\textbf{curiosity} [n] (plural \textbf{curiosities}) \textbf{1.} [uncountable, singular] a strong desire to know about something; \textbf{2.} [countable] \textbf{curiosity (of something)} an unusual \& interesting thing.} \& brings into play your inventive\footnote{\textbf{inventive} [a] \textbf{1.} (especially of people) able to create or design new things or think of new ideas; \textbf{2.} (of ideas) new \& interesting.} faculties\footnote{\textbf{faculty} [n] (plural \textbf{faculties}) \textbf{1.} [countable] a physical or mental ability, especially one that people are born with; \textbf{2.} [countable] \textbf{faculty (of something)} a department or group of related departments in a college or university; \textbf{3.} [countable $+$ singular or plural verb] all the teachers in a faculty of a college or university; \textbf{4.} [countable, uncountable] (\textit{NAE}) all the teachers of a particular university or college.}, \& if you solve it by your own means, you may experience the tension\footnote{\textbf{tension} [n] \textbf{1.} [uncountable, countable, usually plural] a situation in which people do not trust each other, or feel unfriendly towards each other, \& which may cause them to attack each other; \textbf{2.} [countable, uncountable] \textbf{tension (between A \& B)} a situation in which the fact that there are different needs or interests causes difficulties; \textbf{3.} [uncountable] a feeling of anxiety \& stress that makes it impossible to relax; \textbf{4.} [uncountable] the feeling of fear \& excitement that is created by a writer or a film director; \textbf{5.} [uncountable] the state of being stretched tight; the extent to which something is stretched tight.} \& enjoy the triumph\footnote{\textbf{triumph} [n] \textbf{1.} [countable, uncountable] a great success, achievement or victory; \textbf{2.} [uncountable] the state of having achieved a great success or victory; the feeling of happiness that you get from this; [v] [intransitive] to defeat somebody\texttt{/}something; to be successful.} of discovery. Such experiences at a susceptible\footnote{\textbf{susceptible} [a] \textbf{1.} [not usually before noun] \textbf{susceptible (to somebody\texttt{/}something)} very likely to be influenced, harmed or affected by somebody\texttt{/}something; \textbf{2.} \textbf{susceptible (of something)} (\textit{formal}) allowing something; capable of something.} age may create a taste for mental work \& leave their imprint\footnote{\textbf{imprint} [v] [often passive] \textbf{1.} to have a great effect on something so that it cannot be forgotten, changed, etc.; \textbf{2.} to print or press a mark or design onto a surface; [n] \textbf{1.} \textbf{imprint (of something) (in\texttt{/}on something)} a mark made by pressing something onto a surface; \textbf{2.} [usually singular] \textbf{imprint (of something) (on somebody\texttt{/}something)} (\textit{formal}) the lasting effect that a person or an experience has on a place or a situation; \textbf{3.} (\textit{specialist}) the name of the publisher of a book, usually printed below the title on the 1st page; a brand name under which books are published.} on mind \& character for a lifetime\footnote{\textbf{lifetime} [n] the length of time that somebody lives or that something lasts.}.

Thus, a teacher of mathematics has a great opportunity. If he fills his allotted\footnote{\textbf{allot} [v] to give time, money, tasks, etc. to somebody\texttt{/}something as a share of what is available, \textsc{synonym}: \textbf{allocate}.} time with drilling his students in routine operations he kills their interest, hampers\footnote{\textbf{hamper} [v] [often passive] to prevent something from being achieved easily or happening normally; to prevent somebody from easily doing something, \textsc{synonym}: \textbf{hinder, impede}.} their intellectual development, \& misuses his opportunity. But if he challenges the curiosity of his students by setting them problems proportionate\footnote{\textbf{proportionate} [a] increasing or decreasing in size, amount or degree according to changes in something else, \textsc{synonym}: \textbf{proportional}.} to their knowledge, \& helps them to solve their problems with stimulating\footnote{\textbf{stimulating} [a] \textbf{1.} full of interesting or exciting ideas; making people feel enthusiastic; \textbf{2.} making you feel more active \& healthy.} questions, he may give them a taste for, \& some means of, independent thinking.

Also a student whose college curriculum\footnote{\textbf{curriculum} [n] (plural \textbf{curricula}) the subjects that are included in a course of study or taught in a school, college or university.} includes some mathematics has a singular\footnote{\textbf{singular} [n] [singular] (\textit{grammar}) a form of a noun or verb that refers to 1 person or thing; [a] \textbf{1.} (\textit{grammar}) connected with or having the form of a noun or verb that refers to 1 person or thing; \textbf{2.} especially great or obvious, \textsc{synonym}: \textbf{outstanding}; \textbf{3.} (\textit{mathematics, physics}) connected with a singularity.} opportunity. This opportunity is lost, of course, if he regards\footnote{\textbf{regard} [v] [often passive] to think about somebody\texttt{/}something in a particular way; \textbf{as regards somebody\texttt{/}something} [idiom] concerning or in connection with somebody\texttt{/}something; [n] \textbf{1.} [uncountable] attention to or thought \& care for somebody\texttt{/}something; \textbf{2.} [uncountable] \textbf{regard (for somebody\texttt{/}something)} respect or admiration for somebody\texttt{/}something. If you \textbf{hold somebody in high regard}, you have a good opinion of them.; \textbf{3.} (\textbf{regards}) [plural] used to send good wishes to somebody at the end of a letter or email; \textbf{have regard to something} [idiom] (\textit{law}) to remember \& think carefully about something; \textbf{in\texttt{/}with regard to somebody\texttt{/}something} [idiom] concerning somebody\texttt{/}something; \textbf{in this\texttt{/}that regard} [idiom] concerning what has just been mentioned.} mathematics as a subject in which he has to earn so \& so much credit \& which he should forget after the final examination as quickly as possible. The opportunity may be lost even if the student has some natural talent for mathematics because he, as everybody else, must discover his talents \& tastes; he cannot know that he likes raspberry pie if he has never tasted raspberry pie. He may manage to find out, however, that a mathematics problem may be as much fun as a crossword puzzle\footnote{\textbf{crossword} [n] (also \textbf{crossword puzzle}) a game in which you have to fit words across \& downwards into spaces with numbers in a square diagram. You find the words by solving clues.}, or that vigorous\footnote{\textbf{vigorous} [a] \textbf{1.} involving physical strength, effort or energy; \textbf{2.} done with determination, energy or enthusiasm; \textbf{3.} strong \& healthy.} mental work may be an exercise as desirable as a fast game of tennis. Having tasted the pleasure in mathematics he will not forget it easily \& then there is a good chance that mathematics will become something for him: a hobby, or a tool of his profession, or a great ambition,

The author remembers the time when he was a student himself, a somewhat ambitious student, eager to understand a little mathematics \& physics. He listened to lectures, read books, tried to take in the solutions \& facts presented, but there was a question that disturbed\footnote{\textbf{disturb} [v] \textbf{1.} \textbf{disturb something} to change the arrangement of something, or affect how something functions; \textbf{2.} \textbf{disturb somebody\texttt{/}something} to interrupt somebody \& prevent them from continuing with what they are doing; \textbf{3.} \textbf{disturb somebody} to make somebody feel anxious or upset.} him again \& again: ``Yes, the solution seems to work, it appears to be correct; but how is it possible to invent such a solution? Yes, this experiment seems to work, this appears to be a fact; but how can people discover such facts? \& how could I invent or discover such things by myself?'' Today the author is teaching mathematics in a university; he thinks or hopes that some of his more eager students ask similar questions \& he tries to satisfy their curiosity. Trying to understand not only the solution of this or that problem but also the motives \& procedures of the solution, \& trying to explain these motives \& procedures to others, he was finally led to write the present book. He hopes that it will be useful to teachers who wish to develop their students' ability to solve problems, \& to students who are keen on developing their own abilities.

Although the present book pays special attention to the requirements of students \& teachers of mathematics, it should interest anybody concerned with the ways \& means of invention \& discovery. Such interest may be more widespread\footnote{\textbf{widespread} [a] existing or happening over a large area or among many people, \textsc{synonym}: \textbf{extensive}.} than one would assume without reflection\footnote{\textbf{reflection} [n] \textbf{1.} [countable] \textbf{reflection of something} an account or description of what somebody\texttt{/}something is like; a thing that is a result of something else; \textbf{2.} [uncountable] careful thought about something, especially your work or studies; \textbf{3.} [countable, usually plural] \textbf{reflections (on something)} written or spoken thoughts about a particular subject; \textbf{4.} [uncountable] \textbf{reflection (of something)} the action or process of sending back light, heat, sound, etc. from a surface; \textbf{5.} (also \textbf{reflexion}) [countable, uncountable] \textbf{reflection (of something)} (\textit{mathematics}) an operation on a shape to produce its mirror image.}. The space devoted by popular newspapers \& magazines to crossword puzzles \& other riddles\footnote{\textbf{riddle} [n] \textbf{1.} a question that is difficult to understand, \& that has a surprising answer, that you ask somebody as a game; \textbf{2.} a mysterious event or situation that you cannot explain, \textsc{synonym}: \textbf{mystery}; [v] \textbf{riddle somebody\texttt{/}something (with something)} to make a lot of holes in; \textbf{be riddle with something} [idiom] to be full of something, especially something bad or unpleasant.} seems to show that people spend some time in solving unpractical\footnote{\textbf{unpractical} [a] \textbf{1.} not sensible or realistic; \textbf{2.} (9of people) not good at doing things that involve using the hands; not good at planning or organizing things, \textsc{opposite}: \textbf{practical}.} problems. Behind the desire to solve this or that problem that confers\footnote{\textbf{confer} [v] \textbf{1.} [transitive] to give somebody a particular power, right or honor; \textbf{2.} [transitive] to give somebody\texttt{/}something a particular advantage; \textbf{3.} [intransitive] \textbf{confer (with somebody) (on\texttt{/}about something)} to discuss something with somebody, in order to exchange opinions or get advice.} no material advantage, there may be a deeper curiosity, a desire to understand the ways \& means, the motives \& procedures, of solution.

The following pages are written somewhat concisely\footnote{\textbf{concise} [a] giving only the information that is necessary \& important, using few words.}, but as simply as possible, \& are based on a long \& serious study of methods of solution. This sort of study, called \textit{heuristic}\footnote{\textbf{heuristic} [a] (\textit{formal}) \textbf{heuristic} teaching or education encourages you to learn by discovering things for yourself.}\,\footnote{\textbf{heuristics} [n] [uncountable] (\textit{formal}) a method of solving problems by finding practical ways of dealing with them, learning from past experience.} by some writers, is not in fashion nowadays but has a long past \&, perhaps, some future.

Studying the methods of solving problems, we perceive\footnote{\textbf{perceive} [v] \textbf{1.} to notice or become aware of something, \textsc{synonym}: \textbf{notice}; \textbf{2.} to be aware of or experience something using the senses; \textbf{3.} [often passive] to understand or think of somebody\texttt{/}something in a particular way; to believe that a particular thing is true, \textsc{synonym}: \textbf{see}.} another face of mathematics. Yes, mathematics has 2 faces; it is the rigorous\footnote{\textbf{rigorous} [a] \textbf{1.} done carefully \& with a lot of attention to detail, \textsc{synonym}: \textbf{thorough}; \textbf{2.} demanding that particular rules or processes are strictly followed, \textsc{synonym}: \textbf{strict}.} science of Euclid but it is also something else. Mathematics presented in the Euclidean way appears as a systematic\footnote{\textbf{systematic} [a] \textbf{1.} done according to a system or plan, in a thorough, efficient or determined way; \textbf{2.} (of an error) happening in the same way all through a process or set of results; caused by the system that is used.}, deductive\footnote{\textbf{deductive} [a] [usually before noun] using knowledge about things that are generally true in order to understand particular situations or problems.} science; but mathematics in the making appears as an experimental\footnote{\textbf{experimental} [a] \textbf{1.} [usually before noun] connected with scientific experiments; \textbf{2.} based on new ideas, forms or methods that are used to find out what effect they have.}, inductive\footnote{\textbf{inductive} [a] (\textit{specialist}) using particular facts \& examples to form general rules \& principles.} science. Both aspects\footnote{\textbf{aspect} [n] \textbf{1.} [countable] a particular feature of a situation, an idea or a process; a way in which something may be considered; \textbf{2.} [countable, usually singular] \textbf{aspect (of something)} (\textit{specialist}) a particular surface or side of an object or a part of the body; the direction in which something faces; \textbf{3.} [uncountable, countable] (\textit{grammar}) the form of a verb that shows, e.g., whether the action happens once or many times, is completed or is still continuing.} are as old as the science of mathematics itself. But the 2nd aspect is new in 1 respect\footnote{\textbf{respect} [n] \textbf{1.} [countable] a particular aspect or detail of something; \textbf{2.} [uncountable, singular] polite behavior towards or reasonable treatment of somebody\texttt{/}something; \textbf{3.} [uncountable, singular] a feeling of admiration for somebody\texttt{/}something because of their good qualities or achievements; \textbf{in respect of something} [idiom] (\textit{formal}) \textbf{1.} concerning; \textbf{2.} in payment for something; \textbf{with respect to something} [idiom] concerning.}; mathematics ``in statu nascendi,'' in the process of being invented, has never before presented in quite this manner to the student, or to the teacher himself, or to the general public.

The subject of heuristic has manifold\footnote{\textbf{manifold} [a] (\textit{formal}) many; of many different types; [n] (\textit{specialist}) a pipe or chamber with several openings, especially 1 for taking gases in \& out of a car engine.} connections; mathematicians, logicians\footnote{\textbf{logician} [n] a person who studies or is skilled in logic.}, psychologists, educationalists\footnote{\textbf{educationalists} [n] (also \textbf{educationist}) a specialist in theories \& methods of teaching.}, even philosophers may claim various parts of it as belonging to their special domains. The author, well aware of the possibility of criticism\footnote{\textbf{criticism} [n] \textbf{1.} [uncountable, countable] the act of expressing disapproval of somebody\texttt{/}something \& opinions about their faults or bad qualities; a statement showing disapproval; \textbf{2.} [uncountable] the work or activity of analyzing \& making fair, careful judgments about somebody\texttt{/}something, especially books, music, etc.} from opposite\footnote{\textbf{opposite} [a] \textbf{1.} [usually before noun] as different as possible from something; involving 2 different extremes; \textbf{2.} [usually before noun] on the other side of something or facing something; [n] \textbf{1.} (\textbf{the opposite}) [singular] the situation, idea or activity that is as different from another situation, etc. as it is possible to be, \textsc{synonym}: \textbf{the reverse}; \textbf{2.} (\textbf{opposites}) [plural] people, ideas or situations that are as different as possible from each other; \textbf{the exact opposite} [idiom] a person or thing that is as different as possible from somebody\texttt{/}something else; [prep] on the other side of a particular area from somebody\texttt{/}something, \& usually facing them.} quarters\footnote{\textbf{quarter} [n] \textbf{1.} (also \textbf{fourth} \textit{especially in NAE}) [countable] 1 of 4 equal parts of something; \textbf{2.} [countable] a period of 3 months, used especially as a period fo which bills are paid or a company's income is calculated; \textbf{3.} [countable] a person or group of people, especially as a source of help, information or a reaction; \textbf{4.} [countable, usually singular] a district or part of a town; \textbf{5.} (\textbf{quarters}) [plural] rooms that are provided for soldiers, servants, etc. to live in; \textbf{at\texttt{/}from close quarters} [idiom] very near.} \& keenly\footnote{\textbf{keenly} [adv] \textbf{1.} very strongly or deeply; \textbf{2.} by people with different opinions that they express strongly.} conscious\footnote{\textbf{conscious} [a] \textbf{1.} [not before noun] aware of something; noticing something, \textsc{opposite}: \textbf{unconscious}; \textbf{2.} able to use your senses \& mental powers to understand what is happening, \textsc{opposite}: \textbf{unconscious}; \textbf{3.} (of actions, feelings, etc.) deliberate or controlled, \textsc{opposite}: \textbf{unconscious}; \textbf{4.} being particularly interested in something.} of his limitations\footnote{\textbf{limitation} [n] \textbf{1.} [countable, usually plural] a limit on what somebody\texttt{/}something can do or how good they\texttt{/}it can be; \textbf{2.} [countable] a rule, fact or condition that limits something, \textsc{synonym}: \textbf{restraint}; \textbf{3.} [uncountable] \textbf{limitation (of something)} the act or process of limiting or controlling somebody\texttt{/}something, \textsc{synonym}: \textbf{restriction}; \textbf{4.} (also \textbf{limitation period}) [countable] (\textit{law}) a legal limit on the period of time within which court proceedings can be taken or for which a property right continues.}, has 1 claim to make: he has some experience in solving problems \& in teaching mathematics on various levels.

The subject is more fully dealt with in a more extensive book by the author which is on the way to completion. \textit{Stanford University, Aug 1, 1944}'' -- \cite[pp. v--vii]{Polya2014}

\section*{From the Preface to the 7th Printing}
``[$\ldots$] The 2 volumes \textit{Induction \& Analogy in Mathematics} \& \textit{Patterns of Plausible Inference} which constitute my recent work \textit{Mathematics \& Plausible Reasoning} continue the line of thinking begun in \textit{How to Solve It}.  \textit{Zurich, Aug 30, 1954}'' -- \cite[p. viii]{Polya2014}

\section*{Preface to the 2nd Edition}
``The present 2nd edition adds, besides a few minor improvements, a new 4th part, ``Problems, Hints, Solutions.''

As this edition was being prepared for print, a study appeared (Educational Testing Service, Princeton, N.J.; \textit{cf. Time}, Jun 18, 1956) which seems to have formulated\footnote{\textbf{formulate} [v] \textbf{1.} \textbf{formulate something} to create or prepare something carefully, giving particular attention to the details; \textbf{2.} \textbf{formulate something} to express your ideas in carefully chosen words.} a few pertinent\footnote{\textbf{pertinent} [a] appropriate to a particular situation, \textsc{synonym}: \textbf{relevant}.} observations -- they are not new to the people in the know, but it was high time to formulate them for the general public--: ``$\ldots$ mathematics has the dubious\footnote{\textbf{dubious} [a] \textbf{1.} that you cannot be sure about; that is probably not good. \textbf{Dubious} is also when you are stating that something is the opposite of a particular good quality. \textbf{2.} [usually before noun] probably not honest, \textsc{synonym}: \textbf{suspicious}; \textbf{3.} [not usually before noun] \textbf{dubious about something} feeling uncertain about something; not knowing whether something is good or bad, \textsc{synonym}: \textbf{doubtful}.}, honor of being the least popular subject in the curriculum $\ldots$ Future teachers pass through the elementary schools learning to detest\footnote{\textbf{detest} [v] (not used in the progressive tenses) to hate somebody\texttt{/}something very much, \textsc{synonym}: \textbf{loathe}.} mathematics $\ldots$ They return to the elementary school to teach a new generation to detest it.''

I hope that the present edition, designed for wider diffusion\footnote{\textbf{diffusion} [n] [uncountable] \textbf{1.} the spreading of something more widely; \textbf{2.} the mixing of substances by the natural movement of their particles; \textbf{3.} the spreading of elements of culture from 1 region or group to another.}, will convince some of its readers that mathematics, besides being a necessary avenue\footnote{\textbf{avenue} [n] a way of approaching a problem or making progress towards something.} to engineering jobs \& scientific knowledge, may be fun \& may also open up a vista\footnote{\textbf{vista} [n] \textbf{1.} (\textit{literary}) a beautiful view, e.g., of the countryside, a city, etc., \textsc{synonym}: \textbf{panorama}; \textbf{2.} (\textit{formal}) a range of things that might happen in the future, \textsc{synonym}: \textbf{prospect}.} of mental activity on the highest level. \textit{Zurich, Jun 30, 1956}'' -- \cite[pp. ix--]{Polya2014}

\section*{``How to Solve It'' list}
\begin{itemize}
	\item[\textbf{1st.}] You have to \textit{understand} the problem.
	
	\textsc{Understanding the Problem.}
	
	\textit{What is the unknown? What are the data? What is the condition?}
	
	It is possible to satisfy the condition? Is the condition sufficient to determine the unknown? Or is it insufficient? Or redundant? Or contradictory?
	
	Draw a figure. Introduce suitable notation.
	
	Separate the various parts of the condition. Can you write them down?
	\item[\textbf{2nd.}] Find the connection between the data \& the unknown. You may be obliged to consider auxiliary problems if an immediate connection cannot be found. You should obtain eventually a \textit{plan} of the solution.
	
	\textsc{Devising a Plan.}
	
	Have you seen it before? Or have you seen the same problem in a slightly different form?
	
	\textit{Do you know a related problem?} Do you know a theorem that could be useful?
	
	\textit{Look at the unknown!} \& try to think of a familiar problem having the same or a similar unknown.
	
	\textit{Here is a problem related to yours \& solved before. Could you use it?} Could you use its result? Could you use its method? Should you introduce some auxiliary element in order to make its use possible?
	
	Could you restate the problem? Could you restate it still differently? Go back to definitions.
	
	If you cannot solve the proposed problem try to solve 1st some related problem. Could you imagine a more accessible related problem? A more general problem? A more special problem? An analogous problem? Could you solve a part of the problem? Keep only a part of the condition, drop the other part; how far is the unknown then determined, how can it vary? Could you derive something useful from the data? Could you think of other data appropriate to determine the unknown? Could you change the unknown or the data, or both if necessary, so that the new unknown \& the new data are nearer to each other?
	
	Did you use all the data? Did you use the whole condition? Have you taken into account all essential notions involved in the problem?
	\item[\textbf{3rd.}] \textit{Carry out} your plan.
	
	\textsc{Carrying out the Plan.}
	
	Carrying out your plan of the solution, \textit{check each step}. Can you see clearly that the step is correct? Can you prove that it is correct?
	\item[\textbf{4th.}] \textit{Examine} the solution obtained.
	
	\textsc{Looking back.}
	
	Can you \textit{check the result?} Can you check the argument?
	
	Can you derive the result differently? Can you see it at a glance?
	
	Can you use the result, or the method, for some other problem?
\end{itemize}
-- \cite[How to solve it, pp. xvi--xvii]{Polya2014}

\section*{Foreword by John H. Conway}
``\textit{How to Solve It} is a wonderful book! This I realized when I 1st read right through it as a student many years ago, but it has taken me a long time to appreciate just \textit{how} wonderful it is. Why is that? 1 part of the answer is that the book is unique. In all my years as a student \& teacher, I have never seen another that lives up to George Polya's title by teaching you how to go about solving problems. A. H. Schoenfeld correctly described its importance in his 1987 article ``Polya, Problem Solving, \& Education'' in \textit{Mathematics Magazine}: ``For mathematics education \& the world of problem solving it marked a line of demarcation\footnote{\textbf{demarcation} [n] [uncountable, countable] a line or limit that separates 2 things, such as types of work, groups of people or areas of land.} between 2 eras\footnote{\textbf{era} [n] \textbf{1.} a period of time, usually in history, that is different from other periods because of particular characteristics or events; \textbf{2.} (\textit{earth sciences}) a major division of time that can itself be divided into periods.}, problem solving before \& after Polya.''

It is 1 of the most successful mathematics books ever written, having sold over a million copies \& been translated into 17 languages since it 1st appeared in 1945. Polya later wrote 2 more books about the art of doing mathematics, \textit{Mathematics \& Plausible Reasoning} (1954) \& \textit{Mathematical Discovery} (2 volumes, 1962 \& 1965).

The book's title makes it seem that it is directed only toward students, but in fact it is addressed just as much to their teachers. Indeed, as Polya remarks in his introduction, the 1st part of the book takes the teacher's viewpoint more often than the student's.

Everybody gains that way. The student who reads the book on his own will find that overhearing\footnote{\textbf{overhear} [v] to hear, especially by accident, a conversation in which you are not involved.} Polya's comments to his non-existent\footnote{\textbf{non-existent} [a] not existing; not real.} teacher can bring that desirable person into being, as an imaginary but very helpful figure leaning over one's shoulder. This is what happened to me, \& naturally I made heavy use of the remarks I'd found most important when I myself started teaching a few years later.

But it was some time before I read the book again, \& when I did, I suddenly realized that it was even more valuable than I'd thought! Many of Polya's remarks that hadn't helped me as a student now made me a better teacher of those whose problems had differed from mine. Polya had met many more students than I had, \& had obviously thought very hard about how to best help all of them learn mathematics. Perhaps his most important point is that \fbox{learning must be active}. As he said in a lecture on teaching, ``Mathematics, you see, is not a spectator\footnote{\textbf{spectator} [n] a person who is watching a performance or an event.} sport. To understand mathematics means to be able to do mathematics. \& what does it mean [to be] doing mathematics? In the 1st place, it means to be able to solve mathematical problems.''

It is often said that to teach any subject well, one has to understand it ``at least as well as one's students do.'' It is a paradoxical\footnote{\textbf{paradoxical} [a] \textbf{1.} (of a person, thing or situation) having 2 opposite features \& therefore seem strange; \textbf{2.} (of a statement) containing 2 opposite ideas that make it seem impossible or unlikely, although it is probably true.} truth that to teach mathematics well, one must also know how to misunderstand it at least to the extent one's students do! If a teacher's statement can be parsed\footnote{\textbf{parse} [v] (\textit{grammar}) \textbf{parse something} to divide a sentence into parts \& describe the grammar of each word or part.} in 2 or more ways, it goes without saying that some students will understand it 1 way \& others another, with results that can vary from the hilarious\footnote{\textbf{hilarious} [a] extremely funny.} to the tragic\footnote{\textbf{tragic} [a] \textbf{1.} making you feel very sad, usually because somebody has died or suffered a lot; \textbf{2.} [usually before noun] connected with tragedy ($=$ the style of literature).}. J. E. Littlewood gives 2 amusing\footnote{\textbf{amusing} [a] funny \& giving pleasure.} examples of assumptions that can easily be made unconsciously \& misleadingly\footnote{\textbf{misleading} [a] giving the wrong idea \& making people believe something that is not true, \textsc{synonym}: \textbf{deceptive}.}. 1st, he remarks that the description of the coordinate axes (``$Ox$ \& $Oy$ as in 2 dimensions, $Oz$ vertical'') in Lamb's book \textit{Mechanics} is incorrect for him, sine he always worked in an armchair\footnote{\textbf{armchair} [n] a comfortable chair with sides on which you can rest your arms; [a] [only before noun] knowing about a subject through books, television, the Internet, etc., rather than by doing it for yourself.} with his feet up! Then, after asking how his reader would present the picture of a closed curve lying all on 1 side of its tangent, he states that there are 4 main schools (to left or right of vertical tangent, or above or below horizontal one) \& that by lecturing without a figure, presuming that the curve was to the right of its vertical tangent, he had unwittingly\footnote{\textbf{unwittingly} [adv] without being aware of what you are doing or the situation that you are involved in, \textsc{opposite}: \textbf{wittingly}.} made nonsense\footnote{\textbf{nonsense} [n] \textbf{1.} [uncountable, countable] ideas, statements or beliefs that you think are silly or not true, \textsc{synonym}: \textbf{rubbish}; \textbf{2.} [uncountable] spoken or written words that have no meaning or make no sense; \textbf{3.} [uncountable] silly or unacceptable behavior; \textbf{make (a) nonsense of something} [idiom] to reduce the value of something by a lot; to make something seem silly.} for the other 3 schools.

I know of no better remedy\footnote{\textbf{remedy} [n] (plural \textbf{remedies}) \textbf{1.} a way of dealing with or improving an unpleasant or difficult situation, \textsc{synonym}: \textbf{solution}; \textbf{2.} a treatment or medicine to cure a disease or to reduce pain that is not very serious; \textbf{3.} (\textit{law}) a way of dealing with a problem, using the processes of the law, \textsc{synonym}: \textbf{redress}; [v] \textbf{remedy something} to correct or improve something.} for such presumptions\footnote{\textbf{presumption} [n] [countable, uncountable] the act of supposing or accepting that something is true or exists, although it has not been proved; a belief that something is true or exists, \textsc{synonym}: \textbf{assumption}. In legal contexts, \textbf{presumption} often means that something is being accepted as true until it is shown not to be true.} than Polya's counsel\footnote{\textbf{counsel} [n] [uncountable, countable] \textbf{1.} (\textit{formal}) advice, especially given by older people or experts; a piece of advice; \textbf{2.} a lawyer or group of lawyers representing somebody in court; [v] (\textit{formal}) \textbf{1.} \textbf{counsel somebody} to listen to \& give support or professional advice to somebody who needs help; \textbf{2.} to advise a particular course of action; to advise somebody to do something.}: before trying to solve a problem, the student should demonstrate his or her understanding of its statement, preferably\footnote{\textbf{preferable} [a] more attractive or more suitable; to be preferred to something.} to a real teacher, but in lieu\footnote{\textbf{lieu} [n] (\textit{formal}) \textbf{in lieu (of something)} [idiom] instead of.} of that, to an imagined one. Experienced mathematicians know that often the hardest part of researching a problem is understanding precisely what that problem says. They often follow Polya's wise advice: ``If you can't solve a problem, then there is an easier problem you can't solve: find it.''

Readers who learn from this book will also want to learn about its author's life.\footnote{The following biographical information is taken from that given by J. J. O'Connor \& E. F. Robertson in the MacTutor History of Mathematics Archive (\url{www-gap.dcs.st-and.ac.uk/~hisotry/}).}

George Polya was born Gy\"orgy P\'olya (he dropped the accents sometime later) on Dec 13, 1887, in Budapest, Hungary, to Jakab P\'olya \& his wife, the former Anna Deutsch. He was baptized into the Roman Catholic faith, to which Jakab, Anna, \& their 3 previous children, Jen\H{o}, Ilona, \& Fl\'ora, had converted from Judaism\footnote{\textbf{Judaism} [n] [uncountable] the religion of the Jewish people, based mainly on the Bible \& the Talmud ($=$ a collection of ancient writings on Jewish law \& traditions).} in the previous year. The 5th child, L\'aszl\'o, was born 4 years later.

Jakab had changed his surname from Poll\'ak to the more Hungarian-sounding P\'olya 5 years before Gy\"orgy was born, believing that this might help him obtain a university post, which he eventually did, but only shortly before his untimely\footnote{\textbf{untimely} [a] (\textit{formally}) \textbf{1.} [usually before noun] happening too soon or sooner than is normal or expected, \textsc{synonym}: \textbf{premature}; \textbf{2.} happening at a time or in a situation that is not suitable, \textsc{synonym}: \textbf{ill-timed}, \textsc{opposite}: \textbf{timely}.} death in 1897.

At the D\'aniel Berzsenyi Gymnasium\footnote{\textbf{gymnasium} (plural \textbf{gymnasiums, gymnasia}) (\textit{formal}) a gym.}, Gy\"orgy studied Greek, Latin, \& German, in addition to Hungarian. It is surprising to learn that there he was seemingly uninterested in mathematics, his work in geometry deemed merely ``satisfactory'' compared with his ``outstanding'' performance in literature, geography, \& other subjects. His favorite subject, outside of literature, was biology.

He enrolled at the University of Budapest in 1905, initially studying law, which he soon dropped because he found it too boring. He then obtained the certification needed to teach Latin \& Hungarian at a gymnasium, a certification that he never used but of which he remained proud. Eventually his professor, Bern\'at Alexander, advised him that to help his studies in philosophy, he should take some mathematics \& physics courses. This was how he came to mathematics. Later, he joked that he ``wasn't good enough for physics, \& was too good for philosophy -- mathematics is in between.''

In Budapest he was taught physics by E\"otv\"os \& mathematics by Fej\'er \& was awarded a doctorate after spending the academic year 1910--11 in Vienna, where he took some courses by Wirtinger \& Mertens. He spent much of the next 2 years in G\"ottingen, where he met many more mathematicians -- Klein, Caratheodory, Hilbert, Runge, Landau, Weyl, Hecke, Courant, \& Toeplitz -- \& in 1914 visited Paris, where he became acquainted\footnote{\textbf{acquainted} [a] [not before noun] \textbf{1.} \textbf{acquainted with something} (\textit{formal}) familiar with something, having read, seen or experienced it; \textbf{2.} not close friends with somebody, but having met a few times before.} with Picard \& Hadamard \& learned that Hurwitz had arranged an appointment for him in Z\"urich. He accepted this position, writing later: ``I went to Z\"urich in order to be near Hurwitz, \& we were in close touch for about 6 years, from my arrival in Z\"urich in 1914 to his passing [in 1919]. I was very much impressed by him \& edited his works.''

Of course, the 1st World War took place during this period. It initially had little effect on Polya, who had been declared unfit for service in the Hungarian army as the result of a soccer wound. But later when the army, more desperately\footnote{\textbf{desperate} [a] \textbf{1.} feeling or showing that you have little hope \& are ready to do anything without worrying about danger to yourself or others; \textbf{2.} [usually before noun] (of an action) giving little hope of success; tried when everything else was failed; \textbf{3.} (of a situation) extremely serious or dangerous.} needing recruits\footnote{\textbf{recruit} [v] \textbf{1.} [transitive, intransitive] to find new people to join a company, an organization, the armed forces, etc.; \textbf{2.} [transitive] to get people to help with or be involved in something; \textbf{3.} [transitive] \textbf{recruit something (from something)} to form a new army, team, etc. by persuading new people to join it; [n] \textbf{1.} a person who has recently joined the armed forces or the police; \textbf{2.}  a person who joins an organization, a company, etc.}, demanded that he return to fight for his country, his strong pacifist\footnote{\textbf{pacifist} [a] [usually before noun] holding or showing the belief that war \& violence are always wrong; [n] a person who believes that war \& violence are always wrong \& refuses to fight in a war.} views led him to refuse. As a consequence, he was unable to visit Hungary for many years, \& in fact did not do so until 1967, 54 years after he left.

In the meantime, he had taken Swiss citizenship \& married a Swiss girl, Stella Vera Weber, in 1918. Between 1918 \& 1919, he published papers on a wide range of mathematical subjects, such as series, number theory, combinatorics, voting systems, astronomy, \& probability. He was made an extraordinary professor at the Z\"urich ETH in 1920, \& a few years later he \& G\'abor Szeg\H{o} published their book \textit{Aufgaben und Lehrsatze aus der Analysis} (``Problems \& Theorems in Analysis''), described by G. L. Alexanderson \& L. H. Lange in their obituary\footnote{\textbf{obituary} [n] (plural \textbf{obituaries}) an article about somebody's life \& achievements, that is printed in a newspaper soon after they have died.} of Polya as ``a mathematical masterpiece\footnote{\textbf{masterpiece} [n] (also \textbf{masterwork}) \textbf{1.} \textbf{masterpiece (of something)} a work of art such as a painting, film, book, etc. that is an excellent, or the best, example of the artist's work; \textbf{2.} \textbf{masterpiece of something} an extremely good example of something.} that assured\footnote{\textbf{assure} [v] \textbf{1.} to tell somebody that something is definitely true or is definitely going to happen, especially when they have doubts about it; \textbf{2.} to make something certain to happen; to make somebody\texttt{/}something certain to get something; \textbf{3.} to make yourself certain about something.} their reputations\footnote{\textbf{reputation} [n] the opinion that people have about what somebody\texttt{/}something is like, based on what has happened in the past.}.''

That book appeared in 1925, after Polya had obtained a Rockefeller Fellowship to work in England, where he collaborated with Hardy \& Littlewood on what later become their book \textit{Inequalities} (Cambridge University Press, 1936). He used a 2nd Rockefeller Fellowship to visit Princeton University in 1933, \& while in the United States was invited by H. F. Blichfeldt to visit Stanford University, which he greatly enjoyed, \& which ultimately became his home. Polya held a professorship at Stanford from 1943 until his retirement in 1953, \& it was there, in 1978, that he taught his last course, in combinatorics; he died on Sep 7, 1985, at the age of 97.

Some readers will want to know about Polya's many contributions to mathematics. Most of them relate to analysis \& are too technical to be understood by non-experts, but a few are worth mentioning.

In probability theory, Polya is responsible for the now-standard term ``Central Limit Theorem'' \& for proving that the Fourier transform of a probability measure is a characteristic function \& that a random walk on the integer lattice closes with probability 1 iff the dimension is at most 2.

In geometry, Polya independently re-enumerated the 17 plane crystallographic\footnote{\textbf{crystallography} [n] [uncountable] the branch of science that deals with crystals.} groups (their 1st enumeration\footnote{\textbf{enumeration} [n] [uncountable, countable] (\textit{formal}) the act of naming things 1 by 1 in a list; a list of this sort.}, by E. S. Fedorov, having been forgotten) \& together with P. Niggli devised\footnote{\textbf{devise} [v] \textbf{devise something} to plan or invent a procedure, system or method, especially one that is new or complicated, by using careful thought, \textsc{synonym}: \textbf{think something up}.} a notation for them.

In combinatorics, Polya's Enumeration Theorem is now a standard way of counting configurations according to their symmetry. It has been described by R. C. Read as ``a remarkable\footnote{\textbf{remarkable} [a] unusual or surprising in a way that causes people to take notice.} theorem in a remarkable paper, \& a landmark\footnote{\textbf{landmark} [n] \textbf{1.} something, such as a large building, that you can see clearly from a distance \& that will help you to know where you are; \textbf{2.} an event, a discovery or an invention that marks an important stage in something.} in the history of combinatorial analysis.''

\textit{How to Solve It} was written in German during Polya's time in Z\"urich, which ended up in 1940, when the European situation forced him to leave for the United States. Despite the book's eventual success, 4 publishers rejected the English version before Princeton University Press brought it out in 1945. In their hands, \textit{How to Solve It} rapidly became -- \& continues to be -- 1 of the most successful mathematical books of all time.'' -- \cite[Foreword, pp. xix--xxiv]{Polya2014}

\section*{Introduction}
``The following consideration are grouped around the preceding list of questions \& suggestions entitled\footnote{\textbf{entitle} [v] \textbf{1.} [often passive] to give somebody the right to have or to do something; \textbf{2.} [usually passive] to give a title to a book, document, film, etc.} ``How to Solve It.'' Any question or suggestion quoted from it will be printed in \textit{italics}, \& the whole list will be referred to simply as ``the list'' or as ``our list.''

The following pages will discuss the purpose of the list, illustrate its practical use by examples, \& explain the underlying notions \& mental operations. By way of preliminary explanation, this much may be said: If, using them properly, you address these questions \& suggestions to yourself, they may help you to solve your problem. If, using them properly, you address the same questions \& suggestions to 1 of your students, you may help him to solve his problem.

The book is divided into 4 parts.

The title of the 1st part is ``In the Classroom.'' It contains 20 sections. Each section will be quoted by its number in heavy type as, e.g., ``sect. \textbf{7.}'' Sects. \textbf{1}--\textbf{5} discuss the ``Purpose'' of our list in general terms. Sects. \textbf{6}--\textbf{17} explain what are the ``Main Divisions, Main Questions'' of the list, \& discuss a 1st practical example. Sects. \textbf{18}--\textbf{20} add ``More Examples.''

The title of the very short 2nd part is ``How to Solve It.'' It is written in dialogue; a somewhat idealized teacher answers short questions of a somewhat idealized student.

The 3rd \& most extensive part is a ``Short Dictionary of Heuristic''; we shall refer to it as the ``Dictionary.'' It contains 67 articles arranged alphabetically. E.g., the meaning of the term \textsc{heuristic} (set in small capitals) is explained in an article with this title on p. 112. When the title of such an article is referred to within the text it will be set in small capitals. Certain paragraphs of a few articles are more technical; they are enclosed\footnote{\textbf{enclose} [v] \textbf{1.} [usually passive] to build a wall, fence, etc. around something; \textbf{2.} \textbf{enclose something} (especially of a wall, fence, etc.) to surround something; \textbf{3.} \textbf{enclose something (with something)} to put something in the same envelope or package as something else.} in square brackets. Some articles are fairly closely connected with the 1st part to which they add further illustrations \& more specific comments. Other articles go somewhat beyond the aim of the 1st part of which they explain the background. There is a key-article on \textsc{modern heuristic}. It explains the connection of the main articles \& the plan underlying the Dictionary; it contains also directions how to find information about particular items of the list. It must be emphasized that there is a common plan \& a certain unity, because the articles of the Dictionary show the greatest outward variety. There are a few longer articles devoted to the systematic though condensed discussion of some general theme; others contain more specific comments; still others cross-references\footnote{\textbf{cross-reference} [v] \textbf{cross-reference something} to give cross references to another text or part of a text.}, or historical data, or quotations, or aphorisms\footnote{\textbf{aphorism} [n] (\textit{formal}) a short phrase that says something true or wise.}, or even jokes.

The Dictionary should not be read too quickly; its text is often condensed, \& now \& then somewhat subtle. The reader may refer to the Dictionary for information about particular points. If these points come from his experience with his own problems or his own students, the reading has a much better chance to be profitable\footnote{\textbf{profitable} [a] \textbf{1.} that makes or is likely to make money; \textbf{2.} that gives somebody an advantage or a useful result.}.

The title of the 4th part is ``Problems, Hints, Solutions.'' It proposes a few problems to the more ambitious reader. Each problem is followed (in proper distance) by a ``hint'' that may reveal a way to the result which is explained in the ``solution.''

We have mentioned repeatedly the ``student'' \& the ``teacher'' \& we shall refer to them again \& again. It may be good to observe that the ``student'' may be a high school student, or a college student, or anyone else who is studying mathematics. Also the ``teacher'' may be a high school teacher, or a college instructor, or anyone interested in the technique of teaching mathematics. The author looks at the situation sometimes from the point of view of the student \& sometimes from that of the teacher (the latter case is preponderant\footnote{\textbf{preponderant} [a] [usually before noun] (\textit{formal}) larger in number or more important than other people or things in a group.} in the 1st part). Yet most of the time (especially in the 3rd part) the point of view is that of a person who is neither teacher nor student but anxious to solve the problem before him.'' -- \cite[Introduction, pp. xxv--]{Polya2014}

\begin{center}
	\huge Part I. In The Classroom
\end{center}

\begin{center}
	\LARGE Purpose
\end{center}

\section{Helping the student}

\section{Questions, recommendations, mental operations}

\section{Generality}

\section{Common sense}

\section{Teacher \& student. Imitation \& practice}

\begin{center}
	\LARGE Main divisions, main questions
\end{center}

\section{4 phases}

\section{Understanding the problem}

\section{Example}

\section{Devising a plan}

\section{Example}

\section{Carrying out the plan}

\section{Example}

\section{Looking back}

\section{Example}

\section{Various approaches}

\section{The teacher's method of questioning}

\section{Good questions \& bad questions}

\begin{center}
	\LARGE More examples
\end{center}

\section{A problem of construction}

\section{A problem to prove}

\section{A rate problem}

\begin{center}
	\huge Part II. How to Solve It
\end{center}

\section*{A dialogue}

\begin{center}
	\huge Part III. Short Dictionary of Heuristic
\end{center}

\section*{Analogy}

\section*{Auxiliary elements}

\section*{Auxiliary problem}

\section*{Bolzano}

\section*{Bright idea}

\section*{Can you check the result?}

\section*{Can you derive the result differently?}

\section*{Can you use the result?}

\section*{Carrying out}

\section*{Condition}

\section*{Contradictory}

\section*{Corollary}

\section*{Could you derive something useful from the data?}

\section*{Could you restate the problem?}

\section*{Decomposing \& recombining}

\section*{Definition}

\section*{Descartes}

\section*{Determination, hope, success}

\section*{Diagnosis}

\section*{Did you see all the data?}

\section*{Do you know a related problem?}

\section*{Draw a figure}

\section*{Examine your guess}

\section*{Figures}

\section*{Generalization}

\section*{Have you seen it before?}

\section*{Here is a problem related to yours \& solved before}

\section*{Heuristic}

\section*{Heuristic reasoning}

\section*{If you cannot solve the proposed problem}

\section*{Induction \& mathematical induction}

\section*{Inventor's paradox}

\section*{Is it possible to satisfy the condition?}

\section*{Leibnitz}

\section*{Lemma}

\section*{Look at the unknown}

\section*{Modern heuristic}

\section*{Notation}

\section*{Pappus}

\section*{Pedantry \& mastery}

\section*{Practical problems}

\section*{Problems to find, problems to prove}

\section*{Progress \& achievement}

\section*{Puzzles}

\section*{Reductio \& absurdum \& indirect proof}

\section*{Redundant}

\section*{Routine problem}

\section*{Rules of discovery}

\section*{Rules of style}

\section*{Rules of teaching}

\section*{Separate the various parts of the condition}

\section*{Setting up equations}

\section*{Signs of progress}

\section*{Specialization}

\section*{Subconscious work}

\section*{Symmetry}

\section*{Terms, old \& new}

\section*{Test by dimension}

\section*{The future mathematician}

\section*{The intelligent problem-solver}

\section*{The intelligent reader}

\section*{The traditional mathematics professor}

\section*{Variation of the problem}

\section*{What is the unknown?}

\section*{Why proofs?}

\section*{Wisdom of proverbs}

\section*{Working backwards}

\begin{center}
	\huge Part IV. Problems, Hints, Solutions
\end{center}

\section*{Problems}

\section*{Hints}

\section*{Solutions}

%------------------------------------------------------------------------------%

%\selectlanguage{english}
%\begin{thebibliography}{99}
%	\bibitem[]{}
%\end{thebibliography}

%------------------------------------------------------------------------------%

\printbibliography[heading=bibintoc]
	
\end{document}