\documentclass[oneside]{book}
\usepackage[backend=biber,natbib=true,style=authoryear]{biblatex}
\addbibresource{/home/hong/1_NQBH/reference/bib.bib}
\usepackage[vietnamese,english]{babel}
\usepackage{tocloft}
\renewcommand{\cftsecleader}{\cftdotfill{\cftdotsep}}
\usepackage[colorlinks=true,linkcolor=blue,urlcolor=red,citecolor=magenta]{hyperref}
\usepackage{amsmath,amssymb,amsthm,mathtools,float,graphicx}
\allowdisplaybreaks
\numberwithin{equation}{section}
\newtheorem{assumption}{Assumption}[chapter]
\newtheorem{conjecture}{Conjecture}[chapter]
\newtheorem{corollary}{Corollary}[chapter]
\newtheorem{definition}{Definition}[chapter]
\newtheorem{example}{Example}[chapter]
\newtheorem{lemma}{Lemma}[chapter]
\newtheorem{notation}{Notation}[chapter]
\newtheorem{principle}{Principle}[chapter]
\newtheorem{problem}{Problem}[chapter]
\newtheorem{proposition}{Proposition}[chapter]
\newtheorem{question}{Question}[chapter]
\newtheorem{remark}{Remark}[chapter]
\newtheorem{theorem}{Theorem}[chapter]
\usepackage[left=0.5in,right=0.5in,top=1.5cm,bottom=1.5cm]{geometry}
\usepackage{fancyhdr}
\pagestyle{fancy}
\fancyhf{}
\lhead{\small \textsc{Sect.} ~\thesection}
\rhead{\small \nouppercase{\leftmark}}
\renewcommand{\sectionmark}[1]{\markboth{#1}{}}
\cfoot{\thepage}
\def\labelitemii{$\circ$}

\title{Elementary Mathematics}
\author{\selectlanguage{vietnamese} Nguyễn Quản Bá Hồng\footnote{Independent Researcher, Ben Tre City, Vietnam\\e-mail: \texttt{nguyenquanbahong@gmail.com}}}
\date{\today}

\begin{document}
\maketitle
\setcounter{secnumdepth}{4}
\setcounter{tocdepth}{4}
\tableofcontents

%------------------------------------------------------------------------------%

\chapter{Wikipedia's}

\section{\href{https://en.wikipedia.org/wiki/How_to_Solve_It}{Wikipedia\texttt{/}How to Solve It}}
``\textit{How to Solve It} (1945) is a small volume by mathematician \href{https://en.wikipedia.org/wiki/George_P%C3%B3lya}{George P\'olya} describing methods of \href{https://en.wikipedia.org/wiki/Problem_solving}{problem solving}.'' -- \href{https://en.wikipedia.org/wiki/How_to_Solve_It}{Wikipedia\texttt{/}how to solve it}

\subsection{4 principles}
``\textit{How to Solve It} suggests the following steps when solving a \href{https://en.wikipedia.org/wiki/Mathematical_problem}{mathematical problem}:
\begin{enumerate}
	\item 1st, you have to \textit{understand the problem}.
	\item After understanding, \textit{make a plan}.
	\item \textit{Carry out the plan}.
	\item \textit{Look back} on your work. How could it be better?
\end{enumerate}
If this technique fails, P\'olya advises: ``If you can't solve a problem, then there is an easier problem you can solve: find it.'' Or: ``If you cannot solve the proposed problem, try to solve 1st some related problem. Could you imagine a more accessible related problem?'''' -- \href{https://en.wikipedia.org/wiki/How_to_Solve_It#Four_principles}{Wikipedia\texttt{/}how to solve it\texttt{/}4 principles}

\subsubsection{1st principle: Understand the problem}
``Understanding the problem'' is often neglected as being obvious \& is not even mentioned in many mathematics classes. Yet students are often stymied in their efforts to solve it, simply because they don't understand it fully, or even in part. In order to remedy this oversight, P\'olya taught teachers how to prompt each student with appropriate questions, depending on the situation, such as:
\begin{itemize}
	\item What are you asked to find or show?
	\item Can you restate the problem in your own words?
	\item Can you think of a picture of a diagram that might help you understand the problem?
	\item Is there enough information to enable you to find a solution?
	\item Do you understand all the words used in stating the problem?
	\item Do you need to ask a question to get the answer?
\end{itemize}
The teacher is to select the question with the appropriate level of difficulty for each student to ascertain if each student understands at their own level, moving up or down the list to prompt each student, until each one can respond with something constructive.'' -- \href{https://en.wikipedia.org/wiki/How_to_Solve_It#First_principle:_Understand_the_problem}{Wikipedia\texttt{/}how to solve it\texttt{/}4 principles\texttt{/}1st principle: understand the problem}

\subsubsection{2nd principle: Devise a plan}
``P\'olya mentions that there are many reasonable ways to solve problems. The skill at choosing an appropriate strategy is best learned by solving many problems. You will find choosing a strategy increasingly easy. A partial list of strategies is included:
\begin{itemize}
	\item Guess \& check
	\item Make an orderly list
	\item Eliminate possibilities
	\item Use symmetry
	\item Consider special cases
	\item Use direct reasoning
	\item Solve an equation
\end{itemize}
Also suggested:
\begin{itemize}
	\item Look for a pattern
	\item Draw a picture
	\item Solve a simpler problem
	\item Use a model
	\item Work backward
	\item Use a formula
	\item Be creative
	\item Applying these rules to devise a plan takes your own skill \& judgment.
\end{itemize}
P\'olya lays a big emphasis on the teachers' behavior. A teacher should support students with devising their own plan with a question method that goes from the most general questions to more particular questions, with the goal that the last step to having a plan is made by the student. He maintains that just showing students a plan, no matter how good it is, does not help them.'' -- \href{https://en.wikipedia.org/wiki/How_to_Solve_It#Second_principle:_Devise_a_plan}{Wikipedia\texttt{/}how to solve it\texttt{/}4 principles\texttt{/}2nd principle: devise a plan}

\subsubsection{3rd principle: Carry out the plan}
``This step is usually easier than devising the plan. In general, all you need is care \& patience, given that you have the necessary skills. Persist with the plan that you have chosen. If it continues not to work, discard it \& choose another. Don't be misled; this is how mathematics is done, even by professionals.'' -- \href{https://en.wikipedia.org/wiki/How_to_Solve_It#Third_principle:_Carry_out_the_plan}{Wikipedia\texttt{/}how to solve it\texttt{/}4 principles\texttt{/}3rd principle: carry out the plan}

\subsubsection{4th principle: Review\texttt{/}extend}
``P\'olya mentions that much can be gained by taking the time to reflect \& look back at what you have done, what worked \& what did not, \& with thinking about other problems where this could be useful. Doing this will enable you to predict what strategy to use to solve future problems, if these relate to the original problem.'' -- \href{https://en.wikipedia.org/wiki/How_to_Solve_It#Fourth_principle:_Review/extend}{Wikipedia\texttt{/}how to solve it\texttt{/}4 principles\texttt{/}4th principle: review\texttt{/}extend}

\subsection{Heuristics}
``The book contains a dictionary-style set of \href{https://en.wikipedia.org/wiki/Heuristics}{heuristics}, many of which have to do with generating a more accessible problem. E.g.:

\textbf{Heuristic $|$ Informal Description $|$ Formal analogue}
\begin{itemize}
	\item \href{https://en.wikipedia.org/wiki/Analogy}{Analogy} $|$ Can you find a problem analogous to your problem \& solve that? $|$ \href{https://en.wikipedia.org/wiki/Map_(mathematics)}{map}
	\item Auxiliary Elements $|$ Can you add some new element to your problem to get closer to a solution? $|$ \href{https://en.wikipedia.org/wiki/Extension_(predicate_logic)}{Extension}
	\item \href{https://en.wikipedia.org/wiki/Generalization}{Generalization} $|$ Can you find a problem more general than your problem? $|$ \href{https://en.wikipedia.org/wiki/Generalization}{Generalization}
	\item \href{https://en.wikipedia.org/wiki/Induction_(philosophy)}{Induction} $|$ Can you solve your problem by deriving a generalization from some examples? $|$ \href{https://en.wikipedia.org/wiki/Induction_(philosophy)}{Induction}
	\item Variation of the Problem $|$ Can you vary or change your problem to create a new problem (or set of problems) whose solution(s) will help you solve your original problem? $|$ \href{https://en.wikipedia.org/wiki/Search_algorithm}{Search}
	\item Auxiliary Problem $|$ Can you find a subproblem or side problem whose solution will help you solve your problem? $|$ \href{https://en.wikipedia.org/wiki/Subgoal}{Subgoal}
	\item Here is a problem related to yours \& solved before $|$ Can you find a problem related to yours that has already been solved \& use that to solve your problem? $|$ \href{https://en.wikipedia.org/wiki/Pattern_recognition}{Pattern recognization}, \href{https://en.wikipedia.org/wiki/Pattern_matching}{Pattern matching}, \href{https://en.wikipedia.org/wiki/Reduction_(complexity)}{Reduction}
	\item \href{https://en.wikipedia.org/wiki/Special_case}{Specialization} $|$ Can you find a problem more specialized? $|$ \href{https://en.wikipedia.org/wiki/Special_case}{Specialization}
	\item \href{https://en.wikipedia.org/wiki/Decomposition_(computer_science)}{Decomposing} \& Recombining $|$ Can you decompose the problem \& ``recombine its elements in some new manner''? $|$ \href{https://en.wikipedia.org/wiki/Divide_and_conquer_algorithm}{Divide \& conquer}
	\item \href{https://en.wikipedia.org/wiki/Working_backward_from_the_goal}{Working backward} $|$ Can you start with the goal \& work backwards to something you already know? $|$ \href{https://en.wikipedia.org/wiki/Backward_chaining}{Backward chaining}
	\item Draw a Figure $|$ Can you draw a picture of the problem? $|$ \href{https://en.wikipedia.org/wiki/Diagrammatic_Reasoning}{Diagrammatic Reasoning}
\end{itemize}
'' -- \href{https://en.wikipedia.org/wiki/How_to_Solve_It#Heuristics}{Wikipedia\texttt{/}how to solve it\texttt{/}heuristics}

\subsection{Influence}
\begin{itemize}
	\item ``The book has been translated into several languages \& has sold over a million copies, \& has been continuously in print since its 1st publication.
	\item \href{https://en.wikipedia.org/wiki/Marvin_Minsky}{Marvin Minsky} said in his paper \textit{Steps Toward Artificial Intelligence} that ``everyone should know the work of George P\'olya on how to solve problems.''
	\item P\'olya's book has had a large influence on mathematics textbooks as evidenced by the bibliographies for \href{https://en.wikipedia.org/wiki/Mathematics_education}{mathematics education}.
	\item Russian inventor \href{https://en.wikipedia.org/wiki/Genrich_Altshuller}{Genrich Altshuller} developed an elaborate set of methods for problem solving known as \href{https://en.wikipedia.org/wiki/TRIZ}{TRIZ}, which in many aspects reproduces or parallels P\'olya's work.
	\item \href{https://en.wikipedia.org/wiki/How_to_Solve_it_by_Computer}{How to Solve it by Computer} is a computer science book by R. G. Dromey. It was inspired by P\'olya's work.'' -- \href{https://en.wikipedia.org/wiki/How_to_Solve_It#Influence}{Wikipedia\texttt{/}how to solve it\texttt{/}influence}
\end{itemize}

%------------------------------------------------------------------------------%

\chapter{\cite{Andreescu_Dospinescu2010}. Problems from the Book}

\begin{quotation}
	\textit{``Math isn't the art of answering mathematical questions, it is the art of asking the right questions, the questions that give you insight, the ones that lead you in interesting directions, the ones that connect with lots of other interesting questions -- the ones with beautiful answers.''} -- G. Chaitin
\end{quotation}

\section*{Preface}
``What can a new book of problems in elementary mathematics possibly contribute to the vast existing collection of journals, articles, \& books? This was our main concern when we decided to write this book. The inevitability\footnote{\textbf{inevitability} [n] [uncountable, countable] (plural \textbf{inevitabilities}) the fact that something cannot be avoided or prevented.} of this question does not facilitate\footnote{\textbf{facilitate} [v] (\textit{formal}) \textbf{facilitate something} to make an action or a process possible or easier.} the answer, because after 5 years of writing \& rewriting we still had something to add. It could be a new problem, a comment we considered pertinent\footnote{\textbf{pertinent} [a] appropriate to a particular solution, \textsc{synonym}: \textbf{relevant}.}, or a solution that escaped our rationale\footnote{\textbf{rationale} [n] the principles or reasons that explain a particular decision, course of action or belief.} until this predictive\footnote{\textbf{predictive} [a] connected with the ability to show what will happen in the future.} moment, when we were supposed to bring it under the scrutiny\footnote{\textbf{scrutiny} [n] [uncountable] careful \& thorough examination, \textsc{synonym}: \textbf{inspection}.} of a specialist in the field.

A mere perusal\footnote{\textbf{perusal} [n] [uncountable, singular] (\textit{formal or humorous}) the act of reading something; especially in a careful way.} of this book should be efficient to identify its target audience: students \& coaches preparing for mathematical Olympiads, national or international. It takes more effort to realize that these are not the only potential beneficiaries\footnote{\textbf{beneficiary} [n] (plural \textbf{beneficiaries}) \textbf{beneficiary (of something)} a person who gains a result of something.} of this work. While the book is rife\footnote{\textbf{rife} [a] [not before noun] \textbf{1.} if something bad or unpleasant is \textbf{rife} in a place, it is very common there, \textsc{synonym}: \textbf{widespread}; \textbf{2.} \textbf{rife (with something)} full of something bad or unpleasant.} with problems collected from various mathematical competitions \& journals, one cannot neglect the classical results of mathematics, which naturally exceed the level of time-constrained competitions. \& no, \fbox{classical does not mean easy!} These mathematical beauties are more than just proof that elementary mathematics can produce jewels. They serve as an invitation to mathematics beyond competitions, regarded by many to be the ``true mathematics''. In this context, the audience is more diverse than one might think.

Even so, as it will be easily discovered, many of the problems in this book are very difficult. Thus, the theoretical portions are short, while the emphasis is squarely placed on the problems. Certainly, more subtle results like quadratic reciprocity \& existence of primitive roots are related to the basic results in linear algebra or mathematical analysis. Whenever their proofs are particularly useful, they are provided. We will assume of the reader a certain familiarity with classical theorems of elementary mathematics, which we will use freely. The selection of problems was made by weighing the need for routine exercises that engender\footnote{\textbf{engender} [v] \textbf{engender something} to make a feeling or situation exist.} familiarity with the joy of the difficult problems in which we find the truly beautiful ideas. We stroke to select only those problems, easy \& hard, that best illustrate the ideas we wanted to exhibit.

Allow us to discuss in brief the structure of the book. What will most likely surprise the reader when browsing just the table of contents is the absence of any chapters on geometry. This book was not intended to be an exhaustive treatment of elementary mathematics; if even such a book appears, it will be \fbox{a sad day for mathematics}. Rather, we tried to assemble\footnote{\textbf{assemble} [v] \textbf{1.} [intransitive, transitive] to come together as a group; to bring people or things together as a group; \textbf{2.} [transitive] to fit together all the separate parts of something; \textbf{3.} [transitive] \textbf{assemble something} (\textit{computing}) to change instructions that a human can read in an assembly language program into a code that a computer can understand \& act on.} problems that enchanted\footnote{\textbf{enchant} [v] \textbf{1.} \textbf{enchant somebody} (\textit{formal}) to attract somebody strongly \& make them feel very interested, excited, etc., \textsc{synonym}: \textbf{delight}; \textbf{2.} \textbf{enchant somebody\texttt{/}something} to place somebody\texttt{/}something under a magic spell ($=$ magic words that have special powers), \textsc{synonym}: \textbf{bewitch}.} us in order to give a sense of techniques \& ideas that become leitmotifs\footnote{\textbf{leitmotif} [n] (also \textbf{leitmotiv}) (\textit{from German}) \textbf{1.} (music) a short tune in a piece of music that is often repeated \& is connected with a particular person, thing or idea; \textbf{2.} an idea or a phrase that is repeated often in a book or work of art, or is typical of a particular person or group.} not just in problem solving but in all of mathematics.

Furthermore, there are excellent books on geometry, \& it was not hard to realize that it would be beyond our ability to create something new to add to this area of study. Thus, we preferred to elaborate\footnote{\textbf{elaborate} [v] [usually before noun] very complicated \& details; carefully prepared \& organized; [v] \textbf{1.} [intransitive, transitive] to explain or describe something in a more detailed way; \textbf{2.} [transitive] \textbf{elaborate something} to develop a plan, an idea, etc. \& make it complicated or detailed; \textbf{3.} [transitive] (\textit{biology}) (of a natural process) to produce a substance or structure from its elements or simpler constituents.} more on 3 important fields of elementary mathematics: algebra, number theory, \& combinatorics. Even after this narrowing of focus there are many topics that are simply left out, either in consideration of the available space or else because of the fine existing literature on the subject. This is, e.g., the fate\footnote{\textbf{fate} [n] \textbf{1.} [countable] \textbf{fate (of somebody\texttt{/}something)} the things, especially bad things, that will happen or have happened to somebody\texttt{/}something; \textbf{2.} [uncountable] the power that is believed to control everything that happens \& that cannot be stopped or changed.} of functional equations, a field which can spawn\footnote{\textbf{spawn} [v] \textbf{1.} [intransitive, transitive] (of fish \& some other creatures) to lay eggs; \textbf{2.} [transitive] to cause something to develop or be produced in large numbers.} extremely difficult, intriguing\footnote{\textbf{intriguing} [a] very interesting because of being unusual or not having an obvious answer.} problems, but one which does not have obvious recurring\footnote{\textbf{recur} [v] [intransitive] to happen again or a number of times..} themes that tie everything together.

Hoping that you have not abandoned the book because of these omissions, which might be considered major by many who do not keep in mind the stated objectives, we continue by elaborating on the contents of the chapters. To start out, we ordered the chapters in ascending order of difficulty of the mathematical tools used. Thus, the exposition starts out lightly with some classical substitution techniques in algebra, emphasizing a large number of examples \& applications. These are followed by a topic dear to us: the Cauchy--Schwarz inequality \& its variations. A sizable chapter presents applications of the Lagrange interpolation formula, which is known by most only through r\^ote, straightforward applications. The interested reader will find some genuine\footnote{\textbf{genuine} [a] \textbf{1.} real; exactly what it appears\texttt{/}they appear to be; \textbf{2.} honest; that can be trusted, \textsc{synonym}: \textbf{sincere}.} pearls in this chapter, which should be enough to change his or her opinion about this useful mathematical tool. 2 rather difficult chapters, in which mathematical analysis mixes with algebra, are given at the end of the book. 1 of them is quite original, showing how simple consideration of integral calculus can solve very difficult inequalities. The other discusses properties of equidistribution \& dense numerical series. Too many books consider the Weyl equidistribution theorem to be ``much too difficult'' to include, \& we cannot resist contradicting them by presenting an elementary proof. Furthermore, the reader will quickly realize that for elementary problems we have not shied away from presenting the so-called \textit{non-elementary solutions} which use mathematical analysis or advanced algebra. It would be a crime to consider these 2 types of mathematics as 2 different entities, \& it would be even worse to present laborious elementary solutions without admitting the possibility of generalization for problems that have conceptual \& easy non-elementary solutions. In the end we devote a whole chapter to discussing criteria for polynomial irreducibility. We observe that some extremely efficient criteria (like those of Peron \& Capelli) are virtually unknown, even though they are more efficient than the well-known \textit{Eisenstein criterion}.

The section dedicated to number theory is the largest. Some introductory chapters related to prime numbers of the form $4k + 3$ \& to the other of an element are included to provide a better understanding of fundamental results which are used later in the book. A large chapter develops a tool which is as simple as it is useful: the exponent of a prime in the factorization of an integer. Some mathematical diamonds belonging to Paul Erd\H{o}s \& others appear within. \& even though quadratic reciprocity is brought up in many books, we included an entire chapter on this topic because the problems available to us were too ingenious\footnote{\textbf{ingenious} [a] \textbf{1.} (of an object, a plan, an idea, etc.) very suitable for a particular purpose \& resulting from clever new ideas; \textbf{2.} (of a person) having a lot of clever new ideas \& good at inventing things.} to exclude. Next come some difficult chapters concerning arithmetic properties of polynomials, the geometry of numbers (in which we present some arithmetic application of the famous Minkowski's theorem), \& the properties of algebraic numbers. A special chapter studies some applications of the extremely simple idea that a convergent series of integers is eventually stationary! The reader will have the chance to realize that \fbox{in mathematics even simple ideas have great impact}: consider, e.g., the fundamental idea that in the interval $(-1,1)$ the only integer is $0$. But how many fantastic results concerning irrational numbers follow simply from that! Another chapter dear to us concerns the sum of digits, a subject that always yields unexpected \& fascinating problems, but for which we could not find a unique approach.

Finally, some words about the combinatorics section. The reader will immediately observe that our presentation of this topic takes an algebraic slant\footnote{\textbf{slant} [v] \textbf{1.} [intransitive, transitive] to slope or to make something slope in a particular direction or at a particular angle; \textbf{2.} [transitive] \textbf{slant something ($+$ adv.\texttt{/}prep.)} (\textit{sometimes disapproving}) to present information based on a particular way of thinking, especially in an unfair way; [n] \textbf{1.} a sloping position; \textbf{2.} \textbf{slant (on something\texttt{/}somebody)} a way of thinking about something, especially one that shows support for a particular opinion or point of view.}, which was, in fact, our intention. In this way we tried to present some unexpected applications of complex numbers in combinatorics, \& a whole chapter is dedicated to useful formal series. Another chapter shows how useful linear algebra can be when solving problems on set combinatorics. Of course, we are traditional in presenting applications of Turan's theorem \& of graph theory in general, \& the pigeonhole principle could not be omitted. We faced difficulties here, because this topic is covered extensively in other books, though rarely in a satisfying way. For this reason, we tried to present lesser-known problems, because this topic is so dear to elementary mathematics lovers. At the end, we included a chapter on special applications of polynomials in number theory \& combinatorics, emphasizing the Combinatorial Nullstellensatz, a recent \& extremely useful theorem by Noga Alon.

We end our description with some remarks on the structure of the chapters. In general, the main theoretical results are stated, \& if they are sufficiently profound or obscure, a proof is given. Following the theoretical part, we present between 10 \& 15  examples, most from mathematical contests or from journals such as Kvant, Komal, \& American Mathematical Monthly. Others are new problems or classical results. Each chapter ends with a series of problems, the majority of which stem from the theoretical results.

Finally, a change that will please some \& scare others: the end-of-chapter problems do not have solutions! We had several reasons for this. The 1st \& most practical consideration was minimizing the mass of the book. But the 2nd \& more important factor was this: we consider solving problems to necessarily include the inevitably lengthy process of trial \& research to which the inclusion of solutions provides perhaps too tempting of a shortcut. Keeping this in mind, the selection of the problems was made with the goal that the diligent\footnote{\textbf{diligent} [a] (\textit{formal}) showing care \& effort in your work or duties.} reader could solve about $\frac{1}{3}$ of them, make some progress in the $\frac{2}{3}$ \& have at least the satisfaction of looking for a solution in the remainder.'' -- \cite[Preface, pp. vii--xi]{Andreescu_Dospinescu2010}

%------------------------------------------------------------------------------%

\section{Some Useful Substitutions}

\subsection{Theory \& examples}

\subsection{Practice problems}

%------------------------------------------------------------------------------%

\section{Always Cauchy--Schwarz $\ldots$}

\subsection{Theory \& examples}

\subsection{Practice problems}

%------------------------------------------------------------------------------%

\section{Look at the Exponent}

\subsection{Theory \& examples}

\subsection{Practice problems}

%------------------------------------------------------------------------------%

\section{Primes \& Squares}

\subsection{Theory \& examples}

\subsection{Practice problems}

%------------------------------------------------------------------------------%

\section{T2's Lemma}

\subsection{Theory \& examples}

\subsection{Practice problems}

%------------------------------------------------------------------------------%

\section{Some Classical Problems in Extremal Graph Theory}

\subsection{Theory \& examples}

\subsection{Practice problems}

%------------------------------------------------------------------------------%

\section{Complex Combinatorics}

\subsection{Theory \& examples}

\subsection{Practice problems}

%------------------------------------------------------------------------------%

\section{Formal Series Revisited}

\subsection{Theory \& examples}

\subsection{Practice problems}

%------------------------------------------------------------------------------%

\section{A Brief Introduction to Algebraic Number Theory}

\subsection{Theory \& examples}

\subsection{Practice problems}

%------------------------------------------------------------------------------%

\section{Arithmetic Properties of Polynomials}

\subsection{Theory \& examples}

\subsection{Practice problems}

%------------------------------------------------------------------------------%

\section{Lagrange Interpolation Formula}

\subsection{Theory \& examples}

\subsection{Practice problems}

%------------------------------------------------------------------------------%

\section{Higher Algebra in Combinatorics}

\subsection{Theory \& examples}

\subsection{Practice problems}

%------------------------------------------------------------------------------%

\section{Geometry \& Numbers}

\subsection{Theory \& examples}

\subsection{Practice problems}

%------------------------------------------------------------------------------%

\section{The Smaller, the Better}

\subsection{Theory \& examples}

\subsection{Practice problems}

%------------------------------------------------------------------------------%

\section{Density \& Regular Distribution}

\subsection{Theory \& examples}

\subsection{Practice problems}

%------------------------------------------------------------------------------%

\section{The Digit Sum of a Positive Integer}

\subsection{Theory \& examples}

\subsection{Practice problems}

%------------------------------------------------------------------------------%

\section{At the Border of Analysis \& Number Theory}

\subsection{Theory \& examples}

\subsection{Practice problems}

%------------------------------------------------------------------------------%

\section{Quadratic Reciprocity}

\subsection{Theory \& examples}

\subsection{Practice problems}

%------------------------------------------------------------------------------%

\section{Solving Elementary Inequalities Using Integrals}

\subsection{Theory \& examples}

\subsection{Practice problems}

%------------------------------------------------------------------------------%

\section{Pigeonhole Principle Revisited}

\subsection{Theory \& examples}

\subsection{Practice problems}

%------------------------------------------------------------------------------%

\section{Some Useful Irreducibility Criteria}

\subsection{Theory \& examples}

\subsection{Practice problems}

%------------------------------------------------------------------------------%

\section{Cycles, Paths, \& Other Ways}

\subsection{Theory \& examples}

\subsection{Practice problems}

%------------------------------------------------------------------------------%

\section{Some Special Applications of Polynomials}

\subsection{Theory \& examples}

\subsection{Practice problems}

%------------------------------------------------------------------------------%

\chapter{\cite{Polya2014}. How to Solve It: A New Aspect of Mathematical Methods}

\section*{From the Preface to the 1st Printing}
``A great discovery solves a great problem but there is a grain\footnote{\textbf{grain} [n] \textbf{1.} [uncountable, countable] the small hard seeds of food plants such as wheat, rice, etc.; a single seed of such a plant; \textbf{2.} [countable] \textbf{grain (of something)} a small piece of a particular substance; usually a hard substance; \textbf{3.} [countable, usually singular] \textbf{grain of something} a very small amount; \textbf{4.} [countable] an individual particle or crystal in metal, rock, etc., usually explained with a lens or microscope.} of discovery in the solution of any problem. Your problem may be modest\footnote{\textbf{modest} [a] \textbf{1.} fairly limited or small in amount; \textbf{2.} not expensive, rich or impressive; \textbf{3.} (of people, especially women, or their clothes) not showing too much of the body; not intended to attract attention, especially in a sexual way; \textbf{4.} (\textit{approving}) not talking much about your own abilities or possessions.}; but it challenges your curiosity\footnote{\textbf{curiosity} [n] (plural \textbf{curiosities}) \textbf{1.} [uncountable, singular] a strong desire to know about something; \textbf{2.} [countable] \textbf{curiosity (of something)} an unusual \& interesting thing.} \& brings into play your inventive\footnote{\textbf{inventive} [a] \textbf{1.} (especially of people) able to create or design new things or think of new ideas; \textbf{2.} (of ideas) new \& interesting.} faculties\footnote{\textbf{faculty} [n] (plural \textbf{faculties}) \textbf{1.} [countable] a physical or mental ability, especially one that people are born with; \textbf{2.} [countable] \textbf{faculty (of something)} a department or group of related departments in a college or university; \textbf{3.} [countable $+$ singular or plural verb] all the teachers in a faculty of a college or university; \textbf{4.} [countable, uncountable] (\textit{NAE}) all the teachers of a particular university or college.}, \& if you solve it by your own means, you may experience the tension\footnote{\textbf{tension} [n] \textbf{1.} [uncountable, countable, usually plural] a situation in which people do not trust each other, or feel unfriendly towards each other, \& which may cause them to attack each other; \textbf{2.} [countable, uncountable] \textbf{tension (between A \& B)} a situation in which the fact that there are different needs or interests causes difficulties; \textbf{3.} [uncountable] a feeling of anxiety \& stress that makes it impossible to relax; \textbf{4.} [uncountable] the feeling of fear \& excitement that is created by a writer or a film director; \textbf{5.} [uncountable] the state of being stretched tight; the extent to which something is stretched tight.} \& enjoy the triumph\footnote{\textbf{triumph} [n] \textbf{1.} [countable, uncountable] a great success, achievement or victory; \textbf{2.} [uncountable] the state of having achieved a great success or victory; the feeling of happiness that you get from this; [v] [intransitive] to defeat somebody\texttt{/}something; to be successful.} of discovery. Such experiences at a susceptible\footnote{\textbf{susceptible} [a] \textbf{1.} [not usually before noun] \textbf{susceptible (to somebody\texttt{/}something)} very likely to be influenced, harmed or affected by somebody\texttt{/}something; \textbf{2.} \textbf{susceptible (of something)} (\textit{formal}) allowing something; capable of something.} age may create a taste for mental work \& leave their imprint\footnote{\textbf{imprint} [v] [often passive] \textbf{1.} to have a great effect on something so that it cannot be forgotten, changed, etc.; \textbf{2.} to print or press a mark or design onto a surface; [n] \textbf{1.} \textbf{imprint (of something) (in\texttt{/}on something)} a mark made by pressing something onto a surface; \textbf{2.} [usually singular] \textbf{imprint (of something) (on somebody\texttt{/}something)} (\textit{formal}) the lasting effect that a person or an experience has on a place or a situation; \textbf{3.} (\textit{specialist}) the name of the publisher of a book, usually printed below the title on the 1st page; a brand name under which books are published.} on mind \& character for a lifetime\footnote{\textbf{lifetime} [n] the length of time that somebody lives or that something lasts.}.

Thus, a teacher of mathematics has a great opportunity. If he fills his allotted\footnote{\textbf{allot} [v] to give time, money, tasks, etc. to somebody\texttt{/}something as a share of what is available, \textsc{synonym}: \textbf{allocate}.} time with drilling his students in routine operations he kills their interest, hampers\footnote{\textbf{hamper} [v] [often passive] to prevent something from being achieved easily or happening normally; to prevent somebody from easily doing something, \textsc{synonym}: \textbf{hinder, impede}.} their intellectual development, \& misuses his opportunity. But if he challenges the curiosity of his students by setting them problems proportionate\footnote{\textbf{proportionate} [a] increasing or decreasing in size, amount or degree according to changes in something else, \textsc{synonym}: \textbf{proportional}.} to their knowledge, \& helps them to solve their problems with stimulating\footnote{\textbf{stimulating} [a] \textbf{1.} full of interesting or exciting ideas; making people feel enthusiastic; \textbf{2.} making you feel more active \& healthy.} questions, he may give them a taste for, \& some means of, independent thinking.

Also a student whose college curriculum\footnote{\textbf{curriculum} [n] (plural \textbf{curricula}) the subjects that are included in a course of study or taught in a school, college or university.} includes some mathematics has a singular\footnote{\textbf{singular} [n] [singular] (\textit{grammar}) a form of a noun or verb that refers to 1 person or thing; [a] \textbf{1.} (\textit{grammar}) connected with or having the form of a noun or verb that refers to 1 person or thing; \textbf{2.} especially great or obvious, \textsc{synonym}: \textbf{outstanding}; \textbf{3.} (\textit{mathematics, physics}) connected with a singularity.} opportunity. This opportunity is lost, of course, if he regards\footnote{\textbf{regard} [v] [often passive] to think about somebody\texttt{/}something in a particular way; \textbf{as regards somebody\texttt{/}something} [idiom] concerning or in connection with somebody\texttt{/}something; [n] \textbf{1.} [uncountable] attention to or thought \& care for somebody\texttt{/}something; \textbf{2.} [uncountable] \textbf{regard (for somebody\texttt{/}something)} respect or admiration for somebody\texttt{/}something. If you \textbf{hold somebody in high regard}, you have a good opinion of them.; \textbf{3.} (\textbf{regards}) [plural] used to send good wishes to somebody at the end of a letter or email; \textbf{have regard to something} [idiom] (\textit{law}) to remember \& think carefully about something; \textbf{in\texttt{/}with regard to somebody\texttt{/}something} [idiom] concerning somebody\texttt{/}something; \textbf{in this\texttt{/}that regard} [idiom] concerning what has just been mentioned.} mathematics as a subject in which he has to earn so \& so much credit \& which he should forget after the final examination as quickly as possible. The opportunity may be lost even if the student has some natural talent for mathematics because he, as everybody else, must discover his talents \& tastes; he cannot know that he likes raspberry pie if he has never tasted raspberry pie. He may manage to find out, however, that a mathematics problem may be as much fun as a crossword puzzle\footnote{\textbf{crossword} [n] (also \textbf{crossword puzzle}) a game in which you have to fit words across \& downwards into spaces with numbers in a square diagram. You find the words by solving clues.}, or that vigorous\footnote{\textbf{vigorous} [a] \textbf{1.} involving physical strength, effort or energy; \textbf{2.} done with determination, energy or enthusiasm; \textbf{3.} strong \& healthy.} mental work may be an exercise as desirable as a fast game of tennis. Having tasted the pleasure in mathematics he will not forget it easily \& then there is a good chance that mathematics will become something for him: a hobby, or a tool of his profession, or a great ambition,

The author remembers the time when he was a student himself, a somewhat ambitious student, eager to understand a little mathematics \& physics. He listened to lectures, read books, tried to take in the solutions \& facts presented, but there was a question that disturbed\footnote{\textbf{disturb} [v] \textbf{1.} \textbf{disturb something} to change the arrangement of something, or affect how something functions; \textbf{2.} \textbf{disturb somebody\texttt{/}something} to interrupt somebody \& prevent them from continuing with what they are doing; \textbf{3.} \textbf{disturb somebody} to make somebody feel anxious or upset.} him again \& again: ``Yes, the solution seems to work, it appears to be correct; but how is it possible to invent such a solution? Yes, this experiment seems to work, this appears to be a fact; but how can people discover such facts? \& how could I invent or discover such things by myself?'' Today the author is teaching mathematics in a university; he thinks or hopes that some of his more eager students ask similar questions \& he tries to satisfy their curiosity. Trying to understand not only the solution of this or that problem but also the motives \& procedures of the solution, \& trying to explain these motives \& procedures to others, he was finally led to write the present book. He hopes that it will be useful to teachers who wish to develop their students' ability to solve problems, \& to students who are keen on developing their own abilities.

Although the present book pays special attention to the requirements of students \& teachers of mathematics, it should interest anybody concerned with the ways \& means of invention \& discovery. Such interest may be more widespread\footnote{\textbf{widespread} [a] existing or happening over a large area or among many people, \textsc{synonym}: \textbf{extensive}.} than one would assume without reflection\footnote{\textbf{reflection} [n] \textbf{1.} [countable] \textbf{reflection of something} an account or description of what somebody\texttt{/}something is like; a thing that is a result of something else; \textbf{2.} [uncountable] careful thought about something, especially your work or studies; \textbf{3.} [countable, usually plural] \textbf{reflections (on something)} written or spoken thoughts about a particular subject; \textbf{4.} [uncountable] \textbf{reflection (of something)} the action or process of sending back light, heat, sound, etc. from a surface; \textbf{5.} (also \textbf{reflexion}) [countable, uncountable] \textbf{reflection (of something)} (\textit{mathematics}) an operation on a shape to produce its mirror image.}. The space devoted by popular newspapers \& magazines to crossword puzzles \& other riddles\footnote{\textbf{riddle} [n] \textbf{1.} a question that is difficult to understand, \& that has a surprising answer, that you ask somebody as a game; \textbf{2.} a mysterious event or situation that you cannot explain, \textsc{synonym}: \textbf{mystery}; [v] \textbf{riddle somebody\texttt{/}something (with something)} to make a lot of holes in; \textbf{be riddle with something} [idiom] to be full of something, especially something bad or unpleasant.} seems to show that people spend some time in solving unpractical\footnote{\textbf{unpractical} [a] \textbf{1.} not sensible or realistic; \textbf{2.} (9of people) not good at doing things that involve using the hands; not good at planning or organizing things, \textsc{opposite}: \textbf{practical}.} problems. Behind the desire to solve this or that problem that confers\footnote{\textbf{confer} [v] \textbf{1.} [transitive] to give somebody a particular power, right or honor; \textbf{2.} [transitive] to give somebody\texttt{/}something a particular advantage; \textbf{3.} [intransitive] \textbf{confer (with somebody) (on\texttt{/}about something)} to discuss something with somebody, in order to exchange opinions or get advice.} no material advantage, there may be a deeper curiosity, a desire to understand the ways \& means, the motives \& procedures, of solution.

The following pages are written somewhat concisely\footnote{\textbf{concise} [a] giving only the information that is necessary \& important, using few words.}, but as simply as possible, \& are based on a long \& serious study of methods of solution. This sort of study, called \textit{heuristic}\footnote{\textbf{heuristic} [a] (\textit{formal}) \textbf{heuristic} teaching or education encourages you to learn by discovering things for yourself.}\,\footnote{\textbf{heuristics} [n] [uncountable] (\textit{formal}) a method of solving problems by finding practical ways of dealing with them, learning from past experience.} by some writers, is not in fashion nowadays but has a long past \&, perhaps, some future.

Studying the methods of solving problems, we perceive\footnote{\textbf{perceive} [v] \textbf{1.} to notice or become aware of something, \textsc{synonym}: \textbf{notice}; \textbf{2.} to be aware of or experience something using the senses; \textbf{3.} [often passive] to understand or think of somebody\texttt{/}something in a particular way; to believe that a particular thing is true, \textsc{synonym}: \textbf{see}.} another face of mathematics. Yes, mathematics has 2 faces; it is the rigorous\footnote{\textbf{rigorous} [a] \textbf{1.} done carefully \& with a lot of attention to detail, \textsc{synonym}: \textbf{thorough}; \textbf{2.} demanding that particular rules or processes are strictly followed, \textsc{synonym}: \textbf{strict}.} science of Euclid but it is also something else. Mathematics presented in the Euclidean way appears as a systematic\footnote{\textbf{systematic} [a] \textbf{1.} done according to a system or plan, in a thorough, efficient or determined way; \textbf{2.} (of an error) happening in the same way all through a process or set of results; caused by the system that is used.}, deductive\footnote{\textbf{deductive} [a] [usually before noun] using knowledge about things that are generally true in order to understand particular situations or problems.} science; but mathematics in the making appears as an experimental\footnote{\textbf{experimental} [a] \textbf{1.} [usually before noun] connected with scientific experiments; \textbf{2.} based on new ideas, forms or methods that are used to find out what effect they have.}, inductive\footnote{\textbf{inductive} [a] (\textit{specialist}) using particular facts \& examples to form general rules \& principles.} science. Both aspects\footnote{\textbf{aspect} [n] \textbf{1.} [countable] a particular feature of a situation, an idea or a process; a way in which something may be considered; \textbf{2.} [countable, usually singular] \textbf{aspect (of something)} (\textit{specialist}) a particular surface or side of an object or a part of the body; the direction in which something faces; \textbf{3.} [uncountable, countable] (\textit{grammar}) the form of a verb that shows, e.g., whether the action happens once or many times, is completed or is still continuing.} are as old as the science of mathematics itself. But the 2nd aspect is new in 1 respect\footnote{\textbf{respect} [n] \textbf{1.} [countable] a particular aspect or detail of something; \textbf{2.} [uncountable, singular] polite behavior towards or reasonable treatment of somebody\texttt{/}something; \textbf{3.} [uncountable, singular] a feeling of admiration for somebody\texttt{/}something because of their good qualities or achievements; \textbf{in respect of something} [idiom] (\textit{formal}) \textbf{1.} concerning; \textbf{2.} in payment for something; \textbf{with respect to something} [idiom] concerning.}; mathematics ``in statu nascendi,'' in the process of being invented, has never before presented in quite this manner to the student, or to the teacher himself, or to the general public.

The subject of heuristic has manifold\footnote{\textbf{manifold} [a] (\textit{formal}) many; of many different types; [n] (\textit{specialist}) a pipe or chamber with several openings, especially 1 for taking gases in \& out of a car engine.} connections; mathematicians, logicians\footnote{\textbf{logician} [n] a person who studies or is skilled in logic.}, psychologists, educationalists\footnote{\textbf{educationalists} [n] (also \textbf{educationist}) a specialist in theories \& methods of teaching.}, even philosophers may claim various parts of it as belonging to their special domains. The author, well aware of the possibility of criticism\footnote{\textbf{criticism} [n] \textbf{1.} [uncountable, countable] the act of expressing disapproval of somebody\texttt{/}something \& opinions about their faults or bad qualities; a statement showing disapproval; \textbf{2.} [uncountable] the work or activity of analyzing \& making fair, careful judgments about somebody\texttt{/}something, especially books, music, etc.} from opposite\footnote{\textbf{opposite} [a] \textbf{1.} [usually before noun] as different as possible from something; involving 2 different extremes; \textbf{2.} [usually before noun] on the other side of something or facing something; [n] \textbf{1.} (\textbf{the opposite}) [singular] the situation, idea or activity that is as different from another situation, etc. as it is possible to be, \textsc{synonym}: \textbf{the reverse}; \textbf{2.} (\textbf{opposites}) [plural] people, ideas or situations that are as different as possible from each other; \textbf{the exact opposite} [idiom] a person or thing that is as different as possible from somebody\texttt{/}something else; [prep] on the other side of a particular area from somebody\texttt{/}something, \& usually facing them.} quarters\footnote{\textbf{quarter} [n] \textbf{1.} (also \textbf{fourth} \textit{especially in NAE}) [countable] 1 of 4 equal parts of something; \textbf{2.} [countable] a period of 3 months, used especially as a period fo which bills are paid or a company's income is calculated; \textbf{3.} [countable] a person or group of people, especially as a source of help, information or a reaction; \textbf{4.} [countable, usually singular] a district or part of a town; \textbf{5.} (\textbf{quarters}) [plural] rooms that are provided for soldiers, servants, etc. to live in; \textbf{at\texttt{/}from close quarters} [idiom] very near.} \& keenly\footnote{\textbf{keenly} [adv] \textbf{1.} very strongly or deeply; \textbf{2.} by people with different opinions that they express strongly.} conscious\footnote{\textbf{conscious} [a] \textbf{1.} [not before noun] aware of something; noticing something, \textsc{opposite}: \textbf{unconscious}; \textbf{2.} able to use your senses \& mental powers to understand what is happening, \textsc{opposite}: \textbf{unconscious}; \textbf{3.} (of actions, feelings, etc.) deliberate or controlled, \textsc{opposite}: \textbf{unconscious}; \textbf{4.} being particularly interested in something.} of his limitations\footnote{\textbf{limitation} [n] \textbf{1.} [countable, usually plural] a limit on what somebody\texttt{/}something can do or how good they\texttt{/}it can be; \textbf{2.} [countable] a rule, fact or condition that limits something, \textsc{synonym}: \textbf{restraint}; \textbf{3.} [uncountable] \textbf{limitation (of something)} the act or process of limiting or controlling somebody\texttt{/}something, \textsc{synonym}: \textbf{restriction}; \textbf{4.} (also \textbf{limitation period}) [countable] (\textit{law}) a legal limit on the period of time within which court proceedings can be taken or for which a property right continues.}, has 1 claim to make: he has some experience in solving problems \& in teaching mathematics on various levels.

The subject is more fully dealt with in a more extensive book by the author which is on the way to completion. \textit{Stanford University, Aug 1, 1944}'' -- \cite[pp. v--vii]{Polya2014}

\section*{From the Preface to the 7th Printing}
``[$\ldots$] The 2 volumes \textit{Induction \& Analogy in Mathematics} \& \textit{Patterns of Plausible Inference} which constitute my recent work \textit{Mathematics \& Plausible Reasoning} continue the line of thinking begun in \textit{How to Solve It}.  \textit{Zurich, Aug 30, 1954}'' -- \cite[p. viii]{Polya2014}

\section*{Preface to the 2nd Edition}
``The present 2nd edition adds, besides a few minor improvements, a new 4th part, ``Problems, Hints, Solutions.''

As this edition was being prepared for print, a study appeared (Educational Testing Service, Princeton, N.J.; \textit{cf. Time}, Jun 18, 1956) which seems to have formulated\footnote{\textbf{formulate} [v] \textbf{1.} \textbf{formulate something} to create or prepare something carefully, giving particular attention to the details; \textbf{2.} \textbf{formulate something} to express your ideas in carefully chosen words.} a few pertinent\footnote{\textbf{pertinent} [a] appropriate to a particular situation, \textsc{synonym}: \textbf{relevant}.} observations -- they are not new to the people in the know, but it was high time to formulate them for the general public--: ``$\ldots$ mathematics has the dubious\footnote{\textbf{dubious} [a] \textbf{1.} that you cannot be sure about; that is probably not good. \textbf{Dubious} is also when you are stating that something is the opposite of a particular good quality. \textbf{2.} [usually before noun] probably not honest, \textsc{synonym}: \textbf{suspicious}; \textbf{3.} [not usually before noun] \textbf{dubious about something} feeling uncertain about something; not knowing whether something is good or bad, \textsc{synonym}: \textbf{doubtful}.}, honor of being the least popular subject in the curriculum $\ldots$ Future teachers pass through the elementary schools learning to detest\footnote{\textbf{detest} [v] (not used in the progressive tenses) to hate somebody\texttt{/}something very much, \textsc{synonym}: \textbf{loathe}.} mathematics $\ldots$ They return to the elementary school to teach a new generation to detest it.''

I hope that the present edition, designed for wider diffusion\footnote{\textbf{diffusion} [n] [uncountable] \textbf{1.} the spreading of something more widely; \textbf{2.} the mixing of substances by the natural movement of their particles; \textbf{3.} the spreading of elements of culture from 1 region or group to another.}, will convince some of its readers that mathematics, besides being a necessary avenue\footnote{\textbf{avenue} [n] a way of approaching a problem or making progress towards something.} to engineering jobs \& scientific knowledge, may be fun \& may also open up a vista\footnote{\textbf{vista} [n] \textbf{1.} (\textit{literary}) a beautiful view, e.g., of the countryside, a city, etc., \textsc{synonym}: \textbf{panorama}; \textbf{2.} (\textit{formal}) a range of things that might happen in the future, \textsc{synonym}: \textbf{prospect}.} of mental activity on the highest level. \textit{Zurich, Jun 30, 1956}'' -- \cite[pp. ix--]{Polya2014}

\section*{``How to Solve It'' list}
\begin{itemize}
	\item[\textbf{1st.}] You have to \textit{understand} the problem.
	
	\textsc{Understanding the Problem.}
	
	\textit{What is the unknown? What are the data? What is the condition?}
	
	It is possible to satisfy the condition? Is the condition sufficient to determine the unknown? Or is it insufficient? Or redundant? Or contradictory?
	
	Draw a figure. Introduce suitable notation.
	
	Separate the various parts of the condition. Can you write them down?
	\item[\textbf{2nd.}] Find the connection between the data \& the unknown. You may be obliged to consider auxiliary problems if an immediate connection cannot be found. You should obtain eventually a \textit{plan} of the solution.
	
	\textsc{Devising a Plan.}
	
	Have you seen it before? Or have you seen the same problem in a slightly different form?
	
	\textit{Do you know a related problem?} Do you know a theorem that could be useful?
	
	\textit{Look at the unknown!} \& try to think of a familiar problem having the same or a similar unknown.
	
	\textit{Here is a problem related to yours \& solved before. Could you use it?} Could you use its result? Could you use its method? Should you introduce some auxiliary element in order to make its use possible?
	
	Could you restate the problem? Could you restate it still differently? Go back to definitions.
	
	If you cannot solve the proposed problem try to solve 1st some related problem. Could you imagine a more accessible related problem? A more general problem? A more special problem? An analogous problem? Could you solve a part of the problem? Keep only a part of the condition, drop the other part; how far is the unknown then determined, how can it vary? Could you derive something useful from the data? Could you think of other data appropriate to determine the unknown? Could you change the unknown or the data, or both if necessary, so that the new unknown \& the new data are nearer to each other?
	
	Did you use all the data? Did you use the whole condition? Have you taken into account all essential notions involved in the problem?
	\item[\textbf{3rd.}] \textit{Carry out} your plan.
	
	\textsc{Carrying out the Plan.}
	
	Carrying out your plan of the solution, \textit{check each step}. Can you see clearly that the step is correct? Can you prove that it is correct?
	\item[\textbf{4th.}] \textit{Examine} the solution obtained.
	
	\textsc{Looking back.}
	
	Can you \textit{check the result?} Can you check the argument?
	
	Can you derive the result differently? Can you see it at a glance?
	
	Can you use the result, or the method, for some other problem?
\end{itemize}
-- \cite[How to solve it, pp. xvi--xvii]{Polya2014}

\section*{Foreword by John H. Conway}
``\textit{How to Solve It} is a wonderful book! This I realized when I 1st read right through it as a student many years ago, but it has taken me a long time to appreciate just \textit{how} wonderful it is. Why is that? 1 part of the answer is that the book is unique. In all my years as a student \& teacher, I have never seen another that lives up to George Polya's title by teaching you how to go about solving problems. A. H. Schoenfeld correctly described its importance in his 1987 article ``Polya, Problem Solving, \& Education'' in \textit{Mathematics Magazine}: ``For mathematics education \& the world of problem solving it marked a line of demarcation\footnote{\textbf{demarcation} [n] [uncountable, countable] a line or limit that separates 2 things, such as types of work, groups of people or areas of land.} between 2 eras\footnote{\textbf{era} [n] \textbf{1.} a period of time, usually in history, that is different from other periods because of particular characteristics or events; \textbf{2.} (\textit{earth sciences}) a major division of time that can itself be divided into periods.}, problem solving before \& after Polya.''

It is 1 of the most successful mathematics books ever written, having sold over a million copies \& been translated into 17 languages since it 1st appeared in 1945. Polya later wrote 2 more books about the art of doing mathematics, \textit{Mathematics \& Plausible Reasoning} (1954) \& \textit{Mathematical Discovery} (2 volumes, 1962 \& 1965).

The book's title makes it seem that it is directed only toward students, but in fact it is addressed just as much to their teachers. Indeed, as Polya remarks in his introduction, the 1st part of the book takes the teacher's viewpoint more often than the student's.

Everybody gains that way. The student who reads the book on his own will find that overhearing\footnote{\textbf{overhear} [v] to hear, especially by accident, a conversation in which you are not involved.} Polya's comments to his non-existent\footnote{\textbf{non-existent} [a] not existing; not real.} teacher can bring that desirable person into being, as an imaginary but very helpful figure leaning over one's shoulder. This is what happened to me, \& naturally I made heavy use of the remarks I'd found most important when I myself started teaching a few years later.

But it was some time before I read the book again, \& when I did, I suddenly realized that it was even more valuable than I'd thought! Many of Polya's remarks that hadn't helped me as a student now made me a better teacher of those whose problems had differed from mine. Polya had met many more students than I had, \& had obviously thought very hard about how to best help all of them learn mathematics. Perhaps his most important point is that \fbox{learning must be active}. As he said in a lecture on teaching, ``Mathematics, you see, is not a spectator\footnote{\textbf{spectator} [n] a person who is watching a performance or an event.} sport. To understand mathematics means to be able to do mathematics. \& what does it mean [to be] doing mathematics? In the 1st place, it means to be able to solve mathematical problems.''

It is often said that to teach any subject well, one has to understand it ``at least as well as one's students do.'' It is a paradoxical\footnote{\textbf{paradoxical} [a] \textbf{1.} (of a person, thing or situation) having 2 opposite features \& therefore seem strange; \textbf{2.} (of a statement) containing 2 opposite ideas that make it seem impossible or unlikely, although it is probably true.} truth that to teach mathematics well, one must also know how to misunderstand it at least to the extent one's students do! If a teacher's statement can be parsed\footnote{\textbf{parse} [v] (\textit{grammar}) \textbf{parse something} to divide a sentence into parts \& describe the grammar of each word or part.} in 2 or more ways, it goes without saying that some students will understand it 1 way \& others another, with results that can vary from the hilarious\footnote{\textbf{hilarious} [a] extremely funny.} to the tragic\footnote{\textbf{tragic} [a] \textbf{1.} making you feel very sad, usually because somebody has died or suffered a lot; \textbf{2.} [usually before noun] connected with tragedy ($=$ the style of literature).}. J. E. Littlewood gives 2 amusing\footnote{\textbf{amusing} [a] funny \& giving pleasure.} examples of assumptions that can easily be made unconsciously \& misleadingly\footnote{\textbf{misleading} [a] giving the wrong idea \& making people believe something that is not true, \textsc{synonym}: \textbf{deceptive}.}. 1st, he remarks that the description of the coordinate axes (``$Ox$ \& $Oy$ as in 2 dimensions, $Oz$ vertical'') in Lamb's book \textit{Mechanics} is incorrect for him, sine he always worked in an armchair\footnote{\textbf{armchair} [n] a comfortable chair with sides on which you can rest your arms; [a] [only before noun] knowing about a subject through books, television, the Internet, etc., rather than by doing it for yourself.} with his feet up! Then, after asking how his reader would present the picture of a closed curve lying all on 1 side of its tangent, he states that there are 4 main schools (to left or right of vertical tangent, or above or below horizontal one) \& that by lecturing without a figure, presuming that the curve was to the right of its vertical tangent, he had unwittingly\footnote{\textbf{unwittingly} [adv] without being aware of what you are doing or the situation that you are involved in, \textsc{opposite}: \textbf{wittingly}.} made nonsense\footnote{\textbf{nonsense} [n] \textbf{1.} [uncountable, countable] ideas, statements or beliefs that you think are silly or not true, \textsc{synonym}: \textbf{rubbish}; \textbf{2.} [uncountable] spoken or written words that have no meaning or make no sense; \textbf{3.} [uncountable] silly or unacceptable behavior; \textbf{make (a) nonsense of something} [idiom] to reduce the value of something by a lot; to make something seem silly.} for the other 3 schools.

I know of no better remedy\footnote{\textbf{remedy} [n] (plural \textbf{remedies}) \textbf{1.} a way of dealing with or improving an unpleasant or difficult situation, \textsc{synonym}: \textbf{solution}; \textbf{2.} a treatment or medicine to cure a disease or to reduce pain that is not very serious; \textbf{3.} (\textit{law}) a way of dealing with a problem, using the processes of the law, \textsc{synonym}: \textbf{redress}; [v] \textbf{remedy something} to correct or improve something.} for such presumptions\footnote{\textbf{presumption} [n] [countable, uncountable] the act of supposing or accepting that something is true or exists, although it has not been proved; a belief that something is true or exists, \textsc{synonym}: \textbf{assumption}. In legal contexts, \textbf{presumption} often means that something is being accepted as true until it is shown not to be true.} than Polya's counsel\footnote{\textbf{counsel} [n] [uncountable, countable] \textbf{1.} (\textit{formal}) advice, especially given by older people or experts; a piece of advice; \textbf{2.} a lawyer or group of lawyers representing somebody in court; [v] (\textit{formal}) \textbf{1.} \textbf{counsel somebody} to listen to \& give support or professional advice to somebody who needs help; \textbf{2.} to advise a particular course of action; to advise somebody to do something.}: before trying to solve a problem, the student should demonstrate his or her understanding of its statement, preferably\footnote{\textbf{preferable} [a] more attractive or more suitable; to be preferred to something.} to a real teacher, but in lieu\footnote{\textbf{lieu} [n] (\textit{formal}) \textbf{in lieu (of something)} [idiom] instead of.} of that, to an imagined one. Experienced mathematicians know that often the hardest part of researching a problem is understanding precisely what that problem says. They often follow Polya's wise advice: ``If you can't solve a problem, then there is an easier problem you can't solve: find it.''

Readers who learn from this book will also want to learn about its author's life.\footnote{The following biographical information is taken from that given by J. J. O'Connor \& E. F. Robertson in the MacTutor History of Mathematics Archive (\url{www-gap.dcs.st-and.ac.uk/~hisotry/}).}

George Polya was born Gy\"orgy P\'olya (he dropped the accents sometime later) on Dec 13, 1887, in Budapest, Hungary, to Jakab P\'olya \& his wife, the former Anna Deutsch. He was baptized into the Roman Catholic faith, to which Jakab, Anna, \& their 3 previous children, Jen\H{o}, Ilona, \& Fl\'ora, had converted from Judaism\footnote{\textbf{Judaism} [n] [uncountable] the religion of the Jewish people, based mainly on the Bible \& the Talmud ($=$ a collection of ancient writings on Jewish law \& traditions).} in the previous year. The 5th child, L\'aszl\'o, was born 4 years later.

Jakab had changed his surname from Poll\'ak to the more Hungarian-sounding P\'olya 5 years before Gy\"orgy was born, believing that this might help him obtain a university post, which he eventually did, but only shortly before his untimely\footnote{\textbf{untimely} [a] (\textit{formally}) \textbf{1.} [usually before noun] happening too soon or sooner than is normal or expected, \textsc{synonym}: \textbf{premature}; \textbf{2.} happening at a time or in a situation that is not suitable, \textsc{synonym}: \textbf{ill-timed}, \textsc{opposite}: \textbf{timely}.} death in 1897.

At the D\'aniel Berzsenyi Gymnasium\footnote{\textbf{gymnasium} (plural \textbf{gymnasiums, gymnasia}) (\textit{formal}) a gym.}, Gy\"orgy studied Greek, Latin, \& German, in addition to Hungarian. It is surprising to learn that there he was seemingly uninterested in mathematics, his work in geometry deemed merely ``satisfactory'' compared with his ``outstanding'' performance in literature, geography, \& other subjects. His favorite subject, outside of literature, was biology.

He enrolled at the University of Budapest in 1905, initially studying law, which he soon dropped because he found it too boring. He then obtained the certification needed to teach Latin \& Hungarian at a gymnasium, a certification that he never used but of which he remained proud. Eventually his professor, Bern\'at Alexander, advised him that to help his studies in philosophy, he should take some mathematics \& physics courses. This was how he came to mathematics. Later, he joked that he ``wasn't good enough for physics, \& was too good for philosophy -- mathematics is in between.''

In Budapest he was taught physics by E\"otv\"os \& mathematics by Fej\'er \& was awarded a doctorate after spending the academic year 1910--11 in Vienna, where he took some courses by Wirtinger \& Mertens. He spent much of the next 2 years in G\"ottingen, where he met many more mathematicians -- Klein, Caratheodory, Hilbert, Runge, Landau, Weyl, Hecke, Courant, \& Toeplitz -- \& in 1914 visited Paris, where he became acquainted\footnote{\textbf{acquainted} [a] [not before noun] \textbf{1.} \textbf{acquainted with something} (\textit{formal}) familiar with something, having read, seen or experienced it; \textbf{2.} not close friends with somebody, but having met a few times before.} with Picard \& Hadamard \& learned that Hurwitz had arranged an appointment for him in Z\"urich. He accepted this position, writing later: ``I went to Z\"urich in order to be near Hurwitz, \& we were in close touch for about 6 years, from my arrival in Z\"urich in 1914 to his passing [in 1919]. I was very much impressed by him \& edited his works.''

Of course, the 1st World War took place during this period. It initially had little effect on Polya, who had been declared unfit for service in the Hungarian army as the result of a soccer wound. But later when the army, more desperately\footnote{\textbf{desperate} [a] \textbf{1.} feeling or showing that you have little hope \& are ready to do anything without worrying about danger to yourself or others; \textbf{2.} [usually before noun] (of an action) giving little hope of success; tried when everything else was failed; \textbf{3.} (of a situation) extremely serious or dangerous.} needing recruits\footnote{\textbf{recruit} [v] \textbf{1.} [transitive, intransitive] to find new people to join a company, an organization, the armed forces, etc.; \textbf{2.} [transitive] to get people to help with or be involved in something; \textbf{3.} [transitive] \textbf{recruit something (from something)} to form a new army, team, etc. by persuading new people to join it; [n] \textbf{1.} a person who has recently joined the armed forces or the police; \textbf{2.}  a person who joins an organization, a company, etc.}, demanded that he return to fight for his country, his strong pacifist\footnote{\textbf{pacifist} [a] [usually before noun] holding or showing the belief that war \& violence are always wrong; [n] a person who believes that war \& violence are always wrong \& refuses to fight in a war.} views led him to refuse. As a consequence, he was unable to visit Hungary for many years, \& in fact did not do so until 1967, 54 years after he left.

In the meantime, he had taken Swiss citizenship \& married a Swiss girl, Stella Vera Weber, in 1918. Between 1918 \& 1919, he published papers on a wide range of mathematical subjects, such as series, number theory, combinatorics, voting systems, astronomy, \& probability. He was made an extraordinary professor at the Z\"urich ETH in 1920, \& a few years later he \& G\'abor Szeg\H{o} published their book \textit{Aufgaben und Lehrsatze aus der Analysis} (``Problems \& Theorems in Analysis''), described by G. L. Alexanderson \& L. H. Lange in their obituary\footnote{\textbf{obituary} [n] (plural \textbf{obituaries}) an article about somebody's life \& achievements, that is printed in a newspaper soon after they have died.} of Polya as ``a mathematical masterpiece\footnote{\textbf{masterpiece} [n] (also \textbf{masterwork}) \textbf{1.} \textbf{masterpiece (of something)} a work of art such as a painting, film, book, etc. that is an excellent, or the best, example of the artist's work; \textbf{2.} \textbf{masterpiece of something} an extremely good example of something.} that assured\footnote{\textbf{assure} [v] \textbf{1.} to tell somebody that something is definitely true or is definitely going to happen, especially when they have doubts about it; \textbf{2.} to make something certain to happen; to make somebody\texttt{/}something certain to get something; \textbf{3.} to make yourself certain about something.} their reputations\footnote{\textbf{reputation} [n] the opinion that people have about what somebody\texttt{/}something is like, based on what has happened in the past.}.''

That book appeared in 1925, after Polya had obtained a Rockefeller Fellowship to work in England, where he collaborated with Hardy \& Littlewood on what later become their book \textit{Inequalities} (Cambridge University Press, 1936). He used a 2nd Rockefeller Fellowship to visit Princeton University in 1933, \& while in the United States was invited by H. F. Blichfeldt to visit Stanford University, which he greatly enjoyed, \& which ultimately became his home. Polya held a professorship at Stanford from 1943 until his retirement in 1953, \& it was there, in 1978, that he taught his last course, in combinatorics; he died on Sep 7, 1985, at the age of 97.

Some readers will want to know about Polya's many contributions to mathematics. Most of them relate to analysis \& are too technical to be understood by non-experts, but a few are worth mentioning.

In probability theory, Polya is responsible for the now-standard term ``Central Limit Theorem'' \& for proving that the Fourier transform of a probability measure is a characteristic function \& that a random walk on the integer lattice closes with probability 1 iff the dimension is at most 2.

In geometry, Polya independently re-enumerated the 17 plane crystallographic\footnote{\textbf{crystallography} [n] [uncountable] the branch of science that deals with crystals.} groups (their 1st enumeration\footnote{\textbf{enumeration} [n] [uncountable, countable] (\textit{formal}) the act of naming things 1 by 1 in a list; a list of this sort.}, by E. S. Fedorov, having been forgotten) \& together with P. Niggli devised\footnote{\textbf{devise} [v] \textbf{devise something} to plan or invent a procedure, system or method, especially one that is new or complicated, by using careful thought, \textsc{synonym}: \textbf{think something up}.} a notation for them.

In combinatorics, Polya's Enumeration Theorem is now a standard way of counting configurations according to their symmetry. It has been described by R. C. Read as ``a remarkable\footnote{\textbf{remarkable} [a] unusual or surprising in a way that causes people to take notice.} theorem in a remarkable paper, \& a landmark\footnote{\textbf{landmark} [n] \textbf{1.} something, such as a large building, that you can see clearly from a distance \& that will help you to know where you are; \textbf{2.} an event, a discovery or an invention that marks an important stage in something.} in the history of combinatorial analysis.''

\textit{How to Solve It} was written in German during Polya's time in Z\"urich, which ended up in 1940, when the European situation forced him to leave for the United States. Despite the book's eventual success, 4 publishers rejected the English version before Princeton University Press brought it out in 1945. In their hands, \textit{How to Solve It} rapidly became -- \& continues to be -- 1 of the most successful mathematical books of all time.'' -- \cite[Foreword, pp. xix--xxiv]{Polya2014}

\section*{Introduction}
``The following consideration are grouped around the preceding list of questions \& suggestions entitled\footnote{\textbf{entitle} [v] \textbf{1.} [often passive] to give somebody the right to have or to do something; \textbf{2.} [usually passive] to give a title to a book, document, film, etc.} ``How to Solve It.'' Any question or suggestion quoted from it will be printed in \textit{italics}, \& the whole list will be referred to simply as ``the list'' or as ``our list.''

The following pages will discuss the purpose of the list, illustrate its practical use by examples, \& explain the underlying notions \& mental operations. By way of preliminary explanation, this much may be said: If, using them properly, you address these questions \& suggestions to yourself, they may help you to solve your problem. If, using them properly, you address the same questions \& suggestions to 1 of your students, you may help him to solve his problem.

The book is divided into 4 parts.

The title of the 1st part is ``In the Classroom.'' It contains 20 sections. Each section will be quoted by its number in heavy type as, e.g., ``sect. \textbf{7.}'' Sects. \textbf{1}--\textbf{5} discuss the ``Purpose'' of our list in general terms. Sects. \textbf{6}--\textbf{17} explain what are the ``Main Divisions, Main Questions'' of the list, \& discuss a 1st practical example. Sects. \textbf{18}--\textbf{20} add ``More Examples.''

The title of the very short 2nd part is ``How to Solve It.'' It is written in dialogue; a somewhat idealized teacher answers short questions of a somewhat idealized student.

The 3rd \& most extensive part is a ``Short Dictionary of Heuristic''; we shall refer to it as the ``Dictionary.'' It contains 67 articles arranged alphabetically. E.g., the meaning of the term \textsc{heuristic} (set in small capitals) is explained in an article with this title on p. 112. When the title of such an article is referred to within the text it will be set in small capitals. Certain paragraphs of a few articles are more technical; they are enclosed\footnote{\textbf{enclose} [v] \textbf{1.} [usually passive] to build a wall, fence, etc. around something; \textbf{2.} \textbf{enclose something} (especially of a wall, fence, etc.) to surround something; \textbf{3.} \textbf{enclose something (with something)} to put something in the same envelope or package as something else.} in square brackets. Some articles are fairly closely connected with the 1st part to which they add further illustrations \& more specific comments. Other articles go somewhat beyond the aim of the 1st part of which they explain the background. There is a key-article on \textsc{modern heuristic}. It explains the connection of the main articles \& the plan underlying the Dictionary; it contains also directions how to find information about particular items of the list. It must be emphasized that there is a common plan \& a certain unity, because the articles of the Dictionary show the greatest outward variety. There are a few longer articles devoted to the systematic though condensed discussion of some general theme; others contain more specific comments; still others cross-references\footnote{\textbf{cross-reference} [v] \textbf{cross-reference something} to give cross references to another text or part of a text.}, or historical data, or quotations, or aphorisms\footnote{\textbf{aphorism} [n] (\textit{formal}) a short phrase that says something true or wise.}, or even jokes.

The Dictionary should not be read too quickly; its text is often condensed, \& now \& then somewhat subtle. The reader may refer to the Dictionary for information about particular points. If these points come from his experience with his own problems or his own students, the reading has a much better chance to be profitable\footnote{\textbf{profitable} [a] \textbf{1.} that makes or is likely to make money; \textbf{2.} that gives somebody an advantage or a useful result.}.

The title of the 4th part is ``Problems, Hints, Solutions.'' It proposes a few problems to the more ambitious reader. Each problem is followed (in proper distance) by a ``hint'' that may reveal a way to the result which is explained in the ``solution.''

We have mentioned repeatedly the ``student'' \& the ``teacher'' \& we shall refer to them again \& again. It may be good to observe that the ``student'' may be a high school student, or a college student, or anyone else who is studying mathematics. Also the ``teacher'' may be a high school teacher, or a college instructor, or anyone interested in the technique of teaching mathematics. The author looks at the situation sometimes from the point of view of the student \& sometimes from that of the teacher (the latter case is preponderant\footnote{\textbf{preponderant} [a] [usually before noun] (\textit{formal}) larger in number or more important than other people or things in a group.} in the 1st part). Yet most of the time (especially in the 3rd part) the point of view is that of a person who is neither teacher nor student but anxious to solve the problem before him.'' -- \cite[Introduction, pp. xxv--xxvii]{Polya2014}

\begin{center}
	\huge Part I. In The Classroom
\end{center}

\begin{center}
	\LARGE Purpose
\end{center}

\section{Helping the student}
``1 of the most important tasks of the teacher is to help his students. This task is not quite easy; it demands time, practice, devotion\footnote{\textbf{devotion} [n] [uncountable, singular] \textbf{1.} devotion (of somebody) (to somebody\texttt{/}something) great love, care \& support for somebody\texttt{/}something; \textbf{2.} \textbf{devotion (to somebody\texttt{/}something)} the action of spending a lot of time or energy on something, \textsc{synonym}: \textbf{dedication}; \textbf{3.} great religious feeling.}, \& sound principles.

The student should acquire as much experience of independent work as possible. But if he is left alone with his problem without any help or with insufficient help, he may make no progress at all. If the teacher helps too much, nothing is left to the student. The teacher should help, but not too much \& not too little, so that the student shall have a \textit{reasonable share of the work}.

If the student is not able to do much, the teacher should leave him at least some illusion\footnote{\textbf{illusion} [n] \textbf{1.} [countable, uncountable] a false idea or belief; \textbf{2.} [countable] something that seems to exist but in fact does not, or seems to be something that it is not.} of independent work. In order to do so, the teacher should help the student discreetly\footnote{\textbf{discreetly} [adv] in a careful way, in order to keep something secret or to avoid causing difficulty for somebody or making them feel embarrassed, \textsc{synonym}: \textbf{tactfully}.}, \textit{unobtrusively}\footnote{\textbf{unobtrusively} [adv] (\textit{formal, often approving}) in a way that does not attract unnecessary attention, \textsc{opposite}: \textbf{obtrusively}.}.

The best is, however, to help the student naturally. The teacher should put himself in the student's place, he should see the student's case, he should try to understand what is going on in the student's mind, \& ask a question or indicate\footnote{\textbf{indicate} [v] \textbf{1.} to show that something is true or exists; \textbf{2.} to be a sign of something; to show that something is possible or likely, \textsc{synonym}: \textbf{suggest}; \textbf{3.} \textbf{indicate something} to represent information without using words; \textbf{4.} to give information in writing; \textbf{5.} [usually passive] to suggest something as a necessary or recommend course of action; \textbf{6.} to mention something, especially in an indirect or brief way; \textbf{7.} \textbf{indicate something} (of an instrument for measuring things) to show a particular measurement.} a step that \textit{could have occurred to the student himself}.'' -- \cite[p. 1]{Polya2014}

\section{Questions, recommendations, mental operations}
``Trying to help the student effectively but unobtrusively \& naturally, the teacher is led to ask the same questions \& to indicate the same steps again \& again. Thus, in countless problems, we have to ask the question: \textit{What is the unknown?} We may vary the words, \& ask the same thing in many different ways: What is required? What do you want to find? What are you supposed to seek? The aim of these questions is to focus the student's attention upon the unknown. Sometimes, we obtain the same effect more naturally with a suggestion: \textit{Look at the unknown!} Question \& suggestion aim at the same effect; they tend to provoke\footnote{\textbf{provoke} [v] \textbf{1.} \textbf{provoke something} to cause a particular reaction or have a particular effect; \textbf{2.} to say or do something in order to produce a strong reaction from somebody, usually anger.} the same mental operation.

It seemed to the author that it might be worth while\footnote{\textbf{worth somebody's while (to do something\texttt{/}doing something)} [idiom] interesting or useful for somebody to do.} to collect \& to group questions \& suggestions which are typically\footnote{\textbf{typically} [adv] \textbf{1.} used to say that something usually happens in the way that you are stating; \textbf{2.} in a way that shows the usual qualities or features of a particular type of person, thing or group.} helpful in discussing problems with students. The list we study contains questions \& suggestions of this sort, carefully chosen \& arranged; they are equally useful to the problem-solver who works by himself. If the reader is sufficiently acquainted with the list \& can see, behind the suggestion, the action suggested, he may realize that the list enumerates, indirectly, \textit{mental operations typically useful for the solution of problems}. These operations are listed in the order in which they are most likely to occur.'' -- \cite[pp. 1--2]{Polya2014}

\section{Generality}
``Generality\footnote{\textbf{generality} [n] (plural \textbf{generalities}) \textbf{1.} [uncountable] \textbf{generality (of something)} the quality of a theory or model that can be applied being across a wide range of cases \& situations. The phrase \textbf{without loss of generality} means that a statement about 1 particular case can be easily applied to all other cases.; \textbf{2.} [countable, usually plural] a statement that makes general points rather than giving details or particular examples; \textbf{3.} (\textbf{the generality}) [singular $+$ singular or plural verb] \textbf{generality (of somebody\texttt{/}something)} (\textit{formal}) the greater part of a group of people or things, \textsc{synonym}: \textbf{majority}.} is an important characteristic of the questions \& suggestions contained in our list. Take the questions: \textit{What is the unknown? What are the data? What is the condition?} These questions are generally applicable\footnote{\textbf{applicable} [a] [not usually before noun] true about or appropriate to a particular situation, group of people, etc.}, we can ask them with good effect dealing with all sorts of problems. Their use is not restricted to any subject-matter\footnote{\textbf{subject matter} [n] [uncountable] \textbf{subject matter (of something)} the ideas or information contained in a book, speech, painting, etc.}. Our problem may be algebraic or geometric, mathematical or nonmathematical, theoretical or practical, a serious problem or a mere puzzle; it makes no difference, the questions make sense \& might help us to solve the problem.

There is a \textbf{restriction}\footnote{\textbf{restriction} [n] \textbf{1.} [countable] a rule or law that limits what you can do or what can happen; \textbf{2.} [uncountable] the act of limiting or controlling somebody\texttt{/}something.}, in fact, but it has nothing to do with the subject-matter. Certain questions \& suggestions of the list are applicable to ``problems to find'' only, not to ``problems to prove.'' If we have a problem of the latter kind we must use different questions; see \textsc{problems to find, problems to prove}.'' -- \cite[pp. 2--3]{Polya2014}

\section{Common sense}
``The questions \& suggestions of our list are general, but, except for their generality, they are natural, simple, obvious, \& proceed from plain common sense. Take the suggestion: \textit{Look at the unknown! \& try to think of a familiar problem having the same or a similar unknown}. This suggestion advises you to do what you would do anyhow\footnote{\textbf{anyhow} [adv] \textbf{1.} (also \textbf{anyway}, also NAE, informal \textbf{anyways}) used when adding something to support an idea or argument; \textbf{2.} (also \textbf{anyway}, also NAE, informal \textbf{anyways}) despite something; even so; \textbf{3.} (also \textbf{anyway}, also NAE, informal \textbf{anyways}) used when changing the subject of a conversation, ending the conversation or returning to a subject; \textbf{4.} (also \textbf{anyway}, also NAE, informal \textbf{anyways}) used to correct or slightly change what you have said; \textbf{5.} in a careless way; not arranged in an order.}, without any advice, if you were seriously concerned with your problem. Are you hungry? You wish to obtain food \& you think of familiar ways of obtaining food. Have you a problem of geometric construction? You wish to construct a triangle \& you think of familiar ways of constructing a triangle. Have you a problem of any kind? You wish to find a certain unknown, or some similar unknown. If you do so you follow exactly the suggestion we quoted from our list. \& you are on the right track, too; the suggestion is a good one, it suggests to you a procedure which is very frequently successful.

All the questions \& suggestions of our list are natural, simple, obvious, just plain common sense; but they state plain common sense in general terms. They suggest a certain conduct which comes naturally to any person who is seriously concerned with his problem \& has some common sense. But the person who behaves the right way usually does not care to express his behavior in clear words \&, possibly, he cannot express it so; our list tries to express it so.'' -- \cite[p. 3]{Polya2014}

\section{Teacher \& student. Imitation \& practice}
``There are 2 aims which the teacher may have in view when addressing to his students a question or a suggestion of the list: 1st, to help the student to solve the problem at hand. 2nd, to develop the student's ability so that he may solve future problems by himself.

Experience shows that the questions \& suggestions of our list, appropriately used, very frequently help the student. They have 2 common characteristics, common sense \& generality. As they proceed from plain common sense they very often come naturally; they could have occurred to the student himself. As they are general, they help unobtrusively; they just indicate a general direction \& leave plenty for the student to do.

But the 2 aims we mentioned before are closely connected; if the student succeeds in solving the problem at hand, he adds a little to his ability to solve problems. Then, we should not forget that our questions are general, applicable in many cases. If the same question is repeatedly helpful, the student will scarcely\footnote{\textbf{scarcely} [adv] only just; almost not.} fail to notice it \& he will be induced\footnote{\textbf{induce} [v] \textbf{1.} \textbf{induce something} to cause something; \textbf{2.} \textbf{induce somebody to do something} to persuade or influence somebody to do something; \textbf{3.} \textbf{induce something} (\textit{physics}) to produce an electric charge or current, or a magnetic state by induction; \textbf{4.} \textbf{induce something (from something)} to use particular facts \& examples to form a general rule or principle.} to ask the question by himself in a similar situation. Asking the question repeatedly, he may succeed once in eliciting\footnote{\textbf{elicit} [v] to get information or a reaction from somebody\texttt{/}something.} the right idea. By such a success, he discovers the right way of using the question, \& then he has really assimilated\footnote{\textbf{assimilate} [v] \textbf{1.} [intransitive, transitive] to become a part of a country or community rather than remaining in a separate group; to allow or cause people to do this; \textbf{2.} [transitive] \textbf{assimilate something} (of the body or any biological system) to absorb or take in a substance; \textbf{3.} \textbf{assimilate something} to think deeply about something \& understand it fully, so that you can use it, \textsc{synonym}: \textbf{absorb}; \textbf{4.} [transitive, often passive] \textbf{assimilate something (into\texttt{/}to something)} to accept an idea, information or activity; to make it fit into something.} it.

The student may absorb\footnote{\textbf{absorb} [v] \textbf{1.} to take in a liquid, gas or other substance from the surface or space around; \textbf{2.} \textbf{absorb something} to take in \& keep heat, light or other forms of energy, instead of reflecting it; \textbf{3.} [often passive] to take control of a smaller unit or group \& make it part of something larger; \textbf{4.} to take something into the mind \& learn or understand it, \textsc{synonym}: \textbf{take something in}; \textbf{5.} \textbf{absorb something} to deal with or reduce the effects of changes or costs; \textbf{6.} \textbf{absorb something} to use up a large supply of something, especially money or time; \textbf{7.} \textbf{be absorbed in something} to be so interested in something that you pay no attention to anything else.} a few questions of our list so well that he is finally able to put to himself the right question in the right moment \& to perform the corresponding mental operation naturally \& vigorously\footnote{\textbf{vigorously} [adv] \textbf{1.} with determination, energy or enthusiasm; \textbf{2.} in a way that involves physical strength, effort or energy.}. Such a student has certainly derived the greatest possible profit from our list. What can the teacher do in order to obtain this best possible result?

Solving problems is a practical skill like, let us say, swimming. We acquire any practical skill by imitation\footnote{\textbf{imitation} [n] \textbf{1.} [countable] a copy of something, especially something expensive; \textbf{2.} [uncountable] the act of copying somebody\texttt{/}something.} \& practice. Trying to swim, you imitate what other people do with their hands \& feet to keep their heads above water, \&, finally, you learn to swim by practicing swimming. Trying to solve problems, you have to observe \& to imitate what other people do when solving problems \&, finally, you learn to do problems by doing them.

The teacher who wishes to develop his students' ability to do problems must instill\footnote{\textbf{instil} [v] (BE) (NAE \textbf{instill}) to graduate put an idea or attitude into somebody's mind; to make somebody feel, think or behave in a particular way over a period of time.} some interest for problems into their minds \& give them plenty of opportunity for imitation \& practice. If the teacher wishes to develop in his students the mental operations which correspond to the questions \& suggestions of our list, he puts these questions \& suggestions to the students as often as he can do so naturally. Moreover, when the teacher solves a problem before the class, he should dramatize\footnote{\textbf{dramatize} [v] (BE also \textbf{dramatise}) \textbf{1.} to present a book or an event as a play or film; \textbf{2.} \textbf{dramatize something} to make something seem more exciting or important than it really is.} his ideas a little \& he should put to himself the same questions which he uses when helping the students. Thanks to such guidance, the student will eventually discover the right use of these questions \& suggestions, \& doing so he will acquire something that is more important than the knowledge of any particular mathematical fact.'' -- \cite[pp. 3--5]{Polya2014}

\begin{center}
	\LARGE Main divisions, main questions
\end{center}

\section{4 phases}
``Trying to find the solution, we may repeatedly change our point of view, our way of looking at the problem. We have to shift our position again \& again. Our conception\footnote{\textbf{conception} [n] \textbf{1.} [countable, uncountable] an understanding or a belief of what something is or what something should be; \textbf{2.} [uncountable] the process of forming an idea or a plan; \textbf{3.} [uncountable, countable] the process of an egg being fertilized inside a woman's body so that he becomes pregnant.} of the problem is likely to be rather incomplete when we start the work; our outlook is different when we have made some progress; it is again different when we have almost obtained the solution.

In order to group conveniently the questions \& suggestions of our list, we shall distinguish 4 phases of the work. 1st, we have to \textit{understand} the problem; we have to see clearly what is required. 2nd, we have to see how the various items are connected, how the unknown is liked to the data, in order to obtain the idea of the solution, to make a \textit{plan}. 3rd, we \textit{carry out} our plan. 4th, we \textit{look back} at the completed solution, we review \& discuss it.

Each of these phases has its importance. It may happen that a student hits upon an exceptionally bright idea \& jumping all preparations blurts\footnote{\textbf{blurt} [v] \textbf{blurt something (out) $|$ blurt that $\ldots$ $|$ blurt what, how, etc. $\ldots$ $|$ $+$ speech} to say something suddenly \& without thinking carefully enough.} out with the solution. Such lucky ideas, of course, are most desirable\footnote{\textbf{desirable} [a] that you would like to have or do; worth having or doing, \textsc{opposite}: \textbf{undesirable}.}, but something very undesirable\footnote{\textbf{undesirable} [a] not wanted or approved of; likely to cause trouble or problems, \textsc{opposite}: \textbf{desirable}.} \& unfortunate may result if the student leaves out any of the 4 phases without having a good idea. The worst may happen if the student embarks upon computations or constructions without having \textit{understood} the problem. It is generally useless to carry out details without having seen the main connection, or having made a sort of \textit{plan}. Many mistakes can be avoided if, carrying out his plan, the student \textit{checks each step}. Some of the best effects may be lost if the student fails to reexamine \& to \textit{reconsider} the completed solution.'' -- \cite[pp. 5--6]{Polya2014}

\section{Understanding the problem}
``It is foolish\footnote{\textbf{foolish} [a] \textbf{1.} not showing good sense or judgment, \textsc{synonym}: \textbf{silly, stupid}; \textbf{2.} [not usually before noun] made to feel or look silly \& embarrassed, \textsc{synonym}: \textbf{silly, stupid}.} to answer a question that you do not understand. It is sad to work for an end that you do not desire. Such foolish \& sad things often happen, in \& out of school, but the teacher should try to prevent them from happening in his class. The student should understand the problem. But he should not only understand it, he should also desire its solution. If the student is lacking in understanding or in interest, it is not always his fault; the problem should be well chosen, not too difficult \& not too easy, natural \& interesting, \& some time should be allowed for natural \& interesting presentation.

1st of all, the verbal statement of the problem must be understood. The teacher can check this, up to a certain extent; he asks the student to repeat the statement \& the student should be able to state the problem fluently. The student should also be able to point out the principal parts of the problem, the unknown, the data, the condition. Hence, the teacher can seldom afford to miss the questions: \textit{What is the unknown? What are the data? What is the condition?}

The student should consider the principle parts of the problem attentively\footnote{\textbf{attentive} [a] \textbf{1.} reading, listening or watching carefully \& with interest; \textbf{2.} helpful; making sure that people have what they need.}, repeatedly, \& from various sides. If there is a figure connected with the problem he should \textit{draw a figure} \& point out on it the unknown \& the data. If it is necessary to give names to these objects he should \textit{introduce suitable notation}; devoting some attention to the appropriate choice of signs, he is obliged to consider the objects for which the signs have to be chosen. There is another question which may be useful in this preparatory\footnote{\textbf{preparatory} [a] done in order to prepare for something.} stage provided that we do not expect a definitive\footnote{\textbf{definitive} [a] \textbf{1.} final; that cannot be changed; \textbf{2.} [usually before noun] considered to be the best of its kind \& almost impossible to improve.} answer but just a provisional\footnote{\textbf{provisional} [a] made for the present time, possibly to be changed when more information is available or when a more permanent arrangement can be made.} answer, a guess: \textit{Is it possible to satisfy the condition?}

(In the exposition of Part II [p. 33] ``Understanding the problem'' is subdivided into 2 stages: ``Getting acquainted'' \& ``Working for better understanding.'')'' -- \cite[pp. 6--7]{Polya2014}

\section{Example}
``Let us illustrate some of the points explained in the foregoing\footnote{\textbf{foregoing} [a] [only before noun] \textbf{1.} used to refer to something that has just been mentioned, \textsc{opposite}: \textbf{following}; \textbf{2.} (\textbf{the foregoing}) [n, singular $+$ singular or plural verb] what has just been mentioned.} section. We take the following simple problem: \textit{Find the diagonal of a rectangular parallelepiped of which the length, the width, \& the height are known}.

In order to discuss this problem profitably, the students must be familiar with the theorem of Pythagoras, \& with some of its applications in plane geometry, but they may have very little systematic knowledge in solid geometry. The teacher may rely here upon the student's unsophisticated\footnote{\textbf{unsophisticated} [a] \textbf{1.} not having or showing much experience of the world \&  social situations; \textbf{2.} simple \& basic; not complicated, \textsc{synonym}: \textbf{crude}, \textsc{opposite}: \textbf{sophisticated}.} familiarity\footnote{\textbf{familiarity} [n] \textbf{1.} [uncountable, singular] \textbf{familiarity with something} the state of knowing somebody\texttt{/}something well; the state of recognizing somebody\texttt{/}something; \textbf{2.} [uncountable] the fact of being well known to you.} with spatial\footnote{\textbf{spatial} [a] connected with space \& the position, size, shape, etc. of things in it.} relations.

The teacher can make the problem interesting by making it concrete\footnote{\textbf{concrete} [a] \textbf{1.} made of concrete; \textbf{2.} based on facts or actions, not on ideas, guesses or intentions; \textbf{3.} a concrete object is one that you can see \& feel, \textsc{opposite}: \textbf{abstract}; [n] [uncountable] building material that is made by mixing together cement, sand, small stones \& water.}. The classroom is a rectangular parallelepiped whose dimensions could be measured, \& can be estimated; the students have to find, to ``measure indirectly,'' the diagonal of the classroom. The teacher points out the length, the width, \& the height of the classroom, indicates the diagonal with a gesture, \& enlivens\footnote{\textbf{enliven} [v] (\textit{formal}) \textbf{enliven something} to make something more interesting or more fun.} his figure, drawn on the blackboard, by referring repeatedly to the classroom.

The dialogue between the teacher \& the students may start as follows:

\textit{``What is the unknown?''}

``The length of the diagonal of a parallelepiped.''

\textit{``What are the data?''}

``The length, the width, \& the height of the parallelepiped.''

\textit{Introduce suitable notation}. Which letter should denote the unknown?''

``$x$.''

``Which letters would you choose for the length, the width, \& the height?''

``$a,b,c$.''

\textit{``What is the condition}, linking $a,b,c$, \& $x$?''

``$x$ is the diagonal of the parallelepiped of which $a,b$, \& $c$ are the length, the width, \& the height.''

``Is it a reasonable problem? I mean, \textit{is the condition sufficient to determine the unknown?}''

``Yes, it is. If we know $a,b,c$, we know the parallelepiped. If the parallelepiped is determined, the diagonal is determined.'''' -- \cite[pp. 7--8]{Polya2014}

\section{Devising a plan}
``We have a plan when we know, or know at least in outline, which calculations, computations, or constructions we have to perform in order to obtain the unknown. The way from understanding the problem to conceiving\footnote{\textbf{conceive} [v] \textbf{1.} [transitive] to form an idea or plan in your mind; \textbf{2.} [transitive, intransitive] to think of something in a particular way; to imagine something; \textbf{3.} [intransitive, transitive] (of a woman) to become pregnant.} a plan may be long \& tortuous\footnote{\textbf{tortuous} [a] [usually before noun] (\textit{formal}) \textbf{1.} (\textit{usually disapproving}) not simple \& direct; long, complicated \& difficult to understand, \textsc{synonym}: \textbf{convoluted}; \textbf{2.} (of a road, path, etc.) full of bends, \textsc{synonym}: \textbf{winding}.}. In fact, the main achievement in the solution of a problem is to conceive the idea of a plan. This idea may emerge gradually. Or, after apparently\footnote{\textbf{apparently} [adv] according to what you have heard or read; according to the way something appears.} unsuccessful trials \& a period of hesitation\footnote{\textbf{hesitate} [v] \textbf{1.} [intransitive] \textbf{hesitate to do something} to be unwilling to do something, especially because you are not sure that it is right or appropriate; \textbf{2.} [intransitive] to be slow to speak or act because you feel uncertain or nervous.}, it may occur suddenly\footnote{\textbf{suddenly} [adv] quickly \& unexpectedly, \textsc{opposite}: \textbf{gradually}.}, in a flash\footnote{\textbf{flash} [n] \textbf{1.} \textbf{flash (of something)} a sudden bright light that shines for a moment \& then disappears; \textbf{2.} \textbf{flash of something} a particular feeling or idea that suddenly comes into your mind or shows in your face; [v] \textbf{1.} [intransitive, transitive] \textbf{flash (something)} to shine very brightly for a short time; to make something shine in this way; \textbf{2.} [intransitive, transitive] to appear on a television screen, computer screen, etc. for a short time; to make something do this.}, as a ``bright idea.'' The best that the teacher can do for the student is to procure\footnote{\textbf{procure} [v] \textbf{procure something (for somebody\texttt{/}something)} (\textit{formal}) to obtain something, especially with effort.} for him, by unobtrusive help, a bright idea. The questions \& suggestions we are going to discuss tend to provoke such an idea.

In order to be able to see the student's position, the teacher should think of his own experience, of his difficulties \& successes in solving problems.

We know, of course, that it is hard to have a good idea if we have little knowledge of the subject, \& impossible to have it if we have no knowledge. Good ideas are based on past experience \& formerly\footnote{\textbf{formerly} [adv] in earlier times.} acquired knowledge. Mere remembering is not enough for a good idea, but we cannot have any good idea without recollecting some pertinent\footnote{\textbf{pertinent} [a] appropriate to a particular situation, \textsc{synonym}: \textbf{relevant}.} facts; materials alone are not enough for constructing a house but we cannot construct a house without collecting the necessary materials. The materials necessary for solving a mathematical problem are certain relevant items of our formerly acquired mathematical knowledge, as formerly solved problems, or formerly proved theorems. Thus, it is often appropriate to start the work with the question: \textit{Do you know a related problem?}

The difficulty is that there are usually too many problems which are somewhat related to our present problem, i.e., have some point in common with it. How can we choose the one, or the few, which are really useful? There is a suggestion that puts our finger on an essential common point: \textit{Look at the unknown! \& try to think of a familiar problem having the same or a similar unknown}.

If we succeed in recalling a formerly solved problem which is closely related to our present problem, we are lucky. We should try to deserve such luck; we may deserve it by exploiting\footnote{\textbf{exploit} [v] \textbf{1.} \textbf{exploit something} to use something well in order to gain as much from it as possible; \textbf{2.} to develop or use something for business or industry; \textbf{3.} \textbf{exploit somebody\texttt{/}something (for something)} (\textit{disapproving}) to treat a person or situation as an opportunity to gain an advantage for yourself; \textbf{4.} \textbf{exploit somebody} (\textit{disapproving}) to treat somebody unfairly by making them work \& not giving them much in return.} it. \textit{Here is a problem related to yours \& solved before. Could you use it?}

The foregoing\footnote{\textbf{foregoing} [a] [only before noun] \textbf{1.} used to refer to something that has just been mentioned, \textsc{opposite}: \textbf{following}; \textbf{2.} (\textbf{the foregoing}) [n, singular $+$ singular or plural verb] what has just been mentioned.} questions, well understood \& seriously considered, very often help to start the right train of ideas; but they cannot help always, they cannot work magic. If they do not work, we must look around for some other appropriate point of contact, \& explore the various aspects of our problem; we have to vary, to transform, to modify the problem. \textit{Could you restate the problem?} Some of the questions of our list hint specific means to vary the problems, as generalization, specialization, use of analogy, dropping a part of the condition, \& so on; the details are important but we cannot go into them now. Variation of the problem may lead to some appropriate auxiliary problem: \textit{If you cannot solve the proposed problem try to solve 1st some related problem}.

Trying to apply various known problems or theorems, considering various modifications\footnote{\textbf{modification} [n] [uncountable, countable] the act or process of changing something in order to improve it or make it more suitable; a change that is made, \textsc{synonym}: \textbf{adaptation}.}, experimenting with various auxiliary problems, we may stray so far from our original problem that we are in danger of losing it altogether. Yet there is a good question that may bring us back to it: \textit{Did you use all the data? Did you use the whole condition?}'' -- \cite[pp. 8--9]{Polya2014}

\section{Example}
``We return to the example considered in Sect. 8. As we left it, the students just succeeded in understanding the problem \& showed some mild\footnote{\textbf{mild} [a] (\textbf{milder, mildest}) \textbf{1.} not severe or strong; \textbf{2.} (of weather) not very cold, \& therefore pleasant.} interest in it. They could now have some ideas of their own, some initiative\footnote{\textbf{initiative} [n] \textbf{1.} [countable] a new plan for dealing with a particular problem or for achieving a particular purpose; \textbf{2.} [uncountable] the ability to decide \& act on your own without waiting for somebody to tell you what to do; \textbf{3.} (\textbf{the initiative}) [singular] the power or opportunity to act before other people do.}. If the teacher, having watched sharply\footnote{\textbf{sharply} [adv] \textbf{1.} suddenly \& by a large amount; \textbf{2.} in a way that clearly shows the differences between 2 things; in a way that clearly emphasizes something; \textbf{3.} in a critical way; \textbf{4.} used to emphasize that something has a sharp point or edge.}, cannot detect\footnote{\textbf{detect} [v] to discover or notice something that is difficult to discover or notice.} any sign of such initiative he has to resume\footnote{\textbf{resume} [v] (\textit{formal}) [intransitive, transitive] (of an activity) to begin again after an interruption; to begin an activity again after an interruption.} carefully his dialogue with the students. He must be prepared to repeat with some modification the questions which the student do not answer. He must be prepared to meet often with the disconcerning silence of the students (which will be indicated by dots $\ldots\ldots$).

\textit{``Do you know a related problem?''}

$\ldots\ldots$

\textit{``Look at the unknown! Do you know a problem having the same unknown?''}

$\ldots\ldots$

``Well, \textit{what is the unknown?''}

``The diagonal of a parallelepiped.''

``Do you know any \textit{problem with the same unknown?''}

``No. We have not had any problem yet about the diagonal of a parallelepiped.''

``Do you know any \textit{problem with a similar unknown?''}

$\ldots\ldots$

``You see, the diagonal is a segment, the segment of a straight line. Did you never solve a problem whose unknown was the length of a line?''

``Of course, we have solved such problems. E.g., to find a side of a right triangle.''

``Good! \textit{Here is a problem related to yours \& solved before. Could you use it?''}

$\ldots\ldots$

``You were lucky enough to remember a problem which is related to your present one \& which you solved before. Would you like to use it? \textit{Could you introduce some auxiliary element in order to make its use possible?''}

$\ldots\ldots$

``Look here, the problem you remembered is about a triangle. Have you any triangle in your figure?''

Let us hope that the last hint was explicit enough to provoke the idea of the solution which is to introduce a right triangle, (emphasized in Fig. 1) of which the required diagonal is the hypotenuse\footnote{\textbf{hypotenuse} [n] (\textit{geometry}) the side opposite the right angle of a right-angled triangle.}. Yet the teacher should be prepared for the case that even this fairly\footnote{\textbf{fairly} [adv] \textbf{1.} (before adjectives \& adverbs) quite but not very; \textbf{2.} in a fair way; in a way that treats people equally \& according to the rules or law.} explicit\footnote{\textbf{explicit} [a] \textbf{1.} saying something clearly \& exactly; \textbf{2.} showing or referring to sex in a very obvious or detailed way.} hint is insufficient to shake the torpor\footnote{\textbf{torpor} [n] [uncountable, singular] (\textit{formal}) the state of not being active \& having no energy or enthusiasm, \textsc{synonym}: \textbf{lethargy}.} of the students; \& so he should be prepared to use a whole gamut\footnote{\textbf{the gamut} [n] [singular] the complete range of a particular kind of thing.} of more \& more explicit hints.

``Would you like to have a triangle in the figure?''

``What sort of triangle would you like to have in the figure?''

``You cannot find yet the diagonal; but you said that you could find the side of a triangle. Now, what will you do?''

``Could you find the diagonal, if it were a side of a triangle?''

When, eventually, with more or less help, the students succeed in introducing the decisive auxiliary element, the right triangle emphasized in Fig. 1, the teacher should convince himself that the students see sufficiently far ahead before encouraging them to go into actual calculations.

``I think that it was a good idea to draw that triangle. You have now a triangle; but have you the unknown?''

``The unknown is the hypotenuse of the triangle; we can calculate it by the theorem of Pythagoras.''

``You can, if both legs are known; but are they?''

``1 leg is given, it is $c$. \& the other, I think, is not difficult to find. Yes, the other leg is the hypotenuse of another right triangle.''

``Very good! Now I see that you have a plan.'''' -- \cite[pp. 10--12]{Polya2014}

\section{Carrying out the plan}
``To devise\footnote{\textbf{devise} [v] \textbf{devise something} to plan or invent a procedure, system or method, especially one that is new or complicated, by using careful thought, \textsc{synonym}: \textbf{think something up}.} a plan, to conceive the idea of the solution is not easy. It takes so much to succeed; formerly acquired knowledge, good mental habits, concentration upon the purpose, \& 1 more thing: good luck. To carry out the plan is much easier; what we need is mainly patience\footnote{\textbf{patience} [n] [uncountable] \textbf{1.} the ability to stay  calm \& accept delay, problems or suffering without complaining; \textbf{2.} the ability to spend a lot of time doing something difficult that needs a lot of attention \& effort.}.

The plan gives a general outline; we have to convince ourselves that the details fit into the outline, \& so we have to examine the details one after the other, patiently\footnote{\textbf{patient} [a] able to stay calm \& accept delay, problems or suffering without complaining; showing this, \textsc{opposite}: \textbf{impatient}.}, till everything is perfectly clear, \& no obscure\footnote{\textbf{obscure} [v] to cover something; to make it difficult to see, hear or understand something.} corner remains in which an error could be hidden.

If the student has really conceived a plan, the teacher has now a relatively peaceful time. The main danger is that the student forgets his plan. This may easily happen if the student received his plan from outside, \& accepted it on the authority of the teacher; but if he worked for it himself, even with some help, \& conceived the final idea with satisfaction, he will not lose this idea easily. Yet the teacher must insist\footnote{\textbf{insist} [v] \textbf{1.} [intransitive, transitive] to say firmly that something is true, especially when other people do not believe you; \textbf{2.} [intransitive, transitive] to demand that something happens or that somebody agrees to do something; \textbf{insist on\texttt{/}upon something} [idiom] to demand something \& refuse to be persuaded to accept anything else.} that the student should \textit{check each step}.

We may convince ourselves of the correctness of a step in our reasoning either ``intuitively\footnote{\textbf{intuitively} [adv] by using feelings rather than by considering facts.}'' or ``formally\footnote{\textbf{formally} [adv] \textbf{1.} officially, \textsc{opposite}: \textbf{informally}; \textbf{2.} in the way that something appears or is presented; \textbf{3.} \textbf{formally trained\texttt{/}educated\texttt{/}taught} trained, etc. in a school, college or other institution.}.'' We may concentrate upon the point in question till we see it so clearly \& distinctly that we have no doubt that the step is correct; or we may derive the point in question according to formal rules. (The difference between ``insight'' \& ``formal proof'' is clear enough in many important cases; we may leave further discussion to philosophers.)

The main point is that the student should be honestly convinced of the correctness of each step. In certain cases, the teacher may emphasize the difference between ``seeing'' \& ``proving'' \textit{Can you see clearly that the step is correct?} But can you also \textit{prove that the step is correct?}'' -- \cite[pp. 12--13]{Polya2014}

\section{Example}
``Let us resume our work at the point where we left it at the end of Sect. 10. The student, at last, has got the idea of the solution. He sees the right triangle of which the unknown $x$ is the hypotenuse \& the given height $c$ is 1 of the legs; the other leg is the diagonal of a face. The student must, possibly, be urged to introduce suitable notation. He should choose $y$ to denote that other leg, the diagonal of the face whose sides are $a$ \& $b$. Thus, he may see more clearly the idea of the solution which is to introduce an auxiliary problem whose unknown is $y$. Finally, working at 1 right triangle after the other, he may obtain (see Fig. 1) $x^2 = y^2 + c^2$, $y^2 = a^2 + b^2$, \& hence, eliminating the auxiliary unknown $y$, $x^2 = a^2 + b^2 + c^2$, $x = \sqrt{a^2 + b^2 + c^2}$.

The teacher has no reason to interrupt the student if he carries out these details correctly except, possibly, to warn him that he should \textit{check each step}. Thus, the teacher may ask:

``Can you \textit{see clearly} that the triangle with sides $x,y,c$ is a right triangle?''

To this question the student may answer honestly ``Yes'' but he could be much embarrassed if the teacher, not satisfied with the intuitive conviction\footnote{\textbf{conviction} [n] \textbf{1.} [countable, uncountable] the act of finding somebody guilty of a crime in court; the fact of having been found guilty; \textbf{2.} [countable, uncountable] a strong opinion or belief; \textbf{3.} [uncountable] the feeling of believing something strongly \& of being sure about it.} of the student, should go on asking:

``But can you \textit{prove} that this triangle is a right triangle?''

Thus, the teacher should rather suppress\footnote{\textbf{suppress something} (of a government or ruler) to stop something by force, especially an activity or group that is believed to threaten authority; \textbf{2.} \textbf{suppress something} to prevent something from growing, developing or continuing; \textbf{3.} \textbf{suppress something} (\textit{usually disapproving}) to prevent something from being published or made known; \textbf{4.} \textbf{suppress something} to prevent yourself from having or expressing a feeling or an emotion.} this question unless the class has had a good initiation\footnote{\textbf{initiation} [n] [uncountable] \textbf{1.} the act of starting something; \textbf{2.} \textbf{initiation (into something)} the act of somebody becoming a member of a group, often with a special ceremony; the act of introducing somebody to an activity or a skill.} in solid geometry. Even in the latter case, there is some danger that the answer to an incidental\footnote{\textbf{incidental} [a] \textbf{1.} \textbf{incidental (to something)} happening in connection with something else, but not as important as it, or not intended; \textbf{2.} \textbf{incidental to something} (\textit{specialist}) happening as a natural result of something.} question may become the main difficulty for the majority of the students.'' -- \cite[pp. 13--14]{Polya2014}

\section{Looking back}
``Even fairly good students, when they have obtained the solution of the problem \& written down neatly\footnote{\textbf{neatly} [a] (\textbf{neater, neatest}) \textbf{1.} in good order; carefully done or arranged; \textbf{2.} simple but clever; \textbf{3.} containing or made out of just 1 substance; not mixed with anything else.} the argument, shut their books \& look for something else. Doing so, they miss an important \& instructive\footnote{\textbf{instructive} [a] giving a lot of useful information.} phase of the work. By looking back at the completed solution, by reconsidering\footnote{\textbf{reconsider} [v] [transitive, intransitive] to think about something again, especially because you might want to change a previous decision or opinion.} \& reexamining\footnote{\textbf{reexamine} [v] \textbf{reexamine something} to examine or think about something again, especially because you may need to change your opinion, \textsc{synonym}: \textbf{reassess}.} the result \& the path that led to it, they could consolidate\footnote{\textbf{consolidate} [v] \textbf{1.} \textbf{consolidate something} to make a position of power or success stronger so that it is more likely to continue; \textbf{2.} (\textit{specialist}) to join things together into a single more effective whole; to join financial accounts or sums of money into a single overall account or sum.} their knowledge \& develop their ability to solve problems. A good teacher should understand \& impress\footnote{\textbf{impress} [v] [transitive, intransitive] if a person or thing impresses you, you feel admiration for them or it; \textbf{impress something on\texttt{/}upon somebody} [phrasal verb] to make somebody understand how important something is by emphasizing it; \textbf{impress something\texttt{/}itself on\texttt{/}upon somebody} [phrasal verb] to have a great effect on something, especially somebody's mind or imagination.} on his students the view that no problem whatever is completely exhausted\footnote{\textbf{exhausted} [a] \textbf{1.} completely used or finished; \textbf{2.} very tired.}. There remains always something to do; with sufficient study \& penetration, we could improve any solution, \&, in any case, we can always improve our understanding of the solution.

The student has now carried through his plan. He has written down the solution, checking each step. Thus, he should have good reasons to believe that his solution is correct. Nevertheless, errors are always possible, especially if the argument is long \& involved. Hence, verifications\footnote{\textbf{verification} [n] [uncountable] (\textit{formal}) the act of showing or checking that something is true or accurate, \textsc{synonym}: \textbf{confirmation}.} are desirable\footnote{\textbf{desirable} [a] that you would like to have or do; worth having or doing.}. Especially, if there is some rapid \& intuitive procedure to test either the result or the argument, it should not be overlooked\footnote{\textbf{overlook} [v] \textbf{1.} \textbf{overlook something} to fail to see or notice something, \textsc{synonym}: \textbf{miss}; \textbf{2.} \textbf{overlook something} if a building, etc. overlooks a place, you can see that place from the building; \textbf{3.} \textbf{overlook somebody (for something)} to not consider somebody for a job or position, even though they might be suitable.}. \textit{Can you check the result? Can you check the argument?}

In order to convince ourselves of the presence or of the quality of an object, we like to see \& to touch it. \& as we prefer perception\footnote{\textbf{perception} [n] \textbf{1.} [uncountable, countable] an idea, a belief or an image you have as a result of how you see or understand something; \textbf{2.} [uncountable] the way you notice things or the ability to notice things with the senses. In biology, \textbf{perception} refers to the process in the nervous system by which a living thing becomes aware of events \& things outside itself.; \textbf{3.} [uncountable] the ability to understand the true nature of something, \textsc{synonym}: \textbf{insight}.} through 2 different senses, so we prefer conviction by 2 different proofs: \textit{Can you derive the result differently?} We prefer, of course, a short \& intuitive argument to a long \& heavy one: \textit{Can you see it at a glance?}

1 of the 1st \& foremost\footnote{\textbf{foremost} [a] the most important or famous; in a position at the front; [adv] \textbf{1st \& foremost} [idiom] more than anything else.} duties of the teacher is not to give his students the impression that mathematical problems have little connection with each other, \& no connection at all with anything else. We have a natural opportunity to investigate the connections of a problem when looking back at its solution. The students will find looking back at the solution really interesting if they have made an honest effort, \& have the consciousness\footnote{\textbf{consciousness} [n] [uncountable] \textbf{1.} the state of being able to use your senses \& mental powrs to understand what is happening; \textbf{2.} the state of being aware of something, \textsc{synonym}: \textbf{awareness}; \textbf{3.} the ideas \& opinions of a person or group.} of having done well. Then they are eager\footnote{\textbf{eager} [a] very interested \& excited by something that is going to happen or about something that you want to do, \textsc{synonym}: \textbf{keen}.} to see what else they could accomplish with that effort, \& how they could do equally well another time. The teacher should encourage the students to imagine cases in which they could utilize again the procedure used, or apply the result obtained. \textit{Can you use the result, or the method for some other problem?}'' -- \cite[pp. 14--16]{Polya2014}

\section{Example}
``In Sect. 12, the students finally obtained the solution: If the 3 edges of a rectangular parallelogram, issued from the same corner, are $a,b,c$, the diagonal is $\sqrt{a^2 + b^2 + c^2}$.

\textit{Can you check the result?} The teacher cannot expect a good answer to this question from inexperienced students. The students, however, should acquire fairly early the experience that problems ``in letters'' have a great advantage over purely numerical problems; if the problem is given ``in letters'' its result is accessible to several tests to which a problem ``in numbers'' is not susceptible at all. Our example, although fairly simple, is sufficient to show this. The teacher can ask several questions about the result which the students may readily\footnote{\textbf{readily} [adv] \textbf{1.} quickly \& without difficulty, \textsc{synonym}: \textbf{freely}; \textbf{2.} in a way that shows that you do not object to something, \textsc{synonym}: \textbf{willingly}.} answer with ``Yes''; but an answer ``No'' would show a serious flaw in the result.

\textit{Did you use all the data?} Do all the data $a,b,c$ appear in your formula for the diagonal?''

``Length, width, \& height play the same role in our question; our problem is symmetric w.r.t. $a,b,c$. Is the expression you obtained for the diagonal symmetric in $a,b,c$? Does it remain unchanged when $a,b,c$ are interchanged?

``Our problem is a problem of solid geometry: to find the diagonal of a parallelepiped with given dimensions $a,b,c$. Our problem is analogous to a problem of plane geometry: to find the diagonal of a rectangle with given dimensions $a,b$. Is the result of our `solid' problem analogous to the result of the `plane' problem?''

``If the height $c$ decreases, \& finally vanishes, the parallelepiped becomes a parallelogram. If you put $c = 0$ in your formula, do you obtain the correct formula for the diagonal of the rectangular parallelogram?''

``If the height $c$ increases, the diagonal increases. Does your formula show this?''

``If all 3 measures $a,b,c$ of the parallelepiped increase in the same proportion, the diagonal also increases in the same proportion, the diagonal also increases in the same proportion. If, in your formula, you substitute $12a,12b,12c$ for $a,b,c$ respectively, the expression of the diagonal, owing to this substitution, should also be multiplied by $12$. Is that so?''

``If $a,b,c$ are measured in feet, your formula gives the diagonal measured in feet too; but if you change all measures into inches, the formula should remain correct. Is that so?''

(The 2 last questions are essentially equivalent; see \textsc{test by dimension}.)

These questions have several good effects. 1st, an intelligent student cannot help being impressed by the fact that the formula passes so many tests. He was convinced before that the formula is correct because he derived it carefully. But now he is more convinced, \& his gain in confidence comes from a different source; it is due to a sort of ``experimental evidence.'' Then, thanks to the foregoing questions, the details of the formula acquire new significance, \& are linked up with various facts. The formula has therefore a better chance of being remembered, the knowledge of the student is consolidated. Finally, these questions can be easily transferred to similar problems. After some experience with similar problems, an intelligent student may perceive the underlying general ideas: use of all relevant data, variation of the data, symmetry, analogy. If he gets into the habit of directing his attention to such points, his ability to solve problems may definitely profit.

\textit{Can you check the argument?} To recheck the argument step by step may be necessary in difficult \& important cases. Usually, it is enough to pick out ``touchy'' points for rechecking. In our case, it may be advisable to discuss retrospectively the question which was less advisable to discuss as the solution was not yet attained: Can you \textit{prove} that the triangle with sides $x,y,c$ is a right triangle (See the end of Sect. 12.)

\textit{Can you use the result or the method for some other problem?} With a little encouragement, \& after 1 or 2 examples, the students easily find applications which consist essentially in giving some \textit{concrete interpretation} to the abstract mathematical elements of the problem. The teacher himself used such a concrete interpretation as he took the room in which the discussion takes place for the parallelepiped of the problem. A dull student may propose, as application, to calculate the diagonal of the cafeteria instead of the diagonal of the classroom. If the students do not volunteer more imaginative remarks, the teacher himself may put a slightly different problem, e.g.: ``Being given the length, the width, \& the height of a rectangular parallelepiped, find the distance of the center from 1 of the corners.''

The students may use the \textit{result} of the problem they just solved, observing that the distance required is $\frac{1}{2}$ of the diagonal they just calculated. Or they may use the \textit{method}, introducing suitable right triangles (the latter alternative is less obvious \& somewhat more clumsy in the present case).

After this application, the teacher may discuss the configuration of the 4 diagonals of the parallelepiped, \& the 6 pyramids of which the 6 faces are the bases, the center the common vertex, \& the seidiagonals the lateral edges. When the geometric imagination of the students is sufficiently enlivened, the teacher should come back to his question: \textit{Can you use the result, or the method, for some other problem?} Now there is a better chance that the students may find some more interesting concrete interpretation, e.g., the following:

``In the center of the flat rectangular top of  building which is 21 yards long \& 16 yards wide, a flagpole\footnote{\textbf{flagpole} [n] (also \textbf{flagstaff}) a tall thin straight piece of wood or metal on which a flag is hung.} is to be erected\footnote{\textbf{erect} [v] \textbf{1.} \textbf{erect something} to build something; \textbf{2.} \textbf{erect something} to create or establish something.}, 8 yards high. To support the pole, we need 4 equal cables. The tables should start from the same point, 2 yards under the top of the pole, \& end at the 4 corners of the top of the building. How long is each cable?''

The students may use the \textit{method} of the problem they solved in detail introducing a right triangle in a vertical plane, \& another one in a horizontal plane. Or they may use the \textit{result}, imagining a rectangular parallelepiped of which the diagonal, $x$, is 1 of the 4 cables \& the edges are $a = 10.5$, $b = 8$, $c = 6$. By straightforward application of the formula, $x = 14.5$.

For more examples, see \textsc{can you use the result?}'' -- \cite[pp. 16--19]{Polya2014}

\section{Various approaches}
``Let us still retain\footnote{\textbf{retain} [v] \textbf{1.} \textbf{retain somebody\texttt{/}something} to keep somebody\texttt{/}something; to continue to have something \& not lose it or get rid of it; \textbf{2.} \textbf{retain something} to take in a substance \& keep holding it; \textbf{3.} \textbf{retain something} to remember or continue to hold something; \textbf{4.} \textbf{retain somebody\texttt{/}something} (\textit{law}) to employ a professional person such as a lawyer or doctor; to make regular payments to such a person in order to keep their services.}, for a while, the problem we considered in the foregoing Sects. 8, 10, 12, 14. The main work, the discovery of the plan, was described in Sect. 10. Let us observe that the teacher could have proceeded differently. Starting from the same point as in Sect. 10, he could have followed a somewhat different line, asking the following questions:

\textit{``Do you know any related problem?''}

``Do you know an \textit{analogous} problem?''

``You see, the proposed problem is a problem of solid geometry. Could you think of a simpler analogous problem of plane geometry?''

``You see, the proposed problem is about a figure in space, it is concerned with the diagonal of a rectangular parallelepiped. What might be an analogous problem about a figure in the plane? It should be concerned with -- the diagonal--of--a rectangular--''

``Parallelogram.''

The students, even if they are very slow \& indifferent\footnote{\textbf{indifferent} [a] [not usually before noun] \textbf{indifferent (to somebody\texttt{/}something)} having or showing no interest in somebody\texttt{/}something.}, \& were not able to guess anything before, are obliged\footnote{\textbf{oblige} [v] [transitive, usually passive] to make somebody do something, by law or because it is a rule or a duty.} finally to contribute at least a minute part of the idea. Besides, if the students are so slow, the teacher should not take up the present problem about the parallelepiped without having discussed before, in order to prepare the students, the analogous problem about the parallelogram. Then, he can go on now as follows:

``Here is a problem related to yours \& solved before. Can you use it?''

``Should you introduce some auxiliary element in order to make its use possible?''

Eventually, the teacher may succeed in suggesting to the students the desirable idea. It consists in conceiving the diagonal of the given parallelepiped as the diagonal of a suitable parallelogram which must be introduced into the figure (as intersection of the parallelepiped with a plane passing through 2 opposite edges). The idea is essentially the same as before (Sect. 10) but the approach is different. In Sect. 10, the contact with the available knowledge of the students was established through the unknown; a formerly solved problem was recollected because its unknown was the same as that of the proposed problem. In the present section analogy provides the contact with the idea of the solution.'' -- \cite[pp. 19--20]{Polya2014}

\section{The teacher's method of questioning}
``shown in the foregoing Sects. 8, 10, 12, 14, 15 is essentially this: Begin with a general question or suggestion of our list, \&, if necessary, come down gradually to more specific \& concrete questions or suggestions till you reach one which elicits\footnote{\textbf{elicit} [v] to get information or a reaction from somebody\texttt{/}something.} a response in the student's mind. If you have to help the student exploit his idea, start again, if possible, from a general question or suggestion contained in the list, \& return again to some more special one if necessary; \& so on.

Of course, our list is just a 1st list of this kind; it seems to be sufficient for the majority of simple cases, but there is no doubt that it could be perfected. It is important, however, that the suggestions from which we start should be simple, natural, \& general, \& that there list should be short.

The suggestions must be simple \& natural because otherwise they cannot be \textit{unobtrusive}.

The suggestions must be general, applicable not only to the present problem but to problems of all sorts, if they are to help develop the \textit{ability} of the student \& not just a special technique.

The list must be short in order that the questions may be often repeated, unartificially\footnote{\textbf{unartificially} [adv]}, \& under varying circumstances; thus, there is a chance that they will be eventually assimilated\footnote{\textbf{assimilate} [v] \textbf{1.} [intransitive, transitive] to become a part of a country or community rather than remaining in a separate group; to allow or cause people to do this; \textbf{2.} [transitive] \textbf{assimilate something} (of the body or any biological system) to absorb or take in a substance; \textbf{3.} [transitive] \textbf{assimilate something} to think deeply about something \& understand it fully, so that you can use it, \textsc{synonym}: \textbf{absorb}; \textbf{4.} [transitive, often passive] \textbf{assimilate something (into\texttt{/}to something)} to accept an idea, information or activity; to make it fit into something.} by the student \& will contribute to the development of a \textit{mental habit}.

It is necessary to come down gradually to specific suggestions, in order that the student may have as great a \textit{share of the work} as possible.

This method of questioning is not a rigid one; fortunately so, because, in these matters, any rigid, mechanical, pedantical\footnote{\textbf{pedantic} [a] (\textit{disapproving}) too worried about small details or rules.} procedure is necessarily bad. Our method admits a certain elasticity \& variation, it admits various approaches (Sect. 15), it can be \& should be so applied that questions asked by the teacher \textit{could have occurred to the student himself}.

If a reader wishes to try the method here proposed in his class he should, of course, proceed with caution. He should study carefully the example introduced in Sect. 8, \& the following examples in Sects. 18--20. He should prepare carefully the examples which he intends to discuss, considering also various approaches. He should start with a few trials \& find out gradually how he can manage the method, how the students take it, \& how much time it takes.'' -- \cite[pp. 20--22]{Polya2014}

\section{Good questions \& bad questions}
``If the method of questioning formulated in the foregoing section is well understood it helps to judge, by comparison, the quality of certain suggestions which may be offered with the intention of helping the students.

Let us go back to the situation as it presented itself at the beginning of Sect. 10 when the question was asked: \textit{Do you know a related problem?} Instead of this, with the best intention to help the students, the question may be offered: \textit{Could you apply the theorem of Pythagoras?}

The intention may be the best, but the question is about the worst. We must realize in what situation it was offered; then we shall see that there is a long sequence of objections against that sort of ``help.''
\begin{enumerate}
	\item If the student is near to the solution, he may understand the suggestion implied by the question; but if he is not, he quite possibly will not see at all the point at which the question is driving. Thus the question fails to help where help is most needed.
	\item If the suggestion is understood, it gives the whole secret away, very little remains for the student to do.
	\item The suggestion is of too special a nature. Even if the student can make use of it in solving the present problem, nothing is learned for future problems. The question is not instructive.
	\item Even if he understands the suggestion, the student can scarcely\footnote{\textbf{scarcely} [adv] only just; almost not, \textsc{synonym}: \textbf{hardly}.} understand how the teacher came to the idea of putting such a question. \& how could he, the student, find such a question by himself? It appears as an unnatural\footnote{\textbf{unnatural} [a] \textbf{1.} different from what is normal or expected, or from what is generally accepted as being right, \textsc{opposite}: \textbf{natural, normal}; \textbf{2.} different from anything in nature, \textsc{opposite}: \textbf{natural}.} surprise, as a rabbit pulled out of a hat; it is really not instructive.
\end{enumerate}
None of these objections can be raised against the procedure described in Sect. 10, or against that in Sect. 15.'' -- \cite[pp. 22--23]{Polya2014}

\begin{center}
	\LARGE More examples
\end{center}

\section{A problem of construction}
``Inscribe\footnote{\textbf{inscribe} [v] \textbf{1.} [often passive] to write or cut words, your name, etc. onto something; \textbf{2.} [often passive] \textbf{inscribe something $+$ adv.\texttt{/}prep.} to make something present in, on, etc. something.} a square in a given triangle. 2 vertices of the square should be on the base of the triangle, the 2 other vertices of the square on the 2 other sides of the triangle, one on each.

\textit{``What is the unknown?''}

``A square.''

\textit{``What are the data?''}

``A triangle is given, nothing else.''

\textit{``What is the condition?''}

``The 4 corners of the square should be on the perimeter of the triangle, 2 corners on the base, 1 corner on each of the other 2 sides.''

\textit{``Is it possible to satisfy the condition?''}

``I think so. I am not so sure.''

``You do not seem to find the problem too easy. \textit{If you cannot solve the proposed problem, try to solve 1st some related problem}. Could you satisfy a \textit{part of the condition?''}

``You see, the condition is concerned with all the vertices of the square. How many vertices are there?''

``4.''

``A part of the condition would be concerned with $< 4$ vertices. \textit{Keep only a part of the condition, drop the other part}. What part of the condition is easy to satisfy?''

``It is easy to draw a square with 2 vertices on the perimeter of the triangle -- or even one with 3 vertices on the perimeter!''

\textit{``Draw a figure!''}

The student draws Fig. 2 (\cite[p. 24]{Polya2014})

\textit{``You kept only a part of the condition}, \& you \textit{dropped the other part. How far is the unknown now determined?''}

``The square is not determined if it has only 3 vertices on the perimeter of the triangle.''

``Good! \textit{Draw a figure.''}

The student draws Fig. 3 (\cite[p. 24]{Polya2014})

``The square, as you said, is not determined by the \textit{part of the condition you kept. How can it vary?''}

$\ldots\ldots$

``3 corners of your square are on the perimeter of the triangle but the 4th corner is not yet there where it should be. Your square, as you said, is undetermined, it can vary; the same is true of its 4th corner. \textit{How can it vary?''}

$\ldots\ldots$

``Try it experimentally, if you wish. Draw more squares with 3 corners on the perimeter in the same way as the 2 squares already in the figure. Draw small squares \& large squares. What seems to be the locus\footnote{\textbf{locus} [n] (plural \textbf{loci}) \textbf{1.} \textbf{locus (of something)} (\textit{formal}) the exact place or position where something happens or which is thought to be the center of something; \textbf{2.} (\textit{biology}) the position of a gene or mutation on a chromosome; \textbf{3.} \textbf{locus (of something)} the set of all points that share a particular property.} of the 4th corner? \textit{How can it vary?''}

The teacher brought the student very near to the idea of the solution. If the student is able to guess that the locus of the 4th corner is a straight line, he has got it.'' -- \cite[pp. 23--25]{Polya2014}

\section{A problem to prove}
``\textit{2 angles are in different planes but each side of one is parallel to the corresponding side of the other, \& has also the same direction. Prove that such angles are equal.}

What we have to prove is a fundamental theorem of solid geometry. The problem may be proposed to students who are familiar with plane geometry \& acquainted with those few facts of solid geometry which prepare the present theorem in Euclid's Elements. (The theorem that we have stated \& are going to prove is the proposition 10 of Book XI of Euclid.) Not only questions \& suggestions quoted from our list are printed in italics but also others which correspond to them as ``problems to prove'' correspond to ``problems to find.'' (The correspondence is worked out systematically in \textsc{problems to find, problems to prove 5, 6}.)

\textit{``What is the hypothesis?''}

``2 angles are in different planes. Each side of one is parallel to the corresponding side of the other, \& has also the same direction.

\textit{``What is the conclusion?''}

``The angles are equal.''

\textit{``Draw a figure. Introduce suitable notation.''}

The student draws the lines of Fig. 4 \& chooses, helped more or less by the teacher, the letters as in Fig. 4.

\textit{``What is the hypothesis?} Say it, please, using your notation.''

``$A,B,C$ are not in the same plane as $A',B',C'$. \& $AB||A'B'$, $AC||A'C'$. Also $AB$ has the same direction as $A'B'$, \& $AC$ the same as $A'C'$.''

\textit{``What is the conclusion?''}

``$\angle BAC = \angle B'A'C'$.''

\textit{``Look at the conclusion! \& try to think of a familiar theorem having the same or a similar conclusion.''}

``If 2 triangles are congruent\footnote{\textbf{congruent} [a] \textbf{1.} (\textit{geometry}) having the same size \& shape; \textbf{2.} \textbf{congruent (with something)} (\textit{formal}) in agreement with something; similar to something \& not in conflict with it, \textsc{synonym}: \textbf{compatible}.}, the corresponding angles are equal.''

``Very good! Now \textit{here is a theorem related to yours \& proved before. Could you use it?''}

``I think so but I do not see yet quite how.''

\textit{``Should you introduce some auxiliary element in order to make its use possible?''}

$\ldots\ldots$

``Well, the theorem which you quoted so well is about triangles, about a pair of congruent triangles. Have you any triangles in your figure?''

``No. But I could introduce some. Let me join $B$ to $C$, \& $B'$ to $C'$. Then there are 2 triangles, $\Delta ABC$, $\Delta A'B'C'$.''

``Well done. But what are these triangles good for?''

``To prove the conclusion, $\angle BAC = \angle B'A'C'$.''

``Good! If you wish to prove this, what kind of triangles do you need?''

``Congruent triangles. Yes, of course, I may choose $B,C,B',C'$ so that $AB = A'B',AC = A'C'$.

``Very good! Now, what do you wish to prove?''

``I wish to prove that the triangles are congruent, $\Delta ABC = \Delta A'B'C'$.

If I could prove this, the conclusion $\angle BAC = \angle B'A'C'$ would follow immediately.''

``Fine! You have a new aim, you aim at a new conclusion. \textit{Look at the conclusion! \& try to think of a familiar theorem having the same or a similar conclusion.''}

``2 triangles are congruent if---if the 3 sides of the one are equal respectively to the 3 sides of the other.''

``Well done. You could have chosen a worse one. Now \textit{here is a theorem related to yours \& proved before. Could you use it?''}

``I could use it if I knew that $BC = B'C'$.''

``That is right! Thus, what is your aim?''

``To prove that $BC = B'C'$.''

\textit{``Try to think of a familiar theorem having the same or a similar conclusion.''}

``Yes, I know a theorem finishing: `$\ldots$ then the 2 lines are equal.' But it does not fit in.''

\textit{``Should you introduce some auxiliary element in order to make its use possible?''}

$\ldots\ldots$

``You see, how could you prove $BC = B'C'$ when there is no connection in the figure between $BC$ \& $B'C'$?''

$\ldots\ldots$

\textit{``Did you use the hypothesis? What is the hypothesis?''}

``We suppose that $AB||A'B',AC||A'C'$. Yes, of course, I must use that.''

\textit{``Did you use the whole hypothesis?} You say that $AB||A'B'$. Is that all that you know about these lines?''

``No; $AB$ is also equal to $A'B'$, by construction. They are parallel \& equal to each other. \& so are $AC$ \& $A'C'$.''

``2 parallel lines of equal length -- it is an interesting configuration. \textit{Have you seen it before?''}

``Of course! Yes! Parallelogram! Let me join $A$ to $A'$, $B$ to $B'$, \& $C$ to $C'$.''

``The idea is not so bad. How many parallelograms have you now in your figure?''

``2. No, 3. No, 2. I mean, there are 2 of which you can prove immediately that they are parallelograms. There is a 3rd which seems to be a parallelogram; I hope I can prove that it is one. \& then the proof will be finished!''

We could have gathered from this foregoing answers that the student is intelligent. But after this last remark of his, there is no doubt.

This student is able to guess a mathematical result \& to distinguish clearly between proof \& guess. He knows also that guesses can be more or less plausible. Really, he did profit something from his mathematics classes; he has some real experience in solving problems; he can conceive \& exploit a good idea.'' -- \cite[pp. 25--29]{Polya2014}

\section{A rate problem}
``\textit{Water is flowing into a conical\footnote{\textbf{conical} [a] shaped like a cone.} vessel at the rate $r$. The vessel has the shape of a right circular cone, with horizontal base, the vertex pointing downwards; the radius of the base is $a$, the altitude of the cone $b$. Find the rate at which the surface is rising when the depth of the water is $y$. Finally, obtain the numerical value of the unknown supposing that $a = 4$ ft., $b = 3$ ft., $r = 2$ cu. ft. per minute, \& $y = 1$ ft.}

The students are supposed to know the simplest rules of differentiation \& the notion of ``rate of change.''

\textit{``What are the data?''}

``The radius of the base of the cone $a = 4$ ft., the altitude of the cone $b = 3$ ft., the rate at which the water is flowing into the vessel $r = 2$ cu. ft. per minute, \& the depth of the water at a certain moment, $y = 1$ ft.''

``Correct. The statement of the problem seems to suggest that you should disregard, provisionally\footnote{\textbf{provisional} [a] made for the present time, possibly to be changed when more information is available or when a more permanent arrangement can be made.}, the numerical values, work with the letters, express the unknown in terms of $a,b,r,y$ \& only finally, after having obtained the expression of the unknown in letters, substitute the numerical values. I would follow this suggestion. Now, \textit{what is the unknown?''}

``The rate at which the surface is rising when the depth of the water is $y$.''

``What is that? Could you say it in other terms?''

``The rate at which the depth of the water is increasing.''

``What is that? \textit{Could you restate it still differently?''}

``The rate of change of the depth of the water.''

``That is right, the rate of change of $y$. But what is the rate of change? \textit{Go back to the definition.''}

``The derivative is the rate of change of a function.''

``Correct. Now, is $y$ a function? As we said before, we disregard the numerical value of $y$. Can you imagine that $y$ changes?''

``Yes, $y$, the depth of the water, increases as the time goes by.''

``Thus, $y$ is a function of what?''

``Of the time $t$.''

``Good. \textit{Introduce suitable notation}. How would you press it in terms of $a,b,r,y$. By the way, 1 of these data is a `rate.' Which one?''

``$r$ is the rate at which water is flowing into the vessel.''

``What is that? Could you say it in other terms?''

``$r$ is the rate of change of the volume of the water in the vessel.''

``What is that? \textit{Could you restate it still differently?} How would you write it in \textit{suitable notation?''}

``$r = \frac{dV}{dt}$.''

``What is $V$?''

``The volume of the water in the vessel at the time $t$.''

``Good. Thus, you have to express $\frac{dy}{dt}$ in terms of $a,b,\frac{dV}{dt},y$. How will you do it?''

$\ldots\ldots$

\textit{``If you cannot solve the proposed problem try to solve 1st some related problem}. If you do not see yet the connection between $\frac{dy}{dt}$ \& the data, try to bring in some simpler connection that could serve as a stepping stone.''

$\ldots\ldots$

``Do you not see that there are other connections? E.g., are $y$ \& $V$ independent of each other?''

``No. When $y$ increases, $V$ must increase too.''

``Thus, there is a connection. What is the connection?''

``Well, $V$ is the volume of a cone of which the altitude is $y$. But I do not know yet the radius of the base.''

``You may consider it, nevertheless. Call it something, say $x$.''

``$V = \frac{\pi x^2y}{3}$.''

``Correct. Now, what about $x$? Is it independent of $y$?''

``No. When the depth of the water, $y$, increases the radius of the free surface, $x$, increases too.''

``Thus, there is a connection. What is the connection?''

``Of course, similar triangles. $x:y = a:b$.''

``1 more connection, you see. I would not miss profiting from it. Do not forget, you wished to know the connection between $V$ \& $y$.''

``I have $x = \frac{ay}{b}$, $V = \frac{\pi a^2y^3}{3^{b^2}}$.''

``Very good. This looks like a stepping stone, does it not? But you should not forget your goal. \textit{What is the unknown?''}

``Well, $\frac{dy}{dt}$.''

``You have to find a connection between $\frac{dy}{dt},\frac{dV}{dt}$, \& other quantities. \& here you have one between $y,V$, \& other quantities. What to do?''

``Differentiate! Of course! $\frac{dV}{dt} = \frac{\pi a^2y^2}{b^2}\frac{dy}{dt}$. Here it is.''

``Fine! \& what about the numerical values?''

``If $a = 4$, $b = 3$, $\frac{dV}{dt} = r = 2$, $y = 1$, then $2 = \frac{\pi\cdot 16\cdot 1}{9}\frac{dy}{dt}$.'''' -- \cite[pp. 29--32]{Polya2014}

\begin{center}
	\huge Part II. How to Solve It
\end{center}

\section{A dialogue}

\subsection{Getting Acquainted}
``\textit{Where should I start?} Start from the statement of the problem.

\textit{What can I do?} Visualize the problem as a whole as clearly \& as vividly\footnote{\textbf{vivid} [a] \textbf{1.} (of memories, a description, etc.) producing very clear pictures in your mind, \textsc{synonym}: \textbf{graphic}; \textbf{2.} (of light, colors, etc.) very bright.} as you can. Do not concern yourself with details for the moment.

\textit{What can I gain by doing so?} You should understand the problem, familiarize yourself with it, impress its purpose on your mind. The attention bestowed\footnote{\textbf{bestow} [v] (\textit{formal}) \textbf{bestow something (on\texttt{/}upon somebody)} to give something to somebody, especially to show how much they are respected.} on the problem may also stimulate\footnote{\textbf{stimulate} [v] \textbf{1.} \textbf{stimulate something} to make something develop or become more active, especially in a positive way; \textbf{2.} to make somebody interested  \& excited about something; \textbf{3.} \textbf{stimulate something} (\textit{biology}) to make a part of the body function.} your memory \& prepare for the recollection\footnote{\textbf{recollection} [n] (\textit{formal}) \textbf{1.} [uncountable] the ability to remember something; the act of remembering something, \textsc{synonym}: \textbf{memory}; \textbf{2.} [countable] a thing that you remember from the past, \textsc{synonym}: \textbf{memory}.} of relevant points.'' -- \cite[p. 33]{Polya2014}

\subsection{Working for Better Understanding}
``\textit{Where should I start?} Start again from the statement of the problem. Start when this statement is so clear to you \& so well impressed on your mind that you may lose sight of it for a while without fear of losing it altogether.

\textit{What can I do?} Isolate\footnote{\textbf{isolate} [v] \textbf{1.} to separate somebody\texttt{/}something physically or socially from other people or things; \textbf{2.} to separate a single substance, cell, etc. from other so that you can study it; \textbf{3.} to separate a part of a situation, a problem or an idea so that you can see what it is \& deal with it separately.} the principal parts of your problem. The hypothesis \& the conclusion are the principal parts of a ``problem to prove''; the unknown, the data, \& the conditions are the principal parts of your problem, consider them 1 by 1, consider them in turn, consider them in various combinations, relating each detail to other details \& each to the whole of the problem.

\textit{What can I gain by doing so?} You should prepare \& clarify details which are likely to play a role afterwards.'' -- \cite[p. 33]{Polya2014}

\subsection{Hunting for the Helpful Idea}
``\textit{Where should I start?} Start from the consideration of the principal parts of your problem. Start when these principal parts are distinctly arranged \& clearly conceived, thanks to your previous work, \& when your memory seems responsive.

\textit{What can I do?} Consider your problem from various sides \& seek contacts with your formerly acquired knowledge.

Consider your problem from various sides. Emphasize different parts, examine different details, examine the same details repeatedly but in different ways, combine the details differently, approach them from different sides. Try to see some new meaning in each detail, some new interpretation\footnote{\textbf{interpretation} [n] \textbf{1.} [countable] the particular way in which something is understood or explained; \textbf{2.} [uncountable] \textbf{interpretation (of something)} the action of explaining the meaning of something. If something is \textbf{open to interpretation}, its meaning is not clear \& can be understood in different ways.} of the whole.

Seek contacts with your formerly acquired knowledge. Try to think of what helped you in similar situations in the past. Try to recognize something familiar in what you examine, try to perceive something useful in what you recognize.

\textit{What could I perceive?} A helpful idea, perhaps a decisive idea that shows you at a glance the way to the very end.

\textit{How can an idea be helpful?} It shows you the whole of the way or a part of the way; it suggests to you more or less distinctly\footnote{\textbf{distinctly} [adv] in a way that is clear \& easily noticed; showing a clear difference.} how you can proceed. Ideas are more or less complete. You are lucky if you have any idea at all.

\textit{What can I do with an incomplete idea?} You should consider it. If it looks advantageous you should consider it longer. If it looks reliable\footnote{\textbf{reliable} [a] \textbf{1.} likely to be correct or true, \textsc{opposite}: \textbf{unreliable}; \textbf{2.} that can be trusted to do something well; that can be relied on, \textsc{opposite}: \textbf{unreliable}.} you should ascertain\footnote{\textbf{ascertain} [v] (\textit{formal}) to find out the true or correct information about something.} how far it leads you, \& reconsider the situation. The situation has changed, thanks to your helpful idea. Consider the new situation from various sides \& seek contacts with your formerly acquired knowledge.

\textit{What can I gain by doing so again?} You may be lucky \& have another idea. Perhaps your next idea will lead you to the solution right away. Perhaps you need a few more helpful ideas after the next. Perhaps you will be led astray\footnote{\textbf{astray} [adv] \textbf{1.} \textbf{go astray} [idiom] to become lost; to be stolen; \textbf{2.} \textbf{lead somebody astray} [idiom] to make somebody go in the wrong direction or do things that are wrong.} by some of your ideas. Nevertheless you should be grateful for all new ideas, also for the lesser ones, also for the hazy\footnote{\textbf{hazy} [a] (\textbf{hazier, haziest}) \textbf{1.} not clear because of haze; \textbf{2.} not clear because of a lack of memory, understanding or detail, \textsc{synonym}: \textbf{vague}; \textbf{3.} (of a person) uncertain or confused about something.} ones, also for the supplementary ideas adding some precision\footnote{\textbf{precision} [n] [uncountable] the quality of being exact \& accurate, \textsc{synonym}: \textbf{accuracy}.} to a hazy one, or attempting the correction of a less fortunate one. Even if you do not have any appreciable\footnote{\textbf{appreciable} [a] large or important enough to be noticed, \textsc{synonym}: \textbf{considerable}.} new ideas for a while you should be grateful if your conception of the problem becomes more complete or more coherent\footnote{\textbf{coherent} [a] \textbf{1.} (of an argument, theory, statement or policy) logical \& well organized; easy to understand \& clear, \textsc{opposite}: \textbf{incoherent}; \textbf{2.} (of a person) able to talk \& express yourself clearly; showing this, \textsc{opposite}: \textbf{incoherent}; \textbf{3.} made up of different parts tht fit or work well together; \textbf{4.} (\textit{physics}) (of waves) in phase with each other, \textsc{opposite}: \textbf{incoherent}.}, more homogeneous\footnote{\textbf{homogeneous} [a] \textbf{1.} (\textit{formal}) consisting of things or people that are all the same or all of the same type, \textsc{opposite}: \textbf{heterogeneous}; \textbf{2.} (\textit{chemistry}) used to describe a process involving substances in the same phase (solid, liquid or gas), \textsc{opposite}: \textbf{heterogeneous}.} or better balanced\footnote{\textbf{balanced} [a] [usually before noun] (\textit{approving}) \textbf{1.} having all parts in equal, correct or good amounts; \textbf{2.} giving careful thought to all opinions on a particular subject.}.'' -- \cite[pp. 33--35]{Polya2014}

\subsection{Carrying Out the Plan}
``\textit{Where should I start?} Start from the lucky idea that led you to the solution. Start when you feel sure of your grasp\footnote{\textbf{grasp} [v] \textbf{1.} to understand something completely; \textbf{2.} \textbf{grasp an opportunity} to take an opportunity without hesitating \& use it; \textbf{3.} \textbf{grasp somebody\texttt{/}something} to take a firm hold of somebody\texttt{/}something, \textsc{synonym}: \textbf{grip}; [n] [usually singular] \textbf{1.} a person's understanding of a subject; \textbf{2.} a firm hold of somebody\texttt{/}something or control over somebody\texttt{/}something; \textbf{3.} the ability to get or achieve something.} of the main connection \& you feel confident that you can supply the minor details that may be wanting.

\textit{What can I do?} Make your grasp quite secure. Carry through in detail all the algebraic or geometric operations which you have recognized previously as feasible\footnote{\textbf{feasible} [a] that is possible \& likely to be achieved, \textsc{synonym}: \textbf{practicable}.}. Convince yourself of the correctness of each step by formal reasoning, or by intuitive insight, or both ways if you can. If your problem is very complex you may distinguish ``great'' steps \& ``small'' steps, each great step being composed of several small ones. Check 1st the great steps, \& get down to the smaller ones afterwards.

\textit{What can I gain by doing so?} A presentation of the solution each step of which is correct beyond doubt.'' -- \cite[pp. 33--35]{Polya2014}

\subsection{Looking Back}
``\textit{Where should I start?} From the solution, complete \& correct in each detail.

\textit{What can I do?} Consider the solution from various sides \& seek contacts with your formerly acquired knowledge.

Consider the details of the solution \& try to make them as simple as you can; survey more extensive parts of the solution \& try to make them shorter; try to see the whole solution at a glance. Try to modify to their advantage smaller or larger parts of the solution, try to improve the whole solution, to make it intuitive, to fit it into your formerly acquired knowledge as naturally as possible. Scrutinize\footnote{\textbf{scrutinize} [v] (BE also \textbf{scrutinise}) \textbf{scrutinize something} to look at or examine something carefully.} the method that led you to the solution, try to see its point, \& try to make use of it for other problems.

\textit{What can I gain by doing so?} You may find a new \& better solution, you may discover new \& interesting facts. In any case, if you get into the habit of surveying \& scrutinizing your solutions in this way, you will acquire some knowledge well ordered \& ready to use, \& you will develop your ability of solving problems.'' -- \cite[p. 35]{Polya2014}

\begin{center}
	\huge Part III. Short Dictionary of Heuristic
\end{center}

\section{Analogy}
``\textbf{Analogy}\footnote{\textbf{analogy} [n] (plural \textbf{analogies}) [countable, uncountable] a comparison of 1 thing with another thing that has similar features, usually in order to explain it; a feature that is similar.} is a sort of similarity\footnote{\textbf{similarity} [n] (plural \textbf{similarities}) \textbf{1.} [uncountable, singular] the state of being like somebody\texttt{/}something but not exactly the same, \textsc{synonym}: \textbf{resemblance}, \textsc{opposite}: \textbf{difference, dissimilarity}; \textbf{2.} [countable] a feature that things or people have that makes them like each other, \textsc{opposite}: \textbf{difference, dissimilarity}.}. Similar objects agree with each other in some respect, analogous objects \textit{agree in certain relations} of their respective parts.

'' -- \cite[pp. 37--]{Polya2014}

\section{Auxiliary elements}

\section{Auxiliary problem}

\section{Bolzano}

\section{Bright idea}

\section{Can you check the result?}

\section{Can you derive the result differently?}

\section{Can you use the result?}

\section{Carrying out}

\section{Condition}

\section{Contradictory}

\section{Corollary}

\section{Could you derive something useful from the data?}

\section{Could you restate the problem?}

\section{Decomposing \& recombining}

\section{Definition}

\section{Descartes}

\section{Determination, hope, success}

\section{Diagnosis}

\section{Did you see all the data?}

\section{Do you know a related problem?}

\section{Draw a figure}

\section{Examine your guess}

\section{Figures}

\section{Generalization}

\section{Have you seen it before?}

\section{Here is a problem related to yours \& solved before}

\section{Heuristic}

\section{Heuristic reasoning}

\section{If you cannot solve the proposed problem}

\section{Induction \& mathematical induction}

\section{Inventor's paradox}

\section{Is it possible to satisfy the condition?}

\section{Leibnitz}

\section{Lemma}

\section{Look at the unknown}

\section{Modern heuristic}

\section{Notation}

\section{Pappus}

\section{Pedantry \& mastery}

\section{Practical problems}

\section{Problems to find, problems to prove}

\section{Progress \& achievement}

\section{Puzzles}

\section{Reductio \& absurdum \& indirect proof}

\section{Redundant}

\section{Routine problem}

\section{Rules of discovery}

\section{Rules of style}

\section{Rules of teaching}

\section{Separate the various parts of the condition}

\section{Setting up equations}

\section{Signs of progress}

\section{Specialization}

\section{Subconscious work}

\section{Symmetry}

\section{Terms, old \& new}

\section{Test by dimension}

\section{The future mathematician}

\section{The intelligent problem-solver}

\section{The intelligent reader}

\section{The traditional mathematics professor}

\section{Variation of the problem}

\section{What is the unknown?}

\section{Why proofs?}

\section{Wisdom of proverbs}

\section{Working backwards}

\begin{center}
	\huge Part IV. Problems, Hints, Solutions
\end{center}

\section{Problems}

\section{Hints}

\section{Solutions}

%------------------------------------------------------------------------------%

%\selectlanguage{english}
%\begin{thebibliography}{99}
%	\bibitem[]{}
%\end{thebibliography}

%------------------------------------------------------------------------------%

\printbibliography[heading=bibintoc]
	
\end{document}