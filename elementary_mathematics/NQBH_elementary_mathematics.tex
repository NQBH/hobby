\documentclass[oneside]{book}
\usepackage[backend=biber,natbib=true,style=authoryear]{biblatex}
\addbibresource{/home/hong/1_NQBH/reference/bib.bib}
\usepackage[vietnamese,english]{babel}
\usepackage{tocloft}
\renewcommand{\cftsecleader}{\cftdotfill{\cftdotsep}}
\usepackage[colorlinks=true,linkcolor=blue,urlcolor=red,citecolor=magenta]{hyperref}
\usepackage{amsmath,amssymb,amsthm,mathtools,float,graphicx}
\allowdisplaybreaks
\numberwithin{equation}{section}
\newtheorem{assumption}{Assumption}[chapter]
\newtheorem{conjecture}{Conjecture}[chapter]
\newtheorem{corollary}{Corollary}[chapter]
\newtheorem{definition}{Definition}[chapter]
\newtheorem{example}{Example}[chapter]
\newtheorem{lemma}{Lemma}[chapter]
\newtheorem{notation}{Notation}[chapter]
\newtheorem{principle}{Principle}[chapter]
\newtheorem{problem}{Problem}[chapter]
\newtheorem{proposition}{Proposition}[chapter]
\newtheorem{question}{Question}[chapter]
\newtheorem{remark}{Remark}[chapter]
\newtheorem{theorem}{Theorem}[chapter]
\usepackage[left=0.5in,right=0.5in,top=1.5cm,bottom=1.5cm]{geometry}
\usepackage{fancyhdr}
\pagestyle{fancy}
\fancyhf{}
\lhead{\small \textsc{Sect.} ~\thesection}
\rhead{\small \nouppercase{\leftmark}}
\renewcommand{\sectionmark}[1]{\markboth{#1}{}}
\cfoot{\thepage}
\def\labelitemii{$\circ$}

\title{Elementary Mathematics}
\author{\selectlanguage{vietnamese} Nguyễn Quản Bá Hồng\footnote{Independent Researcher, Ben Tre City, Vietnam\\e-mail: \texttt{nguyenquanbahong@gmail.com}}}
\date{\today}

\begin{document}
\maketitle
\setcounter{secnumdepth}{4}
\setcounter{tocdepth}{4}
\tableofcontents

%------------------------------------------------------------------------------%

\chapter{Wikipedia's}

\section{\href{https://en.wikipedia.org/wiki/How_to_Solve_It}{Wikipedia\texttt{/}How to Solve It}}
``\textit{How to Solve It} (1945) is a small volume by mathematician \href{https://en.wikipedia.org/wiki/George_P%C3%B3lya}{George P\'olya} describing methods of \href{https://en.wikipedia.org/wiki/Problem_solving}{problem solving}.'' -- \href{https://en.wikipedia.org/wiki/How_to_Solve_It}{Wikipedia\texttt{/}how to solve it}

\subsection{4 principles}
``\textit{How to Solve It} suggests the following steps when solving a \href{https://en.wikipedia.org/wiki/Mathematical_problem}{mathematical problem}:
\begin{enumerate}
	\item 1st, you have to \textit{understand the problem}.
	\item After understanding, \textit{make a plan}.
	\item \textit{Carry out the plan}.
	\item \textit{Look back} on your work. How could it be better?
\end{enumerate}
If this technique fails, P\'olya advises: ``If you can't solve a problem, then there is an easier problem you can solve: find it.'' Or: ``If you cannot solve the proposed problem, try to solve 1st some related problem. Could you imagine a more accessible related problem?'''' -- \href{https://en.wikipedia.org/wiki/How_to_Solve_It#Four_principles}{Wikipedia\texttt{/}how to solve it\texttt{/}4 principles}

\subsubsection{1st principle: Understand the problem}
``Understanding the problem'' is often neglected as being obvious \& is not even mentioned in many mathematics classes. Yet students are often stymied in their efforts to solve it, simply because they don't understand it fully, or even in part. In order to remedy this oversight, P\'olya taught teachers how to prompt each student with appropriate questions, depending on the situation, such as:
\begin{itemize}
	\item What are you asked to find or show?
	\item Can you restate the problem in your own words?
	\item Can you think of a picture of a diagram that might help you understand the problem?
	\item Is there enough information to enable you to find a solution?
	\item Do you understand all the words used in stating the problem?
	\item Do you need to ask a question to get the answer?
\end{itemize}
The teacher is to select the question with the appropriate level of difficulty for each student to ascertain if each student understands at their own level, moving up or down the list to prompt each student, until each one can respond with something constructive.'' -- \href{https://en.wikipedia.org/wiki/How_to_Solve_It#First_principle:_Understand_the_problem}{Wikipedia\texttt{/}how to solve it\texttt{/}4 principles\texttt{/}1st principle: understand the problem}

\subsubsection{2nd principle: Devise a plan}
``P\'olya mentions that there are many reasonable ways to solve problems. The skill at choosing an appropriate strategy is best learned by solving many problems. You will find choosing a strategy increasingly easy. A partial list of strategies is included:
\begin{itemize}
	\item Guess \& check
	\item Make an orderly list
	\item Eliminate possibilities
	\item Use symmetry
	\item Consider special cases
	\item Use direct reasoning
	\item Solve an equation
\end{itemize}
Also suggested:
\begin{itemize}
	\item Look for a pattern
	\item Draw a picture
	\item Solve a simpler problem
	\item Use a model
	\item Work backward
	\item Use a formula
	\item Be creative
	\item Applying these rules to devise a plan takes your own skill \& judgment.
\end{itemize}
P\'olya lays a big emphasis on the teachers' behavior. A teacher should support students with devising their own plan with a question method that goes from the most general questions to more particular questions, with the goal that the last step to having a plan is made by the student. He maintains that just showing students a plan, no matter how good it is, does not help them.'' -- \href{https://en.wikipedia.org/wiki/How_to_Solve_It#Second_principle:_Devise_a_plan}{Wikipedia\texttt{/}how to solve it\texttt{/}4 principles\texttt{/}2nd principle: devise a plan}

\subsubsection{3rd principle: Carry out the plan}
``This step is usually easier than devising the plan. In general, all you need is care \& patience, given that you have the necessary skills. Persist with the plan that you have chosen. If it continues not to work, discard it \& choose another. Don't be misled; this is how mathematics is done, even by professionals.'' -- \href{https://en.wikipedia.org/wiki/How_to_Solve_It#Third_principle:_Carry_out_the_plan}{Wikipedia\texttt{/}how to solve it\texttt{/}4 principles\texttt{/}3rd principle: carry out the plan}

\subsubsection{4th principle: Review\texttt{/}extend}
``P\'olya mentions that much can be gained by taking the time to reflect \& look back at what you have done, what worked \& what did not, \& with thinking about other problems where this could be useful. Doing this will enable you to predict what strategy to use to solve future problems, if these relate to the original problem.'' -- \href{https://en.wikipedia.org/wiki/How_to_Solve_It#Fourth_principle:_Review/extend}{Wikipedia\texttt{/}how to solve it\texttt{/}4 principles\texttt{/}4th principle: review\texttt{/}extend}

\subsection{Heuristics}
``The book contains a dictionary-style set of \href{https://en.wikipedia.org/wiki/Heuristics}{heuristics}, many of which have to do with generating a more accessible problem. E.g.:

\textbf{Heuristic $|$ Informal Description $|$ Formal analogue}
\begin{itemize}
	\item \href{https://en.wikipedia.org/wiki/Analogy}{Analogy} $|$ Can you find a problem analogous to your problem \& solve that? $|$ \href{https://en.wikipedia.org/wiki/Map_(mathematics)}{map}
	\item Auxiliary Elements $|$ Can you add some new element to your problem to get closer to a solution? $|$ \href{https://en.wikipedia.org/wiki/Extension_(predicate_logic)}{Extension}
	\item \href{https://en.wikipedia.org/wiki/Generalization}{Generalization} $|$ Can you find a problem more general than your problem? $|$ \href{https://en.wikipedia.org/wiki/Generalization}{Generalization}
\end{itemize}

'' -- \href{https://en.wikipedia.org/wiki/How_to_Solve_It#Heuristics}{Wikipedia\texttt{/}how to solve it\texttt{/}heuristics}

\subsection{Influence}

%------------------------------------------------------------------------------%

\selectlanguage{english}
\begin{thebibliography}{99}
	\bibitem[]{}
\end{thebibliography}

%------------------------------------------------------------------------------%

\printbibliography[heading=bibintoc]
	
\end{document}