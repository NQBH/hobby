\documentclass[oneside]{book}
\usepackage[backend=biber,natbib=true,style=authoryear]{biblatex}
\addbibresource{/home/hong/1_NQBH/reference/bib.bib}
\usepackage[vietnamese,english]{babel}
\usepackage{tocloft}
\renewcommand{\cftsecleader}{\cftdotfill{\cftdotsep}}
\usepackage[colorlinks=true,linkcolor=blue,urlcolor=red,citecolor=magenta]{hyperref}
\usepackage{amsmath,amssymb,amsthm,mathtools,float,graphicx}
\allowdisplaybreaks
\numberwithin{equation}{section}
\newtheorem{assumption}{Assumption}[chapter]
\newtheorem{conjecture}{Conjecture}[chapter]
\newtheorem{corollary}{Corollary}[chapter]
\newtheorem{definition}{Definition}[chapter]
\newtheorem{example}{Example}[chapter]
\newtheorem{lemma}{Lemma}[chapter]
\newtheorem{notation}{Notation}[chapter]
\newtheorem{principle}{Principle}[chapter]
\newtheorem{problem}{Problem}[chapter]
\newtheorem{proposition}{Proposition}[chapter]
\newtheorem{question}{Question}[chapter]
\newtheorem{remark}{Remark}[chapter]
\newtheorem{theorem}{Theorem}[chapter]
\usepackage[left=0.5in,right=0.5in,top=1.5cm,bottom=1.5cm]{geometry}
\usepackage{fancyhdr}
\pagestyle{fancy}
\fancyhf{}
\lhead{\small \textsc{Sect.} ~\thesection}
\rhead{\small \nouppercase{\leftmark}}
\renewcommand{\sectionmark}[1]{\markboth{#1}{}}
\cfoot{\thepage}
\def\labelitemii{$\circ$}

\title{Elementary Mathematics}
\author{\selectlanguage{vietnamese} Nguyễn Quản Bá Hồng\footnote{Independent Researcher, Ben Tre City, Vietnam\\e-mail: \texttt{nguyenquanbahong@gmail.com}}}
\date{\today}

\begin{document}
\maketitle
\setcounter{secnumdepth}{4}
\setcounter{tocdepth}{4}
\tableofcontents

%------------------------------------------------------------------------------%

\chapter{Wikipedia's}

\section{\href{https://en.wikipedia.org/wiki/How_to_Solve_It}{Wikipedia\texttt{/}How to Solve It}}
``\textit{How to Solve It} (1945) is a small volume by mathematician \href{https://en.wikipedia.org/wiki/George_P%C3%B3lya}{George P\'olya} describing methods of \href{https://en.wikipedia.org/wiki/Problem_solving}{problem solving}.'' -- \href{https://en.wikipedia.org/wiki/How_to_Solve_It}{Wikipedia\texttt{/}how to solve it}

\subsection{4 principles}
``\textit{How to Solve It} suggests the following steps when solving a \href{https://en.wikipedia.org/wiki/Mathematical_problem}{mathematical problem}:
\begin{enumerate}
	\item 1st, you have to \textit{understand the problem}.
	\item After understanding, \textit{make a plan}.
	\item \textit{Carry out the plan}.
	\item \textit{Look back} on your work. How could it be better?
\end{enumerate}
If this technique fails, P\'olya advises: ``If you can't solve a problem, then there is an easier problem you can solve: find it.'' Or: ``If you cannot solve the proposed problem, try to solve 1st some related problem. Could you imagine a more accessible related problem?'''' -- \href{https://en.wikipedia.org/wiki/How_to_Solve_It#Four_principles}{Wikipedia\texttt{/}how to solve it\texttt{/}4 principles}

\subsubsection{1st principle: Understand the problem}
``Understanding the problem'' is often neglected as being obvious \& is not even mentioned in many mathematics classes. Yet students are often stymied in their efforts to solve it, simply because they don't understand it fully, or even in part. In order to remedy this oversight, P\'olya taught teachers how to prompt each student with appropriate questions, depending on the situation, such as:
\begin{itemize}
	\item What are you asked to find or show?
	\item Can you restate the problem in your own words?
	\item Can you think of a picture of a diagram that might help you understand the problem?
	\item Is there enough information to enable you to find a solution?
	\item Do you understand all the words used in stating the problem?
	\item Do you need to ask a question to get the answer?
\end{itemize}
The teacher is to select the question with the appropriate level of difficulty for each student to ascertain if each student understands at their own level, moving up or down the list to prompt each student, until each one can respond with something constructive.'' -- \href{https://en.wikipedia.org/wiki/How_to_Solve_It#First_principle:_Understand_the_problem}{Wikipedia\texttt{/}how to solve it\texttt{/}4 principles\texttt{/}1st principle: understand the problem}

\subsubsection{2nd principle: Devise a plan}
``P\'olya mentions that there are many reasonable ways to solve problems. The skill at choosing an appropriate strategy is best learned by solving many problems. You will find choosing a strategy increasingly easy. A partial list of strategies is included:
\begin{itemize}
	\item Guess \& check
	\item Make an orderly list
	\item Eliminate possibilities
	\item Use symmetry
	\item Consider special cases
	\item Use direct reasoning
	\item Solve an equation
\end{itemize}
Also suggested:
\begin{itemize}
	\item Look for a pattern
	\item Draw a picture
	\item Solve a simpler problem
	\item Use a model
	\item Work backward
	\item Use a formula
	\item Be creative
	\item Applying these rules to devise a plan takes your own skill \& judgment.
\end{itemize}
P\'olya lays a big emphasis on the teachers' behavior. A teacher should support students with devising their own plan with a question method that goes from the most general questions to more particular questions, with the goal that the last step to having a plan is made by the student. He maintains that just showing students a plan, no matter how good it is, does not help them.'' -- \href{https://en.wikipedia.org/wiki/How_to_Solve_It#Second_principle:_Devise_a_plan}{Wikipedia\texttt{/}how to solve it\texttt{/}4 principles\texttt{/}2nd principle: devise a plan}

\subsubsection{3rd principle: Carry out the plan}
``This step is usually easier than devising the plan. In general, all you need is care \& patience, given that you have the necessary skills. Persist with the plan that you have chosen. If it continues not to work, discard it \& choose another. Don't be misled; this is how mathematics is done, even by professionals.'' -- \href{https://en.wikipedia.org/wiki/How_to_Solve_It#Third_principle:_Carry_out_the_plan}{Wikipedia\texttt{/}how to solve it\texttt{/}4 principles\texttt{/}3rd principle: carry out the plan}

\subsubsection{4th principle: Review\texttt{/}extend}
``P\'olya mentions that much can be gained by taking the time to reflect \& look back at what you have done, what worked \& what did not, \& with thinking about other problems where this could be useful. Doing this will enable you to predict what strategy to use to solve future problems, if these relate to the original problem.'' -- \href{https://en.wikipedia.org/wiki/How_to_Solve_It#Fourth_principle:_Review/extend}{Wikipedia\texttt{/}how to solve it\texttt{/}4 principles\texttt{/}4th principle: review\texttt{/}extend}

\subsection{Heuristics}
``The book contains a dictionary-style set of \href{https://en.wikipedia.org/wiki/Heuristics}{heuristics}, many of which have to do with generating a more accessible problem. E.g.:

\textbf{Heuristic $|$ Informal Description $|$ Formal analogue}
\begin{itemize}
	\item \href{https://en.wikipedia.org/wiki/Analogy}{Analogy} $|$ Can you find a problem analogous to your problem \& solve that? $|$ \href{https://en.wikipedia.org/wiki/Map_(mathematics)}{map}
	\item Auxiliary Elements $|$ Can you add some new element to your problem to get closer to a solution? $|$ \href{https://en.wikipedia.org/wiki/Extension_(predicate_logic)}{Extension}
	\item \href{https://en.wikipedia.org/wiki/Generalization}{Generalization} $|$ Can you find a problem more general than your problem? $|$ \href{https://en.wikipedia.org/wiki/Generalization}{Generalization}
	\item \href{https://en.wikipedia.org/wiki/Induction_(philosophy)}{Induction} $|$ Can you solve your problem by deriving a generalization from some examples? $|$ \href{https://en.wikipedia.org/wiki/Induction_(philosophy)}{Induction}
	\item Variation of the Problem $|$ Can you vary or change your problem to create a new problem (or set of problems) whose solution(s) will help you solve your original problem? $|$ \href{https://en.wikipedia.org/wiki/Search_algorithm}{Search}
	\item Auxiliary Problem $|$ Can you find a subproblem or side problem whose solution will help you solve your problem? $|$ \href{https://en.wikipedia.org/wiki/Subgoal}{Subgoal}
	\item Here is a problem related to yours \& solved before $|$ Can you find a problem related to yours that has already been solved \& use that to solve your problem? $|$ \href{https://en.wikipedia.org/wiki/Pattern_recognition}{Pattern recognization}, \href{https://en.wikipedia.org/wiki/Pattern_matching}{Pattern matching}, \href{https://en.wikipedia.org/wiki/Reduction_(complexity)}{Reduction}
	\item \href{https://en.wikipedia.org/wiki/Special_case}{Specialization} $|$ Can you find a problem more specialized? $|$ \href{https://en.wikipedia.org/wiki/Special_case}{Specialization}
	\item \href{https://en.wikipedia.org/wiki/Decomposition_(computer_science)}{Decomposing} \& Recombining $|$ Can you decompose the problem \& ``recombine its elements in some new manner''? $|$ \href{https://en.wikipedia.org/wiki/Divide_and_conquer_algorithm}{Divide \& conquer}
	\item \href{https://en.wikipedia.org/wiki/Working_backward_from_the_goal}{Working backward} $|$ Can you start with the goal \& work backwards to something you already know? $|$ \href{https://en.wikipedia.org/wiki/Backward_chaining}{Backward chaining}
	\item Draw a Figure $|$ Can you draw a picture of the problem? $|$ \href{https://en.wikipedia.org/wiki/Diagrammatic_Reasoning}{Diagrammatic Reasoning}
\end{itemize}
'' -- \href{https://en.wikipedia.org/wiki/How_to_Solve_It#Heuristics}{Wikipedia\texttt{/}how to solve it\texttt{/}heuristics}

\subsection{Influence}
\begin{itemize}
	\item ``The book has been translated into several languages \& has sold over a million copies, \& has been continuously in print since its 1st publication.
	\item \href{https://en.wikipedia.org/wiki/Marvin_Minsky}{Marvin Minsky} said in his paper \textit{Steps Toward Artificial Intelligence} that ``everyone should know the work of George P\'olya on how to solve problems.''
	\item P\'olya's book has had a large influence on mathematics textbooks as evidenced by the bibliographies for \href{https://en.wikipedia.org/wiki/Mathematics_education}{mathematics education}.
	\item Russian inventor \href{https://en.wikipedia.org/wiki/Genrich_Altshuller}{Genrich Altshuller} developed an elaborate set of methods for problem solving known as \href{https://en.wikipedia.org/wiki/TRIZ}{TRIZ}, which in many aspects reproduces or parallels P\'olya's work.
	\item \href{https://en.wikipedia.org/wiki/How_to_Solve_it_by_Computer}{How to Solve it by Computer} is a computer science book by R. G. Dromey. It was inspired by P\'olya's work.'' -- \href{https://en.wikipedia.org/wiki/How_to_Solve_It#Influence}{Wikipedia\texttt{/}how to solve it\texttt{/}influence}
\end{itemize}

%------------------------------------------------------------------------------%

\chapter{\cite{Polya2014}. How to Solve It: A New Aspect of Mathematical Methods}

\section*{From the Preface to the 1st Printing}
``A great discovery solves a great problem but there is a grain\footnote{\textbf{grain} [n] \textbf{1.} [uncountable, countable] the small hard seeds of food plants such as wheat, rice, etc.; a single seed of such a plant; \textbf{2.} [countable] \textbf{grain (of something)} a small piece of a particular substance; usually a hard substance; \textbf{3.} [countable, usually singular] \textbf{grain of something} a very small amount; \textbf{4.} [countable] an individual particle or crystal in metal, rock, etc., usually explained with a lens or microscope.} of discovery in the solution of any problem. Your problem may be modest\footnote{\textbf{modest} [a] \textbf{1.} fairly limited or small in amount; \textbf{2.} not expensive, rich or impressive; \textbf{3.} (of people, especially women, or their clothes) not showing too much of the body; not intended to attract attention, especially in a sexual way; \textbf{4.} (\textit{approving}) not talking much about your own abilities or possessions.}; but it challenges your curiosity\footnote{\textbf{curiosity} [n] (plural \textbf{curiosities}) \textbf{1.} [uncountable, singular] a strong desire to know about something; \textbf{2.} [countable] \textbf{curiosity (of something)} an unusual \& interesting thing.} \& brings into play your inventive\footnote{\textbf{inventive} [a] \textbf{1.} (especially of people) able to create or design new things or think of new ideas; \textbf{2.} (of ideas) new \& interesting.} faculties\footnote{\textbf{faculty} [n] (plural \textbf{faculties}) \textbf{1.} [countable] a physical or mental ability, especially one that people are born with; \textbf{2.} [countable] \textbf{faculty (of something)} a department or group of related departments in a college or university; \textbf{3.} [countable $+$ singular or plural verb] all the teachers in a faculty of a college or university; \textbf{4.} [countable, uncountable] (\textit{North American English}) all the teachers of a particular university or college.}, \& if you solve it by your own means, you may experience the tension\footnote{\textbf{tension} [n] \textbf{1.} [uncountable, countable, usually plural] a situation in which people do not trust each other, or feel unfriendly towards each other, \& which may cause them to attack each other; \textbf{2.} [countable, uncountable] \textbf{tension (between A \& B)} a situation in which the fact that there are different needs or interests causes difficulties; \textbf{3.} [uncountable] a feeling of anxiety \& stress that makes it impossible to relax; \textbf{4.} [uncountable] the feeling of fear \& excitement that is created by a writer or a film director; \textbf{5.} [uncountable] the state of being stretched tight; the extent to which something is stretched tight.} \& enjoy the triumph\footnote{\textbf{triumph} [n] \textbf{1.} [countable, uncountable] a great success, achievement or victory; \textbf{2.} [uncountable] the state of having achieved a great success or victory; the feeling of happiness that you get from this; [v] [intransitive] to defeat somebody\texttt{/}something; to be successful.} of discovery. Such experiences at a susceptible\footnote{\textbf{susceptible} [a] \textbf{1.} [not usually before noun] \textbf{susceptible (to somebody\texttt{/}something)} very likely to be influenced, harmed or affected by somebody\texttt{/}something; \textbf{2.} \textbf{susceptible (of something)} (\textit{formal}) allowing something; capable of something.} age may create a taste for mental work \& leave their imprint\footnote{\textbf{imprint} [v] [often passive] \textbf{1.} to have a great effect on something so that it cannot be forgotten, changed, etc.; \textbf{2.} to print or press a mark or design onto a surface; [n] \textbf{1.} \textbf{imprint (of something) (in\texttt{/}on something)} a mark made by pressing something onto a surface; \textbf{2.} [usually singular] \textbf{imprint (of something) (on somebody\texttt{/}something)} (\textit{formal}) the lasting effect that a person or an experience has on a place or a situation; \textbf{3.} (\textit{specialist}) the name of the publisher of a book, usually printed below the title on the 1st page; a brand name under which books are published.} on mind \& character for a lifetime\footnote{\textbf{lifetime} [n] the length of time that somebody lives or that something lasts.}.

Thus, a teacher of mathematics has a great opportunity. If he fills his allotted\footnote{\textbf{allot} [v] to give time, money, tasks, etc. to somebody\texttt{/}something as a share of what is available, \textsc{synonym}: \textbf{allocate}.} time with drilling his students in routine operations he kills their interest, hampers\footnote{\textbf{hamper} [v] [often passive] to prevent something from being achieved easily or happening normally; to prevent somebody from easily doing something, \textsc{synonym}: \textbf{hinder, impede}.} their intellectual development, \& misuses his opportunity. But if he challenges the curiosity of his students by setting them problems proportionate\footnote{\textbf{proportionate} [a] increasing or decreasing in size, amount or degree according to changes in something else, \textsc{synonym}: \textbf{proportional}.} to their knowledge, \& helps them to solve their problems with stimulating\footnote{\textbf{stimulating} [a] \textbf{1.} full of interesting or exciting ideas; making people feel enthusiastic; \textbf{2.} making you feel more active \& healthy.} questions, he may give them a taste for, \& some means of, independent thinking.

Also a student whose college curriculum\footnote{\textbf{curriculum} [n] (plural \textbf{curricula}) the subjects that are included in a course of study or taught in a school, college or university.} includes some mathematics has a singular\footnote{\textbf{singular} [n] [singular] (\textit{grammar}) a form of a noun or verb that refers to 1 person or thing; [a] \textbf{1.} (\textit{grammar}) connected with or having the form of a noun or verb that refers to 1 person or thing; \textbf{2.} especially great or obvious, \textsc{synonym}: \textbf{outstanding}; \textbf{3.} (\textit{mathematics, physics}) connected with a singularity.} opportunity. This opportunity is lost, of course, if he regards\footnote{\textbf{regard} [v] [often passive] to think about somebody\texttt{/}something in a particular way; \textbf{as regards somebody\texttt{/}something} [idiom] concerning or in connection with somebody\texttt{/}something; [n] \textbf{1.} [uncountable] attention to or thought \& care for somebody\texttt{/}something; \textbf{2.} [uncountable] \textbf{regard (for somebody\texttt{/}something)} respect or admiration for somebody\texttt{/}something. If you \textbf{hold somebody in high regard}, you have a good opinion of them.; \textbf{3.} (\textbf{regards}) [plural] used to send good wishes to somebody at the end of a letter or email; \textbf{have regard to something} [idiom] (\textit{law}) to remember \& think carefully about something; \textbf{in\texttt{/}with regard to somebody\texttt{/}something} [idiom] concerning somebody\texttt{/}something; \textbf{in this\texttt{/}that regard} [idiom] concerning what has just been mentioned.} mathematics as a subject in which he has to earn so \& so much credit \& which he should forget after the final examination as quickly as possible. The opportunity may be lost even if the student has some natural talent for mathematics because he, as everybody else, must discover his talents \& tastes; he cannot know that he likes raspberry pie if he has never tasted raspberry pie. He may manage to find out, however, that a mathematics problem may be as much fun as a crossword puzzle\footnote{\textbf{crossword} [n] (also \textbf{crossword puzzle}) a game in which you have to fit words across \& downwards into spaces with numbers in a square diagram. You find the words by solving clues.}, or that vigorous\footnote{\textbf{vigorous} [a] \textbf{1.} involving physical strength, effort or energy; \textbf{2.} done with determination, energy or enthusiasm; \textbf{3.} strong \& healthy.} mental work may be an exercise as desirable as a fast game of tennis. Having tasted the pleasure in mathematics he will not forget it easily \& then there is a good chance that mathematics will become something for him: a hobby, or a tool of his profession, or a great ambition,

The author remembers the time when he was a student himself, a somewhat ambitious student, eager to understand a little mathematics \& physics. He listened to lectures, read books, tried to take in the solutions \& facts presented, but there was a question that disturbed\footnote{\textbf{disturb} [v] \textbf{1.} \textbf{disturb something} to change the arrangement of something, or affect how something functions; \textbf{2.} \textbf{disturb somebody\texttt{/}something} to interrupt somebody \& prevent them from continuing with what they are doing; \textbf{3.} \textbf{disturb somebody} to make somebody feel anxious or upset.} him again \& again: ``Yes, the solution seems to work, it appears to be correct; but how is it possible to invent such a solution? Yes, this experiment seems to work, this appears to be a fact; but how can people discover such facts? \& how could I invent or discover such things by myself?'' Today the author is teaching mathematics in a university; he thinks or hopes that some of his more eager students ask similar questions \& he tries to satisfy their curiosity. Trying to understand not only the solution of this or that problem but also the motives \& procedures of the solution, \& trying to explain these motives \& procedures to others, he was finally led to write the present book. He hopes that it will be useful to teachers who wish to develop their students' ability to solve problems, \& to students who are keen on developing their own abilities.

Although the present book pays special attention to the requirements of students \& teachers of mathematics, it should interest anybody concerned with the ways \& means of invention \& discovery. Such interest may be more widespread\footnote{\textbf{widespread} [a] existing or happening over a large area or among many people, \textsc{synonym}: \textbf{extensive}.} than one would assume without reflection. The space devoted by popular newspapers \& magazines to crossword puzzles \& other riddles seems to show that people spend some time in solving unpractical problems.

'' -- \cite[pp. v--]{Polya2014}

%------------------------------------------------------------------------------%

%\selectlanguage{english}
%\begin{thebibliography}{99}
%	\bibitem[]{}
%\end{thebibliography}

%------------------------------------------------------------------------------%

\printbibliography[heading=bibintoc]
	
\end{document}