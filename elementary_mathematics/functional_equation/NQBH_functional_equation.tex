\documentclass{article}
\usepackage[backend=biber,natbib=true,style=authoryear]{biblatex}
\addbibresource{/home/nqbh/reference/bib.bib}
\usepackage[utf8]{vietnam}
\usepackage{tocloft}
\renewcommand{\cftsecleader}{\cftdotfill{\cftdotsep}}
\usepackage[colorlinks=true,linkcolor=blue,urlcolor=red,citecolor=magenta]{hyperref}
\usepackage{amsmath,amssymb,amsthm,mathtools,float,graphicx,algpseudocode,algorithm,tcolorbox,tikz,tkz-tab,subcaption}
\DeclareMathOperator{\arccot}{arccot}
\usepackage[inline]{enumitem}
\allowdisplaybreaks
\numberwithin{equation}{section}
\newtheorem{assumption}{Assumption}[section]
\newtheorem{nhanxet}{Nhận xét}[section]
\newtheorem{conjecture}{Conjecture}[section]
\newtheorem{corollary}{Corollary}[section]
\newtheorem{hequa}{Hệ quả}[section]
\newtheorem{definition}{Definition}[section]
\newtheorem{dinhnghia}{Định nghĩa}[section]
\newtheorem{example}{Example}[section]
\newtheorem{vidu}{Ví dụ}[section]
\newtheorem{lemma}{Lemma}[section]
\newtheorem{notation}{Notation}[section]
\newtheorem{principle}{Principle}[section]
\newtheorem{problem}{Problem}[section]
\newtheorem{baitoan}{Bài toán}[section]
\newtheorem{proposition}{Proposition}[section]
\newtheorem{menhde}{Mệnh đề}[section]
\newtheorem{question}{Question}[section]
\newtheorem{cauhoi}{Câu hỏi}[section]
\newtheorem{quytac}{Quy tắc}
\newtheorem{remark}{Remark}[section]
\newtheorem{luuy}{Lưu ý}[section]
\newtheorem{theorem}{Theorem}[section]
\newtheorem{tiende}{Tiên đề}[section]
\newtheorem{dinhly}{Định lý}[section]
\usepackage[left=0.5in,right=0.5in,top=1.5cm,bottom=1.5cm]{geometry}
\usepackage{fancyhdr}
\pagestyle{fancy}
\fancyhf{}
\lhead{\small Subsect.~\thesubsection}
\rhead{\small\nouppercase{\leftmark}}
\renewcommand{\subsectionmark}[1]{\markboth{#1}{}}
\cfoot{\thepage}
\def\labelitemii{$\circ$}

\title{Functional Equation}
\author{Nguyễn Quản Bá Hồng\footnote{Independent Researcher, Ben Tre City, Vietnam\\e-mail: \texttt{nguyenquanbahong@gmail.com}; website: \url{https://nqbh.github.io}.}}
\date{\today}

\begin{document}
\maketitle
\begin{abstract}
	
\end{abstract}
\setcounter{secnumdepth}{4}
\setcounter{tocdepth}{3}
\tableofcontents
\newpage

%------------------------------------------------------------------------------%

\section{\href{https://en.wikipedia.org/wiki/Functional_equation}{Wikipedia\texttt{/}Functional Equation}}
``In mathematics, a \textit{functional equation} is, in the broadest meaning, an equation in which 1 or several functions appear as \href{https://en.wikipedia.org/wiki/Unknown_(mathematics)}{unknowns}. So, \href{https://en.wikipedia.org/wiki/Differential_equation}{differential equations} \& \href{https://en.wikipedia.org/wiki/Integral_equation}{integral equations} are functional equations. However, a more restricted meaning is often used, where a \textit{functional equation} is an equation that relates several rules of the same function. E.g., the \href{https://en.wikipedia.org/wiki/Logarithm_function}{logarithm functions} are \href{https://en.wikipedia.org/wiki/Logarithm#Characterization_by_the_product_formula}{essentially characterized} by the \textit{logarithmic functional equation} $\log(xy) = \log x + \log y$.

In the \href{https://en.wikipedia.org/wiki/Domain_of_a_function}{domain} of the unknown function is supposed to be the \href{https://en.wikipedia.org/wiki/Natural_number}{natural numbers}, the function is generally viewed as a \href{https://en.wikipedia.org/wiki/Sequence_(mathematics)}{sequence}, \&, in this case, a functional equation (in the narrower meaning) is called a \href{https://en.wikipedia.org/wiki/Recurrence_relation}{recurrence relation}. Thus the term \textit{functional equation} is used mainly for \href{https://en.wikipedia.org/wiki/Real_function}{real functions} \& \href{https://en.wikipedia.org/wiki/Complex_function}{complex functions}. Moreover a \href{https://en.wikipedia.org/wiki/Smooth_function}{smoothness condition} is often assumed for the solutions, since without such a condition, most functional equations have very irregular solutions. E.g., the \href{https://en.wikipedia.org/wiki/Gamma_function}{gamma function} is a function that satisfies the functional equation $f(x + 1) = xf(x)$ \& the initial value $f(1) = 1$. There are many functions that satisfy these conditions, but the gamma function is the unique one that is \href{https://en.wikipedia.org/wiki/Meromorphic_function}{meromorphic} in the whole complex plane, \& \href{https://en.wikipedia.org/wiki/Logarithmically_convex_function}{logarithmically convex} for $x$ real \& positive (\href{https://en.wikipedia.org/wiki/Bohr%E2%80%93Mollerup_theorem}{Bohr--Mollerup theorem}).'' -- \href{https://en.wikipedia.org/wiki/Functional_equation}{Wikipedia\texttt{/}functional equation}

\subsection{Examples}
\begin{enumerate*}
	\item[$\bullet$] ``\href{https://en.wikipedia.org/wiki/Recurrence_relation}{Recurrence relations} can be seen as functional equations in functions over the integers or natural numbers, in which the differences between terms' indexes can be seen as an application of the \href{https://en.wikipedia.org/wiki/Shift_operator}{shift operator}. E.g., the recurrence relation defining the \href{https://en.wikipedia.org/wiki/Fibonacci_numbers}{Fibonacci numbers}, $F_n = F_{n-1} + F_{n-2}$, where $F_0 = 0$ \& $F_1 = 1$.
	\item[$\bullet$] $f(x + P) = f(x)$, which characterizes the \href{https://en.wikipedia.org/wiki/Periodic_function}{periodic functions}.
	\item[$\bullet$] $f(x) = f(-x)$, which characterizes the \href{https://en.wikipedia.org/wiki/Even_function}{even functions}, \& likewise $f(x) = -f(-x)$, which characterizes the \href{https://en.wikipedia.org/wiki/Odd_function}{odd functions}.
	\item[$\bullet$] $f(f(x)) = g(x)$, which characterizes the \href{https://en.wikipedia.org/wiki/Functional_square_root}{functional square root} of the function $g$.
	\item[$\bullet$] $f(x + y) = f(x) + f(y)$ (\href{https://en.wikipedia.org/wiki/Cauchy%27s_functional_equation}{Cauchy's functional equation}), satisfied by \href{https://en.wikipedia.org/wiki/Linear_map}{linear maps}. The equation may, contigent on the \href{https://en.wikipedia.org/wiki/Axiom_of_choice}{axiom of choice}, also have other pathological nonlinear solutions, whose existence can be proven with a \href{https://en.wikipedia.org/wiki/Hamel_basis}{Hamel basis} for the real numbers.
	\item[$\bullet$] $f(x + y) = f(x) + f(y)$, satisfied by all \href{https://en.wikipedia.org/wiki/Exponential_function}{exponential functions}. Like Cauchy's additive functional equation, this too may have pathological, discontinuous solutions. $\ldots$
\end{enumerate*}
'' -- \href{https://en.wikipedia.org/wiki/Functional_equation#Examples}{Wikipedia\texttt{/}functional equation\texttt{/}example}

\subsection{Solution}
``1 method of solving elementary functional equations is substitution. Some solutions to functional equations have exploited \href{https://en.wikipedia.org/wiki/Surjective}{surjectivity}, \href{https://en.wikipedia.org/wiki/Injective_function}{injectivity}, \href{https://en.wikipedia.org/wiki/Odd_function}{oddness}, \& \href{https://en.wikipedia.org/wiki/Even_function}{evenness}.

Some functional equations have been solved with the use of \href{https://en.wikipedia.org/wiki/Ansatz}{ansatzes}, \href{https://en.wikipedia.org/wiki/Mathematical_induction}{mathematical induction}.

Some classes of functional equations can be solved by computer-assisted techniques.

In \href{https://en.wikipedia.org/wiki/Dynamic_programming}{dynamic programming} a variety of successive approximation methods are used to solve \href{https://en.wikipedia.org/wiki/Bellman_equation}{Bellman's functional equation}, including methods based on \href{https://en.wikipedia.org/wiki/Fixed_point_iteration}{fixed point iterations}.'' -- \href{https://en.wikipedia.org/wiki/Functional_equation#Solution}{Wikipedia\texttt{/}functional equation\texttt{/}solution}

%------------------------------------------------------------------------------%

\printbibliography[heading=bibintoc]
	
\end{document}