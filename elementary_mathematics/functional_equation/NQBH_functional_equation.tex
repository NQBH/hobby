\documentclass{article}
\usepackage[backend=biber,natbib=true,style=authoryear]{biblatex}
\addbibresource{/home/nqbh/reference/bib.bib}
\usepackage[utf8]{vietnam}
\usepackage{tocloft}
\renewcommand{\cftsecleader}{\cftdotfill{\cftdotsep}}
\usepackage[colorlinks=true,linkcolor=blue,urlcolor=red,citecolor=magenta]{hyperref}
\usepackage{amsmath,amssymb,amsthm,mathtools,float,graphicx,algpseudocode,algorithm,tcolorbox,tikz,tkz-tab,subcaption}
\DeclareMathOperator{\arccot}{arccot}
\usepackage[inline]{enumitem}
\allowdisplaybreaks
\numberwithin{equation}{section}
\newtheorem{assumption}{Assumption}[section]
\newtheorem{nhanxet}{Nhận xét}[section]
\newtheorem{conjecture}{Conjecture}[section]
\newtheorem{corollary}{Corollary}[section]
\newtheorem{hequa}{Hệ quả}[section]
\newtheorem{definition}{Definition}[section]
\newtheorem{dinhnghia}{Định nghĩa}[section]
\newtheorem{example}{Example}[section]
\newtheorem{vidu}{Ví dụ}[section]
\newtheorem{lemma}{Lemma}[section]
\newtheorem{notation}{Notation}[section]
\newtheorem{principle}{Principle}[section]
\newtheorem{problem}{Problem}[section]
\newtheorem{baitoan}{Bài toán}[section]
\newtheorem{proposition}{Proposition}[section]
\newtheorem{menhde}{Mệnh đề}[section]
\newtheorem{question}{Question}[section]
\newtheorem{cauhoi}{Câu hỏi}[section]
\newtheorem{quytac}{Quy tắc}
\newtheorem{remark}{Remark}[section]
\newtheorem{luuy}{Lưu ý}[section]
\newtheorem{theorem}{Theorem}[section]
\newtheorem{tiende}{Tiên đề}[section]
\newtheorem{dinhly}{Định lý}[section]
\usepackage[left=0.5in,right=0.5in,top=1.5cm,bottom=1.5cm]{geometry}
\usepackage{fancyhdr}
\pagestyle{fancy}
\fancyhf{}
\lhead{\small Subsect.~\thesubsection}
\rhead{\small\nouppercase{\leftmark}}
\renewcommand{\subsectionmark}[1]{\markboth{#1}{}}
\cfoot{\thepage}
\def\labelitemii{$\circ$}

\title{Functional Equation}
\author{Nguyễn Quản Bá Hồng\footnote{Independent Researcher, Ben Tre City, Vietnam\\e-mail: \texttt{nguyenquanbahong@gmail.com}; website: \url{https://nqbh.github.io}.}}
\date{\today}

\begin{document}
\maketitle
\begin{abstract}
	
\end{abstract}
\setcounter{secnumdepth}{4}
\setcounter{tocdepth}{3}
\tableofcontents
\newpage

%------------------------------------------------------------------------------%

\section{\href{https://en.wikipedia.org/wiki/Functional_equation}{Wikipedia\texttt{/}Functional Equation}}
``In mathematics, a \textit{functional equation} is, in the broadest meaning, an equation in which 1 or several functions appear as \href{https://en.wikipedia.org/wiki/Unknown_(mathematics)}{unknowns}. So, \href{https://en.wikipedia.org/wiki/Differential_equation}{differential equations} \& \href{https://en.wikipedia.org/wiki/Integral_equation}{integral equations} are functional equations. However, a more restricted meaning is often used, where a \textit{functional equation} is an equation that relates several rules of the same function. E.g., the \href{https://en.wikipedia.org/wiki/Logarithm_function}{logarithm functions} are \href{https://en.wikipedia.org/wiki/Logarithm#Characterization_by_the_product_formula}{essentially characterized} by the \textit{logarithmic functional equation} $\log(xy) = \log x + \log y$.'' -- \href{https://en.wikipedia.org/wiki/Functional_equation}{Wikipedia\texttt{/}functional equation}

%------------------------------------------------------------------------------%

\printbibliography[heading=bibintoc]
	
\end{document}