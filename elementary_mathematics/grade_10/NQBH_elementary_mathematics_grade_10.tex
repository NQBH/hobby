\documentclass[oneside]{book}
\usepackage[backend=biber,natbib=true,style=authoryear]{biblatex}
\addbibresource{/home/hong/1_NQBH/reference/bib.bib}
\usepackage[utf8]{vietnam}
\usepackage{tocloft}
\renewcommand{\cftsecleader}{\cftdotfill{\cftdotsep}}
\usepackage[colorlinks=true,linkcolor=blue,urlcolor=red,citecolor=magenta]{hyperref}
\usepackage{amsmath,amssymb,amsthm,mathtools,float,graphicx,algpseudocode,algorithm,tcolorbox,tikz,tkz-tab}
\DeclareMathOperator{\arccot}{arccot}
\usepackage[inline]{enumitem}
\allowdisplaybreaks
\numberwithin{equation}{section}
\newtheorem{assumption}{Assumption}[section]
\newtheorem{nhanxet}{Nhận xét}[section]
\newtheorem{conjecture}{Conjecture}[section]
\newtheorem{corollary}{Corollary}[section]
\newtheorem{hequa}{Hệ quả}[section]
\newtheorem{definition}{Definition}[section]
\newtheorem{dinhnghia}{Định nghĩa}[section]
\newtheorem{example}{Example}[section]
\newtheorem{vidu}{Ví dụ}[section]
\newtheorem{lemma}{Lemma}[section]
\newtheorem{notation}{Notation}[section]
\newtheorem{principle}{Principle}[section]
\newtheorem{problem}{Problem}[section]
\newtheorem{baitoan}{Bài toán}[section]
\newtheorem{proposition}{Proposition}[section]
\newtheorem{menhde}{Mệnh đề}[section]
\newtheorem{question}{Question}[section]
\newtheorem{cauhoi}{Câu hỏi}[section]
\newtheorem{remark}{Remark}[section]
\newtheorem{luuy}{Lưu ý}[section]
\newtheorem{theorem}{Theorem}[section]
\newtheorem{dinhly}{Định lý}[section]
\usepackage[left=0.5in,right=0.5in,top=1.5cm,bottom=1.5cm]{geometry}
\usepackage{fancyhdr}
\pagestyle{fancy}
\fancyhf{}
\lhead{\small \textsc{Sect.} ~\thesection}
\rhead{\small \nouppercase{\leftmark}}
\renewcommand{\sectionmark}[1]{\markboth{#1}{}}
\cfoot{\thepage}
\def\labelitemii{$\circ$}

\title{Some Topics in Elementary Mathematics\texttt{/}Grade 10}
\author{Nguyễn Quản Bá Hồng\footnote{Independent Researcher, Ben Tre City, Vietnam\\e-mail: \texttt{nguyenquanbahong@gmail.com}; website: \url{https://nqbh.github.io}.}}
\date{\today}

\begin{document}
\frontmatter
\maketitle
\setcounter{secnumdepth}{4}
\setcounter{tocdepth}{3}
\tableofcontents
\newpage

%------------------------------------------------------------------------------%

\mainmatter

\chapter*{Preface}

Tóm tắt kiến thức Toán lớp 10 theo chương trình giáo dục của Việt Nam \& một số chủ đề nâng cao.

%------------------------------------------------------------------------------%

\chapter{Mệnh Đề Toán học. Tập Hợp}

\begin{quotation}
	\textbf{Nội dung.} \textit{Mệnh đề toán học, tập hợp \& các phép toán trên tập hợp}.
\end{quotation}

\section{Mệnh Đề Toán Học}

\subsection{Mệnh đề toán học}

\begin{dinhnghia}[Mệnh đề toán học]
	1 mệnh đề khẳng định về 1 sự kiện trong toán học, gọi là \emph{mệnh đề toán học}.
\end{dinhnghia}
``Khi không sợ nhầm lẫn, ta thường gọi tắt mệnh đề toán học là \textit{mệnh đề}.'' -- \cite[p. 5]{SGK_Toan_10_Canh_Dieu_tap_1}

\begin{menhde}
	 Mỗi mệnh đề toán học phải hoặc đúng hoặc sai. 1 mệnh đề toán học không thể vừa đúng, vừa sai.
\end{menhde}

\begin{dinhnghia}[Mệnh đề đúng\texttt{/}sai]
	Khi mệnh đề toán học là đúng, ta gọi mệnh đề đó là 1 \textit{mệnh đề đúng}. Khi mệnh đề toán học là sai, ta gọi mệnh đề đó là 1 \textit{mệnh đề sai}.
\end{dinhnghia}

\subsection{Mệnh đề chứa biến}

\begin{dinhnghia}[Mệnh đề chứa biến]
	Với mỗi bộ giá trị cụ thể của bộ biến $(x_1,\ldots,x_n)$, $n\in\mathbb{N}^\star$, mệnh đề $P(x_1,\ldots,x_n)$ cho ta 1 mệnh đề toán học mà ta có thể khẳng định được tính đúng sai của mệnh đề đó. Khi đó, $P(x_1,\ldots,x_n)$ được gọi là \emph{mệnh đề chứa biến}.
\end{dinhnghia}
``Ta thường ký hiệu mệnh đề chứa biến $n$ là $P(n)$; mệnh đề chứa biến $x,y$ là $P(x,y)$; $\ldots$'' -- \cite[p. 6]{SGK_Toan_10_Canh_Dieu_tap_1}

\subsection{Phủ định của 1 mệnh đề}

\begin{dinhnghia}[Mệnh đề phủ định]
	Cho mệnh đề $P$. Mệnh đề ``Không phải $P$'' được gọi là \emph{mệnh đề phủ định} của mệnh đề $P$ \& ký hiệu là $\overline{P}$.
\end{dinhnghia}
``Mệnh đề $\overline{P}$ \textit{đúng} khi $P$ sai. Mệnh đề $\overline{P}$ \textit{sai} khi $P$ đúng.'' ``Để phủ định 1 mệnh đề, ta chỉ cần thêm\texttt{/}bớt từ ``không'' (hoặc ``không phải'') vào trước vị ngữ của mệnh đề đó.'' -- \cite[p. 7]{SGK_Toan_10_Canh_Dieu_tap_1}

\subsection{Mệnh đề kéo theo}

\begin{dinhnghia}[Mệnh đề kéo theo]
	Cho 2 mệnh đề $P$ \& $Q$. Mệnh đề ``Nếu $P$ thì $Q$'' được gọi là \emph{mệnh đề kéo theo} \& ký hiệu là $P\Rightarrow Q$. mệnh đề $P\Rightarrow Q$ sai khi $P$ đúng, $Q$ sai \& đúng trong các trường hợp còn lại.
\end{dinhnghia}
``Tùy theo nội dung cụ thể, đôi khi người ta còn phát biểu mệnh đề $P\Rightarrow Q$ là ``$P$ kéo theo $Q$'' hay ``$P$ suy ra $Q$'' hay ``Vì $P$ nên $Q$'' $\ldots$'' ``Các định lý toán học là những mệnh đề đúng \& thường phát biểu ở dạng mệnh đề kéo theo $P\Rightarrow Q$. Khi đó ta nói: $P$ là \textit{giả thiết}, $Q$ là \textit{kết luận} của định lý, hay $P$ là \textit{điều kiện đủ} để có $Q$, hoặc $Q$ là \textit{điều kiện cần} để có $P$.'' -- \cite[p. 7]{SGK_Toan_10_Canh_Dieu_tap_1}

\subsection{Mệnh đề đảo. 2 mệnh đề tương đương}

\begin{dinhnghia}[Mệnh đề đảo, 2 mệnh đề tương đương]
	Mệnh đề $Q\Rightarrow P$ được gọi là \emph{mệnh đề đảo} của mệnh đề $P\Rightarrow Q$. Nếu cả 2 mệnh đề $P\Rightarrow Q$ \& $Q\Rightarrow P$ đều đúng thì ta nói $P$ \& $Q$ là \emph{2 mệnh đề tương đương}, ký hiệu $P\Leftrightarrow Q$.
\end{dinhnghia}
``Mệnh đề $P\Leftrightarrow Q$ có thể phát biểu ở những dạng như sau: ``$P$ tương đương $Q$''; ``$P$ là điều kiện cần \& đủ để có $Q$''; ``$P$ khi \& chỉ khi $Q$''; ``$P$ nếu \& chỉ nếu $Q$''.'' -- \cite[p. 8]{SGK_Toan_10_Canh_Dieu_tap_1}

``Trong toán học, những câu khẳng định đúng phát biểu ở dạng ``$P\Leftrightarrow Q$'' cũng được coi là 1 mệnh đề toán học, gọi là \textit{mệnh đề tương đương}.'' -- \cite[p. 9]{SGK_Toan_10_Canh_Dieu_tap_1}

\subsection{Ký hiệu $\forall$ \& $\exists$}
$\forall$: ``với mọi'', $\exists$: ``tồn tại'' hoặc ``có 1'' (tồn tại 1) hoặc ``có ít nhất 1'' (tồn tại ít nhất 1). Phương pháp chứng minh 1 mệnh đề có ký hiệu ``$\forall$'', ``$\exists$'', là đúng hoặc sai.

\begin{menhde}
	Cho mệnh đề ``$P(x)$, $x\in X$''. Phủ định của mệnh đề $\forall x\in X$, $P(x)$'' là mệnh đề ``$\exists x\in X$, $\overline{P(x)}$''. Phủ định của mệnh đề $\exists x\in X$, $P(x)$'' là mệnh đề ``$\forall x\in X$, $\overline{P(x)}$''.
\end{menhde}

\section{Tập Hợp. Các Phép Toán Trên Tập Hợp}

\subsection{Tập hợp}


%------------------------------------------------------------------------------%

\chapter{Bất Phương Trình \& Hệ Bất Phương Trình Bậc Nhất 2 Ẩn}

\section{Bất Phương Trình Bậc Nhất 2 Ẩn}

\section{Hệ Bất Phương Trình Bậc Nhất 2 Ẩn}

%------------------------------------------------------------------------------%

\chapter{Hàm Số \& Đồ Thị}

\section{Hàm Số \& Đồ Thị}

\section{Hàm Số Bậc 2. Đồ Thị Hàm Số Bậc 2 \& Ứng Dụng}

\section{Dấu của Tam Thức Bậc 2}

\section{Bất Phương Trình Bậc 2 1 Ẩn}

\section{2 Dạng Phương Trình Quy về Phương Trình Bậc 2}

%------------------------------------------------------------------------------%

\chapter{Hệ Thức Lượng Trong Tam Giác. Vector}

\section{Giá Trị Lượng Giác của 1 Góc$\in[0^\circ;180^\circ]$. Định Lý Côsin \& Định Lý Sin Trong Tam Giác}

\section{Giải Tam Giác}

\section{Khái Niệm Vector}

\section{Tổng \& Hiệu của 2 Vector}

\section{Tích của 1 Số với 1 Vector}

\section{Tích Vô Hướng của 2 Vector}

%------------------------------------------------------------------------------%

\begin{thebibliography}{99}
	\bibitem[NQBH\texttt{/}elementary math]{NQBH/elementary math} Nguyễn Quản Bá Hồng. \href{https://github.com/NQBH/hobby/blob/master/elementary_mathematics/some_topics_in_elementary_mathematics_problems_theories_applications_bridges_to_advanced_mathematics/NQBH_some_topics_in_elementary_mathematics_problems_theories_applications_bridges_to_advanced_mathematics.pdf}{\textit{Some Topics in Elementary Mathematics: Problems, Theories, Applications, \& Bridges to Advanced Mathematics}}. Mar 2022--now.
\end{thebibliography}

%------------------------------------------------------------------------------%

\printbibliography[heading=bibintoc]

\end{document}