\documentclass[oneside]{book}
\usepackage[backend=biber,natbib=true,style=authoryear]{biblatex}
\addbibresource{/home/hong/1_NQBH/reference/bib.bib}
\usepackage[utf8]{vietnam}
\usepackage{tocloft}
\renewcommand{\cftsecleader}{\cftdotfill{\cftdotsep}}
\usepackage[colorlinks=true,linkcolor=blue,urlcolor=red,citecolor=magenta]{hyperref}
\usepackage{amsmath,amssymb,amsthm,mathtools,float,graphicx,algpseudocode,algorithm,tcolorbox,tikz,tkz-tab}
\DeclareMathOperator{\arccot}{arccot}
\usepackage[inline]{enumitem}
\allowdisplaybreaks
\numberwithin{equation}{section}
\newtheorem{assumption}{Assumption}[section]
\newtheorem{nhanxet}{Nhận xét}[section]
\newtheorem{conjecture}{Conjecture}[section]
\newtheorem{corollary}{Corollary}[section]
\newtheorem{hequa}{Hệ quả}[section]
\newtheorem{definition}{Definition}[section]
\newtheorem{dinhnghia}{Định nghĩa}[section]
\newtheorem{example}{Example}[section]
\newtheorem{vidu}{Ví dụ}[section]
\newtheorem{lemma}{Lemma}[section]
\newtheorem{notation}{Notation}[section]
\newtheorem{principle}{Principle}[section]
\newtheorem{problem}{Problem}[section]
\newtheorem{baitoan}{Bài toán}[section]
\newtheorem{proposition}{Proposition}[section]
\newtheorem{menhde}{Mệnh đề}[section]
\newtheorem{question}{Question}[section]
\newtheorem{cauhoi}{Câu hỏi}[section]
\newtheorem{remark}{Remark}[section]
\newtheorem{luuy}{Lưu ý}[section]
\newtheorem{theorem}{Theorem}[section]
\newtheorem{dinhly}{Định lý}[section]
\usepackage[left=0.5in,right=0.5in,top=1.5cm,bottom=1.5cm]{geometry}
\usepackage{fancyhdr}
\pagestyle{fancy}
\fancyhf{}
\lhead{\small \textsc{Sect.} ~\thesection}
\rhead{\small \nouppercase{\leftmark}}
\renewcommand{\sectionmark}[1]{\markboth{#1}{}}
\cfoot{\thepage}
\def\labelitemii{$\circ$}

\title{Some Topics in Elementary Mathematics\texttt{/}Grade 10}
\author{Nguyễn Quản Bá Hồng\footnote{Independent Researcher, Ben Tre City, Vietnam\\e-mail: \texttt{nguyenquanbahong@gmail.com}; website: \url{https://nqbh.github.io}.}}
\date{\today}

\begin{document}
\frontmatter
\maketitle
\setcounter{secnumdepth}{4}
\setcounter{tocdepth}{3}
\tableofcontents
\newpage

%------------------------------------------------------------------------------%

\mainmatter

\chapter*{Preface}

Tóm tắt kiến thức Toán lớp 10 theo chương trình giáo dục của Việt Nam \& một số chủ đề nâng cao.

%------------------------------------------------------------------------------%

\chapter{Mệnh Đề Toán học. Tập Hợp}

\begin{quotation}
	\textbf{Nội dung.} \textit{Mệnh đề toán học, tập hợp \& các phép toán trên tập hợp}.
\end{quotation}

\section{Mệnh Đề Toán Học}

\subsection{Mệnh đề toán học}

\begin{dinhnghia}[Mệnh đề toán học]
	1 mệnh đề khẳng định về 1 sự kiện trong toán học, gọi là \emph{mệnh đề toán học}.
\end{dinhnghia}
``Khi không sợ nhầm lẫn, ta thường gọi tắt mệnh đề toán học là \textit{mệnh đề}.'' -- \cite[p. 5]{SGK_Toan_10_Canh_Dieu_tap_1}

\begin{menhde}
	 Mỗi mệnh đề toán học phải hoặc đúng hoặc sai. 1 mệnh đề toán học không thể vừa đúng, vừa sai.
\end{menhde}

\begin{dinhnghia}[Mệnh đề đúng\texttt{/}sai]
	Khi mệnh đề toán học là đúng, ta gọi mệnh đề đó là 1 \textit{mệnh đề đúng}. Khi mệnh đề toán học là sai, ta gọi mệnh đề đó là 1 \textit{mệnh đề sai}.
\end{dinhnghia}

\subsection{Mệnh đề chứa biến}

\begin{dinhnghia}[Mệnh đề chứa biến]
	Với mỗi bộ giá trị cụ thể của bộ biến $(x_1,\ldots,x_n)$, $n\in\mathbb{N}^\star$, mệnh đề $P(x_1,\ldots,x_n)$ cho ta 1 mệnh đề toán học mà ta có thể khẳng định được tính đúng sai của mệnh đề đó. Khi đó, $P(x_1,\ldots,x_n)$ được gọi là \emph{mệnh đề chứa biến}.
\end{dinhnghia}
``Ta thường ký hiệu mệnh đề chứa biến $n$ là $P(n)$; mệnh đề chứa biến $x,y$ là $P(x,y)$; $\ldots$'' -- \cite[p. 6]{SGK_Toan_10_Canh_Dieu_tap_1}

\subsection{Phủ định của 1 mệnh đề}

\begin{dinhnghia}[Mệnh đề phủ định]
	Cho mệnh đề $P$. Mệnh đề ``Không phải $P$'' được gọi là \emph{mệnh đề phủ định} của mệnh đề $P$ \& ký hiệu là $\overline{P}$.
\end{dinhnghia}
``Mệnh đề $\overline{P}$ \textit{đúng} khi $P$ sai. Mệnh đề $\overline{P}$ \textit{sai} khi $P$ đúng.'' ``Để phủ định 1 mệnh đề, ta chỉ cần thêm\texttt{/}bớt từ ``không'' (hoặc ``không phải'') vào trước vị ngữ của mệnh đề đó.'' -- \cite[p. 7]{SGK_Toan_10_Canh_Dieu_tap_1}

\subsection{Mệnh đề kéo theo}

\begin{dinhnghia}[Mệnh đề kéo theo]
	Cho 2 mệnh đề $P$ \& $Q$. Mệnh đề ``Nếu $P$ thì $Q$'' được gọi là \emph{mệnh đề kéo theo} \& ký hiệu là $P\Rightarrow Q$. mệnh đề $P\Rightarrow Q$ sai khi $P$ đúng, $Q$ sai \& đúng trong các trường hợp còn lại.
\end{dinhnghia}
``Tùy theo nội dung cụ thể, đôi khi người ta còn phát biểu mệnh đề $P\Rightarrow Q$ là ``$P$ kéo theo $Q$'' hay ``$P$ suy ra $Q$'' hay ``Vì $P$ nên $Q$'' $\ldots$'' ``Các định lý toán học là những mệnh đề đúng \& thường phát biểu ở dạng mệnh đề kéo theo $P\Rightarrow Q$. Khi đó ta nói: $P$ là \textit{giả thiết}, $Q$ là \textit{kết luận} của định lý, hay $P$ là \textit{điều kiện đủ} để có $Q$, hoặc $Q$ là \textit{điều kiện cần} để có $P$.'' -- \cite[p. 7]{SGK_Toan_10_Canh_Dieu_tap_1}

\subsection{Mệnh đề đảo. 2 mệnh đề tương đương}

\begin{dinhnghia}[Mệnh đề đảo, 2 mệnh đề tương đương]
	Mệnh đề $Q\Rightarrow P$ được gọi là \emph{mệnh đề đảo} của mệnh đề $P\Rightarrow Q$. Nếu cả 2 mệnh đề $P\Rightarrow Q$ \& $Q\Rightarrow P$ đều đúng thì ta nói $P$ \& $Q$ là \emph{2 mệnh đề tương đương}, ký hiệu $P\Leftrightarrow Q$.
\end{dinhnghia}
``Mệnh đề $P\Leftrightarrow Q$ có thể phát biểu ở những dạng như sau: ``$P$ tương đương $Q$''; ``$P$ là điều kiện cần \& đủ để có $Q$''; ``$P$ khi \& chỉ khi $Q$''; ``$P$ nếu \& chỉ nếu $Q$''.'' -- \cite[p. 8]{SGK_Toan_10_Canh_Dieu_tap_1}

``Trong toán học, những câu khẳng định đúng phát biểu ở dạng ``$P\Leftrightarrow Q$'' cũng được coi là 1 mệnh đề toán học, gọi là \textit{mệnh đề tương đương}.'' -- \cite[p. 9]{SGK_Toan_10_Canh_Dieu_tap_1}

\subsection{Ký hiệu $\forall$ \& $\exists$}
$\forall$: ``với mọi'', $\exists$: ``tồn tại'' hoặc ``có 1'' (tồn tại 1) hoặc ``có ít nhất 1'' (tồn tại ít nhất 1). Phương pháp chứng minh 1 mệnh đề có ký hiệu ``$\forall$'', ``$\exists$'', là đúng hoặc sai.

\begin{menhde}
	Cho mệnh đề ``$P(x)$, $x\in X$''. Phủ định của mệnh đề $\forall x\in X$, $P(x)$'' là mệnh đề ``$\exists x\in X$, $\overline{P(x)}$''. Phủ định của mệnh đề $\exists x\in X$, $P(x)$'' là mệnh đề ``$\forall x\in X$, $\overline{P(x)}$''.
\end{menhde}

\section{Tập Hợp. Các Phép Toán Trên Tập Hợp}

\subsection{Tập hợp}
``Người ta còn minh họa tập hợp bằng 1 vòng kín, mỗi phần tử của tập hợp được biểu diễn bởi 1 chấm bên trong vòng kín, còn phần tử không thuộc tập hợp đó được biểu diễn bởi 1 chấm bên ngoài vòng kín. Cách minh họa tập hợp như vậy được gọi là biểu đồ Venn.'' -- \cite[p. 12]{SGK_Toan_10_Canh_Dieu_tap_1}

``Tập hợp không chứa phần tử nào được gọi là tập hợp rỗng, ký hiệu là $\emptyset$. 1 tập hợp có thể không có phần tử nào, cũng có thể có 1 phần tử, có nhiều phần tử, có vô số phần tử. Khi tập hợp $C$ là tập hợp rỗng, ta viết $C = \emptyset$ \& không được viết là $C = \{\emptyset\}$.'' -- \cite[p. 13]{SGK_Toan_10_Canh_Dieu_tap_1}

\subsection{Tập con \& tập hợp bằng nhau}

\subsubsection{Tập con}

\begin{dinhnghia}
	Nếu mọi phần tử của tập hợp $A$ đều là phần tử của tập hợp $B$ thì ta nói $A$ là 1 \emph{tập con} của tập hợp $B$ \& viết là $A\subset B$. Ta còn đọc là $A$ chứa trong $B$.
\end{dinhnghia}
``\textit{Quy ước:} Tập hợp rỗng được coi là tập con của mọi tập hợp.'' ``$A\subset B\Leftrightarrow(\forall x,\ x\in A\Rightarrow x\in B)$. Khi $A\subset B$, ta cũng viết $B\supset A$ (đọc là $B$ chứa $A$). Nếu $A$ không phải là tập con của $B$, ta viết $A\not\subset B$.'' -- \cite[p. 13]{SGK_Toan_10_Canh_Dieu_tap_1}

\begin{menhde}
	$A\subset A$ với mọi tập hợp $A$. Nếu $A\subset B$ \& $B\subset C$ thì $A\subset C$.
\end{menhde}
Tính chất $((A\subset B)\land(B\subset C))\Rightarrow(A\subset C)$ được gọi là \textit{tính chất bắc cầu}.

\subsubsection{Tập hợp bằng nhau}

\begin{dinhnghia}
	Khi $A\subset B$ \& $B\subset A$ thì ta nói 2 tập hợp $A$ \& $B$ \emph{bằng nhau}, viết là $A = B$.
\end{dinhnghia}

\subsection{Giao của 2 tập hợp}

\begin{dinhnghia}[Giao của 2 tập hợp]
	Tập hợp gồm tất cả các phần tử vừa thuộc $A$ vừa thuộc $B$ được gọi là \emph{giao} của $A$ \& $B$, ký hiệu $A\cap B$.
\end{dinhnghia}
``Vậy $A\cap B = \{x|x\in A\mbox{ \& } x\in B\}$.'' ``$x\in A\cap B$ khi \& chỉ khi $x\in A$ \& $x\in B$.'' -- \cite[p. 14]{SGK_Toan_10_Canh_Dieu_tap_1}, i.e., $(x\in A\cap B)\Leftrightarrow((x\in A)\land(x\in B))$.

\subsection{Hợp của 2 tập hợp}

\begin{dinhnghia}[Hợp của 2 tập hợp]
	Tập hợp gồm các phần tử thuộc $A$ hoặc thuộc $B$ được gọi là \emph{hợp} của $A$ \& $B$, ký hiệu $A\cup B$.
\end{dinhnghia}
``Vậy $A\cup B = \{x|x\in A\mbox{ hoặc } x\in B\}$.'' $x\in A\cup B$ khi \& chỉ khi $x\in A$ \& $x\in B$.'' -- \cite[p. 15]{SGK_Toan_10_Canh_Dieu_tap_1}, i.e., $(x\in A\cup B)\Leftrightarrow((x\in A)\lor(x\in B))$.

\begin{vidu}
	Với tập hợp $\mathbb{Q}$ các số hũu tỷ \& tập hợp $I$ các số vô tỷ. $\mathbb{Q}\cap I = \emptyset$, $\mathbb{Q}\cup I = \mathbb{R}$.
\end{vidu}

\subsection{Phần bù. Hiệu của 2 tập hợp}
``Tập hợp $\mathbb{Q}$ các số hữu tỷ là phần bù của tập hợp $I$ các số vô tỷ trong tập hợp  $\mathbb{R}$.'' -- \cite[p. 15]{SGK_Toan_10_Canh_Dieu_tap_1}

\begin{dinhnghia}[Phần bù]
	Cho tập hợp $A$ là tập con của tập hợp $B$. Tập hợp những phần tử $B$ mà không phải là phần tử của $A$ được gọi là \emph{phần bù} của $A$ trong $B$, ký hiệu $C_BA$.
\end{dinhnghia}
$B = A\cup C_BA$ \& $C_BA\cap A = \emptyset$, $\forall$ tập hợp $A,B$.

\begin{dinhnghia}[Hiệu của 2 tập hợp]
	Tập hợp gồm các phần tử thuộc $A$ nhưng không thuộc $B$ được gọi là \emph{hiệu} của $A$ \& $B$, ký hiệu $A\backslash B$.
\end{dinhnghia}
``Vậy $A\backslash B = \{x|x\in A\mbox{ \& } x\notin B\}$.'' ``$x\in A\backslash B$ khi \& chỉ khi $x\in A$ \& $x\notin B$.'' ``Nếu $B\subset A$ thì $A\backslash B = C_AB$.'' -- \cite[p. 16]{SGK_Toan_10_Canh_Dieu_tap_1}

\subsection{Các tập hợp số}

\subsubsection{Các tập hợp số đã học}
``Ta đã biết $\mathbb{N},\mathbb{Z},\mathbb{Q},\mathbb{R}$ lần lượt là tập hợp số tự nhiên, tập hợp số nguyên, tập hợp số hữu tỷ, tập hợp số thực. Ta có quan hệ sau: $\mathbb{N}\subset\mathbb{Z}\subset\mathbb{Q}\subset\mathbb{R}$.'' -- \cite[p. 17]{SGK_Toan_10_Canh_Dieu_tap_1}

\subsubsection{1 số tập con thường dùng của tập hợp số thực}
``$\mathbb{R}$: tập hợp số thực $(-\infty;+\infty)$. $\{x\in\mathbb{R}|a\le x\le b\}$: đoạn $[a;b]$. $\{x\in\mathbb{R}|a < x < b\}$: khoảng $(a;b)$. $\{x\in\mathbb{R}|x  > a\}$: khoảng $(a;+\infty)$. $\{x\in\mathbb{R}|x < b\}$: khoảng $(-\infty;b)$. $\{x\in\mathbb{R}|a\le x < b\}$: nửa khoảng $[a;b)$. $\{x\in\mathbb{R}|a < x\le b\}$: nửa khoảng $(a;b]$. $\{x\in\mathbb{R}|x\ge a\}$: nửa khoảng $[a;+\infty)$. $\{x\in\mathbb{R}|x\le b\}$: nửa khoảng $(-\infty;b]$. Ký hiệu $-\infty$ đọc là \textit{âm vô cực}, ký hiệu $+\infty$ đọc là \textit{dương vô cực}; $a$ \& $b$ được gọi là \textit{đầu mút} của các đoạn, khoảng, nửa khoảng. Ta cũng có thể biểu diễn tập hợp trên trục số bằng cách gạch bỏ phần không thuộc tập đó.'' -- \cite[p. 17]{SGK_Toan_10_Canh_Dieu_tap_1}

%------------------------------------------------------------------------------%

\chapter{Bất Phương Trình \& Hệ Bất Phương Trình Bậc Nhất 2 Ẩn}

\begin{quotation}
	\textbf{Nội dung.} \textit{Bất phương trình bậc nhất 2 ẩn; hệ bất phương trình bậc nhất 2 ẩn \& ứng dụng của chúng vào bài toán thực tiễn}.
\end{quotation}

\section{Bất Phương Trình Bậc Nhất 2 Ẩn}

\subsection{Bất phương trình bậc nhất 2 ẩn}

\begin{dinhnghia}[Bất phương trình bậc nhất 2 ẩn]
	\emph{Bất phương trình bậc nhất 2 ẩn $x,y$} là bất phương trình có 1 trong các dạng sau: $ax + by < c$, $ax + by > c$, $ax + by\le c$, $ax + by\ge c$, trong đó $a,b,c$ là những số cho trước với $a,b$ không đồng thời bằng $0$, $x$ \& $y$ là các ẩn.
	
	Cho bất phương trình bậc nhất 2 ẩn $ax + by < c$. Mỗi cặp số $(x_0;y_0)$ sao cho $ax_0 + by_0 < c\ (\star)$  được gọi là 1 \emph{nghiệm} của bất phương trình $(\star)$. Trong mặt phẳng tọa độ $Oxy$, tập hợp các điểm có tọa độ là nghiệm của bất phương trình $(\star)$ được gọi là \emph{miền nghiệm} của bất phương trình đó.
\end{dinhnghia}
``Nghiệm \& miền nghiệm của các bất phương trình dạng $ax + by > c$, $ax + by\le c$ \& $ax + by\ge c$ được định nghĩa tương tự.'' -- \cite[p. 21]{SGK_Toan_10_Canh_Dieu_tap_1}

\subsection{Biểu diễn miền nghiệm của bất phương trình bậc nhất 2 ẩn}

\subsubsection{Mô tả miền nghiệm của bất phương trình bậc nhất 2 ẩn}
``Người ta chứng minh được định lý sau: 

\begin{dinhly}
	Trong mặt phẳng tọa độ $Oxy$, phương trình $ax + by = c$ (với $a$ \& $b$ không đồng thời bằng $0$) xác định 1 đường thẳng $d$ như sau:
	\begin{itemize}
		\item $d$ có phương trình là $x = \frac{c}{a}$ nếu $b = 0$;
		\item $d$ có phương trình là $y = -\frac{a}{b}x + \frac{c}{b}$ nếu $b\ne 0$.
	\end{itemize}
\end{dinhly}
Ngoài ra, người ta cũng chứng minh được định lý sau:

\begin{dinhly}
	Trong mặt phẳng tọa độ $Oxy$, đường thẳng $d:ax + by = c$ chia mặt phẳng thành 2 nửa mặt phẳng. 1 trong 2 nửa mặt phẳng (không kể $d$) là \emph{miền nghiệm} của bất phương trình $ax + by < c$, nửa mặt phẳng còn lại (không kể $d$) là \emph{miền nghiệm} của bất phương trình $ax + by > c$.
\end{dinhly}
Đối với bất phương trình dạng $ax + by\le c$ hoặc $ax + by\ge c$ thì miền nghiệm là nửa mặt phẳng kể cả đường thẳng $d$.'' -- \cite[p. 22]{SGK_Toan_10_Canh_Dieu_tap_1}

\subsubsection{Biểu diễn miền nghiệm của bất phương trình bậc nhất 2 ẩn}
``Quy tắc thực hành biểu diễn miền nghiệm của bất phương trình bậc nhất 2 ẩn như sau:

\begin{tcolorbox}
	Các bước biểu diễn miền nghiệm của bất phương trình $ax + by < c$ trong mặt phẳng tọa độ $Oxy$:
	\begin{enumerate}
		\item Vẽ đường thẳng $d:ax + by = c$. Đường thẳng $d$ chia mặt phẳng tọa độ thành 2 nửa mặt phẳng.
		\item Lấy 1 điểm $M(x_0;y_0)$ không nằm trên $d$ (thường lấy gốc tọa độ $O$ nếu $c\ne 0$). Tính $ax_0 + by_0$ \& so sánh với $c$.
		\item Kết luận:
		\begin{itemize}
			\item Nếu $ax_0 + by_0 < c$ thì nửa mặt phẳng (không kể $d$) chứa điểm $M$ là miền nghiệm của bất phương trình $ax + by < c$.
			\item Nếu $ax_0 + by_0 > c$ thì nửa mặt phẳng (không kể $d$) không chứa điểm $M$ là miền nghiệm của bất phương trình $ax + by < c$.
		\end{itemize}
	\end{enumerate}
\end{tcolorbox}
Thông thường khi sử dụng phần mềm toán học để biểu diễn miền nghiệm của bất phương trình bậc nhất 2 ẩn, miền nghiệm của bất phương trình đó được tô màu.'' --\cite[pp. 23--24]{SGK_Toan_10_Canh_Dieu_tap_1}

\section{Hệ Bất Phương Trình Bậc Nhất 2 Ẩn}

\subsection{Hệ bất phương trình bậc nhất 2 ẩn}

\begin{dinhnghia}[Hệ bất phương trình bậc nhất 2 ẩn]
	\emph{Hệ bất phương trình bậc nhất 2 ẩn $x,y$} là 1 hệ gồm 1 số bất phương trình bậc nhất 2 ẩn $x,y$. Mỗi nghiệm chung của các bất phương trình trong hệ được gọi là 1 \emph{nghiệm} của hệ bất phương trình đó.
\end{dinhnghia}

\subsection{Biểu diễn miền nghiệm của hệ bất phương trình bậc nhất 2 ẩn}
``Cũng như bất phương trình bậc nhất 2 ẩn, ta có thể biểu diễn miền nghiệm của hệ bất phương trình bậc nhất 2 ẩn trên mặt phẳng tọa độ.'' --\cite[p. 26]{SGK_Toan_10_Canh_Dieu_tap_1}

\begin{dinhnghia}[Miền nghiệm của hệ bất phương trình bậc nhất 2 ẩn]
	\emph{Miền nghiệm} của hệ bất phương trình là giao các miền nghiệm của các bất phương trình trong hệ.
\end{dinhnghia}

\begin{tcolorbox}
	``Để biểu diễn miền nghiệm của hệ bất phương trình bậc nhất 2 ẩn, ta làm như sau:
	\begin{itemize}
		\item Trong cùng mặt phẳng tọa độ, biểu diễn miền nghiệm của mỗi bất phương trình trong hệ bằng cách gạch bỏ phần không thuộc miền nghiệm của nó.
		\item Phần không bị gạch là miền nghiệm cần tìm.'' --\cite[p. 27]{SGK_Toan_10_Canh_Dieu_tap_1}
	\end{itemize}
\end{tcolorbox}

%------------------------------------------------------------------------------%

\chapter{Hàm Số \& Đồ Thị}

\section{Hàm Số \& Đồ Thị}

\section{Hàm Số Bậc 2. Đồ Thị Hàm Số Bậc 2 \& Ứng Dụng}

\section{Dấu của Tam Thức Bậc 2}

\section{Bất Phương Trình Bậc 2 1 Ẩn}

\section{2 Dạng Phương Trình Quy về Phương Trình Bậc 2}

%------------------------------------------------------------------------------%

\chapter{Hệ Thức Lượng Trong Tam Giác. Vector}

\section{Giá Trị Lượng Giác của 1 Góc$\in[0^\circ;180^\circ]$. Định Lý Côsin \& Định Lý Sin Trong Tam Giác}

\section{Giải Tam Giác}

\section{Khái Niệm Vector}

\section{Tổng \& Hiệu của 2 Vector}

\section{Tích của 1 Số với 1 Vector}

\section{Tích Vô Hướng của 2 Vector}

%------------------------------------------------------------------------------%

\begin{thebibliography}{99}
	\bibitem[NQBH\texttt{/}elementary math]{NQBH/elementary math} Nguyễn Quản Bá Hồng. \href{https://github.com/NQBH/hobby/blob/master/elementary_mathematics/some_topics_in_elementary_mathematics_problems_theories_applications_bridges_to_advanced_mathematics/NQBH_some_topics_in_elementary_mathematics_problems_theories_applications_bridges_to_advanced_mathematics.pdf}{\textit{Some Topics in Elementary Mathematics: Problems, Theories, Applications, \& Bridges to Advanced Mathematics}}. Mar 2022--now.
\end{thebibliography}

%------------------------------------------------------------------------------%

\printbibliography[heading=bibintoc]

\end{document}