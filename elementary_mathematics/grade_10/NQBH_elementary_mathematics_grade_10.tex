\documentclass[oneside]{book}
\usepackage[backend=biber,natbib=true,style=authoryear]{biblatex}
\addbibresource{/home/hong/1_NQBH/reference/bib.bib}
\usepackage[utf8]{vietnam}
\usepackage{tocloft}
\renewcommand{\cftsecleader}{\cftdotfill{\cftdotsep}}
\usepackage[colorlinks=true,linkcolor=blue,urlcolor=red,citecolor=magenta]{hyperref}
\usepackage{amsmath,amssymb,amsthm,mathtools,float,graphicx,algpseudocode,algorithm,tcolorbox,tikz,tkz-tab,diagbox}
\DeclareMathOperator{\arccot}{arccot}
\usepackage[inline]{enumitem}
\allowdisplaybreaks
\numberwithin{equation}{section}
\newtheorem{assumption}{Assumption}[section]
\newtheorem{nhanxet}{Nhận xét}[section]
\newtheorem{conjecture}{Conjecture}[section]
\newtheorem{corollary}{Corollary}[section]
\newtheorem{hequa}{Hệ quả}[section]
\newtheorem{definition}{Definition}[section]
\newtheorem{dinhnghia}{Định nghĩa}[section]
\newtheorem{example}{Example}[section]
\newtheorem{vidu}{Ví dụ}[section]
\newtheorem{lemma}{Lemma}[section]
\newtheorem{notation}{Notation}[section]
\newtheorem{principle}{Principle}[section]
\newtheorem{problem}{Problem}[section]
\newtheorem{baitoan}{Bài toán}[section]
\newtheorem{proposition}{Proposition}[section]
\newtheorem{menhde}{Mệnh đề}[section]
\newtheorem{question}{Question}[section]
\newtheorem{cauhoi}{Câu hỏi}[section]
\newtheorem{remark}{Remark}[section]
\newtheorem{luuy}{Lưu ý}[section]
\newtheorem{theorem}{Theorem}[section]
\newtheorem{dinhly}{Định lý}[section]
\usepackage[left=0.5in,right=0.5in,top=1.5cm,bottom=1.5cm]{geometry}
\usepackage{fancyhdr}
\pagestyle{fancy}
\fancyhf{}
\lhead{\small \textsc{Sect.} ~\thesection}
\rhead{\small \nouppercase{\leftmark}}
\renewcommand{\sectionmark}[1]{\markboth{#1}{}}
\cfoot{\thepage}
\def\labelitemii{$\circ$}

\title{Some Topics in Elementary Mathematics\texttt{/}Grade 10}
\author{Nguyễn Quản Bá Hồng\footnote{Independent Researcher, Ben Tre City, Vietnam\\e-mail: \texttt{nguyenquanbahong@gmail.com}; website: \url{https://nqbh.github.io}.}}
\date{\today}

\begin{document}
\frontmatter
\maketitle
\setcounter{secnumdepth}{4}
\setcounter{tocdepth}{3}
\tableofcontents
\newpage

%------------------------------------------------------------------------------%

\mainmatter

\chapter*{Preface}

Tóm tắt kiến thức Toán lớp 10 theo chương trình giáo dục của Việt Nam \& một số chủ đề nâng cao.

%------------------------------------------------------------------------------%

\chapter{Mệnh Đề Toán học. Tập Hợp}

\begin{quotation}
	\textbf{Nội dung.} \textit{Mệnh đề toán học, tập hợp \& các phép toán trên tập hợp}.
\end{quotation}

\section{Mệnh Đề Toán Học}

\subsection{Mệnh đề toán học}

\begin{dinhnghia}[Mệnh đề toán học]
	1 mệnh đề khẳng định về 1 sự kiện trong toán học, gọi là \emph{mệnh đề toán học}.
\end{dinhnghia}
``Khi không sợ nhầm lẫn, ta thường gọi tắt mệnh đề toán học là \textit{mệnh đề}.'' -- \cite[p. 5]{SGK_Toan_10_Canh_Dieu_tap_1}

\begin{menhde}
	 Mỗi mệnh đề toán học phải hoặc đúng hoặc sai. 1 mệnh đề toán học không thể vừa đúng, vừa sai.
\end{menhde}

\begin{dinhnghia}[Mệnh đề đúng\texttt{/}sai]
	Khi mệnh đề toán học là đúng, ta gọi mệnh đề đó là 1 \textit{mệnh đề đúng}. Khi mệnh đề toán học là sai, ta gọi mệnh đề đó là 1 \textit{mệnh đề sai}.
\end{dinhnghia}

\subsection{Mệnh đề chứa biến}

\begin{dinhnghia}[Mệnh đề chứa biến]
	Với mỗi bộ giá trị cụ thể của bộ biến $(x_1,\ldots,x_n)$, $n\in\mathbb{N}^\star$, mệnh đề $P(x_1,\ldots,x_n)$ cho ta 1 mệnh đề toán học mà ta có thể khẳng định được tính đúng sai của mệnh đề đó. Khi đó, $P(x_1,\ldots,x_n)$ được gọi là \emph{mệnh đề chứa biến}.
\end{dinhnghia}
``Ta thường ký hiệu mệnh đề chứa biến $n$ là $P(n)$; mệnh đề chứa biến $x,y$ là $P(x,y)$; $\ldots$'' -- \cite[p. 6]{SGK_Toan_10_Canh_Dieu_tap_1}

\subsection{Phủ định của 1 mệnh đề}

\begin{dinhnghia}[Mệnh đề phủ định]
	Cho mệnh đề $P$. Mệnh đề ``Không phải $P$'' được gọi là \emph{mệnh đề phủ định} của mệnh đề $P$ \& ký hiệu là $\overline{P}$.
\end{dinhnghia}
``Mệnh đề $\overline{P}$ \textit{đúng} khi $P$ sai. Mệnh đề $\overline{P}$ \textit{sai} khi $P$ đúng.'' ``Để phủ định 1 mệnh đề, ta chỉ cần thêm\texttt{/}bớt từ ``không'' (hoặc ``không phải'') vào trước vị ngữ của mệnh đề đó.'' -- \cite[p. 7]{SGK_Toan_10_Canh_Dieu_tap_1}

\subsection{Mệnh đề kéo theo}

\begin{dinhnghia}[Mệnh đề kéo theo]
	Cho 2 mệnh đề $P$ \& $Q$. Mệnh đề ``Nếu $P$ thì $Q$'' được gọi là \emph{mệnh đề kéo theo} \& ký hiệu là $P\Rightarrow Q$. mệnh đề $P\Rightarrow Q$ sai khi $P$ đúng, $Q$ sai \& đúng trong các trường hợp còn lại.
\end{dinhnghia}
``Tùy theo nội dung cụ thể, đôi khi người ta còn phát biểu mệnh đề $P\Rightarrow Q$ là ``$P$ kéo theo $Q$'' hay ``$P$ suy ra $Q$'' hay ``Vì $P$ nên $Q$'' $\ldots$'' ``Các định lý toán học là những mệnh đề đúng \& thường phát biểu ở dạng mệnh đề kéo theo $P\Rightarrow Q$. Khi đó ta nói: $P$ là \textit{giả thiết}, $Q$ là \textit{kết luận} của định lý, hay $P$ là \textit{điều kiện đủ} để có $Q$, hoặc $Q$ là \textit{điều kiện cần} để có $P$.'' -- \cite[p. 7]{SGK_Toan_10_Canh_Dieu_tap_1}

\subsection{Mệnh đề đảo. 2 mệnh đề tương đương}

\begin{dinhnghia}[Mệnh đề đảo, 2 mệnh đề tương đương]
	Mệnh đề $Q\Rightarrow P$ được gọi là \emph{mệnh đề đảo} của mệnh đề $P\Rightarrow Q$. Nếu cả 2 mệnh đề $P\Rightarrow Q$ \& $Q\Rightarrow P$ đều đúng thì ta nói $P$ \& $Q$ là \emph{2 mệnh đề tương đương}, ký hiệu $P\Leftrightarrow Q$.
\end{dinhnghia}
``Mệnh đề $P\Leftrightarrow Q$ có thể phát biểu ở những dạng như sau: ``$P$ tương đương $Q$''; ``$P$ là điều kiện cần \& đủ để có $Q$''; ``$P$ khi \& chỉ khi $Q$''; ``$P$ nếu \& chỉ nếu $Q$''.'' -- \cite[p. 8]{SGK_Toan_10_Canh_Dieu_tap_1}

``Trong toán học, những câu khẳng định đúng phát biểu ở dạng ``$P\Leftrightarrow Q$'' cũng được coi là 1 mệnh đề toán học, gọi là \textit{mệnh đề tương đương}.'' -- \cite[p. 9]{SGK_Toan_10_Canh_Dieu_tap_1}

\subsection{Ký hiệu $\forall$ \& $\exists$}
$\forall$: ``với mọi'', $\exists$: ``tồn tại'' hoặc ``có 1'' (tồn tại 1) hoặc ``có ít nhất 1'' (tồn tại ít nhất 1). Phương pháp chứng minh 1 mệnh đề có ký hiệu ``$\forall$'', ``$\exists$'', là đúng hoặc sai.

\begin{menhde}
	Cho mệnh đề ``$P(x)$, $x\in X$''. Phủ định của mệnh đề $\forall x\in X$, $P(x)$'' là mệnh đề ``$\exists x\in X$, $\overline{P(x)}$''. Phủ định của mệnh đề $\exists x\in X$, $P(x)$'' là mệnh đề ``$\forall x\in X$, $\overline{P(x)}$''.
\end{menhde}

\section{Tập Hợp. Các Phép Toán Trên Tập Hợp}

\subsection{Tập hợp}
``Người ta còn minh họa tập hợp bằng 1 vòng kín, mỗi phần tử của tập hợp được biểu diễn bởi 1 chấm bên trong vòng kín, còn phần tử không thuộc tập hợp đó được biểu diễn bởi 1 chấm bên ngoài vòng kín. Cách minh họa tập hợp như vậy được gọi là biểu đồ Venn.'' -- \cite[p. 12]{SGK_Toan_10_Canh_Dieu_tap_1}

``Tập hợp không chứa phần tử nào được gọi là tập hợp rỗng, ký hiệu là $\emptyset$. 1 tập hợp có thể không có phần tử nào, cũng có thể có 1 phần tử, có nhiều phần tử, có vô số phần tử. Khi tập hợp $C$ là tập hợp rỗng, ta viết $C = \emptyset$ \& không được viết là $C = \{\emptyset\}$.'' -- \cite[p. 13]{SGK_Toan_10_Canh_Dieu_tap_1}

\subsection{Tập con \& tập hợp bằng nhau}

\subsubsection{Tập con}

\begin{dinhnghia}
	Nếu mọi phần tử của tập hợp $A$ đều là phần tử của tập hợp $B$ thì ta nói $A$ là 1 \emph{tập con} của tập hợp $B$ \& viết là $A\subset B$. Ta còn đọc là $A$ chứa trong $B$.
\end{dinhnghia}
``\textit{Quy ước:} Tập hợp rỗng được coi là tập con của mọi tập hợp.'' ``$A\subset B\Leftrightarrow(\forall x,\ x\in A\Rightarrow x\in B)$. Khi $A\subset B$, ta cũng viết $B\supset A$ (đọc là $B$ chứa $A$). Nếu $A$ không phải là tập con của $B$, ta viết $A\not\subset B$.'' -- \cite[p. 13]{SGK_Toan_10_Canh_Dieu_tap_1}

\begin{menhde}
	$A\subset A$ với mọi tập hợp $A$. Nếu $A\subset B$ \& $B\subset C$ thì $A\subset C$.
\end{menhde}
Tính chất $((A\subset B)\land(B\subset C))\Rightarrow(A\subset C)$ được gọi là \textit{tính chất bắc cầu}.

\subsubsection{Tập hợp bằng nhau}

\begin{dinhnghia}
	Khi $A\subset B$ \& $B\subset A$ thì ta nói 2 tập hợp $A$ \& $B$ \emph{bằng nhau}, viết là $A = B$.
\end{dinhnghia}

\subsection{Giao của 2 tập hợp}

\begin{dinhnghia}[Giao của 2 tập hợp]
	Tập hợp gồm tất cả các phần tử vừa thuộc $A$ vừa thuộc $B$ được gọi là \emph{giao} của $A$ \& $B$, ký hiệu $A\cap B$.
\end{dinhnghia}
``Vậy $A\cap B = \{x|x\in A\mbox{ \& } x\in B\}$.'' ``$x\in A\cap B$ khi \& chỉ khi $x\in A$ \& $x\in B$.'' -- \cite[p. 14]{SGK_Toan_10_Canh_Dieu_tap_1}, i.e., $(x\in A\cap B)\Leftrightarrow((x\in A)\land(x\in B))$.

\subsection{Hợp của 2 tập hợp}

\begin{dinhnghia}[Hợp của 2 tập hợp]
	Tập hợp gồm các phần tử thuộc $A$ hoặc thuộc $B$ được gọi là \emph{hợp} của $A$ \& $B$, ký hiệu $A\cup B$.
\end{dinhnghia}
``Vậy $A\cup B = \{x|x\in A\mbox{ hoặc } x\in B\}$.'' $x\in A\cup B$ khi \& chỉ khi $x\in A$ \& $x\in B$.'' -- \cite[p. 15]{SGK_Toan_10_Canh_Dieu_tap_1}, i.e., $(x\in A\cup B)\Leftrightarrow((x\in A)\lor(x\in B))$.

\begin{vidu}
	Với tập hợp $\mathbb{Q}$ các số hũu tỷ \& tập hợp $I$ các số vô tỷ. $\mathbb{Q}\cap I = \emptyset$, $\mathbb{Q}\cup I = \mathbb{R}$.
\end{vidu}

\subsection{Phần bù. Hiệu của 2 tập hợp}
``Tập hợp $\mathbb{Q}$ các số hữu tỷ là phần bù của tập hợp $I$ các số vô tỷ trong tập hợp  $\mathbb{R}$.'' -- \cite[p. 15]{SGK_Toan_10_Canh_Dieu_tap_1}

\begin{dinhnghia}[Phần bù]
	Cho tập hợp $A$ là tập con của tập hợp $B$. Tập hợp những phần tử $B$ mà không phải là phần tử của $A$ được gọi là \emph{phần bù} của $A$ trong $B$, ký hiệu $C_BA$.
\end{dinhnghia}
$B = A\cup C_BA$ \& $C_BA\cap A = \emptyset$, $\forall$ tập hợp $A,B$.

\begin{dinhnghia}[Hiệu của 2 tập hợp]
	Tập hợp gồm các phần tử thuộc $A$ nhưng không thuộc $B$ được gọi là \emph{hiệu} của $A$ \& $B$, ký hiệu $A\backslash B$.
\end{dinhnghia}
``Vậy $A\backslash B = \{x|x\in A\mbox{ \& } x\notin B\}$.'' ``$x\in A\backslash B$ khi \& chỉ khi $x\in A$ \& $x\notin B$.'' ``Nếu $B\subset A$ thì $A\backslash B = C_AB$.'' -- \cite[p. 16]{SGK_Toan_10_Canh_Dieu_tap_1}

\subsection{Các tập hợp số}

\subsubsection{Các tập hợp số đã học}
``Ta đã biết $\mathbb{N},\mathbb{Z},\mathbb{Q},\mathbb{R}$ lần lượt là tập hợp số tự nhiên, tập hợp số nguyên, tập hợp số hữu tỷ, tập hợp số thực. Ta có quan hệ sau: $\mathbb{N}\subset\mathbb{Z}\subset\mathbb{Q}\subset\mathbb{R}$.'' -- \cite[p. 17]{SGK_Toan_10_Canh_Dieu_tap_1}

\subsubsection{1 số tập con thường dùng của tập hợp số thực}
``$\mathbb{R}$: tập hợp số thực $(-\infty;+\infty)$. $\{x\in\mathbb{R}|a\le x\le b\}$: đoạn $[a;b]$. $\{x\in\mathbb{R}|a < x < b\}$: khoảng $(a;b)$. $\{x\in\mathbb{R}|x  > a\}$: khoảng $(a;+\infty)$. $\{x\in\mathbb{R}|x < b\}$: khoảng $(-\infty;b)$. $\{x\in\mathbb{R}|a\le x < b\}$: nửa khoảng $[a;b)$. $\{x\in\mathbb{R}|a < x\le b\}$: nửa khoảng $(a;b]$. $\{x\in\mathbb{R}|x\ge a\}$: nửa khoảng $[a;+\infty)$. $\{x\in\mathbb{R}|x\le b\}$: nửa khoảng $(-\infty;b]$. Ký hiệu $-\infty$ đọc là \textit{âm vô cực}, ký hiệu $+\infty$ đọc là \textit{dương vô cực}; $a$ \& $b$ được gọi là \textit{đầu mút} của các đoạn, khoảng, nửa khoảng. Ta cũng có thể biểu diễn tập hợp trên trục số bằng cách gạch bỏ phần không thuộc tập đó.'' -- \cite[p. 17]{SGK_Toan_10_Canh_Dieu_tap_1}

%------------------------------------------------------------------------------%

\chapter{Bất Phương Trình \& Hệ Bất Phương Trình Bậc Nhất 2 Ẩn}

\begin{quotation}
	\textbf{Nội dung.} \textit{Bất phương trình bậc nhất 2 ẩn; hệ bất phương trình bậc nhất 2 ẩn \& ứng dụng của chúng vào bài toán thực tiễn}.
\end{quotation}

\section{Bất Phương Trình Bậc Nhất 2 Ẩn}

\subsection{Bất phương trình bậc nhất 2 ẩn}

\begin{dinhnghia}[Bất phương trình bậc nhất 2 ẩn]
	\emph{Bất phương trình bậc nhất 2 ẩn $x,y$} là bất phương trình có 1 trong các dạng sau: $ax + by < c$, $ax + by > c$, $ax + by\le c$, $ax + by\ge c$, trong đó $a,b,c$ là những số cho trước với $a,b$ không đồng thời bằng $0$, $x$ \& $y$ là các ẩn.
	
	Cho bất phương trình bậc nhất 2 ẩn $ax + by < c$. Mỗi cặp số $(x_0;y_0)$ sao cho $ax_0 + by_0 < c\ (\star)$  được gọi là 1 \emph{nghiệm} của bất phương trình $(\star)$. Trong mặt phẳng tọa độ $Oxy$, tập hợp các điểm có tọa độ là nghiệm của bất phương trình $(\star)$ được gọi là \emph{miền nghiệm} của bất phương trình đó.
\end{dinhnghia}
``Nghiệm \& miền nghiệm của các bất phương trình dạng $ax + by > c$, $ax + by\le c$ \& $ax + by\ge c$ được định nghĩa tương tự.'' -- \cite[p. 21]{SGK_Toan_10_Canh_Dieu_tap_1}

\subsection{Biểu diễn miền nghiệm của bất phương trình bậc nhất 2 ẩn}

\subsubsection{Mô tả miền nghiệm của bất phương trình bậc nhất 2 ẩn}
``Người ta chứng minh được định lý sau: 

\begin{dinhly}
	Trong mặt phẳng tọa độ $Oxy$, phương trình $ax + by = c$ (với $a$ \& $b$ không đồng thời bằng $0$) xác định 1 đường thẳng $d$ như sau:
	\begin{itemize}
		\item $d$ có phương trình là $x = \frac{c}{a}$ nếu $b = 0$;
		\item $d$ có phương trình là $y = -\frac{a}{b}x + \frac{c}{b}$ nếu $b\ne 0$.
	\end{itemize}
\end{dinhly}
Ngoài ra, người ta cũng chứng minh được định lý sau:

\begin{dinhly}
	Trong mặt phẳng tọa độ $Oxy$, đường thẳng $d:ax + by = c$ chia mặt phẳng thành 2 nửa mặt phẳng. 1 trong 2 nửa mặt phẳng (không kể $d$) là \emph{miền nghiệm} của bất phương trình $ax + by < c$, nửa mặt phẳng còn lại (không kể $d$) là \emph{miền nghiệm} của bất phương trình $ax + by > c$.
\end{dinhly}
Đối với bất phương trình dạng $ax + by\le c$ hoặc $ax + by\ge c$ thì miền nghiệm là nửa mặt phẳng kể cả đường thẳng $d$.'' -- \cite[p. 22]{SGK_Toan_10_Canh_Dieu_tap_1}

\subsubsection{Biểu diễn miền nghiệm của bất phương trình bậc nhất 2 ẩn}
``Quy tắc thực hành biểu diễn miền nghiệm của bất phương trình bậc nhất 2 ẩn như sau:

\begin{tcolorbox}
	Các bước biểu diễn miền nghiệm của bất phương trình $ax + by < c$ trong mặt phẳng tọa độ $Oxy$:
	\begin{enumerate}
		\item Vẽ đường thẳng $d:ax + by = c$. Đường thẳng $d$ chia mặt phẳng tọa độ thành 2 nửa mặt phẳng.
		\item Lấy 1 điểm $M(x_0;y_0)$ không nằm trên $d$ (thường lấy gốc tọa độ $O$ nếu $c\ne 0$). Tính $ax_0 + by_0$ \& so sánh với $c$.
		\item Kết luận:
		\begin{itemize}
			\item Nếu $ax_0 + by_0 < c$ thì nửa mặt phẳng (không kể $d$) chứa điểm $M$ là miền nghiệm của bất phương trình $ax + by < c$.
			\item Nếu $ax_0 + by_0 > c$ thì nửa mặt phẳng (không kể $d$) không chứa điểm $M$ là miền nghiệm của bất phương trình $ax + by < c$.
		\end{itemize}
	\end{enumerate}
\end{tcolorbox}
Thông thường khi sử dụng phần mềm toán học để biểu diễn miền nghiệm của bất phương trình bậc nhất 2 ẩn, miền nghiệm của bất phương trình đó được tô màu.'' --\cite[pp. 23--24]{SGK_Toan_10_Canh_Dieu_tap_1}

\section{Hệ Bất Phương Trình Bậc Nhất 2 Ẩn}

\subsection{Hệ bất phương trình bậc nhất 2 ẩn}

\begin{dinhnghia}[Hệ bất phương trình bậc nhất 2 ẩn]
	\emph{Hệ bất phương trình bậc nhất 2 ẩn $x,y$} là 1 hệ gồm 1 số bất phương trình bậc nhất 2 ẩn $x,y$. Mỗi nghiệm chung của các bất phương trình trong hệ được gọi là 1 \emph{nghiệm} của hệ bất phương trình đó.
\end{dinhnghia}

\subsection{Biểu diễn miền nghiệm của hệ bất phương trình bậc nhất 2 ẩn}
``Cũng như bất phương trình bậc nhất 2 ẩn, ta có thể biểu diễn miền nghiệm của hệ bất phương trình bậc nhất 2 ẩn trên mặt phẳng tọa độ.'' --\cite[p. 26]{SGK_Toan_10_Canh_Dieu_tap_1}

\begin{dinhnghia}[Miền nghiệm của hệ bất phương trình bậc nhất 2 ẩn]
	\emph{Miền nghiệm} của hệ bất phương trình là giao các miền nghiệm của các bất phương trình trong hệ.
\end{dinhnghia}

\begin{tcolorbox}
	``Để biểu diễn miền nghiệm của hệ bất phương trình bậc nhất 2 ẩn, ta làm như sau:
	\begin{itemize}
		\item Trong cùng mặt phẳng tọa độ, biểu diễn miền nghiệm của mỗi bất phương trình trong hệ bằng cách gạch bỏ phần không thuộc miền nghiệm của nó.
		\item Phần không bị gạch là miền nghiệm cần tìm.'' -- \cite[p. 27]{SGK_Toan_10_Canh_Dieu_tap_1}
	\end{itemize}
\end{tcolorbox}

%------------------------------------------------------------------------------%

\chapter{Hàm Số \& Đồ Thị}

\begin{quotation}
	\textbf{Nội dung.} \textit{Hàm số \& đồ thị, hàm số bậc 2 \& ứng dụng, dấu của tam thức bậc 2, bất phương trình bậc 2 1 ẩn, cách giải 2 dạng phương trình vô tỷ}.
\end{quotation}

\section{Hàm Số \& Đồ Thị}
``Galileo Galilei (1564--1642), sinh tại thành phố Pisa (Italia), là nhà bác học vĩ đại của thời kỳ Phục Hưng. Ông được mệnh danh là ``cha đẻ của khoa học hiện đại''. Trước Galileo, người ta tin rằng vật nặng rơi nhanh hơn vật nhẹ, ông đã bác bỏ điều này bằng thí nghiệm nổi tiếng ở tháp nghiêng Pisa. Từ thí nghiệm của Galileo, các nhà khoa học sau này được truyền cảm hứng rằng chúng ta chỉ có thể rút ra tri thức khoa học từ các quy luật khách quan của tự nhiên, chứ không phải từ niềm tin.'' -- \cite[p. 31]{SGK_Toan_10_Canh_Dieu_tap_1}

\subsection{Hàm số}

\subsubsection{Định nghĩa}

\begin{dinhnghia}[Hàm số]
	Cho tập hợp khác rỗng $D\subset\mathbb{R}$. Nếu với mỗi giá trị của $x$ thuộc $D$ có 1 \& chỉ 1 giá trị tương ứng của $y$ thuộc tập hợp số thức $\mathbb{R}$ thì ta có 1 \emph{hàm số}. Ta gọi $x$ là \emph{biến số} \& $y$ là \emph{hàm số} của $x$. Tập hợp $D$ được gọi là \emph{tập xác định} của hàm số. Ký hiệu hàm số: $y = f(x)$, $x\in D$.
\end{dinhnghia}

\subsubsection{Cách cho hàm số}

\paragraph{Hàm số cho bằng 1 công thức.} ``Cùng với cách nói hàm số cho bằng công thức, ta cũng nói hàm số cho bằng biểu thức.'' -- \cite[p. 32]{SGK_Toan_10_Canh_Dieu_tap_1}

\begin{dinhnghia}[Tập xác định của hàm số]
	\emph{Tập xác định của hàm số} $y = f(x)$ là tập hợp tất cả các số thực $x$ sao cho biểu thức $f(x)$ có nghĩa.
\end{dinhnghia}

\paragraph{Hàm số cho bằng nhiều công thức.} ``1 hàm số có thể được cho bằng nhiều công thức.'' ``Cho hàm số $y = f(x)$ với tập xác định là $D$. Khi biến số $x$ thay đổi trong tập $D$ thì tập hợp các giá trị $y$ tương ứng được gọi là \textit{tập giá trị} của hàm số đã cho.'' -- \cite[p. 33]{SGK_Toan_10_Canh_Dieu_tap_1}

\paragraph{Hàm số không cho bằng công thức.} ``Trong thực tiễn, có những tình huống dẫn tới những hàm số không thể cho bằng công thức (hoặc nhiều công thức).'' -- \cite[p. 33]{SGK_Toan_10_Canh_Dieu_tap_1}

\subsection{Đồ thị của hàm số}
``Với mỗi giá trị của biến số $x$, ta có thể xác định được điểm $M(x;y)$ với $y = f(x)$ trong mặt phẳng tọa độ. Khi biến số $x$ thay đổi trên tập xác định, điểm $M(x;y)$ sẽ thay đổi theo trong mặt phẳng tọa độ $Oxy$ \& vạch nên 1 đường. Đường đó gọi là \textit{đồ thị của hàm số} $y = f(x)$.'' -- \cite[p. 34]{SGK_Toan_10_Canh_Dieu_tap_1}

\begin{dinhnghia}[Đồ thị của hàm số]
	\emph{Đồ thị của hàm số} $y = f(x)$ xác định trên tập hợp $D$ là tập hợp tất cả các điểm $M(x;f(x))$ trong mặt phẳng tọa độ $Oxy$ với mọi $x$ thuộc $D$.
\end{dinhnghia}
``Điểm $M(a;b)$ trong mặt phẳng tọa độ $\mathbb{R}^2$ \textit{thuộc} đồ thị hàm số $y = f(x)$, $x\in D$ khi \& chỉ khi
\begin{equation*}
	\left\{\begin{split}
		a&\in D,\\
		b &= f(a).
	\end{split}\right.
\end{equation*}
Để chứng tỏ điểm $M(a;b)$ trong mặt phẳng tọa độ \textit{không thuộc} đồ thị hàm số $y = f(x)$, $x\in D$, ta có thể kiểm tra 1 trong 2 khả năng sau:
\begin{itemize}
	\item \textit{Khả năng 1}: Chứng tỏ rằng $a\notin D$.
	\item \textit{Khả năng 2}: Khi $a\in D$ thì chứng tỏ rằng $b\ne f(a)$.'' -- \cite[pp. 34--35]{SGK_Toan_10_Canh_Dieu_tap_1}
\end{itemize}

\subsection{Sự biến  thiên của hàm số}

\subsubsection{Khái niệm}

\begin{dinhnghia}[Hàm số đồng biến\texttt{/}nghịch biến]
	Cho hàm số $y = f(x)$ xác định trên khoảng $(a;b)$. Hàm số $y = f(x)$ gọi là \emph{đồng biến} trên khoảng $(a;b)$ nếu
	\begin{align*}
		\forall x_1,x_2\in(a;b),\ x_1 < x_2\Rightarrow f(x_1) < f(x_2).
	\end{align*}
	Hàm số $y = f(x)$ gọi là \emph{nghịch biến} trên khoảng $(a;b)$ nếu
	\begin{align*}
		\forall x_1,x_2\in(a;b),\ x_1 < x_2\Rightarrow f(x_1) > f(x_2).
	\end{align*}
\end{dinhnghia}
``Xét sự biến thiên của 1 hàm số là tìm các khoảng hàm số đồng biến \& các khoảng hàm số nghịch biến. Kết quả xét sự biến thiên được tổng kết trong 1 \textit{bảng biến thiên}.'' -- \cite[p. 36]{SGK_Toan_10_Canh_Dieu_tap_1}

\subsubsection{Mô tả hàm số đồng biến, hàm số nghịch biến bằng đồ thịs}
``Hàm số đồng biến trên khoảng $(a;b)$ khi \& chỉ khi đồ thị hàm số ``đi lên'' trên khoảng đó. Hàm số nghịch biến trên khoảng $(a;b)$ khi \& chỉ khi đồ thị hàm số ``đi xuống'' trên khoảng đó. Khi nói đồ thị ``đi lên'' hay ``đi xuống'', ta luôn kể theo chiều tăng của biến số, i.e., kể từ trái qua phải.'' -- \cite[p. 37]{SGK_Toan_10_Canh_Dieu_tap_1}

\section{Hàm Số Bậc 2. Đồ Thị Hàm Số Bậc 2 \& Ứng Dụng}

\subsection{Hàm số bậc 2}

\begin{dinhnghia}[Hàm số bậc 2]
	\emph{Hàm số bậc 2} là hàm số được cho bằng biểu thức có dạng $y = ax^2 + bx + c$, trong đó $a,b,c$ là những hằng số \& $a\ne 0$. Tập xác định của hàm số là $\mathbb{R}$.
\end{dinhnghia}

\subsection{Đồ thị hàm số bậc 2}

\begin{dinhnghia}
	\emph{Đồ thị hàm số bậc 2 $y = ax^2 + bx + c$} ($a\ne 0$) là 1 đường parabol có đỉnh là điểm với tọa độ $\left(-\frac{b}{2a};-\frac{\Delta}{4a}\right)$ \& trục đối xứng là đường thẳng $x = -\frac{b}{2a}$.
\end{dinhnghia}
``Cho hàm số $f(x) = ax^2 + bx + c$ ($a\ne 0$), ta có: $-\frac{\Delta}{4a} = f\left(-\frac{b}{2a}\right)$. Để vẽ đồ thị hàm số $y = ax^2 + bx + c$ ($a\ne 0$), ta thực hiện các bước:
\begin{itemize}
	\item Xác định tọa độ đỉnh: $\left(-\frac{b}{2a};-\frac{\Delta}{4a}\right)$;
	\item Vẽ trục đối xứng $x = -\frac{b}{2a}$;
	\item Xác định 1 số điểm đặc biệt, e.g.: giao điểm với trục tung (có tọa độ $(0;c)$) \& trục hoành (nếu có), điểm đối xứng với điểm có tọa độ $(0;c)$ qua trục đối xứng $x = -\frac{b}{2a}$.
	\item Vẽ đường parabol đi qua các điểm đã xác định ta nhận được đồ thị hàm số $y = ax^2 + bx + c$.
\end{itemize}
Nếu $a > 0$ thì parabol có bề lõm quay lên trên, nếu $a < 0$ thì parabol có bề lõm quay xuống dưới.'' ``Cho hàm số bậc 2 $y = ax^2 + bx + c$ ($a\ne 0$). Nếu $a > 0$ thì hàm số nghịch biến trên khoảng $\left(-\infty;-\frac{b}{2a}\right)$; đồng biến trên khoảng $\left(-\frac{b}{2a};+\infty\right)$. Nếu $a < 0$ thì hàm số đồng biến trên khoảng $\left(-\infty;-\frac{b}{2a}\right)$; nghịch biến trên khoảng $\left(-\frac{b}{2a};+\infty\right)$. Ta có bảng biến thiên của hàm số bậc 2 như sau:

\begin{figure}[H]
	\centering
	\includegraphics[scale=0.2]{BBT_ham_so_bac_2}
	\caption{Bảng biến thiên của hàm số bậc 2 $y = ax^2 + bx + c$ ($a\ne 0$).}
\end{figure}
'' -- \cite[pp. 40--41]{SGK_Toan_10_Canh_Dieu_tap_1}

\subsection{Ứng dụng}
``Các hàm số bậc 2 có nhiều ứng dụng trong việc giải quyết những vấn đề thực tiễn.'' -- \cite[p. 42]{SGK_Toan_10_Canh_Dieu_tap_1}

\section{Dấu của Tam Thức Bậc 2}
``Đa thức $f(x) = ax^2 + bx + c$ ($a\ne 0$) còn gọi là \textit{tam thức bậc 2}.'' -- \cite[p. 44]{SGK_Toan_10_Canh_Dieu_tap_1}

\subsection{Dấu của tam thức bậc 2}
``Xét tam thức bậc 2 $f(x) = ax^2 + bx + c$ ($a\ne 0$). Ta đã biết: $ax^2 + bx + c > 0$ ứng với phần parabol $y = ax^2 + bx + c$ nằm phía trên trục hoành. $ax^2 + bx + c < 0$ ứng với phần parabol $y = ax^2 + bx + c$ nằm phía dưới trục hoành. Như vậy, ta có thể nhận ra dấu của tam thức bậc 2 $f(x) = ax^2 + bx + c$ là ``$+$'' (hoặc ``$-$'') thông qua việc nhận ra phần parabol $y = ax^2 + bx + c$ nằm phía trên (hoặc phía dưới) trục hoành.'' -- \cite[p. 44]{SGK_Toan_10_Canh_Dieu_tap_1}

\begin{luuy}[Dấu của $\Delta$]
	\begin{itemize}
		\item ``Nếu $\Delta < 0$ thì $f(x)$ cùng dấu với hệ số $a$ với mọi $x\in\mathbb{R}$.
		\item Nếu $\Delta = 0$ thì $f(x)$ cùng dấu với hệ số $a$ với mọi $x\in\mathbb{R}\backslash\left\{-\frac{b}{2a}\right\}$.
		\item Nếu $\Delta > 0$ thì $f(x)$ cùng dấu với hệ số $a$ với mọi $x$ thuộc các khoảng $(-\infty;x_1)$ \& $(x_2;+\infty)$; $f(x)$ trái dấu với hệ số $a$ với mọi $x$ thuộc khoảng $(x_1;x_2)$, trong đó $x_1,x_2$ là 2 nghiệm của $f(x)$ \& $x_1 < x_2$.'' -- \cite[pp. 44--45]{SGK_Toan_10_Canh_Dieu_tap_1}
	\end{itemize}
\end{luuy}
``Người ta đã chứng minh được định lý về dấu tam thức bậc 2 sau:

\begin{dinhly}
	Cho tam thức bậc 2 $f(x) = ax^2 + bx + c$ ($a\ne 0$), $\Delta = b^2 - 4ac$.
	\begin{itemize}
		\item Nếu $\Delta < 0$ thì $f(x)$ cùng dấu với hệ số $a$ với mọi $x\in\mathbb{R}$.
		\item Nếu $\Delta = 0$ thì $f(x)$ cùng dấu với hệ số $a$ với mọi $x\in\mathbb{R}\backslash\left\{-\frac{b}{2a}\right\}$.
		\item Nếu $\Delta > 0$ thì $f(x)$ có 2 nghiệm $x_1,x_2$ ($x_1 < x_2$). Khi đó: $f(x)$ cùng dấu với hệ số $a$ với mọi $x$ thuộc các khoảng $(-\infty;x_1)$ \& $(x_2;+\infty)$; $f(x)$ trái dấu với hệ số $a$ với mọi $x$ thuộc khoảng $(x_1;x_2)$.
	\end{itemize}
\end{dinhly}
Trong định lý, có thể thay biệt thức $\Delta = b^2 - 4ac$ bằng biệt thức thu gọn $\Delta' = (b')^2 - ac$ với $b = 2b'$.'' -- \cite[p. 46]{SGK_Toan_10_Canh_Dieu_tap_1}

\section{Bất Phương Trình Bậc 2 1 Ẩn}

\subsection{Bất phương trình bậc 2 1 ẩn}

\begin{dinhnghia}[Bất phương trình bậc 2 1 ẩn]
	\emph{Bất phương trình bậc 2 ẩn $x$} là bất phương trình có 1 trong các dạng sau: $ax^2 + bx + c < 0$, $ax^2 + bx + c\le 0$, $ax^2 + bx + c > 0$, $ax^2 + bx + c\ge 0$, trong đó $a,b,c$ là các số thực đã cho, $a\ne 0$. Đối với bất phương trình bậc 2 có dạng $ax^2 + bx + c < 0$, mỗi số $x_0\in\mathbb{R}$ sao cho $ax_0^2 + bx_0 + c < 0$ được gọi là 1 \emph{nghiệm} của bất phương trình đó. Tập hợp các nghiệm $x_0$ như thế còn được gọi là \emph{tập nghiệm} của bất phương trình bậc 2 đã cho. Nghiệm \& tập nghiệm của các dạng bất phương trình bậc 2 ẩn $x$ còn lại được định nghĩa tương tự.
\end{dinhnghia}
``Giải bất phương trình bậc 2 ẩn $x$ là đi tìm tập nghiệm của bất phương trình đó.'' -- \cite[p. 49]{SGK_Toan_10_Canh_Dieu_tap_1}

\subsection{Giải bất phương trình bậc 2 1 ẩn}

\subsubsection{Giải bất phương trình bậc 2 1 ẩn bằng cách xét dấu của tam thức bậc 2}
``Để giải bất phương trình bậc 2 (1 ẩn) có dạng $f(x) > 0$ ($f(x) = ax^2 + bx + c$), ta chuyển viêc giải bất phương trình đó về việc tìm tập hợp những giá trị của $x$ sao cho $f(x)$ mang dấu ``$+$''. Cụ thể, ta làm như sau:
\begin{enumerate}
	\item Xác định dấu của hệ số $a$ \& tìm nghiệm của $f(x)$ (nếu có).
	\item Sử dụng định lý về dấu của tam thức bậc 2 để tìm tập hợp những giá trị của $x$ sao cho $f(x)$ mang dấu ``$+$''.
\end{enumerate}
Các bất phương trình bậc 2 có dạng $f(x) < 0$, $f(x)\ge 0$, $f(x)\le 0$ được giải bằng cách tương tự.'' -- \cite[p. 50]{SGK_Toan_10_Canh_Dieu_tap_1}

\subsubsection{Giải bất phương trình bậc 2 1 ẩn bằng cách sử dụng đồ thị}
``\textit{Giải bất phương trình bậc 2 $ax^2 + bx + c > 0$} là tìm tập hợp những giá trị của $x$ ứng với phần parabol $y = ax^2 + bx + c$ nằm phía trên trục hoành. Tương tự, giải bất phương trình bậc 2 $ax^2 + bx + c < 0$ là tìm tập hợp những giá trị của $x$ ứng với phần parabol $y = ax^2 + bx + c$ nằm phía dưới trục hoành. Như vậy, để giải bất phương trình bậc 2 (1 ẩn) có dạng $f(x) > 0$ ($f(x) = ax^2 + bx + c$) bằng cách sử dụng đồ thị, ta có thể làm như sau: Dựa vào parabol $y = ax^2 + bx + c$, ta tìm tập hợp những giá trị của $x$ ứng với phần parabol đó nằm phía trên trục hoành. Đối với các bất phương trình bậc 2 có dạng $f(x) < 0$, $f(x)\ge 0$, $f(x)\le 0$, ta cũng làm tương tự.'' -- \cite[p. 51]{SGK_Toan_10_Canh_Dieu_tap_1}

\subsection{Ứng dụng của bất phương trình bậc 2 1 ẩn}
``Bất phương trình bậc 2 1 ẩn có nhiều ứng dụng, e.g.: giải 1 số hệ bất phương trình; ứng dụng vào tính toán lợi nhuận trong kinh doanh; tính toán điểm rơi trong pháo binh; $\ldots$'' -- \cite[p. 52]{SGK_Toan_10_Canh_Dieu_tap_1}

Xem \cite[p. 55]{SGK_Toan_10_Canh_Dieu_tap_1} bảng tổng kết các trường hợp có thể xảy ra khi giải bất phương trình bậc 2 $ax^2 + bx + c > 0$ (*) ($a\ne 0$). Đặt $f(x) = ax^2 + bx + c$.

\begin{figure}[H]
	\centering
	\includegraphics[scale=0.35]{bat_phuong_trinh_bac_2}
	\caption{Các trường hợp có thể xảy ra khi giải bất phương trình bậc 2 $y = ax^2 + bx + c$ ($a\ne 0$).}
\end{figure}
Bảng tổng kết các trường hợp có thể xảy ra khi giải các bất phương trình bậc 2 $ax^2 + bx + c < 0$, $ax^2 + bx + c\ge 0$, $ax^2 + bx + c\le 0$ ($a\ne 0$).

\section{2 Dạng Phương Trình Quy về Phương Trình Bậc 2}

\subsection{Giải phương trình có dạng $\sqrt{f(x)} = \sqrt{g(x)}$ (I) $f(x) = ax^2 + bx + c$ \& $g(x) = mx^2 + nx + p$ với $a\ne m$}
``Để giải phương trình (I), ta làm như sau:
\begin{enumerate}
	\item Bình phương 2 vế của (I) dẫn đến phương trình $f(x) = g(x)$ rồi tìm nghiệm của phương trình này.
	\item Thay từng nghiệm của phương trình $f(x) = g(x)$ vào bất phương trình $f(x)\ge 0$ (hoặc $g(x)\ge 0$). Nghiệm nào thỏa mãn bất phương trình đó thì giữ lại, nghiệm nào không thỏa mãn thì loại đi.
	\item Trên cơ sở những nghiệm giữ lại ở Bước 2, ta kết luận nghiệm của phương trình (I).
\end{enumerate}
Trong 2 bất phương trình $f(x)\ge 0$ \& $g(x)\ge 0$, ta thường chọn bất phương trình có dạng đơn giản hơn để thực hiện Bước 2. Người ta chứng minh được rằng tập hợp (số thực) giữ lại ở Bước 2 chính là tập nghiệm của phương trình (I).'' -- \cite[p. 56]{SGK_Toan_10_Canh_Dieu_tap_1}

\subsection{Giải phương trình có dạng $\sqrt{f(x)} = g(x)$ (II) $f(x) = ax^2 + bx + c$ \& $g(x) = dx + e$ với $a\ne d^2$}
``Để giải phương trình (II), ta làm như sau:
\begin{enumerate}
	\item Giải bất phương trình $g(x)\ge 0$ để tìm tập nghiệm của bất phương trình đó.
	\item Bình phương 2 vế của (II) dẫn đến phương trình $f(x) = [g(x)]^2$ rồi tìm tập nghiệm của phương trình đó.
	\item Trong những nghiệm của phương trình $f(x) = [g(x)]^2$, ta chỉ giữ lại những nghiệm thuộc tập nghiệm của bất phương trình $g(x)\ge 0$. Tập nghiệm giữ lại đó chính là tập nghiệm của phương trình (II).'' -- \cite[p. 57]{SGK_Toan_10_Canh_Dieu_tap_1}
\end{enumerate}

%------------------------------------------------------------------------------%

\chapter{Hệ Thức Lượng Trong Tam Giác. Vector}

\begin{quotation}
	\textbf{Nội dung.} \textit{Giá trị lượng giác của 1 góc $\in[0^\circ;180^\circ]$, định lý côsin \& định lý sin trong tam giác, giải tam giác; vector, tổng \& hiệu 2 vector, tích của 1 số với 1 vector, tích vô hướng của 2 vector; ứng dụng vào giải các bài toán thực tiễn}.
\end{quotation}

\section{Giá Trị Lượng Giác của 1 Góc $\in[0^\circ;180^\circ]$. Định Lý Côsin \& Định Lý Sin Trong Tam Giác}

\subsection{Giá trị lượng giác của 1 góc $\in[0^\circ;180^\circ]$}
``Cho $\Delta ABC$ vuông tại $A$ có góc $\widehat{ABC} = \alpha$:
\begin{align*}
	\sin\alpha &= \frac{AC}{BC},\ \cos\alpha = \frac{AB}{BC},\ \tan\alpha = \frac{AC}{AB},\ \cot\alpha = \frac{AB}{AC},\\
	\sin(90^\circ - \alpha) &= \cos\alpha,\ \cos(90^\circ - \alpha) = \sin\alpha,\ \tan(90^\circ - \alpha) = \cot\alpha,\ \cot(90^\circ - \alpha) = \tan\alpha.
\end{align*}
Trong mặt phẳng tọa độ $Oxy$, nửa đường tròn tâm $O$ nằm phía trên trục hoành bán kính $R = 1$ được gọi là \textit{nửa đường tròn đơn vị} (Fig. \ref{fig:nua duong tron don vi}). 

\begin{figure}[H]
	\centering
	\includegraphics[scale=0.2]{nua_duong_tron_don_vi_1}
	\caption{Nửa đường tròn đơn vị.}
	\label{fig:nua duong tron don vi}
\end{figure}
Với mỗi góc nhọn $\alpha$ ta có thể xác định 1 điểm $M$ duy nhất trên nửa đường tròn đơn vị sao cho $\widehat{xOM} = \alpha$. Giả sử điểm $M$ có tọa độ $(x_0,y_0)$. Xét $\Delta OMH$ vuông tại $H$, ta có:
\begin{align*}
	\sin\alpha = \frac{MH}{OM} = \frac{y_0}{1} = y_0,\ \cos\alpha = \frac{OH}{OM} = \frac{x_0}{1} = x_0,\ \tan\alpha = \frac{MH}{OH} = \frac{y_0}{x_0},\ \cot\alpha = \frac{OH}{MH} = \frac{x_0}{y_0}.
\end{align*}
Mở rộng khái niệm tỷ số lượng giác đối với góc nhọn cho những góc $\alpha\in[0^\circ,180^\circ]$, ta có định nghĩa sau đây: Với mỗi góc $\alpha$ ($0^\circ\le\alpha\le 180^\circ$), ta xác định 1 điểm $M(x_0,y_0)$ trên nửa đường tròn đơn vị sao cho $\widehat{xOM} = \alpha$ (Fig. \ref{fig:nua duong tron don vi 2}). Khi đó:

\begin{figure}[H]
	\centering
	\includegraphics[scale=0.2]{nua_duong_tron_don_vi_2}
	\caption{Nửa đường tròn đơn vị.}
	\label{fig:nua duong tron don vi 2}
\end{figure}
sin của góc $\alpha$, ký hiệu là $\sin\alpha$, được xác định bởi $\sin\alpha = y_0$, côsin của góc $\alpha$, ký hiệu là $\cos\alpha$, được xác định bởi $\cos\alpha = x_0$, tang của góc $\alpha$, ký hiệu là $\tan\alpha$, được xác định bởi $\tan\alpha = \frac{y_0}{x_0}$ ($x_0\ne 0$), côtang của góc $\alpha$, ký hiệu là $\cot\alpha$, được xác định bởi $\cot\alpha = \frac{x_0}{y_0}$ ($y_0\ne 0$). Các số $\sin\alpha,\cos\alpha,\tan\alpha,\cot\alpha$ được gọi là các \textit{giá trị lượng giác} của góc $\alpha$.'' -- \cite[pp. 63--64]{SGK_Toan_10_Canh_Dieu_tap_1}
\begin{align*}
	\tan\alpha &= \frac{\sin\alpha}{\cos\alpha},\ (\alpha\ne 90^\circ),\ \cot\alpha = \frac{\cos\alpha}{\sin\alpha}\ (0 < \alpha < 180^\circ),\\
	\sin(90^\circ - \alpha) &= \cos\alpha\ (0^\circ\le\alpha\le 90^\circ),\ \cos(90^\circ - \alpha) = \sin\alpha\ (0^\circ\le\alpha\le 90^\circ),\\
	\tan(90^\circ - \alpha) &= \cot\alpha\ (0^\circ < \alpha\le 90^\circ),\ \cot(90^\circ - \alpha) = \tan\alpha\ (0^\circ\le\alpha < 90^\circ).
\end{align*}

\begin{menhde}
	Với $0^\circ\le\alpha\le 180^\circ$ thì: $\sin(180^\circ - \alpha) = \sin\alpha$, $\cos(180^\circ - \alpha) = -\cos\alpha$, $\tan(180^\circ - \alpha) = -\tan\alpha$ ($\alpha\ne 90^\circ$), $\cot(180^\circ - \alpha) = -\cot\alpha$ ($\alpha\ne 0^\circ$, $\alpha\ne 180^\circ$).
\end{menhde}
Bảng giá trị lượng giác (GTLG) của 1 số góc đặc biệt:

\begin{table}[H]
	\centering
	\begin{tabular}{|c|c|c|c|c|c|c|c|c|c|}
		\hline
		\diagbox{GTLG}{$\alpha$}& $0^\circ$ & $30^\circ$ & $45^\circ$ & $60^\circ$ & $90^\circ$ & $120^\circ$ & $135^\circ$ & $150^\circ$ & $180^\circ$ \\
		\hline
		$\sin\alpha$ & 0 & $\frac{1}{2}$ & $\frac{\sqrt{2}}{2}$ & $\frac{\sqrt{3}}{2}$ & $1$ & $\frac{\sqrt{3}}{2}$ & $\frac{\sqrt{2}}{2}$ & $\frac{1}{2}$ & $0$ \\
		\hline
		$\cos\alpha$ & 1 & $\frac{\sqrt{3}}{2}$ & $\frac{\sqrt{2}}{2}$ & $\frac{1}{2}$ & 0 & $-\frac{1}{2}$ & $-\frac{\sqrt{2}}{2}$ & $-\frac{\sqrt{3}}{2}$ & $-1$ \\
		\hline
		$\tan\alpha$ & 0 & $\frac{\sqrt{3}}{2}$ & $1$ & $\sqrt{3}$ & $||$ & $-\sqrt{3}$ & $-1$ & $-\frac{\sqrt{3}}{3}$ & $0$ \\
		\hline
		$\cot\alpha$ & $||$ & $\sqrt{3}$ & $1$ & $\frac{\sqrt{3}}{3}$ & $0$ & $-\frac{\sqrt{3}}{3}$ & $-1$ & $-\sqrt{3}$ & $||$ \\
		\hline
	\end{tabular}
	\caption{Bảng giá trị lượng giác (GTLG) của 1 số góc đặc biệt.}
\end{table}
``Ta có thể tìm giá trị lượng giác (đúng hoặc gần đúng) của 1 góc (từ $0^\circ$ đến $180^\circ$) bằng cách sử dụng các phím: $\boxed{\sin},\boxed{\cos},\boxed{\tan}$ trên máy tính cầm tay.'' ``Ta có thể tìm số đo (đúng hoặc gần đúng) của 1 góc từ $0^\circ$ đến $180^\circ$ khi biết giá trị lượng giác của góc đó bằng cách sử dụng các phím: \fbox{SHIFT} cùng với $\boxed{\sin},\boxed{\cos},\boxed{\tan}$ trên máy tính cầm tay.'' ``Khi tìm góc $\alpha$ ($0^\circ\le\alpha\le 180^\circ$) nếu đã biết $\sin\alpha$, trên máy tính chỉ hiện lên kết quả góc $\alpha$ trong khoảng từ $0^\circ$ đến $90^\circ$. Giá trị còn lại cần tìm là $180^\circ - \alpha$.'' -- \cite[p. 66]{SGK_Toan_10_Canh_Dieu_tap_1}

\subsection{Định lý côsin -- cos theorem}

\begin{dinhly}[Định lý côsin]
	\label{thm: cos theorem}
	Cho $\Delta ABC$ có $BC = a$, $CA = b$, $AB = c$. Khi đó:
	\begin{equation*}
		\left\{\begin{split}
			a^2 &= b^2 + c^2 - 2bc\cos A,\\
			b^2 &= c^2 + a^2 - 2ca\cos B,\\
			c^2 &= a^2 + b^2 - 2ab\cos C,
		\end{split}\right.\ \ \left\{\begin{split}
			\cos A &= \frac{b^2 + c^2 - a^2}{2bc},\\
			\cos B &= \frac{c^2 + a^2 - b^2}{2ca},\\
			\cos C &= \frac{a^2 + b^2 - c^2}{2ab}.
	\end{split}\right.		
	\end{equation*}
\end{dinhly}
Nếu đặt $\alpha\coloneqq\widehat{A}$, $\beta = \widehat{B}$, $\gamma\coloneqq\widehat{C}$ (theo số đo độ hoặc số đo radian -- nhưng phải sử dụng chỉ 1 trong 2 loại này để các công thức tương thích\texttt{/}consistent với nhau), các công thức trong định lý côsin \ref{thm: cos theorem} có thể viết lại thành:
\begin{equation*}
	\left\{\begin{split}
		a^2 &= b^2 + c^2 - 2bc\cos\alpha,\\
		b^2 &= c^2 + a^2 - 2ca\cos\beta,\\
		c^2 &= a^2 + b^2 - 2ab\cos\gamma,
	\end{split}\right.\ \ \left\{\begin{split}
		\cos\alpha &= \frac{b^2 + c^2 - a^2}{2bc},\\
		\cos\beta &= \frac{c^2 + a^2 - b^2}{2ca},\\
		\cos\gamma &= \frac{a^2 + b^2 - c^2}{2ab}.
	\end{split}\right.		
\end{equation*}

\begin{luuy}
	\label{luu y:khoang gia tri 3 goc tam giac}
	Vì $1 + \cos A = \frac{(b + c)^2 - a^2}{2bc} > 0$ do $b + c > a > 0$ (i.e., \textit{bất đẳng thức tam giác}\emph{\texttt{/}}\textit{triangle inequality}: trong 1 tam giác, tổng độ dài 2 cạnh bất kỳ luôn lớn hơn độ dài cạnh còn lại), nên $\cos A > -1$, suy ra $\widehat{A}\ne 180^\circ$ (hay $\widehat{A}\ne\pi$ nếu sử dụng số đo radian thay vì số đo độ). Tương tự, vì $1 - \cos A = \frac{a^2 - (b - c)^2}{2bc} < 0$ do $a < |b - c|$ (suy ra trực tiếp từ bất đẳng thức tam giác vừa đề cập bằng cách xét dấu để bỏ dấu giá trị tuyệt đối), nên $\cos A < 1$, suy ra $\widehat{A}\ne 0^\circ$ (hay $\widehat{A}\ne 0$  nếu sử dụng số đo radian thay vì số đo độ). Cả 2 điều này hợp lý vì tổng 3 góc bất kỳ của 1 tam giác bằng $180^\circ$, i.e., với $\Delta ABC$, $\widehat{A} + \widehat{B} + \widehat{C} = 180^\circ$, mà $\widehat{A} > 0^\circ$, $\widehat{B} > 0^\circ$, $\widehat{C} > 0^\circ$, nên ta phải có $\widehat{A},\widehat{B},\widehat{C}\in(0^\circ;180^\circ)$ (hay $\widehat{A},\widehat{B},\widehat{C}\in(0;\pi)$ nếu sử dụng số đo radian thay vì số đo độ). Các lý luận này cho thấy sự hợp lý của các công thức trong định lý côsin \ref{thm: cos theorem}.
\end{luuy}

\subsection{Định lý sin -- sin theorem}

\begin{dinhly}[Định lý sin]
	\label{thm: sin theorem}
	Cho $\Delta ABC$ có $BC = a$, $CA = b$, $AB = c$ \& bán kính đường tròn ngoại tiếp là $R$. Khi đó:
	\begin{align*}
		\frac{a}{\sin A} = \frac{b}{\sin B} = \frac{c}{\sin C} = 2R,\ \ a = 2R\sin A,\ b = 2R\sin B,\ c = 2R\sin C.
	\end{align*}
\end{dinhly}

\begin{hequa}
	\label{corollary: sin theorem}
	Trong 1 tam giác, tỷ số của 2 cạnh bất kỳ bằng tỷ số sin của 2 góc tương ứng đối diện 2 cạnh đó. Cụ thể, với $\Delta ABC$ có $BC = a$, $CA = b$, $AB = c$:
	\begin{align*}
		 \frac{a}{b} = \frac{\sin A}{\sin B},\ \frac{b}{c} = \frac{\sin B}{\sin C},\ \frac{c}{a} = \frac{\sin C}{\sin A}.
	\end{align*}
\end{hequa}
Sử dụng ký hiệu $\alpha,\beta,\gamma$ vừa giới thiệu, 2 công thức trên được viết lại thành:
\begin{align*}
	\frac{a}{\sin\alpha} = \frac{b}{\sin\beta} = \frac{c}{\sin\gamma} = 2R,\ \ a = 2R\sin\alpha,\ b = 2R\sin\beta,\ c = 2R\sin\gamma,\ \frac{a}{b} = \frac{\sin\alpha}{\sin\beta},\ \frac{b}{c} = \frac{\sin\beta}{\sin\gamma},\ \frac{c}{a} = \frac{\sin\gamma}{\sin\alpha}.
\end{align*}

\section{Giải Tam Giác -- Solve Triangle}
``Từ xa xưa, con người đã cần đo đạc các khoảng cách mà không thể trực tiếp đo được. E.g., để đo khoảng cách từ vị trí $A$ trên bờ biển tới 1 hòn đảo (hay con tàu, $\ldots$) trên biển, người xưa đã tìm ra 1 cách đo khoảng cách đó như sau: Từ vị trí $A$, đo góc nghiêng $\alpha$ so với bờ biển tới 1 vị trí $C$ quan sát được trên đảo. Sau đó di chuyển dọc bờ biển đến vị trí $B$ cách $A$ 1 khoảng $d$ \& tiếp tục đo góc nghiêng $\beta$ so với bờ biển tới vị trí $C$ đã chọn. Bằng cách giải $\Delta ABC$, họ tính được khoảng cách $AC$.'' -- \cite[p. 72]{SGK_Toan_10_Canh_Dieu_tap_1}

\subsection{Tính các cạnh \& góc của tam giác dựa trên 1 số điều kiện cho trước}
``Như ta đã biết, 1 tam giác hoàn toàn xác định nếu biết 1 trong những dữ kiện sau:
\begin{enumerate*}
	\item Biết độ dài 2 cạnh \& độ lớn góc xen giữa 2 cạnh đó;
	\item Biết độ dài 3 cạnh;
	\item Biết độ dài 1 cạnh \& độ lớn 2 góc kề với cạnh đó.
\end{enumerate*}
\textit{Giải tam giác} là tính các cạnh \& các góc của tam giác dựa trên những dữ kiện cho trước.'' -- \cite[p. 72]{SGK_Toan_10_Canh_Dieu_tap_1}

\begin{baitoan}
	Giải tam giác khi biết 2 độ dài 2 cạnh \& độ lớn góc xen giữa 2 cạnh đó.
\end{baitoan}

\begin{proof}[Giải]
	Gọi tam giác đã cho là $\Delta ABC$. Giả sử $AB = c$, $AC = b$, $\widehat{A} = \alpha$ cho trước. Áp dụng định lý côsin \ref{thm: cos theorem}, ta được $a\coloneqq BC = \sqrt{b^2 + c^2 - 2bc\cos A} = \sqrt{b^2 + c^2 - 2bc\cos\alpha}$, $\cos B = \frac{c^2 + a^2 - b^2}{2ca}$, nên $\beta\coloneqq\widehat{B} = \arccos\frac{c^2 + a^2 - b^2}{2ca}\in(0^\circ,180^\circ)$, tương tự, $\cos C = \frac{a^2 + b^2 - c^2}{2ab}$, nên $\gamma\coloneqq\widehat{C} = \arccos\frac{a^2 + b^2 - c^2}{2ab}\in(0^\circ,180^\circ)$. Việc giải $\Delta ABC$ hoàn tất.
\end{proof}

\begin{luuy}[$\arccos$]
	``Dễ thấy rằng với mọi số $m\in\mathbb{R}$ cho trước mà $|m|\le 1$, phương trình $\cos x = m$ có đúng 1 nghiệm nằm trong đoạn $[0;\pi]$. Người ta thường ký hiệu nghiệm đó là $\arccos m$. Khi đó
	\begin{equation*}
		\cos x = m\Leftrightarrow\left[\begin{split}
			x &= \arccos m + k2\pi,\\
			x &= -\arccos m + k2\pi,
		\end{split}\right.
	\end{equation*}
	mà cũng thường được viết là $x = \pm\arccos m + k2\pi$.'' -- \cite[p. 24]{SGK_Toan_11_dai_so_giai_tich_nang_cao}. Sử dụng công thức này với $m = \frac{c^2 + a^2 - b^2}{2ca}$, suy ra nghiệm của phương trình $\cos x = \frac{c^2 + a^2 - b^2}{2ca}$ có nghiệm $x = \pm\arccos\frac{c^2 + a^2 - b^2}{2ca} + k2\pi$. Nhưng $\beta\coloneqq\widehat{B}\in(0^\circ,180^\circ)$ (i.e., $\widehat{B} = \beta\in(0;\pi)$ nếu sử dụng số đo radian thay vì số đo độ). Kết hợp 2 điều này, suy ra $\widehat{B} = \arccos\frac{c^2 + a^2 - b^2}{2ca}\in(0^\circ;180^\circ)$ như trong lời giải trên.
\end{luuy}
Ngoài ra, áp dụng tiếp định lý sin \ref{thm: sin theorem}, đẳng thức $\frac{a}{\sin\alpha} = \frac{b}{\sin\beta} = \frac{c}{\sin\gamma} = 2R$ có thể viết lại thành:
\begin{align*}
	\frac{\sqrt{b^2 + c^2 - 2bc\cos\alpha}}{\sin\alpha} = \frac{b}{\sin\arccos\frac{c^2 + a^2 - b^2}{2ca}} = \frac{c}{\sin\arccos\frac{a^2 + b^2 - c^2}{2ab}} = 2R.
\end{align*}
Thay tiếp $a = \sqrt{b^2 + c^2 - 2bc\cos\alpha}$ vào đẳng vừa thu được cho ta đẳng thức sau được biểu diễn chỉ bởi dữ kiện $(b,c,\alpha)$:
\begin{align*}
	\frac{\sqrt{b^2 + c^2 - 2bc\cos\alpha}}{\sin\alpha} = \frac{b}{\sin\arccos\frac{c - b\cos\alpha}{\sqrt{b^2 + c^2 - 2bc\cos\alpha}}} = \frac{c}{\sin\arccos\frac{b - c\cos\alpha}{\sqrt{b^2 + c^2 - 2bc\cos\alpha}}} = 2R,
\end{align*}
dùng để tính đường kính của hình tròn ngoại tiếp $\Delta ABC$, suy ra các công thức sau để tính bán kính $R$ của đường tròn ngoại tiếp $\Delta ABC$:
\begin{align*}
	R = \frac{\sqrt{b^2 + c^2 - 2bc\cos\alpha}}{2\sin\alpha} = \frac{b}{2\sin\arccos\frac{c - b\cos\alpha}{\sqrt{b^2 + c^2 - 2bc\cos\alpha}}} = \frac{c}{2\sin\arccos\frac{b - c\cos\alpha}{\sqrt{b^2 + c^2 - 2bc\cos\alpha}}}.
\end{align*}
Đẳng thức đầu tiên hiển nhiên có nhiều lợi thế hơn 2 đẳng thức còn lại để tính $R$ do không chứa hàm hợp $\sin\arccos(\cdot)$.

Vì vai trò của $a,b,c$ là bình đẳng, không mất tính tổng quát (without loss of generality, abbr., w.l.o.g.), 

\begin{baitoan}
	Giải tam giác khi biết độ dài 3 cạnh.
\end{baitoan}

\begin{proof}[Giải]
	Gọi tam giác đã cho là $\Delta ABC$. Giả sử $AB = c$, $BC = a$, $CA = b$ cho trước. Áp dụng định lý côsin \ref{thm: cos theorem}, ta được $\cos A = \frac{b^2 + c^2 - a^2}{2bc}$, $\cos B = \frac{c^2 + a^2 - b^2}{2ca}$, $\cos C = \frac{a^2 + b^2 - c^2}{2ab}$. Tương tự lời giải bài toán trước, suy ra $\alpha\coloneqq\widehat{A} = \arccos\frac{b^2 + c^2 - a^2}{2bc}$, $\beta\coloneqq\widehat{B} = \arccos\frac{c^2 + a^2 - b^2}{2ca}$, $\gamma\coloneqq\widehat{C} = \arccos\frac{a^2 + b^2 - c^2}{2ab}$. Theo Lưu ý \ref{luu y:khoang gia tri 3 goc tam giac}, $\alpha,\beta,\gamma\in(0^\circ;180^\circ)$ nếu sử dụng số đo độ hay $\alpha,\beta,\gamma\in(0;\pi)$ nếu sử dụng số đo radian. Việc giải $\Delta ABC$ hoàn tất.
\end{proof}
Tương tự, áp dụng tiếp định lý sin \ref{thm: sin theorem}, đẳng thức $\frac{a}{\sin\alpha} = \frac{b}{\sin\beta} = \frac{c}{\sin\gamma} = 2R$ có thể viết lại thành:
\begin{align*}
	\frac{a}{\sin\arccos\frac{b^2 + c^2 - a^2}{2bc}} = \frac{b}{\sin\arccos\frac{c^2 + a^2 - b^2}{2ca}} = \frac{c}{\sin\arccos\frac{a^2 + b^2 - c^2}{2ab}} = 2R.
\end{align*}

\begin{baitoan}
	Giải tam giác khi biết độ dài 1 cạnh \& độ lớn 2 góc kề với cạnh đó.
\end{baitoan}

\begin{proof}[Giải]
	Gọi tam giác đã cho là $\Delta ABC$. Giả sử $BC = a$, $\widehat{B} = \beta$, $\widehat{C} = \gamma$ cho trước. Ngay lập tức, ta có $\alpha\coloneqq\widehat{A} = 180^\circ - \widehat{B} - \widehat{C} = 180^\circ - \beta - \gamma$ (hay $\alpha = \pi - \beta - \gamma$ nếu sử dụng số đo radian). Áp dụng định lý sin \ref{thm: sin theorem} hoặc trực tiếp hơn là Hệ quả \ref{corollary: sin theorem}, ta có: $b\coloneqq AC = a\frac{\sin\beta}{\sin\alpha} = \frac{a\sin\beta}{\sin(180^\circ - \beta - \gamma)}$, $c\coloneqq AB = a\frac{\sin\gamma}{\sin\alpha} = \frac{a\sin\gamma}{\sin(180^\circ - \beta - \gamma)}$. Việc giải $\Delta ABC$ hoàn tất.
\end{proof}
Do lời giải của bài toán này đã áp dụng định lý sin \ref{thm: sin theorem}, ta áp dụng tiếp định lý côsin \ref{thm: cos theorem} để thu được:
\begin{align*}
	\frac{a}{\sin(180^\circ)} = 2R.
\end{align*}



\section{Khái Niệm Vector}

\section{Tổng \& Hiệu của 2 Vector}

\section{Tích của 1 Số với 1 Vector}

\section{Tích Vô Hướng của 2 Vector}

%------------------------------------------------------------------------------%

\begin{thebibliography}{99}
	\bibitem[NQBH\texttt{/}elementary math]{NQBH/elementary math} Nguyễn Quản Bá Hồng. \href{https://github.com/NQBH/hobby/blob/master/elementary_mathematics/some_topics_in_elementary_mathematics_problems_theories_applications_bridges_to_advanced_mathematics/NQBH_some_topics_in_elementary_mathematics_problems_theories_applications_bridges_to_advanced_mathematics.pdf}{\textit{Some Topics in Elementary Mathematics: Problems, Theories, Applications, \& Bridges to Advanced Mathematics}}. Mar 2022--now.
\end{thebibliography}

%------------------------------------------------------------------------------%

\printbibliography[heading=bibintoc]

\end{document}