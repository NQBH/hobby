\documentclass{article}
\usepackage[backend=biber,natbib=true,style=authoryear]{biblatex}
\addbibresource{/home/hong/1_NQBH/reference/bib.bib}
\usepackage[utf8]{vietnam}
\usepackage{tocloft}
\renewcommand{\cftsecleader}{\cftdotfill{\cftdotsep}}
\usepackage[colorlinks=true,linkcolor=blue,urlcolor=red,citecolor=magenta]{hyperref}
\usepackage{amsmath,amssymb,amsthm,mathtools,float,graphicx,algpseudocode,algorithm,tcolorbox}
\usepackage[inline]{enumitem}
\allowdisplaybreaks
\numberwithin{equation}{section}
\newtheorem{assumption}{Assumption}[section]
\newtheorem{conjecture}{Conjecture}[section]
\newtheorem{corollary}{Corollary}[section]
\newtheorem{hequa}{Hệ quả}[section]
\newtheorem{definition}{Definition}[section]
\newtheorem{dinhnghia}{Định nghĩa}[section]
\newtheorem{example}{Example}[section]
\newtheorem{vidu}{Ví dụ}[section]
\newtheorem{lemma}{Lemma}[section]
\newtheorem{notation}{Notation}[section]
\newtheorem{principle}{Principle}[section]
\newtheorem{problem}{Problem}[section]
\newtheorem{baitoan}{Bài toán}[section]
\newtheorem{proposition}{Proposition}[section]
\newtheorem{question}{Question}[section]
\newtheorem{cauhoi}{Câu hỏi}[section]
\newtheorem{remark}{Remark}[section]
\newtheorem{luuy}{Lưu ý}[section]
\newtheorem{theorem}{Theorem}[section]
\newtheorem{dinhly}{Định lý}[section]
\usepackage[left=0.5in,right=0.5in,top=1.5cm,bottom=1.5cm]{geometry}
\usepackage{fancyhdr}
\pagestyle{fancy}
\fancyhf{}
\lhead{\small Subsect.~\thesubsection}
\rhead{\small\nouppercase{\leftmark}}
\renewcommand{\subsectionmark}[1]{\markboth{#1}{}}
\cfoot{\thepage}
\def\labelitemii{$\circ$}

\title{Some Topics in Elementary Mathematics\texttt{/}Grade 10}
\author{Nguyễn Quản Bá Hồng\footnote{Independent Researcher, Ben Tre City, Vietnam\\e-mail: \texttt{nguyenquanbahong@gmail.com}; website: \url{https://nqbh.github.io}.}}
\date{\today}

\begin{document}
\maketitle
\begin{abstract}
	1 bộ sưu tập các bài toán chọn lọc từ cơ bản đến nâng cao cho Toán sơ cấp lớp 10. Tài liệu này là phần bài tập bổ sung cho tài liệu chính \href{https://github.com/NQBH/hobby/blob/master/elementary_mathematics/grade_10/NQBH_elementary_mathematics_grade_10.pdf}{GitHub\texttt{/}NQBH\texttt{/}hobby\texttt{/}elementary mathematics\texttt{/}grade 10\texttt{/}lecture}\footnote{\textsc{url}: \url{https://github.com/NQBH/hobby/blob/master/elementary_mathematics/grade_10/NQBH_elementary_mathematics_grade_10.pdf}.} của tác giả viết cho Toán lớp 10. Phiên bản mới nhất của tài liệu này được lưu trữ ở link sau: \href{https://github.com/NQBH/hobby/blob/master/elementary_mathematics/grade_10/problem/NQBH_elementary_mathematics_grade_10_problem.pdf}{GitHub\texttt{/}NQBH\texttt{/}hobby\texttt{/}elementary mathematics\texttt{/}grade 10\texttt{/}problem}\footnote{\textsc{url}: \url{https://github.com/NQBH/hobby/blob/master/elementary_mathematics/grade_10/problem/NQBH_elementary_mathematics_grade_10_problem.pdf}.}.
\end{abstract}
\tableofcontents
\newpage

%------------------------------------------------------------------------------%

\section{Mệnh Đề, Tập Hợp, Ánh Xạ}

\subsection{Mệnh Đề}

\begin{baitoan}[\cite{TL_chuyen_Toan_Dai_So_10}, Ví dụ 3, p. 8]
	Tìm phủ định của các mệnh đề sau:
	\begin{enumerate*}
		\item[(a)] Hôm nay là thứ 5.
		\item[(b)] $2007$ là số nguyên tố.
		\item[(c)] $\sqrt{2}$ là số hữu tỷ.
	\end{enumerate*}
\end{baitoan}

\begin{baitoan}[\cite{TL_chuyen_Toan_Dai_So_10}, Ví dụ 7, p. 11]
	Cho mệnh đề H: ``Nếu hôm nay trời mưa thì bể bơi Tây Hồ đóng cửa.'' Lập các mệnh đề đảo, mệnh đề phản, \& mệnh đề phản đảo của mệnh đề H.
\end{baitoan}

\begin{baitoan}[\cite{TL_chuyen_Toan_Dai_So_10}, Ví dụ 9, pp. 12--13]
	Chứng minh:
	\begin{enumerate*}
		\item[(a)] $P\Rightarrow Q = \overline{P}\lor Q$.
		\item[(b)] $P\Rightarrow Q = \overline{Q}\Rightarrow\overline{P}$, i.e., mệnh đề thuận \& mệnh đề phản đảo là tương đương.
		\item[(c)] $Q\Rightarrow P = \overline{P}\Rightarrow\overline{Q}$, i.e., mệnh đề đảo \& mệnh đề phản là tương đương.
	\end{enumerate*}
\end{baitoan}

\begin{baitoan}[\cite{TL_chuyen_Toan_Dai_So_10}, Ví dụ 10, p. 14]
	Chứng minh $\overline{P\Rightarrow Q} = P\land\overline{Q}$.
\end{baitoan}

\begin{baitoan}[\cite{TL_chuyen_Toan_Dai_So_10}, \textbf{1.}, p. 15]
	Cho P \& Q là các mệnh đề: P: ``Hôm nay ở Sapa nhiệt độ dưới $0^\circ$'', Q: ``Hôm nay ở Sapa có tuyết rơi''. Biểu diễn các mệnh đề sau theo P \& Q bằng các phép toán logic;
	\begin{enumerate*}
		\item[(a)] Hôm nay ở Sapa nhiệt độ dưới $0^\circ$ \& có tuyết rơi.
		\item[(b)] Hôm nay ở Sapa nhiệt độ dưới $0^\circ$ nhưng không có tuyết rơi.
		\item[(c)] Hôm nay ở Sapa nhiệt độ dưới $0^\circ$ hoặc có tuyết rơi.
		\item[(d)] Hôm nay ở Sapa nếu nhiệt độ dưới $0^\circ$ thì tuyết rơi.
		\item[(e)] Hôm nay ở Sapa có tuyết rơi khi \& chỉ khi nhiệt độ dưới $0^\circ$.
	\end{enumerate*}
\end{baitoan}

\begin{baitoan}[\cite{TL_chuyen_Toan_Dai_So_10}, \textbf{2.}, p. 15]
	Phát biểu mệnh đề đảo \& phản đảo của các mệnh đề sau đây:
	\begin{enumerate*}
		\item[(a)] Nếu ngày mai có tuyết rơi thì tôi sẽ đi trượt tuyết.
		\item[(b)] Nếu ngày mai có bài kiểm tra thì hôm nay tôi sẽ học đến khuya.
	\end{enumerate*}
\end{baitoan}

\begin{baitoan}[\cite{TL_chuyen_Toan_Dai_So_10}, \textbf{3.}, p. 15]
	Cho 2 mệnh đề P \& Q. Xét mệnh đề ``$P\Rightarrow\overline{Q}$''. Trong các mệnh đề sau đây, mệnh đề nào tương đương với mệnh đề trên?
	\begin{enumerate*}
		\item[(A)] $P\lor\overline{Q}$.
		\item[(B)] $\overline{Q}\Rightarrow P$.
		\item[(C)] $\overline{P}\Rightarrow Q$.
		\item[(D)] $Q\Rightarrow\overline{P}$.
	\end{enumerate*}
\end{baitoan}

\begin{baitoan}[\cite{TL_chuyen_Toan_Dai_So_10}, \textbf{4.}, p. 15]
	Chứng minh: $P = P\lor(P\land Q)$, $P = P\land(P\lor Q)$.
\end{baitoan}

\begin{baitoan}[\cite{TL_chuyen_Toan_Dai_So_10}, \textbf{5.}, p. 15]
	Chứng minh: mệnh đề $(P\Rightarrow Q)\Rightarrow R$ không tương đương với mệnh đề $P\Rightarrow(Q\Rightarrow R)$.
\end{baitoan}

\begin{baitoan}[\cite{TL_chuyen_Toan_Dai_So_10}, \textbf{6.}, p. 15]
	Chứng minh $\overline{P\Leftrightarrow Q} = \overline{P}\Leftrightarrow Q$.
\end{baitoan}

%------------------------------------------------------------------------------%

\subsection{Mệnh Đề Chứa Biến}

%------------------------------------------------------------------------------%

\subsection{Áp Dụng Mệnh Đề vào Suy Luận Toán Học}

%------------------------------------------------------------------------------%

\subsection{Tập Hợp}

%------------------------------------------------------------------------------%

\subsection{Ánh Xạ}

%------------------------------------------------------------------------------%

\subsection{Số Gần Đúng \& Sai Số}

%------------------------------------------------------------------------------%

\section{Hàm Số}

\subsection{Khái Niệm Hàm Số}

%------------------------------------------------------------------------------%

\subsection{Các Phép Toán Trên Hàm Số}

%------------------------------------------------------------------------------%

\subsection{Các Tính Chất Cơ Bản của Hàm Số}

%------------------------------------------------------------------------------%

\subsection{1 Số Hàm Số Cơ Bản}

%------------------------------------------------------------------------------%

\subsection{Các Phép Biến Đổi trên Đồ Thị Hàm Số}

%------------------------------------------------------------------------------%

\subsection{Đại Cương về Phương Trình Hàm}

%------------------------------------------------------------------------------%

\section{Bất Đẳng Thức}

\subsection{Số Thực}

%------------------------------------------------------------------------------%

\subsection{Khái Niệm Bất Đẳng Thức}

%------------------------------------------------------------------------------%

\subsection{Các Đại Lượng Trung Bình của 2 Số Không Âm}

%------------------------------------------------------------------------------%

\subsection{Bất Đẳng Thức giữa Trung Bình Cộng \& Trung Bình Nhân}

%------------------------------------------------------------------------------%

\subsection{Bất Đẳng Thức Cauchy}

%------------------------------------------------------------------------------%

\section{Phương Trình, Bất Phương Trình Đại Số}

\subsection{Đại Cương về Phương Trình, Bất Phương Trình}

%------------------------------------------------------------------------------%

\subsection{Phương Trình, Bất Phương Trình Bậc Nhất \& Bậc 2}

%------------------------------------------------------------------------------%

\subsection{1 Số Dạng Phương Trình, Bất Phương Trình Thường Gặp}

%------------------------------------------------------------------------------%

\subsection{Các Phương Pháp Đặc Biệt Giải Phương Trình}

%------------------------------------------------------------------------------%

\section{Hệ Phương Trình, Hệ Bất Phương Trình Đại Số}

%------------------------------------------------------------------------------%

\subsection{Đại Cương về Hệ Phương Trình, Hệ Bất Phương Trình}

%------------------------------------------------------------------------------%

\subsection{1 Số Dạng Hệ Phương Trình}

%------------------------------------------------------------------------------%

\subsection{1 Số Dạng Hệ Bất Phương Trình}

%------------------------------------------------------------------------------%

\printbibliography[heading=bibintoc]
	
\end{document}