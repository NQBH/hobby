\documentclass{article}
\usepackage[backend=biber,natbib=true,style=authoryear,maxbibnames=50]{biblatex}
\addbibresource{/home/nqbh/reference/bib.bib}
\usepackage[utf8]{vietnam}
\usepackage{tocloft}
\renewcommand{\cftsecleader}{\cftdotfill{\cftdotsep}}
\usepackage[colorlinks=true,linkcolor=blue,urlcolor=red,citecolor=magenta]{hyperref}
\usepackage{amsmath,amssymb,amsthm,mathtools,float,graphicx,algpseudocode,algorithm,tcolorbox}
\usepackage[inline]{enumitem}
\allowdisplaybreaks
\numberwithin{equation}{section}
\newtheorem{assumption}{Assumption}[section]
\newtheorem{baitoan}{Bài toán}
\newtheorem{cauhoi}{Câu hỏi}[section]
\newtheorem{conjecture}{Conjecture}[section]
\newtheorem{corollary}{Corollary}[section]
\newtheorem{dangtoan}{Dạng toán}[section]
\newtheorem{definition}{Definition}[section]
\newtheorem{dinhly}{Định lý}[section]
\newtheorem{dinhnghia}{Định nghĩa}[section]
\newtheorem{example}{Example}[section]
\newtheorem{ghichu}{Ghi chú}[section]
\newtheorem{hequa}{Hệ quả}[section]
\newtheorem{hypothesis}{Hypothesis}[section]
\newtheorem{lemma}{Lemma}[section]
\newtheorem{luuy}{Lưu ý}[section]
\newtheorem{nhanxet}{Nhận xét}[section]
\newtheorem{notation}{Notation}[section]
\newtheorem{note}{Note}[section]
\newtheorem{principle}{Principle}[section]
\newtheorem{problem}{Problem}[section]
\newtheorem{proposition}{Proposition}[section]
\newtheorem{question}{Question}[section]
\newtheorem{remark}{Remark}[section]
\newtheorem{theorem}{Theorem}[section]
\newtheorem{vidu}{Ví dụ}[section]
\usepackage[left=0.5in,right=0.5in,top=1.5cm,bottom=1.5cm]{geometry}
\usepackage{fancyhdr}
\pagestyle{fancy}
\fancyhf{}
\lhead{\small Sect.~\thesection}
\rhead{\small\nouppercase{\leftmark}}
\renewcommand{\subsectionmark}[1]{\markboth{#1}{}}
\cfoot{\thepage}
\def\labelitemii{$\circ$}

\title{3D Vector -- Vector Trong Không Gian}
\author{Nguyễn Quản Bá Hồng\footnote{Independent Researcher, Ben Tre City, Vietnam\\e-mail: \texttt{nguyenquanbahong@gmail.com}; website: \url{https://nqbh.github.io}.}}
\date{\today}

\begin{document}
\maketitle
\begin{abstract}
	\textsc{[en]} This text is a collection of problems, from easy to advanced, about 3D vector. This text is also a supplementary material for my lecture note on Elementary Mathematics grade 11, which is stored \& downloadable at the following link: \href{https://github.com/NQBH/hobby/blob/master/elementary_mathematics/grade_11/NQBH_elementary_mathematics_grade_11.pdf}{GitHub\texttt{/}NQBH\texttt{/}hobby\texttt{/}elementary mathematics\texttt{/}grade 11\texttt{/}lecture}\footnote{\textsc{url}: \url{https://github.com/NQBH/hobby/blob/master/elementary_mathematics/grade_11/NQBH_elementary_mathematics_grade_11.pdf}.}. The latest version of this text has been stored \& downloadable at the following link: \href{https://github.com/NQBH/hobby/blob/master/elementary_mathematics/grade_11/3D_vector/NQBH_3D_vector.pdf}{GitHub\texttt{/}NQBH\texttt{/}hobby\texttt{/}elementary mathematics\texttt{/}grade 11\texttt{/}3D vector}\footnote{\textsc{url}: \url{https://github.com/NQBH/hobby/blob/master/elementary_mathematics/grade_11/3D_vector/NQBH_3D_vector.pdf}.}.
	\vspace{2mm}
	
	\textsc{[vi]} Tài liệu này là 1 bộ sưu tập các bài tập chọn lọc từ cơ bản đến nâng cao về biểu thức đại số. Tài liệu này là phần bài tập bổ sung cho tài liệu chính -- bài giảng \href{https://github.com/NQBH/hobby/blob/master/elementary_mathematics/grade_11/NQBH_elementary_mathematics_grade_11.pdf}{GitHub\texttt{/}NQBH\texttt{/}hobby\texttt{/}elementary mathematics\texttt{/}grade 11\texttt{/}lecture} của tác giả viết cho Toán Sơ Cấp lớp 11. Phiên bản mới nhất của tài liệu này được lưu trữ \& có thể tải xuống ở link sau: \href{https://github.com/NQBH/hobby/blob/master/elementary_mathematics/grade_11/3D_vector/NQBH_3D_vector.pdf}{GitHub\texttt{/}NQBH\texttt{/}hobby\texttt{/}elementary mathematics\texttt{/}grade 11\texttt{/}3D vector}.
	
	\textsf{\textbf{Nội dung.} Vector trong không gian, 2 đường thẳng vuông góc trong không gian, đường thẳng vuông góc với mặt phẳng, 2 mặt phẳng vuông góc, khoảng cách trong không gian.}
\end{abstract}
\tableofcontents
\newpage

%------------------------------------------------------------------------------%

\section{Vector Trong Không Gian}

%------------------------------------------------------------------------------%

\section{2 Đường Thẳng Vuông Góc}

\begin{baitoan}[\cite{SGK_Toan_11_hinh_hoc_co_ban}, 1., p. 97]
	Cho hình lập phương $ABCD.EFGH$. Xác định góc giữa các cặp vector: (a) $\overrightarrow{AB},\overrightarrow{EG}$; (b)  $\overrightarrow{AF},\overrightarrow{EG}$; (c) $\overrightarrow{AB},\overrightarrow{DH}$.
\end{baitoan}

\begin{baitoan}[\cite{SGK_Toan_11_hinh_hoc_co_ban}, 2., p. 97]
	Cho tứ diện $ABCD$. (a) Chứng minh $\overrightarrow{AB}\cdot\overrightarrow{CD} + \overrightarrow{AC}\cdot\overrightarrow{DB} + \overrightarrow{AD}\cdot\overrightarrow{BC} = 0$. (b) Từ đẳng thức trên suy ra nếu tứ diện $ABCD$ có $AB\bot CD$ \& $AC\bot DB$ thì $AD\bot BC$.
\end{baitoan}

\begin{baitoan}[\cite{SGK_Toan_11_hinh_hoc_co_ban}, 3., p. 97]
	(a) Trong không gian nếu 2 đường thẳng $a,b$ vuông góc với đường thẳng $c$ thì $a,b$ có song song với nhau không? (b) Trong không gian nếu đường thẳng $a$ vuông góc với đường thẳng $b$ \& đường thẳng $b$ vuông góc với đường thẳng $c$ thì $a$ có vuông góc với $c$ không?
\end{baitoan}

\begin{baitoan}[\cite{SGK_Toan_11_hinh_hoc_co_ban}, 4., p. 98]
	Trong không gian cho 2 tam giác đều $ABC$ \& $ABC'$ có chung cạnh $AB$ \& nằm trong 2 mặt phẳng khác nhau. Gọi $M,N,P,Q$ lần lượt là trung điểm của các cạnh $AC,CB,BC',C'A$. Chứng minh: (a) $AB\bot CC'$; (b) Tứ giác $MNPQ$ là hình chữ nhật.
\end{baitoan}

\begin{baitoan}[\cite{SGK_Toan_11_hinh_hoc_co_ban}, 5., p. 98]
	Cho hình chóp tam giác $S.ABC$ có $SA = SB = SC$ \& $\widehat{ASB} = \widehat{BSC} = \widehat{CSA}$. Chứng minh $SA\bot BC$, $SB\bot AC$, $SC\bot AB$.
\end{baitoan}

\begin{baitoan}[\cite{SGK_Toan_11_hinh_hoc_co_ban}, 6., p. 98]
	Trong không gian cho 2 hình vuông $ABCD$ \& $ABC'D'$ có chung cạnh $AB$ \& nằm trong 2 mặt phẳng khác nhau, lần lượt có tâm $O$ \& $O'$. Chứng minh: $AB\bot OO'$ \& tứ giác $CDD'C'$ là hình chữ nhật.
\end{baitoan}

\begin{baitoan}[\cite{SGK_Toan_11_hinh_hoc_co_ban}, 7., p. 98]
	Cho $S$ là diện tích của $\Delta ABC$. Chứng minh: $S = \frac{1}{2}\sqrt{\overrightarrow{AB}^2\cdot\overrightarrow{AC}^2 - (\overrightarrow{AB}\cdot\overrightarrow{AC})^2}$.
\end{baitoan}

\begin{baitoan}[\cite{SGK_Toan_11_hinh_hoc_co_ban}, 8., p. 98]
	Cho tứ diện $ABCD$ có $AB = AC = AD$ \& $\widehat{BAC} = \widehat{BAD} = 60^\circ$. Chứng minh: (a) $AB\bot CD$; (b) Nếu $M,N$ lần lượt là trung điểm của $AB,CD$ thì $MN\bot AB$ \& $MN\bot CD$.
\end{baitoan}

%------------------------------------------------------------------------------%

\section{Đường Thẳng Vuông Góc với Mặt Phẳng}

\begin{baitoan}[\cite{SGK_Toan_11_hinh_hoc_co_ban}, 2., p. 104]
	Cho 2 đường thẳng phân biệt $a,b$ \& mặt phẳng $(\alpha)$. \emph{Đ\texttt{/}S?} (a)  Nếu $a\parallel(\alpha)$ \& $b\bot(\alpha)$ thì $a\bot b$. (b) Nếu $a\parallel(\alpha)$ \& $b\bot a$ thì $b\bot(\alpha)$. (c) Nếu $a\parallel(\alpha)$ \& $b\parallel(\alpha)$ thì $b\parallel a$. (d) Nếu $a\bot(\alpha)$ \& $b\bot a$ thì $b\parallel(\alpha)$.
\end{baitoan}

\begin{baitoan}[\cite{SGK_Toan_11_hinh_hoc_co_ban}, 2., p. 104]
	Cho tứ diện $ABCD$ có 2 mặt $ABC$ \& $BCD$ là 2 tam giác cân có chung cạnh đáy $BC$. Gọi $I$ là trung điểm của cạnh $BC$. (a) Chứng minh $BC\bot(ADI)$. (b) Gọi $AH$ là đường cao của $\Delta ADI$, chứng minh $AH\bot(BCD)$.
\end{baitoan}

\begin{baitoan}[\cite{SGK_Toan_11_hinh_hoc_co_ban}, 3., pp. 104--105]
	Cho hình chóp $S.ABCD$. có đáy là hình thoi $ABCD$ \& có $SA = SB = SC = SD$. Gọi $O$ là giao điểm của $AC,BD$. Chứng minh: (a) $SO\bot(ABCD)$; (b) $AC\bot(SBD)$ \& $BD\bot(SAC)$.
\end{baitoan}

\begin{baitoan}[\cite{SGK_Toan_11_hinh_hoc_co_ban}, 4., p. 105]
	Cho tứ diện $OABC$ có 3 cạnh $OA,OB,OC$ đôi một vuông góc. Gọi $H$ là chân đường vuông góc hạ từ $O$ tới mặt phẳng $(ABC)$. Chứng minh: (a) $H$ là trực tâm của $\Delta ABC$. (b) $\frac{1}{OH^2} = \frac{1}{OA^2} + \frac{1}{OB^2} + \frac{1}{OC^2}$. 
\end{baitoan}

\begin{baitoan}[\cite{SGK_Toan_11_hinh_hoc_co_ban}, 5., p. 105]
	Trên mặt phẳng $(\alpha)$ cho hình bình hành $ABCD$. Gọi $O$ là giao điểm của $AC$ \& $BD$, $S$ là 1 điểm nằm ngoài mặt phẳng $(\alpha)$ sao cho $SA = SC$, $SB = SD$. Chứng minh: (a) $SO\bot(\alpha)$; (b) Nếu trong mặt phẳng $(SAB)$ kẻ $SH$ vuông góc với $AB$ tại $H$ thì $AB$ vuông góc với mặt phẳng $(SOH)$.
\end{baitoan}

\begin{baitoan}[\cite{SGK_Toan_11_hinh_hoc_co_ban}, 6., p. 105]
	Cho hình chóp $S.ABCD$ có đáy là hình thoi $ABCD$ \& có cạnh $SA$ vuông góc với mặt phẳng $(ABCD$). Gọi $I,K$ là 2 điểm lần lượt lấy trên 2 cạnh $SB,SD$ sao cho $\frac{SI}{SB} = \frac{SK}{SD}$. Chứng minh: (a) $BD\bot SC$; (b) $IK\bot(SAC)$.
\end{baitoan}

\begin{baitoan}[\cite{SGK_Toan_11_hinh_hoc_co_ban}, 7., p. 105]
	Cho tứ diện $SABC$ có cạnh $SA$ vuông góc với mặt phẳng $(ABC)$ \& có $\Delta ABC$ vuông tại $B$. Trong mặt phẳng $(SAB)$ kẻ $AM$ vuông góc với $SB$ tại $M$. Trên cạnh $SC$ lấy điểm $N$ sao cho $\frac{SM}{SB} = \frac{SN}{SC}$. Chứng minh: (a) $BC\bot(SAB)$ \& $AM\bot(SBC)$; (b) $SB\bot AN$.
\end{baitoan}

\begin{baitoan}[\cite{SGK_Toan_11_hinh_hoc_co_ban}, 8., p. 105]
	Cho điểm $S$ không thuộc mặt phẳng $(\alpha)$ có hình chiếu trên $(\alpha)$ là điểm $H$. Với điểm $M$ bất kỳ trên $(\alpha)$ \& $M$ không trùng với $H$, gọi $SM$ là đường xiên \& đoạn $HM$ là hình chiếu của đường xiên đó. Chứng minh: (a) 2 đường xiên bằng nhau $\Leftrightarrow$ 2 hình chiếu của chúng bằng nhau. (b) Với 2 đường xiên cho trước, đường xiên nào lớn hơn thì có hình chiếu lớn hơn \& ngược lại đường xiên nào có hình chiếu lớn hơn thì lớn hơn.
\end{baitoan}

%------------------------------------------------------------------------------%

\section{2 Mặt Phẳng Vuông Góc}

%------------------------------------------------------------------------------%

\section{Khoảng Cách}

%------------------------------------------------------------------------------%

\printbibliography[heading=bibintoc]
	
\end{document}