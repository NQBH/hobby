\documentclass{article}
\usepackage[backend=biber,natbib=true,style=authoryear,maxbibnames=50]{biblatex}
\addbibresource{/home/nqbh/reference/bib.bib}
\usepackage[utf8]{vietnam}
\usepackage{tocloft}
\renewcommand{\cftsecleader}{\cftdotfill{\cftdotsep}}
\usepackage[colorlinks=true,linkcolor=blue,urlcolor=red,citecolor=magenta]{hyperref}
\usepackage{amsmath,amssymb,amsthm,mathtools,float,graphicx,algpseudocode,algorithm,tcolorbox}
\usepackage[inline]{enumitem}
\allowdisplaybreaks
\numberwithin{equation}{section}
\newtheorem{assumption}{Assumption}[section]
\newtheorem{baitoan}{Bài toán}
\newtheorem{cauhoi}{Câu hỏi}[section]
\newtheorem{conjecture}{Conjecture}[section]
\newtheorem{corollary}{Corollary}[section]
\newtheorem{dangtoan}{Dạng toán}[section]
\newtheorem{definition}{Definition}[section]
\newtheorem{dinhly}{Định lý}[section]
\newtheorem{dinhnghia}{Định nghĩa}[section]
\newtheorem{example}{Example}[section]
\newtheorem{ghichu}{Ghi chú}[section]
\newtheorem{hequa}{Hệ quả}[section]
\newtheorem{hypothesis}{Hypothesis}[section]
\newtheorem{lemma}{Lemma}[section]
\newtheorem{luuy}{Lưu ý}[section]
\newtheorem{nhanxet}{Nhận xét}[section]
\newtheorem{notation}{Notation}[section]
\newtheorem{note}{Note}[section]
\newtheorem{principle}{Principle}[section]
\newtheorem{problem}{Problem}[section]
\newtheorem{proposition}{Proposition}[section]
\newtheorem{question}{Question}[section]
\newtheorem{remark}{Remark}[section]
\newtheorem{theorem}{Theorem}[section]
\newtheorem{vidu}{Ví dụ}[section]
\usepackage[left=0.5in,right=0.5in,top=1.5cm,bottom=1.5cm]{geometry}
\usepackage{fancyhdr}
\pagestyle{fancy}
\fancyhf{}
\lhead{\small Sect.~\thesection}
\rhead{\small\nouppercase{\leftmark}}
\renewcommand{\subsectionmark}[1]{\markboth{#1}{}}
\cfoot{\thepage}
\def\labelitemii{$\circ$}

\title{3D Vector -- Vector Trong Không Gian}
\author{Nguyễn Quản Bá Hồng\footnote{Independent Researcher, Ben Tre City, Vietnam\\e-mail: \texttt{nguyenquanbahong@gmail.com}; website: \url{https://nqbh.github.io}.}}
\date{\today}

\begin{document}
\maketitle
\begin{abstract}
	\textsc{[en]} This text is a collection of problems, from easy to advanced, about 3D vector. This text is also a supplementary material for my lecture note on Elementary Mathematics grade 11, which is stored \& downloadable at the following link: \href{https://github.com/NQBH/hobby/blob/master/elementary_mathematics/grade_11/NQBH_elementary_mathematics_grade_11.pdf}{GitHub\texttt{/}NQBH\texttt{/}hobby\texttt{/}elementary mathematics\texttt{/}grade 11\texttt{/}lecture}\footnote{\textsc{url}: \url{https://github.com/NQBH/hobby/blob/master/elementary_mathematics/grade_11/NQBH_elementary_mathematics_grade_11.pdf}.}. The latest version of this text has been stored \& downloadable at the following link: \href{https://github.com/NQBH/hobby/blob/master/elementary_mathematics/grade_11/3D_vector/NQBH_3D_vector.pdf}{GitHub\texttt{/}NQBH\texttt{/}hobby\texttt{/}elementary mathematics\texttt{/}grade 11\texttt{/}3D vector}\footnote{\textsc{url}: \url{https://github.com/NQBH/hobby/blob/master/elementary_mathematics/grade_11/3D_vector/NQBH_3D_vector.pdf}.}.
	\vspace{2mm}
	
	\textsc{[vi]} Tài liệu này là 1 bộ sưu tập các bài tập chọn lọc từ cơ bản đến nâng cao về biểu thức đại số. Tài liệu này là phần bài tập bổ sung cho tài liệu chính -- bài giảng \href{https://github.com/NQBH/hobby/blob/master/elementary_mathematics/grade_11/NQBH_elementary_mathematics_grade_11.pdf}{GitHub\texttt{/}NQBH\texttt{/}hobby\texttt{/}elementary mathematics\texttt{/}grade 11\texttt{/}lecture} của tác giả viết cho Toán Sơ Cấp lớp 11. Phiên bản mới nhất của tài liệu này được lưu trữ \& có thể tải xuống ở link sau: \href{https://github.com/NQBH/hobby/blob/master/elementary_mathematics/grade_11/3D_vector/NQBH_3D_vector.pdf}{GitHub\texttt{/}NQBH\texttt{/}hobby\texttt{/}elementary mathematics\texttt{/}grade 11\texttt{/}3D vector}.
	
	\textsf{\textbf{Nội dung.} Vector trong không gian, 2 đường thẳng vuông góc trong không gian, đường thẳng vuông góc với mặt phẳng, 2 mặt phẳng vuông góc, khoảng cách trong không gian.}
\end{abstract}
\tableofcontents
\newpage

%------------------------------------------------------------------------------%

\section{Vector Trong Không Gian}

\begin{baitoan}[\cite{SGK_Toan_11_hinh_hoc_co_ban}, 1, p. 85]
	Cho tứ diện $ABCD$. Chỉ ra các vector có điểm đầu là $A$ \& điểm cuối là các đỉnh còn lại của hình tứ diện. Các vector đó có cùng nằm trong 1 mặt phẳng không?
\end{baitoan}

\begin{baitoan}[\cite{SGK_Toan_11_hinh_hoc_co_ban}, 2, p. 85]
	Cho hình hộp $ABCD.A'B'C'D'$. Kể tên các vector có điểm đầu \& điểm cuối là các đỉnh của hình hộp \& bằng $\overrightarrow{AB}$.
\end{baitoan}

\begin{baitoan}[\cite{SGK_Toan_11_hinh_hoc_co_ban}, Ví dụ 1, p. 86]
	Cho tứ diện $ABCD$. Chứng minh: $\overrightarrow{AC} + \overrightarrow{BD} = \overrightarrow{AD} + \overrightarrow{BC}$.
\end{baitoan}

\begin{baitoan}[\cite{SGK_Toan_11_hinh_hoc_co_ban}, 3, p. 86]
	Cho hình hộp $ABCD.EFGH$. Thực hiện các phép toán: (a) $\overrightarrow{AB} + \overrightarrow{CD} + \overrightarrow{EF} + \overrightarrow{GH}$; (b) $\overrightarrow{BE} - \overrightarrow{CH}$.
\end{baitoan}

\begin{baitoan}[\cite{SGK_Toan_11_hinh_hoc_co_ban}, Ví dụ 2, p. 87]
	Cho tứ diện $ABCD$. Gọi $M,N$ lần lượt là trung điểm của $AD,BC$, \& $G$ là trọng tâm của $\Delta BCD$. Chứng minh: (a) $\overrightarrow{MN} = \frac{1}{2}(\overrightarrow{AB} + \overrightarrow{DC})$; (b) $\overrightarrow{AB} + \overrightarrow{AC} + \overrightarrow{AD} = 3\overrightarrow{AG}$.
\end{baitoan}

\begin{baitoan}[\cite{SGK_Toan_11_hinh_hoc_co_ban}, 4, p. 87]
	Trong không gian cho 2 vector $\vec{a},\vec{b}$ đều khác vector không. Xác định các vector $\vec{m} = 2\vec{a}$, $\vec{n} = -3\vec{b}$, \& $\vec{p} = \vec{m} + \vec{n}$.
\end{baitoan}

\begin{baitoan}[\cite{SGK_Toan_11_hinh_hoc_co_ban}, Ví dụ 3, p. 88]
	Cho tứ diện $ABCD$. Gọi $M,N$ lần lượt là trung điểm của $AB,CD$. Chứng minh 3 vector $\overrightarrow{BC},\overrightarrow{AD},\overrightarrow{MN}$ đồng phẳng.
\end{baitoan}

\begin{baitoan}[\cite{SGK_Toan_11_hinh_hoc_co_ban}, 5, p. 89]
	Cho hình hộp $ABCD.EFGH$. Gọi $I,K$ lần lượt là trung điểm của $AB,BC$. Chứng minh các đường thẳng $IK,ED$ song song với mặt phẳng $(AFC)$. Từ đó suy ra 3 vector $\overrightarrow{AF},\overrightarrow{IK},\overrightarrow{ED}$ đồng phẳng.
\end{baitoan}

\begin{baitoan}[\cite{SGK_Toan_11_hinh_hoc_co_ban}, 6, p. 89]
	Cho 2 vector $\vec{a},\vec{b}$ đều khác vector $\vec{0}$. Xác định vector $\vec{c} = 2\vec{a} - \vec{b}$ \& giải thích tại sao 3 vector $\vec{a},\vec{b},\vec{c}$ đồng phẳng.
\end{baitoan}

\begin{baitoan}[\cite{SGK_Toan_11_hinh_hoc_co_ban}, 7, p. 89]
	Cho 3 vector $\vec{a},\vec{b},\vec{c}$ trong không gian. Chứng minh nếu $m\vec{a} + n\vec{b} + p\vec{c} = \vec{0}$ \& 1 số trong 3 số $m,n,p\in\mathbb{R}$ khác $0$ thì 3 vector $\vec{a},\vec{b},\vec{c}$ đồng phẳng.
\end{baitoan}

\begin{baitoan}[\cite{SGK_Toan_11_hinh_hoc_co_ban}, Ví dụ 4, p. 89]
	Cho tứ diện $ABCD$. Gọi $M,N$ lần lượt là trung điểm của $AB$ \& $CD$. Trên các cạnh $AD,BC$ lần lượt lấy các điểm $P,Q$ sao cho $\overrightarrow{AP} = \frac{2}{3}\overrightarrow{AD}$ \& $\overrightarrow{BQ} = \frac{2}{3}\overrightarrow{BC}$. Chứng minh 4 điểm $M,N,P,Q$ cùng thuộc 1 mặt phẳng.
\end{baitoan}

\begin{baitoan}[\cite{SGK_Toan_11_hinh_hoc_co_ban}, Ví dụ 5, p. 91]
	Cho hình hộp $ABCD.EFGH$ có $\overrightarrow{AB} = \vec{a}$, $\overrightarrow{AD} = \vec{b}$, $\overrightarrow{AE} = \vec{c}$. Gọi $I$ là trung điểm của $BG$. Biểu thị vector $\overrightarrow{AI}$ qua 3 vector $\vec{a},\vec{b},\vec{c}$.
\end{baitoan}

\begin{baitoan}[\cite{SGK_Toan_11_hinh_hoc_co_ban}, 1., p. 91]
	Cho hình lăng trụ tứ giác $ABCD.A'B'C'D'$. Mặt phẳng $(P0$ cắt các cạnh bên $AA',BB',CC',DD'$ lần lượt tại $I,K,L,M$. Xét các vector có các điểm đầu là các điểm $I,K,L,M$ \& có các điểm cuối là các đỉnh của hình lăng trụ. Chỉ ra các vector: (a) Cùng phương với $\overrightarrow{IA}$; (b) Cùng hướng với $\overrightarrow{IA}$; (c) Ngược hướng với $\overrightarrow{IA}$.
\end{baitoan}

\begin{baitoan}[\cite{SGK_Toan_11_hinh_hoc_co_ban}, 2., p. 91]
	Cho hình hộp $ABCD.A'B'C'D'$. Chứng minh: (a) $\overrightarrow{AB} + \overrightarrow{B'C'} + \overrightarrow{DD'} = \overrightarrow{AC'}$; (b) $\overrightarrow{BD} - \overrightarrow{D'D} - \overrightarrow{B'D'} = \overrightarrow{BB'}$; (c) $\overrightarrow{AC} + \overrightarrow{BA'} + \overrightarrow{DB} + \overrightarrow{C'D} = \vec{0}$.
\end{baitoan}

\begin{baitoan}[\cite{SGK_Toan_11_hinh_hoc_co_ban}, 3., p. 91]
	Cho hình bình hành $ABCD$. Gọi $S$ là 1 điểm nằm ngoài mặt phẳng chứa hình bình hành. Chứng minh: $\overrightarrow{SA} + \overrightarrow{SC} = \overrightarrow{SB} + \overrightarrow{SD}$
\end{baitoan}

\begin{baitoan}[\cite{SGK_Toan_11_hinh_hoc_co_ban}, 4., p. 92]
	Cho tứ diện $ABCD$. Gọi $M,N$ lần lượt là trung điểm của $AB,CD$. Chứng minh: (a) $\overrightarrow{MN} = \frac{1}{2}(\overrightarrow{AD} + \overrightarrow{BC})$; (b) $\overrightarrow{MN} = \frac{1}{2}(\overrightarrow{AC} + \overrightarrow{BD})$.
\end{baitoan}

\begin{baitoan}[\cite{SGK_Toan_11_hinh_hoc_co_ban}, 5., p. 92]
	Cho tứ diện $ABCD$. Xác định 2 điểm $E,F$ sao cho: (a) $\overrightarrow{AE} = \overrightarrow{AB} + \overrightarrow{AC} + \overrightarrow{AD}$; (b) $\overrightarrow{AF} = \overrightarrow{AB} + \overrightarrow{AC} - \overrightarrow{AD}$.
\end{baitoan}

\begin{baitoan}[\cite{SGK_Toan_11_hinh_hoc_co_ban}, 6., p. 92]
	Cho tứ diện $ABCD$. Gọi $G$ là trọng tâm của $\Delta ABC$. Chứng minh: $\overrightarrow{DA} + \overrightarrow{DB} + \overrightarrow{DC} = 3\overrightarrow{DG}$.
\end{baitoan}

\begin{baitoan}[\cite{SGK_Toan_11_hinh_hoc_co_ban}, 7., p. 92]
	Gọi $M,N$ lần lượt là trung điểm của $AC,BD$ của tứ diện $ABCD$. Gọi $I$ là trung điểm của $MN$ \& $P$ là 1 điểm bất kỳ trong không gian. Chứng minh: (a) $\overrightarrow{IA} + \overrightarrow{IB} + \overrightarrow{IC} + \overrightarrow{ID} = \vec{0}$; (b) $\overrightarrow{PI} = \frac{1}{4}(\overrightarrow{PA} + \overrightarrow{PB} + \overrightarrow{PC} + \overrightarrow{PD})$.
\end{baitoan}

\begin{baitoan}[\cite{SGK_Toan_11_hinh_hoc_co_ban}, 8., p. 92]
	Cho hình lăng trụ tam giác $ABC.A'B'C'$ có $\overrightarrow{AA'} = \vec{a}$, $\overrightarrow{AB} = \vec{b}$, $\overrightarrow{AC} = \vec{c}$. Phân tích (hay biểu thị) các vector $\overrightarrow{B'C},\overrightarrow{BC'}$ qua các vector $\vec{a},\vec{b},\vec{c}$.
\end{baitoan}

\begin{baitoan}[\cite{SGK_Toan_11_hinh_hoc_co_ban}, 9., p. 92]
	Cho $\Delta ABC$. Lấy điểm $S$ nằm ngoài mặt phẳng $(ABC)$. Trên đoạn $SA$ lấy điểm $M$ sao cho $\overrightarrow{MS} = -2\overrightarrow{MA}$ \& trên đoạn $BC$ lấy điểm $N$ sao cho $\overrightarrow{NB} = -\frac{1}{2}\overrightarrow{NC}$. Chứng minh 3 vector $\overrightarrow{AB},\overrightarrow{MN},\overrightarrow{SC}$ đồng phẳng.
\end{baitoan}

\begin{baitoan}[\cite{SGK_Toan_11_hinh_hoc_co_ban}, 10., p. 92]
	Cho hình hộp $ABCD.EFGH$. Gọi $K$ là giao điểm của $AH$ \& $DE$, $I$ là giao điểm của $BH$ \& $DF$. Chứng minh 3 vector $\overrightarrow{AC},\overrightarrow{KI},\overrightarrow{FG}$ đồng phẳng.
\end{baitoan}

%------------------------------------------------------------------------------%

\section{2 Đường Thẳng Vuông Góc}

\begin{baitoan}[\cite{SGK_Toan_11_hinh_hoc_co_ban}, 1, p. 93]
	Cho tứ diện đều $ABCD$ có $H$ là trung điểm của $AB$. Tính góc giữa các cặp vector: (a) $\overrightarrow{AB},\overrightarrow{BC}$; (c) $\overrightarrow{CH},\overrightarrow{AC}$.
\end{baitoan}

\begin{baitoan}[\cite{SGK_Toan_11_hinh_hoc_co_ban}, Ví dụ 1, p. 93]
	Cho tứ diện $OABC$ có các cạnh $OA,OB,OC$ đôi một vuông góc \& $OA = OB = OC = 1$. Gọi $M$ là trung điểm của $AB$. Tính góc giữa 2 vector $\overrightarrow{OM}$ \& $\overrightarrow{BC}$.
\end{baitoan}

\begin{baitoan}[\cite{SGK_Toan_11_hinh_hoc_co_ban}, 2, p. 94]
	Cho hình lập phương $ABCD.A'B'C'D'$. (a) Phân tích các vector $\overrightarrow{AC'},\overrightarrow{BD}$ theo 3 vector $\overrightarrow{AB},\overrightarrow{AD},\overrightarrow{AA'}$. (b) Tính $\cos(\overrightarrow{AC'},\overrightarrow{BD})$ \& từ đó suy ra $\overrightarrow{AC'},\overrightarrow{BD}$ vuông góc với nhau.
\end{baitoan}

\begin{baitoan}[\cite{SGK_Toan_11_hinh_hoc_co_ban}, 3, p. 95]
	Cho hình lập phương $ABCD.A'B'C'D'$. Tính góc giữa các cặp đường thẳng sau: (a) $AB,B'C'$' (b) $AC,B'C'$; (c) $A'C',B'C$.
\end{baitoan}

\begin{baitoan}[\cite{SGK_Toan_11_hinh_hoc_co_ban}, Ví dụ 2, p. 96]
	Cho hình chóp $S.ABC$ có $SA = SB = SC = AB = AC = a$ \& $BC = a\sqrt{2}$. Tính góc giữa 2 đường thẳng $AB,SC$.
\end{baitoan}

\begin{baitoan}[\cite{SGK_Toan_11_hinh_hoc_co_ban}, Ví dụ 3, p. 97]
	Cho tứ diện $ABCD$ có $AB\bot AC$ \& $AB\bot BD$. Gọi $P,Q$ lần lượt là trung điểm của $AB,CD$. Chứng minh $AB,PQ$ là 2 đường thẳng vuông góc với nhau.
\end{baitoan}

\begin{baitoan}[\cite{SGK_Toan_11_hinh_hoc_co_ban}, 4, p. 97]
	Cho hình lập phương $ABCD.A'B'C'D'$. Nêu tên các đường thẳng đi qua 2 đỉnh của hình lập phương đã cho \& vuông góc với: (a) đường thẳng $AB$; (b) đường thẳng $AC$.
\end{baitoan}

\begin{baitoan}[\cite{SGK_Toan_11_hinh_hoc_co_ban}, 5, p. 97]
	Tìm những hình ảnh trong thực tế minh họa cho sự vuông góc của 2 đường thẳng trong không gian (trường hợp cắt nhau \& trường hợp chéo nhau).
\end{baitoan}

\begin{baitoan}[\cite{SGK_Toan_11_hinh_hoc_co_ban}, 1., p. 97]
	Cho hình lập phương $ABCD.EFGH$. Xác định góc giữa các cặp vector: (a) $\overrightarrow{AB},\overrightarrow{EG}$; (b) $\overrightarrow{AF},\overrightarrow{EG}$; (c) $\overrightarrow{AB},\overrightarrow{DH}$.
\end{baitoan}

\begin{baitoan}[\cite{SGK_Toan_11_hinh_hoc_co_ban}, 2., p. 97]
	Cho tứ diện $ABCD$. (a) Chứng minh $\overrightarrow{AB}\cdot\overrightarrow{CD} + \overrightarrow{AC}\cdot\overrightarrow{DB} + \overrightarrow{AD}\cdot\overrightarrow{BC} = 0$. (b) Từ đẳng thức trên suy ra: Nếu tứ diện $ABCD$ có $AB\bot CD$ \& $AC\bot DB$ thì $AD\bot BC$.
\end{baitoan}

\begin{baitoan}[\cite{SGK_Toan_11_hinh_hoc_co_ban}, 3., p. 97]
	(a) Trong không gian nếu 2 đường thẳng $a,b$ cùng vuông góc với đường thẳng $c$ thì $a,b$ có song song với nhau không? (b) Trong không gian nếu đường thẳng $a$ vuông góc với đường thẳng $b$ \& đường thẳng $b$ vuông góc với đường thẳng $c$ thì $a$ có vuông góc với $c$ không?
\end{baitoan}

\begin{baitoan}[\cite{SGK_Toan_11_hinh_hoc_co_ban}, 4., p. 98]
	Trong không gian cho 2 tam giác đều $ABC,A'B'C'$ có chung cạnh $AB$ \& nằm trong 2 mặt phẳng khác nhau. Gọi $M,N,P,Q$ lần lượt là trung điểm của $AC,CB,BC',C'A$. Chứng minh: (a) $AB\bot CC'$; (b) Tứ giác $MNPQ$ là hình chữ nhật.
\end{baitoan}

\begin{baitoan}[\cite{SGK_Toan_11_hinh_hoc_co_ban}, 5., p. 98]
	Cho hình chóp tam giác $S.ABC$ có $SA = SB = SC$ \& có $\widehat{ASB} = \widehat{BSC} = \widehat{CSA}$. Chứng minh $SA\bot BC$, $SB\bot AC$, $SC\bot AB$.
\end{baitoan}

\begin{baitoan}[\cite{SGK_Toan_11_hinh_hoc_co_ban}, 6., p. 98]
	Trong không gian cho 2 hình vuông $ABCD,ABC'D'$ có chung cạnh $AB$ \& nằm trong 2 mặt phẳng khác nhau, lần lượt có tâm $O,O'$. Chứng minh $AB\bot OO'$ \& tứ giác $CDD'C'$ là hình chữ nhật.
\end{baitoan}

\begin{baitoan}[\cite{SGK_Toan_11_hinh_hoc_co_ban}, 7., p. 98]
	Cho $S$ là diện tích của $\Delta ABC$. Chứng minh: $S = \frac{1}{2}\sqrt{\overrightarrow{AB}^2\cdot\overrightarrow{AC}^2 - (\overrightarrow{AB}\cdot\overrightarrow{AC})^2}$.
\end{baitoan}

\begin{baitoan}[\cite{SGK_Toan_11_hinh_hoc_co_ban}, 8., p. 98]
	Cho tứ diện $ABCD$ có $AB = AC = AD$ \& $\widehat{BAC} = \widehat{BAD} = 60^\circ$. Chứng minh: (a) $AB\bot CD$; (b) Nếu $M,N$ lần lượt là trung điểm của $AB,CD$ thì $MN\bot AB$ \& $MN\bot CD$.
\end{baitoan}

%------------------------------------------------------------------------------%

\section{Đường Thẳng Vuông Góc với Mặt Phẳng}

\begin{baitoan}[\cite{SGK_Toan_11_hinh_hoc_co_ban}, 1, p. 100]
	Muốn chứng minh đường thẳng $d$ vuông góc với 1 mặt phẳng $(\alpha)$, người ta phải làm như thế nào?
\end{baitoan}

\begin{baitoan}[\cite{SGK_Toan_11_hinh_hoc_co_ban}, 2, p. 100]
	Cho 2 đường thẳng $a,b$ song song với nhau. 1 đường thẳng $d$ vuông góc với $a,b$. Khi đó đường thẳng $d$ có vuông góc với mặt phẳng xác định bởi 2 đường thẳng song song $a,b$ không?
\end{baitoan}

\begin{baitoan}[\cite{SGK_Toan_11_hinh_hoc_co_ban}, Ví dụ 1, p. 102]
	Cho hình chóp $S.ABC$ có đáy là $\Delta ABC$ vuông tại $B$ \& có cạnh $SA$ vuông góc với mặt phẳng $(ABC)$. (a) Chứng minh $BC\bot(SAB)$. (b) Gọi $AH$ là đường cao của $\Delta SAB$. Chứng minh $AH\bot SC$.
\end{baitoan}

\begin{baitoan}[\cite{SGK_Toan_11_hinh_hoc_co_ban}, Ví dụ 2, pp. 103--104]
	Cho hình chóp $S.ABCD$ có đáy là hình vuông $ABCD$ cạnh $a$, có cạnh $SA = a\sqrt{2}$ \& $SA$ vuông góc với mặt phẳng $(ABCD)$. (a) Gọi $M,N$ lần lượt là hình chiếu của điểm $A$ lên các đường thẳng $SB,SD$. Tính góc giữa đường thẳng $SC$ \& mặt phẳng $(AMN)$. (b) Tính góc giữa đường thẳng $SC$ \& mặt phẳng $(ABCD)$.
\end{baitoan}

\begin{baitoan}[\cite{SGK_Toan_11_hinh_hoc_co_ban}, 1., p. 104]
	Cho 2 đường thẳng phân biệt $a,b$ \& mặt phẳng $(\alpha)$. \emph{Đ\texttt{/}S?} (a)  Nếu $a\parallel(\alpha)$ \& $b\bot(\alpha)$ thì $a\bot b$. (b) Nếu $a\parallel(\alpha)$ \& $b\bot a$ thì $b\bot(\alpha)$. (c) Nếu $a\parallel(\alpha)$ \& $b\parallel(\alpha)$ thì $b\parallel a$. (d) Nếu $a\bot(\alpha)$ \& $b\bot a$ thì $b\parallel(\alpha)$.
\end{baitoan}

\begin{baitoan}[\cite{SGK_Toan_11_hinh_hoc_co_ban}, 2., p. 104]
	Cho tứ diện $ABCD$ có 2 mặt $ABC$ \& $BCD$ là 2 tam giác cân có chung cạnh đáy $BC$. Gọi $I$ là trung điểm của cạnh $BC$. (a) Chứng minh $BC\bot(ADI)$. (b) Gọi $AH$ là đường cao của $\Delta ADI$, chứng minh $AH\bot(BCD)$.
\end{baitoan}

\begin{baitoan}[\cite{SGK_Toan_11_hinh_hoc_co_ban}, 3., pp. 104--105]
	Cho hình chóp $S.ABCD$. có đáy là hình thoi $ABCD$ \& có $SA = SB = SC = SD$. Gọi $O$ là giao điểm của $AC,BD$. Chứng minh: (a) $SO\bot(ABCD)$; (b) $AC\bot(SBD)$ \& $BD\bot(SAC)$.
\end{baitoan}

\begin{baitoan}[\cite{SGK_Toan_11_hinh_hoc_co_ban}, 4., p. 105]
	Cho tứ diện $OABC$ có 3 cạnh $OA,OB,OC$ đôi một vuông góc. Gọi $H$ là chân đường vuông góc hạ từ $O$ tới mặt phẳng $(ABC)$. Chứng minh: (a) $H$ là trực tâm của $\Delta ABC$. (b) $\frac{1}{OH^2} = \frac{1}{OA^2} + \frac{1}{OB^2} + \frac{1}{OC^2}$. 
\end{baitoan}

\begin{baitoan}[\cite{SGK_Toan_11_hinh_hoc_co_ban}, 5., p. 105]
	Trên mặt phẳng $(\alpha)$ cho hình bình hành $ABCD$. Gọi $O$ là giao điểm của $AC$ \& $BD$, $S$ là 1 điểm nằm ngoài mặt phẳng $(\alpha)$ sao cho $SA = SC$, $SB = SD$. Chứng minh: (a) $SO\bot(\alpha)$; (b) Nếu trong mặt phẳng $(SAB)$ kẻ $SH$ vuông góc với $AB$ tại $H$ thì $AB$ vuông góc với mặt phẳng $(SOH)$.
\end{baitoan}

\begin{baitoan}[\cite{SGK_Toan_11_hinh_hoc_co_ban}, 6., p. 105]
	Cho hình chóp $S.ABCD$ có đáy là hình thoi $ABCD$ \& có cạnh $SA$ vuông góc với mặt phẳng $(ABCD$). Gọi $I,K$ là 2 điểm lần lượt lấy trên 2 cạnh $SB,SD$ sao cho $\frac{SI}{SB} = \frac{SK}{SD}$. Chứng minh: (a) $BD\bot SC$; (b) $IK\bot(SAC)$.
\end{baitoan}

\begin{baitoan}[\cite{SGK_Toan_11_hinh_hoc_co_ban}, 7., p. 105]
	Cho tứ diện $SABC$ có cạnh $SA$ vuông góc với mặt phẳng $(ABC)$ \& có $\Delta ABC$ vuông tại $B$. Trong mặt phẳng $(SAB)$ kẻ $AM$ vuông góc với $SB$ tại $M$. Trên cạnh $SC$ lấy điểm $N$ sao cho $\frac{SM}{SB} = \frac{SN}{SC}$. Chứng minh: (a) $BC\bot(SAB)$ \& $AM\bot(SBC)$; (b) $SB\bot AN$.
\end{baitoan}

\begin{baitoan}[\cite{SGK_Toan_11_hinh_hoc_co_ban}, 8., p. 105]
	Cho điểm $S$ không thuộc mặt phẳng $(\alpha)$ có hình chiếu trên $(\alpha)$ là điểm $H$. Với điểm $M$ bất kỳ trên $(\alpha)$ \& $M$ không trùng với $H$, gọi $SM$ là đường xiên \& đoạn $HM$ là hình chiếu của đường xiên đó. Chứng minh: (a) 2 đường xiên bằng nhau $\Leftrightarrow$ 2 hình chiếu của chúng bằng nhau. (b) Với 2 đường xiên cho trước, đường xiên nào lớn hơn thì có hình chiếu lớn hơn \& ngược lại đường xiên nào có hình chiếu lớn hơn thì lớn hơn.
\end{baitoan}

%------------------------------------------------------------------------------%

\section{2 Mặt Phẳng Vuông Góc}

\begin{baitoan}[\cite{SGK_Toan_11_hinh_hoc_co_ban}, Ví dụ, p. 107]
	Cho hình chóp $S.ABC$ có đáy là $\Delta ABC$ đều cạnh $a$, cạnh bên $SA$ vuông góc với mặt phẳng $(ABC)$ \& $SA = \frac{a}{2}$. (a) Tính góc giữa 2 mặt phẳng $(ABC)$ \& $(SBC)$. (b) Tính $S_{\Delta SBC}$.
\end{baitoan}

\begin{baitoan}[\cite{SGK_Toan_11_hinh_hoc_co_ban}, 1, p. 109]
	Cho 2 mặt phẳng $(\alpha),(\beta)$ vuông góc với nhau \& cắt nahu theo giao tuyến $d$. Chứng minh nếu có 1 đường thẳng $\Delta$ nằm trong $(\alpha)$ \& $\Delta$ vuông góc với $d$ thì $\Delta$ vuông góc với $(\beta)$.
\end{baitoan}

\begin{baitoan}[\cite{SGK_Toan_11_hinh_hoc_co_ban}, 2, p. 109]
	Cho tứ diện $ABCD$ có 3 cạnh $AB,AC,AD$ đôi một vuông góc với nhau. Chứng minh các mặt phẳng $(ABC),(ACD),(ADB)$ cũng đôi một vuông góc với nhau.
\end{baitoan}

\begin{baitoan}[\cite{SGK_Toan_11_hinh_hoc_co_ban}, 3, p. 109]
	Cho hình vuông $ABCD$. Dựng đoạn thẳng $AS$ vuông góc với mặt phẳng chứa hình vuông $ABCD$. (a) Nêu tên các mặt phẳng lần lượt chứa các đường thẳng $SB,SC,SD$ \& vuông góc với mặt phẳng $(ABCD)$. (b) Chứng minh mặt phẳng $(SAC)$ vuông góc với mặt phẳng $(SBD)$.
\end{baitoan}

\begin{baitoan}[\cite{SGK_Toan_11_hinh_hoc_co_ban}, 4, p. 111]
	\emph{Đ\texttt{/}S?} (a) Hình hộp là hình lăng trụ đứng. (b) Hình hộp chữ nhật là hình lăng trụ đứng. (c) Hình lăng trụ là hình hộp. (d) Có hình lăng trụ không phải là hình hộp.
\end{baitoan}

\begin{baitoan}[\cite{SGK_Toan_11_hinh_hoc_co_ban}, 5, p. 111]
	$6$ mặt của hình hộp chữ nhật có phải là những hình chữ nhật không?
\end{baitoan}

\begin{baitoan}[\cite{SGK_Toan_11_hinh_hoc_co_ban}, Ví dụ, p. 111]
	Cho hình lập phương $ABCD.A'B'C'D'$ có cạnh bằng $a$. Tính diện tích thiết diện của hình lập phương bị cắt bởi mặt phẳng trung trực $(\alpha)$ của đoạn $AC'$.
\end{baitoan}

\begin{baitoan}[\cite{SGK_Toan_11_hinh_hoc_co_ban}, 6, p. 112]
	Chứng minh hình chóp đều có các mặt bên là những tam giác cân bằng nhau.
\end{baitoan}

\begin{baitoan}[\cite{SGK_Toan_11_hinh_hoc_co_ban}, 7, p. 112]
	Có tồn tại 1 hình chóp tứ giác $S.ABCD$ có 2 mặt bên $(SAB),(SCD)$ cùng vuông góc với mặt phẳng đáy hay không?
\end{baitoan}

\begin{baitoan}[\cite{SGK_Toan_11_hinh_hoc_co_ban}, 1., p. 113]
	Cho 3 mặt phẳng $(\alpha),(\beta),(\gamma)$. \emph{Đ\texttt{/}S?} (a) Nếu $(\alpha)\bot(\beta)$ \& $(\alpha)\parallel(\gamma)$ thì $(\beta)\bot(\gamma)$; (b) Nếu $(\alpha)\bot(\beta)$ \& $(\alpha)\bot(\gamma)$ thì $(\beta)\parallel(\gamma)$.
\end{baitoan}

\begin{baitoan}[\cite{SGK_Toan_11_hinh_hoc_co_ban}, 2., p. 113]
	Cho 2 mặt phẳng $(\alpha),(\beta)$ vuông góc với nhau. Người ta lấy trên giao tuyến $\Delta$ của 2 mặt phẳng đó 2 điểm $A,B$ sao cho $AB = 8$\emph{cm}. Gọi $C$ là 1 điểm trên $(\alpha)$ \& $D$ là 1 điểm trên $(\beta)$ sao cho $AC,BD$ cùng vuông góc với giao tuyến $\Delta$ \& $AC = 6$\emph{cm}, $BD = 24$\emph{cm}. Tính độ dài đoạn $CD$.
\end{baitoan}

\begin{baitoan}[\cite{SGK_Toan_11_hinh_hoc_co_ban}, 3., pp. 113--114]
	Trong mặt phẳng $(\alpha)$ cho $\Delta ABC$ vuông ở $B$. 1 đoạn thẳng $AD$ vuông góc với $(\alpha)$ tại $A$. Chứng minh: (a) $\widehat{ABD}$ là góc giữa 2 mặt phẳng $(ABC),(DBC)$; (b) Mặt phẳng $(ABD)$ vuông góc với mặt phẳng $(BCD)$; (c) $HK\parallel BC$ với $H,K$ lần lượt là giao điểm của $DB,DC$ với mặt phẳng $(P)$ đi qua $A$ \& vuông góc với $DB$.
\end{baitoan}

\begin{baitoan}[\cite{SGK_Toan_11_hinh_hoc_co_ban}, 4., p. 114]
	Cho 2 mặt phẳng $(\alpha),(\beta)$ cắt nhau \& 1 điểm $M$ không thuộc $(\alpha)$ \& không thuộc $(\beta)$. Chứng minh qua điểm $M$ có 1 \& chỉ 1 mặt phẳng $(P)$ vuông góc với $(\alpha),(\beta)$. Nếu $(\alpha)$ song song với $(\beta)$ thì kết quả trên sẽ thay đổi như thế nào?
\end{baitoan}

\begin{baitoan}[\cite{SGK_Toan_11_hinh_hoc_co_ban}, 5., p. 114]
	Cho hình lập phương $ABCD.A'B'C'D'$. Chứng minh: (a) Mặt phẳng $(AB'C'D)$ vuông góc với mặt phẳng $(BCD'A')$; (b) Đường thẳng $AC'$ vuông góc với mặt phẳng $(A'BD)$.
\end{baitoan}

\begin{baitoan}[\cite{SGK_Toan_11_hinh_hoc_co_ban}, 6., p. 114]
	Cho hình chóp $S.ABCD$ có đáy $ABCD$ là 1 hình thoi cạnh $a$ \& có $SA = SB = SC = a$. Chứng minh: (a) Mặt phẳng $(ABCD)$ vuông góc với mặt phẳng $(SBD)$; (b) $\Delta SBD$ là tam giác vuông.
\end{baitoan}

\begin{baitoan}[\cite{SGK_Toan_11_hinh_hoc_co_ban}, 7., p. 114]
	Cho hình hộp chữ nhật $ABCD.A'B'C'D'$ có $AB = a$, $BC = b$, $CC' = c$. (a) Chứng minh mặt phẳng $(ADC'B')$ vuông góc với mặt phẳng $(ABB'A')$. (b) Tính độ dài đường chéo $AC'$ theo $a,b,c$.
\end{baitoan}

\begin{baitoan}[\cite{SGK_Toan_11_hinh_hoc_co_ban}, 8., p. 114]
	Tính độ dài đường chéo của 1 hình lập phương cạnh $a$.
\end{baitoan}

\begin{baitoan}[\cite{SGK_Toan_11_hinh_hoc_co_ban}, 9., p. 114]
	Cho hình chóp tam giác đều $S.ABC$ có $SH$ là đường cao. Chứng minh $SA\bot BC$, $SB\bot AC$.
\end{baitoan}

\begin{baitoan}[\cite{SGK_Toan_11_hinh_hoc_co_ban}, 10., p. 114]
	Cho hình chóp tứ giác đều $S.ABCD$ có các cạnh bên \& các cạnh đáy đều bằng $a$. Gọi $O$ là tâm của hình vuông $ABCD$. (a) Tính độ dài đoạn $SO$. (b) Gọi $M$ là trung điểm của đoạn $SC$. Chứng minh 2 mặt phẳng $(MBD)$ \& $(SAC)$ vuông góc với nhau. (c) Tính độ dài đoạn $OM$ \& tính góc giữa 2 mặt phẳng $(MBD),(ABCD)$.
\end{baitoan}

\begin{baitoan}[\cite{SGK_Toan_11_hinh_hoc_co_ban}, 11., p. 114]
	Cho hình chóp $S.ABCD$ có đáy $ABCD$ là 1 hình thoi tâm $I$ cạnh $a$ \& có $\widehat{A} = 60^\circ$, $SC = \frac{a\sqrt{6}}{2}$, \& $SC$ vuông góc với mặt phẳng $(ABCD)$. (a) Chứng minh mặt phẳng $(SBD)$ vuông góc với mặt phẳng $(SAC)$. (b) Trong $\Delta SCA$ kẻ $IK$ vuông góc với $SA$ tại $K$. Tính độ dài $IK$. (c) Chứng minh $\widehat{BKD} = 90^\circ$ \& từ đó suy ra mặt phẳng $(SAB)$ vuông góc với mặt phẳng $(SAD)$.
\end{baitoan}

%------------------------------------------------------------------------------%

\section{Khoảng Cách}

\begin{baitoan}[\cite{SGK_Toan_11_hinh_hoc_co_ban}, 1, p. 115]
	Cho điểm $O$ \& đường thẳng $a$. Chứng minh khoảng cách từ điểm $O$ đến đường thẳng $a$ là bé nhất so với các khoảng cách từ $O$ đến 1 điểm bất kỳ của đường thẳng $a$.
\end{baitoan}

\begin{baitoan}[\cite{SGK_Toan_11_hinh_hoc_co_ban}, 2, p. 115]
	Cho điểm $O$ \& mặt phẳng $(\alpha)$. Chứng minh khoảng cách từ điểm $O$ đến mặt phẳng $(\alpha)$ là bé nhất so với các khoảng cách từ $O$ tới 1 điểm bất kỳ của mặt phẳng $(\alpha)$.
\end{baitoan}

\begin{baitoan}[\cite{SGK_Toan_11_hinh_hoc_co_ban}, 3, p. 116]
	Cho đường thẳng $a$ song song với mặt phẳng $(\alpha)$. Chứng minh khoảng cách giữa $a$ \& $(\alpha)$ là bé nhất so với khoảng cách từ 1 điểm bất kỳ thuộc $a$ tới 1 điểm bất kỳ thuộc mặt phẳng $(\alpha)$.
\end{baitoan}

\begin{baitoan}[\cite{SGK_Toan_11_hinh_hoc_co_ban}, 4, p. 116]
	Cho 2 mặt phẳng $(\alpha),(\beta)$. Chứng minh khoảng cách giữa 2 mặt phẳng song song $(\alpha),(\beta)$ là nhỏ nhất trong các khoảng cách từ 1 điểm bất kỳ của mặt phẳng này tới 1 điểm bất kỳ của mặt phẳng kia.
\end{baitoan}

\begin{baitoan}[\cite{SGK_Toan_11_hinh_hoc_co_ban}, 5, p. 116]
	Cho tứ diện đều $ABCD$. Gọi $M,N$ lần lượt là trung điểm của $BC,AD$. Chứng minh $MN\bot BC$, $MN\bot AD$.
\end{baitoan}

\begin{baitoan}[\cite{SGK_Toan_11_hinh_hoc_co_ban}, 6, p. 118]
	Chứng minh khoảng cách giữa 2 đường thẳng chéo nhau là bé nhất so với khoảng cách giữa 2 điểm bất kỳ lần lượt nằm trên 2 đường thẳng ấy.
\end{baitoan}

\begin{baitoan}[\cite{SGK_Toan_11_hinh_hoc_co_ban}, Ví dụ, p. 118]
	Cho hình chóp $S.ABCD$ có đáy là hình vuông $ABCD$ cạnh $a$, cạnh $SA$ vuông góc với mặt phẳng $(ABCD)$ \& $SA = a$. Tính khoảng cách giữa 2 đường thẳng chéo nhau $SC,BD$.
\end{baitoan}

\begin{baitoan}[\cite{SGK_Toan_11_hinh_hoc_co_ban}, 1., p. 119]
	\emph{Đ\texttt{/}S?} (a) Đường thẳng $\Delta$ là đường vuông góc chung của 2 đường thẳng $a,b$ nếu $\Delta$ vuông góc với $a$ \& $\Delta$ vuông góc với $b$. (b) Gọi $(P)$ là mặt phẳng song song với cả 2 đường thẳng $a,b$ chéo nhau. Khi đó đường thẳng vuông góc chung $\Delta$ của $a,b$ luôn luôn vuông góc với $(P)$. (c) Gọi $\Delta$ là đường vuông góc chung của 2 đường thẳng chéo nhau $a,b$ thì $\Delta$ là giao tuyến của 2 mặt phẳng $(a,\Delta)$ \& $(b,\Delta)$. (d) Cho 2 đường thẳng chéo nhau $a,b$. Đường thẳng nào đi qua 1 điểm $M$ trên $a$ đồng thời cắt $b$ tại $N$ \& vuông góc với $b$ thì đó là đường vuông góc chung của $a,b$. (e) Đường vuông góc chung $\Delta$ của 2 đường thẳng chéo nhau $a,b$ nằm trong mặt phẳng chứa đường này \& vuông góc với đường kia.
\end{baitoan}

\begin{baitoan}[\cite{SGK_Toan_11_hinh_hoc_co_ban}, 2., p. 119]
	Cho tứ diện $S.ABC$ có $SA$ vuông góc với mặt phẳng $(ABC)$. Gọi $H,K$ lần lượt là trực tâm của $\Delta ABC,\Delta SBC$. (a) Chứng minh 3 đường thẳng $AH,SK,BC$ đồng quy. (b) Chứng minh $SC$ vuông góc với mặt phẳng $(BHK)$ \& $HK$ vuông góc với mặt phẳng $(SBC)$. (c) Xác định đường vuông góc chung của $BC,SA$.
\end{baitoan}

\begin{baitoan}[\cite{SGK_Toan_11_hinh_hoc_co_ban}, 3., p. 119]
	Cho hình lập phương $ABCD.A'B'C'D'$ cạnh $a$. Chứng minh các khoảng cách từ các điểm $B,C,D,A',B',D'$ đến đường chéo $AC'$ đều bằng nhau. Tính khoảng cách đó.
\end{baitoan}

\begin{baitoan}[\cite{SGK_Toan_11_hinh_hoc_co_ban}, 4., p. 119]
	Cho hình hộp chữ nhật $ABCD.A'B'C'D'$ có $AB = a$, $BC = b$, $CC' = c$. (a) Tính khoảng cách từ $B$ đến mặt phẳng $(ACC'A')$. (b) Tính khoảng cách giữa 2 đường thẳng $BB',AC'$.
\end{baitoan}

\begin{baitoan}[\cite{SGK_Toan_11_hinh_hoc_co_ban}, 5., p. 119]
	Cho hình lập phương $ABCD.A'B'C'D'$ cạnh $a$. (a) Chứng minh $B'D$ vuông góc với mặt phẳng $(BA'C')$. (b) Tính khoảng cách giữa 2 mặt phẳng $(BA'C'),(ACD')$. (c) Tính khoảng cách giữa 2 đường thẳng $BC',CD'$.
\end{baitoan}

\begin{baitoan}[\cite{SGK_Toan_11_hinh_hoc_co_ban}, 6., p. 119]
	Chứng minh nếu đường thẳng nối trung điểm 2 cạnh $AB,CD$ của tứ diện $ABCD$ là đường vuông góc chung của $AB,CD$ thì $AC = BD$, $AD = BC$.
\end{baitoan}

\begin{baitoan}[\cite{SGK_Toan_11_hinh_hoc_co_ban}, 7., p. 120]
	Cho hình chóp tam giác đều $S.ABC$ có cạnh đáy bằng $3a$, cạnh bên bằng $2a$. Tính khoảng cách từ $S$ tới mặt đáy $(ABC)$.
\end{baitoan}

\begin{baitoan}[\cite{SGK_Toan_11_hinh_hoc_co_ban}, 8., p. 120]
	Cho tứ diện đều $ABCD$ cạnh $a$. Tính khoảng cách giữa 2 cạnh đối của tứ diện đều đó.
\end{baitoan}

%------------------------------------------------------------------------------%

\section{Miscellaneous}

%------------------------------------------------------------------------------%

\printbibliography[heading=bibintoc]
	
\end{document}