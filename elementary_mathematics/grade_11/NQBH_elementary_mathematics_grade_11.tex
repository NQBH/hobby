\documentclass[oneside]{book}
\usepackage[backend=biber,natbib=true,style=authoryear]{biblatex}
\addbibresource{/home/hong/1_NQBH/reference/bib.bib}
\usepackage[utf8]{vietnam}
\usepackage{tocloft}
\renewcommand{\cftsecleader}{\cftdotfill{\cftdotsep}}
\usepackage[colorlinks=true,linkcolor=blue,urlcolor=red,citecolor=magenta]{hyperref}
\usepackage{amsmath,amssymb,amsthm,mathtools,float,graphicx,algpseudocode,algorithm,tcolorbox,tikz,tkz-tab}
\DeclareMathOperator{\arccot}{arccot}
\usepackage[inline]{enumitem}
\allowdisplaybreaks
\numberwithin{equation}{section}
\newtheorem{assumption}{Assumption}[section]
\newtheorem{nhanxet}{Nhận xét}[section]
\newtheorem{conjecture}{Conjecture}[section]
\newtheorem{corollary}{Corollary}[section]
\newtheorem{hequa}{Hệ quả}[section]
\newtheorem{definition}{Definition}[section]
\newtheorem{dinhnghia}{Định nghĩa}[section]
\newtheorem{example}{Example}[section]
\newtheorem{vidu}{Ví dụ}[section]
\newtheorem{lemma}{Lemma}[section]
\newtheorem{notation}{Notation}[section]
\newtheorem{principle}{Principle}[section]
\newtheorem{problem}{Problem}[section]
\newtheorem{baitoan}{Bài toán}[section]
\newtheorem{proposition}{Proposition}[section]
\newtheorem{menhde}{Mệnh đề}[section]
\newtheorem{question}{Question}[section]
\newtheorem{cauhoi}{Câu hỏi}[section]
\newtheorem{remark}{Remark}[section]
\newtheorem{luuy}{Lưu ý}[section]
\newtheorem{theorem}{Theorem}[section]
\newtheorem{tiende}{Tiên đề}[section]
\newtheorem{dinhly}{Định lý}[section]
\usepackage[left=0.5in,right=0.5in,top=1.5cm,bottom=1.5cm]{geometry}
\usepackage{fancyhdr}
\pagestyle{fancy}
\fancyhf{}
\lhead{\small \textsc{Sect.} ~\thesection}
\rhead{\small \nouppercase{\leftmark}}
\renewcommand{\sectionmark}[1]{\markboth{#1}{}}
\cfoot{\thepage}
\def\labelitemii{$\circ$}

\title{Some Topics in Elementary Mathematics\texttt{/}Grade 11}
\author{Nguyễn Quản Bá Hồng\footnote{Independent Researcher, Ben Tre City, Vietnam\\e-mail: \texttt{nguyenquanbahong@gmail.com}; website: \url{https://nqbh.github.io}.}}
\date{\today}

\begin{document}
\frontmatter
\maketitle
\setcounter{secnumdepth}{4}
\setcounter{tocdepth}{3}
\tableofcontents
\newpage

%------------------------------------------------------------------------------%

\mainmatter

\part{Đại Số \& Giải Tích -- Algebra \& Analysis}

\chapter{Hàm Số Lượng Giác \& Phương Trình Lượng Giác -- Trigonometric Function \& Trigonometric Equation}

``Nhiều hiện tượng tuần hoàn đơn giản trong thực tế được mô tả bởi những hàm số lượng giác. Chương này cung cấp những kiến thức cơ bản về các \textit{hàm số lượng giác} \& cách giải các \textit{phương trình lượng giác} đơn giản.'' -- \cite[p. 3]{SGK_Toan_11_dai_so_giai_tich_nang_cao}

\begin{quotation}
	\textbf{Nội dung.} \textit{Tính chất tuần hoàn của các hàm số lượng giác \& phương pháp sử dụng đường tròn lượng giác để tìm nghiệm của các phương trình lượng giác cơ bản, kỹ năng biến đổi lượng giác \& kỹ năng giải các dạng phương trình lượng giác}.
\end{quotation}

\section{Công Thức Lượng Giác}
\textbf{Nội dung.} \textit{1 số công thức lượng giác cơ bản, trình bày số phức dưới dạng lượng giác \& ứng dụng}.

\subsection{Công thức lượng giác cơ bản}

\subsubsection{Công thức cộng}
\begin{equation}
	\label{ctc}
	\tag{ctc}
	\boxed{\left\{\begin{split}
		\cos(\alpha - \beta) &= \cos\alpha\cos\beta + \sin\alpha\sin\beta,\\
		\cos(\alpha + \beta) &= \cos\alpha\cos\beta - \sin\alpha\sin\beta,\\
		\sin(\alpha - \beta) &= \sin\alpha\cos\beta - \cos\alpha\sin\beta,\\
		\sin(\alpha + \beta) &= \sin\alpha\cos\beta + \cos\alpha\sin\beta.
	\end{split}\right.\ \forall\alpha,\beta\in\mathbb{R}.}
\end{equation}
Có thể viết tắt \eqref{ctc}bằng cách sử dụng các ký hiệu $\pm,\mp$ như sau:
\begin{align*}
	\boxed{\cos(\alpha\pm\beta) = \cos\alpha\cos\beta \mp \sin\alpha\sin\beta,\ \sin(\alpha\pm\beta) = \sin\alpha\cos\beta\pm\cos\alpha\sin\beta,\ \forall\alpha,\beta\in\mathbb{R}.}
\end{align*}

\begin{proof}[Chứng minh \eqref{ctc}]
	\begin{enumerate*}
		\item[\textbf{(a)}] ``Ta chỉ cần chứng minh công thức đầu tiên rồi từ đó dùng giá trị lượng giác của các góc liên kết để suy ra các công thức còn lại. Giả sử các điểm $M$ \& $N$ nằm trên đường tròn lượng giác tâm $O$, gốc $A$ sao cho góc lượng giác $(OA,OM) = \alpha$, $(OA,ON) = \beta$ thì $\overrightarrow{OM}$ có tọa độ $(\cos\alpha;\sin\alpha)$, $\overrightarrow{ON}$ có tọa độ $(\cos\beta;\sin\beta)$, từ đó tích vô hướng $\overrightarrow{OM}\cdot\overrightarrow{ON} = \cos\alpha\cos\beta + \sin\alpha\sin\beta$. Mặt khác, $\overrightarrow{OM}\cdot\overrightarrow{ON} = |\overrightarrow{OM}||\overrightarrow{ON}|\cos\widehat{NOM} = \cos\widehat{NOM} = \cos(ON,OM)$ $= \cos[(OA,OM) - (OA,ON)] = \cos(\alpha - \beta)$, nên suy ra $\cos(\alpha - \beta) = \cos\alpha\cos\beta + \sin\alpha\sin\beta$.'' -- \cite[pp. 5--6]{TL_chuyen_Toan_Dai_So_Giai_Tich_11}.
		\item[\textbf{(b)}] Thay $\beta$ trong công thức vừa thu được ở (a) bởi $-\beta$, $\cos(\alpha + \beta) = \cos\alpha\cos(-\beta) + \sin\alpha\sin(-\beta) = \cos\alpha\cos\beta - \sin\alpha\sin\beta$.
		\item[\textbf{(c)}] $\sin(\alpha - \beta) = \cos\left(\frac{\pi}{2} - \alpha + \beta\right) = \cos\left(\frac{\pi}{2} - \alpha\right)\cos\beta - \sin\left(\frac{\pi}{2} - \alpha\right)\sin\beta = \sin\alpha\cos\beta - \cos\alpha\sin\beta$, trong đó ta sử dụng công thức $\sin x = \cos\left(\frac{\pi}{2} - x\right)$, $\cos x = \sin\left(\frac{\pi}{2} - x\right)$, $\forall x\in\mathbb{R}$\footnote{I.e., với 2 góc phụ nhau, sin 1 góc bất kỳ bằng côsin góc còn lại.} \& công thức vừa chứng minh ở (b) với $\alpha$ được thay bởi $\frac{\pi}{2} - \alpha$.
		\item[\textbf{(d)}] Thay $\beta$ trong công thức vừa thu được ở (c) bởi $-\beta$.
	\end{enumerate*}
	Các công thức cộng \eqref{ctc} được chứng minh.
\end{proof}
Kiểm tra nhanh tính hợp lý của \eqref{ctc}:
\begin{align*}
	&\sin^2(\alpha - \beta) + \cos^2(\alpha - \beta) = (\sin\alpha\cos\beta - \cos\alpha\sin\beta)^2 + (\cos\alpha\cos\beta + \sin\alpha\sin\beta)^2\\
	&\hspace{5mm}= \sin^2\alpha\cos^2\beta - 2\sin\alpha\cos\beta\cos\alpha\sin\beta + \cos^2\alpha\sin^2\beta + \cos^2\alpha\cos^2\beta + 2\cos\alpha\cos\beta\sin\alpha\sin\beta + \sin^2\alpha\sin^2\beta\\
	&\hspace{5mm}= \sin^2\alpha\cos^2\beta + \cos^2\alpha\sin^2\beta + \cos^2\alpha\cos^2\beta + \sin^2\alpha\sin^2\beta = (\sin^2\alpha + \cos^2\alpha)(\sin^2\beta + \cos^2\beta)  = 1,\\
	&\sin^2(\alpha + \beta) + \cos^2(\alpha + \beta) = (\sin\alpha\cos\beta + \cos\alpha\sin\beta)^2 + (\cos\alpha\cos\beta - \sin\alpha\sin\beta)^2\\
	&\hspace{5mm}= \sin^\alpha\cos^2\beta + 2\sin\alpha\cos\beta\cos\alpha\sin\beta + \cos^2\alpha\sin^2\beta + \cos^2\alpha\cos^2\beta - 2\cos\alpha\cos\beta\sin\alpha\sin\beta + \sin^2\alpha\sin^2\beta\\
	&\hspace{5mm}= \sin^2\alpha\cos^2\beta + \cos^2\alpha\sin^2\beta + \cos^2\alpha\cos^2\beta + \sin^2\alpha\sin^2\beta = (\sin^2\alpha + \cos^2\alpha)(\sin^2\beta + \cos^2\beta)  = 1.
\end{align*}
Từ \eqref{ctc} dễ suy ra:
\begin{align}
	\label{ctc1}
	\tag{ctc'}
	\boxed{\tan(\alpha + \beta) = \frac{\tan\alpha + \tan\beta}{1 - \tan\alpha\tan\beta},\ \tan(\alpha - \beta) = \frac{\tan\alpha - \tan\beta}{1 + \tan\alpha\tan\beta},\ \forall\alpha,\beta\in\mathbb{R}\mbox{ s.t. } \cos\alpha\cos\beta\cos(\alpha\pm\beta)\ne 0,\,\tan\alpha\tan\beta\ne\pm 1.}
\end{align}
Công thức \eqref{ctc1} cũng được gọi là công thức cộng. Có thể viết tắt \eqref{ctc1}bằng cách sử dụng các ký hiệu $\pm,\mp$ như sau:
\begin{align*}
	\boxed{\tan(\alpha\pm\beta) = \frac{\tan\alpha\pm\tan\beta}{1\mp\tan\alpha\tan\beta},\ \forall\alpha,\beta\in\mathbb{R}\mbox{ s.t. } \cos\alpha\cos\beta\cos(\alpha\pm\beta)\ne 0,\,\tan\alpha\tan\beta\ne\pm 1.}
\end{align*}

\begin{proof}[1st Chứng minh \eqref{ctc1}]
	Sử dụng \eqref{ctc}, với $\alpha,\beta\in\mathbb{R}$ thỏa giả thiết,
	\begin{align*}
		\tan(\alpha + \beta) = \frac{\sin(\alpha + \beta)}{\cos(\alpha + \beta)} = \frac{\sin\alpha\cos\beta + \cos\alpha\sin\beta}{\cos\alpha\cos\beta - \sin\alpha\sin\beta} = \frac{\frac{\sin\alpha}{\cos\alpha} + \frac{\sin\beta}{\cos\beta}}{1 - \frac{\sin\alpha\sin\beta}{\cos\alpha\cos\beta}} = \frac{\tan\alpha + \tan\beta}{1 - \tan\alpha\tan\beta},
	\end{align*}
	trong đó đẳng thức thứ 3 thu được bằng cách chia cả tử thức \& mẫu thức cho $\cos\alpha\cos\beta$ (phép chia này có nghĩa vì $\cos\alpha\cos\beta\ne 0$). Thay $\beta$ bởi $-\beta$ trong biểu thức vừa thu được, ta thu được biểu thức còn lại.
\end{proof}
Ta vừa chứng minh \eqref{ctc1} từ vế trái sang vế phải (i.e., LHS $= \cdots =$ RHS\footnote{LHS is the abbreviation of `Left Hand Side' \& RHS is the abbreviation of `Right Hand Side'. In many English texts in mathematics, the abbreviations l.h.s. \& r.h.s. are also used. In Vietnamese texts in mathematics, the abbreviations VT (vế trái) \& VP\texttt{/}VF (vế phải) are commonly used.}, hay VT $=\cdots =$ VP), cách chứng minh sau đi theo chiều ngược lại (i.e., RHS $= \cdots =$ LHS, hay VP $=\cdots =$ VT).

\begin{proof}[2nd Chứng minh \eqref{ctc1}]
	Với $\alpha,\beta\in\mathbb{R}$ thỏa giả thiết,
	\begin{align*}
		\frac{\tan\alpha + \tan\beta}{1 - \tan\alpha\tan\beta} = \frac{\frac{\sin\alpha}{\cos\alpha} + \frac{\sin\beta}{\cos\beta}}{1 - \frac{\sin\alpha\sin\beta}{\cos\alpha\cos\beta}} = \frac{\sin\alpha\cos\beta + \cos\alpha\sin\beta}{\cos\alpha\cos\beta - \sin\alpha\sin\beta} = \frac{\sin(\alpha + \beta)}{\cos(\alpha + \beta)} = \tan(\alpha + \beta),
	\end{align*}
	trong đó đẳng thức thứ 3 sử dụng \eqref{ctc}. Thay $\beta$ bởi $-\beta$ trong biểu thức vừa thu được, ta thu được biểu thức còn lại.
\end{proof}

\subsubsection{Công thức nhân đôi, nhân 3}
Áp dụng công thức cộng \eqref{ctc}--\eqref{ctc1} với $\alpha = \beta$,
\begin{equation}
	\label{ctn2}
	\tag{ctn2}
	\boxed{\left\{\begin{split}
		\cos2\alpha &= \cos^2\alpha - \sin^2\alpha = 2\cos^2\alpha - 1 = 1 - 2\sin^2\alpha,\ \sin2\alpha = 2\sin\alpha\cos\alpha,\ \forall\alpha\in\mathbb{R}\\
		\tan2\alpha &= \frac{2\tan\alpha}{1 - \tan^2\alpha},\ \forall\alpha\in\mathbb{R}\mbox{ s.t. }\cos2\alpha\ne 0,\,\tan\alpha\ne\pm 1.
	\end{split}\right.}
\end{equation}
\& áp dụng tiếp \eqref{ctc} với $(\alpha,\beta) = (\alpha,2\alpha)$,
\begin{align}
	\label{ctn3}
	\tag{ctn3}
	\boxed{\cos3\alpha = 4\cos^3\alpha - 3\cos\alpha,\ \sin3\alpha = 3\sin\alpha - 4\sin^3\alpha,\ \forall\alpha\in\mathbb{R}.}
\end{align}

\begin{proof}[Chứng minh \eqref{ctn2}]
	Với $\alpha\in\mathbb{R}$ bất kỳ, áp dụng \eqref{ctc} với $(\alpha,\beta) = (\alpha,2\alpha)$:
	\begin{align*}
		\cos3\alpha &= \cos(\alpha + 2\alpha) = \cos\alpha(2\cos^2\alpha - 1) - 2\sin^2\alpha\cos\alpha = \cos\alpha(2\cos^2\alpha - 1) - 2(1 - \cos^2\alpha)\cos\alpha = 4\cos^3\alpha - 3\cos\alpha,\\
		\sin3\alpha &= \sin(\alpha + 2\alpha) = \sin\alpha\cos2\alpha + \cos\alpha\sin2\alpha = \sin\alpha(1 - 2\sin^2\alpha) + 2\sin\alpha\cos^2\alpha\\
		&= \sin\alpha(1 - 2\sin^2\alpha) + 2\sin\alpha(1 - \sin^2\alpha) = 3\sin\alpha - 4\sin^3\alpha.
	\end{align*}
	Hoàn tất chứng minh.
\end{proof}

\begin{luuy}[Các khai triển khác của $\sin3\alpha,\cos3\alpha$]
	Sử dụng các biểu thức khác của $\sin2\alpha$ \& $\cos2\alpha$, ta cũng thu được:
	\begin{align*}
		\cos3\alpha &= \cos^3\alpha - 3\sin^2\alpha\cos\alpha = 2\cos^3\alpha - \cos\alpha - 2\sin^2\alpha\cos\alpha = \cos\alpha - 4\sin^2\alpha\cos\alpha,\\
		\sin3\alpha &= 3\sin\alpha\cos^2\alpha - \sin^3\alpha = 4\sin\alpha\cos^2\alpha - \sin\alpha = \sin\alpha - 2\sin^3\alpha + 2\sin\alpha\cos^2\alpha.
	\end{align*}
	Tuy nhiên, 2 công thức của \eqref{ctn3} mang lại nhiều lợi thế hơn do chúng là 2 đa thức bậc 3 của hàm $\sin\alpha$ \& $\cos\alpha$, chứ không phải là 1 biểu thức đại số gồm cả $\sin\alpha$ \& $\cos\alpha$.
\end{luuy}

\subsubsection{Công thức hạ bậc}
Từ công thức nhân đôi \eqref{ctn2} suy ra công thức hạ bậc:
\begin{align}
	\label{cthb}
	\tag{cthb}
	\boxed{\cos^2\alpha = \frac{1 + \cos2\alpha}{2},\ \sin^2\alpha = \frac{1 - \cos2\alpha}{2},\ \forall\alpha\in\mathbb{R}.}
\end{align}

\subsubsection{Công thức biến đổi tích thành tổng \& công thức biến đổi tổng thành tích}
``Từ các công thức cộng \eqref{ctc}, dễ dàng suy ra
\begin{equation*}
	\boxed{\left\{\begin{split}
		\cos\alpha\cos\beta &= \frac{1}{2}[\cos(\alpha + \beta) + \cos(\alpha - \beta)],&&\sin\alpha\sin\beta = -\frac{1}{2}[\cos(\alpha + \beta) - \cos(\alpha - \beta)],\\
		\sin\alpha\cos\beta &= \frac{1}{2}[\sin(\alpha + \beta) + \sin(\alpha - \beta)],&&\cos\alpha\sin\beta = \frac{1}{2}[\sin(\alpha + \beta) - \sin(\alpha - \beta)],
	\end{split}\right.\ \forall\alpha,\beta\in\mathbb{R}.}
\end{equation*}
Trong các công thức đó, đặt $x\coloneqq\alpha + \beta$, $y\coloneqq\alpha - \beta$ thì suy ra
\begin{equation*}
	\boxed{\left\{\begin{split}
		\cos x + \cos y &= 2\cos\frac{x + y}{2}\cos\frac{x - y}{2},&&\cos x - \cos y = -2\sin\frac{x + y}{2}\sin\frac{x - y}{2},\\
		\sin x + \sin y &= 2\sin\frac{x + y}{2}\cos\frac{x - y}{2},&&\sin x - \sin y = 2\cos\frac{x + y}{2}\sin\frac{x - y}{2},
	\end{split}\right.\ \forall x,y\in\mathbb{R}.}
\end{equation*}
Tất cả các công thức trên được dùng nhiều khi giải phương trình lượng giác.'' -- \cite[p. 7]{TL_chuyen_Toan_Dai_So_Giai_Tich_11}

\subsection{Số phức \& dạng lượng giác của nó -- Complex number \& its trigonometric representation}

\subsubsection{Số phức -- Complex number}
``Người ta xây dựng được 1 tập hợp số gọi là \textit{tập hợp số phức}, ký hiệu $\mathbb{C}$, chứa tập hợp số thực $\mathbb{R}$, trong đó có 2 phép toán cộng \& nhân (mà khi thu hẹp lên $\mathbb{R}$ thì đó là phép toán cộng, nhân số thực) thỏa mãn các tính chất tương tự phép toán cộng \& nhân số thực (giao, hoán, kết hợp, phân phối, $\ldots$), trong đó mọi số thực âm đều có căn bậc 2, mọi phương trình đa thức đều có nghiệm. Cụ thể là:
\begin{itemize}
	\item[(a)] Mỗi số phức được viết dưới dạng $z = a + bi$, $a,b\in\mathbb{R}$, $i$ là \textit{đơn vị ảo} ($i^2 = -1$), $a$ gọi là phần thực của $z$, $b$ gọi là \textit{phần ảo của $z$}.
\end{itemize}
\texttt{Skipped due to Toan 12}'' -- \cite[p. 7]{TL_chuyen_Toan_Dai_So_Giai_Tich_11}

%------------------------------------------------------------------------------%

\section{Các Hàm Số Lượng Giác -- Trigonometric Functions}
``Các hàm số lượng giác\texttt{/}trigonometric\footnote{\textbf{trigonometric} [a] (also \textbf{trigonometrical}) (\textit{mathematics}) connected with the types of mathematics that deals with the relationship between the sides \& angles of triangles.}\,\footnote{\textbf{trigonometry} [n] [uncountable] the type of mathematics that deals with the relationship between the sides \& angles of triangles.} functions thường được dùng để mô tả những hiện tượng thay đổi 1 cách tuần hoàn hay gặp trong thực tiễn, khoa học \& kỹ thuật.'' -- \cite[p. 4]{SGK_Toan_11_dai_so_giai_tich_nang_cao}

\subsection{Các hàm số $y = \sin x$ \& $y = \cos x$}

\subsubsection{Khái niệm}

\begin{dinhnghia}[Hàm số $\sin,\cos$]
	Quy tắc đặt tương ứng mỗi số thực $x\in\mathbb{R}$ với $\sin$ của góc lượng giác có số đo radian bằng $x$ được gọi là \emph{hàm số $\sin$}, ký hiệu là $y = \sin x$. Quy tắc đặt tương ứng mỗi số thực $x\in\mathbb{R}$ với côsin của góc lượng giác có số đo radian bằng $x$ được gọi là \emph{hàm số côsin}, ký hiệu là $y = \cos x$.
\end{dinhnghia}
``Tập xác định của các hàm số $y = \sin x$, $y = \cos x$ là $\mathbb{R}$. Do đó các hàm số sin \& côsin được viết là:
\begin{equation*}
	\begin{split}
		\sin:\mathbb{R}&\to\mathbb{R}\\
		x&\mapsto\sin x
	\end{split},\ \begin{split}
		\cos:\mathbb{R}&\to\mathbb{R}\\
		x&\mapsto\cos x
	\end{split}.
\end{equation*}
Hàm số $y = \sin x$ là 1 \textit{hàm số lẻ} vì $\sin(-x) = -\sin(x)$, $\forall x\in\mathbb{R}$, trong khi hàm số $y = \cos x$ là 1 \textit{hàm số chẵn} vì $\cos(-x) = \cos x$, $\forall x\in\mathbb{R}$.'' -- \cite[p. 4]{SGK_Toan_11_dai_so_giai_tich_nang_cao}. Về định nghĩa \& tính chất của hàm số chẵn \& hàm số lẻ, xem Sect. \ref{sect: even & odd functions}. Có thể xem thêm \href{https://vi.wikipedia.org/wiki/H%C3%A0m_s%E1%BB%91_ch%E1%BA%B5n_v%C3%A0_l%E1%BA%BB}{Wikipedia\texttt{/}hàm số chẵn \& lẻ} \& \href{https://en.wikipedia.org/wiki/Even_and_odd_functions}{Wikipedia\texttt{/}even \& odd functions}.

\subsubsection{Tính chất tuần hoàn của các hàm số $y = \sin x$ \& $y = \cos x$}
``Với mỗi $k\in\mathbb{Z}$, số $k2\pi$ thỏa mãn: $\sin(x + k2\pi) = \sin x$, $\forall x\in\mathbb{R},\,\forall k\in\mathbb{Z}$. Ngược lại, có thể chứng minh rằng số $T$ sao cho $\sin(x + T) = \sin x$, $\forall x\in\mathbb{R}$ phải có dạng $T = k2\pi$, với $k\in\mathbb{Z}$. Rõ ràng, trong các số dạng $k2\pi$ ($k\in\mathbb{Z}$), số dương nhỏ nhất là $2\pi$. Vậy đối với hàm số $y = \sin x$, số $T = 2\pi$ là số dương nhỏ nhất thỏa mãn $\sin(x + T) = \sin x$, $\forall x\in\mathbb{R}$. Hàm số $y = \cos x$ cũng có tinh chất tương tự. Ta nói 2 hàm số đó là những \textit{hàm số tuần hoàn với chu kỳ $2\pi$}.

Từ tính chất tuần hoàn với chu kỳ $2\pi$, ta thấy khi biết giá trị các hàm số $y = \sin x$ \& $y = \cos x$ trên 1 đoạn có độ dài $2\pi$ (e.g., đoạn $[0;2\pi]$ hay đoạn $[-\pi;\pi]$) thì ta tính được giá trị của chúng tại mọi $x\in\mathbb{R}$. (Cứ mỗi khi biến số được cộng thêm $2\pi$ thì giá trị của các hàm số đó lại trở về như cũ; điều này giải thích từ ``tuần hoàn'').'' -- \cite[p. 4--5]{SGK_Toan_11_dai_so_giai_tich_nang_cao}

\subsubsection{Sự biến thiên \& đồ thị của hàm số $y = \sin x$}
``Do hàm số $y = \sin x$ là hàm số tuần hoàn với chu kỳ $2\pi$ nên ta chỉ cần khảo sát hàm số đó trên 1 đoạn có độ dài $2\pi$, e.g., trên đoạn $[-\pi;\pi]$.''
\begin{itemize}
	\item \textbf{Chiều biến thiên.} \textit{Bảng biến thiên của hàm số $y = \sin x$ trên đoạn $[-\pi;\pi]$}:
	
	\begin{figure}[H]
		\centering
		\includegraphics[scale=0.15]{bang_bien_thien_sin}
		\caption{Bảng biến thiên của hàm số $y = \sin x$ trên đoạn $[-\pi;\pi]$.}
	\end{figure}
	
	\item \textbf{Đồ thị.} ``Khi vẽ đồ thị của hàm số $y = \sin x$ trên đoạn $[-\pi;\pi]$, ta nên để ý rằng: Hàm số $y = \sin x$ là 1 hàm số lẻ, do đó đồ thị của nó nhận gốc tọa độ làm tâm đối xứng. Vì vậy, đầu tiên ta vẽ đồ thị của hàm số $y = \sin x$ trên đoạn $[0;\pi]$.
	
	\begin{figure}[H]
		\centering
		\includegraphics[scale=0.2]{graph_sin}
		\caption{Đồ thị của hàm số $y = \sin x$ trên đoạn $[0,\pi]$.}
		\label{fig:graph of sin}
	\end{figure}
	Trên đoạn $[0;\pi]$, đồ thị của hàm số $y = \sin x$ (Fig. \ref{fig:graph of sin}) đi qua các điểm có tọa độ $(x;y)$ trong bảng sau:
	\begin{table}[H]
		\centering
		\begin{tabular}{|c|c|c|c|c|c|c|c|c|c|}
			\hline
			$x$ & $0$ & $\frac{\pi}{6}$ & $\frac{\pi}{4}$ & $\frac{\pi}{3}$ & $\frac{\pi}{2}$ & $\frac{2\pi}{3}$ & $\frac{3\pi}{4}$ & $\frac{5\pi}{6}$ & $\pi$ \\
			\hline
			$y = \sin x$ & $0$ & $\frac{1}{2}$ & $\frac{\sqrt{2}}{2}$ & $\frac{\sqrt{3}}{2}$ & $1$ & $\frac{\sqrt{3}}{2}$ & $\frac{\sqrt{2}}{2}$ & $\frac{1}{2}$ & $0$ \\
			\hline
		\end{tabular}
		\caption{Các giá trị của hàm $y = \sin x$ tại 1 số điểm $\in[0;\pi]$.}
	\end{table}
	Phần đồ thị của hàm số $y = \sin x$ trên đoạn $[0;\pi]$ cùng với hình đối xứng của nó qua gốc $O$ lập thành đồ thị của hàm số $y = \sin x$ trên đoạn $[-\pi,\pi]$ (Fig. \ref{fig:duong hinh sin}).
	
	\begin{figure}[H]
		\centering
		\includegraphics[scale=0.2]{duong_hinh_sin}
		\caption{Đồ thị của hàm số $y = \sin x$ trên $\mathbb{R}$ -- \textit{đường hình sin}.}
		\label{fig:duong hinh sin}
	\end{figure}
	Tịnh tiến phần đồ thị vừa vẽ sang trái, sang phải những đoạn có độ dài $2\pi,4\pi,6\pi,\ldots$ thì được toàn bộ đồ thị hàm số $y = \sin x$. Đồ thị đó được gọi là 1 \textit{đường hình sin} (Fig. \ref{fig:duong hinh sin}).'' -- \cite[pp. 6--7]{SGK_Toan_11_dai_so_giai_tich_nang_cao}
\end{itemize}

\begin{nhanxet}
	\begin{enumerate}
		\item ``Khi $x$ thay đổi, hàm số $y = \sin x$ nhận mọi giá trị thuộc đoạn $[-1;1]$. Ta nói \emph{tập giá trị} của hàm số $y = \sin x$ là đoạn $[-1;1]$.
		\item Hàm số $y = \sin x$ đồng biến trên khoảng $\left(-\frac{\pi}{2};\frac{\pi}{2}\right)$. Từ đó, do tính chất tuần hoàn với chu kỳ $2\pi$, hàm số $y = \sin x$ đồng biến trên mỗi khoảng $\left(-\frac{\pi}{2} + k2\pi;\frac{\pi}{2} + k2\pi\right)$, $k\in\mathbb{Z}$.'' -- \cite[p. 7]{SGK_Toan_11_dai_so_giai_tich_nang_cao}
	\end{enumerate}
\end{nhanxet}

\subsubsection{Sự biến thiên \& đồ thị của hàm số $y = \cos x$}
``Ta có thể tiến hành khảo sát sự biến thiên \& vẽ đồ thị của hàm số $y = \cos x$ tương tự như đã làm đối với hàm số $y = \sin x$ trên đây. Tuy nhiên, ta nhận thấy $\cos x = \sin\left(x + \frac{\pi}{2}\right)$, $\forall x\in\mathbb{R}$, nên bằng cách tịnh tiến đồ thị hàm số $y = \sin x$ sang trái 1 đoạn có độ dài $\frac{\pi}{2}$, ta được đồ thị hàm số $y = \cos x$ (nó cùng được gọi là 1 \textit{đường hình sin}) (Fig. \ref{fig:graph cos}).

\begin{figure}[H]
	\centering
	\includegraphics[scale=0.2]{graph_cos}
	\caption{Đồ thị của hàm số $y = \cos x$ trên $\mathbb{R}$.}
	\label{fig:graph cos}
\end{figure}
Căn cứ vào đồ thị của hàm số $y = \cos x$, ta lập được bảng biến thiên của hàm số đó trên đoạn $[-\pi;\pi]$ (Fig. \ref{fig:bang bien thien cos}):

\begin{figure}[H]
	\centering
	\includegraphics[scale=0.15]{bang_bien_thien_cos}
	\caption{Bảng biến thiên của hàm số $y = \cos x$ trên đoạn $[-\pi;\pi]$.}
	\label{fig:bang bien thien cos}
\end{figure}

\begin{nhanxet}
	\begin{enumerate}
		\item Khi $x$ thay đổi, hàm số $y = \cos x$ nhận mọi giá trị thuộc đoạn $[-1;1]$. Ta nói \emph{tập giá trị} của hàm số $y = \cos x$ là đoạn $[-1;1]$.
		\item Do hàm số $y = \cos x$ là hàm số chẵn nên đồ thị của hàm số $y = \cos x$ nhận trục tung làm trục đối xứng.
		\item Hàm số $y = \cos x$ đồng biến trên khoảng $(-\pi;0)$. Từ đó do tính chất tuần hoàn với chu kỳ $2\pi$, hàm số $y = \cos x$ đồng biến trên mỗi khoảng $(-\pi + k2\pi;k2\pi)$, $k\in\mathbb{Z}$.'' -- \cite[pp. 8--9]{SGK_Toan_11_dai_so_giai_tich_nang_cao}
	\end{enumerate}
\end{nhanxet}

\begin{table}[H]
	\centering
	\begin{tabular}{|p{9cm}|p{9cm}|}
		\hline
		\textbf{Hàm số $y = \sin x$} & \textbf{Hàm số $y = \cos x$} \\
		\hline
		Có tập xác định là $\mathbb{R}$ & Có tập xác định là $\mathbb{R}$ \\
		\hline
		Có tập giá trị là $[-1;1]$ & Có tập giá trị là $[-1;1]$ \\
		\hline
		Là hàm số lẻ & Là hàm số chẵn \\
		\hline
		Là hàm số tuần hoàn với chu kỳ $2\pi$ & Là hàm số tuần hoàn với chu kỳ $2\pi$ \\
		\hline
		Đồng biến trên mỗi khoảng $\left(-\frac{\pi}{2} + k2\pi;\frac{\pi}{2} + k2\pi\right)$ \& nghịch biến trên mỗi khoảng $\left(\frac{\pi}{2} + k2\pi;\frac{3\pi}{2} + k2\pi\right)$, $k\in\mathbb{Z}$ & Đồng biến trên mỗi khoảng $(-\pi + k2\pi;k2\pi)$ \& nghịch biến trên mỗi khoảng $(k2\pi;\pi + k2\pi)$, $k\in\mathbb{Z}$ \\
		\hline
		Có đồ thị là 1 đường hình sin & Có đồ thị là 1 đường hình sin \\
		\hline
	\end{tabular}
	\caption{So sánh tính chất của 2 hàm số $y = \sin x$ \& $y = \cos x$.}
\end{table}

\subsection{Các hàm số $y = \tan x$ \& $y = \cot x$}

\subsubsection{Định nghĩa}
\begin{itemize}
	\item ``Với mỗi số thực $x\in\mathbb{R}$ mà $\cos x\ne 0$, i.e., $x\ne\frac{\pi}{2} + k\pi$ ($k\in\mathbb{Z}$), ta xác định được số thực $\tan x = \frac{\sin x}{\cos x}$. Đặt $\mathcal{D}_1\coloneqq\mathbb{R}\backslash\left\{\frac{\pi}{2} + k\pi|k\in\mathbb{Z}\right\}$.
	
	\begin{dinhnghia}[Hàm số $\tan$]
		Quy tắc đặt tương ứng mỗi số $x\in\mathcal{D}_1$ với số thực $\tan x = \frac{\sin x}{\cos x}$ được gọi là \emph{hàm số tang}, ký hiệu là $y = \tan x$.
	\end{dinhnghia}
	Vậy hàm số $y = \tan x$ có tập xác định $\mathcal{D}_1$; ta viết
	\begin{align*}
		\tan:\mathcal{D}_1&\to\mathbb{R}\\
		x&\mapsto\tan x.
	\end{align*}
	\item Với mỗi số thực $x\in\mathbb{R}$ mà $\sin x\ne 0$, i.e., $x\ne k\pi$tan ($k\in\mathbb{Z}$), ta xác định được số thực $\cot x = \frac{\cos x}{\sin x}$. Đặt $\mathcal{D}_2\coloneqq\mathbb{R}\backslash\{k\pi|k\in\mathbb{Z}\}$.
	
	\begin{dinhnghia}[Hàm số $\cot$]
		Quy tắc đặt tương ứng mỗi số $x\in\mathcal{D}_2$ với số thực $\cot x = \frac{\cos x}{\sin x}$ được gọi là \emph{hàm số côtang}, ký hiệu là $y = \cot x$.
	\end{dinhnghia}
	Vậy hàm số $y = \cot x$ có tập xác định là $\mathcal{D}_2$; ta viết
	\begin{align*}
		\cot:\mathcal{D}_2&\to\mathbb{R}\\
		x&\mapsto\cot x.
	\end{align*}
	
	\begin{figure}[H]
		\centering
		\includegraphics[scale=0.15]{truc_tan_truc_cot}
		\caption{Trục tang \& trục côtang.}
		\label{fig:truc tan truc cot}
	\end{figure}
	Trên hình \ref{fig:truc tan truc cot}, ta có $(OA,OM) = x$, $\tan x = \overline{AT}$, $\cot x = \overline{BS}$.
	
	\begin{nhanxet}
		\begin{enumerate}
			\item Hàm số $y = \tan x$ là 1 \emph{hàm số lẻ} vì nếu $x\in\mathcal{D}_1$ thì $-x\in\mathcal{D}_1$ \& $\tan(-x) = -\tan x$.
			\item Hàm số $y = \cot x$ cũng là 1 \emph{hàm số lẻ} vì nếu $x\in\mathcal{D}_2$ thì $-x\in\mathcal{D}_2$ \& $\cot(-x) = -\cot x$.'' -- \cite[pp. 9--10]{SGK_Toan_11_dai_so_giai_tich_nang_cao}
		\end{enumerate}
	\end{nhanxet}	
\end{itemize}

\subsubsection{Tính chất tuần hoàn}
``Có thể chứng minh rằng $T = \pi$ là số dương nhỏ nhất thỏa mãn $\tan(x + T) = \tan x$, $\forall x\in\mathcal{D}_1$, \& $T = \pi$ cũng là số dương nhỏ nhất thỏa mãn $\cot(x + T) = \cot x$, $\forall x\in\mathcal{D}_2$. Ta nói các hàm số $y = \tan x$ \& $y = \cot x$ là những \textit{hàm số tuần hoàn với chu kỳ $\pi$}.'' -- \cite[p. 10]{SGK_Toan_11_dai_so_giai_tich_nang_cao}

\subsubsection{Sự biến thiên \& đồ thị của hàm số $y = \tan x$}
``Do tính chất tuần hoanf với chu kỳ $\pi$ của hàm số $y = \tan x$, ta chỉ cần khảo sát sự biến thiên \& vẽ đồ thị của nó trên khoảng $\left(-\frac{\pi}{2};\frac{\pi}{2}\right)\subset\mathcal{D}_1$, rồi tịnh tiến phần đồ thị vừa vẽ sang trái, sang phải các đoạn của độ dài $\pi,2\pi,3\pi,\ldots$ thì được toàn bộ đồ thị của hàm số $y = \tan x$.
\begin{itemize}
	\item \textit{Chiều biến thiên}:
	
	\begin{figure}[H]
		\centering
		\includegraphics[scale=0.15]{chieu_bien_thien_tan}
		\caption{Chiều biến thiên của hàm $y = \tan x$.}
		\label{fig:chieu bien thien tan}
	\end{figure}
	Khi cho $x = (OA,OM)$ tăng từ $-\frac{\pi}{2}$ đến $\frac{\pi}{2}$ (không kể $\pm\frac{\pi}{2}$) thì điểm $M$ chạy trên đường tròn lượng giác theo chiều dương từ $B'$ đến $B$ (không kể $B'$ \& $B$). Khi đó điểm $T$ thuộc trục tang $At$ sao cho $\overline{AT} = \tan x$ chạy dọc theo $At$ suốt từ dưới lên trên, nên $\tan x$ \textit{tăng từ $-\infty$ đến $+\infty$} (qua quá trị $0$ khi $x = 0$).''
	\item \textit{Đồ thị}: ``Đồ thị của hàm số $y = \tan x$ có dạng như ở hình \ref{fig:graph tan}.
	
	\begin{figure}[H]
		\centering
		\includegraphics[scale=0.15]{graph_tan}
		\caption{Đồ thị của hàm $y = \tan x$.}
		\label{fig:graph tan}
	\end{figure}

	\begin{nhanxet}
		\begin{enumerate}
			\item Khi $x$ thay đổi, hàm số $y = \tan x$ nhận mọi giá trị thực. Ta nói \emph{tập giá trị} của hàm số $y = \tan x$ là $\mathbb{R}$.
			\item Vì hàm số $y = \tan x$ là hàm số lẻ nên đồ thị của nó nhận gốc tọa độ làm tâm đối xứng.
			\item Hàm số $y = \tan x$ không xác định tại $x = \frac{\pi}{2} + k\pi$ ($k\in\mathbb{Z}$). Với mỗi $k\in\mathbb{Z}$, đường thẳng vuông góc với trục hành, đi qua điểm $\left(\frac{\pi}{2} + k\pi;0\right)$ gọi là 1 \emph{đường tiệm cận} của đồ thị hàm số $y = \tan x$. (Từ ``tiệm cận'' có nghĩa là ngày càng gần. E.g., nói đường thẳng $x = \frac{\pi}{2}$ là 1 đường tiệm cận của đồ thị hàm số $y = \tan x$ nhằm diễn tả tính chất: điểm $M$ trên đồ thị có hoành độ càng gần $\frac{\pi}{2}$ thì $M$ càng gần đường thẳng $x = \frac{\pi}{2}$).'' -- \cite[pp. 11--12]{SGK_Toan_11_dai_so_giai_tich_nang_cao}
		\end{enumerate}
	\end{nhanxet}	   
\end{itemize}

\subsubsection{Sự biến thiên \& đồ thị của hàm số $y = \cot x$}
``Hàm số $y = \cot x$ xác định trên $\mathcal{D}_2 = \mathbb{R}\backslash\{k\pi|k\in\mathbb{Z}\}$ là 1 hàm số tuần hoàn với chu kỳ $\pi$. Ta có thể khảo sát sự biến thiên \& vẽ đồ thị của nó tương tự như đã làm đối với hàm số $y = \tan x$. Đồ thị của hàm số $y = \cot x$ có dạng như hình \ref{fig:graph cot}.

\begin{figure}[H]
	\centering
	\includegraphics[scale=0.15]{graph_cot}
	\caption{Đồ thị của hàm $y = \cot x$.}
	\label{fig:graph cot}
\end{figure}
Nó nhận mỗi đường thẳng vuông góc với trục hoành, đi qua điểm $(k\pi;0)$, $k\in\mathbb{Z}$ làm 1 đường tiệm cận.'' -- \cite[p. 12]{SGK_Toan_11_dai_so_giai_tich_nang_cao}

\begin{table}[H]
	\centering
	\begin{tabular}{|p{9cm}|p{9cm}|}
		\hline
		\textbf{Hàm số $y = \tan x$} & \textbf{Hàm số $y = \cot x$} \\
		\hline
		Có tập xác định là $\mathcal{D}_1 = \mathbb{R}\backslash\left\{\frac{\pi}{2} + k\pi|k\in\mathbb{Z}\right\}$ & Có tập xác định là $\mathcal{D}_2 = \mathbb{R}\backslash\{k\pi|k\in\mathbb{Z}\}$ \\
		\hline
		Có tập giá trị là $\mathbb{R}$ & Có tập giá trị là $\mathbb{R}$ \\
		\hline
		Là hàm số lẻ & Là hàm số lẻ \\
		\hline
		Là hàm số tuần hoàn với chu kỳ $\pi$ & Là hàm số tuần hoàn với chu kỳ $\pi$ \\
		\hline
		Đồng biến trên mỗi khoảng $\left(-\frac{\pi}{2} + k\pi;\frac{\pi}{2} + k\pi\right)$, $k\in\mathbb{Z}$ & Nghịch biến trên mỗi khoảng $(k\pi;\pi + k\pi)$, $k\in\mathbb{Z}$ \\
		\hline
		Có đồ thị nhận mỗi đường thẳng $x = \frac{\pi}{2} + k\pi$ ($k\in\mathbb{Z}$) làm 1 đường tiệm cận & Có đồ thị nhận mỗi đường thẳng $x = k\pi$ ($k\in\mathbb{Z}$) làm 1 đường tiệm cận \\
		\hline
	\end{tabular}
	\caption{So sánh tính chất của 2 hàm số $y = \tan x$ \& $y = \cot x$.}
\end{table}

\subsection{Về khái niệm hàm số tuần hoàn}
``Các hàm số $y = \sin x$, $y = \cos x$ là những hàm số tuần hoàn với chu kỳ $2\pi$; các hàm số $y = \tan x$, $y = \cot x$ là những hàm số tuần hoàn với chu kỳ $\pi$. 1 cách tổng quát:

\begin{dinhnghia}[Hàm số tuần hoàn]
	Hàm số $y = f(x)$ xác định trên tập hợp $\mathcal{D}$ được gọi là \emph{hàm số tuần hoàn} nếu có số $T\ne 0$ sao cho với mọi $x\in\mathcal{D}$ ta có $x + T\in\mathcal{D}$, $x - T\in\mathcal{D}$ \& $f(x + T) = f(x)$. Nếu có số $T$ dương nhỏ nhất thỏa mãn các điều kiện trên thì hàm số đó được gọi là 1 \emph{hàm số tuần hoàn với chu kỳ $T$}.'' -- \cite[p. 13]{SGK_Toan_11_dai_so_giai_tich_nang_cao}
\end{dinhnghia}

\begin{vidu}
	Các hàm số có dạng $y = a\sin bx$, với $a,b\in\mathbb{R}^\star\coloneqq\mathbb{R}\backslash\{0\}$ là những hàm số tuần hoàn.
\end{vidu}

\subsection{Dao động điều hòa}
``Nhiều hiện tượng tự nhiên thay đổi có tính chất tuần hoàn (lặp đi lặp lại sau khoảng thời gian xác định) như: Chuyển động của các hành tinh trong hệ mặt trời, chuyển động của guồng nước quay, chuyển động của quả lắc đồng hồ, sự biến thiên của cường độ dòng điện xoay chiều, $\ldots$. Hiện tượng tuần hoàn đơn giản nhất là \textit{dao động điều hòa} được mô tả bởi hàm số $y = A\sin(\omega x + \alpha) + B$, trong đó $A,B,\omega$ \& $\alpha$ là những hằng số; $A$ \& $\omega$ khác $0$. Đó là hàm số tuần hoàn với chu kỳ $\frac{2\pi}{|\omega|}$; $|A|$ gọi là \textit{biên độ}. Đồ thị của nó là 1 \textit{đường hình sin} có được từ đồ thị của hàm số $y = A\sin\omega x$ bằng cách tịnh tiến thích hợp (theo vector $-\frac{\alpha}{\omega}\vec{i}$ rồi theo vector $B\vec{j}$, i.e., tịnh tiến theo vector $-\frac{\alpha}{\omega}\vec{i} + B\vec{j}$).'' -- \cite[pp. 15--16]{SGK_Toan_11_dai_so_giai_tich_nang_cao}

\subsection{Âm thanh}
``Âm thanh được tạo nên bởi sự thay đổi áp suất của môi trường vật chất (chất khí, chất lỏng, chất rắn) 1 cách tuần hoàn theo thời gian (dao động tuần hoàn) \& được lan truyền trong môi trường đó (sóng âm thanh).

Nếu dao động tuần hoàn ấy có chu kỳ $T$ (đo bằng đơn vị thời gian là giây) thì $\frac{1}{T}$ gọi là \textit{tần số} của dao động (i.e., số chu kỳ trong 1 giây); đơn vị của tần số là Hertz (abbr., Hz). Âm thanh tai người nghe được là dao động có tần số trong khoảng từ 17--20 Hz đến 20000 Hz. Dao động có tần số cao hơn 20000 Hz được gọi là \textit{siêu âm}.

Trong âm nhạc (nghệ thuật phối hợp các âm thanh) người ta thường dùng những nốt nhạc để ghi những âm có tần số xác định. Tần số dao động càng lớn thì âm càng cao. Khi tăng tần số 1 âm lên gấp đôi thì ta nói cao độ của âm đó được tăng thêm 1 quãng 8. Người ta thường chia quãng 8 đó thành 12 quãng bằng nhau, mỗi quãng gọi là 1 bán cung để đo chênh lệch cao độ giữa các âm (xem SGK Âm nhạc \& Mỹ thuật lớp 7). Với 2 âm cách nhau 1 bán cung, tỷ số các tần số của chúng bằng $\sqrt[12]{2}$; với 2 âm cách nhau 1 cung (i.e., 2 bán cung), tỷ số các tần số của chúng bằng $(\sqrt[12]{2})^2 = \sqrt[6]{2}$. Ở khuông nhạc dưới đây có ghi các nốt nhạc của 1 ``âm giai'' (quãng 8) cùng khoảng cách cao độ giữa 2 âm ứng với 2 nốt kề nhau. Âm \textit{la} của âm giai đó có tần số 440 Hz (do đó, e.g., âm \textit{si} kế đó có tần số $440\sqrt[6]{2}$ Hz).

\begin{figure}[H]
	\centering
	\includegraphics[scale=0.2]{khuong_nhac}
	\caption{Khuông nhạc.}
	\label{fig:khuong nhac}
\end{figure}
Trong âm nhạc, ngoài các âm riêng lẻ còn có hợp âm (kết hợp các âm thanh). Nhà toán học Pháp \href{https://en.wikipedia.org/wiki/Joseph_Fourier}{Joseph Fourier} (1768--1830) đã chứng minh rằng 1 hàm số tuần hoàn với chu kỳ $T$ có thể phân tích thành ``tổng'' của 1 hằng số với những hàm số tuần hoàn có đồ thị là những đường hình sin với chu kỳ $\frac{T}{n}$ ($n\in\mathbb{N}^\star$). Điều đó giúp ta hiểu sâu hơn về hợp âm, hòa âm, âm bội \& âm sắc.'' -- \cite[p. 18]{SGK_Toan_11_dai_so_giai_tich_nang_cao}

%------------------------------------------------------------------------------%

\section{Phương Trình Lượng Giác Cơ Bản -- Basic Trigonometric Equation}
``Trên thực tế, có nhiều bài toán dẫn đến việc giải các phương trình có 1 trong các dạng $\sin x = m$, $\cos x = m$, $\tan x = m$, \& $\cot x = m$, trong đó $x$ là ẩn số ($x\in\mathbb{R}$) \& $m$ là 1 số cho trước. Đó là các \textit{phương trình lượng giác cơ bản}.'' -- \cite[p. 19]{SGK_Toan_11_dai_so_giai_tich_nang_cao}

\subsection{Phương trình $\sin x = m$}
``Giả sử $m$ là 1 số đã cho. Xét phương trình
\begin{align}
	\label{sin}
	\tag{sin}
	\sin x = m.
\end{align}
Hiển nhiên phương trình \eqref{sin} xác định với mọi $x\in\mathbb{R}$. Ta đã biết $|\sin x|\le 1$ với mọi $x\in\mathbb{R}$. Do đó phương trình \eqref{sin} vô nghiệm khi $|m| > 1$. Mặt khác, khi $x$ thay đổi, $\sin x$ nhận mọi giá trị từ $-1$ đến $1$ nên phương trình \eqref{sin} luôn có nghiệm khi $|m|\le 1$.'' -- \cite[p. 20]{SGK_Toan_11_dai_so_giai_tich_nang_cao}
\begin{tcolorbox}
	Nếu $\alpha$ là 1 nghiệm của phương trình \eqref{sin}, i.e., $\sin\alpha = m$ thì
	\begin{equation}
		\label{root sin}
		\sin x = m\Leftrightarrow\left[\begin{split}
			x &= \alpha + k2\pi\\
			x &= \pi - \alpha + k2\pi
		\end{split}\right.\ (k\in\mathbb{Z}).
	\end{equation}
\end{tcolorbox}
``Ta nói rằng $x = \alpha + k2\pi$ \& $x = \pi - \alpha + k2\pi$ là 2 \textit{họ nghiệm} của phương trình \eqref{sin}.

Kể từ đây, để cho gọn ta quy ước rằng nếu trong 1 biểu thức nghiệm của phương trình lượng giác có chứa $k$ mà không giải thích gì thêm thì ta hiểu rằng $k$ nhận mọi giá trị thuộc $\mathbb{Z}$. E.g., $x = \alpha + k2\pi$ có nghĩa là $x$ lấy mọi giá trị thuộc tập hợp $\{\alpha,\alpha\pm 2\pi,\alpha\pm 4\pi,\alpha\pm 6\pi,\ldots\}$.'' -- \cite[p. 21]{SGK_Toan_11_dai_so_giai_tich_nang_cao}

``Trong mặt phẳng tọa độ, nếu vẽ đồ thị $(G)$ của hàm số $y = \sin x$ \& đường thẳng $(d)$: $y = m$ thì hoành độ mỗi giao điểm của $(d)$ \& $(G)$ (nếu có) là 1 nghiệm của phương trình $\sin x = m$.'' -- \cite[p. 22]{SGK_Toan_11_dai_so_giai_tich_nang_cao}

\begin{luuy}
	\begin{enumerate}
		\item ``Khi $m\in\{0;\pm 1\}$, công thức \eqref{root sin} có thể viết gọn như sau:
		\begin{align*}
			\sin x = 1\Leftrightarrow x = \frac{\pi}{2} + k2\pi,\ \sin x = -1\Leftrightarrow x = -\frac{\pi}{2} + k2\pi,\ \sin x = 0\Leftrightarrow x = k\pi.
		\end{align*}
		\item Dễ thấy rằng với $m$ cho trước mà $|m|\le 1$, phương trình $\sin x = m$ có đúng 1 nghiệm nằm trong đoạn $\left[-\frac{\pi}{2};\frac{\pi}{2}\right]$. Người ta thường ký hiệu đó là $\arcsin m$. Khi đó
		\begin{equation*}
			\boxed{\sin x = m \Leftrightarrow\left[\begin{split}
				x &= \arcsin m + k2\pi,\\
				x &= \pi - \arcsin m + k2\pi.
			\end{split}\right.}
		\end{equation*}
		\item Từ \eqref{root sin} ta thấy rằng: Nếu $\alpha$ \& $\beta$ là 2 số thực thì $\sin\beta = \sin\alpha$ khi \& chỉ khi có số nguyên $k$ để $\beta = \alpha + k2\pi$ hoặc $\beta = \pi - \alpha + k2\pi$, $k\in\mathbb{Z}$.'' -- \cite[pp. 22--23]{SGK_Toan_11_dai_so_giai_tich_nang_cao}
	\end{enumerate}
\end{luuy}

\subsection{Phương trình $\cos x = m$}
``Xét phương trình
\begin{align*}
	\label{cos}
	\tag{cos}
	\cos x = m,
\end{align*}
trong đó $m$ là 1 số cho trước. Hiển nhiên phương trình \eqref{cos} xác định với mọi $x\in\mathbb{R}$. Dễ thấy rằng: Khi $|m| > 1$, phương trình \eqref{cos} vô nghiệm. Khi $|m|\le 1$, phương trình (II) luôn có nghiệm. Để tìm tất cả các nghiệm của (II), trên trục côsin ta lấy điểm $H$ sao cho $\overline{OH} = m$. Gọi $(l)$ là đường thẳng đi qua $H$ \& vuông góc với trục côsin (Fig. \ref{fig:truc cos}).

\begin{figure}[H]
	\centering
	\includegraphics[scale=0.15]{truc_cos}
	\caption{Trục côsin.}
	\label{fig:truc cos}
\end{figure}
Do $|m|\le 1$ nên đường thẳng $(l)$ cắt đường tròn lượng giác tại 2 điểm $M_1$ \& $M_2$. 2 điểm này đối xứng với nhau qua trục côsin (chúng trùng nhau nếu $m = \pm 1$). Ta thấy số đo của các góc lượng giác $(OA,OM_1)$ \& $(OA,OM_2)$ là tất cả các nghiệm của \eqref{cos}. Nếu $\alpha$ là số đo của 1 góc trong chúng, nói cách khác, nếu $\alpha$ là 1 nghiệm của \eqref{cos} thì các góc đó có các số đo là $\pm\alpha + k2\pi$. Vậy ta có

\begin{tcolorbox}
	Nếu $\alpha$ là 1 nghiệm của phương trình \eqref{cos}, i.e., $\cos\alpha = m$ thì
	\begin{equation}
		\label{root cos}
		\cos x = m\Leftrightarrow\left[\begin{split}
			x &= \alpha + k2\pi,\\
			x &= -\alpha + k2\pi.
		\end{split}\right.
	\end{equation}
\end{tcolorbox}

\begin{luuy}
	\begin{enumerate}
		\item Đặc biệt, khi $m\in\{0;\pm 1\}$, công thức \eqref{root cos} có thể viết gọn như sau:
		\begin{align*}
			\cos x = 1\Leftrightarrow x = k2\pi,\ \cos x = -1\Leftrightarrow x = \pi + k2\pi,\ \cos x = 0\Leftrightarrow x = \frac{\pi}{2} + k\pi.
		\end{align*}
		\item Dễ thấy rằng với mọi số $m$ cho trước mà $|m|\le 1$, phương trình $\cos x = m$ có đúng 1 nghiệm nằm trong đoạn $[0;\pi]$. Người ta thường ký hiệu nghiệm đó là $\arccos m$. Khi đó
		\begin{equation*}
			\cos x = m\Leftrightarrow\left[\begin{split}
				x &= \arccos m + k2\pi,\\
				x &= -\arccos m + k2\pi,
			\end{split}\right.
		\end{equation*}
		mà cũng thường được viết là $x = \pm\arccos m + k2\pi$.
		\item Từ \eqref{root cos} ta thấy rằng: Nếu $\alpha$ \& $\beta$ là 2 số thực thì $\cos\beta = \cos\alpha$ khi \& chỉ khi có số nguyên $k$ để $\beta = \alpha + k2\pi$ hoặc $\beta = -\alpha + k2\pi$, $k\in\mathbb{Z}$.'' -- \cite[pp. 23--24]{SGK_Toan_11_dai_so_giai_tich_nang_cao}
	\end{enumerate}
\end{luuy}

\subsection{Phương trình $\tan x = m$}
``Cho $m$ là 1 số tùy ý. Xét phương trình
\begin{align}
	\label{tan}
	\tag{tan}
	\tan x = m.
\end{align}
Điều kiện xác định (ĐKXĐ) của phương trình \eqref{tan} là $\cos x\ne 0$. Ta đã biết, khi $x$ thay đổi, $\tan x$ nhận mọi giá trị từ $-\infty$ đến $+\infty$. Do đó phương trình \eqref{tan} luôn có nghiệm. Để tìm tất cả các nghiệm của \eqref{tan}, trên tục tang, ta lấy điểm $T$ sao cho $\overline{AT} = m$. Đường thẳng $OT$ cắt đường tròn lượng giác tại 2 điểm $M_1$ \& $M_2$ (Fig. \ref{fig:truc tan}).

\begin{figure}[H]
	\centering
	\includegraphics[scale=0.15]{truc_tan}
	\caption{Trục tang.}
	\label{fig:truc tan}
\end{figure}
Ta có: $\tan(OA,OM_1) = \tan(OA,OM_2) = \overline{AT} = m$. Gọi số đo của 1 trong các góc lượng giác $(OA,OM_1)$ \& $(OA,OM_2)$ là $\alpha$; i.e., $\alpha$ là 1 nghiệm nào đó của phương trình \eqref{tan}. Khi đó, các góc lượng giác $(OA,OM_1)$ \& $(OA,OM_2)$. Khi đó, các góc lượng giác $(OA,OM_1)$ \& $(OA,OM_2)$ có các số đo là $\alpha + k\pi$. Đó là tất cả các nghiệm của phương trình \eqref{tan} (hiển nhiên chúng thỏa mãn ĐKXĐ của \eqref{tan}). Vậy ta có:

\begin{tcolorbox}
	Nếu $\alpha$ là 1 nghiệm của phương trình \eqref{tan}, i.e., $\tan\alpha = m$ thì
	\begin{align}
		\label{root tan}
		\tan x = m\Leftrightarrow x = \alpha + k\pi.
	\end{align}
\end{tcolorbox}

\begin{luuy}
	\begin{enumerate}
		\item Dễ thấy rằng với mọi số $m\in\mathbb{R}$ cho trước, phương trình $\tan x = m$ có đúng 1 nghiệm nằm trong khoảng $\left(-\frac{\pi}{2};\frac{\pi}{2}\right)$. Người ta thường ký hiệu nghiệm đó là $\arctan m$. Khi đó
		\begin{align*}
			\boxed{\tan x = m\Leftrightarrow x = \arctan m + k\pi.}
		\end{align*}
		\item Từ \eqref{root tan} ta thấy rằng: Nếu $\alpha$ \& $\beta$ là 2 số thực mà $\tan\alpha$, $\tan\beta$ xác định thì $\tan\beta = \tan\alpha$ khi \& chỉ khi có số nguyên $k$ để $\beta = \alpha + k\pi$.'' -- \cite[pp. 25--26]{SGK_Toan_11_dai_so_giai_tich_nang_cao}
	\end{enumerate}
\end{luuy}

\subsection{Phương trình $\cot x = m$}
``Cho $m\in\mathbb{R}$ là 1 số tùy ý, xét phương trình
\begin{align}
	\label{cot}
	\tag{cot}
	\cot x = m.
\end{align}
ĐKXĐ của phương trình \eqref{cot} là $\sin x\ne 0$. Tương tự như đối với phương trình $\tan x = m$, ta có

\begin{tcolorbox}
	Nếu $\alpha$ là 1 nghiệm của phương trình \eqref{cot}, i.e., $\cot\alpha = m$ thì
	\begin{align}
		\label{root cot}
		\cot x = m\Leftrightarrow x = \alpha + k\pi.
	\end{align}
\end{tcolorbox}
'' -- \cite[pp. 26--27]{SGK_Toan_11_dai_so_giai_tich_nang_cao}

\begin{luuy}
	Dễ thấy rằng với mọi số $m\in\mathbb{R}$ cho trước, phương trình $\cot x = m$ có đúng 1 nghiệm nằm trong khoảng $(0;\pi)$. Người ta thường ký hiệu nghiệm đó là $\arccot m$. Khi đó:
	\begin{align*}
		\boxed{\cot x = m\Leftrightarrow x = \arccot m + k\pi.}
	\end{align*}
\end{luuy}

\subsection{1 số điều cần lưu ý}
\begin{enumerate}
	\item Khi đã cho số $m$, ta có thể tính được các giá trị $\arcsin m,\arccos m$ (với $|m|\le 1$), $\arctan m$ bằng máy tính bỏ túi với các phím $\sin^{-1},\cos^{-1}$ \& $\tan^{-1}$.
	\item $\arcsin m,\arccos m$ (với $|m|\le 1$), $\arctan m$ \& $\arccot m$ có giá trị là những số thực. Do đó ta viết, e.g., $\arctan 1 = \frac{\pi}{4}$ mà không viết $\arctan 1 = 45^\circ$.
	\item Khi xét các phương trình lượng giác ta đã coi ẩn số $x$ là số đo radian của các góc lượng giác. Trên thực tế, ta còn gặp những bài toán yêu cầu tìm số đo độ của các góc (cung) lượng giác sao cho sin (côsin, tang hoặc côtang) của chúng bằng số $m\in\mathbb{R}$ cho trước e.g. $\sin(x + 20^\circ) = \frac{\sqrt{3}}{2}$. Khi giải các phương trình này (mà làm dụng ngôn ngữ, ta vẫn gọi là giải các phương trình lượng giác), ta có thể áp dụng các công thức nêu trên \& lưu ý sử dụng ký hiệu số đo độ trong ``công thức nghiệm'' cho thống nhất, e.g., viết $x = 30^\circ + k360^\circ$ chứ không viết $x = 30^\circ + k2\pi$.
	
	Tuy nhiên, ta quy ước rằng nếu không có giải thích gì thêm hoặc trong phương trình lượng giác không sử dụng đơn vị đo góc là độ thì mặc nhiên ẩn số là số đo radian của góc lượng giác.'' -- \cite[p. 27]{SGK_Toan_11_dai_so_giai_tich_nang_cao}
\end{enumerate}

\subsection{Dùng máy tính bỏ túi để tìm 1 góc khi biết 1 giá trị lượng giác của nó}
``Các phím $\sin^{-1},\cos^{-1}$ \& $\tan^{-1}$ của máy tính bỏ túi \textsc{Casio} \textit{fx-500MS} được dùng để tìm số đo (độ hoặc radian) của 1 góc khi biết 1 trong các giá trị lượng giác của nó. Muốn thế đối với máy tính CASIO \textit{fx-500MS} ta thực hiện 2 bước sau:
\begin{enumerate}
	\item \textit{Ấn định đơn vị đo góc (độ hoặc radian)}. Muốn tìm số đo độ, ta ấn \fbox{MODE} \fbox{MODE} \fbox{MODE} \fbox{1}. Lúc này dòng trên cùng của màn hình xuất hiện chữ nhỏ \fbox{D}. Muốn tìm số đo radian, ta ấn \fbox{MODE} \fbox{MODE} \fbox{MODE} \fbox{2}. Lúc này dòng trên cùng của màn hình xuất hiện chữ nhỏ \fbox{R}.
	\item \textit{Tìm số đo góc}. Khi biết sin, côsin hay tang của góc $\alpha$ cần tìm bằng $m$, ta lần lượt ấn phím \fbox{SHIFT}, \& 1 trong các phím $\boxed{\sin^{-1}}$, $\boxed{\cos^{-1}}$, $\boxed{\tan^{-1}}$, rồi nhập giá trị lượng giác $m$ \& cuối cùng ấn phím $=$. Lúc này, trên màn hình cho kết quả là số đo của góc $\alpha$ (độ hay radian tùy theo bước 1).
\end{enumerate}

\begin{luuy}
	\begin{enumerate}
		\item Ở chế độ số đo radian, các phím $\sin^{-1},\cos^{-1}$ cho kết quả (khi $|m|\le 1$) là $\arcsin m,\arccos m$; phím $\tan^{-1}$ cho kết quả là $\arctan m$.
		\item Ở chế độ số đo độ, các phím $\sin^{-1}$ \& $\tan^{-1}$ cho kết quả là số đo góc $\alpha$ từ $-90^\circ$ đến $90^\circ$; phím $\cos^{-1}$ cho kết quả là số đo góc $\alpha$ từ $0^\circ$ đến $180^\circ$. Các kết quả ấy được hiển thị dưới dạng số thập phân.'' -- \cite[p. 27]{SGK_Toan_11_dai_so_giai_tich_nang_cao}
	\end{enumerate}
\end{luuy}
Xem \cite[Ví dụ 1--3, p. 31]{SGK_Toan_11_dai_so_giai_tich_nang_cao} để biết chi tiết thao tác bấm phím trên máy tính cầm tay.

%------------------------------------------------------------------------------%

\section{1 Số Dạng Phương Trình Lượng Giác Cơ Bản}

\subsection{1 số dạng phương trình lượng giác đơn giản}

\subsubsection{Phương trình bậc nhất \& bậc 2 đối với 1 hàm số lượng giác}
Để giải các phương trình lượng giác có dạng $P(\sin x) = 0$, $P(\cos x) = 0$, $P(\tan x) = 0$, $P(\cot x) = 0$ với $P$ là 1 đa thức có bậc 1 hoặc 2 (i.e., $\deg P\in\{1,2\}$ \footnote{Ký $\deg$ là viết tắt của từ ``degree'' tức là ``bậc''.}), ta chọn 1 biểu thức lượng giác thích hợp có mặt trong phương trình làm ẩn phụ \& quy về phương trình bậc nhất hoặc bậc 2 đối với ẩn phụ đó (có thể nêu hoặc không nêu ký hiệu ẩn phụ).

\paragraph{Phương trình bậc nhất đối với 1 hàm số lượng giác.} Xét các phương trình lượng giác có dạng $P(\sin x) = 0$, $P(\cos x) = 0$, $P(\tan x) = 0$, $P(\cot x) = 0$ với $P$ là 1 đa thức có bậc 1 (i.e., $\deg P = 1$), i.e.:
\begin{align*}
	a\sin(mx + n) + b = 0,\ a\cos(mx + n) + b = 0,\ a\tan(mx + n) + b = 0,\ a\cot(mx + n) + b = 0,\ a,b,m,n\in\mathbb{R},\,a\ne 0,\,m\ne 0.
\end{align*}
Tổng quát hơn, giải các phương trình lượng giác sau:
\begin{align*}
	a\sin f(x) + b = 0,\ a\cos f(x) + b = 0,\ a\tan f(x) + b = 0,\ a\cot f(x) + b = 0,
\end{align*}
trong đó $a,b\in\mathbb{R}$, $a\ne 0$, \& $f$ là 1 hàm số (đa thức, phân thức, hàm căn thức) sao cho phương trình $f(x) = m$ có thể giải được\texttt{/}solvable (có nghiệm hoặc vô nghiệm) trên tập số thực $\mathbb{R}$.

\paragraph{Phương trình bậc 2 đối với 1 hàm số lượng giác.} Xét các phương trình lượng giác có dạng $P(\sin x) = 0$, $P(\cos x) = 0$, $P(\tan x) = 0$, $P(\cot x) = 0$ với $P$ là 1 đa thức có bậc 2 (i.e., $\deg P = 2$), i.e.,
\begin{align*}
	a\sin^2(mx + n) + b\sin(mx + n) + c &= 0,\ a\cos^2(mx + n) + b\cos(mx + n) + c = 0,\\
	a\tan^2(mx + n) + b\tan(mx + n) + c &= 0,\ a\cot^2(mx + n) + b\tan(mx + n) + c = 0,\ 
\end{align*}
trong đó $a,b,c,m,n\in\mathbb{R}$, $a\ne 0$, $m\ne 0$. Tổng quát hơn, giải các phương trình lượng giác sau:
\begin{align*}
	a\sin^2f(x) + b\sin f(x) + c &= 0,\ a\cos^2f(x) + b\cos f(x) + c = 0,\\
	a\tan^2f(x) + b\tan f(x) + c &= 0,\ a\cot^2f(x) + b\cot f(x) + c = 0,\ ,
\end{align*}
trong đó $a,b,c\in\mathbb{R}$, $a\ne 0$, \& $f$ là 1 hàm số (đa thức, phân thức, hàm căn thức) sao cho phương trình $f(x) = m$ có thể giải được (solvable) trên tập số thực $\mathbb{R}$. 

\paragraph{Phương trình bậc $n\in\mathbb{N}$ đối với 1 hàm số lượng giác.} Xét các phương trình lượng giác có dạng $P(\sin x) = 0$, $P(\cos x) = 0$, $P(\tan x) = 0$, $P(\cot x) = 0$ với $P$ là 1 đa thức có bậc $n\in\mathbb{N}$ (i.e., $\deg P = n$), i.e., với $P(x) = \sum_{i=0}^n a_ix^i$, $a_i\in\mathbb{R}$, $i = 0,\ldots,n$, hệ số cao nhất $a_n\ne 0$, xét các phương trình lượng giác có dạng
\begin{align*}
	P(\sin(mx + n)) &= \sum_{i=0}^n a_i\sin^i(mx + n) = 0,\ P(\cos(mx + n)) = \sum_{i=0}^n a_i\cos^i(mx + n)  = 0,\\
	P(\tan(mx + n)) &= \sum_{i=0}^n a_i\tan^i(mx + n)  = 0,\ P(\cot(mx + n)) = \sum_{i=0}^n a_i\cot^i(mx + n) = 0.
\end{align*}
Tổng quát hơn, giải các phương trình lượng giác sau:
\begin{align*}
	P(\sin f(x)) &= \sum_{i=0}^n a_i\sin^if(x) = 0,\ P(\cos f(x)) = \sum_{i=0}^n a_i\cos^if(x) = 0,\\
	P(\tan f(x)) &= \sum_{i=0}^n a_i\tan^if(x) = 0,\ P(\cot f(x)) = \sum_{i=0}^n a_i\cot^if(x) = 0.
\end{align*}
Về đa thức tổng quát bậc $n$ \& các tính chất liên quan, có thể xem các tài liệu chuyên khảo về đa thức hoặc phần đầu của tài liệu của tác giả cho chương trình Toán lớp 8 ở link sau: \href{https://github.com/NQBH/hobby/blob/master/elementary_mathematics/grade_8/NQBH_elementary_mathematics_grade_8.pdf}{GitHub\texttt{/}NQBH\texttt{/}hobby\texttt{/}elementary mathematics\texttt{/}grade 8\texttt{/}lecture}\footnote{Explicitly, \url{https://github.com/NQBH/hobby/blob/master/elementary_mathematics/grade_8/NQBH_elementary_mathematics_grade_8.pdf}.}.

%------------------------------------------------------------------------------%

\chapter{Thống Kê -- Statistics}

See \href{https://www.facebook.com/statsystem}{FaceBook\texttt{/}Statsystem} -- a funny Facebook page, including hilarious mathematical \& statistical jokes, etc.

\section{Mẫu Số Liệu \& Trình Bày Mẫu Số Liệu -- Data Sample \& Representation of Data Sample}

\subsection{Định nghĩa của thống kê -- Definiton of Statistics}
``Những thông tin dưới dạng số liệu rất phổ biến trong khoa học \& đời sống. Khi đọc 1 tờ báo, nghe 1 bản tin trên truyền hình, $\ldots$ chúng ta thường bắt gặp các con số thống kê.

\begin{dinhnghia}[Thống kê]
	\emph{Thống kê} là khoa học về phương pháp thu thập, tổ chức, trình bày, phân tích \& xử lý số liệu.
\end{dinhnghia}
Thống kê giúp ta thu thập, phân tích các số liệu 1 cách khoa học \& rút ra các tri thức, thông tin chứa đựng trong các số liệu đó. Trên cơ sở này, chúng ta mới có thể đưa ra được những dự báo \& những quyết định đúng đắn. Chính vì thế, thống kê đóng 1 vai trò cực kỳ quan trọng, 1 vai trò không thể thiếu trong rất nhiều hoạt động của con người, từ khoa học tự nhiên, kinh tế, nông nghiệp, y học cho tới khoa học xã hội, khoa học quản lý \& hoạch định chính sách. Lenin\footnote{See, e.g., \href{https://vi.wikipedia.org/wiki/Vladimir_Ilyich_Lenin}{Wikipedia\texttt{/}Vladimir Ilyich Lenin} \& \href{https://en.wikipedia.org/wiki/Vladimir_Lenin}{Wikipedia\texttt{/}Vladimir Lenin}.} đã từng ví von rằng thống kê giống như tai, như mắt của Nhà nước; không có thống kê, Nhà nước như người mù \& điếc. Ngay từ đầu thế kỷ 20, nhà khoa học người Anh H. D. Well đã cho rằng: \texttt{[translated]} ``Trong tương lai không xa, kiến thức thống kê \& tư duy thống kê phải trở thành 1 yếu tố không thể thiếu được trong học vấn phổ thông của mỗi công dân, giống như khả năng biết đọc biết viết vậy''.'' -- \cite[p. ]{TL_chuyen_Toan_Dai_So_Giai_Tich_11}

Có thể xem thêm \href{https://en.wikipedia.org/wiki/Statistics}{Wikipedia\texttt{/}Statistics}.

\subsection{Mẫu số liệu -- Data sample}
Xem \cite[Chap. IV: 1 Số Yếu Tố Thống Kê \& Xác Suất, pp. 3--24]{SGK_Toan_6_Canh_Dieu_tap_2} \& tài liệu của tác giả cho chương trình Toán lớp 6 \href{https://github.com/NQBH/hobby/blob/master/elementary_mathematics/grade_6/NQBH_elementary_mathematics_grade_6.pdf}{GitHub\texttt{/}NQBH\texttt{/}hobby\texttt{/}elementary mathematics\texttt{/}grade 6\texttt{/}lecture}.

\begin{dinhnghia}[Mẫu số liệu, kích thước mẫu, số liệu của mẫu]
	\begin{enumerate*}
		\item[(i)] ``1 tập con hữu hạn các đơn vị điều tra được gọi là 1 \emph{mẫu}. Số phần tử của 1 mẫu được gọi là \emph{kích thước mẫu}. Các giá trị của dấu hieuj thu được trên mẫu được gọi là 1 \emph{mẫu số liệu}. Mỗi giá trị trong mẫu số liệu được gọi là 1 \emph{số liệu của mẫu}.
		\item[(ii)] Nếu thực hiện việc điều tra trên mọi đơn vị điều tra thì đó là \emph{điều tra toàn bộ}.
		\item[(iii)] Nếu chỉ điều tra trên 1 mẫu thì đó là điều tra mẫu.'' -- \cite[p. 62]{TL_chuyen_Toan_Dai_So_Giai_Tich_11}
	\end{enumerate*}
\end{dinhnghia}

\begin{luuy}
	``Điều tra toàn bộ nói chung không được thực hiện khi số lượng các đơn vị điều tra quá lớn hoặc khi điều tra thì phải phá hủy đơn vị điều tra. Người ta thường chỉ điều tra mẫu \& dựa trên các thông tin thu được, phân tích, suy diễn để rút ra những kết luận \& dự báo cần thiết liên quan tới toàn bộ đơn vị điều tra.'' -- \cite[p. 62]{TL_chuyen_Toan_Dai_So_Giai_Tich_11}
\end{luuy}

\subsection{Trình bày 1 mẫu số liệu -- Representation of a data sample}

\subsubsection{Bảng phân bố tần số -- tần suất}

\begin{dinhnghia}
	Giả sử trong 1 mẫu số liệu kích thước $N$ có $m$ giá trị khác nhau $x_1 < x_2 < \cdots < x_m$.
	\begin{enumerate*}
		\item[(i)] \emph{Tần số} của giá trị $x_i$ (ký hiệu là $n_i$) là số lần xuất hiện của $x_i$ trong mẫu số liệu.
		\item[(ii)] \emph{Tần suất} của giá trị $x_i$ (ký hiệu là $f_i$) là tỷ số giữa tần số $n_i$ \& kích thước mẫu $N$, $f_i = \frac{n_i}{N}$. Người ta thường viết tần suất dưới dạng phần trăm (\%).
		\item[(iii)] Bảng sau đây được gọi là \emph{bảng phân bố tần số -- tần suất} (gọi tắt là \emph{bảng tần số -- tần suất}):
	\end{enumerate*}
\end{dinhnghia}
\begin{table}[H]
	\centering
	\begin{tabular}{|c|c|c|c|c|c|}
		\hline
		Giá trị & $x_1$ & $x_2$ & $\cdots$ & $x_m$ &  \\
		\hline
		Tần số & $n_1$ & $n_2$ & $\cdots$ & $n_m$ & $N = \sum_{i=1}^m n_i$ \\
		\hline
		Tần suất (\%) & $f_1$ & $f_2$ & $\cdots$ & $f_m$ & \\
		\hline
	\end{tabular}
\end{table}

\begin{luuy}
	``Bảng tần số -- tần suất ở trên có dạng ``ngang'' với 3 dòng \& $m + 2$ cột. Ta có thể trình bày bảng tần số -- tần suất dưới dạng ``dọc'' (chuyển hàng thành cột). Khi đó bảng sẽ có 3 cột \& $m + 2$ dòng.'' -- \cite[p. 63]{TL_chuyen_Toan_Dai_So_Giai_Tich_11}
\end{luuy}

\subsubsection{Bảng phân bố tần số -- tần suất ghép lớp}
``Trong trường hợp ta có mẫu số liệu với kích thước lớn, ta thường thực hiện việc ghép số liệu thành các lớp sao cho mỗi số liệu thuộc vào 1 \& chỉ 1 lớp. Mỗi lớp thường là 1 đoạn hoặc nửa khoảng. Việc phân lớp thế nào là tùy nhu cầu của ta trong mỗi tình huống cụ thể.

\begin{dinhnghia}[Tần số\texttt{/}tần suất ghép lớp]
	Giả sử ta xác định $m$ lớp $C_1,C_2,\ldots,C_m$.
	\begin{enumerate*}
		\item[(i)] \emph{Tần số của lớp $C_i$} (ký hiệu $n_i$) là số số liệu của mẫu nằm trong lớp $C_i$.
		\item[(ii)] \emph{Tần suất của lớp $C_i$} (ký hiệu $f_i$) là tỷ số giữa tần số $n_i$ của lớp $C_i$ \& kích thước mẫu $N$, $f_i = \frac{n_i}{N}$.
	\end{enumerate*}
\end{dinhnghia}
Bảng sau đây được gọi là \textit{bảng phân bố tần số -- tần suất ghép lớp của mẫu số liệu}:

\begin{table}[H]
	\centering
	\begin{tabular}{|c|c|c|}
		\hline
		Lớp & Tần số & Tần suất (\%) \\
		\hline
		$C_1$ & $n_1$ & $f_1$ \\
		\hline
		$C_2$ & $n_2$ & $f_2$ \\
		\hline
		$\vdots$ & $\vdots$ & $\vdots$ \\
		\hline
		$C_m$ & $n_m$ & $f_m$ \\
		\hline
		& $N = \sum_{i=1}^m n_i$ &  \\
		\hline
	\end{tabular}
	\caption{Bảng phân bố tần số -- tần suất ghép lớp của mẫu số liệu.}
\end{table}
Trong nhiều trường hợp, ta ghép lớp theo các nửa khoảng sao cho mút bên phải của 1 nửa khoảng cũng là mút bên trái của nửa khoảng tiếp theo.'' -- \cite[pp. 64--65]{TL_chuyen_Toan_Dai_So_Giai_Tich_11}

\subsubsection{Biểu đồ}
``Tục ngữ có những câu: ``Trăm nghe không bằng 1 thấy''; ``1 hình ảnh có giá trị hơn ngàn lời nói''. Chính vì thế, để trình bày mẫu số liệu 1 cách trực quan sinh động, dễ nhớ \& gây ấn tượng, người ta sử dụng biểu đồ. Sau đây là 1 số biểu đồ thông dụng nhất:
\begin{itemize}
	\item[(i)] \textit{Biểu đồ tần số, tần suất hình cột.} Đây là cách thể hiện rất tốt bảng phân bố tần số -- tần suất. Trên mỗi đoạn (hay nửa khoảng) xác định lớp, ta dựng 1 hình chữ nhật với đáy là đoạn đó (hay nửa khoảng đó) \& chiều cao bằng tần số của lớp. Khi đó ta có biểu đồ tần số hình cột.'' ``Nếu trên mỗi đoạn (hay nửa khoảng) xác định lớp, ta dựng 1 hình chữ nhật với đáy là đoạn đó (hay nửa khoảng đó) \& chiều cao bằng tần suất của lớp thì ta có biểu đồ tần suất hình cột.
	\item[(ii)] \textit{Biểu đồ tần suất hình quạt.} Biểu đồ hình quạt rất thích hợp cho việc thể hiện bảng phân bố tần suất ghép lớp. Hình tròn được chia thành những hình quạt. Mỗi lớp được tương ứng với 1 hình quạt mà diện tích của nó tỷ lệ với tần suất của lớp đó.'' -- \cite[pp. 65--66]{TL_chuyen_Toan_Dai_So_Giai_Tich_11}
\end{itemize}

%------------------------------------------------------------------------------%

\section{Các Số Đặc Trưng của Mẫu Số Liệu}
``Để nhanh chóng nắm bắt được những thông tin quan trọng chứa đựng trong mẫu số liệu, ta đưa ra 1 vài chỉ số gọi là \textit{các số đặc trưng của mẫu số liệu}.'' -- \cite[p. 69]{TL_chuyen_Toan_Dai_So_Giai_Tich_11}

\subsection{Số trung bình}
``Cho mẫu số liệu kích thước $N$: $\{x_i\}_{i=1}^N = \{x_1,\ldots,x_N\}$. \textit{Số trung bình} của mẫu số liệu này, ký hiệu là $\overline{x}$, được tính bởi công thức sau:
\begin{align}
	\overline{x}\coloneqq\frac{\sum_{i=1}^N x_i}{N}.
\end{align}
Nếu mẫu số liệu được cho dưới dạng bảng tần số:

\begin{table}[H]
	\centering
	\begin{tabular}{|c|c|c|c|c|c|}
		\hline
		Giá trị & $x_1$ & $x_2$ & $\cdots$ & $x_m$ &  \\
		\hline
		Tần số & $n_1$ & $n_2$ & $\ldots\ldots$ & $n_m$ & $N = \sum_{i=1}^m n_i$ \\
		\hline
	\end{tabular}
\end{table}
\noindent thì trong tổng $\sum_{i=1}^N x_i$ mỗi giá trị $x_i$ xuất hiện đúng $n_i$ lần. Thành thử công thứ tính số trung bình trở thành
\begin{align*}
	\overline{x}\coloneqq\frac{\sum_{i=1}^m n_ix_i}{N} = \frac{\sum_{i=1}^m n_ix_i}{\sum_{i=1}^m n_i}.
\end{align*}
Nếu mẫu số liệu được cho dưới dạng bảng tần số -- tần suất ghép lớp thì ta thường tính gần đúng số trung bình theo cách sau: Giả sử các số liệu được chia thành $m$ lớp $C_1,C_2,\ldots,C_m$, trong đó mỗi lớp $C_i$ là 1 đoạn $[a_i;b_i]$ hoặc 1 nửa khoảng $[a_i;b_i)$. Ta gọi giá trị $x_i = \frac{a_i + b_i}{2}$ là \textit{giá trị đại diện} của lớp $C_i$. Các gía trị thuộc lớp $C_i$ có thể coi như xấp xỉ bằng $x_i$. Gọi $n_i$ là tần số của lớp $C_i$. Khi đó số trung bình của mẫu số liệu này được tính xấp xỉ theo công thức
\begin{align*}
	\overline{x}\approx\frac{\sum_{i=1}^m n_ix_i}{N} = \frac{\sum_{i=1}^m n_i(a_i + b_i)}{2\sum_{i=1}^m n_i}.
\end{align*}
\textbf{Ý nghĩa của số trung bình.} \textit{Số trung bình của mẫu số liệu dùng làm đại diện cho các giá tri trong mẫu số liệu}.'' -- \cite[pp. 69--70]{TL_chuyen_Toan_Dai_So_Giai_Tich_11}

\subsection{Số trung vị}
``Giả sử ta có 1 mẫu số liệu kích thước $N$. Sắp xếp các số liệu trong mẫu theo thứ tự không giảm $x_1\le x_2\le\cdots\le x_N$. Nếu $N$ là 1 số lẻ thì số liệu $x_{\frac{N+1}{2}}$ (số liệu đứng chính giữa) được gọi là \textit{số trung vị}. Nếu $N$ là 1 số chẵn thì trung bình cộng của $x_{\frac{N}{2}}$ \& $x_{\frac{N}{2} + 1}$, i.e., $\frac{1}{2}\left(x_{\frac{N}{2}} + x_{\frac{N}{2} + 1}\right)$ được gọi là \textit{số trung vị}. Số trung vị được ký hiệu là $M_e$.'' -- \cite[p. 70]{TL_chuyen_Toan_Dai_So_Giai_Tich_11}

\begin{luuy}
	``Khi các số liệu trong mẫu có sự chênh lệch rất lớn với nhau thì số trung bình \& số trung vị cũng khác biệt lớn. Khi đó số trung bình không đại diện tốt cho các số liệu trong mẫu, số trung vị làm đại diện tốt hơn.'' ``Khi các số liệu trong mẫu không có sự chênh lệch rất lớn với nhau thì số trung bình \& số trung vị xấp xỉ nhau.'' -- \cite[p. 71]{TL_chuyen_Toan_Dai_So_Giai_Tich_11}
\end{luuy}
``Nếu mẫu số liệu được cho dưới dạng bảng tần số -- tần suất ghép lớp thì ta thường tính gần đúng số trung vị theo cách sau: Giả sử mẫu số liệu kích thước $N$ được chia làm $m$ lớp, trong đó mỗi lớp là 1 nửa khoảng có độ dài $h$:

\begin{table}[H]
	\centering
	\begin{tabular}{|c|c|}
		\hline
		Lớp & Tần số \\
		\hline
		$[a_1;a_2)$ & $n_1$ \\
		\hline
		$[a_2;a_3)$ & $n_2$ \\
		\hline
		$\vdots$ & $\vdots$ \\
		\hline
		$[a_m;a_{m+1})$ & $n_m$ \\
		\hline
		& $N$ \\
		\hline
	\end{tabular}
\end{table}
ký hiệu $S_l = \sum_{i=1}^l n_i$. Ta xác định số nguyên dương $k$ thỏa mãn bất đẳng thức $S_{k-1} < \frac{N}{2}\le S_k$, i.e., trong dãy $S_1 < S_2 < \cdots < S_m = N$, $k$ là số nguyên dương đầu tiên thỏa mãn $S_k\ge\frac{N}{2}$. Đặt $p\coloneqq\frac{N}{2} - S_{k-1}$. Khi đó số trung vị được tính xấp xỉ bởi công thức $M_e\approx a_k + \frac{hp}{n_k}$.'' -- \cite[pp. 71--72]{TL_chuyen_Toan_Dai_So_Giai_Tich_11}

\subsection{Mốt -- Mode}
``Cho mẫu số liệu dưới dạng bảng phân bố tần số:

\begin{table}[H]
	\centering
	\begin{tabular}{|c|c|c|c|c|c|}
		\hline
		Giá trị & $x_1$ & $x_2$ & $\cdots$ &  $x_m$ &  \\
		\hline
		Tần số & $n_1$ & $n_2$ & $\cdots$ & $n_m$ & $N = \sum_{i=1}^m n_i$ \\
		\hline
	\end{tabular}
	\caption{Bảng phân bố tần số.}
	\label{tab:bang phan bo tan so}
\end{table}

\begin{dinhnghia}[Mốt]
	Giá trị có tần số lớn nhất được gọi là \emph{mốt} của mẫu số liệu \& ký hiệu là $M_0$, i.e., $M_0 = x_k$ nếu $n_k\ge n_i$, $\forall i = 1,\ldots,m$.
\end{dinhnghia}

\begin{luuy}
	Từ đinh nghĩa ta thấy 1 mẫu số liệu có thể có 1 hay nhiều mốt.'' -- \cite[p. 73]{TL_chuyen_Toan_Dai_So_Giai_Tich_11}
\end{luuy}
``Nếu mẫu số liệu được cho dưới dạng bảng tần số -- tần suất ghép lớp thì ta thường tính gần đúng mốt theo cách sau: Giả sử mẫu số liệu kích thước $N$ được chia làm $m$ lớp, trong đó mỗi lớp là 1 đoạn hoặc 1 nửa khoảng. Ta xác định lớp có tần số cao nhất \& giá trị đại diện của lớp có tần số cao nhất này được xem là mốt của bảng phân bố tần số ghép lớp.'' -- \cite[p. 74]{TL_chuyen_Toan_Dai_So_Giai_Tich_11}

\subsection{Phương sai \& độ lệch chuẩn}
``Cho mẫu số liệu kích thước $N$ là $\{x_1,x_2,\ldots,x_N\}$ có $\overline{x}$ là số trung bình. \textit{Phương sai} của mẫu số liệu này, ký hiệu là $s^2$, được tính bởi công thức sau:
\begin{align}
	\label{phuong sai}
	s^2 = \frac{\sum_{i=1}^N \left(x_i - \overline{x}\right)^2}{N}.
\end{align}
Căn bậc 2 của phương sai, ký hiệu là $s$, được gọi là \textit{độ lệch chuẩn}
\begin{align}
	\label{do lech chuan}
	s = \sqrt{\frac{\sum_{i=1}^N \left(x_i - \overline{x}\right)^2}{N}}.
\end{align}

\begin{luuy}
	Phương sai có thể tính theo công thức sau đây:
	\begin{align}
		\label{phuong sai 1}
		s^2 = \frac{1}{N}\sum_{i=1}^N x_i^2 - \frac{1}{N^2}\left(\sum_{i=1}^N x_i\right)^2 = \frac{C}{N} - \frac{B^2}{N^2},\mbox{ với } B = \sum_{i=1}^N x_i,\ C = \sum_{i=1}^N x_i^2.
	\end{align}
	Công thức này thường được sử dụng trong tính toán. Thật vậy, ta có $B = N\overline{x}$. Vậy $\sum_{i=1}^N \left(x_i - \overline{x}\right)^2 = C - 2B\overline{x} + N\overline{x}^2 = C - 2N\overline{x}^2 + N\overline{x}^2 = C - N\overline{x}^2 = C - \frac{B^2}{N}$. Suy ra $s^2 = \frac{C}{N} - \frac{B^2}{N^2}$.'' -- \cite[pp. 74--75]{TL_chuyen_Toan_Dai_So_Giai_Tich_11}
\end{luuy}
``Nếu mẫu số liệu được cho dưới dạng bảng tần số \ref{tab:bang phan bo tan so} thì trong tổng $\sum_{i=1}^N x_i$ \& $\sum_{i=1}^N x_i^2$, mỗi giá trị $x_i$ xuất hiện đúng $n_i$ lần. Thành thử công thức tính phương sai \eqref{phuong sai 1} trở thành:
\begin{align*}
	s^ 2 = \frac{1}{N}\sum_{i=1}^m n_ix_i^2 - \frac{1}{N^2}\left(\sum_{i=1}^m n_ix_i\right)^2.
\end{align*}
Nếu mẫu số liệu được cho dưới dạng bảng tần số -- tần suất ghép lớp thì ta thường tính gần đúng số trung bình theo cách sau: Các số liệu được chia thành $m$ lớp $C_1,C_2,\ldots,C_m$, trong đó mỗi lớp $C_i$ là 1 đoạn $[a_i;b_i]$ hoặc 1 nửa khoảng $[a_i;b_i)$. Ta gọi giá trị $x_i = \frac{a_i + b_i}{2}$ là giá trị đại diện của lớp $C_i$. Các giá trị thuộc lớp $C_i$ có thể coi như xấp xỉ bằng $x_i$. Gọi $n_i$ là tần số của lớp $C_i$. Khi đó phương sai của mẫu số liệu này được tính xấp xỉ theo công thức
\begin{align*}
	s^2\approx\frac{1}{N}\sum_{i=1}^m n_ix_i^2 - \frac{1}{N^2}\left(\sum_{i=1}^m n_ix_i\right)^2.
\end{align*}
\textbf{Ý nghĩa của phương sai \& độ lệch chuẩn.} Từ công thức tính phương sai trong định nghĩa ta thấy phương sai là trung bình cộng của bình phương khoảng cách từ mỗi số liệu tới số trung bình. Như vậy phương sai đo mức độ phân tán của các số liệu trong mẫu quanh số trung bình. Khi số liệu có đơn vị (thứ nguyên) thì đơn vị của phương sai là bình phương đơn vị của số liệu. Độ lệch chuẩn là căn bậc 2 của phương sai, do đó nó cũng là 1 số đo mức độ phân tán của các số liệu trong mẫu quanh số trung bình. Khi số liệu có đơn vị (thứ nguyên) thì độ lệch chuẩn \& số liệu có cùng đơn vị.'' -- \cite[pp. 76--77]{TL_chuyen_Toan_Dai_So_Giai_Tich_11}

%------------------------------------------------------------------------------%

\chapter{Tổ Hợp \& Xác Suất -- Combinatorics \& Probability}

\section{2 Quy Tắc Đếm Cơ Bản}
\textbf{Nội dung.} \textit{2 quy tắc đếm cơ bản} -- [goal] nhờ đó có thể tính chính xác \& nhanh chóng số phần tử của 1 tập hợp mà không cần đếm trực tiếp bằng cách liệt kê.

\subsection{Quy tắc cộng}

\subsubsection{Quy tắc cộng}
``Quy tắc cộng có thể phát biểu dưới dạng sau:
\begin{enumerate*}
	\item[(i)] Giả sử 1 công việc nào đó có thể thực hiện theo phương án $A$ hoặc phương án $B$. Có $n$ cách thực hiện phương án $A$ \& có $m$ cách thực hiện phương án $B$. Khi đó công việc có thể thực hiện theo $n + m$ cách.
	\item[(ii)] 1 cách tổng quát, giả sử 1 công việc nào đó có thể thực hiện theo 1 trong $k$ phương án $A_1,A_2,\ldots,A_k$. Phương án $A_i$ có $n_i$ cách thực hiện ($i = 1,2,\ldots,k$). Khi đó công việc có thể thực hiện theo $\sum_{i=1}^k n_i = n_1 + n_2 + \cdots + n_k$ cách.'' -- \cite[p. 79]{TL_chuyen_Toan_Dai_So_Giai_Tich_11}
\end{enumerate*}

\subsubsection{Tính số phần tử của hợp 2 tập hợp}
``Bản chất toán học của quy tắc cộng (i) là công thức tính số phần tử của hợp 2 tập hợp không giao nhau: \textit{Nếu $A$ \& $B$ là 2 tập hợp hữu hạn không giao nhau thì $|A\cup B| = |A| + |B|$}. 1 cách tổng quát, bản chất toán học của quy tắc cộng (ii) là công thức tính số phần tử của tập $n$ tập hợp hữu hạn đôi một không giao nhau. Quy tắc cộng cho nhiều phần tử đôi một không giao nhau được phát biểu như sau: \textit{Cho $n$ tập hợp $A_1,A_2,\ldots,A_k$ đôi một không giao nhau. Khi đó: $|A_1\cup A_2\cup\cdots\cup A_n| = \sum_{i=1}^k |A_i|$}. Trong nhiều bài toán tổ hợp, chúng ta phải tính số phần tử của hợp 2 tập hợp bất kỳ (có thể không rời nhau).'' -- \cite[pp. 79--80]{TL_chuyen_Toan_Dai_So_Giai_Tich_11}

\begin{dinhly}[Công thức tính số phần tử của hợp 2 tập hợp bất kỳ]
	\label{thm:công thức tính số phần tử của hợp 2 tập hợp bất kỳ}
	Cho $A$ \& $B$ là 2 tập hợp hữu hạn bất kỳ. Khi đó ta có
	\begin{align}
		\label{so phan tu cua hop 2 tap hop bat ky}
		|A\cup B| = |A| + |B| - |A\cap B|.
	\end{align}
\end{dinhly}

\begin{proof}[Chứng minh]
	$B$ \& $A\backslash B$ là 2 tập hợp không giao nhau \& $A\cup B = B\cup(A\backslash B)$ nên $|A\cup B| = |B| + |A\backslash B|$. Mặt khác, $A\cap B$ \& $A\backslash B$ là 2 tập hợp không giao nhau \& $A = (A\cap B)\cup(A\backslash B)$ nên $|A| = |A\cap B| + |A\backslash B|$, do đó $|A\backslash B| = |A| - |A\cap B|$. Kết hợp 2 biểu thức vừa thu được suy ra \eqref{so phan tu cua hop 2 tap hop bat ky}.
\end{proof}

\subsubsection{Tính số phần tử của hợp 3 tập hợp}
Với 3 tập hợp bất kỳ, ta có định lý sau:

\begin{dinhly}[Công thức tính số phần tử của hợp 3 tập hợp bất kỳ]
	Cho $A,B,C$ là 3 tập hợp. Khi đó:
	\begin{align}
		\label{so phan tu cua hop 3 tap hop bat ky}
		|A\cup B\cup C| = |A| + |B| + |C| - |A\cap B| - |B\cap C| - |C\cap A| + |A\cap B\cap C|.
	\end{align}
\end{dinhly}

\begin{proof}[Chứng minh]
	Theo định lý \ref{thm:công thức tính số phần tử của hợp 2 tập hợp bất kỳ} ta có $|A\cup B\cup C| = |A\cup (B\cup C)| = |A| + |B\cup C| - |A\cap(B\cup C)|$. Mặt khác cũng theo Định lý 1, $|B\cup C| = |B| + |C| - |B\cap C|$ \& $|A\cap(B\cup C)| = |(A\cap B)\cup(A\cap C)| = |A\cap B| + |A\cap C| - |(A\cap B)\cap(A\cap C)| = |A\cap B| + |A\cap C| - |A\cap B\cap C|$. Kết hợp 3 biểu thức vừa thu được suy ra \eqref{so phan tu cua hop 3 tap hop bat ky}.
\end{proof}

\subsection{Quy tắc nhân}

\subsubsection{Quy tắc nhân}
\begin{enumerate*}
	\item[(i)] ``Giả sử 1 công việc nào đó bao gồm 2 công đoạn $A$ \& $B$. Công đoạn $A$ có thể làm theo $n$ cách. Với mỗi cách thực hiện công đoạn $A$ thì công đoạn $B$ có thể làm theo $m$ cách. Khi đó công việc có thể thực hiện theo $mn$ cách.
	\item[(ii)] Giả sử 1 công việc nào đó bao gồm $k$ công đoạn $A_1,A_2,\ldots,A_k$. Giả sử rằng công đoạn $A_1$ có thể làm theo $n_1$ cách. Với mỗi $i\ge 2$ \& với mỗi cách thực hiện các công đoạn $A_1,A_2,\ldots,A_{i-1}$ thì công đoạn $A_i$ có thể thực hiện theo $n_i$ cách. Khi đó công việc có thể thực hiện theo $\prod_{i=1}^k n_i = n_1n_2\cdots n_k$ cách.'' -- \cite[p. 82]{TL_chuyen_Toan_Dai_So_Giai_Tich_11}
\end{enumerate*}

\subsubsection{Tính số phần tử của tích Descartes của 2 tập hợp}
``Giả sử công đoạn đầu có thể tiến hành theo $n$ cách: $a_1,a_2,\ldots,a_n$. Công đoạn thứ 2 có thể tiến hành theo $m$ cách $b_1,b_2,\ldots,b_m$. Như vậy nếu công đoạn đầu tiến hành theo cách $a_i$ ($i = 1,2,\ldots,n$), công đoạn thứ 2 tiến hành theo cách $b_j$ ($j = 1,2,\ldots,m$) thì việc thực hiện công việc được mô tả bởi cặp $(a_i,b_j)$. Thành thử tập hợp tất cả các cách thực hiện công việc được mô tả bởi tập hợp tất cả các cặp $\{(a_i,b_j)\}$  ($i = 1,2,\ldots,n$; $j = 1,2,\ldots,m$), i.e., tích Descartes $A\times B$ của 2 tập hợp $A$ \& $B$.'' -- \cite[p. 82]{TL_chuyen_Toan_Dai_So_Giai_Tich_11}. Như vậy bản chất toán học của quy tắc nhân là:

\begin{dinhly}
	Số phần tử của tích Descartes $A\times B$ của 2 tập hợp hữu hạn $A$ \& $B$ bằng số phần tử của $A$ nhân với số phần tử của $B$, $|A\times B| = |A||B|$.
\end{dinhly}

\subsubsection{Tính số phần tử của tích Descartes của nhiều tập hợp}
``1 cách tổng quát, bản chất toán học của quy tắc nhân (ii) cho công việc nhiều công đoạn là công thức tính số phần tử của tích Descartes của nhiều tập hợp.'' -- \cite[p. 82]{TL_chuyen_Toan_Dai_So_Giai_Tich_11}

\begin{dinhly}
	Cho $k$ tập hợp $A_1,A_2,\ldots,A_k$. Tập hợp tất cả các bộ $\{(a_1,a_2,\ldots,a_k)\}$ với $a_i\in A_i$ ($i = 1,2,\ldots,k$) được gọi là \emph{tích Descartes của $k$ tập hợp $A_1,A_2,\ldots,A_k$} \& ký hiệu là $A_1\times A_2\times\cdots\times A_k$. Ta có quy tắc nhân sau đây:
	\begin{align*}
		|A_1\times A_2\times\cdots\times A_k| = \prod_{i=1}^k |A_i| = |A_1||A_2|\cdots|A_k|.
	\end{align*}
\end{dinhly}

\begin{proof}[Chứng minh]
	Sử dụng phương pháp quy nạp \& nhận xét $A_1\times A_2\times\cdots\times A_k\times A_{k+1} = (A_1\times A_2\times\cdots\times A_k)\times A_{k+1}$.
\end{proof}

%------------------------------------------------------------------------------%

\section{Hoán Vị, Chỉnh Hợp \& Tổ Hợp -- Permutation, Arrangement, \& Combination}

\subsection{Hoán vị -- Permutation}

\begin{dinhnghia}[Hoán vị]
	Cho tập hợp $A$ có $n$ phần tử. Mỗi cách sắp xếp $n$ phần tử này theo 1 thứ tự cho ta 1 hoán vị của tập hợp $A$. Số các hoán vị của tập hợp có $n$ phần tử được ký hiệu là $P_n$.
\end{dinhnghia}
Ký hiệu $P$ được lấy từ chữ cái đầu của từ \textit{permutation}\footnote{\textbf{permutation} [n] any of the different ways in which a set or number of things can be ordered or arranged.}, i.e., hoán vị.

\begin{dinhly}[Công thức tính số các hoán vị]
	\begin{align}
		\label{hoan vi}
		P_n = n!\coloneqq\prod_{i=1}^n i = n(n - 1)\cdots 2\cdot 1,\ \forall n\in\mathbb{N}^\star.
	\end{align}
\end{dinhly}

\begin{proof}[Chứng minh]
	``Việc sắp xếp thứ tự $n$ phần tử của tập hợp $A$ có $n$ phần tử là công việc gồm $n$ công đoạn. Công đoạn 1 là chọn phần tử để xếp vào vị trí thứ nhất: có $n$ cách thực hiện. Sau khi thực hiện công đoạn 1, công đoạn 2 là chọn phần tử xếp vào vị trí thứ 2: có $n - 1$ cách thực hiện. Sau khi thực hiện xong $i - 1$ công đoạn (chọn $i - 1$ phần tử của $A$ vào các vị trí thứ $1,2,\ldots,i - 1$), công đoạn thứ $i$ tiếp theo là chọn phần tử xếp vào vị trí thứ $i$: có $n - i + 1$ cách thực hiện. Công đoạn cuối cùng (công đoạn thứ $n$) có 1 cách thực hiện. Theo quy tắc nhân, ta có $n(n - 1)\cdots 2\cdot 1 = n!$ cách sắp xếp thứ tự $n$ phần tử của tập $A$, i.e., có $n!$ hoán vị.'' -- \cite[p. 85]{TL_chuyen_Toan_Dai_So_Giai_Tich_11}
\end{proof}

\subsection{Chỉnh hợp -- Arrangement}

\begin{dinhnghia}[Chỉnh hợp]
	Cho tập hợp $A$ gồm $n$ phần tử \& số nguyên dương $k$ với $1\le k\le n$. Khi lấy ra $k$ phần tử của $A$ \& sắp xếp chúng theo 1 thứ tự ta được 1 \emph{chỉnh hợp chập $k$ của $n$ phần tử của $A$} (gọi tắt là 1 \emph{chỉnh hợp chập $k$ của $A$}). Số các chỉnh hợp chập $k$ của tập hợp có $n$ phần tử được ký hiệu là $A_n^k$.
\end{dinhnghia}

\begin{nhanxet}
	``Từ định nghĩa, ta thấy 1 hoán vị của tập hợp $A$ có $n$ phần tử là 1 chỉnh hợp chập $n$ của $A$.'' -- \cite[p. 85]{TL_chuyen_Toan_Dai_So_Giai_Tich_11}
\end{nhanxet}

\begin{dinhly}[Công thức tính số các chỉnh hợp]
	\label{thm:chinh hop}
	\begin{align}
		\label{chinh hop}
		A_n^k = \prod_{i = n - k + 1}^n i = n(n - 1)\cdots(n - k + 1) = \frac{n!}{(n - k)!} = \frac{P_n}{P_{n-k}},\ \forall n\in\mathbb{N}^\star,\,\forall k\in\mathbb{N},\,k\le n.
	\end{align}
\end{dinhly}

\begin{proof}[Chứng minh]
	``Việc thiết lập 1 chỉnh hợp chập $k$ của tập $A$ có $n$ phần tử là công việc gồm $k$ công đoạn. Công đoạn 1 là chọn phần tử để xếp vào vị trí thứ nhất: có $n$ cách thực hiện. Sau khi thực hiện công đoạn 1, công đoạn 2 là chọn phần tử xếp vào vị trí thứ 2: có $n - 1$ cách thực hiện. Sau khi thực hiện xong $i - 1$ công đoạn (chọn $i - 1$ phần tử của $A$ vào các vị trí thứ $1,2,\ldots,i - 1$), công đoạn thứ $i$ tiếp theo là chọn phần tử xếp vào vị trí thứ $i$: có $n - i + 1$ cách thực hiện. Công đoạn cuối cùng (công đoạn thứ $k$) có $n - k + 1$ cách thực hiện. Theo quy tắc nhân, ta có $\prod_{i = n - k + 1}^n i = n(n - 1)\cdots(n - k + 1)$ cách lập 1 chỉnh hợp chập $k$ của tập $A$ có $n$ phần tử, i.e., có $\prod_{i = n - k + 1}^n i = n(n - 1)\cdots(n - k + 1)$ chỉnh hợp chập $k$ của tập $A$ có $n$ phần tử.'' -- \cite[p. 86]{TL_chuyen_Toan_Dai_So_Giai_Tich_11}
\end{proof}

\subsection{Tổ hợp -- Combination}

\begin{dinhnghia}[Tổ hợp]
	Cho tập hợp $A$ gồm $n$ phần tử \& $k\in\mathbb{N}^\star$ với $1\le k\le n$. Mỗi tập con có $k$ phần tử của $A$ được gọi là 1 \emph{tổ hợp chập $k$ của $n$ phần tử của $A$} (gọi tắt là 1 \emph{tổ hợp chập $k$} của $A$). Số các tổ hợp chập $k$ của tập hợp có $n$ phần tử được ký hiệu là $C_n^k$.
\end{dinhnghia}

\begin{dinhly}[Công thức tính số các tổ hợp]
	\label{thm:to hop}
	\begin{align*}
		C_n^k = \frac{A_n^k}{n!} = \frac{n(n - 1)\cdots(n - k + 1)}{k!} = \frac{n!}{k!(n - k)!} = \frac{P_n}{P_kP_{n-k}},\ \forall n\in\mathbb{N}^\star,\,\forall k\in\mathbb{N},\,k\le n.
	\end{align*}
\end{dinhly}

\begin{proof}[Chứng minh]
	``Từ định nghĩa, ta có mỗi hoán vị của 1 tổ hợp chập $k$ của $A$ cho ta 1 chỉnh hợp chập $k$ của $A$. Do đó, từ 1 tổ hợp chập $k$ của $A$, ta lập được $k!$ chỉnh hợp chập $k$ của $A$. Vậy $A_n^k = k!C_n^k$.'' -- \cite[p. 86]{TL_chuyen_Toan_Dai_So_Giai_Tich_11}
\end{proof}
\textit{Quy ước}: $0! = 1$ \& $C_n^0 = A_n^0 = 1$, $\forall n\in\mathbb{N}^\star$. Với quy ước đó thì định lý \ref{thm:to hop}--\ref{thm:to hop} đúng cho cả $k = 0$ \& $k = n$.

\begin{dinhly}[2 tính chất căn bản của số $C_n^k$]
	\begin{itemize}
		\item[(a)] $C_n^k = C_n^{n-k}$, $\forall n\in\mathbb{N}^\star$, $\forall k\in\mathbb{N}$, $k\le n$.
		\item[(b)] \emph{(Hằng đẳng thức Pascal)}
		\begin{align}
			\label{hang dang thuc Pascal}
			\tag{hdtP}
			C_{n+1}^k = C_n^k + C_n^{k-1},\ \forall n,k\in\mathbb{N}^\star,\,k\le n.
		\end{align}
	\end{itemize}
\end{dinhly}

\begin{proof}[Chứng minh]
	\begin{enumerate*}
		\item[(a)] $C_n^{n-k} = \frac{n!}{(n - k)!(n - (n - k))!} = \frac{n!}{(n - k)!k!} = C_n^k$, $\forall n\in\mathbb{N}^\star$, $\forall k\in\mathbb{N}$, $k\le n$.
		\item[(b)] $C_n^k + C_n^{k-1} = \frac{n!}{k!(n - k)!} + \frac{n!}{(k - 1)!(n - k + 1)!}$ $= \frac{n!(n - k + 1)}{k!(n - k + 1)!} + \frac{n!k}{k!(n - k + 1)!} = \frac{(n + 1)!}{k!(n - k + 1)!} = C_{n+1}^k$, $\forall n,k\in\mathbb{N}^\star$, $k\le n$.
	\end{enumerate*}
\end{proof}

%------------------------------------------------------------------------------%

\section{Nhị Thức Newton -- Newton's Binomial}

\subsection{Công thức nhị thức Newton -- Newton's binomial theorem}
``1 cách tổng quát, khai triển của $(a + b)^n$ được cho bởi công thức sau:

\begin{dinhly}[Công thức nhị thức Newton]
	\begin{align}
		(a + b)^n = \sum_{k=0}^n C_n^k a^{n-k}b^k,\ \forall a,b\in\mathbb{R},\,\forall n\in\mathbb{N}^\star.
	\end{align}
\end{dinhly}
Công thức này được gọi là \textit{công thức nhị thức Newton} (gọi tắt là \textit{nhị thức Newton}).'' See, e.g., \href{https://vi.wikipedia.org/wiki/%C4%90%E1%BB%8Bnh_l%C3%BD_nh%E1%BB%8B_th%E1%BB%A9c}{Wikipedia\texttt{/}định lý nhị thức} \& \href{https://en.wikipedia.org/wiki/Binomial_theorem}{Wikipedia\texttt{/}binomial theorem}.

\begin{proof}[Chứng minh]
	``Trước hết ta chứng minh khẳng định $P(n)$ sau: $(1 + x)^n = \sum_{k=0}^n C_n^kx^k$, $\forall x\in\mathbb{R}$, $\forall n\in\mathbb{N}^\star$. Chứng minh bằng quy nạp theo $n$. Rõ ràng $P(1)$ đúng. Giả sử $P(n)$ đúng. Ta có $(1 + x)^{n+1} = (1 + x)(1 + x)^n = (1 + x)\sum_{k=0}^n C_n^kx^k = \sum_{k=0}^n C_n^kx^k + \sum_{k=0}^n C_n^kx^{k+1}$. Lại có $\sum_{k=0}^n C_n^k x^k = 1 + \sum_{k=1}^n C_n^kx^k$, $\sum_{k=0}^n C_n^kx^{k+1} = \sum_{k=1}^n C_n^{k-1}x^k + x^{n+1}$. Kết hợp 3 đẳng thức vừa thu được \& áp dụng hằng đẳng thức Pascal \eqref{hang dang thuc Pascal}, ta được $(1 + x)^{n+1} = 1 + \sum_{k=1}^n (C_n^k + C_n^{k-1})x^k + x^{n+1} = 1 + \sum_{k=1}^n C_{n+1}^kx^k + x^{n+1} = \sum_{k=0}^{n+1} C_{n+1}^kx^k$. Vậy $P(n+1)$ đúng. Theo nguyên lý quy nạp ta có $P(n)$ đúng, $\forall n\in\mathbb{N}^\star$. Trở lại định lý. Nếu $a = 0$ thì công thức hiển nhiên đúng. Giả sử $a\ne 0$. Đặt $x = \frac{b}{a}$ \& áp dụng $P(n)$ ta có $\left(1 + \frac{b}{a}\right)^n = \sum_{k=0}^n C_n^k\frac{b^k}{a^k}$. Thành thử $(a + b)^n = a^n\left(1 + \frac{b}{a}\right)^n = a^n\sum_{k=0}^n C_n^k\frac{b^k}{a^k} = \sum_{k=0}^n C_n^ka^{n-k}b^k$.'' -- \cite[pp. 90--91]{TL_chuyen_Toan_Dai_So_Giai_Tich_11}
\end{proof}

\begin{luuy}
	Công thức trên là khai triển của $(a + b)^n$ theo lũy thừa giảm của $a$ \& lũy thừa tăng của $b$. Ta cũng có thể viết khai triển của $(a + b)^n$ theo lũy thừa tăng của $a$ \& lũy thừa giảm của $b$. $(a + b) = \sum_{k=0}^n C_n^ka^kb^{n - k}$, $\forall a,b\in\mathbb{R}$, $\forall n\in\mathbb{N}^\star$.
\end{luuy}

\begin{hequa}[Khai triển lũy thừa của hiệu]
	\begin{align*}
		(a - b)^n = \sum_{k=0}^n (-1)^kC_n^ka^{n-k}b^k,\ \forall a,b\in\mathbb{R},\,\forall n\in\mathbb{N}^\star.
	\end{align*}
\end{hequa}
Suy ra
\begin{align*}
	(a - b)^n = a^n - C_n^1a^{n-1}b + \cdots - C_n^{n-1}ab^{n-1} + b^n,\ \forall a,b\in\mathbb{R},\,\forall n\in\mathbb{N}^\star,\,n\ \vdots\ 2,\\
	(a - b)^n = a^n - C_n^1a^{n-1}b + \cdots + C_n^{n-1}ab^{n-1} - b^n,\ \forall a,b\in\mathbb{R},\,\forall n\in\mathbb{N}^\star,\,n\not\vdots\ 2.
\end{align*}
Có thể gộp chung 2 công thức thành:
\begin{equation*}
	(a - b)^n = \left\{\begin{split}
		&a^n - C_n^1a^{n-1}b + \cdots - C_n^{n-1}ab^{n-1} + b^n,&&\mbox{if } n\ \vdots\ 2,\\
		&a^n - C_n^1a^{n-1}b + \cdots + C_n^{n-1}ab^{n-1} - b^n,&&\mbox{if } n\not\vdots\ 2,
	\end{split}\right.\ \forall n\in\mathbb{N}^\star.
\end{equation*}

\begin{baitoan}[\cite{TL_chuyen_Toan_Dai_So_Giai_Tich_11}, Ví dụ 2, p. 92]
	Chứng minh công thức khai triển nhị thức Newton bằng suy luận tổ hợp.
\end{baitoan}

\begin{proof}[Chứng minh]
	``Khi khai triển $(a + b)^n = (a + b)(a + b)\cdots(a + b)$ ($n$ thừa số $(a + b)$) theo quy tắc phân phối của phép nhân, ta được 1 tổng các đơn thức dạng $x_1x_2\cdots x_n$ trong đó $x_i\in\{a,b\}$. Bây giờ ta thực hiện việc ghép các số hạng đồng dạng. Khi tất cả $n$ giá trị $x_i$ bằng $a$, ta nhận được đơn thức $a^n$. Chỉ có 1 đơn thức như vậy. Khi $n - 1$ giá trị $x_i$ bằng $a$ \& giá trị còn lại bằng $b$, ta nhận được đơn thức $a^{n-1}b$. Số các đơn thức như thế bằng số cách chọn 1 số bằng $b$ trong $n$ số $x_1,x_2,\ldots,x_n$, i.e., bằng $C_n^1$. Do đó ta thu được số hạng $C_n^1a^{n-1}b$. Khi $n - 2$ giá trị $x_i$ bằng $a$ \& 2 giá trị còn lại bằng $b$, ta nhận được đơn thức $a^{n-2}b^2$. Số các đơn thức như thế bằng số cách chọn 2 số bằng $b$ trong $n$ số $x_1,x_2,\ldots,x_n$, i.e., bằng $C_n^2$. Do đó ta thu được số hạng $C_n^2a^{n-2}b^2$. Tiếp tục như vậy, khi $n - k$ giá trị $x_i$ bằng $a$ \& $k$ giá trị còn lại bằng $b$, ta nhận được đơn thức $a^{n-k}b^k$. Số các đơn thức như thế bằng số cách chọn $k$ số bằng $b$ trong $n$ số $x_1,\ldots,x_n$, i.e., bằng $C_n^k$. Do đó ta thu được số hạng $C_n^ka^{n-k}b^k$. Cuối cùng, khi tất cả $n$ giá trị bằng $b$, ta được đơn thức $b^n$. Chỉ có 1 đơn thức như vậy. Vậy $(a + b)^n = \sum_{k=0}^n C_n^ka^{n-k}b^k = a^n + C_n^1a^{n-1}b + \cdots + C_n^{n-1}ab^{n-1} + b^n$, $\forall a,b\in\mathbb{R}$, $\forall n\in\mathbb{N}^\star$.'' -- \cite[pp. 92--93]{TL_chuyen_Toan_Dai_So_Giai_Tich_11}
\end{proof}

\subsection{Tam giác Pascal -- Pascal triangle}
``Tam giác Pascal là 1 bảng số được lập theo quy luật sau: Đỉnh của tam giác được ghi số 1. Hàng thứ nhất được ghi 2 số 1. Nếu đã có hàng thứ $n\in\mathbb{N}^\star$ thì hàng thứ $n + 1$ tiếp theo được thiết lập bằng cách cộng 2 số liên tiếp của hàng thứ $n$ rồi viết kết quả xuống hàng dưới ở vị trị giữa 2 số này. Sau đó viết số 1 ở đầu \& cuối hàng.

\begin{dinhly}
	Các số ở hàng thứ $n\in\mathbb{N}^\star$ trong tam giác Pascal là dãy gồm các số $C_n^0,C_n^1,C_n^2,\ldots,C_n^{n-1},C_n^n$.
\end{dinhly}

\begin{proof}[Chứng minh]
	Chứng minh bằng quy nạp. Với $n = 1$, khẳng định đúng. Giả sử khẳng định đúng với $n$. Xét hàng thứ $n + 1$: Giả sử các số ở hàng này là $a_0,a_1,\ldots,a_n,a_{n+1}$. Theo cách xây dựng tam giác, ta có $a_0 = 1 = C_{n+1}^0$, $a_{n+1} = 1 = C_{n+1}^{n+1}$. Xét $1\le k\le n$. Theo cách xây dựng tam giác \& giả thiết quy nạp, ta có $a_k = C_n^k + C_n^{k-1}$. Mặt khác, theo hằng đẳng thức Pascal $C_{n+1}^k = C_n^k + C_n^{k-1}$. Thành thử $a_k = C_{n+1}^k$. Vậy khẳng định đúng với $n + 1$. Theo nguyên lý quy nạp, khẳng định đúng với mọi $n\in\mathbb{N}^\star$.'' -- \cite[pp. 93--94]{TL_chuyen_Toan_Dai_So_Giai_Tich_11}
\end{proof}

%------------------------------------------------------------------------------%

\section{Biến Cố \& Xác Suất của Biến Cố -- Event \& Probability of Event}

\subsection{Phép thử ngẫu nhiên \& không gian mẫu -- Random trial \& sample space}

\subsubsection{Phép thử ngẫu nhiên -- Random trial}

\begin{dinhnghia}[Phép thử ngẫu nhiên]
	``\emph{Phép thử ngẫu nhiên} (gọi tắt là \emph{phép thử}) là 1 thí nghiệm hay 1 hành động mà:
	\begin{enumerate*}
		\item[$\bullet$] Kết quả của nó không dự đoán trước được.
		\item[$\bullet$] Có thể xác định được tập hợp tất cả các kết quả có thể của nó.
	\end{enumerate*}
	Phép thử thường được ký hiệu bởi chữ $T$.'' -- \cite[p. 95]{TL_chuyen_Toan_Dai_So_Giai_Tich_11}
\end{dinhnghia}
Ký hiệu $T$ của phép thử được lấy từ chữ cái đầu của từ \textit{trial}\footnote{\textbf{trial} [n] [countable, uncountable] \textbf{1.} the process of testing something\texttt{/}somebody to find out whether they are effective, successful, suitable, etc.; \textbf{2.} a formal examination of evidence in court by a judge \& often a jury, to decide if somebody accused of a crime is guilty or not; \textbf{trial \& error} [idiom] the process of solving a problem by trying various methods, amounts, etc. until you find one that is successful; [v] \textbf{trial something} to test something\texttt{/}somebody to find out whether they are effective, successful, suitable, etc.}, i.e., thử, phép thử.

\subsubsection{Không gian mẫu -- Sample space}

\begin{dinhnghia}[Không gian mẫu]
	``Tập hợp tất cả các kết quả có thể của phép thử được gọi là \emph{không gian mẫu} của phép thử \& được ký hiệu bởi chữ $\Omega$.'' -- \cite[p. 95]{TL_chuyen_Toan_Dai_So_Giai_Tich_11}
\end{dinhnghia}

\subsection{Biến cố -- Event}

\begin{dinhnghia}[Biến cố]
	\begin{enumerate*}
		\item[(a)] ``1 \emph{biến cố} $A$ (còn gọi là \emph{sự kiện} $A$) liên quan tới phép thử $T$ là biến cố mà việc xảy ra hay không xảy ra của nó tùy thuộc vào kết quả $T$. Mỗi kết quả của phép thử $T$ làm cho biến cố $A$ xảy ra được gọi là 1 \emph{kết quả thuận lợi} cho $A$.
		\item[(b)] Tập hợp các kết quả thuận lợi cho $A$ được ký hiệu bởi $\Omega_A$. Để đơn giản, ta có thể dùng chính chữ $A$ để ký hiệu tập hợp các kết quả thuận lợi cho $A$. Khi đó ta cũng nói biến cố $A$ được mô tả bởi tập $A$.
		\item[(c)] \emph{Biến cố chắc chắn} là biến cố luôn luôn xảy ra khi thực hiện phép thử $T$. Biến cố chắc chắn được mô tả bởi tập $\Omega$ \& được ký hiệu là $\Omega$.
		\item[(d)] \emph{Biến cố không thể} là biến cố không bao giờ xảy ra khi thực hiện phép thử $T$. Biến cố không thể được mô tả bởi tập $\emptyset$ \& được ký hiệu là $\emptyset$.'' -- \cite[pp. 95--96]{TL_chuyen_Toan_Dai_So_Giai_Tich_11}
	\end{enumerate*}
\end{dinhnghia}

\subsection{Xác suất của biến cố -- Probability of event}
``Trong cuộc sống hàng ngày, khi nói về biến cố, ta thường nói biến cố này có nhiều khả năng xảy ra, biến cố kia có ít khả năng xảy ra, biến cố này có nhiều khả năng xảy ra hơn biến cố kia. Toán học đã định lượng hóa các khả năng này bằng cách gán cho mỗi biến cố 1 số $\in[0;1]$ gọi là \textit{xác suất của biến cố} đó. \textit{Xác suất của biến cố $A$} được ký hiệu bởi $P(A)$. Nó đo lường khả năng khách quan xuất hiện biến cố $A$.'' -- \cite[p. 96]{TL_chuyen_Toan_Dai_So_Giai_Tich_11}

\subsubsection{Định nghĩa cố điển của xác suất -- Classical definition of probability}

\begin{dinhnghia}[Xác suất]
	``Giả sử phép thử $T$ có 1 số hữu hạn kết quả có thể đồng khả năng. Khi đó \emph{xác suất của 1 biến cố  $A$} liên quan tới $T$ là tỷ số giữa số kết quả thuận lợi cho $A$ \& số kết quả có thể: $P(A)\coloneqq\frac{|A|}{|\Omega|}$.
\end{dinhnghia}
Như vậy, việc giải 1 bài toán tính xác suất của 1 biến cố $A$ theo định nghĩa cổ điển sẽ được quy về 1 bài toán tổ hợp: đếm số kết quả có thể của $T$ \& đếm số kết quả thuận lợi cho $A$. Cụ thể chúng ta có 3 bước sau:
\begin{enumerate}
	\item Xác định không gian mẫu $\Omega$ rồi tính số phần tử của $\Omega$, i.e., đếm số kết quả có thể của phép thử $T$.
	\item Xác định tập con $A$ mô tả biến cố $A$ rồi tính số phần tử của $A$, i.e., đếm số kết quả thuận lợi cho $A$.
	\item Lấy kết quả của bước 2 chia cho bước 1.
\end{enumerate}
Nói chung việc tính số các kết quả có thể (bước 1) thường dễ dàng hơn nhiều so với việc tính số các kết quả thuận lợi cho $A$ (bước 2). Để giải tốt các bài toán tính xác suất, các bạn phải nắm chắc phần tổ hợp. 1 giả thiết quan trọng khi áp dụng định nghĩa cố điển là các kết quả của phép thử $T$ (i.e., các phần tử của $\Omega$) được coi là \textit{đồng khả năng}. I.e., tất cả các kết quả của $T$ đều có khả năng xuất hiện như nhau, ta không có 1 lý do gì để cho rằng kết quả này lại hay xảy ra hơn kết quả kia. E.g.: Nếu phép thử $T$ liên quan tới gieo súc sắc, tung đồng tiền, ta giả thiết con súc sắc, đồng tiền được chế tạo cân đối, đồng chất thì các kết quả của $T$ là đồng khả năng. Khi nói tới việc ``chọn ngẫu nhiên'' 1 phần tử trong 1 tập hợp nào đó, ta phải hiểu là việc chọn vô tư, không thiên vị, do đó mỗi phần tử đều có khả năng được chọn như nhau.'' -- \cite[pp. 96--97]{TL_chuyen_Toan_Dai_So_Giai_Tich_11}

\subsubsection{Định nghĩa thống kê của xác suất -- Statistical definition of probability}
``Trong định nghĩa cổ điển của xác suất, ta cần giả thiết phép thử $T$ có 1 số hữu hạn kết qủa có thể đồng khả năng. Nếu giả thiết đó bị vi phạm thì định nghĩa đó sẽ được thay bởi định nghĩa sau gọi là định nghĩa thống kê của xác suất.

\begin{dinhnghia}[Xác suất]
	Xét phép thử $T$ \& biến cố $A$ liên quan tới $T$. Ta tiến hành lặp đi lặp lại $N$ lần phép thử $T$. Giả sử trong $N$ lần thực hiện phép thử $T$ đó, biến cố $A$ xuất hiện $k = k(N)$ lần. Người ta chứng minh được rằng khi $N\to+\infty$ thì tỷ số $\frac{k(N)}{N}$ luôn dần tới 1 giới hạn xác định. Giới hạn đó được gọi là \emph{xác suất} của $A$, i.e., $P(A) = \lim\frac{k(N)}{N}$.
\end{dinhnghia}
Trong trường hợp phép thử $T$ có 1 số hữu hạn kết quả có thể đồng khả năng thì xác suất của biến cố $A$ theo định nghĩa thống kê cũng trùng với xác suất của biến cố $A$ theo định nghĩa cổ điển. Tỷ số $\frac{k(N)}{N}$ được gọi là \emph{tần suất} của $A$ trong $N$ lần thực hiện phép thử $T$. Khi $N$ càng lớn thì tần suất càng gần với xác suất. Thành thử tần suất được xem như giá trị gần đúng của xác suất. Số phép thử $N$ càng lớn thì sai số giữa tần suất \& xác suất càng bé.'' -- \cite[pp. 98--99]{TL_chuyen_Toan_Dai_So_Giai_Tich_11}

\subsection{Quy tắc cộng xác suất -- Addition rule of probability}

\subsubsection{Biến cố hợp}

\begin{dinhnghia}[Biến cố hợp]
	``Cho 2 biến cố $A$ \& $B$. Biến cố ``$A$ hoặc $B$ xảy ra'', ký hiệu là $A\cup B$, được gọi là \emph{hợp} của 2 biến cố $A$ \& $B$. 1 cách tổng quát, cho $k$ biến cố $A_1,A_2,\ldots,A_k$. Biến cố: ``Có ít nhất 1 trong các biến cố $A_1,A_2,\ldots,A_k$ xảy ra'', ký hiệu là $A_1\cup A_2\cup\cdots\cup A_k$, được gọi là \emph{hợp} của $k$ biến cố đó.'' -- \cite[p. 99]{TL_chuyen_Toan_Dai_So_Giai_Tich_11}
\end{dinhnghia}

\subsubsection{Biến cố xung khắc}

\begin{dinhnghia}[Biến cố xung khắc]
	``2 biến cố $A$ \& $B$ được gọi là \emph{xung khắc} với nhau nếu biến cố này xảy ra thì biến cố kia không xảy ra.'' -- \cite[p. 99]{TL_chuyen_Toan_Dai_So_Giai_Tich_11}
\end{dinhnghia}

\subsubsection{Biến cố đối -- Addition rule of probability}

\begin{dinhnghia}[Biến cố đối]
	``Cho $A$ là 1 biến cố. Khi đó biến cố ``Không xảy ra $A$'' được gọi là \emph{biến cố đối} của $A$. Biến cố đối của $A$ ký hiệu là $\overline{A}$.'' -- \cite[p. 99]{TL_chuyen_Toan_Dai_So_Giai_Tich_11}
\end{dinhnghia}
``Rõ ràng $A$ \& $\overline{A}$ là 2 biến cố xung khắc \& hợp của chúng là 1 biến cố chắc chắn: $\Omega = A\cup\overline{A}$.'' -- \cite[p. 100]{TL_chuyen_Toan_Dai_So_Giai_Tich_11}

\subsubsection{Quy tắc cộng xác suất}

\begin{dinhly}[Quy tắc cộng xác suất]
	\begin{itemize}
		\item[(a)] Nếu 2 biến cố $A,B$ xung khắc với nhau thì $P(A\cup B) = P(A) + P(B)$.
		\item[(b)] Nếu $A_1,A_2,\ldots,A_k$ là $k$ biến cố đôi một xung khắc với nhau thì $P\left(A_1\cup A_2\cup\cdots\cup A_k\right) = \sum_{i=1}^k P(A_i) = P(A_1) + P(A_2) + \cdots + P(A_k)$, i.e.,
		\begin{align*}
			P\left(\bigcup_{i=1}^k A_i\right) = \sum_{i=1}^k P(A_i).
		\end{align*}
	\end{itemize}	
\end{dinhly}

\begin{proof}[Chứng minh]
	\begin{enumerate*}
		\item[(a)] Sử dụng công thức $|A\cup B| = |A| + |B| - |A\cap B|$ với mọi tập $A,B$, \& nếu có thêm $A\cap B = \empty$, i.e., $A$ \& $B$ xung khắc, thì $|A\cup B| = |A| + |B|$. Chia 2 vế cho $|\Omega|$, thu được $P(A\cup B) = P(A) + P(B)$.
		\item[(b)] Có thể chứng minh bằng quy nạp dựa vào bước cơ sở $k = 2$ chính là (a) hoặc sử dụng trực tiếp công thức tương ứng ở ngôn ngữ tập hợp: $|\bigcup_{i=1}^k A_i| = \sum_{i=1}^k |A_i|$, với mọi tập hợp $A_i$, $i = 1,\ldots,k$, đôi một không giao nhau.
	\end{enumerate*}
\end{proof}

\subsubsection{Công thức tính xác suất biến cố đối}
``Xác suất của biến cố đối $\overline{A}$ của biến cố $A$ là $P(\overline{A}) = 1 - P(A)$. Thật vậy, vì $P(\Omega) = 1$; $A$ \& $\overline{A}$ là 2 biến cố xung khắc \& $\Omega = A\cup\overline{A}$ nên từ quy tắc cộng suy ra $1 = P(A) + P(\overline{A})$.'' -- \cite[p. 100]{TL_chuyen_Toan_Dai_So_Giai_Tich_11}

\subsection{Quy tắc nhân xác suất -- Multiplication rule of probability}

\subsubsection{Biến cố giao}

\begin{dinhnghia}
	``Cho 2 biến cố $A$ \& $B$. Biến cố ``Cả $A$ \& $B$ đều xảy ra'' ký hiệu là $AB$ được gọi là \emph{giao} của 2 biến cố $A$ \& $B$. 1 cách tổng quát, cho $k$ biến cố $A_1,A_2,\ldots,A_k$. Biến cố: ``Tất cả $k$ biến số $A_1,A_2,\ldots,A_k$ đều xảy ra'', ký hiệu là $A_1A_2\ldots A_k$, được gọi là \emph{giao} của $k$ biến cố đó.'' -- \cite[p. 101]{TL_chuyen_Toan_Dai_So_Giai_Tich_11}
\end{dinhnghia}

\subsubsection{Biến cố độc lập}

\begin{dinhnghia}[Biến cố độc lập]
	``2 biến cố $A$ \& $B$ được gọi là \emph{độc lập với nhau} nếu việc xảy ra hay không xảy ra của biến cố này không làm ảnh hưởng tới xác suất xảy ra của biến cố kia. 1 cách tổng quát, cho $k$ biến cố $A_1,A_2,\ldots,A_k$. Chúng được goị là \emph{độc lập với nhau} nếu việc xảy ra hay không xảy ra của 1 nhóm bất kỳ trong các biến cố trên không làm ảnh hưởng tới xác suất xảy ra của các biến cố còn lại.'' -- \cite[p. 101]{TL_chuyen_Toan_Dai_So_Giai_Tich_11}
\end{dinhnghia}

\subsubsection{Quy tắc nhân xác suất}

\begin{dinhly}[Quy tắc nhân xác suất]
	Nếu $A,B$ là 2 biến cố độc lập thì $P(AB) = P(A)P(B)$. 1 cách tổng quát, nếu $k$ biến cố $A_1,A_2,\ldots,A_k$ là độc lập thì $P(A_1A_2\ldots A_k) = \prod_{i=1}^k P(A_i) = P(A_1)P(A_2)\cdots P(A_k)$.
\end{dinhly}

\begin{proof}[Chứng minh]
	***
\end{proof}


%------------------------------------------------------------------------------%

\section{Các Quy Tắc Tính Xác Suất -- Rules of Probability}

%------------------------------------------------------------------------------%

\section{Biến Ngẫu Nhiên Rời Rạc -- Discrete Random Variable}

%------------------------------------------------------------------------------%

\chapter{Dãy Số. Cấp Số Cộng \& Cấp Số Nhân -- Series. Arithmetic Progression\texttt{/}Sequence \& Geometric Progression\texttt{/}Sequence}

\section{Phương Pháp Quy Nạp Toán Học}

%------------------------------------------------------------------------------%

\section{Dãy Số}

%------------------------------------------------------------------------------%

\section{Cấp Số Cộng}

%------------------------------------------------------------------------------%

\section{Cấp Số Nhân}

%------------------------------------------------------------------------------%

\chapter{Giới Hạn -- Limit}

\section{Dãy Số Có Giới Hạn $0$}

%------------------------------------------------------------------------------%

\section{Dãy Số Có Giới Hạn Hữu Hạn}

%------------------------------------------------------------------------------%

\section{Dãy Số Có Giới Hạn Vô Cực}

%------------------------------------------------------------------------------%

\section{Định Nghĩa \& 1 Số Định Lý về Giới Hạn của Hàm Số}

%------------------------------------------------------------------------------%

\section{Giới Hạn 1 Bên}

%------------------------------------------------------------------------------%

\section{1 Vài Quy Tắc Tìm Giới Hạn Vô Cực}

%------------------------------------------------------------------------------%

\section{Các Dạng Vô Hình}

%------------------------------------------------------------------------------%

\section{Hàm Số Liên Tục}

%------------------------------------------------------------------------------%

\chapter{Đạo Hàm -- Derivative}

\section{Khái Niệm Đạo Hàm}

%------------------------------------------------------------------------------%

\section{Các Quy Tắc Tính Đạo Hàm}

%------------------------------------------------------------------------------%

\section{Đạo Hàm của Các Hàm Số Lượng Giác}

%------------------------------------------------------------------------------%

\section{Vi Phân}

%------------------------------------------------------------------------------%

\section{Đạo Hàm Cấp Cao}

%------------------------------------------------------------------------------%

\part{Hình Học -- Geometry}

\chapter{Phép Dời Hình \& Phép Đồng Dạng Trong Mặt Phẳng}

``Bức tranh của họa sĩ Hà Lan M.C. Escher gồm những hình bằng nhau mô tả các chiến binh trên lưng ngựa. Các hình này phủ kín mặt phẳng. 2 chiến binh \& ngựa cùng màu (trắng hoặc đen) tương ứng với nhau qua 1 phép tịnh tiến. 2 chiến binh \& ngựa khác màu thì tương ứng với nhau qua 1 phép đối xứng trục \& tiếp theo là 1 phép tịnh tiến. Nghệ thuật dùng những hình bằng nhau để lấp đầy mặt phẳng được phát triển mạnh mẽ vào thế kỷ XIII ở nước Ý\texttt{/}Italia.'' -- \cite[p. 3]{SGK_Toan_11_hinh_hoc_nang_cao}

\begin{quotation}
	\textbf{Nội dung.} \textit{Các phép dời hình \& đồng dạng trong mặt phẳng: phép tịnh tiến, phép đối xứng trục, phép quay, phép vị tự, $\ldots$; 2 hình bằng nhau, 2 hình đồng dạng 1 cách tổng quát}.
\end{quotation}

\section{Mở Đầu về Phép Biến Hình}

\subsection{Phép biến hình}
Khái niệm ``hàm số'' -- 1 khái niệm quan trọng trong Đại số: ``Nếu có 1 quy tắc để với mỗi số $x\in\mathbb{R}$, xác định được 1 số duy nhất $y\in\mathbb{R}$ thì quy tắc đó gọi là \textit{1 hàm số xác định trên tập số thực $\mathbb{R}$}. Bây giờ, trong mệnh đề trên ta thay \textit{số thực} bằng \textit{điểm thuộc mặt phẳng} thì ta được khái niệm về phép biến hình trong mặt phẳng. Cụ thể là: Nếu có 1 quy tắc để với mỗi điểm $M$ thuộc mặt phẳng, xác định được 1 điểm duy nhất $M'$ thuộc mặt phẳng ấy thì quy tắc đó gọi là \textit{1 phép biến hình (trong mặt phẳng)}.'' -- \cite[p. 4]{SGK_Toan_11_hinh_hoc_nang_cao}

\begin{dinhnghia}[Phép biến hình]
	\begin{itemize}
		\item ``Quy tắc đặt tương ứng mỗi điểm $M$ của mặt phẳng với 1 điểm xác định duy nhất $M'$ của mặt phẳng đó được gọi là \emph{phép biến hình} trong mặt phẳng.'' ``Nếu ký hiệu phép biến hình là $F$ thì ta viết $F(M) = M'$ hay $M' = F(M)$ \& gọi điểm $M'$ là \emph{ảnh} của điểm $M$ qua phép biến hình $F$.'' -- \cite[p. 4]{SGK_Toan_11_hinh_hoc_co_ban}
		\item ``\emph{Phép biến hình} (trong mặt phẳng) là 1 quy tắc để với mỗi điểm $M$ thuộc mặt phẳng, xác định được 1 điểm duy nhất $M'$ thuộc mặt phẳng ấy. Điểm $M'$ gọi là \emph{ảnh} của điểm $M$ qua phép biến hình đó.'' -- \cite[p. 4]{SGK_Toan_11_hinh_hoc_nang_cao}
	\end{itemize}	
\end{dinhnghia}

\begin{vidu}[Phép chiếu vuông góc lên 1 đường thẳng]
	``Cho đường thẳng $d$. Với mỗi điểm $M$, ta xác định $M'$ là hình chiếu (vuông góc) của $M$ trên $d$ thì ta được 1 phép biến hình.
	
	\begin{figure}[H]
		\centering
		\includegraphics[scale=0.15]{phep_chieu_vuong_goc}
		\caption{Phép chiếu vuông góc lên đường thẳng $d$.}
		\label{fig:phep chieu vuong goc len duong thang}
	\end{figure}
	Phép biến hình này gọi là \emph{phép chiếu (vuông góc) lên đường thẳng $d$}.'' -- \cite[p. 4]{SGK_Toan_11_hinh_hoc_nang_cao}
\end{vidu}

\begin{vidu}[Phép tịnh tiến theo vector]
	``Cho vector $\vec{u}$, với mỗi điểm $M$ ta xác định điểm $M'$ theo quy tắc $\overrightarrow{MM'} = \vec{u}$ (Fig. \ref{fig:phep tinh tien theo vector}). Như vậy ta cũng có 1 phép biến hình. Phép biến hình đó gọi là \emph{phép tịnh tiến theo vector $\vec{u}$}.'' -- \cite[p. 4]{SGK_Toan_11_hinh_hoc_nang_cao}
	
	\begin{figure}[H]
		\centering
		\includegraphics[scale=0.15]{phep_tinh_tien_theo_vector}
		\caption{Phép tịnh tiến theo vector $\vec{u}$.}
		\label{fig:phep tinh tien theo vector}
	\end{figure}	
\end{vidu}

\begin{vidu}[Phép đồng nhất]
	``Với mỗi điểm $M$, ta xác định điểm $M'$ trùng với $M$ thì ta cũng được 1 phép biến hình.'' -- \cite[p. 5]{SGK_Toan_11_hinh_hoc_nang_cao}
\end{vidu}

\begin{dinhnghia}[Phép đồng nhất]
	``Phép biến hình biến mỗi điểm $M$ thành chính nó được gọi là \emph{phép đồng nhất}.'' -- \cite[p. 4]{SGK_Toan_11_hinh_hoc_co_ban}. Phép đồng nhất thường được ký hiệu là $\operatorname{id}$ (identity mapping), $\operatorname{id}(M) = M$, $\forall M\in\mathbb{R}^2$, \& $\operatorname{id}(\mathcal{H}) = \mathcal{H}$, với mọi hình $\mathcal{H}\subset\mathbb{R}^2$.
\end{dinhnghia}

\begin{vidu}
	Cho trước số dương $a\in(0;+\infty)$, với mỗi điểm $M$ trong mặt phẳng, gọi $M'$ là 1 điểm sao cho $MM' = a$. Khi đó tập hợp các điểm $M'$ thỏa mãn điều kiện này là đường tròn tâm $M$ bán kính $a$, i.e., $\{M'\in\mathbb{R}^2|MM' = a\} = \operatorname{circle}(M;a)$, là 1 tập có vô hạn không đếm được các phần tử, thậm chí lực lượng\emph{\texttt{/}}cardinality\footnote{See, e.g., \href{https://en.wikipedia.org/wiki/Cardinality}{Wikipedia\texttt{/}cardinality}.} của 1 hình tròn với bán kính là 1 số dương bất kỳ bằng lực lượng của $\mathbb{R}$ \& bằng $\mathfrak{c}$ (cardinality of the continuum\footnote{See, e.g., \href{https://en.wikipedia.org/wiki/Cardinality_of_the_continuum}{Wikipedia\texttt{/}cardinality of the continuum}.}). Quy tắc này hiển nhiên không là 1 phép biến hình do vi phạm yêu cầu về tính xác định duy nhất của ảnh.
\end{vidu}
Về lực lượng \& các tính chất sâu sắc hơn của tập hợp, có thể tham khảo \cite{Halmos1960, Halmos1974, Kaplansky1972, Kaplansky1977}\footnote{Đây là những quyển sách đầu tiên tác giả đọc khi bắt đầu học Toán Cao Cấp ở bậc Đại học}.

\subsection{Ký hiệu \& thuật ngữ}
\begin{itemize}
	\item ``Nếu $\mathcal{H}$ là 1 hình nào đó trong mặt phẳng thì ta ký hiệu $\mathcal{H}'\coloneqq F(\mathcal{H})$ là tập các điểm $M' = F(M)$, $\forall M\in\mathcal{H}$. Khi đó ta nói $F$ \textit{biến hình $\mathcal{H}$ thành hình $\mathcal{H}'$}, hay \textit{hình $\mathcal{H}'$ là ảnh của hình $\mathcal{H}$} qua phép biến hình $F$.
	'' -- \cite[p. 4]{SGK_Toan_11_hinh_hoc_co_ban}
	\item ``Nếu ta ký hiệu 1 phép biến hình nào đó là $F$ \& điểm $M'$ là ảnh của điểm $M$ qua phép biến hình $F$ thì ta viết $M' = F(M)$, hoặc $F(M) = M'$. Khi đó, ta còn nói \textit{phép biến hình $F$ biến điểm $M$ thành điểm $M'$}. Với mỗi hình $\mathcal{H}$, ta gọi hình $\mathcal{H}'$ gồm các điểm $M' = F(M)$, trong đó $M\in\mathcal{H}$, là \textit{ảnh của $\mathcal{H}$ qua phép biến hình $F$}, \& viết $\mathcal{H}' = F(\mathcal{H})$.'' -- \cite[p. 5]{SGK_Toan_11_hinh_hoc_nang_cao}, i.e.,
	\begin{align*}
		\mathcal{H}'\coloneqq\{M'\in\mathbb{R}^2|\exists M\in\mathcal{H},\ M' = F(M)\} = \{F(M)|M\in\mathcal{H}\} = F(\mathcal{H}).
	\end{align*}
\end{itemize}

%------------------------------------------------------------------------------%

\section{Phép Tịnh Tiến}

\subsection{Định nghĩa phép tịnh tiến}

\begin{dinhnghia}[Phép tịnh tiến]
	``\emph{Phép tịnh tiến} theo vector $\vec{u}$ là 1 phép biến hình biến điểm $M$ thành điểm $M'$ sao cho $\overrightarrow{MM'} = \vec{u}$.'' -- \cite[p. 5]{SGK_Toan_11_hinh_hoc_nang_cao}
\end{dinhnghia}
``Phép tịnh tiến theo vector $\vec{u}$ thường được ký hiệu là $T$ hoặc $T_{\vec{u}}$. Vector $\vec{u}$ được gọi là \textit{vector tịnh tiến}.'' -- \cite[p. 5]{SGK_Toan_11_hinh_hoc_nang_cao}.

``Như vậy, $T_{\vec{u}}(M) = M'\Leftrightarrow\overrightarrow{MM'} = \vec{u}$. Phép tịnh tiến theo vector $\vec{0}$ chính là \textit{phép đồng nhất}.'' -- \cite[p. 5]{SGK_Toan_11_hinh_hoc_co_ban}

\begin{baitoan}[\cite{SGK_Toan_11_dai_so_giai_tich_co_ban}, \textbf{1.}, p. 7]
	\label{prob:phep tinh tien}
	Chứng minh: $M' = T_{\vec{u}}(M)\Leftrightarrow M = T_{-\vec{u}}(M')$.
\end{baitoan}

\subsection{Các tính chất của phép tịnh tiến}

\begin{dinhly}[Phép tịnh tiến bảo toàn khoảng cách]
	\label{thm:phep tinh tien bao toan khoang cach}
	Nếu phép tịnh tiến biến 2 điểm $M$ \& $N$ lần lượt thành 2 điểm $M'$ \& $N'$ thì $M'N' = MN$.
\end{dinhly}
``Người ta diễn tả tính chất trên của phép tịnh tiến là: \textit{Phép tịnh tiến không làm thay đổi khoảng cách giữa 2 điểm bất kỳ}.'' -- \cite[p. 6]{SGK_Toan_11_hinh_hoc_nang_cao}. ``Nếu $T_{\vec{u}}(M) = M'$, $T_{\vec{u}}(N) = N'$ thì $\overrightarrow{M'N'} = \overrightarrow{MN}$ \& từ đó suy ra $M'N' = MN$.'' ``Nói cách khác, phép tịnh tiến bảo toàn khoảng cách giữa 2 điểm bất kỳ.'' -- \cite[p. 6]{SGK_Toan_11_hinh_hoc_co_ban}, i.e.,
\begin{align}
	\label{phep tinh tien bao toan khoang cach}
	((T_{\vec{u}}(M) = M')\land(T_{\vec{u}}(N) = N'))\Rightarrow\overrightarrow{M'N'} = \overrightarrow{MN}\Rightarrow M'N' = MN.
\end{align}

\begin{proof}[Chứng minh \eqref{phep tinh tien bao toan khoang cach}]
	Vì $\overrightarrow{MM'} = \overrightarrow{NN'} = \vec{u}$ \& $\overrightarrow{M'M} = -\vec{u}$, $\overrightarrow{M'N'} = \overrightarrow{M'M} + \overrightarrow{MN} + \overrightarrow{NN'} = -\vec{u} + \overrightarrow{MN} + \vec{u} = \overrightarrow{MN}$. Suy ra $|\overrightarrow{M'N'}| = |\overrightarrow{MN}|$, i.e., $M'N' = MN$.
\end{proof}

\begin{dinhly}[Phép tịnh tiến bảo toàn tính chất thẳng hàng \& thứ tự các điểm thẳng hàng]
	Phép tịnh tiến biến 3 điểm thẳng hàng thành 3 điểm thẳng hàng \& không làm thay đổi thứ tự 3 điểm đó.
\end{dinhly}

\begin{proof}[Chứng minh]
	``Giả sử phép tịnh tiến biến 3 điểm $A,B,C$ thành 3 điểm $A',B',C'$. Theo Định lý \ref{thm: phep tinh tien bao toan khoang cach}, ta có $A'B' = AB$, $B'C' = BC$, \& $A'C' = AC$. Nếu $A,B,C$ thẳng hàng, $B$ nằm giữa $A$ \& $C$ thì $AB + AC = AC$. Do đó ta cùng có $A'B' + B'C' = A'C'$, i.e., $A',B',C'$ thẳng hàng, trong đó $B'$ nằm giữa $A'$ \& $C'$.
\end{proof}
Từ định lý \ref{thm: phep tinh tien bao toan khoang cach}, suy ra:

\begin{hequa}
	\label{cor:phep tinh tien}
	Phép tịnh tiến biến đường thẳng thành đường thẳng song song hoặc trùng với nó, biến tia thành tia song song cùng hướng hoặc trùng với nó, biến đoạn thẳng thành đoạn thẳng bằng nó, biến tam giác thành tam giác bằng nó, biến đường tròn thành đường tròn có cùng bán kính, biến góc thành góc bằng nó.
\end{hequa}

\subsection{Biểu thức tọa độ của phép tịnh tiến}
``Trong mặt phẳng với hệ trục tọa độ $Oxy$, cho phép tịnh tiến theo vector $\vec{u}$. Biết tọa độ của $\vec{u}$ là $(a;b)$. Giả sử điểm $M(x;y)$ biến thành điểm $M(x';y')$ (Fig. \ref{fig:phep tinh tien tren he truc toa do}).

\begin{figure}[H]
	\centering
	\includegraphics[scale=0.15]{phep_tinh_tien_tren_he_truc_toa_do}
	\caption{Phép tịnh tiến trên hệ trục tọa độ.}
	\label{fig:phep tinh tien tren he truc toa do}
\end{figure}
Khi đó ta có:
\begin{equation*}
	\boxed{\left\{\begin{split}
		x' &= x + a,\\
		y' &= y + b.
	\end{split}\right.}
\end{equation*}
Công thức trên gọi là \textit{biểu thức tọa độ của phép tịnh tiến $T_{\vec{u}}$ theo vector $\vec{u}(a;b)$}.'' -- \cite[pp. 6--7]{SGK_Toan_11_hinh_hoc_nang_cao}

\begin{baitoan}
	Trong mặt phẳng tọa độ $Oxy$ cho vector $\vec{u} = (u_1;u_2)$, 2 điểm $A(a_1;a_2)\not\equiv B(b_1,b_2)$ (i.e., phân biệt\emph{\texttt{/}}không trùng nhau) \& đường thẳng $d$ có phương trình $ax + by + c = 0$ với $a,b\in\mathbb{R}$ sao cho $a^2 + b^2\ne 0$.
	\begin{enumerate*}
		\item[(a)] Tìm tọa độ của các điểm $A',B'$ theo thứ tự là ảnh của $A,B$ qua phép tịnh tiến theo $\vec{u}$.
		\item[(b)] Tìm tọa độ của điểm $C$ sao cho $A$ là ảnh của $C$ qua phép tịnh tiến theo $\vec{u}$.
		\item[(c)] Tìm phương trình của đường thẳng $d'$  là ảnh của $d$ qua phép tịnh tiến theo $\vec{u}$.
	\end{enumerate*}
\end{baitoan}

\begin{proof}[Giải]
	\begin{enumerate*}
		\item[(a)] $A' = T_{\vec{u}}(A)$, $B' = T_{\vec{u}}(B)$, sử dụng biểu thức tọa độ của phép tịnh tiến $T_{\vec{u}}$, thu được $A'(a_1 + u_1;a_2 + u_2)$, $B'(b_1 + u_1;b_2 + u_2)$.
		\item[(b)] Giả sử $C(c_1;c_2)$, $A = T_{\vec{u}}(C)$,\footnote{Sử dụng bài toán \ref{prob:phep tinh tien} cho ta $C = T_{-\vec{u}}(A)$, có thể sử dụng biểu thức tọa độ của $T_{-\vec{u}}$ để thu được trực tiếp $c_i = a_i - u_i$, $i = 1,2$.} sử dụng biểu thức tọa độ của $T_{\vec{u}}$, thu được $a_i = c_i + u_i$, $i = 1,2$, suy ra $c_i = a_i - u_i$, $i = 1,2$, hay $C(a_1 - u_1,a_2 - u_2)$.
		\item[(c)] Nếu $\vec{u} = \vec{0}$ thì $d\equiv d'$ \& có chung phương trình $ax + by + c = 0$. Nếu $\vec{u}\ne\vec{0}$. Giả sử đường thẳng $d'$ có phương trình $a'x + b'y + c' = 0$ với $a',b'\in\mathbb{R}$ với $(a')^2 + (b')^2\ne 0$. Sử dụng hệ quả \ref{cor:phep tinh tien} thu được $d'\equiv d$ hoặc $d'\parallel d$, nên $\frac{a}{a'} = \frac{b}{b'} = \frac{c}{c'} = k = \mbox{const}$ với quy ước nếu mẫu số bằng $0$ thì tử bằng $0$. W.l.o.g., có thể giả sử $a' = a$, $b' = b$ (chia các hệ số của $d'$ cho hằng số $k$ trong dãy tỷ số bằng nhau vừa thu được), ta cần tìm $c'$ bằng cách xác định 1 điểm thuộc $d'$. Vì $a^2 + b^2\ne 0$, 1 trong số $a,b$ phải khác $0$. W.l.o.g., giả sử $a\ne 0$, thì điểm $D_1\left(-\frac{c}{a};0\right)\in d$, khi đó ảnh của $D_1$ qua $T_{\vec{u}}$ là $D_1'\coloneqq T_{\vec{u}}(D_1) = \left(-\frac{c}{a} + u_1;u_2\right)\in d'$, i.e., $a\left(-\frac{c}{a} + u_1\right) + bu_2 + c' = 0$, hay $c' = c - au_1 - bu_2$. (Nếu $b\ne 0$, thì có thể lấy điểm $D_2\left(0;-\frac{c}{b}\right)$, khi đó $D_2'\coloneqq T_{\vec{u}}(D_2) = \left(u_1;-\frac{c}{b} + u_2\right)\in d'$, i.e., $au_1 + b\left(-\frac{c}{b} + u_2\right) + c' = 0$, cũng cho ta $c' = c - au_1 - bu_2$.). Vậy phương trình đường thẳng $d' = T_{\vec{u}}(d)$ là $ax + by + c - au_1 - bu_2 = 0$.
	\end{enumerate*}
\end{proof}

\begin{nhanxet}
	Vài nhận xét về lời giải trên:
	\begin{enumerate}
		\item Nếu $\vec{u} = \vec{0}$, i.e., $u_1 = u_2 = 0$, phương trình của $d'$ thu được ở (c) trùng với phương trình của $d$ như đã lý luận ở đầu lời giải của (c) (consistency).
		\item Nếu vector tịnh tiến $\vec{u}$  có tọa độ trùng với hệ số của $d$, i.e., $u_1 = a$, $u_2 = b$, $\vec{u}(a;b)$, thì $d\not\equiv d'$. Thật vậy, phương trình của $d'$ thu được ở (c) trở thành: $ax + by + c - a^2 - b^2 = 0$, \& vì $a^2 + b^2\ne 0$ (ít nhất 1 trong 2 số phải khác $0$) nên $c - a^2 - b^2\ne c$, nên $d'\parallel d$ nhưng $d\not\equiv d'$ trong trường hợp này.
		\item 1 ý tưởng giải khác cho (c) là tìm 2 điểm phân biệt thuộc $d$ (e.g., 2 điểm $\left(0;-\frac{c}{b}\right)$, $\left(-\frac{c}{a};0\right)$ nếu $ab\ne 0$; còn nếu $ab = 0$ thì phải tìm thêm 1 điểm khác), tìm 2 ảnh của 2 điểm đó qua $T_{\vec{u}}$ như (a), (b), rồi viết phương trình đường thẳng $d'$ đi qua 2 điểm ảnh vừa tìm được. Cách này cũng sẽ cho cùng kết quả với lời giải (c) ở trên nhưng không hay \&\emph{\texttt{/}}vì sẽ tốn nhiều công sức tính toán hơn, bởi vì không tận dụng trực tiếp tính chất của phép tịnh tiến biến đường thẳng thành đường thẳng song song hoặc trùng với nó. Cho nên, cần ưu tiên các lập luận logic (thông minh) sử dụng các tính chất đã biết của 1 đối tượng toán học nói chung hoặc 1 phép biến hình nói riêng, để tiết kiệm công sức tính toán \& thời gian đi tìm phương hướng tiếp cận khi giải 1 bài toán bất kỳ:
		\begin{align*}
			\boxed{\mbox{Smart strategies}\gg\mbox{heavy calculation\emph{\texttt{/}}computation skills.}}
		\end{align*}
		The symbol ``$\gg$'' in the last inequality means ``are much more important'', ``remarkably dominate'', or ``dramatically outweigh''. To be able to be lazy in doing mathematics, you need smart strategies, not heavy computations: Work less but effective -- laziness as its finest.
	\end{enumerate}
\end{nhanxet}

\subsection{Ứng dụng của phép tịnh tiến}

\begin{baitoan}[\cite{SGK_Toan_11_hinh_hoc_nang_cao}, p. 7]
	Cho 2 điểm $B,C$ cố định trên đường tròn $(O;R)$ \& 1 điểm $A$ thay đổi trên đường tròn đó. Chứng minh rằng trực tâm $\Delta ABC$ nằm trên 1 đường tròn cố định.
\end{baitoan}

\begin{proof}[Giải]
	Nếu $BC$ là đường kính thì trực tâm $H$ của $\Delta ABC$ chính là $A$. Vậy $H$ nằm trên đường tròn cố định $(O;R)$. Nếu $BC$ không phải là đường kính, vẽ đường kính $BB'$ của đường tròn. Nếu $H$ là trực tâm của $\Delta ABC$ thì $\overrightarrow{AH} = \overrightarrow{B'C}$ (suy ra từ nhận xét tứ giác $AHCB'$ là hình bình hành). Như vậy, phép tịnh tiến theo vector cố định $\overrightarrow{B'C}$ biến điểm $A$ thành điểm $H$. Do đó, khi $A$ thay đổi trên $(O;R)$ thì trực tâm $H$ luôn nằm trên đường tròn cố định là ảnh của đường tròn $(O;R)$ qua phép tịnh tiến nói trên.
\end{proof}

\begin{baitoan}
	2 thôn nằm ở 2 vị trí $A$ \& $B$ cách nhau 1 con sông (xem rằng 2 bờ sông là 2 đường thẳng song song). Người ta dự định xây 1 chiếc cầu $MN$ bắt qua sông (cố nhiên cầu phải vuông góc với bờ sông) \& làm 2 đoạn thẳng từ $A$ đến $M$ \& từ $B$ đến $N$. Hãy xác định vị trí chiếc cầu $MN$ sao cho $AM + BN$ ngắn nhất.
\end{baitoan}

\begin{proof}[Hint]
	Trường hợp tổng quát có thể đưa về trường hợp con sông rất hẹp -- hẹp đến mức 2 bờ sông $a$ \& $b$ xem như trùng nhau bằng 1 phép tịnh tiến theo vector $\overrightarrow{MN}$ để $a$ trùng $b$. Khi đó điểm $A$ biến thành điểm $A'$ sao cho $\overrightarrow{AA'} = \overrightarrow{MN}$ \& do đó $A'N = AM$.
\end{proof}

\begin{baitoan}[\cite{SGK_Toan_11_hinh_hoc_nang_cao}, \textbf{1.}, p. 9]
	Qua phép tịnh tiến $T$ theo vector $\vec{u}\ne\vec{0}$, đường thẳng $d$ biến thành đường thẳng $d'$. Trong trường hợp nào thì: $d\equiv d'$? $d\parallel d'$? $d$ cắt $d'$? 
\end{baitoan}

\begin{baitoan}[\cite{SGK_Toan_11_hinh_hoc_nang_cao}, \textbf{2.}, p. 9]
	Cho 2 đường thẳng song song $a$ \& $a'$. Tìm tất cả những phép tịnh tiến biến $a$ thành $a'$.
\end{baitoan}

\begin{baitoan}[\cite{SGK_Toan_11_hinh_hoc_nang_cao}, \textbf{3.}, p. 9]
	Cho 2 phép tịnh tiến $T_{\vec{u}}$ \& $T_{\vec{v}}$. Với điểm $M$ bất kỳ, $T_{\vec{u}}$ biến $M$ thành điểm $M'$, $T_{\vec{v}}$ biến $M'$ thành điểm $M''$. Chứng tỏ rằng phép biến hình biến $M$ thành $M''$ là 1 phép tịnh tiến.
\end{baitoan}

\begin{baitoan}[\cite{SGK_Toan_11_hinh_hoc_nang_cao}, \textbf{4.}, p. 9]
	Cho đường tròn $(O)$ \& 2 điểm $A,B$. 1 điểm $M$ thay đổi trên đường tròn $(O)$. Tìm quỹ tích điểm $M'$ sao cho $\overrightarrow{MM'} + \overrightarrow{MA} = \overrightarrow{MB}$.
\end{baitoan}

\begin{baitoan}[\cite{SGK_Toan_11_hinh_hoc_nang_cao}, \textbf{5.}, p. 9]
	Trong mặt phẳng tọa độ $Oxy$, với $\alpha,a,b$ là những số cho trước, xét phép biến hình $F$ biến mỗi điểm $M(x;y)$ thành điểm $M'(x';y')$, trong đó
	\begin{equation*}
		\left\{\begin{split}
			x' &= x\cos\alpha - y\sin\alpha + a,\\
			y' &= x\sin\alpha + y\cos\alpha + b.
		\end{split}\right.
	\end{equation*}
	\begin{enumerate*}
		\item[(a)] Cho 2 điểm $M(x_1;y_1)$, $N(x_2;y_2)$ \& gọi $M',N'$ lần lượt là ảnh của $M,N$ qua phép $F$. Tìm tọa độ của $M'$ \& $N'$.s
		\item[(b)] Tính khoảng cách $d$ giữa $M$ \& $N$; khoảng cách $d'$ giữa $M'$ \& $N'$.
		\item[(c)] Phép $F$ có phải là phép dời hình hay không?
		\item Khi $\alpha = 0$, chứng tỏ rằng $F$ là phép tịnh tiến.
	\end{enumerate*}
\end{baitoan}
Tổng quát hơn của bài toán \cite[\textbf{6.}, p. 9]{SGK_Toan_11_hinh_hoc_nang_cao}:
\begin{baitoan}
	Trong mặt phẳng tọa độ $Oxy$, xét các phép biến hình sau đây: Phép biến hình $F$ biến mỗi điểm $M(x;y)$ thành điểm $M'(f(x,y);g(x;y))$ với $f,g:\mathbb{R}^2\to\mathbb{R}$ là 2 hàm sốt. Với $f,g$ thỏa điều kiện nào thì $F$ là 1 phép dời hình?
\end{baitoan}

%------------------------------------------------------------------------------%

\section{Phép Dời Hình Phẳng}

\subsection{Đại cương về các phép dời hình phẳng}
``Không phải chỉ có phép tịnh tiến ``không làm thay đổi khoảng cách giữa 2 điểm'' mà còn nhiều phép biến hình khác cũng có tính chất đó (tính chất này còn được gọi là tính chất \textit{bảo toàn khoảng cách} giữa 2 điểm). Người ta gọi các phép biến hình như vậy là phép dời hình.'' -- \cite[p. 8]{SGK_Toan_11_hinh_hoc_nang_cao}

\subsubsection{Định nghĩa phép dời hình}

\begin{dinhnghia}[Phép dời hình]
	\begin{itemize}
		\item ``\emph{Phép dời hình} là phép biến hình không làm thay đổi khoảng cách giữa 2 điểm bất kỳ.'' -- \cite[p. 8]{SGK_Toan_11_hinh_hoc_nang_cao}
		\item ``1 phép biến hình $f:\mathcal{P}\to\mathcal{P}$ được gọi là 1 \emph{phép dời hình} của mặt phẳng, ký hiệu là $\mathcal{D}$, nếu với 2 điểm bất kỳ $M,N$ nào của $\mathcal{P}$ \& các ảnh $M' = f(M)$, $N' = f(N)$ của chúng, ta đều có $M'N' = MN$.
		
		Nói 1 cách ngắn gọn, phép dời hình của mặt phẳng, hay gọi vắn tắt là \emph{phép dời hình phẳng}, là phép biến hình bảo toàn khoảng cách giữa 2 điểm bất kỳ nào của mặt phẳng. Vậy là, nếu ký hiệu tập hợp các phép dời hình của mặt phẳng là $\{\mathcal{D}\}$ thì: $f\in\{\mathcal{D}\}$ của $\mathcal{P}\Leftrightarrow f(M)f(N) = MN$, $\forall M,N\in\mathcal{P}$.'' -- \cite[p. 5]{TL_chuyen_Toan_Hinh_Hoc_11}
	\end{itemize}
\end{dinhnghia}
``Chính vì phép dời hình bảo toàn khoảng cách giữa 2 điểm bất kỳ nào nên người ta còn gọi nó là \textit{phép biến hình đẳng cự}, hay vắn tắt là \textit{phép đẳng cự}.'' -- \cite[p. 5]{TL_chuyen_Toan_Hinh_Hoc_11}. Từ định nghĩa của phép dời hình ta suy ra:

\begin{hequa}[\cite{TL_chuyen_Toan_Hinh_Hoc_11}, p. 5]
	\begin{enumerate*}
		\item[(a)] Phép biến hình đồng nhất $\operatorname{id}$ là 1 phép dời hình.
		\item[(b)] Phép biến hình đảo ngược của 1 phép dời hình cũng là 1 phép dời hình.
		\item[(c)] Hợp thành (i.e., tích) của 2, hay $n$ ($n\in\mathbb{N}$, $n > 2$) phép dời hình là 1 phép dời hình.
		\item[(d)] Phép đối xứng trục, phép đối xứng tâm, phép tịnh tiến, phép quay (xung quanh 1 điểm) là những phép dời hình phẳng.
	\end{enumerate*}
\end{hequa}

\subsubsection{Các tính chất của phép dời hình}
Chú ý rằng các tính chất đã nêu của phép tịnh tiến được chứng minh dựa vào tính chất ``\textit{không làm thay đổi khoảng cách giữa 2 điểm}''. Bởi vậy, các phép dời hình cũng có những tính chất đó. Cụ thể ta có:

\begin{dinhly}[\cite{SGK_Toan_11_hinh_hoc_nang_cao}, p. 8; \cite{TL_chuyen_Toan_Hinh_Hoc_11}, p. 6]
	Phép dời hình biến 3 điểm thẳng hàng thành 3 điểm thẳng hàng \& không làm thay đổi thứ tự 3 điểm đó, biến 1 đường thẳng thành 1 đường thẳng, biến 1 tia thành 1 tia, biến 1 đoạn thẳng thành 1 đoạn thẳng bằng nó, biến 1 tam giác thành 1 tam giác bằng nó, biến 1 đường tròn thành đường tròn có cùng bán kính, trong đó tâm biến thành tâm, biến 1 góc thành 1 góc bằng nó.
\end{dinhly}

\begin{dinhly}[\cite{TL_chuyen_Toan_Hinh_Hoc_11}, p. 6]
	Phép dời hình bảo toàn sự thẳng hàng của 3 điểm \& thứ tự của chúng trên đường thẳng chứa 3 điểm đó.
\end{dinhly}
``Cụ thể là: Phép dời hình biến 3 điểm $A,B,C$ thẳng hàng, trong đó $B$ ở giữa $A$ \& $C$ thành 3 điểm $A',B',C'$ thẳng hàng cũng theo thứ tự đó.'' -- \cite[p. 6]{TL_chuyen_Toan_Hinh_Hoc_11}

\begin{dinhly}[\cite{TL_chuyen_Toan_Hinh_Hoc_11}, p. 6]
	1 phép dời hình phẳng có 3 điểm bất động không thẳng hàng là phép biến hình đồng nhất.
\end{dinhly}

\begin{proof}[Chứng minh]
	``Giả sử $f:\mathcal{P}\to\mathcal{P}$ là 1 phép dời hình phẳng có 3 điểm bất động không thẳng hàng: $A = A' = f(A)$, $B = B' = f(B)$, $C = C' = f(C)$. Thế thì theo tính chất của phép dời hình, bất kỳ 1 điểm nào trên các đường thẳng $(BC),(CA)$, hoặc $(AB)$ đều là điểm bất động. Từ đó dễ dàng suy ra mọi điểm $M$ của mặt phẳng $(ABC)$ đều là điểm bất động, \& do đó $f = \operatorname{id}$.'' -- \cite[p. 6]{TL_chuyen_Toan_Hinh_Hoc_11}
\end{proof}

\begin{hequa}[\cite{TL_chuyen_Toan_Hinh_Hoc_11}, p. 6]
	1 phép dời hình phẳng $\mathcal{D}\ne\operatorname{id}$ thì hoặc không có điểm bất động nào, hoặc có 1 điểm bất động duy nhất, hoặc có 1 đường thẳng mà mọi điểm của nó đều là điểm bất động (i.e., có 1 đường thẳng cố định).
\end{hequa}
Các ví dụ về các phép dời hình có 0 điểm bất động, 1 điểm bất động, \& 1 tập hợp các điểm bất động là 1 đường thẳng cố định trong hệ quả vừa phát biểu lần lượt là phép tịnh tiến theo 1 vector khác $\vec{0}$, phép đối xứng tâm, \& phép đối xứng trục.

\subsubsection{Khái niệm về 2 hình bằng nhau}
``Phép dời hình biến $\Delta ABC$ thành $\Delta A'B'C'$ bằng nó, trong đó các cạnh tương ứng bằng nhau \& các góc tương ứng bằng nhau: $B'C' = BC$, $C'A' = CA$, $A'B' = AB$; $\widehat{A'} = \widehat{A}$, $\widehat{B'} = \widehat{B}$, $\widehat{C'} = \widehat{C}$. 1 cách tổng quát, giả sử 1 phép dời hình $\mathcal{D}$ biến 1 hình (phẳng) $\mathcal{H}$ thành 1 hình $\mathcal{H}'$, ký hiệu $\mathcal{H}' = \mathcal{D}(\mathcal{H})$.'' -- \cite[pp. 6--7]{TL_chuyen_Toan_Hinh_Hoc_11}

\begin{dinhnghia}[2 hình bằng nhau]
	``2 hình $\mathcal{H}$ \& $\mathcal{H}'$ được gọi là \emph{bằng nhau}, nếu có 1 phép dời hình $\mathcal{D}$ biến hình $\mathcal{H}$ thành hình $\mathcal{H}'$ (\& do đó, phép dời hình $\mathcal{D}^{-1}$, đảo ngược của phép dời hình $\mathcal{D}$ biến $\mathcal{H}'$ thành $\mathcal{H}$), ký hiệu như thông thường: $\mathcal{H}'\coloneqq\mathcal{H}$.'' -- \cite[p. 7]{TL_chuyen_Toan_Hinh_Hoc_11}
\end{dinhnghia}

\subsection{Sự xác định 1 phép dời hình phẳng}
\begin{itemize}
	\item ``Khi chúng ta nói cho 1 phép dời hình $\mathcal{D}$ mà tổng quát hơn là cho 1 phép biến hình $f$ của $\mathcal{P}$ (mặt phẳng), i.e., chỉ ra đầy đủ các yếu tốt để xác định hoàn toàn phép dời hình (hay phép biến hình) đó của $\mathcal{P}$. I.e.: Với 1 điểm $M$ bất kỳ của $\mathcal{P}$, ta phải chỉ ra cách (quy tắc) dựng, cũng là cách xác định được điểm tương ứng (ảnh) $M'$ của nó qua phép dời hình (hay phép biến hình) này.
	\item Về phép dời hình, ta đã biết rằng 1 phép dời hình biến 1 $\Delta ABC$ thành 1 $\Delta A'B'C'$ bằng nó, trong đó các cạnh tương ứng bằng nhau, các góc tương ứng bằng nhau.
\end{itemize}
Mệnh đề sau đây khẳng định điều ngược lại.

\begin{dinhly}[Về sự xác định 1 phép dời hình phẳng]
	$\Delta ABC$ \& $\Delta A'B'C'$ là 2 tam giác bằng nhau cho trước trong mặt phẳng $\mathcal{P}$ ($B'C' = BC$, $C'A' = CA$, $A'B' = AB$). Bao giờ cũng có 1 \& chỉ 1 phép dời hình $\mathcal{D}:\mathcal{P}\to\mathcal{P}$ biến $A$ thành $A'$, $B$ thành $B'$ \& $C$ thành $C'$. Đồng thời, phép dời hình $\mathcal{D}$ này có thể phân tích thành tích của không quá 3 phép đối xứng trục.
\end{dinhly}

\begin{proof}[Chứng minh]
	Xem \cite[p. 7]{TL_chuyen_Toan_Hinh_Hoc_11}
\end{proof}

%------------------------------------------------------------------------------%

\section{Phép Đối Xứng Trục}

%------------------------------------------------------------------------------%

\section{Phép Quay \& Phép Đối Xứng Tâm}
\texttt{Quick notes.} Hình có tâm đối xứng phải có chẵn đỉnh (i.e., số đỉnh của hình đó phải là 1 số chẵn). Hình có số đỉnh là 1 số lẻ chắc chắn không phải là hình có tâm đối xứng.

Hình có vô số tâm đối xứng, e.g.:
\begin{itemize}
	\item 1 đường thẳng với tâm đối xứng là bất kỳ điểm nào thuộc nó, i.e., tập hợp các tâm đối xứng của 1 đường thẳng là chính nó. 
	\item 2 đường thẳng song song có tâm đối xứng là 1 điểm bất kỳ nằm trên đường thẳng song song \& cách đều 2 đường thẳng đó. Đường thẳng này cũng là tập hợp các tâm đối xứng của hình gồm 2 đường thẳng song song này.
\end{itemize}

đẳng cự

%------------------------------------------------------------------------------%

\section{2 Hình bằng Nhau}

%------------------------------------------------------------------------------%

\section{Phép Vị Tự}

%------------------------------------------------------------------------------%

\section{Phép Đồng Dạng}

%------------------------------------------------------------------------------%

\section{Hình Tự Đồng Dạng \& Hình Học Fractal}

%------------------------------------------------------------------------------%

\chapter{Đường Thẳng \& Mặt Phẳng Trong Không Gian. Quan Hệ Song Song -- Line \& Plane in Euclidean Space $\mathbb{R}^n$. Parallelism}

``Từ trước đến nay trong phần hình học phẳng, chúng ta xem xét các vấn đề, các bài toán trong đó các đối tượng, các hình thể \& các chuyển động đều được cho trong 1 mặt phẳng. Tuy nhiên, trong thực tế, không gian mà chúng đang sống không chỉ bó hẹp trong 1 mặt phẳng; các vật thể, các dịch chuyển mà chúng ta vẫn quan sát hàng ngày không phải lúc nào cũng có thể được mô tả bằng những mô hình đã có trong hình học phẳng. \textit{Hình học không gian}\texttt{/}\textit{hình học khối} đã phần nào giúp ta khắc phục những thiếu sót đó''. ``Bên cạnh điểm \& đường thẳng thư đã có trong hình học phẳng, chúng ta cần xem thêm 1 đối tượng cơ bản nữa là mặt phẳng \& các hình thể được xét đến trong hình học không gian chủ yếu đều có dạng ``khối'' như tứ diện, hình chóp, hình lăng trụ, hình cầu, $\ldots$'' -- \cite[p. 43]{TL_chuyen_Toan_Hinh_Hoc_11}

\section{Đại Cương về Đường Thẳng \& Mặt Phẳng}

\subsection{Điểm, đường thẳng, mặt phẳng, \& các tiên đề}
``$\ldots$ trong hình học phẳng, điểm, đường thẳng là các đối tượng cơ bản \& xuất phát từ các đối tượng này cũng như từ các mối tương quan ban đầu giữa chúng (mà người ta thường gọi đó là các \textit{tiên đề}), ta có thể xây dựng \& định nghĩa các đối tượng, các khái niệm khác \& thiết lập các mối tương quan mới giữa chúng. Trong hình học không gian, ta sẽ xuất phát từ các đối tượng cơ bản là \textit{điểm, đường thẳng, mặt phẳng}. Đây là các đối tượng được thừa nhận từ đầu, không qua định nghĩa. Có thể nói rằng chính chúng là sự thể hiện của những hình tượng, những đường nét quen thuộc mà chúng ta vẫn thường gặp \& cảm nhận được khi quan sát các vật thể xung quanh cùng những chuyển động của chúng trong cuộc sống hằng ngày. E.g., điểm có thể được xem là biểu tượng chung của những ngôi sao lấp lánh trên bầu trời về đêm hoặc những lỗ khoan trên 1 tấm bảng gỗ, $\ldots$ đường thẳng là đường nét chính hiện hữu trong những tia sáng mặt trời hoặc trong những sợi dây được kéo căng, $\ldots$ Còn mặt biển yên ả không gợn sóng hoặc bề mặt 1 bức tường là những hình ảnh thực tế của 1 phần mặt phẳng, $\ldots$

Trong hình học không gian các điểm thường được biểu thị trên hình vẽ bằng các chấm nhỏ \& được ký hiệu bằng các chữ in lớn như: $A,B,C,\ldots$. Các đường thẳng thường được biểu thị bằng các nét kẻ thẳng \& ký hiệu bằng các chữ in thường như: $a,b,c,\ldots$. Còn các mặt phẳng thường được biểu thị bằng các hình bình hành \& ký hiệu bằng chữ cái Hy Lạp hoặc chữ in hoa được ghi trong ngoặc đơn như: $(\alpha),(\beta),(\gamma),\ldots$ hoặc $(P),(Q),\ldots$. Đường thẳng \& mặt phẳng là các tập hợp điểm. Nếu điểm $A$ thuộc đường thẳng $a$, ký hiệu $A\in a$ \& đôi khi còn nói rằng đường thẳng $a$ đi qua điểm $A$. Nếu điểm $A$ thuộc mặt phẳng $(\alpha)$, ký hiệu $A\in(\alpha)$ \& đôi khi còn nói rằng mặt phẳng $(\alpha)$ đi qua điểm $A$. Nếu đường thẳng $a$ chứa trong mặt phẳng $(\alpha)$, ký hiệu $a\subset(\alpha)$ \& đôi khi còn nói rằng mặt phẳng $(\alpha)$ đi qua (hoặc chứa) đường thẳng $a$.

Bên cạnh việc lấy điểm, đường thẳng, mặt phẳng làm các đối tượng cơ bản (mà không định nghĩa), chúng ta còn thừa nhận 1 số tương quan cơ bản giữa chúng (mà không chứng minh). Các tương quan này thường xuất phát từ những nhận định đơn giản nhất về các mối quan hệ giữa điểm, đường thẳng, mặt phẳng có trong thực tế \& thường được gọi là các \textit{tiên đề}. Trên cơ sở các đối tượng cơ bản là điểm, đường thẳng, mặt phẳng, \& các tiên đề được thừa nhận, người ta xây dựng các đối tượng, các khái niệm khác \& thiết lập 1 cách chặt chẽ các tính chất cũng như các mối tương quan giữa các đối tượng mới này.

\begin{tiende}
	Trong không gian với 2 điểm phân biệt cho trước có 1 \& chỉ 1 đường thẳng đi qua.
\end{tiende}
Như vậy với 2 điểm phân $A,B$ tồn tại duy nhất 1 đường thẳng chứa cả 2 điểm này \& đường thẳng đó thường được gọi là \textit{đường thẳng $AB$}.

\begin{tiende}
	\label{TLCT Hinh Hoc 11 Tien de 2 p. 45}
	Trong không gian, với 3 điểm cho trước không cùng thuộc 1 đường thẳng, có 1 \& chỉ 1 mặt phẳng đi qua.
\end{tiende}
Do đó đối với 3 điểm $A,B,C$ không cùng thuộc 1 đường thẳng, tồn tại duy nhất 1 mặt phẳng chứa cả 3 điểm này \& mặt phẳng đó thường được gọi lfa \textit{mặt phẳng đi qua $A,B,C$} \& ký hiệu là $(ABC)$.

\begin{tiende}
	\label{TLCT Hinh Hoc 11 Tien de 3 p. 45}
	Trong không gian, 2 mặt phẳng phân biệt có 1 điểm chung thì phải có điểm chung thứ 2.
\end{tiende}

\begin{tiende}
	Trong không gian có ít nhất 4 điểm không cùng thuộc bất cứ mặt phẳng nào.
\end{tiende}
Nếu có 1 số điểm cùng thuộc 1 mặt phẳng thì ta thường nói rằng các điểm đó \textit{đồng phẳng}. Tiên đề \ref{TLCT Hinh Hoc 11 Tien de 2 p. 45} cho thấy rằng 3 điểm bất kỳ thì luôn luôn đồng phẳng. Nhưng đối với 4 điểm, tiên đề 4 cho thấy rằng điều tương tự không phải bao giờ cũng đúng.

\begin{tiende}
	Trong mỗi mặt phẳng của không gian, các tiên đề của hình học phẳng đều đúng.
\end{tiende}
Như vậy tiên đề này cho phép chúng ta sử dụng các kết quả đã có của hình học phẳng trong trường hợp nếu các đối tượng mà ta đang lưu ý xem xét \& các chuyển động của chúng được giới hạn trong phạm vi 1 mặt phẳng nào đó của không gian.

Từ các tiên đề vừa được phát biểu, chúng ta có thể thiết lập được 1 tổng thể về vị trí tương đối của các đường thẳng \& mặt phẳng trong không gian.'' -- \cite[pp. 43--45]{TL_chuyen_Toan_Hinh_Hoc_11}

\subsection{Vị trí tương đối của các đường thẳng \& mặt phẳng trong không gian}

\subsubsection{Vị trí tương đối của 1 đường thẳng \& 1 mặt phẳng}
``Cho đường thẳng $d$ \& mặt phẳng $(\alpha)$. Có thể xảy ra 1 trong các khả năng sau:
\begin{itemize}
	\item Đường thẳng $d$ \& mặt phẳng $(\alpha)$ không có điểm chung. Trong trường hợp này ta sẽ nói rằng đường thẳng $d$ song song với mặt phẳng $(\alpha)$ \& ký hiệu $d\parallel(\alpha)$.
	\item Đường thẳng $d$ \& mặt phẳng $(\alpha)$ có đúng 1 điểm chung. Trong trường hợp này ta sẽ nói rằng đường thẳng $d$ cắt mặt phẳng $(\alpha)$ tại điểm $A$ \& ký hiệu $d\cap(\alpha) = \{A\}$.
	\item Đường thẳng $d$ \& mặt phẳng $(\alpha)$ có nhiều hơn 1 điểm chung. Lúc này ta sẽ chứng minh rằng đường thẳng $d$ thuộc mặt phẳng $(\alpha)$ \& ký hiệu $d\subset(\alpha)$. Thật vậy, giả sử $d$ \& $(\alpha)$ có 2 điểm chung là $P$ \& $Q$. Xét đường thẳng $a$ đi qua $P$ \& $Q$ trong mặt phẳng $(\alpha)$. Theo tiên đề 1 thì $d$ \& $a$ phải trùng nhau, thành thử đường thẳng $d$ phải thuộc mặt phẳng $(\alpha)$.
\end{itemize}
Vậy \textit{với 1 đường thẳng \& 1 mặt phẳng cho trước thì: Đường thẳng hoặc song song, hoặc cắt mặt phẳng tại 1 điểm hoặc thuộc mặt phẳng đó}.'' -- \cite[p. 46]{TL_chuyen_Toan_Hinh_Hoc_11}

\subsubsection{Vị trí tương đối của 2 mặt phẳng}
``Cho 2 mặt phẳng phân biệt $(\alpha)$ \& $(\beta)$. Có thể xảy ra 1 trong các khả năng sau:
\begin{itemize}
	\item Các mặt phẳng $(\alpha)$ \& $(\beta)$ không có điểm chung. Trong trường hợp này ta sẽ nói rằng các mặt phẳng $(\alpha)$ \& $(\beta)$ song song với nhau \& ký hiệu $(\alpha)\parallel(\beta)$.
	\item Các mặt phẳng $(\alpha)$ \& $(\beta)$ có ít nhất 1 điểm chung. Lúc này ta sẽ chứng minh rằng các mặt phẳng $(\alpha)$ \& $(\beta)$ có phần chung là 1 đường thẳng. Thật vậy theo tiên đề \ref{TLCT Hinh Hoc 11 Tien de 3 p. 45} phải có ít nhất 2 điểm chung là $A$ \& $B$. Theo phần vị trí tương đối của đường thẳng \& mặt phẳng ở trên thì $(\alpha)$ \& $(\beta)$ đều chứa đường thẳng $AB$. Hơn nữa có thể thấy chúng không thể có điểm chung nào khác ngoài đường thẳng $AB$. Nếu không, gọi $C$ là điểm như vậy thì các mặt phẳng $(\alpha)$ \& $(\beta)$ đều đi qua 3 điểm $A,B,C$ không thẳng hàng nên phải trùng nhau. Vô lý. Vậy $(\alpha)$ \& $(\beta)$ có phần chung là đường thẳng $d$. Trong trường hợp này ta thường nói rằng các mặt phẳng $(\alpha)$ \& $(\beta)$ cắt nhau theo \textit{giao tuyến} là đường thẳng $d$ \& ký hiệu $(\alpha)\cap(\beta) = d$.
\end{itemize}
Vậy \textit{với 2 mặt phẳng phân biệt cho trước thì: 2 mặt phẳng đó hoặc song song hoặc cắt nhau theo giao tuyến là 1 đường thẳng}.'' -- \cite[pp. 46--47]{TL_chuyen_Toan_Hinh_Hoc_11}

\subsubsection{Vị trí tương đối của 2 đường thẳng}
``Cho 2 đường thẳng phân biệt $a$ \& $b$. Có thể xảy ra 1 trong các khả năng sau:
\begin{itemize}
	\item Các đường thẳng $a$ \& $b$ cùng thuộc 1 mặt phẳng. Trong trường hợp này ta thường nói rằng các đường thẳng $a$ \& $b$ đồng phẳng \& trong hình học phẳng ta biết rằng $a$ \& $b$ lúc đó hoặc cắt nhau tại 1 điểm hoặc không cắt nhau. Trường hợp 2 đường thẳng $a$ \& $b$ đồng phẳng \& không cắt nhau, cũng như trong hình học phẳng, ta sẽ nói rằng \textit{$a$ \& $b$ song song với nhau} \& ký hiệu $a\parallel b$.
	\item Các đường thẳng $a$ \& $b$ không cùng thuộc bất cứ 1 mặt phẳng nào. Trong trường hợp này ta thường nói rằng \textit{các đường thẳng $a$ \& $b$ chéo nhau}. Nhận xét rằng nếu các đường thẳng $a$ \& $b$ chéo nhau thì chúng không có điểm chung. Thật vậy, nếu $a$ \& $b$ chéo nhau \& có điểm chung là $O$. Lấy trên $a$ \& $b$ các điểm $A$ \& $B$ khác $O$. Rõ ràng $O,A,B$ không cùng thuộc 1 đường thẳng. Do đó theo vị trí tương đối giữa đường thẳng \& mặt phẳng thì các đường thẳng $a$ \& $b$ đều thuộc mặt phẳng $(OAB)$. Vô lý! Vậy $a$ \& $b$ không có điểm chung.
\end{itemize}
Như vậy có thể thấy rằng khác với hình học phẳng, trong hình học không gian, 2 đường thẳng không có điểm chung không nhất thiết phải song song với nhau mà có thể chéo nhau.

Vậy \textit{với 2 đường thẳng phân biệt trong không gian thì: 2 đường thẳng đó hoặc đồng phẳng hoặc chéo nhau. Trong trường hợp đồng phẳng, chúng hoặc cắt nhau tại 1 điểm hoặc song song với nhau}.'' -- \cite[pp. 47--48]{TL_chuyen_Toan_Hinh_Hoc_11}

\subsection{Xác định mặt phẳng trong không gian}
``Từ tiên đề \ref{TLCT Hinh Hoc 11 Tien de 2 p. 45} ở trên, ta biết rằng 1 mặt phẳng được xác định bởi 3 điểm cho trước không cùng thuộc 1 đường thẳng. Trong các phần lý thuyết \& bài tập về sau, mặt phẳng còn có thể được xác định bằng các cách thức khác. Ta nói rằng mặt phẳng được xác định bởi 1 cách thức nếu tồn tại duy nhất 1 mặt phẳng thỏa mãn tất cả các yêu cầu của cách thức đó. Mệnh đề sau đây cho ta 1 số cách thức xác định mặt phẳng quan trọng nhất.

\begin{menhde}
	1 mặt phẳng trong không gian có thể được xác định bởi 1 trong các cách thức sau:
	\begin{enumerate*}
		\item[(a)] Mặt phẳng đó đi qua 3 điểm không cùng thuộc 1 đường thẳng;
		\item[(b)] Mặt phẳng đó đi qua 1 đường thẳng \& 1 điểm ngoài đường thẳng ấy;
		\item[(c)] Mặt phẳng đó đi qua 2 đường thẳng cắt nhau;
		\item[(d)] Mặt phẳng đó đi qua 2 đưởng thẳng song song với nhau.
	\end{enumerate*}
\end{menhde}

\begin{proof}[Chứng minh]
	\begin{enumerate*}
		\item[\textbf{(a)}] Tiên đề \ref{TLCT Hinh Hoc 11 Tien de 2 p. 45}.
		\item[\textbf{(b)}] Cho đường thẳng $d$ \& điểm $A$ ở ngoài $d$. Ta chứng minh tồn tại duy nhất 1 mặt phẳng $(\alpha)$ đi qua $d$ \& $A$. Thật vậy, lấy trên đường thẳng $d$ 2 điểm $B,C$ \& gọi $(\alpha)$ là mặt phẳng đi qua 3 điểm $A,B,C$ thì đường thẳng $d$ phải thuộc mặt phẳng $(\alpha)$ thành thử $(\alpha)$ đi qua $d$ \& $A$. Do mọi mặt phẳng chứa $d$ \& $A$ đều phải đi qua 3 điểm $A,B,C$ nên theo tiên đề \ref{TLCT Hinh Hoc 11 Tien de 2 p. 45}, $(\alpha)$ chính là mặt phẳng duy nhất đi qua đường thẳng $d$ \& điểm $A$.
		\item[\textbf{(c)}--\textbf{(d)}] Cho 2 đường thẳng $a$ \& $b$ cắt nhau hoặc song song với nhau. Theo vị trí tương đối của 2 đường thẳng thì $a$ \& $b$ cùng thuộc 1 mặt phẳng $(\alpha)$. Theo vị trí tương đối của 2 mặt phẳng thì 2 mặt phẳng phân biệt không có chung 2 đường thẳng khác nhau nên $(\alpha)$ chính là mặt phẳng duy nhất đi qua các đường thẳng $a$ \& $b$.
	\end{enumerate*}
\end{proof}
Mặt phẳng đi qua đường thẳng $d$ \& điểm $A$ ở ngoài nó thường được ký hiệu là mặt phẳng $(d;A)$. Mặt phăng đi qua 2 đường thẳng $a$ \& $b$ cắt nhau hoặc song song với nhau thường được ký hiệu là mặt phẳng $(a;b)$.'' -- \cite[pp. 48--49]{TL_chuyen_Toan_Hinh_Hoc_11}

\begin{nhanxet}
	``Giao tuyến của 2 mặt phẳng là 1 đường thẳng; do vậy việc xác định giao tuyến của 2 mặt phẳng tương đương với việc xác định 2 điểm cùng thuộc đồng thời 2 mặt phẳng đã cho. Ngoài ra, nếu biết được rằng 3 điểm cũng thuộc đồng thời 2 mặt phẳng thì 3 điểm đó phải nằm trên 1 đường thẳng.'' -- \cite[p. 50]{TL_chuyen_Toan_Hinh_Hoc_11}
\end{nhanxet}

\subsection{Hình chóp \& tứ diện}

\subsubsection{Hình chóp}
``Hình chóp $S.A_1A_2\ldots A_n$ là hình được lập thành tử 1 đa giác $A_1A_2\ldots A_n$ \& điểm $S$ nằm ngoài mặt phẳng chứa đa giác. Điểm $S$ được gọi là \textit{đỉnh} của hình chóp. Đa giác $A_1A_2\ldots A_n$ được gọi là \textit{đáy} của hình chóp \& các đoạn $A_iA_{i+1}$, $i = 1,\ldots,n$, với $A_{n+1}\coloneqq A_1$, được gọi là các \textit{cạnh đáy} của hình chóp. Các $\Delta SA_iA_{i+1}$, $i = 1,\ldots,n$, được gọi là các \textit{mặt bên} của hình chóp. Các đoạn $SA_i$, $i = 1,\ldots,n$ được gọi là các \textit{cạnh bên} của hình chóp. Nếu đáy là tam giác, tứ giác hoặc ngũ giác, $\ldots$ thì hình chóp tương ứng được gọi là \textit{hình chóp tam giác, tứ giác hoặc ngũ giác}, $\ldots$'' -- \cite[pp. 51--52]{TL_chuyen_Toan_Hinh_Hoc_11}

\subsubsection{Tứ diện}
``Tứ diện $ABCD$ là hình được lập thanh từ 4 điểm không đồng phẳng $A,B,C,D$. Các điểm $A,B,C,D$ được gọi là các \textit{đỉnh} của tứ diện, các $\Delta BCD,\Delta CDA,\Delta DAB,\Delta ABC$ được gọi là các \textit{mặt} của tứ diện đối diện với các đỉnh $A,B,C,D$ \& các đoạn $AB,CD,AC,BD,AD,BC$ được gọi là \textit{cạnh} của tứ diện. Trong đó các cặp cạnh $AB$ \& $CD$, $AC$ \& $BD$, $AD$ \& $BC$ thường được gọi là các \textit{cặp cạnh đối} của tứ diện. Như vậy khác với hình chóp tam giác khi 1 đỉnh đã được chọn trước \& 3 điểm còn lại lập thành đáy hình chóp, trong tứ diện mỗi 1 trong 4 điểm đã cho đều là đỉnh \& 3 điểm còn lại lập thành mặt đối của đỉnh đó.'' -- \cite[p. 52]{TL_chuyen_Toan_Hinh_Hoc_11}

\begin{nhanxet}
	\begin{enumerate*}
		\item[(a)] ``Để xác định giao điểm của đường thẳng $d$ với mặt phẳng $(\alpha)$, ta thường đi tìm trước 1 đường thẳng $a$ trong mặt phẳng $(\alpha)$ sao cho $a$ \& $d$ đồng phẳng \& lấy giao điểm của $d$ \& $a$ Việc tìm đường thẳng $a$ như vậy thường yêu cầu phải dựng thêm 1 vài điểm bổ sung trong mặt phẳng $(\alpha)$.
		\item[(b)] Việc xác định giao tuyến của 2 mặt phẳng \& xác định giao điểm của đường thẳng \& mặt phẳng chính là phương tiện giúp chúng ta giải quyết vấn đề xác định thiết diện của 1 hình cắt bởi 1 mặt phẳng.'' ``Cho hình chóp $S.A_1A_2\ldots A_n$ \& mặt phẳng $(\alpha)$. Lúc đó mặt phẳng $(\alpha)$ có thể cắt 1 số mặt của hình chóp (${\rm mp}(\alpha)$ có thể không cắt hết các mặt của hình chóp). Mỗi mặt như vậy được $(\alpha)$ cắt theo 1 đoạn thẳng gọi là \emph{đoạn giao tuyến} (Fig. \ref{fig:thiet_dien_hinh_chop}). Khi sắp các đoạn giao tuyến đó 1 cách liên tiếp: $P_iP_{i+1}$, $i = 1,\ldots,m$, với $P_{m+1}\coloneqq P_1$ (điểm đầu của đoạn sau là điểm cuối của đoạn trước), ta được các cạnh liền nhau của 1 đa giác $P_1P_2\ldots P_m$. Đa giác này chính là \emph{thiết diện} (đôi khi còn gọi là \emph{mặt cắt}) của hình chóp cắt bởi mặt phẳng $(\alpha)$.'' -- \cite[pp. 53--54]{TL_chuyen_Toan_Hinh_Hoc_11}
	\end{enumerate*}
\end{nhanxet}

\begin{figure}[H]
	\centering
	\includegraphics[scale=0.13]{thiet_dien_hinh_chop}
	\caption{Thiết diện của hình chóp, \cite[Hình 2.10, p. 53]{TL_chuyen_Toan_Hinh_Hoc_11}.}
	\label{fig:thiet_dien_hinh_chop}
\end{figure}

\begin{nhanxet}
	``Để xác định thiết diện của hình chóp $S.A_1A_2\ldots A_n$ cắt bởi mặt phẳng $(\alpha)$ với các đường thẳng chứa các cạnh của hình chóp. Thiết diện cần xác định chính là đa giác với các đỉnh phải là các giao điểm của $(\alpha)$ với các cạnh của hình chóp (\& mỗi cạnh của thiết diện phải là 1 đoạn giao tuyến với 1 mặt của hình chóp).'' -- \cite[p. 55]{TL_chuyen_Toan_Hinh_Hoc_11}
\end{nhanxet}

%------------------------------------------------------------------------------%

\section{Quan Hệ Song Song -- Parallelism}
``Quá trình xác định thiết diện của 1 hình cắt bởi 1 mặt phẳng thường được thực hiện thông qua việc tìm giao tuyến của 2 mặt phẳng cũng như tìm giao điểm của 1 đường thẳng \& 1 mặt phẳng. Trong quá trình đó, mỗi đỉnh của 1 thiết diện cần dựng có thể xác định như là giao điểm của 2 đường thẳng đồng phẳng được chọn lựa phù hợp với tình hình cụ thể của từng bài toán. Trong 1 vài trường hợp, khi 2 đường thẳng đồng phẳng được chọn lựa không thể cắt nhau, việc tìm các yếu tố cần thiết cho bài toán xác định thiết diện (như giao tuyến, giao điểm) có thể được thực hiện dựa trên 1 số tính chất liên quan đến các mối quan hệ song song. Các tính chất này còn cho phép chúng ta bổ sung thêm những cách thức khác để xác định 1 đường thẳng hoặc 1 mặt phẳng trong không gian.'' -- \cite[p. 59]{TL_chuyen_Toan_Hinh_Hoc_11}

\subsection{Đường thẳng song song với đường thẳng}

\subsubsection{Định nghĩa}
``2 đường thẳng phân biệt bất kỳ hoặc chéo nhau hoặc đồng phẳng \& nếu đồng phẳng thì 2 đường thẳng đó hoặc cắt nhau tại 1 điểm hoặc không cắt nhau. Nếu 2 đường thẳng phân biệt đồng phẳng \& không cắt nhau thì cũng như trong hình học phẳng ta nói rằng 2 đường thẳng đó song song với nhau. Vậy, trong không gian: 2 đường thẳng $a$ \& $b$ được gọi là \textit{song song} với nhau, ký hiệu $a\parallel b$, nếu chúng đồng phẳng \& không cắt nhau.'' -- \cite[pp. 59--60]{TL_chuyen_Toan_Hinh_Hoc_11}

\subsubsection{Tính chất}
``Trong hình học phẳng, ta biết rằng với 1 đường thẳng $d$ cho trước \& 1 điểm $A$ cho trước nằm ngoài $d$, tồn tại duy nhất 1 đường thẳng $a$ đi qua $A$ \& song song với $d$. Khẳng định này cho ta 1 cách xác định đường thẳng trong mặt phẳng \& được thừa nhận như là 1 tiên đề. Tiên đề này thường được gọi là \textit{tiên đề 5 của hình học phẳng}. Trong không gian, tính chất này cũng đúng nhưng khác với trong mặt phẳng, nó có thể được chứng minh bằng cách sử dụng tiên đề 5 của hình học phẳng.

\begin{dinhly}
	Trong không gian, cho đường thẳng $d$ \& điểm $A$ ngoài $d$. Lúc đó tồn tại duy nhất 1 đường thẳng $a$ đi qua điểm $A$ \& song song với đường thẳng $d$.
\end{dinhly}

\begin{proof}[Chứng minh]
	Xét mặt phẳng $(d;A)$. Trong mặt phẳng này theo tiên đề 5 của hình học phẳng, tồn tại 1 đường thẳng $a$ đi qua $A$ \& song song với đường thẳng $d$. Nếu trong không gian còn có 1 đường thẳng $b$ cũng đi qua $A$ \& song song với $d$ thì mặt phẳng $(d;b)$ xác định bởi các đường thẳng song song $b$ \& $d$ cũng chính là mặt phẳng $(d;A)$. Theo tính chất của tiên đề 5 trong mặt phẳng này thì các đường thẳng $a$ \& $b$ phải trùng nhau. Vậy đường thẳng $a$ tồn tại \& duy nhất.
\end{proof}
Định lý vừa được chứng minh cho ta thêm 1 cách xác định đường thẳng trong không gian: đó là đường thẳng đi qua 1 điểm \& song song với 1 đường thẳng cho trước không chứa điểm đó. Kết hợp với mệnh đề dưới đây, nó còn cho 1 phương thức bổ sung để tìm giao tuyến của 2 mặt phẳng.

\begin{menhde}
	\label{TLCT Hinh Hoc 11 Menh de 1 p. 61}
	Nếu 2 mặt phẳng chứa lần lượt 2 đường thẳng song song với nhau \& 2 mặt phẳng đó cắt nhau theo 1 đường thẳng thì đường thẳng này song song với cả 2 đường thẳng trên hoặc trùng với 1 trong chúng.
\end{menhde}

\begin{proof}[Chứng minh]
	Giả sử các mặt phẳng $(\alpha)$ \& $(\beta)$ chứa lần lượt các đường thẳng $a$ \& $b$ song song với nhau \& $(\alpha),(\beta)$ cắt nhau theo giao tuyến $d$. Xét mặt phẳng $(a;b)$. Nếu mặt phẳng này trùng với $(\alpha)$ thì ta có: $(\alpha)\cap(\beta) = b$ \& do đó $b$ trùng với $d$. Tương tự nếu mặt phẳng $(a;b)$ trùng với $(\beta)$ thì $a$ trùng với $d$ \& mệnh đề được chứng minh. Bây giờ giả sử rằng $(a;b)$ không trùng với cả $(\alpha)$ lẫn $(\beta)$. Lấy điểm $P$ bất kỳ trên $d$. Nếu $P\in a$ thì $P\notin b$, suy ra các mặt phẳng $(\beta)$ \& $(a;b)$ phải trùng nhau vì cùng chứa $P$ \& $b$. Vô lý! Thành thử với mọi điểm $P\in d$ thì $P\notin a$ nên $d\parallel a$. Lập luận tương tự ta cũng được $d\parallel b$. Vậy $d$ song song với cả $a$ \& $b$.
\end{proof}
Từ mệnh đề \ref{TLCT Hinh Hoc 11 Menh de 1 p. 61}, có thể chứng minh tính chất sau:

\begin{menhde}
	2 đường thẳng phân biệt cùng song song với 1 đường thẳng thứ 3 thì song song với nhau.
\end{menhde}
Mệnh đề này thường được gọi là \textit{tính bắc cầu trong quan hệ song song giữa các đường thẳng} \& có thể được ghi vắn tắt dưới dạng: $(d_1\parallel d_2)\land(d_2\parallel d_3)\Rightarrow(d_1\parallel d_3)$. Trong hình học phẳng, đây là 1 tính chất quen thuộc \& là hệ quả của tiên đề 5. Còn trong hình học không gian, ta chứng minh nó như sau:

\begin{proof}[Chứng minh]
	Giả sử 2 đường thẳng $a,b$ cùng song song với đường thẳng $c$. Nếu $a,b$, \& $c$ cùng thuộc 1 mặt phẳng thì theo tính chất quen thuộc đã biết trong hình học phẳng, hệ quả được chứng minh. Giả sử 3 đường thẳng này không cùng thuộc 1 mặt phẳng. Lấy 1 điểm $P\in a$. Các mặt phẳng $(b;P)$ \& $(c;a)$ không trùng nhau \& có điểm $P$ chung nên chúng cắt nhau theo giao tuyến là 1 đường thẳng. Đường thẳng này cũng đi qua $P$ \& song song với $c$ như đường thẳng $a$ nên đó chính là đường thẳng $a$. Theo mệnh đề \ref{TLCT Hinh Hoc 11 Menh de 1 p. 61}, giao tuyến $a$ song song với $b$ \& $c$. Mệnh đề được chứng minh.'' -- \cite[pp. 60--62]{TL_chuyen_Toan_Hinh_Hoc_11}
\end{proof}

\subsection{Đường thẳng song song với mặt phẳng}

%------------------------------------------------------------------------------%

\section{2 Mặt Phẳng Song Song}

%------------------------------------------------------------------------------%

\section{Phép Chiếu Song Song}

%------------------------------------------------------------------------------%

\section{Phương Pháp Tiên Đề Trong Hình Học}

%------------------------------------------------------------------------------%

\chapter{Vector Trong Không Gian. Quan Hệ Vuông Góc -- Vector in Euclidean Space $\mathbb{R}^n$. Perpendicular Relation}

\section{Vector Trong Không Gian. Sự Đồng Phẳng của Các Vector}

%------------------------------------------------------------------------------%

\section{2 Đường Thẳng Vuông Góc}

%------------------------------------------------------------------------------%

\section{Đường Thẳng Vuông Góc với Mặt Phẳng}

%------------------------------------------------------------------------------%

\section{2 Mặt Phẳng Vuông Góc}

%------------------------------------------------------------------------------%

\section{Khoảng Cách}

%------------------------------------------------------------------------------%

\appendix

\chapter{Phụ Lục -- Appendix}

\section{Hàm Số Chẵn \& Hàm Số Lẻ -- Even \& Odd Functions}
\label{sect: even & odd functions}
``Trong toán học, \textit{hàm số chẵn} \& \textit{hàm số lẻ} là các \href{https://vi.wikipedia.org/wiki/H%C3%A0m_s%E1%BB%91}{hàm số} thỏa mãn các quan hệ \href{https://vi.wikipedia.org/wiki/%C4%90%E1%BB%91i_x%E1%BB%A9ng}{đối xứng} nhất định khi lấy \href{https://vi.wikipedia.org/wiki/Ngh%E1%BB%8Bch_%C4%91%E1%BA%A3o_ph%C3%A9p_c%E1%BB%99ng}{nghịch đảo phép cộng}. Chúng rất quan trọng trong nhiều lĩnh vực của \href{https://vi.wikipedia.org/wiki/Gi%E1%BA%A3i_t%C3%ADch_to%C3%A1n}{giải tích toán}, đặc biệt trong lý thuyết chuỗi lũy thừa \& \href{https://vi.wikipedia.org/wiki/Chu%E1%BB%97i_Fourier}{chuỗi Fourier}. Chúng được đặt tên theo \href{https://vi.wikipedia.org/wiki/T%C3%ADnh_ch%E1%BA%B5n_l%E1%BA%BB}{tính chẵn lẻ} của số mũ lũy thừa của \href{https://vi.wikipedia.org/wiki/L%C5%A9y_th%E1%BB%ABa}{hàm lũy thừa} thỏa mãn từng điều kiện: hàm số $f(x) = x^n$ là 1 hàm chẵn nếu $n$ là 1 số nguyên chẵn, \& nó là hàm lẻ nếu $n$ là 1 số nguyên lẻ.'' -- \href{https://vi.wikipedia.org/wiki/H%C3%A0m_s%E1%BB%91_ch%E1%BA%B5n_v%C3%A0_l%E1%BA%BB}{Wikipedia\texttt{/}hàm số chẵn \& lẻ}

\subsection{Hàm số chẵn -- Even function}

\begin{dinhnghia}[Hàm số chẵn]
	``Cho $f$ là 1 hàm số giá trị thực của 1 đối số thực, $f$ là \emph{hàm số chẵn} nếu điều kiện sau được thỏa mãn với mọi $x$ sao cho cả $x$ \& $-x$ đều thuộc miền xác định của $f$: $f(x) = f(-x)$, $\forall x\in\operatorname{dom}(f)$, với $\operatorname{dom}(f)$ ký hiệu miền xác định của $f$, hoặc phát biểu 1 cách tương đương, nếu phương trình sau thỏa mãn $f(x) - f(-x) = 0$, $\forall x\in\operatorname{dom}(f)$.
\end{dinhnghia}
Về mặt hình học, đồ thị của 1 hàm số chẵn \href{https://vi.wikipedia.org/wiki/%C4%90%E1%BB%91i_x%E1%BB%A9ng}{đối xứng} qua trục $y$, nghĩa là đồ thị của nó giữ không đổi sau phép \href{https://vi.wikipedia.org/wiki/%C4%90%E1%BB%91i_x%E1%BB%A9ng_tr%E1%BB%A5c}{lấy đối xứng qua trục $y$}.'' -- \href{https://vi.wikipedia.org/wiki/H%C3%A0m_s%E1%BB%91_ch%E1%BA%B5n_v%C3%A0_l%E1%BA%BB}{Wikipedia\texttt{/}hàm số chẵn \& lẻ}

\begin{vidu}[Hàm chẵn]
	\href{https://vi.wikipedia.org/wiki/Gi%C3%A1_tr%E1%BB%8B_tuy%E1%BB%87t_%C4%91%E1%BB%91i}{Hàm trị tuyệt đối} $x\mapsto|x|$, các hàm đơn thức dạng $x\mapsto x^{2n}$, \href{https://vi.wikipedia.org/wiki/H%C3%A0m_l%C6%B0%E1%BB%A3ng_gi%C3%A1c}{hàm cosin} $\cos$, \href{https://vi.wikipedia.org/wiki/H%C3%A0m_hyperbolic}{hàm cosin hyperbolic} $\cosh$.
\end{vidu}

\subsection{Hàm số chẵn -- Odd function}

\begin{dinhnghia}[Hàm số lẻ]
	Cho $f$ là 1 hàm số giá trị thực của 1 đối số (biến) thực, $f$ là hàm số \emph{lẻ} nếu điều kiện sau được thỏa mãn với mọi $x$ sao cho cả $x$ \& $-x$ đều thuộc miền xác định của $f$: $f(-x) = -f(x)$, $\forall x\in\operatorname{dom}(f)$, với $\operatorname{dom}(f)$ ký hiệu miền xác định của $f$, hoặc phát biểu 1 cách tương đương, nếu phương trình sau thỏa mãn $f(x) + f(-x) = 0$, $\forall x\in\operatorname{dom}(f)$.
\end{dinhnghia}
``Về mặt hình học, đồ thị của 1 hàm lẻ có tính đối xứng tâm quay qua gốc tọa độ, i.e., đồ thị của nó không đổi sau khi thực hiện phép quay $180^\circ$ quanh điểm gốc.'' -- \href{https://vi.wikipedia.org/wiki/H%C3%A0m_s%E1%BB%91_ch%E1%BA%B5n_v%C3%A0_l%E1%BA%BB}{Wikipedia\texttt{/}hàm số chẵn \& lẻ}

\begin{vidu}[Hàm số lẻ]
	Hàm đồng nhất $x\mapsto x$, các hàm đơn thức dạng $x\mapsto x^{2n + 1}$, \href{https://vi.wikipedia.org/wiki/Sin}{hàm sin} $\sin$, \href{https://vi.wikipedia.org/wiki/H%C3%A0m_hyperbolic}{hàm sin hyperbol} $\sinh$, \href{https://vi.wikipedia.org/wiki/H%C3%A0m_l%E1%BB%97i}{hàm lỗi} $\operatorname{erf}$.
\end{vidu}

\subsection{Các tính chất cơ bản}

\subsubsection{Tính duy nhất}
\begin{itemize}
	\item ``Nếu 1 hàm số vừa chẵn \& vừa lẻ, nó bằng $0$ ở mọi điểm mà nó được xác định.
	\item Nếu 1 hàm là lẻ thì \href{https://vi.wikipedia.org/wiki/Gi%C3%A1_tr%E1%BB%8B_tuy%E1%BB%87t_%C4%91%E1%BB%91i}{giá trị tuyệt đối} của hàm đó là 1 hàm chẵn.'' -- \href{https://vi.wikipedia.org/wiki/H%C3%A0m_s%E1%BB%91_ch%E1%BA%B5n_v%C3%A0_l%E1%BA%BB}{Wikipedia\texttt{/}hàm số chẵn \& lẻ}
\end{itemize}

\subsubsection{Cộng \& trừ hàm số chẵn lẻ}
\begin{itemize}
	\item Tổng \& hiệu của 2 hàm số chẵn là 2 hàm số chẵn.
	\item Tổng \& hiệu của 2 hàm lẻ là 2 hàm lẻ.
	\item Tổng của 1 hàm chẵn \& 1 hàm lẻ thì không chẵn cũng không lẻ, trừ khi 1 trong các hàm ấy bằng $0$ trên miền đã cho.
\end{itemize}

\subsubsection{Nhân \& chia hàm số chẵn lẻ}
\begin{itemize}
	\item Tích \& thương của 2 hàm chẵn là 2 hàm chẵn.
	\item Tích \& thương của 2 hàm lẻ là 2 hàm chẵn.
	\item Tích \& thương của 1 hàm chẵn với 1 hàm lẻ là 2 hàm lẻ.
\end{itemize}

\subsubsection{Hàm hợp (tích ánh xạ)}
\begin{itemize}
	\item Hàm hợp của 2 hàm chẵn là hàm chẵn.
	\item Hàm hợp của 2 hàm lẻ là hàm lẻ.
	\item 1 hàm chẵn hợp với 1 hàm lẻ là hàm chẵn.
	\item Hàm hợp của bất kỳ hàm số nào với 1 hàm chẵn là hàm chẵn (nhưng điều ngược lại không đúng).
\end{itemize}

\subsection{Phân tích chẵn--lẻ}
``Mọi hàm có thể được phân tích duy nhất thành tổng của 1 hàm chẵn \& 1 hàm lẻ, được gọi tương ứng là \textit{phần chẵn} \& \textit{phần lẻ} của 1 hàm số, nếu ta đặt như sau:
\begin{align*}
	f_{\rm e}(x)\coloneqq\frac{f(x) + f(-x)}{2},\ f_{\rm o}(x)\coloneqq\frac{f(x) - f(-x)}{2},
\end{align*}
sau đó $f_{\rm e}$ là hàm chẵn, $f_{\rm o}$ là hàm lẻ, \& $f(x) = f_{\rm e}(x) + f_{\rm o}(x)$. Ngược lại nếu $f(x) = g(x) + h(x)$, trong đó $g$ là chẵn \& $h$ là lẻ, thì $g = f_{\rm e}$ \& $h = f_{\rm o}$, bởi vì
\begin{align*}
	2f_{\rm e}(x) &= f(x) + f(-x) = g(x) + g(-x) + h(x) + h(-x) = 2g(x),\\
	2f_{\rm o}(x) &= f(x) - f(-x) = g(x) - g(-x) + h(x) - h(-x) = 2h(x).
\end{align*}

\begin{vidu}
	Hàm \href{https://vi.wikipedia.org/wiki/H%C3%A0m_hyperbolic}{cosin hyperbolic} \& \href{https://vi.wikipedia.org/wiki/H%C3%A0m_hyperbolic}{sin hyperbolic} có thể được coi là các phần chẵn \& phần lẻ của hàm số lũy thừa tự nhiên, bởi vì hàm thứ nhất là chẵn, hàm thứ 2 là lẻ, \& $e^x = \sinh x + \cosh x$.'' -- \href{https://vi.wikipedia.org/wiki/H%C3%A0m_s%E1%BB%91_ch%E1%BA%B5n_v%C3%A0_l%E1%BA%BB}{Wikipedia\emph{\texttt{/}}hàm số chẵn \& lẻ}
\end{vidu}

%------------------------------------------------------------------------------%

\begin{thebibliography}{99}
	\bibitem[NQBH\texttt{/}elementary math]{NQBH/elementary math} Nguyễn Quản Bá Hồng. \href{https://github.com/NQBH/hobby/blob/master/elementary_mathematics/some_topics_in_elementary_mathematics_problems_theories_applications_bridges_to_advanced_mathematics/NQBH_some_topics_in_elementary_mathematics_problems_theories_applications_bridges_to_advanced_mathematics.pdf}{\textit{Some Topics in Elementary Mathematics: Problems, Theories, Applications, \& Bridges to Advanced Mathematics}}. Mar 2022--now.
\end{thebibliography}

%------------------------------------------------------------------------------%

\printbibliography[heading=bibintoc]
	
\end{document}