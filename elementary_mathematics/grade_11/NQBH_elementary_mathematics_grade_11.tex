\documentclass[oneside]{book}
\usepackage[backend=biber,natbib=true,style=authoryear]{biblatex}
\addbibresource{/home/hong/1_NQBH/reference/bib.bib}
\usepackage[utf8]{vietnam}
\usepackage{tocloft}
\renewcommand{\cftsecleader}{\cftdotfill{\cftdotsep}}
\usepackage[colorlinks=true,linkcolor=blue,urlcolor=red,citecolor=magenta]{hyperref}
\usepackage{amsmath,amssymb,amsthm,mathtools,float,graphicx,algpseudocode,algorithm,tcolorbox}
\usepackage[inline]{enumitem}
\allowdisplaybreaks
\numberwithin{equation}{section}
\newtheorem{assumption}{Assumption}[section]
\newtheorem{conjecture}{Conjecture}[section]
\newtheorem{corollary}{Corollary}[section]
\newtheorem{hequa}{Hệ quả}[section]
\newtheorem{definition}{Definition}[section]
\newtheorem{dinhnghia}{Định nghĩa}[section]
\newtheorem{example}{Example}[section]
\newtheorem{vidu}{Ví dụ}[section]
\newtheorem{lemma}{Lemma}[section]
\newtheorem{notation}{Notation}[section]
\newtheorem{principle}{Principle}[section]
\newtheorem{problem}{Problem}[section]
\newtheorem{baitoan}{Bài toán}[section]
\newtheorem{proposition}{Proposition}[section]
\newtheorem{menhde}{Mệnh đề}[section]
\newtheorem{question}{Question}[section]
\newtheorem{cauhoi}{Câu hỏi}[section]
\newtheorem{remark}{Remark}[section]
\newtheorem{luuy}{Lưu ý}[section]
\newtheorem{theorem}{Theorem}[section]
\newtheorem{dinhly}{Định lý}[section]
\usepackage[left=0.5in,right=0.5in,top=1.5cm,bottom=1.5cm]{geometry}
\usepackage{fancyhdr}
\pagestyle{fancy}
\fancyhf{}
\lhead{\small \textsc{Sect.} ~\thesection}
\rhead{\small \nouppercase{\leftmark}}
\renewcommand{\sectionmark}[1]{\markboth{#1}{}}
\cfoot{\thepage}
\def\labelitemii{$\circ$}

\title{Some Topics in Elementary Mathematics\texttt{/}Grade 11}
\author{Nguyễn Quản Bá Hồng\footnote{Independent Researcher, Ben Tre City, Vietnam\\e-mail: \texttt{nguyenquanbahong@gmail.com}; website: \url{https://nqbh.github.io}.}}
\date{\today}

\begin{document}
\frontmatter
\maketitle
\setcounter{secnumdepth}{4}
\setcounter{tocdepth}{3}
\tableofcontents
\newpage

%------------------------------------------------------------------------------%

\mainmatter

\part{Đại Số \& Giải Tích}

\chapter{Hàm Số Lượng Giác \& Phương Trình Lượng Giác}

\section{Các Hàm Số Lượng Giác}

\section{Phương Trình Lượng Giác Cơ Bản}

\section{1 Số Dạng Phương Trình Lượng Giác Cơ Bản}

%------------------------------------------------------------------------------%

\chapter{Tổ Hợp \& Xác Suất}

\section{2 Quy Tắc Đếm Cơ Bản}

\section{Hoán Vị, Chỉnh Hợp \& Tổ Hợp}

\section{Nhị Thức Newton}

\section{Biến Cố \& Xác Suất của Biến Cố}

\section{Các Quy Tắc Tính Xác Suất}

\section{Biến Ngẫu Nhiên Rời Rạc}

%------------------------------------------------------------------------------%

\chapter{Dãy Số. Cấp Số Cộng \& Cấp Số Nhân}

\section{Phương Pháp Quy Nạp Toán Học}

\section{Dãy Số}

\section{Cấp Số Cộng}

\section{Cấp Số Nhân}

%------------------------------------------------------------------------------%

\chapter{Giới Hạn}

\section{Dãy Số Có Giới Hạn $0$}

\section{Dãy Số Có Giới Hạn Hữu Hạn}

\section{Dãy Số Có Giới Hạn Vô Cực}

\section{Định Nghĩa \& 1 Số Định Lý về Giới Hạn của Hàm Số}

\section{Giới Hạn 1 Bên}

\section{1 Vài Quy Tắc Tìm Giới Hạn Vô Cực}

\section{Các Dạng Vô Hình}

\section{Hàm Số Liên Tục}

%------------------------------------------------------------------------------%

\chapter{Đạo Hàm}

\section{Khái Niệm Đạo Hàm}

\section{Các Quy Tắc Tính Đạo Hàm}

\section{Đạo Hàm của Các Hàm Số Lượng Giác}

\section{Vi Phân}

\section{Đạo Hàm Cấp Cao}

%------------------------------------------------------------------------------%

\part{Hình Học}

\chapter{Phép Dời Hình \& Phép Đồng Dạng Trong Mặt Phẳng}

\section{Mở Đầu về Phép Biến Hình}

\section{Phép Tịnh Tiến \& Phép Dời Hình}

\section{Phép Đối Xứng Trục}

\section{Phép Quay \& Phép Đối Xứng Tâm}

\section{2 Hình bằng Nhau}

\section{Phép Vị Tự}

\section{Phép Đồng Dạng}

\section{Hình Tự Đồng Dạng \& Hình Học Fractal}

%------------------------------------------------------------------------------%

\chapter{Đường Thẳng \& Mặt Phẳng Trong Không Gian}

\section{Đại Cương về Đường Thẳng \& Mặt Phẳng}

\section{2 Đường Thẳng Song Song}

\section{Đường Thẳng Song Song với Mặt Phẳng}

\section{2 Mặt Phẳng Song Song}

\section{Phép Chiếu Song Song}

\section{Phương Pháp Tiên Đề Trong Hình Học}

%------------------------------------------------------------------------------%

\chapter{Vector Trong Không Gian. Quan Hệ Vuông Góc}

\section{Vector Trong Không Gian. Sự Đồng Phẳng của Các Vector}

\section{2 Đường Thẳng Vuông Góc}

\section{Đường Thẳng Vuông Góc với Mặt Phẳng}

\section{2 Mặt Phẳng Vuông Góc}

\section{Khoảng Cách}

%------------------------------------------------------------------------------%

\begin{thebibliography}{99}
	\bibitem[NQBH\texttt{/}elementary math]{NQBH/elementary math} Nguyễn Quản Bá Hồng. \href{https://github.com/NQBH/hobby/blob/master/elementary_mathematics/some_topics_in_elementary_mathematics_problems_theories_applications_bridges_to_advanced_mathematics/NQBH_some_topics_in_elementary_mathematics_problems_theories_applications_bridges_to_advanced_mathematics.pdf}{\textit{Some Topics in Elementary Mathematics: Problems, Theories, Applications, \& Bridges to Advanced Mathematics}}. Mar 2022--now.
\end{thebibliography}

%------------------------------------------------------------------------------%

\printbibliography[heading=bibintoc]
	
\end{document}