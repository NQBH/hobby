\documentclass[oneside]{book}
\usepackage[backend=biber,natbib=true,style=authoryear]{biblatex}
\addbibresource{/home/hong/1_NQBH/reference/bib.bib}
\usepackage[utf8]{vietnam}
\usepackage{tocloft}
\renewcommand{\cftsecleader}{\cftdotfill{\cftdotsep}}
\usepackage[colorlinks=true,linkcolor=blue,urlcolor=red,citecolor=magenta]{hyperref}
\usepackage{amsmath,amssymb,amsthm,mathtools,float,graphicx,algpseudocode,algorithm,tcolorbox,tikz,tkz-tab}
\DeclareMathOperator{\arccot}{arccot}
\usepackage[inline]{enumitem}
\allowdisplaybreaks
\numberwithin{equation}{section}
\newtheorem{assumption}{Assumption}[section]
\newtheorem{nhanxet}{Nhận xét}[section]
\newtheorem{conjecture}{Conjecture}[section]
\newtheorem{corollary}{Corollary}[section]
\newtheorem{hequa}{Hệ quả}[section]
\newtheorem{definition}{Definition}[section]
\newtheorem{dinhnghia}{Định nghĩa}[section]
\newtheorem{example}{Example}[section]
\newtheorem{vidu}{Ví dụ}[section]
\newtheorem{lemma}{Lemma}[section]
\newtheorem{notation}{Notation}[section]
\newtheorem{principle}{Principle}[section]
\newtheorem{problem}{Problem}[section]
\newtheorem{baitoan}{Bài toán}[section]
\newtheorem{proposition}{Proposition}[section]
\newtheorem{menhde}{Mệnh đề}[section]
\newtheorem{question}{Question}[section]
\newtheorem{cauhoi}{Câu hỏi}[section]
\newtheorem{remark}{Remark}[section]
\newtheorem{luuy}{Lưu ý}[section]
\newtheorem{theorem}{Theorem}[section]
\newtheorem{dinhly}{Định lý}[section]
\usepackage[left=0.5in,right=0.5in,top=1.5cm,bottom=1.5cm]{geometry}
\usepackage{fancyhdr}
\pagestyle{fancy}
\fancyhf{}
\lhead{\small \textsc{Sect.} ~\thesection}
\rhead{\small \nouppercase{\leftmark}}
\renewcommand{\sectionmark}[1]{\markboth{#1}{}}
\cfoot{\thepage}
\def\labelitemii{$\circ$}

\title{Some Topics in Elementary Mathematics\texttt{/}Grade 11}
\author{Nguyễn Quản Bá Hồng\footnote{Independent Researcher, Ben Tre City, Vietnam\\e-mail: \texttt{nguyenquanbahong@gmail.com}; website: \url{https://nqbh.github.io}.}}
\date{\today}

\begin{document}
\frontmatter
\maketitle
\setcounter{secnumdepth}{4}
\setcounter{tocdepth}{3}
\tableofcontents
\newpage

%------------------------------------------------------------------------------%

\mainmatter

\part{Đại Số \& Giải Tích -- Algebra \& Analysis}

\chapter{Hàm Số Lượng Giác \& Phương Trình Lượng Giác -- Trigonometric Function \& Trigonometric Equation}

``Nhiều hiện tượng tuần hoàn đơn giản trong thực tế được mô tả bởi những hàm số lượng giác. Chương này cung cấp những kiến thức cơ bản về các \textit{hàm số lượng giác} \& cách giải các \textit{phương trình lượng giác} đơn giản.'' -- \cite[p. 3]{SGK_Toan_11_dai_so_giai_tich_nang_cao}

\begin{quotation}
	\textbf{Nội dung.} \textit{Tính chất tuần hoàn của các hàm số lượng giác \& phương pháp sử dụng đường tròn lượng giác để tìm nghiệm của các phương trình lượng giác cơ bản, kỹ năng biến đổi lượng giác \& kỹ năng giải các dạng phương trình lượng giác}.
\end{quotation}

\section{Các Hàm Số Lượng Giác -- Trigonometric Functions}
``Các hàm số lượng giác\texttt{/}trigonometric\footnote{\textbf{trigonometric} [a] (also \textbf{trigonometrical}) (\textit{mathematics}) connected with the types of mathematics that deals with the relationship between the sides \& angles of triangles.}\,\footnote{\textbf{trigonometry} [n] [uncountable] the type of mathematics that deals with the relationship between the sides \& angles of triangles.} functions thường được dùng để mô tả những hiện tượng thay đổi 1 cách tuần hoàn hay gặp trong thực tiễn, khoa học \& kỹ thuật.'' -- \cite[p. 4]{SGK_Toan_11_dai_so_giai_tich_nang_cao}

\subsection{Các hàm số $y = \sin x$ \& $y = \cos x$}

\subsubsection{Khái niệm}

\begin{dinhnghia}[Hàm số $\sin,\cos$]
	Quy tắc đặt tương ứng mỗi số thực $x\in\mathbb{R}$ với $\sin$ của góc lượng giác có số đo radian bằng $x$ được gọi là \emph{hàm số $\sin$}, ký hiệu là $y = \sin x$. Quy tắc đặt tương ứng mỗi số thực $x\in\mathbb{R}$ với côsin của góc lượng giác có số đo radian bằng $x$ được gọi là \emph{hàm số côsin}, ký hiệu là $y = \cos x$.
\end{dinhnghia}
``Tập xác định của các hàm số $y = \sin x$, $y = \cos x$ là $\mathbb{R}$. Do đó các hàm số sin \& côsin được viết là:
\begin{equation*}
	\begin{split}
		\sin:\mathbb{R}&\to\mathbb{R}\\
		x&\mapsto\sin x
	\end{split},\ \begin{split}
		\cos:\mathbb{R}&\to\mathbb{R}\\
		x&\mapsto\cos x
	\end{split}.
\end{equation*}
Hàm số $y = \sin x$ là 1 \textit{hàm số lẻ} vì $\sin(-x) = -\sin(x)$, $\forall x\in\mathbb{R}$, trong khi hàm số $y = \cos x$ là 1 \textit{hàm số chẵn} vì $\cos(-x) = \cos x$, $\forall x\in\mathbb{R}$.'' -- \cite[p. 4]{SGK_Toan_11_dai_so_giai_tich_nang_cao}. Về định nghĩa \& tính chất của hàm số chẵn \& hàm số lẻ, xem Sect. \ref{sect: even & odd functions}. Có thể xem thêm \href{https://vi.wikipedia.org/wiki/H%C3%A0m_s%E1%BB%91_ch%E1%BA%B5n_v%C3%A0_l%E1%BA%BB}{Wikipedia\texttt{/}hàm số chẵn \& lẻ} \& \href{https://en.wikipedia.org/wiki/Even_and_odd_functions}{Wikipedia\texttt{/}even \& odd functions}.

\subsubsection{Tính chất tuần hoàn của các hàm số $y = \sin x$ \& $y = \cos x$}
``Với mỗi $k\in\mathbb{Z}$, số $k2\pi$ thỏa mãn: $\sin(x + k2\pi) = \sin x$, $\forall x\in\mathbb{R},\,\forall k\in\mathbb{Z}$. Ngược lại, có thể chứng minh rằng số $T$ sao cho $\sin(x + T) = \sin x$, $\forall x\in\mathbb{R}$ phải có dạng $T = k2\pi$, với $k\in\mathbb{Z}$. Rõ ràng, trong các số dạng $k2\pi$ ($k\in\mathbb{Z}$), số dương nhỏ nhất là $2\pi$. Vậy đối với hàm số $y = \sin x$, số $T = 2\pi$ là số dương nhỏ nhất thỏa mãn $\sin(x + T) = \sin x$, $\forall x\in\mathbb{R}$. Hàm số $y = \cos x$ cũng có tinh chất tương tự. Ta nói 2 hàm số đó là những \textit{hàm số tuần hoàn với chu kỳ $2\pi$}.

Từ tính chất tuần hoàn với chu kỳ $2\pi$, ta thấy khi biết giá trị các hàm số $y = \sin x$ \& $y = \cos x$ trên 1 đoạn có độ dài $2\pi$ (e.g., đoạn $[0;2\pi]$ hay đoạn $[-\pi;\pi]$) thì ta tính được giá trị của chúng tại mọi $x\in\mathbb{R}$. (Cứ mỗi khi biến số được cộng thêm $2\pi$ thì giá trị của các hàm số đó lại trở về như cũ; điều này giải thích từ ``tuần hoàn'').'' -- \cite[p. 4--5]{SGK_Toan_11_dai_so_giai_tich_nang_cao}

\subsubsection{Sự biến thiên \& đồ thị của hàm số $y = \sin x$}
``Do hàm số $y = \sin x$ là hàm số tuần hoàn với chu kỳ $2\pi$ nên ta chỉ cần khảo sát hàm số đó trên 1 đoạn có độ dài $2\pi$, e.g., trên đoạn $[-\pi;\pi]$.''
\begin{itemize}
	\item \textbf{Chiều biến thiên.} \textit{Bảng biến thiên của hàm số $y = \sin x$ trên đoạn $[-\pi;\pi]$}:
	
	\begin{figure}[H]
		\centering
		\includegraphics[scale=0.15]{bang_bien_thien_sin}
		\caption{Bảng biến thiên của hàm số $y = \sin x$ trên đoạn $[-\pi;\pi]$.}
	\end{figure}
	
	\item \textbf{Đồ thị.} ``Khi vẽ đồ thị của hàm số $y = \sin x$ trên đoạn $[-\pi;\pi]$, ta nên để ý rằng: Hàm số $y = \sin x$ là 1 hàm số lẻ, do đó đồ thị của nó nhận gốc tọa độ làm tâm đối xứng. Vì vậy, đầu tiên ta vẽ đồ thị của hàm số $y = \sin x$ trên đoạn $[0;\pi]$.
	
	\begin{figure}[H]
		\centering
		\includegraphics[scale=0.2]{graph_sin}
		\caption{Đồ thị của hàm số $y = \sin x$ trên đoạn $[0,\pi]$.}
		\label{fig:graph of sin}
	\end{figure}
	Trên đoạn $[0;\pi]$, đồ thị của hàm số $y = \sin x$ (Fig. \ref{fig:graph of sin}) đi qua các điểm có tọa độ $(x;y)$ trong bảng sau:
	\begin{table}[H]
		\centering
		\begin{tabular}{|c|c|c|c|c|c|c|c|c|c|}
			\hline
			$x$ & $0$ & $\frac{\pi}{6}$ & $\frac{\pi}{4}$ & $\frac{\pi}{3}$ & $\frac{\pi}{2}$ & $\frac{2\pi}{3}$ & $\frac{3\pi}{4}$ & $\frac{5\pi}{6}$ & $\pi$ \\
			\hline
			$y = \sin x$ & $0$ & $\frac{1}{2}$ & $\frac{\sqrt{2}}{2}$ & $\frac{\sqrt{3}}{2}$ & $1$ & $\frac{\sqrt{3}}{2}$ & $\frac{\sqrt{2}}{2}$ & $\frac{1}{2}$ & $0$ \\
			\hline
		\end{tabular}
		\caption{Các giá trị của hàm $y = \sin x$ tại 1 số điểm $\in[0;\pi]$.}
	\end{table}
	Phần đồ thị của hàm số $y = \sin x$ trên đoạn $[0;\pi]$ cùng với hình đối xứng của nó qua gốc $O$ lập thành đồ thị của hàm số $y = \sin x$ trên đoạn $[-\pi,\pi]$ (Fig. \ref{fig:duong hinh sin}).
	
	\begin{figure}[H]
		\centering
		\includegraphics[scale=0.2]{duong_hinh_sin}
		\caption{Đồ thị của hàm số $y = \sin x$ trên $\mathbb{R}$ -- \textit{đường hình sin}.}
		\label{fig:duong hinh sin}
	\end{figure}
	Tịnh tiến phần đồ thị vừa vẽ sang trái, sang phải những đoạn có độ dài $2\pi,4\pi,6\pi,\ldots$ thì được toàn bộ đồ thị hàm số $y = \sin x$. Đồ thị đó được gọi là 1 \textit{đường hình sin} (Fig. \ref{fig:duong hinh sin}).'' -- \cite[pp. 6--7]{SGK_Toan_11_dai_so_giai_tich_nang_cao}
\end{itemize}

\begin{nhanxet}
	\begin{enumerate}
		\item ``Khi $x$ thay đổi, hàm số $y = \sin x$ nhận mọi giá trị thuộc đoạn $[-1;1]$. Ta nói \emph{tập giá trị} của hàm số $y = \sin x$ là đoạn $[-1;1]$.
		\item Hàm số $y = \sin x$ đồng biến trên khoảng $\left(-\frac{\pi}{2};\frac{\pi}{2}\right)$. Từ đó, do tính chất tuần hoàn với chu kỳ $2\pi$, hàm số $y = \sin x$ đồng biến trên mỗi khoảng $\left(-\frac{\pi}{2} + k2\pi;\frac{\pi}{2} + k2\pi\right)$, $k\in\mathbb{Z}$.'' -- \cite[p. 7]{SGK_Toan_11_dai_so_giai_tich_nang_cao}
	\end{enumerate}
\end{nhanxet}

\subsubsection{Sự biến thiên \& đồ thị của hàm số $y = \cos x$}
``Ta có thể tiến hành khảo sát sự biến thiên \& vẽ đồ thị của hàm số $y = \cos x$ tương tự như đã làm đối với hàm số $y = \sin x$ trên đây. Tuy nhiên, ta nhận thấy $\cos x = \sin\left(x + \frac{\pi}{2}\right)$, $\forall x\in\mathbb{R}$, nên bằng cách tịnh tiến đồ thị hàm số $y = \sin x$ sang trái 1 đoạn có độ dài $\frac{\pi}{2}$, ta được đồ thị hàm số $y = \cos x$ (nó cùng được gọi là 1 \textit{đường hình sin}) (Fig. \ref{fig:graph cos}).

\begin{figure}[H]
	\centering
	\includegraphics[scale=0.2]{graph_cos}
	\caption{Đồ thị của hàm số $y = \cos x$ trên $\mathbb{R}$.}
	\label{fig:graph cos}
\end{figure}
Căn cứ vào đồ thị của hàm số $y = \cos x$, ta lập được bảng biến thiên của hàm số đó trên đoạn $[-\pi;\pi]$ (Fig. \ref{fig:bang bien thien cos}):

\begin{figure}[H]
	\centering
	\includegraphics[scale=0.15]{bang_bien_thien_cos}
	\caption{Bảng biến thiên của hàm số $y = \cos x$ trên đoạn $[-\pi;\pi]$.}
	\label{fig:bang bien thien cos}
\end{figure}

\begin{nhanxet}
	\begin{enumerate}
		\item Khi $x$ thay đổi, hàm số $y = \cos x$ nhận mọi giá trị thuộc đoạn $[-1;1]$. Ta nói \emph{tập giá trị} của hàm số $y = \cos x$ là đoạn $[-1;1]$.
		\item Do hàm số $y = \cos x$ là hàm số chẵn nên đồ thị của hàm số $y = \cos x$ nhận trục tung làm trục đối xứng.
		\item Hàm số $y = \cos x$ đồng biến trên khoảng $(-\pi;0)$. Từ đó do tính chất tuần hoàn với chu kỳ $2\pi$, hàm số $y = \cos x$ đồng biến trên mỗi khoảng $(-\pi + k2\pi;k2\pi)$, $k\in\mathbb{Z}$.'' -- \cite[pp. 8--9]{SGK_Toan_11_dai_so_giai_tich_nang_cao}
	\end{enumerate}
\end{nhanxet}

\begin{table}[H]
	\centering
	\begin{tabular}{|p{9cm}|p{9cm}|}
		\hline
		\textbf{Hàm số $y = \sin x$} & \textbf{Hàm số $y = \cos x$} \\
		\hline
		Có tập xác định là $\mathbb{R}$ & Có tập xác định là $\mathbb{R}$ \\
		\hline
		Có tập giá trị là $[-1;1]$ & Có tập giá trị là $[-1;1]$ \\
		\hline
		Là hàm số lẻ & Là hàm số chẵn \\
		\hline
		Là hàm số tuần hoàn với chu kỳ $2\pi$ & Là hàm số tuần hoàn với chu kỳ $2\pi$ \\
		\hline
		Đồng biến trên mỗi khoảng $\left(-\frac{\pi}{2} + k2\pi;\frac{\pi}{2} + k2\pi\right)$ \& nghịch biến trên mỗi khoảng $\left(\frac{\pi}{2} + k2\pi;\frac{3\pi}{2} + k2\pi\right)$, $k\in\mathbb{Z}$ & Đồng biến trên mỗi khoảng $(-\pi + k2\pi;k2\pi)$ \& nghịch biến trên mỗi khoảng $(k2\pi;\pi + k2\pi)$, $k\in\mathbb{Z}$ \\
		\hline
		Có đồ thị là 1 đường hình sin & Có đồ thị là 1 đường hình sin \\
		\hline
	\end{tabular}
	\caption{So sánh tính chất của 2 hàm số $y = \sin x$ \& $y = \cos x$.}
\end{table}

\subsection{Các hàm số $y = \tan x$ \& $y = \cot x$}

\subsubsection{Định nghĩa}
\begin{itemize}
	\item ``Với mỗi số thực $x\in\mathbb{R}$ mà $\cos x\ne 0$, i.e., $x\ne\frac{\pi}{2} + k\pi$ ($k\in\mathbb{Z}$), ta xác định được số thực $\tan x = \frac{\sin x}{\cos x}$. Đặt $\mathcal{D}_1\coloneqq\mathbb{R}\backslash\left\{\frac{\pi}{2} + k\pi|k\in\mathbb{Z}\right\}$.
	
	\begin{dinhnghia}[Hàm số $\tan$]
		Quy tắc đặt tương ứng mỗi số $x\in\mathcal{D}_1$ với số thực $\tan x = \frac{\sin x}{\cos x}$ được gọi là \emph{hàm số tang}, ký hiệu là $y = \tan x$.
	\end{dinhnghia}
	Vậy hàm số $y = \tan x$ có tập xác định $\mathcal{D}_1$; ta viết
	\begin{align*}
		\tan:\mathcal{D}_1&\to\mathbb{R}\\
		x&\mapsto\tan x.
	\end{align*}
	\item Với mỗi số thực $x\in\mathbb{R}$ mà $\sin x\ne 0$, i.e., $x\ne k\pi$tan ($k\in\mathbb{Z}$), ta xác định được số thực $\cot x = \frac{\cos x}{\sin x}$. Đặt $\mathcal{D}_2\coloneqq\mathbb{R}\backslash\{k\pi|k\in\mathbb{Z}\}$.
	
	\begin{dinhnghia}[Hàm số $\cot$]
		Quy tắc đặt tương ứng mỗi số $x\in\mathcal{D}_2$ với số thực $\cot x = \frac{\cos x}{\sin x}$ được gọi là \emph{hàm số côtang}, ký hiệu là $y = \cot x$.
	\end{dinhnghia}
	Vậy hàm số $y = \cot x$ có tập xác định là $\mathcal{D}_2$; ta viết
	\begin{align*}
		\cot:\mathcal{D}_2&\to\mathbb{R}\\
		x&\mapsto\cot x.
	\end{align*}
	
	\begin{figure}[H]
		\centering
		\includegraphics[scale=0.15]{truc_tan_truc_cot}
		\caption{Trục tang \& trục côtang.}
		\label{fig:truc tan truc cot}
	\end{figure}
	Trên hình \ref{fig:truc tan truc cot}, ta có $(OA,OM) = x$, $\tan x = \overline{AT}$, $\cot x = \overline{BS}$.
	
	\begin{nhanxet}
		\begin{enumerate}
			\item Hàm số $y = \tan x$ là 1 \emph{hàm số lẻ} vì nếu $x\in\mathcal{D}_1$ thì $-x\in\mathcal{D}_1$ \& $\tan(-x) = -\tan x$.
			\item Hàm số $y = \cot x$ cũng là 1 \emph{hàm số lẻ} vì nếu $x\in\mathcal{D}_2$ thì $-x\in\mathcal{D}_2$ \& $\cot(-x) = -\cot x$.'' -- \cite[pp. 9--10]{SGK_Toan_11_dai_so_giai_tich_nang_cao}
		\end{enumerate}
	\end{nhanxet}	
\end{itemize}

\subsubsection{Tính chất tuần hoàn}
``Có thể chứng minh rằng $T = \pi$ là số dương nhỏ nhất thỏa mãn $\tan(x + T) = \tan x$, $\forall x\in\mathcal{D}_1$, \& $T = \pi$ cũng là số dương nhỏ nhất thỏa mãn $\cot(x + T) = \cot x$, $\forall x\in\mathcal{D}_2$. Ta nói các hàm số $y = \tan x$ \& $y = \cot x$ là những \textit{hàm số tuần hoàn với chu kỳ $\pi$}.'' -- \cite[p. 10]{SGK_Toan_11_dai_so_giai_tich_nang_cao}

\subsubsection{Sự biến thiên \& đồ thị của hàm số $y = \tan x$}
``Do tính chất tuần hoanf với chu kỳ $\pi$ của hàm số $y = \tan x$, ta chỉ cần khảo sát sự biến thiên \& vẽ đồ thị của nó trên khoảng $\left(-\frac{\pi}{2};\frac{\pi}{2}\right)\subset\mathcal{D}_1$, rồi tịnh tiến phần đồ thị vừa vẽ sang trái, sang phải các đoạn của độ dài $\pi,2\pi,3\pi,\ldots$ thì được toàn bộ đồ thị của hàm số $y = \tan x$.
\begin{itemize}
	\item \textit{Chiều biến thiên}:
	
	\begin{figure}[H]
		\centering
		\includegraphics[scale=0.15]{chieu_bien_thien_tan}
		\caption{Chiều biến thiên của hàm $y = \tan x$.}
		\label{fig:chieu bien thien tan}
	\end{figure}
	Khi cho $x = (OA,OM)$ tăng từ $-\frac{\pi}{2}$ đến $\frac{\pi}{2}$ (không kể $\pm\frac{\pi}{2}$) thì điểm $M$ chạy trên đường tròn lượng giác theo chiều dương từ $B'$ đến $B$ (không kể $B'$ \& $B$). Khi đó điểm $T$ thuộc trục tang $At$ sao cho $\overline{AT} = \tan x$ chạy dọc theo $At$ suốt từ dưới lên trên, nên $\tan x$ \textit{tăng từ $-\infty$ đến $+\infty$} (qua quá trị $0$ khi $x = 0$).''
	\item \textit{Đồ thị}: ``Đồ thị của hàm số $y = \tan x$ có dạng như ở hình \ref{fig:graph tan}.
	
	\begin{figure}[H]
		\centering
		\includegraphics[scale=0.15]{graph_tan}
		\caption{Đồ thị của hàm $y = \tan x$.}
		\label{fig:graph tan}
	\end{figure}

	\begin{nhanxet}
		\begin{enumerate}
			\item Khi $x$ thay đổi, hàm số $y = \tan x$ nhận mọi giá trị thực. Ta nói \emph{tập giá trị} của hàm số $y = \tan x$ là $\mathbb{R}$.
			\item Vì hàm số $y = \tan x$ là hàm số lẻ nên đồ thị của nó nhận gốc tọa độ làm tâm đối xứng.
			\item Hàm số $y = \tan x$ không xác định tại $x = \frac{\pi}{2} + k\pi$ ($k\in\mathbb{Z}$). Với mỗi $k\in\mathbb{Z}$, đường thẳng vuông góc với trục hành, đi qua điểm $\left(\frac{\pi}{2} + k\pi;0\right)$ gọi là 1 \emph{đường tiệm cận} của đồ thị hàm số $y = \tan x$. (Từ ``tiệm cận'' có nghĩa là ngày càng gần. E.g., nói đường thẳng $x = \frac{\pi}{2}$ là 1 đường tiệm cận của đồ thị hàm số $y = \tan x$ nhằm diễn tả tính chất: điểm $M$ trên đồ thị có hoành độ càng gần $\frac{\pi}{2}$ thì $M$ càng gần đường thẳng $x = \frac{\pi}{2}$).'' -- \cite[pp. 11--12]{SGK_Toan_11_dai_so_giai_tich_nang_cao}
		\end{enumerate}
	\end{nhanxet}	   
\end{itemize}

\subsubsection{Sự biến thiên \& đồ thị của hàm số $y = \cot x$}
``Hàm số $y = \cot x$ xác định trên $\mathcal{D}_2 = \mathbb{R}\backslash\{k\pi|k\in\mathbb{Z}\}$ là 1 hàm số tuần hoàn với chu kỳ $\pi$. Ta có thể khảo sát sự biến thiên \& vẽ đồ thị của nó tương tự như đã làm đối với hàm số $y = \tan x$. Đồ thị của hàm số $y = \cot x$ có dạng như hình \ref{fig:graph cot}.

\begin{figure}[H]
	\centering
	\includegraphics[scale=0.15]{graph_cot}
	\caption{Đồ thị của hàm $y = \cot x$.}
	\label{fig:graph cot}
\end{figure}
Nó nhận mỗi đường thẳng vuông góc với trục hoành, đi qua điểm $(k\pi;0)$, $k\in\mathbb{Z}$ làm 1 đường tiệm cận.'' -- \cite[p. 12]{SGK_Toan_11_dai_so_giai_tich_nang_cao}

\begin{table}[H]
	\centering
	\begin{tabular}{|p{9cm}|p{9cm}|}
		\hline
		\textbf{Hàm số $y = \tan x$} & \textbf{Hàm số $y = \cot x$} \\
		\hline
		Có tập xác định là $\mathcal{D}_1 = \mathbb{R}\backslash\left\{\frac{\pi}{2} + k\pi|k\in\mathbb{Z}\right\}$ & Có tập xác định là $\mathcal{D}_2 = \mathbb{R}\backslash\{k\pi|k\in\mathbb{Z}\}$ \\
		\hline
		Có tập giá trị là $\mathbb{R}$ & Có tập giá trị là $\mathbb{R}$ \\
		\hline
		Là hàm số lẻ & Là hàm số lẻ \\
		\hline
		Là hàm số tuần hoàn với chu kỳ $\pi$ & Là hàm số tuần hoàn với chu kỳ $\pi$ \\
		\hline
		Đồng biến trên mỗi khoảng $\left(-\frac{\pi}{2} + k\pi;\frac{\pi}{2} + k\pi\right)$, $k\in\mathbb{Z}$ & Nghịch biến trên mỗi khoảng $(k\pi;\pi + k\pi)$, $k\in\mathbb{Z}$ \\
		\hline
		Có đồ thị nhận mỗi đường thẳng $x = \frac{\pi}{2} + k\pi$ ($k\in\mathbb{Z}$) làm 1 đường tiệm cận & Có đồ thị nhận mỗi đường thẳng $x = k\pi$ ($k\in\mathbb{Z}$) làm 1 đường tiệm cận \\
		\hline
	\end{tabular}
	\caption{So sánh tính chất của 2 hàm số $y = \tan x$ \& $y = \cot x$.}
\end{table}

\subsection{Về khái niệm hàm số tuần hoàn}
``Các hàm số $y = \sin x$, $y = \cos x$ là những hàm số tuần hoàn với chu kỳ $2\pi$; các hàm số $y = \tan x$, $y = \cot x$ là những hàm số tuần hoàn với chu kỳ $\pi$. 1 cách tổng quát:

\begin{dinhnghia}[Hàm số tuần hoàn]
	Hàm số $y = f(x)$ xác định trên tập hợp $\mathcal{D}$ được gọi là \emph{hàm số tuần hoàn} nếu có số $T\ne 0$ sao cho với mọi $x\in\mathcal{D}$ ta có $x + T\in\mathcal{D}$, $x - T\in\mathcal{D}$ \& $f(x + T) = f(x)$. Nếu có số $T$ dương nhỏ nhất thỏa mãn các điều kiện trên thì hàm số đó được gọi là 1 \emph{hàm số tuần hoàn với chu kỳ $T$}.'' -- \cite[p. 13]{SGK_Toan_11_dai_so_giai_tich_nang_cao}
\end{dinhnghia}

\begin{vidu}
	Các hàm số có dạng $y = a\sin bx$, với $a,b\in\mathbb{R}^\star\coloneqq\mathbb{R}\backslash\{0\}$ là những hàm số tuần hoàn.
\end{vidu}

\subsection{Dao động điều hòa}
``Nhiều hiện tượng tự nhiên thay đổi có tính chất tuần hoàn (lặp đi lặp lại sau khoảng thời gian xác định) như: Chuyển động của các hành tinh trong hệ mặt trời, chuyển động của guồng nước quay, chuyển động của quả lắc đồng hồ, sự biến thiên của cường độ dòng điện xoay chiều, $\ldots$. Hiện tượng tuần hoàn đơn giản nhất là \textit{dao động điều hòa} được mô tả bởi hàm số $y = A\sin(\omega x + \alpha) + B$, trong đó $A,B,\omega$ \& $\alpha$ là những hằng số; $A$ \& $\omega$ khác $0$. Đó là hàm số tuần hoàn với chu kỳ $\frac{2\pi}{|\omega|}$; $|A|$ gọi là \textit{biên độ}. Đồ thị của nó là 1 \textit{đường hình sin} có được từ đồ thị của hàm số $y = A\sin\omega x$ bằng cách tịnh tiến thích hợp (theo vector $-\frac{\alpha}{\omega}\vec{i}$ rồi theo vector $B\vec{j}$, i.e., tịnh tiến theo vector $-\frac{\alpha}{\omega}\vec{i} + B\vec{j}$).'' -- \cite[pp. 15--16]{SGK_Toan_11_dai_so_giai_tich_nang_cao}

\subsection{Âm thanh}
``Âm thanh được tạo nên bởi sự thay đổi áp suất của môi trường vật chất (chất khí, chất lỏng, chất rắn) 1 cách tuần hoàn theo thời gian (dao động tuần hoàn) \& được lan truyền trong môi trường đó (sóng âm thanh).

Nếu dao động tuần hoàn ấy có chu kỳ $T$ (đo bằng đơn vị thời gian là giây) thì $\frac{1}{T}$ gọi là \textit{tần số} của dao động (i.e., số chu kỳ trong 1 giây); đơn vị của tần số là Hertz (abbr., Hz). Âm thanh tai người nghe được là dao động có tần số trong khoảng từ 17--20 Hz đến 20000 Hz. Dao động có tần số cao hơn 20000 Hz được gọi là \textit{siêu âm}.

Trong âm nhạc (nghệ thuật phối hợp các âm thanh) người ta thường dùng những nốt nhạc để ghi những âm có tần số xác định. Tần số dao động càng lớn thì âm càng cao. Khi tăng tần số 1 âm lên gấp đôi thì ta nói cao độ của âm đó được tăng thêm 1 quãng 8. Người ta thường chia quãng 8 đó thành 12 quãng bằng nhau, mỗi quãng gọi là 1 bán cung để đo chênh lệch cao độ giữa các âm (xem SGK Âm nhạc \& Mỹ thuật lớp 7). Với 2 âm cách nhau 1 bán cung, tỷ số các tần số của chúng bằng $\sqrt[12]{2}$; với 2 âm cách nhau 1 cung (i.e., 2 bán cung), tỷ số các tần số của chúng bằng $(\sqrt[12]{2})^2 = \sqrt[6]{2}$. Ở khuông nhạc dưới đây có ghi các nốt nhạc của 1 ``âm giai'' (quãng 8) cùng khoảng cách cao độ giữa 2 âm ứng với 2 nốt kề nhau. Âm \textit{la} của âm giai đó có tần số 440 Hz (do đó, e.g., âm \textit{si} kế đó có tần số $440\sqrt[6]{2}$ Hz).

\begin{figure}[H]
	\centering
	\includegraphics[scale=0.2]{khuong_nhac}
	\caption{Khuông nhạc.}
	\label{fig:khuong nhac}
\end{figure}
Trong âm nhạc, ngoài các âm riêng lẻ còn có hợp âm (kết hợp các âm thanh). Nhà toán học Pháp \href{https://en.wikipedia.org/wiki/Joseph_Fourier}{Joseph Fourier} (1768--1830) đã chứng minh rằng 1 hàm số tuần hoàn với chu kỳ $T$ có thể phân tích thành ``tổng'' của 1 hằng số với những hàm số tuần hoàn có đồ thị là những đường hình sin với chu kỳ $\frac{T}{n}$ ($n\in\mathbb{N}^\star$). Điều đó giúp ta hiểu sâu hơn về hợp âm, hòa âm, âm bội \& âm sắc.'' -- \cite[p. 18]{SGK_Toan_11_dai_so_giai_tich_nang_cao}

\section{Phương Trình Lượng Giác Cơ Bản -- Basic Trigonometric Equation}
``Trên thực tế, có nhiều bài toán dẫn đến việc giải các phương trình có 1 trong các dạng $\sin x = m$, $\cos x = m$, $\tan x = m$, \& $\cot x = m$, trong đó $x$ là ẩn số ($x\in\mathbb{R}$) \& $m$ là 1 số cho trước. Đó là các \textit{phương trình lượng giác cơ bản}.'' -- \cite[p. 19]{SGK_Toan_11_dai_so_giai_tich_nang_cao}

\subsection{Phương trình $\sin x = m$}
``Giả sử $m$ là 1 số đã cho. Xét phương trình
\begin{align}
	\label{sin}
	\tag{sin}
	\sin x = m.
\end{align}
Hiển nhiên phương trình \eqref{sin} xác định với mọi $x\in\mathbb{R}$. Ta đã biết $|\sin x|\le 1$ với mọi $x\in\mathbb{R}$. Do đó phương trình \eqref{sin} vô nghiệm khi $|m| > 1$. Mặt khác, khi $x$ thay đổi, $\sin x$ nhận mọi giá trị từ $-1$ đến $1$ nên phương trình \eqref{sin} luôn có nghiệm khi $|m|\le 1$.'' -- \cite[p. 20]{SGK_Toan_11_dai_so_giai_tich_nang_cao}
\begin{tcolorbox}
	Nếu $\alpha$ là 1 nghiệm của phương trình \eqref{sin}, i.e., $\sin\alpha = m$ thì
	\begin{equation}
		\label{root sin}
		\sin x = m\Leftrightarrow\left[\begin{split}
			x &= \alpha + k2\pi\\
			x &= \pi - \alpha + k2\pi
		\end{split}\right.\ (k\in\mathbb{Z}).
	\end{equation}
\end{tcolorbox}
``Ta nói rằng $x = \alpha + k2\pi$ \& $x = \pi - \alpha + k2\pi$ là 2 \textit{họ nghiệm} của phương trình \eqref{sin}.

Kể từ đây, để cho gọn ta quy ước rằng nếu trong 1 biểu thức nghiệm của phương trình lượng giác có chứa $k$ mà không giải thích gì thêm thì ta hiểu rằng $k$ nhận mọi giá trị thuộc $\mathbb{Z}$. E.g., $x = \alpha + k2\pi$ có nghĩa là $x$ lấy mọi giá trị thuộc tập hợp $\{\alpha,\alpha\pm 2\pi,\alpha\pm 4\pi,\alpha\pm 6\pi,\ldots\}$.'' -- \cite[p. 21]{SGK_Toan_11_dai_so_giai_tich_nang_cao}

``Trong mặt phẳng tọa độ, nếu vẽ đồ thị $(G)$ của hàm số $y = \sin x$ \& đường thẳng $(d)$: $y = m$ thì hoành độ mỗi giao điểm của $(d)$ \& $(G)$ (nếu có) là 1 nghiệm của phương trình $\sin x = m$.'' -- \cite[p. 22]{SGK_Toan_11_dai_so_giai_tich_nang_cao}

\begin{luuy}
	\begin{enumerate}
		\item ``Khi $m\in\{0;\pm 1\}$, công thức \eqref{root sin} có thể viết gọn như sau:
		\begin{align*}
			\sin x = 1\Leftrightarrow x = \frac{\pi}{2} + k2\pi,\ \sin x = -1\Leftrightarrow x = -\frac{\pi}{2} + k2\pi,\ \sin x = 0\Leftrightarrow x = k\pi.
		\end{align*}
		\item Dễ thấy rằng với $m$ cho trước mà $|m|\le 1$, phương trình $\sin x = m$ có đúng 1 nghiệm nằm trong đoạn $\left[-\frac{\pi}{2};\frac{\pi}{2}\right]$. Người ta thường ký hiệu đó là $\arcsin m$. Khi đó
		\begin{equation*}
			\boxed{\sin x = m \Leftrightarrow\left[\begin{split}
				x &= \arcsin m + k2\pi,\\
				x &= \pi - \arcsin m + k2\pi.
			\end{split}\right.}
		\end{equation*}
		\item Từ \eqref{root sin} ta thấy rằng: Nếu $\alpha$ \& $\beta$ là 2 số thực thì $\sin\beta = \sin\alpha$ khi \& chỉ khi có số nguyên $k$ để $\beta = \alpha + k2\pi$ hoặc $\beta = \pi - \alpha + k2\pi$, $k\in\mathbb{Z}$.'' -- \cite[pp. 22--23]{SGK_Toan_11_dai_so_giai_tich_nang_cao}
	\end{enumerate}
\end{luuy}

\subsection{Phương trình $\cos x = m$}
``Xét phương trình
\begin{align*}
	\label{cos}
	\tag{cos}
	\cos x = m,
\end{align*}
trong đó $m$ là 1 số cho trước. Hiển nhiên phương trình \eqref{cos} xác định với mọi $x\in\mathbb{R}$. Dễ thấy rằng: Khi $|m| > 1$, phương trình \eqref{cos} vô nghiệm. Khi $|m|\le 1$, phương trình (II) luôn có nghiệm. Để tìm tất cả các nghiệm của (II), trên trục côsin ta lấy điểm $H$ sao cho $\overline{OH} = m$. Gọi $(l)$ là đường thẳng đi qua $H$ \& vuông góc với trục côsin (Fig. \ref{fig:truc cos}).

\begin{figure}[H]
	\centering
	\includegraphics[scale=0.15]{truc_cos}
	\caption{Trục côsin.}
	\label{fig:truc cos}
\end{figure}
Do $|m|\le 1$ nên đường thẳng $(l)$ cắt đường tròn lượng giác tại 2 điểm $M_1$ \& $M_2$. 2 điểm này đối xứng với nhau qua trục côsin (chúng trùng nhau nếu $m = \pm 1$). Ta thấy số đo của các góc lượng giác $(OA,OM_1)$ \& $(OA,OM_2)$ là tất cả các nghiệm của \eqref{cos}. Nếu $\alpha$ là số đo của 1 góc trong chúng, nói cách khác, nếu $\alpha$ là 1 nghiệm của \eqref{cos} thì các góc đó có các số đo là $\pm\alpha + k2\pi$. Vậy ta có

\begin{tcolorbox}
	Nếu $\alpha$ là 1 nghiệm của phương trình \eqref{cos}, i.e., $\cos\alpha = m$ thì
	\begin{equation}
		\label{root cos}
		\cos x = m\Leftrightarrow\left[\begin{split}
			x &= \alpha + k2\pi,\\
			x &= -\alpha + k2\pi.
		\end{split}\right.
	\end{equation}
\end{tcolorbox}

\begin{luuy}
	\begin{enumerate}
		\item Đặc biệt, khi $m\in\{0;\pm 1\}$, công thức \eqref{root cos} có thể viết gọn như sau:
		\begin{align*}
			\cos x = 1\Leftrightarrow x = k2\pi,\ \cos x = -1\Leftrightarrow x = \pi + k2\pi,\ \cos x = 0\Leftrightarrow x = \frac{\pi}{2} + k\pi.
		\end{align*}
		\item Dễ thấy rằng với mọi số $m$ cho trước mà $|m|\le 1$, phương trình $\cos x = m$ có đúng 1 nghiệm nằm trong đoạn $[0;\pi]$. Người ta thường ký hiệu nghiệm đó là $\arccos m$. Khi đó
		\begin{equation*}
			\cos x = m\Leftrightarrow\left[\begin{split}
				x &= \arccos m + k2\pi,\\
				x &= -\arccos m + k2\pi,
			\end{split}\right.
		\end{equation*}
		mà cũng thường được viết là $x = \pm\arccos m + k2\pi$.
		\item Từ \eqref{root cos} ta thấy rằng: Nếu $\alpha$ \& $\beta$ là 2 số thực thì $\cos\beta = \cos\alpha$ khi \& chỉ khi có số nguyên $k$ để $\beta = \alpha + k2\pi$ hoặc $\beta = -\alpha + k2\pi$, $k\in\mathbb{Z}$.'' -- \cite[pp. 23--24]{SGK_Toan_11_dai_so_giai_tich_nang_cao}
	\end{enumerate}
\end{luuy}

\subsection{Phương trình $\tan x = m$}
``Cho $m$ là 1 số tùy ý. Xét phương trình
\begin{align}
	\label{tan}
	\tag{tan}
	\tan x = m.
\end{align}
Điều kiện xác định (ĐKXĐ) của phương trình \eqref{tan} là $\cos x\ne 0$. Ta đã biết, khi $x$ thay đổi, $\tan x$ nhận mọi giá trị từ $-\infty$ đến $+\infty$. Do đó phương trình \eqref{tan} luôn có nghiệm. Để tìm tất cả các nghiệm của \eqref{tan}, trên tục tang, ta lấy điểm $T$ sao cho $\overline{AT} = m$. Đường thẳng $OT$ cắt đường tròn lượng giác tại 2 điểm $M_1$ \& $M_2$ (Fig. \ref{fig:truc tan}).

\begin{figure}[H]
	\centering
	\includegraphics[scale=0.15]{truc_tan}
	\caption{Trục tang.}
	\label{fig:truc tan}
\end{figure}
Ta có: $\tan(OA,OM_1) = \tan(OA,OM_2) = \overline{AT} = m$. Gọi số đo của 1 trong các góc lượng giác $(OA,OM_1)$ \& $(OA,OM_2)$ là $\alpha$; i.e., $\alpha$ là 1 nghiệm nào đó của phương trình \eqref{tan}. Khi đó, các góc lượng giác $(OA,OM_1)$ \& $(OA,OM_2)$. Khi đó, các góc lượng giác $(OA,OM_1)$ \& $(OA,OM_2)$ có các số đo là $\alpha + k\pi$. Đó là tất cả các nghiệm của phương trình \eqref{tan} (hiển nhiên chúng thỏa mãn ĐKXĐ của \eqref{tan}). Vậy ta có:

\begin{tcolorbox}
	Nếu $\alpha$ là 1 nghiệm của phương trình \eqref{tan}, i.e., $\tan\alpha = m$ thì
	\begin{align}
		\label{root tan}
		\tan x = m\Leftrightarrow x = \alpha + k\pi.
	\end{align}
\end{tcolorbox}

\begin{luuy}
	\begin{enumerate}
		\item Dễ thấy rằng với mọi số $m\in\mathbb{R}$ cho trước, phương trình $\tan x = m$ có đúng 1 nghiệm nằm trong khoảng $\left(-\frac{\pi}{2};\frac{\pi}{2}\right)$. Người ta thường ký hiệu nghiệm đó là $\arctan m$. Khi đó
		\begin{align*}
			\boxed{\tan x = m\Leftrightarrow x = \arctan m + k\pi.}
		\end{align*}
		\item Từ \eqref{root tan} ta thấy rằng: Nếu $\alpha$ \& $\beta$ là 2 số thực mà $\tan\alpha$, $\tan\beta$ xác định thì $\tan\beta = \tan\alpha$ khi \& chỉ khi có số nguyên $k$ để $\beta = \alpha + k\pi$.'' -- \cite[pp. 25--26]{SGK_Toan_11_dai_so_giai_tich_nang_cao}
	\end{enumerate}
\end{luuy}

\subsection{Phương trình $\cot x = m$}
``Cho $m\in\mathbb{R}$ là 1 số tùy ý, xét phương trình
\begin{align}
	\label{cot}
	\tag{cot}
	\cot x = m.
\end{align}
ĐKXĐ của phương trình \eqref{cot} là $\sin x\ne 0$. Tương tự như đối với phương trình $\tan x = m$, ta có

\begin{tcolorbox}
	Nếu $\alpha$ là 1 nghiệm của phương trình \eqref{cot}, i.e., $\cot\alpha = m$ thì
	\begin{align}
		\label{root cot}
		\cot x = m\Leftrightarrow x = \alpha + k\pi.
	\end{align}
\end{tcolorbox}
'' -- \cite[pp. 26--27]{SGK_Toan_11_dai_so_giai_tich_nang_cao}

\begin{luuy}
	Dễ thấy rằng với mọi số $m\in\mathbb{R}$ cho trước, phương trình $\cot x = m$ có đúng 1 nghiệm nằm trong khoảng $(0;\pi)$. Người ta thường ký hiệu nghiệm đó là $\arccot m$. Khi đó:
	\begin{align*}
		\boxed{\cot x = m\Leftrightarrow x = \arccot m + k\pi.}
	\end{align*}
\end{luuy}

\subsection{1 số điều cần lưu ý}
\begin{enumerate}
	\item Khi đã cho số $m$, ta có thể tính được các giá trị $\arcsin m,\arccos m$ (với $|m|\le 1$), $\arctan m$ bằng máy tính bỏ túi với các phím $\sin^{-1},\cos^{-1}$ \& $\tan^{-1}$.
	\item $\arcsin m,\arccos m$ (với $|m|\le 1$), $\arctan m$ \& $\arccot m$ có giá trị là những số thực. Do đó ta viết, e.g., $\arctan 1 = \frac{\pi}{4}$ mà không viết $\arctan 1 = 45^\circ$.
	\item Khi xét các phương trình lượng giác ta đã coi ẩn số $x$ là số đo radian của các góc lượng giác. Trên thực tế, ta còn gặp những bài toán yêu cầu tìm số đo độ của các góc (cung) lượng giác sao cho sin (côsin, tang hoặc côtang) của chúng bằng số $m\in\mathbb{R}$ cho trước e.g. $\sin(x + 20^\circ) = \frac{\sqrt{3}}{2}$. Khi giải các phương trình này (mà làm dụng ngôn ngữ, ta vẫn gọi là giải các phương trình lượng giác), ta có thể áp dụng các công thức nêu trên \& lưu ý sử dụng ký hiệu số đo độ trong ``công thức nghiệm'' cho thống nhất, e.g., viết $x = 30^\circ + k360^\circ$ chứ không viết $x = 30^\circ + k2\pi$.
	
	Tuy nhiên, ta quy ước rằng nếu không có giải thích gì thêm hoặc trong phương trình lượng giác không sử dụng đơn vị đo góc là độ thì mặc nhiên ẩn số là số đo radian của góc lượng giác.'' -- \cite[p. 27]{SGK_Toan_11_dai_so_giai_tich_nang_cao}
\end{enumerate}

\subsection{Dùng máy tính bỏ túi để tìm 1 góc khi biết 1 giá trị lượng giác của nó}
``Các phím $\sin^{-1},\cos^{-1}$ \& $\tan^{-1}$ của máy tính bỏ túi \textsc{Casio} \textit{fx-500MS} được dùng để tìm số đo (độ hoặc radian) của 1 góc khi biết 1 trong các giá trị lượng giác của nó. Muốn thế đối với máy tính CASIO \textit{fx-500MS} ta thực hiện 2 bước sau:
\begin{enumerate}
	\item \textit{Ấn định đơn vị đo góc (độ hoặc radian)}. Muốn tìm số đo độ, ta ấn \fbox{MODE} \fbox{MODE} \fbox{MODE} \fbox{1}. Lúc này dòng trên cùng của màn hình xuất hiện chữ nhỏ \fbox{D}. Muốn tìm số đo radian, ta ấn \fbox{MODE} \fbox{MODE} \fbox{MODE} \fbox{2}. Lúc này dòng trên cùng của màn hình xuất hiện chữ nhỏ \fbox{R}.
	\item \textit{Tìm số đo góc}. Khi biết sin, côsin hay tang của góc $\alpha$ cần tìm bằng $m$, ta lần lượt ấn phím \fbox{SHIFT}, \& 1 trong các phím $\boxed{\sin^{-1}}$, $\boxed{\cos^{-1}}$, $\boxed{\tan^{-1}}$, rồi nhập giá trị lượng giác $m$ \& cuối cùng ấn phím $=$. Lúc này, trên màn hình cho kết quả là số đo của góc $\alpha$ (độ hay radian tùy theo bước 1).
\end{enumerate}

\begin{luuy}
	\begin{enumerate}
		\item Ở chế độ số đo radian, các phím $\sin^{-1},\cos^{-1}$ cho kết quả (khi $|m|\le 1$) là $\arcsin m,\arccos m$; phím $\tan^{-1}$ cho kết quả là $\arctan m$.
		\item Ở chế độ số đo độ, các phím $\sin^{-1}$ \& $\tan^{-1}$ cho kết quả là số đo góc $\alpha$ từ $-90^\circ$ đến $90^\circ$; phím $\cos^{-1}$ cho kết quả là số đo góc $\alpha$ từ $0^\circ$ đến $180^\circ$. Các kết quả ấy được hiển thị dưới dạng số thập phân.'' -- \cite[p. 27]{SGK_Toan_11_dai_so_giai_tich_nang_cao}
	\end{enumerate}
\end{luuy}
Xem \cite[Ví dụ 1--3, p. 31]{SGK_Toan_11_dai_so_giai_tich_nang_cao} để biết chi tiết thao tác bấm phím trên máy tính cầm tay.

\section{1 Số Dạng Phương Trình Lượng Giác Cơ Bản}

\subsection{1 số dạng phương trình lượng giác đơn giản}

\subsubsection{Phương trình bậc nhất \& bậc 2 đối với 1 hàm số lượng giác}
Để giải các phương trình lượng giác có dạng $P(\sin x) = 0$, $P(\cos x) = 0$, $P(\tan x) = 0$, $P(\cot x) = 0$ với $P$ là 1 đa thức có bậc 1 hoặc 2 (i.e., $\deg P\in\{1,2\}$ \footnote{Ký $\deg$ là viết tắt của từ ``degree'' tức là ``bậc''.}), ta chọn 1 biểu thức lượng giác thích hợp có mặt trong phương trình làm ẩn phụ \& quy về phương trình bậc nhất hoặc bậc 2 đối với ẩn phụ đó (có thể nêu hoặc không nêu ký hiệu ẩn phụ).

\paragraph{Phương trình bậc nhất đối với 1 hàm số lượng giác.} Xét các phương trình lượng giác có dạng $P(\sin x) = 0$, $P(\cos x) = 0$, $P(\tan x) = 0$, $P(\cot x) = 0$ với $P$ là 1 đa thức có bậc 1 (i.e., $\deg P = 1$), i.e.:
\begin{align*}
	a\sin(mx + n) + b = 0,\ a\cos(mx + n) + b = 0,\ a\tan(mx + n) + b = 0,\ a\cot(mx + n) + b = 0,\ a,b,m,n\in\mathbb{R},\,a\ne 0,\,m\ne 0.
\end{align*}
Tổng quát hơn, giải các phương trình lượng giác sau:
\begin{align*}
	a\sin f(x) + b = 0,\ a\cos f(x) + b = 0,\ a\tan f(x) + b = 0,\ a\cot f(x) + b = 0,
\end{align*}
trong đó $a,b\in\mathbb{R}$, $a\ne 0$, \& $f$ là 1 hàm số (đa thức, phân thức, hàm căn thức) sao cho phương trình $f(x) = m$ có thể giải được\texttt{/}solvable (có nghiệm hoặc vô nghiệm) trên tập số thực $\mathbb{R}$.

\paragraph{Phương trình bậc 2 đối với 1 hàm số lượng giác.} Xét các phương trình lượng giác có dạng $P(\sin x) = 0$, $P(\cos x) = 0$, $P(\tan x) = 0$, $P(\cot x) = 0$ với $P$ là 1 đa thức có bậc 2 (i.e., $\deg P = 2$), i.e.,
\begin{align*}
	a\sin^2(mx + n) + b\sin(mx + n) + c &= 0,\ a\cos^2(mx + n) + b\cos(mx + n) + c = 0,\\
	a\tan^2(mx + n) + b\tan(mx + n) + c &= 0,\ a\cot^2(mx + n) + b\tan(mx + n) + c = 0,\ 
\end{align*}
trong đó $a,b,c,m,n\in\mathbb{R}$, $a\ne 0$, $m\ne 0$. Tổng quát hơn, giải các phương trình lượng giác sau:
\begin{align*}
	a\sin^2f(x) + b\sin f(x) + c &= 0,\ a\cos^2f(x) + b\cos f(x) + c = 0,\\
	a\tan^2f(x) + b\tan f(x) + c &= 0,\ a\cot^2f(x) + b\cot f(x) + c = 0,\ ,
\end{align*}
trong đó $a,b,c\in\mathbb{R}$, $a\ne 0$, \& $f$ là 1 hàm số (đa thức, phân thức, hàm căn thức) sao cho phương trình $f(x) = m$ có thể giải được (solvable) trên tập số thực $\mathbb{R}$. 

\paragraph{Phương trình bậc $n\in\mathbb{N}$ đối với 1 hàm số lượng giác.} Xét các phương trình lượng giác có dạng $P(\sin x) = 0$, $P(\cos x) = 0$, $P(\tan x) = 0$, $P(\cot x) = 0$ với $P$ là 1 đa thức có bậc $n\in\mathbb{N}$ (i.e., $\deg P = n$), i.e., với $P(x) = \sum_{i=0}^n a_ix^i$, $a_i\in\mathbb{R}$, $i = 0,\ldots,n$, hệ số cao nhất $a_n\ne 0$, xét các phương trình lượng giác có dạng
\begin{align*}
	P(\sin(mx + n)) &= \sum_{i=0}^n a_i\sin^i(mx + n) = 0,\ P(\cos(mx + n)) = \sum_{i=0}^n a_i\cos^i(mx + n)  = 0,\\
	P(\tan(mx + n)) &= \sum_{i=0}^n a_i\tan^i(mx + n)  = 0,\ P(\cot(mx + n)) = \sum_{i=0}^n a_i\cot^i(mx + n) = 0.
\end{align*}
Tổng quát hơn, giải các phương trình lượng giác sau:
\begin{align*}
	P(\sin f(x)) &= \sum_{i=0}^n a_i\sin^if(x) = 0,\ P(\cos f(x)) = \sum_{i=0}^n a_i\cos^if(x) = 0,\\
	P(\tan f(x)) &= \sum_{i=0}^n a_i\tan^if(x) = 0,\ P(\cot f(x)) = \sum_{i=0}^n a_i\cot^if(x) = 0.
\end{align*}
Về đa thức tổng quát bậc $n$ \& các tính chất liên quan, có thể xem các tài liệu chuyên khảo về đa thức hoặc phần đầu của tài liệu của tác giả cho chương trình Toán lớp 8 ở link sau: \href{https://github.com/NQBH/hobby/blob/master/elementary_mathematics/grade_8/NQBH_elementary_mathematics_grade_8.pdf}{GitHub\texttt{/}NQBH\texttt{/}hobby\texttt{/}elementary mathematics\texttt{/}grade 8\texttt{/}lecture}\footnote{Explicitly, \url{https://github.com/NQBH/hobby/blob/master/elementary_mathematics/grade_8/NQBH_elementary_mathematics_grade_8.pdf}.}.



%------------------------------------------------------------------------------%

\chapter{Tổ Hợp \& Xác Suất}

\section{2 Quy Tắc Đếm Cơ Bản}

\section{Hoán Vị, Chỉnh Hợp \& Tổ Hợp}

\section{Nhị Thức Newton}

\section{Biến Cố \& Xác Suất của Biến Cố}

\section{Các Quy Tắc Tính Xác Suất}

\section{Biến Ngẫu Nhiên Rời Rạc}

%------------------------------------------------------------------------------%

\chapter{Dãy Số. Cấp Số Cộng \& Cấp Số Nhân}

\section{Phương Pháp Quy Nạp Toán Học}

\section{Dãy Số}

\section{Cấp Số Cộng}

\section{Cấp Số Nhân}

%------------------------------------------------------------------------------%

\chapter{Giới Hạn}

\section{Dãy Số Có Giới Hạn $0$}

\section{Dãy Số Có Giới Hạn Hữu Hạn}

\section{Dãy Số Có Giới Hạn Vô Cực}

\section{Định Nghĩa \& 1 Số Định Lý về Giới Hạn của Hàm Số}

\section{Giới Hạn 1 Bên}

\section{1 Vài Quy Tắc Tìm Giới Hạn Vô Cực}

\section{Các Dạng Vô Hình}

\section{Hàm Số Liên Tục}

%------------------------------------------------------------------------------%

\chapter{Đạo Hàm}

\section{Khái Niệm Đạo Hàm}

\section{Các Quy Tắc Tính Đạo Hàm}

\section{Đạo Hàm của Các Hàm Số Lượng Giác}

\section{Vi Phân}

\section{Đạo Hàm Cấp Cao}

%------------------------------------------------------------------------------%

\part{Hình Học -- Geometry}

\chapter{Phép Dời Hình \& Phép Đồng Dạng Trong Mặt Phẳng}

``Bức tranh của họa sĩ Hà Lan M.C. Escher gồm những hình bằng nhau mô tả các chiến binh trên lưng ngựa. Các hình này phủ kín mặt phẳng. 2 chiến binh \& ngựa cùng màu (trắng hoặc đen) tương ứng với nhau qua 1 phép tịnh tiến. 2 chiến binh \& ngựa khác màu thì tương ứng với nhau qua 1 phép đối xứng trục \& tiếp theo là 1 phép tịnh tiến. Nghệ thuật dùng những hình bằng nhau để lấp đầy mặt phẳng được phát triển mạnh mẽ vào thế kỷ XIII ở nước Ý\texttt{/}Italia.'' -- \cite[p. 3]{SGK_Toan_11_hinh_hoc_nang_cao}

\begin{quotation}
	\textbf{Nội dung.} \textit{Các phép dời hình \& đồng dạng trong mặt phẳng: phép tịnh tiến, phép đối xứng trục, phép quay, phép vị tự, $\ldots$; 2 hình bằng nhau, 2 hình đồng dạng 1 cách tổng quát}.
\end{quotation}

\section{Mở Đầu về Phép Biến Hình}

\subsection{Phép biến hình}
Khái niệm ``hàm số'' -- 1 khái niệm quan trọng trong Đại số: ``Nếu có 1 quy tắc để với mỗi số $x\in\mathbb{R}$, xác định được 1 số duy nhất $y\in\mathbb{R}$ thì quy tắc đó gọi là \textit{1 hàm số xác định trên tập số thực $\mathbb{R}$}. Bây giờ, trong mệnh đề trên ta thay \textit{số thực} bằng \textit{điểm thuộc mặt phẳng} thì ta được khái niệm về phép biến hình trong mặt phẳng. Cụ thể là: Nếu có 1 quy tắc để với mỗi điểm $M$ thuộc mặt phẳng, xác định được 1 điểm duy nhất $M'$ thuộc mặt phẳng ấy thì quy tắc đó gọi là \textit{1 phép biến hình (trong mặt phẳng)}.'' -- \cite[p. 4]{SGK_Toan_11_hinh_hoc_nang_cao}

\begin{dinhnghia}[Phép biến hình]
	\emph{Phép biến hình} (trong mặt phẳng) là 1 quy tắc để với mỗi điểm $M$ thuộc mặt phẳng, xác định được 1 điểm duy nhất $M'$ thuộc mặt phẳng ấy. Điểm $M'$ gọi là \emph{ảnh} của điểm $M$ qua phép biến hình đó.
\end{dinhnghia}

\begin{vidu}[Phép chiếu vuông góc lên 1 đường thẳng]
	``Cho đường thẳng $d$. Với mỗi điểm $M$, ta xác định $M'$ là hình chiếu (vuông góc) của $M$ trên $d$ thì ta được 1 phép biến hình.
	
	\begin{figure}[H]
		\centering
		\includegraphics[scale=0.15]{phep_chieu_vuong_goc}
		\caption{Phép chiếu vuông góc lên đường thẳng $d$.}
		\label{fig:phep chieu vuong goc len duong thang}
	\end{figure}
	Phép biến hình này gọi là \emph{phép chiếu (vuông góc) lên đường thẳng $d$}.'' -- \cite[p. 4]{SGK_Toan_11_hinh_hoc_nang_cao}
\end{vidu}

\begin{vidu}
	``Cho vector $\vec{u}$, với mỗi điểm $M$ ta xác định điểm $M'$ theo quy tắc $\overrightarrow{MM'} = \vec{u}$ (Fig. \ref{fig:phep tinh tien theo vector}).
	
	\begin{figure}[H]
		\centering
		\includegraphics[scale=0.15]{phep_tinh_tien_theo_vector}
		\caption{Phép tịnh tiến theo vector $\vec{u}$.}
		\label{fig:phep tinh tien theo vector}
	\end{figure}
	Như vậy ta cũng có 1 phép biến hình. Phép biến hình đó gọi là \emph{phép tịnh tiến theo vector $\vec{u}$}.'' -- \cite[p. 4]{SGK_Toan_11_hinh_hoc_nang_cao}
\end{vidu}

\begin{vidu}[Phép đồng nhất]
	Với mỗi điểm $M$, ta xác định điểm $M'$ trùng với $M$ thì ta cũng được 1 phép biến hình.
\end{vidu}

\subsection{Ký hiệu \& thuật ngữ}
``Nếu ta ký hiệu 1 phép biến hình nào đó là $F$ \& điểm $M'$ là ảnh của điểm $M$ qua phép biến hình $F$ thì ta viết $M' = F(M)$, hoặc $F(M) = M'$. Khi đó, ta còn nói \textit{phép biến hình $F$ biến điểm $M$ thành điểm $M'$}.

Với mỗi hình $\mathcal{H}$, ta gọi hình $\mathcal{H}'$ gồm các điểm $M' = F(M)$, trong đó $M\in\mathcal{H}$, là \textit{ảnh của $\mathcal{H}$ qua phép biến hình $F$}, \& viết $\mathcal{H}' = F(\mathcal{H})$.'' -- \cite[p. 5]{SGK_Toan_11_hinh_hoc_nang_cao}

\section{Phép Tịnh Tiến \& Phép Dời Hình}

\subsection{Định nghĩa phép tịnh tiến}

\begin{dinhnghia}[Phép tịnh tiến]
	\emph{Phép tịnh tiến} theo vector $\vec{u}$ là 1 phép biến hình biến điểm $M$ thành điểm $M'$ sao cho $\overrightarrow{MM'} = \vec{u}$.
\end{dinhnghia}
``Phép tịnh tiến theo vector $\vec{u}$ thường được ký hiệu là $T$ hoặc $T_{\vec{u}}$. Vector $\vec{u}$ được gọi là \textit{vector tịnh tiến}.'' -- \cite[p. 5]{SGK_Toan_11_hinh_hoc_nang_cao}. Phép đồng nhất là phép tịnh tiến theo vector $\vec{u} = \vec{0}$.

\subsection{Các tính chất của phép tịnh tiến}

\begin{dinhly}[Phép tịnh tiến bảo toàn khoảng cách]
	\label{thm: phep tinh tien bao toan khoang cach}
	Nếu phép tịnh tiến biến 2 điểm $M$ \& $N$ lần lượt thành 2 điểm $M'$ \& $N'$ thì $M'N' = MN$.
\end{dinhly}
``Người ta diễn tả tính chất trên của phép tịnh tiến là: \textit{Phép tịnh tiến không làm thay đổi khoảng cách giữa 2 điểm bất kỳ}.'' -- \cite[p. 6]{SGK_Toan_11_hinh_hoc_nang_cao}

\begin{dinhly}[Phép tịnh tiến bảo toàn tính chất thẳng hàng \& thứ tự các điểm thẳng hàng]
	Phép tịnh tiến biến 3 điểm thẳng hàng thành 3 điểm thẳng hàng \& không làm thay đổi thứ tự 3 điểm đó.
\end{dinhly}

\begin{proof}[Chứng minh]
	``Giả sử phép tịnh tiến biến 3 điểm $A,B,C$ thành 3 điểm $A',B',C'$. Theo Định lý \ref{thm: phep tinh tien bao toan khoang cach}, ta có $A'B' = AB$, $B'C' = BC$, \& $A'C' = AC$. Nếu $A,B,C$ thẳng hàng, $B$ nằm giữa $A$ \& $C$ thì $AB + AC = AC$. Do đó ta cùng có $A'B' + B'C' = A'C'$, i.e., $A',B',C'$ thẳng hàng, trong đó $B'$ nằm giữa $A'$ \& $C'$.
\end{proof}

\begin{hequa}
	Phép tịnh tiến biến đường thẳng thành đường thẳng, biến tia thành tia, biến đoạn thẳng thành đoạn thẳng bằng nó, biến tam giác thành tam giác bằng nó, biến đường tròn thành đường tròn có cùng bán kính, biến góc thành góc bằng nó.
\end{hequa}

\subsection{Biểu thức tọa độ của phép tịnh tiến}
``Trong mặt phẳng với hệ trục tọa độ $Oxy$, cho phép tịnh tiến theo vector $\vec{u}$. Biết tọa độ của $\vec{u}$ là $(a;b)$. Giả sử điểm $M(x;y)$ biến thành điểm $M(x';y')$ (Fig. \ref{fig:phep tinh tien tren he truc toa do}).

\begin{figure}[H]
	\centering
	\includegraphics[scale=0.15]{phep_tinh_tien_tren_he_truc_toa_do}
	\caption{Phép tịnh tiến trên hệ trục tọa độ.}
	\label{fig:phep tinh tien tren he truc toa do}
\end{figure}
Khi đó ta có:
\begin{equation*}
	\boxed{\left\{\begin{split}
		x' &= x + a,\\
		y' &= y + b.
	\end{split}\right.}
\end{equation*}
Công thức trên gọi là \textit{biểu thức tọa độ của phép tịnh tiến theo vector $\vec{u}(a;b)$}.'' -- \cite[pp. 6--7]{SGK_Toan_11_hinh_hoc_nang_cao}

\subsection{Ứng dụng của phép tịnh tiến}

\begin{baitoan}[\cite{SGK_Toan_11_hinh_hoc_nang_cao}, p. 7]
	Cho 2 điểm $B,C$ cố định trên đường tròn $(O;R)$ \& 1 điểm $A$ thay đổi trên đường tròn đó. Chứng minh rằng trực tâm $\Delta ABC$ nằm trên 1 đường tròn cố định.
\end{baitoan}

\begin{proof}[Giải]
	Nếu $BC$ là đường kính thì trực tâm $H$ của $\Delta ABC$ chính là $A$. Vậy $H$ nằm trên đường tròn cố định $(O;R)$. Nếu $BC$ không phải là đường kính, vẽ đường kính $BB'$ của đường tròn. Nếu $H$ là trực tâm của $\Delta ABC$ thì $\overrightarrow{AH} = \overrightarrow{B'C}$ (suy ra từ nhận xét tứ giác $AHCB'$ là hình bình hành). Như vậy, phép tịnh tiến theo vector cố định $\overrightarrow{B'C}$ biến điểm $A$ thành điểm $H$. Do đó, khi $A$ thay đổi trên $(O;R)$ thì trực tâm $H$ luôn nằm trên đường tròn cố định là ảnh của đường tròn $(O;R)$ qua phép tịnh tiến nói trên.
\end{proof}

\begin{baitoan}
	2 thôn nằm ở 2 vị trí $A$ \& $B$ cách nhau 1 con sông (xem rằng 2 bờ sông là 2 đường thẳng song song). Người ta dự định xây 1 chiếc cầu $MN$ bắt qua sông (cố nhiên cầu phải vuông góc với bờ sông) \& làm 2 đoạn thẳng từ $A$ đến $M$ \& từ $B$ đến $N$. Hãy xác định vị trí chiếc cầu $MN$ sao cho $AM + BN$ ngắn nhất.
\end{baitoan}

\begin{proof}[Hint]
	Trường hợp tổng quát có thể đưa về trường hợp con sông rất hẹp -- hẹp đến mức 2 bờ sông $a$ \& $b$ xem như trùng nhau bằng 1 phép tịnh tiến theo vector $\overrightarrow{MN}$ để $a$ trùng $b$. Khi đó điểm $A$ biến thành điểm $A'$ sao cho $\overrightarrow{AA'} = \overrightarrow{MN}$ \& do đó $A'N = AM$.
\end{proof}

\subsection{Phép dời hình}
``Không phải chỉ có phép tịnh tiến ``không làm thay đổi khoảng cách giữa 2 điểm'' mà còn nhiều phép biến hình khác cũng có tính chất đó (tính chất này còn được gọi là tính chất \textit{bảo toàn khoảng cách} giữa 2 điểm). Người ta gọi các phép biến hình như vậy là phép dời hình.

\begin{dinhnghia}[Phép dời hình]
	\emph{Phép dời hình} là phép biến hình không làm thay đổi khoảng cách giữa 2 điểm bất kỳ.
\end{dinhnghia}
Chú ý rằng các tính chất đã nêu của phép tịnh tiến được chứng minh dựa vào tính chất ``\textit{không làm thay đổi khoảng cách giữa 2 điểm}''. Bởi vậy, các phép dời hình cũng có những tính chất đó. Cụ thể ta có:

\begin{dinhly}
	Phép dời hình biến 3 điểm thẳng hàng thành 3 điểm thẳng hàng \& không làm thay đổi thứ tự 3 điểm đó, biến đường thằng thành đường thẳng, biến tia thành tia, biến đoạn thẳng thành đoạn thẳng bằng nó, biến tam giác thành tam giác bằng nó, biến đường tròn thành đường tròn có cùng bán kính, biến góc thành góc bằng nó.
\end{dinhly}
'' -- \cite[p. 8]{SGK_Toan_11_hinh_hoc_nang_cao}

\section{Phép Đối Xứng Trục}

\section{Phép Quay \& Phép Đối Xứng Tâm}

\section{2 Hình bằng Nhau}

\section{Phép Vị Tự}

\section{Phép Đồng Dạng}

\section{Hình Tự Đồng Dạng \& Hình Học Fractal}

%------------------------------------------------------------------------------%

\chapter{Đường Thẳng \& Mặt Phẳng Trong Không Gian}

\section{Đại Cương về Đường Thẳng \& Mặt Phẳng}

\section{2 Đường Thẳng Song Song}

\section{Đường Thẳng Song Song với Mặt Phẳng}

\section{2 Mặt Phẳng Song Song}

\section{Phép Chiếu Song Song}

\section{Phương Pháp Tiên Đề Trong Hình Học}

%------------------------------------------------------------------------------%

\chapter{Vector Trong Không Gian. Quan Hệ Vuông Góc}

\section{Vector Trong Không Gian. Sự Đồng Phẳng của Các Vector}

\section{2 Đường Thẳng Vuông Góc}

\section{Đường Thẳng Vuông Góc với Mặt Phẳng}

\section{2 Mặt Phẳng Vuông Góc}

\section{Khoảng Cách}

%------------------------------------------------------------------------------%

\appendix

\chapter{Phụ Lục -- Appendices}

\section{Hàm Số Chẵn \& Hàm Số Lẻ -- Even \& Odd Functions}
\label{sect: even & odd functions}
``Trong toán học, \textit{hàm số chẵn} \& \textit{hàm số lẻ} là các \href{https://vi.wikipedia.org/wiki/H%C3%A0m_s%E1%BB%91}{hàm số} thỏa mãn các quan hệ \href{https://vi.wikipedia.org/wiki/%C4%90%E1%BB%91i_x%E1%BB%A9ng}{đối xứng} nhất định khi lấy \href{https://vi.wikipedia.org/wiki/Ngh%E1%BB%8Bch_%C4%91%E1%BA%A3o_ph%C3%A9p_c%E1%BB%99ng}{nghịch đảo phép cộng}. Chúng rất quan trọng trong nhiều lĩnh vực của \href{https://vi.wikipedia.org/wiki/Gi%E1%BA%A3i_t%C3%ADch_to%C3%A1n}{giải tích toán}, đặc biệt trong lý thuyết chuỗi lũy thừa \& \href{https://vi.wikipedia.org/wiki/Chu%E1%BB%97i_Fourier}{chuỗi Fourier}. Chúng được đặt tên theo \href{https://vi.wikipedia.org/wiki/T%C3%ADnh_ch%E1%BA%B5n_l%E1%BA%BB}{tính chẵn lẻ} của số mũ lũy thừa của \href{https://vi.wikipedia.org/wiki/L%C5%A9y_th%E1%BB%ABa}{hàm lũy thừa} thỏa mãn từng điều kiện: hàm số $f(x) = x^n$ là 1 hàm chẵn nếu $n$ là 1 số nguyên chẵn, \& nó là hàm lẻ nếu $n$ là 1 số nguyên lẻ.'' -- \href{https://vi.wikipedia.org/wiki/H%C3%A0m_s%E1%BB%91_ch%E1%BA%B5n_v%C3%A0_l%E1%BA%BB}{Wikipedia\texttt{/}hàm số chẵn \& lẻ}

\subsection{Hàm số chẵn -- Even function}

\begin{dinhnghia}[Hàm số chẵn]
	``Cho $f$ là 1 hàm số giá trị thực của 1 đối số thực, $f$ là \emph{hàm số chẵn} nếu điều kiện sau được thỏa mãn với mọi $x$ sao cho cả $x$ \& $-x$ đều thuộc miền xác định của $f$: $f(x) = f(-x)$, $\forall x\in\operatorname{dom}(f)$, với $\operatorname{dom}(f)$ ký hiệu miền xác định của $f$, hoặc phát biểu 1 cách tương đương, nếu phương trình sau thỏa mãn $f(x) - f(-x) = 0$, $\forall x\in\operatorname{dom}(f)$.
\end{dinhnghia}
Về mặt hình học, đồ thị của 1 hàm số chẵn \href{https://vi.wikipedia.org/wiki/%C4%90%E1%BB%91i_x%E1%BB%A9ng}{đối xứng} qua trục $y$, nghĩa là đồ thị của nó giữ không đổi sau phép \href{https://vi.wikipedia.org/wiki/%C4%90%E1%BB%91i_x%E1%BB%A9ng_tr%E1%BB%A5c}{lấy đối xứng qua trục $y$}.'' -- \href{https://vi.wikipedia.org/wiki/H%C3%A0m_s%E1%BB%91_ch%E1%BA%B5n_v%C3%A0_l%E1%BA%BB}{Wikipedia\texttt{/}hàm số chẵn \& lẻ}

\begin{vidu}[Hàm chẵn]
	\href{https://vi.wikipedia.org/wiki/Gi%C3%A1_tr%E1%BB%8B_tuy%E1%BB%87t_%C4%91%E1%BB%91i}{Hàm trị tuyệt đối} $x\mapsto|x|$, các hàm đơn thức dạng $x\mapsto x^{2n}$, \href{https://vi.wikipedia.org/wiki/H%C3%A0m_l%C6%B0%E1%BB%A3ng_gi%C3%A1c}{hàm cosin} $\cos$, \href{https://vi.wikipedia.org/wiki/H%C3%A0m_hyperbolic}{hàm cosin hyperbolic} $\cosh$.
\end{vidu}

\subsection{Hàm số chẵn -- Odd function}

\begin{dinhnghia}[Hàm số lẻ]
	Cho $f$ là 1 hàm số giá trị thực của 1 đối số (biến) thực, $f$ là hàm số \emph{lẻ} nếu điều kiện sau được thỏa mãn với mọi $x$ sao cho cả $x$ \& $-x$ đều thuộc miền xác định của $f$: $f(-x) = -f(x)$, $\forall x\in\operatorname{dom}(f)$, với $\operatorname{dom}(f)$ ký hiệu miền xác định của $f$, hoặc phát biểu 1 cách tương đương, nếu phương trình sau thỏa mãn $f(x) + f(-x) = 0$, $\forall x\in\operatorname{dom}(f)$.
\end{dinhnghia}
``Về mặt hình học, đồ thị của 1 hàm lẻ có tính đối xứng tâm quay qua gốc tọa độ, i.e., đồ thị của nó không đổi sau khi thực hiện phép quay $180^\circ$ quanh điểm gốc.'' -- \href{https://vi.wikipedia.org/wiki/H%C3%A0m_s%E1%BB%91_ch%E1%BA%B5n_v%C3%A0_l%E1%BA%BB}{Wikipedia\texttt{/}hàm số chẵn \& lẻ}

\begin{vidu}[Hàm số lẻ]
	Hàm đồng nhất $x\mapsto x$, các hàm đơn thức dạng $x\mapsto x^{2n + 1}$, \href{https://vi.wikipedia.org/wiki/Sin}{hàm sin} $\sin$, \href{https://vi.wikipedia.org/wiki/H%C3%A0m_hyperbolic}{hàm sin hyperbol} $\sinh$, \href{https://vi.wikipedia.org/wiki/H%C3%A0m_l%E1%BB%97i}{hàm lỗi} $\operatorname{erf}$.
\end{vidu}

\subsection{Các tính chất cơ bản}

\subsubsection{Tính duy nhất}
\begin{itemize}
	\item ``Nếu 1 hàm số vừa chẵn \& vừa lẻ, nó bằng $0$ ở mọi điểm mà nó được xác định.
	\item Nếu 1 hàm là lẻ thì \href{https://vi.wikipedia.org/wiki/Gi%C3%A1_tr%E1%BB%8B_tuy%E1%BB%87t_%C4%91%E1%BB%91i}{giá trị tuyệt đối} của hàm đó là 1 hàm chẵn.'' -- \href{https://vi.wikipedia.org/wiki/H%C3%A0m_s%E1%BB%91_ch%E1%BA%B5n_v%C3%A0_l%E1%BA%BB}{Wikipedia\texttt{/}hàm số chẵn \& lẻ}
\end{itemize}

\subsubsection{Cộng \& trừ hàm số chẵn lẻ}
\begin{itemize}
	\item Tổng \& hiệu của 2 hàm số chẵn là 2 hàm số chẵn.
	\item Tổng \& hiệu của 2 hàm lẻ là 2 hàm lẻ.
	\item Tổng của 1 hàm chẵn \& 1 hàm lẻ thì không chẵn cũng không lẻ, trừ khi 1 trong các hàm ấy bằng $0$ trên miền đã cho.
\end{itemize}

\subsubsection{Nhân \& chia hàm số chẵn lẻ}
\begin{itemize}
	\item Tích \& thương của 2 hàm chẵn là 2 hàm chẵn.
	\item Tích \& thương của 2 hàm lẻ là 2 hàm chẵn.
	\item Tích \& thương của 1 hàm chẵn với 1 hàm lẻ là 2 hàm lẻ.
\end{itemize}

\subsubsection{Hàm hợp (tích ánh xạ)}
\begin{itemize}
	\item Hàm hợp của 2 hàm chẵn là hàm chẵn.
	\item Hàm hợp của 2 hàm lẻ là hàm lẻ.
	\item 1 hàm chẵn hợp với 1 hàm lẻ là hàm chẵn.
	\item Hàm hợp của bất kỳ hàm số nào với 1 hàm chẵn là hàm chẵn (nhưng điều ngược lại không đúng).
\end{itemize}

\subsection{Phân tích chẵn--lẻ}
``Mọi hàm có thể được phân tích duy nhất thành tổng của 1 hàm chẵn \& 1 hàm lẻ, được gọi tương ứng là \textit{phần chẵn} \& \textit{phần lẻ} của 1 hàm số, nếu ta đặt như sau:
\begin{align*}
	f_{\rm e}(x)\coloneqq\frac{f(x) + f(-x)}{2},\ f_{\rm o}(x)\coloneqq\frac{f(x) - f(-x)}{2},
\end{align*}
sau đó $f_{\rm e}$ là hàm chẵn, $f_{\rm o}$ là hàm lẻ, \& $f(x) = f_{\rm e}(x) + f_{\rm o}(x)$. Ngược lại nếu $f(x) = g(x) + h(x)$, trong đó $g$ là chẵn \& $h$ là lẻ, thì $g = f_{\rm e}$ \& $h = f_{\rm o}$, bởi vì
\begin{align*}
	2f_{\rm e}(x) &= f(x) + f(-x) = g(x) + g(-x) + h(x) + h(-x) = 2g(x),\\
	2f_{\rm o}(x) &= f(x) - f(-x) = g(x) - g(-x) + h(x) - h(-x) = 2h(x).
\end{align*}

\begin{vidu}
	Hàm \href{https://vi.wikipedia.org/wiki/H%C3%A0m_hyperbolic}{cosin hyperbolic} \& \href{https://vi.wikipedia.org/wiki/H%C3%A0m_hyperbolic}{sin hyperbolic} có thể được coi là các phần chẵn \& phần lẻ của hàm số lũy thừa tự nhiên, bởi vì hàm thứ nhất là chẵn, hàm thứ 2 là lẻ, \& $e^x = \sinh x + \cosh x$.'' -- \href{https://vi.wikipedia.org/wiki/H%C3%A0m_s%E1%BB%91_ch%E1%BA%B5n_v%C3%A0_l%E1%BA%BB}{Wikipedia\texttt{/}hàm số chẵn \& lẻ}
\end{vidu}

%------------------------------------------------------------------------------%

\begin{thebibliography}{99}
	\bibitem[NQBH\texttt{/}elementary math]{NQBH/elementary math} Nguyễn Quản Bá Hồng. \href{https://github.com/NQBH/hobby/blob/master/elementary_mathematics/some_topics_in_elementary_mathematics_problems_theories_applications_bridges_to_advanced_mathematics/NQBH_some_topics_in_elementary_mathematics_problems_theories_applications_bridges_to_advanced_mathematics.pdf}{\textit{Some Topics in Elementary Mathematics: Problems, Theories, Applications, \& Bridges to Advanced Mathematics}}. Mar 2022--now.
\end{thebibliography}

%------------------------------------------------------------------------------%

\printbibliography[heading=bibintoc]
	
\end{document}