\documentclass{article}
\usepackage[backend=biber,natbib=true,style=authoryear,maxbibnames=10]{biblatex}
\addbibresource{/home/nqbh/reference/bib.bib}
\usepackage[utf8]{vietnam}
\usepackage{tocloft}
\renewcommand{\cftsecleader}{\cftdotfill{\cftdotsep}}
\usepackage[colorlinks=true,linkcolor=blue,urlcolor=red,citecolor=magenta]{hyperref}
\usepackage{amsmath,amssymb,amsthm,float,graphicx,mathtools,soul}
\allowdisplaybreaks
\newtheorem{assumption}{Assumption}
\newtheorem{baitoan}{Bài toán}
\newtheorem{cauhoi}{Câu hỏi}
\newtheorem{conjecture}{Conjecture}
\newtheorem{corollary}{Corollary}
\newtheorem{dangtoan}{Dạng toán}
\newtheorem{definition}{Definition}
\newtheorem{dinhly}{Định lý}
\newtheorem{dinhnghia}{Định nghĩa}
\newtheorem{example}{Example}
\newtheorem{ghichu}{Ghi chú}
\newtheorem{hequa}{Hệ quả}
\newtheorem{hypothesis}{Hypothesis}
\newtheorem{lemma}{Lemma}
\newtheorem{luuy}{Lưu ý}
\newtheorem{nhanxet}{Nhận xét}
\newtheorem{notation}{Notation}
\newtheorem{note}{Note}
\newtheorem{principle}{Principle}
\newtheorem{problem}{Problem}
\newtheorem{proposition}{Proposition}
\newtheorem{question}{Question}
\newtheorem{remark}{Remark}
\newtheorem{theorem}{Theorem}
\newtheorem{vidu}{Ví dụ}
\usepackage[left=1cm,right=1cm,top=5mm,bottom=5mm,footskip=4mm]{geometry}
\def\labelitemii{$\circ$}
\DeclareRobustCommand{\divby}{%
	\mathrel{\vbox{\baselineskip.65ex\lineskiplimit0pt\hbox{.}\hbox{.}\hbox{.}}}%
}

\title{Limit -- Giới Hạn}
\author{Nguyễn Quản Bá Hồng\footnote{Independent Researcher, Ben Tre City, Vietnam\\e-mail: \texttt{nguyenquanbahong@gmail.com}; website: \url{https://nqbh.github.io}.}}
\date{\today}

\begin{document}
\maketitle
\begin{abstract}
	\textsc{[en]} This text is a collection of problems, from easy to advanced, about limit. This text is also a supplementary material for my lecture note on Elementary Mathematics grade 11, which is stored \& downloadable at the following link: \href{https://github.com/NQBH/hobby/blob/master/elementary_mathematics/grade_11/NQBH_elementary_mathematics_grade_11.pdf}{GitHub\texttt{/}NQBH\texttt{/}hobby\texttt{/}elementary mathematics\texttt{/}grade 11\texttt{/}lecture}\footnote{\textsc{url}: \url{https://github.com/NQBH/hobby/blob/master/elementary_mathematics/grade_11/NQBH_elementary_mathematics_grade_11.pdf}.}. The latest version of this text has been stored \& downloadable at the following link: \href{https://github.com/NQBH/hobby/blob/master/elementary_mathematics/grade_11/limit/NQBH_limit.pdf}{GitHub\texttt{/}NQBH\texttt{/}hobby\texttt{/}elementary mathematics\texttt{/}grade 11\texttt{/}limit}\footnote{\textsc{url}: \url{https://github.com/NQBH/hobby/blob/master/elementary_mathematics/grade_11/limit/NQBH_limit.pdf}.}.
	\vspace{2mm}
	
	\textsc{[vi]} Tài liệu này là 1 bộ sưu tập các bài tập chọn lọc từ cơ bản đến nâng cao về biểu thức đại số. Tài liệu này là phần bài tập bổ sung cho tài liệu chính -- bài giảng \href{https://github.com/NQBH/hobby/blob/master/elementary_mathematics/grade_11/NQBH_elementary_mathematics_grade_11.pdf}{GitHub\texttt{/}NQBH\texttt{/}hobby\texttt{/}elementary mathematics\texttt{/}grade 11\texttt{/}lecture} của tác giả viết cho Toán Sơ Cấp lớp 11. Phiên bản mới nhất của tài liệu này được lưu trữ \& có thể tải xuống ở link sau: \href{https://github.com/NQBH/hobby/blob/master/elementary_mathematics/grade_11/limit/NQBH_limit.pdf}{GitHub\texttt{/}NQBH\texttt{/}hobby\texttt{/}elementary mathematics\texttt{/}grade 11\texttt{/}limit}.
	
	\textsf{\textbf{Nội dung.} Giới hạn của dãy số, giới hạn của hàm số, quy tắc tìm giới hạn.}
\end{abstract}
\tableofcontents
\newpage

%------------------------------------------------------------------------------%

\section{Giới Hạn của Dãy Số}

\begin{dinhnghia}[Dãy số có giới hạn là 0]
	Dãy số $(u_n)$ \emph{có giới hạn là $0$} khi $n$ dần tới dương vô cực, nếu $|u_n|$ có thể nhỏ hơn 1 số dương bé tùy ý, kể từ 1 số hạng nào đó trở đi. Ký hiệu: $\lim_{n\to+\infty} u_n = 0$ hay $u_n\to0$ khi $n\to+\infty$.
\end{dinhnghia}
Như vậy, $(u_n)$ có giới hạn là 0 khi $n\to+\infty$ nếu $u_n$ có thể gần 0 bao nhiêu cũng được, miễn là $n$ đủ lớn.

\begin{baitoan}[\cite{SGK_Toan_11_dai_so_giai_tich_co_ban}, Ví dụ 1, p. 113]
	Cho dãy số $(u_n)$ với $u_n = \frac{(-1)^n}{n^2}$. Chứng minh $\lim_{n\to+\infty} u_n = 0$.
\end{baitoan}

\begin{dinhnghia}[Giới hạn của dãy số]
	Dãy số $(v_n)$ có \emph{giới hạn} là số $a$ (hay $v_n$ dần tới $a$) khi $n\to+\infty$, nếu $\lim_{n\to+\infty} (v_n - a) = 0$. Ký hiệu: $\lim_{n\to+\infty} v_n = a$ hay $v_n\to a$ khi $n\to+\infty$.
\end{dinhnghia}

\begin{baitoan}[\cite{SGK_Toan_11_dai_so_giai_tich_co_ban}, Ví dụ 2, p. 114]
	Cho dãy số $(v_n)$ với $v_n = \frac{2n + 1}{n}$. Chứng minh $\lim_{n\to+\infty} v_n = 2$.
\end{baitoan}

\begin{baitoan}[Mở rộng \cite{SGK_Toan_11_dai_so_giai_tich_co_ban}, Ví dụ 2, p. 114]
	Cho dãy số $(v_n)$ với $v_n = \frac{an + b}{cn + d}$ với $a,b,c,d\in\mathbb{R}$, $c^2 + d^2\ne0$. Tính $\lim_{n\to+\infty} v_n$.
\end{baitoan}

\begin{baitoan}[\cite{SGK_Toan_11_dai_so_giai_tich_co_ban}, Ví dụ 3, p. 115]
	Tìm $\lim_{n\to+\infty} \frac{3n^2 - n}{1 + n^2}$.
\end{baitoan}

\begin{baitoan}[\cite{SGK_Toan_11_dai_so_giai_tich_co_ban}, Ví dụ 4, p. 115]
	Tìm $\lim_{n\to+\infty} \frac{\sqrt{1 + 4n^2}}{1 - 2n}$.
\end{baitoan}

\begin{baitoan}[\cite{SGK_Toan_11_dai_so_giai_tich_co_ban}, Ví dụ 5, p. 116]
	(a) Tính tổng của cấp số nhân lùi vô hạn $(u_n)$, với $u_n = \frac{1}{3^n}$. (b) Tính tổng $1 - \frac{1}{2} + \frac{1}{4} - \frac{1}{8} + \cdots + \left(-\frac{1}{2}\right)^{n-1} + \cdots$.
\end{baitoan}

\begin{dinhnghia}[Giới hạn vô cực]
	Ta nói dãy số $(u_n)$ \emph{có giới hạn $+\infty$} khi $n\to+\infty$, nếu $u_n$ có thể lớn hơn 1 số dương bất kỳ, kể từ 1 số hạng nào đó trở đi. Ký hiệu: $\lim_{n\to+\infty} u_n = +\infty$ hay $u_n\to+\infty$ khi $n\to+\infty$. Dãy số $(u_n)$ được gọi là \emph{có giới hạn $-\infty$} khi $n\to+\infty$ nếu $\lim_{n\to+\infty} (-u_n) = +\infty$. Ký hiệu: $\lim_{n\to+\infty} u_n = -\infty$ hay $u_n\to-\infty$ khi $n\to+\infty$.
\end{dinhnghia}

\begin{baitoan}[\cite{SGK_Toan_11_dai_so_giai_tich_co_ban}, Ví dụ 6, p. 118]
	Cho dãy số $(u_n)$ với $u_n = n^2$. Chứng minh $\lim_{n\to+\infty} u_n = +\infty$.
\end{baitoan}

\begin{baitoan}[\cite{SGK_Toan_11_dai_so_giai_tich_co_ban}, Ví dụ 7, p. 119]
	Tìm $\lim_{n\to+\infty} \frac{2n + 5}{n\cdot3^n}$.
\end{baitoan}

\begin{baitoan}[\cite{SGK_Toan_11_dai_so_giai_tich_co_ban}, Ví dụ 8, p. 119]
	Tính $\lim_{n\to+\infty} (n^2 - 2n - 1)$.
\end{baitoan}

\begin{baitoan}[\cite{SGK_Toan_11_dai_so_giai_tich_co_ban}, \textbf{1.}, p. 121]
	Có $1$\emph{kg} chất phóng xạ độc hại. Biết cứ sau 1 khoảng thời gian $T = 24000$ năm thì 1 nửa số chất phóng xạ này bị phân rã thành chất khác không độc hại đối với sức khỏe của con người ($T$ được gọi là \emph{chu kỳ bán rã}). Gọi $u_n$ là khối lượng chất phóng xạ còn lại sau chu kỳ thứ $n$. (a) Tìm số hạng tổng quát $u_n$ của dãy số $(u_n)$. (b) Chứng minh $(u_n)$ có giới hạn là $0$. (c) Từ kết quả (b), chứng minh sau 1 số năm nào đó khối lượng chất phóng xạ đã cho ban đầu không còn độc hại đối với con người, cho biết chất phóng xạ này sẽ không độc hại nữa nếu khối lượng chất phóng xạ còn lại bé hơn $10^{-6}$\emph{g}.
\end{baitoan}

\begin{baitoan}[\cite{SGK_Toan_11_dai_so_giai_tich_co_ban}, \textbf{2.}, p. 121]
	Biết dãy số $(u_n)$ thỏa mãn $|u_n - 1| < \frac{1}{n^3}$, $\forall n\in\mathbb{N}^\star$. Chứng minh $\lim_{n\to+\infty} u_n = 1$.
\end{baitoan}

\begin{baitoan}[\cite{SGK_Toan_11_dai_so_giai_tich_co_ban}, \textbf{3.}, p. 121]
	Tính: (a) $\lim_{n\to+\infty} \frac{6n - 1}{3n + 2}$; (b) $\lim_{n\to+\infty} \frac{3n^2 + n - 5}{2n^2 + 1}$; (c) $\lim_{n\to+\infty} \frac{3^n + 5\cdot4^n}{4^n + 2^n}$; (d) $\lim_{n\to+\infty} \frac{\sqrt{9n^2 - n + 1}}{4n - 2}$.
\end{baitoan}
Bài tập phụ thuộc hình vẽ: \cite[\textbf{4.}, p. 121]{SGK_Toan_11_dai_so_giai_tich_co_ban}.

\begin{baitoan}[\cite{SGK_Toan_11_dai_so_giai_tich_co_ban}, \textbf{5.}, p. 122]
	Tính $S = -1 + \frac{1}{10} - \frac{1}{10^2} + \cdots + \frac{(-1)^n}{10^{n-1}} + \cdots$.
\end{baitoan}

\begin{baitoan}[\cite{SGK_Toan_11_dai_so_giai_tich_co_ban}, \textbf{6.}, p. 122]
	Cho số thập phân vô hạn tuần hoàn $a = 1.(02) = 1.020202\ldots$ (chu kỳ $02$). Viết $a$ dưới dạng 1 phân số.
\end{baitoan}

\begin{baitoan}[\cite{SGK_Toan_11_dai_so_giai_tich_co_ban}, \textbf{7.}, p. 122]
	Tính: (a) $\lim_{n\to+\infty} (n^3 + 2n^2 - n + 1)$; (b) $\lim_{n\to+\infty} (-n^2 + 5n - 2)$; (c) $\lim_{n\to+\infty} \left(\sqrt{n^2 - n} - n\right)$; (d) $\lim_{n\to+\infty} \left(\sqrt{n^2 - n} + n\right)$.
\end{baitoan}

\begin{baitoan}[\cite{SGK_Toan_11_dai_so_giai_tich_co_ban}, \textbf{8.}, p. 122]
	Cho 2 dãy số $(u_n),(v_n)$. Biết $\lim_{n\to+\infty} u_n = 3$, $\lim_{n\to+\infty} v_n = +\infty$. Tính: (a) $\lim_{n\to+\infty} \frac{3u_n - 1}{u_n + 1}$; (b) $\lim_{n\to+\infty} \frac{v_n + 2}{v_n^2 - 1}$.
\end{baitoan}

%------------------------------------------------------------------------------%

\section{Giới Hạn của Hàm Số}

\begin{baitoan}[\cite{SBT_Toan_11_dai_so_giai_tich_co_ban}, Ví dụ 1, p. 153]
	Cho hàm số $f(x) = \frac{2x^2 + x - 3}{x - 1}$. Dùng định nghĩa chứng minh $\lim_{x\to1} f(x) = 5$.
\end{baitoan}

\begin{proof}[Giải]
	Hàm số đã cho xác định trên $D_f = \mathbb{R}\backslash\{1\}$. Giả sử $(x_n)$ là dãy số bất kỳ, $x_n\ne1$, \& $x_n\to1$.
	\begin{align*}
		\lim_{n\to+\infty} f(x_n) = \frac{2x_n^2 + x_n - 3}{x_n - 1} \lim_{n\to+\infty} \frac{2(x_n - 1)\left(x_n + \frac{3}{2}\right)}{x_n - 1} = \lim_{n\to+\infty} 2\left(x_n + \frac{3}{2}\right) = 2\left(1 + \frac{3}{2}\right) = 5.
	\end{align*}
	Do đó, $\lim_{x\to1} f(x) = 5$.
\end{proof}

\begin{baitoan}[\cite{SBT_Toan_11_dai_so_giai_tich_co_ban}, Ví dụ 2, p. 154]
	Cho hàm số
	\begin{equation*}
		f(x) = \left\{\begin{split}
			&x,&&\mbox{nếu }x\ge0,\\
			&1 - x,&&\mbox{nếu }x < 0.
		\end{split}\right.
	\end{equation*}
	Dùng định nghĩa chứng minh hàm số $f(x)$ không có giới hạn khi $x\to0$, i.e., $\bar\exists\lim_{x\to0} f(x)$.
\end{baitoan}

\begin{proof}[Giải]
	Hàm số đã cho xác định trên $D_f = \mathbb{R}$. Lấy dãy số $(x_n)$ với $x_n = \frac{1}{n}$. Có $x_n\to0$ \& $\lim_{n\to+\infty} f(x_n) = \lim_{n\to+\infty} x_n = \lim_{n\to+\infty} \frac{1}{n} = 0$ (1). Lấy dãy số $(y_n)$ với $y_n = -\frac{1}{n}$. Có $y_n\to0$ \& $\lim_{n\to+\infty} f(y_n) = \lim_{n\to+\infty} (1 - y_n) = \lim_{n\to+\infty} \left(1 + \frac{1}{n}\right) = 1$ (2). Từ (1) \& (2) suy ra hàm số $f(x)$ không có giới hạn khi $x\to0$.
\end{proof}

\begin{nhanxet}
	``Để dùng định nghĩa chứng minh hàm số $y = f(x)$ không có giới hạn khi $x\to x_0$, ta thường làm như sau: (a) Chọn 2 dãy số khác nhau $(a_n),(b_n)$ thỏa mãn: $a_n,b_n$ thuộc tập xác định $D_f$ của hàm số $y = f(x)$ \& khác $x_0$, $a_n\to x_0$, $b_n\to x_0$. (b) Chứng minh $\lim_{n\to+\infty} f(a_n)\ne\lim_{n\to+\infty} f(b_n)$ hoặc chứng minh 1 trong các giới hạn này không tồn tại. Trường hợp $x\to x_0^+$, $x\to x_0^-$, hay $x\to\pm\infty$ chứng minh tương tự.'' -- \cite[pp. 154--155]{SBT_Toan_11_dai_so_giai_tich_co_ban}
\end{nhanxet}

\begin{baitoan}[\cite{SBT_Toan_11_dai_so_giai_tich_co_ban}, Ví dụ 3, p. 155]
	Tính: (a) $\lim_{x\to-2} \left(\sqrt{x^2 + 5} - 1\right)$. (b) $\lim_{x\to3^-} \frac{x + 1}{x - 2}$. (c) $\lim_{x\to-\infty} (-x^3 + x^2 - x + 1)$. (d) $\lim_{x\to4} \frac{1 - x}{(x - 4)^2}$. (e) $\lim_{x\to3^-} \frac{2x - 1}{x - 3}$.
\end{baitoan}

\begin{proof}[Giải]
	(a) $-2\in D_f = \mathbb{R}$, $\lim_{x\to-2} \left(\sqrt{x^2 + 5} - 1\right) = \sqrt{(-2)^2 + 5} - 1 = 2$. (b) $3\in D_f = \mathbb{R}\backslash\{2\}$, $\lim_{x\to3^-} \frac{x + 1}{x - 2} = \frac{3 + 1}{3 - 2} = 4$. (c) $\lim_{x\to-\infty} (-x^3 + x^2 - x + 1) = \lim_{x\to-\infty} x^3\left(-1 + \frac{1}{x} - \frac{1}{x^2} + \frac{1}{x^3}\right) = +\infty$. (d) Chú ý $4\notin D_f = \mathbb{R}\backslash\{4\}$. Có $\lim_{x\to4} (1 - x) = -3 < 0$ (1), $\lim_{x\to4} (x - 4)^2 = 0$ \& $(x - 4)^2 > 0$, $\forall x\ne4$ (2). Áp dụng quy tắc về giới hạn vô cực đối với thương $\frac{f(x)}{g(x)}$, từ (1) \& (2) suy ra  $\lim_{x\to4} \frac{1 - x}{(x - 4)^2} = -\infty$. (e) $3\notin D_f = \mathbb{R}\backslash\{3\}$, có $\lim_{x\to3^-} (2x - 1) = 5 > 0$, $\lim_{x\to3^-} (x - 3) = 0$ \& $x - 3 < 0$, $\forall x < 3$. Suy ra $\lim_{x\to3^-} \frac{2x - 1}{x - 3} = -\infty$.
\end{proof}
Lời giải các ví dụ  trên đã dùng trực tiếp các định lý về giới hạn của tổng, hiệu, tích, thương, \& căn của các hàm số hoặc các quy tắc về giới hạn vô cực.

\begin{baitoan}[\cite{SBT_Toan_11_dai_so_giai_tich_co_ban}, Ví dụ 4, p. 156]
	Tính: (a) $\lim_{x\to1} \frac{x^2 + 2x - 3}{2x^2 - x - 1}$. (b) $\lim_{x\to2} \frac{2 - x}{\sqrt{x + 7} - 3}$. (c) $\lim_{x\to+\infty} \frac{2x^3 + 3x - 4}{-x^3 - x^2 + 1}$. (d) $\lim_{x\to-\infty} \frac{\sqrt{x^2 - x} - \sqrt{4x^2 + 1}}{2x + 3}$. (e) $\lim_{x\to0^-} \frac{1}{x}\left(\frac{1}{x + 1} - 1\right)$. (f) $\lim_{x\to-\infty} \left(\sqrt{4x^2 - x} + 2x\right)$.
\end{baitoan}

\begin{proof}[Giải]
	(a) $\lim_{x\to1} \frac{x^2 + 2x - 3}{2x^2 - x - 1} = \lim_{x\to1} \frac{(x - 1)(x + 3)}{2(x - 1)\left(x + \frac{1}{2}\right)} = \lim_{x\to1} \frac{x + 3}{2x + 1} = \frac{4}{3}$. (b) $\lim_{x\to2} \frac{2 - x}{\sqrt{x + 7} - 3} = \lim_{x\to2} \frac{(2 - x)(\sqrt{x + 7} + 3)}{(\sqrt{x + 7} - 3)(\sqrt{x + 7} + 3)}$\\$= \lim_{x\to2} \frac{(2 - x)(\sqrt{x + 7} + 3)}{x - 2} = \lim_{x\to2} -(\sqrt{x + 7} + 3) = -6$. (c) $\ldots$ 
\end{proof}

\begin{baitoan}[\cite{SBT_Toan_11_dai_so_giai_tich_co_ban}, 2.1., p. 158]
	Dùng định nghĩa tìm các giới hạn: (a) $\lim_{x\to5} \frac{x + 3}{3 - x}$. (b) $\lim_{x\to+\infty} \frac{x^3 + 1}{x^2 + 1}$.
\end{baitoan}

\begin{baitoan}[\cite{SBT_Toan_11_dai_so_giai_tich_co_ban}, 2.2., p. 158]
	Cho hàm số
	\begin{equation*}
		f(x) = \left\{\begin{split}
			&x^2,&&\mbox{nếu }x\ge0,\\
			&x^2 - 1,&&\mbox{nếu }x < 0.
		\end{split}\right.
	\end{equation*}
	(a) Vẽ đồ thị của hàm số $f(x)$. Từ đó dự đoán về giới hạn của $f(x)$ khi $x\to0$. (b) Dùng định nghĩa chứng minh dự đoán trên.
\end{baitoan}

\begin{baitoan}[\cite{SBT_Toan_11_dai_so_giai_tich_co_ban}, 2.3, p. 158]	
	(a) Chứng minh hàm số $y = \sin x$ không có giới hạn khi $x\to+\infty$. (b) Giải thích bằng đồ thị kết luận ở (a).
\end{baitoan}
\cite{SBT_Toan_11_dai_so_giai_tich_co_ban}, 2.4--2.11.

%------------------------------------------------------------------------------%

\section{Hàm Số Liên Tục}

\begin{baitoan}[\cite{SBT_Toan_11_dai_so_giai_tich_co_ban}, Ví dụ 1 , p. 161]
	Xét tính liên tục của hàm số
	\begin{equation*}
		f(x) = \left\{\begin{split}
			&\frac{x + 3}{x - 1},&&\mbox{nếu }x\ne-1,\\
			&2,&&\mbox{nếu }x = 1,
		\end{split}\right.
	\end{equation*} 
	tại điểm $x = -1$.
\end{baitoan}

\begin{proof}[Giải]
	$-1\in D_f = \mathbb{R}$. Có $f(-1) = 2$, $\lim_{x\to-1} \frac{x + 3}{x - 1} = \frac{-1 + 3}{-1 - 1} = -1\ne f(-1)$. Do đó, hàm số $f(x)$ không liên tục tại $x = -1$.
\end{proof}

\begin{baitoan}[\cite{SBT_Toan_11_dai_so_giai_tich_co_ban}, Ví dụ 2 , p. 161]
	Xét tính liên tục của hàm số
	\begin{equation*}
		f(x) = \left\{\begin{split}
			&\frac{x^2 - 2x - 3}{x - 3},&&\mbox{nếu }x\ne3,\\
			&5,&&\mbox{nếu }x = 3,
		\end{split}\right.
	\end{equation*} 
	trên tập xác định của nó.
\end{baitoan}

\begin{baitoan}[\cite{SBT_Toan_11_dai_so_giai_tich_co_ban}, Ví dụ 3 , p. 162]
	Chứng minh phương trình sau có ít nhất 2 nghiệm: $2x^3 - 10x - 7 = 0$.
\end{baitoan}

\begin{proof}[Giải]
	Xét hàm số $f(x) = 2x^3 - 10x - 7$. Hàm số này là hàm đa thức nên liên tục trên $\mathbb{R}$. Do đó $f(x)$ liên tục trên các đoạn $[-1;0]$ \& $[0;3]$ (1). Có $f(-1) = 1$, $f(0) = -7$, $f(3) = 17$, nên $f(-1)f(0) < 0$, $f(0)f(3) < 0$ (2). Từ (1) \& (2) suy ra phương trình $f(x) = 0$, i.e.,  $2x^3 - 10x - 7 = 0$ có ít nhất 2 nghiệm, 1 nghiệm thuộc khoảng $(-1;0)$, còn nghiệm kia thuộc khoảng $(0;3)$.
\end{proof}

\begin{baitoan}[\cite{SBT_Toan_11_dai_so_giai_tich_co_ban}, Ví dụ 4 , p. 163]
	Chứng minh phương trình sau luôn có nghiệm với mọi giá trị của tham số $m$: $(1 - m^2)x^5 - 3x - 1 = 0$.
\end{baitoan}

\begin{proof}[Giải]
	Xét hàm số $f(x) = 	(1 - m^2)x^5 - 3x - 1$. Vì $f(0) = -1 < 0$, $f(-1) = m^2 + 1 > 0$ nên $f(-1)f(0) < 0$, $\forall m\in\mathbb{R}$ (1). Vì $f(x)$ là hàm đa thức, liên tục trên $\mathbb{R}$, nên liên tục trên đoạn $[-1;0]$ (2). Từ (1) \& (2) suy ra phương trình $f(x) = 0$ có ít nhất 1 nghiệm trong khoảng $(-1;0)$, i.e., phương trình $(1 - m^2)x^5 - 3x - 1 = 0$ luôn có nghiệm\footnote{\& biết được thêm là có ít nhất 1 nghiệm trong khoảng $(-1;0)$.} $\forall m\in\mathbb{R}$.
\end{proof}

\begin{nhanxet}
	``Để chứng minh phương trình $f(x) = 0$ có ít nhất 1 nghiệm, chỉ cần tìm được 2 số $a,b\in\mathbb{R}$ sao cho $f(a)f(b) < 0$ \& hàm số $f(x)$ liên tục trên đoạn $[a;b]$. Nếu phương trình chứa tham số, thì chọn $a,b\in\mathbb{R}$ sao cho: $f(a),f(b)$ không còn chứa tham số hay chứa tham số nhưng có dấu không đổi; hoặc $f(a)f(b)$ chứa tham số nhưng tích $f(a)f(b)$ luôn âm.'' -- \cite[p. 163]{SBT_Toan_11_dai_so_giai_tich_co_ban}
\end{nhanxet}

\begin{baitoan}[\cite{SBT_Toan_11_dai_so_giai_tich_co_ban}, 3.1., p. 163]
	Cho hàm số $f(x) = \frac{(x - 1)|x|}{x}$. Vẽ đồ thị của hàm số này. Từ đồ thị dự đoán các khoảng trên đó hàm số liên tục \& chứng minh dự đoán đó.
\end{baitoan}

\begin{baitoan}[\cite{SBT_Toan_11_dai_so_giai_tich_co_ban}, 3.2., p. 163]
	Cho ví dụ về 1 hàm số liên tục trên $(a;b]$ \& trên $(b;c)$ nhưng không liên tục trên $(a;c)$.
\end{baitoan}

\begin{baitoan}[\cite{SBT_Toan_11_dai_so_giai_tich_co_ban}, 3.3., p. 164]
	Chứng minh nếu 1 hàm số liên tục trên $(a;b]$ \& trên $[b;c)$ thì nó liên tục trên $(a;c)$.
\end{baitoan}

\begin{baitoan}[\cite{SBT_Toan_11_dai_so_giai_tich_co_ban}, 3.4., p. 164]
	Cho hàm số $y = f(x)$ xác định trên khoảng $(a;b)$ chứa điểm $x_0$. Chứng minh nếu $\lim_{x\to x_0} \frac{f(x) - f(x_0)}{x - x_0} = L$ thì hàm số $f(x)$ liên tục tại điểm $x_0$.
\end{baitoan}
\noindent\textit{Hint.} Đặt $g(x)\coloneqq\frac{f(x) - f(x_0)}{x - x_0} - L$ \& biểu diễn $f(x)$ qua $g(x)$.

%------------------------------------------------------------------------------%

\printbibliography[heading=bibintoc]
	
\end{document}