\documentclass{article}
\usepackage[backend=biber,natbib=true,style=authoryear]{biblatex}
\addbibresource{/home/nqbh/reference/bib.bib}
\usepackage[utf8]{vietnam}
\usepackage{tocloft}
\renewcommand{\cftsecleader}{\cftdotfill{\cftdotsep}}
\usepackage[colorlinks=true,linkcolor=blue,urlcolor=red,citecolor=magenta]{hyperref}
\usepackage{amsmath,amssymb,amsthm,mathtools,float,graphicx,algpseudocode,algorithm,tcolorbox}
\usepackage[inline]{enumitem}
\allowdisplaybreaks
\numberwithin{equation}{section}
\newtheorem{assumption}{Assumption}[section]
\newtheorem{baitoan}{Bài toán}
\newtheorem{cauhoi}{Câu hỏi}[section]
\newtheorem{conjecture}{Conjecture}[section]
\newtheorem{corollary}{Corollary}[section]
\newtheorem{dangtoan}{Dạng toán}[section]
\newtheorem{definition}{Definition}[section]
\newtheorem{dinhly}{Định lý}[section]
\newtheorem{dinhnghia}{Định nghĩa}[section]
\newtheorem{example}{Example}[section]
\newtheorem{ghichu}{Ghi chú}[section]
\newtheorem{hequa}{Hệ quả}[section]
\newtheorem{hypothesis}{Hypothesis}[section]
\newtheorem{lemma}{Lemma}[section]
\newtheorem{luuy}{Lưu ý}[section]
\newtheorem{nhanxet}{Nhận xét}[section]
\newtheorem{notation}{Notation}[section]
\newtheorem{note}{Note}[section]
\newtheorem{principle}{Principle}[section]
\newtheorem{problem}{Problem}[section]
\newtheorem{proposition}{Proposition}[section]
\newtheorem{question}{Question}[section]
\newtheorem{remark}{Remark}[section]
\newtheorem{theorem}{Theorem}[section]
\newtheorem{vidu}{Ví dụ}[section]
\usepackage[left=0.5in,right=0.5in,top=1.5cm,bottom=1.5cm]{geometry}
\usepackage{fancyhdr}
\pagestyle{fancy}
\fancyhf{}
\lhead{\small Sect.~\thesection}
\rhead{\small\nouppercase{\leftmark}}
\renewcommand{\subsectionmark}[1]{\markboth{#1}{}}
\cfoot{\thepage}
\def\labelitemii{$\circ$}

\title{Limit -- Giới Hạn}
\author{Nguyễn Quản Bá Hồng\footnote{Independent Researcher, Ben Tre City, Vietnam\\e-mail: \texttt{nguyenquanbahong@gmail.com}; website: \url{https://nqbh.github.io}.}}
\date{\today}

\begin{document}
\maketitle
\begin{abstract}
	\textsc{[en]} This text is a collection of problems, from easy to advanced, about limit. This text is also a supplementary material for my lecture note on Elementary Mathematics grade 11, which is stored \& downloadable at the following link: \href{https://github.com/NQBH/hobby/blob/master/elementary_mathematics/grade_11/NQBH_elementary_mathematics_grade_11.pdf}{GitHub\texttt{/}NQBH\texttt{/}hobby\texttt{/}elementary mathematics\texttt{/}grade 11\texttt{/}lecture}\footnote{\textsc{url}: \url{https://github.com/NQBH/hobby/blob/master/elementary_mathematics/grade_11/NQBH_elementary_mathematics_grade_11.pdf}.}. The latest version of this text has been stored \& downloadable at the following link: \href{https://github.com/NQBH/hobby/blob/master/elementary_mathematics/grade_11/limit/NQBH_limit.pdf}{GitHub\texttt{/}NQBH\texttt{/}hobby\texttt{/}elementary mathematics\texttt{/}grade 11\texttt{/}limit}\footnote{\textsc{url}: \url{https://github.com/NQBH/hobby/blob/master/elementary_mathematics/grade_11/limit/NQBH_limit.pdf}.}.
	\vspace{2mm}
	
	\textsc{[vi]} Tài liệu này là 1 bộ sưu tập các bài tập chọn lọc từ cơ bản đến nâng cao về biểu thức đại số. Tài liệu này là phần bài tập bổ sung cho tài liệu chính -- bài giảng \href{https://github.com/NQBH/hobby/blob/master/elementary_mathematics/grade_11/NQBH_elementary_mathematics_grade_11.pdf}{GitHub\texttt{/}NQBH\texttt{/}hobby\texttt{/}elementary mathematics\texttt{/}grade 11\texttt{/}lecture} của tác giả viết cho Toán Sơ Cấp lớp 11. Phiên bản mới nhất của tài liệu này được lưu trữ \& có thể tải xuống ở link sau: \href{https://github.com/NQBH/hobby/blob/master/elementary_mathematics/grade_11/limit/NQBH_limit.pdf}{GitHub\texttt{/}NQBH\texttt{/}hobby\texttt{/}elementary mathematics\texttt{/}grade 11\texttt{/}limit}.
	
	\textsf{\textbf{Nội dung.} Giới hạn của dãy số, giới hạn của hàm số, quy tắc tìm giới hạn.}
\end{abstract}
\tableofcontents
\newpage

%------------------------------------------------------------------------------%

\section{Giới Hạn của Dãy Số}

\begin{dinhnghia}[Dãy số có giới hạn là 0]
	Dãy số $(u_n)$ \emph{có giới hạn là $0$} khi $n$ dần tới dương vô cực, nếu $|u_n|$ có thể nhỏ hơn 1 số dương bé tùy ý, kể từ 1 số hạng nào đó trở đi. Ký hiệu: $\lim_{n\to+\infty} u_n = 0$ hay $u_n\to0$ khi $n\to+\infty$.
\end{dinhnghia}
Như vậy, $(u_n)$ có giới hạn là 0 khi $n\to+\infty$ nếu $u_n$ có thể gần 0 bao nhiêu cũng được, miễn là $n$ đủ lớn.

\begin{baitoan}[\cite{SGK_Toan_11_dai_so_giai_tich_co_ban}, Ví dụ 1, p. 113]
	Cho dãy số $(u_n)$ với $u_n = \frac{(-1)^n}{n^2}$. Chứng minh $\lim_{n\to+\infty} u_n = 0$.
\end{baitoan}

\begin{dinhnghia}[Giới hạn của dãy số]
	Dãy số $(v_n)$ có \emph{giới hạn} là số $a$ (hay $v_n$ dần tới $a$) khi $n\to+\infty$, nếu $\lim_{n\to+\infty} (v_n - a) = 0$. Ký hiệu: $\lim_{n\to+\infty} v_n = a$ hay $v_n\to a$ khi $n\to+\infty$.
\end{dinhnghia}

\begin{baitoan}[\cite{SGK_Toan_11_dai_so_giai_tich_co_ban}, Ví dụ 2, p. 114]
	Cho dãy số $(v_n)$ với $v_n = \frac{2n + 1}{n}$. Chứng minh $\lim_{n\to+\infty} v_n = 2$.
\end{baitoan}

\begin{baitoan}[Mở rộng \cite{SGK_Toan_11_dai_so_giai_tich_co_ban}, Ví dụ 2, p. 114]
	Cho dãy số $(v_n)$ với $v_n = \frac{an + b}{cn + d}$ với $a,b,c,d\in\mathbb{R}$, $c^2 + d^2\ne0$. Tính $\lim_{n\to+\infty} v_n$.
\end{baitoan}

\begin{baitoan}[\cite{SGK_Toan_11_dai_so_giai_tich_co_ban}, Ví dụ 3, p. 115]
	Tìm $\lim_{n\to+\infty} \frac{3n^2 - n}{1 + n^2}$.
\end{baitoan}

\begin{baitoan}[\cite{SGK_Toan_11_dai_so_giai_tich_co_ban}, Ví dụ 4, p. 115]
	Tìm $\lim_{n\to+\infty} \frac{\sqrt{1 + 4n^2}}{1 - 2n}$.
\end{baitoan}

\begin{baitoan}[\cite{SGK_Toan_11_dai_so_giai_tich_co_ban}, Ví dụ 5, p. 116]
	(a) Tính tổng của cấp số nhân lùi vô hạn $(u_n)$, với $u_n = \frac{1}{3^n}$. (b) Tính tổng $1 - \frac{1}{2} + \frac{1}{4} - \frac{1}{8} + \cdots + \left(-\frac{1}{2}\right)^{n-1} + \cdots$.
\end{baitoan}

\begin{dinhnghia}[Giới hạn vô cực]
	Ta nói dãy số $(u_n)$ \emph{có giới hạn $+\infty$} khi $n\to+\infty$, nếu $u_n$ có thể lớn hơn 1 số dương bất kỳ, kể từ 1 số hạng nào đó trở đi. Ký hiệu: $\lim_{n\to+\infty} u_n = +\infty$ hay $u_n\to+\infty$ khi $n\to+\infty$. Dãy số $(u_n)$ được gọi là \emph{có giới hạn $-\infty$} khi $n\to+\infty$ nếu $\lim_{n\to+\infty} (-u_n) = +\infty$. Ký hiệu: $\lim_{n\to+\infty} u_n = -\infty$ hay $u_n\to-\infty$ khi $n\to+\infty$.
\end{dinhnghia}

\begin{baitoan}[\cite{SGK_Toan_11_dai_so_giai_tich_co_ban}, Ví dụ 6, p. 118]
	Cho dãy số $(u_n)$ với $u_n = n^2$. Chứng minh $\lim_{n\to+\infty} u_n = +\infty$.
\end{baitoan}

\begin{baitoan}[\cite{SGK_Toan_11_dai_so_giai_tich_co_ban}, Ví dụ 7, p. 119]
	Tìm $\lim_{n\to+\infty} \frac{2n + 5}{n\cdot3^n}$.
\end{baitoan}

\begin{baitoan}[\cite{SGK_Toan_11_dai_so_giai_tich_co_ban}, Ví dụ 8, p. 119]
	Tính $\lim_{n\to+\infty} (n^2 - 2n - 1)$.
\end{baitoan}

\begin{baitoan}[\cite{SGK_Toan_11_dai_so_giai_tich_co_ban}, \textbf{1.}, p. 121]
	Có $1$\emph{kg} chất phóng xạ độc hại. Biết cứ sau 1 khoảng thời gian $T = 24000$ năm thì 1 nửa số chất phóng xạ này bị phân rã thành chất khác không độc hại đối với sức khỏe của con người ($T$ được gọi là \emph{chu kỳ bán rã}). Gọi $u_n$ là khối lượng chất phóng xạ còn lại sau chu kỳ thứ $n$. (a) Tìm số hạng tổng quát $u_n$ của dãy số $(u_n)$. (b) Chứng minh $(u_n)$ có giới hạn là $0$. (c) Từ kết quả (b), chứng minh sau 1 số năm nào đó khối lượng chất phóng xạ đã cho ban đầu không còn độc hại đối với con người, cho biết chất phóng xạ này sẽ không độc hại nữa nếu khối lượng chất phóng xạ còn lại bé hơn $10^{-6}$\emph{g}.
\end{baitoan}

\begin{baitoan}[\cite{SGK_Toan_11_dai_so_giai_tich_co_ban}, \textbf{2.}, p. 121]
	Biết dãy số $(u_n)$ thỏa mãn $|u_n - 1| < \frac{1}{n^3}$, $\forall n\in\mathbb{N}^\star$. Chứng minh $\lim_{n\to+\infty} u_n = 1$.
\end{baitoan}

\begin{baitoan}[\cite{SGK_Toan_11_dai_so_giai_tich_co_ban}, \textbf{3.}, p. 121]
	Tính: (a) $\lim_{n\to+\infty} \frac{6n - 1}{3n + 2}$; (b) $\lim_{n\to+\infty} \frac{3n^2 + n - 5}{2n^2 + 1}$; (c) $\lim_{n\to+\infty} \frac{3^n + 5\cdot4^n}{4^n + 2^n}$; (d) $\lim_{n\to+\infty} \frac{\sqrt{9n^2 - n + 1}}{4n - 2}$.
\end{baitoan}
Bài tập phụ thuộc hình vẽ: \cite[\textbf{4.}, p. 121]{SGK_Toan_11_dai_so_giai_tich_co_ban}.

\begin{baitoan}[\cite{SGK_Toan_11_dai_so_giai_tich_co_ban}, \textbf{5.}, p. 122]
	Tính $S = -1 + \frac{1}{10} - \frac{1}{10^2} + \cdots + \frac{(-1)^n}{10^{n-1}} + \cdots$.
\end{baitoan}

\begin{baitoan}[\cite{SGK_Toan_11_dai_so_giai_tich_co_ban}, \textbf{6.}, p. 122]
	Cho số thập phân vô hạn tuần hoàn $a = 1.(02) = 1.020202\ldots$ (chu kỳ $02$). Viết $a$ dưới dạng 1 phân số.
\end{baitoan}

\begin{baitoan}[\cite{SGK_Toan_11_dai_so_giai_tich_co_ban}, \textbf{7.}, p. 122]
	Tính: (a) $\lim_{n\to+\infty} (n^3 + 2n^2 - n + 1)$; (b) $\lim_{n\to+\infty} (-n^2 + 5n - 2)$; (c) $\lim_{n\to+\infty} \left(\sqrt{n^2 - n} - n\right)$; (d) $\lim_{n\to+\infty} \left(\sqrt{n^2 - n} + n\right)$.
\end{baitoan}

\begin{baitoan}[\cite{SGK_Toan_11_dai_so_giai_tich_co_ban}, \textbf{8.}, p. 122]
	Cho 2 dãy số $(u_n),(v_n)$. Biết $\lim_{n\to+\infty} u_n = 3$, $\lim_{n\to+\infty} v_n = +\infty$. Tính: (a) $\lim_{n\to+\infty} \frac{3u_n - 1}{u_n + 1}$; (b) $\lim_{n\to+\infty} \frac{v_n + 2}{v_n^2 - 1}$.
\end{baitoan}

%------------------------------------------------------------------------------%

\section{Giới Hạn của Hàm Số}

%------------------------------------------------------------------------------%

\section{Hàm Số Liên Tục}

%------------------------------------------------------------------------------%

\printbibliography[heading=bibintoc]
	
\end{document}