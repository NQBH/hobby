\documentclass{article}
\usepackage[backend=biber,natbib=true,style=authoryear]{biblatex}
\addbibresource{/home/hong/1_NQBH/reference/bib.bib}
\usepackage[utf8]{vietnam}
\usepackage{tocloft}
\renewcommand{\cftsecleader}{\cftdotfill{\cftdotsep}}
\usepackage[colorlinks=true,linkcolor=blue,urlcolor=red,citecolor=magenta]{hyperref}
\usepackage{amsmath,amssymb,amsthm,mathtools,float,graphicx,algpseudocode,algorithm,tcolorbox}
\usepackage[inline]{enumitem}
\allowdisplaybreaks
\numberwithin{equation}{section}
\newtheorem{assumption}{Assumption}[section]
\newtheorem{conjecture}{Conjecture}[section]
\newtheorem{corollary}{Corollary}[section]
\newtheorem{hequa}{Hệ quả}[section]
\newtheorem{definition}{Definition}[section]
\newtheorem{dinhnghia}{Định nghĩa}[section]
\newtheorem{example}{Example}[section]
\newtheorem{vidu}{Ví dụ}[section]
\newtheorem{lemma}{Lemma}[section]
\newtheorem{notation}{Notation}[section]
\newtheorem{principle}{Principle}[section]
\newtheorem{problem}{Problem}[section]
\newtheorem{baitoan}{Bài toán}[section]
\newtheorem{proposition}{Proposition}[section]
\newtheorem{question}{Question}[section]
\newtheorem{cauhoi}{Câu hỏi}[section]
\newtheorem{remark}{Remark}[section]
\newtheorem{luuy}{Lưu ý}[section]
\newtheorem{theorem}{Theorem}[section]
\newtheorem{dinhly}{Định lý}[section]
\usepackage[left=0.5in,right=0.5in,top=1.5cm,bottom=1.5cm]{geometry}
\usepackage{fancyhdr}
\pagestyle{fancy}
\fancyhf{}
\lhead{\small \textsc{Sect.} ~\thesection}
\rhead{\small \nouppercase{\leftmark}}
\renewcommand{\sectionmark}[1]{\markboth{#1}{}}
\cfoot{\thepage}
\def\labelitemii{$\circ$}

\title{Problems in Elementary Mathematics\texttt{/}Grade 11}
\author{Nguyễn Quản Bá Hồng\footnote{Independent Researcher, Ben Tre City, Vietnam\\e-mail: \texttt{nguyenquanbahong@gmail.com}; website: \url{https://nqbh.github.io}.}}
\date{\today}

\begin{document}
\maketitle
\begin{abstract}
	1 bộ sưu tập các bài toán chọn lọc từ cơ bản đến nâng cao cho Toán học sơ cấp lớp 11. Tài liệu này là phần bài tập bổ sung cho tài liệu chính \href{https://github.com/NQBH/hobby/blob/master/elementary_mathematics/grade_11/NQBH_elementary_mathematics_grade_11.pdf}{GitHub\texttt{/}NQBH\texttt{/}hobby\texttt{/}elementary mathematics\texttt{/}grade 11\texttt{/}lecture}\footnote{Explicitly, \url{https://github.com/NQBH/hobby/blob/master/elementary_mathematics/grade_11/NQBH_elementary_mathematics_grade_11.pdf}.} của tác giả viết cho Toán lớp 11. Phiên bản mới nhất của tài liệu này được lưu trữ ở link sau: \href{https://github.com/NQBH/hobby/blob/master/elementary_mathematics/grade_11/problem/NQBH_elementary_mathematics_grade_11_problem.pdf}{GitHub\texttt{/}NQBH\texttt{/}hobby\texttt{/}elementary mathematics\texttt{/}grade 11\texttt{/}problem}\footnote{Explicitly, \url{https://github.com/NQBH/hobby/blob/master/elementary_mathematics/grade_11/problem/NQBH_elementary_mathematics_grade_11_problem.pdf}.}.
\end{abstract}
\tableofcontents
\newpage

%------------------------------------------------------------------------------%

\section{Hàm Số Lượng Giác \& Phương Trình Lượng Giác -- Trigonometric Function \& Trigonometric Equation}

%------------------------------------------------------------------------------%

\section{Tổ Hợp \& Xác Suất -- Combinatorics \& Probability}

\subsection{2 quy tắc đếm cơ bản -- 2 basic rules of counting}

\begin{baitoan}[\cite{TL_chuyen_Toan_Dai_So_Giai_Tich_11}, Ví dụ 2, p. 81]
	Trong 1 kỳ thi đại học, trong số các thí sinh dự thi vào trường Đại học Sư phạm ở khối A có $51$ em đạt điểm giỏi môn Toán, $73$ em đạt điểm giỏi môn Vật lý, $64$ em đạt điểm giỏi môn Hóa học, $32$ em đạt điểm giỏi cả 2 môn Toán \& Vật lý, $45$ em đạt điểm giỏi cả 2 môn Vật lý \& Hóa học, $21$ em đạt điểm giỏi cả 2 môn Toán \& Hóa học \& $10$ em đạt điểm giỏi cả 3 môn Toán, Vật lý, Hóa học. Có $767$ em mà cả 3 môn đều không có môn nào đạt điểm giỏi. Hỏi có bao nhiêu thí sinh dự thi vào trường Đại học Sư phạm ở khối A?
\end{baitoan}

\begin{baitoan}[\cite{TL_chuyen_Toan_Dai_So_Giai_Tich_11}, \textbf{1.}, p. 83]
	Có bao nhiêu số nguyên dương không vượt quá $1000$ mà chia hết cho $3$ hoặc chia hết cho $5$.
\end{baitoan}

\begin{baitoan}[\cite{TL_chuyen_Toan_Dai_So_Giai_Tich_11}, \textbf{2.}, p. 83]
	Trong 1 khu phố gồm $53$ hộ, thống kê cho thấy có $30$ hộ đặt mua báo A, $18$ hộ đặt mua báo B \& $26$ hộ đặt mua báo C. Có $9$ hộ đặt mua báo A \& B; $16$ hộ đặt mua báo A \& C; $8$ hộ đặt mua báo B \& C. Có $47$ hộ đặt mua ít nhất 1 tờ báo. Hỏi:
	\begin{enumerate*}
		\item[(a)] Có bao nhiêu hộ không mua tờ báo nào?
		\item[(b)] Có bao nhiêu hộ mua cả 3 tờ báo?
		\item[(c)] Có bao nhiêu hộ mua báo A \& B nhưng không mua báo C?
		\item[(d)] Có bao nhiêu hộ chỉ mua báo A mà không mua báo B \& C?
	\end{enumerate*}
\end{baitoan}

\begin{baitoan}[\cite{TL_chuyen_Toan_Dai_So_Giai_Tich_11}, \textbf{3.}, p. 83]
	1 nhóm $9$ người gồm 3 đàn ông, 4 phụ nữ \& 2 đứa trẻ đi xem phim. Hỏi có bao nhiêu cách xếp họ ngồi trên 1 hàng ghế sao cho mỗi đứa trẻ ngồi giữa 2 người phụ nữ \& không có 2 người đàn ông nào ngồi cạnh nhau?
\end{baitoan}

\begin{baitoan}[\cite{TL_chuyen_Toan_Dai_So_Giai_Tich_11}, \textbf{4.}, p. 83]
	Tìm số các số nguyên dương không lớn hơn $1000$ mà chia hết cho $4$ hoặc cho $7$.
\end{baitoan}

\begin{baitoan}[\cite{TL_chuyen_Toan_Dai_So_Giai_Tich_11}, \textbf{5.}, p. 84]
	Người ta phỏng vấn $100$ người về 3 bộ phim A, B, C đang chiếu thì thu được kết quả sau: Bộ phim A có $28$ người đã xem. Bộ phim B có $26$ người đã xem. Bộ phim $C$ có $14$ người đã xem. Có $8$ người đã xem 2 bộ phim A \& B. Có $4$ người đã xem 2 bộ phim B \& C. Có $3$ người đã xem 2 bộ phim A \& C. Có $2$ người xem cả 3 bộ phim A, B, C. Xác định số người không đi xem bất cứ phim nào trong 3 bộ phim ấy.
\end{baitoan}

\begin{baitoan}[\cite{TL_chuyen_Toan_Dai_So_Giai_Tich_11}, \textbf{6.}, p. 84]
	Trong 1 trường có 3 câu lạc bộ (CLB) Toán, Văn, \& Ngoại ngữ. Có $28$ học sinh tham gia ít nhất 1 trong 3 CLB. Biết rằng:
	\begin{enumerate*}
		\item[(a)] Số học sinh chỉ tham gia CLB Toán, Văn bằng số học sinh chỉ tham gia duy nhất CLB Toán.
		\item[(b)] Số học sinh chỉ tham gia CLB Văn, Ngoại ngữ gấp $5$ lần số học sinh tham gia cả 3 CLB.
		\item[(c)] Có 6 học sinh chỉ tham gia CLB Toán, Ngoại ngữ.
		\item[(d)] Không có học sinh nào chỉ tham gia duy nhất 1 CLB Văn hoặc duy nhất 1 CLB ngoại ngữ.
		\item[(e)] Số học sinh tham gia cả 3 CLB là 1 số nguyên dương chẵn.
	\end{enumerate*}
	Tìm số học sinh chỉ tham gia CLB Toán \& Văn \& số học sinh tham gia cả 3 CLB.
\end{baitoan}

\begin{baitoan}[\cite{TL_chuyen_Toan_Dai_So_Giai_Tich_11}, \textbf{7.}, p. 84]
	1 con bò có thể mang virus A, virus B, hoặc virus C; có thể mang đồng thời 2 hoặc nhiều hơn các virus nói trên; \& cũng có thể không mang virus nào. Trong bản báo cáo của 1 nông trường nuôi bò cho biết: ``Kiểm tra $1200$ con bò thì có $675$ con có virus A; $682$ con có virus B; $684$ con có virus C; $195$ con có virus A \& B; $467$ con có virus A \& C; $318$ con có virus B \& C; $165$ con có virus A, B, C''. Chỉ ra rằng các số liệu trong báo cáo là không chính xác.
\end{baitoan}

%------------------------------------------------------------------------------%

\subsection{Hoán vị, chỉnh hợp, tổ hợp -- Permutation, arrangement, combination}
Mở rộng \cite[Ví dụ 1, p. 87]{TL_chuyen_Toan_Dai_So_Giai_Tich_11} cho trường hợp tổng quát $n$ điểm phân biệt:
 
\begin{baitoan}
	\begin{enumerate*}
		\item[(a)] Trong mặt phẳng cho 1 tập hợp $P$ gồm $6$ điểm phân biệt. Có bao nhiêu vector (khác $\vec{0}$) có điểm đầu \& điểm cuối thuộc $P$?
		\item[(b)] Trong mặt phẳng cho 1 tập hợp $Q$ gồm $7$ điểm phân biệt, trong đó không có 3 điểm nào thẳng hàng. Có bao nhiêu tam giác có 3 đỉnh đều thuộc $Q$?
	\end{enumerate*}
\end{baitoan}

\begin{baitoan}[\cite{TL_chuyen_Toan_Dai_So_Giai_Tich_11}, Ví dụ 3, p. 88]
	Chứng minh hằng đẳng thức
	\begin{align*}
		C_{2n}^2 = 2C_n^2 + n^2,\ \forall n\in\mathbb{N}.
	\end{align*}
	\begin{enumerate*}
		\item[(a)] Bằng biến đổi đại số.
		\item[(b)] Bằng suy luận tổ hợp.
	\end{enumerate*}
\end{baitoan}

\begin{baitoan}[\cite{TL_chuyen_Toan_Dai_So_Giai_Tich_11}, \textbf{8.}, p. 88]
	Hỏi có bao nhiêu số có $5$ chữ số khác nhau chia hết cho $5$ mà trong biểu diễn thập phân của nó không có các chữ số $7,8,9$?
\end{baitoan}

\begin{baitoan}[\cite{TL_chuyen_Toan_Dai_So_Giai_Tich_11}, \textbf{9.}, p. 88]
	Chứng minh
	\begin{align*}
		C_n^k + 3C_n^{k-1} + 3C_n^{k-2} + C_n^{k-3} = C_{n+3}^k,\ \forall n,k\in\mathbb{N}^\star,\,3\le k\le n.
	\end{align*}
\end{baitoan}

\begin{baitoan}[\cite{TL_chuyen_Toan_Dai_So_Giai_Tich_11}, \textbf{10.}, p. 88]
	Chứng minh
	\begin{align*}
		\sum_{k=0}^r C_n^kC_m^{r-k} = C_{m+n}^r,\ \forall m,n,r\in\mathbb{N}^\star,\,r < \min\{m,n\}.
	\end{align*}
\end{baitoan}

\begin{baitoan}[\cite{TL_chuyen_Toan_Dai_So_Giai_Tich_11}, \textbf{11.}, p. 88]
	Chứng minh bằng quy nạp
	\begin{align*}
		\sum_{k=0}^r C_{n+k}^k = C_{n+r+1}^r,\ \forall n,r\in\mathbb{N}^\star.
	\end{align*}
\end{baitoan}

\begin{baitoan}[\cite{TL_chuyen_Toan_Dai_So_Giai_Tich_11}, \textbf{12.}, p. 88]
	Trong 1 nhóm có $5$ người $A,B,C,D,E$.
	\begin{enumerate*}
		\item[(a)] Có bao nhiêu cách xếp $5$ người này thành hàng ngang sao cho $A$ \& $B$ đứng cạnh nhau?
		\item[(b)] Có bao nhiêu cách xếp $5$ người này thành hàng ngang sao cho $C$ \& $D$ không đứng cạnh nhau?
	\end{enumerate*}
\end{baitoan}

\begin{baitoan}[\cite{TL_chuyen_Toan_Dai_So_Giai_Tich_11}, \textbf{13.}, p. 88]
	Có bao nhiêu số có $5$ chữ số khác nhau mà biểu diễn thập phân không có các chữ số $6,7,8,9$?
\end{baitoan}

\begin{baitoan}[\cite{TL_chuyen_Toan_Dai_So_Giai_Tich_11}, \textbf{14.}, p. 88]
	1 lớp học có $n$ học sinh ($n > 3$). Thầy chủ nhiệm cần chọn ra 1 nhóm \& chỉ định 1 em trong nhóm làm nhóm trưởng. Số học sinh trong nhóm phải $\in(1;n)$. Gọi $T$ là số cách chọn.
	\begin{enumerate*}
		\item[(a)] Chứng minh $T = \sum_{k=2}^{n-1} kC_n^k$.
		\item[(b)] Chứng minh $T = n\left(2^{n-1} - 2\right)$.
		\item[(c)] Từ đó suy ra đẳng thức $\sum_{k=1}^n kC_n^k = n2^{n-1}$.
	\end{enumerate*}
\end{baitoan}

%------------------------------------------------------------------------------%

\subsection{Nhị thức Newton -- Newton's binomial theorem}

\begin{baitoan}[\cite{TL_chuyen_Toan_Dai_So_Giai_Tich_11}, Ví dụ 2, p. 92]
	Chứng minh công thức khai triển nhị thức Newton bằng suy luận tổ hợp.
\end{baitoan}

\begin{baitoan}[\cite{TL_chuyen_Toan_Dai_So_Giai_Tich_11}, \textbf{15.}, p. 94]
	Tìm số hạng không chứa $x$ (số hạng tự do) trong khai triển của $\left(x\sqrt{x} + \frac{1}{x^4}\right)^n$ nếu biết $C_n^2 - C_n^1 = 44$.
\end{baitoan}

\begin{baitoan}[\cite{TL_chuyen_Toan_Dai_So_Giai_Tich_11}, \textbf{16.}, p. 94]
	Tìm hệ số của $x^8$ trong khai triển của $(1 + x^2 - x^3)^8$.
\end{baitoan}

\begin{baitoan}[\cite{TL_chuyen_Toan_Dai_So_Giai_Tich_11}, \textbf{16.}, p. 94]
	Tìm hệ số của $x^4$ trong khai triển của $\left(x^2 - \frac{2}{x}\right)^8$.
\end{baitoan}

\begin{baitoan}[\cite{TL_chuyen_Toan_Dai_So_Giai_Tich_11}, \textbf{17.}, p. 94]
	Tìm hệ số của $x^7$ trong khai triển của $\left(\sqrt{x} + \sqrt[4]{x}\right)^{20}$.
\end{baitoan}

\begin{baitoan}[\cite{TL_chuyen_Toan_Dai_So_Giai_Tich_11}, \textbf{18.}, p. 94]
	Tìm hệ số của $x^7$ trong khai triển của $\left(\sqrt{x} + \frac{1}{x}\right)^{12}$.
\end{baitoan}

\begin{baitoan}[\cite{TL_chuyen_Toan_Dai_So_Giai_Tich_11}, \textbf{18.}, p. 94]
	Tìm hệ số của số hạng chứa $x^8$ trong khai triển nhị thức Newton của $\left(\frac{1}{x^3} + \sqrt{x^5}\right)^n$ biết $C_{n+4}^{n+1} - C_{n+3}^n = 7(n + 3)$.
\end{baitoan}
1 vài tổng quát cho các bài toán trên.

%------------------------------------------------------------------------------%

\subsection{Biến Cố \& Xác Suất của Biến Cố -- Event \& Probability of Event}

\begin{baitoan}[\cite{TL_chuyen_Toan_Dai_So_Giai_Tich_11}, Ví dụ 2, p. 97]
	Trong 1 nhóm có $k$ người. Biết không có ai sinh vào năm nhuận.
	\begin{enumerate*}
		\item[(a)] Mô tả không gian mẫu.
		\item[(b)] Gọi $A$ là biến cố: ``Trong nhóm $k$ người không có 2 người nào có ngày sinh trùng nhau''. Xác định tập con $A$ mô tả biến cố $A$.
		\item[(c)] Gọi $B$ là biến cố: ``Không có người nào trong nhóm có ngày sinh trùng với ngày Quốc khánh 2.9''. Xác định tập con $B$ mô tả biến cố $B$.
		\item[(d)] Giả thiết rằng các phần tử của không gian mẫu là đồng khả năng. Tính $P(A)$ \& $P(B)$.
		\item[(e)] Ký hiệu $P(k)$ là xác suất để trong nhóm $k$ người không có 2 người nào có ngày sinh trùng nhau \& $Q(k)$ là xác suất để trong nhóm $k$ người không có người nào trong nhóm có ngày sinh trùng với ngày Quốc khánh 2.9. Chứng minh $P(k),Q(k)$ giảm theo $k$.
	\end{enumerate*}
\end{baitoan}

\begin{baitoan}[\cite{TL_chuyen_Toan_Dai_So_Giai_Tich_11}, Ví dụ 4, p. 100]
	1 chiếc hộp có 9 thẻ đánh số từ $1$--$9$. Rút ngẫu nhiên 2 thẻ rồi nhân 2 số ghi trên 2 thẻ với nhau. Tính xác suất để kết quả nhận được là 1 số chẵn, bằng ít nhất 2 cách khác nhau.
\end{baitoan}

\begin{baitoan}[\cite{TL_chuyen_Toan_Dai_So_Giai_Tich_11}, Ví dụ 5, p. 100]
	1 hộp đựng $4$ viên bi xanh, $3$ viên bi đỏ, \& $2$ viên bi vàng. Chọn ngẫu nhiên 2 viên bị.
	\begin{enumerate*}
		\item[(a)] Tính xác suất để chọn được 2 viên bi cùng màu.
		\item[(b)] Tính xác suất để chọn được 2 viên bi khác màu.
	\end{enumerate*}
\end{baitoan}

%------------------------------------------------------------------------------%

\section{Dãy Số. Cấp Số Cộng \& Cấp Số Nhân -- Series. Arithmetic Progression\texttt{/}Sequence \& Geometric Progression\texttt{/}Sequence}

\begin{baitoan}
	Chứng minh bằng quy nạp quy tắc cộng \& quy tắc nhân tổng quát.
\end{baitoan}

\begin{baitoan}
	Chứng minh bằng quy nạp công thức tính số phần tử của hợp $n$, $n\in\mathbb{N}^\star$, $n\ge 2$ tập hợp bất kỳ.
\end{baitoan}

%------------------------------------------------------------------------------%

\section{Giới Hạn -- Limits}

%------------------------------------------------------------------------------%

\section{Đạo Hàm -- Derivative}

%------------------------------------------------------------------------------%

\section{Phép Dời Hình \& Phép Đồng Dạng Trong Mặt Phẳng}

\subsection{Phép tịnh tiến}

\begin{baitoan}[\cite{SGK_Toan_11_dai_so_giai_tich_co_ban}, \textbf{1.}, p. 7]
	\label{prob:phep tinh tien}
	Chứng minh: $M' = T_{\vec{u}}(M)\Leftrightarrow M = T_{-\vec{u}}(M')$.
\end{baitoan}

\begin{baitoan}
	Trong mặt phẳng tọa độ $Oxy$ cho vector $\vec{u} = (u_1;u_2)$, 2 điểm $A(a_1;a_2)\not\equiv B(b_1,b_2)$ (i.e., phân biệt\emph{\texttt{/}}không trùng nhau) \& đường thẳng $d$ có phương trình $ax + by + c = 0$ với $a,b\in\mathbb{R}$ sao cho $a^2 + b^2\ne 0$.
	\begin{enumerate*}
		\item[(a)] Tìm tọa độ của các điểm $A',B'$ theo thứ tự là ảnh của $A,B$ qua phép tịnh tiến theo $\vec{u}$.
		\item[(b)] Tìm tọa độ của điểm $C$ sao cho $A$ là ảnh của $C$ qua phép tịnh tiến theo $\vec{u}$.
		\item[(c)] Tìm phương trình của đường thẳng $d'$  là ảnh của $d$ qua phép tịnh tiến theo $\vec{u}$.
		\item[(d)] Cho đường tròn tâm $O_1(o_1;o_2)$ bán kính $R > 0$ có phương trình: $(x - o_1)^2 + (y - o_2)^2 = R^2$. Tìm ảnh của đường tròn đó qua phép tịnh tiến theo $\vec{u}$.
	\end{enumerate*}
\end{baitoan}
Có thể mở rộng bài toán trên cho các đường conic, i.e., ellipse, parabol, \& hyperbol.

\begin{baitoan}[\cite{SGK_Toan_11_hinh_hoc_nang_cao}, p. 7]
	Cho 2 điểm $B,C$ cố định trên đường tròn $(O;R)$ \& 1 điểm $A$ thay đổi trên đường tròn đó. Chứng minh rằng trực tâm $\Delta ABC$ nằm trên 1 đường tròn cố định.
\end{baitoan}

\begin{baitoan}
	2 thôn nằm ở 2 vị trí $A$ \& $B$ cách nhau 1 con sông (xem rằng 2 bờ sông là 2 đường thẳng song song). Người ta dự định xây 1 chiếc cầu $MN$ bắt qua sông (cố nhiên cầu phải vuông góc với bờ sông) \& làm 2 đoạn thẳng từ $A$ đến $M$ \& từ $B$ đến $N$. Hãy xác định vị trí chiếc cầu $MN$ sao cho $AM + BN$ ngắn nhất.
\end{baitoan}

\begin{proof}[Hint]
	Trường hợp tổng quát có thể đưa về trường hợp con sông rất hẹp -- hẹp đến mức 2 bờ sông $a$ \& $b$ xem như trùng nhau bằng 1 phép tịnh tiến theo vector $\overrightarrow{MN}$ để $a$ trùng $b$. Khi đó điểm $A$ biến thành điểm $A'$ sao cho $\overrightarrow{AA'} = \overrightarrow{MN}$ \& do đó $A'N = AM$.
\end{proof}

\begin{baitoan}[\cite{SGK_Toan_11_hinh_hoc_nang_cao}, \textbf{1.}, p. 9]
	Qua phép tịnh tiến $T$ theo vector $\vec{u}\ne\vec{0}$, đường thẳng $d$ biến thành đường thẳng $d'$. Trong trường hợp nào thì: $d\equiv d'$? $d\parallel d'$? $d$ cắt $d'$? 
\end{baitoan}

\begin{baitoan}[\cite{SGK_Toan_11_hinh_hoc_nang_cao}, \textbf{2.}, p. 9]
	Cho 2 đường thẳng song song $a$ \& $a'$. Tìm tất cả những phép tịnh tiến biến $a$ thành $a'$.
\end{baitoan}

\begin{baitoan}[\cite{SGK_Toan_11_hinh_hoc_nang_cao}, \textbf{3.}, p. 9]
	Cho 2 phép tịnh tiến $T_{\vec{u}}$ \& $T_{\vec{v}}$. Với điểm $M$ bất kỳ, $T_{\vec{u}}$ biến $M$ thành điểm $M'$, $T_{\vec{v}}$ biến $M'$ thành điểm $M''$. Chứng tỏ rằng phép biến hình biến $M$ thành $M''$ là 1 phép tịnh tiến.
\end{baitoan}

\begin{baitoan}[\cite{SGK_Toan_11_hinh_hoc_nang_cao}, \textbf{4.}, p. 9]
	Cho đường tròn $(O)$ \& 2 điểm $A,B$. 1 điểm $M$ thay đổi trên đường tròn $(O)$. Tìm quỹ tích điểm $M'$ sao cho $\overrightarrow{MM'} + \overrightarrow{MA} = \overrightarrow{MB}$.
\end{baitoan}

\begin{baitoan}[\cite{SGK_Toan_11_hinh_hoc_nang_cao}, \textbf{5.}, p. 9]
	Trong mặt phẳng tọa độ $Oxy$, với $\alpha,a,b$ là những số cho trước, xét phép biến hình $F$ biến mỗi điểm $M(x;y)$ thành điểm $M'(x';y')$, trong đó
	\begin{equation*}
		\left\{\begin{split}
			x' &= x\cos\alpha - y\sin\alpha + a,\\
			y' &= x\sin\alpha + y\cos\alpha + b.
		\end{split}\right.
	\end{equation*}
	\begin{enumerate*}
		\item[(a)] Cho 2 điểm $M(x_1;y_1)$, $N(x_2;y_2)$ \& gọi $M',N'$ lần lượt là ảnh của $M,N$ qua phép $F$. Tìm tọa độ của $M'$ \& $N'$.s
		\item[(b)] Tính khoảng cách $d$ giữa $M$ \& $N$; khoảng cách $d'$ giữa $M'$ \& $N'$.
		\item[(c)] Phép $F$ có phải là phép dời hình hay không?
		\item Khi $\alpha = 0$, chứng tỏ rằng $F$ là phép tịnh tiến.
	\end{enumerate*}
\end{baitoan}
Tổng quát hơn của bài toán \cite[\textbf{6.}, p. 9]{SGK_Toan_11_hinh_hoc_nang_cao}:
\begin{baitoan}
	Trong mặt phẳng tọa độ $Oxy$, xét các phép biến hình sau đây: Phép biến hình $F$ biến mỗi điểm $M(x;y)$ thành điểm $M'(f(x,y);g(x;y))$ với $f,g:\mathbb{R}^2\to\mathbb{R}$ là 2 hàm sốt. Với $f,g$ thỏa điều kiện nào thì $F$ là 1 phép dời hình?
\end{baitoan}

%------------------------------------------------------------------------------%

\subsection{Phép đối xứng trục}

\begin{baitoan}
	Trong mặt phẳng tọa độ $Oxy$ cho điểm $A(a_1;a_2)$ \& đường thẳng $d$ có phương trình $ax + by + c = 0$ ($a^2 + b^2\ne 0$). Tìm ảnh của $A$ \& đường thẳng $d$ qua phép đối xứng trục $d_1$ có phương trình $a_1x + b_1y + c_1 = 0$ ($a_1^2 + b_1^2\ne 0$). Suy ra các trường hợp riêng khi $d_1$ là $Ox$ hoặc $Oy$. Cho đường tròn tâm $O_1(o_1;o_2)$ bán kính $R > 0$ có phương trình: $(x - o_1)^2 + (y - o_2)^2 = R^2$. Tìm ảnh của đường tròn đó qua phép đối xứng trục $d$.
\end{baitoan}
Có thể mở rộng bài toán trên cho các đường conic, i.e., ellipse, parabol, \& hyperbol.

%------------------------------------------------------------------------------%

\subsection{Phép đối xứng tâm}

\begin{baitoan}
	Trong mặt phẳng tọa độ $Oxy$ cho điểm $A(a_1;a_2)$ \& đường thẳng $d$ có phương trình $ax + by + c = 0$ ($a^2 + b^2\ne 0$). Tìm ảnh của $A$ \& đường thẳng $d$ qua phép đối xứng tâm $O$ \& phép đối xứng tâm $O_1(o_1;o_2)$. Cho đường tròn tâm $O_1(o_1;o_2)$ bán kính $R > 0$ có phương trình: $(x - o_1)^2 + (y - o_2)^2 = R^2$. Tìm ảnh của đường tròn đó qua phép đối xứng tâm $O$ \& phép đối xứng tâm $O_1$.
\end{baitoan}
Có thể mở rộng bài toán trên cho các đường conic, i.e., ellipse, parabol, \& hyperbol.

%------------------------------------------------------------------------------%

\subsection{Phép quay}

\begin{baitoan}
	Trong mặt phẳng toaj độ $Oxy$ cho điểm $A(a_1;a_2)$ \& đường thẳng $d$ có phương trình $ax + by + c = 0$ ($a^2 + b^2\ne 0$). Tìm ảnh của $A$ \& $d$ qua phép quay tâm $O$ góc $\alpha$ (radian). Cho đường tròn tâm $O_1(o_1;o_2)$ bán kính $R > 0$ có phương trình: $(x - o_1)^2 + (y - o_2)^2 = R^2$. Tìm ảnh của đường tròn đó qua phép quay tâm $O$ góc $\alpha$ (radian).
\end{baitoan}


%------------------------------------------------------------------------------%

\section{Đường Thẳng \& Mặt Phẳng Trong Không Gian -- Line \& Plane in Euclidean Space $\mathbb{R}^n$}

%------------------------------------------------------------------------------%

\section{Vector Trong Không Gian. Quan Hệ Vuông Góc -- Vector in Euclidean Space $\mathbb{R}^n$. Perpendicular Relation}

%------------------------------------------------------------------------------%

\section{Solutions}

%------------------------------------------------------------------------------%

Tài liệu: \cite{SGK_Toan_11_dai_so_giai_tich_co_ban, SGK_Toan_11_dai_so_giai_tich_nang_cao, SGK_Toan_11_hinh_hoc_co_ban, SGK_Toan_11_hinh_hoc_nang_cao, TL_chuyen_Toan_Dai_So_Giai_Tich_11, TL_chuyen_Toan_Hinh_Hoc_11}.

\printbibliography[heading=bibintoc]
	
\end{document}