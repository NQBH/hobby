\documentclass[oneside]{book}
\usepackage[backend=biber,natbib=true,style=authoryear]{biblatex}
\addbibresource{/home/hong/1_NQBH/reference/bib.bib}
\usepackage[utf8]{vietnam}
\usepackage{tocloft}
\renewcommand{\cftsecleader}{\cftdotfill{\cftdotsep}}
\usepackage[colorlinks=true,linkcolor=blue,urlcolor=red,citecolor=magenta]{hyperref}
\usepackage{amsmath,amssymb,amsthm,mathtools,float,graphicx,algpseudocode,algorithm,tcolorbox,tikz,tkz-tab}
\DeclareMathOperator{\arccot}{arccot}
\usepackage[inline]{enumitem}
\allowdisplaybreaks
\numberwithin{equation}{section}
\newtheorem{assumption}{Assumption}[section]
\newtheorem{nhanxet}{Nhận xét}[section]
\newtheorem{conjecture}{Conjecture}[section]
\newtheorem{corollary}{Corollary}[section]
\newtheorem{hequa}{Hệ quả}[section]
\newtheorem{definition}{Definition}[section]
\newtheorem{dinhnghia}{Định nghĩa}[section]
\newtheorem{example}{Example}[section]
\newtheorem{vidu}{Ví dụ}[section]
\newtheorem{lemma}{Lemma}[section]
\newtheorem{notation}{Notation}[section]
\newtheorem{principle}{Principle}[section]
\newtheorem{problem}{Problem}[section]
\newtheorem{baitoan}{Bài toán}[section]
\newtheorem{proposition}{Proposition}[section]
\newtheorem{menhde}{Mệnh đề}[section]
\newtheorem{question}{Question}[section]
\newtheorem{cauhoi}{Câu hỏi}[section]
\newtheorem{remark}{Remark}[section]
\newtheorem{luuy}{Lưu ý}[section]
\newtheorem{theorem}{Theorem}[section]
\newtheorem{tiende}{Tiên đề}[section]
\newtheorem{dinhly}{Định lý}[section]
\usepackage[left=0.5in,right=0.5in,top=1.5cm,bottom=1.5cm]{geometry}
\usepackage{fancyhdr}
\pagestyle{fancy}
\fancyhf{}
\lhead{\small \textsc{Sect.} ~\thesection}
\rhead{\small \nouppercase{\leftmark}}
\renewcommand{\sectionmark}[1]{\markboth{#1}{}}
\cfoot{\thepage}
\def\labelitemii{$\circ$}

\title{Some Topics in Elementary Mathematics\texttt{/}Grade 12}
\author{Nguyễn Quản Bá Hồng\footnote{Independent Researcher, Ben Tre City, Vietnam\\e-mail: \texttt{nguyenquanbahong@gmail.com}; website: \url{https://nqbh.github.io}.}}
\date{\today}

\begin{document}
\frontmatter
\maketitle
\setcounter{secnumdepth}{4}
\setcounter{tocdepth}{3}
\tableofcontents
\newpage

%------------------------------------------------------------------------------%

\mainmatter

\part{Đại Số \& Giải Tích -- Algebra \& Analysis}

\chapter{Ứng Dụng Đạo Hàm Để Khảo Sát \& Vẽ Đồ Thị của Hàm Số}

\section{Tính Đơn Điệu của Hàm Số}

%------------------------------------------------------------------------------%

\section{Cực Trị của Hàm Số}

%------------------------------------------------------------------------------%

\section{Giá Trị Lớn Nhất \& Giá Trị Nhỏ Nhất của Hàm Số}

%------------------------------------------------------------------------------%

\section{Đồ Thị của Hàm Số \& Phép Tịnh Tiến Hệ Tọa Độ}

%------------------------------------------------------------------------------%

\section{Đường Tiệm Cận của Đồ Thị Hàm Số}

%------------------------------------------------------------------------------%

\section{Khảo Sát Sự Biến Thiên \& Vẽ Đồ Thị của 1 Số Hàm Đa Thức}

%------------------------------------------------------------------------------%

\section{Khảo Sát Sự Biến Thiên \& Vẽ Đồ Thị của 1 Số Hàm Phân Thức Hữu Tỷ}

%------------------------------------------------------------------------------%

\section{1 Số Bài Toán Thường Gặp về Đồ Thị}

%------------------------------------------------------------------------------%

\chapter{Hàm Số Lũy Thừa, Hàm Số Mũ, \& Hàm Số Logarith}

\section{Lũy Thừa với Số Mũ Hữu Tỷ}

%------------------------------------------------------------------------------%

\section{Lũy Thừa với Số Mũ Thực}

%------------------------------------------------------------------------------%

\section{Logarithm}

%------------------------------------------------------------------------------%

\section{Số $\rm e$ \& Logarith Tự Nhiên}

%------------------------------------------------------------------------------%

\section{Hàm Số Mũ \& Hàm Số Logarithm}

%------------------------------------------------------------------------------%

\section{Hàm Số Lũy Thừa}

%------------------------------------------------------------------------------%

\section{Phương Trình Mũ \& Logarithm}

%------------------------------------------------------------------------------%

\section{Hệ Phương Trình Mũ \& Logarithm}

%------------------------------------------------------------------------------%

\section{Bất Phương Trình Mũ \& Logarithm}

%------------------------------------------------------------------------------%

\chapter{Nguyên Hàm, Tích Phân, \& Ứng Dụng}

\section{Nguyên Hàm}

%------------------------------------------------------------------------------%

\section{1 Số Phương Pháp Tìm Nguyên Hàm}

%------------------------------------------------------------------------------%

\section{Tích Phân}

%------------------------------------------------------------------------------%

\section{1 Số Phương Pháp Tính Tích Phân}

%------------------------------------------------------------------------------%

\section{Ứng Dụng Tích Phân Để Tính Diện Tích Hình Phẳng}

%------------------------------------------------------------------------------%

\section{Ứng Dụng Tích Phân Để Tính Thể Tích Vật Thể}

%------------------------------------------------------------------------------%

\chapter{Số Phức}

\section{Số Phức}

%------------------------------------------------------------------------------%

\section{Căn Bậc 2 của Số Phức \& Phương Trình Bậc 2}

%------------------------------------------------------------------------------%

\section{Dạng Lượng Giác của Số Phức \& Ứng Dụng}

%------------------------------------------------------------------------------%

\part{Hình Học -- Geometry}

\chapter{Khối Đa Diện \& Thể Tích của Chúng}

\section{Khái Niệm về Khối Đa Diện}

%------------------------------------------------------------------------------%

\section{Phép Đối Xứng qua Mặt Phẳng \& Sự Bằng Nhau của Các Khối Đa Diện}

%------------------------------------------------------------------------------%

\section{Phép Vị Tự \& Sự Đồng Dạng của Các Khối Đa Diện. Các Khối Đa Diện Đều}

%------------------------------------------------------------------------------%

\section{Thể Tích của Khối Đa Diện}

%------------------------------------------------------------------------------%

\chapter{Mặt Cầu, Mặt Trụ, Mặt Nón}

\section{Mặt Cầu, Khối Cầu}

%------------------------------------------------------------------------------%

\section{Khái Niệm về Mặt Tròn Xoay}

%------------------------------------------------------------------------------%

\section{Mặt Trụ, Hình Trụ, \& Khối Trụ}

%------------------------------------------------------------------------------%

\section{Mặt Nón, Hình Nón, \& Khối Nón}

%------------------------------------------------------------------------------%

\chapter{Phương Pháp Tọa Độ Trong Không Gian}

\section{Hệ Tọa Độ Trong Không Gian}

%------------------------------------------------------------------------------%

\section{Phương Trình Mặt Phẳng}

%------------------------------------------------------------------------------%

\section{Phương Trình Đường Thẳng}

%------------------------------------------------------------------------------%

\printbibliography[heading=bibintoc]
	
\end{document}