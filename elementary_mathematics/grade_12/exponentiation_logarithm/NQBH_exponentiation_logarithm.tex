\documentclass{article}
\usepackage[backend=biber,natbib=true,style=authoryear]{biblatex}
\addbibresource{/home/nqbh/reference/bib.bib}
\usepackage[utf8]{vietnam}
\usepackage{tocloft}
\renewcommand{\cftsecleader}{\cftdotfill{\cftdotsep}}
\usepackage[colorlinks=true,linkcolor=blue,urlcolor=red,citecolor=magenta]{hyperref}
\usepackage{amsmath,amssymb,amsthm,mathtools,float,graphicx,algpseudocode,algorithm,tcolorbox,tikz,tkz-tab,subcaption}
\DeclareMathOperator{\arccot}{arccot}
\usepackage[inline]{enumitem}
\allowdisplaybreaks
\numberwithin{equation}{section}
\newtheorem{assumption}{Assumption}[section]
\newtheorem{nhanxet}{Nhận xét}[section]
\newtheorem{conjecture}{Conjecture}[section]
\newtheorem{corollary}{Corollary}[section]
\newtheorem{hequa}{Hệ quả}[section]
\newtheorem{definition}{Definition}[section]
\newtheorem{dinhnghia}{Định nghĩa}[section]
\newtheorem{example}{Example}[section]
\newtheorem{vidu}{Ví dụ}[section]
\newtheorem{lemma}{Lemma}[section]
\newtheorem{notation}{Notation}[section]
\newtheorem{principle}{Principle}[section]
\newtheorem{problem}{Problem}[section]
\newtheorem{baitoan}{Bài toán}[section]
\newtheorem{proposition}{Proposition}[section]
\newtheorem{menhde}{Mệnh đề}[section]
\newtheorem{question}{Question}[section]
\newtheorem{cauhoi}{Câu hỏi}[section]
\newtheorem{quytac}{Quy tắc}
\newtheorem{remark}{Remark}[section]
\newtheorem{luuy}{Lưu ý}[section]
\newtheorem{theorem}{Theorem}[section]
\newtheorem{tiende}{Tiên đề}[section]
\newtheorem{dinhly}{Định lý}[section]
\usepackage[left=0.5in,right=0.5in,top=1.5cm,bottom=1.5cm]{geometry}
\usepackage{fancyhdr}
\pagestyle{fancy}
\fancyhf{}
\lhead{\small Sect.~\thesection}
\rhead{\small\nouppercase{\leftmark}}
\renewcommand{\subsectionmark}[1]{\markboth{#1}{}}
\cfoot{\thepage}
\def\labelitemii{$\circ$}
\DeclareRobustCommand{\divby}{%
	\mathrel{\vbox{\baselineskip.65ex\lineskiplimit0pt\hbox{.}\hbox{.}\hbox{.}}}%
}

\title{Exponentiation \textit{\&} Logarithm\\Hàm Số Lũy Thừa, Hàm Số Mũ, \textit{\&} Hàm Số Logarith}
\author{Nguyễn Quản Bá Hồng\footnote{Independent Researcher, Ben Tre City, Vietnam\\e-mail: \texttt{nguyenquanbahong@gmail.com}; website: \url{https://nqbh.github.io}.}}
\date{\today}

\begin{document}
\maketitle
\begin{abstract}
	
\end{abstract}
\setcounter{secnumdepth}{4}
\setcounter{tocdepth}{3}
\tableofcontents
\newpage

%------------------------------------------------------------------------------%

\section{Cheatsheet}

\begin{dinhly}[So sánh các lũy thừa cùng cơ số]
	\label{thm: so sanh luy thua cung co so}
	$\forall m,n\in\mathbb{Z}$, $a^m > a^n\Leftrightarrow m > n$, $\forall a > 1$; $a^m > a^n\Leftrightarrow m < n$, $\forall a\in(0,1)$.
\end{dinhly}

%------------------------------------------------------------------------------%

\section{Problem}

\begin{baitoan}[\cite{TL_chuyen_Toan_Giai_Tich_12}, Ví dụ 1, p. 42]
	Không dùng máy tính, so sánh $99^{100} + 100^{100}$ \& $101^{100}$.
\end{baitoan}

\begin{proof}[Giải]
	Có $99^{100} + 100^{100}\le 2\cdot 100^{100}$, cần chứng minh $2\cdot 100^{100} < 101^{100}$. Thật vậy, theo bất đẳng thức Bernoulli: $\left(\frac{101}{100}\right)^{100} = \left(1 + \frac{1}{100}\right)^{100} > 1 + 100\cdot\frac{1}{100} = 2\Rightarrow 2\cdot 100^{100} < 101^{100}$. Do đó $99^{100} + 100^{100} < 101^{100}$.
\end{proof}
``Các bất đẳng thức dạng này khá yếu \& thường khi giải bất phương trình mũ, ta sẽ dùng các đánh giá trung gian đưa về cùng số mũ hoặc cùng cơ số rồi so sánh dựa vào định lý \ref{thm: so sanh luy thua cung co so}.'' -- \cite[p. 42]{TL_chuyen_Toan_Giai_Tich_12}

``In mathematics, \textit{Bernoulli's inequality} (named after \href{https://en.wikipedia.org/wiki/Jacob_Bernoulli}{Jacob Bernoulli}) is an \href{https://en.wikipedia.org/wiki/Inequality_(mathematics)}{inequality} that approximates \href{https://en.wikipedia.org/wiki/Exponentiation}{exponentiations} of $1 + x$. It is often employed in \href{https://en.wikipedia.org/wiki/Real_analysis}{real analysis}. It has several useful variants:
\begin{enumerate*}
	\item[$\bullet$] $(1 + x)^r\ge 1 + rx$, $\forall r\in\mathbb{N}$, $\forall x\in\mathbb{R}$, $x > - 1$. The inequality is strict if $x\ne 0$ \& $r\ge 2$.
	\item[$\bullet$] $(1 + x)^r\ge 1 + rx$, $\forall r\in\mathbb{N}$, $r\divby 2$, $\forall x\in\mathbb{R}$.
	\item[$\bullet$] $(1 + x)^r\ge 1 + rx$, $\forall r\in\mathbb{N}$, $\forall x\ge -2$.
	\item[$\bullet$] $(1 + x)^r\ge 1 + rx$, $\forall r\in[1,\infty)$, $x\ge -1$. The inequalities are strict if $x\ne 0$ \& $r\notin\{0,1\}$.
	 \item[$\bullet$] $(1 + x)^r\le 1 + rx$, $\forall r\in[0,1]$, $x\ge -1$.'' -- \href{https://en.wikipedia.org/wiki/Bernoulli%27s_inequality}{Wikipedia\texttt{/}Bernoulli's inequality}
\end{enumerate*}

\begin{dinhly}[Bernoulli's inequality]
	$(1 + x)^r\ge 1 + rx$, $\forall r\in\mathbb{N}$, $\forall x\in\mathbb{R}$, $x > - 1$.
\end{dinhly}

\begin{baitoan}[Mở rộng \cite{TL_chuyen_Toan_Giai_Tich_12}, Ví dụ 1, p. 42]
	So sánh $m^n + (m + 1)^n$ \& $(m + 2)^n$.
\end{baitoan}

\begin{baitoan}[\cite{TL_chuyen_Toan_Giai_Tich_12}, Ví dụ 2, p. 42]
	Chứng minh:
	\begin{align*}
		\frac{a^7 + b^7 + c^7}{7} = \frac{a^4 + b^4 + c^4}{2}\frac{a^3 + b^3 + c^3}{3},\ \forall a,b,c\in\mathbb{R},\,a + b + c = 0.
	\end{align*}
\end{baitoan}

\begin{baitoan}[\cite{TL_chuyen_Toan_Giai_Tich_12}, H1, p. 42]
	Với những giá trị nguyên dương nào của $n$ thì $\sum_{i=1}^{9} i^n = 1^n + 2^n + \cdots + 9^n < 10^n$?
\end{baitoan}

\begin{baitoan}[\cite{TL_chuyen_Toan_Giai_Tich_12}, Ví dụ 3, p. 43]
	Chứng minh: $\sqrt{x + 4\sqrt{x - 4}} + \sqrt{x - 4\sqrt{x - 4}} = \mbox{const}$, $\forall x\in[4,8]$.
\end{baitoan}

\begin{proof}[Giải]
	$\sqrt{x + 4\sqrt{x - 4}} + \sqrt{x - 4\sqrt{x - 4}} = \sqrt{(\sqrt{x - 4} + 2)^2} + \sqrt{(\sqrt{x - 4} - 2)^2} = |\sqrt{x - 4} + 2| + |\sqrt{x - 4} - 2| = \sqrt{x - 4} + 2 + 2 - \sqrt{x - 4} = 4$, $\forall x\in[4,8]$, trong đó $|\sqrt{x - 4} - 2| = 2 - \sqrt{x - 4}$ vì $x\le 8$, nên $\sqrt{x - 4}\le\sqrt{8 - 4} = 2$.
\end{proof}

\begin{baitoan}[Mở rộng \cite{TL_chuyen_Toan_Giai_Tich_12}, Ví dụ 3, p. 43]
	Biện luận theo tham số $a$ để rút gọn biểu thức $A = \sqrt{x + 2a\sqrt{x - a^2}} + \sqrt{x - 2a\sqrt{x - a^2}}$ \& $B = \sqrt{x + 2a\sqrt{x - a^2}} - \sqrt{x - 2a\sqrt{x - a^2}}$.
\end{baitoan}

\begin{baitoan}[\cite{TL_chuyen_Toan_Giai_Tich_12}, H2, p. 43]
	Rút gọn biểu thức $M = \sqrt[3]{11\sqrt{2} + 9\sqrt{3}} + \sqrt[3]{11\sqrt{2} - 9\sqrt{3}}$.
\end{baitoan}
\noindent\textit{Phân tích.} Dưới dấu $\sqrt[3]{\cdot}$ là biểu thức có dạng $A\sqrt{2} + B\sqrt{3}$, ta nghĩ ngay đến $(a\sqrt{2} + b\sqrt{3})^3 = 2a^3\sqrt{2} + 6a^2b\sqrt{3} + 9ab^2\sqrt{2} + 3b^3\sqrt{3} = (2a^3 + 9ab^2)\sqrt{2} + (6a^2b + 3b^3)\sqrt{3}$. Đồng nhất hệ số: $2a^3 + 9ab^2 = 11$ \& $6a^2b + 3b^3 = 9$, suy ra $a = b = 1$, hay $11\sqrt{2}\pm9\sqrt{3} = (\sqrt{2}\pm\sqrt{3})^3$.

\begin{proof}[Giải]
	$M = \sqrt[3]{11\sqrt{2} + 9\sqrt{3}} + \sqrt[3]{11\sqrt{2} - 9\sqrt{3}} = \sqrt[3]{(\sqrt{2} + \sqrt{3})^3} + \sqrt[3]{(\sqrt{2} - \sqrt{3})^3} = \sqrt{2} + \sqrt{3} + \sqrt{2} - \sqrt{3} = 2\sqrt{2}$.
\end{proof}
Từ phân tích trên, ta có 1 mở rộng của bài toán vừa giải như sau:

\begin{baitoan}[Mở rộng \cite{TL_chuyen_Toan_Giai_Tich_12}, H2, p. 43]
	Rút gọn biểu thức
	\begin{align*}
		A = \sqrt[3]{(2a^3 + 9ab^2)\sqrt{2} + (6a^2b + 3b^3)\sqrt{3}} + \sqrt[3]{(2a^3 + 9ab^2)\sqrt{2} - (6a^2b + 3b^3)\sqrt{3}},\ \forall a,b\in\mathbb{R}.
	\end{align*}
\end{baitoan}

\begin{proof}[Giải]
	$A = \sqrt[3]{(2a^3 + 9ab^2)\sqrt{2} + (6a^2b + 3b^3)\sqrt{3}} + \sqrt[3]{(2a^3 + 9ab^2)\sqrt{2} - (6a^2b + 3b^3)\sqrt{3}} = \sqrt[3]{(a\sqrt{2} + b\sqrt{3})^3} + \sqrt[3]{(a\sqrt{2} - b\sqrt{3})^3} = a\sqrt{2} + b\sqrt{3} + a\sqrt{2} - b\sqrt{3} = 2a\sqrt{2}$, $\forall a,b\in\mathbb{R}$.
\end{proof}

\begin{luuy}
	Kết quả rút gọn của biểu thức $A$ chỉ phụ thuộc vào mỗi tham số $a$, \& độc lập với tham số $b$.
\end{luuy}
Mở rộng hơn nữa bằng cách thay $\sqrt{2},\sqrt{3}$ bởi $\sqrt{m},\sqrt{n}$:

\begin{baitoan}[Mở rộng \cite{TL_chuyen_Toan_Giai_Tich_12}, H2, p. 43]
	Rút gọn biểu thức
	
\end{baitoan}
Mở rộng hơn nữa bằng cách thay $\sqrt{\cdot},\sqrt[3]{\cdot}$ bởi $\sqrt[n]{\cdot}$.
\begin{baitoan}[Mở rộng \cite{TL_chuyen_Toan_Giai_Tich_12}, H2, p. 43]
	Rút gọn biểu thức
	
\end{baitoan}

%------------------------------------------------------------------------------%

\section{Phương Trình, Bất Phương Trình Mũ \& Logarithm -- Exponential \& Logarithmic Equation \& Inequation}
\textit{1 số tính chất cơ bản của phương trình mũ \& logarithm}:
\begin{enumerate*}
	\item[$\bullet$] Phương trình $a^x = m$, $0 < a\ne 1$. Nếu $m\le 0$ thì phương trình vô nghiệm. Nếu $m > 0$ thì phương trình có nghiệm duy nhất là $x = \log_am$.
	\item[$\bullet$] Phương trình $\log_ax = m$, $0 < a\ne 1$ luôn có nghiệm duy nhất là $x = a^m$.
\end{enumerate*}

\subsection{Phương trình có dạng $a^{f(x)} = b^{g(x)}$}
``\textit{Cách giải chung}:
\begin{enumerate*}
	\item[$\bullet$] Nếu $a = b$ thì theo tính chất của hàm số mũ, ta có $f(x) = g(x)$, đưa bài toán về dạng đơn giản hơn -- \textit{phương pháp đưa về cùng cơ số}.
	\item[$\bullet$] Nếu $a\ne b$ thì lấy logarith cơ số $a$ (hoặc $b$) 2 vế đưa về $f(x) = \log_ab\cdot g(x)$. Do $\log_ab$ cũng là 1 hằng số nên tính chất của phương trình này cũng tương tự trường hợp $a = b$.'' -- \cite[p. 71]{TL_chuyen_Toan_Giai_Tich_12}
\end{enumerate*}

\begin{baitoan}[\cite{TL_chuyen_Toan_Giai_Tich_12}, Ví dụ 1, p. 71]
	Giải phương trình sau:
	\begin{enumerate*}
		\item[(a)] $(5^{x+2})^{x+1} + (5^x)^{x+3} = (2^{x+1})^{x+5} - 6(2^{x+6})^x$.
		\item[(b)] $(10 + 6\sqrt{3})^{2\sin x} = \sqrt{(\sqrt{3} + 1)^{\sin 4x}}$.
		\item[(c)] $\left(\frac{8}{3}\right)^{x^2 - x + 1}\left(\frac{3}{5}\right)^{2x^2 - 3x + 2}\left(\frac{5}{7}\right)^{3x^2 - 4x + 3}\left(\frac{7}{2}\right)^{4x^2 - 5x + 4} = 210^{(x - 1)^2}$.
	\end{enumerate*}
\end{baitoan}

\begin{baitoan}[\cite{TL_chuyen_Toan_Giai_Tich_12}, H1, p. 72]
	\begin{enumerate*}
		\item[(a)] Giải các phương trình sau: $2^x\cdot 3^x\cdot 4^{x^2} = 4\cdot 36^{\frac{x}{x + 1}}$.
		\item[(b)] Với $a > 1$, giải phương trình $\left(\frac{a}{a^2 + 1}\right)^x\left(\frac{a^2 - 1}{a^2 - 1}\right)^{2x} = 6$.
	\end{enumerate*}
\end{baitoan}



%------------------------------------------------------------------------------%

\printbibliography[heading=bibintoc]
	
\end{document}