\documentclass{article}
\usepackage[backend=biber,natbib=true,style=authoryear,maxbibnames=50]{biblatex}
\addbibresource{/home/nqbh/reference/bib.bib}
\usepackage[utf8]{vietnam}
\usepackage{tocloft}
\renewcommand{\cftsecleader}{\cftdotfill{\cftdotsep}}
\usepackage[colorlinks=true,linkcolor=blue,urlcolor=red,citecolor=magenta]{hyperref}
\usepackage{amsmath,amssymb,amsthm,mathtools,float,graphicx,algpseudocode,algorithm,tcolorbox}
\usepackage[inline]{enumitem}
\allowdisplaybreaks
\numberwithin{equation}{section}
\newtheorem{assumption}{Assumption}[section]
\newtheorem{baitoan}{Bài toán}
\newtheorem{cauhoi}{Câu hỏi}[section]
\newtheorem{conjecture}{Conjecture}[section]
\newtheorem{corollary}{Corollary}[section]
\newtheorem{dangtoan}{Dạng toán}[section]
\newtheorem{definition}{Definition}[section]
\newtheorem{dinhly}{Định lý}[section]
\newtheorem{dinhnghia}{Định nghĩa}[section]
\newtheorem{example}{Example}[section]
\newtheorem{ghichu}{Ghi chú}[section]
\newtheorem{hequa}{Hệ quả}[section]
\newtheorem{hypothesis}{Hypothesis}[section]
\newtheorem{lemma}{Lemma}[section]
\newtheorem{luuy}{Lưu ý}[section]
\newtheorem{nhanxet}{Nhận xét}[section]
\newtheorem{notation}{Notation}[section]
\newtheorem{note}{Note}[section]
\newtheorem{principle}{Principle}[section]
\newtheorem{problem}{Problem}[section]
\newtheorem{proposition}{Proposition}[section]
\newtheorem{question}{Question}[section]
\newtheorem{remark}{Remark}[section]
\newtheorem{theorem}{Theorem}[section]
\newtheorem{vidu}{Ví dụ}[section]
\usepackage[left=0.5in,right=0.5in,top=1.5cm,bottom=1.5cm]{geometry}
\usepackage{fancyhdr}
\pagestyle{fancy}
\fancyhf{}
\lhead{\small Sect.~\thesection}
\rhead{\small\nouppercase{\leftmark}}
\renewcommand{\subsectionmark}[1]{\markboth{#1}{}}
\cfoot{\thepage}
\def\labelitemii{$\circ$}

\title{Antiderivative \& Integration -- Nguyên Hàm \& Tích Phân}
\author{Nguyễn Quản Bá Hồng\footnote{Independent Researcher, Ben Tre City, Vietnam\\e-mail: \texttt{nguyenquanbahong@gmail.com}; website: \url{https://nqbh.github.io}.}}
\date{\today}

\begin{document}
\maketitle
\begin{abstract}
	\textsc{[en]} This text is a collection of problems, from easy to advanced, about antiderivative \& integration. This text is also a supplementary material for my lecture note on Elementary Mathematics grade 12, which is stored \& downloadable at the following link: \href{https://github.com/NQBH/hobby/blob/master/elementary_mathematics/grade_12/NQBH_elementary_mathematics_grade_12.pdf}{GitHub\texttt{/}NQBH\texttt{/}hobby\texttt{/}elementary mathematics\texttt{/}grade 12\texttt{/}lecture}\footnote{\textsc{url}: \url{https://github.com/NQBH/hobby/blob/master/elementary_mathematics/grade_12/NQBH_elementary_mathematics_grade_12.pdf}.}. The latest version of this text has been stored \& downloadable at the following link: \href{https://github.com/NQBH/hobby/blob/master/elementary_mathematics/grade_12/integration/NQBH_integration.pdf}{GitHub\texttt{/}NQBH\texttt{/}hobby\texttt{/}elementary mathematics\texttt{/}grade 12\texttt{/}integration}\footnote{\textsc{url}: \url{https://github.com/NQBH/hobby/blob/master/elementary_mathematics/grade_12/integration/NQBH_integration.pdf}.}.
	\vspace{2mm}
	
	\textsc{[vi]} Tài liệu này là 1 bộ sưu tập các bài tập chọn lọc từ cơ bản đến nâng cao về nguyên hàm \& tích phân. Tài liệu này là phần bài tập bổ sung cho tài liệu chính -- bài giảng \href{https://github.com/NQBH/hobby/blob/master/elementary_mathematics/grade_12/NQBH_elementary_mathematics_grade_12.pdf}{GitHub\texttt{/}NQBH\texttt{/}hobby\texttt{/}elementary mathematics\texttt{/}grade 12\texttt{/}lecture} của tác giả viết cho Toán Sơ Cấp lớp 12. Phiên bản mới nhất của tài liệu này được lưu trữ \& có thể tải xuống ở link sau: \href{https://github.com/NQBH/hobby/blob/master/elementary_mathematics/grade_12/integration/NQBH_integration.pdf}{GitHub\texttt{/}NQBH\texttt{/}hobby\texttt{/}elementary mathematics\texttt{/}grade 12\texttt{/}integration}.
\end{abstract}
\tableofcontents
\newpage

%------------------------------------------------------------------------------%

\section{Nguyên Hàm}

\begin{baitoan}[\cite{SGK_Toan_12_Giai_Tich_co_ban}, 1, p. 93]
	Tìm hàm số $F(x)$ sao cho $F'(x) = f(x)$ nếu: (a) $f(x) = 3x^2$, $\forall x\in\mathbb{R}$; (b) $f(x) = \frac{1}{\cos^2x}$, $\forall x\in\left(-\frac{\pi}{2};\frac{\pi}{2}\right)$.
\end{baitoan}

\begin{baitoan}[\cite{SGK_Toan_12_Giai_Tich_co_ban}, Ví dụ 6, p. 97]
	Tính: (a) $\int \left(2x^ + \frac{1}{\sqrt[3]{x^2}}\right)\,{\rm d}x$ trên khoảng $(0;+\infty)$; (b) $\int (3\cos x - 3^{x-1})\,{\rm d}x$ trên khoảng $(-\infty;+\infty)$.
\end{baitoan}

\begin{baitoan}[\cite{SGK_Toan_12_Giai_Tich_co_ban}, 6, p. 98]
	(a) Cho $\int (x - 1)^{10}\,{\rm d}x$. Đặt $u = x - 1$, viết $(x - 1)^{10}\,{\rm d}x$ theo $u$ \& ${\rm d}u$. (b) Cho $\int\frac{\ln x}{x}\,{\rm d}x$. Đặt $x = e^t$, viết $\frac{\ln x}{x}\,{\rm d}x$ theo $t$ \& ${\rm d}t$.
\end{baitoan}

\begin{baitoan}[\cite{SGK_Toan_12_Giai_Tich_co_ban}, Ví dụ 7, p. 98]
	Tính: (a) $\int \sin(3x - 1)\,{\rm d}x$. (b) $\int \sin(ax + b)\,{\rm d}x$. (c) $\int \cos(ax + b)\,{\rm d}x$
\end{baitoan}

\begin{baitoan}[\cite{SGK_Toan_12_Giai_Tich_co_ban}, Ví dụ 8, p. 99]
	Tính $\int \frac{x}{(x + 1)^5}\,{\rm d}x$.
\end{baitoan}

\begin{baitoan}[Mở rộng \cite{SGK_Toan_12_Giai_Tich_co_ban}, Ví dụ 8, p. 99]
	Tính $\int \frac{x}{(x + 1)^n}\,{\rm d}x$ với $n\in\mathbb{N}$.
\end{baitoan}

\begin{baitoan}[\cite{SGK_Toan_12_Giai_Tich_co_ban}, Ví dụ 8, p. 100]
	Tính: (a) $\int xe^x\,{\rm d}x$; (b) $\int x\cos x\,{\rm d}x$; (c) $\int \ln x\,{\rm d}x$.
\end{baitoan}

\begin{baitoan}[\cite{SGK_Toan_12_Giai_Tich_co_ban}, 8, p. 100]
	Cho $P(x)$ là đa thức của $x$. Tính $\int P(x)e^x\,{\rm d}x$, $\int P(x)\cos x\,{\rm d}x$, $\int P(x)\ln x\,{\rm d}x$.
\end{baitoan}

\begin{baitoan}[\cite{SGK_Toan_12_Giai_Tich_co_ban}, \textbf{1.}, p. 100]
	Trong các cặp hàm số dưới đây, hàm số nào là 1 nguyên hàm của hàm số còn lại? (a) $e^{-x}$ \& $-e^{-x}$; (b) $\sin2x$ \& $\sin^2x$; (c) $\left(1 - \frac{2}{x}\right)^2e^x$ \& $\left(1 - \frac{4}{x}\right)e^x$.
\end{baitoan}

\begin{baitoan}[\cite{SGK_Toan_12_Giai_Tich_co_ban}, \textbf{2.}, pp. 100--101]
	Tìm nguyên hàm của các hàm số sau: (a) $f(x) = \frac{x + \sqrt{x} + 1}{\sqrt[3]{x}}$; (b) $f(x) = \frac{2^x - 1}{e^x}$; (c) $f(x) = \frac{1}{\sin^2x\cos^2x}$; (d) $f(x) = \sin5x\cos3x$; (e) $f(x) = \tan^2x$; (g) $f(x) = e^{3-2x}$; (h) $f(x) = \frac{1}{(1 + x)(1 - 2x)}$.
\end{baitoan}

\begin{baitoan}[\cite{SGK_Toan_12_Giai_Tich_co_ban}, \textbf{3.}, p. 101]
	Sử dụng phương pháp đổi biến số, tính: (a) $\int (1 - x)^9\,{\rm d}x$ (đặt $u = 1 - x$); (b) $\int x(1 + x^2)^{\frac{3}{2}}\,{\rm d}x$ (đặt $u = 1 + x^2$); (c) $\int \cos^3x\sin x\,{\rm d}x$ (đặt $t = \cos x$); (d) $\int \frac{{\rm d}x}{e^x + e^{-x} + 2}$ (đặt $u = e^x + 1$).
\end{baitoan}

\begin{baitoan}[\cite{SGK_Toan_12_Giai_Tich_co_ban}, \textbf{4.}, p. 101]
	Sử dụng phương pháp tính nguyên hàm từng phần, tính: (a) $\int x\ln(1 + x)\,{\rm d}x$; (b) $\int (x^2 + 2x - 1)e^x\,{\rm d}x$; (c) $\int x\sin(2x + 1)\,{\rm d}x$; (d) $\int (1 - x)\cos x\,{\rm d}x$.
\end{baitoan}

\begin{baitoan}[\cite{SBT_Toan_12_Giai_Tich_co_ban}, Ví dụ 1, p. 144]
	Tính: $\int \frac{\sin^3x}{\cos^4x}\,{\rm d}x$.
\end{baitoan}

\begin{baitoan}[\cite{SBT_Toan_12_Giai_Tich_co_ban}, Ví dụ 2, p. 144]
	Tính: $\int \frac{\ln(\sin x)}{\cos^2x}\,{\rm d}x$.
\end{baitoan}

\begin{baitoan}[\cite{SBT_Toan_12_Giai_Tich_co_ban}, Ví dụ 3, p. 145]
	Tính: $\int \cos\sqrt{x}\,{\rm d}x$.
\end{baitoan}

\begin{baitoan}[\cite{SBT_Toan_12_Giai_Tich_co_ban}, \textbf{3.1}, p. 145]
	Kiểm tra xem hàm số nào là 1 nguyên hàm của hàm số còn lại trong mỗi cặp hàm số sau: (a) $f(x) = \ln(x + \sqrt{1 + x^2})$, $g(x) = \frac{1}{\sqrt{1 + x^2}}$; (b) $f(x) = e^{\sin x}\cos x$, $g(x) = e^{\sin x}$; (c) $f(x) = \sin^2\frac{1}{x}$, $g(x) = -\frac{1}{x^2}\sin\frac{2}{x}$; (d) $f(x) = \frac{x - 1}{\sqrt{x^2 - 2x + 2}}$, $g(x) = \sqrt{x^2 - 2x + 2}$; (e) $f(x) = x^2e^{\frac{1}{x}}$, $g(x) = (2x - 1)e^{\frac{1}{x}}$. 
\end{baitoan}

\begin{baitoan}[\cite{SBT_Toan_12_Giai_Tich_co_ban}, \textbf{3.2}, pp. 145--146]
	Chứng minh các hàm số $F(x)$ \& $G(x)$ sau đều là 1 nguyên hàm của cùng 1 hàm số: (a) $F(x) = \frac{x^2 + 6x + 1}{2x - 3}$, $G(x) = \frac{x^2 + 10}{2x - 3}$; (b) $F(x) = \frac{1}{\sin^2x}$, $G(x) = 10 + \cot^2x$; (c) $F(x) = 5 + 2\sin^2x$, $G(x) = 1 - \cos2x$.
\end{baitoan}

\begin{baitoan}[\cite{SBT_Toan_12_Giai_Tich_co_ban}, \textbf{3.3}, p. 146]
	Tìm nguyên hàm của các hàm số sau: (a) $f(x) = (x - 9)^4$; (b) $f(x) = \frac{1}{(2 - x)^2}$; (c) $f(x) = \frac{x}{\sqrt{1 - x^2}}$; (d) $f(x) = \frac{1}{\sqrt{2x + 1}}$; (e) $f(x) = \frac{1 - \cos2x}{\cos^2x}$; (f) $f(x) = \frac{2x + 1}{x^2 + x + 1}$.
\end{baitoan}

\begin{baitoan}[\cite{SBT_Toan_12_Giai_Tich_co_ban}, \textbf{3.4}, p. 146]
	Tính các nguyên hàm sau bằng phương pháp đổi biến số: (a) $\int x^2\sqrt[3]{1 + x^3}\,{\rm d}x$ với $x > -1$ (đặt $t = 1 + x^3$); (b) $\int xe^{-x^2}\,{\rm d}x$ (đặt $t = x^2$); (c) $\int \frac{x}{(1 + x^2)^2}\,{\rm d}x$ (đặt $t = 1 + x^2$); (d) $\int \frac{1}{(1 - x)\sqrt{x}}\,{\rm d}x$ (đặt $t = \sqrt{x}$); (e) $\int \sin\frac{1}{x}\cdot\frac{1}{x^2}\,{\rm d}x$ (đặt $t = \frac{1}{x}$); (f) $\int \frac{(\ln x)^2}{x}\,{\rm d}x$ (đặt $t = \ln x$); (g) $\int \frac{\sin x}{\sqrt[3]{\cos^2x}}\,{\rm d}x$ (đặt $t = \cos x$); (h) $\int \cos x\sin^3x\,{\rm d}x$ (đặt $t = \sin x$); (i) $\int \frac{1}{e^x - e^{-x}}\,{\rm d}x$ (đặt $t = e^x$); (j) $\int \frac{\cos x + \sin x}{\sqrt{\sin x - \cos x}}\,{\rm d}x$ (đặt $t = \sin x - \cos x$).
\end{baitoan}

\begin{baitoan}[\cite{SBT_Toan_12_Giai_Tich_co_ban}, \textbf{3.5}, p. 146]
	Áp dụng phương pháp tính nguyên hàm từng phần, tính: (a) $\int (1 - 2x)e^x\,{\rm d}x$; (b) $\int xe^{-x}\,{\rm d}x$; (c) $\int x\ln(1 - x)\,{\rm d}x$; (d) $\int x\sin^2x\,{\rm d}x$; (e) $\int \ln(1 + \sqrt{1 + x^2})$
\end{baitoan}


%------------------------------------------------------------------------------%

\section{Tích Phân}

%------------------------------------------------------------------------------%

\section{Ứng Dụng của Tích Phân Trong Hình Học}

%------------------------------------------------------------------------------%

\printbibliography[heading=bibintoc]
	
\end{document}