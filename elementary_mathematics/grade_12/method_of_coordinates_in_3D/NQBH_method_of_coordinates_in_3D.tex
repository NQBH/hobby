\documentclass{article}
\usepackage[backend=biber,natbib=true,style=authoryear,maxbibnames=10]{biblatex}
\addbibresource{/home/nqbh/reference/bib.bib}
\usepackage[utf8]{vietnam}
\usepackage{tocloft}
\renewcommand{\cftsecleader}{\cftdotfill{\cftdotsep}}
\usepackage[colorlinks=true,linkcolor=blue,urlcolor=red,citecolor=magenta]{hyperref}
\usepackage{amsmath,amssymb,amsthm,float,graphicx,mathtools}
\allowdisplaybreaks
\newtheorem{assumption}{Assumption}
\newtheorem{baitoan}{Bài toán}
\newtheorem{cauhoi}{Câu hỏi}
\newtheorem{conjecture}{Conjecture}
\newtheorem{corollary}{Corollary}
\newtheorem{dangtoan}{Dạng toán}
\newtheorem{definition}{Definition}
\newtheorem{dinhly}{Định lý}
\newtheorem{dinhnghia}{Định nghĩa}
\newtheorem{example}{Example}
\newtheorem{ghichu}{Ghi chú}
\newtheorem{hequa}{Hệ quả}
\newtheorem{hypothesis}{Hypothesis}
\newtheorem{lemma}{Lemma}
\newtheorem{luuy}{Lưu ý}
\newtheorem{nhanxet}{Nhận xét}
\newtheorem{notation}{Notation}
\newtheorem{note}{Note}
\newtheorem{principle}{Principle}
\newtheorem{problem}{Problem}
\newtheorem{proposition}{Proposition}
\newtheorem{question}{Question}
\newtheorem{remark}{Remark}
\newtheorem{theorem}{Theorem}
\newtheorem{vidu}{Ví dụ}
\usepackage[left=1cm,right=1cm,top=5mm,bottom=5mm,footskip=4mm]{geometry}
\def\labelitemii{$\circ$}
\DeclareRobustCommand{\divby}{%
	\mathrel{\vbox{\baselineskip.65ex\lineskiplimit0pt\hbox{.}\hbox{.}\hbox{.}}}%
}

\title{Method of Coordinates in 3D -- Phương Pháp Tọa Độ Trong Không Gian}
\author{Nguyễn Quản Bá Hồng\footnote{Independent Researcher, Ben Tre City, Vietnam\\e-mail: \texttt{nguyenquanbahong@gmail.com}; website: \url{https://nqbh.github.io}.}}
\date{\today}

\begin{document}
\maketitle
\begin{abstract}
	\textsc{[en]} This text is a collection of problems, from easy to advanced, about \textit{method of coordinates in 3D}. This text is also a supplementary material for my lecture note on Elementary Mathematics grade 12, which is stored \& downloadable at the following link: \href{https://github.com/NQBH/hobby/blob/master/elementary_mathematics/grade_12/NQBH_elementary_mathematics_grade_12.pdf}{GitHub\texttt{/}NQBH\texttt{/}hobby\texttt{/}elementary mathematics\texttt{/}grade 12\texttt{/}lecture}\footnote{\textsc{url}: \url{https://github.com/NQBH/hobby/blob/master/elementary_mathematics/grade_12/NQBH_elementary_mathematics_grade_12.pdf}.}. The latest version of this text has been stored \& downloadable at the following link: \href{https://github.com/NQBH/hobby/blob/master/elementary_mathematics/grade_12/method_of_coordinates_in_3D/NQBH_method_of_coordinates_in_3D.pdf}{GitHub\texttt{/}NQBH\texttt{/}hobby\texttt{/}elementary mathematics\texttt{/}grade 12\texttt{/}method of coordinates in 3D}\footnote{\textsc{url}: \url{https://github.com/NQBH/hobby/blob/master/elementary_mathematics/grade_12/method_of_coordinates_in_3D/NQBH_method_of_coordinates_in_3D.pdf}.}.
	\vspace{2mm}
	
	\textsc{[vi]} Tài liệu này là 1 bộ sưu tập các bài tập chọn lọc từ cơ bản đến nâng cao về \textit{phương pháp tọa độ trong mặt phẳng}. Tài liệu này là phần bài tập bổ sung cho tài liệu chính -- bài giảng \href{https://github.com/NQBH/hobby/blob/master/elementary_mathematics/grade_12/NQBH_elementary_mathematics_grade_12.pdf}{GitHub\texttt{/}NQBH\texttt{/}hobby\texttt{/}elementary mathematics\texttt{/}grade 12\texttt{/}lecture} của tác giả viết cho Mathematics Sơ Cấp lớp 12. Phiên bản mới nhất của tài liệu này được lưu trữ \& có thể tải xuống ở link sau: \href{https://github.com/NQBH/hobby/blob/master/elementary_mathematics/grade_12/method_of_coordinates_in_3D/NQBH_method_of_coordinates_in_3D.pdf}{GitHub\texttt{/}NQBH\texttt{/}hobby\texttt{/}elementary mathematics\texttt{/}grade 12\texttt{/}method of coordinates in 3D}.
	
	\textsf{\textbf{Nội dung.} Hệ tọa độ trong không gian, phương trình mặt phẳng, phương trình đường thẳng.}
\end{abstract}
\tableofcontents
\newpage

%------------------------------------------------------------------------------%

\section{Hệ Tọa Độ Trong Không Gian}

\begin{baitoan}[\cite{SGK_Toan_12_hinh_hoc_co_ban}, 1., p. 68]
	Cho 3 vector $\vec{a} = (2,-5,3),\vec{b} = (0,2,-1),\vec{c} = (1,7,2)$. (a) Tính tọa độ của vector $\vec{d} = 4\vec{a} - \frac{1}{3}\vec{b} + 3\vec{c}$. (b) Tính tọa độ của vector $\vec{e} = \vec{a} - 4\vec{b} - 2\vec{c}$.
\end{baitoan}

\begin{baitoan}[\cite{SGK_Toan_12_hinh_hoc_co_ban}, 2., p. 68]
	Cho 3 điểm $A = (1,-1,1),B = (0,1,2),C = (1,0,1)$. Tìm tọa độ trọng tâm $G$ của $\Delta ABC$.
\end{baitoan}

\begin{baitoan}[\cite{SGK_Toan_12_hinh_hoc_co_ban}, 3., p. 68]
	Cho hình hộp $ABCD.A'B'C'D'$ biết $A = (1,0,1),B = (2,1,2),D = (1,-1,1),C' = (4,5,-5)$. Tính tọa độ các đỉnh còn lại của hình hộp.
\end{baitoan}

\begin{baitoan}[\cite{SGK_Toan_12_hinh_hoc_co_ban}, 4., p. 68]
	Tính: (a) $\vec{a}\cdot\vec{b}$ với $\vec{a} = (3,0,-6),\vec{b} = (2,-4,0)$. (b) $\vec{c}\cdot\vec{d}$ với $\vec{c} = (1,-5,2),\vec{d} = (4,3,-5)$.
\end{baitoan}

\begin{baitoan}[\cite{SGK_Toan_12_hinh_hoc_co_ban}, 5., p. 68]
	Tìm tâm \& bán kính của các mặt cầu có phương trình sau: (a) $x^2 + y^2 + z^2 - 8x - 2y + 1 = 0$; (b) $3x^2 + 3y^2 + 3z^2 - 6x + 8y + 15z - 3 = 0$.
\end{baitoan}

\begin{baitoan}[\cite{SGK_Toan_12_hinh_hoc_co_ban}, 6., p. 68]
	Lập phương trình mặt cầu trong 2 mặt cầu trong 2 trường hợp sau: (a) Có đường kính $AB$ với $A = (4,-3,7),B = (2,1,3)$. (b) Đi qua điểm $A = (5,-2,1)$ \& có tâm $C = (3,-3,1)$.
\end{baitoan}

%------------------------------------------------------------------------------%

\section{Phương Trình Mặt Phẳng}

\begin{baitoan}[\cite{SGK_Toan_12_hinh_hoc_co_ban}, 1., p. 80]
	Viết phương trình của mặt phẳng: (a) Đi qua điểm $M(1,-2,4)$ \& nhận $\vec{n} = (2,3,5)$ làm vector pháp tuyến; (b) Đi qua điểm $A(0,-1,2)$ \& song song với giá của mỗi vector $\vec{u} = (3,2,1)$ \& $\vec{v} = (-3,0,1)$; (c) Đi qua 3 điểm $A(-3,0,0)$, $B(0,-2,0)$, \& $C(0,0,-1)$.
\end{baitoan}

\begin{baitoan}[\cite{SGK_Toan_12_hinh_hoc_co_ban}, 2., p. 80]
	Viết phương trình mặt phẳng trung trực của đoạn thẳng $AB$ với $A(2,3,7)$, $B(4,1,3)$.
\end{baitoan}

\begin{baitoan}[\cite{SGK_Toan_12_hinh_hoc_co_ban}, 3., p. 80]
	(a) Lập phương trình của các mặt phẳng tọa độ $(Oxy),(Oyz),(Oxz)$. (b) Lập phương trình của các mặt phẳng đi qua điểm $M(2,6,-3)$ \& lần lượt song song với các mặt phẳng tọa độ.
\end{baitoan}

\begin{baitoan}[\cite{SGK_Toan_12_hinh_hoc_co_ban}, 4., p. 80]
	Lập phương trình của mặt phẳng: (a) Chứa trục $Ox$ \& điểm $P(4,-1,2)$; (b) Chứa trục $Oy$ \& điểm $Q(1,4,-3)$; (c) Chứa trục $Oz$ \& điểm $R(3,-4,7)$.
\end{baitoan}

\begin{baitoan}[\cite{SGK_Toan_12_hinh_hoc_co_ban}, 5., p. 80]
	Cho tứ diện có các đỉnh là $A(5,1,3),B(1,6,2),C(5,0,4),D(4,0,6)$. (a) Viết phương trình của các mặt phẳng $(ACD),(BCD)$. (b) Viết phương trình mặt phẳng $(\alpha)$ đi qua cạnh $AB$ \& song song với cạnh $CD$.
\end{baitoan}

\begin{baitoan}[\cite{SGK_Toan_12_hinh_hoc_co_ban}, 6., p. 80]
	Viết phương trình mặt phẳng $(\alpha)$ đi qua điểm $M(2,-1,2)$ \& song song với mặt phẳng $(\beta)$: $2x - y + 3z + 4 = 0$.
\end{baitoan}

\begin{baitoan}[\cite{SGK_Toan_12_hinh_hoc_co_ban}, 7., p. 80]
	Lập phương trình mặt phẳng $(\alpha)$ đi qua 2 điểm $A(1,0,1),B(5,2,3)$ \& vuông góc với mặt phẳng $(\beta)$: $2x - y + z - 7 = 0$.
\end{baitoan}

\begin{baitoan}[\cite{SGK_Toan_12_hinh_hoc_co_ban}, 8., p. 81]
	Xác định các giá trị của $m,n$ để mỗi cặp mặt phẳng sau đây là 1 cặp mặt phẳng song song với nhau: (a) $2x + my + 3z - 5 = 0$ \& $nx - 8y - 6z + 2 = 0$. (b) $3x - 5y + mz - 3 = 0$ \& $2x + ny - 3z + 1 = 0$.
\end{baitoan}

\begin{baitoan}[\cite{SGK_Toan_12_hinh_hoc_co_ban}, 9., p. 81]
	Tính khoảng cách từ điểm $A(2,4,-3)$ lần lượt đến các mặt phẳng sau: (a) $2x - y + 2z - 9 = 0$; (b) $12x - 5z + 5 = 0$; (c) $x = 0$.
\end{baitoan}

\begin{baitoan}[\cite{SGK_Toan_12_hinh_hoc_co_ban}, 10., p. 81]
	Giải bài toán sau đây bằng phương pháp tọa độ: Cho hình lập phương $ABCD.A'B'C'D'$ cạnh bằng $1$. (a) Chứng minh 2 mặt phẳng $(AB'D'),(BC'D)$ song song với nhau. (b) Tính khoảng cách giữa 2 mặt phẳng nói trên.
\end{baitoan}

%------------------------------------------------------------------------------%

\section{Phương Trình Đường Thẳng Trong Không Gian}

%------------------------------------------------------------------------------%

\printbibliography[heading=bibintoc]
	
\end{document}