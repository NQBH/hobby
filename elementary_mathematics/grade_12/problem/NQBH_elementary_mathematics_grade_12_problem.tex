\documentclass{article}
\usepackage[backend=biber,natbib=true,style=authoryear]{biblatex}
\addbibresource{/home/hong/1_NQBH/reference/bib.bib}
\usepackage[utf8]{vietnam}
\usepackage{tocloft}
\renewcommand{\cftsecleader}{\cftdotfill{\cftdotsep}}
\usepackage[colorlinks=true,linkcolor=blue,urlcolor=red,citecolor=magenta]{hyperref}
\usepackage{amsmath,amssymb,amsthm,mathtools,float,graphicx,algpseudocode,algorithm,tcolorbox}
\usepackage[inline]{enumitem}
\allowdisplaybreaks
\numberwithin{equation}{section}
\newtheorem{assumption}{Assumption}[section]
\newtheorem{conjecture}{Conjecture}[section]
\newtheorem{corollary}{Corollary}[section]
\newtheorem{hequa}{Hệ quả}[section]
\newtheorem{definition}{Definition}[section]
\newtheorem{dinhnghia}{Định nghĩa}[section]
\newtheorem{example}{Example}[section]
\newtheorem{vidu}{Ví dụ}[section]
\newtheorem{lemma}{Lemma}[section]
\newtheorem{notation}{Notation}[section]
\newtheorem{principle}{Principle}[section]
\newtheorem{problem}{Problem}[section]
\newtheorem{baitoan}{Bài toán}[section]
\newtheorem{proposition}{Proposition}[section]
\newtheorem{question}{Question}[section]
\newtheorem{cauhoi}{Câu hỏi}[section]
\newtheorem{remark}{Remark}[section]
\newtheorem{luuy}{Lưu ý}[section]
\newtheorem{theorem}{Theorem}[section]
\newtheorem{dinhly}{Định lý}[section]
\usepackage[left=0.5in,right=0.5in,top=1.5cm,bottom=1.5cm]{geometry}
\usepackage{fancyhdr}
\pagestyle{fancy}
\fancyhf{}
\lhead{\small Sect.~\thesection}
\rhead{\small \nouppercase{\leftmark}}
\renewcommand{\sectionmark}[1]{\markboth{#1}{}}
\cfoot{\thepage}
\def\labelitemii{$\circ$}

\title{Some Topics in Elementary Mathematics\texttt{/}Grade 12}
\author{Nguyễn Quản Bá Hồng\footnote{Independent Researcher, Ben Tre City, Vietnam\\e-mail: \texttt{nguyenquanbahong@gmail.com}; website: \url{https://nqbh.github.io}.}}
\date{\today}

\begin{document}
\maketitle
\begin{abstract}
	1 bộ sưu tập các bài toán chọn lọc từ cơ bản đến nâng cao cho Toán sơ cấp lớp 12. Tài liệu này là phần bài tập bổ sung cho tài liệu chính \href{https://github.com/NQBH/hobby/blob/master/elementary_mathematics/grade_12/NQBH_elementary_mathematics_grade_12.pdf}{GitHub\texttt{/}NQBH\texttt{/}hobby\texttt{/}elementary mathematics\texttt{/}grade 12\texttt{/}lecture}\footnote{\textsc{url}: \url{https://github.com/NQBH/hobby/blob/master/elementary_mathematics/grade_12/NQBH_elementary_mathematics_grade_12.pdf}.} của tác giả viết cho Toán lớp 12. Phiên bản mới nhất của tài liệu này được lưu trữ ở link sau: \href{https://github.com/NQBH/hobby/blob/master/elementary_mathematics/grade_12/problem/NQBH_elementary_mathematics_grade_12_problem.pdf}{GitHub\texttt{/}NQBH\texttt{/}hobby\texttt{/}elementary mathematics\texttt{/}grade 12\texttt{/}problem}\footnote{\textsc{url}: \url{https://github.com/NQBH/hobby/blob/master/elementary_mathematics/grade_12/problem/NQBH_elementary_mathematics_grade_12_problem.pdf}.}.
\end{abstract}
\tableofcontents
\newpage

%------------------------------------------------------------------------------%

\section{Ứng Dụng Đạo Hàm Để Khảo Sát \& Vẽ Đồ Thị của Hàm Số}

\subsection{Tính Đơn Điệu của Hàm Số}

%------------------------------------------------------------------------------%

\subsection{Cực Trị của Hàm Số}

%------------------------------------------------------------------------------%

\subsection{Giá Trị Lớn Nhất \& Giá Trị Nhỏ Nhất của Hàm Số}

%------------------------------------------------------------------------------%

\subsection{Đồ Thị của Hàm Số \& Phép Tịnh Tiến Hệ Tọa Độ}

%------------------------------------------------------------------------------%

\subsection{Đường Tiệm Cận của Đồ Thị Hàm Số}

%------------------------------------------------------------------------------%

\subsection{Khảo Sát Sự Biến Thiên \& Vẽ Đồ Thị của 1 Số Hàm Đa Thức}

%------------------------------------------------------------------------------%

\subsection{Khảo Sát Sự Biến Thiên \& Vẽ Đồ Thị của 1 Số Hàm Phân Thức Hữu Tỷ}

%------------------------------------------------------------------------------%

\subsection{1 Số Bài Toán Thường Gặp về Đồ Thị}

%------------------------------------------------------------------------------%

\section{Hàm Số Lũy Thừa, Hàm Số Mũ, \& Hàm Số Logarith}

\subsection{Lũy Thừa với Số Mũ Hữu Tỷ}

%------------------------------------------------------------------------------%

\subsection{Lũy Thừa với Số Mũ Thực}

%------------------------------------------------------------------------------%

\subsection{Logarithm}

%------------------------------------------------------------------------------%

\subsection{Số $\rm e$ \& Logarith Tự Nhiên}

%------------------------------------------------------------------------------%

\subsection{Hàm Số Mũ \& Hàm Số Logarithm}

%------------------------------------------------------------------------------%

\subsection{Hàm Số Lũy Thừa}

%------------------------------------------------------------------------------%

\subsection{Phương Trình Mũ \& Logarithm}

%------------------------------------------------------------------------------%

\subsection{Hệ Phương Trình Mũ \& Logarithm}

%------------------------------------------------------------------------------%

\subsection{Bất Phương Trình Mũ \& Logarithm}

%------------------------------------------------------------------------------%

\section{Nguyên Hàm, Tích Phân, \& Ứng Dụng}

\subsection{Nguyên Hàm}

%------------------------------------------------------------------------------%

\subsection{1 Số Phương Pháp Tìm Nguyên Hàm}

%------------------------------------------------------------------------------%

\subsection{Tích Phân}

%------------------------------------------------------------------------------%

\subsection{1 Số Phương Pháp Tính Tích Phân}

%------------------------------------------------------------------------------%

\subsection{Ứng Dụng Tích Phân Để Tính Diện Tích Hình Phẳng}

%------------------------------------------------------------------------------%

\subsection{Ứng Dụng Tích Phân Để Tính Thể Tích Vật Thể}

%------------------------------------------------------------------------------%

\section{Số Phức}

\subsection{Số Phức}

%------------------------------------------------------------------------------%

\subsection{Căn Bậc 2 của Số Phức \& Phương Trình Bậc 2}

%------------------------------------------------------------------------------%

\subsection{Dạng Lượng Giác của Số Phức \& Ứng Dụng}

%------------------------------------------------------------------------------%

\section{Khối Đa Diện \& Thể Tích của Chúng}

\subsection{Khái Niệm về Khối Đa Diện}

%------------------------------------------------------------------------------%

\subsection{Phép Đối Xứng qua Mặt Phẳng \& Sự Bằng Nhau của Các Khối Đa Diện}

%------------------------------------------------------------------------------%

\subsection{Phép Vị Tự \& Sự Đồng Dạng của Các Khối Đa Diện. Các Khối Đa Diện Đều}

%------------------------------------------------------------------------------%

\subsection{Thể Tích của Khối Đa Diện}

%------------------------------------------------------------------------------%

\section{Mặt Cầu, Mặt Trụ, Mặt Nón}

\subsection{Mặt Cầu, Khối Cầu}

%------------------------------------------------------------------------------%

\subsection{Khái Niệm về Mặt Tròn Xoay}

%------------------------------------------------------------------------------%

\subsection{Mặt Trụ, Hình Trụ, \& Khối Trụ}

%------------------------------------------------------------------------------%

\subsection{Mặt Nón, Hình Nón, \& Khối Nón}

%------------------------------------------------------------------------------%

\section{Phương Pháp Tọa Độ Trong Không Gian}

\subsection{Hệ Tọa Độ Trong Không Gian}

%------------------------------------------------------------------------------%

\subsection{Phương Trình Mặt Phẳng}

%------------------------------------------------------------------------------%

\subsection{Phương Trình Đường Thẳng}

%------------------------------------------------------------------------------%

\printbibliography[heading=bibintoc]
	
\end{document}