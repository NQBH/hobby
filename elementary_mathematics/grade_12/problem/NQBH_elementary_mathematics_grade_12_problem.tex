\documentclass[12pt]{article}
\usepackage[backend=biber,natbib=true,style=authoryear]{biblatex}
\addbibresource{/home/hong/1_NQBH/reference/bib.bib}
\usepackage[utf8]{vietnam}
\usepackage{tocloft}
\renewcommand{\cftsecleader}{\cftdotfill{\cftdotsep}}
\usepackage[colorlinks=true,linkcolor=blue,urlcolor=red,citecolor=magenta]{hyperref}
\usepackage{amsmath,amssymb,amsthm,mathtools,float,graphicx,algpseudocode,algorithm,tcolorbox}
\usepackage[inline]{enumitem}
\allowdisplaybreaks
\numberwithin{equation}{section}
\newtheorem{assumption}{Assumption}[section]
\newtheorem{conjecture}{Conjecture}[section]
\newtheorem{corollary}{Corollary}[section]
\newtheorem{dangtoan}{Dạng toán}[section]
\newtheorem{hequa}{Hệ quả}[section]
\newtheorem{definition}{Definition}[section]
\newtheorem{dinhnghia}{Định nghĩa}[section]
\newtheorem{example}{Example}[section]
\newtheorem{vidu}{Ví dụ}[section]
\newtheorem{lemma}{Lemma}[section]
\newtheorem{notation}{Notation}[section]
\newtheorem{principle}{Principle}[section]
\newtheorem{problem}{Problem}[section]
\newtheorem{baitoan}{Bài toán}[section]
\newtheorem{proposition}{Proposition}[section]
\newtheorem{question}{Question}[section]
\newtheorem{cauhoi}{Câu hỏi}[section]
\newtheorem{remark}{Remark}[section]
\newtheorem{luuy}{Lưu ý}[section]
\newtheorem{theorem}{Theorem}[section]
\newtheorem{dinhly}{Định lý}[section]
\usepackage[left=0.5in,right=0.5in,top=1.5cm,bottom=1.5cm]{geometry}
\usepackage{fancyhdr}
\pagestyle{fancy}
\fancyhf{}
\lhead{\small Sect.~\thesection}
\rhead{\small \nouppercase{\leftmark}}
\renewcommand{\sectionmark}[1]{\markboth{#1}{}}
\cfoot{\thepage}
\def\labelitemii{$\circ$}

\title{Problems in Elementary Mathematics\texttt{/}Grade 12}
\author{Nguyễn Quản Bá Hồng\footnote{Independent Researcher, Ben Tre City, Vietnam\\e-mail: \texttt{nguyenquanbahong@gmail.com}; website: \url{https://nqbh.github.io}.}}
\date{\today}

\begin{document}
\maketitle
\begin{abstract}
	1 bộ sưu tập các bài toán chọn lọc từ cơ bản đến nâng cao cho Toán sơ cấp lớp 12. Tài liệu này là phần bài tập bổ sung cho tài liệu chính \href{https://github.com/NQBH/hobby/blob/master/elementary_mathematics/grade_12/NQBH_elementary_mathematics_grade_12.pdf}{GitHub\texttt{/}NQBH\texttt{/}hobby\texttt{/}elementary mathematics\texttt{/}grade 12\texttt{/}lecture}\footnote{\textsc{url}: \url{https://github.com/NQBH/hobby/blob/master/elementary_mathematics/grade_12/NQBH_elementary_mathematics_grade_12.pdf}.} của tác giả viết cho Toán lớp 12. Phiên bản mới nhất của tài liệu này được lưu trữ ở link sau: \href{https://github.com/NQBH/hobby/blob/master/elementary_mathematics/grade_12/problem/NQBH_elementary_mathematics_grade_12_problem.pdf}{GitHub\texttt{/}NQBH\texttt{/}hobby\texttt{/}elementary mathematics\texttt{/}grade 12\texttt{/}problem}\footnote{\textsc{url}: \url{https://github.com/NQBH/hobby/blob/master/elementary_mathematics/grade_12/problem/NQBH_elementary_mathematics_grade_12_problem.pdf}.}.
\end{abstract}
\tableofcontents
\newpage

%------------------------------------------------------------------------------%

\section{Ứng Dụng Đạo Hàm Để Khảo Sát \& Vẽ Đồ Thị của Hàm Số}

\subsection{Tính Đơn Điệu của Hàm Số}
\begin{dangtoan}
	Cho hàm số $y = f(x)$ có đồ thị như hình vẽ. Tìm khoảng đồng biến \& nghịch biến của hàm số $y = f(x)$.
\end{dangtoan}

\begin{proof}[Cách giải]
	Nhìn vào đồ thị, từ trái sang phải, khoảng nào hàm số $y = f(x)$ đi lên là khoảng đồng biến của hàm số $y = f(x)$, khoảng nào hàm số $y = f(x)$ đi xuống là khoảng nghịch biến của hàm số $y = f(x)$.
\end{proof}

%------------------------------------------------------------------------------%

\subsection{Cực Trị của Hàm Số}

%------------------------------------------------------------------------------%

\subsection{Giá Trị Lớn Nhất \& Giá Trị Nhỏ Nhất của Hàm Số}

%------------------------------------------------------------------------------%

\subsection{Đồ Thị của Hàm Số \& Phép Tịnh Tiến Hệ Tọa Độ}

\begin{baitoan}[\cite{TL_chuyen_Toan_Giai_Tich_12}, Ví dụ 1, p. 7]
	Cho đường cong $(C)$ có phương trình $y = \frac{1}{2}(x - 2)^3 - 1$ \& điểm $I(2;-1)$.
	\begin{enumerate*}
		\item[(a)] Viết công thức chuyển hệ tọa độ trong phép tịnh tiến theo vector $\overrightarrow{OI}$ \& viết phương trình của đường cong $(C)$ đối với hệ tọa độ $IXY$.
		\item[(b)] Từ đó suy ra rằng $I$ là tâm đối xứng của đường cong $(C)$.
	\end{enumerate*}
\end{baitoan}

\begin{baitoan}[\cite{TL_chuyen_Toan_Giai_Tich_12}, \textbf{1.}, p. 9]
	Xác định đỉnh $I$ của mỗi parabol $(P)$ sau đây. Viết công thức chuyển tọa độ trong phép tịnh tiến theo vector $\overrightarrow{OI}$ \& viết phương trình của parabol $(P)$ đối với hệ tọa độ $IXY$.
	\begin{enumerate*}
		\item[(a)] $y = 2x^2 - 4x + 1$;
		\item[(b)] $y = \frac{1}{2}x^2 - x - 2$;
		\item[(c)] $y = x - x^2$;
		\item[(d)] $y = x^2 + 3x + 2$.
	\end{enumerate*}
\end{baitoan}

\begin{baitoan}[\cite{TL_chuyen_Toan_Giai_Tich_12}, \textbf{2.}, p. 9]
	Cho hàm số $f(x) = x^3 - 3x^2 + 1$.
	\begin{enumerate*}
		\item[(a)] Xác định điểm $I$ thuộc đồ thị $(C)$ của hàm số, biết rằng hoành độ của điểm $I$ là nghiệm của phương trình $f''(x) = 0$.
		\item[(b)] Viết công thức chuyển hệ tọa độ trong phép tịnh tiến theo vector $\overrightarrow{OI}$ \& viết phương trình của đường cong $(C)$ đôi với hệ tọa độ $IXY$. Từ đó suy ra rằng $I$ là tâm đối xứng của đường cong $(C)$.
		\item[(c)] Viết phương trình tiếp tuyến của đường cong $(C)$ tại điểm $I$ đối với hệ tọa độ $Oxy$. Chứng minh rằng trên khoảng $(-\infty;1)$, đường cong $(C)$ nằm phía dưới tiếp tuyến tại $I$ của $(C)$ \& trên khoảng $(1;+\infty)$, đường cong $(C)$ nằm phía trên tiếp tuyến đó.
	\end{enumerate*}
\end{baitoan}

\begin{baitoan}[\cite{TL_chuyen_Toan_Giai_Tich_12}, \textbf{3.}, p. 9]
	Xác định tâm đối xứng của đồ thị mỗi hàm số sau:
	\begin{enumerate*}
		\item[(a)] $y = \frac{2}{x - 1} + 1$;
		\item[(b)] $y = \frac{3x - 1}{x + 1}$;
		\item[(c)] $y = (x - 2)^3 - 1$;
		\item[(d)] $y = x^3 - 3x + 2$.
	\end{enumerate*}
\end{baitoan}

\begin{baitoan}[\cite{TL_chuyen_Toan_Giai_Tich_12}, \textbf{4.}, p. 10]
	Cho đường cong $(C)$ có phương trình $y = ax + b + \frac{c}{x - x_0}$, trong đó $a\ne 0$, $c\ne 0$, \& điểm $I$ có tọa độ $(x_0;y_0)$ thỏa mãn $y_0 = ax_0 + b$. Viết công thức chuyển hệ tọa độ trong phép tịnh tiến vector $\overrightarrow{OI}$ \& phương trình của $(C)$ đối với hệ tọa độ $IXY$. Từ đó suy ra rằng $I$ là tâm đối xứng của đường cong $(C)$.
\end{baitoan}

\begin{baitoan}[\cite{TL_chuyen_Toan_Giai_Tich_12}, \textbf{5.}, p. 10]
	\begin{enumerate*}
		\item[(a)] Vẽ đồ thị $(G)$ của hàm số $y = |x|$.
		\item[(b)] Từ đồ thị $(G)$, suy ra đồ thị của hàm sô $y = |x - 3|$.
		\item[(c)] Từ đồ thị $(G)$, suy ra đồ thị của hàm số $y = 2|x|$.
	\end{enumerate*}
\end{baitoan}

\begin{baitoan}[\cite{TL_chuyen_Toan_Giai_Tich_12}, \textbf{6.}, p. 10]
	Từ đồ thị $(G)$ của hàm số $y = x^2 - 2x$, suy ra đồ thị các hàm số sau:
	\begin{enumerate*}
		\item[(a)] $y = |x^2 - 2x|$;
		\item[(b)] $y = 2x^2 - 4x$;
		\item[(c)] $y = |x|(x - 2)$.
	\end{enumerate*}
\end{baitoan}

\begin{baitoan}[\cite{TL_chuyen_Toan_Giai_Tich_12}, \textbf{7.}, p. 10]
	Từ đồ thị hàm số $y = \sin x$, suy ra đồ thị các hàm số $y = \cos x$, $y = \sin 2x$ bằng các phép biến đổi đồ thị thích hợp.
\end{baitoan}

%------------------------------------------------------------------------------%

\subsection{Đường Tiệm Cận của Đồ Thị Hàm Số}

\begin{baitoan}[\cite{TL_chuyen_Toan_Giai_Tich_12}, Ví dụ 1, p. 12]
	Tìm tiệm cận ngang \& tiệm cận đứng của đồ thị hàm số $y = \frac{2x + 1}{x + 2}$.
\end{baitoan}

\begin{baitoan}[\cite{TL_chuyen_Toan_Giai_Tich_12}, Ví dụ 2, p. 12]
	Tìm tiệm cận ngang \& tiệm cận đứng của đồ thị hàm số $y = \frac{\sqrt{x^2 + 1}}{x}$.
\end{baitoan}

\begin{baitoan}[\cite{TL_chuyen_Toan_Giai_Tich_12}, H1, p. 13]
	Tìm tiệm cận ngang \& tiệm cận đứng của đồ thị hàm số $y = \frac{1 - 4x^2}{1 - x^2}$.
\end{baitoan}

\begin{baitoan}[\cite{TL_chuyen_Toan_Giai_Tich_12}, Ví dụ 3, p. 14]
	Tìm tiệm cận xiên của đồ thị hàm số $f(x) = x + \frac{x}{x^2 + 1}$.
\end{baitoan}

\begin{baitoan}[\cite{TL_chuyen_Toan_Giai_Tich_12}, H1, p. 14]
	Chứng minh rằng đường thẳng $y = x + 1$ là đường tiệm cận xiên của đồ thị hàm số $y = \frac{x^2}{x - 1}$.
\end{baitoan}

\begin{baitoan}[\cite{TL_chuyen_Toan_Giai_Tich_12}, Ví dụ 4, p. 15]
	Tìm tiệm cận xiên của đồ thị hàm số $f(x) = \frac{x^3}{x^2 - 1}$.
\end{baitoan}

\begin{baitoan}[\cite{TL_chuyen_Toan_Giai_Tich_12}, \textbf{8.}, p. 15]
	Tìm các đường tiệm cận của đồ thị mỗi hàm số sau:
	\begin{enumerate*}
		\item[(a)] $y = \frac{2x - 3}{3x - 2}$;
		\item[(b)] $y = x + 2 - \frac{1}{x - 3}$;
		\item[(c)] $y = \frac{x + 2}{x^2 - 1}$;
		\item[(d)] $y = \frac{x^2 - 3x + 5}{2x + 1}$;
		\item[(e)] $y = \frac{x^3 + 2}{x^2 - 1}$;
		\item[(f)] $y = \frac{x^2 + x + 1}{x^2 - x + 1}$.
	\end{enumerate*}
\end{baitoan}

\begin{baitoan}[\cite{TL_chuyen_Toan_Giai_Tich_12}, \textbf{9.}, p. 15]
	Tìm các đường tiệm cận của đồ thị mỗi hàm số sau:
	\begin{enumerate*}
		\item[(a)] $y = \sqrt{x^2 - 1}$;
		\item[(b)] $y = 2x + \sqrt{x^2 - 1}$;
		\item[(c)] $y = x + \sqrt{x^2 + 1}$.
	\end{enumerate*}
\end{baitoan}

\begin{baitoan}[\cite{TL_chuyen_Toan_Giai_Tich_12}, \textbf{10.}, p. 15]
	Tìm các đường tiệm cận của đồ thị mỗi hàm số sau:
	\begin{enumerate*}
		\item[(a)] $y = \sqrt{x^2 + x + 1}$;
		\item[(b)] $y = \frac{x^2 + 1}{x^2 - 4}$;
		\item[(c)] $y = \frac{\lfloor x\rfloor}{x}$.
	\end{enumerate*}
\end{baitoan}

\begin{baitoan}[\cite{TL_chuyen_Toan_Giai_Tich_12}, \textbf{11.}, p. 16]
	\begin{enumerate*}
		\item[(a)] Tìm tiệm cận đứng \& tiệm cận xiên của đồ thị $(C)$ của hàm số $y = \frac{x^2 + x}{x - 2}$.
		\item[(b)] Xác định giao điểm $I$ của 2 tiệm cận trên \& viết công thức chuyển hệ tọa độ trong phép tịnh tiến theo vector $\overrightarrow{OI}$.
		\item[(c)] Viết phương trình đường cong $(C)$ đối với hệ tọa độ $IXY$.
	\end{enumerate*}
	Từ đó suy ra rằng $I$ là tâm đối xứng của đường cong $(C)$.
\end{baitoan}

\begin{baitoan}[\cite{TL_chuyen_Toan_Giai_Tich_12}, \textbf{12.}, p. 16]
	Cho $(C_m)$ là đường cong có phương trình $y = \frac{2x^2 + (m + 1)x - 3}{x + m}$.
	\begin{enumerate*}
		\item[(a)] Tìm $m$ để tiệm cận xiên của $(C_m)$ đi qua $A(1;1)$.
		\item[(b)] Tìm $m$ để giao điểm của 2 tiệm cận nằm trên đường cong $(P)$: $y = x^2 + 3$.
	\end{enumerate*}
\end{baitoan}

\begin{baitoan}[\cite{TL_chuyen_Toan_Giai_Tich_12}, \textbf{13.}, p. 16]
	Cho $(C)$: $y = \frac{x^2 + x}{x - 2}$. Chứng minh rằng tích các khoảng cách từ điểm $M$ bất kỳ trên $(C)$ đến 2 tiệm cận của $(C)$ bằng 1 hằng số.
\end{baitoan}

\begin{baitoan}[\cite{TL_chuyen_Toan_Giai_Tich_12}, \textbf{14.}, p. 16]
	Tìm những điểm trên đường cong $(C)$ có phương trình $y = \frac{x^2 + x + 1}{x + 2}$ sao cho tổng khoảng cách từ điểm đó đến 2 tiệm cận là nhỏ nhất.
\end{baitoan}


%------------------------------------------------------------------------------%

\subsection{Khảo Sát Sự Biến Thiên \& Vẽ Đồ Thị của 1 Số Hàm Đa Thức}

\begin{dangtoan}
		Khảo sát sự biến thiên \& vẽ đồ thị hàm số $y = f(x)$ với hàm số $f$ cho trước có thể chứa tham số.
\end{dangtoan}

\begin{baitoan}[\cite{TL_chuyen_Toan_Giai_Tich_12}, Ví dụ 1, p. 17]
	Khảo sát sự biến thiên \& vẽ đồ thị hàm số $y = \sqrt{x^2 + x + 1}$.
\end{baitoan}

\subsubsection{Khảo sát sự biến thiên \& vẽ đồ thị của hàm số bậc 1 $y = ax + b$, $a\ne 0$}

\begin{baitoan}
	Khảo sát sự biến thiên \& vẽ đồ thị hàm số bậc nhất $y = f(x) = ax + b$, $a\ne 0$.
\end{baitoan}

\subsubsection{Khảo sát sự biến thiên \& vẽ đồ thị của hàm số bậc 2 $y = ax^2 + bx + c$, $a\ne 0$}

\begin{baitoan}
	Khảo sát sự biến thiên \& vẽ đồ thị hàm số\emph{\texttt{/}}tam thức bậc 2 $y = f(x) = ax^2 + bx + c$, $a\ne 0$.
\end{baitoan}

\subsubsection{Khảo sát sự biến thiên \& vẽ đồ thị của hàm số bậc 3 $y = ax^3 + bx^2 + cx + d$, $a\ne 0$}

\begin{baitoan}
	Khảo sát sự biến thiên \& vẽ đồ thị hàm số bậc 3 $y = f(x) = ax^3 + bx^2 + cx + d$, $a\ne 0$.
\end{baitoan}

\begin{baitoan}[\cite{TL_chuyen_Toan_Giai_Tich_12}, Ví dụ 2, p. 19]
	Cho hàm số bậc 3 $y = x^3 - 3x^2 + mx$ với $m$ là tham số.
	\begin{enumerate*}
		\item[(a)] Khảo sát \& vẽ đồ thị hàm số ứng với $m = 0$.
		\item[(b)] Tìm tất cả các giá trị của tham số $m$ sao cho đồ thị của hàm số có điểm cực đại, cực tiểu. Trong trường hợp đó, viết phương trình đường thẳng đi qua 2 điểm cực trị.
	\end{enumerate*}
\end{baitoan}

\begin{baitoan}[\cite{TL_chuyen_Toan_Giai_Tich_12}, \textbf{15.}, p. 22]
	\begin{enumerate*}
		\item[(a)] Biết rằng đồ thị của hàm số $y = (3a^2 - 1)x^3 - (b^3 + 1)x^2 + 3c^2x + 4d$ có 2 điểm cực trị là $(1;-7)$, $(2,-8)$. Xác định $M = a^2 + b^2 + c^2 + d^2$.
		\item[(b)] Chứng minh rằng đồ thị hàm số $y = x^4 + 2m^2x^2 + 1$ luôn cắt đường thẳng $y = x + 1$ tại đúng 2 điểm phân biệt với mọi giá trị $m$.
	\end{enumerate*}
\end{baitoan}

\begin{baitoan}[\cite{TL_chuyen_Toan_Giai_Tich_12}, \textbf{16.}, p. 23]
	Cho hàm số $y = -x^3 - 3x^2 + mx + 4$ với $m$ là tham số thực.
	\begin{enumerate*}
		\item[(a)] Khảo sát sự biến thiên \& vẽ đồ thị hàm số khi $m = 0$.
		\item[(b)] Tìm tất cả các giá trị của tham số $m$ để hàm số đã cho nghịch biến trên $(0;+\infty)$.
	\end{enumerate*}
\end{baitoan}

\begin{baitoan}[\cite{TL_chuyen_Toan_Giai_Tich_12}, \textbf{18.}, p. 23]
	Cho hàm số $y = f(x) = mx^3 + 3mx^2 - (m - 1)x - 1$ với $m$ là tham số.
	\begin{enumerate*}
		\item[(a)] Khảo sát sự biến thiên \& vẽ đồ thị hàm số khi $m = 1$.
		\item[(b)] Xác định tất cả các giá trị $m$ để hàm số $y = f(x)$ không có cực trị.
	\end{enumerate*}
\end{baitoan}

\begin{baitoan}[\cite{TL_chuyen_Toan_Giai_Tich_12}, \textbf{19.}, p. 23]
	Cho hàm số $y = -2x^3 + 6x^2 - 5$ có đồ thị $(C)$.
	\begin{enumerate*}
		\item[(a)] Khảo sát sự biến thiên \& vẽ đồ thị hàm số $(C)$.
		\item[(b)] Viết phương trình tiếp tuyến của $(C)$ đi qua điểm $A(-1;-13)$.
	\end{enumerate*}
\end{baitoan}

\begin{baitoan}[\cite{TL_chuyen_Toan_Giai_Tich_12}, \textbf{20.}, p. 23]
	Cho hàm số $y = x^3 - 3x^2 - 9x + m$ với tham số $m$.
	\begin{enumerate*}
		\item[(a)] Khảo sát sự biến thiên \& vẽ đồ thị hàm số đã cho khi $m = 0$.
		\item[(b)] Tìm tất cả các giá trị $m$ để đồ thị hàm số cắt trục hoành tại 3 điểm phân biệt có hoành độ lập thành cấp số cộng.
	\end{enumerate*}
\end{baitoan}

\subsubsection{Khảo sát sự biến thiên \& vẽ đồ thị của hàm số bậc 4 dạng trùng phương $y = ax^4 + bx^2 + c$, $a\ne 0$}

\begin{baitoan}
	Khảo sát sự biến thiên \& vẽ đồ thị hàm số bậc 4 dạng trùng phương $y = f(x) = ax^4 + bx^2 + c$, $a\ne 0$.
\end{baitoan}

\begin{baitoan}[\cite{TL_chuyen_Toan_Giai_Tich_12}, Ví dụ 3, p. 19]
	Cho hàm số $y = x^4 - 2m^2x^2 + 1$ $(C_m)$ với $m$ là tham số.
	\begin{enumerate*}
		\item[(a)] Khảo sát \& vẽ đồ thị hàm số khi $m = 1$.
		\item[(b)] Tìm $m$ để đồ thị $(C_m)$ có 3 điểm cực trị tạo thành 1 tam giác có diện tích bằng $32$.
	\end{enumerate*}
\end{baitoan}

\begin{baitoan}[\cite{TL_chuyen_Toan_Giai_Tich_12}, \textbf{17.}, p. 23]
	Cho hàm số $y = f(x) = 8x^4 - 9x^2 + 1$.
	\begin{enumerate*}
		\item[(a)] Khảo sát sự biến thiên \& vẽ đồ thị hàm số trên.
		\item[(b)] Dựa vào đồ thị trên, biện luận theo $m$ số nghiệm của phương trình lượng giác $8\cos^4x - 9\cos^2x + m = 0$ với $x\in[0;\pi]$.
	\end{enumerate*}
\end{baitoan}

\begin{baitoan}[\cite{TL_chuyen_Toan_Giai_Tich_12}, \textbf{21.}, p. 23]
	Cho hàm số $y = f(x) = x^4 - 2x^2$ có đồ thị $(C)$.
	\begin{enumerate*}
		\item[(a)] Khảo sát \& vẽ đồ thị $(C)$.
		\item[(b)] Trên đồ thị $(C)$ lấy 2 điểm phân biệt là $A$ \& $B$ có hoành độ lần lượt là $a,b$. Tìm điều kiện của $a,b$ để tiếp tuyến tại $(C)$ tại các điểm $A$ \& $B$ song song với nhau.
	\end{enumerate*}
\end{baitoan}

\subsubsection{Khảo sát sự biến thiên \& vẽ đồ thị của hàm số bậc 4 $y = ax^4 + bx^3 + cx^2 + dx + e$ $[\star]$}

\begin{baitoan}[$\star$]
	Khảo sát sự biến thiên \& vẽ đồ thị hàm số bậc 4 $y = f(x) = ax^4 + bx^3 + cx^2 + dx + e$, $a\ne 0$.
\end{baitoan}
%------------------------------------------------------------------------------%

\subsection{Khảo Sát Sự Biến Thiên \& Vẽ Đồ Thị của 1 Số Hàm Phân Thức Hữu Tỷ}

\subsubsection{Khảo sát sự biến thiên \& vẽ đồ thị của hàm số nhất biến $y = \frac{ax + b}{cx + d}$, $c\ne 0$, $ad - bc\ne 0$}

\begin{baitoan}
	Khảo sát sự biến thiên \& vẽ đồ thị hàm số nhất biến $y = \frac{ax + b}{cx + d}$, $c\ne 0$, $ad - bc\ne 0$.
\end{baitoan}

\begin{baitoan}[\cite{TL_chuyen_Toan_Giai_Tich_12}, Ví dụ 1, p. 24]
	Cho hàm số $y = \frac{2x + 1}{x - 1}$.
	\begin{enumerate*}
		\item[(a)] Khảo sát sự biến thiên \& vẽ đồ thị hàm số.
		\item[(b)] Gọi $M$ là 1 điểm di động trên $(C)$. Tiếp tuyến tại $M$ của đồ thị $(C)$ cắt 2 đường tiệm cận tại $A$ \& $B$. Tìm giá trị nhỏ nhất của độ dài đoạn $AB$.
	\end{enumerate*}
\end{baitoan}

\begin{baitoan}[\cite{TL_chuyen_Toan_Giai_Tich_12}, \textbf{22.}, p. 29]
	Biết rằng đồ thị hàm số $y = \frac{ax + b}{cx + d}$, $ac\ne 0$, $ad - bc\ne 0$ có tâm đối xứng là $I\left(2;\frac{1}{2}\right)$ \& đi qua gốc tọa độ. Xác định tung độ của điểm có hoành độ là $1$ thuộc đồ thị.
\end{baitoan}

\begin{baitoan}[\cite{TL_chuyen_Toan_Giai_Tich_12}, \textbf{23.}, p. 29]
	\begin{enumerate*}
		\item[(a)] Chứng minh rằng $\forall m\ne 1$ thì đồ thị của hàm số $y = \frac{(2m - 1)x - m^2}{x - 1}$ luôn tiếp xúc với đường phân giác của góc phần tư thứ nhất.
		\item[(b)] Tìm $m$ để tiệm cận xiên của đồ thị hàm số $y = \frac{x^2 + (m + 2)x + 2m + 2}{x + 2}$ tiếp xúc với đường cong $(C):y = x^3 - 3x^2 - 8x$.
	\end{enumerate*}
\end{baitoan}

\begin{baitoan}[\cite{TL_chuyen_Toan_Giai_Tich_12}, \textbf{25.}, pp. 29--30]
	Cho hàm số $y = \frac{x}{4(x - 3)}$ có đồ thị $(C)$.
	\begin{enumerate*}
		\item[(a)] Khảo sát \& vẽ đồ thị của hàm số đã cho.
		\item[(b)] Tìm tọa độ điểm $M\in(C)$ sao cho tiếp tuyến của $(C)$ tại $M$ cắt 2 trục tọa độ $Ox,Oy$ lần lượt tại 2 điểm $A,B$ \& diện tích $\Delta OAB$ là $\frac{3}{8}$.
	\end{enumerate*}
\end{baitoan}

\begin{baitoan}[\cite{TL_chuyen_Toan_Giai_Tich_12}, \textbf{26.}, p. 30]
	Cho hàm số $y = \frac{x - 1}{2x + 3}$ có đồ thị $(C)$.
	\begin{enumerate*}
		\item[(a)] Khảo sát \& vẽ đồ thị hàm số.
		\item[(b)] Với mỗi điểm $M$ bất kỳ thuộc $(C)$, tìm giá trị nhỏ nhất của tổng khoảng cách từ $M$ đến 2 trục tọa độ.
	\end{enumerate*}
\end{baitoan}

\subsubsection{Khảo sát sự biến thiên \& vẽ đồ thị của hàm số $y = \frac{ax^2 + bx + c}{a'x + b'}$, $a\ne 0$, $a'\ne 0$ \& tử thức $\not\vdots$ mẫu thức}

\begin{baitoan}
	Khảo sát sự biến thiên \& vẽ đồ thị hàm số $y = \frac{ax^2 + bx + c}{a'x + b'}$, $a\ne 0$, $a'\ne 0$ \& tử thức không chia hết cho mẫu thức.
\end{baitoan}

\begin{baitoan}[\cite{TL_chuyen_Toan_Giai_Tich_12}, Ví dụ 2, p. 27]
	Cho hàm số $y = \frac{x^2 + x + 1}{x + 1}$ có đồ thị $(C)$.
	\begin{enumerate*}
		\item[(a)] Khảo sát \& vẽ đồ thị hàm số.
		\item[(b)] Biết rằng $A$ \& $B$ là 2 điểm phân biệt trên đồ thị sao cho tiếp tuyến tại 2 điểm này song song với nhau. Chứng minh rằng $A,B$ đối xứng với nhau qua tâm đối xứng của đồ thị $(C)$.
	\end{enumerate*}
\end{baitoan}

\begin{baitoan}[\cite{TL_chuyen_Toan_Giai_Tich_12}, \textbf{27.}, p. 30]
	Cho hàm số $y = 2x - 1 + \frac{1}{x - 1}$.
	\begin{enumerate*}
		\item[(a)] Khảo sát \& vẽ đồ thị hàm số.
		\item[(b)] Tìm tọa độ điểm $M$ thuộc đồ thị sao cho tổng khoảng cách từ $M$ đến 2 đường tiệm cận nhỏ nhất.
	\end{enumerate*}
\end{baitoan}

\subsubsection{Khảo sát sự biến thiên \& vẽ đồ thị của hàm số $y = \frac{ax + b}{a'x^2 + b'x + c}$, $a\ne 0$, $a'\ne 0$ \& mẫu thức $\not\vdots$ tử thức}

\begin{baitoan}
	Khảo sát sự biến thiên \& vẽ đồ thị hàm số $y = \frac{ax + b}{a'x^2 + b'x + c}$, $a\ne 0$, $a'\ne 0$, \& mẫu thức không chia hết cho tử thức.
\end{baitoan}

\subsubsection{Khảo sát sự biến thiên \& vẽ đồ thị của hàm số $y = \frac{ax^2 + bx + c}{a'x^2 + b'x + c'}$, $a\ne 0$, $a'\ne 0$, tử thức \& mẫu thức không có nhân tử chung}

\begin{baitoan}
	Khảo sát sự biến thiên \& vẽ đồ thị hàm số $y = \frac{ax^2 + bx + c}{a'x^2 + b'x + c'}$, $a\ne 0$, $a'\ne 0$, \& tử thức \& mẫu thức không có nhân tử chung.
\end{baitoan}

\begin{baitoan}[\cite{TL_chuyen_Toan_Giai_Tich_12}, \textbf{24.}, p. 29]
	\begin{enumerate*}
		\item[(a)] Cho hàm số $y = \frac{x^2 + px + q}{x^2 + 1}$ trong đó $p\ne 0$, $p^2 + q^2 = 1$. Tìm tất cả các giá trị $p,q$ sao cho khoảng cách giữa 2 điểm cực trị là $\sqrt{10}$.
		\item[(b)] Chứng minh rằng $\forall m\in\mathbb{R}$ thì đồ thị của hàm số $y = \frac{x^2 + (m + 1)x + m + 1}{x + 1}$ luôn có 2 điểm cực trị \& khoảng cách giữa chúng không đổi.
	\end{enumerate*}
\end{baitoan}

\subsubsection{Miscellaneous}

\begin{baitoan}[\cite{TL_chuyen_Toan_Giai_Tich_12}, \textbf{28.}, p. 30]
	Cho hàm số $y = \frac{(x - 1)^3 + a + 1}{x}$.
	\begin{enumerate*}
		\item[(a)] Tìm các giá trị của $a$ để đồ thị của hàm số có 3 cực trị \& chứng minh rằng với các giá trị đó thì các cực trị này sẽ nằm trên 1 parabol cố định.
		\item[(b)] Chứng minh rằng $\forall a\in\mathbb{R}$, đồ thị của hàm số $y = \frac{x + a}{x^2 + x + 1}$ luôn có 3 điểm uốn thẳng hàng.
	\end{enumerate*}
\end{baitoan}
	
%------------------------------------------------------------------------------%

\subsection{1 Số Bài Toán Thường Gặp về Đồ Thị}

\subsubsection{Viết phương trình đường thẳng đi qua các điểm đặc biệt của đồ thị hàm số}

\begin{baitoan}[\cite{TL_chuyen_Toan_Giai_Tich_12}, Ví dụ 1, p. 31]
	Viết phương trình đường thẳng đi qua các điểm cực trị của đồ thị hàm số $y = x^3 - 3x + 2$.
\end{baitoan}

\begin{baitoan}[\cite{TL_chuyen_Toan_Giai_Tich_12}, Ví dụ 2, p. 31]
	Chứng minh rằng $\forall a\in\mathbb{R}$, đồ thị của hàm số $y = \frac{x + a}{x^2 + 1}$ luôn có 3 điểm thẳng hàng.
\end{baitoan}

\subsubsection{Họ đường cong phụ thuộc tham số}

\begin{baitoan}[\cite{TL_chuyen_Toan_Giai_Tich_12}, Ví dụ 3, p. 32]
	Cho hàm số $y = x^3 - mx^2 + (2m + 1)x - m - 2$ $(C_m)$.
	\begin{enumerate*}
		\item[(a)] Tìm điểm cố định của họ $(C_m)$ khi $m$ thay đổi.
		\item[(b)] Tìm $m$ để $(C_m)$ cắt $y = 0$ tại 3 điểm phân biệt có hoành độ dương.
	\end{enumerate*}
\end{baitoan}

\begin{baitoan}[\cite{TL_chuyen_Toan_Giai_Tich_12}, Ví dụ 4, p. 33]
	Cho hàm số $y = (m + 1)x^3 - (2m + 1)x - m + 1$ $(C_m)$.
	\begin{enumerate*}
		\item[(a)] Chứng minh rằng $\forall m\in\mathbb{R}$, $(C_m)$ luôn đi qua 3 điểm cố định thẳng hàng.
		\item[(b)] Với giá trị nào của $m$ thì $(C_m)$ có tiếp tuyến vuông góc với đường thẳng chứa 3 điểm cố định nói trong câu (a).
	\end{enumerate*}
\end{baitoan}

\begin{baitoan}[\cite{TL_chuyen_Toan_Giai_Tich_12}, Ví dụ 5, p. 34]
	Cho hàm số $y = \frac{(3m + 1)x - (m^2 - m)}{x + m}$ ($m\ne 0$). Tìm tất cả các điểm trên mặt phẳng mà đồ thị không thể đi qua khi $m$ thay đổi.
\end{baitoan}

\begin{baitoan}[\cite{TL_chuyen_Toan_Giai_Tich_12}, Ví dụ 6, p. 35]
	Cho hàm số $y = x^3 - 3mx^2 + 3(m^2 - 1)x + 1 - m^2$ $(C_m)$. Tìm $m$ để $(C_m)$ có 2 điểm phân biệt đối xứng với nhau qua gốc tọa độ.
\end{baitoan}

\begin{baitoan}[\cite{TL_chuyen_Toan_Giai_Tich_12}, Ví dụ 7, p. 35]
	Cho hàm số $y = x^3 - 3x^2 + m^2x + m$. Tìm tất cả các giá trị của $m$ để hàm số có các điểm cực đại, cực tiểu đối xứng nhau qua đường thẳng $(d)$: $x - 2y - 5 = 0$.
\end{baitoan}

\begin{baitoan}[\cite{TL_chuyen_Toan_Giai_Tich_12}, Ví dụ 8, p. 36]
	Cho hàm số $y = \frac{2x + 1}{x - 2}$ có đồ thị $(C)$. Tìm phương trình đường cong $(C')$ đối xứng với $(C)$ qua đường thẳng $(d)$ có phương trình $y = 2$. 
\end{baitoan}

\subsubsection{Ứng dụng đồ thị hàm số trong các bài toán biện luận số nghiệm của phương trình}

\begin{baitoan}[\cite{TL_chuyen_Toan_Giai_Tich_12}, Ví dụ 9, p. 36]
	Biện luận số nghiệm của phương trình sau theo tham số $m$: $x^3 - 3x + 2 = m$.
\end{baitoan}

\begin{baitoan}[\cite{TL_chuyen_Toan_Giai_Tich_12}, Ví dụ 10, p. 37]
	Biện luận theo $m$ số nghiệm của phương trình $4|x|^3 - 3|x| = m$.
\end{baitoan}

\begin{baitoan}[\cite{TL_chuyen_Toan_Giai_Tich_12}, Ví dụ 11, p. 37]
	Tìm $m$ để phương trình sau có 1 nghiệm duy nhất: $x^3 - x^2 = mx - 1$.
\end{baitoan}

\begin{baitoan}[\cite{TL_chuyen_Toan_Giai_Tich_12}, H3, p. 38]
	Tìm tất cả các giá trị $m$ sao cho phương trình $x^3 - x^2 = (m^2 + m)x - 1$ có nghiệm duy nhất.
\end{baitoan}

\begin{baitoan}[\cite{TL_chuyen_Toan_Giai_Tich_12}, \textbf{29.}, p. 39]
	Cho $(C_m)$ có phương trình $y = x^3 + (m - 1)x - (m + 3)x - 1$.
	\begin{enumerate*}
		\item[(a)] Khảo sát \& vẽ đồ thị $(C)$ của hàm số khi $m = 1$.
		\item[(b)] Chứng minh rằng $\forall m\in\mathbb{R}$, hàm số có cực đại, cực tiểu. Viết phương trình đường thẳng đi qua các điểm cực đại \& cực tiểu của đồ thị.
		\item[(c)] Tìm những cặp điểm nguyên trên $(C)$ đối xứng với nhau qua đường thẳng $y = x$ \& không nằm trên đường thẳng đó.
	\end{enumerate*}
\end{baitoan}

\begin{baitoan}[\cite{TL_chuyen_Toan_Giai_Tich_12}, \textbf{30.}, p. 39]
	Cho họ đường cong $(C_m)$: $y = \frac{-x^2 + mx - m^2}{x - m}$.
	\begin{enumerate*}
		\item[(a)] Khảo sát \& vẽ đồ thị $(C)$ của hàm số khi $m = 1$.
		\item[(b)] Xác định $m$ để hàm số có cực đại, cực tiểu. Viết phương trình đường thẳng đi qua các điểm cực đại \& cực tiểu của đồ thị hàm số.
		\item[(c)] Tìm các điểm trong mặt phẳng sao cho có đúng 2 đường của của họ $(C_m)$ đi qua.
	\end{enumerate*}
\end{baitoan}

\begin{baitoan}[\cite{TL_chuyen_Toan_Giai_Tich_12}, \textbf{31.}, p. 39]
	Cho hàm số $y = x^3 - 3(m + 1)x^2 + 2(m^2 + 4m + 1)x - 4m(m + 1)$ $(C_m)$.
	\begin{enumerate*}
		\item[(a)] Chứng minh rằng $(C_m)$ luôn đi qua 1 điểm cố định khi $m$ thay đổi.
		\item[(b)] Tìm $m$ sao cho $(C_m)$ cắt trục hoành tại 3 điểm phân biệt.
	\end{enumerate*}
\end{baitoan}

\begin{baitoan}[\cite{TL_chuyen_Toan_Giai_Tich_12}, \textbf{32.}, p. 39]
	Cho hàm số $y = \frac{x^2}{x - 1}$ $(C)$.
	\begin{enumerate*}
		\item[(a)] Khảo sát sự biến thiên \& vẽ đồ thị $(C)$.
		\item[(b)] Tìm 2 điểm $A,B\in(C)$ \& đối xứng nhau qua đường thẳng $y = x - 1$.
	\end{enumerate*}
\end{baitoan}

\begin{baitoan}[\cite{TL_chuyen_Toan_Giai_Tich_12}, \textbf{33.}, p. 39]
	Cho hàm số $y = \frac{2x^2 + (6 - m)x + 4}{mx + 2}$. Chứng minh rằng $\forall m\in\mathbb{R}$, đồ thị hàm số luôn đi qua 1 điểm cố định duy nhất. Xác định tọa độ của điểm đó.
\end{baitoan}

\begin{baitoan}[\cite{TL_chuyen_Toan_Giai_Tich_12}, \textbf{34.}, p. 39]
	Cho hàm số $y = \frac{(x - 1)^2}{x + 2}$.
	\begin{enumerate*}
		\item[(a)] Khảo sát sự biến thiên \& vẽ đồ thị hàm số đã cho.
		\item[(b)] Biện luận theo $m$ số nghiệm của phương trình $\frac{(x - 1)^2}{|x + 2|} = m$.
	\end{enumerate*}
\end{baitoan}

\begin{baitoan}[\cite{TL_chuyen_Toan_Giai_Tich_12}, \textbf{35.}, p. 39]
	Tìm $m$ để phương trình sau có 4 nghiệm phân biệt: $4|x|^3 - 3|x| - 1 = mx - m$.
\end{baitoan}

%------------------------------------------------------------------------------%

\section{Hàm Số Lũy Thừa, Hàm Số Mũ, \& Hàm Số Logarith}

\subsection{Lũy Thừa với Số Mũ Hữu Tỷ}

%------------------------------------------------------------------------------%

\subsection{Lũy Thừa với Số Mũ Thực}

%------------------------------------------------------------------------------%

\subsection{Logarithm}

%------------------------------------------------------------------------------%

\subsection{Số $\rm e$ \& Logarith Tự Nhiên}

%------------------------------------------------------------------------------%

\subsection{Hàm Số Mũ \& Hàm Số Logarithm}

%------------------------------------------------------------------------------%

\subsection{Hàm Số Lũy Thừa}

%------------------------------------------------------------------------------%

\subsection{Phương Trình Mũ \& Logarithm}

%------------------------------------------------------------------------------%

\subsection{Hệ Phương Trình Mũ \& Logarithm}

%------------------------------------------------------------------------------%

\subsection{Bất Phương Trình Mũ \& Logarithm}

%------------------------------------------------------------------------------%

\section{Nguyên Hàm, Tích Phân, \& Ứng Dụng}

\subsection{Nguyên Hàm}

%------------------------------------------------------------------------------%

\subsection{1 Số Phương Pháp Tìm Nguyên Hàm}

%------------------------------------------------------------------------------%

\subsection{Tích Phân}

%------------------------------------------------------------------------------%

\subsection{1 Số Phương Pháp Tính Tích Phân}

%------------------------------------------------------------------------------%

\subsection{Ứng Dụng Tích Phân Để Tính Diện Tích Hình Phẳng}

%------------------------------------------------------------------------------%

\subsection{Ứng Dụng Tích Phân Để Tính Thể Tích Vật Thể}

%------------------------------------------------------------------------------%

\section{Số Phức}

\subsection{Số Phức}

%------------------------------------------------------------------------------%

\subsection{Căn Bậc 2 của Số Phức \& Phương Trình Bậc 2}

%------------------------------------------------------------------------------%

\subsection{Dạng Lượng Giác của Số Phức \& Ứng Dụng}

%------------------------------------------------------------------------------%

\section{Khối Đa Diện \& Thể Tích của Chúng}

\subsection{Khái Niệm về Khối Đa Diện}

%------------------------------------------------------------------------------%

\subsection{Phép Đối Xứng qua Mặt Phẳng \& Sự Bằng Nhau của Các Khối Đa Diện}

%------------------------------------------------------------------------------%

\subsection{Phép Vị Tự \& Sự Đồng Dạng của Các Khối Đa Diện. Các Khối Đa Diện Đều}

%------------------------------------------------------------------------------%

\subsection{Thể Tích của Khối Đa Diện}

%------------------------------------------------------------------------------%

\section{Mặt Cầu, Mặt Trụ, Mặt Nón}

\subsection{Mặt Cầu, Khối Cầu}

%------------------------------------------------------------------------------%

\subsection{Khái Niệm về Mặt Tròn Xoay}

%------------------------------------------------------------------------------%

\subsection{Mặt Trụ, Hình Trụ, \& Khối Trụ}

%------------------------------------------------------------------------------%

\subsection{Mặt Nón, Hình Nón, \& Khối Nón}

%------------------------------------------------------------------------------%

\section{Phương Pháp Tọa Độ Trong Không Gian}

\subsection{Hệ Tọa Độ Trong Không Gian}

%------------------------------------------------------------------------------%

\subsection{Phương Trình Mặt Phẳng}

%------------------------------------------------------------------------------%

\subsection{Phương Trình Đường Thẳng}

%------------------------------------------------------------------------------%

\printbibliography[heading=bibintoc]
	
\end{document}