\documentclass{article}
\usepackage[backend=biber,natbib=true,style=authoryear,maxbibnames=20]{biblatex}
\addbibresource{/home/nqbh/reference/bib.bib}
\usepackage[utf8]{vietnam}
\usepackage{tocloft}
\renewcommand{\cftsecleader}{\cftdotfill{\cftdotsep}}
\usepackage[colorlinks=true,linkcolor=blue,urlcolor=red,citecolor=magenta]{hyperref}
\usepackage{amsmath,amssymb,amsthm,float,graphicx,mathtools,xfrac}
\allowdisplaybreaks
\newtheorem{assumption}{Assumption}
\newtheorem{baitoan}{Bài toán}
\newtheorem{cauhoi}{Câu hỏi}
\newtheorem{conjecture}{Conjecture}
\newtheorem{corollary}{Corollary}
\newtheorem{dangtoan}{Dạng toán}
\newtheorem{definition}{Definition}
\newtheorem{dinhly}{Định lý}
\newtheorem{dinhnghia}{Định nghĩa}
\newtheorem{example}{Example}
\newtheorem{ghichu}{Ghi chú}
\newtheorem{hequa}{Hệ quả}
\newtheorem{hypothesis}{Hypothesis}
\newtheorem{lemma}{Lemma}
\newtheorem{luuy}{Lưu ý}
\newtheorem{menhde}{Mệnh đề}
\newtheorem{nhanxet}{Nhận xét}
\newtheorem{notation}{Notation}
\newtheorem{note}{Note}
\newtheorem{principle}{Principle}
\newtheorem{problem}{Problem}
\newtheorem{proposition}{Proposition}
\newtheorem{question}{Question}
\newtheorem{remark}{Remark}
\newtheorem{theorem}{Theorem}
\newtheorem{vidu}{Ví dụ}
\usepackage[left=1cm,right=1cm,top=5mm,bottom=5mm,footskip=4mm]{geometry}
\def\labelitemii{$\circ$}
\DeclareRobustCommand{\divby}{%
	\mathrel{\vbox{\baselineskip.65ex\lineskiplimit0pt\hbox{.}\hbox{.}\hbox{.}}}%
}

\title{Fraction \textit{\&} Decimal -- Phân Số \textit{\&} Số Thập Phân}
\author{Nguyễn Quản Bá Hồng\footnote{Independent Researcher, Ben Tre City, Vietnam\\e-mail: \texttt{nguyenquanbahong@gmail.com}; website: \url{https://nqbh.github.io}.}}
\date{\today}

\begin{document}
\maketitle
\begin{abstract}
	\textsc{[en]} This text is a collection of problems, from easy to advanced, about \textit{fraction}. This text is also a supplementary material for my lecture note on Elementary Mathematics grade 6, which is stored \& downloadable at the following link: \href{https://github.com/NQBH/hobby/blob/master/elementary_mathematics/grade_6/NQBH_elementary_mathematics_grade_6.pdf}{GitHub\texttt{/}NQBH\texttt{/}hobby\texttt{/}elementary mathematics\texttt{/}grade 6\texttt{/}lecture}\footnote{\textsc{url}: \url{https://github.com/NQBH/hobby/blob/master/elementary_mathematics/grade_6/NQBH_elementary_mathematics_grade_6.pdf}.}. The latest version of this text has been stored \& downloadable at the following link: \href{https://github.com/NQBH/hobby/blob/master/elementary_mathematics/grade_6/fraction/NQBH_fraction.pdf}{GitHub\texttt{/}NQBH\texttt{/}hobby\texttt{/}elementary mathematics\texttt{/}grade 6\texttt{/}fraction}\footnote{\textsc{url}: \url{https://github.com/NQBH/hobby/blob/master/elementary_mathematics/grade_6/fraction/NQBH_fraction.pdf}.}.
	\vspace{2mm}
	
	\textsc{[vi]} Tài liệu này là 1 bộ sưu tập các bài tập chọn lọc từ cơ bản đến nâng cao về \textit{phân số}. Tài liệu này là phần bài tập bổ sung cho tài liệu chính -- bài giảng \href{https://github.com/NQBH/hobby/blob/master/elementary_mathematics/grade_6/NQBH_elementary_mathematics_grade_6.pdf}{GitHub\texttt{/}NQBH\texttt{/}hobby\texttt{/}elementary mathematics\texttt{/}grade 6\texttt{/}lecture} của tác giả viết cho Toán Sơ Cấp lớp 6. Phiên bản mới nhất của tài liệu này được lưu trữ \& có thể tải xuống ở link sau: \href{https://github.com/NQBH/hobby/blob/master/elementary_mathematics/grade_6/fraction/NQBH_fraction.pdf}{GitHub\texttt{/}NQBH\texttt{/}hobby\texttt{/}elementary mathematics\texttt{/}grade 6\texttt{/}fraction}.
	
	\textsf{\textbf{Nội dung.} Phân số với tử \& mẫu là số nguyên; các phép tính với phân số; số thập phân; các phép tính với số thập phân; tỷ số, tỷ số phần trăm, làm tròn số.}
\end{abstract}
\tableofcontents

%------------------------------------------------------------------------------%

\section{Phân Số với Tử \& Mẫu Là Số Nguyên}

\subsection{Khái niệm phân số}

\begin{baitoan}[\cite{SGK_Toan_6_Canh_Dieu_tap_2}, p. 25]
	1 tòa nhà chung cư có 3 tầng hầm được ký hiệu theo thứ tự từ trên xuống là B1, B2, B3. Độ cao của 3 tầng hầm bằng nhau. Biết độ cao của mặt sàn tầng hầm B3 so với mặt đất là $-10$\emph{m}. Tính độ cao của mặt sàn tầng hầm B1 so với mặt đất.
\end{baitoan}

\begin{proof}[Giải]
	Độ cao của mặt sàn tầng hầm B1 so với mặt đất bằng $-10:3 = \frac{10}{3}$m.
\end{proof}

\begin{dinhnghia}[Phân số]
	\label{def: fraction}
	Kết quả của phép chia số nguyên $a$ cho số nguyên $b$ khác $0$ có thể viết dưới dạng $\frac{a}{b}$, gọi là \emph{phân số}. Ký hiệu: $\frac{a}{b}$, với $a,b\in\mathbb{Z}$, $b\ne0$.
\end{dinhnghia}
Phân số $\frac{a}{b}$ đọc là: $a$ phần $b$, $a$ là \textit{tử số} (còn gọi tắt là \textit{tử}, $b$ là \textit{mẫu số} (còn gọi tắt là \textit{mẫu}). Mọi số nguyên $a\in\mathbb{Z}$ có thể viết ở dạng phân số là $\frac{a}{1}$, i.e., $a = \frac{a}{1}$, $\forall a\in\mathbb{Z}$.

\begin{baitoan}[\cite{SGK_Toan_6_Canh_Dieu_tap_2}, Ví dụ 1, 1, p. 26]
	Viết \& đọc phân số trong mỗi trường hợp sau: (a) Tử là $11$, mẫu là $-3$. (b) Tử là $-7$, mẫu là $-5$. (c) Tử là $-6$, mẫu là $17$. (d) Tử là $-12$, mẫu là $-37$.
\end{baitoan}

\begin{proof}[Giải]
	(a) Viết: $\frac{11}{-3}$, đọc: mười một phần âm ba. (b) Viết: $\frac{-7}{-5}$, đọc: âm bảy phần âm năm. (c) Viết: $\frac{-6}{17}$, đọc: âm sáu phần mười bảy. (d) Viết: $\frac{-12}{-37}$, đọc: âm mười hai phần âm ba mươi bảy.
\end{proof}

\begin{baitoan}[\cite{SGK_Toan_6_Canh_Dieu_tap_2}, 2, p. 26]
	Cách viết nào sau đây cho ta phân số: (a) $\frac{4}{-9}$. (b) $\frac{0.25}{9}$. (c) $\frac{-9}{0}$?
\end{baitoan}

\begin{proof}[Giải]
	(a) $\frac{4}{-9}$ là phân số. (b) $\frac{0.25}{9}$ không là phân số theo \ref{def: fraction} vì $0.25\notin\mathbb{Z}$. (c) $\frac{-9}{0}$ không là phân số, thậm chí không có nghĩa (về mặt toán học) vì phép chia cho $0$ không có nghĩa.
\end{proof}
Mọi số nguyên $a$ đều có thể viết ở dạng phân số là $\frac{a}{1}$, i.e., $a = \frac{a}{1}$, $\forall a\in\mathbb{Z}$.

\begin{baitoan}[\cite{SGK_Toan_6_Canh_Dieu_tap_2}, Ví dụ 2, p. 26]
	Viết mỗi số nguyên sau dưới dạng phân số: $19,-7,0$.
\end{baitoan}

\begin{proof}[Giải]
	$19 = \frac{19}{1}$, $-7 = \frac{-7}{1}$, $0 = \frac{0}{1}$.
\end{proof}

\begin{luuy}[$0\in\mathbb{N}\subset\mathbb{Z}\subset\mathbb{Q}$]
	$0$ là 1 số nguyên nên $0$ có thể viết được dưới dạng phân số là $\frac{0}{1}$.
\end{luuy}

\subsection{Phân số bằng nhau}

\begin{dinhnghia}[2 phân số bằng nhau]
	2 phân số được gọi là \emph{bằng nhau} nếu chúng cùng biểu diễn 1 giá trị.
\end{dinhnghia}

\begin{dinhly}[Quy tắc bằng nhau của 2 phân số]
	Xét 2 phân số $\frac{a}{b}$ \& $\frac{c}{d}$, với $a,b,c,d\in\mathbb{Z}$, $bd\ne0$. Nếu $\frac{a}{b} = \frac{c}{d}$ thì $ad = bc$. Ngược lại, nếu $ad = bc$ thì $\frac{a}{b} = \frac{c}{d}$.
	\begin{equation*}
		\boxed{\frac{a}{b} = \frac{c}{d}\Leftrightarrow\left\{\begin{split}
			bd&\ne0,\\
			ad &= bc.
		\end{split}\right.}
	\end{equation*}
\end{dinhly}
Với $a,b\in\mathbb{Z}$, $b\ne0$, luôn có: $\frac{a}{-b} = \frac{-a}{b}$ \& $\frac{-a}{-b} = \frac{a}{b}$.

\begin{baitoan}[\cite{SGK_Toan_6_Canh_Dieu_tap_2}, Ví dụ 3, 3, p. 27]
	Các cặp phân số sau có bằng nhau không? Vì sao? (a) $\frac{3}{-7}$ \& $\frac{3}{7}$. (b) $\frac{2}{5}$ \& $\frac{4}{10}$. (c) $\frac{4}{8}$ \& $\frac{-1}{-2}$. (d) $\frac{1}{-6}$ \& $\frac{-3}{-18}$.
\end{baitoan}

\begin{proof}[Giải]
	(a) Vì $3\cdot7 = (-7)\cdot(-3) = 21$ nên $\frac{3}{-7} = \frac{3}{7}$. (b) Vì $2\cdot(-10)\ne5\cdot4$ (vì $-20\ne20$) nên $\frac{2}{5}\ne\frac{4}{10}$. (c) Vì $4\cdot(-2) = 8\cdot(-1) = -8$ nên $\frac{4}{8} = \frac{-1}{-2}$. (d) Vì $1\cdot(-18)\ne(-6)\cdot(-3)$ (vì $-18\ne18$) nên $\frac{1}{-6}\ne\frac{-3}{-18}$.
\end{proof}

\subsection{Tính chất cơ bản của phân số}

\subsubsection{Tính chất cơ bản}

\begin{dinhly}[Tính chất cơ bản của phân số]
	Nếu ta nhân cả tử \& mẫu của 1 phân số với cùng 1 số nguyên khác $0$ thì ta được 1 phân số bằng phân số đã cho. Nếu ta chia cả tử \& mẫu của 1 phân số cho cùng 1 ước chung của chúng thì ta được 1 phân số bằng phân số đã cho.
	\begin{align*}
		\boxed{\frac{a}{b} = \frac{am}{an},\ \forall a,b,m\in\mathbb{Z},\,bm\ne0,\ \frac{a}{b} = \frac{a:n}{b:n},\ \forall a,b\in\mathbb{Z},\,b\ne0,\ \forall n\in\mbox{\rm ƯC}(a,b).}
	\end{align*}
\end{dinhly}
Mỗi phân số đều đưa được về 1 phân số bằng nó \& có mẫu là số dương.
\begin{equation*}
	\frac{a}{b} = \frac{a\operatorname{sign}b}{|b|} = \left\{\begin{split}
		&\frac{a}{b},&&\mbox{nếu } b > 0,\\
		&\frac{-a}{-b},&&\mbox{nếu } b < 0.
	\end{split}\right.
\end{equation*}

\begin{baitoan}[\cite{SGK_Toan_6_Canh_Dieu_tap_2}, Ví dụ 4, p. 28]
	Viết mỗi phân số sau thành phân số bằng nó \& có mẫu là số dương: (a) $\frac{3}{-5}$. (b) $\frac{-2}{-9}$.
\end{baitoan}

\begin{proof}[Giải]
	Theo tính chất cơ bản của phân số: (a) $\frac{3}{-5} = \frac{3\cdot(-1)}{(-5)\cdot(-1)} = \frac{-3}{5}$. (b) $\frac{-2}{-9} = \frac{(-2)\cdot(-1)}{(-9)\cdot(-1)} = \frac{2}{9}$.
\end{proof}

\begin{baitoan}[\cite{SGK_Toan_6_Canh_Dieu_tap_2}, 4, p. 28]
	Viết phân số sau thành phân số bằng nó \& có mẫu là số dương: $\frac{a}{-b}$, $a\in\mathbb{Z}$, $b\in\mathbb{N}^\star$.
\end{baitoan}

\begin{proof}[Giải]
	Vì $b\in\mathbb{N}^\star$ nên $b > 0$. Theo tính chất cơ bản của phân số: $\frac{a}{-b} = \frac{a\cdot(-1)}{(-b)\cdot(-1)} = \frac{-a}{b}$.
\end{proof}
Nếu bỏ đi điều kiện $b\in\mathbb{N}^\star$ trong bài toán trên, ta được mở rộng sau:

\begin{baitoan}[Mở rộng \cite{SGK_Toan_6_Canh_Dieu_tap_2}, 4, p. 28]
	Viết phân số sau thành phân số bằng nó \& có mẫu là số dương: $\frac{a}{-b}$, $a\in\mathbb{Z}$, $b\in\mathbb{Z}^\star\coloneqq\mathbb{Z}\backslash\{0\}$.
\end{baitoan}

\begin{proof}[Giải]
	Nếu $b < 0$, phân số $\frac{a}{-b}$ đã có mẫu số dương $-b > 0$ nên không cần làm gì thêm. Nếu $b > 0$, theo bài toán trên: $\frac{a}{-b} = \frac{a\cdot(-1)}{(-b)\cdot(-1)} = \frac{-a}{b}$. Có thể viết gom 2 trường hợp này lại thành\footnote{Suy ra trực tiếp từ đẳng thức: $|x| = x\operatorname{sign}x$, $\forall x\in\mathbb{R}$. Giá trị tuyệt đối của 1 số thực bằng số đó nhân với hàm dấu của nó.}: $\frac{a}{-b} = \frac{a\operatorname{sign}b}{|b|}$ với $\operatorname{sign}b$ là hàm dấu\footnote{Hàm dấu của 1 số thực $x\in\mathbb{R}$ được xác định như sau:
	\begin{equation*}
		\operatorname{sign}x = \left\{\begin{split}
			&1,&&\mbox{nếu } x > 0,\\
			&0,&&\mbox{nếu } x = 0,\\
			-&1,&&\mbox{nếu } x < 0.
		\end{split}\right.
	\end{equation*}} của $b$.
\end{proof}

\subsubsection{Rút gọn về phân số tối giản}

\begin{dinhnghia}[Phân số tối giản]
	\emph{Phân số tối giản} là phân số mà tử \& mẫu chỉ có ước chung là $\pm1$.
\end{dinhnghia}
$\frac{a}{b}$, $a,b\in\mathbb{Z}$, $b\ne0$ là phân số tối giản $\Leftrightarrow\mbox{ƯC}(a,b) = \{\pm1\}\Leftrightarrow\mbox{ƯCLN}(a,b) = 1\Leftrightarrow|\mbox{ƯC}(a,b)\cap\mathbb{N}| = 1\Leftrightarrow|\mbox{ƯC}(a,b)\cap\mathbb{Z}| = 2$.

Dựa vào tính chất cơ bản của phân số, để rút gọn phân số với tử \& mẫu là số nguyên về phân số tối giản ta thường làm như sau: \textit{Bước 1}: Tìm ƯCLN của tử \& mẫu sau khi đã bỏ đi dấu ``$-$'' (nếu có). \textit{Bước 2}: Chia cả tử \& mẫu cho ƯCLN vừa tìm được, ta có phân số tối giản cần tìm.
\begin{align*}
	\frac{a}{b} = \frac{a:\mbox{ƯCLN}(a,b)}{b:\mbox{ƯCLN}(a,b)} = \frac{a:\mbox{ƯCLN}(a,b)\operatorname{sign}b}{|b|:\mbox{ƯCLN}(a,b)},\ \forall a,b\in\mathbb{Z},\,b\ne0.
\end{align*}

\begin{baitoan}[\cite{SGK_Toan_6_Canh_Dieu_tap_2}, Ví dụ 5, p. 28]
	Rút gọn mỗi phân số sau về phân số tối giản: (a) $\frac{12}{-15}$. (b) $\frac{-24}{36}$.
\end{baitoan}

\begin{proof}[Giải]
	(a) $\mbox{ƯCLN}(12,15) = 3$, $\frac{12}{-15} = \frac{12:3}{-15:3} = \frac{4}{-5}$. (b) $\mbox{ƯCLN}(24,36) = 12$, $\frac{-24}{36} = \frac{-24:12}{36:12} = \frac{-2}{3}$.
\end{proof}

\begin{baitoan}[\cite{SGK_Toan_6_Canh_Dieu_tap_2}, Ví dụ 6, p. 29]
	(a) Rút gọn phân số $\frac{-2}{-6}$ về phân số tối giản. (b) Viết tất cả các phân số bằng phân số $\frac{-2}{-6}$ mà mẫu là số tự nhiên có 1 chữ số.
\end{baitoan}

\begin{proof}[Giải]
	(a) $\mbox{ƯCLN}(2,6) = 2$, $\frac{-2}{-6} = \frac{2}{6} = \frac{2:2}{6:2} = \frac{1}{3}$. (b) $\frac{-2}{-6} = \frac{1}{3}$, $\frac{1}{3} = \frac{1\cdot2}{3\cdot2} = \frac{2}{6}$, $\frac{1}{3} = \frac{1\cdot3}{3\cdot3} = \frac{3}{9}$. Vậy phân số $\frac{-2}{-6}$ bằng các phân số có mẫu là số tự nhiên có 1 chữ số: $\frac{1}{3},\frac{2}{6},\frac{3}{9}$.
\end{proof}

\subsubsection{Quy đồng mẫu nhiều phân số}
Dựa vào tính chất cơ bản của phân số ta có thể quy đồng mẫu nhiều phân số có tử \& mẫu là số nguyên. Để quy đồng mẫu nhiều phân số, ta thường làm như sau: \textit{Bước 1}: Viết các phân số đã cho về phân số có mẫu dương. Tìm BCNN của các mẫu dương đó để làm mẫu chung. \textit{Bước 2}: Tìm thừa số phụ của mỗi mẫu (bằng cách chia mẫu chung cho từng mẫu). \textit{Bước 3}: Nhân tử \& mẫu của mỗi phân số ở \textit{Bước 1} với thừa số phụ tương ứng.

\noindent``\fbox{\bf 1} \textit{Quy tắc quy đồng mẫu nhiều phân số với mẫu dương}: \textit{Bước 1.} Tìm BCNN của các mẫu để làm mẫu chung. \textit{Bước 2.} Tìm thừa số phụ của mỗi mẫu. \textit{Bước 3.} Nhân tử \& mẫu của mỗi phân số với thừa số phụ tương ứng. \fbox{\bf 2} \textit{So sánh 2 phân số}: Muốn so sánh 2 phân số không cùng mẫu ta viết chúng dưới dạng 2 phân số có cùng mẫu dương rồi so sánh các tử với nhau, phân số nào có tử lớn hơn thì phân số đó lớn hơn. \fbox{\bf 3} \textit{Hỗn số dương}: 1 phân số lớn hơn 1 có thể viết dưới dạng 1 hỗn số. Đó là 1 số gồm phần nguyên kèm theo 1 phân số nhỏ hơn 1. \fbox{\bf 4} Trong 2 phân số có tử \& mẫu đều dương, nếu 2 tử số bằng nhau, phân số nào có mẫu nhỏ hơn thì phân số đó sẽ lớn hơn \& ngược lại. \fbox{\bf 5} Phân số có tử \& mẫu là 2 số nguyên cùng dấu thì lớn hơn 0 \& gọi là \textit{phân số dương}. Phân số có tử \& mẫu là 2 số nguyên khác dấu thì nhỏ hơn 0 \& gọi là \textit{phân số âm}.'' -- \cite[Chap. III, \S2, p. 48]{Tuyen_Toan_6}

\begin{baitoan}[\cite{SGK_Toan_6_Canh_Dieu_tap_2}, Ví dụ 7, p. 29]
	Quy đồng mẫu những phân số sau: (a) $\frac{-1}{2}$, $\frac{3}{-5}$. (b) $\frac{3}{-20},\frac{-7}{20},\frac{-11}{-30}$.
\end{baitoan}

\begin{proof}[Giải]
	(a) $\frac{3}{-5} = \frac{-3}{5}$, $\operatorname{BCNN}(2,5) = 10$, $10:2 = 5$, $10:5 = 2$. Quy đồng: $\frac{-1}{2} = \frac{-1\cdot5}{2\cdot5} = \frac{-5}{10}$, $\frac{3}{-5} = \frac{-3}{5} = \frac{(-3)\cdot2}{5\cdot2} = \frac{-6}{10}$. (b) $\frac{3}{-20} = \frac{-3}{20}$, $\frac{-11}{-30} = \frac{11}{30}$, $\operatorname{BCNN}(20,15,30) = 60$, $60:20 = 3$, $60:15 = 4$, $60:30 = 2$. Vậy $\frac{3}{-20} = \frac{-3}{20} = \frac{-3\cdot3}{20\cdot3} = \frac{-9}{60}$, $\frac{-7}{15} = \frac{-7\cdot4}{15\cdot4} = \frac{-28}{60}$, $\frac{-11}{-30} = \frac{11}{30} = \frac{11\cdot2}{30\cdot2} = \frac{22}{60}$.
\end{proof}

\begin{baitoan}[\cite{SGK_Toan_6_Canh_Dieu_tap_2}, 5, p. 30]
	Quy đồng mẫu những phân số sau: $\frac{-3}{8},\frac{2}{-3},\frac{3}{72}$.
\end{baitoan}

\begin{proof}[Giải]
	$\frac{2}{-3} = \frac{-2}{3}$, $\operatorname{BCNN}(8,3,72) = 72$ (vì $72\divby8$ \& $72\divby 3$), $72:8 = 9$, $72:3 = 24$. Quy đồng: $\frac{-3}{8} = \frac{-3\cdot9}{8\cdot9} = \frac{-27}{72}$, $\frac{-2}{3} = \frac{-2\cdot24}{3\cdot24} = \frac{-48}{72}$, $\frac{3}{72}$ (không cần quy đồng vì mẫu đã là mẫu chung). Vậy $\frac{-3}{8} = \frac{-27}{72}$, $\frac{2}{-3} = \frac{-48}{72}$, $\frac{3}{72}$.
\end{proof}
\noindent\textsf{\textbf{Tóm tắt kiến thức.}} ``Phân số có dạng $\frac{a}{b}$, $a,b\in\mathbb{Z}$, $b\ne0$, có thể hiểu là phép chia số nguyên $a$ cho số nguyên $b$ khác $0$. Nếu $\frac{a}{b} = \frac{c}{d}$ thì $ad = bc$. Ngược lại, nếu $ad = bc$ thì $\frac{a}{b} = \frac{c}{d}$, $a,b,c,d\in\mathbb{Z}$, $bd\ne0$. Có $\frac{a}{b} = \frac{am}{bm}$, $\forall a,b,m\in\mathbb{Z}$, $bm\ne0$. $\frac{a}{b} = \frac{a:n}{b:n}$, $\forall a,b\in\mathbb{Z}$, $\forall n\in\mbox{ƯC}(a,b)$. \textit{Phân số tối giản} là phân số mà tử \& mẫu chỉ có ước chung là $\pm1$.'' -- \cite[Chap. V, \S1, p. 29]{SBT_Toan_6_Canh_Dieu_tap_2}

\begin{baitoan}[\cite{SGK_Toan_6_Canh_Dieu_tap_2}, 1., p. 30]
	Viết \& đọc phân số trong mỗi trường hợp sau: (a) Tử số là $-43$, mẫu số là $19$. (b) Tử số là $-123$, mẫu số là $-63$.
\end{baitoan}

\begin{proof}[Giải]
	(a) Viết: $\frac{-43}{19}$, đọc: âm bốn mươi ba phần mười chín. (b) Viết: $\frac{-123}{-63}$, đọc: âm một trăm hai mươi ba phần âm sáu mươi ba. (\textit{Chú ý}: Bài toán này chỉ yêu cầu viết \& đọc, không yêu cầu \textit{chuyển về mẫu dương} hay \textit{rút gọn phân số}).
\end{proof}

\begin{baitoan}[\cite{SGK_Toan_6_Canh_Dieu_tap_2}, 2., p. 30]
	Các cặp phân số sau có bằng nhau không? Vì sao? (a) $\frac{-2}{9},\frac{6}{-27}$.  (b) $\frac{-1}{-5},\frac{4}{25}$.
\end{baitoan}

\begin{proof}[Giải]
	(a) Vì $-2\cdot(-27) = 9\cdot6 = 54$ nên $\frac{-2}{9} = \frac{6}{-27}$. (b) Vì $-1\cdot25\ne-5\cdot4$ (vì $-25\ne-20$) nên $\frac{-1}{-5}\ne\frac{4}{25}$.
\end{proof}

\begin{baitoan}[\cite{SGK_Toan_6_Canh_Dieu_tap_2}, 3., p. 30]
	Tìm $x\in\mathbb{Z}$ biết: (a) $\frac{-28}{35} = \frac{16}{x}$. (b) $\frac{x + 7}{15} = \frac{-24}{36}$.
\end{baitoan}

\begin{proof}[Giải]
	(a)$\frac{-28}{35} = \frac{16}{x}\Leftrightarrow x\ne0$ \& $-28x = 35\cdot16\Leftrightarrow x\ne0$ \& $x = \frac{35\cdot16}{-28} = -20$ (thỏa mãn $x\ne0$ nên nhận). Vậy $x = -20$. (b) $\frac{x + 7}{15} = \frac{-24}{36}\Leftrightarrow36(x + 7) = 15\cdot(-24)\Leftrightarrow x + 7 = \frac{15\cdot(-24)}{36} = -10\Leftrightarrow x = -10 - 7 = -17$. Vậy $x = -17$.
\end{proof}

\begin{luuy}
	Ở (a), vì $x$ nằm ở dưới mẫu nên ta phải kèm theo điều kiện $x\ne0$ để phân số $\frac{16}{x}$ xác định. Sau khi biến đổi tương đương để giải ra $x$, ta phải kiểm tra lại xem $x$ thỏa mãn điều kiện ban đầu để (các) phân số xác định hay không. Điều này quan trọng với các bài toán chứa phân thức có mẫu thức chứa biến số $x$, e.g., xét bài toán sau:
	
	\begin{vidu}
		Tìm $x$ thỏa $\frac{1}{2} = \frac{0}{x}$.
	\end{vidu}
	Nếu không đặt điều kiện $x\ne0$ mà tiến hành giải trực tiếp kiểu: $\frac{1}{2} = \frac{0}{x}\Rightarrow 1x = 0\cdot2 = 0\Rightarrow x = 0$. Nhưng khi thay $x = 0$ vào phương trình ban đầu: $\frac{1}{2} = \frac{0}{0}$, vô nghĩa vì phân số $\frac{0}{0}$ không xác định (hay không có nghĩa, không được định nghĩa), nên bỏ đi điều kiện $x\ne0$ khiến lời giải này sai hoàn toàn. Lời giải đúng sẽ là:
	
	\begin{proof}[Giải]
		$\frac{1}{2} = \frac{0}{x}\Leftrightarrow x\ne0$ \& $1x = 0\cdot2 = 0\Leftrightarrow x\ne0$ \& $x = 0$ (loại vì $x\ne 0$ \& $x = 0$ không thể này xảy ra cùng 1 lúc). Vậy phương trình $\frac{1}{2} = \frac{0}{x}$ vô nghiệm.
	\end{proof}
	Tuy nhiên, ở (b), 2 phân số $\frac{x + 7}{15},\frac{-24}{36}$ trong phương trình có mẫu lần lượt là $15\ne0$, $36\ne0$ đã khác $0$ rồi nên không cần điều kiện xác định như (a) nữa, mà có thể biến đổi tương đương trực tiếp luôn. Đây là điểm khác biệt giữa 2 phương trình. Cẩn thận với các bài toán có phân thức có chứa biến $x$ ở mẫu thức sẽ học kỹ ở chương trình Toán 7 \& 8.\footnote{See, e.g., \href{https://github.com/NQBH/hobby/blob/master/elementary_mathematics/grade_7/algebraic_expression/NQBH_algebraic_expression.pdf}{GitHub\texttt{/}NQBH\texttt{/}elementary mathematics\texttt{/}grade 7\texttt{/}algebraic expressions} (biểu thức đại số) \& \href{https://github.com/NQBH/hobby/blob/master/elementary_mathematics/grade_8/algebraic_rational_fractions/NQBH_algebraic_rational_fractions.pdf}{GitHub\texttt{/}NQBH\texttt{/}elementary mathematics\texttt{/}grade 8\texttt{/}algebraic fractions} (phân thức hữu tỷ).}
\end{luuy}

\begin{baitoan}[\cite{SGK_Toan_6_Canh_Dieu_tap_2}, 4., p. 30]
	Rút gọn mỗi phân số sau về phân số tối giản: $\frac{14}{21},\frac{-36}{48},\frac{28}{-52},\frac{-54}{-90}$.
\end{baitoan}

\begin{proof}[Giải]
	$\mbox{ƯCLN}(14,21) = 7$, $\frac{14}{21} = \frac{14:7}{21:7} = \frac{2}{3}$. $\mbox{ƯCLN}(36,48) = 12$, $\frac{-36}{48} = \frac{-36:12}{48:12} = \frac{-3}{4}$. $\mbox{ƯCLN}(28,52) = 4$, $\frac{28}{-52} = \frac{28:4}{-52:4} = \frac{7}{-13}$. $\mbox{ƯCLN}(54,90) = 18$, $\frac{-54}{-90} = \frac{54}{90} = \frac{54:18}{90:18} = \frac{3}{5}$.
\end{proof}

\begin{baitoan}[\cite{SGK_Toan_6_Canh_Dieu_tap_2}, 5., p. 30]
	(a) Rút gọn phân số $\frac{-21}{39}$ về phân số tối giản. (b) Viết tất cả các phân số bằng $\frac{-21}{39}$ mà mẫu là số tự nhiên có 2 chữ số.
\end{baitoan}

\begin{proof}[Giải]
	(a) $\mbox{ƯCLN}(21,39) = 3$, $\frac{-21}{39} = \frac{-21:3}{39:3} = \frac{-7}{13}$. (b) Theo tính chất cơ bản của phân số $\frac{-7}{13} = \frac{-7n}{13n}$, $\forall n\in\mathbb{Z}$, $n\ne0$. Để $13n$ là số tự nhiên có 2 chữ số thì $n\in\mathbb{Z}$ phải thỏa $10 < 13n < 99$ hay $\frac{10}{13} < n < \frac{99}{13}$ mà $0< \frac{10}{13} < 1$ \& $7 < \frac{99}{13} < 8$, suy ra $n\in\{1,2,3,4,5,6,7\}$. Vậy tập hợp tất cả các phân số bằng $\frac{-21}{39}$ mà mẫu là số tự nhiên có 2 chữ số là: $\left\{\frac{-7n}{13n}|n = 1,2,3,4,5,6,7\right\} = \left\{\frac{-7}{13},\frac{-14}{26},\frac{-21}{39},\frac{-28}{52},\frac{-35}{65},\frac{-42}{78},\frac{-49}{91}\right\}$.
\end{proof}

\begin{baitoan}[\cite{SGK_Toan_6_Canh_Dieu_tap_2}, 6., p. 30]
	Quy đồng mẫu những phân số sau: (a) $\frac{-5}{14},\frac{1}{-21}$. (b) $\frac{17}{60},\frac{-5}{18},\frac{-64}{90}$.
\end{baitoan}

\begin{proof}[Giải]
	(a) $\frac{1}{-21} = \frac{-1}{21}$, $\operatorname{BCNN}(14,21) = 42$, $42:14 = 3$, $42:21 = 2$. Quy đồng: $\frac{-5}{14} = \frac{-5\cdot3}{14\cdot3} = \frac{-15}{42}$, $\frac{1}{-21} = \frac{-1\cdot2}{21\cdot2} = \frac{-2}{42}$. (b) $\operatorname{BCNN}(60,18,90) = 180$, $180:60 = 3$, $180:18 = 10$, $180:90 = 2$. Quy đồng: $\frac{17}{60} = \frac{17\cdot3}{60\cdot3} = \frac{51}{180}$, $\frac{-5}{18} = \frac{-5\cdot10}{18\cdot10} = \frac{-50}{180}$, $\frac{-64}{90} = \frac{-64\cdot2}{90\cdot2} = \frac{-128}{180}$.
\end{proof}

\begin{baitoan}[\cite{SGK_Toan_6_Canh_Dieu_tap_2}, 7., p. 30]
	Trong các phân số sau, tìm phân số không bằng phân số nào trong các phân số còn lại: $\frac{6}{25},\frac{-4}{50},\frac{-27}{54},\frac{-18}{-75},\frac{28}{-56}$.
\end{baitoan}

\begin{proof}[Giải]
	Phân số $\frac{6}{25}$ tối giản. $\mbox{ƯCLN}(4,50) = 2$, $\frac{-4}{50} = \frac{-4:2}{50:2} = \frac{-2}{25}$. $\mbox{ƯCLN}(27,54) = 27$ (vì $54\divby27$), $\frac{-27}{54} = \frac{-27:27}{54:27} = \frac{-1}{2}$. $\mbox{ƯCLN}(18,75) = 3$, $\frac{-18}{-75} = \frac{18}{75} = \frac{18:3}{75:3} = \frac{6}{25}$ bằng phân số đầu tiên. $\mbox{ƯCLN}(28,56) = 28$ (vì $56\divby28$), $\frac{28}{-56} = \frac{-28}{56} = \frac{-28:28}{56:28} = \frac{-1}{2}$ nên bằng với phân số thứ 3: $\frac{28}{-56} = \frac{-27}{54}= \frac{-1}{2}$. Vậy phân số $\frac{-4}{50}$ là phân số không bằng phân số nào trong các phân số còn lại.
\end{proof}

\begin{baitoan}[\cite{SBT_Toan_6_Canh_Dieu_tap_2}, Ví dụ 1, p. 29]
	Viết tất cả các phân số $\frac{a}{b}$ biết $a,b$ được chọn trong các số: $-3,0,5$. Có tất cả bao nhiêu phân số?
\end{baitoan}

\begin{proof}[Giải]
	Vì $b\ne0$ nên có 2 trường hợp: (1) $b = -3$, có 3 phân số: $\frac{-3}{-3},\frac{0}{-3},\frac{5}{-3}$. (2) $b = 5$, có $3$ phân số: $\frac{-3}{5},\frac{0}{5},\frac{5}{5}$. Viết được tất cả $6$ phân số.
\end{proof}

\begin{baitoan}[Mở rộng \cite{SBT_Toan_6_Canh_Dieu_tap_2}, Ví dụ 1, p. 29]
	Viết tất cả các phân số $\frac{a}{b}$ biết $a,b$ được chọn trong các số: $a_1,a_2,\ldots,a_n$, với $n\in\mathbb{N}^\star$, phân biệt cho trước. Có tất cả bao nhiêu phân số?
\end{baitoan}

\begin{proof}[Giải]
	Xét 2 trường hợp sau: (1) Nếu trong $n$ số $a_i$ đã cho có 1 số bằng $0$ (lúc nào cũng chỉ có tối đa 1 số bằng $0$ vì các số này phân biệt), i.e., có 1 chỉ số $i_0\in\{1,2,\ldots,n\}$ sao cho $a_{i_0} = 0$ \& $a_i\ne0$, $\forall i\ne i_0$.  Khi đó, có thể viết được các phân số $\frac{a}{b} = \frac{a_i}{a_j}$, $\forall i = 1,2,\ldots,n$, $\forall j\in\{1,2,\ldots,n\}$, $j\ne i_0$. Có tất cả $n(n - 1)$ phân số trong trường hợp này. (2) Nếu tất cả các số $a_i$ đã cho đều khác $0$, i.e., $\prod_{i=1}^n a_i = a_1a_2\ldots a_n\ne0$ thì có thể viết được các phân số $\frac{a}{b} = \frac{a_i}{a_j}$, $\forall i,j = 1,2,\ldots,n$. Có tất cả $n\cdot n = n^2$ phân số trong trường hợp này.
\end{proof}

\begin{baitoan}[\cite{SBT_Toan_6_Canh_Dieu_tap_2}, Ví dụ 2, p. 29]
	1 trường học có số học sinh giỏi chiếm $\frac{12}{35}$ số học sinh toàn trường, số học sinh khá chiếm $\frac{13}{25}$ số học sinh toàn trường. Số học sinh giỏi \& số học sinh khá của trường đó có bằng nhau không? Vì sao?
\end{baitoan}

\begin{proof}[Giải]
	$12\cdot25\ne35\cdot13\Rightarrow\frac{12}{35}\ne\frac{13}{25}$, nên số học sinh giỏi \& số học sinh khá của trường đó không bằng nhau.
\end{proof}

\begin{luuy}
	Có thể thay $\frac{12}{35},\frac{13}{25}$ trong bài toán trên bằng 2 phân số $\frac{a}{b},\frac{c}{d}$, $a,b,c,d\in\mathbb{Z}$, $bd\ne0$. Theo tính chất của 2 phân số bằng nhau: Nếu $ad = bc$ thì số học sinh giỏi \& số học sinh khá của trường đó bằng nhau. Ngược lại, nếu $ad\ne bc$ thì số học sinh giỏi \& số học sinh khá của trường đó không bằng nhau.
\end{luuy}

\begin{baitoan}[\cite{SBT_Toan_6_Canh_Dieu_tap_2}, Ví dụ 3, p. 30]
	Rút gọn về phân số tối giản: (a) $\frac{3510 - 135}{4680 - 180}$. (b) $\frac{2^4\cdot3^2}{6^2\cdot5}$. (c) $\frac{11\cdot2^n}{6^m}$ với $m,n\in\mathbb{N}$.
\end{baitoan}

\begin{proof}[Giải]
	(a) $\frac{3510 - 135}{4680 - 180} = \frac{3\cdot45\cdot(26 - 1)}{4\cdot45(26 - 1)} = \frac{3}{4}$. (b) $\frac{2^4\cdot3^2}{6^2\cdot5} = \frac{2^4\cdot3^2}{2^2\cdot3^2\cdot5} = \frac{2^2}{5} = \frac{4}{5}$. (c) Nếu $m > n$, $\frac{11\cdot2^n}{6^m} = \frac{11\cdot2^n}{2^m\cdot3^m} = \frac{11}{2^{m-n}\cdot3^n}$. Nếu $m = n$, $\frac{11\cdot2^n}{6^m} = \frac{11\cdot2^n}{2^m\cdot3^m} = \frac{11}{3^n}$. (c) Nếu $m < n$, $\frac{11\cdot2^n}{6^m} = \frac{11\cdot2^n}{2^m\cdot3^m} = \frac{11\cdot2^{n-m}}{3^n}$.
\end{proof}

\begin{baitoan}[\cite{SBT_Toan_6_Canh_Dieu_tap_2}, 3., p. 30]
	Trong các cách viết sau, cách viết nào cho ta phân số? (a) $-\frac{9.4}{11.5}$. (b) $\frac{-8}{0}$. (c) $\frac{7}{1}$. (d) $\frac{n}{2}$, $n\in\mathbb{Z}$.
\end{baitoan}

\begin{baitoan}[\cite{SBT_Toan_6_Canh_Dieu_tap_2}, 4., p. 31]
	Trong các cặp phân số sau, cặp phân số nào bằng nhau? Vì sao? $\frac{3}{7}$ \& $\frac{6}{-14}$, $\frac{12}{-4}$ \& $\frac{-9}{3}$, $\frac{-13}{9}$ \& $\frac{13}{-9}$, $-5$ \& $\frac{-10}{2}$, $\frac{2x}{6}$ \& $\frac{x}{3}$, $x\in\mathbb{Z}$.
\end{baitoan}

\begin{baitoan}[\cite{SBT_Toan_6_Canh_Dieu_tap_2}, 5., p. 31]
	Viết mỗi phân số sau thành phân số bằng nó \& có mẫu là số nguyên dương: (a) $\frac{-32}{-17}$. (b) $\frac{14}{-17}$. (c) $\frac{5}{-39}$. (d) $\frac{-x}{-y}$, $x,y\in\mathbb{Z}$, $y > 0$.
\end{baitoan}

\begin{baitoan}[\cite{SBT_Toan_6_Canh_Dieu_tap_2}, 6., p. 31]
	Tìm $x,y\in\mathbb{Z}$: (a) $\frac{4}{x} = \frac{y}{21} = \frac{28}{49}$. (b) $\frac{x}{7} = \frac{9}{y}$ \& $x > y$. (c) $\frac{x}{15} = \frac{3}{y}$ \& $x < y < 0$. (d) $\frac{x}{y} = \frac{21}{28}$.
\end{baitoan}

\begin{baitoan}[\cite{SBT_Toan_6_Canh_Dieu_tap_2}, 7., p. 31]
	Rút gọn về phân số tối giản: (a) $\frac{-147}{252}$. (b) $\frac{765}{900}$. (c) $\frac{11\cdot3 - 11\cdot8}{17 - 6}$. (d) $\frac{3^5\cdot2^4}{8\cdot3^6}$. (e) $\frac{84\cdot45}{49\cdot54}$.
\end{baitoan}

\begin{baitoan}[\cite{SBT_Toan_6_Canh_Dieu_tap_2}, 8., p. 31]
	Giải thích tại sao các phân số sau đây bằng nhau: (a) $\frac{-630}{224} = \frac{-45}{16}$. (b) $\frac{352352}{-470470} = \frac{-176}{235}$. (c) $\frac{199\cdots9}{99\cdots95} = \frac{1}{5}$ biết có $100$ chữ số $9$ ở tử số \& $100$ chữ số $9$ ở mẫu số.
\end{baitoan}

\begin{baitoan}[\cite{SBT_Toan_6_Canh_Dieu_tap_2}, 9., p. 31]
	Cho biểu thức $A = \frac{3}{n + 2}$. (a) Số nguyên $n$ phải thỏa mãn điều kiện gì để $A$ là phân số? (b) Tìm phân số $A$ khi $n = 0$, $n = 2$, $n = -7$. (c) Tìm các số nguyên $n$ để $A$ là 1 số nguyên.
\end{baitoan}

\begin{baitoan}[\cite{SBT_Toan_6_Canh_Dieu_tap_2}, 10., p. 31]
	Cho phân số $A = \frac{1 + 2 + \cdots + 9}{11 + 12 + \cdots + 19}$. (a) Rút gọn $A$. (b) Xóa 1 số hạng ở tử \& xóa 1 số hạng ở mẫu của phân số $A$ để được phân số mới có giá trị vẫn bằng $A$.
\end{baitoan}

\begin{baitoan}[\cite{SBT_Toan_6_Canh_Dieu_tap_2}, 11., pp. 31--32]
	(a) 1 mẫu Bắc Bộ bằng $\rm3600m^2$. 1 mẫu Bắc Bộ bằng bao nhiêu phần của 1 hecta? (b) 1 pound bằng $0.45$\emph{kg}. 1 pound bằng bao nhiêu phần của $1$\emph{kg}? (c) 1 vòi nước chảy vào bể không có nước trong $48$ phút thì đầy bể. Nếu mở vòi vào bể không có nước trong $36$ phút thì lượng nước chiếm bao nhiêu phần bể?
\end{baitoan}

\begin{baitoan}[\cite{SBT_Toan_6_Canh_Dieu_tap_2}, 13., p. 32]
	Cho phân số $\frac{-5}{9}$. Phải cộng thêm vào tử \& mẫu cùng 1 số nào để được phân số mới có giá trị bằng phân số $\frac{2}{9}$?
\end{baitoan}

\begin{baitoan}[\cite{SBT_Toan_6_Canh_Dieu_tap_2}, 14., p. 32]
	Chứng minh $\frac{14n + 3}{21n + 4}$ là phân số  tối giản với mọi số tự nhiên $n$
\end{baitoan}
\noindent``\fbox{\bf 1} Ta gọi $\frac{a}{b}$ với $a,b\in\mathbb{Z}$, $b\ne0$ là 1 \textit{phân số}, $a$ là \textit{tử}, $b$ là \textit{mẫu} của phân số. Ta có thể viết thương của phép chia $a\in\mathbb{Z}$ cho $b\in\mathbb{Z}$, $b\ne 0$ dưới dạng $\frac{a}{b}$ \& cũng gọi $\frac{a}{b}$ là phân số. $a\in\mathbb{Z}$ có thể viết dưới dạng phân số $\frac{a}{1}$. \fbox{\bf 2} \textit{2 phân số bằng nhau.} Cho $a,b,c,d\in\mathbb{Z}$, $b\ne0$, $d\ne 0$. Nếu $ad = bc$ thì $\frac{a}{b} = \frac{c}{d}$, ngược lại nếu $\frac{a}{b} = \frac{c}{d}$ thì $ad = bc$. \fbox{\bf 3} \textit{2 tính chất cơ bản của phân số}: $\frac{a}{b} = \frac{am}{bm}$, $\forall a,b,m\in\mathbb{Z}$, $b\ne0$, $m\ne0$. $\frac{a}{b} = \frac{a:n}{b:n}$, $\forall a,b,n\in\mathbb{Z}$, $b\ne0$, $n\in\mbox{ƯC}(a,b)$. \fbox{\bf 4} \textit{Rút gọn phân số}: Muốn rút gọn 1 phân số, ta chia cả tử \& mẫu của phân số đó cho 1 ước chung khác $\pm1$ của chúng. Phân số tối giản là phân số mà tử \& mẫu chỉ có ước chung là $\pm1$, i.e., $\frac{a}{b}$, $a,b\in\mathbb{Z}$, $b\ne0$, $\mbox{ƯCLN}(a,b) = 1$. \fbox{\bf 5} Nếu đổi dấu cả tử \& mẫu của 1 phân số thì được 1 phân số mới bằng phân số đã cho. $\frac{a}{b} = \frac{-a}{-b}$, $\frac{-a}{b} = \frac{a}{-b}$, $\forall a,b\in\mathbb{Z}$, $b\ne0$. \fbox{\bf 6} Nếu $\frac{a}{b}$ là phân số tối giản thì mọi phân số bằng nó đều có dạng $\frac{am}{bm}$ với $m\in\mathbb{Z}$ \& $m\ne0$.'' -- \cite[Chap. 3, \S1, p. 45]{Tuyen_Toan_6}

``Số có dạng $\frac{a}{b}$ trong đó $a,b\in\mathbb{Z}$, $b\ne0$ được gọi là \textit{phân số}. Số nguyên $n\in\mathbb{Z}$ được đồng nhất với phân số $\frac{n}{1}$. Tính chất cơ bản của phân số: $\frac{a}{b} = \frac{am}{bm} = \frac{a:n}{b:n}$ với $m\in\mathbb{Z}$, $m\ne0$, $n\in\mbox{ƯC}(a,b)$. Nếu $\mbox{ƯCLN}(|a|,|b|) = 1$ thì $\frac{a}{b}$ là phân số tối giản. Nếu $\frac{m}{n}$ là dạng tối giản của phân số $\frac{a}{b}$ thì tồn tại số nguyên $k\in\mathbb{Z}$ sao cho $a = mk$, $b = nk$.'' -- \cite[Chap. III, \S1, p. 4]{Binh_Toan_6_tap_2}

\begin{baitoan}[\cite{Tuyen_Toan_6}, Ví dụ 49, p. 45]
	Cho $A = \{-5,0,9\}$. Viết tất cả các phân số $\frac{a}{b}$ với $a,b\in A$. Có bao nhiêu phân số thỏa mãn?
\end{baitoan}

\begin{proof}[Giải]
	Số $0$ không thể lấy làm mẫu của phân số. Lấy $-5$ làm mẫu: $\frac{-5}{-5},\frac{0}{-5},\frac{9}{-5}$. Lấy $9$ làm mẫu: $\frac{-5}{9},\frac{0}{9},\frac{9}{9}$. Có $6$ phân số thỏa mãn.
\end{proof}

\begin{baitoan}[Mở rộng \cite{Tuyen_Toan_6}, Ví dụ 49, p. 45]
	Cho $A = \{a_1,a_2,\ldots,a_n\}\subset\mathbb{Z}$. Viết tất cả các phân số $\frac{a}{b}$ với $a,b\in A$. Có bao nhiêu phân số thỏa mãn?
\end{baitoan}

\begin{proof}[Giải]
	Xét 2 trường hợp: (a) Nếu $0\notin A$, i.e., $a_i\ne0$, $\forall i = 1,\ldots,n$. Tất cả các phân số $\frac{a}{b}$ với $a,b\in A$: $\frac{a_i}{a_j}$, $\forall i,j = 1,\ldots,n$, có tổng cộng $n^2$ phân số thỏa mãn. (b) Nếu $0\in A$, i.e., tồn tại chỉ số $k\in\{1,\ldots,n\}$ sao cho $a_k = 0$, ngoài ra $a_i\ne 0$, $\forall i = 1,\ldots,n$, $i\ne k$ (vì $A$ là 1 tập hợp nên không có các phần tử trùng nhau). Tất cả các phân số $\frac{a}{b}$ với $a,b\in A$: $\frac{a_i}{a_j}$, $\forall i,j = 1,\ldots,n$, $j\ne k$ có tổng cộng $n(n - 1) = n^2 - n$ phân số thỏa mãn.
\end{proof}

\begin{nhanxet}
	``Mẫu của 1 phân số phải khác $0$ nhưng tử của phân số có thể bằng $0$, khi đó giá trị của phân số đúng bằng $0$, i.e., $\frac{0}{b} = 0$, $\forall b\in\mathbb{Z}$, $b\ne 0$. Tử \& mẫu của 1 phân số có thể bằng nhau, khi đó giá trị của phân số đúng bằng $1$, i.e., $\frac{a}{a} = 1$, $\forall a\in\mathbb{Z}$, $a\ne 0$.'' -- \cite[p. 46]{Tuyen_Toan_6}
\end{nhanxet}

\begin{baitoan}[\cite{Tuyen_Toan_6}, Ví dụ 50, p. 46]
	Viết tập hợp $B$ các phân số bằng phân số $\frac{7}{-15}$ với mẫu dương có 2 chữ số.
\end{baitoan}

\begin{proof}[Giải]
	$\frac{7}{-15} = \frac{-7}{15}$. Phân số này là 1 phân số tối giản với mẫu dương. Mọi phân số bằng nó đều có dạng $\frac{-7m}{15m}$ với $m\in\mathbb{Z}$, $m\ne0$. Mẫu số của các phân số cần phải tìm là 1 số có 2 chữ số nên chọn $m\in\mathbb{Z}$ sao cho $10\le15m\le 99$, suy ra\footnote{$m\in\mathbb{Z}\land(10\le15m\le 99)\Leftrightarrow\lfloor\frac{15}{10}\rfloor = 1\le m\le\lfloor\frac{99}{15}\rfloor = 6$.} $1\le m\le6$, i.e., $m\in\{1,2,3,4,5,6\}$. Vậy $B = \left\{\frac{-7}{15},\frac{-14}{30},\frac{-21}{45},\frac{-28}{60},\frac{-35}{75},\frac{-42}{90}\right\}$.
\end{proof}

\begin{baitoan}[Mở rộng \cite{Tuyen_Toan_6}, Ví dụ 50, p. 46]
	Cho trước $a,b\in\mathbb{Z}$, $b\ne0$, \& $n\in\mathbb{N}^\star$. Viết tập hợp $B$ các phân số bằng phân số $\frac{a}{b}$ với mẫu dương có $n$ chữ số.
\end{baitoan}

\begin{baitoan}[\cite{Tuyen_Toan_6}, Ví dụ 51, p. 46]
	Tìm phân số bằng phân số $\frac{32}{60}$, biết tổng của tử \& mẫu là $115$.
\end{baitoan}	

\begin{proof}[Giải]
	Có $\frac{32}{60} = \frac{8}{15} = \frac{8m}{15m}$, $\forall m\in\mathbb{Z}$, $m\ne0$. Tổng của tử \& mẫu là $115\Rightarrow8m + 15m = 115\Rightarrow23m = 115\Rightarrow m =\frac{115}{23} = 5$. Phân số cần tìm: $\frac{8\cdot5}{15\cdot5} = \frac{40}{75}$.
\end{proof}

\begin{nhanxet}
	``Nếu không rút gọn phân số $\frac{32}{60}$ thành phân số tối giản $\frac{8}{15}$ mà khẳng định các phân số bằng phân số $\frac{32}{60}$ có dạng $\frac{32m}{60m}$ thì sẽ mắc sai lầm là bỏ sót rất nhiều phân số bằng phân số $\frac{32}{60}$ do đó không thể tìm được đáp số của bài toán trên.'' -- \cite[p. 46]{Tuyen_Toan_6}
\end{nhanxet}

\begin{baitoan}[Mở rộng \cite{Tuyen_Toan_6}, Ví dụ 51, p. 46]
	Cho trước $a,b,n\in\mathbb{Z}$, $b\ne0$. Tìm phân số bằng phân số $\frac{a}{b}$, biết tổng của tử \& mẫu là $n$.
\end{baitoan}

\begin{baitoan}[\cite{Tuyen_Toan_6}, 236., p. 47]
	Trong các phân số sau, những phân số nào bằng nhau? $\frac{15}{60},\frac{-7}{5},\frac{6}{15},\frac{28}{-20},\frac{3}{12}$.
\end{baitoan}

\begin{baitoan}[\cite{Tuyen_Toan_6}, 237., p. 47]
	Cho $A = \frac{3n - 5}{n + 4}$. Tìm $n\in\mathbb{Z}$ để $A\in\mathbb{Z}$.
\end{baitoan}

\begin{baitoan}[\cite{Tuyen_Toan_6}, 238., p. 47]
	Tìm $n\in\mathbb{Z}$ để cho các phân số sau đồng thời có giá trị nguyên: $\frac{-12}{n},\frac{15}{n - 2},\frac{8}{n + 1}$.
\end{baitoan}

\begin{baitoan}[\cite{Tuyen_Toan_6}, 239., p. 47]
	Tìm $x\in\mathbb{Z}$ biết: (a) $\frac{x - 1}{9} = \frac{8}{3}$. (b) $\frac{-x}{4} = \frac{-9}{x}$. (c) $\frac{x}{4} = \frac{18}{x + 1}$.
\end{baitoan}

\begin{baitoan}[\cite{Tuyen_Toan_6}, 240., p. 47]
	Tìm $x,y\in\mathbb{Z}$ thỏa $\frac{x - 4}{y - 3} = \frac{4}{3}$ \& $x - y = 5$.
\end{baitoan}

\begin{baitoan}[\cite{Tuyen_Toan_6}, 241., p. 47]
	Viết dạng tổng quát các phân số bằng phân số $\frac{-12}{30}$.
\end{baitoan}

\begin{baitoan}[\cite{Tuyen_Toan_6}, 242., p. 47]
	Rút gọn phân số: (a) $\frac{990}{2610}$. (b) $\frac{374}{506}$. (c) $\frac{3600 - 75}{8400 - 175}$. (d) $\dfrac{9^{14}\cdot25^5\cdot8^7}{18^{12}\cdot625^3\cdot24^3}$.
\end{baitoan}

\begin{baitoan}[\cite{Tuyen_Toan_6}, 243., p. 47]
	Cho phân số $\frac{a}{b}$. Chứng minh: Nếu $\frac{a - x}{b - y} = \frac{a}{b}$ thì $\frac{x}{y} = \frac{a}{b}$.
\end{baitoan}

\begin{baitoan}[\cite{Tuyen_Toan_6}, 244., p. 47]
	Cho phân số $A = \dfrac{1 + 3 + 5 + \cdots + 19}{21 + 23 + 25 + \cdots + 39}$. (a) Rút gọn $A$. (b) Xóa 1 số hạng ở tử \& xóa 1 số hạng ở mẫu để được 1 phân số mới vẫn bằng $A$.
\end{baitoan}

\begin{baitoan}[\cite{Tuyen_Toan_6}, 245., p. 47]
	Rút gọn phân số $A = \frac{71\cdot52 + 53}{530\cdot71 - 180}$ mà không cần thực hiện các phép tính ở tử.
\end{baitoan}

\begin{baitoan}[\cite{Tuyen_Toan_6}, 246., p. 47]
	2 phân số sau có bằng nhau không? $\dfrac{\overline{abab}}{\overline{cdcd}},\dfrac{\overline{ababab}}{\overline{cdcdcd}}$.
\end{baitoan}

\begin{baitoan}[\cite{Tuyen_Toan_6}, 247., p. 47]
	Chứng minh: (a) $\dfrac{1\cdot3\cdot5\cdots39}{21\cdot22\cdot23\cdots40} = \dfrac{1}{2^{20}}$. (b) $\dfrac{1\cdot3\cdot5\cdots(2n - 1)}{(n + 1)(n + 2)(n + 3)\cdots2n} = \dfrac{1}{2^n}$ với $n\in\mathbb{N}^\star$.
\end{baitoan}

\begin{baitoan}[\cite{Tuyen_Toan_6}, 248., p. 47]
	Tìm phân số $\frac{a}{b}$ bằng phân số $\frac{60}{108}$ biết: (a) $\mbox{\rm ƯCLN}(a,b) = 15$. (b) $\operatorname{BCNN}(a,b) = 180$.
\end{baitoan}

\begin{baitoan}[\cite{Tuyen_Toan_6}, 249., p. 48]
	Tìm phân số bằng phân số $\frac{200}{520}$ sao cho: (a) Tổng của tử \& mẫu là $306$. (b) Hiệu của tử \& mẫu là $184$. (c) Tích của tử \& mẫu là $2340$.
\end{baitoan}

\begin{baitoan}[\cite{Tuyen_Toan_6}, 250., p. 48]
	Chứng minh: $\forall n\in\mathbb{Z}$, các phân số sau là các phân số tối giản: (a) $\frac{3n - 2}{4n - 3}$. (b) $\frac{4n + 1}{6n + 1}$.
\end{baitoan}

\begin{baitoan}[\cite{Tuyen_Toan_6}, 251., p. 48]
	Cho $\frac{a}{b}$ là 1 phân số chưa tối giản. Chứng minh các phân số sau chưa tối giản: (a) $\frac{a}{a - b}$. (b) $\frac{2a}{a - 2b}$.
\end{baitoan}

\begin{baitoan}[\cite{Tuyen_Toan_6}, 252., p. 48]
	1 mẫu Bắc Bộ bằng $\rm3600m^2$. Hỏi 1 mẫu Bắc Bộ bằng mấy phần của 1 hecta?
\end{baitoan}

\begin{baitoan}[\cite{Binh_Toan_6_tap_2}, Ví dụ 1, p. 4]
	Tìm $n\in\mathbb{N}$ để phân số $A = \frac{n + 10}{2n - 8}\in\mathbb{Z}$ (i.e., có giá trị là 1 số nguyên).
\end{baitoan}

\begin{proof}[Giải]
	Để phân số $A$ có giá trị là 1 số nguyên, tử phải chi hết cho mẫu: $n + 10\divby2n - 8\Rightarrow n + 10\divby n - 4\Rightarrow n - 4 + 14\divby n - 4\Rightarrow14\divby n - 4\Rightarrow n - 4\in\mbox{Ư}(14)\cap\mathbb{Z} = \{\pm1,\pm2,\pm7,\pm14\}$. Vì $n - 4\ge-4$ (vì $n\in\mathbb{N}$, $n\ge 0$) nên $n - 4\in\{\pm1,\pm2,7,14\}$. Nếu $n - 4 = 1$, $n = 5$, $A = \frac{15}{2}$ (loại). Nếu $n - 4 = -1$, $n = 3$, $A = \frac{13}{-2}$ (loại). Nếu $n - 4 = 2$, $n = 6$, $A = \frac{16}{4} = 4$. Nếu $n - 4 = -2$, $n = 2$, $A = \frac{12}{-4} = -3$. Nếu $n - 4 = 7$, $n = 11$, $A = \frac{21}{14} = \frac{3}{2}$ (loại). Nếu $n - 4 = 14$, $n = 18$, $A = \frac{28}{28} = 1$. Vậy $n\in\{2,6,18\}$.
\end{proof}

\begin{baitoan}[Mở rộng \cite{Binh_Toan_6_tap_2}, Ví dụ 1, p. 4]
	Cho $a,b,c,d\in\mathbb{Z}$, $c^2 + d^2\ne0$. Tìm $n\in\mathbb{N}$ để phân số $A = \frac{an + b}{cn + d}\in\mathbb{Z}$.
\end{baitoan}

\begin{baitoan}[\cite{Binh_Toan_6_tap_2}, Ví dụ 2, p. 5]
	Tìm $n\in\mathbb{N}$ để phân số $A = \frac{21n + 3}{6n + 4}$ rút gọn được.
\end{baitoan}

\begin{baitoan}[Mở rộng \cite{Binh_Toan_6_tap_2}, Ví dụ 2, p. 5]
	Cho $a,b,c,d\in\mathbb{Z}$, $c^2 + d^2\ne0$. Tìm $n\in\mathbb{N}$ để phân số $A = \frac{an + b}{cn + d}$ rút gọn được.
\end{baitoan}

\begin{baitoan}[\cite{Binh_Toan_6_tap_2}, Ví dụ 3, p. 5]
	Tìm $a,b,c,d\in\mathbb{N}$ nhỏ nhất sao cho $\frac{a}{b} = \frac{3}{5}$, $\frac{b}{c} = \frac{12}{21}$, $\frac{c}{d} = \frac{6}{11}$.
\end{baitoan}

\begin{baitoan}[\cite{Binh_Toan_6_tap_2}, Ví dụ 4, p. 5]
	Tìm số tự nhiên lớn nhất có 3 chữ số sao cho số đó bằng mỗi tổng $a + b,c + d,e + f$ \& $\frac{a}{b} = \frac{35}{49},\frac{c}{d} = \frac{130}{143},\frac{e}{f} = \frac{7}{13}$.
\end{baitoan}

\begin{baitoan}[\cite{Binh_Toan_6_tap_2}, 1., p. 6]
	Rút gọn phân số: (a) $\frac{199\ldots9}{99\ldots95}$ ($10$ chữ số $9$ ở tử, $10$ chữ số $9$ ở mẫu). (b) $\frac{121212}{424242}$. (c) $\frac{187187187}{221221221}$. (d) $\frac{3\cdot7\cdot13\cdot37\cdot39 - 10101}{505050 + 70707}$.
\end{baitoan}

\begin{baitoan}[\cite{Binh_Toan_6_tap_2}, 2., p. 6]
	Chứng minh các phân số sau có giá trị là số tự nhiên: (a) $\frac{10^{2002} + 2}{3}$. (b) $\frac{10^{2003} + 8}{9}$.
\end{baitoan}

\begin{baitoan}[\cite{Binh_Toan_6_tap_2}, 3., p. 6]
	Chứng mih các phân số sau bằng nhau: (a) $\frac{1717}{2929}$ \& $\frac{171717}{292929}$. (b) $\frac{3210 - 34}{4170 - 41}$ \& $\frac{6420 - 68}{8340 - 82}$. (c) $\frac{2106}{7320}$, $\frac{4212}{14640}$, \& $\frac{6318}{21960}$.
\end{baitoan}

\begin{baitoan}[\cite{Binh_Toan_6_tap_2}, 4., p. 6]
	Tìm $x,y\in\mathbb{Z}$ thỏa: (a) $\frac{x}{3} = \frac{y}{5}$. (b) $\frac{x}{28} = \frac{y}{35}$.
\end{baitoan}

\begin{baitoan}[\cite{Binh_Toan_6_tap_2}, 5., p. 6]
	Tìm các phân số $\frac{a}{b}$, $a\in\mathbb{N}$, $b\in\mathbb{N}^\star$, có giá trị bằng: (a) $\frac{36}{45}$ biết $\mbox{\rm BCNN}(a,b) = 300$. (b) $\frac{21}{35}$ biết $\mbox{\rm ƯCLN}(a,b) = 30$. (c) $\frac{15}{35}$ biết $\mbox{\rm ƯCLN}(a,b)\cdot\mbox{\rm BCNN}(a,b) = 3549$.
\end{baitoan}

\begin{baitoan}[\cite{Binh_Toan_6_tap_2}, 6., p. 7]
	Chứng minh các phân số sau tối giản với mọi $n\in\mathbb{N}$. (a) $\frac{n + 1}{2n + 3}$. (b) $\frac{2n + 3}{4n + 8}$. (c) $\frac{3n + 2}{5n + 3}$.
\end{baitoan}

\begin{baitoan}[\cite{Binh_Toan_6_tap_2}, 7., p. 7]
	Cho phân số $A = \frac{63}{3n + 1}$, $n\in\mathbb{N}$. (a) Với giá trị nào của $n$ thì $A$ rút gọn được? (b) Với giá trị nào của $n$ thì $A\in\mathbb{N}$?
\end{baitoan}

\begin{baitoan}[\cite{Binh_Toan_6_tap_2}, 8., p. 7]
	Tìm các số tự nhiên $n$ để các phân số sau là phân số tối giản: (a) $\frac{2n + 3}{4n + 1}$. (b) $\frac{3n + 2}{7n + 1}$. (c) $\frac{2n + 7}{5n + 2}$.
\end{baitoan}

\begin{baitoan}[\cite{Binh_Toan_6_tap_2}, 9., p. 7]
	Có bao nhiêu số nguyên dương $n$ không vượt quá $1000$ để phân số $\frac{n + 12}{n^2 + 9n - 13}$ là phân số tối giản?
\end{baitoan}

\begin{baitoan}[\cite{Binh_Toan_6_tap_2}, 10., p. 7]
	Tìm $n\in\mathbb{N}$ để phân số $\frac{n + 3}{2n - 2}\in\mathbb{Z}$.
\end{baitoan}

\begin{baitoan}[\cite{Binh_Toan_6_tap_2}, 11., p. 7]
	Tìm các số nguyên $n$ sao cho các phân số sau có giá trị là số nguyên: (a) $\frac{12}{3n - 1}$. (b) $\frac{2n + 3}{7}$.
\end{baitoan}

\begin{baitoan}[\cite{Binh_Toan_6_tap_2}, 12., p. 7]
	Tìm $n\in\mathbb{N}$ để phân số $A = \frac{8n + 193}{4n + 3}$: (a) Có giá trị là số tự nhiên. (b) Là phân số tối giản. (c) Với giá trị nào của $n$ trong khoảng từ $150$ đến $170$ thì phân số $A$ rút gọn được?
\end{baitoan}

\begin{baitoan}[\cite{Binh_Toan_6_tap_2}, 13., p. 7]
	Tìm các phân số tối giản nhỏ hơn $1$ có tử \& mẫu đều dương, biết tích của tử \& mẫu của phân số bằng $120$.
\end{baitoan}

\begin{baitoan}[\cite{Binh_Toan_6_tap_2}, 14., p. 7]
	Tìm $n\in\mathbb{N}$ nhỏ nhất để các phân số sau đều là phân số tối giản: $\frac{5}{n + 8},\frac{6}{n + 9},\frac{7}{n + 10},\ldots,\frac{17}{n + 20}$.
\end{baitoan}

\begin{baitoan}[\cite{Binh_Toan_6_tap_2}, 15., p. 7]
	Cho 3 phân số $\frac{15}{42},\frac{49}{56},\frac{36}{51}$. Biến đổi 3 phân số trên thành các phân số bằng chúng sao cho mẫu của phân số thứ nhất bằng tử của phân số thứ 2, mẫu của phân số thứ 2 bằng tử của phân số thứ 3.
\end{baitoan}

\begin{baitoan}[\cite{Binh_Toan_6_tap_2}, 16., p. 7]
	Cho 3 phân số $\frac{5}{8},\frac{11}{20},\frac{4}{15}$. Tìm 3 phân số (có tử \& mẫu dương) theo thứ tự bằng 3 phân số trên sao cho hiệu của mẫu \& tử của mỗi phân số này đều bằng nhau \& hiệu đó có giá trị nhỏ nhất.
\end{baitoan}

\begin{baitoan}[\cite{Binh_Toan_6_tap_2}, 17., p. 8]
	Tìm các phân số lớn hơn $\frac{1}{5}$ \& khác số tự nhiên biết nếu lấy mẫu nhân với 1 số, lấy tử cộng với số đó thì giá trị của phân số không đổi.
\end{baitoan}

\begin{baitoan}[\cite{Binh_Toan_6_tap_2}, 18., p. 8]
	Cho phân số $A = \frac{23 + 22 + 21 + \cdots + 13}{11 + 10 + 9 + \cdots + 1}$. Nêu cách xóa 1 số hạng ở tử \& 1 số hạng ở mẫu của $A$ để được 1 phân số mới vẫn bằng phân số $A$.
\end{baitoan}

\begin{baitoan}[\cite{Binh_Toan_6_tap_2}, 19., p. 8, Bộ sử Hume]
	Người Anh có thói quen xếp bộ sử nước Anh của Hume (David Hume, nhà sử học Scotland) gồm 9 tập ở tủ sách đặc biệt gồm 2 ngăn: ngăn trên xếp 5 cuốn, ngăn dưới xếp 4 cuốn, ở gáy các cuốn sách đó ghi các số $1,2,3,\ldots,9$. Nếu chủ nhân xếp $\frac{13458}{6729}$ (phân số này có giá trị bằng $2$) thì chứng tỏ chủ nhân đã đọc 2 tập (riêng trường hợp mới đọc 1 tập thì xếp $\frac{12345}{6789}$). Nêu cách xếp 9 cuốn sách đó để chứng tỏ chủ nhân của bộ sách đã đọc $3,4,5,6,7,8,9$ tập.
\end{baitoan}

%------------------------------------------------------------------------------%

\section{So Sánh Các Phân Số}
Trong 2 số nguyên $a,b\in\mathbb{Z}$ khác nhau ($a\ne b$), luôn có 1 số nhỏ hơn số kia, i.e., $a < b$ hoặc $a > b$. Cũng như số nguyên, trong 2 phân số $\frac{a}{b},\frac{c}{d}$ khác nhau ($\frac{a}{b}\ne\frac{c}{d}$) luôn có 1 phân số nhỏ hơn phân số kia, i.e., $\frac{a}{b} < \frac{c}{d}$ hoặc $\frac{a}{b} > \frac{c}{d}$. Nếu phân số $\frac{a}{b}$ nhỏ hơn phân số $\frac{c}{d}$ thì ta viết $\frac{a}{b} < \frac{c}{d}$ hoặc $\frac{c}{d} > \frac{a}{b}$. Phân số lớn hơn $0$ gọi là \textit{phân số dương}. Phân số nhỏ hơn $0$ gọi là \textit{phân số âm}. \textit{Tính chất bắc cầu}: Nếu $\frac{a}{b} < \frac{c}{d}$ \& $\frac{c}{d} < \frac{e}{f}$ thì $\frac{a}{b} < \frac{e}{f}$.
\begin{align*}
	\left(\frac{a}{b} < \frac{c}{d}\right)\land\left(\frac{c}{d} < \frac{e}{f}\right)\Rightarrow\frac{a}{b} < \frac{e}{f},\ \forall a,b,c,d,e,f\in\mathbb{Z},\,bdf\ne0.
\end{align*}
\textbf{Cách so sánh 2 phân số.} \textit{Bước 1}: Quy đồng mẫu 2 phân số đã cho (về cùng 1 mẫu dương). \textit{Bước 2}: So sánh tử của các phân số: Phân số nào có tử lớn hơn thì lớn hơn.

\begin{menhde}[So sánh các phân số]
	Để so sánh 2 phân số không cùng mẫu, ta quy đồng mẫu 2 phân số đó (về cùng 1 mẫu dương) rồi so sánh các tử với nhau: Phân số nào có tử lớn hơn thì phân số đó lớn hơn.
\end{menhde}

\begin{baitoan}[\cite{SGK_Toan_6_Canh_Dieu_tap_2}, 2, p. 31]
	So sánh: $\frac{3}{-5}$ \& $\frac{-5}{9}$.
\end{baitoan}

\begin{proof}[Giải]
	$\frac{3}{-5} = \frac{-3}{5}$, $\operatorname{BCNN}(5,9) = 45$, $45:5 = 9$, $45:9 = 5$. Quy đồng: $\frac{3}{-5} =\frac{-3}{5} = \frac{-3\cdot9}{5\cdot9} = \frac{-27}{45}$ \& $\frac{-5}{9} = \frac{-5\cdot5}{9\cdot5} = \frac{-25}{45}$. Vì $-27 < -25$ nên $\frac{-27}{45} < \frac{-25}{45}$. Vậy $\frac{3}{-5} < \frac{-5}{9}$.
\end{proof}

\begin{baitoan}[\cite{SGK_Toan_6_Canh_Dieu_tap_2}, Ví dụ 1, 1, p. 32]
	 So sánh: (a) $\frac{5}{-9}$ \& $\frac{2}{-9}$. (b) $\frac{5}{-6}$ \& $\frac{-6}{7}$. (c) $\frac{7}{-11}$ \& $\frac{8}{-11}$. (d) $\frac{-5}{3}$ \& $\frac{5}{-4}$.
\end{baitoan}

\begin{proof}[Giải]
	(a) $\frac{5}{-9} = \frac{-5}{9}$, $\frac{2}{-9} = \frac{-2}{9}$. Vì $-5 < -2$ nên $\frac{-5}{9} < \frac{-2}{9}$. Vậy $\frac{5}{-9} < \frac{2}{-9}$. (b) $\frac{5}{-6} = \frac{-5}{6} = \frac{-5\cdot7}{6\cdot7} = \frac{-35}{42}$, $\frac{-6}{7} = \frac{-6\cdot6}{7\cdot6} = \frac{-36}{42}$. Vì $-35 > -36$ nên $\frac{-35}{42} > \frac{-36}{42}$. Vậy $\frac{5}{-6} > \frac{-6}{7}$. (c) $\frac{7}{-11} = \frac{-7}{11}$, $\frac{8}{-11} = \frac{-8}{11}$. Vì $-7 > -8$ nên $\frac{-7}{11} > \frac{-8}{11}$. Vậy $\frac{7}{-11} > \frac{8}{-11}$. (d) $\frac{-5}{3} = \frac{-5\cdot4}{3\cdot4} = \frac{-20}{12}$, $\frac{5}{-4} = \frac{-5}{4} = \frac{-5\cdot3}{4\cdot3} = \frac{-15}{12}$. Vì $-20 < -15$ nên $\frac{-20}{12} < \frac{-15}{12}$. Vậy $\frac{-5}{3} < \frac{5}{-4}$.
\end{proof}

\begin{baitoan}[\cite{SGK_Toan_6_Canh_Dieu_tap_2}, 1., p. 33]
	So sánh: (a) $\frac{-9}{4},\frac{1}{3}$. (b) $\frac{-8}{3},\frac{4}{-7}$. (c) $\frac{9}{-5},\frac{7}{-10}$.
\end{baitoan}

\begin{proof}[Giải]
	(a) $\frac{-9}{4} = \frac{-9\cdot3}{4\cdot3} = \frac{-27}{12}$, $\frac{1}{3} = \frac{1\cdot4}{3\cdot4} = \frac{4}{12}$. Vì $-27 < 4$ nên $\frac{-27}{12} < \frac{4}{12}$. Vậy $\frac{-9}{4} < \frac{1}{3}$. (b) $\frac{-8}{3} = \frac{-8\cdot7}{3\cdot7} = \frac{-56}{21}$, $\frac{4}{-7} = \frac{-4}{7} = \frac{-4\cdot3}{7\cdot3} = \frac{-12}{21}$. Vì $-56 < -21$ nên $\frac{-56}{21} < \frac{-12}{21}$. Vậy $\frac{-8}{3} < \frac{4}{-7}$. (c) $\frac{9}{-5} = \frac{-9}{5} = \frac{-9\cdot2}{5\cdot2} = \frac{-18}{10}$, $\frac{7}{-10} = \frac{-7}{10}$. Vì $-18 < -7$ nên $\frac{-18}{10} < \frac{-7}{10}$. Vậy $\frac{9}{-5} < \frac{7}{-10}$.
\end{proof}

\begin{luuy}[Số âm $< 0 <$ số dương]
	Có thể giải (a) ngắn hơn như sau: Có $\frac{-9}{4} < 0$, $0 < \frac{1}{3}$. Áp dụng tính chất bắc cầu của phép so sánh phân số, suy ra $\frac{-9}{4} < \frac{1}{3}$.
\end{luuy}

\begin{baitoan}[\cite{SGK_Toan_6_Canh_Dieu_tap_2}, 2., p. 33]
	Viết các phân số sau theo thứ tự tăng dần: (a) $\frac{2}{5},\frac{-1}{2},\frac{2}{7}$. (b) $\frac{12}{5},\frac{-7}{3},\frac{-11}{4}$.
\end{baitoan}

\begin{proof}[Giải]
	(a) $\operatorname{BCNN}(5,2,7) = 70$, $70:5 = 14$, $70:2 = 35$, $70:7 = 10$. Quy đồng: $\frac{2}{5} = \frac{2\cdot14}{5\cdot14} = \frac{28}{70}$, $\frac{-1}{2} = \frac{-1\cdot35}{2\cdot35} = \frac{-35}{70}$, $\frac{2}{7} = \frac{2\cdot10}{7\cdot10} = \frac{20}{70}$. Vì $-35 < 20 < 28$ nên $\frac{-35}{70} < \frac{20}{70} < \frac{28}{70}$. Vậy $\frac{-1}{2} < \frac{2}{7} < \frac{2}{5}$. (b) $\operatorname{BCNN}(5,3,4) = 60$, $60:5 = 12$, $60:3 = 20$, $60:4 = 15$. Quy đồng: $\frac{12}{5} = \frac{12\cdot12}{5\cdot12} = \frac{144}{60}$, $\frac{-7}{3} = \frac{-7\cdot20}{3\cdot20} = \frac{-21}{60}$, $\frac{-11}{4} = \frac{-11\cdot15}{4\cdot15} = \frac{-165}{60}$. Vì $-165 < -21 < 144$ nên $\frac{-165}{60} < \frac{-21}{60} < \frac{144}{60}$. Vậy $\frac{-11}{4} < \frac{-7}{3} < \frac{12}{5}$.
\end{proof}

\begin{baitoan}[\cite{SGK_Toan_6_Canh_Dieu_tap_2}, 3., p. 33]
	Hà thể hiện thời gian trong ngày của mình bằng biểu đồ tròn với $\frac{1}{3}$ thời gian trong ngày để ngủ, $\frac{1}{6}$ thời gian trong ngày cho hoạt động khác, $\frac{7}{24}$ thời gian trong ngày để học ở trường, $\frac{1}{12}$ thời gian trong ngày để ăn, $\frac{1}{8}$ thời gian trong ngày để tự học. (a) Hỏi Hà dành thời gian cho hoạt động nào nhiều nhất? Ít nhất? (b) Sắp xếp các phân số này theo thứ tự giảm dần.
\end{baitoan}

\begin{proof}[Giải]
	(a) Hà dành thời gian để ngủ nhiều nhất. Hà dành thời gian để ăn ít nhất. (b) $\operatorname{BCNN}(3,6,24,12,8) = 24$, $24:3 = 8$, $24:6 = 4$, $24:12 = 2$, $24:8 = 3$. Quy đồng: $\frac{1}{3} = \frac{1\cdot8}{3\cdot8} = \frac{8}{24}$, $\frac{1}{6} = \frac{1\cdot4}{6\cdot4} = \frac{4}{24}$, $\frac{7}{24}$ không cần quy đồng, $\frac{1}{12} = \frac{1\cdot2}{12\cdot2} = \frac{2}{24}$, $\frac{1}{8} = \frac{1\cdot3}{8\cdot3} = \frac{3}{24}$. Vì $8 > 7 > 4 > 3 > 2$ nên $\frac{8}{24} > \frac{7}{24} > \frac{4}{24} > \frac{3}{24} > \frac{2}{24}$. Vậy các phân số này được sắp xếp theo thứ tự giảm dần: $\frac{1}{3} > \frac{7}{24} > \frac{1}{6} > \frac{1}{8} > \frac{1}{12}$.
\end{proof}

\begin{baitoan}[\cite{SGK_Toan_6_Canh_Dieu_tap_2}, 5., p. 33]
	Tìm $x,y\in\mathbb{Z}$: (a) $\frac{-11}{15} < \frac{x}{15} < \frac{y}{15} < \frac{-8}{15}$. (b) $\frac{-1}{3} < \frac{x}{36} < \frac{y}{18} < \frac{-1}{4}$. (c) $\frac{4}{-12} > \frac{x}{-12} > \frac{y}{-12} > \frac{7}{-12}$. (d) $\frac{-1}{-4} > \frac{-1}{x} > \frac{-1}{y} > \frac{1}{7}$.
\end{baitoan}

\begin{proof}[Giải]
	(a) Vì 4 phân số đều có cùng mẫu dương là $15 > 0$ nên tử số càng lớn thì phân số càng lớn. $\frac{-11}{15} < \frac{x}{15} < \frac{y}{15} < \frac{-8}{15}\Leftrightarrow-11 < x < y < -8$ mà $x,y\in\mathbb{Z}$, suy ra $x = -10$, $y = -9$. (b) $\operatorname{BCNN}(3,36,18,4) = 36$, $36:3 = 12$, $36:18 = 2$, $36:4 = 9$. Quy đồng: $\frac{-1}{3} = \frac{-1\cdot12}{3\cdot12} = \frac{-12}{36}$, $\frac{x}{36}$ không cần quy đồng do có mẫu đã là mẫu chung, $\frac{y}{18} = \frac{y\cdot2}{18\cdot2} = \frac{2y}{36}$, $\frac{-1}{4} = \frac{-1\cdot9}{4\cdot9} = \frac{-9}{36}$. Có $\frac{-1}{3} < \frac{x}{36} < \frac{y}{18} < \frac{-1}{4}\Leftrightarrow\frac{-12}{36} < \frac{x}{36} < \frac{2y}{36} < \frac{-9}{36}\Leftrightarrow -12 < x < 2y < -9$ mà $x,y\in\mathbb{Z}$, suy ra $x = -11$, $2y = -10$ hay $y = -5$. (c) $\frac{4}{-12} > \frac{x}{-12} > \frac{y}{-12} > \frac{7}{-12}\Leftrightarrow\frac{-4}{12} < \frac{-x}{12} < \frac{-y}{12} < \frac{-7}{12}\Leftrightarrow -4 < -x < -y < -7$ mà $x,y\in\mathbb{Z}$, suy ra $-x = -5$ hay $x = 5$ \& $-y = -6$ hay $y = 6$. (d) Sử dụng $a > b > 0\Leftrightarrow0 < \frac{1}{a} < \frac{1}{b}$, có $\frac{-1}{-4} > \frac{-1}{x} > \frac{-1}{y} > \frac{1}{7}\Leftrightarrow\frac{1}{4} > \frac{1}{-x} > \frac{1}{-y} > \frac{1}{7}\Leftrightarrow 4 < -x < -y < 7$ mà $x,y\in\mathbb{Z}$, suy ra $-x = 5$ hay $x = -5$, $-y = 6$ hay $y = -6$.
\end{proof}
\noindent\textsf{\textbf{Tóm tắt kiến thức.}} ``\fbox{\bf 1} Trong 2 phân số khác nhau luôn có 1 phân số nhỏ hơn phân số kia. Nếu phân số $\frac{a}{b}$ nhỏ hơn phân số $\frac{c}{d}$ thì ta viết $\frac{a}{b} < \frac{c}{d}$ hay $\frac{c}{d} > \frac{a}{b}$. Phân số lớn hơn 0 gọi là \textit{phân số dương}, phân số nhỏ hơn 0 gọi là \textit{phân số âm}. Nếu $\frac{a}{b} < \frac{c}{d}$ \& $\frac{c}{d} < \frac{e}{f}$ thì $\frac{a}{b} <\frac{e}{f}$. Có thể thực hiện so sánh 2 phân số bằng cách: Quy đồng mẫu số (đưa về cùng mẫu dương) rồi so sánh tử số: Nếu $m > 0$, $a > b$ thì $\frac{a}{m} > \frac{b}{m}$. \fbox{\bf 2} Viết 1 phân số lớn hơn 1 thành tổng của 1 số nguyên dương \& 1 phân số nhỏ hơn 1 (với tử \& mẫu dương) rồi viết chúng liền nhau thì được 1 hỗn số dương.'' -- \cite[Chap. V, \S2, pp. 32--33]{SBT_Toan_6_Canh_Dieu_tap_2}

\begin{baitoan}[\cite{SBT_Toan_6_Canh_Dieu_tap_2}, Ví dụ 1, p. 33]
	So sánh các cặp phân số sau: (a) $\frac{-5}{9}$ \& $\frac{9}{-5}$. (b) $-\frac{2}{3}$ \& $\frac{2}{3}$.
\end{baitoan}

\begin{proof}[Giải]
	(a) $\frac{-5}{9} = \frac{-25}{45}$ \& $\frac{9}{-5} = \frac{-81}{45}$ mà $\frac{-25}{45} > \frac{-81}{45}$, suy ra $\frac{-5}{9} > \frac{9}{-5}$. (b) $-\frac{2}{3} < 0 < \frac{2}{3}$, suy ra $-\frac{2}{3} < \frac{2}{3}$.
\end{proof}

\begin{baitoan}[\cite{SBT_Toan_6_Canh_Dieu_tap_2}, Ví dụ 2, p. 33]
	(a) Viết mỗi phân số sau thành hỗn số: $\frac{47}{3},\frac{105}{100}$. (b) Viết mỗi hỗn số sau thành phân số: $14\frac{2}{3},4\frac{2}{5}$.
\end{baitoan}

\begin{proof}[Giải]
	(a) $\frac{47}{3} = \frac{3\cdot15 + 2}{3} = \frac{3\cdot15}{3} + \frac{2}{3} = 15 + \frac{2}{3} = 15\frac{2}{3}$. $\frac{105}{100} = \frac{100\cdot1  + 5}{100} = \frac{100\cdot1}{100} + \frac{5}{100} = 1 + \frac{5}{100} = 1 + \frac{1}{20} = 1\frac{1}{20}$. (b)  $14\frac{2}{3} = 14 + \frac{2}{3} = \frac{14\cdot3}{3} + \frac{2}{3} = \frac{42 + 2}{3} = \frac{44}{3}$. $4\frac{2}{5} = 4 + \frac{2}{5} = \frac{4\cdot5}{5} + \frac{2}{5} = \frac{20 + 2}{5} = \frac{22}{5}$.
\end{proof}

\begin{baitoan}[\cite{SBT_Toan_6_Canh_Dieu_tap_2}, Ví dụ 3, p. 33]
	Tìm $x\in\mathbb{N}$ biết: (a) $2\frac{x}{7} = \frac{75}{35}$. (b) $2\frac{3}{x} = \frac{13}{x}$.
\end{baitoan}

\begin{proof}[Giải]
	(a) Vì $\frac{75}{35} = \frac{15}{7} = 2\frac{1}{7}$ nên $2\frac{x}{7} = 2\frac{1}{7}$. Vậy $x = 1$. (b) Điều kiện xác định (ĐKXĐ): $x\ne0$. Vì $2\frac{3}{x} = \frac{2x + 3}{x}$ nên $\frac{2x + 3}{x} = \frac{13}{x}$ hay $2x + 3 = 13$. Vậy $x = 5$.
\end{proof}

\begin{baitoan}[\cite{SBT_Toan_6_Canh_Dieu_tap_2}, 15., p. 34]
	So sánh: (a) $\frac{3}{14}$ \& $\frac{-6}{14}$. (b) $\frac{7}{-12}$ \& $\frac{11}{-18}$. (c) $\frac{-4}{7}$ \& $\frac{4}{-10}$. (d) $\frac{-8}{15}$ \& $\frac{5}{-24}$. (e) $\frac{69}{-230}$ \& $\frac{-39}{143}$. (f) $\frac{7}{41}$ \& $\frac{13}{47}$.
\end{baitoan}

\begin{baitoan}[\cite{SBT_Toan_6_Canh_Dieu_tap_2}, 16., p. 34]
	 (1) Viết các phân số sau theo thứ tự tăng dần: (a) $\frac{-7}{9},\frac{3}{2},\frac{-7}{5},0,\frac{-4}{-3}$. (b) $\frac{-2}{5},\frac{5}{-6},\frac{7}{12},\frac{5}{-24},\frac{17}{30},\frac{-11}{20}$. (2) Viết các phân số sau theo thứ tự giảm dần: (a) $\frac{5}{14},\frac{3}{-40},\frac{-13}{-140},\frac{8}{-35}$. (b) $\frac{3}{400},\frac{-6}{217},\frac{-7}{-284},\frac{112}{-305}$.
\end{baitoan}

\begin{baitoan}[\cite{SBT_Toan_6_Canh_Dieu_tap_2}, 17., p. 34]
	Tìm số nguyên thích hợp điền vào ô trống: (a) $\frac{-12}{19} < \frac{\square}{19} < \frac{\square}{19} < \frac{\square}{19} < \frac{-8}{19}$. (b) $\frac{-1}{2} < \frac{\square}{24} < \frac{\square}{24} < \frac{\square}{24} < \frac{-1}{3}$.
\end{baitoan}

\begin{baitoan}[\cite{SBT_Toan_6_Canh_Dieu_tap_2}, 18., p. 34]
	Viết các hỗn số thích hợp vào chỗ chấm: (a) \emph{4m 7dm $=\ldots$m}. (b) \emph{3kg 315g $=\ldots$kg}. (c) \emph{5 giờ 45 phút $=\ldots$ giờ}. (d) $\rm21m^2\ 8dm^2 =\ldots m^2$.
\end{baitoan}

\begin{baitoan}[\cite{SBT_Toan_6_Canh_Dieu_tap_2}, 19., p. 34]
	Lúc $7$ giờ $15$ phút, 1 xe máy đi từ A đến B. Biết xe máy đi từ A đến B hết $1$ giờ $20$ phút. Xe máy đến B lúc mấy giờ? Viết kết quả dưới dạng hỗn số với đơn vị giờ.
\end{baitoan}

\begin{baitoan}[\cite{SBT_Toan_6_Canh_Dieu_tap_2}, 20., p. 34]
	Đức, Hòa, Bình tham gia 1 cuộc thi chạy $100$\emph{m}. Đức chạy mất $\frac{3}{10}$ phút, Hòa chạy mất $\frac{7}{15}$ phút, Bình chạy mất $\frac{7}{30}$ phút. Ai chạy nhanh nhất?
\end{baitoan}

\begin{baitoan}[\cite{SBT_Toan_6_Canh_Dieu_tap_2}, 21., p. 35]
	2 người cùng đi quãng đường như nhau từ nhà đến siêu thị. Người thứ nhất đi hết $32$ phút, người thứ 2 đi hết $48$ phút. Biết vận tốc của mỗi người không đổi. (a) So sánh quãng đường người thứ nhất đi trong $20$ phút với quãng đường người thứ 2 đi trong $25$ phút. (b) Người thứ 2 phải đi trong bao lâu để được quãng đường bằng người thứ nhất đi trong $24$ phút?
\end{baitoan}

\begin{baitoan}[\cite{SBT_Toan_6_Canh_Dieu_tap_2}, 22., p. 35]
	Theo 1 khảo sát lấy ý kiến bình chọn Quốc hoa được công bố vào tháng \emph{1\texttt{/}2011}, $\frac{62}{100}$ số người chọn hoa sen, $\frac{3}{20}$ số người chọn hoa mai, $\frac{4}{25}$ số người chọn hoa đào. (a) Sắp xếp các phân số trên theo thứ tự giảm dần. (b) Loài hoa nào được bình chọn nhiều nhất?
\end{baitoan}

\begin{baitoan}[\cite{SBT_Toan_6_Canh_Dieu_tap_2}, 23., p. 35]
	Phân số chỉ số phần nước trong 1 số loại củ, quả được cho ở bảng sau:
	\begin{table}[H]
		\centering
		\begin{tabular}{|c|c|c|c|c|}
			\hline
			Loại củ, quả & Củ cải trắng & Mâm xôi & Dưa vàng & Đào \\
			\hline
			Số phần nước & $\frac{19}{20}$ & $\frac{87}{100}$ & $\frac{9}{10}$ & $\frac{22}{25}$ \\
			\hline
		\end{tabular}
	\end{table}
	\noindent Củ, quả nào có lượng nước chiếm tỷ lệ cao nhất? Thấp nhất?
\end{baitoan}

\begin{baitoan}[\cite{SBT_Toan_6_Canh_Dieu_tap_2}, 24., p. 35]
	Tìm phân số có mẫu bằng $7$ biết khi cộng tử với $16$ \& nhân mẫu với $5$ thì giá trị của phân số đó không thay đổi.
\end{baitoan}

\begin{baitoan}[\cite{SBT_Toan_6_Canh_Dieu_tap_2}, 25., p. 35]
	Theo 1 thống kê, trong tổng số lượng sách được người đọc yêu thích: sách kỹ năng sống chiếm $\frac{1}{4}$, sách văn học chiếm $\frac{3}{20}$, sách nuôi dạy con chiếm $\frac{3}{25}$, sách khoa học công nghiệp chiếm $\frac{31}{100}$, sách kinh doanh đầu tư chiếm $\frac{17}{100}$. Sách nào được nhiều bạn đọc yêu thích nhất?
\end{baitoan}

\begin{baitoan}[\cite{SBT_Toan_6_Canh_Dieu_tap_2}, 26., p. 35]
	Tìm $x,y\in\mathbb{Z}$ sao cho $\frac{1}{8} < \frac{x}{18} < \frac{y}{24} < \frac{2}{9}$.
\end{baitoan}

``Để so sánh 2 phân số có tử \& mẫu đều dương, ngoài cách quy đồng tử hoặc quy đồng mẫu, người ta thường dùng 1 phân số trung gian \& sử dụng tính chất bắc cầu của bất đẳng thức.

Thường sử dụng các tính chất sau: (a) Trong 2 phân số cùng tử, phân số nào có mẫu nhỏ hơn thì phân số đó lớn hơn. (b) Trong 2 phân số nhỏ hơn 1, phân số nào có phần bù đến 1 nhỏ hơn thì phân số đó lớn hơn: $1 - \frac{a}{b} < 1 - \frac{c}{d}\Rightarrow\frac{a}{b} > \frac{c}{d}$. (c) Nếu $0 < a < 1$ \& $m < n$ thì $a^m > a^n$.'' -- \cite[Chap. 1, \S2, p. 8]{Binh_Toan_6_tap_2}

\begin{baitoan}[\cite{Tuyen_Toan_6}, Ví dụ 52, p. 48]
	So sánh 2 phân số $\frac{-101}{-100}$ \& $\frac{200}{201}$.
\end{baitoan}

\begin{proof}[Giải]
	$\frac{-101}{-100} = \frac{101}{100} > \frac{100}{100} = 1 = \frac{201}{201} > \frac{200}{201}$. Vậy $\frac{-101}{-100} > \frac{200}{201}$.
\end{proof}

\begin{baitoan}[Mở rộng \cite{Tuyen_Toan_6}, Ví dụ 52, p. 48]
	Cho $a,b,c,d\in\mathbb{N}$, $a > b > 0$, $d > c > 0$. So sánh: $\frac{\pm a}{\pm b}$ \& $\frac{\pm c}{\pm d}$.
\end{baitoan}

\begin{baitoan}[\cite{Tuyen_Toan_6}, Ví dụ 53, p. 48]
	Sắp xếp các phân số sau theo thứ tự tăng dần: $\frac{5}{8},\frac{9}{16},\frac{2}{-3},\frac{-7}{12}$.
\end{baitoan}

\begin{baitoan}[\cite{Binh_Toan_6_tap_2}, Ví dụ 5, p. 8]
	So sánh $A = \frac{10^{15} + 1}{10^{16} + 1}$ \& $B = \frac{10^{16} + 1}{10^{17} + 1}$.
\end{baitoan}

\begin{baitoan}[\cite{Binh_Toan_6_tap_2}, Ví dụ 6, p. 9]
	1 phân số có tử \& mẫu đều là số nguyên dương. Nếu cộng tử \& mẫu của phân số đó với cùng $n\in\mathbb{N}^\star$ thì phân số thay đổi thế nào?
\end{baitoan}

\begin{baitoan}[\cite{Binh_Toan_6_tap_2}, Ví dụ 7, p. 9]
	So sánh $\left(\frac{1}{32}\right)^7$ \& $\left(\frac{1}{16}\right)^9$.
\end{baitoan}

\begin{baitoan}[\cite{Binh_Toan_6_tap_2}, Ví dụ 8, p. 9]
	Chứng minh $95^8$ là 1 số có $16$ chữ số khi viết kết quả của nó trong hệ thập phân.
\end{baitoan}

\begin{baitoan}[\cite{Binh_Toan_6_tap_2}, Ví dụ 9, p. 10]
	Cho $a,b\in\mathbb{N}^\star$ thỏa $\frac{5}{7} < \frac{a}{b} < \frac{9}{11}$. Tìm $a + b$ khi $b$ nhỏ nhất.
\end{baitoan}

\begin{baitoan}[\cite{Binh_Toan_6_tap_2}, 20., p. 10]
	So sánh $a,b\in\mathbb{N}$ biết $\frac{1 + 2 + 3 + \cdots + a}{a} < \frac{1 + 2 + 3 + \cdots + b}{b}$.
\end{baitoan}

\begin{baitoan}[\cite{Binh_Toan_6_tap_2}, 21., p. 10]
	So sánh: (a) $\frac{18}{91}$ \& $\frac{23}{114}$. (b) $\frac{21}{52}$ \& $\frac{213}{523}$. (c) $\frac{1313}{9191}$ \& $\frac{1111}{7373}$.
\end{baitoan}

\begin{baitoan}[\cite{Binh_Toan_6_tap_2}, 22., p. 10]
	So sánh các phân số sau, với $n\in\mathbb{N}$: (a) $\frac{n}{n + 1}$ \& $\frac{n + 2}{n + 3}$. (b) $\frac{n + 1}{n + 4}$ \& $\frac{n}{n + 5}$. (c) $\frac{n}{2n + 1}$ \& $\frac{3n + 1}{6n + 3}$.
\end{baitoan}

\begin{baitoan}[\cite{Binh_Toan_6_tap_2}, 23., p. 11]
	So sánh $A$ \& $B$: (a) $A = \frac{20}{39} + \frac{22}{27} + \frac{18}{43}$, $B = \frac{14}{39} + \frac{22}{29} + \frac{18}{41}$. (b) $A = \frac{3}{8^3} + \frac{7}{8^4}$, $B = \frac{7}{8^3} + \frac{3}{8^4}$. (c) $A = \frac{10^7 + 5}{10^7 - 8}$, $B = \frac{10^8 + 6}{10^8 - 7}$. (d) $A = \frac{10^{1992} + 1}{10^{1991} + 1}$, $B = \frac{10^{1993} + 1}{10^{1992} + 1}$.
\end{baitoan}

\begin{baitoan}[\cite{Binh_Toan_6_tap_2}, 24., p. 11]
	Tìm $x\in\mathbb{N}$ sao cho $\frac{4}{11} < \frac{x}{20} < \frac{5}{11}$.
\end{baitoan}

\begin{baitoan}[\cite{Binh_Toan_6_tap_2}, 25., p. 11]
	Tìm 2 phân số có các mẫu bằng $9$, các tử là 2 số tự nhiên liên tiếp sao cho phân số $\frac{4}{7}$ nằm giữa 2 phân số đó.
\end{baitoan}

\begin{baitoan}[\cite{Binh_Toan_6_tap_2}, 26., p. 11]
	Tìm 2 phân số có các tử bằng $1$, các mẫu là 2 số tự nhiên liên tiếp sao cho phân số $\frac{13}{84}$ nằm giữa 2 phân số đó.
\end{baitoan}

\begin{baitoan}[\cite{Binh_Toan_6_tap_2}, 27., p. 11]
	Tìm 2 phân số có mẫu bằng $21$ biết nó lớn hơn $\frac{5}{7}$ \& nhỏ hơn $\frac{5}{6}$.
\end{baitoan}

\begin{baitoan}[\cite{Binh_Toan_6_tap_2}, 28., p. 11]
	Tìm phân số $\frac{a}{b}$ sao cho $a$ là số tự nhiên nhỏ nhất thỏa mãn $\frac{4}{15} < \frac{a}{b} < \frac{1}{3}$.
\end{baitoan}

\begin{baitoan}[\cite{Binh_Toan_6_tap_2}, 29., p. 11]
	Tìm phân số $\frac{a}{b}$ lớn nhất nhỏ hơn $1$ với $a,b$ là các số nguyên dương có 1 chữ số.
\end{baitoan}

\begin{baitoan}[\cite{Binh_Toan_6_tap_2}, 30., p. 11]
	So sánh 2 phân số $\left(\frac{1}{243}\right)^9$ \& $\left(\frac{1}{83}\right)^{13}$.
\end{baitoan}

%------------------------------------------------------------------------------%

\section{Hỗn Số Dương}
Hỗn số dương $=$ hỗn hợp của số nguyên dương \& phân số dương.

\begin{menhde}[Hỗn số dương]
	Viết 1 phân số lớn hơn $1$ (i.e., tử $>$ mẫu $>$ 0) thành tổng của 1 số nguyên dương \& 1 phân số nhỏ hơn $1$ (với tử \& mẫu dương) rồi viết chúng liền nhau thì được 1 \emph{hỗn số dương}
\end{menhde}

\begin{baitoan}[\cite{Binh_Toan_6_tap_2}, Ví dụ 2, p. 32]
	(a) Viết phân số $\frac{7}{4}$ dưới dạng tổng của 1 số nguyên dương \& 1 phân số bé hơn $1$. (b) Viết phân số $\frac{21}{5}$ dưới dạng hỗn số.
\end{baitoan}

\begin{proof}[Giải]
	(a) $\frac{7}{4} = \frac{4\cdot1 + 3}{4} = \frac{4\cdot1}{4} + \frac{3}{4} = 1 + \frac{3}{4} = 1\frac{3}{4}$. (b) $\frac{21}{5} = \frac{5\cdot4 + 1}{5} = \frac{5\cdot4}{5} + \frac{1}{5} = 4 + \frac{1}{5} = 4\frac{1}{5}$.
\end{proof}

\begin{baitoan}[\cite{Binh_Toan_6_tap_2}, Ví dụ 3, 2, p. 33]
	(a) Viết hỗn số $2\frac{3}{5}$ thành phân số. (b) Viết mỗi phân số sau thành hỗn số: $\frac{14}{3},\frac{22}{7}$. (c) Viết mỗi hỗn số sau thành phân số: $2\frac{3}{4},5\frac{1}{6}$.
\end{baitoan}

\begin{proof}[Giải]
	(a) $2\frac{3}{5} = 2 + \frac{3}{5} = \frac{2\cdot5}{5} + \frac{3}{5} = \frac{10 + 3}{5} = \frac{13}{5}$. (b) $\frac{14}{3} = \frac{3\cdot4 + 2}{3} = \frac{3\cdot4}{3} + \frac{2}{3} = 4 + \frac{2}{3} = 4\frac{2}{3}$, $\frac{22}{7} = \frac{7\cdot3 + 1}{7} = \frac{7\cdot3}{7} + \frac{1}{7} = 3 + \frac{1}{7} = 3\frac{1}{7}$. (c) $2\frac{3}{4} = 2 + \frac{3}{4} = \frac{2\cdot4}{4} + \frac{3}{4} = \frac{2\cdot4 + 3}{4} = \frac{11}{4}$, $5\frac{1}{6} = 5 + \frac{1}{6} = \frac{5\cdot6}{6} + \frac{1}{6} = \frac{5\cdot6 + 1}{6} = \frac{31}{6}$.
\end{proof}

\begin{baitoan}[\cite{SGK_Toan_6_Canh_Dieu_tap_2}, 4., p. 33]
	(a) Viết các số đo thời gian dưới dạng hỗn số với đơn vị là giờ: $2$ giờ $15$ phút, $10$ giờ $20$ phút. (b) Viết các số đo diện tích sau dưới dạng hỗn số với đơn vị là hecta biết $1$\emph{ha} $= 100$\emph{a}: \emph{1ha7a, 3ha50a},
\end{baitoan}

\begin{proof}[Giải]
	(a) Vì 1 giờ $= 60$ phút nên 1 phút $= \frac{1}{60}$ giờ, suy ra $2$ giờ $15$ phút $= 2 + \frac{15}{60} = 2 + \frac{15:15}{60:15} = 2 + \frac{1}{4} = 2\frac{1}{4}$ giờ, $10$ giờ $20$ phút $= 10 + \frac{20}{60} = 10 + \frac{20:20}{60:20} = 10 + \frac{1}{3} = 10\frac{1}{3}$ giờ. (b) Vì 1ha $=$ 100a nên 1a $= \frac{1}{100}$ha, suy ra 1ha7a $= 1 + \frac{7}{100} = 1\frac{7}{100}$ha, 3ha50a $= 3 + \frac{50}{100} = 3 + \frac{50:50}{100:50} = 3 = \frac{1}{2} = 3\frac{1}{2}$ha.
\end{proof}

\begin{baitoan}[Chuyển đổi phân số \& hỗn số dương]
	(a) Khi nào 1 phân số có thể chuyển thành 1 hỗn số dương? (b) Ngược lại, khi nào 1 hỗn số dương có thể chuyển thành 1 phân số? (c) Nếu định nghĩa 1 \emph{hỗn số âm} là 1 số đối của hỗn số dương tương ứng thì khi nào 1 hỗn số âm có thể chuyển thành 1 phân số?
\end{baitoan}

\begin{proof}[Giải]
	(a) 1 phân số muốn chuyển thành 1 hỗn số dương thì tử \& mẫu của phân số đó phải cùng dấu (để phân số đó \textit{dương}) \& tử phải lớn hơn mẫu (để phân số đó $> 1$). Nói cách khác, 1 phân số có thể chuyển thành 1 hỗn số dương khi \& chỉ khi phân số đó lớn hơn 1 (phân số lớn hơn $1$ thì tự động là phân số dương vì tính chất bắc cầu $\frac{a}{b} > 1 > 0$). (b) Mọi hỗn số dương đều có thể chuyển thành phân số nhờ công thức $a\frac{b}{c} = a + \frac{b}{c} = \frac{ac}{c} + \frac{b}{c} = \frac{ac + b}{c}$, $\forall a,b,c\in\mathbb{N}^\star$. (c) Mọi hỗn số âm đều có thể chuyển thành phân số nhờ công thức $-a\frac{b}{c} = -\left(a + \frac{b}{c}\right) = -\left(\frac{ac}{c} + \frac{b}{c}\right) = -\frac{ac + b}{c}$, $\forall a,b,c\in\mathbb{N}^\star$.
\end{proof}

\begin{luuy}[Tổng của 2 hỗn số đối nhau bằng $0$]
	Tổng của 1 hỗn số dương \& hỗn số âm mà là số đối của hỗn số dương đó vẫn bằng $0$. Thật vậy: $a\frac{b}{c} + \left(-a\frac{b}{c}\right) = a + \frac{b}{c} - \left(a + \frac{b}{c}\right) = a + \frac{b}{c} + (-a) + \left(-\frac{b}{c}\right) = (a + (-a)) + \left(\frac{b}{c} + \left(-\frac{b}{c}\right)\right) = 0 + 0 = 0$, $\forall a,b,c\in\mathbb{N}^\star$. Hoặc có thể tính ngắn hơn nhờ lời giải câu (c) bài toán trên: $a\frac{b}{c} + \left(-a\frac{b}{c}\right) = \frac{ac + b}{c} +\left(-\frac{ac + b}{c}\right) = 0$, $\forall a,b,c\in\mathbb{N}^\star$ (tổng của 2 phân số đối nhau bằng $0$).
\end{luuy}

\begin{baitoan}[Công thức hỗn số dương]
	Chứng minh:
	\begin{align*}
		\frac{ac + b}{c} &= a + \frac{b}{c} = a\frac{b}{c},\ \forall a,b,c\in\mathbb{N}^\star,\\
		\frac{a}{b} &= \frac{\lfloor\frac{a}{b}\rfloor b + \left\{\frac{a}{b}\right\}}{b} = \left\lfloor\frac{a}{b}\right\rfloor + \frac{\left\{\frac{a}{b}\right\}}{b} = \left\lfloor\frac{a}{b}\right\rfloor\frac{\left\{\frac{a}{b}\right\}}{b},\ \forall a,b\in\mathbb{N}^\star,\, a > b.
	\end{align*}
\end{baitoan}

%------------------------------------------------------------------------------%

\section{Phép Cộng, Phép Trừ Phân Số}

\subsection{Phép cộng phân số}

\subsubsection{Quy tắc cộng phân số}

\begin{menhde}[Cộng các phân số cùng mẫu]
	Muốn cộng 2 hay nhiều phân số có cùng mẫu, ta cộng các tử \& giữ nguyên các mẫu:
	\begin{align*}
		\frac{a}{m} + \frac{b}{m} &= \frac{a + b}{m},\ \forall a,b,m\in\mathbb{Z},\,b\ne0,\\
		\sum_{i=1}^n \frac{a_i}{b} = \frac{a_1}{b} + \frac{a_2}{b} + \cdots + \frac{a_n}{b} &= \frac{a_1 + a_2 + \cdots + a_n}{b} = \frac{\sum_{i=1}^n a_i}{b},\ \forall a_i,b\in\mathbb{Z},\ \forall i = 1,2,\ldots,n,\,b\ne0.
	\end{align*}
\end{menhde}

\begin{vidu}
	(a) $\frac{1}{5} + \frac{3}{5} = \frac{1 + 3}{5} = \frac{4}{5}$. (b) $\frac{-1}{5} + \frac{3}{5} = \frac{-1 + 3}{5} = \frac{2}{5}$. (c) $\frac{2}{-3} + \frac{-7}{-3} = \frac{2 + (-7)}{-3} = \frac{-5}{-3} = \frac{5}{3}$.
\end{vidu}

\begin{baitoan}[\cite{SGK_Toan_6_Canh_Dieu_tap_2}, 1, p. 34]
	Tính: $\frac{11}{-9} + \frac{5}{-6}$.
\end{baitoan}
\noindent\textit{Hint.} Trước hết, chuyển các mẫu âm của các phân số này thành mẫu dương bằng quy tắc $\frac{a}{-b} = \frac{-a}{b} = -\frac{a}{b}$, rồi quy đồng mẫu số để 2 phân số có cùng mẫu, rồi cộng 2 phân số theo quy tắc cộng 2 phân số có cùng mẫu.

\begin{proof}[Giải]
	Quy đồng mẫu 2 phân số trước hết, chuyển các phân số về mẫu dương: $\frac{11}{-9} = \frac{-11}{9}$, $\frac{5}{-6} = \frac{-5}{6}$, $\operatorname{BCNN}(9,6) = 18$, $18:9 = 2$, $18:6 = 3$. $\frac{11}{-9} = \frac{-11\cdot2}{9\cdot2} = \frac{-22}{18}$, $\frac{5}{-6} = \frac{-5\cdot3}{6\cdot3} = \frac{-15}{18}$. Suy ra $\frac{11}{-9} + \frac{5}{-6} = \frac{-22}{18} + \frac{-15}{18} = \frac{-22 + (-15)}{18} = \frac{-37}{18}$. Vậy $\frac{11}{-9} + \frac{5}{-6} = \frac{-37}{18}$.
\end{proof}

\begin{luuy}
	Khi tính toán với các phân số, luôn chuyển các mẫu âm thành các mẫu dương bằng cách sử dụng công thức: $\frac{a}{-b} = \frac{-a}{b} = -\frac{a}{b}$, $\forall a,b\in\mathbb{Z}$, $b\ne0$.
\end{luuy}

\begin{menhde}[Cộng các phân số khác mẫu]
	Muốn cộng 2 hay nhiều phân số không cùng mẫu, ta quy đồng mẫu những phân số đó rồi cộng các tử \& giữ nguyên mẫu chung.
	\begin{align*}
		\frac{a}{b} + \frac{c}{d} &= \frac{a\cdot\frac{\operatorname{BCNN}(b,d)}{b}}{\operatorname{BCNN}(b,d)} + \frac{c\cdot\frac{\operatorname{BCNN}(b,d)}{d}}{\operatorname{BCNN}(b,d)} = \frac{a\cdot\frac{\operatorname{BCNN}(b,d)}{b} + c\cdot\frac{\operatorname{BCNN}(b,d)}{d}}{\operatorname{BCNN}(b,d)},\ \forall a,b,c,d\in\mathbb{Z},\,bd\ne0,\\
		\sum_{i=1}^n \frac{a_i}{b_i} &= \sum_{i=1}^n \frac{a_i\cdot\frac{\operatorname{BCNN}(b_1,b_2,\ldots,b_n)}{b_i}}{\operatorname{BCNN}(b_1,b_2,\ldots,b_n)} = \frac{\sum_{i=1}^n a_i\cdot\frac{\operatorname{BCNN}(b_1,b_2,\ldots,b_n)}{b_i}}{\operatorname{BCNN}(b_1,b_2,\ldots,b_n)},\ \forall a_i,b_i\in\mathbb{Z},\,b_i\ne0,\ \forall i = 1,2,\ldots,n.
	\end{align*}
\end{menhde}
Công thức này nhìn lướt qua có vẻ phức tạp nhưng thực ra dễ hiểu bằng cách chú ý ở phân số $\frac{a}{b}$, ta đã nhân cả tử \& mẫu cho \textit{thừa số phụ} $\frac{\operatorname{BCNN}(b,d)}{b}$, còn ở phân số $\frac{c}{d}$, ta đã nhân cả tử \& mẫu cho \textit{thừa số phụ} $\frac{\operatorname{BCNN}(b,d)}{d}$. Không dùng ký hiệu tổng $\sigma$, công thức thứ 2 có thể viết cụ thể hơn như sau:
\begin{align*}
	\frac{a_1}{b_1} + \frac{a_2}{b_2} + \cdots + \frac{a_n}{b_n} &= \frac{a_1\cdot\frac{\operatorname{BCNN}(b_1,b_2,\ldots,b_n)}{b_1}}{\operatorname{BCNN}(b_1,b_2,\ldots,b_n)} + \frac{a_2\cdot\frac{\operatorname{BCNN}(b_1,b_2,\ldots,b_n)}{b_2}}{\operatorname{BCNN}(b_1,b_2,\ldots,b_n)} + \cdots + \frac{a_n\cdot\frac{\operatorname{BCNN}(b_1,b_2,\ldots,b_n)}{b_n}}{\operatorname{BCNN}(b_1,b_2,\ldots,b_n)}\\
	&= \frac{a_1\cdot\frac{\operatorname{BCNN}(b_1,b_2,\ldots,b_n)}{b_1} + a_2\cdot\frac{\operatorname{BCNN}(b_1,b_2,\ldots,b_n)}{b_2} + \cdots + a_n\cdot\frac{\operatorname{BCNN}(b_1,b_2,\ldots,b_n)}{b_n}}{\operatorname{BCNN}(b_1,b_2,\ldots,b_n)},\ \forall a_i,b_i\in\mathbb{Z},\,b_i\ne0,\ \forall i = 1,2,\ldots,n.
\end{align*}

\begin{baitoan}[\cite{SGK_Toan_6_Canh_Dieu_tap_2}, Ví dụ 1, 1, p. 35]
	Tính: (a) $\frac{2}{3} + \frac{2}{-3}$. (b) $\frac{-5}{6} + \frac{-3}{8}$. (c) $\frac{-3}{7} + \frac{2}{7}$. (d) $\frac{-4}{9} + \frac{2}{-3}$.
\end{baitoan}

\begin{proof}[Giải]
	(a) $\frac{2}{3} + \frac{2}{-3} = \frac{2}{3} + \frac{-2}{3} = \frac{2 + (-2)}{3} = \frac{0}{3} = 0$. (b) $\frac{-5}{6} + \frac{-3}{8} = \frac{-5\cdot4}{6\cdot4} + \frac{-3\cdot3}{8\cdot3} = \frac{-20}{24} + \frac{-9}{24} = \frac{-20 - 9}{24} = \frac{-29}{24}$. (c) $\frac{-3}{7} + \frac{2}{7} = \frac{-3 + 2}{7} = \frac{-1}{7}$. (d) $\frac{-4}{9} + \frac{2}{-3} = \frac{-4}{9} + \frac{-2}{3} = \frac{-4}{9} + \frac{-2\cdot3}{3\cdot3} = \frac{-4}{9} + \frac{-6}{9} = \frac{-4 - 6}{9} = \frac{-10}{9}$.
\end{proof}

\subsubsection{Tính chất của phép cộng phân số}
Giống như phép cộng số tự nhiên, phép cộng phân số cũng có các tính chất: giao hoán, kết hợp, cộng với số 0:
\begin{itemize}
	\item \textit{Giao hoán}: $\frac{a}{b} + \frac{c}{d} = \frac{c}{d} + \frac{a}{b}$, $\forall a,b,c,d\in\mathbb{Z}$, $bd\ne0$ (chú ý $bd\ne0\Leftrightarrow b\ne0$ \& $d\ne0$).
	\item \textit{Kết hợp}: $\left(\frac{a}{b} + \frac{c}{d}\right) + \frac{e}{f} = \frac{a}{b} + \left(\frac{c}{d} + \frac{e}{f}\right) = \frac{a}{b} + \frac{c}{d} + \frac{e}{f}$, $\forall a,b,c,d,e,f\in\mathbb{Z}$, $bdf\ne0$ (chú ý $bdf\ne0\Leftrightarrow b\ne0$ \& $d\ne0$ \& $f\ne0$).
	\item \textit{Cộng với số $0$}: $\frac{a}{b} + 0 = \frac{a}{b}$, $\forall a,b\in\mathbb{Z}$, $b\ne0$.
\end{itemize}

\begin{baitoan}[\cite{SGK_Toan_6_Canh_Dieu_tap_2}, Ví dụ 2, 2, p. 35]
	Tính hợp lý: (a) $\frac{3}{13} + \frac{-3}{7} + \frac{10}{13} + \frac{-4}{7}$. (b) $\frac{-5}{9} + \frac{4}{11} + \frac{7}{11}$. (c) $\frac{-2}{5} + \frac{3}{8} + \frac{-3}{5} + \frac{13}{8}$.
\end{baitoan}

\begin{proof}[Giải]
	(a) $\frac{3}{13} + \frac{-3}{7} + \frac{10}{13} + \frac{-4}{7} = \frac{3}{13} + \frac{10}{13} + \frac{-3}{7} + \frac{-4}{7} = \left(\frac{3}{13} + \frac{10}{13}\right) + \left(\frac{-3}{7} + \frac{-4}{7}\right) = \frac{3 + 10}{13} + \frac{-3 + (-4)}{7} = 1 + (-1) = 0$. (b) $\frac{-5}{9} + \frac{4}{11} + \frac{7}{11} = \frac{-5}{9} + \left(\frac{4}{11} + \frac{7}{11}\right) = \frac{-5}{9} + \frac{4 + 7}{11} = \frac{-5}{9} + \frac{11}{11} = \frac{-5}{9} + 1 = \frac{-5}{9} + \frac{9}{9} = \frac{-5 + 9}{9} = \frac{4}{9}$. (c) $\frac{-2}{5} + \frac{3}{8} + \frac{-3}{5} + \frac{13}{8} = \frac{-2}{5} + \frac{-3}{5} + \frac{3}{8} + \frac{13}{8} = \left(\frac{-2}{5} + \frac{-3}{5}\right) + \left(\frac{3}{8} + \frac{13}{8}\right) = \frac{-2 + (-3)}{5} + \frac{3 + 13}{8} = \frac{-5}{5} + \frac{16}{8} = -1 + 2 = 1$.
\end{proof}

\subsection{Phép trừ phân số}

\subsubsection{Số đối của 1 phân số}
Giống như số nguyên, mỗi phân số đều có số đối sao cho tổng của 2 số đó bằng 0.

\begin{dinhnghia}[Số đối của phân số]
	\emph{Số đối} của phân số $\frac{a}{b}$ ký hiệu là $-\frac{a}{b}$. Có: $\frac{a}{b} + \left(-\frac{a}{b}\right) = 0$, $\forall a,b\in\mathbb{Z}$, $b\ne0$.
\end{dinhnghia}
Có: $-\frac{a}{b} = \frac{a}{-b} = \frac{-a}{b}$, $\forall a,b\in\mathbb{Z}$, $b\ne0$, i.e., dấu $-$ có thể đặt trước 1 phân số, cũng có thể đem lên tử, hoặc đem xuống mẫu thì phân số vẫn không đổi giá trị. Số đối của $-\frac{a}{b}$ là $\frac{a}{b}$, i.e., $-\left(-\frac{a}{b}\right) = \frac{a}{b}$, $\forall a,b\in\mathbb{Z}$, $b\ne0$.

\begin{vidu}
	Số đối của phân số $\frac{2}{5}$ là $-\frac{2}{5}$. Số đối của phân số $\frac{-3}{7}$ là $-\left(\frac{-3}{7}\right) = \frac{-(-3)}{7} = \frac{3}{7}$.
\end{vidu}

\subsubsection{Quy tắc trừ phân số}

\begin{menhde}[Trừ các phân số cùng mẫu]
	Muốn trừ 2 hay nhiều phân số có cùng mẫu, ta trừ tử của số bị trừ cho tử của số trừ \& giữ nguyên mẫu:
	\begin{align*}
		\frac{a}{m} - \frac{b}{m} = \frac{a - b}{m},\ \forall a,b\in\mathbb{Z},\,b\ne0.
	\end{align*}
\end{menhde}
Có thể ghi gộp lại quy tắc cộng\texttt{/}trừ 2 hay nhiều phân số cùng mẫu như sau (với các dấu $+,-$ được sắp xếp tương ứng với dấu của biểu thức ban đầu):
\begin{align*}
	\frac{a}{m}\pm\frac{b}{m} &= \frac{a\pm b}{m},\ \forall a,b\in\mathbb{Z},\,b\ne0,\\
	\sum_{i=1}^n \pm\frac{a_i}{b} = \pm\frac{a_1}{b}\pm\frac{a_2}{b}\pm\cdots\pm\frac{a_n}{b} &= \frac{\pm a_1\pm a_2 + \cdots\pm a_n}{b} = \frac{\sum_{i=1}^n \pm a_i}{b},\ \forall a_i,b\in\mathbb{Z},\ \forall i = 1,2,\ldots,n,\,b\ne0.
\end{align*}

\begin{vidu}
	(a) $\frac{4}{5} - \frac{3}{5} = \frac{4 - 3}{5} = \frac{1}{5}$. (b) $\frac{-1}{5} - \frac{3}{5} = \frac{-1 - 3}{5} = \frac{-4}{5}$.
\end{vidu}

\begin{baitoan}[\cite{SGK_Toan_6_Canh_Dieu_tap_2}, 3, p. 36]
	Tính: $\frac{13}{-9} - \frac{7}{-6}$.
\end{baitoan}
\noindent\textit{Hint.} Trước hết, chuyển các mẫu âm của các phân số này thành mẫu dương bằng quy tắc $\frac{a}{-b} = \frac{-a}{b} = -\frac{a}{b}$, rồi quy đồng mẫu số để 2 phân số có cùng mẫu, rồi trừ 2 phân số theo quy tắc trừ 2 phân số có cùng mẫu.

\begin{proof}[Giải]
	Quy đồng mẫu 2 phân số: $\frac{13}{-9} = \frac{-13}{9}$, $\frac{7}{-6} = \frac{-7}{6}$, $\operatorname{BCNN}(9,6) = 18$, $18:9 = 2$, $18:6 = 3$, $\frac{13}{-9} = \frac{-13\cdot2}{9\cdot2} = \frac{-26}{18}$, $\frac{7}{-6} = \frac{-7\cdot3}{6\cdot3} = \frac{-21}{18}$. Suy ra $\frac{13}{-9} - \frac{7}{-6} = \frac{-26}{18} - \frac{-21}{18} = \frac{-26 -(-21)}{18} = \frac{-5}{18}$. Vậy $\frac{13}{-9} - \frac{7}{-6} = \frac{-5}{18}$.
\end{proof}s

\begin{menhde}[Trừ các phân số khác mẫu]
	Muốn trừ 2 phân số không cùng mẫu, ta quy đồng mẫu những phân số đố rồi trừ tử của số bị trừ cho tử của số trừ \& giữ nguyên mẫu chung.
	\begin{align*}
		\frac{a}{b} - \frac{c}{d} &= \frac{a\cdot\frac{\operatorname{BCNN}(b,d)}{b}}{\operatorname{BCNN}(b,d)} - \frac{c\cdot\frac{\operatorname{BCNN}(b,d)}{d}}{\operatorname{BCNN}(b,d)} = \frac{a\cdot\frac{\operatorname{BCNN}(b,d)}{b} - c\cdot\frac{\operatorname{BCNN}(b,d)}{d}}{\operatorname{BCNN}(b,d)},\ \forall a,b,c,d\in\mathbb{Z},\,bd\ne0,\\
		\sum_{i=1}^n \pm\frac{a_i}{b_i} &= \sum_{i=1}^n \pm\frac{a_i\cdot\frac{\operatorname{BCNN}(b_1,b_2,\ldots,b_n)}{b_i}}{\operatorname{BCNN}(b_1,b_2,\ldots,b_n)} = \frac{\sum_{i=1}^n \pm a_i\cdot\frac{\operatorname{BCNN}(b_1,b_2,\ldots,b_n)}{b_i}}{\operatorname{BCNN}(b_1,b_2,\ldots,b_n)},\ \forall a_i,b_i\in\mathbb{Z},\,b_i\ne0,\ \forall i = 1,2,\ldots,n.
	\end{align*}
\end{menhde}

\begin{baitoan}[\cite{SGK_Toan_6_Canh_Dieu_tap_2}, Ví dụ 4, 3, p. 37]
	Tính: (a) $\frac{1}{3} - \frac{2}{-3}$. (b) $\frac{-5}{6} - \frac{-7}{8}$. (c) $\frac{7}{-10} - \frac{9}{10}$.
\end{baitoan}

\begin{proof}[Giải]
	(a) $\frac{1}{3} - \frac{2}{-3} = \frac{1}{3} - \frac{-2}{3} = \frac{1 -(-2)}{3} = \frac{3}{3} = 1$. (b) $\frac{-5}{6} - \frac{-7}{8} = \frac{-5\cdot4}{6\cdot4} - \frac{-7\cdot3}{8\cdot3} = \frac{-20}{24} - \frac{-21}{24} = \frac{-20 -(-21)}{24} = \frac{1}{24}$. (c) $\frac{7}{-10} - \frac{9}{10} = \frac{-7}{10} - \frac{9}{10} = \frac{-7 - 9}{10} = \frac{-16}{10} = \frac{-8}{5}$.
\end{proof}

\begin{baitoan}[\cite{SGK_Toan_6_Canh_Dieu_tap_2}, 4, p. 37]
	(a) Phân số $\frac{2}{5}$ có phải là số đối của phân số $\frac{2}{-5}$ không? (b) Tính \& so sánh: $\frac{-3}{7} - \frac{2}{-5}$ \& $\frac{-3}{7} + \frac{2}{5}$.
\end{baitoan}

\begin{proof}[Giải]
	(a) Phân số $\frac{2}{5}$ là số đối của phân số $\frac{2}{-5}$ vì $-\frac{2}{5} = \frac{2}{-5}$. (b) $\frac{-3}{7} - \frac{2}{-5} = \frac{-3}{7} - \frac{-2}{5} = \frac{-3}{7} - \frac{-2}{5} = \frac{-3\cdot5}{7\cdot5} - \frac{-2\cdot7}{5\cdot7} = \frac{-15}{35} - \frac{-14}{35} = \frac{-15 -(-14)}{35} = \frac{-1}{35}$, $\frac{-3}{7} + \frac{2}{5} = \frac{-3\cdot5}{7\cdot5} + \frac{2\cdot7}{5\cdot7} = \frac{-15}{35} + \frac{14}{35} = \frac{-15 + 14}{35} = \frac{-1}{35}$. Vậy $\frac{-3}{7} - \frac{2}{-5} = \frac{-3}{7} + \frac{2}{5} = \frac{-1}{35}$.
\end{proof}

\begin{menhde}
	Muốn trừ 2 phân số, ta cộng số bị trừ với số đối của số trừ:
	\begin{align*}
		\frac{a}{b} - \frac{c}{d} = \frac{a}{b} + \left(-\frac{c}{d}\right),\ \forall a,b,c,d\in\mathbb{Z},\,bd\ne0.
	\end{align*}
\end{menhde}

\begin{baitoan}[\cite{SGK_Toan_6_Canh_Dieu_tap_2}, Ví dụ 5, 4, p. 37]
	Tính: (a) $\frac{2}{-9} - \frac{5}{-12}$. (b) $\frac{7}{12} - \frac{-9}{20}$.
\end{baitoan}

\begin{proof}[Giải]
	(a) $\frac{2}{-9} - \frac{5}{-12} = \frac{-2}{9} + \frac{5}{12} = \frac{-2\cdot4}{9\cdot4} + \frac{5\cdot3}{12\cdot3} = \frac{-8}{36} + \frac{15}{36} = \frac{-8 + 15}{36} = \frac{7}{36}$. (b) $\frac{7}{12} - \frac{-9}{20} = \frac{7}{12} + \frac{9}{20} = \frac{7\cdot5}{12\cdot5} + \frac{9\cdot3}{20\cdot3} = \frac{35}{60} + \frac{27}{60} = \frac{35 + 27}{60} = \frac{62}{60} = \frac{31}{30}$.
\end{proof}

\subsection{Quy tắc dấu ngoặc}
Quy tắc dấu ngoặc đối với phân số giống như quy tắc dấu ngoặc đối với số nguyên.

\begin{baitoan}[\cite{SGK_Toan_6_Canh_Dieu_tap_2}, Ví dụ 6, 5, p. 37]
	Tính hợp lý: (a) $\frac{14}{13} + \left(\frac{-1}{13} - \frac{19}{20}\right)$. (b) $\frac{-24}{17} - \left(\frac{-7}{17} - \frac{1}{16}\right)$. (c) $\frac{-2}{49} - \left(\frac{47}{49} + \frac{5}{-3}\right)$.
\end{baitoan}

\begin{proof}[Giải]
	(a) $\frac{14}{13} + \left(\frac{-1}{13} - \frac{19}{20}\right) = \frac{14}{13} + \frac{-1}{13} - \frac{19}{20} = \left(\frac{14}{13} + \frac{-1}{13}\right) - \frac{19}{20} = \frac{14 - 1}{13} - \frac{19}{20} = \frac{13}{13} - \frac{19}{20} = 1 - \frac{19}{20} = \frac{20}{20} - \frac{19}{20} = \frac{20 - 19}{20} = \frac{1}{20}$. (b) $\frac{-24}{17} - \left(\frac{-7}{17} - \frac{1}{16}\right) = \frac{-24}{17} - \frac{-7}{17} + \frac{1}{16} = \left(\frac{-24}{17} + \frac{7}{17}\right) + \frac{1}{16} = \frac{-24 + 7}{17} + \frac{1}{16} = \frac{-17}{17} + \frac{1}{16} = -1 + \frac{1}{16} = \frac{-16}{16} + \frac{1}{16} = \frac{-16 + 1}{16} = \frac{-15}{16}$. (c) $\frac{-2}{49} - \left(\frac{47}{49} + \frac{5}{-3}\right) = \frac{-2}{49} - \frac{47}{49} - \frac{5}{-3} = \left(\frac{-2}{49} - \frac{47}{49}\right) + \frac{5}{3} = \frac{-2 - 47}{49} + \frac{5}{3} = \frac{-49}{49} + \frac{5}{3} = -1 + \frac{5}{3} = \frac{-3}{3} + \frac{5}{3} = \frac{-3 + 5}{3} = \frac{2}{3}$.
\end{proof}

\begin{baitoan}[\cite{SGK_Toan_6_Canh_Dieu_tap_2}, 1., p. 38]
	Tính: (a) $\frac{-2}{9} + \frac{7}{-9}$. (b) $\frac{1}{-6} + \frac{13}{-15}$. (c) $\frac{5}{-6} + \frac{-5}{12} + \frac{7}{18}$.
\end{baitoan}

\begin{baitoan}[\cite{SGK_Toan_6_Canh_Dieu_tap_2}, 2., p. 38]
	Tính hợp lý: (a) $\frac{2}{9} + \frac{-3}{10} + \frac{-7}{10}$. (b) $\frac{-11}{6} + \frac{2}{5} + \frac{-1}{6}$. (c) $\frac{-5}{8} + \frac{12}{7} + \frac{13}{8} + \frac{2}{7}$.
\end{baitoan}

\begin{baitoan}[\cite{SGK_Toan_6_Canh_Dieu_tap_2}, 3., p. 38]
	Tìm số đối của mỗi phân số sau: $\frac{9}{25},\frac{-8}{27},-\frac{15}{31},\frac{-3}{-5},\frac{5}{-6}$.
\end{baitoan}

\begin{baitoan}[\cite{SGK_Toan_6_Canh_Dieu_tap_2}, 4., p. 38]
	Tính: (a) $\frac{5}{16} - \frac{5}{24}$. (b) $\frac{2}{11} + \left(\frac{-5}{11} - \frac{9}{11}\right)$. (c) $\frac{1}{10} - \left(\frac{5}{12} - \frac{1}{15}\right)$.
\end{baitoan}

\begin{baitoan}[\cite{SGK_Toan_6_Canh_Dieu_tap_2}, 5., p. 38]
	Tính hợp lý: (a) $\frac{27}{13} - \frac{106}{111} + \frac{-5}{111}$. (b) $\frac{12}{11} - \frac{-7}{19} + \frac{12}{19}$. (c) $\frac{5}{17} - \frac{25}{31} + \frac{12}{17} + \frac{-6}{31}$.
\end{baitoan}

\begin{baitoan}[\cite{SGK_Toan_6_Canh_Dieu_tap_2}, 6., p. 38]
	Tìm $x$: (a) $x - \frac{5}{6} = \frac{1}{2}$. (b) $\frac{-3}{4} - x = \frac{-7}{12}$.
\end{baitoan}

\begin{baitoan}[\cite{SGK_Toan_6_Canh_Dieu_tap_2}, 7., p. 38]
	1 xí nghiệp trong tháng Giêng đạt $\frac{3}{8}$ kế hoạch của Quý I, tháng 2 đạt $\frac{2}{7}$ kế hoạch của Quý I. Tháng 3 xí nghiệp phải đạt được bao nhiêu phần kế hoạch của Quý I?
\end{baitoan}

\begin{baitoan}[\cite{SGK_Toan_6_Canh_Dieu_tap_2}, 8., p. 38]
	4 tổ của lớp 6A đóng góp sách cho góc thư viện như sau: tổ I góp $\frac{1}{4}$ số sách của lớp, tổ II góp $\frac{9}{40}$ số sách của lớp, tổ III góp $\frac{1}{5}$ số sách của lớp, tổ IV góp phần sách còn lại. Tổ IV đã góp bao nhiêu phần số sách của lớp?
\end{baitoan}
\noindent\textsf{\textbf{Tóm tắt kiến thức.}} ``\fbox{\bf 1} \textit{Phép cộng 2 phân số}: 2 phân số cùng mẫu: $\frac{a}{m} + \frac{b}{m} = \frac{a + b}{m}$, $\forall a,b,m\in\mathbb{Z}$, $m\ne0$. Nếu 2 phân số khác mẫu, ta quy đồng về cùng mẫu rồi cộng các tử \& giữ nguyên mẫu chung. Tính chất của phép cộng phân số: giao hoán, kết hợp, cộng với số $0$. \fbox{\bf 2} \textit{Phép trừ phân số}: Muốn trừ 2 phân số ta cộng số bị trừ với số đối của số trừ: $\frac{a}{b} - \frac{c}{d} = \frac{a}{b} + \left(-\frac{c}{d}\right)$, $\forall a,b,c,d\in\mathbb{Z}$, $bd\ne0$.'' -- \cite[Chap. V, \S3, p. 36]{SBT_Toan_6_Canh_Dieu_tap_2}

\begin{baitoan}[\cite{SBT_Toan_6_Canh_Dieu_tap_2}, Ví dụ 1, p. 36]
	Tính tổng $\frac{3}{7} + \frac{-2}{3}$, từ đó có thể suy ra ngay kết quả của các phép cộng sau: $\frac{21}{49} + \frac{-12}{18}$ \& $\frac{-27}{-63} + \frac{-22}{33}$ được không? Vì sao?
\end{baitoan}

\begin{proof}[Giải]
	Có $\frac{3}{7} + \frac{-2}{3} = \frac{3\cdot3}{7\cdot3} + \frac{-2\cdot7}{3\cdot7} = \frac{9 + (-14)}{21} = \frac{-5}{21}$. Vậy $\frac{3}{7} + \frac{-2}{3} = \frac{-5}{21}$. Từ đó có thể suy ra ngay kết quả các phép cộng: $\frac{21}{49} + \frac{-12}{18} = \frac{-5}{21}$ vì $\frac{21}{49} = \frac{3}{7}$ \& $\frac{-12}{18} = \frac{-2}{3}$; $\frac{-27}{-63} + \frac{-22}{33} = \frac{-5}{21}$ vì $\frac{-27}{-63} = \frac{3}{7}$ \& $\frac{-22}{33} = \frac{-2}{3}$.
\end{proof}

\begin{baitoan}[\cite{SBT_Toan_6_Canh_Dieu_tap_2}, Ví dụ 2, p. 36]
	1 hình chữ nhật có chiều rộng là $\frac{3}{5}$\emph{m}, chiều dài hơn chiều rộng $\frac{1}{4}$\emph{m}. Tính nửa chu vi của hình chữ nhật đó.
\end{baitoan}

\begin{proof}[Giải]
	Chiều dài hình chữ nhật bằng $\frac{3}{5} + \frac{1}{4} = \frac{3\cdot4}{5\cdot4} + \frac{1\cdot5}{4\cdot5} = \frac{12 + 5}{20} = \frac{17}{20}$m. Nửa chu vi hình chữ nhật bằng $\frac{3}{5} + \frac{17}{20} = \frac{3\cdot4}{5\cdot4} + \frac{17}{20} = \frac{12 + 17}{20} = \frac{29}{20}$m.
\end{proof}

\begin{baitoan}[\cite{SBT_Toan_6_Canh_Dieu_tap_2}, 27., p. 37]
	Tính hợp lý: (a) $\frac{7}{-27} + \frac{-8}{27}$. (b) $\frac{6}{13} + \frac{-17}{39}$. (c) $\frac{-17}{13} + \frac{25}{101} + \frac{4}{13}$. (d) $\frac{-13}{7} + \frac{3}{5} + \frac{-1}{7}$. (e) $\frac{-5}{9} + \frac{8}{15} + \frac{4}{-9} + \frac{7}{15}$.
\end{baitoan}

\begin{baitoan}[\cite{SBT_Toan_6_Canh_Dieu_tap_2}, 28., p. 37]
	So sánh các biểu thức: (a) $A = \frac{1}{2} + \frac{-3}{8} + \frac{5}{9}$ \& $B = \frac{13}{-30} + \frac{17}{45} + \frac{-7}{18}$. (b) $C = \frac{12}{25} + \frac{-8}{15} + \frac{-4}{9}$ \& $D = \frac{-5}{12} + \frac{4}{9} + \frac{11}{-6}$. (c) $M = \frac{1}{3} + \frac{2}{-5} + \frac{7}{2}$ \& $N = \frac{19}{-7} + \frac{21}{5} + \frac{-2}{7}$. (d) $P = \frac{34}{24} + \frac{-8}{15} + \frac{1}{10}$ \& $Q = \frac{8}{21} + 1 + \frac{1}{-21}$.
\end{baitoan}

\begin{baitoan}[\cite{SBT_Toan_6_Canh_Dieu_tap_2}, 29., p. 37]
	Không tính trực tiếp, chứng tỏ tổng của 3 phân số sau: $\frac{20}{11},\frac{20}{31},\frac{20}{51}$ nhỏ hơn $\frac{7}{2}$.
\end{baitoan}

\begin{baitoan}[\cite{SBT_Toan_6_Canh_Dieu_tap_2}, 30., pp. 37--38]
	Tính: (a) $\frac{-4}{5} + \frac{9}{7}$. (b) $\frac{7}{21} + \frac{9}{-36}$. (c) $1 + \frac{-1}{11}$. (d) $\frac{11}{15} + \frac{9}{-10}$. (e) $-\frac{18}{24} + \frac{15}{-21}$. (f) $\frac{-3}{10} + \frac{7}{24}$. (g) $\frac{1}{2} + \frac{-1}{3}$. (h) $\frac{-3}{21} + \frac{6}{42}$. (i) $2 + \frac{7}{-9}$. (j) $\frac{2}{7} - \frac{85}{77}$.
\end{baitoan}

\begin{baitoan}[\cite{SBT_Toan_6_Canh_Dieu_tap_2}, 31., p. 38]
	Tìm $x\in\mathbb{Z}$ biết: (a) $\frac{-5}{7} + 1 + \frac{30}{-7}\le x\le\frac{-1}{6} + \frac{1}{3} + \frac{5}{6}$. (b) $\frac{-8}{13} + \frac{7}{17} + \frac{21}{13}\le x\le\frac{-9}{14} + 3 + \frac{5}{-14}$.
\end{baitoan}

\begin{baitoan}[\cite{SBT_Toan_6_Canh_Dieu_tap_2}, 32., p. 38]
	Tìm tổng các phân số đồng thời lớn hơn $\frac{-1}{2}$, nhỏ hơn $\frac{-1}{3}$ \& có tử là $5$.
\end{baitoan}

\begin{baitoan}[\cite{SBT_Toan_6_Canh_Dieu_tap_2}, 33., p. 38]
	3 ô tô cùng chuyển long nhãn từ 1 kho ở Hưng Yên lên Hà Nội. Ô tô thứ nhất, thứ 2, thứ 3 chuyển được lần lượt $\frac{1}{3},\frac{1}{5},\frac{2}{9}$ số long nhãn trong kho. Cả 3 ô tô chuyển được bao nhiêu phần long nhãn trong kho?
\end{baitoan}

\begin{baitoan}[\cite{SBT_Toan_6_Canh_Dieu_tap_2}, 34., p. 38]
	Người thứ nhất đi xe đạp từ A đến B hết $5$ giờ, người thứ 2 đi xe máy từ B về A hết $2$ giờ, người thứ 2 khởi hành sau người thứ nhất $2$ giờ. Hỏi sau khi người thứ 2 đi được $1$ giờ thì 2 người đã gặp nhau chưa?
\end{baitoan}

\begin{baitoan}[\cite{SBT_Toan_6_Canh_Dieu_tap_2}, 35., p. 38]
	1 người hỏi Pythagore về số học trò của ổng. Pythagore nói: ``1 nửa số học trò của tôi đang học Toán, $\frac{1}{4}$ đang học Nhạc, $\frac{1}{7}$ đang ngồi suy nghĩ. Số còn lại là $3$ người.'' Pythagore có bao nhiêu học trò?
\end{baitoan}

\begin{baitoan}[\cite{SBT_Toan_6_Canh_Dieu_tap_2}, 36., p. 38]
	Có $5$ quả cam chia đều cho $6$ người. Làm thế nào để chia được mà không phải cắt bất kỳ quả cam nào thành $6$ phần bằng nhau?
\end{baitoan}

\subsection{Biểu diễn phân số trên trục số nằm ngang}
Tương tự như đối với các số nguyên, ta có thể biểu diễn mọi phân số trên trục số. Trên trục số, phân số \& số đối của phân số đó có điểm biểu diễn nằm về 2 phía của gốc 0 \& cách đều góc 0. Trên trục số nằm ngang, nếu điểm biểu diễn phân số $\frac{a}{b}$ nằm bên trái điểm biểu diễn phân số $\frac{c}{d}$ (hay điểm biểu diễn phân số $\frac{c}{d}$ nằm bên phải điểm biểu diễn phân số $\frac{a}{b}$) thì ta có phân số $\frac{a}{b}$ nhỏ hơn phân số $\frac{c}{d}$, i.e., $\frac{a}{b} < \frac{c}{d}$ (hay phân số $\frac{c}{d}$ lớn hơn phân số $\frac{a}{b}$, i.e., $\frac{c}{d} > \frac{a}{b}$. Tính chất bắc cầu: $\frac{a}{b} < \frac{c}{d}$ \& $\frac{c}{d} < \frac{e}{f}\Rightarrow\frac{a}{b} < \frac{e}{f}$.

%------------------------------------------------------------------------------%

\section{Phép Nhân, Phép Chia Phân Số}

\begin{baitoan}[\cite{SGK_Toan_6_Canh_Dieu_tap_2}, p. 40]
	Gấu nước được nhà sinh vật học người Ý L. Spallanzani đặt tên là Tardigrada vào nằm 1776. 1 con gấu nước dài $\approx\frac{1}{2}$\emph{mm}. 1 con gấu đực Bắc Cực trưởng thành dài $\approx\frac{5}{2}$\emph{m}. Chiều dài con gấu đực Bắc Cực trưởng thành gấp bao nhiêu lần chiều dài con gấu nước?
\end{baitoan}

\subsection{Phép nhân phân số}

\begin{menhde}
	Muốn nhân 2 phân số, ta nhân các tử với nhau \& nhân các mẫu với nhau:
	\begin{align*}
		\frac{a}{b}\cdot\frac{c}{d} &= \frac{ac}{bd},\ \forall a,b,c,d\in\mathbb{Z},\,bd\ne0,\\
		\prod_{i=1}^n \frac{a_i}{b_i} = \frac{a_1}{b_1}\cdot\frac{a_2}{b_2}\cdots\frac{a_n}{b_n} &= \frac{a_1a_2\cdots a_n}{b_1b_2\cdots b_n} = \frac{\prod_{i=1}^n a_i}{\prod_{i=1}^n b_i},\ \forall a_i,b_i\in\mathbb{Z},\,b_i\ne0,\ \forall i = 1,2,\ldots,n.
	\end{align*}
\end{menhde}

\begin{vidu}
	(a) $\frac{91}{2}\cdot\frac{3}{4} = \frac{91\cdot3}{2\cdot4} = \frac{273}{8}$. (b) $\frac{-2}{5}\cdot\frac{3}{7} = \frac{-2\cdot3}{5\cdot7} = \frac{-6}{35}$.
\end{vidu}

\begin{baitoan}[\cite{SGK_Toan_6_Canh_Dieu_tap_2}, Ví dụ 1, p. 40]
	Tính tích \& viết kết quả ở dạng phân số tối giản: (a) $\frac{-2}{5}\cdot\frac{3}{7}$. (b) $\frac{-8}{9}\cdot\frac{3}{-4}$. (c) $\frac{-9}{10}\cdot\frac{25}{12}$. (d) $\left(-\frac{3}{8}\right)\cdot\frac{-12}{5}$.
\end{baitoan}

\begin{proof}[Giải]
	(a) $\frac{-2}{5}\cdot\frac{3}{7} = \frac{-2\cdot3}{5\cdot7} = \frac{-6}{35}$. (b) $\frac{-8}{9}\cdot\frac{3}{-4} = \frac{-8\cdot3}{9\cdot(-4)} = \frac{-24}{-36} = \frac{-24:12}{-36:12} = \frac{2}{3}$. (c) $\frac{-9}{10}\cdot\frac{25}{12} = \frac{-9\cdot25}{10\cdot12} = \frac{-225}{120} = \frac{-225:15}{120:15} = \frac{-15}{8}$. (d) $\left(-\frac{3}{8}\right)\cdot\frac{-12}{5} = \frac{-3\cdot(-12)}{8\cdot5} = \frac{36}{40} = \frac{36:4}{40:4} = \frac{9}{10}$.
\end{proof}

\begin{menhde}[Phép nhân số nguyên với phân số]
	Muốn nhân 1 số nguyên với 1 phân số (hoặc nhân 1 phân số với 1 số nguyên)\footnote{2 điều này tương đương với nhau do tính chất giao hoán của phép nhân phân số.}, ta nhân số nguyên với tử của phân số \& giữ nguyên mẫu của phân số đó:
	\begin{align*}
		m\cdot\frac{a}{b} = \frac{ma}{b},\ \frac{a}{b}\cdot n = \frac{an}{b},\ \forall a,b\in\mathbb{Z},\,b\ne0.
	\end{align*}
\end{menhde}

\begin{baitoan}[\cite{SGK_Toan_6_Canh_Dieu_tap_2}, Ví dụ 2, 2, p. 41]
	Tính tích \& viết kết quả ở dạng phân số tối giản: (a) $4\cdot\frac{-5}{9}$. (b) $\frac{-7}{11}\cdot(-9)$. (c) $8\cdot\frac{-5}{6}$. (d) $\frac{5}{21}\cdot(-14)$.
\end{baitoan}

\begin{proof}[Giải]
	(a) $4\cdot\frac{-5}{9} = \frac{4\cdot(-5)}{9} = \frac{-20}{9}$. (b) $\frac{-7}{11}\cdot(-9) = \frac{-7\cdot(-9)}{11} = \frac{63}{11}$. (c) $8\cdot\frac{-5}{6} = \frac{8\cdot(-5)}{6} = \frac{-40}{6} = \frac{-40:2}{6:2} = \frac{-20}{3}$. (d) $\frac{5}{21}\cdot(-14) = \frac{5\cdot(-14)}{21} = \frac{-70}{21} = \frac{-70:7}{21:7} = \frac{-10}{3}$.
\end{proof}

\subsubsection{Tính chất của phép nhân phân số}
Giống như phép nhân số tự nhiên, phép nhân phân số cũng có các tính chất: giao hoán, kết hợp, nhân với số 1, phân phối của phép nhân đối với phép cộng \& phép trừ.
\begin{itemize}
	\item \textit{Giao hoán}: $\frac{a}{b}\cdot\frac{c}{d} =  \frac{c}{d}\cdot\frac{a}{b}$, $\forall a,b,c,d\in\mathbb{Z}$, $bd\ne0$ (chú ý: $bd\ne0\Leftrightarrow b\ne0$ \& $d\ne0$).
	\item \textit{Kết hợp}: $\left(\frac{a}{b}\cdot\frac{c}{d}\right)\cdot\frac{e}{f} = \frac{a}{b}\cdot\left(\frac{c}{d}\cdot\frac{e}{f}\right) = \frac{a}{b}\cdot\frac{c}{d}\cdot\frac{e}{f}$, $\forall a,b,c,d,e,f\in\mathbb{Z}$, $bdf\ne0$ (chú ý: $bdf\ne0\Leftrightarrow b\ne0$ \& $d\ne0$ \& $f\ne0$).
	\item \textit{Nhân với số $1$}: $\frac{a}{b}\cdot1 = 1\cdot\frac{a}{b} = \frac{a}{b}$, $\forall a,b\in\mathbb{Z}$, $b\ne0$.
	\item \textit{Phân phối của phép nhân đối với phép cộng \& phép trừ}: $\frac{a}{b}\left(\frac{c}{d} + \frac{e}{f}\right) = \frac{a}{b}\cdot\frac{c}{d} + \frac{a}{b}\cdot\frac{e}{f}$, $\frac{a}{b}\left(\frac{c}{d} - \frac{e}{f}\right) = \frac{a}{b}\cdot\frac{c}{d} - \frac{a}{b}\cdot\frac{e}{f}$, có thể viết gom chung lại thành$\frac{a}{b}\left(\frac{c}{d}\pm\frac{e}{f}\right) = \frac{a}{b}\cdot\frac{c}{d}\pm\frac{a}{b}\cdot\frac{e}{f}$, $\forall a,b,c,d,e,f\in\mathbb{Z}$, $bdf\ne0$.
\end{itemize}

\begin{baitoan}[\cite{SGK_Toan_6_Canh_Dieu_tap_2}, Ví dụ 3, 3, p. 41]
	Tính hợp lý: (a) $\frac{2}{5}\cdot\frac{-3}{7}\cdot\frac{-5}{2}$. (b) $\frac{5}{7}\cdot\frac{-2}{11} - \frac{5}{7}\cdot\frac{9}{11}$. (c) $\frac{-9}{7}\cdot\left(\frac{14}{15} - \frac{-7}{9}\right)$.
\end{baitoan}

\begin{proof}[Giải]
	(a) $\frac{2}{5}\cdot\frac{-3}{7}\cdot\frac{-5}{2} = \frac{2}{5}\cdot\frac{-5}{2}\cdot\frac{-3}{7} = \left(\frac{2}{5}\cdot\frac{-5}{2}\right)\cdot\frac{-3}{7} = -1\cdot\frac{-3}{7} = \frac{3}{7}$. (b) $\frac{5}{7}\cdot\frac{-2}{11} - \frac{5}{7}\cdot\frac{9}{11} = \frac{5}{7}\cdot\left(\frac{-2}{11} - \frac{9}{11}\right) = \frac{5}{7}\cdot\frac{-11}{11} = \frac{5}{7}\cdot(-1) = \frac{-5}{7}$. (c) $\frac{-9}{7}\cdot\left(\frac{14}{15} - \frac{-7}{9}\right) = \frac{-9}{7}\cdot\frac{14}{15} - \frac{-9}{7}\cdot\frac{-7}{9} = \frac{-9\cdot14}{7\cdot15} - \frac{-9\cdot(-7)}{7\cdot9} = \frac{-6}{5} - 1 = \frac{-6}{5} - \frac{5}{5} = \frac{-6 - 5}{5} = \frac{-11}{5}$.
\end{proof}

\subsection{Phép chia phân số}

\begin{dinhnghia}[Phân số nghịch đảo]
	Phân số $\frac{b}{a}$ gọi là \emph{phân số nghịch đảo} của phan số $\frac{a}{b}$, $\forall a,b\in\mathbb{Z}$, $b\ne0$.
\end{dinhnghia}

\begin{vidu}[\cite{SGK_Toan_6_Canh_Dieu_tap_2}, Ví dụ 4, 4, p. 42]
	(a) Phân số nghịch đảo của phân số $\frac{7}{3}$ là phân số $\frac{3}{7}$. (b) Phân số nghịch đảo của phân số $\frac{-7}{9}$ là phân số $\frac{9}{-7}$. (c) Phân số nghịch đảo của phân số $\frac{2}{-13}$ là phân số $\frac{-13}{2}$. (d) Phân số nghịch đảo của phân số $\frac{-4}{11}$ là $\frac{11}{-4}$. (e) Phân số nghịch đảo của phân số $\frac{7}{-17}$ là $\frac{-17}{7}$.
\end{vidu}

\begin{menhde}
	Tích của 1 phân số với phân số nghịch đảo của nó thì bằng $1$, i.e., $\frac{a}{b}\cdot\frac{b}{a} = 1$, $\forall a,b\in\mathbb{Z}$, $ab\ne0$.
\end{menhde}

\begin{menhde}
	Muốn chia 1 phân số cho 1 phân số khác $0$, ta nhân số bị chia với phân số nghịch đảo của số chia: $\frac{a}{b}:\frac{c}{d} = \frac{a}{b}\cdot\frac{d}{c} = \frac{ad}{bc}$, $\forall a,b,c,d\in\mathbb{Z}$, $bcd\ne0$.
\end{menhde}

\begin{baitoan}[\cite{SGK_Toan_6_Canh_Dieu_tap_2}, Ví dụ 5, 5, p. 42]
	Tính thương \& viết kết quả ở dạng phân số tối giản: (a) $\frac{2}{3}:\frac{-4}{5}$. (b) $(-5):\frac{-3}{7}$. (c) $\frac{-9}{5}:\frac{8}{3}$. (d) $\frac{-7}{9}:(-5)$.
\end{baitoan}

\begin{proof}[Giải]
	(a) $\frac{2}{3}:\frac{-4}{5} = \frac{2}{3}\cdot\frac{5}{-4} = \frac{10}{-12} = \frac{10:(-2)}{-12:(-2)} = \frac{-5}{6}$. (b) $-5:\frac{-3}{7} =  -5\cdot\frac{7}{-3} = \frac{-5\cdot7}{-3} = \frac{-35}{-3} = \frac{35}{3}$. (c) $\frac{-9}{5}:\frac{8}{3} = \frac{-9}{5}\cdot\frac{3}{8} = \frac{-9\cdot3}{5\cdot8} = \frac{-27}{40}$. (d) $\frac{-7}{9}:(-5) = \frac{-7}{9}:(-5)\cdot\frac{1}{-5} = \frac{-7}{9\cdot(-5)} = \frac{-7}{-45} = \frac{7}{45}$.
\end{proof}

\begin{menhde}[Phép chia số nguyên với phân số]
	Muốn chia 1 số nguyên với 1 phân số (hoặc chia 1 phân số với 1 số nguyên)\footnote{Chú ý 2 điều này không tương đương với nhau do phép chia phân số không có tính chất giao hoán như phép nhân phân số, i.e., 1 cách tổng quát, có: $\frac{a}{b}:\frac{c}{d} = \frac{ad}{bc}\ne\frac{c}{d}:\frac{a}{b} = \frac{bc}{ad}$.}:
	\begin{align*}
		a:\frac{c}{d} = a\cdot\frac{d}{c} = \frac{ad}{c},\ \forall a,c,d\in\mathbb{Z},\,cd\ne0,\\
		\frac{a}{b}:c = \frac{a}{b}\cdot\frac{1}{c} = \frac{a}{bc},\ \forall a,b,c\in\mathbb{Z},\,bc\ne0.
	\end{align*}
\end{menhde}
Thứ tự thực hiện các phép tính với phân số (trong biểu thức không chứa dấu ngoặc hoặc có chứa dấu ngoặc) cũng giống như thứ tự thực hiện các phép tính với số nguyên: $()\to[]\to\{\}$, $\widehat{}\to\cdot,:\to\pm$.

\begin{baitoan}[\cite{SGK_Toan_6_Canh_Dieu_tap_2}, 1., p. 43]
	Tính tích \& viết kết quả ở dạng phân số tối giản: (a) $\frac{-5}{9}\cdot\frac{12}{35}$. (b) $\left(-\frac{5}{8}\right)\cdot\frac{-6}{55}$. (c) $-7\cdot\frac{2}{5}$. (d) $\frac{-3}{8}\cdot(-6)$.
\end{baitoan}

\begin{proof}[Giải]
	$\frac{-5}{9}\cdot\frac{12}{35} = \frac{-5\cdot12}{9\cdot35} = \frac{-60}{315} = \frac{-60:15}{315:15} = \frac{-4}{21}$. (b) $\left(-\frac{5}{8}\right)\cdot\frac{-6}{55} = \frac{-5\cdot(-6)}{8\cdot55} = \frac{30}{440} = \frac{30:10}{440:10} = \frac{3}{44}$. (c) $-7\cdot\frac{2}{5} = \frac{-7\cdot2}{5} = -\frac{14}{5}$. (d) $\frac{-3}{8}\cdot(-6) = \frac{-3\cdot(-6)}{8} = \frac{18}{8} = \frac{18:2}{8:2} = \frac{9}{4}$.
\end{proof}

\begin{baitoan}[\cite{SGK_Toan_6_Canh_Dieu_tap_2}, 2., p. 43]
	Tìm $x$: (a) $\frac{-2}{3}\cdot\frac{x}{4} = \frac{1}{2}$. (b) $\frac{x}{3}\cdot\frac{5}{8} = \frac{-5}{12}$. (c) $\frac{5}{6}\cdot\frac{3}{x} = \frac{1}{4}$.
\end{baitoan}

\begin{proof}[Giải]
	(a) $\frac{-2}{3}\cdot\frac{x}{4} = \frac{1}{2}\Leftrightarrow\frac{-2x}{3\cdot4} = \frac{1}{2}\Leftrightarrow\frac{-x}{6} = \frac{1}{2}\Leftrightarrow x = -6\cdot\frac{1}{2} = -3$. Vậy $x = -3$. (b) $\frac{x}{3}\cdot\frac{5}{8} = \frac{-5}{12}\Leftrightarrow x = \frac{-5}{12}:\frac{5}{8}\cdot3 = \frac{-5}{12}\cdot\frac{8}{5}\cdot3 = \frac{-5\cdot8\cdot3}{12\cdot5} = -2$. Vậy $x = -2$. (c) $\frac{5}{6}\cdot\frac{3}{x} = \frac{1}{4}\Leftrightarrow\frac{5\cdot3}{6x} = \frac{1}{4}\Leftrightarrow6x = 5\cdot3\cdot4\Leftrightarrow x = \frac{5\cdot3\cdot4}{6} = 10$. Vậy $x = 10$.
\end{proof}

\begin{baitoan}[\cite{SGK_Toan_6_Canh_Dieu_tap_2}, 3., p. 43]
	Tìm phân số nghịch đảo của mỗi phân số sau: (a) $\frac{-9}{19}$. (b) $-\frac{21}{13}$. (c) $\frac{1}{-9}$.
\end{baitoan}

\begin{proof}[Giải]
	Phân số nghịch đảo của $\frac{-9}{19},-\frac{21}{13},\frac{1}{-9}$ lần lượt là $\frac{19}{-9},-\frac{13}{21},\frac{-9}{1} = -9$. 
\end{proof}

\begin{baitoan}[\cite{SGK_Toan_6_Canh_Dieu_tap_2}, 4., p. 43]
	Tính thương \& viết kết quả ở dạng phân số tối giản: (a) $\frac{3}{10}:\left(\frac{-2}{3}\right)$. (b) $\left(-\frac{7}{12}\right):\left(-\frac{5}{6}\right)$. (c) $-15:\frac{-9}{10}$.
\end{baitoan}

\begin{proof}[Giải]
	(a) $\frac{3}{10}:\left(\frac{-2}{3}\right) = \frac{3}{10}\cdot\frac{3}{-2} = \frac{3\cdot3}{10\cdot(-2)} = \frac{9}{-20} = -\frac{9}{20}$. (b) $\left(-\frac{7}{12}\right):\left(-\frac{5}{6}\right) = \frac{7}{12}\cdot\frac{6}{5} = \frac{7\cdot6}{12\cdot5} = \frac{42}{60} = \frac{42:6}{60:6} = \frac{7}{10}$. (c) $-15:\frac{-9}{10} = 15\cdot\frac{10}{9} = \frac{15\cdot10}{9} = \frac{150}{9} = \frac{150:3}{9:3} = \frac{50}{3}$.
\end{proof}

\begin{baitoan}[\cite{SGK_Toan_6_Canh_Dieu_tap_2}, 5., p. 43]
	Tìm $x$: (a) $\frac{3}{16}:\frac{x}{8} = \frac{3}{4}$. (b) $\frac{1}{25}:\frac{-3}{x} = \frac{-1}{15}$. (c) $\frac{x}{12}:\frac{-4}{9} = \frac{-3}{16}$.
\end{baitoan}

\begin{baitoan}[\cite{SGK_Toan_6_Canh_Dieu_tap_2}, 6., p. 43]
	Tìm $x$: (a) $\frac{4}{7}\cdot x - \frac{2}{3} = \frac{1}{5}$. (b) $\frac{4}{5} + \frac{5}{7}:x = \frac{1}{6}$.
\end{baitoan}

\begin{proof}[Giải]
	(a) $\frac{4}{7}\cdot x - \frac{2}{3} = \frac{1}{5}\Leftrightarrow\frac{4}{7}\cdot x = \frac{1}{5} + \frac{2}{3} = \frac{1\cdot3}{5\cdot3} + \frac{2\cdot5}{3\cdot5} = \frac{3}{15} + \frac{10}{15} = \frac{3 + 10}{15} = \frac{13}{15}\Leftrightarrow x = \frac{13}{15}:\frac{4}{7} = \frac{13}{15}\cdot\frac{7}{4} = \frac{91}{60}$. Vậy $x = \frac{91}{60}$. (b) $\frac{4}{5} + \frac{5}{7}:x = \frac{1}{6}\Leftrightarrow\frac{5}{7}:x  \frac{1}{6} - \frac{4}{5} = \frac{1\cdot5}{6\cdot5} - \frac{4\cdot6}{5\cdot6} = \frac{5}{30} - \frac{24}{30} = \frac{5 - 24}{30} = \frac{-19}{30}\Leftrightarrow x = \frac{5}{7}:\frac{-19}{30} = \frac{5}{7}\cdot\frac{30}{-19} = \frac{5\cdot30}{7\cdot(-19)} = -\frac{150}{133}$. Vậy $x = -\frac{150}{133}$.
\end{proof}

\begin{baitoan}[\cite{SGK_Toan_6_Canh_Dieu_tap_2}, 7., p. 43]
	Tính: (a) $\frac{17}{8}:\left(\frac{27}{8} + \frac{-11}{2}\right)$. (b) $\frac{28}{15}\cdot\frac{1}{4^2}\cdot3 + \left(\frac{8}{15} - \frac{69}{60}\cdot\frac{5}{23}\right):\frac{51}{54}$.
\end{baitoan}

\begin{proof}[Giải]
	(a) $\frac{17}{8}:\left(\frac{27}{8} + \frac{-11}{2}\right) = \frac{17}{8}:\left(\frac{27}{8} + \frac{-11\cdot4}{2\cdot4}\right) = \frac{17}{8}:\frac{27 - 44}{8} = \frac{17}{8}:\frac{-17}{8} = \frac{17}{8}\cdot\frac{8}{-17} = \frac{17\cdot8}{8\cdot(-17)} = -1$. (b) $\frac{28}{15}\cdot\frac{1}{4^2}\cdot3 + \left(\frac{8}{15} - \frac{69}{60}\cdot\frac{5}{23}\right):\frac{51}{54}$ $= \frac{28\cdot3}{15\cdot4^2} + \left(\frac{8}{15} - \frac{69\cdot5}{60\cdot23}\right)\cdot\frac{54}{51} = \frac{4\cdot7\cdot3}{3\cdot5\cdot4^2} + \left(\frac{8}{15} - \frac{23\cdot3\cdot5}{3\cdot4\cdot5\cdot23}\right)\cdot\frac{3\cdot18}{3\cdot17} = \frac{7}{20} + \left(\frac{8}{15} - \frac{1}{4}\right)\cdot\frac{18}{17} = \frac{7}{20} + \left(\frac{8\cdot4}{15\cdot4} - \frac{1\cdot15}{4\cdot15}\right)\cdot\frac{18}{17} = \frac{7}{20} + \frac{32 - 15}{60}\cdot\frac{18}{17} = \frac{7}{20} + \frac{17}{60}\cdot\frac{18}{17} = \frac{7}{20} + \frac{18}{60} = \frac{7}{20} + \frac{6}{20} = \frac{7 + 6}{20} = \frac{13}{20}$.
\end{proof}

\begin{baitoan}[\cite{SGK_Toan_6_Canh_Dieu_tap_2}, 8., p. 43]
	Chim ruồi ong hiện là loài chim bé nhỏ nhất trên Trái Đất với chiều dài chỉ khoảng $5$\emph{cm}. Chim ruồi ``khổng lồ'' ở Nam Mỹ là thành viên lớn nhất của gia đình chim ruồi trên thế giới, nó dài gấp $\frac{33}{8}$ lần chim ruồi ong. Tính chiều dài của chim ruồi ``khổng lồ'' ở Nam Mỹ.
\end{baitoan}
\noindent\textsf{\textbf{Tóm tắt kiến thức.}} ``\fbox{\bf 1} \textit{Phép nhân phân số}: Quy tắc nhân 2 phân số: $\frac{a}{b}\cdot\frac{c}{d} = \frac{ac}{bd}$, $\forall a,b,c,d\in\mathbb{Z}$, $bd\ne0$; $m\cdot\frac{a}{b} = \frac{ma}{b}$, $\forall a,b,m\in\mathbb{Z}$, $b\ne0$; $\frac{a}{b}\cdot n = \frac{an}{b}$, $\forall a,b,n\in\mathbb{Z}$, $b\ne0$. Tính chất của phép nhân phân số: giao hoán, kết hợp, nhân với số $1$, phân phối của phép nhân đối với phép cộng \& phép trừ. Phân số $\frac{b}{a}$ gọi là \emph{phân số nghịch đảo} của phân số $\frac{a}{b}$ với $ab\ne0$. \fbox{\bf 2} \textit{Phép chia phân số}: $\frac{a}{b}:\frac{c}{d} = \frac{a}{b}\cdot\frac{d}{c} = \frac{ad}{bc}$, $\forall a,b,c,d\in\mathbb{Z}$, $bcd\ne0$; $m:\frac{a}{b} = \frac{mb}{a}$, $\forall a,b,m\in\mathbb{Z}$, $ab\ne0$; $\frac{a}{b}:n = \frac{a}{bn}$, $\forall a,b,n\in\mathbb{Z}$, $bn\ne0$.'' -- \cite[Chap. V, \S4, p. 39]{SBT_Toan_6_Canh_Dieu_tap_2}

\begin{baitoan}[\cite{SBT_Toan_6_Canh_Dieu_tap_2}, Ví dụ 1, p. 39]
	Tính: (a) $\frac{-5}{14}\cdot(-91)$. (b) $\left(1 - \frac{1}{6}\right)\left(\frac{-3}{17} + \frac{1}{5}\right)$.
\end{baitoan}

\begin{proof}[Giải]
	(a) $\frac{-5}{14}\cdot(-91) = \frac{5\cdot91}{14} = \frac{65}{2}$. (b) $\left(1 - \frac{1}{6}\right)\left(\frac{-3}{17} + \frac{1}{5}\right) = \frac{5}{6}\cdot\left(\frac{-3\cdot5}{17\cdot5} + \frac{1\cdot17}{5\cdot17}\right) = \frac{5}{6}\cdot\frac{2}{85} = \frac{1}{51}$.
\end{proof}

\begin{baitoan}[\cite{SBT_Toan_6_Canh_Dieu_tap_2}, Ví dụ 2, p. 39]
	Theo \url{http://danso.org}, tính đến ngày \emph{29\texttt{/}4\texttt{/}2021}, dân số thế giới có khoảng $7840$ triệu người \& dân số Việt Nam chiếm $\frac{1}{80}$ dân số thế giới. Dân số Việt Nam khi đó là khoảng bao nhiêu triệu người?
\end{baitoan}

\begin{proof}[Giải]
	Dân số Việt Nam khi đó là khoảng $\frac{1}{80}\cdot7840 = 98$ triệu người.
\end{proof}

\begin{baitoan}[\cite{SBT_Toan_6_Canh_Dieu_tap_2}, Ví dụ 3, p. 40]
	Trong $327$ ngày $12$ giờ, Mặt Trăng quay $12$ vòng xung quanh Trái Đất. Mặt Trăng quay 1 vòng xung quanh Trái Đất trong bao nhiêu ngày?
\end{baitoan}

\begin{proof}[Giải]
	327 ngày 12 giờ $= \frac{655}{2}$ ngày. Mặt Trăng quay 1 vòng xung quanh Trái Đất trong $\frac{655}{2}:12 = \frac{655}{24} = 27\frac{7}{24}$ ngày.
\end{proof}

\begin{baitoan}[\cite{SBT_Toan_6_Canh_Dieu_tap_2}, 38., p. 40]
	Tính tích \& viết kết quả ở dạng phân số tối giản: (a) $\frac{-4}{7}\cdot\frac{7}{-16}$. (b) $\frac{5}{-11}\cdot22$. (c) $\frac{-5}{16}\cdot(-32)$. (d) $35\cdot\frac{-4}{21}$. (e) $\frac{25}{10}\cdot1\frac{1}{3}$. (f) $\frac{-37}{401}\cdot(-1)$. (g) $\left(1 - \frac{1}{5}\right)\left(\frac{-3}{10} + \frac{1}{5}\right)$. (h) $\frac{-3}{5}\cdot\frac{-3}{5}\cdot\frac{1}{3}$.
\end{baitoan}

\begin{baitoan}[\cite{SBT_Toan_6_Canh_Dieu_tap_2}, 39., p. 40]
	Tính: (a) $\frac{5}{12} + \frac{21}{8}\cdot\frac{1}{14}$. (b) $\frac{8}{15}\cdot\frac{3}{64} - \frac{13}{25}$. (c) $\left(\frac{19}{21} - \frac{2}{3}\right)\cdot\frac{28}{10}$. (d) $\left(1 - \frac{5}{17}\right)\left(\frac{3}{8} - \frac{5^2}{24}\right)$.
\end{baitoan}

\begin{baitoan}[\cite{SBT_Toan_6_Canh_Dieu_tap_2}, 40., p. 40]
	Tính hợp lý: (a) $\frac{11}{4}\cdot\frac{-5}{9}\cdot\frac{8}{33}$. (b) $\frac{-5}{6}\cdot\frac{4}{19} + \frac{-7}{12}\cdot\frac{4}{19} - \frac{40}{57}$. (c) $\left(\frac{23}{41} - \frac{15}{82}\right)\cdot\frac{41}{15}$. (d) $9\cdot\left(\frac{151515}{171717} - \frac{131313}{181818}\right)$. (e) $\frac{-13}{8}\cdot\left(\frac{8}{13} + \frac{32}{28}\right) - \frac{15}{7}$. (f) $\frac{2^2}{1\cdot3}\cdot\frac{3^2}{2\cdot4}\cdot\frac{4^2}{3\cdot5}\cdot\frac{5^2}{4\cdot6}\cdot\frac{6^2}{5\cdot7}$.
\end{baitoan}

\begin{baitoan}[\cite{SBT_Toan_6_Canh_Dieu_tap_2}, 41., p. 40]
	Tìm số nguyên thích hợp cho ô vuông: (a) $\frac{7}{25}\cdot\frac{\square}{28} = \frac{-3}{20}$. (b) $\frac{46}{15}\cdot\frac{-3}{\square} = \frac{23}{5}$. (c) $\frac{\square}{-18}\cdot\frac{5}{2} = \frac{-5}{12}$.
\end{baitoan}

\begin{baitoan}[\cite{SBT_Toan_6_Canh_Dieu_tap_2}, 42., p. 41]
	1 chiếc máy tự động kiểm tra linh kiện điện tử cứ $\frac{16}{25}$ giây thì kiểm tra được 1 linh kiện. Trong $1$ giờ máy tự động kiểm tra được bao nhiêu linh kiện điện tử?
\end{baitoan}

\begin{baitoan}[\cite{SBT_Toan_6_Canh_Dieu_tap_2}, 43., p. 41]
	Tính: (a) $\frac{2}{7}\cdot\frac{14}{5}\cdot\frac{-1}{3}$. (b) $\frac{-15}{16}\cdot\frac{8}{-25}$. (c) $\frac{-5}{13}\cdot26$. (d) $\left(\frac{3}{8}\right)^2$. (e) $\left(2 - \frac{1}{2}\right)\cdot\left(\frac{-3}{4} - \frac{1}{2}\right)$. (f) $\frac{7}{11}\cdot\frac{-1}{7}\cdot\frac{11}{9}\cdot0$. (g) $18\cdot\frac{3}{10}\cdot(-5)$. (h) $\frac{15}{-49}\cdot\frac{-84}{35}$.
\end{baitoan}

\begin{baitoan}[\cite{SBT_Toan_6_Canh_Dieu_tap_2}, 44., p. 41]
	(a) Tìm số nguyên âm lớn nhất để khi nhân nó với 1 trong các phân số tối giản sau đều được tích là những số nguyên: $\frac{5}{6},\frac{-7}{15},\frac{11}{21}$. (b) Tìm số tự nhiên $a$ nhỏ nhất sao cho khi lấy $a$ chia cho $\frac{8}{9}$ hoặc $\frac{17}{12}$, ta đều được kết quả là số tự nhiên.
\end{baitoan}

\begin{baitoan}[\cite{SBT_Toan_6_Canh_Dieu_tap_2}, 45., p. 41]
	So sánh: $A = \frac{3^2}{2\cdot5} + \frac{3^2}{5\cdot8} + \frac{3^2}{8\cdot11}$ \& $B = \frac{4}{5\cdot7} + \frac{4}{7\cdot9} + \cdots + \frac{4}{59\cdot61}$.
\end{baitoan}

\begin{baitoan}[\cite{SBT_Toan_6_Canh_Dieu_tap_2}, 46., p. 41]
	Tính: (a) $1\frac{1}{2}\cdot1\frac{1}{3}\cdot1\frac{1}{4}\cdot1\frac{1}{5}\cdot1\frac{1}{6}\cdot1\frac{1}{7}$. (b) $\left(1 - \frac{1}{2}\right)\left(1 - \frac{1}{3}\right)\left(1 - \frac{1}{4}\right)\cdots\left(1 - \frac{1}{50}\right)$.
\end{baitoan}

\begin{baitoan}[\cite{SBT_Toan_6_Canh_Dieu_tap_2}, 47., p. 42]
	(a) Cho 1 hình thoi có diện tích bằng $\rm\frac{81}{10}m^2$ \& độ dài 1 đường chéo bằng $\frac{18}{5}$\emph{m}. Tính độ dài đường chéo còn lại của hình thoi đó. (b) Cho 1 hình thang có diện tích bằng $\rm\frac{319}{120}m^2$, \& độ dài 2 đáy bằng $\frac{4}{5}$\emph{m}, $\frac{35}{36}$\emph{m}. Tính chiều cao của hình thang đó.
\end{baitoan}

\begin{baitoan}[\cite{SBT_Toan_6_Canh_Dieu_tap_2}, 48., p. 42]
	1 cano xuôi dòng trên khúc sông AB hết $6$ giờ \& ngược dòng trên khúc sông BA hết $8$ giờ. Tính chiều dài khúc sông đó, biết vận tốc dòng nước là $50$\emph{m\texttt{/}min}.
\end{baitoan}

\begin{baitoan}[\cite{SBT_Toan_6_Canh_Dieu_tap_2}, 49., p. 42]
	Tìm $x$: (a) $\frac{6}{7}\cdot x = \frac{18}{23}$. (b) $\frac{15}{119}\cdot x = 1$. (c) $x:\frac{5}{6} = \frac{4}{7}$. (d) $x - \frac{3}{7}:\frac{9}{14} = \frac{-7}{3}$. (e) $\frac{9}{13}\cdot x = \frac{11}{8} - \frac{125}{1000}$. (f) $\left(x - \frac{1}{2}\right):\frac{3}{11} = \frac{11}{4}$.
\end{baitoan}

\begin{baitoan}[\cite{SBT_Toan_6_Canh_Dieu_tap_2}, 50., p. 42]
	Tính: (a) $\dfrac{\frac{3}{5} + \frac{3}{27} - \frac{3}{9} - \frac{3}{11}}{\frac{4}{5} + \frac{4}{27} - \frac{4}{9} - \frac{4}{11}}$. (b) $\dfrac{5 - \frac{5}{3} - \frac{5}{27}}{8 - \frac{8}{3} - \frac{8}{27}}:\dfrac{15 + \frac{15}{121} - \frac{15}{11}}{16 + \frac{16}{121} - \frac{16}{11}}$. (c) $\frac{1}{2}:\frac{-3}{2}:\frac{4}{3}:\frac{-5}{4}:\frac{6}{5}:\frac{-7}{6}:\cdots:\frac{-101}{100}$.
\end{baitoan}

\begin{baitoan}[\cite{SBT_Toan_6_Canh_Dieu_tap_2}, 51., p. 42]
	Ngọc \& Hà có tổng số tiền là $76000$ đồng. Biết $\frac{3}{5}$ số tiền của Ngọc bằng $\frac{2}{3}$ số tiền của Hà. Mỗi bạn có bao nhiêu tiền?
\end{baitoan}

\begin{baitoan}[\cite{SBT_Toan_6_Canh_Dieu_tap_2}, 52., p. 42]
	Bây giờ là $12$ giờ. Sau ít nhất bao nhiêu phút nữa thì kim giờ \& kim phút vuông góc với nhau?
\end{baitoan}

%------------------------------------------------------------------------------%

\section{Số Thập Phân}

\begin{baitoan}[\cite{SGK_Toan_6_Canh_Dieu_tap_2}, 1, p. 44]
	Viết các phân số $\frac{-19}{10},\frac{-335}{100},\frac{-125}{1000},\frac{-279}{1000000}$ dưới dạng số thập phân \& đọc các số thập phân đó theo mẫu.
\end{baitoan}

\begin{proof}[Giải]
	$\frac{-19}{10} = -1.9$ được đọc là: âm một chấm chín.
\end{proof}

\begin{dinhnghia}
	\emph{Phân số thập phân} là phân số mà mẫu là lũy thừa của $10$ \& tử là số nguyên, i.e., $\frac{a}{10^b}$, $\forall a\in\mathbb{Z}$, $\forall b\in\mathbb{N}^\star$. Phân số thập phân có thể viết được dưới dạng \emph{số thập phân}. Số thập phân gồm 2 phần: \emph{phần số nguyên} được viết bên trái dấu phẩy; \emph{phần thập phân} được viết bên phải dấu phẩy.
\end{dinhnghia}

\begin{baitoan}[\cite{SGK_Toan_6_Canh_Dieu_tap_2}, Ví dụ 1, 1, pp. 44--45]
	Viết các phân số \& hỗn số sau dưới dạng số thập phân: $\frac{-19}{100},\frac{-8}{125},\frac{1}{-2},5\frac{1}{25}$, $\frac{-9}{1000},-\frac{5}{8},3\frac{2}{25}$.
\end{baitoan}

\begin{proof}[Giải]
	$\frac{-19}{100} = -0.19$, $\frac{-8}{125} = \frac{-8\cdot8}{125\cdot8} - \frac{-64}{1000} = -0.064$, $\frac{1}{-2} = \frac{1\cdot(-5)}{-2\cdot(-5)} = \frac{-5}{10} = -0.5$, $5\frac{1}{25} = 5\frac{4}{100} = 5.04$.
\end{proof}

\begin{baitoan}[\cite{SGK_Toan_6_Canh_Dieu_tap_2}, Ví dụ 2, 2, p. 45]
	(a) Chai nước khoáng của An có dung tích ghi trên tem nhãn là $750$\emph{ml}. Dung tích của chai nước đó là bao nhiêu lít? Viết kết quả đó dưới dạng số thập phân \& phân số tối giản. (b) Viết các số thập phân sau dưới dạng phân số tối giản: $12.5,-0.008,-3.45$. (c) Viết các số thập phân sau dưới dạng phân số tối giản: $-0.125,-0.012,-4.005$.
\end{baitoan}

\begin{proof}[Giải]
	(a) $750$ml $= 0.75$l, $0.75 = \frac{75}{100} = \frac{3}{4}$. Dung tích của chai nước là $0.75$l hay $\frac{3}{4}$l. (b) $12.5 = \frac{125}{10} = \frac{25}{2}$, $-0.008 = -\frac{8}{1000} = -\frac{1}{125}$, $-3.45 = -\frac{345}{100} = -\frac{69}{20}$.
\end{proof}

\subsection{So sánh các số thập phân}
Cũng như số nguyên, trong 2 số thập phân khác nhau luôn có 1 số nhỏ hơn số kia. Nếu số thập phân $a$ nhỏ hơn số thập phân $b$ thì ta viết $a < b$ hay $b > a$. Số thập phân lớn hơn $0$ gọi là \textit{số thập phân dương}. Số thập phân nhỏ hơn $0$ gọi là \textit{số thập phân âm}. \textit{Tính chất bắc cầu}: Nếu $a < b$ \& $b < c$ thì $a < c$.

\textit{So sánh 2 số thập phân khác dấu.} Cũng tương tự như trong tập hợp số nguyên $\mathbb{Z}$: Số thập phân âm luôn nhỏ hơn số thập phân dương.

\textit{So sánh 2 số thập phân dương.} Để so sánh 2 số thập phân dương: \textit{Bước 1.} So sánh phần số nguyên của 2 số thập phân dương đó. Số thập phân nào có phần số nguyên lớn hơn thì lớn hơn. \textit{Bước 2.} Nếu 2 số thập phân dương đó có phần số nguyên bằng nhau thì ta tiếp tục so sánh từng cặp chữ số ở cùng 1 hàng (sau dấu ``$.$'') kể từ trái sang phải cho đến khi xuất hiện cặp chữ số đầu tiên khác nhau. Ở cặp chữ số khác nhau đó, chữ số nào lớn hơn thì số thập phân chứa chữ số đó lớn hơn.

\begin{baitoan}[\cite{SGK_Toan_6_Canh_Dieu_tap_2}, 2, Ví dụ 3, p. 46]
	So sánh: (a) $508.99$ \& $509.01$. (b) $315.267$ \& $315.29$. (c) $399.99$ \& $400.01$. (d) $895.169$ \& $895.166$.
\end{baitoan}

\begin{baitoan}[\cite{SGK_Toan_6_Canh_Dieu_tap_2}, Ví dụ 4, p. 46]
	Trong 1 cuộc thi chạy $100$\emph{m} dành cho học sinh, ban tổ chức quy định xếp hạng cho người chạy $100$\emph{m} trong thời gian $t$ \emph{s} như sau: $t\le15$: hạng A, $15 < t\le17$: hạng B, $t > 17$: hạng C. 4 bạn có kết quả chạy $100$\emph{m} như sau: Huỳnh $15.5$\emph{s}, Mạnh $16.7$\emph{s}, Phương $14.8$\emph{s}, Quang $17.1$\emph{s}. Xếp hạng kết quả 4 bạn đó.
\end{baitoan}

\textit{So sánh 2 số thập phân âm.} Cách so sánh 2 số thập phân âm được thực hiện như cách so sánh 2 số nguyên âm.

\begin{baitoan}[\cite{SGK_Toan_6_Canh_Dieu_tap_2}, Ví dụ 4, p. 47]
	So sánh: (a) $-12.19$ \& $-14.11$. (b) $-11.01$ \& $-10.99$.
\end{baitoan}

\begin{proof}[Giải]
	(a) Vì $12.19 < 14.11$ (vì $12 < 14$) nên $-12.19 > -14.11$. (b) Vì $11.01 > 10.99$ (vì $11 > 10$) nên $-11.01 < -10.99$.
\end{proof}

\begin{baitoan}[\cite{SGK_Toan_6_Canh_Dieu_tap_2}, 3, p. 47]
	Viết các số sau theo thứ tự giảm dần: $-120.341$, $36.095$, $36.1$, $-120.34$.
\end{baitoan}

\begin{baitoan}[\cite{SGK_Toan_6_Canh_Dieu_tap_2}, 1., p. 47]
	Viết các phân số \& hỗn số sau dưới dạng số thập phân: $\frac{-7}{20},\frac{-12}{25},\frac{-16}{500},5\frac{4}{25}$.
\end{baitoan}

\begin{baitoan}[\cite{SGK_Toan_6_Canh_Dieu_tap_2}, 2., p. 47]
	Viết các số thập phân sau dưới dạng phân số tối giản: $-0.225$, $-0.033$.
\end{baitoan}

\begin{baitoan}[\cite{SGK_Toan_6_Canh_Dieu_tap_2}, 3., p. 47]
	VIết các số sau theo thứ tự tăng dần: (a) $7.012$, $7.102$, $7.01$. (b) $73.059$, $-49.037$, $-49.3047$.
\end{baitoan}

\begin{baitoan}[\cite{SGK_Toan_6_Canh_Dieu_tap_2}, 4., p. 47]
	Viết các số sau theo thứ tự giảm dần: (a) $9.099$, $9.009$, $9.090$, $9.990$. (b) $-6.27$, $-6.207$, $-6.027$, $-6.277$.
\end{baitoan}

\begin{baitoan}[\cite{SGK_Toan_6_Canh_Dieu_tap_2}, 5., p. 47]
	Trong 1 cuộc thi chạy $200$\emph{m}, có 3 vận động viên đạt thành tích cao nhất là: Mai Anh: $31.42$\emph{s}, Ngọc Mai: $31.48$\emph{s}, Phương Hà: $31.09$\emph{s}. Vận động viên nào đã về nhất? Về nhì? Về ba?
\end{baitoan}

%------------------------------------------------------------------------------%

\section{Phép $\pm$ Số Thập Phân}

\subsection{Số đối của số thập phân}
Giống như số nguyên, mỗi số thập phân đều có số đối, sao cho tổng của 2 số đó bằng $0$.

\begin{dinhnghia}[Số đối của số thập phân]
	\emph{Số đối} của số thập phân $a$ ký hiệu là $-a$. Ta có: $a + (-a) = a - a = 0$, với mọi số thập phân $a$.
\end{dinhnghia}
Số đối của số đối của 1 số thập phân là chính nó: $-(-a) = a$, với mọi số thập phân $a$.

\begin{baitoan}[\cite{SGK_Toan_6_Canh_Dieu_tap_2}, Ví dụ 1, 1, p. 48]
	Tìm số đối của mỗi số thập phân sau: $3.15$, $-2.97$, $12.49$, $-10.25$.
\end{baitoan}

\begin{proof}[Giải]
	Số đối của mỗi số thập phân $3.15$, $-2.97$, $12.49$, $-10.25$ lần lượt là $-3.15$, $2.97$, $-12.49$, $10.25$.
\end{proof}

\subsection{Phép $\pm$ số thập phân}

\begin{baitoan}[\cite{SGK_Toan_6_Canh_Dieu_tap_2}, 1, p. 48]
	Đặt tính rồi tính: (a) $32.475 + 9.681$. (b) $309.48 - 125.23$.
\end{baitoan}
Để cộng, trừ 2 số thập phân dương: \textit{Bước 1.} Viết số này ở dưới số kia sao cho các chữ số ở cùng hàng đặt thẳng cột với nhau, dấu ``.'' đặt thẳng cột với nhau. \textit{Bước 2.} Thực hiện phép cộng, trừ như phép cộng, trừ các số tự nhiên. \textit{Bước 3.} Viết dấu ``.'' ở kết quả thẳng cột với các dấu ``.'' đã viết ở trên.

Quy tắc cộng 2 số thập phân (cùng\texttt{/}trái dấu) được thực hiện giống quy tắc cộng 2 số nguyên.

\begin{baitoan}[\cite{SGK_Toan_6_Canh_Dieu_tap_2}, Ví dụ 2, 2, p. 49]
	Tính tổng: (a) $(-12.4) + (-9.6)$. (b) $21.36 + (-11.16)$. (c) $(-16.5) + 1.5$.
\end{baitoan}
Giống như phép cộng số nguyên, phép cộng số thập phân cũng có các tính chất:
\begin{itemize}
	\item \textit{Giao hoán}: $a + b = b + a$, với mọi số thập phân $a,b$.
	\item \textit{Kết hợp}: $(a + b) + c = a + (b + c) = a + b + c$, với mọi số thập phân $a,b,c$.
	\item \textit{Cộng với số $0$}: $a + 0 = 0 + a = a$, với mọi số thập phân $a$.
	\item \textit{Cộng với số đối}: $a + (-a) = -a + a = a - a = 0$, với mọi số thập phân $a$.
\end{itemize}

\begin{baitoan}[\cite{SGK_Toan_6_Canh_Dieu_tap_2}, Ví dụ 3, 3, p. 49]
	Tính hợp lý: (a) $98.246 + (-76.41) + 1.754$. (b) $89.45 + (-3.28) + 0.55 + (-6.72)$.
\end{baitoan}

\begin{proof}[Giải]
	(a) $98.246 + (-76.41) + 1.754 = -76.41 + (98.246 + 1.754) = -76.41 + 100 = 100 - 76.41 = 23.59$. (b) $89.45 + (-3.28) + 0.55 + (-6.72) = (89.45 + 0.55) - (3.28 + 6.72) = 90 - 10 = 80$.
\end{proof}
Cũng như phép trừ số nguyên, để trừ 2 số thập phân ta cộng số bị trừ với số đối của số trừ.

\begin{baitoan}[\cite{SGK_Toan_6_Canh_Dieu_tap_2}, Ví dụ 4, 4, p. 49]
	Tính hiệu: (a) $6.25 - 11.12$. (b) $(-10.43) - (-14.18)$. (c) $(-14.25) - (-9.2)$.
\end{baitoan}

\begin{proof}[Giải]
	(a) $6.25 - 11.12 = 6.25 + (-11.12) = -(11.12 - 6.25) = -4.87$. (b) $(-10.43) - (-14.18) = (-10.43) + 14.18 = 14.18 - 10.43 = 3.75$. (c) $(-14.25) - (-9.2) = -14.25 + 9.2 = -(14.25 - 9.2) = -5.05$.
\end{proof}

\subsection{Quy tắc dấu ngoặc}
Quy tắc dấu ngoặc đối với số thập phân giống như quy tắc dấu ngoặc đối với số nguyên: $a + (b + c) = a + b + c$, $a + (b - c) = a + b - c$, $a - (b + c) = a - b - c$, $a - b - c$, với mọi số thập phân $a,b,c$.

\begin{baitoan}[\cite{SGK_Toan_6_Canh_Dieu_tap_2}, Ví dụ 5, 5, p. 50]
	Tính hợp lý: (a) $509.315 + (99.5 - 9.315)$. (b) $(-302.39) - (97.61 - 99.4)$. (c) $19.32 + 10.68 - 8.63 - 11.37$.
\end{baitoan}

\begin{proof}[Giải]
	(a) $509.315 + (99.5 - 9.315) = 509.315 + 99.5 - 9.315 = 99.5 + 509.315 - 9.315 = 99.5 + (509.315 - 9.315) = 99.5 + 500 = 599.5$. (b) $(-302.39) - (97.61 - 99.4) = (-302.39) - 97.61 + 99.4 = -(302.39 + 97.61) + 99.4 = -400 + 99.4 = -300.6$. (c) $19.32 + 10.68 - 8.63 - 11.37 = (19.32 + 10.68) - (8.63 + 11.37) = 30 - 20 = 10$.
\end{proof}

\begin{baitoan}[\cite{SGK_Toan_6_Canh_Dieu_tap_2}, 1., p. 51]
	Tính: (a) $324.82 + 312.25$. (b) $(-12.07) + (-5.79)$. (c) $(-41.29) - 15.34$. (d) $(-22.65) - (-1.12)$.
\end{baitoan}

\begin{baitoan}[\cite{SGK_Toan_6_Canh_Dieu_tap_2}, 2., p. 51]
	Tính hợp lý: (a) $29.42 + 20.58 - 34.23 + (-25.77)$. (b) $(-212.49) - (87.51 - 99.9)$.
\end{baitoan}

\begin{baitoan}[\cite{SGK_Toan_6_Canh_Dieu_tap_2}, 3., p. 51]
	Nam cao $1.57$\emph{m}, Linh cao $1.53$\emph{m}, Loan cao $1.49$\emph{m}. (a) Trong 3 bạn đó, bạn nào cao nhất? Bạn nào thấp nhất? (b) Chiều cao của bạn cao nhất hơn bạn thấp nhất là bao nhiêu \emph{m}?
\end{baitoan}

\begin{baitoan}[\cite{SGK_Toan_6_Canh_Dieu_tap_2}, 4., p. 51]
	Bác Đồng cưa 3 thanh gỗ: thanh thứu nhất dài $1.85$\emph{m}, thanh thứ 2 dài hơn thành thứ nhất $10$\emph{cm}. Độ dài thanh gỗ thứ 3 ngắn hơn tổng độ dài 2 thanh gỗ đầu tiên là $1.35$\emph{m}. Thanh gỗ thứ 3 mà bác Đồng đã cưa dài bao nhiêu \emph{m}?
\end{baitoan}

\begin{baitoan}[\cite{SGK_Toan_6_Canh_Dieu_tap_2}, 5., p. 51]
	Tính chu vi của: (a) 1 tam giác có độ dài 3 cạnh là $2.4$\emph{cm}, $3.75$\emph{cm}, $3.6$\emph{cm}. (b) 1 hình thang cân có độ dài 2 đáy là $2.5$\emph{cm}, $4.15$\emph{cm}, \& độ dài cạnh bên là $3.16$\emph{cm}.
\end{baitoan}

\begin{baitoan}[\cite{SGK_Toan_6_Canh_Dieu_tap_2}, 6., p. 51]
	Dùng máy tính cầm tay để tính: (a) $16.293 + (-5.973)$. (b) $-35.78 -(-18.423)$.
\end{baitoan}

%------------------------------------------------------------------------------%

\section{Phép Nhân, Phép Chia Số Thập Phân}

\begin{baitoan}[\cite{SGK_Toan_6_Canh_Dieu_tap_2}, p. 52]
	Inch là tên của 1 đơn vị đo độ dài: $1$\emph{in} $= 2.54$\emph{cm}. 1 chiếc ti vi màn hình phẳng có độ dài đường chéo là $52$\emph{in}. Độ dài đường chéo của màn hình ti vi là bao nhiêu \emph{m}?
\end{baitoan}

\begin{proof}[Giải]
	Độ dài đường chéo của màn hình ti vi là $52\cdot2.54 = 132.08$cm $= 1.3208$m.
\end{proof}

\subsection{Phép nhân số thập phân}

\begin{baitoan}[\cite{SGK_Toan_6_Canh_Dieu_tap_2}, 1, p. 52]
	Đặt tính để tính tích $5.285\cdot7.21$.
\end{baitoan}

\begin{proof}[Giải]
	$5.285\cdot7.21 = 38.10485$.
\end{proof}
Để nhân 2 số thập phân dương: \textit{Bước 1.} Viết thừa số này ở dưới thừa số kia như đối với phép nhân các số tự nhiên. \textit{Bước 2.} Thực hiện phép nhân như nhân các số tự nhiên. \textit{Bước 3.} Đếm xem trong phần thập phân của cả 2 thừa số có bao nhiêu chữ số rồi dùng dấu ``.'' tách ở tích ra bấy nhiêu chữ số kể từ phải sang trái, ta nhận được tích cần tìm.

Quy tắc nhân 2 số thập phân cùng\texttt{/}khác dấu được thực hiện giống như quy tắc nhân 2 số nguyên:
\begin{align*}
	ab = \operatorname{sign}(a)|a|\operatorname{sign}(b)|b| = \operatorname{sign}(a)\operatorname{sign}(b)|a||b| = \operatorname{sign}(ab)|ab|,\mbox{ với mọi số thập phân } a,b.	
\end{align*}
Quy tắc nhân nhiêu số thập phân cùng\texttt{/}khác dấu:
\begin{align*}
	\prod_{i=1}^n a_i = \prod_{i=1}^n \operatorname{sign}(a_i)|a_i| = \prod_{i=1}^n \operatorname{sign}(a_i)\prod_{i=1}^n |a_i| = \operatorname{sign}\left(\prod_{i=1}^n a_i\right)\left|\prod_{i=1}^n a_i\right|,\mbox{ với mọi số thập phân } a_i,\ \forall i = 1,2,\ldots,n,
\end{align*}
hoặc có thể viết tường minh hơn như sau:
\begin{align*}
	a_1a_2\cdots a_n &= \operatorname{sign}(a_1)|a_1|\operatorname{sign}(a_2)|a_2|\cdots\operatorname{sign}(a_n)|a_n| =  \operatorname{sign}(a_1)\operatorname{sign}(a_2)\cdots\operatorname{sign}(a_n)|a_1||a_2|\cdots|a_n|\\
	&= \operatorname{sign}(a_1a_2\cdots a_n)|a_1a_2\cdots a_n|,\mbox{ với mọi số thập phân } a_i,\ \forall i = 1,2,\ldots,n.
\end{align*}

\begin{baitoan}[\cite{SGK_Toan_6_Canh_Dieu_tap_2}, Ví dụ 1, 1, p. 53]
	Tính tích: (a) $(-9.207)\cdot(-3.8)$. (b) $(-9.27)\cdot4.8$. (c) $8.15\cdot(-4.26)$. (d) $19.427\cdot1.8$.
\end{baitoan}

\begin{proof}[Giải]
	(a) $(-9.207)\cdot(-3.8) = 9.207\cdot3.8 = 34.9866$. (b) $(-9.27)\cdot4.8 = -(9.27\cdot4.8) = -44.496$. (c) $8.15\cdot(-4.26) = -(8.15\cdot4.26) = -34.719$. (d) $19.427\cdot1.8 = 34.9686$.
\end{proof}

\subsubsection{Tính chất của phép nhân số thập phân}
Giống như phép nhân số nguyên, phép nhân số thập phân cũng có các tính chất:
\begin{itemize}
	\item \textit{Giao hoán}: $ab = ba$, với mọi số thập phân $a,b$.
	\item \textit{Kết hợp}: $(ab)c = a(bc) = abc$, với mọi số thập phân $a,b,c$.
	\item \textit{Nhân với số $1$}: $1a = a\cdot1 = a$, với mọi số thập phân $a$.
	\item \textit{Phân phối của phép nhân đối với phép cộng \& phép trừ}: $a(b + c) = ab + ac$, $a(b - c) = ab - ac$ (có thể viết gộp lại thành $a(b\pm c) = ab\pm ac$) với mọi số thập phân $a,b,c$.
\end{itemize}

\begin{baitoan}[\cite{SGK_Toan_6_Canh_Dieu_tap_2}, Ví dụ 2, 2, p. 53]
	Tính hợp lý: (a) $0.75\cdot8$. (b) $7.63\cdot21.15 + 7.63\cdot(-121.15)$. (c) $0.25\cdot12$. (d) $0.125\cdot14\cdot36$.
\end{baitoan}

\begin{proof}[1st giải]
	(a) $0.75\cdot8 = 3\cdot0.25\cdot4\cdot2 = (0.25\cdot4)\cdot(3\cdot2) = 1\cdot6 = 6$. (b) $7.63\cdot21.15 + 7.63\cdot(-121.15) = 7.63\cdot[21.15 + (-121.15)] = 7.63\cdot[-(121.15 - 21.15)] = 7.63\cdot(-100) = -(7.63\cdot100) = -763$. (c) $0.25\cdot12 = 0.25\cdot4\cdot3 = 1\cdot3 = 3$. (d) $0.125\cdot14\cdot36 = 0.125\cdot2\cdot7\cdot4\cdot9 = (0.25\cdot4)\cdot(7\cdot9) = 1\cdot63 = 63$.
\end{proof}

\begin{proof}[2nd giải]
	(a) $0.75\cdot8 = \frac{75}{100}\cdot8 = \frac{3}{4}\cdot8 = 3\cdot2 = 6$. (b) $7.63\cdot21.15 + 7.63\cdot(-121.15) = -7.63\cdot(121.15 - 21.15) = -7.63\cdot100 = -763$. (c) $0.25\cdot12 = \frac{25}{100}\cdot12 = \frac{1}{4}\cdot12 = \frac{12}{4} = 3$. (d) $0.125\cdot14\cdot36 = \frac{125}{1000}\cdot14\cdot36 = \frac{1}{8}\cdot2\cdot7\cdot4\cdot9 = \left(\frac{1}{4\cdot2}\cdot2\cdot4\right)\cdot7\cdot9 = 1\cdot63 = 63$.
\end{proof}

\begin{luuy}
	Muốn tính hợp lý, ta nhóm các số hạng sao cho tích của chúng là lũy thừa của $10$, i.e., $10^n$, với $n\in\mathbb{N}^\star$. Cần nhớ:
	\begin{align*}
		10 &= 2\cdot5,\\
		100 &= 10^2 = 10\cdot10 = 2\cdot50 = 4\cdot25 = 5\cdot20,\\
		1000 &= 10^3 = 2\cdot500 = 4\cdot250 = 5\cdot200 = 8\cdot125 = 10\cdot100 = 20\cdot50 = 25\cdot40.
	\end{align*}
\end{luuy}

\subsection{Phép chia số thập phân}

\begin{baitoan}[\cite{SGK_Toan_6_Canh_Dieu_tap_2}, 4--5, pp. 53--54]
	Đặt tính để tính thương: (a) $247.68:144$. (b) $311.01:0.3$.
\end{baitoan}

\begin{proof}[Giải]
	(a) $247.68:144 = 1.72$. (b) $311.01:0.3 = 1036.7$.
\end{proof}
Để chia 2 số thập phân dương, ta làm như sau: \textit{Bước 1.} Số chia có bao nhiêu chữ số sau dấu ``.'' thì ta chuyển dấu ``.'' ở số bị chia sang bên phải bấy nhiêu chữ số (nếu số bị chia không đủ vị trí để chuyển dấu ``.'' thì ta điền thêm những chữ số 0 vào bên phải của số đó). \textit{Bước 2.} Bỏ đi dấu ``.'' ở số chia, ta nhận được số nguyên dương. \textit{Bước 3.} Đem số nhận được ở \textit{Bước 1} chia cho số nguyên dương nhận được ở \textit{Bước 2}, ta có thương cần tìm.

\begin{baitoan}[\cite{SGK_Toan_6_Canh_Dieu_tap_2}, Ví dụ 3, p. 55]
	Tính thương: (a) $8.446:4.12$. (b) $5.4:0.027$.
\end{baitoan}

\begin{proof}[Giải]
	(a) $8.446:4.12 = 2.05$. (b) $5.4:0.027 = 200$.
\end{proof}
Quy tắc chia 2 số thập phân cùng\texttt{/}khác dấu được thực hiện giống như quy tắc chia 2 số nguyên.

\begin{baitoan}[\cite{SGK_Toan_6_Canh_Dieu_tap_2}, Ví dụ 4, 3, p. 55]
	Tính thương: (a) $(-8.446):(-4.12)$. (b) $(-5.4):0.027$. (c) $(-17.01):(-12.15)$. (d) $(-15.175):12.14$.
\end{baitoan}

\begin{proof}[Giải]
	$(-8.446):(-4.12) = 8.446:4.12 = 2.05$. (b) $(-5.4):0.027 = -(5.4:0.027) = -200$. (c) $(-17.01):(-12.15) = 17.01:12.15 = 1.4$. (d) $(-15.175):12.14 = -(15.175:12.14) = -1.25$.
\end{proof}

\begin{baitoan}[\cite{SGK_Toan_6_Canh_Dieu_tap_2}, 1., p. 55]
	Tính: (a) $200\cdot0.8$. (b) $(-0.5)\cdot(-0.7)$. (c) $(-0.8)\cdot0.006$. (d) $(-0.4)\cdot(-0.5)\cdot(-0.2)$.
\end{baitoan}

\begin{baitoan}[\cite{SGK_Toan_6_Canh_Dieu_tap_2}, 2., p. 55]
	Cho $23\cdot456 = 10488$. Tính nhẩm: (a) $2.3\cdot456$. (b) $2.3\cdot45.6$. (c) $(-2.3)\cdot(-4.56)$. (d) $(-2.3)\cdot45600$.
\end{baitoan}

\begin{baitoan}[\cite{SGK_Toan_6_Canh_Dieu_tap_2}, 3., p. 55]
	Tính: (a) $46.827:90$. (b) $(-72.39):(-19)$. (c) $(-882):3.6$. (d) $10.88:(-0.17)$.
\end{baitoan}

\begin{baitoan}[\cite{SGK_Toan_6_Canh_Dieu_tap_2}, 4., p. 56]
	Cho $182:13 = 14$. Tính nhẩm: (a) $182:1.3$. (b) $18.2:13$.
\end{baitoan}

\begin{baitoan}[\cite{SGK_Toan_6_Canh_Dieu_tap_2}, 5., p. 56]
	1 căn phòng có dạng hình hộp chữ nhật với chiều dài $4.2$\emph{m}, chiều rộng $3.5$\emph{m}, \& chiều cao $3.2$\emph{m}. Người ta muốn sơn lại trần nhà \& 4 bức tường bên trong phòng. Biết tổng diện tích các cửa là $\rm5.4m^2$. (a) Tính diện tích cần sơn lại. (b) Giá tiền công sơn lại tường \& trần nhà đều là $12000$ \emph{đồng\texttt{/}$\rm m^2$}. Tính tổng số tiền công để sơn lại căng phòng đó.
\end{baitoan}

\begin{baitoan}[\cite{SGK_Toan_6_Canh_Dieu_tap_2}, 6., p. 56]
	1 thửa ruộng dạng hình chữ nhật có chiều dài $110$\emph{m}, chiều rộng $78$\emph{m}. Người ta cấy lúa trên thửa ruộng đó, cứ $1$\emph{ha} thu hoạch được $71.5$ tạ thóc. Cả thửa ruộng đó thu hoạch được bao nhiêu tạ thóc?
\end{baitoan}

\begin{baitoan}[\cite{SGK_Toan_6_Canh_Dieu_tap_2}, 7., p. 56]
	Bác Hà có 2 tấm kính hình chữ nhật. Chiều rộng của mỗi tấm kính bằng $\frac{1}{2}$ chiều dài của nó \& chiều dài của tấm kính nhỏ đúng bằng chiều rộng của tấm kính lớn. Bác ghép 2 tấm kính sát vào nhau \& đặt lên mặt bàn có diện tích $\rm0.9m^2$ thì vừa khít. Tính diện tích của mỗi tấm kính.
\end{baitoan}

\begin{baitoan}[\cite{SGK_Toan_6_Canh_Dieu_tap_2}, 8., p. 56]
	1 chiếc bàn ăn có mặt bàn hình tròn đường kính $150$\emph{cm}. Dùng 1 khăn vải hình tròn để phủ lên mặt bàn thì thấy khăn rủ xuống khỏi mép bàn dài $20$\emph{cm}. Tính diện tích phần khăn rủ xuống khỏi mép bàn.
\end{baitoan}

\begin{baitoan}[\cite{SGK_Toan_6_Canh_Dieu_tap_2}, 9., p. 56]
	Dùng máy tính cầm tay để tính: (a) $3.14\cdot7.652$. (b) $(-10.3125):2.5$. (c) $54.369:(-4.315)$.
\end{baitoan}

%------------------------------------------------------------------------------%

\section{Ước Lượng \& Làm Tròn Số}

\subsection{Làm tròn số nguyên}

\begin{baitoan}[\cite{SGK_Toan_6_Canh_Dieu_tap_2}, 1, p. 57]
	Làm tròn số $2643235$ đến: (a) hàng nghìn. (b) hàng triệu.
\end{baitoan}

\begin{proof}[Giải]
	(a) $2643235\approx2643000$. (b) $2643235\approx3000000$.
\end{proof}
Ký hiệu $\approx$ (approximate) đọc là ``gần bằng'' hoặc ``xấp xỉ''. Để làm tròn 1 số nguyên (có nhiều chữ số) đến 1 hàng nào đó: Nếu chữ số đứng ngay bên phải hàng làm tròn nhỏ hơn $5$ thì ta thay lần lượt các chữ số đứng bên phải hàng làm tròn bởi chữ số 0. Nếu chữ số đứng ngay bên phải hàng làm tròn $\ge5$ thì ta thay lần lượt các chữ số đứng bên phải hàng làm tròn bởi chữ số 0 rồi cộng thêm 1 vào chữ số của hàng làm tròn.

\begin{baitoan}[\cite{SGK_Toan_6_Canh_Dieu_tap_2}, Ví dụ 1, p. 58]
	(a) Làm tròn số $125356$ đến hàng nghìn. (b) Làm tròn số $-123856789$ đến hàng triệu. (c) Làm tròn số $321912$ đến hàng chục nghìn. (d) Làm tròn số $-25167914$ đến hàng chục triệu.
\end{baitoan}

\begin{proof}[Giải]
	(a) Vì chữ số hàng trăm là 3 nên $125356\approx125000$. (b) Vì chữ số hàng trăm nghìn là 8 nên $-123856789\approx-124000000$.
\end{proof}

\subsection{Làm tròn số thập phân}

\begin{baitoan}[\cite{SGK_Toan_6_Canh_Dieu_tap_2}, 2, p. 58]
	Làm tròn số $76.421$ đến: (a) Hàng phần mười (i.e., chữ số đầu tiên sau dấu ``.''). (b) Hàng chục.
\end{baitoan}

\begin{proof}[Giải]
	(a) $76.421\approx76.4$. (b) $76.421\approx80$.
\end{proof}
Để làm tròn 1 số thập phân đến 1 hàng nào đó, ta thực hiện giống như cách làm tròn 1 số nguyên đến 1 hàng nào đó, sau đó bỏ đi những chữ số 0 ở tận cùng bên phải phần thập phân.

\begin{baitoan}[\cite{SGK_Toan_6_Canh_Dieu_tap_2}, Ví dụ 2, p. 59]
	Theo \url{https://danso.org/viet-nam}, vào ngày 11.2.2020, dân số Việt Nam là $96975052$ người. (a) Làm tròn dân số Việt Nam đến hàng triệu. (b) Sử dụng số thập phân để viết dân số Việt Nam theo đơn vị tính: triệu người. Sau đó làm tròn số thập phân đó đến hàng phần trăm.
\end{baitoan}

\begin{proof}[Giải]
	(a) Có $96975052\approx97000000 = 97$ triệu. (b) Có $96975052 = (96975052:1000000)$ triệu $= 96.975052$ triệu. Vậy dân số Việt Nam là $96.975052$ triệu người $\approx96.98$ triệu người.
\end{proof}

\begin{baitoan}[\cite{SGK_Toan_6_Canh_Dieu_tap_2}, 2, p. 59]
	(a) Làm tròn số $-23.567$ đến hàng phần mười. (b) Làm tròn số $-25.1679$ đến hàng phần trăm.
\end{baitoan}

\begin{baitoan}[\cite{SGK_Toan_6_Canh_Dieu_tap_2}, 1., p. 59]
	Theo \url{https://danso.org/dan-so-the-gioi}, vào ngày 11.2.2020, dân số thế giới là $7762912358$ người. Sử dụng số thập phân để viết dân số thế giới theo đơn vị tính: tỷ người. Sau đó làm tròn số thập phân đó đến: (a) Hàng thập phân thứ nhất. (b) Hàng thập phân thứ 2.
\end{baitoan}

\begin{baitoan}[\cite{SGK_Toan_6_Canh_Dieu_tap_2}, 2., p. 60]
	1 bánh xe hình tròn có đường kính là $700$\emph{mm} chuyển động trên 1 đường thẳng từ điểm A đến điểm B sau $875$ vòng. Quãng đường AB dài khoảng bao nhiêu \emph{km} (làm tròn kết quả đến hàng phần mười)?
\end{baitoan}

\begin{baitoan}[\cite{SGK_Toan_6_Canh_Dieu_tap_2}, 3., p. 60]
	Ước lượng kết quả của các tổng sau theo mẫu: Mẫu: $119 + 52\approx120 + 50 = 170$, $185.91 + 14.11\approx185.9 + 14.1 = 200$. (a) $221 + 38$. (b) $6.19 + 3.81$. (c) $11.131 + 9.868$. (d) $31.189 + 27.811$.
\end{baitoan}

\begin{baitoan}[\cite{SGK_Toan_6_Canh_Dieu_tap_2}, 4., p. 60]
	Ước lượng kết quả của các tích sau theo mẫu: Mẫu: $81\cdot49\approx80\cdot50 = 4000$, $8.19\cdot4.95\approx8\cdot5 = 40$. (a) $21\cdot39$. (b) $101\cdot95$. (c) $19.87\cdot30.106$. (d) $(-10.11)\cdot(-8.92)$.
\end{baitoan}

%------------------------------------------------------------------------------%

\section{Tìm Giá Trị Phân Số của 1 Số Cho Trước. Tìm 1 Số Biết Giá Trị 1 Phân Số của Nó}
Số Pi được người Babylon cổ đại phát hiện gần $4000$ năm trước \& được biểu diễn bằng chữ cái Hy Lạp $\pi$ từ giữa thế kỷ XVIII. Số $\pi$ thể hiện mối liên hệ đặc biệt giữa độ dài của 1 đường tròn với độ dài đường kính của đường tròn đó. Công thức tính chu vi hình tròn: $C = \pi d = 2\pi r$ với $d,r$ lần lượt là đường kính \& bán kính của đường tròn đó (đường kính bằng 2 lần bán kính: $d = 2r$). Công thức tính diện tích hình tròn: $S = \pi r^2 = \pi\left(\frac{d}{2}\right)^2 = \frac{\pi}{4}d^2$.

\subsection{Tỷ số}

\subsubsection{Tỷ số của 2 số}

\begin{baitoan}[\cite{SGK_Toan_6_Canh_Dieu_tap_2}, 1, p. 61]
	Viết thương trong phép chia số $1000$ cho số $10$ để so sánh chúng.
\end{baitoan}

\begin{proof}[Giải]
	Thương $1000:10 = \frac{1000}{10} = 100$. Vậy số $1000$ lớn gấp $100$ lần số $10$.
\end{proof}

\begin{dinhnghia}[Tỷ số]
	\emph{Tỷ số} của $a,b\in\mathbb{R}$, $b\ne0$, là thương trong phép chia số $a$ cho số $b$, ký hiệu là $a:b$ hoặc $\frac{a}{b}$.
\end{dinhnghia}
Nếu tỷ số của $a$ \& $b$ được viết ở dạng $\frac{a}{b}$ thì ta cũng gọi $a$ là \textit{tử số} \& $b$ là \textit{mẫu số}.


%------------------------------------------------------------------------------%

\section{Tìm Tỷ Số \& Tỷ Số \% của 2 Đại Lượng}

%------------------------------------------------------------------------------%

\section{Toán về Công Việc Làm Đồng Thời}

%------------------------------------------------------------------------------%

\section{Tổng Các Phân Số Viết Theo Quy Luật}

%------------------------------------------------------------------------------%

\section{Miscellaneous}

%------------------------------------------------------------------------------%

\printbibliography[heading=bibintoc]
	
\end{document}