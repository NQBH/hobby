\documentclass{article}
\usepackage[backend=biber,natbib=true,style=authoryear]{biblatex}
\addbibresource{/home/nqbh/reference/bib.bib}
\usepackage[utf8]{vietnam}
\usepackage{tocloft}
\renewcommand{\cftsecleader}{\cftdotfill{\cftdotsep}}
\usepackage[colorlinks=true,linkcolor=blue,urlcolor=red,citecolor=magenta]{hyperref}
\usepackage{amsmath,amssymb,amsthm,mathtools,float,graphicx,algpseudocode,algorithm,tcolorbox}
\usepackage[inline]{enumitem}
\allowdisplaybreaks
\numberwithin{equation}{section}
\newtheorem{assumption}{Assumption}[section]
\newtheorem{baitoan}{Bài toán}
\newtheorem{cauhoi}{Câu hỏi}[section]
\newtheorem{conjecture}{Conjecture}[section]
\newtheorem{corollary}{Corollary}[section]
\newtheorem{definition}{Definition}[section]
\newtheorem{dinhly}{Định lý}[section]
\newtheorem{dinhnghia}{Định nghĩa}[section]
\newtheorem{example}{Example}[section]
\newtheorem{hequa}{Hệ quả}[section]
\newtheorem{lemma}{Lemma}[section]
\newtheorem{luuy}{Lưu ý}[section]
\newtheorem{nhanxet}{Nhận xét}[section]
\newtheorem{notation}{Notation}[section]
\newtheorem{principle}{Principle}[section]
\newtheorem{problem}{Problem}[section]
\newtheorem{proposition}{Proposition}[section]
\newtheorem{question}{Question}[section]
\newtheorem{remark}{Remark}[section]
\newtheorem{theorem}{Theorem}[section]
\newtheorem{vidu}{Ví dụ}[section]
\usepackage[left=0.5in,right=0.5in,top=1.5cm,bottom=1.5cm]{geometry}
\usepackage{fancyhdr}
\pagestyle{fancy}
\fancyhf{}
\lhead{\small Sect.~\thesection}
\rhead{\small\nouppercase{\leftmark}}
\renewcommand{\subsectionmark}[1]{\markboth{#1}{}}
\cfoot{\thepage}
\def\labelitemii{$\circ$}
\DeclareRobustCommand{\divby}{%
	\mathrel{\vbox{\baselineskip.65ex\lineskiplimit0pt\hbox{.}\hbox{.}\hbox{.}}}%
}

\title{Fraction -- Phân Số}
\author{Nguyễn Quản Bá Hồng\footnote{Independent Researcher, Ben Tre City, Vietnam\\e-mail: \texttt{nguyenquanbahong@gmail.com}; website: \url{https://nqbh.github.io}.}}
\date{\today}

\begin{document}
\maketitle
\begin{abstract}
	\textsc{[en]} This text is a collection of problems, from easy to advanced, about fraction. This text is also a supplementary material for my lecture note on Elementary Mathematics grade 6, which is stored \& downloadable at the following link: \href{https://github.com/NQBH/hobby/blob/master/elementary_mathematics/grade_6/NQBH_elementary_mathematics_grade_6.pdf}{GitHub\texttt{/}NQBH\texttt{/}hobby\texttt{/}elementary mathematics\texttt{/}grade 6\texttt{/}lecture}\footnote{\textsc{url}: \url{https://github.com/NQBH/hobby/blob/master/elementary_mathematics/grade_6/NQBH_elementary_mathematics_grade_6.pdf}.}. The latest version of this text has been stored \& downloadable at the following link: \href{https://github.com/NQBH/hobby/blob/master/elementary_mathematics/grade_6/fraction/NQBH_fraction.pdf}{GitHub\texttt{/}NQBH\texttt{/}hobby\texttt{/}elementary mathematics\texttt{/}grade 6\texttt{/}fraction}\footnote{\textsc{url}: \url{https://github.com/NQBH/hobby/blob/master/elementary_mathematics/grade_6/fraction/NQBH_fraction.pdf}.}.
	\vspace{2mm}
	
	\textsc{[vi]} Tài liệu này là 1 bộ sưu tập các bài tập chọn lọc từ cơ bản đến nâng cao về phân số. Tài liệu này là phần bài tập bổ sung cho tài liệu chính -- bài giảng \href{https://github.com/NQBH/hobby/blob/master/elementary_mathematics/grade_6/NQBH_elementary_mathematics_grade_6.pdf}{GitHub\texttt{/}NQBH\texttt{/}hobby\texttt{/}elementary mathematics\texttt{/}grade 6\texttt{/}lecture} của tác giả viết cho Toán Sơ Cấp lớp 6. Phiên bản mới nhất của tài liệu này được lưu trữ \& có thể tải xuống ở link sau: \href{https://github.com/NQBH/hobby/blob/master/elementary_mathematics/grade_6/fraction/NQBH_fraction.pdf}{GitHub\texttt{/}NQBH\texttt{/}hobby\texttt{/}elementary mathematics\texttt{/}grade 6\texttt{/}fraction}.
\end{abstract}
\tableofcontents
\newpage

%------------------------------------------------------------------------------%

\section{Phân Số. Tính chất Cơ Bản của Phân Số. Rút Gọn Phân Số}
``\textbf{1.} Ta gọi $\frac{a}{b}$ với $a,b\in\mathbb{Z}$, $b\ne0$ là 1 \textit{phân số}, $a$ là \textit{tử}, $b$ là \textit{mẫu} của phân số. Ta có thể viết thương của phép chia $a\in\mathbb{Z}$ cho $b\in\mathbb{Z}$, $b\ne 0$ dưới dạng $\frac{a}{b}$ \& cũng gọi $\frac{a}{b}$ là phân số. $a\in\mathbb{Z}$ có thể viết dưới dạng phân số $\frac{a}{1}$. \textbf{2.} \textit{2 phân số bằng nhau.} Cho $a,b,c,d\in\mathbb{Z}$, $b\ne0$, $d\ne 0$. Nếu $ad = bc$ thì $\frac{a}{b} = \frac{c}{d}$, ngược lại nếu $\frac{a}{b} = \frac{c}{d}$ thì $ad = bc$. \textbf{3.} \textit{2 tính chất cơ bản của phân số}: $\frac{a}{b} = \frac{am}{bm}$, $\forall a,b,m\in\mathbb{Z}$, $b\ne0$, $m\ne0$. $\frac{a}{b} = \frac{a:n}{b:n}$, $\forall a,b,n\in\mathbb{Z}$, $b\ne0$, $n\in\mbox{ƯC}(a,b)$. \textbf{4.} \textit{Rút gọn phân số}: Muốn rút gọn 1 phân số, ta chia cả tử \& mẫu của phân số đó cho 1 ước chung khác $\pm1$ của chúng. Phân số tối giản là phân số mà tử \& mẫu chỉ có ước chung là $\pm1$, i.e., $\frac{a}{b}$, $a,b\in\mathbb{Z}$, $b\ne0$, $\mbox{ƯCLN}(a,b) = 1$. \textbf{5.} Nếu đổi dấu cả tử \& mẫu của 1 phân số thì được 1 phân số mới bằng phân số đã cho. $\frac{a}{b} = \frac{-a}{-b}$, $\frac{-a}{b} = \frac{a}{-b}$, $\forall a,b\in\mathbb{Z}$, $b\ne0$. \textbf{6.} Nếu $\frac{a}{b}$ là phân số tối giản thì mọi phân số bằng nó đều có dạng $\frac{am}{bm}$ với $m\in\mathbb{Z}$ \& $m\ne0$.'' -- \cite[Chap. 3, \S1, p. 45]{Tuyen_Toan_6}

``Số có dạng $\frac{a}{b}$ trong đó $a,b\in\mathbb{Z}$, $b\ne0$ được gọi là \textit{phân số}. Số nguyên $n\in\mathbb{Z}$ được đồng nhất với phân số $\frac{n}{1}$. Tính chất cơ bản của phân số: $\frac{a}{b} = \frac{am}{bm} = \frac{a:n}{b:n}$ với $m\in\mathbb{Z}$, $m\ne0$, $n\in\mbox{ƯC}(a,b)$. Nếu $\mbox{ƯCLN}(|a|,|b|) = 1$ thì $\frac{a}{b}$ là phân số tối giản. Nếu $\frac{m}{n}$ là dạng tối giản của phân số $\frac{a}{b}$ thì tồn tại số nguyên $k\in\mathbb{Z}$ sao cho $a = mk$, $b = nk$.'' -- \cite[Chap. III, \S1, p. 4]{Binh_Toan_6_tap_2}

\begin{baitoan}[\cite{Tuyen_Toan_6}, Ví dụ 49, p. 45]
	Cho $A = \{-5,0,9\}$. Viết tất cả các phân số $\frac{a}{b}$ với $a,b\in A$. Có bao nhiêu phân số thỏa mãn?
\end{baitoan}

\begin{proof}[Giải]
	Số $0$ không thể lấy làm mẫu của phân số. Lấy $-5$ làm mẫu: $\frac{-5}{-5},\frac{0}{-5},\frac{9}{-5}$. Lấy $9$ làm mẫu: $\frac{-5}{9},\frac{0}{9},\frac{9}{9}$. Có $6$ phân số thỏa mãn.
\end{proof}

\begin{baitoan}[Mở rộng \cite{Tuyen_Toan_6}, Ví dụ 49, p. 45]
	Cho $A = \{a_1,a_2,\ldots,a_n\}\subset\mathbb{Z}$. Viết tất cả các phân số $\frac{a}{b}$ với $a,b\in A$. Có bao nhiêu phân số thỏa mãn?
\end{baitoan}

\begin{proof}[Giải]
	Xét 2 trường hợp: (a) Nếu $0\notin A$, i.e., $a_i\ne0$, $\forall i = 1,\ldots,n$. Tất cả các phân số $\frac{a}{b}$ với $a,b\in A$: $\frac{a_i}{a_j}$, $\forall i,j = 1,\ldots,n$, có tổng cộng $n^2$ phân số thỏa mãn. (b) Nếu $0\in A$, i.e., tồn tại chỉ số $k\in\{1,\ldots,n\}$ sao cho $a_k = 0$, ngoài ra $a_i\ne 0$, $\forall i = 1,\ldots,n$, $i\ne k$ (vì $A$ là 1 tập hợp nên không có các phần tử trùng nhau). Tất cả các phân số $\frac{a}{b}$ với $a,b\in A$: $\frac{a_i}{a_j}$, $\forall i,j = 1,\ldots,n$, $j\ne k$ có tổng cộng $n(n - 1) = n^2 - n$ phân số thỏa mãn.
\end{proof}

\begin{nhanxet}
	``Mẫu của 1 phân số phải khác $0$ nhưng tử của phân số có thể bằng $0$, khi đó giá trị của phân số đúng bằng $0$, i.e., $\frac{0}{b} = 0$, $\forall b\in\mathbb{Z}$, $b\ne 0$. Tử \& mẫu của 1 phân số có thể bằng nhau, khi đó giá trị của phân số đúng bằng $1$, i.e., $\frac{a}{a} = 1$, $\forall a\in\mathbb{Z}$, $a\ne 0$.'' -- \cite[p. 46]{Tuyen_Toan_6}
\end{nhanxet}

\begin{baitoan}[\cite{Tuyen_Toan_6}, Ví dụ 50, p. 46]
	Viết tập hợp $B$ các phân số bằng phân số $\frac{7}{-15}$ với mẫu dương có 2 chữ số.
\end{baitoan}

\begin{proof}[Giải]
	$\frac{7}{-15} = \frac{-7}{15}$. Phân số này là 1 phân số tối giản với mẫu dương. Mọi phân số bằng nó đều có dạng $\frac{-7m}{15m}$ với $m\in\mathbb{Z}$, $m\ne0$. Mẫu số của các phân số cần phải tìm là 1 số có 2 chữ số nên chọn $m\in\mathbb{Z}$ sao cho $10\le15m\le 99$, suy ra\footnote{$m\in\mathbb{Z}\land(10\le15m\le 99)\Leftrightarrow\lfloor\frac{15}{10}\rfloor = 1\le m\le\lfloor\frac{99}{15}\rfloor = 6$.} $1\le m\le6$, i.e., $m\in\{1,2,3,4,5,6\}$. Vậy $B = \left\{\frac{-7}{15},\frac{-14}{30},\frac{-21}{45},\frac{-28}{60},\frac{-35}{75},\frac{-42}{90}\right\}$.
\end{proof}

\begin{baitoan}[Mở rộng \cite{Tuyen_Toan_6}, Ví dụ 50, p. 46]
	Cho trước $a,b\in\mathbb{Z}$, $b\ne0$, \& $n\in\mathbb{N}^\star$. Viết tập hợp $B$ các phân số bằng phân số $\frac{a}{b}$ với mẫu dương có $n$ chữ số.
\end{baitoan}

\begin{baitoan}[\cite{Tuyen_Toan_6}, Ví dụ 51, p. 46]
	Tìm phân số bằng phân số $\frac{32}{60}$, biết tổng của tử \& mẫu là $115$.
\end{baitoan}	

\begin{proof}[Giải]
	Có $\frac{32}{60} = \frac{8}{15} = \frac{8m}{15m}$, $\forall m\in\mathbb{Z}$, $m\ne0$. Tổng của tử \& mẫu là $115\Rightarrow8m + 15m = 115\Rightarrow23m = 115\Rightarrow m =\frac{115}{23} = 5$. Phân số cần tìm: $\frac{8\cdot5}{15\cdot5} = \frac{40}{75}$.
\end{proof}

\begin{nhanxet}
	``Nếu không rút gọn phân số $\frac{32}{60}$ thành phân số tối giản $\frac{8}{15}$ mà khẳng định các phân số bằng phân số $\frac{32}{60}$ có dạng $\frac{32m}{60m}$ thì sẽ mắc sai lầm là bỏ sót rất nhiều phân số bằng phân số $\frac{32}{60}$ do đó không thể tìm được đáp số của bài toán trên.'' -- \cite[p. 46]{Tuyen_Toan_6}
\end{nhanxet}

\begin{baitoan}[Mở rộng \cite{Tuyen_Toan_6}, Ví dụ 51, p. 46]
	Cho trước $a,b,n\in\mathbb{Z}$, $b\ne0$. Tìm phân số bằng phân số $\frac{a}{b}$, biết tổng của tử \& mẫu là $n$.
\end{baitoan}

\begin{baitoan}[\cite{Tuyen_Toan_6}, \textbf{236.}, p. 47]
	Trong các phân số sau, những phân số nào bằng nhau? $\frac{15}{60},\frac{-7}{5},\frac{6}{15},\frac{28}{-20},\frac{3}{12}$.
\end{baitoan}

\begin{baitoan}[\cite{Tuyen_Toan_6}, \textbf{237.}, p. 47]
	Cho $A = \frac{3n - 5}{n + 4}$. Tìm $n\in\mathbb{Z}$ để $A\in\mathbb{Z}$.
\end{baitoan}

\begin{baitoan}[\cite{Tuyen_Toan_6}, \textbf{238.}, p. 47]
	Tìm $n\in\mathbb{Z}$ để cho các phân số sau đồng thời có giá trị nguyên: $\frac{-12}{n},\frac{15}{n - 2},\frac{8}{n + 1}$.
\end{baitoan}

\begin{baitoan}[\cite{Tuyen_Toan_6}, \textbf{239.}, p. 47]
	Tìm $x\in\mathbb{Z}$ biết: (a) $\frac{x - 1}{9} = \frac{8}{3}$; (b) $\frac{-x}{4} = \frac{-9}{x}$; (c) $\frac{x}{4} = \frac{18}{x + 1}$.
\end{baitoan}

\begin{baitoan}[\cite{Tuyen_Toan_6}, \textbf{240.}, p. 47]
	Tìm $x,y\in\mathbb{Z}$ thỏa $\frac{x - 4}{y - 3} = \frac{4}{3}$ \& $x - y = 5$.
\end{baitoan}

\begin{baitoan}[\cite{Tuyen_Toan_6}, \textbf{241.}, p. 47]
	Viết dạng tổng quát các phân số bằng phân số $\frac{-12}{30}$.
\end{baitoan}

\begin{baitoan}[\cite{Tuyen_Toan_6}, \textbf{242.}, p. 47]
	Rút gọn phân số: (a) $\frac{990}{2610}$; (b) $\frac{374}{506}$; (c) $\frac{3600 - 75}{8400 - 175}$; (d) $\dfrac{9^{14}\cdot25^5\cdot8^7}{18^{12}\cdot625^3\cdot24^3}$.
\end{baitoan}

\begin{baitoan}[\cite{Tuyen_Toan_6}, \textbf{243.}, p. 47]
	Cho phân số $\frac{a}{b}$. Chứng minh: Nếu $\frac{a - x}{b - y} = \frac{a}{b}$ thì $\frac{x}{y} = \frac{a}{b}$.
\end{baitoan}

\begin{baitoan}[\cite{Tuyen_Toan_6}, \textbf{244.}, p. 47]
	Cho phân số $A = \dfrac{1 + 3 + 5 + \cdots + 19}{21 + 23 + 25 + \cdots + 39}$. (a) Rút gọn $A$; (b) Xóa 1 số hạng ở tử \& xóa 1 số hạng ở mẫu để được 1 phân số mới vẫn bằng $A$.
\end{baitoan}

\begin{baitoan}[\cite{Tuyen_Toan_6}, \textbf{245.}, p. 47]
	Rút gọn phân số $A = \frac{71\cdot52 + 53}{530\cdot71 - 180}$ mà không cần thực hiện các phép tính ở tử.
\end{baitoan}

\begin{baitoan}[\cite{Tuyen_Toan_6}, \textbf{246.}, p. 47]
	2 phân số sau có bằng nhau không? $\dfrac{\overline{abab}}{\overline{cdcd}},\dfrac{\overline{ababab}}{\overline{cdcdcd}}$.
\end{baitoan}

\begin{baitoan}[\cite{Tuyen_Toan_6}, \textbf{247.}, p. 47]
	Chứng minh: (a) $\dfrac{1\cdot3\cdot5\cdots39}{21\cdot22\cdot23\cdots40} = \dfrac{1}{2^{20}}$; (b) $\dfrac{1\cdot3\cdot5\cdots(2n - 1)}{(n + 1)(n + 2)(n + 3)\cdots2n} = \dfrac{1}{2^n}$ với $n\in\mathbb{N}^\star$.
\end{baitoan}

\begin{baitoan}[\cite{Tuyen_Toan_6}, \textbf{248.}, p. 47]
	Tìm phân số $\frac{a}{b}$ bằng phân số $\frac{60}{108}$ biết: (a) $\mbox{\rm ƯCLN}(a,b) = 15$; (b) $\operatorname{BCNN}(a,b) = 180$.
\end{baitoan}

\begin{baitoan}[\cite{Tuyen_Toan_6}, \textbf{249.}, p. 48]
	Tìm phân số bằng phân số $\frac{200}{520}$ sao cho: (a) Tổng của tử \& mẫu là $306$; (b) Hiệu của tử \& mẫu là $184$; (c) Tích của tử \& mẫu là $2340$.
\end{baitoan}

\begin{baitoan}[\cite{Tuyen_Toan_6}, \textbf{250.}, p. 48]
	Chứng minh: $\forall n\in\mathbb{Z}$, các phân số sau là các phân số tối giản: (a) $\frac{3n - 2}{4n - 3}$; (b) $\frac{4n + 1}{6n + 1}$.
\end{baitoan}

\begin{baitoan}[\cite{Tuyen_Toan_6}, \textbf{251.}, p. 48]
	Cho $\frac{a}{b}$ là 1 phân số chưa tối giản. Chứng minh các phân số sau chưa tối giản: (a) $\frac{a}{a - b}$; (b) $\frac{2a}{a - 2b}$.
\end{baitoan}

\begin{baitoan}[\cite{Tuyen_Toan_6}, \textbf{252.}, p. 48]
	1 mẫu Bắc Bộ bằng $\rm3600m^2$. Hỏi 1 mẫu Bắc Bộ bằng mấy phần của 1 hecta?
\end{baitoan}

\begin{baitoan}[\cite{Binh_Toan_6_tap_2}, Ví dụ 1, p. 4]
	Tìm $n\in\mathbb{N}$ để phân số $A = \frac{n + 10}{2n - 8}\in\mathbb{Z}$ (i.e., có giá trị là 1 số nguyên).
\end{baitoan}

\begin{proof}[Giải]
	Để phân số $A$ có giá trị là 1 số nguyên, tử phải chi hết cho mẫu: $n + 10\divby2n - 8\Rightarrow n + 10\divby n - 4\Rightarrow n - 4 + 14\divby n - 4\Rightarrow14\divby n - 4\Rightarrow n - 4\in\mbox{Ư}(14)\cap\mathbb{Z} = \{\pm1,\pm2,\pm7,\pm14\}$. Vì $n - 4\ge-4$ (vì $n\in\mathbb{N}$, $n\ge 0$) nên $n - 4\in\{\pm1,\pm2,7,14\}$. Nếu $n - 4 = 1$, $n = 5$, $A = \frac{15}{2}$ (loại). Nếu $n - 4 = -1$, $n = 3$, $A = \frac{13}{-2}$ (loại). Nếu $n - 4 = 2$, $n = 6$, $A = \frac{16}{4} = 4$. Nếu $n - 4 = -2$, $n = 2$, $A = \frac{12}{-4} = -3$. Nếu $n - 4 = 7$, $n = 11$, $A = \frac{21}{14} = \frac{3}{2}$ (loại). Nếu $n - 4 = 14$, $n = 18$, $A = \frac{28}{28} = 1$. Vậy $n\in\{2,6,18\}$.
\end{proof}

\begin{baitoan}[Mở rộng \cite{Binh_Toan_6_tap_2}, Ví dụ 1, p. 4]
	Cho $a,b,c,d\in\mathbb{Z}$, $c^2 + d^2\ne0$. Tìm $n\in\mathbb{N}$ để phân số $A = \frac{an + b}{cn + d}\in\mathbb{Z}$.
\end{baitoan}

\begin{baitoan}[\cite{Binh_Toan_6_tap_2}, Ví dụ 2, p. 5]
	Tìm $n\in\mathbb{N}$ để phân số $A = \frac{21n + 3}{6n + 4}$ rút gọn được.
\end{baitoan}

\begin{baitoan}[Mở rộng \cite{Binh_Toan_6_tap_2}, Ví dụ 2, p. 5]
	Cho $a,b,c,d\in\mathbb{Z}$, $c^2 + d^2\ne0$. Tìm $n\in\mathbb{N}$ để phân số $A = \frac{an + b}{cn + d}$ rút gọn được.
\end{baitoan}

\begin{baitoan}[\cite{Binh_Toan_6_tap_2}, Ví dụ 3, p. 5]
	Tìm $a,b,c,d\in\mathbb{N}$ nhỏ nhất sao cho $\frac{a}{b} = \frac{3}{5}$, $\frac{b}{c} = \frac{12}{21}$, $\frac{c}{d} = \frac{6}{11}$.
\end{baitoan}

\begin{baitoan}[\cite{Binh_Toan_6_tap_2}, Ví dụ 4, p. 5]
	Tìm số tự nhiên lớn nhất có 3 chữ số sao cho số đó bằng mỗi tổng $a + b,c + d,e + f$ \& $\frac{a}{b} = \frac{35}{49},\frac{c}{d} = \frac{130}{143},\frac{e}{f} = \frac{7}{13}$.
\end{baitoan}

\begin{baitoan}[\cite{Binh_Toan_6_tap_2}, \textbf{1.}, p. 6]
	Rút gọn phân số: (a) $\frac{199\ldots9}{99\ldots95}$ ($10$ chữ số $9$ ở tử, $10$ chữ số $9$ ở mẫu); (b) $\frac{121212}{424242}$; (c) $\frac{187187187}{221221221}$; (d) $\frac{3\cdot7\cdot13\cdot37\cdot39 - 10101}{505050 + 70707}$.
\end{baitoan}

\begin{baitoan}[\cite{Binh_Toan_6_tap_2}, \textbf{2.}, p. 6]
	Chứng minh các phân số sau có giá trị lfa số tự nhiên: (a) $\frac{10^{2002} + 2}{3}$; (b) $\frac{10^{2003} + 8}{9}$.
\end{baitoan}

\begin{baitoan}[\cite{Binh_Toan_6_tap_2}, \textbf{3.}, p. 6]
	Chứng mih các phân số sau bằng nhau: (a) $\frac{1717}{2929}$ \& $\frac{171717}{292929}$; (b) $\frac{3210 - 34}{4170 - 41}$ \& $\frac{6420 - 68}{8340 - 82}$; (c) $\frac{2106}{7320}$, $\frac{4212}{14640}$, \& $\frac{6318}{21960}$.
\end{baitoan}

\begin{baitoan}[\cite{Binh_Toan_6_tap_2}, \textbf{4.}, p. 6]
	Tìm $x,y\in\mathbb{Z}$ thỏa: (a) $\frac{x}{3} = \frac{y}{5}$; (b) $\frac{x}{28} = \frac{y}{35}$.
\end{baitoan}

\begin{baitoan}[\cite{Binh_Toan_6_tap_2}, \textbf{5.}, p. 6]
	Tìm các phân số $\frac{a}{b}$, $a\in\mathbb{N}$, $b\in\mathbb{N}^\star$, có giá trị bằng: (a) $\frac{36}{45}$ biết $\mbox{\rm BCNN}(a,b) = 300$; (b) $\frac{21}{35}$ biết $\mbox{\rm ƯCLN}(a,b) = 30$; (c) $\frac{15}{35}$ biết $\mbox{\rm ƯCLN}(a,b)\cdot\mbox{\rm BCNN}(a,b) = 3549$.
\end{baitoan}

\begin{baitoan}[\cite{Binh_Toan_6_tap_2}, \textbf{6.}, p. 7]
	Chứng minh các phân số sau tối giản với mọi $n\in\mathbb{N}$. (a) $\frac{n + 1}{2n + 3}$; (b) $\frac{2n + 3}{4n + 8}$; (c) $\frac{3n + 2}{5n + 3}$.
\end{baitoan}

\begin{baitoan}[\cite{Binh_Toan_6_tap_2}, \textbf{7.}, p. 7]
	Cho phân số $A = \frac{63}{3n + 1}$, $n\in\mathbb{N}$. (a) Với giá trị nào của $n$ thì $A$ rút gọn được? (b) Với giá trị nào của $n$ thì $A\in\mathbb{N}$?
\end{baitoan}

\begin{baitoan}[\cite{Binh_Toan_6_tap_2}, \textbf{8.}, p. 7]
	Tìm các số tự nhiên $n$ để các phân số sau là phân số tối giản: (a) $\frac{2n + 3}{4n + 1}$; (b) $\frac{3n + 2}{7n + 1}$; (c) $\frac{2n + 7}{5n + 2}$.
\end{baitoan}

\begin{baitoan}[\cite{Binh_Toan_6_tap_2}, \textbf{9.}, p. 7]
	Có bao nhiêu số nguyên dương $n$ không vượt quá $1000$ để phân số $\frac{n + 12}{n^2 + 9n - 13}$ là phân số tối giản?
\end{baitoan}

\begin{baitoan}[\cite{Binh_Toan_6_tap_2}, \textbf{10.}, p. 7]
	Tìm $n\in\mathbb{N}$ để phân số $\frac{n + 3}{2n - 2}\in\mathbb{Z}$.
\end{baitoan}

\begin{baitoan}[\cite{Binh_Toan_6_tap_2}, \textbf{11.}, p. 7]
	Tìm các số nguyên $n$ sao cho các phân số sau có giá trị là số nguyên: (a) $\frac{12}{3n - 1}$; (b) $\frac{2n + 3}{7}$.
\end{baitoan}

\begin{baitoan}[\cite{Binh_Toan_6_tap_2}, \textbf{12.}, p. 7]
	Tìm $n\in\mathbb{N}$ để phân số $A = \frac{8n + 193}{4n + 3}$: (a) Có giá trị là số tự nhiên; (b) Là phân số tối giản; (c) Với giá trị nào của $n$ trong khoảng từ $150$ đến $170$ thì phân số $A$ rút gọn được?
\end{baitoan}

\begin{baitoan}[\cite{Binh_Toan_6_tap_2}, \textbf{13.}, p. 7]
	Tìm các phân số tối giản nhỏ hơn $1$ có tử \& mẫu đều dương, biết tích của tử \& mẫu của phân số bằng $120$.
\end{baitoan}

\begin{baitoan}[\cite{Binh_Toan_6_tap_2}, \textbf{14.}, p. 7]
	Tìm $n\in\mathbb{N}$ nhỏ nhất để các phân số sau đều là phân số tối giản: $\frac{5}{n + 8},\frac{6}{n + 9},\frac{7}{n + 10},\ldots,\frac{17}{n + 20}$.
\end{baitoan}

\begin{baitoan}[\cite{Binh_Toan_6_tap_2}, \textbf{15.}, p. 7]
	Cho 3 phân số $\frac{15}{42},\frac{49}{56},\frac{36}{51}$. Biến đổi 3 phân số trên thành các phân số bằng chúng sao cho mẫu của phân số thứ nhất bằng tử của phân số thứ 2, mẫu của phân số thứ 2 bằng tử của phân số thứ 3.
\end{baitoan}

\begin{baitoan}[\cite{Binh_Toan_6_tap_2}, \textbf{16.}, p. 7]
	Cho 3 phân số $\frac{5}{8},\frac{11}{20},\frac{4}{15}$. Tìm 3 phân số (có tử \& mẫu dương) theo thứ tự bằng 3 phân số trên sao cho hiệu của mẫu \& tử của mỗi phân số này đều bằng nhau \& hiệu đó có giá trị nhỏ nhất.
\end{baitoan}

\begin{baitoan}[\cite{Binh_Toan_6_tap_2}, \textbf{17.}, p. 8]
	Tìm các phân số lớn hơn $\frac{1}{5}$ \& khác số tự nhiên biết nếu lấy mẫu nhân với 1 số, lấy tử cộng với số đó thì giá trị của phân số không đổi.
\end{baitoan}

\begin{baitoan}[\cite{Binh_Toan_6_tap_2}, \textbf{18.}, p. 8]
	Cho phân số $A = \frac{23 + 22 + 21 + \cdots + 13}{11 + 10 + 9 + \cdots + 1}$. Nêu cách xóa 1 số hạng ở tử \& 1 số hạng ở mẫu của $A$ để được 1 phân số mới vẫn bằng phân số $A$.
\end{baitoan}

\begin{baitoan}[\cite{Binh_Toan_6_tap_2}, \textbf{19.}, p. 8, Bộ sử Hume]
	Người Anh có thói quen xếp bộ sử nước Anh của Hume (David Hume, nhà sử học Scotland) gồm 9 tập ở tủ sách đặc biệt gồm 2 ngăn: ngăn trên xếp 5 cuốn, ngăn dưới xếp 4 cuốn, ở gáy các cuốn sách đó ghi các số $1,2,3,\ldots,9$. Nếu chủ nhân xếp $\frac{13458}{6729}$ (phân số này có giá trị bằng $2$) thì chứng tỏ chủ nhân đã đọc 2 tập (riêng trường hợp mới đọc 1 tập thì xếp $\frac{12345}{6789}$). Nêu cách xếp 9 cuốn sách đó để chứng tỏ chủ nhân của bộ sách đã đọc $3,4,5,6,7,8,9$ tập.
\end{baitoan}

%------------------------------------------------------------------------------%

\section{Quy Đồng Mẫu Số Nhiều Phân Số. So Sánh Phân Số. Hỗn Số Dương}
``\textbf{1.} \textit{Quy tắc quy đồng mẫu nhiều phân số với mẫu dương}: \textit{Bước 1.} Tìm BCNN của các mẫu để làm mẫu chung. \textit{Bước 2.} Tìm thừa số phụ của mỗi mẫu. \textit{Bước 3.} Nhân tử \& mẫu của mỗi phân số với thừa số phụ tương ứng. \textbf{2.} \textit{So sánh 2 phân số}: Muốn so sánh 2 phân số không cùng mẫu ta viết chúng dưới dạng 2 phân số có cùng mẫu dương rồi so sánh các tử với nhau, phân số nào có tử lớn hơn thì phân số đó lớn hơn. \textbf{3.} \textit{Hỗn số dương}: 1 phân số lớn hơn 1 có thể viết dưới dạng 1 hỗn số. Đó là 1 số gồm phần nguyên kèm theo 1 phân số nhỏ hơn 1. \textbf{4.} Trong 2 phân số có tử \& mẫu đều dương, nếu 2 tử số bằng nhau, phân số nào có mẫu nhỏ hơn thì phân số đó sẽ lớn hơn \& ngược lại. \textbf{5.} Phân số có tử \& mẫu là 2 số nguyên cùng dấu thì lớn hơn 0 \& gọi là \textit{phân số dương}. Phân số có tử \& mẫu là 2 số nguyên khác dấu thì nhỏ hơn 0 \& gọi là \textit{phân số âm}.'' -- \cite[Chap. III, \S2, p. 48]{Tuyen_Toan_6}

\begin{baitoan}[Công thức hỗn số dương]
	Chứng minh:
	\begin{align*}
		\frac{ac + b}{c} = a + \frac{b}{c} = a\frac{b}{c},\ \forall a,b,c\in\mathbb{Z},\,c\ne0.\ \ \frac{a}{b} = \frac{\lfloor\frac{a}{b}\rfloor b + \left\{\frac{a}{b}\right\}}{b} = \lfloor\frac{a}{b}\rfloor + \frac{\left\{\frac{a}{b}\right\}}{b} = \lfloor\frac{a}{b}\rfloor\frac{\left\{\frac{a}{b}\right\}}{b}.
	\end{align*}
\end{baitoan}

\begin{baitoan}[\cite{Tuyen_Toan_6}, Ví dụ 52, p. 48]
	So sánh 2 phân số $\frac{-101}{-100}$ \& $\frac{200}{201}$.
\end{baitoan}

\begin{proof}[Giải]
	$\frac{-101}{-100} = \frac{101}{100} > \frac{100}{100} = 1 = \frac{201}{201} > \frac{200}{201}$. Vậy $\frac{-101}{-100} > \frac{200}{201}$.
\end{proof}

\begin{baitoan}[Mở rộng \cite{Tuyen_Toan_6}, Ví dụ 52, p. 48]
	Cho $a,b,c,d\in\mathbb{N}$, $a > b > 0$, $d > c > 0$. So sánh: $\frac{\pm a}{\pm b}$ \& $\frac{\pm c}{\pm d}$.
\end{baitoan}

\begin{baitoan}[\cite{Tuyen_Toan_6}, Ví dụ 53, p. 48]
	Sắp xếp các phân số sau theo thứ tự tăng dần: $\frac{5}{8},\frac{9}{16},\frac{2}{-3},\frac{-7}{12}$.
\end{baitoan}

%------------------------------------------------------------------------------%

\section{1 Số Phương Pháp Đặc Biệt để So Sánh 2 Phân Số}

%------------------------------------------------------------------------------%

\section{$\pm$ Phân Số}

%------------------------------------------------------------------------------%

\section{$\cdot,:$ Phân Số}

%------------------------------------------------------------------------------%

\section{Tổng Các Phân Số Viết Theo Quy Luật}

%------------------------------------------------------------------------------%

\section{Số Thập Phân. Làm Tròn Số Thập Phân. Các Phép Tính với Số Thập Phân}

%------------------------------------------------------------------------------%

\section{Tìm Giá Trị Phân Số của 1 Số Cho Trước. Tìm 1 Số Biết Giá Trị 1 Phân Số của Nó}

%------------------------------------------------------------------------------%

\section{Tìm Tỷ Số \& Tỷ Số \% của 2 Đại Lượng}

%------------------------------------------------------------------------------%

\section{Toán về Công Việc Làm Đồng Thời}

%------------------------------------------------------------------------------%

\section{Miscellaneous}

%------------------------------------------------------------------------------%

\printbibliography[heading=bibintoc]
	
\end{document}