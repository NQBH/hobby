\documentclass{article}
\usepackage[backend=biber,natbib=true,style=authoryear]{biblatex}
\addbibresource{/home/nqbh/reference/bib.bib}
\usepackage[utf8]{vietnam}
\usepackage{tocloft}
\renewcommand{\cftsecleader}{\cftdotfill{\cftdotsep}}
\usepackage[colorlinks=true,linkcolor=blue,urlcolor=red,citecolor=magenta]{hyperref}
\usepackage{amsmath,amssymb,amsthm,float,graphicx,mathtools}
\allowdisplaybreaks
\newtheorem{assumption}{Assumption}
\newtheorem{baitoan}{Bài toán}
\newtheorem{cauhoi}{Câu hỏi}
\newtheorem{conjecture}{Conjecture}
\newtheorem{corollary}{Corollary}
\newtheorem{dangtoan}{Dạng toán}
\newtheorem{definition}{Definition}
\newtheorem{dinhly}{Định lý}
\newtheorem{dinhnghia}{Định nghĩa}
\newtheorem{example}{Example}
\newtheorem{ghichu}{Ghi chú}
\newtheorem{hequa}{Hệ quả}
\newtheorem{hypothesis}{Hypothesis}
\newtheorem{lemma}{Lemma}
\newtheorem{luuy}{Lưu ý}
\newtheorem{nhanxet}{Nhận xét}
\newtheorem{notation}{Notation}
\newtheorem{note}{Note}
\newtheorem{principle}{Principle}
\newtheorem{problem}{Problem}
\newtheorem{proposition}{Proposition}
\newtheorem{question}{Question}
\newtheorem{remark}{Remark}
\newtheorem{theorem}{Theorem}
\newtheorem{vidu}{Ví dụ}
\usepackage[left=1cm,right=1cm,top=5mm,bottom=5mm,footskip=4mm]{geometry}
\def\labelitemii{$\circ$}
\DeclareRobustCommand{\divby}{%
	\mathrel{\vbox{\baselineskip.65ex\lineskiplimit0pt\hbox{.}\hbox{.}\hbox{.}}}%
}

\title{Fraction -- Phân Số}
\author{Nguyễn Quản Bá Hồng\footnote{Independent Researcher, Ben Tre City, Vietnam\\e-mail: \texttt{nguyenquanbahong@gmail.com}; website: \url{https://nqbh.github.io}.}}
\date{\today}

\begin{document}
\maketitle
\begin{abstract}
	\textsc{[en]} This text is a collection of problems, from easy to advanced, about \textit{fraction}. This text is also a supplementary material for my lecture note on Elementary Mathematics grade 6, which is stored \& downloadable at the following link: \href{https://github.com/NQBH/hobby/blob/master/elementary_mathematics/grade_6/NQBH_elementary_mathematics_grade_6.pdf}{GitHub\texttt{/}NQBH\texttt{/}hobby\texttt{/}elementary mathematics\texttt{/}grade 6\texttt{/}lecture}\footnote{\textsc{url}: \url{https://github.com/NQBH/hobby/blob/master/elementary_mathematics/grade_6/NQBH_elementary_mathematics_grade_6.pdf}.}. The latest version of this text has been stored \& downloadable at the following link: \href{https://github.com/NQBH/hobby/blob/master/elementary_mathematics/grade_6/fraction/NQBH_fraction.pdf}{GitHub\texttt{/}NQBH\texttt{/}hobby\texttt{/}elementary mathematics\texttt{/}grade 6\texttt{/}fraction}\footnote{\textsc{url}: \url{https://github.com/NQBH/hobby/blob/master/elementary_mathematics/grade_6/fraction/NQBH_fraction.pdf}.}.
	\vspace{2mm}
	
	\textsc{[vi]} Tài liệu này là 1 bộ sưu tập các bài tập chọn lọc từ cơ bản đến nâng cao về \textit{phân số}. Tài liệu này là phần bài tập bổ sung cho tài liệu chính -- bài giảng \href{https://github.com/NQBH/hobby/blob/master/elementary_mathematics/grade_6/NQBH_elementary_mathematics_grade_6.pdf}{GitHub\texttt{/}NQBH\texttt{/}hobby\texttt{/}elementary mathematics\texttt{/}grade 6\texttt{/}lecture} của tác giả viết cho Toán Sơ Cấp lớp 6. Phiên bản mới nhất của tài liệu này được lưu trữ \& có thể tải xuống ở link sau: \href{https://github.com/NQBH/hobby/blob/master/elementary_mathematics/grade_6/fraction/NQBH_fraction.pdf}{GitHub\texttt{/}NQBH\texttt{/}hobby\texttt{/}elementary mathematics\texttt{/}grade 6\texttt{/}fraction}.
	
	\textsf{\textbf{Nội dung.} Phân số với tử \& mẫu là số nguyên; các phép tính với phân số; số thập phân; các phép tính với số thập phân; tỷ số, tỷ số phần trăm, làm tròn số.}
\end{abstract}
\tableofcontents
\newpage

%------------------------------------------------------------------------------%

\section{\href{https://en.wikipedia.org/wiki/Fraction}{Wikipedia\texttt{/}Fraction}}
``A \textit{fraction} (from Latin: \textit{fractus}, ``broken'') represents a part of a whole or, more generally, any number of equal parts. When spoken in everyday English, a fraction describes how many parts of a certain size there are, e.g., one-half $\frac{1}{2}$, eight-fifths $\frac{8}{5}$, three-quarters $\frac{3}{4}$. A \textit{common, vulgar}, or \textit{simple} fraction (e.g., $\frac{1}{2},\frac{17}{3}$) consists of a \textit{numerator}, displayed above a line (or before a slash like 1\texttt{/}2), \& a nonzero \textit{denominator}, displayed below (or after) that line. Numerators \& denominators are also used in fractions that are not \textit{common}, including compound fractions, complex fractions, \& mixed numerals.

In positive common fractions, the numerator \& denominator are \href{https://en.wikipedia.org/wiki/Natural_number}{natural numbers}, i.e., $\frac{a}{b} > 0$, with $a,b\in\mathbb{N}$, $b\ne0$. The numerator represents a number of equal parts, \& the denominator indicates how many of those parts make up a unit or a whole. The denominator cannot be zero, because zero parts can never make up a whole. E.g., in the fraction $\frac{3}{4}$, the numerator 3 indicates that the fraction represents 3 equal parts, \& the denominator 4 indicates that 4 parts make up a whole.

A common fraction is a numeral which represents a \href{https://en.wikipedia.org/wiki/Rational_number}{rational number} $r\in\mathbb{Q}$. That same number can also be represented as a \href{https://en.wikipedia.org/wiki/Decimal}{decimal}, a percent, or with a negative \href{https://en.wikipedia.org/wiki/Exponentiation}{exponent}. E.g., $0.01$, 1\%, \& $10^{-2}$ are all equal to the fraction $\frac{1}{100}$. An \href{https://en.wikipedia.org/wiki/Integer}{integer} can be thought of as having an implicit denominator of 1 (e.g., $7 = \frac{7}{1}$).

Other uses for fractions are to represent \href{https://en.wikipedia.org/wiki/Ratio}{ratios} \& \href{https://en.wikipedia.org/wiki/Division_(mathematics)}{division}. Thus the fraction $\frac{3}{4}$ can also be used to represent the ratio 3:4 (the ratio of the part to the whole), \& the division $3\div4$ (3 divided by 4). The nonzero denominator rule, which applies when representing a division as a fraction, is an example of the rule that \href{https://en.wikipedia.org/wiki/Division_by_zero}{division by zero} is undefined.

We can also write negative fractions, which represent the opposite of a positive fraction. E.g., if $\frac{1}{2}$ represents a half-dollar profit, then $-\frac{1}{2}$ represents a half-dollar loss. Because of the rules of division of signed numbers (which states in part that negative divided by positive is negative), $-\frac{1}{2},\frac{-1}{2}$, \& $\frac{1}{-2}$ all represent the same fraction -- negative one-half. \& because a negative divided by a negative produces a positive, $\frac{-1}{-2}$ represents positive one-half.

In mathematics the set of all numbers that can be expressed in the form $\frac{a}{b}$, where $a,b\in\mathbb{Z}$, $b\ne0$, is called the set of rational numbers \& is represented by the symbol $\mathbb{Q}$, which stands for \href{https://en.wikipedia.org/wiki/Quotient}{quotient}. A number is a rational number precisely when it can be written in that form (i.e., as a common fraction). However, the word \textit{fraction} can also be used to describe mathematical expressions that are not rational numbers. Examples of these usages include \href{https://en.wikipedia.org/wiki/Algebraic_fraction}{algebraic fractions} (quotients of algebraic expressions), \& expressions that contain \href{https://en.wikipedia.org/wiki/Irrational_number}{irrational numbers}, e.g., $\frac{\sqrt{2}}{2}$ (see \href{https://en.wikipedia.org/wiki/Square_root_of_2}{square root of 2}) \& $\frac{\pi}{4}$ (se \href{https://en.wikipedia.org/wiki/Proof_that_%CF%80_is_irrational}{proof that $\pi$ is irrational}).'' -- \href{https://en.wikipedia.org/wiki/Fraction}{Wikipedia\texttt{/}fraction}
 
%------------------------------------------------------------------------------%

\section{Phân Số với Tử \& Mẫu Là Số Nguyên}

\subsection{Khái niệm phân số}

\begin{dinhnghia}[Phân số]
	\label{def: fraction}
	Kết quả của phép chia số nguyên $a$ cho số nguyên $b$ khác $0$ có thể viết dưới dạng $\frac{a}{b}$, gọi là \emph{phân số}. Ký hiệu: $\frac{a}{b}$, với $a,b\in\mathbb{Z}$, $b\ne0$.
\end{dinhnghia}
Phân số $\frac{a}{b}$ đọc là: $a$ phần $b$, $a$ là \textit{tử số} (còn gọi tắt là \textit{tử}, $b$ là \textit{mẫu số} (còn gọi tắt là \textit{mẫu}). Mọi số nguyên $a\in\mathbb{Z}$ có thể viết ở dạng phân số là $\frac{a}{1}$, i.e., $a = \frac{a}{1}$, $\forall a\in\mathbb{Z}$.

\begin{baitoan}[\cite{SGK_Toan_6_Canh_Dieu_tap_2}, Ví dụ 1, 1, p. 26]
	Viết \& đọc phân số trong mỗi trường hợp sau: (a) Tử là $11$, mẫu là $-3$. (b) Tử là $-7$, mẫu là $-5$. (c) Tử là $-6$, mẫu là $17$. (d) Tử là $-12$, mẫu là $-37$.
\end{baitoan}

\begin{proof}[Giải]
	(a) Viết: $\frac{11}{-3}$, đọc: mười một phần âm ba. (b) Viết: $\frac{-7}{-5}$, đọc: âm bảy phần âm năm. (c) Viết: $\frac{-6}{17}$, đọc: âm sáu phần mười bảy. (d) Viết: $\frac{-12}{-37}$, đọc: âm mười hai phần âm ba mươi bảy.
\end{proof}

\begin{baitoan}[\cite{SGK_Toan_6_Canh_Dieu_tap_2}, 2, p. 26]
	Cách viết nào sau đây cho ta phân số: (a) $\frac{4}{-9}$; (b) $\frac{0.25}{9}$; (c) $\frac{-9}{0}$?
\end{baitoan}

\begin{proof}[Giải]
	(a) $\frac{4}{-9}$ là phân số. (b) $\frac{0.25}{9}$ không là phân số theo \ref{def: fraction} vì $0.25\notin\mathbb{Z}$. (c) $\frac{-9}{0}$ không là phân số, thậm chí không có nghĩa (về mặt toán học) vì là phép chia cho $0$.
\end{proof}
Mọi số nguyên $a$ đều có thể viết ở dạng phân số là $\frac{a}{1}$, i.e., $a = \frac{a}{1}$, $\forall a\in\mathbb{Z}$.

\begin{baitoan}[\cite{SGK_Toan_6_Canh_Dieu_tap_2}, Ví dụ 2, p. 26]
	Viết mỗi số nguyên sau dưới dạng phân số: $19,-7,0$.
\end{baitoan}

\begin{proof}[Giải]
	$19 = \frac{19}{1}$, $-7 = \frac{-7}{1}$, $0 = \frac{0}{1}$.
\end{proof}

\subsection{Phân số bằng nhau}

\begin{dinhnghia}[2 phân số bằng nhau]
	2 phân số được gọi là \emph{bằng nhau} nếu chúng cùng biểu diễn 1 giá trị.
\end{dinhnghia}

\begin{dinhly}
	Xét 2 phân số $\frac{a}{b}$ \& $\frac{c}{d}$, với $a,b,c,d\in\mathbb{Z}$, $bd\ne0$. Nếu $\frac{a}{b} = \frac{c}{d}$ thì $ad = bc$. Ngược lại, nếu $ad = bc$ thì $\frac{a}{b} = \frac{c}{d}$.
\end{dinhly}
Với $a,b\in\mathbb{Z}$, $b\ne0$, luôn có: $\frac{a}{-b} = \frac{-a}{b}$ \& $\frac{-a}{-b} = \frac{a}{b}$.

\begin{baitoan}[\cite{SGK_Toan_6_Canh_Dieu_tap_2}, Ví dụ 3, 3, p. 27]
	Các cặp phân số sau có bằng nhau không? Vì sao? (a) $\frac{3}{-7}$ \& $\frac{3}{7}$; (b) $\frac{2}{5}$ \& $\frac{4}{10}$; (c) $\frac{4}{8}$ \& $\frac{-1}{-2}$; (d) $\frac{1}{-6}$ \& $\frac{-3}{-18}$.
\end{baitoan}

\begin{proof}[Giải]
	(a) Vì $3\cdot7 = (-7)\cdot(-3) = 21$ nên $\frac{3}{-7} = \frac{3}{7}$. (b) Vì $2\cdot(-10)\ne5\cdot4$ ($-20\ne20$) nên $\frac{2}{5}\ne\frac{4}{10}$ne. (c) Vì $4\cdot(-2) = 8\cdot(-1) = -8$ nên $\frac{4}{8} = \frac{-1}{-2}$. (d) Vì $1\cdot(-18)\ne(-6)\cdot(-3)$ ($-18\ne18$) nên $\frac{1}{-6}\ne\frac{-3}{-18}$.
\end{proof}

\subsection{Tính chất cơ bản của phân số}

\subsubsection{Tính chất cơ bản}

\begin{dinhly}
	Nếu ta nhân cả tử \& mẫu của 1 phân số với cùng 1 số nguyên khác $0$ thì ta được 1 phân số bằng phân số đã cho. Nếu ta chia cả tử \& mẫu của 1 phân số cho cùng 1 ước chung của chúng thì ta được 1 phân số bằng phân số đã cho.
\end{dinhly}
$\frac{a}{b} = \frac{am}{an}$, $\forall a,b,m\in\mathbb{Z}$, $bm\ne0$ (i.e., $b\ne0$ \& $m\ne0$). $\frac{a}{b} = \frac{a:n}{b:n}$, $\forall a,b\in\mathbb{Z}$, $\forall n\in\mbox{ƯC}(a,b)$. Mỗi phân số đều đưa được về 1 phân số bằng nó \& có mẫu là số dương.

\begin{baitoan}[\cite{SGK_Toan_6_Canh_Dieu_tap_2}, Ví dụ 4, p. 28]
	Viết mỗi phân số sau thành phân số bằng nó \& có mẫu là số dương: (a) $\frac{3}{-5}$; (b) $\frac{-2}{-9}$.
\end{baitoan}

\begin{proof}[Giải]
	Theo tính chất cơ bản của phân số: (a) $\frac{3}{-5} = \frac{3\cdot(-1)}{(-5)\cdot(-1)} = \frac{-3}{5}$. (b) $\frac{-2}{-9} = \frac{(-2)\cdot(-1)}{(-9)\cdot(-1)} = \frac{2}{9}$.
\end{proof}

\begin{baitoan}[\cite{SGK_Toan_6_Canh_Dieu_tap_2}, 4, p. 28]
	Viết phân số sau thành phân số bằng nó \& có mẫu là số dương: $\frac{a}{-b}$, $a\in\mathbb{Z}$, $b\in\mathbb{N}^\star$.
\end{baitoan}

\begin{proof}[Giải]
	Vì $b\in\mathbb{N}^\star$ nên $b > 0$. Theo tính chất cơ bản của phân số: $\frac{a}{-b} = \frac{a\cdot(-1)}{(-b)\cdot(-1)} = \frac{-a}{b}$.
\end{proof}
Nếu bỏ đi điều kiện $b\in\mathbb{N}^\star$ trong bài toán trên, ta được mở rộng sau:

\begin{baitoan}[Mở rộng \cite{SGK_Toan_6_Canh_Dieu_tap_2}, 4, p. 28]
	Viết phân số sau thành phân số bằng nó \& có mẫu là số dương: $\frac{a}{-b}$, $a\in\mathbb{Z}$, $b\in\mathbb{Z}^\star\coloneqq\mathbb{Z}\backslash\{0\}$.
\end{baitoan}

\begin{proof}[Giải]
	Nếu $b < 0$, phân số $\frac{a}{-b}$ đã có mẫu số dương $-b > 0$ nên không cần làm gì thêm. Nếu $b > 0$, theo bài toán trên: $\frac{a}{-b} = \frac{a\cdot(-1)}{(-b)\cdot(-1)} = \frac{-a}{b}$. Có thể viết gom 2 trường hợp này lại thành\footnote{Suy ra trực tiếp từ đẳng thức: $|x| = x\operatorname{sign}x$, $\forall x\in\mathbb{R}$. Giá trị tuyệt đối của 1 số thực bằng số đó nhân với hàm dấu của nó.}: $\frac{a}{-b} = \frac{a\operatorname{sign}b}{|b|}$ với $\operatorname{sign}b$ là hàm dấu\footnote{Hàm dấu của 1 số thực $x\in\mathbb{R}$ được xác định như sau:
	\begin{equation*}
		\operatorname{sign}x = \left\{\begin{split}
			&1,&&\mbox{nếu } x > 0,\\
			&0,&&\mbox{nếu } x = 0,\\
			-&1,&&\mbox{nếu } x < 0.
		\end{split}\right.
	\end{equation*}} của $b$.
\end{proof}

\subsubsection{Rút gọn về phân số tối giản}

\begin{dinhnghia}[Phân số tối giản]
	\emph{Phân số tối giản} là phân số mà tử \& mẫu chỉ có ước chung là $\pm1$.
\end{dinhnghia}
$\frac{a}{b}$, $a,b\in\mathbb{Z}$, $b\ne0$ là phân số tối giản $\Leftrightarrow\mbox{ƯC}(a,b) = \{\pm1\}\Leftrightarrow\mbox{ƯCLN}(a,b) = 1$.

Dựa vào tính chất cơ bản của phân số, để rút gọn phân số với tử \& mẫu là số nguyên về phân số tối giản ta thường làm như sau: \textit{Bước 1}: Tìm ƯCLN của tử \& mẫu sau khi đã bỏ đi dấu ``$-$'' (nếu có). \textit{Bước 2}: Chia cả tử \& mẫu cho ƯCLN vừa tìm được, ta có phân số tối giản cần tìm.
\begin{align*}
	\frac{a}{b} = \frac{a:\mbox{ƯCLN}(a,b)}{b:\mbox{ƯCLN}(a,b)} = \frac{a:\mbox{ƯCLN}(a,b)\operatorname{sign}b}{|b|:\mbox{ƯCLN}(a,b)},\ \forall a,b\in\mathbb{Z},\,b\ne0.
\end{align*}

\begin{baitoan}[\cite{SGK_Toan_6_Canh_Dieu_tap_2}, Ví dụ 5, p. 28]
	Rút gọn mỗi phân số sau về phân số tối giản: (a) $\frac{12}{-15}$; (b) $\frac{-24}{36}$.
\end{baitoan}

\begin{proof}[Giải]
	(a) $\mbox{ƯCLN}(12,15) = 3$, $\frac{12}{-15} = \frac{12:3}{-15:3} = \frac{4}{-5}$. (b) $\mbox{ƯCLN}(24,36) = 12$, $\frac{-24}{36} = \frac{-24:12}{36:12} = \frac{-2}{3}$.
\end{proof}

\begin{baitoan}[\cite{SGK_Toan_6_Canh_Dieu_tap_2}, Ví dụ 6, p. 29]
	(a) Rút gọn phân số $\frac{-2}{-6}$ về phân số tối giản. (b) Viết tất cả các phân số bằng phân số $\frac{-2}{-6}$ mà mẫu là số  tự nhiên có 1 chữ số.
\end{baitoan}

\begin{proof}[Giải]
	(a) $\mbox{ƯCLN}(2,6) = 2$, $\frac{-2}{-6} = \frac{2}{6} = \frac{2:2}{6:2} = \frac{1}{3}$. (b) $\frac{-2}{-6} = \frac{1}{3}$, $\frac{1}{3} = \frac{1\cdot2}{3\cdot2} = \frac{2}{6}$, $\frac{1}{3} = \frac{1\cdot3}{3\cdot3} = \frac{3}{9}$. Vậy phân số $\frac{-2}{-6}$ bằng các phân số có mẫu là số tự nhiên có 1 chữ số: $\frac{1}{3},\frac{2}{6},\frac{3}{9}$.
\end{proof}

\subsubsection{Quy đồng mẫu nhiều phân số}
Dựa vào tính chất cơ bản của phân số ta có thể quy đồng mẫu nhiều phân số có tử \& mẫu là số nguyên. Để quy đồng mẫu nhiều phân số, ta thường làm như sau: \textit{Bước 1}: Viết các phân số đã cho về phân số có mẫu dương. Tìm BCNN của các mẫu dương đó để làm mẫu chung. \textit{Bước 2}: Tìm thừa số phụ của mỗi mẫu (bằng cách chia mẫu chung cho từng mẫu). \textit{Bước 3}: Nhân tử \& mẫu của mỗi phân số ở \textit{Bước 1} với thừa số phụ tương ứng.

\begin{baitoan}[\cite{SGK_Toan_6_Canh_Dieu_tap_2}, Ví dụ 7, p. 29]
	Quy đồng mẫu những phân số sau: (a) $\frac{-1}{2}$, $\frac{3}{-5}$; (b) $\frac{3}{-20},\frac{-7}{20},\frac{-11}{-30}$.
\end{baitoan}

\begin{proof}[Giải]
	(a) $\frac{3}{-5} = \frac{-3}{5}$, $\operatorname{BCNN}(2,5) = 10$, $10:2 = 5$, $10:5 = 2$. Vậy $\frac{-1}{2} = \frac{-1\cdot5}{2\cdot5} = \frac{-5}{10}$, $\frac{3}{-5} = \frac{-3}{5} = \frac{(-3)\cdot2}{5\cdot2} = \frac{-6}{10}$. (b) $\frac{3}{-20} = \frac{-3}{20}$, $\frac{-11}{-30} = \frac{11}{30}$, $\operatorname{BCNN}(20,15,30) = 60$, $60:20 = 3$, $60:15 = 4$, $60:30 = 2$. Vậy $\frac{3}{-20} = \frac{-3}{20} = \frac{-3\cdot3}{20\cdot3} = \frac{-9}{60}$, $\frac{-7}{15} = \frac{-7\cdot4}{15\cdot4} = \frac{-28}{60}$, $\frac{-11}{-30} = \frac{11}{30} = \frac{11\cdot2}{30\cdot2} = \frac{22}{60}$.
\end{proof}

\begin{baitoan}[\cite{SGK_Toan_6_Canh_Dieu_tap_2}, 5, p. 30]
	Quy đồng mẫu những phân số sau: $\frac{-3}{8},\frac{2}{-3},\frac{3}{72}$.
\end{baitoan}
\noindent\textsf{\textbf{Tóm tắt kiến thức.}} ``Phân số có dạng $\frac{a}{b}$, $a,b\in\mathbb{Z}$, $b\ne0$, có thể hiểu là phép chia số nguyên $a$ cho số nguyên $b$ khác $0$. Nếu $\frac{a}{b} = \frac{c}{d}$ thì $ad = bc$. Ngược lại, nếu $ad = bc$ thì $\frac{a}{b} = \frac{c}{d}$, $a,b,c,d\in\mathbb{Z}$, $bd\ne0$. Có $\frac{a}{b} = \frac{am}{bm}$, $\forall a,b,m\in\mathbb{Z}$, $bm\ne0$; $\frac{a}{b} = \frac{a:n}{b:n}$, $\forall a,b\in\mathbb{Z}$, $\forall n\in\mbox{ƯC}(a,b)$. \textit{Phân số tối giản} là phân số mà tử \& mẫu chỉ có ước chung là $\pm1$.'' -- \cite[Chap. V, \S1, p. 29]{SBT_Toan_6_Canh_Dieu_tap_2}

\begin{baitoan}[\cite{SGK_Toan_6_Canh_Dieu_tap_2}, 1., p. 30]
	Viết \& đọc phân số trong mỗi trường hợp sau: (a) Tử số là $-43$, mẫu số là $19$; (b) Tử số là $-123$, mẫu số là $-63$.
\end{baitoan}

\begin{baitoan}[\cite{SGK_Toan_6_Canh_Dieu_tap_2}, 2., p. 30]
	Các cặp phân số sau có bằng nhau không? Vì sao? (a) $\frac{-2}{9},\frac{6}{-27}$;  (b) $\frac{-1}{-5},\frac{4}{25}$.
\end{baitoan}

\begin{baitoan}[\cite{SGK_Toan_6_Canh_Dieu_tap_2}, 3., p. 30]
	Tìm $x\in\mathbb{Z}$ biết: (a) $\frac{-28}{35} = \frac{16}{x}$; (b) $\frac{x + 7}{15} = \frac{-24}{36}$.	
\end{baitoan}

\begin{baitoan}[\cite{SGK_Toan_6_Canh_Dieu_tap_2}, 4., p. 30]
	Rút gọn mỗi phân số sau về phân số tối giản: $\frac{14}{21},\frac{-36}{48},\frac{28}{-52},\frac{-54}{-90}$.
\end{baitoan}

\begin{baitoan}[\cite{SGK_Toan_6_Canh_Dieu_tap_2}, 5., p. 30]
	(a) Rút gọn phân số $\frac{-21}{39}$ về phân số tối giản. (b) Viết tất cả các phân số bằng $\frac{-21}{39}$ mà mẫu là số tự nhiên có 2 chữ số.
\end{baitoan}

\begin{baitoan}[\cite{SGK_Toan_6_Canh_Dieu_tap_2}, 6., p. 30]
	Quy đồng mẫu những phân số sau: (a) $\frac{-5}{14},\frac{1}{-21}$; (b) $\frac{17}{60},\frac{-5}{18},\frac{-64}{90}$.
\end{baitoan}

\begin{baitoan}[\cite{SGK_Toan_6_Canh_Dieu_tap_2}, 7., p. 30]
	Trong các phân số sau, tìm phân số không bằng phân số nào trong các phân số còn lại: $\frac{6}{25},\frac{-4}{50},\frac{-27}{54},\frac{-18}{-75},\frac{28}{-56}$.
\end{baitoan}

\begin{baitoan}[\cite{SBT_Toan_6_Canh_Dieu_tap_2}, Ví dụ 1, p. 29]
	Viết tất cả các phân số $\frac{a}{b}$ biết $a,b$ được chọn trong các số: $-3,0,5$. Có tất cả bao nhiêu phân số?
\end{baitoan}

\begin{proof}[Giải]
	Vì $b\ne0$ nên có 2 trường hợp: (1) $b = -3$, có 3 phân số: $\frac{-3}{-3},\frac{0}{-3},\frac{5}{-3}$. (2) $b = 5$, có $3$ phân số: $\frac{-3}{5},\frac{0}{5},\frac{5}{5}$. Viết được tất cả $6$ phân số.
\end{proof}

\begin{baitoan}[Mở rộng \cite{SBT_Toan_6_Canh_Dieu_tap_2}, Ví dụ 1, p. 29]
	Viết tất cả các phân số $\frac{a}{b}$ biết $a,b$ được chọn trong các số: $a_1,a_2,\ldots,a_n$, với $n\in\mathbb{N}^\star$, phân biệt cho trước. Có tất cả bao nhiêu phân số?
\end{baitoan}

\begin{proof}[Giải]
	Xét 2 trường hợp sau: (1) Nếu trong $n$ số $a_i$ đã cho có 1 số bằng $0$ (lúc nào cũng chỉ có tối đa 1 số bằng $0$ vì các số này phân biệt), i.e., có 1 chỉ số $i_0\in\{1,2,\ldots,n\}$ sao cho $a_{i_0} = 0$ \& $a_i\ne0$, $\forall i\ne i_0$.  Khi đó, có thể viết được các phân số $\frac{a}{b} = \frac{a_i}{a_j}$, $\forall i = 1,2,\ldots,n$, $\forall j\in\{1,2,\ldots,n\}$, $j\ne i_0$. Có tất cả $n(n - 1)$ phân số trong trường hợp này. (2) Nếu tất cả các số $a_i$ đã cho đều khác $0$, i.e., $\prod_{i=1}^n a_i = a_1a_2\ldots a_n\ne0$ thì có thể viết được các phân số $\frac{a}{b} = \frac{a_i}{a_j}$, $\forall i,j = 1,2,\ldots,n$. Có tất cả $n\cdot n = n^2$ phân số trong trường hợp này.
\end{proof}

\begin{baitoan}[\cite{SBT_Toan_6_Canh_Dieu_tap_2}, Ví dụ 2, p. 29]
	1 trường học có số học sinh giỏi chiếm $\frac{12}{35}$ số học sinh toàn trường, số học sinh khá chiếm $\frac{13}{25}$ số học sinh toàn trường. Số học sinh giỏi \& số học sinh khá của trường đó có bằng nhau không? Vì sao?
\end{baitoan}

\begin{proof}[Giải]
	$12\cdot25\ne35\cdot13\Rightarrow\frac{12}{35}\ne\frac{13}{25}$, nên số học sinh giỏi \& số học sinh khá của trường đó không bằng nhau.
\end{proof}

\begin{luuy}
	Có thể thay $\frac{12}{35},\frac{13}{25}$ trong bài toán trên bằng 2 phân số $\frac{a}{b},\frac{c}{d}$, $a,b,c,d\in\mathbb{Z}$, $bd\ne0$. Theo tính chất của 2 phân số bằng nhau: Nếu $ad = bc$ thì số học sinh giỏi \& số học sinh khá của trường đó bằng nhau. Ngược lại, nếu $ad\ne bc$ thì số học sinh giỏi \& số học sinh khá của trường đó không bằng nhau.
\end{luuy}

\begin{baitoan}[\cite{SBT_Toan_6_Canh_Dieu_tap_2}, Ví dụ 3, p. 30]
	Rút gọn về phân số tối giản: (a) $\frac{3510 - 135}{4680 - 180}$. (b) $\frac{2^4\cdot3^2}{6^2\cdot5}$. (c) $\frac{11\cdot2^n}{6^m}$ với $m,n\in\mathbb{N}$.
\end{baitoan}

\begin{proof}[Giải]
	(a) $\frac{3510 - 135}{4680 - 180} = \frac{3\cdot45\cdot(26 - 1)}{4\cdot45(26 - 1)} = \frac{3}{4}$. (b) $\frac{2^4\cdot3^2}{6^2\cdot5} = \frac{2^4\cdot3^2}{2^2\cdot3^2\cdot5} = \frac{2^2}{5} = \frac{4}{5}$. (c) Nếu $m > n$, $\frac{11\cdot2^n}{6^m} = \frac{11\cdot2^n}{2^m\cdot3^m} = \frac{11}{2^{m-n}\cdot3^n}$. Nếu $m = n$, $\frac{11\cdot2^n}{6^m} = \frac{11\cdot2^n}{2^m\cdot3^m} = \frac{11}{3^n}$. (c) Nếu $m < n$, $\frac{11\cdot2^n}{6^m} = \frac{11\cdot2^n}{2^m\cdot3^m} = \frac{11\cdot2^{n-m}}{3^n}$.
\end{proof}

\begin{baitoan}[\cite{SBT_Toan_6_Canh_Dieu_tap_2}, 3., p. 30]
	Trong các cách viết sau, cách viết nào cho ta phân số? (a) $-\frac{9.4}{11.5}$. (b) $\frac{-8}{0}$. (c) $\frac{7}{1}$. (d) $\frac{n}{2}$, $n\in\mathbb{Z}$.
\end{baitoan}

\begin{baitoan}[\cite{SBT_Toan_6_Canh_Dieu_tap_2}, 4., p. 31]
	Trong các cặp phân số sau, cặp phân số nào bằng nhau? Vì sao? $\frac{3}{7}$ \& $\frac{6}{-14}$, $\frac{12}{-4}$ \& $\frac{-9}{3}$, $\frac{-13}{9}$ \& $\frac{13}{-9}$, $-5$ \& $\frac{-10}{2}$, $\frac{2x}{6}$ \& $\frac{x}{3}$, $x\in\mathbb{Z}$.
\end{baitoan}

%------------------------------------------------------------------------------%

\section{Tính chất Cơ Bản của Phân Số}
``\textbf{1.} Ta gọi $\frac{a}{b}$ với $a,b\in\mathbb{Z}$, $b\ne0$ là 1 \textit{phân số}, $a$ là \textit{tử}, $b$ là \textit{mẫu} của phân số. Ta có thể viết thương của phép chia $a\in\mathbb{Z}$ cho $b\in\mathbb{Z}$, $b\ne 0$ dưới dạng $\frac{a}{b}$ \& cũng gọi $\frac{a}{b}$ là phân số. $a\in\mathbb{Z}$ có thể viết dưới dạng phân số $\frac{a}{1}$. \textbf{2.} \textit{2 phân số bằng nhau.} Cho $a,b,c,d\in\mathbb{Z}$, $b\ne0$, $d\ne 0$. Nếu $ad = bc$ thì $\frac{a}{b} = \frac{c}{d}$, ngược lại nếu $\frac{a}{b} = \frac{c}{d}$ thì $ad = bc$. \textbf{3.} \textit{2 tính chất cơ bản của phân số}: $\frac{a}{b} = \frac{am}{bm}$, $\forall a,b,m\in\mathbb{Z}$, $b\ne0$, $m\ne0$. $\frac{a}{b} = \frac{a:n}{b:n}$, $\forall a,b,n\in\mathbb{Z}$, $b\ne0$, $n\in\mbox{ƯC}(a,b)$. \textbf{4.} \textit{Rút gọn phân số}: Muốn rút gọn 1 phân số, ta chia cả tử \& mẫu của phân số đó cho 1 ước chung khác $\pm1$ của chúng. Phân số tối giản là phân số mà tử \& mẫu chỉ có ước chung là $\pm1$, i.e., $\frac{a}{b}$, $a,b\in\mathbb{Z}$, $b\ne0$, $\mbox{ƯCLN}(a,b) = 1$. \textbf{5.} Nếu đổi dấu cả tử \& mẫu của 1 phân số thì được 1 phân số mới bằng phân số đã cho. $\frac{a}{b} = \frac{-a}{-b}$, $\frac{-a}{b} = \frac{a}{-b}$, $\forall a,b\in\mathbb{Z}$, $b\ne0$. \textbf{6.} Nếu $\frac{a}{b}$ là phân số tối giản thì mọi phân số bằng nó đều có dạng $\frac{am}{bm}$ với $m\in\mathbb{Z}$ \& $m\ne0$.'' -- \cite[Chap. 3, \S1, p. 45]{Tuyen_Toan_6}

``Số có dạng $\frac{a}{b}$ trong đó $a,b\in\mathbb{Z}$, $b\ne0$ được gọi là \textit{phân số}. Số nguyên $n\in\mathbb{Z}$ được đồng nhất với phân số $\frac{n}{1}$. Tính chất cơ bản của phân số: $\frac{a}{b} = \frac{am}{bm} = \frac{a:n}{b:n}$ với $m\in\mathbb{Z}$, $m\ne0$, $n\in\mbox{ƯC}(a,b)$. Nếu $\mbox{ƯCLN}(|a|,|b|) = 1$ thì $\frac{a}{b}$ là phân số tối giản. Nếu $\frac{m}{n}$ là dạng tối giản của phân số $\frac{a}{b}$ thì tồn tại số nguyên $k\in\mathbb{Z}$ sao cho $a = mk$, $b = nk$.'' -- \cite[Chap. III, \S1, p. 4]{Binh_Toan_6_tap_2}

\begin{baitoan}[\cite{Tuyen_Toan_6}, Ví dụ 49, p. 45]
	Cho $A = \{-5,0,9\}$. Viết tất cả các phân số $\frac{a}{b}$ với $a,b\in A$. Có bao nhiêu phân số thỏa mãn?
\end{baitoan}

\begin{proof}[Giải]
	Số $0$ không thể lấy làm mẫu của phân số. Lấy $-5$ làm mẫu: $\frac{-5}{-5},\frac{0}{-5},\frac{9}{-5}$. Lấy $9$ làm mẫu: $\frac{-5}{9},\frac{0}{9},\frac{9}{9}$. Có $6$ phân số thỏa mãn.
\end{proof}

\begin{baitoan}[Mở rộng \cite{Tuyen_Toan_6}, Ví dụ 49, p. 45]
	Cho $A = \{a_1,a_2,\ldots,a_n\}\subset\mathbb{Z}$. Viết tất cả các phân số $\frac{a}{b}$ với $a,b\in A$. Có bao nhiêu phân số thỏa mãn?
\end{baitoan}

\begin{proof}[Giải]
	Xét 2 trường hợp: (a) Nếu $0\notin A$, i.e., $a_i\ne0$, $\forall i = 1,\ldots,n$. Tất cả các phân số $\frac{a}{b}$ với $a,b\in A$: $\frac{a_i}{a_j}$, $\forall i,j = 1,\ldots,n$, có tổng cộng $n^2$ phân số thỏa mãn. (b) Nếu $0\in A$, i.e., tồn tại chỉ số $k\in\{1,\ldots,n\}$ sao cho $a_k = 0$, ngoài ra $a_i\ne 0$, $\forall i = 1,\ldots,n$, $i\ne k$ (vì $A$ là 1 tập hợp nên không có các phần tử trùng nhau). Tất cả các phân số $\frac{a}{b}$ với $a,b\in A$: $\frac{a_i}{a_j}$, $\forall i,j = 1,\ldots,n$, $j\ne k$ có tổng cộng $n(n - 1) = n^2 - n$ phân số thỏa mãn.
\end{proof}

\begin{nhanxet}
	``Mẫu của 1 phân số phải khác $0$ nhưng tử của phân số có thể bằng $0$, khi đó giá trị của phân số đúng bằng $0$, i.e., $\frac{0}{b} = 0$, $\forall b\in\mathbb{Z}$, $b\ne 0$. Tử \& mẫu của 1 phân số có thể bằng nhau, khi đó giá trị của phân số đúng bằng $1$, i.e., $\frac{a}{a} = 1$, $\forall a\in\mathbb{Z}$, $a\ne 0$.'' -- \cite[p. 46]{Tuyen_Toan_6}
\end{nhanxet}

\begin{baitoan}[\cite{Tuyen_Toan_6}, Ví dụ 50, p. 46]
	Viết tập hợp $B$ các phân số bằng phân số $\frac{7}{-15}$ với mẫu dương có 2 chữ số.
\end{baitoan}

\begin{proof}[Giải]
	$\frac{7}{-15} = \frac{-7}{15}$. Phân số này là 1 phân số tối giản với mẫu dương. Mọi phân số bằng nó đều có dạng $\frac{-7m}{15m}$ với $m\in\mathbb{Z}$, $m\ne0$. Mẫu số của các phân số cần phải tìm là 1 số có 2 chữ số nên chọn $m\in\mathbb{Z}$ sao cho $10\le15m\le 99$, suy ra\footnote{$m\in\mathbb{Z}\land(10\le15m\le 99)\Leftrightarrow\lfloor\frac{15}{10}\rfloor = 1\le m\le\lfloor\frac{99}{15}\rfloor = 6$.} $1\le m\le6$, i.e., $m\in\{1,2,3,4,5,6\}$. Vậy $B = \left\{\frac{-7}{15},\frac{-14}{30},\frac{-21}{45},\frac{-28}{60},\frac{-35}{75},\frac{-42}{90}\right\}$.
\end{proof}

\begin{baitoan}[Mở rộng \cite{Tuyen_Toan_6}, Ví dụ 50, p. 46]
	Cho trước $a,b\in\mathbb{Z}$, $b\ne0$, \& $n\in\mathbb{N}^\star$. Viết tập hợp $B$ các phân số bằng phân số $\frac{a}{b}$ với mẫu dương có $n$ chữ số.
\end{baitoan}

\begin{baitoan}[\cite{Tuyen_Toan_6}, Ví dụ 51, p. 46]
	Tìm phân số bằng phân số $\frac{32}{60}$, biết tổng của tử \& mẫu là $115$.
\end{baitoan}	

\begin{proof}[Giải]
	Có $\frac{32}{60} = \frac{8}{15} = \frac{8m}{15m}$, $\forall m\in\mathbb{Z}$, $m\ne0$. Tổng của tử \& mẫu là $115\Rightarrow8m + 15m = 115\Rightarrow23m = 115\Rightarrow m =\frac{115}{23} = 5$. Phân số cần tìm: $\frac{8\cdot5}{15\cdot5} = \frac{40}{75}$.
\end{proof}

\begin{nhanxet}
	``Nếu không rút gọn phân số $\frac{32}{60}$ thành phân số tối giản $\frac{8}{15}$ mà khẳng định các phân số bằng phân số $\frac{32}{60}$ có dạng $\frac{32m}{60m}$ thì sẽ mắc sai lầm là bỏ sót rất nhiều phân số bằng phân số $\frac{32}{60}$ do đó không thể tìm được đáp số của bài toán trên.'' -- \cite[p. 46]{Tuyen_Toan_6}
\end{nhanxet}

\begin{baitoan}[Mở rộng \cite{Tuyen_Toan_6}, Ví dụ 51, p. 46]
	Cho trước $a,b,n\in\mathbb{Z}$, $b\ne0$. Tìm phân số bằng phân số $\frac{a}{b}$, biết tổng của tử \& mẫu là $n$.
\end{baitoan}

\begin{baitoan}[\cite{Tuyen_Toan_6}, 236., p. 47]
	Trong các phân số sau, những phân số nào bằng nhau? $\frac{15}{60},\frac{-7}{5},\frac{6}{15},\frac{28}{-20},\frac{3}{12}$.
\end{baitoan}

\begin{baitoan}[\cite{Tuyen_Toan_6}, 237., p. 47]
	Cho $A = \frac{3n - 5}{n + 4}$. Tìm $n\in\mathbb{Z}$ để $A\in\mathbb{Z}$.
\end{baitoan}

\begin{baitoan}[\cite{Tuyen_Toan_6}, 238., p. 47]
	Tìm $n\in\mathbb{Z}$ để cho các phân số sau đồng thời có giá trị nguyên: $\frac{-12}{n},\frac{15}{n - 2},\frac{8}{n + 1}$.
\end{baitoan}

\begin{baitoan}[\cite{Tuyen_Toan_6}, 239., p. 47]
	Tìm $x\in\mathbb{Z}$ biết: (a) $\frac{x - 1}{9} = \frac{8}{3}$; (b) $\frac{-x}{4} = \frac{-9}{x}$; (c) $\frac{x}{4} = \frac{18}{x + 1}$.
\end{baitoan}

\begin{baitoan}[\cite{Tuyen_Toan_6}, 240., p. 47]
	Tìm $x,y\in\mathbb{Z}$ thỏa $\frac{x - 4}{y - 3} = \frac{4}{3}$ \& $x - y = 5$.
\end{baitoan}

\begin{baitoan}[\cite{Tuyen_Toan_6}, 241., p. 47]
	Viết dạng tổng quát các phân số bằng phân số $\frac{-12}{30}$.
\end{baitoan}

\begin{baitoan}[\cite{Tuyen_Toan_6}, 242., p. 47]
	Rút gọn phân số: (a) $\frac{990}{2610}$; (b) $\frac{374}{506}$; (c) $\frac{3600 - 75}{8400 - 175}$; (d) $\dfrac{9^{14}\cdot25^5\cdot8^7}{18^{12}\cdot625^3\cdot24^3}$.
\end{baitoan}

\begin{baitoan}[\cite{Tuyen_Toan_6}, 243., p. 47]
	Cho phân số $\frac{a}{b}$. Chứng minh: Nếu $\frac{a - x}{b - y} = \frac{a}{b}$ thì $\frac{x}{y} = \frac{a}{b}$.
\end{baitoan}

\begin{baitoan}[\cite{Tuyen_Toan_6}, 244., p. 47]
	Cho phân số $A = \dfrac{1 + 3 + 5 + \cdots + 19}{21 + 23 + 25 + \cdots + 39}$. (a) Rút gọn $A$; (b) Xóa 1 số hạng ở tử \& xóa 1 số hạng ở mẫu để được 1 phân số mới vẫn bằng $A$.
\end{baitoan}

\begin{baitoan}[\cite{Tuyen_Toan_6}, 245., p. 47]
	Rút gọn phân số $A = \frac{71\cdot52 + 53}{530\cdot71 - 180}$ mà không cần thực hiện các phép tính ở tử.
\end{baitoan}

\begin{baitoan}[\cite{Tuyen_Toan_6}, 246., p. 47]
	2 phân số sau có bằng nhau không? $\dfrac{\overline{abab}}{\overline{cdcd}},\dfrac{\overline{ababab}}{\overline{cdcdcd}}$.
\end{baitoan}

\begin{baitoan}[\cite{Tuyen_Toan_6}, 247., p. 47]
	Chứng minh: (a) $\dfrac{1\cdot3\cdot5\cdots39}{21\cdot22\cdot23\cdots40} = \dfrac{1}{2^{20}}$; (b) $\dfrac{1\cdot3\cdot5\cdots(2n - 1)}{(n + 1)(n + 2)(n + 3)\cdots2n} = \dfrac{1}{2^n}$ với $n\in\mathbb{N}^\star$.
\end{baitoan}

\begin{baitoan}[\cite{Tuyen_Toan_6}, 248., p. 47]
	Tìm phân số $\frac{a}{b}$ bằng phân số $\frac{60}{108}$ biết: (a) $\mbox{\rm ƯCLN}(a,b) = 15$; (b) $\operatorname{BCNN}(a,b) = 180$.
\end{baitoan}

\begin{baitoan}[\cite{Tuyen_Toan_6}, 249., p. 48]
	Tìm phân số bằng phân số $\frac{200}{520}$ sao cho: (a) Tổng của tử \& mẫu là $306$; (b) Hiệu của tử \& mẫu là $184$; (c) Tích của tử \& mẫu là $2340$.
\end{baitoan}

\begin{baitoan}[\cite{Tuyen_Toan_6}, 250., p. 48]
	Chứng minh: $\forall n\in\mathbb{Z}$, các phân số sau là các phân số tối giản: (a) $\frac{3n - 2}{4n - 3}$; (b) $\frac{4n + 1}{6n + 1}$.
\end{baitoan}

\begin{baitoan}[\cite{Tuyen_Toan_6}, 251., p. 48]
	Cho $\frac{a}{b}$ là 1 phân số chưa tối giản. Chứng minh các phân số sau chưa tối giản: (a) $\frac{a}{a - b}$; (b) $\frac{2a}{a - 2b}$.
\end{baitoan}

\begin{baitoan}[\cite{Tuyen_Toan_6}, 252., p. 48]
	1 mẫu Bắc Bộ bằng $\rm3600m^2$. Hỏi 1 mẫu Bắc Bộ bằng mấy phần của 1 hecta?
\end{baitoan}

\begin{baitoan}[\cite{Binh_Toan_6_tap_2}, Ví dụ 1, p. 4]
	Tìm $n\in\mathbb{N}$ để phân số $A = \frac{n + 10}{2n - 8}\in\mathbb{Z}$ (i.e., có giá trị là 1 số nguyên).
\end{baitoan}

\begin{proof}[Giải]
	Để phân số $A$ có giá trị là 1 số nguyên, tử phải chi hết cho mẫu: $n + 10\divby2n - 8\Rightarrow n + 10\divby n - 4\Rightarrow n - 4 + 14\divby n - 4\Rightarrow14\divby n - 4\Rightarrow n - 4\in\mbox{Ư}(14)\cap\mathbb{Z} = \{\pm1,\pm2,\pm7,\pm14\}$. Vì $n - 4\ge-4$ (vì $n\in\mathbb{N}$, $n\ge 0$) nên $n - 4\in\{\pm1,\pm2,7,14\}$. Nếu $n - 4 = 1$, $n = 5$, $A = \frac{15}{2}$ (loại). Nếu $n - 4 = -1$, $n = 3$, $A = \frac{13}{-2}$ (loại). Nếu $n - 4 = 2$, $n = 6$, $A = \frac{16}{4} = 4$. Nếu $n - 4 = -2$, $n = 2$, $A = \frac{12}{-4} = -3$. Nếu $n - 4 = 7$, $n = 11$, $A = \frac{21}{14} = \frac{3}{2}$ (loại). Nếu $n - 4 = 14$, $n = 18$, $A = \frac{28}{28} = 1$. Vậy $n\in\{2,6,18\}$.
\end{proof}

\begin{baitoan}[Mở rộng \cite{Binh_Toan_6_tap_2}, Ví dụ 1, p. 4]
	Cho $a,b,c,d\in\mathbb{Z}$, $c^2 + d^2\ne0$. Tìm $n\in\mathbb{N}$ để phân số $A = \frac{an + b}{cn + d}\in\mathbb{Z}$.
\end{baitoan}

\begin{baitoan}[\cite{Binh_Toan_6_tap_2}, Ví dụ 2, p. 5]
	Tìm $n\in\mathbb{N}$ để phân số $A = \frac{21n + 3}{6n + 4}$ rút gọn được.
\end{baitoan}

\begin{baitoan}[Mở rộng \cite{Binh_Toan_6_tap_2}, Ví dụ 2, p. 5]
	Cho $a,b,c,d\in\mathbb{Z}$, $c^2 + d^2\ne0$. Tìm $n\in\mathbb{N}$ để phân số $A = \frac{an + b}{cn + d}$ rút gọn được.
\end{baitoan}

\begin{baitoan}[\cite{Binh_Toan_6_tap_2}, Ví dụ 3, p. 5]
	Tìm $a,b,c,d\in\mathbb{N}$ nhỏ nhất sao cho $\frac{a}{b} = \frac{3}{5}$, $\frac{b}{c} = \frac{12}{21}$, $\frac{c}{d} = \frac{6}{11}$.
\end{baitoan}

\begin{baitoan}[\cite{Binh_Toan_6_tap_2}, Ví dụ 4, p. 5]
	Tìm số tự nhiên lớn nhất có 3 chữ số sao cho số đó bằng mỗi tổng $a + b,c + d,e + f$ \& $\frac{a}{b} = \frac{35}{49},\frac{c}{d} = \frac{130}{143},\frac{e}{f} = \frac{7}{13}$.
\end{baitoan}

\begin{baitoan}[\cite{Binh_Toan_6_tap_2}, 1., p. 6]
	Rút gọn phân số: (a) $\frac{199\ldots9}{99\ldots95}$ ($10$ chữ số $9$ ở tử, $10$ chữ số $9$ ở mẫu); (b) $\frac{121212}{424242}$; (c) $\frac{187187187}{221221221}$; (d) $\frac{3\cdot7\cdot13\cdot37\cdot39 - 10101}{505050 + 70707}$.
\end{baitoan}

\begin{baitoan}[\cite{Binh_Toan_6_tap_2}, 2., p. 6]
	Chứng minh các phân số sau có giá trị lfa số tự nhiên: (a) $\frac{10^{2002} + 2}{3}$; (b) $\frac{10^{2003} + 8}{9}$.
\end{baitoan}

\begin{baitoan}[\cite{Binh_Toan_6_tap_2}, 3., p. 6]
	Chứng mih các phân số sau bằng nhau: (a) $\frac{1717}{2929}$ \& $\frac{171717}{292929}$; (b) $\frac{3210 - 34}{4170 - 41}$ \& $\frac{6420 - 68}{8340 - 82}$; (c) $\frac{2106}{7320}$, $\frac{4212}{14640}$, \& $\frac{6318}{21960}$.
\end{baitoan}

\begin{baitoan}[\cite{Binh_Toan_6_tap_2}, 4., p. 6]
	Tìm $x,y\in\mathbb{Z}$ thỏa: (a) $\frac{x}{3} = \frac{y}{5}$; (b) $\frac{x}{28} = \frac{y}{35}$.
\end{baitoan}

\begin{baitoan}[\cite{Binh_Toan_6_tap_2}, 5., p. 6]
	Tìm các phân số $\frac{a}{b}$, $a\in\mathbb{N}$, $b\in\mathbb{N}^\star$, có giá trị bằng: (a) $\frac{36}{45}$ biết $\mbox{\rm BCNN}(a,b) = 300$; (b) $\frac{21}{35}$ biết $\mbox{\rm ƯCLN}(a,b) = 30$; (c) $\frac{15}{35}$ biết $\mbox{\rm ƯCLN}(a,b)\cdot\mbox{\rm BCNN}(a,b) = 3549$.
\end{baitoan}

\begin{baitoan}[\cite{Binh_Toan_6_tap_2}, 6., p. 7]
	Chứng minh các phân số sau tối giản với mọi $n\in\mathbb{N}$. (a) $\frac{n + 1}{2n + 3}$; (b) $\frac{2n + 3}{4n + 8}$; (c) $\frac{3n + 2}{5n + 3}$.
\end{baitoan}

\begin{baitoan}[\cite{Binh_Toan_6_tap_2}, 7., p. 7]
	Cho phân số $A = \frac{63}{3n + 1}$, $n\in\mathbb{N}$. (a) Với giá trị nào của $n$ thì $A$ rút gọn được? (b) Với giá trị nào của $n$ thì $A\in\mathbb{N}$?
\end{baitoan}

\begin{baitoan}[\cite{Binh_Toan_6_tap_2}, 8., p. 7]
	Tìm các số tự nhiên $n$ để các phân số sau là phân số tối giản: (a) $\frac{2n + 3}{4n + 1}$; (b) $\frac{3n + 2}{7n + 1}$; (c) $\frac{2n + 7}{5n + 2}$.
\end{baitoan}

\begin{baitoan}[\cite{Binh_Toan_6_tap_2}, 9., p. 7]
	Có bao nhiêu số nguyên dương $n$ không vượt quá $1000$ để phân số $\frac{n + 12}{n^2 + 9n - 13}$ là phân số tối giản?
\end{baitoan}

\begin{baitoan}[\cite{Binh_Toan_6_tap_2}, 10., p. 7]
	Tìm $n\in\mathbb{N}$ để phân số $\frac{n + 3}{2n - 2}\in\mathbb{Z}$.
\end{baitoan}

\begin{baitoan}[\cite{Binh_Toan_6_tap_2}, 11., p. 7]
	Tìm các số nguyên $n$ sao cho các phân số sau có giá trị là số nguyên: (a) $\frac{12}{3n - 1}$; (b) $\frac{2n + 3}{7}$.
\end{baitoan}

\begin{baitoan}[\cite{Binh_Toan_6_tap_2}, 12., p. 7]
	Tìm $n\in\mathbb{N}$ để phân số $A = \frac{8n + 193}{4n + 3}$: (a) Có giá trị là số tự nhiên; (b) Là phân số tối giản; (c) Với giá trị nào của $n$ trong khoảng từ $150$ đến $170$ thì phân số $A$ rút gọn được?
\end{baitoan}

\begin{baitoan}[\cite{Binh_Toan_6_tap_2}, 13., p. 7]
	Tìm các phân số tối giản nhỏ hơn $1$ có tử \& mẫu đều dương, biết tích của tử \& mẫu của phân số bằng $120$.
\end{baitoan}

\begin{baitoan}[\cite{Binh_Toan_6_tap_2}, 14., p. 7]
	Tìm $n\in\mathbb{N}$ nhỏ nhất để các phân số sau đều là phân số tối giản: $\frac{5}{n + 8},\frac{6}{n + 9},\frac{7}{n + 10},\ldots,\frac{17}{n + 20}$.
\end{baitoan}

\begin{baitoan}[\cite{Binh_Toan_6_tap_2}, 15., p. 7]
	Cho 3 phân số $\frac{15}{42},\frac{49}{56},\frac{36}{51}$. Biến đổi 3 phân số trên thành các phân số bằng chúng sao cho mẫu của phân số thứ nhất bằng tử của phân số thứ 2, mẫu của phân số thứ 2 bằng tử của phân số thứ 3.
\end{baitoan}

\begin{baitoan}[\cite{Binh_Toan_6_tap_2}, 16., p. 7]
	Cho 3 phân số $\frac{5}{8},\frac{11}{20},\frac{4}{15}$. Tìm 3 phân số (có tử \& mẫu dương) theo thứ tự bằng 3 phân số trên sao cho hiệu của mẫu \& tử của mỗi phân số này đều bằng nhau \& hiệu đó có giá trị nhỏ nhất.
\end{baitoan}

\begin{baitoan}[\cite{Binh_Toan_6_tap_2}, 17., p. 8]
	Tìm các phân số lớn hơn $\frac{1}{5}$ \& khác số tự nhiên biết nếu lấy mẫu nhân với 1 số, lấy tử cộng với số đó thì giá trị của phân số không đổi.
\end{baitoan}

\begin{baitoan}[\cite{Binh_Toan_6_tap_2}, 18., p. 8]
	Cho phân số $A = \frac{23 + 22 + 21 + \cdots + 13}{11 + 10 + 9 + \cdots + 1}$. Nêu cách xóa 1 số hạng ở tử \& 1 số hạng ở mẫu của $A$ để được 1 phân số mới vẫn bằng phân số $A$.
\end{baitoan}

\begin{baitoan}[\cite{Binh_Toan_6_tap_2}, 19., p. 8, Bộ sử Hume]
	Người Anh có thói quen xếp bộ sử nước Anh của Hume (David Hume, nhà sử học Scotland) gồm 9 tập ở tủ sách đặc biệt gồm 2 ngăn: ngăn trên xếp 5 cuốn, ngăn dưới xếp 4 cuốn, ở gáy các cuốn sách đó ghi các số $1,2,3,\ldots,9$. Nếu chủ nhân xếp $\frac{13458}{6729}$ (phân số này có giá trị bằng $2$) thì chứng tỏ chủ nhân đã đọc 2 tập (riêng trường hợp mới đọc 1 tập thì xếp $\frac{12345}{6789}$). Nêu cách xếp 9 cuốn sách đó để chứng tỏ chủ nhân của bộ sách đã đọc $3,4,5,6,7,8,9$ tập.
\end{baitoan}

%------------------------------------------------------------------------------%

\section{Rút Gọn Phân Số}

%------------------------------------------------------------------------------%

\section{Quy Đồng Mẫu Số Nhiều Phân Số}

%------------------------------------------------------------------------------%

\section{So Sánh Các Phân Số}
Trong 2 số nguyên $a,b\in\mathbb{Z}$ khác nhau ($a\ne b$), luôn có 1 số nhỏ hơn số kia, i.e., $a < b$ hoặc $a > b$. Cũng như số nguyên, trong 2 phân số $\frac{a}{b},\frac{c}{d}$ khác nhau ($\frac{a}{b}\ne\frac{c}{d}$) luôn có 1 phân số nhỏ hơn phân số kia, i.e., $\frac{a}{b} < \frac{c}{d}$ hoặc $\frac{a}{b} > \frac{c}{d}$. Nếu phân số $\frac{a}{b}$ nhỏ hơn phân số $\frac{c}{d}$ thì ta viết $\frac{a}{b} < \frac{c}{d}$ hoặc $\frac{c}{d} > \frac{a}{b}$. Phân số lớn hơn $0$ gọi là \textit{phân số dương}. Phân số nhỏ hơn $0$ gọi là \textit{phân số âm}. Tính chất bắc cầu: Nếu $\frac{a}{b} < \frac{c}{d}$ \& $\frac{c}{d} < \frac{e}{f}$ thì $\frac{a}{b} < \frac{e}{f}$.
\begin{align*}
	\left(\frac{a}{b} < \frac{c}{d}\right)\land\left(\frac{c}{d} < \frac{e}{f}\right)\Rightarrow\frac{a}{b} < \frac{e}{f},\ \forall a,b,c,d,e,f\in\mathbb{Z},\,bd\ne0.
\end{align*}
``\textbf{1.} \textit{Quy tắc quy đồng mẫu nhiều phân số với mẫu dương}: \textit{Bước 1.} Tìm BCNN của các mẫu để làm mẫu chung. \textit{Bước 2.} Tìm thừa số phụ của mỗi mẫu. \textit{Bước 3.} Nhân tử \& mẫu của mỗi phân số với thừa số phụ tương ứng. \textbf{2.} \textit{So sánh 2 phân số}: Muốn so sánh 2 phân số không cùng mẫu ta viết chúng dưới dạng 2 phân số có cùng mẫu dương rồi so sánh các tử với nhau, phân số nào có tử lớn hơn thì phân số đó lớn hơn. \textbf{3.} \textit{Hỗn số dương}: 1 phân số lớn hơn 1 có thể viết dưới dạng 1 hỗn số. Đó là 1 số gồm phần nguyên kèm theo 1 phân số nhỏ hơn 1. \textbf{4.} Trong 2 phân số có tử \& mẫu đều dương, nếu 2 tử số bằng nhau, phân số nào có mẫu nhỏ hơn thì phân số đó sẽ lớn hơn \& ngược lại. \textbf{5.} Phân số có tử \& mẫu là 2 số nguyên cùng dấu thì lớn hơn 0 \& gọi là \textit{phân số dương}. Phân số có tử \& mẫu là 2 số nguyên khác dấu thì nhỏ hơn 0 \& gọi là \textit{phân số âm}.'' -- \cite[Chap. III, \S2, p. 48]{Tuyen_Toan_6}

``Để so sánh 2 phân số có tử \& mẫu đều dương, ngoài cách quy đồng tử hoặc quy đồng mẫu, người ta thường dùng 1 phân số trung gian \& sử dụng tính chất bắc cầu của bất đẳng thức.

Thường sử dụng các tính chất sau: (a) Trong 2 phân số cùng tử, phân số nào có mẫu nhỏ hơn thì phân số đó lớn hơn. (b) Trong 2 phân số nhỏ hơn 1, phân số nào có phần bù đến 1 nhỏ hơn thì phân số đó lớn hơn: $1 - \frac{a}{b} < 1 - \frac{c}{d}\Rightarrow\frac{a}{b} > \frac{c}{d}$. (c) Nếu $0 < a < 1$ \& $m < n$ thì $a^m > a^n$.'' -- \cite[Chap. 1, \S2, p. 8]{Binh_Toan_6_tap_2}

\begin{baitoan}[Công thức hỗn số dương]
	Chứng minh:
	\begin{align*}
		\frac{ac + b}{c} = a + \frac{b}{c} = a\frac{b}{c},\ \forall a,b,c\in\mathbb{Z},\,c\ne0.\ \ \frac{a}{b} = \frac{\lfloor\frac{a}{b}\rfloor b + \left\{\frac{a}{b}\right\}}{b} = \left\lfloor\frac{a}{b}\right\rfloor + \frac{\left\{\frac{a}{b}\right\}}{b} = \left\lfloor\frac{a}{b}\right\rfloor\frac{\left\{\frac{a}{b}\right\}}{b}.
	\end{align*}
\end{baitoan}

\begin{baitoan}[\cite{Tuyen_Toan_6}, Ví dụ 52, p. 48]
	So sánh 2 phân số $\frac{-101}{-100}$ \& $\frac{200}{201}$.
\end{baitoan}

\begin{proof}[Giải]
	$\frac{-101}{-100} = \frac{101}{100} > \frac{100}{100} = 1 = \frac{201}{201} > \frac{200}{201}$. Vậy $\frac{-101}{-100} > \frac{200}{201}$.
\end{proof}

\begin{baitoan}[Mở rộng \cite{Tuyen_Toan_6}, Ví dụ 52, p. 48]
	Cho $a,b,c,d\in\mathbb{N}$, $a > b > 0$, $d > c > 0$. So sánh: $\frac{\pm a}{\pm b}$ \& $\frac{\pm c}{\pm d}$.
\end{baitoan}

\begin{baitoan}[\cite{Tuyen_Toan_6}, Ví dụ 53, p. 48]
	Sắp xếp các phân số sau theo thứ tự tăng dần: $\frac{5}{8},\frac{9}{16},\frac{2}{-3},\frac{-7}{12}$.
\end{baitoan}

\begin{baitoan}[\cite{Binh_Toan_6_tap_2}, Ví dụ 5, p. 8]
	So sánh $A = \frac{10^{15} + 1}{10^{16} + 1}$ \& $B = \frac{10^{16} + 1}{10^{17} + 1}$.
\end{baitoan}

\begin{baitoan}[\cite{Binh_Toan_6_tap_2}, Ví dụ 6, p. 9]
	1 phân số có tử \& mẫu đều là số nguyên dương. Nếu cộng tử \& mẫu của phân số đó với cùng $n\in\mathbb{N}^\star$ thì phân số thay đổi thế nào?
\end{baitoan}

\begin{baitoan}[\cite{Binh_Toan_6_tap_2}, Ví dụ 7, p. 9]
	So sánh $\left(\frac{1}{32}\right)^7$ \& $\left(\frac{1}{16}\right)^9$.
\end{baitoan}

\begin{baitoan}[\cite{Binh_Toan_6_tap_2}, Ví dụ 8, p. 9]
	Chứng minh $95^8$ là 1 số có $16$ chữ số khi viết kết quả của nó trong hệ thập phân.
\end{baitoan}

\begin{baitoan}[\cite{Binh_Toan_6_tap_2}, Ví dụ 9, p. 10]
	Cho $a,b\in\mathbb{N}^\star$ thỏa $\frac{5}{7} < \frac{a}{b} < \frac{9}{11}$. Tìm $a + b$ khi $b$ nhỏ nhất.
\end{baitoan}

\begin{baitoan}[\cite{Binh_Toan_6_tap_2}, 20., p. 10]
	So sánh $a,b\in\mathbb{N}$ biết $\frac{1 + 2 + 3 + \cdots + a}{a} < \frac{1 + 2 + 3 + \cdots + b}{b}$.
\end{baitoan}

\begin{baitoan}[\cite{Binh_Toan_6_tap_2}, 21., p. 10]
	So sánh: (a) $\frac{18}{91}$ \& $\frac{23}{114}$; (b) $\frac{21}{52}$ \& $\frac{213}{523}$; (c) $\frac{1313}{9191}$ \& $\frac{1111}{7373}$.
\end{baitoan}

\begin{baitoan}[\cite{Binh_Toan_6_tap_2}, 22., p. 10]
	So sánh các phân số sau, với $n\in\mathbb{N}$: (a) $\frac{n}{n + 1}$ \& $\frac{n + 2}{n + 3}$; (b) $\frac{n + 1}{n + 4}$ \& $\frac{n}{n + 5}$; (c) $\frac{n}{2n + 1}$ \& $\frac{3n + 1}{6n + 3}$.
\end{baitoan}

\begin{baitoan}[\cite{Binh_Toan_6_tap_2}, 23., p. 11]
	So sánh $A$ \& $B$: (a) $A = \frac{20}{39} + \frac{22}{27} + \frac{18}{43}$, $B = \frac{14}{39} + \frac{22}{29} + \frac{18}{41}$; (b) $A = \frac{3}{8^3} + \frac{7}{8^4}$, $B = \frac{7}{8^3} + \frac{3}{8^4}$; (c) $A = \frac{10^7 + 5}{10^7 - 8}$, $B = \frac{10^8 + 6}{10^8 - 7}$; (d) $A = \frac{10^{1992} + 1}{10^{1991} + 1}$, $B = \frac{10^{1993} + 1}{10^{1992} + 1}$.
\end{baitoan}

\begin{baitoan}[\cite{Binh_Toan_6_tap_2}, 24., p. 11]
	Tìm $x\in\mathbb{N}$ sao cho $\frac{4}{11} < \frac{x}{20} < \frac{5}{11}$.
\end{baitoan}

\begin{baitoan}[\cite{Binh_Toan_6_tap_2}, 25., p. 11]
	Tìm 2 phân số có các mẫu bằng $9$, các tử là 2 số tự nhiên liên tiếp sao cho phân số $\frac{4}{7}$ nằm giữa 2 phân số đó.
\end{baitoan}

\begin{baitoan}[\cite{Binh_Toan_6_tap_2}, 26., p. 11]
	Tìm 2 phân số có các tử bằng $1$, các mẫu là 2 số tự nhiên liên tiếp sao cho phân số $\frac{13}{84}$ nằm giữa 2 phân số đó.
\end{baitoan}

\begin{baitoan}[\cite{Binh_Toan_6_tap_2}, 27., p. 11]
	Tìm 2 phân số có mẫu bằng $21$ biết nó lớn hơn $\frac{5}{7}$ \& nhỏ hơn $\frac{5}{6}$.
\end{baitoan}

\begin{baitoan}[\cite{Binh_Toan_6_tap_2}, 28., p. 11]
	Tìm phân số $\frac{a}{b}$ sao cho $a$ là số tự nhiên nhỏ nhất thỏa mãn $\frac{4}{15} < \frac{a}{b} < \frac{1}{3}$.
\end{baitoan}

\begin{baitoan}[\cite{Binh_Toan_6_tap_2}, 29., p. 11]
	Tìm phân số $\frac{a}{b}$ lớn nhất nhỏ hơn $1$ với $a,b$ là các số nguyên dương có 1 chữ số.
\end{baitoan}

\begin{baitoan}[\cite{Binh_Toan_6_tap_2}, 30., p. 11]
	So sánh 2 phân số $\left(\frac{1}{243}\right)^9$ \& $\left(\frac{1}{83}\right)^{13}$.
\end{baitoan}

%------------------------------------------------------------------------------%

\section{Hỗn Số Dương}

%------------------------------------------------------------------------------%

\section{1 Số Phương Pháp Đặc Biệt để So Sánh 2 Phân Số}

%------------------------------------------------------------------------------%

\section{$\pm$ Phân Số}

%------------------------------------------------------------------------------%

\section{$\cdot,:$ Phân Số}

%------------------------------------------------------------------------------%

\section{Tổng Các Phân Số Viết Theo Quy Luật}

%------------------------------------------------------------------------------%

\section{Số Thập Phân. Làm Tròn Số Thập Phân. Các Phép Tính với Số Thập Phân}

%------------------------------------------------------------------------------%

\section{Tìm Giá Trị Phân Số của 1 Số Cho Trước. Tìm 1 Số Biết Giá Trị 1 Phân Số của Nó}

%------------------------------------------------------------------------------%

\section{Tìm Tỷ Số \& Tỷ Số \% của 2 Đại Lượng}

%------------------------------------------------------------------------------%

\section{Toán về Công Việc Làm Đồng Thời}

%------------------------------------------------------------------------------%

\section{Miscellaneous}

%------------------------------------------------------------------------------%

\printbibliography[heading=bibintoc]
	
\end{document}