\documentclass{article}
\usepackage[backend=biber,natbib=true,style=authoryear]{biblatex}
\addbibresource{/home/nqbh/reference/bib.bib}
\usepackage[utf8]{vietnam}
\usepackage{tocloft}
\renewcommand{\cftsecleader}{\cftdotfill{\cftdotsep}}
\usepackage[colorlinks=true,linkcolor=blue,urlcolor=red,citecolor=magenta]{hyperref}
\usepackage{amsmath,amssymb,amsthm,mathtools,float,graphicx,algpseudocode,algorithm,tcolorbox}
\usepackage[inline]{enumitem}
\allowdisplaybreaks
\numberwithin{equation}{section}
\newtheorem{assumption}{Assumption}[section]
\newtheorem{baitoan}{Bài toán}
\newtheorem{cauhoi}{Câu hỏi}[section]
\newtheorem{conjecture}{Conjecture}[section]
\newtheorem{corollary}{Corollary}[section]
\newtheorem{definition}{Definition}[section]
\newtheorem{dinhly}{Định lý}[section]
\newtheorem{dinhnghia}{Định nghĩa}[section]
\newtheorem{example}{Example}[section]
\newtheorem{hequa}{Hệ quả}[section]
\newtheorem{lemma}{Lemma}[section]
\newtheorem{luuy}{Lưu ý}[section]
\newtheorem{notation}{Notation}[section]
\newtheorem{principle}{Principle}[section]
\newtheorem{problem}{Problem}[section]
\newtheorem{proposition}{Proposition}[section]
\newtheorem{question}{Question}[section]
\newtheorem{remark}{Remark}[section]
\newtheorem{theorem}{Theorem}[section]
\newtheorem{vidu}{Ví dụ}[section]
\usepackage[left=0.5in,right=0.5in,top=1.5cm,bottom=1.5cm]{geometry}
\usepackage{fancyhdr}
\pagestyle{fancy}
\fancyhf{}
\lhead{\small Sect.~\thesection}
\rhead{\small\nouppercase{\leftmark}}
\renewcommand{\subsectionmark}[1]{\markboth{#1}{}}
\cfoot{\thepage}
\def\labelitemii{$\circ$}
\DeclareRobustCommand{\divby}{%
	\mathrel{\vbox{\baselineskip.65ex\lineskiplimit0pt\hbox{.}\hbox{.}\hbox{.}}}%
}

\title{Integer -- Số Nguyên $\mathbb{Z}$}
\author{Nguyễn Quản Bá Hồng\footnote{Independent Researcher, Ben Tre City, Vietnam\\e-mail: \texttt{nguyenquanbahong@gmail.com}; website: \url{https://nqbh.github.io}.}}
\date{\today}

\begin{document}
\maketitle
\begin{abstract}
	\textsc{[en]} This text is a collection of problems, from easy to advanced, about integer. This text is also a supplementary material for my lecture note on Elementary Mathematics grade 6, which is stored \& downloadable at the following link: \href{https://github.com/NQBH/hobby/blob/master/elementary_mathematics/grade_6/NQBH_elementary_mathematics_grade_6.pdf}{GitHub\texttt{/}NQBH\texttt{/}hobby\texttt{/}elementary mathematics\texttt{/}grade 6\texttt{/}lecture}\footnote{\textsc{url}: \url{https://github.com/NQBH/hobby/blob/master/elementary_mathematics/grade_6/NQBH_elementary_mathematics_grade_6.pdf}.}. The latest version of this text has been stored \& downloadable at the following link: \href{https://github.com/NQBH/hobby/blob/master/elementary_mathematics/grade_6/integer/NQBH_integer.pdf}{GitHub\texttt{/}NQBH\texttt{/}hobby\texttt{/}elementary mathematics\texttt{/}grade 6\texttt{/}integer $\mathbb{Z}$}\footnote{\textsc{url}: \url{https://github.com/NQBH/hobby/blob/master/elementary_mathematics/grade_6/integer/NQBH_integer.pdf}.}.
	\vspace{2mm}
	
	\textsc{[vi]} Tài liệu này là 1 bộ sưu tập các bài tập chọn lọc từ cơ bản đến nâng cao về số nguyên. Tài liệu này là phần bài tập bổ sung cho tài liệu chính -- bài giảng \href{https://github.com/NQBH/hobby/blob/master/elementary_mathematics/grade_6/NQBH_elementary_mathematics_grade_6.pdf}{GitHub\texttt{/}NQBH\texttt{/}hobby\texttt{/}elementary mathematics\texttt{/}grade 6\texttt{/}lecture} của tác giả viết cho Toán Sơ Cấp lớp 6. Phiên bản mới nhất của tài liệu này được lưu trữ \& có thể tải xuống ở link sau: \href{https://github.com/NQBH/hobby/blob/master/elementary_mathematics/grade_6/integer/NQBH_integer.pdf}{GitHub\texttt{/}NQBH\texttt{/}hobby\texttt{/}elementary mathematics\texttt{/}grade 6\texttt{/}integer $\mathbb{Z}$}.
\end{abstract}
\tableofcontents

%------------------------------------------------------------------------------%

\section{Số Nguyên}
``Tập hợp $\mathbb{Z}$ các số nguyên gồm các số tự nhiên \& các số $-1,-2,-3,\ldots$. $\mathbb{Z} = \{\ldots,-3,-2,-1,0,1,2,3,\ldots\}$. Ta xác định trên $\mathbb{Z}$ 1 thứ tự như sau: $a < b$ khi \& chỉ khi điểm $a$ ở bên trái điểm $b$ trên trục số ($a,b\in\mathbb{Z}$). Ta xác định trên $\mathbb{Z}$ 2 phép toán: phép cộng \& phép nhân. Phép cộng có 4 tính chất: giao hoán, kết hợp, cộng với số $0$, cộng với số đối. Phép nhân có 3 tính chất: giao hoán, kết hợp, nhân với số $1$. Giữa phép nhân \& phép cộng có quan hệ: phép nhân phân phối đối với phép cộng. Giữa thứ tự \& phép toán có quan hệ: $a < b\Rightarrow a + c < b + c$, $a < b\Rightarrow ac < bc$ với $c > 0$, $ac > bc$ với $c < 0$. Trừ đi 1 số là cộng với số đối của số trừ. Phép trừ 2 số nguyên bao giờ cũng thực hiện được\footnote{Phép trừ 2 số tự nhiên sẽ không thực hiện được (i.e., kết quả không phải là 1 số tự nhiên, hay không còn nằm trong $\mathbb{N}$) nếu số bị trừ nhỏ hơn số trừ.}. Phép chia chỉ thực hiện được trong phạm vi số nguyên khi số bị chia chia hết cho số chia. Trong trường hợp $a\divby b$, ta nói: $a$ là \textit{bội} của $b$ \& $b$ là \textit{ước} của $a$. \textit{Ước chung} (hoặc \textit{bội chung}) của 2 hay nhiều số là ước (hoặc bội) của tất cả các số đó.'' -- \cite[Chap. II, p. 41]{Binh_Toan_6_tap_1}

%------------------------------------------------------------------------------%

\subsection{Thứ Tự Trên $\mathbb{Z}$}

\begin{baitoan}[\cite{Binh_Toan_6_tap_1}, Ví dụ 48, p. 41]
	Cho $a\in\mathbb{Z}$. Gọi khoảng cách từ điểm $a$ đến điểm gốc trên trục số là \emph{giá trị tuyệt đối} của số $a$ \& ký hiệu là $|a|$. Điền vào chỗ trống các dấu $\ge,\le,>,<,=$ để các khẳng định sau là đúng:
	\begin{enumerate*}
		\item[(a)] $|a|\ldots a$, $\forall a\in\mathbb{Z}$.
		\item[(a)] $|a|\ldots 0$, $\forall a\in\mathbb{Z}$.
		\item[(c)] Nếu $a > 0$ thì $a\ldots|a|$.
		\item[(d)] Nếu $a = 0$ thì $a\ldots|a|$.
		\item[(e)] Nếu $a < 0$ thì $a\ldots|a|$.
	\end{enumerate*}
\end{baitoan}

\begin{baitoan}[\cite{Binh_Toan_6_tap_1}, \textbf{247.}, p. 42]
	Điền vào chỗ trống $\ldots$ các từ ``nhỏ hơn'' hoặc ``lớn hơn'' cho đúng:
	\begin{enumerate*}
		\item[(a)] Mọi số nguyên dương đều $\ldots$ số $0$.
		\item[(b)] Mọi số nguyên âm đều $\ldots$ số $0$.
		\item[(c)] Mỗi số nguyên dương đều $\ldots$ mọi số nguyên âm.
		\item[(d)] Trong 2 số nguyên dương, số nào có giá trị tuyệt đối lớn hơn thì số ấy $\ldots$.
		\item[(e)] Trong 2 số nguyên âm, số nào có giá trị tuyệt đối lớn hơn thì số ấy $\ldots$.
	\end{enumerate*}
\end{baitoan}

\begin{baitoan}[\cite{Binh_Toan_6_tap_1}, \textbf{248.}, p. 42]
	Tìm:
	\begin{enumerate*}
		\item[(a)] Số nguyên dương lớn nhất có 2 chữ số.
		\item[(a)] Số nguyên âm lớn nhất có 2 chữ số.
	\end{enumerate*}
\end{baitoan}

\begin{baitoan}[\cite{Binh_Toan_6_tap_1}, \textbf{249.}, p. 42]
	Tính $|b| - |a|$ biết: 
	\begin{enumerate*}
		\item[(a)] $a = -3$, $b = 7$;
		\item[(b)] $a = 5$, $b = -6$;
		\item[(c)] $a = 5$, $b = -5$;
	\end{enumerate*}
\end{baitoan}

\begin{baitoan}[\cite{Binh_Toan_6_tap_1}, \textbf{250.}, p. 42]
	Các khẳng định sau có đúng $\forall a,b\in\mathbb{Z}$ hay không? Cho ví dụ.
	\begin{enumerate*}
		\item[(a)] $|a| = |b|\Rightarrow a = b$.
		\item[(b)] $a > b\Rightarrow|a| > |b|$.
	\end{enumerate*}
\end{baitoan}

%------------------------------------------------------------------------------%

\subsection{$\pm$ Trêm $\mathbb{Z}$}

\begin{baitoan}[\cite{Binh_Toan_6_tap_1}, Ví dụ 49, p. 42]
	Tìm $x\in\mathbb{Z}$, biết $10 = 10 + 9 + 8 + \cdots + x$, trong đó vế phải là tổng các số nguyên liên tiếp viết theo thứ tự giảm dần.
\end{baitoan}

\begin{baitoan}[\cite{Binh_Toan_6_tap_1}, \textbf{251.}, p. 42]
	Tìm tổng của số nguyên âm nhỏ nhất có 1 chữ số \& số nguyên dương lớn nhất có 1 chữ số.
\end{baitoan}

\begin{baitoan}[\cite{Binh_Toan_6_tap_1}, \textbf{252.}, p. 42]
	Điền vào chỗ trống cho đúng:
	\begin{enumerate*}
		\item[(a)] Số đối của 1 số nguyên âm là 1 số $\ldots$.
		\item[(b)] 2 số nguyên đối nhau thì có giá trị tuyệt đối $\ldots$.
		\item[(c)] 2 số nguyên có giá trị tuyệt đối bằng nhau thì $\ldots$.
		\item[(d)] Số $\ldots$ thì nhỏ hơn số đối của nó.
		\item[(e)] Nếu $a\ldots$ thì $-a > 0$.
		\item[(f)] Nếu $a < 0$ thì $|a| = \ldots$.
		\item[(g)] Nếu $a < 0$ thì $a + |a| = \ldots$.
	\end{enumerate*}
\end{baitoan}

\begin{baitoan}[\cite{Binh_Toan_6_tap_1}, \textbf{253.}, p. 43]
	Tìm $x\in\mathbb{Z}$ biết:
	\begin{enumerate*}
		\item[(a)] $x + 13 = 5$.
		\item[(b)] $x - 1 = -9$.
		\item[(c)] $25 - |x| = 10$.
		\item[(d)] $|x - 2| + 7 = 12$.
		\item[(e)] $x + 4$ là số nguyên dương nhỏ nhất.
		\item[(f)] $10 - x$ là số nguyên âm lớn nhất.
	\end{enumerate*}
\end{baitoan}

\begin{baitoan}[\cite{Binh_Toan_6_tap_1}, \textbf{254.}, p. 43]
	\begin{itemize}
		\item[(a)] Cho bảng vuông $3\times 3$ ô:
		\begin{table}[H]
			\centering
			\begin{tabular}{|c|c|c|}
				\hline
				$-8$ & $7$ &  \\
				\hline
				$\ \ 5$ &  & $\ \ 9$ \\
				\hline
				& $5$ & $-6$ \\
				\hline
			\end{tabular}
		\end{table}
		Điền số vào các ô trống sao cho tổng các số ở 3 dòng 1,2,3 lần lượt bằng $-5,11,1$. Tính tổng các số ở mỗi cột.
		\item[(b)] Cho bảng vuông $3\times 3$ ô. Có thể điền được hay không 9 số nguyên vào 9 ô của bảng sao cho tổng các số ở 3 dòng lần lượt bằng $5,-3,2$ \& tổng các số ở 3 cột lần lượt bằng $-1,2,2$?
	\end{itemize}
\end{baitoan}

\begin{baitoan}[\cite{Binh_Toan_6_tap_1}, \textbf{255.}, p. 43]
	\begin{enumerate*}
		\item[(a)] Có $10$ ô liên tiếp trong đó ô đầu tiên ghi số $6$, ô thứ $8$ ghi số $-4$. Điền số vào các ô trống để tổng 3 số ở 3 ô liền nhau bằng $0$.
		\item[(b)] 1 bảng vuông $4\times 4$ ô có 2 ô ở góc trên ghi số $-3$ \& $2$. Điền số vào các ô còn lại, sao cho tổng 2 số ở 2 ô liền nhau thì bằng nhau (2 ô liền nhau là 2 ô có 1 cạnh chung).
	\end{enumerate*}
\end{baitoan}

\begin{baitoan}[\cite{Binh_Toan_6_tap_1}, \textbf{256.}, p. 43]
	Tìm $x\in\mathbb{Z}$ biết $x + (x + 1) + (x + 2) + \cdots + 19 + 20 = 20$, trong đó vế trái là tổng các số nguyên liên tiếp viết theo thứ tự tăng dần.
\end{baitoan}

\begin{baitoan}[\cite{Binh_Toan_6_tap_1}, \textbf{257.}, p. 43]
	Tìm các số nguyên $a$ sao cho:
	\begin{enumerate*}
		\item[(a)] $a > -a$.
		\item[(b)] $a = -a$.
		\item[(c)] $a < -a$.
	\end{enumerate*}
\end{baitoan}

\begin{baitoan}[\cite{Binh_Toan_6_tap_1}, \textbf{258.}, p. 43]
	Tìm $a,b,c\in\mathbb{Z}$ biết: $a + b = 11$, $b + c = 3$, $c + a = 2$.
\end{baitoan}

\begin{baitoan}[\cite{Binh_Toan_6_tap_1}, \textbf{259.}, p. 43]
	Tìm $a,b,c,d\in\mathbb{Z}$ biết $a + b + c + d = 1$, $a + c + d = 2$, $a + b + d = 3$, $a + b + c = 4$.
\end{baitoan}

\begin{baitoan}[\cite{Binh_Toan_6_tap_1}, \textbf{260.}, p. 43]
	Cho $\sum_{i=1}^{51} x_i = x_1 + x_2 + \cdots + x_{50} + x_{51} = 0$ \& $x_1 + x_2 = x_3 + x_4 = \cdots = x_{47} + x_{48} = x_{49} + x_{50} = x_{50} + x_{51} = 1$. Tính $x_{50}$.
\end{baitoan}

%------------------------------------------------------------------------------%

\subsection{$\cdot,:$ Trên $\mathbb{Z}$}

\begin{baitoan}[\cite{Binh_Toan_6_tap_1}, Ví dụ 50, p. 43]
	\begin{itemize}
		\item[(a)] Cho bảng vuông $3\times 3$ ô:
		\begin{table}[H]
			\centering
			\begin{tabular}{|c|c|c|}
				\hline
				$\ \ 5$ & $\ \ 2$ & $-4$ \\
				\hline
				$-2$ & $-4$ & $-3$ \\
				\hline
				$-6$ & $\ \ 5$ & $\ \ 7$ \\
				\hline
			\end{tabular}
		\end{table}
		Tìm tích các số ở mỗi dòng, tích các số ở mỗi cột.
		\item[(b)] Viết $9$ số nguyên khác $0$ vào 1 bảng vuông $3\times 3$. Biết tích các số ở mỗi dòng đều là số âm. Chứng minh luôn luôn tồn tại 1 cột mà tích các số trong cột ấy là số âm.
	\end{itemize}
\end{baitoan}

\begin{baitoan}[\cite{Binh_Toan_6_tap_1}, Ví dụ 51, p. 44]
	Thay các dấu $\star$ trong biểu thức $1\star2\star3$ bằng các phép tính $+,-,\cdot,:$ \& thêm các dấu ngoặc để được kết quả là: số lớn nhất, số nhỏ nhất.
\end{baitoan}

\begin{baitoan}[\cite{Binh_Toan_6_tap_1}, \textbf{261.}, p. 44]
	Thực hiện các phép tính sau 1 cách nhanh chóng:
	\begin{enumerate*}
		\item[(a)] $(-14)\cdot(-125)\cdot3\cdot(-8)$;
		\item[(b)] $(-127)\cdot57 + (-127)\cdot43$;
		\item[(c)] $(-13)\cdot34 - 87\cdot34$;
		\item[(d)] $(-25)\cdot68 + (-34)\cdot(-250)$;
		\item[(e)] $A = 1 - 2 + 3 - 4 + \cdots + 99 - 100$;
		\item[(f)] $B = 1 + 3 - 5 - 7 + 9 + 11 - \cdots - 397 - 399$;
		\item[(g)] $C = 1 - 2 - 3 + 4 + 5 - 6 - 7 + \cdots + 97 - 98 - 99 + 100$;
		\item[(h)] $D = 2^{200} - 2^{99} - 2^{98} - \cdots - 2^2 - 2 - 1$.
	\end{enumerate*}
\end{baitoan}

\begin{baitoan}[\cite{Binh_Toan_6_tap_1}, \textbf{262.}, p. 44]
	Thay các dấu  $\star$ trong biểu thức $1\star2\star3\star4$ bằng dấu các phép tính $+,-,\cdot,:$ \& thêm các dấu ngoặc để được kết quả là: số lớn nhất, số nhỏ nhất.
\end{baitoan}

\begin{baitoan}[\cite{Binh_Toan_6_tap_1}, \textbf{263.}, p. 44]
	Tìm $x\in\mathbb{Z}$ sao cho:
	\begin{enumerate*}
		\item[(a)] $(x - 1)^2 = 0$;
		\item[(b)] $x(x - 1) = 0$;
		\item[(c)] $(x + 1)(x - 2) = 0$.
	\end{enumerate*}
\end{baitoan}

\begin{baitoan}[\cite{Binh_Toan_6_tap_1}, \textbf{264.}, p. 44]
	Cho dãy số $a_1,a_2,\ldots,a_{100}$ trong đó $a_1 = 1$, $a_2 = -1$, $a_k = a_{k-2}a_{k-1}$, $k\in\mathbb{N}$, $k\ge 3$. Tính $a_{100}$.
\end{baitoan}

\begin{baitoan}[\cite{Binh_Toan_6_tap_1}, \textbf{265.}, p. 44]
	Gọi $a,b,c,d,e,f,g,h$ là các số khác nhau trong tập hợp số $\{-7,-5,-3,-2,2,4,6,13\}$. Tính giá trị lớn nhất của biểu thức $A = (a + b + c + d)^2 + (e + f + g + h)^2$.
\end{baitoan}

%------------------------------------------------------------------------------%

\subsection{Tính Chia hết Trên $\mathbb{Z}$}

\begin{baitoan}[\cite{Binh_Toan_6_tap_1}, Ví dụ 52, p. 44]
	Số $36$ chia cho $a\in\mathbb{Z}$ rồi trừ đi $a$. Lấy kết quả này chia cho $a$ rồi trừ đi $a$. Lại lấy kết quả này chia cho $a$ rồi trừ đi $a$. Cuối cùng ta được số $-a$. Tìm $a$.\hfill\textsf{Ans:} $3$.
\end{baitoan}

\begin{baitoan}[\cite{Binh_Toan_6_tap_1}, \textbf{266.}, p. 44]
	Tìm $x,y\in\mathbb{Z}$ biết:
	\begin{enumerate*}
		\item[(a)] $(x + 2)(y - 3) = 5$;
		\item[(b)] $(x + 1)(xy - 1) = 3$.
	\end{enumerate*}
\end{baitoan}

\begin{baitoan}[\cite{Binh_Toan_6_tap_1}, \textbf{267.}, p. 44]
	Tính tổng $A + B$ biết $A$ là tổng các số nguyên âm lẻ có 2 chữ số, $B$ là tổng các số nguyên dương chẵn có 2 chữ số.
\end{baitoan}

\begin{baitoan}[\cite{Binh_Toan_6_tap_1}, \textbf{268.}, p. 44]
	Cho $A = 2 - 5 + 8 - 11 + 14 - 17 + \cdots + 98 - 101$.
	\begin{enumerate*}
		\item[(a)] Viết dạng tổng quát của số hạng thứ $n$ của $A$.
		\item[(b)] Tính giá trị của biểu thức $A$.
	\end{enumerate*}
\end{baitoan}

\begin{baitoan}[\cite{Binh_Toan_6_tap_1}, \textbf{269.}, p. 44]
	Cho $A = 1 + 2 - 3 - 4 + 5 + 6 - \cdots - 99 - 100$.
	\begin{enumerate*}
		\item[(a)] $A$ có chia hết cho $2$, cho $3$, cho $5$ hay không?
		\item[(b)] $A$ có bao nhiêu ước nguyên, có bao nhiêu ước tự nhiên?
	\end{enumerate*}
\end{baitoan}

%------------------------------------------------------------------------------%

\printbibliography[heading=bibintoc]
	
\end{document}