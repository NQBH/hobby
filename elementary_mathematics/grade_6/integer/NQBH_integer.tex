\documentclass{article}
\usepackage[backend=biber,natbib=true,style=authoryear]{biblatex}
\addbibresource{/home/nqbh/reference/bib.bib}
\usepackage[utf8]{vietnam}
\usepackage{tocloft}
\renewcommand{\cftsecleader}{\cftdotfill{\cftdotsep}}
\usepackage[colorlinks=true,linkcolor=blue,urlcolor=red,citecolor=magenta]{hyperref}
\usepackage{amsmath,amssymb,amsthm,mathtools,float,graphicx,algpseudocode,algorithm,tcolorbox}
\usepackage[inline]{enumitem}
\allowdisplaybreaks
\numberwithin{equation}{section}
\newtheorem{assumption}{Assumption}[section]
\newtheorem{baitoan}{Bài toán}
\newtheorem{cauhoi}{Câu hỏi}[section]
\newtheorem{conjecture}{Conjecture}[section]
\newtheorem{corollary}{Corollary}[section]
\newtheorem{definition}{Definition}[section]
\newtheorem{dinhly}{Định lý}[section]
\newtheorem{dinhnghia}{Định nghĩa}[section]
\newtheorem{example}{Example}[section]
\newtheorem{hequa}{Hệ quả}[section]
\newtheorem{lemma}{Lemma}[section]
\newtheorem{luuy}{Lưu ý}[section]
\newtheorem{notation}{Notation}[section]
\newtheorem{principle}{Principle}[section]
\newtheorem{problem}{Problem}[section]
\newtheorem{proposition}{Proposition}[section]
\newtheorem{question}{Question}[section]
\newtheorem{remark}{Remark}[section]
\newtheorem{theorem}{Theorem}[section]
\newtheorem{vidu}{Ví dụ}[section]
\usepackage[left=0.5in,right=0.5in,top=1.5cm,bottom=1.5cm]{geometry}
\usepackage{fancyhdr}
\pagestyle{fancy}
\fancyhf{}
\lhead{\small Sect.~\thesection}
\rhead{\small\nouppercase{\leftmark}}
\renewcommand{\subsectionmark}[1]{\markboth{#1}{}}
\cfoot{\thepage}
\def\labelitemii{$\circ$}
\DeclareRobustCommand{\divby}{%
	\mathrel{\vbox{\baselineskip.65ex\lineskiplimit0pt\hbox{.}\hbox{.}\hbox{.}}}%
}

\title{Integer -- Số Nguyên $\mathbb{Z}$}
\author{Nguyễn Quản Bá Hồng\footnote{Independent Researcher, Ben Tre City, Vietnam\\e-mail: \texttt{nguyenquanbahong@gmail.com}; website: \url{https://nqbh.github.io}.}}
\date{\today}

\begin{document}
\maketitle
\begin{abstract}
	\textsc{[en]} This text is a collection of problems, from easy to advanced, about integer. This text is also a supplementary material for my lecture note on Elementary Mathematics grade 6, which is stored \& downloadable at the following link: \href{https://github.com/NQBH/hobby/blob/master/elementary_mathematics/grade_6/NQBH_elementary_mathematics_grade_6.pdf}{GitHub\texttt{/}NQBH\texttt{/}hobby\texttt{/}elementary mathematics\texttt{/}grade 6\texttt{/}lecture}\footnote{\textsc{url}: \url{https://github.com/NQBH/hobby/blob/master/elementary_mathematics/grade_6/NQBH_elementary_mathematics_grade_6.pdf}.}. The latest version of this text has been stored \& downloadable at the following link: \href{https://github.com/NQBH/hobby/blob/master/elementary_mathematics/grade_6/integer/NQBH_integer.pdf}{GitHub\texttt{/}NQBH\texttt{/}hobby\texttt{/}elementary mathematics\texttt{/}grade 6\texttt{/}integer $\mathbb{Z}$}\footnote{\textsc{url}: \url{https://github.com/NQBH/hobby/blob/master/elementary_mathematics/grade_6/integer/NQBH_integer.pdf}.}.
	\vspace{2mm}
	
	\textsc{[vi]} Tài liệu này là 1 bộ sưu tập các bài tập chọn lọc từ cơ bản đến nâng cao về số nguyên. Tài liệu này là phần bài tập bổ sung cho tài liệu chính -- bài giảng \href{https://github.com/NQBH/hobby/blob/master/elementary_mathematics/grade_6/NQBH_elementary_mathematics_grade_6.pdf}{GitHub\texttt{/}NQBH\texttt{/}hobby\texttt{/}elementary mathematics\texttt{/}grade 6\texttt{/}lecture} của tác giả viết cho Toán Sơ Cấp lớp 6. Phiên bản mới nhất của tài liệu này được lưu trữ \& có thể tải xuống ở link sau: \href{https://github.com/NQBH/hobby/blob/master/elementary_mathematics/grade_6/integer/NQBH_integer.pdf}{GitHub\texttt{/}NQBH\texttt{/}hobby\texttt{/}elementary mathematics\texttt{/}grade 6\texttt{/}integer $\mathbb{Z}$}.
\end{abstract}
\tableofcontents

%------------------------------------------------------------------------------%

\section*{Số Nguyên}
``Tập hợp $\mathbb{Z}$ các số nguyên gồm các số tự nhiên \& các số $-1,-2,-3,\ldots$ $\mathbb{Z} = \{\ldots,-3,-2,-1,0,1,2,3,\ldots\}$. Ta xác định trên $\mathbb{Z}$ 1 thứ tự như sau: $a < b$ khi \& chỉ khi điểm $a$ ở bên trái điểm $b$ trên trục số ($a,b\in\mathbb{Z}$). Ta xác định trên $\mathbb{Z}$ 2 phép toán: phép cộng \& phép nhân. Phép cộng có 4 tính chất: giao hoán, kết hợp, cộng với số $0$, cộng với số đối. Phép nhân có 3 tính chất: giao hoán, kết hợp, nhân với số $1$. Giữa phép nhân \& phép cộng có quan hệ: phép nhân phân phối đối với phép cộng. Giữa thứ tự \& phép toán có quan hệ: $a < b\Rightarrow a + c < b + c$, $a < b\Rightarrow ac < bc$ với $c > 0$, $ac > bc$ với $c < 0$. Trừ đi 1 số là cộng với số đối của số trừ. Phép trừ 2 số nguyên bao giờ cũng thực hiện được\footnote{Phép trừ 2 số tự nhiên sẽ không thực hiện được (i.e., kết quả không phải là 1 số tự nhiên, hay không còn nằm trong $\mathbb{N}$) nếu số bị trừ nhỏ hơn số trừ.}. Phép chia chỉ thực hiện được trong phạm vi số nguyên khi số bị chia chia hết cho số chia. Trong trường hợp $a\divby b$, ta nói: $a$ là \textit{bội} của $b$ \& $b$ là \textit{ước} của $a$. \textit{Ước chung} (hoặc \textit{bội chung}) của 2 hay nhiều số là ước (hoặc bội) của tất cả các số đó.'' -- \cite[Chap. II, p. 41]{Binh_Toan_6_tap_1}

%------------------------------------------------------------------------------%

\section{Thứ Tự Trên $\mathbb{Z}$}

\begin{baitoan}[\cite{Binh_Toan_6_tap_1}, Ví dụ 48, p. 41]
	Cho $a\in\mathbb{Z}$. Gọi khoảng cách từ điểm $a$ đến điểm gốc trên trục số là \emph{giá trị tuyệt đối} của số $a$ \& ký hiệu là $|a|$. Điền vào chỗ trống các dấu $\ge,\le,>,<,=$ để các khẳng định sau là đúng:
	\begin{enumerate*}
		\item[(a)] $|a|\ldots a$, $\forall a\in\mathbb{Z}$.
		\item[(a)] $|a|\ldots 0$, $\forall a\in\mathbb{Z}$.
		\item[(c)] Nếu $a > 0$ thì $a\ldots|a|$.
		\item[(d)] Nếu $a = 0$ thì $a\ldots|a|$.
		\item[(e)] Nếu $a < 0$ thì $a\ldots|a|$.
	\end{enumerate*}
\end{baitoan}

\begin{baitoan}[\cite{Binh_Toan_6_tap_1}, \textbf{247.}, p. 42]
	Điền vào chỗ trống $\ldots$ các từ ``nhỏ hơn'' hoặc ``lớn hơn'' cho đúng:
	\begin{enumerate*}
		\item[(a)] Mọi số nguyên dương đều $\ldots$ số $0$.
		\item[(b)] Mọi số nguyên âm đều $\ldots$ số $0$.
		\item[(c)] Mỗi số nguyên dương đều $\ldots$ mọi số nguyên âm.
		\item[(d)] Trong 2 số nguyên dương, số nào có giá trị tuyệt đối lớn hơn thì số ấy $\ldots$
		\item[(e)] Trong 2 số nguyên âm, số nào có giá trị tuyệt đối lớn hơn thì số ấy $\ldots$
	\end{enumerate*}
\end{baitoan}

\begin{baitoan}[\cite{Binh_Toan_6_tap_1}, \textbf{248.}, p. 42]
	Tìm:
	\begin{enumerate*}
		\item[(a)] Số nguyên dương lớn nhất có 2 chữ số.
		\item[(a)] Số nguyên âm lớn nhất có 2 chữ số.
	\end{enumerate*}
\end{baitoan}

\begin{baitoan}[\cite{Binh_Toan_6_tap_1}, \textbf{249.}, p. 42]
	Tính $|b| - |a|$ biết: 
	\begin{enumerate*}
		\item[(a)] $a = -3$, $b = 7$;
		\item[(b)] $a = 5$, $b = -6$;
		\item[(c)] $a = 5$, $b = -5$;
	\end{enumerate*}
\end{baitoan}

\begin{baitoan}[\cite{Binh_Toan_6_tap_1}, \textbf{250.}, p. 42]
	Các khẳng định sau có đúng $\forall a,b\in\mathbb{Z}$ hay không? Cho ví dụ.
	\begin{enumerate*}
		\item[(a)] $|a| = |b|\Rightarrow a = b$.
		\item[(b)] $a > b\Rightarrow|a| > |b|$.
	\end{enumerate*}
\end{baitoan}

%------------------------------------------------------------------------------%

\section{$\pm$ Trêm $\mathbb{Z}$}

\begin{baitoan}[\cite{Binh_Toan_6_tap_1}, Ví dụ 49, p. 42]
	Tìm $x\in\mathbb{Z}$, biết $10 = 10 + 9 + 8 + \cdots + x$, trong đó vế phải là tổng các số nguyên liên tiếp viết theo thứ tự giảm dần.
\end{baitoan}

\begin{baitoan}[\cite{Binh_Toan_6_tap_1}, \textbf{251.}, p. 42]
	Tìm tổng của số nguyên âm nhỏ nhất có 1 chữ số \& số nguyên dương lớn nhất có 1 chữ số.
\end{baitoan}

\begin{baitoan}[\cite{Binh_Toan_6_tap_1}, \textbf{252.}, p. 42]
	Điền vào chỗ trống cho đúng:
	\begin{enumerate*}
		\item[(a)] Số đối của 1 số nguyên âm là 1 số $\ldots$
		\item[(b)] 2 số nguyên đối nhau thì có giá trị tuyệt đối $\ldots$
		\item[(c)] 2 số nguyên có giá trị tuyệt đối bằng nhau thì $\ldots$
		\item[(d)] Số $\ldots$ thì nhỏ hơn số đối của nó.
		\item[(e)] Nếu $a\ldots$ thì $-a > 0$.
		\item[(f)] Nếu $a < 0$ thì $|a| = \ldots$
		\item[(g)] Nếu $a < 0$ thì $a + |a| = \ldots$
	\end{enumerate*}
\end{baitoan}

\begin{baitoan}[\cite{Binh_Toan_6_tap_1}, \textbf{253.}, p. 43]
	Tìm $x\in\mathbb{Z}$ biết:
	\begin{enumerate*}
		\item[(a)] $x + 13 = 5$.
		\item[(b)] $x - 1 = -9$.
		\item[(c)] $25 - |x| = 10$.
		\item[(d)] $|x - 2| + 7 = 12$.
		\item[(e)] $x + 4$ là số nguyên dương nhỏ nhất.
		\item[(f)] $10 - x$ là số nguyên âm lớn nhất.
	\end{enumerate*}
\end{baitoan}

\begin{baitoan}[\cite{Binh_Toan_6_tap_1}, \textbf{254.}, p. 43]
	\begin{itemize}
		\item[(a)] Cho bảng vuông $3\times 3$ ô:
		\begin{table}[H]
			\centering
			\begin{tabular}{|c|c|c|}
				\hline
				$-8$ & $7$ &  \\
				\hline
				$\ \ 5$ &  & $\ \ 9$ \\
				\hline
				& $5$ & $-6$ \\
				\hline
			\end{tabular}
		\end{table}
		Điền số vào các ô trống sao cho tổng các số ở 3 dòng 1,2,3 lần lượt bằng $-5,11,1$. Tính tổng các số ở mỗi cột.
		\item[(b)] Cho bảng vuông $3\times 3$ ô. Có thể điền được hay không 9 số nguyên vào 9 ô của bảng sao cho tổng các số ở 3 dòng lần lượt bằng $5,-3,2$ \& tổng các số ở 3 cột lần lượt bằng $-1,2,2$?
	\end{itemize}
\end{baitoan}

\begin{baitoan}[\cite{Binh_Toan_6_tap_1}, \textbf{255.}, p. 43]
	\begin{enumerate*}
		\item[(a)] Có $10$ ô liên tiếp trong đó ô đầu tiên ghi số $6$, ô thứ $8$ ghi số $-4$. Điền số vào các ô trống để tổng 3 số ở 3 ô liền nhau bằng $0$.
		\item[(b)] 1 bảng vuông $4\times 4$ ô có 2 ô ở góc trên ghi số $-3$ \& $2$. Điền số vào các ô còn lại, sao cho tổng 2 số ở 2 ô liền nhau thì bằng nhau (2 ô liền nhau là 2 ô có 1 cạnh chung).
	\end{enumerate*}
\end{baitoan}

\begin{baitoan}[\cite{Binh_Toan_6_tap_1}, \textbf{256.}, p. 43]
	Tìm $x\in\mathbb{Z}$ biết $x + (x + 1) + (x + 2) + \cdots + 19 + 20 = 20$, trong đó vế trái là tổng các số nguyên liên tiếp viết theo thứ tự tăng dần.
\end{baitoan}

\begin{baitoan}[\cite{Binh_Toan_6_tap_1}, \textbf{257.}, p. 43]
	Tìm các số nguyên $a$ sao cho:
	\begin{enumerate*}
		\item[(a)] $a > -a$.
		\item[(b)] $a = -a$.
		\item[(c)] $a < -a$.
	\end{enumerate*}
\end{baitoan}

\begin{baitoan}[\cite{Binh_Toan_6_tap_1}, \textbf{258.}, p. 43]
	Tìm $a,b,c\in\mathbb{Z}$ biết: $a + b = 11$, $b + c = 3$, $c + a = 2$.
\end{baitoan}

\begin{baitoan}[\cite{Binh_Toan_6_tap_1}, \textbf{259.}, p. 43]
	Tìm $a,b,c,d\in\mathbb{Z}$ biết $a + b + c + d = 1$, $a + c + d = 2$, $a + b + d = 3$, $a + b + c = 4$.
\end{baitoan}

\begin{baitoan}[\cite{Binh_Toan_6_tap_1}, \textbf{260.}, p. 43]
	Cho $\sum_{i=1}^{51} x_i = x_1 + x_2 + \cdots + x_{50} + x_{51} = 0$ \& $x_1 + x_2 = x_3 + x_4 = \cdots = x_{47} + x_{48} = x_{49} + x_{50} = x_{50} + x_{51} = 1$. Tính $x_{50}$.
\end{baitoan}

%------------------------------------------------------------------------------%

\section{$\cdot,:$ Trên $\mathbb{Z}$}

\begin{baitoan}[\cite{Binh_Toan_6_tap_1}, Ví dụ 50, p. 43]
	\begin{itemize}
		\item[(a)] Cho bảng vuông $3\times 3$ ô:
		\begin{table}[H]
			\centering
			\begin{tabular}{|c|c|c|}
				\hline
				$\ \ 5$ & $\ \ 2$ & $-4$ \\
				\hline
				$-2$ & $-4$ & $-3$ \\
				\hline
				$-6$ & $\ \ 5$ & $\ \ 7$ \\
				\hline
			\end{tabular}
		\end{table}
		Tìm tích các số ở mỗi dòng, tích các số ở mỗi cột.
		\item[(b)] Viết $9$ số nguyên khác $0$ vào 1 bảng vuông $3\times 3$. Biết tích các số ở mỗi dòng đều là số âm. Chứng minh luôn luôn tồn tại 1 cột mà tích các số trong cột ấy là số âm.
	\end{itemize}
\end{baitoan}

\begin{baitoan}[\cite{Binh_Toan_6_tap_1}, Ví dụ 51, p. 44]
	Thay các dấu $\star$ trong biểu thức $1\star2\star3$ bằng các phép tính $+,-,\cdot,:$ \& thêm các dấu ngoặc để được kết quả là: số lớn nhất, số nhỏ nhất.
\end{baitoan}

\begin{baitoan}[\cite{Binh_Toan_6_tap_1}, \textbf{261.}, p. 44]
	Thực hiện các phép tính sau 1 cách nhanh chóng:
	\begin{enumerate*}
		\item[(a)] $(-14)\cdot(-125)\cdot3\cdot(-8)$;
		\item[(b)] $(-127)\cdot57 + (-127)\cdot43$;
		\item[(c)] $(-13)\cdot34 - 87\cdot34$;
		\item[(d)] $(-25)\cdot68 + (-34)\cdot(-250)$;
		\item[(e)] $A = 1 - 2 + 3 - 4 + \cdots + 99 - 100$;
		\item[(f)] $B = 1 + 3 - 5 - 7 + 9 + 11 - \cdots - 397 - 399$;
		\item[(g)] $C = 1 - 2 - 3 + 4 + 5 - 6 - 7 + \cdots + 97 - 98 - 99 + 100$;
		\item[(h)] $D = 2^{200} - 2^{99} - 2^{98} - \cdots - 2^2 - 2 - 1$.
	\end{enumerate*}
\end{baitoan}

\begin{baitoan}[\cite{Binh_Toan_6_tap_1}, \textbf{262.}, p. 44]
	Thay các dấu  $\star$ trong biểu thức $1\star2\star3\star4$ bằng dấu các phép tính $+,-,\cdot,:$ \& thêm các dấu ngoặc để được kết quả là: số lớn nhất, số nhỏ nhất.
\end{baitoan}

\begin{baitoan}[\cite{Binh_Toan_6_tap_1}, \textbf{263.}, p. 44]
	Tìm $x\in\mathbb{Z}$ sao cho:
	\begin{enumerate*}
		\item[(a)] $(x - 1)^2 = 0$;
		\item[(b)] $x(x - 1) = 0$;
		\item[(c)] $(x + 1)(x - 2) = 0$.
	\end{enumerate*}
\end{baitoan}

\begin{baitoan}[\cite{Binh_Toan_6_tap_1}, \textbf{264.}, p. 44]
	Cho dãy số $a_1,a_2,\ldots,a_{100}$ trong đó $a_1 = 1$, $a_2 = -1$, $a_k = a_{k-2}a_{k-1}$, $k\in\mathbb{N}$, $k\ge 3$. Tính $a_{100}$.
\end{baitoan}

\begin{baitoan}[\cite{Binh_Toan_6_tap_1}, \textbf{265.}, p. 44]
	Gọi $a,b,c,d,e,f,g,h$ là các số khác nhau trong tập hợp số $\{-7,-5,-3,-2,2,4,6,13\}$. Tính giá trị lớn nhất của biểu thức $A = (a + b + c + d)^2 + (e + f + g + h)^2$.
\end{baitoan}

%------------------------------------------------------------------------------%

\section{Tính Chia hết Trên $\mathbb{Z}$}

\begin{baitoan}[\cite{Binh_Toan_6_tap_1}, Ví dụ 52, p. 44]
	Số $36$ chia cho $a\in\mathbb{Z}$ rồi trừ đi $a$. Lấy kết quả này chia cho $a$ rồi trừ đi $a$. Lại lấy kết quả này chia cho $a$ rồi trừ đi $a$. Cuối cùng ta được số $-a$. Tìm $a$.\hfill\textsf{Ans:} $3$.
\end{baitoan}

\begin{baitoan}[\cite{Binh_Toan_6_tap_1}, \textbf{266.}, p. 45]
	Tìm $x,y\in\mathbb{Z}$ biết:
	\begin{enumerate*}
		\item[(a)] $(x + 2)(y - 3) = 5$;
		\item[(b)] $(x + 1)(xy - 1) = 3$.
	\end{enumerate*}
\end{baitoan}

\begin{baitoan}[\cite{Binh_Toan_6_tap_1}, \textbf{267.}, p. 45]
	Tính tổng $A + B$ biết $A$ là tổng các số nguyên âm lẻ có 2 chữ số, $B$ là tổng các số nguyên dương chẵn có 2 chữ số.
\end{baitoan}

\begin{baitoan}[\cite{Binh_Toan_6_tap_1}, \textbf{268.}, p. 45]
	Cho $A = 2 - 5 + 8 - 11 + 14 - 17 + \cdots + 98 - 101$.
	\begin{enumerate*}
		\item[(a)] Viết dạng tổng quát của số hạng thứ $n$ của $A$.
		\item[(b)] Tính giá trị của biểu thức $A$.
	\end{enumerate*}
\end{baitoan}

\begin{baitoan}[\cite{Binh_Toan_6_tap_1}, \textbf{269.}, p. 45]
	Cho $A = 1 + 2 - 3 - 4 + 5 + 6 - \cdots - 99 - 100$.
	\begin{enumerate*}
		\item[(a)] $A$ có chia hết cho $2$, cho $3$, cho $5$ hay không?
		\item[(b)] $A$ có bao nhiêu ước nguyên, có bao nhiêu ước tự nhiên?
	\end{enumerate*}
\end{baitoan}

\begin{baitoan}[\cite{Binh_Toan_6_tap_1}, \textbf{270.}, p. 45]
	Cho dãy số $1,-3,5,-7,9,-11,13,-15,17,-19$. Có thể tìm được hay không 5 số trong các số trên, sao cho đặt dấu ``$+$'' hoặc ``$-$'' nối các số đó với nhau, ta được kết quả bằng:
	\begin{enumerate*}
		\item[(a)] $15$;
		\item[(b)] $20$?
	\end{enumerate*}
\end{baitoan}

\begin{baitoan}[\cite{Binh_Toan_6_tap_1}, \textbf{271.}, p. 45]
	Thay các dấu  $\star$ trong biểu thức $1\star2\star3\star4\star5\star6\star7\star8\star9$ bởi các dấu ``$+$'' hoặc ``$-$'' để giá trị của biểu thức bằng:
	\begin{enumerate*}
		\item[(a)] $-13$;
		\item[(b)] $-4$?
	\end{enumerate*}
\end{baitoan}

\begin{baitoan}[\cite{Binh_Toan_6_tap_1}, \textbf{272.}, p. 45]
	Tìm $n\in\mathbb{Z}$ sao cho:
	\begin{enumerate*}
		\item[(a)] $n + 5\divby n - 2$;
		\item[(b)] $2n + 1\divby n - 5$;
		\item[(c)] $n^2 + 3n - 13\divby n + 3$;
		\item[(d)] $n^2 + 3\divby n - 1$.
	\end{enumerate*}
\end{baitoan}

\begin{baitoan}[\cite{Binh_Toan_6_tap_1}, \textbf{273.}, p. 45]
	Tìm các số $a,b,c,d,m$ khác nhau thuộc tập hợp $\{-2,-1,0,1,2\}$ sao cho $a < b < \min\{c,d\}$, với $\min\{c,d\}$ là số nhỏ hơn trong 2 số $c,d$, \& đặt $m$ nằm ở trung tâm, các số $a,b,c,d$ lần lượt nằm ở bên trái, bên trên, bên phải, bên dưới của $m$, \& tổng của 3 số trên đường nằm ngang bằng tổng của 3 số trên đường thẳng đứng.
\end{baitoan}

\begin{baitoan}[\cite{Binh_Toan_6_tap_1}, \textbf{274.}${}^\star$, p. 45]
	Cho $n$ số nguyên (có thể có số âm) với $n > 1$ mà tổng \& tích của chúng đều bằng $505$. Tìm giá trị nhỏ nhất của $n$.
\end{baitoan}

%------------------------------------------------------------------------------%

\section{Điền Chữ Số}
``Các bài toán về điền chữ số không chỉ yêu cầu kỹ năng tính toán đúng mà còn đòi hỏi cả lập luận chính xác \& hợp lý. Ta quy ước rằng khi ở đề bài cho các chữ $a,b,c,\ldots$ mà không chú thích gì thêm, ta hiểu rằng các chữ khác nhau biểu thị các chữ số khác nhau.'' -- \cite[p. 46]{Binh_Toan_6_tap_1}

\begin{baitoan}[\cite{Binh_Toan_6_tap_1}, Ví dụ 53, p. 46]
	Thay các chữ bởi các chữ số thích hợp: $\overline{abc} + \overline{acb} = \overline{bca}$.
\end{baitoan}

\begin{baitoan}[\cite{Binh_Toan_6_tap_1}, Ví dụ 54, p. 46]
	Tìm các chữ số $a,b,c$ biết tổng $a + b + c$ bằng tổng của 4 số chẵn liên tiếp \& các chữ số $a,b,c$ thỏa mãn cả 2 phép trừ sau: $\overline{abc} - \overline{cba} = 99$ \& $\overline{bac} - \overline{abc} = 270$.
\end{baitoan}

\begin{baitoan}[\cite{Binh_Toan_6_tap_1}, Ví dụ 55, p. 46]
	Thay các dấu {\bf*} bằng các chữ số thích hợp trong phép chia:
	\begin{figure}[H]
		\centering
		\includegraphics[scale=0.13]{Binh_vi_du_55_p_47}
	\end{figure}
\end{baitoan}

\begin{baitoan}[\cite{Binh_Toan_6_tap_1}, Ví dụ 56, p. 47]
	Thay các chữ $a,b,c$ bằng các chữ số khác nhau thích hợp trong phép nhân sau: $\overline{ab}\cdot\overline{cc}\cdot\overline{abc} = \overline{abcabc}$.
\end{baitoan}

\begin{baitoan}[\cite{Binh_Toan_6_tap_1}, Ví dụ 57, p. 47]
	Tìm số tự nhiên có 3 chữ số, biết trong 2 cách viết: viết thêm chữ số $5$ vào đằng sau số đó hoặc viết thêm chữ số $1$ vào đằng trước số đó thì cách viết thứ nhất cho số lớn gấp $5$ lần so với cách viết thứ 2.
\end{baitoan}

\begin{baitoan}[\cite{Binh_Toan_6_tap_1}, Ví dụ 58, p. 48]
	Điền các chữ số thích hợp vào các chữ trong phép nhân sau: $2\overline{abcdmn} = \overline{cdmnab}$.
\end{baitoan}

\begin{baitoan}[\cite{Binh_Toan_6_tap_1}, Ví dụ 59, p. 48]
	Điền các chữ số thích hợp vào các dấu $\star$ trong phép nhân sau: $\star\star\cdot\star\star = \star\star\star$ biết cả 2 thừa số đều chẵn \& tích là số có 3 chữ số như nhau.
\end{baitoan}

\begin{baitoan}[\cite{Binh_Toan_6_tap_1}, Ví dụ 60, p. 48]
	Tìm các chữ số $a$ \& $b$, biết $900:(a + b) = \overline{ab}$.
\end{baitoan}

\begin{baitoan}[\cite{Binh_Toan_6_tap_1}, Ví dụ 61, p. 49]
	Chứng minh không thể thay các chữ bằng các chữ số để có phép tính đúng:
	\begin{enumerate*}
		\item[(a)] \emph{HỌC VUI $-$ VUI HỌC} $= 1991$;
		\item[(b)] \emph{TOÁN $+$ LÝ $+$ SỬ $+$ VẼ} $= 1992$.
	\end{enumerate*}
\end{baitoan}
Thay các dấu $\star$ \& các chữ bởi các số thích hợp:

\begin{baitoan}[\cite{Binh_Toan_6_tap_1}, \textbf{275.}, p. 49]
	$\overline{ab} + \overline{bc} + \overline{ca} = \overline{abc}$.
\end{baitoan}

\begin{baitoan}[\cite{Binh_Toan_6_tap_1}, \textbf{276.}, p. 49]
	\begin{enumerate*}
		\item[(a)] $\overline{abc} + \overline{ab} + a = 874$;
		\item[(b)] $\overline{abc} + \overline{ab} + a = 1037$.
	\end{enumerate*}
\end{baitoan}

\begin{baitoan}[\cite{Binh_Toan_6_tap_1}, \textbf{277.}, p. 49]
	\begin{enumerate*}
		\item[(a)] $\overline{acc}\cdot b = \overline{dba}$ biết $a$ là chữ số lẻ;
		\item[(b)] $\overline{ac}\cdot\overline{ac} = \overline{acc}$;
		\item[(c)] $\overline{ab}\cdot\overline{ab} = \overline{acc}$.
	\end{enumerate*}
\end{baitoan}

\begin{baitoan}[\cite{Binh_Toan_6_tap_1}, \textbf{278.}, p. 49]
	\begin{enumerate*}
		\item[(a)] $2\overline{1bac} = \overline{abc8}$;
		\item[(b)] $\overline{ab} = 9b$.
	\end{enumerate*}
\end{baitoan}

\begin{baitoan}[\cite{Binh_Toan_6_tap_1}, \textbf{279.}, p. 49]
	$4\overline{abcdef} = \overline{fabcde}$ \& $\overline{abcde} + f = 15390$.
\end{baitoan}

\begin{baitoan}[\cite{Binh_Toan_6_tap_1}, \textbf{280.}, p. 49]
	$\overline{abc} - \overline{ca} = \overline{ca} - \overline{ac}$.
\end{baitoan}

\begin{baitoan}[\cite{Binh_Toan_6_tap_1}, \textbf{281.}, p. 49]
	$\overline{abcd} + \overline{abc} = 3576$.
\end{baitoan}

\begin{baitoan}[\cite{Binh_Toan_6_tap_1}, \textbf{282.}, p. 49]
	$\overline{abcd0} - \overline{abcd} = \overline{3462\star}$.
\end{baitoan}

\begin{baitoan}[\cite{Binh_Toan_6_tap_1}, \textbf{283.}, p. 49]
	Thay các dấu {\bf*} bởi các số thích hợp:
	\begin{figure}[H]
		\centering
		\includegraphics[scale=0.13]{Binh_194_p_49}
	\end{figure}
	biết số bị nhân có tổng các chữ số bằng $18$ \& không đổi khi đọc từ phải sang trái.
\end{baitoan}

\begin{baitoan}[\cite{Binh_Toan_6_tap_1}, \textbf{284.}, p. 49]
	\begin{enumerate*}
		\item[(a)] $\overline{ab}\cdot b = \overline{1ab}$;
		\item[(b)] $\overline{abc} = 9\overline{bc}$.
	\end{enumerate*}
\end{baitoan}

\begin{baitoan}[\cite{Binh_Toan_6_tap_1}, \textbf{285.}, p. 50]
	$\overline{260abc}:\overline{abc} = 626$.
\end{baitoan}

\begin{baitoan}[\cite{Binh_Toan_6_tap_1}, \textbf{286.}, p. 50]
	Thay các dấu {\bf*} bởi các số thích hợp:
	\begin{figure}[H]
		\centering
		\includegraphics[scale=0.13]{Binh_286_p_50}
	\end{figure}
\end{baitoan}

\begin{baitoan}[\cite{Binh_Toan_6_tap_1}, \textbf{287.}, p. 50]
	\begin{enumerate*}
		\item[(a)] $\overline{ab}\cdot\overline{cb} = \overline{ddd}$;
		\item[(b)] $\star\star\cdot\,\star = \star\star\star$;
		\item[(c)] $\overline{ab}\cdot\overline{cd} = bbb$. Biết tích là số có 3 chữ số như nhau.
	\end{enumerate*}
\end{baitoan}

\begin{baitoan}[\cite{Binh_Toan_6_tap_1}, \textbf{288.}, p. 50]
	$6\overline{abcdef} = \overline{defabc}$.
\end{baitoan}

\begin{baitoan}[\cite{Binh_Toan_6_tap_1}, \textbf{289.}, p. 50]
	$20\star\star:13 = \star\star7$.
\end{baitoan}

\begin{baitoan}[\cite{Binh_Toan_6_tap_1}, \textbf{290.}, p. 50]
	Thay các dấu {\bf*} bởi các số thích hợp:
	\begin{figure}[H]
		\centering
		\includegraphics[scale=0.13]{Binh_290_p_50}
	\end{figure}
\end{baitoan}

\begin{baitoan}[\cite{Binh_Toan_6_tap_1}, \textbf{291.}, p. 50]
	$\overline{abc}:11 = a + b + c$.
\end{baitoan}

\begin{baitoan}[\cite{Binh_Toan_6_tap_1}, \textbf{292.}, p. 50]
	$(\overline{ab} + \overline{cd})(\overline{ab} - \overline{cd}) = 2002$.
\end{baitoan}

\begin{baitoan}[\cite{Binh_Toan_6_tap_1}, \textbf{293.}, p. 50]
	\begin{enumerate*}
		\item[(a)] $a\cdot\overline{bc} = d\cdot\overline{ef} = 156$ (các chữ khác các chữ số đã có);
		\item[(b)] $\overline{ab}\cdot\overline{cde} = 16038$ (các chữ khác các chữ số đã có).
	\end{enumerate*}
\end{baitoan}

\begin{baitoan}[\cite{Binh_Toan_6_tap_1}, \textbf{294.}, p. 50]
	Tìm chữ số $a$ sao cho $n = \overline{\underbrace{4\ldots4}_{\scriptsize55\mbox{ số}}a\underbrace{6\ldots6}_{\scriptsize55\mbox{ số}}}\divby13$.
\end{baitoan}

\begin{baitoan}[\cite{Binh_Toan_6_tap_1}, \textbf{295.}, p. 50]
	Tìm chữ số $a$ \& $x\in\mathbb{N}$ sao cho: $(12 + 3x)^2 = \overline{1a96}$.
\end{baitoan}

\begin{baitoan}[\cite{Binh_Toan_6_tap_1}, \textbf{296.}, p. 50]
	Tìm số tự nhiên có 5 chữ số, biết rằng nếu viết thêm chữ số $7$ vào đằng trước số đó thì được 1 số lớn gấp $4$ lần so với số có được bằng cách viết thêm chữ số $7$ vào sau số đó.
\end{baitoan}

\begin{baitoan}[\cite{Binh_Toan_6_tap_1}, \textbf{297.}, p. 50]
	Tìm số tự nhiên có 2 chữ số, biết rằng nếu viết thêm 1 chữ số $2$ vào bên phải \& 1 chữ số $2$ vào bên trái của nó thì số ấy tăng gấp $36$ lần.
\end{baitoan}

\begin{baitoan}[\cite{Binh_Toan_6_tap_1}, \textbf{298.}, p. 50]
	Tìm số tự nhiên có 2 chữ số, biết rằng nếu viết xen vào giữa 2 chữ số của nó chính số đó thì số đó tăng gấp $99$ lần.
\end{baitoan}

\begin{baitoan}[\cite{Binh_Toan_6_tap_1}, \textbf{299.}, p. 50]
	Tìm số tự nhiên có 4 chữ số, sao cho khi nhân số đó với $4$ ta được số gồm 4 chữ số ấy viết theo thứ tự ngược lại.
\end{baitoan}

\begin{baitoan}[\cite{Binh_Toan_6_tap_1}, \textbf{300.}, p. 50]
	Tìm số tự nhiên có 4 chữ số, sao cho nhân nó với $9$ ta được số gồm chính các chữ số của số ấy viết theo thứ tự ngược lại.
\end{baitoan}

\begin{baitoan}[\cite{Binh_Toan_6_tap_1}, \textbf{301.}, p. 51]
	Tìm số tự nhiên có 5 chữ số, sao cho nhân nó với $9$ ta được số gồm chính các chữ số của số ấy viết theo thứ tự ngược lại.
\end{baitoan}

\begin{baitoan}[\cite{Binh_Toan_6_tap_1}, \textbf{302.}, p. 51]
	\begin{enumerate*}
		\item[(a)] Tìm số tự nhiên có 3 chữ số, biết rằng nếu xóa chữ số hàng trăm thì số ấy giảm $9$ lần.
		\item[(b)] Giải bài toán trên nếu không cho biết chữ số bị xóa thuộc hàng nào.
	\end{enumerate*}
\end{baitoan}

\begin{baitoan}[\cite{Binh_Toan_6_tap_1}, \textbf{303.}, p. 51]
	Tìm $n\in\mathbb{N}$ có 3 chữ số khác nhau, biết rằng nếu xóa bất kỳ chữ số nào của nó ta cũng được 1 số là ước của $n$.	
\end{baitoan}

\begin{baitoan}[\cite{Binh_Toan_6_tap_1}, \textbf{304.}, p. 51]
	Tìm số tự nhiên có 4 chữ số, biết rằng nếu xóa chữ số hàng nhìn thì số ấy giảm $9$ lần.
\end{baitoan}

\begin{baitoan}[\cite{Binh_Toan_6_tap_1}, \textbf{305.}, p. 51]
	\begin{enumerate*}
		\item[(a)] Tìm số tự nhiên có 4 chữ số, biết rằng chữ số hàng trăm bằng $0$ \& nếu xóa chữ số $0$ đó thì số ấy giảm $9$ lần.
		\item[(b)] 1 số tự nhiên tăng gấp $9$ lần nếu viết thêm 1 chữ số $0$ vào giữa các chữ số hàng chục \& hàng đơn vị của nó. Tìm số ấy.
	\end{enumerate*}	
\end{baitoan}

\begin{baitoan}[\cite{Binh_Toan_6_tap_1}, \textbf{306.}, p. 51]
	Tìm $A\in\mathbb{N}$, biết rằng nếu xóa 1 hoặc nhiều chữ số tận cùng của nó thì được số $B$ mà $A = 130B$.
\end{baitoan}

\begin{baitoan}[\cite{Binh_Toan_6_tap_1}, \textbf{307.}${}^\star$, p. 51]
	Tìm $x\in\mathbb{N}$ có chữ số tận cùng bằng $2$, biết rằng $x,2x,3x$ đều là các số có 3 chữ số \& 9 chữ số của 3 số đó đều khác nhau \& khác $0$.
\end{baitoan}

\begin{baitoan}[\cite{Binh_Toan_6_tap_1}, \textbf{308.}${}^\star$, p. 51]
	Tìm $x\in\mathbb{N}$ có 6 chữ số, biết rằng các tích $2x,3x,4x,5x,6x$ cũng là số có 6 chữ số gồm cả 6 chữ số ấy.
	\begin{enumerate*}
		\item[(a)] Cho biết 6 chữ số của số phải tìm là $1,2,4,5,7,8$.
		\item[(b)] Giải bài toán nếu không cho điều kiện (a).
	\end{enumerate*}
\end{baitoan}

%------------------------------------------------------------------------------%

\section{Dãy Các Số Viết Theo Quy Luật}

\subsection{Dãy cộng}
``Xét các dãy số sau:
\begin{enumerate*}
	\item[(a)] Dãy số tự nhiên: $0,1,2,3,\ldots$;
	\item[(b)] Dãy số lẻ: $1,3,5,7,\ldots$;
	\item[(c)] Dãy các số chia cho $3$ dư $1$: $1,4,7,10,\ldots$
\end{enumerate*}
Trong các dãy số trên, mỗi số hạng, kể từ số hạng thứ 2, đều lớn hơn số hạng đứng liền trước nó cùng 1 đơn vị, số đơn vị này là $1$ ở dãy (a), là $2$ ở dãy (b), là $3$ ở dãy (c). Ta gọi các dãy trên là \textit{dãy cộng}.

Xét dãy cộng $4,7,10,13,16,19,\ldots$ Hiệu giữa 2 số liên tiếp của dãy là $3$. Số hạng thứ 6 của dãy này là $19$, bằng: $4 + (6 - 1)\cdot3$; số hạng thứ $10$ của dãy này là $4 + (10 - 1)\cdot3 = 31$. Tổng quát, nếu 1 dãy cộng có số hạng đầu là $a_1$ \& hiệu giữa 2 số hạng liên tiếp là $d$ thì số hạng thứ $n$ của dãy cộng đó (ký hiệu $a_n$) bằng: $a_n = a_1 + (n - 1)d$, $\forall n\in\mathbb{N}^\star$. Để tính tổng các số hạng của dãy cộng $4 + 7 + 10 + \cdots + 25 + 28 + 31$ (gồm $10$ số) ta viết: $A = 4 + 7 + 10 + \cdots + 25 + 28 + 31$, $A = 31 + 28 + 25 + \cdots + 10 + 7 + 4$ nên $2A = (4 + 31) + (7 + 28) + \cdots + (28 + 7) + (31 + 4) = (4 + 31)\cdot10$. Do đó $A = \frac{(4 + 31)\cdot10}{2} = 175$.

Tổng quát, nếu 1 dãy cộng có $n$ số hạng, số hạng đầu là $a_1$, số hạng cuối là $a_n$ thì tổng của $n$ số hạng đó được tính như sau: $S = \frac{(a_1 + a_n)\cdot n}{2}$. Quy tắc dân gian: dĩ đầu, cộng vĩ, chiết bán, nhân chi (lấy số đầu cộng với số cuối, chia đôi, nhân với số số hạng). Trường hợp đặc biệt, tổng của $n$ số tự nhiên liên tiếp bắt đầu từ 1 bằng: $\sum_{i=1}^n i = 1 + 2 + \cdots + n = \frac{1}{2}n(n + 1)$.'' -- \cite[Chuyên đề 2, pp. 51--52]{Binh_Toan_6_tap_1} (Cho $a_1 = 1$, $a_n = n$ trong công thức $S = \frac{1}{2}n(a_1 + a_n)$.)

\begin{dinhnghia}[Dãy cộng]
	\emph{Dãy cộng} là dãy có dạng $\{a + n b\}_{n=0}^\infty = a, a + b, a + 2b, a + 3b,\ldots$, với $a,b\in\mathbb{N}$, $b\ne 0$.
\end{dinhnghia}
Trong các dãy số cộng, mỗi số hạng, kể từ số hạng thứ 2, đều lớn hơn số hạng đứng trước nó cùng 1 số đơn vị là $b$.

\begin{vidu}
	\begin{enumerate*}
		\item[(a)] $a = 0$, $b = 1$, dãy $\{a + n b\}_{n=0}^\infty = \{n\}_{n=0}^\infty = \mathbb{N} = 0,1,2,3,\ldots$ là dãy các số tự nhiên.
		\item[(b)] $a = 1$, $b = 2$, dãy $\{a + n b\}_{n=0}^\infty = \{1 + 2n\}_{n=0}^\infty = 1,3,5,7,\ldots$ là dãy các số tự nhiên lẻ.
		\item[(c)] $a = 0$, $b = 2$, $\{a + n b\}_{n=0}^\infty = \{2n\}_{n=0}^\infty = 0,2,4,6,\ldots$ là dãy các số tự nhiên chẵn.
		\item[(d)] Với $b\in\mathbb{N}^\star$, $b\ge 2$, $a\in\mathbb{N}$, $a < b$, dãy $\{a + n b\}_{n=0}^\infty$ là dãy các số tự nhiên chia cho $b$ dư $a$.
	\end{enumerate*}
\end{vidu}

\begin{baitoan}[\cite{Binh_Toan_6_tap_1}, Ví dụ 62, p. 52]
	Bạn Lâm đánh số trang 1 cuốn sách dày $284$ trang bằng dãy số chẵn $2,4,6,8,\ldots$
	\begin{enumerate*}
		\item[(a)] Biết mỗi chữ số viết mất $1$ giây. Hỏi bạn Lâm cần bao nhiêu phút để đánh số trang cuốn sách?
		\item[(b)] Chữ số thứ $300$ mà bạn Lâm viết là chữ số nào?
	\end{enumerate*}
\end{baitoan}

\begin{baitoan}[\cite{Binh_Toan_6_tap_1}, Ví dụ 63${}^\star$, p. 52]
	Tìm $n\in\mathbb{N}$ lớn nhất để tích các số tự nhiên từ $1$ đến $1000$ chia hết cho $5^n$.
\end{baitoan}
``\textit{Tổng quát}: Số thừa số $a$ khi phân tích $n! = \prod_{i=1}^n i = 1\cdot 2\cdot 3\cdots n$ ra thừa số nguyên tố là: $\sum_{i=1}^k \lfloor\frac{n}{a^i}\rfloor = \lfloor\frac{n}{a}\rfloor + \lfloor\frac{n}{a^2}\rfloor + \cdots + \lfloor\frac{n}{a^k}\rfloor$ với $k$ là số mũ lớn nhất sao cho $a^k\le n$. Ký hiệu $\lfloor\frac{n}{m}\rfloor$ là số tự nhiên lớn nhất không vượt quá $\frac{n}{m}$ (nếu $n\divby m$ thì $\lfloor\frac{n}{m}\rfloor$ là thương đúng, nếu $n\not\,\divby m$ thì $\lfloor\frac{n}{m}\rfloor$ là thương hụt, ta gọi $\lfloor\frac{n}{m}\rfloor$ là \emph{phần nguyên} của $\frac{n}{m}$).'' -- \cite[p. 53]{Binh_Toan_6_tap_1}

\begin{baitoan}[\cite{Binh_Toan_6_tap_1}, Ví dụ 64, p. 53]
	Có bao nhiêu số tự nhiên chia hết cho $13$ trong dãy $111,1111,\ldots,\underbrace{1\ldots 1}_{\scriptsize1993\mbox{ số}}$.
\end{baitoan}

\subsection{Các dãy khác}

\begin{baitoan}[\cite{Binh_Toan_6_tap_1}, Ví dụ 65, p. 53]
	Tìm số hạng thứ $100$ của các dãy được viết theo quy luật:
	\begin{enumerate*}
		\item[(a)] $3,8,15,24,35,\ldots$;
		\item[(b)] $3,24,63,120,195,\ldots$;
		\item[(c)] $1,3,6,10,15,\ldots$;
		\item[(d)] $1,2,4,7,11$;
		\item[(e)] $2,5,10,17,26,\ldots$
	\end{enumerate*}
\end{baitoan}
\noindent\textit{Hint.} 2 số hạng đầu của các dãy trên có thể viết dưới dạng: dãy (a): $1\cdot3,2\cdot4$; dãy (b): $1\cdot3,4\cdot6$; dãy (c): $\frac{1\cdot2}{2},\frac{2\cdot3}{2}$; (c) dãy (e): $1 + 1^2,1 + 2^2$.

\begin{baitoan}[\cite{Binh_Toan_6_tap_1}, Ví dụ 66, p. 54]
	\begin{enumerate*}
		\item[(a)] Tính tổng $A = \sum_{i=1}^{98} i(i + 1) = 1\cdot2 + 2\cdot3 + 3\cdot4 + \cdots + 98\cdot99$.
		\item[(b)] Sử dụng kết quả của (a), tính $B = \sum_{i=1}^{98} i^2 = 1^2 + 2^2 + 3^2 + \cdots + 97^2 + 98^2$.
	\end{enumerate*}
\end{baitoan}
Tổng quát:
\begin{align*}
	\sum_{i=1}^n i^2 = 1^2 + 2^2 + 3^2 + \cdots + n^2 = \frac{n(n + 1)(n + 2)}{3} - \frac{n(n + 1)}{2} = \frac{n(n + 1)(2n + 1)}{6},\ \forall n\in\mathbb{N}^\star.
\end{align*}

\begin{baitoan}[\cite{Binh_Toan_6_tap_1}, \textbf{309.}, p. 55]
	Tìm chữ số thứ $1000$ khi viết liên tiếp liền nhau các số hạng của dãy số lẻ $1,3,5,7,\ldots$
\end{baitoan}

\begin{baitoan}[\cite{Binh_Toan_6_tap_1}, \textbf{310.}, p. 55]
	\begin{enumerate*}
		\item[(a)] Tính tổng các số lẻ có 2 chữ số.
		\item[(b)] Tính tổng các số chẵn có 2 chữ số.
	\end{enumerate*}
\end{baitoan}

\begin{baitoan}[\cite{Binh_Toan_6_tap_1}, \textbf{311.}, p. 55]
	Có số hạng nào của dãy sau tận cùng bằng $2$ hay không? $1, 1 + 2, 1 + 2 + 3, 1 + 2 + 3 + 4,\ldots$
\end{baitoan}

\begin{baitoan}[\cite{Binh_Toan_6_tap_1}, \textbf{312.}, p. 55]
	\begin{enumerate*}
		\item[(a)] Viết liên tiếp các số hạng của dãy số tự nhiên từ $1$ đến $100$ tạo thành 1 số $A$. Tính tổng các chữ số của $A$.
		\item[(b)] Cũng hỏi như trên nếu viết từ $1$ đến $1000000$.
	\end{enumerate*}
\end{baitoan}

\begin{baitoan}[\cite{Binh_Toan_6_tap_1}, \textbf{313.}, p. 55]
	Có $n$ em bé được nhận quà. Cô giáo đã xếp cho các em đứng thành 1 hàng ngang, có số tuổi nhỏ dần kể từ trái sang phải. Lần lượt từ trái sang phải em thứ nhất được $1$ chiếc, em thứ 2 được $2$ chiếc, cứ như vậy em nhận sau được chia nhiều hơn em nhận trước $1$ chiếc kẹo. Đến lượt chia thứ 2, cô giao cũng chia kẹo từ trái sang phải sao cho em nhận sau được chia nhiều hơn em nhận trước $1$ chiếc kẹo (lưu ý: ở lượt thứ 2 thì em thứ nhất nhận không phải $1$ chiếc kẹo mà là $n + 1$ chiếc kẹo). Tính $n$ biết số kẹo chia ở lượt thứ 2 nhiều hơn số kẹo chia ở lượt thứ nhất là $36$ chiếc. 
\end{baitoan}

\begin{baitoan}[\cite{Binh_Toan_6_tap_1}, \textbf{314.}, p. 55]
	Bạn A viết dãy số tự nhiên như sau: $3,4,5,\ldots,345$ (1). Bạn B thay mỗi số của dãy (1) bởi tổng các chữ số của nó \& được dãy (2). Bạn C thay mỗi số của dãy (2) bởi tổng các chữ số của nó \& được dãy (3). Bạn D thay mỗi số của dãy (3) bởi tổng các chữ số của nó \& được dãy (4).
	\begin{enumerate*}
		\item[(a)] Chứng tỏ chỉ có dãy (4) mới có mọi số hạng đều là số có 1 chữ số.
		\item[(b)] Số nào xuất hiện nhiều nhất ở dãy (4)?
	\end{enumerate*}
\end{baitoan}

\begin{baitoan}[\cite{Binh_Toan_6_tap_1}, \textbf{315.}, p. 55]
	\begin{enumerate*}
		\item[(a)] Khi phân tích ra thừa số nguyên tố, số $1000!$ chứa thừa số nguyên tố $7$ với số mũ bằng bao nhiêu?
		\item[(b)] Tích $A = 500!$ tận cùng bằng bao nhiêu chữ số $0$?
	\end{enumerate*}	
\end{baitoan}

\begin{baitoan}[\cite{Binh_Toan_6_tap_1}, \textbf{316.}, p. 55]
	\begin{enumerate*}
		\item[(a)] Tích $B = 38\cdot 39\cdot 40\cdots 74$ có bao nhiêu thừa số $2$ khi phân tích ra thừa số nguyên tố?
		\item[(b)] Tích $C = 31\cdot 32\cdot 33\cdots 90$ có bao nhiêu thừa số $3$ khi phân tích ra thừa số nguyên tố?
	\end{enumerate*}
\end{baitoan}

\begin{baitoan}[\cite{Binh_Toan_6_tap_1}, \textbf{317.}, p. 55]
	2 con châu chấu cùng nhảy 1 lúc từ 1 chỗ \& về cùng 1 phía. Khi con I nhảy 1 bước thì con II cũng nhảy 1 bước. Con I nhảy mỗi bước dài $4$\emph{m}. Con II nhảy bước thứ nhất dài $1$\emph{m}, mỗi bước sau tăng hơn so với bước liền trước $1$\emph{m} cho đến khi đuổi kịp con I. Hỏi sau bao nhiêu bước nhảy thì con II đuổi kịp con I?
\end{baitoan}

\begin{baitoan}[\cite{Binh_Toan_6_tap_1}, \textbf{318.}, p. 56]
	Cho 3 dãy các số tự nhiên liên tiếp: $1,2,3,\ldots,95,96,97$; $1,2,3,\dots,95,96$; $1,2,3,\ldots,95$. Trong dãy nào có thể chia các số của dãy thành 2 nhóm để tổng các số trong mỗi nhóm bằng nhau?
\end{baitoan}

\begin{baitoan}[\cite{Binh_Toan_6_tap_1}, \textbf{319.}, p. 56]
	\begin{enumerate*}
		\item[(a)] Viết số hạng thứ $n$ của dãy $1,4,7,10,13,\ldots$;
		\item[(b)] Viết 2 số hạng tiếp theo của dãy $1,3,2,6,3,9,4,12,5,\ldots$;
		\item[(c)] Viết số hạng thứ $n$ của dãy $1,2,4,7,11,\ldots$
	\end{enumerate*}
\end{baitoan}

\begin{baitoan}[\cite{Binh_Toan_6_tap_1}, \textbf{320.}, p. 56]
	Có bao nhiêu số tự nhiên đồng thời là các số hạng của cả 2 dãy sau: $3,7,11,15,\ldots,407$ \& $2,9,16,23,\ldots,709$.
\end{baitoan}

\begin{baitoan}[\cite{Binh_Toan_6_tap_1}, \textbf{321.}, p. 56]
	Cho 1 dãy gồm $30$ số chẵn liên tiếp tăng dần có tổng bằng $1470$. Tìm số hạng đầu \& số hạng cuối của dãy.
\end{baitoan}

\begin{baitoan}[\cite{Binh_Toan_6_tap_1}, \textbf{322.}, p. 56]
	\begin{enumerate*}
		\item[(a)] Tính tổng của $n$ số lẻ liên tiếp bắt đầu từ $1$.s
		\item[(b)] Xếp các hộp thàng hàng, hàng thứ nhất có $1$ hộp, hàng thứ 2 có $3$ hộp, hàng thứ 3 có $5$ hộp, etc., sao cho các hộp ở giữa mỗi hàng tạo thành 1 cột. Có tất cả bao nhiêu hộp từ hàng thứ nhất tới hàng thứ $10$? Có tất cả bao nhiêu  hộp từ hàng thứ nhất tới hàng thứ $n$, $n\in\mathbb{N}^\star$?
	\end{enumerate*}
\end{baitoan}

\begin{baitoan}[\cite{Binh_Toan_6_tap_1}, \textbf{323.}, p. 56]
	Trong dãy số $1,2,3,\ldots,1990$, có thể chọn được nhiều nhất bao nhiêu số để tổng 2 số bất kỳ được chọn chia hết cho $38$?
\end{baitoan}

\begin{baitoan}[\cite{Binh_Toan_6_tap_1}, \textbf{324.}, p. 56]
	1 đồng hồ reo chuông vào các thời điểm sau: 4:10, 5:20, 6:40, 8:10, $\ldots$ Theo quy luật trên, đồng hồ reo chuông lần tiếp theo vào lúc nào?
\end{baitoan}

\begin{baitoan}[\cite{Binh_Toan_6_tap_1}, \textbf{325.}${}^\star$, p. 56, Theo nội dung bài toán bò ăn cở của Newton]
	Có 3 cánh đồng cỏ như nhau \& cỏ luôn mọc đều như nhau trên toàn bộ cánh đồng. $9$ con bò ăn hết số cỏ có sẵn \& số cỏ mọc thêm của cánh đồng I trong $2$ tuần, $6$ con bò ăn hết số cỏ có sẵn \& số cỏ mọc thêm của cánh đồng II trong $4$ tuần. Hỏi bao nhiêu con bò ăn hết cỏ có sẵn \& số cỏ mọc thêm của cánh đồng II trong $6$ tuần? (mỗi con bò đều ăn số cỏ như nhau).
\end{baitoan}

\begin{baitoan}[\cite{Binh_Toan_6_tap_1}, \textbf{326.}${}^\star$, p. 56]
	Chia dãy số tự nhiên kể từ $1$ thành từng nhóm (các số cùng nhóm được đặt trong dấu ngoặc) $(1),(2,3),(4,5,6),(7,8,9,10),(11,12,13,14,15),\ldots$
	\begin{enumerate*}
		\item[(a)] Tìm số hạng đầu tiên của nhóm thứ $100$.
		\item[(b)] Tính tổng các số thuộc nhóm thứ $100$.
	\end{enumerate*}
\end{baitoan}

\begin{baitoan}[\cite{Binh_Toan_6_tap_1}, \textbf{327.}, p. 56]
	Cho $S_1 = 1 + 2$, $S_2 = 3 + 4 + 5$, $S_3 = 6 + 7 + 8 + 9$, $S_4 = 10 + 11 + 12 + 13 + 14,\ldots$ Tính $S_{100}$.
\end{baitoan}
Bài tập phụ thuộc vào hình vẽ: \cite[\textbf{328.}--\textbf{331.}, p. 57]{Binh_Toan_6_tap_1}.

\texttt{pause here ...}

\begin{baitoan}[\cite{Binh_Toan_6_tap_1}, \textbf{230.}, p. 49]
	Tính số hạng thứ $50$ của các dãy sau:
	\begin{enumerate*}
		\item[(a)] $1\cdot 6,2\cdot 7,3\cdot 8,\ldots$;
		\item[(b)] $1\cdot 4,4\cdot 7,7\cdot 10,\ldots$
	\end{enumerate*}
\end{baitoan}

\begin{baitoan}[\cite{Binh_Toan_6_tap_1}, \textbf{231.}, p. 49]
	Cho $A = 1 + 3 + 3^2 + 3^3 + \cdots + 3^{20} = \sum_{i=0}^{20} 3^i$, $B = 3^{21}:2$. Tính $B - A$.
\end{baitoan}

\begin{baitoan}[\cite{Binh_Toan_6_tap_1}, \textbf{232.}, p. 49]
	Cho $A = 1 + 4 + 4^2 + 4^3 + \cdots + 4^{99}$, $B = 4^{100}$. Chứng minh rằng $A < \frac{B}{3}$.
\end{baitoan}

\begin{baitoan}[\cite{Binh_Toan_6_tap_1}, \textbf{233.}, p. 49]
	Tính giá trị của biểu thức:
	
	\begin{enumerate*}
		\item[(a)] $A = 9 + 99 + 999 + \cdots + \underbrace{9\ldots 9}_{50's}$;
		\item[(b)] $B = 9 + 99 + 999 + \cdots + \underbrace{9\ldots 9}_{200's}$.
	\end{enumerate*}
\end{baitoan}

\subsection{Đếm số}

\begin{baitoan}[\cite{Binh_Toan_6_tap_1}, Ví dụ 43, p. 49]
	Có bao nhiêu số $\overline{abcd}$ mà $\overline{ab} < \overline{cd}$?
\end{baitoan}

\begin{baitoan}[\cite{Binh_Toan_6_tap_1}, Ví dụ 44, p. 49]
	Có bao nhiêu số tự nhiên chia hết cho $4$ gồm 4 chữ số, chữ số tận cùng bằng $2$?
\end{baitoan}

\begin{luuy}
	``Nếu việc chọn đối tượng $A$ có thể thực hiện bởi $m$ cách \& với mỗi cách chọn của $A$ có thể chọn đối tượng $B$ bởi $n$ cách thì việc chọn $A$ \& $B$ theo thứ tự đó có thể thực hiện bởi $mn$ cách chọn.'' -- \cite[p. 50]{Binh_Toan_6_tap_1} \emph{Quy tắc nhân trong phép đếm} \& khái niệm \emph{tổ hợp, chỉnh hợp} sẽ được học ở môn Tổ hợp, trong chương trình Toán 10.
\end{luuy}

\begin{baitoan}[\cite{Binh_Toan_6_tap_1}, Ví dụ 45, p. 50]
	Có bao nhiêu số tự nhiên có $4$ chữ số $\overline{abcd}$, trong đó $b - a = 1$, $d - c = 1$?
\end{baitoan}

\begin{baitoan}[\cite{Binh_Toan_6_tap_1}, Ví dụ 46, p. 50]
	Có bao nhiêu số tự nhiên có 3 chữ số trong đó có đúng 1 chữ số $5$?
\end{baitoan}
``Trong nhiều trường hợp, để đếm các số có tính chất nào đó, ta lại đếm trước hết các số không có tính chất ấy.'' -- \cite[p. 51]{Binh_Toan_6_tap_1}

\begin{baitoan}[\cite{Binh_Toan_6_tap_1}, Ví dụ 47, p. 50]
	Có bao nhiêu số chứa ít nhất 1 chữ số $1$ trong các số tự nhiên:
	\begin{enumerate*}
		\item[(a)] có 3 chữ số;
		\item[(b)] từ $1$ đến $999$.
	\end{enumerate*}
\end{baitoan}

\begin{baitoan}[\cite{Binh_Toan_6_tap_1}, Ví dụ 48, p. 51]
	Viết $999$ số tự nhiên liên tiếp kể từ $1$. Hỏi:
	\begin{enumerate*}
		\item[(a)] Chữ số $2$ có mặt bao nhiêu lần?
		\item[(b)] Chữ số $0$ có mặt bao nhiêu lần?
	\end{enumerate*}
\end{baitoan}

\begin{baitoan}[\cite{Binh_Toan_6_tap_1}, \textbf{234.}, p. 52]
	Bạn Tâm đánh số trang của 1 cuốn vở có $110$ trang bằng cách viết dãy số tự nhiên $1,2,\ldots,110$. Bạn Tâm phải viết tất cả bao nhiêu chữ số?
\end{baitoan}

\begin{baitoan}[\cite{Binh_Toan_6_tap_1}, \textbf{235.}, p. 52]
	1 cô nhân viên đánh máy liên tục dãy số chẵn bắt đầu từ $2$: $2,4,6,8,10,12,\ldots$ Cô phải đánh tất cả $2000$ chữ số. Tìm chữ số cuối cùng mà cô đã đánh.
\end{baitoan}

\begin{baitoan}[\cite{Binh_Toan_6_tap_1}, \textbf{236.}, p. 52]
	Bạn Mai viết dãy số lẻ $1,3,5,\ldots,245$.
	\begin{enumerate*}
		\item[(a)] Bạn Mai phải viết tất cả bao nhiêu chữ số?
		\item[(b)] Nếu mỗi chữ số viết mất 1 giây thì viết đến số $245$ mất bao nhiêu giây? Sau $5$ phút, bạn Mai viết đến chữ số nào?
	\end{enumerate*}
\end{baitoan}

\begin{baitoan}[\cite{Binh_Toan_6_tap_1}, \textbf{237.}, p. 52]
	Bạn Hùng viết dãy số lẻ $1,3,5,7,\ldots$ để đánh số trang 1 cuốn sách. Tính xem chữ số $200$ mà bạn Hùng viết là chữ số nào?
\end{baitoan}

\begin{baitoan}[\cite{Binh_Toan_6_tap_1}, \textbf{238.}, p. 52]
	Để đánh số trang của 1 cuốn sách, người ta viết dãy số tự nhiên bắt đầu từ $1$ \& phải dùng tất cả $1998$ chữ số.
	\begin{enumerate*}
		\item[(a)] Hỏi cuốn sách có bao nhiêu trang?
		\item[(b)] Chữ số thứ $1010$ là chữ số nào?
	\end{enumerate*}
\end{baitoan}

\begin{baitoan}[\cite{Binh_Toan_6_tap_1}, \textbf{239.}, p. 52]
	Có bao nhiêu số tự nhiên chia hết cho $3$, có 4 chữ số \& tận cùng bằng $5$?
\end{baitoan}

\begin{baitoan}[\cite{Binh_Toan_6_tap_1}, \textbf{240.}, pp. 52--53]
	Tuấn muốn đến nhà bạn, nhưng không nhớ số nhà, chỉ biết rằng số nhà của bạn là số chia hết cho $3$ \& có 2 chữ số. Biết số nhà cuối của dãy phố đó là $135$. Hỏi Tuấn phải gõ cửa nhiều nhất bao nhiêu số nhà? (các số nhà không đánh số $a,b,\ldots$).
\end{baitoan}

\begin{baitoan}[\cite{Binh_Toan_6_tap_1}, \textbf{241.}, p. 53]
	Tìm số lượng các số tự nhiên có 4 chữ số mà:
	\begin{enumerate*}
		\item[(a)] Số tạo bởi 2 chữ số đầu (theo thứ tự ấy) cộng với số tạo bởi 2 chữ số cuối (theo thứ tự ấy) nhỏ hơn $100$.
		\item[(b)] Số tạo bởi 2 chữ số đầu (theo thứ tự ấy) lớn hơn số tạo bởi 2 chữ số cuối (theo thứ tự ấy)?
	\end{enumerate*}
\end{baitoan}

\begin{baitoan}[\cite{Binh_Toan_6_tap_1}, \textbf{242.}, p. 53]
	Trong các số tự nhiên từ $1$ đến $252$, xóa các số chia hết cho $2$ nhưng không chia hết cho $5$, rồi xóa các số chia hết cho $5$ nhưng không chia hết cho $2$. Còn lại bao nhiêu số?
\end{baitoan}

\begin{baitoan}[\cite{Binh_Toan_6_tap_1}, \textbf{243.}, p. 53]
	Có bao nhiêu số tự nhiên có 3 chữ số mà:
	\begin{enumerate*}
		\item[(a)] Các chữ số đều chẵn?
		\item[(b)] Chữ số hàng chục là chữ số lẻ?
	\end{enumerate*}
\end{baitoan}

\begin{baitoan}[\cite{Binh_Toan_6_tap_1}, \textbf{244.}, p. 53]
	Có bao nhiêu số tự nhiên có 4 chữ số mà:
	\begin{enumerate*}
		\item[(a)] Mỗi chữ số đều chẵn?
		\item[(b)] Tổng các chữ số là số chẵn?
	\end{enumerate*}
\end{baitoan}

\begin{baitoan}[\cite{Binh_Toan_6_tap_1}, \textbf{245.}, p. 53]
	Có bao nhiêu biển số xe máy khác nhau, mỗi số xe lập bởi 2 chữ cái đứng đầu \& 3 chữ số đứng sau? (bảng chữ cái có $25$ chữ, không có biển số $000$).
\end{baitoan}

\begin{baitoan}[\cite{Binh_Toan_6_tap_1}, \textbf{246.}, p. 53]
	Trong các số tự nhiên có 3 chữ số, có bao nhiêu số:
	\begin{enumerate*}
		\item[(a)] Chứa đúng 1 chữ số $4$?
		\item[(b)] Chứa đúng 2 chữ số $4$?
	\end{enumerate*}
\end{baitoan}

\begin{baitoan}[\cite{Binh_Toan_6_tap_1}, \textbf{247.}, p. 53]
	Có bao nhiêu số tự nhiên chia hết cho $5$, có 4 chữ số, có đúng 1 chữ số $5$?
\end{baitoan}

\begin{baitoan}[\cite{Binh_Toan_6_tap_1}, \textbf{248.}, p. 53]
	Có bao nhiêu số tự nhiên có 3 chữ số, biết rằng cộng nó với số gồm 3 chữ số ấy viết theo thứ tự ngược lại thì được 1 số chia hết cho $5$?
\end{baitoan}

\begin{baitoan}[\cite{Binh_Toan_6_tap_1}, \textbf{249.}, p. 53]
	Có bao nhiêu số chẵn có 3 chữ số, các chữ số khác nhau?
\end{baitoan}

\begin{baitoan}[\cite{Binh_Toan_6_tap_1}, \textbf{250.}, p. 53]
	Có bao nhiêu số tự nhiên có 3 chữ số trong đó có ít nhất 2 chữ số như nhau?
\end{baitoan}

\begin{baitoan}[\cite{Binh_Toan_6_tap_1}, \textbf{251.}, p. 53]
	Trong các số tự nhiên có 4 chữ số, có bao nhiêu số trong đó có đúng 3 chữ số như nhau?
\end{baitoan}

\begin{baitoan}[\cite{Binh_Toan_6_tap_1}, \textbf{252.}, p. 53]
	Trong các số tự nhiên có 3 chữ số, có bao nhiêu số:
	\begin{enumerate*}
		\item[(a)] Chia hết cho $5$, có chứa chữ số $5$?
		\item[(b)] Chia hết cho $4$, có chứa chữ số $4$?
		\item[(c)] Chia hết cho $3$, không chứa chữ số $3$?
	\end{enumerate*}
\end{baitoan}

\begin{baitoan}[\cite{Binh_Toan_6_tap_1}, \textbf{253.}, p. 54]
	Viết liên tiếp các số tự nhiên từ $1$ đến $999$ ta được 1 số tự nhiên $A$.
	\begin{enumerate*}
		\item[(a)] Số $A$ có bao nhiêu chữ số?
		\item[(b)] Tính tổng các chữ số của số $A$.
	\end{enumerate*}
\end{baitoan}

\begin{baitoan}[\cite{Binh_Toan_6_tap_1}, $\bf 254^\star.$, p. 54]
	Viết dãy số tự nhiên từ $1$ đến $999$.
	\begin{enumerate*}
		\item[(a)] Chữ số $1$ được viết bao nhiêu lần?
		\item[(b)] Chữ số $0$ được viết bao nhiêu lần?
	\end{enumerate*}
\end{baitoan}

\begin{baitoan}[\cite{Binh_Toan_6_tap_1}, \textbf{255.}, p. 54]
	Trong các số tự nhiên có 3 chữ số, có bao nhiêu số chứa ít nhất 1 chữ số $4$?
\end{baitoan}

\begin{baitoan}[\cite{Binh_Toan_6_tap_1}, $\bf 256^\star.$, p. 54]
	Trong các số tự nhiên từ $1$ đến $10000$:
	\begin{enumerate*}
		\item[(a)] Có bao nhiêu số chứa chữ số $0$?
		\item[(b)] Số chứa chữ số $1$ hay số không chứa chữ số $1$ có nhiều hơn?
	\end{enumerate*}
\end{baitoan}

\begin{baitoan}[\cite{Binh_Toan_6_tap_1}, \textbf{257.}, p. 54]
	Viết dãy số chẵn $100,102,\ldots,390$. Hỏi chữ số $2$ được viết bao nhiêu lần?
\end{baitoan}

\begin{baitoan}[\cite{Binh_Toan_6_tap_1}, \textbf{258.}, p. 54]
	Từ các chữ số $1,2,3,4,5,6,7$, lập tất cả các số tự nhiên có 7 chữ số trong đó mỗi chữ số trên đều có mặt. Chứng minh rằng tổng tất cả các số đó chia hết cho $9$.
\end{baitoan}

\begin{baitoan}[\cite{Binh_Toan_6_tap_1}, \textbf{259.}, p. 54]
	Cho 3 chữ số $a,b,c$ khác nhau \& khác $0$. Gọi $A$ là tập hợp các số tự nhiên có 3 chữ số lập bởi cả 3 chữ số trên.
	\begin{enumerate*}
		\item[(a)] Tập hợp $A$ có bao nhiêu phần tử?
		\item[(b)] Tính tổng các phần tử của tập hợp $A$, biết rằng $a + b + c = 17$.
	\end{enumerate*}
\end{baitoan}

\begin{baitoan}[\cite{Binh_Toan_6_tap_1}, \textbf{260.}, p. 54]
	Từ các chữ số $1,2,3,4$, lập tất cả các số tự nhiên mà mỗi chữ số trên đều có mặt đúng 1 lần. Tìm tổng các số ấy.
\end{baitoan}

\begin{baitoan}[\cite{Binh_Toan_6_tap_1}, \textbf{261.}, p. 54]
	Tìm tổng các số tự nhiên có 3 chữ số lập bởi các chữ số $2,3,0,7$ trong đó:
	\begin{enumerate*}
		\item[(a)] Các chữ số có thể giống nhau;
		\item[(b)] Các chữ số đều khác nhau.
	\end{enumerate*}
\end{baitoan}

%------------------------------------------------------------------------------%

\printbibliography[heading=bibintoc]
	
\end{document}