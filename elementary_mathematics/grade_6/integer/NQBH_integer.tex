\documentclass{article}
\usepackage[backend=biber,natbib=true,style=authoryear,maxbibnames=99]{biblatex}
\addbibresource{/home/nqbh/reference/bib.bib}
\usepackage[utf8]{vietnam}
\usepackage{tocloft}
\renewcommand{\cftsecleader}{\cftdotfill{\cftdotsep}}
\usepackage[colorlinks=true,linkcolor=blue,urlcolor=red,citecolor=magenta]{hyperref}
\usepackage{amsmath,amssymb,amsthm,mathtools,float,graphicx,algpseudocode,algorithm,tcolorbox}
\usepackage[inline]{enumitem}
\allowdisplaybreaks
\numberwithin{equation}{section}
\newtheorem{assumption}{Assumption}[section]
\newtheorem{baitoan}{Bài toán}
\newtheorem{cauhoi}{Câu hỏi}[section]
\newtheorem{conjecture}{Conjecture}[section]
\newtheorem{corollary}{Corollary}[section]
\newtheorem{definition}{Definition}[section]
\newtheorem{dinhly}{Định lý}[section]
\newtheorem{dinhnghia}{Định nghĩa}[section]
\newtheorem{example}{Example}[section]
\newtheorem{hequa}{Hệ quả}[section]
\newtheorem{lemma}{Lemma}[section]
\newtheorem{luuy}{Lưu ý}[section]
\newtheorem{nhanxet}{Nhận xét}[section]
\newtheorem{notation}{Notation}[section]
\newtheorem{principle}{Principle}[section]
\newtheorem{problem}{Problem}[section]
\newtheorem{proposition}{Proposition}[section]
\newtheorem{question}{Question}[section]
\newtheorem{remark}{Remark}[section]
\newtheorem{theorem}{Theorem}[section]
\newtheorem{tip}{Tip}[section]
\newtheorem{vidu}{Ví dụ}[section]
\usepackage[left=0.5in,right=0.5in,top=1.5cm,bottom=1.5cm]{geometry}
\usepackage{fancyhdr}
\pagestyle{fancy}
\fancyhf{}
\lhead{\small Sect.~\thesection}
\rhead{\small\nouppercase{\leftmark}}
\renewcommand{\subsectionmark}[1]{\markboth{#1}{}}
\cfoot{\thepage}
\def\labelitemii{$\circ$}
\DeclareRobustCommand{\divby}{%
	\mathrel{\vbox{\baselineskip.65ex\lineskiplimit0pt\hbox{.}\hbox{.}\hbox{.}}}%
}

\title{Integer -- Số Nguyên $\mathbb{Z}$}
\author{Nguyễn Quản Bá Hồng\footnote{Independent Researcher, Ben Tre City, Vietnam\\e-mail: \texttt{nguyenquanbahong@gmail.com}; website: \url{https://nqbh.github.io}.}}
\date{\today}

\begin{document}
\maketitle
\begin{abstract}
	\textsc{[en]} This text is a collection of problems, from easy to advanced, about integer. This text is also a supplementary material for my lecture note on Elementary Mathematics grade 6, which is stored \& downloadable at the following link: \href{https://github.com/NQBH/hobby/blob/master/elementary_mathematics/grade_6/NQBH_elementary_mathematics_grade_6.pdf}{GitHub\texttt{/}NQBH\texttt{/}hobby\texttt{/}elementary mathematics\texttt{/}grade 6\texttt{/}lecture}\footnote{\textsc{url}: \url{https://github.com/NQBH/hobby/blob/master/elementary_mathematics/grade_6/NQBH_elementary_mathematics_grade_6.pdf}.}. The latest version of this text has been stored \& downloadable at the following link: \href{https://github.com/NQBH/hobby/blob/master/elementary_mathematics/grade_6/integer/NQBH_integer.pdf}{GitHub\texttt{/}NQBH\texttt{/}hobby\texttt{/}elementary mathematics\texttt{/}grade 6\texttt{/}integer $\mathbb{Z}$}\footnote{\textsc{url}: \url{https://github.com/NQBH/hobby/blob/master/elementary_mathematics/grade_6/integer/NQBH_integer.pdf}.}.
	\vspace{2mm}
	
	\textsc{[vi]} Tài liệu này là 1 bộ sưu tập các bài tập chọn lọc từ cơ bản đến nâng cao về số nguyên. Tài liệu này là phần bài tập bổ sung cho tài liệu chính -- bài giảng \href{https://github.com/NQBH/hobby/blob/master/elementary_mathematics/grade_6/NQBH_elementary_mathematics_grade_6.pdf}{GitHub\texttt{/}NQBH\texttt{/}hobby\texttt{/}elementary mathematics\texttt{/}grade 6\texttt{/}lecture} của tác giả viết cho Toán Sơ Cấp lớp 6. Phiên bản mới nhất của tài liệu này được lưu trữ \& có thể tải xuống ở link sau: \href{https://github.com/NQBH/hobby/blob/master/elementary_mathematics/grade_6/integer/NQBH_integer.pdf}{GitHub\texttt{/}NQBH\texttt{/}hobby\texttt{/}elementary mathematics\texttt{/}grade 6\texttt{/}integer $\mathbb{Z}$}.
\end{abstract}
\tableofcontents
\newpage

%------------------------------------------------------------------------------%

\section{Tập Hợp $\mathbb{Z}$ Các Số Nguyên. Thứ Tự Trên $\mathbb{Z}$}
``\textbf{1.} Tập hợp $\{\ldots,-3,-2,-1,0,1,2,3,\ldots\}$ gồm có số $0$, các số $1,2,3,\ldots$ (\textit{số nguyên dương}) \& các số $-1,-2,-3,\ldots$ (\textit{số nguyên âm}) được gọi là \textit{tập hợp các số nguyên}, ký hiệu là $\mathbb{Z}$. \textbf{2.} \textit{Biểu diễn trên trục số}: Điểm biểu diễn số nguyên $a\in\mathbb{Z}$ được gọi là \textit{điểm $a$}. \textbf{3.} \textit{Số đối}: Cho $a\in\mathbb{Z}$, trên trục số, 2 điểm $a$ \& $-a$ (viết gom lại là $\pm a$) cách đều điểm gốc $0$. Ta nói $a$ \& $-a$ là \textit{2 số đối nhau}. Tính chất: $a + (-a) = (-a) + a = a - a = 0$, $|a| = |-a|$, $\forall a\in\mathbb{Z}$. \textbf{4.} \textit{Thứ tự trong $\mathbb{Z}$}: Trên trục số, điểm $a$ nằm bên trái điểm $b$ thì $a < b$ hay $b > a$. Suy ra: số nguyên âm $< 0 < $ số nguyên dương, i.e., $-a < 0 < b$, $\forall a,b\in\mathbb{N}^\star$. Có $a > b\Leftrightarrow-a < -b$, $a = b\Leftrightarrow-a = -b$, $\forall a,b\in\mathbb{N}^\star$.'' -- \cite[Chap. II, \S1, p. 35]{Tuyen_Toan_6} 

Ký hiệu ``or''\texttt{/}``hoặc'', ký hiệu ``and''\texttt{/}``\&'':
\begin{equation*}
	\left[\begin{split}
		A\\B
	\end{split}\right.\mbox{ hay } A\lor B\mbox{ nghĩa là $A$ hoặc $B$};\ \left\{\begin{split}
		A\\B
	\end{split}\right.\mbox{ hay } A\land B\mbox{ nghĩa là $A$ \& $B$}.
\end{equation*}
E.g., các mệnh đề chứa phép so sánh:
\begin{equation*}
	x > a \mbox{ hoặc } x < b\Leftrightarrow x > a\lor x < b\Leftrightarrow\left[\begin{split}
		x > a\\x < b
	\end{split}\right.;\ x\ge a \mbox{ hoặc } x < b\Leftrightarrow x\ge a\lor x < b\Leftrightarrow\left[\begin{split}
		x\ge a\\x < b
	\end{split}\right.;
\end{equation*}

\begin{equation*}
	x > a \mbox{ hoặc } x\le b\Leftrightarrow x > a\lor x\le b\Leftrightarrow\left[\begin{split}
		x > a\\x\le b
	\end{split}\right.;\ x\ge a \mbox{ hoặc } x\le b\Leftrightarrow x\ge a\lor x\le b\Leftrightarrow\left[\begin{split}
		x\ge a\\x\le b
	\end{split}\right..
\end{equation*}
``Tập hợp $\mathbb{Z}$ các số nguyên gồm các số tự nhiên \& các số $-1,-2,-3,\ldots$ $\mathbb{Z} = \{\ldots,-3,-2,-1,0,1,2,3,\ldots\}$. Ta xác định trên $\mathbb{Z}$ 1 thứ tự như sau: $a < b$ khi \& chỉ khi điểm $a$ ở bên trái điểm $b$ trên trục số ($a,b\in\mathbb{Z}$). Ta xác định trên $\mathbb{Z}$ 2 phép toán: phép cộng \& phép nhân. Phép cộng có 4 tính chất: giao hoán, kết hợp, cộng với số $0$, cộng với số đối. Phép nhân có 3 tính chất: giao hoán, kết hợp, nhân với số $1$. Giữa phép nhân \& phép cộng có quan hệ: phép nhân phân phối đối với phép cộng. Giữa thứ tự \& phép toán có quan hệ: $a < b\Rightarrow a + c < b + c$, $a < b\Rightarrow ac < bc$ với $c > 0$, $ac > bc$ với $c < 0$. Trừ đi 1 số là cộng với số đối của số trừ. Phép trừ 2 số nguyên bao giờ cũng thực hiện được\footnote{Phép trừ 2 số tự nhiên sẽ không thực hiện được (i.e., kết quả không phải là 1 số tự nhiên, hay không còn nằm trong $\mathbb{N}$) nếu số bị trừ nhỏ hơn số trừ.}. Phép chia chỉ thực hiện được trong phạm vi số nguyên khi số bị chia chia hết cho số chia. Trong trường hợp $a\divby b$, ta nói: $a$ là \textit{bội} của $b$ \& $b$ là \textit{ước} của $a$. \textit{Ước chung} (hoặc \textit{bội chung}) của 2 hay nhiều số là ước (hoặc bội) của tất cả các số đó.'' -- \cite[Chap. II, p. 41]{Binh_Toan_6_tap_1}

\begin{baitoan}[\cite{Tuyen_Toan_6}, Ví dụ 38, p. 35]
	\label{Ví dụ 38, p. 35}
	Viết tập hợp 3 số nguyên liên tiếp trong đó có số $0$.
\end{baitoan}
\noindent\textit{Phân tích.} Có 3 vị trí cho số $0$: đầu, giữa, cuối.

\begin{proof}[Giải]
	Tập hợp 3 số số nguyên liên tiếp trong đó có số $0$: $\{0,1,2\},\{-1,0,1\},\{-2,-1,0\}$.
\end{proof}

\begin{baitoan}[Mở rộng \cite{Tuyen_Toan_6}, Ví dụ 38, p. 35]
	\label{Mở rộng Ví dụ 38, p. 35}
	Viết tập hợp 3 số nguyên liên tiếp trong đó có số $a\in\mathbb{Z}$ cho trước.
\end{baitoan}
\noindent\textit{Phân tích.} Có 3 vị trí cho số $a$: đầu, giữa, cuối. Bài toán \ref{Ví dụ 38, p. 35} là 1 trường hợp riêng của bài toán này với $a = 0$.

\begin{proof}[Giải]
	Tập hợp 3 số số nguyên liên tiếp trong đó có số $0$: $\{a,a + 1,a + 2\},\{a - 1,a,a + 1\},\{a - 2,a - 1,a\}$.
\end{proof}

\begin{baitoan}[Mở rộng \cite{Tuyen_Toan_6}, Ví dụ 38, p. 35]
	\label{prob: Mở rộng Ví dụ 38, p. 35}
	Cho trước $n\in\mathbb{N}^\star$, $n\ge2$ \& $a\in\mathbb{Z}$. Viết tập hợp $n$ số nguyên liên tiếp trong đó có số $a$.
\end{baitoan}
\noindent\textit{Phân tích.} Có $n$ vị trí cho số $a$: 1st, 2nd, 3rd, $\ldots$, $n$th. Bài toán \ref{Ví dụ 38, p. 35} là 1 trường hợp riêng của bài toán này với $n = 3$ \& $a = 0$. Bài toán \ref{Mở rộng Ví dụ 38, p. 35} là 1 trường hợp riêng của bài toán này với $n = 3$.

\begin{proof}[Giải]
	Tập hợp $n$ số số nguyên liên tiếp trong đó có số $0$: $\{a,a + 1,\ldots,a + n - 1\},\{a - 1,a,\ldots,a + n - 2\},\ldots,\{a - n + 1,\ldots,a - 1,a\}$. (Có thể viết gọn $n$ tập hợp này lại thành 1 công thức duy nhất: $\{a - k,a - k + 1,\ldots,a - k + n - 1\}$, $\forall k = 0,1,\ldots,n - 1$.)
\end{proof}

\begin{baitoan}[\cite{Tuyen_Toan_6}, Ví dụ 39, p. 36]
	Cho 3 số nguyên khác nhau $a,b,0$. Biết $a$ là 1 số âm \& $a < b$. Sắp xếp các số đó theo thứ tự tăng dần.
\end{baitoan}

\begin{proof}[Giải]
	Nếu $b$ là số nguyên âm thì $a < b < 0$. Nếu $b$ là số nguyên dương thì $a < 0 < b$.
\end{proof}

\begin{baitoan}[Mở rộng \cite{Tuyen_Toan_6}, Ví dụ 39, p. 36]
	Cho 3 số nguyên khác nhau $a,b,0$ \& $a < b$. Sắp xếp các số đó theo thứ tự tăng dần.
\end{baitoan}

\begin{proof}[Giải]
	Nếu $a > 0$ thì $0 < a < b$. Nếu $b < 0$ thì $a < b < 0$. Nếu $a < 0$ \& $b > 0$ thì $a < 0 < b$.
\end{proof}

\begin{baitoan}[\cite{Tuyen_Toan_6}, \textbf{177.}, p. 36]
	Số nguyên âm \& số nguyên dương thường được sử dụng để biểu thị các đại lượng có 2 hướng ngược nhau. Điền cho đủ các câu sau: (a) Nếu $+8^\circ$C biểu diễn nhiệt độ $8^\circ$C trên $0^\circ$C thì $-8^\circ$C biểu diễn nhiệt độ $\ldots$. (b) Nếu $+8848$\emph{m} biểu diễn độ cao của đỉnh núi Everest là $8848$\emph{m} trên mực nước biển thì $\ldots$ biểu diễn độ sâu của thềm lục địa Việt Nam là $65$\emph{m} dưới mực nước biển. (c) Nếu $-3$ biểu diễn số tầng ngầm dưới mặt đất của 1 chung cư thì $+27$ biểu diễn $\ldots$.
\end{baitoan}

\begin{proof}[Giải]
	(a) $8^\circ$C dưới $0^\circ$C; (b) $-65$m; (c) Số tầng nhà ở trên mặt đất của chung cư đó.
\end{proof}

\begin{baitoan}[\cite{Tuyen_Toan_6}, \textbf{178.}, p. 36]
	Các suy luận sau đúng hay sai? (a) Nếu $a\in\mathbb{N}$ thì $a\in\mathbb{Z}$. (b) Nếu $a\in\mathbb{Z}$ thì $a\in\mathbb{N}$. (c) Nếu $a\notin\mathbb{Z}$ thì $a\notin\mathbb{N}$.\hfill{\sf Ans:} (a) Đ. (b) S. (c) Đ.
\end{baitoan}

\begin{baitoan}[\cite{Tuyen_Toan_6}, \textbf{179.}, p. 36]
	Trên trục số, điểm A cách gốc $2$ đơn vị về bên trái, điểm B cách A là $3$ đơn vị. Hỏi: (a) Điểm A biểu diễn số nguyên nào? (b) Điểm B biểu diễn số nguyên nào?\hfill{\sf Ans:} (a) $-2$. (b) $-5$ hoặc $1$.
\end{baitoan}

\begin{baitoan}[Mở rộng \cite{Tuyen_Toan_6}, \textbf{179.}, p. 36]
	Cho trước $a,b\in\mathbb{N}$. Trên trục số, điểm A cách gốc $a$ đơn vị về bên trái, điểm B cách A là $b$ đơn vị. Hỏi: (a) Điểm A biểu diễn số nguyên nào? (b) Điểm B biểu diễn số nguyên nào?\\\mbox{}\hfill{\sf Ans:} (a) $-a$. (b) $a + b$ hoặc $-a - b$.
\end{baitoan}

\begin{baitoan}[\cite{Tuyen_Toan_6}, \textbf{180.}, p. 36]
	Cho dãy số $15,-4,0,-76,100,99$. (a) Sắp xếp các số trong dãy theo thứ tự giảm dần. (b) Sắp xếp số đối của các số trong dãy theo thứ tự tăng dần.
\end{baitoan}

\begin{proof}[Giải]
	(a) Sắp xếp các số trong dãy theo thứ tự giảm dần: $100 > 99 > 15 > 0 > -4 > -76$. (b) Sắp xếp số đối của các số trong dãy theo thứ tự tăng dần: $-100 < -99 < -15 < 0 < 4 < 76$.
\end{proof}

\begin{baitoan}[\cite{Tuyen_Toan_6}, \textbf{181.}, p. 36]
	Viết 4 số nguyên liên tiếp trong đó có số $0$.
\end{baitoan}
Bài này là 1 trường hợp riêng của Bài toán \ref{prob: Mở rộng Ví dụ 38, p. 35} với $n = 4$ \& $a = 0$.

\begin{proof}[Giải]
	Có 4 cách: $-3,-2,-1,0$; $-2,-1,0,1$; $-1,0,1,2$; $0,1,2,3$.
\end{proof}

\begin{baitoan}[\cite{Tuyen_Toan_6}, \textbf{182.}, p. 36]
	Viết tập hợp các số nguyên $x$ sao cho: (a) $-4 < x < 3$; (b) $-2\le x\le 2$.
\end{baitoan}

\begin{proof}[Giải]
	(a) $A = \{-3,-2,-1,0,1,2\}$. (b) $B = \{-2,-1,0,1,2\}$.
\end{proof}

\begin{baitoan}[Mở rộng \cite{Tuyen_Toan_6}, \textbf{182.}, p. 36]
	Cho trước $a,b\in\mathbb{Z}$. Viết tập hợp các số nguyên $x$ sao cho: (a) $a < x < b$; (b) $a\le x < b$; (c) $a < x\le b$; (d) $a\le x\le b$.
\end{baitoan}

\begin{proof}[Giải]
	Đặt $A = \{x\in\mathbb{Z}|a < x < b\}$, $B = \{x\in\mathbb{Z}|a\le x < b\}$, $C = \{x\in\mathbb{Z}|a < x\le b\}$, $D = \{x\in\mathbb{Z}|a\le x\le b\}$. Xét các trường hợp sau: (a) Nếu $a > b$ thì $A = B = C = D = \emptyset$. (b) Nếu $a = b$ thì $A = B = C = \emptyset$, $D = \{a\}$. (c) Nếu $a < b$ thì $A = \{a + 1,a + 2,\ldots,b - 1\}$, $B = \{a,a + 1,a + 2,\ldots,b - 1\}$, $C = \{a + 1,a + 2,\ldots,b - 1,b\}$, $D = \{a,a + 1,a + 2,\ldots,b - 1,b\}$.
\end{proof}

\begin{baitoan}[\cite{Tuyen_Toan_6}, \textbf{183.}, p. 36]
	Cho các tập hợp $A = \{x\in\mathbb{Z}|x > -9\}$, $B = \{x\in\mathbb{Z}|x < -4\}$, $C = \{x\in\mathbb{Z}|x\ge-2\}$. Tìm $x$ sao cho: (a) $x\in A$ \& $x\in B$; (b) $x\in B$ \& $x\in C$; (c) $x\in C$ \& $x\in A$.
\end{baitoan}

\begin{proof}[Giải]
	(a) $x\in A\cap B = \{-5,-6,-7,-8\}$. (b) Không tồn tại $x$ thỏa. (c) có $C \subset A$ nên $x\in A\cap C = C = \{-2,-1,0,1,2,\ldots\}$.
\end{proof}

\begin{baitoan}[\cite{Tuyen_Toan_6}, \textbf{184.}, p. 36]
	Số nguyên âm lớn nhất có $3$ chữ số \& số nguyên âm nhỏ nhất có $2$ chữ số có phải là 2 số nguyên liền nhau không?
\end{baitoan}

\begin{proof}[Giải]
	Số nguyên âm lớn nhất có $3$ chữ số: $-100$. Số nguyên âm nhỏ nhất có $2$ chữ số: $-99$. Có $-100,-99$ là 2 số nguyên liền nhau\texttt{/}liên tiếp.
\end{proof}

\begin{baitoan}[\cite{Tuyen_Toan_6}, \textbf{185.}, p. 36]
	Tìm các giá trị thích hợp của $a,b$: (a) $\overline{a00} > -111$; (b) $-\overline{a99} > -600$; (c) $-\overline{cb3} < -\overline{cba}$; (d) $-\overline{cab} < -\overline{c85}$.
\end{baitoan}

\begin{proof}[Giải]
	(a) Có $\overline{a00} > 0 > -111$, $\forall a\in\{1,2,\ldots,9\}$. (b) $-\overline{a99} > -600\Leftrightarrow\overline{a99} < 600\Leftrightarrow a\in\{1,2,3,4,5\}$. (c) $-\overline{cb3} < -\overline{cba}\Leftrightarrow\overline{cb3} > \overline{cba}\Rightarrow 3 > a\Rightarrow a\in\{0,1,2\}$ (với $b\in\{0,1,\ldots,9\}$, $c\in\{1,2,\ldots,9\}$ tùy ý). (d) $-\overline{cab} < -\overline{c85}\Leftrightarrow\overline{cab} > \overline{c85}\Leftrightarrow\overline{ab}\in\{86,87,\ldots,99\}$ (với $c\in\{1,2,\ldots,9\}$ tùy ý).
\end{proof}

\begin{baitoan}[\cite{Binh_Toan_6_tap_1}, Ví dụ 48, p. 41]
	Cho $a\in\mathbb{Z}$. Gọi khoảng cách từ điểm $a$ đến điểm gốc trên trục số là \emph{giá trị tuyệt đối} của số $a$ \& ký hiệu là $|a|$. Điền vào chỗ trống các dấu $\ge,\le,>,<,=$ để các khẳng định sau là đúng:
	\begin{enumerate*}
		\item[(a)] $|a|\ldots a$, $\forall a\in\mathbb{Z}$.
		\item[(b)] $|a|\ldots 0$, $\forall a\in\mathbb{Z}$.
		\item[(c)] Nếu $a > 0$ thì $a\ldots|a|$.
		\item[(d)] Nếu $a = 0$ thì $a\ldots|a|$.
		\item[(e)] Nếu $a < 0$ thì $a\ldots|a|$.
	\end{enumerate*}
\end{baitoan}

\begin{baitoan}[\cite{Binh_Toan_6_tap_1}, \textbf{247.}, p. 42]
	Điền vào chỗ trống $\ldots$ các từ ``nhỏ hơn'' hoặc ``lớn hơn'' cho đúng:
	\begin{enumerate*}
		\item[(a)] Mọi số nguyên dương đều $\ldots$ số $0$.
		\item[(b)] Mọi số nguyên âm đều $\ldots$ số $0$.
		\item[(c)] Mỗi số nguyên dương đều $\ldots$ mọi số nguyên âm.
		\item[(d)] Trong 2 số nguyên dương, số nào có giá trị tuyệt đối lớn hơn thì số ấy $\ldots$
		\item[(e)] Trong 2 số nguyên âm, số nào có giá trị tuyệt đối lớn hơn thì số ấy $\ldots$
	\end{enumerate*}
\end{baitoan}

\begin{baitoan}[\cite{Binh_Toan_6_tap_1}, \textbf{248.}, p. 42]
	Tìm:
	\begin{enumerate*}
		\item[(a)] Số nguyên dương lớn nhất có 2 chữ số.
		\item[(a)] Số nguyên âm lớn nhất có 2 chữ số.
	\end{enumerate*}
\end{baitoan}

\begin{baitoan}[\cite{Binh_Toan_6_tap_1}, \textbf{249.}, p. 42]
	Tính $|b| - |a|$ biết: 
	\begin{enumerate*}
		\item[(a)] $a = -3$, $b = 7$;
		\item[(b)] $a = 5$, $b = -6$;
		\item[(c)] $a = 5$, $b = -5$;
	\end{enumerate*}
\end{baitoan}

\begin{baitoan}[\cite{Binh_Toan_6_tap_1}, \textbf{250.}, p. 42]
	Các khẳng định sau có đúng $\forall a,b\in\mathbb{Z}$ hay không? Cho ví dụ.
	\begin{enumerate*}
		\item[(a)] $|a| = |b|\Rightarrow a = b$.
		\item[(b)] $a > b\Rightarrow|a| > |b|$.
	\end{enumerate*}
\end{baitoan}

%------------------------------------------------------------------------------%

\section{$\pm$ Trên $\mathbb{Z}$}
``\textbf{1.} \textit{Cộng 2 số nguyên cùng dấu.} $\bullet$ Muốn cộng 2 số nguyên dương, ta cộng chúng như cộng 2 số tự nhiên. Muốn cộng 2 số nguyên âm, ta cộng 2 số đối của chúng rồi thêm dấu trừ $-$ đằng trước. $(+a) + (+b) = a + b$, $(-a) + (-b) = -(a + b)$, $\forall a,b\in\mathbb{N}^\star$. \textbf{2.} \textit{Cộng 2 số nguyên khác dấu.} $\bullet$ Tổng 2 số nguyên đối nhau thì bằng $0$: $a + (-a) = a - a = 0$, $\forall a\in\mathbb{Z}$. $\bullet$ Cộng 2 số nguyên khác dấu không đối nhau: Với 2 số nguyên dương $a,b$: $\circ$ $a > b\Rightarrow a + (-b) = a - b$. $\circ$ $a < b\Rightarrow a + (-b) = -(b - a)$. \textbf{3.} \textit{Tính chất của phép cộng 2 số nguyên.} Các tính chất của phép cộng trong $\mathbb{N}$ vẫn còn đúng trong $\mathbb{Z}$. Khi thực hiện cộng nhiều số nguyên ta có thể thay đổi tùy ý thứ tự các số hạng, nhóm các số hạng 1 cách tùy ý. \textbf{4.} \textit{Phép trừ 2 số nguyên.} Muốn lấy số nguyên $a$ trừ đi số nguyên $b$ ta cộng $a$ với số đối của $b$: $a - b = a + (-b)$, $\forall a,b\in\mathbb{Z}$. \textbf{5.} $a > b\Leftrightarrow a - b > 0$, $a < b\Leftrightarrow a - b < 0$, $\forall a,b\in\mathbb{Z}$.'' -- \cite[Chap. II, \S2, p. 37]{Tuyen_Toan_6}

\begin{luuy}
	Khi gặp bài tính nhanh\emph{\texttt{/}}hợp lý, phải: (a) Tìm các số đối nhau để gom lại: $a + (-a) = (-a) + a = a - a = 0$, $\forall a\in\mathbb{Z}$. (b) Tìm \& nhóm các số sao cho khi cộng, trừ, nhân, chia chúng được các số tròn chục, tròn trăm, tròn nghìn, etc., i.e., làm xuất hiện các lũy thừa của $10$ là $10^n$ với $n\in\mathbb{N}^\star$. (c) Nếu trong 1 tổng đại số không xuất hiện 2 trường hợp vừa nêu, i.e., không có các cặp số đối nhau \& cũng không có các nhóm số sao cho khi cộng, trừ, nhân, chia chúng được các số tròn chục, tròn trăm, tròn nghìn, etc., i.e., làm xuất hiện $10^n$, thì ta sẽ gom các nhóm số khác dấu nhưng có giá trị tuyệt đối gần bằng nhau, i.e., gom các số làm xuất hiện $a + b$ với $a,b$ trái dấu: $ab < 0$ \& $|a|\approx|b|$. (d) Nếu cả 3 trường hợp trên đều không xảy ra, thì gom 1 nhóm tất cả các số âm \& 1 nhóm tất cả các số dương lại để tính.
\end{luuy}

\begin{baitoan}[\cite{Trong_Toan_6_2021}, \textbf{9.}, p. 59]
	Tính hợp lý: (a) $152 + (-73) - (-18) - 127$; (b) $7 + 8 + (-9) + (-10)$.
\end{baitoan}

\begin{proof}[Giải]
	(a) $152 + (-73) - (-18) - 127 = (152 + 18) - (73 + 127) = 170 - 200 = -30$. (b) $7 + 8 + (-9) + (-10) = (7 - 9) + (8 - 10) = -2 - 2 = -4$.
\end{proof}

\begin{baitoan}[\cite{Trong_Toan_6_2021}, \textbf{10.}, p. 59]
	Tính giá trị của biểu thức $(-156) - x$ khi: (a) $x = -26$; (b) $x = 76$; (c) $x = (-28) - (-143)$.
\end{baitoan}

\begin{baitoan}[\cite{Trong_Toan_6_2021}, \textbf{11.}, p. 59]
	Thay mỗi dấu $\star$ bằng 1 chữ số thích hợp: (a) $(-\overline{6\star}) + (-34) = -100$; (b) $(-789) + \overline{2\star\star} = -515$.
\end{baitoan}

\begin{baitoan}[\cite{Trong_Toan_6_2021}, \textbf{12.}, p. 59]
	Liệt kê các phần tử của tập hợp sau rồi tính tổng của chúng: (a) $A = \{x\in\mathbb{Z}|- 5 < x < 5\}$; (b) $B = \{x\in\mathbb{Z}|-7\le x < 1\}$.
\end{baitoan}

\begin{baitoan}[Mở rộng \cite{Trong_Toan_6_2021}, \textbf{12.}, p. 59]
	Cho trước $a,b\in\mathbb{Z}$. Liệt kê các phần tử của tập hợp sau rồi tính tổng của chúng: (a) $A = \{x\in\mathbb{Z}|a < x < b\}$; (b) $B = \{x\in\mathbb{Z}|a\le x < b\}$; (c) $C = \{x\in\mathbb{Z}|a < x\le b\}$; (d) $D = \{x\in\mathbb{Z}|a\le x\le b\}$; trong các trường hợp: (1) $a\ge b$; (2) $0 < a < b$; (3) $a < 0 < b$; (4) $a < b < 0$.
\end{baitoan}

\begin{baitoan}[\cite{Tuyen_Toan_6}, Ví dụ 40, p. 37]
	Tính tổng $S = (-351) + (-74) + 51 + (-126) + 149$.
\end{baitoan}

\begin{proof}[1st giải]
	$S = [(-315) + (-74) + (-126)] + (51 + 149) = -551 + 200 = -351$.
\end{proof}

\begin{proof}[2nd giải]
	$S = [(-351) + 51] + [(-74) + (-126)] + 149 = (-300) + (-200) + 149 = -500 + 149 = -351$.
\end{proof}

\begin{nhanxet}
	``Trong 1st Giải, để cộng nhiều số ta cộng số âm với số âm, số dương với số dương rồi cộng 2 kết quả với nhau. Cách này có ưu điểm là đỡ nhầm dấu. Trong 2nd Giải, ta kết hợp từng nhóm có tổng là 1 số tròn trăm. Cách giải này có ưu điểm là có thể nhẩm ra kết quả.'' -- \cite[p. 37]{Tuyen_Toan_6}
\end{nhanxet}

\begin{baitoan}[\cite{Tuyen_Toan_6}, Ví dụ 41, p. 38]
	Với $a,b\in\mathbb{Z}$, chứng minh $a - b$ \& $b - a$ là 2 số đối nhau.
\end{baitoan}
\noindent\textit{Phân tích.} Để chứng minh 2 số đối nhau ta chứng minh tổng của chúng bằng $0$.
 
\begin{proof}[1st giải]
	Có $(a - b) + (b - a) = a - b + b - a = (a - a) + (b - b) = 0 + 0 = 0$.
\end{proof}

\begin{proof}[2nd giải]
	Có $(a - b) + (b - a) = [a + (-b)] + [b + (-a)] = [a + (-a)] + [b + (-b)] = 0 + 0 = 0$.
\end{proof}

\begin{nhanxet}
	``Do $a - b$ \& $b - a$ là 2 số đối nhau nên nếu biết hiệu $a - b$ thì không cần làm phép trừ ta cũng tính được $b - a$ 1 cách nhanh chóng.'' -- \cite[p. 38]{Tuyen_Toan_6}
\end{nhanxet}

\begin{baitoan}[\cite{Tuyen_Toan_6}, \textbf{186.}, p. 38]
	Tính nhanh: (a) $-37 + 54 + (-70) + (-163) + 246$; (b) $-359 + 181 + (-123) + 350 + (-172)$; (c) $-69 + 53 + 46 + (-94) + (-14) + 78$.
\end{baitoan}

\begin{proof}[Giải]
	(a) $-37 + 54 + (-70) + (-163) + 246 = (54 + 246) + [(-37) + (-163)] + (-70) = 300 + (-200) + (-70) = 100 - 70 = 30$.\\(b) \textit{Cách 1}: $-359 + 181 + (-123) + 350 + (-172) = (181 + 350) + [-359 + (-172)] + (-123) = 531 + (-531) + (-123) = 0 - 123 = -123$. \textit{Cách 2}: $-359 + 181 + (-123) + 350 + (-172) = (350 - 359) + (181 - 172) - 123 = -9 + 9 - 123 = 0 - 123 = 0$.\\(c) $-69 + 53 + 46 + (-94) + (-14) + 78 = [-69 + (-94) + (-14)] + (53 + 46 + 78) = -177 + 177 = 0$.
\end{proof}

\begin{baitoan}[\cite{Tuyen_Toan_6}, \textbf{187.}, p. 38]
	Tính tổng các số nguyên $x$ biết: $-17\le x\le18$.
\end{baitoan}

\begin{proof}[Giải]
	$x\in\mathbb{Z}\land(-17\le x\le18)\Leftrightarrow x\in\{-17,-16,\ldots,-1,0,1,\ldots,18\}$. Tổng của chúng bằng $(-17) + (-16) + \cdots + (-1) + 0 + 1 + \cdots + 16 + 17 + 18 = (-17 + 17) + (-16 + 16) + \cdots + (-1 + 1) + 0 + 18 = 0 + 0 + \cdots + 0 + 0 + 18 = 18$. 
\end{proof}

\begin{baitoan}[\cite{Tuyen_Toan_6}, \textbf{188.}, p. 38]
	Cho $S_1 = 1 + (-3) + 5 + (-7) + \cdots + 17$, $S_2 = -2 + 4 + (-6) + 8 + \cdots + (-18)$. Tính $S_1 + S_2$.
\end{baitoan}

\begin{proof}[Giải]
	$S_1 + S_2 = 1 + (-3) + 5 + (-7) + \cdots + 17 - 2 + 4 + (-6) + 8 + \cdots + (-18) = [1 + (-3) + (-2) + 4] + [5 + (-7) + (-6) + 8] + \cdots + [13 + (-15) + (-14) + 16] + [17 + (-18)] = 0 + 0 + \cdots + 0 + (-1) = -1$.
\end{proof}

\begin{baitoan}[\cite{Tuyen_Toan_6}, \textbf{189.}, p. 38]
	Cho $x\in\{-3,-2,-1,0,1,2,\ldots,10\}$, $y\in\{-1,0,1,2,\ldots,5\}$. Biết $x + y = 3$, tìm $x,y$.
\end{baitoan}

\begin{proof}[Giải]
	$(x,y)\in\{(-2,5),(-1,4),(0,3),(1,2),(2,1),(3,0),(4,-1)\}$.
\end{proof}

\begin{baitoan}[\cite{Tuyen_Toan_6}, \textbf{190.}, p. 38]
	1 thủ quỹ ghi số tiền thu chi trong ngày (đơn vị là nghìn đồng) như sau: $+7250,+13485,-10964,+5000,-1380,+24750,-9771$. Đầu ngày trong két có $500$ (nghìn đồng). Hỏi cuối ngày trong két có bao nhiêu?\hfill{\sf Ans:} $28870$ nghìn đồng.
\end{baitoan}

\begin{baitoan}[\cite{Tuyen_Toan_6}, \textbf{191.}, p. 38]
	Chứng minh số đối của tổng 2 số bằng tổng 2 số đối của chúng.
\end{baitoan}

\begin{proof}[1st giải]
	Ta phải chứng minh $-(x + y) = (-x) + (-y)$. Xét tổng $(x + y) + [(-x) + (-y)] = [x + (-x)] + [y + (-y)] = 0 + 0 = 0$. Tổng của 2 số bằng 0 vậy chúng đối nhau. Suy ra $-(x + y) = (-x) + (-y)$, $\forall x,y\in\mathbb{Z}$.
\end{proof}

\begin{proof}[2nd giải]
	Có $-(a + b) = -a - b = (-a) + (-b)$, $\forall a,b\in\mathbb{Z}$.
\end{proof}

\begin{baitoan}[Mở rộng \cite{Tuyen_Toan_6}, \textbf{191.}, p. 38]
	Chứng minh số đối của tổng $n$ số bằng tổng $n$ số đối của chúng với $n\in\mathbb{N}^\star$ cho trước.
\end{baitoan}

\begin{proof}[1st giải]
	Ta phải chứng minh $-(a_1 + a_2 + \cdots + a_n) = (-a_1) + \cdots + (-a_n)$. Xét tổng $(a_1 + a_2 + \cdots + a_n) + [(-a_1) + (-a_2) + \cdots + (-a_n)] = [a_1 + (-a_1)] + [a_2 + (-a_2)] + \cdots + [a_n + (-a_n)] = 0 + 0 + \cdots + 0 = 0$. Tổng của 2 số bằng 0 vậy chúng đối nhau. Suy ra $-(a_1 + a_2 + \cdots + a_n) = (-a_1) + \cdots + (-a_n)$, $\forall a_i\in\mathbb{Z}$, $\forall i = 1,2,\ldots,n$.
\end{proof}

\begin{proof}[2nd giải]
	Có $-(a_1 + a_2 + \cdots + a_n) = -a_1 - a_2 - \cdots - a_n = (-a_1) + (-a_2) + \cdots + (-a_n)$, $\forall a_i\in\mathbb{Z}$, $\forall i = 1,2,\ldots,n$.
\end{proof}

\begin{baitoan}[\cite{Tuyen_Toan_6}, \textbf{192.}, p. 38]
	Cho $18$ số nguyên sao cho tổng của $6$ số bất kỳ trong các số đó đều là 1 số âm. Giải thích vì sao tổng của $18$ số đó cũng là 1 số âm. Bài toán còn đúng không nếu thay $18$ số bởi $19$ số?
\end{baitoan}

\begin{proof}[Giải]
	Ta chia 18 số thành 3 nhóm, mỗi nhóm 6 số. Vì tổng của 6 số bất kỳ là 1 số âm nên tổng của các số trong mỗi nhóm là 1 số âm. Vậy tổng của 3 nhóm tức là tổng của 18 số là 1 số âm. Nếu thay 18 số bằng 19 số thì trong 19 số ít nhất cũng có 1 số âm (vì nếu không có 1 số âm nào thì tổng của 6 số bất kỳ không thể là số âm). Ta tách riêng số âm đó ra còn lại 18 số. Theo phần trước thì tổng của 18 số là 1 số âm, cộng với số âm đã tách riêng ra từ đầu sẽ được 1 số âm, i.e., tổng của 19 số đã cho cũng là 1 số âm.
\end{proof}

\begin{baitoan}[\cite{Tuyen_Toan_6}, \textbf{192.}, p. 38]
	Cho trước $m,n\in\mathbb{N}^\star$. Cho $m$ số nguyên sao cho tổng của $n$ số bất kỳ trong các số đó đều là 1 số âm. Tổng của $m$ số đó có là 1 số âm hay không? Biện luận theo $m,n$.
\end{baitoan}

\begin{baitoan}[\cite{Tuyen_Toan_6}, \textbf{193.}, p. 38]
	Cho $x = \pm5$, $y = \pm11$ với $x,y\in\mathbb{Z}$. Tính $x + y$.
\end{baitoan}

\begin{proof}[Giải]
	Xét 4 trường hợp: $(x,y)\in\{(5,11),(-5,11),(5,-11),(-5,-11)\}$ được $x + y\in\{16,-16,-6,6\} = \{\pm6,\pm16\}$.
\end{proof}

\begin{baitoan}[\cite{Tuyen_Toan_6}, \textbf{194.}, p. 38]
	Cho $x = \pm7$, $y = \pm20$ với $x,y\in\mathbb{Z}$. Tính $x - y$.
\end{baitoan}

\begin{proof}[Giải]
	Xét 4 trường hợp: $(x,y)\in\{(7,20),(-7,20),(7,-20),(-7,-20)\}$ được $x + y\in\{-13,-27,27,13\} = \{\pm13,\pm27\}$.
\end{proof}

\begin{baitoan}[\cite{Tuyen_Toan_6}, \textbf{195.}, p. 38]
	Cho $-3\le x\le3$ \& $-5\le y\le5$ với $x,y\in\mathbb{Z}$. Biết $x - y = 2$, tìm $x,y$.
\end{baitoan}

\begin{proof}[Giải]
	$(x,y)\in\{(-3,-5),(-2,-4),(-1,-3),(0,-2),(1,-1),(2,0),(3,1)\}$.
\end{proof}

\begin{baitoan}[\cite{Tuyen_Toan_6}, \textbf{196.}, p. 38]
	Cho $x\in\{-2,-1,0,1,\ldots,11\}$, $y\in\{-89,-88,-87,\ldots,-1,0,1\}$. Tìm giá trị lớn nhất (GTLN hoặc $\max$) \& giá trị nhỏ nhất (GTNN hoặc $\min$) của hiệu $x - y$.
\end{baitoan}
\noindent\textit{Phân tích.} ``Trong 1 phép trừ, hiệu sẽ lớn nhất khi số bị trừ lớn nhất còn số trừ phải nhỏ nhất. Ngược lại, hiệu sẽ nhỏ nhất nếu số bị trừ nhỏ nhất còn số trừ lớn nhất.'' -- \cite[p. 130]{Tuyen_Toan_6}

\begin{proof}[Giải]
	Đặt $A\coloneqq\{-2,-1,0,1,\ldots,11\}$, $B\coloneqq\{-89,-88,-87,\ldots,-1,0,1\}$. $\min_{x\in A,y\in B} x - y = 11 - (-89) = 11 + 89 = 100$. $\max_{x\in A,y\in B} x - y = -2 - 1 = -3$. Vậy $\min_{x\in A,y\in B} x - y = 100$ \& $\max_{x\in A,y\in B} x - y = -3$.
\end{proof}

\begin{baitoan}[\cite{Tuyen_Toan_6}, \textbf{197.}, p. 38]
	Quan sát các số sau \& các số còn thiếu (?) để tìm giá trị của $x$:
	\begin{table}[H]
		\centering
		\begin{tabular}{ccccccc}
			$40$ &  & $32$ &  & $21$ &  &  $15$ \\
			& $8$ &  & ? &  & $6$ &  \\
			&  & ? &  & ? &  &  \\
			&  &  & $x$ &  &  &  \\
		\end{tabular}
	\end{table}
\end{baitoan}

\begin{proof}[Giải]
	Trên mỗi hàng, xét 2 số liền nhau, lấy số bên trái trừ đi số bên phải ta được hiệu là số ở hàng dưới (viết ở giữa 2 số vừa lấy hiệu), được:
	\begin{table}[H]
		\centering
		\begin{tabular}{ccccccc}
			$40$ &  & $32$ &  & $21$ &  &  $15$ \\
			& $8$ &  & $11$ &  & $6$ &  \\
			&  & $-3$ &  & $5$ &  &  \\
			&  &  & $-8$ &  &  &  \\
		\end{tabular}
	\end{table}
	Vậy $x = -8$.
\end{proof}

\begin{baitoan}[\cite{Binh_Toan_6_tap_1}, Ví dụ 49, p. 42]
	Tìm $x\in\mathbb{Z}$, biết $10 = 10 + 9 + 8 + \cdots + x$, trong đó vế phải là tổng các số nguyên liên tiếp viết theo thứ tự giảm dần.
\end{baitoan}

\begin{baitoan}[\cite{Binh_Toan_6_tap_1}, \textbf{251.}, p. 42]
	Tìm tổng của số nguyên âm nhỏ nhất có 1 chữ số \& số nguyên dương lớn nhất có 1 chữ số.
\end{baitoan}

\begin{baitoan}[\cite{Binh_Toan_6_tap_1}, \textbf{252.}, p. 42]
	Điền vào chỗ trống cho đúng:
	\begin{enumerate*}
		\item[(a)] Số đối của 1 số nguyên âm là 1 số $\ldots$
		\item[(b)] 2 số nguyên đối nhau thì có giá trị tuyệt đối $\ldots$
		\item[(c)] 2 số nguyên có giá trị tuyệt đối bằng nhau thì $\ldots$
		\item[(d)] Số $\ldots$ thì nhỏ hơn số đối của nó.
		\item[(e)] Nếu $a\ldots$ thì $-a > 0$.
		\item[(f)] Nếu $a < 0$ thì $|a| = \ldots$
		\item[(g)] Nếu $a < 0$ thì $a + |a| = \ldots$
	\end{enumerate*}
\end{baitoan}

\begin{baitoan}[\cite{Binh_Toan_6_tap_1}, \textbf{253.}, p. 43]
	Tìm $x\in\mathbb{Z}$ biết:
	\begin{enumerate*}
		\item[(a)] $x + 13 = 5$.
		\item[(b)] $x - 1 = -9$.
		\item[(c)] $25 - |x| = 10$.
		\item[(d)] $|x - 2| + 7 = 12$.
		\item[(e)] $x + 4$ là số nguyên dương nhỏ nhất.
		\item[(f)] $10 - x$ là số nguyên âm lớn nhất.
	\end{enumerate*}
\end{baitoan}

\begin{baitoan}[\cite{Binh_Toan_6_tap_1}, \textbf{254.}, p. 43]
	\begin{itemize}
		\item[(a)] Cho bảng vuông $3\times 3$ ô:
		\begin{table}[H]
			\centering
			\begin{tabular}{|c|c|c|}
				\hline
				$-8$ & $7$ &  \\
				\hline
				$\ \ 5$ &  & $\ \ 9$ \\
				\hline
				& $5$ & $-6$ \\
				\hline
			\end{tabular}
		\end{table}
		Điền số vào các ô trống sao cho tổng các số ở 3 dòng 1,2,3 lần lượt bằng $-5,11,1$. Tính tổng các số ở mỗi cột.
		\item[(b)] Cho bảng vuông $3\times 3$ ô. Có thể điền được hay không 9 số nguyên vào 9 ô của bảng sao cho tổng các số ở 3 dòng lần lượt bằng $5,-3,2$ \& tổng các số ở 3 cột lần lượt bằng $-1,2,2$?
	\end{itemize}
\end{baitoan}

\begin{baitoan}[\cite{Binh_Toan_6_tap_1}, \textbf{255.}, p. 43]
	\begin{enumerate*}
		\item[(a)] Có $10$ ô liên tiếp trong đó ô đầu tiên ghi số $6$, ô thứ $8$ ghi số $-4$. Điền số vào các ô trống để tổng 3 số ở 3 ô liền nhau bằng $0$.
		\item[(b)] 1 bảng vuông $4\times 4$ ô có 2 ô ở góc trên ghi số $-3$ \& $2$. Điền số vào các ô còn lại, sao cho tổng 2 số ở 2 ô liền nhau thì bằng nhau (2 ô liền nhau là 2 ô có 1 cạnh chung).
	\end{enumerate*}
\end{baitoan}

\begin{baitoan}[\cite{Binh_Toan_6_tap_1}, \textbf{256.}, p. 43]
	Tìm $x\in\mathbb{Z}$ biết $x + (x + 1) + (x + 2) + \cdots + 19 + 20 = 20$, trong đó vế trái là tổng các số nguyên liên tiếp viết theo thứ tự tăng dần.
\end{baitoan}

\begin{baitoan}[\cite{Binh_Toan_6_tap_1}, \textbf{257.}, p. 43]
	Tìm các số nguyên $a$ sao cho:
	\begin{enumerate*}
		\item[(a)] $a > -a$.
		\item[(b)] $a = -a$.
		\item[(c)] $a < -a$.
	\end{enumerate*}
\end{baitoan}

\begin{baitoan}[\cite{Binh_Toan_6_tap_1}, \textbf{258.}, p. 43]
	Tìm $a,b,c\in\mathbb{Z}$ biết: $a + b = 11$, $b + c = 3$, $c + a = 2$.
\end{baitoan}

\begin{baitoan}[\cite{Binh_Toan_6_tap_1}, \textbf{259.}, p. 43]
	Tìm $a,b,c,d\in\mathbb{Z}$ biết $a + b + c + d = 1$, $a + c + d = 2$, $a + b + d = 3$, $a + b + c = 4$.
\end{baitoan}

\begin{baitoan}[\cite{Binh_Toan_6_tap_1}, \textbf{260.}, p. 43]
	Cho $\sum_{i=1}^{51} x_i = x_1 + x_2 + \cdots + x_{50} + x_{51} = 0$ \& $x_1 + x_2 = x_3 + x_4 = \cdots = x_{47} + x_{48} = x_{49} + x_{50} = x_{50} + x_{51} = 1$. Tính $x_{50}$.
\end{baitoan}

%------------------------------------------------------------------------------%

\section{Quy Tắc Dấu Ngoặc}
``\textbf{1.} \textit{Quy tắc dấu ngoặc}: Khi bỏ dấu ngoặc có dấu ``$+$'' đằng trước, ta giữ nguyên dấu của các số hạng trong ngoặc. Khi bỏ dấu ngoặc có dấu ``$-$'' đằng trước, ta phải đổi dấu tất cả của các số hạng trong ngoặc. \textbf{2.} \textit{Tổng đại số}: 1 dãy các phép tính cộng, trừ các số nguyên được gọi là 1 \textit{tổng đại số} hay 1 \textit{tổng}. Để viết gọn 1 tổng đại số ta làm như sau: (1) Thay các phép trừ bằng phép cộng với số đối. (2) Bỏ đi dấu của phép cộng \& dấu ngoặc. (3) Đổi vị trí của các số hạng (nếu cần). \textbf{3.} \textit{Quy tắc đặt dấu ngoặc}: Khi đặt dấu ngoặc có dấu ``$+$'' đằng trước ta giữ nguyên dấu của các số hạng đặt vào trong ngoặc. Khi đặt dấu ngoặc có dấu ``$-$'' đằng trước ta phải đổi dấu tất cả các số hạng đặt vào trong ngoặc.'' -- \cite[Chap. II, \S3, p. 39]{Tuyen_Toan_6}

\begin{baitoan}[\cite{Tuyen_Toan_6}, Ví dụ 42, p. 39]
	Cho $a$ là 1 số nguyên âm, còn $b,c\in\mathbb{Z}$. Chứng minh số $M = (-a + b) - (b + c - a) + (c - a)$ là 1 số nguyên dương.
\end{baitoan}

\begin{proof}[Giải]
	$M = -a + b - b - c + a + c - a = (-a + a) + (b - b) + (-c + c) - a = -a > 0$ do $a < 0$. Hiển nhiên $M\in\mathbb{Z}$ vì $M$ là 1 tổng (đại số) của các số nguyên. Vậy $M$ là 1 số nguyên dương.
\end{proof}

\begin{baitoan}[\cite{Tuyen_Toan_6}, \textbf{198.}, p. 39]
	Tính hợp lý: (a) $-2021 + (-22 + 87 + 2021)$; (b) $1152 - (374 + 1152) + (-65 + 374)$.
\end{baitoan}

\begin{proof}[Giải]
	(a) $-2021 + (-22 + 87 + 2021) = -2021 - 22 + 87 + 2021 = (-2021 + 2021) + (-22 + 87) = 0 + 65 = 65$. (b) $1152 - (374 + 1152) + (-65 + 374) = 1152 - 374 - 1152 - 65 + 374 = (1152 - 1152) + (-374 + 374) - 65 = 0 + 0 - 65 = -65$.
\end{proof}

\begin{baitoan}[\cite{Tuyen_Toan_6}, \textbf{199.}, p. 39]
	Đặt dấu ngoặc 1 cách thích hợp để tính các tổng đại số sau: (a) $942 - 2567 + 2563 - 1942$; (b) $13 - 12 + 11 + 10 - 9 + 8 - 7 - 6 + 5 - 4 + 3 + 2 - 1$.
\end{baitoan}

\begin{proof}[Giải]
	(a) $942 - 2567 + 2563 - 1942 = (942 - 1942) - (2567 - 2563) = -1000 - 4 = -1004$. (b) $13 - 12 + 11 + 10 - 9 + 8 - 7 - 6 + 5 - 4 + 3 + 2 - 1 = 13 - (12 - 11 - 10 + 9) + (8 - 7 - 6 + 5) - (4 - 3 - 2 + 1) = 13 - 0 + 0 - 0 = 13$.
\end{proof}

\begin{baitoan}[\cite{Tuyen_Toan_6}, \textbf{200.}, p. 39]
	Tìm $x\in\mathbb{Z}$ thỏa: (a) $461 + (x - 45) = 387$; (b) $11 - (-53 + x) = 97$; (c) $-(x + 84) + 213 = -16$.
\end{baitoan}

\begin{proof}[1st giải]
	Xử lý biểu thức trong ngoặc trước: (a) $461 + (x - 45) = 387\Leftrightarrow 461 + x - 45 = 387\Leftrightarrow x = 387 + 45 - 461 = 432 - 461 = -29$. (b) $11 - (-53 + x) = 97\Leftrightarrow 11 + 53 - x = 97\Leftrightarrow x = 11 + 53 - 97 = 64 - 97 = -33$. (c) $-(x + 84) + 213 = -16\Leftrightarrow-x - 84 + 213 = -16\Leftrightarrow x = 213 + 16 - 84 = 229 - 84 = 145$.
\end{proof}

\begin{proof}[2nd giải]
	Xử lý biểu thức trong ngoặc sau: (a) $461 + (x - 45) = 387\Leftrightarrow x - 45 = 387 - 461 = -74\Leftrightarrow x = 45 - 74 = - 29$. (b) $11 - (-53 + x) = 97\Leftrightarrow x - 53 = 11 - 97 = -86\Leftrightarrow x = 53 - 86 = -33$. (c) $-(x + 84) + 213 = -16\Leftrightarrow x + 84 = 213 + 16 = 229\Leftrightarrow x = 229 - 84 = 145$.
\end{proof}

\begin{baitoan}[\cite{Tuyen_Toan_6}, \textbf{201.}, p. 39]
	Chứng minh: $-(-a + b + c) + (b + c - 1) = (b - c + 6) - (7 - a + b) + c$, $\forall a,b,c\in\mathbb{Z}$.
\end{baitoan}

\begin{luuy}[Viết tắt\texttt{/}Abbreviation]
	Người ta thường viết tắt (cả sách Việt Nam lẫn sách Toán tiếng Anh) ``vế trái'' là VT hoặc LHS hoặc l.h.s. (left-hand side), ``vế phải'' là VP, VF, RHS, hoặc r.h.s. (right-hand side).
\end{luuy}

\begin{proof}[Giải]
	VT $= -(-a + b + c) + (b + c - 1) = a - b - c + b + c - 1 = a + (-b + b) + (-c + c) - 1 = a + 0 + 0 - 1 = a - 1$. VF $= (b - c + 6) - (7 - a + b) + c = b - c + 6 - 7 + a - b + c = (b - b) + (-c + c) + a + (6 - 7) = 0 + 0 + a - 1 = a - 1$. Vậy VT $=$ VF, $\forall a,b,c\in\mathbb{Z}$.
\end{proof}

\begin{baitoan}[\cite{Tuyen_Toan_6}, \textbf{202.}, p. 40]
	Cho $a,b,c\in\mathbb{Z}$ \& $A = a + b - 5$, $B = -b - c + 1$, $C = b - c - 4$, $D = b - a$. Chứng minh $A + B = C - D$.
\end{baitoan}

\begin{proof}[Giải]
	$A + B = a + b - 5 - b - c + 1 = a + (b - b) - c + (1 - 5) = a - c - 4$. $C - D = b - c - 4 - (b - a) = b - c - 4 - b + a = (b - b) - c - 4 + a = a - c - 4$. Suy ra $A + B = C - D$.
\end{proof}

\begin{baitoan}[\cite{Tuyen_Toan_6}, \textbf{203.}, p. 40]
	Cho $S = -(a - b - c) + (-c + b + a) - (a + b)$ trong đó $a > b$ \& $a,b,c\in\mathbb{Z}$. Chứng minh $S$ là 1 số nguyên âm.
\end{baitoan}

\begin{proof}[Giải]
	$S = -(a - b - c) + (-c + b + a) - (a + b) = -a + b + c - c + b + a - a - b = (-a + a) + (b - b) + (c - c) + b - a = 0 + 0 + 0 + b - a = b - a < 0$ do $a > b$, mà $S\in\mathbb{Z}$ (do $S$ là tổng (đại số) của các số nguyên), nên $S$ là 1 số nguyên âm.
\end{proof}

\begin{baitoan}[\cite{Tuyen_Toan_6}, \textbf{204.}, p. 40]
	Viết $5$ số nguyên vào $5$ đỉnh của 1 ngôi sao 5 cánh sao cho tổng của 2 số tại 2 đỉnh liền nhau luôn bằng $-6$. Tìm $5$ số nguyên đó.
\end{baitoan}

\begin{baitoan}[\cite{Tuyen_Toan_6}, \textbf{205.}, p. 40]
	Cho $1001$ số tự nhiên từ $1$ đến $1001$ sắp xếp theo thứ tự tùy ý. Lấy số thứ nhất trừ đi $1$, lấy số thứ 2 trừ đi $2$, lấy số thứ 3 trừ đi $3$, $\ldots$, lấy số thứ 1001 trừ đi $1001$. Tính tổng của $1001$ số mới.
\end{baitoan}

%------------------------------------------------------------------------------%

\section{$\cdot,:$ Trên $\mathbb{Z}$}
``\textbf{1.} \textit{Nhân 2 số nguyên cùng dấu}: Muốn nhân 2 số nguyên dương ta nhân chúng như nhân 2 số tự nhiên. Muốn nhân 2 số nguyên âm ta nhân 2 số đối của chúng. \textbf{2.} \textit{Nhân 2 số nguyên khác dấu}: Với $a,b$ là 2 số nguyên dương thì $(+a)\cdot(-b) = (-a)\cdot(+b) = -(ab) = -ab$. \textbf{3.} $a\cdot1 = 1a = a$, $a\cdot0 = 0a = 0$, $\forall a\in\mathbb{Z}$. $ab = 0\Leftrightarrow a = 0\lor b = 0$. \textbf{4.} \textit{Tính chất của phép nhân 2 số nguyên}: Các tính chất của phép nhân trong $\mathbb{N}$ vẫn còn đúng trong $\mathbb{Z}$. Phép nhân các số nguyên còn có tính chất phân phối đối với phép trừ: $a(b - c) = ab - ac$. Trong phép nhân nhiều thừa số khác $0$, nếu thừa số âm chẵn thì tích mang dấu ``$+$''. Nếu thừa số âm lẻ thì tích mang dấu ``$-$''. Khi đổi dấu 1 thừa số thì tích đổi dấu. Khi đổi dấu 2 thừa số thì tích không thay đổi. Tổng quát hơn, khi đổi dấu 1 số lẻ các thừa số thì tích đổi dấu \& khi đổi dấu 1 số chẵn các thừa số thì tích không thay đổi. \textbf{5.} Lũy thừa bậc chẵn của 1 số âm là 1 số dương, i.e., $(-a)^{2n} > 0$, $\forall a,n\in\mathbb{N}$, $a\ne 0$. Lũy thừa bậc lẻ của 1 số âm là 1 số âm, i.e., $(-a)^{2n + 1} < 0$, $\forall a,n\in\mathbb{N}$, $a\ne 0$. \textbf{6.} $a^2\ge 0$, $\forall a\in\mathbb{Z}$ (dấu ``$=$'' xảy ra $\Leftrightarrow a = 0$).'' -- \cite[Chap. II, \S4, p. 40]{Tuyen_Toan_6}

\begin{baitoan}[\cite{Tuyen_Toan_6}, Ví dụ 43, p. 40]
	Tìm $a,b\in\mathbb{Z}$ biết $ab = 24$ \& $a + b = -10$.
\end{baitoan}

\begin{proof}[Giải]
	Vì $ab > 0$ nên $a,b$ cùng dấu, \& vì $a + b < 0$ nên $a,b$ cùng âm. Có $ab = 24 = (-1)\cdot(-24) = (-2)\cdot(-12) = (-3)\cdot(-8) = (-4)\cdot(-6)$. Trong các trường hợp trên chỉ có $(-4) + (-6) = -10$. Vậy $a = -4$, $b = -6$ hoặc $a = -6$ hoặc $b = -4$.\footnote{Có thể viết gom kết luận lại thành $\{a,b\} = \{-4,-6\}$, i.e., dùng tập hợp để không cần phân biệt thứ tự của $a,b$.}
\end{proof}

\begin{nhanxet}
	``Trong ví dụ trên ta cũng có thể biểu diễn số $-10$ dưới dạng tổng của 2 số nguyên âm, i.e., $-10 = (-1) + (-9) = (-2) + (-8) = (-3) + (-7) = (-4) + (-6) = (-5) + (-5)$. Có tất cả $9$ trường hợp. Xét $9$ trường hợp này có 1 trường hợp có đáp số như trên. Cách này chưa hay vì phải xét nhiều trường hợp.'' -- \cite[p. 41]{Tuyen_Toan_6}
\end{nhanxet}

\begin{baitoan}[\cite{Tuyen_Toan_6}, Ví dụ 44, p. 41]
	Tìm tất cả các cặp số nguyên sao cho tổng bằng tích.
\end{baitoan}

\begin{proof}[Giải]
	Gọi cặp số nguyên cần tìm là $x,y\in\mathbb{Z}$. Có
	\begin{equation*}
		x + y = xy\Leftrightarrow xy - x - y + 1 = 1\Leftrightarrow x(y - 1) - (y - 1) = 1\Leftrightarrow(x - 1)(y - 1) = 1\Leftrightarrow\left[\begin{split}
			y - 1 = x - 1 = 1,\\
			y - 1 = x - 1 = -1,
		\end{split}\right.\Leftrightarrow\left[\begin{split}
			x = y = 2,\\
			x = y = 0.
	\end{split}\right.
	\end{equation*}
	Vậy $(x,y)\in\{(0,0),(2,2)\}$.
\end{proof}

\begin{baitoan}[\cite{Tuyen_Toan_6}, \textbf{206.}, p. 41]
	Tìm $x\in\mathbb{Z}$ thỏa: (a) $x(x + 3) = 0$; (b) $(x - 2)(5 - x) = 0$; (c) $(x - 1)(x^2 + 1) = 0$.
\end{baitoan}

\begin{proof}[Giải]
	(a) $x(x + 3) = 0\Leftrightarrow x = 0\lor x + 3 = 0\Leftrightarrow x = 0\lor x = -3$. Vậy $x\in\{-3,0\}$. (b) $(x - 2)(5 - x) = 0\Leftrightarrow x - 2 = 0\lor5 - x = 0\Leftrightarrow x = 2\lor x = 5$. Vậy $x\in\{2,5\}$. (c) $(x - 1)(x^2 + 1) = 0\Leftrightarrow x - 1 = 0\lor x^2 + 1 = 0\Leftrightarrow x = 1$, trong đó $x^2 + 1 = 0$ vô nghiệm vì $x^2\ge 0$, $\forall x\in\mathbb{Z}$, nên $x^2 + 1\ge0 + 1 = 1 > 0$, $\forall x\in\mathbb{Z}$. Vậy $x = 1$.
\end{proof}

\begin{baitoan}[\cite{Tuyen_Toan_6}, \textbf{207.}, p. 41]
	Thu gọn các biểu thức sau với $x,y\in\mathbb{Z}$: (a) $7x - 19x + 6x$; (b) $-xy - xy$.
\end{baitoan}

\begin{proof}[Giải]
	(a) $7x - 19x + 6x = (7 - 19 + 6)x = -6x$. (b) $-xy - xy = xy(-1 - 1) = -2xy$.
\end{proof}

\begin{baitoan}[\cite{Tuyen_Toan_6}, \textbf{208.}, p. 41]
	Cho $A = -36m^2n^3$ với $m,n\in\mathbb{Z}$. Với giá trị nào của $m,n$ thì $A > 0$?
\end{baitoan}

\begin{proof}[Giải]
	$A > 0\Leftrightarrow-36m^2n^3 > 0\Leftrightarrow m^2n^3 < 0\Leftrightarrow m^2 > 0\land n^3 < 0\Leftrightarrow m\ne0\land n < 0$. Vậy $m,n\in\mathbb{Z}$, $m\ne0$, $n < 0\Rightarrow A > 0$.\footnote{Có thể viết gọn bằng công thức toán học 1 cách chặt chẽ như sau: $(m,n)\in\mathbb{Z}^\star\times\mathbb{Z}_{< 0}\Rightarrow A > 0$ trong đó $\mathbb{Z}^\star\coloneqq\mathbb{Z}\backslash\{0\} = \{a\in\mathbb{Z}|a\ne 0\}$ \& $\mathbb{Z}_{< 0}\coloneqq\{a\in\mathbb{Z}|a < 0\}$, \& $A\times B\coloneqq\{(a,b)|a\in A,\,b\in B\}$ ký hiệu \textit{tích Descartes} của 2 tập hợp $A,B$ sẽ được học sâu hơn ở Toán sơ cấp bậc THPT \& Toán cao cấp bậc Đại học.} 
\end{proof}

\begin{baitoan}[\cite{Tuyen_Toan_6}, \textbf{209.}, p. 41]
	Tìm $x\in\mathbb{Z}$ thỏa: (a) $-12(x - 5) + 7(3 - x) = 5$; (b) $30(x + 2) - 6(x - 5) - 24x = 100$.
\end{baitoan}

\begin{proof}[Giải]
	(a) $-12(x - 5) + 7(3 - x) = 5\Leftrightarrow-12x + 60 + 21 - 7x = 5\Leftrightarrow-(12 + 7)x = 5 - 60 - 21\Leftrightarrow-19x = -76\Leftrightarrow x = \frac{-76}{-19} = 4$. (b) $30(x + 2) - 6(x - 5) - 24x = 100\Leftrightarrow 30x + 60 - 6x + 30 - 24x = 100\Leftrightarrow(30 - 6 - 24)x = 100 - 60 - 30\Leftrightarrow0x = 10\Leftrightarrow0 = 10$, vô lý, không tồn tại $x$ thỏa mãn đẳng thức đã cho.
\end{proof}

\begin{baitoan}[\cite{Tuyen_Toan_6}, \textbf{210.}, p. 41]
	Tìm $x,y\in\mathbb{Z}$ biết: (a) $(x - 3)(2y + 1) = 7$; (b) $(x - 7)(x + 3) < 0$; (c) $(x - 7)(x + 3)\ge0$.
\end{baitoan}

\begin{proof}[Giải]
	(a) $(x - 3)(2y + 1) = 7 = 1\cdot 7 = (-1)(-7) = 7\cdot 1 = (-7)\cdot(-1)$ suy ra:
	\begin{equation*}
		\left[\begin{split}
			x - 3 &= 1,&&2y + 1 = 7,\\
			x - 3 &= -1,&&2y + 1 = -7,\\
			x - 3 &= 7,&&2y + 1 = 1,\\
			x - 3 &= -7,&&2y + 1 = -1,
		\end{split}\right.\Leftrightarrow
		\left[\begin{split}
			x &= 4,&&y = 3,\\
			x &= 2,&&y = -4,\\
			x &= 10,&&y = 0,\\
			x &= -4,&&y = -1,
		\end{split}\right.
	\end{equation*}
	Vậy $(x,y)\in\{(4,3),(2,-4),(10,0),(-4,-1)\}$. (b) $(x - 7)(x + 3) < 0\Leftrightarrow x - 7$ \& $x + 3$ trái dấu, vì $x - 7 < x + 3$ ($-7 < 3$) nên $x - 7< 0 < x + 3\Leftrightarrow-3 < x < 7$. Vậy $x\in\{-2,-1,0,1,2,3,4,5,6\}$. (c) 
\end{proof}

\begin{baitoan}[Mở rộng \cite{Tuyen_Toan_6}, \textbf{210.}, p. 41]
	Cho trước $a,b,c,d\in\mathbb{Z}$, \& $p$ là 1 số nguyên tố. Tìm $x,y\in\mathbb{Z}$ biết: (a) $(ax + b)(cy + d) = p$ với ; (b) $(x + a)(x + b) < 0$; (c) $(x + a)(x + b) > 0$; (d) $(x + a)(x + b)\le0$; (d) $(x + a)(x + b)\ge0$; (e) $(x + a)(x + b)(x + c) < 0$; (f) $(x + a)(x + b)(x + c) > 0$; (g) $(x + a)(x + b)(x + c)\le0$; (h) $(x + a)(x + b)(x + c)\ge0$; (i) $(x + a)(x + b)(x + c)(x + d) < 0$; (j) $(x + a)(x + b)(x + c)(x + d) > 0$; (k) $(x + a)(x + b)(x + c)(x + d)\le0$; (l) $(x + a)(x + b)(x + c)(x + d)\ge0$.
\end{baitoan}

\begin{baitoan}[\cite{Tuyen_Toan_6}, \textbf{211.}, p. 41]
	Tính hợp lý: (a) $125\cdot(-61)\cdot(-2)^3\cdot(-1)^{2n}$ với $n\in\mathbb{N}^\star$; (b) $136\cdot(-47) + 36\cdot47$; (c) $(-48)\cdot72 + 36\cdot(-304)$.
\end{baitoan}

\begin{proof}[Giải]
	(a) $125\cdot(-61)\cdot(-2)^3\cdot(-1)^{2n}= 125\cdot(-8)\cdot(-61) = -1000\cdot(-61) = 61000$. (b) $136\cdot(-47) + 36\cdot47 = 47(-136 + 36) = 47(-100) = -4700$. (c) $(-48)\cdot72 + 36\cdot(-304) = -48\cdot36\cdot2 + 36\cdot(-304) = -36(48\cdot2 + 304) = -36(96 + 304) = -36\cdot400 = -14400$.
\end{proof}

\begin{luuy}[Dấu của lũy thừa chẵn \& lẻ của số nguyên âm]
	$(-1)^{2n} = 1$, $(-1)^{2n+1} = -1$, $\forall n\in\mathbb{N}$.
\end{luuy}

\begin{baitoan}[\cite{Tuyen_Toan_6}, \textbf{212.}, p. 41]
	Tìm $x\in\mathbb{Z}$ thỏa: (a) $(x + 1) + (x + 3) + (x + 5) + \cdots + (x + 99) = 0$; (b) $(x - 3) + (x - 2) + (x - 1) + \cdots + 9 + 10 = 0$.
\end{baitoan}

\begin{baitoan}[\cite{Tuyen_Toan_6}, \textbf{213.}, p. 41]
	Cho $16$ số nguyên. Tích của 3 số bất kỳ luôn là 1 số âm. Chứng minh tích của $16$ số đó là 1 số dương.
\end{baitoan}

\begin{baitoan}[\cite{Tuyen_Toan_6}, \textbf{214.}, p. 41]
	Cho $A^2 = b(a - c) - c(a - b)$ với $a,b,c\in\mathbb{Z}$. Tính $A$ với $a = -20$, $b - c = -5$.
\end{baitoan}

\begin{baitoan}[\cite{Tuyen_Toan_6}, \textbf{215.}, p. 41]
	Biến đổi tổng thành tích: (a) $ab - ac + ad$; (b) $ac + ad - bc - bd$.
\end{baitoan}

\begin{baitoan}[\cite{Tuyen_Toan_6}, \textbf{216.}, p. 42]
	Cho $a,b,c\in\mathbb{Z}$. Biết $ab - ac + bc - c^2 = -1$. Chứng minh $a,c$ đối nhau.
\end{baitoan}

\begin{baitoan}[\cite{Tuyen_Toan_6}, \textbf{216.}, p. 42]
	1 tài khoản ngân hàng có số dư đầu tháng là $48$ triệu đồng. Trong tháng này người chủ tài khoản có giao dịch 5 lần trong đó 2 lần, mỗi lần $+9$ triệu đồng \& 3 lần, mỗi lần $-12$ triệu đồng. Tính số dư của tài khoản vào cuối tháng.
\end{baitoan}

\begin{baitoan}[\cite{Binh_Toan_6_tap_1}, Ví dụ 50, p. 43]
	\begin{itemize}
		\item[(a)] Cho bảng vuông $3\times 3$ ô:
		\begin{table}[H]
			\centering
			\begin{tabular}{|c|c|c|}
				\hline
				$\ \ 5$ & $\ \ 2$ & $-4$ \\
				\hline
				$-2$ & $-4$ & $-3$ \\
				\hline
				$-6$ & $\ \ 5$ & $\ \ 7$ \\
				\hline
			\end{tabular}
		\end{table}
		Tìm tích các số ở mỗi dòng, tích các số ở mỗi cột.
		\item[(b)] Viết $9$ số nguyên khác $0$ vào 1 bảng vuông $3\times 3$. Biết tích các số ở mỗi dòng đều là số âm. Chứng minh luôn luôn tồn tại 1 cột mà tích các số trong cột ấy là số âm.
	\end{itemize}
\end{baitoan}

\begin{baitoan}[\cite{Binh_Toan_6_tap_1}, Ví dụ 51, p. 44]
	Thay các dấu $\star$ trong biểu thức $1\star2\star3$ bằng các phép tính $+,-,\cdot,:$ \& thêm các dấu ngoặc để được kết quả là: số lớn nhất, số nhỏ nhất.
\end{baitoan}

\begin{baitoan}[\cite{Binh_Toan_6_tap_1}, \textbf{261.}, p. 44]
	Thực hiện các phép tính sau 1 cách nhanh chóng:
	\begin{enumerate*}
		\item[(a)] $(-14)\cdot(-125)\cdot3\cdot(-8)$;
		\item[(b)] $(-127)\cdot57 + (-127)\cdot43$;
		\item[(c)] $(-13)\cdot34 - 87\cdot34$;
		\item[(d)] $(-25)\cdot68 + (-34)\cdot(-250)$;
		\item[(e)] $A = 1 - 2 + 3 - 4 + \cdots + 99 - 100$;
		\item[(f)] $B = 1 + 3 - 5 - 7 + 9 + 11 - \cdots - 397 - 399$;
		\item[(g)] $C = 1 - 2 - 3 + 4 + 5 - 6 - 7 + \cdots + 97 - 98 - 99 + 100$;
		\item[(h)] $D = 2^{200} - 2^{99} - 2^{98} - \cdots - 2^2 - 2 - 1$.
	\end{enumerate*}
\end{baitoan}

\begin{baitoan}[\cite{Binh_Toan_6_tap_1}, \textbf{262.}, p. 44]
	Thay các dấu  $\star$ trong biểu thức $1\star2\star3\star4$ bằng dấu các phép tính $+,-,\cdot,:$ \& thêm các dấu ngoặc để được kết quả là: số lớn nhất, số nhỏ nhất.
\end{baitoan}

\begin{baitoan}[\cite{Binh_Toan_6_tap_1}, \textbf{263.}, p. 44]
	Tìm $x\in\mathbb{Z}$ sao cho:
	\begin{enumerate*}
		\item[(a)] $(x - 1)^2 = 0$;
		\item[(b)] $x(x - 1) = 0$;
		\item[(c)] $(x + 1)(x - 2) = 0$.
	\end{enumerate*}
\end{baitoan}

\begin{luuy}
	Bình phương của mọi số nguyên đều không âm, i.e., $x^2\ge0$, $\forall x\in\mathbb{Z}$. Dấu bằng\emph{\texttt{/}}đẳng thức xảy ra $\Leftrightarrow x = 0$.
\end{luuy}

\begin{proof}[Giải]
	(a) $(x - 1)^2 = 0\Leftrightarrow x - 1 = 0\Leftrightarrow x = 1$. (b) \& (c):
	\begin{equation*}
		x(x - 1) = 0\Leftrightarrow\left[\begin{split}
			x &= 0\\
			x - 1 &= 0
		\end{split}\right.\Leftrightarrow\left[\begin{split}
			x &= 0\\
			x &= 1
		\end{split}\right.,\ (x + 1)(x - 2) = 0\Leftrightarrow\left[\begin{split}
			x + 1 &= 0\\
			x - 2 &= 0
		\end{split}\right.\Leftrightarrow\left[\begin{split}
			x &= -1\\
			x &= 2
		\end{split}\right..
	\end{equation*}
\end{proof}

\begin{baitoan}[\cite{Binh_Toan_6_tap_1}, \textbf{264.}, p. 44]
	Cho dãy số $a_1,a_2,\ldots,a_{100}$ trong đó $a_1 = 1$, $a_2 = -1$, $a_k = a_{k-2}a_{k-1}$, $k\in\mathbb{N}$, $k\ge 3$. Tính $a_{100}$.
\end{baitoan}

\begin{baitoan}[\cite{Binh_Toan_6_tap_1}, \textbf{265.}, p. 44]
	Gọi $a,b,c,d,e,f,g,h$ là các số khác nhau trong tập hợp số $\{-7,-5,-3,-2,2,4,6,13\}$. Tính giá trị lớn nhất của biểu thức $A = (a + b + c + d)^2 + (e + f + g + h)^2$.
\end{baitoan}

%------------------------------------------------------------------------------%

\section{Tính Chia hết Trên $\mathbb{Z}$. Bội \& Ước của 1 Số Nguyên}
``\textbf{1.} Cho $a,b\in\mathbb{Z}$, $b\ne0$. Nếu có 1 số nguyên $q$ sao cho $a = bq$ thì ta nói $a$ \textit{chia hết cho} $b$, ký hiệu $a\divby b$, \& $a$ chia cho $b$ được $q$ (viết là $a:b = q$ hoặc $\frac{a}{b} = q$). Ta còn nói $a$ là \textit{bội} của $b$ \& $b$ là \textit{ước} của $a$. \textbf{2.} Quy tắc về dấu của phép chia 2 số nguyên cũng giống như quy tắc về dấu của phép nhân 2 số nguyên. \textbf{3.} Số $0$ là bội của mọi số nguyên khác $0$, i.e., $0\divby a$, $0\in\mbox{B}(a)$, $\forall a\in\mathbb{Z}$, $a\ne0$. Số $0$ không phải là ước của bất kỳ số nguyên nào, i.e., $0\notin\mbox{Ư}(a)$, $\forall a\in\mathbb{Z}$. Số $1$ \& $-1$ là ước của mọi số nguyên, i.e., $a\divby\pm1$, $\pm1\in\mbox{Ư}(a)$, $\forall a\in\mathbb{Z}$. \textbf{4.} \textit{Tính chất của phép chia hết}: Tính chất bắc cầu: $a\divby b\land b\divby c\Rightarrow a\divby c$, $\forall a,b,c\in\mathbb{Z}$. $a\divby b\Rightarrow an\divby b$, $\forall n\in\mathbb{Z}$. $a\divby n\land b\divby n\Rightarrow a\pm b\divby n$. Tổng quát hơn, $a\divby n\land b\divby n\Rightarrow xa\pm yb\divby n$, $\forall x,y\in\mathbb{Z}$. \textbf{5.} Nếu $a$ là bội của $b$ thì $-a$ cũng là bội của $b$, i.e., $a\divby b\Rightarrow-a\divby b$, hay $a\in\mbox{B}(b)\Rightarrow-a\in\mbox{B}(b)$. Nếu $b$ là ước của $a$ thì $-b$ cũng là ước của $a$, i.e., $b|a\Leftrightarrow-b|a$, hay $b\in\mbox{Ư}(a)\Leftrightarrow-b\in\mbox{Ư}(a)$. Do đó nếu 1 số nguyên $a$ có $k$ ước tự nhiên thì $a$ có thêm $k$ ước âm (là số đối của các ước tự nhiên)\footnote{Số phần tử của tập hợp các ước nguyên của 1 số sẽ gấp đôi số phần tử của tập hợp các ước tự nhiên của 1 số vì phải thêm dấu $-$ vào trước tất cả các ước tự nhiên của số đó, i.e., $|\mbox{Ư}(a)\cap\mathbb{Z}| = 2|\mbox{Ư}(a)\cap\mathbb{N}|$ với $|A|$ ký hiệu số phần tử của 1 tập hợp $A$.}.'' -- \cite[Chap. II, \S5, p. 42]{Tuyen_Toan_6}

Các ước nguyên của 1 số nguyên sẽ bao gồm các ước tự nhiên của số nguyên đó \& các số đối của các ước tự nhiên đó, i.e., $\mbox{Ư}(a)\cap\mathbb{Z} = (\mbox{Ư}(a)\cap\mathbb{N})\cap\{-n|n\in\mbox{Ư}(a)\cap\mathbb{N}\} = \{\pm n|n\in\mbox{Ư}(a)\cap\mathbb{N}\}$, $\forall a\in\mathbb{Z}$. 2 tập hợp ước nguyên của 2 số nguyên đối nhau trùng nhau, i.e., $\mbox{Ư}(a)\cap\mathbb{Z} = \mbox{Ư}(-a)\cap\mathbb{Z}$, $\forall a\in\mathbb{Z}$.

\begin{baitoan}[\cite{Tuyen_Toan_6}, Ví dụ 45, p. 42]
	Tìm tất cả các ước của $-24$.
\end{baitoan}

\begin{proof}[Giải]
	Vì $\mbox{Ư}(-24) = \mbox{Ư}(24)$ nên chỉ cần tìm $\mbox{Ư}(24)$. Các ước tự nhiên của $24$ là: $1,2,3,4,6,8,12,24$.\footnote{Cũng có thể viết bằng công thức toán học 1 cách chặt chẽ là: $\mbox{Ư}(24)\cap\mathbb{N} = \{1,2,3,4,6,8,12,24\}$.} Thêm các số đối của chúng, được $\mbox{Ư}(-24) = \mbox{Ư}(24) = \{\pm1,\pm2,\pm3,\pm4,\pm6,\pm8,\pm12,\pm24\}$.
\end{proof}

\begin{nhanxet}
	``Để tìm tất cả các ước của 1 số nguyên âm ta chỉ cần tìm tất cả các ước của số đối của số nguyên âm đó. Trước tiên, tìm các ước tự nhiên rồi thêm các ước đối của chúng.'' -- \cite[p. 42]{Tuyen_Toan_6}
\end{nhanxet}

\begin{baitoan}[\cite{Tuyen_Toan_6}, Ví dụ 46, p. 43]
	Cho $a,b\in\mathbb{Z}$, $a\ne0$, $b\ne0$. Biết $a\divby b$ \& $b\divby a$. Chứng minh $a = \pm b$.
\end{baitoan}

\begin{proof}[Giải]
	$a\divby b\Rightarrow a = bq_1$, với $q_1\in\mathbb{Z}$. $b\divby a\Rightarrow b = aq_2$, với $q_2\in\mathbb{Z}$. Suy ra $a = bq_1 = (aq_2)q_1 = a(q_1q_2)$, vì $a\ne0$ suy ra $q_1q_2 = 1$, suy ra $q_1 = q_2 = 1$ hoặc $q_1 = q_2 = -1$, hay $a = b$ hoặc $a = -b$. Vậy $a = \pm b$.
\end{proof}

\begin{baitoan}[\cite{Tuyen_Toan_6}, \textbf{218.}, p. 43]
	Các số sau có bao nhiêu ước? (a) $54$; (b) $-196$.
\end{baitoan}

\begin{proof}[1st giải]
	Giải bằng cách liệt kê ước: (a) $\mbox{Ư}(54)\cap\mathbb{Z} = \{\pm1,\pm2,\pm3,\pm6,\pm9,\pm18,\pm27,\pm54\}$ có tất cả 16 ước. (b) $\mbox{Ư}(196)\cap\mathbb{Z} = \{\pm1,\pm2,\pm4,\pm7,\pm14,\pm28,\pm49,\pm98,\pm196\}$ có tất cả $18$ ước.
\end{proof}

\begin{lemma}
	\label{lemma: number of divisor}
	Nếu $a\in\mathbb{N}^\star$, $a\ge2$, có phân tích ra thừa số nguyên tố là $a = p_1^{a_1}p_2^{a_2}\cdots p_n^{a_n}$ với $a_i\in\mathbb{N}^\star$, $p_i$ là số nguyên tố, $\forall i = 1,\ldots,n$, thì số ước dương của $a$ là $(p_1 + 1)(p_2 + 1)\cdots(p_n + 1)$ \& do đó số ước nguyên của $a$ là $2(p_1 + 1)(p_2 + 1)\cdots(p_n + 1)$, i.e., $|\mbox{Ư}(a)\cap\mathbb{N}| = (p_1 + 1)(p_2 + 1)\cdots(p_n + 1)$ \& $|\mbox{Ư}(a)\cap\mathbb{Z}| = 2(p_1 + 1)(p_2 + 1)\cdots(p_n + 1)$.
\end{lemma}

\begin{proof}[2nd giải]
	Sử dụng bổ đề \ref{lemma: number of divisor}: (a) $54 = 2\cdot3^3\Rightarrow|\mbox{Ư}(54)\cap\mathbb{Z}| = 2(1 + 1)(3 + 1) = 16$. Vậy $54$ có $16$ ước nguyên. (b) $196 =2^2\cdot7^2\Rightarrow|\mbox{Ư}(196)\cap\mathbb{Z}| = 2(2 + 1)(2 + 1) = 18$. Vậy $196$ có $18$ ước nguyên.
\end{proof}

\begin{baitoan}[\cite{Tuyen_Toan_6}, \textbf{219.}, p. 43]
	Cho $a,b,x,y\in\mathbb{Z}$, trong đó $x,y$ không đối nhau. Chứng minh nếu $(ax - by)\divby(x + y)$  thì $(ay - bx)\divby(x + y)$.
\end{baitoan}

\begin{baitoan}[\cite{Tuyen_Toan_6}, \textbf{220.}, p. 43]
	Cho $S = 1 - 3 + 3^2 - 3^3 + \cdots + 3^{98} - 3^{99}$. (a) Chứng minh $S$ là bội của $-20$; (b) Tính $S$ từ đó suy ra $3^{100}$ chia cho $4$ dư $1$.
\end{baitoan}

\begin{baitoan}[\cite{Tuyen_Toan_6}, \textbf{221.}, p. 43]
	Tìm $n\in\mathbb{N}^\star$ sao cho $n + 2$ là ước của $111$ còn $n - 2$ là bội của $11$.
\end{baitoan}

\begin{baitoan}[\cite{Tuyen_Toan_6}, \textbf{222.}, p. 43]
	Tìm $n\in\mathbb{Z}$ thỏa: (a) $(4n - 5)\divby n$; (b) $-11$ là bội của $n - 1$; (c) $2n - 1$ là ước của $3n + 2$.
\end{baitoan}

\begin{baitoan}[\cite{Tuyen_Toan_6}, \textbf{223.}, p. 43]
	Tìm $n\in\mathbb{Z}$ sao cho: $n - 1$ là bội của $n + 5$ \& $n + 5$ là bội của $n - 1$.
\end{baitoan}

\begin{baitoan}[\cite{Tuyen_Toan_6}, \textbf{224.}, p. 43]
	Ở 1 thành phố xứ lạnh, nhiệt độ thấp nhất trong mỗi ngày của 1 tuần lễ là: $-8^\circ$C, $-6^\circ$C, $-5^\circ$C, $-4^\circ$C, $-1^\circ$C, $+1^\circ$C, \& $2^\circ$C. Tính nhiệt độ trung bình thấp nhất trong tuần lễ đó của thành phố này. 
\end{baitoan}

\begin{baitoan}[\cite{Tuyen_Toan_6}, \textbf{225.}, p. 43]
	Hà làm bài kiểm tra trắc nghiệm gồm $25$ câu. Mỗi câu làm đúng được $3$ điểm, làm sai được $(-2)$ điểm \& không làm câu nào thì câu ấy không có điểm. Biết Hà làm đúng được $20$ câu nhưng chỉ được $54$ điểm. Hỏi Hà đã làm sai mấy câu?
\end{baitoan}

\begin{baitoan}[\cite{Binh_Toan_6_tap_1}, Ví dụ 52, p. 44]
	Số $36$ chia cho $a\in\mathbb{Z}$ rồi trừ đi $a$. Lấy kết quả này chia cho $a$ rồi trừ đi $a$. Lại lấy kết quả này chia cho $a$ rồi trừ đi $a$. Cuối cùng ta được số $-a$. Tìm $a$.\hfill{\sf Ans:} $3$.
\end{baitoan}

\begin{proof}[Giải]
	Có phương trình: $\left[\left(\frac{36}{a} - a\right):a - a\right]:a - a = -a\Leftrightarrow\left[\left(\frac{36}{a} - a\right):a - a\right]:a = 0\Rightarrow\left(\frac{36}{a} - a\right):a - a = 0\Rightarrow\left(\frac{36}{a} - a\right):a = a\Rightarrow\frac{36}{a} - a = a^2\Rightarrow\frac{36}{a} = a^2 + a = a(a + 1)\Rightarrow36 = a^2(a + 1)$. Do $a\in\mathbb{Z}$ nên $a^2\in\mbox{Ư}(36)$, nên $a^2\{1,4,9,36\}$ suy ra $a\in A\coloneqq\{\pm1,\pm2,\pm3,\pm4\}$. Từ phương trình cuối suy ra $a + 1 = \frac{36}{a^2}\in\{36,9,4,1\}$, \& tương ứng với $a\in B\coloneqq\{35,8,3,0\}$. Tổng hợp lại, $a\in A\cap B = \{3\}$. Vậy $a = 3$.
\end{proof}

\begin{baitoan}[\cite{Binh_Toan_6_tap_1}, \textbf{266.}, p. 45]
	Tìm $x,y\in\mathbb{Z}$ biết:
	\begin{enumerate*}
		\item[(a)] $(x + 2)(y - 3) = 5$;
		\item[(b)] $(x + 1)(xy - 1) = 3$.
	\end{enumerate*}
\end{baitoan}

\begin{proof}[Giải]
	(a) $x + 2\in\mbox{Ư}(5) = \{\pm1,\pm5\}\Rightarrow x\in\{-1,-3,3,-7\}$. Đáp số: $(x,y)\in\{(-1,8),(-3,-2),(3,4),(-7,2)\}$. (b) $(x,y)\in\{(-2,1),(2,1),(-4,0)\}$.
\end{proof}

\begin{baitoan}[\cite{Binh_Toan_6_tap_1}, \textbf{267.}, p. 45]
	Tính tổng $A + B$ biết $A$ là tổng các số nguyên âm lẻ có 2 chữ số, $B$ là tổng các số nguyên dương chẵn có 2 chữ số.
\end{baitoan}

\begin{baitoan}[\cite{Binh_Toan_6_tap_1}, \textbf{268.}, p. 45]
	Cho $A = 2 - 5 + 8 - 11 + 14 - 17 + \cdots + 98 - 101$.
	\begin{enumerate*}
		\item[(a)] Viết dạng tổng quát của số hạng thứ $n$ của $A$.
		\item[(b)] Tính giá trị của biểu thức $A$.
	\end{enumerate*}
\end{baitoan}

\begin{baitoan}[\cite{Binh_Toan_6_tap_1}, \textbf{269.}, p. 45]
	Cho $A = 1 + 2 - 3 - 4 + 5 + 6 - \cdots - 99 - 100$.
	\begin{enumerate*}
		\item[(a)] $A$ có chia hết cho $2$, cho $3$, cho $5$ hay không?
		\item[(b)] $A$ có bao nhiêu ước nguyên, có bao nhiêu ước tự nhiên?
	\end{enumerate*}
\end{baitoan}

\begin{baitoan}[\cite{Binh_Toan_6_tap_1}, \textbf{270.}, p. 45]
	Cho dãy số $1,-3,5,-7,9,-11,13,-15,17,-19$. Có thể tìm được hay không 5 số trong các số trên, sao cho đặt dấu ``$+$'' hoặc ``$-$'' nối các số đó với nhau, ta được kết quả bằng:
	\begin{enumerate*}
		\item[(a)] $15$;
		\item[(b)] $20$?
	\end{enumerate*}
\end{baitoan}

\begin{baitoan}[\cite{Binh_Toan_6_tap_1}, \textbf{271.}, p. 45]
	Thay các dấu  $\star$ trong biểu thức $1\star2\star3\star4\star5\star6\star7\star8\star9$ bởi các dấu ``$+$'' hoặc ``$-$'' để giá trị của biểu thức bằng:
	\begin{enumerate*}
		\item[(a)] $-13$;
		\item[(b)] $-4$?
	\end{enumerate*}
\end{baitoan}

\begin{baitoan}[\cite{Binh_Toan_6_tap_1}, \textbf{272.}, p. 45]
	Tìm $n\in\mathbb{Z}$ sao cho:
	\begin{enumerate*}
		\item[(a)] $n + 5\divby n - 2$;
		\item[(b)] $2n + 1\divby n - 5$;
		\item[(c)] $n^2 + 3n - 13\divby n + 3$;
		\item[(d)] $n^2 + 3\divby n - 1$.
	\end{enumerate*}
\end{baitoan}

\begin{baitoan}[\cite{Binh_Toan_6_tap_1}, \textbf{273.}, p. 45]
	Tìm các số $a,b,c,d,m$ khác nhau thuộc tập hợp $\{-2,-1,0,1,2\}$ sao cho $a < b < \min\{c,d\}$, với $\min\{c,d\}$ là số nhỏ hơn trong 2 số $c,d$, \& đặt $m$ nằm ở trung tâm, các số $a,b,c,d$ lần lượt nằm ở bên trái, bên trên, bên phải, bên dưới của $m$, \& tổng của 3 số trên đường nằm ngang bằng tổng của 3 số trên đường thẳng đứng.
\end{baitoan}

\begin{baitoan}[\cite{Binh_Toan_6_tap_1}, \textbf{274.}${}^\star$, p. 45]
	Cho $n$ số nguyên (có thể có số âm) với $n > 1$ mà tổng \& tích của chúng đều bằng $505$. Tìm giá trị nhỏ nhất của $n$.
\end{baitoan}

%------------------------------------------------------------------------------%

\section{Điền Chữ Số}
``Các bài toán về điền chữ số không chỉ yêu cầu kỹ năng tính toán đúng mà còn đòi hỏi cả lập luận chính xác \& hợp lý. Ta quy ước rằng khi ở đề bài cho các chữ $a,b,c,\ldots$ mà không chú thích gì thêm, ta hiểu rằng các chữ khác nhau biểu thị các chữ số khác nhau.'' -- \cite[p. 46]{Binh_Toan_6_tap_1}

\begin{baitoan}[\cite{Binh_Toan_6_tap_1}, Ví dụ 53, p. 46]
	Thay các chữ bởi các chữ số thích hợp: $\overline{abc} + \overline{acb} = \overline{bca}$.
\end{baitoan}

\begin{baitoan}[\cite{Binh_Toan_6_tap_1}, Ví dụ 54, p. 46]
	Tìm các chữ số $a,b,c$ biết tổng $a + b + c$ bằng tổng của 4 số chẵn liên tiếp \& các chữ số $a,b,c$ thỏa mãn cả 2 phép trừ sau: $\overline{abc} - \overline{cba} = 99$ \& $\overline{bac} - \overline{abc} = 270$.
\end{baitoan}

\begin{baitoan}[\cite{Binh_Toan_6_tap_1}, Ví dụ 55, p. 46]
	Thay các dấu {\bf*} bằng các chữ số thích hợp trong phép chia:
	\begin{figure}[H]
		\centering
		\includegraphics[scale=0.13]{Binh_vi_du_55_p_47}
	\end{figure}
\end{baitoan}

\begin{baitoan}[\cite{Binh_Toan_6_tap_1}, Ví dụ 56, p. 47]
	Thay các chữ $a,b,c$ bằng các chữ số khác nhau thích hợp trong phép nhân sau: $\overline{ab}\cdot\overline{cc}\cdot\overline{abc} = \overline{abcabc}$.
\end{baitoan}

\begin{baitoan}[\cite{Binh_Toan_6_tap_1}, Ví dụ 57, p. 47]
	Tìm số tự nhiên có 3 chữ số, biết trong 2 cách viết: viết thêm chữ số $5$ vào đằng sau số đó hoặc viết thêm chữ số $1$ vào đằng trước số đó thì cách viết thứ nhất cho số lớn gấp $5$ lần so với cách viết thứ 2.
\end{baitoan}

\begin{baitoan}[\cite{Binh_Toan_6_tap_1}, Ví dụ 58, p. 48]
	Điền các chữ số thích hợp vào các chữ trong phép nhân sau: $2\overline{abcdmn} = \overline{cdmnab}$.
\end{baitoan}

\begin{baitoan}[\cite{Binh_Toan_6_tap_1}, Ví dụ 59, p. 48]
	Điền các chữ số thích hợp vào các dấu $\star$ trong phép nhân sau: $\star\star\cdot\star\star = \star\star\star$ biết cả 2 thừa số đều chẵn \& tích là số có 3 chữ số như nhau.
\end{baitoan}

\begin{baitoan}[\cite{Binh_Toan_6_tap_1}, Ví dụ 60, p. 48]
	Tìm các chữ số $a$ \& $b$, biết $900:(a + b) = \overline{ab}$.
\end{baitoan}

\begin{baitoan}[\cite{Binh_Toan_6_tap_1}, Ví dụ 61, p. 49]
	Chứng minh không thể thay các chữ bằng các chữ số để có phép tính đúng:
	\begin{enumerate*}
		\item[(a)] \emph{HỌC VUI $-$ VUI HỌC} $= 1991$;
		\item[(b)] \emph{TOÁN $+$ LÝ $+$ SỬ $+$ VẼ} $= 1992$.
	\end{enumerate*}
\end{baitoan}
Thay các dấu $\star$ \& các chữ bởi các số thích hợp:

\begin{baitoan}[\cite{Binh_Toan_6_tap_1}, \textbf{275.}, p. 49]
	$\overline{ab} + \overline{bc} + \overline{ca} = \overline{abc}$.
\end{baitoan}

\begin{baitoan}[\cite{Binh_Toan_6_tap_1}, \textbf{276.}, p. 49]
	\begin{enumerate*}
		\item[(a)] $\overline{abc} + \overline{ab} + a = 874$;
		\item[(b)] $\overline{abc} + \overline{ab} + a = 1037$.
	\end{enumerate*}
\end{baitoan}

\begin{baitoan}[\cite{Binh_Toan_6_tap_1}, \textbf{277.}, p. 49]
	\begin{enumerate*}
		\item[(a)] $\overline{acc}\cdot b = \overline{dba}$ biết $a$ là chữ số lẻ;
		\item[(b)] $\overline{ac}\cdot\overline{ac} = \overline{acc}$;
		\item[(c)] $\overline{ab}\cdot\overline{ab} = \overline{acc}$.
	\end{enumerate*}
\end{baitoan}

\begin{baitoan}[\cite{Binh_Toan_6_tap_1}, \textbf{278.}, p. 49]
	\begin{enumerate*}
		\item[(a)] $2\overline{1bac} = \overline{abc8}$;
		\item[(b)] $\overline{ab} = 9b$.
	\end{enumerate*}
\end{baitoan}

\begin{baitoan}[\cite{Binh_Toan_6_tap_1}, \textbf{279.}, p. 49]
	$4\overline{abcdef} = \overline{fabcde}$ \& $\overline{abcde} + f = 15390$.
\end{baitoan}

\begin{baitoan}[\cite{Binh_Toan_6_tap_1}, \textbf{280.}, p. 49]
	$\overline{abc} - \overline{ca} = \overline{ca} - \overline{ac}$.
\end{baitoan}

\begin{baitoan}[\cite{Binh_Toan_6_tap_1}, \textbf{281.}, p. 49]
	$\overline{abcd} + \overline{abc} = 3576$.
\end{baitoan}

\begin{baitoan}[\cite{Binh_Toan_6_tap_1}, \textbf{282.}, p. 49]
	$\overline{abcd0} - \overline{abcd} = \overline{3462\star}$.
\end{baitoan}

\begin{baitoan}[\cite{Binh_Toan_6_tap_1}, \textbf{283.}, p. 49]
	Thay các dấu {\bf*} bởi các số thích hợp:
	\begin{figure}[H]
		\centering
		\includegraphics[scale=0.13]{Binh_194_p_49}
	\end{figure}
	biết số bị nhân có tổng các chữ số bằng $18$ \& không đổi khi đọc từ phải sang trái.
\end{baitoan}

\begin{baitoan}[\cite{Binh_Toan_6_tap_1}, \textbf{284.}, p. 49]
	\begin{enumerate*}
		\item[(a)] $\overline{ab}\cdot b = \overline{1ab}$;
		\item[(b)] $\overline{abc} = 9\overline{bc}$.
	\end{enumerate*}
\end{baitoan}

\begin{baitoan}[\cite{Binh_Toan_6_tap_1}, \textbf{285.}, p. 50]
	$\overline{260abc}:\overline{abc} = 626$.
\end{baitoan}

\begin{baitoan}[\cite{Binh_Toan_6_tap_1}, \textbf{286.}, p. 50]
	Thay các dấu {\bf*} bởi các số thích hợp:
	\begin{figure}[H]
		\centering
		\includegraphics[scale=0.13]{Binh_286_p_50}
	\end{figure}
\end{baitoan}

\begin{baitoan}[\cite{Binh_Toan_6_tap_1}, \textbf{287.}, p. 50]
	\begin{enumerate*}
		\item[(a)] $\overline{ab}\cdot\overline{cb} = \overline{ddd}$;
		\item[(b)] $\star\star\cdot\,\star = \star\star\star$;
		\item[(c)] $\overline{ab}\cdot\overline{cd} = bbb$. Biết tích là số có 3 chữ số như nhau.
	\end{enumerate*}
\end{baitoan}

\begin{baitoan}[\cite{Binh_Toan_6_tap_1}, \textbf{288.}, p. 50]
	$6\overline{abcdef} = \overline{defabc}$.
\end{baitoan}

\begin{baitoan}[\cite{Binh_Toan_6_tap_1}, \textbf{289.}, p. 50]
	$20\star\star:13 = \star\star7$.
\end{baitoan}

\begin{baitoan}[\cite{Binh_Toan_6_tap_1}, \textbf{290.}, p. 50]
	Thay các dấu {\bf*} bởi các số thích hợp:
	\begin{figure}[H]
		\centering
		\includegraphics[scale=0.13]{Binh_290_p_50}
	\end{figure}
\end{baitoan}

\begin{baitoan}[\cite{Binh_Toan_6_tap_1}, \textbf{291.}, p. 50]
	$\overline{abc}:11 = a + b + c$.
\end{baitoan}

\begin{baitoan}[\cite{Binh_Toan_6_tap_1}, \textbf{292.}, p. 50]
	$(\overline{ab} + \overline{cd})(\overline{ab} - \overline{cd}) = 2002$.
\end{baitoan}

\begin{baitoan}[\cite{Binh_Toan_6_tap_1}, \textbf{293.}, p. 50]
	\begin{enumerate*}
		\item[(a)] $a\cdot\overline{bc} = d\cdot\overline{ef} = 156$ (các chữ khác các chữ số đã có);
		\item[(b)] $\overline{ab}\cdot\overline{cde} = 16038$ (các chữ khác các chữ số đã có).
	\end{enumerate*}
\end{baitoan}

\begin{baitoan}[\cite{Binh_Toan_6_tap_1}, \textbf{294.}, p. 50]
	Tìm chữ số $a$ sao cho $n = \overline{\underbrace{4\ldots4}_{\scriptsize55\mbox{ số}}a\underbrace{6\ldots6}_{\scriptsize55\mbox{ số}}}\divby13$.
\end{baitoan}

\begin{baitoan}[\cite{Binh_Toan_6_tap_1}, \textbf{295.}, p. 50]
	Tìm chữ số $a$ \& $x\in\mathbb{N}$ sao cho: $(12 + 3x)^2 = \overline{1a96}$.
\end{baitoan}

\begin{baitoan}[\cite{Binh_Toan_6_tap_1}, \textbf{296.}, p. 50]
	Tìm số tự nhiên có 5 chữ số, biết rằng nếu viết thêm chữ số $7$ vào đằng trước số đó thì được 1 số lớn gấp $4$ lần so với số có được bằng cách viết thêm chữ số $7$ vào sau số đó.
\end{baitoan}

\begin{baitoan}[\cite{Binh_Toan_6_tap_1}, \textbf{297.}, p. 50]
	Tìm số tự nhiên có 2 chữ số, biết rằng nếu viết thêm 1 chữ số $2$ vào bên phải \& 1 chữ số $2$ vào bên trái của nó thì số ấy tăng gấp $36$ lần.
\end{baitoan}

\begin{baitoan}[\cite{Binh_Toan_6_tap_1}, \textbf{298.}, p. 50]
	Tìm số tự nhiên có 2 chữ số, biết rằng nếu viết xen vào giữa 2 chữ số của nó chính số đó thì số đó tăng gấp $99$ lần.
\end{baitoan}

\begin{baitoan}[\cite{Binh_Toan_6_tap_1}, \textbf{299.}, p. 50]
	Tìm số tự nhiên có 4 chữ số, sao cho khi nhân số đó với $4$ ta được số gồm 4 chữ số ấy viết theo thứ tự ngược lại.
\end{baitoan}

\begin{baitoan}[\cite{Binh_Toan_6_tap_1}, \textbf{300.}, p. 50]
	Tìm số tự nhiên có 4 chữ số, sao cho nhân nó với $9$ ta được số gồm chính các chữ số của số ấy viết theo thứ tự ngược lại.
\end{baitoan}

\begin{baitoan}[\cite{Binh_Toan_6_tap_1}, \textbf{301.}, p. 51]
	Tìm số tự nhiên có 5 chữ số, sao cho nhân nó với $9$ ta được số gồm chính các chữ số của số ấy viết theo thứ tự ngược lại.
\end{baitoan}

\begin{baitoan}[\cite{Binh_Toan_6_tap_1}, \textbf{302.}, p. 51]
	\begin{enumerate*}
		\item[(a)] Tìm số tự nhiên có 3 chữ số, biết rằng nếu xóa chữ số hàng trăm thì số ấy giảm $9$ lần.
		\item[(b)] Giải bài toán trên nếu không cho biết chữ số bị xóa thuộc hàng nào.
	\end{enumerate*}
\end{baitoan}

\begin{baitoan}[\cite{Binh_Toan_6_tap_1}, \textbf{303.}, p. 51]
	Tìm $n\in\mathbb{N}$ có 3 chữ số khác nhau, biết rằng nếu xóa bất kỳ chữ số nào của nó ta cũng được 1 số là ước của $n$.	
\end{baitoan}

\begin{baitoan}[\cite{Binh_Toan_6_tap_1}, \textbf{304.}, p. 51]
	Tìm số tự nhiên có 4 chữ số, biết rằng nếu xóa chữ số hàng nhìn thì số ấy giảm $9$ lần.
\end{baitoan}

\begin{baitoan}[\cite{Binh_Toan_6_tap_1}, \textbf{305.}, p. 51]
	\begin{enumerate*}
		\item[(a)] Tìm số tự nhiên có 4 chữ số, biết rằng chữ số hàng trăm bằng $0$ \& nếu xóa chữ số $0$ đó thì số ấy giảm $9$ lần.
		\item[(b)] 1 số tự nhiên tăng gấp $9$ lần nếu viết thêm 1 chữ số $0$ vào giữa các chữ số hàng chục \& hàng đơn vị của nó. Tìm số ấy.
	\end{enumerate*}	
\end{baitoan}

\begin{baitoan}[\cite{Binh_Toan_6_tap_1}, \textbf{306.}, p. 51]
	Tìm $A\in\mathbb{N}$, biết rằng nếu xóa 1 hoặc nhiều chữ số tận cùng của nó thì được số $B$ mà $A = 130B$.
\end{baitoan}

\begin{baitoan}[\cite{Binh_Toan_6_tap_1}, \textbf{307.}${}^\star$, p. 51]
	Tìm $x\in\mathbb{N}$ có chữ số tận cùng bằng $2$, biết rằng $x,2x,3x$ đều là các số có 3 chữ số \& 9 chữ số của 3 số đó đều khác nhau \& khác $0$.
\end{baitoan}

\begin{baitoan}[\cite{Binh_Toan_6_tap_1}, \textbf{308.}${}^\star$, p. 51]
	Tìm $x\in\mathbb{N}$ có 6 chữ số, biết rằng các tích $2x,3x,4x,5x,6x$ cũng là số có 6 chữ số gồm cả 6 chữ số ấy.
	\begin{enumerate*}
		\item[(a)] Cho biết 6 chữ số của số phải tìm là $1,2,4,5,7,8$.
		\item[(b)] Giải bài toán nếu không cho điều kiện (a).
	\end{enumerate*}
\end{baitoan}

%------------------------------------------------------------------------------%

\section{Dãy Các Số Viết Theo Quy Luật}

\subsection{Dãy cộng}
``Xét các dãy số sau:
\begin{enumerate*}
	\item[(a)] Dãy số tự nhiên: $0,1,2,3,\ldots$;
	\item[(b)] Dãy số lẻ: $1,3,5,7,\ldots$;
	\item[(c)] Dãy các số chia cho $3$ dư $1$: $1,4,7,10,\ldots$
\end{enumerate*}
Trong các dãy số trên, mỗi số hạng, kể từ số hạng thứ 2, đều lớn hơn số hạng đứng liền trước nó cùng 1 đơn vị, số đơn vị này là $1$ ở dãy (a), là $2$ ở dãy (b), là $3$ ở dãy (c). Ta gọi các dãy trên là \textit{dãy cộng}.

Xét dãy cộng $4,7,10,13,16,19,\ldots$ Hiệu giữa 2 số liên tiếp của dãy là $3$. Số hạng thứ 6 của dãy này là $19$, bằng: $4 + (6 - 1)\cdot3$; số hạng thứ $10$ của dãy này là $4 + (10 - 1)\cdot3 = 31$. Tổng quát, nếu 1 dãy cộng có số hạng đầu là $a_1$ \& hiệu giữa 2 số hạng liên tiếp là $d$ thì số hạng thứ $n$ của dãy cộng đó (ký hiệu $a_n$) bằng: $a_n = a_1 + (n - 1)d$, $\forall n\in\mathbb{N}^\star$. Để tính tổng các số hạng của dãy cộng $4 + 7 + 10 + \cdots + 25 + 28 + 31$ (gồm $10$ số) ta viết: $A = 4 + 7 + 10 + \cdots + 25 + 28 + 31$, $A = 31 + 28 + 25 + \cdots + 10 + 7 + 4$ nên $2A = (4 + 31) + (7 + 28) + \cdots + (28 + 7) + (31 + 4) = (4 + 31)\cdot10$. Do đó $A = \frac{(4 + 31)\cdot10}{2} = 175$.

Tổng quát, nếu 1 dãy cộng có $n$ số hạng, số hạng đầu là $a_1$, số hạng cuối là $a_n$ thì tổng của $n$ số hạng đó được tính như sau: $S = \frac{(a_1 + a_n)\cdot n}{2}$. Quy tắc dân gian: dĩ đầu, cộng vĩ, chiết bán, nhân chi (lấy số đầu cộng với số cuối, chia đôi, nhân với số số hạng). Trường hợp đặc biệt, tổng của $n$ số tự nhiên liên tiếp bắt đầu từ 1 bằng: $\sum_{i=1}^n i = 1 + 2 + \cdots + n = \frac{1}{2}n(n + 1)$.'' -- \cite[Chuyên đề 2, pp. 51--52]{Binh_Toan_6_tap_1} (Cho $a_1 = 1$, $a_n = n$ trong công thức $S = \frac{1}{2}n(a_1 + a_n)$.)

\begin{dinhnghia}[Dãy cộng]
	\emph{Dãy cộng} là dãy có dạng $\{a + n b\}_{n=0}^\infty = a, a + b, a + 2b, a + 3b,\ldots$, với $a,b\in\mathbb{N}$, $b\ne 0$.
\end{dinhnghia}
Trong các dãy số cộng, mỗi số hạng, kể từ số hạng thứ 2, đều lớn hơn số hạng đứng trước nó cùng 1 số đơn vị là $b$.

\begin{vidu}
	\begin{enumerate*}
		\item[(a)] $a = 0$, $b = 1$, dãy $\{a + n b\}_{n=0}^\infty = \{n\}_{n=0}^\infty = \mathbb{N} = 0,1,2,3,\ldots$ là dãy các số tự nhiên.
		\item[(b)] $a = 1$, $b = 2$, dãy $\{a + n b\}_{n=0}^\infty = \{1 + 2n\}_{n=0}^\infty = 1,3,5,7,\ldots$ là dãy các số tự nhiên lẻ.
		\item[(c)] $a = 0$, $b = 2$, $\{a + n b\}_{n=0}^\infty = \{2n\}_{n=0}^\infty = 0,2,4,6,\ldots$ là dãy các số tự nhiên chẵn.
		\item[(d)] Với $b\in\mathbb{N}^\star$, $b\ge 2$, $a\in\mathbb{N}$, $a < b$, dãy $\{a + n b\}_{n=0}^\infty$ là dãy các số tự nhiên chia cho $b$ dư $a$.
	\end{enumerate*}
\end{vidu}

\begin{baitoan}[\cite{Binh_Toan_6_tap_1}, Ví dụ 62, p. 52]
	Bạn Lâm đánh số trang 1 cuốn sách dày $284$ trang bằng dãy số chẵn $2,4,6,8,\ldots$
	\begin{enumerate*}
		\item[(a)] Biết mỗi chữ số viết mất $1$ giây. Hỏi bạn Lâm cần bao nhiêu phút để đánh số trang cuốn sách?
		\item[(b)] Chữ số thứ $300$ mà bạn Lâm viết là chữ số nào?
	\end{enumerate*}
\end{baitoan}

\begin{baitoan}[\cite{Binh_Toan_6_tap_1}, Ví dụ 63${}^\star$, p. 52]
	Tìm $n\in\mathbb{N}$ lớn nhất để tích các số tự nhiên từ $1$ đến $1000$ chia hết cho $5^n$.
\end{baitoan}
``\textit{Tổng quát}: Số thừa số $a$ khi phân tích $n! = \prod_{i=1}^n i = 1\cdot 2\cdot 3\cdots n$ ra thừa số nguyên tố là: $\sum_{i=1}^k \lfloor\frac{n}{a^i}\rfloor = \lfloor\frac{n}{a}\rfloor + \lfloor\frac{n}{a^2}\rfloor + \cdots + \lfloor\frac{n}{a^k}\rfloor$ với $k$ là số mũ lớn nhất sao cho $a^k\le n$. Ký hiệu $\lfloor\frac{n}{m}\rfloor$ là số tự nhiên lớn nhất không vượt quá $\frac{n}{m}$ (nếu $n\divby m$ thì $\lfloor\frac{n}{m}\rfloor$ là thương đúng, nếu $n\not\,\divby m$ thì $\lfloor\frac{n}{m}\rfloor$ là thương hụt, ta gọi $\lfloor\frac{n}{m}\rfloor$ là \emph{phần nguyên} của $\frac{n}{m}$).'' -- \cite[p. 53]{Binh_Toan_6_tap_1}

\begin{baitoan}[\cite{Binh_Toan_6_tap_1}, Ví dụ 64, p. 53]
	Có bao nhiêu số tự nhiên chia hết cho $13$ trong dãy $111,1111,\ldots,\underbrace{1\ldots 1}_{\scriptsize1993\mbox{ số}}$.
\end{baitoan}

\subsection{Các dãy khác}

\begin{baitoan}[\cite{Binh_Toan_6_tap_1}, Ví dụ 65, p. 53]
	Tìm số hạng thứ $100$ của các dãy được viết theo quy luật:
	\begin{enumerate*}
		\item[(a)] $3,8,15,24,35,\ldots$;
		\item[(b)] $3,24,63,120,195,\ldots$;
		\item[(c)] $1,3,6,10,15,\ldots$;
		\item[(d)] $1,2,4,7,11$;
		\item[(e)] $2,5,10,17,26,\ldots$
	\end{enumerate*}
\end{baitoan}
\noindent\textit{Hint.} 2 số hạng đầu của các dãy trên có thể viết dưới dạng: dãy (a): $1\cdot3,2\cdot4$; dãy (b): $1\cdot3,4\cdot6$; dãy (c): $\frac{1\cdot2}{2},\frac{2\cdot3}{2}$; (c) dãy (e): $1 + 1^2,1 + 2^2$.

\begin{baitoan}[\cite{Binh_Toan_6_tap_1}, Ví dụ 66, p. 54]
	\begin{enumerate*}
		\item[(a)] Tính tổng $A = \sum_{i=1}^{98} i(i + 1) = 1\cdot2 + 2\cdot3 + 3\cdot4 + \cdots + 98\cdot99$.
		\item[(b)] Sử dụng kết quả của (a), tính $B = \sum_{i=1}^{98} i^2 = 1^2 + 2^2 + 3^2 + \cdots + 97^2 + 98^2$.
	\end{enumerate*}
\end{baitoan}
Tổng quát:
\begin{align*}
	\sum_{i=1}^n i^2 = 1^2 + 2^2 + 3^2 + \cdots + n^2 = \frac{n(n + 1)(n + 2)}{3} - \frac{n(n + 1)}{2} = \frac{n(n + 1)(2n + 1)}{6},\ \forall n\in\mathbb{N}^\star.
\end{align*}

\begin{baitoan}[\cite{Binh_Toan_6_tap_1}, \textbf{309.}, p. 55]
	Tìm chữ số thứ $1000$ khi viết liên tiếp liền nhau các số hạng của dãy số lẻ $1,3,5,7,\ldots$
\end{baitoan}

\begin{baitoan}[\cite{Binh_Toan_6_tap_1}, \textbf{310.}, p. 55]
	\begin{enumerate*}
		\item[(a)] Tính tổng các số lẻ có 2 chữ số.
		\item[(b)] Tính tổng các số chẵn có 2 chữ số.
	\end{enumerate*}
\end{baitoan}

\begin{baitoan}[\cite{Binh_Toan_6_tap_1}, \textbf{311.}, p. 55]
	Có số hạng nào của dãy sau tận cùng bằng $2$ hay không? $1, 1 + 2, 1 + 2 + 3, 1 + 2 + 3 + 4,\ldots$
\end{baitoan}

\begin{baitoan}[\cite{Binh_Toan_6_tap_1}, \textbf{312.}, p. 55]
	\begin{enumerate*}
		\item[(a)] Viết liên tiếp các số hạng của dãy số tự nhiên từ $1$ đến $100$ tạo thành 1 số $A$. Tính tổng các chữ số của $A$.
		\item[(b)] Cũng hỏi như trên nếu viết từ $1$ đến $1000000$.
	\end{enumerate*}
\end{baitoan}

\begin{baitoan}[\cite{Binh_Toan_6_tap_1}, \textbf{313.}, p. 55]
	Có $n$ em bé được nhận quà. Cô giáo đã xếp cho các em đứng thành 1 hàng ngang, có số tuổi nhỏ dần kể từ trái sang phải. Lần lượt từ trái sang phải em thứ nhất được $1$ chiếc, em thứ 2 được $2$ chiếc, cứ như vậy em nhận sau được chia nhiều hơn em nhận trước $1$ chiếc kẹo. Đến lượt chia thứ 2, cô giao cũng chia kẹo từ trái sang phải sao cho em nhận sau được chia nhiều hơn em nhận trước $1$ chiếc kẹo (lưu ý: ở lượt thứ 2 thì em thứ nhất nhận không phải $1$ chiếc kẹo mà là $n + 1$ chiếc kẹo). Tính $n$ biết số kẹo chia ở lượt thứ 2 nhiều hơn số kẹo chia ở lượt thứ nhất là $36$ chiếc. 
\end{baitoan}

\begin{baitoan}[\cite{Binh_Toan_6_tap_1}, \textbf{314.}, p. 55]
	Bạn A viết dãy số tự nhiên như sau: $3,4,5,\ldots,345$ (1). Bạn B thay mỗi số của dãy (1) bởi tổng các chữ số của nó \& được dãy (2). Bạn C thay mỗi số của dãy (2) bởi tổng các chữ số của nó \& được dãy (3). Bạn D thay mỗi số của dãy (3) bởi tổng các chữ số của nó \& được dãy (4).
	\begin{enumerate*}
		\item[(a)] Chứng tỏ chỉ có dãy (4) mới có mọi số hạng đều là số có 1 chữ số.
		\item[(b)] Số nào xuất hiện nhiều nhất ở dãy (4)?
	\end{enumerate*}
\end{baitoan}

\begin{baitoan}[\cite{Binh_Toan_6_tap_1}, \textbf{315.}, p. 55]
	\begin{enumerate*}
		\item[(a)] Khi phân tích ra thừa số nguyên tố, số $1000!$ chứa thừa số nguyên tố $7$ với số mũ bằng bao nhiêu?
		\item[(b)] Tích $A = 500!$ tận cùng bằng bao nhiêu chữ số $0$?
	\end{enumerate*}	
\end{baitoan}

\begin{baitoan}[\cite{Binh_Toan_6_tap_1}, \textbf{316.}, p. 55]
	\begin{enumerate*}
		\item[(a)] Tích $B = 38\cdot 39\cdot 40\cdots 74$ có bao nhiêu thừa số $2$ khi phân tích ra thừa số nguyên tố?
		\item[(b)] Tích $C = 31\cdot 32\cdot 33\cdots 90$ có bao nhiêu thừa số $3$ khi phân tích ra thừa số nguyên tố?
	\end{enumerate*}
\end{baitoan}

\begin{baitoan}[\cite{Binh_Toan_6_tap_1}, \textbf{317.}, p. 55]
	2 con châu chấu cùng nhảy 1 lúc từ 1 chỗ \& về cùng 1 phía. Khi con I nhảy 1 bước thì con II cũng nhảy 1 bước. Con I nhảy mỗi bước dài $4$\emph{m}. Con II nhảy bước thứ nhất dài $1$\emph{m}, mỗi bước sau tăng hơn so với bước liền trước $1$\emph{m} cho đến khi đuổi kịp con I. Hỏi sau bao nhiêu bước nhảy thì con II đuổi kịp con I?
\end{baitoan}

\begin{baitoan}[\cite{Binh_Toan_6_tap_1}, \textbf{318.}, p. 56]
	Cho 3 dãy các số tự nhiên liên tiếp: $1,2,3,\ldots,95,96,97$; $1,2,3,\dots,95,96$; $1,2,3,\ldots,95$. Trong dãy nào có thể chia các số của dãy thành 2 nhóm để tổng các số trong mỗi nhóm bằng nhau?
\end{baitoan}

\begin{baitoan}[\cite{Binh_Toan_6_tap_1}, \textbf{319.}, p. 56]
	\begin{enumerate*}
		\item[(a)] Viết số hạng thứ $n$ của dãy $1,4,7,10,13,\ldots$;
		\item[(b)] Viết 2 số hạng tiếp theo của dãy $1,3,2,6,3,9,4,12,5,\ldots$;
		\item[(c)] Viết số hạng thứ $n$ của dãy $1,2,4,7,11,\ldots$
	\end{enumerate*}
\end{baitoan}

\begin{baitoan}[\cite{Binh_Toan_6_tap_1}, \textbf{320.}, p. 56]
	Có bao nhiêu số tự nhiên đồng thời là các số hạng của cả 2 dãy sau: $3,7,11,15,\ldots,407$ \& $2,9,16,23,\ldots,709$.
\end{baitoan}

\begin{baitoan}[\cite{Binh_Toan_6_tap_1}, \textbf{321.}, p. 56]
	Cho 1 dãy gồm $30$ số chẵn liên tiếp tăng dần có tổng bằng $1470$. Tìm số hạng đầu \& số hạng cuối của dãy.
\end{baitoan}

\begin{baitoan}[\cite{Binh_Toan_6_tap_1}, \textbf{322.}, p. 56]
	\begin{enumerate*}
		\item[(a)] Tính tổng của $n$ số lẻ liên tiếp bắt đầu từ $1$.s
		\item[(b)] Xếp các hộp thàng hàng, hàng thứ nhất có $1$ hộp, hàng thứ 2 có $3$ hộp, hàng thứ 3 có $5$ hộp, etc., sao cho các hộp ở giữa mỗi hàng tạo thành 1 cột. Có tất cả bao nhiêu hộp từ hàng thứ nhất tới hàng thứ $10$? Có tất cả bao nhiêu  hộp từ hàng thứ nhất tới hàng thứ $n$, $n\in\mathbb{N}^\star$?
	\end{enumerate*}
\end{baitoan}

\begin{baitoan}[\cite{Binh_Toan_6_tap_1}, \textbf{323.}, p. 56]
	Trong dãy số $1,2,3,\ldots,1990$, có thể chọn được nhiều nhất bao nhiêu số để tổng 2 số bất kỳ được chọn chia hết cho $38$?
\end{baitoan}

\begin{baitoan}[\cite{Binh_Toan_6_tap_1}, \textbf{324.}, p. 56]
	1 đồng hồ reo chuông vào các thời điểm sau: 4:10, 5:20, 6:40, 8:10, $\ldots$ Theo quy luật trên, đồng hồ reo chuông lần tiếp theo vào lúc nào?
\end{baitoan}

\begin{baitoan}[\cite{Binh_Toan_6_tap_1}, \textbf{325.}${}^\star$, p. 56, Theo nội dung bài toán bò ăn cở của Newton]
	Có 3 cánh đồng cỏ như nhau \& cỏ luôn mọc đều như nhau trên toàn bộ cánh đồng. $9$ con bò ăn hết số cỏ có sẵn \& số cỏ mọc thêm của cánh đồng I trong $2$ tuần, $6$ con bò ăn hết số cỏ có sẵn \& số cỏ mọc thêm của cánh đồng II trong $4$ tuần. Hỏi bao nhiêu con bò ăn hết cỏ có sẵn \& số cỏ mọc thêm của cánh đồng II trong $6$ tuần? (mỗi con bò đều ăn số cỏ như nhau).
\end{baitoan}

\begin{baitoan}[\cite{Binh_Toan_6_tap_1}, \textbf{326.}${}^\star$, p. 56]
	Chia dãy số tự nhiên kể từ $1$ thành từng nhóm (các số cùng nhóm được đặt trong dấu ngoặc) $(1),(2,3),(4,5,6),(7,8,9,10),(11,12,13,14,15),\ldots$
	\begin{enumerate*}
		\item[(a)] Tìm số hạng đầu tiên của nhóm thứ $100$.
		\item[(b)] Tính tổng các số thuộc nhóm thứ $100$.
	\end{enumerate*}
\end{baitoan}

\begin{baitoan}[\cite{Binh_Toan_6_tap_1}, \textbf{327.}, p. 56]
	Cho $S_1 = 1 + 2$, $S_2 = 3 + 4 + 5$, $S_3 = 6 + 7 + 8 + 9$, $S_4 = 10 + 11 + 12 + 13 + 14,\ldots$ Tính $S_{100}$.
\end{baitoan}
Bài tập phụ thuộc vào hình vẽ: \cite[\textbf{328.}--\textbf{331.}, p. 57]{Binh_Toan_6_tap_1}.

\texttt{pause here ...}

\begin{baitoan}[\cite{Binh_Toan_6_tap_1}, \textbf{230.}, p. 49]
	Tính số hạng thứ $50$ của các dãy sau:
	\begin{enumerate*}
		\item[(a)] $1\cdot 6,2\cdot 7,3\cdot 8,\ldots$;
		\item[(b)] $1\cdot 4,4\cdot 7,7\cdot 10,\ldots$
	\end{enumerate*}
\end{baitoan}

\begin{baitoan}[\cite{Binh_Toan_6_tap_1}, \textbf{231.}, p. 49]
	Cho $A = 1 + 3 + 3^2 + 3^3 + \cdots + 3^{20} = \sum_{i=0}^{20} 3^i$, $B = 3^{21}:2$. Tính $B - A$.
\end{baitoan}

\begin{baitoan}[\cite{Binh_Toan_6_tap_1}, \textbf{232.}, p. 49]
	Cho $A = 1 + 4 + 4^2 + 4^3 + \cdots + 4^{99}$, $B = 4^{100}$. Chứng minh rằng $A < \frac{B}{3}$.
\end{baitoan}

\begin{baitoan}[\cite{Binh_Toan_6_tap_1}, \textbf{233.}, p. 49]
	Tính giá trị của biểu thức:
	
	\begin{enumerate*}
		\item[(a)] $A = 9 + 99 + 999 + \cdots + \underbrace{9\ldots 9}_{50's}$;
		\item[(b)] $B = 9 + 99 + 999 + \cdots + \underbrace{9\ldots 9}_{200's}$.
	\end{enumerate*}
\end{baitoan}

\section{Đếm số}

\begin{baitoan}[\cite{Binh_Toan_6_tap_1}, Ví dụ 43, p. 49]
	Có bao nhiêu số $\overline{abcd}$ mà $\overline{ab} < \overline{cd}$?
\end{baitoan}

\begin{baitoan}[\cite{Binh_Toan_6_tap_1}, Ví dụ 44, p. 49]
	Có bao nhiêu số tự nhiên chia hết cho $4$ gồm 4 chữ số, chữ số tận cùng bằng $2$?
\end{baitoan}

\begin{luuy}
	``Nếu việc chọn đối tượng $A$ có thể thực hiện bởi $m$ cách \& với mỗi cách chọn của $A$ có thể chọn đối tượng $B$ bởi $n$ cách thì việc chọn $A$ \& $B$ theo thứ tự đó có thể thực hiện bởi $mn$ cách chọn.'' -- \cite[p. 50]{Binh_Toan_6_tap_1} \emph{Quy tắc nhân trong phép đếm} \& khái niệm \emph{tổ hợp, chỉnh hợp} sẽ được học ở môn Tổ hợp, trong chương trình Toán 10.
\end{luuy}

\begin{baitoan}[\cite{Binh_Toan_6_tap_1}, Ví dụ 45, p. 50]
	Có bao nhiêu số tự nhiên có $4$ chữ số $\overline{abcd}$, trong đó $b - a = 1$, $d - c = 1$?
\end{baitoan}

\begin{baitoan}[\cite{Binh_Toan_6_tap_1}, Ví dụ 46, p. 50]
	Có bao nhiêu số tự nhiên có 3 chữ số trong đó có đúng 1 chữ số $5$?
\end{baitoan}
``Trong nhiều trường hợp, để đếm các số có tính chất nào đó, ta lại đếm trước hết các số không có tính chất ấy.'' -- \cite[p. 51]{Binh_Toan_6_tap_1}

\begin{baitoan}[\cite{Binh_Toan_6_tap_1}, Ví dụ 47, p. 50]
	Có bao nhiêu số chứa ít nhất 1 chữ số $1$ trong các số tự nhiên:
	\begin{enumerate*}
		\item[(a)] có 3 chữ số;
		\item[(b)] từ $1$ đến $999$.
	\end{enumerate*}
\end{baitoan}

\begin{baitoan}[\cite{Binh_Toan_6_tap_1}, Ví dụ 48, p. 51]
	Viết $999$ số tự nhiên liên tiếp kể từ $1$. Hỏi:
	\begin{enumerate*}
		\item[(a)] Chữ số $2$ có mặt bao nhiêu lần?
		\item[(b)] Chữ số $0$ có mặt bao nhiêu lần?
	\end{enumerate*}
\end{baitoan}

\begin{baitoan}[\cite{Binh_Toan_6_tap_1}, \textbf{234.}, p. 52]
	Bạn Tâm đánh số trang của 1 cuốn vở có $110$ trang bằng cách viết dãy số tự nhiên $1,2,\ldots,110$. Bạn Tâm phải viết tất cả bao nhiêu chữ số?
\end{baitoan}

\begin{baitoan}[\cite{Binh_Toan_6_tap_1}, \textbf{235.}, p. 52]
	1 cô nhân viên đánh máy liên tục dãy số chẵn bắt đầu từ $2$: $2,4,6,8,10,12,\ldots$ Cô phải đánh tất cả $2000$ chữ số. Tìm chữ số cuối cùng mà cô đã đánh.
\end{baitoan}

\begin{baitoan}[\cite{Binh_Toan_6_tap_1}, \textbf{236.}, p. 52]
	Bạn Mai viết dãy số lẻ $1,3,5,\ldots,245$.
	\begin{enumerate*}
		\item[(a)] Bạn Mai phải viết tất cả bao nhiêu chữ số?
		\item[(b)] Nếu mỗi chữ số viết mất 1 giây thì viết đến số $245$ mất bao nhiêu giây? Sau $5$ phút, bạn Mai viết đến chữ số nào?
	\end{enumerate*}
\end{baitoan}

\begin{baitoan}[\cite{Binh_Toan_6_tap_1}, \textbf{237.}, p. 52]
	Bạn Hùng viết dãy số lẻ $1,3,5,7,\ldots$ để đánh số trang 1 cuốn sách. Tính xem chữ số $200$ mà bạn Hùng viết là chữ số nào?
\end{baitoan}

\begin{baitoan}[\cite{Binh_Toan_6_tap_1}, \textbf{238.}, p. 52]
	Để đánh số trang của 1 cuốn sách, người ta viết dãy số tự nhiên bắt đầu từ $1$ \& phải dùng tất cả $1998$ chữ số.
	\begin{enumerate*}
		\item[(a)] Hỏi cuốn sách có bao nhiêu trang?
		\item[(b)] Chữ số thứ $1010$ là chữ số nào?
	\end{enumerate*}
\end{baitoan}

\begin{baitoan}[\cite{Binh_Toan_6_tap_1}, \textbf{239.}, p. 52]
	Có bao nhiêu số tự nhiên chia hết cho $3$, có 4 chữ số \& tận cùng bằng $5$?
\end{baitoan}

\begin{baitoan}[\cite{Binh_Toan_6_tap_1}, \textbf{240.}, pp. 52--53]
	Tuấn muốn đến nhà bạn, nhưng không nhớ số nhà, chỉ biết rằng số nhà của bạn là số chia hết cho $3$ \& có 2 chữ số. Biết số nhà cuối của dãy phố đó là $135$. Hỏi Tuấn phải gõ cửa nhiều nhất bao nhiêu số nhà? (các số nhà không đánh số $a,b,\ldots$).
\end{baitoan}

\begin{baitoan}[\cite{Binh_Toan_6_tap_1}, \textbf{241.}, p. 53]
	Tìm số lượng các số tự nhiên có 4 chữ số mà:
	\begin{enumerate*}
		\item[(a)] Số tạo bởi 2 chữ số đầu (theo thứ tự ấy) cộng với số tạo bởi 2 chữ số cuối (theo thứ tự ấy) nhỏ hơn $100$.
		\item[(b)] Số tạo bởi 2 chữ số đầu (theo thứ tự ấy) lớn hơn số tạo bởi 2 chữ số cuối (theo thứ tự ấy)?
	\end{enumerate*}
\end{baitoan}

\begin{baitoan}[\cite{Binh_Toan_6_tap_1}, \textbf{242.}, p. 53]
	Trong các số tự nhiên từ $1$ đến $252$, xóa các số chia hết cho $2$ nhưng không chia hết cho $5$, rồi xóa các số chia hết cho $5$ nhưng không chia hết cho $2$. Còn lại bao nhiêu số?
\end{baitoan}

\begin{baitoan}[\cite{Binh_Toan_6_tap_1}, \textbf{243.}, p. 53]
	Có bao nhiêu số tự nhiên có 3 chữ số mà:
	\begin{enumerate*}
		\item[(a)] Các chữ số đều chẵn?
		\item[(b)] Chữ số hàng chục là chữ số lẻ?
	\end{enumerate*}
\end{baitoan}

\begin{baitoan}[\cite{Binh_Toan_6_tap_1}, \textbf{244.}, p. 53]
	Có bao nhiêu số tự nhiên có 4 chữ số mà:
	\begin{enumerate*}
		\item[(a)] Mỗi chữ số đều chẵn?
		\item[(b)] Tổng các chữ số là số chẵn?
	\end{enumerate*}
\end{baitoan}

\begin{baitoan}[\cite{Binh_Toan_6_tap_1}, \textbf{245.}, p. 53]
	Có bao nhiêu biển số xe máy khác nhau, mỗi số xe lập bởi 2 chữ cái đứng đầu \& 3 chữ số đứng sau? (bảng chữ cái có $25$ chữ, không có biển số $000$).
\end{baitoan}

\begin{baitoan}[\cite{Binh_Toan_6_tap_1}, \textbf{246.}, p. 53]
	Trong các số tự nhiên có 3 chữ số, có bao nhiêu số:
	\begin{enumerate*}
		\item[(a)] Chứa đúng 1 chữ số $4$?
		\item[(b)] Chứa đúng 2 chữ số $4$?
	\end{enumerate*}
\end{baitoan}

\begin{baitoan}[\cite{Binh_Toan_6_tap_1}, \textbf{247.}, p. 53]
	Có bao nhiêu số tự nhiên chia hết cho $5$, có 4 chữ số, có đúng 1 chữ số $5$?
\end{baitoan}

\begin{baitoan}[\cite{Binh_Toan_6_tap_1}, \textbf{248.}, p. 53]
	Có bao nhiêu số tự nhiên có 3 chữ số, biết rằng cộng nó với số gồm 3 chữ số ấy viết theo thứ tự ngược lại thì được 1 số chia hết cho $5$?
\end{baitoan}

\begin{baitoan}[\cite{Binh_Toan_6_tap_1}, \textbf{249.}, p. 53]
	Có bao nhiêu số chẵn có 3 chữ số, các chữ số khác nhau?
\end{baitoan}

\begin{baitoan}[\cite{Binh_Toan_6_tap_1}, \textbf{250.}, p. 53]
	Có bao nhiêu số tự nhiên có 3 chữ số trong đó có ít nhất 2 chữ số như nhau?
\end{baitoan}

\begin{baitoan}[\cite{Binh_Toan_6_tap_1}, \textbf{251.}, p. 53]
	Trong các số tự nhiên có 4 chữ số, có bao nhiêu số trong đó có đúng 3 chữ số như nhau?
\end{baitoan}

\begin{baitoan}[\cite{Binh_Toan_6_tap_1}, \textbf{252.}, p. 53]
	Trong các số tự nhiên có 3 chữ số, có bao nhiêu số:
	\begin{enumerate*}
		\item[(a)] Chia hết cho $5$, có chứa chữ số $5$?
		\item[(b)] Chia hết cho $4$, có chứa chữ số $4$?
		\item[(c)] Chia hết cho $3$, không chứa chữ số $3$?
	\end{enumerate*}
\end{baitoan}

\begin{baitoan}[\cite{Binh_Toan_6_tap_1}, \textbf{253.}, p. 54]
	Viết liên tiếp các số tự nhiên từ $1$ đến $999$ ta được 1 số tự nhiên $A$.
	\begin{enumerate*}
		\item[(a)] Số $A$ có bao nhiêu chữ số?
		\item[(b)] Tính tổng các chữ số của số $A$.
	\end{enumerate*}
\end{baitoan}

\begin{baitoan}[\cite{Binh_Toan_6_tap_1}, $\bf 254^\star.$, p. 54]
	Viết dãy số tự nhiên từ $1$ đến $999$.
	\begin{enumerate*}
		\item[(a)] Chữ số $1$ được viết bao nhiêu lần?
		\item[(b)] Chữ số $0$ được viết bao nhiêu lần?
	\end{enumerate*}
\end{baitoan}

\begin{baitoan}[\cite{Binh_Toan_6_tap_1}, \textbf{255.}, p. 54]
	Trong các số tự nhiên có 3 chữ số, có bao nhiêu số chứa ít nhất 1 chữ số $4$?
\end{baitoan}

\begin{baitoan}[\cite{Binh_Toan_6_tap_1}, $\bf 256^\star.$, p. 54]
	Trong các số tự nhiên từ $1$ đến $10000$:
	\begin{enumerate*}
		\item[(a)] Có bao nhiêu số chứa chữ số $0$?
		\item[(b)] Số chứa chữ số $1$ hay số không chứa chữ số $1$ có nhiều hơn?
	\end{enumerate*}
\end{baitoan}

\begin{baitoan}[\cite{Binh_Toan_6_tap_1}, \textbf{257.}, p. 54]
	Viết dãy số chẵn $100,102,\ldots,390$. Hỏi chữ số $2$ được viết bao nhiêu lần?
\end{baitoan}

\begin{baitoan}[\cite{Binh_Toan_6_tap_1}, \textbf{258.}, p. 54]
	Từ các chữ số $1,2,3,4,5,6,7$, lập tất cả các số tự nhiên có 7 chữ số trong đó mỗi chữ số trên đều có mặt. Chứng minh rằng tổng tất cả các số đó chia hết cho $9$.
\end{baitoan}

\begin{baitoan}[\cite{Binh_Toan_6_tap_1}, \textbf{259.}, p. 54]
	Cho 3 chữ số $a,b,c$ khác nhau \& khác $0$. Gọi $A$ là tập hợp các số tự nhiên có 3 chữ số lập bởi cả 3 chữ số trên.
	\begin{enumerate*}
		\item[(a)] Tập hợp $A$ có bao nhiêu phần tử?
		\item[(b)] Tính tổng các phần tử của tập hợp $A$, biết rằng $a + b + c = 17$.
	\end{enumerate*}
\end{baitoan}

\begin{baitoan}[\cite{Binh_Toan_6_tap_1}, \textbf{260.}, p. 54]
	Từ các chữ số $1,2,3,4$, lập tất cả các số tự nhiên mà mỗi chữ số trên đều có mặt đúng 1 lần. Tìm tổng các số ấy.
\end{baitoan}

\begin{baitoan}[\cite{Binh_Toan_6_tap_1}, \textbf{261.}, p. 54]
	Tìm tổng các số tự nhiên có 3 chữ số lập bởi các chữ số $2,3,0,7$ trong đó:
	\begin{enumerate*}
		\item[(a)] Các chữ số có thể giống nhau;
		\item[(b)] Các chữ số đều khác nhau.
	\end{enumerate*}
\end{baitoan}

%------------------------------------------------------------------------------%

\section{Miscellaneous}
\textsf{\textbf{Nội dung.} Số âm, số dương; tập hợp $\mathbb{Z}$ các số nguyên; thứ tự trong $\mathbb{Z}$; các phép tính về số nguyên; quy tắc dấu ngoặc; bội, ước của 1 số nguyên.}

\begin{baitoan}[\cite{Tuyen_Toan_6}, Ví dụ 47, p. 43]
	Tìm $x,y,z\in\mathbb{Z}$ thỏa $x - y = -9$, $y - z = -10$, $z + x = 11$.
\end{baitoan}

\begin{proof}[Giải]
	Có $(x - y) + (y - z) + (z + x) = (-9) + (-10) + 11\Leftrightarrow(x + x) + (-y + y) + (-z + z) = -9\Leftrightarrow2x = -8\Leftrightarrow x = \frac{-8}{2} = -4$. Vì $x - y = -9$ nên $y = x + 9 = -4 + 9 = 5$. Vì $x + z = 11$ nên $z = 11 - x = 11 -(-4) = 15$. Vậy $(x,y,z) = (-4,5,15)$.
\end{proof}

\begin{baitoan}[\cite{Tuyen_Toan_6}, Ví dụ 48, p. 44]
	Cho $x\in\mathbb{Z}$. So sánh $x^2$ với $x^3$.
\end{baitoan}

\begin{proof}[Giải]
	Nếu $x < 0$ thì $x^2 > 0$, $x^3 < 0$ nên $x^2 > 0 > x^3$. Nếu $x = 0$ hoặc $x = 1$ thì $x^2 = x^3$ (tương ứng $=0,1$). Nếu $x > 1$ thì $x^2 - x^3 = x^2(1 - x) < 0$ (vì $x^2 > 0$ \& $1 - x < 0$), nên $x^2 < x^3$.
\end{proof}

\begin{nhanxet}
	``Trong cách giải trên ta đã phân chia tập hợp $\mathbb{Z}$ các số nguyên thành 3 tập hợp là tập hợp các số nguyên âm, tập hợp các số $0$ \& $1$, tập hợp các số nguyên lớn hơn $1$. Cách phân chia như vậy đảm bảo được yêu cầu không bỏ sót số nguyên nào cũng như không có số nguyên nào thuộc 2 tập hợp.'' -- \cite[p. 44]{Tuyen_Toan_6}
\end{nhanxet}

\begin{baitoan}[\cite{Tuyen_Toan_6}, \textbf{226.}, p. 44]
	Tính giá trị của biểu thức $A = -125\cdot(\underbrace{x + x + \cdots + x}_8 - \underbrace{y - y - \cdots - y}_8)$ với $x = -43$, $y = 17$.
\end{baitoan}

\begin{proof}[Giải]
	$A = -125(8x - 8y) = -125\cdot8(x - y) = -1000(-43 - 17) = -1000\cdot-60 = 60000$.
\end{proof}

\begin{baitoan}[\cite{Tuyen_Toan_6}, \textbf{227.}, p. 44]
	Tìm $x\in\mathbb{Z}$ thỏa: (a) $-4x + 5 = 41$; (b) $7x + 1 = \pm20$.
\end{baitoan}

\begin{proof}[Giải]
	(a) $-4x + 5 = 41\Leftrightarrow -4x = 41 - 5 = -36\Leftrightarrow x = \frac{-36}{-4} = 9$. (b)
	\begin{equation*}
		7x + 1 = \pm20\Leftrightarrow\left[\begin{split}
			7x + 1 &= 20,\\
			7x + 1 &= -20,
		\end{split}\right.\Leftrightarrow\left[\begin{split}
		7x &= 20 - 1 = 19,\\
		7x &= -20 - 1 = -21,
	\end{split}\right.\Leftrightarrow\left[\begin{split}
		x &= \frac{19}{7}\notin\mathbb{Z}\mbox{ (loại)},\\
		x &= \frac{-21}{7} = -3,
	\end{split}\right.
	\end{equation*}
	Vậy phương trình chỉ có nghiệm nguyên duy nhất $x = -3$.
\end{proof}

\begin{baitoan}[\cite{Tuyen_Toan_6}, \textbf{228.}, p. 44]
	Cho $A = \{6,7,8,9\}$, $B = \{-1,-2,-3,4,8\}$. (a) Có bao nhiêu hiệu dạng $a - b$ với $a\in A,b\in B$? (b) Có bao nhiêu hiệu chia hết cho $5$? (c) Có bao nhiệu hiệu là số nguyên âm?
\end{baitoan}

\begin{baitoan}[\cite{Tuyen_Toan_6}, \textbf{229.}, p. 44]
	Tìm $x\in\mathbb{Z}$ thỏa $(x + 5)(3x - 12) > 0$.
\end{baitoan}

\begin{baitoan}[\cite{Tuyen_Toan_6}, \textbf{230.}, p. 44]
	Tìm $x\in\mathbb{Z}$ thỏa $(x^3 + 5)(x^3 + 10)(x^3 + 15)(x^3 + 30) < 0$.
\end{baitoan}

\begin{baitoan}[\cite{Tuyen_Toan_6}, \textbf{231.}, p. 44]
	Tìm $x,y\in\mathbb{Z}$ thỏa $(x - 7)(xy + 1) = 5$.
\end{baitoan}

\begin{baitoan}[\cite{Tuyen_Toan_6}, \textbf{232.}, p. 44]
	Cho $a,b,c,d\in\mathbb{Z}$. Biết tích $ab$ là liền sau của tích $cd$ \& $a + b = c + d$. Chứng minh $a = b$.
\end{baitoan}

\begin{baitoan}[\cite{Tuyen_Toan_6}, \textbf{233.}, p. 44]
	Tìm 2 số nguyên mà tích của chúng bằng hiệu của chúng.
\end{baitoan}

\begin{baitoan}[\cite{Tuyen_Toan_6}, \textbf{234.}, p. 44]
	Máy cấp đông (làm lạnh nhanh) làm thay đổi nhiệt độ được $-2^\circ$C sau 1 phút. Khi bắt đầu chạy, nhiệt độ trong tủ đông là $+10^\circ$C. Hỏi sau mấy phút thì nhiệt độ trong tủ đông là $-4^\circ$C.
\end{baitoan}

\begin{baitoan}[\cite{Tuyen_Toan_6}, \textbf{235.}, p. 44]
	1 công ty nhỏ trong quý I mỗi tháng thu nhập $-100$ triệu đồng. Trong quý II mỗi tháng được $-20$ triệu đồng. Sang quý III mỗi tháng thu nhập $+60$ triệu đồng. Cuối năm tổng kết lại, kết quả kinh doanh của công ty thu nhập được $+420$ triệu đồng. Hỏi mỗi tháng trong quý IV công ty thu nhập được bao nhiêu, biết thu nhập trong các tháng như nhau?
\end{baitoan}

%------------------------------------------------------------------------------%

\printbibliography[heading=bibintoc]
	
\end{document}