\documentclass{article}
\usepackage[backend=biber,natbib=true,style=authoryear]{biblatex}
\addbibresource{/home/nqbh/reference/bib.bib}
\usepackage[utf8]{vietnam}
\usepackage{tocloft}
\renewcommand{\cftsecleader}{\cftdotfill{\cftdotsep}}
\usepackage[colorlinks=true,linkcolor=blue,urlcolor=red,citecolor=magenta]{hyperref}
\usepackage{amsmath,amssymb,amsthm,mathtools,float,graphicx,algpseudocode,algorithm,tcolorbox,tikz,tkz-tab,subcaption}
\DeclareMathOperator{\arccot}{arccot}
\usepackage[inline]{enumitem}
\allowdisplaybreaks
\numberwithin{equation}{section}
\newtheorem{assumption}{Assumption}[section]
\newtheorem{baitoan}{Bài toán}
\newtheorem{cauhoi}{Câu hỏi}[section]
\newtheorem{conjecture}{Conjecture}[section]
\newtheorem{corollary}{Corollary}[section]
\newtheorem{definition}{Definition}[section]
\newtheorem{dinhly}{Định lý}[section]
\newtheorem{dinhnghia}{Định nghĩa}[section]
\newtheorem{example}{Example}[section]
\newtheorem{hequa}{Hệ quả}[section]
\newtheorem{lemma}{Lemma}[section]
\newtheorem{luuy}{Lưu ý}[section]
\newtheorem{notation}{Notation}[section]
\newtheorem{principle}{Principle}[section]
\newtheorem{problem}{Problem}[section]
\newtheorem{proposition}{Proposition}[section]
\newtheorem{question}{Question}[section]
\newtheorem{remark}{Remark}[section]
\newtheorem{theorem}{Theorem}[section]
\newtheorem{vidu}{Ví dụ}[section]
\usepackage[left=0.5in,right=0.5in,top=1.5cm,bottom=1.5cm]{geometry}
\usepackage{fancyhdr}
\pagestyle{fancy}
\fancyhf{}
\lhead{\small Sect.~\thesection}
\rhead{\small\nouppercase{\leftmark}}
\renewcommand{\subsectionmark}[1]{\markboth{#1}{}}
\cfoot{\thepage}
\def\labelitemii{$\circ$}

\title{Natural -- Số Tự Nhiên $\mathbb{N}$}
\author{Nguyễn Quản Bá Hồng\footnote{Independent Researcher, Ben Tre City, Vietnam\\e-mail: \texttt{nguyenquanbahong@gmail.com}; website: \url{https://nqbh.github.io}.}}
\date{\today}

\begin{document}
\maketitle
\begin{abstract}
	\textsc{[en]} This text is a collection of problems, from easy to advanced, about natural. This text is also a supplementary material for my lecture note on Elementary Mathematics grade 6, which is stored \& downloadable at the following link: \href{https://github.com/NQBH/hobby/blob/master/elementary_mathematics/grade_6/NQBH_elementary_mathematics_grade_6.pdf}{GitHub\texttt{/}NQBH\texttt{/}hobby\texttt{/}elementary mathematics\texttt{/}grade 6\texttt{/}lecture}\footnote{\textsc{url}: \url{https://github.com/NQBH/hobby/blob/master/elementary_mathematics/grade_6/NQBH_elementary_mathematics_grade_6.pdf}.}. The latest version of this text has been stored \& downloadable at the following link: \href{https://github.com/NQBH/hobby/blob/master/elementary_mathematics/grade_6/natural/NQBH_natural.pdf}{GitHub\texttt{/}NQBH\texttt{/}hobby\texttt{/}elementary mathematics\texttt{/}grade 6\texttt{/}natural $\mathbb{N}$}\footnote{\textsc{url}: \url{https://github.com/NQBH/hobby/blob/master/elementary_mathematics/grade_6/natural/NQBH_natural.pdf}.}.
	\vspace{2mm}
	
	\textsc{[vi]} Tài liệu này là 1 bộ sưu tập các bài tập chọn lọc từ cơ bản đến nâng cao về số tự nhiên. Tài liệu này là phần bài tập bổ sung cho tài liệu chính -- bài giảng \href{https://github.com/NQBH/hobby/blob/master/elementary_mathematics/grade_6/NQBH_elementary_mathematics_grade_6.pdf}{GitHub\texttt{/}NQBH\texttt{/}hobby\texttt{/}elementary mathematics\texttt{/}grade 6\texttt{/}lecture} của tác giả viết cho Toán Sơ Cấp lớp 6. Phiên bản mới nhất của tài liệu này được lưu trữ \& có thể tải xuống ở link sau: \href{https://github.com/NQBH/hobby/blob/master/elementary_mathematics/grade_6/natural/NQBH_natural.pdf}{GitHub\texttt{/}NQBH\texttt{/}hobby\texttt{/}elementary mathematics\texttt{/}grade 6\texttt{/}natural $\mathbb{N}$}.
\end{abstract}
\setcounter{secnumdepth}{4}
\setcounter{tocdepth}{3}
\tableofcontents

%------------------------------------------------------------------------------%

\section{Tập Hợp $\mathbb{N}$ Các Số Tự Nhiên}
``Tập hợp các số $0,1,2,3,\ldots$ gọi là tập hợp $\mathbb{N}$ các số tự nhiên. Ta xác định trên $\mathbb{N}$ 1 thứ tự như sau:
\begin{enumerate*}
	\item[(a)] $0$ là số tự nhiên nhỏ nhất;
	\item[(b)] $a < b$ khi \& chỉ khi $a$ ở bên trái điểm $b$ trên tia số nằm ngang.
\end{enumerate*}
Để dễ dàng ghi \& đọc các số tự nhiên, người ta dùng hệ ghi số: Khi được 1 số đơn vị nhất định ở 1 hàng, ta thay nó bằng 1 đơn vị ở hàng liền trước nó. Hệ ghi số thường dùng nhất là \textit{hệ thập phân}. Trong hệ thập phân, người ta dùng 10 ký hiệu để ghi nó, đó là các chữ số $0,1,2,\ldots,9$ \& cứ 10 đơn vị ở 1 hàng thì làm thành 1 đơn vị ở hàng liền trước nó. Trong hệ thập phân, có: $\overline{ab} = 10a + b$, $\overline{abc} = 100a + 10b + c = 10^2a + 10b + c$, $\overline{a_na_{n-1}\ldots a_2a_1a_0} = \sum_{i=0}^n 10^ia_i = 10^na_n + 10^{n-1}a_{n-1} + \cdots + 10^2a_2 + 10a_1 + a_0$.'' -- \cite[\S1, p. 4]{Binh_Toan_6_tap_1}

\begin{baitoan}[\cite{Binh_Toan_6_tap_1}, Ví dụ 1, p. 4]
	Viết các tập hợp sau rồi tìm số phần tử của mỗi tập hợp đó:
	\begin{enumerate*}
		\item[(a)] Tập hợp $A$ các số tự nhiên $x$ mà $8:x = 2$.
		\item[(b)] Tập hợp $B$ các số tự nhiên $x$ mà $x + 3 < 5$.
		\item[(c)] Tập hợp $C$ các số tự nhiên $x$ mà $x - 2 = x + 2$.
		\item[(d)] Tập hợp $D$ các số tự nhiên $x$ mà $x:2 = x:4$.
		\item[(e)] Tập hợp $E$ các số tự nhiên $x$ mà $x + 0 = x$.
	\end{enumerate*}\\\mbox{}\hfill\textsf{Ans:} (a) $\{4\}$. (b) $\{0,1\}$. (c) $\emptyset$. (d) $\{0\}$. (e) $\mathbb{N}$.
\end{baitoan}

\begin{baitoan}[\cite{Binh_Toan_6_tap_1}, Ví dụ 2, p. 5]
	Viết các tập hợp sau bằng cách liệt kê các phần tử của nó:
	\begin{enumerate*}
		\item[(a)] Tập hợp $A$ các số tự nhiên có 2 chữ số, trong đó chữ số hàng chục lớn hơn chữ số hàng đơn vị là $2$.
		\item[(b)] Tập hợp $B$ các số tự nhiên có 3 chữ số mà tổng các chữ số bằng $3$.
	\end{enumerate*}\hfill\textsf{Ans:} (a) $\{20,31,42,53,64,75,86,97\}$. (b) $\{300,201,210,102,111,120\}$.
\end{baitoan}

\begin{baitoan}[\cite{Binh_Toan_6_tap_1}, Ví dụ 3, p. 5]
	Tìm số tự nhiên có 5 chữ số, biết nếu viết thêm chữ số $2$ vào đằng sau số đó thì được số lớn gấp 3 lần số có được bằng cách viết thêm chữ số $2$ vào đằng trước số đó.\hfill\textsf{Ans:} $85714$.
\end{baitoan}

\begin{baitoan}[\cite{Binh_Toan_6_tap_1}, Mở rộng Ví dụ 3, p. 5]
	Tìm số tự nhiên nhỏ nhất có chữ số đầu tiên ở bên trái là $2$, khi chuyển chữ số $2$ này xuống cuối cùng thì số đó tăng gấp 3 lần.\hfill\textsf{Ans:} $285714$.
\end{baitoan}

\begin{baitoan}[\cite{Binh_Toan_6_tap_1}, Mở rộng Ví dụ 3, p. 6]
	Tìm số tự nhiên có 5 chữ số, biết nếu viết thêm 1 chữ số vào đằng sau số đó thì được số lớn gấp 3 lần số có được nếu viết thêm chính chữ số ấy vào đằng trước số đó.\hfill\textsf{Ans:} $85714$.
\end{baitoan}

\begin{baitoan}[\cite{Binh_Toan_6_tap_1}, \textbf{1.}, p. 6]
	Các tập hợp $A,B,C,D$ được cho bởi sơ đồ sau:
	\begin{figure}[H]
		\centering
		\includegraphics[scale=0.12]{Binh_1_p_6}
	\end{figure}
	Viết các tập hợp $A,B,C,D$ bằng cách liệt kê các phần tử của tập hợp.
\end{baitoan}

\begin{baitoan}[\cite{Binh_Toan_6_tap_1}, \textbf{2.}, p. 6]
	Xác định các tập hợp sau bằng cách chỉ ra tính chất đặc trưng của các phần tử thuộc tập hợp đó:
	\begin{enumerate*}
		\item[(a)] $A = \{1,3,5,7,\ldots,49\}$;
		\item[(b)] $B = \{11,22,33,44,\ldots,99\}$;\\
		\item[(c)] $C = \{\mbox{tháng } 1,\mbox{tháng } 3,\mbox{tháng } 5,\mbox{tháng } 7,\mbox{tháng } 8,\mbox{tháng } 10,\mbox{tháng } 12\}$.
	\end{enumerate*}
\end{baitoan}

\begin{baitoan}[\cite{Binh_Toan_6_tap_1}, \textbf{3.}, p. 6]
	Tìm tập hợp các số tự nhiên $x$ sao cho:
	\begin{enumerate*}
		\item[(a)] $x + 3 = 4$;
		\item[(b)] $8 - x = 5$;
		\item[(c)] $x:2 = 0$;
		\item[(d)] $0:x = 0$;
		\item[(e)] $5x = 12$.
	\end{enumerate*}
\end{baitoan}

\begin{baitoan}[\cite{Binh_Toan_6_tap_1}, \textbf{4.}, p. 6]
	Tìm $a,b\in\mathbb{N}$ sao cho $12 < a < b < 16$.
\end{baitoan}

\begin{baitoan}[\cite{Binh_Toan_6_tap_1}, \textbf{5.}, p. 6]
	Viết các số tự nhiên có 4 chữ số trong đó có 2 chữ số $3$, 1 chữ số $2$, 1 chữ số $1$.
\end{baitoan}

\begin{baitoan}[\cite{Binh_Toan_6_tap_1}, \textbf{6.}, p. 6]
	Với cả 2 chữ số I \& X, viết được bao nhiêu số La Mã? (Mỗi chữ số có thể viết nhiều lần, nhưng không viết liên tiếp quá 3 lần).
\end{baitoan}

\begin{baitoan}[\cite{Binh_Toan_6_tap_1}, \textbf{7.}, pp. 6--7]
	\begin{enumerate*}
		\item[(a)] Dùng 3 que diêm, xếp được các số La Mã nào?
		\item[(b)] Để viết các số La Mã từ 4000 trở lên, e.g. số 19520, người ta viết XIXmDXX (chữ m biểu thị \emph{1 nghìn}, m là chữ đầu của từ \emph{mille}, tiếng Latin là 1 nghìn). Hãy viết các số sau bằng chữ số La Mã: 7203, 121512.
	\end{enumerate*}
\end{baitoan}

\begin{baitoan}[\cite{Binh_Toan_6_tap_1}, \textbf{8.}, p. 7]
	Tìm số tự nhiên có tận cùng bằng $3$, biết rằng nếu xóa chữ số hàng đơn vị thì số đó giảm đi $1992$ đơn vị.
\end{baitoan}

\begin{baitoan}[\cite{Binh_Toan_6_tap_1}, \textbf{9.}, p. 7]
	Tìm số tự nhiên có 6 chữ số, biết rằng chữ số hàng đơn vị là $4$ \& nếu chuyển chữ số đó lên hàng đầu tiên thì số đó tăng gấp 4 lần.
\end{baitoan}

\begin{baitoan}[\cite{Binh_Toan_6_tap_1}, \textbf{10.}, p. 7]
	Cho 4 chữ số $a,b,c,d$ khác nhau \& khác $0$. Lập số tự nhiên lớn nhất \& số tự nhiên nhỏ nhất có 4 chữ số gồm cả 4 chữ số ấy. Tổng của 2 số này bằng $11330$. Tìm tổng các chữ số $a + b + c + d$.
\end{baitoan}

\begin{baitoan}[\cite{Binh_Toan_6_tap_1}, \textbf{11.}, p. 7]
	Cho 3 chữ số $a,b,c$ sao cho $0 < a < b < c$.
	\begin{enumerate*}
		\item[(a)] Viết tập hợp $A$ các số tự nhiên có 3 chữ số gồm cả 3 chữ số $a,b,c$.
		\item[(b)] Biết tổng 2 số nhỏ nhất trong tập hợp $A$ bằng $488$. Tìm 3 chữ số $a,b,c$ nói trên.
	\end{enumerate*}
\end{baitoan}

\begin{baitoan}[\cite{Binh_Toan_6_tap_1}, \textbf{12.}, p. 7]
	Tìm 3 chữ số khác nhau \& khác $0$, biết rằng nếu dùng cả 3 chữ số này lập thành các số tự nhiên có 3 chữ số thì 2 số lớn nhất có tổng bằng $1444$.
\end{baitoan}

%------------------------------------------------------------------------------%

\section{$\pm,\cdot,:$ Trên $\mathbb{N}$}
``Ta xác định trên $\mathbb{N}$ 2 phép toán: phép cộng \& phép nhân. Phép cộng có 3 tính chất: giao hoán, kết hợp, cộng với số $0$. Phép nhân có 3 tính chất: giao hoán, kết hợp, nhân với số $1$. Giữa phép nhân \& phép cộng có quan hệ: phép nhân phân phối đối với phép cộng. Giữa thứ tự \& phép toán có quan hệ: $a < b\Rightarrow a + c < b + c$, $a < b\Rightarrow ac < bc$ với $c > 0$.

Trong phạm vi số tự nhiên, phép trừ chỉ thực hiện được khi số bị trừ lớn hơn hoặc bằng số trừ, phép chia chỉ thực hiện được khi số bị chia chia hết cho số chia. Với mọi cặp số tự nhiên $a$ \& $b$ bất kỳ, $b\ne 0$, bao giờ cũng tồn tại duy nhất 2 số tự nhiên $q$ \& $r$ sao cho $a = bq + r$ với $0\le r < b$. Nếu $r = 0$, ta được phép chia hết, khi đó $q$ là thương. Nếu $r\ne0$, ta được phép chia có dư, khi đó $q$ là \textit{thương} \& $r$ là số dư trong phép chia $a$ cho $b$. Trong trường hợp chia $a$ cho $b$ mà chỉ quan tâm đến thương mà không quan tâm đến số dư, ta dùng ký hiệu $[a:b]$ để chỉ thương của phép chia, e.g., $[6:4] = 1$.'' -- \cite[\S2, p. 7]{Binh_Toan_6_tap_1}

\begin{baitoan}[\cite{Binh_Toan_6_tap_1}, Ví dụ 4, p. 7]
	Cho $A = 137\cdot454 + 206$, $B = 453\cdot138 - 110$. Không tính giá trị của $A$ \& $B$, chứng minh $A = B$.
\end{baitoan}

\begin{baitoan}[\cite{Binh_Toan_6_tap_1}, Ví dụ 5, p. 8]
	Tìm kết quả của phép nhân: $A = \underbrace{33\ldots3}_{\scriptsize50\,\mbox{\rm số}}\cdot\underbrace{99\ldots9}_{\scriptsize50\,\mbox{\rm số}}$.
\end{baitoan}

\begin{baitoan}[\cite{Binh_Toan_6_tap_1}, Ví dụ 6, p. 8]
	Tổng của 2 số tự nhiên khác nhau gấp 3 hiệu của chúng. Tìm thương của 2 số tự nhiên đó.
\end{baitoan}

\begin{baitoan}[\cite{Binh_Toan_6_tap_1}, Ví dụ 7, p. 8]
	Khi chia $a\in\mathbb{N}$ cho $54$ được số dư là $38$. Chia số $a$ cho $18$ được thương là $14$ \& còn dư. Tìm $a$.
\end{baitoan}

\begin{baitoan}[\cite{Binh_Toan_6_tap_1}, Ví dụ 8, p. 8]
	Tìm 2 số tự nhiên lớn hơn $0$ sao cho tích của 2 số ấy gấp đôi tổng của chúng.
\end{baitoan}

\begin{baitoan}[\cite{Binh_Toan_6_tap_1}, Ví dụ 9, p. 9]
	Điền các số tự nhiên $1,2,3,4,5$ vào các dấu $\star$ để kết quả phép tính bằng $6$: $\star+\star-\star\cdot\star:\star$.
\end{baitoan}

\begin{baitoan}[\cite{Binh_Toan_6_tap_1}, Ví dụ 10, p. 9]
	Giá tiền $7$ quyển vở nhiều hơn giá tiền $8$ bút chì. Hỏi giá tiền $8$ quyển vở \& giá tiền $9$ bút chì, đằng nào nhiều hơn?
\end{baitoan}

\begin{baitoan}[\cite{Binh_Toan_6_tap_1}, Ví dụ 11, p. 9]
	Cho 6 số tự nhiên khác nhau có tổng bằng $50$. Chứng minh trong 6 số đó tồn tại 3 số có tổng lớn hơn hoặc bằng $30$.
\end{baitoan}

\begin{baitoan}[\cite{Binh_Toan_6_tap_1}, \textbf{13.}, p. 10]
	Có thể viết được hay không 9 số vào 1 bảng vuông $3\times3$ sao cho: Tổng các số trong 3 dòng theo thứ tự bằng $352,463,541$; tổng các số trong 3 cột theo thứ tự bằng $335,687,234$?
\end{baitoan}

\begin{baitoan}[\cite{Binh_Toan_6_tap_1}, \textbf{14.}, p. 10]
	Cho 9 số xếp vào 9 ô thành 1 hàng ngang, trong đó số đầu tiên là $4$, số cuối cùng là $8$ \& tổng 3 số ở 3 ô liền nhau bất kỳ bằng $17$. Tìm 9 số đó.
\end{baitoan}

\begin{baitoan}[\cite{Binh_Toan_6_tap_1}, \textbf{15.}, p. 10]
	Tìm số có 3 chữ số, biết chữ số hàng trăm gấp 4 lần chữ số hàng đơn vị \& nếu viết số ấy theo thứ tự ngược lại thì nó giảm đi $594$ đơn vị.
\end{baitoan}

\begin{baitoan}[\cite{Binh_Toan_6_tap_1}, \textbf{16.}, p. 10]
	Thay các dấu $\star$ bởi các chữ số thích hợp: $\star\star\star\star - \star\star\star = \star\star$, biết số bị trừ, số trừ, \& hiệu đều không đổi nếu đọc mỗi số từ phải sang trái.
\end{baitoan}

\begin{baitoan}[\cite{Binh_Toan_6_tap_1}, \textbf{17.}, p. 10]
	Xếp 9 số $1,2,3,4,5,6,7,8,9$ vào các hình tròn đặt trên các cạnh của tam giác:
	\begin{figure}[H]
		\centering
		\includegraphics[scale=0.1]{Binh_17_p_10}
	\end{figure}
	sao cho tổng các số trên cạnh nào của tam giác cũng bằng $17$.
\end{baitoan}
Bài tập phụ thuộc vào hình vẽ: \cite[\textbf{18.}, p. 10]{Binh_Toan_6_tap_1}.

\begin{baitoan}[\cite{Binh_Toan_6_tap_1}, \textbf{19.}, p. 10]
	Hiệu của 2 số là $4$. Nếu gấp 1 số lên $3$ lần, giữ nguyên số kia thì hiệu của chúng bằng $60$. Tìm 2 số đó.
\end{baitoan}

\begin{baitoan}[\cite{Binh_Toan_6_tap_1}, \textbf{20.}, p. 10]
	Cho số $123456789$. Đặt 1 số dấu ``$+$'' \& ``$-$'' vào giữa các chữ số để kết quả của phép tính bằng $100$.
\end{baitoan}

\begin{baitoan}[\cite{Binh_Toan_6_tap_1}, \textbf{21.}, p. 10]
	Cho số $987654321$. Đặt 1 số dấu ``$+$'' \& ``$-$'' vào giữa các chữ số để kết quả của phép tính bằng:
	\begin{enumerate*}
		\item[(a)] $100$;
		\item[(b)] $99$.
	\end{enumerate*}
\end{baitoan}

\begin{baitoan}[\cite{Binh_Toan_6_tap_1}, \textbf{22.}, p. 10]
	Tìm giá trị lớn nhất của hiệu $\overline{bd} - \overline{ac}$ biết $a < b < c < d$.
\end{baitoan}

\begin{baitoan}[\cite{Binh_Toan_6_tap_1}, \textbf{23.}, p. 10]
	Tìm 6 chữ số khác nhau $a,b,c,d,e,f$ sao cho $A = \overline{abc} - \overline{def}$ có giá trị lớn nhất.
\end{baitoan}

\begin{baitoan}[\cite{Binh_Toan_6_tap_1}, \textbf{24.}, p. 11]
	Cho 6 chữ số khác nhau $a,b,c,d,e,f$. Gọi $A = \overline{abc} + \overline{bcd} + \overline{cde} + \overline{def}$.
	\begin{enumerate*}
		\item[(a)] Tìm giá trị lớn nhất của $A$;
		\item[(b)] Tìm giá trị nhỏ nhất của $A$.
	\end{enumerate*}
\end{baitoan}

\begin{baitoan}[\cite{Binh_Toan_6_tap_1}, \textbf{25.}, p. 11]
	Tìm 2 số, biết tổng của chúng gấp $5$ lần hiệu của chúng, tích của chúng gấp $24$ lần hiệu của chúng.
\end{baitoan}

\begin{baitoan}[\cite{Binh_Toan_6_tap_1}, \textbf{26.}, p. 11]
	Tìm 2 số biết tổng của chúng gấp $7$ lần hiệu của chúng, còn tích của chúng gấp $192$ lần hiệu của chúng.
\end{baitoan}

\begin{baitoan}[\cite{Binh_Toan_6_tap_1}, \textbf{27.}, p. 11]
	Tích của 2 số là $6210$. Nếu giảm 1 thừa số đi $7$ đơn vị thì tích mới là $5265$. Tìm các thừa số của tích.
\end{baitoan}

\begin{baitoan}[\cite{Binh_Toan_6_tap_1}, \textbf{28.}, p. 11]
	Bảo làm 1 phép nhân, trong đó số nhân là $102$. Nhưng khi viết số nhân, Bảo đã quên không viết chữ số $0$ nên tích bị giảm đi $21870$ đơn vị so với tích đúng. Tìm số bị nhân của phép nhân đó.
\end{baitoan}

\begin{baitoan}[\cite{Binh_Toan_6_tap_1}, \textbf{29.}, p. 11]
	1 học sinh nhân $78$ với số nhân là số có 2 chữ số, trong đó chữ số hàng chục gấp 3 lần chữ số hàng đơn vị. Do nhầm lẫn bạn đó viết đổi thứ tự 2 chữ số của số nhân, nên tích giảm đi $2808$ đơn vị so với tích đúng. Tìm số nhân đúng.
\end{baitoan}

\begin{baitoan}[\cite{Binh_Toan_6_tap_1}, \textbf{30.}, p. 11]
	1 học sinh nhân 1 số với $463$. Vì bạn đó viết các chữ số tận cùng của các tích riêng ở cùng 1 cột nên tích bằng $30524$. Tìm số bị nhân.
\end{baitoan}

\begin{baitoan}[\cite{Binh_Toan_6_tap_1}, \textbf{31.}, p. 11]
	Chứng minh $11111111 - 2222$ có thể viết được thành 1 tích của 2 thừa số bằng nhau.
\end{baitoan}

\begin{baitoan}[\cite{Binh_Toan_6_tap_1}, \textbf{32.}, p. 11]
	Chỉ ra 2 số khác nhau sao cho, nếu nhân mỗi số với $7$ thì ta được kết quả là các số gồm toàn các chữ số $9$.
\end{baitoan}

\begin{baitoan}[\cite{Binh_Toan_6_tap_1}, \textbf{33.}${}^\star$, p. 11]
	Tính $\underbrace{33\ldots3}_{\scriptsize50\,\mbox{\rm số}}\cdot\underbrace{33\ldots3}_{\scriptsize50\,\mbox{\rm số}}$.
\end{baitoan}

\begin{baitoan}[\cite{Binh_Toan_6_tap_1}, \textbf{34.}${}^\star$, p. 11]
	Tìm tổng các chữ số của tích:
	\begin{enumerate*}
		\item[(a)] $\underbrace{88\ldots8}_{\scriptsize21\,\mbox{\rm số}}\cdot\underbrace{99\ldots9}_{\scriptsize21\,\mbox{\rm số}}$.
		\item[(b)] $\underbrace{99\ldots9}_{\scriptsize94\,\mbox{\rm số}}\cdot\underbrace{44\ldots4}_{\scriptsize94\,\mbox{\rm số}}$.
	\end{enumerate*}
\end{baitoan}

\begin{baitoan}[\cite{Binh_Toan_6_tap_1}, \textbf{35.}, p. 11]
	Chứng minh các số sau có thể viết được thành 1 tích của 2 số tự nhiên liên tiếp:
	\begin{enumerate*}
		\item[(a)] $111222$;
		\item[(b)] $444222$.
	\end{enumerate*}
\end{baitoan}

\begin{baitoan}[\cite{Binh_Toan_6_tap_1}, \textbf{36.}, p. 11]
	Tìm 2 số tự nhiên có thương bằng $35$, biết nếu số bị chia tăng thêm $1056$ đơn vị thì thương bằng $57$.
\end{baitoan}

\begin{baitoan}[\cite{Binh_Toan_6_tap_1}, \textbf{37.}, p. 11]
	Tìm số bị chia \& số chia biết thương bằng $6$, số dư bằng $49$, tổng của số bị chia, số chia \& số dư bằng $595$.
\end{baitoan}

\begin{baitoan}[\cite{Binh_Toan_6_tap_1}, \textbf{38.}, p. 11]
	1 phép chia có thương bằng $4$, số dư bằng $25$. Tổng của số bị chia, số chia, \& số dư bằng $210$. Tìm số bị chia \& số chia.
\end{baitoan}

\begin{baitoan}[\cite{Binh_Toan_6_tap_1}, \textbf{39.}, p. 11]
	Trong hội trường có $680$ người ngồi. Tất cả có $25$ dãy ghế, mỗi dãy ghế có $30$ chỗ ngồi. Ít nhất có bao nhiêu dãy ghế có số chỗ ngồi như nhau?
\end{baitoan}

\begin{baitoan}[\cite{Binh_Toan_6_tap_1}, \textbf{40.}, p. 12]
	\begin{enumerate*}
		\item[(a)] Trong 1 năm có, có ít nhất bao nhiêu ngày Chủ nhật? Có nhiều nhất bao nhiêu ngày Chủ nhật?
		\item[(b)] Ngày 1-1 năm nay rơi vào ngày Chủ nhật. Ngày 1-1 năm sau rơi vào ngày thứ mấy?
	\end{enumerate*}
\end{baitoan}

\begin{baitoan}[\cite{Binh_Toan_6_tap_1}, \textbf{41.}, p. 12]
	Tháng $8$ của 1 năm có 4 ngày thứ 5 \& 5 ngày thứ Tư. Hỏi ngày đầu tiên của tháng đó là ngày thứ mấy?
\end{baitoan}

\begin{baitoan}[\cite{Binh_Toan_6_tap_1}, \textbf{42.}, p. 12]
	Ngày 19-8-2020 vào ngày thứ 2. Tính xem ngày 19-8-1945 vào ngày nào trong tuần?
\end{baitoan}

\begin{baitoan}[\cite{Binh_Toan_6_tap_1}, \textbf{43.}, p. 12]
	Tìm thương của 1 phép chia, biết nếu thêm $15$ vào số bị chia \& thêm $5$ vào số chia thì thương \& số dư không đổi.
\end{baitoan}

\begin{baitoan}[\cite{Binh_Toan_6_tap_1}, \textbf{44.}, p. 12]
	Tìm thương của 1 phép chia, biết nếu tăng số bị chia $90$ đơn vị, tăng số chia $6$ đơn vị thì thương \& số dư không đổi.
\end{baitoan}

\begin{baitoan}[\cite{Binh_Toan_6_tap_1}, \textbf{45.}, p. 12]
	Tìm thương của 1 phép chia, biết nếu tăng số bị chia $73$ đơn vị, tăng số chia $4$ đơn vị thì thương không đổi, còn số dư tăng $5$ đơn vị.
\end{baitoan}

\begin{baitoan}[\cite{Binh_Toan_6_tap_1}, \textbf{46.}, p. 12]
	Xác định phép chia, biết số bị chia, số chia, thương, \& số dư là 4 số trong các số sau:
	\begin{enumerate*}
		\item[(a)] $3,4,16,256,772$;
		\item[(b)] $2,3,9,27,81,243,567$.
	\end{enumerate*}
\end{baitoan}

\begin{baitoan}[\cite{Binh_Toan_6_tap_1}, \textbf{47.}, p. 12]
	Khi chia 1 số tự nhiên gồm 3 chữ số như nhau cho 1 số tự nhiên gồm 3 chữ số như nhau, ta được thương là $2$ là còn dư. Nếu xóa 1 chữ số ở số bị chia \& xóa 1 chữ số ở số chia thì thương của phép chia vẫn bằng $2$ nhưng số dư giảm hơn trước là $100$. Tìm số bị chia \& số chia lúc đầu.
\end{baitoan}

\begin{baitoan}[\cite{Binh_Toan_6_tap_1}, \textbf{48.}, p. 12]
	Trong 1 phép chia có dư, số bị chia gồm 4 chữ số như nhau, số chia gồm 3 chữ số như nhau, thương bằng $13$ \& còn dư. Nếu xóa 1 chữ số ở số bị chia, xóa 1 chữ số ở số chia thì thương không đổi, còn số dư giảm hơn trước là $100$ đơn vị. Tìm số bị chia \& số chia lúc đầu.
\end{baitoan}

\begin{baitoan}[\cite{Binh_Toan_6_tap_1}, \textbf{49.}, p. 12]
	Tính nhanh:
	\begin{enumerate*}
		\item[(a)] $19\cdot64 + 76\cdot34$;
		\item[(b)] $35\cdot12 + 65\cdot13$;
		\item[(c)] $136\cdot68 + 16\cdot272$;
		\item[(d)] $(2 + 4 + 6 + \cdots + 100)\cdot(36\cdot333 - 108\cdot111)$;
		\item[(e)] $19991999\cdot1998 - 19981998\cdot1999$.
	\end{enumerate*}
\end{baitoan}

\begin{baitoan}[\cite{Binh_Toan_6_tap_1}, \textbf{50.}, p. 12]
	Không tính cụ thể các giá trị của $A$ \& $B$, cho biết số nào lớn hơn \& lớn hơn bao nhiêu?
	\begin{enumerate*}
		\item[(a)] $A = 1998\cdot1998$, $B = 1996\cdot2000$;
		\item[(b)] $A = 2000\cdot2000$, $B = 1990\cdot2010$;
		\item[(c)] $A = 25\cdot33 - 10$, $B = 31\cdot26 + 10$;
		\item[(d)] $A = 32\cdot53 - 31$, $B = 53\cdot31 + 32$.
	\end{enumerate*}
\end{baitoan}

\begin{baitoan}[\cite{Binh_Toan_6_tap_1}, \textbf{51.}, p. 12]
	Tìm thương của phép chia sau mà không tính kết quả cụ thể của số bị chia \& số chia: $\dfrac{37\cdot13 - 13}{24 + 37\cdot12}$.
\end{baitoan}

\begin{baitoan}[\cite{Binh_Toan_6_tap_1}, \textbf{52.}, p. 13]
	Tính:
	\begin{enumerate*}
		\item[(a)] $A = \dfrac{101 + 100 + 99 + 98 + \cdots + 3 + 2 + 1}{101 - 100 + 99 - 98 + \cdots + 3 - 2 + 1}$;
		\item[(b)] $B = \dfrac{3737\cdot43 - 4343\cdot37}{2 + 4 + 6 + \cdots + 100}$.
	\end{enumerate*}
\end{baitoan}

\begin{baitoan}[\cite{Binh_Toan_6_tap_1}, \textbf{53.}, p. 13]
	Vận dụng tính chất các phép tính để tìm các kết quả bằng cách nhanh chóng:
	\begin{enumerate*}
		\item[(a)] $1990\cdot1990 - 1992\cdot1988$;
		\item[(b)] $(1374\cdot57 + 687\cdot86):(26\cdot13 + 74\cdot14)$;
		\item[(c)] $(124\cdot237 + 152):(870 + 235\cdot122)$;
		\item[(d)] $\dfrac{423134\cdot846267 - 423133}{846267\cdot423133 + 423134}$.
	\end{enumerate*}
\end{baitoan}

\begin{baitoan}[\cite{Binh_Toan_6_tap_1}, \textbf{54.}, p. 13]
	Tìm $a\in\mathbb{N}$ biết:
	\begin{enumerate*}
		\item[(a)] $697:\dfrac{15a + 364}{a} = 17$;
		\item[(b)] $92\cdot4 - 27 = \dfrac{a + 350}{a} + 315$.
	\end{enumerate*}
\end{baitoan}

\begin{baitoan}[\cite{Binh_Toan_6_tap_1}, \textbf{55.}, p. 13]
	Tìm $x\in\mathbb{N}$ biết:
	\begin{enumerate*}
		\item[(a)] $720:[41 - (2x - 5)] = 40$;\\
		\item[(b)] $\sum_{i=1}^{100} (x + i) = (x + 1) + (x + 2) + \cdots + (x + 100) = 5750$.
	\end{enumerate*}
\end{baitoan}

\begin{baitoan}[\cite{Binh_Toan_6_tap_1}, \textbf{56.}, p. 13]
	Cho số $12345678$. Đặt các dấu phép tính \& dấu ngoặc để kết quả của phép tính bằng $9$.
\end{baitoan}

\begin{baitoan}[\cite{Binh_Toan_6_tap_1}, \textbf{57.}, p. 13]
	Viết 5 dãy tính có kết quả bằng $100$, với 6 chữ số $5$ cùng với dấu các phép tính (\& dấu ngoặc nếu cần).
\end{baitoan}

\begin{baitoan}[\cite{Binh_Toan_6_tap_1}, \textbf{58.}, p. 13]
	\begin{enumerate*}
		\item[(a)] Viết dãy tính có kết quả bằng $100$, với 5 chữ số như nhau cùng với dấu các phép tính (\& dấu ngoặc nếu cần).
		\item[(b)] Cũng hỏi như trên với 6 chữ số như nhau.
	\end{enumerate*}
\end{baitoan}

\begin{baitoan}[\cite{Binh_Toan_6_tap_1}, \textbf{59.}, p. 13]
	\begin{enumerate*}
		\item[(a)] Thực hiện các phép tính sau (kết quả khá đặc biệt): $1\cdot8 + 1$, $12\cdot8 + 1$, $123\cdot8 + 1$, $1234\cdot8 + 1$.
		\item[(b)] Viết tiếp 4 dòng nữa theo quy luật trên.
	\end{enumerate*}
\end{baitoan}

\begin{baitoan}[\cite{Binh_Toan_6_tap_1}, \textbf{60.}, p. 13]
	Điền các số $1,2,3,4,5$ vào các dấu $\star$ để kết quả của phép tính bằng $3$: $\star+\star-\star\cdot\star:\star$.
\end{baitoan}

\begin{baitoan}[\cite{Binh_Toan_6_tap_1}, \textbf{61.}, p. 13]
	Giá tiền $1$ quyển sách, $6$ quyển vở, $3$ chiếc bút là $7700$ đồng, giá tiền $8$ quyển sách, $6$ quyển vở, $6$ chiếc bút là $16000$ đồng. So sánh giá tiền 1 quyển sách \& 1 quyển vở.
\end{baitoan}

\begin{baitoan}[\cite{Binh_Toan_6_tap_1}, \textbf{62.}, p. 13]
	Viết liên tiếp các số tự nhiên từ $1$ đến $15$, ta được: $A = 1234\ldots1415$. Xóa đi $15$ chữ số của số $A$ để các chữ số còn lại (vẫn giữ nguyên thứ tự như trước) tạo thành:
	\begin{enumerate*}
		\item[(a)] Số nhỏ nhất;
		\item[(b)] Số lớn nhất.
	\end{enumerate*}
\end{baitoan}

\begin{baitoan}[\cite{Binh_Toan_6_tap_1}, \textbf{63.}, p. 14]
	Cho số $123\ldots20$ (viết liên tiếp các số tự nhiên từ $1$ đến $20$). Xóa đi $20$ chữ số để số còn lại có giá trị:
	\begin{enumerate*}
		\item[(a)] Nhỏ nhất;
		\item[(b)] Lớn nhất.
	\end{enumerate*}
\end{baitoan}

\begin{baitoan}[\cite{Binh_Toan_6_tap_1}, \textbf{64.}, p. 14]
	Tìm giá trị nhỏ nhất của hiệu giữa 1 số tự nhiên có 2 chữ số với tổng các chữ số của nó.
\end{baitoan}

\begin{baitoan}[\cite{Binh_Toan_6_tap_1}, \textbf{65.}, p. 14]
	Tìm số chia \& số dư biết số bị chia bằng $113$, thương bằng $5$.
\end{baitoan}

\begin{baitoan}[\cite{Binh_Toan_6_tap_1}, \textbf{66.}, p. 14]
	Tìm số chia \& số dư biết số bị chia bằng $813$, thương bằng $15$, số dư gồm 2 chữ số như nhau.
\end{baitoan}

\begin{baitoan}[\cite{Binh_Toan_6_tap_1}, \textbf{67.}, p. 14]
	Tìm số chia \& số dư của phép chia $542$ cho 1 số tự nhiên biết thương bằng $12$.
\end{baitoan}

\begin{baitoan}[\cite{Binh_Toan_6_tap_1}, \textbf{68.}, p. 14]
	1 học sinh trong 5 năm học từ lớp $5$ đến lớp $9$ đã qua $31$ kỳ thi, trong đó số kỳ thi ở năm sau nhiều hơn số kỳ thi ở năm trước, \& số kỳ thi ở năm cuối gấp 3 lần số kỳ thi ở năm đầu. Hỏi học sinh đó thi bao nhiêu kỳ ở năm thứ 4?
\end{baitoan}

\begin{baitoan}[\cite{Binh_Toan_6_tap_1}, \textbf{69.}, p. 14]
	Tìm 2 số tự nhiên sao cho tổng của 2 số ấy bằng tích của chúng.
\end{baitoan}

\begin{baitoan}[\cite{Binh_Toan_6_tap_1}, \textbf{70.}, p. 14]
	Tìm 3 số tự nhiên khác $0$ biết tổng của 3 số ấy bằng tích của chúng.
\end{baitoan}

%------------------------------------------------------------------------------%

\section{Lũy Thừa}

%------------------------------------------------------------------------------%

\printbibliography[heading=bibintoc]
	
\end{document}