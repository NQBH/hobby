\documentclass{article}
\usepackage[backend=biber,natbib=true,style=alphabetic,maxbibnames=50]{biblatex}
\addbibresource{/home/nqbh/reference/bib.bib}
\usepackage[utf8]{vietnam}
\usepackage{tocloft}
\renewcommand{\cftsecleader}{\cftdotfill{\cftdotsep}}
\usepackage[colorlinks=true,linkcolor=blue,urlcolor=red,citecolor=magenta]{hyperref}
\usepackage{amsmath,amssymb,amsthm,float,graphicx,mathtools}
\allowdisplaybreaks
\newtheorem{assumption}{Assumption}
\newtheorem{baitoan}{}
\newtheorem{cauhoi}{Câu hỏi}
\newtheorem{conjecture}{Conjecture}
\newtheorem{corollary}{Corollary}
\newtheorem{dangtoan}{Dạng toán}
\newtheorem{definition}{Definition}
\newtheorem{dinhly}{Định lý}
\newtheorem{dinhnghia}{Định nghĩa}
\newtheorem{example}{Example}
\newtheorem{ghichu}{Ghi chú}
\newtheorem{hequa}{Hệ quả}
\newtheorem{hypothesis}{Hypothesis}
\newtheorem{lemma}{Lemma}
\newtheorem{luuy}{Lưu ý}
\newtheorem{nhanxet}{Nhận xét}
\newtheorem{notation}{Notation}
\newtheorem{note}{Note}
\newtheorem{principle}{Principle}
\newtheorem{problem}{Problem}
\newtheorem{proposition}{Proposition}
\newtheorem{question}{Question}
\newtheorem{remark}{Remark}
\newtheorem{theorem}{Theorem}
\newtheorem{vidu}{Ví dụ}
\usepackage[left=1cm,right=1cm,top=5mm,bottom=5mm,footskip=4mm]{geometry}
\def\labelitemii{$\circ$}
\DeclareRobustCommand{\divby}{%
	\mathrel{\vbox{\baselineskip.65ex\lineskiplimit0pt\hbox{.}\hbox{.}\hbox{.}}}%
}
\title{Problem: Calculus on $\mathbb{N}$\\Bài Tập: Các Phép Tính Trên Tập Hợp Các Số Tự Nhiên}
\date{}

\begin{document}
\maketitle
\vspace{-2.5cm}

%------------------------------------------------------------------------------%

\section{$\pm,\cdot,:$ on $\mathbb{N}$}

\begin{baitoan}[\cite{Tuyen_Toan_6}, Ví dụ 3, p. 8]
	1 học sinh khi nhân 1 số với $31$ đã đặt các tích riêng thẳng hàng như trong phép cộng nên tích đã giảm đi $540$ đơn vị so với tích đúng. Tìm tích đúng.
\end{baitoan}

\begin{baitoan}[\cite{Tuyen_Toan_6}, Ví dụ 4, p. 8]
	Cho 2 số không chia hết cho $3$, khi chia cho $3$, khi chia cho $3$ được các số dư khác nhau. Chứng minh tổng của 2 số đó chia hết cho $3$.
\end{baitoan}

\begin{baitoan}[\cite{Tuyen_Toan_6}, 14., p. 9]
	Tính hợp lý: (a) $38 + 41 + 117 + 159 + 62$. (b) $73 + 86 + 978 + 914 + 3022$. (c) $341\cdot67 + 341\cdot16 + 659\cdot83$. (d) $42\cdot53 + 47\cdot156 - 47\cdot114$.
\end{baitoan}

\begin{baitoan}[\cite{Tuyen_Toan_6}, 15., p. 9]
	Tính giá trị của biểu thức: (a) $A = (100 - 1)\cdot(100 - 2)\cdots(100 - n)$ với $n\in\mathbb{N}^\star$ \& tích trên có đúng $100$ thừa số. (b) $B = 13a + 19b + 4a - 2b$ với $a + b = 100$.
\end{baitoan}

\begin{baitoan}[\cite{Tuyen_Toan_6}, 16., p. 9]
	Không tính giá trị cụ thể, so sánh giá trị 2 biểu thức: (a) $A = 199\cdot201$ \& $B = 200\cdot200$. (b) $C = 35'\cdot53 - 18$ \& $D = 35 + 53\cdot34$.
\end{baitoan}

\begin{baitoan}[\cite{Tuyen_Toan_6}, 17., p. 9]
	Tính hợp lý: (a) $(44\cdot52\cdot60):(11\cdot13\cdot15)$. (b) $123\cdot456456 - 456\cdot123123$. (c) $(98\cdot7676 - 9898\cdot76):(2021\cdot2022\cdot2023\cdots2030)$.
\end{baitoan}

\begin{baitoan}[\cite{Tuyen_Toan_6}, 18., p. 9]
	Tìm giá trị nhỏ nhất của biểu thức: $A = 2021 - 1021:(999 - x)$.
\end{baitoan}

\begin{baitoan}[\cite{Tuyen_Toan_6}, 20., p. 9]
	Tìm số hạng thứ 5, thứ $n$ của dãy số: (a) $2,3,7,25,\ldots$. (b) $8,30,72,140,\ldots$.
\end{baitoan}

\begin{baitoan}[\cite{Tuyen_Toan_6}, 21., p. 9]
	Tìm $x$: (a) $(x + 74) - 318 = 200$. (b) $3636:(12x - 91) = 36$. (c) $(x:23 + 45)\cdot67 = 8911$.
\end{baitoan}

\begin{baitoan}[\cite{Tuyen_Toan_6}, 22., p. 9]
	1 nong tằm là $5$ nong kén. 1 nong kén là $9$ nén tơ. Hỏi muốn được $540$ nén tơ thì phải chăn bao nhiêu nong tằm?
\end{baitoan}

\begin{baitoan}[\cite{Tuyen_Toan_6}, 23., p. 9]
	2 số tự nhiên $a,b$ chia hết cho $m$ có cùng số dư, $a\ge b$. Chứng tỏ $a - b$ chia hết cho $m$.
\end{baitoan}

\begin{baitoan}[\cite{Tuyen_Toan_6}, 24., p. 9]
	Trong 1 phép chia có số bị chia là $155$, số dư là $12$. Tìm số chia \& thương.
\end{baitoan}

\begin{baitoan}[\cite{Tuyen_Toan_6}, 25., p. 9]
	Viết tập hợp $A$ các số tự nhiên $x$ biết lấy $x$ chia cho $12$ ta được thương bằng số dư.
\end{baitoan}

\begin{baitoan}[\cite{Tuyen_Toan_6}, 26., p. 10]
	Chia $129$ cho 1 số ta được số dư là $10$. Chia $61$ cho số đó ta cũng được số dư là $10$. Tìm số chia.
\end{baitoan}

\begin{baitoan}[\cite{Tuyen_Toan_6}, 27., p. 10]
	Cho 3 chữ số $a,b,c$ khác nhau \& khác $0$. Với cùng cả 3 chữ số này có thể lập được bao nhiêu số có 3 chữ số?
\end{baitoan}

\begin{baitoan}[\cite{Tuyen_Toan_6}, 28., p. 10]
	Cho 4 chữ số khác nhau \& khác $0$. (a) Với cùng cả 4 chữ số này có thể lập được bao nhiêu số có 4 chữ số? (b) Có thể lập được bao nhiêu số có 2 chữ số khác nhau trong 4 chữ số đã cho?
\end{baitoan}

\begin{baitoan}[\cite{Tuyen_Toan_6}, 29., p. 10]
	Cho 4 chữ số khác nhau trong đó có 1 chữ số $0$. Với cùng cả 4 chữ số này có thể lập được bao nhiêu số có 4 chữ số?
\end{baitoan}

\begin{baitoan}[\cite{Tuyen_Toan_6}, 30., p. 10]
	Anh Bách đi mua bánh kẹo tại 1 siêu thị, thanh toán bằng 1 phiếu mua hàng trị giá $100000$ đồng. Siêu thị không trả lại số tiền thừa. Giúp anh Bách chọn mua vừa hết số tiền ghi trong phiếu. Bảng giá 1 số mặt hàng có bán:
	\begin{table}[H]
		\centering
		\begin{tabular}{|l|c|c|c|}
			\hline
			STT & Tên hàng & Đơn vị & Giá bán \\
			\hline
			1 & Bánh đậu xanh & Hộp & 31 500 đồng \\
			\hline
			2 & Bánh bông lan & Gói & 23 500 đồng \\
			\hline
			3 & Bánh gạo & Gói & 19 000 đồng \\
			\hline
			4 & Bánh gạo chiên & Gói & 17 800 đồng \\
			\hline
			5 & Bánh quy & Gói & 13 500 đồng \\
			\hline
			6 & Bánh xốp & Gói & 5300 đồng \\
			\hline
			7 & Kẹo hương dâu & Gói & 2 500 đồng \\
			\hline
		\end{tabular}
	\end{table}
\end{baitoan}

%------------------------------------------------------------------------------%

\section{Exponentiation on $\mathbb{N}$ -- Lũy Thừa với Số Mũ Tự Nhiên}

\begin{dinhnghia}[Số chính phương]
	{\rm Số chính phương} là số bằng bình phương của 1 số tự nhiên, i.e., $a$ là số chính phương $\Leftrightarrow a = n^2$ với $n\in\mathbb{N}$ nào đó.
\end{dinhnghia}

\begin{baitoan}[\cite{Tuyen_Toan_6}, Ví dụ 5, p. 11]
	Chứng minh tổng $\sum_{i=1}^5 i^3 = 1^3 + 2^3 + 3^3 + 4^3 + 5^3$ là 1 số chính phương.
\end{baitoan}

\begin{baitoan}[\cite{Tuyen_Toan_6}, Ví dụ 6, p. 11]
	Tìm $x\in\mathbb{N}$ biết $2\cdot3^x = 162$.
\end{baitoan}

\begin{baitoan}[\cite{Tuyen_Toan_6}, Ví dụ 7, p. 11]
	Tìm $x\in\mathbb{N}$ biết $(x + 2)^4 = 5^2\cdot25$.
\end{baitoan}

\begin{baitoan}[\cite{Tuyen_Toan_6}, 31., p. 11]
	Trong các số $2^4,3^4,4^2,4^3,99^0,0^{99},1^n$ với $n\in\mathbb{N}^\star$, các số nào bằng nhau? Số nào nhỏ nhất? Số nào lớn nhất?
\end{baitoan}

\begin{baitoan}[\cite{Tuyen_Toan_6}, 32., p. 11]
	Kiểm tra đẳng thức $152 - 5^3 = 10^2$ đúng hay sai. Nếu sai, di chuyển 1 chữ số đến vị trí khác để được đẳng thức đúng.
\end{baitoan}

\begin{baitoan}[\cite{Tuyen_Toan_6}, 33., p. 11]
	Chứng minh mỗi tổng{\tt/}hiệu sau là 1 số chính phương: (a) $3^2 + 4^2$. (b) $13^2 - 5^2$. (c) $1^3 + 2^3 + 3^3 + 4^3$.
\end{baitoan}

\begin{baitoan}[\cite{Tuyen_Toan_6}, 34., pp. 11--12]
	Viết các tổng{\tt/}hiệu sau dưới dạng 1 lũy thừa với số mũ lớn hơn $1$. (a) $17^2 - 15^2$. (b) $4^3 - 2^3 + 5^2$. 
\end{baitoan}

\begin{baitoan}[\cite{Tuyen_Toan_6}, 35., p. 12]
	Viết số $729$ dưới dạng 1 lũy thừa với 3 cơ số khác nhau \& số mũ lớn hơn $1$.
\end{baitoan}

\begin{baitoan}[\cite{Tuyen_Toan_6}, 36., p. 12]
	Viết các tích{\tt/}thương sau dưới dạng lũy thừa của 1 số: (a) $2^5\cdot8^4$. (b) $25^6\cdot125^3$. (c) $625^5:25^7$. (d) $12^3\cdot3^3$.
\end{baitoan}

\begin{baitoan}[\cite{Tuyen_Toan_6}, 37., p. 12]
	Tính $6^{3^1},3^{2^3},7^{1^{2^{3^4}}},2020^{3^{0^{1^0}}}$.
\end{baitoan}

\begin{baitoan}[\cite{Tuyen_Toan_6}, 38., p. 12]
	Tìm $x\in\mathbb{N}$ biết: (a) $(3x - 2)^3 = 64$. (b) $(2x + 5)^4 = 3^4\cdot5^4$.
\end{baitoan}

\begin{baitoan}[\cite{Tuyen_Toan_6}, 39., p. 12]
	Tìm $x\in\mathbb{N}$ biết: (a) $5^x + 5^{x + 2} = 650$. (b) $3^{x + 4} = 9^{2x - 1}$.
\end{baitoan}

\begin{baitoan}[\cite{Tuyen_Toan_6}, 40., p. 12]
	Tìm $x\in\mathbb{N}$ biết: (a) $2^x - 15 = 17$. (b) $(7x - 11)^3 = 2^5\cdot5^2 + 200$.
\end{baitoan}

\begin{baitoan}[\cite{Tuyen_Toan_6}, 41., p. 12]
	Tìm $x\in\mathbb{N}$ biết: (a) $x^{10} = 1^x$. (b) $x^{10} = x$. (c) $(2x - 15)^5 = (2x - 15)^3$.
\end{baitoan}

\begin{baitoan}[\cite{Tuyen_Toan_6}, 42., p. 12]
	Tìm $m,n\in\mathbb{N}$ thỏa $2^m + 2^n = 40$.
\end{baitoan}

\begin{baitoan}[\cite{Tuyen_Toan_6}, 43., p. 12]
	Số $4^6\cdot5^{14}$ có bao nhiêu chữ số nếu viết trong hệ thập phân ở dạng thông thường (không có số mũ)?
\end{baitoan}

\begin{baitoan}[\cite{Tuyen_Toan_6}, 44., p. 12]
	Trong âm nhạc, về trường đột thì: 1 nốt tròn bằng 2 nốt trắng, 1 nốt trắng bằng 2 nốt đen, 1 nốt đen bằng 2 nốt móc đơn, 1 nốt móc đơn bằng 2 nốt móc kép, 1 nốt móc kép bằng 2 nốt móc 3, 1 nốt móc 3 bằng 2 nốt móc 4. Dùng lũy thừa của 1 số tự nhiên để: (a) Diễn tả mối quan hệ về trường độ giữa nốt tròn với các nốt nhạc khác. (b) Cho biết nốt nhạc có trường độ gấp $8$ lần nốt móc 3 là nốt nhạc nào?
\end{baitoan}

\begin{baitoan}[\cite{Tuyen_Toan_6}, 45., p. 12, Phân bào]
	Tế báo lớn lên đến 1 kích thước nhất định thì phân chia thành $2$ tế bào con. Mỗi tế bào con tiếp tục lớn lên cho đến khi bằng tế bào mẹ, sau đó phân chia thành $2$ tế bào, quá trình cứ thế tiếp tục. Cho biết: (a) Số tế bào con sau lần phân chia thứ $3$, thứ $5$, thứ $n\in\mathbb{N}^\star$. Viết kết quả dưới dạng lũy thừa cơ số $2$. (b) Sau mấy lần phân chia thì số tế bào con là $128$?
\end{baitoan}
Về phân bào, see, e.g., \href{https://vi.wikipedia.org/wiki/Ph%C3%A2n_b%C3%A0o}{Wikipedia{\tt/}phân bào} \& \href{https://en.wikipedia.org/wiki/Spindle_apparatus}{Wikipedia{\tt/}spindle apparatus}.

%------------------------------------------------------------------------------%

\subsection{Compare Exponentiations -- So Sánh Các Lũy Thừa}

\begin{baitoan}[\cite{Tuyen_Toan_6}, Ví dụ 8, p. 13]
	So sánh $3^7$ \& $2^{11}$.
\end{baitoan}

\begin{baitoan}[\cite{Tuyen_Toan_6}, Ví dụ 9, p. 13]
	So sánh $16^{19}$ \& $8^{25}$.
\end{baitoan}

\begin{baitoan}[\cite{Tuyen_Toan_6}, Ví dụ 10, p. 13]
	So sánh $3^{4040}$ \& $2^{6060}$.
\end{baitoan}

\begin{baitoan}[\cite{Tuyen_Toan_6}, 46., p. 14]
	So sánh: (a) $27^{11}$ \& $81^8$. (b) $625^5$ \& $125^7$.
\end{baitoan}

\begin{baitoan}[\cite{Tuyen_Toan_6}, 47., p. 14]
	So sánh: (a) $5^{36}$ \& $11^{24}$. (b) $3^{2n}$ \& $2^{3n}$, $\forall n\in\mathbb{N}^\star$.
\end{baitoan}

\begin{baitoan}[\cite{Tuyen_Toan_6}, 48., p. 14]
	So sánh $A = 2\cdot3^{54}$ \& $B = 6\cdot5^{32}$.
\end{baitoan}

\begin{baitoan}[\cite{Tuyen_Toan_6}, 49., p. 14]
	Chứng minh: $5^{60n} < 2^{140n} < 3^{100n}$, $\forall n\in\mathbb{N}^\star$.
\end{baitoan}

\begin{baitoan}[\cite{Tuyen_Toan_6}, 50., p. 14]
	Sắp xếp 3 số $3^{539},7^{308},2^{847}$ theo thứ tự tăng dần.
\end{baitoan}

\begin{baitoan}[\cite{Tuyen_Toan_6}, 51., p. 14]
	So sánh: (a) $5^{75}$ \& $7^{60}$. (b) $3^{21}$ \& $2^{31}$.
\end{baitoan}

\begin{baitoan}[\cite{Tuyen_Toan_6}, 52., p. 14]
	So sánh: (a) $5^{23}$ \& $6\cdot5^{22}$. (b) $7\cdot2^{13}$ \& $2^{16}$. (c) $21^{15}$ \& $27^5\cdot49^8$.
\end{baitoan}

\begin{baitoan}[\cite{Tuyen_Toan_6}, 53., p. 14]
	So sánh: (a) $199^{20}$ \& $2003^{15}$. (b) $3^{39}$ \& $11^{21}$.
\end{baitoan}

\begin{baitoan}[\cite{Tuyen_Toan_6}, 54., p. 14]
	So sánh 2 hiệu $A = 72^{45} - 72^{44}$ \& $B = 72^{44} - 72^{43}$.
\end{baitoan}

\begin{baitoan}[\cite{Tuyen_Toan_6}, 55., p. 14]
	Tìm $x\in\mathbb{N}$ biết: (a) $16^x < 128^4$. (b) $5^x\cdot5^{x + 1}\cdot5^{x + 2}\le1\underbrace{00\ldots0}_{18}:2^{18}$.
\end{baitoan}

\begin{baitoan}[\cite{Tuyen_Toan_6}, 56., p. 14]
	Tìm $n\in\mathbb{N}$ biết $2^5\cdot3^n\cdot3^{n + 2}\le32\cdot3^6\cdot3^4$.
\end{baitoan}

\begin{baitoan}[\cite{Tuyen_Toan_6}, 57., p. 14]
	So sánh $A = \sum_{i=0}^9 2^i = 1 + 2 + 2^2 + 2^3 + \cdots + 2^9$ \& $B = 5\cdot2^8$.
\end{baitoan}

\begin{baitoan}[\cite{Tuyen_Toan_6}, 58., p. 14]
	Viết số lớn nhất bằng cách dùng $3$ chữ số $1,2,3$ với điều kiện mỗi chữ số chỉ dùng 1 lần.
\end{baitoan}

%------------------------------------------------------------------------------%

\printbibliography[heading=bibintoc]

\end{document}