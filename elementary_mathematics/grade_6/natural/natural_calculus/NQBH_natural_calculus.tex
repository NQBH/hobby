\documentclass{article}
\usepackage[backend=biber,natbib=true,style=alphabetic,maxbibnames=50]{biblatex}
\addbibresource{/home/nqbh/reference/bib.bib}
\usepackage[utf8]{vietnam}
\usepackage{tocloft}
\renewcommand{\cftsecleader}{\cftdotfill{\cftdotsep}}
\usepackage[colorlinks=true,linkcolor=blue,urlcolor=red,citecolor=magenta]{hyperref}
\usepackage{amsmath,amssymb,amsthm,float,graphicx,mathtools}
\allowdisplaybreaks
\newtheorem{assumption}{Assumption}
\newtheorem{baitoan}{}
\newtheorem{cauhoi}{Câu hỏi}
\newtheorem{conjecture}{Conjecture}
\newtheorem{corollary}{Corollary}
\newtheorem{dangtoan}{Dạng toán}
\newtheorem{definition}{Definition}
\newtheorem{dinhly}{Định lý}
\newtheorem{dinhnghia}{Định nghĩa}
\newtheorem{example}{Example}
\newtheorem{ghichu}{Ghi chú}
\newtheorem{hequa}{Hệ quả}
\newtheorem{hypothesis}{Hypothesis}
\newtheorem{lemma}{Lemma}
\newtheorem{luuy}{Lưu ý}
\newtheorem{nhanxet}{Nhận xét}
\newtheorem{notation}{Notation}
\newtheorem{note}{Note}
\newtheorem{principle}{Principle}
\newtheorem{problem}{Problem}
\newtheorem{proposition}{Proposition}
\newtheorem{question}{Question}
\newtheorem{remark}{Remark}
\newtheorem{theorem}{Theorem}
\newtheorem{vidu}{Ví dụ}
\usepackage[left=1cm,right=1cm,top=5mm,bottom=5mm,footskip=4mm]{geometry}
\def\labelitemii{$\circ$}
\DeclareRobustCommand{\divby}{%
	\mathrel{\vbox{\baselineskip.65ex\lineskiplimit0pt\hbox{.}\hbox{.}\hbox{.}}}%
}
\title{Problem: Calculus on $\mathbb{N}$\\Bài Tập: Các Phép Tính Trên Tập Hợp Các Số Tự Nhiên}
\date{}

\begin{document}
\maketitle
\vspace{-2.5cm}

%------------------------------------------------------------------------------%

\section{$\pm,\cdot,:$ on $\mathbb{N}$}

\begin{baitoan}[\cite{Tuyen_Toan_6}, Ví dụ 3, p. 8]
	1 học sinh khi nhân 1 số với $31$ đã đặt các tích riêng thẳng hàng như trong phép cộng nên tích đã giảm đi $540$ đơn vị so với tích đúng. Tìm tích đúng.
\end{baitoan}

\begin{baitoan}[\cite{Tuyen_Toan_6}, Ví dụ 4, p. 8]
	Cho 2 số không chia hết cho $3$, khi chia cho $3$, khi chia cho $3$ được các số dư khác nhau. Chứng minh tổng của 2 số đó chia hết cho $3$.
\end{baitoan}

\begin{baitoan}[\cite{Tuyen_Toan_6}, 14., p. 9]
	Tính hợp lý: (a) $38 + 41 + 117 + 159 + 62$. (b) $73 + 86 + 978 + 914 + 3022$. (c) $341\cdot67 + 341\cdot16 + 659\cdot83$. (d) $42\cdot53 + 47\cdot156 - 47\cdot114$.
\end{baitoan}

\begin{baitoan}[\cite{Tuyen_Toan_6}, 15., p. 9]
	Tính giá trị của biểu thức: (a) $A = (100 - 1)\cdot(100 - 2)\cdots(100 - n)$ với $n\in\mathbb{N}^\star$ \& tích trên có đúng $100$ thừa số. (b) $B = 13a + 19b + 4a - 2b$ với $a + b = 100$.
\end{baitoan}

\begin{baitoan}[\cite{Tuyen_Toan_6}, 16., p. 9]
	Không tính giá trị cụ thể, so sánh giá trị 2 biểu thức: (a) $A = 199\cdot201$ \& $B = 200\cdot200$. (b) $C = 35'\cdot53 - 18$ \& $D = 35 + 53\cdot34$.
\end{baitoan}

\begin{baitoan}[\cite{Tuyen_Toan_6}, 17., p. 9]
	Tính hợp lý: (a) $(44\cdot52\cdot60):(11\cdot13\cdot15)$. (b) $123\cdot456456 - 456\cdot123123$. (c) $(98\cdot7676 - 9898\cdot76):(2021\cdot2022\cdot2023\cdots2030)$.
\end{baitoan}

\begin{baitoan}[\cite{Tuyen_Toan_6}, 18., p. 9]
	Tìm giá trị nhỏ nhất của biểu thức: $A = 2021 - 1021:(999 - x)$.
\end{baitoan}

\begin{baitoan}[\cite{Tuyen_Toan_6}, 20., p. 9]
	Tìm số hạng thứ 5, thứ $n$ của dãy số: (a) $2,3,7,25,\ldots$. (b) $8,30,72,140,\ldots$.
\end{baitoan}

\begin{baitoan}[\cite{Tuyen_Toan_6}, 21., p. 9]
	Tìm $x$: (a) $(x + 74) - 318 = 200$. (b) $3636:(12x - 91) = 36$. (c) $(x:23 + 45)\cdot67 = 8911$.
\end{baitoan}

\begin{baitoan}[\cite{Tuyen_Toan_6}, 22., p. 9]
	1 nong tằm là $5$ nong kén. 1 nong kén là $9$ nén tơ. Hỏi muốn được $540$ nén tơ thì phải chăn bao nhiêu nong tằm?
\end{baitoan}

\begin{baitoan}[\cite{Tuyen_Toan_6}, 23., p. 9]
	2 số tự nhiên $a,b$ chia hết cho $m$ có cùng số dư, $a\ge b$. Chứng tỏ $a - b$ chia hết cho $m$.
\end{baitoan}

\begin{baitoan}[\cite{Tuyen_Toan_6}, 24., p. 9]
	Trong 1 phép chia có số bị chia là $155$, số dư là $12$. Tìm số chia \& thương.
\end{baitoan}

\begin{baitoan}[\cite{Tuyen_Toan_6}, 25., p. 9]
	Viết tập hợp $A$ các số tự nhiên $x$ biết lấy $x$ chia cho $12$ ta được thương bằng số dư.
\end{baitoan}

\begin{baitoan}[\cite{Tuyen_Toan_6}, 26., p. 10]
	Chia $129$ cho 1 số ta được số dư là $10$. Chia $61$ cho số đó ta cũng được số dư là $10$. Tìm số chia.
\end{baitoan}

\begin{baitoan}[\cite{Tuyen_Toan_6}, 27., p. 10]
	Cho 3 chữ số $a,b,c$ khác nhau \& khác $0$. Với cùng cả 3 chữ số này có thể lập được bao nhiêu số có 3 chữ số?
\end{baitoan}

\begin{baitoan}[\cite{Tuyen_Toan_6}, 28., p. 10]
	Cho 4 chữ số khác nhau \& khác $0$. (a) Với cùng cả 4 chữ số này có thể lập được bao nhiêu số có 4 chữ số? (b) Có thể lập được bao nhiêu số có 2 chữ số khác nhau trong 4 chữ số đã cho?
\end{baitoan}

\begin{baitoan}[\cite{Tuyen_Toan_6}, 29., p. 10]
	Cho 4 chữ số khác nhau trong đó có 1 chữ số $0$. Với cùng cả 4 chữ số này có thể lập được bao nhiêu số có 4 chữ số?
\end{baitoan}

\begin{baitoan}[\cite{Tuyen_Toan_6}, 30., p. 10]
	Anh Bách đi mua bánh kẹo tại 1 siêu thị, thanh toán bằng 1 phiếu mua hàng trị giá $100000$ đồng. Siêu thị không trả lại số tiền thừa. Giúp anh Bách chọn mua vừa hết số tiền ghi trong phiếu. Bảng giá 1 số mặt hàng có bán:
	\begin{table}[H]
		\centering
		\begin{tabular}{|l|c|c|c|}
			\hline
			STT & Tên hàng & Đơn vị & Giá bán \\
			\hline
			1 & Bánh đậu xanh & Hộp & 31 500 đồng \\
			\hline
			2 & Bánh bông lan & Gói & 23 500 đồng \\
			\hline
			3 & Bánh gạo & Gói & 19 000 đồng \\
			\hline
			4 & Bánh gạo chiên & Gói & 17 800 đồng \\
			\hline
			5 & Bánh quy & Gói & 13 500 đồng \\
			\hline
			6 & Bánh xốp & Gói & 5300 đồng \\
			\hline
			7 & Kẹo hương dâu & Gói & 2 500 đồng \\
			\hline
		\end{tabular}
	\end{table}
\end{baitoan}

%------------------------------------------------------------------------------%

\section{Exponentiation on $\mathbb{N}$ -- Lũy Thừa với Số Mũ Tự Nhiên}

%------------------------------------------------------------------------------%

\printbibliography[heading=bibintoc]

\end{document}