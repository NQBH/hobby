\documentclass{article}
\usepackage[backend=biber,natbib=true,style=alphabetic,maxbibnames=50]{biblatex}
\addbibresource{/home/nqbh/reference/bib.bib}
\usepackage[utf8]{vietnam}
\usepackage{tocloft}
\renewcommand{\cftsecleader}{\cftdotfill{\cftdotsep}}
\usepackage[colorlinks=true,linkcolor=blue,urlcolor=red,citecolor=magenta]{hyperref}
\usepackage{amsmath,amssymb,amsthm,float,graphicx,mathtools}
\allowdisplaybreaks
\newtheorem{assumption}{Assumption}
\newtheorem{baitoan}{Bài toán}
\newtheorem{cauhoi}{Câu hỏi}
\newtheorem{conjecture}{Conjecture}
\newtheorem{corollary}{Corollary}
\newtheorem{dangtoan}{Dạng toán}
\newtheorem{definition}{Definition}
\newtheorem{dinhly}{Định lý}
\newtheorem{dinhnghia}{Định nghĩa}
\newtheorem{example}{Example}
\newtheorem{ghichu}{Ghi chú}
\newtheorem{hequa}{Hệ quả}
\newtheorem{hypothesis}{Hypothesis}
\newtheorem{lemma}{Lemma}
\newtheorem{luuy}{Lưu ý}
\newtheorem{nhanxet}{Nhận xét}
\newtheorem{notation}{Notation}
\newtheorem{note}{Note}
\newtheorem{principle}{Principle}
\newtheorem{problem}{Problem}
\newtheorem{proposition}{Proposition}
\newtheorem{question}{Question}
\newtheorem{remark}{Remark}
\newtheorem{theorem}{Theorem}
\newtheorem{vidu}{Ví dụ}
\usepackage[left=1cm,right=1cm,top=5mm,bottom=5mm,footskip=4mm]{geometry}
\def\labelitemii{$\circ$}
\DeclareRobustCommand{\divby}{%
	\mathrel{\vbox{\baselineskip.65ex\lineskiplimit0pt\hbox{.}\hbox{.}\hbox{.}}}%
}

\title{Problem: Set -- Bài Tập: Tập Hợp}
\author{Nguyễn Quản Bá Hồng\footnote{Independent Researcher, Ben Tre City, Vietnam\\e-mail: \texttt{nguyenquanbahong@gmail.com}; website: \url{https://nqbh.github.io}.}}
\date{\today}

\begin{document}
\maketitle

%------------------------------------------------------------------------------%

\begin{baitoan}[\cite{SGK_Toan_6_CTST_tap_1}, ]
	
\end{baitoan}

\begin{baitoan}
	Viết tập hợp theo 2 cách:(a) Tập hợp các số tự nhiên. (b) Tập hợp các số nguyên dương. (c) Tập hợp các số tự nhiên nhỏ hơn 1 số tự nhiên $a$ cho trước. (d) Tập hợp các số tự nhiên lớn hơn 1 số tự nhiên $a$ cho trước. (e) Tập hợp các số tự nhiên nhỏ hơn hoặc bằng 1 số tự nhiên $a$ cho trước. (f) Tập hợp các số tự nhiên lớn hơn hoặc bằng 1 số tự nhiên $a$ cho trước.
\end{baitoan}

\begin{baitoan}
	Với $a,b$ là 2 số tự nhiên cho trước, viết tập hợp theo 2 cách: (a) Tập hợp các số tự nhiên lớn hơn $a$ \& nhỏ hơn $b$. (b) Tập hợp các số tự nhiên lớn hơn hoặc bằng $a$ \& nhỏ hơn $b$. (c) Tập hợp các số tự nhiên lớn hơn $a$ \& nhỏ hơn hoặc bằng $b$. (d) Tập hợp các số tự nhiên lớn hơn hoặc bằng $a$ \& nhỏ hơn hoặc bằng $b$.
\end{baitoan}

\begin{baitoan}
	Với $b,r$ là 2 số nguyên dương cho trước (i.e., số tự nhiên khác $0$), viết tập hợp theo 2 cách: (a) Tập hợp các số tự nhiên chia hết cho $b$. (b) Tập hợp các số tự nhiên chia cho $b$ dư $r$.
\end{baitoan}
Integer version.

%------------------------------------------------------------------------------%

\printbibliography[heading=bibintoc]
	
\end{document}