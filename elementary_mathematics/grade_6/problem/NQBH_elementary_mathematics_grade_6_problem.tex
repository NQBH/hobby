\documentclass{article}
\usepackage[backend=biber,natbib=true,style=authoryear]{biblatex}
\addbibresource{/home/hong/1_NQBH/reference/bib.bib}
\usepackage[utf8]{vietnam}
\usepackage{tocloft}
\renewcommand{\cftsecleader}{\cftdotfill{\cftdotsep}}
\usepackage[colorlinks=true,linkcolor=blue,urlcolor=red,citecolor=magenta]{hyperref}
\usepackage{amsmath,amssymb,amsthm,mathtools,float,graphicx,algpseudocode,algorithm,tcolorbox}
\usepackage[inline]{enumitem}
\allowdisplaybreaks
\numberwithin{equation}{section}
\newtheorem{assumption}{Assumption}[section]
\newtheorem{conjecture}{Conjecture}[section]
\newtheorem{corollary}{Corollary}[section]
\newtheorem{hequa}{Hệ quả}[section]
\newtheorem{definition}{Definition}[section]
\newtheorem{dinhnghia}{Định nghĩa}[section]
\newtheorem{example}{Example}[section]
\newtheorem{vidu}{Ví dụ}[section]
\newtheorem{lemma}{Lemma}[section]
\newtheorem{notation}{Notation}[section]
\newtheorem{principle}{Principle}[section]
\newtheorem{problem}{Problem}[section]
\newtheorem{baitoan}{Bài toán}[section]
\newtheorem{proposition}{Proposition}[section]
\newtheorem{question}{Question}[section]
\newtheorem{cauhoi}{Câu hỏi}[section]
\newtheorem{remark}{Remark}[section]
\newtheorem{luuy}{Lưu ý}[section]
\newtheorem{theorem}{Theorem}[section]
\newtheorem{dinhly}{Định lý}[section]
\usepackage[left=0.5in,right=0.5in,top=1.5cm,bottom=1.5cm]{geometry}
\usepackage{fancyhdr}
\pagestyle{fancy}
\fancyhf{}
\lhead{\small \textsc{Sect.} ~\thesection}
\rhead{\small \nouppercase{\leftmark}}
\renewcommand{\sectionmark}[1]{\markboth{#1}{}}
\cfoot{\thepage}
\def\labelitemii{$\circ$}

\title{Some Topics in Elementary Mathematics\texttt{/}Grade 6}
\author{Nguyễn Quản Bá Hồng\footnote{Independent Researcher, Ben Tre City, Vietnam\\e-mail: \texttt{nguyenquanbahong@gmail.com}; website: \url{https://nqbh.github.io}.}}
\date{\today}

\title{Problems in Elementary Mathematics\texttt{/}Grade 6}
\author{Nguyễn Quản Bá Hồng\footnote{Independent Researcher, Ben Tre City, Vietnam\\e-mail: \texttt{nguyenquanbahong@gmail.com}; website: \url{https://nqbh.github.io}.}}
\date{\today}

\begin{document}
\maketitle
\begin{abstract}
	1 bộ sưu tập các bài toán chọn lọc từ cơ bản đến nâng cao cho Toán sơ cấp lớp 6. Tài liệu này là phần bài tập bổ sung cho tài liệu chính \href{https://github.com/NQBH/hobby/blob/master/elementary_mathematics/grade_6/NQBH_elementary_mathematics_grade_6.pdf}{GitHub\texttt{/}NQBH\texttt{/}hobby\texttt{/}elementary mathematics\texttt{/}grade 6\texttt{/}lecture}\footnote{Explicitly, \url{https://github.com/NQBH/hobby/blob/master/elementary_mathematics/grade_6/NQBH_elementary_mathematics_grade_6.pdf}.} của tác giả viết cho Toán lớp 6. Phiên bản mới nhất của tài liệu này được lưu trữ ở link sau: \href{https://github.com/NQBH/hobby/blob/master/elementary_mathematics/grade_6/problem/NQBH_elementary_mathematics_grade_6_problem.pdf}{GitHub\texttt{/}NQBH\texttt{/}hobby\texttt{/}elementary mathematics\texttt{/}grade 6\texttt{/}problem}\footnote{Explicitly, \url{https://github.com/NQBH/hobby/blob/master/elementary_mathematics/grade_6/problem/NQBH_elementary_mathematics_grade_6_problem.pdf}.}.
\end{abstract}

\tableofcontents
\newpage

%------------------------------------------------------------------------------%

\section{Tập Hợp Các Số Tự Nhiên}

\subsection{Tập hợp}
\textsc{Kiến thức cần nhớ.}
\begin{tcolorbox}	
	Tên tập hợp được viết bằng chữ cái in hoa. Cho $A = \{a;b;c\}$. Khi đó, $a,b,c$ là các phần tử của tập hợp $A$. $a\in A$ đọc là \textit{$a$ thuộc tập hợp $A$} hay \textit{$a$ là phần tử của tập hợp $A$}. $d\notin A$, đọc là \textit{$d$ không thuộc tập hợp $A$} hay \textit{$d$ không là phần tử của tập hợp $A$}.
	
	Cách viết tập hợp có 2 cách:
	\begin{itemize}
		\item \textit{Cách 1.} Liệt kê các phần tử của tập hợp: các phần tử của 1 tập hợp được viết trong 2 dấu ngoặc nhọn $\{\ \}$, cách nhau bởi dấu ``;'' (nếu có phần tử là số) hoặc dấu ``,''. Mỗi phần tử được liệt kê 1 lần, thứ tự liệt kê tùy ý.
		\item \textit{Cách 2.} Chỉ ra tính chất đặc trưng của các phần tử của tập hợp.
	\end{itemize}
	\textbf{Các ký hiệu.} $\mathbb{N}$: tập hợp các số tự nhiên. $\mathbb{N}^\star$: tập hợp các số tự nhiên khác $0$. $|$: sao cho, thỏa mãn. $\ge$: lớn hơn hoặc bằng ($>$ hoặc $=$). $\le$: nhỏ hơn hoặc bằng ($<$ hoặc $=$). $\emptyset$: tập hợp rỗng, i.e., tập hợp không có phần tử nào.
\end{tcolorbox}
Các bài tập SGK \cite[\textbf{1}--\textbf{4}, pp. 7--8]{SGK_Toan_6_Canh_Dieu_tap_1} \& SBT \cite[Ví dụ 1, 2, p. 5; \textbf{1}--\textbf{8}, pp. 6--7]{SBT_Toan_6_Canh_Dieu_tap_1}.

\begin{baitoan}[\cite{Trong_Toan_6_2021}, \textbf{7.}, p. 6]
	Tập hợp $M$ gồm các chữ cái của từ ``THANG LONG''. Hãy viết tập $M$ bằng cách liệt kê các phần tử.
\end{baitoan}

\begin{baitoan}[\cite{Trong_Toan_6_2021}, \textbf{8.}, p. 6]
	Tập hợp $B$ gồm các chữ cái của từ ``NGOẠI NGỮ''. Hãy viết tập $B$ bằng cách liệt kê các phần tử.
\end{baitoan}

\begin{baitoan}[\cite{Trong_Toan_6_2021}, \textbf{9.}, p. 6]
	Tập hợp $A$ gồm các số tự nhiên nhỏ hơn $3$. Viết tập hợp $A$ bằng cách liệt kê các phần tử.
\end{baitoan}

\begin{baitoan}[\cite{Trong_Toan_6_2021}, \textbf{10.}, p. 6]
	Tập hợp $E$ gồm các số chẵn nhỏ hơn $5$. Viết tập hợp $E$ bằng cách liệt kê các phần tử.
\end{baitoan}

\begin{baitoan}[\cite{Trong_Toan_6_2021}, \textbf{11.}, p. 6]
	Tập hợp $H$ gồm các số lẻ nhỏ hơn $8$. Viết tập hợp $H$ bằng cách liệt kê các phần tử.
\end{baitoan}

\begin{baitoan}[\cite{Trong_Toan_6_2021}, \textbf{12.}, p. 7]
	Tập hợp $C$ gồm các số tự nhiên nhỏ hơn hoặc bằng $4$. Viết tập hợp $C$ bằng cách liệt kê các phần tử.
\end{baitoan}

\begin{baitoan}[\cite{Trong_Toan_6_2021}, \textbf{13.}, p. 7]
	Tập hợp $E$ gồm các số tự nhiên không vượt quá $11$. Viết tập hợp $E$ bằng cách liệt kê các phần tử.
\end{baitoan}

\begin{baitoan}[\cite{Trong_Toan_6_2021}, \textbf{14.}, p. 7]
	Tập hợp $C$ gồm các số tự nhiên lớn hơn $1$ \& nhỏ hơn $5$. Viết tập hợp $C$ bằng cách liệt kê các phần tử.
\end{baitoan}

\begin{baitoan}[\cite{Trong_Toan_6_2021}, \textbf{15.}, p. 7]
	Tập hợp $D$ gồm các số tự nhiên lớn hơn hoặc bằng $6$ \& nhỏ hơn $12$. Viết tập hợp $D$ bằng cách liệt kê các phần tử.
\end{baitoan}

\begin{baitoan}[\cite{Trong_Toan_6_2021}, \textbf{16.}, p. 7]
	Tập hợp $E$ gồm các số tự nhiên lớn hơn $4$ \& nhỏ hơn $9$. Viết tập hợp $E$ bằng cách liệt kê các phần tử.
\end{baitoan}

\begin{baitoan}[\cite{Trong_Toan_6_2021}, \textbf{17.}, p. 7]
	Tập hợp $A$ gồm các số tự nhiên nhỏ hơn $3$. Viết tập hợp $A$ bằng cách chỉ ra các tính chất đặc trưng cho các phần tử của tập hợp.
\end{baitoan}

\begin{baitoan}[\cite{Trong_Toan_6_2021}, \textbf{18.}, p. 7]
	Tập hợp $B$ gồm các số tự nhiên nhỏ hơn $8$. Viết tập hợp $B$ bằng cách chỉ ra các tính chất đặc trưng cho các phần tử của tập hợp.
\end{baitoan}

\begin{baitoan}[\cite{Trong_Toan_6_2021}, \textbf{19.}, p. 7]
	Tập hợp $C$ gồm các số tự nhiên lớn hơn $11$. Viết tập hợp $C$ bằng cách chỉ ra các tính chất đặc trưng cho các phần tử của tập hợp.
\end{baitoan}

\begin{baitoan}[\cite{Trong_Toan_6_2021}, \textbf{20.}, p. 7]
	Tập hợp $A$ gồm các số tự nhiên lớn hơn hoặc bằng $8$. Viết tập hợp $A$ bằng cách chỉ ra các tính chất đặc trưng cho các phần tử của tập hợp.
\end{baitoan}

\begin{baitoan}[\cite{Trong_Toan_6_2021}, \textbf{21.}, p. 7]
	Tập hợp $B$ gồm các số tự nhiên lớn hơn $7$ \& nhỏ hơn $17$. Viết tập hợp $B$ bằng cách chỉ ra các tính chất đặc trưng cho các phần tử của tập hợp.
\end{baitoan}

\begin{baitoan}[\cite{Trong_Toan_6_2021}, \textbf{22.}, p. 7]
	Tập hợp $C$ gồm các số tự nhiên lớn hơn hoặc bằng $7$ \& nhỏ hơn $14$. Viết tập hợp $C$ bằng cách chỉ ra các tính chất đặc trưng cho các phần tử của tập hợp.
\end{baitoan}

\begin{baitoan}[\cite{Trong_Toan_6_2021}, \textbf{23.}, p. 7]
	Tập hợp $A$ gồm các số tự nhiên khác $0$ \& nhỏ hơn hoặc bằng $5$. Viết tập hợp $A$ bằng cách chỉ ra các tính chất đặc trưng cho các phần tử của tập hợp.
\end{baitoan}

\begin{baitoan}[\cite{Trong_Toan_6_2021}, \textbf{24.}, p. 7]
	Cho $A$ là tập hợp các số tự nhiên nhỏ hơn $5$. Viết tập hợp $A$ bằng 2 cách:
	\begin{enumerate*}
		\item Liệt kê các phần tử.
		\item Chỉ ra tính chất đặc trưng cho các phần tử của tập hợp.
	\end{enumerate*}
\end{baitoan}

\begin{baitoan}[\cite{Trong_Toan_6_2021}, \textbf{25.}, p. 7]
	Cho $A$ là tập hợp các số tự nhiên lớn hơn $4$ \& nhỏ hơn $8$. Viết tập hợp $A$ bằng 2 cách:
	\begin{enumerate*}
		\item Liệt kê các phần tử.
		\item Chỉ ra tính chất đặc trưng cho các phần tử của tập hợp.
	\end{enumerate*}
\end{baitoan}

\begin{baitoan}[\cite{Trong_Toan_6_2021}, \textbf{26.}, p. 7]
	Tìm tập hợp $B$ gồm các số tự nhiên lớn hơn hoặc bằng $5$ \& nhỏ hơn hoặc bằng $6$ rồi viết tập hợp $B$ bằng 2 cách: liệt kê các phần tử \& nêu tính chất đặc trưng của các phần tử.
\end{baitoan}

\begin{baitoan}[\cite{Trong_Toan_6_2021}, \textbf{27.}, p. 7]
	Viết tập hợp $K$ những người sống trên mặt trăng.
\end{baitoan}

\begin{baitoan}[\cite{Trong_Toan_6_2021}, \textbf{28.}, p. 8]
	$A$ là tập hợp các số tự nhiên không quá $4$.
	\begin{enumerate*}
		\item[(a)] Viết tập hợp $A$ bằng cách liệt kê \& cách chỉ ra tính chất đặc trưng của các phần tử.
		\item[(b)] Điền vào chỗ trống dùng ký hiệu $\in,\notin$: $4\square A$, $3\square A$, $0\square A$, $6\square A$, $1\square A$, $\frac{1}{2}\square A$.
	\end{enumerate*}
\end{baitoan}

\begin{baitoan}[\cite{Trong_Toan_6_2021}, \textbf{29.}, p. 8]
	Viết tập hợp $C$ các số tự nhiên lớn hơn $5$ \& nhỏ hơn $6$ bằng 2 cách.
\end{baitoan}

\begin{baitoan}[\cite{Trong_Toan_6_2021}, \textbf{30.}, p. 8]
	Cho $A$ là tập hợp các số tự nhiên nhỏ hơn $7$ \& $B$ là tập hợp các số tự nhiên chẵn nhỏ hơn $8$.
	\begin{enumerate*}
		\item[(a)] Viết các tập $A$ \& $B$ bằng cách liệt kê phần tử.
		\item[(b)] Điền vào ô trống dùng các ký hiệu: $\subset,\in,\notin$: $B\square A$, $5\square B$, $6\square A$, $7\square B$, $6\square B$, $4\square A$, $4\square B$, $5\square A$, $0\square A$, $0\square B$.
	\end{enumerate*}
\end{baitoan}

\begin{baitoan}[\cite{Trong_Toan_6_2021}, \textbf{31.}, p. 8]
	 Cho $A$ là tập hợp các số tự nhiên nhỏ hơn $4$.
	 \begin{enumerate*}
	 	\item[(a)] Viết tập $A$ bằng 2 cách.
	 	\item[(b)] Xét tính đúng sai của các cách viết sau: $0\in A$, $1\notin A$, $4\in A$, $3\in A$, $5\notin A$, $2\in A$.
	 	\item[(c)] Điền vào ô trống dùng ký hiệu $\in,\notin$: $3\square A$, $5\square A$, $4\square A$, $0\square A$, $1\square A$, $2\square A$.
	 \end{enumerate*}
\end{baitoan}
\noindent\textsc{Kiến thức cần nhớ.}
\begin{tcolorbox}
	\begin{enumerate}
		\item \textbf{Số phần tử của tập hợp.} 1 tập hợp có thể có 1 phần tử, có nhiều phần tử, có vô số phần tử hoặc không có phần tử nào.
		\item \textbf{Tập hợp con.} Nếu mọi phần tử của tập hợp $A$ đều thuộc tập hợp $B$ thì tập hợp $A$ là \textit{tập hợp con} của tập hợp $B$. Ký hiệu: $A\subset B$ hay $B\supset A$.
		\item \textbf{Tập hợp bằng nhau.} Nếu các phần tử của tập hợp $A$ \& tập hợp $B$ giống nhau thì tập hợp $A$ bằng tập hợp $B$.
		
		\begin{luuy}
			Tập hợp rỗng $\emptyset$ là tập hợp con của mọi tập hợp. Nếu $A\subset B$ \& $B\supset A$ thì $A = B$. Mỗi tập hợp đều là tập hợp con của chính nó, i.e., $A\subset A$ với mọi tập hợp $A$.
		\end{luuy}
	\end{enumerate}
\end{tcolorbox}

\begin{baitoan}[\cite{Trong_Toan_6_2021}, \textbf{32.}, p. 8]
	Cho tập hợp $A = \{1;3\}$.
	\begin{enumerate*}
		\item[(a)] Viết các tập hợp con của tập hợp $A$ sao cho mỗi tập hợp con đó có đúng 1 phần tử.
		\item[(b)] Viết các tập hợp con của tập hợp $A$ sao cho mỗi tập hợp con đó có đúng 2 phần tử.
		\item[(c)] Viết tất cả các tập hợp con của tập hợp $A$.
	\end{enumerate*}
\end{baitoan}

\begin{baitoan}[\cite{Trong_Toan_6_2021}, \textbf{33.}, p. 9]
	Cho tập hợp $A = \{3;4;5\}$.
	\begin{enumerate*}
		\item[(a)] Viết các tập hợp con của tập hợp $A$ sao cho mỗi tập hợp con đó có đúng 1 phần tử.
		\item[(b)] Viết các tập hợp con của tập hợp $A$ sao cho mỗi tập hợp con đó có đúng 2 phần tử.
		\item[(c)] Viết tất cả các tập hợp con của tập hợp $A$.
	\end{enumerate*}
\end{baitoan}
	
\begin{baitoan}[\cite{Trong_Toan_6_2021}, \textbf{34.}, p. 9]
	Cho tập hợp $B = \{a;b;c\}$. Viết tất cả các tập hợp con của tập hợp $B$.		
\end{baitoan}

\begin{baitoan}[\cite{Trong_Toan_6_2021}, \textbf{35.}, p. 9]
	Cho $A$ là tập hợp các số tự nhiên nhỏ hơn $8$ \& $B$ là tập hợp các số tự nhiên nhỏ hơn $5$.
	\begin{enumerate*}
		\item[(a)] Hãy viết các tập hợp $A$ \& $B$ bằng cách liệt kê các phần tử.
		\item[(b)] Dùng ký hiệu $\subset$ để thể hiện quan hệ giữa 2 tập hợp $A$ \& $B$.
	\end{enumerate*}
\end{baitoan}

\begin{baitoan}[\cite{Trong_Toan_6_2021}, \textbf{36.}, p. 9]
	Cho 2 tập hợp $A = \{x\in\mathbb{N}|x < 7\}$; $B = \{x\in\mathbb{N};x < 6\}$.
	\begin{enumerate*}
		\item[(a)] Viết các tập hợp $A$ \& $B$ bằng cách liệt kê các phần tử \& cho biết số phần tử của mỗi tập hợp.
		\item[(b)] Dùng ký hiệu $\subset$ để thể hiện quan hệ giữa 2 tập hợp $A$ \& $B$.
	\end{enumerate*} 
\end{baitoan}

\begin{baitoan}[\cite{Trong_Toan_6_2021}, \textbf{37.}, p. 9]
	Cho 2 tập hợp $C = \{x\in\mathbb{N}^\star|x < 6\}$; $D = \{x\in\mathbb{N}^\star|x < 9\}$.
	\begin{enumerate*}
		\item[(a)] Viết các tập hợp $C$ \& $D$ bằng cách liệt kê các phần tử \& cho biết số phần tử của mỗi tập hợp.
		\item[(b)] Dùng ký hiệu $\subset$ để thể hiện quan hệ giữa 2 tập hợp $C$ \& $D$.
	\end{enumerate*} 
\end{baitoan}

\begin{baitoan}[\cite{Trong_Toan_6_2021}, \textbf{38.}, p. 9]
	Cho $A$ là tập hợp các số tự nhiên nhỏ hơn $8$, $B$ là tập hợp các số tự nhiên lẻ nhỏ hơn $7$.
	\begin{enumerate*}
		\item[(a)] Viết tập hợp $A$ \& $B$ bằng cách liệt kê các phần tử.
		\item[(b)] Viết các tập con của $B$.
		\item[(c)] Dùng các ký hiệu đã học điền vào ô trống. $1\square A$, $2\square B$, $0\square A$, $\{1;3\}\square B$, $B\square A$, $\{0;1\}\in A$.
	\end{enumerate*}
\end{baitoan}

\begin{baitoan}[\cite{Trong_Toan_6_2021}, \textbf{39.}, p. 9]
	$A$ là tập hợp các số tự nhiên khác $0$ \& nhỏ hơn $7$.
	\begin{enumerate*}
		\item[(a)] Viết tập $A$ bằng 2 cách: Liệt kê các phần tử. Nêu tính chất đặc trưng của các phần tử.
		\item[(b)] Viết các tập con của $A$ sao cho mỗi tập con đó có đúng 2 phần tử.
	\end{enumerate*}
\end{baitoan}

\begin{baitoan}[\cite{Trong_Toan_6_2021}, \textbf{40.}, p. 9]
	$A$ là tập hợp các số tự nhiên lớn hơn $5$ \& nhỏ hơn $9$.
	\begin{enumerate*}
		\item[(a)] Viết tập $A$ bằng 2 cách: Liệt kê các phần tử. Nêu tính chất đặc trưng của các phần tử.
		\item[(b)] Tìm các tập con của $A$.
		\item[(c)] Điền vào ô trống: $1\square A$, $5\square A$, $7\square A$, $\{6,7\}\square A$, $\{0,1,2\}\square A$.
	\end{enumerate*}
\end{baitoan}

\begin{baitoan}[\cite{Trong_Toan_6_2021}, \textbf{41.}, p. 10]
	Cho $A = \{1;2;3;4;5;6\}$, $B = \{x\in\mathbb{N}^\star|x\le 5\}$.
	\begin{enumerate*}
		\item[(a)] Viết tập hợp $A$ bằng cách nêu các tính chất chung của các phần tử \& viết tập $B$ bằng cách liệt kê các phần tử.
		\item[(b)] Dùng ký hiệu để biểu thị sự quan hệ giữa $A$ \& $B$.
	\end{enumerate*}
\end{baitoan}

\begin{baitoan}[\cite{Trong_Toan_6_2021}, \textbf{42.}, p. 10]
	Cho $A = \{x\in\mathbb{N}|30 < x < 50,\ x\ \vdots\ 5\}$, $B = \{x\in\mathbb{N}|30 < x < 50,\ x\ \vdots\ 2\}$.
	\begin{enumerate*}
		\item[(a)] Viết các tập hợp $A,B$ bằng cách liệt kê các phần tử.
		\item[(c)] Tìm các tập con của $A$.
	\end{enumerate*}
\end{baitoan}

\begin{baitoan}[\cite{Trong_Toan_6_2021}, \textbf{43.}, p. 10]
	Cho $A = \{x\in\mathbb{N}|x\le 4\}$, $B = \{x\in\mathbb{N}|x < 7\}$. Liệt kê các phần tử của tập hợp $A$ \& $B$.
\end{baitoan}

\begin{baitoan}[\cite{Trong_Toan_6_2021}, \textbf{45.}, p. 10]
	Cho $A = \{x\in\mathbb{N}|20\le x < 40,\ x\ \vdots\ 3\}$, $B = \{x\in\mathbb{N}|30\le x\le 40,\ x\ \vdots\ 5\}$, $C = \{x\in\mathbb{N}|30\le x\le 40,\ x\ \vdots\ 4\}$. Viết các tập hợp $A,B,C$ bằng cách liệt kê.
\end{baitoan}
\noindent\textsc{Kiến thức cần nhớ.}
\begin{tcolorbox}
	Công thức tính số phần tử của tập hợp là các dãy số đặc biệt:
	\begin{align*}
		\mbox{số phần tử} = \frac{\mbox{số lớn nhất} - \mbox{số bé nhất}}{\mbox{khoảng cách giữa 2 số liên tiếp}} + 1.
	\end{align*}
\end{tcolorbox}

\begin{baitoan}[\cite{Trong_Toan_6_2021}, \textbf{46.}, p. 10]
	Cho tập hợp $A = \{1;3;5;\ldots;39\}$. Tính số phần tử của tập hợp $A$.
\end{baitoan}

\begin{baitoan}[\cite{Trong_Toan_6_2021}, \textbf{47.}, p. 10]
	Cho $E = \{5;10;15;20;\ldots;195\}$. Tính số phần tử của tập hợp $E$.
\end{baitoan}

\begin{baitoan}[\cite{Trong_Toan_6_2021}, \textbf{48.}, p. 10]
	Cho $E = \{3;5;7;9;\ldots;113;115\}$. Tính số phần tử của tập hợp $F$.
\end{baitoan}

\begin{baitoan}[\cite{Trong_Toan_6_2021}, \textbf{49.}, p. 10]
	Để đánh số trang của cuốn sách dày $98$ trang người ta dùng tất cả bao nhiêu chữ số?
\end{baitoan}

\begin{baitoan}[\cite{Trong_Toan_6_2021}, \textbf{50.}, p. 10]
	Để đánh số trang của cuốn sách dày $150$ trang ta cần dùng bao nhiêu chữ số?
\end{baitoan}

\begin{baitoan}[\cite{Trong_Toan_6_2021}, \textbf{51.}, p. 10]
	Người ta dùng $1002$ chữ số để đánh số trang 1 cuốn sách từ 1 đến hết. Hỏi cuốn sách đó dày nhiêu trang?
\end{baitoan}

\begin{baitoan}[\cite{Trong_Toan_6_2021}, \textbf{52.}, p. 10]
	Để đánh số trang 1 quyển sách người ta dùng hết $831$ chữ số. Hỏi quyển sách đó có bao nhiêu trang?
\end{baitoan}

\subsection{Tập hợp các số tự nhiên. Cộng, trừ, nhân, chia số tự nhiên}
\textsc{Kiến thức cần nhớ.}
\begin{tcolorbox}
	Cách ghi số tự nhiên trong hệ thập phân:
	\begin{enumerate*}
		\item[(a)] Trong hệ thập phân, mỗi số tự nhiên được viết dưới dạng 1 dãy những số lấy trong 10 chữ số $0,1,2,3,4,5,6,7,8$, \& $9$; vị trí của các chữ số trong dãy gọi là hàng.
		\item[(b)] Cứ 10 đơn vị ở 1 hàng thì bằng 1 đơn vị của hàng liền trước nó. E.g., $10$ chục thì bằng 1 trăm; 10 trăm thì bằng 1 nghìn; $\ldots$
	\end{enumerate*}
	Trong tập hợp số tự nhiên, số liền sau hơn số liền trước 1 đơn vị.
\end{tcolorbox}
Các bài tập SGK \cite[\textbf{1}--\textbf{4}, pp. 7--8]{SGK_Toan_6_Canh_Dieu_tap_1} \& SBT \cite[Ví dụ 1--3, pp. 7--8; \textbf{9}--\textbf{14}, pp. 8--9]{SBT_Toan_6_Canh_Dieu_tap_1}.

\begin{baitoan}[\cite{Trong_Toan_6_2021}, \textbf{1.}, p. 11]
	Trong các khẳng định sau, khẳng định nào là đúng, khẳng định nào là sai?
	\begin{enumerate*}
		\item[(a)] $1999 > 2003$;
		\item[(b)] $100000$ là số tự nhiên nhỏ lớn nhất;
		\item[(c)] $5\le 5$;
		\item[(d)] Số $1$ là số tự nhiên nhỏ nhất.
	\end{enumerate*}
\end{baitoan}

\begin{baitoan}[\cite{Trong_Toan_6_2021}, \textbf{2.}, p. 11]
	Thay mỗi chữ cái dưới đây bằng 1 số tự nhiên phù hợp trong những trường hợp sau:
	\begin{enumerate*}
		\item[(a)] $17,a,b$ là 3 số lẻ liên tiếp tăng dần.
		\item[(b)] $m,101,n,p$ là 4 số tự nhiên liên tiếp giảm dần.
	\end{enumerate*}
\end{baitoan}

\begin{baitoan}[\cite{Trong_Toan_6_2021}, \textbf{3.}, p. 11]
	\begin{enumerate*}
		\item[(a)] Viết số tự nhiên nhỏ nhất có 4 chữ số;
		\item[(b)] Viết số tự nhiên nhỏ nhất có 4 chữ số khác nhau;
		\item[(c)] Viết số tự nhiên nhỏ nhất có 4 chữ số khác nhau \& đều là số chẵn;
		\item[(d)] Viết số tự nhiên nhỏ nhất có 4 chữ số khác nhau \& đều là số lẻ.
	\end{enumerate*}
\end{baitoan}

\begin{baitoan}[\cite{Trong_Toan_6_2021}, \textbf{5.}, p. 11]
	Dùng các chữ số $0,3$, \& $5$ viết 1 số tự nhiên có 3 chữ số khác nhau mà chữ số $5$ có giá trị là $50$.
\end{baitoan}

\begin{baitoan}[\cite{Trong_Toan_6_2021}, \textbf{6.}, p. 11]
	\emph{Số chẵn} là số tự nhiên có chữ số tận cùng là $0,2,4,6,8$; \emph{số lẻ} là số tự nhiên có chữ số tận cùng là $1,3,5,7,9$. 2 số chẵn (hoặc lẻ) \emph{liên tiếp} thì hơn kém nhau $2$ đơn vị.
	\begin{enumerate*}
		\item[(a)] Viết tập hợp $A$ các số chẵn nhỏ hơn $15$.
		\item[(b)] Viết tập hợp $B$ các số lẻ lớn hơn $5$ nhưng nhỏ hơn $17$.
		\item[(c)] Viết tập hợp $C$ 3 số chẵn liên tiếp, trong đó số lớn nhất là $46$.
	\end{enumerate*}
\end{baitoan}

\begin{baitoan}[\cite{Trong_Toan_6_2021}, \textbf{9.}, p. 12]
	Trong 1 cửa hàng bánh kẹo, người ta đóng gói kẹo thành các loại: mỗi gói có $10$ cái kẹo; mỗi hộp có $10$ gói; mỗi thùng có $10$ hộp. 1 người mua $9$ thùng, $9$ hộp \& $9$ gói kẹo. Hỏi người đó đã mua tất cả bao nhiêu cái kẹo?
\end{baitoan}
\noindent\textsc{Kiến thức cần nhớ.}
\begin{tcolorbox}
	Mỗi số tự nhiên viết trong hệ thập phân đều biểu diễn được thành \textit{tổng giá trị các chữ số của nó}. 
\end{tcolorbox}

\begin{baitoan}[\cite{Trong_Toan_6_2021}, \textbf{14.}, \textbf{15.}, p. 12]
	Viết các số sau dưới dạng tổng giá trị các chữ số của nó:
	\begin{align*}
		\overline{5at},\overline{ab},\overline{xyz},\overline{a5b},\overline{xyzt},\overline{xt5z},\overline{a2yb3}.
	\end{align*}
\end{baitoan}
\noindent\textsc{Kiến thức cần nhớ.}
\begin{tcolorbox}
	``Ngoài cách ghi số trong hệ thập phân gồm các chữ số từ $0$ đến $9$ \& các hàng (đơn vị, chục, trăm, nghìn, $\ldots$) như trên, còn có cách ghi số La Mã như sau: $\rm I = 1,V = 5,X = 10$. Mỗi chữ số La Mã có giá trị không phụ thuộc vào vị trí của nó trong số La Mã. Mỗi số La Mã biểu diễn 1 số tự nhiên bằng tổng giá trị của các thành phần viết nên số đó. Không có số La Mã nào biểu diễn số $0$.'' -- \cite[p. 13]{Trong_Toan_6_2021}
\end{tcolorbox}

\begin{baitoan}[\cite{Trong_Toan_6_2021}, \textbf{16.}, p. 13]
	Viết giá trị tương ứng trong hệ thập phân của các số La Mã: $\rm XIV,XVI,XXIII$.
\end{baitoan}

\begin{baitoan}[\cite{Trong_Toan_6_2021}, \textbf{17.}, p. 13]
	Viết các số sau bằng số La Mã: $18$, $25$.
\end{baitoan}

\begin{baitoan}[\cite{Trong_Toan_6_2021}, \textbf{18.}, p. 13]
	Sắp xếp theo thứ tự từ lớn đến bé: $\rm I,VII,IX,XI,V,IV,II,XVIII$.
\end{baitoan}
\noindent\textsc{Kiến thức cần nhớ.}
\begin{tcolorbox}
	``Đối với biểu thức có phép toán cộng, trừ, nhân, chia, ta thực hiện phép tính nhân, chia trước, cộng, trừ sau.'' -- \cite[p. 13]{Trong_Toan_6_2021}. ``Phép cộng \& phép nhân có tính chất giao hoán \& kết hợp: Tính chất giao hoán: $a + b = b + a$, $ab = ba$. Tính chất kết hợp: $(a + b) + c = a + (b + c)$, $(ab)c = a(bc)$.'' -- \cite[p. 14]{Trong_Toan_6_2021}
	
	``Tính chất phân phối của phép nhân đối với phép cộng: Muốn nhân 1 số với 1 tổng, ta lấy số đó nhân với từng số hạng của tổng, i.e., $a(b + c) = ab + ac$. Tính chất cộng với số $0$, nhân với số $1$: $a + 0 = a$ \& $a\cdot 1 = a$. Ngược với phép nhân phân phối là lấy thừa số chung.'' -- \cite[p. 14]{Trong_Toan_6_2021}
	
	``Muốn tính biểu thức 1 cách hợp lý, ta sử dụng tính chất giao hoán, kết hợp để xuất hiện các phép tính có kết quả tròn chuc, tròn trăm, tròn nghìn, $\ldots$\footnote{I.e., làm xuất hiện $a\cdot 10^n$ với $a,n\in\mathbb{N}^\star$.}'' -- \cite[p. 15]{Trong_Toan_6_2021}
\end{tcolorbox}

\begin{baitoan}[\cite{Trong_Toan_6_2021}, \textbf{26.}, p. 15]
	Tính hợp lý:
	\begin{enumerate*}
		\item[(a)] $1 + 7 + 9$;
		\item[(b)] $2 + 5 + 8$;
		\item[(c)] $11 + 2 + 8 + 9$;
		\item[(d)] $5\cdot 3\cdot 4$;
		\item[(e)] $2\cdot 3\cdot 50$;
		\item[(f)] $9\cdot 6 + 9\cdot 4$;
		\item[(g)] $2\cdot 8 + 2\cdot 12$;
		\item[(h)] $4\cdot 7 + 4 \cdot 13$;
		\item[(i)] $7\cdot 3 + 7\cdot 17$;
		\item[(j)] $11\cdot 13 + 37\cdot 11$.
	\end{enumerate*}
\end{baitoan}

\begin{baitoan}[\cite{Trong_Toan_6_2021}, \textbf{27.}, p. 15]
	Tính nhanh:
	\begin{enumerate*}
		\item[(a)] $46 + 17 + 54$;
		\item[(b)] $87\cdot 36 + 87\cdot 64$.
	\end{enumerate*}
\end{baitoan}

\begin{baitoan}[\cite{Trong_Toan_6_2021}, \textbf{28.}, p. 15]
	Áp dụng các tính chất của phép cộng \& phép nhân để tính nhanh:
	\begin{enumerate*}
		\item[(a)] $86 + 357 + 14$;
		\item[(b)] $772 + 69 + 128$;
		\item[(c)] $25\cdot 5\cdot 4\cdot 27\cdot 2$;
		\item[(d)] $28\cdot 64 + 28\cdot 36$.
	\end{enumerate*}
\end{baitoan}

\begin{baitoan}[\cite{Trong_Toan_6_2021}, \textbf{29.}, p. 15]
	Áp dụng các tính chất của phép cộng \& phép nhân để tính nhanh:
	\begin{enumerate*}
		\item[(a)] $25 + 39 + 21$;
		\item[(b)] $997 + 29 + 3 + 51$;
		\item[(c)] $578 + 125 + 422 + 375$;
		\item[(d)] $198 + 789 + 502 + 311$;
		\item[(e)] $158 + 445 + 342 + 555$;
		\item[(f)] $714 + 382 + 286 + 318$;
		\item[(g)] $15\cdot 6\cdot 4\cdot 125\cdot 8$;
		\item[(h)] $14\cdot 25\cdot 6\cdot 7$;
		\item[(i)] $24\cdot 3\cdot 5\cdot 10$;
		\item[(j)] $18\cdot 26\cdot 25\cdot 9$;
		\item[(k)] $25(187 + 18 + 1382)$;
		\item[(l)] $125\cdot 98\cdot 2\cdot 8\cdot 25$;
		\item[(m)] $1122\cdot 34 + 2244\cdot 83$;
		\item[(n)] $8466\cdot 15 + 170\cdot 4233$;
		\item[(o)] $1 + 2 + 3 + 4 + 5 + 6 + 7 + 8$;
		\item[(p)] $3 + 4 + 5 + 6 + 7 + 8 + 9 + 10 + 11$.
	\end{enumerate*}
\end{baitoan}

\begin{baitoan}[\cite{Trong_Toan_6_2021}, \textbf{30.}, p. 15]
	Tính nhanh:
	\begin{enumerate*}
		\item[(a)] $285 + 470 + 115 + 230$;
		\item[(b)] $571 + 216 + 129 + 124$.
	\end{enumerate*}
\end{baitoan}

\begin{baitoan}[\cite{Trong_Toan_6_2021}, \textbf{31.}, p. 15]
	Tìm các tích bằng nhau mà không cần tính kết quả của mỗi tích: $15\cdot 2\cdot 6$, $4\cdot 4\cdot 9$, $5\cdot 3\cdot 12$, $15\cdot 3\cdot 4$, $8\cdot 2\cdot 9$.
\end{baitoan}

\begin{baitoan}[\cite{Trong_Toan_6_2021}, \textbf{32.}, p. 15]
	Tính nhanh:
	\begin{enumerate*}
		\item[(a)] $13\cdot 58\cdot 4 + 32\cdot 26\cdot 2 + 52\cdot 10$;
		\item[(b)] $15\cdot 37\cdot 4 + 120\cdot 21 + 21\cdot 5\cdot 12$;
		\item[(c)] $14\cdot 35\cdot 5 + 10\cdot 25\cdot 7 + 20\cdot 70$;
		\item[(d)] $15(27 + 18 + 6) + 15(23 + 12)$;
		\item[(e)] $24(15 + 49) + 12(50 + 42)$;
		\item[(f)] $10(81 + 19) + 100 + 50(91 + 9)$;
		\item[(g)] $53(51 + 4) + 53(49 + 96) + 53$;
		\item[(h)] $42(15 + 96) + 6(25 + 4)\cdot 7$;
		\item[(i)] $45(13 + 78) + 9(87 + 22)\cdot 5$;
		\item[(j)] $16(27 + 75) + 8(53 + 25)\cdot 2$.
	\end{enumerate*}
\end{baitoan}
\noindent\textsc{Kiến thức cần nhớ.}
\begin{tcolorbox}
	``Muốn tìm số hạng chưa biết, ta lấy tổng trừ đi số hạng đã biết. Muốn tìm số bị trừ, ta lấy hiệu cộng với số trừ. Muốn tìm số trừ, ta lấy số bị trừ trừ đi hiệu. Muốn tìm thừa số chưa biết, ta lấy tích chia cho thừa số đã biết. Muốn tìm số bị chia, ta lấy thương nhân với số chia. Muốn tìm số chia, ta lấy số bị chia chia cho thương.'' -- \cite[p. 16]{Trong_Toan_6_2021}
	
	``Cho 2 số tự nhiên $a$ \& $b$, trong đó $b\ne 0$, ta luôn tìm được 2 số tự nhiên $q$ \& $r$ duy nhất sao cho:
	
	\fbox{số bị chia $=$ số chia $\cdot$ thương $+$ số dư}, i.e., $a = bq + r$, trong đó $0\le r < b$. Nếu $r = 0$ thì ta có phép chia hết. Nếu $r\ne 0$ thì ta có phép chia có dư. Điều kiện để thực hiện phép trừ các số tự nhiên là số bị trừ lớn hơn hoặc bằng số trừ. Số chia bao giờ cũng khác $0$.'' -- \cite[p. 18]{Trong_Toan_6_2021}
\end{tcolorbox}

\begin{baitoan}[\cite{Trong_Toan_6_2021}, \textbf{42.}, p. 19]
	Tìm $a\in\mathbb{N}$ biết khi chia $a$ cho $4$ thì được thương là $14$ \& có số dư là $12$.
\end{baitoan}

\begin{baitoan}[\cite{Trong_Toan_6_2021}, \textbf{43.}, p. 19]
	Tìm $m\in\mathbb{N}$ biết khi chia $m$ cho $13$ thì được thương là $4$ \& có số dư là $12$.
\end{baitoan}

\begin{baitoan}[\cite{Trong_Toan_6_2021}, \textbf{44.}, p. 19]
	Tìm $a\in\mathbb{N}$ biết khi chia $58$ cho $a$ thì được thương là $4$ \& có số dư là $2$.
\end{baitoan}

\begin{baitoan}[\cite{Trong_Toan_6_2021}, \textbf{45.}, p. 19]
	Tìm $b\in\mathbb{N}$ biết khi chia $64$ cho $b$ thì được thương là $4$ \& có số dư là $12$.
\end{baitoan}

\begin{baitoan}[\cite{Trong_Toan_6_2021}, \textbf{46.}, p. 19]
	Tìm $a\in\mathbb{N}$ biết khi chia $a$ cho $13$ thì được thương là $4$ \& có số dư $r$ lớn hơn $11$.
\end{baitoan}

\begin{baitoan}[\cite{Trong_Toan_6_2021}, \textbf{47.}, p. 19]
	Tìm $a\in\mathbb{N}$ biết khi chia $a$ cho $13$ thì được thương là $4$ \& số dư là số lớn nhất có thể được trong phép chia ấy.
\end{baitoan}

\begin{baitoan}[\cite{Trong_Toan_6_2021}, \textbf{48.}, p. 19]
	Tìm $a\in\mathbb{N}$, biết khi chia $a$ cho $17$ thì được thương là $6$ \& số dư là số lớn nhất có thể có trong phép chia ấy.
\end{baitoan}

\begin{baitoan}[\cite{Trong_Toan_6_2021}, \textbf{49.}, p. 19]
	Tìm $a\in\mathbb{N}$, biết khi chia $a$ cho $17$ thì được thương là $6$ \& số dư lớn hơn $15$.
\end{baitoan}

\begin{baitoan}[\cite{Trong_Toan_6_2021}, \textbf{50.}, p. 19]
	\begin{enumerate*}
		\item Minh dùng $23000$đ để mua bút. Mỗi cây bút giá $2000$đ. Hỏi Minh mua được nhiều nhất bao nhiêu cây bút? \& còn dư mấy ngàn?
		\item Lan dùng $5000$đ để mua bút. 1 cây bút giá $2000$đ. Hỏi Lan mua được nhiều nhất mấy cây bút? \& còn dư mấy ngàn?
	\end{enumerate*}
\end{baitoan}

\begin{baitoan}[\cite{Trong_Toan_6_2021}, \textbf{51.}, p. 19]
	1 trường có $50$ phòng học, mỗi phòng có $11$ bộ bàn ghế, mỗi bộ bàn ghế có thể xếp cho $4$ học sinh ngồi. Trường có thể nhận nhiều nhất bao nhiêu học sinh để mọi học sinh đều có chỗ ngồi?
\end{baitoan}

\begin{baitoan}[\cite{Trong_Toan_6_2021}, \textbf{52.}, p. 19]
	1 trường Trung học cơ sở có $997$ học sinh tham dự lễ tổng kết cuối năm. Ban tổ chức đã chuẩn bị những chiếc băng $5$ chỗ ngồi. Phải có ít nhất bao nhiêu ghế băng như vậy để tất cả học sinh đều có chỗ ngồi?
\end{baitoan}

\begin{baitoan}[\cite{Trong_Toan_6_2021}, \textbf{53.}, p. 19]
	1 tàu hỏa cần chở $900$ khách. Mỗi toa tàu chứa được $88$ khách. Hỏi cần ít nhất bao nhiêu toa để chở hết khách?
\end{baitoan}

\begin{baitoan}[\cite{Trong_Toan_6_2021}, \textbf{54.}, p. 19]
	Tỉnh Bắc Giang có dân số $1803905$ \& đứng thứ $12$ về dân số trong $63$ tỉnh thành toàn quốc. Tính dân số Thanh Hóa (tỉnh đông dân thứ 3), biết rằng gấp đôi số dân Bắc Giang vẫn còn kém dân số Thanh Hóa $32228$ người.
\end{baitoan}

\begin{baitoan}[\cite{Trong_Toan_6_2021}, \textbf{55.}, p. 19]
	1 tàu hỏa cần chở $980$ khách. Mỗi toa tàu có $11$ khoang, mỗi khoang có $8$ chỗ ngồi. Hỏi cần có ít nhất bao nhiêu toa để chở hết khách?
\end{baitoan}

\begin{baitoan}[\cite{Trong_Toan_6_2021}, \textbf{56.}, p. 19]
	Mỗi hội trường có $32$ chỗ ngồi cho 1 hàng ghế. Nếu có $890$ đại biểu tham dự họp thì phải dùng ít nhất bao nhiêu hàng ghế?
\end{baitoan}

\begin{baitoan}[\cite{Trong_Toan_6_2021}, \textbf{57.}, p. 19]
	Tìm $a,b\in\mathbb{N}$, biết $ab + 13 = 200$.
\end{baitoan}

\begin{baitoan}[\cite{Trong_Toan_6_2021}, \textbf{58.}, p. 19]
	Trong 1 phép chia có số bị chia là $200$, số dư là $13$. Tìm số chia \& thương.
\end{baitoan}

\begin{baitoan}[\cite{Trong_Toan_6_2021}, \textbf{59.}, p. 19]
	Trong tháng $7$ nhà ông Khánh dùng hết $115$ số điện. Hỏi ông Khánh phải trả bao nhiêu tiền điện, biết đơn giá điện như sau: Giá tiền cho $50$ số đầu tiên là $1678$đ\emph{\texttt{/}}số. Giá tiền cho $50$ số tiếp theo $51$--$100$) là $1734$đ\emph{\texttt{/}}số. Giá tiền cho $100$ số tiếp theo ($101$--$200$) là $2014$đ\emph{\texttt{/}}số.
\end{baitoan}

\begin{baitoan}[\cite{Trong_Toan_6_2021}, \textbf{60.}, p. 20]
	1 phòng chiếu phim có $18$ hàng ghế, mỗi hàng có $18$ ghế. Giá 1 vé xem phim là $50000$đ.
	\begin{enumerate*}
		\item[(a)] Tối thứ $7$, tất cả các vé đều được bán hết. Số tiền bán vé thu được là bao nhiêu?
		\item[(b)] Tối thứ $6$, số tiền bán vé thu được là $10550000$đ. Hỏi có bao nhiêu vé không bán được?
		\item[(c)] Chủ Nhật còn $41$ vé không bán được. Hỏi số tiền bán vé thu được là bao nhiêu?
	\end{enumerate*}
\end{baitoan}
	
\subsection{Lũy thừa của 1 số tự nhiên}
\textsc{Kiến thức cần nhớ.}
\begin{tcolorbox}
	``\textbf{Dạng.} Lũy thừa là tích của nhiều thừa số giống nhau.
	\begin{align}
		a^n = \underbrace{a\cdot a\cdots a}_{n\mbox{ \footnotesize thừa số }},\ \forall n\in\mathbb{N}^\star.
	\end{align}
	$a^n$, trong đó $a$ là \textit{cơ số}, $n$ là \textit{số mũ}. Quy ước: $a^0 = 1$, $\forall a\in\mathbb{N}^\star$.'' -- \cite[p. 20]{Trong_Toan_6_2021}
\end{tcolorbox}

\begin{baitoan}[\cite{Trong_Toan_6_2021}, \textbf{13.}, p. 21]
	Tìm $c\in\mathbb{N}$, biết rằng với mọi $n\in\mathbb{N}^\star$ ta có:
	\begin{enumerate*}
		\item[(a)] $c^n = 1$;
		\item[(b)] $c^n = 0$.
	\end{enumerate*}
\end{baitoan}

\begin{baitoan}[\cite{Trong_Toan_6_2021}, \textbf{14.}, p. 21]
	Tính $11^2,111^2$. Từ đó dự đoán kết quả của $1111^2,11111^2$, \& $1\ldots 1^2$ với $n$ số $1$.
\end{baitoan}

\begin{baitoan}[\cite{Trong_Toan_6_2021}, \textbf{15.}, p. 21]
	Ta có: $1 + 3 + 5 = 9 = 3^2$. Viết các tổng sau dưới dạng bình phương của 1 số tự nhiên:
	\begin{enumerate*}
		\item[(a)] $1 + 3 + 5 + 7$;
		\item[(b)] $1 + 3 + 5 + 7 + 9$;
		\item[(c)] $1 + 3 + 5 + 7 + 9 + 11$;
		\item[(d)] Tổng của $n$ số lẻ đầu tiên: $1 + 3 + \cdots + (2n - 3) + (2n - 1) = \sum_{i=1}^n (2i - 1)$.
	\end{enumerate*}
\end{baitoan}

\begin{baitoan}[\cite{Trong_Toan_6_2021}, \textbf{16.}, p. 21]
	\emph{Số chính phương} là số bằng bình phương của 1 số tự nhiên (e.g., $0,1,4,9,16,\ldots$). Mỗi tổng sau có là 1 số chính phương không?
	\begin{enumerate*}
		\item[(a)] $1^3 + 2^3$;
		\item[(b)] $1^3 + 2^3 + 3^3$;
		\item[(c)] $1^3 + 2^3 + 3^3 + 4^3$.
		\item[(d)] $\sum_{i=1}^n i^3 = 1^3 + 2^3 + \cdots + n^3$.
	\end{enumerate*}
\end{baitoan}
\noindent\textsc{Kiến thức cần nhớ.}
\begin{tcolorbox}
	``Khi nhân 2 hay nhiều lũy thừa cùng cơ số, ta giữa nguyên cơ số \& cộng các số mũ. $a^ma^n = a^{m + n}$, $\forall a,m,n\in\mathbb{N}$. Khi chia 2 lũy thừa cùng cơ số (khác $0$), ta giữ nguyên cơ số \& trừ các số mũ. $a^m:a^n = a^{m-n}$, $\forall a,m,n\in\mathbb{N}$, $a\ne 0$, $m\ge n$.'' -- \cite[p. 22]{Trong_Toan_6_2021}
\end{tcolorbox}

\begin{baitoan}[\cite{Trong_Toan_6_2021}, \textbf{22.}, p. 22]
	Biết rằng khối lượng của Trái Đất khoảng $600\ldots 00$ tấn với $21$ chữ số $0$, khối lượng của Mặt Trăng khoảng $7500\ldots 00$ với $18$ chữ số $0$.
	\begin{enumerate*}
		\item[(a)] Viết khối lượng Trái Đất \& khối lượng Mặt Trăng dưới dạng tích của 1 số với lũy thừa của $10$.
		\item[(b)] Khối lượng của Trái Đất gấp bao nhiêu lần khối lượng Mặt Trăng?
	\end{enumerate*}
\end{baitoan}
\noindent\textsc{Kiến thức cần nhớ.}
\begin{tcolorbox}
	``Muốn tìm $x$ ở số mũ, ta đưa về lũy thừa cùng cơ số rồi suy ra số mũ bằng số mũ. Trong tập hợp số tự nhiên $\mathbb{N}$, muốn tìm $x$ ở cơ số, ta đưa về lũy thừa cùng số mũ, suy ra cơ số bằng cơ số.'' -- \cite[p. 22]{Trong_Toan_6_2021}
\end{tcolorbox}

\begin{baitoan}
	Giải \& biện luận phương trình $m^x = m^n$ với $m,n\in\mathbb{N}$ cho trước. Tương tự, giải \& biện luận phương trình $m^{f(x)} = m^n$ với $m,n\in\mathbb{N}$ với $f$ là 1 hàm số sao cho phương trình $f(x) = m$ giải được \& có các nghiệm tự nhiên là các số $x_i$, $i = 1,\ldots,k$, $k\in\mathbb{N}$.
\end{baitoan}

\begin{baitoan}
	Giải \& biện luận phương trình $(ax + b)^n = m^n$ với $a,b,m,n\in\mathbb{N}$ cho trước. Tương tự, giải \& biện luận phương trình $(f(x))^n = m^n$ với $f$ là 1 hàm số sao cho phương trình $f(x) = \pm m$ giải được \& có các nghiệm tự nhiên là các số $x_i$, $i = 1,\ldots,k$, $k\in\mathbb{N}$.
\end{baitoan}

\subsection{Thứ tự thực hiện phép tính}
\textsc{Kiến thức cần nhớ.}
\begin{tcolorbox}
	``Đối với biểu thức không có dấu ngoặc, ta thực hiện phép tính lũy thừa, rồi đến phép tính nhân, chia, rồi đến phép tính cộng \& trừ. Đối với biểu thức có dấu ngoặc, ta thực hiện các phép tính trong dấu ngoặc tròn ( ), rồi đến các phép tính trong dấu ngoặc vuông [ ], rồi đến các phép tính trong dấu ngoặc nhọn \{ \}.'' -- \cite[p. 24]{Trong_Toan_6_2021}
\end{tcolorbox}

\begin{baitoan}[\cite{Trong_Toan_6_2021}, \textbf{3.}, p. 24]
	Tính:
	\begin{enumerate*}
		\item[(a)] $13 + 21\cdot 5 - (198:11 - 8)$;
		\item[(b)] $272:16 - 5 + 4(30 - 5 - 255:17)$;
		\item[(c)] $15\cdot 24 - 14\cdot 5(145:5 - 27)$;
		\item[(d)] $18\cdot 3 - 18\cdot 2 + 3(51:17)$;
		\item[(e)] $(64 + 115 + 36) - 25\cdot 8$;
		\item[(f)] $15\cdot 8 - (17 - 30 + 83) - 144:6$;
		\item[(g)] $250:50 - (46 - 75 + 54)$;
		\item[(h)] $13(17 - 95 + 83):5 - 18:9$;
		\item[(i)] $140 - 180(47 - 90 + 43) + 7$;
		\item[(j)] $24(15 + 30 + 85 - 120):10$;
		\item[(k)] $27 + 73 - 30(25 - 10)$;
		\item[(l)] $18 - 4(27 - 90 + 73):10$.
	\end{enumerate*}
\end{baitoan}

\begin{baitoan}[\cite{Trong_Toan_6_2021}, \textbf{4.}, pp. 24--25]
	Tính:
	\begin{enumerate*}
		\item[(a)] $140 - [25 :(4^2 - 11) + 4]$;
		\item[(b)] $40 - [6 - (5 - 1)]$;
		\item[(c)] $4\cdot 3 + [8 - (2 + 3)]$;
		\item[(d)] $36:\{46 - [4(17 - 7)]\}$;
		\item[(e)] $2\cdot\{19 + [12:(8 - 4)] + 5\}$;
		\item[(f)] $12:\{18:[9 - (4 + 2)]\}$;
		\item[(g)] $40:\{5[10 - (6 + 3)]\}$;
		\item[(h)] $25\{16:[12 - 4 + 4(4:2)]\}$;
		\item[(i)] $3[(15\cdot 2):(5 + 5\cdot 2)]$;
		\item[(j)] $30:\{15:[8 - (1 + 2)]\}$;
		\item[(k)] $15 - \{15:[6 - (1 + 2)]\}$.
	\end{enumerate*}
\end{baitoan}

\begin{baitoan}[\cite{Trong_Toan_6_2021}, \textbf{5.}, p. 25]
	Tính:
	\begin{enumerate*}
		\item[(a)] $(6:2) + 4^2$;
		\item[(b)] $(5\cdot 2^2 - 20):5$;
		\item[(c)] $2^3(7 + 3)$;
		\item[(d)] $(4\cdot 5 - 2^3)\cdot 2$;
		\item[(e)] $(5^2\cdot 2 - 10)\cdot 4$;
		\item[(f)] $(1^{10} + 80):3^2$;
		\item[(g)] $2^3\cdot 5 - (15 - 10)$;
		\item[(h)] $2^2 + [10^5:10^4 - (2 + 3\cdot 2)]$;
		\item[(i)] $2^2 + [5^3:5^2 + (6:2)]$;
		\item[(j)] $3^2 + [4^5:4^3 - (12:3)]$.
	\end{enumerate*}
\end{baitoan}

\begin{baitoan}[\cite{Trong_Toan_6_2021}, \textbf{6.}, p. 25]
	Tính:
	\begin{enumerate*}
		\item[(a)] $(2^{2007} + 2^{2006}):2^{2006}$;
		\item[(b)] $(3^{2011} + 3^{2010}):3^{2010}$;
		\item[(c)] $(5^{2001} + 5^{2000}):5^{2000}$;
		\item[(d)] $(4^{2001} - 4^{2000}):4^{2000}$;
		\item[(e)] $(6^{2005} - 6^{2004}):6^{2004}$;
		\item[(f)] $(7^{2011} - 7^{2010}):7^{2010}$.
	\end{enumerate*}
\end{baitoan}

\begin{baitoan}[\cite{Trong_Toan_6_2021}, \textbf{7.}, p. 25]
	Tính:
	\begin{enumerate*}
		\item[(a)] $9\cdot(8^2 - 15)$;
		\item[(b)] $75:3 + 6\cdot 9^2$;
		\item[(c)] $39\cdot 213 + 87\cdot 39$;
		\item[(d)] $80 - [130 - (12 - 4)^2]$.
	\end{enumerate*}
\end{baitoan}

\begin{baitoan}[\cite{Trong_Toan_6_2021}, \textbf{8.}, p. 25]
	Tính:
	\begin{enumerate*}
		\item[(a)] $25:5\cdot 7$;
		\item[(b)] $30:2\cdot 8\cdot 4$;
		\item[(c)] $20:2^2\cdot 14$;
		\item[(d)] $125:5^3\cdot 170$;
		\item[(e)] $64:2^5\cdot 30\cdot 4$;
		\item[(f)] $(25:5^2\cdot 30):15\cdot 7$;
		\item[(g)] $[(5^2\cdot 2:10)\cdot 4]:(2^2\cdot 5:2)$;
		\item[(h)] $(15:3\cdot 5^2):(20:2^2)$;
		\item[(i)] $2^2\cdot 3^2 - 5\cdot 2\cdot 3$;
		\item[(j)] $3^2\cdot 5 - 2^2\cdot 7 + 1\cdot 5$;
		\item[(k)] $5^2\cdot 2 - 3^2\cdot 4$;
		\item[(l)] $7^2\cdot 3 - 5^2\cdot 3$;
		\item[(m)] $(5\cdot 2^2 - 20):5 + 3^2\cdot 6$;
		\item[(n)] $(24\cdot 5 - 5^2\cdot 2):(5\cdot 2) - 3$;
		\item[(o)] $[(5^2\cdot 2^3 - 7^2\cdot 2):2]\cdot 6 - 7\cdot 2^5$;
		\item[(p)] $(6\cdot 5^2 - 13\cdot 7)\cdot 2 - 2^3(7 + 3)$.
	\end{enumerate*}
\end{baitoan}

\begin{baitoan}[\cite{Trong_Toan_6_2021}, \textbf{9.}, p. 26]
	Tính:
	\begin{enumerate*}
		\item[(a)] $2^3 - 5^3:5^2 + 12\cdot 2^2$;
		\item[(b)] $5[(85 - 35:7):8 + 90] - 50$;
		\item[(c)] $2[(2 - 3^3:3^2):2^2 + 99] - 100$;
		\item[(d)] $2^7:2^2 + 5^4:5^3\cdot 2^4 - 3\cdot 2^5$;
		\item[(e)] $5\cdot 2^2\cdot 2^3 - 4(5^8:5^6)$;
		\item[(e)] $(3^5\cdot 3^7):3^{10} + 5\cdot 2^4 - 7^3:7$;
		\item[(f)] $15:(3^5:3^4) - 2^9:2^7$;
		\item[(g)] $5\cdot 3^5:(3^8:3^5) - 2^3\cdot 5$;
		\item[(h)] $4[(3 + 3^7:3^4):10 + 97] - 300$;
		\item[(i)] $5[(92 + 2^5:2^2):5^2 + 2^4] - 7^2$;
		\item[(j)] $3^2[(5^2 - 3):11] - 2^4 + 2\cdot 10^3$;
		\item[(k)] $2^2\cdot 5[(5^2 + 2^3):11 - 2] - 3^2\cdot 2$;
		\item[(l)] $(6^{2007} - 6^{2006}):6^{2006}$;
		\item[(m)] $(5^{2001} - 5^{2000}):5^{2000}$;
		\item[(n)] $(7^{2005} + 7^{2004}):7^{2004}$;
		\item[(o)] $(11^{2023} + 11^{2022}):11^{2022}$;
		\item[(p)] $(5^7 + 5^9)(6^8 + 6^{10})(2^4 - 4^2)$;
		\item[(q)] $(7^3 + 7^5)(5^4 + 5^6)(3^3\cdot 3 - 9^2)$.		
	\end{enumerate*}
\end{baitoan}	

\begin{baitoan}[\cite{Trong_Toan_6_2021}, \textbf{10.}, p. 26]
	Trong 8 tháng đầu năm, 1 cửa hàng bán được $1264$ chiếc ti vi. Trong 4 tháng cuối năm, trung bình mỗi tháng cửa hàng bán được $164$ ti vi. Hỏi trong cả năm, trung bình mỗi tháng cửa hàng đó bán được bao nhiêu ti vi? Viết biểu thức viết kết quả.
\end{baitoan}

\begin{baitoan}[\cite{Trong_Toan_6_2021}, \textbf{13.}, p. 26]
	Trang đố Nga dùng 4 chữ số $2$ cùng với dấu phép tính \& dấu ngoặc (nếu cần) viết dãy tính có kết quả lần lượt bằng $0,1,2,3,4$. Giúp Nga làm điều đó.
\end{baitoan}
\noindent\textsc{Kiến thức cần nhớ.}
\begin{tcolorbox}
	``Muốn tính biểu thức 1 cách hợp lý, ta sử dụng tính chất giao hoán, kết hợp để xuất hiện các phép tính có kết quả tròn chục, tròn trăm, tròn nghìn, $\ldots$'' -- \cite[p. 26]{Trong_Toan_6_2021}, i.e., làm xuất hiện $a10^n$ với $a,n\in\mathbb{N}^\star$ 1 cách hợp lý.
\end{tcolorbox}

\begin{baitoan}[\cite{Trong_Toan_6_2021}, \textbf{15.}, p. 27]
	Tính hợp lý:
	\begin{enumerate*}
		\item[(a)] $4\cdot 24\cdot 5^2 - (3^3\cdot 18 + 3^3\cdot 12)$;
		\item[(b)] $2^3\cdot 7\cdot 5^3 - (5^2\cdot 65 + 5^2\cdot 35)$;
		\item[(c)] $2^2\cdot 74\cdot 5^2 + 5^2\cdot 26\cdot 4 - 7000$;
		\item[(d)] $31\cdot 15\cdot 7^2\cdot 4 - 31\cdot 49\cdot 40$;
		\item[(e)] $55\cdot 2^2\cdot 5 + 4\cdot 89\cdot 5^2 - 3^2\cdot 10^3$.
	\end{enumerate*}
\end{baitoan}
\noindent\textsc{Kiến thức cần nhớ.}
\begin{tcolorbox}
	``Ta có thể tính tổng các số hạng cách đều nhau dựa vào công thức sau:
	\begin{align*}
		\mbox{số số hạng} &= (\mbox{số lớn nhất} - \mbox{số bé nhất}):\mbox{khoảng cách giữa 2 số liên tiếp} + 1,\\
		\mbox{tổng} &= [(\mbox{số đầu} + \mbox{số cuối})\cdot\mbox{số số hạng}]:2.
	\end{align*}
	'' -- \cite[p. 29]{Trong_Toan_6_2021}
\end{tcolorbox}
\begin{baitoan}[\cite{Trong_Toan_6_2021}, \textbf{15.}, p. 27]
	Tính hợp lý:
	\begin{enumerate*}
		\item[(a)] $1 + 2 + 3 + \cdots + 9 + 10 = \sum_{i=1}^{10} i$;
		\item[(b)] $2 + 4 + 6 + \cdots + 16 + 18 = \sum_{i=1}^9 2i$;
		\item[(c)] $1 + 3 + 5 + \cdots 17 + 19 = \sum_{i=0}^9 (2i + 1)$;
		\item[(d)] $1 + 4 + 7 + \cdots + 25 + 28 = \sum_{i=0}^9 (3i + 1)$;
		\item[(e)] $2 + 6 + 10 + \cdots + 30 + 34 = \sum_{i=0}^8 (4i + 2)$;
		\item[(f)] $3 + 8 + 13 + \cdots + 38 + 43 = \sum_{i=0}^8 (5i + 3)$;
		\item[(g)] $5 + 8 + 11 + \cdots + 26 + 29 = \sum_{i=1}^9 (3i + 2)$;
		\item[(h)] $7 + 11 + 15 + \cdots + 43 + 47 = \sum_{i=1}^{11} (4i + 3)$;
		\item[(i)] $1 + 6 + 11 + \cdots + 46 + 51 = \sum_{i=0}^{10} (5i + 1)$;
		\item[(j)] $4 + 10 + 16 + \cdots + 58 + 64 = \sum_{i=0}^{10} (6i + 4)$;
		\item[(k)] $10 + 13 + 16 + \cdots + 37 + 40 = \sum_{i=3}^{13} (3i + 1)$;
		\item[(l)] $2 + 4 + 6 + 8 + 10 + 12 + 1 + 4 + 7 + 10 + 13 + 16 + 19$;
		\item[(m)] $5 + 7 + 9 + 11 + 13 + 15 + 17 + 3 + 8 + 13 + 18 + 23 + 28$;
		\item[(n)] $4 + 7 + 10 + 13 + 16 + 19 + 5 + 9 + 13 + 17 + 21 + 25$;
		\item[(o)] $7 + 12 + 17 + 22 + 27 + 8 + 10 + 12 + 14 + 16 + 18 + 20$.
	\end{enumerate*}
\end{baitoan}

%------------------------------------------------------------------------------%

\section{Tính Chất Chia Hết Trong Tập Hợp Các Số Tự Nhiên}
\textsc{Kiến thức cần nhớ.}
\begin{tcolorbox}
	``Số có chữ số tận cùng là $0,2,4,6,8$ thì chia hết cho $2$. Số có tổng các chữ số chia hết cho $3$ thì chia hết cho $3$. Số có chữ số tận cùng là $0$ hoặc $5$ thì chia hết cho $5$. Số có tổng các chữ số chia hết cho $9$ thì chia hết cho $9$.
	
	Ký hiệu: $a\ \vdots\ b$ đọc là $a$ chia hết cho $b$; $a\not\vdots\ b$ đọc là $a$ không chia hết cho $b$. Các số chia hết cho $9$ thì luôn chia hết cho $3$ nhưng các số chia hết cho $3$ thì có thể không chia hết cho $9$.'' -- \cite[p. 30]{Trong_Toan_6_2021}
\end{tcolorbox}

\begin{baitoan}[\cite{Trong_Toan_6_2021}, \textbf{10.}, p. 31]
	Thay dấu * bằng 1 chữ số để các số sau:
	\begin{enumerate*}
		\item[(a)] $\overline{1373*}$ chia hết cho $2$ \& cho $9$;
		\item[(b)] $\overline{158*}$ chia hết cho $2$ \& cho $3$;
		\item[(c)] $\overline{1475*}$ chia hết cho $2$ \& cho $5$;
		\item[(d)] $\overline{171*}$ chia hết cho $5$ \& cho $9$;
		\item[(e)] *
	\end{enumerate*}
\end{baitoan}
















\subsection{Dấu hiệu chia hết}

\subsection{Tính chất chia hết của 1 tổng, 1 hiệu}

\subsection{Ước \& bội}

\subsection{Số nguyên tố. Hợp số}

\subsection{Ước chung \& bội chung}

\subsection{Ước chung lớn nhất}

\subsection{Bội chung nhỏ nhất}

%------------------------------------------------------------------------------%

\section{Số Nguyên}

\subsection{Tập hợp các số nguyên}

\subsection{Phép cộng \& phép trừ số nguyên}

\subsection{Quy tắc dấu ngoặc}

\subsection{Quy tắc chuyển vế}

\subsection{Phép nhân \& phép chia hết 2 số nguyên}

%------------------------------------------------------------------------------%

\newpage
\section{Hình Học Trực Quan}

\subsection{Tam giác đều -- hình vuông -- lục giác đều}

\subsection{Hình chữ nhật -- hình thoi -- hình bình hành -- hình thang cân}

\subsection{Chu vi \& diện tích của 1 số tứ giác đã học}

%------------------------------------------------------------------------------%

\section{Tính Đối Xứng của Hình Phẳng Tự Nhiên}

\subsection{Hình có trục đối xứng}

\subsection{Hình có tâm đối xứng}

%------------------------------------------------------------------------------%

\newpage
\section{Phân Số}

\subsection{Mở rộng khái niệm phân số}

\subsection{Phân số bằng nhau}

\subsection{Tính chất cơ bản của phân số}

\subsection{So sánh phân số}

\subsection{Phép cộng \& trừ phân số}

\subsection{Phép nhân \& chia phân số}

\subsection{Hỗn số}

\subsection{Tìm giá trị phân số của 1 số cho trước}

\subsection{Tìm 1 số biết giá trị 1 phân số của nó}

%------------------------------------------------------------------------------%

\section{Số Thập Phân}

\subsection{Số thập phân. Phần trăm}

\subsection{Tính toán với số thập phân}

\subsection{Làm tròn số thập phân \& ước lượng kết quả}

\subsection{Tỷ số \& tỷ số phần trăm}

\subsection{2 bài toán về tỷ số phần trăm}

%------------------------------------------------------------------------------%

\newpage
\section{Những Hình Học Cơ Bản}

\subsection{Điểm \& đường thẳng}

\subsection{Điểm nằm giữa 2 điểm. Tia}

\subsection{Đoạn thẳng \& độ dài đoạn thẳng}

\subsection{Trung điểm của đoạn thẳng}

\subsection{Nửa mặt phẳng}

\subsection{Góc}

\subsection{Số đo góc}

%------------------------------------------------------------------------------%

\section{Xác Suất Thống Kê}

\subsection{Phép thử nghiệm -- Sự kiện}

\subsection{Thu thập \& phân loại dữ liệu}

\subsection{Biểu diễn dữ liệu trên bảng}

\subsection{Bảng thống kê \& biểu dồ tranh}

\subsection{Biểu đồ cột}

\subsection{Biểu đồ cột kép}

\subsection{Xác suất thực nghiệm}

\subsection{Hoạt động thực hành \& trải nghiệm}

%------------------------------------------------------------------------------%

%\selectlanguage{english}
%\begin{thebibliography}{99}
%	\bibitem[]{}
%\end{thebibliography}

%------------------------------------------------------------------------------%

\section{Solutions}

\newpage
Tài liệu: \cite{SGK_Toan_6_Canh_Dieu_tap_1, SGK_Toan_6_Canh_Dieu_tap_2, SBT_Toan_6_Canh_Dieu_tap_1, Binh_Toan_6_tap_1, Binh_Toan_6_tap_2, Trong_Toan_6_2021}.

\printbibliography[heading=bibintoc]
	
\end{document}