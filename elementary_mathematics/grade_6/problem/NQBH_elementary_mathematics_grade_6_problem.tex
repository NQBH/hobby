\documentclass{article}
\usepackage[backend=biber,natbib=true,style=authoryear]{biblatex}
\addbibresource{/home/hong/1_NQBH/reference/bib.bib}
\usepackage[utf8]{vietnam}
\usepackage{tocloft}
\renewcommand{\cftsecleader}{\cftdotfill{\cftdotsep}}
\usepackage[colorlinks=true,linkcolor=blue,urlcolor=red,citecolor=magenta]{hyperref}
\usepackage{amsmath,amssymb,amsthm,mathtools,float,graphicx,algpseudocode,algorithm,tcolorbox}
\usepackage[inline]{enumitem}
\allowdisplaybreaks
\numberwithin{equation}{section}
\newtheorem{assumption}{Assumption}[section]
\newtheorem{conjecture}{Conjecture}[section]
\newtheorem{corollary}{Corollary}[section]
\newtheorem{hequa}{Hệ quả}[section]
\newtheorem{definition}{Definition}[section]
\newtheorem{dinhnghia}{Định nghĩa}[section]
\newtheorem{example}{Example}[section]
\newtheorem{vidu}{Ví dụ}[section]
\newtheorem{lemma}{Lemma}[section]
\newtheorem{notation}{Notation}[section]
\newtheorem{principle}{Principle}[section]
\newtheorem{problem}{Problem}[section]
\newtheorem{baitoan}{Bài toán}[section]
\newtheorem{proposition}{Proposition}[section]
\newtheorem{question}{Question}[section]
\newtheorem{cauhoi}{Câu hỏi}[section]
\newtheorem{remark}{Remark}[section]
\newtheorem{luuy}{Lưu ý}[section]
\newtheorem{theorem}{Theorem}[section]
\newtheorem{dinhly}{Định lý}[section]
\usepackage[left=0.5in,right=0.5in,top=1.5cm,bottom=1.5cm]{geometry}
\usepackage{fancyhdr}
\pagestyle{fancy}
\fancyhf{}
\lhead{\small \textsc{Sect.} ~\thesection}
\rhead{\small \nouppercase{\leftmark}}
\renewcommand{\sectionmark}[1]{\markboth{#1}{}}
\cfoot{\thepage}
\def\labelitemii{$\circ$}

\title{Some Topics in Elementary Mathematics\texttt{/}Grade 6}
\author{Nguyễn Quản Bá Hồng\footnote{Independent Researcher, Ben Tre City, Vietnam\\e-mail: \texttt{nguyenquanbahong@gmail.com}; website: \url{https://nqbh.github.io}.}}
\date{\today}

\title{Problems in Elementary Mathematics\texttt{/}Grade 6}
\author{Nguyễn Quản Bá Hồng\footnote{Independent Researcher, Ben Tre City, Vietnam\\e-mail: \texttt{nguyenquanbahong@gmail.com}; website: \url{https://nqbh.github.io}.}}
\date{\today}

\begin{document}
\maketitle
\begin{abstract}
	1 bộ sưu tập các bài toán chọn lọc từ cơ bản đến nâng cao cho Toán sơ cấp lớp 6. Tài liệu này là phần bài tập bổ sung cho tài liệu chính \href{https://github.com/NQBH/hobby/blob/master/elementary_mathematics/grade_6/NQBH_elementary_mathematics_grade_6.pdf}{GitHub\texttt{/}NQBH\texttt{/}hobby\texttt{/}elementary mathematics\texttt{/}grade 6\texttt{/}lecture}\footnote{Explicitly, \url{https://github.com/NQBH/hobby/blob/master/elementary_mathematics/grade_6/NQBH_elementary_mathematics_grade_6.pdf}.}. Phiên bản mới nhất của tài liệu này được lưu trữ ở link sau: \href{https://github.com/NQBH/hobby/blob/master/elementary_mathematics/grade_6/problem/NQBH_elementary_mathematics_grade_6_problem.pdf}{GitHub\texttt{/}NQBH\texttt{/}hobby\texttt{/}elementary mathematics\texttt{/}grade 6\texttt{/}problem}\footnote{Explicitly, \url{https://github.com/NQBH/hobby/blob/master/elementary_mathematics/grade_6/problem/NQBH_elementary_mathematics_grade_6_problem.pdf}.}.
\end{abstract}

\tableofcontents
\newpage

%------------------------------------------------------------------------------%

\section{Tập Hợp Các Số Tự Nhiên}

\subsection{Tập hợp}
\textsc{Kiến thức cần nhớ.}
\begin{tcolorbox}	
	Tên tập hợp được viết bằng chữ cái in hoa. Cho $A = \{a;b;c\}$. Khi đó, $a,b,c$ là các phần tử của tập hợp $A$. $a\in A$ đọc là \textit{$a$ thuộc tập hợp $A$} hay \textit{$a$ là phần tử của tập hợp $A$}. $d\notin A$, đọc là \textit{$d$ không thuộc tập hợp $A$} hay \textit{$d$ không là phần tử của tập hợp $A$}.
	
	Cách viết tập hợp có 2 cách:
	\begin{itemize}
		\item \textit{Cách 1.} Liệt kê các phần tử của tập hợp: các phần tử của 1 tập hợp được viết trong 2 dấu ngoặc nhọn $\{\ \}$, cách nhau bởi dấu ``;'' (nếu có phần tử là số) hoặc dấu ``,''. Mỗi phần tử được liệt kê 1 lần, thứ tự liệt kê tùy ý.
		\item \textit{Cách 2.} Chỉ ra tính chất đặc trưng của các phần tử của tập hợp.
	\end{itemize}
	\textbf{Các ký hiệu.} $\mathbb{N}$: tập hợp các số tự nhiên. $\mathbb{N}^\star$: tập hợp các số tự nhiên khác $0$. $|$: sao cho, thỏa mãn. $\ge$: lớn hơn hoặc bằng ($>$ hoặc $=$). $\le$: nhỏ hơn hoặc bằng ($<$ hoặc $=$). $\emptyset$: tập hợp rỗng, i.e., tập hợp không có phần tử nào.
\end{tcolorbox}
Các bài tập SGK \cite[\textbf{1}--\textbf{4}, pp. 7--8]{SGK_Toan_6_Canh_Dieu_tap_1} \& SBT \cite[Ví dụ 1, 2, p. 5; \textbf{1}--\textbf{8}, pp. 6--7]{SBT_Toan_6_Canh_Dieu_tap_1}.

\begin{baitoan}[\cite{Trong_Toan_6_2021}, \textbf{7.}, p. 6]
	Tập hợp $M$ gồm các chữ cái của từ ``THANG LONG''. Hãy viết tập $M$ bằng cách liệt kê các phần tử.
\end{baitoan}

\begin{baitoan}[\cite{Trong_Toan_6_2021}, \textbf{8.}, p. 6]
	Tập hợp $B$ gồm các chữ cái của từ ``NGOẠI NGỮ''. Hãy viết tập $B$ bằng cách liệt kê các phần tử.
\end{baitoan}

\begin{baitoan}[\cite{Trong_Toan_6_2021}, \textbf{9.}, p. 6]
	Tập hợp $A$ gồm các số tự nhiên nhỏ hơn $3$. Viết tập hợp $A$ bằng cách liệt kê các phần tử.
\end{baitoan}

\begin{baitoan}[\cite{Trong_Toan_6_2021}, \textbf{10.}, p. 6]
	Tập hợp $E$ gồm các số chẵn nhỏ hơn $5$. Viết tập hợp $E$ bằng cách liệt kê các phần tử.
\end{baitoan}

\begin{baitoan}[\cite{Trong_Toan_6_2021}, \textbf{11.}, p. 6]
	Tập hợp $H$ gồm các số lẻ nhỏ hơn $8$. Viết tập hợp $H$ bằng cách liệt kê các phần tử.
\end{baitoan}

\begin{baitoan}[\cite{Trong_Toan_6_2021}, \textbf{12.}, p. 7]
	Tập hợp $C$ gồm các số tự nhiên nhỏ hơn hoặc bằng $4$. Viết tập hợp $C$ bằng cách liệt kê các phần tử.
\end{baitoan}

\begin{baitoan}[\cite{Trong_Toan_6_2021}, \textbf{13.}, p. 7]
	Tập hợp $E$ gồm các số tự nhiên không vượt quá $11$. Viết tập hợp $E$ bằng cách liệt kê các phần tử.
\end{baitoan}

\begin{baitoan}[\cite{Trong_Toan_6_2021}, \textbf{14.}, p. 7]
	Tập hợp $C$ gồm các số tự nhiên lớn hơn $1$ \& nhỏ hơn $5$. Viết tập hợp $C$ bằng cách liệt kê các phần tử.
\end{baitoan}

\begin{baitoan}[\cite{Trong_Toan_6_2021}, \textbf{15.}, p. 7]
	Tập hợp $D$ gồm các số tự nhiên lớn hơn hoặc bằng $6$ \& nhỏ hơn $12$. Viết tập hợp $D$ bằng cách liệt kê các phần tử.
\end{baitoan}

\begin{baitoan}[\cite{Trong_Toan_6_2021}, \textbf{16.}, p. 7]
	Tập hợp $E$ gồm các số tự nhiên lớn hơn $4$ \& nhỏ hơn $9$. Viết tập hợp $E$ bằng cách liệt kê các phần tử.
\end{baitoan}

\begin{baitoan}[\cite{Trong_Toan_6_2021}, \textbf{17.}, p. 7]
	Tập hợp $A$ gồm các số tự nhiên nhỏ hơn $3$. Viết tập hợp $A$ bằng cách chỉ ra các tính chất đặc trưng cho các phần tử của tập hợp.
\end{baitoan}

\begin{baitoan}[\cite{Trong_Toan_6_2021}, \textbf{18.}, p. 7]
	Tập hợp $B$ gồm các số tự nhiên nhỏ hơn $8$. Viết tập hợp $A$ bằng cách chỉ ra các tính chất đặc trưng cho các phần tử của tập hợp.
\end{baitoan}

\begin{baitoan}[\cite{Trong_Toan_6_2021}, \textbf{19.}, p. 7]
	Tập hợp $C$ gồm các số tự nhiên lớn hơn $11$. Viết tập hợp $C$ bằng cách chỉ ra các tính chất đặc trưng cho các phần tử của tập hợp.
\end{baitoan}

\begin{baitoan}[\cite{Trong_Toan_6_2021}, \textbf{20.}, p. 7]
	Tập hợp $A$ gồm các số tự nhiên lớn hơn hoặc bằng $8$. Viết tập hợp $A$ bằng cách chỉ ra các tính chất đặc trưng cho các phần tử của tập hợp.
\end{baitoan}

\begin{baitoan}[\cite{Trong_Toan_6_2021}, \textbf{21.}, p. 7]
	Tập hợp $B$ gồm các số tự nhiên lớn hơn $7$ \& nhỏ hơn $17$. Viết tập hợp $B$ bằng cách chỉ ra các tính chất đặc trưng cho các phần tử của tập hợp.
\end{baitoan}

\begin{baitoan}[\cite{Trong_Toan_6_2021}, \textbf{22.}, p. 7]
	Tập hợp $C$ gồm các số tự nhiên lớn hơn hoặc bằng $7$ \& nhỏ hơn $14$. Viết tập hợp $C$ bằng cách chỉ ra các tính chất đặc trưng cho các phần tử của tập hợp.
\end{baitoan}

\begin{baitoan}[\cite{Trong_Toan_6_2021}, \textbf{23.}, p. 7]
	Tập hợp $A$ gồm các số tự nhiên khác $0$ \& nhỏ hơn hoặc bằng $5$. Viết tập hợp $A$ bằng cách chỉ ra các tính chất đặc trưng cho các phần tử của tập hợp.
\end{baitoan}

\begin{baitoan}[\cite{Trong_Toan_6_2021}, \textbf{24.}, p. 7]
	Cho $A$ là tập hợp các số tự nhiên nhỏ hơn $5$. Viết tập hợp $A$ bằng 2 cách:
	\begin{enumerate*}
		\item Liệt kê các phần tử.
		\item Chỉ ra tính chất đặc trưng cho các phần tử của tập hợp.
	\end{enumerate*}
\end{baitoan}

\begin{baitoan}[\cite{Trong_Toan_6_2021}, \textbf{25.}, p. 7]
	Cho $A$ là tập hợp các số tự nhiên lớn hơn $4$ \& nhỏ hơn $8$. Viết tập hợp $A$ bằng 2 cách:
	\begin{enumerate*}
		\item Liệt kê các phần tử.
		\item Chỉ ra tính chất đặc trưng cho các phần tử của tập hợp.
	\end{enumerate*}
\end{baitoan}

\begin{baitoan}[\cite{Trong_Toan_6_2021}, \textbf{26.}, p. 7]
	Tìm tập hợp $B$ gồm các số tự nhiên lớn hơn hoặc bằng $5$ \& nhỏ hơn hoặc bằng $6$ rồi viết tập hợp $B$ bằng 2 cách: liệt kê các phần tử \& nêu tính chất đặc trưng của các phần tử.
\end{baitoan}

\begin{baitoan}[\cite{Trong_Toan_6_2021}, \textbf{27.}, p. 7]
	Viết tập hợp $K$ những người sống trên mặt trăng.
\end{baitoan}

\begin{baitoan}[\cite{Trong_Toan_6_2021}, \textbf{28.}, p. 8]
	$A$ là tập hợp các số tự nhiên không quá $4$.
	\begin{enumerate*}
		\item Viết tập hợp $A$ bằng cách liệt kê \& cách chỉ ra tính chất đặc trưng của các phần tử.
		\item Điền vào chỗ trống dùng ký hiệu $\in,\notin$: $4\square A$, $3\square A$, $0\square A$, $6\square A$, $1\square A$, $\frac{1}{2}\square A$.
	\end{enumerate*}
\end{baitoan}

\begin{baitoan}[\cite{Trong_Toan_6_2021}, \textbf{29.}, p. 8]
	Viết tập hợp $C$ các số tự nhiên lớn hơn $5$ \& nhỏ hơn $6$ bằng 2 cách.
\end{baitoan}

\begin{baitoan}[\cite{Trong_Toan_6_2021}, \textbf{30.}, p. 8]
	Cho $A$ là tập hợp các số tự nhiên nhỏ hơn $7$ \& $B$ là tập hợp các số tự nhiên chẵn nhỏ hơn $8$.
	\begin{enumerate*}
		\item Viết các tập $A$ \& $B$ bằng cách liệt kê phần tử.
		\item Điền vào ô trống dùng các ký hiệu: $\subset,\in,\notin$: $B\square A$, $5\square B$, $6\square A$, $7\square B$, $6\square B$, $4\square A$, $4\square B$, $5\square A$, $0\square A$, $0\square B$.
	\end{enumerate*}
\end{baitoan}

\begin{baitoan}[\cite{Trong_Toan_6_2021}, \textbf{31.}, p. 8]
	 Cho $A$ là tập hợp các số tự nhiên nhỏ hơn $4$.
	 \begin{enumerate*}
	 	\item Viết tập $A$ bằng 2 cách.
	 	\item Xét tính đúng sai của các cách viết sau: $0\in A$, $1\notin A$, $4\in A$, $3\in A$, $5\notin A$, $2\in A$.
	 	\item Điền vào ô trống dùng ký hiệu $\in,\notin$: $3\square A$, $5\square A$, $4\square A$, $0\square A$, $1\square A$, $2\square A$.
	 \end{enumerate*}
\end{baitoan}
\noindent\textsc{Kiến thức cần nhớ.}
\begin{tcolorbox}
	\begin{enumerate}
		\item \textbf{Số phần tử của tập hợp.} 1 tập hợp có thể có 1 phần tử, có nhiều phần tử, có vô số phần tử hoặc không có phần tử nào.
		\item \textbf{Tập hợp con.} Nếu mọi phần tử của tập hợp $A$ đều thuộc tập hợp $B$ thì tập hợp $A$ là \textit{tập hợp con} của tập hợp $B$. Ký hiệu: $A\subset B$ hay $B\supset A$.
		\item \textbf{Tập hợp bằng nhau.} Nếu các phần tử của tập hợp $A$ \& tập hợp $B$ giống nhau thì tập hợp $A$ bằng tập hợp $B$.
		
		\begin{luuy}
			Tập hợp rỗng $\emptyset$ là tập hợp con của mọi tập hợp. Nếu $A\subset B$ \& $B\supset A$ thì $A = B$. Mỗi tập hợp đều là tập hợp con của chính nó, i.e., $A\subset A$ với mọi tập hợp $A$.
		\end{luuy}
	\end{enumerate}
\end{tcolorbox}

\begin{baitoan}[\cite{Trong_Toan_6_2021}, \textbf{32.}, p. 8]
	Cho tập hợp $A = \{1;3\}$.
	\begin{enumerate*}
		\item Viết các tập hợp con của tập hợp $A$ sao cho mỗi tập hợp con đó có đúng 1 phần tử.
		\item Viết các tập hợp con của tập hợp $A$ sao cho mỗi tập hợp con đó có đúng 2 phần tử.
		\item Viết tất cả các tập hợp con của tập hợp $A$.
	\end{enumerate*}
\end{baitoan}

\begin{baitoan}[\cite{Trong_Toan_6_2021}, \textbf{33.}, p. 9]
	Cho tập hợp $A = \{3;4;5\}$.
	\begin{enumerate*}
		\item Viết các tập hợp con của tập hợp $A$ sao cho mỗi tập hợp con đó có đúng 1 phần tử.
		\item Viết các tập hợp con của tập hợp $A$ sao cho mỗi tập hợp con đó có đúng 2 phần tử.
		\item Viết tất cả các tập hợp con của tập hợp $A$.
	\end{enumerate*}
\end{baitoan}
	
\begin{baitoan}[\cite{Trong_Toan_6_2021}, \textbf{34.}, p. 9]
	Cho tập hợp $B = \{a;b;c\}$. Viết tất cả các tập hợp con của tập hợp $B$.		
\end{baitoan}

\begin{baitoan}[\cite{Trong_Toan_6_2021}, \textbf{35.}, p. 9]
	Cho $A$ là tập hợp các số tự nhiên nhỏ hơn $8$ \& $B$ là tập hợp các số tự nhiên nhỏ hơn $5$.
	\begin{enumerate*}
		\item Hãy viết các tập hợp $A$ \& $B$ bằng cách liệt kê các phần tử.
		\item Dùng ký hiệu $\subset$ để thể hiện quan hệ giữa 2 tập hợp $A$ \& $B$.
	\end{enumerate*}
\end{baitoan}

\begin{baitoan}[\cite{Trong_Toan_6_2021}, \textbf{36.}, p. 9]
	Cho 2 tập hợp $A = \{x\in\mathbb{N}|x < 7\}$; $B = \{x\in\mathbb{N};x < 6\}$.
	\begin{enumerate*}
		\item Viết các tập hợp $A$ \& $B$ bằng cách liệt kê các phần tử \& cho biết số phần tử của mỗi tập hợp.
		\item Dùng ký hiệu $\subset$ để thể hiện quan hệ giữa 2 tập hợp $A$ \& $B$.
	\end{enumerate*} 
\end{baitoan}

\begin{baitoan}[\cite{Trong_Toan_6_2021}, \textbf{37.}, p. 9]
	Cho 2 tập hợp $C = \{x\in\mathbb{N}^\star|x < 6\}$; $D = \{x\in\mathbb{N}^\star|x < 9\}$.
	\begin{enumerate*}
		\item Viết các tập hợp $C$ \& $D$ bằng cách liệt kê các phần tử \& cho biết số phần tử của mỗi tập hợp.
		\item Dùng ký hiệu $\subset$ để thể hiện quan hệ giữa 2 tập hợp $C$ \& $D$.
	\end{enumerate*} 
\end{baitoan}

\begin{baitoan}[\cite{Trong_Toan_6_2021}, \textbf{38.}, p. 9]
	Cho $A$ là tập hợp các số tự nhiên nhỏ hơn $8$, $B$ là tập hợp các số tự nhiên lẻ nhỏ hơn $7$.
	\begin{enumerate*}
		\item Viết tập hợp $A$ \& $B$ bằng cách liệt kê các phần tử.
		\item Viết các tập con của $B$.
		\item Dùng các ký hiệu đã học điền vào ô trống. $1\square A$, $2\square B$, $0\square A$, $\{1;3\}\square B$, $B\square A$, $\{0;1\}\in A$.
	\end{enumerate*}
\end{baitoan}

\begin{baitoan}[\cite{Trong_Toan_6_2021}, \textbf{39.}, p. 9]
	$A$ là tập hợp các số tự nhiên khác $0$ \& nhỏ hơn $7$.
	\begin{enumerate*}
		\item Viết tập $A$ bằng 2 cách: Liệt kê các phần tử. Nêu tính chất đặc trưng của các phần tử.
		\item Viết các tập con của $A$ sao cho mỗi tập con đó có đúng 2 phần tử.
	\end{enumerate*}
\end{baitoan}

\begin{baitoan}[\cite{Trong_Toan_6_2021}, \textbf{40.}, p. 9]
	$A$ là tập hợp các số tự nhiên lớn hơn $5$ \& nhỏ hơn $9$.
	\begin{enumerate*}
		\item Viết tập $A$ bằng 2 cách: Liệt kê các phần tử. Nêu tính chất đặc trưng của các phần tử.
		\item Tìm các tập con của $A$.
		\item Điền vào ô trống: $1\square A$, $5\square A$, $7\square A$, $\{6,7\}\square A$, $\{0,1,2\}\square A$.
	\end{enumerate*}
\end{baitoan}

\begin{baitoan}[\cite{Trong_Toan_6_2021}, \textbf{41.}, p. 10]
	Cho $A = \{1;2;3;4;5;6\}$, $B = \{x\in\mathbb{N}^\star|x\le 5\}$.
	\begin{enumerate*}
		\item Viết tập hợp $A$ bằng cách nêu các tính chất chung của các phần tử \& viết tập $B$ bằng cách liệt kê các phần tử.
		\item Dùng ký hiệu để biểu thị sự quan hệ giữa $A$ \& $B$.
	\end{enumerate*}
\end{baitoan}

\begin{baitoan}[\cite{Trong_Toan_6_2021}, \textbf{42.}, p. 10]
	Cho $A = \{x\in\mathbb{N}|30 < x < 50,\ x\ \vdots\ 5\}$, $B = \{x\in\mathbb{N}|30 < x < 50,\ x\ \vdots\ 2\}$.
	\begin{enumerate*}
		\item Viết các tập hợp $A,B$ bằng cách liệt kê các phần tử.
		\item Tìm các tập con của $A$.
	\end{enumerate*}
\end{baitoan}

\begin{baitoan}[\cite{Trong_Toan_6_2021}, \textbf{43.}, p. 10]
	Cho $A = \{x\in\mathbb{N}|x\le 4\}$, $B = \{x\in\mathbb{N}|x < 7\}$. Liệt kê các phần tử của tập hợp $A$ \& $B$.
\end{baitoan}

\begin{baitoan}[\cite{Trong_Toan_6_2021}, \textbf{44.}, p. 10]
	Cho $A = \{x\in\mathbb{N}|30 < x < 50,\ x\ \vdots\ 5\}$, $B = \{x\in\mathbb{N}|30 < x < 50,\ x\ \vdots\ 2\}$.	
	\begin{enumerate*}
		\item Viết các tập hợp $A,B$ bằng cách liệt kê các phần tử.
		\item Tìm các tập con của $A$.
	\end{enumerate*}
\end{baitoan}

\begin{baitoan}[\cite{Trong_Toan_6_2021}, \textbf{45.}, p. 10]
	Cho $A = \{x\in\mathbb{N}|20\le x < 40,\ x\ \vdots\ 3\}$, $B = \{x\in\mathbb{N}|30\le x\le 40,\ x\ \vdots\ 5\}$, $C = \{x\in\mathbb{N}|30\le x\le 40,\ x\ \vdots\ 4\}$. Viết các tập hợp $A,B,C$ bằng cách liệt kê.
\end{baitoan}
\noindent\textsc{Kiến thức cần nhớ.}
\begin{tcolorbox}
	Công thức tính số phần tử của tập hợp là các dãy số đặc biệt:
	\begin{align*}
		\mbox{số phần tử} = \frac{\mbox{số lớn nhất} - \mbox{số bé nhất}}{\mbox{khoảng cách giữa 2 số liên tiếp}} + 1.
	\end{align*}
\end{tcolorbox}

\begin{baitoan}[\cite{Trong_Toan_6_2021}, \textbf{46.}, p. 10]
	Cho tập hợp $A = \{1;3;5;\ldots;39\}$. Tính số phần tử của tập hợp $A$.
\end{baitoan}

\begin{baitoan}[\cite{Trong_Toan_6_2021}, \textbf{47.}, p. 10]
	Cho $E = \{5;10;15;20;\ldots;195\}$. Tính số phần tử của tập hợp $E$.
\end{baitoan}

\begin{baitoan}[\cite{Trong_Toan_6_2021}, \textbf{48.}, p. 10]
	Cho $E = \{3;5;7;9;\ldots;113;115\}$. Tính số phần tử của tập hợp $F$.
\end{baitoan}

\begin{baitoan}[\cite{Trong_Toan_6_2021}, \textbf{49.}, p. 10]
	Để đánh số trang của cuốn sách dày $98$ trang người ta dùng tất cả bao nhiêu chữ số?
\end{baitoan}

\begin{baitoan}[\cite{Trong_Toan_6_2021}, \textbf{50.}, p. 10]
	Để đánh số trang của cuốn sách dày $150$ trang ta cần dùng bao nhiêu chữ số?
\end{baitoan}

\begin{baitoan}[\cite{Trong_Toan_6_2021}, \textbf{51.}, p. 10]
	Người ta dùng $1002$ chữ số để đánh số trang 1 cuốn sách từ 1 đến hết. Hỏi cuốn sách đó dày nhiêu trang?
\end{baitoan}

\begin{baitoan}[\cite{Trong_Toan_6_2021}, \textbf{52.}, p. 10]
	Để đánh số trang 1 quyển sách người ta dùng hết $831$ chữ số. Hỏi quyển sách đó có bao nhiêu trang?
\end{baitoan}

\subsection{Tập hợp các số tự nhiên. Cộng, trừ, nhân, chia số tự nhiên}
\textsc{Kiến thức cần nhớ.}
\begin{tcolorbox}
	Cách ghi số tự nhiên trong hệ thập phân:
	\begin{enumerate*}
		\item[(a)] Trong hệ thập phân, mỗi số tự nhiên được viết dưới dạng 1 dãy những số lấy trong 10 chữ số $0,1,2,3,4,5,6,7,8$, \& $9$; vị trí của các chữ số trong dãy gọi là hàng.
		\item[(b)] Cứ 10 đơn vị ở 1 hàng thì bằng 1 đơn vị của hàng liền trước nó. E.g., $10$ chục thì bằng 1 trăm; 10 trăm thì bằng 1 nghìn; $\ldots$
	\end{enumerate*}
	Trong tập hợp số tự nhiên, số liền sau hơn số liền trước 1 đơn vị.
\end{tcolorbox}
Các bài tập SGK \cite[\textbf{1}--\textbf{4}, pp. 7--8]{SGK_Toan_6_Canh_Dieu_tap_1} \& SBT \cite[Ví dụ 1--3, pp. 7--8; \textbf{9}--\textbf{14}, pp. 8--9]{SBT_Toan_6_Canh_Dieu_tap_1}.

\begin{baitoan}[\cite{Trong_Toan_6_2021}, \textbf{1.}, p. 11]
	Trong các khẳng định sau, khẳng định nào là đúng, khẳng định nào là sai?
	\begin{enumerate*}
		\item[(a)] $1999 > 2003$;
		\item[(b)] $100000$ là số tự nhiên nhỏ lớn nhất;
		\item[(c)] $5\le 5$;
		\item[(d)] Số $1$ là số tự nhiên nhỏ nhất.
	\end{enumerate*}
\end{baitoan}

\begin{baitoan}[\cite{Trong_Toan_6_2021}, \textbf{2.}, p. 11]
	Thay mỗi chữ cái dưới đây bằng 1 số tự nhiên phù hợp trong những trường hợp sau:
	\begin{enumerate*}
		\item[(a)] $17,a,b$ là 3 số lẻ liên tiếp tăng dần.
		\item[(b)] $m,101,n,p$ là 4 số tự nhiên liên tiếp giảm dần.
	\end{enumerate*}
\end{baitoan}

\begin{baitoan}[\cite{Trong_Toan_6_2021}, \textbf{3.}, p. 11]
	\begin{enumerate*}
		\item[(a)] Viết số tự nhiên nhỏ nhất có 4 chữ số;
		\item[(b)] Viết số tự nhiên nhỏ nhất có 4 chữ số khác nhau;
		\item[(c)] Viết số tự nhiên nhỏ nhất có 4 chữ số khác nhau \& đều là số chẵn;
		\item[(d)] Viết số tự nhiên nhỏ nhất có 4 chữ số khác nhau \& đều là số lẻ.
	\end{enumerate*}
\end{baitoan}

\begin{baitoan}[\cite{Trong_Toan_6_2021}, \textbf{5.}, p. 11]
	Dùng các chữ số $0,3$, \& $5$ viết 1 số tự nhiên có 3 chữ số khác nhau mà chữ số $5$ có giá trị là $50$.
\end{baitoan}

\begin{baitoan}[\cite{Trong_Toan_6_2021}, \textbf{6.}, p. 11]
	\emph{Số chẵn} là số tự nhiên có chữ số tận cùng là $0,2,4,6,8$; \emph{số lẻ} là số tự nhiên có chữ số tận cùng là $1,3,5,7,9$. 2 số chẵn (hoặc lẻ) \emph{liên tiếp} thì hơn kém nhau $2$ đơn vị.
	\begin{enumerate*}
		\item[(a)] Viết tập hợp $A$ các số chẵn nhỏ hơn $15$.
		\item[(b)] Viết tập hợp $B$ các số lẻ lớn hơn $5$ nhưng nhỏ hơn $17$.
		\item[(c)] Viết tập hợp $C$ 3 số chẵn liên tiếp, trong đó số lớn nhất là $46$.
	\end{enumerate*}
\end{baitoan}

\begin{baitoan}[\cite{Trong_Toan_6_2021}, \textbf{9.}, p. 12]
	Trong 1 cửa hàng bánh kẹo, người ta đóng gói kẹo thành các loại: mỗi gói có $10$ cái kẹo; mỗi hộp có $10$ gói; mỗi thùng có $10$ hộp. 1 người mua $9$ thùng, $9$ hộp \& $9$ gói kẹo. Hỏi người đó đã mua tất cả bao nhiêu cái kẹo?
\end{baitoan}
\noindent\textsc{Kiến thức cần nhớ.}
\begin{tcolorbox}
	Mỗi số tự nhiên viết trong hệ thập phân đều biểu diễn được thành \textit{tổng giá trị các chữ số của nó}. 
\end{tcolorbox}

\begin{baitoan}[\cite{Trong_Toan_6_2021}, \textbf{14.}, \textbf{15.}, p. 12]
	Viết các số sau dưới dạng tổng giá trị các chữ số của nó:
	\begin{align*}
		\overline{5at},\overline{ab},\overline{xyz},\overline{a5b},\overline{xyzt},\overline{xt5z},\overline{a2yb3}.
	\end{align*}
\end{baitoan}
\noindent\textsc{Kiến thức cần nhớ.}
\begin{tcolorbox}
	``Ngoài cách ghi số trong hệ thập phân gồm các chữ số từ $0$ đến $9$ \& các hàng (đơn vị, chục, trăm, nghìn, $\ldots$) như trên, còn có cách ghi số La Mã như sau: $\rm I = 1,V = 5,X = 10$. Mỗi chữ số La Mã có giá trị không phụ thuộc vào vị trí của nó trong số La Mã. Mỗi số La Mã biểu diễn 1 số tự nhiên bằng tổng giá trị của các thành phần viết nên số đó. Không có số La Mã nào biểu diễn số $0$.'' -- \cite[p. 13]{Trong_Toan_6_2021}
\end{tcolorbox}

\begin{baitoan}[\cite{Trong_Toan_6_2021}, \textbf{16.}, p. 13]
	Viết giá trị tương ứng trong hệ thập phân của các số La Mã: $\rm XIV,XVI,XXIII$.
\end{baitoan}

\begin{baitoan}[\cite{Trong_Toan_6_2021}, \textbf{17.}, p. 13]
	Viết các số sau bằng số La Mã: $18$, $25$.
\end{baitoan}

\begin{baitoan}[\cite{Trong_Toan_6_2021}, \textbf{18.}, p. 13]
	Sắp xếp theo thứ tự từ lớn đến bé: $\rm I,VII,IX,XI,V,IV,II,XVIII$.
\end{baitoan}
\noindent\textsc{Kiến thức cần nhớ.}
\begin{tcolorbox}
	``Đối với biểu thức có phép toán cộng, trừ, nhân, chia, ta thực hiện phép tính nhân, chia trước, cộng, trừ sau.'' -- \cite[p. 13]{Trong_Toan_6_2021}. ``Phép cộng \& phép nhân có tính chất giao hoán \& kết hợp: Tính chất giao hoán: $a + b = b + a$, $ab = ba$. Tính chất kết hợp: $(a + b) + c = a + (b + c)$, $(ab)c = a(bc)$.'' -- \cite[p. 14]{Trong_Toan_6_2021}
	
	``Tính chất phân phối của phép nhân đối với phép cộng: Muốn nhân 1 số với 1 tổng, ta lấy số đó nhân với từng số hạng của tổng, i.e., $a(b + c) = ab + ac$. Tính chất cộng với số $0$, nhân với số $1$: $a + 0 = a$ \& $a\cdot 1 = a$. Ngược với phép nhân phân phối là lấy thừa số chung.'' -- \cite[p. 14]{Trong_Toan_6_2021}
	
	``Muốn tính biểu thức 1 cách hợp lý, ta sử dụng tính chất giao hoán, kết hợp để xuất hiện các phép tính có kết quả tròn chuc, tròn trăm, tròn nghìn, $\ldots$\footnote{I.e., làm xuất hiện $a\cdot 10^n$ với $a,n\in\mathbb{N}^\star$.}'' -- \cite[p. 15]{Trong_Toan_6_2021}
\end{tcolorbox}

\begin{baitoan}[\cite{Trong_Toan_6_2021}, \textbf{26.}, p. 15]
	Tính hợp lý:
	\begin{enumerate*}
		\item[(a)] $1 + 7 + 9$;
		\item[(b)] $2 + 5 + 8$;
		\item[(c)] $11 + 2 + 8 + 9$;
		\item[(d)] $5\cdot 3\cdot 4$;
		\item[(e)] $2\cdot 3\cdot 50$;
		\item[(f)] $9\cdot 6 + 9\cdot 4$;
		\item[(g)] $2\cdot 8 + 2\cdot 12$;
		\item[(h)] $4\cdot 7 + 4 \cdot 13$;
		\item[(i)] $7\cdot 3 + 7\cdot 17$;
		\item[(j)] $11\cdot 13 + 37\cdot 11$.
	\end{enumerate*}
\end{baitoan}

\begin{baitoan}[\cite{Trong_Toan_6_2021}, \textbf{27.}, p. 15]
	Tính nhanh:
	\begin{enumerate*}
		\item[(a)] $46 + 17 + 54$;
		\item[(b)] $87\cdot 36 + 87\cdot 64$.
	\end{enumerate*}
\end{baitoan}

\begin{baitoan}[\cite{Trong_Toan_6_2021}, \textbf{28.}, p. 15]
	Áp dụng các tính chất của phép cộng \& phép nhân để tính nhanh:
	\begin{enumerate*}
		\item[(a)] $86 + 357 + 14$;
		\item[(b)] $772 + 69 + 128$;
		\item[(c)] $25\cdot 5\cdot 4\cdot 27\cdot 2$;
		\item[(d)] $28\cdot 64 + 28\cdot 36$.
	\end{enumerate*}
\end{baitoan}

\begin{baitoan}[\cite{Trong_Toan_6_2021}, \textbf{29.}, p. 15]
	Áp dụng các tính chất của phép cộng \& phép nhân để tính nhanh:
	\begin{enumerate*}
		\item[(a)] $25 + 39 + 21$;
		\item[(b)] $997 + 29 + 3 + 51$;
		\item[(c)] $578 + 125 + 422 + 375$;
		\item[(d)] $198 + 789 + 502 + 311$;
		\item[(e)] $158 + 445 + 342 + 555$;
		\item[(f)] $714 + 382 + 286 + 318$;
		\item[(g)] $15\cdot 6\cdot 4\cdot 125\cdot 8$;
		\item[(h)] $14\cdot 25\cdot 6\cdot 7$;
		\item[(i)] $24\cdot 3\cdot 5\cdot 10$;
		\item[(j)] $18\cdot 26\cdot 25\cdot 9$;
		\item[(k)] $25(187 + 18 + 1382)$;
		\item[(l)] $125\cdot 98\cdot 2\cdot 8\cdot 25$;
		\item[(m)] $1122\cdot 34 + 2244\cdot 83$;
		\item[(n)] $8466\cdot 15 + 170\cdot 4233$;
		\item[(o)] $1 + 2 + 3 + 4 + 5 + 6 + 7 + 8$;
		\item[(p)] $3 + 4 + 5 + 6 + 7 + 8 + 9 + 10 + 11$.
	\end{enumerate*}
\end{baitoan}

\begin{baitoan}[\cite{Trong_Toan_6_2021}, \textbf{30.}, p. 15]
	Tính nhanh:
	\begin{enumerate*}
		\item[(a)] $285 + 470 + 115 + 230$;
		\item[(b)] $571 + 216 + 129 + 124$.
	\end{enumerate*}
\end{baitoan}

\begin{baitoan}[\cite{Trong_Toan_6_2021}, \textbf{31.}, p. 15]
	Tìm các tích bằng nhau mà không cần tính kết quả của mỗi tích: $15\cdot 2\cdot 6$, $4\cdot 4\cdot 9$, $5\cdot 3\cdot 12$, $15\cdot 3\cdot 4$, $8\cdot 2\cdot 9$.
\end{baitoan}

\begin{baitoan}[\cite{Trong_Toan_6_2021}, \textbf{32.}, p. 15]
	Tính nhanh:
	\begin{enumerate*}
		\item[(a)] $13\cdot 58\cdot 4 + 32\cdot 26\cdot 2 + 52\cdot 10$;
		\item[(b)] $15\cdot 37\cdot 4 + 120\cdot 21 + 21\cdot 5\cdot 12$;
		\item[(c)] $14\cdot 35\cdot 5 + 10\cdot 25\cdot 7 + 20\cdot 70$;
		\item[(d)] $15(27 + 18 + 6) + 15(23 + 12)$;
		\item[(e)] $24(15 + 49) + 12(50 + 42)$;
		\item[(f)] $10(81 + 19) + 100 + 50(91 + 9)$;
		\item[(g)] $53(51 + 4) + 53(49 + 96) + 53$;
		\item[(h)] $42(15 + 96) + 6(25 + 4)\cdot 7$;
		\item[(i)] $45(13 + 78) + 9(87 + 22)\cdot 5$;
		\item[(j)] $16(27 + 75) + 8(53 + 25)\cdot 2$.
	\end{enumerate*}
\end{baitoan}
\noindent\textsc{Kiến thức cần nhớ.}
\begin{tcolorbox}
	``Muốn tìm số hạng chưa biết, ta lấy tổng trừ đi số hạng đã biết. Muốn tìm số bị trừ, ta lấy hiệu cộng với số trừ. Muốn tìm số trừ, ta lấy số bị trừ trừ đi hiệu. Muốn tìm thừa số chưa biết, ta lấy tích chia cho thừa số đã biết. Muốn tìm số bị chia, ta lấy thương nhân với số chia. Muốn tìm số chia, ta lấy số bị chia chia cho thương.'' -- \cite[p. 16]{Trong_Toan_6_2021}
	
	
\end{tcolorbox}

\subsection{Lũy thừa của 1 số tự nhiên}

\subsection{Thứ tự thực hiện phép tính}

%------------------------------------------------------------------------------%

\section{Tính Chất Chia Hết Trong Tập Hợp Các Số Tự Nhiên}

\subsection{Dấu hiệu chia hết}

\subsection{Tính chất chia hết của 1 tổng, 1 hiệu}

\subsection{Ước \& bội}

\subsection{Số nguyên tố. Hợp số}

\subsection{Ước chung \& bội chung}

\subsection{Ước chung lớn nhất}

\subsection{Bội chung nhỏ nhất}

%------------------------------------------------------------------------------%

\section{Số Nguyên}

\subsection{Tập hợp các số nguyên}

\subsection{Phép cộng \& phép trừ số nguyên}

\subsection{Quy tắc dấu ngoặc}

\subsection{Quy tắc chuyển vế}

\subsection{Phép nhân \& phép chia hết 2 số nguyên}

%------------------------------------------------------------------------------%

\newpage
\section{Hình Học Trực Quan}

\subsection{Tam giác đều -- hình vuông -- lục giác đều}

\subsection{Hình chữ nhật -- hình thoi -- hình bình hành -- hình thang cân}

\subsection{Chu vi \& diện tích của 1 số tứ giác đã học}

%------------------------------------------------------------------------------%

\section{Tính Đối Xứng của Hình Phẳng Tự Nhiên}

\subsection{Hình có trục đối xứng}

\subsection{Hình có tâm đối xứng}

%------------------------------------------------------------------------------%

\newpage
\section{Phân Số}

\subsection{Mở rộng khái niệm phân số}

\subsection{Phân số bằng nhau}

\subsection{Tính chất cơ bản của phân số}

\subsection{So sánh phân số}

\subsection{Phép cộng \& trừ phân số}

\subsection{Phép nhân \& chia phân số}

\subsection{Hỗn số}

\subsection{Tìm giá trị phân số của 1 số cho trước}

\subsection{Tìm 1 số biết giá trị 1 phân số của nó}

%------------------------------------------------------------------------------%

\section{Số Thập Phân}

\subsection{Số thập phân. Phần trăm}

\subsection{Tính toán với số thập phân}

\subsection{Làm tròn số thập phân \& ước lượng kết quả}

\subsection{Tỷ số \& tỷ số phần trăm}

\subsection{2 bài toán về tỷ số phần trăm}

%------------------------------------------------------------------------------%

\newpage
\section{Những Hình Học Cơ Bản}

\subsection{Điểm \& đường thẳng}

\subsection{Điểm nằm giữa 2 điểm. Tia}

\subsection{Đoạn thẳng \& độ dài đoạn thẳng}

\subsection{Trung điểm của đoạn thẳng}

\subsection{Nửa mặt phẳng}

\subsection{Góc}

\subsection{Số đo góc}

%------------------------------------------------------------------------------%

\section{Xác Suất Thống Kê}

\subsection{Phép thử nghiệm -- Sự kiện}

\subsection{Thu thập \& phân loại dữ liệu}

\subsection{Biểu diễn dữ liệu trên bảng}

\subsection{Bảng thống kê \& biểu dồ tranh}

\subsection{Biểu đồ cột}

\subsection{Biểu đồ cột kép}

\subsection{Xác suất thực nghiệm}

\subsection{Hoạt động thực hành \& trải nghiệm}

%------------------------------------------------------------------------------%

%\selectlanguage{english}
%\begin{thebibliography}{99}
%	\bibitem[]{}
%\end{thebibliography}

%------------------------------------------------------------------------------%

\newpage
Tài liệu: \cite{SGK_Toan_6_Canh_Dieu_tap_1, SGK_Toan_6_Canh_Dieu_tap_2, SBT_Toan_6_Canh_Dieu_tap_1, Binh_Toan_6_tap_1, Binh_Toan_6_tap_2, Trong_Toan_6_2021}.

\printbibliography[heading=bibintoc]
	
\end{document}