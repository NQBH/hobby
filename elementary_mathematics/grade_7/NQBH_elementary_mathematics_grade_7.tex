\documentclass[oneside]{book}
\usepackage[backend=biber,natbib=true,style=authoryear]{biblatex}
\addbibresource{/home/hong/1_NQBH/reference/bib.bib}
\usepackage[utf8]{vietnam}
\usepackage{tocloft}
\renewcommand{\cftsecleader}{\cftdotfill{\cftdotsep}}
\usepackage[colorlinks=true,linkcolor=blue,urlcolor=red,citecolor=magenta]{hyperref}
\usepackage{amsmath,amssymb,amsthm,mathtools,float,graphicx,algpseudocode,algorithm,tcolorbox}
\usepackage[inline]{enumitem}
\allowdisplaybreaks
\numberwithin{equation}{section}
\newtheorem{assumption}{Assumption}[section]
\newtheorem{conjecture}{Conjecture}[section]
\newtheorem{corollary}{Corollary}[section]
\newtheorem{hequa}{Hệ quả}[section]
\newtheorem{definition}{Definition}[section]
\newtheorem{dinhnghia}{Định nghĩa}[section]
\newtheorem{example}{Example}[section]
\newtheorem{vidu}{Ví dụ}[section]
\newtheorem{lemma}{Lemma}[section]
\newtheorem{notation}{Notation}[section]
\newtheorem{principle}{Principle}[section]
\newtheorem{problem}{Problem}[section]
\newtheorem{baitoan}{Bài toán}[section]
\newtheorem{proposition}{Proposition}[section]
\newtheorem{menhde}{Mệnh đề}[section]
\newtheorem{question}{Question}[section]
\newtheorem{cauhoi}{Câu hỏi}[section]
\newtheorem{remark}{Remark}[section]
\newtheorem{luuy}{Lưu ý}[section]
\newtheorem{theorem}{Theorem}[section]
\newtheorem{dinhly}{Định lý}[section]
\usepackage[left=0.5in,right=0.5in,top=1.5cm,bottom=1.5cm]{geometry}
\usepackage{fancyhdr}
\pagestyle{fancy}
\fancyhf{}
\lhead{\small \textsc{Sect.} ~\thesection}
\rhead{\small \nouppercase{\leftmark}}
\renewcommand{\sectionmark}[1]{\markboth{#1}{}}
\cfoot{\thepage}
\def\labelitemii{$\circ$}

\title{Some Topics in Elementary Mathematics\texttt{/}Grade 7}
\author{Nguyễn Quản Bá Hồng\footnote{Independent Researcher, Ben Tre City, Vietnam\\e-mail: \texttt{nguyenquanbahong@gmail.com}; website: \url{https://nqbh.github.io}.}}
\date{\today}

\begin{document}
\maketitle
\setcounter{secnumdepth}{4}
\setcounter{tocdepth}{3}
\tableofcontents
\newpage

%------------------------------------------------------------------------------%

\chapter{Số Hữu Tỷ -- Rational Number\texttt{/}Rational}

\begin{quotation}
	\textbf{Nội dung.} \textit{Tập hợp các số hữu tỷ $\mathbb{Q}$; các phép tính trong tập hợp các số hữu tỷ; thứ tự thực hiện các phép tính; quy tắc chuyển vế \& quy tắc dấu ngoặc; biểu diễn thập phân của số hữu tỷ}.
\end{quotation}

\section{Tập Hợp $\mathbb{Q}$ Các Số Hữu Tỷ -- Set $\mathbb{Q}$ of Rationals}
Ký hiệu $\mathbb{Q}$ được lấy từ chữ cái đầu \textit{Q} của từ \textit{quotient}\footnote{\textbf{quotient} [n] \textbf{1.} (in compounds) a degree or amount of a particular quality or characteristic; \textbf{2.} (\textit{mathematics}) a number which is the result when 1 number is divided by another.}, i.e., thương số.

\subsection{Số hữu tỷ}

\begin{dinhnghia}[Số hữu tỷ]
	\emph{Số hữu tỷ} là số viết được dưới dạng phân số $\frac{a}{b}$ với $a,b\in\mathbb{Z}$, $b\ne 0$. \emph{Tập hợp các số hữu tỷ} được ký hiệu là $\mathbb{Q}$, i.e., $\mathbb{Q}\coloneqq\left\{\frac{a}{b}|a,b\in\mathbb{Z},\ b\ne 0\right\}$.
\end{dinhnghia}
``Mỗi số nguyên là 1 số hữu tỷ.'' i.e., $n\in\mathbb{Z}\Rightarrow n\in\mathbb{Q}$ bởi vì $\mathbb{Z}\subset\mathbb{Q}$\footnote{Rộng hơn, $\mathbb{N}^\star\subset\mathbb{N}\subset\mathbb{Z}\subset\mathbb{Q}\subset\mathbb{R}\subset\mathbb{C}$ với $\mathbb{R}$ là \textit{tập hợp các số thực} sẽ đề cập ở Chương 2 chương trình Toán 7 (\& cả tài liệu này), \& $\mathbb{C}$ là \textit{tập hợp các số phức}, sẽ được học ở chương trình Toán 12, phần Đại số.}. Nhưng 1 số hữu tỷ bất kỳ chưa chắc là 1 số nguyên, i.e., $a\in\mathbb{Q}\not\Rightarrow a\in\mathbb{Z}$. ``Các phân số bằng nhau là các cách viết khác nhau của cùng 1 số hữu tỷ.'' -- \cite[p. 6]{SGK_Toan_7_Canh_Dieu_tap_1}

\subsection{Biểu diễn số hữu tỷ trên trục số}
``Tương tự như đối với số nguyên, ta có thể biểu diễn mọi số hữu tỷ trên trục số. Trên trục số, điểm biểu diễn số hữu tỷ $a\in\mathbb{Q}$ được gọi là điểm $a$. Do các phân số bằng nhau cùng biểu diễn 1 số hữu tỷ nên khi biểu diễn số hữu tỷ trên trục số, ta có thể chọn 1 trong những phân số đó để biểu diễn số hữu tỷ trên trục số. Thông thường, ta chọn phân số tối giản để biểu diễn số hữu tỷ đó.'' -- \cite[p. 6]{SGK_Toan_7_Canh_Dieu_tap_1}. Vì $-\frac{a}{b} = \frac{a}{-b} = \frac{-a}{b}$, $\forall a,b\in\mathbb{Z}$, $b\ne 0$ nên 3 điểm biểu diễn 3 phân số này trùng nhau.

\subsection{Số đối của 1 số hữu tỷ}

\begin{dinhnghia}[2 số hữu tỷ đối nhau]
	Trên trục số, 2 số hữu tỷ (phân biệt) có điểm biểu diễn nằm về 2 phía của điểm gốc $0$ \& cách đều điểm gốc $0$ được gọi là \emph{2 số đối nhau}. Số đối của số hữu tỷ $a\in\mathbb{Q}$, ký hiệu là $-a\in\mathbb{Q}$. Số đối của số $0$ là $0$.
\end{dinhnghia}
``Số đối của số $-a$ là số $a$, i.e., $-(-a) = a$.'' -- \cite[p. 8]{SGK_Toan_7_Canh_Dieu_tap_1}

\subsection{So sánh các số hữu tỷ}

\subsubsection{So sánh 2 số hữu tỷ}
``Cũng như số nguyên, trong 2 số hữu tỷ khác nhau luôn có 1 số nhỏ hơn số kia. Nếu số hữu tỷ $a\in\mathbb{Q}$ nhỏ hơn số hữu tỷ $b\in\mathbb{Q}$ thì ta viết $a < b$ hay $b > a$. Số hữu tỷ lớn hơn $0$ gọi là \emph{số hữu tỷ dương}. Số hữu tỷ nhỏ hơn $0$ gọi là \emph{số hữu tỷ âm}. Số hữu tỷ $0$ không là số hữu tỷ dương, cũng không là số hữu tỷ âm. Nếu $a < b$ \& $b < c$ thì $a < c$'' -- \cite[p. 8]{SGK_Toan_7_Canh_Dieu_tap_1} Tính chất cuối cùng được gọi là \textit{tính chất bắc cầu của thứ tự các số hữu tỷ}, được viết dưới dạng mệnh đề toán học bằng ký hiệu như sau: $((a < b)\land(b < c))\Rightarrow(a < c)$, $\forall a,b,c\in\mathbb{Q}$.

\subsubsection{Cách so sánh 2 số hữu tỷ}
``Ở lớp 6, ta đã biết cách so sánh 2 phân số \& cách so sánh 2 số thập phân.'' ``Khi 2 số hữu tỷ cùng là phân số hoặc cùng là số thập phân, ta so sánh chúng theo những quy tắc đã biết ở lớp 6. Ngoài 2 trường hợp trên, để so sánh 2 số hữu tỷ, ta viết chúng về cùng dạng phân số (hoặc cùng dạng số thập phân) rồi so sánh chúng.'' -- \cite[p. 9]{SGK_Toan_7_Canh_Dieu_tap_1}

\subsection{Minh họa trên trục số}
``Giả sử 2 điểm $x,y$ lần lượt biểu diễn 2 số hữu tỷ $x,y$ trên trục số nằm ngang. Khi so sánh 2 số hữu tỷ, ta viết chúng ở dạng phân số có cùng mẫu số dương rồi so sánh 2 tử số, tức là so sánh 2 số nguyên. Vì vậy, cũng như số nguyên, nếu $x < y$ hay $y > x$ thì điểm $x$ nằm bên trái điểm $y$. Tương tự, nếu $x < y$ hay $y > x$ thì điểm $x$ nằm phía dưới điểm $y$ trên trục số thẳng đứng.'' -- \cite[pp. 9--10]{SGK_Toan_7_Canh_Dieu_tap_1}

\section{Cộng, Trừ, Nhân, Chia Số Hữu Tỷ -- Addition, Subtraction, Multiplication, Division on Rationals}

\subsection{Cộng, trừ 2 số hữu tỷ. Quy tắc chuyển vế}

\subsubsection{Quy tắc cộng, trừ 2 số hữu tỷ}
``Vì mọi số hữu tỷ đều viết được dưới dạng phân số nên ta có thể cộng, trừ 2 số hữu tỷ bằng cách viết chúng dưới dạng phân số rồi áp dụng quy tắc cộng, trừ phân số. Tuy nhiên, khi 2 số hữu tỷ cùng viết ở dạng số thập phân (với hữu hạn chữ số khác $0$ ở phần thập phân) thì ta có thể cộng, trừ 2 số đó theo quy tắc cộng, trừ số thập phân.'' -- \cite[p. 12]{SGK_Toan_7_Canh_Dieu_tap_1}

\subsubsection{Tính chất của phép cộng các số hữu tỷ}
``Giống như phép cộng các số nguyên, phép cộng các số hữu tỷ cũng có các tính chất: giao hoán, kết hợp, cộng với $0$, cộng với số đối. Ta có thể chuyển phép trừ cho 1 số hữu tỷ thành phép cộng với số đối của số hữu tỷ đó. Vì thế, trong 1 biểu thức số chỉ gồm các phép cộng \& phép trừ, ta có thể thay đổi tùy ý vị trí các số hạng kèm theo dấu của chúng.'' -- \cite[p. 13]{SGK_Toan_7_Canh_Dieu_tap_1}

\subsubsection{Quy tắc chuyển vế}
Ta có quy tắc ``chuyển vế'' đối với số hữu tỷ như sau:

\begin{menhde}
	Khi chuyển 1 số hạng từ vế này sang vế kia của 1 đẳng thức, ta phải đối dấu số hạng đó:
	\begin{align*}
		x + y = z\Rightarrow x = z - y,\ x - y = z\Rightarrow x = z + y,\ \forall x,y,z\in\mathbb{Q}.
	\end{align*}
\end{menhde}

\subsection{Nhân, chia 2 số hữu tỷ}

\subsubsection{Quy tắc nhân, chia 2 số hữu tỷ}
``Vì mọi số hữu tỷ đều viết được dưới dạng phân số nên ta có thể nhân, chia 2 số hữu tỷ bằng cách viết chúng dưới dạng phân số rồi áp dụng quy tắc nhân, chia phân số. Tuy nhiên, khi 2 số hữu tỷ cùng viết ở dạng số thập phân (với hữu hạn chữ số khác $0$ ở phần thập phân) thì ta có thể nhân, chia 2 số đó theo quy tắc nhân, chia số thập phân.'' -- \cite[p. 14]{SGK_Toan_7_Canh_Dieu_tap_1}

\subsubsection{Tính chất của phép nhân các số hữu tỷ}
Ký hiệu $\mathbb{Q}^\star\coloneqq\mathbb{Q}\backslash\{0\}$ là tập các số hữu tỷ khác $0$. ``Giống như phép nhân các số nguyên, phép nhân các số hữu tỷ cũng có các tính chất: giao hoán, kết hợp, nhân với số $1$, phân phối của phép nhân đối với phép cộng \& phép trừ.'' ``Mỗi số hữu tỷ $a$ khác $0$ (i.e., $a\in\mathbb{Q}^\star$) đều có số nghịch đảo sao cho tích của số đó với $a$ bằng $1$.'' ``\textit{Số nghịch đảo} của số hữu tỷ $a$ khác $0$ (i.e., $a\in\mathbb{Q}^\star$) ký hiệu là $\frac{1}{a}$, ta có $a\cdot\frac{1}{a} = 1$, $\forall a\in\mathbb{Q}^\star$. Số nghịch đảo của số hữu tỷ $\frac{1}{a}$ là $a$. Nếu $a,b\in\mathbb{Q}$ \& $b\ne 0$ thì $a:b = a\cdot\frac{1}{b}$.'' -- \cite[p. 15]{SGK_Toan_7_Canh_Dieu_tap_1}

\section{Phép Tính Lũy Thừa với Số Mũ Tự Nhiên của 1 Số Hữu Tỷ}

\subsection{Phép tính lũy thừa với số mũ tự nhiên}
Tương tự như đối với số tự nhiên, với số hữu tỷ ta cũng có:

\begin{dinhnghia}[Lũy thừa của số hữu tỷ]
	\emph{Lũy thừa bậc $n$} của 1 số hữu tỷ $x\in\mathbb{Q}$, ký hiệu $x^n$, là tích của $n$ thừa số $x$: $x^n = \underbrace{x\cdot x\cdots x}_{n\mbox{ \footnotesize thừa số } x}$ với $n\in\mathbb{N}^\star$. Số $x$ được gọi là \emph{cơ số}, $n$ được gọi là \emph{số mũ}. Quy ước: $x^1 = x$, $\forall x\in\mathbb{Q}$. 
\end{dinhnghia}
``$x^n$ đọc là ``$x$ mũ $n$'' hoặc ``$x$ lũy thừa $n$'' hoặc ``lũy thừa bậc $n$ của $x$''; $x^2$ còn được đọc là ``$x$ bình phương'' hay ``bình phương của $x$''; $x^3$ còn được đọc là ``$x$ lập phương'' hay ``lập phương của $x$''.'' -- \cite[p. 17]{SGK_Toan_7_Canh_Dieu_tap_1}. ``Để viết lũy thừa bậc $n$ của phân số $\frac{a}{b}$, ta phải viết $\frac{a}{b}$ trong dấu ngoặc $(\ )$, i.e., $\left(\frac{a}{b}\right)^n$.'' -- \cite[p. 18]{SGK_Toan_7_Canh_Dieu_tap_1}

\subsection{Tích \& thương của 2 lũy thừa cùng cơ số}
Cũng như lũy thừa với cơ số là số tự nhiên, đối với cơ số là số hữu tỷ, ta có các quy tắc sau:

\begin{dinhly}
	Khi nhân 2 lũy thừa cùng cơ số, ta giữ nguyên cơ số \& cộng các số mũ: $x^m\cdot x^n = x^{m + n}$, $\forall x\in\mathbb{Q},\,\forall m,n\in\mathbb{N}$. Khi chia 2 lũy thừa cùng cơ số (khác $0$), ta giữ nguyên cơ số \& lấy số mũ của lũy thừa bị chia trừ đi số mũ của lũy thừa chia: $x^m:x^n = \frac{x^m}{x^n} = x^{m - n}$, $\forall x\in\mathbb{Q}^\star,\,\forall m,n\in\mathbb{N},\,m\ge n$. Quy ước: $x^0 = 1$, $\forall x\in\mathbb{Q}^\star$.
\end{dinhly}

\subsection{Lũy thừa của 1 lũy thừa}
Đối với lũy thừa mà cơ số là số hữu tỷ, ta có:

\begin{dinhly}
	Khi tính lũy thừa của 1 lũy thừa, ta giữ nguyên cơ số \& nhân 2 số mũ: $(x^m)^n = x^{mn}$, $\forall x\in\mathbb{Q},\,\forall m,n\in\mathbb{N}$.
\end{dinhly}

\subsection{Lũy thừa của 1 tích, 1 thương}

\subsubsection{Lũy thừa của 1 tích}
Lũy thừa của 1 tích bằng tích các lũy thừa: $(xy)^n = x^ny^n$, $\forall x,y\in\mathbb{Q},\,\forall n\in\mathbb{N}$.

\subsubsection{Lũy thừa của 1 thương}
Lũy thừa của 1 thương bằng thương các lũy thừa: $\left(\dfrac{x}{y}\right)^n = \dfrac{x^n}{y^n}$, $\forall x,y\in\mathbb{Q},\,y\ne 0,\,\forall n\in\mathbb{N}$.

\section{Thứ Tự Thực Hiện Các Phép Tính. Quy Tắc Dấu Ngoặc}

\subsection{Thứ tự thực hiện các phép tính}
``Ở lớp 6, ta đã học thứ tự thực hiện các phép tính đối với số tự nhiên, số nguyên, phân số, số thập phân. Thứ tự thực hiện các phép tính đối với số hữu tỷ cũng tương tự thứ tự thực hiện các phép tính đối với các loại số trên.'' -- \cite[p. 23]{SGK_Toan_7_Canh_Dieu_tap_1}

\subsection{Quy tắc dấu ngoặc}
``Ở lớp 6, ta đã học quy tắc dấu ngoặc đối với số nguyên, phân số, số thập phân. Quy tắc dấu ngoặc đối với số hữu tỷ cũng tương tự quy tắc dấu ngoặc đối với các loại số trên.
\begin{itemize}
	\item Khi bỏ dấu ngoặc có dấu ``$+$'' đằng trước, ta giữ nguyên dấu của các số hạng trong dấu ngoặc.
	\begin{align*}
		a + (b + c) = a + b + c,\ a + (b - c) = a + b - c,\ \forall a,b,c\in\mathbb{Q}.
	\end{align*}
	\item Khi bỏ dấu ngoặc có dấu ``$-$'' đằng trước, ta phải đổi dấu của các số hạng trong dấu ngoặc: dấu ``$+$'' đổi thành dấu ``$-$'' \& dấu ``$-$'' đổi thành dấu ``$+$''.
	\begin{align*}
		a - (b + c) = a - b - c,\ a - (b - c) = a - b + c.
	\end{align*}
\end{itemize}
Nếu đưa các số hạng vào trong dấu ngoặc có dấu ``$-$'' đằng trước thì phải đối dấu các số hạng đó.'' -- \cite[p. 24]{SGK_Toan_7_Canh_Dieu_tap_1}

\section{Biểu Diễn Thập Phân của Số Hữu Tỷ}

\subsection{Số thập phân hữu hạn \& số thập phân vô hạn tuần hoàn}
``Các số thập phân chỉ gồm hữu hạn chữ số sau dấu ``,'' được gọi là \textit{số thập phân hữu hạn}.'' -- \cite[p. 27]{SGK_Toan_7_Canh_Dieu_tap_1} Các số thập phân vô hạn tuần hoàn có tính chất: Trong phần thập phân, bắt đầu từ 1 hàng nào đó, có \textit{1 chữ số} hay \textit{1 cụm chữ số liền nhau} xuất hiện liên tiếp mãi, \& chữ số đó hoặc cụm chữ số đó được gọi là \textit{chu kỳ} của số thập phân vô hạn tuần hoàn đó, e.g., $\overline{a_na_{n-1}\ldots a_1a_0,a_{-1}a_{-2}\ldots a_{-m}}(\overline{b_kb_{k-1}\ldots b_1b_0}) = \overline{a_na_{n-1}\ldots a_1a_0,a_{-1}a_{-2}\ldots a_{-m}b_kb_{k-1}\ldots b_1b_0b_kb_{k-1}\ldots}$.

\subsection{Biểu diễn thập phân của số hữu tỷ}
``Ta đã biết mỗi số hữu tỷ đều viết được dưới dạng phân số $\frac{a}{b}$ với $a,b\in\mathbb{Z}$, $b > 0$. Thực hiện phép tính $a:b$, ta có thể biểu diễn số hữu tỷ đó dưới dạng số thập phân hữu hạn hoặc vô hạn tuần hoàn. Mỗi số hữu tỷ được biểu diễn bởi 1 số thập phân hữu hạn hoặc vô hạn tuần hoàn.'' -- \cite[p. 28]{SGK_Toan_7_Canh_Dieu_tap_1}

\subsection{Dạng biểu diễn thập phân của số hữu tỷ}
``Ta đã biết mỗi số hữu tỷ được biểu diễn bởi 1 số thập phân hữu hạn hoặc vô hạn tuần hoàn. Vấn đề đặt ra là biểu diễn thập phân của số hữu tỷ khi nào là số thập phân hữu hạn? Khi nào là số thập phân vô hạn tuần hoàn? Giả sử số hữu tỷ $r$ viết được dưới dạng phân số tối giản $\frac{a}{b}$, $a,b\in\mathbb{Z}$, $b> 0$. Người ta đã chứng minh được định lý sau:
\begin{itemize}
	\item Các phân số tối giản với mẫu dương mà mẫu không có ước nguyên tố khác $2$ \& $5$ thì viết được dưới dạng số thập phân hữu hạn \& chỉ những phân số đó mới viết được dưới dạng số thập phân hữu hạn.
	\item Các phân số tối giản với mẫu dương \& mẫu có ước nguyên tố khác $2$ \& $5$ thì viết được dưới dạng số thập phân vô hạn tuần hoàn \& chỉ những phân số đó mới viết được dưới dạng số thập phân vô hạn tuần hoàn.
\end{itemize}
Từ định lý trên, ta có sơ đồ phân loại biểu diễn thập phân của số hữu tỷ như sau:

\fbox{Biểu diễn thập phân của số hữu tỷ $\frac{a}{b}$, $a,b\in\mathbb{Z}$, $\frac{a}{b}$ là phân số tối giản} bao gồm:
\begin{itemize}
	\item \fbox{Biểu diễn bằng số thập phân hữu hạn} $\leftrightarrow$ Mẫu $b$ không có ước nguyên tố khác $2$ \& $5$.
	\item \fbox{Biểu diễn bằng số thập phân vô hạn tuần hoàn} $\leftrightarrow$ Mẫu $b$ có ước nguyên tố khác $2$ \& $5$.'' -- \cite[p. 29]{SGK_Toan_7_Canh_Dieu_tap_1}
\end{itemize}

%------------------------------------------------------------------------------%

\chapter{Số Thực -- Real Number\texttt{/}Real}

\begin{quotation}
	\textbf{Nội dung.} \textit{Số vô tỷ; căn bậc 2 số học; tập hợp các số thực $\mathbb{R}$; giá trị tuyệt đối của 1 số thực; làm tròn \& ước lượng; tỷ lệ thức, dãy tỷ số bằng nhau; đại lượng tỷ lệ thuận, đại lượng tỷ lệ nghịch \& áp dụng vào bài toán thực tế}.
\end{quotation}

\section{Số Vô Tỷ. Căn Bậc 2 Số Học -- Irrational Number\texttt{/}Irrational. Square Root}
``Ngay từ thời xa xưa, phân số đã gắn bó với đời sống thực tiễn của con người trong suốt quá trình đo đạc, tính toán. Các nhà toán học Hy Lạp cổ đại thuộc trường phái Pythagoras còn cho rằng: ``Tất cả các hiện tượng trong vũ trụ có thể được thu gọn thành các số nguyên \& tỷ số của chúng''. Họ gọi các số nguyên \& tỷ số của chúng là số \textit{rational}, tức là những số \textit{có lý}, mà ngày nay chúng ta quen gọi là số hữu tỷ. Tuy nhiên, vào thế kỷ V trước Công nguyên, nhà toán học Hippasus (530--450 trước Công nguyên) đã phát hiện ra rằng có những đối tượng trong thế giới tự nhiên không biểu thị được qua số hữu tỷ, e.g., tỷ số giữa độ dài đường chéo hình vuông với cạnh của hình vuông đó thì không thể là số hữu tỷ. Phát minh của ông không được chấp nhận trong 1 thời gian dài, thậm chí những số như thế còn bị gọi là \textit{irrational}, tức là những số \textit{vô lý} hay \textit{không có lý}. (Nguồn: M. Kline, \textit{Mathematical Thought from Ancient to Modern Times}, Vol. 1, Oxford University Press, New York, 1990)''  -- \cite[p. 32]{SGK_Toan_7_Canh_Dieu_tap_1}

\section{Tập Hợp $\mathbb{R}$ Các Số Thực}

\section{Giá Trị Tuyệt Đối của 1 Số Thực}

\section{Làm Tròn \& Ước Lượng}

\section{Tỷ Lệ Thức}

\section{Dãy Tỷ Số Bằng Nhau}

\section{Đại Lượng Tỷ Lệ Thuận}

\section{Đại Lượng Tỷ Lệ Nghịch}

\section{Hoạt Động Thực Hành \& Trải Nghiệm: 1 Số Hình Thức Khuyến Mãi trong Kinh Doanh}

%------------------------------------------------------------------------------%

\chapter{Hình Học Trực Quan -- Visual Geometry}

\section{Hình Hộp Chữ Nhật. Hình Lập Phương}

\section{Hình Lăng Trụ Đứng Tam Giác. Hình Lăng Trụ Đứng Tứ Giác}

\section{Hoạt Động Thực Hành \& Trải Nghiệm: Tạo Đồ Dùng Dạng Hình Lăng Trụ Đứng}

%------------------------------------------------------------------------------%

\chapter{Góc. Đường Thẳng Song Song -- Angle\texttt{/}Paralleling Lines}

\section{Góc ở Vị Trí Đặc Biệt}

\section{Tia Phân Giác của 1 Góc}

\section{2 Đường Thẳng Song Song}

\section{Định Lý}

%------------------------------------------------------------------------------%

\chapter{1 Số Yếu Tố Thống Kê \& Xác Suất -- Some Ingredients in Statistics \& Probability}

\section{Thu Thập, Phân Loại \& Biểu Diễn Dữ Liệu}

\section{Phân Tích \& Xử Lý Dữ Liệu}

\section{Biểu Đồ Đoạn Thẳng}

\section{Biểu Đồ Hình Quạt Tròn}

\section{Biến Cố Trong 1 Số Trò Chơi Đơn Giản}

\section{Xác Suất của Biến Cố Ngẫu Nhiên trong 1 Số Trò Chơi Đơn Giản}

\section{Hoạt Động Thực Hành \& Trải Nghiệm: Dung Tích Phổi}

%------------------------------------------------------------------------------%

\chapter{Biểu Thức Đại Số -- Algebraic Expression}

\section{Biểu Thức Số. Biểu Thức Đại Số}

\section{Đa Thức 1 Biến. Nghiệm của Đa Thức 1 Biến}

\section{Phép Cộng, Phép Trừ Đa Thức 1 Biến}

\section{Phép Nhân Đa Thức 1 Biến}

\section{Phép Chia Đa Thức 1 Biến}

%------------------------------------------------------------------------------%

\chapter{Tam Giác -- Triangle}

\section{Tổng Các Góc của 1 Tam Giác}

\section{Quan Hệ Giữa Góc \& Cạnh Đối Diện. Bất Đẳng Thức Tam Giác}

\section{2 Tam Giác Bằng Nhau}

\section{Trường Hợp Bằng Nhau Thứ Nhất của Tam Giác: Cạnh -- Cạnh -- Cạnh}

\section{Trường Hợp Bằng Nhau Thứ 2 của Tam Giác: Cạnh -- Góc -- Cạnh}

\section{Trường Hợp Bằng Nhau Thứ 3 của Tam Giác: Góc -- Cạnh -- Góc}

\section{Tam Giác Cân}

\section{Đường Vuông Góc \& Đường Xiên}

\section{Đường Trung Trực của 1 Đoạn Thẳng}

\section{Tính Chất 3 Đường Trung Tuyến của Tam Giác}

\section{Tính Chất 3 Đường Phân Giác của Tam Giác}

\section{Tính Chất 3 Đường Trung Trực của Tam Giác}

\section{Tính Chất 3 Đường Cao của Tam Giác}

%------------------------------------------------------------------------------%

\begin{thebibliography}{99}
	\bibitem[NQBH\texttt{/}elementary math]{NQBH/elementary math} Nguyễn Quản Bá Hồng. \href{https://github.com/NQBH/hobby/blob/master/elementary_mathematics/some_topics_in_elementary_mathematics_problems_theories_applications_bridges_to_advanced_mathematics/NQBH_some_topics_in_elementary_mathematics_problems_theories_applications_bridges_to_advanced_mathematics.pdf}{\textit{Some Topics in Elementary Mathematics: Problems, Theories, Applications, \& Bridges to Advanced Mathematics}}. Mar 2022--now.
\end{thebibliography}

%------------------------------------------------------------------------------%

\printbibliography[heading=bibintoc]
	
\end{document}