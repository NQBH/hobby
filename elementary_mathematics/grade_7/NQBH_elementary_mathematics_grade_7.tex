\documentclass[oneside]{book}
\usepackage[backend=biber,natbib=true,style=authoryear]{biblatex}
\addbibresource{/home/hong/1_NQBH/reference/bib.bib}
\usepackage[utf8]{vietnam}
\usepackage{tocloft}
\renewcommand{\cftsecleader}{\cftdotfill{\cftdotsep}}
\usepackage[colorlinks=true,linkcolor=blue,urlcolor=red,citecolor=magenta]{hyperref}
\usepackage{amsmath,amssymb,amsthm,mathtools,float,graphicx,algpseudocode,algorithm,tcolorbox}
\usepackage[inline]{enumitem}
\allowdisplaybreaks
\numberwithin{equation}{section}
\newtheorem{assumption}{Assumption}[section]
\newtheorem{conjecture}{Conjecture}[section]
\newtheorem{corollary}{Corollary}[section]
\newtheorem{hequa}{Hệ quả}[section]
\newtheorem{definition}{Definition}[section]
\newtheorem{dinhnghia}{Định nghĩa}[section]
\newtheorem{example}{Example}[section]
\newtheorem{vidu}{Ví dụ}[section]
\newtheorem{lemma}{Lemma}[section]
\newtheorem{notation}{Notation}[section]
\newtheorem{principle}{Principle}[section]
\newtheorem{problem}{Problem}[section]
\newtheorem{baitoan}{Bài toán}[section]
\newtheorem{proposition}{Proposition}[section]
\newtheorem{question}{Question}[section]
\newtheorem{cauhoi}{Câu hỏi}[section]
\newtheorem{remark}{Remark}[section]
\newtheorem{luuy}{Lưu ý}[section]
\newtheorem{theorem}{Theorem}[section]
\newtheorem{dinhly}{Định lý}[section]
\usepackage[left=0.5in,right=0.5in,top=1.5cm,bottom=1.5cm]{geometry}
\usepackage{fancyhdr}
\pagestyle{fancy}
\fancyhf{}
\lhead{\small \textsc{Sect.} ~\thesection}
\rhead{\small \nouppercase{\leftmark}}
\renewcommand{\sectionmark}[1]{\markboth{#1}{}}
\cfoot{\thepage}
\def\labelitemii{$\circ$}

\title{Some Topics in Elementary Mathematics\texttt{/}Grade 7}
\author{Nguyễn Quản Bá Hồng\footnote{Independent Researcher, Ben Tre City, Vietnam\\e-mail: \texttt{nguyenquanbahong@gmail.com}; website: \url{https://nqbh.github.io}.}}
\date{\today}

\begin{document}
\maketitle
\setcounter{secnumdepth}{4}
\setcounter{tocdepth}{3}
\tableofcontents
\newpage

%------------------------------------------------------------------------------%

\chapter{Số Hữu Tỷ -- Rational Number\texttt{/}Rational}

\section{Tập Hợp $\mathbb{Q}$ Các Số Hữu Tỷ -- Set $\mathbb{Q}$ of Rationals}

\section{Cộng, Trừ, Nhân, Chia Số Hữu Tỷ}

\section{Phép Tính Lũy Thừa với Số Mũ Tự Nhiên của 1 Số Hữu Tỷ}

\section{Thứ Tự Thực Hiện Các Phép Tính. Quy Tắc Dấu Ngoặc}

\section{Biểu Diễn Thập Phân của Số Hữu Tỷ}

%------------------------------------------------------------------------------%

\chapter{Số Thực -- Real Number\texttt{/}Real}

\section{Số Vô Tỷ. Căn Bậc 2 Số Học}

\section{Tập Hợp $\mathbb{R}$ Các Số Thực}

\section{Giá Trị Tuyệt Đối của 1 Số Thực}

\section{Làm Tròn \& Ước Lượng}

\section{Tỷ Lệ Thức}

\section{Dãy Tỷ Số Bằng Nhau}

\section{Đại Lượng Tỷ Lệ Thuận}

\section{Đại Lượng Tỷ Lệ Nghịch}

\section{Hoạt Động Thực Hành \& Trải Nghiệm: 1 Số Hình Thức Khuyến Mãi trong Kinh Doanh}

%------------------------------------------------------------------------------%

\chapter{Hình Học Trực Quan -- Visual Geometry}

\section{Hình Hộp Chữ Nhật. Hình Lập Phương}

\section{Hình Lăng Trụ Đứng Tam Giác. Hình Lăng Trụ Đứng Tứ Giác}

\section{Hoạt Động Thực Hành \& Trải Nghiệm: Tạo Đồ Dùng Dạng Hình Lăng Trụ Đứng}

%------------------------------------------------------------------------------%

\chapter{Góc. Đường Thẳng Song Song -- Angle\texttt{/}Paralleling Lines}

\section{Góc ở Vị Trí Đặc Biệt}

\section{Tia Phân Giác của 1 Góc}

\section{2 Đường Thẳng Song Song}

\section{Định Lý}

%------------------------------------------------------------------------------%

\chapter{1 Số Yếu Tố Thống Kê \& Xác Suất -- Some Ingredients in Statistics \& Probability}

\section{Thu Thập, Phân Loại \& Biểu Diễn Dữ Liệu}

\section{Phân Tích \& Xử Lý Dữ Liệu}

\section{Biểu Đồ Đoạn Thẳng}

\section{Biểu Đồ Hình Quạt Tròn}

\section{Biến Cố Trong 1 Số Trò Chơi Đơn Giản}

\section{Xác Suất của Biến Cố Ngẫu Nhiên trong 1 Số Trò Chơi Đơn Giản}

\section{Hoạt Động Thực Hành \& Trải Nghiệm: Dung Tích Phổi}

%------------------------------------------------------------------------------%

\chapter{Biểu Thức Đại Số -- Algebraic Expression}

\section{Biểu Thức Số. Biểu Thức Đại Số}

\section{Đa Thức 1 Biến. Nghiệm của Đa Thức 1 Biến}

\section{Phép Cộng, Phép Trừ Đa Thức 1 Biến}

\section{Phép Nhân Đa Thức 1 Biến}

\section{Phép Chia Đa Thức 1 Biến}

%------------------------------------------------------------------------------%

\chapter{Tam Giác -- Triangle}

\section{Tổng Các Góc của 1 Tam Giác}

\section{Quan Hệ Giữa Góc \& Cạnh Đối Diện. Bất Đẳng Thức Tam Giác}

\section{2 Tam Giác Bằng Nhau}

\section{Trường Hợp Bằng Nhau Thứ Nhất của Tam Giác: Cạnh -- Cạnh -- Cạnh}

\section{Trường Hợp Bằng Nhau Thứ 2 của Tam Giác: Cạnh -- Góc -- Cạnh}

\section{Trường Hợp Bằng Nhau Thứ 3 của Tam Giác: Góc -- Cạnh -- Góc}

\section{Tam Giác Cân}

\section{Đường Vuông Góc \& Đường Xiên}

\section{Đường Trung Trực của 1 Đoạn Thẳng}

\section{Tính Chất 3 Đường Trung Tuyến của Tam Giác}

\section{Tính Chất 3 Đường Phân Giác của Tam Giác}

\section{Tính Chất 3 Đường Trung Trực của Tam Giác}

\section{Tính Chất 3 Đường Cao của Tam Giác}

%------------------------------------------------------------------------------%

\begin{thebibliography}{99}
	\bibitem[NQBH\texttt{/}elementary math]{NQBH/elementary math} Nguyễn Quản Bá Hồng. \href{https://github.com/NQBH/hobby/blob/master/elementary_math/NQBH_elementary_math.pdf}{\textit{Some Topics in Elementary Mathematics: Problems, Theories, Applications, \& Bridges to Advanced Mathematics}}. Mar 2022--now.
\end{thebibliography}

%------------------------------------------------------------------------------%

\printbibliography[heading=bibintoc]
	
\end{document}