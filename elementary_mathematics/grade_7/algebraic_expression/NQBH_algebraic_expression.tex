\documentclass{article}
\usepackage[backend=biber,natbib=true,style=authoryear]{biblatex}
\addbibresource{/home/nqbh/reference/bib.bib}
\usepackage[utf8]{vietnam}
\usepackage{tocloft}
\renewcommand{\cftsecleader}{\cftdotfill{\cftdotsep}}
\usepackage[colorlinks=true,linkcolor=blue,urlcolor=red,citecolor=magenta]{hyperref}
\usepackage{amsmath,amssymb,amsthm,mathtools,float,graphicx,algpseudocode,algorithm,tcolorbox,tikz,tkz-tab,subcaption}
\DeclareMathOperator{\arccot}{arccot}
\usepackage[inline]{enumitem}
\allowdisplaybreaks
\numberwithin{equation}{section}
\newtheorem{assumption}{Assumption}[section]
\newtheorem{nhanxet}{Nhận xét}[section]
\newtheorem{conjecture}{Conjecture}[section]
\newtheorem{corollary}{Corollary}[section]
\newtheorem{hequa}{Hệ quả}[section]
\newtheorem{definition}{Definition}[section]
\newtheorem{dinhnghia}{Định nghĩa}[section]
\newtheorem{example}{Example}[section]
\newtheorem{vidu}{Ví dụ}[section]
\newtheorem{lemma}{Lemma}[section]
\newtheorem{notation}{Notation}[section]
\newtheorem{principle}{Principle}[section]
\newtheorem{problem}{Problem}[section]
\newtheorem{baitoan}{Bài toán}[section]
\newtheorem{proposition}{Proposition}[section]
\newtheorem{menhde}{Mệnh đề}[section]
\newtheorem{question}{Question}[section]
\newtheorem{cauhoi}{Câu hỏi}[section]
\newtheorem{quytac}{Quy tắc}
\newtheorem{remark}{Remark}[section]
\newtheorem{luuy}{Lưu ý}[section]
\newtheorem{theorem}{Theorem}[section]
\newtheorem{tiende}{Tiên đề}[section]
\newtheorem{dinhly}{Định lý}[section]
\usepackage[left=0.5in,right=0.5in,top=1.5cm,bottom=1.5cm]{geometry}
\usepackage{fancyhdr}
\pagestyle{fancy}
\fancyhf{}
\lhead{\small Subsect.~\thesubsection}
\rhead{\small\nouppercase{\leftmark}}
\renewcommand{\subsectionmark}[1]{\markboth{#1}{}}
\cfoot{\thepage}
\def\labelitemii{$\circ$}

\title{Algebraic Expression -- Biểu Thức Đại Số}
\author{Nguyễn Quản Bá Hồng\footnote{Independent Researcher, Ben Tre City, Vietnam\\e-mail: \texttt{nguyenquanbahong@gmail.com}; website: \url{https://nqbh.github.io}.}}
\date{\today}

\begin{document}
\maketitle
\begin{abstract}
	\textsf{\textbf{Nội dung.} Biểu thức số, biểu thức đại số; đa thức 1 biến, nghiệm của đa thức 1 biến; phép cộng, phép trừ đa thức 1 biến; phép nhân đa thức 1 biến; phép chia đa thức 1 biến.}
\end{abstract}
\setcounter{secnumdepth}{4}
\setcounter{tocdepth}{3}
\tableofcontents
\newpage

%------------------------------------------------------------------------------%

\section{Biểu Thức Số. Biểu Thức Đại Số}

\begin{baitoan}[\cite{SGK_Toan_7_Canh_Dieu_tap_2}, \textbf{6.}, p. 46]
	Lãi suất ngân hành quy định cho kỳ hạn 1 năm là $r$\%\emph{\texttt{/}}năm. Viết biểu thức đại số biểu thị số tiền lãi \& tổng tiền gốc lẫn tiền lãi khi hết kỳ hạn 1 năm nếu gửi ngân hàng $A$ đồng.
\end{baitoan}

\begin{baitoan}[\cite{SGK_Toan_7_Canh_Dieu_tap_2}, \textbf{7.}, p. 46]
	Các nhà khoa học đã đưa ra cách ước tính chiều cao của trẻ em khi trưởng thành dựa trên chiều cao $b$ của bố \& chiều cao $m$ của mẹ ($b,m$ tính theo đơn vị cm) như sau: Chiều cao của con trai $= \frac{1}{2}\cdot1.08(b + m)$, Chiều cao của con gái $= \frac{1}{2}(0.923b + m)$. (a) Với chiều cao nào của bố, mẹ thì con trai cao hơn, bằng, thấp hơn con gái?
\end{baitoan}

%------------------------------------------------------------------------------%

\section{Đa Thức 1 Biến. Nghiệm của Đa Thức 1 Biến}

%------------------------------------------------------------------------------%

\section{Phép $\pm$ Đa Thức 1 Biến}

%------------------------------------------------------------------------------%

\section{Phép Nhân Đa Thức 1 Biến}

%------------------------------------------------------------------------------%

\section{Phép Chia Đa Thức 1 Biến}

%------------------------------------------------------------------------------%

\printbibliography[heading=bibintoc]
	
\end{document}