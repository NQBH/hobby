\documentclass{article}
\usepackage[backend=biber,natbib=true,style=authoryear,maxbibnames=50]{biblatex}
\addbibresource{/home/nqbh/reference/bib.bib}
\usepackage[utf8]{vietnam}
\usepackage{tocloft}
\renewcommand{\cftsecleader}{\cftdotfill{\cftdotsep}}
\usepackage[colorlinks=true,linkcolor=blue,urlcolor=red,citecolor=magenta]{hyperref}
\usepackage{amsmath,amssymb,amsthm,mathtools,float,graphicx,algpseudocode,algorithm,tcolorbox,tikz,tkz-tab,subcaption}
\DeclareMathOperator{\arccot}{arccot}
\usepackage[inline]{enumitem}
\allowdisplaybreaks
\numberwithin{equation}{section}
\newtheorem{assumption}{Assumption}[section]
\newtheorem{baitoan}{Bài toán}
\newtheorem{cauhoi}{Câu hỏi}[section]
\newtheorem{conjecture}{Conjecture}[section]
\newtheorem{corollary}{Corollary}[section]
\newtheorem{dangtoan}{Dạng toán}[section]
\newtheorem{definition}{Definition}[section]
\newtheorem{dinhly}{Định lý}[section]
\newtheorem{dinhnghia}{Định nghĩa}[section]
\newtheorem{example}{Example}[section]
\newtheorem{ghichu}{Ghi chú}[section]
\newtheorem{hequa}{Hệ quả}[section]
\newtheorem{hypothesis}{Hypothesis}[section]
\newtheorem{lemma}{Lemma}[section]
\newtheorem{luuy}{Lưu ý}[section]
\newtheorem{nhanxet}{Nhận xét}[section]
\newtheorem{notation}{Notation}[section]
\newtheorem{note}{Note}[section]
\newtheorem{principle}{Principle}[section]
\newtheorem{problem}{Problem}[section]
\newtheorem{proposition}{Proposition}[section]
\newtheorem{question}{Question}[section]
\newtheorem{remark}{Remark}[section]
\newtheorem{theorem}{Theorem}[section]
\newtheorem{vidu}{Ví dụ}[section]
\usepackage[left=0.5in,right=0.5in,top=1.5cm,bottom=1.5cm]{geometry}
\usepackage{fancyhdr}
\pagestyle{fancy}
\fancyhf{}
\lhead{\small Sect.~\thesection}
\rhead{\small\nouppercase{\leftmark}}
\renewcommand{\subsectionmark}[1]{\markboth{#1}{}}
\cfoot{\thepage}
\def\labelitemii{$\circ$}

\title{Algebraic Expression -- Biểu Thức Đại Số}
\author{Nguyễn Quản Bá Hồng\footnote{Independent Researcher, Ben Tre City, Vietnam\\e-mail: \texttt{nguyenquanbahong@gmail.com}; website: \url{https://nqbh.github.io}.}}
\date{\today}

\begin{document}
\maketitle
\begin{abstract}
	\textsf{\textbf{Nội dung.} Biểu thức số, biểu thức đại số; đa thức 1 biến, nghiệm của đa thức 1 biến; phép cộng, phép trừ đa thức 1 biến; phép nhân đa thức 1 biến; phép chia đa thức 1 biến.}
\end{abstract}
\setcounter{secnumdepth}{4}
\setcounter{tocdepth}{3}
\tableofcontents
\newpage

%------------------------------------------------------------------------------%

\section{Biểu Thức Số. Biểu Thức Đại Số -- Algebraic Expression}
``In mathematics, an \textit{algebraic expression} is an \href{https://en.wikipedia.org/wiki/Expression_(mathematics)}{expression} built up from constant \href{https://en.wikipedia.org/wiki/Algebraic_number}{algebraic numbers}, \href{https://en.wikipedia.org/wiki/Variable_(mathematics)}{variables}, \& the \href{https://en.wikipedia.org/wiki/Algebraic_operation}{algebraic operations} (addition, subtraction, multiplication, division, \& exponentiation by an exponent that is a rational number).'' -- \href{https://en.wikipedia.org/wiki/Algebraic_expression}{Wikipedia\texttt{/}algebraic expression}

``\fbox{\bf 1} Các số được nối với nhau bởi dấu của các phép tính tạo thành 1 biểu thức số. Đặc biệt mỗi số đều được coi là 1 biểu thức số. Khi thực hiện các phép tính trong 1 biểu thức số ta được 1 số. Số này gọi là \textit{giá trị} của biểu thức đã cho. \fbox{\bf 2} Biểu thức chỉ chứa chữ hoặc chứa cả số \& chữ gọi chung là \textit{biểu thức đại số}. Đặc biệt biểu thức số cũng được coi là biểu thức đại số. Trong 1 biểu thức đại số, các chữ (nếu có) dùng để thay thế hay đại diện cho những số nào đó được gọi là các \textit{biến số} (gọi tắt là các \textit{biến}). Khi thực hiện các phép tính trên các chữ ta có thể áp dụng những tính chất, quy tắc các phép tính như trên các số. \fbox{\bf 3} Để cho gọn: Ta không viết dấu nhân giữa các biến cũng như giữa số \& biến. Trong các biểu thức đại số như $\pm1x$, ta không viết thừa số 1. \fbox{\bf 4} Muốn tính giá trị của 1 biểu thức đại số tại những giá trị cho trước của biến ta thay giá trị đã cho của mỗi biến vào biểu thức rồi thực hiện các phép tính. \fbox{\bf 5} Quy ước đọc \& viết 1 biểu thức đại số có nhiều phép tính: Phép tính nào làm sau cùng thì đọc trước tiên. Phép tính nào làm trước tiên thì đọc sau.'' -- \cite[Chap. III, \S1, p. 37]{Tuyen_Toan_7}
\begin{table}[H]
	\centering
	\begin{tabular}{|c|l|l|}
		\hline
		\textbf{Biểu thức} & \textbf{Thứ tự thực hiện các phép tính} & \textbf{Cách đọc}\\
		\hline
		$(x + y)^2$ & Tính tổng $\to$ Tính bình phương & Bình phương của tổng 2 số $x,y$\\
		\hline
		$(x - y)^2$ & Tính hiệu $\to$ Tính bình phương & Bình phương của hiệu 2 số $x,y$\\
		\hline
		$(x + y)^3$ & Tính tổng $\to$ Tính lập phương & Lập phương của tổng 2 số $x,y$ \\
		\hline		
		$(x - y)^3$ & Tính hiệu $\to$ Tính lập phương & Lập phương của hiệu 2 số $x,y$ \\
		\hline
		$x^2 + y^2$ & Tính bình phương của $x,y$ $\to$ Tính tổng & Tổng các bình phương của 2 số $x,y$\\
		\hline
		$x^2 - y^2$ & Tính bình phương của $x,y$ $\to$ Tính hiệu & Hiệu các bình phương của 2 số $x,y$\\
		\hline
		$x^3 + y^3$ & Tính lập phương của $x,y$ $\to$ Tính tổng & Tổng các lập phương của 2 số $x,y$\\
		\hline
		$x^3 - y^3$ & Tính lập phương của $x,y$ $\to$ Tính hiệu & Hiệu các lập phương của 2 số $x,y$\\
		\hline
		$(x + y)(x - y)$ & Tính tổng \& hiệu $\to$ Tính tích & Tích của tổng 2 số $x,y$ với hiệu của chúng\\
		\hline
	\end{tabular}
\end{table}

\begin{baitoan}[\cite{SGK_Toan_7_Canh_Dieu_tap_2}, \textbf{6.}, p. 46]
	Lãi suất ngân hành quy định cho kỳ hạn 1 năm là $r$\%\emph{\texttt{/}}năm. Viết biểu thức đại số biểu thị số tiền lãi \& tổng tiền gốc lẫn tiền lãi khi hết kỳ hạn 1 năm nếu gửi ngân hàng $A$ đồng.
\end{baitoan}

\begin{baitoan}[\cite{SGK_Toan_7_Canh_Dieu_tap_2}, \textbf{7.}, p. 46]
	Các nhà khoa học đã đưa ra cách ước tính chiều cao của trẻ em khi trưởng thành dựa trên chiều cao $b$ của bố \& chiều cao $m$ của mẹ ($b,m$ tính theo đơn vị cm) như sau: Chiều cao của con trai $= \frac{1}{2}\cdot1.08(b + m)$, Chiều cao của con gái $= \frac{1}{2}(0.923b + m)$. (a) Với chiều cao nào của bố, mẹ thì con trai cao hơn, bằng, thấp hơn con gái?
\end{baitoan}

\begin{baitoan}[\cite{Tuyen_Toan_7}, Ví dụ 42, p. 37]
	Cho $y = 5x$, tính giá trị của biểu thức $A = \frac{4x + y}{6x - y}$.
\end{baitoan}

\begin{baitoan}[Mở rộng \cite{Tuyen_Toan_7}, Ví dụ 42, p. 37]
	\label{mo rong Tuyen_Toan_7 vi du 42}
	Cho $y = kx$, tính giá trị của biểu thức:
	\begin{align*}
		A &= \frac{ax + by}{cx + dy},\ B = \frac{ax^2 + bxy + cy^2}{dx^2 + exy + fy^2},\ C = \frac{a_1x^3 + a_2x^2y + a_3xy^2 + a_4y^3}{b_1x^3 + b_2x^2y + b_3xy^2 + b_4y^3},\\
		D &= \frac{\sum_{i=0}^n a_ix^{n-i}y^i}{\sum_{i=0}^n b_ix^{n-i}y^i} = \frac{a_1x^n + a_2x^{n-1}y + \cdots + a_{n-1}xy^{n-1} + a_ny^n}{b_1x^n + b_2x^{n-1}y + \cdots + b_{n-1}xy^{n-1} + b_ny^n},
	\end{align*}
	với $a,b,c,d,e,f,a_i,b_i\in\mathbb{R}$, $\forall i = 1,\ldots,n$.
\end{baitoan}

\begin{baitoan}[\cite{Tuyen_Toan_7}, Ví dụ 43, p. 37]
	Tính giá trị của biểu thức: $B = x^2 + 4xy - 3y^3$ với $|x| = 5$, $|y| = 1$.
\end{baitoan}

\begin{nhanxet}
	``Biểu thức $B$ có chứa 2 biến $x,y$. Biến $x$ nhận 2 giá trị, biến $y$ nhận 2 giá trị do đó ta phải xét đủ 4 trường hợp các cặp giá trị của $x,y$ dẫn đến biểu thức $B$ có 4 giá trị khác nhau.'' -- \cite[p. 38]{Tuyen_Toan_7}
\end{nhanxet}

\begin{baitoan}[\cite{Tuyen_Toan_7}, \textbf{150.}, p. 38]
	Cho $A$ là tổng lập phương các số tự nhiên từ $1$ đến $n$ \& $B$ là bình phương của tổng các số tự nhiên từ $1$ đến $n$. Người ta đã chứng minh được $A = B$. Kiểm nghiệm lại bằng cách cho $n = 1,2,3,4,5,6$.
\end{baitoan}

\begin{baitoan}[\cite{Tuyen_Toan_7}, \textbf{151.}, p. 38]
	Tính giá trị của các biểu thức sau với $x = \sqrt{2}$: (a) $(x + 1)(x^2 - 2)$; (b) $(x - 1)(x^2 + 1) + 3$.
\end{baitoan}

\begin{baitoan}[\cite{Tuyen_Toan_7}, \textbf{152.}, p. 38]
	Tính giá trị của biểu thức $M = \frac{2x^2 + 3x - 2}{x + 2}$ tại: (a) $x = -1$; (b) $|x| = 3$.
\end{baitoan}

\begin{baitoan}[\cite{Tuyen_Toan_7}, \textbf{153.}, p. 38]
	Tính giá trị của biểu thức $N = \frac{6x^2 + x - 3}{2x - 1}$ với $|x| = \frac{1}{2}$.
\end{baitoan}

\begin{baitoan}[\cite{Tuyen_Toan_7}, \textbf{154.}, p. 38]
	Tính giá trị của biểu thức $P = 9x^2 - 7x|y| - \frac{1}{4}y^3$ tại $x = \frac{1}{3}$, $y = -6$.
\end{baitoan}

\begin{baitoan}[\cite{Tuyen_Toan_7}, \textbf{155.}, p. 38]
	Tìm các giá trị của biến để: (a) Biểu thức $(x + 1)(y^2 - 6)$ có giá trị bằng $0$. (b) Biểu thức $x^2 - 12x + 7$ có giá trị lớn hơn $7$.
\end{baitoan}

\begin{baitoan}[\cite{Tuyen_Toan_7}, \textbf{156.}, p. 38]
	Tính giá trị của biểu thức $Q = \frac{5x^2 + 3y^2}{10x^2 - 3y^2}$ với $\frac{x}{3} = \frac{y}{5}$.
\end{baitoan}
Bài toán này là 1 trường hợp nhỏ của Bài toán \ref{mo rong Tuyen_Toan_7 vi du 42}: $B = \frac{ax^2 + bxy + cy^2}{dx^2 + exy + fy^2}$ khi $k = \frac{5}{3}, a = 5, b = 0, c = 3, d = 10, e = 0, f = -3$.

\begin{baitoan}[\cite{Tuyen_Toan_7}, \textbf{157.}, p. 38]
	Cho $x,y,z\in\mathbb{R}$, $x,y,z\ne0$, $x - y - z = 0$. Tính giá trị của biểu thức $M = \left(1 - \frac{z}{x}\right)\left(1 - \frac{x}{y}\right)\left(1 + \frac{y}{z}\right)$.
\end{baitoan}

\begin{baitoan}[\cite{Tuyen_Toan_7}, \textbf{158.}, p. 38]
	(a) Tìm GTNN của biểu thứcG: $A = (x + 2)^2 + \left(y - \frac{1}{5}\right)^2 - 10$. (b) Tìm GTLN của biểu thức: $B = \frac{4}{(2x - 3)^2 + 5}$.
\end{baitoan}

\begin{baitoan}[\cite{Tuyen_Toan_7}, \textbf{159.}, p. 38]
	Cho biểu thức $C = \frac{5 - x}{x - 2}$. Tìm giá trị nguyên của $x$ để: (a) $C$ có giá trị nguyên; (b) $C$ có giá trị nhỏ nhất.
\end{baitoan}

%------------------------------------------------------------------------------%

\section{Đa Thức 1 Biến. Nghiệm của Đa Thức 1 Biến}

\begin{baitoan}[\cite{SGK_Toan_7_Canh_Dieu_tap_2}, \textbf{3.}, p. 52--53]
	Cho 2 đa thức $P(y) = -12y^4 + 5y^4 + 13y^3 - 6y^3 + y -1 + 9$, $Q(y) = -20y^3 + 31y^3 + 6y - 8y + y - 7 + 11$. (a) Thu gọn mỗi đa thức trên rồi sắp xếp mỗi đa thức theo số mũ giảm dần của biến. (b) Tìm bậc, hệ số cao nhất \& hệ số tự do của mỗi đa thức đó.
\end{baitoan}

\begin{baitoan}[\cite{SGK_Toan_7_Canh_Dieu_tap_2}, \textbf{3.}, pp. 52--53]
	Cho đa thức $P(x) = ax^2 + bx + c$, $a\ne0$. Chứng tỏ: (a) $P(0) = c$; (b) $P(1) = a + b + c$; (c) $P(-1) = a - b + c$. (d) Tính $P(2),P(-2),P(3),P(-3),P\left(\frac{1}{2}\right),P\left(-\frac{1}{2}\right)$. (e) Tính $P(x) + P(-x)$ với $x\in\mathbb{R}$. (f) Tính $P(x) + P\left(\frac{1}{x}\right)$ với $x\in\mathbb{R}$.
\end{baitoan}

\begin{baitoan}[\cite{SGK_Toan_7_Canh_Dieu_tap_2}, \textbf{6.}, p. 53]
	Theo tiêu chuẩn của Tổ chức Y tế Thế Giới (WHO), đối với bé gái, công thức tính cân nặng chuẩn là $C = 9 + 2(N - 1)$ \emph{kg}, công thức tính chiều cao chuẩn là $H = 75 + 5(N - 1)$ \emph{cm}, trong đó $N$ là số tuổi của bé gái. (a) Tính cân nặng chuẩn, chiều cao chuẩn của 1 bé gái $3$ tuổi. (b) 1 bé gái $3$ tuổi nặng $13.5$\emph{kg} \& cao $86$\emph{cm}. Bé gái đó có đạt tiêu chuẩn về cân nặng \& chiều cao của Tổ chức Y tế Thế giới hay không?
\end{baitoan}

\begin{baitoan}[\cite{SGK_Toan_7_Canh_Dieu_tap_2}, \textbf{7.}, p. 52--53]
	Nhà bác học Galileo Galilei (1564--1642) là người đầu tiên phát hiện ra quãng đường chuyển động của vật rơi tự do tỷ lệ thuận với bình phương của thời gian chuyển động. Quan hệ giữa quãng đường chuyển động $y$ \emph{m} \& thời gian chuyển động $x$ \emph{s} được biểu diễn gần đúng bởi công thức $y = 5x^2$. Trong 1 thí nghiệm vật lý, người ta thả 1 vật nặng từ độ cao $180$\emph{m} xuống đất (coi sức cản của không khí không đáng kể). (a) Sau $3$\emph{s} thì vật nặng còn cách mặt đất bao nhiêu \emph{m}? (b) Khi vật nặng còn cách mặt đất $100$\emph{m} thì nó đã rơi được thời gian bao lâu? (c) Sau bao lâu thì vật chạm đất?
\end{baitoan}

\begin{baitoan}[\cite{SGK_Toan_7_Canh_Dieu_tap_2}, \textbf{8.}, p. 53]
	Pound là 1 đơn vị đo khối lượng truyền thống của Anh, Mỹ \& 1 số quốc gia khác. Công thức tính khối lượng $y$ \emph{kg} theo $x$ pound là $y = 0.45359237x$. (a) Tính giá trị của $y$ \emph{kg} khi $x = 100$ pound. (b) 1 hãng hàng không quốc tế quy định mỗi hành khác được mang 2 va li không tính cước; mỗi va li cân nặng không vượt quá $23$ \emph{kg}. Hỏi với va li cân nặng $50.99$ pound sau khi quy đổi sang kilogram \& được phép làm tròn đến hàng đơn vị thì có vượt quá quy định trên hay không?
\end{baitoan}

%------------------------------------------------------------------------------%

\section{Phép $\pm$ Đa Thức 1 Biến}

\begin{baitoan}
	Tính: (a) $ax^k + bx^k$, $\forall a,b\in\mathbb{R}$, $\forall k\in\mathbb{N}$. (b) $ax^k + bx^k + cx^k$, $\forall a,b,c\in\mathbb{R}$, $\forall k\in\mathbb{N}$. (c) $\sum_{i=1}^n a_ix^k = a_1x^k + a_2x^k + \cdots + a_nx^k$, $\forall a_i\in\mathbb{R}$, $\forall i = 1,\ldots,n$, $\forall k\in\mathbb{N}$.
\end{baitoan}

\begin{baitoan}[\cite{SGK_Toan_7_Canh_Dieu_tap_2}, Ví dụ 1, p. 55]
	Tính tổng của 2 đa thức: $P(x) = 5x^3 + 2x^2 + 3x + 1$ \& $Q(x) = 2x^3 - 4x^2 + 2x + 2$.
\end{baitoan}

\begin{baitoan}[\cite{SGK_Toan_7_Canh_Dieu_tap_2}, \textbf{1.}, p. 59]
	Cho 2 đa thức: $R(x) = -8x^4 + 6x^3 + 2x^2 - 5x + 1$ \& $S(x) = x^4 - 8x^3 + 2x + 3$. Tính: (a) $R(x) + S(x)$; (b) $R(x) - S(x)$.
\end{baitoan}

\begin{baitoan}[\cite{SGK_Toan_7_Canh_Dieu_tap_2}, \textbf{2.}, p. 59]
	Xác định bậc của 2 đa thức là tổng, hiệu của: $A(x) = -8x^5 + 6x^4 + 2x^2 - 5x + 1$ \& $B(x) = 8x^5 + 8x^3 + 2x - 3$.
\end{baitoan}

\begin{baitoan}[\cite{SGK_Toan_7_Canh_Dieu_tap_2}, \textbf{3.}, p. 59]
	Bác Ngọc gửi ngân hàng thứ nhất $90$ triệu đồng với kỳ hạn 1 năm, lãi suất $x$\%\emph{\texttt{/}}năm. Bác Ngọc gửi ngân hàng thứ 2 $80$ triệu đồng với kỳ hạn 1 năm, lãi suất $(x + 1.5)$\%\emph{\texttt{/}}năm. Hết kỳ hạn 1 năm, bác Ngọc có được cả gốc \& lãi là bao nhiêu: (a) Ở ngân hàng thứ 2? (b) Ở cả 2 ngân hàng?
\end{baitoan}

\begin{baitoan}[\cite{SGK_Toan_7_Canh_Dieu_tap_2}, \textbf{4.}, p. 59]
	Người ta rót nước từ 1 can đựng $10$ lít nước sang 1 bể rỗng có dạng hình lập phương với độ dài cạnh $20$\emph{cm}. Khi mực nước trong bể cao $h$ \emph{cm} thì thể tích nước trong can còn lại là bao nhiêu? Biết $1{\rm l} = 1{\rm dm}^3$.
\end{baitoan}

\begin{baitoan}[\cite{SGK_Toan_7_Canh_Dieu_tap_2}, \textbf{5.}, p. 59]
	Đ hay S? (a) Tổng của 2 đa thức bậc 4 luôn luôn là đa thức bậc 4. (b) Hiệu của 2 đa thức bậc 4 luôn luôn là đa thức bậc 4. (c) Tổng \& hiệu của 2 đa thức bậc $n\in\mathbb{N}$ luôn là đa thức bậc $n$.
\end{baitoan}

%------------------------------------------------------------------------------%

\section{Phép Nhân Đa Thức 1 Biến}

%------------------------------------------------------------------------------%

\section{Phép Chia Đa Thức 1 Biến}

%------------------------------------------------------------------------------%

\printbibliography[heading=bibintoc]
	
\end{document}