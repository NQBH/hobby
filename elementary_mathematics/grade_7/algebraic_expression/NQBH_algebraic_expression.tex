\documentclass{article}
\usepackage[backend=biber,natbib=true,style=authoryear,maxbibnames=50]{biblatex}
\addbibresource{/home/nqbh/reference/bib.bib}
\usepackage[utf8]{vietnam}
\usepackage{tocloft}
\renewcommand{\cftsecleader}{\cftdotfill{\cftdotsep}}
\usepackage[colorlinks=true,linkcolor=blue,urlcolor=red,citecolor=magenta]{hyperref}
\usepackage{amsmath,amssymb,amsthm,mathtools,float,graphicx,algpseudocode,algorithm,tcolorbox,tikz,tkz-tab,subcaption}
\DeclareMathOperator{\arccot}{arccot}
\usepackage[inline]{enumitem}
\allowdisplaybreaks
\numberwithin{equation}{section}
\newtheorem{assumption}{Assumption}[section]
\newtheorem{nhanxet}{Nhận xét}[section]
\newtheorem{conjecture}{Conjecture}[section]
\newtheorem{corollary}{Corollary}[section]
\newtheorem{hequa}{Hệ quả}[section]
\newtheorem{definition}{Definition}[section]
\newtheorem{dinhnghia}{Định nghĩa}[section]
\newtheorem{example}{Example}[section]
\newtheorem{vidu}{Ví dụ}[section]
\newtheorem{lemma}{Lemma}[section]
\newtheorem{notation}{Notation}[section]
\newtheorem{principle}{Principle}[section]
\newtheorem{problem}{Problem}[section]
\newtheorem{baitoan}{Bài toán}[section]
\newtheorem{proposition}{Proposition}[section]
\newtheorem{menhde}{Mệnh đề}[section]
\newtheorem{question}{Question}[section]
\newtheorem{cauhoi}{Câu hỏi}[section]
\newtheorem{quytac}{Quy tắc}
\newtheorem{remark}{Remark}[section]
\newtheorem{luuy}{Lưu ý}[section]
\newtheorem{theorem}{Theorem}[section]
\newtheorem{tiende}{Tiên đề}[section]
\newtheorem{dinhly}{Định lý}[section]
\usepackage[left=0.5in,right=0.5in,top=1.5cm,bottom=1.5cm]{geometry}
\usepackage{fancyhdr}
\pagestyle{fancy}
\fancyhf{}
\lhead{\small Subsect.~\thesubsection}
\rhead{\small\nouppercase{\leftmark}}
\renewcommand{\subsectionmark}[1]{\markboth{#1}{}}
\cfoot{\thepage}
\def\labelitemii{$\circ$}

\title{Algebraic Expression -- Biểu Thức Đại Số}
\author{Nguyễn Quản Bá Hồng\footnote{Independent Researcher, Ben Tre City, Vietnam\\e-mail: \texttt{nguyenquanbahong@gmail.com}; website: \url{https://nqbh.github.io}.}}
\date{\today}

\begin{document}
\maketitle
\begin{abstract}
	\textsf{\textbf{Nội dung.} Biểu thức số, biểu thức đại số; đa thức 1 biến, nghiệm của đa thức 1 biến; phép cộng, phép trừ đa thức 1 biến; phép nhân đa thức 1 biến; phép chia đa thức 1 biến.}
\end{abstract}
\setcounter{secnumdepth}{4}
\setcounter{tocdepth}{3}
\tableofcontents
\newpage

%------------------------------------------------------------------------------%

\section{Biểu Thức Số. Biểu Thức Đại Số}

\begin{baitoan}[\cite{SGK_Toan_7_Canh_Dieu_tap_2}, \textbf{6.}, p. 46]
	Lãi suất ngân hành quy định cho kỳ hạn 1 năm là $r$\%\emph{\texttt{/}}năm. Viết biểu thức đại số biểu thị số tiền lãi \& tổng tiền gốc lẫn tiền lãi khi hết kỳ hạn 1 năm nếu gửi ngân hàng $A$ đồng.
\end{baitoan}

\begin{baitoan}[\cite{SGK_Toan_7_Canh_Dieu_tap_2}, \textbf{7.}, p. 46]
	Các nhà khoa học đã đưa ra cách ước tính chiều cao của trẻ em khi trưởng thành dựa trên chiều cao $b$ của bố \& chiều cao $m$ của mẹ ($b,m$ tính theo đơn vị cm) như sau: Chiều cao của con trai $= \frac{1}{2}\cdot1.08(b + m)$, Chiều cao của con gái $= \frac{1}{2}(0.923b + m)$. (a) Với chiều cao nào của bố, mẹ thì con trai cao hơn, bằng, thấp hơn con gái?
\end{baitoan}

%------------------------------------------------------------------------------%

\section{Đa Thức 1 Biến. Nghiệm của Đa Thức 1 Biến}

\begin{baitoan}[\cite{SGK_Toan_7_Canh_Dieu_tap_2}, \textbf{3.}, p. 52--53]
	Cho 2 đa thức $P(y) = -12y^4 + 5y^4 + 13y^3 - 6y^3 + y -1 + 9$, $Q(y) = -20y^3 + 31y^3 + 6y - 8y + y - 7 + 11$. (a) Thu gọn mỗi đa thức trên rồi sắp xếp mỗi đa thức theo số mũ giảm dần của biến. (b) Tìm bậc, hệ số cao nhất \& hệ số tự do của mỗi đa thức đó.
\end{baitoan}

\begin{baitoan}[\cite{SGK_Toan_7_Canh_Dieu_tap_2}, \textbf{3.}, pp. 52--53]
	Cho đa thức $P(x) = ax^2 + bx + c$, $a\ne0$. Chứng tỏ: (a) $P(0) = c$; (b) $P(1) = a + b + c$; (c) $P(-1) = a - b + c$. (d) Tính $P(2),P(-2),P(3),P(-3),P\left(\frac{1}{2}\right),P\left(-\frac{1}{2}\right)$. (e) Tính $P(x) + P(-x)$ với $x\in\mathbb{R}$. (f) Tính $P(x) + P\left(\frac{1}{x}\right)$ với $x\in\mathbb{R}$.
\end{baitoan}

\begin{baitoan}[\cite{SGK_Toan_7_Canh_Dieu_tap_2}, \textbf{6.}, p. 53]
	Theo tiêu chuẩn của Tổ chức Y tế Thế Giới (WHO), đối với bé gái, công thức tính cân nặng chuẩn là $C = 9 + 2(N - 1)$ \emph{kg}, công thức tính chiều cao chuẩn là $H = 75 + 5(N - 1)$ \emph{cm}, trong đó $N$ là số tuổi của bé gái. (a) Tính cân nặng chuẩn, chiều cao chuẩn của 1 bé gái $3$ tuổi. (b) 1 bé gái $3$ tuổi nặng $13.5$\emph{kg} \& cao $86$\emph{cm}. Bé gái đó có đạt tiêu chuẩn về cân nặng \& chiều cao của Tổ chức Y tế Thế giới hay không?
\end{baitoan}

\begin{baitoan}[\cite{SGK_Toan_7_Canh_Dieu_tap_2}, \textbf{7.}, p. 52--53]
	Nhà bác học Galileo Galilei (1564--1642) là người đầu tiên phát hiện ra quãng đường chuyển động của vật rơi tự do tỷ lệ thuận với bình phương của thời gian chuyển động. Quan hệ giữa quãng đường chuyển động $y$ \emph{m} \& thời gian chuyển động $x$ \emph{s} được biểu diễn gần đúng bởi công thức $y = 5x^2$. Trong 1 thí nghiệm vật lý, người ta thả 1 vật nặng từ độ cao $180$\emph{m} xuống đất (coi sức cản của không khí không đáng kể). (a) Sau $3$\emph{s} thì vật nặng còn cách mặt đất bao nhiêu \emph{m}? (b) Khi vật nặng còn cách mặt đất $100$\emph{m} thì nó đã rơi được thời gian bao lâu? (c) Sau bao lâu thì vật chạm đất?
\end{baitoan}

\begin{baitoan}[\cite{SGK_Toan_7_Canh_Dieu_tap_2}, \textbf{8.}, p. 53]
	Pound là 1 đơn vị đo khối lượng truyền thống của Anh, Mỹ \& 1 số quốc gia khác. Công thức tính khối lượng $y$ \emph{kg} theo $x$ pound là $y = 0.45359237x$. (a) Tính giá trị của $y$ \emph{kg} khi $x = 100$ pound. (b) 1 hãng hàng không quốc tế quy định mỗi hành khác được mang 2 va li không tính cước; mỗi va li cân nặng không vượt quá $23$ \emph{kg}. Hỏi với va li cân nặng $50.99$ pound sau khi quy đổi sang kilogram \& được phép làm tròn đến hàng đơn vị thì có vượt quá quy định trên hay không?
\end{baitoan}

%------------------------------------------------------------------------------%

\section{Phép $\pm$ Đa Thức 1 Biến}

%------------------------------------------------------------------------------%

\section{Phép Nhân Đa Thức 1 Biến}

%------------------------------------------------------------------------------%

\section{Phép Chia Đa Thức 1 Biến}

%------------------------------------------------------------------------------%

\printbibliography[heading=bibintoc]
	
\end{document}