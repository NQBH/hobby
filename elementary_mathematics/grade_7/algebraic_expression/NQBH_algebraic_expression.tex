\documentclass{article}
\usepackage[backend=biber,natbib=true,style=authoryear,maxbibnames=50]{biblatex}
\addbibresource{/home/nqbh/reference/bib.bib}
\usepackage[utf8]{vietnam}
\usepackage{tocloft}
\renewcommand{\cftsecleader}{\cftdotfill{\cftdotsep}}
\usepackage[colorlinks=true,linkcolor=blue,urlcolor=red,citecolor=magenta]{hyperref}
\usepackage{amsmath,amssymb,amsthm,mathtools,float,graphicx,algpseudocode,algorithm,tcolorbox,tikz,tkz-tab,subcaption}
\DeclareMathOperator{\arccot}{arccot}
\usepackage[inline]{enumitem}
\allowdisplaybreaks
\numberwithin{equation}{section}
\newtheorem{assumption}{Assumption}[section]
\newtheorem{baitoan}{Bài toán}
\newtheorem{cauhoi}{Câu hỏi}[section]
\newtheorem{conjecture}{Conjecture}[section]
\newtheorem{corollary}{Corollary}[section]
\newtheorem{dangtoan}{Dạng toán}[section]
\newtheorem{definition}{Definition}[section]
\newtheorem{dinhly}{Định lý}[section]
\newtheorem{dinhnghia}{Định nghĩa}[section]
\newtheorem{example}{Example}[section]
\newtheorem{ghichu}{Ghi chú}[section]
\newtheorem{hequa}{Hệ quả}[section]
\newtheorem{hypothesis}{Hypothesis}[section]
\newtheorem{lemma}{Lemma}[section]
\newtheorem{luuy}{Lưu ý}[section]
\newtheorem{nhanxet}{Nhận xét}[section]
\newtheorem{notation}{Notation}[section]
\newtheorem{note}{Note}[section]
\newtheorem{principle}{Principle}[section]
\newtheorem{problem}{Problem}[section]
\newtheorem{proposition}{Proposition}[section]
\newtheorem{question}{Question}[section]
\newtheorem{remark}{Remark}[section]
\newtheorem{theorem}{Theorem}[section]
\newtheorem{vidu}{Ví dụ}[section]
\usepackage[left=0.5in,right=0.5in,top=1.5cm,bottom=1.5cm]{geometry}
\usepackage{fancyhdr}
\pagestyle{fancy}
\fancyhf{}
\lhead{\small Sect.~\thesection}
\rhead{\small\nouppercase{\leftmark}}
\renewcommand{\subsectionmark}[1]{\markboth{#1}{}}
\cfoot{\thepage}
\def\labelitemii{$\circ$}
\DeclareRobustCommand{\divby}{%
	\mathrel{\vbox{\baselineskip.65ex\lineskiplimit0pt\hbox{.}\hbox{.}\hbox{.}}}%
}

\title{Algebraic Expression -- Biểu Thức Đại Số}
\author{Nguyễn Quản Bá Hồng\footnote{Independent Researcher, Ben Tre City, Vietnam\\e-mail: \texttt{nguyenquanbahong@gmail.com}; website: \url{https://nqbh.github.io}.}}
\date{\today}

\begin{document}
\maketitle
\begin{abstract}
	\textsf{\textbf{Nội dung.} Biểu thức số, biểu thức đại số; đa thức 1 biến, nghiệm của đa thức 1 biến; phép cộng, phép trừ đa thức 1 biến; phép nhân đa thức 1 biến; phép chia đa thức 1 biến.}
\end{abstract}
\setcounter{secnumdepth}{4}
\setcounter{tocdepth}{3}
\tableofcontents
\newpage

%------------------------------------------------------------------------------%

\section{Biểu Thức Số. Biểu Thức Đại Số -- Algebraic Expression}
``In mathematics, an \textit{algebraic expression} is an \href{https://en.wikipedia.org/wiki/Expression_(mathematics)}{expression} built up from constant \href{https://en.wikipedia.org/wiki/Algebraic_number}{algebraic numbers}, \href{https://en.wikipedia.org/wiki/Variable_(mathematics)}{variables}, \& the \href{https://en.wikipedia.org/wiki/Algebraic_operation}{algebraic operations} (addition, subtraction, multiplication, division, \& exponentiation by an exponent that is a rational number).'' -- \href{https://en.wikipedia.org/wiki/Algebraic_expression}{Wikipedia\texttt{/}algebraic expression}

``\fbox{\bf 1} Các số được nối với nhau bởi dấu của các phép tính tạo thành 1 biểu thức số. Đặc biệt mỗi số đều được coi là 1 biểu thức số. Khi thực hiện các phép tính trong 1 biểu thức số ta được 1 số. Số này gọi là \textit{giá trị} của biểu thức đã cho. \fbox{\bf 2} Biểu thức chỉ chứa chữ hoặc chứa cả số \& chữ gọi chung là \textit{biểu thức đại số}. Đặc biệt biểu thức số cũng được coi là biểu thức đại số. Trong 1 biểu thức đại số, các chữ (nếu có) dùng để thay thế hay đại diện cho những số nào đó được gọi là các \textit{biến số} (gọi tắt là các \textit{biến}). Khi thực hiện các phép tính trên các chữ ta có thể áp dụng những tính chất, quy tắc các phép tính như trên các số. \fbox{\bf 3} Để cho gọn: Ta không viết dấu nhân giữa các biến cũng như giữa số \& biến. Trong các biểu thức đại số như $\pm1x$, ta không viết thừa số 1. \fbox{\bf 4} Muốn tính giá trị của 1 biểu thức đại số tại những giá trị cho trước của biến ta thay giá trị đã cho của mỗi biến vào biểu thức rồi thực hiện các phép tính. \fbox{\bf 5} Quy ước đọc \& viết 1 biểu thức đại số có nhiều phép tính: Phép tính nào làm sau cùng thì đọc trước tiên. Phép tính nào làm trước tiên thì đọc sau.'' -- \cite[Chap. III, \S1, p. 37]{Tuyen_Toan_7}
\begin{table}[H]
	\centering
	\begin{tabular}{|c|l|l|}
		\hline
		\textbf{Biểu thức} & \textbf{Thứ tự thực hiện các phép tính} & \textbf{Cách đọc}\\
		\hline
		$(x + y)^2$ & Tính tổng $\to$ Tính bình phương & Bình phương của tổng 2 số $x,y$\\
		\hline
		$(x - y)^2$ & Tính hiệu $\to$ Tính bình phương & Bình phương của hiệu 2 số $x,y$\\
		\hline
		$(x + y)^3$ & Tính tổng $\to$ Tính lập phương & Lập phương của tổng 2 số $x,y$ \\
		\hline		
		$(x - y)^3$ & Tính hiệu $\to$ Tính lập phương & Lập phương của hiệu 2 số $x,y$ \\
		\hline
		$x^2 + y^2$ & Tính bình phương của $x,y$ $\to$ Tính tổng & Tổng các bình phương của 2 số $x,y$\\
		\hline
		$x^2 - y^2$ & Tính bình phương của $x,y$ $\to$ Tính hiệu & Hiệu các bình phương của 2 số $x,y$\\
		\hline
		$x^3 + y^3$ & Tính lập phương của $x,y$ $\to$ Tính tổng & Tổng các lập phương của 2 số $x,y$\\
		\hline
		$x^3 - y^3$ & Tính lập phương của $x,y$ $\to$ Tính hiệu & Hiệu các lập phương của 2 số $x,y$\\
		\hline
		$(x + y)(x - y)$ & Tính tổng \& hiệu $\to$ Tính tích & Tích của tổng 2 số $x,y$ với hiệu của chúng\\
		\hline
	\end{tabular}
\end{table}

\begin{baitoan}[\cite{SGK_Toan_7_Canh_Dieu_tap_2}, \textbf{6.}, p. 46]
	Lãi suất ngân hành quy định cho kỳ hạn 1 năm là $r$\%\emph{\texttt{/}}năm. Viết biểu thức đại số biểu thị số tiền lãi \& tổng tiền gốc lẫn tiền lãi khi hết kỳ hạn 1 năm nếu gửi ngân hàng $A$ đồng.
\end{baitoan}

\begin{baitoan}[\cite{SGK_Toan_7_Canh_Dieu_tap_2}, \textbf{7.}, p. 46]
	Các nhà khoa học đã đưa ra cách ước tính chiều cao của trẻ em khi trưởng thành dựa trên chiều cao $b$ của bố \& chiều cao $m$ của mẹ ($b,m$ tính theo đơn vị cm) như sau: Chiều cao của con trai $= \frac{1}{2}\cdot1.08(b + m)$, Chiều cao của con gái $= \frac{1}{2}(0.923b + m)$. (a) Với chiều cao nào của bố, mẹ thì con trai cao hơn, bằng, thấp hơn con gái?
\end{baitoan}

\begin{baitoan}[\cite{Tuyen_Toan_7}, Ví dụ 42, p. 37]
	Cho $y = 5x$, tính giá trị của biểu thức $A = \frac{4x + y}{6x - y}$.
\end{baitoan}

\begin{baitoan}[Mở rộng \cite{Tuyen_Toan_7}, Ví dụ 42, p. 37]
	\label{mo rong Tuyen_Toan_7 vi du 42}
	Cho $y = kx$, tính giá trị của biểu thức:
	\begin{align*}
		A &= \frac{ax + by}{cx + dy},\ B = \frac{ax^2 + bxy + cy^2}{dx^2 + exy + fy^2},\ C = \frac{a_1x^3 + a_2x^2y + a_3xy^2 + a_4y^3}{b_1x^3 + b_2x^2y + b_3xy^2 + b_4y^3},\\
		D &= \frac{\sum_{i=0}^n a_ix^{n-i}y^i}{\sum_{i=0}^n b_ix^{n-i}y^i} = \frac{a_1x^n + a_2x^{n-1}y + \cdots + a_{n-1}xy^{n-1} + a_ny^n}{b_1x^n + b_2x^{n-1}y + \cdots + b_{n-1}xy^{n-1} + b_ny^n},
	\end{align*}
	với $a,b,c,d,e,f,a_i,b_i\in\mathbb{R}$, $\forall i = 1,2,\ldots,n$.
\end{baitoan}

\begin{baitoan}[\cite{Tuyen_Toan_7}, Ví dụ 43, p. 37]
	Tính giá trị của biểu thức: $B = x^2 + 4xy - 3y^3$ với $|x| = 5$, $|y| = 1$.
\end{baitoan}

\begin{nhanxet}
	``Biểu thức $B$ có chứa 2 biến $x,y$. Biến $x$ nhận 2 giá trị, biến $y$ nhận 2 giá trị do đó ta phải xét đủ 4 trường hợp các cặp giá trị của $x,y$ dẫn đến biểu thức $B$ có 4 giá trị khác nhau.'' -- \cite[p. 38]{Tuyen_Toan_7}
\end{nhanxet}

\begin{baitoan}[\cite{Tuyen_Toan_7}, \textbf{150.}, p. 38]
	Cho $A$ là tổng lập phương các số tự nhiên từ $1$ đến $n$ \& $B$ là bình phương của tổng các số tự nhiên từ $1$ đến $n$. Người ta đã chứng minh được $A = B$. Kiểm nghiệm lại bằng cách cho $n = 1,2,3,4,5,6$.
\end{baitoan}

\begin{baitoan}[\cite{Tuyen_Toan_7}, \textbf{151.}, p. 38]
	Tính giá trị của các biểu thức sau với $x = \sqrt{2}$: (a) $(x + 1)(x^2 - 2)$; (b) $(x - 1)(x^2 + 1) + 3$.
\end{baitoan}

\begin{baitoan}[\cite{Tuyen_Toan_7}, \textbf{152.}, p. 38]
	Tính giá trị của biểu thức $M = \frac{2x^2 + 3x - 2}{x + 2}$ tại: (a) $x = -1$; (b) $|x| = 3$.
\end{baitoan}

\begin{baitoan}[\cite{Tuyen_Toan_7}, \textbf{153.}, p. 38]
	Tính giá trị của biểu thức $N = \frac{6x^2 + x - 3}{2x - 1}$ với $|x| = \frac{1}{2}$.
\end{baitoan}

\begin{baitoan}[\cite{Tuyen_Toan_7}, \textbf{154.}, p. 38]
	Tính giá trị của biểu thức $P = 9x^2 - 7x|y| - \frac{1}{4}y^3$ tại $x = \frac{1}{3}$, $y = -6$.
\end{baitoan}

\begin{baitoan}[\cite{Tuyen_Toan_7}, \textbf{155.}, p. 38]
	Tìm các giá trị của biến để: (a) Biểu thức $(x + 1)(y^2 - 6)$ có giá trị bằng $0$. (b) Biểu thức $x^2 - 12x + 7$ có giá trị lớn hơn $7$.
\end{baitoan}

\begin{baitoan}[\cite{Tuyen_Toan_7}, \textbf{156.}, p. 38]
	Tính giá trị của biểu thức $Q = \frac{5x^2 + 3y^2}{10x^2 - 3y^2}$ với $\frac{x}{3} = \frac{y}{5}$.
\end{baitoan}
Bài toán này là 1 trường hợp nhỏ của Bài toán \ref{mo rong Tuyen_Toan_7 vi du 42}: $B = \frac{ax^2 + bxy + cy^2}{dx^2 + exy + fy^2}$ khi $k = \frac{5}{3}, a = 5, b = 0, c = 3, d = 10, e = 0, f = -3$.

\begin{baitoan}[\cite{Tuyen_Toan_7}, \textbf{157.}, p. 38]
	Cho $x,y,z\in\mathbb{R}$, $x,y,z\ne0$, $x - y - z = 0$. Tính giá trị của biểu thức $M = \left(1 - \frac{z}{x}\right)\left(1 - \frac{x}{y}\right)\left(1 + \frac{y}{z}\right)$.
\end{baitoan}

\begin{baitoan}[\cite{Tuyen_Toan_7}, \textbf{158.}, p. 38]
	(a) Tìm GTNN của biểu thứcG: $A = (x + 2)^2 + \left(y - \frac{1}{5}\right)^2 - 10$. (b) Tìm GTLN của biểu thức: $B = \frac{4}{(2x - 3)^2 + 5}$.
\end{baitoan}

\begin{baitoan}[\cite{Tuyen_Toan_7}, \textbf{159.}, p. 38]
	Cho biểu thức $C = \frac{5 - x}{x - 2}$. Tìm giá trị nguyên của $x$ để: (a) $C$ có giá trị nguyên; (b) $C$ có giá trị nhỏ nhất.
\end{baitoan}

%------------------------------------------------------------------------------%

\section{Đa Thức 1 Biến. Nghiệm của Đa Thức 1 Biến}
``\fbox{\bf 1} \textit{Đơn thức 1 biến} là biểu thức đại số chỉ gồm 1 số hoặc 1 tích của 1 số thực với lũy thừa có số mũ nguyên dương của biến đó. Số thực gọi là \textit{hệ số}, số mũ của lũy thừa gọi là \textit{bậc} của đơn thức. E.g., đơn thức $-3x^4$ có hệ số là $-3$, bậc $4$. Số $5$ là đơn thức có hệ số là $5$, bậc $0$ (vì $5 = 5x^0$). Số $0$ cũng coi là 1 đơn thức nhưng nó không có bậc. \fbox{\bf 2} Với các đơn thức 1 biến ta có thể: Cộng\texttt{/}trừ 2 đơn thức cùng bậc bằng cách cộng\texttt{/}trừ các hệ số với nhau \& giữ nguyên lũy thừa của biến. Nhân 2 đơn thức tùy ý bằng cách nhân 2 hệ số với nhau \& nhân 2 lũy thừa của biến với nhau. \fbox{\bf 3} Đa thức 1 biến là tổng của những đơn thức của cùng 1 biến. Mỗi đơn thức trong tổng gọi là 1 \textit{hạng tử} của đa thức. Đặc biệt, số 0 cũng được coi là 1 đa thức, gọi là \textit{đa thức không}. Ta thường ký hiệu đa thức (1 biến) bằng 1 chữ cái in hoa. E.g., $A(x)$ là đa thức 1 biến $x$ còn $A(y)$ là đa thức 1 biến $y$. \fbox{\bf 4} Thu gọn đa thức (1 biến) là làm cho đa thức đó không còn 2 đơn thức nào có cùng bậc của biến. Sắp xếp đa thức (1 biến) theo số mũ giảm dần (hoặc tăng dần) của biến là sắp xếp các hạng tử trong dạng đã thu gọn của đa thức đó theo số mũ giảm dần (hoặc tăng dần). \fbox{\bf 5} Bậc của 1 đa thức 1 biến đã thu gọn (khác đa thức không) là số mũ cao nhất của biến trong đa thức đó. \textit{Chú ý}: Trong dạng thu gọn của đa thức, hệ số của lũy thừa với số mũ cao nhất của biến gọi là \textit{hệ số cao nhất} của đa thức. Hạng tử không chứa biến gọi là \textit{hạng tử tự do} của đa thức. \fbox{\bf 6} Nghiệm của đa thức 1 biến: Nếu tại $x = a$ mà đa thức $P(x)$ có giá trị bằng $0$ (i.e., $P(a) = 0$) thì $x = a$ là 1 nghiệm của đa thức. \fbox{\bf 7} 1 đa thức (khác đa thức không) có thể có 1 nghiệm, 2 nghiệm, $\ldots$ hoặc không có nghiệm nào (0 nghiệm). 1 đa thức bậc $n$ có không quá $n$ nghiệm. \fbox{\bf 8} Mỗi đa thức bậc nhất biến $x$ đều có thể viết dưới dạng $ax + b$ trong đó hệ số $a,b$ là các số cho trước (hằng số), $a\ne0$. Ta gọi đa thức $ax + b$ như thế là \textit{nhị thức bậc nhất}. Mỗi đa thức bậc 2 biến $x$ đều có thể viết dưới dạng $ax^2 + bx + c$ trong đó các hệ số $a,b,c\in\mathbb{R}$, $a\ne 0$\footnote{Vì nếu $a = 0$, $ax^2 + bx + c$ trở thành đa thức bậc nhất $bx + c$ chứ không còn là 1 đa thức bậc 2 nữa.}, là các số cho trước (hằng số thực). Ta gọi đa thức $ax^2 + bx + c$ như thế là \textit{tam thức bậc 2}.'' -- \cite[Chap. III, \S2, p. 39]{Tuyen_Toan_7}

\begin{baitoan}[\cite{SGK_Toan_7_Canh_Dieu_tap_2}, \textbf{3.}, p. 52--53]
	Cho 2 đa thức $P(y) = -12y^4 + 5y^4 + 13y^3 - 6y^3 + y -1 + 9$, $Q(y) = -20y^3 + 31y^3 + 6y - 8y + y - 7 + 11$. (a) Thu gọn mỗi đa thức trên rồi sắp xếp mỗi đa thức theo số mũ giảm dần của biến. (b) Tìm bậc, hệ số cao nhất \& hệ số tự do của mỗi đa thức đó.
\end{baitoan}

\begin{baitoan}[\cite{SGK_Toan_7_Canh_Dieu_tap_2}, \textbf{3.}, pp. 52--53]
	Cho đa thức $P(x) = ax^2 + bx + c$, $a\ne0$. Chứng tỏ: (a) $P(0) = c$; (b) $P(1) = a + b + c$; (c) $P(-1) = a - b + c$. (d) Tính $P(2),P(-2),P(3),P(-3),P\left(\frac{1}{2}\right),P\left(-\frac{1}{2}\right)$. (e) Tính $P(x) + P(-x)$ với $x\in\mathbb{R}$. (f) Tính $P(x) + P\left(\frac{1}{x}\right)$ với $x\in\mathbb{R}$.
\end{baitoan}

\begin{baitoan}[\cite{SGK_Toan_7_Canh_Dieu_tap_2}, \textbf{6.}, p. 53]
	Theo tiêu chuẩn của Tổ chức Y tế Thế Giới (WHO), đối với bé gái, công thức tính cân nặng chuẩn là $C = 9 + 2(N - 1)$ \emph{kg}, công thức tính chiều cao chuẩn là $H = 75 + 5(N - 1)$ \emph{cm}, trong đó $N$ là số tuổi của bé gái. (a) Tính cân nặng chuẩn, chiều cao chuẩn của 1 bé gái $3$ tuổi. (b) 1 bé gái $3$ tuổi nặng $13.5$\emph{kg} \& cao $86$\emph{cm}. Bé gái đó có đạt tiêu chuẩn về cân nặng \& chiều cao của Tổ chức Y tế Thế giới hay không?
\end{baitoan}

\begin{baitoan}[\cite{SGK_Toan_7_Canh_Dieu_tap_2}, \textbf{7.}, p. 52--53]
	Nhà bác học Galileo Galilei (1564--1642) là người đầu tiên phát hiện ra quãng đường chuyển động của vật rơi tự do tỷ lệ thuận với bình phương của thời gian chuyển động. Quan hệ giữa quãng đường chuyển động $y$ \emph{m} \& thời gian chuyển động $x$ \emph{s} được biểu diễn gần đúng bởi công thức $y = 5x^2$. Trong 1 thí nghiệm vật lý, người ta thả 1 vật nặng từ độ cao $180$\emph{m} xuống đất (coi sức cản của không khí không đáng kể). (a) Sau $3$\emph{s} thì vật nặng còn cách mặt đất bao nhiêu \emph{m}? (b) Khi vật nặng còn cách mặt đất $100$\emph{m} thì nó đã rơi được thời gian bao lâu? (c) Sau bao lâu thì vật chạm đất?
\end{baitoan}

\begin{baitoan}[\cite{SGK_Toan_7_Canh_Dieu_tap_2}, \textbf{8.}, p. 53]
	Pound là 1 đơn vị đo khối lượng truyền thống của Anh, Mỹ \& 1 số quốc gia khác. Công thức tính khối lượng $y$ \emph{kg} theo $x$ pound là $y = 0.45359237x$. (a) Tính giá trị của $y$ \emph{kg} khi $x = 100$ pound. (b) 1 hãng hàng không quốc tế quy định mỗi hành khác được mang 2 va li không tính cước; mỗi va li cân nặng không vượt quá $23$ \emph{kg}. Hỏi với va li cân nặng $50.99$ pound sau khi quy đổi sang kilogram \& được phép làm tròn đến hàng đơn vị thì có vượt quá quy định trên hay không?
\end{baitoan}

\begin{baitoan}[\cite{Tuyen_Toan_7}, Ví dụ 44, p. 40]
	Cho các đơn thức $A = -\frac{4}{9}ax^3$, $B = \frac{3}{8}ax^5$ trong đó $a\in\mathbb{R}$ là số đã biết (hằng số). Có giá trị nào của biến $x$ làm cho $A$ \& $B$ cùng có giá trị âm không?
\end{baitoan}

\begin{nhanxet}
	``Trong đơn thức cũng như trong đa thức nói chung, ngoài chữ chỉ biến số có thể còn có những chữ khác đại diện cho những số đã biết mà ta gọi là \emph{hằng số}.'' -- \cite[p. 40]{Tuyen_Toan_7}
\end{nhanxet}

\begin{baitoan}[\cite{Tuyen_Toan_7}, Ví dụ 45, p. 40]
	Cho các đa thức $f(x) = ax^3 + 4x(x^2 - 1) + 8$, $g(x) = x^3 - 4x(bx + 1) + c - 3$ trong đó $a,b,c\in\mathbb{R}$ là những hằng số. (a) Thu gọn \& sắp xếp mỗi đa  thức trên theo số mũ giảm dần của biến. (b) Xác định các hệ số $a,b,c$ để $f(x) = g(x)$.
\end{baitoan}

\begin{nhanxet}
	``2 đa thức cùng biến bằng nhau $\Leftrightarrow$ các hệ số của lũy thừa cùng bậc bằng nhau.'' -- \cite[p. 40]{Tuyen_Toan_7}
\end{nhanxet}
I.e., với $P(x) = \sum_{i=0}^n a_ix^i = a_nx^n + a_{n-1}x^{n-1} + \cdots + a_1x + a_0$, $Q(x) = \sum_{i=0}^m b_ix^i = b_mx^m + b_{m-1}x^{m-1} + \cdots + b_1x + b_0$, $m,n\in\mathbb{N}$, $a_i,b_j\in\mathbb{R}$, $\forall i = 1,2,\ldots,n$, $\forall j = 1,\ldots,m$, thì
\begin{equation*}
	P(x) = Q(x),\ \forall x\in\mathbb{R}\Leftrightarrow\left\{\begin{split}
		m &= n,\\
		a_i &= b_i,\ \forall i = 1,2,\ldots,n.
	\end{split}\right.
\end{equation*}

\begin{baitoan}[\cite{Tuyen_Toan_7}, Ví dụ 46, p. 40]
	Cho đa thức $f(x) = x^2 + 4x - 5$. (a) Số $-5$ có phải là nghiệm của $f(x)$ không? (b) Viết tập hợp $S$ tất cả ác nghiệm của $f(x)$.
\end{baitoan}

\begin{nhanxet}
	``Đa thức có tổng các hệ số bằng $0$ thì có 1 nghiệm là $1$. Nếu tổng các hệ số bậc chẵn bằng tổng các hệ số bậc lẻ thì đa thức có 1 nghiệm là $-1$'' -- \cite[p. 40]{Tuyen_Toan_7}
\end{nhanxet}
I.e., với $P(x) = \sum_{i=0}^n a_ix^i = a_nx^n + a_{n-1}x^{n-1} + \cdots + a_1x + a_0$, $n\in\mathbb{N}$, $a_i\in\mathbb{R}$, $\forall i = 1,2,\ldots,n$, thì
\begin{align*}
	\sum_{i=0}^n a_i = 0&\Leftrightarrow P(1) = 0,\\
	\sum_{i=0,\,i\divby 2}^n a_i = \sum_{i=0,\,i\not\,\divby 2}^n a_i&\Leftrightarrow P(-1) = 0.
\end{align*}
Thật vậy, vì $P(1) = \sum_{i=0}^n a_i1^i = a_n1^n + a_{n-1}1^{n-1} + \cdots + a_11 + a_0 = a_n + a_{n-1} + \cdots + a_1 + a_0 = \sum_{i=0}^n a_i$ \& $P(-1) = \sum_{i=0}^n a_i(-1)^i = a_n(-1)^n + a_{n-1}(-1)^{n-1} + \cdots + a_1(-1) + a_0 = \sum_{i=0,\,i\divby 2}^n a_i - \sum_{i=0,\,i\not\,\divby 2}^n a_i$.

\begin{baitoan}[\cite{Tuyen_Toan_7}, \textbf{160.}, p. 40]
	Cho biểu thức $M = (4a + 1)x^3$ với $a\in\mathbb{R}$ là hằng số. Hỏi biểu thức $M$ có phải là đơn thức không? Nếu $M$ là đơn thức thì cho biết bậc của $M$ \& hệ số của nó.
\end{baitoan}

\begin{baitoan}[\cite{Tuyen_Toan_7}, \textbf{161.}, p. 40]
	Viết đơn thức $64x^6$ dưới dạng lũy thừa của 1 đơn thức.
\end{baitoan}

\begin{baitoan}[\cite{Tuyen_Toan_7}, \textbf{162.}, p. 40]
	Cho 3 đơn thức $M = -5x$, $N = 11x$, $P = \frac{7}{5}x^2$. Chứng minh 3 đơn thức này không thể có cùng giá trị dương.
\end{baitoan}

\begin{baitoan}[\cite{Tuyen_Toan_7}, \textbf{163.}, p. 41]
	Cho đơn thức $A = 5m(x^2)^2$, $B = -\frac{2}{m}x^4$ trong đó $m$ là hằng số dương. (a) 2 đơn thức $A$ \& $B$ có cùng bậc không? (b) Tính hiệu $A - B$. (c) Tính giá trị nhỏ nhất của hiệu $A - B$.
\end{baitoan}

\begin{baitoan}[\cite{Tuyen_Toan_7}, \textbf{164.}, p. 41]
	Viết các số tự nhiên sau dưới dạng 1 đa thức thu gọn: (a) $\overline{xxx}$; (b) $\overline{x1x2}$.
\end{baitoan}

\begin{baitoan}[\cite{Tuyen_Toan_7}, \textbf{165.}, p. 41]
	Cho đa thức $A(x) = x^8 - 101x^7 + 101x^6 - 101x^5 + \cdots + 101x^2 - 101x + 25$. Tính $A(100)$.
\end{baitoan}

\begin{baitoan}[\cite{Tuyen_Toan_7}, \textbf{166.}, p. 41]
	Cho $f(x) = (8x^2 + 5x - 10)^{49}(3x^3 - 10x^2 + 6x + 1)^{50}$. Sau khi thu gọn thì tổng các hệ số của $f(x)$ là bao nhiêu?
\end{baitoan}

\begin{baitoan}[\cite{Tuyen_Toan_7}, \textbf{167.}, p. 41]
	Cho tam thức bậc 2 $f(x) = ax^2 + bx + c$ trong đó $7a + b = 0$. Hỏi $f(10)f(-3)$ có thể là số âm không?
\end{baitoan}

\begin{baitoan}[\cite{Tuyen_Toan_7}, \textbf{168.}, p. 41]
	Cho nhị thức bậc nhất $f(x) = ax + b$. Xác định các hệ số $a,b$. Biết $f(1) = 2$, $f(3) = 8$.
\end{baitoan}

\begin{baitoan}[\cite{Tuyen_Toan_7}, \textbf{169.}, p. 41]
	Cho tam thức bậc 2 $f(x) = ax^2 + bx + c$. Xác định các hệ số $a,b,c$ biết $f(1) = 4$, $f(-1) = 8$, \& $a - c = -4$.
\end{baitoan}

\begin{baitoan}[\cite{Tuyen_Toan_7}, \textbf{170.}, p. 41]
	Cho $f(x) = 2x^2 + ax + 4$ ($a\in\mathbb{R}$ là hằng số), $g(x) = x^2 - 5x - b$ ($b\in\mathbb{R}$ là hằng số). Tìm các hệ số $a,b$ sao cho $f(1) = g(2)$ \& $f(-1) = g(5)$.
\end{baitoan}

\begin{baitoan}[\cite{Tuyen_Toan_7}, \textbf{171.}, p. 41]
	Tìm nghiệm của đa thức sau: (a) $(x - 3)(4 - 5x)$; (b) $x^2 - 2$; (c) $x^2 - \sqrt{3}$; (d) $x^2 + 2x$.
\end{baitoan}

\begin{baitoan}[\cite{Tuyen_Toan_7}, \textbf{172.}, p. 41]
	Thu gọn rồi tìm nghiệm của đa thức sau: (a) $f(x) = x(1 - 2x) + (2x^2 - x + 4)$; (b) $g(x) = x(x - 5) - x(x + 2) + 7x$.
\end{baitoan}

\begin{baitoan}[\cite{Tuyen_Toan_7}, \textbf{173.}, p. 41]
	Xác định hệ số $m$ để các đa thức sau nhận $1$ là nghiệm: (a) $mx^2 + 2x + 8$; (b) $7x^2 + mx - 1$; (c) $x^5 - 3x^2 + m$.
\end{baitoan}

\begin{baitoan}[\cite{Tuyen_Toan_7}, \textbf{174.}, p. 41]
	Cho đa thức $f(x) = x^2 + mx + 2$. (a) Xác định $m$ để $f(x)$ nhận $-2$ là 1 nghiệm. (b) Tìm tập hợp các nghiệm của $f(x)$ ứng với giá trị vừa tìm được của $m$.
\end{baitoan}

\begin{baitoan}[\cite{Tuyen_Toan_7}, \textbf{175.}, p. 41]
	Cho các nhị thức bậc nhất $f(x) = ax + b$ \& $g(x) = bx + a$. Chứng minh nếu $x_0$ là nghiệm của $f(x)$ thì $\frac{1}{x_0}$ là nghiệm của $g(x)$.
\end{baitoan}

\begin{baitoan}[\cite{Tuyen_Toan_7}, \textbf{176.}, p. 41]
	Cho biết $(x - 1)f(x) = (x + 4)f(x + 8)$, $\forall x\in\mathbb{R}$. Chứng minh $f(x)$ có ít nhất 2 nghiệm.
\end{baitoan}

\begin{baitoan}[Mở rộng \cite{Tuyen_Toan_7}, \textbf{176.}, p. 41]
	Cho biết $(x + a)f(x + b) = (x + c)f(x + d)$, $\forall x\in\mathbb{R}$d, với $a,b,c,d\in\mathbb{R}$, khác nhau đôi một. Chứng minh $f(x)$ có ít nhất 2 nghiệm.
\end{baitoan}

%------------------------------------------------------------------------------%

\section{Phép $\pm$ Đa Thức 1 Biến}
``\fbox{\bf 1} Để cộng 2 đa thức 1 biến theo hàng ngang ta thực hiện theo các bước sau: (a) Viết mỗi đa thức vào trong ngoặc \& nối với nhau bởi dấu cộng. (b) Bỏ dấu ngoặc rồi nhóm các hạng tử cùng bậc theo thứ tự giảm dần\texttt{/}tăng dần. (c) Thực hiện phép tính trong từng nhóm ta được tổng cần tìm. Ta cũng có thể cộng 2 đa thức 1 biến theo cột dọc bằng cách: (a) Thu gọn \& sắp xếp mỗi đa thức theo thứ tự giảm dần\texttt{/}tăng dần. (b) Đặt các đa thức theo cột dọc, các hạng tử cùng bậc thẳng cột với nhau. (c) Cộng từng cột ta được tổng cần tìm. \fbox{\bf 2} Để trừ 2 đa thức 1 biến theo hàng ngang ta thực hiện như sau: (a) Viết mỗi đa thức vào trong ngoặc \& nối với nhau bởi dấu trừ. (b) Bỏ dấu ngoặc rồi nhóm các hạng tử cùng bậc theo thứ tự bậc giảm dần\texttt{/}tăng dần. (c) Thực hiện phép tính trong từng nhóm ta được hiệu cần tìm. Ta cũng có thể trừ 2 đa thức theo cột dọc, tương tự như cộng 2 đa thức theo cột dọc. \fbox{\bf 3} Phép cộng đa thức cũng có tính chất như phép cộng các số thực.'' -- \cite[Chap. III, \S3, p. 42]{Tuyen_Toan_7}

\begin{baitoan}
	Tính: (a) $ax^k + bx^k$, $\forall a,b\in\mathbb{R}$, $\forall k\in\mathbb{N}$. (b) $ax^k + bx^k + cx^k$, $\forall a,b,c\in\mathbb{R}$, $\forall k\in\mathbb{N}$. (c) $\sum_{i=1}^n a_ix^k = a_1x^k + a_2x^k + \cdots + a_nx^k$, $\forall a_i\in\mathbb{R}$, $\forall i = 1,2,\ldots,n$, $\forall k\in\mathbb{N}$.
\end{baitoan}

\begin{baitoan}[\cite{SGK_Toan_7_Canh_Dieu_tap_2}, Ví dụ 1, p. 55]
	Tính tổng của 2 đa thức: $P(x) = 5x^3 + 2x^2 + 3x + 1$ \& $Q(x) = 2x^3 - 4x^2 + 2x + 2$.
\end{baitoan}

\begin{baitoan}[\cite{SGK_Toan_7_Canh_Dieu_tap_2}, \textbf{1.}, p. 59]
	Cho 2 đa thức: $R(x) = -8x^4 + 6x^3 + 2x^2 - 5x + 1$ \& $S(x) = x^4 - 8x^3 + 2x + 3$. Tính: (a) $R(x) + S(x)$; (b) $R(x) - S(x)$.
\end{baitoan}

\begin{baitoan}[\cite{SGK_Toan_7_Canh_Dieu_tap_2}, \textbf{2.}, p. 59]
	Xác định bậc của 2 đa thức là tổng, hiệu của: $A(x) = -8x^5 + 6x^4 + 2x^2 - 5x + 1$ \& $B(x) = 8x^5 + 8x^3 + 2x - 3$.
\end{baitoan}

\begin{baitoan}[\cite{SGK_Toan_7_Canh_Dieu_tap_2}, \textbf{3.}, p. 59]
	Bác Ngọc gửi ngân hàng thứ nhất $90$ triệu đồng với kỳ hạn 1 năm, lãi suất $x$\%\emph{\texttt{/}}năm. Bác Ngọc gửi ngân hàng thứ 2 $80$ triệu đồng với kỳ hạn 1 năm, lãi suất $(x + 1.5)$\%\emph{\texttt{/}}năm. Hết kỳ hạn 1 năm, bác Ngọc có được cả gốc \& lãi là bao nhiêu: (a) Ở ngân hàng thứ 2? (b) Ở cả 2 ngân hàng?
\end{baitoan}

\begin{baitoan}[\cite{SGK_Toan_7_Canh_Dieu_tap_2}, \textbf{4.}, p. 59]
	Người ta rót nước từ 1 can đựng $10$ lít nước sang 1 bể rỗng có dạng hình lập phương với độ dài cạnh $20$\emph{cm}. Khi mực nước trong bể cao $h$ \emph{cm} thì thể tích nước trong can còn lại là bao nhiêu? Biết $1{\rm l} = 1{\rm dm}^3$.
\end{baitoan}

\begin{baitoan}[\cite{SGK_Toan_7_Canh_Dieu_tap_2}, \textbf{5.}, p. 59]
	Đ hay S? (a) Tổng của 2 đa thức bậc 4 luôn luôn là đa thức bậc 4. (b) Hiệu của 2 đa thức bậc 4 luôn luôn là đa thức bậc 4. (c) Tổng \& hiệu của 2 đa thức bậc $n\in\mathbb{N}$ luôn là đa thức bậc $n$.
\end{baitoan}

\begin{baitoan}[\cite{Tuyen_Toan_7}, Ví dụ 47, p. 42]
	Cho các đa thức biến $x$: $A = 7x + 5a$, $B = 2x - 9a$, $C = x + 10a$, trong đó $a$ là hằng số, $a,x\in\mathbb{Z}$. Không cần thực hiện phép nhân, cho biết tích $ABC$ có giá trị là 1 số chẵn hay lẻ?
\end{baitoan}

\begin{nhanxet}
	``Trong phép nhân các số nguyên, tích là 1 số lẻ thì tất cả các thừa số đều là số lẻ. Tích là số chẵn thì có ít nhất 1 thừa số là số chẵn.''
\end{nhanxet}
I.e., $\prod_{i=1}^n a_i = a_1a_2\cdots a_n\divby2\Leftrightarrow\exists i\in\{1,2,\ldots,n\}$ s.t. $a_i\divby2$ (ký hiệu $\exists$ là \textit{tồn tại}). $\prod_{i=1}^n a_i = a_1a_2\cdots a_n\not\,\divby2\Leftrightarrow a_i\not\,\divby2$, $\forall i = 1,2,\ldots,n$.

\begin{baitoan}[\cite{Tuyen_Toan_7}, \textbf{177.}, p. 42]
	Cho đa thức $A = 7x^4 - 2x^3 + x - 9$, $B = -5x^4 + 2x^3 - 4x^2 - 6x - 1$. Tính tổng $A + B$ \& hiệu $A - B$ bằng 2 cách.
\end{baitoan}

\begin{baitoan}[\cite{Tuyen_Toan_7}, \textbf{178.}, p. 42]
	Tính tổng $S = \overline{a1} + \overline{a17} + \overline{1a} - \overline{1a7}$.
\end{baitoan}

\begin{baitoan}[\cite{Tuyen_Toan_7}, \textbf{179.}, p. 42]
	Chứng minh tổng của 4 số lẻ liên tiếp thì chia hết cho $8$.
\end{baitoan}

\begin{baitoan}[\cite{Tuyen_Toan_7}, \textbf{180.}, p. 42]
	Cho đa thức $A = 16x^4 - 8x^3 + 7x^2 - 9$, $B = -15x^4 + 3x^3 - 5x^2 - 6$, $C = 5x^3 + 3x^2 + 18$. Chứng minh ít nhất 1 trong 3 đa thức này có giá trị dương với mọi $x\in\mathbb{R}$.
\end{baitoan}

\begin{baitoan}[\cite{Tuyen_Toan_7}, \textbf{181.}, p. 42]
	Cho đa thức $A = 2x^2 + |7x - 1| - (5 - x + 2x^2)$. (a) Thu gọn $A$. (b) Tìm $x\in\mathbb{R}$ để $A$ có giá trị bằng $2$.
\end{baitoan}

\begin{baitoan}[\cite{Tuyen_Toan_7}, \textbf{182.}, p. 43]
	Cho $f(x) + g(x) = 6x^4 - 3x^2 - 5$, $f(x) - g(x) = 4x^4 - 6x^3 + 7x^2 + 8x - 9$. Tìm các đa thức $f(x),g(x)$.
\end{baitoan}

\begin{baitoan}[\cite{Tuyen_Toan_7}, \textbf{183.}, p. 43]
	Cho $f(x) = x^{2n} - x^{2n-1} + \cdots + x^2 - x + 1$, $g(x) = -x^{2n+1} + x^{2n} - x^{2n-1} + \cdots + x^2 - x + 1$, với $n\in\mathbb{N}$. Tính giá trị của hiệu $f(x) - g(x)$ tại $x = \frac{1}{10}$.
\end{baitoan}

\begin{baitoan}[\cite{Tuyen_Toan_7}, \textbf{184.}, p. 43]
	Bên trong khu đất hình vuông cạnh $3x$ \emph{m} có khu vực chăn nuôi hình chữ nhật kích thước $x$ \emph{m} \& $5$\emph{m}. (a) Tính diện tích $S$ còn lại để làm vườn cây. (b) Tìm nghiệm của đa thức $S$.
\end{baitoan}

%------------------------------------------------------------------------------%

\section{Phép Nhân Đa Thức 1 Biến}
``\fbox{\bf 1} Muốn nhân 1 đơn thức với 1 đa thức, ta nhân đơn thức với từng hạng tử của đa thức rồi cộng các tích với nhau. $A(B + C) = AB + AC$. \fbox{\bf 2} Muốn nhân 1 đa thức với 1 đa thức, ta nhân mỗi hạng tử của đa thức này với từng hạng tử của đa thức kia rồi cộng các tích với nhau. \fbox{\bf 3} Phép nhân đa thức cũng có các tính chất giao hoán, kết hợp, phân phối của phép nhân đối với phép cộng.'' -- \cite[Chap. III, \S4, p. 43]{Tuyen_Toan_7}

\begin{baitoan}[\cite{Tuyen_Toan_7}, Ví dụ 48, p. 43]
	Rút gọn biểu thức $A =  (x + 5)(2x - 3) - 2x(x + 3) - (x - 15)$ rồi cho biết bậc của đa thức kết quả.
\end{baitoan}

\begin{baitoan}[\cite{Tuyen_Toan_7}, Ví dụ 49, p. 43]
	Cho biểu thức $C = x(x + x^3) + (x - 1)(x^2 + x^3) + 1$. Rút gọn biểu thức $C$ rồi chứng minh với 2 giá trị đối nhau của $x$ thì biểu thức $C$ có cùng 1 giá trị.
\end{baitoan}

\begin{baitoan}[\cite{Tuyen_Toan_7}, \textbf{185.}, p. 44]
	Cho biểu thức $B = 5x^2(3x - 2) - (4x + 7)(6x^2 - x) - (7x - 9x^3)$. Rút gọn rồi  tính giá trị của biểu thức $B$ với $x = -\frac{3}{4}$.
\end{baitoan}

\begin{baitoan}[\cite{Tuyen_Toan_7}, \textbf{186.}, p. 44]
	Chứng minh giá trị của biểu thức sau không phụ thuộc vào giá trị của biến: $A = (2x - 3)(x + 7) - 2x(x + 5) - x$. 
\end{baitoan}

\begin{baitoan}[\cite{Tuyen_Toan_7}, \textbf{187.}, p. 44]
	Cho $ab = 1$. Chứng minh đẳng thức: $a(b + 1) + b(a + 1) = (a + 1)(b + 1)$.
\end{baitoan}

\begin{baitoan}[\cite{Tuyen_Toan_7}, \textbf{188.}, p. 44]
	Tìm $x$ biết: $3(x - 2)(x + 3) - x(3x + 1) = 2$.
\end{baitoan}

\begin{baitoan}[\cite{Tuyen_Toan_7}, \textbf{189.}, p. 44]
	Tính giá trị của biểu thức sau bằng cách hợp lý: (a) $A = x^5 - 100x^4 + 100x^3 - 100x^2 + 100x + 9$ tại $x = 99$; (b) $B = x^6 - 20x^5 - 20x^4 - 20x^3 - 20x^2 - 20x + 3$ tại $x = 21$; (c) $C = x^7 - 26x^6 + 27x^5 - 47x^4 - 77x^3 + 50x^2 + x - 24$ tại $x = 25$.
\end{baitoan}

\begin{baitoan}[\cite{Tuyen_Toan_7}, \textbf{190.}, p. 44]
	Cho 4 số lẻ liên tiếp. Chứng minh hiệu của tích 2 số cuối với tích của 2 số đầu chia hết cho $16$.
\end{baitoan}

\begin{baitoan}[\cite{Tuyen_Toan_7}, \textbf{191.}, p. 44]
	Cho 4 số nguyên liên tiếp. Hỏi tích của số đầu với số cuối nhỏ hơn tích của 2 số giữa bao nhiêu đơn vị?
\end{baitoan}

\begin{baitoan}[\cite{Tuyen_Toan_7}, \textbf{192.}, p. 44]
	Cho $b + c = 100$, chứng minh đẳng thức $(10a + b)(10a + c) = 100a(a + 1) + bc$. Áp dụng để tính nhẩm $62\cdot68$, $43\cdot47$.
\end{baitoan}

\begin{baitoan}[\cite{Tuyen_Toan_7}, \textbf{193.}, p. 44]
	Xác định các hệ số $a,b,c\in\mathbb{R}$ biết: (a) $(2x - 5)(3x + b) = ax^2 + bx + c$, $\forall x\in\mathbb{R}$. (b) $(ax + b)(x^2 - x - 1) = ax^3 + cx^2 - 1$, $\forall x\in\mathbb{R}$.
\end{baitoan}

\begin{baitoan}[\cite{Tuyen_Toan_7}, \textbf{194.}, p. 44]
	Cho $m\in\mathbb{N}^\star$, $m < 30$. Có bao nhiêu giá trị của $m$ để đa thức $x^2 + mx + 72$ là tích của 2 đa thức bậc nhất với hệ số nguyên.
\end{baitoan}

%------------------------------------------------------------------------------%

\section{Phép Chia Đa Thức 1 Biến}
``\fbox{\bf 1} Chia đơn thức $A$ cho đơn thức $B$, $B\ne0$, khi số mũ của biến trong $A$ lớn hơn hoặc bằng số mũ của biến đó trong $B$ ta làm như sau: (a) Chia hệ số của $A$ cho hệ số của $B$. (b) Chia lũy thừa của biến trong $A$ cho lũy thừa của biến đó trong $B$. (c) Nhân các kết quả với nhau: $ax^m:bx^n = \frac{ax^m}{bx^n} = \frac{a}{b}x^{m-n}$, $m\ge n$. \fbox{\bf 2} Muốn chia đa thức $P$ cho đơn thức $Q$, $Q\ne0$, khi số mũ của mỗi biến ở đơn thức $P$ lớn hơn hoặc bằng số mũ của biến đó trong $Q$ ta chia mỗi đơn thức của $P$ cho đơn thức $Q$ rồi cộng các thương với nhau. \fbox{\bf 3} Để chia 1 đa thức cho 1 đa thức khác đa thức không (cả 2 đa thức đều đã thu gọn \& sắp xếp các đa thức theo số mũ giảm dần của biến), bậc của đa thức bị chia lớn hơn hoặc bằng bậc của đa thức chia ta làm như sau: \textit{Bước 1}: (a) Chia đơn thức bậc cao nhất của đa thức bị chia cho đơn thức bậc cao nhất của đa thức chia. (b) Nhân kết quả trên với đa thức chia \& đặt tích dưới đa thức bị chia sao cho 2 đơn thức có cùng số mũ của biến ở từng cột. (c) Lấy đa thức bị chia trừ đi tích đặt ở dưới để được đa thức mới (gọi là \textit{đa thức dư thứ nhất}. \textit{Bước 2}: Tiếp tục quá trình trên cho đến khi nhận được đa thức không hoặc đa thức có bậc nhỏ hơn bậc của đa thức chia. \fbox{\bf 4} Nhận xét: $\bullet$ Khi chia đa thức $A$ cho đa thức $B$ của cùng 1 biến, $B\ne0$, có 2 khả năng xảy ra: (a) Phép chia có đa thức dư là đa thức không. Ta nói đa thức $A$ \textit{chia hết cho} đa thức $B$. (b) Phép chia có đa thức dư là đa thức $R\ne0$ có bậc nhỏ hơn bậc của $B$ ($\deg R < \deg B$). Ta nói phép chia này là phép chia có dư. $\bullet$ Đối với 2 đa thức tùy ý $A,B$ của cùng 1 biến, $B\ne0$, tồn tại duy nhất 1 cặp đa thức $Q,R$ áo cho $A = BQ + R$ trong đó $R = 0$ hoặc bậc của $R$ nhỏ hơn bậc của $B$ ($\deg R < \deg B$). Như vậy đa thức $A$ chia hết cho đa thức $B\Leftrightarrow R = 0$.

\begin{dinhly}[B\'ezout]
	Số dư trong phép chia đa thức $f(x)$ cho nhị thức bậc nhất $x - a$ đúng bằng $f(a)$.
\end{dinhly}

\begin{hequa}
	Nếu $a$ là nghiệm của đa thức $f(x)$ thì $f(x)$ chia hết cho $x - a$.
\end{hequa}
Đặc biệt: Nếu tổng các hệ số của đa thức $f(x)$ bằng 0 thì 1 là nghiệm \& $f(x)$ chia hết cho $x - 1$. Nếu $f(x)$ có tổng các hệ số bậc chẵn bằng tổng các hệ số bậc lẻ thì $-1$ là nghiệm \& $f(x)$ chia hết cho $x - (-1)$, i.e., $f(x)$ chia hết cho $x + 1$.'' -- \cite[Chap. III, \S5, pp. 44--45]{Tuyen_Toan_7}

\begin{baitoan}[\cite{Tuyen_Toan_7}, Ví dụ 50, p. 45]
	Tìm $n\in\mathbb{N}$ để cả 2 phép chia sau đồng thời là phép chia không còn dư: (a) $6x^5:3x^n$; (b) $15x^{n+2}:5x^4$.
\end{baitoan}

\begin{baitoan}[\cite{Tuyen_Toan_7}, Ví dụ 51, p. 45]
	Cho các đa thức $A = 2x^4 + 3x^3 - 3x^2 + mx - 5$, $B = x^2 + 1$. Tìm giá trị của $m$ để $A$ chia hết cho $B$.
\end{baitoan}

\begin{baitoan}[\cite{Tuyen_Toan_7}, Ví dụ 52, p. 46]
	Cho các đa thức $A = 6x^3 - 15x^2 - 4x + 13$, $B = 2x - 5$. Tìm các giá trị nguyên của $x$ để giá trị của $A$ chia hết cho giá trị của $B$.
\end{baitoan}

\begin{baitoan}[\cite{Tuyen_Toan_7}, \textbf{195.}, p. 46]
	Tìm $n\in\mathbb{N}$ để cả 2 phép chia sau đồng thời là phép chia không còn dư: $15x^{n+2}:3x^3$ \& $-\frac{1}{5}x^{n+3}:\frac{3}{10}x^{2n}$.
\end{baitoan}

\begin{baitoan}[\cite{Tuyen_Toan_7}, \textbf{196.}, p. 46]
	Tính: (a) $(x^3 + 2x + 3):(x + 1)$; (b) $(x^4 - 3x^3 + 3x - 1):(x^2 - 1)$.
\end{baitoan}

\begin{baitoan}[\cite{Tuyen_Toan_7}, \textbf{197.}, p. 46]
	Xác định các hệ số $a,b\in\mathbb{R}$ sao cho đa thức $x^4 + ax^3 + b$ chia hết cho đa thức $x^2 - 1$.
\end{baitoan}

\begin{baitoan}[\cite{Tuyen_Toan_7}, \textbf{198.}, p. 46]
	Tìm các giá trị nguyên của $x$ để thương có giá trị nguyên: (a) $(3x^3 + 13x^2 - 7x + 5):(3x - 2)$; (b) $(2x^5 + 4x^4 + 7x^3 - 49x - 44):(2x^2 - 7)$.
\end{baitoan}

\begin{baitoan}[\cite{Tuyen_Toan_7}, \textbf{199.}, p. 46]
	Chứng minh không tồn tại $n\in\mathbb{N}$ để cho giá trị của biểu thức $n^6 - n^4 - 2n^2 + 9$ chia hết cho giá trị của biểu thức $n^4 + n^2$.
\end{baitoan}

\begin{baitoan}[\cite{Tuyen_Toan_7}, \textbf{200.}, p. 47]
	Không thực hiện phép chia đa thức, tìm số dư trong phép chia đa thức $f(x)$ cho đa thức $g(x)$ trong các trường hợp sau: (a) $f(x) = x^{21} + x^{20} + x^{19} + 101$, $g(x) = x + 1$; (b) $f(x) = 3x^3 + 4x^2 - 2x + 7$, $g(x) = x + 2$; (c) $f(x) = x^4 - 5x^3 + 2x - 10$, $g(x) = x - 5$.
\end{baitoan}

\begin{baitoan}[\cite{Tuyen_Toan_7}, \textbf{201.}, p. 47]
	Chứng minh $f(x) = (x^2 - 3x + 1)^{31} - (x^2 - 4x - 5)^{30} + 2$ chia hết cho $x - 2$.
\end{baitoan}

\begin{baitoan}[\cite{Tuyen_Toan_7}, \textbf{202.}, p. 47]
	Tìm đa thức dư trong phép chia $(x^{54} + x^{45} + x^{36} + \cdots + x^9 + 1):(x^2 - 1)$.
\end{baitoan}

\begin{baitoan}[\cite{Tuyen_Toan_7}, \textbf{203.}, p. 47]
	Xác định đa thức $f(x)$ thỏa mãn cả 3 điều kiện sau: (a) Khi chia cho $x - 1$ dư $4$; (b) Khi chia cho $x + 2$ dư $1$; (c) Khi chia cho $(x - 1)(x + 2)$ thì được thương là $5x^2$ \& còn dư.
\end{baitoan}

\begin{baitoan}[\cite{Tuyen_Toan_7}, \textbf{204.}, p. 47]
	Cho đa thức $A = ax^2 + bx + c$. Xác định hệ số $b$ biết khi chia $A$ cho $x - 1$ hoặc chia $A$ cho $x + 1$ đều có cùng 1 đa thức dư.
\end{baitoan}

\begin{baitoan}[\cite{Tuyen_Toan_7}, \textbf{205.}, p. 47]
	Chứng minh nếu $x^4 - 4x^3 + 5ax^2 - 4bx + c$ chia hết cho $x^3 + 3x^2 - 9x - 3$ thì $a + b + c = 0$.
\end{baitoan}

%------------------------------------------------------------------------------%

\printbibliography[heading=bibintoc]
	
\end{document}