\documentclass{article}
\usepackage[backend=biber,natbib=true,style=authoryear,maxbibnames=10]{biblatex}
\addbibresource{/home/nqbh/reference/bib.bib}
\usepackage[utf8]{vietnam}
\usepackage{tocloft}
\renewcommand{\cftsecleader}{\cftdotfill{\cftdotsep}}
\usepackage[colorlinks=true,linkcolor=blue,urlcolor=red,citecolor=magenta]{hyperref}
\usepackage{amsmath,amssymb,amsthm,float,graphicx,mathtools,soul}
\allowdisplaybreaks
\newtheorem{assumption}{Assumption}
\newtheorem{baitoan}{Bài toán}
\newtheorem{cauhoi}{Câu hỏi}
\newtheorem{conjecture}{Conjecture}
\newtheorem{corollary}{Corollary}
\newtheorem{dangtoan}{Dạng toán}
\newtheorem{definition}{Definition}
\newtheorem{dinhly}{Định lý}
\newtheorem{dinhnghia}{Định nghĩa}
\newtheorem{example}{Example}
\newtheorem{ghichu}{Ghi chú}
\newtheorem{hequa}{Hệ quả}
\newtheorem{hypothesis}{Hypothesis}
\newtheorem{lemma}{Lemma}
\newtheorem{luuy}{Lưu ý}
\newtheorem{nhanxet}{Nhận xét}
\newtheorem{notation}{Notation}
\newtheorem{note}{Note}
\newtheorem{principle}{Principle}
\newtheorem{problem}{Problem}
\newtheorem{proposition}{Proposition}
\newtheorem{question}{Question}
\newtheorem{remark}{Remark}
\newtheorem{theorem}{Theorem}
\newtheorem{vidu}{Ví dụ}
\usepackage[left=1cm,right=1cm,top=5mm,bottom=5mm,footskip=4mm]{geometry}
\def\labelitemii{$\circ$}
\DeclareRobustCommand{\divby}{%
	\mathrel{\vbox{\baselineskip.65ex\lineskiplimit0pt\hbox{.}\hbox{.}\hbox{.}}}%
}

\title{Congruent Triangles -- Các Tam Giác Bằng Nhau}
\author{Nguyễn Quản Bá Hồng\footnote{Independent Researcher, Ben Tre City, Vietnam\\e-mail: \texttt{nguyenquanbahong@gmail.com}; website: \url{https://nqbh.github.io}.}}
\date{\today}

\begin{document}
\maketitle
\begin{abstract}
	\textsc{[en]} This text is a collection of problems, from easy to advanced, about \textit{congruent triangles}. This text is also a supplementary material for my lecture note on Elementary Mathematics grade 7, which is stored \& downloadable at the following link: \href{https://github.com/NQBH/hobby/blob/master/elementary_mathematics/grade_7/NQBH_elementary_mathematics_grade_7.pdf}{GitHub\texttt{/}NQBH\texttt{/}hobby\texttt{/}elementary mathematics\texttt{/}grade 7\texttt{/}lecture}\footnote{\textsc{url}: \url{https://github.com/NQBH/hobby/blob/master/elementary_mathematics/grade_7/NQBH_elementary_mathematics_grade_7.pdf}.}. The latest version of this text has been stored \& downloadable at the following link: \href{https://github.com/NQBH/hobby/blob/master/elementary_mathematics/grade_7/congruent_triangle/NQBH_congruent_triangle.pdf}{GitHub\texttt{/}NQBH\texttt{/}hobby\texttt{/}elementary mathematics\texttt{/}grade 7\texttt{/}congruent triangle}\footnote{\textsc{url}: \url{https://github.com/NQBH/hobby/blob/master/elementary_mathematics/grade_7/congruent_triangle/NQBH_congruent_triangle.pdf}.}.
	\vspace{2mm}
	
	\textsc{[vi]} Tài liệu này là 1 bộ sưu tập các bài tập chọn lọc từ cơ bản đến nâng cao về \textit{các tam giác bằng nhau}. Tài liệu này là phần bài tập bổ sung cho tài liệu chính -- bài giảng \href{https://github.com/NQBH/hobby/blob/master/elementary_mathematics/grade_7/NQBH_elementary_mathematics_grade_7.pdf}{GitHub\texttt{/}NQBH\texttt{/}hobby\texttt{/}elementary mathematics\texttt{/}grade 7\texttt{/}lecture} của tác giả viết cho Toán Sơ Cấp lớp 7. Phiên bản mới nhất của tài liệu này được lưu trữ \& có thể tải xuống ở link sau: \href{https://github.com/NQBH/hobby/blob/master/elementary_mathematics/grade_7/congruent_triangle/NQBH_congruent_triangle.pdf}{GitHub\texttt{/}NQBH\texttt{/}hobby\texttt{/}elementary mathematics\texttt{/}grade 7\texttt{/}congruent triangle}.
	
	\textsf{\textbf{Nội dung.} Tổng các góc của 1 tam giác; quan hệ giữa góc \& cạnh đối diện trong 1 tam giác; bất đẳng thức tam giác; 2 tam giác bằng nhau; các trường hợp bằng nhau của 2 tam giác; tam giác cân; đường vuông góc \& đường xiên; đường trung trực của 1 đoạn thẳng; tính chất 3 đường trung tuyến, 3 đường phân giác, 3 đường trung trực, 3 đường cao của tam giác.}
\end{abstract}
\setcounter{secnumdepth}{4}
\setcounter{tocdepth}{3}
\tableofcontents

%------------------------------------------------------------------------------%

\section{Tổng Các Góc của 1 Tam Giác}giác
``\fbox{\bf 1} Tam giác $ABC$ là hình gồm 3 đoạn thẳng $AB,BC,CA$ khi 3 điểm $A,B,C$ không thẳng hàng. Tam giác $ABC$ được ký hiệu là $\Delta ABC$. 3 cạnh của tam giác: $AB,BC,CA$. 3 góc của tam giác: góc $A$, góc $B$, góc $C$. Nhận biết được điểm nằm trong \& điểm nằm ngoài 1 tam giác. \fbox{\bf 2} Tổng 3 góc của 1 tam giác bằng $180^\circ$: $\widehat{A} + \widehat{B} + \widehat{C} = 180^\circ$. \fbox{\bf 3} Tam giác vuông là tam giác có 1 góc vuông. Cạnh đối diện với góc vuông gọi là \textit{cạnh huyền}, cạnh huyền là cạnh lớn nhất trong tam giác vuông. 2 góc phụ nhau là 2 góc có tổng bằng $90^\circ$. Góc ngoài của 1 tam giác là góc kề bù với 1 góc trong của tam giác ấy. \fbox{\bf 4} 1 số hệ quả của định lý tổng 3 góc của tam giác:

\begin{hequa}
	Trong 1 tam giác vuông 2 góc nhọn phụ nhau. $\Delta ABC$, $\widehat{A} = 90^\circ\Rightarrow\widehat{B} + \widehat{C} = 90^\circ$.
\end{hequa}

\begin{hequa}
	1 góc ngoài của 1 tam giác bằng tổng 2 góc trong không kề với nó. 1 góc ngoài của 1 tam giác lớn hơn mỗi góc trong không kề với nó.
\end{hequa}
$\Delta ABC$, $\widehat{ACx}$ là góc ngoài tại đỉnh $C$: $\widehat{ACx} = \widehat{A} + \widehat{B}$, $\widehat{ACx} > \widehat{A}$, $\widehat{ACx} > \widehat{B}$ (có thể viết gộp 2 bất đẳng thức cuối thành $\widehat{ACx} > \min\{\widehat{A},\widehat{B}\}$ trong đó $\min\{a_1,a_2,\ldots,a_n\}$ là số nhỏ nhất trong $n$ số $a_i$, $i = 1,2,\ldots,n$). \fbox{\bf 5} Tam giác nhọn là tam giác có 3 góc nhọn. Tam giác tù là tam giác có 1 góc tù. Nếu 2 tam giác có 2 cặp góc bằng nhau từng đôi một thì cặp góc còn lại cũng bằng nhau.'' -- \cite[Chap. IV, \S1, p.65]{Tuyen_Toan_7}

\begin{baitoan}[\cite{SBT_Toan_7_Canh_Dieu_tap_2}, Ví dụ 1, p. 67]
	Tháp nghiêng Pisa ở Italy nghiêng $5^\circ$ so với phương thẳng đứng. Tính độ nghiêng của tháp đó so với phương nằm ngang.
\end{baitoan}

\begin{baitoan}[\cite{SBT_Toan_7_Canh_Dieu_tap_2}, 3., p. 68]
	(a) Cho biết 1 góc nhọn của tam giác vuông bằng $\alpha^\circ$, $\alpha\in(0,90)$. Tính số đo góc còn lại. (b) Cho 1 tam giác vuông có 2 góc bằng nhau. Tính số đo mỗi góc.
\end{baitoan}

\begin{baitoan}[\cite{SBT_Toan_7_Canh_Dieu_tap_2}, 4., p. 68]
	Đ hay S? Không có $\Delta ABC$ nào mà $\widehat{A} = 3\widehat{B}$, $\widehat{B} = 3\widehat{C}$, \& $C = 14^\circ$.
\end{baitoan}

\begin{baitoan}[\cite{Tuyen_Toan_7}, Ví dụ 15, p. 65]
	Cho 2 đường thẳng $a,b$ cắt nhau tại 1 điểm ở ngoài mép tờ giấy. Trong tay chỉ có thước đo góc, làm thế nào để đo được góc nhọn giữa 2 đường thẳng $a,b$ (đoạn thẳng $AB$ nằm trong góc đó).
\end{baitoan}
Để tính số đo 1 góc của tam giác ta lấy $180^\circ$ trừ đi tổng số đo của 2 góc còn lại.

\begin{baitoan}[\cite{Tuyen_Toan_7}, Ví dụ 16, p. 66]
	Cho $\Delta ABC$, các tia phân giác của góc $B$, góc $C$ cắt nhau tại $O$. Chứng minh: (a) $\widehat{BOC} = 90^\circ + \frac{\widehat{A}}{2}$; (b) Nếu $\widehat{BOC} = 135^\circ$ thì $\Delta ABC$ vuông tại $A$.
\end{baitoan}

\begin{baitoan}[\cite{Tuyen_Toan_7}, 58., p. 66]
	Cho $\Delta ABC$ vuông tại $A$. Trên tia đối tia $CA$ lấy điểm $E$ khác $C$. Gọi $D$ là hình chiếu vuông góc của $E$ lên đường thẳng $BC$. Chứng minh: $\widehat{B} = \widehat{CED}$.
\end{baitoan}

\begin{baitoan}[\cite{Tuyen_Toan_7}, 59., p. 66]
	Cho $\Delta ABC$ vuông tại $A$, $\widehat{C} = 25^\circ$. Tia phân giác của góc $A$ cắt $BC$ tại $D$. Vẽ $AH\bot BC$. Tính $\widehat{HAD}$.
\end{baitoan}

\begin{baitoan}[\cite{Tuyen_Toan_7}, 60., p. 66]
	Cho $\Delta ABC$ vuông tại $A$. Tia phân giác của góc $C$ cắt $AB$ tại $D$. (a) Chứng minh góc $BDC$ là góc tù. (b) Giả sử $\widehat{BDC} = 105^\circ$, tính $\widehat{B}$.
\end{baitoan}

\begin{baitoan}[\cite{Tuyen_Toan_7}, 61., p. 66]
	Cho $\Delta ABC$ \& điểm $O$ nằm trong tam giác đó. So sánh góc $BOC$ \& $BAC$.
\end{baitoan}

\begin{baitoan}[\cite{Tuyen_Toan_7}, 62., p. 66]
	Cho $\Delta ABC$ vuông tại $A$. Vẽ $AH\bot BC$. Vẽ các tia phân giác của góc $B$ \& góc $HAC$ cắt nhau tại $O$. Chứng minh $\Delta AOB$ là tam giác vuông.
\end{baitoan}

\begin{baitoan}[\cite{Tuyen_Toan_7}, 63., p. 66]
	Chứng minh với mỗi tam giác bao giờ cũng tồn tại 1 góc ngoài không lớn hơn $120^\circ$.
\end{baitoan}

\begin{baitoan}[\cite{Tuyen_Toan_7}, 64., pp. 66--67]
	Cho $\Delta ABC$ có $\widehat{B} > \widehat{C}$. Vẽ tia phân giác của góc $A$ cắt $BC$ tại $D$. (a) Chứng minh $\widehat{ADC} - \widehat{ADB} = \widehat{ABC} - \widehat{C}$. (b) Đường thẳng chứa tia phân giác ngoài tại đỉnh $A$ của $\Delta ABC$ cắt đường thẳng $BC$ tại $E$. Chứng minh $\widehat{AEB} = \frac{\widehat{ABC} - \widehat{C}}{2}$.
\end{baitoan}

\begin{baitoan}[\cite{Tuyen_Toan_7}, 65., p. 66]
	Trên lá cờ đỏ sao vàng của Việt Nam có ngôi sao $5$ cánh. Tính tổng các góc ở $5$ đỉnh của ngôi sao đó.
\end{baitoan}

%------------------------------------------------------------------------------%

\section{Quan Hệ Giữa Góc \& Cạnh Đối Diện. Bất Đẳng Thức Tam Giác}

\begin{baitoan}[\cite{SGK_Toan_7_Canh_Dieu_tap_2}, p. 75]
	Cho $\Delta ABC$ có $AB = 2$\emph{cm}, $BC = 4$\emph{cm}. So sánh $AB$ \& $AC$.
\end{baitoan}

\begin{baitoan}
	Đ hay sai? (a) Nếu 1 tam giác có 1 cạnh dài gấp đôi 1 cạnh khác, thì 2 cạnh đó lần lượt là cạnh dài nhất \& ngắn nhất của tam giác đó. (b) Nếu 1 tam giác có 1 cạnh dài hơn gấp đôi 1 cạnh khác, thì 2 cạnh đó lần lượt là cạnh dài nhất \& ngắn nhất của tam giác đó.
\end{baitoan}

%------------------------------------------------------------------------------%

\section{2 Tam Giác Bằng Nhau}

\begin{dinhnghia}[2 tam giác bằng nhau]
	\emph{2 tam giác bằng nhau} là 2 tam giác có các cạnh tương ứng bằng nhau, các góc tương ứng bằng nhau.
\end{dinhnghia}

%------------------------------------------------------------------------------%

\section{Trường Hợp Bằng Nhau Thứ 1 của Tam Giác: Cạnh - Cạnh - Cạnh}

\begin{dinhly}
	Nếu 3 cạnh của tam giác này bằng 3 cạnh của tam giác kia thì 2 tam giác đó bằng nhau.
\end{dinhly}
\begin{align*}
	\Delta ABC = \Delta A'B'C'&\Leftrightarrow AB = A'B', BC = B'C', CA = C'A', \widehat{A} = \widehat{A'}, \widehat{B} = \widehat{B'}, \widehat{C} = \widehat{C'},\\
	\Delta ABC = \Delta A'B'C'&\Leftrightarrow AB = A'B', BC = B'C', CA = C'A',\\
	\Delta ABC = \Delta A'B'C'&\not\Leftrightarrow\widehat{A} = \widehat{A'}, \widehat{B} = \widehat{B'}, \widehat{C} = \widehat{C'}.
\end{align*}

\begin{luuy}
	2 tam giác có các cặp góc tương ứng bằng nhau chưa chắc đã bằng nhau: 2 tam giác đó chỉ đồng dạng, i.e., cùng hình dạng nhưng khác nhau về kích cỡ.
\end{luuy}

\begin{baitoan}[\cite{SGK_Toan_7_Canh_Dieu_tap_2}, Ví dụ 2, p. 81]
	Cho góc $xOy$. (a) Dùng thước thẳng (có chia đơn vị) \& compa vẽ hình theo các bước sau: Vẽ 1 phần đường tròn tâm $O$ bán kính $2$\emph{cm} cắt $Ox,Oy$ lần lượt tại $A,B$. Vẽ 1 phần đường tròn tâm $A$ bán kính $3$\emph{cm}. Vẽ 1 phần đường tròn tâm $B$ bán kính $3$\emph{cm} cắt phần đường tròn tâm $A$ bán kính $3$\emph{cm} tại $C$ nằm trong góc $xOy$. Vẽ tia $Oz$ đi qua điểm $C$. (b) Chứng minh: $\Delta OAC = \Delta OBC$. Tia $Oz$ là tia phân giác của góc $xOy$.
\end{baitoan}

\begin{baitoan}
	Chứng minh: Nếu cạnh huyền \& 1 cạnh góc vuông của tam giác vuông này bằng cạnh huyền \& 1 cạnh góc vuông của tam giác vuông kia thì 2 tam giác vuông đó bằng nhau.
\end{baitoan}

\begin{dinhly}
	Nếu cạnh huyền \& 1 cạnh góc vuông của tam giác vuông này bằng cạnh huyền \& 1 cạnh góc vuông của tam giác vuông kia thì 2 tam giác vuông đó bằng nhau.
\end{dinhly}

\begin{baitoan}
	Nếu 2 cặp cạnh góc vuông tương ứng của 2 tam giác vuông bằng nhau thì 2 tam giác vuông đó có bằng nhau hay không?
\end{baitoan}

\begin{baitoan}[\cite{SGK_Toan_7_Canh_Dieu_tap_2}, Ví dụ 3, p. 82]
	Cho $\Delta ABC$ có $AB = AC$, $AH\bot BC$. Chứng minh: (a) $\Delta AHB = \Delta AHC$. (b) $AH$ là tia phân giác của góc $BAC$.
\end{baitoan}

\begin{baitoan}[\cite{SGK_Toan_7_Canh_Dieu_tap_2}, \textbf{1.}, p. 83]
	Cho tứ giác $MNPQ$ sao cho $MN = QN$, $MP = QP$. Chứng minh $\widehat{MNP} = \widehat{QNP}$, $\widehat{MPN} = \widehat{QPN}$, $\widehat{NMP} = \widehat{NQP}$. 
\end{baitoan}

\begin{baitoan}[\cite{SGK_Toan_7_Canh_Dieu_tap_2}, \textbf{2.}, p. 83]
	Cho tứ giác $ABCD$ có $AB = AD$, $\widehat{ABC} = \widehat{ADC} = 90^\circ$. Chứng minh $\widehat{ACB} = \widehat{ACD}$, $\widehat{BAC} = \widehat{DAC}$, $BC = CD$.
\end{baitoan}

%------------------------------------------------------------------------------%

\section{Trường Hợp Bằng Nhau Thứ 2 của Tam Giác: Cạnh - Góc - Cạnh}

%------------------------------------------------------------------------------%

\section{Trường Hợp Bằng Nhau Thứ 3 của Tam Giác: Góc - Cạnh - Góc}

\begin{baitoan}[\cite{SGK_Toan_7_Canh_Dieu_tap_2}, 1., p. 91]
	Cho $\Delta ABC,\Delta A'B'C'$ thỏa mãn: $AB = A'B'$, $\widehat{A} = \widehat{A'}$, $\widehat{C} = \widehat{C'}$. $\Delta ABC,\Delta A'B'C'$ có bằng nhau không? Vì sao?
\end{baitoan}

\begin{proof}[1st giải]
	Có $\widehat{B} = 180^\circ - \widehat{A} - \widehat{C} = 180^\circ - \widehat{A'} - \widehat{C'} = \widehat{B'}$. Vì $\widehat{A} = \widehat{A'}$, $\widehat{B} = \widehat{B'}$, $AB = A'B'$ nên $\Delta ABC = \Delta A'B'C'$ (g.c.g).
\end{proof}

\begin{proof}[2nd giải]
	Vì $\widehat{B} = 180^\circ - \widehat{A} - \widehat{C}$, $\widehat{B'} = 180^\circ - \widehat{A'} - \widehat{C'}$, mà $\widehat{A} = \widehat{A'}$, $\widehat{C} = \widehat{C'}$, nên $\widehat{B} = \widehat{B'}$. Xét $\Delta ABC$ \& $\Delta A'B'C'$: $\widehat{A} = \widehat{A'}$, $\widehat{B} = \widehat{B'}$, $AB = A'B'$. Suy ra $\Delta ABC = \Delta A'B'C'$ (g.c.g).
\end{proof}

\begin{baitoan}[\cite{SGK_Toan_7_Canh_Dieu_tap_2}, 2., p. 92]
	Cho tứ giác $ANBM$ có 2 đường chéo $AB,MN$ cắt nhau tại $O$, $AM = BN$, $\widehat{OAM} = \widehat{OBN}$. Chứng minh $OA = OB$, $OM = ON$, $\widehat{OMA} = \widehat{ONB}$.
\end{baitoan}

\begin{proof}[1st giải]
	Vì $\widehat{M} = 180^\circ - \widehat{AOM} - \widehat{A}$, $\widehat{N} = 180^\circ - \widehat{BON} - \widehat{B}$, mà $\widehat{AOM} = \widehat{BON}$ (2 góc đối đỉnh), $\widehat{A} = \widehat{B}$, nên $\widehat{M} = \widehat{N}$. Xét $\Delta AOM$ \& $\Delta BON$: $AM = BN$, $\widehat{A} = \widehat{B}$, $\widehat{M} = \widehat{N}$. Suy ra $\Delta AOM = \Delta BON$, suy ra $OA = OB$, $OM = ON$, $\widehat{OMA} = \widehat{ONB}$.
\end{proof}

\begin{proof}[2nd giải]
	$\widehat{A} = \widehat{B}\Rightarrow AM\parallel BN$ (2 góc so le trong). $AM\parallel BN\Rightarrow\widehat{M} = \widehat{N}$ (2 góc so le trong). Xét $\Delta AOM$ \& $\Delta BON$: $AM = BN$, $\widehat{A} = \widehat{B}$, $\widehat{M} = \widehat{N}$. Suy ra $\Delta AOM = \Delta BON$, suy ra $OA = OB$, $OM = ON$, $\widehat{OMA} = \widehat{ONB}$.
\end{proof}

\begin{baitoan}[\cite{SGK_Toan_7_Canh_Dieu_tap_2}, 3., p. 92]
	Cho tứ giác $MNPQ$, $\widehat{MNQ} = \widehat{MPQ} = 90^\circ$, $\widehat{NQM} = \widehat{PMQ}$. Chứng minh $MN = PQ$, $MP = NQ$.
	\begin{figure}[H]
		\centering
		\includegraphics[scale=0.15]{66}
		\caption{\cite{SGK_Toan_7_Canh_Dieu_tap_2}, Hình 66, p. 92.}
	\end{figure}
\end{baitoan}

\begin{proof}[1st giải]
	Xét $\Delta NMQ$ \& $\Delta PQM$: $\widehat{MNQ} = \widehat{MPQ} = 90^\circ$, $\widehat{NQM} = \widehat{PMQ}$, $MQ$ cạnh chung. Suy ra $\Delta NMQ = \Delta PQM$ (ch.g), suy ra $MN = PQ$, $MP = NQ$.
\end{proof}

\begin{baitoan}[\cite{SGK_Toan_7_Canh_Dieu_tap_2}, 4., p. 92]
	Cho hình sau có $\widehat{AHD} = \widehat{BKC} = 90^\circ$, $DH = CK$, $\widehat{DAB} = \widehat{CBA}$. Chứng minh $AD = BC$.
	\begin{figure}[H]
		\centering
		\includegraphics[scale=0.15]{67}
		\caption{\cite{SGK_Toan_7_Canh_Dieu_tap_2}, Hình 67, p. 92.}
	\end{figure}
\end{baitoan}

\begin{baitoan}[\cite{SGK_Toan_7_Canh_Dieu_tap_2}, 5., p. 92]
	Cho $\Delta ABC$ có $\widehat{B} > \widehat{C}$. Tia phân giác góc $BAC$ cắt cạnh $BC$ tại điểm $D$. (a) Chứng minh $\widehat{ADB} < \widehat{ADC}$. (b) Kẻ tia $Dx$ nằm trong góc $ADC$ sao cho $\widehat{ADx} = \widehat{ADB}$. Giả sử tia $Dx$ cắt cạnh $AC$ tại điểm $E$. Chứng minh $\Delta ABD = \Delta AED$, $AB < AC$.
\end{baitoan}

\begin{baitoan}[\cite{SGK_Toan_7_Canh_Dieu_tap_2}, 6., p. 92]
	Cho $\Delta ABC = \Delta MNP$. Tia phân giác của góc $BAC$ \& $NMP$ lần lượt cắt các cạnh $BC,NP$ tại $D,Q$. Chứng minh $AD = MQ$.
\end{baitoan}

%------------------------------------------------------------------------------%

\section{Tam Giác Cân}

%------------------------------------------------------------------------------%

\section{Đường Vuông Góc \& Đường Xiên}

%------------------------------------------------------------------------------%

\section{Đường Trung Trực của 1 Đoạn Thẳng}

%------------------------------------------------------------------------------%

\section{Tính Chất 3 Đường Trung Tuyến của Tam Giác}

%------------------------------------------------------------------------------%

\section{Tính Chất 3 Đường Phân Giác của Tam Giác}

%------------------------------------------------------------------------------%

\section{Tính Chất 3 Đường Trung Trực của Tam Giác}

%------------------------------------------------------------------------------%

\section{Tính Chất 3 Đường Cao của Tam Giác}

%------------------------------------------------------------------------------%

\printbibliography[heading=bibintoc]
	
\end{document}