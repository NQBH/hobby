\documentclass{article}
\usepackage[backend=biber,natbib=true,style=authoryear]{biblatex}
\addbibresource{/home/hong/1_NQBH/reference/bib.bib}
\usepackage[utf8]{vietnam}
\usepackage{tocloft}
\renewcommand{\cftsecleader}{\cftdotfill{\cftdotsep}}
\usepackage[colorlinks=true,linkcolor=blue,urlcolor=red,citecolor=magenta]{hyperref}
\usepackage{amsmath,amssymb,amsthm,mathtools,float,graphicx,algpseudocode,algorithm,tcolorbox}
\usepackage[inline]{enumitem}
\allowdisplaybreaks
\numberwithin{equation}{section}
\newtheorem{assumption}{Assumption}[section]
\newtheorem{conjecture}{Conjecture}[section]
\newtheorem{corollary}{Corollary}[section]
\newtheorem{hequa}{Hệ quả}[section]
\newtheorem{definition}{Definition}[section]
\newtheorem{dinhnghia}{Định nghĩa}[section]
\newtheorem{example}{Example}[section]
\newtheorem{vidu}{Ví dụ}[section]
\newtheorem{lemma}{Lemma}[section]
\newtheorem{notation}{Notation}[section]
\newtheorem{principle}{Principle}[section]
\newtheorem{problem}{Problem}[section]
\newtheorem{baitoan}{Bài toán}[section]
\newtheorem{proposition}{Proposition}[section]
\newtheorem{question}{Question}[section]
\newtheorem{cauhoi}{Câu hỏi}[section]
\newtheorem{remark}{Remark}[section]
\newtheorem{luuy}{Lưu ý}[section]
\newtheorem{theorem}{Theorem}[section]
\newtheorem{dinhly}{Định lý}[section]
\usepackage[left=0.5in,right=0.5in,top=1.5cm,bottom=1.5cm]{geometry}
\usepackage{fancyhdr}
\pagestyle{fancy}
\fancyhf{}
\lhead{\small Subsect.~\thesubsection}
\rhead{\small\nouppercase{\leftmark}}
\renewcommand{\subsectionmark}[1]{\markboth{#1}{}}
\cfoot{\thepage}
\def\labelitemii{$\circ$}

\title{Some Topics in Elementary Mathematics\texttt{/}Grade 7}
\author{Nguyễn Quản Bá Hồng\footnote{Independent Researcher, Ben Tre City, Vietnam\\e-mail: \texttt{nguyenquanbahong@gmail.com}; website: \url{https://nqbh.github.io}.}}
\date{\today}

\begin{document}
\maketitle
\begin{abstract}
	1 bộ sưu tập các bài toán chọn lọc từ cơ bản đến nâng cao cho Toán sơ cấp lớp 7. Tài liệu này là phần bài tập bổ sung cho tài liệu chính \href{https://github.com/NQBH/hobby/blob/master/elementary_mathematics/grade_7/NQBH_elementary_mathematics_grade_7.pdf}{GitHub\texttt{/}NQBH\texttt{/}hobby\texttt{/}elementary mathematics\texttt{/}grade 7\texttt{/}lecture}\footnote{\textsc{url}: \url{https://github.com/NQBH/hobby/blob/master/elementary_mathematics/grade_7/NQBH_elementary_mathematics_grade_7.pdf}.} của tác giả viết cho Toán lớp 7. Phiên bản mới nhất của tài liệu này được lưu trữ ở link sau: \href{https://github.com/NQBH/hobby/blob/master/elementary_mathematics/grade_7/problem/NQBH_elementary_mathematics_grade_7_problem.pdf}{GitHub\texttt{/}NQBH\texttt{/}hobby\texttt{/}elementary mathematics\texttt{/}grade 7\texttt{/}problem}\footnote{\textsc{url}: \url{https://github.com/NQBH/hobby/blob/master/elementary_mathematics/grade_7/problem/NQBH_elementary_mathematics_grade_7_problem.pdf}.}.
\end{abstract}
\tableofcontents
\newpage

%------------------------------------------------------------------------------%

\section{Số Hữu Tỷ}

\subsection{Tập Hợp Các Số Hữu Tỷ $\mathbb{Q}$}
Với phân số $\frac{a}{b}$ tối giản, $a,\in\mathbb{Z}$, $b\ne 0$, $\mbox{ƯCLN}(a,b) = 1$ thì các phân số có dạng $\frac{na}{nb}$, $\forall n\in\mathbb{Z}^\star$, đều biểu diễn phân số $\frac{a}{b}$. Để so sánh 2 hay nhiều số hữu tỷ, chuyển chúng về cùng 1 trong 2 dạng: dạng phân số hoặc dạng biểu diễn thập phân, rồi so sánh chúng dựa vào các quy tắc đã học ở Toán 7.

\begin{baitoan}[\cite{Trong_Toan_7_2022}, \textbf{10.}, p. 6]
	So sánh 2 số hữu tỷ $\frac{a}{b}$ ($a,b\in\mathbb{Z}$, $b\ne 0$) với số $0$ khi $a,b$ cùng dấu \& khi $a,b$ khác dấu.
\end{baitoan}

\begin{proof}[Giải]
	$a = 0\Rightarrow\frac{a}{b} = 0$, $ab > 0\Rightarrow\frac{a}{b} > 0$, \& $ab < 0\Rightarrow\frac{a}{b} < 0$.
\end{proof}

\begin{baitoan}[\cite{Trong_Toan_7_2022}, \textbf{11.}, p. 6]
	Giả sử $x = \frac{a}{m},y = \frac{b}{m},\ (a,b,m\in\mathbb{Z},\,m > 0)$ \& $x < y$. Chứng minh $x < z < y$ với $z\coloneqq\frac{a + b}{2m}$ ($z$ là \emph{trung bình cộng} của $x$ \& $y$, i.e., $z = \frac{x + y}{2}$).
\end{baitoan}

\begin{proof}[Chứng minh]
	$x < y\Rightarrow x + x < x + y < y + y\Rightarrow x < \frac{x + y}{2} < y$, mà $\frac{x + y}{2} = \frac{1}{2}\left(\frac{a}{m} + \frac{b}{m}\right) = \frac{a + b}{2m} = z$, nên $x < z < y$.
\end{proof}

%------------------------------------------------------------------------------%

\subsection{$\pm$ trên $\mathbb{Q}$}
Tính tổng các phân số cùng mẫu số:
\begin{align*}
	\sum_{i=1}^{n} \frac{a_i}{b} = \frac{\sum_{i=1}^n a_i}{b},\mbox{ i.e., } \frac{a_1}{b} + \cdots + \frac{a_n}{b} = \frac{a_1 + \cdots + a_n}{b},\ \forall a_i,b\in\mathbb{Z},\,b\ne 0,\,\forall i = 1,\ldots,n.
\end{align*}
Tính tổng các phân số khác mẫu số: Quy đồng mẫu số các phân số đó với mẫu số chung là BCNN của các mẫu số các phân số đó rồi cộng lại:
\begin{align*}
	\sum_{i=1}^{n} \frac{a_i}{b_i} = \frac{\sum_{i=1}^n a_i\frac{\operatorname{BCNN}(b_1,\ldots,b_n)}{b_i}}{\operatorname{BCNN}(b_1,\ldots,b_n)},\mbox{i.e., }\frac{a_1}{b_1} + \cdots + \frac{a_n}{b_n} = \frac{a_1\frac{\operatorname{BCNN}(b_1,\ldots,b_n)}{b_1} + \cdots + a_n\frac{\operatorname{BCNN}(b_1,\ldots,b_n)}{b_n}}{\operatorname{BCNN}(b_1,\ldots,b_n)},&\\\forall a_i,b_i\in\mathbb{Z},\,b_i\ne 0,\,\forall i = 1,\ldots,n.&
\end{align*}

\begin{baitoan}[\cite{Binh_Toan_7_tap_1}, \S1, \textbf{2.}]
	Tìm $2$ phân số có tử bằng $9$, biết rằng giá trị của mỗi phân số đó lớn hơn $\frac{-11}{13}$ \& nhỏ hơn $\frac{-11}{15}$.
\end{baitoan}

\begin{baitoan}[\cite{Binh_Toan_7_tap_1}, \S1, \textbf{3.}]
	Cho các số hữu tỷ $\frac{a}{b}$ \& $\frac{c}{d}$ với mẫu dương, trong đó $\frac{a}{b} < \frac{c}{d}$. Chứng minh:
	\begin{enumerate*}
		\item[(a)] $ab < bc$;
		\item[(b)] $\frac{a}{b} < \frac{a + c}{b + d} < \frac{c}{d}$.
	\end{enumerate*}
\end{baitoan}

\begin{baitoan}[\cite{Binh_Toan_7_tap_1}, \S1, \textbf{4.}]
	Ký hiệu $\lfloor x\rfloor$ là số nguyên lớn nhất không vượt quá $x$, được gọi là \emph{phần nguyên} của $x$, e.g., $\lfloor 1.5\rfloor = 1$, $\lfloor 5\rfloor = 5$, $\lfloor -2.5\rfloor = -3$.
	\begin{enumerate*}
		\item[(a)] Tính $\lfloor-\frac{1}{7}\rfloor,\lfloor 3.7\rfloor,\lfloor-4\rfloor,\lfloor-\frac{43}{10}\rfloor$.
		\item[(b)] Cho $x = 3.7$. So sánh: $A = \lfloor x\rfloor + \lfloor x + \frac{1}{5}\rfloor + \lfloor x + \frac{2}{5}\rfloor + \lfloor x + \frac{3}{5}\rfloor$ $+ \lfloor x + \frac{4}{5}\rfloor$ \& $B = \lfloor 5x\rfloor$.
		\item[(c)] Tính $ \lfloor\frac{100}{3}\rfloor + \lfloor\frac{100}{3^2}\rfloor + \lfloor\frac{100}{3^3}\rfloor + \lfloor\frac{100}{3^4}\rfloor$.
		\item[(d)] Tính $ \lfloor\frac{50}{2}\rfloor + \lfloor\frac{50}{2^2}\rfloor + \lfloor\frac{50}{2^3}\rfloor + \lfloor\frac{50}{2^4}\rfloor + \lfloor\frac{50}{2^5}\rfloor$.
		\item[(e)] Cho $x\in\mathbb{Q}$. So sánh $\lfloor x\rfloor$ với $x$, so sánh $\lfloor x\rfloor$ với $y$ trong đó $y\in\mathbb{Z}$, $y < x$.
	\end{enumerate*}
\end{baitoan}

%------------------------------------------------------------------------------%

\subsection{$\cdot,:$ trên $\mathbb{Q}$}
``\textit{Phép nhân 2 hay nhiều số hữu tỷ}:
\begin{enumerate*}
	\item[$\bullet$] Xác định dấu bằng cách đếm các thừa số âm, nếu chẵn thì kết quả dương, nếu lẻ thì kết quả âm.
	\item[$\bullet$] Nhân phần số tự nhiên của tử với tử, mẫu với mẫu rồi rút gọn.
\end{enumerate*}

\noindent\textit{Phép chia 2 số hữu tỷ}: Ta lấy số hữu tỷ bị chia nhân với nghịch đảo số hữu tỷ chia rồi làm như phép nhân. Thương của phép chia $x\in\mathbb{Q}$ cho số hữu tỷ $y\in\mathbb{Q}^\star$ gọi là \textit{tỷ số} của 2 số $x$ \& $y$, ký hiệu là $\frac{x}{y}$ hay $x:y$.'' -- \cite[\S3, p. 10]{Trong_Toan_7_2022}

\begin{baitoan}[\cite{Trong_Toan_7_2022}, \textbf{3.}, p. 13]
	Cho số hữu tỷ $\frac{a}{b}$ với $a,b\in\mathbb{Z}$, $b > 0$. Chứng minh:
	\begin{enumerate*}
		\item[(a)] $\frac{a}{b} > 1\Leftrightarrow a > b$;
		\item[(b)] $\frac{a}{b} < 1\Leftrightarrow a < b$;
		\item[(c)] $((a < b)\land(a,c > 0))\Rightarrow\frac{a}{b} < \frac{a + c}{b + c}$;
		\item[(d)] $((a > b)\land(c > 0))\Rightarrow\frac{a}{b} > \frac{a + c}{b + c}$.
	\end{enumerate*}
\end{baitoan}

\begin{baitoan}[\cite{Binh_Toan_7_tap_1}, \S1, Ví dụ 1]
	Cho phân số $\frac{a}{b}\ne 1$. Tìm phân số $\frac{c}{d}$ sao cho $\frac{a}{b} + \frac{c}{d} = \frac{a}{b}\cdot\frac{c}{d}$.
\end{baitoan}

\begin{baitoan}[\cite{Binh_Toan_7_tap_1}, \S1, \textbf{5.}]
	Thực hiện các phép tính:
	\begin{enumerate*}
		\item[(a)] $\frac{-2}{3} + \frac{3}{4} - \frac{-1}{6} + \frac{-2}{5}$;
		\item[(b)] $\frac{-2}{3} + \frac{-1}{5} + \frac{3}{4} - \frac{5}{6} - \frac{-7}{10}$;
		\item[(c)] $\frac{1}{2} - \frac{-2}{5} + \frac{1}{3} + \frac{5}{7} - \frac{-1}{6} + \frac{-4}{35} + \frac{1}{41}$;
		\item[(d)] $\frac{1}{100\cdot 99} - \frac{1}{99\cdot 98} - \frac{1}{98\cdot 97} - \cdots - \frac{1}{3\cdot 2} - \frac{1}{2\cdot 1}$.
	\end{enumerate*}
\end{baitoan}

\begin{baitoan}[\cite{Binh_Toan_7_tap_1}, \S1, \textbf{6.}]
	Cho các số hữu tỷ $x$ bằng $1.4089,0.1398,-0.4771,-1.2592$.
	\begin{enumerate*}
		\item[(a)] Viết các số đó dưới dạng tổng của 1 số nguyên $a$ \& 1 số thập phân $b$ không âm nhỏ hơn $1$.\footnote{Trong cách viết này, $a$ là phần nguyên của $x$, còn $b$ là phần lẻ của $x$. Ký hiệu phần lẻ của $x$ là $\{x\}$ thì $x = \lfloor x\rfloor + \{x\}$.}
		\item[(b)] Tính tổng các số hữu tỷ trên bằng cách $2$ cách: tính thông thường, tính tổng các số được viết dưới dạng ở câu (a).
		\item[(c)] So sánh $a$ \& $\lfloor x\rfloor$ trong từng trường hợp ở câu (a).
	\end{enumerate*}
\end{baitoan}

\begin{baitoan}[\cite{Binh_Toan_7_tap_1}, \S1, \textbf{7.}]
	Tìm $n\in\mathbb{Z}$ để phân số sau có giá trị là 1 số nguyên \& tính giá trị đó:
	\begin{enumerate*}
		\item[(a)] $A = \frac{3n + 9}{n - 4}$;
		\item[(b)] $B = \frac{6n + 5}{2n - 1}$.
	\end{enumerate*}
\end{baitoan}

\begin{baitoan}[\cite{Binh_Toan_7_tap_1}, \S1, \textbf{8.}]
	Tìm $x,y\in\mathbb{Z}$, biết: $\frac{5}{x} + \frac{y}{4} = \frac{1}{8}$.
\end{baitoan}

\begin{baitoan}[\cite{Binh_Toan_7_tap_1}, \S1, \textbf{9.}]
	Viết tất cả các số nguyên có giá trị tuyệt đối nhỏ hơn $20$ theo thứ tự tùy ý. Lấy mỗi số trừ đi số thứ tự của nó ta được 1 hiệu. Tổng của tất cả các hiệu đó bằng bao nhiêu?
\end{baitoan}

\begin{baitoan}[\cite{Binh_Toan_7_tap_1}, \S1, \textbf{10.}]
	Tính:
	\begin{enumerate*}
		\item[(a)] $\dfrac{\left(\frac{3}{10} - \frac{4}{15} - \frac{7}{20}\right)\cdot\frac{5}{19}}{\left(\frac{1}{14} + \frac{1}{7} - \frac{-3}{35}\right)\cdot\frac{-4}{3}}$;
		\item[(b)] $\dfrac{(1 + 2 + \cdots + 100)\left(\frac{1}{3} - \frac{1}{5} - \frac{1}{7} - \frac{1}{9}\right)\cdot(6.3\cdot 12 - 21\cdot 3.6)}{\frac{1}{2} + \frac{1}{3} + \cdots + \frac{1}{100}}$;
		\item[(c)] $\dfrac{\frac{1}{9} - \frac{1}{7} - \frac{1}{11}}{\frac{4}{9} - \frac{4}{7} - \frac{4}{11}} + \dfrac{\frac{3}{5} - \frac{3}{25} - \frac{3}{125} - \frac{3}{625}}{\frac{4}{5} - \frac{4}{25} - \frac{4}{125} - \frac{4}{625}}$.
	\end{enumerate*}
\end{baitoan}

\begin{baitoan}[\cite{Binh_Toan_7_tap_1}, \S1, \textbf{11.}]
	Tìm $x\in\mathbb{Q}$, biết:
	\begin{enumerate*}
		\item[(a)] $\frac{2}{3}x + 4 = -12$;
		\item[(b)] $\frac{3}{4} + \frac{1}{4}:x = -3$;
		\item[(c)] $|3x - 5| = 4$;
		\item[(d)] $\frac{x + 1}{10} + \frac{x + 1}{11} + \frac{x + 1}{12} = \frac{x + 1}{13} + \frac{x + 1}{14}$;
		\item[(e)] $\frac{x + 4}{2000} + \frac{x + 3}{2001} = \frac{x + 2}{2002} + \frac{x + 1}{2003}$.
	\end{enumerate*}
\end{baitoan}

\begin{baitoan}[\cite{Binh_Toan_7_tap_1}, \S1, \textbf{12.}]
	Chứng minh $\sum_{i=1}^{99} \frac{i}{(i+1)!} = \frac{1}{2!} + \frac{2}{3!} + \frac{3}{4!} + \cdots + \frac{99}{100!} < 1$.
\end{baitoan}

\begin{baitoan}[\cite{Binh_Toan_7_tap_1}, \S1, \textbf{13.}]
	Chứng minh $\sum_{i=1}^{99} \frac{i(i + 1) - 1}{(i+1)!} = \frac{1\cdot 2 - 1}{2!} + \frac{2\cdot 3 - 1}{3!} + \frac{3\cdot 4 - 1}{4!} + \cdots + \frac{99\cdot 100 - 1}{100!} < 2$.
\end{baitoan}

\begin{baitoan}[\cite{Binh_Toan_7_tap_1}, \S1, \textbf{14.}]
	\begin{enumerate*}
		\item[(a)] Người ta viết $7$ số hữu tỷ trên 1 vòng tròn. Tìm các số đó, biết rằng tích của $2$ số bất kỳ cạnh nhau bằng $16$.
		\item[(b)] Cũng hỏi như trên đối với $n$ số.
	\end{enumerate*}
\end{baitoan}

\begin{baitoan}[\cite{Binh_Toan_7_tap_1}, \S1, \textbf{15.}]
	Có tồn tại hay không $2$ số dương $a,b$ khác nhau sao cho $\frac{1}{a} - \frac{1}{b} = \frac{1}{a - b}$?
\end{baitoan}

\begin{baitoan}[\cite{Binh_Toan_7_tap_1}, \S1, \textbf{16.}]
	Chứng minh: $\frac{1}{1\cdot 2} + \frac{1}{3\cdot 4} + \frac{1}{5\cdot 6} + \cdots + \frac{1}{49\cdot 50} = \frac{1}{26} + \frac{1}{27} + \frac{1}{28} + \cdots + \frac{1}{50}$.
\end{baitoan}

\begin{baitoan}[\cite{Binh_Toan_7_tap_1}, \S1, \textbf{17.}]
	Cho $A = \frac{1}{1\cdot 2} + \frac{1}{3\cdot 4} + \frac{1}{5\cdot 6} + \cdots + \frac{1}{99\cdot 100}$. Chứng minh $\frac{7}{12} < A < \frac{5}{6}$.
\end{baitoan}

\begin{baitoan}[\cite{Binh_Toan_7_tap_1}, \S1, \textbf{18.}]
	Tìm $a,b\in\mathbb{Q}$ sao cho: $a - b = 2(a + b) = a:b$.
\end{baitoan}

\begin{baitoan}[\cite{Binh_Toan_7_tap_1}, \S1, \textbf{19.}]
	Tìm $a,b\in\mathbb{Q}$ sao cho $a + b = ab = a:b$.
\end{baitoan}

\begin{baitoan}[\cite{Binh_Toan_7_tap_1}, \S1, \textbf{20.}]
	Tìm $x\in\mathbb{Q}$, sao cho tổng của số đó với số nghịch đảo của nó là 1 số nguyên.
\end{baitoan}

%------------------------------------------------------------------------------%

\subsection{Lũy thừa của 1 số hữu tỷ}

\begin{baitoan}[\cite{Binh_Toan_7_tap_1}, \S3, Ví dụ 2]
	Cho $x\in\mathbb{Q}$. Khi nào thì:
	\begin{enumerate*}
		\item[(a)] $x^2 = x$;
		\item[(b)] $x^2 > x$;
		\item[(c)] $x^2 < x$.
	\end{enumerate*}
\end{baitoan}

\begin{baitoan}[\cite{Binh_Toan_7_tap_1}, \S3, Ví dụ 3]
	Tìm $a,b,c\in\mathbb{Q}$ biết $ab = 2$, $bc = 3$, $ca = 54$.
\end{baitoan}

\begin{baitoan}[\cite{Binh_Toan_7_tap_1}, \S3, Ví dụ 4]
	Rút gọn $A = \sum_{i=0}^{50} 5^i = 1 + 5 + 5^2 + \cdots + 5^{50}$.
\end{baitoan}

\begin{baitoan}[\cite{Binh_Toan_7_tap_1}, \S3, Ví dụ 5]
	Cho $B = \sum_{i=1}^{99} \left(\frac{1}{2}\right)^i = \frac{1}{2} + \left(\frac{1}{2}\right)^2 + \cdots + \left(\frac{1}{2}\right)^{99}$. Chứng minh $B < 1$.
\end{baitoan}

\begin{baitoan}[\cite{Binh_Toan_7_tap_1}, \S3, \textbf{21.}]
	Chứng minh:
	\begin{enumerate*}
		\item[(a)] $7^6 + 7^5 - 7^4\ \vdots\ 55$;
		\item[(b)] $16^5 + 2^{15}\ \vdots\ 33$;
		\item[(c)] $81^7 - 27^9 - 9^{13}\ \vdots\ 405$.
	\end{enumerate*}
\end{baitoan}

\begin{baitoan}[\cite{Binh_Toan_7_tap_1}, \S3, \textbf{22.}]
	Điền vào chỗ trống $\ldots$ các từ ``bằng nhau'' hoặc ``đối nhau'' cho đúng:
	\begin{enumerate*}
		\item[(a)] Nếu $2$ số đối nhau thì bình phương của chúng $\ldots$
		\item[(b)] Nếu $2$ số đối nhau thì lập phương của chúng $\ldots$
		\item[(c)] Lũy thừa chẵn cùng bậc của 2 số đối nhau thì $\ldots$
		\item[(d)] Lũy thừa lẻ cùng bậc của 2 số đối nhau thì $\ldots$
	\end{enumerate*}
\end{baitoan}

\begin{baitoan}[\cite{Binh_Toan_7_tap_1}, \S3, \textbf{23.} \& mở rộng]
	Các đẳng thức sau có đúng với mọi $a,b\in\mathbb{Q}$ hay không?
	\begin{enumerate*}
		\item[(a)] $-a^3 = (-a)^3$;
		\item[(b)] $-a^5 = (-a)^5$;
		\item[(c)] $-a^2 = (-a)^2$;
		\item[(d)] $-a^4 = (-a)^4$;
		\item[(e)] $-a^{2n+1} = (-a)^{2n+1}$, $\forall n\in\mathbb{N}$;
		\item[(f)] $a^{2n} = (-a)^{2n}$, $\forall n\in\mathbb{N}$;
		\item[(g)] $(a - b)^2 = (b - a)^2$;
		\item[(h)] $(a - b)^3 = -(b - a)^3$;
		\item[(i)] $(a - b)^{2n} = (b - a)^{2n}$, $\forall n\in\mathbb{N}$;
		\item[(j)] $(a - b)^{2n+1} = -(b - a)^{2n+1}$, $\forall n\in\mathbb{N}$.
	\end{enumerate*}
\end{baitoan}

\begin{baitoan}[\cite{Binh_Toan_7_tap_1}, \S3, \textbf{24.}]
	Tính:
	\begin{enumerate*}
		\item[(a)] $\left(\frac{1}{2}\right)^{15}\cdot\left(\frac{1}{4}\right)^{20}$;
		\item[(b)] $\left(\frac{1}{9}\right)^{25}:\left(\frac{1}{3}\right)^{30}$;
		\item[(c)] $\left(\frac{1}{16}\right)^3:\left(\frac{1}{8}\right)^2$;
		\item[(d)] $(x^3)^2:(x^2)^3$ với $x\ne 0$.
	\end{enumerate*}
\end{baitoan}

\begin{baitoan}[\cite{Binh_Toan_7_tap_1}, \S3, \textbf{25.}]
	Viết số $64$ dưới dạng $a^n$ với $a\in\mathbb{Z}$. Có bao nhiêu cách viết?
\end{baitoan}

\begin{baitoan}[\cite{Binh_Toan_7_tap_1}, \S3, \textbf{26.}]
	Rút gọn biểu thức: $A = \dfrac{4^5\cdot 9^4 - 2\cdot 6^9}{2^{10}\cdot 3^8 + 6^8\cdot 20}$.
\end{baitoan}

\begin{baitoan}[\cite{Binh_Toan_7_tap_1}, \S3, \textbf{27.}]
	Cho $S_n = \sum_{i=1}^{n-1} (-1)^{i-1}i = 1 - 2 + 3 - 4 + \cdots + (-1)^{n-1}n$ với $n\in\mathbb{N}^\star$. Tính $S_{35} + S_{60}$.
\end{baitoan}

\begin{baitoan}[\cite{Binh_Toan_7_tap_1}, \S3, \textbf{28.}]
	Cho $A = 1 - 5 + 9 - 13 + 17 - 21 + 25 - \cdots$ ($n$ số hạng, giá trị tuyệt đối của số sau lớn hơn giá trị tuyệt đối của số hạng trước $4$ đơn vị, các dấu $+$ \& $-$ xen kẽ).
	\begin{enumerate*}
		\item[(a)] Tính $A$ theo $n$.
		\item[(b)] Viết số hạng thứ $n$ của biểu thức $A$ theo $n$ (chú ý dùng lũy thừa để biểu thị dấu của số hạng đó).
	\end{enumerate*}
\end{baitoan}

\begin{baitoan}[\cite{Binh_Toan_7_tap_1}, \S3, \textbf{29.}]
	Với giá trị nào của các chữ thì các biểu thức sau có giá trị là số $0$, số dương, số âm?
	\begin{enumerate*}
		\item[(a)] $P = \frac{a^2b}{c}$;
		\item[(b)] $Q = \frac{x^3}{yz}$.
	\end{enumerate*}
\end{baitoan}

\begin{baitoan}[\cite{Binh_Toan_7_tap_1}, \S3, \textbf{30.}]
	Cho $2$ số hữu tỷ $a$ \& $b$ trái dấu trong đó $|a| = b^5$. Xác định dấu của mỗi số.
\end{baitoan}

\begin{baitoan}[\cite{Binh_Toan_7_tap_1}, \S3, \textbf{31.}]
	Viết các số sau dưới dạng lũy thừa của $2$: $16,64,1,\frac{1}{32},\frac{1}{8},0.5,0.25$.
\end{baitoan}

\begin{baitoan}[\cite{Binh_Toan_7_tap_1}, \S3, \textbf{32.}]
	\begin{enumerate*}
		\item[(a)] Viết các số sau thành lũy thừa với số mũ âm: $\frac{1}{1000000},0.00000002$.
		\item[(b)] Viết các số sau dưới dạng số thập phân: $10^{-7}$, $2.5\cdot 10^{-6}$.
	\end{enumerate*}
\end{baitoan}

\begin{baitoan}[\cite{Binh_Toan_7_tap_1}, \S3, \textbf{33.}]
	Tính xem $A$ gấp mấy lần $B$:
	\begin{enumerate*}
		\item[(a)] $A = 3.4\cdot 10^{-8}$, $B = 34\cdot 10^{-9}$;
		\item[(b)] $A = 10^{-4} + 10^{-3} + 10^{-2}$, $B = 10^{-9}$.
	\end{enumerate*}
\end{baitoan}

\begin{baitoan}[\cite{Binh_Toan_7_tap_1}, \S3, \textbf{34.}]
	So sánh:
	\begin{enumerate*}
		\item[(a)] $\left(-\frac{1}{16}\right)^{100}$ \& $\left(-\frac{1}{2}\right)^{500}$;
		\item[(b)] $(-32)^9$ \& $(-18)^{13}$.
	\end{enumerate*}
\end{baitoan}

\begin{baitoan}[\cite{Binh_Toan_7_tap_1}, \S3, \textbf{35.}]
	Sắp xếp $a,b,c\in\mathbb{Q}$ theo thứ tự từ nhỏ đến lớn: $a = 2^{100}$, $b = 3^{75}$, $c = 5^{50}$.
\end{baitoan}

\begin{baitoan}[\cite{Binh_Toan_7_tap_1}, \S3, \textbf{36.}]
	Trong các câu sau, câu nào đúng với mọi $a\in\mathbb{Q}$?
	\begin{enumerate*}
		\item[(a)] Nếu $a < 0$ thì $a^2 > 0$;
		\item[(b)] Nếu $a^2 > 0$ thì $a > 0$;
		\item[(c)] Nếu $a < 0$ thì $a^2 > a$;
		\item[(d)] Nếu $a^2 > a$ thì $a > 0$;
		\item[(e)] Nếu $a^2 > a$ thì $a < 0$.
	\end{enumerate*}
\end{baitoan}

\begin{baitoan}[\cite{Binh_Toan_7_tap_1}, \S3, \textbf{37.}]
	\begin{enumerate*}
		\item[(a)] Cho $a^m = a^n$ ($a\in\mathbb{Q}$, $m,n\in\mathbb{N}$). Tìm $m,n$.
		\item[(b)] Cho $a^m > a^n$ ($a\in\mathbb{Q}$, $a > 0$, $m,n\in\mathbb{N}$). So sánh $m$ \& $n$.
	\end{enumerate*}
\end{baitoan}

\begin{baitoan}[\cite{Binh_Toan_7_tap_1}, \S3, \textbf{38.}]
	Tìm $x\in\mathbb{Q}$, biết rằng:
	\begin{enumerate*}
		\item[(a)] $(2x - 1)^4 = 81$;
		\item[(b)] $(x - 1)^5 = -32$;
		\item[(c)] $(2x - 1)^6 = (2x - 1)^8$.
	\end{enumerate*}
\end{baitoan}

\begin{baitoan}[\cite{Binh_Toan_7_tap_1}, \S3, \textbf{39.}]
	Tìm $x\in\mathbb{N}$, biết rằng:
	\begin{enumerate*}
		\item[(a)] $5^x + 5^{x+2} = 650$;
		\item[(b)] $3^{x-1} + 5\cdot 3^{x-1} = 162$.
	\end{enumerate*}
\end{baitoan}

\begin{baitoan}[\cite{Binh_Toan_7_tap_1}, \S3, \textbf{40.}]
	Tìm $x,y\in\mathbb{N}$, biết rằng:
	\begin{enumerate*}
		\item[(a)] $2^{x+1}\cdot 3^y = 12^x$;
		\item[(b)] $10^x:5^y = 20^y$;
		\item[(c)] $2^x = 4^{y-1}$ \& $27^y = 3^{x+8}$.
	\end{enumerate*}
\end{baitoan}

\begin{baitoan}[\cite{Binh_Toan_7_tap_1}, \S3, \textbf{41.}]
	Tìm $a,b,c\in\mathbb{Q}$, biết rằng:
	\begin{enumerate*}
		\item[(a)] $ab = \frac{3}{5}$, $bc = \frac{4}{5}$, $ca = \frac{3}{4}$.
		\item[(b)] $a(a + b + c) = -12$, $b(a + b + c) = 18$, $c(a + b + c) = 30$;
		\item[(c)] $ab = c$, $bc = 4a$, $ac = 9b$.
	\end{enumerate*}
\end{baitoan}

\begin{baitoan}[\cite{Binh_Toan_7_tap_1}, \S3, \textbf{42.}]
	Cho $a,b,c,d,e\in\mathbb{N}$ thỏa mãn $a^b = b^c = c^d = d^e = e^a$. Chứng minh $a = b = c = d = e$.
\end{baitoan}

\begin{baitoan}[\cite{Binh_Toan_7_tap_1}, \S3, \textbf{43.}]
	Cho $A = \prod_{i=2}^{100} \left(\frac{1}{i^2} - 1\right) = \left(\frac{1}{2^2} - 1\right)\left(\frac{1}{3^2} - 1\right)\left(\frac{1}{4^2} - 1\right)\cdots\left(\frac{1}{100^2} - 1\right)$. So sánh $A$ với $-\frac{1}{2}$.
\end{baitoan}

\begin{baitoan}[\cite{Binh_Toan_7_tap_1}, \S3, \textbf{44.}]
	Rút gọn $A = \sum_{i=1}^{100} (-1)^i2^i = 2^{100} - 2^{99} + 2^{98} - 2^{97} + \cdots + 2^2 - 2$.
\end{baitoan}

\begin{baitoan}[\cite{Binh_Toan_7_tap_1}, \S3, \textbf{45.}]
	Rút gọn $B = \sum_{i=
	}^{100} (-1)^i3^i = 3^{100} - 3^{99} + 3^{98} - 3^{97} + \cdots + 3^2 - 3 + 1$.
\end{baitoan}

\begin{baitoan}[\cite{Binh_Toan_7_tap_1}, \S3, \textbf{46.}]
	Cho $C = \sum_{i=1}^{99} \frac{1}{3^i} = \frac{1}{3} + \frac{1}{3^2} + \cdots + \frac{1}{3^{99}}$. Chứng minh $C < \frac{1}{2}$.
\end{baitoan}

\begin{baitoan}[\cite{Binh_Toan_7_tap_1}, \S3, \textbf{47.}]
	Chứng minh $\frac{3}{1^2\cdot 2^2} + \frac{5}{2^2\cdot 3^2} + \frac{7}{3^2\cdot 4^2} + \cdots + \frac{19}{9^2\cdot 10^2} < 1$.
\end{baitoan}

\begin{baitoan}[\cite{Binh_Toan_7_tap_1}, \S3, \textbf{48.}]
	Chứng minh $\sum_{i=1}^{100} \frac{i}{3^i} = \frac{1}{3} + \frac{2}{3^2} + \frac{3}{3^3} + \cdots + \frac{100}{3^{100}} < \frac{3}{4}$.
\end{baitoan}

\begin{baitoan}[\cite{Binh_Toan_7_tap_1}, \S3, \textbf{49.}]
	Ta không có $2^m + 2^n = 2^{m+n}$, $\forall m,n\in\mathbb{N}^\star$. Nhưng có những số nguyên dương $m,n$ có tính chất trên. Tìm các số đó.
\end{baitoan}

\begin{baitoan}[\cite{Binh_Toan_7_tap_1}, \S3, \textbf{50.}]
	Tìm $m,n\in\mathbb{N}^\star$ sao cho $2^m - 2^n = 256$.
\end{baitoan}

\begin{baitoan}[\cite{Binh_Toan_7_tap_1}, \S3, \textbf{51.}]
	Cho 1 bảng vuông $3\times 3$ ô. Trong mỗi ô của bảng viết số $1$ hoặc số $-1$. Gọi $d_i$ là tích các số trên dòng $i$ ($i = 1,2,3$), $c_k$ là tích các số trên cột $k$ ($k = 1,2,3$).
	\begin{enumerate*}
		\item[(a)] Chứng minh rằng không thể xảy ra $d_1 + d_2 + d_3 + c_1 + c_2 + c_3 = 0$.
		\item[(b)] Xét bài toán trên đối với bảng vuông $n\times n$.
	\end{enumerate*}
\end{baitoan}

\begin{baitoan}[\cite{Binh_Toan_7_tap_1}, \S3, \textbf{52.}]
	Cho $n$ số $x_1,\ldots,x_n$, mỗi số bằng $1$ hoặc $-1$. Biết rằng tổng của $n$ tích $x_1x_2$, $x_2x_3$, $x_3x_4,\ldots,x_nx_1$ bằng $0$. Chứng minh $n\ \vdots\ 4$.
\end{baitoan}

%------------------------------------------------------------------------------%

\subsection{Tỷ Lệ Thức}
``Tỷ lệ thức là 1 đẳng thức của 2 tỷ lệ. Trong tỷ lệ thức $\frac{a}{b} = \frac{c}{d}$ (hoặc $a:b = c:d$) các số hạng $a$ \& $d$ được gọi là \textit{ngoại tỷ}, các số hạng $b$ \& $c$ được gọi là \textit{trung tỷ}. Khi viết tỷ lệ thức $\frac{a}{b} = \frac{c}{d}$, ta luôn giả thiết $b\ne 0$, $d\ne 0$. Từ tỷ lệ thức $\frac{a}{b} = \frac{c}{d}$ ta suy ra $ad = bc$. Đảo lại, nếu $ad = bc$ (cả $4$ số $a,b,c,d$ khác $0$\footnote{I.e., $a^2 + b^2 + c^2 + d^2\ne 0$.}) thì ta có các tỷ lệ thức: $\frac{a}{b} = \frac{c}{d},\frac{a}{c} = \frac{b}{d},\frac{d}{d} = \frac{c}{a},\frac{d}{c} = \frac{b}{a}$. Như vậy trong tỷ lệ thức, ta có thể hoán vị các ngoại tỷ với nhau, hoán vị các trung tỷ với nhau, hoán vị cả ngoại tỷ với nhau \& trung tỷ với nhau. Từ đẳng thức $ad = bc$, ta lập được 4 tỷ lệ thức với các số hạng là $a,b,c,d$ (với quy ước 2 tỷ lệ thức $\frac{a}{b} = \frac{c}{d}$ \& $\frac{c}{d} = \frac{a}{b}$ chỉ kể là 1 tỷ lệ thức).'' -- \cite[\S4]{Binh_Toan_7_tap_1}

\begin{baitoan}[\cite{Binh_Toan_7_tap_1}, \S4, Ví dụ 6]
	Cho 3 số $6,8,24$.
	\begin{enumerate*}
		\item[(a)] Tìm số $x$, sao cho $x$ cùng với 3 số trên lập thành 1 tỷ lệ thức.
		\item[(b)] Có thể lập được tất cả bao nhiêu tỷ lệ thức?
	\end{enumerate*}
\end{baitoan}

\begin{baitoan}[\cite{Binh_Toan_7_tap_1}, \S4, Ví dụ 7]
	Cho tỷ lệ thức $\frac{a}{b} = \frac{c}{d}$. Chứng minh: $\frac{a}{a - b} = \frac{c}{c - d}$ (giả thiết $a\ne b$, $c\ne d$ \& mỗi số $a,b,c,d\ne 0$).
\end{baitoan}

\begin{baitoan}[\cite{Binh_Toan_7_tap_1}, Ví dụ 8, \S4]
	Cho tỷ lệ thức $\frac{x}{2} = \frac{y}{5}$. Biết $xy = 90$. Tính $x$ \& $y$.
\end{baitoan}

\begin{baitoan}[\cite{Binh_Toan_7_tap_1}, \S4, \textbf{53.}]
	Tìm $x\in\mathbb{Q}$ trong tỷ lệ thức:
	\begin{enumerate*}
		\item[(a)] $0.4:x = x:0.9$.
		\item[(b)] $13\frac{1}{3}:1\frac{1}{3} = 26:(2x - 1)$.
		\item[(c)] $0.2:1\frac{1}{5} = \frac{2}{3}:(6x + 7)$.
		\item[(d)] $\frac{37 - x}{x + 13} = \frac{3}{7}$.
	\end{enumerate*}
\end{baitoan}

\begin{baitoan}[\cite{Binh_Toan_7_tap_1}, \S4, \textbf{54.}]
	Cho tỷ lệ thức $\frac{3x - y}{x + y} = \frac{3}{4}$. Tìm giá trị của tỷ số $\frac{x}{y}$.
\end{baitoan}

\begin{baitoan}[\cite{Binh_Toan_7_tap_1}, \S4, \textbf{55.}]
	Cho tỷ lệ thức $\frac{a}{b} = \frac{c}{d}$. Chứng minh các tỷ lệ thức sau (giả thiết các tỷ lệ thức đều có nghĩa):
	\begin{enumerate*}
		\item[(a)] $\frac{2a + 3b}{2a - 3b} = \frac{2c + 3d}{2c - 3d}$.
		\item[(b)] $\frac{ab}{cd} = \frac{a^2 - b^2}{c^2 - d^2}$.
		\item[(c)] $\left(\frac{a + b}{c + d}\right)^2 = \frac{a^2 + b^2}{c^2 + d^2}$.
	\end{enumerate*}
\end{baitoan}

\begin{baitoan}[\cite{Binh_Toan_7_tap_1}, \S4, \textbf{56.}]
	Chứng minh: ta có tỷ lệ thức $\frac{a}{b} = \frac{c}{d}$ nếu có 1 trong các đẳng thức sau (giả thiết các tỷ lệ thức đều có nghĩa):
	\begin{enumerate*}
		\item[(a)] $\frac{a + b}{a - b} = \frac{c + d}{c - d}$.
		\item[(b)] $(a + b + c + d)(a - b - c + d) = (a - b + c - d)(a + b - c - d)$.
	\end{enumerate*}
\end{baitoan}

\begin{baitoan}[\cite{Binh_Toan_7_tap_1}, \S4, \textbf{57.}]
	Cho tỷ lệ thức $\frac{a + b + c}{a + b - c} = \frac{a - b + c}{a - b - c}$ trong đó $b\ne 0$. Chứng minh $c = 0$.
\end{baitoan}

\begin{baitoan}[\cite{Binh_Toan_7_tap_1}, \S4, \textbf{58.}]
	Cho tỷ lệ thức $\frac{a + b}{b + c} = \frac{c + d}{d + a}$. Chứng minh: $a = c$ hoặc $a + b + c + d = 0$.
\end{baitoan}

\begin{baitoan}[\cite{Binh_Toan_7_tap_1}, \S4, \textbf{59.}]
	Có thể lập được 1 tỷ lệ thức từ 4 trong các số sau không (mỗi số chỉ chọn 1 lần)? Nếu có thì lập được bao nhiêu tỷ lệ thức?
	\begin{enumerate*}
		\item[(a)] $3,4,5,6,7$.
		\item[(b)] $1,2,4,8,16$.
		\item[(c)] $1,3,9,27,81,243$.
	\end{enumerate*}
\end{baitoan}

\begin{baitoan}[\cite{Binh_Toan_7_tap_1}, \S4, \textbf{60.}]
	Cho 4 số $2,4,8,16$. Tìm $x\in\mathbb{Q}$ cùng với 3 trong 4 số trên lập được thành 1 tỷ lệ thức.
\end{baitoan}

%------------------------------------------------------------------------------%

\subsection{Tính Chất của Dãy Tỷ Số Bằng Nhau}
``Nếu có $n$ tỷ số bằng nhau ($n\ge 2$): $\frac{a_1}{b_1} = \frac{a_2}{b_2} = \cdots = \frac{a_n}{b_n}$ thì $\frac{a_1}{b_1} = \frac{\sum_{i=1}^n c_ia_i}{\sum_{i=1}^n c_ib_i} = \frac{c_1a_1 + c_2a_2 + \cdots + c_na_n}{c_1b_1 + c_2b_2 + \cdots + c_nb_n}$ (nếu đặt dấu ``$-$'' trước số hạng trên của tỷ số nào thì cũng đặt dấu ``$-$'' trước số hạng dưới của tỷ số đó). Ta gọi tính chất này là \textit{tính chất dãy tỷ số bằng nhau}. Tính chất dãy tỷ số bằng nhau cho ta 1 khả năng rộng rãi để từ 1 số tỷ số bằng nhau cho trước, ta lập được những tỷ số mới bằng các tỷ số đã cho, trong đó số hạng trên hoặc số hạng dưới của nó có dạng thuận lợi nhằm sử dụng các dữ kiện của bài toán.'' -- \cite[\S5]{Binh_Toan_7_tap_1}

\begin{baitoan}[\cite{Binh_Toan_7_tap_1}, \S5, Ví dụ 9]
	Tìm các số $x,y,z$ biết $\frac{x}{3} = \frac{y}{4}$, $\frac{y}{5} = \frac{z}{7}$ \& $2x + 3y - z = 186$.
\end{baitoan}

\begin{baitoan}[\cite{Binh_Toan_7_tap_1}, \S5, Ví dụ 10]
	Tìm các số $x,y,z$ biết $\frac{y + z + 1}{x} = \frac{x + z + 2}{y} = \frac{x + y - 3}{z} = \frac{1}{x + y + z}$.
\end{baitoan}

\begin{baitoan}[\cite{Binh_Toan_7_tap_1}, \S5, \textbf{61.}]
	Tìm các số $x,y,z$ biết:
	\begin{enumerate*}
		\item[(a)] $\frac{x}{10} = \frac{y}{10} = \frac{z}{21}$ \& $5x + y - 2z = 28$.
		\item[(b)] $3x = 2y$, $7y = 5z$, $x - y + z = 32$.
		\item[(c)] $\frac{x}{3} = \frac{y}{4}$, $\frac{y}{3} = \frac{z}{5}$, $2x - 3y + z = 6$.
		\item[(d)] $\frac{2x}{3} = \frac{3y}{4} = \frac{4z}{5}$ \& $x + y + z = 49$.
		\item[(e)] $\frac{x - 1}{2} = \frac{y - 2}{3} = \frac{z - 3}{4}$ \& $2x + 3y - z = 50$.
		\item[(g)] $\frac{x}{2} = \frac{y}{3} = \frac{z}{5}$ \& $xyz = 810$.
	\end{enumerate*}
\end{baitoan}

\begin{baitoan}[\cite{Binh_Toan_7_tap_1}, \S5, \textbf{62.}]
	Tìm $x$ biết $\frac{1 + 2y}{18} = \frac{1 + 4y}{24} = \frac{1 + 6y}{6x}$.
\end{baitoan}

\begin{baitoan}[\cite{Binh_Toan_7_tap_1}, \S5, \textbf{63.}]
	Tìm phân số $\frac{a}{b}$ biết nếu cộng thêm cùng 1 số khác $0$ vào tử \& mẫu thì giá trị của phân số đó không đổi.
\end{baitoan}

\begin{baitoan}[\cite{Binh_Toan_7_tap_1}, \S5, \textbf{64.}]
	Cho $\frac{a}{b} = \frac{b}{c} = \frac{c}{d}$. Chứng minh $\left(\frac{a + b + c}{b + c + d}\right)^3 = \frac{a}{d}$.
\end{baitoan}

\begin{baitoan}[\cite{Binh_Toan_7_tap_1}, \S5, \textbf{65.}]
	Cho $\frac{a}{b} = \frac{b}{c} = \frac{c}{a}$. Chứng minh $a = b = c$.
\end{baitoan}

\begin{baitoan}[\cite{Binh_Toan_7_tap_1}, \S5, \textbf{66.}]
	Vì sao tỷ số của 2 hỗn số dạng $a\frac{1}{b}$ \& $b\frac{1}{a}$ luôn luôn bằng phân số $\frac{a}{b}$?
\end{baitoan}

\begin{baitoan}[\cite{Binh_Toan_7_tap_1}, \S5, \textbf{67.}]
	Cho 3 tỷ số bằng nhau là $\frac{a}{b + c},\frac{b}{c + a},\frac{c}{a + b}$. Tìm giá trị của mỗi tỷ số đó.
\end{baitoan}

\subsection{Chia Tỷ Lệ}
``Trong các bài toán về chia 1 số thành các phần tỷ lệ thuận hoặc tỷ lệ nghịch với các số cho trước, cần chú ý:
\begin{enumerate*}
	\item[\textbf{1.}] $x,y,z$ tỷ lệ thuận với $a,b,c\Leftrightarrow x:y:z = a:b:c\Leftrightarrow\frac{x}{a} = \frac{y}{b} = \frac{z}{c}$.
	\item[\textbf{2.}] $x,y,z$ tỷ lệ nghịch với $m,n,p\Leftrightarrow x:y:z = \frac{1}{m}:\frac{1}{n}:\frac{1}{p}$.'' -- \cite{Binh_Toan_7_tap_1}
\end{enumerate*}

\begin{baitoan}[\cite{Binh_Toan_7_tap_1}, Ví dụ 16]
	2 xe ô tô cùng khởi hành 1 lúc từ 2 địa điểm A \& B. Xe thứ nhất đi quãng đường AB hết $\rm4h15ph$, xe thứ 2 đi quãng đường BA hết $\rm3h45ph$. Đến chỗ gặp nhau, xe thứ 2 đi được quãng đường dài hơn quãng đường xe thứ nhất đã đi là $20$\emph{km}. Tính quãng đường AB.
\end{baitoan}

\begin{baitoan}[\cite{Binh_Toan_7_tap_1}, Ví dụ 17]
	Để đi từ A đến B có thể dùng các phương tiện: máy bay, ô tô, xe lửa. Vận tốc của máy bay, ô tô, xe lửa có tỷ lệ với $6$; $2$; $1$. Biết thời gian đi từ A đến B bằng máy bay ít hơn so với đi bằng ô tô là $6$ giờ. Hỏi thời gian xe lửa đi quãng đường AB là bao lâu?
\end{baitoan}

\begin{baitoan}[\cite{Binh_Toan_7_tap_1}, Ví dụ 18]
	3 kho A, B, C chứa 1 số gạo. Người ta nhập vào kho A thêm $\frac{1}{7}$ số gạo của kho đó, xuất ở kho B đi $\frac{1}{9}$ số gạo của kho đó, xuất ở kho C đi $\frac{2}{7}$ số gạo của kho đó. Khi đó số gạo của 3 kho bằng nhau. Tính số gạo ở mỗi kho lúc đầu, biết kho B chứa nhiều hơn kho A là $20$ tạ gạo.
\end{baitoan}

\begin{baitoan}[\cite{Binh_Toan_7_tap_1}, Ví dụ 19]
	3 đội công nhân I, II, III phải vận chuyển tổng cộng $1530$\emph{kg} hàng từ kho theo thứ tự đến 3 địa điểm cách kho $1500$\emph{m}, $2000$\emph{m}, $3000$\emph{m}. Phân chia số hàng cho mỗi đội sao cho khối lượng hàng tỷ lệ nghịch với khoảng cách cần chuyển.
\end{baitoan}

\begin{baitoan}[\cite{Binh_Toan_7_tap_1}, Ví dụ 20]
	3 xí nghiệp cùng xây dựng chung 1 cái cầu hết $38$ triệu đồng. Xí nghiệp I có $40$ xe ở cách cầu $1.5$\emph{km}, xí nghiệp II có $20$ xe ở cách cầu $3$\emph{km}, xí nghiệp III có $30$ xe ở cách cầu $1$\emph{km}. Hỏi mỗi xí nghiệp phải trả cho việc xây dựng cầu bao nhiêu tiền, biết số tiền phải trả tỷ lệ thuận với số xe \& tỷ lệ nghịch với khoảng cách từ xí nghiệp đến cầu?
\end{baitoan}

\begin{baitoan}[\cite{Binh_Toan_7_tap_1}, \textbf{106.}]
	\begin{enumerate*}
		\item[(a)] Tính thời gian từ lúc 2 kim đồng hồ gặp nhau lần trước đến lúc chúng gặp nhau lần tiếp theo.
		\item[(b)] Trong 1 ngày, 2 kim đồng hồ tạo với nhau góc vuông bao nhiêu lần?
	\end{enumerate*}
\end{baitoan}

\begin{baitoan}[\cite{Binh_Toan_7_tap_1}, \textbf{107.}]
	1 ống dài được kéo bởi 1 máy kéo trên đường. Tuấn chạy dọc từ đầu ống đến cuối ống theo hướng chuyển động của máy kéo thì đếm được $140$ bước. Sau đó Tuấn quay lại chạy dọc ống theo chiều ngược lại thì đếm được $20$ bước. Biết mỗi bước chạy của Tuấn dài $1$\emph{m}. Tính độ dài của ống.
\end{baitoan}

\begin{baitoan}[\cite{Binh_Toan_7_tap_1}, \textbf{108.}]
	5 lớp 7A, 7B, 7C, 7D, 7E nhận chăm sóc vườn trường có diện tích $300\rm m^2$. Lớp 7A nhận $15$\% diện tích vườn, lớp 7B nhận $\frac{1}{5}$ diện tích còn lại. Diện tích còn lại của vườn sau khi 2 lớp trên nhận được đem chia cho 3 lớp 7C, 7D, 7E tỷ lệ với $\frac{1}{2},\frac{1}{4},\frac{5}{16}$. Tính diện tích vườn giao cho mỗi lớp.
\end{baitoan}

\begin{baitoan}[\cite{Binh_Toan_7_tap_1}, \textbf{109.}]
	3 công nhân được thưởng $100000$ đồng, số tiền thưởng được phân chia tỷ lệ với mức sản xuất của mỗi người. Biết mức sản xuất của người thứ nhất so với mức sản xuất của người thứ 2 bằng $5:3$, mức sản xuất của người thứ 3 bằng $25$\% tổng số mức sản xuất của 2 người kia. Tính số tiền mỗi người được thưởng.
\end{baitoan}

\begin{baitoan}[\cite{Binh_Toan_7_tap_1}, \textbf{110.}]
	1 công trường dự định phân chia số đất cho 3 đội I, II, III tỷ lệ với $7,6,5$. Nhưng sau đó vì số người của các đội thay đổi nên đã chia lại tỷ lệ với $6,5,4$. Như vậy có 1 đội làm nhiều hơn so với dự định là $\rm6m^3$ đất. Tính số đất đã phân chia cho mỗi đội.
\end{baitoan}

\begin{baitoan}[\cite{Binh_Toan_7_tap_1}, \textbf{111.}]
	Trong 1 đợt lao đông, 3 khối 7, 8, 9 chuyển được $\rm912m^3$ đất. Trung bình mỗi học sinh khối 7, 8, 9 theo thứ tự làm được $\rm1.2m^3,1.4m^3,1.6m^3$. Số học sinh khối 7 \& khối 8 tỷ lệ với 1 \& 3, số học sinh khối 8 \& khối 9 tỷ lệ với 4 \& 5. Tính số học sinh của mỗi khối.
\end{baitoan}

\begin{baitoan}[\cite{Binh_Toan_7_tap_1}, \textbf{112.}]
	3 tổ công nhân có mức sản xuất tỷ lệ với $5,4,3$. Tổ I tăng năng suất $10$\%, tổ II tăng năng suất $20$\%, tổ III tăng năng suất $10$\%. Do đó trong cùng 1 thời gian, tổ I làm được nhiều hơn tổ II là $7$ sản phẩm. Tính số sản phẩm mỗi tổ đã làm được trong thời gian đó.
\end{baitoan}

\begin{baitoan}[\cite{Binh_Toan_7_tap_1}, \textbf{113.}]
	Tìm 3 số tự nhiên, biết $\operatorname{BCNN}$ của chúng bằng $3150$, tỷ số của số thứ nhất \& số thứ 2 là $5:9$, tỷ số của số thứ nhất \& số thứ 3 là $10:7$.
\end{baitoan}

\begin{baitoan}[\cite{Binh_Toan_7_tap_1}, \textbf{114.}]
	3 tấm vải theo thứ tự giá $120000$ đồng, $192000$ đồng, \& $144000$ đồng. Tấm thứ nhất \& tấm thứ 2 có cùng chiều dài, tấm thứ 2, \& tấm thứ 3 có cùng chiều rộng. Tổng của 3 chiều dài là $110$\emph{m}, tổng của 3 chiều rộng là $2.1$\emph{m}. Tính kích thước của mỗi tấm vải, biết giá $\rm1m^2$ của 3 tấm vải bằng nhau.
\end{baitoan}

\begin{baitoan}[\cite{Binh_Toan_7_tap_1}, \textbf{115.}]
	Có 3 gói tiền: gói thứ nhất gồm toàn tờ $500$ đồng, gói thứ 2 gồm toàn tờ $2000$ đồng, gói thứ 3 gồm toàn tờ $5000$ đồng. Biết tổng số tờ giấy bạc của 3 gói là $540$ tờ \& số tiền ở các gói bằng nhau. Tính số tờ giấy bạc mỗi loại.
\end{baitoan}

\begin{baitoan}[\cite{Binh_Toan_7_tap_1}, \textbf{116.}]
	3 công nhân tiện được tất cả $860$ dụng cụ trong cùng 1 thời gian. Để tiện 1 dụng cụ, người thứ nhất cần $5$\emph{ph}, người thứ 2 cần $6$\emph{ph}, người thứ 3 cần $9$\emph{ph}. Tính số dụng cụ mỗi người tiện được.
\end{baitoan}

\begin{baitoan}[\cite{Binh_Toan_7_tap_1}, \textbf{117.}]
	3 em bé: Ánh $5$ tuổi, Bích $6$ tuổi, Châu $10$ tuổi được bà chia cho $42$ chiếc kẹo. Số kẹo được chia tỷ lệ nghịch với số tuổi của mỗi em. Hỏi mỗi em được chia bao nhiêu chiếc kẹo?
\end{baitoan}

\begin{baitoan}[\cite{Binh_Toan_7_tap_1}, \textbf{118.}]
	Tìm 3 phân số, biết tổng của chúng bằng $3\frac{3}{70}$, các tử của chúng tỷ lệ với $3,4,5$, các mẫu của chúng tỷ lệ với $5,1,2$.
\end{baitoan}

\begin{baitoan}[\cite{Binh_Toan_7_tap_1}, \textbf{119.}]
	Tìm số tự nhiên có 3 chữ số, biết số đó là bội của $72$ \& các chữ số của nó nếu xếp từ nhỏ đến lớn thì tỷ lệ với $1,2,3$.
\end{baitoan}

\begin{baitoan}[\cite{Binh_Toan_7_tap_1}, \textbf{120.}]
	Độ dài 3 cạnh của 1 tam giác tỷ lệ với $2,3,4$. 3 chiều cao tương ứng với 3 cạnh đó tỷ lệ với 3 số nào?
\end{baitoan}

\begin{baitoan}[\cite{Binh_Toan_7_tap_1}, \textbf{121.}]
	3 đường cao của $\Delta ABC$ có độ dài bằng $4,12,x$. Biết $x\in\mathbb{N}^\star$. Tìm $x$ (cho biết \emph{bất đẳng thức tam giác}: mỗi cạnh của tam giác nhỏ hơn tổng 2 cạnh kia \& lớn hơn hiệu của chúng).
\end{baitoan}

\begin{baitoan}[\cite{Binh_Toan_7_tap_1}, \textbf{122.}]
	Cho $\Delta ABC$. Có góc ngoài của tam giác tại $A,B,C$ tỷ lệ với $4,5,6$. Các góc trong tương ứng tỷ lệ với các số nào?
\end{baitoan}

\begin{baitoan}[\cite{Binh_Toan_7_tap_1}, \textbf{123.}]
	Tìm 2 số khác $0$ biết tổng, hiệu, tích của chúng tỷ lệ với $5,1,12$.
\end{baitoan}

%------------------------------------------------------------------------------%

\subsection{Số Thập Phân Hữu Hạn. Số Thập Phân Vô Hạn Tuần Hoàn}

\begin{baitoan}[\cite{Binh_Toan_7_tap_1}, Ví dụ 21]
	Viết các phân số sau dưới dạng số thập phân:
	\begin{enumerate*}
		\item[(a)] $\frac{7}{25},\frac{3}{40}$.
		\item[(b)] $\frac{7}{33},\frac{1}{7},\frac{7}{22}$.
	\end{enumerate*}
\end{baitoan}

%------------------------------------------------------------------------------%

\section{Số Thực}

\begin{baitoan}
	Chứng minh: $(x^2 + m^2)(x^2 + n^2) = 0\Leftrightarrow x^2 + m^2n^2 = 0$ \& $(x^2 + m^2)(x^2 + n^2)\ne 0\Leftrightarrow x^2 + m^2n^2\ne 0$, $\forall x,m,n\in\mathbb{R}$.
\end{baitoan}
\textit{Ý nghĩa}: Điều kiện để các công thức nhân chia lũy thừa cùng cơ số xác định.

\subsection{Số Vô Tỷ. Căn Bậc 2. Số Thực}
``Mọi số hữu tỷ đều biểu diễn được dưới dạng số thập phân hữu hạn hoặc vô hạn tuần hoàn. Ngược lại, mỗi số thập phân hữu hạn hoặc vô hạn tuần hoàn đều biểu diễn 1 số hữu tỷ. Số vô tỷ là số viết được dưới dạng số thập phân vô hạn không tuần hoàn. Tập hợp các số thực $\mathbb{R}$ bao gồm tập hợp số hữu tỷ $\mathbb{Q}$ \& tập hợp số vô tỷ $\mathbb{I}$. Cho số $a$ không âm. Căn bậc 2 của $a$ là số $x$ mà $x^2 = a$. Căn bậc 2 không âm của $a$ ký hiệu là $\sqrt{a}$. Nếu $n\in\mathbb{N}$ không là số chính phương thì $\sqrt{n}$ là số vô tỷ.'' -- \cite[\S7]{Binh_Toan_7_tap_1}

\begin{baitoan}[\cite{Binh_Toan_7_tap_1}, \S7, Ví dụ 11]
	Chứng minh:
	\begin{enumerate*}
		\item[(a)] $\sqrt{15}$ là số vô tỷ.
		\item[(b)] Nếu số tự nhiên $a$ không phải là số chính phương thì $\sqrt{a}$ là số vô tỷ.
	\end{enumerate*}
\end{baitoan}

\begin{baitoan}[\cite{Binh_Toan_7_tap_1}, \S7, \textbf{69.}]
	Tìm $x$ biết:
	\begin{enumerate*}
		\item[(a)] $x^2 = 81$.
		\item[(b)] $(x - 1)^2 = \frac{9}{16}$.
		\item[(c)] $x - 2\sqrt{x} = 0$.
		\item[(d)] $x = \sqrt{x}$.
	\end{enumerate*}
\end{baitoan}

\begin{baitoan}[\cite{Binh_Toan_7_tap_1}, \S7, \textbf{70.}]
	Cho $A = \frac{\sqrt{x} + 1}{\sqrt{x} - 1}$. Chứng minh với $x = \frac{16}{9}$ \& $x = \frac{25}{9}$ thì $A$ có giá trị là số nguyên.
\end{baitoan}

\begin{baitoan}[\cite{Binh_Toan_7_tap_1}, \S7, \textbf{71.}]
	Cho $A = \frac{\sqrt{x} + 1}{\sqrt{x} - 3}$. Tìm số nguyên $x$ để $A$ có giá trị là 1 số nguyên.
\end{baitoan}

\begin{baitoan}[\cite{Binh_Toan_7_tap_1}, \S7, \textbf{72.}]
	Chứng minh:
	\begin{enumerate*}
		\item[(a)] $\sqrt{2}$ là số vô tỷ.
		\item[(b)] $5 - \sqrt{2}$ là số vô tỷ.
	\end{enumerate*}
\end{baitoan}

\begin{baitoan}[\cite{Binh_Toan_7_tap_1}, \S7, \textbf{73.}]
	\begin{enumerate*}
		\item[(a)] Có 2 số vô tỷ nào mà tích là 1 số hữu tỷ hay không?
		\item[(b)] Có 2 số vô tỷ dương nào mà tổng là 1 số hữu tỷ hay không?
	\end{enumerate*}
\end{baitoan}

\begin{baitoan}[\cite{Binh_Toan_7_tap_1}, \S7, \textbf{74.}]
	Ký hiệu $\lfloor x\rfloor$ là số nguyên lớn nhất không vượt quá $x$. Tính giá trị của tổng: $\sum_{i=1}^{35} \lfloor\sqrt{i}\rfloor = \lfloor\sqrt{1}\rfloor + \lfloor\sqrt{2}\rfloor + \cdots + \lfloor\sqrt{35}\rfloor$.
\end{baitoan}

\begin{baitoan}[\cite{Binh_Toan_7_tap_1}, \S7, \textbf{75.}]
	Cho $a,b\in\mathbb{R}$ sao cho các tập hợp $\{a^2 + a;b\}$ \& $\{b^2 + b;b\}$ bằng nhau. Chứng minh $a = b$.
\end{baitoan}

%------------------------------------------------------------------------------%

\newpage
\section{Hình Học Trực Quan}

%------------------------------------------------------------------------------%

\section{Góc. Đường Thẳng Song Song}

%------------------------------------------------------------------------------%

\section{1 Số Yếu Tố Thống Kê \& Xác Suất}

%------------------------------------------------------------------------------%

\section{Biểu Thức Đại Số}

%------------------------------------------------------------------------------%

\section{Tam Giác}

%------------------------------------------------------------------------------%

\printbibliography[heading=bibintoc]
	
\end{document}