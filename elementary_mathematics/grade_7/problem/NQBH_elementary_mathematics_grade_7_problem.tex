\documentclass{article}
\usepackage[backend=biber,natbib=true,style=authoryear]{biblatex}
\addbibresource{/home/hong/1_NQBH/reference/bib.bib}
\usepackage[utf8]{vietnam}
\usepackage{tocloft}
\renewcommand{\cftsecleader}{\cftdotfill{\cftdotsep}}
\usepackage[colorlinks=true,linkcolor=blue,urlcolor=red,citecolor=magenta]{hyperref}
\usepackage{amsmath,amssymb,amsthm,mathtools,float,graphicx,algpseudocode,algorithm,tcolorbox}
\usepackage[inline]{enumitem}
\allowdisplaybreaks
\numberwithin{equation}{section}
\newtheorem{assumption}{Assumption}[section]
\newtheorem{conjecture}{Conjecture}[section]
\newtheorem{corollary}{Corollary}[section]
\newtheorem{hequa}{Hệ quả}[section]
\newtheorem{definition}{Definition}[section]
\newtheorem{dinhnghia}{Định nghĩa}[section]
\newtheorem{example}{Example}[section]
\newtheorem{vidu}{Ví dụ}[section]
\newtheorem{lemma}{Lemma}[section]
\newtheorem{notation}{Notation}[section]
\newtheorem{principle}{Principle}[section]
\newtheorem{problem}{Problem}[section]
\newtheorem{baitoan}{Bài toán}[section]
\newtheorem{proposition}{Proposition}[section]
\newtheorem{question}{Question}[section]
\newtheorem{cauhoi}{Câu hỏi}[section]
\newtheorem{remark}{Remark}[section]
\newtheorem{luuy}{Lưu ý}[section]
\newtheorem{theorem}{Theorem}[section]
\newtheorem{dinhly}{Định lý}[section]
\usepackage[left=0.5in,right=0.5in,top=1.5cm,bottom=1.5cm]{geometry}
\usepackage{fancyhdr}
\pagestyle{fancy}
\fancyhf{}
\lhead{\small \textsc{Sect.} ~\thesection}
\rhead{\small \nouppercase{\leftmark}}
\renewcommand{\sectionmark}[1]{\markboth{#1}{}}
\cfoot{\thepage}
\def\labelitemii{$\circ$}

\title{Some Topics in Elementary Mathematics\texttt{/}Grade }
\author{Nguyễn Quản Bá Hồng\footnote{Independent Researcher, Ben Tre City, Vietnam\\e-mail: \texttt{nguyenquanbahong@gmail.com}; website: \url{https://nqbh.github.io}.}}
\date{\today}

\begin{document}
\maketitle
\begin{abstract}
	1 bộ sưu tập các bài toán chọn lọc từ cơ bản đến nâng cao cho Toán sơ cấp lớp 7. Tài liệu này là phần bài tập bổ sung cho tài liệu chính \href{https://github.com/NQBH/hobby/blob/master/elementary_mathematics/grade_7/NQBH_elementary_mathematics_grade_7.pdf}{GitHub\texttt{/}NQBH\texttt{/}hobby\texttt{/}elementary mathematics\texttt{/}grade 6\texttt{/}lecture}\footnote{\textsc{url}: \url{https://github.com/NQBH/hobby/blob/master/elementary_mathematics/grade_7/NQBH_elementary_mathematics_grade_7.pdf}.} của tác giả viết cho Toán lớp 6. Phiên bản mới nhất của tài liệu này được lưu trữ ở link sau: \href{https://github.com/NQBH/hobby/blob/master/elementary_mathematics/grade_7/problem/NQBH_elementary_mathematics_grade_7_problem.pdf}{GitHub\texttt{/}NQBH\texttt{/}hobby\texttt{/}elementary mathematics\texttt{/}grade 6\texttt{/}problem}\footnote{\textsc{url}: \url{https://github.com/NQBH/hobby/blob/master/elementary_mathematics/grade_7/problem/NQBH_elementary_mathematics_grade_7_problem.pdf}.}.
\end{abstract}
\tableofcontents
\newpage

%------------------------------------------------------------------------------%

\section{Số Hữu Tỷ}

\subsection{Tập Hợp Các Số Hữu Tỷ $\mathbb{Q}$}
Với phân số $\frac{a}{b}$ tối giản, $a,\in\mathbb{Z}$, $b\ne 0$, $\mbox{ƯCLN}(a,b) = 1$ thì các phân số có dạng $\frac{na}{nb}$, $\forall n\in\mathbb{Z}^\star$, đều biểu diễn phân số $\frac{a}{b}$. Để so sánh 2 hay nhiều số hữu tỷ, chuyển chúng về cùng 1 trong 2 dạng: dạng phân số hoặc dạng biểu diễn thập phân, rồi so sánh chúng dựa vào các quy tắc đã học ở Toán 6.

\begin{baitoan}[\cite{Trong_Toan_7_2022}, \textbf{10.}, p. 6]
	So sánh 2 số hữu tỷ $\frac{a}{b}$ ($a,b\in\mathbb{Z}$, $b\ne 0$) với số $0$ khi $a,b$ cùng dấu \& khi $a,b$ khác dấu.
\end{baitoan}

\begin{proof}[Giải]
	$a = 0\Rightarrow\frac{a}{b} = 0$, $ab > 0\Rightarrow\frac{a}{b} > 0$, \& $ab < 0\Rightarrow\frac{a}{b} < 0$.
\end{proof}

\begin{baitoan}[\cite{Trong_Toan_7_2022}, \textbf{11.}, p. 6]
	Giả sử $x = \frac{a}{m},y = \frac{b}{m},\ (a,b,m\in\mathbb{Z},\,m > 0)$ \& $x < y$. Chứng minh $x < z < y$ với $z\coloneqq\frac{a + b}{2m}$ ($z$ là \emph{trung bình cộng} của $x$ \& $y$, i.e., $z = \frac{x + y}{2}$).
\end{baitoan}

\begin{proof}[Chứng minh]
	$x < y\Rightarrow x + x < x + y < y + y\Rightarrow x < \frac{x + y}{2} < y$, mà $\frac{x + y}{2} = \frac{1}{2}\left(\frac{a}{m} + \frac{b}{m}\right) = \frac{a + b}{2m} = z$, nên $x < z < y$.
\end{proof}

%------------------------------------------------------------------------------%

\subsection{$\pm$ on $\mathbb{Q}$}
Tính tổng các phân số cùng mẫu số:
\begin{align*}
	\sum_{i=1}^{n} \frac{a_i}{b} = \frac{\sum_{i=1}^n a_i}{b},\mbox{ i.e., } \frac{a_1}{b} + \cdots + \frac{a_n}{b} = \frac{a_1 + \cdots + a_n}{b},\ \forall a_i,b\in\mathbb{Z},\,b\ne 0,\,\forall i = 1,\ldots,n.
\end{align*}
Tính tổng các phân số khác mẫu số: Quy đồng mẫu số các phân số đó với mẫu số chung là BCNN của các mẫu số các phân số đó rồi cộng lại:
\begin{align*}
	\sum_{i=1}^{n} \frac{a_i}{b_i} = \frac{\sum_{i=1}^n a_i\frac{\operatorname{BCNN}(b_1,\ldots,b_n)}{b_i}}{\operatorname{BCNN}(b_1,\ldots,b_n)},\mbox{i.e., }\frac{a_1}{b_1} + \cdots + \frac{a_n}{b_n} = \frac{a_1\frac{\operatorname{BCNN}(b_1,\ldots,b_n)}{b_1} + \cdots + a_n\frac{\operatorname{BCNN}(b_1,\ldots,b_n)}{b_n}}{\operatorname{BCNN}(b_1,\ldots,b_n)},&\\\forall a_i,b_i\in\mathbb{Z},\,b_i\ne 0,\,\forall i = 1,\ldots,n.&
\end{align*}

%------------------------------------------------------------------------------%

\subsection{$\cdot,:$ on $\mathbb{Q}$}
``\textit{Phép nhân 2 hay nhiều số hữu tỷ}:
\begin{enumerate*}
	\item[$\bullet$] Xác định dấu bằng cách đếm các thừa số âm, nếu chẵn thì kết quả dương, nếu lẻ thì kết quả âm.
	\item[$\bullet$] Nhân phần số tự nhiên của tử với tử, mẫu với mẫu rồi rút gọn.
\end{enumerate*}

\noindent\textit{Phép chia 2 số hữu tỷ}: Ta lấy số hữu tỷ bị chia nhân với nghịch đảo số hữu tỷ chia rồi làm như phép nhân. Thương của phép chia $x\in\mathbb{Q}$ cho số hữu tỷ $y\in\mathbb{Q}^\star$ gọi là \textit{tỷ số} của 2 số $x$ \& $y$, ký hiệu là $\frac{x}{y}$ hay $x:y$.'' -- \cite[\S3, p. 10]{Trong_Toan_7_2022}

%------------------------------------------------------------------------------%

\section{Số Thực}

%------------------------------------------------------------------------------%

\section{Hình Học Trực Quan}

%------------------------------------------------------------------------------%

\section{Góc. Đường Thẳng Song Song}

%------------------------------------------------------------------------------%

\section{1 Số Yếu Tố Thống Kê \& Xác Suất}

%------------------------------------------------------------------------------%

\section{Biểu Thức Đại Số}

%------------------------------------------------------------------------------%

\section{Tam Giác}

%------------------------------------------------------------------------------%

\printbibliography[heading=bibintoc]
	
\end{document}