\documentclass{article}
\usepackage[backend=biber,natbib=true,style=authoryear]{biblatex}
\addbibresource{/home/hong/1_NQBH/reference/bib.bib}
\usepackage[utf8]{vietnam}
\usepackage{tocloft}
\renewcommand{\cftsecleader}{\cftdotfill{\cftdotsep}}
\usepackage[colorlinks=true,linkcolor=blue,urlcolor=red,citecolor=magenta]{hyperref}
\usepackage{amsmath,amssymb,amsthm,mathtools,float,graphicx,algpseudocode,algorithm,tcolorbox}
\usepackage[inline]{enumitem}
\allowdisplaybreaks
\numberwithin{equation}{section}
\newtheorem{assumption}{Assumption}[section]
\newtheorem{conjecture}{Conjecture}[section]
\newtheorem{corollary}{Corollary}[section]
\newtheorem{hequa}{Hệ quả}[section]
\newtheorem{definition}{Definition}[section]
\newtheorem{dinhnghia}{Định nghĩa}[section]
\newtheorem{example}{Example}[section]
\newtheorem{vidu}{Ví dụ}[section]
\newtheorem{lemma}{Lemma}[section]
\newtheorem{notation}{Notation}[section]
\newtheorem{principle}{Principle}[section]
\newtheorem{problem}{Problem}[section]
\newtheorem{baitoan}{Bài toán}[section]
\newtheorem{proposition}{Proposition}[section]
\newtheorem{question}{Question}[section]
\newtheorem{cauhoi}{Câu hỏi}[section]
\newtheorem{remark}{Remark}[section]
\newtheorem{luuy}{Lưu ý}[section]
\newtheorem{theorem}{Theorem}[section]
\newtheorem{dinhly}{Định lý}[section]
\usepackage[left=0.5in,right=0.5in,top=1.5cm,bottom=1.5cm]{geometry}
\usepackage{fancyhdr}
\pagestyle{fancy}
\fancyhf{}
\lhead{\small Sect.~\thesection}
\rhead{\small \nouppercase{\leftmark}}
\renewcommand{\sectionmark}[1]{\markboth{#1}{}}
\cfoot{\thepage}
\def\labelitemii{$\circ$}

\title{Some Topics in Elementary Mathematics\texttt{/}Grade }
\author{Nguyễn Quản Bá Hồng\footnote{Independent Researcher, Ben Tre City, Vietnam\\e-mail: \texttt{nguyenquanbahong@gmail.com}; website: \url{https://nqbh.github.io}.}}
\date{\today}

\begin{document}
\maketitle
\begin{abstract}
	1 bộ sưu tập các bài toán chọn lọc từ cơ bản đến nâng cao cho Toán sơ cấp lớp 7. Tài liệu này là phần bài tập bổ sung cho tài liệu chính \href{https://github.com/NQBH/hobby/blob/master/elementary_mathematics/grade_7/NQBH_elementary_mathematics_grade_7.pdf}{GitHub\texttt{/}NQBH\texttt{/}hobby\texttt{/}elementary mathematics\texttt{/}grade 6\texttt{/}lecture}\footnote{\textsc{url}: \url{https://github.com/NQBH/hobby/blob/master/elementary_mathematics/grade_7/NQBH_elementary_mathematics_grade_7.pdf}.} của tác giả viết cho Toán lớp 6. Phiên bản mới nhất của tài liệu này được lưu trữ ở link sau: \href{https://github.com/NQBH/hobby/blob/master/elementary_mathematics/grade_7/problem/NQBH_elementary_mathematics_grade_7_problem.pdf}{GitHub\texttt{/}NQBH\texttt{/}hobby\texttt{/}elementary mathematics\texttt{/}grade 6\texttt{/}problem}\footnote{\textsc{url}: \url{https://github.com/NQBH/hobby/blob/master/elementary_mathematics/grade_7/problem/NQBH_elementary_mathematics_grade_7_problem.pdf}.}.
\end{abstract}
\tableofcontents
\newpage

%------------------------------------------------------------------------------%

\section{Số Hữu Tỷ}

\subsection{Tập Hợp Các Số Hữu Tỷ $\mathbb{Q}$}
Với phân số $\frac{a}{b}$ tối giản, $a,\in\mathbb{Z}$, $b\ne 0$, $\mbox{ƯCLN}(a,b) = 1$ thì các phân số có dạng $\frac{na}{nb}$, $\forall n\in\mathbb{Z}^\star$, đều biểu diễn phân số $\frac{a}{b}$. Để so sánh 2 hay nhiều số hữu tỷ, chuyển chúng về cùng 1 trong 2 dạng: dạng phân số hoặc dạng biểu diễn thập phân, rồi so sánh chúng dựa vào các quy tắc đã học ở Toán 6.

\begin{baitoan}[\cite{Trong_Toan_7_2022}, \textbf{10.}, p. 6]
	So sánh 2 số hữu tỷ $\frac{a}{b}$ ($a,b\in\mathbb{Z}$, $b\ne 0$) với số $0$ khi $a,b$ cùng dấu \& khi $a,b$ khác dấu.
\end{baitoan}

\begin{proof}[Giải]
	$a = 0\Rightarrow\frac{a}{b} = 0$, $ab > 0\Rightarrow\frac{a}{b} > 0$, \& $ab < 0\Rightarrow\frac{a}{b} < 0$.
\end{proof}

\begin{baitoan}[\cite{Trong_Toan_7_2022}, \textbf{11.}, p. 6]
	Giả sử $x = \frac{a}{m},y = \frac{b}{m},\ (a,b,m\in\mathbb{Z},\,m > 0)$ \& $x < y$. Chứng minh $x < z < y$ với $z\coloneqq\frac{a + b}{2m}$ ($z$ là \emph{trung bình cộng} của $x$ \& $y$, i.e., $z = \frac{x + y}{2}$).
\end{baitoan}

\begin{proof}[Chứng minh]
	$x < y\Rightarrow x + x < x + y < y + y\Rightarrow x < \frac{x + y}{2} < y$, mà $\frac{x + y}{2} = \frac{1}{2}\left(\frac{a}{m} + \frac{b}{m}\right) = \frac{a + b}{2m} = z$, nên $x < z < y$.
\end{proof}

%------------------------------------------------------------------------------%

\subsection{$\pm$ on $\mathbb{Q}$}
Tính tổng các phân số cùng mẫu số:
\begin{align*}
	\sum_{i=1}^{n} \frac{a_i}{b} = \frac{\sum_{i=1}^n a_i}{b},\mbox{ i.e., } \frac{a_1}{b} + \cdots + \frac{a_n}{b} = \frac{a_1 + \cdots + a_n}{b},\ \forall a_i,b\in\mathbb{Z},\,b\ne 0,\,\forall i = 1,\ldots,n.
\end{align*}
Tính tổng các phân số khác mẫu số: Quy đồng mẫu số các phân số đó với mẫu số chung là BCNN của các mẫu số các phân số đó rồi cộng lại:
\begin{align*}
	\sum_{i=1}^{n} \frac{a_i}{b_i} = \frac{\sum_{i=1}^n a_i\frac{\operatorname{BCNN}(b_1,\ldots,b_n)}{b_i}}{\operatorname{BCNN}(b_1,\ldots,b_n)},\mbox{i.e., }\frac{a_1}{b_1} + \cdots + \frac{a_n}{b_n} = \frac{a_1\frac{\operatorname{BCNN}(b_1,\ldots,b_n)}{b_1} + \cdots + a_n\frac{\operatorname{BCNN}(b_1,\ldots,b_n)}{b_n}}{\operatorname{BCNN}(b_1,\ldots,b_n)},&\\\forall a_i,b_i\in\mathbb{Z},\,b_i\ne 0,\,\forall i = 1,\ldots,n.&
\end{align*}

%------------------------------------------------------------------------------%

\subsection{$\cdot,:$ on $\mathbb{Q}$}
``\textit{Phép nhân 2 hay nhiều số hữu tỷ}:
\begin{enumerate*}
	\item[$\bullet$] Xác định dấu bằng cách đếm các thừa số âm, nếu chẵn thì kết quả dương, nếu lẻ thì kết quả âm.
	\item[$\bullet$] Nhân phần số tự nhiên của tử với tử, mẫu với mẫu rồi rút gọn.
\end{enumerate*}

\noindent\textit{Phép chia 2 số hữu tỷ}: Ta lấy số hữu tỷ bị chia nhân với nghịch đảo số hữu tỷ chia rồi làm như phép nhân. Thương của phép chia $x\in\mathbb{Q}$ cho số hữu tỷ $y\in\mathbb{Q}^\star$ gọi là \textit{tỷ số} của 2 số $x$ \& $y$, ký hiệu là $\frac{x}{y}$ hay $x:y$.'' -- \cite[\S3, p. 10]{Trong_Toan_7_2022}

\begin{baitoan}[\cite{Trong_Toan_7_2022}, \textbf{3.}, p. 13]
	Cho số hữu tỷ $\frac{a}{b}$ với $a,b\in\mathbb{Z}$, $b > 0$. Chứng minh:
	\begin{enumerate*}
		\item[(a)] $\frac{a}{b} > 1\Leftrightarrow a > b$;
		\item[(b)] $\frac{a}{b} < 1\Leftrightarrow a < b$;
		\item[(c)] $((a < b)\land(a,c > 0))\Rightarrow\frac{a}{b} < \frac{a + c}{b + c}$;
		\item[(d)] $((a > b)\land(c > 0))\Rightarrow\frac{a}{b} > \frac{a + c}{b + c}$.
	\end{enumerate*}
\end{baitoan}

\begin{baitoan}[\cite{Binh_Toan_7_tap_1}, \S1, Ví dụ 1]
	Cho phân số $\frac{a}{b}\ne 1$. Tìm phân số $\frac{c}{d}$ sao cho $\frac{a}{b} + \frac{c}{d} = \frac{a}{b}\cdot\frac{c}{d}$.
\end{baitoan}

%------------------------------------------------------------------------------%

\subsection{Lũy thừa của 1 số hữu tỷ}

\begin{baitoan}[\cite{Binh_Toan_7_tap_1}, \S3, Ví dụ 2]
	Cho $x\in\mathbb{Q}$. Khi nào thì:
	\begin{enumerate*}
		\item[(a)] $x^2 = x$;
		\item[(b)] $x^2 > x$;
		\item[(c)] $x^2 < x$.
	\end{enumerate*}
\end{baitoan}

\begin{baitoan}[\cite{Binh_Toan_7_tap_1}, \S3, Ví dụ 3]
	Tìm $a,b,c\in\mathbb{Q}$ biết $ab = 2$, $bc = 3$, $ca = 54$.
\end{baitoan}

\begin{baitoan}[\cite{Binh_Toan_7_tap_1}, \S3, Ví dụ 4]
	Rút gọn $A = \sum_{i=0}^{50} 5^i = 1 + 5 + 5^2 + \cdots + 5^{50}$.
\end{baitoan}

\begin{baitoan}[\cite{Binh_Toan_7_tap_1}, \S3, Ví dụ 5]
	Cho $B = \sum_{i=1}^{99} \left(\frac{1}{2}\right)^i = \frac{1}{2} + \left(\frac{1}{2}\right)^2 + \cdots + \left(\frac{1}{2}\right)^{99}$. Chứng minh $B < 1$.
\end{baitoan}

\begin{baitoan}[\cite{Binh_Toan_7_tap_1}, \S3, \textbf{21.}]
	Chứng minh:
	\begin{enumerate*}
		\item[(a)] $7^6 + 7^5 - 7^4\ \vdots\ 55$;
		\item[(b)] $16^5 + 2^{15}\ \vdots\ 33$;
		\item[(c)] $81^7 - 27^9 - 9^{13}\ \vdots\ 405$.
	\end{enumerate*}
\end{baitoan}

\begin{baitoan}[\cite{Binh_Toan_7_tap_1}, \S3, \textbf{22.}]
	Điền vào chỗ trống $\ldots$ các từ ``bằng nhau'' hoặc ``đối nhau'' cho đúng:
	\begin{enumerate*}
		\item[(a)] Nếu $2$ số đối nhau thì bình phương của chúng $\ldots$
		\item[(b)] Nếu $2$ số đối nhau thì lập phương của chúng $\ldots$
		\item[(c)] Lũy thừa chẵn cung bậc của 2 số đối nhau thì $\ldots$
		\item[(d)] Lũy thừa lẻ cùng bậc của 2 số đối nhau thì $\ldots$
	\end{enumerate*}
\end{baitoan}

\begin{baitoan}[\cite{Binh_Toan_7_tap_1}, \S3, \textbf{23.} \& mở rộng]
	Các đẳng thức sau có đúng với mọi $a,b\in\mathbb{Q}$ hay không?
	\begin{enumerate*}
		\item[(a)] $-a^3 = (-a)^3$;
		\item[(b)] $-a^5 = (-a)^5$;
		\item[(c)] $-a^2 = (-a)^2$;
		\item[(d)] $-a^4 = (-a)^4$;
		\item[(e)] $-a^{2n+1} = (-a)^{2n+1}$, $\forall n\in\mathbb{N}$;
		\item[(f)] $a^{2n} = (-a)^{2n}$, $\forall n\in\mathbb{N}$;
		\item[(g)] $(a - b)^2 = (b - a)^2$;
		\item[(h)] $(a - b)^3 = -(b - a)^3$;
		\item[(g)] $(a - b)^{2n} = (b - a)^{2n}$, $\forall n\in\mathbb{N}$;
		\item[(h)] $(a - b)^{2n+1} = -(b - a)^{2n+1}$, $\forall n\in\mathbb{N}$.
	\end{enumerate*}
\end{baitoan}

\begin{baitoan}[\cite{Binh_Toan_7_tap_1}, \S3, \textbf{24.}]
	Tính:
	\begin{enumerate*}
		\item[(a)] $\left(\frac{1}{2}\right)^{15}\cdot\left(\frac{1}{4}\right)^{20}$;
		\item[(b)] $\left(\frac{1}{9}\right)^{25}:\left(\frac{1}{3}\right)^{30}$;
		\item[(c)] $\left(\frac{1}{16}\right)^3:\left(\frac{1}{8}\right)^2$;
		\item[(d)] $(x^3)^2:(x^2)^3$ với $x\ne 0$.
	\end{enumerate*}
\end{baitoan}

\begin{baitoan}[\cite{Binh_Toan_7_tap_1}, \S3, \textbf{25.}]
	Viết số $64$ dưới dạng $a^n$ với $a\in\mathbb{Z}$. Có bao nhiêu cách viết?
\end{baitoan}

\begin{baitoan}[\cite{Binh_Toan_7_tap_1}, \S3, \textbf{26.}]
	Rút gọn biểu thức: $A = \dfrac{4^5\cdot 9^4 - 2\cdot 6^9}{2^{10}\cdot 3^8 + 6^8\cdot 20}$.
\end{baitoan}

\begin{baitoan}[\cite{Binh_Toan_7_tap_1}, \S3, \textbf{27.}]
	Cho $S_n = \sum_{i=1}^{n-1} (-1)^{i-1}i = 1 - 2 + 3 - 4 + \cdots + (-1)^{n-1}n$ với $n\in\mathbb{N}^\star$. Tính $S_{35} + S_{60}$.
\end{baitoan}

\begin{baitoan}[\cite{Binh_Toan_7_tap_1}, \S3, \textbf{28.}]
	Cho $A = 1 - 5 + 9 - 13 + 17 - 21 + 25 - \cdots$ ($n$ số hạng, giá trị tuyệt đối của số sau lớn hơn giá trị tuyệt đối của số hạng trước $4$ đơn vị, các dấu $+$ \& $-$ xen kẽ).
	\begin{enumerate*}
		\item[(a)] Tính $A$ theo $n$.
		\item[(b)] Viết số hạng thứ $n$ của biểu thức $A$ theo $n$ (chú ý dùng lũy thừa để biểu thị dấu của số hạng đó).
	\end{enumerate*}
\end{baitoan}

\begin{baitoan}[\cite{Binh_Toan_7_tap_1}, \S3, \textbf{29.}]
	Với giá trị nào của các chữ thì các biểu thức sau có giá trị là số $0$, số dương, số âm?
	\begin{enumerate*}
		\item[(a)] $P = \frac{a^2b}{c}$;
		\item[(b)] $Q = \frac{x^3}{yz}$.
	\end{enumerate*}
\end{baitoan}

\begin{baitoan}[\cite{Binh_Toan_7_tap_1}, \S3, \textbf{30.}]
	Cho $2$ số hữu tỷ $a$ \& $b$ trái dấu trong đó $|a| = b^5$. Xác định dấu của mỗi số.
\end{baitoan}

\begin{baitoan}[\cite{Binh_Toan_7_tap_1}, \S3, \textbf{31.}]
	Viết các số sau dưới dạng lũy thừa của $2$: $16,64,1,\frac{1}{32},\frac{1}{8},0.5,0.25$.
\end{baitoan}

\begin{baitoan}[\cite{Binh_Toan_7_tap_1}, \S3, \textbf{32.}]
	\begin{enumerate*}
		\item[(a)] Viết các số sau thành lũy thừa với số mũ âm: $\frac{1}{1000000},0.00000002$.
		\item[(b)] Viết các số sau dưới dạng số thập phân: $10^{-7}$, $2.5\cdot 10^{-6}$.
	\end{enumerate*}
\end{baitoan}

\begin{baitoan}[\cite{Binh_Toan_7_tap_1}, \S3, \textbf{33.}]
	Tính xem $A$ gấp mấy lần $B$:
	\begin{enumerate*}
		\item[(a)] $A = 3.4\cdot 10^{-8}$, $B = 34\cdot 10^{-9}$;
		\item[(b)] $A = 10^{-4} + 10^{-3} + 10^{-2}$, $B = 10^{-9}$.
	\end{enumerate*}
\end{baitoan}

\begin{baitoan}[\cite{Binh_Toan_7_tap_1}, \S3, \textbf{34.}]
	So sánh:
	\begin{enumerate*}
		\item[(a)] $\left(-\frac{1}{16}\right)^{100}$ \& $\left(-\frac{1}{2}\right)^{500}$;
		\item[(b)] $(-32)^9$ \& $(-18)^{13}$.
	\end{enumerate*}
\end{baitoan}

\begin{baitoan}[\cite{Binh_Toan_7_tap_1}, \S3, \textbf{35.}]
	Sắp xếp $a,b,c\in\mathbb{Q}$ theo thứ tự từ nhỏ đến lớn: $a = 2^{100}$, $b = 3^{75}$, $c = 5^{50}$.
\end{baitoan}

\begin{baitoan}[\cite{Binh_Toan_7_tap_1}, \S3, \textbf{36.}]
	Trong các câu sau, câu nào đúng với mọi $a\in\mathbb{Q}$?
	\begin{enumerate*}
		\item[(a)] Nếu $a < 0$ thì $a^2 > 0$;
		\item[(b)] Nếu $a^2 > 0$ thì $a > 0$;
		\item[(c)] Nếu $a < 0$ thì $a^2 > a$;
		\item[(d)] Nếu $a^2 > a$ thì $a > 0$;
		\item[(e)] Nếu $a^2 > a$ thì $a < 0$.
	\end{enumerate*}
\end{baitoan}

\begin{baitoan}[\cite{Binh_Toan_7_tap_1}, \S3, \textbf{37.}]
	\begin{enumerate*}
		\item[(a)] Cho $a^m = a^n$ ($a\in\mathbb{Q}$, $m,n\in\mathbb{N}$). Tìm $m,n$.
		\item[(b)] Cho $a^m > a^n$ ($a\in\mathbb{Q}$, $a > 0$, $m,n\in\mathbb{N}$). So sánh $m$ \& $n$.
	\end{enumerate*}
\end{baitoan}

\begin{baitoan}[\cite{Binh_Toan_7_tap_1}, \S3, \textbf{38.}]
	Tìm $x\in\mathbb{Q}$, biết rằng:
	\begin{enumerate*}
		\item[(a)] $(2x - 1)^4 = 81$;
		\item[(b)] $(x - 1)^5 = -32$;
		\item[(c)] $(2x - 1)^6 = (2x - 1)^8$.
	\end{enumerate*}
\end{baitoan}

\begin{baitoan}[\cite{Binh_Toan_7_tap_1}, \S3, \textbf{39.}]
	Tìm $x\in\mathbb{N}$, biết rằng:
	\begin{enumerate*}
		\item[(a)] $5^x + 5^{x+2} = 650$;
		\item[(b)] $3^{x-1} + 5\cdot 3^{x-1} = 162$.
	\end{enumerate*}
\end{baitoan}

\begin{baitoan}[\cite{Binh_Toan_7_tap_1}, \S3, \textbf{40.}]
	Tìm $x,y\in\mathbb{N}$, biết rằng:
	\begin{enumerate*}
		\item[(a)] $2^{x+1}\cdot 3^y = 12^x$;
		\item[(b)] $10^x:5^y = 20^y$;
		\item[(c)] $2^x = 4^{y-1}$ \& $27^y = 3^{x+8}$.
	\end{enumerate*}
\end{baitoan}

\begin{baitoan}[\cite{Binh_Toan_7_tap_1}, \S3, \textbf{41.}]
	Tìm $a,b,c\in\mathbb{Q}$, biết rằng:
	\begin{enumerate*}
		\item[(a)] $ab = \frac{3}{5}$, $bc = \frac{4}{5}$, $ca = \frac{3}{4}$.
		\item[(b)] $a(a + b + c) = -12$, $b(a + b + c) = 18$, $c(a + b + c) = 30$;
		\item[(c)] $ab = c$, $bc = 4a$, $ac = 9b$.
	\end{enumerate*}
\end{baitoan}

\begin{baitoan}[\cite{Binh_Toan_7_tap_1}, \S3, \textbf{42.}]
	Cho $a,b,c,d,e\in\mathbb{N}$ thỏa mãn $a^b = b^c = c^d = d^e = e^a$. Chứng minh $a = b = c = d = e$.
\end{baitoan}

\begin{baitoan}[\cite{Binh_Toan_7_tap_1}, \S3, \textbf{43.}]
	Cho $A = \prod_{i=2}^{100} \frac{1}{i^2} - 1 = \left(\frac{1}{2^2} - 1\right)\left(\frac{1}{3^2} - 1\right)\left(\frac{1}{4^2} - 1\right)\cdots\left(\frac{1}{100^2} - 1\right)$. So sánh $A$ với $-\frac{1}{2}$.
\end{baitoan}

\begin{baitoan}[\cite{Binh_Toan_7_tap_1}, \S3, \textbf{44.}]
	Rút gọn $A = \sum_{i=1}^{100} (-1)^i2^i = 2^{100} - 2^{99} + 2^{98} - 2^{97} + \cdots + 2^2 - 2$.
\end{baitoan}

\begin{baitoan}[\cite{Binh_Toan_7_tap_1}, \S3, \textbf{45.}]
	Rút gọn $B = \sum_{i=
	}^{100} (-1)^i3^i = 3^{100} - 3^{99} + 3^{98} - 3^{97} + \cdots + 3^2 - 3 + 1$.
\end{baitoan}

\begin{baitoan}[\cite{Binh_Toan_7_tap_1}, \S3, \textbf{46.}]
	Cho $C = \sum_{i=1}^{99} \frac{1}{3^i} = \frac{1}{3} + \frac{1}{3^2} + \cdots + \frac{1}{3^{99}}$. Chứng minh $C < \frac{1}{2}$.
\end{baitoan}

\begin{baitoan}[\cite{Binh_Toan_7_tap_1}, \S3, \textbf{47.}]
	Chứng minh $\frac{3}{1^2\cdot 2^2} + \frac{5}{2^2\cdot 3^2} + \frac{7}{3^2\cdot 4^2} + \cdots + \frac{19}{9^2\cdot 10^2} < 1$.
\end{baitoan}

\begin{baitoan}[\cite{Binh_Toan_7_tap_1}, \S3, \textbf{48.}]
	Chứng minh $\sum_{i=1}^{100} \frac{i}{3^i} = \frac{1}{3} + \frac{2}{3^2} + \frac{3}{3^3} + \cdots + \frac{100}{3^{100}} < \frac{3}{4}$.
\end{baitoan}

\begin{baitoan}[\cite{Binh_Toan_7_tap_1}, \S3, \textbf{49.}]
	Ta không có $2^m + 2^n = 2^{m+n}$, $\forall m,n\in\mathbb{N}^\star$. Nhưng có những số nguyên dương $m,n$ có tính chất trên. Tìm các số đó.
\end{baitoan}

\begin{baitoan}[\cite{Binh_Toan_7_tap_1}, \S3, \textbf{50.}]
	Tìm $m,n\in\mathbb{N}^\star$ sao cho $2^m - 2^n = 256$.
\end{baitoan}

\begin{baitoan}[\cite{Binh_Toan_7_tap_1}, \S3, \textbf{51.}]
	Cho 1 bảng vuông $3\times 3$ ô. Trong mỗi ô của bảng viết số $1$ hoặc số $-1$. Gọi $d_i$ là tích các số trên dòng $i$ ($i = 1,2,3$), $c_k$ là tích các số trên cột $k$ ($k = 1,2,3$).
	\begin{enumerate*}
		\item[(a)] Chứng minh rằng không thể xảy ra $d_1 + d_2 + d_3 + c_1 + c_2 + c_3 = 0$.
		\item[(b)] Xét bài toán trên đối với bảng vuông $n\times n$.
	\end{enumerate*}
\end{baitoan}

\begin{baitoan}[\cite{Binh_Toan_7_tap_1}, \S3, \textbf{52.}]
	Cho $n$ số $x_1,\ldots,x_n$, mỗi số bằng $1$ hoặc $-1$. Biết rằng tổng của $n$ tích $x_1x_2$, $x_2x_3$, $x_3x_4,\ldots,x_nx_1$ bằng $0$. Chứng minh $n\ \vdots\ 4$.
\end{baitoan}

%------------------------------------------------------------------------------%

\section{Số Thực}

\begin{baitoan}
	Chứng minh: $(x^2 + m^2)(x^2 + n^2) = 0\Leftrightarrow x^2 + m^2n^2 = 0$ \& $(x^2 + m^2)(x^2 + n^2)\ne 0\Leftrightarrow x^2 + m^2n^2\ne 0$, $\forall x,m,n\in\mathbb{R}$.
\end{baitoan}
\textit{Ý nghĩa}: Điều kiện để các công thức nhân chia lũy thừa cùng cơ số xác định.

%------------------------------------------------------------------------------%

\section{Hình Học Trực Quan}

%------------------------------------------------------------------------------%

\section{Góc. Đường Thẳng Song Song}

%------------------------------------------------------------------------------%

\section{1 Số Yếu Tố Thống Kê \& Xác Suất}

%------------------------------------------------------------------------------%

\section{Biểu Thức Đại Số}

%------------------------------------------------------------------------------%

\section{Tam Giác}

%------------------------------------------------------------------------------%

\printbibliography[heading=bibintoc]
	
\end{document}