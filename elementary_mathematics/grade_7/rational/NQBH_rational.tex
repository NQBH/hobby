\documentclass{article}
\usepackage[backend=biber,natbib=true,style=authoryear]{biblatex}
\addbibresource{/home/nqbh/reference/bib.bib}
\usepackage[utf8]{vietnam}
\usepackage{tocloft}
\renewcommand{\cftsecleader}{\cftdotfill{\cftdotsep}}
\usepackage[colorlinks=true,linkcolor=blue,urlcolor=red,citecolor=magenta]{hyperref}
\usepackage{amsmath,amssymb,amsthm,mathtools,float,graphicx,algpseudocode,algorithm,tcolorbox,tikz,tkz-tab,subcaption}
\DeclareMathOperator{\arccot}{arccot}
\usepackage[inline]{enumitem}
\allowdisplaybreaks
\numberwithin{equation}{section}
\newtheorem{assumption}{Assumption}[section]
\newtheorem{nhanxet}{Nhận xét}[section]
\newtheorem{conjecture}{Conjecture}[section]
\newtheorem{corollary}{Corollary}[section]
\newtheorem{hequa}{Hệ quả}[section]
\newtheorem{definition}{Definition}[section]
\newtheorem{dinhnghia}{Định nghĩa}[section]
\newtheorem{example}{Example}[section]
\newtheorem{vidu}{Ví dụ}[section]
\newtheorem{lemma}{Lemma}[section]
\newtheorem{notation}{Notation}[section]
\newtheorem{principle}{Principle}[section]
\newtheorem{problem}{Problem}[section]
\newtheorem{baitoan}{Bài toán}[section]
\newtheorem{proposition}{Proposition}[section]
\newtheorem{menhde}{Mệnh đề}[section]
\newtheorem{question}{Question}[section]
\newtheorem{cauhoi}{Câu hỏi}[section]
\newtheorem{quytac}{Quy tắc}
\newtheorem{remark}{Remark}[section]
\newtheorem{luuy}{Lưu ý}[section]
\newtheorem{theorem}{Theorem}[section]
\newtheorem{tiende}{Tiên đề}[section]
\newtheorem{dinhly}{Định lý}[section]
\usepackage[left=0.5in,right=0.5in,top=1.5cm,bottom=1.5cm]{geometry}
\usepackage{fancyhdr}
\pagestyle{fancy}
\fancyhf{}
\lhead{\small Sect.~\thesection}
\rhead{\small\nouppercase{\leftmark}}
\renewcommand{\subsectionmark}[1]{\markboth{#1}{}}
\cfoot{\thepage}
\def\labelitemii{$\circ$}
\DeclareRobustCommand{\divby}{%
	\mathrel{\vbox{\baselineskip.65ex\lineskiplimit0pt\hbox{.}\hbox{.}\hbox{.}}}%
}

\title{Rational -- Số Hữu Tỷ $\mathbb{Q}$}
\author{Nguyễn Quản Bá Hồng\footnote{Independent Researcher, Ben Tre City, Vietnam\\e-mail: \texttt{nguyenquanbahong@gmail.com}; website: \url{https://nqbh.github.io}.}}
\date{\today}

\begin{document}
\maketitle
\begin{abstract}
	\textsc{[en]} This text is a collection of problems, from easy to advanced, about rational. This text is also a supplementary material for my lecture note on Elementary Mathematics grade 7, which is stored \& downloadable at the following link: \href{https://github.com/NQBH/hobby/blob/master/elementary_mathematics/grade_7/NQBH_elementary_mathematics_grade_7.pdf}{GitHub\texttt{/}NQBH\texttt{/}hobby\texttt{/}elementary mathematics\texttt{/}grade 7\texttt{/}lecture}\footnote{\textsc{url}: \url{https://github.com/NQBH/hobby/blob/master/elementary_mathematics/grade_7/NQBH_elementary_mathematics_grade_7.pdf}.}. The latest version of this text has been stored \& downloadable at the following link: \href{https://github.com/NQBH/hobby/blob/master/elementary_mathematics/grade_7/rational/NQBH_rational.pdf}{GitHub\texttt{/}NQBH\texttt{/}hobby\texttt{/}elementary mathematics\texttt{/}grade 7\texttt{/}rational $\mathbb{Q}$}\footnote{\textsc{url}: \url{https://github.com/NQBH/hobby/blob/master/elementary_mathematics/grade_7/rational/NQBH_rational.pdf}.}.
	\vspace{2mm}
	
	\textsc{[vi]} Tài liệu này là 1 bộ sưu tập các bài tập chọn lọc từ cơ bản đến nâng cao về phân thức đại số \& phân thức đại số hữu tỷ. Tài liệu này là phần bài tập bổ sung cho tài liệu chính -- bài giảng \href{https://github.com/NQBH/hobby/blob/master/elementary_mathematics/grade_7/NQBH_elementary_mathematics_grade_7.pdf}{GitHub\texttt{/}NQBH\texttt{/}hobby\texttt{/}elementary mathematics\texttt{/}grade 7\texttt{/}lecture} của tác giả viết cho Toán Sơ Cấp lớp 8. Phiên bản mới nhất của tài liệu này được lưu trữ \& có thể tải xuống ở link sau: \href{https://github.com/NQBH/hobby/blob/master/elementary_mathematics/grade_7/rational/NQBH_rational.pdf}{GitHub\texttt{/}NQBH\texttt{/}hobby\texttt{/}elementary mathematics\texttt{/}grade 7\texttt{/}rational $\mathbb{Q}$}.
\end{abstract}
\setcounter{secnumdepth}{4}
\setcounter{tocdepth}{3}
\tableofcontents

%------------------------------------------------------------------------------%

\section{Problem}

\subsection{Tập Hợp $\mathbb{Q}$ Các Số Hữu Tỷ}
``\begin{enumerate*}
	\item[\textbf{1.}] \textit{Số hữu tỷ} là số viết được dưới dạng $\frac{a}{b}$ với $a,b\in\mathbb{Z}$, $b\ne 0$. \textit{Tập hợp các số hữu tỷ} được ký hiệu là $\mathbb{Q}$.
	\item[\textbf{2.}] \textit{Biểu diễn số hữu tỷ trên trục số}: Trên trục số, điểm biểu diễn số hữu tỷ $x$ gọi là \textit{điểm $x$}. Các số khác nhau biểu diễn bởi những điểm khác nhau.
	\item[\textbf{3.}] \textit{Số đối} của 1 số hữu tỷ: Trên trục số, 2 số hữu tỷ có điểm biểu diễn nằm về 2 phía của gốc $O$ \& cách đều gốc $O$ được gọi là \textit{2 số đối nhau}. \textit{Số đối} của số hữu tỷ $x$ ký hiệu là $-x$. Số đối của $0$ là $0$.
	\item[\textbf{4.}] \textit{So sánh các số hữu tỷ}: Để so sánh $x,y\in\mathbb{Q}$ ta làm như sau: Viết $x,y$ dưới dạng 2 phân số có cùng mẫu dương $x = \frac{a}{m}$, $y = \frac{b}{m}$, $m > 0$. Sau đó so sánh các tử số:
	\begin{enumerate*}
		\item[$\bullet$] Nếu $a < b$ thì $x < y$.
		\item[$\bullet$] Nếu $a = b$ thì $x = y$.
		\item[$\bullet$] Nếu $a > b$ thì $x > y$.
	\end{enumerate*}
	Số hữu tỷ lớn hơn $0$ gọi là \textit{số hữu tỷ dương}. Số hữu tỷ nhỏ hơn $0$ gọi là \textit{số hữu tỷ âm}. Số $0$ không là số hữu tỷ dương, cũng không là số hữu tỷ âm.
\end{enumerate*}

Cho các số hữu tỷ $x = \frac{a}{b}$ \& $y = \frac{c}{d}$, $a,b,c,d\in\mathbb{Z}$, $b > 0$, $d > 0$. Luôn có:
\begin{enumerate*}
	\item[$\bullet$] $x = y\Leftrightarrow ad = bc$, $x < y\Leftrightarrow ad < bc$, $x > y\Leftrightarrow ad > bc$.
	\item[$\bullet$] $\frac{a}{b} < \frac{a + c}{b + d} < \frac{c}{d}$.
\end{enumerate*}'' -- \cite[pp. 4--5]{Tuyen_Toan_7}

\begin{baitoan}[\cite{Tuyen_Toan_7}, Ví dụ 1, p. 5]
	Cho $x = \frac{12}{b - 15}$ với $b\in\mathbb{Z}$. Xác định $b$ để:
	\begin{enumerate*}
		\item[(a)] $x$ là 1 số hữu tỷ;
		\item[(b)] $x$ là 1 số hữu tỷ dương;
		\item[(c)] $x$ là 1 số hữu tỷ âm;
		\item[(d)] $0 < x < 1$.
	\end{enumerate*}
\end{baitoan}

\begin{baitoan}[\cite{Tuyen_Toan_7}, Ví dụ 2, p. 5]
	So sánh các số hữu tỷ sau: $\frac{-16}{27}$, $\frac{-16}{29}$, $\frac{-19}{27}$.
\end{baitoan}

\begin{baitoan}[\cite{Tuyen_Toan_7}, \textbf{1.}, p. 5]
	Cho các số hữu tỷ $x = \frac{-5}{7}$, $y = \frac{-2}{3}$. Các số hữu tỷ này còn được biểu diễn bởi phân số nào trong các phân số sau: $\frac{9}{11}$, $\frac{4}{-6}$, $\frac{15}{-21}$, $\frac{-35}{49}$, $\frac{-10}{15}$, $\frac{-6}{-9}$.
\end{baitoan}

\begin{baitoan}[\cite{Tuyen_Toan_7}, \textbf{2.}, p. 6]
	Sắp xếp các số hữu tỷ sau theo thứ tự tăng dần:
	\begin{enumerate*}
		\item[(a)] $\frac{19}{33}$, $\frac{6}{11}$, $\frac{13}{22}$;
		\item[(b)] $\frac{-18}{12}$, $\frac{-10}{7}$; $\frac{-8}{5}$.
	\end{enumerate*}
\end{baitoan}

\begin{baitoan}[\cite{Tuyen_Toan_7}, \textbf{3.}, p. 6]
	So sánh các số hữu tỷ sau bằng cách nhanh nhất:
	\begin{enumerate*}
		\item[(a)] $-5$ \& $\frac{1}{63}$;
		\item[(b)] $\frac{-18}{17}$ \& $\frac{-999}{1000}$;
		\item[(c)] $\frac{-17}{35}$ \& $\frac{-43}{85}$;
		\item[(d)] $-0.76$ \& $\frac{-19}{28}$.
	\end{enumerate*}
\end{baitoan}

\begin{baitoan}[\cite{Tuyen_Toan_7}, \textbf{4.}, p. 6]
	Tìm các số hữu tỷ biểu diễn dưới dạng phân số có mẫu số bằng $10$, lớn hơn $\frac{-7}{13}$ nhưng nhỏ hơn $-\frac{-4}{13}$.
\end{baitoan}

\begin{baitoan}[\cite{Tuyen_Toan_7}, \textbf{5.}, p. 6]
	Dùng $4$ chữ số $1$ \& dấu $-$ (nếu cần thiết) để biểu diễn (không dùng phép tính lũy thừa):
	\begin{enumerate*}
		\item[(a)] Các số nguyên $-1$, $-111$;
		\item[(b)] Số hữu tỷ âm lớn nhất.
	\end{enumerate*}
\end{baitoan}

\begin{baitoan}[\cite{Tuyen_Toan_7}, \textbf{6.}, p. 6]
	Cho các số nguyên dương $a < b < c < d < m < n$. Chứng minh: $\frac{a + c + m}{ a + b + c + d + m + n} < \frac{1}{2}$.
\end{baitoan}

\begin{baitoan}[\cite{Tuyen_Toan_7}, \textbf{7.}, p. 6]
	Với cùng 1 khối lượng thành phẩm, vàng 4 số 9 \& vàng 3 số 9, loại nào có hàm lượng vàng nhiều hơn?
\end{baitoan}

%------------------------------------------------------------------------------%

\subsection{$\pm,\cdot,:$ Trên $\mathbb{Q}$}

\begin{baitoan}[\cite{Binh_Toan_7_tap_1}, Ví dụ 1, p. 3]
	Tính $A = \frac{1}{2} - \frac{1}{3} - \frac{1}{6} - \frac{1}{2} - \frac{1}{3} - \frac{1}{6} - \frac{1}{2} - \frac{1}{3} - \frac{1}{6} - \cdots$ ($A$ có $300$ số hạng).
\end{baitoan}

\begin{baitoan}[\cite{Binh_Toan_7_tap_1}, Ví dụ 2, p. 4]
	Cho phân số $\frac{a}{b}\ne 1$.
	\begin{enumerate*}
		\item[(a)] Tìm phân số $x$ sao cho nhân $x$ với $\frac{a}{b}$ cũng bằng cộng $x$ với $\frac{a}{b}$.
		\item[(b)] Tìm giá trị của $x$ trong câu (a) nếu $\frac{a}{b} = \frac{7}{5}$, nếu $\frac{a}{b} = \frac{8}{11}$.
	\end{enumerate*}
\end{baitoan}

\begin{baitoan}[\cite{Binh_Toan_7_tap_1}, Ví dụ 3, p. 4]
	Tìm $x\in\mathbb{Q}$, $x < 0$ để $\frac{4}{x - 1}\in\mathbb{Z}$.
\end{baitoan}

\begin{baitoan}[\cite{Binh_Toan_7_tap_1}, Ví dụ 4, p. 5]
	Tân đạp xe từ trường về nhà với thời gian dự kiến. Nhưng Tân đã dùng $\frac{2}{3}$ thời gian dự kiến để đi $\frac{3}{4}$ quãng đường với vận tốc $v_1$, rồi đi quãng đường còn lại với vận tốc $v_2$ \& đã về nhà đúng thời điểm dự kiến. Tính tỷ số $v_1:v_2$.
\end{baitoan}

\begin{baitoan}[\cite{Binh_Toan_7_tap_1}, Mở rộng Ví dụ 4, p. 5]
	Tân đạp xe từ trường về nhà với thời gian dự kiến. Nhưng Tân đã dùng $a$ thời gian dự kiến để đi $b$ quãng đường với vận tốc $v_1$, $a,b > 0$, $a + b < 1$, rồi đi quãng đường còn lại với vận tốc $v_2$ \& đã về nhà đúng thời điểm dự kiến. Tính tỷ số $v_1:v_2$ theo $a,b$.
\end{baitoan}

\begin{baitoan}[\cite{Binh_Toan_7_tap_1}, \textbf{1.}, p. 5]
	So sánh các số hữu tỷ:
	\begin{enumerate*}
		\item[(a)] $-\frac{18}{91}$ \& $-\frac{23}{114}$;
		\item[(b)] $-\frac{22}{35}$ \& $-\frac{103}{177}$.
	\end{enumerate*}
\end{baitoan}

\begin{baitoan}[\cite{Binh_Toan_7_tap_1}, \textbf{2.}, p. 5]
	Tìm $2$ phân số có tử bằng $9$, biết giá trị của mỗi phân số đó lớn hơn $-\frac{11}{13}$ \& nhỏ hơn $-\frac{11}{15}$.
\end{baitoan}

\begin{baitoan}[\cite{Binh_Toan_7_tap_1}, \textbf{3.}, p. 5]
	Cho các số hữu tỷ $\frac{a}{b}$ \& $\frac{c}{d}$ với mẫu dương, trong đó $\frac{a}{b} < \frac{c}{d}$. Chứng minh:
	\begin{enumerate*}
		\item[(a)] $ab < bc$;
		\item[(b)] $\frac{a}{b} < \frac{a + c}{b + d} < \frac{c}{d}$.
	\end{enumerate*}
\end{baitoan}

\begin{baitoan}[\cite{Binh_Toan_7_tap_1}, \textbf{4.}, p. 5]
	Tính:
	\begin{enumerate*}
		\item[(a)] $\frac{-2}{3} + \frac{3}{4} - \frac{-1}{6} + \frac{-2}{5}$;
		\item[(b)] $\frac{-2}{3} + \frac{-1}{5} + \frac{3}{4} - \frac{5}{6} - \frac{-7}{10}$;
		\item[(c)] $\frac{1}{2} - \frac{-2}{5} + \frac{1}{3} + \frac{5}{7} - \frac{-1}{6} + \frac{-4}{35} + \frac{1}{41}$;
		\item[(d)] $\frac{1}{100\cdot 99} - \frac{1}{99\cdot 98} - \frac{1}{98\cdot 97} - \cdots - \frac{1}{3\cdot 2} - \frac{1}{2\cdot 1}$.
	\end{enumerate*}
\end{baitoan}

\begin{baitoan}[\cite{Binh_Toan_7_tap_1}, \textbf{5.}, pp. 5--6]
	Ký hiệu $\lfloor x\rfloor$ là số nguyên lớn nhất không vượt quá $x$, được gọi là \emph{phần nguyên} của $x$, e.g., $\lfloor 1.5\rfloor = 1$, $\lfloor 5\rfloor = 5$, $\lfloor -2.5\rfloor = -3$.
	\begin{enumerate*}
		\item[(a)] Tính $\lfloor-\frac{1}{7}\rfloor,\lfloor 3.7\rfloor,\lfloor-4\rfloor,\lfloor-\frac{43}{10}\rfloor$.
		\item[(b)] Cho $x = 3.7$. So sánh: $A = \lfloor x\rfloor + \lfloor x + \frac{1}{5}\rfloor + \lfloor x + \frac{2}{5}\rfloor + \lfloor x + \frac{3}{5}\rfloor$ $+ \lfloor x + \frac{4}{5}\rfloor$ \& $B = \lfloor 5x\rfloor$.
		\item[(c)] Tính $ \lfloor\frac{100}{3}\rfloor + \lfloor\frac{100}{3^2}\rfloor + \lfloor\frac{100}{3^3}\rfloor + \lfloor\frac{100}{3^4}\rfloor$.
		\item[(d)] Tính $ \lfloor\frac{50}{2}\rfloor + \lfloor\frac{50}{2^2}\rfloor + \lfloor\frac{50}{2^3}\rfloor + \lfloor\frac{50}{2^4}\rfloor + \lfloor\frac{50}{2^5}\rfloor$.
		\item[(e)] Cho $x\in\mathbb{Q}$. So sánh $\lfloor x\rfloor$ với $x$, so sánh $\lfloor x\rfloor$ với $y$ trong đó $y\in\mathbb{Z}$, $y < x$.
	\end{enumerate*}
\end{baitoan}

\begin{baitoan}[\cite{Binh_Toan_7_tap_1}, \textbf{6.}, p. 6]
	Cho các số hữu tỷ $x$ bằng $1.4089, 0.1398, -0.4771, -1.2592$.
	\begin{enumerate*}
		\item[(a)] Viết các số đó dưới dạng tổng của 1 số nguyên $a$ \& 1 số thập phân $b$ không âm nhỏ hơn $1$.
		\item[(b)] Tính tổng các số hữu tỷ trên bằng 2 cách: tính theo cách thông thường, tính tổng các số được viết dưới dạng ở (a).
		\item[(c)] So sánh $a$ \& $\lfloor x\rfloor$ trong trường hợp ở câu (a). Lưu ý: Trong cách viết này, $a$ là \emph{phần nguyên} của $x$, còn $b$ là \emph{phần lẻ} của $x$. Ký hiệu phần lẻ của $x$ là $\{x\}$ thì $x = \lfloor x\rfloor + \{x\}$.
	\end{enumerate*}
\end{baitoan}

\begin{baitoan}[\cite{Binh_Toan_7_tap_1}, \textbf{7.}, p. 6]
	Tìm $n\in\mathbb{Z}$ để phân số sau có giá trị là 1 số nguyên \& tính giá trị đó:
	\begin{enumerate*}
		\item[(a)] $A = \frac{3n + 9}{n - 4}$;
		\item[(b)] $B = \frac{6n + 5}{2n - 1}$.
	\end{enumerate*}
\end{baitoan}

\begin{baitoan}[\cite{Binh_Toan_7_tap_1}, \textbf{8.}, p. 6]
	Tìm $x,y\in\mathbb{Z}$, biết: $\frac{5}{x} + \frac{y}{4} = \frac{1}{8}$.
\end{baitoan}

\begin{baitoan}[\cite{Binh_Toan_7_tap_1}, \textbf{9.}, p. 6]
	Viết tất cả các số nguyên có giá trị tuyệt đối nhỏ hơn $20$ theo thứ tự tùy ý. Lấy mỗi số trừ đi số thứ tự của nó ta được 1 hiệu. Tổng của tất cả các hiệu đó bằng bao nhiêu?
\end{baitoan}

\begin{baitoan}[\cite{Binh_Toan_7_tap_1},  \textbf{10.}, p. 6]
	Tính:
	\begin{enumerate*}
		\item[(a)] $\dfrac{\left(\frac{3}{10} - \frac{4}{15} - \frac{7}{20}\right)\cdot\frac{5}{19}}{\left(\frac{1}{14} + \frac{1}{7} - \frac{-3}{35}\right)\cdot\frac{-4}{3}}$;
		\item[(b)] $\dfrac{(1 + 2 + \cdots + 100)\left(\frac{1}{3} - \frac{1}{5} - \frac{1}{7} - \frac{1}{9}\right)\cdot(6.3\cdot 12 - 21\cdot 3.6)}{\frac{1}{2} + \frac{1}{3} + \cdots + \frac{1}{100}}$;
		\item[(c)] $\dfrac{\frac{1}{9} - \frac{1}{7} - \frac{1}{11}}{\frac{4}{9} - \frac{4}{7} - \frac{4}{11}} + \dfrac{\frac{3}{5} - \frac{3}{25} - \frac{3}{125} - \frac{3}{625}}{\frac{4}{5} - \frac{4}{25} - \frac{4}{125} - \frac{4}{625}}$.
	\end{enumerate*}
\end{baitoan}

\begin{baitoan}[\cite{Binh_Toan_7_tap_1}, \textbf{11.}, p. 7]
	Tìm $x\in\mathbb{Q}$, biết:
	\begin{enumerate*}
		\item[(a)] $\frac{2}{3}x - 4 = -12$;
		\item[(b)] $\frac{3}{4} + \frac{1}{4}:x = -3$;
		\item[(c)] $|3x - 5| = 4$;
		\item[(d)] $\frac{x + 1}{10} + \frac{x + 1}{11} + \frac{x + 1}{12} = \frac{x + 1}{13} + \frac{x + 1}{14}$;
		\item[(e)] $\frac{x + 4}{2000} + \frac{x + 3}{2001} = \frac{x + 2}{2002} + \frac{x + 1}{2003}$.
	\end{enumerate*}
\end{baitoan}

\begin{baitoan}[\cite{Binh_Toan_7_tap_1}, \textbf{12.}, p. 7]
	Cho phân số $\frac{a}{b}$ với $a,b\in\mathbb{N}^\star$. Tìm phân số $x$ sao cho $\frac{a}{b} - x = \frac{a}{b}\cdot x$.
\end{baitoan}

\begin{baitoan}[\cite{Binh_Toan_7_tap_1}, \textbf{13.}, p. 7]
	Trung bình cộng của 2 số lớn hơn số thứ nhất $75$\% thì nhỏ hơn số thứ 2 bao nhiêu \%?
\end{baitoan}

\begin{baitoan}[\cite{Binh_Toan_7_tap_1}, \textbf{14.}, p. 7]
	Chứng minh:\\	
	\begin{enumerate*}
		\item[(a)] $\sum_{i=1}^{99} \frac{i}{(i+1)!} = \frac{1}{2!} + \frac{2}{3!} + \frac{3}{4!} + \cdots + \frac{99}{100!} < 1$.
		\item[(b)] $\sum_{i=1}^{99} \frac{i(i + 1) - 1}{(i+1)!} = \frac{1\cdot 2 - 1}{2!} + \frac{2\cdot 3 - 1}{3!} + \frac{3\cdot 4 - 1}{4!} + \cdots + \frac{99\cdot 100 - 1}{100!} < 2$.
	\end{enumerate*}
\end{baitoan}

\begin{baitoan}[\cite{Binh_Toan_7_tap_1}, \textbf{15.}, p. 7]
	\begin{enumerate*}
		\item[(a)] Người ta viết $7$ số hữu tỷ trên 1 vòng tròn. Tìm các số đó, biết tích của $2$ số bất kỳ cạnh nhau bằng $16$.
		\item[(b)] Cũng hỏi như trên đối với $n$ số.
	\end{enumerate*}
\end{baitoan}

\begin{baitoan}[\cite{Binh_Toan_7_tap_1}, \textbf{16.}, p. 7]
	Có tồn tại hay không $2$ số dương $a,b$ khác nhau sao cho $\frac{1}{a} - \frac{1}{b} = \frac{1}{a - b}$?
\end{baitoan}

\begin{baitoan}[\cite{Binh_Toan_7_tap_1}, \textbf{17.}${}^\star$, p. 7]
	\begin{enumerate*}
		\item[(a)] Chứng minh: $\frac{1}{1\cdot 2} + \frac{1}{3\cdot 4} + \frac{1}{5\cdot 6} + \cdots + \frac{1}{49\cdot 50} = \frac{1}{26} + \frac{1}{27} + \frac{1}{28} + \cdots + \frac{1}{50}$.
		\item[(b)] Cho $B = \frac{1}{1\cdot 2} + \frac{1}{3\cdot 4} + \frac{1}{5\cdot 6} + \cdots + \frac{1}{99\cdot 100}$. Chứng minh $\frac{7}{12} < B < \frac{5}{6}$.
	\end{enumerate*}	
\end{baitoan}

\begin{baitoan}[\cite{Binh_Toan_7_tap_1}, \textbf{18.}, p. 7]
	Tìm $a,b\in\mathbb{Q}$ sao cho:
	\begin{enumerate*}
		\item[(a)] $a - b = 2(a + b) = a:b$.
		\item[(b)] $a + b = ab = a:b$.
	\end{enumerate*}
\end{baitoan}

\begin{baitoan}[\cite{Binh_Toan_7_tap_1}, \textbf{19.}${}^\star$, p. 7]
	Tìm $x\in\mathbb{Q}$, sao cho tổng của số đó với số nghịch đảo của nó là 1 số nguyên.
\end{baitoan}

\begin{baitoan}[\cite{Binh_Toan_7_tap_1}, \textbf{20.}${}^\star$, p. 8]
	Viết tất cả các số hữu tỷ dương  thành dãy gồm các nhóm phân số có tổng của tử \& mẫu lần lượt bằng $2,3,4,5,\ldots$, các phân số trong cùng 1 nhóm được đặt trong dấu ngoặc: $\left(\frac{1}{1}\right),\left(\frac{2}{1},\frac{1}{2}\right),\left(\frac{3}{1},\frac{2}{2},\frac{1}{3}\right),\left(\frac{4}{1},\frac{3}{2},\frac{2}{3},\frac{1}{4}\right),\ldots$. Tìm phân số thứ $200$ của dãy.
\end{baitoan}

%------------------------------------------------------------------------------%

\subsection{Lũy Thừa của 1 Số Hữu Tỷ}

\begin{baitoan}[\cite{Binh_Toan_7_tap_1}, Ví dụ 5, p. 8]
	\begin{enumerate*}
		\item[(a)] Chứng minh: $2^{10}\approx 10^3$ \& $9^{10}\approx 80^5$.
		\item[(b)] Dùng nhận xét ở (a) để chứng minh $9^{10}\approx 3.2\cdot 10^9$.
	\end{enumerate*}
\end{baitoan}

\begin{baitoan}[\cite{Binh_Toan_7_tap_1}, Ví dụ 6, p. 8]
	Tính: $A = \sum_{i=1}^{10} \frac{i}{2^i} = \frac{1}{2} + \frac{2}{2^2} + \frac{3}{2^3} + \cdots + \frac{10}{2^{10}}$.
\end{baitoan}

\begin{baitoan}[\cite{Binh_Toan_7_tap_1}, Ví dụ 7, p. 9]
	\begin{enumerate*}
		\item[(a)] Có thể khẳng định $x^2$ luôn luôn lớn hơn $x$ hay không?
		\item[(b)] Khi nào thì $x^2 < x$?
	\end{enumerate*}
\end{baitoan}

\begin{baitoan}[\cite{Binh_Toan_7_tap_1}, Ví dụ 8, p. 9]
	Tìm $a,b,c\in\mathbb{Q}$, biết: $ab = 2$, $bc = 3$, $ca = 54$.
\end{baitoan}

\begin{baitoan}[\cite{Binh_Toan_7_tap_1}, Ví dụ 9, p. 9]
	Rút gọn: $A = \sum_{i=0}^{50} 5^i = 1 + 5 + 5^2 + \cdots + 5^{49} + 5^{50}$.
\end{baitoan}

\begin{baitoan}[\cite{Binh_Toan_7_tap_1}, Ví dụ 10, p. 9]
	Cho $B = \sum_{i=1}^{99} \left(\frac{1}{2}\right)^i = \frac{1}{2} + \left(\frac{1}{2}\right)^2 + \cdots + \left(\frac{1}{2}\right)^{98} + \left(\frac{1}{2}\right)^{99}$. Chứng minh $B < 1$.
\end{baitoan}

\begin{baitoan}[\cite{Binh_Toan_7_tap_1}, \textbf{21.}, p. 10]
	Chứng minh:
	\begin{enumerate*}
		\item[(a)] $7^6 + 7^5 - 7^4\divby 55$;
		\item[(b)] $16^5 + 2^{15}\divby 33$;
		\item[(c)] $81^7 - 27^9 - 9^{13}\divby 405$.
	\end{enumerate*}
\end{baitoan}

\begin{baitoan}[\cite{Binh_Toan_7_tap_1}, \textbf{22.}, p. 10]
	Điền vào chỗ chấm ($\cdots$) các từ ``bằng nhau'' hoặc ``đối nhau'' cho đúng:
	\begin{enumerate*}
		\item[(a)] Nếu 2 số đối nhau thì bình phương của chúng $\ldots$.
		\item[(b)] Nếu 2 số đối nhau thì lập phương của chúng $\ldots$.
		\item[(c)] Lũy thừa chẵn cùng bậc của 2 số đối nhau thì $\ldots$.
		\item[(d)] Lũy thừa lẻ cùng bậc của 2 số đối nhau thì $\ldots$.
	\end{enumerate*}
\end{baitoan}

\begin{baitoan}[\cite{Binh_Toan_7_tap_1}, \textbf{23.}, p. 10 \& mở rộng]
	Các đẳng thức sau có đúng với mọi $a,b\in\mathbb{Q}$ hay không?
	\begin{enumerate*}
		\item[(a)] $-a^3 = (-a)^3$;
		\item[(b)] $-a^5 = (-a)^5$;
		\item[(c)] $-a^2 = (-a)^2$;
		\item[(d)] $-a^4 = (-a)^4$;
		\item[(e)] $-a^{2n+1} = (-a)^{2n+1}$, $\forall n\in\mathbb{N}$;
		\item[(f)] $a^{2n} = (-a)^{2n}$, $\forall n\in\mathbb{N}$;
		\item[(g)] $(a - b)^2 = (b - a)^2$;
		\item[(h)] $(a - b)^3 = -(b - a)^3$;
		\item[(i)] $(a - b)^{2n} = (b - a)^{2n}$, $\forall n\in\mathbb{N}$;
		\item[(j)] $(a - b)^{2n+1} = -(b - a)^{2n+1}$, $\forall n\in\mathbb{N}$.
	\end{enumerate*}
\end{baitoan}

\begin{baitoan}[\cite{Binh_Toan_7_tap_1}, \textbf{24.}, p. 10]
	Tính:
	\begin{enumerate*}
		\item[(a)] $\left(\frac{1}{2}\right)^{15}\cdot\left(\frac{1}{4}\right)^{20}$;
		\item[(b)] $\left(\frac{1}{9}\right)^{25}:\left(\frac{1}{3}\right)^{30}$;
		\item[(c)] $\left(\frac{1}{16}\right)^3:\left(\frac{1}{8}\right)^2$;
		\item[(d)] $(x^3)^2:(x^2)^3$ với $x\ne 0$.
	\end{enumerate*}
\end{baitoan}

\begin{baitoan}[\cite{Binh_Toan_7_tap_1}, \textbf{25.}, p. 10]
	Viết số $64$ dưới dạng $a^n$ với $a\in\mathbb{Z}$. Có bao nhiêu cách viết?
\end{baitoan}

\begin{baitoan}[\cite{Binh_Toan_7_tap_1}, \textbf{26.}, p. 10]
	Rút gọn biểu thức: $A = \dfrac{4^5\cdot 9^4 - 2\cdot 6^9}{2^{10}\cdot 3^8 + 6^8\cdot 20}$.
\end{baitoan}

\begin{baitoan}[\cite{Binh_Toan_7_tap_1}, \textbf{27.}, p. 10]
	\begin{enumerate*}
		\item[(a)] Chứng minh: $2^{10}\approx 10^3$ \& $3^{16}\approx 80^4$.
		\item[(b)] Dùng nhận xét ở (a) để chứng minh $3^{16}\approx 40000000$.
	\end{enumerate*}
\end{baitoan}

\begin{baitoan}[\cite{Binh_Toan_7_tap_1}, \textbf{28.}, p. 10]
	Cho $S_n = \sum_{i=1}^{n-1} (-1)^{i-1}i = 1 - 2 + 3 - 4 + \cdots + (-1)^{n-1}n$ với $n\in\mathbb{N}^\star$. Tính $S_{35} + S_{60}$.
\end{baitoan}

\begin{baitoan}[\cite{Binh_Toan_7_tap_1}, \textbf{29.}, p. 10]
	Cho $A = 1 - 5 + 9 - 13 + 17 - 21 + 25 - \cdots$ ($n$ số hạng, giá trị tuyệt đối của số sau lớn hơn giá trị tuyệt đối của số hạng trước $4$ đơn vị, các dấu $+$ \& $-$ xen kẽ).
	\begin{enumerate*}
		\item[(a)] Tính $A$ theo $n$.
		\item[(b)] Viết số hạng thứ $n$ của biểu thức $A$ theo $n$ (chú ý dùng lũy thừa để biểu thị dấu của số hạng đó).
	\end{enumerate*}
\end{baitoan}

\begin{baitoan}[\cite{Binh_Toan_7_tap_1}, \textbf{30.}, p. 11]
	Với giá trị nào của các chữ thì các biểu thức sau có giá trị là số $0$, số dương, số âm?
	\begin{enumerate*}
		\item[(a)] $P = \frac{a^2b}{c}$;
		\item[(b)] $Q = \frac{x^3}{yz}$.
	\end{enumerate*}
\end{baitoan}

\begin{baitoan}[\cite{Binh_Toan_7_tap_1}, \textbf{31.}, p. 11]
	Cho $2$ số hữu tỷ $a$ \& $b$ trái dấu trong đó $|a| = b^5$. Xác định dấu của mỗi số.
\end{baitoan}

\begin{baitoan}[\cite{Binh_Toan_7_tap_1}, \textbf{32.}, p. 11]
	Viết các số sau dưới dạng lũy thừa của $2$: $16,64,1,\frac{1}{32},\frac{1}{8},0.5,0.25$.
\end{baitoan}

\begin{baitoan}[\cite{Binh_Toan_7_tap_1}, \textbf{33.}, p. 11]
	\begin{enumerate*}
		\item[(a)] Viết các số sau thành lũy thừa với số mũ âm: $\frac{1}{1000000},0.00000002$.
		\item[(b)] Viết các số sau dưới dạng số thập phân: $10^{-7}$, $2.5\cdot 10^{-6}$.
	\end{enumerate*}
\end{baitoan}

\begin{baitoan}[\cite{Binh_Toan_7_tap_1}, \textbf{34.}, p. 11]
	Tính xem $A$ gấp mấy lần $B$:
	\begin{enumerate*}
		\item[(a)] $A = 3.4\cdot 10^{-8}$, $B = 34\cdot 10^{-9}$;
		\item[(b)] $A = 10^{-4} + 10^{-3} + 10^{-2}$, $B = 10^{-9}$.
	\end{enumerate*}
\end{baitoan}

\begin{baitoan}[\cite{Binh_Toan_7_tap_1}, \textbf{35.}, p. 11]
	So sánh:
	\begin{enumerate*}
		\item[(a)] $\left(-\frac{1}{16}\right)^{100}$ \& $\left(-\frac{1}{2}\right)^{500}$;
		\item[(b)] $(-32)^9$ \& $(-18)^{13}$;
		\item[(c)] $a = 2^{100}$, $b = 3^{75}$, $c = 5^{50}$.
	\end{enumerate*}
\end{baitoan}

\begin{baitoan}[\cite{Binh_Toan_7_tap_1}, \textbf{36.}, p. 11]
	Trong các câu sau, câu nào đúng với mọi $a\in\mathbb{Q}$?
	\begin{enumerate*}
		\item[(a)] Nếu $a < 0$ thì $a^2 > 0$;
		\item[(b)] Nếu $a^2 > 0$ thì $a > 0$;
		\item[(c)] Nếu $a < 0$ thì $a^2 > a$;
		\item[(d)] Nếu $a^2 > a$ thì $a > 0$;
		\item[(e)] Nếu $a^2 > a$ thì $a < 0$.
	\end{enumerate*}
\end{baitoan}

\begin{baitoan}[\cite{Binh_Toan_7_tap_1}, \textbf{37.}, p. 11]
	\begin{enumerate*}
		\item[(a)] Cho $a^m = a^n$ ($a\in\mathbb{Q}$, $m,n\in\mathbb{N}$). Tìm $m,n$.
		\item[(b)] Cho $a^m > a^n$ ($a\in\mathbb{Q}$, $a > 0$, $m,n\in\mathbb{N}$). So sánh $m$ \& $n$.
	\end{enumerate*}
\end{baitoan}

\begin{baitoan}[\cite{Binh_Toan_7_tap_1}, \textbf{38.}, p. 11]
	Tìm $x\in\mathbb{Q}$, biết:
	\begin{enumerate*}
		\item[(a)] $(2x - 1)^4 = 81$;
		\item[(b)] $(x - 1)^5 = -32$;
		\item[(c)] $(2x - 1)^6 = (2x - 1)^8$.
	\end{enumerate*}
\end{baitoan}

\begin{baitoan}[\cite{Binh_Toan_7_tap_1}, \textbf{39.}, p. 11]
	Tìm $x\in\mathbb{N}$, biết:
	\begin{enumerate*}
		\item[(a)] $5^x + 5^{x+2} = 650$;
		\item[(b)] $3^{x-1} + 5\cdot 3^{x-1} = 162$.
	\end{enumerate*}
\end{baitoan}

\begin{baitoan}[\cite{Binh_Toan_7_tap_1}, \textbf{40.}, p. 11]
	Tìm $x,y\in\mathbb{N}$, biết:
	\begin{enumerate*}
		\item[(a)] $2^{x+1}\cdot 3^y = 12^x$;
		\item[(b)] $10^x:5^y = 20^y$;
		\item[(c)] $2^x = 4^{y-1}$ \& $27^y = 3^{x+8}$.
	\end{enumerate*}
\end{baitoan}

\begin{baitoan}[\cite{Binh_Toan_7_tap_1}, \textbf{41.}, p. 11]
	Tìm $a,b,c\in\mathbb{Q}$, biết:
	\begin{enumerate*}
		\item[(a)] $ab = \frac{3}{5}$, $bc = \frac{4}{5}$, $ca = \frac{3}{4}$.
		\item[(b)] $a(a + b + c) = -12$, $b(a + b + c) = 18$, $c(a + b + c) = 30$;
		\item[(c)] $ab = c$, $bc = 4a$, $ac = 9b$.
	\end{enumerate*}
\end{baitoan}

\begin{baitoan}[\cite{Binh_Toan_7_tap_1}, \textbf{42.}${}^\star$, p. 12]
	Cho $a,b,c,d,e\in\mathbb{N}$ thỏa mãn $a^b = b^c = c^d = d^e = e^a$. Chứng minh $a = b = c = d = e$.
\end{baitoan}

\begin{baitoan}[\cite{Binh_Toan_7_tap_1}, \textbf{43.}, p. 12]
	Cho $A = \prod_{i=2}^{100} \left(\frac{1}{i^2} - 1\right) = \left(\frac{1}{2^2} - 1\right)\left(\frac{1}{3^2} - 1\right)\left(\frac{1}{4^2} - 1\right)\cdots\left(\frac{1}{100^2} - 1\right)$. So sánh $A$ với $-\frac{1}{2}$.
\end{baitoan}

\begin{baitoan}[\cite{Binh_Toan_7_tap_1}, \textbf{44.}, p. 12]
	Rút gọn $A = \sum_{i=1}^{100} (-1)^i2^i = 2^{100} - 2^{99} + 2^{98} - 2^{97} + \cdots + 2^2 - 2$.
\end{baitoan}

\begin{baitoan}[\cite{Binh_Toan_7_tap_1}, \textbf{45.}, p. 12]
	Rút gọn $B = \sum_{i=
	}^{100} (-1)^i3^i = 3^{100} - 3^{99} + 3^{98} - 3^{97} + \cdots + 3^2 - 3 + 1$.
\end{baitoan}

\begin{baitoan}[\cite{Binh_Toan_7_tap_1}, \textbf{46.}, p. 12]
	Cho $C = \sum_{i=1}^{99} \frac{1}{3^i} = \frac{1}{3} + \frac{1}{3^2} + \cdots + \frac{1}{3^{99}}$. Chứng minh $C < \frac{1}{2}$.
\end{baitoan}

\begin{baitoan}[\cite{Binh_Toan_7_tap_1}, \textbf{47.}, p. 12]
	Chứng minh $\frac{3}{1^2\cdot 2^2} + \frac{5}{2^2\cdot 3^2} + \frac{7}{3^2\cdot 4^2} + \cdots + \frac{19}{9^2\cdot 10^2} < 1$.
\end{baitoan}

\begin{baitoan}[\cite{Binh_Toan_7_tap_1}, \textbf{48.}${}^\star$, p. 12]
	Chứng minh $\sum_{i=1}^{100} \frac{i}{3^i} = \frac{1}{3} + \frac{2}{3^2} + \frac{3}{3^3} + \cdots + \frac{100}{3^{100}} < \frac{3}{4}$.
\end{baitoan}

\begin{baitoan}[\cite{Binh_Toan_7_tap_1}, \textbf{49.}, p. 12]
	Ta không có $2^m + 2^n = 2^{m+n}$, $\forall m,n\in\mathbb{N}^\star$. Nhưng có những số nguyên dương $m,n$ có tính chất trên. Tìm các số đó.
\end{baitoan}

\begin{baitoan}[\cite{Binh_Toan_7_tap_1}, \textbf{50.}${}^\star$, p. 12]
	Tìm $m,n\in\mathbb{N}^\star$ sao cho $2^m - 2^n = 256$.
\end{baitoan}

\begin{baitoan}[\cite{Binh_Toan_7_tap_1}, \textbf{51.}${}^\star$, p. 12]
	Cho 1 bảng vuông $3\times 3$ ô. Trong mỗi ô của bảng viết số $1$ hoặc số $-1$. Gọi $d_i$ là tích các số trên dòng $i$ ($i = 1,2,3$), $c_k$ là tích các số trên cột $k$ ($k = 1,2,3$).
	\begin{enumerate*}
		\item[(a)] Chứng minh không thể xảy ra $d_1 + d_2 + d_3 + c_1 + c_2 + c_3 = 0$.
		\item[(b)] Xét bài toán trên đối với bảng vuông $n\times n$.
	\end{enumerate*}
\end{baitoan}

\begin{baitoan}[\cite{Binh_Toan_7_tap_1}, \textbf{52.}${}^\star$, p. 12]
	Cho $n$ số $x_1,\ldots,x_n$, mỗi số bằng $1$ hoặc $-1$. Biết tổng của $n$ tích $x_1x_2$, $x_2x_3$, $x_3x_4,\ldots,x_nx_1$ bằng $0$. Chứng minh $n\ \vdots\ 4$.
\end{baitoan}

%------------------------------------------------------------------------------%

\printbibliography[heading=bibintoc]
	
\end{document}