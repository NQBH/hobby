\documentclass{article}
\usepackage[backend=biber,natbib=true,style=authoryear]{biblatex}
\addbibresource{/home/nqbh/reference/bib.bib}
\usepackage[utf8]{vietnam}
\usepackage{tocloft}
\renewcommand{\cftsecleader}{\cftdotfill{\cftdotsep}}
\usepackage[colorlinks=true,linkcolor=blue,urlcolor=red,citecolor=magenta]{hyperref}
\usepackage{amsmath,amssymb,amsthm,mathtools,float,graphicx,algpseudocode,algorithm,tcolorbox,tikz,tkz-tab,subcaption}
\DeclareMathOperator{\arccot}{arccot}
\usepackage[inline]{enumitem}
\allowdisplaybreaks
\numberwithin{equation}{section}
\newtheorem{assumption}{Assumption}[section]
\newtheorem{nhanxet}{Nhận xét}[section]
\newtheorem{conjecture}{Conjecture}[section]
\newtheorem{corollary}{Corollary}[section]
\newtheorem{hequa}{Hệ quả}[section]
\newtheorem{definition}{Definition}[section]
\newtheorem{dinhnghia}{Định nghĩa}[section]
\newtheorem{example}{Example}[section]
\newtheorem{vidu}{Ví dụ}[section]
\newtheorem{lemma}{Lemma}[section]
\newtheorem{notation}{Notation}[section]
\newtheorem{principle}{Principle}[section]
\newtheorem{problem}{Problem}[section]
\newtheorem{baitoan}{Bài toán}[section]
\newtheorem{proposition}{Proposition}[section]
\newtheorem{menhde}{Mệnh đề}[section]
\newtheorem{question}{Question}[section]
\newtheorem{cauhoi}{Câu hỏi}[section]
\newtheorem{quytac}{Quy tắc}
\newtheorem{remark}{Remark}[section]
\newtheorem{luuy}{Lưu ý}[section]
\newtheorem{theorem}{Theorem}[section]
\newtheorem{tiende}{Tiên đề}[section]
\newtheorem{dinhly}{Định lý}[section]
\usepackage[left=0.5in,right=0.5in,top=1.5cm,bottom=1.5cm]{geometry}
\usepackage{fancyhdr}
\pagestyle{fancy}
\fancyhf{}
\lhead{\small Sect.~\thesection}
\rhead{\small\nouppercase{\leftmark}}
\renewcommand{\subsectionmark}[1]{\markboth{#1}{}}
\cfoot{\thepage}
\def\labelitemii{$\circ$}
\DeclareRobustCommand{\divby}{%
	\mathrel{\vbox{\baselineskip.65ex\lineskiplimit0pt\hbox{.}\hbox{.}\hbox{.}}}%
}

\title{Rational -- Số Hữu Tỷ $\mathbb{Q}$}
\author{Nguyễn Quản Bá Hồng\footnote{Independent Researcher, Ben Tre City, Vietnam\\e-mail: \texttt{nguyenquanbahong@gmail.com}; website: \url{https://nqbh.github.io}.}}
\date{\today}

\begin{document}
\maketitle
\begin{abstract}
	\textsc{[en]} This text is a collection of problems, from easy to advanced, about rational. This text is also a supplementary material for my lecture note on Elementary Mathematics grade 7, which is stored \& downloadable at the following link: \href{https://github.com/NQBH/hobby/blob/master/elementary_mathematics/grade_7/NQBH_elementary_mathematics_grade_7.pdf}{GitHub\texttt{/}NQBH\texttt{/}hobby\texttt{/}elementary mathematics\texttt{/}grade 7\texttt{/}lecture}\footnote{\textsc{url}: \url{https://github.com/NQBH/hobby/blob/master/elementary_mathematics/grade_7/NQBH_elementary_mathematics_grade_7.pdf}.}. The latest version of this text has been stored \& downloadable at the following link: \href{https://github.com/NQBH/hobby/blob/master/elementary_mathematics/grade_7/rational/NQBH_rational.pdf}{GitHub\texttt{/}NQBH\texttt{/}hobby\texttt{/}elementary mathematics\texttt{/}grade 7\texttt{/}rational $\mathbb{Q}$}\footnote{\textsc{url}: \url{https://github.com/NQBH/hobby/blob/master/elementary_mathematics/grade_7/rational/NQBH_rational.pdf}.}.
	\vspace{2mm}
	
	\textsc{[vi]} Tài liệu này là 1 bộ sưu tập các bài tập chọn lọc từ cơ bản đến nâng cao về phân thức đại số \& phân thức đại số hữu tỷ. Tài liệu này là phần bài tập bổ sung cho tài liệu chính -- bài giảng \href{https://github.com/NQBH/hobby/blob/master/elementary_mathematics/grade_7/NQBH_elementary_mathematics_grade_7.pdf}{GitHub\texttt{/}NQBH\texttt{/}hobby\texttt{/}elementary mathematics\texttt{/}grade 7\texttt{/}lecture} của tác giả viết cho Toán Sơ Cấp lớp 8. Phiên bản mới nhất của tài liệu này được lưu trữ \& có thể tải xuống ở link sau: \href{https://github.com/NQBH/hobby/blob/master/elementary_mathematics/grade_7/rational/NQBH_rational.pdf}{GitHub\texttt{/}NQBH\texttt{/}hobby\texttt{/}elementary mathematics\texttt{/}grade 7\texttt{/}rational $\mathbb{Q}$}.
\end{abstract}
\setcounter{secnumdepth}{4}
\setcounter{tocdepth}{3}
\tableofcontents

%------------------------------------------------------------------------------%

\section{Problem}

\subsection{Tập Hợp $\mathbb{Q}$ Các Số Hữu Tỷ}
``\begin{enumerate*}
	\item[\textbf{1.}] \textit{Số hữu tỷ} là số viết được dưới dạng $\frac{a}{b}$ với $a,b\in\mathbb{Z}$, $b\ne 0$. \textit{Tập hợp các số hữu tỷ} được ký hiệu là $\mathbb{Q}$.
	\item[\textbf{2.}] \textit{Biểu diễn số hữu tỷ trên trục số}: Trên trục số, điểm biểu diễn số hữu tỷ $x$ gọi là \textit{điểm $x$}. Các số khác nhau biểu diễn bởi những điểm khác nhau.
	\item[\textbf{3.}] \textit{Số đối} của 1 số hữu tỷ: Trên trục số, 2 số hữu tỷ có điểm biểu diễn nằm về 2 phía của gốc $O$ \& cách đều gốc $O$ được gọi là \textit{2 số đối nhau}. \textit{Số đối} của số hữu tỷ $x$ ký hiệu là $-x$. Số đối của $0$ là $0$.
	\item[\textbf{4.}] \textit{So sánh các số hữu tỷ}: Để so sánh $x,y\in\mathbb{Q}$ ta làm như sau: Viết $x,y$ dưới dạng 2 phân số có cùng mẫu dương $x = \frac{a}{m}$, $y = \frac{b}{m}$, $m > 0$. Sau đó so sánh các tử số:
	\begin{enumerate*}
		\item[$\bullet$] Nếu $a < b$ thì $x < y$.
		\item[$\bullet$] Nếu $a = b$ thì $x = y$.
		\item[$\bullet$] Nếu $a > b$ thì $x > y$.
	\end{enumerate*}
	Số hữu tỷ lớn hơn $0$ gọi là \textit{số hữu tỷ dương}. Số hữu tỷ nhỏ hơn $0$ gọi là \textit{số hữu tỷ âm}. Số $0$ không là số hữu tỷ dương, cũng không là số hữu tỷ âm.
	\item[\textbf{5.}] Cho các số hữu tỷ $x = \frac{a}{b}$ \& $y = \frac{c}{d}$, $a,b,c,d\in\mathbb{Z}$, $b > 0$, $d > 0$. Luôn có:
	\begin{enumerate*}
		\item[$\bullet$] $x = y\Leftrightarrow ad = bc$, $x < y\Leftrightarrow ad < bc$, $x > y\Leftrightarrow ad > bc$.
		\item[$\bullet$] $\frac{a}{b} < \frac{a + c}{b + d} < \frac{c}{d}$.
	\end{enumerate*}'' -- \cite[\S1, pp. 4--5]{Tuyen_Toan_7}
\end{enumerate*}

\begin{baitoan}[\cite{Tuyen_Toan_7}, Ví dụ 1, p. 5]
	Cho $x = \frac{12}{b - 15}$ với $b\in\mathbb{Z}$. Xác định $b$ để:
	\begin{enumerate*}
		\item[(a)] $x$ là 1 số hữu tỷ;
		\item[(b)] $x$ là 1 số hữu tỷ dương;
		\item[(c)] $x$ là 1 số hữu tỷ âm;
		\item[(d)] $0 < x < 1$.
	\end{enumerate*}
\end{baitoan}

\begin{baitoan}[\cite{Tuyen_Toan_7}, Ví dụ 2, p. 5]
	So sánh các số hữu tỷ sau: $\frac{-16}{27}$, $\frac{-16}{29}$, $\frac{-19}{27}$.
\end{baitoan}

\begin{baitoan}[\cite{Tuyen_Toan_7}, \textbf{1.}, p. 5]
	Cho các số hữu tỷ $x = \frac{-5}{7}$, $y = \frac{-2}{3}$. Các số hữu tỷ này còn được biểu diễn bởi phân số nào trong các phân số sau: $\frac{9}{11}$, $\frac{4}{-6}$, $\frac{15}{-21}$, $\frac{-35}{49}$, $\frac{-10}{15}$, $\frac{-6}{-9}$.
\end{baitoan}

\begin{baitoan}[\cite{Tuyen_Toan_7}, \textbf{2.}, p. 6]
	Sắp xếp các số hữu tỷ sau theo thứ tự tăng dần:
	\begin{enumerate*}
		\item[(a)] $\frac{19}{33}$, $\frac{6}{11}$, $\frac{13}{22}$;
		\item[(b)] $\frac{-18}{12}$, $\frac{-10}{7}$; $\frac{-8}{5}$.
	\end{enumerate*}
\end{baitoan}

\begin{baitoan}[\cite{Tuyen_Toan_7}, \textbf{3.}, p. 6]
	So sánh các số hữu tỷ sau bằng cách nhanh nhất:
	\begin{enumerate*}
		\item[(a)] $-5$ \& $\frac{1}{63}$;
		\item[(b)] $\frac{-18}{17}$ \& $\frac{-999}{1000}$;
		\item[(c)] $\frac{-17}{35}$ \& $\frac{-43}{85}$;
		\item[(d)] $-0.76$ \& $\frac{-19}{28}$.
	\end{enumerate*}
\end{baitoan}

\begin{baitoan}[\cite{Tuyen_Toan_7}, \textbf{4.}, p. 6]
	Tìm các số hữu tỷ biểu diễn dưới dạng phân số có mẫu số bằng $10$, lớn hơn $\frac{-7}{13}$ nhưng nhỏ hơn $-\frac{-4}{13}$.
\end{baitoan}

\begin{baitoan}[\cite{Tuyen_Toan_7}, \textbf{5.}, p. 6]
	Dùng $4$ chữ số $1$ \& dấu $-$ (nếu cần thiết) để biểu diễn (không dùng phép tính lũy thừa):
	\begin{enumerate*}
		\item[(a)] Các số nguyên $-1$, $-111$;
		\item[(b)] Số hữu tỷ âm lớn nhất.
	\end{enumerate*}
\end{baitoan}

\begin{baitoan}[\cite{Tuyen_Toan_7}, \textbf{6.}, p. 6]
	Cho các số nguyên dương $a < b < c < d < m < n$. Chứng minh: $\frac{a + c + m}{ a + b + c + d + m + n} < \frac{1}{2}$.
\end{baitoan}

\begin{baitoan}[\cite{Tuyen_Toan_7}, \textbf{7.}, p. 6]
	Với cùng 1 khối lượng thành phẩm, vàng 4 số 9 \& vàng 3 số 9, loại nào có hàm lượng vàng nhiều hơn?
\end{baitoan}

%------------------------------------------------------------------------------%

\subsection{$\pm,\cdot,:$ Trên $\mathbb{Q}$}
``\begin{enumerate*}
	\item[\textbf{1.}] Ta có thể cộng, trừ, nhân, chia 2 số hữu tỷ bằng cách viết chúng dưới dạng phân số rồi áp dụng quy tắc cộng, trừ, nhân, chia phân số. Nếu 2 số hữu tỷ cùng viết dưới dạng số thập phân thì có thể cộng, trừ, nhân, chia 2 số đó theo quy tắc cộng, trừ, nhân, chia số thập phân.
	\item[\textbf{2.}] Tính chất:
	\begin{enumerate*}
		\item[$\bullet$] Phép cộng các số hữu tỷ cũng có các tính chất giao hoán, kết hợp, cộng với số $0$, cộng với số đối như phép cộng các số nguyên.
		\item[$\bullet$] Phép trừ 2 số hữu tỷ có thể chuyển thành phép cộng với số đối của số trừ. $x - y = x + (-y)$.
		\item[$\bullet$] Phép nhân các số hữu tỷ cũng có các tính chất giao hoán, kết hợp, nhân với $1$, phân phối đối với phép cộng \& phép trừ.
		\item[$\bullet$] Phép chia số hữu tỷ $x$ cho số hữu tỷ $y\ne 0$ có thể chuyển thành phép nhân với số nghịch đảo của số chia $x:y = x\cdot\frac{1}{y}$, $y\ne 0$.
	\end{enumerate*}
	\item[\textbf{3.}] \textit{Quy tắc chuyển vế}: Khi chuyển 1 số hạng từ vế này sang vế kia của 1 đẳng thức, ta phải đổi dấu số hạng đó. $x + y = z\Leftrightarrow x - z = -y$, $x - y = z\Leftrightarrow x - z = y$.
	\item[\textbf{4.}] Trong $\mathbb{Q}$ cũng có những tổng đại số trong đó có thể đổi chỗ các số hạng, đặt dấu ngoặc để nhóm các số hạng 1 cách tùy ý như các tổng đại số trong $\mathbb{Z}$.
	\item[\textbf{5.}] $-(x\cdot y) = (-x)\cdot y = x\cdot(-y)$.'' -- \cite[\S2, pp. 6--7]{Tuyen_Toan_7}
\end{enumerate*}

\begin{baitoan}[\cite{Tuyen_Toan_7}, Ví dụ 3, p. 7]
	Tính bằng cách hợp lý (nếu có thể):
	\begin{enumerate*}
		\item[(a)] $-\frac{5}{18} + \frac{32}{45} - \frac{9}{10}$;
		\item[(b)] $\left(-\frac{1}{4} + \frac{7}{33} - \frac{5}{3}\right) - \left(-\frac{15}{12} + \frac{6}{11} - \frac{48}{49}\right)$.
	\end{enumerate*}
\end{baitoan}

\begin{baitoan}[\cite{Tuyen_Toan_7}, Ví dụ 4, p. 7]
	So sánh các tích sau bằng cách hợp lý nhất:
	\begin{align*}
		P_1 = \left(-\frac{43}{51}\right)\cdot\left(\frac{-19}{80}\right),\ P_2 = \left(-\frac{7}{13}\right)\cdot\left(-\frac{4}{65}\right)\cdot\left(-\frac{8}{31}\right),\ P_3 = \frac{-5}{10}\cdot\frac{-4}{10}\cdot\frac{-3}{10}\cdots\frac{3}{10}\cdot\frac{4}{10}\cdot\frac{5}{10}.
	\end{align*}
\end{baitoan}

\begin{baitoan}[\cite{Tuyen_Toan_7}, Ví dụ 5, p. 7]
	Tìm giá trị của $x\in\mathbb{Q}$ để biểu thức sau có giá trị dương  $P = (x + 5)(x + 9)$.
\end{baitoan}

\begin{baitoan}[\cite{Tuyen_Toan_7}, \textbf{8.}, p. 7]
	Tìm $x$ biết: $\frac{11}{13} - \left(\frac{5}{42} - x\right) = -\left(\frac{15}{28} - \frac{11}{13}\right)$.
\end{baitoan}

\begin{baitoan}[\cite{Tuyen_Toan_7}, \textbf{9.}, p. 7]
	Cho $S = (a + b + c) - (a - b + c) + (a - b - c) + c$ với $a = 0.1$, $b = 0.01$, $c = 0.001$. Tính $S$.
\end{baitoan}

\begin{baitoan}[\cite{Tuyen_Toan_7}, \textbf{10.}, p. 7]
	Tính bằng cách hợp lý:\\
	\begin{enumerate*}
		\item[(a)] $\frac{11}{125} - \frac{17}{18} - \frac{5}{7} + \frac{4}{9} + \frac{17}{14}$;
		\item[(b)] $1 - \frac{1}{2} + 2 - \frac{2}{3} + 3 - \frac{3}{4} + 4 - \frac{1}{4} - 3 - \frac{1}{3} - 2 - \frac{1}{2} - 1$.
	\end{enumerate*}
\end{baitoan}

\begin{baitoan}[\cite{Tuyen_Toan_7}, \textbf{11.}, p. 7]
	Cho các số hữu tỷ $x = \frac{a}{9}$ \& $y = \frac{b}{9}$ trong đó $a$ là các số nguyên âm liên tiếp từ $-5$ đến $-1$; $b$ là các số nguyên dương liên tiếp từ $1$ đến $8$. Tính tổng $x + y$.
\end{baitoan}

\begin{baitoan}[\cite{Tuyen_Toan_7}, \textbf{12.}, p. 8]
	Cho $A = \frac{1}{2} + \frac{1}{4} + \frac{1}{8} + \frac{1}{16} + \frac{1}{32}$; $B = \frac{3}{2} + \frac{5}{4} + \frac{9}{8} + \frac{17}{16} + \frac{33}{32} - 6$. Tính $A$ \& $B$.
\end{baitoan}

\begin{baitoan}[\cite{Tuyen_Toan_7}, \textbf{13.}, p. 8]
	Cho $31$ số hữu tỷ sao cho bất kỳ 3 số nào trong chúng cũng có tổng là 1 số âm. Chứng minh tổng của $31$ số đó là 1 số âm.
\end{baitoan}

\begin{baitoan}[\cite{Tuyen_Toan_7}, \textbf{14.}, p. 8]
	Tìm $x$ biết:
	\begin{enumerate*}
		\item[(a)] $\left(\frac{1}{7}x - \frac{2}{7}\right)\left(-\frac{1}{5}x + \frac{3}{5}\right)\left(\frac{1}{3}x + \frac{4}{3}\right) = 0$;
		\item[(b)] $\frac{1}{6}x + \frac{1}{10}x - \frac{4}{15}x + 1 = 0$.
	\end{enumerate*}
\end{baitoan}

\begin{baitoan}[\cite{Tuyen_Toan_7}, \textbf{15.}, p. 8]
	Tính sau bằng cách hợp lý nhất:
	\begin{enumerate*}
		\item[(a)] $\left(-\frac{40}{51}\cdot 0.32\cdot\frac{17}{20}\right):\frac{64}{75}$;
		\item[(b)] $-\frac{10}{11}\cdot\frac{8}{9} + \frac{7}{18}\cdot\frac{10}{11}$;
		\item[(c)] $\frac{3}{14}:\frac{1}{28} - \frac{13}{21}:\frac{1}{28} + \frac{29}{42}:\frac{1}{28} - 8$;
		\item[(d)] $-1\frac{5}{7}\cdot 15 + \frac{2}{7}(-15) + (-105)\cdot\left(\frac{2}{3} - \frac{4}{5} + \frac{1}{7}\right)$.
	\end{enumerate*}
\end{baitoan}

\begin{baitoan}[\cite{Tuyen_Toan_7}, \textbf{16.}, p. 8]
	Tính giá trị các biểu thức sau:
	\begin{enumerate*}
		\item[(a)] $A = 7x - 2x - \frac{2}{3}y + \frac{7}{9}y$ với $x = -\frac{1}{10}$, $y = 4.8$;
		\item[(b)] $B = x + \dfrac{0.2 - 0.375 + \frac{5}{11}}{-0.3 + \frac{9}{16} - \frac{15}{22}}$ với $x = -\frac{1}{3}$. 
	\end{enumerate*}
\end{baitoan}

\begin{baitoan}[\cite{Tuyen_Toan_7}, \textbf{17.}, p. 8]
	Tìm giá trị của $x$ để các biểu thức sau có giá trị dương:
	\begin{enumerate*}
		\item[(a)] $A = x^2 + 4x$;
		\item[(b)] $B = (x - 3)(x + 7)$;
		\item[(c)] $C = \left(\frac{1}{2} - x\right)\left(\frac{1}{3} - x\right)$.
	\end{enumerate*}
\end{baitoan}

\begin{baitoan}[\cite{Tuyen_Toan_7}, \textbf{18.}, p. 8]
	Tìm các giá trị của $x$ để các biểu thức sau có giá trị âm:
	\begin{enumerate*}
		\item[(a)] $D = x^2 - \frac{2}{5}x$;
		\item[(b)] $E = \frac{x - 2}{x - 6}$.
	\end{enumerate*}
\end{baitoan}

\begin{baitoan}[\cite{Tuyen_Toan_7}, \textbf{19.}, p. 8]
	Tìm $x,y\in\mathbb{Q}$, $y\ne 0$ thỏa $x - y = xy = x:y$.
\end{baitoan}

\begin{baitoan}[\cite{Tuyen_Toan_7}, \textbf{20.}, p. 8]
	Cho $100$ số hữu tỷ trong đó tích của bất kỳ 3 số nào cũng là 1 số âm. Chứng minh:
	\begin{enumerate*}
		\item[(a)] Tích của $100$ số đó là 1 số dương;
		\item[(b)] Tất cả $100$ số đó đều là số âm.
	\end{enumerate*}
\end{baitoan}

\begin{baitoan}[\cite{Binh_Toan_7_tap_1}, Ví dụ 1, p. 3]
	Tính $A = \frac{1}{2} - \frac{1}{3} - \frac{1}{6} - \frac{1}{2} - \frac{1}{3} - \frac{1}{6} - \frac{1}{2} - \frac{1}{3} - \frac{1}{6} - \cdots$ ($A$ có $300$ số hạng).
\end{baitoan}

\begin{baitoan}[\cite{Binh_Toan_7_tap_1}, Ví dụ 2, p. 4]
	Cho phân số $\frac{a}{b}\ne 1$.
	\begin{enumerate*}
		\item[(a)] Tìm phân số $x$ sao cho nhân $x$ với $\frac{a}{b}$ cũng bằng cộng $x$ với $\frac{a}{b}$.
		\item[(b)] Tìm giá trị của $x$ trong câu (a) nếu $\frac{a}{b} = \frac{7}{5}$, nếu $\frac{a}{b} = \frac{8}{11}$.
	\end{enumerate*}
\end{baitoan}

\begin{baitoan}[\cite{Binh_Toan_7_tap_1}, Ví dụ 3, p. 4]
	Tìm $x\in\mathbb{Q}$, $x < 0$ để $\frac{4}{x - 1}\in\mathbb{Z}$.
\end{baitoan}

\begin{baitoan}[\cite{Binh_Toan_7_tap_1}, Ví dụ 4, p. 5]
	Tân đạp xe từ trường về nhà với thời gian dự kiến. Nhưng Tân đã dùng $\frac{2}{3}$ thời gian dự kiến để đi $\frac{3}{4}$ quãng đường với vận tốc $v_1$, rồi đi quãng đường còn lại với vận tốc $v_2$ \& đã về nhà đúng thời điểm dự kiến. Tính tỷ số $v_1:v_2$.
\end{baitoan}

\begin{baitoan}[\cite{Binh_Toan_7_tap_1}, Mở rộng Ví dụ 4, p. 5]
	Tân đạp xe từ trường về nhà với thời gian dự kiến. Nhưng Tân đã dùng $a$ thời gian dự kiến để đi $b$ quãng đường với vận tốc $v_1$, $a,b > 0$, $a + b < 1$, rồi đi quãng đường còn lại với vận tốc $v_2$ \& đã về nhà đúng thời điểm dự kiến. Tính tỷ số $v_1:v_2$ theo $a,b$.
\end{baitoan}

\begin{baitoan}[\cite{Binh_Toan_7_tap_1}, \textbf{1.}, p. 5]
	So sánh các số hữu tỷ:
	\begin{enumerate*}
		\item[(a)] $-\frac{18}{91}$ \& $-\frac{23}{114}$;
		\item[(b)] $-\frac{22}{35}$ \& $-\frac{103}{177}$.
	\end{enumerate*}
\end{baitoan}

\begin{baitoan}[\cite{Binh_Toan_7_tap_1}, \textbf{2.}, p. 5]
	Tìm $2$ phân số có tử bằng $9$, biết giá trị của mỗi phân số đó lớn hơn $-\frac{11}{13}$ \& nhỏ hơn $-\frac{11}{15}$.
\end{baitoan}

\begin{baitoan}[\cite{Binh_Toan_7_tap_1}, \textbf{3.}, p. 5]
	Cho các số hữu tỷ $\frac{a}{b}$ \& $\frac{c}{d}$ với mẫu dương, trong đó $\frac{a}{b} < \frac{c}{d}$. Chứng minh:
	\begin{enumerate*}
		\item[(a)] $ab < bc$;
		\item[(b)] $\frac{a}{b} < \frac{a + c}{b + d} < \frac{c}{d}$.
	\end{enumerate*}
\end{baitoan}

\begin{baitoan}[\cite{Binh_Toan_7_tap_1}, \textbf{4.}, p. 5]
	Tính:
	\begin{enumerate*}
		\item[(a)] $\frac{-2}{3} + \frac{3}{4} - \frac{-1}{6} + \frac{-2}{5}$;
		\item[(b)] $\frac{-2}{3} + \frac{-1}{5} + \frac{3}{4} - \frac{5}{6} - \frac{-7}{10}$;
		\item[(c)] $\frac{1}{2} - \frac{-2}{5} + \frac{1}{3} + \frac{5}{7} - \frac{-1}{6} + \frac{-4}{35} + \frac{1}{41}$;
		\item[(d)] $\frac{1}{100\cdot 99} - \frac{1}{99\cdot 98} - \frac{1}{98\cdot 97} - \cdots - \frac{1}{3\cdot 2} - \frac{1}{2\cdot 1}$.
	\end{enumerate*}
\end{baitoan}

\begin{baitoan}[\cite{Binh_Toan_7_tap_1}, \textbf{5.}, pp. 5--6]
	Ký hiệu $\lfloor x\rfloor$ là số nguyên lớn nhất không vượt quá $x$, được gọi là \emph{phần nguyên} của $x$, e.g., $\lfloor 1.5\rfloor = 1$, $\lfloor 5\rfloor = 5$, $\lfloor -2.5\rfloor = -3$.
	\begin{enumerate*}
		\item[(a)] Tính $\lfloor-\frac{1}{7}\rfloor,\lfloor 3.7\rfloor,\lfloor-4\rfloor,\lfloor-\frac{43}{10}\rfloor$.
		\item[(b)] Cho $x = 3.7$. So sánh: $A = \lfloor x\rfloor + \lfloor x + \frac{1}{5}\rfloor + \lfloor x + \frac{2}{5}\rfloor + \lfloor x + \frac{3}{5}\rfloor$ $+ \lfloor x + \frac{4}{5}\rfloor$ \& $B = \lfloor 5x\rfloor$.
		\item[(c)] Tính $ \lfloor\frac{100}{3}\rfloor + \lfloor\frac{100}{3^2}\rfloor + \lfloor\frac{100}{3^3}\rfloor + \lfloor\frac{100}{3^4}\rfloor$.
		\item[(d)] Tính $ \lfloor\frac{50}{2}\rfloor + \lfloor\frac{50}{2^2}\rfloor + \lfloor\frac{50}{2^3}\rfloor + \lfloor\frac{50}{2^4}\rfloor + \lfloor\frac{50}{2^5}\rfloor$.
		\item[(e)] Cho $x\in\mathbb{Q}$. So sánh $\lfloor x\rfloor$ với $x$, so sánh $\lfloor x\rfloor$ với $y$ trong đó $y\in\mathbb{Z}$, $y < x$.
	\end{enumerate*}
\end{baitoan}

\begin{baitoan}[\cite{Binh_Toan_7_tap_1}, \textbf{6.}, p. 6]
	Cho các số hữu tỷ $x$ bằng $1.4089, 0.1398, -0.4771, -1.2592$.
	\begin{enumerate*}
		\item[(a)] Viết các số đó dưới dạng tổng của 1 số nguyên $a$ \& 1 số thập phân $b$ không âm nhỏ hơn $1$.
		\item[(b)] Tính tổng các số hữu tỷ trên bằng 2 cách: tính theo cách thông thường, tính tổng các số được viết dưới dạng ở (a).
		\item[(c)] So sánh $a$ \& $\lfloor x\rfloor$ trong trường hợp ở câu (a). Lưu ý: Trong cách viết này, $a$ là \emph{phần nguyên} của $x$, còn $b$ là \emph{phần lẻ} của $x$. Ký hiệu phần lẻ của $x$ là $\{x\}$ thì $x = \lfloor x\rfloor + \{x\}$.
	\end{enumerate*}
\end{baitoan}

\begin{baitoan}[\cite{Binh_Toan_7_tap_1}, \textbf{7.}, p. 6]
	Tìm $n\in\mathbb{Z}$ để phân số sau có giá trị là 1 số nguyên \& tính giá trị đó:
	\begin{enumerate*}
		\item[(a)] $A = \frac{3n + 9}{n - 4}$;
		\item[(b)] $B = \frac{6n + 5}{2n - 1}$.
	\end{enumerate*}
\end{baitoan}

\begin{baitoan}[\cite{Binh_Toan_7_tap_1}, \textbf{8.}, p. 6]
	Tìm $x,y\in\mathbb{Z}$, biết: $\frac{5}{x} + \frac{y}{4} = \frac{1}{8}$.
\end{baitoan}

\begin{baitoan}[\cite{Binh_Toan_7_tap_1}, \textbf{9.}, p. 6]
	Viết tất cả các số nguyên có giá trị tuyệt đối nhỏ hơn $20$ theo thứ tự tùy ý. Lấy mỗi số trừ đi số thứ tự của nó ta được 1 hiệu. Tổng của tất cả các hiệu đó bằng bao nhiêu?
\end{baitoan}

\begin{baitoan}[\cite{Binh_Toan_7_tap_1},  \textbf{10.}, p. 6]
	Tính:
	\begin{enumerate*}
		\item[(a)] $\dfrac{\left(\frac{3}{10} - \frac{4}{15} - \frac{7}{20}\right)\cdot\frac{5}{19}}{\left(\frac{1}{14} + \frac{1}{7} - \frac{-3}{35}\right)\cdot\frac{-4}{3}}$;
		\item[(b)] $\dfrac{(1 + 2 + \cdots + 100)\left(\frac{1}{3} - \frac{1}{5} - \frac{1}{7} - \frac{1}{9}\right)\cdot(6.3\cdot 12 - 21\cdot 3.6)}{\frac{1}{2} + \frac{1}{3} + \cdots + \frac{1}{100}}$;
		\item[(c)] $\dfrac{\frac{1}{9} - \frac{1}{7} - \frac{1}{11}}{\frac{4}{9} - \frac{4}{7} - \frac{4}{11}} + \dfrac{\frac{3}{5} - \frac{3}{25} - \frac{3}{125} - \frac{3}{625}}{\frac{4}{5} - \frac{4}{25} - \frac{4}{125} - \frac{4}{625}}$.
	\end{enumerate*}
\end{baitoan}

\begin{baitoan}[\cite{Binh_Toan_7_tap_1}, \textbf{11.}, p. 7]
	Tìm $x\in\mathbb{Q}$, biết:
	\begin{enumerate*}
		\item[(a)] $\frac{2}{3}x - 4 = -12$;
		\item[(b)] $\frac{3}{4} + \frac{1}{4}:x = -3$;
		\item[(c)] $|3x - 5| = 4$;
		\item[(d)] $\frac{x + 1}{10} + \frac{x + 1}{11} + \frac{x + 1}{12} = \frac{x + 1}{13} + \frac{x + 1}{14}$;
		\item[(e)] $\frac{x + 4}{2000} + \frac{x + 3}{2001} = \frac{x + 2}{2002} + \frac{x + 1}{2003}$.
	\end{enumerate*}
\end{baitoan}

\begin{baitoan}[\cite{Binh_Toan_7_tap_1}, \textbf{12.}, p. 7]
	Cho phân số $\frac{a}{b}$ với $a,b\in\mathbb{N}^\star$. Tìm phân số $x$ sao cho $\frac{a}{b} - x = \frac{a}{b}\cdot x$.
\end{baitoan}

\begin{baitoan}[\cite{Binh_Toan_7_tap_1}, \textbf{13.}, p. 7]
	Trung bình cộng của 2 số lớn hơn số thứ nhất $75$\% thì nhỏ hơn số thứ 2 bao nhiêu \%?
\end{baitoan}

\begin{baitoan}[\cite{Binh_Toan_7_tap_1}, \textbf{14.}, p. 7]
	Chứng minh:\\	
	\begin{enumerate*}
		\item[(a)] $\sum_{i=1}^{99} \frac{i}{(i+1)!} = \frac{1}{2!} + \frac{2}{3!} + \frac{3}{4!} + \cdots + \frac{99}{100!} < 1$.
		\item[(b)] $\sum_{i=1}^{99} \frac{i(i + 1) - 1}{(i+1)!} = \frac{1\cdot 2 - 1}{2!} + \frac{2\cdot 3 - 1}{3!} + \frac{3\cdot 4 - 1}{4!} + \cdots + \frac{99\cdot 100 - 1}{100!} < 2$.
	\end{enumerate*}
\end{baitoan}

\begin{baitoan}[\cite{Binh_Toan_7_tap_1}, \textbf{15.}, p. 7]
	\begin{enumerate*}
		\item[(a)] Người ta viết $7$ số hữu tỷ trên 1 vòng tròn. Tìm các số đó, biết tích của $2$ số bất kỳ cạnh nhau bằng $16$.
		\item[(b)] Cũng hỏi như trên đối với $n$ số.
	\end{enumerate*}
\end{baitoan}

\begin{baitoan}[\cite{Binh_Toan_7_tap_1}, \textbf{16.}, p. 7]
	Có tồn tại hay không $2$ số dương $a,b$ khác nhau sao cho $\frac{1}{a} - \frac{1}{b} = \frac{1}{a - b}$?
\end{baitoan}

\begin{baitoan}[\cite{Binh_Toan_7_tap_1}, \textbf{17.}${}^\star$, p. 7]
	\begin{enumerate*}
		\item[(a)] Chứng minh: $\frac{1}{1\cdot 2} + \frac{1}{3\cdot 4} + \frac{1}{5\cdot 6} + \cdots + \frac{1}{49\cdot 50} = \frac{1}{26} + \frac{1}{27} + \frac{1}{28} + \cdots + \frac{1}{50}$.
		\item[(b)] Cho $B = \frac{1}{1\cdot 2} + \frac{1}{3\cdot 4} + \frac{1}{5\cdot 6} + \cdots + \frac{1}{99\cdot 100}$. Chứng minh $\frac{7}{12} < B < \frac{5}{6}$.
	\end{enumerate*}	
\end{baitoan}

\begin{baitoan}[\cite{Binh_Toan_7_tap_1}, \textbf{18.}, p. 7]
	Tìm $a,b\in\mathbb{Q}$ sao cho:
	\begin{enumerate*}
		\item[(a)] $a - b = 2(a + b) = a:b$.
		\item[(b)] $a + b = ab = a:b$.
	\end{enumerate*}
\end{baitoan}

\begin{baitoan}[\cite{Binh_Toan_7_tap_1}, \textbf{19.}${}^\star$, p. 7]
	Tìm $x\in\mathbb{Q}$, sao cho tổng của số đó với số nghịch đảo của nó là 1 số nguyên.
\end{baitoan}

\begin{baitoan}[\cite{Binh_Toan_7_tap_1}, \textbf{20.}${}^\star$, p. 8]
	Viết tất cả các số hữu tỷ dương  thành dãy gồm các nhóm phân số có tổng của tử \& mẫu lần lượt bằng $2,3,4,5,\ldots$, các phân số trong cùng 1 nhóm được đặt trong dấu ngoặc: $\left(\frac{1}{1}\right),\left(\frac{2}{1},\frac{1}{2}\right),\left(\frac{3}{1},\frac{2}{2},\frac{1}{3}\right),\left(\frac{4}{1},\frac{3}{2},\frac{2}{3},\frac{1}{4}\right),\ldots$. Tìm phân số thứ $200$ của dãy.
\end{baitoan}

%------------------------------------------------------------------------------%

\subsection{Lũy Thừa của 1 Số Hữu Tỷ}
``\begin{enumerate*}
	\item[\textbf{1.}] Lũy thừa với số mũ tự nhiên $x^n = \underbrace{x\cdot x\cdots x}_{\scriptsize n\mbox{ thừa số } x}$, $\forall x\in\mathbb{Q}$, $\forall n\in\mathbb{N}$, $n > 1$. Nếu $x = \frac{a}{b}$ thì $\left(\frac{a}{b}\right)^n = \frac{a^n}{b^n}$, $\forall a,b\in\mathbb{Z}$, $b\ne 0$. Quy ước: $x^0 = 1$, $\forall x\in\mathbb{Q}$, $x\ne 0$, $x^1 = x$, $\forall x\in\mathbb{Q}$.
	\item[\textbf{2.}] $x^m\cdot x^n = x^{m + n}$, $x^m:x^n = \frac{x^m}{x^n} = x^{m-n}$, $(x^m)^n = x^{mn}$, $\forall x,y\in\mathbb{Q}$, $x\ne 0$, $\forall m,n\in\mathbb{N}^\star$, $m\ge n$; $\left(\frac{x}{y}\right)^n = \frac{x^n}{y^n}$, $y\ne 0$.
	\item[\textbf{3.}] Lũy thừa với số mũ nguyên âm $x^{-n} = \frac{1}{x^n}$, $\forall x\in\mathbb{Q}$, $x\ne 0$, $\forall n\in\mathbb{N}^\star$.
	\item[\textbf{4.}] So sánh 2 lũy thừa:
	\begin{enumerate*}
		\item[(a)] \textit{Cùng cơ số}: Với $m > n > 0$ thì:
		\begin{enumerate*}
			\item[$\bullet$] $x > 1\Rightarrow x^m > x^n$;
			\item[$\bullet$] $x = 1\Rightarrow x^m = x^n$;
			\item[$\bullet$] $0 < x < 1\Rightarrow x^m > x^n$.
		\end{enumerate*}
		\item[(b)] \textit{Cùng số mũ}: $\forall n\in\mathbb{N}^\star$:
		\begin{enumerate*}
			\item[$\bullet$] Với $x,y > 0$, nếu $x > y$ thì $x^n > y^n$;
			\item[$\bullet$] $x > y\Leftrightarrow x^{2n + 1} > y^{2n + 1}$;
			\item[$\bullet$] $(-x)^{2n} = x^{2n}$;
			\item[$\bullet$] $(-x)^{2n + 1} = -x^{2n + 1}$.'' -- \cite[\S3, pp. 8--9]{Tuyen_Toan_7}
		\end{enumerate*} 
	\end{enumerate*}
\end{enumerate*}

\begin{baitoan}[\cite{Tuyen_Toan_7}, Ví dụ 6, p. 9]
	Chứng minh: Không tồn tại 3 số hữu tỷ $x,y,z$ sao cho $xy = \frac{13}{15}$, $yz = \frac{11}{3}$, $zx = -\frac{3}{13}$.
\end{baitoan}

\begin{baitoan}[\cite{Tuyen_Toan_7}, Ví dụ 7, p. 9]
	Tìm $x$ biết $(3^x)^2:3^3 = \frac{1}{243}$.
\end{baitoan}

\begin{baitoan}[\cite{Tuyen_Toan_7}, Ví dụ 8, p. 9]
	Tìm $x$ biết: $(3x^2 - 51)^{2n} = (-24)^{2n}$, $n\in\mathbb{N}^\star$.
\end{baitoan}

\begin{baitoan}[\cite{Tuyen_Toan_7}, \textbf{21.}, p. 10]
	Viết các số sau dưới dạng 1 lũy thừa với số mũ tự nhiên lớn hơn $1$:
	\begin{enumerate*}
		\item[(a)] $64$, $81$, $-216$;
		\item[(b)] $-\frac{1}{27}$, $\frac{8}{729}$, $\frac{16}{625}$.
	\end{enumerate*}
\end{baitoan}

\begin{baitoan}[\cite{Tuyen_Toan_7}, \textbf{22.}, p. 10]
	Dùng lũy thừa với số mũ nguyên âm để viết gọn các số sau:
	\begin{enumerate*}
		\item[(a)] Đường kính của nguyên tử cỡ $0.000 000 001$\emph{m};
		\item[(a)] Đường kính của hạt nhân nguyên tử cỡ $0.000 000 000 000 001$\emph{m};
		\item[(c)] Khối lượng hạt nhân nguyên tử cỡ $0.\underbrace{000\ldots 00}_{\scriptsize 23\mbox{ số } 0}1$ gam.
	\end{enumerate*}
\end{baitoan}

\begin{baitoan}[\cite{Tuyen_Toan_7}, \textbf{23.}, p. 10]
	Viết các biểu thức sau dưới dạng lũy thừa của 1 số nguyên:
	\begin{enumerate*}
		\item[(a)] $12^3:(3^{-4}\cdot 64)$;
		\item[(b)] $\left(\frac{3}{7}\right)^5\cdot\left(\frac{7}{3}\right)^{-1}\cdot\left(\frac{5}{3}\right)^6:\left(\frac{343}{625}\right)^{-2}$;
		\item[(c)] $5^4\cdot 125\cdot(2.5)^{-5}\cdot 0.04$.
	\end{enumerate*}
\end{baitoan}

\begin{baitoan}[\cite{Tuyen_Toan_7}, \textbf{24.}, p. 10]
	Cho $A = (ax + by)^2$, $B = (a^2 + b^2)(x^2 + y^2)$. So sánh giá trị của 2 biểu thức $A$ \& $B$ biết: $a = 2$, $b = -1$, $x = \frac{8}{11}$, $y = \frac{-5}{11}$.
\end{baitoan}

\begin{baitoan}[\cite{Tuyen_Toan_7}, \textbf{25.}, p. 10]
	So sánh $\left(\frac{1}{8}\right)^6$ với $\left(\frac{1}{32}\right)^4$.
\end{baitoan}

\begin{baitoan}[\cite{Tuyen_Toan_7}, \textbf{26.}, p. 10]
	So sánh $4^{30}$ với $1000\cdot 32^{10}$.
\end{baitoan}

\begin{baitoan}[\cite{Tuyen_Toan_7}, \textbf{27.}, p. 10]
	Tìm $x$ biết:
	\begin{enumerate*}
		\item[(a)] $5^x\cdot(5^3)^2 = 625$;
		\item[(b)] $\left(\frac{12}{25}\right)^x = \left(\frac{3}{5}\right)^2 - \left(-\frac{3}{5}\right)^4$;
		\item[(c)] $\left(-\frac{3}{4}\right)^{3x - 1} = \frac{256}{81}$.
	\end{enumerate*}
\end{baitoan}

\begin{baitoan}[\cite{Tuyen_Toan_7}, \textbf{28.}, p. 10]
	Tìm $x\in\mathbb{N}$ biết:
	\begin{enumerate*}
		\item[(a)] $8 < 2^x\le 2^9:2^5$;
		\item[(b)] $27 < 81^3:3^x < 243$;
		\item[(c)] $\left(\frac{2}{5}\right)^x > \left(\frac{5}{2}\right)^{-3}\cdot\left(-\frac{2}{5}\right)^2$.
	\end{enumerate*}
\end{baitoan}

\begin{baitoan}[\cite{Tuyen_Toan_7}, \textbf{29.}, p. 10]
	Tìm $x$ biết:
	\begin{enumerate*}
		\item[(a)] $(5x + 1)^2 = \frac{36}{49}$;
		\item[(b)] $\left(x - \frac{2}{9}\right)^3 = \left(\frac{2}{3}\right)^6$;
		\item[(c)] $(8x - 1)^{2n + 1} = 5^{2n + 1}$, $n\in\mathbb{N}$.
	\end{enumerate*}
\end{baitoan}

\begin{baitoan}[\cite{Tuyen_Toan_7}, \textbf{30.}, p. 10]
	 Tìm $x,y$ biết:
	 \begin{enumerate*}
	 	\item[(a)] $x^2 + \left(y - \frac{1}{10}\right)^4 = 0$;
	 	\item[(b)] $\left(\frac{1}{2}x - 5\right)^{20} + \left(y^2 - \frac{1}{4}\right)^{10}\le 0$.
	\end{enumerate*}
\end{baitoan}

\begin{baitoan}[\cite{Tuyen_Toan_7}, \textbf{31.}, p. 10]
	Tìm $x\in\mathbb{Z}$ biết: $(x - 7)^{x + 1} - (x - 7)^{x + 11} = 0$.
\end{baitoan}

\begin{baitoan}[\cite{Tuyen_Toan_7}, \textbf{32.}, p. 10]
	Tìm $x,y$ biết: $x(x - y) = \frac{3}{10}$, $y(x - y) = -\frac{3}{50}$.
\end{baitoan}

\begin{baitoan}[\cite{Tuyen_Toan_7}, \textbf{33.}, p. 11]
	Tìm:
	\begin{enumerate*}
		\item[(a)] Giá trị nhỏ nhất (GTNN) của biểu thức $A = \left(2x + \frac{1}{3}\right)^2 - 1$;
		\item[(b)] Giá trị lớn nhất (GTLN) của biểu thức $B = -\left(\frac{4}{9}x - \frac{2}{15}\right)^6 + 3$.
	\end{enumerate*}
\end{baitoan}

\begin{baitoan}[\cite{Binh_Toan_7_tap_1}, Ví dụ 5, p. 8]
	\begin{enumerate*}
		\item[(a)] Chứng minh: $2^{10}\approx 10^3$ \& $9^{10}\approx 80^5$.
		\item[(b)] Dùng nhận xét ở (a) để chứng minh $9^{10}\approx 3.2\cdot 10^9$.
	\end{enumerate*}
\end{baitoan}

\begin{baitoan}[\cite{Binh_Toan_7_tap_1}, Ví dụ 6, p. 8]
	Tính: $A = \sum_{i=1}^{10} \frac{i}{2^i} = \frac{1}{2} + \frac{2}{2^2} + \frac{3}{2^3} + \cdots + \frac{10}{2^{10}}$.
\end{baitoan}

\begin{baitoan}[\cite{Binh_Toan_7_tap_1}, Ví dụ 7, p. 9]
	\begin{enumerate*}
		\item[(a)] Có thể khẳng định $x^2$ luôn luôn lớn hơn $x$ hay không?
		\item[(b)] Khi nào thì $x^2 < x$?
	\end{enumerate*}
\end{baitoan}

\begin{baitoan}[\cite{Binh_Toan_7_tap_1}, Ví dụ 8, p. 9]
	Tìm $a,b,c\in\mathbb{Q}$, biết: $ab = 2$, $bc = 3$, $ca = 54$.
\end{baitoan}

\begin{baitoan}[\cite{Binh_Toan_7_tap_1}, Ví dụ 9, p. 9]
	Rút gọn: $A = \sum_{i=0}^{50} 5^i = 1 + 5 + 5^2 + \cdots + 5^{49} + 5^{50}$.
\end{baitoan}

\begin{baitoan}[\cite{Binh_Toan_7_tap_1}, Ví dụ 10, p. 9]
	Cho $B = \sum_{i=1}^{99} \left(\frac{1}{2}\right)^i = \frac{1}{2} + \left(\frac{1}{2}\right)^2 + \cdots + \left(\frac{1}{2}\right)^{98} + \left(\frac{1}{2}\right)^{99}$. Chứng minh $B < 1$.
\end{baitoan}

\begin{baitoan}[\cite{Binh_Toan_7_tap_1}, \textbf{21.}, p. 10]
	Chứng minh:
	\begin{enumerate*}
		\item[(a)] $7^6 + 7^5 - 7^4\divby 55$;
		\item[(b)] $16^5 + 2^{15}\divby 33$;
		\item[(c)] $81^7 - 27^9 - 9^{13}\divby 405$.
	\end{enumerate*}
\end{baitoan}

\begin{baitoan}[\cite{Binh_Toan_7_tap_1}, \textbf{22.}, p. 10]
	Điền vào chỗ chấm ($\cdots$) các từ ``bằng nhau'' hoặc ``đối nhau'' cho đúng:
	\begin{enumerate*}
		\item[(a)] Nếu 2 số đối nhau thì bình phương của chúng $\ldots$.
		\item[(b)] Nếu 2 số đối nhau thì lập phương của chúng $\ldots$.
		\item[(c)] Lũy thừa chẵn cùng bậc của 2 số đối nhau thì $\ldots$.
		\item[(d)] Lũy thừa lẻ cùng bậc của 2 số đối nhau thì $\ldots$.
	\end{enumerate*}
\end{baitoan}

\begin{baitoan}[\cite{Binh_Toan_7_tap_1}, \textbf{23.}, p. 10 \& mở rộng]
	Các đẳng thức sau có đúng với mọi $a,b\in\mathbb{Q}$ hay không?
	\begin{enumerate*}
		\item[(a)] $-a^3 = (-a)^3$;
		\item[(b)] $-a^5 = (-a)^5$;
		\item[(c)] $-a^2 = (-a)^2$;
		\item[(d)] $-a^4 = (-a)^4$;
		\item[(e)] $-a^{2n+1} = (-a)^{2n+1}$, $\forall n\in\mathbb{N}$;
		\item[(f)] $a^{2n} = (-a)^{2n}$, $\forall n\in\mathbb{N}$;
		\item[(g)] $(a - b)^2 = (b - a)^2$;
		\item[(h)] $(a - b)^3 = -(b - a)^3$;
		\item[(i)] $(a - b)^{2n} = (b - a)^{2n}$, $\forall n\in\mathbb{N}$;
		\item[(j)] $(a - b)^{2n+1} = -(b - a)^{2n+1}$, $\forall n\in\mathbb{N}$.
	\end{enumerate*}
\end{baitoan}

\begin{baitoan}[\cite{Binh_Toan_7_tap_1}, \textbf{24.}, p. 10]
	Tính:
	\begin{enumerate*}
		\item[(a)] $\left(\frac{1}{2}\right)^{15}\cdot\left(\frac{1}{4}\right)^{20}$;
		\item[(b)] $\left(\frac{1}{9}\right)^{25}:\left(\frac{1}{3}\right)^{30}$;
		\item[(c)] $\left(\frac{1}{16}\right)^3:\left(\frac{1}{8}\right)^2$;
		\item[(d)] $(x^3)^2:(x^2)^3$ với $x\ne 0$.
	\end{enumerate*}
\end{baitoan}

\begin{baitoan}[\cite{Binh_Toan_7_tap_1}, \textbf{25.}, p. 10]
	Viết số $64$ dưới dạng $a^n$ với $a\in\mathbb{Z}$. Có bao nhiêu cách viết?
\end{baitoan}

\begin{baitoan}[\cite{Binh_Toan_7_tap_1}, \textbf{26.}, p. 10]
	Rút gọn biểu thức: $A = \dfrac{4^5\cdot 9^4 - 2\cdot 6^9}{2^{10}\cdot 3^8 + 6^8\cdot 20}$.
\end{baitoan}

\begin{baitoan}[\cite{Binh_Toan_7_tap_1}, \textbf{27.}, p. 10]
	\begin{enumerate*}
		\item[(a)] Chứng minh: $2^{10}\approx 10^3$ \& $3^{16}\approx 80^4$.
		\item[(b)] Dùng nhận xét ở (a) để chứng minh $3^{16}\approx 40000000$.
	\end{enumerate*}
\end{baitoan}

\begin{baitoan}[\cite{Binh_Toan_7_tap_1}, \textbf{28.}, p. 10]
	Cho $S_n = \sum_{i=1}^{n-1} (-1)^{i-1}i = 1 - 2 + 3 - 4 + \cdots + (-1)^{n-1}n$ với $n\in\mathbb{N}^\star$. Tính $S_{35} + S_{60}$.
\end{baitoan}

\begin{baitoan}[\cite{Binh_Toan_7_tap_1}, \textbf{29.}, p. 10]
	Cho $A = 1 - 5 + 9 - 13 + 17 - 21 + 25 - \cdots$ ($n$ số hạng, giá trị tuyệt đối của số sau lớn hơn giá trị tuyệt đối của số hạng trước $4$ đơn vị, các dấu $+$ \& $-$ xen kẽ).
	\begin{enumerate*}
		\item[(a)] Tính $A$ theo $n$.
		\item[(b)] Viết số hạng thứ $n$ của biểu thức $A$ theo $n$ (chú ý dùng lũy thừa để biểu thị dấu của số hạng đó).
	\end{enumerate*}
\end{baitoan}

\begin{baitoan}[\cite{Binh_Toan_7_tap_1}, \textbf{30.}, p. 11]
	Với giá trị nào của các chữ thì các biểu thức sau có giá trị là số $0$, số dương, số âm?
	\begin{enumerate*}
		\item[(a)] $P = \frac{a^2b}{c}$;
		\item[(b)] $Q = \frac{x^3}{yz}$.
	\end{enumerate*}
\end{baitoan}

\begin{baitoan}[\cite{Binh_Toan_7_tap_1}, \textbf{31.}, p. 11]
	Cho $2$ số hữu tỷ $a$ \& $b$ trái dấu trong đó $|a| = b^5$. Xác định dấu của mỗi số.
\end{baitoan}

\begin{baitoan}[\cite{Binh_Toan_7_tap_1}, \textbf{32.}, p. 11]
	Viết các số sau dưới dạng lũy thừa của $2$: $16,64,1,\frac{1}{32},\frac{1}{8},0.5,0.25$.
\end{baitoan}

\begin{baitoan}[\cite{Binh_Toan_7_tap_1}, \textbf{33.}, p. 11]
	\begin{enumerate*}
		\item[(a)] Viết các số sau thành lũy thừa với số mũ âm: $\frac{1}{1000000},0.00000002$.
		\item[(b)] Viết các số sau dưới dạng số thập phân: $10^{-7}$, $2.5\cdot 10^{-6}$.
	\end{enumerate*}
\end{baitoan}

\begin{baitoan}[\cite{Binh_Toan_7_tap_1}, \textbf{34.}, p. 11]
	Tính xem $A$ gấp mấy lần $B$:
	\begin{enumerate*}
		\item[(a)] $A = 3.4\cdot 10^{-8}$, $B = 34\cdot 10^{-9}$;
		\item[(b)] $A = 10^{-4} + 10^{-3} + 10^{-2}$, $B = 10^{-9}$.
	\end{enumerate*}
\end{baitoan}

\begin{baitoan}[\cite{Binh_Toan_7_tap_1}, \textbf{35.}, p. 11]
	So sánh:
	\begin{enumerate*}
		\item[(a)] $\left(-\frac{1}{16}\right)^{100}$ \& $\left(-\frac{1}{2}\right)^{500}$;
		\item[(b)] $(-32)^9$ \& $(-18)^{13}$;
		\item[(c)] $a = 2^{100}$, $b = 3^{75}$, $c = 5^{50}$.
	\end{enumerate*}
\end{baitoan}

\begin{baitoan}[\cite{Binh_Toan_7_tap_1}, \textbf{36.}, p. 11]
	Trong các câu sau, câu nào đúng với mọi $a\in\mathbb{Q}$?
	\begin{enumerate*}
		\item[(a)] Nếu $a < 0$ thì $a^2 > 0$;
		\item[(b)] Nếu $a^2 > 0$ thì $a > 0$;
		\item[(c)] Nếu $a < 0$ thì $a^2 > a$;
		\item[(d)] Nếu $a^2 > a$ thì $a > 0$;
		\item[(e)] Nếu $a^2 > a$ thì $a < 0$.
	\end{enumerate*}
\end{baitoan}

\begin{baitoan}[\cite{Binh_Toan_7_tap_1}, \textbf{37.}, p. 11]
	\begin{enumerate*}
		\item[(a)] Cho $a^m = a^n$ ($a\in\mathbb{Q}$, $m,n\in\mathbb{N}$). Tìm $m,n$.
		\item[(b)] Cho $a^m > a^n$ ($a\in\mathbb{Q}$, $a > 0$, $m,n\in\mathbb{N}$). So sánh $m$ \& $n$.
	\end{enumerate*}
\end{baitoan}

\begin{baitoan}[\cite{Binh_Toan_7_tap_1}, \textbf{38.}, p. 11]
	Tìm $x\in\mathbb{Q}$, biết:
	\begin{enumerate*}
		\item[(a)] $(2x - 1)^4 = 81$;
		\item[(b)] $(x - 1)^5 = -32$;
		\item[(c)] $(2x - 1)^6 = (2x - 1)^8$.
	\end{enumerate*}
\end{baitoan}

\begin{baitoan}[\cite{Binh_Toan_7_tap_1}, \textbf{39.}, p. 11]
	Tìm $x\in\mathbb{N}$, biết:
	\begin{enumerate*}
		\item[(a)] $5^x + 5^{x+2} = 650$;
		\item[(b)] $3^{x-1} + 5\cdot 3^{x-1} = 162$.
	\end{enumerate*}
\end{baitoan}

\begin{baitoan}[\cite{Binh_Toan_7_tap_1}, \textbf{40.}, p. 11]
	Tìm $x,y\in\mathbb{N}$, biết:
	\begin{enumerate*}
		\item[(a)] $2^{x+1}\cdot 3^y = 12^x$;
		\item[(b)] $10^x:5^y = 20^y$;
		\item[(c)] $2^x = 4^{y-1}$ \& $27^y = 3^{x+8}$.
	\end{enumerate*}
\end{baitoan}

\begin{baitoan}[\cite{Binh_Toan_7_tap_1}, \textbf{41.}, p. 11]
	Tìm $a,b,c\in\mathbb{Q}$, biết:
	\begin{enumerate*}
		\item[(a)] $ab = \frac{3}{5}$, $bc = \frac{4}{5}$, $ca = \frac{3}{4}$.
		\item[(b)] $a(a + b + c) = -12$, $b(a + b + c) = 18$, $c(a + b + c) = 30$;
		\item[(c)] $ab = c$, $bc = 4a$, $ac = 9b$.
	\end{enumerate*}
\end{baitoan}

\begin{baitoan}[\cite{Binh_Toan_7_tap_1}, \textbf{42.}${}^\star$, p. 12]
	Cho $a,b,c,d,e\in\mathbb{N}$ thỏa mãn $a^b = b^c = c^d = d^e = e^a$. Chứng minh $a = b = c = d = e$.
\end{baitoan}

\begin{baitoan}[\cite{Binh_Toan_7_tap_1}, \textbf{43.}, p. 12]
	Cho $A = \prod_{i=2}^{100} \left(\frac{1}{i^2} - 1\right) = \left(\frac{1}{2^2} - 1\right)\left(\frac{1}{3^2} - 1\right)\left(\frac{1}{4^2} - 1\right)\cdots\left(\frac{1}{100^2} - 1\right)$. So sánh $A$ với $-\frac{1}{2}$.
\end{baitoan}

\begin{baitoan}[\cite{Binh_Toan_7_tap_1}, \textbf{44.}, p. 12]
	Rút gọn $A = \sum_{i=1}^{100} (-1)^i2^i = 2^{100} - 2^{99} + 2^{98} - 2^{97} + \cdots + 2^2 - 2$.
\end{baitoan}

\begin{baitoan}[\cite{Binh_Toan_7_tap_1}, \textbf{45.}, p. 12]
	Rút gọn $B = \sum_{i=
	}^{100} (-1)^i3^i = 3^{100} - 3^{99} + 3^{98} - 3^{97} + \cdots + 3^2 - 3 + 1$.
\end{baitoan}

\begin{baitoan}[\cite{Binh_Toan_7_tap_1}, \textbf{46.}, p. 12]
	Cho $C = \sum_{i=1}^{99} \frac{1}{3^i} = \frac{1}{3} + \frac{1}{3^2} + \cdots + \frac{1}{3^{99}}$. Chứng minh $C < \frac{1}{2}$.
\end{baitoan}

\begin{baitoan}[\cite{Binh_Toan_7_tap_1}, \textbf{47.}, p. 12]
	Chứng minh $\frac{3}{1^2\cdot 2^2} + \frac{5}{2^2\cdot 3^2} + \frac{7}{3^2\cdot 4^2} + \cdots + \frac{19}{9^2\cdot 10^2} < 1$.
\end{baitoan}

\begin{baitoan}[\cite{Binh_Toan_7_tap_1}, \textbf{48.}${}^\star$, p. 12]
	Chứng minh $\sum_{i=1}^{100} \frac{i}{3^i} = \frac{1}{3} + \frac{2}{3^2} + \frac{3}{3^3} + \cdots + \frac{100}{3^{100}} < \frac{3}{4}$.
\end{baitoan}

\begin{baitoan}[\cite{Binh_Toan_7_tap_1}, \textbf{49.}, p. 12]
	Ta không có $2^m + 2^n = 2^{m+n}$, $\forall m,n\in\mathbb{N}^\star$. Nhưng có những số nguyên dương $m,n$ có tính chất trên. Tìm các số đó.
\end{baitoan}

\begin{baitoan}[\cite{Binh_Toan_7_tap_1}, \textbf{50.}${}^\star$, p. 12]
	Tìm $m,n\in\mathbb{N}^\star$ sao cho $2^m - 2^n = 256$.
\end{baitoan}

\begin{baitoan}[\cite{Binh_Toan_7_tap_1}, \textbf{51.}${}^\star$, p. 12]
	Cho 1 bảng vuông $3\times 3$ ô. Trong mỗi ô của bảng viết số $1$ hoặc số $-1$. Gọi $d_i$ là tích các số trên dòng $i$ ($i = 1,2,3$), $c_k$ là tích các số trên cột $k$ ($k = 1,2,3$).
	\begin{enumerate*}
		\item[(a)] Chứng minh không thể xảy ra $d_1 + d_2 + d_3 + c_1 + c_2 + c_3 = 0$.
		\item[(b)] Xét bài toán trên đối với bảng vuông $n\times n$.
	\end{enumerate*}
\end{baitoan}

\begin{baitoan}[\cite{Binh_Toan_7_tap_1}, \textbf{52.}${}^\star$, p. 12]
	Cho $n$ số $x_1,\ldots,x_n$, mỗi số bằng $1$ hoặc $-1$. Biết tổng của $n$ tích $x_1x_2$, $x_2x_3$, $x_3x_4,\ldots,x_nx_1$ bằng $0$. Chứng minh $n\ \vdots\ 4$.
\end{baitoan}

%------------------------------------------------------------------------------%

\subsection{Thứ Tự Thực Hiện Các Phép Tính. Quy Tắc Chuyển Vế}
``\begin{enumerate*}
	\item[\textbf{1.}] Thứ tự thực hiện các phép tính đối với số hữu tỷ cũng tương tự như thứ tự thực hiện các phép tính đối với số tự nhiên, số nguyên, phân số.
	\item[\textbf{2.}] \textit{Quy tắc chuyển vế}: Khi chuyển 1 số hạng từ vế này sang vế kia của 1 đẳng thức, ta phải đổi dấu số hạng đó.
	\item[\textbf{3.}] Quy tắc dấu ngoặc đối với số hữu tỷ cũng tương tự quy tắc dấu ngoặc đối với các số nguyên, phân số, số thập phân.
	\item[\textbf{4.}] Nếu đưa các số hạng vào trong dấu ngoặc \& có dấu ``$-$'' đằng trước thì phải đổi dấu các số hạng đó.'' -- \cite[\S4, p. 11]{Tuyen_Toan_7} 
\end{enumerate*}

\begin{baitoan}[\cite{Tuyen_Toan_7}, Ví dụ 9, p. 11]
	Tính: $A = \frac{2}{5} - \left(\frac{7}{10}\right)^2:\frac{28}{25} + \left(\frac{1}{2}\right)^3\cdot(-3)$.
\end{baitoan}

\begin{baitoan}[\cite{Tuyen_Toan_7}, Ví dụ 10, p. 11]
	Tính: $B = \dfrac{3 + \frac{1}{6} - \frac{2}{5}}{5 - \frac{1}{6} + \frac{7}{10}} - \dfrac{3}{2}$.
\end{baitoan}

\begin{baitoan}[\cite{Tuyen_Toan_7}, Ví dụ 11, p. 12]
	Tìm $x$ biết: $\frac{3}{7}\left(x - \frac{14}{9}\right) = -\frac{11}{7}\left(x + \frac{14}{11}\right)$.
\end{baitoan}

\begin{baitoan}[\cite{Tuyen_Toan_7}, \textbf{34.}, p. 12]
	Tính: $2\frac{1}{8}:1\frac{11}{40} - \left(2^4 - 7\frac{13}{18}\right):11\frac{1}{27}$.
\end{baitoan}

\begin{baitoan}[\cite{Tuyen_Toan_7}, \textbf{35.}, p. 12]
	Tính: $1\frac{13}{15}\cdot\frac{3}{4} - \left[\frac{2^3}{4^2 - 1} + \left(\frac{1}{2}\right)^2\right]\cdot\frac{24}{47}$.
\end{baitoan}

\begin{baitoan}[\cite{Tuyen_Toan_7}, \textbf{36.}, p. 12]
	Tính giá trị của biểu thức: $A = \dfrac{2 - \frac{1}{3} + 2^{-2}}{2 + \frac{1}{6} - 2^{-2}} - 2^0$.
\end{baitoan}

\begin{baitoan}[\cite{Tuyen_Toan_7}, \textbf{37.}, p. 12]
	Tìm $x$ biết: $2\frac{1}{4}\cdot\left(x - 7\frac{1}{3}\right) = 1.5$.
\end{baitoan}

\begin{baitoan}[\cite{Tuyen_Toan_7}, \textbf{38.}, p. 12]
	Tìm $x$ biết: $\left(12\frac{7}{18} - 10\frac{13}{18}\right):x - 1\frac{7}{33}:\frac{8}{11} = 1\frac{2}{3}$.
\end{baitoan}

\begin{baitoan}[\cite{Tuyen_Toan_7}, \textbf{39.}, p. 12]
	Cho biểu thức $A = \frac{12}{17}:\frac{5}{51} - \frac{8}{35}\cdot 7$.
	\begin{enumerate*}
		\item[(a)] Tính giá trị của biểu thức $A$.
		\item[(b)] Đặt thêm dấu ngoặc để biểu thức $A$ có giá trị là $48.8$.
	\end{enumerate*}
\end{baitoan}

\begin{baitoan}[\cite{Tuyen_Toan_7}, \textbf{40.}, p. 12]
	Ông Phú gửi tiết kiệm $100$ triệu đồng tại 1 ngân hàng với kỳ hạn 1 năm, lãi suất $5$\%\emph{\texttt{/}}năm. Hết thời hạn 1 năm, tiền lãi gộp vào sổ tiền gửi ban đầu \& lại gửi theo thể thức cũ. Cứ như thế sau 3 năm thì số tiền cả gốc lẫn lãi là bao nhiêu?
\end{baitoan}

%------------------------------------------------------------------------------%

\subsection{Biểu Diễn Thập Phân của Số Hữu Tỷ}
``\begin{enumerate*}
	\item[\textbf{1.}] Số thập phân hữu hạn \& số thập phân vô hạn tuần hoàn:
	\begin{enumerate*}
		\item[$\bullet$] Xét phép chia $47:20 = 2.35$. Số thập phân $2.35$ chỉ có 2 chữ số sau dấu phẩy được gọi là \textit{số thập phân hữu hạn}.
		\item[$\bullet$] Xét phép chia $49:30 = 1.63333\ldots$. Trong phần thập phân của thương, chữ số $3$ xuất hiện liên tiếp mãi. Ta nói số thập phân $1.63333\ldots$ là \textit{số thập phân vô hạn tuần hoàn} có chu kỳ là $3$ \& viết gọn $1.63333\ldots = 1.6(3)$. Chu kỳ của 1 số thập phân vô hạn tuần hoàn có thể có 1 chữ số hoặc nhiều chữ số, có thể bắt đầu ngay sau dấu phẩy hoặc không bắt đầu ngay sau dấu phẩy.
	\end{enumerate*}
	\item[\textbf{2.}] \textit{Biểu diễn thập phân của số hữu tỷ}: Mỗi số hữu tỷ được biểu diễn bởi 1 số thập phân hữu hạn hoặc vô hạn tuần hoàn. Ngược lại, mỗi số thập phân hữu hạn hoặc vô hạn tuần hoàn biểu diễn 1 số hữu tỷ.
	\begin{enumerate*}
		\item[$\bullet$] Các phân số tối giản với mẫu dương mà mẫu không có ước nguyên tố khác $2$ \& $5$ thì viết được dưới dạng số thập phân hữu hạn.
		\item[$\bullet$] Các phân số tối giản với mẫu dương mà mẫu có ước nguyên tố khác $2$ \& $5$ thì được viết dưới dạng số thập phân vô hạn tuần hoàn.
	\end{enumerate*}
	\item[\textbf{3.}] $\frac{1}{9} = 0.111\ldots = 0.(1)$, $\frac{1}{99} = 0.0101\ldots = 0.(01)$, $\frac{1}{999} = 0.001001\ldots = 0.(001)$. Ta thừa nhận kết quả sau: $0.(1) = \frac{1}{9}$, $0.(01) = \frac{1}{99}$, $0.(001) = \frac{1}{999}$.'' -- \cite[\S5, pp. 12--13]{Tuyen_Toan_7}
\end{enumerate*}

Tổng quát,
\begin{align*}
	0.(\underbrace{0\ldots 0}_{\scriptsize n\mbox{ số } 0}1) = \frac{1}{\underbrace{9\ldots 9}_{\scriptsize n + 1\mbox{ số } 9}},\ \forall n\in\mathbb{N}.
\end{align*}

\begin{proof}[Chứng minh]
	$\forall n\in\mathbb{N}$, đặt $a_n = 0.(\underbrace{0\ldots 0}_{\scriptsize n\mbox{ số } 0}1)$. Có $10^{n+1}a_n = 1.(\underbrace{0\ldots 0}_{\scriptsize n\mbox{ số } 0}1) = 1 + 0.(\underbrace{0\ldots 0}_{\scriptsize n\mbox{ số } 0}1) = 1 + a_n\Leftrightarrow(10^{n+1} - 1)a_n = 1\Leftrightarrow a_n = \dfrac{1}{10^{n+1} - 1} = \dfrac{1}{1\underbrace{0\ldots 0}_{\scriptsize n + 1\mbox{ số } 0}}= \dfrac{1}{\underbrace{9\ldots 9}_{\scriptsize n + 1\mbox{ số } 9}}$.
\end{proof}

\begin{baitoan}[\cite{Tuyen_Toan_7}, Ví dụ 12, p. 13]
	Viết các số thập phân sau dưới dạng phân số tối giản:
	\begin{enumerate*}
		\item[(a)] $0.555$;
		\item[(b)] $0.555\ldots$.
	\end{enumerate*}
\end{baitoan}

\begin{luuy}[Số thập phân vô hạn tuần hoàn đơn]
	``Các số $0.555\ldots$, $4.272727\ldots$ là những số thập phân vô hạn tuần hoàn có chu kỳ ngay sau dấu phẩy gọi là \emph{số thập phân vô hạn tuần hoàn đơn}.'' -- \cite[p. 13]{Tuyen_Toan_7}
\end{luuy}

\begin{baitoan}[\cite{Tuyen_Toan_7}, Ví dụ 13, p. 13]
	Viết số thập phân $0.25454\ldots$ dưới dạng phân số tối giản.
\end{baitoan}

\begin{luuy}[Số thập phân vô hạn tuần hoàn tạp]
	Số $0.25454\ldots$ có chu kỳ không bắt đầu ngay sau dấu phẩy gọi là \emph{số thập phân vô hạn tuần hoàn tạp}. Để viết các số thập phân vô hạn tuần hoàn tạp dưới dạng phân số tối giản, trước hết phải đưa chúng về dạng vô hạn tuần hoàn đơn.'' -- \cite[p. 13]{Tuyen_Toan_7}
\end{luuy}

\begin{baitoan}[\cite{Tuyen_Toan_7}, Ví dụ 14, p. 13]
	Tính: $1.(6)\cdot 2.(3):0.(7)$.
\end{baitoan}
\noindent\textit{Cách giải}: ``Trước hết viết các số thập phân vô hạn tuần hoàn dưới dạng phân số tối giản rồi làm các phép tính đối với phân số.'' -- \cite[p. 13]{Tuyen_Toan_7}

\begin{baitoan}[\cite{Tuyen_Toan_7}, \textbf{41.}, p. 14]
	Không làm phép chia, cho biết trong các phân số sau, phân số nào viết được dưới dạng số thập phân hữu hạn, phân số nào viết được dưới dạng số thập phân vô hạn tuần hoàn? $\frac{7}{32}$, $\frac{2}{35}$, $\frac{6}{75}$, $\frac{-35}{42}$, $\frac{3^2}{11^2 - 1}$.
\end{baitoan}

\begin{baitoan}[\cite{Tuyen_Toan_7}, \textbf{42.}, p. 14]
	Viết các số thập phân vô hạn tuần hoàn đơn dưới dạng phân số tối giản:
	\begin{enumerate*}
		\item[(a)] $0.333\ldots$;
		\item[(b)] $0.454545\ldots$;
		\item[(c)] $0.162162\ldots$;
		\item[(d)] $5.272727\ldots$.
	\end{enumerate*}
\end{baitoan}

\begin{baitoan}[\cite{Tuyen_Toan_7}, \textbf{43.}, p. 14]
	Viết các số thập phân vô hạn tuần hoàn tạp dưới dạng phân số tối giản:
	\begin{enumerate*}
		\item[(a)] $0.7666\ldots$;
		\item[(b)] $0.507575\ldots$;
		\item[(c)] $1.2148148\ldots$.
	\end{enumerate*}
\end{baitoan}

\begin{baitoan}[\cite{Tuyen_Toan_7}, \textbf{44.}, p. 14]
	Tính:\\
	\begin{enumerate*}
		\item[(a)] $0.2777\ldots 0.3555\ldots$;
		\item[(b)] $1.5454\ldots - 0.8181\ldots - 0.75$;
		\item[(c)] $1:10.2(6):0.41(6)\cdot 0.42(7)$.
	\end{enumerate*}
\end{baitoan}

\begin{baitoan}[\cite{Tuyen_Toan_7}, \textbf{45.}, p. 14]
	Cho $x$ \& $y$ là các số nguyên tố. Tìm $x$ \& $y$ để các phân số sau viết được dưới dạng số thập phân hữu hạn:
	\begin{enumerate*}
		\item[(a)] $P = \frac{x}{3\cdot 5\cdot y}$;
		\item[(b)] $Q = \frac{15x}{14y}$.
	\end{enumerate*}
\end{baitoan}

\begin{baitoan}[\cite{Tuyen_Toan_7}, \textbf{46.}, p. 14]
	Tìm 1 phân số dương tối giản nhỏ hơn $1$, biết khi chia tử cho mẫu ta được 1 số thập phân vô hạn tuần hoàn đơn chu kỳ có $3$ chữ số \& phân số này bằng lập phương của 1 phân số khác.
\end{baitoan}

%------------------------------------------------------------------------------%

\subsection{Phần Nguyên, Phần Lẻ của 1 Số Hữu Tỷ}
``\begin{enumerate*}
	\item[\textbf{1.}] \textit{Phần nguyên} của $x\in\mathbb{Q}$, ký hiệu $\lfloor x\rfloor$ là số nguyên lớn nhất không vượt quá $x$. Như vậy $\lfloor x\rfloor$ là 1 số nguyên sao cho: $\lfloor x\rfloor\le x < \lfloor x\rfloor + 1$. Khi $x\in\mathbb{Z}$ thì $\lfloor x\rfloor = x$. E.g., $\lfloor 8.9\rfloor = 8$, $\lfloor-3.2\rfloor = -4$, $\lfloor-2\rfloor = -2$.
	\item[\textbf{2.}] \textit{Phần lẻ} của 1 số hữu tỷ $x\in\mathbb{Q}$, ký hiệu $\{x\}$ \& $\{x\} = x - \lfloor x\rfloor$. Như vậy $\{x\}$ là 1 số hữu tỷ sao cho $0\le\{x\} < 1$. Khi $x\in\mathbb{Z}$ thì $\{x\} = 0$. E.g., $\{8.9\} = 8.9 - \lfloor 8.9\rfloor = 8.9 - 8 = 0.9$, $\{-3.2\} = -3.2 - \lfloor-3.2\rfloor = -3.2 -(-4) = 0.8$, $\{-2\} = -2 - \lfloor-2\rfloor = -2 -(-2) = 0$.
	\item[\textbf{3.}] Tính chất quan trọng để tìm phần nguyên của 1 số hữu tỷ: Nếu số hữu tỷ $x$ bị ``kẹp giữa'' 2 số nguyên liên tiếp thì $\lfloor x\rfloor$ bằng số nhỏ hơn trong 2 số đó. Nếu $a\le x < a + 1$, $a\in\mathbb{Z}$ thì $\lfloor x\rfloor = a$.
	\item[\textbf{4.}] Với khái niệm phần nguyên, ta có thể trình bày nguyên lý Dirichlet 1 cách tổng quát như sau: Nếu nhốt $a$ con thỏ vào $b$ chiếc chuồng mà phép chia $\frac{a}{b}$ còn dư thì tồn tại 1 chuồng nhốt $\lfloor\frac{a}{b}\rfloor + 1$ con thỏ trở lên.'' -- \cite[\S6, pp. 14--15]{Tuyen_Toan_7}
\end{enumerate*}

\begin{baitoan}[\cite{Tuyen_Toan_7}, Ví dụ 15, p. 15]
	Tìm $\lfloor x\rfloor$ biết $x < 9 < x + 0.4$.
\end{baitoan}

\begin{baitoan}[\cite{Tuyen_Toan_7}, Ví dụ 16, p. 15]
	Tìm $\lfloor x\rfloor$ biết $\lfloor\frac{x}{3}\rfloor = -5$.
\end{baitoan}

\begin{baitoan}[\cite{Tuyen_Toan_7}, Ví dụ 17, p. 15]
	Tích $A = 1000! = \prod_{i=1}^{1000} = 1\cdot 2\cdot 3\cdots 1000$ có bao nhiêu thừa số $3$ khi phân tích ra thừa số nguyên tố?
\end{baitoan}
``Tổng quát, số thừa số nguyên tố $p$ khi phân tích $A = n! = \prod_{i=1}^n i = 1\cdot 2\cdot 3\cdots n$ ra thừa số nguyên tố là: $\sum_{i=1}^k \lfloor\frac{n}{p^i}\rfloor = \lfloor\frac{n}{p}\rfloor + \lfloor\frac{n}{p^2}\rfloor + \cdots \lfloor\frac{n}{p^k}\rfloor$ với $k$ là số mũ lớn nhất sao cho $p^k\le n$.'' -- \cite[p. 15]{Tuyen_Toan_7} 

\begin{baitoan}[\cite{Tuyen_Toan_7}, \textbf{47.}, p. 16]
	Tìm phần nguyên, phần lẻ của $x\in\mathbb{Q}$ biết:
	\begin{enumerate*}
		\item[(a)] $x = -3$;
		\item[(b)] $x = 6.1$;
		\item[(c)] $x = -\frac{6}{5}$;
		\item[(d)] $x = \frac{1}{8}$.
	\end{enumerate*}
\end{baitoan}

\begin{baitoan}[\cite{Tuyen_Toan_7}, \textbf{48.}, p. 16]
	Tìm $\lfloor x\rfloor$ của $x\in\mathbb{Q}$ biết:
	\begin{enumerate*}
		\item[(a)] $13 < x < 13.4$;
		\item[(b)] $-9.2 < x < -9$.
	\end{enumerate*}
\end{baitoan}

\begin{baitoan}[\cite{Tuyen_Toan_7}, \textbf{49.}, p. 16]
	So sánh phần nguyên của các số hữu tỷ sau:
	\begin{enumerate*}
		\item[(a)] $x = \frac{25}{8}$, $y  = \frac{24}{6}$, $z = \frac{23}{7}$;
		\item[(b)] $x = -3\frac{1}{9}$, $y = -3\frac{8}{9}$, $z = -4$.
	\end{enumerate*}
\end{baitoan}

\begin{baitoan}[\cite{Tuyen_Toan_7}, \textbf{50.}, p. 16]
	Cho $x\in\mathbb{Z}$, $y\in\mathbb{Q}$. So sánh $\{x\}$ \& $\{y\}$.
\end{baitoan}

\begin{baitoan}[\cite{Tuyen_Toan_7}, \textbf{51.}, p. 16]
	Tìm $\lfloor x\rfloor$ của $x\in\mathbb{Q}$ biết:
	\begin{enumerate*}
		\item[(a)] $x - 0.7 < 8 < x$;
		\item[(b)] $x < -5 < x + \frac{1}{3}$.
	\end{enumerate*}
\end{baitoan}

\begin{baitoan}[\cite{Tuyen_Toan_7}, \textbf{52.}, p. 16]
	Tính:
	\begin{enumerate*}
		\item[(a)] $\lfloor\frac{12}{2}\rfloor + \lfloor\frac{13}{2}\rfloor$;
		\item[(b)] $\lfloor\frac{12}{3}\rfloor + \lfloor\frac{13}{3}\rfloor + \lfloor\frac{14}{3}\rfloor$;
		\item[(c)] $\lfloor\frac{-12}{3}\rfloor + \lfloor\frac{-13}{3}\rfloor + \lfloor\frac{-14}{3}\rfloor$.
	\end{enumerate*}
\end{baitoan}

\begin{baitoan}[\cite{Tuyen_Toan_7}, \textbf{53.}, p. 16]
	Cho $A = \lfloor\frac{n}{2}\rfloor + \lfloor\frac{n + 1}{2}\rfloor$, $B = \lfloor\frac{n}{3}\rfloor + \lfloor\frac{n + 1}{3}\rfloor + \lfloor\frac{n + 2}{3}\rfloor$ với giá trị nào của $n\in\mathbb{Z}$ thì:
	\begin{enumerate*}
		\item[(a)] $A\divby 2$;
		\item[(b)] $B\divby 3$.
	\end{enumerate*}
\end{baitoan}

\begin{baitoan}[\cite{Tuyen_Toan_7}, \textbf{54.}, p. 16]
	Tìm $x$ biết:
	\begin{enumerate*}
		\item[(a)] $\lfloor 2x\rfloor = -1$;
		\item[(b)] $\lfloor x + 0.4\rfloor = 3$;
		\item[(c)] $\lfloor\frac{2}{3}x - 5\rfloor = 3$.
	\end{enumerate*}
\end{baitoan}

\begin{baitoan}[\cite{Tuyen_Toan_7}, \textbf{55.}, p. 16]
	Chứng ming: $\lfloor x + y\rfloor = \lfloor x\rfloor + \lfloor y\rfloor$, $\forall x,y\in\mathbb{Z}$.
\end{baitoan}

\begin{baitoan}[\cite{Tuyen_Toan_7}, \textbf{56.}, p. 16]
	Tìm $x$ biết:
	\begin{enumerate*}
		\item[(a)] $\lfloor 3x - 4\rfloor = x$;
		\item[(b)] $\lfloor x + 8\rfloor = -3x$;
		\item[(c)] $\lfloor 5x - 3\rfloor = 2x + 1$.
	\end{enumerate*}
\end{baitoan}

\begin{baitoan}[\cite{Tuyen_Toan_7}, \textbf{57.}, p. 16]
	Tích $C = \prod_{i=201}^{600}$ có bao nhiêu thừa số $3$ khi phân tích ra thừa số nguyên tố?
\end{baitoan}

\begin{baitoan}[\cite{Tuyen_Toan_7}, \textbf{58.}, p. 16]
	Số $300!$ có tận cùng bằng bao nhiêu chữ số $0$?
\end{baitoan}

\begin{baitoan}[\cite{Tuyen_Toan_7}, \textbf{59.}, p. 16]
	1 lớp học có $44$ học sinh làm bài kiểm tra Toán. Điểm số là 1 số tự nhiên từ $6$ đến $10$. Biết cả lớp có $6$ học sinh được điểm $10$. Chứng minh: Ít nhất cũng có $10$ học sinh có cùng 1 loại điểm.
\end{baitoan}

\begin{baitoan}[\cite{Tuyen_Toan_7}, \textbf{60.}, p. 16]
	1 tổ $11$ học sinh thảo luận về học tập. Có 1 học sinh phát biểu $4$ lần, các học sinh khác đều phát biểu nhưng số lần phát biểu ít hơn. Chứng minh: Ít nhất cũng có $4$ học sinh có số lần phát biểu như nhau.
\end{baitoan}

\begin{baitoan}[\cite{Tuyen_Toan_7}, \textbf{61.}, p. 16]
	Có $50$ quyển vở chia cho $11$ học sinh. Chứng minh:
	\begin{enumerate*}
		\item[(a)] Ít nhất cũng có 1 học sinh được $5$ quyển trở lên.
		\item[(b)] Với mọi cách chia (kể cả trường hợp có học sinh không được quyển nào) bao giờ cũng có ít nhất $2$ học sinh được số quyển vở như nhau.
	\end{enumerate*}
\end{baitoan}

%------------------------------------------------------------------------------%

\subsection{Miscellaneous}
\textsf{\textbf{Nội dung.} Định nghĩa số hữu tỷ; các phép tính cộng, trừ, nhân, chia, \& lũy thừa với số mũ tự nhiên của 1 số hữu tỷ; thứ tự thực hiện các phép tính; quy tắc chuyển vế; số thập phân hữu hạn \& số thập phân vô hạn tuần hoàn.}

\begin{baitoan}[\cite{Tuyen_Toan_7}, Ví dụ 18, p. 17]
	Cho biểu thức $A = \left(\frac{1}{2.5  -1}\right)^2 - \left(\frac{1}{3\frac{1}{2} - 1}\right)^2$. Tính giá trị của $A$ rồi viết kết quả dưới dạng lũy thừa với số mũ tự nhiên của 1 số hữu tỷ.
\end{baitoan}

\begin{baitoan}[\cite{Tuyen_Toan_7}, Ví dụ 19, p. 17]
	Tìm $x$ biết: $\left(1 - \frac{3}{10} - x\right):\left(\frac{19}{10} - 1 - \frac{2}{5}\right) + \frac{4}{5} = 1$.
\end{baitoan}

\begin{baitoan}[\cite{Tuyen_Toan_7}, \textbf{62.}, p. 18]
	Tính:
	\begin{enumerate*}
		\item[(a)] $9.6\cdot\left(\frac{3}{4} - \frac{5}{6}\right)^2$;
		\item[(b)] $6\cdot\left(-\frac{2}{3}\right) + 12\left(-\frac{2}{3}\right)^2 + 18\cdot\left(-\frac{2}{3}\right)^3$.
	\end{enumerate*}
\end{baitoan}

\begin{baitoan}[\cite{Tuyen_Toan_7}, \textbf{63.}, p. 18]
	Cho $A = \frac{17}{24}\cdot 9\frac{1}{2} - 3\frac{1}{4}\cdot\frac{17}{24}$, $B = 3\frac{1}{2}\cdot 2\frac{13}{36} + 2\frac{13}{36}\cdot 2\frac{3}{4}$.
	\begin{enumerate*}
		\item[(a)] Tính giá trị của thương $\frac{A}{B}$;
		\item[(b)] Tính $\left(\frac{A}{B} - \frac{1}{5}\right)^{-2}$.
	\end{enumerate*}
\end{baitoan}

\begin{baitoan}[\cite{Tuyen_Toan_7}, \textbf{64.}, p. 18]
	Tính: $\dfrac{3^6\cdot 45^4 - 15^{13}\cdot\left(\frac{1}{5}\right)^9}{27^4\cdot 25^3 + 45^6}$.
\end{baitoan}
	
\begin{baitoan}[\cite{Tuyen_Toan_7}, \textbf{65.}, p. 18]
	Tính: $\dfrac{\left(\frac{2}{5}\right)^7\cdot 5^7 + \left(\frac{9}{4}\right)^3:\left(\frac{3}{16}\right)^3}{2^7\cdot 5^2 + 512}$.
\end{baitoan}

\begin{baitoan}[\cite{Tuyen_Toan_7}, \textbf{66.}, p. 18]
	Tìm $x\in\mathbb{Q}$ biết: $30\left(x - \frac{7}{12}\right) - 24x = 100 + 6\left(x - \frac{3}{4}\right)$.
\end{baitoan}

\begin{baitoan}[\cite{Tuyen_Toan_7}, \textbf{67.}, p. 18]
	Tìm $x$ sao cho:
	\begin{enumerate*}
		\item[(a)] $(x - 3)(x + 4) > 0$;
		\item[(b)] $(x + 5)(x - 1) < 0$.
	\end{enumerate*}
\end{baitoan}

\begin{baitoan}[\cite{Tuyen_Toan_7}, \textbf{68.}, p. 18]
	Muốn biết ngày $\ldots$ tháng $\ldots$ năm $\ldots$ nào đó là ngày thứ mấy, làm theo 2 bước sau:
	\begin{itemize}
		\item \textit{Bước 1}: Tìm $S$ theo công thức: $S = x - 1 + \lfloor\frac{x - 1}{4}\rfloor -  \lfloor\frac{x - 1}{100}\rfloor + \lfloor\frac{x - 1}{400}\rfloor + C$, trong đó $x$ là năm dương lịch, $C$ là số ngày từ mồng 1 tháng giêng năm đó đến ngày cần tìm (kể cả ngày đầu tiên).
		\item \textit{Bước 2}: Tìm số dư trong phép chia $\frac{S}{7}$ rồi đối chiều với bảng sau:
		
		\begin{table}[H]
			\centering
			\begin{tabular}{|c|c|c|c|c|c|c|c|}
				\hline
				Thứ & CN & 2 & 3 & 4 & 5 & 6 & 7 \\
				\hline
				Số dư & 0 & 1 & 2 & 3 & 4 & 5 & 6 \\
				\hline
			\end{tabular}
		\end{table}
	\end{itemize}
	\begin{enumerate*}
		\item[(a)] Ngày \emph{01\texttt{/}01\texttt{/}2001} là ngày thứ mấy?
		\item[(b)] Ngày \emph{03\texttt{/}02\texttt{/}2030} là ngày thứ mấy?
	\end{enumerate*}
\end{baitoan}



%------------------------------------------------------------------------------%

\printbibliography[heading=bibintoc]
	
\end{document}