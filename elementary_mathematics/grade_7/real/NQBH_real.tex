\documentclass{article}
\usepackage[backend=biber,natbib=true,style=authoryear]{biblatex}
\addbibresource{/home/nqbh/reference/bib.bib}
\usepackage[utf8]{vietnam}
\usepackage{tocloft}
\renewcommand{\cftsecleader}{\cftdotfill{\cftdotsep}}
\usepackage[colorlinks=true,linkcolor=blue,urlcolor=red,citecolor=magenta]{hyperref}
\usepackage{amsmath,amssymb,amsthm,mathtools,float,graphicx,algpseudocode,algorithm,tcolorbox,tikz,tkz-tab,subcaption}
\DeclareMathOperator{\arccot}{arccot}
\usepackage[inline]{enumitem}
\allowdisplaybreaks
\numberwithin{equation}{section}
\newtheorem{assumption}{Assumption}[section]
\newtheorem{nhanxet}{Nhận xét}[section]
\newtheorem{conjecture}{Conjecture}[section]
\newtheorem{corollary}{Corollary}[section]
\newtheorem{hequa}{Hệ quả}[section]
\newtheorem{definition}{Definition}[section]
\newtheorem{dinhnghia}{Định nghĩa}[section]
\newtheorem{example}{Example}[section]
\newtheorem{vidu}{Ví dụ}[section]
\newtheorem{lemma}{Lemma}[section]
\newtheorem{notation}{Notation}[section]
\newtheorem{principle}{Principle}[section]
\newtheorem{problem}{Problem}[section]
\newtheorem{baitoan}{Bài toán}
\newtheorem{proposition}{Proposition}[section]
\newtheorem{menhde}{Mệnh đề}[section]
\newtheorem{question}{Question}[section]
\newtheorem{cauhoi}{Câu hỏi}[section]
\newtheorem{quytac}{Quy tắc}
\newtheorem{remark}{Remark}[section]
\newtheorem{luuy}{Lưu ý}[section]
\newtheorem{theorem}{Theorem}[section]
\newtheorem{tiende}{Tiên đề}[section]
\newtheorem{dinhly}{Định lý}[section]
\usepackage[left=0.5in,right=0.5in,top=1.5cm,bottom=1.5cm]{geometry}
\usepackage{fancyhdr}
\pagestyle{fancy}
\fancyhf{}
\lhead{\small Sect.~\thesection}
\rhead{\small\nouppercase{\leftmark}}
\renewcommand{\subsectionmark}[1]{\markboth{#1}{}}
\cfoot{\thepage}
\def\labelitemii{$\circ$}
\DeclareRobustCommand{\divby}{%
	\mathrel{\vbox{\baselineskip.65ex\lineskiplimit0pt\hbox{.}\hbox{.}\hbox{.}}}%
}

\title{Real -- Số Thực $\mathbb{R}$}
\author{Nguyễn Quản Bá Hồng\footnote{Independent Researcher, Ben Tre City, Vietnam\\e-mail: \texttt{nguyenquanbahong@gmail.com}; website: \url{https://nqbh.github.io}.}}
\date{\today}

\begin{document}
\maketitle
\begin{abstract}
	\textsc{[en]} This text is a collection of problems, from easy to advanced, about real. This text is also a supplementary material for my lecture note on Elementary Mathematics grade 7, which is stored \& downloadable at the following link: \href{https://github.com/NQBH/hobby/blob/master/elementary_mathematics/grade_7/NQBH_elementary_mathematics_grade_7.pdf}{GitHub\texttt{/}NQBH\texttt{/}hobby\texttt{/}elementary mathematics\texttt{/}grade 7\texttt{/}lecture}\footnote{\textsc{url}: \url{https://github.com/NQBH/hobby/blob/master/elementary_mathematics/grade_7/NQBH_elementary_mathematics_grade_7.pdf}.}. The latest version of this text has been stored \& downloadable at the following link: \href{https://github.com/NQBH/hobby/blob/master/elementary_mathematics/grade_7/real/NQBH_real.pdf}{GitHub\texttt{/}NQBH\texttt{/}hobby\texttt{/}elementary mathematics\texttt{/}grade 7\texttt{/}real $\mathbb{R}$}\footnote{\textsc{url}: \url{https://github.com/NQBH/hobby/blob/master/elementary_mathematics/grade_7/real/NQBH_real.pdf}.}.
	\vspace{2mm}
	
	\textsc{[vi]} Tài liệu này là 1 bộ sưu tập các bài tập chọn lọc từ cơ bản đến nâng cao về phân thức đại số \& phân thức đại số thực. Tài liệu này là phần bài tập bổ sung cho tài liệu chính -- bài giảng \href{https://github.com/NQBH/hobby/blob/master/elementary_mathematics/grade_7/NQBH_elementary_mathematics_grade_7.pdf}{GitHub\texttt{/}NQBH\texttt{/}hobby\texttt{/}elementary mathematics\texttt{/}grade 7\texttt{/}lecture} của tác giả viết cho Toán Sơ Cấp lớp 7. Phiên bản mới nhất của tài liệu này được lưu trữ \& có thể tải xuống ở link sau: \href{https://github.com/NQBH/hobby/blob/master/elementary_mathematics/grade_7/real/NQBH_real.pdf}{GitHub\texttt{/}NQBH\texttt{/}hobby\texttt{/}elementary mathematics\texttt{/}grade 7\texttt{/}real $\mathbb{R}$}.
\end{abstract}
\setcounter{secnumdepth}{4}
\setcounter{tocdepth}{3}
\tableofcontents

%------------------------------------------------------------------------------%

\section{Problem}

\subsection{Số Vô Tỷ. Căn Bậc 2 Số Học}

\begin{dinhnghia}[Số vô tỷ]
	``\emph{Số vô tỷ} là số viết được dưới dạng số thập phân vô hạn không tuần hoàn.
\end{dinhnghia}

\begin{dinhnghia}[Căn bậc 2 số học]
	\emph{Căn bậc 2 số học} của số $a\ge 0$ là số $x\ge 0$ sao cho $x^2 = a$.
\end{dinhnghia}
Căn bậc 2 số học của số $a$ ký hiệu là $\sqrt{2}$. Căn bậc 2 số học của $0$ là số $0$, viết là $\sqrt{0} = 0$. Nếu số nguyên dương $a$ không phải là bình phương của bất kỳ số nguyên dương nào thì $\sqrt{a}$ là 1 số vô tỷ. Như vậy $\sqrt{2},\sqrt{3},\sqrt{5},\sqrt{6},\ldots$ đều là số vô tỷ. $\sqrt{2}$ là độ dài đường chéo của hình vuông có độ dài cạnh bằng $1$.'' -- \cite[Chap. 2, \S1, p. 19]{Tuyen_Toan_7}

``Mọi số hữu tỷ đều biểu diễn được dưới dạng số thập phân hữu hạn hoặc vô hạn tuần hoàn. Ngược lại, mỗi số thập phân hữu hạn hoặc vô hạn tuần hoàn đều biểu diễn 1 số hữu tỷ. Số vô tỷ là số viết được dưới dạng số thập phân vô hạn không tuần hoàn. Tập hợp các số thực $\mathbb{R}$ bao gồm tập hợp số hữu tỷ $\mathbb{Q}$ \& tập hợp số vô tỷ $\mathbb{I}$. Cho số $a$ không âm. Căn bậc 2 của $a$ là số $x$ mà $x^2 = a$. Căn bậc 2 không âm của $a$ ký hiệu là $\sqrt{a}$. Nếu $n\in\mathbb{N}$ không là số chính phương thì $\sqrt{n}$ là số vô tỷ.'' -- \cite[\S7]{Binh_Toan_7_tap_1}

\begin{baitoan}[\cite{Tuyen_Toan_7}, Ví dụ 20, p. 19]
	Tính độ dài mỗi cạnh của 1 sân hình vuông có diện tích lần lượt là $\rm16m^2$, $\rm6.25m^2$, $\rm6m^2$. Trong mỗi trường hợp, cho biết độ dài mỗi cạnh được biểu diễn bằng số hữu tỷ hay vô tỷ?\hfill\textsf{Ans:} (a) $4\in\mathbb{Q}$. (b) $2.5\in\mathbb{Q}$. (c) $\sqrt{6}\in\mathbb{R}\backslash\mathbb{Q}$.
\end{baitoan}

\begin{baitoan}[\cite{Tuyen_Toan_7}, Ví dụ 21, p. 19]
	Cho $A = \frac{5}{\sqrt{x} - 3}$. Tìm số chính phương $x$ để biểu thức $A$ có giá trị nguyên.\\\mbox{}\hfill\textsf{Ans:} $4$, $16$, $64$.
\end{baitoan}

\begin{baitoan}[\cite{Tuyen_Toan_7}, \textbf{69.}, p. 19]
	Tính:
	\begin{enumerate*}
		\item[(a)] $\sqrt{(-5)^2} + \sqrt{5^2} - \sqrt{(-3)^2} - \sqrt{3^2}$;
		\item[(b)] $\left[\sqrt{4^2} + \sqrt{(-4)^2}\right]\cdot\sqrt{\frac{1}{4^3}} - \sqrt{\frac{1}{3^4}}$.
	\end{enumerate*}
\end{baitoan}

\begin{baitoan}[\cite{Tuyen_Toan_7}, \textbf{70.}, pp. 19--20]
	Tìm $x > 0$ biết:
	\begin{enumerate*}
		\item[(a)] $4x^2 - 1 = 0$;
		\item[(b)] $2x^2 + 0.82 = 1$.
	\end{enumerate*}
\end{baitoan}

\begin{baitoan}[\cite{Tuyen_Toan_7}, \textbf{71.}, p. 20]
	Tìm $x\ge 0$ biết:
	\begin{enumerate*}
		\item[(a)] $7 - \sqrt{x} = 0$;
		\item[(b)] $3\sqrt{x} + 1 = 40$;
		\item[(c)] $\frac{5}{12}\sqrt{x} - \frac{1}{6} = \frac{1}{3}$;
		\item[(d)] $\sqrt{x + 1} + 2 = 0$.
	\end{enumerate*}
\end{baitoan}

\begin{baitoan}[\cite{Tuyen_Toan_7}, \textbf{72.}, p. 20]
	Cho $M = \sqrt{\sqrt{x} - 1}{2}$. Tìm số chính phương $x < 50$ để $M$ có giá trị nguyên.
\end{baitoan}

\begin{baitoan}[\cite{Tuyen_Toan_7}, \textbf{73.}, p. 20]
	Cho $N = \sqrt{9}{\sqrt{x} - 5}$. Tìm số chính phương $x$ để $N$ có giá trị nguyên.
\end{baitoan}

\begin{baitoan}[\cite{Tuyen_Toan_7}, \textbf{74.}, p. 20]
	Bên trong 1 hình vuông cạnh $5$ có $76$ điểm. Chứng minh: Tồn tại $4$ điểm trong các điểm đó thuộc 1 hình tròn có bán kính là $\frac{3}{4}$.
\end{baitoan}

\begin{baitoan}[\cite{Binh_Toan_7_tap_1}, \S7, Ví dụ 11]
	Chứng minh:
	\begin{enumerate*}
		\item[(a)] $\sqrt{15}$ là số vô tỷ.
		\item[(b)] Nếu số tự nhiên $a$ không phải là số chính phương thì $\sqrt{a}$ là số vô tỷ.
	\end{enumerate*}
\end{baitoan}

\begin{baitoan}[\cite{Binh_Toan_7_tap_1}, \S7, \textbf{69.}]
	Tìm $x$ biết:
	\begin{enumerate*}
		\item[(a)] $x^2 = 81$.
		\item[(b)] $(x - 1)^2 = \frac{9}{16}$.
		\item[(c)] $x - 2\sqrt{x} = 0$.
		\item[(d)] $x = \sqrt{x}$.
	\end{enumerate*}
\end{baitoan}

\begin{baitoan}[\cite{Binh_Toan_7_tap_1}, \S7, \textbf{70.}]
	Cho $A = \frac{\sqrt{x} + 1}{\sqrt{x} - 1}$. Chứng minh với $x = \frac{16}{9}$ \& $x = \frac{25}{9}$ thì $A$ có giá trị là số nguyên.
\end{baitoan}

\begin{baitoan}[\cite{Binh_Toan_7_tap_1}, \S7, \textbf{71.}]
	Cho $A = \frac{\sqrt{x} + 1}{\sqrt{x} - 3}$. Tìm số nguyên $x$ để $A$ có giá trị là 1 số nguyên.
\end{baitoan}

\begin{baitoan}[\cite{Binh_Toan_7_tap_1}, \S7, \textbf{72.}]
	Chứng minh:
	\begin{enumerate*}
		\item[(a)] $\sqrt{2}$ là số vô tỷ.
		\item[(b)] $5 - \sqrt{2}$ là số vô tỷ.
	\end{enumerate*}
\end{baitoan}

\begin{baitoan}[\cite{Binh_Toan_7_tap_1}, \S7, \textbf{73.}]
	\begin{enumerate*}
		\item[(a)] Có 2 số vô tỷ nào mà tích là 1 số hữu tỷ hay không?
		\item[(b)] Có 2 số vô tỷ dương nào mà tổng là 1 số hữu tỷ hay không?
	\end{enumerate*}
\end{baitoan}

\begin{baitoan}[\cite{Binh_Toan_7_tap_1}, \S7, \textbf{74.}]
	Ký hiệu $\lfloor x\rfloor$ là số nguyên lớn nhất không vượt quá $x$. Tính giá trị của tổng: $\sum_{i=1}^{35} \lfloor\sqrt{i}\rfloor = \lfloor\sqrt{1}\rfloor + \lfloor\sqrt{2}\rfloor + \cdots + \lfloor\sqrt{35}\rfloor$.
\end{baitoan}

\begin{baitoan}[\cite{Binh_Toan_7_tap_1}, \S7, \textbf{75.}]
	Cho $a,b\in\mathbb{R}$ sao cho các tập hợp $\{a^2 + a;b\}$ \& $\{b^2 + b;b\}$ bằng nhau. Chứng minh $a = b$.
\end{baitoan}

%------------------------------------------------------------------------------%

\subsection{Tập Hợp $\mathbb{R}$ Các Số Thực}
``\begin{enumerate*}
	\item[\textbf{1.}] Số hữu tỷ \& số vô tỷ được gọi chung là \textit{số thực}. Tập hợp các số thực ký hiệu là $\mathbb{R}$. Mỗi số thực được biểu diễn bởi 1 điểm trên trục số. Ngược lại mỗi điểm trên trục số biểu diễn 1 số thực.
	\item[\textbf{2.}] \textit{Số đối của 1 số thực}: Trên trục số, 2 số thực phân biệt có điểm biểu diễn nằm về 2 phía gốc $O$ \& cách đều điểm $O$ được gọi là \textit{2 số đối nhau}. Số đối của số thực $a$ ký hiệu là $-a$. Số đối của $0$ là $0$.
	\item[\textbf{3.}] \textit{So sánh 2 số thực}: Có thể so sánh 2 số thực tương tự như so sánh 2 số hữu tỷ viết dưới dạng số thập phân.
	\begin{enumerate*}
		\item[$\bullet$] Số thực lớn hơn $0$ gọi là \textit{số thực dương}.
		\item[$\bullet$] Số thực nhỏ hơn $0$ gọi là \textit{số thực âm}.
		\item[$\bullet$] Nếu $a < b$ \& $b < c$ thì $a < c$.
		\item[$\bullet$] Đặc biệt nếu $0 < a < b$ thì $\sqrt{a} < \sqrt{b}$.
	\end{enumerate*}
	\item[\textbf{4.}] Trong tập hợp các số thực cũng có các phép tính cộng, trừ, nhân, chia, lũy thừa với số mũ tự nhiên với các tính chất tương tự như các phép tính trong tập hợp các số hữu tỷ. Thứ tự thực hiện các phép tính, quy tắc chuyển vế, quy tắc dấu ngoặc trong tập hợp số thực cũng giống như trong tập hợp số hữu tỷ.'' -- \cite[Chap. 2, \S2, p. 20]{Tuyen_Toan_7}
\end{enumerate*}

\begin{baitoan}[\cite{Tuyen_Toan_7}, Ví dụ 22, p. 20]
	Không dùng bảng số hoặc máy tính, so sánh $\sqrt{50 + 2}$ \& $\sqrt{50} + \sqrt{2}$.
\end{baitoan}
\noindent\textit{Hint.} Tìm các số chính phương gần với các số đã cho dưới dấu căn để loại bỏ căn thức.

\begin{proof}[Giải]
	$\sqrt{50} + \sqrt{2} > \sqrt{49} + \sqrt{1} = 7 + 1 = 8 = \sqrt{64} > \sqrt{52} = \sqrt{50 + 2}$.
\end{proof}

\begin{baitoan}[Mở rộng \cite{Tuyen_Toan_7}, Ví dụ 22, p. 20]
	So sánh $\sqrt{a + b}$ \& $\sqrt{a} + \sqrt{b}$, với $a,b\ge 0$.
\end{baitoan}

\begin{baitoan}[\cite{Tuyen_Toan_7}, \textbf{75.}, p. 21]
	Tìm $x$ biết: $6\sqrt{x} + \sqrt{12.25} = 8$.
\end{baitoan}

\begin{baitoan}[\cite{Tuyen_Toan_7}, \textbf{76.}, p. 21]
	So sánh:
	\begin{enumerate*}
		\item[(a)] $4\frac{8}{33}$ \& $3\sqrt{2}$;
		\item[(b)] $5\sqrt{(-10)^2}$ \& $10\sqrt{(-5)^2}$.
	\end{enumerate*}
\end{baitoan}

\begin{baitoan}[\cite{Tuyen_Toan_7}, \textbf{77.}, p. 21]
	Không dùng bảng số hoặc máy tính, so sánh:
	\begin{enumerate*}
		\item[(a)] $\sqrt{26} + \sqrt{17}$ \& $9$.
		\item[(b)] $\sqrt{8} - \sqrt{5}$ \& $1$;
		\item[(c)] $\sqrt{63 - 27}$ \& $\sqrt{63} - \sqrt{27}$.
	\end{enumerate*}
\end{baitoan}

\begin{baitoan}[Mở rộng \cite{Tuyen_Toan_7}, \textbf{77.}, p. 21]
	So sánh $\sqrt{a - b}$ \& $\sqrt{a} - \sqrt{b}$ với $a\ge b\ge 0$.
\end{baitoan}

\begin{baitoan}[\cite{Tuyen_Toan_7}, \textbf{78.}, p. 21]
	So sánh $A$ \& $B$ biết: $A = \sqrt{225} - \frac{1}{\sqrt{5}} - 1$, $B = \sqrt{196} - \frac{1}{\sqrt{6}}$.
\end{baitoan}

\begin{baitoan}[\cite{Tuyen_Toan_7}, \textbf{79.}, p. 21]
	Cho $P = \frac{1}{2} + \sqrt{x}$, $Q = 7 - 2\sqrt{x - 1}$. Tìm:
	\begin{enumerate*}
		\item[(a)] Giá trị nhỏ nhất của $P$;
		\item[(b)] Giá trị lớn nhất của $Q$.
	\end{enumerate*}
\end{baitoan}

\begin{baitoan}[\cite{Tuyen_Toan_7}, \textbf{80.}, p. 21]
	Xét xem các số $x$ \& $y$ có thể là số vô tỷ không nếu biết:
	\begin{enumerate*}
		\item[(a)] $x + y$ \& $x - y$ đều là số hữu tỷ;
		\item[(b)] $x + y$ \& $\frac{x}{y}$ đều là số hữu tỷ.
	\end{enumerate*}
\end{baitoan}

%------------------------------------------------------------------------------%

\subsection{Giá Trị Tuyệt Đối của  Số Thực \& Biểu Thức}
``\begin{enumerate*}
	\item[\textbf{1.}] \textit{Giá trị tuyệt đối} của $x\in\mathbb{R}$, ký hiệu $|x|$, là khoảng cách từ điểm $x$ đến gốc $O$ trên trục số.
	\item[\textbf{2.}] $\forall x\in\mathbb{R}$:
	\begin{enumerate*}
		\item[$\bullet$] Giá trị tuyệt đối của 1 số thì không âm: $|x|\ge 0$ (dấu ``$=$'' xảy ra khi \& chỉ khi $x = 0$).
		\item[$\bullet$] Giá trị tuyệt đối của 1 số thì lớn hơn hoặc bằng số đó: $|x|\ge x$ (dấu ``$=$'' xảy ra khi \& chỉ khi $x\ge 0$).
		\item[$\bullet$] Giá trị tuyệt đối của 2 số đối nhau thì bằng nhau: $|x| = |-x|$.
	\end{enumerate*}
\end{enumerate*}
Với $m > 0$ thì: $|x| < m\Leftrightarrow -m < x < m$ \& $|x| > m\Leftrightarrow((x > m)\lor(x < - m))$.'' -- \cite[Chap. 2, \S3, pp. 21--22]{Tuyen_Toan_7}

\begin{dinhnghia}[Giá trị tuyệt đối]
	``\emph{Giá trị tuyệt đối} của 1 số $a$, ký hiệu $|a|$, là số đo của khoảng cách từ điểm $a$ đến điểm gốc trên trục số.
\end{dinhnghia}
Ta thường sử dụng định nghĩa trên dưới dạng:
\begin{equation*}
	|a| = \left\{\begin{split}
		&a,&&\mbox{ nếu }a\ge 0,\\
		-&a,&&\mbox{ nếu }a < 0.
	\end{split}\right.
\end{equation*}
\textbf{Tính chất.}
\begin{enumerate*}
	\item[$\bullet$] Nếu $a = 0$ thì $|a| = 0$, nếu $a\ne 0$ thì $|a| > 0$. Ta có: \textit{Giá trị tuyệt đối của 1 số thì không âm}: $|a|\ge 0$, $\forall a\in\mathbb{R}$.
	\item[$\bullet$] Nếu $a\ge 0$ thì $|a| = a$, nếu $a < 0$ thì $|a| > a$. Ta có: \textit{Giá trị tuyệt đối của 1 số thì lớn hơn hoặc bằng số đó}: $|a|\ge a$, $\forall a\in\mathbb{R}$.
\end{enumerate*}

\begin{baitoan}[\cite{Tuyen_Toan_7}, Ví dụ 24, p. 22]
	Cho $A = |\sqrt{2} - \sqrt{3}| - |-\sqrt{7}| + |\sqrt{7} - \sqrt{3}|$, $B = |\sqrt{5} - \sqrt{7}| - |\sqrt{7} - \sqrt{6}|$. So sánh $A$ \& $B$.
\end{baitoan}	

\begin{baitoan}[\cite{Tuyen_Toan_7}, Ví dụ 25, p. 22]
	Tìm $x,y\in\mathbb{R}$ biết: $|x + y| + |y - \sqrt{11}| = 0$.
\end{baitoan}

\begin{baitoan}[\cite{Tuyen_Toan_7}, Ví dụ 26, p. 22]
	Tìm $x$ biết: $|x + \sqrt{2}| = \sqrt{3}$.
\end{baitoan}

\begin{baitoan}[\cite{Tuyen_Toan_7}, \textbf{81.}, p. 22]
	Tính:
	\begin{enumerate*}
		\item[(a)] $|-2.15| - |-3.75| + \left|\frac{4}{3} + \frac{4}{15}\right|$;
		\item[(b)] $|-\sqrt{42} - \sqrt{53}| - |\sqrt{53} - \sqrt{61}| + |\sqrt{61} - \sqrt{42}| - |-\sqrt{53}|$;
		\item[(c)] $|-150| - |100|:|-4| + |37|\cdot|-3|$.
	\end{enumerate*}
\end{baitoan}

\begin{baitoan}[\cite{Tuyen_Toan_7}, \textbf{82.}, p. 22]
	Tìm $x$ biết:
	\begin{enumerate*}
		\item[(a)] $\left|5x - \frac{3}{4}\right| + \frac{7}{4} = 3$;
		\item[(b)] $9 - |x - \sqrt{10}| = 10$.
	\end{enumerate*}
\end{baitoan}

\begin{baitoan}[\cite{Tuyen_Toan_7}, \textbf{83.}, p. 22]
	Tìm $x,y$ biết $|x - \sqrt{3}| + |y + \sqrt{5}| = 0$.
\end{baitoan}

\begin{baitoan}[Mở rộng \cite{Tuyen_Toan_7}, \textbf{83.}, p. 22]
	Tìm $x,y$ biết $|x - a| + |y - b| = 0$ với $a,b\in\mathbb{R}$ cho trước.
\end{baitoan}

\begin{baitoan}[\cite{Tuyen_Toan_7}, \textbf{84.}, p. 23]
	Tìm $x,y$ biết $\left|\frac{1}{2} - \frac{1}{3} + x\right| = -\frac{1}{4} - |y|$.
\end{baitoan}

\begin{baitoan}[\cite{Tuyen_Toan_7}, \textbf{85.}, p. 23]
	Cho $x,y$ là 2 số thực cùng dấu \& $|x| > |y|$. So sánh $x$ \& $y$.
\end{baitoan}

\begin{baitoan}[\cite{Tuyen_Toan_7}, \textbf{86.}, p. 23]
	Tìm $x$ thỏa mãn các bất đẳng thức sau:
	\begin{enumerate*}
		\item[(a)] $\left|x - \frac{5}{3}\right| < \frac{1}{3}$;
		\item[(b)] $\left|x + \frac{11}{2}\right| > |-5.5|$.
	\end{enumerate*}
\end{baitoan}

\begin{baitoan}[\cite{Tuyen_Toan_7}, \textbf{87.}, p. 23]
	Tìm giá trị nhỏ nhất của biểu thức:
	\begin{enumerate*}
		\item[(a)] $M = \left|x + \frac{15}{19}\right|$;
		\item[(b)] $N = \left|x - \frac{4}{7}\right| - \frac{1}{2}$.
	\end{enumerate*}
\end{baitoan}

\begin{baitoan}[\cite{Tuyen_Toan_7}, \textbf{88.}, p. 23]
	Tìm giá trị lớn nhất của biểu thức:
	\begin{enumerate*}
		\item[(a)] $P = -\left|\frac{5}{3} - x\right|$;
		\item[(b)] $Q = 9 - \left|x - \frac{1}{10}\right|$.
	\end{enumerate*}
\end{baitoan}

\begin{baitoan}[\cite{Tuyen_Toan_7}, \textbf{89.}, p. 23]
	Tìm giá trị nhỏ nhất của biểu thức: $A = |x - 1| + |9 - x|$.
\end{baitoan}

\begin{baitoan}[\cite{Tuyen_Toan_7}, \textbf{90.}, p. 23]
	Tìm giá trị lớn nhất của biểu thức: $B = |x - 4| + |x - 7|$.
\end{baitoan}

\subsubsection{Tính giá trị của 1 biểu thức}

\begin{baitoan}[\cite{Binh_Toan_7_tap_1}, Ví dụ 15, p. 19]
	Tính giá trị của biểu thức $A = 3x^2 - 2x + 1$ với $|x| = \frac{1}{2}$.
\end{baitoan}

\subsubsection{Rút gọn biểu thức chứa dấu giá trị tuyệt đối}

\begin{baitoan}[\cite{Binh_Toan_7_tap_1}, Ví dụ 16, p. 20]
	Rút gọn biểu thức $|a| + a$.
\end{baitoan}

\subsubsection{Tìm giá trị của biến trong đẳng thức chứa dấu giá trị tuyệt đối}

\begin{baitoan}[\cite{Binh_Toan_7_tap_1}, Ví dụ 17, p. 20]
	Tìm $x$ thỏa $2|3x - 1| + 1 = 5$.
\end{baitoan}

\begin{baitoan}[\cite{Binh_Toan_7_tap_1}, Ví dụ 18, p. 20]
	Tìm $x$ thỏa $|x - 5| - x = 3$.
\end{baitoan}

\begin{baitoan}[\cite{Binh_Toan_7_tap_1}, Ví dụ 19, p. 20]
	Tìm $x$ thỏa $|x - 2| = 2x - 3$.
\end{baitoan}

\begin{baitoan}[\cite{Binh_Toan_7_tap_1}, Ví dụ 20, p. 20]
	Với giá trị nào của $a,b$ thì đẳng thức $|a(b - 2)| = a(2 - b)$ đúng?
\end{baitoan}

\begin{baitoan}[\cite{Binh_Toan_7_tap_1}, Ví dụ 21, p. 21]
	Tìm các số $a,b\in\mathbb{R}$ thỏa $a + b = |a| - |b|$.
\end{baitoan}

\subsubsection{Tìm giá trị nhỏ nhất, giá trị lớn nhất của biểu thức chứa dấu giá trị tuyệt đối}

\begin{baitoan}[\cite{Binh_Toan_7_tap_1}, Ví dụ 22, p. 21]
	Tìm giá trị nhỏ nhất của biểu thức $A = 2|3x - 1| - 4$.
\end{baitoan}

\begin{baitoan}[\cite{Binh_Toan_7_tap_1}, Ví dụ 23, p. 21]
	Tìm giá trị lớn nhất của biểu thức $B = 10 - 4|x - 2|$.
\end{baitoan}

\begin{baitoan}[\cite{Binh_Toan_7_tap_1}, Ví dụ 24, p. 21]
	Tìm giá trị nhỏ nhất của biểu thức $C = \frac{6}{|x| - 3}$ với $x\in\mathbb{Z}$.
\end{baitoan}

\begin{baitoan}[\cite{Binh_Toan_7_tap_1}, Ví dụ 25, p. 21]
	Tìm giá trị lớn nhất của biểu thức $A = x - |x|$.
\end{baitoan}

\begin{baitoan}[\cite{Binh_Toan_7_tap_1}, \textbf{70.}, p. 22]
	Tìm tất cả các số $a$ thỏa mãn 1 trong các điều kiện sau:
	\begin{enumerate*}
		\item[(a)] $a = |a|$;
		\item[(b)] $a < |a|$;
		\item[(c)] $a > |a|$;
		\item[(d)] $|a| = -a$;
		\item[(e)] $a\le|a|$.
	\end{enumerate*}
\end{baitoan}

\begin{baitoan}[\cite{Binh_Toan_7_tap_1}, \textbf{71.}, p. 22]
	Bổ sung các điều kiện để các khẳng định sau là đúng:
	\begin{enumerate*}
		\item[(a)] $|a| = |b|\Rightarrow a = b$;
		\item[(b)] $a > b\Rightarrow|a| > |b|$.
	\end{enumerate*}
\end{baitoan}

\begin{baitoan}[\cite{Binh_Toan_7_tap_1}, \textbf{72.}, p. 22]
	Cho $|x| = |y|$, $x < 0$, $y > 0$. Trong các khẳng định sau, khẳng định nào sai?
	\begin{enumerate*}
		\item[(a)] $x^2y > 0$;
		\item[(b)] $x + y = 0$;
		\item[(c)] $xy < 0$;
		\item[(d)] $\frac{1}{x} - \frac{1}{y} = 0$;
		\item[(e)] $\frac{x}{y} + 1 = 0$.
	\end{enumerate*}
\end{baitoan}

\begin{baitoan}[\cite{Binh_Toan_7_tap_1}, \textbf{73.}, p. 22]
	Tìm giá trị của các biểu thức:
	\begin{enumerate*}
		\item[(a)] $A = 6x^3 - 3x^2 + 2|x| + 4$ với $x = -\frac{2}{3}$;
		\item[(b)] $B = 2|x| - 3|y|$ với $x = \frac{1}{2}$, $y = -3$;
		\item[(c)] $C = 2|x - 2| - 3|1 - x|$ với $x = 4$;
		\item[(d)] $D = \frac{5x^2 - 7x + 1}{3x - 1}$ với $|x| = \frac{1}{2}$.
	\end{enumerate*}
\end{baitoan}

\begin{baitoan}[\cite{Binh_Toan_7_tap_1}, \textbf{74.}, p. 22]
	Rút gọn các biểu thức:
	\begin{enumerate*}
		\item[(a)] $|a| - a$;
		\item[(b)] $|a|a$;
		\item[(c)] $|a|:a$.
	\end{enumerate*}
\end{baitoan}

\begin{baitoan}[\cite{Binh_Toan_7_tap_1}, \textbf{75.}, p. 22]
	Tìm $x$ trong các đẳng thức:
	\begin{enumerate*}
		\item[(a)] $|2x - 3| = 5$;
		\item[(b)] $|2x - 1| = |2x + 3|$;
		\item[(c)] $|x - 1| + 3x = 1$;
		\item[(d)] $|5x - 3| - x = 7$.
	\end{enumerate*}
\end{baitoan}

\begin{baitoan}[\cite{Binh_Toan_7_tap_1}, \textbf{76.}, p. 23]
	Tìm các số $a$ \& $b$ thỏa mãn 1 trong các điều kiện sau:
	\begin{enumerate*}
		\item[(a)] $a + b = |a| + |b|$:
		\item[(b)] $a + b = |b| - |a|$.
	\end{enumerate*}
\end{baitoan}

\begin{baitoan}[\cite{Binh_Toan_7_tap_1}, \textbf{77.}, p. 23]
	Có bao nhiêu cặp số nguyên $(x;y)$ thỏa mãn 1 trong các điều kiện sau:
	\begin{enumerate*}
		\item[(a)] $|x| + |y| = 20$;
		\item[(b)] $|x| + |y| < 20$.
	\end{enumerate*}
\end{baitoan}

\begin{baitoan}[\cite{Binh_Toan_7_tap_1}, \textbf{78.}, p. 23]
	Điền vào chỗ chấm các dấu $\ge,\le,=$ để các khẳng định sau đúng với mọi $a,b$. Phát biểu mỗi khẳng định đó thành 1 tính chất \& chỉ rõ khi nào xảy ra dấu đẳng thức?
	\begin{enumerate*}
		\item[(a)] $|a + b|\ldots|a| + |b|$;
		\item[(b)] $|a - b|\ldots|a| - |b|$ với $|a|\ge|b|$;
		\item[(c)] $|ab|\ldots|a||b|$;
		\item[(d)] $\left|\frac{a}{b}\right|\ldots\frac{|a|}{|b|}$.
	\end{enumerate*}
\end{baitoan}

\begin{baitoan}[\cite{Binh_Toan_7_tap_1}, \textbf{79.}, p. 23]
	Tìm giá trị nhỏ nhất của các biểu thức:
	\begin{enumerate*}
		\item[(a)] $A = 2|3x - 2| - 1$;
		\item[(b)] $B = 5|1 - 4x| - 1$;
		\item[(c)] $C = x^2 + 3|y - 2| - 1$;
		\item[(d)] $D = x + |x|$.
	\end{enumerate*}
\end{baitoan}

\begin{baitoan}[\cite{Binh_Toan_7_tap_1}, \textbf{80.}, p. 23]
	Tìm giá trị lớn nhất của các biểu thức:
	\begin{enumerate*}
		\item[(a)] $A = 5 - |2x - 1|$;
		\item[(b)] $B = \frac{1}{|x - 2| + 3}$.
	\end{enumerate*}
\end{baitoan}

\begin{baitoan}[\cite{Binh_Toan_7_tap_1}, \textbf{81.}, p. 23]
	Tìm giá trị lớn nhất của biểu thức: $C = \frac{x + 2}{|x|}$ với $x\in\mathbb{Z}$.
\end{baitoan}

%------------------------------------------------------------------------------%

\subsection{Làm Tròn Số \& Ước Lượng Kết Quả}
``\begin{enumerate*}
	\item[\textbf{1.}] Trong thực tiễn ta thường làm tròn số để thuận tiện trong việc ghi nhớ, đo đạc hay tính toán. Làm tròn số thập phân vô hạn cũng thực hiện tương tự như làm tròn số thập phân hữu hạn. Cần chú ý:
	\begin{enumerate*}
		\item[$\bullet$] Ta phải viết số định làm tròn dưới dạng số thập phân trước khi làm tròn.
		\item[$\bullet$] Khi làm tròn số ta không quan tâm đến dấu của nó.
	\end{enumerate*}
	\item[\textbf{2.}] Làm tròn số căn cứ vào độ chính xác cho trước. Khi làm tròn đến 1 hàng nào đó, kết quả làm tròn có độ chính xác bằng $\frac{1}{2}$ đơn vị của hàng làm tròn.
\end{enumerate*}
\begin{table}[H]
	\centering
	\begin{tabular}{|c|c|c|c|c|c|c|c|c|}
		\hline
		Hàng làm tròn & $\ldots$ & nghìn & trăm & chục & đơn vị & phần mười & phần trăm & $\ldots$ \\
		\hline
		Độ chính xác & $\ldots$ & $500$ & $50$ & $5$ & $0.5$ & $0.05$ & $0.005$ & $\ldots$ \\
		\hline
	\end{tabular}
\end{table}
Như vậy 1 đề toán về làm tròn số có thể hỏi theo 2 cách khác nhau:
\begin{enumerate*}
	\item[$\bullet$] Làm tròn số đến 1 hàng nào đó.
	\item[$\bullet$] Làm tròn số với độ chính xác nào đó.
\end{enumerate*}
\textbf{3.} \textit{Ước lượng kết quả các phép tính}: Khi ta không quan tâm đến tính chính xác của kết quả tính toán mà chỉ cần tìm 1 số gần sát với kết quả chính xác thì ta tìm cách ước lượng kết quả. Ta áp dụng quy tắc làm tròn số để ước lượng kết quả.'' -- \cite[Chap. 2, \S4, pp. 23--24]{Tuyen_Toan_7}

\begin{baitoan}[\cite{Tuyen_Toan_7}, Ví dụ 27, p. 24]
	Làm tròn:
	\begin{enumerate*}
		\item[(a)] Số $348.62$ đến hàng chục;
		\item[(b)] Số $-67.(506)$ đến hàng phần mười \& hàng phần trăm.
	\end{enumerate*}
\end{baitoan}

\begin{baitoan}[\cite{Tuyen_Toan_7}, Ví dụ 28, p. 24]
	Làm tròn:
	\begin{enumerate*}
		\item[(a)] Số $924578$ với độ chính xác $500$;
		\item[(b)] Số $569827$ với độ chính xác $0.5$ \& độ chính xác $0.005$.
	\end{enumerate*}
\end{baitoan}

\begin{baitoan}[\cite{Tuyen_Toan_7}, Ví dụ 29, p. 24]
	1 khu đất hình chữ nhật có kích thước $7.56$\emph{m} \& $5.173$\emph{m}. Tính diện tích khu đất đó bằng 2 cách.
	\begin{enumerate*}
		\item[$\bullet$] Cách 1: Làm tròn số trước rồi mới thực hiện các phép tính sau (làm tròn đến hàng đơn vị).
		\item[$\bullet$] Cách 2: Thực hiện các phép tính trước rồi làm tròn kết quả sau (làm tròn đến hàng đơn vị).
	\end{enumerate*}
\end{baitoan}

\begin{baitoan}[\cite{Tuyen_Toan_7}, \textbf{91.}, p. 24]
	Đầu năm $2021$ dân số nước ta nếu làm tròn đến hàng triệu thì được $98000000$ người. Hỏi dân số lúc đó:
	\begin{enumerate*}
		\item[(a)] Nhiều nhất là tới bao nhiêu người?
		\item[(b)] Ít nhất là có bao nhiêu người? 
	\end{enumerate*}
\end{baitoan}

\begin{baitoan}[\cite{Tuyen_Toan_7}, \textbf{92.}, p. 24]
	1 trận đấu bóng đá có $198 792$ khán giả. Để dễ nhớ người ta nói trên trận đấu này có khoảng $200000$ khán giả. Hỏi số liệu đó đã được làm tròn đến hàng nào?
\end{baitoan}

\begin{baitoan}[\cite{Tuyen_Toan_7}, \textbf{93.}, p. 24]
	Cho số $\pi = 3.141592\ldots$. Làm tròn số đó với độ chính xác lần lượt là $0.5$, $0.005$, $0.00005$.
\end{baitoan}

\begin{baitoan}[\cite{Tuyen_Toan_7}, \textbf{94.}, p. 24]
	Thực hiện phép chia $19:24$ rồi làm tròn kết quả với độ chính xác $0.05$.
\end{baitoan}

\begin{baitoan}[\cite{Tuyen_Toan_7}, \textbf{95.}, p. 24]
	Dùng máy  tính để tính $\sqrt{148} + \sqrt{65}$ rồi làm tròn kết quả với độ chính xác $0.5$.
\end{baitoan}

\begin{baitoan}[\cite{Tuyen_Toan_7}, \textbf{96.}, p. 25]
	Áp dụng quy tắc làm tròn số để ước lượng giá trị của biểu thức sau: $A = \frac{53.7\cdot 12.8}{24.56}$.
\end{baitoan}

\begin{baitoan}[\cite{Tuyen_Toan_7}, \textbf{97.}, p. 25]
	Trong học kỳ vừa qua điểm kiểm tra môn Toán của Bình như sau: Điểm kiểm tra thường xuyên (hệ số 1): $8$, $9$, $8$, $9$. Điểm kiểm tra giữa kỳ (hệ số 2): $9$. Điểm kiểm tra cuối kỳ (hệ số 3): $8$. Tính điểm trung bình môn Toán của Bình (làm tròn kết quả đến hàng phần mười).
\end{baitoan}

\begin{baitoan}[\cite{Tuyen_Toan_7}, \textbf{98.}, p. 25]
	Để tính số năm tăng gấp đôi tổng sản phẩm quốc nội (GDP) của 1 quốc gia ta có thể dùng công thức $n = \frac{72}{g}$, trong đó: $g$\% là \emph{tốc độ tăng trưởng GDP} trong giai đoạn đang xét; $n$ là số năm để tăng gấp đôi GDP. Hỏi:
	\begin{enumerate*}
		\item[(a)] Nếu tốc độ tăng trưởng GDP trong giai đoạn hiện nay của Việt Nam khoảng $7.1$\% thì sau bao nhiêu năm nữa GDP của nước ta tăng gấp đôi (làm tròn đến hàng đơn vị)?
		\item[(b)] Nếu muốn sau $7$ năm, GDP tăng gấp đôi thì tốc độ tăng trưởng hằng năm là bao nhiêu \% (làm tròn đến hàng phần mười)?
	\end{enumerate*}
\end{baitoan}

%------------------------------------------------------------------------------%

\subsection{Tỷ Lệ Thức}
``\begin{enumerate*}
	\item[\textbf{1.}] \textit{Tỷ lệ thức} là đẳng thức của 2 tỷ số. Dạng tổng quát: $\frac{a}{b} = \frac{c}{d}$ hoặc $a:b = c:d$. Các số hạng $a$ \& $d$ gọi là \textit{ngoại tỷ}, $b$ \& $c$ gọi là \textit{trung tỷ}.
	\item[\textbf{2.}] Tính chất:
	\begin{enumerate*}
		\item[$\bullet$] \textit{Tính chất cơ bản}: Trong 1 tỷ lệ thức, tích các ngoại tỷ bằng tích các trung tỷ. $\frac{a}{b} = \frac{c}{d}\Rightarrow ad = bc$. Đảo lại, nếu $ad = bc$ \& $abcd\ne 0$ thì $\frac{a}{b} = \frac{c}{d}$.
		\item[$\bullet$] \textit{Tính chất hoán vị}: Từ tỷ lệ thức $\frac{a}{b} = \frac{c}{d}$, với $abcd\ne 0$, ta có thể suy ra 3 tỷ lệ thức khác bằng cách:
		\begin{enumerate*}
			\item[$\circ$] Đổi chỗ các ngoại tỷ cho nhau;
			\item[$\circ$] Đổi chỗ các trung tỷ cho nhau;
			\item[$\circ$] Đổi chỗ các ngoại tỷ đồng thời đổi chỗ các trung tỷ cho nhau.
		\end{enumerate*}
		$((ad = bc)\land(abcd\ne 0))\Leftrightarrow\frac{a}{b} = \frac{c}{d}\Leftrightarrow\frac{d}{b} = \frac{c}{a}\Leftrightarrow\frac{a}{c} = \frac{b}{d}\Leftrightarrow\frac{d}{c} = \frac{b}{a}$.
		\item[$\bullet$] Tính chất của dãy tỷ số bằng nhau: Nếu $\frac{a}{b} = \frac{c}{d} = \frac{e}{f} = k$ thì $\frac{a\pm c\pm e}{b\pm d\pm f} = k$ (giả thiết các tỷ số đều có nghĩa).
	\end{enumerate*}
	\item[\textbf{3.}] \textit{Chú ý}: Các số $x,y,z$ tỷ lệ với các số $a,b,c$ được viết là $\frac{x}{a} = \frac{y}{b} = \frac{z}{c}$. Ta còn viết $x:y:z = a:b:c$.
	\item[\textbf{4.}] \textit{Tính chất tổng hoặc hiệu tỷ lệ}: $\frac{a}{b} = \frac{c}{d}\Leftrightarrow\frac{a\pm b}{b} = \frac{c\pm d}{d}\Leftrightarrow\frac{a\pm b}{a} = \frac{c\pm d}{d}$.'' -- \cite[Chap. 2, \S5, pp. 25--26]{Tuyen_Toan_7}
\end{enumerate*}

``Tỷ lệ thức là 1 đẳng thức của 2 tỷ lệ. Trong tỷ lệ thức $\frac{a}{b} = \frac{c}{d}$ (hoặc $a:b = c:d$) các số hạng $a$ \& $d$ được gọi là \textit{ngoại tỷ}, các số hạng $b$ \& $c$ được gọi là \textit{trung tỷ}. Khi viết tỷ lệ thức $\frac{a}{b} = \frac{c}{d}$, ta luôn giả thiết $b\ne 0$, $d\ne 0$. Từ tỷ lệ thức $\frac{a}{b} = \frac{c}{d}$ ta suy ra $ad = bc$. Đảo lại, nếu $ad = bc$ (cả $4$ số $a,b,c,d$ khác $0$\footnote{I.e., $a^2 + b^2 + c^2 + d^2\ne 0$.}) thì ta có các tỷ lệ thức: $\frac{a}{b} = \frac{c}{d},\frac{a}{c} = \frac{b}{d},\frac{d}{d} = \frac{c}{a},\frac{d}{c} = \frac{b}{a}$. Như vậy trong tỷ lệ thức, ta có thể hoán vị các ngoại tỷ với nhau, hoán vị các trung tỷ với nhau, hoán vị cả ngoại tỷ với nhau \& trung tỷ với nhau. Từ đẳng thức $ad = bc$, ta lập được 4 tỷ lệ thức với các số hạng là $a,b,c,d$ (với quy ước 2 tỷ lệ thức $\frac{a}{b} = \frac{c}{d}$ \& $\frac{c}{d} = \frac{a}{b}$ chỉ kể là 1 tỷ lệ thức).'' -- \cite[\S4]{Binh_Toan_7_tap_1}

\begin{baitoan}[\cite{Tuyen_Toan_7}, Ví dụ 30, p. 26]
	Tìm $x\in\mathbb{R}$ biết: $\frac{x + 6}{32} = \frac{8}{x + 6}$.
\end{baitoan}

\begin{baitoan}[\cite{Tuyen_Toan_7}, Ví dụ 31, p. 26]
	Tìm $x,y,z\in\mathbb{R}$ biết $\frac{x}{y} = \frac{10}{9}$, $\frac{y}{z} = \frac{3}{4}$ \& $x - y + z = 78$.
\end{baitoan}

\begin{baitoan}[\cite{Tuyen_Toan_7}, Ví dụ 32, p. 26]
	Cho 3 số $a,b,c$ sao cho $a + b + c\ne 0$. Biết $\frac{b + c}{a} = \frac{c + a}{b} = \frac{a + b}{c} = k$. Tính giá trị của $k$.
\end{baitoan}

\begin{baitoan}[\cite{Tuyen_Toan_7}, \textbf{99.}, p. 27]
	Lập các tỷ lệ thức có thể được từ 4 số sau: $3$, $-2$, $-9$, $6$.
\end{baitoan}

\begin{baitoan}[\cite{Tuyen_Toan_7}, \textbf{100.}, p. 27]
	Tìm $x$ trong mỗi tỷ lệ thức sau:
	\begin{enumerate*}
		\item[(a)] $\frac{x - 3}{x + 5} = \frac{5}{7}$;
		\item[(b)] $\frac{x + 4}{20} = \frac{5}{x + 4}$;
		\item[(c)] $\frac{x + 1}{x^2} = \frac{1}{x}$.
	\end{enumerate*}
\end{baitoan}

\begin{baitoan}[\cite{Tuyen_Toan_7}, \textbf{101.}, p. 27]
	Chứng minh nếu $\frac{a}{b} = \frac{c}{d}$ thì $\frac{a^2 + b^2}{c^2 + d^2} = \frac{ab}{cd}$.
\end{baitoan}

\begin{baitoan}[\cite{Tuyen_Toan_7}, \textbf{102.}, p. 27]
	Cho $P = \frac{x + 2y - 3z}{x - 2y + 3z}$. Tính giá trị của $P$ biết các số $x,y,z$ tỷ lệ với các số $5,4,3$.
\end{baitoan}

\begin{baitoan}[\cite{Tuyen_Toan_7}, \textbf{103.}, p. 27]
	Cho các số $A,B,C$ tỷ lệ với các số $a,b,c$. Chứng minh giá trị của biểu thức $Q = \frac{Ax + By + C}{ax + by + c}$ không phụ thuộc vào giá trị của $x$ \& $y$.
\end{baitoan}

\begin{baitoan}[\cite{Tuyen_Toan_7}, \textbf{104.}, p. 27]
	Tìm các số $x,y,z$ biết:
	\begin{enumerate*}
		\item[(a)] $\frac{x}{4} = \frac{y}{3} = \frac{z}{9}$ \& $x - 3y + 4z = 62$;
		\item[(b)] $\frac{x}{y} = \frac{9}{7}$, $\frac{y}{z} = \frac{7}{3}$, \& $x - y + z = -15$;
		\item[(c)] $\frac{x}{y} = \frac{7}{20}$, $\frac{y}{z} = \frac{5}{8}$, \& $2x + 5y - 2z = 100$.
	\end{enumerate*}
\end{baitoan}

\begin{baitoan}[\cite{Tuyen_Toan_7}, \textbf{105.}, p. 27]
	Tìm các số $x,y,z$ biết:
	\begin{enumerate*}
		\item[(a)] $5x = 8y = 20z$ \& $x - y - z = 3$;
		\item[(b)] $\frac{6}{11}x = \frac{9}{2}y = \frac{18}{5}z$ \& $-x + y + z = -120$.
	\end{enumerate*}
\end{baitoan}

\begin{baitoan}[\cite{Tuyen_Toan_7}, \textbf{106.}, p. 27]
	1 hộp đựng $70$ quả bóng. Tỷ số giữa số bóng đỏ \& số bóng trắng là $2:3$. Tỷ số giữa số bóng trắng \& số bóng xanh là $3:5$. Tính số bóng đỏ \& số bóng xanh.
\end{baitoan}

\begin{baitoan}[\cite{Tuyen_Toan_7}, \textbf{107.}, p. 27]
	3 kho có tất cả $710$ tấn thóc. Sau khi chuyển $\frac{1}{5}$ số thóc ở kho I, $\frac{1}{6}$ số thóc ở kho II \& $\frac{1}{11}$ số thóc ở kho III thì số thóc còn lại ở 3 kho bằng nhau. Hỏi lúc đầu mỗi kho có bao nhiêu tấn thóc?
\end{baitoan}

\begin{baitoan}[\cite{Tuyen_Toan_7}, \textbf{108.}, p. 28]
	Chia số $x$ thành 3 phần theo thứ tự tỷ lệ với $2,3,4$ rồi lại chia $x$ theo thứ tự tỷ lệ với $3,5,7$ thì có 1 phần giảm đi $1$. Tìm $x$.
\end{baitoan}

\begin{baitoan}[\cite{Tuyen_Toan_7}, \textbf{109.}, p. 28]
	1 khu vườn hình chữ nhật có diện tích $\rm300m^2$, 2 cạnh tỷ lệ với $4$ \& $3$. Tính chiều dài, chiều rộng của khu vườn.
\end{baitoan}

\begin{baitoan}[\cite{Tuyen_Toan_7}, \textbf{110.}, p. 28]
	Tìm $x,y,z$ biết: $\frac{x}{12} = \frac{y}{9} = \frac{z}{5}$ \& $xyz = 20$.
\end{baitoan}

\begin{baitoan}[\cite{Tuyen_Toan_7}, \textbf{111.}, p. 28]
	Tìm $x,y,z$ biết: $\frac{x}{5} = \frac{y}{7} = \frac{z}{3}$ \& $x^2 + y^2 + z^2 = 385$.
\end{baitoan}

\begin{baitoan}[\cite{Tuyen_Toan_7}, \textbf{112.}, p. 28]
	Tìm 2 phân số tối giản biết hiệu của chúng là $\frac{3}{196}$, các tử số tỷ lệ với $3$ \& $5$; các mẫu số tỷ lệ với $4$ \& $7$.
\end{baitoan}

\begin{baitoan}[\cite{Tuyen_Toan_7}, \textbf{113.}, p. 28]
	Tìm $x,y,z$ biết: $\frac{12x - 15y}{7} = \frac{20z - 12x}{9} = \frac{15y - 20z}{11}$ \& $x + y + z = 48$.
\end{baitoan}

\begin{baitoan}[\cite{Tuyen_Toan_7}, \textbf{114.}, p. 28]
	Cho dãy tỷ số bằng nhau: $\frac{2a + b + c + d}{a} = \frac{a + 2b + c + d}{b} = \frac{a + b + 2c + d}{c} = \frac{a + b + c + 2d}{d}$. Tính giá trị của biểu thức $M = \frac{a + b}{c + d} + \frac{b + c}{d + a} + \frac{c + d}{a + b} + \frac{d + a}{b + c}$.
\end{baitoan}

\begin{baitoan}[\cite{Binh_Toan_7_tap_1}, \S4, Ví dụ 6]
	Cho 3 số $6,8,24$.
	\begin{enumerate*}
		\item[(a)] Tìm số $x$, sao cho $x$ cùng với 3 số trên lập thành 1 tỷ lệ thức.
		\item[(b)] Có thể lập được tất cả bao nhiêu tỷ lệ thức?
	\end{enumerate*}
\end{baitoan}

\begin{baitoan}[\cite{Binh_Toan_7_tap_1}, \S4, Ví dụ 7]
	Cho tỷ lệ thức $\frac{a}{b} = \frac{c}{d}$. Chứng minh: $\frac{a}{a - b} = \frac{c}{c - d}$ (giả thiết $a\ne b$, $c\ne d$ \& mỗi số $a,b,c,d\ne 0$).
\end{baitoan}

\begin{baitoan}[\cite{Binh_Toan_7_tap_1}, Ví dụ 8, \S4]
	Cho tỷ lệ thức $\frac{x}{2} = \frac{y}{5}$. Biết $xy = 90$. Tính $x$ \& $y$.
\end{baitoan}

\begin{baitoan}[\cite{Binh_Toan_7_tap_1}, Ví dụ 29, p. 25]
	Cho dãy số $10,11,\ldots,n$. Tìm số $n$ nhỏ nhất để trong dãy đó ta chọn được 4 số khác nhau lập thành 1 tỷ lệ thức.
\end{baitoan}

\begin{baitoan}[\cite{Binh_Toan_7_tap_1}, \S4, \textbf{53.}]
	Tìm $x\in\mathbb{Q}$ trong tỷ lệ thức:
	\begin{enumerate*}
		\item[(a)] $0.4:x = x:0.9$.
		\item[(b)] $13\frac{1}{3}:1\frac{1}{3} = 26:(2x - 1)$.
		\item[(c)] $0.2:1\frac{1}{5} = \frac{2}{3}:(6x + 7)$.
		\item[(d)] $\frac{37 - x}{x + 13} = \frac{3}{7}$.
	\end{enumerate*}
\end{baitoan}

\begin{baitoan}[\cite{Binh_Toan_7_tap_1}, \S4, \textbf{54.}]
	Cho tỷ lệ thức $\frac{3x - y}{x + y} = \frac{3}{4}$. Tìm giá trị của tỷ số $\frac{x}{y}$.
\end{baitoan}

\begin{baitoan}[\cite{Binh_Toan_7_tap_1}, \S4, \textbf{55.}]
	Cho tỷ lệ thức $\frac{a}{b} = \frac{c}{d}$. Chứng minh các tỷ lệ thức sau (giả thiết các tỷ lệ thức đều có nghĩa):
	\begin{enumerate*}
		\item[(a)] $\frac{2a + 3b}{2a - 3b} = \frac{2c + 3d}{2c - 3d}$.
		\item[(b)] $\frac{ab}{cd} = \frac{a^2 - b^2}{c^2 - d^2}$.
		\item[(c)] $\left(\frac{a + b}{c + d}\right)^2 = \frac{a^2 + b^2}{c^2 + d^2}$.
	\end{enumerate*}
\end{baitoan}

\begin{baitoan}[\cite{Binh_Toan_7_tap_1}, \S4, \textbf{56.}]
	Chứng minh: ta có tỷ lệ thức $\frac{a}{b} = \frac{c}{d}$ nếu có 1 trong các đẳng thức sau (giả thiết các tỷ lệ thức đều có nghĩa):
	\begin{enumerate*}
		\item[(a)] $\frac{a + b}{a - b} = \frac{c + d}{c - d}$.
		\item[(b)] $(a + b + c + d)(a - b - c + d) = (a - b + c - d)(a + b - c - d)$.
	\end{enumerate*}
\end{baitoan}

\begin{baitoan}[\cite{Binh_Toan_7_tap_1}, \S4, \textbf{57.}]
	Cho tỷ lệ thức $\frac{a + b + c}{a + b - c} = \frac{a - b + c}{a - b - c}$ trong đó $b\ne 0$. Chứng minh $c = 0$.
\end{baitoan}

\begin{baitoan}[\cite{Binh_Toan_7_tap_1}, \S4, \textbf{58.}]
	Cho tỷ lệ thức $\frac{a + b}{b + c} = \frac{c + d}{d + a}$. Chứng minh: $a = c$ hoặc $a + b + c + d = 0$.
\end{baitoan}

\begin{baitoan}[\cite{Binh_Toan_7_tap_1}, \S4, \textbf{59.}]
	Có thể lập được 1 tỷ lệ thức từ 4 trong các số sau không (mỗi số chỉ chọn 1 lần)? Nếu có thì lập được bao nhiêu tỷ lệ thức?
	\begin{enumerate*}
		\item[(a)] $3,4,5,6,7$.
		\item[(b)] $1,2,4,8,16$.
		\item[(c)] $1,3,9,27,81,243$.
	\end{enumerate*}
\end{baitoan}

\begin{baitoan}[\cite{Binh_Toan_7_tap_1}, \S4, \textbf{60.}]
	Cho 4 số $2,4,8,16$. Tìm $x\in\mathbb{Q}$ cùng với 3 trong 4 số trên lập được thành 1 tỷ lệ thức.
\end{baitoan}

%------------------------------------------------------------------------------%

\subsection{Chứng Minh Tỷ Lệ Thức}
``\begin{enumerate*}
	\item[\textbf{1.}] Học về tỷ lệ thức có nhiều lợi ích. Từ 1 tỷ lệ thức ta có thể chuyển thành 1 đẳng thức giữa 2 tích. Trong 1 tỷ lệ thức, nếu biết 3 số hạng thì có thể tìm được số hạng thứ 4. Khi học về đại lượng tỷ lệ thuận, tỷ lệ nghịch ta sẽ thấy tỷ lệ thức là 1 phương tiện quan trọng giúp ta giải toán. Trong hình học, để học được định lý Thales, tam giác đồng dạng thì không thể thiếu kiến thức về tỷ lệ thức.
	\item[\textbf{2.}] Có nhiều phương pháp chứng minh tỷ lệ thức.'' -- \cite[Chap. 2, \S6, p. 28]{Tuyen_Toan_7}
\end{enumerate*}

\begin{baitoan}[\cite{Tuyen_Toan_7}, Ví dụ 33, p. 28]
	Cho tỷ lệ thức $\frac{a}{b} = \frac{c}{d}\ne 1$ với $a,b,c,d\ne 0$. Chứng minh $\frac{a - b}{a} = \frac{c - d}{c}$.
\end{baitoan}

\begin{baitoan}[\cite{Tuyen_Toan_7}, Ví dụ 34, p. 29]
	Cho $a = b + c$ \& $c = \frac{bd}{b - d}$, $b\ne 0$, $d\ne 0$. Chứng minh $\frac{a}{b} = \frac{c}{d}$.
\end{baitoan}

\begin{baitoan}[\cite{Tuyen_Toan_7}, \textbf{115.}, p. 29]
	Cho $\frac{a}{k} = \frac{x}{a}$, $\frac{b}{k} = \frac{y}{b}$ với $y\ne 0$. Chứng minh $\frac{a^2}{b^2} = \frac{x}{y}$.
\end{baitoan}

\begin{baitoan}[\cite{Tuyen_Toan_7}, \textbf{116.}, p. 29]
	Cho tỷ lệ thức $\frac{x}{y} = \frac{a}{b}$ \& $c = x + y$. Chứng minh $\frac{1}{x} = \frac{a + b}{ac}$ (giả thiết các tỷ số đều có nghĩa).
\end{baitoan}

\begin{baitoan}[\cite{Tuyen_Toan_7}, \textbf{117.}, p. 30]
	Cho tỷ lệ thức $\frac{a}{b} = \frac{c}{d}\ne\pm 1$ \& $c\ne 0$. Chứng minh:
	\begin{enumerate*}
		\item[(a)] $\left(\frac{a - b}{c - d}\right)^2 = \frac{ab}{cd}$;
		\item[(b)] $\left(\frac{a + b}{c + d}\right)^3 = \frac{a^3 - b^3}{c^3 - d^3}$.
	\end{enumerate*}
\end{baitoan}

\begin{baitoan}[\cite{Tuyen_Toan_7}, \textbf{118.}, p. 30]
	Cho tỷ lệ thức $\frac{a}{b} = \frac{c}{d}$, $c\ne\pm\frac{3}{5}d$. Chứng minh $\frac{5a + 3b}{5c + 3d} = \frac{5a - 3b}{5c - 3d}$.
\end{baitoan}

\begin{baitoan}[\cite{Tuyen_Toan_7}, \textbf{119.}, p. 30]
	Cho $b^2 = ac$, $c^2 = bd$ với $b,c,d\ne 0$, $b + c\ne d$, \& $b^3 + c^3\ne d^3$. Chứng minh $\frac{a^3 + b^3 - c^3}{b^3 + c^3 - d^3} = \left(\frac{a + b - c}{b + c - d}\right)^3$.
\end{baitoan}

\begin{baitoan}[\cite{Tuyen_Toan_7}, \textbf{120.}, p. 30]
	Chứng minh nếu $2(x + y) = 5(y + z) = 3(z + x)$ thì $\frac{x - y}{4} = \frac{y - z}{5}$.
\end{baitoan}

\begin{baitoan}[\cite{Tuyen_Toan_7}, \textbf{121.}, p. 30]
	Cho $b^2 = ac$, $a,b,c\ne 0$. Chứng minh $\frac{a^2 + b^2}{b^2 + c^2} = \frac{a}{c}$.
\end{baitoan}

\begin{baitoan}[\cite{Tuyen_Toan_7}, \textbf{122.}, p. 30]
	Cho $\frac{a + b}{a - b} = \frac{c + a}{c - a}$. Chứng minh nếu 3 số $a,b,c$ đều khác $0$ thì từ 3 số này (có 1 số được dùng $2$ lần) có thể lập thành 1 tỷ lệ thức.
\end{baitoan}

\begin{baitoan}[\cite{Tuyen_Toan_7}, \textbf{123.}, p. 30]
	Chứng minh nếu $a^2 = bc$, $a\ne b$, $a\ne c$ thì $\frac{a + b}{a - b} = \frac{c + a}{c - a}$.
\end{baitoan}

\begin{baitoan}[\cite{Tuyen_Toan_7}, \textbf{124.}, p. 30]
	Cho biểu thức $M = \frac{ax + by}{cx + dy}$, $c,d\ne 0$. Chứng minh nếu giá trị của biểu thức $M$ không phụ thuộc vào giá trị của $x$ \& $y$ thì 4 số $a,b,c,d$ lập thành 1 tỷ lệ thức.
\end{baitoan}

\begin{baitoan}[\cite{Tuyen_Toan_7}, \textbf{125.}, p. 30]
	Cho $\frac{a}{x + 2y + z} = \frac{b}{2x + y - z} = \frac{c}{4x - 4y + z}$. Chứng minh $\frac{x}{a + 2b + c} = \frac{y}{2a + b - c} = \frac{z}{4a - 4b + c}$ với $xyz\ne 0$ \& các mẫu số đều khác $0$.
\end{baitoan}

%------------------------------------------------------------------------------%

\subsection{Tính Chất của Dãy Tỷ Số Bằng Nhau}
``Nếu có $n$ tỷ số bằng nhau ($n\ge 2$): $\frac{a_1}{b_1} = \frac{a_2}{b_2} = \cdots = \frac{a_n}{b_n}$ thì $\frac{a_1}{b_1} = \frac{\sum_{i=1}^n c_ia_i}{\sum_{i=1}^n c_ib_i} = \frac{c_1a_1 + c_2a_2 + \cdots + c_na_n}{c_1b_1 + c_2b_2 + \cdots + c_nb_n}$ (nếu đặt dấu ``$-$'' trước số hạng trên của tỷ số nào thì cũng đặt dấu ``$-$'' trước số hạng dưới của tỷ số đó). Ta gọi tính chất này là \textit{tính chất dãy tỷ số bằng nhau}. Tính chất dãy tỷ số bằng nhau cho ta 1 khả năng rộng rãi để từ 1 số tỷ số bằng nhau cho trước, ta lập được những tỷ số mới bằng các tỷ số đã cho, trong đó số hạng trên hoặc số hạng dưới của nó có dạng thuận lợi nhằm sử dụng các dữ kiện của bài toán.'' -- \cite[\S5]{Binh_Toan_7_tap_1}

\begin{baitoan}[\cite{Binh_Toan_7_tap_1}, \S5, Ví dụ 9]
	Tìm các số $x,y,z$ biết $\frac{x}{3} = \frac{y}{4}$, $\frac{y}{5} = \frac{z}{7}$ \& $2x + 3y - z = 186$.
\end{baitoan}

\begin{baitoan}[\cite{Binh_Toan_7_tap_1}, \S5, Ví dụ 10]
	Tìm các số $x,y,z$ biết $\frac{y + z + 1}{x} = \frac{x + z + 2}{y} = \frac{x + y - 3}{z} = \frac{1}{x + y + z}$.
\end{baitoan}

\begin{baitoan}[\cite{Binh_Toan_7_tap_1}, \S5, \textbf{61.}]
	Tìm các số $x,y,z$ biết:
	\begin{enumerate*}
		\item[(a)] $\frac{x}{10} = \frac{y}{10} = \frac{z}{21}$ \& $5x + y - 2z = 28$.
		\item[(b)] $3x = 2y$, $7y = 5z$, $x - y + z = 32$.
		\item[(c)] $\frac{x}{3} = \frac{y}{4}$, $\frac{y}{3} = \frac{z}{5}$, $2x - 3y + z = 6$.
		\item[(d)] $\frac{2x}{3} = \frac{3y}{4} = \frac{4z}{5}$ \& $x + y + z = 49$.
		\item[(e)] $\frac{x - 1}{2} = \frac{y - 2}{3} = \frac{z - 3}{4}$ \& $2x + 3y - z = 50$.
		\item[(g)] $\frac{x}{2} = \frac{y}{3} = \frac{z}{5}$ \& $xyz = 810$.
	\end{enumerate*}
\end{baitoan}

\begin{baitoan}[\cite{Binh_Toan_7_tap_1}, \S5, \textbf{62.}]
	Tìm $x$ biết $\frac{1 + 2y}{18} = \frac{1 + 4y}{24} = \frac{1 + 6y}{6x}$.
\end{baitoan}

\begin{baitoan}[\cite{Binh_Toan_7_tap_1}, \S5, \textbf{63.}]
	Tìm phân số $\frac{a}{b}$ biết nếu cộng thêm cùng 1 số khác $0$ vào tử \& mẫu thì giá trị của phân số đó không đổi.
\end{baitoan}

\begin{baitoan}[\cite{Binh_Toan_7_tap_1}, \S5, \textbf{64.}]
	Cho $\frac{a}{b} = \frac{b}{c} = \frac{c}{d}$. Chứng minh $\left(\frac{a + b + c}{b + c + d}\right)^3 = \frac{a}{d}$.
\end{baitoan}

\begin{baitoan}[\cite{Binh_Toan_7_tap_1}, \S5, \textbf{65.}]
	Cho $\frac{a}{b} = \frac{b}{c} = \frac{c}{a}$. Chứng minh $a = b = c$.
\end{baitoan}

\begin{baitoan}[\cite{Binh_Toan_7_tap_1}, \S5, \textbf{66.}]
	Vì sao tỷ số của 2 hỗn số dạng $a\frac{1}{b}$ \& $b\frac{1}{a}$ luôn luôn bằng phân số $\frac{a}{b}$?
\end{baitoan}

\begin{baitoan}[\cite{Binh_Toan_7_tap_1}, \S5, \textbf{67.}]
	Cho 3 tỷ số bằng nhau là $\frac{a}{b + c},\frac{b}{c + a},\frac{c}{a + b}$. Tìm giá trị của mỗi tỷ số đó.
\end{baitoan}

%------------------------------------------------------------------------------%

\subsection{Đại Lượng Tỷ Lệ Thuận}
``\begin{enumerate*}
	\item[\textbf{1.}] Nếu đại lượng $y$ liên hệ với đại lượng $x$ theo công thức $y = kx$ (với $k$ là 1 hằng số khác $0$) thì ta nói $y$ tỷ lệ thuận với $x$ theo hệ số tỷ lệ $k$. Nếu $y$ tỷ lệ thuận với $x$ theo hệ số tỷ lệ $k$ thì $x$ tỷ lệ thuận với $y$ theo hệ số tỷ lệ $\frac{1}{k}$.
	\item[\textbf{2.}] Nếu 2 đại lượng tỷ lệ thuận với nhau thì:
	\begin{enumerate*}
		\item[$\bullet$] Tỷ số 2 giá trị tương ứng của chúng luôn không đổi.
		\item[$\bullet$] Tỷ số 2 giá trị bất kỳ của đại lượng này bằng tỷ số 2 giá trị tương ứng của đại lượng kia. $\frac{y_1}{x_1} = \frac{y_2}{x_2} = \frac{y_3}{x_3} = \cdots = k$, i.e., $\frac{x_i}{y_i} = k$, $\forall i\in\mathbb{N}^\star$. $\frac{x_1}{x_2} = \frac{y_1}{y_2}$, $\frac{x_1}{x_3} = \frac{y_1}{y_3}$, $\frac{x_2}{x_3} = \frac{y_2}{y_3}$ $\ldots$, i.e., $\frac{x_i}{y_j} = \frac{x_j}{y_i}$, $\forall i,j\in\mathbb{N}^\star$.
		\end{enumerate*}
	\item[\textbf{3.}] Chia số $M$ cho trước thành những phần $x,y,z$ tỷ lệ thuận với các số $a,b,c$ có nghĩa là tìm $x,y,z$, biết $\frac{x}{a} = \frac{y}{b} = \frac{z}{c}$ \& $x + y + z = M$. \textit{Cách giải}: dựa vào tính chất của dãy tỷ số bằng nhau.
	\item[\textbf{4.}] Nếu $z$ tỷ lệ thuận với $y$ theo hệ số tỷ lệ $k_1$, $y$ tỷ lệ thuận với $x$ theo hệ số tỷ lệ $k_2$ thì $z$ tỷ lệ thuận với $x$ theo hệ số tỷ lệ $k_1k_2$.'' -- \cite[Chap. 2, \S7, pp. 30--31]{Tuyen_Toan_7}
\end{enumerate*}

\begin{baitoan}[\cite{Tuyen_Toan_7}, Ví dụ 35, p. 31]
	Cho $y$ tỷ lệ thuận với $x$ với hệ số tỷ lệ là 1 số âm. Biết tổng các bình phương 2 giá trị của $y$ là $18$, tổng các bình phương 2 giá trị tương ứng của $x$ là $2$. Viết công thức liên hệ giữa $y$ \& $x$.
\end{baitoan}

\begin{baitoan}[\cite{Tuyen_Toan_7}, Ví dụ 36, p. 31]
	1 xe tải chạy từ $A$ đến $B$ mất $6$\emph{h}, trong khi đó 1 xe con chạy từ $B$ đến $A$ chỉ mất $3$\emph{h}. Nếu 2 xe khởi hành cùng 1 lúc thì sau bao lâu sẽ gặp nhau?
\end{baitoan}

\begin{baitoan}[\cite{Tuyen_Toan_7}, Ví dụ 37, p. 32]
	Mức nước sinh hoạt của nhà Thủy được thống kê trong bảng sau:
	
	\begin{table}[H]
		\centering
		\begin{tabular}{|c|c|c|c|c|}
			\hline
			Thời điểm & Cuối tháng $6$ & Cuối tháng $7$ & Cuối tháng $8$ & Cuối tháng $9$ \\
			\hline
			Chỉ số đồng hồ đo nước $\rm m^3$ & $204$ & $220$ & $237$ & $250$ \\
			\hline
		\end{tabular}
	\end{table}
	Biết tổng số tiền nước nhà Thủy phải trả trong quý III là $184000$ đồng. Tính tiền nước phải trả trong mỗi tháng $7,8,9$.
\end{baitoan}

\begin{baitoan}[\cite{Tuyen_Toan_7}, \textbf{126.}, p. 32]
	1 số $M$ được chia làm $3$ phần sao cho phần thứ nhất \& phần thứ 2 tỷ lệ với $5$ \& $6$, phần thứ 2 \& phần thứ 3 tỷ lệ với $8$ \& $9$. Biết phần thứ 3 hơn phần thứ 2 là $150$. Tìm số $M$.
\end{baitoan}
	
\begin{baitoan}[\cite{Tuyen_Toan_7}, \textbf{127.}, p. 32]
	1 đội thủy lợi có $10$ người làm trong $8$ ngày đào đắp được $\rm200m^3$ đất. 1 đội khác có $12$ người làm trong $7$ ngày đào đắp được bao nhiêu mét khối đất? (giả thiết năng suất của mỗi người đều như nhau).
\end{baitoan}

\begin{baitoan}[\cite{Tuyen_Toan_7}, \textbf{128.}, p. 32]
	2 bể nước hình hộp chữ nhật có diện tích bằng nhau. Biết hiệu thể tích nước trong 2 bể là $\rm1.8m^3$, hiệu chiều cao nước trong 2 bể là $\rm0.6m$. Tính diện tích đáy của mỗi bể?
\end{baitoan}

\begin{baitoan}[\cite{Tuyen_Toan_7}, \textbf{129.}, p. 32]
	Đoạn đường $AB$ dài $\rm275km$. Cùng 1 lúc, 1 ô tô chạy từ A \& 1 xe máy chạy từ B đi ngược chiều để gặp nhau. Vận tốc của ô tô là $60$\emph{km\texttt{/}h}, vận tốc của xe máy là $50$\emph{km\texttt{/}h}. Tính xem đến khi gặp nhau thì mỗi xe đã đi được quãng đường bao nhiêu.
\end{baitoan}

\begin{baitoan}[\cite{Tuyen_Toan_7}, \textbf{130.}, p. 32]
	Vận tốc riêng của 1 canô là $21$\emph{km\texttt{/}h}, vận tốc dòng nước là $3$\emph{km\texttt{/}h}. Hỏi với thời gian để canô chạy ngược dòng được $30$\emph{km} thì canô chạy xuôi dòng được bao nhiêu \emph{km}?
\end{baitoan}

\begin{baitoan}[\cite{Tuyen_Toan_7}, \textbf{131.}, p. 32]
	1 ô tô chạy từ $A$ đến $B$ với vận tốc $65$\emph{km\texttt{/}h}, cùng lúc đó 1 xe máy chạy từ $B$ đến $A$ với vận tốc $40$\emph{km\texttt{/}h}. Biết khoảng cách $AB$ là $540$\emph{km} \& $M$ là trung điểm của $AB$. Hỏi sau khi khởi hành bao lâu thì ô tô cách $M$ 1 khoảng cách bằng $\frac{1}{2}$ khoảng cách từ xe máy đến $M$?
\end{baitoan}

\begin{baitoan}[\cite{Tuyen_Toan_7}, \textbf{132.}, p. 32]
	Người ta trồng cây ở 1 bên của đoạn đường dài $30$\emph{m} với khoảng cách giữa 2 cây liên tiếp là $5$\emph{m}. Nếu cả 2 đầu đường đều trồng cây thì số cây được trồng là $\frac{30}{5} + 1 = 7$ (cây). Nếu đoạn đường dài $300$\emph{m}, gấp $10$ lần đoạn đường $30$\emph{m} thì số cây trồng phải gấp $10$ lần, tức là phải trồng $7\cdot 10 = 70$ (cây). Lập luận đó đúng hay sai?
\end{baitoan}

%------------------------------------------------------------------------------%

\subsection{Đại Lượng Tỷ Lệ Nghịch}
``\begin{enumerate*}
	\item[\textbf{1.}] Nếu đại lượng $y$ liên hệ với đại lượng $x$ theo công thức $y = \frac{a}{x}$ hay $xy = a$ (với $a$ là hằng số khác $0$) thì ta nói $y$ tỷ lệ nghịch với $x$ theo hệ số tỷ lệ $a$. Nếu $y$ tỷ lệ nghịch với $x$ theo hệ số tỷ lệ $a$ thì $x$ cũng tỷ lệ nghịch với $y$ theo hệ số tỷ lệ $a$.
	\item[\textbf{2.}] Nếu 2 đại lượng tỷ lệ nghịch với nhau thì:
	\begin{enumerate*}
		\item[$\bullet$] Tích 2 giá trị tương ứng của chúng luôn không đổi (bằng hệ số tỷ lệ).
		\item[$\bullet$] Tỷ số 2 giá trị của đại lượng này bằng nghịch đảo của tỷ số 2 giá trị tương ứng của đại lượng kia. $x_1y_1 = x_2y_2 = x_3y_3 = \cdots = a$, i.e., $x_iy_i = \cdots = a$, $\forall i\in\mathbb{N}^\star$ hay $\frac{x_1}{\frac{1}{y_1}} = \frac{x_2}{\frac{1}{y_2}} = \frac{x_3}{\frac{1}{y_3}} = \cdots = a$, i.e., $\frac{x_i}{\frac{1}{y_i}} = a$, $\forall i\in\mathbb{N}^\star$. $\frac{x_1}{x_2} = \frac{y_2}{y_1}$, $\frac{x_1}{x_3} = \frac{y_3}{y_1}$, $\frac{x_2}{x_3} = \frac{y_3}{y_2}$, $\ldots$, i.e., $\frac{x_i}{x_j} = \frac{y_j}{y_i}$, $\forall i,j\in\mathbb{N}^\star$.
	\end{enumerate*}
	\item[\textbf{3.}] Chia số $M$ cho trước thành những phần $x,y,z$ tỷ lệ nghịch với các số $a,b,c$ có nghĩa là tìm $x,y,z$ biết $ax = by = cz$ hoặc $\frac{x}{\frac{1}{a}} = \frac{y}{\frac{1}{b}} = \frac{z}{\frac{1}{c}}$ với $x + y + z = M$.
	\item[\textbf{4.}] Nếu $z$ tỷ lệ nghịch với $y$ theo hệ số tỷ lệ $a_1$, $y$ tỷ lệ nghịch với $x$ theo hệ số tỷ lệ là $a_2$ thì $z$ tỷ lệ thuận với $x$ theo hệ số tỷ lệ $\frac{a_1}{a_2}$.\footnote{Tương tự như quy tắc dấu: `âm' của `âm' thành `dương' ($--$ thành $+$), `tỷ lệ nghịch' của `tỷ lệ nghịch' thành `tỷ lệ thuận'. Trong khi đó: `dương' của `dương' vẫn là `dương' ($++$ vẫn là $+$), `tỷ lệ thuận' của `tỷ lệ thuận' vẫn là `tỷ lệ thuận'.}'' -- \cite[Chap. 2, \S8, p. 33]{Tuyen_Toan_7}
\end{enumerate*}

\begin{baitoan}[\cite{Tuyen_Toan_7}, Ví dụ 38, p. 33]
	2 ô tô cùng khởi hành từ $A$ đến $B$. Vận tốc của ô tô I là $50$\emph{km\texttt{/}h}, vận tốc của ô tô II là $60$\emph{km\texttt{/}h}. Ô tô I đi đến $B$ sau ô tô II là $36$ phút. Tính quãng đường $AB$.
\end{baitoan}

\begin{baitoan}[\cite{Tuyen_Toan_7}, Ví dụ 39, p. 34]
	1 số $M$ được chia thành 3 phần tỷ lệ nghịch với $5,2,4$. Biết tổng các lập phương của 3 phần đó là $9512$. Tìm số $M$.
\end{baitoan}

\begin{baitoan}[\cite{Tuyen_Toan_7}, \textbf{133.}, p. 34]
	2 bác mua gạo hết cùng 1 số tiền. Bác thứ nhất mua loại $24000$ \emph{đồng\texttt{/}kg}, bác thứ 2 mua loại $28800$ \emph{đồng\texttt{/}kg}. Biết bác thứ nhất mua nhiều hơn bác thứ 2 là $2$\emph{kg}. Hỏi mỗi bác mua bao nhiêu \emph{kg} gạo?
\end{baitoan}

\begin{baitoan}[\cite{Tuyen_Toan_7}, \textbf{134.}, p. 34]
	2 cạnh của 1 tam giác dài $25$\emph{cm} \& $36$\emph{cm}. Tổng độ dài 2 đường cao tương ứng là $48.8$\emph{cm}. Tính độ dài của mỗi đường cao nói trên.
\end{baitoan}

\begin{baitoan}[\cite{Tuyen_Toan_7}, \textbf{135.}, p. 34]
	1 xe ô tô chạy từ $A$ đến $B$ gồm 3 chặng đường dài như nhau nhưng chất lượng mặt đường tốt xấu khác nhau. Vận tốc trên mỗi chặng đường lần lượt là $72$\emph{km\texttt{/}h}, $60$\emph{km\texttt{/}h}, $40$\emph{km\texttt{/}h}. Biết tổng thời gian xe chạy từ $A$ đến $B$ là $4$\emph{h}. Tính quãng đường $AB$.
\end{baitoan}

\begin{baitoan}[\cite{Tuyen_Toan_7}, \textbf{136.}, p. 34]	
	1 xe ô tô chạy từ $A$ đến $B$ với vận tốc $50$\emph{km\texttt{/}h} rồi chạy từ $B$ về $A$ với vận tốc $40$\emph{km\texttt{/}h}. Cả đi lẫn về mất $\rm4h30m$. Tính thời gian đi \& về.
\end{baitoan}

\begin{baitoan}[\cite{Tuyen_Toan_7}, \textbf{137.}, p. 34]
	1 ô tô dự định chạy từ $A$ đến $B$ trong thời gian nhất định. Nếu xe chạy với vận tốc $54$\emph{km\texttt{/}h} thì đến nơi sớm hơn $1$\emph{h}. Nếu xe chạy với vận tốc $63$\emph{km\texttt{/}h} thì đến nơi sớm hơn $2$\emph{h}. Tính quãng đường $AB$ \& thời gian dự định đi.
\end{baitoan}

\begin{baitoan}[\cite{Tuyen_Toan_7}, \textbf{138.}, p. 34]
	Để làm xong 1 công việc thì $21$ công nhân cần làm trong $15$ ngày. Do cải tiến công cụ lao động nên năng suất lao động của mỗi người tăng thêm $25$\%. Hỏi $18$ công nhân phải làm bao lâu mới xong công việc đó?
\end{baitoan}

\begin{baitoan}[\cite{Tuyen_Toan_7}, \textbf{139.}, p. 34]
	Để làm xong 1 công việc, 1 số công nhân cần làm trong 1 số ngày. 1 bạn học sinh lập luận: Nếu số công nhân tăng thêm $\frac{1}{3}$ thì thời gian sẽ giảm đi $\frac{1}{3}$. Đúng hay sai?
\end{baitoan}

%------------------------------------------------------------------------------%

\subsection{Chia Tỷ Lệ}
``Trong các bài toán về chia 1 số thành các phần tỷ lệ thuận hoặc tỷ lệ nghịch với các số cho trước, cần chú ý:
\begin{enumerate*}
	\item[\textbf{1.}] $x,y,z$ tỷ lệ thuận với $a,b,c\Leftrightarrow x:y:z = a:b:c\Leftrightarrow\frac{x}{a} = \frac{y}{b} = \frac{z}{c}$.
	\item[\textbf{2.}] $x,y,z$ tỷ lệ nghịch với $m,n,p\Leftrightarrow x:y:z = \frac{1}{m}:\frac{1}{n}:\frac{1}{p}$.'' -- \cite{Binh_Toan_7_tap_1}
\end{enumerate*}

\begin{baitoan}[\cite{Binh_Toan_7_tap_1}, Ví dụ 16]
	2 xe ô tô cùng khởi hành 1 lúc từ 2 địa điểm A \& B. Xe thứ nhất đi quãng đường AB hết $\rm4h15ph$, xe thứ 2 đi quãng đường BA hết $\rm3h45ph$. Đến chỗ gặp nhau, xe thứ 2 đi được quãng đường dài hơn quãng đường xe thứ nhất đã đi là $20$\emph{km}. Tính quãng đường AB.
\end{baitoan}

\begin{baitoan}[\cite{Binh_Toan_7_tap_1}, Ví dụ 17]
	Để đi từ A đến B có thể dùng các phương tiện: máy bay, ô tô, xe lửa. Vận tốc của máy bay, ô tô, xe lửa có tỷ lệ với $6$; $2$; $1$. Biết thời gian đi từ A đến B bằng máy bay ít hơn so với đi bằng ô tô là $6$ giờ. Hỏi thời gian xe lửa đi quãng đường AB là bao lâu?
\end{baitoan}

\begin{baitoan}[\cite{Binh_Toan_7_tap_1}, Ví dụ 18]
	3 kho A, B, C chứa 1 số gạo. Người ta nhập vào kho A thêm $\frac{1}{7}$ số gạo của kho đó, xuất ở kho B đi $\frac{1}{9}$ số gạo của kho đó, xuất ở kho C đi $\frac{2}{7}$ số gạo của kho đó. Khi đó số gạo của 3 kho bằng nhau. Tính số gạo ở mỗi kho lúc đầu, biết kho B chứa nhiều hơn kho A là $20$ tạ gạo.
\end{baitoan}

\begin{baitoan}[\cite{Binh_Toan_7_tap_1}, Ví dụ 19]
	3 đội công nhân I, II, III phải vận chuyển tổng cộng $1530$\emph{kg} hàng từ kho theo thứ tự đến 3 địa điểm cách kho $1500$\emph{m}, $2000$\emph{m}, $3000$\emph{m}. Phân chia số hàng cho mỗi đội sao cho khối lượng hàng tỷ lệ nghịch với khoảng cách cần chuyển.
\end{baitoan}

\begin{baitoan}[\cite{Binh_Toan_7_tap_1}, Ví dụ 20]
	3 xí nghiệp cùng xây dựng chung 1 cái cầu hết $38$ triệu đồng. Xí nghiệp I có $40$ xe ở cách cầu $1.5$\emph{km}, xí nghiệp II có $20$ xe ở cách cầu $3$\emph{km}, xí nghiệp III có $30$ xe ở cách cầu $1$\emph{km}. Hỏi mỗi xí nghiệp phải trả cho việc xây dựng cầu bao nhiêu tiền, biết số tiền phải trả tỷ lệ thuận với số xe \& tỷ lệ nghịch với khoảng cách từ xí nghiệp đến cầu?
\end{baitoan}

\begin{baitoan}[\cite{Binh_Toan_7_tap_1}, \textbf{106.}]
	\begin{enumerate*}
		\item[(a)] Tính thời gian từ lúc 2 kim đồng hồ gặp nhau lần trước đến lúc chúng gặp nhau lần tiếp theo.
		\item[(b)] Trong 1 ngày, 2 kim đồng hồ tạo với nhau góc vuông bao nhiêu lần?
	\end{enumerate*}
\end{baitoan}

\begin{baitoan}[\cite{Binh_Toan_7_tap_1}, \textbf{107.}]
	1 ống dài được kéo bởi 1 máy kéo trên đường. Tuấn chạy dọc từ đầu ống đến cuối ống theo hướng chuyển động của máy kéo thì đếm được $140$ bước. Sau đó Tuấn quay lại chạy dọc ống theo chiều ngược lại thì đếm được $20$ bước. Biết mỗi bước chạy của Tuấn dài $1$\emph{m}. Tính độ dài của ống.
\end{baitoan}

\begin{baitoan}[\cite{Binh_Toan_7_tap_1}, \textbf{108.}]
	5 lớp 7A, 7B, 7C, 7D, 7E nhận chăm sóc vườn trường có diện tích $300\rm m^2$. Lớp 7A nhận $15$\% diện tích vườn, lớp 7B nhận $\frac{1}{5}$ diện tích còn lại. Diện tích còn lại của vườn sau khi 2 lớp trên nhận được đem chia cho 3 lớp 7C, 7D, 7E tỷ lệ với $\frac{1}{2},\frac{1}{4},\frac{5}{16}$. Tính diện tích vườn giao cho mỗi lớp.
\end{baitoan}

\begin{baitoan}[\cite{Binh_Toan_7_tap_1}, \textbf{109.}]
	3 công nhân được thưởng $100000$ đồng, số tiền thưởng được phân chia tỷ lệ với mức sản xuất của mỗi người. Biết mức sản xuất của người thứ nhất so với mức sản xuất của người thứ 2 bằng $5:3$, mức sản xuất của người thứ 3 bằng $25$\% tổng số mức sản xuất của 2 người kia. Tính số tiền mỗi người được thưởng.
\end{baitoan}

\begin{baitoan}[\cite{Binh_Toan_7_tap_1}, \textbf{110.}]
	1 công trường dự định phân chia số đất cho 3 đội I, II, III tỷ lệ với $7,6,5$. Nhưng sau đó vì số người của các đội thay đổi nên đã chia lại tỷ lệ với $6,5,4$. Như vậy có 1 đội làm nhiều hơn so với dự định là $\rm6m^3$ đất. Tính số đất đã phân chia cho mỗi đội.
\end{baitoan}

\begin{baitoan}[\cite{Binh_Toan_7_tap_1}, \textbf{111.}]
	Trong 1 đợt lao đông, 3 khối 7, 8, 9 chuyển được $\rm912m^3$ đất. Trung bình mỗi học sinh khối 7, 8, 9 theo thứ tự làm được $\rm1.2m^3,1.4m^3,1.6m^3$. Số học sinh khối 7 \& khối 8 tỷ lệ với 1 \& 3, số học sinh khối 8 \& khối 9 tỷ lệ với 4 \& 5. Tính số học sinh của mỗi khối.
\end{baitoan}

\begin{baitoan}[\cite{Binh_Toan_7_tap_1}, \textbf{112.}]
	3 tổ công nhân có mức sản xuất tỷ lệ với $5,4,3$. Tổ I tăng năng suất $10$\%, tổ II tăng năng suất $20$\%, tổ III tăng năng suất $10$\%. Do đó trong cùng 1 thời gian, tổ I làm được nhiều hơn tổ II là $7$ sản phẩm. Tính số sản phẩm mỗi tổ đã làm được trong thời gian đó.
\end{baitoan}

\begin{baitoan}[\cite{Binh_Toan_7_tap_1}, \textbf{113.}]
	Tìm 3 số tự nhiên, biết $\operatorname{BCNN}$ của chúng bằng $3150$, tỷ số của số thứ nhất \& số thứ 2 là $5:9$, tỷ số của số thứ nhất \& số thứ 3 là $10:7$.
\end{baitoan}

\begin{baitoan}[\cite{Binh_Toan_7_tap_1}, \textbf{114.}]
	3 tấm vải theo thứ tự giá $120000$ đồng, $192000$ đồng, \& $144000$ đồng. Tấm thứ nhất \& tấm thứ 2 có cùng chiều dài, tấm thứ 2, \& tấm thứ 3 có cùng chiều rộng. Tổng của 3 chiều dài là $110$\emph{m}, tổng của 3 chiều rộng là $2.1$\emph{m}. Tính kích thước của mỗi tấm vải, biết giá $\rm1m^2$ của 3 tấm vải bằng nhau.
\end{baitoan}

\begin{baitoan}[\cite{Binh_Toan_7_tap_1}, \textbf{115.}]
	Có 3 gói tiền: gói thứ nhất gồm toàn tờ $500$ đồng, gói thứ 2 gồm toàn tờ $2000$ đồng, gói thứ 3 gồm toàn tờ $5000$ đồng. Biết tổng số tờ giấy bạc của 3 gói là $540$ tờ \& số tiền ở các gói bằng nhau. Tính số tờ giấy bạc mỗi loại.
\end{baitoan}

\begin{baitoan}[\cite{Binh_Toan_7_tap_1}, \textbf{116.}]
	3 công nhân tiện được tất cả $860$ dụng cụ trong cùng 1 thời gian. Để tiện 1 dụng cụ, người thứ nhất cần $5$\emph{ph}, người thứ 2 cần $6$\emph{ph}, người thứ 3 cần $9$\emph{ph}. Tính số dụng cụ mỗi người tiện được.
\end{baitoan}

\begin{baitoan}[\cite{Binh_Toan_7_tap_1}, \textbf{117.}]
	3 em bé: Ánh $5$ tuổi, Bích $6$ tuổi, Châu $10$ tuổi được bà chia cho $42$ chiếc kẹo. Số kẹo được chia tỷ lệ nghịch với số tuổi của mỗi em. Hỏi mỗi em được chia bao nhiêu chiếc kẹo?
\end{baitoan}

\begin{baitoan}[\cite{Binh_Toan_7_tap_1}, \textbf{118.}]
	Tìm 3 phân số, biết tổng của chúng bằng $3\frac{3}{70}$, các tử của chúng tỷ lệ với $3,4,5$, các mẫu của chúng tỷ lệ với $5,1,2$.
\end{baitoan}

\begin{baitoan}[\cite{Binh_Toan_7_tap_1}, \textbf{119.}]
	Tìm số tự nhiên có 3 chữ số, biết số đó là bội của $72$ \& các chữ số của nó nếu xếp từ nhỏ đến lớn thì tỷ lệ với $1,2,3$.
\end{baitoan}

\begin{baitoan}[\cite{Binh_Toan_7_tap_1}, \textbf{120.}]
	Độ dài 3 cạnh của 1 tam giác tỷ lệ với $2,3,4$. 3 chiều cao tương ứng với 3 cạnh đó tỷ lệ với 3 số nào?
\end{baitoan}

\begin{baitoan}[\cite{Binh_Toan_7_tap_1}, \textbf{121.}]
	3 đường cao của $\Delta ABC$ có độ dài bằng $4,12,x$. Biết $x\in\mathbb{N}^\star$. Tìm $x$ (cho biết \emph{bất đẳng thức tam giác}: mỗi cạnh của tam giác nhỏ hơn tổng 2 cạnh kia \& lớn hơn hiệu của chúng).
\end{baitoan}

\begin{baitoan}[\cite{Binh_Toan_7_tap_1}, \textbf{122.}]
	Cho $\Delta ABC$. Có góc ngoài của tam giác tại $A,B,C$ tỷ lệ với $4,5,6$. Các góc trong tương ứng tỷ lệ với các số nào?
\end{baitoan}

\begin{baitoan}[\cite{Binh_Toan_7_tap_1}, \textbf{123.}]
	Tìm 2 số khác $0$ biết tổng, hiệu, tích của chúng tỷ lệ với $5,1,12$.
\end{baitoan}

%------------------------------------------------------------------------------%

\subsection{Miscellaneous}
\textsf{\textbf{Nội dung.} Định nghĩa số vô tỷ, căn bậc 2 số học; tập hợp $\mathbb{R}$ các số thực; giá trị tuyệt đối của 1 số thực; làm tròn số; tỷ lệ thức \& dãy tỷ số bằng nhau; đại lượng tỷ lệ thuận, tỷ lệ nghịch.}

\begin{baitoan}[\cite{Tuyen_Toan_7}, Ví dụ 40, p. 35]
	So sánh $\sqrt{24} + \sqrt{14}$ \& $\sqrt{84}$.
\end{baitoan}

\begin{baitoan}[\cite{Tuyen_Toan_7}, Ví dụ 41, p. 35]
	3 công nhân được lĩnh tổng cộng $18 500 000$ đồng tiền thưởng. Số tiền thưởng của mỗi người tỷ lệ nghịch với số ngày nghỉ của họ. Biết số ngày nghỉ lần lượt là $5,4,6$ ngày. Tính số tiền thưởng của mỗi người.
\end{baitoan}

\begin{baitoan}[\cite{Tuyen_Toan_7}, \textbf{140.}, p. 35]
	Trong các số sau, những số nào là số hữu tỷ, những số nào là số vô tỷ? $0.4343\ldots$, $-13.9$, $\pi$, $59.8637$, $3.464101615\ldots$, $\sqrt{10}$, $6 + \sqrt{2}$, $\frac{61}{172}$, số $x > 0$ mà $x^2 = 7$, số $y > 0$ mà $y^2 = 121$.
\end{baitoan}

\begin{baitoan}[\cite{Tuyen_Toan_7}, \textbf{141.}, p. 36]
	So sánh:
	\begin{enumerate*}
		\item[(a)] $5 + \sqrt{99}$ \& $\sqrt{21} + \sqrt{93}$;
		\item[(b)] $\sqrt{54} + \sqrt{230}$ \& $22$.
	\end{enumerate*}
\end{baitoan}

\begin{baitoan}[\cite{Tuyen_Toan_7}, \textbf{142.}, p. 36]
	Viết các số sau dưới dạng số thập phân hữu hạn hoặc vô hạn (làm tròn đến hàng phần trăm). Sắp xếp kết quả theo thứ tự tăng dần: $\frac{37}{7}$, $\frac{43}{8}$, $\sqrt{29}$.
\end{baitoan}

\begin{baitoan}[\cite{Tuyen_Toan_7}, \textbf{143.}, p. 36]
	Tìm $x$ biết: $\left|x + \frac{1}{101}\right| + \left|x + \frac{2}{101}\right| = \left|x + \frac{3}{101}\right| + \cdots + \left|x + \frac{100}{101}\right| = 101x$.
\end{baitoan}

\begin{baitoan}[\cite{Tuyen_Toan_7}, \textbf{144.}, p. 36]
	Cho $2026$ số thực $a_1,a_2,a_3,\ldots,a_{2026}$ sao cho bất kỳ $5$ số nào trong chúng cũng có tổng bằng $0$. Tìm $a_{2026}$.
\end{baitoan}

\begin{baitoan}[\cite{Tuyen_Toan_7}, \textbf{145.}, p. 36]
	Tìm 3 phân số tối giản biết tổng của chúng bằng $5\frac{25}{63}$, tử số của chúng tỷ lệ nghịch với $20,4,5$ \& mẫu số của chúng tỷ lệ thuận với $1,3,7$.
\end{baitoan}

\begin{baitoan}[\cite{Tuyen_Toan_7}, \textbf{146.}, p. 36]
	Chu vi 1 tam giác là $60$\emph{cm}. Các đường cao có độ dài là $12$\emph{cm}, $15$\emph{cm}, $20$\emph{cm}. Tính độ dài mỗi cạnh của tam giác đó.
\end{baitoan}

\begin{baitoan}[\cite{Tuyen_Toan_7}, \textbf{147.}, p. 36]
	Nếu ta cộng từng 2 cạnh của 1 tam giác thì 3 tổng tỷ lệ với $5,6,7$. Chứng minh tam giác này có 1 đường cao dài gấp $2$ lần 1 đường cao khác.
\end{baitoan}

\begin{baitoan}[\cite{Tuyen_Toan_7}, \textbf{148.}, p. 36]
	1 xe ô tô khởi hành từ $A$, dự định chạy với vận tốc $60$\emph{km\texttt{/}h} thì sẽ tới $B$ lúc 11:00. Sau khi chạy được nửa quãng đường vì đường hẹp \& xấu nên vận tốc ô tô giảm xuống còn $40$\emph{km\texttt{/}h} do đó đến 11:00 xe vẫn còn cách $B$ là $40$\emph{km}.
	\begin{enumerate*}
		\item[(a)] Tính khoảng cách $AB$;
		\item[(b)] Xe khởi hành lúc mấy giờ?
	\end{enumerate*}
\end{baitoan}

\begin{baitoan}[\cite{Tuyen_Toan_7}, \textbf{149.}, p. 36]
	1 đơn vị làm đường lúc đầu đặt kế hoạch giao cho 3 đội I, II, III, mỗi đội làm 1 đoạn đường có chiều dài tỷ lệ với $7,8,9$. Về sau do thiết bị máy móc \& nhân lực của các đội thay đổi nên kế hoạch đã được điều chỉnh, mỗi đội làm 1 đoạn đường có chiều dài tỷ lệ với $6,7,8$. Như vậy đội III phải làm nhiều hơn so với kế hoạch ban đầu là $0.5$\emph{km} đường. Tính chiều dài đoạn đường mà mỗi đội phải làm theo kế hoạch mới.
\end{baitoan}

%------------------------------------------------------------------------------%

\printbibliography[heading=bibintoc]
	
\end{document}