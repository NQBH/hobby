\documentclass{article}
\usepackage[backend=biber,natbib=true,style=authoryear,maxbibnames=50]{biblatex}
\addbibresource{/home/nqbh/reference/bib.bib}
\usepackage[utf8]{vietnam}
\usepackage{tocloft}
\renewcommand{\cftsecleader}{\cftdotfill{\cftdotsep}}
\usepackage[colorlinks=true,linkcolor=blue,urlcolor=red,citecolor=magenta]{hyperref}
\usepackage{amsmath,amssymb,amsthm,mathtools,float,graphicx,algpseudocode,algorithm,tcolorbox,tikz,tkz-tab,subcaption}
\DeclareMathOperator{\arccot}{arccot}
\usepackage[inline]{enumitem}
\allowdisplaybreaks
\numberwithin{equation}{section}
\newtheorem{assumption}{Assumption}[section]
\newtheorem{baitoan}{Bài toán}
\newtheorem{cauhoi}{Câu hỏi}[section]
\newtheorem{conjecture}{Conjecture}[section]
\newtheorem{corollary}{Corollary}[section]
\newtheorem{dangtoan}{Dạng toán}[section]
\newtheorem{definition}{Definition}[section]
\newtheorem{dinhly}{Định lý}[section]
\newtheorem{dinhnghia}{Định nghĩa}[section]
\newtheorem{example}{Example}[section]
\newtheorem{ghichu}{Ghi chú}[section]
\newtheorem{hequa}{Hệ quả}[section]
\newtheorem{hypothesis}{Hypothesis}[section]
\newtheorem{lemma}{Lemma}[section]
\newtheorem{luuy}{Lưu ý}[section]
\newtheorem{nhanxet}{Nhận xét}[section]
\newtheorem{notation}{Notation}[section]
\newtheorem{note}{Note}[section]
\newtheorem{principle}{Principle}[section]
\newtheorem{problem}{Problem}[section]
\newtheorem{proposition}{Proposition}[section]
\newtheorem{question}{Question}[section]
\newtheorem{remark}{Remark}[section]
\newtheorem{theorem}{Theorem}[section]
\newtheorem{vidu}{Ví dụ}[section]
\usepackage[left=0.5in,right=0.5in,top=1.5cm,bottom=1.5cm]{geometry}
\usepackage{fancyhdr}
\pagestyle{fancy}
\fancyhf{}
\lhead{\small Sect.~\thesection}
\rhead{\small\nouppercase{\leftmark}}
\renewcommand{\subsectionmark}[1]{\markboth{#1}{}}
\cfoot{\thepage}
\def\labelitemii{$\circ$}

\title{Triangle -- Tam Giác}
\author{Nguyễn Quản Bá Hồng\footnote{Independent Researcher, Ben Tre City, Vietnam\\e-mail: \texttt{nguyenquanbahong@gmail.com}; website: \url{https://nqbh.github.io}.}}
\date{\today}

\begin{document}
\maketitle
\begin{abstract}
	\textsc{[en]} This text is a collection of problems, from easy to advanced, about triangle. This text is also a supplementary material for my lecture note on Elementary Mathematics grade 7, which is stored \& downloadable at the following link: \href{https://github.com/NQBH/hobby/blob/master/elementary_mathematics/grade_7/NQBH_elementary_mathematics_grade_7.pdf}{GitHub\texttt{/}NQBH\texttt{/}hobby\texttt{/}elementary mathematics\texttt{/}grade 7\texttt{/}lecture}\footnote{\textsc{url}: \url{https://github.com/NQBH/hobby/blob/master/elementary_mathematics/grade_7/NQBH_elementary_mathematics_grade_7.pdf}.}. The latest version of this text has been stored \& downloadable at the following link: \href{https://github.com/NQBH/hobby/blob/master/elementary_mathematics/grade_7/triangle/NQBH_triangle.pdf}{GitHub\texttt{/}NQBH\texttt{/}hobby\texttt{/}elementary mathematics\texttt{/}grade 7\texttt{/}triangle}\footnote{\textsc{url}: \url{https://github.com/NQBH/hobby/blob/master/elementary_mathematics/grade_7/triangle/NQBH_triangle.pdf}.}.
	\vspace{2mm}
	
	\textsc{[vi]} Tài liệu này là 1 bộ sưu tập các bài tập chọn lọc từ cơ bản đến nâng cao về tam giác. Tài liệu này là phần bài tập bổ sung cho tài liệu chính -- bài giảng \href{https://github.com/NQBH/hobby/blob/master/elementary_mathematics/grade_7/NQBH_elementary_mathematics_grade_7.pdf}{GitHub\texttt{/}NQBH\texttt{/}hobby\texttt{/}elementary mathematics\texttt{/}grade 7\texttt{/}lecture} của tác giả viết cho Toán Sơ Cấp lớp 7. Phiên bản mới nhất của tài liệu này được lưu trữ \& có thể tải xuống ở link sau: \href{https://github.com/NQBH/hobby/blob/master/elementary_mathematics/grade_7/triangle/NQBH_triangle.pdf}{GitHub\texttt{/}NQBH\texttt{/}hobby\texttt{/}elementary mathematics\texttt{/}grade 7\texttt{/}triangle}.
	
	\textsf{\textbf{Nội dung.} Tổng các góc của 1 tam giác; quan hệ giữa góc \& cạnh đối diện trong 1 tam giác; bất đẳng thức tam giác; 2 tam giác bằng nhau; các trường hợp bằng nhau của 2 tam giác; tam giác cân; đường vuông góc \& đường xiên; đường trung trực của 1 đoạn thẳng; tính chất 3 đường trung tuyến, 3 đường phân giác, 3 đường trung trực, 3 đường cao của tam giác.}
\end{abstract}
\setcounter{secnumdepth}{4}
\setcounter{tocdepth}{3}
\tableofcontents

%------------------------------------------------------------------------------%

\section{Tổng Các Góc của 1 Tam Giác}giác
``\fbox{\bf 1} Tam giác $ABC$ là hình gồm 3 đoạn thẳng $AB,BC,CA$ khi 3 điểm $A,B,C$ không thẳng hàng. Tam giác $ABC$ được ký hiệu là $\Delta ABC$. 3 cạnh của tam giác: $AB,BC,CA$. 3 góc của tam giác: góc $A$, góc $B$, góc $C$. Nhận biết được điểm nằm trong \& điểm nằm ngoài 1 tam giác. \fbox{\bf 2} Tổng 3 góc của 1 tam giác bằng $180^\circ$: $\widehat{A} + \widehat{B} + \widehat{C} = 180^\circ$. \fbox{\bf 3} Tam giác vuông là tam giác có 1 góc vuông. Cạnh đối diện với góc vuông gọi là \textit{cạnh huyền}, cạnh huyền là cạnh lớn nhất trong tam giác vuông. 2 góc phụ nhau là 2 góc có tổng bằng $90^\circ$. Góc ngoài của 1 tam giác là góc kề bù với 1 góc trong của tam giác ấy. \fbox{\bf 4} 1 số hệ quả của định lý tổng 3 góc của tam giác:

\begin{hequa}
	Trong 1 tam giác vuông 2 góc nhọn phụ nhau. $\Delta ABC$, $\widehat{A} = 90^\circ\Rightarrow\widehat{B} + \widehat{C} = 90^\circ$.
\end{hequa}

\begin{hequa}
	1 góc ngoài của 1 tam giác bằng tổng 2 góc trong không kề với nó. 1 góc ngoài của 1 tam giác lớn hơn mỗi góc trong không kề với nó.
\end{hequa}
$\Delta ABC$, $\widehat{ACx}$ là góc ngoài tại đỉnh $C$: $\widehat{ACx} = \widehat{A} + \widehat{B}$, $\widehat{ACx} > \widehat{A}$, $\widehat{ACx} > \widehat{B}$ (có thể viết gộp 2 bất đẳng thức cuối thành $\widehat{ACx} > \min\{\widehat{A},\widehat{B}\}$ trong đó $\min\{a_1,a_2,\ldots,a_n\}$ là số nhỏ nhất trong $n$ số $a_i$, $i = 1,2,\ldots,n$). \fbox{\bf 5} Tam giác nhọn là tam giác có 3 góc nhọn. Tam giác tù là tam giác có 1 góc tù. Nếu 2 tam giác có 2 cặp góc bằng nhau từng đôi một thì cặp góc còn lại cũng bằng nhau.'' -- \cite[Chap. IV, \S1, p.65]{Tuyen_Toan_7}

\begin{baitoan}[\cite{SBT_Toan_7_Canh_Dieu_tap_2}, Ví dụ 1, p. 67]
	Tháp nghiêng Pisa ở Italy nghiêng $5^\circ$ so với phương thẳng đứng. Tính độ nghiêng của tháp đó so với phương nằm ngang.
\end{baitoan}

\begin{baitoan}[\cite{SBT_Toan_7_Canh_Dieu_tap_2}, \textbf{3.}, p. 68]
	(a) Cho biết 1 góc nhọn của tam giác vuông bằng $\alpha^\circ$, $\alpha\in(0,90)$. Tính số đo góc còn lại. (b) Cho 1 tam giác vuông có 2 góc bằng nhau. Tính số đo mỗi góc.
\end{baitoan}

\begin{baitoan}[\cite{SBT_Toan_7_Canh_Dieu_tap_2}, \textbf{4.}, p. 68]
	Đ hay S? Không có $\Delta ABC$ nào mà $\widehat{A} = 3\widehat{B}$, $\widehat{B} = 3\widehat{C}$, \& $C = 14^\circ$.
\end{baitoan}

\begin{baitoan}[\cite{Tuyen_Toan_7}, Ví dụ 15, p. 65]
	Cho 2 đường thẳng $a,b$ cắt nhau tại 1 điểm ở ngoài mép tờ giấy. Trong tay chỉ có thước đo góc, làm thế nào để đo được góc nhọn giữa 2 đường thẳng $a,b$ (đoạn thẳng $AB$ nằm trong góc đó).
\end{baitoan}
Để tính số đo 1 góc của tam giác ta lấy $180^\circ$ trừ đi tổng số đo của 2 góc còn lại.

\begin{baitoan}[\cite{Tuyen_Toan_7}, Ví dụ 16, p. 66]
	Cho $\Delta ABC$, các tia phân giác của góc $B$, góc $C$ cắt nhau tại $O$. Chứng minh: (a) $\widehat{BOC} = 90^\circ + \frac{\widehat{A}}{2}$; (b) Nếu $\widehat{BOC} = 135^\circ$ thì $\Delta ABC$ vuông tại $A$.
\end{baitoan}

\begin{baitoan}[\cite{Tuyen_Toan_7}, \textbf{58.}, p. 66]
	Cho $\Delta ABC$ vuông tại $A$. Trên tia đối tia $CA$ lấy điểm $E$ khác $C$. Gọi $D$ là hình chiếu vuông góc của $E$ lên đường thẳng $BC$. Chứng minh: $\widehat{B} = \widehat{CED}$.
\end{baitoan}

\begin{baitoan}[\cite{Tuyen_Toan_7}, \textbf{59.}, p. 66]
	Cho $\Delta ABC$ vuông tại $A$, $\widehat{C} = 25^\circ$. Tia phân giác của góc $A$ cắt $BC$ tại $D$. Vẽ $AH\bot BC$. Tính $\widehat{HAD}$.
\end{baitoan}

\begin{baitoan}[\cite{Tuyen_Toan_7}, \textbf{60.}, p. 66]
	Cho $\Delta ABC$ vuông tại $A$. Tia phân giác của góc $C$ cắt $AB$ tại $D$. (a) Chứng minh góc $BDC$ là góc tù. (b) Giả sử $\widehat{BDC} = 105^\circ$, tính $\widehat{B}$.
\end{baitoan}

\begin{baitoan}[\cite{Tuyen_Toan_7}, \textbf{61.}, p. 66]
	Cho $\Delta ABC$ \& điểm $O$ nằm trong tam giác đó. So sánh góc $BOC$ \& $BAC$.
\end{baitoan}

\begin{baitoan}[\cite{Tuyen_Toan_7}, \textbf{62.}, p. 66]
	Cho $\Delta ABC$ vuông tại $A$. Vẽ $AH\bot BC$. Vẽ các tia phân giác của góc $B$ \& góc $HAC$ cắt nhau tại $O$. Chứng minh $\Delta AOB$ là tam giác vuông.
\end{baitoan}

\begin{baitoan}[\cite{Tuyen_Toan_7}, \textbf{63.}, p. 66]
	Chứng minh với mỗi tam giác bao giờ cũng tồn tại 1 góc ngoài không lớn hơn $120^\circ$.
\end{baitoan}

\begin{baitoan}[\cite{Tuyen_Toan_7}, \textbf{64.}, pp. 66--67]
	Cho $\Delta ABC$ có $\widehat{B} > \widehat{C}$. Vẽ tia phân giác của góc $A$ cắt $BC$ tại $D$. (a) Chứng minh $\widehat{ADC} - \widehat{ADB} = \widehat{ABC} - \widehat{C}$. (b) Đường thẳng chứa tia phân giác ngoài tại đỉnh $A$ của $\Delta ABC$ cắt đường thẳng $BC$ tại $E$. Chứng minh $\widehat{AEB} = \frac{\widehat{ABC} - \widehat{C}}{2}$.
\end{baitoan}

\begin{baitoan}[\cite{Tuyen_Toan_7}, \textbf{65.}, p. 66]
	Trên lá cờ đỏ sao vàng của Việt Nam có ngôi sao $5$ cánh. Tính tổng các góc ở $5$ đỉnh của ngôi sao đó.
\end{baitoan}

%------------------------------------------------------------------------------%

\section{Quan Hệ Giữa Góc \& Cạnh Đối Diện. Bất Đẳng Thức Tam Giác}

\begin{baitoan}[\cite{SGK_Toan_7_Canh_Dieu_tap_2}, p. 75]
	Cho $\Delta ABC$ có $AB = 2$\emph{cm}, $BC = 4$\emph{cm}. So sánh $AB$ \& $AC$.
\end{baitoan}

\begin{baitoan}
	Đ hay sai? (a) Nếu 1 tam giác có 1 cạnh dài gấp đôi 1 cạnh khác, thì 2 cạnh đó lần lượt là cạnh dài nhất \& ngắn nhất của tam giác đó. (b) Nếu 1 tam giác có 1 cạnh dài hơn gấp đôi 1 cạnh khác, thì 2 cạnh đó lần lượt là cạnh dài nhất \& ngắn nhất của tam giác đó.
\end{baitoan}

%------------------------------------------------------------------------------%

\section{2 Tam Giác Bằng Nhau}

%------------------------------------------------------------------------------%

\section{Trường Hợp Bằng Nhau Thứ 1 của Tam Giác: Cạnh - Cạnh - Cạnh}

%------------------------------------------------------------------------------%

\section{Trường Hợp Bằng Nhau Thứ 2 của Tam Giác: Cạnh - Góc - Cạnh}

%------------------------------------------------------------------------------%

\section{Trường Hợp Bằng Nhau Thứ 3 của Tam Giác: Góc - Cạnh - Góc}

%------------------------------------------------------------------------------%

\section{Tam Giác Cân}

%------------------------------------------------------------------------------%

\section{Đường Vuông Góc \& Đường Xiên}

%------------------------------------------------------------------------------%

\section{Đường Trung Trực của 1 Đoạn Thẳng}

%------------------------------------------------------------------------------%

\section{Tính Chất 3 Đường Trung Tuyến của Tam Giác}

%------------------------------------------------------------------------------%

\section{Tính Chất 3 Đường Phân Giác của Tam Giác}

%------------------------------------------------------------------------------%

\section{Tính Chất 3 Đường Trung Trực của Tam Giác}

%------------------------------------------------------------------------------%

\section{Tính Chất 3 Đường Cao của Tam Giác}

%------------------------------------------------------------------------------%

\printbibliography[heading=bibintoc]
	
\end{document}