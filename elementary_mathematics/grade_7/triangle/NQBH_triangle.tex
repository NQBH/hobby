\documentclass{article}
\usepackage[backend=biber,natbib=true,style=authoryear,maxbibnames=50]{biblatex}
\addbibresource{/home/nqbh/reference/bib.bib}
\usepackage[utf8]{vietnam}
\usepackage{tocloft}
\renewcommand{\cftsecleader}{\cftdotfill{\cftdotsep}}
\usepackage[colorlinks=true,linkcolor=blue,urlcolor=red,citecolor=magenta]{hyperref}
\usepackage{amsmath,amssymb,amsthm,mathtools,float,graphicx,algpseudocode,algorithm,tcolorbox,tikz,tkz-tab,subcaption}
\DeclareMathOperator{\arccot}{arccot}
\usepackage[inline]{enumitem}
\allowdisplaybreaks
\numberwithin{equation}{section}
\newtheorem{assumption}{Assumption}[section]
\newtheorem{baitoan}{Bài toán}
\newtheorem{cauhoi}{Câu hỏi}[section]
\newtheorem{conjecture}{Conjecture}[section]
\newtheorem{corollary}{Corollary}[section]
\newtheorem{dangtoan}{Dạng toán}[section]
\newtheorem{definition}{Definition}[section]
\newtheorem{dinhly}{Định lý}[section]
\newtheorem{dinhnghia}{Định nghĩa}[section]
\newtheorem{example}{Example}[section]
\newtheorem{ghichu}{Ghi chú}[section]
\newtheorem{hequa}{Hệ quả}[section]
\newtheorem{hypothesis}{Hypothesis}[section]
\newtheorem{lemma}{Lemma}[section]
\newtheorem{luuy}{Lưu ý}[section]
\newtheorem{nhanxet}{Nhận xét}[section]
\newtheorem{notation}{Notation}[section]
\newtheorem{note}{Note}[section]
\newtheorem{principle}{Principle}[section]
\newtheorem{problem}{Problem}[section]
\newtheorem{proposition}{Proposition}[section]
\newtheorem{question}{Question}[section]
\newtheorem{remark}{Remark}[section]
\newtheorem{theorem}{Theorem}[section]
\newtheorem{vidu}{Ví dụ}[section]
\usepackage[left=0.5in,right=0.5in,top=1.5cm,bottom=1.5cm]{geometry}
\usepackage{fancyhdr}
\pagestyle{fancy}
\fancyhf{}
\lhead{\small Sect.~\thesection}
\rhead{\small\nouppercase{\leftmark}}
\renewcommand{\subsectionmark}[1]{\markboth{#1}{}}
\cfoot{\thepage}
\def\labelitemii{$\circ$}

\title{Triangle -- Tam Giác}
\author{Nguyễn Quản Bá Hồng\footnote{Independent Researcher, Ben Tre City, Vietnam\\e-mail: \texttt{nguyenquanbahong@gmail.com}; website: \url{https://nqbh.github.io}.}}
\date{\today}

\begin{document}
\maketitle
\begin{abstract}
	\textsf{\textbf{Nội dung.} Biểu thức số, biểu thức đại số; đa thức 1 biến, nghiệm của đa thức 1 biến; phép cộng, phép trừ đa thức 1 biến; phép nhân đa thức 1 biến; phép chia đa thức 1 biến.}
\end{abstract}
\setcounter{secnumdepth}{4}
\setcounter{tocdepth}{3}
\tableofcontents

%------------------------------------------------------------------------------%

\section{Tổng Các Góc của 1 Tam Giác}

\begin{baitoan}[\cite{SBT_Toan_7_Canh_Dieu_tap_2}, Ví dụ 1, p. 67]
	Tháp nghiêng Pisa ở Italy nghiêng $5^\circ$ so với phương thẳng đứng. Tính độ nghiêng của tháp đó so với phương nằm ngang.
\end{baitoan}

\begin{baitoan}[\cite{SBT_Toan_7_Canh_Dieu_tap_2}, \textbf{3.}, p. 68]
	(a) Cho biết 1 góc nhọn của tam giác vuông bằng $\alpha^\circ$, $\alpha\in(0,90)$. Tính số đo góc còn lại. (b) Cho 1 tam giác vuông có 2 góc bằng nhau. Tính số đo mỗi góc.
\end{baitoan}

\begin{baitoan}[\cite{SBT_Toan_7_Canh_Dieu_tap_2}, \textbf{4.}, p. 68]
	Đ hay S? Không có $\Delta ABC$ nào mà $\widehat{A} = 3\widehat{B}$, $\widehat{B} = 3\widehat{C}$, \& $C = 14^\circ$.
\end{baitoan}

%------------------------------------------------------------------------------%

\section{Quan Hệ Giữa Góc \& Cạnh Đối Diện. Bất Đẳng Thức Tam Giác}

\begin{baitoan}[\cite{SGK_Toan_7_Canh_Dieu_tap_2}, p. 75]
	Cho $\Delta ABC$ có $AB = 2$\emph{cm}, $BC = 4$\emph{cm}. So sánh $AB$ \& $AC$.
\end{baitoan}

\begin{baitoan}
	Đ hay sai? (a) Nếu 1 tam giác có 1 cạnh dài gấp đôi 1 cạnh khác, thì 2 cạnh đó lần lượt là cạnh dài nhất \& ngắn nhất của tam giác đó. (b) Nếu 1 tam giác có 1 cạnh dài hơn gấp đôi 1 cạnh khác, thì 2 cạnh đó lần lượt là cạnh dài nhất \& ngắn nhất của tam giác đó.
\end{baitoan}

%------------------------------------------------------------------------------%

\section{2 Tam Giác Bằng Nhau}

%------------------------------------------------------------------------------%

\section{Trường Hợp Bằng Nhau Thứ 1 của Tam Giác: Cạnh - Cạnh - Cạnh}

%------------------------------------------------------------------------------%

\section{Trường Hợp Bằng Nhau Thứ 2 của Tam Giác: Cạnh - Góc - Cạnh}

%------------------------------------------------------------------------------%

\section{Trường Hợp Bằng Nhau Thứ 3 của Tam Giác: Góc - Cạnh - Góc}

%------------------------------------------------------------------------------%

\section{Tam Giác Cân}

%------------------------------------------------------------------------------%

\section{Đường Vuông Góc \& Đường Xiên}

%------------------------------------------------------------------------------%

\section{Đường Trung Trực của 1 Đoạn Thẳng}

%------------------------------------------------------------------------------%

\section{Tính Chất 3 Đường Trung Tuyến của Tam Giác}

%------------------------------------------------------------------------------%

\section{Tính Chất 3 Đường Phân Giác của Tam Giác}

%------------------------------------------------------------------------------%

\section{Tính Chất 3 Đường Trung Trực của Tam Giác}

%------------------------------------------------------------------------------%

\section{Tính Chất 3 Đường Cao của Tam Giác}

%------------------------------------------------------------------------------%

\printbibliography[heading=bibintoc]
	
\end{document}