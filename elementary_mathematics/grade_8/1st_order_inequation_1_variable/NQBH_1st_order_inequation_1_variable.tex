\documentclass{article}
\usepackage[backend=biber,natbib=true,style=authoryear,maxbibnames=10]{biblatex}
\addbibresource{/home/nqbh/reference/bib.bib}
\usepackage[utf8]{vietnam}
\usepackage{tocloft}
\renewcommand{\cftsecleader}{\cftdotfill{\cftdotsep}}
\usepackage[colorlinks=true,linkcolor=blue,urlcolor=red,citecolor=magenta]{hyperref}
\usepackage{amsmath,amssymb,amsthm,float,graphicx,mathtools,soul}
\allowdisplaybreaks
\newtheorem{assumption}{Assumption}
\newtheorem{baitoan}{Bài toán}
\newtheorem{cauhoi}{Câu hỏi}
\newtheorem{conjecture}{Conjecture}
\newtheorem{corollary}{Corollary}
\newtheorem{dangtoan}{Dạng toán}
\newtheorem{definition}{Definition}
\newtheorem{dinhly}{Định lý}
\newtheorem{dinhnghia}{Định nghĩa}
\newtheorem{example}{Example}
\newtheorem{ghichu}{Ghi chú}
\newtheorem{hequa}{Hệ quả}
\newtheorem{hypothesis}{Hypothesis}
\newtheorem{lemma}{Lemma}
\newtheorem{luuy}{Lưu ý}
\newtheorem{menhde}{Mệnh đề}
\newtheorem{nhanxet}{Nhận xét}
\newtheorem{notation}{Notation}
\newtheorem{note}{Note}
\newtheorem{principle}{Principle}
\newtheorem{problem}{Problem}
\newtheorem{proposition}{Proposition}
\newtheorem{question}{Question}
\newtheorem{remark}{Remark}
\newtheorem{theorem}{Theorem}
\newtheorem{vidu}{Ví dụ}
\usepackage[left=1cm,right=1cm,top=5mm,bottom=5mm,footskip=4mm]{geometry}
\def\labelitemii{$\circ$}
\DeclareRobustCommand{\divby}{%
	\mathrel{\vbox{\baselineskip.65ex\lineskiplimit0pt\hbox{.}\hbox{.}\hbox{.}}}%
}

\title{1st-Order Inequation with 1 Variable -- Bất Phương Trình Bậc Nhất 1 Ẩn $ax + b >,<,\ge,\le0$}
\author{Nguyễn Quản Bá Hồng\footnote{Independent Researcher, Ben Tre City, Vietnam\\e-mail: \texttt{nguyenquanbahong@gmail.com}; website: \url{https://nqbh.github.io}.}}
\date{\today}

\begin{document}
\maketitle
\begin{abstract}
	\textsc{[en]} This text is a collection of problems, from easy to advanced, about \textit{1st-order inequation with 1 variable}. This text is also a supplementary material for my lecture note on Elementary Mathematics grade 8, which is stored \& downloadable at the following link: \href{https://github.com/NQBH/hobby/blob/master/elementary_mathematics/grade_8/NQBH_elementary_mathematics_grade_8.pdf}{GitHub\texttt{/}NQBH\texttt{/}hobby\texttt{/}elementary mathematics\texttt{/}grade 8\texttt{/}lecture}\footnote{\textsc{url}: \url{https://github.com/NQBH/hobby/blob/master/elementary_mathematics/grade_8/NQBH_elementary_mathematics_grade_8.pdf}.}. The latest version of this text has been stored \& downloadable at the following link: \href{https://github.com/NQBH/hobby/blob/master/elementary_mathematics/grade_8/1st_order_inequation_1_variable/NQBH_1st_order_inequation_1_variable.pdf}{GitHub\texttt{/}NQBH\texttt{/}hobby\texttt{/}elementary mathematics\texttt{/}grade 8\texttt{/}1st order inequation with 1 variable}\footnote{\textsc{url}: \url{https://github.com/NQBH/hobby/blob/master/elementary_mathematics/grade_8/1st_order_inequation_1_variable/NQBH_1st_order_inequation_1_variable.pdf}.}.
	\vspace{2mm}
	
	\textsc{[vi]} Tài liệu này là 1 bộ sưu tập các bài tập chọn lọc từ cơ bản đến nâng cao về \textit{bất phương trình bậc nhất 1 ẩn}. Tài liệu này là phần bài tập bổ sung cho tài liệu chính -- bài giảng \href{https://github.com/NQBH/hobby/blob/master/elementary_mathematics/grade_8/NQBH_elementary_mathematics_grade_8.pdf}{GitHub\texttt{/}NQBH\texttt{/}hobby\texttt{/}elementary mathematics\texttt{/}grade 8\texttt{/}lecture} của tác giả viết cho Toán Sơ Cấp lớp 8. Phiên bản mới nhất của tài liệu này được lưu trữ \& có thể tải xuống ở link sau: \href{https://github.com/NQBH/hobby/blob/master/elementary_mathematics/grade_8/1st_order_inequation_1_variable/NQBH_1st_order_inequation_1_variable.pdf}{GitHub\texttt{/}NQBH\texttt{/}hobby\texttt{/}elementary mathematics\texttt{/}grade 8\texttt{/}1st order inequation with 1 variable}.
\end{abstract}
\setcounter{secnumdepth}{4}
\setcounter{tocdepth}{3}
\tableofcontents
\newpage

%------------------------------------------------------------------------------%

\section{Liên Hệ Giữa Thứ Tự \& Phép Cộng}

\subsection{Thứ tự trên tập số thực $\mathbb{R}$}
Trên tập hợp số thực $\mathbb{R}$, khi so sánh 2 số $a,b\in\mathbb{R}$, xảy ra 1 trong 3 trường hợp sau: Số $a$ \textit{bằng} số $b$, ký hiệu $a = b$. Số $a$ \textit{nhỏ hơn} số $b$, ký hiệu $a < b$. Số $a$ \textit{lớn hơn} số $b$, ký hiệu $a > b$. Khi biểu diễn số thực trên trục số vẽ theo phương nằm ngang (hoặc thẳng đứng), điểm biểu diễn số nhỏ hơn ở bên trái (ở bên dưới) điểm biểu diễn số lớn hơn. Chính điều đó cho ta hình dung về thứ tự trên tập số thực $\mathbb{R}$.

Nếu số $a$ không nhỏ hơn số $b$, thì phải có hoặc $a > b$, hoặc $a = b$. Khi đó, ta nói gọn là $a$ \textit{lớn hơn hoặc bằng} $b$, ký hiệu $a\ge b$, e.g., $x^2\ge0$, $\forall x\in\mathbb{R}$; nếu $c\in\mathbb{R}$ là số không âm thì ta viết $c\ge0$. Nếu số $a$ không lớn hơn số $b$, thì phải có hoặc $a < b$, hoặc $a = b$. Khi đó, ta nói gọn là $a$ \textit{nhỏ hơn hoặc bằng} $b$, ký hiệu $a\le b$, e.g., $-x^2\le0$, $\forall x\in\mathbb{R}$; nếu số $y\in\mathbb{R}$ không lớn hơn $3$ thì ta viết $y\le3$.

\subsection{Bất đẳng thức}

\begin{dinhnghia}[Bất đẳng thức]
	Các hệ thức dạng $a < b$, $a > b$, $a\le b$, $a\ge b$h được gọi là \emph{bất đẳng thức} \& gọi $a$ là \emph{vế trái}, $b$ là \emph{vế phải} của bất đẳng thức.
\end{dinhnghia}

\subsection{Liên Hệ Giữa Thứ Tự \& Phép Cộng}

\begin{menhde}
	Khi cộng\emph{\texttt{/}}trừ cùng 1 số vào cả 2 vế của 1 bất đẳng thức ta được bất đẳng thức mới cùng chiều với bất đẳng thức đã cho: $a < b\Leftrightarrow a + c < b + c,\ a\le b\Leftrightarrow a + c\le b + c,\ a > b\Leftrightarrow a + c > b + c,\ a\ge b\Leftrightarrow a + c\ge b + c,\ \forall a,b,c,d\in\mathbb{R}$.
\end{menhde}
Có thể áp dụng tính chất trên để so sánh 2 số, hoặc chứng minh bất đẳng thức.

\begin{baitoan}[\cite{SGK_Toan_8_tap_2}, ?3, p. 36]
	So sánh $\sqrt{2} + 2$ \& $5$.
\end{baitoan}

\begin{proof}[Giải]
	$2 < 9\Rightarrow\sqrt{2} < \sqrt{9} = 3\Rightarrow\sqrt{2} + 2 < 3 + 2 = 5$. Vậy $\sqrt{2} + 2 < 5$.
\end{proof}

\begin{luuy}
	Tính chất của thứ tự cũng chính là tính chất của bất đẳng thức.
\end{luuy}

\begin{baitoan}[Tổng các bình phương thì không âm]
	Chứng minh: (a) $x^2 + a\ge a$, $\forall x,a\in\mathbb{R}$. (b) $\sum_{i=1}^n (a_ix + b_i)^2 = (a_1x + b_1)^2 + (a_2x + b_2)^2 + \cdots + (a_nx + b_n)^2\ge0$, $\forall x,a_i,b_i\in\mathbb{R}$, $\forall i = 1,2,\ldots,n$. (c) $\sum_{i=1}^n (a_ix + b_iy + c_i)^2 = (a_1x + b_1y + c_1)^2 + (a_2x + b_2y + c_2)^2 + \cdots + (a_nx + b_ny + c_n)^2\ge0$, $\forall x,y,a_i,b_i,c_i\in\mathbb{R}$, $\forall i = 1,2,\ldots,n$. (d) $\sum_{i=1}^m\left(\sum_{j=1}^n a_{ij}x_j + a_i\right)^2 = (a_{11}x_1 + a_{12}x_2 + \cdots + a_{1n}x_n + a_1)^2 + (a_{21}x_1 + a_{22}x_2 + \cdots + a_{2n}x_n + a_2)^2 + \cdots + (a_{m1}x_1 + a_{m2}x_2 + \cdots + a_{mn}x_n + a_m)^2\ge0$, $\forall a_{ij},a_i,x_j\in\mathbb{R}$, $\forall i = 1,2,\ldots,m$, $\forall j = 1,2,\ldots,n$. (e) 1 cách tổng quát, tổng các bình phương của các hàm nhiều biến thì không âm: $\sum_{i=1}^m (f_i(x_1,x_2,\ldots,x_m))^2\ge0$, $\forall x_i\in\mathbb{R}$, $\forall i = 1,\ldots,m$, với mọi hàm $m$ biến $f_i$, $\forall i = 1,2,\ldots,n$.
\end{baitoan}

\begin{baitoan}[\cite{SGK_Toan_8_tap_2}, 4., p. 37]
	1 biển báo giao thông với nền trắng, số $20$ màu đen, viền đỏ cho biết vận tốc tối đa mà các phương tiện giao thông được đi trên quãng đường có biển quy định là $20$\emph{km\texttt{/}h}. Nếu 1 ôtô đi trên đường đó có vận tốc là $a$\emph{km\texttt{/}h} thì $a$ phải thỏa mãn điều kiện nào trong các điều kiện sau: $a > 20$, $a < 20$, $a\le20$, $a\ge20$?
\end{baitoan}

%------------------------------------------------------------------------------%

\section{Liên Hệ Giữa Thứ Tự \& Phép Nhân}

\subsection{Liên hệ giữa thứ tự \& phép nhân với số dương \& số âm}

\begin{menhde}
	Khi nhân\emph{\texttt{/}}chia cả 2 vế của bất đẳng thức với cùng 1 số dương ta được bất đẳng thức mới cùng chiều với bất đẳng thức đã cho: $a < b\Leftrightarrow ac < bc$, $a\le b\Leftrightarrow ac\le bc$, $a > b\Leftrightarrow ac > bc$, $a\ge b\Leftrightarrow ac\ge bc$, $\forall a,b,c\in\mathbb{R}$, $c > 0$. Ngược lại, khi nhân\emph{\texttt{/}}chia cả 2 vế của bất đẳng thức với cùng 1 số âm ta được bất đẳng thức mới ngược chiều với bất đẳng thức đã cho: $a < b\Leftrightarrow ac > bc$, $a\le b\Leftrightarrow ac\ge bc$, $a > b\Leftrightarrow ac < bc$, $a\ge b\Leftrightarrow ac\le bc$, $\forall a,b,c\in\mathbb{R}$, $c < 0$.
\end{menhde}

\begin{proof}[Chứng minh]
	Dễ thấy nhờ xét dấu\footnote{$\operatorname{sign}(ac - bc) = \operatorname{sign}(c(a - b)) = \operatorname{sign}(c)\operatorname{sign}(a - b)$, $\forall a,b,c\in\mathbb{R}$.} của tích $ac - bc = c(a - b)$ theo dấu của $c$ \& dấu của $a - b$. 
\end{proof}

\subsection{Tính chất bắc cầu của thứ tự}

\begin{menhde}[Tính chất bắc cầu]
	($a < b$ \& $b < c$) $\Rightarrow a < c$, ($a > b$ \& $b > c$) $\Rightarrow a > c$, ($a\le b$ \& $b\le c$) $\Rightarrow a\le c$, ($a\ge b$ \& $b\ge c$) $\Rightarrow a\ge c$ $\forall a,b,c\in\mathbb{R}$.
\end{menhde}

\begin{baitoan}[\cite{SGK_Toan_8_tap_2}, Ví dụ, p. 39]
	Cho $a > b$. Chứng minh $a + 2 > b + 1$.
\end{baitoan}

\begin{proof}[1st chứng minh]
	Cộng 2 vào 2 vế của bất đẳng thức $a > b$, được: $a + 2 > b + 2$ (1). Cộng $b$ vào 2 vế của bất đẳng thức $2 > -1$, được: $b + 2 > b - 1$ (2). Từ (1) \& (2), theo tính chất bắc cầu, suy ra $a + 2 > b - 1$.
\end{proof}

\begin{proof}[2nd chứng minh]
	$a > b\Rightarrow a - b > 0$. Có $a + 2 - (b + 1) = a + 2 - b - 1 = a - b + 1\ge0 + 1 = 1 > 0$, suy ra $a + 2 > b - 1$.
\end{proof}

\begin{baitoan}[\cite{SGK_Toan_8_tap_2}, 6., p. 39]
	Cho $a < b$, so sánh: (a) $2a$ \& $2b$. (b) $2a$ \& $a + b$. (c) $-a$ \& $-b$.
\end{baitoan}

\begin{baitoan}[\cite{SGK_Toan_8_tap_2}, 7., p. 40]
	Số $a\in\mathbb{R}$ là số âm hay dương nếu: (a) $12a < 15a$? (b) $4a < 3a$? (c) $-3a > -5a$?
\end{baitoan}

\begin{baitoan}[\cite{SGK_Toan_8_tap_2}, 8., p. 40]
	Cho $a < b$. Chứng minh: (a) $2a - 3 < 2b - 3$. (b) $2a - 3 < 2b + 5$.
\end{baitoan}

\begin{baitoan}[\cite{SGK_Toan_8_tap_2}, 9., p. 40]
	Cho $\Delta ABC$. \emph{Đ\texttt{/}S?} (a) $\widehat{A} + \widehat{B} + \widehat{C} > 180^\circ$. (b) $\widehat{A} + \widehat{B} < 180^\circ$. (c) $\widehat{B} + \widehat{C}\le180^\circ$. (d) $\widehat{A} + \widehat{B}\ge180^\circ$.
\end{baitoan}

\begin{baitoan}[\cite{SGK_Toan_8_tap_2}, 11., p. 40]
	Cho $a < b$. Chứng minh: (a) $3a + 1 < 3b + 1$. (b) $-2a - 5 > -2b - 5$.
\end{baitoan}

\begin{baitoan}[\cite{SGK_Toan_8_tap_2}, 13., p. 40]
	So sánh $a,b$ nếu: (a) $a + 5 < b + 5$. (b) $-3a > -3b$. (c) $5a - 6\ge5b - 6$. (d) $-2a + 3\le-2b + 3$. 
\end{baitoan}

\begin{baitoan}[\cite{SGK_Toan_8_tap_2}, 14., p. 40]
	Cho $a < b$. So sánh: (a) $2a + 1$ với $2b + 1$. (b) $2a + 1$ với $2b + 3$.
\end{baitoan}
Bất đẳng thức giữa \textit{trung bình cộng} $\frac{a + b}{2}$ \& \textit{trung bình nhân} $\sqrt{ab}$:

\begin{dinhly}[Bất đẳng thức Cauchy\texttt{/}AM--GM cho 2 số]
	 $\frac{a + b}{2}\ge\sqrt{ab}$, $\forall a,b\in\mathbb{R}$, $a\ge0$, $b\ge0$. Đẳng thức xảy ra khi \& chỉ khi $a = b$.
\end{dinhly}

\begin{proof}[Chứng minh]
	$\frac{a + b}{2} - \sqrt{ab} = \frac{a = b - 2\sqrt{ab}}{2} = \frac{(\sqrt{a} - \sqrt{b})^2}{2}\ge0$, $\forall a,b\in\mathbb{R}$, $a\ge0$, $b\ge0$. ``$=$'' $\Leftrightarrow\sqrt{a} - \sqrt{b} = 0\Leftrightarrow\sqrt{a} = \sqrt{b}\Leftrightarrow a = b$.
\end{proof}

%------------------------------------------------------------------------------%

\section{Bất Phương Trình 1 Ẩn}
\textit{Phân biệt}: Phương trình 1 ẩn có dạng $f(x) = 0$, còn bất phương trình 1 ẩn có 1 trong 4 dạng sau: $f(x) < 0$, $f(x) > 0$, $f(x)\le0$, $f(x)\ge 0$ (i.e., thay dấu $=$ bởi các dấu so sánh $<,>,\le,\ge$).

\begin{baitoan}[\cite{SGK_Toan_8_tap_2}, p. 41]
	Nam có $25000$ đồng. Nam muốn mua 1 cái bút giá $4000$ đồng \& 1 số quyển vở loại $2200$ đồng\emph{\texttt{/}}quyển. Tính số quyển vở Nam có thể mua được.
\end{baitoan}

\begin{proof}[Giải]
	Gọi $x\in\mathbb{N}$ là số quyển vở Nam có thể mua thì $x$ phải thỏa mãn hệ thức $2200x + 4000\le25000\Leftrightarrow x\le\frac{25000 - 4000}{2200} = 9.(54)$, mà $x\in\mathbb{N}$, suy ra $x\in\{1,2,\ldots,9\}$. Vậy $S = \{1,2,\ldots,9\}$.
\end{proof}

\begin{baitoan}[\cite{SGK_Toan_8_tap_2}, ?1, p. 41]
	Giải bất phương trình bậc 2 $x^2\le6x - 5$.
\end{baitoan}

\begin{proof}[Giải]
	$x^2\le6x - 5\Leftrightarrow(x - 1)(x - 5)\le0\Leftrightarrow1\le x\le 5$. Vậy $S = [1,5] = \{x\in\mathbb{R}|1\le x\le5\}$.
\end{proof}

\subsection{Tập nghiệm của bất phương trình}

\begin{dinhnghia}[Tập nghiệm của bất phương trình, giải bất phương trình]
	Tập hợp tất cả các nghiệm của 1 bất phương trình được gọi là \emph{tập nghiệm} của bất phương trình đó. \emph{Giải bất phương trình} là tìm tập nghiệm của bất phương trình đó.
\end{dinhnghia}

\begin{baitoan}[Bất phương trình bậc nhất]
	Giải bất phương trình với $a,b,c\in\mathbb{R}$ cho trước: (a) $x < a$. (b) $x > a$. (c) $x\le a$. (d) $x\ge a$. (e) $ax + b < c$. (f) $ax + b > c$. (g) $ax + b\le c$. (h) $ax + b\ge c$.
\end{baitoan}

\begin{proof}[Giải]
	(a) $S = (-\infty,a)\coloneqq\{x\in\mathbb{R}|x < a\}$. (b) $S = (a,+\infty)\coloneqq\{x\in\mathbb{R}|x > a\}$. (c) $S = (-\infty,a]\coloneqq\{x\in\mathbb{R}|x\le a\}$. (d) $S = [a,+\infty)\coloneqq\{x\in\mathbb{R}|x\ge a\}$.
\end{proof}

\subsection{Bất phương trình tương đương}

\begin{dinhnghia}[Bất phương trình tương đương]
	2 bất phương trình có cùng tập nghiệm được gọi là \emph{2 bất phương trình tương đương} \& dùng ký hiệu ``$\Leftrightarrow$'' để chỉ sự tương đương đó.
\end{dinhnghia}

\begin{vidu}
	(a) $x < a\Leftrightarrow a > x$ vì có chung tập nghiệm $S = (-\infty,a)\coloneqq\{x\in\mathbb{R}|x < a\}$. (b) $x > a\Leftrightarrow a < x$ vì có chung tập nghiệm $S = (a,+\infty)\coloneqq\{x\in\mathbb{R}|x > a\}$. (c) $x\le a\Leftrightarrow a\ge x$ vì có chung tập nghiệm $S = (-\infty,a]\coloneqq\{x\in\mathbb{R}|x\le a\}$. (d) $x\ge a\Leftrightarrow a\le x$ vì có chung tập nghiệm $S = [a,+\infty)\coloneqq\{x\in\mathbb{R}|x\ge a\}$.
\end{vidu}

\begin{baitoan}[\cite{SGK_Toan_8_tap_2}, 18., p. 43]
	Quãng đường từ A đến B dài $50$\emph{km}. 1 ôtô đi từ A đến B, khởi hành lúc 7:00. Hỏi ôtô phải đi với vận tốc bao nhiêu \emph{km\texttt{/}h} để đến B trước 9:00 cùng ngày?
\end{baitoan}

%------------------------------------------------------------------------------%

\section{Bất Phương Trình Bậc Nhất 1 Ẩn}

%------------------------------------------------------------------------------%

\section{Phương Trình Chứa Dấu Giá Trị Tuyệt Đối}

%------------------------------------------------------------------------------%

\section{Miscellaneous}

%------------------------------------------------------------------------------%

\printbibliography[heading=bibintoc]
	
\end{document}