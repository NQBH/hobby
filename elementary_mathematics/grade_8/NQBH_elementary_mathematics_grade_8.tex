\documentclass{article}
\usepackage[backend=biber,natbib=true,style=authoryear]{biblatex}
\addbibresource{/home/hong/1_NQBH/reference/bib.bib}
\usepackage[utf8]{vietnam}
\usepackage{tocloft}
\renewcommand{\cftsecleader}{\cftdotfill{\cftdotsep}}
\usepackage[colorlinks=true,linkcolor=blue,urlcolor=red,citecolor=magenta]{hyperref}
\usepackage{amsmath,amssymb,amsthm,mathtools,float,graphicx,algpseudocode,algorithm,tcolorbox}
\usepackage[inline]{enumitem}
\allowdisplaybreaks
\numberwithin{equation}{section}
\newtheorem{assumption}{Assumption}[section]
\newtheorem{conjecture}{Conjecture}[section]
\newtheorem{corollary}{Corollary}[section]
\newtheorem{hequa}{Hệ quả}[section]
\newtheorem{definition}{Definition}[section]
\newtheorem{dinhnghia}{Định nghĩa}[section]
\newtheorem{example}{Example}[section]
\newtheorem{vidu}{Ví dụ}[section]
\newtheorem{lemma}{Lemma}[section]
\newtheorem{notation}{Notation}[section]
\newtheorem{principle}{Principle}[section]
\newtheorem{problem}{Problem}[section]
\newtheorem{baitoan}{Bài toán}[section]
\newtheorem{proposition}{Proposition}[section]
\newtheorem{menhde}{Mệnh đề}[section]
\newtheorem{question}{Question}[section]
\newtheorem{cauhoi}{Câu hỏi}[section]
\newtheorem{remark}{Remark}[section]
\newtheorem{luuy}{Lưu ý}[section]
\newtheorem{theorem}{Theorem}[section]
\newtheorem{dinhly}{Định lý}[section]
\usepackage[left=0.5in,right=0.5in,top=1.5cm,bottom=1.5cm]{geometry}
\usepackage{fancyhdr}
\pagestyle{fancy}
\fancyhf{}
\lhead{\small \textsc{Sect.} ~\thesection}
\rhead{\small \nouppercase{\leftmark}}
\renewcommand{\sectionmark}[1]{\markboth{#1}{}}
\cfoot{\thepage}
\def\labelitemii{$\circ$}

\title{Some Topics in Elementary Mathematics\texttt{/}Grade 8}
\author{Nguyễn Quản Bá Hồng\footnote{Independent Researcher, Ben Tre City, Vietnam\\e-mail: \texttt{nguyenquanbahong@gmail.com}; website: \url{https://nqbh.github.io}.}}
\date{\today}

\begin{document}
\maketitle
\setcounter{secnumdepth}{4}
\setcounter{tocdepth}{3}
\tableofcontents
\newpage

%------------------------------------------------------------------------------%

\section*{Ký Hiệu, Viết Tắt, Quy Ước -- Notation, Abbreviation, Convention}
\addcontentsline{toc}{section}{\protect\numberline{}Ký Hiệu, Viết Tắt, Quy Ước -- Notation, Abbreviation, Convention}

\subsection*{Ký Hiệu -- Notation}
\begin{itemize}
	\item $\land$: và, (logical) and.
	\item $\lor$: hoặc, (logical) or.
	\item $\Sigma$: tổng, sum, e.g., $\sum_{i=a}^b f(i) = f(a) + f(a + 1) + \cdots + f(b - 1) + f(b)$, $\forall a,b\in\mathbb{Z}$, $a\le b$.
	\item $\prod$: tích, product, e.g., $\prod_{i=a}^b f(i) = f(a)f(a + 1)\cdots f(b - 1)f(b)$, $\forall a,b\in\mathbb{Z}$, $a\le b$.
\end{itemize}

\subsection*{Viết Tắt -- Abbreviation}
\begin{itemize}
	\item \textbf{abbr.} (abbr., abbreviation): viết tắt, abbreviation, for short.
	\item \textbf{i.e.} stands for the Latin \textit{id est}, or `that is,' \& is used in front of a word or phrase that restates what has been said previously: tức là, nghĩa là, that is, that means, in another term.
	\item \textbf{e.g.} stands for \textit{exempli gratia} in Latin: ví dụ là, chẳng hạn, for example, for instance.
	\item \textbf{w.l.o.g.} (abbr., without loss of generality): không mất tính tổng quát.
\end{itemize}

\subsection*{Quy Ước -- Convention}

%------------------------------------------------------------------------------%

\section{Phép Nhân \& Phép Chia Các Đa Thức}

\subsection{Nhân Đơn Thức với Đa Thức}

\subsubsection{Quy tắc}
``Muốn nhân 1 đơn thức với 1 đa thức, ta nhân đơn thức với từng hạng tử của đa thức rồi cộng các tích với nhau.'' -- \cite[p. 4]{SGK_Toan_8_tap_1}.

\begin{vidu}[Đơn thức 1 biến nhân đa thức 1 biến]
	Phép nhân 1 đơn thức 1 biến $ax^m$ với 1 đa thức bậc $n$ được thực hiện như sau:
	\begin{align*}
		ax^m\sum_{i=0}^n a_ix^i &= ax^m\left(a_nx^n + a_{n-1}x^{n-1} + \cdots + a_1x + a_0\right)\\
		&= aa_nx^{m+n} + aa_{n-1}x^{m+ n-1} + \cdots + aa_1x^{m+1} + aa_0x^m,\ \forall a,a_i\in\mathbb{R},\,i = 0,\ldots,n,\,\forall m,n\in\mathbb{N}.
	\end{align*}
\end{vidu}

\begin{vidu}[Đơn thức $\le 2$ biến nhân đa thức $\le 2$ biến]
	Phép nhân 1 đơn thức 2 biến $ax^{m_1}y^{m_2}$ với 1 đa thức 2 biến được thực hiện như sau:
	\begin{align*}
		ax^{m_1}y^{m_2}\cdot\left(\sum_{i=0}^{n_1}\sum_{j=0}^{n_2} a_{ij}x^iy^j\right) = \sum_{i=0}^{n_1}\sum_{j=0}^{n_2} aa_{ij}x^{m_1 + i}y^{m_2 + j},\ \forall a,a_{ij}\in\mathbb{R},\,i = 1,\ldots,n_1,\,j = 0,\ldots,n_2,\ \forall m_i,n_i\in\mathbb{N},\,i = 1,2.
	\end{align*}
\end{vidu}
Tổng quát,
\begin{vidu}[Đơn thức $\le k$ biến nhân đa thức $\le k$ biến]
	Với $k\in\mathbb{N}$, $k\ge 2$ cho trước. Phép nhân 1 đơn thức $k$ biến $ax_1^{m_1}x_2^{m_2}\cdots x_k^{m_k} = a\prod_{i=1}^k x_i^{m_i}$ với 1 đa thức $k$ biến được thực hiện như sau:
	\begin{align*}
		a\prod_{i=1}^k x_i^{m_i}\left(\sum_{i_1 = 0}^{n_1}\ldots\sum_{i_k = 0}^{n_k} a_{i_1\ldots i_k}\prod_{j=1}^k x_j^{i_j}\right) &= ax_1^{m_1}\ldots x_k^{m_k}\sum_{i_1 = 0}^{n_1}\ldots\sum_{i_k = 0}^{n_k} a_{i_1\ldots i_k}x_1^{i_1}\ldots x_k^{i_k}\\
		&= \sum_{i_1 = 0}^{n_1}\ldots\sum_{i_k = 0}^{n_k} aa_{i_1\ldots i_k}x_1^{m_1 + i_1}\ldots x_k^{m_k + i_k},
	\end{align*}
	$\forall a,a_{i_1\ldots i_k}\in\mathbb{R}$, $i_1 = 0,\ldots,n_1;\ldots;i_k = 0,\ldots,n_k$, $\forall m_i,n_i\in\mathbb{N}$, $i = 1,\ldots,k$.\footnote{Điều kiện $i_1 = 0,\ldots,n_1;\ldots;i_k = 0,\ldots,n_k$ có thể viết gọn hơn thành $(i_1,\ldots,i_k)\in\overline{0,n_1}\times\cdots\times\overline{0,n_k}$ với ký hiệu $\overline{0,n}\coloneqq\{0,1,\ldots,n\}$, $\forall n\in\mathbb{N}$.}
\end{vidu}

\subsection{Nhân Đa Thức với Đa Thức}

\subsubsection{Quy tắc}
``Muốn nhân 1 đa thức với 1 đa thức, ta nhân mỗi hạng tử của đa thức này với từng hạng tử của đa thức kia rồi cộng các tích với nhau.'' ``Tích của 2 đa thức là 1 đa thức.'' -- \cite[p. 7]{SGK_Toan_8_tap_1}. Tổng quát, muốn nhân 2 đa thức bậc $P,Q$ lần lượt có bậc $m,n$ (ký hiệu $\deg P = m,\deg Q = n$), $P(x) = \sum_{i=0}^m a_ix^i$, $Q(x) = \sum_{i=0}^n b_ix^i$.

\subsection{Những Hằng Đẳng Thức Đáng Nhớ}

\subsubsection{Bình phương của 1 tổng -- Square of a sum}
\begin{align}
	\label{square of sum}
	\tag{sos}
	(a + b)^2 = a^2 + 2ab + b^2,\ \forall a,b\in\mathbb{R}.
\end{align}
``Với $a > 0$, $b > 0$, công thức này được minh họa bởi diện tích các hình vuông \& hình chữ nhật trong hình vuông với cạnh có độ dài $a + b$. Với $A$ \& $B$ là các biểu thức tùy ý, ta cũng có
\begin{align*}
	(A + B)^2 = A^2 + 2AB + B^2.
\end{align*}
'' -- \cite[p. 9]{SGK_Toan_8_tap_1}. Bình phương của 1 tổng 2 số bằng tổng của tổng bình phương 2 số đó với 2 lần tích 2 số đó.

\subsubsection{Bình phương của 1 hiệu -- square of a difference}
\begin{align}
	\label{square of difference}
	\tag{sod}
	(a - b)^2 = a^2 - 2ab + b^2,\ \forall a,b\in\mathbb{R}.
\end{align}
Đẳng thức \eqref{square of difference} có thể thu được trực tiếp từ đẳng thức \eqref{square of sum} bằng cách thay $b$ bởi $-b$. ``Với 2 biểu thức tùy ý $A$ \& $B$, ta cũng có:
\begin{align*}
	(A - B)^2 = A^2 - 2AB + B^2.
\end{align*}
Bình phương của 1 hiệu 2 số bằng hiệu của tổng bình phương 2 số đó với 2 lần tích 2 số đó. 

\subsubsection{Hiệu 2 bình phương -- Difference of 2 squares}
\begin{align}
	a^2 - b^2 = (a + b)(a - b),\ \forall a,b\in\mathbb{R}.
\end{align}
Với $A$ \& $B$ là các biểu thức tùy ý, ta cũng có:
\begin{align}
	A^2 - B^2 = (A + B)(A - B).
\end{align}

\begin{baitoan}[\cite{SGK_Toan_8_tap_1}, \textbf{23.}, p. 12]
	Chứng minh các đẳng thức sau:
	\begin{align*}
		(a + b)^2 = (a - b)^2 + 4ab,\ (a - b)^2 = (a + b)^2 - 4ab,\ \forall a,b\in\mathbb{R}.
	\end{align*}
\end{baitoan}

\begin{baitoan}[\cite{SGK_Toan_8_tap_1}, \textbf{25.}, p. 12]
	Tính
	\begin{enumerate*}
		\item[(a)] $(a + b + c)^2$;
		\item[(b)] $(a + b - c)^2$;
		\item[(c)] $a - b - c)^2$.
	\end{enumerate*}
\end{baitoan}
Tổng quát hơn,
\begin{baitoan}
	Với $n\in\mathbb{N}^\star$ cho trước, tính $\left(\sum_{i=1}^n a_i\right)^2 = (a_1 + \cdots + a_n)^2$, sau đó phát biểu đẳng thức tìm được bằng lời. Từ đó suy ra kết quả của $\left(\sum_{i=1}^n \pm a_i\right)^2 = (\pm a_1\pm\cdots\pm a_n)^2$.
\end{baitoan}

\subsubsection{Lập phương của 1 tổng -- Cube of a sum}
\begin{align}
	\label{cube of a sum}
	(a + b)^3 = a^3 + 3a^2b + 3ab^2 + b^3,\ \forall a,b\in\mathbb{R}.
\end{align}
Với $A$ \& $B$ là các biểu thức tùy ý ta cũng có:
\begin{align*}
	(A + B)^3 = A^3 + 3A^2B + 3AB^2 + B^3.
\end{align*}

\subsubsection{Lập phương của 1 hiệu -- Cube of a difference}
\begin{align}
	\label{cube of a difference}
	(a - b)^3 = a^3 - 3a^2b + 3ab^2 - b^3,\ \forall a,b\in\mathbb{R}.
\end{align}
Với $A$ \& $B$ là các biểu thức tùy ý ta cũng có:
\begin{align*}
	(A - B)^3 = A^3 - 3A^2B + 3AB^2 - B^3.
\end{align*}

\begin{luuy}
	Vì $x^{2n} = (-x)^{2n}$, $x^{2n + 1} = -(-x)^{2n + 1}$, $\forall x\in\mathbb{R}$, $\forall n\in\mathbb{N}$, nên
	\begin{align*}
		(a - b)^{2n} = (b - a)^{2n},\ (a - b)^{2n + 1} = -(b - a)^{2n + 1},\ \forall a,b\in\mathbb{R},\ \forall n\in\mathbb{N}.
	\end{align*}
\end{luuy}

\subsubsection{Tổng 2 lập phương -- Sum of cubes}
\begin{align}
	\label{sum of cubes}
	a^3 + b^3 = (a + b)(a^2 - ab + b^2),\ \forall a,b\in\mathbb{R}.
\end{align}
Với $A$ \& $B$ là các biểu thức tùy ý ta cũng có:
\begin{align*}
	A^3 + B^3 = (A + B)(A^2 - AB + B^2).
\end{align*}

\begin{luuy}
	``Ta quy ước gọi $A^2 - AB + B^2$ là \emph{bình phương thiếu của hiệu $A - B$}.'' -- \cite[p. 15]{SGK_Toan_8_tap_1}
\end{luuy}

\subsubsection{Hiệu 2 lập phương -- Difference of cubes}
\begin{align}
	\label{sum of cubes}
	a^3 - b^3 = (a - b)(a^2 + ab + b^2),\ \forall a,b\in\mathbb{R}.
\end{align}
Với $A$ \& $B$ là các biểu thức tùy ý ta cũng có:
\begin{align*}
	A^3 - B^3 = (A - B)(A^2 + AB + B^2).
\end{align*}

\begin{luuy}
	``Ta quy ước gọi $A^2 + AB + B^2$ là \emph{bình phương thiếu của hiệu $A + B$}.'' -- \cite[p. 15]{SGK_Toan_8_tap_1}
\end{luuy}

\begin{tcolorbox}
	\centering
	\textsc{7 hằng đẳng thức đáng nhớ.}
	\begin{align*}
			(A + B)^2 &= A^2 + 2AB + B^2,\\
			(A - B)^2 &= A^2 - 2AB + B^2,\\
			A^2 - B^2 &= (A + B)(A - B),\\
			(A + B)^3 &= A^3 + 3A^2B + 3AB^2 + B^3,\\
			(A - B)^3 &= A^3 - 3A^2B + 3AB^2 - B^3,\\
			A^3 + B^3 &= (A + B)(A^2 - AB + B^2),\\
			A^3 - B^3 &= (A - B)(A^2 + AB + B^2).
	\end{align*}
\end{tcolorbox}

\begin{baitoan}[\cite{SGK_Toan_8_tap_1}, \textbf{31.}, p. 16]
	Chứng minh rằng:
	\begin{align*}
		a^3 + b^3 = (a + b)^3 - 3ab(a + b),\ a^3 - b^3 = (a - b)^3 + 3ab(a - b),\ \forall a,b\in\mathbb{R}.
	\end{align*}
	Áp dụng: Tính $a^3 + b^3$ biết $ab = m$ \& $a + b = n$ với $m,n\in\mathbb{R}$ cho trước. Tính $a^3 - b^3$ biết $ab = m$ \& $a - b = k$ với $m,k\in\mathbb{R}$ cho trước.
\end{baitoan}

\subsection{Phân Tích Đa Thức Thành Nhân Tử Bằng Phương Pháp Đặt Nhân Tử Chung}

\begin{dinhnghia}[Phân tích đa thức thành nhân tử]
	\emph{Phân tích đa thức thành nhân tử (hay thừa số)} là biến đổi đa thức đó thành 1 tích của những đa thức.	
\end{dinhnghia}
Phân tích đa thức thành nhân tử bằng phương pháp đặt nhân tử chung. ``Nhiều khi để làm xuất hiện nhân tử chung ta cần đổi dấu các hạng tử (lưu ý tới tính chất $A = -(-A)$).'' -- \cite[p. 18]{SGK_Toan_8_tap_1}

\subsection{Phân Tích Đa Thức Thành Nhân Tử Bằng Phương Pháp Dùng Hằng Đẳng Thức}
Phân tích đa thức thành nhân tử bằng phương pháp dùng hằng đẳng thức.

\subsection{Phân Tích Đa Thức Thành Nhân Tử Bằng Phương Pháp Nhóm Hạng Tử}
Phân tích đa thức thành nhân tử bằng phương pháp nhóm hạng tử. ``Đối với 1 đa thức có thể có nhiều cách nhóm những hạng tử thích hợp.'' -- \cite[p. 21]{SGK_Toan_8_tap_1}

\subsection{Phân Tích Đa Thức Thành Nhân Tử Bằng Cách Phối Hợp Nhiều Phương Pháp}

\begin{baitoan}[\cite{SGK_Toan_8_tap_1}, \textbf{58.}, p. 25]
	Chứng minh rằng $n^3 - n\ \vdots\ 6$, $\forall n\in\mathbb{Z}$.
\end{baitoan}

\subsection{Chia Đơn Thức Cho Đơn Thức}
``Cho $A$ \& $B$ là 2 đa thức, $B\ne 0$. Ta nói đa thức $A$ chia hết cho đa thức $B$ nếu tìm được 1 đa thức $Q$ sao cho $A = B\cdot Q$. $A$ được gọi là \textit{đa thức bị chia}, $B$ được gọi là đa thức chia, $Q$ được gọi là \textit{đa thức thương} (gọi tắt \textit{thương}). Ký hiệu $Q = A:B$ hoặc $Q = \frac{A}{B}$. Trong \cite[\S10]{SGK_Toan_8_tap_1}, ta xét trường hợp đơn giản nhất của phép chia 2 đa thức, đó là phép chia đơn thức cho đơn thức.'' -- \cite[p. 25]{SGK_Toan_8_tap_1}

\subsubsection{Quy tắc}
``Ở lớp 7 ta đã biết:
\begin{equation*}
	x^m:x^n = \left\{\begin{split}
		&x^{m - n}&&\mbox{ nếu } m > n,\\
		&1&&\mbox{ nếu } m = n.
	\end{split}\right.\ \forall x\in\mathbb{R}\backslash\{0\},\ \forall m,n\in\mathbb{N},\ m\ge n.
\end{equation*}
Đơn thức $A$ chia hết cho đơn thức $B$ khi mỗi biến của $B$ đều là biến của $A$ với số mũ không lớn hơn số mũ của nó trong $A$.

\textbf{Quy tắc.} Muốn chia đơn thức $A$ cho đơn thức $B$ (trường hợp $A$ chia hết cho $B$) ta làm như sau:
\begin{itemize}
	\item Chia hệ số của đơn thức $A$ cho hệ số của đơn thức $B$.
	\item Chia lũy thừa của từng biến trong $A$ cho lũy thừa của cùng biến đó trong $B$.
	\item Nhân các kết quả vừa tìm được với nhau.
\end{itemize}

\begin{vidu}[Chia 2 đơn thức 1 biến]
	\begin{align*}
		ax^m:bx^n = \frac{a}{b}x^{m-n},\ \forall a,b\in\mathbb{R},\,b\ne 0,\ \forall m,n\in\mathbb{N},\,m\ge n.
	\end{align*}
\end{vidu}

\begin{vidu}[Chia 2 đơn thức 2 biến]
	\begin{align*}
		ax^{m_1}y^{m_2}:bx^{n_1}y^{n_2} = \frac{a}{b}x^{m_1 - n_1}y^{m_2 - n_2},\ \forall a,b\in\mathbb{R},\,b\ne 0,\ \forall m_i,n_i\in\mathbb{N},\,m_i\ge n_i,\,i = 1,2.
	\end{align*}
\end{vidu}

\begin{vidu}[Chia 2 đơn thức $k$ biến]
	Cho $k\in\mathbb{N}$, $k\ge 2$.
	\begin{align*}
		ax_1^{m_1}\cdots x_k^{m_k}:bx_1^{n_1}\cdots x_k^{n_k} = \frac{a}{b}x_1^{m_1 - n_1}\cdots x_k^{m_k - n_k},\ \forall a,b\in\mathbb{R},\,b\ne 0,\ \forall m_i,n_i\in\mathbb{N},,\,m_i\ge n_i,\,i = 1,\ldots,k.
	\end{align*}
\end{vidu}

\subsection{Chia Đa Thức Cho Đơn Thức}
``Ta có quy tắc chia đa thức cho đơn thức (trường hợp các hạng tử của đa thức $A$ đều chia hết cho đơn thức $B$) như sau:

\textbf{Quy tắc.} Muốn chia đa thức $A$ cho đơn thức $B$ (trường hợp các hạng tử của đa thức $A$ đều chia hết cho đơn thức $B$), ta chia mỗi hạng tử của $A$ cho $B$ rồi cộng các kết quả với nhau.'' -- \cite[p. 27]{SGK_Toan_8_tap_1}. ``Trong thực hành ta có thể tính nhẩm \& bỏ bớt 1 số phép tính trung gian.'' -- \cite[p. 28]{SGK_Toan_8_tap_1}

\subsection{Chia Đa Thức 1 Biến Đã Sắp Xếp}

\subsubsection{Phép chia hết}
``Phép chia có dư bằng $0$ là phép chia hết.'' -- \cite[p. 30]{SGK_Toan_8_tap_1}

\subsubsection{Phép chia có dư}
``Người ta chứng minh được rằng đối với 2 đa thức tùy ý $A$ \& $B$ của cùng 1 biến ($B\ne 0$), tồn tại duy nhất 1 cặp đa thức $Q$ \& $R$ sao cho $A = B\cdot Q + R$, trong đó $R$ bằng 0 hoặc bậc của $R$ nhỏ hơn bậc của $B$ ($R$ được gọi là \textit{dư} trong phép chia $A$ cho $B$). Khi $R = 0$ phép chia $A$ cho $B$ là phép chia hết.'' -- \cite[p. 31]{SGK_Toan_8_tap_1}

%------------------------------------------------------------------------------%

\section{Phân Thức Đại Số}

\begin{align*}
	\frac{a}{b}\in\mathbb{Q},\ a,b\in\mathbb{Z},\,b\ne 0\longrightarrow\frac{A(x)}{B(x)},\ \forall A(x),B(x)\mbox{ là 2 đa thức 1 biến } x,\,B(x)\ne 0.
\end{align*}
``Ở lớp 7 ta đã biết, từ tập hợp các số nguyên $\mathbb{Z}$ ta thiết lập được tập hợp các số hữu tỷ $\mathbb{Q}$. Khi đó, mỗi số nguyên cũng là 1 số hữu tỷ. Tương tự, bây giờ từ tập hợp các đa thức ta sẽ thiết lập 1 tập hợp mới gồm những biểu thức gọi là những \textit{phân  thức đại số}.''

\begin{quotation}
	\textbf{Nội dung.} \textit{Phân thức đại số, các quy tắc làm tính trên các phân thức đại số\footnote{Những quy tắc này tương tự như các quy tắc làm tính trên các phân số}}.
\end{quotation}

\subsection{Phân Thức Đại Số}

\subsubsection{Định nghĩa}

\begin{dinhnghia}[Phân thức đại số]
	\emph{1 phân thức đại số} (hay nói gọn là \emph{phân thức}) là 1 biểu thức có dạng $\frac{A}{B}$, trong đó $A,B$ là những đa thức \& $B$ khác đa thức $0$. $A$ được gọi là \emph{tử thức} (hay \emph{tử}), $B$ được gọi là \emph{mẫu thức} (hay \emph{mẫu}).
\end{dinhnghia}
``Mỗi đa thức cũng được coi như 1 phân thức với mẫu thức bằng $1$.'' ``Số $0$, số $1$ cũng là những phân thức đại số.'' -- \cite[p. 35]{SGK_Toan_8_tap_1}

\subsubsection{2 phân thức bằng nhau}

\begin{dinhnghia}[2 phân thức bằng nhau]
	2 phân thức $\frac{A}{B}$ \& $\frac{C}{D}$ gọi là \emph{bằng nhau} nếu $A\cdot D = B\cdot C$. Ta viết
	\begin{align*}
		\frac{A}{B} = \frac{C}{D}\mbox{ nếu } A\cdot D = B\cdot C.
	\end{align*}
\end{dinhnghia}

\subsection{Tính Chất Cơ Bản của Phân Thức}
``Phân thức đại số có tính chất cơ bản sau:

\begin{menhde}
	Nếu nhân cả tử \& mẫu của 1 phân thức với cùng 1 đa thức khác đa thức $0$ thì được 1 phân thức bằng phân thức đã cho:
	\begin{align*}
		\frac{A}{B} = \frac{A\cdot M}{B\cdot M}\ (M\mbox{ là 1 đa thức khác đa thức } 0).
	\end{align*}
	Nếu chia cả tử \& mẫu của 1 phân thức cho 1 nhân tử chung của chúng thì được 1 phân thức bằng phân thức đã cho:
	\begin{align*}
		\frac{A}{B} = \frac{A:N}{B:N}\ (N\mbox{ là 1 nhân tử chung}).
	\end{align*}
\end{menhde}
Tính chất này được gọi là \textit{tính chất cơ bản của phân thức}.'' -- \cite[p. 37]{SGK_Toan_8_tap_1}

\subsubsection{Quy tắc đổi dấu}

\begin{menhde}
	Nếu đổi dấu cả tử \& mẫu của 1 phân thức thì được 1 phân thức bằng phân thức đã cho:
	\begin{align*}
		\frac{A}{B} = \frac{-A}{-B}.
	\end{align*}
\end{menhde}

\subsection{Rút Gọn Phân Thức}
``Muốn rút gọn 1 phân thức ta có thể:
\begin{enumerate*}
	\item Phân tích tử \& mẫu thành nhân tử (nếu cần) để tìm nhân tử chung;
	\item Chia cả tử \& mẫu cho nhân tử chung.''
\end{enumerate*}
``Có khi cần đổi dấu ở tử hoặc mẫu để nhận ra nhân tử chugn của tử \& mẫu (lưu ý tới tính chất $A = -(-A)$).'' -- \cite[p. 39]{SGK_Toan_8_tap_1}

\subsection{Quy Đồng Mẫu thức Nhiều Phân Thức}

\begin{dinhnghia}[Quy đồng mẫu thức nhiều phân thức]
	\emph{Quy đồng mẫu thức nhiều phân thức} là biến đổi các phân thức đã cho thành những phân thức mới có cùng mẫu thức \& lần lượt bằng các phân thức đã cho.
\end{dinhnghia}
Ta thường ký hiệu ``mẫu thức chung'' bởi MTC. Để quy đồng mẫu thức nhiều phân thức, trước hết ta hãy xem có thể tìm mẫu thức chung của những phân thức mới này như thế nào.'' -- \cite[p. 41]{SGK_Toan_8_tap_1}

\subsubsection{Tìm mẫu thức chung}
``Có thể chọn mẫu thức chung là 1 tích chia hết cho mẫu thức của mỗi phân thức đã cho.'' -- \cite[p. 41]{SGK_Toan_8_tap_1}

``Khi quy đồng mẫu thức nhiều phân thức, muốn tìm mẫu thức chung ta có thể làm như sau: 
\begin{enumerate}
	\item Phân tích mẫu thức của các phân thức đã cho thành nhân tử;
	\item Mẫu thức chung cần tìm là 1 tích mà các nhân tử được chọn như sau:
	\begin{itemize}
		\item Nhân tử bằng số của mẫu thức chung là tích các nhân tử bằng số ở các mẫu thức của các phân thức đã cho. (Nếu các nhân tử bằng số ở các mẫu thức là những số nguyên dương thì nhân tử bằng số của mẫu thức chung là BCNN của chúng);
		\item Với mỗi lũy thừa của cùng 1 biểu thức có mặt trong các biểu thức, ta chọn lũy thừa với số mũ cao nhất.'' -- \cite[p. 42]{SGK_Toan_8_tap_1}		
	\end{itemize}
\end{enumerate}

\subsubsection{Quy đồng mẫu thức}
``Muốn quy đồng mẫu thức nhiều phân thức ta có thể làm như sau:
\begin{enumerate*}
	\item Phân tích các mẫu thức thành nhân tử rồi tìm mẫu thức chung;
	\item Tìm nhân tử phụ của mỗi mẫu thức;
	\item Nhân cả tử \& mẫu của mỗi phân thức với nhân tử phụ tương ứng.'' -- \cite[p. 42]{SGK_Toan_8_tap_1}
\end{enumerate*}

\subsection{Phép Cộng Các Phân Thức Đại Số}

\subsubsection{Cộng 2 phân thức cùng mẫu thức}
``Quy tắc cộng 2 phân thức cũng tương tự như quy tắc cộng 2 phân số.

\textbf{Quy tắc.} Muốn cộng 2 phân thức có cùng mẫu thức, ta cộng các tử thức với nhau \& giữ nguyên mẫu thức.'' -- \cite[p. 44]{SGK_Toan_8_tap_1}

\subsubsection{Cộng 2 phân thức có mẫu thức khác nhau}
``Ta đã biết quy đồng mẫu thức 2 phân thức \& quy tắc cộng 2 phân thức cùng mẫu thức. Có thể áp dụng những điều đó để cộng 2 phân thức có mẫu thức khác nhau.'' ``Ta có quy tắc cộng 2 phân thức có mẫu thức khác nhau như sau:

\textbf{Quy tắc.} Muốn cộng 2 phân thức có mẫu thức khác nhau, ta quy đồng mẫu thức rồi cộng các phân thức có cùng mẫu thức vừa tìm được.

Kết quả của phép cộng 2 phân thức được gọi là \textit{tổng} của 2 phân thức ấy. Ta thường viết tổng này dưới dạng rút gọn.'' ``Phép cộng các phân thức cũng có các tính chất sau:
\begin{enumerate}
	\item Giao hoán: $\frac{A}{B} + \frac{C}{D} = \frac{C}{D} + \frac{A}{B}$;
	\item Kết hợp: $\left(\frac{A}{B} + \frac{C}{D}\right) + \frac{E}{F} = \frac{A}{B} + \left(\frac{C}{D} + \frac{E}{F}\right)$.
\end{enumerate}
Nhờ tính chất kết hợp, trong 1 dãy phép cộng nhiều phân thức, ta không cần đặt dấu ngoặc.'' -- \cite[p. 45]{SGK_Toan_8_tap_1}

\subsection{Phép Trừ Các Phân Thức Đại Số}

\subsection{Phép Nhân Các Phân Thức Đại Số}

\subsection{Phép Chia Các Phân Thức Đại Số}

\subsection{Biến Đổi Các Biểu Thức Hữu Tỷ. Giá Trị của Phân Thức}

%------------------------------------------------------------------------------%

\section{Phương Trình Đại Số 1 Ẩn -- Algebraic Equation with 1 Unknown}

\subsection{Mở Đầu về Phương Trình}

\subsection{Phương Trình Bậc Nhất 1 Ẩn \& Cách Giải}

\subsection{Phương Trình Đưa Được về Dạng $ax + b = 0$}

\subsection{Phương Trình Tích}

\subsection{Phương Trình Chứa Ẩn ở Mẫu}

\subsection{Giải Bài Toán Bằng Cách Lập Phương Trình}

%------------------------------------------------------------------------------%

\section{Bất Phương Trình Bậc Nhất 1 Ẩn -- Algebraic Inequation with 1 Unknown}

\subsection{Liên Hệ Giữa Thứ Tự \& Phép Cộng}

\subsection{Liên Hệ Giữa Thứ Tự \& Phép Nhân}

\subsection{Bất Phương Trình 1 Ẩn}

\subsection{Bất Phương Trình Bậc Nhất 1 Ẩn}

\subsection{Phương Trình Chứa Dấu Giá Trị Tuyệt Đối}

%------------------------------------------------------------------------------%

\section{Tứ Giác}

\subsection{Tứ Giác}

\subsection{Hình Thang}

\subsection{Hình Thang Cân}

\subsection{Đường Trung Bình của Tam Giác, của Hình Thang}

\subsection{Dựng Hình Bằng Thước \& Compa. Dựng Hình thang}

\subsection{Đối Xứng Trục}

\subsection{Hình Bình Hành}

\subsection{Đối Xứng Tâm}

\subsection{Hình Chữ Nhật}

\subsection{Đường Thẳng Song Song với 1 Đường Thẳng Cho Trước}

\subsection{Hình Thoi}

\subsection{Hình Vuông}

%------------------------------------------------------------------------------%

\section{Đa Giác. Diện Tích Đa Giác}

\subsection{Đa Giác. Đa Giác Đều}

\subsection{Diện Tích Hình Chữ Nhật}

\subsection{Diện Tích Tam Giác}

\subsection{Diện Tích Hình Thang}

\subsection{Diện Tích Hình Thoi}

\subsection{Diện Tích Đa Giác}

%------------------------------------------------------------------------------%

\section{Tam Giác Đồng Dạng}

\subsection{Định Lý Thales Trong Tam Giác}

\subsection{Định Lý Đảo \& Hệ Quả của Định Lý Thales}

\subsection{Tính Chất Đường Phân Giác của Tam Giác}

\subsection{Khái Niệm 2 Tam Giác Đồng Dạng}

\subsection{Trường Hợp Đồng Dạng Thứ Nhất}

\subsection{Trường Hợp Đồng Dạng Thứ 2}

\subsection{Trường Hợp Đồng Dạng Thứ 3}

\subsection{Các Trường Hợp Đồng Dạng của Tam Giác Vuông}

\subsection{Ứng Dụng Thực Tế của Tam Giác Đồng Dạng}

%------------------------------------------------------------------------------%

\section{Hình Lăng Trụ Đứng. Hình Chóp Đều}

\begin{center}
	\Large A -- Hình Lăng Trụ Đứng
\end{center}

\subsection{Hình Hộp Chữ Nhật}

\subsection{Thể Tích của Hình Hộp Chữ Nhật}

\subsection{Hình Lăng Trụ Đứng}

\subsection{Diện Tích Xung Quanh của Hình Lăng Trụ Đứng}

\subsection{Thể Tích của Hình Lăng Trụ Đứng}

\begin{center}
	\Large B -- Hình Chóp Đều
\end{center}

\subsection{Hình Chóp Đều \& Hình Chóp Cụt Đều}

\subsection{Diện Tích Xung Quanh của Hình Chóp Đều}

\subsection{Thể Tích của Hình Chóp Đều}

%------------------------------------------------------------------------------%

\begin{thebibliography}{99}
	\bibitem[NQBH\texttt{/}elementary math]{NQBH/elementary math} Nguyễn Quản Bá Hồng. \href{https://github.com/NQBH/hobby/blob/master/elementary_mathematics/NQBH_elementary_mathematics.pdf}{\textit{Some Topics in Elementary Mathematics: Problems, Theories, Applications, \textit{\&} Bridges to Advanced Mathematics}}. Mar 2022--now.
\end{thebibliography}

%------------------------------------------------------------------------------%

\printbibliography[heading=bibintoc]
	
\end{document}