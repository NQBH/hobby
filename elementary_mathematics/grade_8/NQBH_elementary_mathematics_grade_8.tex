\documentclass[oneside]{book}
\usepackage[backend=biber,natbib=true,style=authoryear]{biblatex}
\addbibresource{/home/hong/1_NQBH/reference/bib.bib}
\usepackage[utf8]{vietnam}
\usepackage{tocloft}
\renewcommand{\cftsecleader}{\cftdotfill{\cftdotsep}}
\usepackage[colorlinks=true,linkcolor=blue,urlcolor=red,citecolor=magenta]{hyperref}
\usepackage{amsmath,amssymb,amsthm,mathtools,float,graphicx,algpseudocode,algorithm,tcolorbox}
\usepackage[inline]{enumitem}
\allowdisplaybreaks
\numberwithin{equation}{section}
\newtheorem{assumption}{Assumption}[section]
\newtheorem{conjecture}{Conjecture}[section]
\newtheorem{corollary}{Corollary}[section]
\newtheorem{hequa}{Hệ quả}[section]
\newtheorem{definition}{Definition}[section]
\newtheorem{dinhnghia}{Định nghĩa}[section]
\newtheorem{example}{Example}[section]
\newtheorem{vidu}{Ví dụ}[section]
\newtheorem{lemma}{Lemma}[section]
\newtheorem{notation}{Notation}[section]
\newtheorem{principle}{Principle}[section]
\newtheorem{problem}{Problem}[section]
\newtheorem{baitoan}{Bài toán}[section]
\newtheorem{proposition}{Proposition}[section]
\newtheorem{question}{Question}[section]
\newtheorem{cauhoi}{Câu hỏi}[section]
\newtheorem{remark}{Remark}[section]
\newtheorem{luuy}{Lưu ý}[section]
\newtheorem{theorem}{Theorem}[section]
\newtheorem{dinhly}{Định lý}[section]
\usepackage[left=0.5in,right=0.5in,top=1.5cm,bottom=1.5cm]{geometry}
\usepackage{fancyhdr}
\pagestyle{fancy}
\fancyhf{}
\lhead{\small \textsc{Sect.} ~\thesection}
\rhead{\small \nouppercase{\leftmark}}
\renewcommand{\sectionmark}[1]{\markboth{#1}{}}
\cfoot{\thepage}
\def\labelitemii{$\circ$}

\title{Some Topics in Elementary Mathematics\texttt{/}Grade 8}
\author{Nguyễn Quản Bá Hồng\footnote{Independent Researcher, Ben Tre City, Vietnam\\e-mail: \texttt{nguyenquanbahong@gmail.com}; website: \url{https://nqbh.github.io}.}}
\date{\today}

\begin{document}
\frontmatter
\maketitle
\setcounter{secnumdepth}{4}
\setcounter{tocdepth}{3}
\tableofcontents
\newpage

%------------------------------------------------------------------------------%

\part{Đại Số -- Algebra}

\chapter{Phép Nhân \& Phép Chia Các Đa Thức}

\section{Nhân Đơn Thức với Đa Thức}

\section{Nhân Đa Thức với Đa Thức}

\section{Những Hằng Đẳng Thức Đáng Nhớ}

\section{Phân Tích Đa Thức Thành Nhân Tử Bằng Phương Pháp Đặt Nhân Tử Chung}

\section{Phân Tích Đa Thức Thành Nhân Tử Bằng Phương Pháp Dùng Hằng Đẳng Thức}

\section{Phân Tích Đa Thức Thành Nhân Tử Bằng Phương Pháp Nhóm Hạng Tử}

\section{Phân Tích Đa Thức Thành Nhân Tử Bằng Cách Phối Hợp Nhiều Phương Pháp}

\section{Chia Đơn Thức Cho Đơn Thức}

\section{Chia Đa Thức Cho Đơn Thức}

\section{Chia Đa Thức 1 Biến Đã Sắp Xếp}

%------------------------------------------------------------------------------%

\chapter{Phân Thức Đại Số}

\section{Phân Thức Đại Số}

\section{Tính Chất Cơ Bản của Phân Thức}

\section{Rút Gọn Phân Thức}

\section{Quy Đồng Mẫu thức Nhiều Phân Thức}

\section{Phép Cộng Các Phân Thức Đại Số}

\section{Phép Trừ Các Phân Thức Đại Số}

\section{Phép Nhân Các Phân Thức Đại Số}

\section{Phép Chia Các Phân Thức Đại Số}

\section{Biến Đổi Các Biểu Thức Hữu Tỷ. Giá Trị của Phân Thức}

%------------------------------------------------------------------------------%

\chapter{Phương Trình Đại Số 1 Ẩn -- Algebraic Equation with 1 Unknown}

\section{Mở Đầu về Phương Trình}

\section{Phương Trình Bậc Nhất 1 Ẩn \& Cách Giải}

\section{Phương Trình Đưa Được về Dạng $ax + b = 0$}

\section{Phương Trình Tích}

\section{Phương Trình Chứa Ẩn ở Mẫu}

\section{Giải Bài Toán Bằng Cách Lập Phương Trình}

%------------------------------------------------------------------------------%

\chapter{Bất Phương Trình Bậc Nhất 1 Ẩn -- Algebraic Inequation with 1 Unknown}

\section{Liên Hệ Giữa Thứ Tự \& Phép Cộng}

\section{Liên Hệ Giữa Thứ Tự \& Phép Nhân}

\section{Bất Phương Trình 1 Ẩn}

\section{Bất Phương Trình Bậc Nhất 1 Ẩn}

\section{Phương Trình Chứa Dấu Giá Trị Tuyệt Đối}

%------------------------------------------------------------------------------%

\part{Hình Học -- Geometry}

\chapter{Tứ Giác}

\section{Tứ Giác}

\section{Hình Thang}

\section{Hình Thang Cân}

\section{Đường Trung Bình của Tam Giác, của Hình Thang}

\section{Dựng Hình Bằng Thước \& Compa. Dựng Hình thang}

\section{Đối Xứng Trục}

\section{Hình Bình Hành}

\section{Đối Xứng Tâm}

\section{Hình Chữ Nhật}

\section{Đường Thẳng Song Song với 1 Đường Thẳng Cho Trước}

\section{Hình Thoi}

\section{Hình Vuông}

%------------------------------------------------------------------------------%

\chapter{Đa Giác. Diện Tích Đa Giác}

\section{Đa Giác. Đa Giác Đều}

\section{Diện Tích Hình Chữ Nhật}

\section{Diện Tích Tam Giác}

\section{Diện Tích Hình Thang}

\section{Diện Tích Hình Thoi}

\section{Diện Tích Đa Giác}

%------------------------------------------------------------------------------%

\chapter{Tam Giác Đồng Dạng}

\section{Định Lý Thales Trong Tam Giác}

\section{Định Lý Đảo \& Hệ Quả của Định Lý Thales}

\section{Tính Chất Đường Phân Giác của Tam Giác}

\section{Khái Niệm 2 Tam Giác Đồng Dạng}

\section{Trường Hợp Đồng Dạng Thứ Nhất}

\section{Trường Hợp Đồng Dạng Thứ 2}

\section{Trường Hợp Đồng Dạng Thứ 3}

\section{Các Trường Hợp Đồng Dạng của Tam Giác Vuông}

\section{Ứng Dụng Thực Tế của Tam Giác Đồng Dạng}

%------------------------------------------------------------------------------%

\chapter{Hình Lăng Trụ Đứng. Hình Chóp Đều}

\begin{center}
	\Large A -- Hình Lăng Trụ Đứng
\end{center}

\section{Hình Hộp Chữ Nhật}

\section{Thể Tích của Hình Hộp Chữ Nhật}

\section{Hình Lăng Trụ Đứng}

\section{Diện Tích Xung Quanh của Hình Lăng Trụ Đứng}

\section{Thể Tích của Hình Lăng Trụ Đứng}

\begin{center}
	\Large B -- Hình Chóp Đều
\end{center}

\section{Hình Chóp Đều \& Hình Chóp Cụt Đều}

\section{Diện Tích Xung Quanh của Hình Chóp Đều}

\section{Thể Tích của Hình Chóp Đều}

%------------------------------------------------------------------------------%

\begin{thebibliography}{99}
	\bibitem[NQBH\texttt{/}elementary math]{NQBH/elementary math} Nguyễn Quản Bá Hồng. \href{https://github.com/NQBH/hobby/blob/master/elementary_mathematics/NQBH_elementary_mathematics.pdf}{\textit{Some Topics in Elementary Mathematics: Problems, Theories, Applications, \textit{\&} Bridges to Advanced Mathematics}}. Mar 2022--now.
\end{thebibliography}

%------------------------------------------------------------------------------%

\printbibliography[heading=bibintoc]
	
\end{document}