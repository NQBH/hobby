\documentclass[oneside]{book}
\usepackage[backend=biber,natbib=true,style=authoryear]{biblatex}
\addbibresource{/home/hong/1_NQBH/reference/bib.bib}
\usepackage[utf8]{vietnam}
\usepackage{tocloft}
\renewcommand{\cftsecleader}{\cftdotfill{\cftdotsep}}
\usepackage[colorlinks=true,linkcolor=blue,urlcolor=red,citecolor=magenta]{hyperref}
\usepackage{amsmath,amssymb,amsthm,mathtools,float,graphicx,algpseudocode,algorithm,tcolorbox}
\usepackage[inline]{enumitem}
\allowdisplaybreaks
\numberwithin{equation}{section}
\newtheorem{assumption}{Assumption}[section]
\newtheorem{conjecture}{Conjecture}[section]
\newtheorem{corollary}{Corollary}[section]
\newtheorem{hequa}{Hệ quả}[section]
\newtheorem{definition}{Definition}[section]
\newtheorem{dinhnghia}{Định nghĩa}[section]
\newtheorem{example}{Example}[section]
\newtheorem{vidu}{Ví dụ}[section]
\newtheorem{lemma}{Lemma}[section]
\newtheorem{notation}{Notation}[section]
\newtheorem{principle}{Principle}[section]
\newtheorem{problem}{Problem}[section]
\newtheorem{baitoan}{Bài toán}[section]
\newtheorem{proposition}{Proposition}[section]
\newtheorem{question}{Question}[section]
\newtheorem{cauhoi}{Câu hỏi}[section]
\newtheorem{remark}{Remark}[section]
\newtheorem{luuy}{Lưu ý}[section]
\newtheorem{theorem}{Theorem}[section]
\newtheorem{dinhly}{Định lý}[section]
\usepackage[left=0.5in,right=0.5in,top=1.5cm,bottom=1.5cm]{geometry}
\usepackage{fancyhdr}
\pagestyle{fancy}
\fancyhf{}
\lhead{\small \textsc{Sect.} ~\thesection}
\rhead{\small \nouppercase{\leftmark}}
\renewcommand{\sectionmark}[1]{\markboth{#1}{}}
\cfoot{\thepage}
\def\labelitemii{$\circ$}

\title{Some Topics in Elementary Mathematics\texttt{/}Grade 8}
\author{Nguyễn Quản Bá Hồng\footnote{Independent Researcher, Ben Tre City, Vietnam\\e-mail: \texttt{nguyenquanbahong@gmail.com}; website: \url{https://nqbh.github.io}.}}
\date{\today}

\begin{document}
\frontmatter
\maketitle
\setcounter{secnumdepth}{4}
\setcounter{tocdepth}{3}
\tableofcontents
\newpage

%------------------------------------------------------------------------------%

\chapter*{Preface}

\section*{Ký Hiệu, Viết Tắt, Quy Ước -- Notation, Abbreviation, Convention}
\addcontentsline{toc}{section}{\protect\numberline{}Ký Hiệu, Viết Tắt, Quy Ước -- Notation, Abbreviation, Convention}

\subsection*{Ký Hiệu -- Notation}
\begin{itemize}
	\item $\land$: và, (logical) and.
	\item $\lor$: hoặc, (logical) or.
	\item $\Sigma$: tổng, sum, e.g., $\sum_{i=a}^b f(i) = f(a) + f(a + 1) + \cdots + f(b - 1) + f(b)$, $\forall a,b\in\mathbb{Z}$, $a\le b$.
	\item $\prod$: tích, product, e.g., $\prod_{i=a}^b f(i) = f(a)f(a + 1)\cdots f(b - 1)f(b)$, $\forall a,b\in\mathbb{Z}$, $a\le b$.
\end{itemize}

\subsection*{Viết Tắt -- Abbreviation}
\begin{itemize}
	\item \textbf{abbr.} (abbr., abbreviation): viết tắt, abbreviation, for short.
	\item \textbf{i.e.} stands for the Latin \textit{id est}, or `that is,' \& is used in front of a word or phrase that restates what has been said previously: tức là, nghĩa là, that is, that means, in another term.
	\item \textbf{e.g.} stands for \textit{exempli gratia} in Latin: ví dụ là, chẳng hạn, for example, for instance.
	\item \textbf{w.l.o.g.} (abbr., without loss of generality): không mất tính tổng quát.
\end{itemize}

\subsection*{Quy Ước -- Convention}

%------------------------------------------------------------------------------%

\mainmatter
\part{Đại Số -- Algebra}

\chapter{Phép Nhân \& Phép Chia Các Đa Thức}

\section{Nhân Đơn Thức với Đa Thức}

\subsection{Quy tắc}
``Muốn nhân 1 đơn thức với 1 đa thức, ta nhân đơn thức với từng hạng tử của đa thức rồi cộng các tích với nhau.'' -- \cite[p. 4]{SGK_Toan_8_tap_1}.

\begin{vidu}[Đơn thức 1 biến nhân đa thức 1 biến]
	Phép nhân 1 đơn thức 1 biến $ax^m$ với 1 đa thức bậc $n$ được thực hiện như sau:
	\begin{align*}
		ax^m\sum_{i=0}^n a_ix^i &= ax^m\left(a_nx^n + a_{n-1}x^{n-1} + \cdots + a_1x + a_0\right)\\
		&= aa_nx^{m+n} + aa_{n-1}x^{m+ n-1} + \cdots + aa_1x^{m+1} + aa_0x^m,\ \forall a,a_i\in\mathbb{R},\,i = 0,\ldots,n,\,\forall m,n\in\mathbb{N}.
	\end{align*}
\end{vidu}

\begin{vidu}[Đơn thức $\le 2$ biến nhân đa thức $\le 2$ biến]
	Phép nhân 1 đơn thức 2 biến $ax^{m_1}y^{m_2}$ với 1 đa thức 2 biến được thực hiện như sau:
	\begin{align*}
		ax^{m_1}y^{m_2}\cdot\left(\sum_{i=0}^{n_1}\sum_{j=0}^{n_2} a_{ij}x^iy^j\right) = \sum_{i=0}^{n_1}\sum_{j=0}^{n_2} aa_{ij}x^{m_1 + i}y^{m_2 + j},\ \forall a,a_{ij}\in\mathbb{R},\,i = 1,\ldots,n_1,\,j = 0,\ldots,n_2,\ \forall m_i,n_i\in\mathbb{N},\,i = 1,2.
	\end{align*}
\end{vidu}
Tổng quát,
\begin{vidu}[Đơn thức $\le k$ biến nhân đa thức $\le k$ biến]
	Với $k\in\mathbb{N}$, $k\ge 2$ cho trước. Phép nhân 1 đơn thức $k$ biến $ax_1^{m_1}x_2^{m_2}\cdots x_k^{m_k} = a\prod_{i=1}^k x_i^{m_i}$ với 1 đa thức $k$ biến được thực hiện như sau:
	\begin{align*}
		a\prod_{i=1}^k x_i^{m_i}\left(\sum_{i_1 = 0}^{n_1}\ldots\sum_{i_k = 0}^{n_k} a_{i_1\ldots i_k}\prod_{j=1}^k x_j^{i_j}\right) &= ax_1^{m_1}\ldots x_k^{m_k}\sum_{i_1 = 0}^{n_1}\ldots\sum_{i_k = 0}^{n_k} a_{i_1\ldots i_k}x_1^{i_1}\ldots x_k^{i_k}\\
		&= \sum_{i_1 = 0}^{n_1}\ldots\sum_{i_k = 0}^{n_k} aa_{i_1\ldots i_k}x_1^{m_1 + i_1}\ldots x_k^{m_k + i_k},
	\end{align*}
	$\forall a,a_{i_1\ldots i_k}\in\mathbb{R}$, $i_1 = 0,\ldots,n_1;\ldots;i_k = 0,\ldots,n_k$, $\forall m_i,n_i\in\mathbb{N}$, $i = 1,\ldots,k$.\footnote{Điều kiện $i_1 = 0,\ldots,n_1;\ldots;i_k = 0,\ldots,n_k$ có thể viết gọn hơn thành $(i_1,\ldots,i_k)\in\overline{0,n_1}\times\cdots\times\overline{0,n_k}$ với ký hiệu $\overline{0,n}\coloneqq\{0,1,\ldots,n\}$, $\forall n\in\mathbb{N}$.}
\end{vidu}

\section{Nhân Đa Thức với Đa Thức}

\subsection{Quy tắc}
``Muốn nhân 1 đa thức với 1 đa thức, ta nhân mỗi hạng tử của đa thức này với từng hạng tử của đa thức kia rồi cộng các tích với nhau.'' ``Tích của 2 đa thức là 1 đa thức.'' -- \cite[p. 7]{SGK_Toan_8_tap_1}. Tổng quát, muốn nhân 2 đa thức bậc $P,Q$ lần lượt có bậc $m,n$ (ký hiệu $\deg P = m,\deg Q = n$), $P(x) = \sum_{i=0}^m a_ix^i$, $Q(x) = \sum_{i=0}^n b_ix^i$.

\section{Những Hằng Đẳng Thức Đáng Nhớ}

\section{Phân Tích Đa Thức Thành Nhân Tử Bằng Phương Pháp Đặt Nhân Tử Chung}

\section{Phân Tích Đa Thức Thành Nhân Tử Bằng Phương Pháp Dùng Hằng Đẳng Thức}

\section{Phân Tích Đa Thức Thành Nhân Tử Bằng Phương Pháp Nhóm Hạng Tử}

\section{Phân Tích Đa Thức Thành Nhân Tử Bằng Cách Phối Hợp Nhiều Phương Pháp}

\section{Chia Đơn Thức Cho Đơn Thức}

\section{Chia Đa Thức Cho Đơn Thức}

\section{Chia Đa Thức 1 Biến Đã Sắp Xếp}

%------------------------------------------------------------------------------%

\chapter{Phân Thức Đại Số}

\section{Phân Thức Đại Số}

\section{Tính Chất Cơ Bản của Phân Thức}

\section{Rút Gọn Phân Thức}

\section{Quy Đồng Mẫu thức Nhiều Phân Thức}

\section{Phép Cộng Các Phân Thức Đại Số}

\section{Phép Trừ Các Phân Thức Đại Số}

\section{Phép Nhân Các Phân Thức Đại Số}

\section{Phép Chia Các Phân Thức Đại Số}

\section{Biến Đổi Các Biểu Thức Hữu Tỷ. Giá Trị của Phân Thức}

%------------------------------------------------------------------------------%

\chapter{Phương Trình Đại Số 1 Ẩn -- Algebraic Equation with 1 Unknown}

\section{Mở Đầu về Phương Trình}

\section{Phương Trình Bậc Nhất 1 Ẩn \& Cách Giải}

\section{Phương Trình Đưa Được về Dạng $ax + b = 0$}

\section{Phương Trình Tích}

\section{Phương Trình Chứa Ẩn ở Mẫu}

\section{Giải Bài Toán Bằng Cách Lập Phương Trình}

%------------------------------------------------------------------------------%

\chapter{Bất Phương Trình Bậc Nhất 1 Ẩn -- Algebraic Inequation with 1 Unknown}

\section{Liên Hệ Giữa Thứ Tự \& Phép Cộng}

\section{Liên Hệ Giữa Thứ Tự \& Phép Nhân}

\section{Bất Phương Trình 1 Ẩn}

\section{Bất Phương Trình Bậc Nhất 1 Ẩn}

\section{Phương Trình Chứa Dấu Giá Trị Tuyệt Đối}

%------------------------------------------------------------------------------%

\part{Hình Học -- Geometry}

\chapter{Tứ Giác}

\section{Tứ Giác}

\section{Hình Thang}

\section{Hình Thang Cân}

\section{Đường Trung Bình của Tam Giác, của Hình Thang}

\section{Dựng Hình Bằng Thước \& Compa. Dựng Hình thang}

\section{Đối Xứng Trục}

\section{Hình Bình Hành}

\section{Đối Xứng Tâm}

\section{Hình Chữ Nhật}

\section{Đường Thẳng Song Song với 1 Đường Thẳng Cho Trước}

\section{Hình Thoi}

\section{Hình Vuông}

%------------------------------------------------------------------------------%

\chapter{Đa Giác. Diện Tích Đa Giác}

\section{Đa Giác. Đa Giác Đều}

\section{Diện Tích Hình Chữ Nhật}

\section{Diện Tích Tam Giác}

\section{Diện Tích Hình Thang}

\section{Diện Tích Hình Thoi}

\section{Diện Tích Đa Giác}

%------------------------------------------------------------------------------%

\chapter{Tam Giác Đồng Dạng}

\section{Định Lý Thales Trong Tam Giác}

\section{Định Lý Đảo \& Hệ Quả của Định Lý Thales}

\section{Tính Chất Đường Phân Giác của Tam Giác}

\section{Khái Niệm 2 Tam Giác Đồng Dạng}

\section{Trường Hợp Đồng Dạng Thứ Nhất}

\section{Trường Hợp Đồng Dạng Thứ 2}

\section{Trường Hợp Đồng Dạng Thứ 3}

\section{Các Trường Hợp Đồng Dạng của Tam Giác Vuông}

\section{Ứng Dụng Thực Tế của Tam Giác Đồng Dạng}

%------------------------------------------------------------------------------%

\chapter{Hình Lăng Trụ Đứng. Hình Chóp Đều}

\begin{center}
	\Large A -- Hình Lăng Trụ Đứng
\end{center}

\section{Hình Hộp Chữ Nhật}

\section{Thể Tích của Hình Hộp Chữ Nhật}

\section{Hình Lăng Trụ Đứng}

\section{Diện Tích Xung Quanh của Hình Lăng Trụ Đứng}

\section{Thể Tích của Hình Lăng Trụ Đứng}

\begin{center}
	\Large B -- Hình Chóp Đều
\end{center}

\section{Hình Chóp Đều \& Hình Chóp Cụt Đều}

\section{Diện Tích Xung Quanh của Hình Chóp Đều}

\section{Thể Tích của Hình Chóp Đều}

%------------------------------------------------------------------------------%

\begin{thebibliography}{99}
	\bibitem[NQBH\texttt{/}elementary math]{NQBH/elementary math} Nguyễn Quản Bá Hồng. \href{https://github.com/NQBH/hobby/blob/master/elementary_mathematics/NQBH_elementary_mathematics.pdf}{\textit{Some Topics in Elementary Mathematics: Problems, Theories, Applications, \textit{\&} Bridges to Advanced Mathematics}}. Mar 2022--now.
\end{thebibliography}

%------------------------------------------------------------------------------%

\printbibliography[heading=bibintoc]
	
\end{document}