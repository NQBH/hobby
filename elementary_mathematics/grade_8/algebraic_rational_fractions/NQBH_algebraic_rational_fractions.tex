\documentclass{article}
\usepackage[backend=biber,natbib=true,style=authoryear]{biblatex}
\addbibresource{/home/nqbh/reference/bib.bib}
\usepackage[utf8]{vietnam}
\usepackage{tocloft}
\renewcommand{\cftsecleader}{\cftdotfill{\cftdotsep}}
\usepackage[colorlinks=true,linkcolor=blue,urlcolor=red,citecolor=magenta]{hyperref}
\usepackage{amsmath,amssymb,amsthm,mathtools,float,graphicx,algpseudocode,algorithm,tcolorbox,tikz,tkz-tab,subcaption}
\DeclareMathOperator{\arccot}{arccot}
\usepackage[inline]{enumitem}
\allowdisplaybreaks
\numberwithin{equation}{section}
\newtheorem{assumption}{Assumption}[section]
\newtheorem{baitoan}{Bài toán}
\newtheorem{cauhoi}{Câu hỏi}[section]
\newtheorem{conjecture}{Conjecture}[section]
\newtheorem{corollary}{Corollary}[section]
\newtheorem{definition}{Definition}[section]
\newtheorem{dinhly}{Định lý}[section]
\newtheorem{dinhnghia}{Định nghĩa}[section]
\newtheorem{example}{Example}[section]
\newtheorem{hequa}{Hệ quả}[section]
\newtheorem{lemma}{Lemma}[section]
\newtheorem{luuy}{Lưu ý}[section]
\newtheorem{notation}{Notation}[section]
\newtheorem{principle}{Principle}[section]
\newtheorem{problem}{Problem}[section]
\newtheorem{proposition}{Proposition}[section]
\newtheorem{question}{Question}[section]
\newtheorem{remark}{Remark}[section]
\newtheorem{theorem}{Theorem}[section]
\newtheorem{vidu}{Ví dụ}[section]
\usepackage[left=0.5in,right=0.5in,top=1.5cm,bottom=1.5cm]{geometry}
\usepackage{fancyhdr}
\pagestyle{fancy}
\fancyhf{}
\lhead{\small Sect.~\thesection}
\rhead{\small\nouppercase{\leftmark}}
\renewcommand{\subsectionmark}[1]{\markboth{#1}{}}
\cfoot{\thepage}
\def\labelitemii{$\circ$}

\title{Algebraic \& Rational Fractions\\Phân Thức Đại Số \textit{\&} Phân Thức Đại Số Hữu Tỷ}
\author{Nguyễn Quản Bá Hồng\footnote{Independent Researcher, Ben Tre City, Vietnam\\e-mail: \texttt{nguyenquanbahong@gmail.com}; website: \url{https://nqbh.github.io}.}}
\date{\today}

\begin{document}
\maketitle
\begin{abstract}
	\textsc{[en]} This text is a collection of problems, from easy to advanced, about algebraic \& rational fractions. This text is also a supplementary material for my lecture note on Elementary Mathematics grade 8, which is stored \& downloadable at the following link: \href{https://github.com/NQBH/hobby/blob/master/elementary_mathematics/grade_8/NQBH_elementary_mathematics_grade_8.pdf}{GitHub\texttt{/}NQBH\texttt{/}hobby\texttt{/}elementary mathematics\texttt{/}grade 8\texttt{/}lecture}\footnote{\textsc{url}: \url{https://github.com/NQBH/hobby/blob/master/elementary_mathematics/grade_8/NQBH_elementary_mathematics_grade_8.pdf}.}. The latest version of this text has been stored \& downloadable at the following link: \href{https://github.com/NQBH/hobby/blob/master/elementary_mathematics/grade_8/algebraic_rational_fractions/NQBH_algebraic_rational_fractions.pdf}{GitHub\texttt{/}NQBH\texttt{/}hobby\texttt{/}elementary mathematics\texttt{/}grade 8\texttt{/}algebraic \& rational fractions}\footnote{\textsc{url}: \url{https://github.com/NQBH/hobby/blob/master/elementary_mathematics/grade_8/algebraic_rational_fractions/NQBH_algebraic_rational_fractions.pdf}.}.
	\vspace{2mm}
	
	\textsc{[vi]} Tài liệu này là 1 bộ sưu tập các bài tập chọn lọc từ cơ bản đến nâng cao về phân thức đại số \& phân thức đại số hữu tỷ. Tài liệu này là phần bài tập bổ sung cho tài liệu chính -- bài giảng \href{https://github.com/NQBH/hobby/blob/master/elementary_mathematics/grade_8/NQBH_elementary_mathematics_grade_8.pdf}{GitHub\texttt{/}NQBH\texttt{/}hobby\texttt{/}elementary mathematics\texttt{/}grade 8\texttt{/}lecture} của tác giả viết cho Toán Sơ Cấp lớp 8. Phiên bản mới nhất của tài liệu này được lưu trữ \& có thể tải xuống ở link sau: \href{https://github.com/NQBH/hobby/blob/master/elementary_mathematics/grade_8/algebraic_rational_fractions/NQBH_algebraic_rational_fractions.pdf}{GitHub\texttt{/}NQBH\texttt{/}hobby\texttt{/}elementary mathematics\texttt{/}grade 8\texttt{/}algebraic \& rational fractions}.
\end{abstract}
\setcounter{secnumdepth}{4}
\setcounter{tocdepth}{3}
\tableofcontents

%------------------------------------------------------------------------------%

\section*{Định Nghĩa \& Tính Chất Cơ Bản}

\begin{definition}[Algebraic fraction]
	``In \href{https://en.wikipedia.org/wiki/Algebra}{algebra}, an \emph{algebraic fraction} is a \href{https://en.wikipedia.org/wiki/Fraction_(mathematics)}{fraction} whose numerator \& denominator are \href{https://en.wikipedia.org/wiki/Algebraic_expression}{algebraic expressions}.'' -- \href{https://en.wikipedia.org/wiki/Algebraic_fraction}{Wikipedia\emph{\texttt{/}}algebraic fraction}
\end{definition}
``Algebraic fractions are subject to the same laws as \href{https://en.wikipedia.org/wiki/Arithmetic_fraction}{algebraic expressions}.'' -- \href{https://en.wikipedia.org/wiki/Algebraic_fraction}{Wikipedia\texttt{/}algebraic fraction}

\begin{definition}[Rational fraction]
	A \emph{rational fraction} is an algebraic fraction whose numerator \& denominator are both \href{https://en.wikipedia.org/wiki/Polynomial}{polynomials}.
\end{definition}
``\begin{enumerate*}
	\item[\textbf{1.}] \textit{Phân thức đại số} là 1 biểu thức có dạng $\frac{A}{B}$, trong đó $A,B$ là những đa thức \& $B\ne 0$. Đặc biệt: Mỗi đa thức cũng được coi như 1 phân thức với mẫu thức bằng 1.
	\item[\textbf{2.}] $\frac{A}{B} = \frac{C}{D}$ nếu $AD = BC$, $B\ne 0$, $D\ne 0$.
	\item[\textbf{3.}] Tính chất cơ bản của phân thức: $\frac{A}{B} = \frac{AM}{AM}$, $M$ là đa thức khác đa thức không 0; $\frac{A}{B} = \frac{A:N}{B:N}$, $N$ là 1 nhân tử chung của $A$ \& $B$. Đặc biệt với $N = -1$, $\frac{A}{B} = \frac{-A}{-B}$ (quy tắc đổi dấu).
	\item[\textbf{4.}] Rút gọn phân thức: Phân tích tử \& mẫu thành nhân tử (nếu cần) để tìm nhân tử chung; Chia cả tử \& mẫu cho nhân tử chung (nếu có).
	\item[\textbf{5.}] Quy đồng mẫu của nhiều phân thức: Phân tích các mẫu thành nhân tử rồi tìm mẫu thức chung; Tìm nhân tử phụ của mỗi mẫu thức; Nhân tử \& mẫu của mỗi phân thức với nhân tử phụ tương ứng.
\end{enumerate*}

\textit{Bổ sung}. Phân số $\frac{a}{b}$ là 1 trường hợp đặc biệt của phân thức $\frac{A}{B}$ khi $A,B$ là những đa thức bậc 0. Vì vậy tính chất cơ bản của phân số là 1 trường hợp đặc biệt của tính chất cơ bản của phân thức đại số.'' -- \cite[pp. 37--38]{Tuyen_Toan_8}

%------------------------------------------------------------------------------%

\section{Tính Chất Cơ Bản của Phân Thức. Rút Gọn Phân Thức}

\begin{baitoan}[\cite{Binh_Nam_Ngoc_Son_Toan_8_Dai_So}, Ví dụ 5.1, p. 39]
	Dùng định nghĩa 2 phân thức bằng nhau, chứng minh 2 phân thức sau bằng nhau: $\dfrac{a^2 - 2ab - 3b^2}{a^2 - 4ab + 3b^2}$ \& $\dfrac{a + b}{a - b}$ với $a\ne b$ \& $a\ne 3b$.
\end{baitoan}

\begin{baitoan}[\cite{Binh_Nam_Ngoc_Son_Toan_8_Dai_So}, Ví dụ 5.2, p. 39]
	Dùng định nghĩa 2 phân thức bằng nhau, xét sự bằng nhau của 2 phân thức $\dfrac{(3x + 2)(x + 5)}{4(3x + 2)}$ \& $\dfrac{x + 5}{4}$ trong các trường hợp biến $x$ thuộc các tập hợp sau:
	\begin{enumerate*}
		\item[(a)] $x\in\mathbb{N}$;
		\item[(b)] $x\in\mathbb{Z}$;
		\item[(c)] $x\in\mathbb{Q}$.
	\end{enumerate*}
\end{baitoan}

\begin{baitoan}[\cite{Binh_Nam_Ngoc_Son_Toan_8_Dai_So}, Ví dụ 5.3, p. 39]
	So sánh $C = \dfrac{2013^2 - 2012^2}{2013^2 + 2012^2}$ với $D = \dfrac{2013 - 2012}{2013 + 2012}$.
\end{baitoan}

\begin{baitoan}[\cite{Binh_Nam_Ngoc_Son_Toan_8_Dai_So}, Ví dụ 5.4, p. 40]
	Chứng minh: $\sum_{i=0}^{63} a^i = \prod_{i=0}^{5} (1 + a^{2^i})$, i.e., $1 + a + a^2 + \cdots + a^{63} = (1 + a)(1 + a^2)(1 + a^4)\cdots(1 + a^{32})$.
\end{baitoan}

\begin{baitoan}[\cite{Binh_Nam_Ngoc_Son_Toan_8_Dai_So}, Ví dụ 5.5, p. 40]
	Rút gọn phân thức: $A = \dfrac{x^3 - 7x + 6}{x^3 + 5x^2 - 2x - 24}$.
\end{baitoan}

\begin{baitoan}[\cite{Binh_Nam_Ngoc_Son_Toan_8_Dai_So}, Ví dụ 5.6, p. 40]
	Rút gọn phân thức: $B = \dfrac{a^{30} + a^{20} + a^{10} + 1}{a^{2042} + a^{2032} + a^{2022} + a^{2012} + a^{30} + a^{20} + a^{10} + 1}$.
\end{baitoan}

\begin{baitoan}[\cite{Binh_Nam_Ngoc_Son_Toan_8_Dai_So}, \textbf{5.1}, p. 41]
	Dùng định nghĩa 2 phân thức bằng nhau, tìm đa thức $A$ trong các trường hợp sau:
	\begin{enumerate*}
		\item[(a)] $\dfrac{A}{3x - 2} = \dfrac{15x^2 + 10x}{9x^2 - 4}$;
		\item[(b)] $\dfrac{3x^2 - 5x - 2}{A} = \dfrac{x - 2}{2x - 3}$;
		\item[(c)] $\dfrac{x^2 - 4}{x^2 + x - 6} = \dfrac{x^2 + 4x + 4}{A}$;
		\item[(d)] $\dfrac{2x + 1}{x^3 + x^2 - x + 2} = \dfrac{A}{x^3 + 1}$.
	\end{enumerate*}
\end{baitoan}

\begin{baitoan}[\cite{Binh_Nam_Ngoc_Son_Toan_8_Dai_So}, \textbf{5.2}, p. 41]
	Biến đổi mỗi phân thức sau thành 1 phân thức bằng nó \& có tử thức là đa thức $B$ cho sau đây:
	\begin{enumerate*}
		\item[(a)] $\dfrac{2x - 5}{3x^2 + 4}$ \& $B = 2x^2 - 3x - 5$;
		\item[(b)] $\dfrac{(x + 1)(x^2 + x - 6)}{(x^2 - 9)(x^2 + 3x + 2)}$ \& $B = x - 2$.
	\end{enumerate*}
\end{baitoan}

\begin{baitoan}[\cite{Binh_Nam_Ngoc_Son_Toan_8_Dai_So}, \textbf{5.3}, p. 41]
	Rút gọn biểu thức:
	\begin{enumerate*}
		\item[(a)] $\dfrac{2^{18}\cdot 54^3 + 15\cdot 4^{10}\cdot 9^4}{2\cdot 12^9 + 6^{10}\cdot 2^{10}}$;
		\item[(b)] $\dfrac{4^{15}\cdot 27^6\cdot 42 - 3\cdot 72^{10}}{4^4\cdot 25\cdot 36^{10}  - 4^5\cdot 6^{19}\cdot 35}$;
		\item[(c)] $\dfrac{880\cdot(15^2\cdot 3^{18} + 27^7)}{4^2\cdot 15^4\cdot 3^{16} - 2^4\cdot 9^{11}}$.
	\end{enumerate*}
\end{baitoan}

\begin{baitoan}[\cite{Binh_Nam_Ngoc_Son_Toan_8_Dai_So}, \textbf{5.4}, p. 41]
	Rút gọn:
	\begin{enumerate*}
		\item[(a)] $M = \dfrac{4024\cdot 2014 - 2}{2011 + 2012\cdot 2013}$;
		\item[(b)] $N = \dfrac{2012\cdot 2013 + 2014}{2010 - 2012\cdot 2015}$;
		\item[(c)] $P = \dfrac{66666\cdot 87564 - 33333}{22222\cdot 87560 + 77777}$.
	\end{enumerate*}
\end{baitoan}

\begin{baitoan}[\cite{Binh_Nam_Ngoc_Son_Toan_8_Dai_So}, \textbf{5.5}, p. 41]
	Rút gọn các phân thức sau:
	\begin{enumerate*}
		\item[(a)] $Q = \dfrac{x^2 + 2x - 8}{x^2 + x - 12}$;
		\item[(b)] $R = \dfrac{3x^2 + 5xy - 2y^2}{3x^2 - 7xy + 2y^2}$;
		\item[(c)] $S = \dfrac{x^6 - 14x^4 + 49x^2 - 36}{x^4 + 4x^3 - x^2 - 16x - 12}$;
		\item[(d)] $T = \dfrac{x^6 - y^6}{x^6 + 2x^4y^2 + 2x^2y^4 + y^6}$.
	\end{enumerate*}
\end{baitoan}

\begin{baitoan}[\cite{Binh_Nam_Ngoc_Son_Toan_8_Dai_So}, \textbf{5.6}, pp. 41--42]
	Rút gọn:
	\begin{enumerate*}
		\item[(a)] $A = \dfrac{a^4 - 5a^2 + 4}{a^4 - a^2 + 4a - 4}$;
		\item[(b)] $B = \dfrac{a^3 - 3a + 2}{2a^3 - 7a^2 + 8a - 3}$;
		\item[(c)] $C = \dfrac{a^2 - 2ab + b^2 - c^2}{a^2 + b^2 + c^2 - 2ab - 2bc + 2ca}$;
		\item[(d)] $D = \dfrac{a^3 - 7a + 6}{a^2(a + 3)^3 - 4a(a + 3)^3 + 4(a + 3)^3}$;
		\item[(e)] $E = \dfrac{a^3 + b^3 + c^3 - 3abc}{(a - b)^2 + (b - c)^2 + (c - a)^2}$.
	\end{enumerate*}
\end{baitoan}

\begin{baitoan}[\cite{Binh_Nam_Ngoc_Son_Toan_8_Dai_So}, \textbf{5.7}, p. 42]
	Rút gọn các phân thức sau:
	\begin{enumerate*}
		\item[(a)] $A = \dfrac{xy^2 - xz^2 - y^3 + yz^2}{x^2(z - y) + y^2(x - z) + z^2(y - x)}$;
		\item[(b)] $B = \dfrac{x^4(y^2 - z^2) + y^4(z^2 - x^2) + z^4(x^2 - y^2)}{x^2(y - z) + y^2(z - x) + z^2(x - y)}$.
	\end{enumerate*}
\end{baitoan}

\begin{baitoan}[\cite{Binh_Nam_Ngoc_Son_Toan_8_Dai_So}, \textbf{5.8}, p. 42]
	Rút gọn các phân thức sau:
	\begin{enumerate*}
		\item[(a)] $A = \dfrac{(x + y + z)^2 - 3xy - 3yz - 3zx}{9xyz - 3x^3 - 3y^3 - 3z^3}$;
		\item[(b)] $B = \dfrac{x^3 - y^3 + z^3 + 3xyz}{(x + y)^2 + (y + z)^2 + (z - x)^2}$;
		\item[(c)] $C = \dfrac{(x - y)^3 + (y - z)^3 + (z - x)^3}{(x^2 - y^2)^3 + (y^2 - z^2)^3 + (z^2 - x^2)^3}$.
	\end{enumerate*}
\end{baitoan}

\begin{baitoan}[\cite{Binh_Nam_Ngoc_Son_Toan_8_Dai_So}, \textbf{5.9}, p. 42]
	Rút gọn các phân thức sau với $n\in\mathbb{N}^\star$:
	\begin{enumerate*}
		\item[(a)] $\dfrac{(n + 2)!}{n!(n + 2)(n + 3)}$;
		\item[(b)] $\dfrac{n!}{n! + (n - 1)!}$;
		\item[(c)] $\dfrac{(n + 3)! - (n + 2)!}{(n + 2)! + (n + 3)!}$.
	\end{enumerate*}
\end{baitoan}

\begin{baitoan}[\cite{Binh_Nam_Ngoc_Son_Toan_8_Dai_So}, \textbf{5.10}, p. 42]
	Chứng minh các phân số sau là tối giản $\forall n\in\mathbb{N}$:
	\begin{enumerate*}
		\item[(a)] $\dfrac{3n + 2}{4n + 3}$;
		\item[(b)] $\dfrac{12n + 1}{2(10n + 1)}$;
		\item[(c)] $\dfrac{2n + 3}{2n^2 + 4n + 1}$.
	\end{enumerate*}
\end{baitoan}

\begin{baitoan}[\cite{Binh_Nam_Ngoc_Son_Toan_8_Dai_So}, \textbf{5.11}, p. 42]
	Chứng minh phân số $\dfrac{n^7 + 2n^2 + n + 2}{n^8 + n^2 + 2n + 2}$ không tối giản, $\forall n\in\mathbb{N}^\star$.
\end{baitoan}

\begin{baitoan}[\cite{Binh_Nam_Ngoc_Son_Toan_8_Dai_So}, \textbf{5.12}, p. 42]
	Viết gọn biểu thức sau dưới dạng 1 phân thức: $P = (x^4 - x^2 + 1)(x^8 - x^4 + 1)(x^{16} - x^8 + 1)(x^{32} + x^{16} + 1)$.
\end{baitoan}

\begin{baitoan}[\cite{Binh_Nam_Ngoc_Son_Toan_8_Dai_So}, \textbf{5.13}, p. 42]
	Rút gọn phân thức:
	\begin{enumerate*}
		\item[(a)] $\dfrac{|x - 2| + |x - 1| + x}{2x^2 - 7x + 3}$ với $x < 1$;
		\item[(b)] $\dfrac{|x - 4||x - 5|}{x^3 - 9x^2 + 20x}$ với $4 < x < 5$.
	\end{enumerate*}
\end{baitoan}

\begin{baitoan}[\cite{Binh_Nam_Ngoc_Son_Toan_8_Dai_So}, \textbf{5.14}, p. 43]
	Rút gọn các phân thức sau:
	\begin{enumerate*}
		\item[(a)] $T = \dfrac{(x + 2)(x + 3)(x + 4)(x + 5) + 1}{x^2 + 7x + 11}$;
		\item[(b)] $U = \dfrac{x^3 - 53x + 88}{(x - 1)(x - 3)(x - 5)(x - 7) + 16}$.
	\end{enumerate*}
\end{baitoan}

\begin{baitoan}[\cite{Binh_Nam_Ngoc_Son_Toan_8_Dai_So}, \textbf{5.15}, p. 43]
	Cho $\dfrac{a}{x} = \dfrac{b}{y} = \dfrac{c}{z}$ \& $x,y,z\ne 0$. Chứng minh: $\dfrac{x^2 + y^2 + z^2}{(ax + by + cz)^2} = \dfrac{1}{a^2 + b^2 + c^2}$.
\end{baitoan}

\begin{baitoan}[\cite{Binh_Nam_Ngoc_Son_Toan_8_Dai_So}, \textbf{5.16}, p. 43]
	Cho $ax + by + cz = 0$. Rút gọn phân thức: $V = \dfrac{ax^2 + by^2 + cz^2}{bc(y - z)^2 + ca(z - x)^2 + ab(x - y)^2}$.
\end{baitoan}

\begin{baitoan}[\cite{Binh_Nam_Ngoc_Son_Toan_8_Dai_So}, \textbf{5.17}, p. 43]
	Cho $x + y + z = 0$. Chứng minh: $\dfrac{9(x^2 + y^2 + z^2)}{(x - y)^2 + (y - z)^2 + (z - x)^2} = 3$.
\end{baitoan}

\begin{baitoan}[\cite{Binh_Nam_Ngoc_Son_Toan_8_Dai_So}, \textbf{5.18}, p. 43]
	Chứng minh: $\dfrac{x^2 + y^2 - z^2 - 2zt + 2xy - t^2}{x + y - z - t} = \dfrac{x^2 - y^2 + z^2 - 2zt + 2xz - t^2}{x - y + z - t}$.
\end{baitoan}

\begin{baitoan}[\cite{Binh_Nam_Ngoc_Son_Toan_8_Dai_So}, \textbf{5.19}, p. 43]
	Rút gọn: $X = \dfrac{(2^4 + 4)(6^4 + 4)(10^4 + 4)(14^4 + 4)}{(4^4 + 4)(8^4 + 4)(12^4 + 4)(16^4 + 4)}$.
\end{baitoan}

\begin{baitoan}[\cite{Tuyen_Toan_8}, Ví dụ 16, p. 38]
	Cho $\dfrac{xy}{x^2 + y^2} = \dfrac{5}{8}$, rút gọn phân thức $P = \dfrac{x^2 - 2xy + y^2}{x^2 + 2xy + y^2}$.
\end{baitoan}

\begin{proof}[Giải]
	$\frac{xy}{x^2 + y^2} = \frac{5}{8}\Rightarrow x^2 + y^2 = \frac{8}{5}xy\Rightarrow P = \dfrac{\frac{8}{5}xy - 2xy}{\frac{8}{5}xy + xy} = \dfrac{\left(\frac{8}{5} - 2\right)xy}{\left(\frac{8}{5} + 2\right)xy} = \dfrac{\frac{8}{5} - 2}{\frac{8}{5} + 2} = -\frac{1}{9}$.
\end{proof}

\begin{baitoan}[\cite{Tuyen_Toan_8}, \textbf{151.}, p. 38]
	So sánh:
	\begin{enumerate*}
		\item[(a)] $A = \dfrac{201 - 200}{201 + 200}$ \& $B = \dfrac{201^2 - 200^2}{201^2 + 200^2}$.
		\item[(b)] $C = \dfrac{1999\cdot 4001 + 2000}{2000\cdot 4001 - 2001}$ \& $D = \dfrac{1501\cdot 1503 - 1500\cdot 1498}{6002}$.
	\end{enumerate*}
\end{baitoan}

\begin{proof}[Giải]
	\begin{enumerate*}
		\item[(a)] Có $A = \frac{201 - 200}{201 + 200} = \frac{(201 - 200)(201 + 200)}{(201 + 200)^2} = \frac{201^2 - 200^2}{201^2 + 2\cdot200\cdot201 + 200^2} < \frac{201^2 - 200^2}{201^2 + 200^2} = B$.
		\item[(b)] Đặt $x\coloneqq2000$, $y\coloneqq1500$. $C = \frac{(x - 1)(2x + 1) + x}{x(2x + 1) - (x + 1)} = \frac{2x^2 - 1}{2x^2 - 1} = 1$. $D = \frac{(y + 1)(y + 3) - y(y - 2)}{4y + 2} = \frac{3(2y + 1)}{2(2y + 1)} = \frac{3}{2}$. Suy ra $C < D$.
	\end{enumerate*}	
\end{proof}

\begin{baitoan}[Mở rộng \cite{Tuyen_Toan_8}, \textbf{151.} (a), p. 38]
	Biện luận theo các tham số $a,b\in\mathbb{R}$, $a\ne-b$, $(a,b)\ne(0,0)$ để so sánh $A = \dfrac{a - b}{a + b}$ \& $B = \dfrac{a^2 - b^2}{a^2 + b^2}$.
\end{baitoan}

\begin{proof}[Giải]
	$A - B = \frac{a - b}{a + b} - \frac{a^2 - b^2}{a^2 + b^2} = \frac{(a - b)(a^2 + b^2) - (a + b)(a^2 - b^2)}{(a + b)(a^2 + b^2)} = $.
\end{proof}

\begin{baitoan}[\cite{Tuyen_Toan_8}, \textbf{152.}, p. 39]
	Chứng minh: $\forall n\in\mathbb{N}$,
	\begin{enumerate*}
		\item[(a)] Phân số $A = \dfrac{n^3 - 1}{n^5 + n + 1}$ không tối giản;
		\item[(b)] Phân số $B = \dfrac{6n + 1}{8n + 1}$ tối giản;
		\item[(c)] Phân số $C = \dfrac{10n^2 + 9n + 4}{20n^2 + 20n + 9}$ tối giản.
	\end{enumerate*}
	Có thể mở rộng từ $\mathbb{N}$ lên $\mathbb{Z}$ được không?
\end{baitoan}

\begin{baitoan}[\cite{Tuyen_Toan_8}, \textbf{153.}, p. 39]
	Viết mỗi đa thức sau dưới dạng 1 phân thức đại số với tử \& mẫu là những đa thức có 2 hạng tử:
	\begin{enumerate*}
		\item[(a)] $A = \sum_{i=0}^{19} x^i = x^{19} + x^{18} + x^{17} + \cdots + x + 1$;
		\item[(b)] $B = (x + 1)(x^2 + 1)(x^4 + 1)\cdots(x^{32} + 1)$.
	\end{enumerate*}
\end{baitoan}

\begin{baitoan}[\cite{Tuyen_Toan_8}, \textbf{154.}, p. 39]
	Rút gọn các phân thức:
	\begin{enumerate*}
		\item[(a)] $A = \dfrac{n!}{(n - 1)!(n + 1)}$;
		\item[(b)] $\dfrac{(n + 1)! - n!}{(n + 1)! + n!}$.
	\end{enumerate*}
\end{baitoan}

\begin{baitoan}[\cite{Tuyen_Toan_8}, \textbf{155.}, p. 39]
	Rút gọn các phân thức:
	
	\begin{enumerate*}
		\item[(a)] $A = \dfrac{(x^2 - y)(y + 1) + x^2y^2 - 1}{(x^2 + y)(y + 1) + x^2y^2 + 1}$;
		\item[(b)] $B = \dfrac{x^2(y - z) + y^2(z - x) + z^2(x - y)}{x^2y - x^2z + y^2z - y^3}$.
	\end{enumerate*}
\end{baitoan}

\begin{baitoan}[\cite{Tuyen_Toan_8}, \textbf{156.}, p. 39]
	Rút gọn các phân thức:
	\begin{enumerate*}
		\item[(a)] $\dfrac{x^4 - 4x^2 + 3}{x^4 + 6x^2 - 7}$;
		\item[(b)] $\dfrac{x^4 + x^3 - x - 1}{x^4 + x^3 + 2x^2 + x + 1}$;
		\item[(c)] $\dfrac{x^3 + 3x^2 - 4}{x^3 - 3x + 2}$.
	\end{enumerate*}
\end{baitoan}

\begin{baitoan}[\cite{Tuyen_Toan_8}, \textbf{157.}, p. 39]
	Rút gọn các phân thức:
	\begin{enumerate*}
		\item[(a)] $\dfrac{x^3 + x^2 - 4x - 4}{x^3 + 8x^2 + 17x + 10}$;
		\item[(b)] $\dfrac{x^4 + 6x^3 + 9x^2 - 1}{x^4 + 6x^3 + 7x^2 - 6x + 1}$.
	\end{enumerate*}
\end{baitoan}

\begin{baitoan}[\cite{Tuyen_Toan_8}, \textbf{158.}, p. 39]
	Cho $\dfrac{x}{a} = \dfrac{y}{b} = \dfrac{z}{c}$, rút gọn phân thức $P = \dfrac{x^2 + y^2 + z^2}{(ax + by + cz)^2}$.
\end{baitoan}

\begin{baitoan}[\cite{Tuyen_Toan_8}, \textbf{159.}, p. 40]
	Cho $x + y + z = 0$, \& $x,y,z\ne 0$, rút gọn các phân thức sau:
	
	\begin{enumerate*}
		\item[(a)] $P = \dfrac{x^2 + y^2 + z^2}{(x - y)^2 + (y - z)^2 + (z - x)^2}$;
		\item[(b)] $Q = \dfrac{(x^2 + y^2 - z^2)(y^2 + z^2 - x^2)(z^2 + x^2 - y^2)}{16xyz}$.
	\end{enumerate*}
\end{baitoan}

\begin{baitoan}[\cite{Tuyen_Toan_8}, \textbf{160.}, p. 40]
	Cho $x^3 + y^3 + z^3 = 3xyz$, rút gọn phân thức $P = \dfrac{xyz}{(x + y)(y + z)(z + x)}$.
\end{baitoan}

%------------------------------------------------------------------------------%

\section{Các Phép Toán về Phân Thức Đại Số}

\subsection{Phép Cộng \& Phép Trừ Các Phân Thức Đại Số}
``\textbf{1.} (a) Muốn cộng 2 phân thức cũng mẫu, ta cộng các tử thức với nhau \& giữ nguyên mẫu thức. (b) Muốn cộng 2 phân thức có mẫu thức khác nhau, ta quy đồng mẫu thức rồi cộng các phân thức cùng mẫu vừa tìm được. \textbf{2.} Phép cộng các phân thức cũng có các tính chất giao hoán, kết hợp. \textbf{3.} 2 phân thức được gọi là \textit{đối nhau} nếu tổng của chúng bằng 0. $-\frac{A}{B} = \frac{-A}{B}$, $-\frac{-A}{B} = \frac{A}{B}$. \textbf{4.} $\frac{A}{B} - \frac{C}{D} = \frac{A}{B} + \left(-\frac{C}{D}\right)$. \textbf{5.} Từ cách tìm phân thức đối của 1 phân thức ta có quy tắc đổi dấu (thứ 2): Nếu đổi dấu của tử thức (hoặc mẫu thức) đồng thời đổi dấu đứng trước phân thức thì được 1 phân thức bằng phân thức đã cho. $\frac{A}{B} = -\frac{-A}{B} = -\frac{A}{-B}$.'' -- \cite[Chap. 2, \S2, p. 40]{Tuyen_Toan_8}

\begin{baitoan}[\cite{Tuyen_Toan_8}, Ví dụ 17, p. 41]
	Thực hiện các phép tính: $A = \frac{x^2}{(x - y)(x - z)} + \frac{y^2}{(y - z)(y - x)} + \frac{z^2}{(z - x)(z - y)}$.
\end{baitoan}

\begin{baitoan}[\cite{Tuyen_Toan_8}, Ví dụ 18, p. 41]
	Thực hiện các phép tính: $B = \frac{x^2 - yz}{(x + y)(x + z)} + \frac{y^2 - zx}{(y + z)(y + x)} + \frac{z^2 - xy}{(z + x)(z + y)}$.
\end{baitoan}
%------------------------------------------------------------------------------%

\printbibliography[heading=bibintoc]
	
\end{document}