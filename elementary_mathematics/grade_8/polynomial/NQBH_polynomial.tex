\documentclass{article}
\usepackage[backend=biber,natbib=true,style=authoryear]{biblatex}
\addbibresource{/home/nqbh/reference/bib.bib}
\usepackage[utf8]{vietnam}
\usepackage{tocloft}
\renewcommand{\cftsecleader}{\cftdotfill{\cftdotsep}}
\usepackage[colorlinks=true,linkcolor=blue,urlcolor=red,citecolor=magenta]{hyperref}
\usepackage{amsmath,amssymb,amsthm,mathtools,float,graphicx,algpseudocode,algorithm,tcolorbox,tikz,tkz-tab,subcaption}
\DeclareMathOperator{\arccot}{arccot}
\usepackage[inline]{enumitem}
\allowdisplaybreaks
\numberwithin{equation}{section}
\newtheorem{assumption}{Assumption}[section]
\newtheorem{baitoan}{Bài toán}
\newtheorem{cauhoi}{Câu hỏi}[section]
\newtheorem{conjecture}{Conjecture}[section]
\newtheorem{corollary}{Corollary}[section]
\newtheorem{definition}{Definition}[section]
\newtheorem{dinhly}{Định lý}[section]
\newtheorem{dinhnghia}{Định nghĩa}[section]
\newtheorem{example}{Example}[section]
\newtheorem{hequa}{Hệ quả}[section]
\newtheorem{lemma}{Lemma}[section]
\newtheorem{luuy}{Lưu ý}[section]
\newtheorem{notation}{Notation}[section]
\newtheorem{principle}{Principle}[section]
\newtheorem{problem}{Problem}[section]
\newtheorem{proposition}{Proposition}[section]
\newtheorem{question}{Question}[section]
\newtheorem{remark}{Remark}[section]
\newtheorem{theorem}{Theorem}[section]
\newtheorem{vidu}{Ví dụ}[section]
\usepackage[left=0.5in,right=0.5in,top=1.5cm,bottom=1.5cm]{geometry}
\usepackage{fancyhdr}
\pagestyle{fancy}
\fancyhf{}
\lhead{\small Sect.~\thesection}
\rhead{\small\nouppercase{\leftmark}}
\renewcommand{\subsectionmark}[1]{\markboth{#1}{}}
\cfoot{\thepage}
\def\labelitemii{$\circ$}
\DeclareRobustCommand{\divby}{%
	\mathrel{\vbox{\baselineskip.65ex\lineskiplimit0pt\hbox{.}\hbox{.}\hbox{.}}}%
}

\title{Polynomial -- Đa Thức}
\author{Nguyễn Quản Bá Hồng\footnote{Independent Researcher, Ben Tre City, Vietnam\\e-mail: \texttt{nguyenquanbahong@gmail.com}; website: \url{https://nqbh.github.io}.}}
\date{\today}

\begin{document}
\maketitle
\begin{abstract}
	\textsc{[en]} This text is a collection of problems, from easy to advanced, about polynomial. This text is also a supplementary material for my lecture note on Elementary Mathematics grade 8, which is stored \& downloadable at the following link: \href{https://github.com/NQBH/hobby/blob/master/elementary_mathematics/grade_8/NQBH_elementary_mathematics_grade_8.pdf}{GitHub\texttt{/}NQBH\texttt{/}hobby\texttt{/}elementary mathematics\texttt{/}grade 8\texttt{/}lecture}\footnote{\textsc{url}: \url{https://github.com/NQBH/hobby/blob/master/elementary_mathematics/grade_8/NQBH_elementary_mathematics_grade_8.pdf}.}. The latest version of this text has been stored \& downloadable at the following link: \href{https://github.com/NQBH/hobby/blob/master/elementary_mathematics/grade_8/polynomial/NQBH_polynomial.pdf}{GitHub\texttt{/}NQBH\texttt{/}hobby\texttt{/}elementary mathematics\texttt{/}grade 8\texttt{/}polynomial}\footnote{\textsc{url}: \url{https://github.com/NQBH/hobby/blob/master/elementary_mathematics/grade_8/polynomial/NQBH_polynomial.pdf}.}.
	\vspace{2mm}
	
	\textsc{[vi]} Tài liệu này là 1 bộ sưu tập các bài tập chọn lọc từ cơ bản đến nâng cao về đa thức. Tài liệu này là phần bài tập bổ sung cho tài liệu chính -- bài giảng \href{https://github.com/NQBH/hobby/blob/master/elementary_mathematics/grade_8/NQBH_elementary_mathematics_grade_8.pdf}{GitHub\texttt{/}NQBH\texttt{/}hobby\texttt{/}elementary mathematics\texttt{/}grade 8\texttt{/}lecture} của tác giả viết cho Toán Sơ Cấp lớp 8. Phiên bản mới nhất của tài liệu này được lưu trữ \& có thể tải xuống ở link sau: \href{https://github.com/NQBH/hobby/blob/master/elementary_mathematics/grade_8/polynomial/NQBH_polynomial.pdf}{GitHub\texttt{/}NQBH\texttt{/}hobby\texttt{/}elementary mathematics\texttt{/}grade 8\texttt{/}polynomial}.
\end{abstract}
\setcounter{secnumdepth}{4}
\setcounter{tocdepth}{3}
\tableofcontents

%------------------------------------------------------------------------------%

\section{Nhân Đa Thức}
``\begin{enumerate*}
	\item[\textbf{1.}] Muốn nhân 1 đơn thức với 1 đa thức, ta nhân đơn thức với từng hạng tử của đa thức rồi cộng các  tích với nhau.
	\item[\textbf{2.}] Muốn nhân 1 đa thức với 1 đa thức, ta nhân mỗi hạng tử của đa thức này với từng hạng tử của đa thức kia rồi cộng các tích với nhau.
	\item[\textbf{3.}] Quy tắc nhân 1 đơn thức với 1 đa thức còn được vận dụng theo chiều ngược lại: $AB + AC = A(B + C)$.
	\item[\textbf{4.}] Nếu 2 đa thức $P(x),Q(x)$ luôn có giá trị bằng nhau với mọi giá trị của biến thì 2 đa thức đó gọi là \textit{2 đa thức đồng nhất}, ký hiệu $P(x)\equiv Q(x)$. 2 đa thức $P(x),Q(x)$ (viết dưới dạng thu gọn) là \textit{đồng nhất} khi \& chỉ khi hệ số của các lũy thừa cùng bậc bằng nhau. Đặc biệt, nếu $P(x) = \sum_{i=0}^n a_ix^i = a_nx^n + a_{n-1}x^{n-1} + \cdots + a_1x + a_0$ luôn bằng $0$ với mọi $x\in\mathbb{R}$ thì $a_0 = a_1 = \cdots = a_n = 0$, i.e., $a_i = 0$, $\forall i = 0,1,\ldots,n$.'' -- \cite[Chap. 1, \S1, p. 4]{Tuyen_Toan_8}
\end{enumerate*}

\begin{baitoan}[\cite{Tuyen_Toan_8}, Ví dụ 1, p. 4]
	Cho $P = (x + 5)(ax^2 + bx + 25)$ \& $Q = x^3 + 125$.
	\begin{enumerate*}
		\item[(a)] Viết $P$ dưới dạng 1 đa thức thu gọn theo lũy thừa giảm dần của $x$.
		\item[(b)] Với giá trị nào của $a,b$ thì $P = Q$, $\forall x\in\mathbb{R}$.
	\end{enumerate*}	
\end{baitoan}

\begin{proof}[Giải]
	\begin{enumerate*}
		\item[(a)] $P = (x + 5)(ax^2 + bx + 25) = ax^3 + bx^2 + 25x + 5ax^2 + 5bx + 125 = ax^3 + (5a + b)x^2 + (5b + 25)x + 125$.
		\item[(b)] $P = Q$, $\forall x\in\mathbb{R}\Leftrightarrow ax^3 + (5a + b)x^2 + (5b + 25)x + 125 = x^3 + 125$, $\forall x\in\mathbb{R}\Leftrightarrow(a = 1)\land(5a + b = 0)\land(5b + 25 = 0)\Leftrightarrow(a = 1)\land(b = -5)$.
	\end{enumerate*}
\end{proof}
\noindent\textit{Nhận xét.} ``Phương pháp giải (b) dựa vào tính chất: 2 đa thức $P,Q$ (viết dưới dạng thu gọn) là đồng nhất khi \& chỉ khi mọi hệ số của các đơn thức đồng dạng chứa trong 2 đa thức đó phải bằng nhau.'' -- \cite[p. 5]{Tuyen_Toan_8}


\begin{baitoan}[\cite{Tuyen_Toan_8}, \textbf{1.}, p. 5]
	Tính giá trị của các biểu thức sau bằng cách hợp lý:\\
	\begin{enumerate*}
		\item[(a)] $A = x^5 - 100x^4 + 100x^3 - 100x^2 + 100x - 9$ tại $x = 99$.
		\item[(b)] $B = x^6 - 20x^5 - 20x^4 - 20x^3 - 20x^2 - 20x + 3$ tại $x  = 21$.
		\item[(c)] $C = x^7 - 26x^6 + 27x^5 - 47x^4 - 77x^3 + 50x^2 + x - 24$ tại $x = 25$.
	\end{enumerate*}
\end{baitoan}

\begin{baitoan}[\cite{Tuyen_Toan_8}, \textbf{2.}, p. 5]
	Cho $x,y\in\mathbb{Z}$. Chứng minh:
	\begin{enumerate*}
		\item[(a)] Nếu $A = 5x + y\divby19$ thì $B = 4x - 3y\divby19$.
		\item[(b)] Nếu $C = 4x + 3y\divby13$ thì $D = 7x + 2y\divby13$.
	\end{enumerate*}
\end{baitoan}

\begin{baitoan}[\cite{Tuyen_Toan_8}, \textbf{3.}, p. 5]
	Cho 4 số lẻ liên tiếp. Chứng minh hiệu của tích 2 số cuối với tích 2 số đầu chia hết cho $16$.
\end{baitoan}

\begin{baitoan}[\cite{Tuyen_Toan_8}, \textbf{4.}, pp. 5--6]
	Cho 4 số nguyên liên tiếp.
	\begin{enumerate*}
		\item[(a)] Hỏi tích của số đầu với số cuối nhỏ hơn tích của 2 số ở giữa bao nhiêu đơn vị?
		\item[(b)] Giả sử tích của số đầu với số thứ 3 nhỏ hơn tích của số thứ 2 \& số thứ 4 là $99$, tìm 4 số nguyên đó.
	\end{enumerate*}
\end{baitoan}

\begin{baitoan}[\cite{Tuyen_Toan_8}, \textbf{5.}, p. 6]
	Cho $b + c = 10$. Chứng minh đẳng thức $(10a + b)(10a + c) = 100a(a + 1) + bc$. Áp dụng để tính nhẩm: $62\cdot68$, $43\cdot47$.
\end{baitoan}

\begin{baitoan}[\cite{Tuyen_Toan_8}, \textbf{6.}, p. 6]
	Xác định các hệ số $a,b,c$ biết:
	\begin{enumerate*}
		\item[(a)] $(2x - 5)(3x + b) = ax^2 + x + c$.\\
		\item[(b)] $(ax + b)(x^2 - x - 1) = ax^3 + cx^2 - 1$.
	\end{enumerate*}
\end{baitoan}

\begin{baitoan}[\cite{Tuyen_Toan_8}, \textbf{7.}, p. 6]
	Cho $m\in\mathbb{N}^\star$, $m < 30$. Có bao nhiêu giá trị của $m$ để đa thức $x^2 + mx + 72$ là tích của 2 đa thức bậc nhất với hệ số nguyên?
\end{baitoan}

%------------------------------------------------------------------------------%

\section{Các Hằng Đẳng Thức Đáng Nhớ}
``\textbf{1.} $(A + B)^2 = A^2 + 2AB + B^2$. \textbf{2.} $(A - B)^2 = A^2 - 2AB + B^2$. \textbf{3.} $(A - B)(A + B) = A^2 - B^2$. \textbf{4.} $(A + B)^3 = A^3 + 3A^2B + 3AB^2 + B^3 = A^3 + B^3 + 3AB(A + B)$. \textbf{5.} $(A - B)^3 = A^3 - 3A^2B + 3AB^2 - B^3 = A^3 - B^3 - 3AB(A + B)$. \textbf{6.} $(A + B)(A^2 - AB + B^2) = A^3 + B^3$. \textbf{7.} $(A - B)(A^2 + AB + B^2) = A^3 - B^3$. \textbf{8.} Bình phương của đa thức: $(a + b + c)^2 = a^2 + b^2 + c^2 + 2ab + 2bc + 2ca$, $(a + b + c + d)^2 = a^2 + b^2 + c^2 + d^2 + 2ab + 2ac + 2ad + 2bc + 2bd + 2cd$, $\ldots$. \textbf{9.} Lũy thừa bậc $n$ của 1 nhị thức (nhị thức Newton):
\begin{align*}
	(a + b)^0 &= 1,\\
	(a + b)^1 &= 1a + 1b,\\
	(a + b)^2 &= 1a^2 + 2ab + 1b^2,\\
	(a + b)^3 &= 1a^3 + 3a^2b + 3ab^2 + 1b^3,\\
	(a + b)^4 &= 1a^4 + 4a^3b + 6a^2b^2 + 4ab^3 + 1b^4,\\
	(a + b)^5 &= 1a^5 + 5a^4b + 10a^3b^2 + + 10a^2b^3 + 5ab^4 + 1b^5.
\end{align*}
Ta thấy khi khai triển $(a + b)^n$ ta được 1 đa thức có $n + 1$ hạng tử, hạng tử dầu là $a^n$, hạng tử cuối là $b^n$, các hạng tử còn lại đều chứa các nhân tử $a$ \& $b$. Vì vậy $(a + b)^n = B(a) + b^n = B(b) + a^n$. \textbf{10.} Nếu viết riêng các hệ số ở vế phải ta được bảng sau (gọi là \textit{tam giác Pascal}):
\begin{align*}
	&1\\
	&1\ \ 1\\
	&1\ \ 2\ \ 1\\
	&1\ \ 3\ \ 3\ \ 1\\
	&1\ \ 4\ \ 6\ \ 4\ \ 1\\
	&1\ \ 5\ \ 10\ \ 10\ \ 5\ \ 1\\
	&\ldots
\end{align*}
Nhận xét: Mỗi dòng đều bắt đầu bằng 1 \& kết thúc bằng 1. Mỗi số trên 1 dòng kể từ dòng thứ 2 đều bằng số liền trên cộng với số bên trái của số liền trên.'' -- \cite[\S2, pp. 6--7]{Tuyen_Toan_8}

\begin{baitoan}[\cite{Tuyen_Toan_8}, Ví dụ 2, p. 7]
	Cho $x + y = 9$, $xy = 14$. Tính giá trị của các biểu thức sau:
	\begin{enumerate*}
		\item[(a)] $x - y$;
		\item[(b)] $x^2 + y^2$;
		\item[(c)] $x^3 + y^3$.
	\end{enumerate*}
\end{baitoan}

\begin{proof}[Giải]
	\begin{enumerate*}
		\item[(a)] $(x - y)^2 = (x + y)^2 - 4xy = 9^2 - 4\cdot14 = 25\Rightarrow x - y = \pm5$.
		\item[(b)] $x^2 + y^2 = (x + y)^2 - 2xy = 9^2 - 2\cdot14 = 53$.
		\item[(c)] $x^3 + y^3 = (x + y)^3 - 3xy(x + y) = 9^3 - 3\cdot14\cdot9 = 351$.
	\end{enumerate*}
\end{proof}

\begin{luuy}
	``2 số có bình phương bằng nhau thì chúng đối nhau hoặc bằng nhau. Ngược lại, 2 số đối nhau hoặc bằng nhau thì có bình phương bằng nhau. $(a - b)^2 = (b - a)^2 = a^2 - 2ab + b^2$, $\forall a,b\in\mathbb{R}$.'' -- \cite[p. 8]{Tuyen_Toan_8}
\end{luuy}

\begin{baitoan}[Mở rộng \cite{Tuyen_Toan_8}, p. 7]
	Cho $x + y = a$, $xy = b$, với $a,b\in\mathbb{R}$, $a^2\ge 4b$. Tính giá trị của các biểu thức sau theo $a,b$:
	\begin{enumerate*}
		\item[(a)] $x - y$;
		\item[(b)] $x^2 + y^2$;
		\item[(c)] $x^2 - y^2$;
		\item[(d)] $x^3 + y^3$;
		\item[(e)] $x^3 - y^3$.
	\end{enumerate*}
\end{baitoan}

\begin{proof}[Giải]
	(a) $(x - y)^2 = (x + y)^2 - 4xy = a^2 - 4\cdot b\Rightarrow x - y = \pm\sqrt{a^2 - 4b}$. (b) $x^2 + y^2 = (x + y)^2 - 2xy = a^2 - 2\cdot b$. (c) $x^2 - y^2 = (x - y)(x + y) = \pm a\sqrt{a^2 - 4b}$. (d) $x^3 + y^3 = (x + y)^3 - 3xy(x + y) = a^3 - 3ab$. (e) $x^3 - y^3 = (x - y)(x^2 + xy + y^2) = \pm\sqrt{a^2 - 4b}(a^2 - 2b + b) = \pm\sqrt{a^2 - 4b}(a^2 - b)$.
\end{proof}

\begin{baitoan}[\cite{Tuyen_Toan_8}, Ví dụ 3, p. 8]
	Tìm giá trị nhỏ nhất của biểu thức: $A = (x + 3y - 5)^2 - 6xy + 26$.
\end{baitoan}

\begin{proof}[Giải]
	$A = x^2 + 9y^2 + 25 + 6xy - 10x - 30y - 6xy + 26 = (x^2 - 10x + 25) + (9y^2 - 30y + 25) + 1 = (x - 5)^2 + (3y - 5)^2 + 1\ge 1$, $\forall x,y\in\mathbb{R}\Rightarrow\min A = 1\Leftrightarrow(x = 5)\land\left(y = \frac{5}{3}\right)$.
\end{proof}

\begin{luuy}
	\begin{enumerate*}
		\item[(a)] ``Các hằng đẳng thức được vận dụng theo 2 chiều ngược nhau, e.g., $(a - b)^2 = a^2 - 2ab + b^2$ hoặc ngược lại $a^2 - 2ab + b^2 = (a - b)^2$.
		\item[(b)] Bình phương của mọi số thực đều không âm: $x^2\ge 0$, $\forall x\in\mathbb{R}$, ``$=$'' xảy ra $\Leftrightarrow x = 0$; hay tương đương với $(a - b)^2\ge 0$, $\forall a,b\in\mathbb{R}$, ``$=$'' xảy ra $\Leftrightarrow a = b$.'' -- \cite[p. 9]{Tuyen_Toan_8}
	\end{enumerate*}
\end{luuy}

\begin{baitoan}[\cite{Tuyen_Toan_8}, \textbf{8.}, p. 9]
	Chứng minh các đẳng thức:
	\begin{enumerate*}
		\item[(a)] $(2 + 1)(2^2 + 1)(2^4 + 1)(2^8 + 1)(2^{16} + 1) = 2^{32} - 1$;
		\item[(b)] $100^2 + 103^2 + 105^2 + 94^2 = 101^2 + 98^2 + 96^2 + 107^2$.
	\end{enumerate*}
\end{baitoan}

\begin{baitoan}[\cite{Tuyen_Toan_8}, \textbf{9.}, p. 9]
	Tính giá trị của biểu thức bằng cách hợp lý:
	\begin{enumerate*}
		\item[(a)] $A = \frac{258^2 - 242^2}{254^2 - 246^2}$;
		\item[(b)] $B = 263^2 + 74\cdot263 + 37^2$;
		\item[(c)] $C = 136^2 - 92\cdot136 + 46^2$;
		\item[(d)] $D = (50^2 + 48^2 + 46^2 + \cdots + 2^2) - (49^2 + 47^2 + 45^2 + \cdots + 1^2)$.
	\end{enumerate*}
\end{baitoan}

\begin{baitoan}[\cite{Tuyen_Toan_8}, \textbf{10.}, p. 9]
	Cho biết $2(a^2 + b^2) = (a - b)^2$. Chứng minh $a$ \& $b$ đối nhau.
\end{baitoan}

\begin{baitoan}[\cite{Tuyen_Toan_8}, \textbf{11.}, p. 9]
	Cho $a,b,x,y\in\mathbb{R}\backslash\{0\}$. Biết $(a^2 + b^2)(x^2 + y^2) = (ax + by)^2$. Tìm hệ thức giữa 4 số $a,b,x,y$.
\end{baitoan}

\begin{baitoan}[\cite{Tuyen_Toan_8}, \textbf{12.}, p. 9]
	Cho $a^2 + b^2 + c^2 = ab + bc + ca$. Chứng minh $a = b = c$.
\end{baitoan}

\begin{baitoan}[\cite{Tuyen_Toan_8}, \textbf{13.}, p. 9]
	Chứng minh không có các số $x,y\in\mathbb{R}$ nào thỏa mãn mỗi đẳng thức sau: (a) $3x^2 + y^2 + 10x - 2xy + 26 = 0$; (b) $4x^2 + 3y^2 - 4x + 30y + 78 = 0$; (c) $3x^2 + 6y^2 - 12x - 20y + 40 = 0$.
\end{baitoan}

\begin{baitoan}[\cite{Tuyen_Toan_8}, \textbf{14.}, p. 10]
	Tìm $x\in\mathbb{R},n\in\mathbb{N}$ thỏa $x^2 + 2x + 4^n - 2^{n + 1} + 2 = 0$.
\end{baitoan}

\begin{baitoan}[\cite{Tuyen_Toan_8}, \textbf{15.}, p. 10]
	Chứng minh: (a) Biểu thức $A = x^2 + x + 1$ luôn luôn dương với mọi số thực $x$; (b) Biểu thức $B = x^2 - xy + y^2$ luôn luôn dương với mọi số thực $x,y$ không đồng thời bằng $0$; (c) Biểu thức $C = 4x - 10 - x^2$ luôn luôn âm với mọi số thực $x$.
\end{baitoan}

\begin{baitoan}[\cite{Tuyen_Toan_8}, \textbf{16.}, p. 10]
	Tìm giá trị nhỏ nhất của biểu thức: (a) $A = 25x^2 + 3y^2 - 10x + 11$; (b) $B = (x - 3)^2 + (x - 11)^2$; (c) $C = (x + 1)(x - 2)(x - 3)(x - 6)$.
\end{baitoan}

\begin{baitoan}[\cite{Tuyen_Toan_8}, \textbf{17.}, p. 10]
	Tìm giá trị lớn nhất của biểu thức: (a) $A = 2x - x^2$; (b) $B = 19 - 6x - 9x^2$.
\end{baitoan}

\begin{baitoan}[\cite{Tuyen_Toan_8}, \textbf{18.}, p. 10]
	Chứng minh: (a) 2 số chẵn hơn kém nhau $4$ đơn vị thì hiệu các bình phương của chúng chia hết cho $16$; (b) 2 số lẻ hơn kém nhau $6$ đơn vị thì hiệu các bình phương của chúng chia hết cho $24$.
\end{baitoan}

\begin{baitoan}[\cite{Tuyen_Toan_8}, \textbf{19.}, p. 10]
	Cho $x > y > 0$ \& $x - y = 7$, $xy = 60$. Không tính $x,y$, tính: (a) $x^2 - y^2$; (b) $x^4 + y^4$.
\end{baitoan}

\begin{baitoan}[\cite{Tuyen_Toan_8}, \textbf{20.}, p. 10]
	Cho $a + b + c = 2p$. Chứng minh: (a) $a^2 - b^2 - c^2 + 2abc = 4(p - b)(p - c)$; (b) $p^2 + (p - a)^2 + (p - b)^2 + (p - c)^2 = a^2 + b^2 + c^2$.
\end{baitoan}

\begin{baitoan}[\cite{Tuyen_Toan_8}, \textbf{21.}, p. 10]
	Cho $a = m^2 + n^2$, $b = m^2 - n^2$, $c = 2mn$. Chứng minh nếu $m > n > 0$ thì $a,b,c$ là độ dài 3 cạnh của 1 tam giác vuông.
\end{baitoan}

\begin{baitoan}[\cite{Tuyen_Toan_8}, \textbf{22.}, p. 11]
	Tính giá trị của biểu thức: $A = x^3 + 9x^2 + 27x + 27$ với $x = -103$; (b) $B = x^3 - 15x^2 + 75x$ với $x = 25$; (c) $C = (x + 1)(x - 1)(x^2 + x + 1)(x^2 - x + 1)$ với $x = -3$.
\end{baitoan}

\begin{baitoan}[\cite{Tuyen_Toan_8}, \textbf{23.}, p. 11]
	Cho $x - y = 2$, tính giá trị của biểu thức: $A = 2(x^3 - y^3) - 3(x + y)^2$.
\end{baitoan}

\begin{baitoan}[\cite{Tuyen_Toan_8}, \textbf{24.}, p. 11]
	Cho $x + y + z = 0$. Chứng minh $x^3 + y^3 + z^3 = 3xyz$.
\end{baitoan}

\begin{baitoan}[\cite{Tuyen_Toan_8}, \textbf{25.}, p. 11]
	Rút gọn biểu thức $A = (x - y - 1)^3 - (x - y + 1)^3 + 6(x - y)^2$.
\end{baitoan}

\begin{baitoan}[\cite{Tuyen_Toan_8}, \textbf{26.}, p. 11]
	Giải hệ phương trình:
	\begin{equation*}
		\left\{\begin{split}
			(x + 2y)(x^2 - 2xy + 4y^2) &= 0,\\
			(x - 2y)(x^2 + 2xy + 4y^2) &= 16.
		\end{split}\right.
	\end{equation*}
\end{baitoan}

\begin{baitoan}[\cite{Tuyen_Toan_8}, \textbf{27.}, p. 11]
	Chứng minh: (a) $742^3 - 692^3\divby200$; (b) $685^3 + 315^3\divby25000$.
\end{baitoan}

\begin{baitoan}[\cite{Tuyen_Toan_8}, \textbf{28${}^\star$.}, p. 11]
	Cho $a + b + c + d = 0$. Chứng minh $a^3 + b^3 + c^3 + d^3 = 3(b + c)(ad - bc)$.
\end{baitoan}

\begin{baitoan}[\cite{Tuyen_Toan_8}, \textbf{29.}, p. 11]
	Cho $a + b + c = 0$. Chứng minh: (a) $(ab + bc + ca)^2 = a^2b^2 + b^2c^2 + c^2a^2$; (b) $a^4 + b^4 + c^4 = 2(ab + bc + ca)^2$.
\end{baitoan}

\begin{baitoan}[\cite{Tuyen_Toan_8}, \textbf{30.}, p. 11]
	Xác định các hệ số $a,b$ để đa thức $A = x^4 - 2x^3 + 3x^2 + ax + b$ là bình phương của 1 đa thức.
\end{baitoan}

\begin{baitoan}[\cite{Tuyen_Toan_8}, \textbf{31.}, p. 11]
	Cho $a + b + c = 0$, $a^2 + b^2 + c^2 = 1$. Chứng minh $a^4 + b^4 + c^4 = \frac{1}{2}$.
\end{baitoan}

\begin{baitoan}[\cite{Tuyen_Toan_8}, \textbf{32.}, pp. 11--12]
	Cho $a,b,c$ là 3 số thực không đồng thời bằng $0$. Chứng minh có ít nhất 1 trong các biểu thức sau có giá trị dương: $x = (a - b + c)^2 + 8ab$, $y = (a - b +  c)^2 + 8bc$, $z = (a - b + c)^2 - 8ca$.
\end{baitoan}

\begin{baitoan}[\cite{Tuyen_Toan_8}, \textbf{33.}, p. 12]
	Tính tổng các hệ số của tất cả các hạng tử trong khai triển của nhị thức: (a) $(5x - 3)^6$; (b) $(3x - 4y)^{20}$.
\end{baitoan}

\begin{baitoan}[\cite{Tuyen_Toan_8}, \textbf{34.}, p. 12]
	Đa thức $(x + y)^5$ được khai triển theo lũy thừa giảm của $x$. Biết hạng tử thứ 2 \& hạng tử thứ 3 có giá trị bằng nhau khi cho $x = a$, $y = b$ trong đó $a,b$ là các số thực dương \& $a - b = 1$. Tìm $a,b$.
\end{baitoan}

\begin{baitoan}[\cite{Tuyen_Toan_8}, \textbf{35.}, p. 12]
	Tính: (a) $(x + 2)^2$; (b) $(x - 1)^6$; (c) $(x - 1)^5$.
\end{baitoan}

\begin{baitoan}[\cite{Tuyen_Toan_8}, \textbf{36.}, p. 12]
	Tìm số dư của phép chia $38^{10}$ cho $13$ \& $38^9$ cho $13$.
\end{baitoan}

\begin{baitoan}[\cite{Tuyen_Toan_8}, \textbf{37.}, p. 12]
	Chứng minh 2 chữ số tận cùng của $7^{43}$ là $43$.
\end{baitoan}

%------------------------------------------------------------------------------%

\section{Phân Tích Đa Thức Thành Nhân Tử}
``\textbf{1.} Phân tích đa thức thành nhân tử là biến đổi đa thức đó thành 1 tích của những đa thức. \textbf{2.} Các phương pháp thông thường: $\bullet$ \textit{Phương pháp đặt nhân tử chung}: $AB + AC - AD = A(B + C - D)$. $\bullet$ \textit{Phương pháp dùng hằng đẳng thức}: $A^2\pm2AB + B^2 = (A\pm B)^2$, $A^3\pm3A^2B + 3AB^2\pm B^3 = (A\pm B)^3$, $A^2 - B^2 = (A - B)(A + B)$, $A^3 - B^3 = (A - B)(A^2 + AB + B^2)$, $A^3 + B^3 = (A + B)(A^2 - AB + B^2)$. $\bullet$ \textit{Phương pháp nhóm các hạng tử}: $AC - AD + BC - BD = A(C - D) + B(C - D) = (A + B)(C - D)$. \textbf{3.} Dạng tổng quát của các hằng đẳng thức hiệu 2 bình phương, hiệu 2 lập phương: $A^n - B^n = (A - B)\sum_{i=0}^{n-1} A^{n - 1 - i}B^i = (A - B)(A^{n-1} + A^{n-2}B + \cdots + AB^{n-2} + B^{n-1})$. \textbf{4.} Dạng tổng quát của hằng đẳng thức tổng 2 lập phương: $A^n + B^n = (A + B)\sum_{i=0}^{n-1} (-1)^iA^{n - 1 - i}B^i = (A - B)(A^{n-1} - A^{n-2}B + A^{n-3}B^2 - \cdots + AB^{n-2} + B^{n-1})$ với $n$ lẻ. \textbf{5.} Áp dụng vào tính chất chia hết: $A^n - B^n\divby A - B$, $\forall n\in\mathbb{N}$, $A\ne B$; $A^n + B^n\divby A + B$, $\forall n\in\mathbb{N}$, $n$ lẻ, $A\ne-B$; $A^{2k} - B^{2k}\divby A^2 - B^2$, $\forall k\in\mathbb{N}$, $A\ne B$.'' -- \cite[\S3, pp. 12--13]{Tuyen_Toan_8}

\begin{baitoan}[\cite{Tuyen_Toan_8}, Ví dụ 4, p. 13]
	Cho $x,y\in\mathbb{R}$, $x\ne y$, thỏa mãn điều kiện $9x(x - y) - 10(y - x)^2 = 0$. Chứng minh $x = 10y$.
\end{baitoan}

\begin{proof}[Giải]
	$0 = 9x(x - y) - 10(y - x)^2 = 9x(x - y) - 10(x - y)^2 = (x - y)[9x - 10(x - y)] = (x - y)(-x + 10y)\Rightarrow(x = y)\lor(x = 10y)$, mà $x\ne y$, nên $x = 10y$.
\end{proof}
``Phân tích đa thức thành nhân tử có nhiều ứng dụng như để tính giá trị của biểu thức, chứng minh tính chia hết hoặc như trong ví dụ trên, để tìm mối quan hệ giữa các biến, $\ldots$'' -- \cite[p. 14]{Tuyen_Toan_8}

\begin{baitoan}[\cite{Tuyen_Toan_8}, \textbf{38.}, p. 14]
	Phân tích các đa thức sau thành nhân tử: (a) $5x(x - 2y) + 2(2y - x)^2$; (b) $7x(y - 4)^2 - (4 - y)^3$; (c) $(4x - 8)(x^2 + 6) - (4x - 8)(x + 7) + 9(8 - 4x)$.
\end{baitoan}

\begin{baitoan}[\cite{Tuyen_Toan_8}, \textbf{39.}, p. 14]
	Chứng minh: (a) $43^2 + 43\cdot17\divby60$; (b) $27^5 - 3^{11}\divby80$.
\end{baitoan}

\begin{baitoan}[\cite{Tuyen_Toan_8}, \textbf{40.}, p. 14]
	Tìm 1 số biết 3 lần bình phương của nó đúng bằng 2 lần lập phương của số ấy.
\end{baitoan}

\begin{baitoan}[\cite{Tuyen_Toan_8}, \textbf{41.}, p. 14]
	Có các số nguyên $x,y,z$ nào thỏa mãn hệ phương trình sau không?
	\begin{equation*}
		\left\{\begin{split}
			x^3 + xyz &= 957,\\
			y^3 + xyz &= 795,\\
			z^3 + xyz &= 579.
		\end{split}\right.
	\end{equation*}
\end{baitoan}

\begin{baitoan}[\cite{Tuyen_Toan_8}, \textbf{42.}, p. 14]
	Chứng minh số $\underbrace{11\ldots1}_n\underbrace{22\ldots2}_n$ là tích của 2 số nguyên liên tiếp.
\end{baitoan}

\begin{baitoan}[\cite{Tuyen_Toan_8}, \textbf{43.}, p. 15]
	Phân tích các đa thức sau thành nhân tử: (a) $100x^2 - (x^2 + 25)^2$; (b) $(x - y + 5)^2 - 2(x - y + 5) + 1$.
\end{baitoan}

\begin{baitoan}[\cite{Tuyen_Toan_8}, \textbf{44.}, p. 15]
	Phân tích đa thức thành nhân tử: $(x^2 + 4y^2 - 5)^2 - 16(x^2y^2 + 2xy + 1)$.
\end{baitoan}

\begin{baitoan}[\cite{Tuyen_Toan_8}, \textbf{45.}, p. 15]
	Cho $A = 4a^2b^2 - (a^2 + b^2 + c^2)$ trong đó $a,b,c$ là độ dài 3 cạnh của 1 tam giác. Chứng minh $A > 0$.
\end{baitoan}

\begin{baitoan}[\cite{Tuyen_Toan_8}, \textbf{46.}, p. 15]
	Chứng minh: (a) $21^{10} - 1\divby200$; (b) $39^{20} + 39^{13}\divby40$; (c) $2^{60} + 5^{30}\divby41$; (d) $2005^{2007} + 2007^{2005}\divby2006$.
\end{baitoan}

\begin{baitoan}[\cite{Tuyen_Toan_8}, \textbf{47.}, p. 15]
	Cho $n$ là 1 số tự nhiên lẻ. Chứng minh $24^n + 1$ chia hết cho $25$ nhưng không chia hết cho $23$.
\end{baitoan}

\begin{baitoan}[\cite{Tuyen_Toan_8}, \textbf{48.}, p. 15]
	Cho $a$ là 1 số nguyên lẻ, $a > 1$. Chứng minh $(a - 1)^{\frac{1}{2}(a - 1)} - 1\divby a - 2$.
\end{baitoan}

\begin{baitoan}[\cite{Tuyen_Toan_8}, \textbf{49.}, p. 15]
	Phân tích các đa thức sau thành nhân tử: (a) $x^2 - xz - 9y^2 + 3yz$; (b) $x^3 - x^2 - 5x + 125$; (c) $x^3 + 2x^2 - 6x - 27$; (d) $12x^3 + 4x^2 - 27x - 9$.
\end{baitoan}

\begin{baitoan}[\cite{Tuyen_Toan_8}, \textbf{50.}, p. 15]
	Phân tích các đa thức sau thành nhân tử: (a) $x^4 - 25x^2 + 20x - 4$; (b) $x^2(x^2 - 6) - x^2 + 9$; (c) $ab(x^2 + y^2) - xy(a^2 + b^2)$.
\end{baitoan}

\begin{baitoan}[\cite{Tuyen_Toan_8}, \textbf{51.}, p. 15]
	Tìm $x,y\in\mathbb{R}$ sao cho $x - y = xy - 1$.
\end{baitoan}

\begin{baitoan}[\cite{Tuyen_Toan_8}, \textbf{52.}, p. 15]
	Cho $x,y\in\mathbb{R}$, $x\ne y$ sao cho $x^2 - y = y^2 - x$. Tính giá trị của biểu thức $A = x^2 + 2xy + y^2 - 3x - 3y$.
\end{baitoan}

\begin{baitoan}[\cite{Tuyen_Toan_8}, \textbf{53.}, p. 16]
	Cho $\frac{a - b}{b - c} = \frac{c - d}{d - a}$. Chứng minh hoặc $a = c$ hoặc $a + c = b + d$.
\end{baitoan}

\begin{baitoan}[\cite{Tuyen_Toan_8}, \textbf{54.}, p. 16]
	Phân tích các đa thức sau thành nhân tử: (a) $4x^4 + 4x^3 - x^2 - x$; (b) $x^6 - x^4 - 9x^3 + 9x^2$; (c) $x^4 - 4x^3 + 8x^2 - 16x + 16$.
\end{baitoan}

\begin{baitoan}[\cite{Tuyen_Toan_8}, \textbf{55.}, p. 16]
	Phân tích các đa thức sau thành nhân tử: (a) $(xy + 4)^2 - 4(x + y)^2$; (b) $(ab - xy)^2 - (bx - ay)^2$; (c) $(x^2 + 8x - 34)^2 - (3x^2 - 8x - 2)^2$.
\end{baitoan}

\begin{baitoan}[\cite{Tuyen_Toan_8}, \textbf{56.}, p. 16]
	Phân tích các đa thức sau thành nhân tử: (a) $(a + b + c)^2 + (a - b + c)^2 - 4b^2$; (b) $a(b^2 - c^2) - b(c^2 - a^2) + c(a^2 - b^2)$; (c) $a^5 + b^5 - (a + b)^5$.
\end{baitoan}

\begin{baitoan}[\cite{Tuyen_Toan_8}, \textbf{57.}, p. 16]
	Chứng minh: (a) $999^4 + 999$ có tận cùng bằng $3$ chữ số $0$; (b) $49^5 - 49\divby100$.
\end{baitoan}

\begin{baitoan}[\cite{Tuyen_Toan_8}, \textbf{58.}, p. 16]
	Chứng minh: (a) Lập phương của 1 số nguyên trừ đi số nguyên đó thì chia hết cho $6$; (b) Nếu tổng của 3 số nguyên chia hết cho $6$ thì tổng các lập phương của chúng chia hết cho $6$.
\end{baitoan}

\begin{baitoan}[\cite{Tuyen_Toan_8}, \textbf{59.}, p. 16]
	Cho $a\ne\pm b$ \&  $a(a + b)(a + c) = b(b + c)(b + a)$. Chứng minh $a + b + c = 0$.
\end{baitoan}

\begin{baitoan}[\cite{Tuyen_Toan_8}, \textbf{60.}, p. 16]
	Cho $x^2y - y^2x + x^2z - z^2x + y^2z + z^2y = 2xyz$. Chứng minh trong 3 số $x,y,z$ ít nhất cũng có 2 số bằng nhau hoặc đối nhau.
\end{baitoan}

\subsection{Phương Pháp Tách 1 Hạng Tử Thành Nhiều Hạng Tử}

\begin{baitoan}[\cite{Tuyen_Toan_8}, Ví dụ 5, p. 17]
	Phân tích đa thức thành nhân tử: $A = 4x^2 - 8x + 3$.
\end{baitoan}

\begin{proof}[1st Giải]
	(Tách hạng tử cuối) $A = 4x^2 - 8x + 4 - 1 = (2x - 2)^2 - 1^2 = (2x - 2 - 1)(2x - 2 + 1) = (2x - 3)(2x - 1)$.
\end{proof}

\begin{proof}[2nd Giải]
	(Tách hạng tử 2nd) $A = 4x^2 - 2x - 6x + 3 = 2x(2x - 1) - 3(2x - 1) = (2x - 1)(2x - 3)$.
\end{proof}

\begin{proof}[3rd Giải]
	(Tách hạng tử 2nd) $A = 4x^2 - 6x - 2x + 3 = 2x(2x - 3) - (2x - 3) = (2x - 3)(2x - 1)$.
\end{proof}

\begin{proof}[4th Giải]
	$A = 4x^2 - 9 - 8x + 12 = (2x - 3)(2x + 3) - 4(2x - 3) = (2x - 3)(2x + 3 - 4) = (2x - 3)(2x - 1)$.
\end{proof}
\noindent\textit{Nhận xét.} ``\textbf{1.} Ta nhận thấy với các phương pháp thông thường thì không thể phân tích $A$ thành nhân tử được vì $A$ không có nhân tử chung, không có dạng 1 hằng đẳng thức nào. Đa thức $A$ chỉ có 3 hạng tử nên cũng không thể dùng phương pháp nhóm hạng tử. Vì vậy ta đã tách 1 hạng tử thành 2 hạng tử để xuất hiện những nhóm hạng tử sao cho: $\bullet$ Hoặc có thể dùng hằng đẳng thức để phân tích tiếp; $\bullet$ Hoặc có thể đặt nhân tử chung. \textbf{2.} Trong cách giải 2nd, ta đã tách hạng tử thứ 2 là $-8x$ thành $-2x - 6x$. Ta thấy $(-2)(-6) = 12$. Trong khi đó tích các hệ số đầu \& cuối là $4\cdot3 = 12$. 2 tích này đúng bằng nhau. 1 cách tổng quát, để phân tích tam thức bậc 2 $ax^2 + bx + c$ thành nhân tử, ta tách hạng tử bậc nhất $bx$ thành $b_1x + b_2x$ sao cho $b_1b_2 = ac$ sau đó đặt nhân tử chung theo từng nhóm. \textbf{3.} Đối với các đa thức có bậc 3 trở lên thì tùy theo đặc điểm của các hệ số mà có cách tách riêng cho phù hợp.'' -- \cite[pp. 17--18]{Tuyen_Toan_8}

\begin{baitoan}[\cite{Tuyen_Toan_8}, p. 18]
	Phân tích đa thức thành nhân tử: $A = x^3 + 5x^2 + 3x - 9$.
\end{baitoan}

\begin{proof}[1st Giải]
	$A = x^3 - x^2 + 6x^2 - 6x + 9x - 9 = x^2(x - 1) + 6x(x - 1) + 9(x - 1) = (x - 1)(x^2 + 6x + 9) = (x - 1)(x + 3)^2$.
\end{proof}

\begin{proof}[2nd Giải]
	$A = x^3 + 3x^2 + 2x^2 + 6x - 3x - 9 = x^2(x + 3) + 2x(x + 3) - 3(x + 3) = (x + 3)(x^2 + 2x - 3) = (x + 3)(x - 1)(x + 3) = (x - 1)(x + 3)^2$.
\end{proof}
Trong đó có nhiều cách phân tích $B\coloneqq x^2 + 2x - 3$ thành nhân tử, e.g.: $\bullet$ $B = x^2 - x + 3x - 3 = x(x - 1) + 3(x - 1) = (x - 1)(x + 3)$. $\bullet$ $B = x^2 + 3x - x - 3 = x(x + 3) - (x + 3) = (x + 3)(x - 1)$. $\bullet$ $B = x^2 + 2x + 1 - 4 = (x + 1)^2 - 4 = (x + 1 - 2)(x + 1 + 2) = (x - 1)(x + 3)$.

\begin{proof}[3rd Giải]
	$A = x^3 + 6x^2 + 9x - x^2 - 6x - 9 = x(x^2 + 6x - 9) - (x^2 + 6x + 9) = (x - 1)(x^2 + 6x + 9) = (x - 1)(x + 3)^2$.
\end{proof}

\begin{proof}[4th Giải]
	$A = x^3 + 2x^2 - 3x + 3x^2 + 6x - 9 = x(x^2 + 2x - 3) + 3(x^2 + 2x - 3) = (x + 3)(x^2 + 2x - 3) = (x + 3)(x + 3)(x - 1) = (x + 3)^2(x - 1)$, trong đó các cách phân tích $x^2 + 2x - 3$ thành nhân tử đã được trình bày ở 2nd Giải.
\end{proof}

\subsection{Phương Pháp Thêm Bớt Cùng Hạng Tử}

\begin{baitoan}[\cite{Tuyen_Toan_8}, Ví dụ 6, p. 18]
	Phân tích đa thức thành nhân tử $A = 4x^4 + y^4$.
\end{baitoan}

\begin{proof}[1st Giải]
	$A = 4x^4 + 4x^2y^2 + y^4 - 4x^2y^2 = (2x^2 + y^2)^2 - (2xy)^2 = (2x^2 - 2xy + y^2 )(2x^2 + 2xy + y^2)$.
\end{proof}
\noindent\textit{Nhận xét.} ``Ta dễ dàng nhận thấy các phương pháp thông thường không dùng được. Ta tăng thêm các hạng tử của $A$ bằng cách thêm bớt cùng 1 hạng tử là $4x^2y^2$. Lúc này xuất hiện dạng khai triển của bình phương 1 tổng \& ta tiếp tục phân tích bằng cách áp dụng hằng đẳng thức. Như vậy mục đích của việc thêm bớt cùng 1 hạng tử là để xuất hiện những nhóm hạng tử sao cho có thể dùng hằng đẳng thức hoặc đặt nhân tử chung.'' -- \cite[p. 18]{Tuyen_Toan_8}

\subsection{Phương Pháp Đổi Biến}

\begin{baitoan}[\cite{Tuyen_Toan_8}, Ví dụ 7, p. 18]
	Phân tích đa thức thành nhân tử $A = (x^2 - 3x - 1)^2 - 12(x^2 - 3x - 1) + 27$.
\end{baitoan}

\begin{proof}[Giải]
	Đặt $y\coloneqq x^2 - 3x - 1$ ta được $A = y^2 - 12y + 27 = (y - 3)(y - 9) = (x^2 - 3x - 1 - 3)(x^2 - 3x - 1 - 9) = (x^2 - 3x - 4)(x^2 - 3x - 10) = (x + 1)(x - 4)(x + 2)(x - 5)$.
\end{proof}
Trong đó có thể phân tích $B\coloneqq y^2 - 12y + 27$ thành nhân tử bằng nhiều cách, e.g.: $\bullet$ $B = y^2 - 12y + 36 - 9 = (y - 6)^2 - 3^2 = (y - 6 - 3)(y - 6 + 3) = (y - 9)(y - 3)$. $\bullet$ $B = y^2 - 3y - 9y + 27 = y(y - 3) - 9(y - 3) = (y - 3)(y - 9)$. $\bullet$ $B = y^2 - 9y - 3y + 27 = y(y - 9) - 3(y - 9) = (y - 9)(y - 3)$. $\bullet$ $B = y^2 - 3^2 - 12y + 36 = (y - 3)(y + 3) - 12(y - 3) = (y - 3)(y + 3 - 12) = (y - 3)(y - 9)$. $\bullet$ $B = y^2 - 9^2 - 12y + 108 = (y - 9)(y + 9) - 12(y - 9) = (y - 9)(y + 9 - 12) = (y - 9)(y - 3)$.

\noindent\textit{Nhận xét.} ``Trong cách giải trên, nhờ cách đổi biến $y\coloneqq x^2 - 3x - 1$ ta đã đưa 1 đa thức bậc 4 đối với $x$ rất phức tạp trở thành 1 đa thức bậc 2 đối với $y$ rất đơn giản, nhờ đó phân tích thành nhân tử được dễ dàng.'' -- \cite[p. 19]{Tuyen_Toan_8}

\subsection{Phương Pháp Đồng Nhất Hệ Số\texttt{/}Phương Pháp Hệ Số Bất Định}

\begin{baitoan}[\cite{Tuyen_Toan_8}, Ví dụ 8, p. 18]
	Phân tích đa thức $A = x^4 - 3x^3 + 6x^2 - 5x + 3$ thành tích của 2 tam thức bậc 2 với hệ số nguyên.
\end{baitoan}

\begin{proof}[Giải]
	Sau khi phân tích thì $A$ có dạng $(x^2 + ax + 1)(x^2 + bx + 3)$ hoặc $(x^2 + ax - 1)(x^2 + bx - 3)$. Trường hợp 2 hạng tử bậc 2 của mỗi tam thức là $-x^2$ \& $-x^2$ thì ta chỉ cần đổi dấu cả 2 tam thức. Xét trường hợp  $A = (x^2 + ax + 1)(x^2 + bx + 3)$, i.e., $x^4 - 3x^3 + 6x^2 - 5x + 3 = x^4 + (a + b)x^3 + (ab + 4)x^2 + (3a + b)x + 3$. 2 đa thức trên đồng nhất với nhau nên ta có:
	\begin{equation*}
		\left\{\begin{split}
			a + b &= -3,\\
			ab + 4 &= 6,\\
			3a + b &= -5.
		\end{split}\right.
	\end{equation*}
	Suy ra\footnote{Thay vì giải hệ gồm 3 phương trình nhưng chỉ có 2 ẩn trên, ta chỉ cần giải\texttt{/}sử dụng máy tính bỏ túi để giải hệ gồm 2 phương trình đầu \& cuối:
	\begin{equation*}
		\left\{\begin{split}
			a + b &= -3,\\
			3a + b &= -5.
		\end{split}\right.
	\end{equation*}} $a = -1$, $b = -2$. Vậy $A = (x^2 - x + 1)(x^2 - 2x + 3)$.
\end{proof}
\noindent\textit{Nhận xét.} \textbf{1.} Nếu trường hợp 2 hệ số tự do là 1 \& 3 không thỏa mãn thì ta xét trường hợp 2 hệ số tự do là $-1$ \& $-3$ bằng cách tương tự như trên. \textbf{2.} Phương pháp giải như trên gọi là \textit{phương pháp đồng nhất hệ số} (hay \textit{phương pháp hệ số bất định}). Cơ sở phương pháp này là: 2 đa thức (viết dưới dạng thu gọn) là đồng nhất khi \& chỉ khi mọi hệ số của các đơn thức đồng dạng chứa trong 2 đa thức đó phải bằng nhau.'' -- \cite[p. 20]{Tuyen_Toan_8}

Phân tích các đa thức sau thành nhân tử.

\begin{baitoan}[\cite{Tuyen_Toan_8}, \textbf{61.}, p. 20]
	(a) $3x^2 - 11x + 6$; (b) $8x^2 + 10x - 3$; (c) $8x^2 - 2x - 1$.
\end{baitoan}

\begin{baitoan}[\cite{Tuyen_Toan_8}, \textbf{62.}, p. 20]
	(a) $6x^2 + 7xy + 2y^2$; (b) $9x^2 - 9xy - 4y^2$; (c) $x^2 - y^2 + 10x - 6y + 16$.
\end{baitoan}

\begin{baitoan}[\cite{Tuyen_Toan_8}, \textbf{63.}, p. 20]
	(a) $x^3 + x + 2$; (b) $x^3 - 2x - 1$; (c) $x^3 + 3x^2 - 4$.
\end{baitoan}

\begin{baitoan}[\cite{Tuyen_Toan_8}, \textbf{64.}, p. 20]
	(a) $x^3y^3 + x^2y^2 + 4$; (b) $x^3 + 3x^2y - 9xy^2 + 5y^3$.
\end{baitoan}

\begin{baitoan}[\cite{Tuyen_Toan_8}, \textbf{65.}, p. 20]
	(a) $x^4 + x^3 + 6x^2 + 5x + 5$; (b) $x^4 - 2x^3 - 12x^2 + 12x + 36$; (c) $x^8y^8 + x^4y^4 + 1$.
\end{baitoan}

\begin{baitoan}[\cite{Tuyen_Toan_8}, \textbf{66.}, p. 21]
	(a) $x^5 - x^4 + x^3 - x^2 + x - 1$; (b) $x^5 + x^4 - x^3 + x^2 - x + 2$.
\end{baitoan}

\begin{baitoan}[\cite{Tuyen_Toan_8}, \textbf{67.}, p. 21]
	(a) $x^4 + y^4 + (x + y)^4$; (b) $2(x^2 + x + 1)^2 - (2x + 1)^2 - (x^2 + 2x)^2$.
\end{baitoan}

\begin{baitoan}[\cite{Tuyen_Toan_8}, \textbf{68.}, p. 21]
	(a) $xy(x + y) + yz(y + z) + zx(z + x) + 3xyz$; (b) $xy(x + y) - yz(y + z) - zx(z - x)$; (c) $x(y^2 - z^2) + y(z^2 - x^2) + z(x^2 - y^2)$.
\end{baitoan}

\begin{baitoan}[\cite{Tuyen_Toan_8}, \textbf{69.}, p. 21]
	Chứng minh $a^5 - a\divby30$, $\forall a\in\mathbb{Z}$.
\end{baitoan}

\begin{baitoan}[\cite{Tuyen_Toan_8}, \textbf{70${}^\star$.}, p. 21]
	Cho $x,y,z\in\mathbb{R}$, $x > y > z$. Chứng minh biểu thức $A = x^4(y - z) + y^4(z - x) + z^4(x - y)$ luôn dương.
\end{baitoan}

\begin{baitoan}[\cite{Tuyen_Toan_8}, \textbf{71.}, p. 21]
	Cho $x,y,z$ là các số thực dương thỏa $(x + y)(y + z)(z + x) = 8xyz$. Chứng minh $x = y = z$.
\end{baitoan}

\begin{baitoan}[\cite{Tuyen_Toan_8}, \textbf{72.}, p. 21]
	(a) $x^4 + 5x^3 + 10x - 4$; (b) $x^3 + y^3 + z^3 - 3xyz$.
\end{baitoan}

\begin{baitoan}[\cite{Tuyen_Toan_8}, \textbf{73.}, p. 21]
	(a) $x^7 + x^2 + 1$; (b) $x^8 + x + 1$.
\end{baitoan}

\begin{baitoan}[\cite{Tuyen_Toan_8}, \textbf{74.}, p. 21]
	(a) $x^5 + x^4 + 1$; (b) $x^{10} + x^5 + 1$.
\end{baitoan}

\begin{baitoan}[\cite{Tuyen_Toan_8}, \textbf{75.}, p. 21]
	Cho $x\in\mathbb{Z}$. Chứng minh $x^{200} + x^{100} + 1\divby x^4 + x^2 + 1$.
\end{baitoan}

\begin{baitoan}[\cite{Tuyen_Toan_8}, \textbf{76.}, p. 21]
	(a) $A = x^2 - 2xy + y^2 + 3x - 3y - 4$; (b) $B = (12x^2 - 12xy + 3y^2) - 10(2x - y) + 8$.
\end{baitoan}

\begin{baitoan}[\cite{Tuyen_Toan_8}, \textbf{77.}, p. 21]
	(a) $A = (a - b)^3 + (b - c)^3 + (c - a)^3$; (b) $B = (a + b - 2c)^3 + (b + c - 2a)^3 + (c + a - 2b)^3$.
\end{baitoan}

\begin{baitoan}[\cite{Tuyen_Toan_8}, \textbf{78.}, p. 21]
	(a) Chứng minh: $(x + y + z)^3 - x^3 - y^3 - z^3 = 3(x + y)(y + z)(z + x)$. (b) Phân tích đa thức thành nhân tử: $A = (a + b + c)^3 + (a - b - c)^3 + (b - c - a)^3 + (c - a - b)^3$.
\end{baitoan}

\begin{baitoan}[\cite{Tuyen_Toan_8}, \textbf{79.}, p. 21]
	(a) $A = (x^2 - 2x)(x^2 - 2x - 1) - 6$; (b) $B = (x^2 + 4x - 3)^2 - 5x(x^2 + 4x - 3) + 6x^2$; (c) $C = (x^2 + x + 4)^2 + 8x(x^2 + x + 4) + 15x^2$.
\end{baitoan}

\begin{baitoan}[\cite{Tuyen_Toan_8}, \textbf{80.}, p. 21]
	$2(x^2 - 6x + 1)^2 + 5(x^2 -6x + 1)(x^2 + 1) + 2(x^2 + 1)^2$.
\end{baitoan}

\begin{baitoan}[\cite{Tuyen_Toan_8}, \textbf{81.}, p. 21]
	Cho $M = 4(x - 2)(x - 1)(x + 4)(x + 8) + 25x^2$. Chứng minh $M$ không có giá trị âm.
\end{baitoan}

\begin{baitoan}[\cite{Tuyen_Toan_8}, \textbf{82.}, p. 21]
	Phân tích đa thức $A$ thành tích của 1 nhị thức bậc nhất với 1 đa thức bậc 3 với hệ số nguyên sao cho hệ số cao nhất của đa thức bậc 3 là $1$: $A = 3x^4 + 11x^3 - 7x^2 - 2x + 1$.
\end{baitoan}

\begin{baitoan}[\cite{Tuyen_Toan_8}, \textbf{83.}, p. 21]
	Phân tích đa thức $B$ thành tích của 2 tam thức bậc 2 với hệ số nguyên: $B = x^4 - 6x^3 + 11x^2 - 6x + 1$.
\end{baitoan}

\begin{baitoan}[\cite{Tuyen_Toan_8}, \textbf{84.}, p. 21]
	Phân tích đa thức $C$ thành tích của 2 tam thức bậc 2 với hệ số nguyên \& các hệ số cao nhất đều mang dấu dương: $C = x^4 - x^3 + 2x^2 - 11x - 5$.
\end{baitoan}

%------------------------------------------------------------------------------%

\section{Chia Đa Thức}
``\textbf{1.} Chia đơn thức $A$ cho đơn thức $B$: $\bullet$ Chia hệ số của $A$ cho hệ số của $B$; $\bullet$ Chia lũy thừa của từng biến trong $A$ cho lũy thừa của cùng biến đó trong $B$; Nhân các kết quả với nhau. \textbf{2.} Chia đa thức $A$ cho đơn thức $B$: Ta chia mỗi hạng tử của $A$ cho $B$ rồi cộng các kết quả với nhau. \textbf{3.} Chia đa thức $A$ cho đa thức $B$: Cho $A,B$ là 2 đa thức tùy ý của cùng 1 biến, $B\ne 0$, khi đó tồn tại duy nhất 1 cặp đa thức $Q,R$ sao cho $A = BQ + R$, trong đó $R = 0$ hoặc bậc của $R$ nhỏ hơn bậc của $B$. $Q$ gọi là \textit{đa thức thương} \& $R$ là \textit{đa thức dư} của phép chia $A$ cho $B$. Nếu $R = 0$ thì phép chia $A$ cho $B$ là \textit{phép chia hết}.

\begin{dinhly}[Định lý B\'ezout]
	Số dư trong phép chia đa thức $f(x)$ cho nhị thức bậc nhất $x - a$ đúng bằng $f(a)$.
\end{dinhly}

\begin{vidu}
	Với $f(x) = x^3 - 6x + 5$. Số dư trong phép chia $f(x)$ cho $x - 2$ là $f(2) = 8 - 12 + 5 = 1$. Số dư trong phép chia $f(x)$ cho $x - 1$ là $f(1) = 1 - 6 + 5 = 9$, i.e., $f(x)\divby x - 1$.
\end{vidu}

\begin{hequa}
	Nếu $a$ là nghiệm của đa thức $f(x)$ thì $f(x)\divby x - a$. Đặc biệt, nếu tổng các hệ số của đa thức $f(x)$ bằng $0$ thì $1$ là nghiệm \& $f(x)\divby x - 1$; nếu $f(x)$ có tổng các hệ số bậc chẵn bằng tổng các hệ số bậc lẻ thì $-1$ là nghiệm \& $f(x)\divby x -(-1)$, i.e., $f(x)\divby x + 1$.
\end{hequa}
\textbf{4.} \textit{Áp dụng hệ quả của định lý B\'ezout vào việc phân tích đa thức thành nhân tử.} Nếu đa thức $f(x)$ có nghiệm $x = a$ thì khi phân tích $f(x)$ thành nhân tử, tích sẽ chứa nhân tử $(x - a)$. \textbf{5.} \textit{Cách nhẩm nghiệm nguyên, nghiệm hữu tỷ của đa thức $f(x)$ với hệ số nguyên.} Nếu $f(x)$ có nghiệm nguyên thì nghiệm đó phải là ước của hệ số tự do. Nếu $f(x)$ có nghiệm hữu tỷ thì nghiệm đó có dạng $\frac{p}{q}$, $\mbox{ƯCLN}(p,q) = 1$ trong đó $p$ là ước của hệ số tự do, $q$ là ước dương của hệ số cao nhất.'' -- \cite[\S4, pp. 22--24]{Tuyen_Toan_8}

\begin{baitoan}[\cite{Tuyen_Toan_8}, Ví dụ 9, p. 24]
	Xác định các hệ số $a,b$ sao cho $x^4 + ax^3 + b\divby x^2 - 1$.
\end{baitoan}

\begin{baitoan}[\cite{Tuyen_Toan_8}, Ví dụ 10, p. 25]
	Phân tích đa thức thành nhân tử $M = xy(x + y) + yz(y + z) + zx(z + x) + 2xyz$.
\end{baitoan}

\begin{baitoan}[\cite{Tuyen_Toan_8}, Ví dụ 11, p. 26]
	Phân tích đa thức thành nhân tử $A = x^3 - x^2 - 8x + 12$.
\end{baitoan}
``Phương pháp nhẩm nghiệm của đa thức để vận dụng hệ quả của định lý B\'ezout giúp ta định hướng nhanh chóng việc tách 1 hạng tử thành nhiều hạng tử 1 cách thích hợp.'' -- \cite[p. 26]{Tuyen_Toan_8}

\begin{baitoan}[\cite{Tuyen_Toan_8}, \textbf{85.}, p. 27]
	Tìm $n\in\mathbb{N}$ để đơn thức $-7x^{n+1}y^6\divby4x^5y^n$.
\end{baitoan}

\begin{baitoan}[\cite{Tuyen_Toan_8}, \textbf{86.}, p. 27]
	Chứng minh giá trị của biểu thức $A$ luôn luôn không âm với mọi giá trị khác $0$ của $x,y$: $A = (75x^5y^2 - 45x^4y^3):3x^3y^2 - \left(\frac{5}{2}x^2y^4 - 2xy^5\right):\frac{1}{2}xy^3$.
\end{baitoan}

\begin{baitoan}[\cite{Tuyen_Toan_8}, \textbf{87.}, p. 27]
	Tìm $x,y\in\mathbb{R}$ thỏa: $[(x - 2y)(x - 7y) - x^2 + 4y^2]:(x - 2y) = 18$.
\end{baitoan}

\begin{baitoan}[\cite{Tuyen_Toan_8}, \textbf{88.}, p. 27]
	Tìm giá trị nhỏ nhất của thương: $(4x^5 + 2x^4 + 4x^3 - x - 1):(2x^3 + x - 1)$.
\end{baitoan}

\begin{baitoan}[\cite{Tuyen_Toan_8}, \textbf{89.}, p. 27]
	Tìm các giá trị nguyên của $x$ để thương có giá trị nguyên. (a) $(3x^3 + 13x^2 - 7x + 5):(3x - 2)$; (b) $(2x^5 + 4x^4 - 7x^3 - 44):(2x^2 - 7)$.
\end{baitoan}

\begin{baitoan}[\cite{Tuyen_Toan_8}, \textbf{90.}, p. 27]
	Chứng minh không tồn tại $n\in\mathbb{N}$ để giá trị của biểu thức $n^6 - n^4 - 2n^2 + 9$ chia hết cho giá trị của biểu thức $n^4 + n^2$.
\end{baitoan}

\begin{baitoan}[\cite{Tuyen_Toan_8}, \textbf{91.}, p. 27]
	Không làm phép chia đa thức, tìm số dư trong phép chia đa thức $f(x)$ cho đa thức $g(x)$ trong các trường hợp sau: (a) $f(x) = x^{21} + x^{20} + x^{19}  + 101$, $g(x) = x + 1$. (b) $f(x) = 3x^3 + 4x^2 - 2x + 7$, $g(x) = x + 2$. (c) $f(x) = x^4 - 5x^3 + 2x - 10$, $g(x) = x - 5$.
\end{baitoan}

\begin{baitoan}[\cite{Tuyen_Toan_8}, \textbf{92.}, p. 27]
	Chứng minh $f(x) = (x^2 - 3x + 1)^{31} - (x^2 - 4x + 5)^{30} + 2\divby x - 2$.
\end{baitoan}

\begin{baitoan}[\cite{Tuyen_Toan_8}, \textbf{93.}, p. 27]
	Phân tích các đa thức sau thành nhân tử: (a) $x^3 - 9x^2 + 15x + 25$; (b) $x^3 - 4x^2 - 11x + 30$; (c) $2x^4 + x^3 - 22x^2 + 15x - 36$.
\end{baitoan}

\begin{baitoan}[\cite{Tuyen_Toan_8}, \textbf{94.}, p. 27]
	Phân tích các đa thức sau thành nhân tử: (a) $3x^3 + 5x^2 - 14x + 4$; (b) $2x^3 - x^2 - 3x - 1$.
\end{baitoan}

\begin{baitoan}[\cite{Tuyen_Toan_8}, \textbf{95.}, p. 28]
	Tìm đa thức dư trong phép chia $(x^{54} + x^{45} + x^{36} + \cdots + x^9 + 1):(x^2 - 1)$.
\end{baitoan}

\begin{baitoan}[\cite{Tuyen_Toan_8}, \textbf{96.}, p. 28]
	Xác định đa thức $f(x)$ thỏa mãn cả 3 điều kiện sau: (a) Khi chia cho $x - 1$ dư $4$; (b) Khi chia cho $x + 2$ dư $1$; (c) Khi chia cho $(x - 1)(x + 2)$ thì được thương là $5x^2$ \& còn dư.
\end{baitoan}

\begin{baitoan}[\cite{Tuyen_Toan_8}, \textbf{97.}, p. 28]
	Cho đa thức $A = ax^2 + bx + c$. Xác định hệ số $b$ biết khi chia $A$ cho $x - 1$, chia $A$ cho $x + 1$ đều có cùng 1 số dư.
\end{baitoan}

\begin{baitoan}[\cite{Tuyen_Toan_8}, \textbf{98.}, p. 28]
	Chứng minh nếu $x^4 - 4x^3 + 5ax^2 - 4bx + c\divby x^3 + 3x^2 - 9x - 3$ thì $a + b + c = 0$.
\end{baitoan}

\begin{baitoan}[\cite{Tuyen_Toan_8}, \textbf{99.}, p. 28]
	Đa thức $4x^3 + ax + b$ chia hết cho 2 đa thức $x - 2,x + 1$. Tính $2a - 3b$.
\end{baitoan}

%------------------------------------------------------------------------------%

\section{Số Chính Phương}
``\textbf{1.} \textit{Số chính phương} là số bằng bình phương của 1 số tự nhiên. 20 số chính phương đầu tiên: $0,1,4,9,16,25,36,49,4,81,100$, $121,144,169,196,225,256,289,324,361,400$. \textbf{2.} 1 số tính chất của số chính phương: (a) Số chính phương không tận cùng bởi các chữ số $2,3,7,8$. (b) Khi phân tích 1 số chính phương ra thừa số nguyên tố ta được các thừa số là lũy thừa của số nguyên tố với số mũ chẵn, e.g., $3600 = 60^2 = 2^4\cdot3^2\cdot5^2$. Từ đó suy ra số chính phương $N$ chia hết cho 2 thì $N$ chia hết cho $2^2 = 4$; số chính phương $N$ chia hết cho $2^3 = 8$ thì $N$ chia hết cho $2^4 = 16$. Tổng quát, nếu số chính phương $N$ chia hết cho $p^{2k+1}$ thì $N$ chia hết cho $p^{2k+2}$ ($p$ là số nguyên tố, $k\in\mathbb{Z}$). (c) Số chính phương chia cho 3 chỉ có thể dư 0 hoặc dư 1. Thực vậy, xét các trường hợp: $(3k)^2 = 9k^2\divby3$, $(3k + 1)^2 = 9k^2 + 6k + 1$ chia cho 3 dư 1, $(3k + 2)^2 = 9k^2 + 12k + 4$ chia cho 3 dư 1. Tương tự, 1 số chính phương chia cho 4 chỉ có thể dư 0 hoặc dư 1, chia cho 5 dư 0 hoặc dư 1 hoặc dư 4. Số chính phương lẻ chia cho 4 hoặc chia cho 8 đều dư 1. (d) Giữa 2 số chính phương liên tiếp không có số chính phương nào. $n^2 < x^2 < (n + 1)^2\Rightarrow\overline{\exists}x\in\mathbb{Z}$ thỏa mãn $n^2 < x^2 < (n + 1)^2$. $n^2 < x^2 < (n + 2)^2\Rightarrow x^2 = (n + 1)^2$. (e) Nếu 2 số nguyên liên tiếp có tích là 1 số chính phương thì 1 trong 2 số nguyên đó là số 0. \textbf{3.} \textit{Nhận biết 1 số chính phương}: (a) Để chứng minh $N$ là 1 số chính phương ta có thể: $\bullet$ Biến đổi $N$ thành bình phương của 1 số tự nhiên (hoặc số nguyên). $\bullet$ Vận dụng tính chất: nếu $a,b\in\mathbb{N}$ nguyên tố cùng nhau có tích là 1 số chính phương thì mỗi số $a,b$ cũng là 1 số chính phương. (b) Để chứng minh $N$ không phải là số chính phương ta có thể: $\bullet$ Chứng minh $N$ có chữ số tận cùng là $2,3,7,8$ hoặc có \textit{1 số lẻ} chữ số 0 tận cùng. $\bullet$ Chứng minh $N$ chứa số nguyên tố với số mũ lẻ. $\bullet$ Xét số dư khi chia $N$ cho 3 hoặc cho 4 hoặc cho 5, hoặc cho 8, $\ldots$ E.g., nếu $N$ chia cho 3 có số dư là 2, hoặc $N$ chia cho 4, cho 5 có số dư là 2, 3 thì $N$ không phải là số chính phương. $\bullet$ Chứng minh $N$ nằm giữa 2 số chính phương liên tiếp.'' -- \cite[pp. 28--29]{Tuyen_Toan_8}

\begin{baitoan}[\cite{Tuyen_Toan_8}, Ví dụ 12, p. 29]
	Cho $A = \underbrace{11\ldots1}_{2n} - \underbrace{88\ldots8}_n + 1$. Chứng minh $A$ là 1 số chính phương.
\end{baitoan}

\begin{baitoan}[\cite{Tuyen_Toan_8}, Ví dụ 13, p. 30]
	Chứng minh: (a) Tổng của 3 số chính phương liên tiếp không phải là 1 số chính phương. (b) Tổng $S = \sum_{i=1}^{30} i^2 = 1^2 + 2^2 + 3^3 + \cdots + 30^2$ không phải là 1 số chính phương.
\end{baitoan}

\begin{baitoan}[\cite{Tuyen_Toan_8}, \textbf{100.}, p. 31]
	Có 2 số chính phương nào: (a) Có tổng bằng $4567$? (b) Có hiệu bằng $7654$?
\end{baitoan}

\begin{baitoan}[\cite{Tuyen_Toan_8}, \textbf{101.}, p. 31]
	Chứng minh tổng của $20$ số chính phương liên tiếp không thể là số chính phương.
\end{baitoan}

\begin{baitoan}[\cite{Tuyen_Toan_8}, \textbf{102.}, p. 31]
	Cho 5 số chính phương bất kỳ có chữ số hàng đơn vị đều bằng $6$ còn chữ số hàng chục thì khác nhau. Chứng minh tổng các chữ số hàng chục của 5 số chính phương đó cũng là 1 số chính phương.
\end{baitoan}

\begin{baitoan}[\cite{Tuyen_Toan_8}, \textbf{103.}, p. 31]
	Cho $a,b,c$ là các chữ số khác $0$. (a) Tính tổng $S$ của tất cả các số có 3 chữ số tạo thành bởi cả 3 chữ số $a,b,c$; (b) Chứng minh $S$ không phải là số chính phương.
\end{baitoan}

\begin{baitoan}[\cite{Tuyen_Toan_8}, \textbf{104.}, p. 31]
	Tìm 1 số chính phương có 4 chữ số biết 2 chữ số đầu giống nhau, 2 chữ số cuối giống nhau.
\end{baitoan}

\begin{baitoan}[\cite{Tuyen_Toan_8}, \textbf{105.}, p. 31]
	Chứng minh nếu $n + 1$ \& $2n + 1$ đều là số chính phương thì $n\divby24$.
\end{baitoan}

\begin{baitoan}[\cite{Tuyen_Toan_8}, \textbf{106.}, pp. 31--32]
	Tìm $n\in\mathbb{N}$ biết trong 3 mệnh đề sau có 2 mệnh đề đúng \& 1 mệnh đề sai: (a) $n$ có chữ số tận cùng là $2$. (b) $n + 20$ là 1 số chính phương. (c) $n - 69$ là 1 số chính phương.
\end{baitoan}

\begin{baitoan}[\cite{Tuyen_Toan_8}, \textbf{107.}, p. 32]
	Cho $N$ là tổng của 2 số chính phương. Chứng minh: (a) $2N$ cũng là tổng của 2 số chính phương; (b) $N^2$ cũng là tổng của 2 số chính phương.
\end{baitoan}

\begin{baitoan}[\cite{Tuyen_Toan_8}, \textbf{108.}, p. 32]
	Cho $A,B,C,D$ là 4 số chính phương. Chứng minh $(A + B)(C + D)$ là tổng của 2 số chính phương.
\end{baitoan}

\begin{baitoan}[\cite{Tuyen_Toan_8}, \textbf{109.}, p. 32]
	Cho $x,y,z\in\mathbb{Z}$ sao cho $x = y + z$. Chứng minh $2(xy + xz - yz)$ là tổng của 3 số chính phương.
\end{baitoan}

\begin{baitoan}[\cite{Tuyen_Toan_8}, \textbf{110.}, p. 32]
	Cho $a,b,c,d\in\mathbb{Z}$ thỏa mãn điều kiện $a - b = c + d$. Chứng minh $a^2 + b^2 + c^2 + d^2$ luôn là tổng của 3 số chính phương.
\end{baitoan}

\begin{baitoan}[\cite{Tuyen_Toan_8}, \textbf{111.}, p. 32]
	Cho 2 số chính phương liên tiếp. Chứng minh tổng của 2 số đó cộng với tích của chúng là 1 số chính phương lẻ.
\end{baitoan}

\begin{baitoan}[\cite{Tuyen_Toan_8}, \textbf{112.}, p. 32]
	Cho $a_n = \sum_{i=1}^n i = 1 + 2 + \cdots + n$. (a) Tính $a_{n+1}$; (b) Chứng minh $a_n + a_{n+1}$ là 1 số chính phương.
\end{baitoan}

\begin{baitoan}[\cite{Tuyen_Toan_8}, \textbf{113.}, p. 32]
	Cho $M$ là tích của 4 số nguyên liên tiếp. Chứng minh $M + 1$ là 1 số chính phương.
\end{baitoan}

\begin{baitoan}[\cite{Tuyen_Toan_8}, \textbf{114.}, p. 32]
	(a) Cho $a = \underbrace{11\ldots1}_n5$, $b = \underbrace{11\ldots1}_n9$. Chứng minh $ab + 4$ là 1 số chính phương. (b) Cho $a = \underbrace{11\ldots1}_n$, $b = 1\underbrace{00\ldots0}_{n-2}11$, $n\ge 2$. Chứng minh $ab + 4$ là 1 số chính phương.
\end{baitoan}

\begin{baitoan}[\cite{Tuyen_Toan_8}, \textbf{115.}, p. 32]
	Cho $M = \underbrace{11\ldots1}_n\underbrace{55\ldots5}_n + 1$. Chứng minh $M$ là 1 số chính phương.
\end{baitoan}

\begin{baitoan}[\cite{Tuyen_Toan_8}, \textbf{116.}, p. 33]
	Chứng minh các số sau là số chính phương. (a) $M_n = \underbrace{11\ldots1}_{2n} + \underbrace{44\ldots4}_n + 1$, $\forall n\in\mathbb{N}$; (b) $N = \underbrace{11\ldots1}_{2n} + \underbrace{11\ldots1}_{n+1} + \underbrace{66\ldots6}_n + 8$, $\forall n\in\mathbb{N}$.
\end{baitoan}

\begin{baitoan}[\cite{Tuyen_Toan_8}, \textbf{117.}, p. 33]
	Cho $a,b,c\in\mathbb{Z}$ thỏa mãn điều kiện $ab + bc + ca = 1$. Chứng minh $(a^2 + 1)(b^2 + 1)(c^2 + 1)$ là 1 số chính phương.
\end{baitoan}

\begin{baitoan}[\cite{Tuyen_Toan_8}, \textbf{118.}, p. 33]
	Tìm tất cả $n\in\mathbb{N}$ sao cho $n^2 + 1234$ là 1 số chính phương.
\end{baitoan}

\begin{baitoan}[\cite{Tuyen_Toan_8}, \textbf{119.}, p. 33]
	Tìm tất cả $k\in\mathbb{N}$ để cho số $2^k + 2^4 + 2^7$ là 1 số chính phương.
\end{baitoan}

\begin{baitoan}[\cite{Tuyen_Toan_8}, \textbf{120.}, p. 33]
	Tìm tất cả $x\in\mathbb{N}$ sao cho $x^2 + 2x + 200$ là 1 số chính phương.
\end{baitoan}

\begin{baitoan}[\cite{Tuyen_Toan_8}, \textbf{121.}, p. 33]
	Tìm tất cả $x\in\mathbb{Z}$ sao cho $A = x(x - 1)(x - 7)(x - 8)$ là 1 số chính phương.
\end{baitoan}

\begin{baitoan}[\cite{Tuyen_Toan_8}, \textbf{122.}, p. 33]
	Cho $A = p^4$ trong đó $p$ là 1 số nguyên tố. (a) Số $A$ có những ước dương nào? (b) Tìm các giá trị của $p$ để tổng các ước dương của $A$ là 1 số chính phương.
\end{baitoan}

%------------------------------------------------------------------------------%

\section{Miscellaneous}

%------------------------------------------------------------------------------%

\printbibliography[heading=bibintoc]
	
\end{document}