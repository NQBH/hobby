\documentclass{article}
\usepackage[backend=biber,natbib=true,style=authoryear]{biblatex}
\addbibresource{/home/nqbh/reference/bib.bib}
\usepackage[utf8]{vietnam}
\usepackage{tocloft}
\renewcommand{\cftsecleader}{\cftdotfill{\cftdotsep}}
\usepackage[colorlinks=true,linkcolor=blue,urlcolor=red,citecolor=magenta]{hyperref}
\usepackage{amsmath,amssymb,amsthm,mathtools,float,graphicx,algpseudocode,algorithm,tcolorbox,tikz,tkz-tab,subcaption}
\DeclareMathOperator{\arccot}{arccot}
\usepackage[inline]{enumitem}
\allowdisplaybreaks
\numberwithin{equation}{section}
\newtheorem{assumption}{Assumption}[section]
\newtheorem{baitoan}{Bài toán}
\newtheorem{cauhoi}{Câu hỏi}[section]
\newtheorem{conjecture}{Conjecture}[section]
\newtheorem{corollary}{Corollary}[section]
\newtheorem{definition}{Definition}[section]
\newtheorem{dinhly}{Định lý}[section]
\newtheorem{dinhnghia}{Định nghĩa}[section]
\newtheorem{example}{Example}[section]
\newtheorem{hequa}{Hệ quả}[section]
\newtheorem{lemma}{Lemma}[section]
\newtheorem{luuy}{Lưu ý}[section]
\newtheorem{notation}{Notation}[section]
\newtheorem{principle}{Principle}[section]
\newtheorem{problem}{Problem}[section]
\newtheorem{proposition}{Proposition}[section]
\newtheorem{question}{Question}[section]
\newtheorem{remark}{Remark}[section]
\newtheorem{theorem}{Theorem}[section]
\newtheorem{vidu}{Ví dụ}[section]
\usepackage[left=0.5in,right=0.5in,top=1.5cm,bottom=1.5cm]{geometry}
\usepackage{fancyhdr}
\pagestyle{fancy}
\fancyhf{}
\lhead{\small Sect.~\thesection}
\rhead{\small\nouppercase{\leftmark}}
\renewcommand{\subsectionmark}[1]{\markboth{#1}{}}
\cfoot{\thepage}
\def\labelitemii{$\circ$}
\DeclareRobustCommand{\divby}{%
	\mathrel{\vbox{\baselineskip.65ex\lineskiplimit0pt\hbox{.}\hbox{.}\hbox{.}}}%
}

\title{Polynomial -- Đa Thức}
\author{Nguyễn Quản Bá Hồng\footnote{Independent Researcher, Ben Tre City, Vietnam\\e-mail: \texttt{nguyenquanbahong@gmail.com}; website: \url{https://nqbh.github.io}.}}
\date{\today}

\begin{document}
\maketitle
\begin{abstract}
	\textsc{[en]} This text is a collection of problems, from easy to advanced, about polynomial. This text is also a supplementary material for my lecture note on Elementary Mathematics grade 8, which is stored \& downloadable at the following link: \href{https://github.com/NQBH/hobby/blob/master/elementary_mathematics/grade_8/NQBH_elementary_mathematics_grade_8.pdf}{GitHub\texttt{/}NQBH\texttt{/}hobby\texttt{/}elementary mathematics\texttt{/}grade 8\texttt{/}lecture}\footnote{\textsc{url}: \url{https://github.com/NQBH/hobby/blob/master/elementary_mathematics/grade_8/NQBH_elementary_mathematics_grade_8.pdf}.}. The latest version of this text has been stored \& downloadable at the following link: \href{https://github.com/NQBH/hobby/blob/master/elementary_mathematics/grade_8/polynomial/NQBH_polynomial.pdf}{GitHub\texttt{/}NQBH\texttt{/}hobby\texttt{/}elementary mathematics\texttt{/}grade 8\texttt{/}polynomial}\footnote{\textsc{url}: \url{https://github.com/NQBH/hobby/blob/master/elementary_mathematics/grade_8/polynomial/NQBH_polynomial.pdf}.}.
	\vspace{2mm}
	
	\textsc{[vi]} Tài liệu này là 1 bộ sưu tập các bài tập chọn lọc từ cơ bản đến nâng cao về đa thức. Tài liệu này là phần bài tập bổ sung cho tài liệu chính -- bài giảng \href{https://github.com/NQBH/hobby/blob/master/elementary_mathematics/grade_8/NQBH_elementary_mathematics_grade_8.pdf}{GitHub\texttt{/}NQBH\texttt{/}hobby\texttt{/}elementary mathematics\texttt{/}grade 8\texttt{/}lecture} của tác giả viết cho Toán Sơ Cấp lớp 8. Phiên bản mới nhất của tài liệu này được lưu trữ \& có thể tải xuống ở link sau: \href{https://github.com/NQBH/hobby/blob/master/elementary_mathematics/grade_8/polynomial/NQBH_polynomial.pdf}{GitHub\texttt{/}NQBH\texttt{/}hobby\texttt{/}elementary mathematics\texttt{/}grade 8\texttt{/}polynomial}.
\end{abstract}
\setcounter{secnumdepth}{4}
\setcounter{tocdepth}{3}
\tableofcontents

%------------------------------------------------------------------------------%

\section{Nhân Đa Thức}
``\begin{enumerate*}
	\item[\textbf{1.}] Muốn nhân 1 đơn thức với 1 đa thức, ta nhân đơn thức với từng hạng tử của đa thức rồi cộng các  tích với nhau.
	\item[\textbf{2.}] Muốn nhân 1 đa thức với 1 đa thức, ta nhân mỗi hạng tử của đa thức này với từng hạng tử của đa thức kia rồi cộng các tích với nhau.
	\item[\textbf{3.}] Quy tắc nhân 1 đơn thức với 1 đa thức còn được vận dụng theo chiều ngược lại: $AB + AC = A(B + C)$.
	\item[\textbf{4.}] Nếu 2 đa thức $P(x),Q(x)$ luôn có giá trị bằng nhau với mọi giá trị của biến thì 2 đa thức đó gọi là \textit{2 đa thức đồng nhất}, ký hiệu $P(x)\equiv Q(x)$. 2 đa thức $P(x),Q(x)$ (viết dưới dạng thu gọn) là \textit{đồng nhất} khi \& chỉ khi hệ số của các lũy thừa cùng bậc bằng nhau. Đặc biệt, nếu $P(x) = \sum_{i=0}^n a_ix^i = a_nx^n + a_{n-1}x^{n-1} + \cdots + a_1x + a_0$ luôn bằng $0$ với mọi $x\in\mathbb{R}$ thì $a_0 = a_1 = \cdots = a_n = 0$, i.e., $a_i = 0$, $\forall i = 0,1,\ldots,n$.'' -- \cite[Chap. 1, \S1, p. 4]{Tuyen_Toan_8}
\end{enumerate*}

\begin{baitoan}[\cite{Tuyen_Toan_8}, Ví dụ 1, p. 4]
	Cho $P = (x + 5)(ax^2 + bx + 25)$ \& $Q = x^3 + 125$.
	\begin{enumerate*}
		\item[(a)] Viết $P$ dưới dạng 1 đa thức thu gọn theo lũy thừa giảm dần của $x$.
		\item[(b)] Với giá trị nào của $a,b$ thì $P = Q$, $\forall x\in\mathbb{R}$.
	\end{enumerate*}	
\end{baitoan}

\begin{proof}[Giải]
	\begin{enumerate*}
		\item[(a)] $P = (x + 5)(ax^2 + bx + 25) = ax^3 + bx^2 + 25x + 5ax^2 + 5bx + 125 = ax^3 + (5a + b)x^2 + (5b + 25)x + 125$.
		\item[(b)] $P = Q$, $\forall x\in\mathbb{R}\Leftrightarrow ax^3 + (5a + b)x^2 + (5b + 25)x + 125 = x^3 + 125$, $\forall x\in\mathbb{R}\Leftrightarrow(a = 1)\land(5a + b = 0)\land(5b + 25 = 0)\Leftrightarrow(a = 1)\land(b = -5)$.
	\end{enumerate*}
\end{proof}
\noindent\textit{Nhận xét.} ``Phương pháp giải (b) dựa vào tính chất: 2 đa thức $P,Q$ (viết dưới dạng thu gọn) là đồng nhất khi \& chỉ khi mọi hệ số của các đơn thức đồng dạng chứa trong 2 đa thức đó phải bằng nhau.'' -- \cite[p. 5]{Tuyen_Toan_8}


\begin{baitoan}[\cite{Tuyen_Toan_8}, \textbf{1.}, p. 5]
	Tính giá trị của các biểu thức sau bằng cách hợp lý:\\
	\begin{enumerate*}
		\item[(a)] $A = x^5 - 100x^4 + 100x^3 - 100x^2 + 100x - 9$ tại $x = 99$.
		\item[(b)] $B = x^6 - 20x^5 - 20x^4 - 20x^3 - 20x^2 - 20x + 3$ tại $x  = 21$.
		\item[(c)] $C = x^7 - 26x^6 + 27x^5 - 47x^4 - 77x^3 + 50x^2 + x - 24$ tại $x = 25$.
	\end{enumerate*}
\end{baitoan}

\begin{baitoan}[\cite{Tuyen_Toan_8}, \textbf{2.}, p. 5]
	Cho $x,y\in\mathbb{Z}$. Chứng minh:
	\begin{enumerate*}
		\item[(a)] Nếu $A = 5x + y\divby19$ thì $B = 4x - 3y\divby19$.
		\item[(b)] Nếu $C = 4x + 3y\divby13$ thì $D = 7x + 2y\divby13$.
	\end{enumerate*}
\end{baitoan}

\begin{baitoan}[\cite{Tuyen_Toan_8}, \textbf{3.}, p. 5]
	Cho 4 số lẻ liên tiếp. Chứng minh hiệu của tích 2 số cuối với tích 2 số đầu chia hết cho $16$.
\end{baitoan}

\begin{baitoan}[\cite{Tuyen_Toan_8}, \textbf{4.}, pp. 5--6]
	Cho 4 số nguyên liên tiếp.
	\begin{enumerate*}
		\item[(a)] Hỏi tích của số đầu với số cuối nhỏ hơn tích của 2 số ở giữa bao nhiêu đơn vị?
		\item[(b)] Giả sử tích của số đầu với số thứ 3 nhỏ hơn tích của số thứ 2 \& số thứ 4 là $99$, tìm 4 số nguyên đó.
	\end{enumerate*}
\end{baitoan}

\begin{baitoan}[\cite{Tuyen_Toan_8}, \textbf{5.}, p. 6]
	Cho $b + c = 10$. Chứng minh đẳng thức $(10a + b)(10a + c) = 100a(a + 1) + bc$. Áp dụng để tính nhẩm: $62\cdot68$, $43\cdot47$.
\end{baitoan}

\begin{baitoan}[\cite{Tuyen_Toan_8}, \textbf{6.}, p. 6]
	Xác định các hệ số $a,b,c$ biết:
	\begin{enumerate*}
		\item[(a)] $(2x - 5)(3x + b) = ax^2 + x + c$.\\
		\item[(b)] $(ax + b)(x^2 - x - 1) = ax^3 + cx^2 - 1$.
	\end{enumerate*}
\end{baitoan}

\begin{baitoan}[\cite{Tuyen_Toan_8}, \textbf{7.}, p. 6]
	Cho $m\in\mathbb{N}^\star$, $m < 30$. Có bao nhiêu giá trị của $m$ để đa thức $x^2 + mx + 72$ là tích của 2 đa thức bậc nhất với hệ số nguyên?
\end{baitoan}

%------------------------------------------------------------------------------%

\section{Các Hằng Đẳng Thức Đáng Nhớ}
``\textbf{1.} $(A + B)^2 = A^2 + 2AB + B^2$. \textbf{2.} $(A - B)^2 = A^2 - 2AB + B^2$. \textbf{3.} $(A - B)(A + B) = A^2 - B^2$. \textbf{4.} $(A + B)^3 = A^3 + 3A^2B + 3AB^2 + B^3 = A^3 + B^3 + 3AB(A + B)$. \textbf{5.} $(A - B)^3 = A^3 - 3A^2B + 3AB^2 - B^3 = A^3 - B^3 - 3AB(A + B)$. \textbf{6.} $(A + B)(A^2 - AB + B^2) = A^3 + B^3$. \textbf{7.} $(A - B)(A^2 + AB + B^2) = A^3 - B^3$. \textbf{8.} Bình phương của đa thức: $(a + b + c)^2 = a^2 + b^2 + c^2 + 2ab + 2bc + 2ca$, $(a + b + c + d)^2 = a^2 + b^2 + c^2 + d^2 + 2ab + 2ac + 2ad + 2bc + 2bd + 2cd$, $\ldots$. \textbf{9.} Lũy thừa bậc $n$ của 1 nhị thức (nhị thức Newton):
\begin{align*}
	(a + b)^0 &= 1,\\
	(a + b)^1 &= 1a + 1b,\\
	(a + b)^2 &= 1a^2 + 2ab + 1b^2,\\
	(a + b)^3 &= 1a^3 + 3a^2b + 3ab^2 + 1b^3,\\
	(a + b)^4 &= 1a^4 + 4a^3b + 6a^2b^2 + 4ab^3 + 1b^4,\\
	(a + b)^5 &= 1a^5 + 5a^4b + 10a^3b^2 + + 10a^2b^3 + 5ab^4 + 1b^5.
\end{align*}
Ta thấy khi khai triển $(a + b)^n$ ta được 1 đa thức có $n + 1$ hạng tử, hạng tử dầu là $a^n$, hạng tử cuối là $b^n$, các hạng tử còn lại đều chứa các nhân tử $a$ \& $b$. Vì vậy $(a + b)^n = B(a) + b^n = B(b) + a^n$. \textbf{10.} Nếu viết riêng các hệ số ở vế phải ta được bảng sau (gọi là \textit{tam giác Pascal}):
\begin{align*}
	&1\\
	&1\ \ 1\\
	&1\ \ 2\ \ 1\\
	&1\ \ 3\ \ 3\ \ 1\\
	&1\ \ 4\ \ 6\ \ 4\ \ 1\\
	&1\ \ 5\ \ 10\ \ 10\ \ 5\ \ 1\\
	&\ldots
\end{align*}
Nhận xét: Mỗi dòng đều bắt đầu bằng 1 \& kết thúc bằng 1. Mỗi số trên 1 dòng kể từ dòng thứ 2 đều bằng số liền trên cộng với số bên trái của số liền trên.'' -- \cite[\S2, pp. 6--7]{Tuyen_Toan_8}

\begin{baitoan}[\cite{Tuyen_Toan_8}, Ví dụ 2, p. 7]
	Cho $x + y = 9$, $xy = 14$. Tính giá trị của các biểu thức sau:
	\begin{enumerate*}
		\item[(a)] $x - y$;
		\item[(b)] $x^2 + y^2$;
		\item[(c)] $x^3 + y^3$.
	\end{enumerate*}
\end{baitoan}

\begin{proof}[Giải]
	\begin{enumerate*}
		\item[(a)] $(x - y)^2 = (x + y)^2 - 4xy = 9^2 - 4\cdot14 = 25\Rightarrow x - y = \pm5$.
		\item[(b)] $x^2 + y^2 = (x + y)^2 - 2xy = 9^2 - 2\cdot14 = 53$.
		\item[(c)] $x^3 + y^3 = (x + y)^3 - 3xy(x + y) = 9^3 - 3\cdot14\cdot9 = 351$.
	\end{enumerate*}
\end{proof}

\begin{luuy}
	``2 số có bình phương bằng nhau thì chúng đối nhau hoặc bằng nhau. Ngược lại, 2 số đối nhau hoặc bằng nhau thì có bình phương bằng nhau. $(a - b)^2 = (b - a)^2 = a^2 - 2ab + b^2$, $\forall a,b\in\mathbb{R}$.'' -- \cite[p. 8]{Tuyen_Toan_8}
\end{luuy}

\begin{baitoan}[Mở rộng \cite{Tuyen_Toan_8}, p. 7]
	Cho $x + y = a$, $xy = b$, với $a,b\in\mathbb{R}$, $a^2\ge 4b$. Tính giá trị của các biểu thức sau theo $a,b$:
	\begin{enumerate*}
		\item[(a)] $x - y$;
		\item[(b)] $x^2 + y^2$;
		\item[(c)] $x^2 - y^2$;
		\item[(d)] $x^3 + y^3$;
		\item[(e)] $x^3 - y^3$.
	\end{enumerate*}
\end{baitoan}

\begin{proof}[Giải]
	(a) $(x - y)^2 = (x + y)^2 - 4xy = a^2 - 4\cdot b\Rightarrow x - y = \pm\sqrt{a^2 - 4b}$. (b) $x^2 + y^2 = (x + y)^2 - 2xy = a^2 - 2\cdot b$. (c) $x^2 - y^2 = (x - y)(x + y) = \pm a\sqrt{a^2 - 4b}$. (d) $x^3 + y^3 = (x + y)^3 - 3xy(x + y) = a^3 - 3ab$. (e) $x^3 - y^3 = (x - y)(x^2 + xy + y^2) = \pm\sqrt{a^2 - 4b}(a^2 - 2b + b) = \pm\sqrt{a^2 - 4b}(a^2 - b)$.
\end{proof}

\begin{baitoan}[\cite{Tuyen_Toan_8}, Ví dụ 3, p. 8]
	Tìm giá trị nhỏ nhất của biểu thức: $A = (x + 3y - 5)^2 - 6xy + 26$.
\end{baitoan}

\begin{proof}[Giải]
	$A = x^2 + 9y^2 + 25 + 6xy - 10x - 30y - 6xy + 26 = (x^2 - 10x + 25) + (9y^2 - 30y + 25) + 1 = (x - 5)^2 + (3y - 5)^2 + 1\ge 1$, $\forall x,y\in\mathbb{R}\Rightarrow\min A = 1\Leftrightarrow(x = 5)\land\left(y = \frac{5}{3}\right)$.
\end{proof}

\begin{luuy}
	\begin{enumerate*}
		\item[(a)] ``Các hằng đẳng thức được vận dụng theo 2 chiều ngược nhau, e.g., $(a - b)^2 = a^2 - 2ab + b^2$ hoặc ngược lại $a^2 - 2ab + b^2 = (a - b)^2$.
		\item[(b)] Bình phương của mọi số thực đều không âm: $x^2\ge 0$, $\forall x\in\mathbb{R}$, ``$=$'' xảy ra $\Leftrightarrow x = 0$; hay tương đương với $(a - b)^2\ge 0$, $\forall a,b\in\mathbb{R}$, ``$=$'' xảy ra $\Leftrightarrow a = b$.'' -- \cite[p. 9]{Tuyen_Toan_8}
	\end{enumerate*}
\end{luuy}

\begin{baitoan}[\cite{Tuyen_Toan_8}, \textbf{8.}, p. 9]
	Chứng minh các đẳng thức:
	\begin{enumerate*}
		\item[(a)] $(2 + 1)(2^2 + 1)(2^4 + 1)(2^8 + 1)(2^{16} + 1) = 2^{32} - 1$;
		\item[(b)] $100^2 + 103^2 + 105^2 + 94^2 = 101^2 + 98^2 + 96^2 + 107^2$.
	\end{enumerate*}
\end{baitoan}

\begin{baitoan}[\cite{Tuyen_Toan_8}, \textbf{9.}, p. 9]
	Tính giá trị của biểu thức bằng cách hợp lý:
	\begin{enumerate*}
		\item[(a)] $A = \frac{258^2 - 242^2}{254^2 - 246^2}$;
		\item[(b)] $B = 263^2 + 74\cdot263 + 37^2$;
		\item[(c)] $C = 136^2 - 92\cdot136 + 46^2$;
		\item[(d)] $D = (50^2 + 48^2 + 46^2 + \cdots + 2^2) - (49^2 + 47^2 + 45^2 + \cdots + 1^2)$.
	\end{enumerate*}
\end{baitoan}

\begin{baitoan}[\cite{Tuyen_Toan_8}, \textbf{10.}, p. 9]
	Cho biết $2(a^2 + b^2) = (a - b)^2$. Chứng minh $a$ \& $b$ đối nhau.
\end{baitoan}

\begin{baitoan}[\cite{Tuyen_Toan_8}, \textbf{11.}, p. 9]
	Cho $a,b,x,y\in\mathbb{R}\backslash\{0\}$. Biết $(a^2 + b^2)(x^2 + y^2) = (ax + by)^2$. Tìm hệ thức giữa 4 số $a,b,x,y$.
\end{baitoan}

\begin{baitoan}[\cite{Tuyen_Toan_8}, \textbf{12.}, p. 9]
	Cho $a^2 + b^2 + c^2 = ab + bc + ca$. Chứng minh $a = b = c$.
\end{baitoan}

\begin{baitoan}[\cite{Tuyen_Toan_8}, \textbf{13.}, p. 9]
	Chứng minh không có các số $x,y\in\mathbb{R}$ nào thỏa mãn mỗi đẳng thức sau: (a) $3x^2 + y^2 + 10x - 2xy + 26 = 0$; (b) $4x^2 + 3y^2 - 4x + 30y + 78 = 0$; (c) $3x^2 + 6y^2 - 12x - 20y + 40 = 0$.
\end{baitoan}

\begin{baitoan}[\cite{Tuyen_Toan_8}, \textbf{14.}, p. 10]
	Tìm $x\in\mathbb{R},n\in\mathbb{N}$ thỏa $x^2 + 2x + 4^n - 2^{n + 1} + 2 = 0$.
\end{baitoan}

\begin{baitoan}[\cite{Tuyen_Toan_8}, \textbf{15.}, p. 10]
	Chứng minh: (a) Biểu thức $A = x^2 + x + 1$ luôn luôn dương với mọi số thực $x$; (b) Biểu thức $B = x^2 - xy + y^2$ luôn luôn dương với mọi số thực $x,y$ không đồng thời bằng $0$; (c) Biểu thức $C = 4x - 10 - x^2$ luôn luôn âm với mọi số thực $x$.
\end{baitoan}

\begin{baitoan}[\cite{Tuyen_Toan_8}, \textbf{16.}, p. 10]
	Tìm giá trị nhỏ nhất của biểu thức: (a) $A = 25x^2 + 3y^2 - 10x + 11$; (b) $B = (x - 3)^2 + (x - 11)^2$; (c) $C = (x + 1)(x - 2)(x - 3)(x - 6)$.
\end{baitoan}

\begin{baitoan}[\cite{Tuyen_Toan_8}, \textbf{17.}, p. 10]
	Tìm giá trị lớn nhất của biểu thức: (a) $A = 2x - x^2$; (b) $B = 19 - 6x - 9x^2$.
\end{baitoan}

\begin{baitoan}[\cite{Tuyen_Toan_8}, \textbf{18.}, p. 10]
	Chứng minh: (a) 2 số chẵn hơn kém nhau $4$ đơn vị thì hiệu các bình phương của chúng chia hết cho $16$; (b) 2 số lẻ hơn kém nhau $6$ đơn vị thì hiệu các bình phương của chúng chia hết cho $24$.
\end{baitoan}

\begin{baitoan}[\cite{Tuyen_Toan_8}, \textbf{19.}, p. 10]
	Cho $x > y > 0$ \& $x - y = 7$, $xy = 60$. Không tính $x,y$, tính: (a) $x^2 - y^2$; (b) $x^4 + y^4$.
\end{baitoan}

\begin{baitoan}[\cite{Tuyen_Toan_8}, \textbf{20.}, p. 10]
	Cho $a + b + c = 2p$. Chứng minh: (a) $a^2 - b^2 - c^2 + 2abc = 4(p - b)(p - c)$; (b) $p^2 + (p - a)^2 + (p - b)^2 + (p - c)^2 = a^2 + b^2 + c^2$.
\end{baitoan}

\begin{baitoan}[\cite{Tuyen_Toan_8}, \textbf{21.}, p. 10]
	Cho $a = m^2 + n^2$, $b = m^2 - n^2$, $c = 2mn$. Chứng minh nếu $m > n > 0$ thì $a,b,c$ là độ dài 3 cạnh của 1 tam giác vuông.
\end{baitoan}

\begin{baitoan}[\cite{Tuyen_Toan_8}, \textbf{22.}, p. 11]
	Tính giá trị của biểu thức: $A = x^3 + 9x^2 + 27x + 27$ với $x = -103$; (b) $B = x^3 - 15x^2 + 75x$ với $x = 25$; (c) $C = (x + 1)(x - 1)(x^2 + x + 1)(x^2 - x + 1)$ với $x = -3$.
\end{baitoan}

\begin{baitoan}[\cite{Tuyen_Toan_8}, \textbf{23.}, p. 11]
	Cho $x - y = 2$, tính giá trị của biểu thức: $A = 2(x^3 - y^3) - 3(x + y)^2$.
\end{baitoan}

\begin{baitoan}[\cite{Tuyen_Toan_8}, \textbf{24.}, p. 11]
	Cho $x + y + z = 0$. Chứng minh $x^3 + y^3 + z^3 = 3xyz$.
\end{baitoan}

\begin{baitoan}[\cite{Tuyen_Toan_8}, \textbf{25.}, p. 11]
	Rút gọn biểu thức $A = (x - y - 1)^3 - (x - y + 1)^3 + 6(x - y)^2$.
\end{baitoan}

\begin{baitoan}[\cite{Tuyen_Toan_8}, \textbf{26.}, p. 11]
	Giải hệ phương trình:
	\begin{equation*}
		\left\{\begin{split}
			(x + 2y)(x^2 - 2xy + 4y^2) &= 0,\\
			(x - 2y)(x^2 + 2xy + 4y^2) &= 16.
		\end{split}\right.
	\end{equation*}
\end{baitoan}

\begin{baitoan}[\cite{Tuyen_Toan_8}, \textbf{27.}, p. 11]
	Chứng minh: (a) $742^3 - 692^3\divby200$; (b) $685^3 + 315^3\divby25000$.
\end{baitoan}

\begin{baitoan}[\cite{Tuyen_Toan_8}, \textbf{28${}^\star$.}, p. 11]
	Cho $a + b + c + d = 0$. Chứng minh $a^3 + b^3 + c^3 + d^3 = 3(b + c)(ad - bc)$.
\end{baitoan}

\begin{baitoan}[\cite{Tuyen_Toan_8}, \textbf{29.}, p. 11]
	Cho $a + b + c = 0$. Chứng minh: (a) $(ab + bc + ca)^2 = a^2b^2 + b^2c^2 + c^2a^2$; (b) $a^4 + b^4 + c^4 = 2(ab + bc + ca)^2$.
\end{baitoan}

\begin{baitoan}[\cite{Tuyen_Toan_8}, \textbf{30.}, p. 11]
	Xác định các hệ số $a,b$ để đa thức $A = x^4 - 2x^3 + 3x^2 + ax + b$ là bình phương của 1 đa thức.
\end{baitoan}

\begin{baitoan}[\cite{Tuyen_Toan_8}, \textbf{31.}, p. 11]
	Cho $a + b + c = 0$, $a^2 + b^2 + c^2 = 1$. Chứng minh $a^4 + b^4 + c^4 = \frac{1}{2}$.
\end{baitoan}

\begin{baitoan}[\cite{Tuyen_Toan_8}, \textbf{32.}, pp. 11--12]
	Cho $a,b,c$ là 3 số thực không đồng thời bằng $0$. Chứng minh có ít nhất 1 trong các biểu thức sau có giá trị dương: $x = (a - b + c)^2 + 8ab$, $y = (a - b +  c)^2 + 8bc$, $z = (a - b + c)^2 - 8ca$.
\end{baitoan}

\begin{baitoan}[\cite{Tuyen_Toan_8}, \textbf{33.}, p. 12]
	Tính tổng các hệ số của tất cả các hạng tử trong khai triển của nhị thức: (a) $(5x - 3)^6$; (b) $(3x - 4y)^{20}$.
\end{baitoan}

\begin{baitoan}[\cite{Tuyen_Toan_8}, \textbf{34.}, p. 12]
	Đa thức $(x + y)^5$ được khai triển theo lũy thừa giảm của $x$. Biết hạng tử thứ 2 \& hạng tử thứ 3 có giá trị bằng nhau khi cho $x = a$, $y = b$ trong đó $a,b$ là các số thực dương \& $a - b = 1$. Tìm $a,b$.
\end{baitoan}

\begin{baitoan}[\cite{Tuyen_Toan_8}, \textbf{35.}, p. 12]
	Tính: (a) $(x + 2)^2$; (b) $(x - 1)^6$; (c) $(x - 1)^5$.
\end{baitoan}

\begin{baitoan}[\cite{Tuyen_Toan_8}, \textbf{36.}, p. 12]
	Tìm số dư của phép chia $38^{10}$ cho $13$ \& $38^9$ cho $13$.
\end{baitoan}

\begin{baitoan}[\cite{Tuyen_Toan_8}, \textbf{37.}, p. 12]
	Chứng minh 2 chữ số tận cùng của $7^{43}$ là $43$.
\end{baitoan}

%------------------------------------------------------------------------------%

\section{Phân Tích Đa Thức Thành Nhân Tử}
``\textbf{1.} Phân tích đa thức thành nhân tử là biến đổi đa thức đó thành 1 tích của những đa thức. \textbf{2.} Các phương pháp thông thường: $\bullet$ \textit{Phương pháp đặt nhân tử chung}: $AB + AC - AD = A(B + C - D)$. $\bullet$ \textit{Phương pháp dùng hằng đẳng thức}: $A^2\pm2AB + B^2 = (A\pm B)^2$, $A^3\pm3A^2B + 3AB^2\pm B^3 = (A\pm B)^3$, $A^2 - B^2 = (A - B)(A + B)$, $A^3 - B^3 = (A - B)(A^2 + AB + B^2)$, $A^3 + B^3 = (A + B)(A^2 - AB + B^2)$. $\bullet$ \textit{Phương pháp nhóm các hạng tử}: $AC - AD + BC - BD = A(C - D) + B(C - D) = (A + B)(C - D)$. \textbf{3.} Dạng tổng quát của các hằng đẳng thức hiệu 2 bình phương, hiệu 2 lập phương: $A^n - B^n = (A - B)\sum_{i=0}^{n-1} A^{n - 1 - i}B^i = (A - B)(A^{n-1} + A^{n-2}B + \cdots + AB^{n-2} + B^{n-1})$. \textbf{4.} Dạng tổng quát của hằng đẳng thức tổng 2 lập phương: $A^n + B^n = (A + B)\sum_{i=0}^{n-1} (-1)^iA^{n - 1 - i}B^i = (A - B)(A^{n-1} - A^{n-2}B + A^{n-3}B^2 - \cdots + AB^{n-2} + B^{n-1})$ với $n$ lẻ. \textbf{5.} Áp dụng vào tính chất chia hết: $A^n - B^n\divby A - B$, $\forall n\in\mathbb{N}$, $A\ne B$; $A^n + B^n\divby A + B$, $\forall n\in\mathbb{N}$, $n$ lẻ, $A\ne-B$; $A^{2k} - B^{2k}\divby A^2 - B^2$, $\forall k\in\mathbb{N}$, $A\ne B$.'' -- \cite[\S3, pp. 12--13]{Tuyen_Toan_8}

\begin{baitoan}[\cite{Tuyen_Toan_8}, Ví dụ 4, p. 13]
	Cho $x,y\in\mathbb{R}$, $x\ne y$, thỏa mãn điều kiện $9x(x - y) - 10(y - x)^2 = 0$. Chứng minh $x = 10y$.
\end{baitoan}

\begin{proof}[Giải]
	$0 = 9x(x - y) - 10(y - x)^2 = 9x(x - y) - 10(x - y)^2 = (x - y)[9x - 10(x - y)] = (x - y)(-x + 10y)\Rightarrow(x = y)\lor(x = 10y)$, mà $x\ne y$, nên $x = 10y$.
\end{proof}
``Phân tích đa thức thành nhân tử có nhiều ứng dụng như để tính giá trị của biểu thức, chứng minh tính chia hết hoặc như trong ví dụ trên, để tìm mối quan hệ giữa các biến, $\ldots$'' -- \cite[p. 14]{Tuyen_Toan_8}

\begin{baitoan}[\cite{Tuyen_Toan_8}, \textbf{38.}, p. 14]
	Phân tích các đa thức sau thành nhân tử: (a) $5x(x - 2y) + 2(2y - x)^2$; (b) $7x(y - 4)^2 - (4 - y)^3$; (c) $(4x - 8)(x^2 + 6) - (4x - 8)(x + 7) + 9(8 - 4x)$.
\end{baitoan}

\begin{baitoan}[\cite{Tuyen_Toan_8}, \textbf{39.}, p. 14]
	Chứng minh: (a) $43^2 + 43\cdot17\divby60$; (b) $27^5 - 3^{11}\divby80$.
\end{baitoan}

\begin{baitoan}[\cite{Tuyen_Toan_8}, \textbf{40.}, p. 14]
	Tìm 1 số biết 3 lần bình phương của nó đúng bằng 2 lần lập phương của số ấy.
\end{baitoan}

\begin{baitoan}[\cite{Tuyen_Toan_8}, \textbf{41.}, p. 14]
	Có các số nguyên $x,y,z$ nào thỏa mãn hệ phương trình sau không?
	\begin{equation*}
		\left\{\begin{split}
			x^3 + xyz &= 957,\\
			y^3 + xyz &= 795,\\
			z^3 + xyz &= 579.
		\end{split}\right.
	\end{equation*}
\end{baitoan}

\begin{baitoan}[\cite{Tuyen_Toan_8}, \textbf{42.}, p. 14]
	Chứng minh số $\underbrace{11\ldots1}_n\underbrace{22\ldots2}_n$ là tích của 2 số nguyên liên tiếp.
\end{baitoan}

\begin{baitoan}[\cite{Tuyen_Toan_8}, \textbf{43.}, p. 15]
	Phân tích các đa thức sau thành nhân tử: (a) $100x^2 - (x^2 + 25)^2$; (b) $(x - y + 5)^2 - 2(x - y + 5) + 1$.
\end{baitoan}

\begin{baitoan}[\cite{Tuyen_Toan_8}, \textbf{44.}, p. 15]
	Phân tích đa thức thành nhân tử: $(x^2 + 4y^2 - 5)^2 - 16(x^2y^2 + 2xy + 1)$.
\end{baitoan}

\begin{baitoan}[\cite{Tuyen_Toan_8}, \textbf{45.}, p. 15]
	Cho $A = 4a^2b^2 - (a^2 + b^2 + c^2)$ trong đó $a,b,c$ là độ dài 3 cạnh của 1 tam giác. Chứng minh $A > 0$.
\end{baitoan}

\begin{baitoan}[\cite{Tuyen_Toan_8}, \textbf{46.}, p. 15]
	Chứng minh: (a) $21^{10} - 1\divby200$; (b) $39^{20} + 39^{13}\divby40$; (c) $2^{60} + 5^{30}\divby41$; (d) $2005^{2007} + 2007^{2005}\divby2006$.
\end{baitoan}

\begin{baitoan}[\cite{Tuyen_Toan_8}, \textbf{47.}, p. 15]
	Cho $n$ là 1 số tự nhiên lẻ. Chứng minh $24^n + 1$ chia hết cho $25$ nhưng không chia hết cho $23$.
\end{baitoan}

\begin{baitoan}[\cite{Tuyen_Toan_8}, \textbf{48.}, p. 15]
	Cho $a$ là 1 số nguyên lẻ, $a > 1$. Chứng minh $(a - 1)^{\frac{1}{2}(a - 1)} - 1\divby a - 2$.
\end{baitoan}

\begin{baitoan}[\cite{Tuyen_Toan_8}, \textbf{49.}, p. 15]
	Phân tích các đa thức sau thành nhân tử: (a) $x^2 - xz - 9y^2 + 3yz$; (b) $x^3 - x^2 - 5x + 125$; (c) $x^3 + 2x^2 - 6x - 27$; (d) $12x^3 + 4x^2 - 27x - 9$.
\end{baitoan}

\begin{baitoan}[\cite{Tuyen_Toan_8}, \textbf{50.}, p. 15]
	Phân tích các đa thức sau thành nhân tử: (a) $x^4 - 25x^2 + 20x - 4$; (b) $x^2(x^2 - 6) - x^2 + 9$; (c) $ab(x^2 + y^2) - xy(a^2 + b^2)$.
\end{baitoan}

\begin{baitoan}[\cite{Tuyen_Toan_8}, \textbf{51.}, p. 15]
	Tìm $x,y\in\mathbb{R}$ sao cho $x - y = xy - 1$.
\end{baitoan}

\begin{baitoan}[\cite{Tuyen_Toan_8}, \textbf{52.}, p. 15]
	Cho $x,y\in\mathbb{R}$, $x\ne y$ sao cho $x^2 - y = y^2 - x$. Tính giá trị của biểu thức $A = x^2 + 2xy + y^2 - 3x - 3y$.
\end{baitoan}

\begin{baitoan}[\cite{Tuyen_Toan_8}, \textbf{53.}, p. 16]
	Cho $\frac{a - b}{b - c} = \frac{c - d}{d - a}$. Chứng minh hoặc $a = c$ hoặc $a + c = b + d$.
\end{baitoan}

\begin{baitoan}[\cite{Tuyen_Toan_8}, \textbf{54.}, p. 16]
	Phân tích các đa thức sau thành nhân tử: (a) $4x^4 + 4x^3 - x^2 - x$; (b) $x^6 - x^4 - 9x^3 + 9x^2$; (c) $x^4 - 4x^3 + 8x^2 - 16x + 16$.
\end{baitoan}

\begin{baitoan}[\cite{Tuyen_Toan_8}, \textbf{55.}, p. 16]
	Phân tích các đa thức sau thành nhân tử: (a) $(xy + 4)^2 - 4(x + y)^2$; (b) $(ab - xy)^2 - (bx - ay)^2$; (c) $(x^2 + 8x - 34)^2 - (3x^2 - 8x - 2)^2$.
\end{baitoan}

\begin{baitoan}[\cite{Tuyen_Toan_8}, \textbf{56.}, p. 16]
	Phân tích các đa thức sau thành nhân tử: (a) $(a + b + c)^2 + (a - b + c)^2 - 4b^2$; (b) $a(b^2 - c^2) - b(c^2 - a^2) + c(a^2 - b^2)$; (c) $a^5 + b^5 - (a + b)^5$.
\end{baitoan}

\begin{baitoan}[\cite{Tuyen_Toan_8}, \textbf{57.}, p. 16]
	Chứng minh: (a) $999^4 + 999$ có tận cùng bằng $3$ chữ số $0$; (b) $49^5 - 49\divby100$.
\end{baitoan}

\begin{baitoan}[\cite{Tuyen_Toan_8}, \textbf{58.}, p. 16]
	Chứng minh: (a) Lập phương của 1 số nguyên trừ đi số nguyên đó thì chia hết cho $6$; (b) Nếu tổng của 3 số nguyên chia hết cho $6$ thì tổng các lập phương của chúng chia hết cho $6$.
\end{baitoan}

\begin{baitoan}[\cite{Tuyen_Toan_8}, \textbf{59.}, p. 16]
	Cho $a\ne\pm b$ \&  $a(a + b)(a + c) = b(b + c)(b + a)$. Chứng minh $a + b + c = 0$.
\end{baitoan}

\begin{baitoan}[\cite{Tuyen_Toan_8}, \textbf{60.}, p. 16]
	Cho $x^2y - y^2x + x^2z - z^2x + y^2z + z^2y = 2xyz$. Chứng minh trong 3 số $x,y,z$ ít nhất cũng có 2 số bằng nhau hoặc đối nhau.
\end{baitoan}

%------------------------------------------------------------------------------%

\printbibliography[heading=bibintoc]
	
\end{document}