\documentclass{article}
\usepackage[backend=biber,natbib=true,style=authoryear]{biblatex}
\addbibresource{/home/nqbh/reference/bib.bib}
\usepackage[utf8]{vietnam}
\usepackage{tocloft}
\renewcommand{\cftsecleader}{\cftdotfill{\cftdotsep}}
\usepackage[colorlinks=true,linkcolor=blue,urlcolor=red,citecolor=magenta]{hyperref}
\usepackage{amsmath,amssymb,amsthm,mathtools,float,graphicx,algpseudocode,algorithm,tcolorbox,tikz,tkz-tab,subcaption}
\DeclareMathOperator{\arccot}{arccot}
\usepackage[inline]{enumitem}
\allowdisplaybreaks
\numberwithin{equation}{section}
\newtheorem{assumption}{Assumption}[section]
\newtheorem{nhanxet}{Nhận xét}[section]
\newtheorem{conjecture}{Conjecture}[section]
\newtheorem{corollary}{Corollary}[section]
\newtheorem{hequa}{Hệ quả}[section]
\newtheorem{definition}{Definition}[section]
\newtheorem{dinhnghia}{Định nghĩa}[section]
\newtheorem{example}{Example}[section]
\newtheorem{vidu}{Ví dụ}[section]
\newtheorem{lemma}{Lemma}[section]
\newtheorem{notation}{Notation}[section]
\newtheorem{principle}{Principle}[section]
\newtheorem{problem}{Problem}[section]
\newtheorem{baitoan}{Bài toán}[section]
\newtheorem{proposition}{Proposition}[section]
\newtheorem{menhde}{Mệnh đề}[section]
\newtheorem{question}{Question}[section]
\newtheorem{cauhoi}{Câu hỏi}[section]
\newtheorem{quytac}{Quy tắc}
\newtheorem{remark}{Remark}[section]
\newtheorem{luuy}{Lưu ý}[section]
\newtheorem{theorem}{Theorem}[section]
\newtheorem{tiende}{Tiên đề}[section]
\newtheorem{dinhly}{Định lý}[section]
\usepackage[left=0.5in,right=0.5in,top=1.5cm,bottom=1.5cm]{geometry}
\usepackage{fancyhdr}
\pagestyle{fancy}
\fancyhf{}
\lhead{\small Subsect.~\thesubsection}
\rhead{\small\nouppercase{\leftmark}}
\renewcommand{\subsectionmark}[1]{\markboth{#1}{}}
\cfoot{\thepage}
\def\labelitemii{$\circ$}
\DeclareRobustCommand{\divby}{%
	\mathrel{\vbox{\baselineskip.65ex\lineskiplimit0pt\hbox{.}\hbox{.}\hbox{.}}}%
}

\title{Polynomial -- Đa Thức}
\author{Nguyễn Quản Bá Hồng\footnote{Independent Researcher, Ben Tre City, Vietnam\\e-mail: \texttt{nguyenquanbahong@gmail.com}; website: \url{https://nqbh.github.io}.}}
\date{\today}

\begin{document}
\maketitle
\begin{abstract}
	\textsc{[en]} This text is a collection of problems, from easy to advanced, about polynomial. This text is also a supplementary material for my lecture note on Elementary Mathematics grade 8, which is stored \& downloadable at the following link: \href{https://github.com/NQBH/hobby/blob/master/elementary_mathematics/grade_8/NQBH_elementary_mathematics_grade_8.pdf}{GitHub\texttt{/}NQBH\texttt{/}hobby\texttt{/}elementary mathematics\texttt{/}grade 8\texttt{/}lecture}\footnote{\textsc{url}: \url{https://github.com/NQBH/hobby/blob/master/elementary_mathematics/grade_8/NQBH_elementary_mathematics_grade_8.pdf}.}. The latest version of this text has been stored \& downloadable at the following link: \href{https://github.com/NQBH/hobby/blob/master/elementary_mathematics/grade_8/polynomial/NQBH_polynomial.pdf}{GitHub\texttt{/}NQBH\texttt{/}hobby\texttt{/}elementary mathematics\texttt{/}grade 8\texttt{/}polynomial}\footnote{\textsc{url}: \url{https://github.com/NQBH/hobby/blob/master/elementary_mathematics/grade_8/polynomial/NQBH_polynomial.pdf}.}.
	\vspace{2mm}
	
	\textsc{[vi]} Tài liệu này là 1 bộ sưu tập các bài tập chọn lọc từ cơ bản đến nâng cao về đa thức. Tài liệu này là phần bài tập bổ sung cho tài liệu chính -- bài giảng \href{https://github.com/NQBH/hobby/blob/master/elementary_mathematics/grade_8/NQBH_elementary_mathematics_grade_8.pdf}{GitHub\texttt{/}NQBH\texttt{/}hobby\texttt{/}elementary mathematics\texttt{/}grade 8\texttt{/}lecture} của tác giả viết cho Toán Sơ Cấp lớp 8. Phiên bản mới nhất của tài liệu này được lưu trữ \& có thể tải xuống ở link sau: \href{https://github.com/NQBH/hobby/blob/master/elementary_mathematics/grade_8/polynomial/NQBH_polynomial.pdf}{GitHub\texttt{/}NQBH\texttt{/}hobby\texttt{/}elementary mathematics\texttt{/}grade 8\texttt{/}polynomial}.
\end{abstract}
\setcounter{secnumdepth}{4}
\setcounter{tocdepth}{3}
\tableofcontents

%------------------------------------------------------------------------------%

\section{Nhân Đa Thức}
``\begin{enumerate*}
	\item[\textbf{1.}] Muốn nhân 1 đơn thức với 1 đa thức, ta nhân đơn thức với từng hạng tử của đa thức rồi cộng các  tích với nhau.
	\item[\textbf{2.}] Muốn nhân 1 đa thức với 1 đa thức, ta nhân mỗi hạng tử của đa thức này với từng hạng tử của đa thức kia rồi cộng các tích với nhau.
	\item[\textbf{3.}] Quy tắc nhân 1 đơn thức với 1 đa thức còn được vận dụng theo chiều ngược lại: $AB + AC = A(B + C)$.
	\item[\textbf{4.}] Nếu 2 đa thức $P(x),Q(x)$ luôn có giá trị bằng nhau với mọi giá trị của biến thì 2 đa thức đó gọi là \textit{2 đa thức đồng nhất}, ký hiệu $P(x)\equiv Q(x)$. 2 đa thức $P(x),Q(x)$ (viết dưới dạng thu gọn) là \textit{đồng nhất} khi \& chỉ khi hệ số của các lũy thừa cùng bậc bằng nhau. Đặc biệt, nếu $P(x) = \sum_{i=0}^n a_ix^i = a_nx^n + a_{n-1}x^{n-1} + \cdots + a_1x + a_0$ luôn bằng $0$ với mọi $x\in\mathbb{R}$ thì $a_0 = a_1 = \cdots = a_n = 0$, i.e., $a_i = 0$, $\forall i = 0,1,\ldots,n$.'' -- \cite[Chap. 1, \S1, p. 4]{Tuyen_Toan_8}
\end{enumerate*}

\begin{baitoan}[\cite{Tuyen_Toan_8}, Ví dụ 1, p. 4]
	Cho $P = (x + 5)(ax^2 + bx + 25)$ \& $Q = x^3 + 125$.
	\begin{enumerate*}
		\item[(a)] Viết $P$ dưới dạng 1 đa thức thu gọn theo lũy thừa giảm dần của $x$.
		\item[(b)] Với giá trị nào của $a,b$ thì $P = Q$, $\forall x\in\mathbb{R}$.
	\end{enumerate*}	
\end{baitoan}

\begin{proof}[Giải]
	\begin{enumerate*}
		\item[(a)] $P = (x + 5)(ax^2 + bx + 25) = ax^3 + bx^2 + 25x + 5ax^2 + 5bx + 125 = ax^3 + (5a + b)x^2 + (5b + 25)x + 125$.
		\item[(b)] $P = Q$, $\forall x\in\mathbb{R}\Leftrightarrow ax^3 + (5a + b)x^2 + (5b + 25)x + 125 = x^3 + 125$, $\forall x\in\mathbb{R}\Leftrightarrow(a = 1)\land(5a + b = 0)\land(5b + 25 = 0)\Leftrightarrow(a = 1)\land(b = -5)$.
	\end{enumerate*}
\end{proof}
\noindent\textit{Nhận xét.} ``Phương pháp giải (b) dựa vào tính chất: 2 đa thức $P,Q$ (viết dưới dạng thu gọn) là đồng nhất khi \& chỉ khi mọi hệ số của các đơn thức đồng dạng chứa trong 2 đa thức đó phải bằng nhau.'' -- \cite[p. 5]{Tuyen_Toan_8}


\begin{baitoan}[\cite{Tuyen_Toan_8}, \textbf{1.}, p. 5]
	Tính giá trị của các biểu thức sau bằng cách hợp lý:\\
	\begin{enumerate*}
		\item[(a)] $A = x^5 - 100x^4 + 100x^3 - 100x^2 + 100x - 9$ tại $x = 99$.
		\item[(b)] $B = x^6 - 20x^5 - 20x^4 - 20x^3 - 20x^2 - 20x + 3$ tại $x  = 21$.
		\item[(c)] $C = x^7 - 26x^6 + 27x^5 - 47x^4 - 77x^3 + 50x^2 + x - 24$ tại $x = 25$.
	\end{enumerate*}
\end{baitoan}

\begin{baitoan}[\cite{Tuyen_Toan_8}, \textbf{2.}, p. 5]
	Cho $x,y\in\mathbb{Z}$. Chứng minh:
	\begin{enumerate*}
		\item[(a)] Nếu $A = 5x + y\divby19$ thì $B = 4x - 3y\divby19$.
		\item[(b)] Nếu $C = 4x + 3y\divby13$ thì $D = 7x + 2y\divby13$.
	\end{enumerate*}
\end{baitoan}

\begin{baitoan}[\cite{Tuyen_Toan_8}, \textbf{3.}, p. 5]
	Cho 4 số lẻ liên tiếp. Chứng minh hiệu của tích 2 số cuối với tích 2 số đầu chia hết cho $16$.
\end{baitoan}

\begin{baitoan}[\cite{Tuyen_Toan_8}, \textbf{4.}, pp. 5--6]
	Cho 4 số nguyên liên tiếp.
	\begin{enumerate*}
		\item[(a)] Hỏi tích của số đầu với số cuối nhỏ hơn tích của 2 số ở giữa bao nhiêu đơn vị?
		\item[(b)] Giả sử tích của số đầu với số thứ 3 nhỏ hơn tích của số thứ 2 \& số thứ 4 là $99$, tìm 4 số nguyên đó.
	\end{enumerate*}
\end{baitoan}

\begin{baitoan}[\cite{Tuyen_Toan_8}, \textbf{5.}, p. 6]
	Cho $b + c = 10$. Chứng minh đẳng thức $(10a + b)(10a + c) = 100a(a + 1) + bc$. Áp dụng để tính nhẩm: $62\cdot68$, $43\cdot47$.
\end{baitoan}

\begin{baitoan}[\cite{Tuyen_Toan_8}, \textbf{6.}, p. 6]
	Xác định các hệ số $a,b,c$ biết:
	\begin{enumerate*}
		\item[(a)] $(2x - 5)(3x + b) = ax^2 + x + c$.\\
		\item[(b)] $(ax + b)(x^2 - x - 1) = ax^3 + cx^2 - 1$.
	\end{enumerate*}
\end{baitoan}

\begin{baitoan}[\cite{Tuyen_Toan_8}, \textbf{7.}, p. 6]
	Cho $m\in\mathbb{N}^\star$, $m < 30$. Có bao nhiêu giá trị của $m$ để đa thức $x^2 + mx + 72$ là tích của 2 đa thức bậc nhất với hệ số nguyên?
\end{baitoan}

%------------------------------------------------------------------------------%

\section{Các Hằng Đẳng Thức Đáng Nhớ}
``\textbf{1.} $(A + B)^2 = A^2 + 2AB + B^2$. \textbf{2.} $(A - B)^2 = A^2 - 2AB + B^2$. \textbf{3.} $(A - B)(A + B) = A^2 - B^2$. \textbf{4.} $(A + B)^3 = A^3 + 3A^2B + 3AB^2 + B^3 = A^3 + B^3 + 3AB(A + B)$. \textbf{5.} $(A - B)^3 = A^3 - 3A^2B + 3AB^2 - B^3 = A^3 - B^3 - 3AB(A + B)$. \textbf{6.} $(A + B)(A^2 - AB + B^2) = A^3 + B^3$. \textbf{7.} $(A - B)(A^2 + AB + B^2) = A^3 - B^3$. \textbf{8.} Bình phương của đa thức: $(a + b + c)^2 = a^2 + b^2 + c^2 + 2ab + 2bc + 2ca$, $(a + b + c + d)^2 = a^2 + b^2 + c^2 + d^2 + 2ab + 2ac + 2ad + 2bc + 2bd + 2cd$, $\ldots$. \textbf{9.} Lũy thừa bậc $n$ của 1 nhị thức (nhị thức Newton):
\begin{align*}
	(a + b)^0 &= 1,\\
	(a + b)^1 &= 1a + 1b,\\
	(a + b)^2 &= 1a^2 + 2ab + 1b^2,\\
	(a + b)^3 &= 1a^3 + 3a^2b + 3ab^2 + 1b^3,\\
	(a + b)^4 &= 1a^4 + 4a^3b + 6a^2b^2 + 4ab^3 + 1b^4,\\
	(a + b)^5 &= 1a^5 + 5a^4b + 10a^3b^2 + + 10a^2b^3 + 5ab^4 + 1b^5.
\end{align*}
Ta thấy khi khai triển $(a + b)^n$ ta được 1 đa thức có $n + 1$ hạng tử, hạng tử dầu là $a^n$, hạng tử cuối là $b^n$, các hạng tử còn lại đều chứa các nhân tử $a$ \& $b$. Vì vậy $(a + b)^n = B(a) + b^n = B(b) + a^n$. \textbf{10.} Nếu viết riêng các hệ số ở vế phải ta được bảng sau (gọi là \textit{tam giác Pascal}):
\begin{align*}
	&1\\
	&1\ \ 1\\
	&1\ \ 2\ \ 1\\
	&1\ \ 3\ \ 3\ \ 1\\
	&1\ \ 4\ \ 6\ \ 4\ \ 1\\
	&1\ \ 5\ \ 10\ \ 10\ \ 5\ \ 1\\
	&\ldots
\end{align*}
Nhận xét: Mỗi dòng đều bắt đầu bằng 1 \& kết thúc bằng 1. Mỗi số trên 1 dòng kể từ dòng thứ 2 đều bằng số liền trên cộng với số bên trái của số liền trên.'' -- \cite[\S2, pp. 6--7]{Tuyen_Toan_8}

\begin{baitoan}[\cite{Tuyen_Toan_8}, Ví dụ 2, p. 7]
	Cho $x + y = 9$, $xy = 14$. Tính giá trị của các biểu thức sau:
	\begin{enumerate*}
		\item[(a)] $x - y$;
		\item[(b)] $x^2 + y^2$;
		\item[(c)] $x^3 + y^3$.
	\end{enumerate*}
\end{baitoan}

\begin{proof}[Giải]
	\begin{enumerate*}
		\item[(a)] $(x - y)^2 = (x + y)^2 - 4xy = 9^2 - 4\cdot14 = 25\Rightarrow x - y = \pm5$.
		\item[(b)] $x^2 + y^2 = (x + y)^2 - 2xy = 9^2 - 2\cdot14 = 53$.
		\item[(c)] $x^3 + y^3 = (x + y)^3 - 3xy(x + y) = 9^3 - 3\cdot14\cdot9 = 351$.
	\end{enumerate*}
\end{proof}

\begin{baitoan}[Mở rộng \cite{Tuyen_Toan_8}, p. 7]
	Cho $x + y = a$, $xy = b$, với $a,b\in\mathbb{R}$, $a^2\ge 4b$. Tính giá trị của các biểu thức sau theo $a,b$:
	\begin{enumerate*}
		\item[(a)] $x - y$;
		\item[(b)] $x^2 + y^2$;
		\item[(c)] $x^2 - y^2$;
		\item[(d)] $x^3 + y^3$;
		\item[(e)] $x^3 - y^3$.
	\end{enumerate*}
\end{baitoan}

\begin{proof}[Giải]
	(a) $(x - y)^2 = (x + y)^2 - 4xy = a^2 - 4\cdot b\Rightarrow x - y = \pm\sqrt{a^2 - 4b}$. (b) $x^2 + y^2 = (x + y)^2 - 2xy = a^2 - 2\cdot b$. (c) $x^2 - y^2 = (x - y)(x + y) = \pm a\sqrt{a^2 - 4b}$. (d) $x^3 + y^3 = (x + y)^3 - 3xy(x + y) = a^3 - 3ab$. (e) $x^3 - y^3 = (x - y)(x^2 + xy + y^2) = \pm\sqrt{a^2 - 4b}(a^2 - 2b + b) = \pm\sqrt{a^2 - 4b}(a^2 - b)$.
\end{proof}

\begin{baitoan}[\cite{Tuyen_Toan_8}, Ví dụ 3, p. 8]
	Tìm giá trị nhỏ nhất của biểu thức: $A = (x + 3y - 5)^2 - 6xy + 26$.
\end{baitoan}



%------------------------------------------------------------------------------%

\section{Phân Tích Đa Thức Thành Nhân Tử}

%------------------------------------------------------------------------------%

\printbibliography[heading=bibintoc]
	
\end{document}