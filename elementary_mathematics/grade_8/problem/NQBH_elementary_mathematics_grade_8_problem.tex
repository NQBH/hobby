\documentclass{article}
\usepackage[backend=biber,natbib=true,style=authoryear]{biblatex}
\addbibresource{/home/hong/1_NQBH/reference/bib.bib}
\usepackage[utf8]{vietnam}
\usepackage{tocloft}
\renewcommand{\cftsecleader}{\cftdotfill{\cftdotsep}}
\usepackage[colorlinks=true,linkcolor=blue,urlcolor=red,citecolor=magenta]{hyperref}
\usepackage{amsmath,amssymb,amsthm,mathtools,float,graphicx,algpseudocode,algorithm,tcolorbox}
\usepackage[inline]{enumitem}
\allowdisplaybreaks
\numberwithin{equation}{section}
\newtheorem{assumption}{Assumption}[section]
\newtheorem{conjecture}{Conjecture}[section]
\newtheorem{corollary}{Corollary}[section]
\newtheorem{hequa}{Hệ quả}[section]
\newtheorem{definition}{Definition}[section]
\newtheorem{dinhnghia}{Định nghĩa}[section]
\newtheorem{example}{Example}[section]
\newtheorem{vidu}{Ví dụ}[section]
\newtheorem{lemma}{Lemma}[section]
\newtheorem{notation}{Notation}[section]
\newtheorem{principle}{Principle}[section]
\newtheorem{problem}{Problem}[section]
\newtheorem{baitoan}{Bài toán}[section]
\newtheorem{proposition}{Proposition}[section]
\newtheorem{question}{Question}[section]
\newtheorem{cauhoi}{Câu hỏi}[section]
\newtheorem{remark}{Remark}[section]
\newtheorem{luuy}{Lưu ý}[section]
\newtheorem{theorem}{Theorem}[section]
\newtheorem{dinhly}{Định lý}[section]
\usepackage[left=0.5in,right=0.5in,top=1.5cm,bottom=1.5cm]{geometry}
\usepackage{fancyhdr}
\pagestyle{fancy}
\fancyhf{}
\lhead{\small Sect.~\thesection}
\rhead{\small \nouppercase{\leftmark}}
\renewcommand{\sectionmark}[1]{\markboth{#1}{}}
\cfoot{\thepage}
\def\labelitemii{$\circ$}

\title{Problems in Elementary Mathematics\texttt{/}Grade 8}
\author{Nguyễn Quản Bá Hồng\footnote{Independent Researcher, Ben Tre City, Vietnam\\e-mail: \texttt{nguyenquanbahong@gmail.com}; website: \url{https://nqbh.github.io}.}}
\date{\today}

\begin{document}
\maketitle
\begin{abstract}
	1 bộ sưu tập các bài toán chọn lọc từ cơ bản đến nâng cao cho Toán sơ cấp lớp *. Tài liệu này là phần bài tập bổ sung cho tài liệu chính \href{https://github.com/NQBH/hobby/blob/master/elementary_mathematics/grade_8/NQBH_elementary_mathematics_grade_8.pdf}{GitHub\texttt{/}NQBH\texttt{/}hobby\texttt{/}elementary mathematics\texttt{/}grade 8\texttt{/}lecture}\footnote{\textsc{url}: \url{https://github.com/NQBH/hobby/blob/master/elementary_mathematics/grade_8/NQBH_elementary_mathematics_grade_8.pdf}.} của tác giả viết cho Toán lớp 8. Phiên bản mới nhất của tài liệu này được lưu trữ ở link sau: \href{https://github.com/NQBH/hobby/blob/master/elementary_mathematics/grade_8/problem/NQBH_elementary_mathematics_grade_8_problem.pdf}{GitHub\texttt{/}NQBH\texttt{/}hobby\texttt{/}elementary mathematics\texttt{/}grade 8\texttt{/}problem}\footnote{\textsc{url}: \url{https://github.com/NQBH/hobby/blob/master/elementary_mathematics/grade_8/problem/NQBH_elementary_mathematics_grade_8_problem.pdf}.}.
\end{abstract}
\tableofcontents
\newpage

%------------------------------------------------------------------------------%

\section{Phép Nhân \& Phép Chia Các Đa Thức}

\subsection{Nhân Đơn Thức với Đa Thức}

%------------------------------------------------------------------------------%

\subsection{Nhân Đa Thức với Đa Thức}

%------------------------------------------------------------------------------%

\subsection{Những Hằng Đẳng Thức Đáng Nhớ}
\begin{baitoan}[\cite{SGK_Toan_8_tap_1}, \textbf{23.}, p. 12]
	Chứng minh các đẳng thức sau:
	\begin{align*}
		(a + b)^2 = (a - b)^2 + 4ab,\ (a - b)^2 = (a + b)^2 - 4ab,\ \forall a,b\in\mathbb{R}.
	\end{align*}
\end{baitoan}

\begin{baitoan}[\cite{SGK_Toan_8_tap_1}, \textbf{25.}, p. 12]
	Tính
	\begin{enumerate*}
		\item[(a)] $(a + b + c)^2$;
		\item[(b)] $(a + b - c)^2$;
		\item[(c)] $a - b - c)^2$.
	\end{enumerate*}
\end{baitoan}
Tổng quát hơn,
\begin{baitoan}
	Với $n\in\mathbb{N}^\star$ cho trước, tính $\left(\sum_{i=1}^n a_i\right)^2 = (a_1 + \cdots + a_n)^2$, sau đó phát biểu đẳng thức tìm được bằng lời. Từ đó suy ra kết quả của $\left(\sum_{i=1}^n \pm a_i\right)^2 = (\pm a_1\pm\cdots\pm a_n)^2$.
\end{baitoan}

\begin{baitoan}[\cite{SGK_Toan_8_tap_1}, \textbf{31.}, p. 16]
	Chứng minh rằng:
	\begin{align*}
		a^3 + b^3 = (a + b)^3 - 3ab(a + b),\ a^3 - b^3 = (a - b)^3 + 3ab(a - b),\ \forall a,b\in\mathbb{R}.
	\end{align*}
	Áp dụng: Tính $a^3 + b^3$ biết $ab = m$ \& $a + b = n$ với $m,n\in\mathbb{R}$ cho trước. Tính $a^3 - b^3$ biết $ab = m$ \& $a - b = k$ với $m,k\in\mathbb{R}$ cho trước.
\end{baitoan}

%------------------------------------------------------------------------------%

\subsection{Phân Tích Đa Thức Thành Nhân Tử Bằng Phương Pháp Đặt Nhân Tử Chung}


%------------------------------------------------------------------------------%

\subsection{Phân Tích Đa Thức Thành Nhân Tử Bằng Phương Pháp Dùng Hằng Đẳng Thức}

%------------------------------------------------------------------------------%

\subsection{Phân Tích Đa Thức Thành Nhân Tử Bằng Phương Pháp Nhóm Hạng Tử}

%------------------------------------------------------------------------------%

\subsection{Phân Tích Đa Thức Thành Nhân Tử Bằng Cách Phối Hợp Nhiều Phương Pháp}
\begin{baitoan}[\cite{SGK_Toan_8_tap_1}, \textbf{58.}, p. 25]
	Chứng minh rằng $n^3 - n\ \vdots\ 6$, $\forall n\in\mathbb{Z}$.
\end{baitoan}

%------------------------------------------------------------------------------%

\subsection{Chia Đơn Thức Cho Đơn Thức}

%------------------------------------------------------------------------------%

\subsection{Chia Đa Thức Cho Đơn Thức}

%------------------------------------------------------------------------------%

\subsection{Chia Đa Thức 1 Biến Đã Sắp Xếp}

%------------------------------------------------------------------------------%

\section{Phân Thức Đại Số}

\subsection{Phân Thức Đại Số}

%------------------------------------------------------------------------------%

\subsection{Tính Chất Cơ Bản của Phân Thức}

%------------------------------------------------------------------------------%

\subsection{Rút Gọn Phân Thức}

%------------------------------------------------------------------------------%

\subsection{Quy Đồng Mẫu thức Nhiều Phân Thức}

%------------------------------------------------------------------------------%

\subsection{Phép Cộng Các Phân Thức Đại Số}

%------------------------------------------------------------------------------%

\subsection{Phép Trừ Các Phân Thức Đại Số}

%------------------------------------------------------------------------------%

\subsection{Phép Nhân Các Phân Thức Đại Số}

%------------------------------------------------------------------------------%

\subsection{Phép Chia Các Phân Thức Đại Số}

%------------------------------------------------------------------------------%

\subsection{Biến Đổi Các Biểu Thức Hữu Tỷ. Giá Trị của Phân Thức}

%------------------------------------------------------------------------------%

\section{Phương Trình Đại Số 1 Ẩn -- Algebraic Equation with 1 Unknown}

\subsection{Mở Đầu về Phương Trình}

%------------------------------------------------------------------------------%

\subsection{Phương Trình Bậc Nhất 1 Ẩn \& Cách Giải}

%------------------------------------------------------------------------------%

\subsection{Phương Trình Đưa Được về Dạng $ax + b = 0$}

%------------------------------------------------------------------------------%

\subsection{Phương Trình Tích}

%------------------------------------------------------------------------------%

\subsection{Phương Trình Chứa Ẩn ở Mẫu}

%------------------------------------------------------------------------------%

\subsection{Giải Bài Toán Bằng Cách Lập Phương Trình}

%------------------------------------------------------------------------------%

\section{Bất Phương Trình Bậc Nhất 1 Ẩn -- Algebraic Inequation with 1 Unknown}

\subsection{Liên Hệ Giữa Thứ Tự \& Phép Cộng}

%------------------------------------------------------------------------------%

\subsection{Liên Hệ Giữa Thứ Tự \& Phép Nhân}

%------------------------------------------------------------------------------%

\subsection{Bất Phương Trình 1 Ẩn}

%------------------------------------------------------------------------------%

\subsection{Bất Phương Trình Bậc Nhất 1 Ẩn}

%------------------------------------------------------------------------------%

\subsection{Phương Trình Chứa Dấu Giá Trị Tuyệt Đối}

%------------------------------------------------------------------------------%

\section{Tứ Giác}

\subsection{Tứ Giác}

%------------------------------------------------------------------------------%

\subsection{Hình Thang}

%------------------------------------------------------------------------------%

\subsection{Hình Thang Cân}

%------------------------------------------------------------------------------%

\subsection{Đường Trung Bình của Tam Giác, của Hình Thang}

%------------------------------------------------------------------------------%

\subsection{Dựng Hình Bằng Thước \& Compa. Dựng Hình thang}

%------------------------------------------------------------------------------%

\subsection{Đối Xứng Trục}

%------------------------------------------------------------------------------%

\subsection{Hình Bình Hành}

%------------------------------------------------------------------------------%

\subsection{Đối Xứng Tâm}

%------------------------------------------------------------------------------%

\subsection{Hình Chữ Nhật}

%------------------------------------------------------------------------------%

\subsection{Đường Thẳng Song Song với 1 Đường Thẳng Cho Trước}

%------------------------------------------------------------------------------%

\subsection{Hình Thoi}

%------------------------------------------------------------------------------%

\subsection{Hình Vuông}

%------------------------------------------------------------------------------%

\section{Đa Giác. Diện Tích Đa Giác}

\subsection{Đa Giác. Đa Giác Đều}

%------------------------------------------------------------------------------%

\subsection{Diện Tích Hình Chữ Nhật}

%------------------------------------------------------------------------------%

\subsection{Diện Tích Tam Giác}

%------------------------------------------------------------------------------%

\subsection{Diện Tích Hình Thang}

%------------------------------------------------------------------------------%

\subsection{Diện Tích Hình Thoi}

%------------------------------------------------------------------------------%

\subsection{Diện Tích Đa Giác}

%------------------------------------------------------------------------------%

\section{Tam Giác Đồng Dạng}

\subsection{Định Lý Thales Trong Tam Giác}

%------------------------------------------------------------------------------%

\subsection{Định Lý Đảo \& Hệ Quả của Định Lý Thales}

%------------------------------------------------------------------------------%

\subsection{Tính Chất Đường Phân Giác của Tam Giác}

%------------------------------------------------------------------------------%

\subsection{Khái Niệm 2 Tam Giác Đồng Dạng}

%------------------------------------------------------------------------------%

\subsection{Trường Hợp Đồng Dạng Thứ Nhất}

%------------------------------------------------------------------------------%

\subsection{Trường Hợp Đồng Dạng Thứ 2}

%------------------------------------------------------------------------------%

\subsection{Trường Hợp Đồng Dạng Thứ 3}

%------------------------------------------------------------------------------%

\subsection{Các Trường Hợp Đồng Dạng của Tam Giác Vuông}

%------------------------------------------------------------------------------%

\subsection{Ứng Dụng Thực Tế của Tam Giác Đồng Dạng}

%------------------------------------------------------------------------------%

\section{Hình Lăng Trụ Đứng. Hình Chóp Đều}

\begin{center}
	\Large A -- Hình Lăng Trụ Đứng
\end{center}

\subsection{Hình Hộp Chữ Nhật}

%------------------------------------------------------------------------------%

\subsection{Thể Tích của Hình Hộp Chữ Nhật}

%------------------------------------------------------------------------------%

\subsection{Hình Lăng Trụ Đứng}

%------------------------------------------------------------------------------%

\subsection{Diện Tích Xung Quanh của Hình Lăng Trụ Đứng}

%------------------------------------------------------------------------------%

\subsection{Thể Tích của Hình Lăng Trụ Đứng}

%------------------------------------------------------------------------------%

\begin{center}
	\Large B -- Hình Chóp Đều
\end{center}

\subsection{Hình Chóp Đều \& Hình Chóp Cụt Đều}

%------------------------------------------------------------------------------%

\subsection{Diện Tích Xung Quanh của Hình Chóp Đều}

%------------------------------------------------------------------------------%

\subsection{Thể Tích của Hình Chóp Đều}

%------------------------------------------------------------------------------%

\printbibliography[heading=bibintoc]
	
\end{document}