\documentclass{article}
\usepackage[backend=biber,natbib=true,style=authoryear]{biblatex}
\addbibresource{/home/nqbh/reference/bib.bib}
\usepackage[utf8]{vietnam}
\usepackage{tocloft}
\renewcommand{\cftsecleader}{\cftdotfill{\cftdotsep}}
\usepackage[colorlinks=true,linkcolor=blue,urlcolor=red,citecolor=magenta]{hyperref}
\usepackage{amsmath,amssymb,amsthm,mathtools,float,graphicx,algpseudocode,algorithm,tcolorbox}
\usepackage[inline]{enumitem}
\allowdisplaybreaks
\numberwithin{equation}{section}
\newtheorem{assumption}{Assumption}[section]
\newtheorem{conjecture}{Conjecture}[section]
\newtheorem{corollary}{Corollary}[section]
\newtheorem{hequa}{Hệ quả}[section]
\newtheorem{definition}{Definition}[section]
\newtheorem{dinhnghia}{Định nghĩa}[section]
\newtheorem{example}{Example}[section]
\newtheorem{vidu}{Ví dụ}[section]
\newtheorem{lemma}{Lemma}[section]
\newtheorem{notation}{Notation}[section]
\newtheorem{principle}{Principle}[section]
\newtheorem{problem}{Problem}[section]
\newtheorem{baitoan}{Bài toán}[section]
\newtheorem{proposition}{Proposition}[section]
\newtheorem{question}{Question}[section]
\newtheorem{cauhoi}{Câu hỏi}[section]
\newtheorem{remark}{Remark}[section]
\newtheorem{luuy}{Lưu ý}[section]
\newtheorem{theorem}{Theorem}[section]
\newtheorem{dinhly}{Định lý}[section]
\usepackage[left=0.5in,right=0.5in,top=1.5cm,bottom=1.5cm]{geometry}
\usepackage{fancyhdr}
\pagestyle{fancy}
\fancyhf{}
\lhead{\small Subsect.~\thesubsection}
\rhead{\small\nouppercase{\leftmark}}
\renewcommand{\subsectionmark}[1]{\markboth{#1}{}}
\cfoot{\thepage}
\def\labelitemii{$\circ$}

\title{Problems in Elementary Mathematics\texttt{/}Grade 8}
\author{Nguyễn Quản Bá Hồng\footnote{Independent Researcher, Ben Tre City, Vietnam\\e-mail: \texttt{nguyenquanbahong@gmail.com}; website: \url{https://nqbh.github.io}.}}
\date{\today}

\begin{document}
\maketitle
\begin{abstract}
	1 bộ sưu tập các bài toán chọn lọc từ cơ bản đến nâng cao cho Toán sơ cấp lớp 8. Tài liệu này là phần bài tập bổ sung cho tài liệu chính \href{https://github.com/NQBH/hobby/blob/master/elementary_mathematics/grade_8/NQBH_elementary_mathematics_grade_8.pdf}{GitHub\texttt{/}NQBH\texttt{/}hobby\texttt{/}elementary mathematics\texttt{/}grade 8\texttt{/}lecture}\footnote{\textsc{url}: \url{https://github.com/NQBH/hobby/blob/master/elementary_mathematics/grade_8/NQBH_elementary_mathematics_grade_8.pdf}.} của tác giả viết cho Toán lớp 8. Phiên bản mới nhất của tài liệu này được lưu trữ ở link sau: \href{https://github.com/NQBH/hobby/blob/master/elementary_mathematics/grade_8/problem/NQBH_elementary_mathematics_grade_8_problem.pdf}{GitHub\texttt{/}NQBH\texttt{/}hobby\texttt{/}elementary mathematics\texttt{/}grade 8\texttt{/}problem}\footnote{\textsc{url}: \url{https://github.com/NQBH/hobby/blob/master/elementary_mathematics/grade_8/problem/NQBH_elementary_mathematics_grade_8_problem.pdf}.}.
\end{abstract}
\tableofcontents
\newpage

%------------------------------------------------------------------------------%

\subsection*{Ký Hiệu -- Notation}
\begin{enumerate*}
	\item[$\bullet$] Tổng hữu hạn\texttt{/}finite sum: $\sum_{i=a}^b f(i) = f(a) + f(a + 1) + \cdots + f(b)$, $\forall f$: hàm số, $\forall a,b\in\mathbb{Z}$, $a\le b$.
	\item[$\bullet$] Tích hữu hạn\texttt{/}finite product: $\prod_{i=a}^b f(i) = f(a)f(a + 1)\cdots f(b)$, $\forall f$: hàm số, $\forall a,b\in\mathbb{Z}$, $a\le b$.
\end{enumerate*}

\section{Phép Nhân \& Phép Chia Các Đa Thức}

\subsection{Nhân Đơn Thức với Đa Thức}
\begin{baitoan}[Đơn thức nhân đơn thức]
	Tính:
	\begin{enumerate*}
		\item[(a)] \emph{(Tích 2 đơn thức cùng biến)} $ax^mbx^n$, $\forall a,b\in\mathbb{R}$, $\forall m,n\in\mathbb{N}$.
		\item[(b)] \emph{(Tích 2 đơn thức khác biến)} $ax^mby^n$, $\forall a,b\in\mathbb{R}$, $\forall m,n\in\mathbb{N}$.
		\item[(c)] \emph{(Tích nhiều đơn thức cùng biến)} $\prod_{i=1}^n a_ix^{m_i} = a_1x^{m_1}\cdots a_nx^{m_n}$, $\forall n\in\mathbb{N}^\star$, $a_i\in\mathbb{R}$, $m_i\in\mathbb{N}$, $\forall i = 1,\ldots,n$.
		\item[(d)] \emph{(Tích nhiều đơn thức khác biến)} $\prod_{i=1}^n a_ix_i^{m_i} = a_1x_1^{m_1}\cdots a_nx_n^{m_n}$, $\forall n\in\mathbb{N}^\star$, $a_i\in\mathbb{R}$, $m_i\in\mathbb{N}$, $\forall i = 1,\ldots,n$. Trong đó, $x,y,x_i$'s là các biến số của các đơn thức này.
	\end{enumerate*}
\end{baitoan}

\begin{baitoan}[Đơn thức nhân đa thức]
	Tính:
	\begin{enumerate*}
		\item[(a)] \emph{(Đơn thức 1 biến nhân đa thức bậc nhất 1 biến)} $Cx^n(ax + b)$, $\forall C,a,b\in\mathbb{R}$, $\forall n\in\mathbb{N}$.
		\item[(b)] \emph{(Đơn thức 1 biến nhân đa thức bậc 2 1 biến)} $Cx^n(ax^2 + bx + c)$.
		\item[(c)] \emph{(Đơn thức 1 biến nhân đa thức bậc 3 1 biến)} $ax^n(bx^3 + cx^2 + dx + e)$.
		\item[(d)] \emph{(Đơn thức 1 biến nhân đa thức bậc $n$ 1 biến)} $ax^m\sum_{i=0}^n a_ix^i = ax^m(a_nx^n + a_{n-1}x^{n-1} + \cdots + a_1x + a_0)$.
	\end{enumerate*}
\end{baitoan}

\begin{baitoan}[\cite{Binh_Toan_8_tap_1}, Ví dụ 1, p. 5]
	Tính giá trị của biểu thức: $A = x^4 - 17x^3 + 17x^2 - 17x + 20$ tại $x = 16$.
\end{baitoan}

\begin{baitoan}[\cite{Binh_Toan_8_tap_1}, Ví dụ 2, p. 5]
	Tìm 3 số tự nhiên liên tiếp, biết rằng nếu cộng 3 tích của 2 trong 3 số ấy, ta được $242$.
\end{baitoan}

\begin{baitoan}[\cite{Binh_Toan_8_tap_1}, \textbf{1.}, p. 5]
	Thực hiện phép tính:
	\begin{enumerate*}
		\item[(a)] $3x^n(6x^{n-3} + 1) - 2x^n(9x^{n-3} - 1)$.
		\item[(b)] $5^{n+1} - 4\cdot 5^n$.
		\item[(c)] $6^2\cdot 6^4 - 4^3(3^6 - 1)$.
	\end{enumerate*}
\end{baitoan}

\begin{baitoan}[\cite{Binh_Toan_8_tap_1}, \textbf{2.}, p. 6]
	Tìm $x$, biết:
	\begin{enumerate*}
		\item[(a)] $4(18 - 5x) - 12(3x - 7) = 15(2x - 16) - 6(x + 14)$.
		\item[(b)] $5(3x + 5) - 4(2x - 3) =$ $5x + 3(2x + 12) + 1$.
		\item[(c)] $2(5x - 8) - 3(4x - 5) = 4(3x - 4) + 11$.
		\item[(d)] $5x - 3\{4x - 2[4x - 3(5x - 2)]\} = 182$.
	\end{enumerate*}
\end{baitoan}

\begin{baitoan}[\cite{Binh_Toan_8_tap_1}, \textbf{3.}, p. 6]
	Tính giá trị của các biểu thức:
	\begin{enumerate*}
		\item[(a)] $A = x^3 - 30x^2 - 31x + 1$ tại $x = 31$.
		\item[(b)] $B = x^5 - 15x^4$ $+ 16x^3 - 29x^2 + 13x$ tại $x = 14$.
		\item[(c)] $C = x^{14} - 10x^3 + 10x^2 - 10x^{11} + \cdots + 10x^2 - 10x + 10$ tại $x = 9$.
	\end{enumerate*}
\end{baitoan}

\begin{baitoan}[\cite{Binh_Toan_8_tap_1}, \textbf{4.}, p. 6]
	Tính giá trị của biểu thức sau bằng cách thay số bởi chữ 1 cách hợp lý:
	\begin{align*}
		A = 2\frac{1}{315}\cdot\frac{1}{651} - \frac{1}{105}\cdot 3\frac{650}{651} - \frac{4}{315\cdot 651} + \frac{4}{105}.
	\end{align*}
\end{baitoan}

%------------------------------------------------------------------------------%

\subsection{Nhân Đa Thức với Đa Thức}

\begin{baitoan}[Đa thức nhân đa thức]
	Tính:
	\begin{enumerate*}
		\item[(a)] $(ax + b)(cx + d)$.
		\item[(b)] $(ax^2 + bx + c)(dx + e)$.
		\item[(c)] $(ax^3 + bx^2 + cx + d)(ex + f)$.
		\item[(d)] $(ax^2 + bx + c)(dx^2 + ex + f)$.
		\item[(e)] $(ax^3 + bx^2 + cx + d)(ex^2 + fx + g)$.
		\item[(f)] $(ax^3 + bx^2 + cx + d)(ex^3 + fx^2 + gx + h)$.
		\item[(g)] $\left(\sum_{i=0}^m a_ix^i\right)\left(\sum_{j=0}^n b_jx^j\right) = (a_mx^m + \cdots + a_1x + a_0)(b_nx^n + \cdots + b_1x + b_0)$.
	\end{enumerate*}
\end{baitoan}

\begin{baitoan}[\cite{Binh_Toan_8_tap_1}, \textbf{5.}, p. 6]
	Thực hiện phép tính:
	\begin{enumerate*}
		\item[(a)] $(x - 1)(x^5 + x^4 + x^3 + x^2 + x + 1)$.
		\item[(b)] $(x + 1)(x^6 - x^5 + x^4 - x^3 + x^2 - x + 1)$.
	\end{enumerate*}
\end{baitoan}
1 tổng quát của bài toán này:

\begin{baitoan}
	Tính:
	\begin{enumerate*}
		\item[(a)] $(x - 1)\sum_{i=0}^n x^i = (x - 1)(x^n + x^{n-1} + \cdots + x + 1)$, $\forall n\in\mathbb{N}$.
		\item[(b)] $(x + 1)\sum_{i=0}^n (-1)^ix^i$.
	\end{enumerate*}
\end{baitoan}

\begin{baitoan}[\cite{Binh_Toan_8_tap_1}, \textbf{6.}, p. 6]
	Tìm $x$, biết:
	\begin{enumerate*}
		\item[(a)] $(x + 2)(x + 3) - (x - 2)(x + 5) = 6$.
		\item[(b)] $(3x + 2)(2x + 9) - (x + 2)(6x + 1)$ $= (x + 1) - (x - 6)$.
		\item[(c)] $3(2x - 1)(3x - 1) - (2x - 3)(9x - 1) = 0$.
	\end{enumerate*}
\end{baitoan}

\begin{baitoan}[\cite{Binh_Toan_8_tap_1}, \textbf{7.}, p. 6]
	Cho $a + b + c = 0$. Chứng minh: $M = N = P$ với: $M = a(a + b)(a + c)$, $N = b(b + c)(b + a)$, $P = c(c + a)(c + b)$.
\end{baitoan}

\begin{baitoan}[\cite{Binh_Toan_8_tap_1}, \textbf{8.}, p. 6]
	Chứng minh các hằng đẳng thức:
	\begin{enumerate*}
		\item[(a)] $(x + a)(x + b) = x^2 + (a + b)x + ab$.
		\item[(b)] $(x + a)(x + b)(x + c) = x^3 + (a + b + c)x^2 + (ab + bc + ca)x + abc$.
		\item[(c)] $\prod_{i=1}^n x + a_i = (x + a_1)(x + a_2)\cdots(x + a_n)$, $\forall n\in\mathbb{N}^\star$, $\forall a_i\in\mathbb{R}$, $\forall i = 1,\ldots,n$.
	\end{enumerate*}
\end{baitoan}

\begin{baitoan}[\cite{Binh_Toan_8_tap_1}, \textbf{9.}, p. 6]
	Cho $a + b + c = 2p$. Chứng minh hằng đẳng thức: $2bc + b^2 + c^2 - a^2 = 4p(p - a)$.
\end{baitoan}

\begin{baitoan}[\cite{Binh_Toan_8_tap_1}, \textbf{10.}, p. 7]
	Xét các ví dụ: $53\cdot 57 = 3021$, $72\cdot 78 = 5616$. Xây dựng quy tắc nhân nhẩm 2 số có 2 chữ số, trong đó các chữ số hàng chục bằng nhau, còn các chữ số hàng đơn vị có tổng bằng $10$.
\end{baitoan}

\begin{baitoan}[\cite{Binh_Toan_8_tap_1}, \textbf{11.}, p. 7]
	Cho biểu thức $M = (x - a)(x - b) + (x - b)(x - c) + (x - c)(x - a) + x^2$. Tính $M$ theo $a,b,c$, biết $x = \frac{1}{2}(a + b + c)$.
\end{baitoan}

\begin{baitoan}[\cite{Binh_Toan_8_tap_1}, \textbf{12.}, p. 7]
	Cho dãy số $1,3,6,10,15,\ldots,\frac{n(n + 1)}{2},\ldots$. Chứng minh: tổng 2 số hạng liên tiếp của dãy bao giờ cũng là số chính phương.
\end{baitoan}

\begin{baitoan}[\cite{Binh_Toan_8_tap_1}, \textbf{13.}, p. 7]
	Số $a$ gồm $31$ chữ số $1$, số $b$ gồm $38$ chữ số $1$. Chứng minh: $ab - 2\ \vdots\ 3$.
\end{baitoan}

\begin{baitoan}[\cite{Binh_Toan_8_tap_1}, \textbf{14.}, p. 7]
	Số $3^{50} + 1$ có là tích của 2 số tự nhiên liên tiếp không?
\end{baitoan}

\begin{baitoan}[\cite{Binh_Toan_8_tap_1}, \textbf{15.}, p. 7]
	\begin{enumerate*}
		\item[(a)] Thực hiện phép tính: $A = (2^9 + 2^7 + 1)(2^{23} - 2^{21} + 2^{19} - 2^{17} + 2^{14} - 2^{10} + 2^9 - 2^7 + 1)$.
		\item[(b)] Số $2^{32} + 1$ có là số nguyên tố không?
	\end{enumerate*}
\end{baitoan}

%------------------------------------------------------------------------------%

\subsection{Các Hằng Đẳng Thức Đáng Nhớ}
``Thực hiện các phép nhân đa thức, ta được các hằng đẳng thức sau:
\begin{enumerate*}
	\item[\textbf{1.}] $(a + b)^2 = a^2 + 2ab + b^2$.
	\item[\textbf{2.}] $(a - b)^2 = a^2 - 2ab + b^2$.
	\item[\textbf{3.}] $(a + b)(a - b) = a^2 - b^2$.
	\item[\textbf{4.}] $(a + b)^3 = a^3 + 3a^2b + 3ab^2 + b^3 = a^3 + b^3 + 3ab(a + b)$.
	\item[\textbf{5.}] $(a - b)^3 = a^3 - 3a^2b + 3ab^2 - b^3 = a^3 - b^3 - 3ab(a - b)$.
	\item[\textbf{6.}] $(a + b)(a^2 - ab + b^2) = a^3 + b^3$.
	\item[\textbf{7.}] $(a - b)(a^2 + ab + b^2) = a^3 - b^3$. Ta cũng có: $(a + b + c)^2 = a^2 + b^2 + c^2 + 2ab + 2bc + 2ca$.'' Tổng quát hơn, $\left(\sum_{i=1}^n a_i\right)^2 = \sum_{i=1}^n a_i^2 + 2\sum_{1\le i < j\le n} a_ia_j$. ``Tổng quát của các hằng đẳng thức \textbf{3} \& \textbf{7}, ta có hằng đẳng thức:
	\item[\textbf{8.}] $a^n - b^n = (a - b)\sum_{i=0}^{n-1} a^{n-1-i}b^i = (a - b)\left(a^{n-1} + a^{n-2}b + \cdots + ab^{n-2} + b^{n-1}\right)$, $\forall n\in\mathbb{N}^\star$. Tổng quát của hằng đẳng thức \textbf{6}, ta có hằng đẳng thức:
	\item[\textbf{9.}] $a^n + b^n = (a + b)\sum_{i=0}^{n-1} (-1)^ia^{n-1-i}b^i = (a + b)\left(a^{n-1} - a^{n-2}b + \cdots - ab^{n-2} + b^{n-1}\right)$, $\forall n\in\mathbb{N}^\star$, $n$ lẻ. Tổng quát của các hằng đẳng thức \textbf{1}--\textbf{5}, ta có công thức Newton.'' -- \cite[pp. 7--8]{Binh_Toan_8_tap_1}
\end{enumerate*}

\begin{dinhly}[Công thức nhị thức Newton]
	$(a + b)^n = \sum_{i=0}^n C_n^ia^{n-i}b^i = a^n + C_n^1a^{n-1}b + C_n^2a^{n-2}b^2 + \cdots + C_n^{n-1}ab^{n-1} + b^n$, $\forall a,b\in\mathbb{R}$, $\forall n\in\mathbb{N}$, trong đó $C_n^i\coloneqq\frac{n!}{i!(n - i)!}$, $\forall i,n\in\mathbb{N}$, $i\le n$.
\end{dinhly}


\begin{baitoan}[\cite{SGK_Toan_8_tap_1}, \textbf{23.}, p. 12]
	Chứng minh các đẳng thức sau:
	\begin{align*}
		(a + b)^2 = (a - b)^2 + 4ab,\ (a - b)^2 = (a + b)^2 - 4ab,\ \forall a,b\in\mathbb{R}.
	\end{align*}
\end{baitoan}

\begin{baitoan}[\cite{SGK_Toan_8_tap_1}, \textbf{25.}, p. 12]
	Tính
	\begin{enumerate*}
		\item[(a)] $(a + b + c)^2$.
		\item[(b)] $(a + b - c)^2$.
		\item[(c)] $a - b - c)^2$.
	\end{enumerate*}
\end{baitoan}
Tổng quát hơn,
\begin{baitoan}
	Với $n\in\mathbb{N}^\star$ cho trước, tính $\left(\sum_{i=1}^n a_i\right)^2 = (a_1 + \cdots + a_n)^2$, sau đó phát biểu đẳng thức tìm được bằng lời. Từ đó suy ra kết quả của $\left(\sum_{i=1}^n \pm a_i\right)^2 = (\pm a_1\pm\cdots\pm a_n)^2$.
\end{baitoan}

\begin{baitoan}[\cite{SGK_Toan_8_tap_1}, \textbf{31.}, p. 16]
	Chứng minh rằng:
	\begin{align*}
		a^3 + b^3 = (a + b)^3 - 3ab(a + b),\ a^3 - b^3 = (a - b)^3 + 3ab(a - b),\ \forall a,b\in\mathbb{R}.
	\end{align*}
	Áp dụng: Tính $a^3 + b^3$ biết $ab = m$ \& $a + b = n$ với $m,n\in\mathbb{R}$ cho trước. Tính $a^3 - b^3$ biết $ab = m$ \& $a - b = k$ với $m,k\in\mathbb{R}$ cho trước.
\end{baitoan}

\begin{baitoan}[\cite{Binh_Toan_8_tap_1}, Ví dụ 3, p. 8]
	Chứng minh số $3599$ viết được dưới dạng tích của 2 số tự nhiên khác $1$.
\end{baitoan}

\begin{baitoan}[\cite{Binh_Toan_8_tap_1}, Ví dụ 4, p. 8]
	Chứng minh biểu thức sau viết được dưới dạng tổng các bình phương của 2 biểu thức: $x^2 + 2(x + 1)^2 + 3(x + 2)^2 + 4(x + 3)^2$.
\end{baitoan}

\begin{baitoan}[\cite{Binh_Toan_8_tap_1}, Ví dụ 5, p. 8]
	Cho $x + y + z = 0$, $xy + yz + zx = 0$. Chứng minh $x = y = z$.
\end{baitoan}

\begin{baitoan}[\cite{Binh_Toan_8_tap_1}, Ví dụ 6, p. 9]
	\begin{enumerate*}
		\item[(a)] Tính $A = -1^2 + 2^2 - 3^2 + 4^2 - \cdots - 99^2 + 100^2$.
		\item[(b)] Tính $\sum_{i=1}^n (-1)^ii^2$ $= -1^2 + 2^2 - 3^2 + 4^2 - \cdots + (-1)^nn^2$.
	\end{enumerate*}
\end{baitoan}

\begin{baitoan}[\cite{Binh_Toan_8_tap_1}, Ví dụ 7, p. 9]
	Cho $x + y = a + b$, $x^2 + y^2 = a^2 + b^2$. Chứng minh: $x^3 + y^3 = a^3 + b^3$.
\end{baitoan}

\begin{baitoan}[\cite{Binh_Toan_8_tap_1}, Ví dụ 8, p. 10]
	Cho $a + b = m$, $a - b = m$. Tính $ab$ \& $a^3 - b^3$ theo $m$ \& $n$.
\end{baitoan}

\begin{baitoan}[\cite{Binh_Toan_8_tap_1}, \textbf{16.}, p. 10]
	Tính:
	\begin{enumerate*}
		\item[(a)] $\frac{63^2 - 47^2}{215^2 - 105^2}$.
		\item[(b)] $\frac{437^2 - 363^2}{537^2 - 463^2}$.
	\end{enumerate*}
\end{baitoan}

\begin{baitoan}[\cite{Binh_Toan_8_tap_1}, \textbf{17.}, p. 11]
	So sánh $A = 26^2 - 24^2$ \& $B = 27^2 - 25^2$.
\end{baitoan}

\begin{baitoan}[\cite{Binh_Toan_8_tap_1}, \textbf{18.}, p. 11]
	Tìm $x$ thỏa $4(x + 1)^2 + (2x - 1)^2 - 8(x - 1)(x + 1) = 11$.
\end{baitoan}

\begin{baitoan}[\cite{Binh_Toan_8_tap_1}, \textbf{19.}, p. 11]
	Rút gọn các biểu thức:
	\begin{enumerate*}
		\item[(a)] $2x(2x - 1)^2 - 3x(x + 3)(x - 3) - 4x(x + 1)^2$.
		\item[(b)] $(a - b + c)^2  - (b - c)^2 + 2ab - 2ac$.
		\item[(c)] $(3x + 1)^2 - 2(3x + 1)(3x + 5) + (3x + 5)^2$.
		\item[(d)] $(3 + 1)(3^1 + 1)(3^4 + 1)(3^8 + 1)(3^{16} + 1)(3^{32} + 1)$.
		\item[(e)] $(a + b - c)^2 + (a - b + c)^2 - 2(b - c)^2$.
		\item[(f)] $(a + b + c)^2 + (a - b - c)^2 + (b - c - a)^2 + (c - a - b)^2$.
		\item[(g)] $(a + b + c + d)^2 + (a + b - c - d)^2$ $+ (a + c - b - d)^2 + (a + d - b - c)^2$
	\end{enumerate*}
\end{baitoan}

\begin{baitoan}[\cite{Binh_Toan_8_tap_1}, \textbf{20.}, p. 11]
	Cho $x + y = 3$. Tính giá trị của biểu thức $A = x^2 + 2xy + y^2 - 4x - 4y + 1$.
\end{baitoan}

\begin{baitoan}[\cite{Binh_Toan_8_tap_1}, \textbf{21.}, p. 11]
	Cho $a^2 + b^2 + c^2 = m$. Tính giá trị của biểu thức sau theo $m$: $A = (2a + b - c)^2 + (2b + 2c - a)^2 + (2c + 2a - b)^2$.
\end{baitoan}

\begin{baitoan}[\cite{Binh_Toan_8_tap_1}, \textbf{22.}, p. 11]
	Viết các số sau đây dưới dạng tích của 2 số tự nhiên khác $1$:
	\begin{enumerate*}
		\item[(a)] $899$.
		\item[(b)] $9991$.
	\end{enumerate*}
\end{baitoan}

\begin{baitoan}[\cite{Binh_Toan_8_tap_1}, \textbf{23.}, p. 11]
	Chứng minh hiệu sau đây là 1 số gồm toàn các chữ số như nhau: $7778^2 - 2223^2$.
\end{baitoan}

\begin{baitoan}[\cite{Binh_Toan_8_tap_1}, \textbf{24.}, p. 11]
	Chứng minh các hằng đẳng thức:
	\begin{enumerate*}
		\item[(a)] $(a + b + c)^2 + a^2 + b^2 + c^2 = (a + b)^2 + (b + c)^2 + (c + a)^2$.
		\item[(b)] $x^4 + y^4 + (x + y)^4 = 2(x^2 + xy + y^2)^2$.
	\end{enumerate*}
\end{baitoan}

\begin{baitoan}[\cite{Binh_Toan_8_tap_1}, \textbf{25.}, p. 11]
	Cho $a^2 - b^2 = 4c^2$. Chứng minh hằng đẳng thức $(5a - 3b + 8c)(5a - 3b - 8c) = (3a - 5b)^2$.
\end{baitoan}

\begin{baitoan}[\cite{Binh_Toan_8_tap_1}, \textbf{26.}, p. 11, điều kiện để đẳng thức xảy ra trong bất đẳng thức Cauchy--Schwarz 2 cặp biến]
	Chứng minh nếu $(a^2 + b^2)(x^2 + y^2) = (ax + by)^2$ với $x,y\ne 0$ thì $\frac{a}{x} = \frac{b}{y}$.
\end{baitoan}

\begin{baitoan}[\cite{Binh_Toan_8_tap_1}, \textbf{27.}, p. 12, điều kiện để đẳng thức xảy ra trong bất đẳng thức Cauchy--Schwarz 3 cặp biến]
	Chứng minh nếu $(a^2 + b^2 + c^2)(x^2 + y^2 + z^2) = (ax + by + cz)^2$ với $x,y,z\ne 0$ thì $\frac{a}{x} = \frac{b}{y} = \frac{c}{z}$.
\end{baitoan}

\begin{baitoan}[\cite{Binh_Toan_8_tap_1}, \textbf{28.}, p. 11]
	Cho $(a + b)^2 = 2(a^2 + b^2)$. Chứng minh $a = b$.
\end{baitoan}

\begin{baitoan}[\cite{Binh_Toan_8_tap_1}, \textbf{29.}, p. 12]
	Chứng minh $a = b = c$ nếu có 1 trong các điều kiện sau:
	\begin{enumerate*}
		\item[(a)] $a^2 + b^2 + c^2 = ab + bc + ca$.
		\item[(b)] $(a + b + c)^2 = 3(a^2 + b^2 + c^2)$.
		\item[(c)] $(a + b + c)^2 = 3(ab + bc + ca)$.
	\end{enumerate*}
\end{baitoan}

\begin{baitoan}[\cite{Binh_Toan_8_tap_1}, \textbf{30.}, p. 12]
	Viết các biểu thức sau dưới dạng tổng của 3 bình phương:
	\begin{enumerate*}
		\item[(a)] $(a + b + c)^2 + a^2 + b^2 + c^2$.
		\item[(b)] $2(a - b)(c - b) + 2(b - a)(c - a) + 2(b - c)(a - c)$.
	\end{enumerate*}
\end{baitoan}

\begin{baitoan}[\cite{Binh_Toan_8_tap_1}, \textbf{31.}, p. 12]
	Tính giá trị của biểu thức $a^4 + b^4 + c^4$, biết rằng $a + b + c = 0$ \&:
	\begin{enumerate*}
		\item[(a)] $a^2 + b^2 + c^2 = 2$.
		\item[(b)] $a^2 + b^2 + c^2 = 1$.
	\end{enumerate*}
\end{baitoan}

\begin{baitoan}[\cite{Binh_Toan_8_tap_1}, \textbf{32.}, p. 12]
	Cho $a + b + c = 0$. Chứng minh $a^4 + b^4 + c^4$ bằng mỗi biểu thức:
	\begin{enumerate*}
		\item[(a)] $2(a^2b^2 + b^2c^2 + c^2a^2)$.
		\item[(b)] $2(ab + bc + ca)^2$.
		\item[(c)] $\frac{1}{2}(a^2 + b^2 + c^2)^2$.
	\end{enumerate*}
\end{baitoan}

\begin{baitoan}[\cite{Binh_Toan_8_tap_1}, \textbf{33.}, p. 12]
	Chứng minh các biểu thức sau luôn luôn có giá trị dương với mọi giá trị của biến:
	\begin{enumerate*}
		\item[(a)] $9x^2 - 6x + 2$.
		\item[(b)] $x^2 + x + 1$.
		\item[(c)] $2x^2 + 2x + 1$.
	\end{enumerate*}
\end{baitoan}

\begin{baitoan}[\cite{Binh_Toan_8_tap_1}, \textbf{34.}, p. 12]
	Tìm giá trị nhỏ nhất của các biểu thức:
	\begin{enumerate*}
		\item[(a)] $A = x^2 - 3x + 5$.
		\item[(b)] $B = (2x - 1)^2 + (x + 2)^2$.
	\end{enumerate*}
\end{baitoan}

\begin{baitoan}[\cite{Binh_Toan_8_tap_1}, \textbf{35.}, p. 12]
	Tìm giá trị lớn nhất của các biểu thức:
	\begin{enumerate*}
		\item[(a)] $A = 4 - x^2 + 2x$.
		\item[(b)] $B = 4x - x^2$.
	\end{enumerate*}
\end{baitoan}

\begin{baitoan}[\cite{Binh_Toan_8_tap_1}, \textbf{36.}, p. 12]
	Chứng minh:
	\begin{enumerate*}
		\item[(a)] Nếu $p$ \& $p^2 + 8$ là các số nguyên tố thì $p^2 + 2$ cũng là số nguyên tố.
		\item[(b)] Nếu $p$ \& $8p^2 + 1$ là các số nguyên tố thì $2p + 1$ cũng là số nguyên tố.
	\end{enumerate*}
\end{baitoan}

\begin{baitoan}[\cite{Binh_Toan_8_tap_1}, \textbf{37.}, p. 13]
	Chứng minh các số sau là hợp số:
	\begin{enumerate*}
		\item[(a)] $999991$.
		\item[(b)] $1000027$.
	\end{enumerate*}
\end{baitoan}

\begin{baitoan}[\cite{Binh_Toan_8_tap_1}, \textbf{38.}, p. 13]
	Thực hiện phép tính:
	
	\begin{enumerate*}
		\item[(a)] $(x - 2)^3 - x(x + 1)(x - 1) + 6x(x - 3)$.
		\item[(b)] $(x - 2)(x^2 - 2x + 4)(x + 2)(x^2 + 2x + 4)$.
	\end{enumerate*}
\end{baitoan}

\begin{baitoan}[\cite{Binh_Toan_8_tap_1}, \textbf{39.}, p. 13]
	Tìm $x$, biết:
	\begin{enumerate*}
		\item[(a)] $(x - 3)(x^2 + 3x + 9) + x(x + 2)(2 - x) = 1$.
		\item[(b)] $(x + 1)^3 - (x - 1)^3 - 6(x - 1)^2 = -10$.
	\end{enumerate*}
\end{baitoan}

\begin{baitoan}[\cite{Binh_Toan_8_tap_1}, \textbf{40.}, p. 13]
	Rút gọn các biểu thức:
	\begin{enumerate*}
		\item[(a)] $(a + b + c)^3 - (b + c - a)^3 - (a + c - b)^3 - (a + b - c)^3$.
		\item[(b)] $(a + b)^3 + (b + c)^3 + (c + a)^3 - 3(a + b)(b + c)(c + a)$.
	\end{enumerate*}
\end{baitoan}

\begin{baitoan}[\cite{Binh_Toan_8_tap_1}, \textbf{41.}, p. 13]
	Chứng minh các hằng đẳng thức:
	\begin{enumerate*}
		\item[(a)] $(a + b + c)^3 - a^3 - b^3 - c^3 = 3(a + b)(b + c)(c + a)$.
		\item[(b)] $a^3 + b^3 + c^3 - 3abc = (a + b + c)(a^2 + b^2 + c^2 - ab - bc - ca)$.
	\end{enumerate*}
\end{baitoan}

\begin{baitoan}[\cite{Binh_Toan_8_tap_1}, \textbf{42.}, p. 13]
	Cho $a + b + c = 0$. Chứng minh $a^3 + b^3 + c^3 = 3abc$.
\end{baitoan}

\begin{baitoan}[\cite{Binh_Toan_8_tap_1}, \textbf{43.}, p. 13]
	Cho $x + y = a$ \& $xy = b$. Tính giá trị của các biểu thức sau theo $a$ \& $b$:
	\begin{enumerate*}
		\item[(a)] $x^2 + y^2$.
		\item[(b)] $x^3 + y^3$.
		\item[(c)] $x^4 + y^4$.
		\item[(d)] $x^5 + y^5$.
	\end{enumerate*}
\end{baitoan}

\begin{baitoan}[\cite{Binh_Toan_8_tap_1}, \textbf{44.}, p. 13]
	\begin{enumerate*}
		\item[(a)] Cho $x + y = 1$. Tính giá trị của biểu thức $x^3 + y^3 + 3xy$.
		\item[(b)] Cho $x - y = 1$. Tính giá trị của biểu thức $x^3 - y^3 - 3xy$.
	\end{enumerate*}
\end{baitoan}

\begin{baitoan}[\cite{Binh_Toan_8_tap_1}, \textbf{45.}, p. 13]
	Cho $a + b = 1$. Tính giá trị của biểu thức $M = a^3 + b^3 + 3ab(a^2 + b^2) + 6a^2b^2(a + b)$.
\end{baitoan}

\begin{baitoan}[\cite{Binh_Toan_8_tap_1}, \textbf{46.}, p. 13]
	\begin{enumerate*}
		\item[(a)] Cho $x + y = 2$ \& $x^2 + y^2 = 10$. Tính giá trị của biểu thức $x^3 + y^3$.
		\item[(b)] Cho $x + y = a$ \& $x^2 + y^2 = b$. Tính $x^3 + y^3$ theo $a$ \& $b$.
	\end{enumerate*}
\end{baitoan}

\begin{baitoan}[\cite{Binh_Toan_8_tap_1}, \textbf{47.}, pp. 13--14]
	Chứng minh:
	\begin{enumerate*}
		\item[(a)] Nếu số $n$ là tổng của 2 số chính phương thì $2n$ cũng là tổng của 2 số chính phương.
		\item[(b)] Nếu số $2n$ là tổng của 2 số chính phương thì $n$ cũng là tổng của 2 số chính phương.
		\item[(c)] Nếu số $n$ là tổng của 2 số chính phương thì $n^2$ cũng là tổng của 2 số chính phương.
		\item[(d)] Nếu mỗi số $m$ \& $n$ đều là tổng của 2 số chính phương thì tích $mn$ cũng là tổng của 2 số chính phương.
	\end{enumerate*}
\end{baitoan}

\begin{baitoan}[\cite{Binh_Toan_8_tap_1}, \textbf{48.}, p. 14]
	Mỗi số sau là bình phương của số tự nhiên nào?
	\begin{enumerate*}
		\item[(a)] $A = \underbrace{99\ldots 9}_n\underbrace{00\ldots 0}_n25$.
		\item[(b)] $B = \underbrace{99\ldots 9}_n8\underbrace{00\ldots 0}_n1$.
		\item[(c)] $C = \underbrace{44\ldots 4}_n\underbrace{88\ldots 8}_{n-1}9$.
		\item[(d)] $D = \underbrace{11\ldots 1}_n\underbrace{22\ldots 2}_{n+1}5$.
	\end{enumerate*}
\end{baitoan}

\begin{baitoan}[\cite{Binh_Toan_8_tap_1}, \textbf{49.}, p. 14]
	Chứng minh các biểu thức sau là số chính phương:
	\begin{enumerate*}
		\item[(a)] $A = \underbrace{11\ldots 1}_{2n} - \underbrace{22\ldots 2}_n$.
		\item[(b)] $A = \underbrace{11\ldots 1}_{2n} + \underbrace{44\ldots 4}_n + 1$.
	\end{enumerate*}
\end{baitoan}

\begin{baitoan}[\cite{Binh_Toan_8_tap_1}, \textbf{50.}, p. 14]
	\begin{enumerate*}
		\item[(a)] Cho $a = \underbrace{11\ldots 1}_n$, $b = 1\underbrace{00\ldots 0}_{n-1}5$. Chứng minh $ab + 1$ là số chính phương.
		\item[(b)] Cho 1 dãy số có số hạng đầu là $16$, các số hạng sau là số tạo thành bằng cách viết chèn số $15$ vào chính giữa số hạng liền trước: $16,1156,111556,\ldots$ Chứng minh mọi số hạng của dãy đều là số chính phương.
	\end{enumerate*}
\end{baitoan}

\begin{baitoan}[\cite{Binh_Toan_8_tap_1}, \textbf{51.}, p. 14]
	Chứng minh $ab + 1$ là số chính phương với $a = \underbrace{11\ldots 1}_n2$, $b = \underbrace{11\ldots 1}_n4$.
\end{baitoan}

\begin{baitoan}[\cite{Binh_Toan_8_tap_1}, \textbf{52.}, p. 14]
	Chứng minh với mọi $a\in\mathbb{N}$, tồn tại $b\in\mathbb{N}$ sao cho $ab + 4$ là số chính phương.
\end{baitoan}

\begin{baitoan}[\cite{Binh_Toan_8_tap_1}, \textbf{53.}, p. 14]
	Cho $a$ là số gồm $2n$ chữ số $1$, $b$ là số gồm $n + 1$ chữ số $1$, $c$ là số gồm $n$ chữ số $6$. Chứng minh $a + b + c + 8$ là số chính phương.
\end{baitoan}

\begin{baitoan}[\cite{Binh_Toan_8_tap_1}, \textbf{54.}, p. 14]
	Chứng minh biểu thức sau không là lập phương của 1 số tự nhiên: $10^{150} + 5\cdot 10^{50} + 1$.
\end{baitoan}

\begin{baitoan}[\cite{Binh_Toan_8_tap_1}, \textbf{55.}, p. 14]
	Chứng minh tích 3 số nguyên dương liên  tiếp không là lập phương của 1 số tự nhiên.
\end{baitoan}

\begin{baitoan}[\cite{Binh_Toan_8_tap_1}, \textbf{56.}, p. 14]
	Chứng minh số $A = \frac{1}{3}(\underbrace{11\ldots 1}_n - \underbrace{33\ldots 3}_n\underbrace{00\ldots 0}_n)$ là lập phương của 1 số tự nhiên.
\end{baitoan}

\begin{baitoan}[\cite{Binh_Toan_8_tap_1}, \textbf{57.}, p. 15]
	Chia $27$ quả cân có khối lượng $10,20,30,\ldots,270$ gam thành 3 nhóm có khối lượng bằng nhau.
\end{baitoan}

\begin{baitoan}[\cite{Binh_Toan_8_tap_1}, \textbf{58.}, p. 15]
	Chia $18$ quả cân có khối lượng $1^2,2^2,3^2,\ldots,18^2$ gam thành 3 nhóm có khối lượng bằng nhau.
\end{baitoan}

\begin{baitoan}[\cite{Binh_Toan_8_tap_1}, \textbf{59.}, p. 15]
	Chia $27$ quả cân có khối lượng $1^2,2^2,3^2,\ldots,27^2$ gam thành 3 nhóm có khối lượng bằng nhau.
\end{baitoan}

%------------------------------------------------------------------------------%

\subsection{Phân Tích Đa Thức Thành Nhân Tử Bằng Phương Pháp Đặt Nhân Tử Chung}
``Để phân tích 1 đa thức thành nhân tử, ta thường dùng các phương pháp:
\begin{enumerate*}
	\item[$\bullet$] Đặt nhân tử chung.
	\item[$\bullet$] Dùng các hằng đẳng thức đáng nhớ.
	\item[$\bullet$] Nhóm các  hạng tử 1 cách thích hợp nhằm làm xuất hiện dạng hằng đẳng thức hoặc xuất hiện nhân tử chung mới.
\end{enumerate*}
Để phân tích đa thức thành nhân tử, người ta còn dùng các phương pháp khác. Xem chuyên đề \textit{1 số phương pháp phân tích đa thức thành nhân tử}.'' -- \cite[p. 15]{Binh_Toan_8_tap_1}

\begin{baitoan}[\cite{Binh_Toan_8_tap_1}, Ví dụ 9, p. 15]
	Phân tích đa thức sau thành nhân tử: 	$x^4 + x^3 + 2x^2 + x + 1$.
\end{baitoan}

\begin{baitoan}[\cite{Binh_Toan_8_tap_1}, Ví dụ 10, p. 15]
	Cho $a + b + c = 0$. Rút gọn biểu thức $M = a^3 + b^3 + c(a^2 + b^2) - abc$.
\end{baitoan}

\begin{baitoan}[\cite{Binh_Toan_8_tap_1}, Ví dụ 11, p. 16]
	\begin{enumerate*}
		\item[(a)] Phân tích đa thức sau thành nhân tử: $a^3 + b^3 + c^3 - 3abc$.
		\item[(b)] Phân tích đa thức sau thành nhân tử bằng cách áp dụng câu (a): $(x - y)^3 + (y - z)^3 + (z - x)^3$.
	\end{enumerate*}
\end{baitoan}

\begin{baitoan}[\cite{Binh_Toan_8_tap_1}, Ví dụ 12, p. 16]
	Phân tích các đa thức sau thành nhân tử:
	\begin{enumerate*}
		\item[(a)] $(a + b + c)^3 - a^3 - b^3 - c^3$.
		\item[(b)] $8(x + y + z)^3 - (x + y)^3 - (y + z)^3 - (z + x)^3$.
	\end{enumerate*}
\end{baitoan}

\begin{baitoan}[\cite{Binh_Toan_8_tap_1}, Ví dụ 13, p. 17]
	Phân tích đa thức sau thành nhân tử: $P = x^2(y - z) + y^2(z - x) + z^2(x -y)$.
\end{baitoan}

\begin{baitoan}[\cite{Binh_Toan_8_tap_1}, Ví dụ 14, p. 17]
	Xét hằng đẳng thức $(x + 1)^3 = x^3 + 3x^2 + 3x + 1$. Lần lượt cho $x$ bằng $1,2,\ldots,n$ rồi cộng từng vế $n$ đẳng thức trên để tính giá trị của biểu thức: $S = \sum_{i=1}^n i^2 = 1^2 + 2^2 + \cdots + n^2$.
\end{baitoan}

\begin{baitoan}[\cite{Binh_Toan_8_tap_1}, \textbf{60.}, p. 18]
	Phân tích thành nhân tử:
	\begin{enumerate*}
		\item[(a)] $(ab - 1)^2 + (a + b)^2$.
		\item[(b)] $x^3 + 2x^2 + 2x + 1$.
		\item[(c)] $x^3 - 4x^2 + 12x - 27$.
		\item[(d)] $x^4 - 2x^4 + 2x - 1$.
		\item[(e)] $x^4 + 2x^3 + 2x^2 + 2x + 1$.
	\end{enumerate*}
\end{baitoan}

\begin{baitoan}[\cite{Binh_Toan_8_tap_1}, \textbf{61.}, p. 18]
	Phân tích thành nhân tử:
	\begin{enumerate*}
		\item[(a)] $x^2 - 2x - 4y^2 - 4y$.
		\item[(b)] $x^4 + 2x^2 - 4x - 4$.
		\item[(c)] $x^2(1 - x^2) - 4 - 4x^2$.
		\item[(d)] $(1 + 2x)(1 - 2x) - x(x + 2)(x - 2)$.
		\item[(e)] $x^2 + y^2 - x^2y^2 + xy - x - y$.
	\end{enumerate*}
\end{baitoan}

\begin{baitoan}[\cite{Binh_Toan_8_tap_1}, \textbf{62.}, p. 18]
	Chứng minh $199^3 - 199\ \vdots\ 200$.
\end{baitoan}

\begin{baitoan}[\cite{Binh_Toan_8_tap_1}, \textbf{63.}, p. 18]
	Tính giá trị của biểu thức sau, biết $x^3 - x = 6$: $A = x^6 - 2x^4 + x^3 + x^2 - x$.
\end{baitoan}

\begin{baitoan}[\cite{Binh_Toan_8_tap_1}, \textbf{64.}, p. 18]
	Phân tích thành nhân tử:
	\begin{enumerate*}
		\item[(a)] $a(b^2 + c^2 + bc) + b(c^2 + a^2 + ac) + c(a^2 + b^2 + ab)$.
		\item[(b)] $(a + b + c)(ab + bc + ca) - abc$.
		\item[(c)] $a(a + 2b)^3 - b(2a + b)^3$.
	\end{enumerate*}
\end{baitoan}

\begin{baitoan}[\cite{Binh_Toan_8_tap_1}, \textbf{65.}, pp. 18--19]
	Phân tích thành nhân tử:\\
	\begin{enumerate*}
		\item[(a)] $ab(a + b) - bc(b + c) + ac(a - c)$.
		\item[(b)] $a(b^2 + c^2) + b(c^2 + a^2) + c(a^2 + b^2) + 2abc$.
		\item[(c)] $(a + b)(a^2 - b^2) + (b + c)(b^2 - c^2) + (c + a)(c^2 - a^2)$.
		\item[(d)] $a^3(b - c) + b^3(c - a) + c^3(a - b)$.
		\item[(e)] $a^3(b - c) + b^3(c - a) + c^3(a - b)$.
		\item[(f)] $a^3(c - b^2) + b^3(a - c^2) + c^3(b - a^2) + abc(abc - 1)$.
	\end{enumerate*}
\end{baitoan}

\begin{baitoan}[\cite{Binh_Toan_8_tap_1}, \textbf{66.}, p. 19]
	Phân tích thành nhân tử:
	\begin{enumerate*}
		\item[(a)] $a(b + c)^2(b - c) + b(c + a)^2(c - a) + c(a + b)^2(a - b)$.
		\item[(b)] $a(b - c)^3 + b(c - a)^3 + c(a - b)^3$.
		\item[(c)] $a^2b^2(a - b) + b^2c^2(b - c) + c^2a^2(c - a)$.
		\item[(d)] $a(b^2 + c^2) + b(c^2 + a^2) + c(a^2 + b^2) - 2abc - a^3 - b^3 - c^3$.
		\item[(e)] $a^4(b - c) + b^4(c - a) + c^4(a - b)$.
	\end{enumerate*}
\end{baitoan}

\begin{baitoan}[\cite{Binh_Toan_8_tap_1}, \textbf{67.}, p. 19]
	Phân tích thành nhân tử:
	\begin{enumerate*}
		\item[(a)] $(a + b + c)^3 - (a + b - c)^3 - (b + c - a)^3 - (c + a - b)^3$.
		\item[(b)] $abc - (ab + bc + ca) + (a + b + c) - 1$.
	\end{enumerate*}
\end{baitoan}

\begin{baitoan}[\cite{Binh_Toan_8_tap_1}, \textbf{68.}, p. 19]
	Chứng minh rằng trong 3 số $a,b,c$, tồn tại 2 số bằng nhau, nếu: $a^2(b - c) + b^2(c - a) + c^2(a - b) = 0$.
\end{baitoan}

\begin{baitoan}[\cite{Binh_Toan_8_tap_1}, \textbf{69.}, p. 19]
	Chứng minh rằng nếu $a^2 + b^2 = 2ab$ thì $a = b$.
\end{baitoan}

\begin{baitoan}[\cite{Binh_Toan_8_tap_1}, \textbf{70.}, p. 19]
	Chứng minh rằng nếu $a^3 + b^3 + c^3 = 3abc$ \& $a,b,c$ là các số dương thì $a = b = c$.
\end{baitoan}

\begin{baitoan}[\cite{Binh_Toan_8_tap_1}, \textbf{71.}, p. 19]
	Chứng minh rằng nếu $a^4 + b^4 + c^4 + d^4 = 4abcd$ \& $a,b,c,d$ là các số dương thì $a = b = c = d$.
\end{baitoan}

\begin{baitoan}[\cite{Binh_Toan_8_tap_1}, \textbf{72.}, p. 19]
	Chứng minh rằng nếu $m = a + b + c$ thì $(am + bc)(bm + ac)(cm + ab) = (a + b)^2(b + c)^2(c + a)^2$.
\end{baitoan}

\begin{baitoan}[\cite{Binh_Toan_8_tap_1}, \textbf{73.}, p. 19]
	Cho $a^2 + b^2 = 1$, $c^2 + d^2 = 1$, $ac + bd = 0$. Chứng minh rằng $ab + cd = 0$.
\end{baitoan}

\begin{baitoan}[\cite{Binh_Toan_8_tap_1}, \textbf{74.}, p. 19]
	Xét hằng đẳng thức $(x + 1)^2 = x^2 + 2x + 1$. Lần lượt cho $x$ bằng $1,2,\ldots,n$ rồi cộng từng vế $n$ đẳng thức trên để tính giá trị của biểu thức $S_1 = \sum_{i=1}^n i = 1 + 2 + \cdots + n$.
\end{baitoan}

\begin{baitoan}[\cite{Binh_Toan_8_tap_1}, \textbf{75.}, p. 19]
	Tính giá trị của biểu thức $S_3 = \sum_{i=1}^n i^3 = 1^3 + 2^3 + \cdots + n^3$.
\end{baitoan}

\begin{baitoan}[\cite{SGK_Toan_8_tap_1}, \textbf{58.}, p. 25]
	Chứng minh rằng $n^3 - n\ \vdots\ 6$, $\forall n\in\mathbb{Z}$.
\end{baitoan}

%------------------------------------------------------------------------------%

\subsection{Chia Đơn Thức Cho Đơn Thức}

\begin{baitoan}[Đơn thức chia đơn thức]
	Tính:
	\begin{enumerate*}
		\item[(a)] \emph{(Đơn thức 1 biến chia đơn thức 1 biến)} $ax^m:bx^n$, $\forall a,b\in\mathbb{R}$, $b\ne 0$, $\forall m,n\in\mathbb{Z}$.
		\item[(b)] \emph{(Đơn thức 2 biến chia đơn thức 2 biến)} $ax^my^n:bx^py^q$, $\forall a,b\in\mathbb{R}$, $b\ne 0$, $\forall m,n,p,q\in\mathbb{Z}$.
		\item[(c)] \emph{(Đơn thức 3 biến chia đơn thức 3 biến)} $ax^my^nz^p:bx^ty^uz^v$, $\forall a,b\in\mathbb{R}$, $b\ne 0$, $\forall m,n,p,t,u,v\in\mathbb{Z}$.
		\item[(d)] \emph{(Đơn thức $n$ biến chia đơn thức $n$ biến)} $a\prod_{i=1}^n x_i^{a_i}:b\prod_{i=1}^n x_i^{b_i} = ax_1^{a_1}x_2^{a_2}\cdots x_n^{a_n}:bx_1^{b_1}x_2^{b_2}\cdots x_n^{b_n}$, $\forall n\in\mathbb{N}$, $\forall a,b\in\mathbb{R}$, $b\ne 0$, $a_i\in\mathbb{Z}$, $\forall i = 1,\ldots,n$.
	\end{enumerate*}
\end{baitoan}

%------------------------------------------------------------------------------%

\subsection{Chia Đa Thức Cho Đơn Thức}
``\textbf{1.} Chia đơn thức $A$ cho đơn thức $B$:
\begin{enumerate*}
	\item[$\bullet$] Chia hệ số của $A$ cho hệ số của $B$.
	\item[$\bullet$] Chia lũy thừa của từng biến trong $A$ cho lũy thừa của cùng biến đó trong $B$.
	\item[$\bullet$] Nhân các kết quả với nhau.
\end{enumerate*}
\textbf{2.} Chia đa thức $A$ cho đơn thức $B$: Ta chia mỗi hạng tử của $A$ \& $B$ rồi cộng các kết quả với nhau. \textbf{3.} Chia đa thức $A$ cho đa thức $B$: Cho $A$ \& $B$ là 2 đa thức tùy ý của cùng 1 biến, $B\ne 0$, khi đó tồn tại duy nhất 1 cặp đa thức $Q$ \& $R$ sao cho $A = BQ + R$, trong đó $R = 0$ hoặc bậc của $R$ nhỏ hơn bậc của $B$, i.e., $\deg R < \deg B$. $Q$ gọi là \textit{đa thức thương} \& $R$ gọi là \textit{đa thức dư} của phép chia $A$ cho $B$. Nếu $R = 0$ thì phép chia $A$ cho $B$ là \textit{phép chia hết}.

\begin{dinhly}[B\'ezout]
	Số dư trong phép chia đa thức $f(x)$ cho nhị thức bậc nhất $x - a$ đúng bằng $f(a)$.
\end{dinhly}

\begin{hequa}
	Nếu $a$ là nghiệm của đa thức $f(x)$ thì $f(x)$ chia hết cho $(x - a)$.
\end{hequa}
Đặc biệt:
\begin{enumerate*}
	\item[$\bullet$] Nếu tổng các hệ số của đa thức $f(x)$ bằng $0$ thì $1$ là nghiệm \& $f(x)$ chia hết cho $(x - 1)$.
	\item[$\bullet$] Nếu $f(x)$ có tổng các hệ số bậc chẵn bằng tổng các hệ số bậc lẻ thì $-1$ là nghiệm \& $f(x)$ chia hết cho $x - (-1)$, i.e., chia hết cho $x + 1$.
\end{enumerate*}
Áp dụng hệ quả của định lý B\'ezout vào việc phân tích đa thức thành nhân tử: Nếu đa thức $f(x)$ có nghiệm $x = a$ thì khi phân tích $f(x)$ thành nhân tử, tích sẽ chứa nhân tử $(x - a)$.

\textit{Cách nhẩm nghiệm nguyên, nghiệm hữu tỷ của đa thức $f(x)$ với hệ số nguyên.}
\begin{enumerate*}
	\item[$\bullet$] Nếu $f(x)$ có nghiệm nguyên thì nghiệm đó phải là ước của hệ số tự do.
	\item[$\bullet$] Nếu $f(x)$ có nghiệm hữu tỷ thì nghiệm đó có dạng $\frac{p}{q}$, $(p,q) = 1$ trong đó $p$ là ước của hệ số tự do, $q$ là ước dương của hệ số cao nhất.'' -- \cite[pp. 22--23]{Tuyen_Toan_8}
\end{enumerate*}

\begin{baitoan}[Đa thức chia đơn thức]
	Tính:
	\begin{enumerate*}
		\item[(a)] \emph{(Đa thức 1 biến chia đơn thức 1 biến)} $\left(\sum_{i=0}^n a_ix^i\right):ax^m$\\$= (a_nx^n + a_{n-1}x^{n-1} + \cdots + a_1x + a_0):ax^m$, $\forall m,n\in\mathbb{N}$, $a,a_i\in\mathbb{R}$, $\forall i = 1,\ldots,n$, $a\ne 0$.
		\item[(b)] \emph{(Đa thức 2 biến chia đơn thức 2 biến)} $\left(\sum_{i,j=0}^{m,n} a_{ij}x^iy^j\right):ax^py^q = \left(\sum_{i=0}^m\sum_{j=0}^n a_{ij}x^iy^j\right):ax^py^q$, $\forall m,n,p,q\in\mathbb{N}$, $a,a_{ij}\in\mathbb{R}$, $\forall i = 1,\ldots,m$, $\forall j = 1,\ldots,n$.
		\item[(c)] \emph{(Đa thức 3 biến chia đơn thức 3 biến)} $\left(\sum_{i,j,k=0}^{m,n,p} a_{ij}x^iy^jz^k\right):ax^ty^uz^v = \left(\sum_{i=0}^m\sum_{j=0}^n\sum_{k=0}^p a_{ijk}x^iy^jz^k\right):ax^ty^uz^v$.
		\item[(d)] \emph{(Đa thức $n$ biến chia đơn thức $n$ biến)}
	\end{enumerate*}
\end{baitoan}

\begin{baitoan}[\cite{Tuyen_Toan_8}, Ví dụ 9, p. 24]
	Xác định các hệ số $a$ \& $b$ sao cho $x^4 + ax^3 + b$ chia hết cho $x^2 - 1$.
\end{baitoan}
Phương pháp xét giá trị riêng của các biến cũng là 1 phương pháp phân tích đa thức thành nhân tử.

\begin{baitoan}[\cite{Tuyen_Toan_8}, Ví dụ 10, p. 25]
	Phân tích đa thức thành nhân tử $M = xy(x + y) + yz(y + z) + zx(z + x) + 2xyz$.
\end{baitoan}
``Phương pháp nhẩm nghiệm của đa thức để vận dụng hệ quả của định lý B\'ezout giúp ta định hướng nhanh chóng việc tách 1 hạng tử thành nhiều hạng tử 1 cách thích hợp.'' -- \cite[pp. 26]{Tuyen_Toan_8}

\begin{baitoan}[\cite{Tuyen_Toan_8}, Ví dụ 11, p. 26]
	Phân tích đa thức thành nhân tử $A = x^3 - x^2 - 8x + 12$.
\end{baitoan}

\begin{baitoan}[\cite{Tuyen_Toan_8}, \textbf{85.}, p. 27]
	Tìm $n\in\mathbb{N}$ để đơn thức $-7x^{n+1}y^6$ chia hết cho $4x^5y^n$.
\end{baitoan}

\begin{baitoan}[\cite{Tuyen_Toan_8}, \textbf{86.}, p. 27]
	Chứng minh: Giá trị của biểu thức $A$ luôn luôn không âm với mọi giá trị khác $0$ của $x$ \& $y$: $A = (75x^5y^2 - 45x^4y^3):3x^3y^2 - \left(\frac{5}{2}x^2y^4 - 2xy^5\right):\frac{1}{2}xy^3$.
\end{baitoan}

\begin{baitoan}[\cite{Tuyen_Toan_8}, \textbf{87.}, p. 27]
	Tìm $x$ \& $y$ biết: $[(x - 2y)(x - 7y) - x^2 + 4y^2]:(x - 2y) = 18$.
\end{baitoan}

\begin{baitoan}[\cite{Tuyen_Toan_8}, \textbf{88.}, p. 27]
	Tìm giá trị nhỏ nhất của thương: $(4x^5 + 2x^4 + 4x^3 - x - 1):(2x^3 + x - 1)$.
\end{baitoan}

%------------------------------------------------------------------------------%

\subsection{Chia Đa Thức 1 Biến Đã Sắp Xếp}

%------------------------------------------------------------------------------%

\section{Phân Thức Đại Số}

\subsection{Phân Thức Đại Số}

%------------------------------------------------------------------------------%

\subsection{Tính Chất Cơ Bản của Phân Thức}

%------------------------------------------------------------------------------%

\subsection{Rút Gọn Phân Thức}

%------------------------------------------------------------------------------%

\subsection{Quy Đồng Mẫu thức Nhiều Phân Thức}

%------------------------------------------------------------------------------%

\subsection{Phép Cộng Các Phân Thức Đại Số}

%------------------------------------------------------------------------------%

\subsection{Phép Trừ Các Phân Thức Đại Số}

%------------------------------------------------------------------------------%

\subsection{Phép Nhân Các Phân Thức Đại Số}

%------------------------------------------------------------------------------%

\subsection{Phép Chia Các Phân Thức Đại Số}

%------------------------------------------------------------------------------%

\subsection{Biến Đổi Các Biểu Thức Hữu Tỷ. Giá Trị của Phân Thức}

%------------------------------------------------------------------------------%

\section{Phương Trình Đại Số 1 Ẩn -- Algebraic Equation with 1 Unknown}

\subsection{Mở Đầu về Phương Trình}

%------------------------------------------------------------------------------%

\subsection{Phương Trình Bậc Nhất 1 Ẩn \& Cách Giải}

%------------------------------------------------------------------------------%

\subsection{Phương Trình Đưa Được về Dạng $ax + b = 0$}

%------------------------------------------------------------------------------%

\subsection{Phương Trình Tích}

%------------------------------------------------------------------------------%

\subsection{Phương Trình Chứa Ẩn ở Mẫu}

%------------------------------------------------------------------------------%

\subsection{Giải Bài Toán Bằng Cách Lập Phương Trình}

%------------------------------------------------------------------------------%

\section{Bất Phương Trình Bậc Nhất 1 Ẩn -- Algebraic Inequation with 1 Unknown}

\subsection{Liên Hệ Giữa Thứ Tự \& Phép Cộng}

%------------------------------------------------------------------------------%

\subsection{Liên Hệ Giữa Thứ Tự \& Phép Nhân}

%------------------------------------------------------------------------------%

\subsection{Bất Phương Trình 1 Ẩn}

%------------------------------------------------------------------------------%

\subsection{Bất Phương Trình Bậc Nhất 1 Ẩn}

%------------------------------------------------------------------------------%

\subsection{Phương Trình Chứa Dấu Giá Trị Tuyệt Đối}

%------------------------------------------------------------------------------%

\section{Tứ Giác}

\subsection{Tứ Giác}


%------------------------------------------------------------------------------%

\subsection{Hình Thang}

%------------------------------------------------------------------------------%

\subsection{Hình Thang Cân}

%------------------------------------------------------------------------------%

\subsection{Đường Trung Bình của Tam Giác, của Hình Thang}

%------------------------------------------------------------------------------%

\subsection{Dựng Hình Bằng Thước \& Compa. Dựng Hình thang}

%------------------------------------------------------------------------------%

\subsection{Đối Xứng Trục}

%------------------------------------------------------------------------------%

\subsection{Hình Bình Hành}

%------------------------------------------------------------------------------%

\subsection{Đối Xứng Tâm}

%------------------------------------------------------------------------------%

\subsection{Hình Chữ Nhật}

%------------------------------------------------------------------------------%

\subsection{Đường Thẳng Song Song với 1 Đường Thẳng Cho Trước}

%------------------------------------------------------------------------------%

\subsection{Hình Thoi}

%------------------------------------------------------------------------------%

\subsection{Hình Vuông}

%------------------------------------------------------------------------------%

\section{Đa Giác. Diện Tích Đa Giác}

\subsection{Đa Giác. Đa Giác Đều}

%------------------------------------------------------------------------------%

\subsection{Diện Tích Hình Chữ Nhật}

%------------------------------------------------------------------------------%

\subsection{Diện Tích Tam Giác}

%------------------------------------------------------------------------------%

\subsection{Diện Tích Hình Thang}

%------------------------------------------------------------------------------%

\subsection{Diện Tích Hình Thoi}

%------------------------------------------------------------------------------%

\subsection{Diện Tích Đa Giác}

%------------------------------------------------------------------------------%

\section{Tam Giác Đồng Dạng}

\subsection{Định Lý Thales Trong Tam Giác}

%------------------------------------------------------------------------------%

\subsection{Định Lý Đảo \& Hệ Quả của Định Lý Thales}

%------------------------------------------------------------------------------%

\subsection{Tính Chất Đường Phân Giác của Tam Giác}

%------------------------------------------------------------------------------%

\subsection{Khái Niệm 2 Tam Giác Đồng Dạng}

%------------------------------------------------------------------------------%

\subsection{Trường Hợp Đồng Dạng Thứ Nhất}

%------------------------------------------------------------------------------%

\subsection{Trường Hợp Đồng Dạng Thứ 2}

%------------------------------------------------------------------------------%

\subsection{Trường Hợp Đồng Dạng Thứ 3}

%------------------------------------------------------------------------------%

\subsection{Các Trường Hợp Đồng Dạng của Tam Giác Vuông}

%------------------------------------------------------------------------------%

\subsection{Ứng Dụng Thực Tế của Tam Giác Đồng Dạng}

%------------------------------------------------------------------------------%

\section{Hình Lăng Trụ Đứng. Hình Chóp Đều}

\begin{center}
	\Large A -- Hình Lăng Trụ Đứng
\end{center}

\subsection{Hình Hộp Chữ Nhật}

%------------------------------------------------------------------------------%

\subsection{Thể Tích của Hình Hộp Chữ Nhật}

%------------------------------------------------------------------------------%

\subsection{Hình Lăng Trụ Đứng}

%------------------------------------------------------------------------------%

\subsection{Diện Tích Xung Quanh của Hình Lăng Trụ Đứng}

%------------------------------------------------------------------------------%

\subsection{Thể Tích của Hình Lăng Trụ Đứng}

%------------------------------------------------------------------------------%

\begin{center}
	\Large B -- Hình Chóp Đều
\end{center}

\subsection{Hình Chóp Đều \& Hình Chóp Cụt Đều}

%------------------------------------------------------------------------------%

\subsection{Diện Tích Xung Quanh của Hình Chóp Đều}

%------------------------------------------------------------------------------%

\subsection{Thể Tích của Hình Chóp Đều}

%------------------------------------------------------------------------------%

\printbibliography[heading=bibintoc]
	
\end{document}