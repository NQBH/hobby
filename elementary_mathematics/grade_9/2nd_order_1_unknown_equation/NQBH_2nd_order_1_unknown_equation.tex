\documentclass{article}
\usepackage[backend=biber,natbib=true,style=authoryear,maxbibnames=10]{biblatex}
\addbibresource{/home/nqbh/reference/bib.bib}
\usepackage[utf8]{vietnam}
\usepackage{tocloft}
\renewcommand{\cftsecleader}{\cftdotfill{\cftdotsep}}
\usepackage[colorlinks=true,linkcolor=blue,urlcolor=red,citecolor=magenta]{hyperref}
\usepackage{amsmath,amssymb,amsthm,float,graphicx,mathtools,soul,subcaption}
\allowdisplaybreaks
\newtheorem{assumption}{Assumption}
\newtheorem{baitoan}{Bài toán}
\newtheorem{cauhoi}{Câu hỏi}
\newtheorem{conjecture}{Conjecture}
\newtheorem{corollary}{Corollary}
\newtheorem{dangtoan}{Dạng toán}
\newtheorem{definition}{Definition}
\newtheorem{dinhly}{Định lý}
\newtheorem{dinhnghia}{Định nghĩa}
\newtheorem{example}{Example}
\newtheorem{ghichu}{Ghi chú}
\newtheorem{hequa}{Hệ quả}
\newtheorem{hypothesis}{Hypothesis}
\newtheorem{lemma}{Lemma}
\newtheorem{luuy}{Lưu ý}
\newtheorem{nhanxet}{Nhận xét}
\newtheorem{notation}{Notation}
\newtheorem{note}{Note}
\newtheorem{principle}{Principle}
\newtheorem{problem}{Problem}
\newtheorem{proposition}{Proposition}
\newtheorem{question}{Question}
\newtheorem{remark}{Remark}
\newtheorem{theorem}{Theorem}
\newtheorem{vidu}{Ví dụ}
\usepackage[left=1cm,right=1cm,top=5mm,bottom=5mm,footskip=4mm]{geometry}
\def\labelitemii{$\circ$}
\DeclareRobustCommand{\divby}{%
	\mathrel{\vbox{\baselineskip.65ex\lineskiplimit0pt\hbox{.}\hbox{.}\hbox{.}}}%
}

\title{2nd-Order 1-Unknown Equation\\Phương Trình Bậc 2 với 1 Ẩn $y = ax^2 + bx + c$, $a\ne0$}
\author{Nguyễn Quản Bá Hồng\footnote{Independent Researcher, Ben Tre City, Vietnam\\e-mail: \texttt{nguyenquanbahong@gmail.com}; website: \url{https://nqbh.github.io}.}}
\date{\today}

\begin{document}
\maketitle
\begin{abstract}
	\textsc{[en]} This text is a collection of problems, from easy to advanced, about \textit{1st-order 2-unknown system of equations}. This text is also a supplementary material for my lecture note on Elementary Mathematics grade 9, which is stored \& downloadable at the following link: \href{https://github.com/NQBH/hobby/blob/master/elementary_mathematics/grade_9/NQBH_elementary_mathematics_grade_9.pdf}{GitHub\texttt{/}NQBH\texttt{/}hobby\texttt{/}elementary mathematics\texttt{/}grade 9\texttt{/}lecture}\footnote{\textsc{url}: \url{https://github.com/NQBH/hobby/blob/master/elementary_mathematics/grade_9/NQBH_elementary_mathematics_grade_9.pdf}.}. The latest version of this text has been stored \& downloadable at the following link: \href{https://github.com/NQBH/hobby/blob/master/elementary_mathematics/grade_9/1st_order_2_unknown_system_of_equations/NQBH_1st_order_2_unknown_system_of_equations.pdf}{GitHub\texttt{/}NQBH\texttt{/}hobby\texttt{/}elementary mathematics\texttt{/}grade 9\texttt{/}1st-order 2-unknown system of equations}\footnote{\textsc{url}: \url{https://github.com/NQBH/hobby/blob/master/elementary_mathematics/grade_9/similar_triangle/NQBH_1st_order_2_unknown_system_of_equations.pdf}.}.
	\vspace{2mm}
	
	\textsc{[vi]} Tài liệu này là 1 bộ sưu tập các bài tập chọn lọc từ cơ bản đến nâng cao về \textit{hàm số bậc nhất}. Tài liệu này là phần bài tập bổ sung cho tài liệu chính -- bài giảng \href{https://github.com/NQBH/hobby/blob/master/elementary_mathematics/grade_9/NQBH_elementary_mathematics_grade_9.pdf}{GitHub\texttt{/}NQBH\texttt{/}hobby\texttt{/}elementary mathematics\texttt{/}grade 9\texttt{/}lecture} của tác giả viết cho Toán Sơ Cấp lớp 9. Phiên bản mới nhất của tài liệu này được lưu trữ \& có thể tải xuống ở link sau: \href{https://github.com/NQBH/hobby/blob/master/elementary_mathematics/grade_9/1st_order_2_unknown_system_of_equations/NQBH_1st_order_2_unknown_system_of_equations.pdf}{GitHub\texttt{/}NQBH\texttt{/}hobby\texttt{/}elementary mathematics\texttt{/}grade 9\texttt{/}1st-order 2-unknown system of equations}.
	
	\textsf{\textbf{Nội dung.} Hàm số bậc nhất.}
\end{abstract}
\tableofcontents
\newpage

%------------------------------------------------------------------------------%

\section{Hàm Số $y = ax^2$, $a\ne0$}

\noindent\fbox{%
	\parbox{\textwidth}{%
		\noindent\textsf{\textbf{Kiến thức cơ bản.}} Xét hàm số $y = ax^2$, $a\ne0$. \fbox{\bf 1} Hàm số $y = f(x) = ax^2$ có tập xác định $D = \mathbb{R}$. \fbox{\bf 2} Đồ thị của hàm $y = ax^2$ là 1 đường \textit{parabol} với đỉnh $O(0,0)$ \& trục đối xứng là trục tung. \fbox{\bf 3} Với $a > 0$ hàm số $y = ax^2$ nghịch biến khi $x < 0$ \& đồng biến khi $x > 0$. Giá trị nhỏ nhất của $f(x)$ bằng 0 khi $x = 0$. \fbox{\bf 4} Với $a < 0$ hàm số $y = ax^2$ đồng biến khi $x < 0$ \& nghịch biến khi $x > 0$. Giá trị lớn nhất của $f(x)$ bằng 0 khi $x = 0$.
	}%
}

%------------------------------------------------------------------------------%

\section{Đồ Thị của Hàm Số $y = ax^2$, $a\ne0$}

\begin{baitoan}[\cite{TLCT_THCS_Toan_9_dai_so}, Ví dụ 12.1, p. 65]
	Trong mặt phẳng tọa độ $(Oxy)$ cho đường thẳng $(d):y = -1$ \& điểm $F(0,1)$. Tìm tập hợp tất cả những điểm $I$ sao cho khoảng cách từ $I$ đến $(d)$ bằng $IF$.
\end{baitoan}

%------------------------------------------------------------------------------%

\section{2nd-Order 1-Unknown Equation -- Phương Trình Bậc 2 1 Ẩn}

%------------------------------------------------------------------------------%

\section{Công Thức Nghiệm của Phương Trình Bậc 2}

%------------------------------------------------------------------------------%

\section{Công Thức Nghiệm Thu Gọn}

%------------------------------------------------------------------------------%

\section{Hệ Thức Vi\`ete \& Ứng Dụng}

%------------------------------------------------------------------------------%

\section{Phương Trình Quy về Phương Trình Bậc 2}

%------------------------------------------------------------------------------%

\section{Giải Bài Toán Bằng Cách Lập Phương Trình}

%------------------------------------------------------------------------------%

\section{Miscellaneous}

%------------------------------------------------------------------------------%

\printbibliography[heading=bibintoc]
	
\end{document}