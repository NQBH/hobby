\documentclass{article}
\usepackage[backend=biber,natbib=true,style=alphabetic,maxbibnames=50]{biblatex}
\addbibresource{/home/nqbh/reference/bib.bib}
\usepackage[utf8]{vietnam}
\usepackage{tocloft}
\renewcommand{\cftsecleader}{\cftdotfill{\cftdotsep}}
\usepackage[colorlinks=true,linkcolor=blue,urlcolor=red,citecolor=magenta]{hyperref}
\usepackage{amsmath,amssymb,amsthm,float,graphicx,mathtools}
\allowdisplaybreaks
\newtheorem{assumption}{Assumption}
\newtheorem{baitoan}{Bài toán}
\newtheorem{cauhoi}{Câu hỏi}
\newtheorem{conjecture}{Conjecture}
\newtheorem{corollary}{Corollary}
\newtheorem{dangtoan}{Dạng toán}
\newtheorem{definition}{Definition}
\newtheorem{dinhly}{Định lý}
\newtheorem{dinhnghia}{Định nghĩa}
\newtheorem{example}{Example}
\newtheorem{ghichu}{Ghi chú}
\newtheorem{hequa}{Hệ quả}
\newtheorem{hypothesis}{Hypothesis}
\newtheorem{lemma}{Lemma}
\newtheorem{luuy}{Lưu ý}
\newtheorem{nhanxet}{Nhận xét}
\newtheorem{notation}{Notation}
\newtheorem{note}{Note}
\newtheorem{principle}{Principle}
\newtheorem{problem}{Problem}
\newtheorem{proposition}{Proposition}
\newtheorem{question}{Question}
\newtheorem{remark}{Remark}
\newtheorem{theorem}{Theorem}
\newtheorem{vidu}{Ví dụ}
\usepackage[left=1cm,right=1cm,top=5mm,bottom=5mm,footskip=4mm]{geometry}
\def\labelitemii{$\circ$}
\DeclareRobustCommand{\divby}{%
	\mathrel{\vbox{\baselineskip.65ex\lineskiplimit0pt\hbox{.}\hbox{.}\hbox{.}}}%
}
\title{Giải Đề Thi Toán 9 Trường Trung Học Phổ Thông Chuyên Amsterdam Hà Nội}
\author{Nguyễn Quản Bá Hồng\footnote{Independent Researcher, Ben Tre City, Vietnam\\e-mail: \texttt{nguyenquanbahong@gmail.com}; website: \url{https://nqbh.github.io}.}}
\date{\today}

\begin{document}
\maketitle
\tableofcontents

%------------------------------------------------------------------------------%

\section{Đề Kiểm Tra Toán 9 Giữa Học Kỳ I 2019--2020}

\begin{baitoan}
	Cho 2 biểu thức $A = \dfrac{\sqrt{x} + 10}{\sqrt{x}}$ \& $B = \dfrac{1}{\sqrt{x} + 2} - \dfrac{\sqrt{x}}{\sqrt{x} - 2} + \dfrac{2x - \sqrt{x} + 2}{x- 4}$ (với $x > 0$, $x\ne4$). (a) Tính giá trị của $A$ khi $x = 16$. (b) Rút gọn biểu thức $B$. (c) Tìm tất cả các giá trị của $x$ để biểu thức $P = AB$ nhận giá trị nguyên.
\end{baitoan}

\begin{proof}[Giải]
	(a) Khi $x = 16$, $A = \dfrac{\sqrt{16} + 10}{\sqrt{16}} = \dfrac{4 + 10}{4} = \dfrac{14}{4} = \dfrac{7}{2} = 3.5$. (b) Mẫu thức chung (MTC): $x - 4 = (\sqrt{x} + 2)(\sqrt{x} - 2)$.
	\begin{align*}
		B &= \frac{\sqrt{x} - 2 - \sqrt{x}(\sqrt{x} + 2) + 2x - \sqrt{x} + 2}{(\sqrt{x} + 2)(\sqrt{x} - 2)} = \frac{\sqrt{x} - 2 - x - 2\sqrt{x} + 2x - \sqrt{x} + 2}{(\sqrt{x} + 2)(\sqrt{x} - 2)}\\
		&= \frac{\sqrt{x}(1 - \sqrt{x} - 2 + 2\sqrt{x} - 1) - 2 + 2}{(\sqrt{x} + 2)(\sqrt{x} - 2)} = \frac{\sqrt{x}(\sqrt{x} - 2)}{(\sqrt{x} + 2)(\sqrt{x} - 2)} = \frac{\sqrt{x}}{\sqrt{x} + 2}.
	\end{align*}
	(c) Từ kết quả rút gọn của $B$ ở câu (b), ta có:
	\begin{align*}
		P = AB = \frac{\sqrt{x} + 10}{\sqrt{x}}\cdot\frac{\sqrt{x}}{\sqrt{x} + 2} = \frac{\sqrt{x} + 10}{\sqrt{x} + 2} = 1 + \frac{8}{\sqrt{x} + 2}.
	\end{align*}
	Có $P\in\mathbb{Z}\Leftrightarrow\dfrac{8}{\sqrt{x} + 2}\in\mathbb{Z}\Leftrightarrow8\divby(\sqrt{x} + 2)\Leftrightarrow\sqrt{x} + 2\in\mbox{Ư}(8)\cap\mathbb{Z} = \{\pm1,\pm2,\pm4,\pm8\}$, mà $\sqrt{x} + 2\ge2$, $\forall x\in\mathbb{R}$, $x\ge0$, \& quan trọng là ĐKXĐ của $A$ là $x > 0$, ĐKXĐ của $B$ là $x\ge0$ \& $x\ne4$, nên suy ra $P\in\mathbb{Z}\Leftrightarrow\sqrt{x} + 2\in\{2,4,8\}$ \& $x > 0$, $x\ne4$ $\Leftrightarrow\sqrt{x}\in\{0,2,6\}$ \& $x > 0$, $x\ne4$ $\Leftrightarrow x\in\{0,4,36\}$ \& $x > 0$, $x\ne4$ $\Leftrightarrow x = 36$. Vậy $P\in\mathbb{Z}\Leftrightarrow x = 36$.
\end{proof}

\begin{luuy}
	Có nhiều cách rút gọn $B$. Trong lời giải trên ở câu (b), ta đã đặt nhân tử chung là $\sqrt{x}$ ở tử thức để phân tích nhân tử tử thức rồi sau đó đơn giản nhân tử chung $\sqrt{x} - 2$ với mẫu thức. Ngoài cách này, ta có thể nhân phân phối vào như sau:
	\begin{align*}
		B &= \frac{\sqrt{x} - 2 - \sqrt{x}(\sqrt{x} + 2) + 2x - \sqrt{x} + 2}{(\sqrt{x} + 2)(\sqrt{x} - 2)} = \frac{\sqrt{x} - 2 - x - 2\sqrt{x} + 2x - \sqrt{x} + 2}{(\sqrt{x} + 2)(\sqrt{x} - 2)}\\
		&= \frac{(2x - x) + (\sqrt{x} - 2\sqrt{x} - \sqrt{x}) + (2 - 2)}{(\sqrt{x} + 2)(\sqrt{x} - 2)} = \frac{x - 2\sqrt{x}}{(\sqrt{x} + 2)(\sqrt{x} - 2)} = \frac{\sqrt{x}(\sqrt{x} - 2)}{(\sqrt{x} + 2)(\sqrt{x} - 2)} = \frac{\sqrt{x}}{\sqrt{x} + 2}.
	\end{align*}
\end{luuy}

\begin{baitoan}
	Giải phương trình: (a) $\sqrt{x^2 - 6x + 9} = 2x - 1$. (b) $\sqrt{2x - 3} - \sqrt{x + 1} = 0$.
\end{baitoan}

\begin{proof}[Giải]
	(a) ĐKXĐ: $x^2 - 6x  + 9 = (x - 3)^2\ge0$, $\forall x\in\mathbb{R}$ nên phương trình xác định $\forall x\in\mathbb{R}$.
	\begin{equation*}
		\sqrt{x^2 - 6x + 9} = 2x - 1\Leftrightarrow\sqrt{(x - 3)^2} = 2x - 1\Leftrightarrow|x - 3| = 2x - 1\Leftrightarrow\left[\begin{split}
			x - 3 &= 2x - 1\ge0\\
			3 - x &= 2x - 1\ge0
		\end{split}\right.\Leftrightarrow\left[\begin{split}
		x &= -2,\mbox{ loại vì }2\cdot-2 - 1 = -5 < 0\\
		x &= \frac{4}{3},\mbox{ nhận vì }2\cdot \frac{4}{3} - 1 = \frac{5}{3} > 0.
		\end{split}\right.
	\end{equation*}
	Vậy $S = \left\{\frac{4}{3}\right\}$. (b) ĐKXĐ: $2x - 3\ge0$ \& $x + 1\ge0$ $\Leftrightarrow x\ge\frac{3}{2}$ \& $x\ge-1$ $\Leftrightarrow x\ge\frac{3}{2}$. Ta có: $\sqrt{2x - 3} - \sqrt{x + 1} = 0\Leftrightarrow\sqrt{2x - 3} = \sqrt{x + 1}\Rightarrow 2x - 3 = x + 1\Leftrightarrow x = 4$. Thử lại $x = 4$ thấy thỏa mãn: $\sqrt{2\cdot4 - 3} - \sqrt{4 + 1} = \sqrt{5} - \sqrt{5} = 0$. Vậy $S = \left\{4\right\}$.
\end{proof}

\begin{baitoan}
	Cho $\Delta ABC$ vuông tại $A$, $AB < AC$, đường cao $AH$. Các đường phân giác của $BAH$ \& $CAH$, tương ứng cắt cạnh $BC$ tại $M,N$. Gọi $K$ là trung điểm $AM$. (a) Chứng minh $\Delta AMC$ là 1 tam giác cân. (b) Dựng $KI\bot BC$ tại $I$. Chứng minh $MK^2 = MI\cdot MC$ \& $MA^2 = 2MH\cdot MC$. (c) Chứng minh $\dfrac{1}{AH^2} = \dfrac{1}{AM^2} + \dfrac{1}{4CK^2}$.
\end{baitoan}

\begin{proof}[Giải]
	
\end{proof}

\begin{baitoan}
	(a) Cho $a,b,c$ là các số thực không âm thỏa mãn $a + b + c = 3$. Tìm giá trị lớn nhất \& giá trị nhỏ nhất của biểu thức $P = a^4 + b^4 + c^4 - 3abc$. (b) Tìm giá trị lớn nhất \& giá trị nhỏ nhất của biểu thức $P = \sqrt{x - 1} + \sqrt{3 - x}$.
\end{baitoan}

\begin{proof}[Giải]
	
\end{proof}

%------------------------------------------------------------------------------%

\section{Đề Kiểm Tra Toán 9 Giữa Học Kỳ I 2020--2021}

\begin{baitoan}
	Cho 2 biểu thức $A = \dfrac{3(\sqrt{x}  - 2)}{x + 2}$ \& $B = \dfrac{\sqrt{x} - 1}{\sqrt{x} + 2} + \dfrac{5\sqrt{x} - 2}{x - 4}$ với $x\ge0$ \& $x\ne4$. (a) Chứng minh $B = \dfrac{\sqrt{x}}{\sqrt{x} - 2}$. (b) Tìm tất cả các giá trị của $x$ để $B < 0$. (c) Tìm các số thực $x$ sao cho $AB$ nhận giá trị là số nguyên.
\end{baitoan}

\begin{proof}[Giải]
	
\end{proof}

\begin{baitoan}
	Giải phương trình $\sqrt{x^2 - 2x - 1} - \sqrt{2x - 4} = 0$.
\end{baitoan}

\begin{proof}[Giải]
	
\end{proof}

\begin{baitoan}
	(a) Chiều dài của 1 cái bập bênh là {\rm5.2 m}, khi 1 đầu của cái bập bênh chậm đất thì cái bập bênh tạo với mặt đất 1 góc $23^\circ$. Hỏi đầu còn lại của các bập bênh cách mặt đất bao nhiêu {\rm m}? Biết mặt đất phẳng, kết quả làm tròn $2$ chữ số sau dấu phẩy. (b) Cho $\Delta ABC$ vuông tại $A$, $AB < AC$, đường cao $AH$. (a) Cho $AB = 5${\rm cm}, $AC = 12${\rm cm}. Tính tỷ số $\frac{BH}{CH}$. (b) Kẻ $HE, HF$ lần lượt vuông góc với $AB,AC$ tại $E,F$. Chứng minh $EF$ là tiếp tuyến của đường tròn đường kính $HC$. (c) Gọi $O$ là trung điểm của $HC$ \& $d$ là tiếp tuyến tại $C$ của đường tròn đường kính $HC$. Đường thẳng đi qua $H$, vuông góc với $AO$ \& cắt $d$ tại $D$. Chứng minh $\Delta HAC\backsim\Delta COD$.
\end{baitoan}

\begin{proof}[Giải]
	
\end{proof}

\begin{baitoan}
	Cho $x,y$ là các số thực không âm thỏa mãn $x + y = 2020$. Tìm giá trị lớn nhất \& giá trị nhỏ nhất của biểu thức $P = \sqrt{x} + 2\sqrt{y}$.
\end{baitoan}

\begin{proof}[Giải]
	
\end{proof}

%------------------------------------------------------------------------------%

\printbibliography[heading=bibintoc]

\end{document}