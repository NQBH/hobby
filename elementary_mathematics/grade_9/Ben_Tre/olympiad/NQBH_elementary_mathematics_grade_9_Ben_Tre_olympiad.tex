\documentclass{article}
\usepackage[backend=biber,natbib=true,style=alphabetic,maxbibnames=10]{biblatex}
\addbibresource{/home/nqbh/reference/bib.bib}
\usepackage[utf8]{vietnam}
\usepackage{tocloft}
\renewcommand{\cftsecleader}{\cftdotfill{\cftdotsep}}
\usepackage[colorlinks=true,linkcolor=blue,urlcolor=red,citecolor=magenta]{hyperref}
\usepackage{amsmath,amssymb,amsthm,caption,float,graphicx,mathtools,subcaption}
\allowdisplaybreaks
\newtheorem{assumption}{Assumption}
\newtheorem{baitoan}{Bài toán}
\newtheorem{cauhoi}{Câu hỏi}
\newtheorem{conjecture}{Conjecture}
\newtheorem{corollary}{Corollary}
\newtheorem{dangtoan}{Dạng toán}
\newtheorem{definition}{Definition}
\newtheorem{dinhly}{Định lý}
\newtheorem{dinhnghia}{Định nghĩa}
\newtheorem{example}{Example}
\newtheorem{ghichu}{Ghi chú}
\newtheorem{hequa}{Hệ quả}
\newtheorem{hypothesis}{Hypothesis}
\newtheorem{lemma}{Lemma}
\newtheorem{luuy}{Lưu ý}
\newtheorem{nhanxet}{Nhận xét}
\newtheorem{notation}{Notation}
\newtheorem{note}{Note}
\newtheorem{principle}{Principle}
\newtheorem{problem}{Problem}
\newtheorem{proposition}{Proposition}
\newtheorem{question}{Question}
\newtheorem{remark}{Remark}
\newtheorem{theorem}{Theorem}
\newtheorem{vidu}{Ví dụ}
\usepackage[left=1cm,right=1cm,top=5mm,bottom=5mm,footskip=4mm]{geometry}
\def\labelitemii{$\circ$}
\DeclareRobustCommand{\divby}{%
	\mathrel{\vbox{\baselineskip.65ex\lineskiplimit0pt\hbox{.}\hbox{.}\hbox{.}}}%
}

\title{Bến Tre Mathematical Olympiad}
\author{Nguyễn Quản Bá Hồng\footnote{Independent Researcher, Ben Tre City, Vietnam\\e-mail: \texttt{nguyenquanbahong@gmail.com}; website: \url{https://nqbh.github.io}.}}
\date{\today}

\begin{document}
\maketitle
\begin{abstract}
	
\end{abstract}
\tableofcontents

%------------------------------------------------------------------------------%

\section{Châu Thành, Bến Tre 2022--2023}

%------------------------------------------------------------------------------%

\section{Thành Phố Bến Tre 2022--2023}

\begin{baitoan}
	Cho biểu thức:
	\begin{align*}
		A = \frac{a + 1}{\sqrt{a}} + \frac{a\sqrt{a}  -1}{a - \sqrt{a}} + \frac{a^2 - a\sqrt{a} + \sqrt{a} - 1}{\sqrt{a} - a\sqrt{a}}\mbox{ với } a > 0,\,a\ne1.
	\end{align*}
	(a) Chứng minh $A > 4$. (b) Tìm các giá trị của $a$ để biểu thức $\frac{6}{A}$ nhận giá trị nguyên.
\end{baitoan}

\begin{baitoan}
	Phân tích đa thức thành nhân tử: $x^2(y - 2z) + y^2(z - x) + 2z^2(x - y) + xyz$.
\end{baitoan}

\begin{baitoan}
	(a) Giải phương trình: $(x^2 - 4x + 11)(x^4 - 8x^2 + 21) = 35$. (b) Giải hệ phương trình:
	\begin{equation*}
		\left\{\begin{split}
			x^2 + y^2 + 2x + 2y &= (x + 2)(y + 2),\\
			\left(\frac{x}{y + 2}\right)^2 + \left(\frac{y}{x + 2}\right)^2 &= 1.
		\end{split}\right.
	\end{equation*}
\end{baitoan}

\begin{baitoan}
	(a) Tìm tất cả các số tự nhiên $n$ để $B = \dfrac{n(n + 1)(n + 2)}{6} + 1$ là số nguyên tố. (b) Tìm giá trị nhỏ nhất của biểu thức
	\begin{align*}
		C = \frac{\sqrt{x + 6\sqrt{x - 9}} + \sqrt{x - 6\sqrt{x - 9}}}{\sqrt{\dfrac{81}{x^2} - \dfrac{18}{x} + 1}}\mbox{ với } x > 9.
	\end{align*}
\end{baitoan}

\begin{baitoan}
	Cho $\Delta ABC$ nhọn, 3 đường cao $AK,BD,CE$ cắt nhau tại $H$. (a) Chứng minh $BH\cdot BD = BC\cdot BK$ \& $BH\cdot BD + CH\cdot CE = BC^2$. (b) Chứng minh: $BH = AC\cot\widehat{ABC}$. (c) Gọi $M$ là trung điểm của $BC$. Đường thẳng qua $A$ vuông góc với $AM$ cắt các đường thẳng $BD,CE$ lần lượt tại $Q,P$. Chứng minh $MP = MQ$.
\end{baitoan}

%------------------------------------------------------------------------------%

\section{Bến Tre 2022--2023}

\begin{baitoan}
	(a) Tính giá trị biểu thức: $A = \sqrt{4 + \sqrt{15}} + \sqrt{4 - \sqrt{15}} - 2\sqrt{3 - \sqrt{5}}$. (b) Rút gọn biểu thức:
	\begin{align*}
		B = \frac{x - 5 + 2\sqrt{x + 6\sqrt{x} + 9}}{x + 3\sqrt{x} + 2},\ x\ge0,
	\end{align*}
	\& tìm $x$ sao cho $B = \dfrac{2022}{2023}$.
\end{baitoan}

\begin{baitoan}
	Tìm tất cả các cặp số nguyên $(x,y)$ thỏa $\dfrac{x^2 + y^2}{x + y} = \dfrac{85}{13}$.
\end{baitoan}

\begin{baitoan}
	Giải phương trình: $9\left(\dfrac{x - 2}{x + 1}\right)^2 + \left(\dfrac{x + 2}{x - 1}\right)^2 - 10\left(\dfrac{x^2 - 4}{x^2  - 1}\right) = 0$.
\end{baitoan}

\begin{baitoan}
	Cho $a,b,c$ là các số thực không âm. Chứng minh:
	\begin{align*}
		a\sqrt{3a^2 + 6b^2} + b\sqrt{3b^2 + 6c^2} + c\sqrt{3c^2 + 6a^2}\ge(a + b + c)^2.
	\end{align*}
\end{baitoan}

\begin{baitoan}
	Cho $\Delta ABC$ biết $\widehat{ACB} = 45^\circ$, gọi $O$ là tâm đường tròn ngoại tiếp $\Delta ABC$ \& $H$ là trực tâm của $\Delta ABC$. Đường thẳng qua $O$ \& vuông góc với $CO$ cắt $AC$ \& $BC$ lần lượt tại điểm $K$ \& điểm $L$. Chứng minh: chu vi $\Delta HKL$ bằng với đường kính của $(O)$.
\end{baitoan}

\begin{baitoan}
	Cho 2 đường tròn $(O_1),(O_2)$ tiếp xúc ngoài nhau tại điểm $T$. 2 đường tròn này nằm trong đường tròn $(O_3)$ \& tiếp xúc với $(O_3)$ lần lượt tại điểm $M\in(O_1)$ \& điểm $N\in(O_2)$. Tiếp tuyến chung tại $T$ của $(O_1),(O_2)$ cắt $(O_3)$ tại điểm $P$ ($P$ \& $O_3$ nằm cùng phía của đường thẳng $MN$). Đường thẳng $PM$ cắt $(O_1)$ tại $A\ne M$, đường thẳng $PN$ cắt $(O_2)$ tại $D\ne N$, \& đường thẳng $MN$ cắt $(O_1)$ \& $(O_2)$ lần lượt tại $B\ne M$ \& $C\ne N$. Gọi $E$ là giao điểm của $AB$ \& $CD$. (a) Tứ giác $AEDP$ là hình gì? Giải thích. (b) Chứng minh $\widehat{EBC} = \widehat{EDA}$.
\end{baitoan}

%------------------------------------------------------------------------------%

\printbibliography[heading=bibintoc]

\end{document}