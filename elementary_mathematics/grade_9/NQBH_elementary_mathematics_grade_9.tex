\documentclass{article}
\usepackage[backend=biber,natbib=true,style=authoryear]{biblatex}
\addbibresource{/home/hong/1_NQBH/reference/bib.bib}
\usepackage[utf8]{vietnam}
\usepackage{tocloft}
\renewcommand{\cftsecleader}{\cftdotfill{\cftdotsep}}
\usepackage[colorlinks=true,linkcolor=blue,urlcolor=red,citecolor=magenta]{hyperref}
\usepackage{amsmath,amssymb,amsthm,mathtools,float,graphicx,algpseudocode,algorithm,tcolorbox,enumitem}
\allowdisplaybreaks
\numberwithin{equation}{section}
\newtheorem{assumption}{Assumption}[section]
\newtheorem{conjecture}{Conjecture}[section]
\newtheorem{corollary}{Corollary}[section]
\newtheorem{definition}{Definition}[section]
\newtheorem{example}{Example}[section]
\newtheorem{lemma}{Lemma}[section]
\newtheorem{notation}{Notation}[section]
\newtheorem{principle}{Principle}[section]
\newtheorem{problem}{Problem}[section]
\newtheorem{proposition}{Proposition}[section]
\newtheorem{question}{Question}[section]
\newtheorem{remark}{Remark}[section]
\newtheorem{theorem}{Theorem}[section]
\usepackage[left=0.5in,right=0.5in,top=1.5cm,bottom=1.5cm]{geometry}
\usepackage{fancyhdr}
\pagestyle{fancy}
\fancyhf{}
\lhead{\small \textsc{Sect.} ~\thesection}
\rhead{\small \nouppercase{\leftmark}}
\renewcommand{\sectionmark}[1]{\markboth{#1}{}}
\cfoot{\thepage}
\def\labelitemii{$\circ$}

\title{Some Topics in Elementary Mathematics\texttt{/}Grade 9}
\author{Nguyễn Quản Bá Hồng\footnote{Independent Researcher, Ben Tre City, Vietnam\\e-mail: \texttt{nguyenquanbahong@gmail.com}}}
\date{\today}

\begin{document}
\maketitle
\begin{abstract}
	Tóm tắt kiến thức Toán lớp 9 theo chương trình giáo dục của Việt Nam \textit{\&} một số chủ đề nâng cao. Phiên bản mới nhất của tài liệu này được lưu trữ ở link sau: \href{https://github.com/NQBH/hobby/blob/master/elementary_mathematics/grade_9/NQBH_elementary_mathematics_grade_9.pdf}{GitHub\texttt{/}NQBH\texttt{/}hobby\texttt{/}elementary mathematics\texttt{/}grade 9}\footnote{Explicitly, \url{https://github.com/NQBH/hobby/blob/master/elementary_mathematics/grade_9/NQBH_elementary_mathematics_grade_9.pdf}.}.
\end{abstract}
\setcounter{secnumdepth}{4}
\setcounter{tocdepth}{3}
\tableofcontents

\section{Căn Bậc 2, Căn Bậc 3}

\subsection{Căn Bậc 2}
Phép toán ngược của phép bình phương là phép toán lấy căn bậc 2.

\subsubsection{Căn bậc 2 số học}
Căn bậc 2 của 1 số $a\ge 0$ là số $x\in\mathbb{R}$ sao cho $x^2 = a$. Số dương $a > 0$ có đúng 2 căn bậc 2 là 2 số đối nhau: số dương ký hiệu là $\sqrt{a}$ \& số âm ký hiệu là $-\sqrt{a}$, viết chung là $\pm\sqrt{a}$. Số 0 có đúng 1 căn bậc 2 là chính số 0, viết $\sqrt{0} = 0$.

\begin{definition}[Căn bậc 2 số học]
	Với số dương $a$, số $\sqrt{a}$ được gọi là \emph{căn bậc 2 số học} của $a$. Số 0 cũng được gọi là \emph{căn bậc 2 số học của 0}.
\end{definition}
``Với $a\ge 0$, ta có: Nếu $x = \sqrt{a}$ thì $x\ge 0$ \& $x^2 = a$. Nếu $x\ge 0$ \& $x^2 = a$ thì $x = \sqrt{a}$.'' -- \cite[p. 4]{SGK_Toan_9_tap_1}
\begin{equation*}
	x = \sqrt{a}\Leftrightarrow\left\{\begin{split}
		x&\ge 0,\\
		x^2 &= a.
	\end{split}\right.
\end{equation*}
``Phép toán tìm căn bậc 2 số học của số không âm gọi là \textit{phép khai phương} (gọi tắt là \textit{khai phương}). Để khai phương 1 số, người ta có thể dùng máy tính bỏ túy hoặc dùng bảng số. Khi biết căn bậc 2 số học của 1 số, ta dễ dàng xác định được các căn bậc 2 của nó.'' -- \cite[p. 5]{SGK_Toan_9_tap_1}

\subsubsection{So sánh các căn bậc 2 số học}
\begin{theorem}[So sánh các căn bậc 2 số học]
	Với 2 số $a,b\ge 0$, $a < b\Leftrightarrow\sqrt{a} < \sqrt{b}$.
\end{theorem}
``Từ thời xa xưa, người ta đã thấy giữa Hình học \& Đại số có mối liên quan mật thiết. Khái niệm căn bậc 2 cũng có phần xuất phát từ Hình học. Khi biết độ dài cạnh hình vuông, ta tính được diện tích hình đó bằng cách bình phương (hoặc nâng lên lũy thừa bậc 2) độ dài cạnh. Ngược lại, nếu biết diện tích hình vuông, ta tìm được độ dài cạnh của nó nhờ khai phương số đo diện tích. Người ta coi phép lấy căn bậc 2 số học là phép toán ngược của phép bình phương \& coi việc tìm căn 1 số là tìm ``cái gốc, cái nguồn''. Điều này hiện còn thấy trong ngôn ngữ 1 số nước. E.g., ở tiếng Anh, từ \textit{square}\footnote{\textbf{square} [n] \textbf{1.} a shape with 4 straight sides of equal length \& 4 angles of $90^\circ$; a piece of something that has this shape; \textbf{2.} an open area in a town, usually with 4 sides, surrounded by buildings; \textbf{3.} \textbf{square of something} the number obtained when you multiply a number by itself; [a] \textbf{1.} having the shape or approximate shape of a square; \textbf{2.} having angles of $90^\circ$ exactly or approximately; \textbf{3.} \textbf{square kilometer\texttt{/}mile\texttt{/}meter, etc.} used to refer to a unit of measurement equal to the area of a square whose side is of the unit mentioned; \textbf{4.} (abbr., \textbf{sq}) \textbf{3 meters\texttt{/}6 feet\texttt{/}25 miles, etc. square} used to give the size of a square object or space, by giving the length of each of its sides; [v] \textbf{1.} [transitive, usually passive] \textbf{square something} to multiply a number by itself; \textbf{2.} [transitive, intransitive] to make 2 ideas, facts or situations agree or combine well with each other; to agree or be consistent with another idea, fact or situation, \textsc{synonym}: \textbf{reconcile}; \textbf{3.} [transitive] \textbf{square something (off)} to make something have straight edges \& corners.} có nghĩa là \textit{hình vuông} \& cũng có nghĩa là \textit{bình phương}, từ \textit{root}\footnote{\textbf{root} [n] \textbf{1.} [countable] the part of a plant that grows under the ground \& absorbs water \& minerals that it sends to the rest of the plant; \textbf{2.} [countable, usually singular] \textbf{root of something} the main cause of something, such as a problem or difficult situation; \textbf{3.} [countable, usually plural] the basis of something; \textbf{4.} (\textbf{roots}) [plural] the feelings or connections that you have with a place because you have lived there or your family came from there; \textbf{5.} [countable] (\textit{linguistics}) the part of a word that has the main meaning \& that its other forms are based on; a word that other words are formed from; \textbf{6.} [countable] \textbf{root (of something)} (\textit{mathematics}) a quantity which, when multiplied by itself a particular number of times, produces another quantity; \textbf{root \& branch} [idiom] thorough \& complete; \textbf{take root} [idiom] \textbf{1.} (of a plant) to develop roots; \textbf{2.} (of an idea) to become widely accepted; [v] [intransitive] (of a plant) to grow roots; \textbf{root something\texttt{/}somebody out} [phrasal verb] to find a person or thing that is causing a problem \& remove or get rid of them.} có nghĩa là rễ, là nguồn gốc, còn từ \textit{square root}\footnote{\textbf{square root} [n] \textbf{square root (of something)} (\textit{mathematics}) a number which, when multiplied by itself, produces a particular number.} là \textit{căn bậc 2}.'' -- \cite[p. 7]{SGK_Toan_9_tap_1}

\subsection{Căn Thức Bậc 2 \& Hằng Đẳng Thức $\sqrt{A^2} = |A|$}

\subsubsection{Căn thức bậc 2}
\begin{definition}[Căn thức bậc 2]
	Với $A$ là 1 biểu thức đại số, người ta gọi $\sqrt{A}$ là \emph{căn thức bậc 2} của $A$, còn $A$ được gọi là \emph{biểu thức lấy căn} hay \emph{biểu thức dưới dấu căn}. $\sqrt{A}$ xác định (hay có nghĩa) khi $A$ lấy giá trị không âm.
\end{definition}

\subsubsection{Hằng đẳng thức $\sqrt{A^2} = |A|$}
\begin{theorem}
	Với mọi $a\in\mathbb{R}$, ta có $\sqrt{a^2} = |a|$.
\end{theorem}

\begin{proof}
	Theo định nghĩa giá trị tuyệt đối thì $|a|\ge 0$. Nếu $a\ge 0$ thì $|a| = a$, nên $(|a|)^2 = a^2$. Nếu $a < 0$ thì $|a| = -a$, nên $(|a|)^2 = (-a)^2 = a^2$. Do đó, $(|a|)^2 = a^2$, $\forall a\in\mathbb{R}$. Vậy $|a|$ chính là căn bậc 2 số học của $a^2$, i.e., $\sqrt{a^2} = |a|$.
\end{proof}
``1 cách tổng quát, với $A$ là 1 biểu thức ta có $\sqrt{A^2} = |A|$, i.e., $\sqrt{A^2} = A$ nếu $A\ge 0$ (i.e., $A$ lấy giá trị không âm); $\sqrt{A^2} = -A$ nếu $A < 0$ (i.e., $A$ lấy giá trị âm).'' -- \cite[p. 10]{SGK_Toan_9_tap_1}

\subsection{Liên Hệ Giữa Phép Nhân \& Phép Khai Phương}
\begin{theorem}
	\label{theorem:sqrt & product}
	Với mọi $a,b\ge 0$, $\sqrt{ab} = \sqrt{a}\sqrt{b}$.
\end{theorem}

\begin{proof}
	Vì $a\ge 0$, $b\ge 0$ nên $\sqrt{a}\sqrt{b}$ xác định \& không âm. Ta có $(\sqrt{a}\sqrt{b})^2 = (\sqrt{a})^2(\sqrt{b})^2 = ab$. Vậy $\sqrt{a}\sqrt{b}$ là căn bậc 2 số học của $ab$, i.e., $\sqrt{ab} = \sqrt{a}\sqrt{b}$.
\end{proof}

\begin{remark}
	Định lý trên có thể mở rộng cho tích của nhiều số không âm, i.e.,
	\begin{align*}
		\sqrt{a_1\cdots a_n} = \sqrt{a_1}\cdots\sqrt{a_n},\ \forall a_1,\ldots,a_n\ge 0,
	\end{align*}
	or more briefly,
	\begin{align*}
		\sqrt{\prod_{i=1}^n a_i} = \prod_{i=1}^n \sqrt{a_i},\ \forall a_1,\ldots,a_n\ge 0.
	\end{align*}
	Chứng minh dễ dàng bằng phương pháp quy nạp toán học với base case là Định lý \ref{theorem:sqrt & product} cho trường hợp $n = 2$.
\end{remark}

\begin{proposition}[Quy tắc khai phương 1 tích]
	Muốn khai phương 1 tích của các số không âm, ta có thể khai phương từng thừa số rồi nhân các kết quả với nhau.
\end{proposition}

\begin{proposition}[Quy tắc nhân các căn bậc 2]
	Muốn nhân các căn bậc 2 của các số không âm, ta có thể nhân các số dưới dấu căn với nhau rồi khai phương kết quả đó.
\end{proposition}

\begin{remark}
	``1 cách tổng quát, với 2 biểu thức $A$ \& $B$ không âm, ta có $\sqrt{AB} = \sqrt{A}\sqrt{B}$. Đặc biệt, với biểu thức $A$ không âm ta có $(\sqrt{A})^2 = \sqrt{A^2} = A$.'' -- \cite[p. 14]{SGK_Toan_9_tap_1}
\end{remark}

\subsection{Liên Hệ Giữa Phép Chia \& Phép Khai Phương}
\begin{theorem}
	Với số $a\ge 0$ \& số $b > 0$, ta có
	\begin{align*}
		\sqrt{\frac{a}{b}} = \frac{\sqrt{a}}{\sqrt{b}}.
	\end{align*}
\end{theorem}

\begin{proof}
	Vì $a\ge 0$ \& $b > 0$ nên $\frac{\sqrt{a}}{\sqrt{b}}$ xác định \& không âm. Ta có $\left(\frac{\sqrt{a}}{\sqrt{b}}\right)^2 = \frac{(\sqrt{a})^2}{(\sqrt{b})^2} = \frac{a}{b}$. Vậy $\frac{\sqrt{a}}{\sqrt{b}}$ là căn bậc 2 số học của $\frac{a}{b}$, i.e., $\sqrt{\frac{a}{b}} = \frac{\sqrt{a}}{\sqrt{b}}$.
\end{proof}

\begin{proposition}[Quy tắc khai phương 1 thương]
	Muốn khai phương 1 thương $\frac{a}{b}$, trong đó $a\ge 0$ \& $b > 0$, ta có thể lần lượt khai phương số $a$ \& số $b$, rồi lấy kết quả thứ nhất chia cho kết quả thứ 2.
\end{proposition}

\begin{proposition}[Quy tắc chia 2 căn bậc 2]
	Muốn chia căn bậc 2 của số $a\ge 0$ cho căn bậc 2 của số $b > 0$, ta có thể chia số $a$ cho số $b$ rồi khai phương kết quả đó.
\end{proposition}

\begin{remark}
	``1 cách tổng quát, với biểu thức $A\ge 0$ \& biểu thức $B > 0$, ta có $\sqrt{\frac{A}{B}} = \frac{\sqrt{A}}{\sqrt{B}}$.'' -- \cite[p. 18]{SGK_Toan_9_tap_1}
\end{remark}

\subsection{Bảng Căn Bậc 2}
``\textit{1 công cụ tiện lợi để khai phương khi không có máy tính}. Để tìm căn bậc 2 của 1 số dương, người ta có thể sử dụng bảng tính sẵn các căn bậc 2. Trong cuốn ``Bảng số với 4 chữ số thập phân'' của V.M. Bradisor, bảng căn bậc 2 là bảng IV dùng để khai căn bậc 2 của bất cứ số dương nào có nhiều nhất 4 chữ số.'' -- \cite[p. 20]{SGK_Toan_9_tap_1}

\subsubsection{Giới thiệu bảng}
``Bảng căn bậc 2 được chia thành các hàng \& các cột. Ta quy ước gọi tên của các hàng (cột) theo số được ghi ở cột đầu tiên (hàng đầu tiên) của mỗi trang. Căn bậc 2 của các số được viết bởi không qua 3 chữ số từ 1.00 đến 99.9 được ghi sẵn trong bảng ở các cột từ cột \textbf{0}--cột \textbf{9}. Tiếp đó là 9 cột hiệu chính được dùng để hiệu chính chữ số cuối của căn bậc 2 của các số được viết bởi 4 chữ số từ 1.000--99.99.'' -- \cite[pp. 20--21]{SGK_Toan_9_tap_1}

\subsubsection{Cách dùng bảng}

\begin{itemize}
	\item \textbf{Tìm căn bậc 2 của số $\in(1,100)$.} ``Bảng tính sẵn căn bậc 2 của tác giả V.M. Bradisor chỉ cho phép ta tìm trực tiếp căn bậc 2 của số $\in(1,100)$. Tuy nhiên, dựa vào tính chất của căn bậc 2, ta vẫn dùng bảng này để tìm được căn bậc 2 của số không âm $> 100$ hoặc $< 1$.'' -- \cite[p. 21]{SGK_Toan_9_tap_1}
	\item \textbf{Tìm căn bậc 2 của số $\in(100,+\infty)$.} Tách thành tích với $10^{2n}$ với $n\in\mathbb{N}^\star$ thích hợp.
	\item \textbf{Tìm căn bậc 2 của số $\in[0,1)$.} Tách thành thương với $10^{2n}$ với $n\in\mathbb{N}^\star$ thích hợp.
\end{itemize}

\begin{remark}
	``Để thực hành nhanh, khi tìm căn bậc 2 của số không âm $\in(100,+\infty)\cup[0,1)$, ta dùng hướng dẫn của bảng: ``Khi dời dấu phẩy trong số $N$ đi $2,4,6,\ldots$ chữ số thì phải dời dấy phẩy theo cùng chiều trong số $\sqrt{N}$ đi $1,2,3,\ldots$ chữ số.''  -- \cite[p. 22]{SGK_Toan_9_tap_1}
\end{remark}
``Thời xa xưa, con người làm tính bằng cách đếm ngón tay, ngón chân rồi đến đốt ngón tay, đốt ngón chân; khi gặp các số lớn hơn, người ta dùng hòn sỏi, hạt cây. Sau đó, họ làm ra các bàn tính gảy (có thể bắt đầu do ghép xâu các hạt cây lại). Dùng bàn tính gảy, người ta có thể tính toán được với cả các số thập phân. Hiện nay, bàn tính gảy vẫn còn được sử dụng ngay cả ở các nước rất sẵn máy tính bỏ túi.

Sự phát triển của khoa học, kỹ thuật \& nhu cầu thương mại đã đòi hỏi phải đặt ra các bảng tính sẵn. Các nhà thiên văn học, toán học Copenich (Ba Lan), Kepler (Đức), Nepe (Anh) là những người đầu tiên xây dựng kỹ thuật tính toán \& đã lập ra nhiều bảng tính sẵn. Bảng số với 4 chữ số thập phân là 1 dạng bảng tính sẵn như thế.

Ngày nay, những chiếc máy tính bỏ túi gọn nhẹ không chỉ thay thế các bảng tính sẵn để tính 1 cách nhanh chóng mà còn có độ chính xác cao hơn. Tuy nhiên, cũng như các bàn tính gảy, các bảng tính sẵn vẫn có những ưu thế riêng nên người ta vẫn tiếp tục dùng chúng. Mạnh hơn những chiếc máy tính bỏ túi \& cũng dễ dàng mang theo bên người là những chiếc máy tính xách tay.'' -- \cite[pp. 23--24]{SGK_Toan_9_tap_1}

\subsection{Biến Đổi Đơn Giản Biểu Thức Chứa Căn Thức Bậc 2}

\subsubsection{Đưa thừa số ra ngoài dấu căn}
Đẳng thức $\sqrt{a^2b} = a\sqrt{b}$, $\forall a\ge,\,b\ge 0$ cho phép thực hiện phép \textit{đưa thừa số ra ngoài dấu căn}. Đôi khi, ta phải biến đổi biểu thức dưới dấu căn về dạng thích hợp rồi mới thực hiện được phép đưa thừa số ra ngoài dấu căn (tìm \& tách ra các bình phương 1 cách thích hợp). Có thể sử dụng phép đưa thừa số ra ngoài dấu căn để rút gọn biểu thức chứa căn thức bậc 2.

Các biểu thức dạng $m\sqrt{n}$ với $m\in M\subset\mathbb{N}^\star$ và $n\in\mathbb{N}^\star$ được gọi là \textit{đồng dạng} với nhau.

\begin{proposition}
	Với 2 biểu thuwc $A,B$ mà $B\ge 0$, ta có $\sqrt{A^2B} = |A|\sqrt{B}$, i.e.: Nếu $A\ge 0$ \& $B\ge 0$ thì $\sqrt{A^2B} = A\sqrt{B}$. Nếu $A < 0$ \& $B\ge 0$ thì $\sqrt{A^2B} = -A\sqrt{B}$.
\end{proposition}

\subsubsection{Đưa thừa số vào trong dấu căn}
``Phép đưa thừa số ra ngoài dấu căn có phép biến đổi ngược với nó là phép \textit{đưa thừa số vào trong dấu căn}.

\begin{proposition}
	Với $A\ge 0$ \& $B\ge 0$ ta có $A\sqrt{B} = \sqrt{A^2B}$. Với $A < 0$ \& $B\ge 0$ ta có $A\sqrt{B} = -\sqrt{A^2B}$.
\end{proposition}
Có thể sử dụng phép đưa thừa số vào trong (hoặc ra ngoài) dấu căn để so sánh các căn bậc 2.'' -- \cite[p. 26]{SGK_Toan_9_tap_1}

\subsubsection{Khử mẫu của biểu thức lấy căn}
``Khi biến đổi biểu thức chứa căn thức bậc 2, người ta có thể sử dụng phép khử mẫu của biểu thức lấy căn.''

\begin{proposition}
	Với các biểu thức $A,B$ mà $AB\ge 0$ \& $B\ne 0$, ta có $\sqrt{\frac{A}{B}} = \frac{\sqrt{AB}}{|B|}$.
\end{proposition}

\subsubsection{Trục căn thức ở mẫu}
``Trục căn thức ở mẫu cũng là 1 phép biến đổi đơn giản thường gặp.'' -- \cite[p. 28]{SGK_Toan_9_tap_1}. Ta gọi biểu thức $\sqrt{a}\pm\sqrt{b}$ là \textit{2 biểu thức liên hợp với nhau}.

\begin{proposition}
	\begin{itemize}
		\item[(a)] Với các biểu thức $A,B$ mà $B > 0$, ta có $\frac{A}{\sqrt{B}} = \frac{A\sqrt{B}}{B}$.
		\item[(b)] Với các biểu thức $A,B,C$ mà $A\ge 0$ \& $A\ne B^2$, ta có
		\begin{align*}
			\frac{C}{\sqrt{A}\pm B} = \frac{C(\sqrt{A}\mp B)}{A - B^2}.
		\end{align*}
		\item[(c)] Với các biểu thức $A,B,C$ mà $A\ge 0$, $B\ge 0$, \& $A\ne B$, ta có
		\begin{align*}
			\frac{C}{\sqrt{A}\pm\sqrt{B}} = \frac{C(\sqrt{A}\mp\sqrt{B})}{A - B}.
		\end{align*}
	\end{itemize}
\end{proposition}

\subsection{Rút Gọn Biểu Thức Chứa Căn Thức Bậc 2}
``Để rút gọn biểu thức có chứa căn thức bậc 2, ta cần biết vận dụng thích hợp các phép tính \& các phép biến đổi đã biết.'' [$\ldots$] ``Rút gọn biểu thức được áp dụng trong nhiều bài toán về biểu thức có chứa căn thức bậc 2.'' -- \cite[p. 31]{SGK_Toan_9_tap_1}

\begin{problem}[\cite{SGK_Toan_9_tap_1}, p. 31]
	Chứng minh đẳng thức
	\begin{align*}
		\frac{a\sqrt{a} + b\sqrt{b}}{\sqrt{a} + \sqrt{b}} - \sqrt{ab} = \left(\sqrt{a} - \sqrt{b}\right)^2,\ \forall a > 0,\,b > 0.
	\end{align*}
\end{problem}

\texttt{insert MATLAB symbolic}

\subsection{Căn Bậc 3}

\subsubsection{Khái niệm căn bậc 3}
\begin{definition}[Căn bậc 3]
	\emph{Căn bậc 3} của 1 số $a$ là số $x$ sao cho $x^3 = a$.
\end{definition}
``Ta công nhận kết quả sau: \textit{Mỗi số $a\in\mathbb{R}$ đều có duy nhất 1 căn bậc 3.} Căn bậc 3 của số $a$ được ký hiệu là $\sqrt[3]{a}$. Số 3 gọi là chỉ số của căn. Phép tìm căn bậc 3 của 1 số gọi là phép khai căn bậc 3. Từ định nghĩa căn bậc 3, ta có $\left(\sqrt[3]{a}\right)^3 = \sqrt[3]{a^3} = a$. Căn bậc 3 của số dương là số dương. Căn bậc 3 của số âm là số âm. Căn bậc 3 của số 0 là chính số 0.'' -- \cite[p. 35]{SGK_Toan_9_tap_1}

\subsubsection{Tính chất}
``Tương tự tính chất của căn bậc 2, ta có các tính chất sau của căn bậc 3:

\begin{proposition}
	\begin{itemize}
		\item[(a)] $a < b\Leftrightarrow\sqrt[3]{a} < \sqrt[3]{b}$, $\forall a,b\in\mathbb{R}$.
		\item[(b)] $\sqrt[3]{ab} = \sqrt[3]{a}\sqrt[3]{b}$, $\forall a,b\in\mathbb{R}$.
		\item[(c)] $\sqrt[3]{\dfrac{a}{b}} = \dfrac{\sqrt[3]{a}}{\sqrt[3]{b}}$, $\forall a,b\in\mathbb{R}$, $b\ne 0$.
	\end{itemize}
\end{proposition}
Dựa vào các tính chất trên, ta có thể so sánh, tính toán, biến đổi các biểu thức chứa căn bậc 3.'' -- \cite[p. 35]{SGK_Toan_9_tap_1}

\subsubsection{Tìm căn bậc 3 nhờ bảng số \& máy tính bỏ túi}

\paragraph{Tìm căn bậc 3 nhờ bảng số.} ``Trong \textit{Bảng số với 4 chữ số thập phân} của V.M. Bradisor không có bảng tính sẵn căn bậc 3, nhưng ta có thể dùng bảng lập phương (bảng V) để tìm căn bậc 3 của 1 số cho trước.

\subparagraph{Giới thiệu bảng lập phương.} ``Bảng lập phương được chia thành các hàng \& các cột. Ta cũng quy ước gọi tên của các hàng (cột) theo số được ghi ở cột đầu tiên (hàng đầu tiên) của mỗi trang.

Dùng bảng lập phương ta có thể tìm được lập phương của số từ 1.000 đến 10.00. Với những số được viết bởi không quá 3 chữ số, lập phương của nó được tìm trực tiếp từ bảng. Với những số được viết bởi 4 chữ số, ta phải dùng thêm các số ở cột hiệu chính.'' [$\ldots$]

\begin{remark}
	Bảng lập phương có nêu hướng dẫn ``Khi dời dấu phẩy trong số $N$ đi 1 chữ số thì phải dời dấu phẩy trong số $N^3$ đi 3 chữ số'' nên khi tìm căn bậc 3, ta thực hành như sau: Khi dời dấu phẩy trong số $N$ đi $3,6,9,\ldots$ chữ số, ta dời dấu phẩy theo cùng chiều ở số $\sqrt[3]{N}$ đi $1,2,3,\ldots$ chữ số.'' -- \cite[pp. 36--38]{SGK_Toan_9_tap_1}
\end{remark}

\paragraph{Tìm căn bậc 3 bằng máy tính bỏ túi.} ``Có thể dùng máy tính bỏ túi có nút bấm \fbox{$\sqrt[3]{\ }$} để tìm căn bậc 3.'' -- \cite[p. 38]{SGK_Toan_9_tap_1}

\textsc{Các công thức biến đổi căn thức.}
\begin{enumerate}
	\item $\sqrt{A^2} = |A|$.
	\item $\sqrt{AB} = \sqrt{A}\sqrt{B}$, $\forall A\ge 0,\,B\ge 0$.
	\item $\sqrt{\frac{A}{B}} = \frac{\sqrt{A}}{\sqrt{B}}$, $\forall A\ge 0,\,B > 0$.
	\item $\sqrt{A^2B} = |A|\sqrt{B}$, $\forall B\ge 0$.
	\item \begin{equation*}
		A\sqrt{B} = \left\{\begin{split}
			&\sqrt{A^2B},&&A\ge 0,\,B\ge 0,\\
			-&\sqrt{A^2B},&&A < 0,\,B\ge 0.
		\end{split}\right.
	\end{equation*}
\end{enumerate}

\section{Hàm Số Bậc Nhất}

\subsection{Nhắc Lại \& Bổ Sung Các Khái Niệm Về Hàm Số}

\subsection{Hàm Số Bậc Nhất}

\subsection{Đồ Thị Của Hàm Số $y = ax + b$ ($a\ne 0$)}

\subsection{Đường Thẳng Song Song \& Đường Thẳng Cắt Nhau}

\subsection{Hệ Số Góc Của Đường Thẳng $y = ax + b$ ($a\ne 0$)}

\section{Hệ Thức Lượng Trong Tam Giác Vuông}

\subsection{1 Số Hệ Thức Về Cạnh \& Đường Cao Trong Tam Giác Vuông}

\subsection{Tỷ Số Lượng Giác Của Góc Nhọn}

\subsection{Bảng Lượng Giác}

\subsection{1 Số Hệ Thức Về Cạnh \& Góc Trong Tam Giác Vuông}

\subsection{Ứng Dụng Thực Tế Các Tỷ Số Lượng Giác Của Góc Nhọn. Thực Hành Ngoài Trời}

\section{Đường Tròn}

\subsection{Sự Xác Định Đường Tròn. Tính Chất Đối Xứng Của Đường Tròn}

\subsection{Đường Kính \& Dây Của Đường Tròn}

\subsection{Liên Hệ Giữa Dây \& Khoảng Cách từ Tâm đến Dây}

\subsection{Vị Trí Tương Đối của Đường Thẳng \& Đường Tròn}

\subsection{Dấu Hiệu Nhận Biết Tiếp Tuyến của Đường Tròn}

\subsection{Tính Chất của 2 Tiếp Tuyến Cắt Nhau}

\subsection{Vị Trí Tương Đối của 2 Đường Tròn}

%------------------------------------------------------------------------------%

\begin{thebibliography}{99}
	\bibitem[NQBH\texttt{/}elementary math]{NQBH/elementary math} Nguyễn Quản Bá Hồng. \href{https://github.com/NQBH/hobby/blob/master/elementary_mathematics/NQBH_elementary_mathematics.pdf}{\textit{Some Topics in Elementary Mathematics: Problems, Theories, Applications, \textit{\&} Bridges to Advanced Mathematics}}. Mar 2022--now.
\end{thebibliography}

%------------------------------------------------------------------------------%

\printbibliography[heading=bibintoc]
	
\end{document}