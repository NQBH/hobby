\documentclass{article}
\usepackage[backend=biber,natbib=true,style=alphabetic,maxbibnames=10]{biblatex}
\addbibresource{/home/nqbh/reference/bib.bib}
\usepackage[utf8]{vietnam}
\usepackage{tocloft}
\renewcommand{\cftsecleader}{\cftdotfill{\cftdotsep}}
\usepackage[colorlinks=true,linkcolor=blue,urlcolor=red,citecolor=magenta]{hyperref}
\usepackage{amsmath,amssymb,amsthm,float,graphicx,mathtools}
\allowdisplaybreaks
\newtheorem{assumption}{Assumption}
\newtheorem{baitoan}{}
\newtheorem{cauhoi}{Câu hỏi}
\newtheorem{conjecture}{Conjecture}
\newtheorem{corollary}{Corollary}
\newtheorem{dangtoan}{Dạng toán}
\newtheorem{definition}{Definition}
\newtheorem{dinhly}{Định lý}
\newtheorem{dinhnghia}{Định nghĩa}
\newtheorem{example}{Example}
\newtheorem{ghichu}{Ghi chú}
\newtheorem{hequa}{Hệ quả}
\newtheorem{hypothesis}{Hypothesis}
\newtheorem{kyhieu}{Ký hiệu}
\newtheorem{lemma}{Lemma}
\newtheorem{luuy}{Lưu ý}
\newtheorem{nhanxet}{Nhận xét}
\newtheorem{notation}{Notation}
\newtheorem{note}{Note}
\newtheorem{principle}{Principle}
\newtheorem{problem}{Problem}
\newtheorem{proposition}{Proposition}
\newtheorem{question}{Question}
\newtheorem{remark}{Remark}
\newtheorem{theorem}{Theorem}
\newtheorem{vidu}{Ví dụ}
\usepackage[left=1cm,right=1cm,top=5mm,bottom=5mm,footskip=4mm]{geometry}
\def\labelitemii{$\circ$}
\DeclareRobustCommand{\divby}{%
	\mathrel{\vbox{\baselineskip.65ex\lineskiplimit0pt\hbox{.}\hbox{.}\hbox{.}}}%
}

\title{Problem: Algebraic Expression Transformation\\Bài Tập: Biến Đổi Biểu Thức Đại Số}
\date{}

\begin{document}
\maketitle
\vspace{-2cm}

%------------------------------------------------------------------------------%

\section{Rational Expression Transformation -- Biến Đổi Biểu Thức Hữu Tỷ}

\begin{definition}[Rational expression]
	A {\rm rational expression} is the ratio of 2 polynomials. If $f$ is a rational expression then $f$ can be written in the form $\frac{p}{q}$ where $p,q$ are polynomials.
\end{definition}
Like polynomials or any other type of expression, the basic arithmetic operations, namely addition $+$, subtraction $-$, multiplication $\cdot$, \& division $:$ or {\tt/}, can be performed on rational expressions. A nice property of rational expressions is that when any of these operations are performed on 2 rational expressions, the result is always another rational expression. Contrary to polynomials, it is generally easy to multiply or divide but difficult to add or subtract 2 rational expressions.

\begin{notation}[Rational vs. irrational]
	Denote by $\mathbb{Q}_{\rm fn}\coloneqq\left\{\dfrac{a}{2^m\cdot5^n}|a\in\mathbb{Z},\,m,n\in\mathbb{N}\right\}$, $\mathbb{Q}_{\rm ifn}\coloneqq\mathbb{Q}\backslash\mathbb{Q}_{\rm fn}$, \& $\mathbb{R}\backslash\mathbb{Q}$ the set of all finite rationals, the set of all periodic infinite rationals, \& the set of irrationals, respectively.
\end{notation}

\begin{kyhieu}
	Ký hiệu $\mathbb{Q}_{\rm fn}\coloneqq\left\{\dfrac{a}{2^m\cdot5^n}|a\in\mathbb{Z},\,m,n\in\mathbb{N}\right\}$, $\mathbb{Q}_{\rm ifn}\coloneqq\mathbb{Q}\backslash\mathbb{Q}_{\rm fn}$, \& $\mathbb{R}\backslash\mathbb{Q}$ lần lượt là các tập hợp tất cả các số hữu tỷ hữu hạn, các số hữu tỷ vô hạn tuần hoàn, \& các số vô tỷ.
\end{kyhieu}

\begin{baitoan}[\cite{Lam_An_Tuan_Toan_9_dai_so}, Ví dụ 1, p. 5, chuyên Toán Quảng Ngãi 2018--2019]
	Cho biểu thức $A = \dfrac{5x + 1}{x^3 - 1} - \dfrac{1 - 2x}{x^2 + x + 1} - \dfrac{2}{1 - x}$. (a) Tìm {\rm ĐKXĐ}. (b) Rút gọn $A$. (c) Biện luận theo tham số $m\mathbb{R}$ để giải phương trình $A = m$. (d) Tìm $x\in\mathbb{R}$ rồi $x\in\mathbb{Z}$ để lượt $A$ lần lượt thuộc các tập hợp:  $\mathbb{N},\mathbb{Z},\mathbb{Q},\mathbb{Q}_{\rm fn},\mathbb{Q}_{\rm ifn},\mathbb{R}\backslash\mathbb{Q}$.
\end{baitoan}

\begin{baitoan}[\cite{Lam_An_Tuan_Toan_9_dai_so}, Ví dụ 2, p. 5]
	Cho biểu thức $A = \dfrac{2x^3 - 7x^2 - 12x + 45}{3x^3 -19x^2 + 33x - 9}$. (a) Tìm {\rm ĐKXĐ}. (b) Rút gọn $A$. (c) Biện luận theo tham số $m\mathbb{R}$ để giải phương trình $A = m$. (d) Tìm $x\in\mathbb{R}$ rồi $x\in\mathbb{Z}$ để lượt $A$ lần lượt thuộc các tập hợp:  $\mathbb{N},\mathbb{Z},\mathbb{Q},\mathbb{Q}_{\rm fn},\mathbb{Q}_{\rm ifn},\mathbb{R}\backslash\mathbb{Q}$.
\end{baitoan}

\section{Generalization -- Tổng Quát Hóa}

\begin{baitoan}[Phân thức bậc 1{\tt/}bậc 1]
	Cho biểu thức $A = \dfrac{ax + b}{cx + d}$ là phân thức với tử thức \& mẫu thức đều là đa thức bậc nhất ẩn $x$. Tìm điều kiện của $a,b,c,d\in\mathbb{R}$ để $A$ có thể rút gọn, \& biểu diễn biểu thức rút gọn theo $a,b,c,d$.
\end{baitoan}

\begin{baitoan}[Phân thức bậc 2{\tt/}bậc 2]
	(a) Cho biểu thức $A = \dfrac{x^2 + ax + b}{x^2 + cx + d}$ là phân thức với tử thức \& mẫu thức đều là đa thức bậc 2 ẩn $x$, trong đó $a,b,c,d\in\mathbb{R}$. Tìm {\rm ĐKXĐ}. Tìm điều kiện của $a,b,c,d$ để $A$ có thể rút gọn (i.e., rút gọn thành phân thức với tử thức \& mẫu thức đều là đa thức bậc nhất ẩn $x$ hoặc rút gọn thành hằng số), \& biểu diễn biểu thức rút gọn theo $a,b,c,d,x$. (b) Cho biểu thức $A = \dfrac{ax^2 + bx + c}{dx^2 + ex + f}$ là phân thức với tử thức \& mẫu thức đều là đa thức bậc 2 ẩn $x$. Tìm điều kiện của $a,b,c,d,e,f\in\mathbb{R}$, $ad\ne0$, để $A$ có thể rút gọn, \& biểu diễn biểu thức rút gọn theo $a,b,c,d,e,f,x$.
\end{baitoan}

%------------------------------------------------------------------------------%

\printbibliography[heading=bibintoc]

\end{document}