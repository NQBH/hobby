\documentclass{article}
\usepackage[backend=biber,natbib=true,style=alphabetic,maxbibnames=10]{biblatex}
\addbibresource{/home/nqbh/reference/bib.bib}
\usepackage[utf8]{vietnam}
\usepackage{tocloft}
\renewcommand{\cftsecleader}{\cftdotfill{\cftdotsep}}
\usepackage[colorlinks=true,linkcolor=blue,urlcolor=red,citecolor=magenta]{hyperref}
\usepackage{amsmath,amssymb,amsthm,float,graphicx,mathtools}
\allowdisplaybreaks
\newtheorem{assumption}{Assumption}
\newtheorem{baitoan}{Bài toán}
\newtheorem{cauhoi}{Câu hỏi}
\newtheorem{conjecture}{Conjecture}
\newtheorem{corollary}{Corollary}
\newtheorem{dangtoan}{Dạng toán}
\newtheorem{definition}{Definition}
\newtheorem{dinhly}{Định lý}
\newtheorem{dinhnghia}{Định nghĩa}
\newtheorem{example}{Example}
\newtheorem{ghichu}{Ghi chú}
\newtheorem{hequa}{Hệ quả}
\newtheorem{hypothesis}{Hypothesis}
\newtheorem{lemma}{Lemma}
\newtheorem{luuy}{Lưu ý}
\newtheorem{nhanxet}{Nhận xét}
\newtheorem{notation}{Notation}
\newtheorem{note}{Note}
\newtheorem{principle}{Principle}
\newtheorem{problem}{Problem}
\newtheorem{proposition}{Proposition}
\newtheorem{question}{Question}
\newtheorem{remark}{Remark}
\newtheorem{theorem}{Theorem}
\newtheorem{vidu}{Ví dụ}
\usepackage[left=1cm,right=1cm,top=5mm,bottom=5mm,footskip=4mm]{geometry}
\def\labelitemii{$\circ$}
\DeclareRobustCommand{\divby}{%
	\mathrel{\vbox{\baselineskip.65ex\lineskiplimit0pt\hbox{.}\hbox{.}\hbox{.}}}%
}

\title{Problem: Circle -- Bài Tập: Đường Tròn}
\author{Nguyễn Quản Bá Hồng\footnote{Independent Researcher, Ben Tre City, Vietnam\\e-mail: \texttt{nguyenquanbahong@gmail.com}; website: \url{https://nqbh.github.io}.}}
\date{\today}

\begin{document}
\maketitle
\begin{abstract}
	
\end{abstract}
\tableofcontents

%------------------------------------------------------------------------------%

\section{Sự Xác Định Đường Tròn. Tính Chất Đối Xứng của Đường Tròn}

\begin{baitoan}[\cite{Tuyen_Toan_9}, Thí dụ 5, pp. 113--114]
	Trên đường tròn $(O;R)$ đường kính $AB$ lấy 1 điểm $C$. Trên tia $AC$ lấy điểm $M$ sao cho $C$ là trung điểm của $AM$. (a) Xác định vị trí của điểm $C$ để $AM$ có độ dài lớn nhất. (b) Xác định vị trí của điểm $C$ để $AM = 2R\sqrt{3}$. (c) Chứng minh khi $C$ di động trên đường tròn $(O)$ thì điểm $M$ di động trên 1 đường tròn cố định.
\end{baitoan}

\begin{baitoan}[\cite{Tuyen_Toan_9}, 36., p. 114]
	Cho $\Delta ABC$ cân tại $A$, đường cao $AH = BC = a$. Tính bán kính của đường tròn ngoại tiếp $\Delta ABC$.
\end{baitoan}

\begin{baitoan}[\cite{Tuyen_Toan_9}, 37., p. 114]
	Cho $\Delta ABC$. Gọi $D,E,F$ lần lượt là trung điểm của $BC,CA,AB$. Chứng minh: các đường tròn $(AFE),(BFD),(CDE)$ bằng nhau \& cùng đi qua 1 điểm. Xác định điểm chung đó.
\end{baitoan}

\begin{baitoan}[\cite{Tuyen_Toan_9}, 38., p. 114]
	Cho hình thoi $ABCD$ cạnh $1$, 2 đường chéo cắt nhau tại $O$. Gọi $R_1$ \& $R_2$ lần lượt là bán kính các đường tròn ngoại tiếp các $\Delta ABC,\Delta ABD$. Chứng minh: $\dfrac{1}{R_1^2} + \dfrac{1}{R_2^2} = 4$.
\end{baitoan}

\begin{baitoan}[\cite{Tuyen_Toan_9}, 39., p. 115]
	Cho hình bình hành $ABCD$, cạnh $AB$ cố định, đường chéo $AC = 2$ \emph{cm}. Chứng minh điểm $D$ di động trên 1 đường tròn cố định.
\end{baitoan}

\begin{baitoan}[\cite{Tuyen_Toan_9}, 40., p. 115]
	Cho đường tròn $(O;R)$ \& 1 dây $BC$ cố định. Trên đường tròn lấy 1 điểm $A$ ($A\not\equiv B$, $A\not\equiv C$). Gọi $G$ là trọng tâm của $\Delta ABC$. Chứng minh khi $A$ di động trên đường tròn $(O)$ thì điểm $G$ di động trên 1 đường tròn cố định.
\end{baitoan}

\begin{baitoan}[\cite{Tuyen_Toan_9}, 41., p. 115]
	Trong mặt phẳng cho $2n + 1$ điểm, $n\in\mathbb{N}$, sao cho $3$ điểm bất kỳ nào cũng tồn tại $2$ điểm có khoảng cách nhỏ hơn $1$. Chứng minh: trong các điểm này có ít nhất $n + 1$ điểm nằm trong 1 đường tròn có bán kính bằng $1$.
\end{baitoan}

\begin{baitoan}[\cite{Tuyen_Toan_9}, 42., p. 115]
	Cho hình bình hành $ABCD$, 2 đường chéo cắt nhau tại $O$. Vẽ đường tròn tâm $O$ cắt các đường thẳng $AB,BC,CD,DA$ lần lượt tại $M,N,P,Q$. Xác định dạng của tứ giác $MNPQ$.
\end{baitoan}

\begin{baitoan}[\cite{Tuyen_Toan_9}, 43., p. 115]
	2 người chơi 1 trò chơi như sau: Mỗi người lần lượt đặt lên 1 chiếc bàn hình tròn 1 cái cốc. Ai là người cuối cùng đặt được cốc lên bàn thì người đó thắng cuộc. Muốn chắc thắng thì phải chơi theo ``chiến thuật'' nào? (các chiếc cốc đều như nhau).
\end{baitoan}

%------------------------------------------------------------------------------%

\section{Đường Kính \& Dây của Đường Tròn}

%------------------------------------------------------------------------------%

\section{Liên Hệ Giữa Dây \& Khoảng Cách Từ Tâm Đến Dây}

%------------------------------------------------------------------------------%

\section{Vị Trí Tương Đối của Đường Thẳng \& Đường Tròn}

%------------------------------------------------------------------------------%

\section{Dấu Hiệu Nhận Biết Tiếp Tuyến của Đường Tròn}

%------------------------------------------------------------------------------%

\section{Tính Chất của 2 Tiếp Tuyến Cắt Nhau}

%------------------------------------------------------------------------------%

\section{Vị Trí Tương Đối của 2 Đường Tròn}

%------------------------------------------------------------------------------%

\section{Miscellaneous}

%------------------------------------------------------------------------------%

\printbibliography[heading=bibintoc]
	
\end{document}