\documentclass{article}
\usepackage[backend=biber,natbib=true,style=alphabetic,maxbibnames=10]{biblatex}
\addbibresource{/home/nqbh/reference/bib.bib}
\usepackage[utf8]{vietnam}
\usepackage{tocloft}
\renewcommand{\cftsecleader}{\cftdotfill{\cftdotsep}}
\usepackage[colorlinks=true,linkcolor=blue,urlcolor=red,citecolor=magenta]{hyperref}
\usepackage{amsmath,amssymb,amsthm,float,graphicx,mathtools}
\allowdisplaybreaks
\newtheorem{assumption}{Assumption}
\newtheorem{baitoan}{Bài toán}
\newtheorem{cauhoi}{Câu hỏi}
\newtheorem{conjecture}{Conjecture}
\newtheorem{corollary}{Corollary}
\newtheorem{dangtoan}{Dạng toán}
\newtheorem{definition}{Definition}
\newtheorem{dinhly}{Định lý}
\newtheorem{dinhnghia}{Định nghĩa}
\newtheorem{example}{Example}
\newtheorem{ghichu}{Ghi chú}
\newtheorem{hequa}{Hệ quả}
\newtheorem{hypothesis}{Hypothesis}
\newtheorem{lemma}{Lemma}
\newtheorem{luuy}{Lưu ý}
\newtheorem{nhanxet}{Nhận xét}
\newtheorem{notation}{Notation}
\newtheorem{note}{Note}
\newtheorem{principle}{Principle}
\newtheorem{problem}{Problem}
\newtheorem{proposition}{Proposition}
\newtheorem{question}{Question}
\newtheorem{remark}{Remark}
\newtheorem{theorem}{Theorem}
\newtheorem{vidu}{Ví dụ}
\usepackage[left=1cm,right=1cm,top=5mm,bottom=5mm,footskip=4mm]{geometry}
\def\labelitemii{$\circ$}
\DeclareRobustCommand{\divby}{%
	\mathrel{\vbox{\baselineskip.65ex\lineskiplimit0pt\hbox{.}\hbox{.}\hbox{.}}}%
}

\title{Programming Problem: $n$th Roots {\it\&} Trigonometry in Triangles\\Bài Tập Lập Trình: Căn Bậc $n$ \& Lượng Giác trong Tam Giác}
\date{}

\begin{document}
\maketitle
\vspace{-2cm}

%------------------------------------------------------------------------------%

\section{Root}

\begin{baitoan}[Root -- Căn]
	
\end{baitoan}

%------------------------------------------------------------------------------%

\section{Trigonometry in Right Triangles}
``A \textit{right triangle} (\href{https://en.wikipedia.org/wiki/American_English}{American English}) or \textit{right-angled triangle} (\href{https://en.wikipedia.org/wiki/British_English}{British English}), or more formally an \textit{orthogonal triangle}, formerly called a \textit{rectangled triangle} is a \href{https://en.wikipedia.org/wiki/Triangle}{triangle } in which 1 \href{https://en.wikipedia.org/wiki/Angle}{angle} is a \href{https://en.wikipedia.org/wiki/Right_angle}{right angle} (i.e., a $90^\circ$ angle), i.e., in which 2 \href{https://en.wikipedia.org/wiki/Polygon_side}{sides} are \href{https://en.wikipedia.org/wiki/Perpendicular}{perpendicular}. The relation between the sides \& other angles of the right triangle is the basis for \href{https://en.wikipedia.org/wiki/Trigonometry}{trigonometry}.''

``The side opposite to the right angle is called the \href{https://en.wikipedia.org/wiki/Hypotenuse}{\textit{hypotenuse}}. The sides adjacent to the right angle are called \textit{legs} (or \textit{catheti}, singular: \href{https://en.wikipedia.org/wiki/Cathetus}{cathetus}).'' -- \href{https://en.wikipedia.org/wiki/Right_triangle}{Wikipedia{\tt/}right triangle}

Given a right triangle $\Delta ABC$ with $\widehat A = 90^\circ$. Define $a\coloneqq BC$, $b\coloneqq CA$, $c\coloneqq AB$. Side $b$ is the side \textit{adjacent to angle $C$} \& \textit{opposed to angle $B$}, while side $c$ may be identified as the side \textit{adjacent to angle $B$} \& opposed to (or \textit{opposite}) angle $C$.

\subsection{Pythagorean Triple}

\begin{problem}[Pythagorean triple]
	If the lengths of all 3 sides of a right triangle are integers, the triangle is said to be a {\rm Pythagorean triangle} \& its side lengths are collectively known as a \href{https://en.wikipedia.org/wiki/Pythagorean_triple}{Pythagorean triple}. (a) Write {\sf Pascal, Python, C{\tt/}C++} programs to check if 3 integers $a,b,c$ input from the keyboard are Pythagorean triple or not.
\end{problem}

\subsection{Solve Right Triangle}

\begin{baitoan}[Solve right triangle -- Giải tam giác vuông]
	
\end{baitoan}

%------------------------------------------------------------------------------%

\section{Trigonometry in Triangles}
Tổng quát hơn cho tam giác (không suy biến) bất kỳ (i.e., tam giác nhọn, vuông, tù).

\subsection{Solve Triangle}

\begin{baitoan}[Solve triangle -- Giải tam giác]
	
\end{baitoan}

%------------------------------------------------------------------------------%

\printbibliography[heading=bibintoc]

\end{document}