\documentclass{article}
\usepackage[backend=biber,natbib=true,style=alphabetic,maxbibnames=10]{biblatex}
\addbibresource{/home/nqbh/reference/bib.bib}
\usepackage[utf8]{vietnam}
\usepackage{tocloft}
\renewcommand{\cftsecleader}{\cftdotfill{\cftdotsep}}
\usepackage[colorlinks=true,linkcolor=blue,urlcolor=red,citecolor=magenta]{hyperref}
\usepackage{amsmath,amssymb,amsthm,float,graphicx,mathtools}
\allowdisplaybreaks
\newtheorem{assumption}{Assumption}
\newtheorem{baitoan}{Bài toán}
\newtheorem{cauhoi}{Câu hỏi}
\newtheorem{conjecture}{Conjecture}
\newtheorem{corollary}{Corollary}
\newtheorem{dangtoan}{Dạng toán}
\newtheorem{definition}{Definition}
\newtheorem{dinhly}{Định lý}
\newtheorem{dinhnghia}{Định nghĩa}
\newtheorem{example}{Example}
\newtheorem{ghichu}{Ghi chú}
\newtheorem{hequa}{Hệ quả}
\newtheorem{hypothesis}{Hypothesis}
\newtheorem{lemma}{Lemma}
\newtheorem{luuy}{Lưu ý}
\newtheorem{nhanxet}{Nhận xét}
\newtheorem{notation}{Notation}
\newtheorem{note}{Note}
\newtheorem{principle}{Principle}
\newtheorem{problem}{Problem}
\newtheorem{proposition}{Proposition}
\newtheorem{question}{Question}
\newtheorem{remark}{Remark}
\newtheorem{theorem}{Theorem}
\newtheorem{vidu}{Ví dụ}
\usepackage[left=1cm,right=1cm,top=5mm,bottom=5mm,footskip=4mm]{geometry}
\def\labelitemii{$\circ$}
\DeclareRobustCommand{\divby}{%
	\mathrel{\vbox{\baselineskip.65ex\lineskiplimit0pt\hbox{.}\hbox{.}\hbox{.}}}%
}

\title{Programming Problem: $n$th Roots {\it\&} Trigonometry in Triangles\\Bài Tập Lập Trình: Căn Bậc $n$ \& Lượng Giác trong Tam Giác}
\author{Nguyễn Quản Bá Hồng\footnote{Independent Researcher, Ben Tre City, Vietnam\\e-mail: \texttt{nguyenquanbahong@gmail.com}; website: \url{https://nqbh.github.io}.}}
\date{\today}

\begin{document}
\maketitle
\tableofcontents

%------------------------------------------------------------------------------%

\section{Root}

\subsection{Square Root}

\subsection{Cube Root}

\subsection{$n$th Root}

\begin{baitoan}[Root -- Căn]
	
\end{baitoan}

%------------------------------------------------------------------------------%

\section{Trigonometry in Right Triangles}
``A \textit{right triangle} (\href{https://en.wikipedia.org/wiki/American_English}{American English}) or \textit{right-angled triangle} (\href{https://en.wikipedia.org/wiki/British_English}{British English}), or more formally an \textit{orthogonal triangle}, formerly called a \textit{rectangled triangle} is a \href{https://en.wikipedia.org/wiki/Triangle}{triangle } in which 1 \href{https://en.wikipedia.org/wiki/Angle}{angle} is a \href{https://en.wikipedia.org/wiki/Right_angle}{right angle} (i.e., a $90^\circ$ angle), i.e., in which 2 \href{https://en.wikipedia.org/wiki/Polygon_side}{sides} are \href{https://en.wikipedia.org/wiki/Perpendicular}{perpendicular}. The relation between the sides \& other angles of the right triangle is the basis for \href{https://en.wikipedia.org/wiki/Trigonometry}{trigonometry}.''

``The side opposite to the right angle is called the \href{https://en.wikipedia.org/wiki/Hypotenuse}{\textit{hypotenuse}}. The sides adjacent to the right angle are called \textit{legs} (or \textit{catheti}, singular: \href{https://en.wikipedia.org/wiki/Cathetus}{cathetus}).'' -- \href{https://en.wikipedia.org/wiki/Right_triangle}{Wikipedia{\tt/}right triangle}

Given a right triangle $\Delta ABC$ with $\widehat A = 90^\circ$. Define $a\coloneqq BC$, $b\coloneqq CA$, $c\coloneqq AB$. Side $b$ is the side \textit{adjacent to angle $C$} \& \textit{opposed to angle $B$}, while side $c$ may be identified as the side \textit{adjacent to angle $B$} \& opposed to (or \textit{opposite}) angle $C$.

\subsection{Pythagorean Triple}

\begin{definition}[Pythagorean triple]
	If the lengths of all 3 sides of a right triangle are integers, the triangle is said to be a {\rm Pythagorean triangle} \& its side lengths are collectively known as a \href{https://en.wikipedia.org/wiki/Pythagorean_triple}{\rm Pythagorean triple}.
\end{definition}
``A \textit{Pythagorean triple} consists of 3 positive integers $a,b,c$, such that $a^2 = b^2 + c^2$. Such a triple is commonly written $(b,c,a)$, \& a well-known example is $(3,4,5)$. If $(b,c,a)$ is a Pythagorean triple, then so is $(kb,kc,ka)$ for any positive integer $k$. A \textit{primitive Pythagorean triple} is one in which $a,b,c$ are \href{https://en.wikipedia.org/wiki/Coprime}{coprime} (i.e., they have no common divisor larger than 1), e.g., $(3,4,5)$ is a primitive Pythagorean triple whereas $(6,8,10)$ is not. A triangle whose sides form a Pythagorean triple is called a \textit{Pythagorean triangle}, \& is necessarily a \href{https://en.wikipedia.org/wiki/Right_triangle}{right triangle}.

The name is derived from the \href{https://en.wikipedia.org/wiki/Pythagorean_theorem}{Pythagorean theorem}, stating that every right triangle has side lengths satisfying the formula $a^2 = b^2 + c^2$; thus, Pythagorean triples describe the 3 integer side lengths of a right triangle. However, right triangles with non-integer sides do not form Pythagorean triples. E.g., the triangle with sides $(b,c,a) = (1,1,\sqrt{2})$ is a right triangle, but $(1,1,\sqrt{2})$ is not a Pythagorean triple because $\sqrt{2}$ is not an integer.\footnote{$\sqrt{2}\in\mathbb{R}\backslash\mathbb{Q}$, i.e., $\sqrt{2}$ is an irrational number (i.e., a real number which is not a rational number).} Moreover, $1$ \& $\sqrt{2}$ do not have an integer common multiple because $\sqrt{2}$ is \href{https://en.wikipedia.org/wiki/Irrational_number#History}{irrational}.''

``When searching for integer solutions, the equation $b^2 + c^2 = a^2$ is a \href{https://en.wikipedia.org/wiki/Diophantine_equation}{Diophantine equation}. Thus Pythagorean triples are among the oldest known solutions of a \href{https://en.wikipedia.org/wiki/Linear_equation}{nonlinear} Diophantine equation.'' -- \href{https://en.wikipedia.org/wiki/Pythagorean_triple}{Wikipedia{\tt/}Pythagorean triple}

\begin{problem}[Pythagorean triple]
	 Write {\sf Pascal, Python, C{\tt/}C++} programs to check if 3 integers $a,b,c$ input from the keyboard: (a) form a Pythagorean triangle or not. (b) form a primitive Pythagorean triple or not. If not, find \& print out their primitive Pythagorean triple.
\end{problem}

\begin{problem}[List of primitive \& non-primitive Pythagorean triples]
	 Let $N$ be an integer input from the keyboard. Write {\sf Pascal, Python, C{\tt/}C++} programs to print out all: (a) primitive Pythagorean triples of numbers up to $N$. (b) Pythagorean triples of numbers up to $N$.
\end{problem}
Sample: ``There are 16 primitive Pythagorean triples of numbers up to 100:\\$(3, 4, 5)$, $(5, 12, 13)$, $(8, 15, 17)$, $(7, 24, 25)$, $(20, 21, 29)$, $(12, 35, 37)$, $(9, 40, 41)$, $(28, 45, 53)$, $(11, 60, 61)$, $(16, 63, 65)$, $(33, 56, 65)$, $(48, 55, 73)$, $(13, 84, 85)$, $(36, 77, 85)$, $(39, 80, 89)$, $(65, 72, 97)$.

Other small Pythagorean triples such as $(6, 8, 10)$ are not listed because they are not primitive; for instance $(6, 8, 10)$ is a multiple of $(3, 4, 5)$.'' [$\ldots$] ``Additionally, these are the remaining primitive Pythagorean triples of numbers up to 300:\\$(20, 99, 101)$, $(60, 91, 109)$, $(15, 112, 113)$, $(44, 117, 125)$, $(88, 105, 137)$, $(17, 144, 145)$, $(24, 143, 145)$, $(51, 140, 149)$, $(85, 132, 157)$, $(119, 120, 169)$, $(52, 165, 173)$, $(19, 180, 181)$, $(57, 176, 185)$, $(104, 153, 185)$, $(95, 168, 193)$, $(28, 195, 197)$, $(84, 187, 205)$, $(133, 156, 205)$, $(21, 220, 221)$, $(140, 171, 221)$, $(60, 221, 229)$, $(105, 208, 233)$, $(120, 209, 241) $, $(32, 255, 257)$, $(23, 264, 265)$, $(96, 247, 265)$, $(69, 260, 269)$, $(115, 252, 277)$, $(160, 231, 281)$, $(161, 240, 289)$, $(68, 285, 293)$.'' -- \href{https://en.wikipedia.org/wiki/Pythagorean_triple#Examples}{Wikipedia{\tt/}Pythagorean triple{\tt/}examples}

``\textit{Euclid's formula} is a fundamental formula for generating Pythagorean triples given an arbitrary pair of integers $m,n$ with $m > n > 0$. The formula states that the integers
\begin{align}
	\label{Euclid formula}
	\boxed{b = m^2 - n^2,\ c = 2mn,\ a = m^2 + n^2,\mbox{ where } m,n\in\mathbb{N}^\star,\,m > n,}
\end{align}
form a Pythagorean triple. The triple generated by Euclid's formula is primitive iff $m,n$ are \href{https://en.wikipedia.org/wiki/Coprime}{coprime} \& 1 of them is even. When both $m,n$ are odd, then $a,b,c$ will be even, \& the triple will not be primitive; however, dividing $a,b,c$ by 2 will yield a primitive triple when $m,n$ are coprime.

\textit{Every} primitive triple arises (after the exchange of $b$ \& $c$, if $b$ is even) from a \textit{unique pair} of coprime numbers $m,n$, one of which is even. It follows that there are infinitely many primitive Pythagorean triples.'' [$\ldots$] ``Despite generating all primitive triples, Euclid's formula does not produce all triples, e.g., $(9,12,15)$ cannot be generated using integer $m,n$. This can be remedied by inserting an additional parameter $k$ to the formula. The following will generate all Pythagorean triples uniquely:
\begin{align}
	\label{Euclid formula 1}
	\boxed{b = k(m^2 - n^2),\ c = 2kmn,\ a = k(m^2 + n^2),\mbox{ where } m,n,k\in\mathbb{N}^\star,\,m > n,\,{\rm gcd}(m,n) = 1,\,mn\divby2.}
\end{align}
These formulas generate Pythagorean triples can be verified by expanding $b^2 + c^2$ using \href{https://en.wikipedia.org/wiki/Elementary_algebra}{elementary algebra} \& verifying that the result equals $c^2$. Since every Pythagorean triple can be divided through by some integer $k$ to obtain a primitive triple, every triple can be generated uniquely by using the formula with $m,n$ to generate its primitive counterpart \& then multiplying through by $k$ as in the last equation \eqref{Euclid formula 1}.

Choosing $m,n$ from certain integer sequences gives interesting results, e.g., if $m,n$ are consecutive \href{https://en.wikipedia.org/wiki/Pell_number}{Pell numbers}, $a,b$ will differ by 1. Many formulas for generating triples with particular properties have been developed since the time of Euclid.'' -- \href{https://en.wikipedia.org/wiki/Pythagorean_triple#Generating_a_triple}{Wikipedia{\tt/}Pythagorean triple{\tt/}generating a triple}

\begin{problem}
	(a) Prove that $(a,b,c)$ given by either formulas \eqref{Euclid formula} or \eqref{Euclid formula 1} is a Pythagorean triple. (b) Compute $\sin,\cos,\tan,\cot$ of angles $B,C$ in terms of $m,n,k$.
\end{problem}
See \href{https://en.wikipedia.org/wiki/Pythagorean_triple#Proof_of_Euclid's_formula}{Wikipedia{\tt/}formulas for generating Pythagorean triples{\tt/}proof of Euclid's formula} for a (mathematically rigorous) proof. \& \href{https://en.wikipedia.org/wiki/Pythagorean_triple#Interpretation_of_parameters_in_Euclid's_formula}{Wikipedia{\tt/}formulas for generating Pythagorean triples{\tt/}interpretation of parameters in Euclid's formula}.

\begin{proof}[A variant of Euclid's formula for Pythagorean triples]
	The following variant of Euclid's formula is sometimes more convenient, as being more symmetric in $m,n$ (same parity condition on $m,n$). Prove that
	\begin{align}
		\label{Euclid formula 2}
		\boxed{b = mn,\ c = \frac{m^2 - n^2}{2},\ a = \frac{m^2 + n^2}{2},\mbox{ where } m,n,k\in\mathbb{N}^\star,\,m > n,\,{\rm gcd}(m,n) = 1,\,mn\not{\divby}\,2.}
	\end{align}
	are 3 integers that form a Pythagorean triple, which is primitive iff $m,n$ are coprime. Conversely, every primitively Pythagorean triple arises (after the exchange of $b,c$, if $b$ is even) from a unique pair $m > n > 0$ of coprime odd integers.
\end{proof}

\begin{problem}[List of primitive \& non-primitive Pythagorean triples]
	Let $n$ be an integer input from the keyboard. Use Euclid's formulas \eqref{Euclid formula}, \eqref{Euclid formula 1}, \& \eqref{Euclid formula 2} for generating Pythagorean triples, write {\sf Pascal, Python, C{\tt/}C++} programs to print out all: (a) primitive Pythagorean triples of numbers up to $N$. (b) Pythagorean triples of numbers up to $N$.
\end{problem}
See also, \href{https://en.wikipedia.org/wiki/Formulas_for_generating_Pythagorean_triples}{Wikipedia{\tt/}formulas for generating Pythagorean triples}.

\subsection{Principal Properties of Right Triangles}

\subsubsection{Sides -- Cạnh}
``The 3 sides of a right triangle are related by the \href{https://en.wikipedia.org/wiki/Pythagorean_theorem}{Pythagorean theorem}, which in modern algebraic notation can be written $b^2 + c^2 = a^2$, where $a$ is the length of the \textit{hypotenuse} (side opposite the right angle), \& $a,b$ are the lengths of the \textit{legs} (remaining 2 sides). \href{https://en.wikipedia.org/wiki/Pythagorean_triple}{Pythagorean triples} are integer values of $a,b,c$ satisfying this equation. This theorem was proven in antiquity, and is proposition I.47 in \href{https://en.wikipedia.org/wiki/Euclid%27s_Elements}{Euclid's \textit{Elements}}: ``In right-angled triangles the square on the side subtending the right angle is equal to the squares on the sides containing the right angle.''

\subsubsection{Area -- Diện Tích}
``As with any triangle, the area is equal to one half the base multiplied by the corresponding height. In a right triangle, if 1 leg is taken as the base then the other is height, so the area of a right triangle is one half the product of the 2 legs. As a formula, the area $S$ is $S = \frac{1}{2}bc$, where $b,c$ are the legs of the triangle.

If the \href{https://en.wikipedia.org/wiki/Incircle_and_excircles_of_a_triangle}{incircle} is tangent to the hypotenuse $BC$ at point $P$, then denoting the \href{https://en.wikipedia.org/wiki/Semi-perimeter}{semi-perimeter} $\frac{a + b + c}{2}$ as $p$, we have $PB = p - b$, $PC = p - c$, \& the area is given by $S = PB\cdot PC = (p - b)(p - c)$. This formula only applies to right triangles.''

\begin{problem}
	Prove that the formula holds for any right triangle $\Delta ABC$ with the right angle $A$: $S = PB\cdot PC = (p - b)(p - c)$ where $p\coloneqq\frac{a + b + c}{2}$ is its semi-perimeter.
\end{problem}

\subsubsection{Altitudes -- Đường Cao}


\subsection{Solve Right Triangle}

\begin{baitoan}[Solve right triangle -- Giải tam giác vuông]
	
\end{baitoan}

%------------------------------------------------------------------------------%

\section{Trigonometry in Triangles}
Tổng quát hơn cho tam giác (không suy biến) bất kỳ (i.e., tam giác nhọn, vuông, tù).

\subsection{Solve Triangle}

\begin{baitoan}[Solve triangle -- Giải tam giác]
	
\end{baitoan}

%------------------------------------------------------------------------------%

\printbibliography[heading=bibintoc]

\end{document}