\documentclass{article}
\usepackage[backend=biber,natbib=true,style=authoryear,maxbibnames=10]{biblatex}
\addbibresource{/home/nqbh/reference/bib.bib}
\usepackage[utf8]{vietnam}
\usepackage{tocloft}
\renewcommand{\cftsecleader}{\cftdotfill{\cftdotsep}}
\usepackage[colorlinks=true,linkcolor=blue,urlcolor=red,citecolor=magenta]{hyperref}
\usepackage{amsmath,amssymb,amsthm,float,graphicx,mathtools,soul,subcaption}
\allowdisplaybreaks
\newtheorem{assumption}{Assumption}
\newtheorem{baitoan}{Bài toán}
\newtheorem{cauhoi}{Câu hỏi}
\newtheorem{conjecture}{Conjecture}
\newtheorem{corollary}{Corollary}
\newtheorem{dangtoan}{Dạng toán}
\newtheorem{definition}{Definition}
\newtheorem{dinhly}{Định lý}
\newtheorem{dinhnghia}{Định nghĩa}
\newtheorem{example}{Example}
\newtheorem{ghichu}{Ghi chú}
\newtheorem{hequa}{Hệ quả}
\newtheorem{hypothesis}{Hypothesis}
\newtheorem{lemma}{Lemma}
\newtheorem{luuy}{Lưu ý}
\newtheorem{nhanxet}{Nhận xét}
\newtheorem{notation}{Notation}
\newtheorem{note}{Note}
\newtheorem{principle}{Principle}
\newtheorem{problem}{Problem}
\newtheorem{proposition}{Proposition}
\newtheorem{question}{Question}
\newtheorem{remark}{Remark}
\newtheorem{theorem}{Theorem}
\newtheorem{vidu}{Ví dụ}
\usepackage[left=1cm,right=1cm,top=5mm,bottom=5mm,footskip=4mm]{geometry}
\def\labelitemii{$\circ$}
\DeclareRobustCommand{\divby}{%
	\mathrel{\vbox{\baselineskip.65ex\lineskiplimit0pt\hbox{.}\hbox{.}\hbox{.}}}%
}

\title{Square Root \& Cube Root -- Căn Bậc 2 $\sqrt{f(x)}$ \& Căn Bậc 3 $\sqrt[3]{f(x)}$}
\author{Nguyễn Quản Bá Hồng\footnote{Independent Researcher, Ben Tre City, Vietnam\\e-mail: \texttt{nguyenquanbahong@gmail.com}; website: \url{https://nqbh.github.io}.}}
\date{\today}

\begin{document}
\maketitle
\begin{abstract}
	\textsf{[en]} This text is a collection of problems, from easy to advanced, about \textit{square root \& cube root}. This text is also a supplementary material for my lecture note on Elementary Mathematics grade 9, which is stored \& downloadable at the following link: \href{https://github.com/NQBH/hobby/blob/master/elementary_mathematics/grade_9/NQBH_elementary_mathematics_grade_9.pdf}{GitHub\texttt{/}NQBH\texttt{/}hobby\texttt{/}elementary mathematics\texttt{/}grade 9\texttt{/}lecture}\footnote{\textsc{url}: \url{https://github.com/NQBH/hobby/blob/master/elementary_mathematics/grade_9/NQBH_elementary_mathematics_grade_9.pdf}.}. The latest version of this text has been stored \& downloadable at the following link: \href{https://github.com/NQBH/hobby/blob/master/elementary_mathematics/grade_9/square_root_cube_root/NQBH_square_root_cube_root.pdf}{GitHub\texttt{/}NQBH\texttt{/}hobby\texttt{/}elementary mathematics\texttt{/}grade 9\texttt{/}square root \& cube root}\footnote{\textsc{url}: \url{https://github.com/NQBH/hobby/blob/master/elementary_mathematics/grade_9/similar_triangle/NQBH_square_root_cube_root.pdf}.}.
	\vspace{2mm}
	
	\textsf{[vi]} Tài liệu này là 1 bộ sưu tập các bài tập chọn lọc từ cơ bản đến nâng cao về \textit{các tam giác đồng dạng}. Tài liệu này là phần bài tập bổ sung cho tài liệu chính -- bài giảng \href{https://github.com/NQBH/hobby/blob/master/elementary_mathematics/grade_9/NQBH_elementary_mathematics_grade_9.pdf}{GitHub\texttt{/}NQBH\texttt{/}hobby\texttt{/}elementary mathematics\texttt{/}grade 9\texttt{/}lecture} của tác giả viết cho Toán Sơ Cấp lớp 9. Phiên bản mới nhất của tài liệu này được lưu trữ \& có thể tải xuống ở link sau: \href{https://github.com/NQBH/hobby/blob/master/elementary_mathematics/grade_9/square_root_cube_root/NQBH_square_root_cube_root.pdf}{GitHub\texttt{/}NQBH\texttt{/}hobby\texttt{/}elementary mathematics\texttt{/}grade 9\texttt{/}square root \& cube root}.
	
	\textsf{\textbf{Nội dung.} Căn thức bậc 2, 3, $n$.}
\end{abstract}
\tableofcontents
\newpage

%------------------------------------------------------------------------------%

\section*{Notation -- Ký Hiệu}
\begin{itemize}
	\item $\mathbb{R}_{> 0} = (0,+\infty)\coloneqq\{x\in\mathbb{R}|x > 0\}$: Tập hợp các số thực dương.
	\item $\mathbb{R}_{\ge0} = [0,+\infty)\coloneqq\{x\in\mathbb{R}|x\ge0\}$: Tập hợp các số thực không âm.
	\item $\mathbb{R}_{< 0} = (-\infty,0)\coloneqq\{x\in\mathbb{R}|x < 0\}$: Tập hợp các số thực âm.
	\item $\mathbb{R}_{\le0} = (-\infty,0]\coloneqq\{x\in\mathbb{R}|x\le0\}$: Tập hợp các số thực không dương.
\end{itemize}

\section{Square Root -- Căn Bậc 2}

\subsection{\href{https://en.wikipedia.org/wiki/Square_root}{Wikipedia\texttt{/}square root}}

\begin{definition}[Square root]
	``In mathematics, a \textit{square root} of a number $x\in\mathbb{R}$ is a number $y\in\mathbb{R}$ such that $y^2 = x$, i.e., a number $y$ whose \href{https://en.wikipedia.org/wiki/Square_(algebra)}{square} (the result of multiplying the number by itself, or $y\cdot y = y^2$) is $x$.
\end{definition}

\begin{example}
	$\pm4$ (i.e., both $4$ \& $-4$) are square roots of $16$ because $4^2 = (-4)^2 = 16$.
\end{example}
Every nonnegative real number $x$ has a unique nonnegative square root, called the \textit{principal square root}, which is denoted by $\sqrt{x}$, where the symbol $\sqrt{\cdot}$ is called the \href{https://en.wikipedia.org/wiki/Radical_sign}{\textit{radical sign}} or \textit{radix}. E.g., to express the fact that the principal square root of 9 is 3, we write $\sqrt{9} = 3$. The term (or number) whose square root is being considered is known as the \textit{radicand}. The radicand is the number or expression underneath the radical sign, in this case, 9. For nonnegative $x\in\mathbb{R}_{\ge0}$, the principal square root can also be written in \href{https://en.wikipedia.org/wiki/Exponentiation}{exponent} notation, as $x^{\frac{1}{2}}$.

Every \href{https://en.wikipedia.org/wiki/Positive_number}{positive number} $x\in\mathbb{R}_{> 0}$ has 2 square roots: $\sqrt{x}$ (which is positive) \& $-\sqrt{x}$ (which is negative). The 2 roots can be written more concisely using the \href{https://en.wikipedia.org/wiki/Plus%E2%80%93minus_sign}{$\pm$ sign} as $\pm\sqrt{x}$. Although the principle square root of a positive number is only 1 of its 2 square roots, the designation ``\textit{the} square root'' is often used to refer to the principal square root.

Square roots of negative numbers can be discussed within the framework of \href{https://en.wikipedia.org/wiki/Complex_number}{complex numbers}. More generally, square roots can be considered in any context in which a notion of the ``\href{https://en.wikipedia.org/wiki/Square_(algebra)}{square}'' of a mathematical object is defined. These include \href{https://en.wikipedia.org/wiki/Function_space}{function spaces} \& \href{https://en.wikipedia.org/wiki/Square_matrices}{square matrices}, among other \href{https://en.wikipedia.org/wiki/Mathematical_structure}{mathematical structures}.'' -- \href{https://en.wikipedia.org/wiki/Square_root}{Wikipedia\texttt{/}square root}

For the history of square root, see, e.g., \href{https://en.wikipedia.org/wiki/Square_root#History}{Wikipedia\texttt{/}square root\texttt{/}history}.

\subsection{Khái niệm căn bậc 2}
Phép toán ngược của phép cộng là phép trừ: $a + b - b = a$, $-(-a) = +a$, $\forall a\in\mathbb{R}$. Phép toán ngược của phép nhân là phép chia: $ab:b = a$, $1:(1:a) = a$. Phép toán ngược của phép bình phương là phép lấy căn bậc 2. Phép toán ngược của phép lập phương là phép lấy căn bậc 3. Phép toán ngược của phép lũy thừa bậc $n$ là phép lấy căn bậc $n$, $\forall n\in\mathbb{N}^\star$.

\begin{dinhnghia}[Căn bậc 2]
	\emph{Căn bậc 2} của 1 số thực $a$ không âm (i.e., $a\in\mathbb{R}$, $a\ge0$) là số $x\in\mathbb{R}$ sao cho $x^2 = a$.
\end{dinhnghia}

\subsection{Properties \& uses}
``The principal square root function $f:\mathbb{R}_{\ge0}\to\mathbb{R}_{\ge0}$, $f(x) = \sqrt{x}$ (usually just referred to as the ``square root function'') is a function that maps the set of nonnegative real numbers onto itself. In geometrical terms, the square root function maps the area of a square to its side length.
\begin{figure}[H]
	\centering
	\includegraphics[scale=0.4]{square_root}
	\caption{Graph of the function $y = f(x) = \sqrt{x}$.}
\end{figure}
The square root of $x$ is rational iff $x$ is a rational number that can be represented as a ratio of 2 perfect squares. (See \href{https://en.wikipedia.org/wiki/Square_root_of_2}{square root of 2} for proofs that this is an irrational number, \& \href{https://en.wikipedia.org/wiki/Quadratic_irrational}{quadratic irrational} for a proof foor all non-square natural numbers.) The square root function maps rational numbers into \href{https://en.wikipedia.org/wiki/Algebraic_number}{algebraic numbers}, the latter being a \href{https://en.wikipedia.org/wiki/Superset}{superset} of the rational numbers.'' -- \href{https://en.wikipedia.org/wiki/Square_root#Properties_and_uses}{Wikipedia\texttt{/}square root\texttt{/}properties \& use}

Some identities involving square roots:
\begin{equation*}
	\boxed{\sqrt{x^2} = |x| = \left\{\begin{split}
		&x,&&\mbox{if } x\ge0,\\
		-&x,&&\mbox{if } x < 0,
	\end{split}\right.\ \sqrt{x} = x^{\frac{1}{2}},\ \forall x\in\mathbb{R}.}
\end{equation*}
The square root of a product of some nonnegative reals is the product of their square roots:
\begin{equation*}
	\boxed{\left.\begin{split}
		\sqrt{xy} &= \sqrt{x}\sqrt{y},\ \forall x,y\in\mathbb{R}_{\ge0},\\
		\sqrt{\prod_{i=1}^n x_i} = \prod_{i=1}^n \sqrt{x_i},\mbox{ i.e., } \sqrt{x_1x_2\cdots x_n} &= \sqrt{x_1}\sqrt{x_2}\cdots\sqrt{x_n},\ \forall x_i\in\mathbb{R}_{\ge0},\ \forall i = 1,2,\ldots,n.
	\end{split}\right.}
\end{equation*}
The square root of a quotient of a nonnegative real \& a positive real is the quotient of their square roots:
\begin{align*}
	\boxed{\sqrt{\frac{x}{y}} = \frac{\sqrt{x}}{\sqrt{y}},\ \forall x\in\mathbb{R}_{\ge0},\,\forall y\in\mathbb{R}_{> 0}.}
\end{align*}

\subsection{Square roots of positive integers}
``A positive number has 2 square roots, 1 positive, \& 1 negative, which are \href{https://en.wikipedia.org/wiki/Opposite_(mathematics)}{opposite} to each other. When talking of \textit{the} square root of a positive integer, it is usually the positive square root that is meant.

The square roots of an integer are \href{https://en.wikipedia.org/wiki/Algebraic_integer}{algebraic integers} -- more specifically \href{https://en.wikipedia.org/wiki/Quadratic_integer}{quadratic integers}.

The square root of a positive integer is the product of the roots of its \href{https://en.wikipedia.org/wiki/Prime_number}{prime} factors, because the square root of a product is the product of the square roost of the factors. Since $\sqrt{p^{2k}} = p^k$, only roots of those primes having an odd power in the \href{https://en.wikipedia.org/wiki/Integer_factorization}{factorization} are necessary. More precisely, the square root of a prime factorization is
\begin{align*}
	\sqrt{p_1^{2e_1 + 1}\cdots p_k^{2e_k + 1}p_{k+1}^{2e_{k+1}}\cdots p_n^{2e_n}} = p_1^{e_1}\cdots p_n^{e_n}\sqrt{p_1\cdots p_k},\ \forall p_i:\mbox{prime},\,\forall e_i\in\mathbb{N},\,\forall i = 1,2,\ldots,n.
\end{align*}

%------------------------------------------------------------------------------%

\section{Căn Bậc 2 \& Hằng Đẳng Thức $\sqrt{A^2} = |A|$}

%------------------------------------------------------------------------------%

\section{Liên Hệ Giữa Phép Nhân \& Phép Khai Phương}

%------------------------------------------------------------------------------%

\section{Liên Hệ Giữa Phép Chia \& Phép Khai Phương}

%------------------------------------------------------------------------------%

\section{Biến Đổi Đơn Giản Biểu Thức Chứa Căn Thức Bậc 2}

%------------------------------------------------------------------------------%

\section{Rút Gọn Biểu Thức Chứa Căn Thức Bậc 2}

\noindent\fbox{%
	\parbox{\textwidth}{%
		\noindent\textsf{\textbf{Kiến thức cơ bản.}} \fbox{\bf 1} Với số $a\in\mathbb{R}$, $a\ge0$, số $b\in\mathbb{R}$ được gọi là \textit{căn bậc 2} của số $a$ nếu $b^2 = a$. \fbox{\bf 2} Số $a < 0$ không có căn bậc 2. Số $a = 0$ chỉ có 1 căn bậc 2 là số 0. Số $a > 0$ có đúng 2 căn bậc 2 là số $b$ \& số $-b$ (có thể gom lại thành $\pm b$) trong đó $b$ được chọn là số dương, $b > 0$, ký hiệu bởi $\sqrt{a}$, \& được gọi là \textit{căn bậc 2 số học} của $a$. \fbox{\bf 3} Với biểu thức đại số $A$, biểu thức đại số $B$ không âm được gọi là \textit{căn bậc 2} của $A$, ký hiệu $B = \sqrt{A}$, nếu $B^2 = A$, $A$ được gọi là \textit{biểu thức dưới dấu căn bậc 2}. \fbox{\bf 4} Điều kiện để $A$ có căn bậc 2 là $A\ge0$. \fbox{\bf 5} Với biểu thức đại số $A$, ta luôn có $\sqrt{|A^2|} = |A|$. \fbox{\bf 6} Với 2 biểu thức đại số $A,B$ không âm, ta luôn có $\sqrt{AB} = \sqrt{A}\sqrt{B}$, $\sqrt{C^2B} = |C|\sqrt{B}$. \fbox{\bf 7} Với biểu thức đại số $A,B$ thỏa mãn $B\ne0$, $AB\ge0$ luôn có: $\sqrt{\frac{A}{B}} = \frac{\sqrt{|A|}}{\sqrt{|B|}}$, $\frac{A}{B} = \frac{\sqrt{AB}}{|B|}$.
	}%
}

\begin{baitoan}[\cite{TLCT_THCS_Toan_9_dai_so}, Ví dụ 1.1, p. 5]
	Rút gọn biểu thức: $\sqrt{(7 + 4\sqrt{3})(a - 1)^2}$.
\end{baitoan}

\begin{proof}[Giải]
	$\sqrt{(7 + 4\sqrt{3})(a - 1)^2} = \sqrt{7 + 4\sqrt{3}}\sqrt{(a - 1)^2} = \sqrt{(2 + \sqrt{3})^2}\sqrt{(a - 1)^2} = |2 + \sqrt{3}||a - 1| = (2 + \sqrt{3})|a - 1|$.
\end{proof}

\begin{luuy}
	Đẳng thức: \fbox{$(a + b\sqrt{c})^2 = a^2 + 2ab\sqrt{c} + b^2c = (a^2 + b^2c) + 2ab\sqrt{c}$, $\forall a,b,c\in\mathbb{R}$, $c\ge0$.}
\end{luuy}

\begin{baitoan}
	Cho $a,b,c,A,B\in\mathbb{Z}$, $c\ge0$ thỏa mãn đẳng thức $(a + b\sqrt{c})^2 = A + B\sqrt{c}$. (a) Tìm mối quan hệ của $a,b,c,A,B$. Biểu diễn $(A,B)$ theo $(a,b,c)$. (b)${}^\star$ Biểu diễn $(a,b)$ theo $(c,A,B)$.
\end{baitoan}

\begin{baitoan}[\cite{TLCT_THCS_Toan_9_dai_so}, Ví dụ 1.2, p. 6]
	Rút gọn biểu thức: $A = \sqrt{a + 2\sqrt{a - 1}} + \sqrt{a - 2\sqrt{a - 1}}$ với $1 < a < 2$.
\end{baitoan}

\begin{proof}[Giải]
	$A = \sqrt{a + 2\sqrt{a - 1}} + \sqrt{a - 2\sqrt{a - 1}} = \sqrt{(\sqrt{a - 1} + 1)^2} + \sqrt{(\sqrt{a - 1} - 1)^2} = |\sqrt{a - 1} + 1| + |\sqrt{a - 1} - 1| = \sqrt{a - 1} + 1 + + 1 - \sqrt{a - 1} = 2$, trong đó $|\sqrt{a - 1} - 1| = 1 - \sqrt{a - 1}$ vì $1 < a < 2$ nên $\sqrt{a - 1} - 1 < \sqrt{2 - 1} - 1 = \sqrt{1} - 1 = 0$.
\end{proof}


%------------------------------------------------------------------------------%

\section{Cube Root -- Căn Bậc 3}

\begin{definition}[Cube root]
	In mathematics, a \emph{cube root} of a real number $x\in\mathbb{R}$ is a real number $y\in\mathbb{R}$ such that $y^3 = x$.
\end{definition}

\begin{luuy}
	Đẳng thức: \fbox{$(a + b\sqrt[3]{c})^3 = a^3 + 3a^2b\sqrt[3]{c} + 3ab^2\sqrt[3]{c^2} + b^3c = (a^3 + b^3c) + 3a^2b\sqrt[3]{c} + 3ab^2\sqrt[3]{c^2}$, $\forall a,b,c\in\mathbb{R}$.}
\end{luuy}

\begin{baitoan}
	Cho $a,b,c,A,B\in\mathbb{Z}$, $c\ge0$ thỏa mãn đẳng thức $(a + b\sqrt[3]{c})^3 = A + B\sqrt[3]{c} + C\sqrt[3]{c^2}$. (a) Tìm mối quan hệ của $a,b,c,A,B,C$. Biểu diễn $(A,B,C)$ theo $(a,b,c)$. (b)${}^\star$ Biểu diễn $(a,b)$ theo $(c,A,B,C)$.
\end{baitoan}

%------------------------------------------------------------------------------%

\section{$n$th Root -- Căn Bậc $n$}

\begin{definition}[$n$th root]
	``In mathematics, a \emph{$n$th root} of a real number $x\in\mathbb{R}$ is a number $y$ which, when raised to the power $n$, yields $x$: $y^n = x$, where $n\in\mathbb{N}^\star$, sometimes called the \emph{degree} of the root.
\end{definition}
A root of degree 2 is called a \href{https://en.wikipedia.org/wiki/Square_root}{\textit{square root}} \& a root of degree 3, a \href{https://en.wikipedia.org/wiki/Cube_root}{cube root}. Roots of higher degree are referred by using ordinal numbers, as in \textit{4th root, 20th root}, etc. The computation of an $n$th root is a \textit{root extraction}.

%------------------------------------------------------------------------------%

\section{Miscellaneous}

%------------------------------------------------------------------------------%

\printbibliography[heading=bibintoc]
	
\end{document}