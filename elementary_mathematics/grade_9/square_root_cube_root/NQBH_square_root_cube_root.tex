\documentclass{article}
\usepackage[backend=biber,natbib=true,style=authoryear,maxbibnames=10]{biblatex}
\addbibresource{/home/nqbh/reference/bib.bib}
\usepackage[utf8]{vietnam}
\usepackage{tocloft}
\renewcommand{\cftsecleader}{\cftdotfill{\cftdotsep}}
\usepackage[colorlinks=true,linkcolor=blue,urlcolor=red,citecolor=magenta]{hyperref}
\usepackage{amsmath,amssymb,amsthm,float,graphicx,mathtools,soul,subcaption}
\allowdisplaybreaks
\newtheorem{assumption}{Assumption}
\newtheorem{baitoan}{Bài toán}
\newtheorem{cauhoi}{Câu hỏi}
\newtheorem{conjecture}{Conjecture}
\newtheorem{corollary}{Corollary}
\newtheorem{dangtoan}{Dạng toán}
\newtheorem{definition}{Definition}
\newtheorem{dinhly}{Định lý}
\newtheorem{dinhnghia}{Định nghĩa}
\newtheorem{example}{Example}
\newtheorem{ghichu}{Ghi chú}
\newtheorem{hequa}{Hệ quả}
\newtheorem{hypothesis}{Hypothesis}
\newtheorem{lemma}{Lemma}
\newtheorem{luuy}{Lưu ý}
\newtheorem{nhanxet}{Nhận xét}
\newtheorem{notation}{Notation}
\newtheorem{note}{Note}
\newtheorem{principle}{Principle}
\newtheorem{problem}{Problem}
\newtheorem{proposition}{Proposition}
\newtheorem{question}{Question}
\newtheorem{remark}{Remark}
\newtheorem{theorem}{Theorem}
\newtheorem{vidu}{Ví dụ}
\usepackage[left=1cm,right=1cm,top=5mm,bottom=5mm,footskip=4mm]{geometry}
\def\labelitemii{$\circ$}
\DeclareRobustCommand{\divby}{%
	\mathrel{\vbox{\baselineskip.65ex\lineskiplimit0pt\hbox{.}\hbox{.}\hbox{.}}}%
}

\title{Square Root \& Cube Root -- Căn Bậc 2 \& Căn Bậc 3}
\author{Nguyễn Quản Bá Hồng\footnote{Independent Researcher, Ben Tre City, Vietnam\\e-mail: \texttt{nguyenquanbahong@gmail.com}; website: \url{https://nqbh.github.io}.}}
\date{\today}

\begin{document}
\maketitle
\begin{abstract}
	\textsc{[en]} This text is a collection of problems, from easy to advanced, about \textit{square root \& cube root}. This text is also a supplementary material for my lecture note on Elementary Mathematics grade 9, which is stored \& downloadable at the following link: \href{https://github.com/NQBH/hobby/blob/master/elementary_mathematics/grade_9/NQBH_elementary_mathematics_grade_9.pdf}{GitHub\texttt{/}NQBH\texttt{/}hobby\texttt{/}elementary mathematics\texttt{/}grade 9\texttt{/}lecture}\footnote{\textsc{url}: \url{https://github.com/NQBH/hobby/blob/master/elementary_mathematics/grade_9/NQBH_elementary_mathematics_grade_9.pdf}.}. The latest version of this text has been stored \& downloadable at the following link: \href{https://github.com/NQBH/hobby/blob/master/elementary_mathematics/grade_9/square_root_cube_root/NQBH_square_root_cube_root.pdf}{GitHub\texttt{/}NQBH\texttt{/}hobby\texttt{/}elementary mathematics\texttt{/}grade 9\texttt{/}square root \& cube root}\footnote{\textsc{url}: \url{https://github.com/NQBH/hobby/blob/master/elementary_mathematics/grade_9/similar_triangle/NQBH_square_root_cube_root.pdf}.}.
	\vspace{2mm}
	
	\textsc{[vi]} Tài liệu này là 1 bộ sưu tập các bài tập chọn lọc từ cơ bản đến nâng cao về \textit{các tam giác đồng dạng}. Tài liệu này là phần bài tập bổ sung cho tài liệu chính -- bài giảng \href{https://github.com/NQBH/hobby/blob/master/elementary_mathematics/grade_9/NQBH_elementary_mathematics_grade_9.pdf}{GitHub\texttt{/}NQBH\texttt{/}hobby\texttt{/}elementary mathematics\texttt{/}grade 9\texttt{/}lecture} của tác giả viết cho Toán Sơ Cấp lớp 9. Phiên bản mới nhất của tài liệu này được lưu trữ \& có thể tải xuống ở link sau: \href{https://github.com/NQBH/hobby/blob/master/elementary_mathematics/grade_9/square_root_cube_root/NQBH_square_root_cube_root.pdf}{GitHub\texttt{/}NQBH\texttt{/}hobby\texttt{/}elementary mathematics\texttt{/}grade 9\texttt{/}square root \& cube root}.
	
	\textsf{\textbf{Nội dung.} Định lý Thales, tam giác đồng dạng.}
\end{abstract}
\tableofcontents
\newpage

%------------------------------------------------------------------------------%

\section{Square Root -- Căn Bậc 2}

%------------------------------------------------------------------------------%

\section{Căn Bậc 2 \& Hằng Đẳng Thức $\sqrt{A^2} = |A|$}

%------------------------------------------------------------------------------%

\section{Liên Hệ Giữa Phép Nhân \& Phép Khai Phương}

%------------------------------------------------------------------------------%

\section{Liên Hệ Giữa Phép Chia \& Phép Khai Phương}

%------------------------------------------------------------------------------%

\section{Biến Đổi Đơn Giản Biểu Thức Chứa Căn Thức Bậc 2}

%------------------------------------------------------------------------------%

\section{Rút Gọn Biểu Thức Chứa Căn Thức Bậc 2}

\noindent\fbox{%
	\parbox{\textwidth}{%
		\noindent\textsf{\textbf{Kiến thức cơ bản.}} \fbox{\bf 1} Với số $a\in\mathbb{R}$, $a\ge0$, số $b\in\mathbb{R}$ được gọi là \textit{căn bậc 2} của số $a$ nếu $b^2 = a$. \fbox{\bf 2} Số $a < 0$ không có căn bậc 2. Số $a = 0$ chỉ có 1 căn bậc 2 là số 0. Số $a > 0$ có đúng 2 căn bậc 2 là số $b$ \& số $-b$ (có thể gom lại thành $\pm b$) trong đó $b$ được chọn là số dương, $b > 0$, ký hiệu bởi $\sqrt{a}$, \& được gọi là \textit{căn bậc 2 số học} của $a$. \fbox{\bf 3} Với biểu thức đại số $A$, biểu thức đại số $B$ không âm được gọi là \textit{căn bậc 2} của $A$, ký hiệu $B = \sqrt{A}$, nếu $B^2 = A$, $A$ được gọi là \textit{biểu thức dưới dấu căn bậc 2}. \fbox{\bf 4} Điều kiện để $A$ có căn bậc 2 là $A\ge0$. \fbox{\bf 5} Với biểu thức đại số $A$, ta luôn có $\sqrt{|A^2|} = |A|$. \fbox{\bf 6} Với 2 biểu thức đại số $A,B$ không âm, ta luôn có $\sqrt{AB} = \sqrt{A}\sqrt{B}$, $\sqrt{C^2B} = |C|\sqrt{B}$. \fbox{\bf 7} Với biểu thức đại số $A,B$ thỏa mãn $B\ne0$, $AB\ge0$ luôn có: $\sqrt{\frac{A}{B}} = \frac{\sqrt{|A|}}{\sqrt{|B|}}$, $\frac{A}{B} = \frac{\sqrt{AB}}{|B|}$.
	}%
}

\begin{baitoan}[\cite{TLCT_THCS_Toan_9_dai_so}, Ví dụ 1.1, p. 5]
	Rút gọn biểu thức: $\sqrt{(7 + 4\sqrt{3})(a - 1)^2}$.
\end{baitoan}

\begin{proof}[Giải]
	$\sqrt{(7 + 4\sqrt{3})(a - 1)^2} = \sqrt{7 + 4\sqrt{3}}\sqrt{(a - 1)^2} = \sqrt{(2 + \sqrt{3})^2}\sqrt{(a - 1)^2} = |2 + \sqrt{3}||a - 1| = (2 + \sqrt{3})|a - 1|$.
\end{proof}

\begin{luuy}
	Đẳng thức: \fbox{$(a + b\sqrt{c})^2 = a^2 + 2ab\sqrt{c} + b^2c = (a^2 + b^2c) + 2ab\sqrt{c}$, $\forall a,b,c\in\mathbb{R}$, $c\ge0$.}
\end{luuy}

\begin{baitoan}
	Cho $a,b,c,A,B\in\mathbb{Z}$, $c\ge0$ thỏa mãn đẳng thức $(a + b\sqrt{c})^2 = A + B\sqrt{c}$. (a) Tìm mối quan hệ của $a,b,c,A,B$. Biểu diễn $(A,B)$ theo $(a,b,c)$. (b)${}^\star$ Biểu diễn $(a,b)$ theo $(c,A,B)$.
\end{baitoan}

%------------------------------------------------------------------------------%

\section{Cube Root -- Căn Bậc 3}

\begin{luuy}
	Đẳng thức: \fbox{$(a + b\sqrt[3]{c})^3 = a^3 + 3a^2b\sqrt[3]{c} + 3ab^2\sqrt[3]{c^2} + b^3c = (a^3 + b^3c) + 3a^2b\sqrt[3]{c} + 3ab^2\sqrt[3]{c^2}$, $\forall a,b,c\in\mathbb{R}$.}
\end{luuy}

\begin{baitoan}
	Cho $a,b,c,A,B\in\mathbb{Z}$, $c\ge0$ thỏa mãn đẳng thức $(a + b\sqrt[3]{c})^3 = A + B\sqrt[3]{c} + C\sqrt[3]{c^2}$. (a) Tìm mối quan hệ của $a,b,c,A,B,C$. Biểu diễn $(A,B,C)$ theo $(a,b,c)$. (b)${}^\star$ Biểu diễn $(a,b)$ theo $(c,A,B,C)$.
\end{baitoan}

%------------------------------------------------------------------------------%

\printbibliography[heading=bibintoc]
	
\end{document}