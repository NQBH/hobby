\documentclass{article}
\usepackage[backend=biber,natbib=true,style=alphabetic,maxbibnames=10]{biblatex}
\addbibresource{/home/nqbh/reference/bib.bib}
\usepackage[utf8]{vietnam}
\usepackage{tocloft}
\renewcommand{\cftsecleader}{\cftdotfill{\cftdotsep}}
\usepackage[colorlinks=true,linkcolor=blue,urlcolor=red,citecolor=magenta]{hyperref}
\usepackage{amsmath,amssymb,amsthm,float,graphicx,mathtools}
\allowdisplaybreaks
\newtheorem{assumption}{Assumption}
\newtheorem{baitoan}{Bài toán}
\newtheorem{cauhoi}{Câu hỏi}
\newtheorem{conjecture}{Conjecture}
\newtheorem{corollary}{Corollary}
\newtheorem{dangtoan}{Dạng toán}
\newtheorem{definition}{Definition}
\newtheorem{dinhly}{Định lý}
\newtheorem{dinhnghia}{Định nghĩa}
\newtheorem{example}{Example}
\newtheorem{ghichu}{Ghi chú}
\newtheorem{hequa}{Hệ quả}
\newtheorem{hypothesis}{Hypothesis}
\newtheorem{lemma}{Lemma}
\newtheorem{luuy}{Lưu ý}
\newtheorem{nhanxet}{Nhận xét}
\newtheorem{notation}{Notation}
\newtheorem{note}{Note}
\newtheorem{principle}{Principle}
\newtheorem{problem}{Problem}
\newtheorem{proposition}{Proposition}
\newtheorem{question}{Question}
\newtheorem{remark}{Remark}
\newtheorem{theorem}{Theorem}
\newtheorem{vidu}{Ví dụ}
\usepackage[left=1cm,right=1cm,top=5mm,bottom=5mm,footskip=4mm]{geometry}
\def\labelitemii{$\circ$}
\DeclareRobustCommand{\divby}{%
	\mathrel{\vbox{\baselineskip.65ex\lineskiplimit0pt\hbox{.}\hbox{.}\hbox{.}}}%
}

\title{Cheatsheet: Square-, Cube-, \textit{\&} $n$th Roots\\Bảng Tóm Tắt Công Thức: Căn Bậc 2, Căn Bậc 3, \textit{\&} Căn Bậc $n$}
\author{Nguyễn Quản Bá Hồng\footnote{Independent Researcher, Ben Tre City, Vietnam\\e-mail: \texttt{nguyenquanbahong@gmail.com}; website: \url{https://nqbh.github.io}.}}
\date{\today}

\begin{document}
\maketitle
\begin{abstract}
	Cheatsheet for square-, cube-, \& $n$th roots.
\end{abstract}
\tableofcontents

%------------------------------------------------------------------------------%

\fbox{\bf 1} Với số $a\in\mathbb{R}$, $a\ge0$, số $b\in\mathbb{R}$ được gọi là \textit{căn bậc 2} của số $a$ nếu $b^2 = a$. \fbox{\bf 2} Số $a < 0$ không có căn bậc 2. Số $a = 0$ chỉ có 1 căn bậc 2 là số 0. Số $a > 0$ có đúng 2 căn bậc 2 là số $b$ \& số $-b$ (có thể gom lại thành $\pm b$) trong đó $b$ được chọn là số dương, $b > 0$, ký hiệu bởi $\sqrt{a}$, \& được gọi là \textit{căn bậc 2 số học} của $a$. \fbox{\bf 3} Với biểu thức đại số $A$, biểu thức đại số $B$ không âm được gọi là \textit{căn bậc 2} của $A$, ký hiệu $B = \sqrt{A}$, nếu $B^2 = A$, $A$ được gọi là \textit{biểu thức dưới dấu căn bậc 2}. \fbox{\bf 4} Điều kiện để $A$ có căn bậc 2 là $A\ge0$. \fbox{\bf 5} Với biểu thức đại số $A$, ta luôn có $\sqrt{|A^2|} = |A|$. \fbox{\bf 6} Với 2 biểu thức đại số $A,B$ không âm, ta luôn có $\sqrt{AB} = \sqrt{A}\sqrt{B}$, $\sqrt{C^2B} = |C|\sqrt{B}$. \fbox{\bf 7} Với biểu thức đại số $A,B$ thỏa mãn $B\ne0$, $AB\ge0$ luôn có: $\sqrt{\frac{A}{B}} = \frac{\sqrt{|A|}}{\sqrt{|B|}}$, $\frac{A}{B} = \frac{\sqrt{AB}}{|B|}$.

%------------------------------------------------------------------------------%

\printbibliography[heading=bibintoc]

\end{document}