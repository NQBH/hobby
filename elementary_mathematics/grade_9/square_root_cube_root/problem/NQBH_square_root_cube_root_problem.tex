\documentclass{article}
\usepackage[backend=biber,natbib=true,style=authoryear,maxbibnames=10]{biblatex}
\addbibresource{/home/nqbh/reference/bib.bib}
\usepackage[utf8]{vietnam}
\usepackage{tocloft}
\renewcommand{\cftsecleader}{\cftdotfill{\cftdotsep}}
\usepackage[colorlinks=true,linkcolor=blue,urlcolor=red,citecolor=magenta]{hyperref}
\usepackage{amsmath,amssymb,amsthm,float,graphicx,mathtools}
\allowdisplaybreaks
\newtheorem{assumption}{Assumption}
\newtheorem{baitoan}{Bài toán}
\newtheorem{cauhoi}{Câu hỏi}
\newtheorem{conjecture}{Conjecture}
\newtheorem{corollary}{Corollary}
\newtheorem{dangtoan}{Dạng toán}
\newtheorem{definition}{Definition}
\newtheorem{dinhly}{Định lý}
\newtheorem{dinhnghia}{Định nghĩa}
\newtheorem{example}{Example}
\newtheorem{ghichu}{Ghi chú}
\newtheorem{hequa}{Hệ quả}
\newtheorem{hypothesis}{Hypothesis}
\newtheorem{lemma}{Lemma}
\newtheorem{luuy}{Lưu ý}
\newtheorem{nhanxet}{Nhận xét}
\newtheorem{notation}{Notation}
\newtheorem{note}{Note}
\newtheorem{principle}{Principle}
\newtheorem{problem}{Problem}
\newtheorem{proposition}{Proposition}
\newtheorem{question}{Question}
\newtheorem{remark}{Remark}
\newtheorem{theorem}{Theorem}
\newtheorem{vidu}{Ví dụ}
\usepackage[left=1cm,right=1cm,top=5mm,bottom=5mm,footskip=4mm]{geometry}
\def\labelitemii{$\circ$}
\DeclareRobustCommand{\divby}{%
	\mathrel{\vbox{\baselineskip.65ex\lineskiplimit0pt\hbox{.}\hbox{.}\hbox{.}}}%
}

\title{Problem: Square Root \& Cube Root -- Bài Tập Căn Bậc 2 \& 3}
\author{Nguyễn Quản Bá Hồng\footnote{Independent Researcher, Ben Tre City, Vietnam\\e-mail: \texttt{nguyenquanbahong@gmail.com}; website: \url{https://nqbh.github.io}.}}
\date{\today}

\begin{document}
\maketitle
\begin{abstract}
	
\end{abstract}
\tableofcontents

%------------------------------------------------------------------------------%

\section{Square Root \& Irrationals -- Căn Bậc 2 \& Số Vô Tỷ}

\begin{baitoan}[\cite{Binh_Toan_9_tap_1}, Ví dụ 2, p. 5]
	Chứng minh tổng của 1 số hữu tỷ với 1 số vô tỷ là 1 số vô tỷ.
\end{baitoan}

\begin{baitoan}[\cite{Binh_Toan_9_tap_1}, Ví dụ 3, p. 5]
	Xét xem các số $a,b$ có thể là số vô tỷ hay không, nếu: (a) $a + b$ \& $a - b$ là các số hữu tỷ. (b) $a - b$ \& $ab$ là các số hữu tỷ.
\end{baitoan}

\begin{baitoan}[\cite{Binh_Toan_9_tap_1}, Ví dụ 4, p. 5]
	Chứng minh: Nếu số tự nhiên $a$ không là số chính phương thì $\sqrt{a}$ là số vô tỷ.
\end{baitoan}

\begin{baitoan}[\cite{Binh_Toan_9_tap_1}, 2., p. 6]
	Chứng minh các số sau là số vô tỷ: (a) $\sqrt{1 + \sqrt{2}}$. (b) $m + \frac{\sqrt{3}}{n}$ với $m,n\in\mathbb{Q}$, $n\ne0$.
\end{baitoan}

\begin{baitoan}[\cite{Binh_Toan_9_tap_1}, 3., p. 6]
	Xét xem các số $a,b$ có thể là số vô tỷ hay không nếu: (a) $ab$ \& $\frac{a}{b}$ là các số hữu tỷ. (b) $a + b$ \& $\frac{a}{b}$ là các số hữu tỷ ($a + b\ne0$). (c) $a + b$, $a^2$, \& $b^2$ là các số hữu tỷ ($a + b\ne0$).
\end{baitoan}

\begin{baitoan}[\cite{Binh_Toan_9_tap_1}, 4., p. 6]
	So sánh 2 số: (a) $2\sqrt{3}$ \& $3\sqrt{2}$. (b) $6\sqrt{5}$ \& $5\sqrt{6}$. (c) $\sqrt{24} + \sqrt{45}$ \& $12$. (d) $\sqrt{37} - \sqrt{15}$ \& $2$.
\end{baitoan}

\begin{baitoan}[\cite{Binh_Toan_9_tap_1}, 5., p. 6]
	(a) Cho 1 ví dụ để chứng tỏ khẳng định $\sqrt{a}\le a$ với mọi số $a$ không âm là sai. (b) Cho $a > 0$. Với giá trị nào của $a$ thì $\sqrt{a} ? a$?
\end{baitoan}

\begin{baitoan}[\cite{Binh_Toan_9_tap_1}, $\rm6^\star$., pp. 6--7]
	(a) Chỉ ra 1 số thực $x$ mà $x - \frac{1}{x}$ là số nguyên ($x\ne\pm1$). (b) Chứng minh nếu $x - \frac{1}{x}$ là số nguyên \& $x\ne\pm1$ thì $x$ \& $x + \frac{1}{x}$ là số vô tỷ. Khi đó $\left(x + \frac{1}{x}\right)^{2n}$ \& $\left(x + \frac{1}{x}\right)^{2n+1}$ là số hữu tỷ hay số vô tỷ?
\end{baitoan}

%------------------------------------------------------------------------------%

\section{Căn Thức Bậc 2 \& Hằng Đẳng Thức $\sqrt{A^2} = |A|$}

\begin{baitoan}[\cite{Binh_Toan_9_tap_1}, Ví dụ 5, p. 7]
	Cho biểu thức $A = \sqrt{x - \sqrt{x^2 - 4x + 4}}$. (a) Tìm điều kiện xác định của biểu thức $A$. (b) Rút gọn biểu thức $A$.
\end{baitoan}

\begin{baitoan}[\cite{Binh_Toan_9_tap_1}, Ví dụ 6, p. 8]
	Tìm điều kiện xác định của các biểu thức: (a) $A = \frac{1}{\sqrt{x^2 - 2x - 1}}$. (b) $B = \frac{1}{\sqrt{x - \sqrt{2x + 1}}}$.
\end{baitoan}

\begin{baitoan}[\cite{Binh_Toan_9_tap_1}, Ví dụ 7, p. 8]
	Tìm các giá trị của $x$ sao cho $\sqrt{x + 1} < x + 3$.
\end{baitoan}

\begin{baitoan}[\cite{Binh_Toan_9_tap_1}, 7., p. 9]
	Tìm điều kiện xác định của các biểu thức: (a) $3 - \sqrt{1 - 16x^2}$. (b) $\frac{1}{1 - \sqrt{x^2 - 3}}$. (c) $\sqrt{8x - x^2 - 15}$. (d) $\frac{2}{\sqrt{x^2 - x + 1}}$. (e) $A = \frac{1}{\sqrt{x - \sqrt{2x - 1}}}$. (f) $B = \frac{\sqrt{16 - x^2}}{\sqrt{2x + 1}} + \sqrt{x^2 - 8x + 14}$.
\end{baitoan}

\begin{baitoan}[\cite{Binh_Toan_9_tap_1}, 8., p. 9]
	Cho biểu thức $A = \sqrt{x^2 - 6x + 9} - \sqrt{x^2 + 6x + 9}$. (a) Rút gọn biểu thức $A$. (b) Tìm các giá trị của $x$ để $A = 1$.
\end{baitoan}

\begin{baitoan}[\cite{Binh_Toan_9_tap_1}, 9., p. 9]
	Tìm các giá trị của $x$ sao cho: (a) $\sqrt{x^2 - 3}\le x^2 - 3$. (b) $\sqrt{x^2 - 6x + 9} > x - 6$.
\end{baitoan}

\begin{baitoan}[\cite{Binh_Toan_9_tap_1}, 10., p. 9]
	Cho $a + b + c = 0$ \& $a,b,c\ne0$. Chứng minh hằng đẳng thức: $\sqrt{\frac{1}{a^2} + \frac{1}{b^2} + \frac{1}{c^2}} = \left|\frac{1}{a} + \frac{1}{b} + \frac{1}{c}\right|$.
\end{baitoan}

%------------------------------------------------------------------------------%

\section{Liên Hệ Giữa Phép Nhân, Phép Chia \& Phép Khai Phương}

\begin{baitoan}[\cite{Binh_Toan_9_tap_1}, Ví dụ 8, p. 10]
	Rút gọn biểu thức $A = \sqrt{x + \sqrt{2x - 1}} - \sqrt{ x - \sqrt{2x - 1}}$.
\end{baitoan}

\begin{baitoan}[\cite{Binh_Toan_9_tap_1}, Ví dụ 9, p. 11]
	Chứng minh số $\sqrt{2} + \sqrt{3} + \sqrt{5}$ là số vô tỷ.
\end{baitoan}

\begin{baitoan}[\cite{Binh_Toan_9_tap_1}, 11., pp. 11--12]
	Rút gọn biểu thức: (a) $\sqrt{11 - 2\sqrt{10}}$. (b) $\sqrt{9 - 2\sqrt{14}}$. (c) $\sqrt{4 + 2\sqrt{3}} - \sqrt{4 - 2\sqrt{3}}$. (d) $\sqrt{9 - 4\sqrt{5}} - \sqrt{9 + 4\sqrt{5}}$. (e) $\sqrt{4 - \sqrt{7}} - \sqrt{4 + \sqrt{7}}$. (f) $\dfrac{\sqrt{3} + \sqrt{11 + 6\sqrt{2}}- \sqrt{5 + 2\sqrt{6}}}{\sqrt{2} + \sqrt{6 + 2\sqrt{5}} - \sqrt{7 + 2\sqrt{10}}}$. (g) $\sqrt{5\sqrt{3} + 5\sqrt{48 - 10\sqrt{7+ 4\sqrt{3}}}}$. (h) $\sqrt{4 + \sqrt{10 + 2\sqrt{5}}} + \sqrt{4 - \sqrt{10 + 2\sqrt{5}}}$. (i) $\sqrt{94 - 42\sqrt{5}} - \sqrt{94 + 42\sqrt{5}}$.
\end{baitoan}

\begin{baitoan}[\cite{Binh_Toan_9_tap_1}, 12., p. 12]
	Tính: (a) $(4 + \sqrt{15})(\sqrt{10} - \sqrt{6})\sqrt{4 - \sqrt{15}}$. (b) $\sqrt{3 - \sqrt{5}}(\sqrt{10} - \sqrt{2})(3 + \sqrt{5})$. (c) $\dfrac{\sqrt{\sqrt{5} + 2} + \sqrt{\sqrt{5} - 2}}{\sqrt{\sqrt{5} + 1}} - \sqrt{3 - 2\sqrt{2}}$.
\end{baitoan}

\begin{baitoan}[\cite{Binh_Toan_9_tap_1}, 13., p. 12]
	Chứng minh các hằng đẳng thức sau với $b\ge0$, $a\ge\sqrt{b}$: (a) $\sqrt{a + \sqrt{b}}\pm\sqrt{a - \sqrt{b}} = \sqrt{2(a\pm\sqrt{a^2 - b})}$. (b) $\sqrt{a\pm\sqrt{b}} = \sqrt{\dfrac{a + \sqrt{a^2 - b}}{2}}\pm\sqrt{\dfrac{a - \sqrt{a^2 - b}}{2}}$.
\end{baitoan}

\begin{baitoan}[\cite{Binh_Toan_9_tap_1}, 14., p. 12]
	Rút gọn biểu thức $A = \sqrt{x + 2\sqrt{2x - 4}} + \sqrt{x - 2\sqrt{2x - 4}}$.
\end{baitoan}

\begin{baitoan}[\cite{Binh_Toan_9_tap_1}, 15., p. 12]
	Cho biểu thức $A = \dfrac{x + \sqrt{x^2 - 2x}}{x - \sqrt{x^2 - 2x}} - \dfrac{x - \sqrt{x^2 - 2x}}{x + \sqrt{x^2 - 2x}}$. (a) Tìm điều kiện xác định của biểu thức $A$. (b) Rút gọn biểu thức $A$. (c) Tìm giá trị của $x$ để $A < 2$.
\end{baitoan}

\begin{baitoan}[\cite{Binh_Toan_9_tap_1}, 16., p. 12]
	Lập 1 phương trình bậc 2 với các hệ số nguyên, trong đó: (a) $2 + \sqrt{3}$ là 1 nghiệm của phương trình. (b) $6 - 4\sqrt{2}$ là 1 nghiệm của phương trình.
\end{baitoan}

\begin{baitoan}[\cite{Binh_Toan_9_tap_1}, 17., p. 13]
	Chứng minh các số sau là số vô tỷ: (a) $\sqrt{3} - \sqrt{2}$. (b) $2\sqrt{2} + \sqrt{3}$.
\end{baitoan}

\begin{baitoan}[\cite{Binh_Toan_9_tap_1}, 18., p. 13]
	Có tồn tại các số hữu tỷ dương $a,b$ hay không nếu: (a) $\sqrt{a} + \sqrt{b} = \sqrt{2}$. (b) $\sqrt{a} + \sqrt{b} = \sqrt{\sqrt{2}}$.
\end{baitoan}

\begin{baitoan}[\cite{Binh_Toan_9_tap_1}, 19., p. 13]
	Cho 3 số $x,y,\sqrt{x} + \sqrt{y}$ là các số hữu tỷ. Chứng minh mỗi số $\sqrt{x},\sqrt{y}$ đều là số hữu tỷ.
\end{baitoan}

\begin{baitoan}[\cite{Binh_Toan_9_tap_1}, 20., p. 13]
	Cho $a,b,c,d$ là các số dương. Chứng minh tồn tại 1 số dương trong 2 số $2a + b - 2\sqrt{cd}$ \& $2c + d - 2\sqrt{ab}$.
\end{baitoan}

\begin{baitoan}[\cite{Binh_Toan_9_tap_1}, $21^\star$., p. 13]
	(a) Rút gọn biểu thức $A = \sqrt{1 + \frac{1}{a^2} + \frac{1}{(a + 1)^2}}$ với $a > 0$. (b) Tính giá trị của tổng $B = \sum_{i=1}^{99} \sqrt{1 + \frac{1}{i^2} + \frac{1}{(i + 1)^2}} = \sqrt{1 + \frac{1}{1^2} + \frac{1}{2^2}} + \sqrt{1 + \frac{1}{2^2} + \frac{1}{3^2}} + \cdots + \sqrt{1 + \frac{1}{99^2} + \frac{1}{100^2}}$.
\end{baitoan}

\begin{baitoan}[\cite{Binh_Toan_9_tap_1}, $22^\star$., p. 13]
	(a) Nêu 1 cách tính nhẩm $997^2$. (b) Tính tổng các chữ số của $A$ biết $\sqrt{A} = 99\ldots96$ (có $100$ chữ số $9$).
\end{baitoan}

%------------------------------------------------------------------------------%

\printbibliography[heading=bibintoc]
	
\end{document}