\documentclass{article}
\usepackage[backend=biber,natbib=true,style=alphabetic,maxbibnames=10]{biblatex}
\addbibresource{/home/nqbh/reference/bib.bib}
\usepackage[utf8]{vietnam}
\usepackage{tocloft}
\renewcommand{\cftsecleader}{\cftdotfill{\cftdotsep}}
\usepackage[colorlinks=true,linkcolor=blue,urlcolor=red,citecolor=magenta]{hyperref}
\usepackage{amsmath,amssymb,amsthm,float,graphicx,mathtools}
\allowdisplaybreaks
\newtheorem{assumption}{Assumption}
\newtheorem{baitoan}{Bài toán}
\newtheorem{cauhoi}{Câu hỏi}
\newtheorem{conjecture}{Conjecture}
\newtheorem{corollary}{Corollary}
\newtheorem{dangtoan}{Dạng toán}
\newtheorem{definition}{Definition}
\newtheorem{dinhly}{Định lý}
\newtheorem{dinhnghia}{Định nghĩa}
\newtheorem{example}{Example}
\newtheorem{ghichu}{Ghi chú}
\newtheorem{hequa}{Hệ quả}
\newtheorem{hypothesis}{Hypothesis}
\newtheorem{lemma}{Lemma}
\newtheorem{luuy}{Lưu ý}
\newtheorem{nhanxet}{Nhận xét}
\newtheorem{notation}{Notation}
\newtheorem{note}{Note}
\newtheorem{principle}{Principle}
\newtheorem{problem}{Problem}
\newtheorem{proposition}{Proposition}
\newtheorem{question}{Question}
\newtheorem{remark}{Remark}
\newtheorem{theorem}{Theorem}
\newtheorem{vidu}{Ví dụ}
\usepackage[left=1cm,right=1cm,top=5mm,bottom=5mm,footskip=4mm]{geometry}
\def\labelitemii{$\circ$}
\DeclareRobustCommand{\divby}{%
	\mathrel{\vbox{\baselineskip.65ex\lineskiplimit0pt\hbox{.}\hbox{.}\hbox{.}}}%
}

\title{Solution: Square-, Cube-, \textit{\&} $n$th Roots\\Lời Giải: Căn Bậc 2, Căn Bậc 3, \textit{\&} Căn Bậc $n$}
\author{Nguyễn Quản Bá Hồng\footnote{Independent Researcher, Ben Tre City, Vietnam\\e-mail: \texttt{nguyenquanbahong@gmail.com}; website: \url{https://nqbh.github.io}.}}
\date{\today}

\begin{document}
\maketitle
\begin{abstract}
	\textsf{[en]} This text is a collection of problems, from basic to advanced, on \textit{square-, cube-, \& $n$th roots}.
	
	\textsf{\textbf{Keyword.} Square root, cube root, $n$th root.}
	\vspace{2mm}
	
	\textsf{[vi]} Tài liệu này là 1 bộ sưu tập các bài toán, từ cơ bản đến nâng cao, về \textit{căn bậc 2, căn bậc 3, \& căn bậc $n$}.
	
	\textsf{\textbf{Từ khóa.} Căn bậc 2, căn bậc 3, căn bậc $n$, số hữu tỷ, số vô tỷ, căn thức.}
	
	\begin{itemize}
		\item Lecture note -- Bài giảng: \href{https://github.com/NQBH/hobby/blob/master/elementary_mathematics/grade_9/square_root_cube_root/NQBH_square_root_cube_root.pdf}{GitHub\texttt{/}NQBH\texttt{/}hobby\texttt{/}elementary mathematics\texttt{/}grade 9\texttt{/}square- \& cube roots}\footnote{\textsc{url}: \url{https://github.com/NQBH/hobby/blob/master/elementary_mathematics/grade_9/square_root_cube_root/NQBH_square_root_cube_root.pdf}.}.
		\item Cheatsheet -- Công thức: \href{https://github.com/NQBH/hobby/blob/master/elementary_mathematics/grade_9/square_root_cube_root/cheatsheet/NQBH_square_root_cube_root_cheatsheet.pdf}{GitHub\texttt{/}NQBH\texttt{/}hobby\texttt{/}elementary mathematics\texttt{/}grade 9\texttt{/}cheatsheet: square- \& cube roots}\footnote{\url{https://github.com/NQBH/hobby/blob/master/elementary_mathematics/grade_9/square_root_cube_root/cheatsheet/NQBH_square_root_cube_root_cheatsheet.pdf}.}.
		\item Problem -- Bài tập: \href{https://github.com/NQBH/hobby/blob/master/elementary_mathematics/grade_9/square_root_cube_root/problem/NQBH_square_root_cube_root_problem.pdf}{GitHub\texttt{/}NQBH\texttt{/}hobby\texttt{/}elementary mathematics\texttt{/}grade 9\texttt{/}problem: square- \& cube roots}\footnote{\url{https://github.com/NQBH/hobby/blob/master/elementary_mathematics/grade_9/square_root_cube_root/problem/NQBH_square_root_cube_root_problem.pdf}.}.
		\item Solution -- Lời giải: \href{https://github.com/NQBH/hobby/blob/master/elementary_mathematics/grade_9/square_root_cube_root/solution/NQBH_square_root_cube_root_solution.pdf}{GitHub\texttt{/}NQBH\texttt{/}hobby\texttt{/}elementary mathematics\texttt{/}grade 9\texttt{/}solution: square- \& cube roots}\footnote{\url{https://github.com/NQBH/hobby/blob/master/elementary_mathematics/grade_9/square_root_cube_root/solution/NQBH_square_root_cube_root_solution.pdf}.}.
	\end{itemize}
\end{abstract}


\tableofcontents
\newpage

%------------------------------------------------------------------------------%

\section{Square Root \& Irrationals -- Căn Bậc 2 \& Số Vô Tỷ}

\begin{baitoan}[Program to print out 1st $n$ square roots]
	Viết chương trình \textsc{Pascal, C\texttt{/}C++, Python} xuất ra căn bậc 2 của $n$ số tự nhiên đầu tiên với $n\in\mathbb{N}^\star$ được nhập từ bàn phím.
\end{baitoan}
Pascal:
\begin{verbatim}
	program square_root;
	var num, sqrt_num: real;		
	begin
	    write('Enter a number num = ');
	    readln(num);
	    sqrt_num := Sqrt(num);
	    writeln('sqrt of ', num,' = ', sqrt_num)
	end.
\end{verbatim}

\begin{baitoan}[Số chính phương]
	Viết chương trình \textsc{Pascal, C\texttt{/}C++, Python} để kiểm tra 1 số $n\in\mathbb{N}^\star$ được nhập từ bàn phím có phải là số chính phương hay không.
\end{baitoan}

\begin{baitoan}[\cite{Tuyen_Toan_9}, Thí dụ 1, p. 5]
	Cho số thực $x\ge0$. So sánh $\sqrt{x}$ với $x$.
\end{baitoan}

\begin{proof}[Giải]
	Vì $x\ge0$ nên $\sqrt{x}$ có nghĩa\texttt{/}xác định \& $\sqrt{x}\ge0$. Xét các trường hợp: (a) $\sqrt{x} = x\Leftrightarrow x = x^2\Leftrightarrow x - x^2 = 0\Leftrightarrow x(1 - x) = 0\Leftrightarrow x = 0$ hoặc $x = 1$. (b) $\sqrt{x} < x\Leftrightarrow x < x^2\Leftrightarrow x - x^2 < 0\Leftrightarrow x(1 - x) < 0$, mà $x\ge0$ nên suy ra $1 - x < 0$, hay $x > 1$. (c) $\sqrt{x} > x\Leftrightarrow x > x^2\Leftrightarrow x - x^2 > 0\Leftrightarrow x(1 - x) > 0\Leftrightarrow 0 < x < 1$. Vậy: $x\in\{0,1\}\Leftrightarrow\sqrt{x} = x$, $x > 1\Leftrightarrow\sqrt{x} < x$, \& $0 < x < 1\Leftrightarrow\sqrt{x} > x$.
\end{proof}

\begin{nhanxet}
	Về mặt phương pháp để so sánh 2 số không âm ta có thể so sánh các bình phương của 2 số đó: $a\ge b > 0\Leftrightarrow a^2\ge b^2$. Về kết quả, khi so sánh $\sqrt{x}$ với $x$ ta thấy có thể xảy ra cả $3$ trường hợp: lớn hơn, nhỏ hơn, hoặc bằng nhau tùy theo $x$ ở trong khoảng giá trị nào, cụ thể: $x\in\{0,1\}\Leftrightarrow\sqrt{x} = x$, $x > 1\Leftrightarrow\sqrt{x} < x$, \& $0 < x < 1\Leftrightarrow\sqrt{x} > x$.
\end{nhanxet}

\begin{baitoan}[\cite{Binh_Toan_9_tap_1}, Ví dụ 2, p. 5]
	Chứng minh tổng của 1 số hữu tỷ với 1 số vô tỷ là 1 số vô tỷ.
\end{baitoan}

\begin{baitoan}[\cite{Binh_Toan_9_tap_1}, Ví dụ 3, p. 5]
	Xét xem các số $a,b$ có thể là số vô tỷ hay không, nếu: (a) $a + b$ \& $a - b$ là các số hữu tỷ. (b) $a - b$ \& $ab$ là các số hữu tỷ.
\end{baitoan}

\begin{baitoan}[\cite{Binh_Toan_9_tap_1}, Ví dụ 4, p. 5]
	Chứng minh: Nếu số tự nhiên $a$ không là số chính phương thì $\sqrt{a}$ là số vô tỷ.
\end{baitoan}

\begin{baitoan}[\cite{Binh_Toan_9_tap_1}, 2., p. 6]
	Chứng minh các số sau là số vô tỷ: (a) $\sqrt{1 + \sqrt{2}}$. (b) $m + \frac{\sqrt{3}}{n}$ với $m,n\in\mathbb{Q}$, $n\ne0$.
\end{baitoan}

\begin{baitoan}[\cite{Binh_Toan_9_tap_1}, 3., p. 6]
	Xét xem các số $a,b$ có thể là số vô tỷ hay không nếu: (a) $ab$ \& $\frac{a}{b}$ là các số hữu tỷ. (b) $a + b$ \& $\frac{a}{b}$ là các số hữu tỷ ($a + b\ne0$). (c) $a + b$, $a^2$, \& $b^2$ là các số hữu tỷ ($a + b\ne0$).
\end{baitoan}

\begin{baitoan}[\cite{Binh_Toan_9_tap_1}, 4., p. 6]
	So sánh 2 số: (a) $2\sqrt{3}$ \& $3\sqrt{2}$. (b) $6\sqrt{5}$ \& $5\sqrt{6}$. (c) $\sqrt{24} + \sqrt{45}$ \& $12$. (d) $\sqrt{37} - \sqrt{15}$ \& $2$.
\end{baitoan}

\begin{baitoan}[\cite{Binh_Toan_9_tap_1}, 5., p. 6]
	(a) Cho 1 ví dụ để chứng tỏ khẳng định $\sqrt{a}\le a$ với mọi số $a$ không âm là sai. (b) Cho $a > 0$. Với giá trị nào của $a$ thì $\sqrt{a} ? a$?
\end{baitoan}

\begin{baitoan}[\cite{Binh_Toan_9_tap_1}, $\rm6^\star$., pp. 6--7]
	(a) Chỉ ra 1 số thực $x$ mà $x - \frac{1}{x}$ là số nguyên ($x\ne\pm1$). (b) Chứng minh nếu $x - \frac{1}{x}$ là số nguyên \& $x\ne\pm1$ thì $x$ \& $x + \frac{1}{x}$ là số vô tỷ. Khi đó $\left(x + \frac{1}{x}\right)^{2n}$ \& $\left(x + \frac{1}{x}\right)^{2n+1}$ là số hữu tỷ hay số vô tỷ?
\end{baitoan}

%------------------------------------------------------------------------------%

\section{Căn Thức Bậc 2 \& Hằng Đẳng Thức $\sqrt{A^2} = |A|$}

\begin{baitoan}[\cite{Tuyen_Toan_9}, Thí dụ 2, p. 5]
	\label{prob: Tuyen_Toan_9 VD 2 p. 5}
	Cho $a,b,c\in\mathbb{Q}$, $abc\ne0$ \& $a = b + c$. Chứng minh $A = \sqrt{\dfrac{1}{a^2} + \dfrac{1}{b^2} + \dfrac{1}{c^2}}\in\mathbb{Q}$.
\end{baitoan}

\begin{proof}[Giải]
	$\dfrac{1}{a^2} + \dfrac{1}{b^2} + \dfrac{1}{c^2} = \left(\dfrac{1}{a} - \dfrac{1}{b} - \dfrac{1}{c}\right)^2 + 2\left(\dfrac{1}{ab} + \dfrac{1}{ac} - \dfrac{1}{bc}\right) = \left(\dfrac{1}{a} - \dfrac{1}{b} - \dfrac{1}{c}\right)^2 + \dfrac{2(c + b - a)}{abc} = \left(\dfrac{1}{a} - \dfrac{1}{b} - \dfrac{1}{c}\right)^2$ vì $a = b + c$. Suy ra $A = \sqrt{\dfrac{1}{a^2} + \dfrac{1}{b^2} + \dfrac{1}{c^2}} = \sqrt{\left(\dfrac{1}{a} - \dfrac{1}{b} - \dfrac{1}{c}\right)^2} = \left|\dfrac{1}{a} - \dfrac{1}{b} - \dfrac{1}{c}\right|$. Có $a,b,c\in\mathbb{Q}^\star\Rightarrow\dfrac{1}{a},\dfrac{1}{b},\dfrac{1}{c}\in\mathbb{Q}\Rightarrow A = \left|\dfrac{1}{a} - \dfrac{1}{b} - \dfrac{1}{c}\right|\in\mathbb{Q}$.
\end{proof}

\begin{baitoan}
	\label{prob: Mo rong Tuyen_Toan_9 VD 2 p. 5}
	Cho $a,b,c\in\mathbb{Q}$, $abc\ne0$ \& $a + b + c = 0$. Chứng minh $A = \sqrt{\dfrac{1}{a^2} + \dfrac{1}{b^2} + \dfrac{1}{c^2}}\in\mathbb{Q}$.
\end{baitoan}

\begin{proof}[1st giải]
	$\dfrac{1}{a^2} + \dfrac{1}{b^2} + \dfrac{1}{c^2} = \left(\dfrac{1}{a} + \dfrac{1}{b} + \dfrac{1}{c}\right)^2 - 2\left(\dfrac{1}{ab} + \dfrac{1}{bc} + \dfrac{1}{ca}\right) = \left(\dfrac{1}{a} + \dfrac{1}{b} + \dfrac{1}{c}\right)^2 - \dfrac{2(a + b + c)}{abc} = \left(\dfrac{1}{a} + \dfrac{1}{b} + \dfrac{1}{c}\right)^2$ vì $a + b + c = 0$. Suy ra $A = \sqrt{\dfrac{1}{a^2} + \dfrac{1}{b^2} + \dfrac{1}{c^2}} = \sqrt{\left(\dfrac{1}{a} + \dfrac{1}{b} + \dfrac{1}{c}\right)^2} = \left|\dfrac{1}{a} + \dfrac{1}{b} + \dfrac{1}{c}\right|$. Có $a,b,c\in\mathbb{Q}^\star\Rightarrow\dfrac{1}{a},\dfrac{1}{b},\dfrac{1}{c}\in\mathbb{Q}\Rightarrow A =\left|\dfrac{1}{a} + \dfrac{1}{b} + \dfrac{1}{c}\right|\in\mathbb{Q}$.
\end{proof}

\begin{proof}[2nd giải]
	$a + b + c = 0\Leftrightarrow -a = b + c$, nên ta có thể áp dụng bài toán \ref{prob: Tuyen_Toan_9 VD 2 p. 5} cho bộ 3 số $(-a,b,c)\in\mathbb{Q}^3$, $-abc\ne0$ để thu được $\sqrt{\dfrac{1}{(-a)^2} + \dfrac{1}{b^2} + \dfrac{1}{c^2}}\in\mathbb{Q}$, i.e., $A = \sqrt{\dfrac{1}{a^2} + \dfrac{1}{b^2} + \dfrac{1}{c^2}}\in\mathbb{Q}$.
\end{proof}

\begin{nhanxet}[Proof of $\in\mathbb{Q}$]
	Để chứng minh 1 số là số hữu tỷ ta biểu diễn số đó thành 1 biểu thức gồm các phép tính cộng, trừ, nhân, chia (cho 1 số khác $0$) của các số hữu tỷ.
\end{nhanxet}

\begin{baitoan}
	(a) Cho $a,b,c\in\mathbb{R}$, $abc\ne0$, khi nào thì $\left(\dfrac{1}{a} + \dfrac{1}{b} + \dfrac{1}{c}\right)^2 = \dfrac{1}{a^2} + \dfrac{1}{b^2} + \dfrac{1}{c^2}$? (b) Cho $a,b,c,d\in\mathbb{R}$, $abcd\ne0$, khi nào thì $\left(\dfrac{1}{a} + \dfrac{1}{b} + \dfrac{1}{c} + \dfrac{1}{d}\right)^2 = \dfrac{1}{a^2} + \dfrac{1}{b^2} + \dfrac{1}{c^2} + \dfrac{1}{d^2}$? (c) Cho $a,b,c,d,e\in\mathbb{R}$, $abcde\ne0$, khi nào thì $\left(\dfrac{1}{a} + \dfrac{1}{b} + \dfrac{1}{c} + \dfrac{1}{d} + \dfrac{1}{e}\right)^2 = \dfrac{1}{a^2} + \dfrac{1}{b^2} + \dfrac{1}{c^2} + \dfrac{1}{d^2} + \dfrac{1}{e^2}$? (d) Cho $n\in\mathbb{N}^\star$, $a_i\in\mathbb{R}$, $\forall i = 1,2,\ldots,n$, $\prod_{i=1}^n a_i = a_1a_2\ldots a_n\ne0$, khi nào thì xảy ra đẳng thức sau?
	\begin{align*}
		\left(\sum_{i=1}^n \dfrac{1}{a_i}\right)^2 = \sum_{i=1}^n \dfrac{1}{a_i^2}\mbox{, i.e., } \left(\frac{1}{a_1} + \frac{1}{a_2} + \cdots + \frac{1}{a_n}\right)^2 = \frac{1}{a_1^2} + \frac{1}{a_2^2} + \cdots + \frac{1}{a_n^2}.
	\end{align*}
\end{baitoan}

\begin{baitoan}
	Cho $a,b,c,d\in\mathbb{Q}$, $abcd\ne0$ \& $ab + ac + ad + bc + bd + cd = 0$. Chứng minh $A = \sqrt{\dfrac{1}{a^2} + \dfrac{1}{b^2} + \dfrac{1}{c^2} + \dfrac{1}{d^2}}\in\mathbb{Q}$.
\end{baitoan}

\begin{baitoan}
	Cho $a,b,c,d,e\in\mathbb{Q}$, $abcde\ne0$ \& $abc + abd + abe + acd + ace + ade + bcd + bce + bde + cde = 0$. Chứng minh $A = \sqrt{\dfrac{1}{a^2} + \dfrac{1}{b^2} + \dfrac{1}{c^2} + \dfrac{1}{d^2} + \dfrac{1}{e^2}}\in\mathbb{Q}$.
\end{baitoan}

\begin{baitoan}
	Cho $n\in\mathbb{N}^\star$, $a_i\in\mathbb{Q}$, $\forall i = 1,2,\ldots,n$, $\prod_{i=1}^n a_i = a_1a_2\ldots a_n\ne0$, \& $\sum_{\rm cyc} a_1a_2\ldots a_{n-2} = 0$. Chứng minh:
	\begin{align*}
		A = \sqrt{\sum_{i=1}^n \frac{1}{a_i^2}} = \sqrt{\frac{1}{a_1^2} + \frac{1}{a_2^2} + \cdots + \frac{1}{a_n^2}}\in\mathbb{Q}.
	\end{align*}
\end{baitoan}

\begin{luuy}[Cyclic sum]
	Ký hiệu $\sum_{\rm cyc}$ được gọi là tổng cyclic. Xem định nghĩa \& ví dụ tại, e.g., \href{https://artofproblemsolving.com/wiki/index.php/Cyclic_sum}{AoPS\emph{\texttt{/}}cyclic sum}\footnote{\textsc{url}: \url{https://artofproblemsolving.com/wiki/index.php/Cyclic_sum}.}.
\end{luuy}

\begin{baitoan}[\cite{Tuyen_Toan_9}, 1., p. 6]
	Tính $A = \sqrt{\dfrac{8^{10} - 4^{10}}{4^{11} - 8^4}}$.
\end{baitoan}
\noindent\textsf{Phân tích.} $4,8$ đều là lũy thừa của $2$ nên sẽ tiện hơn nếu đưa tất cả các lũy thừa trong $A$ về lũy thừa với cơ số 2.

\begin{proof}[Giải]
	$A = \sqrt{\dfrac{(2^3)^{10} - (2^2)^{10}}{(2^2)^{11} - (2^3)^4}} = \sqrt{\dfrac{2^{30} - 2^{20}}{2^{22} - 2^{12}}} = \sqrt{\dfrac{2^{20}(2^{10} - 1)}{2^{12}(2^{10} - 1)}} = \sqrt{2^8} = 2^4 = 16$.
\end{proof}

\begin{baitoan}[\cite{Tuyen_Toan_9}, 2., p. 6]
	Cho $A = \underbrace{99\ldots9}_{10's}4\underbrace{00\ldots0}_{10's}9$. Tính $\sqrt{A}$.
\end{baitoan}

\begin{proof}[1st giải]
	$A = \underbrace{99\ldots9}_{10's}4\cdot1\underbrace{00\ldots0}_{11's} + 9 = (\underbrace{99\ldots9}_{10's}7 - 3)(\underbrace{99\ldots9}_{10's}7 + 3) + 9 = \underbrace{99\ldots9}_{10's}7^2 - 3^2 + 9 = \underbrace{99\ldots9}_{10's}7^2\Rightarrow\sqrt{A} = \underbrace{99\ldots9}_{10's}7$.
\end{proof}

\begin{proof}[2nd giải]
	$A = (10^{10} - 1)\cdot10^{12} + 4\cdot10^{11} + 9 = 10^{22} - 10^{12} + 4\cdot10^{11} + 9 = 10^{22} - 10\cdot10^{11} + 4\cdot10^{11} + 9 = 10^{22} - 6\cdot10^{11} + 9 = (10^{11} - 3)^2\Rightarrow\sqrt{A} = 10^{11} - 3 = \underbrace{99\ldots9}_{10's}7$.
\end{proof}

\begin{baitoan}[\cite{Tuyen_Toan_9}, 3., p. 6]
	Không dùng máy tính hoặc bảng số, so sánh: (a) $\sqrt{8} + \sqrt{15}$ \& $\sqrt{65} - 1$. (b) $\dfrac{13 - 2\sqrt{3}}{6}$ \& $\sqrt{2}$.
\end{baitoan}
\noindent\textsf{Hint.} Tìm các số chính phương gần với các số dưới dấu căn để đơn giản dấu căn 1 cách hợp lý. 

\begin{proof}[Giải]
	(a) $\sqrt{8} + \sqrt{15} < \sqrt{9} + \sqrt{16} = 3 + 4 = 7$, \& $\sqrt{65} - 1 > \sqrt{64} - 1 = 8 - 1 = 7$. Suy ra $\sqrt{8} + \sqrt{15} < \sqrt{65} - 1$. (b) $\dfrac{13 - 2\sqrt{3}}{6} > \dfrac{13 - 2\sqrt{4}}{6} = \dfrac{3}{2} = 1.5$. Mặt khác, $(1.5)^2 = 2.25 > 2\Leftrightarrow 1.5 > \sqrt{2}$, nên $\dfrac{13 - 2\sqrt{3}}{6} > \sqrt{2}$.
\end{proof}

\begin{baitoan}[\cite{Tuyen_Toan_9}, 4., p. 6]
	Tìm điều kiện xác định (ĐKXĐ) \& tập xác định (TXĐ) của các biểu thức: (a) $\sqrt{2 - x^2}$. (b) $\dfrac{x}{\sqrt{5x^2 - 3}}$. (c) $\sqrt{-4x^2 + 4x - 1}$. (d) $\dfrac{1}{\sqrt{x^2 + x - 2}}$.
\end{baitoan}

\begin{proof}[Giải]
	(a) $\sqrt{2 - x^2}$ xác định $\Leftrightarrow2 - x^2\ge0\Leftrightarrow x^2\le2\Leftrightarrow|x|\le\sqrt{2}\Leftrightarrow-\sqrt{2}\le x\le\sqrt{2}$. ĐKXĐ: $-\sqrt{2}\le x\le\sqrt{2}$. TXĐ: $D = [-\sqrt{2},\sqrt{2}]$. (b) $\frac{x}{\sqrt{5x^2 - 3}}$ xác định $\Leftrightarrow 5x^2 - 3 > 0\Leftrightarrow x^2 > \frac{3}{5}\Leftrightarrow|x| > \sqrt{\frac{3}{5}}\Leftrightarrow x > \sqrt{\frac{3}{5}}$ hoặc $x < -\sqrt{\frac{3}{5}}$. ĐKXĐ: $x > \sqrt{\frac{3}{5}}$ hoặc $x < -\sqrt{\frac{3}{5}}$. TXĐ: $D = \left(-\infty,-\sqrt{\frac{3}{5}}\right)\cup\left(\sqrt{\frac{3}{5}},\infty\right)$. (c) $\sqrt{-4x^2 + 4x - 1}$ xác định $\Leftrightarrow-4x^2 + 4x - 1\ge0\Leftrightarrow-(2x - 1)^2\ge0\Leftrightarrow(2x - 1)^2\le0\Leftrightarrow2x - 1 = 0\Leftrightarrow x = \frac{1}{2}$. ĐKXĐ: $x = \frac{1}{2}$. TXĐ: $D = \left\{\frac{1}{2}\right\}$. (d) $\frac{1}{\sqrt{x^2 + x - 2}}$ xác định $\Leftrightarrow x^2 + x - 2 > 0\Leftrightarrow(x - 1)(x + 2) > 0\Leftrightarrow x > 1$ hoặc $x < -2$. ĐKXĐ: $x > 1$ hoặc $x < -2$. TXĐ: $D = (-\infty,-2)\cup(1,\infty)$.
\end{proof}

\begin{baitoan}[\cite{Tuyen_Toan_9}, 5., p. 6]
	Cho $a,b,c\in\mathbb{Q}$ khác nhau đôi một. Chứng minh $A = \sqrt{\dfrac{1}{(a - b)^2} + \dfrac{1}{(b - c)^2} + \dfrac{1}{(c - a)^2}}\in\mathbb{Q}$.
\end{baitoan}

\begin{proof}[1st giải]
	$\dfrac{1}{(a - b)^2} + \dfrac{1}{(b - c)^2} + \dfrac{1}{(c - a)^2} = \left(\dfrac{1}{a - b} + \dfrac{1}{b - c} + \dfrac{1}{c - a}\right)^2 - 2\left(\dfrac{1}{(a - b)(b - c)} + \dfrac{1}{(b - c)(c - a)} + \dfrac{1}{(c - a)(a - b)}\right) = \left(\dfrac{1}{a - b} + \dfrac{1}{b - c} + \dfrac{1}{c - a}\right)^2 - \dfrac{2(c - a + a - b + b - c)}{(a - b)(b - c)(c - a)} = \left(\dfrac{1}{a - b} + \dfrac{1}{b - c} + \dfrac{1}{c - a}\right)^2\Rightarrow A = \left|\dfrac{1}{a - b} + \dfrac{1}{b - c} + \dfrac{1}{c - a}\right|$. Vì $a,b,c\in\mathbb{Q}$ khác nhau đôi một nghĩa là $(a - b)(b - c)(c - a)\ne0$, suy ra $\dfrac{1}{a - b},\dfrac{1}{b - c},\dfrac{1}{c - a}\in\mathbb{Q}\Rightarrow A = \left|\dfrac{1}{a - b} + \dfrac{1}{b - c} + \dfrac{1}{c - a}\right|\in\mathbb{Q}$.
\end{proof}

\begin{proof}[2nd giải]
	Vì $(a - b) + (b - c) + (c - a) = 0$, \& vì $a,b,c\in\mathbb{Q}$ khác nhau đôi một nghĩa là $(a - b)(b - c)(c - a)\ne0$ nên có thể áp dụng Bài toán \ref{prob: Mo rong Tuyen_Toan_9 VD 2 p. 5} cho bộ 3 số $(a - b,b - c,c - a)$ để thu được $A = \sqrt{\dfrac{1}{(a - b)^2} + \dfrac{1}{(b - c)^2} + \dfrac{1}{(c - a)^2}}\in\mathbb{Q}$.
\end{proof}

\begin{baitoan}[\cite{Tuyen_Toan_9}, 6., p. 6]
	Cho $a,b,c\in\mathbb{Q}$ thỏa mãn $ab + bc + ca = 1$. Chứng minh $A = \sqrt{(a^2 + 1)(b^2 + 1)(c^2 + 1)}\in\mathbb{Q}$.
\end{baitoan}

\begin{proof}[Giải]
	$a^2 + 1 = a^2 + ab + bc + ca = (a + b)(a + c)$, $b^2 + 1 = b^2 + ab + bc + ca = (b + c)(b + a)$, $c^2 + 1 = c^2 + ab + bc + ca = (c + a)(c + b)$, nên $A = \sqrt{(a + b)(a + c)(b + c)(b + a)(c + a)(c + b)} = \sqrt{(a + b)^2(b + c)^2(c + a)^2} = |(a + b)(b + c)(c + a)|$. Có: $a,b,c\in\mathbb{Q}\Rightarrow A = |(a + b)(b + c)(c + a)|\in\mathbb{Q}$.
\end{proof}

\begin{baitoan}[\cite{Tuyen_Toan_9}, 7., p. 6--7]
	(a) Tìm giá trị lớn nhất của biểu thức $A = \sqrt{-x^2 + x + \frac{3}{4}}$. (b) Tìm giá trị nhỏ nhất của biểu thức $B = \sqrt{4x^4 - 4x^2(x + 1) + (x + 1)^2 + 9}$. (c) Tìm giá trị nhỏ nhất của biểu thức $C = \sqrt{25x^2 - 20x + 4} + \sqrt{25x^2}$.
\end{baitoan}

\begin{baitoan}[\cite{Tuyen_Toan_9}, 8., p. 7]
	Cho $x < 0$, rút gọn biểu thức $A = |2x - \sqrt{(5x - 1)^2}|$.
\end{baitoan}

\begin{baitoan}[\cite{Tuyen_Toan_9}, 9., p. 7]
	Cho biểu thức $A = 4x - \sqrt{9x^2 - 12x + 4}$. (a) Rút gọn $A$. (b) Tính giá trị của $A$ với $x = \frac{2}{7}$.
\end{baitoan}

\begin{baitoan}[\cite{Tuyen_Toan_9}, 10., p. 7]
	Cho biểu thức $A = 5x + \sqrt{x^2 + 6x + 9}$. (a) Rút gọn $A$. (b) Tìm $x$ để $B = -9$.
\end{baitoan}

\begin{baitoan}[\cite{Tuyen_Toan_9}, 11., p. 7]
	Tìm $x\in\mathbb{R}$ biết $\sqrt{4x^2 - 4x + 1}\le5 - x$.
\end{baitoan}

\begin{baitoan}[\cite{Tuyen_Toan_9}, 12., p. 7]
	Giải phương trình: (a) $\sqrt{x^2 + 2x + 1} = \sqrt{x + 1}$. (b) $\sqrt{x^2 - 9} + \sqrt{x^2 - 6x + 9} = 0$. (c) $\sqrt{x^2 - 4} - x^2 + 4 = 0$.
\end{baitoan}

\begin{baitoan}[\cite{Tuyen_Toan_9}, 13., p. 7]
	Giải phương trình: (a) $\sqrt{x^2 - 4x + 5} + \sqrt{x^2 - 4x + 8} + \sqrt{x^2 - 4x + 9} = 3 + \sqrt{5}$. (b) $\sqrt{2 - x^2 + 2x} + \sqrt{-x^2 - 6x - 8} = 1 + \sqrt{3}$. (c) $\sqrt{9x^2 - 6x + 2} + \sqrt{45x^2 - 30x + 9} = \sqrt{6x - 9x^2 + 8}$. 
\end{baitoan}

\begin{baitoan}[\cite{Binh_Toan_9_tap_1}, Ví dụ 5, p. 7]
	Cho biểu thức $A = \sqrt{x - \sqrt{x^2 - 4x + 4}}$. (a) Tìm điều kiện xác định của biểu thức $A$. (b) Rút gọn biểu thức $A$.
\end{baitoan}

\begin{baitoan}[\cite{Binh_Toan_9_tap_1}, Ví dụ 6, p. 8]
	Tìm điều kiện xác định của các biểu thức: (a) $A = \dfrac{1}{\sqrt{x^2 - 2x - 1}}$. (b) $B = \dfrac{1}{\sqrt{x - \sqrt{2x + 1}}}$.
\end{baitoan}

\begin{baitoan}[\cite{Binh_Toan_9_tap_1}, Ví dụ 7, p. 8]
	Tìm các giá trị của $x$ sao cho $\sqrt{x + 1} < x + 3$.
\end{baitoan}

\begin{baitoan}[\cite{Binh_Toan_9_tap_1}, 7., p. 9]
	Tìm điều kiện xác định của các biểu thức: (a) $3 - \sqrt{1 - 16x^2}$. (b) $\dfrac{1}{1 - \sqrt{x^2 - 3}}$. (c) $\sqrt{8x - x^2 - 15}$. (d) $\dfrac{2}{\sqrt{x^2 - x + 1}}$. (e) $A = \frac{1}{\sqrt{x - \sqrt{2x - 1}}}$. (f) $B = \dfrac{\sqrt{16 - x^2}}{\sqrt{2x + 1}} + \sqrt{x^2 - 8x + 14}$.
\end{baitoan}

\begin{baitoan}[\cite{Binh_Toan_9_tap_1}, 8., p. 9]
	Cho biểu thức $A = \sqrt{x^2 - 6x + 9} - \sqrt{x^2 + 6x + 9}$. (a) Rút gọn biểu thức $A$. (b) Tìm các giá trị của $x$ để $A = 1$.
\end{baitoan}

\begin{baitoan}[\cite{Binh_Toan_9_tap_1}, 9., p. 9]
	Tìm các giá trị của $x$ sao cho: (a) $\sqrt{x^2 - 3}\le x^2 - 3$. (b) $\sqrt{x^2 - 6x + 9} > x - 6$.
\end{baitoan}

\begin{baitoan}[\cite{Binh_Toan_9_tap_1}, 10., p. 9]
	Cho $a + b + c = 0$ \& $abc\ne0$. Chứng minh hằng đẳng thức: $\sqrt{\dfrac{1}{a^2} + \dfrac{1}{b^2} + \dfrac{1}{c^2}} = \left|\dfrac{1}{a} + \dfrac{1}{b} + \dfrac{1}{c}\right|$.
\end{baitoan}

%------------------------------------------------------------------------------%

\section{Liên Hệ Giữa Phép Nhân, Phép Chia \& Phép Khai Phương}

\begin{baitoan}[\cite{Tuyen_Toan_9}, Thí dụ 3, p. 9]
	Rút gọn biểu thức $A = \sqrt{4 + \sqrt{7}} - \sqrt{4 - \sqrt{7}}$.
\end{baitoan}

\begin{baitoan}[\cite{Tuyen_Toan_9}, Thí dụ 4, p. 10]
	Tìm giá trị lớn nhất của biểu thức $A = \sqrt{x - 5} + \sqrt{13 - x}$.
\end{baitoan}

\begin{baitoan}[\cite{Tuyen_Toan_9}, 14., p. 11]
	Rút gọn biểu thức $A = \dfrac{\sqrt{\sqrt{7} - \sqrt{3}} - \sqrt{\sqrt{7} + \sqrt{3}}}{\sqrt{\sqrt{7} - 2}}$.
\end{baitoan}

\begin{baitoan}[\cite{Tuyen_Toan_9}, 15., p. 11]
	Cho 2 số có tổng bằng $\sqrt{19}$ \& có hiệu bằng $\sqrt{7}$. Tính tích của 2 số đó.
\end{baitoan}

\begin{baitoan}[\cite{Tuyen_Toan_9}, 16., p. 11]
	Tính $\sqrt{A}$ biết: (a) $A = 13 - 2\sqrt{42}$. (b) $A = 46 + 6\sqrt{5}$. (c) $A = 12 - 3\sqrt{15}$.
\end{baitoan}

\begin{baitoan}[\cite{Tuyen_Toan_9}, 17., p. 12]
	Rút gọn biểu thức: (a) $A = \sqrt{6 + 2\sqrt{2}\sqrt{3 - \sqrt{4 + 2\sqrt{3}}}}$. (b) $B = \sqrt{5} - \sqrt{3 - \sqrt{29 - 12\sqrt{5}}}$. (c) $C = \sqrt{3 - \sqrt{5}}(\sqrt{10} - \sqrt{2})(3 + \sqrt{5})$.
\end{baitoan}

\begin{baitoan}[\cite{Tuyen_Toan_9}, 18., p. 12]
	Rút gọn biểu thức $A = \sqrt{x + 2\sqrt{x - 1}} + \sqrt{x - 2\sqrt{x - 1}}$.
\end{baitoan}

\begin{baitoan}[\cite{Tuyen_Toan_9}, 19., p. 12]
	Cho $a > 0$, so sánh $\sqrt{a + 1} + \sqrt{a + 3}$ với $2\sqrt{a + 2}$.
\end{baitoan}

\begin{baitoan}[\cite{Tuyen_Toan_9}, 20., p. 12]
	Cho $a,b,x,y > 0$. Chứng minh $\sqrt{ax} + \sqrt{by}\le\sqrt{(a + b)(x + y)}$.
\end{baitoan}

\begin{baitoan}[\cite{Tuyen_Toan_9}, 21., p. 12]
	(a) Tìm giá trị lớn nhất của biểu thức $A = \sqrt{x + 1} - \sqrt{x - 8}$. (b) Tìm giá trị nhỏ nhất của biểu thức $B = \sqrt{x - 1} + \sqrt{5 - x}$.
\end{baitoan}

\begin{baitoan}[\cite{Tuyen_Toan_9}, 22., p. 12]
	Rút gọn biểu thức:
	\begin{align*}
		A = \frac{\sqrt{1 + \sqrt{1 - x^2}}\left[\sqrt{(1 + x)^3} - \sqrt{(1 - x)^3}\right]}{2 + \sqrt{1 - x^2}}.
	\end{align*}
\end{baitoan}

\begin{baitoan}[\cite{Tuyen_Toan_9}, 23., p. 12]
	Tìm $x,y$ biết $x + y + 12 = 4\sqrt{x} + 6\sqrt{y - 1}$.
\end{baitoan}

\begin{baitoan}[\cite{Tuyen_Toan_9}, 24., p. 12]
	Tìm $x,y,z$ biết $\sqrt{x - a} + \sqrt{y - b} + \sqrt{z - c} = \frac{1}{2}(x + y + z)$, trong đó $a + b + c = 3$.
\end{baitoan}

\begin{baitoan}[\cite{Tuyen_Toan_9}, 25., p. 12]
	Giải phương trình $\sqrt{x + 3 - 4\sqrt{x - 1}} + \sqrt{x + 8 + 6\sqrt{x - 1}} = 5$.
\end{baitoan}

\begin{baitoan}[\cite{Tuyen_Toan_9}, 26., p. 12]
	Giải phương trình $\sqrt{x^2 - 5x + 6} + \sqrt{x + 1} = \sqrt{x - 2} + \sqrt{x^2 - 2x - 3}$.
\end{baitoan}

\begin{baitoan}[\cite{Tuyen_Toan_9}, 27., p. 12]
	Chứng minh bất đẳng thức $\sqrt{n + a} + \sqrt{n - a} < 2\sqrt{n}$ vpwos $0 < |a|\le n$. Áp dụng (không dùng máy tính hoặc bảng số): Chứng minh: $\sqrt{101} - \sqrt{99} > 0.1$.
\end{baitoan}

\begin{baitoan}[\cite{Tuyen_Toan_9}, 28., p. 13]
	Chứng minh: $2(\sqrt{n + 1} - \sqrt{n}) < \dfrac{1}{\sqrt{n}} < 2(\sqrt{n} - \sqrt{n - 1})$, $\forall n\in\mathbb{N}^\star$. Áp dụng: Cho $S = \sum_{i=1}^{100} \frac{1}{\sqrt{i}} = 1 + \frac{1}{\sqrt{2}} + \frac{1}{\sqrt{3}} + \cdots + \frac{1}{\sqrt{100}}$. Chứng minh $18 < S < 19$.
\end{baitoan}

\begin{baitoan}[\cite{Tuyen_Toan_9}, 29., p. 13]
	Chứng minh: $\dfrac{1}{2\sqrt{n + 1}} < \sqrt{n + 1} - \sqrt{n}$, $\forall n\in\mathbb{N}^\star$. Áp dụng: Chứng minh: $S = \sum_{i=1}^{2500} \frac{1}{\sqrt{i}} = 1 + \frac{1}{\sqrt{2}} + \frac{1}{\sqrt{3}} + \cdots + \frac{1}{\sqrt{2500}} < 100$.
\end{baitoan}

\begin{baitoan}[\cite{Tuyen_Toan_9}, 30., p. 13]
	Cho $x,y,z > 0$. Chứng minh $x + y + z\ge\sqrt{xy} + \sqrt{yz} + \sqrt{zx}$.
\end{baitoan}

\begin{baitoan}[\cite{Tuyen_Toan_9}, 31., p. 13]
	Cho $A = \sqrt{x + 3} + \sqrt{5 - x}$. Chứng minh $A\le4$.
\end{baitoan}

\begin{baitoan}[\cite{Tuyen_Toan_9}, 32., p. 13]
	Cho $B = \dfrac{x^3}{1 + y} + \dfrac{y^3}{1 + x}$ trong đó $x,y$ là các số thực dương thỏa mãn điều kiện $xy = 1$. Chứng minh $B\ge1$.
\end{baitoan}

\begin{baitoan}[\cite{Tuyen_Toan_9}, 33., p. 13]
	Cho $x,y,z > 0$ thỏa mãn điều kiện $\dfrac{1}{x + 1} + \dfrac{1}{y + 1} + \dfrac{1}{z + 1} = 2$. Chứng minh $xyz\le\frac{1}{8}$.
\end{baitoan}

\begin{baitoan}[\cite{Tuyen_Toan_9}, 34., p. 13]
	Tìm các số dương $x,y,z$ sao cho $x + y + z = 3$ \& $x^4 + y^4 + z^4 = 3xyz$.
\end{baitoan}

\begin{baitoan}[\cite{Tuyen_Toan_9}, 35., p. 13]
	Cho $\sqrt{x} + 2\sqrt{y} = 10$. Chứng minh: $x + y\ge20$.
\end{baitoan}

\begin{baitoan}[\cite{Tuyen_Toan_9}, 36., p. 13]
	Cho $x,y,z\ge0$ thỏa mãn điều kiện $x + y + z = 1$. Chứng minh: $\sqrt{x + y} + \sqrt{y + z} + \sqrt{z + x}\le\sqrt{6}$.
\end{baitoan}

\begin{baitoan}[\cite{Binh_Toan_9_tap_1}, Ví dụ 8, p. 10]
	Rút gọn biểu thức $A = \sqrt{x + \sqrt{2x - 1}} - \sqrt{ x - \sqrt{2x - 1}}$.
\end{baitoan}

\begin{baitoan}[\cite{Binh_Toan_9_tap_1}, Ví dụ 9, p. 11]
	Chứng minh số $\sqrt{2} + \sqrt{3} + \sqrt{5}$ là số vô tỷ.
\end{baitoan}

\begin{baitoan}[\cite{Binh_Toan_9_tap_1}, 11., pp. 11--12]
	Rút gọn biểu thức: (a) $\sqrt{11 - 2\sqrt{10}}$. (b) $\sqrt{9 - 2\sqrt{14}}$. (c) $\sqrt{4 + 2\sqrt{3}} - \sqrt{4 - 2\sqrt{3}}$. (d) $\sqrt{9 - 4\sqrt{5}} - \sqrt{9 + 4\sqrt{5}}$. (e) $\sqrt{4 - \sqrt{7}} - \sqrt{4 + \sqrt{7}}$. (f) $\dfrac{\sqrt{3} + \sqrt{11 + 6\sqrt{2}}- \sqrt{5 + 2\sqrt{6}}}{\sqrt{2} + \sqrt{6 + 2\sqrt{5}} - \sqrt{7 + 2\sqrt{10}}}$. (g) $\sqrt{5\sqrt{3} + 5\sqrt{48 - 10\sqrt{7+ 4\sqrt{3}}}}$. (h) $\sqrt{4 + \sqrt{10 + 2\sqrt{5}}} + \sqrt{4 - \sqrt{10 + 2\sqrt{5}}}$. (i) $\sqrt{94 - 42\sqrt{5}} - \sqrt{94 + 42\sqrt{5}}$.
\end{baitoan}

\begin{baitoan}[\cite{Binh_Toan_9_tap_1}, 12., p. 12]
	Tính: (a) $(4 + \sqrt{15})(\sqrt{10} - \sqrt{6})\sqrt{4 - \sqrt{15}}$. (b) $\sqrt{3 - \sqrt{5}}(\sqrt{10} - \sqrt{2})(3 + \sqrt{5})$.\\(c) $\dfrac{\sqrt{\sqrt{5} + 2} + \sqrt{\sqrt{5} - 2}}{\sqrt{\sqrt{5} + 1}} - \sqrt{3 - 2\sqrt{2}}$.
\end{baitoan}

\begin{baitoan}[\cite{Binh_Toan_9_tap_1}, 13., p. 12]
	Chứng minh các hằng đẳng thức sau với $b\ge0$, $a\ge\sqrt{b}$: (a) $\sqrt{a + \sqrt{b}}\pm\sqrt{a - \sqrt{b}} = \sqrt{2(a\pm\sqrt{a^2 - b})}$. (b) $\sqrt{a\pm\sqrt{b}} = \sqrt{\dfrac{a + \sqrt{a^2 - b}}{2}}\pm\sqrt{\dfrac{a - \sqrt{a^2 - b}}{2}}$.
\end{baitoan}

\begin{baitoan}[\cite{Binh_Toan_9_tap_1}, 14., p. 12]
	Rút gọn biểu thức $A = \sqrt{x + 2\sqrt{2x - 4}} + \sqrt{x - 2\sqrt{2x - 4}}$.
\end{baitoan}

\begin{baitoan}[\cite{Binh_Toan_9_tap_1}, 15., p. 12]
	Cho biểu thức $A = \dfrac{x + \sqrt{x^2 - 2x}}{x - \sqrt{x^2 - 2x}} - \dfrac{x - \sqrt{x^2 - 2x}}{x + \sqrt{x^2 - 2x}}$. (a) Tìm điều kiện xác định của biểu thức $A$. (b) Rút gọn biểu thức $A$. (c) Tìm giá trị của $x$ để $A < 2$.
\end{baitoan}

\begin{baitoan}[\cite{Binh_Toan_9_tap_1}, 16., p. 12]
	Lập 1 phương trình bậc 2 với các hệ số nguyên, trong đó: (a) $2 + \sqrt{3}$ là 1 nghiệm của phương trình. (b) $6 - 4\sqrt{2}$ là 1 nghiệm của phương trình.
\end{baitoan}

\begin{baitoan}[\cite{Binh_Toan_9_tap_1}, 17., p. 13]
	Chứng minh các số sau là số vô tỷ: (a) $\sqrt{3} - \sqrt{2}$. (b) $2\sqrt{2} + \sqrt{3}$.
\end{baitoan}

\begin{baitoan}[\cite{Binh_Toan_9_tap_1}, 18., p. 13]
	Có tồn tại các số hữu tỷ dương $a,b$ hay không nếu: (a) $\sqrt{a} + \sqrt{b} = \sqrt{2}$. (b) $\sqrt{a} + \sqrt{b} = \sqrt{\sqrt{2}}$.
\end{baitoan}

\begin{baitoan}[\cite{Binh_Toan_9_tap_1}, 19., p. 13]
	Cho 3 số $x,y,\sqrt{x} + \sqrt{y}$ là các số hữu tỷ. Chứng minh mỗi số $\sqrt{x},\sqrt{y}$ đều là số hữu tỷ.
\end{baitoan}

\begin{baitoan}[\cite{Binh_Toan_9_tap_1}, 20., p. 13]
	Cho $a,b,c,d$ là các số dương. Chứng minh tồn tại 1 số dương trong 2 số $2a + b - 2\sqrt{cd}$ \& $2c + d - 2\sqrt{ab}$.
\end{baitoan}

\begin{baitoan}[\cite{Binh_Toan_9_tap_1}, $21^\star$., p. 13]
	(a) Rút gọn biểu thức $A = \sqrt{1 + \frac{1}{a^2} + \frac{1}{(a + 1)^2}}$ với $a > 0$. (b) Tính giá trị của tổng $B = \sum_{i=1}^{99} \sqrt{1 + \frac{1}{i^2} + \frac{1}{(i + 1)^2}} = \sqrt{1 + \frac{1}{1^2} + \frac{1}{2^2}} + \sqrt{1 + \frac{1}{2^2} + \frac{1}{3^2}} + \sqrt{1 + \frac{1}{3^2} + \frac{1}{4^2}} + \cdots + \sqrt{1 + \frac{1}{99^2} + \frac{1}{100^2}}$.
\end{baitoan}

\begin{baitoan}[\cite{Binh_Toan_9_tap_1}, $22^\star$., p. 13]
	(a) Nêu 1 cách tính nhẩm $997^2$. (b) Tính tổng các chữ số của $A$ biết $\sqrt{A} = 99\ldots96$ (có $100$ chữ số $9$).
\end{baitoan}

%------------------------------------------------------------------------------%

\section{Biến Đổi Đơn Giản Biểu Thức Chứa Căn Thức Bậc 2}

\begin{baitoan}[\cite{Binh_Toan_9_tap_1}, Ví dụ 10, p. 14]
	Rút gọn biểu thức $A = \sqrt{5} - \sqrt{3 - \sqrt{29 - 12\sqrt{5}}}$.
\end{baitoan}

\begin{baitoan}[\cite{Binh_Toan_9_tap_1}, Ví dụ 11, p. 14]
	Tính giá trị của biểu thức
	\begin{align*}
		M = \sum_{i=1}^{24} \frac{1}{(i + 1)\sqrt{i} + i\sqrt{i + 1}} = \frac{1}{2\sqrt{1} + 1\sqrt{2}} + \frac{1}{3\sqrt{2} + 2\sqrt{3}} + \frac{1}{4\sqrt{3} + 3\sqrt{4}} + \cdots + \frac{1}{25\sqrt{24} + 24\sqrt{25}}.
	\end{align*}
\end{baitoan}

\begin{baitoan}[\cite{Binh_Toan_9_tap_1}, 23., p. 15]
	Rút gọn biểu thức $A = \sqrt{1 - a} + \sqrt{a(a - 1)} + a\sqrt{\frac{a - 1}{a}}$.
\end{baitoan}

\begin{baitoan}[\cite{Binh_Toan_9_tap_1}, 24., p. 15]
	Chứng minh các hằng đẳng thức: (a) $\sqrt{10 + \sqrt{60} - \sqrt{24} - \sqrt{40}} = \sqrt{3} + \sqrt{5} - \sqrt{2}$. (b) $\sqrt{6 + \sqrt{24} + \sqrt{12} + \sqrt{8}} - \sqrt{3} = \sqrt{2} + 1$.
\end{baitoan}

\begin{baitoan}[\cite{Binh_Toan_9_tap_1}, 25., p. 15]
	Cho $A = \sqrt{10 + \sqrt{24} + \sqrt{40} + \sqrt{60}}$. Biểu diễn $A$ dưới dạng tổng của 3 căn thức.
\end{baitoan}

\begin{baitoan}[\cite{Binh_Toan_9_tap_1}, 26., p. 15]
	Rút gọn biểu thức $A = \dfrac{x + 3 + 2\sqrt{x^2 - 9}}{2x - 6 + \sqrt{x^2 - 9}}$.
\end{baitoan}

\begin{baitoan}[\cite{Binh_Toan_9_tap_1}, 27., p. 15]
	Rút gọn biểu thức $B = \dfrac{x^2 + 5x + 6 + x\sqrt{9 - x^2}}{3x - x^2 + (x + 2)\sqrt{9 - x^2}}$.
\end{baitoan}

\begin{baitoan}[\cite{Binh_Toan_9_tap_1}, 28., p. 15]
	Rút gọn biểu thức:
	\begin{align*}
		A &= \sum_{i=1}^{n-1} \frac{1}{\sqrt{i} + \sqrt{i + 1}} = \frac{1}{\sqrt{1} + \sqrt{2}} + \frac{1}{\sqrt{2} + \sqrt{3}} + \frac{1}{\sqrt{3} + \sqrt{4}} + \cdots + \frac{1}{\sqrt{n - 1} + \sqrt{n}},\\
		B &= \sum_{i=1}^{24} \frac{1}{\sqrt{i} - \sqrt{i + 1}} = \frac{1}{\sqrt{1} - \sqrt{2}} - \frac{1}{\sqrt{2} - \sqrt{3}} + \frac{1}{\sqrt{3} - \sqrt{4}} - \cdots - \frac{1}{\sqrt{24} - \sqrt{25}}.
	\end{align*}
\end{baitoan}

%------------------------------------------------------------------------------%

\section{Rút Gọn Biểu Thức Có Chứa Căn Thức Bậc 2}

\begin{baitoan}[\cite{Tuyen_Toan_9}, Thí dụ 5, p. 14]
	Cho $A = \sqrt{11 + \sqrt{96}}$ \& $B = \dfrac{2\sqrt{2}}{1 + \sqrt{2} - \sqrt{3}}$. Không dùng máy tính hoặc bảng số, so sánh $A$ \& $B$.
\end{baitoan}

\begin{baitoan}[\cite{Tuyen_Toan_9}, Thí dụ 6, p. 15]
	Cho biểu thức $A = \left(\dfrac{1}{\sqrt{x} - \sqrt{x - 1}} - \dfrac{x - 3}{\sqrt{x - 1} - \sqrt{2}}\right)\left(\dfrac{2}{\sqrt{2} - \sqrt{x}} - \dfrac{\sqrt{x} + \sqrt{2}}{\sqrt{2x} - x}\right)$.\\(a) Rút gọn $A$. (b) Tính giá trị của $A$ với $x = 3 - 2\sqrt{2}$.
\end{baitoan}

\begin{baitoan}[\cite{Tuyen_Toan_9}, 37., pp. 15--16]
	Không dùng máy tính hoặc bảng số, so sánh các số sau: (a) $-3\sqrt{11}$ \& $-7\sqrt{2}$. (b) $\frac{7}{2}\sqrt{\frac{1}{12}}$ \& $\frac{9}{4}\sqrt{\frac{1}{5}}$. (c) $\sqrt{\frac{4}{27}}$ \& $\sqrt{\frac{3}{26}}$.
\end{baitoan}

\begin{baitoan}[\cite{Tuyen_Toan_9}, 38., p. 16]
	Không dùng máy tính hoặc bảng số, chứng minh $4\sqrt{5} - 3\sqrt{2} < 5$.
\end{baitoan}

\begin{baitoan}[\cite{Tuyen_Toan_9}, 39., p. 16]
	Cho $A = \sqrt{x^2 + 1} - x - \dfrac{1}{\sqrt{x^2 + 1} - x}$ trong đó $x\in\mathbb{R}$. Xác định $x\in\mathbb{R}$ để giá trị của $A$ là 1 số tự nhiên.
\end{baitoan}

\begin{baitoan}[\cite{Tuyen_Toan_9}, 40., p. 16]
	Trục căn thức ở mẫu của các biểu thức sau: (a) $A = \dfrac{1}{\sqrt{a} + \sqrt{b} + \sqrt{2c}}$ trong đó $a,b,c > 0$ thỏa mãn điều kiện $c$ là trung bình nhân của $a$ \& $b$. (b) $B = \dfrac{1}{\sqrt{a} + \sqrt{b} + \sqrt{c} + \sqrt{d}}$ trong đó $a,b,c,d > 0$ thỏa mãn điều kiện $ab = cd$ \& $a + b\ne c + d$.
\end{baitoan}

\begin{baitoan}[\cite{Tuyen_Toan_9}, 41., p. 16]
	Tìm $x,y\in\mathbb{N}$ sao cho $x > y > 0$ thỏa mãn điều kiện $\sqrt{x} + \sqrt{y} = \sqrt{931}$.
\end{baitoan}

\begin{baitoan}[\cite{Tuyen_Toan_9}, 42., p. 16]
	Chứng minh: $\dfrac{2\sqrt{mn}}{\sqrt{m} + \sqrt{n} + \sqrt{m + n}} = \sqrt{m} + \sqrt{n} - \sqrt{m + n}$. Áp dụng tính $\dfrac{2\sqrt{10}}{\sqrt{2} + \sqrt{5} + \sqrt{7}}$.
\end{baitoan}

\begin{baitoan}[\cite{Tuyen_Toan_9}, 43., p. 16]
	Chứng minh: $\dfrac{1}{(n + 1)\sqrt{n} + n\sqrt{n + 1}} = \dfrac{1}{\sqrt{n}} - \dfrac{1}{\sqrt{n + 1}}$, $\forall n\in\mathbb{N}^\star$. Áp dụng tính tổng: $S = \sum_{i=1}^{399} \dfrac{1}{(i + 1)\sqrt{i} + i\sqrt{i + 1}} = \dfrac{1}{2\sqrt{1} + 1\sqrt{2}} + \dfrac{1}{3\sqrt{2} + 2\sqrt{3}} + \cdots + \dfrac{1}{400\sqrt{399} + 399\sqrt{400}}$.
\end{baitoan}

\begin{baitoan}[\cite{Tuyen_Toan_9}, 44., p. 16]
	Tìm $n\in\mathbb{N}$ nhỏ nhất sao cho $\sqrt{n + 1} - \sqrt{n} < 0.05$.
\end{baitoan}

\begin{baitoan}[\cite{Tuyen_Toan_9}, 45., p. 17]
	Cho $A = \sum_{i=1}^{120} \dfrac{1}{\sqrt{i} + \sqrt{i + 1}} = \dfrac{1}{\sqrt{1} + \sqrt{2}} + \dfrac{1}{\sqrt{2} + \sqrt{3}} + \cdots + \dfrac{1}{\sqrt{120} + \sqrt{121}}$, $B = \sum_{i=1}^{35} \dfrac{1}{\sqrt{i}} = \dfrac{1}{\sqrt{1}} + \dfrac{1}{\sqrt{2}} + \cdots + \dfrac{1}{\sqrt{35}}$. Chứng minh $A < B$.
\end{baitoan}

\begin{baitoan}[\cite{Tuyen_Toan_9}, 46., p. 17]
	Cho $x,y,z > 0$ \& khác nhau đôi một. Chứng minh giá trị của biểu thức
	\begin{align*}
		A = \dfrac{x}{(\sqrt{x} - \sqrt{y})(\sqrt{x} - \sqrt{z})} + \dfrac{y}{(\sqrt{y} - \sqrt{z})(\sqrt{y} - \sqrt{z})} + \dfrac{z}{(\sqrt{z} - \sqrt{x})(\sqrt{z} - \sqrt{y})}
	\end{align*}
	không phụ thuộc vào giá trị của các biến.
\end{baitoan}

\begin{baitoan}[\cite{Tuyen_Toan_9}, 47., p. 17]
	Cho biểu thức $A = \dfrac{1}{\sqrt{x} + 2} - \dfrac{5}{x - \sqrt{x} - 6} - \dfrac{\sqrt{x} - 2}{3 - \sqrt{x}}$. (a) Rút gọn $A$. (b) Tìm giá trị lớn nhất của $A$.
\end{baitoan}

\begin{baitoan}[\cite{Tuyen_Toan_9}, 48., p. 17]
	Cho $A = \left(\dfrac{\sqrt{x} + \sqrt{y}}{1 - \sqrt{xy}} + \dfrac{\sqrt{x} - \sqrt{y}}{1 -+ \sqrt{xy}}\right):\left(1 + \dfrac{x + y + 2xy}{1 - xy}\right)$. (a) Rút gọn $A$. (b) Tính giá trị của $P$ với $x = \dfrac{2}{2 + \sqrt{3}}$. (c) Tìm giá trị lớn nhất của $A$.
\end{baitoan}

\begin{baitoan}[\cite{Tuyen_Toan_9}, 49., p. 17]
	Cho $A = \dfrac{\sqrt{x}}{\sqrt{xy} + \sqrt{x} + 2} + \dfrac{\sqrt{y}}{\sqrt{yz} + \sqrt{y} + 1} + \dfrac{2\sqrt{z}}{\sqrt{zx} + 2\sqrt{z} + 2}$. Biết $xyz = 4$, tính $\sqrt{P}$.
\end{baitoan}

\begin{baitoan}[\cite{Binh_Toan_9_tap_1}, Ví dụ 12, p. 15]
	Tính: $A = \left(\sqrt{\dfrac{1 + a}{1 - a}} + \sqrt{\dfrac{1 - a}{1 + a}}\right):\left(\sqrt{\dfrac{1 + a}{1 - a}} - \sqrt{\dfrac{1 - a}{1 + a}}\right)$.
\end{baitoan}

\begin{baitoan}[\cite{Binh_Toan_9_tap_1}, Ví dụ 13, p. 16]
	Rút gọn biểu thức $A = \dfrac{2 + \sqrt{3}}{\sqrt{2} + \sqrt{2 + \sqrt{3}}} + \dfrac{2 - \sqrt{3}}{\sqrt{2} - \sqrt{2 - \sqrt{3}}}$.
\end{baitoan}

\begin{baitoan}[\cite{Binh_Toan_9_tap_1}, Ví dụ 14, p. 16]
	Cho $A = \dfrac{\sqrt{a} + 6}{\sqrt{a} + 1}$. (a) Tìm các số nguyên $a$ để $A$ là số nguyên. (b) Chứng minh với $a = \frac{4}{9}$ thì $A$ là số nguyên. (c) Tìm các số hữu tỷ $a$ để $A$ là số nguyên.
\end{baitoan}

\begin{baitoan}[\cite{Binh_Toan_9_tap_1}, 29., p. 18]
	Rút gọn biểu thức: (a) $A = \dfrac{1 + \sqrt{5}}{\sqrt{2} + \sqrt{3 + \sqrt{5}}} + \dfrac{1 - \sqrt{5}}{\sqrt{2} - \sqrt{3 - \sqrt{5}}}$.\\(b) $B = \left(\dfrac{1 - a\sqrt{a}}{1 - \sqrt{a}} + \sqrt{a}\right)\left(\dfrac{1 - \sqrt{a}}{1 - a}\right)^2$. (c) $C = \dfrac{\sqrt{x} - \sqrt{y}}{xy\sqrt{xy}}:\left[\left(\dfrac{1}{x} + \dfrac{1}{y}\right)\dfrac{1}{x + y+ 2\sqrt{xy}} + \dfrac{2}{(\sqrt{x} + \sqrt{y})^3}\left(\dfrac{1}{\sqrt{x}} + \dfrac{1}{\sqrt{y}}\right)\right]$ với $x = 2 - \sqrt{3}$ \& $y = 2 + \sqrt{3}$.
\end{baitoan}

\begin{baitoan}[\cite{Binh_Toan_9_tap_1}, 30., p. 18]
	Rút gọn biểu thức $A = \dfrac{1 - \sqrt{x - 1}}{\sqrt{x - 2\sqrt{x - 1}}}$.
\end{baitoan}

\begin{baitoan}[\cite{Binh_Toan_9_tap_1}, 31., p. 18]
	Rút gọn biểu thức $A = \dfrac{\sqrt{x + \sqrt{x^2 - y^2}} - \sqrt{x - \sqrt{x^2 - y^2}}}{\sqrt{2(x - y)}}$ với $x > y > 0$.
\end{baitoan}

\begin{baitoan}[\cite{Binh_Toan_9_tap_1}, 32., p. 18]
	Rút gọn biểu thức $A = \left(\dfrac{1}{\sqrt{x - 1}} + \dfrac{1}{\sqrt{x + 1}}\right):\left(\dfrac{1}{\sqrt{x - 1}} - \dfrac{1}{\sqrt{x + 1}}\right)$ với $x = \dfrac{a^2 + b^2}{2ab}$ \& $b > a > 0$.
\end{baitoan}

\begin{baitoan}[\cite{Binh_Toan_9_tap_1}, 33., p. 18]
	Rút gọn biểu thức $B = \dfrac{2a\sqrt{1 + x^2}}{\sqrt{1 + x^2} - x}$ với $x = \dfrac{1}{2}\left(\sqrt{\dfrac{1 - a}{a}} - \sqrt{\dfrac{a}{1 - a}}\right)$ \& $0 < a < 1$.
\end{baitoan}

\begin{baitoan}[\cite{Binh_Toan_9_tap_1}, 34., p. 18]
	Rút gọn biểu thức $A = a + b - \sqrt{\dfrac{(a^2 + 1)(b^2 + 1)}{c^2 + 1}}$ với $a,b,c > 0$ \& $ab + bc + ca = 1$.
\end{baitoan}

\begin{baitoan}[\cite{Binh_Toan_9_tap_1}, 35., p. 18]
	Rút gọn biểu thức $A = \dfrac{\sqrt{x + 2\sqrt{x - 1}} + \sqrt{x - 2\sqrt{x - 1}}}{\sqrt{x + \sqrt{2x - 1}} + \sqrt{x - \sqrt{2x - 1}}}\cdot\sqrt{2x - 1}$.
\end{baitoan}

\begin{baitoan}[\cite{Binh_Toan_9_tap_1}, 36., p. 18]
	Chứng minh hằng đẳng thức sau với $x\ge2$
	\begin{align*}
		\sqrt{\sqrt{x} + \sqrt{\frac{x^2 - 4}{x}}} + \sqrt{\sqrt{x} - \sqrt{\frac{x^2 - 4}{x}}} = \sqrt{\frac{2x + 4}{\sqrt{x}}}.
	\end{align*}
\end{baitoan}

\begin{baitoan}[\cite{Binh_Toan_9_tap_1}, 37., p. 18]
	Cho $a = \dfrac{-1 + \sqrt{2}}{2}$, $b = \dfrac{-1 - \sqrt{2}}{2}$. Tính $a^7 + b^7$.
\end{baitoan}

\begin{baitoan}[\cite{Binh_Toan_9_tap_1}, 38., p. 19]
	Cho biết $\sqrt{x^2 - 6x + 13} - \sqrt{x^2 - 6x + 10} = 1$. Tính $\sqrt{x^2 - 6x + 13} + \sqrt{x^2 - 6x + 10}$.
\end{baitoan}

\begin{baitoan}[\cite{Binh_Toan_9_tap_1}, 39., p. 19]
	Cho biểu thức $A = \dfrac{\sqrt{a} + 2}{\sqrt{a} - 2}$. (a) Tìm các số nguyên $a$ để $A$ là số nguyên. (b) Tìm các số hữu tỷ $a$ để $A$ là số nguyên.
\end{baitoan}

\begin{baitoan}[\cite{Binh_Toan_9_tap_1}, 40., p. 19]
	Cho $a = \sqrt{2} - 1$. (a) Viết $a^2,a^3$ dưới dạng $\sqrt{m} - \sqrt{m - 1}$ trong đó $m$ là số tự nhiên. (b) Chứng minh với mọi số nguyên dương $n$, số $a^n$ viết được dưới dạng trên.
\end{baitoan}

%------------------------------------------------------------------------------%

\section{Cube Root, $n$th Root -- Căn Bậc 3, Căn Bậc $n$}

\begin{baitoan}[Program to print out 1st $n$ cube roots]
	Viết chương trình \textsc{Pascal, C\texttt{/}C++, Python} xuất ra căn bậc 3 của $n$ số tự nhiên đầu tiên với $n\in\mathbb{N}^\star$ được nhập từ bàn phím.
\end{baitoan}

\begin{baitoan}
	Viết chương trình \textsc{Pascal, C\texttt{/}C++, Python} để kiểm tra 1 số $n\in\mathbb{N}^\star$ được nhập từ bàn phím có phải là lập phương của 1 số tự nhiên hay không.
\end{baitoan}

\begin{baitoan}[Program to print out 1st $n$ $n$th roots]
	Viết chương trình \textsc{Pascal, C\texttt{/}C++, Python} xuất ra căn bậc $n$ của $m$ số tự nhiên đầu tiên với $m,n\in\mathbb{N}^\star$ được nhập từ bàn phím.
\end{baitoan}

\begin{baitoan}
	Viết chương trình \textsc{Pascal, C\texttt{/}C++, Python} để kiểm tra 1 số $m$ được nhập từ bàn phím có phải là lũy thừa bậc $n$ của 1 số tự nhiên hay không với $m,n\in\mathbb{N}^\star$ được nhập từ bàn phím.
\end{baitoan}

\begin{baitoan}[Mở rộng \cite{Tuyen_Toan_9}, Thí dụ 1, p. 5]
	Cho $x\in\mathbb{R}$. So sánh $\sqrt[3]{x}$ với $x$.
\end{baitoan}

\begin{proof}[Giải]
	$\sqrt[3]{x}$ xác định $\forall x\in\mathbb{R}$. Xét các trường hợp: (a) $\sqrt[3]{x} = x\Leftrightarrow x = x^3\Leftrightarrow x - x^3 = 0\Leftrightarrow x(1 - x^2) = 0\Leftrightarrow x(1 - x)(1 + x) = 0\Leftrightarrow x\in\{0,\pm1\}$. (b) $\sqrt[3]{x} < x\Leftrightarrow x < x^3\Leftrightarrow x - x^3 < 0\Leftrightarrow x(1 - x^2) < 0\Leftrightarrow x(1 - x)(1 + x) < 0\Leftrightarrow -1 < x < 0$ hoặc $x > 1$, trong đó phép biến đổi tương đương cuối cùng thu được nhờ lập bảng xét dấu. (c) $\sqrt[3]{x} > x\Leftrightarrow x > x^3\Leftrightarrow x - x^3 > 0\Leftrightarrow x(1 - x^2) > 0\Leftrightarrow x(1 - x)(1 + x) > 0\Leftrightarrow x < -1$ hoặc $0 < x < 1$, trong đó phép biến đổi tương đương cuối cùng cũng thu được nhờ lập bảng xét dấu. Vậy: $\sqrt[3]{x} = x\Leftrightarrow x\in\{0,\pm1\}$, $\sqrt[3]{x} < x\Leftrightarrow x\in(-1,0)\cup(1,+\infty)$, $\sqrt[3]{x} > x\Leftrightarrow x\in(-\infty,-1)\cup(0,1)$.
\end{proof}

\begin{baitoan}[Mở rộng \cite{Tuyen_Toan_9}, Thí dụ 1, p. 5]
	Cho $x\in\mathbb{R}$, $n\in\mathbb{N}^\star$. So sánh $\sqrt[n]{x}$ với $x$.
\end{baitoan}

\begin{baitoan}[\cite{Tuyen_Toan_9}, Thí dụ 7, p. 19]
	Tính $x = \sqrt[3]{17\sqrt{5} + 38} - \sqrt[3]{17\sqrt{5} - 38}$.
\end{baitoan}

\begin{baitoan}[\cite{Tuyen_Toan_9}, Thí dụ 8, p. 20]
	Giải \& biện luận phương trình $(x - a)^n = a^2 - 2a + 1$ với $n\in\mathbb{N}^\star$, $a$ là tham số.
\end{baitoan}

\begin{baitoan}[\cite{Tuyen_Toan_9}, 50., p. 21]
	Tính: (a) $\sqrt[3]{8\sqrt{5} - 16}\sqrt[3]{8\sqrt{5} + 16}$. (b) $\sqrt[3]{7 - 5\sqrt{2}} + \sqrt[6]{8}$. (c) $\sqrt[3]{4}\sqrt[3]{1 - \sqrt{3}}\sqrt[6]{4 + 2\sqrt{3}}$.
\end{baitoan}

\begin{baitoan}[\cite{Tuyen_Toan_9}, 51., p. 21]
	(a) Tính $\dfrac{2}{\sqrt[3]{3} - 1} - \dfrac{4}{\sqrt[3]{9} - \sqrt[3]{3} + 1}$. (b) Cho $x = \dfrac{2}{2\sqrt[3]{2} + 2 + \sqrt[3]{4}}$, $y = \dfrac{6}{2\sqrt[3]{2} - 2 + \sqrt[3]{4}}$. Tính giá trị của biểu thức $P = \dfrac{xy}{x + y}$.
\end{baitoan}

\begin{baitoan}[\cite{Tuyen_Toan_9}, 52., p. 21]
	Cho $x = \dfrac{\sqrt[3]{8 - 3\sqrt{5}} + \sqrt[3]{64 - 12\sqrt{20}}}{\sqrt[3]{57}}\sqrt[3]{8 + 3\sqrt{5}}$, $y = \dfrac{\sqrt[3]{9} - \sqrt{2}}{\sqrt[3]{3} + \sqrt[4]{2}} + \dfrac{\sqrt{2} - 9\sqrt[3]{9}}{\sqrt[4]{2} - \sqrt[3]{81}}$. Tính $xy$.
\end{baitoan}

\begin{baitoan}[\cite{Tuyen_Toan_9}, 53., p. 22]
	Tính: (a) $x = \sqrt[3]{5 + 2\sqrt{13}} + \sqrt[3]{5 - 2\sqrt{13}}$. (b) $x = \sqrt[3]{\sqrt{5} + 2} - \sqrt[3]{\sqrt{5} - 2}$. (c) $x = \sqrt[3]{182 + \sqrt{33125}} + \sqrt[3]{182 - \sqrt{33125}}$.
\end{baitoan}

\begin{baitoan}[\cite{Tuyen_Toan_9}, 54., p. 22]
	Cho $A = \sqrt[3]{60 + \sqrt[3]{60 + \sqrt[3]{60 + \cdots + \sqrt[3]{60}}}}$. Chứng minh $3 < A < 3$. Tìm $\lfloor A\rfloor$.
\end{baitoan}

\begin{baitoan}[\cite{Tuyen_Toan_9}, 55., p. 22]
	Cho $A = \sqrt{20 + \sqrt{20 + \sqrt{20 + \cdots + \sqrt{20}}}}$, $B = \sqrt[3]{24 + \sqrt[3]{24 + \sqrt[3]{24 + \cdots + \sqrt[3]{24}}}}$. Chứng minh $7 < A + B < 8$. Tìm $\lfloor A + B\rfloor$.
\end{baitoan}

\begin{baitoan}[\cite{Tuyen_Toan_9}, 56., p. 22]
	So sánh $a = \sqrt[3]{5\sqrt{2}}$ \& $b = \sqrt{5\sqrt[3]{2}}$.
\end{baitoan}

\begin{baitoan}[\cite{Tuyen_Toan_9}, 57., p. 22]
	Cho $ax^3 = by^3 = cz^3$ \& $\dfrac{1}{x} + \dfrac{1}{y} + \dfrac{1}{z} = 1$. Chứng minh $\sqrt[3]{ax^2 + by^2 + cz^2} = \sqrt[3]{a} + \sqrt[3]{b} + \sqrt[3]{c}$.	
\end{baitoan}

\begin{baitoan}[\cite{Tuyen_Toan_9}, 58., p. 22]
	Giải phương trình: (a) $x^3 + x^2 + x = -\frac{1}{3}$. (b) $x^3 + 2x^2 - 4x = -\frac{8}{3}$.
\end{baitoan}

\begin{baitoan}[\cite{Tuyen_Toan_9}, 59., p. 22]
	Giải phương trình: (a) $\sqrt[3]{x + 2} + \sqrt[3]{x - 2} = \sqrt[3]{5x}$. (b) $2\sqrt[3]{(x + 2)^2} - \sqrt[3]{(x - 2)^2} = \sqrt[3]{x^2 - 4}$.
\end{baitoan}

\begin{baitoan}[\cite{Tuyen_Toan_9}, 60., p. 22]
	Giải phương trình: $\sqrt[3]{x - 5} + \sqrt[3]{2x - 1} - \sqrt[3]{3x + 2} = -2$.
\end{baitoan}

\begin{baitoan}[\cite{Tuyen_Toan_9}, 61., p. 22]
	Giải phương trình: $\sqrt[n]{(x - 2)^2} + 4\sqrt[n]{x^2 - 4} = 5\sqrt[n]{(x + 2)^2}$.
\end{baitoan}

\begin{baitoan}[\cite{Tuyen_Toan_9}, 62., p. 22]
	Cho $A = (a + b)(b + c)(c + a)$ trong đó $a,b,c$ là các số thực dương thỏa mãn điều kiện $abc = 1$. Chứng minh $A + 1\ge3(a + b + c)$.
\end{baitoan}


\begin{baitoan}[\cite{Binh_Toan_9_tap_1}, Ví dụ 15, p. 20]
	Chứng tỏ số $m = \sqrt[3]{\sqrt{5} + 2} - \sqrt[3]{\sqrt{5} - 2}$ là 1 nghiệm của phương trình $x^3 + 3x - 4 = 0$.
\end{baitoan}

\begin{baitoan}[\cite{Binh_Toan_9_tap_1}, Ví dụ 16, p. 20]
	Tính giá trị của biểu thức $A = \sqrt[3]{7 + 5\sqrt{2}} + \sqrt[3]{7 - 5\sqrt{2}}$.
\end{baitoan}

\begin{baitoan}[\cite{Binh_Toan_9_tap_1}, 41., p. 20]
	Tính: (a) $\dfrac{\sqrt[3]{4} + \sqrt[3]{2} + 2}{\sqrt[3]{4} + \sqrt[3]{2} + 1}$. (b) $\sqrt{3 + \sqrt{3} + \sqrt[3]{10 + 6\sqrt{3}}}$. (c) $\dfrac{4 + 2\sqrt{3}}{\sqrt[3]{10 + 6\sqrt{3}}}$.
\end{baitoan}

\begin{baitoan}[\cite{Binh_Toan_9_tap_1}, 42., p. 21]
	Số $m = \sqrt[3]{4 + \sqrt{80}} - \sqrt[3]{4 - \sqrt{80}}$ có phải là nghiệm của phương trình $x^3 + 12x - 8 = 0$ không?
\end{baitoan}

\begin{baitoan}[\cite{Binh_Toan_9_tap_1}, 43., p. 21]
	Lập 1 phương trình bậc 3 với các hệ số nguyên, trong đó: (a) $\sqrt[3]{2} + \sqrt[3]{4}$ là 1 nghiệm của phương trình. (b) $\sqrt[3]{9} - \sqrt[3]{3}$ là 1 nghiệm của phương trình.
\end{baitoan}

\begin{baitoan}[\cite{Binh_Toan_9_tap_1}, 44., p. 21]
	Tính: (a) $A = \sqrt[3]{6\sqrt{3} + 10} - \sqrt[3]{6\sqrt{3} - 10}$. (b) $B = \sqrt[3]{5 + 2\sqrt{13}} + \sqrt[3]{5 - 2\sqrt{13}}$. (c) $C = \sqrt[3]{45 + 29\sqrt{2}} + \sqrt[3]{45 - 29\sqrt{2}}$. (d) $D = \sqrt[3]{2 + 10\sqrt{\frac{1}{27}}} + \sqrt[3]{2 - 10\sqrt{\frac{1}{27}}}$. (e) $E = \sqrt[3]{4 + \frac{5}{3}\sqrt{\frac{31}{3}}} + \sqrt[3]{4 - \frac{5}{3}\sqrt{\frac{31}{3}}}$.
\end{baitoan}

\begin{baitoan}[\cite{Binh_Toan_9_tap_1}, 45., p. 21]
	Tìm $x$biết: (a) $\sqrt[3]{2 + x} + \sqrt[3]{2 - x} = 1$. (b) $2x^3 = (x - 1)^3$.
\end{baitoan}

\begin{baitoan}[\cite{Binh_Toan_9_tap_1}, 46., p. 21]
	Cho $am^3 = bn^3 = cp^3$ \& $\frac{1}{m} + \frac{1}{n} + \frac{1}{p} = 1$. Chứng minh: $\sqrt[3]{a} + \sqrt[3]{b} + \sqrt[3]{c} = \sqrt[3]{am^2 + bn^2 + cp^2}$.
\end{baitoan}

\begin{baitoan}[\cite{Binh_Toan_9_tap_1}, 47., p. 21]
	 Tính: (a) $\sqrt[3]{2 - \sqrt{5}}(\sqrt[6]{9 + 4\sqrt{5}} + \sqrt[3]{2 + \sqrt{5}})$. (b) $\sqrt[4]{17 + 12\sqrt{2}} - \sqrt{2}$. (c) $\sqrt[4]{56 - 24\sqrt{5}}$. (d) $1 + \sqrt[4]{28 - 16\sqrt{3}}$. (e) $\dfrac{2}{\sqrt{4 - 3\sqrt[4]{5} + 2\sqrt{5} - \sqrt[4]{125}}}$.
\end{baitoan}

%------------------------------------------------------------------------------%

\section{Miscellaneous}

\begin{baitoan}[\cite{Tuyen_Toan_9}, Thí dụ 15, pp. 29--30]
	Cho biểu thức $A = \left(\dfrac{1}{1 - \sqrt{x}} - \dfrac{1}{\sqrt{x}}\right):\left(\dfrac{2x + \sqrt{x} - 1}{1 - x} + \dfrac{2x\sqrt{x} + x - \sqrt{x}}{1 + x\sqrt{x}}\right)$. (a) Rút gọn $A$. (b) Tính giá trị của $A$ với $x = 7 - 4\sqrt{3}$. (c) Tìm giá trị lớn nhất của $a$ để $P > a$.
\end{baitoan}

\begin{baitoan}[\cite{Tuyen_Toan_9}, 80., p. 31]
	Chứng minh: $\sqrt{\dfrac{1}{a^2} + \dfrac{1}{b^2} + \dfrac{1}{(a + b)^2}} = \left|\dfrac{1}{a} + \dfrac{1}{b} - \dfrac{1}{a + b}\right|$, $\forall a,b\in\mathbb{R}$, $ab(a + b)\ne0$. Áp dụng tính $A = \sqrt{1 + 999^2 + \dfrac{999^2}{1000^2}} + \dfrac{999}{1000}$.
\end{baitoan}

\begin{baitoan}[\cite{Tuyen_Toan_9}, 81., p. 31]
	Rút gọn biểu thức $A = (4 + \sqrt{15})(\sqrt{10} - \sqrt{6})\sqrt{4 - \sqrt{15}}$.
\end{baitoan}

\begin{baitoan}[\cite{Tuyen_Toan_9}, 82., p. 31]
	Không dùng máy tính hoặc bảng số, chứng minh: $\sqrt{14} - \sqrt{13} < 2\sqrt{3} - \sqrt{11}$.
\end{baitoan}

\begin{baitoan}[\cite{Tuyen_Toan_9}, 83., p. 31]
	Giải phương trình: $\dfrac{1}{\sqrt{x + 3} + \sqrt{x + 2}} + \dfrac{1}{\sqrt{x + 2} + \sqrt{x + 1}} + \dfrac{1}{\sqrt{x + 1} + \sqrt{x}} = 1$. 
\end{baitoan}

\begin{baitoan}[\cite{Tuyen_Toan_9}, 84., p. 31]
	Tìm $x,y,z$ biết $x + y + z + 35 = 2(2\sqrt{x + 1} + 3\sqrt{y + 2} +4\sqrt{z + 3})$.
\end{baitoan}

\begin{baitoan}[\cite{Tuyen_Toan_9}, 85., p. 31]
	Cho $a > 0$, $b > 0$ \& $\frac{1}{a} + \frac{1}{b} = 1$. Chứng minh: $\sqrt{a + b} = \sqrt{a - 1} + \sqrt{b - 1}$.
\end{baitoan}

\begin{baitoan}[\cite{Tuyen_Toan_9}, 86., p. 31]
	Chứng minh: $A = \sqrt{8 + 2\sqrt{10 + 2\sqrt{5}}} + \sqrt{8 - 2\sqrt{10 + 2\sqrt{5}}} = \sqrt{2} + \sqrt{10}$.
\end{baitoan}

\begin{baitoan}[\cite{Tuyen_Toan_9}, 87., p. 31]
	Chứng minh:
	\begin{align*}
		\frac{1}{4} < \dfrac{2 - \sqrt{2 + \sqrt{2 + \sqrt{2 + \cdots + \sqrt{2}}}}}{2 - \sqrt{2 + \sqrt{2 + \sqrt{2 + \cdots + \sqrt{2}}}}} < \frac{3}{10},
	\end{align*}
	(ở tử có $n$ dấu căn, ở mẫu có $n - 1$ dấu căn).
\end{baitoan}

\begin{baitoan}[\cite{Tuyen_Toan_9}, 88., p. 31]
	Giải phương trình: $\sqrt{x + 2 - 3\sqrt{2x - 5}} + \sqrt{x - 2 + 3\sqrt{2x - 5}} = 2\sqrt{2}$.
\end{baitoan}

\begin{baitoan}[\cite{Tuyen_Toan_9}, 89., p. 31]
	Giải phương trình: $\sqrt[3]{(65 + x)^2} + 4\sqrt[3]{(65 - x)^2} = 5\sqrt[3]{65^2 - x^2}$.
\end{baitoan}

\begin{baitoan}[\cite{Tuyen_Toan_9}, 90., p. 32]
	Giải phương trình ẩn $x$: $\dfrac{(a - x)\sqrt[4]{x - b} + (x - b)\sqrt[4]{a - x}}{\sqrt[4]{a - x} + \sqrt[4]{x - b}} = \frac{a - b}{2}$ với $a > b$.
\end{baitoan}

\begin{baitoan}[\cite{Tuyen_Toan_9}, 91., p. 32]
	Cho biểu thức $A = \sum_{i=1}^{199} \dfrac{1}{\sqrt{i(200 - i)}} = \dfrac{1}{\sqrt{1\cdot199}} + \dfrac{1}{\sqrt{2\cdot198}} + \cdots + \dfrac{1}{\sqrt{199\cdot1}}$. Chứng minh $A > 1.99$.
\end{baitoan}

\begin{baitoan}[\cite{Tuyen_Toan_9}, 92., p. 32]
	Cho $n$ số dương $a_1,a_2,\ldots,a_n$. Chứng minh:
	\begin{align*}
		\left(\sum_{i=1}^n a_i\right)\left(\sum_{i=1}^n \frac{1}{a_i}\right) = (a_1 + a_2 + \cdots + a_n)\left(\frac{1}{a_1} + \frac{1}{a_2} + \cdots + \frac{1}{a_n}\right)\ge n^2.
	\end{align*}
\end{baitoan}

\begin{baitoan}[\cite{Tuyen_Toan_9}, 93., p. 32]
	Cho các số thực dương $a,b,c,d$ thỏa mãn điều kiện $abcd = 1$. Chứng minh: $a^2 + b^2 + c^2 + d^2+ a(b + c) + b(c + d) + c(d + a) + d(a + b)\ge12$.
\end{baitoan}

\begin{baitoan}[\cite{Tuyen_Toan_9}, 94., p. 32]
	Giải phương trình: $\sqrt{\dfrac{x^2 + x + 1}{x}} + \sqrt{\dfrac{x}{x^2 + x + 1}} = \frac{7}{4}$.
\end{baitoan}

\begin{baitoan}[\cite{Tuyen_Toan_9}, 95., p. 32]
	Giải phương trình: $\sqrt{x + x^2} + \sqrt{x - x^2} = x + 1$.
\end{baitoan}

\begin{baitoan}[\cite{Tuyen_Toan_9}, 96., p. 32]
	Cho $A = \dfrac{x^2 - \sqrt{x}}{x + \sqrt{x} + 1} - \dfrac{x^2 + \sqrt{x}}{x - \sqrt{x} + 1}$ với $0\le x\le 1$. Rút gọn biểu thức $B = 1 - \sqrt{A + x + 1}$.
\end{baitoan}

\begin{baitoan}[\cite{Tuyen_Toan_9}, 97., p. 32]
	Cho biểu thức $A = \dfrac{x\sqrt{x} - 3}{x - 2\sqrt{x} - 3} - \dfrac{2(\sqrt{x} - 3)}{\sqrt{x} + 1} + \dfrac{\sqrt{x} + 3}{3 - \sqrt{x}}$. (a) Rút gọn $A$. (b) Tính giá trị của $A$ với $x = 14 - 6\sqrt{5}$. (c) Tìm GTNN của $A$.
\end{baitoan}

\begin{baitoan}[\cite{TLCT_THCS_Toan_9_dai_so}, Ví dụ 1.1, p. 5]
	Rút gọn biểu thức $A = \sqrt{(7 + 4\sqrt{3})(a - 1)^2}$.
\end{baitoan}

\begin{baitoan}[\cite{TLCT_THCS_Toan_9_dai_so}, Ví dụ 1.2, p. 6]
	Cho biểu thức $A = \sqrt{a + 2\sqrt{a - 1}} + \sqrt{a - 2\sqrt{a - 1}}$. (a) Tìm điều kiện xác định của $A$. (b) Rút gọn biểu thức $A$ với $1\le a < 2$. (c) Rút gọn biểu thức $A$ với $a\ge2$.
\end{baitoan}

\begin{baitoan}[\cite{TLCT_THCS_Toan_9_dai_so}, Ví dụ 1.3, p. 6]
	Đơn giản biểu thức $A = \left(\sqrt{8 + 2\sqrt{7}} + 2\sqrt{8 - 2\sqrt{7}}\right)(\sqrt{63} + 1)$.
\end{baitoan}

\begin{baitoan}[\cite{TLCT_THCS_Toan_9_dai_so}, Ví dụ 1.4, p. 6]
	Tính tổng $A = \dfrac{1}{\sqrt{1} + \sqrt{2}} + \dfrac{1}{\sqrt{2} + \sqrt{3}} + \dfrac{1}{\sqrt{3} + \sqrt{4}}$.
\end{baitoan}

\begin{baitoan}[\cite{TLCT_THCS_Toan_9_dai_so}, Ví dụ 1.5, p. 6]
	Tính $A = \dfrac{\sqrt{7 - 2\sqrt{10}}(7 + 2\sqrt{10})(74 - 22\sqrt{10})}{\sqrt{125} - 4\sqrt{50} + 5\sqrt{20} + \sqrt{8}}$.
\end{baitoan}

\begin{baitoan}[\cite{TLCT_THCS_Toan_9_dai_so}, Ví dụ 1.6, p. 7]
	Cho $a = \sqrt{3 + \sqrt{5 + 2\sqrt{3}}} + \sqrt{3 - \sqrt{5 + 2\sqrt{3}}}$. Chứng minh: $a^2 - 2a - 2 = 0$.
\end{baitoan}

\begin{baitoan}[\cite{TLCT_THCS_Toan_9_dai_so}, Ví dụ 1.7, p. 7]
	Cho $a = \sqrt{4 + \sqrt{10 + 2\sqrt{5}}} + \sqrt{4 - \sqrt{10 + 2\sqrt{5}}}$. Tính
	\begin{align*}
		A = \dfrac{a^4 - 4a^3 + a^2 + 6a + 4}{a^2 - 2a + 12}.
	\end{align*}
\end{baitoan}

\begin{baitoan}[\cite{TLCT_THCS_Toan_9_dai_so}, Ví dụ 1.8, p. 7]
	Cho $f(x) = \dfrac{1 + \sqrt{1 + x}}{x + 1} + \dfrac{1 + \sqrt{1 - x}}{x - 1}$ \& $a = \dfrac{\sqrt{3}}{2}$. Tính $f(a)$.
\end{baitoan}

\begin{baitoan}[\cite{TLCT_THCS_Toan_9_dai_so}, Ví dụ 1.9, p. 8]
	Giả thiết $x,y,z > 0$ \& $xy + yz + zx = a$. Chứng minh
	\begin{align*}
		x\sqrt{\frac{(a + y^2)(a + z^2)}{a + x^2}} + y\sqrt{\frac{(a + z^2)(a + x^2)}{a + y^2}} + z\sqrt{\frac{(a + x^2)(a + y^2)}{a + z^2}} = 2a.
	\end{align*}
\end{baitoan}

\begin{baitoan}[\cite{TLCT_THCS_Toan_9_dai_so}, 1.1, p. 8]
	Biểu diễn $\sqrt{\dfrac{3 + \sqrt{5}}{2}}$ thành $a + b\sqrt{5}$ với $a,b\in\mathbb{Q}$.
\end{baitoan}

\begin{baitoan}[\cite{TLCT_THCS_Toan_9_dai_so}, 1.2, p. 8]
	Đơn giản biểu thức $A = 3\sqrt{2} + 2\sqrt{3} - \sqrt{18} + \sqrt{28 - 16\sqrt{3}}$.
\end{baitoan}

\begin{baitoan}[\cite{TLCT_THCS_Toan_9_dai_so}, 1.3, p. 8]
	Chứng minh $\sqrt{10 + 2\sqrt{24}} - \sqrt{10 - 2\sqrt{24}} = 4$.
\end{baitoan}

\begin{baitoan}[\cite{TLCT_THCS_Toan_9_dai_so}, 1.4, p. 8]
	Tính $A = \sqrt{2 + \sqrt{3}}\cdot\sqrt{2 + \sqrt{2 + \sqrt{3}}}\cdot\sqrt{2 - \sqrt{2 + \sqrt{3}}}$.
\end{baitoan}

\begin{baitoan}[\cite{TLCT_THCS_Toan_9_dai_so}, 1.5, p. 9]
	Tính tích $ab$ với
	\begin{align*}
		a = \sqrt{2 + \sqrt{2}}\sqrt{3 + \sqrt{7 + \sqrt{2}}},\ b = \sqrt{3 + \sqrt{6 + \sqrt{7 + \sqrt{2}}}}\sqrt{3 - \sqrt{6 + \sqrt{7 + \sqrt{2}}}}.
	\end{align*}
\end{baitoan}

\begin{baitoan}[\cite{TLCT_THCS_Toan_9_dai_so}, 1.6, p. 9]
	Chứng minh $\dfrac{4}{\sqrt{5} - 1} + \dfrac{3}{\sqrt{5} - 2} + \dfrac{16}{\sqrt{5} - 3} = -5$.
\end{baitoan}

\begin{baitoan}[\cite{TLCT_THCS_Toan_9_dai_so}, 1.7, p. 9]
	Chứng minh $\left(\dfrac{2}{\sqrt{6} - 1} + \dfrac{3}{\sqrt{6} - 2} + \dfrac{3}{\sqrt{6} - 3}\right)\dfrac{5}{9\sqrt{6} + 4} = \frac{1}{2}$.
\end{baitoan}

\begin{baitoan}[\cite{TLCT_THCS_Toan_9_dai_so}, 1.8, p. 9]
	Cho $f(x) = \dfrac{x + \sqrt{5}}{\sqrt{x} + \sqrt{x + \sqrt{5}}} + \dfrac{x - \sqrt{5}}{\sqrt{x} - \sqrt{x - \sqrt{5}}}$. Tính $f(3)$.
\end{baitoan}

\begin{baitoan}[\cite{TLCT_THCS_Toan_9_dai_so}, 1.9, p. 9]
	Cho $f(x) = \dfrac{\sqrt{x + 1} + \sqrt{x - 1}}{\sqrt{x + 1} - \sqrt{x - 1}}$ \& $a = \dfrac{4}{\sqrt{3} + \frac{1}{\sqrt{3}}}$. Tính $f(a)$.
\end{baitoan}

\begin{baitoan}[\cite{TLCT_THCS_Toan_9_dai_so}, Ví dụ 2.1, p. 10]
	Chứng minh với $ab\ne0$: $\dfrac{\sqrt[3]{a^5b^7}}{\sqrt[3]{a^2b}} - \dfrac{\sqrt[3]{a^4b^8}}{\sqrt[3]{ab^2}} = 0$.
\end{baitoan}

\begin{baitoan}[\cite{TLCT_THCS_Toan_9_dai_so}, Ví dụ 2.2, p. 10]
	Chứng minh với $abc\ne0$: $\dfrac{\sqrt[3]{a^4b^5c^7}}{\sqrt[3]{ab^2c}} = abc^2$.
\end{baitoan}

\begin{baitoan}[\cite{TLCT_THCS_Toan_9_dai_so}, Ví dụ 2.3, p. 10]
	Với $a\ge2 + \sqrt{2}$ \&
	\begin{align*}
		u = \sqrt[3]{\left(a + \frac{2}{a}\right)^3 - 3a^2 - \frac{12}{a^2} + 3\left(a + \frac{2}{a}\right) - 13},\ v = \sqrt{a^2 + \frac{4}{a^2} - 8\left(a + \frac{2}{a}\right) + 20}.
	\end{align*}
	Chứng minh $u - v = 3$.
\end{baitoan}

\begin{baitoan}[\cite{TLCT_THCS_Toan_9_dai_so}, Ví dụ 2.4, p. 11]
	Đơn giản biểu thức $A = \sqrt[3]{8(7 + 5\sqrt{2})} + \sqrt[3]{216(7 - 5\sqrt{2})} + 4\sqrt{2} - 7$.
\end{baitoan}

\begin{baitoan}[\cite{TLCT_THCS_Toan_9_dai_so}, Ví dụ 2.5, p. 11]
	Chứng minh $\sqrt[3]{2 + \sqrt{5}} + \sqrt[3]{2 - \sqrt{5}} = 1$.
\end{baitoan}

\begin{baitoan}[\cite{TLCT_THCS_Toan_9_dai_so}, Ví dụ 2.6, p. 11]
	Chứng minh nếu $a = \sqrt[3]{\sqrt{5} + 2} - \sqrt[3]{\sqrt{5} - 2}$ thì $a^3 + 3a = 4$.
\end{baitoan}

\begin{baitoan}[\cite{TLCT_THCS_Toan_9_dai_so}, Ví dụ 2.7, p. 11]
	Chứng minh:
	\begin{align*}
		\dfrac{\sqrt{\left(\dfrac{9 - 2\sqrt{3}}{\sqrt{3} - \sqrt[3]{2}} + 3\sqrt[3]{2}\right)\sqrt{3}}}{3 + \sqrt[6]{108}} = \sqrt[3]{\sqrt{5} + 2} - \sqrt[3]{\sqrt{5} - 2}.
	\end{align*}
\end{baitoan}

\begin{baitoan}[\cite{TLCT_THCS_Toan_9_dai_so}, Ví dụ 2.8, p. 12]
	Chứng minh nếu $\sqrt[3]{(a + 1)^2} + \sqrt[3]{a^2 - 1} + \sqrt[3]{(a - 1)^2} = 1$ thì $\sqrt[3]{a + 1} - \sqrt[3]{a - 1} = 2$.
\end{baitoan}

\begin{baitoan}[\cite{TLCT_THCS_Toan_9_dai_so}, Ví dụ 2.9, p. 12]
	Đơn giản biểu thức $A = \dfrac{x + 1}{2\sqrt[3]{\sqrt{3} - \sqrt{2}}\sqrt[6]{5 + 2\sqrt{6}} + x + \frac{1}{x}}$ với $x\notin\{-1,0\}$.
\end{baitoan}

\begin{baitoan}[\cite{TLCT_THCS_Toan_9_dai_so}, Ví dụ 2.10, p. 12]
	Cho $a = \sqrt{2} + \sqrt{7 - \sqrt[3]{61 + 46\sqrt{5}}} + 1$. (a) Chứng minh $a^4 - 14a^2 + 9 = 0$. (b) Giả sử $f(x) = x^5 + 2x^4 - 14x^3 - 28x^2 + 9x + 19$. Tính $f(a)$.
\end{baitoan}

\begin{baitoan}[\cite{TLCT_THCS_Toan_9_dai_so}, Ví dụ 2.11, p. 13]
	Cho $a,b,c > 0$. Giả sử $m,n,p$ là những số nguyên dương lớn hơn $1$ sao cho $bc = \sqrt[m]{a}$, $ca = \sqrt[n]{b}$, \& $ab = \sqrt[p]{c}$. Chứng minh trong 3 số $a,b,c$ phải có ít nhất 1 số bằng $1$.
\end{baitoan}

\begin{baitoan}[\cite{TLCT_THCS_Toan_9_dai_so}, Ví dụ 2.12, p. 13]
	Cho $a = \dfrac{\sqrt[3]{7 + 5\sqrt{2}}}{\sqrt{4 + 2\sqrt{3}} - \sqrt{3}}$. (a) Xác định đa thức với hệ số nguyên bậc dương nhỏ nhất nhận số $a$ làm nghiệm. (b) Giả sử đa thức $f(x) = 3x^6 - 4x^5 - 7x^4 + 6x^3 + 6x^2 + x - 53\sqrt{2}$. Tính $f(a)$.
\end{baitoan}

\begin{baitoan}[\cite{TLCT_THCS_Toan_9_dai_so}, Ví dụ 2.13, p. 14]
	Cho $a = \dfrac{7- 4\sqrt{3}}{\sqrt[3]{26 - 15\sqrt{3}}} - \sqrt[3]{26 + 15\sqrt{3}}$. (a) Xác định đa thức với hệ số nguyên bậc dương nhỏ nhất nhận số $a$ làm nghiệm. (b) Giả sử đa thức $f(x) = \dfrac{x^6 + x^4 + 4x^2}{40(x^4 + 4x^2 - 144)}$. Tính $f(a)$.
\end{baitoan}

\begin{baitoan}[\cite{TLCT_THCS_Toan_9_dai_so}, Ví dụ 2.14, p. 14]
	Cho $a = \sqrt[3]{38 + 17\sqrt{5}} + \sqrt[3]{38 - 17\sqrt{5}}$. Giả sử ta có đa thức $f(x) = (x^3 + 3x + 1935)^{2012}$. Tính $f(a)$.
\end{baitoan}

\begin{baitoan}[\cite{TLCT_THCS_Toan_9_dai_so}, 2.1., p. 14]
	Biểu diễn $\sqrt[3]{2 + \sqrt{5}}$ thành $a + b\sqrt{5}$ với $a,b\in\mathbb{Q}$.
\end{baitoan}

\begin{baitoan}[\cite{TLCT_THCS_Toan_9_dai_so}, 2.2., p. 14]
	Cho $a = \sqrt[3]{\sqrt{5} + 2} + \sqrt[3]{1 - \sqrt{11}}$. Chứng minh $a^9 - 6a^6 + 282a^3 = 8$.
\end{baitoan}

\begin{baitoan}[\cite{TLCT_THCS_Toan_9_dai_so}, 2.3., p. 15]
	Cho $a = (\sqrt[3]{1 + 2\sqrt{6}} - \sqrt[6]{5 + 4\sqrt{6}})\sqrt[3]{2\sqrt{6} - 1} + 1$. (a) Xác định đa thức với hệ số nguyên bậc dương nhỏ nhất nhận $a$ làm nghiệm. (b) Giả sử $f(x) = \sum_{i=1}^{2012} ix^i + 2012$. Tính $f(a)$.
\end{baitoan}

\begin{baitoan}[\cite{TLCT_THCS_Toan_9_dai_so}, 2.4., p. 15]
	Chứng minh:
	\begin{align*}
		\frac{a + 2\sqrt{ab} + 9b}{\sqrt{a} + 3\sqrt{b} - 2\sqrt[4]{ab}} - 2\sqrt{b} = \left(\sqrt[4]{a} + \sqrt[4]{b}\right)^2,\ \forall a,b\in\mathbb{R},\,a,b > 0.
	\end{align*}
\end{baitoan}

\begin{baitoan}[\cite{TLCT_THCS_Toan_9_dai_so}, 2.5., p. 15]
	Chứng minh:
	\begin{align*}
		\left(\sqrt[3]{a^4} + b^2\sqrt[3]{a^2} + b^4\right)\frac{\sqrt[3]{a^8} - b^6 + b^4\sqrt[3]{a^2} - a^2b^2}{a^2b^2 + b^2 - a^2b^8 - b^4} = a^2b^2,\ \forall a,b\in\mathbb{R},\,ab\ne0,\,a\ne b^3.
	\end{align*}
\end{baitoan}

\begin{baitoan}[\cite{TLCT_THCS_Toan_9_dai_so}, 2.6., p. 15]
	Cho $a,b > 0$. Đơn giản biểu thức
	\begin{align*}
		A = \frac{\sqrt{a^3 + 2a^2b} + \sqrt{a^4 + 2a^3b} - \sqrt{a^3} - a^2b}{\sqrt{\left(2a + b - \sqrt{a^2 + 2ab}\right)\left(\sqrt[3]{a^2} - \sqrt[6]{a^5} + a\right)}}.
	\end{align*}
\end{baitoan}

\begin{baitoan}[\cite{TLCT_THCS_Toan_9_dai_so}, 2.7., p. 15]
	Giả sử $u^3\ge v^2$, $u,v\in\mathbb{Q}^+$. Xác định $u,v$ để
	\begin{align*}
		\sqrt{\frac{u - 8\sqrt[6]{u^3v^2} + 4\sqrt[3]{v^2}}{\sqrt{u} - 2\sqrt[3]{v} + 2\sqrt[12]{u^3v^2}} + 3\sqrt[3]{v}} + \sqrt[6]{v} = 1.
	\end{align*}
\end{baitoan}

\begin{baitoan}
	Cho $a,b,c,A,B\in\mathbb{Z}$, $c\ge0$ thỏa mãn đẳng thức $(a + b\sqrt{c})^2 = A + B\sqrt{c}$. (a) Tìm mối quan hệ của $a,b,c,A,B$. Biểu diễn $(A,B)$ theo $(a,b,c)$. (b)${}^\star$ Biểu diễn $(a,b)$ theo $(c,A,B)$.
\end{baitoan}

\begin{baitoan}
	Cho $a,b,c,A,B\in\mathbb{Z}$, $c\ge0$ thỏa mãn đẳng thức $(a + b\sqrt{c})^3 = A + B\sqrt{c}$. (a) Tìm mối quan hệ của $a,b,c,A,B$. Biểu diễn $(A,B)$ theo $(a,b,c)$. (b)${}^\star$ Biểu diễn $(a,b)$ theo $(c,A,B)$.
\end{baitoan}

\begin{baitoan}
	Cho $a,b,c,A,B\in\mathbb{Z}$, $c\ge0$ thỏa mãn đẳng thức $(a + b\sqrt[3]{c})^3 = A + B\sqrt[3]{c} + C\sqrt[3]{c^2}$. (a) Tìm mối quan hệ của $a,b,c,A,B,C$. Biểu diễn $(A,B,C)$ theo $(a,b,c)$. (b)${}^\star$ Biểu diễn $(a,b)$ theo $(c,A,B,C)$.
\end{baitoan}

%------------------------------------------------------------------------------%

\printbibliography[heading=bibintoc]
	
\end{document}