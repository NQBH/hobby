\documentclass{article}
\usepackage[backend=biber,natbib=true,style=alphabetic,maxbibnames=10]{biblatex}
\addbibresource{/home/nqbh/reference/bib.bib}
\usepackage[utf8]{vietnam}
\usepackage{tocloft}
\renewcommand{\cftsecleader}{\cftdotfill{\cftdotsep}}
\usepackage[colorlinks=true,linkcolor=blue,urlcolor=red,citecolor=magenta]{hyperref}
\usepackage{amsmath,amssymb,amsthm,caption,float,graphicx,mathtools,subcaption}
\allowdisplaybreaks
\newtheorem{assumption}{Assumption}
\newtheorem{baitoan}{Bài toán}
\newtheorem{cauhoi}{Câu hỏi}
\newtheorem{conjecture}{Conjecture}
\newtheorem{corollary}{Corollary}
\newtheorem{dangtoan}{Dạng toán}
\newtheorem{definition}{Definition}
\newtheorem{dinhly}{Định lý}
\newtheorem{dinhnghia}{Định nghĩa}
\newtheorem{example}{Example}
\newtheorem{ghichu}{Ghi chú}
\newtheorem{hequa}{Hệ quả}
\newtheorem{hypothesis}{Hypothesis}
\newtheorem{lemma}{Lemma}
\newtheorem{luuy}{Lưu ý}
\newtheorem{nhanxet}{Nhận xét}
\newtheorem{notation}{Notation}
\newtheorem{note}{Note}
\newtheorem{principle}{Principle}
\newtheorem{problem}{Problem}
\newtheorem{proposition}{Proposition}
\newtheorem{question}{Question}
\newtheorem{remark}{Remark}
\newtheorem{theorem}{Theorem}
\newtheorem{vidu}{Ví dụ}
\usepackage[left=1cm,right=1cm,top=5mm,bottom=5mm,footskip=4mm]{geometry}
\def\labelitemii{$\circ$}
\DeclareRobustCommand{\divby}{%
	\mathrel{\vbox{\baselineskip.65ex\lineskiplimit0pt\hbox{.}\hbox{.}\hbox{.}}}%
}

\title{Problem \& Solution: Trigonometry -- Bài Tập Lượng Giác \& Lời Giải}
\author{Nguyễn Quản Bá Hồng\footnote{Independent Researcher, Ben Tre City, Vietnam\\e-mail: \texttt{nguyenquanbahong@gmail.com}; website: \url{https://nqbh.github.io}.}}
\date{\today}

\begin{document}
\maketitle
\begin{abstract}
	
\end{abstract}
\tableofcontents

%------------------------------------------------------------------------------%

\section{Hệ Thức về Cạnh \& Đường Cao Trong Tam Giác Vuông}
Trong 1 tam giác vuông, nếu biết 2 cạnh, hoặc 1 cạnh \& 1 góc nhọn thì có thể tính được các góc \& các cạnh còn lại của tam giác đó.

\begin{baitoan}
	Xét $\Delta ABC$ vuông tại $A$, cạnh huyền $BC = a$, 2 cạnh góc vuông $AC = b$, $AB = c$. Gọi $AH = h$ là đường cao ứng với cạnh huyền $BC$\footnote{$AB,AC$ là đường cao ứng với nhau.} \& $CH = b'$, $BH = c'$ lần lượt là hình chiếu của $AC, AB$ trên cạnh huyền $BC$. Chứng minh: (a) $b^2 = ab'$, $c^2 = ac'$. (b) Định lý Pythagore $a^2 = b^2 + c^2$. (c) $h^2 = b'c'$. (d) $bc = ah$. (e) $\frac{1}{h^2} = \frac{1}{b^2} + \frac{1}{c^2}$.
\end{baitoan}

\begin{proof}[Chứng minh]
	(a) $\Delta AHC\backsim\Delta BAC$ (g.g) vì 2 tam giác vuông này có chung $\widehat{C}$ nhọn, nên $\frac{CH}{AC} = \frac{AC}{BC}\Rightarrow AC^2 = BC\cdot CH\Leftrightarrow b^2 = ab'$. Tương tự, $\Delta BHA\backsim\Delta BAC$ (g.g)  vì 2 tam giác vuông này có chung $\widehat{B}$ nhọn, nên $\frac{BH}{AB} = \frac{AB}{BC}\Leftrightarrow AB^2 = BC\cdot BH\Leftrightarrow c^2 = ac'$. (b) Theo (a), $b^2 + c^2 = ab' + ac' = a(b' + c') = a\cdot a = a^2$. (c) Vì $\Delta AHC\backsim\Delta BAC$ \& $\Delta BHA\backsim\Delta BAC$ nên $\Delta AHC\backsim\Delta BHA$, suy ra $\frac{AH}{CH} = \frac{BH}{AH}\Rightarrow AH^2 = BH\cdot CH\Leftrightarrow h^2 = b'c'$. (d) Tính diện tích $\Delta ABC$ theo 2 cách: $S_{\Delta ABC} = \frac{1}{2}AH\cdot BC = \frac{1}{2}AB\cdot AC\Leftrightarrow AH\cdot BC = AB\cdot AC\Leftrightarrow ah = bc$. (e) $ah = bc\Leftrightarrow a^2h^2 = b^2c^2\Leftrightarrow(b^2 + c^2)h^2 = b^2c^2\Leftrightarrow\frac{1}{h^2} = \frac{b^2 + c^2}{b^2c^2} = \frac{1}{b^2} + \frac{1}{c^2}$.
\end{proof}

\begin{luuy}
	Các hệ thức trên có thể được suy ra trực tiếp từ các tỷ số đồng dạng của bộ 3 tam giác đồng dạng: $\Delta BHA\backsim\Delta AHC\backsim\Delta BAC$. Thật vậy, $\Delta AHC\backsim\Delta BAC\Leftrightarrow\frac{AH}{AB} = \frac{HC}{AC} = \frac{AC}{BC}\Leftrightarrow\frac{h}{c} = \frac{b'}{b} = \frac{b}{a}\Rightarrow bh = b'c$, $b^2 = ab'$, \& $ah = bc$. $\Delta BHA\backsim\Delta BAC\Leftrightarrow\frac{HB}{AB} = \frac{AH}{AC} = \frac{AB}{BC}\Leftrightarrow\frac{c'}{c} = \frac{h}{b} = \frac{c}{a}\Rightarrow hc = bc'$, $ah = bc$, \& $c^2 = ac'$. $\Delta AHC\backsim\Delta BHA\Leftrightarrow\frac{AH}{BH} = \frac{CH}{AH} = \frac{AC}{AB}\Leftrightarrow\frac{h}{c'} = \frac{b'}{h} = \frac{b}{c}\Rightarrow h^2 = b'c'$, $bh = b'c$, \& $ch = bc'$. Hơn nữa, $\widehat{BAH} = \widehat{C}$ \& $\widehat{CAH}= \widehat{B}$.
\end{luuy}

\begin{dinhly}[Hệ thức giữa cạnh góc vuông \& hình chiếu của nó trên cạnh huyền]
	\label{thm: b^2 = ab' c^2 = ac'}
	Trong 1 tam giác vuông, bình phương mỗi cạnh góc vuông bằng tích của cạnh huyền \& hình chiếu của cạnh góc vuông đó trên cạnh huyền. Nói cách khác, mỗi cạnh góc vuông là trung bình nhân của cạnh huyền \& hình chiếu của cạnh góc vuông đó trên cạnh huyền. $b^2 = ab'$, $c^2 = ac'$
\end{dinhly}
3 hệ thức về đường cao trong tam giác vuông:

\begin{dinhly}
	\label{thm: h^2 = b'c'}
	Trong 1 tam giác vuông, bình phương đường cao ứng với cạnh huyền bằng tích 2 hình chiếu của 2 cạnh góc vuông trên cạnh huyền. Nói cách khác, đường cao ứng với cạnh huyền là trung bình nhân của 2 đoạn thẳng mà nó định ra trên cạnh huyền. $h^2 = b'c'$.
\end{dinhly}

\begin{dinhly}
	\label{thm: bc = ah}
	Trong 1 tam giác vuông, tích 2 cạnh góc vuông bằng tích của cạnh huyền \& đường cao tương ứng. $bc = ah$.
\end{dinhly}

\begin{dinhly}
	\label{thm: 1/h^2 = 1/b^2 + 1/c^2}
	Trong 1 tam giác vuông, nghịch đảo của bình phương đường cao ứng với cạnh huyền bằng tổng nghịch đảo của bình phương 2 cạnh góc vuông. $\frac{1}{h^2} = \frac{1}{b^2} + \frac{1}{c^2}$.
\end{dinhly}

\begin{baitoan}
	Cho $\Delta ABC$ có độ dài 2 cạnh góc vuông là $b$ \& $c$. Tính độ dài đường cao $h$ xuất phát từ đỉnh góc vuông theo $b,c$.
\end{baitoan}

\begin{proof}[Giải]
	Có $\frac{1}{h^2} = \frac{1}{b^2} + \frac{1}{c^2}\Rightarrow h^2 = \frac{b^2c^2}{b^2 + c^2}$.
\end{proof}

\begin{baitoan}[\cite{SGK_Toan_9_tap_1}, 1., p. 68]
	Tính $x,y$:
	\begin{figure}[H]
		\centering
		\includegraphics[scale=.25]{SGK_Toan_9_4_p68}
	\end{figure}
\end{baitoan}

\begin{proof}[Giải]
	(a) Áp dụng định lý Pythagore, cạnh huyền dài $x + y = \sqrt{6^2 + 8^2} = 10$. Áp dụng định lý \ref{thm: b^2 = ab' c^2 = ac'} được $6^2 = (x + y)x = 10x\Rightarrow x = \frac{6^2}{10} = 3.6$, $y = 10 - x = 10 - 3.6 = 6.4$ (hoặc $8^2 = (x + y)y = 10y\Rightarrow y = \frac{8^2}{10} = 6.4$). (b) Áp dụng định lý \ref{thm: b^2 = ab' c^2 = ac'} được $12^2 = 20x\Rightarrow x = \frac{12^2}{20} = 7.2$, $y = 20 - x = 20 - 7.2 = 12.8$.
\end{proof}

\begin{baitoan}[\cite{SGK_Toan_9_tap_1}, 2., p. 68]
	Tính $x,y$:
	\begin{figure}[H]
		\centering
		\includegraphics[scale=.25]{SGK_Toan_9_5_p68}
	\end{figure}
\end{baitoan}

\begin{proof}[1st giải]
	Áp dụng định lý \ref{thm: b^2 = ab' c^2 = ac'} được $x^2 = (1 + 4)\cdot1 = 5\Rightarrow x = \sqrt{5}$, $y = (1 + 4)4 = 20\Rightarrow y = \sqrt{20} = 2\sqrt{5}$.
\end{proof}

\begin{proof}[2nd giải]
	Áp dụng định lý \ref{thm: b^2 = ab' c^2 = ac'} được $x^2 = (1 + 4)\cdot1 = 5\Rightarrow x = \sqrt{5}$. Áp dụng định lý Pythagore: $y = \sqrt{(1 + 4)^2 - x^2} = \sqrt{5^2 - 5} = \sqrt{20} = 2\sqrt{5}$.
\end{proof}

\begin{baitoan}[\cite{SGK_Toan_9_tap_1}, 3., p. 69]
	Tính $x,y$:
	\begin{figure}[H]
		\centering
		\includegraphics[scale=.25]{SGK_Toan_9_6_p69}
	\end{figure}
\end{baitoan}

\begin{proof}[1st giải]
	Áp dụng định lý Pythagore: $y = \sqrt{5^2 + 7^2} = \sqrt{74}$. Áp dụng định lý \ref{thm: bc = ah} được $5\cdot7 = xy\Rightarrow x = \frac{5\cdot7}{y} = \frac{35}{\sqrt{74}}$.
\end{proof}

\begin{proof}[2nd giải]
	Áp dụng định lý Pythagore: $y = \sqrt{5^2 + 7^2} = \sqrt{74}$. Áp dụng định lý \ref{thm: 1/h^2 = 1/b^2 + 1/c^2} được $\frac{1}{x^2} = \frac{1}{5^2} + \frac{1}{7^2} = \frac{74}{1225}\Rightarrow x^2 = \frac{1225}{74}\Rightarrow x = \sqrt{\frac{1225}{74}} = \frac{35}{\sqrt{74}}$.
\end{proof}

\begin{proof}[3rd giải]
	Áp dụng định lý \ref{thm: 1/h^2 = 1/b^2 + 1/c^2} được $\frac{1}{x^2} = \frac{1}{5^2} + \frac{1}{7^2} = \frac{74}{1225}\Rightarrow x^2 = \frac{1225}{74}\Rightarrow x = \sqrt{\frac{1225}{74}} = \frac{35}{\sqrt{74}}$, \& định lý \ref{thm: bc = ah} được $5\cdot7 = xy\Rightarrow y = \frac{5\cdot7}{x} = \frac{5\cdot7\sqrt{74}}{35} = \sqrt{74}$.
\end{proof}

\begin{baitoan}[\cite{SGK_Toan_9_tap_1}, 4., p. 69]
	Tính $x,y$:
	\begin{figure}[H]
		\centering
		\includegraphics[scale=.25]{SGK_Toan_9_7_p69}
	\end{figure}
\end{baitoan}

\begin{proof}[1st giải]
	Áp dụng định lý \ref{thm: h^2 = b'c'} được $2^2 = 1\cdot x\Leftrightarrow x = 4$. Áp dụng định lý Pythagore: $y = \sqrt{x^2 + 2^2} = \sqrt{4^2 + 2^2} = \sqrt{20} = 2\sqrt{5}$.
\end{proof}

\begin{proof}[1st giải]
	Áp dụng định lý \ref{thm: h^2 = b'c'} được $2^2 = 1\cdot x\Leftrightarrow x = 4$. Áp dụng định lý \ref{thm: b^2 = ab' c^2 = ac'} được $y^2 = (1 + x)x = (1 + 4)\cdot4 = 20\Rightarrow y = 2\sqrt{5}$.
\end{proof}

\begin{baitoan}[\cite{SGK_Toan_9_tap_1}, 5., p. 69]
	Trong tam giác vuông với các cạnh góc vuông có độ dài là $3$ \& $4$, kẻ đường cao ứng với cạnh huyền. Tính đường cao này \& độ dài các đoạn thẳng mà nó định ra trên cạnh huyền.
\end{baitoan}

\begin{proof}[Giải]
	$b = 3$, $c = 4$. Áp dụng định lý Pythagore: $a = \sqrt{b^2 + c^2} = \sqrt{3^2 + 4^2} = 5$. Áp dụng định lý \ref{thm: bc = ah} được $h = \frac{bc}{a} = \frac{3\cdot4}{5} = 2.4$. Áp dụng định lý \ref{thm: b^2 = ab' c^2 = ac'} được $b^2 = ab'\Rightarrow b' = \frac{b^2}{a} = \frac{3^2}{5} = 1.8$, $c' = a - b' = 5 - 1.8 = 3.2$.
\end{proof}

\begin{baitoan}[\cite{SGK_Toan_9_tap_1}, 6., p. 69]
	Đường cao của 1 tam giác vuông chia cạnh huyền thành 2 đoạn thẳng có độ dài là $1$ \& $2$. Tính các cạnh góc vuông của tam giác này.
\end{baitoan}

\begin{proof}[Giải]
	Không mất tính tổng quát, giả sử $b' = 1$, $c' = 2$. Có $a = b' + c' = 1 + 2 = 3$, $b^2 = ab' = 3\cdot1 = 3\Rightarrow b = \sqrt{3}$, $c^2 = ac' = 3\cdot2 = 6\Rightarrow c = \sqrt{6}$ (hoặc áp dụng định lý Pythagore: $c = \sqrt{a^2 - b^2} = \sqrt{3^2 - 3} = \sqrt{6}$).
\end{proof}

\begin{baitoan}[Mở rộng \cite{SGK_Toan_9_tap_1}, 6., p. 69]
	Đường cao của 1 tam giác vuông chia cạnh huyền thành 2 đoạn thẳng có độ dài là $b'$ \& $c'$. Tính các cạnh góc vuông của tam giác này theo $b',c'$.
\end{baitoan}

\begin{proof}[Giải]
	$a = b' + c'$, $b^2 = ab' = (b' + c')b'\Rightarrow b = \sqrt{b'(b' + c')}$, $c^2 = ac' = (b' + c')c'\Rightarrow c = \sqrt{c'(b' + c')}$ (hoặc áp dụng định lý Pythagore: $c = \sqrt{a^2 - b^2} = \sqrt{(b' + c')^2 - b'(b' + c')} = \sqrt{c'(b' + c')}$).
\end{proof}

\begin{baitoan}[\cite{SGK_Toan_9_tap_1}, 7., p. 69--70]
	Người ta đưa ra 2 cách vẽ đoạn trung bình nhân $x$ của 2 đoạn thẳng $a,b$ (i.e., $x^2 = ab$ hay $x = \sqrt{ab}$) như trong 2 hình sau. Chứng minh các 2 vẽ này là đúng.
	\begin{figure}[H]
		\centering
		\includegraphics[scale=.25]{SGK_Toan_9_8_9_p69}
	\end{figure}
\end{baitoan}
\noindent{\sf Hint.} Nếu 1 tam giác có đường trung tuyến ứng với 1 cạnh bằng nửa cạnh đó thì tam giác đó là tam giác vuông.

\begin{proof}[Chứng minh]
	2 tam giác tạo nội tiếp (nửa) đường tròn ngoại tiếp ở 2 hình đều có trung tuyến ứng với cạnh dài nhất bằng nửa cạnh ấy nên là 2 tam giác vuông (vì trung tuyến là 1 bán kính, còn cạnh ứng với trung tuyến đó là 1 đường kính của hình tròn). Với hình đầu, áp dụng định lý \ref{thm: h^2 = b'c'} được $x^2 = ab$. Với hình sau, áp dụng định lý \ref{thm: b^2 = ab' c^2 = ac'} được $x^2 = ab$. Cả 2 trường hợp đều cho $x^2 = ab$, hay $x = \sqrt{ab}$, i.e., $x$ là trung bình nhân của 2 đoạn thẳng $a,b$.
\end{proof}

\begin{baitoan}
	Chứng minh: Nếu 1 tam giác có đường trung tuyến ứng với 1 cạnh bằng nửa cạnh đó thì tam giác đó là tam giác vuông.
\end{baitoan}

\begin{proof}[1st chứng minh]
	Giả sử $\Delta ABC$ có đường trung tuyến $AM = \frac{1}{2}BC$ với $M$ là trung điểm $BC$. $MA = MB = MC\Rightarrow\Delta AMB,\Delta AMC$ đều cân tại $M$, suy ra $\widehat{B} = \widehat{BAM}$, $\widehat{C} = \widehat{CAM}$. Cộng 2 đẳng thức, vế theo vế, suy ra $\widehat{B} + \widehat{C} = \widehat{BAM} + \widehat{CAM} = \widehat{A}$, mà $\widehat{A} + \widehat{B} + \widehat{C} = 180^\circ$, suy ra $\widehat{A} = \widehat{B} + \widehat{C} = \frac{180^\circ}{2} = 90^\circ$. Vậy $\Delta ABC$ vuông tại $A$.
\end{proof}

\begin{proof}[2nd chứng minh]
	Giả sử $\Delta ABC$ có đường trung tuyến $AM = \frac{1}{2}BC$ với $M$ là trung điểm $BC$. $MA = MB = MC$ nên $\Delta ABC$ nội tiếp đường tròn tâm $M$ đường kính $BC$. Vì $BC$ là đường kính nên $\widehat{A} = 90^\circ$.
\end{proof}

\begin{proof}[3rd chứng minh]
	Giả sử $\Delta ABC$ có đường trung tuyến $AM = \frac{1}{2}BC$ với $M$ là trung điểm $BC$. Công thức tính độ dài đường trung tuyến $m_a\coloneqq AM = \sqrt{\dfrac{2b^2 + 2c^2 - a^2}{4}}$.
	\begin{align*}
		AM = \frac{1}{2}BC\Leftrightarrow m_a^2 = \frac{a^2}{4}\Leftrightarrow\frac{2b^2 + 2c^2 - a^2}{4} = \frac{a^2}{4}\Leftrightarrow2b^2 + 2c^2 - a^2 = a^2\Leftrightarrow2b^2 + 2c^2 = 2a^2\Leftrightarrow b^2 + c^2 = a^2.
	\end{align*}
	Áp dụng định lý Pythagore đảo: $\Delta ABC$ vuông tại $A$.
\end{proof}

\begin{luuy}
	Công thức tính độ dài 3 đường trung tuyến của $\Delta ABC$ (xem, e.g., \href{https://vi.wikipedia.org/wiki/Trung_tuy%E1%BA%BFn}{Wikipedia{\rm\texttt{/}}trung tuyến}):
	\begin{align*}
		m_a = \sqrt{\dfrac{2b^2 + 2c^2 - a^2}{4}},\ m_b = \sqrt{\dfrac{2c^2 + 2a^2 - b^2}{4}},\ m_c = \sqrt{\dfrac{2a^2 + 2b^2 - c^2}{4}}.
	\end{align*}
\end{luuy}

\begin{baitoan}[\cite{SGK_Toan_9_tap_1}, 8., p. 70]
	Tính $x,y$:
	\begin{figure}[H]
		\centering
		\begin{subfigure}[H]{0.3\textwidth}
			\centering
			\includegraphics[scale=0.25]{SGK_Toan_9_10_p70}
		\end{subfigure}
		\begin{subfigure}[H]{0.3\textwidth}
			\centering
			\includegraphics[scale=0.25]{SGK_Toan_9_11_p70}
		\end{subfigure}
		\begin{subfigure}[H]{0.3\textwidth}
			\centering
			\includegraphics[scale=0.2]{SGK_Toan_9_12_p70}
		\end{subfigure}
	\end{figure}
\end{baitoan}

\begin{proof}[1st giải]
	(a) Áp dụng định lý \ref{thm: h^2 = b'c'} được $x^2 = 4\cdot9 = 36\Rightarrow x = \sqrt{36} = 6$. (b) Áp dụng định lý \ref{thm: h^2 = b'c'} được $2^2 = x\cdot x = x^2\Rightarrow x = 2$. Áp dụng định lý Pythagore: $y^2 + y^2 = (x + x)^2\Leftrightarrow 2y^2 = 4x^2\Rightarrow y = x\sqrt{2} = 2\sqrt{2}$. (c) Áp dụng định lý \ref{thm: h^2 = b'c'} được $12^2 = 16x\Rightarrow x = \frac{12^2}{16} = 9$. Áp dụng định lý \ref{thm: b^2 = ab' c^2 = ac'} được $y^2 = (x + 16)x = (9 + 16)\cdot9 = 25\cdot9\Rightarrow y = \sqrt{25\cdot9} = 5\cdot3 = 15$.
\end{proof}

\begin{proof}[2nd giải]
	(b) Tam giác vuông này có đường cao ứng với cạnh huyền đồng thời là trung tuyến nên nó là tam giác vuông cân. Vì trung tuyến dài bằng nửa cạnh huyền nên $x = 2$. Tam giác vuông cân có $y = x\sqrt{2} = 2\sqrt{2}$. (c) Áp dụng định lý \ref{thm: h^2 = b'c'} được $12^2 = 16x\Rightarrow x = \frac{12^2}{16} = 9$. Áp dụng định lý Pythagore: $y = \sqrt{x^2 + 12^2} = \sqrt{9^2 + 12^2} = 15$.
\end{proof}

\begin{baitoan}[\cite{SGK_Toan_9_tap_1}, 9., p. 70]
	Cho hình vuông $ABCD$. Gọi $I$ là 1 điểm nằm giữa $A$ \& $B$. Tia $DI$ \& tia $CB$ cắt nhau ở $K$. Kẻ đường thẳng qua $D$, vuông góc với $DI$. Đường thẳng này cắt đường thẳng $BC$ tại $L$. Chứng minh: (a) $\Delta DIL$ là 1 tam giác cân. (b) Tổng $\dfrac{1}{DI^2} + \dfrac{1}{DK^2}$ không đổi khi $I$ thay đổi trên cạnh $AB$.	
\end{baitoan}

\begin{proof}[Chứng minh]
	(a) Xét 2 tam giác vuông $\Delta DAI$ \& $\Delta DCL$: $AD = CD$ (2 cạnh hình vuông $ABCD$), $\widehat{ADI} = \widehat{CDL}$ (cùng phụ $\widehat{CDI}$). Suy ra $\Delta DAI = \Delta DCL$ (cgv--gn) $\Rightarrow DI = DL\Rightarrow\Delta DIL$ cân tại $D$. (b) Xét $\Delta DKL$ vuông tại $D$, áp dụng định lý \ref{thm: 1/h^2 = 1/b^2 + 1/c^2} được: $\dfrac{1}{DL^2} + \dfrac{1}{DK^2} = \dfrac{1}{CD^2}$, mà $DI = DL$, suy ra $\dfrac{1}{DI^2} + \dfrac{1}{DK^2} = \dfrac{1}{CD^2}$ không đổi khi $I$ thay đổi trên cạnh $AB$. 
\end{proof}

\begin{baitoan}[\cite{SBT_Toan_9_tap_1}, 1., p. 102]
	Tính $x,y$:
	\begin{figure}[H]
		\centering
		\includegraphics[scale=.25]{SBT_Toan_9_1_p102}
	\end{figure}
\end{baitoan}

\begin{proof}[Giải]
	(a) Áp dụng định lý Pythagore: $x + y = \sqrt{5^2 + 7^2} = \sqrt{74}$. Áp dụng định lý \ref{thm: b^2 = ab' c^2 = ac'} được $5^2 = (x + y)x = \sqrt{74}x\Rightarrow x = \frac{5^2}{\sqrt{74}} = \frac{25}{\sqrt{74}}$, $7^2 = (x + y)y = \sqrt{74}y\Rightarrow y = \frac{7^2}{\sqrt{74}} = \frac{49}{\sqrt{74}}$ (hoặc $y = \sqrt{74} - x = \sqrt{74} - \frac{25}{\sqrt{74}} = \frac{49}{\sqrt{74}}$). (b) Áp dụng định lý \ref{thm: b^2 = ab' c^2 = ac'} được $14^2 = 16y\Rightarrow y = \frac{14^2}{16} = 12.25$, $x = 16 - y = 16 - 12.25 = 3.75$.
\end{proof}

\begin{baitoan}[\cite{SBT_Toan_9_tap_1}, 2., p. 102]
	Tính $x,y$:
	\begin{figure}[H]
		\centering
		\includegraphics[scale=.25]{SBT_Toan_9_2_p102}
	\end{figure}
\end{baitoan}

\begin{proof}[Giải]
	(a) Áp dụng định lý \ref{thm: b^2 = ab' c^2 = ac'} được $x^2 = (2 + 6)\cdot2 = 16\Rightarrow x = 4$, $y^2 = (2 + 6)\cdot6 = 48\Rightarrow y = \sqrt{48} = 4\sqrt{3}$ (hoặc áp dụng định lý Pythagore: $y = \sqrt{(2 + 6)^2 - x^2} = \sqrt{8^2 - 16} = \sqrt{48} = 4\sqrt{3}$). (b) Áp dụng định lý \ref{thm: h^2 = b'c'} được $x^2 = 2\cdot8 = 16\Rightarrow x = \sqrt{16} = 4$.
\end{proof}

\begin{baitoan}[\cite{SBT_Toan_9_tap_1}, 3., p. 103]
	Tính $x,y$:
	\begin{figure}[H]
		\centering
		\includegraphics[scale=.25]{SBT_Toan_9_3_p103}
	\end{figure}
\end{baitoan}

\begin{proof}[1st giải]
	(a) Áp dụng định lý Pythagore: $y = \sqrt{7^2 + 9^2} = \sqrt{130}$. Áp dụng định lý \ref{thm: bc = ah} được $xy = 7\cdot 9\Rightarrow x = \frac{7\cdot9}{y} = \frac{63}{\sqrt{130}}$. (b) Áp dụng định lý \ref{thm: h^2 = b'c'} được $5^2 = x\cdot x = x^2\Rightarrow x = 5$. Áp dụng định lý Pythagore: $y^2 + y^2 = (x + x)^2\Leftrightarrow 2y^2 = 4x^2\Rightarrow y = x\sqrt{2} = 5\sqrt{2}$.
\end{proof}

\begin{proof}[2nd giải]
	(a) Áp dụng định lý Pythagore: $y = \sqrt{7^2 + 9^2} = \sqrt{130}$. Áp dụng định lý \ref{thm: 1/h^2 = 1/b^2 + 1/c^2} được $\frac{1}{x^2} = \frac{1}{7^2} + \frac{1}{9^2}\Rightarrow x = \sqrt{\frac{7^2\cdot9^2}{7^2 + 9^2}} = \frac{63}{\sqrt{130}}$. (b) Tam giác vuông này có đường cao ứng với cạnh huyền đồng thời là trung tuyến nên nó là tam giác vuông cân. Vì trung tuyến dài bằng nửa cạnh huyền nên $x = 5$. Tam giác vuông cân có $y = x\sqrt{2} = 5\sqrt{2}$.
\end{proof}

\begin{baitoan}[\cite{SBT_Toan_9_tap_1}, 4., p. 103]
	Tính $x,y$:
	\begin{figure}[H]
		\centering
		\includegraphics[scale=.25]{SBT_Toan_9_4_p103}
	\end{figure}
\end{baitoan}

\begin{proof}[1st giải]
	(a) Áp dụng định lý \ref{thm: h^2 = b'c'} được $3^2 = 2x\Rightarrow x = \frac{3^2}{2} = 4.5$. Áp dụng định lý Pythagore: $y = \sqrt{x^2 + 3^2} = \sqrt{4.5^2 + 3^2} = \frac{3\sqrt{13}}{2}$. (b) $\frac{AB}{AC} = \frac{3}{4}\Rightarrow AC = \frac{4}{3}AB = \frac{4}{3}\cdot15 = 20$. Áp dụng định lý Pythagore: $y = \sqrt{15^2 + 20^2} = 25$. Áp dụng định lý \ref{thm: bc = ah} được $xy = 15\cdot20\Rightarrow x = \frac{15\cdot20}{y} = \frac{15\cdot20}{25} = 12$.
\end{proof}

\begin{proof}[2nd giải]
	Áp dụng định lý \ref{thm: h^2 = b'c'} được $3^2 = 2x\Rightarrow x = \frac{3^2}{2} = 4.5$. Áp dụng định lý \ref{thm: b^2 = ab' c^2 = ac'} được $y^2 = (x + 2)x = (4.5 + 2)\cdot4.5 = \frac{117}{4}\Rightarrow y = \sqrt{\frac{117}{4}} = \frac{3\sqrt{13}}{2}$. (b) $\frac{AB}{AC} = \frac{3}{4}\Rightarrow AC = \frac{4}{3}AB = \frac{4}{3}\cdot15 = 20$. Áp dụng định lý Pythagore: $y = \sqrt{15^2 + 20^2} = 25$. Áp dụng định lý \ref{thm: 1/h^2 = 1/b^2 + 1/c^2} được $\frac{1}{x^2} = \frac{1}{15^2} + \frac{1}{20^2} = \frac{1}{144}\Rightarrow x^2 = 144\Rightarrow x = \sqrt{144} = 12$.
\end{proof}

\begin{baitoan}[\cite{SBT_Toan_9_tap_1}, 5., p. 103]
	Cho $\Delta ABC$ vuông tại $A$, đường cao $AH$:
	\begin{figure}[H]
		\centering
		\includegraphics[scale=.25]{SBT_Toan_9_5_p103}
	\end{figure}
	\noindent Giải bài toán trong mỗi trường hợp sau: (a) Cho $AH = 16$, $BH = 25$. Tính $AB,AC,BC,CH$. (b) Cho $AB = 12$, $BH = 6$. Tính $AH,AC,BC,CH$.
\end{baitoan}

\begin{proof}[Giải]
	(a) Áp dụng định lý Pythagore cho $\Delta ABH$ vuông tại $H$: $AB = \sqrt{AH^2 + BH^2} = \sqrt{16^2 + 25^2} = \sqrt{881}$. \texttt{inserting ...}
\end{proof}

\begin{baitoan}[\cite{SBT_Toan_9_tap_1}, 6., p. 103]
	Cho tam giác vuông với các cạnh góc vuông có độ dài là $5$ \& $7$, kẻ đường cao ứng với cạnh huyền. Tính đường cao này \& các đoạn thẳng mà nó chia ra trên cạnh huyền.
\end{baitoan}

\begin{baitoan}[\cite{SBT_Toan_9_tap_1}, 7., p. 103]
	Đường cao của 1 tam giác vuông chia cạnh huyền thành 2 đoạn thẳng có độ dài là $3$ \& $4$. Tính các cạnh góc vuông của tam giác này.
\end{baitoan}

\begin{baitoan}[\cite{SBT_Toan_9_tap_1}, 8., p. 103]
	Cạnh huyền của 1 tam giác vuông lớn hơn 1 cạnh góc vuông là {\rm1 cm} \& tổng của 2 cạnh góc vuông lớn hơn cạnh huyền {\rm4 cm}. Tính các cạnh của tam giác vuông này.
\end{baitoan}

\begin{baitoan}[\cite{SBT_Toan_9_tap_1}, 9., p. 104]
	1 tam giác vuông có cạnh huyền là $5$ \& đường cao ứng với cạnh huyền là $2$. Tính cạnh nhỏ nhất của tam giác vuông này.
\end{baitoan}

\begin{baitoan}[\cite{SBT_Toan_9_tap_1}, 10., p. 104]
	Cho 1 tam giác vuông. Biết tỷ số 2 cạnh góc vuông là $3:4$ \& cạnh huyền là {\rm125 cm}. Tính độ dài các cạnh góc vuông \& hình chiếu của các cạnh góc vuông trên cạnh huyền.
\end{baitoan}

\begin{baitoan}[\cite{SBT_Toan_9_tap_1}, 11., p. 104]
	Cho $\Delta ABC$ vuông tại $A$. Biết $\frac{AB}{AC} = \frac{5}{6}$, đường cao $AH = 30$ {\rm cm}. Tính $HB,HC$.
\end{baitoan}

\begin{baitoan}[\cite{SBT_Toan_9_tap_1}, 12., p. 104]
	2 vệ tinh đang bay ở vị trí A \& B cùng cách mặt đất {\rm230 km} có nhìn thấy nhau hay không nếu khoảng cách giữa chúng theo đường thẳng là {\rm2200 km}? Biết bán kính $R$ của Trái Đất gần bằng {\rm6370 km} \& 2 vệ tinh nhìn thấy nhau nếu $OH > R$.
	\begin{figure}[H]
		\centering
		\includegraphics[scale=.25]{SBT_Toan_9_6_p104}
	\end{figure}
\end{baitoan}

\begin{baitoan}[\cite{SBT_Toan_9_tap_1}, 13., p. 104]
	Cho 2 đoạn thẳng có độ dài là $a,b$. Dựng các đoạn thẳng có độ dài tương ứng bằng: (a) $\sqrt{a^2 + b^2}$. (b) $\sqrt{a^2 - b^2}$ ($a > b$).
\end{baitoan}

\begin{baitoan}[\cite{SBT_Toan_9_tap_1}, 14., p. 104]
	Cho 2 đoạn thẳng có độ dài là $a,b$. Dựng đoạn thẳng $\sqrt{ab}$ như thế nào?
\end{baitoan}

\begin{baitoan}[\cite{SBT_Toan_9_tap_1}, 15., p. 104]
	Giữa 2 tòa nhà (kho \& phân xưởng) của 1 nhà máy người ta xây dựng 1 băng chuyền $AB$ để chuyển vật liệu. Khoảng cách giữa 2 tòa nhà là {\rm10 m}, còn 2 vòng quay của băng chuyền được đặt ở độ cao {\rm 8 m} \& {\rm 4 m} so với mặt đất. Tìm độ dài $AB$ của băng chuyền.
	\begin{figure}[H]
		\centering
		\includegraphics[scale=.25]{SBT_Toan_9_7_p104}
	\end{figure}
\end{baitoan}

\begin{baitoan}[\cite{SBT_Toan_9_tap_1}, 16., p. 104]
	Cho tam giác có độ dài các cạnh là $5,12,13$. Tìm góc của tam giác đối diện với cạnh có độ dài $13$.
\end{baitoan}

\begin{baitoan}[\cite{SBT_Toan_9_tap_1}, 17., p. 104]
	Cho hình chữ nhật $ABCD$. Đường phân giác của góc $B$ cắt đường chéo $AC$ thành 2 đoạn $4\frac{2}{7}$ {\rm m} \& $5\frac{5}{7}$ {\rm m}. Tính các kích thước của hình chữ nhật.
\end{baitoan}

\begin{baitoan}[\cite{SBT_Toan_9_tap_1}, 18., p. 105]
	Cho $\Delta ABC$ vuông tại $A$, vẽ đường cao $AH$. Chu vi của $\Delta ABH$ là {\rm30 cm} \& chu vi $\Delta ACH$ là {\rm40 cm}. Tính chu vi của $\Delta ABC$.
\end{baitoan}

\begin{baitoan}[\cite{SBT_Toan_9_tap_1}, 19., p. 105]
	Cho $\Delta ABC$ vuông tại $A$ có cạnh $AB = 6$ {\rm cm} \& $AC = 8$ {\rm cm}. Các đường phân giác trong \& ngoài của góc $B$ cắt đường thẳng $ABC$ lần lượt tại $M,N$. Tính độ dài các đoạn thẳng $AM,AN$.
\end{baitoan}

\begin{baitoan}[\cite{SBT_Toan_9_tap_1}, 20., p. 105]
	Cho $\Delta ABC$. Từ 1 điểm $M$ bất kỳ trong tam giác kẻ $MD,ME,MF$ lần lượt vuông góc với các cạnh $BC,CA,AB$. Chứng minh: $BD^2 + CE^2 + AF^2 = CD^2 + AE^2 + BF^2$.
	\begin{figure}[H]
		\centering
		\includegraphics[scale=.25]{SBT_Toan_9_8_p105}
	\end{figure}
\end{baitoan}

\begin{baitoan}[\cite{SBT_Toan_9_tap_1}, 1.1., p. 105]
	Cho $\Delta ABC$ vuông tại $A$ có $AB:AC = 3:4$ \& đường cao $AH$ bằng {\rm9 cm}. Tính độ dài đoạn thẳng $CH$.
\end{baitoan}

\begin{baitoan}[\cite{SBT_Toan_9_tap_1}, 1.2., p. 105]
	Cho $\Delta ABC$ vuông tại $A$ có $AB:AC = 4:5$ \& đường cao $AH$ bằng {\rm12 cm}. Tính độ dài đoạn thẳng $BH$.
\end{baitoan}

\begin{baitoan}[\cite{SBT_Toan_9_tap_1}, 1.3., p. 105]
	(a) Tính $h,b,c$ nếu biết $b' = 36$, $c' = 64$. (b) Tính $h,b,b',c'$ nếu biết $a = 9$, $c = 6$.
\end{baitoan}

\begin{baitoan}[\cite{SBT_Toan_9_tap_1}, 1.4., p. 105]
	Biểu thị $b',c'$ qua $a,b,c$.
\end{baitoan}

\begin{baitoan}[\cite{SBT_Toan_9_tap_1}, 1.5., p. 105]
	Chứng minh: (a) $h = \dfrac{bc}{a}$. (b) $\dfrac{b^2}{c^2} = \dfrac{b'}{c'}$.
\end{baitoan}

\begin{baitoan}[\cite{SBT_Toan_9_tap_1}, 1.6., p. 106]
	Đường cao của 1 tam giác vuông kẻ từ đỉnh góc vuông chia cạnh huyền thành 2 đoạn, trong đó đoạn lớn hơn bằng {\rm9 cm}. Tính cạnh huyền của tam giác vuông đó nếu 2 cạnh góc vuông có tỷ lệ $6:5$.
\end{baitoan}

\begin{baitoan}[\cite{SBT_Toan_9_tap_1}, 1.7., p. 106]
	Trong tam giác có các cạnh là {\rm5 cm, 12 cm, 13 cm}, kẻ đường cao đến cạnh lớn nhất. Tính các đoạn thẳng mà đường cao này chia ra trên cạnh lớn nhất đó.
\end{baitoan}

\begin{baitoan}[\cite{SBT_Toan_9_tap_1}, 1.8., p. 106]
	$\Delta ABC$ vuông tại $A$ có đường cao $AH$ bằng {\rm12 cm}. Tính cạnh huyền $BC$ nếu biết $HB:HC = 1:3$.
\end{baitoan}

\begin{baitoan}[\cite{SBT_Toan_9_tap_1}, 1.9., p. 106]
	Cho $\Delta ABC$ vuông cân tại $A$, đường trung tuyến $BM$. Gọi $D$ là chân đường vuông góc kẻ từ $C$ đến $BM$ \& $H$ là chân đường vuông góc kẻ từ $D$ đến $AC$. {\rm Đ\texttt{/}S?} (a) $\Delta HCD\backsim\Delta ABM$. (b) $AH = 2HD$.
\end{baitoan}

\begin{baitoan}[\cite{SBT_Toan_9_tap_1}, 1.10., p. 106]
	Cho hình thang $ABCD$ vuông tại $A$ có cạnh đáy $AB$ bằng {\rm6 cm}, cạnh bên $AD$ bằng {\rm4 cm} \& 2 đường chéo vuông góc với nhau. Tính độ dài các cạnh $CD,BC$, \& đường chéo $BD$.
\end{baitoan}

\begin{baitoan}[\cite{Tuyen_Toan_9}, Thí dụ 1, p. 103]
	Cho hình thang $ABCD$ có $\widehat{B} = \widehat{C} = 90^\circ$, 2 đường chéo vuông góc với nhau tại $H$. Biết $AB = 3\sqrt{5}$ {\rm cm}, $HA = 3$ {\rm cm}. Chứng minh: (a) $HA:HB:HC:HD = 1:2:4:8$. (b) $\dfrac{1}{AB^2} - \dfrac{1}{CD^2} = \dfrac{1}{HB^2} - \dfrac{1}{HC^2}$.
\end{baitoan}

\begin{baitoan}[\cite{Tuyen_Toan_9}, 1., p. 105]
	Cho hình thang $ABCD$, $AB\parallel CD$, 2 đường chéo vuông góc với nhau. Biết $AC = 16$ {\rm cm}, $BD = 12$ {\rm cm}. Tính chiều cao của hình thang.
\end{baitoan}

\begin{baitoan}[\cite{Tuyen_Toan_9}, 2., p. 105]
	Cho $\Delta ABC$ vuông tại $A$, đường cao $AH$, đường phân giác $AD$. Biết $BH = 63$ {\rm cm}, $CH = 112$ {\rm cm}, tính $HD$.
\end{baitoan}

\begin{baitoan}[\cite{Tuyen_Toan_9}, 3., p. 105]
	Cho $\Delta ABC$ vuông tại $A$. 2 đường trung tuyến $AD,BE$ vuông góc với nhau tại $G$. Biết $AB = \sqrt{6}$ {\rm cm}. Tính cạnh huyền $BC$.
\end{baitoan}

\begin{baitoan}[\cite{Tuyen_Toan_9}, 4., p. 105]
	Gọi $a,b,c$ là các cạnh của 1 tam giác vuông, $h$ là đường cao ứng với cạnh huyền $a$. Chứng minh tam giác có các cạnh $a + h,b + c$, \& $h$ cũng là 1 tam giác vuông.
\end{baitoan}

\begin{baitoan}[\cite{Tuyen_Toan_9}, 5., p. 105]
	Cho $\Delta ABC$ vuông tại $A$, đường cao $AH$. Gọi $I,K$ thứ tự là hình chiếu của $H$ trên $AB,AC$. Đặt $c = AB$, $b = AC$. (a) Tính $AI,AK$ theo $b,c$. (b) Chứng minh $\dfrac{BI}{CK} = \dfrac{c^3}{b^3}$.
\end{baitoan}

\begin{baitoan}[\cite{Tuyen_Toan_9}, 6., p. 105]
	Cho $\Delta ABC$, $AB = 1$, $\widehat{A} = 105^\circ$, $\widehat{B} = 60^\circ$. Trên cạnh $BC$ lấy điểm $E$ sao cho $BE = 1$. Vẽ $ED\parallel AB$, $D\in AC$. Chứng minh: $\dfrac{1}{AC^2} + \dfrac{1}{AD^2} = \dfrac{4}{3}$.
\end{baitoan}

\begin{baitoan}[\cite{Tuyen_Toan_9}, 7., p. 105]
	Cho hình chữ nhật $ABCD$, $AB = 2BC$. Trên cạnh $BC$ lấy điểm $E$. Tia $AE$ cắt đường thẳng $CD$ tại $F$. Chứng minh: $\dfrac{1}{AB^2} = \dfrac{1}{AE^2} + \dfrac{1}{4AF^2}$.
\end{baitoan}

\begin{baitoan}[\cite{Tuyen_Toan_9}, 8., p. 105]
	Cho 3 đoạn thẳng có độ dài $a,b,c$. Dựng đoạn thẳng $x$ sao cho $\dfrac{1}{x^2} = \dfrac{1}{a^2} + \dfrac{1}{b^2} + \dfrac{1}{c^2}$.
\end{baitoan}

\begin{baitoan}[\cite{Tuyen_Toan_9}, 9., p. 105]
	Cho hình thoi $ABCD$ có $\widehat{A} = 120^\circ$. 1 đường thẳng $d$ không cắt các cạnh của hình thoi. Chứng minh: tổng các bình phương hình chiếu của $4$ cạnh với $2$ lần bình phương hình chiếu của đường chéo $AC$ trên đường thẳng $d$ không phụ thuộc vào vị trí của đường thẳng $d$.
\end{baitoan}

\begin{baitoan}[\cite{Tuyen_Toan_9}, 10., p. 106]
	Cho $\Delta ABC$ vuông tại $A$. Từ 1 điểm $O$ ở trong tam giác ta vẽ $OD\bot BC$, $OE\bot CA$, $OF\bot AB$. Xác định vị trí của $O$ để $OD^2 + OE^2 + OF^2$ nhỏ nhất.
\end{baitoan}

\begin{baitoan}[\cite{TLCT_THCS_Toan_9_hinh_hoc}, Ví dụ 1, p. 5]
	Cho $\Delta ABC$ vuông tại $A$, đường cao $AH$. Biết $AB:AC = 3:4$ \& $AB + AC = 21$ {\rm cm}. (a) Tính các cạnh của $\Delta ABC$. (b) Tính độ dài các đoạn $AH,BH,CH$.
\end{baitoan}

\begin{baitoan}[Mở rộng \cite{TLCT_THCS_Toan_9_hinh_hoc}, Ví dụ 1, p. 5]
	Cho $\Delta ABC$ vuông tại $A$, đường cao $AH$. Biết $AB:AC = m:n$ \& $AB + AC = p$ {\rm cm}. (a) Tính các cạnh của $\Delta ABC$. (b) Tính độ dài các đoạn $AH,BH,CH$.
\end{baitoan}

\begin{baitoan}[\cite{TLCT_THCS_Toan_9_hinh_hoc}, Ví dụ 2, p. 6]
	Cho hình thang $ABCD$ có $\widehat{A} = \widehat{D} = 90^\circ$, $\widehat{B} = 60^\circ$, $CD = 30$ {\rm cm}, $CA\bot CB$. Tính diện tích của hình thang.
\end{baitoan}

\begin{baitoan}[\cite{TLCT_THCS_Toan_9_hinh_hoc}, Ví dụ 3, p. 7]
	Cho $\Delta ABC$ nhọn, đường cao $CK$, $H$ là trực tâm. Gọi $M$ là 1 điểm trên $CK$ sao cho $\widehat{AMB} = 90^\circ$. $S,S_1,S_2$ theo thứ tự là diện tích các $\Delta AMB,\Delta ABC,\Delta ABH$. Chứng minh $S = \sqrt{S_1S_2}$.
\end{baitoan}

\begin{baitoan}[\cite{TLCT_THCS_Toan_9_hinh_hoc}, 1.1., p. 7]
	Cho $\Delta ABC$ vuông cân tại $A$ \& điểm $M$ nằm giữa $B$ \& $C$ Gọi $D,E$ lần lượt là hình chiếu của điểm $M$ lên $AB,AC$. Chứng minh $MB^2 + MC^2 = 2MA^2$.
\end{baitoan}

\begin{baitoan}[\cite{TLCT_THCS_Toan_9_hinh_hoc}, 1.2., p. 7]
	Cho hình chữ nhật $ABCD$ \& điểm $O$ nằm trong hình chữ nhật đó. Chứng minh $OA^2 + OC^2 = OB^2 + CD^2$.
\end{baitoan}

\begin{baitoan}[\cite{TLCT_THCS_Toan_9_hinh_hoc}, 1.3., p. 8]
	Cho hình chữ nhật $ABCD$ có $AD = 6$ {\rm cm}, $CD = 8$ {\rm cm}. Đường thẳng kẻ từ $D$ vuông góc với $AC$ tại $E$, cắt cạnh $AB$ tại $F$. Tính độ dài các đoạn thẳng $DE,DF,AE,CE,AF,BF$.
\end{baitoan}

\begin{baitoan}[\cite{TLCT_THCS_Toan_9_hinh_hoc}, 1.4., p. 8]
	Cho $\Delta ABC$ có $AB = 3$  {\rm cm}, $BC = 4$ {\rm cm}, $AC = 5$ {\rm cm}. Đường cao, đường phân giác, đường trung tuyến của tam giác kẻ từ đỉnh $B$ chia tam giác thành $4$ tam giác không có điểm trong chung. Tính diện tích của mỗi tam giác đó.
\end{baitoan}

\begin{baitoan}[\cite{TLCT_THCS_Toan_9_hinh_hoc}, 1.5., p. 8]
	Trong 1 tam giác vuông tỷ số giữa đường cao \& đường trung tuyến kẻ từ đỉnh góc vuông bằng $40:41$. Tính độ dài các cạnh góc vuông của tam giác đó, biết cạnh huyền bằng $\sqrt{41}$ {\rm cm}.
\end{baitoan}

\begin{baitoan}[\cite{TLCT_THCS_Toan_9_hinh_hoc}, 1.6., p. 8]
	Cho $\Delta ABC$ vuông tại $A$, đường cao $AH$. Kẻ $HE\bot AB$, $HF\bot AC$. Gọi $O$ là giao điểm của $AH$ \& $EF$. Chứng minh $HB\cdot HC = 4OE\cdot OF$.
\end{baitoan}

\begin{baitoan}[\cite{TLCT_THCS_Toan_9_hinh_hoc}, 1.7., p. 8]
	Cho $\Delta ABC$ nhọn, 2 đường cao $BD,CE$ cắt nhau tại $H$. Gọi $M,N$ lần lượt là các điểm thuộc $CH,BH$ sao cho $\widehat{AMB} = \widehat{ANC} = 90^\circ$. $\Delta AMN$ là tam giác gì? Vì sao?
\end{baitoan}

\begin{baitoan}[\cite{TLCT_THCS_Toan_9_hinh_hoc}, 1.8., p. 8]
	Cho hình vuông $ABCD$ cạnh bằng $a$. (a) $M$ là 1 điểm trên cạnh $AD$ sao cho $\widehat{ABM} = 30^\circ$. Tính $AM,BM$ theo $a$. (b) Qua $A$ kẻ đường thẳng vuông góc với $BM$ tại $F$, đường thẳng này cắt $CD$ tại $N$. Tính độ dài các đoạn thẳng $AF,MF,BF$ theo $a$.
\end{baitoan}

\begin{baitoan}[\cite{TLCT_THCS_Toan_9_hinh_hoc}, 1.9., p. 8]
	Cho hình vuông $ABCD$ \& điểm $I$ thay đổi nằm giữa $A,B$. Tia $DI$ cắt $BC$ tại $E$. Đường thẳng kẻ qua $D$ vuông góc với $DE$ cắt $BC$ tại $F$. Chứng minh tổng $\dfrac{1}{DI^2} + \dfrac{1}{DE^2}$ không phụ thuộc vào vị trí của điểm $I$.
\end{baitoan}

\begin{baitoan}[\cite{TLCT_THCS_Toan_9_hinh_hoc}, 1.10., p. 8]
	Cho $\Delta ABC$, đường cao $BH$. Đặt $BC = a$, $CA = b$, $AB = c$, $AH = c'$. Chứng minh: (a) Nếu $\widehat{A} < 90^\circ$ thì $a^2 = b^2 + c^2 - 2bc'$. (b) Nếu $\widehat{A} > 90^\circ$ thì $a^2 = b^2 + c^2 + 2bc'$.
\end{baitoan}

\begin{baitoan}[\cite{TLCT_THCS_Toan_9_hinh_hoc}, 1.11., p. 8]
	Cho $\Delta ABC$, đường cao $AH$. Biết $AB = 8$ {\rm cm}, $BC - AC = 2$ {\rm cm}, $\widehat{BAH} = 30^\circ$. Tính diện tích $\Delta ABC$.
\end{baitoan}

\begin{baitoan}[\cite{TLCT_THCS_Toan_9_hinh_hoc}, 1.12., p. 8]
	Cho $\Delta ABC$, các đường cao ứng với các cạnh $a,b,c$ lần lượt là $h_a,h_b,h_c$. Chứng minh nếu $\frac{1}{h_a^2} = \frac{1}{h_b^2} + \frac{1}{h_c^2}$ thì $\Delta ABC$ vuông tại $A$.
\end{baitoan}

\begin{baitoan}[\cite{TLCT_THCS_Toan_9_hinh_hoc}, 1.13., p. 9]
	Cho $\Delta ABC$, 2 đường phân giác $BD,CE$ cắt nhau tại $I$ thỏa mãn $BD\cdot CE= 2BI\cdot CI$. $\Delta ABC$ là tam giác gì? Vì sao?
\end{baitoan}

\begin{baitoan}[\cite{TLCT_THCS_Toan_9_hinh_hoc}, 1.14., p. 9]
	Cho $\Delta ABC$, $\widehat{A} = 90^\circ$, $BC = 2a$, đường cao $AH$. Kẻ $HD\bot AC$, $HE\bot AB$. Tìm giá trị lớn nhất của: (a) Độ dài đoạn thẳng $DE$. (b) Diện tích tứ giác $ADHE$.
\end{baitoan}

\begin{baitoan}[\cite{TLCT_THCS_Toan_9_hinh_hoc}, 1.15., p. 9]
	Cho $\Delta ABC$ đều có cạnh bằng {\rm60 cm}. Trên đoạn $BC$ lấy điểm $D$ sao cho $BD = 20$ {\rm cm}. Đường trung trực của $AD$ cắt $AB$ tại $E$, cắt $AC$ tại $F$. Tính độ dài các cạnh của $\Delta DEF$.
\end{baitoan}

\begin{baitoan}[\cite{TLCT_THCS_Toan_9_hinh_hoc}, 1.16., p. 9]
	Cho $\Delta ABC$. Đường trung tuyến $AD$, đường cao $BH$, đường phân giác $CE$ đồng quy. Chứng minh đẳng thức $(AB + CA)(BC^2 + CA^2 - AB^2) = 2BC\cdot CA^2$ hay $(b + c)(a^2 + b^2 - c^2) = 2ab^2$.
\end{baitoan}

\begin{baitoan}[Tổng quát ${}^\star$]
	Cho $\Delta ABC$ vuông tại $A$. (a) Cho trước 2 trong 8 số $a,b,c,b',c',h,p,S$. Tính 6 số còn lại theo 2 số đã cho. (b) Cho trước 2 trong 14 số $a,b,c,b',c',h,m_a,m_b,m_c,d_a,d_b,d_c,p,S$ với $d_a,d_b,d_c$ lần lượt là 3 đường phân giác ứng với $BC,CA,AB$. Tính 12 số còn lại theo 2 số đã cho.
\end{baitoan}

%------------------------------------------------------------------------------%

\section{Tỷ Số Lượng Giác của Góc Nhọn}

\begin{baitoan}[\cite{Tuyen_Toan_9}, Thí dụ 2, p. 107]
	Cho $\cot\alpha = \dfrac{a^2 - b^2}{2ab}$ trong đó $\alpha$ là góc nhọn, $a > b > 0$. Tính $\cos\alpha$.
\end{baitoan}

\begin{baitoan}[\cite{Tuyen_Toan_9}, 11., p. 108, định lý sin]
	Cho $\Delta ABC$ nhọn, $BC = a$, $CA = b$, $AB = c$. Chứng minh: $\dfrac{a}{\sin A} = \dfrac{b}{\sin B} = \dfrac{c}{\sin C}$. Đẳng thức này còn đúng với tam giác vuông \& tam giác tù hay không?
\end{baitoan}

\begin{baitoan}[\cite{Tuyen_Toan_9}, 12., p. 108]
	Chứng minh: (a) $1 + \tan^2\alpha = \dfrac{1}{\cos^2\alpha}$. (b) $1 + \cot^2\alpha = \dfrac{1}{\sin^2\alpha}$. (c) $\cot^2\alpha - \cos^2\alpha = \cot^2\alpha\cdot\cos^2\alpha$. (d) $\dfrac{1 + \cos\alpha}{\sin\alpha} = \dfrac{\sin\alpha}{1 - \cos\alpha}$.
\end{baitoan}

\begin{baitoan}[\cite{Tuyen_Toan_9}, 13., p. 108]
	Rút gọn biểu thức: (a) $A = \dfrac{1 + 2\sin\alpha\cdot\cos\alpha}{\cos^2\alpha - \sin^2\alpha}$. (b) $B = (1 + \tan^2\alpha)(1 - \sin^2\alpha) - (1 + \cot^2\alpha)(1 - \cos^2\alpha)$. (c) $C = \sin^6\alpha + \cos^6\alpha + 3\sin^2\alpha\cos^2\alpha$.
\end{baitoan}

\begin{baitoan}[\cite{Tuyen_Toan_9}, 14., p. 108]
	Tính giá trị của biểu thức $A = 5\cos^2\alpha + 2\sin^2\alpha$ biết $\sin\alpha = \frac{2}{3}$.
\end{baitoan}

\begin{baitoan}[\cite{Tuyen_Toan_9}, 15., p. 108]
	Không dùng máy tính hoặc bảng số, tính: (a) $A = \cos^2 20^\circ + \cos^2 30^\circ + \cos^2 40^\circ + \cos^2 50^\circ + \cos^2 60^\circ + \cos^2 70^\circ$. (b) $B = \sin^2 5^\circ + \sin^2 25^\circ + \sin^2 45^\circ + \sin^2 65^\circ + \sin^2 85^\circ$.
\end{baitoan}

\begin{baitoan}[\cite{Tuyen_Toan_9}, 16., p. 108]
	Cho $0^\circ < \alpha < 90^\circ$. Chứng minh: $\sin\alpha < \tan\alpha$, $\cos\alpha < \cot\alpha$. Áp dụng: (a) Sắp xếp các số sau theo thứ tự tăng dần: $\sin 65^\circ,\cos 65^\circ,\tan 65^\circ$. (b) Xác định $\alpha$ thỏa mãn điều kiện: $\tan\alpha > \sin\alpha > \cos\alpha$.
\end{baitoan}

\begin{baitoan}[\cite{Tuyen_Toan_9}, 17., p. 108]
	Cho $\Delta ABC$ vuông tại $A$. Biết $\sin B = \frac{1}{4}$, tính $\tan C$.
\end{baitoan}

\begin{baitoan}[\cite{Tuyen_Toan_9}, 18., p. 108]
	Cho biết $\sin\alpha + \cos\alpha = \frac{7}{5}$, $0^\circ < \alpha < 90^\circ$, tính $\tan\alpha$.
\end{baitoan}

\begin{baitoan}[\cite{Tuyen_Toan_9}, 19., p. 109]
	$\Delta ABC$, đường trung tuyến $AM$. Chứng minh nếu $\cot B = 3\cot C$ thì $AM = AC$.
\end{baitoan}

\begin{baitoan}[\cite{Tuyen_Toan_9}, 20., p. 109]
	Cho $\Delta ABC$, trực tâm $H$ là trung điểm của đường cao $AD$. Chứng minh $\tan B\tan C = 2$.
\end{baitoan}

\begin{baitoan}[\cite{Tuyen_Toan_9}, 21., p. 109]
	Cho $\Delta ABC$ nhọn, 2 đường cao $BD,CE$. Chứng minh: (a) $S_{\Delta ADE} = S_{\Delta ABC}\cos^2A$. (b) $S_{BCDE} = S_{\Delta ABC}\sin^2A$.
\end{baitoan}

\begin{baitoan}[\cite{Tuyen_Toan_9}, 22., p. 109]
	Cho $\Delta ABC$ nhọn. Từ 1 điểm $M$ nằm trong tam giác vẽ $MD\bot BC$, $ME\bot AC$, $MF\bot AB$. Chứng minh $\max\{MA,MB,MC\}\ge2\min\{MD,ME,MF\}$, trong đó $\max\{MA,MB,MC\}$ là đoạn thẳng lớn nhất trong các đoạn thẳng $MA,MB,MC$ \& $\min\{MD,ME,MF\}$ là đoạn thẳng nhỏ nhất trong các đoạn thẳng $MD,ME,MF$.
\end{baitoan}

%------------------------------------------------------------------------------%

\section{Hệ Thức về Cạnh \& Góc Trong Tam Giác Vuông}

\begin{baitoan}[\cite{Tuyen_Toan_9}, Thí dụ 3, p. 109]
	Tứ giác $ABCD$ có 2 đường chéo cắt nhau tại $O$. Cho biết $\widehat{AOD} = 70^\circ$, $AC = 5.3$ {\rm cm}, $BD = 4$ {\rm cm}. Tính diện tích tứ giác $ABCD$.
\end{baitoan}

\begin{baitoan}[\cite{Tuyen_Toan_9}, 23., p. 110]
	Chứng minh: (a) Diện tích của 1 tam giác bằng nửa tích của 2 cạnh nhân với sin của góc nhọn tạo bởi các đường thẳng chứa 2 cạnh ấy. (b) Diện tích hình bình hành bằng tích của 2 cạnh kề nhân với sin của góc nhọn tạo bởi các đường thẳng chứa 2 cạnh ấy.
\end{baitoan}

\begin{baitoan}[\cite{Tuyen_Toan_9}, 24., p. 110]
	Cho hình bình hành $ABCD$, $BD\bot BC$. Biết $AB = a$, $\widehat{A} = \alpha$, tính diện tích hình bình hành đó.
\end{baitoan}

\begin{baitoan}[\cite{Tuyen_Toan_9}, 25., p. 110]
	Cho $\Delta ABC$, $\widehat{A} = 120^\circ$, $\widehat{B} = 35^\circ$, $AB = 12.25$ {\rm dm}. Giải $\Delta ABC$.
\end{baitoan}

\begin{baitoan}[\cite{Tuyen_Toan_9}, 26., p. 110]
	Cho $\Delta ABC$ nhọn, $\widehat{A} = 75^\circ$, $AB = 30$ {\rm mm}, $BC = 35$ {\rm mm}. Giải $\Delta ABC$.
\end{baitoan}

\begin{baitoan}[\cite{Tuyen_Toan_9}, 27., p. 110]
	Cho $\Delta ABC$ cân tại $A$, đường cao $BH$. Biết $BH = h$, $\widehat{C} = \alpha$. Giải $\Delta ABC$.
\end{baitoan}

\begin{baitoan}[\cite{Tuyen_Toan_9}, 28., p. 110]
	Hình bình hành $ABCD$ có $\widehat{A} = 120^\circ$, $AB = a$, $BC = b$. Các đường phân giác của 4 góc cắt nhau tạo thành tứ giác $MNPQ$. Tính diện tích tứ giác $MNPQ$.
\end{baitoan}

\begin{baitoan}[\cite{Tuyen_Toan_9}, 29., p. 110]
	Cho $\Delta ABC$, các đường phân giác $AD$, đường cao $BH$, đường trung tuyến $CE$ đồng quy tại điểm $O$. Chứng minh $AC\cos A = BC\cos C$.
\end{baitoan}

%------------------------------------------------------------------------------%

\section{Miscellaneous}

\begin{baitoan}[\cite{Tuyen_Toan_9}, Thí dụ 4, p. 111]
	Cho $\Delta ABC$ vuông tại $A$. Gọi $M,N$ lần lượt là 2 điểm trên cạnh $AB,AC$ sao cho $AM = \frac{1}{3}AB$, $AN = \frac{1}{3}AC$. Biết độ dài $BN = \sin\alpha$, $CM = \cos\alpha$ với $0^\circ < \alpha < 90^\circ$. Tính cạnh huyền $BC$.
\end{baitoan}

\begin{baitoan}[\cite{Tuyen_Toan_9}, 30., p. 112]
	Cho $\Delta ABC$ nhọn, $BC = a$, $AC = b$, $CA = b$ trong đó $b - c = \frac{a}{k}$, $k > 1$. Gọi $h_a,h_b,h_c$ lần lượt là các đường cao hạ từ $A,B,C$. Chứng minh: (a) $\sin A = k(\sin B - \sin C)$. (b) $\frac{1}{h_a} = k\left(\frac{1}{h_b} - \frac{1}{h_c}\right)$.
\end{baitoan}

\begin{baitoan}[\cite{Tuyen_Toan_9}, 31., p. 112]
	Giải $\Delta ABC$ biết $AB = 14$, $BC = 15$, $CA = 13$.
\end{baitoan}

\begin{baitoan}[\cite{Tuyen_Toan_9}, 32., p. 112]
	Cho hình hộp chữ nhật $ABCD.A'B'C'D'$. Biết $\widehat{DC'D'} = 45^\circ$, $\widehat{BC'B'} = 60^\circ$. Tính $\widehat{BC'D}$.
\end{baitoan}

\begin{baitoan}[\cite{Tuyen_Toan_9}, 33., p. 112]
	Cho $\Delta ABC$, $AB = AC = 1$, $\widehat{A} = 2\alpha$, $0^\circ < \alpha < 45^\circ$. Vẽ các đường cao $AD,BE$. (a) Các tỷ số lượng giác $\sin\alpha,\cos\alpha,\sin2\alpha,\cos2\alpha$ được biểu diễn bởi các đoạn thẳng nào? (b) Chứng minh $\Delta ADC\backsim\Delta BEC$, từ đó suy ra các hệ thức sau: $\sin2\alpha = 2\sin\alpha\cos\alpha$, $\cos2\alpha = 1 - 2\sin^2\alpha = 2\cos^2\alpha - 1 = \cos^2\alpha - \sin^2\alpha$. (c) Chứng minh: $\tan2\alpha = \dfrac{2\tan\alpha}{1 - \tan^2\alpha}$, $\cot2\alpha = \dfrac{\cot^2\alpha - 1}{2\cot\alpha}$.
\end{baitoan}

\begin{baitoan}[\cite{Tuyen_Toan_9}, 34., p. 112]
	Cho $\alpha = 22^\circ30'$, tính $\sin\alpha,\cos\alpha,\tan\alpha,\cot\alpha$.
\end{baitoan}

\begin{baitoan}[\cite{Tuyen_Toan_9}, 35., p. 112]
	Cho $\Delta ABC$, đường phân giác $AD$. Biết $AB = c$, $AC = b$, $\widehat{A} = 2\alpha$, $\alpha < 45^\circ$. Chứng minh $AD = \dfrac{2bc\cos\alpha}{b + c}$.
\end{baitoan}

%------------------------------------------------------------------------------%

\printbibliography[heading=bibintoc]
	
\end{document}