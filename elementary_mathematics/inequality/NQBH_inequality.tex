\documentclass{article}
\usepackage[backend=biber,natbib=true,style=alphabetic,maxbibnames=10]{biblatex}
\addbibresource{/home/nqbh/reference/bib.bib}
\usepackage[utf8]{vietnam}
\usepackage{tocloft}
\renewcommand{\cftsecleader}{\cftdotfill{\cftdotsep}}
\usepackage[colorlinks=true,linkcolor=blue,urlcolor=red,citecolor=magenta]{hyperref}
\usepackage{amsmath,amssymb,amsthm,float,graphicx,mathtools}
\allowdisplaybreaks
\newtheorem{assumption}{Assumption}
\newtheorem{baitoan}{Bài toán}
\newtheorem{cauhoi}{Câu hỏi}
\newtheorem{conjecture}{Conjecture}
\newtheorem{corollary}{Corollary}
\newtheorem{dangtoan}{Dạng toán}
\newtheorem{definition}{Definition}
\newtheorem{dinhly}{Định lý}
\newtheorem{dinhnghia}{Định nghĩa}
\newtheorem{example}{Example}
\newtheorem{ghichu}{Ghi chú}
\newtheorem{hequa}{Hệ quả}
\newtheorem{hypothesis}{Hypothesis}
\newtheorem{lemma}{Lemma}
\newtheorem{luuy}{Lưu ý}
\newtheorem{nhanxet}{Nhận xét}
\newtheorem{notation}{Notation}
\newtheorem{note}{Note}
\newtheorem{principle}{Principle}
\newtheorem{problem}{Problem}
\newtheorem{proposition}{Proposition}
\newtheorem{question}{Question}
\newtheorem{remark}{Remark}
\newtheorem{theorem}{Theorem}
\newtheorem{vidu}{Ví dụ}
\usepackage[left=1cm,right=1cm,top=5mm,bottom=5mm,footskip=4mm]{geometry}
\def\labelitemii{$\circ$}
\DeclareRobustCommand{\divby}{%
	\mathrel{\vbox{\baselineskip.65ex\lineskiplimit0pt\hbox{.}\hbox{.}\hbox{.}}}%
}

\title{Elementary Inequality -- Bất Đẳng Thức Sơ Cấp}
\author{Nguyễn Quản Bá Hồng\footnote{Independent Researcher, Ben Tre City, Vietnam\\e-mail: \texttt{nguyenquanbahong@gmail.com}; website: \url{https://nqbh.github.io}.}}
\date{\today}

\begin{document}
\maketitle
\begin{abstract}
	
\end{abstract}
\tableofcontents
\newpage

%------------------------------------------------------------------------------%

\section{Introduction to Inequality}

\begin{definition}[Inequality]
	``In mathematics, an \emph{inequality} is a relation which makes a non-equal comparison between 2 numbers or other mathematical expressions.
\end{definition}
It is used most often to compare 2 numbers on the \href{https://en.wikipedia.org/wiki/Number_line}{number line} by the size. There are several different notations used to represent different kinds of inequalities: The notation $a < b$ means that $a$ is \textit{less than} $b$. The notation $a > b$ means that $a$ is \textit{greater than} $b$. In either case, $a$ is not equal to $b$. These relations are known as \textit{strict inequalities}, meaning that $a$ is strictly less than or strictly greater than $b$. Equivalence is excluded.

In contrast to strict inequalities, there are 2 types of inequality relations that are not strict: The notation $a\le b$ means that $a$ is \textit{less than or equal to} $b$ (or, equivalently, at most $b$, or not greater than $b$). The notation $a\ge b$ means that $a$ is \textit{greater than or equal to} $b$ (or, equivalently, at least $b$, or not less than $b$).

The relation \textit{not great than} can also be represented by $a\not > b$, the symbol for ``greater than'' bisected by a slash, ``not''. The same is true for \textit{not less than} \& $a\not < b$.

The notation $a\ne b$ means that $a$ is not equal to $b$; this \href{https://en.wikipedia.org/wiki/Inequation}{\textit{inequation}} sometimes is considered a form of strict inequality. It does not say that one is greater than the other; it does not even require $a,b$  to be member of an \href{https://en.wikipedia.org/wiki/Ordered_set}{ordered set}.

In engineering science, less formal use of the notation is to state that 1 quantity is ``much greater'' than another, normally by several \href{https://en.wikipedia.org/wiki/Order_of_magnitude}{orders of magnitude}. The notation $a\ll b$ means that $a$ is \textit{much less than} $b$. The notation $a\gg b$ means that $a$ is \textit{much greater than} $b$. This implies that the lesser value can be neglected with little effect on the accuracy of an \href{https://en.wikipedia.org/wiki/Approximation}{approximation} (e.g., the case of \href{https://en.wikipedia.org/wiki/Ultrarelativistic_limit}{ultrarelativistic limit} in physics).

In all of the cases above, any 2 symbols mirroring each other are symmetrical; $a < b$ \& $b > a$ are equivalent, etc.'' -- \href{https://en.wikipedia.org/wiki/Inequality_(mathematics)}{Wikipedia\texttt{/}inequality (mathematics)}

\subsection{Properties on the number line $\mathbb{R}$}



%------------------------------------------------------------------------------%

\printbibliography[heading=bibintoc]

\end{document}