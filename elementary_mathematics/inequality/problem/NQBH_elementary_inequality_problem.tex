\documentclass{article}
\usepackage[backend=biber,natbib=true,style=authoryear]{biblatex}
\addbibresource{/home/nqbh/reference/bib.bib}
\usepackage[utf8]{vietnam}
\usepackage{tocloft}
\renewcommand{\cftsecleader}{\cftdotfill{\cftdotsep}}
\usepackage[colorlinks=true,linkcolor=blue,urlcolor=red,citecolor=magenta]{hyperref}
\usepackage{amsmath,amssymb,amsthm,mathtools,float,graphicx,algpseudocode,algorithm,tcolorbox}
\usepackage[inline]{enumitem}
\allowdisplaybreaks
\numberwithin{equation}{section}
\newtheorem{assumption}{Assumption}[section]
\newtheorem{conjecture}{Conjecture}[section]
\newtheorem{corollary}{Corollary}[section]
\newtheorem{hequa}{Hệ quả}[section]
\newtheorem{definition}{Definition}[section]
\newtheorem{dinhnghia}{Định nghĩa}[section]
\newtheorem{example}{Example}[section]
\newtheorem{vidu}{Ví dụ}[section]
\newtheorem{lemma}{Lemma}[section]
\newtheorem{notation}{Notation}[section]
\newtheorem{principle}{Principle}[section]
\newtheorem{problem}{Problem}[section]
\newtheorem{baitoan}{Bài toán}[section]
\newtheorem{proposition}{Proposition}[section]
\newtheorem{question}{Question}[section]
\newtheorem{cauhoi}{Câu hỏi}[section]
\newtheorem{remark}{Remark}[section]
\newtheorem{luuy}{Lưu ý}[section]
\newtheorem{theorem}{Theorem}[section]
\newtheorem{dinhly}{Định lý}[section]
\usepackage[left=0.5in,right=0.5in,top=1.5cm,bottom=1.5cm]{geometry}
\usepackage{fancyhdr}
\pagestyle{fancy}
\fancyhf{}
\lhead{\small Subsect.~\thesubsection}
\rhead{\small\nouppercase{\leftmark}}
\renewcommand{\subsectionmark}[1]{\markboth{#1}{}}
\cfoot{\thepage}
\def\labelitemii{$\circ$}

\title{Problems in Elementary Inequality}
\author{Nguyễn Quản Bá Hồng\footnote{Independent Researcher, Ben Tre City, Vietnam\\e-mail: \texttt{nguyenquanbahong@gmail.com}; website: \url{https://nqbh.github.io}.}}
\date{\today}

\begin{document}
\maketitle
\begin{abstract}
	A problem set for elementary inequality.
\end{abstract}
\tableofcontents
\newpage

%------------------------------------------------------------------------------%

\section{Easy Problems}

\begin{baitoan}[\cite{Son_Nghiep_Trung_Can2021}, Bổ đề 1.1, p. 5]
	Chứng minh: $4ab\le(a + b)^2\le2(a^2 + b^2)$, hay có thể viết dưới dạng $\frac{a^2 + b^2}{2}\ge\left(\frac{a + b}{2}\right)^2$, $ab\le\frac{(a + b)^2}{4}$, $\forall a,b\in\mathbb{R}$. Đẳng thức xảy ra khi nào?
\end{baitoan}

\begin{proof}[Hint]
	$(a + b)^2 - 4ab = (a - b)^2\ge 0$, $2(a^2 + b^2) - (a + b)^2 = (a - b)^2\ge 0$, $\forall a,b\in\mathbb{R}$. ``$=$'' $\Leftrightarrow a = b$.
\end{proof}

\begin{baitoan}[\cite{Son_Nghiep_Trung_Can2021}, Bổ đề 1.2, p. 5]
	Chứng minh: $3(ab + bc + ca)\le(a + b + c)^2\le3(a^2 + b^2 + c^2)$, hay có thể viết dưới dạng $ab + bc + ca\le\frac{1}{3}(a + b + c)^2$, $\forall a,b,c\in\mathbb{R}$. Đẳng thức xảy ra khi nào?
\end{baitoan}

\begin{proof}[Hint]
	$(a + b + c)^2 - 3(ab + bc + ca) = \frac{1}{2}\left[(a - b)^2 + (b - c)^2 + (c - a)^2\right]\ge 0$, $3(a^2 + b^2 + c^2) - (a + b + c)^2 = (a - b)^2 + (b - c)^2 + (c - a)^2\ge 0$, $\forall a,b,c\in\mathbb{R}$. ``$=$'' $\Leftrightarrow a = b = c$.
\end{proof}

\begin{baitoan}[\cite{Son_Nghiep_Trung_Can2021}, Bổ đề 1.3, p. 6]
	Chứng minh: $\frac{1}{a} + \frac{1}{b}\ge\frac{4}{a + b}$, hay có thể viết dưới dạng $\frac{1}{a + b}\le\frac{1}{4}\left(\frac{1}{a} + \frac{1}{b}\right)$, $\forall a,b > 0$. Đẳng thức xảy ra khi nào?
\end{baitoan}

\begin{proof}[Hint]
	$\frac{1}{a} + \frac{1}{b} - \frac{4}{a + b} = \frac{(a - b)^2}{ab(a + b)}\ge 0$, $\forall a,b > 0$. ``$=$'' $\Leftrightarrow a = b > 0$.
\end{proof}

\begin{baitoan}[\cite{Son_Nghiep_Trung_Can2021}, Bổ đề 1.4, p. 6]
	Chứng minh: $\frac{1}{a} + \frac{1}{b} + \frac{1}{c}\ge\frac{9}{a + b + c}$, hay có thể viết dưới dạng $\frac{1}{a + b + c}\le\frac{1}{9}\left(\frac{1}{a} + \frac{1}{b} + \frac{1}{c}\right)$, $\forall a,b,c > 0$. Đẳng thức xảy ra khi nào?
\end{baitoan}

\begin{baitoan}[\cite{Son_Nghiep_Trung_Can2021}, Mở rộng Bổ đề 1.3--1.4, p. 6 cho $n$ số]
	Chứng minh:
	\begin{align*}
		\frac{1}{a_1} + \ldots + \frac{1}{a_n}\ge\frac{n^2}{a_1 + \cdots + a_n},\mbox{ i.e., }\frac{1}{a_1 + \cdots + a_n}\le\frac{1}{n^2}\left(\frac{1}{a_1} + \cdots + \frac{1}{a_n}\right),\ \forall a_i > 0,\,\forall i = 1,\ldots,n,
	\end{align*}
	hay có thể được viết gọn lại như sau:
	\begin{align*}
		\sum_{i=1}^{n} \frac{1}{a_i}\ge\frac{n^2}{\sum_{i=1}^n a_i},\mbox{ i.e., }\frac{1}{\sum_{i=1}^n a_i}\le\frac{1}{n^2}\sum_{i=1}^n \frac{1}{a_i},\ \forall a_i > 0,\,\forall i = 1,\ldots,n.
	\end{align*}
	Đẳng thức xảy ra khi nào?
\end{baitoan}

\begin{baitoan}[\cite{Son_Nghiep_Trung_Can2021}, Bổ đề 1.5, p. 7]
	Chứng minh: $\sqrt{a + b}\le\sqrt{a} + \sqrt{b}\le\sqrt{2(a + b)}$, $\forall a,b\ge 0$. Đẳng thức xảy ra khi nào?
\end{baitoan}

\begin{baitoan}[\cite{Son_Nghiep_Trung_Can2021}, Mở rộng Bổ đề 1.5, p. 7]
	Chứng minh: $\sqrt{a + b + c}\le\sqrt{a} + \sqrt{b} + \sqrt{c}\le\sqrt{3(a + b + c)}$, $\forall a,b,c\ge 0$. Đẳng thức xảy ra khi nào?
\end{baitoan}

\begin{baitoan}[\cite{Son_Nghiep_Trung_Can2021}, Mở rộng Bổ đề 1.5, p. 7 cho $n$ số]
	Chứng minh: $\sqrt{a_1 + \cdots + a_n}\le\sqrt{a_1} + \cdots + \sqrt{a_n}\le\sqrt{n(a_1 + \cdots + a_n)}$, $\forall a_i\ge 0$, $\forall i = 1,\ldots,n$, hay có thể được viết gọn lại như sau:
	\begin{align*}
		\sqrt{\sum_{i=1}^n a_i}\le\sum_{i=1}^n \sqrt{a_i}\le\sqrt{n\sum_{i=1}^n a_i},\ \forall a_i\ge 0,\,\forall i = 1,\ldots,n.
	\end{align*}
	Đẳng thức xảy ra khi nào?
\end{baitoan}

\begin{baitoan}[\cite{Son_Nghiep_Trung_Can2021}, Bổ đề 1.6, p. 7]
	Chứng minh: $a^3 + b^3\ge ab(a + b)$, $\forall a,b\in\mathbb{R}$, $a + b\ge 0$. Đẳng thức xảy ra khi nào?
\end{baitoan}

\begin{proof}[Hint]
	$a^3 + b^3 - ab(a + b) = (a + b)(a - b)^2\ge 0$, $\forall a,b\in\mathbb{R}$, $a + b\ge 0$. ``$=$'' $\Leftrightarrow a = \pm b$.
\end{proof}

\begin{baitoan}[\cite{Son_Nghiep_Trung_Can2021}, Mở rộng Bổ đề 1.6, p. 7]
	Chứng minh: $a^4 + b^4\ge ab(a^2 + b^2)$, $\forall a,b\in\mathbb{R}$. Đẳng thức xảy ra khi nào?
\end{baitoan}

\begin{baitoan}[\cite{Son_Nghiep_Trung_Can2021}, Bổ đề 1.7, p. 7]
	Chứng minh: $a + b\ge2\sqrt{ab}$, $\forall a,b\ge 0$. Đẳng thức xảy ra khi nào?
\end{baitoan}

\begin{baitoan}[\cite{Son_Nghiep_Trung_Can2021}, Bổ đề 1.8, p. 7]
	Chứng minh: $a + b + c\ge3\sqrt[3]{abc}$, $\forall a,b,c\ge 0$. Đẳng thức xảy ra khi nào?
\end{baitoan}



%------------------------------------------------------------------------------%

\printbibliography[heading=bibintoc]
	
\end{document}