\documentclass{article}
\usepackage[backend=biber,natbib=true,style=alphabetic,maxbibnames=10]{biblatex}
\addbibresource{/home/nqbh/reference/bib.bib}
\usepackage[utf8]{vietnam}
\usepackage{tocloft}
\renewcommand{\cftsecleader}{\cftdotfill{\cftdotsep}}
\usepackage[colorlinks=true,linkcolor=blue,urlcolor=red,citecolor=magenta]{hyperref}
\usepackage{amsmath,amssymb,amsthm,float,graphicx,mathtools}
\allowdisplaybreaks
\newtheorem{assumption}{Assumption}
\newtheorem{baitoan}{Bài toán}
\newtheorem{cauhoi}{Câu hỏi}
\newtheorem{conjecture}{Conjecture}
\newtheorem{corollary}{Corollary}
\newtheorem{dangtoan}{Dạng toán}
\newtheorem{definition}{Definition}
\newtheorem{dinhly}{Định lý}
\newtheorem{dinhnghia}{Định nghĩa}
\newtheorem{example}{Example}
\newtheorem{ghichu}{Ghi chú}
\newtheorem{hequa}{Hệ quả}
\newtheorem{hypothesis}{Hypothesis}
\newtheorem{lemma}{Lemma}
\newtheorem{luuy}{Lưu ý}
\newtheorem{nhanxet}{Nhận xét}
\newtheorem{notation}{Notation}
\newtheorem{note}{Note}
\newtheorem{principle}{Principle}
\newtheorem{problem}{Problem}
\newtheorem{proposition}{Proposition}
\newtheorem{question}{Question}
\newtheorem{remark}{Remark}
\newtheorem{theorem}{Theorem}
\newtheorem{vidu}{Ví dụ}
\usepackage[left=1cm,right=1cm,top=5mm,bottom=5mm,footskip=4mm]{geometry}
\def\labelitemii{$\circ$}
\DeclareRobustCommand{\divby}{%
	\mathrel{\vbox{\baselineskip.65ex\lineskiplimit0pt\hbox{.}\hbox{.}\hbox{.}}}%
}

\title{Problems \textit{\&} Proofs in Elementary Inequality\\Bài Tập Bất Đẳng Thức \textit{\&} Chứng Minh}
\author{Nguyễn Quản Bá Hồng\footnote{Independent Researcher, Ben Tre City, Vietnam\\e-mail: \texttt{nguyenquanbahong@gmail.com}; website: \url{https://nqbh.github.io}.}}
\date{\today}

\begin{document}
\maketitle
\begin{abstract}
	A problem set for elementary inequality.
\end{abstract}
\tableofcontents

%------------------------------------------------------------------------------%

\section{Cauchy-Schwarz Inequality -- Bất Đẳng Thức Cauchy-Schwarz}
The most basic inequality: $x^2\ge0$, $\forall x\in\mathbb{R}$. $x^2 = 0\Leftrightarrow x = 0$. $x^2 > 0\Leftrightarrow x\ne0$.

\begin{baitoan}[Bất đẳng thức Cauchy--Schwarz cho 2 số không âm]
	Chứng minh:
	\begin{align*}
		\boxed{a + b\ge2\sqrt{ab},\ \forall a,b\in\mathbb{R},\,a,b\ge 0.}
	\end{align*}
	Đẳng thức xảy ra khi nào?
\end{baitoan}

\begin{proof}[1st proof]
	$a + b - 2\sqrt{ab} = (\sqrt{a} - \sqrt{b})^2\ge0\Rightarrow a + b\ge2\sqrt{ab}$, $\forall a,b\in\mathbb{R}$, $a,b\ge 0$. ``$=$'' $\Leftrightarrow\sqrt{a} = \sqrt{b}\Leftrightarrow a = b$.
\end{proof}

\begin{proof}[2nd proof]
	$(a + b)^2 - (2\sqrt{ab})^2 = a^2 + 2ab + b^2 - 4ab = a^2 - 2ab + b^2 = (a - b)^2\ge0\Rightarrow(a + b)^2\ge(2\sqrt{ab})^2\Rightarrow a + b\ge2\sqrt{ab}$ (vì $a,b\ge0$ nên $a + b\ge0$ \& $2\sqrt{ab}\ge0$). ``$=$'' $\Leftrightarrow a = b$.
\end{proof}

\begin{luuy}
	Ở 2nd proof, ta đã vận dụng tính chất cơ bản của căn bậc 2: \fbox{$0\le a\le b\Leftrightarrow\sqrt{a}\le\sqrt{b}$, $\forall a,b\in\mathbb{R}$}. Phiên bản chặt\emph{\texttt{/}}ngặt (strict) là: \fbox{$0\le a < b\Leftrightarrow\sqrt{a} < \sqrt{b}$, $\forall a,b\in\mathbb{R}$}. Ý nghĩa hình học của 2 tính chất này: Hình vuông nào có cạnh lớn hơn thì có diện tích lớn hơn \& ngược lại, hình vuông nào có diện tích lớn hơn thì có cạnh lớn hơn.
\end{luuy}

\begin{proof}[3rd proof]
	Đặt $x\coloneqq\sqrt{a}$, $y\coloneqq\sqrt{b}$, $x,y\in\mathbb{R}$, $x,y\ge0$. Có $a + b - 2\sqrt{ab} = a + b - 2\sqrt{a}\sqrt{b} = x^2 + y^2 - 2xy = (x - y)^2\ge0\Rightarrow a + b\ge2\sqrt{ab}$. ``$=$'' $\Leftrightarrow x = y\Leftrightarrow\sqrt{a} = \sqrt{b}\Leftrightarrow a = b$.
\end{proof}

\begin{baitoan}
	Với $m,n,p$ nào thì bất đẳng thức $ma + nb\ge p\sqrt{ab}$ luôn đúng: (a) $\forall a,b\in\mathbb{R}$, $a,b\ge0$. (b) $\forall a,b\in\mathbb{R}$. Đẳng thức xảy ra khi nào?
\end{baitoan}

\begin{luuy}
	Ở 3rd proof, ta đã sử dụng tính chất giao hoán của phép nhân \& phép khai phương: \fbox{$\sqrt{ab} = \sqrt{a}\sqrt{b}$, $\forall a,b\in\mathbb{R}$, $a,b\ge0$}.
\end{luuy}

\begin{baitoan}[Bất đẳng thức Cauchy--Schwarz cho 3 số không âm]
	Chứng minh:
	\begin{align*}
		\boxed{a + b + c\ge3\sqrt[3]{abc},\ \forall a,b,c\in\mathbb{R},\,a,b,c\ge 0.}
	\end{align*}
	Đẳng thức xảy ra khi nào?
\end{baitoan}

\begin{baitoan}
	Với $m,n,p,q$ nào thì bất đẳng thức $ma + nb + pc\ge q\sqrt[3]{abc}$ luôn đúng: (a) $\forall a,b,c\in\mathbb{R}$, $a,b,c\ge0$. (b) $\forall a,b,c\in\mathbb{R}$. Đẳng thức xảy ra khi nào?
\end{baitoan}

\begin{baitoan}[Bất đẳng thức Cauchy--Schwarz cho $n$ số không âm]
	Chứng minh:
	\begin{align*}
		\sum_{i=1}^n a_i\ge n\sqrt[n]{\prod_{i=1}^n a_i},\mbox{ i.e., } a_1 + a_2 + \cdots + a_n\ge\sqrt[n]{a_1a_2\cdots a_n},\ \forall n\in\mathbb{N}^\star,\ \forall a_i\in\mathbb{R},\,a_i\ge0,\,\forall i = 1,2,\ldots,n.
	\end{align*}
	Đẳng thức xảy ra khi nào?
\end{baitoan}

\begin{baitoan}
	Với bộ $(m,m_1,m_2,\ldots,m_n)$ nào thì bất đẳng thức:
	\begin{align*}
		\sum_{i=1}^n m_ia_i\ge m\sqrt[n]{\prod_{i=1}^n a_i},\mbox{ i.e., } m_1a_1 + m_2a_2 + \cdots + m_na_n\ge m\sqrt[n]{a_1a_2\cdots a_n},\ \forall n\in\mathbb{N}^\star,
	\end{align*}
	đúng với: (a) $\forall a_i\in\mathbb{R}$, $a_i\ge0$, $\forall i = 1,2,\ldots,n$. (b) $\forall a_i\in\mathbb{R}$, $\forall i = 1,2,\ldots,n$.
	Đẳng thức xảy ra khi nào?
\end{baitoan}


%------------------------------------------------------------------------------%

\section{Miscellaneous}

\begin{baitoan}[\cite{Son_Nghiep_Trung_Can2021}, Bổ đề 1.1, p. 5]
	Chứng minh: $4ab\le(a + b)^2\le2(a^2 + b^2)$, hay có thể viết dưới dạng $\frac{a^2 + b^2}{2}\ge\left(\frac{a + b}{2}\right)^2$, $ab\le\frac{(a + b)^2}{4}$, $\forall a,b\in\mathbb{R}$. Đẳng thức xảy ra khi nào?
\end{baitoan}

\begin{proof}[Hint]
	$(a + b)^2 - 4ab = (a - b)^2\ge 0$, $2(a^2 + b^2) - (a + b)^2 = (a - b)^2\ge 0$, $\forall a,b\in\mathbb{R}$. ``$=$'' $\Leftrightarrow a = b$.
\end{proof}

\begin{baitoan}[\cite{Son_Nghiep_Trung_Can2021}, Bổ đề 1.2, p. 5]
	Chứng minh: $3(ab + bc + ca)\le(a + b + c)^2\le3(a^2 + b^2 + c^2)$, hay có thể viết dưới dạng $ab + bc + ca\le\frac{1}{3}(a + b + c)^2$, $\forall a,b,c\in\mathbb{R}$. Đẳng thức xảy ra khi nào?
\end{baitoan}

\begin{proof}[Hint]
	$(a + b + c)^2 - 3(ab + bc + ca) = \frac{1}{2}\left[(a - b)^2 + (b - c)^2 + (c - a)^2\right]\ge 0$, $3(a^2 + b^2 + c^2) - (a + b + c)^2 = (a - b)^2 + (b - c)^2 + (c - a)^2\ge 0$, $\forall a,b,c\in\mathbb{R}$. ``$=$'' $\Leftrightarrow a = b = c$.
\end{proof}

\begin{baitoan}[\cite{Son_Nghiep_Trung_Can2021}, Bổ đề 1.3, p. 6]
	Chứng minh: $\frac{1}{a} + \frac{1}{b}\ge\frac{4}{a + b}$, hay có thể viết dưới dạng $\frac{1}{a + b}\le\frac{1}{4}\left(\frac{1}{a} + \frac{1}{b}\right)$, $\forall a,b > 0$. Đẳng thức xảy ra khi nào?
\end{baitoan}

\begin{proof}[Hint]
	$\frac{1}{a} + \frac{1}{b} - \frac{4}{a + b} = \frac{(a - b)^2}{ab(a + b)}\ge 0$, $\forall a,b > 0$. ``$=$'' $\Leftrightarrow a = b > 0$.
\end{proof}

\begin{baitoan}[\cite{Son_Nghiep_Trung_Can2021}, Bổ đề 1.4, p. 6]
	Chứng minh: $\frac{1}{a} + \frac{1}{b} + \frac{1}{c}\ge\frac{9}{a + b + c}$, hay có thể viết dưới dạng $\frac{1}{a + b + c}\le\frac{1}{9}\left(\frac{1}{a} + \frac{1}{b} + \frac{1}{c}\right)$, $\forall a,b,c > 0$. Đẳng thức xảy ra khi nào?
\end{baitoan}

\begin{baitoan}[\cite{Son_Nghiep_Trung_Can2021}, Mở rộng Bổ đề 1.3--1.4, p. 6 cho $n$ số]
	Chứng minh:
	\begin{align*}
		\frac{1}{a_1} + \ldots + \frac{1}{a_n}\ge\frac{n^2}{a_1 + \cdots + a_n},\mbox{ i.e., }\frac{1}{a_1 + \cdots + a_n}\le\frac{1}{n^2}\left(\frac{1}{a_1} + \cdots + \frac{1}{a_n}\right),\ \forall a_i > 0,\,\forall i = 1,\ldots,n,
	\end{align*}
	hay có thể được viết gọn lại như sau:
	\begin{align*}
		\sum_{i=1}^{n} \frac{1}{a_i}\ge\frac{n^2}{\sum_{i=1}^n a_i},\mbox{ i.e., }\frac{1}{\sum_{i=1}^n a_i}\le\frac{1}{n^2}\sum_{i=1}^n \frac{1}{a_i},\ \forall a_i > 0,\,\forall i = 1,\ldots,n.
	\end{align*}
	Đẳng thức xảy ra khi nào?
\end{baitoan}

\begin{baitoan}[\cite{Son_Nghiep_Trung_Can2021}, Bổ đề 1.5, p. 7]
	Chứng minh: $\sqrt{a + b}\le\sqrt{a} + \sqrt{b}\le\sqrt{2(a + b)}$, $\forall a,b\ge 0$. Đẳng thức xảy ra khi nào?
\end{baitoan}

\begin{baitoan}[\cite{Son_Nghiep_Trung_Can2021}, Mở rộng Bổ đề 1.5, p. 7]
	Chứng minh: $\sqrt{a + b + c}\le\sqrt{a} + \sqrt{b} + \sqrt{c}\le\sqrt{3(a + b + c)}$, $\forall a,b,c\ge 0$. Đẳng thức xảy ra khi nào?
\end{baitoan}

\begin{baitoan}[\cite{Son_Nghiep_Trung_Can2021}, Mở rộng Bổ đề 1.5, p. 7 cho $n$ số]
	Chứng minh: $\sqrt{a_1 + \cdots + a_n}\le\sqrt{a_1} + \cdots + \sqrt{a_n}\le\sqrt{n(a_1 + \cdots + a_n)}$, $\forall a_i\ge 0$, $\forall i = 1,\ldots,n$, hay có thể được viết gọn lại như sau:
	\begin{align*}
		\sqrt{\sum_{i=1}^n a_i}\le\sum_{i=1}^n \sqrt{a_i}\le\sqrt{n\sum_{i=1}^n a_i},\ \forall a_i\ge 0,\,\forall i = 1,\ldots,n.
	\end{align*}
	Đẳng thức xảy ra khi nào?
\end{baitoan}

\begin{baitoan}[\cite{Son_Nghiep_Trung_Can2021}, Bổ đề 1.6, p. 7]
	Chứng minh: $a^3 + b^3\ge ab(a + b)$, $\forall a,b\in\mathbb{R}$, $a + b\ge 0$. Đẳng thức xảy ra khi nào?
\end{baitoan}

\begin{proof}[Hint]
	$a^3 + b^3 - ab(a + b) = (a + b)(a - b)^2\ge 0$, $\forall a,b\in\mathbb{R}$, $a + b\ge 0$. ``$=$'' $\Leftrightarrow a = \pm b$.
\end{proof}

\begin{baitoan}[\cite{Son_Nghiep_Trung_Can2021}, Mở rộng Bổ đề 1.6, p. 7]
	Chứng minh: $a^4 + b^4\ge ab(a^2 + b^2)$, $\forall a,b\in\mathbb{R}$. Đẳng thức xảy ra khi nào?
\end{baitoan}





%------------------------------------------------------------------------------%

\printbibliography[heading=bibintoc]
	
\end{document}