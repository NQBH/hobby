\documentclass{article}
\usepackage[backend=biber,natbib=true,style=authoryear]{biblatex}
\addbibresource{/home/hong/1_NQBH/reference/bib.bib}
\usepackage[utf8]{vietnam}
\usepackage{tocloft}
\renewcommand{\cftsecleader}{\cftdotfill{\cftdotsep}}
\usepackage[colorlinks=true,linkcolor=blue,urlcolor=red,citecolor=magenta]{hyperref}
\usepackage{amsmath,amssymb,amsthm,mathtools,float,graphicx}
\allowdisplaybreaks
\numberwithin{equation}{section}
\newtheorem{assumption}{Assumption}[section]
\newtheorem{lemma}{Lemma}[section]
\newtheorem{corollary}{Corollary}[section]
\newtheorem{definition}{Định nghĩa}[section]
\newtheorem{proposition}{Proposition}[section]
\newtheorem{theorem}{Định lý}[section]
\newtheorem{notation}{Notation}[section]
\newtheorem{remark}{Lưu ý}[section]
\newtheorem{example}{Ví dụ}[section]
\newtheorem{question}{Question}[section]
\newtheorem{problem}{Bài toán}[section]
\newtheorem{conjecture}{Conjecture}[section]
\usepackage[left=0.5in,right=0.5in,top=1.5cm,bottom=1.5cm]{geometry}
\usepackage{fancyhdr}
\pagestyle{fancy}
\fancyhf{}
\lhead{\small \textsc{Sect.} ~\thesection}
\rhead{\small \nouppercase{\leftmark}}
\renewcommand{\sectionmark}[1]{\markboth{#1}{}}
\cfoot{\thepage}
\def\labelitemii{$\circ$}

\title{Elementary Mathematics\texttt{/}Principles}
\author{Nguyễn Quản Bá Hồng}
\date{\today}

\begin{document}
\maketitle
\begin{abstract}
	Một vài nguyên tắc cá nhân\texttt{/}personal principles\footnote{\textbf{principle} [n] \textbf{1.} [countable, usually plural, uncountable] a moral rule or a strong belief that influences your actions; \textbf{2.} [countable] a law, a rule or a theory that something is based on; \textbf{3.} [countable] a belief that is accepted as a reason for acting or thinking in a particular way; \textbf{4.} [countable, uncountable] a general or scientific law that explains how something works or why something happens.} trong việc dạy \textit{\&} học Toán Sơ Cấp.
\end{abstract}
\tableofcontents

%------------------------------------------------------------------------------%

\section{Notation}

\begin{enumerate}
	\item abbr.,: abbreviation, viết tắt.
	\item e.g. (abbr., of \textit{exempli gratia}): ``ví dụ'', ``chẳng hạn'', ``for example'', ``for instance''.
	\item i.e. (abbr., of \textit{id est}): ``tức là'', ``nghĩa là'', ``that is'', ``that means'', ``it means''.
	\item w.l.o.g., abbr. of \footnote{abbr. is the abbreviation of abbreviation itself, i.e., abbreviation (abbr., abbr.).} ``without loss of generality'', ``không mất tính tổng quát''.\footnote{Cụm này thường được dùng trong các chứng minh có \textit{chia trường hợp} (hay còn gọi là \textit{kỹ thuật chia để trị}), \textit{\&} điều quan trọng là các trường hợp được xét phải ``bình đẳng''\texttt{/}``đối xứng'' với nhau theo một nghĩa nào đó, thì mới được xử dụng kỹ thuật chia trường hợp, cũng như cụm từ này. Nếu sử dụng cụm từ ``w.l.o.g.'' cho các trường hợp không bình đẳng với nhau thì lời giải sẽ thiếu trường hợp \textit{\&} sai logic ngay từ thời điểm cụm ``w.l.o.g.'' được viết ra.}
	\item Cá nhân mình dùng dấu chấm để ngăn cách phần nguyên \textit{\&} phần thập phân của 1 số thực\texttt{/}phức (nói chung là không nguyên) thay vì dấu ``$,$'' như trong \cite{Thai_Anh_Dat_Ha_Loan_Nam_Quang_Toan_6_tap_1, Thai_Anh_Dat_Ha_Loan_Nam_Quang_Toan_6_tap_2}. Ký hiệu dấu $.$ được sử dụng rộng rãi 1 cách thống nhất trong nhiều ngành Khoa học.
\end{enumerate}

%------------------------------------------------------------------------------%

\section{Moral Principles}

\begin{enumerate}
	\item Học sinh nên\texttt{/}phải dừng ngay người giảng, hoặc ít nhất khi người giảng nói xong câu, nếu phát hiện bất cứ sai xót trong tính toán hoặc nghiêm trọng hơn là lỗi logic (logic là yếu tố quan trọng nhất của Khoa học cơ bản nói chung và Toán học nói riêng).
	\item Cho phép sử dụng sách giải\texttt{/}sách tham khảo. Nhưng không được lạm dụng. Chú ý sử dụng để tham khảo cách trình bày, kiểm tra lại đáp số của mình có đúng hay không. Phải đầu tư suy nghĩ đủ lâu trước khi xem lời giải.
	\item Chý trọng tâm lý học sinh.
	\item Học sinh đừng\texttt{/}không nên ngại hỏi câu hỏi ngu ngốc\texttt{/}ngớ ngẩn.
	\item Đặc biệt chú ý sức khỏe, cả thể chất lẫn tinh thần, đặc biệt là phòng chống những bệnh tâm lý.
\end{enumerate}

%------------------------------------------------------------------------------%

\section{\href{https://terrytao.wordpress.com/career-advice/}{Terence Tao\texttt{/}Career Advice}}
\begin{quotation}
	\textit{``Advice is what we ask for when we already know the answer but wish we didn't.''} -- \href{http://en.wikipedia.org/wiki/Erica_Jong}{Erica Jong}
\end{quotation}

\subsection{Terence Tao\texttt{/}Career Advice\texttt{/}Primary School Level}

\subsubsection{\href{https://terrytao.wordpress.com/career-advice/advice-on-gifted-education/}{Terence Tao\texttt{/}career advice{/}primary school level\texttt{/}advice on gifted education}}
\begin{quotation}
	\textit{``If you can give your son or daughter only 1 gift, let it be enthusiasm\footnote{\textbf{enthusiasm} [n] [uncountable, countable] a feeling of excitement about or interest in something, or of wanting to be involved in something.}.}'' -- \href{http://en.wikipedia.org/wiki/Bruce_Fairchild_Barton}{Bruce Barton}\\
	
	\textit{``Instead of buying your children all the things you never had, you should teach them all the things you were never taught. Material wears out but knowledge stays.''} -- Bruce Lee
\end{quotation}
``Education\footnote{\textbf{education} [n] \textbf{1.} [uncountable, singular] a process of teaching, training \& learning, especially in schools or colleges, to improve knowledge \& develop skills; \textbf{2.} [uncountable] a particular kind of teaching or training; \textbf{3.} (\textbf{Education}) [uncountable] the institutions or people involved in teaching \& training; \textbf{4.} (usually \textbf{Education}) [uncountable] the subject of study that deals with how to teach.} is a complex, multifaceted\footnote{\textbf{multifaceted} [a] (\textit{formal}) having many different aspects to be considered.}, \& painstaking\footnote{\textbf{painstaking} [a] [usually before noun] done with a lot of care, effort \& attention to detail, \textsc{synonym}: \textbf{thorough}.} process, \& being gifted\footnote{\textbf{gifted} [a] [usually before noun] having a lot of natural ability or intelligence.} does not make this less so. I would caution\footnote{\textbf{caution} [n] [uncountable] \textbf{1.} care that you take in order to avoid mistakes or danger; \textbf{2.} a warning or a piece of advice about a possible danger or risk.} against any single ``silver bullet''\footnote{\textbf{silver bullet} [n] (also \textbf{magic bullet}) [usually singular] a fast \& effective solution to a serious problem.} to educating a gifted child, whether it be a special school, private tutoring\footnote{\textbf{tutor} [n] \textbf{1.} a private teacher, especially one who teaches an individual student or a very small group; \textbf{2.} (\textit{especially British English}) a teacher whose job is to pay special attention to the studies or health, etc. of a student or a group of students; \textbf{3.} (\textit{British English}) a teacher, especially one who teaches adult or who has a special role in a school or college; \textbf{4.} (\textit{North American English}) an assistant lecturer in a college; \textbf{5.} a book of instruction in a particular subject, especially music; [v] \textbf{1.} [transitive] \textbf{tutor somebody (in something)} to be a tutor to an individual student or a small group; to teach somebody, especially privately; \textbf{2.} [intransitive] to work as a tutor.}, home schooling\footnote{\textbf{homeschooling} [n] [uncountable] the practice of educating children at home, not in schools.}, grade acceleration\footnote{\textbf{acceleration} [n] \textbf{1.} [uncountable, singular] \textbf{acceleration (in something)} an increase in how fast something happens; \textbf{2.} [uncountable] the rate at which a vehicle increases speed; \textbf{3.} [uncountable] (\textit{physics}) the rate at which the velocity ($=$ speed in a particular direction) of an object changes.}, or anything else; these are all opinions with advantages \& disadvantages, \& need to be weighed against the various requirements \& preferences (both academic \& non-academic) of the child, the parents, \& the school. Since this varies so much from child to child, I cannot give any \textit{specific}\footnote{\textbf{specific} [a] \textbf{1.} [usually before noun] connected with 1 particular person, thing or group only, \textsc{synonym}: \textbf{particular}; \textbf{2.} \textbf{specific to somebody\texttt{/}something} existing only in 1 place; limited to 1 person, thing or group, \textsc{synonym}: \textbf{peculiar}; \textbf{3.} detailed \& exact, \textsc{synonym}: \textbf{precise}.} advice on a given child's situation. [In particular, due to many existing time commitments\footnote{\textbf{commitment} [n] \textbf{1.} [singular, uncountable] a strong belief in a cause or activity \& a promise to support it; \textbf{2.} [countable, uncountable] a promise to do something or to behave in a particular way; \textbf{3.} [uncountable] the willingness to work hard \& give your energy \& time to a job or an activity; \textbf{4.} [countable] (used in compounds) a thing that you have promised or agreed to do, or that you have to do; \textbf{5.} [countable, uncountable] agreeing to use money, time or people in order to achieve something.} \& high volume of requests, I am unable to personally respond to any queries\footnote{\textbf{query} [n] (plural \textbf{queries}) \textbf{1.} a question, especially one asking for information or expressing a doubt about something; \textbf{2.} a search for information that is entered into an Internet search engine; [v] to express doubt about whether something is correct or not.} regarding gifted education.]

I can give a few \textit{general} pieces of advice, though. 1stly, one should not focus overly\footnote{\textbf{overly} [adv] (before an adjective) too; very, \textsc{synonym}: \textbf{excessively}.} much on a specific artificial\footnote{\textbf{artificial} [a] \textbf{1.} made or produced by humans to copy something natural, rather than occurring naturally; \textbf{2.} created by people; not happening naturally.} benchmark\footnote{\textbf{benchmark} [n] something that provides a standard against which other things can be measured or compared; [v] \textbf{benchmark something (against something)} to judge the standard of something in relation to other similar things.}, such as obtaining degree X from \href{https://terrytao.wordpress.com/career-advice/don%E2%80%99t-base-career-decisions-on-glamour-or-fame/}{prestigious\footnote{\textbf{prestigious} [a] [usually before noun] (not usually used about people) respected \& admired; considered to have great importance, quality or value.} institution\footnote{\textbf{institution} [n] \textbf{1.} [countable] an important, often large, organization that has a particular purpose, e.g., a university or bank; \textbf{2.} [countable] a custom or system within society or among a particular group of people; \textbf{3.} [uncountable] \textbf{institution of something} the act of starting or introducing something such as a system or a law; \textbf{4.} [countable] a building where people with special needs are taken care of, e.g. because they are old or mentally ill; the organization that takes care of them.}} Y in only Z years, or \href{https://terrytao.wordpress.com/career-advice/theres-more-to-mathematics-than-grades-and-exams-and-methods/}{on scoring} A on test B at age C. In the long term\footnote{\textbf{long-term} [a] [usually before noun] \textbf{1.} that will last or have an effect over a long period of time into the future; \textbf{2.} that has lasted a long time \& is not likely to change or be solved quickly.}, these feats\footnote{\textbf{feat} [n] (\textit{approving}) an action or a piece of work that needs skill, strength or courage.} will not be the most important or decisive\footnote{\textbf{decisive} [a] \textbf{1.} making the result of something final or certain; \textbf{2.} having or showing the ability to make clear decisions quickly.} moments in the child's career; also, any short-term\footnote{\textbf{short-term} [a] [usually before noun] lasting a short time; designed only for a short period of time in the future. \textbf{Short-term memory} is the ability to remember things that happened a short time ago.} advantage one might gain in working excessively\footnote{\textbf{excessive} [a] \textbf{1.} greater than what seems reasonable or appropriate; \textbf{2.} much greater than what is usual.} towards such benchmarks may be outweighed\footnote{\textbf{outweigh} [v] \textbf{outweigh something} to be greater or more important than something.} by the time \& energy that such a goal takes away from other aspects of a child's social, emotional, academic, physical, or intellectual\footnote{\textbf{intellectual} [a] [usually before noun] connected with or using a person's ability to think in a logical way \& understand things, \textsc{synonym}: \textbf{mental}; [n] a person who is well educated \& enjoys activities in which they have to think seriously about things.} development\footnote{\textbf{development} [n] \textbf{1.} [uncountable] the process of creating a new method, system, product or theory; \textbf{2.} [countable] a new or advanced method, system, product or theory; \textbf{3.} [uncountable] the process of making a country or area richer \& more successful; \textbf{4.} [uncountable] the way in which a child or other living creature grows before \& after birth; \textbf{5.} [uncountable] gradual growth or changes that make somebody\texttt{/}something more advanced, more skilled or stronger; \textbf{6.} [countable] a new event or stage that is likely to affect what happens in a continuing situation; \textbf{7.} [uncountable] \textbf{development of something} the fact of starting to have something such as an illness or a problem; \textbf{8.} [countable] a piece of land with new buildings on it; \textbf{9.} [uncountable] the process of using an area of land, especially to make a profit by building on it.}. Of course, one should still \href{https://terrytao.wordpress.com/career-advice/work-hard/}{work hard}, \& \href{https://terrytao.wordpress.com/career-advice/advice-on-mathematics-competitions/}{participate\footnote{\textbf{participate} [v] [intransitive] \textbf{participate (in something)} to take part in or become involved in an activity.} in competitions\footnote{\textbf{competition} [n] \textbf{1.} [uncountable] (used especially about the world of business) a situation in which somebody\texttt{/}something tries to be more successful than somebody\texttt{/}something else, or tries to get something rather than let somebody\texttt{/}something else get it; \textbf{2.} (\textbf{the competition}) [singular] a person or business that is trying to be more successful than others; goods or services that are intended to be more successful than others; \textbf{3.} [uncountable, countable] (\textit{ecology}) a situation in which animals, plants or other living things try to get resources, with the result that other animals, plants, etc. may not be able to get them; \textbf{4.} [countable] a contest to find out who is the best at something.}} if one wishes; but competitions \& academic achievements\footnote{\textbf{achievement} [n] \textbf{1.} [countable] a thing that somebody has done successfully, especially using their own effort \& skill; \textbf{2.} [uncountable] the fact or process of achieving something; \textbf{3.} [uncountable] a child's or student's progress in a course of learning, especially as measured by standard tests.} should not be viewed as ends in themselves, but rather a way to develop one's talents\footnote{\textbf{talent} [n] \textbf{1.} [countable, uncountable] a natural ability to do something well; \textbf{2.} [uncountable, countable] people or a person with a natural ability to do something well.}, experience, knowledge, \& enjoyment\footnote{\textbf{enjoyment} [n] [uncountable] \textbf{1.} \textbf{enjoyment of something} the fact of having \& using something; \textbf{2.} \textbf{enjoy (of something)} the pleasure that you get from something.} of the subject.

2ndly, I feel that it is important to \href{https://terrytao.wordpress.com/career-advice/enjoy-your-work/}{enjoy one's work}; this is what sustains\footnote{\textbf{sustain} [v] \textbf{1.} \textbf{sustain somebody\texttt{/}something} to provide enough of what somebody\texttt{/}something needs in order to live or exist; \textbf{2.} to make something continue for some time without becoming less, \textsc{synonym}: \textbf{maintain}; \textbf{3.} \textbf{sustain something} (\textit{formal}) to experience something bad, \textsc{synonym}: \textbf{suffer}; \textbf{4.} \textbf{sustain something} to provide evidence to support an opinion, a theory, etc., \textsc{synonym}: \textbf{uphold}; \textbf{5.} \textbf{sustain something} (\textit{law}) to decide that a claim, etc. is valid, \textsc{synonym}: \textbf{uphold}.} \& drives a person throughout\footnote{\textbf{throughout} [prep, adv] \textbf{1.} in or into every part of something; \textbf{2.} during the whole period of time of something.} the duration\footnote{\textbf{duration} [n] [uncountable] \textbf{duration (of something)} the length of time that something lasts or continues.} of his or her career\footnote{\textbf{career} [n] \textbf{1.} the period of time that you spend in your life working or doing a particular thing; \textbf{2.} the series of jobs that a person has in a particular area of work, usually involving more responsibility as time passes.}, \& holds burnout\footnote{\textbf{burnout} [n] [uncountable] the state of being extremely tired or ill, either physically or mentally, because of too much work or stress.} at bay\footnote{\textbf{bay} [n] \textbf{1.} [countable] a part of the sea, or of a large lake, partly surrounded by a wide curve of the land; \textbf{2.} [countable] a marked section of ground either inside or outside a building, e.g. for a vehicle to park in, for storing things, etc.; \textbf{3.} [countable] a curved area of room or building that sticks out from the rest of the building; \textbf{4.} [countable] a horse of a dark brown color; \textbf{5.} [countable] a deep noise, especially the noise made by dogs when hunting; \textbf{6.} (also \textbf{bay tree}) [countable] a small tree with dark green leaves with a sweet smell that are used in cooking; \textbf{7.} [uncountable] the leaves of the bay tree, used in cooking as a herb; \textbf{at bay} [idiom] when an animal that is being hunted is \textbf{at bay}, it must turn \& face the dogs \& hunters because it is impossible to escape from them; \textbf{hold\texttt{/}keep somebody\texttt{/}something at bay} [idiom] to prevent an enemy from coming close or a problem from having a bad effect, \textsc{synonym}: \textbf{ward off}; [v] \textbf{1.} [intransitive] (of a dog or wolf) to make a long deep sound, especially while hunting, \textsc{synonym}: \textbf{howl}; \textbf{2.} [intransitive] \textbf{bay (for something)} (usually used in the progressive tenses) to demand something in a loud \& angry way; [a] (of a horse) dark brown in color.}. It would be a tragedy\footnote{\textbf{tragedy} [n] (plural \textbf{tragedies}) [countable, uncountable] \textbf{1.} a very sad event or situation, especially one that involves death; \textbf{2.} a serious play with a sad ending, especially one in which the main character dies; plays of this type.} if a well-meaning\footnote{\textbf{well-meaning} [a] intending to do what is right \& helpful, but often not succeeding, \textsc{synonym}: well-intentioned.} parent, by pushing too hard (or too little) for the development of their child's gifts in a subject, ended up accidentally extinguishing\footnote{\textbf{extinguish} [v] (\textit{formal}) \textbf{1.} \textbf{extinguish something} to make a fire stop burning or a light stop shining, \textsc{synonym}: \textbf{put out}; \textbf{2.} \textbf{extinguish something} to destroy something.} the child's love for that subject. The pace\footnote{\textbf{pace} [n] \textbf{1.} [uncountable, singular] \textbf{pace (of something)} the speed at which something happens; \textbf{2.} [singular, uncountable] the speed at which somebody\texttt{/}something walks, runs or moves; \textbf{3.} [countable] an act of stepping once when walking or running; the distance traveled when doing this.} of the child's education should be driven more by the eagerness\footnote{\textbf{eagerness} [n] [uncountable, singular] \textbf{eagerness (to do something)} great interest \& excitement about something that is going to happen or about something that you want to do.} of the child than the eagerness of the parent.

3rdly, one should praise\footnote{\textbf{praise} [n] [uncountable] \textbf{1.} (\textit{also less frequent} \textbf{praises} [plural]) words that show that you approve of \& admire somebody\texttt{/}something; \textbf{2.} the expression of thanks to or respect for God; [v] \textbf{1.} to say that you approve of \& admire somebody\texttt{/}something, \textsc{synonym}: \textbf{compliment}; \textbf{2.} \textbf{praise somebody} to express your thanks to or your respect for God; \textbf{praise somebody\texttt{/}something to the skies} [idiom] to praise somebody\texttt{/}something a lot.} one's children for their efforts \& achievements (which they can control), \& not for their innate\footnote{\textbf{innate} [a] (of a quality or feeling) that you have when you are born.} talents (which they cannot). This \href{http://nymag.com/news/features/27840/}{article by Po Bronson} describes this point excellently. See also the Scientific American article ``\href{http://www.sciam.com/article.cfm?id=the-secret-to-raising-smart-kids}{The secret to raising smart kids}'' for a similar viewpoint.

Finally, one should \href{https://terrytao.wordpress.com/career-advice/be-flexible/}{be flexible\footnote{\textbf{flexible} [a] \textbf{1.} able to change to suit new conditions or situations, \textsc{opposite}: \textbf{inflexible}. In economics, \textbf{flexible} is used to describe prices, wages, exchange rates, etc. that are quick to change or react to change. \textsc{opposite}: \textbf{sticky}; \textbf{2.} able to bend easily.}} in one's goals. A child may be initially gifted in field X, but decides that field Y is more enjoyable or is a better fit. This may be a better choice, even if Y is ``\href{https://terrytao.wordpress.com/career-advice/don%E2%80%99t-base-career-decisions-on-glamour-or-fame/}{less prestigious}'' than X; sometimes it is better to work in a less well known field that one feels competent \& comfortable in, than in a ``hot'' but competitive field that one feels unsuitable for. (See also \href{http://en.wikipedia.org/wiki/Comparative_advantage}{Ricardo's law of comparative advantage}.)

My own education is discussed in the following articles. While I am very happy with the way things turned out for me, I would again caution that each child's situation, strengths, \& weaknesses are different, \& that my experience might not necessarily be the ideal template to follow for others.
\begin{itemize}
	\item ``\href{https://link.springer.com/article/10.1007/BF00312075}{Terence Tao}'', Ken Clements, Educational Studies in Mathematics, Aug 1984, Vol. 15, No. 3, 213--238
	\item ``\href{https://www.jstor.org/stable/3482231}{Parental involvement in Gifted Education}'', Billy Tao, Educational Studies in Mathematics, Aug 1986, Vol. 17, No. 3, 313--321
	\item ``\href{https://journals.sagepub.com/doi/10.1177/107621758600900402}{Radical Acceleration in Australia: Terence Tao}'', Miraca Gross, G\texttt{/}C/\texttt{/}T, Jul\texttt{/}Aug 1986
	\item ``\href{http://gcq.sagepub.com/cgi/content/abstract/50/4/307}{Insights from SMPY's greatest former child prodigies: Drs. Terence (``Terry'') Tao \& Lenhard (``Lenny'') Ng reflect on their talent development}'', Michelle Muratori, Julian Stanley, Lenhard Ng, Jack Ng, Miraca Gross, Terence Tao, Billy Tao, Gifted Child Quarterly, Fall 2006, Vol. 50, No. 4, 307--324
\end{itemize}
For professional advice on gifted education, I can recommend the \href{http://cty.jhu.edu/}{Center for Talented Youth}.'' -- Terence Tao

\subsubsection{\href{https://terrytao.wordpress.com/career-advice/don't-base-career-decisions-on-glamour-or-fame/}{Terence Tao\texttt{/}career advice\texttt{/}don't base career decisions on glamour or fame}}
\begin{quotation}
	``One who pursues fame\footnote{\textbf{fame} [n] [uncountable] the state of being known \& talked about by many people.} at the risk\footnote{\textbf{risk} [n] \textbf{1.} [countable, uncountable] the possibility of something bad happening at some time in the future; a situation that could be dangerous or have a bad result; \textbf{2.} [countable] \textbf{risk (to something\texttt{/}something)} a person or thing that is likely to cause problems or danger at some time in the future; \textbf{3.} [countable] \textbf{a good\texttt{/}bad\texttt{/}poor risk} a person or business that a bank or company is willing\texttt{/}unwilling to lend money to, sell insurance to, or do business with because the thank or company is unlikely\texttt{/}likely to lose money; \textbf{at risk (from\texttt{/}of something)} [idiom] in danger of something upleasant or harmful happening; \textbf{at the risk of (doing) something} [idiom] although there may be a particular bad result; \textbf{run a\texttt{/}the risk (of something\texttt{/}of doing something), run risks} [idiom] to be or put yourself in a situation in which something bad could happen to you; \textbf{take a risk, take risks} [idiom] to do something even though you know that something bad could happen as a result.} of losing one's self, is not a scholar\footnote{\textbf{scholar} [n] \textbf{1.} a person who knows a lot about a particular subject because they have studied it in detail. \textbf{Scholar} is usually used in connection with the humanities or social sciences, or in historical contexts. In modern contexts, to talk about natural sciences, use \textbf{scientist}.; \textbf{2.} a person who has a scholarship to study at a university or other institution.}.'' -- \href{http://en.wikipedia.org/wiki/Zhuangzi}{Zhuangzi}, ``[The Grandmaster]''
\end{quotation}
``Going into a field or department\footnote{\textbf{department} [n] (abbr. \textbf{Dept}) a section of a large organization such as government, business or university.} simply because it is glamorous\footnote{\textbf{glamorous} [a] (\textit{also informal} \textbf{glam}) especially attractive \& exciting, \& different from ordinary things or people, \textsc{opposite}: \textbf{unglamorous}.} is not a good idea, nor is focusing on the most famous problems (or mathematicians) within a field, solely\footnote{\textbf{solely} [adv] only; not involving somebody\texttt{/}something else.} because they are famous -- honestly\footnote{\textbf{honestly} [adv] \textbf{1.} in an honest way, \textsc{opposite}: \textbf{dishonestly}; \textbf{2.} used to emphasize that what you are saying is true, however surprising it may seem.}, there isn't that much fame or glamour\footnote{\textbf{glamour} [n] (\textit{North American English also} \textbf{glamor}) [uncountable, singular] \textbf{1.} the attractive \& exciting quality that makes a person, a job or a place seem special, often because of wealth or status; \textbf{2.} physical beauty that also suggests wealth or success.} in mathematics overall\footnote{\textbf{overall} [a] [only before noun] including all the things or people that are involved in a particular situation, \textsc{synonym}: \textbf{general}; [adv] \textbf{1.} including everything or everyone; in total; \textbf{2.} generally; when everything is considered.}, \& it is not worth chasing these things as your primary\footnote{\textbf{primary} [a] \textbf{1.} [usually before noun] main; most important; basic, \textsc{synonym}: \textbf{prime}; \textbf{2.} [usually before noun] developing or happening 1st; earliest. \textbf{Primary} is used especially in biology \& medicine to refer to the 1st stage of development or growth of something.; \textbf{3.} [only before noun] (\textit{especially British English}) connected with the education of children aged around 5--11; \textbf{4.} (\textit{chemistry}) (of an organic compound) having its functional group located on a carbon atom which is bonded to no more than one other carbon atom; containing a nitrogen atom bonded to 1 carbon atom.} goal. Anything glamorous is likely to be highly competitive, \& only those with the most solid\footnote{\textbf{solid} [a] [usually before noun] \textbf{1.} not in the form of a liquid or gas; \textbf{2.} hard or firm, with a surface that does not move when pressed; \textbf{3.} having no holes or empty spaces inside; \textbf{4.} having a strong basis; reliable; \textbf{5.} (\textit{specialist}) having a shape with length, width \& height; \textbf{6.} [only before noun] made completely of the material mentioned; \textbf{7.} (of a line or color) without spaces; [n] \textbf{1.} [countable] a substance that is not a liquid or a gas; \textbf{2.} [countable] (\textit{geometry}) a shape that has length, width \& height; \textbf{3.} (\textbf{solids}) [plural] food that is not liquid.} of backgrounds\footnote{\textbf{background} [n] } (in particular, lots of experience with less glamorous aspects of the field) are likely to get anywhere.

\href{https://terrytao.wordpress.com/career-advice/dont-prematurely-obsess-on-a-single-big-problem-or-big-theory/}{A famous unsolved problem is almost never solved \textit{ab nihilo}}\footnote{\textbf{ab nihilo} [Latin] from nothing.}. One has to 1st spend much time \& \href{https://terrytao.wordpress.com/career-advice/work-hard/}{effort} working on simpler (\& much less famous) model problems, acquiring \href{https://terrytao.wordpress.com/career-advice/learn-the-power-of-other-mathematicians-tools/}{techniques}, \href{https://terrytao.wordpress.com/career-advice/there%E2%80%99s-more-to-mathematics-than-rigour-and-proofs/}{intuition}\footnote{\textbf{intuition} [n] \textbf{1.} [uncountable] the ability to know something by using your feelings rather than considering the facts; \textbf{2.} [countable] an idea or a strong feeling that something is true although it is not proved.}, partial\footnote{\textbf{partial} [a] \textbf{1.} not complete or whole; \textbf{2.} [not usually before noun] showing or feeling too much support for 1 person, team, idea, etc., in a way that is unfair, \textsc{synonym}: \textbf{biased}, \textsc{opposite}: \textbf{impartial}.} results, context\footnote{\textbf{context} [n] [countable, uncountable] \textbf{1.} the situation or set of circumstances in which something happens \& that helps you to understand it; \textbf{2.} the words that come just before \& after a word, phrase or statement \& help you to understand its meaning.}, \& literature, thus enabling fruitful\footnote{\textbf{fruitful} [a] producing many useful results, \textsc{synonym}: \textbf{productive}.} approaches to the problem \& ruling out\footnote{\textbf{rule somebody\texttt{/}something out (as something)} [phrasal verb] \textbf{1.} to state that something is not possible or that somebody\texttt{/}something is not suitable, \textsc{synonym}: \textbf{exclude}; \textbf{2.} to prevent somebody from doing something; to prevent something from happening.} \href{https://terrytao.wordpress.com/career-advice/learn-the-limitations-of-your-tools/}{fruitless ones}, before having any real chance of solving any really big problem in the area. (Occasionally\footnote{\textbf{occasionally} [adv] sometimes but not often}, 1 of these problems falls relatively easily, simply because the right group of people with the right set of tools hadn't had a chance to look at the problem before, but this is usually not the case for the very intensively\footnote{\textbf{intensively} [adv] \textbf{1.} in a way that involves a lot of work or activity in a short time; \textbf{2.} with a lot of attention, effort or care; \textbf{3.} using methods of farming that produce as much food as possible using as little land or as little money as possible.} studied problems -- particularly those which already have a substantial\footnote{\textbf{substantial} [a] large in amount, value or importance, \textsc{synonym}: \textbf{considerable}.} body of ``no go'' theorems \& counterexamples\footnote{\textbf{counterexample} [n] \textbf{counterexample (to something)} an example that provides evidence against an idea or theory.} which rule out entire strategies of attack.)

For similar reasons, one should never make prizes or recognition\footnote{\textbf{recognition} [n] \textbf{1.} [uncountable] the act of remembering who somebody is when you see them, or of identifying what something is; \textbf{2.} [uncountable, singular] the act of accepting that something exists, is true or is official, \textsc{synonym}: \textbf{acknowledgment}; \textbf{3.} [uncountable] public praise \& reward for somebody's work, achievements or actions; \textbf{to change, alter, etc. beyond\texttt{/}out of (all) recognition} [idiom] to change so much that is almost impossible to recognize.} a primary reason for pursuing mathematics; it is a better strategy\footnote{\textbf{strategy} [n] (plural \textbf{strategies}) \textbf{1.} [countable] a plan that is intended to achieve a particular purpose. In ecology, \textbf{strategies} are ways that have evolved ($=$ developed) in plants \& animals that enable them to survive \& be successful in their environment.; \textbf{2.} [uncountable] the process of planning something or putting a plan into operation in a skillful way; \textbf{3.} [uncountable, countable] the skill of planning the movements of armies in a battle or war; an example of doing this.} in the long-term to just produce \href{http://arxiv.org/abs/math.HO/0702396}{good mathematics} \& contribute\footnote{\textbf{contribute} [v] \textbf{1.} [intransitive] \textbf{contribute (to something)} to be 1 of the causes of something; \textbf{2.} [intransitive, transitive] to help to improve or achieve something, especially by adding new ideas; \textbf{3.} [transitive, intransitive] to give something, especially money or goods, to help somebody\texttt{/}something; \textbf{4.} [transitive, intransitive] to write somethign for a newspaper, magazine, website, or a radio or television programme; to speak during a meeting or conversation, especially to give your opinion.} to your field, \& the prizes \& recognition will take care of themselves (\& be well-earned\footnote{\textbf{well earned} [a] well deserved.} when they eventually\footnote{\textbf{eventually} [adv] at the end of a period of time or a series of events. Use \textbf{finally} for the last in a list of things.} do appear).

On the other hand, it can be worth researching \textit{why} a problem or mathematician is famous, or \textit{how} an institution or department earnt its prestige\footnote{\textbf{prestige} [n] [uncountable] respect \& admiration that somebody\texttt{/}something receives, because people consider them\texttt{/}it to have great importance, quality or value.}; such specific information can help you decide whether this problem, mathematician, or department would be of interest to you. See also ``\href{https://terrytao.wordpress.com/career-advice/which-universities-should-one-apply-to/}{which universities should I apply to?}'''' -- Terence Tao

%------------------------------------------------------------------------------%

\section{Miscellaneous}

\begin{enumerate}
	\item Nên\texttt{/}cố gắng tập thể dục mỗi ngày để đầu óc minh mẫn. Không nên làm việc quá sức mà bỏ tập thể dục.
	\item Mình không thích, đúng hơn là cực ghét, việc dịch \& viết phiên âm tiếng Việt của các nhà Khoa học nói chung \& các nhà Toán học nói riêng trong Bộ Sách Giáo Khoa. Mình nghĩ nên viết tên đúng gốc hoặc viết phiên âm tiếng Anh để thể hiện sự tôn trọng \& nhất quán.
	\item Kỹ năng tự học\texttt{/}self-study skill là vua của mọi loại kỹ năng.
\end{enumerate}

%------------------------------------------------------------------------------%

\printbibliography[heading=bibintoc]
	
\end{document}