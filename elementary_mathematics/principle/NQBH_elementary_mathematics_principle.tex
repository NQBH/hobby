\documentclass{article}
\usepackage[backend=biber,natbib=true,style=authoryear]{biblatex}
\addbibresource{/home/hong/1_NQBH/reference/bib.bib}
\usepackage[utf8]{vietnam}
\usepackage{tocloft}
\renewcommand{\cftsecleader}{\cftdotfill{\cftdotsep}}
\usepackage[colorlinks=true,linkcolor=blue,urlcolor=red,citecolor=magenta]{hyperref}
\usepackage{amsmath,amssymb,amsthm,mathtools,float,graphicx}
\allowdisplaybreaks
\numberwithin{equation}{section}
\newtheorem{assumption}{Assumption}[section]
\newtheorem{lemma}{Lemma}[section]
\newtheorem{corollary}{Corollary}[section]
\newtheorem{definition}{Định nghĩa}[section]
\newtheorem{proposition}{Proposition}[section]
\newtheorem{theorem}{Định lý}[section]
\newtheorem{notation}{Notation}[section]
\newtheorem{remark}{Lưu ý}[section]
\newtheorem{example}{Ví dụ}[section]
\newtheorem{question}{Question}[section]
\newtheorem{problem}{Bài toán}[section]
\newtheorem{conjecture}{Conjecture}[section]
\usepackage[left=0.5in,right=0.5in,top=1.5cm,bottom=1.5cm]{geometry}
\usepackage{fancyhdr}
\pagestyle{fancy}
\fancyhf{}
\lhead{\small \textsc{Sect.} ~\thesection}
\rhead{\small \nouppercase{\leftmark}}
\renewcommand{\sectionmark}[1]{\markboth{#1}{}}
\cfoot{\thepage}
\def\labelitemii{$\circ$}

\title{Elementary Mathematics\texttt{/}Principles}
\author{Nguyễn Quản Bá Hồng}
\date{\today}

\begin{document}
\maketitle
\begin{abstract}
	Một vài nguyên tắc cá nhân\texttt{/}personal principles\footnote{\textbf{principle} [n] \textbf{1.} [countable, usually plural, uncountable] a moral rule or a strong belief that influences your actions; \textbf{2.} [countable] a law, a rule or a theory that something is based on; \textbf{3.} [countable] a belief that is accepted as a reason for acting or thinking in a particular way; \textbf{4.} [countable, uncountable] a general or scientific law that explains how something works or why something happens.} trong việc dạy \textit{\&} học Toán Sơ Cấp.
\end{abstract}
\tableofcontents

%------------------------------------------------------------------------------%

\section{Notation}

\begin{enumerate}
	\item abbr.,: abbreviation, viết tắt.
	\item e.g. (abbr., of \textit{exempli gratia}): ``ví dụ'', ``chẳng hạn'', ``for example'', ``for instance''.
	\item i.e. (abbr., of \textit{id est}): ``tức là'', ``nghĩa là'', ``that is'', ``that means'', ``it means''.
	\item w.l.o.g., abbr. of \footnote{abbr. is the abbreviation of abbreviation itself, i.e., abbreviation (abbr., abbr.).} ``without loss of generality'', ``không mất tính tổng quát''.\footnote{Cụm này thường được dùng trong các chứng minh có \textit{chia trường hợp} (hay còn gọi là \textit{kỹ thuật chia để trị}), \textit{\&} điều quan trọng là các trường hợp được xét phải ``bình đẳng''\texttt{/}``đối xứng'' với nhau theo một nghĩa nào đó, thì mới được xử dụng kỹ thuật chia trường hợp, cũng như cụm từ này. Nếu sử dụng cụm từ ``w.l.o.g.'' cho các trường hợp không bình đẳng với nhau thì lời giải sẽ thiếu trường hợp \textit{\&} sai logic ngay từ thời điểm cụm ``w.l.o.g.'' được viết ra.}
	\item Cá nhân tôi dùng dấu chấm để ngăn cách phần nguyên \textit{\&} phần thập phân của 1 số thực\texttt{/}phức (nói chung là không nguyên) thay vì dấu ``$,$'' như trong \cite{Thai_Anh_Dat_Ha_Loan_Nam_Quang_Toan_6_tap_1, Thai_Anh_Dat_Ha_Loan_Nam_Quang_Toan_6_tap_2}. Ký hiệu dấu $.$ được sử dụng rộng rãi 1 cách thống nhất trong nhiều ngành Khoa học.
\end{enumerate}

%------------------------------------------------------------------------------%

\section{Moral Principles}

\begin{enumerate}
	\item Học sinh nên\texttt{/}phải dừng ngay người giảng, hoặc ít nhất khi người giảng nói xong câu, nếu phát hiện bất cứ sai xót trong tính toán hoặc nghiêm trọng hơn là lỗi logic (logic là yếu tố quan trọng nhất của Khoa học cơ bản nói chung và Toán học nói riêng).
	\item Cho phép sử dụng sách giải\texttt{/}sách tham khảo. Nhưng không được lạm dụng. Chú ý sử dụng để tham khảo cách trình bày, kiểm tra lại đáp số của mình có đúng hay không. Phải đầu tư suy nghĩ đủ lâu trước khi xem lời giải.
	\item Chý trọng tâm lý học sinh.
	\item Học sinh đừng\texttt{/}không nên ngại hỏi câu hỏi ngu ngốc\texttt{/}ngớ ngẩn.
	\item Đặc biệt chú ý sức khỏe, cả thể chất lẫn tinh thần, đặc biệt là phòng chống những bệnh tâm lý.
\end{enumerate}

\section{Miscellaneous}

\begin{enumerate}
	\item Nên\texttt{/}cố gắng tập thể dục mỗi ngày để đầu óc minh mẫn. Không nên làm việc quá sức mà bỏ tập thể dục.
	\item Tôi không thích, đúng hơn là cực ghét, việc dịch \& viết phiên âm tiếng Việt của các nhà Khoa học nói chung \& các nhà Toán học nói riêng trong Bộ Sách Giáo Khoa. Tôi nghĩ nên viết tên đúng gốc hoặc viết phiên âm tiếng Anh để thể hiện sự tôn trọng \& nhất quán.
	\item Kỹ năng tự học\texttt{/}self-study skill là vua của mọi loại kỹ năng.
	\item \texttt{To do list: insert Terence Tao's advice}
\end{enumerate}

%------------------------------------------------------------------------------%

\printbibliography[heading=bibintoc]
	
\end{document}