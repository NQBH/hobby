\documentclass[oneside]{book}
\usepackage[backend=biber,natbib=true,style=authoryear]{biblatex}
\addbibresource{/home/hong/1_NQBH/reference/bib.bib}
\usepackage[utf8]{vietnam}
\usepackage{tocloft}
\renewcommand{\cftsecleader}{\cftdotfill{\cftdotsep}}
\usepackage[colorlinks=true,linkcolor=blue,urlcolor=red,citecolor=magenta]{hyperref}
\usepackage{amsmath,amssymb,amsthm,mathtools,float,graphicx,algpseudocode,algorithm,tcolorbox,tikz,tkz-tab,diagbox}
\DeclareMathOperator{\arccot}{arccot}
\usepackage[inline]{enumitem}
\allowdisplaybreaks
\numberwithin{equation}{section}
\newtheorem{assumption}{Assumption}[section]
\newtheorem{nhanxet}{Nhận xét}[section]
\newtheorem{conjecture}{Conjecture}[section]
\newtheorem{corollary}{Corollary}[section]
\newtheorem{hequa}{Hệ quả}[section]
\newtheorem{definition}{Definition}[section]
\newtheorem{dinhnghia}{Định nghĩa}[section]
\newtheorem{example}{Example}[section]
\newtheorem{vidu}{Ví dụ}[section]
\newtheorem{lemma}{Lemma}[section]
\newtheorem{notation}{Notation}[section]
\newtheorem{principle}{Principle}[section]
\newtheorem{problem}{Problem}[section]
\newtheorem{baitoan}{Bài toán}[section]
\newtheorem{proposition}{Proposition}[section]
\newtheorem{menhde}{Mệnh đề}[section]
\newtheorem{question}{Question}[section]
\newtheorem{cauhoi}{Câu hỏi}[section]
\newtheorem{remark}{Remark}[section]
\newtheorem{luuy}{Lưu ý}[section]
\newtheorem{theorem}{Theorem}[section]
\newtheorem{dinhly}{Định lý}[section]
\usepackage[left=0.5in,right=0.5in,top=1.5cm,bottom=1.5cm]{geometry}
\usepackage{fancyhdr}
\pagestyle{fancy}
\fancyhf{}
\lhead{\small \textsc{Sect.} ~\thesection}
\rhead{\small \nouppercase{\leftmark}}
\renewcommand{\sectionmark}[1]{\markboth{#1}{}}
\cfoot{\thepage}
\def\labelitemii{$\circ$}

\title{Some Topics in Elementary Physics\texttt{/}Grade 10}
\author{Nguyễn Quản Bá Hồng\footnote{Independent Researcher, Ben Tre City, Vietnam\\e-mail: \texttt{nguyenquanbahong@gmail.com}; website: \url{https://nqbh.github.io}.}}
\date{\today}

\begin{document}
\frontmatter
\maketitle
\setcounter{secnumdepth}{4}
\setcounter{tocdepth}{3}
\tableofcontents
\newpage

%------------------------------------------------------------------------------%

\mainmatter

\chapter*{Preface}

Tóm tắt kiến thức Vật lý lớp 10 theo chương trình giáo dục của Việt Nam \& một số chủ đề nâng cao.

``Vật lý được biết đến như là 1 trong những ngành của Khoa học tự nhiên xuất hiện sớm nhất trong lịch sử loài người. Vật lý nghiên cứu sự vận hành của vật chất, năng lượng cấu thành vũ trụ \& sự tương tác giữa chúng. Những kiến thức vật lý đã, đang \& sẽ có tác động mạnh mẽ vào sự phát triển của mọi lĩnh vực trong cuộc sống, công nghệ, khoa học kỹ thuật.'' -- \cite[p. 3]{SGK_Vat_Ly_10_Chan_Troi_Sang_Tao}

%------------------------------------------------------------------------------%

\chapter{Mở Đầu}

\section{Khái Quát về Môn Vật Lý}
\textbf{Nội dung.} \textit{Đối tượng nghiên cứu, mục tiêu \& phương pháp nghiên cứu của vật lý; ảnh hưởng của Vật lý đối với cuộc sống \& sự phát triển của khoa học, công nghệ \& kỹ thuật}.

\begin{cauhoi}[What? Why?\texttt{/}For what? How?]
	Vật lý nghiên cứu gì? Nghiên cứu vật lý để làm gì? Nghiên cứu vật lý bằng cách nào?
\end{cauhoi}

\subsection{Đối tượng -- Mục tiêu -- Phương pháp nghiên cứu vật lý}

\subsubsection{Đối tượng nghiên cứu của Vật lý}
``Vật lý là môn khoa học tìm hiểu về thế giới tự nhiên. Trong tiếng Hy Lạp, ``Vật lý'' cũng có nghĩa là ``kiến thức về tự nhiên''. Ngày nay, Vật lý được phân làm rất nhiều lĩnh vực, nhiều phân ngành. Khi xem xét nội dung nghiên cứu thuộc các lĩnh vực \& phân ngành của Vật lý, ta có kết luận sau:

\begin{quotation}
	Đối tượng nghiên cứu của Vật lý gồm: các dạng vận động của \textit{vật chất} \& \textit{năng lượng}.
\end{quotation}
Vào năm 1905, nhà vật lý vĩ đại Albert Einstein (1879--1955) đã đưa ra được biểu thức mô tả mối liên hệ giữa năng lượng \& khối lượng $E = mc^2$ \cite[Hình 1.1, p. 5]{SGK_Vat_Ly_10_Chan_Troi_Sang_Tao}.'' -- \cite[p. 5]{SGK_Vat_Ly_10_Chan_Troi_Sang_Tao}

\subsubsection{Mục tiêu của vật lý}
``Theo các Từ điển bách khoa về Khoa học:

\begin{quotation}
	Mục tiêu của Vật lý là khám phá ra quy luật tổng quát nhất chi phối sự vận động của vật chất \& năng lượng, cũng như tương tác giữa chúng ở mọi cấp độ: \textit{vi mô, vĩ mô}.
\end{quotation}
Đến thời điểm hiện nay, tuy Vật lý chưa đạt tới mục tiêu này, nhưng các định luật vật lý được tìm ra đã \& đang không chỉ giúp loài người giải thích mà còn tiên đoán được rất nhiều hiện tượng tự nhiên. Việc vận dụng các định luật này rất đa dạng, phong phú \& có ý nghĩa thiết thực trong đời sống \& nghiên cứu khoa học.

Học tập môn Vật lý giúp học sinh hiểu được các \textit{quy luật của tự nhiên}, vận dụng kiến thức vào cuộc sống, từ đó hình thành các năng lực khoa học \& công nghệ. Những người có năng khiếu \& đam mê có thể học tiếp lên các bậc cao hơn để trở thành các nhà khoa học trong lĩnh vực Vật lý.'' -- \cite[p. 6]{SGK_Vat_Ly_10_Chan_Troi_Sang_Tao}

\subsubsection{Phương pháp nghiên cứu của Vật lý}
``Phương pháp nghiên cứu của Khoa học nói chung \& Vật lý nói riêng được hình thành qua các thời kỳ phát triển của nền văn minh nhân loại, bao gồm 2 phương pháp chính: \textit{phương pháp thực nghiệm} \& \textit{phương pháp lý thuyết}.'' -- \cite[p. 6]{SGK_Vat_Ly_10_Chan_Troi_Sang_Tao}

\paragraph{Phương pháp thực nghiệm.} ``Thí nghiệm về sự rơi của vật được thực hiện bởi Galileo Galilei tại đỉnh tháp nghiêng Pisa cao 57 m (nước Ý) (\cite[Hình 1.3: \textsf{Galileo Galilei (1564--1642) \& tháp nghiêng Pisa.}, p. 6]{SGK_Vat_Ly_10_Chan_Troi_Sang_Tao}) là 1 ví dụ minh họa cho phương pháp thực nghiệm. Tại đây, Galileo Galilei đã thả rơi 2 vật có khối lượng khác nhau (nhưng cùng hình dạng). Kết quả cho thấy 2 vật rơi \& chạm đất cùng lúc. Nhờ kết quả từ thí nghiệm này, Galileo Galilei đã bác bỏ được nhận định của Aristotle (384 BC--322) (1 triết học gia lỗi lạc thời Hy Lạp cổ đại) cho rằng việc vật nặng rơi nhanh hơn vật nhẹ là bản chất tự nhiên của các vật.'' -- \cite[pp. 6--7]{SGK_Vat_Ly_10_Chan_Troi_Sang_Tao}

\paragraph{Phương pháp lý thuyết.} ``Trong quá trình nghiên cứu khoa học, việc hình thành các giả thuyết khoa học la vô cùng quan trọng. Lý thuyết vật lý được xây dựng dựa trên các quan sát ban đầu \& trực giác của các nhà vật lý, trong nhiều trường hợp có tính định hướng \& dẫn dắt cho thực nghiệm kiểm chứng. 1 ví dụ cụ thể cho phương pháp lý thuyết trong Vật lý là công trình dự đoán sự tồn tại của Hải Vương tinh trong hệ Mặt Trời (Fig. \ref{fig:cac hanh tinh trong he Mat Troi}), được thực hiện độc lập bởi các nhà vật lý Johann Gottfried Galle (1812--1910), Urbain Jean Joseph Le Verrier (1811--1877) \& John Couch Adams (1819--1892) vào thế kỷ XIX.

\begin{figure}[H]
	\centering
	\includegraphics[scale=0.2]{cac_hanh_tinh_trong_he_Mat_Troi}
	\caption{Mô hình mô phỏng vị trí các hành tinh trong hệ Mặt Trời: 1. Thủy tinh; 2. Kim tinh; 3. Trái Đất; 4. Hỏa tinh; 5. Mộc tinh; 6. Thổ tinh; 7. Thiên Vương tinh; 8. Hải Vương tinh, \cite[Hình 1.4, p. 7]{SGK_Vat_Ly_10_Chan_Troi_Sang_Tao}.}
	\label{fig:cac hanh tinh trong he Mat Troi}
\end{figure}
Hải Vương tinh không thể quan sát được bằng kính thiên văn 1 cách thuần túy vào thời đại đó. Việc phát hiện ra Hải Vương tinh là nhờ các nhà thiên văn học tiến hành phân tích các dữ liệu liên quan đến chuyển động của Thiên Vương tinh, họ nhận thấy vị trí của Thiên Vương tinh bị nhiễu loạn khi quan sát vị trí của nó, Thiên Vương tinh không ở đúng vị trí mà các phương trình toán học nghiên cứu chuyển động tiên đoán.

Vào giai đoạn đó, có nhiều giả thuyết về sự không chính xác vị trí của Thiên Vương tinh, 1 số người còn cho là định luật hấp dẫn của Newton (1643--1727) không còn đúng ở khoảng cách quá xa so với Mặt Trời. Vậy điều gì làm cho quỹ đạo chuyển động của Thiên Vương tinh không còn đúng khi tính toán bằng định luật hấp dẫn của Newton?

Vấn đề quỹ đạo của Thiên Vương tinh đã khiến các nhà thiên văn học bắt đầu nghĩ có 1 hành tinh khác xa hơn, có thể ảnh hưởng đến chuyển động của Thiên Vương tinh. Nhà thiên văn học người Pháp Urbain Le Verrier sử dụng toán học để xác định hành tinh bí ẩn này, \& cho ra kết quả vào 6.1845. Nhà thiên văn học người Anh John Couch Adams cũng làm việc trên lý thuyết này cho ra 1 kết quả tương tự. Giả thuyết về 1 hành tinh khác ở gần Thiên Vương tinh được sử dụng \& qua tính toán, các nhà thiên văn học định hướng được vị trí quan sát trên bầu trời để xác định hành tinh này. Lý thuyết này đã có thành công rực rỡ vào 23.9.1846, Galle đã sử dụng các tính toán của Le Verrier để tìm ra Hải Vương tinh, chỉ lệch $1^\circ$ so với các tính toán của Le Verrier. Hành tinh này cũng được xác định lệch $12^\circ$ so với các tính toán của Adams.

Việc hình thành lý thuyết dẫn dắt các thực nghiệm kiểm chứng phụ thuộc rất nhiều yếu tố, các dữ liệu quan sát ban đầu, trực giác của nhà khoa học, sự hoàn thiện của công cụ toán học, tính toán tỉ mỉ, $\ldots$ Thực nghiệm kiểm chứng càng nhiều, lý thuyết càng đúng, nhưng chỉ cần 1 thí nghiệm không phù hợp với lý thuyết, lý thuyết đó hoàn toàn bị bác bỏ, các nhà khoa học lại tiếp tục hành trình xây dựng lại giả thuyết \& lý thuyết mới phù hợp với thực nghiệm. Đó là con đường nghiên cứu khoa học.
\begin{itemize}
	\item Phương pháp thực nghiệm dùng thí nghiệm để phát hiện kết quả mới giúp kiểm chứng, hoàn thiện, bổ sung hay bác bỏ giả thuyết nào đó. Kết quả mới này cần được giải thích bằng lý thuyết đã biết hoặc lý thuyết mới.
	\item Phương pháp lý thuyết sử dụng ngôn ngữ toán học \& suy luận lý thuyết để phát hiện 1 kết quả mới. Kết quả mới này cần được kiểm chứng bằng thực nghiệm.
	\item 2 phương pháp hỗ trợ cho nhau, trong đó phương pháp thực nghiệm có tính quyết định.'' -- \cite[pp. 7--8]{SGK_Vat_Ly_10_Chan_Troi_Sang_Tao}
\end{itemize}

\paragraph{Tìm hiểu thế giới tự nhiên dưới góc độ vật lý.}
``Quá trình nghiên cứu của các nhà khoa học nói chung \& nhà vật lý nói riêng chính là quá trình tìm hiểu thế giới tự nhiên. Quá trình này có tiến trình gồm các bước như sau:
\begin{itemize}
	\item Quan sát hiện tượng để xác định đối tượng nghiên cứu.
	\item Đối chiều với các lý thuyết đang có để đề xuất giả thuyết nghiên cứu.
	\item Thiết kế, xây dựng mô hình lý thuyết hoặc mô hình thực nghiệm để kiểm chứng giả thuyết.
	\item Tiến hành tính toán theo mô hình lý thuyết hoặc thực hiện thí nghiệm để thu nhập dữ liệu. Sau đó xử lý số liệu \& phân tích kết quả để xác nhận, điều chỉnh, bổ sung hay loại bỏ mô hình, giả thuyết ban đầu.
	\item Rút ra kết luận.
\end{itemize}

\begin{luuy}
	Trong mỗi bước của tiến trình, công cụ toán học có tính định hướng \& hỗ trợ các tính toán, đặc biệt là đối với Vật lý hiện đại. Để đạt hiệu quả cao, quá trình học tập môn Vật lý ở trường Trung học phổ thông cần được thực hiện theo tiến trình tương tự, trong đó có sự kết hợp hài hòa giữa phương pháp thực nghiệm \& phương pháp lý thuyết.'' -- \cite[p. 9]{SGK_Vat_Ly_10_Chan_Troi_Sang_Tao}
\end{luuy}

\subsection{Ảnh hưởng của Vật lý đến 1 số lĩnh vực trong đời sống \& kỹ thuật}

\subsubsection{Ảnh hưởng của Vật lý trong 1 số lĩnh vực}
\begin{itemize}
	\item ``\textbf{Thông tin liên lạc.} Ngày nay, nền tảng Internet kết hợp với \textit{điện thoại thông minh} \& \textit{1 số thiết bị công nghệ} đã tạo ra 1 phương tiện thông tin liên lạc vô cùng hữu ích. Tin tức, tiếng nói, hình ảnh được truyền đi nhanh chóng đến mọi nơi trên thế giới. Nhờ đó, khoảng cách địa lý không còn là trở ngại \& thế giới hiện nay ngày càng trở nên ``phẳng'' hơn.
	\item \textbf{Y tế.} Các phương pháp chẩn đoán \& chữa bệnh có áp dụng kiến thức vật lý như \textit{phép nội soi, chụp X-quang, chụp cắt lớp vi tính (CT), chụp cộng hưởng từ (MRI), xạ trị}, $\ldots$ đã giúp cho việc chẩn đoán \& chữa trị của bác sĩ đạt hiệu quả cao. Nhờ đó, sức khỏe của con người ngày càng tăng. Tuổi thọ trung bình của người Việt Nam vào năm 2020 là 73.7 tuổi (theo Cục thống kê).
	\item \textbf{Công nghiệp.} Vật lý là động lực của các cuộc cách mạng công nghiệp. Nhờ vậy, nền sản xuất thủ công nhỏ lẻ được chuyển thành nền sản xuất \textit{dây chuyền, tự động hóa}. Từ đố giải phóng sức lao động của con người. Hiện nay, công nghiệp sản xuất đang bước vào thời kỳ 4.0 với cốt lõi là \textit{Internet vạn vật} (IoT) \& \textit{điện toán đám mây}.
	\item \textbf{Nông nghiệp.} Việc ứng dụng những thành tựu của Vật lý đã chuyển đổi quá trình canh tác truyền thống thành các phương pháp hiện đại với năng suất vượt trội nhờ vào máy móc cơ khí tự động hóa. Ngoài ra, việc tạo ra các giống cây trồng có đặc tính ưu việc dựa vào đột biến bằng việc chiếu xạ cũng ngày càng phổ biến. Công nghệ cảm biến không dây cũng giúp cho quá trình kiểm soát chất lượng nông sản được thuận tiện \& đạt hiệu quả cao (\cite[Hình 1.6: \textsf{Công nghệ cảm biến trong việc kiểm soát chất lượng nông sản.}, p. 10]{SGK_Vat_Ly_10_Chan_Troi_Sang_Tao}).
	\item \textbf{Nghiên cứu khoa học.} Vật lý đã giúp cải tiến thiết bị \& phương pháp nghiên cứu của rất nhiều ngành khoa học. E.g.: \textit{Kính hiển vi điện tử} (\cite[Hình 1.7: \textsf{Kính hiển vi điện tử}, p. 10]{SGK_Vat_Ly_10_Chan_Troi_Sang_Tao}) phóng lớn ảnh hàng trăm nghìn lần giúp quan sát vi khuẩn, virus; \textit{nhiễu xạ tia X} giúp khám phá cấu trúc của phân tử DNA; \textit{máy quang phổ} giúp xác định cấu tạo hóa học; $\ldots$
	
	Trong chính môn Vật lý, việc tìm hiểu kiến thức vật lý cũng tạo ra những phương pháp mới, những thiết bị hiện đại, tối tân giúp các nhà nghiên cứu tìm hiểu sâu hơn về vật chất, năng lượng, vũ trụ. 1 trong những thành tựu nổi bật là \textit{kính thiên văn không gian Hubble} (HST) bay quanh Trái Đất ở độ cao hơn 600 km. Kính này đã chụp được ảnh của thiên hà cách xa Trái Đất hơn 13 tỷ năm ánh sáng \& tạo được kho dữ liệu khổng lồ về không gian \& vũ trụ.'' -- \cite[p. 10]{SGK_Vat_Ly_10_Chan_Troi_Sang_Tao}
\end{itemize}

\begin{itemize}
	\item ``Vật lý ảnh hưởng mạnh mẽ \& có tác động làm thay đổi mọi lĩnh vực hoạt động của con người. Dựa trên nền tảng vật lý, các công nghệ mới được sáng tạo với tốc độ vũ bão.
	\item Kiến thức vật lý trong các phân ngành được áp dụng kết hợp để tạo ra kết quả tối ưu. Các kỹ năng vật lý như tính chính xác, đúng thời điểm \& thời lượng, quan sát, suy luận nhạy bén, $\ldots$ đã thành kỹ năng sống cần có của con người hiện đại.'' -- \cite[p. 11]{SGK_Vat_Ly_10_Chan_Troi_Sang_Tao}
\end{itemize}
``Vào đầu thế kỷ XX, J. J. Thomson đã đề xuất mô hình cấu tạo nguyên tử gồm các electron phân bố đều trong 1 khối điện dương kết cấu tựa như khối mây. Để kiểm chứng giả thuyết này, E. Rutherford đã sử dụng tia alpha gồm các hạt mang điện dương bắn vào các nguyên tử kim loại vàng (\cite[Hình 1P.1: \textsf{Thí nghiệm Rutherford}, p. 11]{SGK_Vat_Ly_10_Chan_Troi_Sang_Tao}). Kết quả của thí nghiệm đã bác bỏ giả thuyết của J. J. Thomson, đồng thời đã giúp khám phá ra hạt nhân nguyên tử.'' -- \cite[p. 11]{SGK_Vat_Ly_10_Chan_Troi_Sang_Tao}

%------------------------------------------------------------------------------%

\section{Vấn Đề An Toàn Trong Vật Lý}
\textbf{Nội dung.} \textit{Quy tắc an toàn trong nghiên cứu \& học tập môn Vật lý}.

\subsection{Vấn đề an toàn trong nghiên cứu \& học tập Vật lý}

\subsubsection{Những quy tắc an toàn trong nghiên cứu \& học tập môn Vật lý}

\paragraph{Vấn đề 1: \textit{Phóng xạ}.} ``Hiện tượng phóng xạ tự nhiên được nhà vật lý người Pháp Becquerel (1852--1908) tình cờ khám phá ra vào cuối thế kỷ XIX \& được phát triển nhờ những nghiên cứu của Marie Curie -- người phụ nữ đầu tiên đoạt 2 giải Nobel (Vật lý \& Hóa học). Việc sử dụng chất phóng xạ không đúng cách sẽ ảnh hưởng nghiêm trọng đến sức khỏe con người. Đã có những trường hợp tử vong bởi phóng xạ do chiến tranh, do vô ý phơi nhiễm hay do bị đầu độc.

Để hạn chế những rủi ro \& sự nguy hiểm do chất phóng xạ gây ra, chúng ta phải đảm bảo 1 số quy tắc an toàn như: giảm thời gian tiếp xúc với nguồn phóng xạ, tăng khoảng cách từ ta đến nguồn phóng xạ, đảm bảo che chắn những cơ quan trọng yếu của cơ thể. Ngày nay, các chất phóng xạ đã được ứng dụng rất rộng rãi trong đời sống: sử dụng trong y học để chẩn đoán hình ảnh \& điều trị ung thư, sử dụng trong nông nghiệp để tạo đột biến cải thiện giống cây trồng, sử dụng trong công nghiệp để phát hiện các khiếm khuyết trong vật liệu, sử dụng trong khảo cổ để xác định tuổi của các mẫu vật, $\ldots$'' -- \cite[pp. 12--13]{SGK_Vat_Ly_10_Chan_Troi_Sang_Tao}

\paragraph{Vấn đề 2: \textit{An toàn trong phòng thí nghiệm}.} ``Trong Vật lý, việc tiến hành các hoạt động học trong phòng thí nghiệm nhằm khảo sát, kiểm chứng kiến thức có vai trò quan trọng trong việc phát triển năng lực tìm hiểu thế giới tự nhiên của học sinh. Tuy nhiên, nếu những vấn đề an toàn không được đảm bảo, quá trình tổ chức hoạt động học tập trong phòng thí nghiệm có thể xảy ra nhiều sự cố nguy hiểm cho học sinh. E.g., học sinh có thể bị bỏng khi xảy ra sự cố chập điện hoặc cháy nổ do lửa, hóa chất. Học sinh cũng có thể bị chấn thương cơ thể khi sử dụng những vật sắc nhọn hoặc thủy tinh trong quá trình tiến hành thí nghiệm không đúng cách. Ngoài ra, những tai nạn liên quan đến điện giật thường gây ra hậu quả nghiêm trọng khi học sinh không đảm bảo những nguyên tắc an toàn khi sử dụng điện. Từ đó, ta thấy rằng trong 1 số trường hợp, đối tượng hoặc hiện tượng cần nghiên cứu có thể đem đến những rủi ro, gây nguy hiểm đến sức khỏe của học sinh \& nhà nghiên cứu.'' -- \cite[pp. 13--14]{SGK_Vat_Ly_10_Chan_Troi_Sang_Tao}

``Khi nghiên cứu \& học tập Vật lý, ta cần phải:
\begin{itemize}
	\item Nắm được thông tin liên quan đến các rủi ro \& nguy hiểm có thể xảy ra.
	\item Tuần thủ \& áp dụng các biện pháp bảo vệ để đảm bảo an toàn cho bản thân \& cộng đồng.
	\item Quan tâm, gìn giữ \& bảo vệ môi trường.
	\item Trong phòng thí nghiệm ở trường học, những rủi ro \& nguy hiểm phải được cảnh báo rõ ràng bởi các biển báo. Học sinh cần chú ý sự nhắc nhở của nhân viên phòng thí nghiệm \& giáo viên về các quy định an toàn. Ngoài ra, các thiết bị bảo hộ cá nhân cần phải được trang bị đầy đủ.'' -- \cite[p. 14]{SGK_Vat_Ly_10_Chan_Troi_Sang_Tao}
\end{itemize}

%------------------------------------------------------------------------------%

\section{Đơn Vị \& Sai Số Trong Vật Lý}
\textbf{Nội dung.} \textit{Đơn vị \& thứ nguyên, các loại sai số đơn giản \& cách hạn chế}.

\subsection{Đơn vị \& thứ nguyên trong vật lý}

\subsubsection{Hệ đơn vị SI, đơn vị cơ bản \& đơn vị dẫn suất}
``Tập hợp của đơn vị được gọi là hệ đơn vị. Trong khoa học có rất nhiều hệ đơn vị được sử dụng, trong đó thông dụng nhất là hệ đơn vị đo lường quốc tế \textbf{SI} (\textbf{S}yst\`eme \textbf{I}nternational d'unit\'es) được xây dựng trên cơ sở của \textit{7 đơn vị cơ bản} (Bảng \ref{tab:cac don vi co ban trong he SI}).
\begin{table}[H]
	\centering
	\begin{tabular}{|c|c|c|c|}
		\hline
		\textbf{STT} & \textbf{Đơn vị} & \textbf{Ký hiệu} & \textbf{Đại lượng} \\
		\hline
		1 & mét & m & chiều dài \\
		\hline
		2 & kilogram & kg & khối lượng \\
		\hline
		3 & giây & s & thời gian \\
		\hline
		4 & kelvin & K & nhiệt độ \\
		\hline
		5 & ampe & A & cường độ dòng điện \\
		\hline
		6 & mol & mol & lượng chất \\
		\hline
		7 & candela & cd & cường độ ánh sáng \\
		\hline
	\end{tabular}
	\caption{Các đơn vị cơ bản trong hệ SI, \cite[Bảng 3.1, p. 16]{SGK_Vat_Ly_10_Chan_Troi_Sang_Tao}.}
	\label{tab:cac don vi co ban trong he SI}
\end{table}
Khi số đo của đại lượng đang xem xét là 1 bội số hoặc ước số thập phân của 10, ta có thể sử dụng tiếp đầu ngữ như trong Bảng \ref{tab:ten & ky hieu tiep dau ngu cua boi so, uoc so thap phan cua don vi} ngay trước đơn vị để phần số đo được trình bày ngắn gọn.'' -- \cite[p. 15]{SGK_Vat_Ly_10_Chan_Troi_Sang_Tao}

\begin{table}[H]
	\centering
	\begin{tabular}{|c|c|c|c|c|c|c|c|c|c|c|}
		\hline
		\textbf{Ký hiệu} & d & c & m & $\mu$ & n & p & f & a & z & y \\
		\hline
		\textbf{Tên đọc} & deci & centi & mili & micro & nano & pico & femto & atto & zepto & yokto \\
		\hline
		\textbf{Hệ số} & $0^{-1}$ & $10^{-2}$ & $10^{-3}$ & $10^{-6}$ & $10^{-9}$ & $10^{-12}$ & $10^{-15}$ & $10^{-18}$ & $10^{-21}$ & $10^{-24}$ \\
		\hline
		\textbf{Ký hiệu} & da & h & k & M & G & T & P & E & Z & Y \\
		\hline
		\textbf{Tên đọc} & deka & hecto & kilo & mega & giga & tera & peta & eta & zetta & yotta \\
		\hline
		\textbf{Hệ số} & $10^1$ & $10^2$ & $10^3$ & $10^6$ & $10^9$ & $10^{12}$ & $10^{15}$ & $10^{18}$ & $10^{21}$ & $10^{24}$ \\
		\hline
	\end{tabular}
	\caption{Tên \& ký hiệu tiếp đầu ngữ của bội số, ước số thập phân của đơn vị, \cite[Bảng 3.2, p. 16]{SGK_Vat_Ly_10_Chan_Troi_Sang_Tao}.}
	\label{tab:ten & ky hieu tiep dau ngu cua boi so, uoc so thap phan cua don vi}
\end{table}
``Ngoài 7 đơn vị cơ bản, những đơn vị còn lại được gọi là \textit{đơn vị dẫn xuất}. Mọi đơn vị dẫn xuất đều có thể phân tích thành các đơn vị cơ bản dựa vào mối liên hệ giữa các đại lượng tương ứng.'' -- \cite[p. 16]{SGK_Vat_Ly_10_Chan_Troi_Sang_Tao}

\subsubsection{Thứ nguyên}
``\textit{Thứ nguyên} của 1 đại lượng là quy luật nêu lên sự phụ thuộc của đơn vị đo đại lượng đó vào các đơn vị cơ bản. Thứ nguyên của 1 đại lượng $X$ được biểu diễn dưới dạng $[X]$. Thứ nguyên của 1 số đại lượng cơ bản thường sử dụng được thể hiện trong Bảng \ref{tab:thu nguyen cua 1 so dai luong co ban}. 1 đại lượng vật lý có thể được biểu diễn bằng nhiều đơn vị khác nhau nhưng chỉ có 1 thứ nguyên duy nhất. 1 số đại lượng vật lý có thể có cùng thứ nguyên. E.g., Tọa độ, quãng đường đi được có thể được biểu diễn bằng đơn vị mét, cây số, hải lý, feet, dặm, $\ldots$ nhưng chỉ có 1 thứ nguyên $L$. Tốc độ, vận tốc có thể được biểu diễn bằng đơn vị m\texttt{/}s, km\texttt{/}h, dặm\texttt{/}giờ nhưng chỉ có 1 thứ nguyên $\rm LT^{-1}$.'' -- \cite[p. 16]{SGK_Vat_Ly_10_Chan_Troi_Sang_Tao}

\begin{table}[H]
	\centering
	\begin{tabular}{|c|c|c|c|c|c|}
		\hline
		\textbf{Đại lượng cơ bản} & [Chiều dài] & [Khối lượng] & [Thời gian] & [Cường độ dòng điện] & [Nhiệt độ] \\
		\hline
		\textbf{Thứ nguyên} & $L$ & $M$ & $T$ & $I$ & $K$ \\
		\hline
	\end{tabular}
	\caption{Thứ nguyên của 1 số đại lượng cơ bản, \cite[Bảng 3.3, p. 16]{SGK_Vat_Ly_10_Chan_Troi_Sang_Tao}.}
	\label{tab:thu nguyen cua 1 so dai luong co ban}
\end{table}

\begin{luuy}
	``Trong các biểu thức vật lý: Các số hạng trong phép cộng (hoặc trừ) phải có cùng thứ nguyên. 2 vế của 1 biểu thức vật lý phải có cùng thứ nguyên.'' -- \cite[p. 17]{SGK_Vat_Ly_10_Chan_Troi_Sang_Tao}
\end{luuy}

\subsubsection{Vận dụng mối liên hệ giữa đơn vị dẫn xuất tới 7 đơn vị cơ bản của hệ SI}
``Trong hệ SI, $s,v$, \& $t$ lần lượt có đơn vị là $\rm m,ms^{-1},s$.'' ``Hiện nay có những đơn vị thường được dùng trong đời sống như picomet (pm), miliampe (mA) (e.g., kích thước của 1 hạt bụi là khoảng $2.5$ pm; cường độ dòng điện dùng trong châm cứu $\approx2$ mA).'' ``Lực cản không khí tác dụng lên vật phụ thuộc vào vận tốc chuyển động của vật theo công thức $F = -kv^2$. Thứ nguyên của lực là $MLT^{-2}$.'' -- \cite[p. 17]{SGK_Vat_Ly_10_Chan_Troi_Sang_Tao}

``Sep 23, 1999, tàu quỹ đạo thăm dò khí hậu của hỏa tinh có trị giá 125 triệu USD của NASA đã bị phá hủy hoàn toàn khi không đáp ứng được độ cao cần thiết so với bề mặt Hỏa tinh. Sau khi tiến hành điều tra, các nhà khoa học của NASA đã phát hiện ra nguyên nhân của vụ tại nạn chính là sự thiếu thống nhất trong việc chuyển đổi giữa hệ đơn vị SI \& hệ đơn vị của Anh đối với nhóm thiết kế \& nhóm thực hiện nhiệm vụ phóng tàu. Đây là 1 trong những ví dụ cho thấy tầm quan trọng của việc xác định chính xác đơn vị khi tiến hành tính toán \& đo đạc, từ đó giúp cho chúng ta phòng tránh được những thiệt hại đáng tiếc.'' -- \cite[p. 18]{SGK_Vat_Ly_10_Chan_Troi_Sang_Tao}

\subsection{Sai số trong phép đo \& cách hạn chế}

\subsubsection{Các phép đo trong Vật lý}

\begin{dinhnghia}[Phép đo các đại lượng vật lý]
	``\emph{Phép đo các đại lượng vật lý} là phép so sánh chúng với đại lượng cùng loại được quy ước làm đơn vị. \emph{Phép đo trực tiếp}: giá trị của đại lượng cần đo được đọc trực tiếp trên dụng cụ đo (e.g., đo khối lượng bằng cân, đo thể tích bằng bình chia độ). \emph{Phép đo gián tiếp}: giá trị của đại lượng cần đo được xác định thông qua các đại lượng được đo trực tiếp (e.g., đo khối lượng riêng).'' -- \cite[p. 18]{SGK_Vat_Ly_10_Chan_Troi_Sang_Tao}
\end{dinhnghia}

\subsubsection{Các loại sai số của phép đo}
``Trong quá trình thực hiện phép đo, chúng ta không thể tránh khỏi sự chênh lệch giữa giá trị thật \& số đo (giá trị đo được). Độ chênh lệch này gọi là \textit{sai số}. Như vậy, mọi phép đo đều tồn tại sai số. Nguyên nhân gây ra sai số là do giới hạn về độ chính xác của dụng cụ đo, do kỹ thuật đo, quy trình đo, chủ quan của người đo, $\ldots$ Xét theo nguyên nhân thì sai số của phép đo được phân thành 2 loại là \textit{sai số hệ thống} \& \textit{sai số ngẫu nhiên}.'' -- \cite[p. 19]{SGK_Vat_Ly_10_Chan_Troi_Sang_Tao}

\begin{dinhnghia}[Sai số hệ thống]
	``\emph{Sai số hệ thống} là sai số có tính quy luật \& được lặp lại ở tất cả các lần đo. Sai số hệ thống làm cho giá trị đo tăng hoặc giảm 1 lượng nhất định so với giá trị thực.
\end{dinhnghia}
Sai số hệ thống thường xuất phát từ dụng cụ đo. E.g., kết quả khối lượng trong mọi lần đo đều lớn hơn giá trị thật 1 lượng xác định khi ta không hiệu chỉnh kim của cân về đúng vị trí. Ngoài ra, sai số hệ thống còn xuất phát từ độ chia nhỏ nhất của dụng cụ đo (gọi là \textit{sai số dụng cụ}). Đối với 1 số dụng cụ đo, sai số này thường được xác định bằng 1 nửa độ chia nhỏ nhất. Trong thực hiện phép đo, cần tìm được nguyên nhân gây ra sai số hệ thống để tìm cách hạn chế. Sai số hệ thống có thể được hạn chế bằng cách thường xuyên hiệu chỉnh dụng cụ đo, sử dụng thiết bị đo có độ chính xác cao.

\begin{dinhnghia}[Sai số ngẫu nhiên]
	\emph{Sai số ngẫu nhiên} là sai số xuất phát từ sai sót, phản xạ của người làm thí nghiệm hoặc từ những yếu tố ngẫu nhiên bên ngoài. Sai số này thường có nguyên nhân không rõ ràng \& dẫn đến sự phân tán của các kết quả đo xung quanh 1 giá trị trung bình.'' -- \cite[p. 19]{SGK_Vat_Ly_10_Chan_Troi_Sang_Tao}
\end{dinhnghia}
``E.g., khi đo thời gian rơi của 1 vật bằng đồng hồ bấm giây, phản xạ của người đo sẽ gây ra sai số ngẫu nhiên. Khi đo khối lượng của 1 vật nhỏ bằng 1 cân hiện số có độ nhạy cao, các yếu tố khách quan như gió, bụi cũng có thể gây ra sai số ngẫu nhiên. Sai số ngẫu nhiên có thể được hạn chế bằng cách thực hiện phép đo nhiều lần \& lấy giá trị trung bình để hạn chế sự phân tán của số liệu đo.'' -- \cite[p. 20]{SGK_Vat_Ly_10_Chan_Troi_Sang_Tao}

\subsubsection{Cách biểu diễn sai số của phép đo}
``Khi tiến hành đo đạc, giá trị $x$ của 1 đại lượng vật lý thường được ghi dưới dạng $x = \overline{x}\pm\Delta x$ với $\overline{x}$ là \textit{giá trị trung bình} của đại lượng cần đo khi tiến hành phép đo nhiều lần: $\overline{x}\coloneqq\frac{1}{n}\sum_{i=1}^n x_i$.  Giá trị trung bình có thể xem là giá trị gần đúng nhất với giá trị thật của đại lượng vật lý cần đo. Sai số của phép đo có thể biểu diễn dưới dạng:
\begin{itemize}
	\item \textbf{Sai số tuyệt đối} là $\Delta x$ trong công thức $x = \overline{x}\pm\Delta x$. \textit{Sai số tuyệt đối ứng với mỗi lần đo} được xác định bằng trị tuyệt đối của hiệu giữa giá trị trung bình \& giá trị của mỗi lần đo. $\Delta x_i\coloneqq|\overline{x} - x_i|$ với $x_i$ là \textit{giá trị đo lần thứ $i$}. \textit{Sai số tuyệt đối trung bình} của $n$ lần đo được xác định theo công thức: $\overline{\Delta x} = \frac{1}{n}\sum_{i=1}^n \Delta x_i$. \textit{Sai số tuyệt đối của phép đo cho biết phạm vi biến thiên của giá trị đo được \& bằng tổng của sai số ngẫu nhiên \& sai số dụng cụ}: $\Delta x = \overline{\Delta x} + \Delta x_{\rm dc}$, trong đó \textit{sai số dụng cụ} $\Delta x_{\rm dc}$ thường được xem có giá trị bằng 1 \textit{nửa độ chia nhỏ nhất} đối với những dụng cụ đơn giản như thước kẻ, cân bàn, bình chia độ, $\ldots$ Trong nhiều trường hợp, sai số dụng cụ thường được cung cấp chính xác bởi nhà sản xuất.
	\item \textbf{Sai số tương đối} được xác định bằng tỷ số giữa sai số tuyệt đối \& giá trị trung bình của đại lượng cần đo theo công thức: $\delta x = \frac{\Delta x}{\overline{x}}\cdot 100\%$. Sai số tương đối cho biết mức độ chính xác của phép đo.'' -- \cite[p. 21]{SGK_Vat_Ly_10_Chan_Troi_Sang_Tao}
\end{itemize}

\subsubsection{Cách xác định sai số trong phép đo gián tiếp}
``Trong đa số trường hợp, 1 đại lượng cần đo (có giá trị $F$) được xác định gián tiếp thông qua việc đo trực tiếp những đại lượng khác (có giá trị $x,y,z,\ldots$). E.g., khối lượng riêng được xác định bằng thương số của khối lượng \& thể tích, chu vi hình chữ nhật được xác định bằng 2 lần tổng của 2 cạnh liên tiếp. Nguyên tắc xác định sai số trong phép đo gián tiếp như sau:
\begin{enumerate*}
	\item[$\bullet$] Sai số tuyệt đối của 1 tổng hay hiệu bằng tổng sai số tuyệt đối của các số hạng: $F = x\pm y\mp z\ldots\Rightarrow\Delta F = \Delta x + \Delta y + \Delta z\ldots$.
	\item[$\bullet$] Sai số tương đối của 1 tích hoặc thương bằng tổng sai số tương đối của các thừa số: $F = x^m\frac{y^n}{z^k}\Rightarrow\delta F = m\delta x + n\delta y + k\delta z$. 
\end{enumerate*}

\begin{luuy}
	$\sqrt[m]{x}$ có thể được viết lại thành $x^n$ với $n = \frac{1}{m}$. 
\end{luuy}
Các chữ số có nghĩa bao gồm: Các chữ số khác $0$, các chữ số $0$ nằm giữa 2 chữ số $\ne 0$ hoặc nằm bên phải của dấu thập phân \& 1 chữ số $\ne 0$.'' -- \cite[pp. 21--22]{SGK_Vat_Ly_10_Chan_Troi_Sang_Tao}

\section{Tổng Kết Chương}
``\textbf{1.} \textsc{Đối tượng nghiên cứu của vật lý.} Các dạng vận động của vật chất \& năng lượng. \textbf{2.} \textsc{Mục tiêu nghiên cứu của vật lý.} Tìm được quy luật tổng quát chi phối sự biến đổi \& vận hành của vật chất, năng lượng. \textbf{3.} \textsc{Phương pháp nghiên cứu của vật lý.} Có 2 phương pháp nghiên cứu vật lý: phương pháp thực nghiệm \& phương pháp lý thuyết. \textbf{4.} \textsc{Ảnh hưởng của vật lý.} Ngày càng rộng khắp, bao trùm mọi lĩnh vực: đời sống, công nghiệp, nông nghiệp, nghiên cứu khoa học. \textbf{5.} \textsc{Vấn đề an toàn trong nghiên cứu \& học tập môn Vật lý.} Hiểu các rủi ro, thực hieej các biện pháp an toàn cho bản thân, cộng đồng, môi trường theo quy định của nơi học tập, làm việc. \textbf{6.} \textsc{Các loại sai số \& cách hạn chế.} Sai số của phép đo gồm sai số hệ thống \& sai số ngẫu nhiên. Sai số của phép đo có thể biểu diễn dưới dạng \textit{sai số tuyệt đối} \& \textit{sai số tương đối}. Hạn chế sai số: thao tác đúng cách, lựa chọn thiết bị phù hợp, tiến hành đo nhiều lần.'' -- \cite[pp. 23]{SGK_Vat_Ly_10_Chan_Troi_Sang_Tao}

%------------------------------------------------------------------------------%

\chapter{Mô Tả Chuyển Động}

\section{Chuyển Động Thẳng}

%------------------------------------------------------------------------------%

\section{Chuyển Động Tổng Hợp}

%------------------------------------------------------------------------------%

\section{Thực Hành Đo Tốc Độ của Vật Chuyển Động Thẳng}

%------------------------------------------------------------------------------%

\chapter{Chuyển Động Biến Đổi}

\section{Gia Tốc -- Chuyển Động Thẳng Biến Đổi Đều}

%------------------------------------------------------------------------------%

\section{Thực Hành Đo Gia Tốc Rơi Tự Do}

%------------------------------------------------------------------------------%

\section{Chuyển Động Ném}

%------------------------------------------------------------------------------%

\chapter{3 Định Luật Newton. 1 Số Lực Trong Thực Tiễn}

\section{3 Định Luật Newton về Chuyển Động}

%------------------------------------------------------------------------------%

\section{1 Số Lực Trong Thực Tiễn}

%------------------------------------------------------------------------------%

\section{Chuyển Động của Vật Trong Chất Lưu}

%------------------------------------------------------------------------------%

\chapter{Moment Lực. Điều Kiện Cân Bằng}

\section{Tổng Hợp Lực -- Phân Tích Lực}

%------------------------------------------------------------------------------%

\section{Moment Lực. Điều kiện Cân Bằng của Vật}

%------------------------------------------------------------------------------%

\chapter{Năng Lượng}

\section{Năng Lượng \& Công}

%------------------------------------------------------------------------------%

\section{Công Suất -- Hiệu Suất}

%------------------------------------------------------------------------------%

\section{Động Năng \& Thế Năng. Định Luật Bảo Toàn Cơ Năng}

%------------------------------------------------------------------------------%

\chapter{Động Lượng}

\section{Động Lượng \& Định Luật Bảo Toàn Động Lượng}

%------------------------------------------------------------------------------%

\section{Các Loại Va Chạm}

%------------------------------------------------------------------------------%

\chapter{Chuyển Động Tròn}

\section{Động Học của Chuyển Động Tròn}

%------------------------------------------------------------------------------%

\section{Động Lực Học của Chuyển Động Tròn. Lực Hướng Tâm}

%------------------------------------------------------------------------------%

\chapter{Biến Dạng của Vật Rắn}

\section{Biến Dạng của Vật Rắn. Đặc Tính của Lò Xo}

%------------------------------------------------------------------------------%
R
\section{Định Luật Hooke}

%------------------------------------------------------------------------------%

\printbibliography[heading=bibintoc]
	
\end{document}