\documentclass[oneside]{book}
\usepackage[backend=biber,natbib=true,style=authoryear]{biblatex}
\addbibresource{/home/hong/1_NQBH/reference/bib.bib}
\usepackage[utf8]{vietnam}
\usepackage{tocloft}
\renewcommand{\cftsecleader}{\cftdotfill{\cftdotsep}}
\usepackage[colorlinks=true,linkcolor=blue,urlcolor=red,citecolor=magenta]{hyperref}
\usepackage{amsmath,amssymb,amsthm,mathtools,float,graphicx,algpseudocode,algorithm,tcolorbox,tikz,tkz-tab,diagbox}
\DeclareMathOperator{\arccot}{arccot}
\usepackage[inline]{enumitem}
\allowdisplaybreaks
\numberwithin{equation}{section}
\newtheorem{assumption}{Assumption}[section]
\newtheorem{nhanxet}{Nhận xét}[section]
\newtheorem{conjecture}{Conjecture}[section]
\newtheorem{corollary}{Corollary}[section]
\newtheorem{hequa}{Hệ quả}[section]
\newtheorem{definition}{Definition}[section]
\newtheorem{dinhnghia}{Định nghĩa}[section]
\newtheorem{example}{Example}[section]
\newtheorem{vidu}{Ví dụ}[section]
\newtheorem{lemma}{Lemma}[section]
\newtheorem{notation}{Notation}[section]
\newtheorem{principle}{Principle}[section]
\newtheorem{problem}{Problem}[section]
\newtheorem{baitoan}{Bài toán}[section]
\newtheorem{proposition}{Proposition}[section]
\newtheorem{menhde}{Mệnh đề}[section]
\newtheorem{question}{Question}[section]
\newtheorem{cauhoi}{Câu hỏi}[section]
\newtheorem{remark}{Remark}[section]
\newtheorem{luuy}{Lưu ý}[section]
\newtheorem{theorem}{Theorem}[section]
\newtheorem{dinhly}{Định lý}[section]
\usepackage[left=0.5in,right=0.5in,top=1.5cm,bottom=1.5cm]{geometry}
\usepackage{fancyhdr}
\pagestyle{fancy}
\fancyhf{}
\lhead{\small \textsc{Sect.} ~\thesection}
\rhead{\small \nouppercase{\leftmark}}
\renewcommand{\sectionmark}[1]{\markboth{#1}{}}
\cfoot{\thepage}
\def\labelitemii{$\circ$}

\title{Some Topics in Elementary Physics\texttt{/}Grade 10}
\author{Nguyễn Quản Bá Hồng\footnote{Independent Researcher, Ben Tre City, Vietnam\\e-mail: \texttt{nguyenquanbahong@gmail.com}; website: \url{https://nqbh.github.io}.}}
\date{\today}

\begin{document}
\frontmatter
\maketitle
\setcounter{secnumdepth}{4}
\setcounter{tocdepth}{3}
\tableofcontents
\newpage

%------------------------------------------------------------------------------%

\mainmatter

\chapter*{Preface}

Tóm tắt kiến thức Vật lý lớp 10 theo chương trình giáo dục của Việt Nam \& một số chủ đề nâng cao.

``Vật lý được biết đến như là 1 trong những ngành của Khoa học tự nhiên xuất hiện sớm nhất trong lịch sử loài người. Vật lý nghiên cứu sự vận hành của vật chất, năng lượng cấu thành vũ trụ \& sự tương tác giữa chúng. Những kiến thức vật lý đã, đang \& sẽ có tác động mạnh mẽ vào sự phát triển của mọi lĩnh vực trong cuộc sống, công nghệ, khoa học kỹ thuật.'' -- \cite[p. 3]{SGK_Vat_Ly_10_Chan_Troi_Sang_Tao}

%------------------------------------------------------------------------------%

\chapter{Mở Đầu}

\section{Khái Quát về Môn Vật Lý}

\begin{quotation}
	\textbf{Nội dung.} \textit{Đối tượng nghiên cứu, mục tiêu \& phương pháp nghiên cứu của vật lý; ảnh hưởng của Vật lý đối với cuộc sống \& sự phát triển của khoa học, công nghệ \& kỹ thuật}.
\end{quotation}

\begin{cauhoi}[What? Why?\texttt{/}For what? How?]
	Vật lý nghiên cứu gì? Nghiên cứu vật lý để làm gì? Nghiên cứu vật lý bằng cách nào?
\end{cauhoi}

\subsection{Đối tượng -- Mục tiêu -- Phương pháp nghiên cứu vật lý}

\subsubsection{Đối tượng nghiên cứu của Vật lý}
``Vật lý là môn khoa học tìm hiểu về thế giới tự nhiên. Trong tiếng Hy Lạp, ``Vật lý'' cũng có nghĩa là ``kiến thức về tự nhiên''. Ngày nay, Vật lý được phân làm rất nhiều lĩnh vực, nhiều phân ngành. Khi xem xét nội dung nghiên cứu thuộc các lĩnh vực \& phân ngành của Vật lý, ta có kết luận sau:

\begin{quotation}
	Đối tượng nghiên cứu của Vật lý gồm: các dạng vận động của \textit{vật chất} \& \textit{năng lượng}.
\end{quotation}
Vào năm 1905, nhà vật lý vĩ đại Albert Einstein (1879--1955) đã đưa ra được biểu thức mô tả mối liên hệ giữa năng lượng \& khối lượng $E = mc^2$ \cite[Hình 1.1, p. 5]{SGK_Vat_Ly_10_Chan_Troi_Sang_Tao}.'' -- \cite[p. 5]{SGK_Vat_Ly_10_Chan_Troi_Sang_Tao}

\subsubsection{Mục tiêu của vật lý}
``Theo các Từ điển bách khoa về Khoa học:

\begin{quotation}
	Mục tiêu của Vật lý là khám phá ra quy luật tổng quát nhất chi phối sự vận động của vật chất \& năng lượng, cũng như tương tác giữa chúng ở mọi cấp độ: \textit{vi mô, vĩ mô}.
\end{quotation}
Đến thời điểm hiện nay, tuy Vật lý chưa đạt tới mục tiêu này, nhưng các định luật vật lý được tìm ra đã \& đang không chỉ giúp loài người giải thích mà còn tiên đoán được rất nhiều hiện tượng tự nhiên. Việc vận dụng các định luật này rất đa dạng, phong phú \& có ý nghĩa thiết thực trong đời sống \& nghiên cứu khoa học.

Học tập môn Vật lý giúp học sinh hiểu được các \textit{quy luật của tự nhiên}, vận dụng kiến thức vào cuộc sống, từ đó hình thành các năng lực khoa học \& công nghệ. Những người có năng khiếu \& đam mê có thể học tiếp lên các bậc cao hơn để trở thành các nhà khoa học trong lĩnh vực Vật lý.'' -- \cite[p. 6]{SGK_Vat_Ly_10_Chan_Troi_Sang_Tao}

\subsubsection{Phương pháp nghiên cứu của Vật lý}
``Phương pháp nghiên cứu của Khoa học nói chung \& Vật lý nói riêng được hình thành qua các thời kỳ phát triển của nền văn minh nhân loại, bao gồm 2 phương pháp chính: \textit{phương pháp thực nghiệm} \& \textit{phương pháp lý thuyết}.'' -- \cite[p. 6]{SGK_Vat_Ly_10_Chan_Troi_Sang_Tao}

\paragraph{Phương pháp thực nghiệm.} ``Thí nghiệm về sự rơi của vật được thực hiện bởi Galileo Galilei tại đỉnh tháp nghiêng Pisa cao 57 m (nước Ý) (\cite[Hình 1.3: \textsf{Galileo Galilei (1564--1642) \& tháp nghiêng Pisa.}, p. 6]{SGK_Vat_Ly_10_Chan_Troi_Sang_Tao}) là 1 ví dụ minh họa cho phương pháp thực nghiệm. Tại đây, Galileo Galilei đã thả rơi 2 vật có khối lượng khác nhau (nhưng cùng hình dạng). Kết quả cho thấy 2 vật rơi \& chạm đất cùng lúc. Nhờ kết quả từ thí nghiệm này, Galileo Galilei đã bác bỏ được nhận định của Aristotle (384 BC--322) (1 triết học gia lỗi lạc thời Hy Lạp cổ đại) cho rằng việc vật nặng rơi nhanh hơn vật nhẹ là bản chất tự nhiên của các vật.'' -- \cite[pp. 6--7]{SGK_Vat_Ly_10_Chan_Troi_Sang_Tao}

\paragraph{Phương pháp lý thuyết.} ``Trong quá trình nghiên cứu khoa học, việc hình thành các giả thuyết khoa học la vô cùng quan trọng. Lý thuyết vật lý được xây dựng dựa trên các quan sát ban đầu \& trực giác của các nhà vật lý, trong nhiều trường hợp có tính định hướng \& dẫn dắt cho thực nghiệm kiểm chứng. 1 ví dụ cụ thể cho phương pháp lý thuyết trong Vật lý là công trình dự đoán sự tồn tại của Hải Vương tinh trong hệ Mặt Trời (Fig. \ref{fig:cac hanh tinh trong he Mat Troi}), được thực hiện độc lập bởi các nhà vật lý Johann Gottfried Galle (1812--1910), Urbain Jean Joseph Le Verrier (1811--1877) \& John Couch Adams (1819--1892) vào thế kỷ XIX.

\begin{figure}[H]
	\centering
	\includegraphics[scale=0.2]{cac_hanh_tinh_trong_he_Mat_Troi}
	\caption{Mô hình mô phỏng vị trí các hành tinh trong hệ Mặt Trời: 1. Thủy tinh; 2. Kim tinh; 3. Trái Đất; 4. Hỏa tinh; 5. Mộc tinh; 6. Thổ tinh; 7. Thiên Vương tinh; 8. Hải Vương tinh, \cite[Hình 1.4, p. 7]{SGK_Vat_Ly_10_Chan_Troi_Sang_Tao}.}
	\label{fig:cac hanh tinh trong he Mat Troi}
\end{figure}
Hải Vương tinh không thể quan sát được bằng kính thiên văn 1 cách thuần túy vào thời đại đó. Việc phát hiện ra Hải Vương tinh là nhờ các nhà thiên văn học tiến hành phân tích các dữ liệu liên quan đến chuyển động của Thiên Vương tinh, họ nhận thấy vị trí của Thiên Vương tinh bị nhiễu loạn khi quan sát vị trí của nó, Thiên Vương tinh không ở đúng vị trí mà các phương trình toán học nghiên cứu chuyển động tiên đoán.

Vào giai đoạn đó, có nhiều giả thuyết về sự không chính xác vị trí của Thiên Vương tinh, 1 số người còn cho là định luật hấp dẫn của Newton (1643--1727) không còn đúng ở khoảng cách quá xa so với Mặt Trời. Vậy điều gì làm cho quỹ đạo chuyển động của Thiên Vương tinh không còn đúng khi tính toán bằng định luật hấp dẫn của Newton?

Vấn đề quỹ đạo của Thiên Vương tinh đã khiến các nhà thiên văn học bắt đầu nghĩ có 1 hành tinh khác xa hơn, có thể ảnh hưởng đến chuyển động của Thiên Vương tinh. Nhà thiên văn học người Pháp Urbain Le Verrier sử dụng toán học để xác định hành tinh bí ẩn này, \& cho ra kết quả vào 6.1845. Nhà thiên văn học người Anh John Couch Adams cũng làm việc trên lý thuyết này cho ra 1 kết quả tương tự. Giả thuyết về 1 hành tinh khác ở gần Thiên Vương tinh được sử dụng \& qua tính toán, các nhà thiên văn học định hướng được vị trí quan sát trên bầu trời để xác định hành tinh này. Lý thuyết này đã có thành công rực rỡ vào 23.9.1846, Galle đã sử dụng các tính toán của Le Verrier để tìm ra Hải Vương tinh, chỉ lệch $1^\circ$ so với các tính toán của Le Verrier. Hành tinh này cũng được xác định lệch $12^\circ$ so với các tính toán của Adams.

Việc hình thành lý thuyết dẫn dắt các thực nghiệm kiểm chứng phụ thuộc rất nhiều yếu tố, các dữ liệu quan sát ban đầu, trực giác của nhà khoa học, sự hoàn thiện của công cụ toán học, tính toán tỉ mỉ, $\ldots$ Thực nghiệm kiểm chứng càng nhiều, lý thuyết càng đúng, nhưng chỉ cần 1 thí nghiệm không phù hợp với lý thuyết, lý thuyết đó hoàn toàn bị bác bỏ, các nhà khoa học lại tiếp tục hành trình xây dựng lại giả thuyết \& lý thuyết mới phù hợp với thực nghiệm. Đó là con đường nghiên cứu khoa học.
\begin{itemize}
	\item Phương pháp thực nghiệm dùng thí nghiệm để phát hiện kết quả mới giúp kiểm chứng, hoàn thiện, bổ sung hay bác bỏ giả thuyết nào đó. Kết quả mới này cần được giải thích bằng lý thuyết đã biết hoặc lý thuyết mới.
	\item Phương pháp lý thuyết sử dụng ngôn ngữ toán học \& suy luận lý thuyết để phát hiện 1 kết quả mới. Kết quả mới này cần được kiểm chứng bằng thực nghiệm.
	\item 2 phương pháp hỗ trợ cho nhau, trong đó phương pháp thực nghiệm có tính quyết định.'' -- \cite[pp. 7--8]{SGK_Vat_Ly_10_Chan_Troi_Sang_Tao}
\end{itemize}

\paragraph{Tìm hiểu thế giới tự nhiên dưới góc độ vật lý.}
``Quá trình nghiên cứu của các nhà khoa học nói chung \& nhà vật lý nói riêng chính là quá trình tìm hiểu thế giới tự nhiên. Quá trình này có tiến trình gồm các bước như sau:
\begin{itemize}
	\item Quan sát hiện tượng để xác định đối tượng nghiên cứu.
	\item Đối chiều với các lý thuyết đang có để đề xuất giả thuyết nghiên cứu.
	\item Thiết kế, xây dựng mô hình lý thuyết hoặc mô hình thực nghiệm để kiểm chứng giả thuyết.
	\item Tiến hành tính toán theo mô hình lý thuyết hoặc thực hiện thí nghiệm để thu nhập dữ liệu. Sau đó xử lý số liệu \& phân tích kết quả để xác nhận, điều chỉnh, bổ sung hay loại bỏ mô hình, giả thuyết ban đầu.
	\item Rút ra kết luận.
\end{itemize}

\begin{luuy}
	Trong mỗi bước của tiến trình, công cụ toán học có tính định hướng \& hỗ trợ các tính toán, đặc biệt là đối với Vật lý hiện đại. Để đạt hiệu quả cao, quá trình học tập môn Vật lý ở trường Trung học phổ thông cần được thực hiện theo tiến trình tương tự, trong đó có sự kết hợp hài hòa giữa phương pháp thực nghiệm \& phương pháp lý thuyết.'' -- \cite[p. 9]{SGK_Vat_Ly_10_Chan_Troi_Sang_Tao}
\end{luuy}

\subsection{Ảnh hưởng của Vật lý đến 1 số lĩnh vực trong đời sống \& kỹ thuật}

\subsubsection{Ảnh hưởng của Vật lý trong 1 số lĩnh vực}
\begin{itemize}
	\item ``\textbf{Thông tin liên lạc.} Ngày nay, nền tảng Internet kết hợp với \textit{điện thoại thông minh} \& \textit{1 số thiết bị công nghệ} đã tạo ra 1 phương tiện thông tin liên lạc vô cùng hữu ích. Tin tức, tiếng nói, hình ảnh được truyền đi nhanh chóng đến mọi nơi trên thế giới. Nhờ đó, khoảng cách địa lý không còn là trở ngại \& thế giới hiện nay ngày càng trở nên ``phẳng'' hơn.
	\item \textbf{Y tế.} Các phương pháp chẩn đoán \& chữa bệnh có áp dụng kiến thức vật lý như \textit{phép nội soi, chụp X-quang, chụp cắt lớp vi tính (CT), chụp cộng hưởng từ (MRI), xạ trị}, $\ldots$ đã giúp cho việc chẩn đoán \& chữa trị của bác sĩ đạt hiệu quả cao. Nhờ đó, sức khỏe của con người ngày càng tăng. Tuổi thọ trung bình của người Việt Nam vào năm 2020 là 73.7 tuổi (theo Cục thống kê).
	\item \textbf{Công nghiệp.} Vật lý là động lực của các cuộc cách mạng công nghiệp. Nhờ vậy, nền sản xuất thủ công nhỏ lẻ được chuyển thành nền sản xuất \textit{dây chuyền, tự động hóa}. Từ đố giải phóng sức lao động của con người. Hiện nay, công nghiệp sản xuất đang bước vào thời kỳ 4.0 với cốt lõi là \textit{Internet vạn vật} (IoT) \& \textit{điện toán đám mây}.
	\item \textbf{Nông nghiệp.} Việc ứng dụng những thành tựu của Vật lý đã chuyển đổi quá trình canh tác truyền thống thành các phương pháp hiện đại với năng suất vượt trội nhờ vào máy móc cơ khí tự động hóa. Ngoài ra, việc tạo ra các giống cây trồng có đặc tính ưu việc dựa vào đột biến bằng việc chiếu xạ cũng ngày càng phổ biến. Công nghệ cảm biến không dây cũng giúp cho quá trình kiểm soát chất lượng nông sản được thuận tiện \& đạt hiệu quả cao (\cite[Hình 1.6: \textsf{Công nghệ cảm biến trong việc kiểm soát chất lượng nông sản.}, p. 10]{SGK_Vat_Ly_10_Chan_Troi_Sang_Tao}).
	\item \textbf{Nghiên cứu khoa học.} Vật lý đã giúp cải tiến thiết bị \& phương pháp nghiên cứu của rất nhiều ngành khoa học. E.g.: \textit{Kính hiển vi điện tử} (\cite[Hình 1.7: \textsf{Kính hiển vi điện tử}, p. 10]{SGK_Vat_Ly_10_Chan_Troi_Sang_Tao}) phóng lớn ảnh hàng trăm nghìn lần giúp quan sát vi khuẩn, virus; \textit{nhiễu xạ tia X} giúp khám phá cấu trúc của phân tử DNA; \textit{máy quang phổ} giúp xác định cấu tạo hóa học; $\ldots$
	
	Trong chính môn Vật lý, việc tìm hiểu kiến thức vật lý cũng tạo ra những phương pháp mới, những thiết bị hiện đại, tối tân giúp các nhà nghiên cứu tìm hiểu sâu hơn về vật chất, năng lượng, vũ trụ. 1 trong những thành tựu nổi bật là \textit{kính thiên văn không gian Hubble} (HST) bay quanh Trái Đất ở độ cao hơn 600 km. Kính này đã chụp được ảnh của thiên hà cách xa Trái Đất hơn 13 tỷ năm ánh sáng \& tạo được kho dữ liệu khổng lồ về không gian \& vũ trụ.'' -- \cite[p. 10]{SGK_Vat_Ly_10_Chan_Troi_Sang_Tao}
\end{itemize}

\begin{itemize}
	\item ``Vật lý ảnh hưởng mạnh mẽ \& có tác động làm thay đổi mọi lĩnh vực hoạt động của con người. Dựa trên nền tảng vật lý, các công nghệ mới được sáng tạo với tốc độ vũ bão.
	\item Kiến thức vật lý trong các phân ngành được áp dụng kết hợp để tạo ra kết quả tối ưu. Các kỹ năng vật lý như tính chính xác, đúng thời điểm \& thời lượng, quan sát, suy luận nhạy bén, $\ldots$ đã thành kỹ năng sống cần có của con người hiện đại.'' -- \cite[p. 11]{SGK_Vat_Ly_10_Chan_Troi_Sang_Tao}
\end{itemize}
``Vào đầu thế kỷ XX, J. J. Thomson đã đề xuất mô hình cấu tạo nguyên tử gồm các electron phân bố đều trong 1 khối điện dương kết cấu tựa như khối mây. Để kiểm chứng giả thuyết này, E. Rutherford đã sử dụng tia alpha gồm các hạt mang điện dương bắn vào các nguyên tử kim loại vàng (\cite[Hình 1P.1: \textsf{Thí nghiệm Rutherford}, p. 11]{SGK_Vat_Ly_10_Chan_Troi_Sang_Tao}). Kết quả của thí nghiệm đã bác bỏ giả thuyết của J. J. Thomson, đồng thời đã giúp khám phá ra hạt nhân nguyên tử.'' -- \cite[p. 11]{SGK_Vat_Ly_10_Chan_Troi_Sang_Tao}

%------------------------------------------------------------------------------%

\section{Vấn Đề An Toàn Trong Vật Lý}

%------------------------------------------------------------------------------%

\section{Đơn Vị \& Sai Số Trong Vật Lý}

%------------------------------------------------------------------------------%

\chapter{Mô Tả Chuyển Động}

\section{Chuyển Động Thẳng}

%------------------------------------------------------------------------------%

\section{Chuyển Động Tổng Hợp}

%------------------------------------------------------------------------------%

\section{Thực Hành Đo Tốc Độ của Vật Chuyển Động Thẳng}

%------------------------------------------------------------------------------%

\chapter{Chuyển Động Biến Đổi}

\section{Gia Tốc -- Chuyển Động Thẳng Biến Đổi Đều}

%------------------------------------------------------------------------------%

\section{Thực Hành Đo Gia Tốc Rơi Tự Do}

%------------------------------------------------------------------------------%

\section{Chuyển Động Ném}

%------------------------------------------------------------------------------%

\chapter{3 Định Luật Newton. 1 Số Lực Trong Thực Tiễn}

\section{3 Định Luật Newton về Chuyển Động}

%------------------------------------------------------------------------------%

\section{1 Số Lực Trong Thực Tiễn}

%------------------------------------------------------------------------------%

\section{Chuyển Động của Vật Trong Chất Lưu}

%------------------------------------------------------------------------------%

\chapter{Moment Lực. Điều Kiện Cân Bằng}

\section{Tổng Hợp Lực -- Phân Tích Lực}

%------------------------------------------------------------------------------%

\section{Moment Lực. Điều kiện Cân Bằng của Vật}

%------------------------------------------------------------------------------%

\chapter{Năng Lượng}

\section{Năng Lượng \& Công}

%------------------------------------------------------------------------------%

\section{Công Suất -- Hiệu Suất}

%------------------------------------------------------------------------------%

\section{Động Năng \& Thế Năng. Định Luật Bảo Toàn Cơ Năng}

%------------------------------------------------------------------------------%

\chapter{Động Lượng}

\section{Động Lượng \& Định Luật Bảo Toàn Động Lượng}

%------------------------------------------------------------------------------%

\section{Các Loại Va Chạm}

%------------------------------------------------------------------------------%

\chapter{Chuyển Động Tròn}

\section{Động Học của Chuyển Động Tròn}

%------------------------------------------------------------------------------%

\section{Động Lực Học của Chuyển Động Tròn. Lực Hướng Tâm}

%------------------------------------------------------------------------------%

\chapter{Biến Dạng của Vật Rắn}

\section{Biến Dạng của Vật Rắn. Đặc Tính của Lò Xo}

%------------------------------------------------------------------------------%
R
\section{Định Luật Hooke}

%------------------------------------------------------------------------------%

\printbibliography[heading=bibintoc]
	
\end{document}