\documentclass{article}
\usepackage[backend=biber,natbib=true,style=authoryear]{biblatex}
\addbibresource{/home/nqbh/reference/bib.bib}
\usepackage[utf8]{vietnam}
\usepackage{tocloft}
\renewcommand{\cftsecleader}{\cftdotfill{\cftdotsep}}
\usepackage[colorlinks=true,linkcolor=blue,urlcolor=red,citecolor=magenta]{hyperref}
\usepackage{amsmath,amssymb,amsthm,mathtools,float,graphicx,algpseudocode,algorithm,tcolorbox,tikz,tkz-tab,subcaption}
\DeclareMathOperator{\arccot}{arccot}
\usepackage[inline]{enumitem}
\allowdisplaybreaks
\numberwithin{equation}{section}
\newtheorem{assumption}{Assumption}[section]
\newtheorem{nhanxet}{Nhận xét}[section]
\newtheorem{conjecture}{Conjecture}[section]
\newtheorem{corollary}{Corollary}[section]
\newtheorem{hequa}{Hệ quả}[section]
\newtheorem{definition}{Definition}[section]
\newtheorem{dinhnghia}{Định nghĩa}[section]
\newtheorem{example}{Example}[section]
\newtheorem{vidu}{Ví dụ}[section]
\newtheorem{lemma}{Lemma}[section]
\newtheorem{notation}{Notation}[section]
\newtheorem{principle}{Principle}[section]
\newtheorem{problem}{Problem}[section]
\newtheorem{baitoan}{Bài toán}[section]
\newtheorem{proposition}{Proposition}[section]
\newtheorem{menhde}{Mệnh đề}[section]
\newtheorem{question}{Question}[section]
\newtheorem{cauhoi}{Câu hỏi}[section]
\newtheorem{quytac}{Quy tắc}
\newtheorem{remark}{Remark}[section]
\newtheorem{luuy}{Lưu ý}[section]
\newtheorem{theorem}{Theorem}[section]
\newtheorem{tiende}{Tiên đề}[section]
\newtheorem{dinhly}{Định lý}[section]
\usepackage[left=0.5in,right=0.5in,top=1.5cm,bottom=1.5cm]{geometry}
\usepackage{fancyhdr}
\pagestyle{fancy}
\fancyhf{}
\lhead{\small Subsect.~\thesubsection}
\rhead{\small\nouppercase{\leftmark}}
\renewcommand{\subsectionmark}[1]{\markboth{#1}{}}
\cfoot{\thepage}
\def\labelitemii{$\circ$}

\title{Dynamics -- Động Lực Học}
\author{Nguyễn Quản Bá Hồng\footnote{Independent Researcher, Ben Tre City, Vietnam\\e-mail: \texttt{nguyenquanbahong@gmail.com}; website: \url{https://nqbh.github.io}.}}
\date{\today}

\begin{document}
\maketitle
\begin{abstract}
	
\end{abstract}
\setcounter{secnumdepth}{4}
\setcounter{tocdepth}{3}
\tableofcontents
\newpage

%------------------------------------------------------------------------------%

\section{\href{https://en.wikipedia.org/wiki/Dynamics_(mechanics)}{Wikipedia\texttt{/}Dynamics (Mechanics)}}
``\textit{Dynamics} is the \href{https://en.wikipedia.org/wiki/Branch_(academia)#Physics}{branch} of \href{https://en.wikipedia.org/wiki/Classical_mechanics}{classical mechanics} that is concerned with the study of \href{https://en.wikipedia.org/wiki/Force_(physics)}{force} \& their effects on \href{https://en.wikipedia.org/wiki/Motion_(physics)}{motion}. \href{https://en.wikipedia.org/wiki/Isaac_Newton}{Isaac Newton} was the 1st to formulate the fundamental \href{https://en.wikipedia.org/wiki/Physical_law}{physical laws} that govern dynamics in classical non-relativistic physics, especially his \href{https://en.wikipedia.org/wiki/Second_law_of_motion}{2nd law of motion}.'' -- \href{https://en.wikipedia.org/wiki/Dynamics_(mechanics)}{Wikipedia\texttt{/}dynammics (mechanics)}

\subsection{Principles}
``Generally speaking, researchers involved in dynamics study how a physical system might develop or alter over time \& study the causes of those changes. In addition, Newton established the fundamental physical laws which govern dynamics in physics. By studying his system of mechanics, dynamics can be understood. In particular, dynamics is mostly related to Newton's 2nd law of motion. However, all 3 laws of motion are taken into account because these are interrelated in any given observation or experiment.'' -- \href{https://en.wikipedia.org/wiki/Dynamics_(mechanics)#Principles}{Wikipedia\texttt{/}dynamics (mechanics)\texttt{/}principles}

\subsection{Linear \& rotational dynamics}
``The study of dynamics falls under 2 categories: linear \& rotational. Linear dynamics pertains to objects moving in a line \& involves such quantities as \href{https://en.wikipedia.org/wiki/Force}{force}, \href{https://en.wikipedia.org/wiki/Mass}{mass}\texttt{/}\href{https://en.wikipedia.org/wiki/Inertia#Mass_and_inertia}{inertia}, \href{https://en.wikipedia.org/wiki/Displacement_(vector)}{displacement} (in units of distance), \href{https://en.wikipedia.org/wiki/Velocity}{velocity} (distance per unit time), \href{https://en.wikipedia.org/wiki/Acceleration}{acceleration} (distance per unit of time squared) \& \href{https://en.wikipedia.org/wiki/Momentum}{momentum} (mass times unit of velocity). Rotational dynamics pertains to obejcts that are rotating or moving in a curved path \& involves such quantities as \href{https://en.wikipedia.org/wiki/Torque}{torque}, \href{https://en.wikipedia.org/wiki/Moment_of_inertia}{moment of inertia}\texttt{/}\href{https://en.wikipedia.org/wiki/Rotational_inertia}{rotational inertia}, \href{https://en.wikipedia.org/wiki/Angular_displacement}{angular displacement} (in radians or less often, degrees), \href{https://en.wikipedia.org/wiki/Angular_velocity}{angular velocity} (radians per unit time), \href{https://en.wikipedia.org/wiki/Angular_acceleration}{angular acceleration} (radians per unit of time squared) \& \href{https://en.wikipedia.org/wiki/Angular_momentum}{angular momentum} (moment of inertia times unit of angular velocity). Very often, objects exhibit linear \& rotational motion.

For classical \href{https://en.wikipedia.org/wiki/Electromagnetism}{electromagnetism}, \href{https://en.wikipedia.org/wiki/Maxwell%27s_equations}{Maxwell's equations} describe the kinematics. The dynamics of classical systems involving both mechanics \& electromagnetism are described by the combination of Newton's laws, Maxwell's equations, \& the \href{https://en.wikipedia.org/wiki/Lorentz_force}{Lorentz force}.'' -- \href{https://en.wikipedia.org/wiki/Dynamics_(mechanics)#Linear_and_rotational_dynamics}{Wikipedia\texttt{/}dynamics (mechanics)\texttt{/}linear \& rotational dynamics}

\subsection{Force}
``Main article: \href{https://en.wikipedia.org/wiki/Force}{Wikipedia\texttt{/}force}. From Newton, force can be defined as an exertion or \href{https://en.wikipedia.org/wiki/Pressure}{pressure} which can cause an object to \href{https://en.wikipedia.org/wiki/Accelerate}{accelerate}. The concept of force is used to describe an influence which causes a \href{https://en.wikipedia.org/wiki/Free_body}{free body} (object) to accelerate. It can be a push or a pull, which causes an object to change direction, have new \href{https://en.wikipedia.org/wiki/Velocity}{velocity}, or to \href{https://en.wikipedia.org/wiki/Deformation_(mechanics)}{deform} temporarily or permanently. Generally speaking, force causes an object's \href{https://en.wikipedia.org/wiki/Motion_(physics)}{state of motion} to change.'' -- \href{https://en.wikipedia.org/wiki/Dynamics_(mechanics)#Force}{Wikipedia\texttt{/}dynamics (mechanics)\texttt{/}force}

\subsection{Newton's laws}
``Main article: \href{https://en.wikipedia.org/wiki/Newton%27s_laws_of_motion}{Wikipedia\texttt{/}Newton's laws of motion}. Newton described force as the ability to cause a mass to accelerate. His 3 laws can be summarized as follows:
\begin{enumerate*}
	\item[$\bullet$] \textit{1st law}: if there is no net force on an object, then its velocity is constant: either the object is at rest (if its velocity is equal to zero), or it moves with constant speed in a single direction.
	\item[$\bullet$] \textit{2nd law}: The rate of change of linear momentum ${\bf P}$ of an object is equal to the net force ${\bf F}_{\rm net}$, i.e., $\frac{{\rm d}{\bf P}}{{\rm d}t} = {\bf F}_{\rm net}$.
	\item[$\bullet$] \textit{3rd law}: When a 1st body exerts a force ${\bf F}_1$ on a 2nd body, the 2nd body simultaneously exerts a force ${\bf F}_2 = -{\bf F}_1$ on the 1st body. I.e., ${\bf F}_1$ \& ${\bf F}_2$ are equal in magnitude \& opposite in direction.
\end{enumerate*}
Newton's laws of motion are valid only in an \href{https://en.wikipedia.org/wiki/Inertial_frame_of_reference}{inertial frame of reference}.'' -- \href{https://en.wikipedia.org/wiki/Dynamics_(mechanics)#Newton's_laws}{Wikipedia\texttt{/}dynamics (mechanics)\texttt{/}Newton's laws}

%------------------------------------------------------------------------------%

\section{Tổng Hợp Lực \& Phân Tích Lực. Cân Bằng Lực}

\subsection{Tổng Hợp Lực}

%------------------------------------------------------------------------------%

\printbibliography[heading=bibintoc]
	
\end{document}