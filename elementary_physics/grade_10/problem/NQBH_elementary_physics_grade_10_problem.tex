\documentclass{article}
\usepackage[backend=biber,natbib=true,style=authoryear]{biblatex}
\addbibresource{/home/hong/1_NQBH/reference/bib.bib}
\usepackage[utf8]{vietnam}
\usepackage{tocloft}
\renewcommand{\cftsecleader}{\cftdotfill{\cftdotsep}}
\usepackage[colorlinks=true,linkcolor=blue,urlcolor=red,citecolor=magenta]{hyperref}
\usepackage{amsmath,amssymb,amsthm,mathtools,float,graphicx,algpseudocode,algorithm,tcolorbox}
\usepackage[inline]{enumitem}
\allowdisplaybreaks
\numberwithin{equation}{section}
\newtheorem{assumption}{Assumption}[section]
\newtheorem{conjecture}{Conjecture}[section]
\newtheorem{corollary}{Corollary}[section]
\newtheorem{dangtoan}{Dạng toán}[section]
\newtheorem{hequa}{Hệ quả}[section]
\newtheorem{definition}{Definition}[section]
\newtheorem{dinhnghia}{Định nghĩa}[section]
\newtheorem{example}{Example}[section]
\newtheorem{vidu}{Ví dụ}[section]
\newtheorem{lemma}{Lemma}[section]
\newtheorem{notation}{Notation}[section]
\newtheorem{principle}{Principle}[section]
\newtheorem{problem}{Problem}[section]
\newtheorem{baitoan}{Bài toán}[section]
\newtheorem{proposition}{Proposition}[section]
\newtheorem{question}{Question}[section]
\newtheorem{cauhoi}{Câu hỏi}[section]
\newtheorem{remark}{Remark}[section]
\newtheorem{luuy}{Lưu ý}[section]
\newtheorem{theorem}{Theorem}[section]
\newtheorem{dinhly}{Định lý}[section]
\usepackage[left=0.5in,right=0.5in,top=1.5cm,bottom=1.5cm]{geometry}
\usepackage{fancyhdr}
\pagestyle{fancy}
\fancyhf{}
\lhead{\small Subsect.~\thesubsection}
\rhead{\small \nouppercase{\leftmark}}
\renewcommand{\subsectionmark}[1]{\markboth{#1}{}}
\cfoot{\thepage}
\def\labelitemii{$\circ$}

\title{Problems in Elementary Mathematics\texttt{/}Grade 11}
\author{Nguyễn Quản Bá Hồng\footnote{Independent Researcher, Ben Tre City, Vietnam\\e-mail: \texttt{nguyenquanbahong@gmail.com}; website: \url{https://nqbh.github.io}.}}
\date{\today}

\begin{document}
\maketitle
\begin{abstract}
	
\end{abstract}
\tableofcontents
\newpage

%------------------------------------------------------------------------------%

\section{Mô Tả Chuyển Động}

\subsection{Chuyển động thẳng}

\begin{dangtoan}
	Bài toán về quãng đường đi.
\end{dangtoan}

\begin{proof}[Phương pháp]
	``Chọn chiều dương là chiều chuyển động. Nếu có nhiều vật chuyển động, có thể chọn chiều dương riêng cho mỗi vật. Áp dụng phương trình $s = vt$ theo điều kiện của đề để giải quyết bài toán.'' -- \cite[p. 5]{Giai_Toan_Vat_Ly_10_tap_1}
\end{proof}

\begin{baitoan}[\cite{Giai_Toan_Vat_Ly_10_tap_1}, Thí dụ 1.1, p. 6]
	2 xe chuyển động thẳng đều trên cùng 1 đường thẳng với các vận tốc không đổi. Nếu đi ngược chiều thì sau $15$\emph{ph} khoảng cách giữa 2 xe giảm $25$\emph{km}. Nếu đi cùng chiều thì sau $15$\emph{ph} khoảng cách giữa 2 xe chỉ giảm $5$\emph{km}. Tính vận tốc của mỗi xe.
\end{baitoan}

\begin{baitoan}[\cite{Giai_Toan_Vat_Ly_10_tap_1}, Thí dụ 1.2, p. 6]
	2 xe chuyển động thẳng đều từ A đến B cách nhau $60$\emph{km}. Xe I có vận tốc $15$\emph{km\texttt{/}h} \& đi liên tục không nghỉ. Xe II khởi hành sớm hơn $1$ giờ nhưng dọc đường phải ngừng $2$ giờ. Hỏi xe II phải có vận tốc nào để tới B cùng lúc với xe I?
\end{baitoan}

\begin{baitoan}[\cite{Giai_Toan_Vat_Ly_10_tap_1}, \textbf{1.3.}, p. 7]
	Năm 1946 người ta đo khoảng cách Trái Đất--Mặt Trăng bằng kỹ thuật phản xạ sóng radar. Tín hiệu radar phát đi từ Trái Đất truyền với vận tốc $c = 3\cdot 10^8$\emph{m\texttt{/}s} phản xạ trên bề mặt của Mặt Trăng \& trở lại Trái Đất. Tín hiệu phản xạ được ghi nhận sau $2.5$\emph{s} kể từ lúc truyền. Coi Trái Đất \& Mặt Trăng có dạng hình cầu bán kính lần lượt là $R_{\rm Earth} = 6400$\emph{km} \& $R_{\rm Moon} = 1740$\emph{km}. Tính khoảng cách $d$ giữa 2 tâm.
\end{baitoan}
``Nhờ các thiết bị phản xạ tia laser do các phi hành gia đặt trên Mặt Trăng, ngày nay dùng tia laser, người ta đo được khoảng cách này với độ chính xác tới cm.'' -- \cite[p. 7]{Giai_Toan_Vat_Ly_10_tap_1}

\begin{baitoan}[\cite{Giai_Toan_Vat_Ly_10_tap_1}, \textbf{1.4.}, pp. 7--8]
	1 canô rời bến chuyển động thẳng đều. Thoạt tiên, canô chạy theo hướng Nam--Bắc trong thời gian $\rm 2m40s$ rồi tức thì rẽ sang hướng Đông--Tây \& chạy thêm $2$\emph{ph} với vận tốc như trước \& dừng lại. Khoảng cách từ nơi xuất phát tới nơi dừng là $1$\emph{km}. Tính vận tốc của canô.
\end{baitoan}

\begin{baitoan}[\cite{Giai_Toan_Vat_Ly_10_tap_1}, \textbf{1.5.}, p. 8]
	1 người đứng tại A trên 1 bờ hồ. Người này muốn tới B trên mặt hồ nhanh nhất. Cho các khoảng cách như trên hình vẽ. Biết người này có thể chạy thẳng dọc theo bờ hồ với vận tốc $v_1$ \& bơi thẳng với vận tốc $v_2$. Xác định cách mà người này phải theo:
	\begin{enumerate*}
		\item[$\bullet$] hoặc bơi thẳng từ A đến B.
		\item[$\bullet$] hoặc chạy dọc theo bờ hồ 1 đoạn rồi sau đó bơi thẳng tới B.
	\end{enumerate*}
	
	\begin{figure}[h]
		\centering
		\includegraphics[scale=0.1]{bo_ho}
	\end{figure}
\end{baitoan}

\begin{baitoan}[\cite{Giai_Toan_Vat_Ly_10_tap_1}, \textbf{1.6.}, p. 8]
	2 tàu A \& B cách nhau 1 khoảng cách $a$ đồng thời chuyển động thẳng đều với cùng độ lớn $v$ của vận tốc từ 2 nơi trên 1 bờ hồ thẳng. Tàu A chuyển động theo hướng vuông góc với bờ trong khi tàu B luôn luôn hướng về tàu A. Sau 1 thời gian đủ lâu, tàu B \& tàu A chuyển động trên cùng 1 đường thẳng nhưng cách nhau 1 khoảng không đổi. Tính khoảng cách này.
\end{baitoan}

\begin{dangtoan}
	Định thời điểm \& vị trí gặp nhau của các vật chuyển động.
\end{dangtoan}

\begin{proof}[Phương pháp]
	``Chọn chiều dương, gốc tọa độ, gốc thời gian. Suy ra vận tốc các vật \& điều kiện ban đầu. Áp dụng phương trình tổng quát để lập phương trình chuyển động của mỗi vật: $x = v(t - t_0) + x_0$. Khi 2 vật gặp nhau, tọa độ của 2 vật bằng nhau: $x_1 = x_2$. Giải phương trình này để tìm thời gian \& tọa độ gặp nhau.'' -- \cite[p. 9]{Giai_Toan_Vat_Ly_10_tap_1}
\end{proof}

\begin{baitoan}[\cite{Giai_Toan_Vat_Ly_10_tap_1}, Thí dụ 2.1, p. 9]
	Lúc 6:00 1 người đi xe đạp đuổi theo 1 người đi bộ đã đi được $8$\emph{km}. Cả 2 chuyển động thẳng đều với các vận tốc $12$\emph{km\texttt{/}h} \& $4$\emph{km\texttt{/}h}. Tìm vị trí \& thời gian người đi xe đạp đuổi kịp người đi bộ.
\end{baitoan}

\begin{baitoan}[\cite{Giai_Toan_Vat_Ly_10_tap_1}, Thí dụ 2.2, p. 10]
	\label{thi du 2.2}
	2 ôtô chuyển động thẳng đều hướng về nhau với các vận tốc $40$\emph{km\texttt{/}h} \& $60$\emph{km\texttt{/}h}. Lúc 7:00, 2 xe cách nhau $150$\emph{km}. Hỏi 2 ôtô sẽ gặp nhau lúc mấy giờ, ở đâu?
\end{baitoan}

\begin{baitoan}[\cite{Giai_Toan_Vat_Ly_10_tap_1}, \textbf{2.3}, p. 11]
	1 xe khởi hành từ A lúc $9$h để về B theo chuyển động thẳng đều với vận tốc $36$\emph{km\texttt{/}h}. Nửa giờ sau, 1 xe đi từ B về A với vận tốc $54$\emph{km\texttt{/}h}. Cho $AB = 108$\emph{km}. Xác định lúc \& nơi 2 xe gặp nhau.
\end{baitoan}

\begin{baitoan}[\cite{Giai_Toan_Vat_Ly_10_tap_1}, \textbf{2.4}, p. 11]
	Lúc 7:00 có 1 xe khởi thành từ A chuyển động về B theo chuyển động thẳng đều với vận tốc $40$\emph{km\texttt{/}h}. Lúc 7:30 1 xe khác khởi hành từ B đi về A theo chuyển động thẳng đều với vận tốc $50$\emph{km\texttt{/}h}. Cho $AB = 110$\emph{km}.
	\begin{enumerate*}
		\item[(a)] Xác định vị trí của mỗi xe \& khoảng cách giữa chúng lúc 8:00 \& lúc 9:00.
		\item[(b)] 2 xe gặp nhau lúc mấy giờ, ở đâu?
	\end{enumerate*}
\end{baitoan}

\begin{baitoan}[\cite{Giai_Toan_Vat_Ly_10_tap_1}, \textbf{2.5}, p. 11]
	Lúc 8:00 1 người đi xe đạp với vận tốc đều $12$\emph{km\texttt{/}h} gặp 1 người đi bộ đi ngược chiều với vận tốc đều $4$\emph{km\texttt{/}h} trên cùng đoạn đường thẳng. Tới 8:30 người đi xe đạp dừng lại, nghỉ $30$\emph{ph} rồi quay trở lại đuổi theo người đi bộ với vận tốc có độ lớn như trước. Xác định lúc \& nơi người đi xe đạp đuổi kịp người đi bộ.
\end{baitoan}

\begin{dangtoan}
	Vẽ đồ thị của chuyển động. Dùng đồ thị để giải bài toán về chuyển động.
\end{dangtoan}

\begin{proof}[Phương pháp]
	Vẽ đồ thị của chuyển động:
	\begin{enumerate*}
		\item[$\bullet$] Dựa vào phương trình, xác định 2 điểm của đồ thị. Lưu ý giới hạn.
		\item[$\bullet$] Xác định điểm biểu diễn điều kiện ban đầu \& vẽ đường thẳng có độ dốc bằng vận tốc.
	\end{enumerate*}
	
	Đặc điểm của chuyển động theo đồ thị:
	\begin{enumerate*}
		\item[$\bullet$] \textit{Đồ thị hướng lên}: $v > 0$ (vật chuyển động theo chiều dương). \textit{Đồ thị hướng xuống}: $v < 0$ (vật chuyển động ngược chiều dương).
		\item[$\bullet$] \textit{2 đồ thị song song}: 2 vật có cùng vận tốc.
		\item[$\bullet$] \textit{2 đồ thị cắt nhau}: giao điểm cho biết lúc \& nơi 2 vật gặp nhau.
		\item[$\bullet$] Đồ thị của 2 chuyển động xác định trên trục $x$ \& trục $t$ khoảng cách \& khoảng chênh lệch thời gian của 2 chuyển động.
	\end{enumerate*}
\end{proof}

\begin{baitoan}[\cite{Giai_Toan_Vat_Ly_10_tap_1}, Thí dụ 3.1, p. 13]
	1 vật chuyển động có đồ thị tọa độ theo thời gian như hình vẽ. Suy ra các thông tin của chuyển động trình bày trên đồ thị.
	
	\begin{figure}[h]
		\centering
		\includegraphics[scale=0.1]{thi_du_3_1}
	\end{figure}
\end{baitoan}

\begin{baitoan}[\cite{Giai_Toan_Vat_Ly_10_tap_1}, Thí dụ 3.2, p. 13]
	Giải lại bài tập \ref{thi du 2.2} bằng phương pháp đồ thị.
\end{baitoan}

\begin{baitoan}[\cite{Giai_Toan_Vat_Ly_10_tap_1}, Thí dụ 3.3, p. 14]
	Lúc 9:00 1 ôtô khởi hành từ TP Hồ Chí Minh chạy về hướng Long An với vận tốc đều $60$\emph{km\texttt{/}h}. Sau khi đi được $45$\emph{ph}, xe dừng $15$\emph{ph} rồi tiếp tục chạy với vận tốc đều như lúc đầu. Lúc 9:30 1 ôtô thứ 2 khởi hành từ TP Hồ Chí Minh đuổi theo xe thứ nhất. Xe thứ 2 có vận tốc đều $70$\emph{km\texttt{/}h}.
	\begin{enumerate*}
		\item[(a)] Vẽ đồ thị tọa độ theo thời gian của mỗi xe.
		\item[(b)] Xác định nơi \& lúc xe sau đuổi kịp xe đầu.
	\end{enumerate*}
\end{baitoan}

\begin{baitoan}[\cite{Giai_Toan_Vat_Ly_10_tap_1}, Thí dụ 3.4, p. 15]
	Giữa 2 bến sông cách nhau $20$\emph{km} theo đường thẳng có 1 đoàn ghe máy phục vụ chở khách. Khi xuôi dòng từ A đến B vận tốc ghe là $20$\emph{km\texttt{/}h}; khi ngược dòng từ B về A vận tốc ghe là $10$\emph{km\texttt{/}h}. Ở mỗi bến cứ $20$\emph{ph} lại có 1 ghe xuất phát. Khi tới bến mỗi ghe ngừng $20$\emph{ph} rồi quay về.
	\begin{enumerate*}
		\item[(a)] Cần bao nhiêu ghe cho đoạn sông?
		\item[(b)] 1 ghe khi đi từ A đến B gặp bao nhiêu ghe? Khi đi từ B về A gặp bao nhiêu ghe?
	\end{enumerate*}
\end{baitoan}

\begin{baitoan}[\cite{Giai_Toan_Vat_Ly_10_tap_1}, \textbf{3.5}, p. 17]
	Chuyển động của 3 xe (1), (2), (3) có các đồ thị tọa độ--thời gian như hình bên.
	\begin{enumerate*}
		\item[(a)] Nêu đặc điểm chuyển động của mỗi xe.
		\item[(b)] Lập phương trình chuyển động của mỗi xe.
		\item[(c)] Xác định vị trí \& thời điểm gặp nhau bằng đồ thị. Kiểm tra lại bằng phép tính.
	\end{enumerate*}

	\begin{figure}[h]
		\centering
		\includegraphics[scale=0.1]{3_5}
	\end{figure}
\end{baitoan}

\begin{baitoan}[\cite{Giai_Toan_Vat_Ly_10_tap_1}, \textbf{3.6}, p. 17]
	Giữa 2 bến sông A, B có 2 tàu chuyển thử chạy thẳng đều. Tàu đi từ A chạy xuôi dòng, tàu đi từ B ngược dòng. Khi gặp nhau \& chuyển thư, mỗi tàu tức thì trở lại bến xuất phát. Nếu khởi hành cùng lúc thì tàu từ A đi \& về mất $3$\emph{h}, tàu từ B đi \& về mất $\rm 1h30ph$. Hỏi nếu muốn thời gian đi \& về của 2 tàu bằng nhau thì tàu từ A phải khởi hành trễ hơn tàu từ B bao lâu? Cho biết:
	\begin{enumerate*}
		\item[$\bullet$] Vận tốc mỗi tàu dối với nước như nhau \& không đổi lúc đi cũng như lúc về.
		\item[$\bullet$] Khi xuôi dòng, vận tốc dòng nước làm tàu chạy nhanh hơn; khi ngược dòng, vận tốc dòng nước làm tàu chạy chậm hơn.
		\item[(a)] Giải bài toán bằng đồ thị.
		\item[(b)] Giải bài toán bằng phương trình.
	\end{enumerate*}
\end{baitoan}

\begin{baitoan}[\cite{Giai_Toan_Vat_Ly_10_tap_1}, \textbf{3.7}, p. 18]
	Hằng ngày có 1 xe hơi đi từ nhà máy tới đón 1 kỹ sư tại trạm đến nhà máy làm việc. 1 hôm, viên kỹ sư tới trạm sớm hơn $1$\emph{h} nên anh đi bộ hướng về nhà máy. Dọc đường anh ta gặp chiếc xe tới đón mình \& cả 2 tới nhà máy sớm hơn bình thường $10$\emph{ph}. Coi các chuyển động là thẳng đều có độ lớn vận tốc nhất định. Tính thời gian mà viên kỹ sư đã đi bộ từ trạm tới khi gặp xe.
\end{baitoan}

\begin{baitoan}[\cite{Giai_Toan_Vat_Ly_10_tap_1}, \textbf{3.8}, p. 18]
	3 người đang ở cùng 1 nơi \& muốn cùng có mặt tại 1 sân vận động cách đó $48$\emph{km}. Đường đi thẳng. Họ có 1 chiếc xe đạp chỉ có thể chở thêm 1 người. 3 người giải quyết bằng cách 2 người đi xe đạp khởi hành cùng lúc với người đi bộ, tới 1 vị trí thích hợp, người được chở bằng xe đạp xuống xe đi bộ tiếp, người đi xe đạp quay về gặp người đi bộ từ đầu \& chở người này quay ngược trở lại. 3 người đến sân vận động cùng lúc.
	\begin{enumerate*}
		\item[(a)] Vẽ đồ thị của các chuyển động. Coi các chuyển động là thẳng đều \& vận tốc có độ lớn không đổi là $12$\emph{km\texttt{/}h} khi đi xe đạp \& $4$\emph{km\texttt{/}h} khi đi bộ.
		\item[(b)] Tính sự phân bố thời gian \& quãng đường.
	\end{enumerate*}	
\end{baitoan}

\begin{dangtoan}
	Đổi hệ quy chiếu để nghiên cứu chuyển động thẳng đều.
\end{dangtoan}

\begin{proof}[Phương pháp]
	Chọn hệ quy chiếu thích hợp. Áp dụng công thức cộng vận tốc để xác định vận tốc của vật trong hệ quy chiếu đã chọn.
	\begin{enumerate*}
		\item[$\bullet$] \textit{Nếu chuyển động cùng phương}: các vận tốc cộng vào nhau hay trừ đi nhau.
		\item[$\bullet$] \textit{Nếu chuyển động khác phương}: dựa vào giản đồ vector \& các tính chất hình học hay lượng giác.
	\end{enumerate*}
	Lập các phương trình theo đề bài để tìm ẩn của bài toán.
\end{proof}

\begin{baitoan}[\cite{Giai_Toan_Vat_Ly_10_tap_1}, Thí dụ 4.1, p. 19]
	1 hành khách trên toa xe lửa chuyển động thẳng đều với vận tốc $54$\emph{km\texttt{/}h} quan sát qua khe cửa thấy 1 đoàn tàu khác chạy cùng phương cùng chiều trên đường sắt bên cạnh. Từ lúc nhìn thấy điểm cuối đến lúc nhìn thấy điểm đầu của đoàn tàu mất $8$\emph{s}. Đoàn tàu mà người này quan sát gồm $20$ toa, mỗi toa dài $4$\emph{m}. Tính vận tốc của nó. (Coi các toa sát nhau).
\end{baitoan}

\begin{baitoan}[\cite{Giai_Toan_Vat_Ly_10_tap_1}, Thí dụ 4.2, p. 20]
	1 đoàn xe cơ giới có đội hình dài $1500$\emph{m} hành quân với vận tooc $40$\emph{km\texttt{/}h}. Người chỉ huy ở xe đầu trao cho 1 chiến sĩ đi mô tô 1 mệnh lệnh chuyển xuống xe cuối. Chiến sĩ ấy đi \& về với cùng 1 vận tốc \& hoàn thành nhiệm vụ trở về báo cáo mất 1 thời gian $\rm 5ph4s$. Tính vận tốc của chiến sĩ đi mô tô.
\end{baitoan}

\begin{baitoan}[\cite{Giai_Toan_Vat_Ly_10_tap_1}, Thí dụ 4.3, p. 21]
	1 chiếc tàu chuyển động thẳng đều với vận tốc $v_1 = 30$\emph{km\texttt{/}h} gặp 1 đoàn xà lan dài $l = 250$\emph{m} đi ngược chiều với vận tốc $v_2 = 15$\emph{km\texttt{/}h}. Trên boong tàu có 1 người đi từ mũi đến lái với vận tốc $v_3 = 5$\emph{km\texttt{/}h}. Hỏi người ấy thấy đoàn xà lan qua trước mặt mình trong bao lâu?
\end{baitoan}

\begin{baitoan}[\cite{Giai_Toan_Vat_Ly_10_tap_1}, Thí dụ 4.4, p. 22]
	2 xe ôtô chạy trên 2 đường thẳng vuông góc với nhau, sau khi gặp nhau ở ngã 4, 1 xe chạy sang phía đông, xe kia chạy lên phía bắc với cùng vận tốc $40$\emph{km\texttt{/}h}.
	\begin{enumerate*}
		\item[(a)] Tính vận tốc tương đối của xe thứ nhất so với xe thứ 2.
		\item[(b)] Ngồi trên xe thứ 2 quan sát thì thấy xe thứ nhất chạy theo hướng nào?
		\item[(c)] Tính khoảng cách 2 xe sau nửa giờ kể từ khi gặp nhau ở ngã 4.
	\end{enumerate*}
\end{baitoan}

\begin{baitoan}[\cite{Giai_Toan_Vat_Ly_10_tap_1}, Thí dụ 4.5, p. 23]
	Ôtô chuyển động thẳng đều với vận tốc $v_1 = 54$\emph{km\texttt{/}h}. 1 hành khách cách ôtô đoạn $a = 400$\emph{m} \& cách đường đoạn $d = 80$\emph{m}, muốn đón ôtô. Hỏi người ấy phải chạy theo hướng nào với vận tốc nhỏ nhất là bao nhiêu để đón được ôtô?
\end{baitoan}

\begin{baitoan}[\cite{Giai_Toan_Vat_Ly_10_tap_1}, Thí dụ 4.6, p. 24]
	Ngồi trên 1 toa xe lửa đang chuyển động thẳng đều với vận tốc $17.32$\emph{m\texttt{/}s}, 1 hành khác thấy các giọt mưa vạch trên cửa kính những đường thẳng nghiêng $30^\circ$ so với phương thẳng đứng. Tính vận tốc rơi của các giọt mưa (coi là rơi thẳng đều theo hướng thẳng đứng).
\end{baitoan}

\begin{baitoan}[\cite{Giai_Toan_Vat_Ly_10_tap_1}, \textbf{4.7}, p. 25]
	Trên 1 tuyến xe bus các xe coi như chuyển động thẳng đều với vận tốc $30$\emph{km\texttt{/}h}; 2 chuyến xe liên tiếp khởi hành cách nhau $10$\emph{ph}. 1 người đi xe đạp ngược lại gặp 2 chuyến xe bus liên tiếp cách nhau $\rm 7ph30s$. Tính vận tốc người đi xe đạp.
\end{baitoan}

\begin{baitoan}[\cite{Giai_Toan_Vat_Ly_10_tap_1}, \textbf{4.8}, p. 25]
	1 chiếc phà chạy xuôi dòng từ A đến B mất $3$\emph{h}; khi chạy về mất $6$\emph{h}. Hỏi nếu phà tắt máy trôi theo dòng nước thì từ A đến B mất bao lâu?
\end{baitoan}

\begin{baitoan}[\cite{Giai_Toan_Vat_Ly_10_tap_1}, \textbf{4.9}, p. 25]
	1 thang cuốn tự động đưa khách từ tầng trệt lên lầu trong $1$\emph{ph}. Nếu thang ngừng thì khách phải đi bộ lên trong $3$\emph{ph}. Hỏi nếu thang chạy mà khách vẫn bước lên thì mất bao lâu?
\end{baitoan}

\begin{baitoan}[\cite{Giai_Toan_Vat_Ly_10_tap_1}, \textbf{4.10}, pp. 25--26]
	1 tàu ngầm đang lặn xuống theo phương thẳng đứng với vận tốc đều $v$. Để dò đáy biển, máy SONAR trên tàu phát 1 tín hiệu âm kéo dài trong thời gian $t_0$ hướng xuống đáy biển. Âm truyền trong nước với vận tốc đều $u$, phản xạ ở đáy biển (coi như nằm ngang) \& truyền trở lại tàu. Tàu thu được tín hiệu âm phản xạ trong thời gian $t$. Tính vận tốc lặn của tàu.
\end{baitoan}

\begin{baitoan}[\cite{Giai_Toan_Vat_Ly_10_tap_1}, \textbf{4.11}, p. 26]
	1 thuyền máy chuyển động thẳng đều \textit{ngược dòng} gặp 1 bè trôi xuôi dòng. Sau khi gặp nhau $1$\emph{h}, động cơ của thuyền bị hỏng \& phải sửa mất $30$\emph{ph}. Trong thời gian sửa, thuyền máy trôi xuôi dòng. Sau khi sửa xong động cơ, thuyền máy chuyển động thẳng đều xuôi dòng với vận tốc so với nước như trước. Thuyền máy gặp bè cách nơi gặp lần trước $7.5$\emph{km}. Tính vận tốc chảy của nước coi là không đổi.
\end{baitoan}

%------------------------------------------------------------------------------%

\subsection{Chuyển động tổng hợp}

%------------------------------------------------------------------------------%

\section{Chuyển Động Biến Đổi}

\subsection{Gia tốc -- Chuyển động thẳng biến đổi đều}

%------------------------------------------------------------------------------%

\subsection{Chuyển động ném}

%------------------------------------------------------------------------------%

\section{3 Định Luật Newton. 1 Số Lực Trong Thực Tiễn}

\subsection{3 định luật Newton về chuyển động}

%------------------------------------------------------------------------------%

\subsection{1 số lực trong thực tiễn}

%------------------------------------------------------------------------------%

\subsection{Chuyển động của vật trong chất lưu}

%------------------------------------------------------------------------------%

\section{Moment Lực. Điều Kiện Cân Bằng}

\subsection{Tổng hợp lực -- Phân tích lực}

%------------------------------------------------------------------------------%

\subsection{Moment lực. Điều kiện cân bằng của vật}

%------------------------------------------------------------------------------%

\section{Năng Lượng}

\subsection{Năng lượng \& công}

%------------------------------------------------------------------------------%

\subsection{Công suất -- Hiệu suất}

%------------------------------------------------------------------------------%

\subsection{Động năng \& thế năng. Định luật bảo toàn cơ năng}

%------------------------------------------------------------------------------%

\section{Động Lượng}

\subsection{Động lượng \& định luật bảo toàn động lượng}

%------------------------------------------------------------------------------%

\subsection{Các loại va chạm}

%------------------------------------------------------------------------------%

\section{Chuyển Động Tròn}

\subsection{Động học của chuyển động tròn}

%------------------------------------------------------------------------------%

\subsection{Động lực học của chuyển động tròn. Lực hướng tâm}

%------------------------------------------------------------------------------%

\section{Biến Dạng của Vật Rắn}

\subsection{Biến dạng của vật rắn. Đặc tính của lò xo}

%------------------------------------------------------------------------------%

\subsection{Định luật Hooke}

%------------------------------------------------------------------------------%

\printbibliography[heading=bibintoc]
	
\end{document}