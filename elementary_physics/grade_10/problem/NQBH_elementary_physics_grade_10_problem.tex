\documentclass{article}
\usepackage[backend=biber,natbib=true,style=authoryear]{biblatex}
\addbibresource{/home/hong/1_NQBH/reference/bib.bib}
\usepackage[utf8]{vietnam}
\usepackage{tocloft}
\renewcommand{\cftsecleader}{\cftdotfill{\cftdotsep}}
\usepackage[colorlinks=true,linkcolor=blue,urlcolor=red,citecolor=magenta]{hyperref}
\usepackage{amsmath,amssymb,amsthm,mathtools,float,graphicx,algpseudocode,algorithm,tcolorbox}
\usepackage[inline]{enumitem}
\allowdisplaybreaks
\numberwithin{equation}{section}
\newtheorem{assumption}{Assumption}[section]
\newtheorem{conjecture}{Conjecture}[section]
\newtheorem{corollary}{Corollary}[section]
\newtheorem{hequa}{Hệ quả}[section]
\newtheorem{definition}{Definition}[section]
\newtheorem{dinhnghia}{Định nghĩa}[section]
\newtheorem{example}{Example}[section]
\newtheorem{vidu}{Ví dụ}[section]
\newtheorem{lemma}{Lemma}[section]
\newtheorem{notation}{Notation}[section]
\newtheorem{principle}{Principle}[section]
\newtheorem{problem}{Problem}[section]
\newtheorem{baitoan}{Bài toán}[section]
\newtheorem{proposition}{Proposition}[section]
\newtheorem{question}{Question}[section]
\newtheorem{cauhoi}{Câu hỏi}[section]
\newtheorem{remark}{Remark}[section]
\newtheorem{luuy}{Lưu ý}[section]
\newtheorem{theorem}{Theorem}[section]
\newtheorem{dinhly}{Định lý}[section]
\usepackage[left=0.5in,right=0.5in,top=1.5cm,bottom=1.5cm]{geometry}
\usepackage{fancyhdr}
\pagestyle{fancy}
\fancyhf{}
\lhead{\small \textsc{Sect.} ~\thesection}
\rhead{\small \nouppercase{\leftmark}}
\renewcommand{\sectionmark}[1]{\markboth{#1}{}}
\cfoot{\thepage}
\def\labelitemii{$\circ$}

\title{Problems in Elementary Mathematics\texttt{/}Grade 11}
\author{Nguyễn Quản Bá Hồng\footnote{Independent Researcher, Ben Tre City, Vietnam\\e-mail: \texttt{nguyenquanbahong@gmail.com}; website: \url{https://nqbh.github.io}.}}
\date{\today}

\begin{document}
\maketitle
\begin{abstract}
	
\end{abstract}
\tableofcontents
\newpage

%------------------------------------------------------------------------------%

\section{Mô Tả Chuyển Động}

\subsection{Chuyển động thẳng}

\subsection{Chuyển động tổng hợp}

%------------------------------------------------------------------------------%

\section{Chuyển Động Biến Đổi}

\subsection{Gia tốc -- Chuyển động thẳng biến đổi đều}

\subsection{Chuyển động ném}

%------------------------------------------------------------------------------%

\section{3 Định Luật Newton. 1 Số Lực Trong Thực Tiễn}

\subsection{3 định luật Newton về chuyển động}

\subsection{1 số lực trong thực tiễn}

\subsection{Chuyển động của vật trong chất lưu}

%------------------------------------------------------------------------------%

\section{Moment Lực. Điều Kiện Cân Bằng}

\subsection{Tổng hợp lực -- Phân tích lực}

\subsection{Moment lực. Điều kiện cân bằng của vật}

%------------------------------------------------------------------------------%

\section{Năng Lượng}

\subsection{Năng lượng \& công}

\subsection{Công suất -- Hiệu suất}

\subsection{Động năng \& thế năng. Định luật bảo toàn cơ năng}

%------------------------------------------------------------------------------%

\section{Động Lượng}

\subsection{Động lượng \& định luật bảo toàn động lượng}

\subsection{Các loại va chạm}

%------------------------------------------------------------------------------%

\section{Chuyển Động Tròn}

\subsection{Động học của chuyển động tròn}

\subsection{Động lực học của chuyển động tròn. Lực hướng tâm}

%------------------------------------------------------------------------------%

\section{Biến Dạng của Vật Rắn}

\subsection{Biến dạng của vật rắn. Đặc tính của lò xo}

\subsection{Định luật Hooke}

%------------------------------------------------------------------------------%

\printbibliography[heading=bibintoc]
	
\end{document}