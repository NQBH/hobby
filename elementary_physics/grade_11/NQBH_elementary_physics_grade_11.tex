\documentclass[oneside]{book}
\usepackage[backend=biber,natbib=true,style=authoryear]{biblatex}
\addbibresource{/home/hong/1_NQBH/reference/bib.bib}
\usepackage[utf8]{vietnam}
\usepackage{tocloft}
\renewcommand{\cftsecleader}{\cftdotfill{\cftdotsep}}
\usepackage[colorlinks=true,linkcolor=blue,urlcolor=red,citecolor=magenta]{hyperref}
\usepackage{amsmath,amssymb,amsthm,mathtools,float,graphicx,algpseudocode,algorithm,tcolorbox,tikz,tkz-tab,subcaption}
\usepackage[inline]{enumitem}
\allowdisplaybreaks
\numberwithin{equation}{section}
\newtheorem{assumption}{Assumption}[section]
\newtheorem{nhanxet}{Nhận xét}[section]
\newtheorem{conjecture}{Conjecture}[section]
\newtheorem{corollary}{Corollary}[section]
\newtheorem{hequa}{Hệ quả}[section]
\newtheorem{definition}{Definition}[section]
\newtheorem{dinhnghia}{Định nghĩa}[section]
\newtheorem{dinhluat}{Định luật}[section]
\newtheorem{example}{Example}[section]
\newtheorem{vidu}{Ví dụ}[section]
\newtheorem{lemma}{Lemma}[section]
\newtheorem{notation}{Notation}[section]
\newtheorem{principle}{Principle}[section]
\newtheorem{problem}{Problem}[section]
\newtheorem{baitoan}{Bài toán}[section]
\newtheorem{proposition}{Proposition}[section]
\newtheorem{menhde}{Mệnh đề}[section]
\newtheorem{nguyenly}{Nguyên lý}[section]
\newtheorem{question}{Question}[section]
\newtheorem{cauhoi}{Câu hỏi}[section]
\newtheorem{remark}{Remark}[section]
\newtheorem{luuy}{Lưu ý}[section]
\newtheorem{theorem}{Theorem}[section]
\newtheorem{dinhly}{Định lý}[section]
\usepackage[left=0.5in,right=0.5in,top=1.5cm,bottom=1.5cm]{geometry}
\usepackage{fancyhdr}
\pagestyle{fancy}
\fancyhf{}
\lhead{\small \textsc{Sect.} ~\thesection}
\rhead{\small \nouppercase{\leftmark}}
\renewcommand{\sectionmark}[1]{\markboth{#1}{}}
\cfoot{\thepage}
\def\labelitemii{$\circ$}

\makeatletter
\let\old@endpart\@endpart
\renewcommand\@endpart[1][]{%
	\begin{quote}#1\end{quote}%
	\old@endpart}
\makeatother

\title{Some Topics in Elementary Physics\texttt{/}Grade 11}
\author{Nguyễn Quản Bá Hồng\footnote{Independent Researcher, Ben Tre City, Vietnam\\e-mail: \texttt{nguyenquanbahong@gmail.com}; website: \url{https://nqbh.github.io}.}}
\date{\today}

\begin{document}
\frontmatter
\maketitle
\setcounter{secnumdepth}{4}
\setcounter{tocdepth}{3}
\tableofcontents
\newpage

%------------------------------------------------------------------------------%

\mainmatter
\part{Điện Học -- Điện Từ Học}
[``Phần Điện học -- Điện từ học đề cập đến các hiện tượng liên quan đến tương tác giữa các điện tích đứng yên \& chuyển động, gọi chung là \textit{hiện tượng điện từ} \& các quy luật chi phối các hiện tượng này. Các hiện tượng điện từ rất phổ biến trong tự nhiên, rất phong phú \& đa dạng. Chúng được ứng dụng rộng rãi trong khoa học \& kỹ thuật, cũng như trong cuộc sống.'' -- \cite{SGK_Vat_Ly_11_nang_cao}, p. 3]

\chapter{Điện Tích -- Điện Trường}

\begin{quotation}
	\textbf{Nội dung.} \textit{Định luật tương tác giữa các điện tích điểm (định luật Coulomb), điện trường, cường độ điện trường của điện tích điểm, hiệu điện thế, điện thế \& công của lực điện, năng lượng điện trường, tụ điện, ghép tụ điện}.
\end{quotation}

\section{Điện Tích. Định Luật Coulomb}
\begin{quotation}
	\textbf{Nội dung.} \textit{1 số khái niệm mở đầu về điện tích (điện tích dương, điện tích âm, sự nhiễm điện của các vật) \& về định luật tương tác giữa 2 điện tích}.
\end{quotation}

\subsection{2 loại điện tích. Sự nhiễm điện của các vật}

\subsubsection{2 loại điện tích}
``Có 2 loại điện tích: điện tích dương, điện tích âm. Các điện tích cùng dấu thì đẩy nhau, các điện tích khác dấu thì hút nhau. Đơn vị điện tích là coulomb\footnote{\textsc{Charles Coulomb} (1736--1806), nhà vật lý người Pháp. Có thể đọc thêm \href{https://vi.wikipedia.org/wiki/Charles-Augustin_de_Coulomb}{Wikipedia\texttt{/}Charles-Augustin de Coulomb
} \& \href{https://en.wikipedia.org/wiki/Charles-Augustin_de_Coulomb}{Charles-Augustin de Coulomb}.}, ký hiệu là C. Điện tích của electron là điện tích âm \& có độ lớn $e = 1,6\cdot 10^{-19}$ C. 1 điện tích $e = 1,6\cdot 10^{-19}$ C được gọi là \textit{điện tích nguyên tố}. Thí nghiệm đã chứng tỏ rằng, trong tự nhiên không có hạt nào có điện tích nhỏ hơn điện tích nguyên tố. Độ lớn của điện tích 1 hạt bao giờ cũng bằng 1 số nguyên lần điện tích nguyên tố.

Dựa vào sự tương tác giữa các điện tích cùng dấu người ta chế tạo ra điện nghiệm.

\begin{figure}[H]
	\centering
	\includegraphics[scale=0.15]{dien_nghiem}
	\caption{Điện nghiệm. 1. Bình thủy tinh; 2. Nút cách điện; 3. Nút kim loại; 4. Thanh kim loại; 5. 2 lá kim loại nhẹ. \cite[Hình 1.1, p. 6]{SGK_Vat_Ly_11_nang_cao}}
\end{figure}
Điện nghiệm dùng để phát hiện điện tích ở 1 vật. Khi 1 vật nhiễm điên chạm vào núm kim loại, thì điện tích truyền đến 2 lá kim loại (nhiễm điện do tiếp xúc). Do đó, 2 lá kim loại đẩy nhau \& xòe ra.'' -- \cite[p. 6]{SGK_Vat_Ly_11_nang_cao}

\subsubsection{Sự nhiễm điện của các vật}

\paragraph{Nhiễm điện do cọ xát.} ``Sau khi cọ xát vào lụa, thanh thủy tinh có thể hút được các mẩu giấy vụn (\cite[Hình 1.2: \textsf{Thanh thủy tin nhiễm điện hút các mẩu giấy}, p. 6]{SGK_Vat_Ly_11_nang_cao}). Người ta nói thanh thủy tinh được \textit{nhiễm điện do cọ xát}.'' -- \cite[p. 6]{SGK_Vat_Ly_11_nang_cao}

\paragraph{Nhiễm điện do tiếp xúc.} ``Cho thanh kim loại không nhiễm điện chạm vào quả cầu đã nhiễm điện thì thanh kim loại nhiễm điện cùng dấu với điện tích của quả cầu (Fig. \ref{fig:nhiem dien do tiep xuc}). Người ta nói thanh kim loại được \textit{nhiễm điện do tiếp xúc}. Đưa thanh kim loại ra xa quả cầu thì thanh kim loại vẫn nhiễm điện.'' -- \cite[p. 7]{SGK_Vat_Ly_11_nang_cao}

\begin{figure}[H]
	\centering
	\includegraphics[scale=0.15]{nhiem_dien_do_tiep_xuc}
	\caption{Nhiễm điện do tiếp xúc, \cite[Hình 1.3, p. 7]{SGK_Vat_Ly_11_nang_cao}.}
	\label{fig:nhiem dien do tiep xuc}
\end{figure}

\paragraph{Nhiễm điện do hưởng ứng.} ``Đưa thanh kim loại không nhiễm điện đến gần quả cầu đã nhiễm điện nhưng không chạm vào quả cầu, thì 2 đầu thanh kim loại đươc nhiễm điện. Đầu gần quả cầu hơn nhiễm điện trái dấu với điện tích của quả cầu, đầu xa hơn nhiễm điện cùng dấu (Fig. \ref{fig:nhiem dien do huong ung}). Đưa thanh kim loại ra xa quả cầu thì thanh kim loại trở về trạng thái không nhiễm điện như lúc đầu.'' ``1 vật được nhiễm điện cũng gọi là vật được tích điện.'' -- \cite[p. 7]{SGK_Vat_Ly_11_nang_cao}

\begin{figure}[H]
	\centering
	\includegraphics[scale=0.15]{nhiem_dien_do_huong_ung}
	\caption{Nhiễm điện do hưởng ứng, \cite[Hình 1.4, p. 7]{SGK_Vat_Ly_11_nang_cao}.}
	\label{fig:nhiem dien do huong ung}
\end{figure}

\subsection{Định luật Coulomb}
``Coulomb đã dùng chiếc cân xoắn (Fig. \ref{fig:can xoan Coulomb}) để khảo sát lực tương tác giữa 2 quả cầu nhiễm điện tích có kích thước nhỏ so với khoảng cách giữa chúng. Các vật nhiễm điện có kích thước nhỏ như vậy gọi là các \textit{điện tích điểm}.

\begin{figure}[H]
	\centering
	\includegraphics[scale=0.15]{can_xoan_Coulomb}
	\caption{Cân xoắn Coulomb, \cite[Hình 1.5, p. 7]{SGK_Vat_Ly_11_nang_cao}. Khoảng cách giữa 2 quả cầu $A,B$ được điều chỉnh nhờ chiếc núm xoay $C$ của cân. Độ xoắn của sợi dây treo cho phép ta xác định lực tương tác giữa 2 quả cầu.}
	\label{fig:can xoan Coulomb}
\end{figure}
Năm 1785, Coulomb tổng kết các kết quả thí nghiệm của mình \& nêu thành định luật sau đây gọi là \textit{định luật Coulomb}:

\begin{dinhluat}[Định luật Coulomb]
	Độ lớn của lực tương tác giữa 2 điện tích điểm tỷ lệ thuận với tích các độ lớn của 2 điện tích đó \& tỷ lệ nghịch với bình phương khoảng cách giữa chúng. Phương của lực tương tác giữa 2 điện tích điểm là đường thẳng nối 2 điện tích điểm đó. 2 điện tích cùng dấu thì đẩy nhau, 2 điện tích trái dấu thì hút nhau (Fig. \ref{fig:luc tuong tac giua 2 dien tich diem}).
\end{dinhluat}

\begin{figure}[H]
	\centering
	\includegraphics[scale=0.15]{luc_tuong_tac_giua_2_dien_tich_diem}
	\caption{Phương \& chiều của lực tương tác giữa 2 điện tích điểm, \cite[Hình 1.6, p. 7]{SGK_Vat_Ly_11_nang_cao}.}
	\label{fig:luc tuong tac giua 2 dien tich diem}
\end{figure}
Lực tương tác giữa 2 điện tích gọi là \textit{lực điện}, hay cũng gọi là \textit{lực Coulomb}.'' -- \cite[p. 7]{SGK_Vat_Ly_11_nang_cao}

``Công thức tính độ lớn của lực tương tác giữa 2 điện tích điểm:
\begin{align}
	\label{luc tuong tac giua 2 dien tich diem}
	F = k\frac{|q_1q_2|}{r^2},
\end{align}
$r$ là khoảng cách giữa 2 điện tích điểm $q_1,q_2$; $k$ là hệ số tỷ lệ phụ thuộc vào hệ đơn vị. Trong hệ SI, $k = 9\cdot 10^9\frac{N\cdot m^2}{C^2}$.'' -- \cite[p. 8]{SGK_Vat_Ly_11_nang_cao}

\subsection{Lực tương tác của các điện tích trong điện môi (chất cách điện)}
``Thí nghiệm chứng tỏ rằng, lực tương tác giữa các điện tích điểm đặt trong điện môi đồng tính, chiếm đầy không gian xung quanh điện tích, giảm đi $\varepsilon$ lần so với khi chúng được đặt trong chân không.
\begin{align}
	\label{luc tuong tac cua cac dien tich trong dien moi}
	F = k\frac{|q_1q_2|}{\varepsilon r^2}.
\end{align}
Đại lượng $\varepsilon$ chỉ phụ thuộc vào tính chất của điện môi mà không phụ thuộc vào độ lớn các điện tích \& khoảng cách giữa các điện tích. $\varepsilon$ được gọi là \textit{hằng số điện môi}.

Người ta quy ước hằng số điện môi của chân không bằng $1$. Trong bảng \ref{tab:hang so dien moi}, ta chú ý hằng số điện môi của không khí gần bằng $1$. Thí nghiệm Coulomb được tiến hành trong không khí, nhưng vì hằng số điện môi của không khí gần bằng $1$ nên kết quả của thí nghiệm cũng được coi là đúng cả trong chân không.'' -- \cite[p. 8]{SGK_Vat_Ly_11_nang_cao}

\begin{table}[H]
	\centering
	\begin{tabular}{|l|l|}
		\hline
		\textbf{Chất} & \textbf{Hằng số điện môi} \\
		\hline
		Thủy tinh & $5\div 10$ \\
		\hline
		Sứ & $5.5$ \\
		\hline
		Êbônit & $2.7$ \\
		\hline
		Cao su & $2.3$ \\
		\hline
		Nước nguyên chất & $81.0$ \\
		\hline
		Dầu hỏa & $2.1$ \\
		\hline
		Không khí & $1.000594$ \\
		\hline
	\end{tabular}
	\caption{Hằng số điện môi của 1 số chất, \cite[Bảng 1.1, p. 8]{SGK_Vat_Ly_11_nang_cao}.}
	\label{tab:hang so dien moi}
\end{table}

\subsection{Máy lọc bụi}
``Sơ đồ của máy lọc bụi được trình bày trên \cite[Hình 1.8: \textsf{Sơ đồ máy lọc bụi}, p. 9]{SGK_Vat_Ly_11_nang_cao}. Không khí có nhiều bụi được quạt vào máy qua lớp lọc bụi thông thường. Tại đây, các hạt bụi có kích thước lớn bị gạt lại. Dòng không khí có lẫn các hạt bụi kích thước nhỏ vẫn bay lên. 2 lưới 1 \& 2 thực chất là 2 điện cực: lưới 1 là điện cực dương, lưới 2 là điện cực âm. Khi bay qua lưới 1 các hạt bụi nhiễm điện dương. Do đó, khi gặp lưới 2 nhiễm điện âm, các hạt bụt bị hút vào lưới. Vì vậy, khi qua lưới 2, không khí đã được lọc sạch bụi. Sau đó có thể cho không khí đi qua lớp lọc bằng than để khử mùi. Bằng cách này có thể loc đến $95\%$ bụi trong không khí. Máy lọc bụi là 1 ứng dụng của lực tương tác giữa các điện tích. Ngoài ra, lực tương tác giữa các điện tích còn có nhiều ứng dụng khác trong công nghiệp cũng như trong đời sống. E.g., kỹ thuật sơn tĩnh điện là 1 trong những ứng dụng đó. Muốn sơn vỏ xe ô tô, người ta làm cho sơn \& vỏ xe nhiễm điện trái dấu nhau. Khi sơn được phun vào vỏ xe, thì các hạt sơn nhỏ li ti sẽ bị hút \& bám chặt vào mặt vỏ xe.'' -- \cite[p. 9]{SGK_Vat_Ly_11_nang_cao}

%------------------------------------------------------------------------------%

\section{Thuyết Electron. Định Luật Bảo Toàn Điện Tích}

\subsection{Thuyết electron}
``Thuyết dựa vào sự có mặt của electron \& chuyển động của chúng để giải thích 1 số hiện tượng điện từ gọi là \textit{thuyết electron}. Thuyết electron trong phạm vi giải thích tính dẫn điện hay cách điện \& sự nhiễm điện của các vật gồm 1 số nội dung chính như sau:
\begin{itemize}
	\item Bình thường tổng đại số tất cả các điện tích trong nguyên tử bằng không, nguyên tử trung hòa về điện (\cite[Hình 2.1: \textsf{Mô hình đơn giản của nguyên tử liti}, p. 10]{SGK_Vat_Ly_11_nang_cao}).
	
	Nếu nguyên tử bị mất đi 1 số electron thì tổng đại số các điện tích trong nguyên tử là 1 số dương, nó là 1 \textit{ion dương}. Ngược lại, nếu nguyên tử nhận thêm 1 số electron thì nó là \textit{ion âm} (\cite[Hình 2.2: \textsf{Mô hình đơn giản của nguyên tử liti. (a) ion dương liti; (b) ion âm liti}, p. 10]{SGK_Vat_Ly_11_nang_cao}).
	\item Khối lượng của electron rất nhỏ nên độ linh động của electron rất lớn. Vì vậy, do 1 số điều kiện nào đó (cọ xát, tiếp xúc, nung nóng, $\ldots$) 1 số electron có thể bứt ra khỏi nguyên tử, di chuyển trong vật hay di chuyển từ vật này sang vật khác. Electron di chuyển từ vật này sang vật khác làm cho các vật \textit{nhiễm điện}. Vật nhiễm điện âm là vật thừa electron, vật nhiễm điện dương là vật thiếu electron.'' -- \cite[p. 10]{SGK_Vat_Ly_11_nang_cao}
\end{itemize}
``Nhiều trường hợp lẽ ra phải nói ``hạt mang điện'' hay ``vật mang điện'' thì người ta lại quen nói gọn là ``điện tích''. Ngoài ra, thuật ngữ ``điện tích'' nhiều khi được dùng với ý nghĩa là điện lượng.'' -- \cite[p. 10]{SGK_Vat_Ly_11_nang_cao}

\subsection{Vật (chất) dẫn điện \& vật (chất) cách điện}
``Xét về tính dẫn điện của môi trường, người ta phân biệt vật dẫn điện (vật dẫn) với vật cách điện (điện môi). \textit{Vật dẫn điện} là những vật có nhiều hạt mang điện có thể di chuyển được trong những khoảng lớn hơn nhiều lần kích thước phân tử của vật. Những hạt đó gọi là các \textit{điện tích tự do}. Kim loại có nhiều electron tự do, các dung dịch muối, axit, bazơ có nhiều ion tự do. Chúng là những chất dẫn điện. Những vật có chứa rất ít điện tích tự do là những \textit{vật điện môi}. Thủy tinh, nước nguyên chất, không khí khô, $\ldots$ có rất ít điện tích tự do. Chúng là những điện môi.'' -- \cite[pp. 10--11]{SGK_Vat_Ly_11_nang_cao}

\subsection{Giải thích 3 hiện tượng nhiễm điện}

\subsubsection{Nhiễm điện do cọ xát}
``Nếu có những điểm tiếp xúc chặt chẽ giữa thanh thủy tinh \& mảnh lụa, thì ở những điểm đó có 1 số electron từ thủy tinh di chuyển sang lụa. Khi thanh thủy tinh cọ xát với lụa thì số điểm tiếp xúc chặt chẽ tăng lên rất lớn. Do đó số electron di chuyển từ thủy tinh sang lụa cũng tăng lên. Vì vậy, thanh thủy tinh nhiễm điện dương, mảnh lụa nhiễm điện âm (\cite[Hình 2.3: \textsf{Nhiễm điện do cọ xát}, p. 11]{SGK_Vat_Ly_11_nang_cao}).'' -- \cite[p. 11]{SGK_Vat_Ly_11_nang_cao}

\subsubsection{Nhiễm điện do tiếp xúc}
``Khi thanh kim loại trung hòa điện tiếp xúc với quả cầu nhiễm điện âm, thì 1 phần trong số electron thừa ở quả cầu di chuyển sang thanh kim loại. Vì thế thanh kim loại cũng thừa electron. Do đó, thanh kim loại nhiễm điện âm (\cite[Hình 2.4: \textsf{Nhiễm điện do tiếp xúc}, p.11]{SGK_Vat_Ly_11_nang_cao}). Ngược lại, nếu thanh kim loại trung hòa điện tiếp xúc với quả cầu nhiễm điện dương, thì 1 số electron tự do từ thanh kim loại sẽ di chuyển sang quả cầu. Vì thế thanh kim loại trở thành thiếu electron. Do đó, thanh kim loại nhiễm điện dương.'' -- \cite[p. 11]{SGK_Vat_Ly_11_nang_cao}

\subsubsection{Nhiễm điện do hưởng ứng}
``Thanh kim loại trung hòa điện đặt gần quả cầu nhiễm điện âm, thì các electron tự do trong thanh kim loại bị đẩy ra xa quả cầu. Do đó, đầu thanh kim loại xa quả cầu thừa electron, nên nhiễm điện âm. Đầu thanh kim loại gần quả cầu thiếu electron, nên nhiễm điện dương (\cite[Hình 2.5: \textsf{Nhiễm điện do hưởng ứng}, p.11]{SGK_Vat_Ly_11_nang_cao}). Thanh kim loại đặt gần quả cầu nhiễm điện dương, thì electron tự do trong thanh kim loại bị hút lại gần quả cầu. Do đó, đầu thanh gần quả cầu thừa electron nên nhiễm điện âm, còn đầu kia thiếu electron nên nhiễm điện dương. Vậy thực chất của sự nhiễm điện do hưởng ứng là sự phân bố lại điện tích trong thanh kim loại.'' -- \cite[pp. 11--12]{SGK_Vat_Ly_11_nang_cao}

\subsection{Định luật bảo toàn điện tích}

\begin{dinhluat}[Định luật bảo toàn điện tích]
	Ở 1 hệ vật cô lập về điện, i.e., hệ không trao đổi điện tích với các hệ khác, thì tổng đại số các điện tích trong hệ là 1 hằng số.
\end{dinhluat}
``1 vật nào đó trong hệ được nhiễm điện không có nghĩa là điện tích được sinh ra mà là các định tích âm \& dương được tách ra \& được phân bố lại trong nội bộ hệ vật. Cho đến nay, định luật bảo toàn điện tích đã được kiểm nghiệm trong nhiều điều kiện khác nhau, nhưng người ta chưa gặp 1 trường hợp nào cho thấy định luật này không được thỏa mãn.'' -- \cite[p. 12]{SGK_Vat_Ly_11_nang_cao}

%------------------------------------------------------------------------------%

\section{Điện Trường}

\subsection{Điện trường}

\subsubsection{Khái niệm điện trường}
``1 vật tác dụng lực hấp dẫn lên các vật khác ở gần nó vì xung quanh vật đó có trường hấp dẫn. Ở đây ta cũng có hiện tượng tương tự. \textit{1 điện tích tác dụng lực điện lên các điện tích khác ở gần nó. Ta nói, xung quanh điện tích có điện trường}. Các điện tích tương tác được với nhau là vì điện trường của điện tích này tác dụng lên điện tích kia. Hiện nay, khoa học chứng tỏ những điều trên là đúng.'' ``Con cá mập đầu búa có thể nhận biết được điện trường.'' -- \cite[p. 13]{SGK_Vat_Ly_11_nang_cao}

\subsubsection{Tính chất cơ bản của điện trường}
``\textit{Tính chất cơ bản của điện trường là nó tác dụng lực điện lên điện tích đặt trong nó}. 1 vật có kích thước nhỏ, mang 1 điện tích nhỏ, được dùng để phát hiện lực điện tác dụng lên nó gọi là \textit{điện tích thử}. Người ta dùng điện tích thử để nhận biết điện trường.'' `Trong \cite[Chap. 1]{SGK_Vat_Ly_11_nang_cao}, ta chỉ xét điện trường của các điện tích đứng yên đối với nhau, i.e., \textit{điện trường tĩnh}, gọi tắt là \textit{điện trường}.'' -- \cite[p. 13]{SGK_Vat_Ly_11_nang_cao}

\subsection{Cường độ điện trường}
``Giả sử ta có 1 số điện tích thử $q_1,q_2,q_3,\ldots$ Đặt lần lượt các điện tích này tại 1 điểm nhất định trong điện trường \& xác định các lực $\overrightarrow{F_1},\overrightarrow{F_2},\overrightarrow{F_3},\ldots$ tác dụng lên chúng. Thí nghiệm cho biết các lực $\overrightarrow{F_1},\overrightarrow{F_2},\overrightarrow{F_3}$, có độ lớn khác nhau, nhưng các thương số dạng $\frac{F}{|q|}$ thì bằng nhau. Nếu để ý đến cả chiều của các lực tác dụng lên các điện tích thử thì các thương dạng $\frac{\overrightarrow{F}}{q}$ cũng không đổi, i.e., $\frac{\overrightarrow{F_i}}{q_i} = \mbox{const}$, $i\in\mathbb{N}$. Làm thí nghiệm ở các điểm khác nhau thì các thương $\frac{\overrightarrow{F}}{q}$ là khác nhau. Thương $\frac{\overrightarrow{F}}{q}$ đặc trưng cho điện trường ở điểm đang xét về mặt tác dụng lực gọi là \textit{cường độ điện trường} \& ký hiệu là $\overrightarrow{E}$.
\begin{align}
	\label{cuong do dien truong}
	\boxed{\overrightarrow{E} = \frac{\overrightarrow{F}}{q}.}
\end{align}
Trong trường hợp đã biết cường độ điện trường, thì từ công thức \eqref{cuong do dien truong} suy ra:
\begin{align}
	\label{cuong do dien truong 1}
	\overrightarrow{F} = q\overrightarrow{E}.
\end{align}
Từ \eqref{cuong do dien truong 1} ta thấy nếu $q > 0$ thì $\overrightarrow{F}$ cùng chiều với $\overrightarrow{E}$ (Fig. \ref{fig:chieu luc dien}), ngược lại nếu $q < 0$ thì $\overrightarrow{F}$ ngược chiều với $\overrightarrow{E}$ (Fig. \ref{fig:chieu luc dien}).

\begin{figure}[H]
	\centering
	\includegraphics[scale=0.15]{chieu_luc_dien}
	\caption{Chiều của lực điện tác dụng lên điện tích, \cite[Hình 3.1, p. 14]{SGK_Vat_Ly_11_nang_cao}.}
	\label{fig:chieu luc dien}
\end{figure}
Trong hệ SI, đơn vị cường độ điện trường có thể là Newton trên Coulomb, nhưng thường dùng đơn vị Von trên mét, ký hiệu là V\texttt{/}m.'' ``Cường độ điện trường ($\overrightarrow{E}$) là đại lượng vector, nhưng nhiều khi người ta cũng gọi độ lớn của $\overrightarrow{E}$, ký hiệu $E\coloneqq| \overrightarrow{E}|$, là cường độ điện trường.'' -- \cite[p. 14]{SGK_Vat_Ly_11_nang_cao}

\subsection{Đường sức điện}

\subsubsection{Định nghĩa}
``Có nhiều cách mô tả điện trường. Cách mô tả có tính trực quan rõ rệt là dùng cách vẽ các đường sức điện.

\begin{dinhnghia}[Đường sức điện]
	\emph{Đường sức điện} là đường được vẽ trong điện trường sao cho tiếp tuyến tại bất kỳ điểm nào trên đường cũng trùng với phương của vector cường độ điện trường tại điểm đó.
\end{dinhnghia}
Tuy nhiên, trong thực tế người ta thường quy định cho đường sức 1 chiều đi sao cho chiều của đường sức \& chiều của vector cường độ điện trường tại các điểm trên đường là trùng nhau. Khi đó, ta hiểu các đường sức là các đường có chiều xác định (Fig. \ref{fig:duong suc dien})

\begin{figure}[H]
	\centering
	\includegraphics[scale=0.15]{duong_suc_dien}
	\caption{Đường sức điện \& vector cường độ điện trường, \cite[Hình 3.2, p. 15]{SGK_Vat_Ly_11_nang_cao}.}
	\label{fig:duong suc dien}
\end{figure}
Các đường sức điện của 1 điện tích điểm \& của hệ 2 điện tích điểm được trình bày trên Figs. \ref{fig:duong suc cua 1 dien tich diem}--\ref{fig:duong suc cua he 2 dien tich diem}.'' --  \cite[pp. 14--15]{SGK_Vat_Ly_11_nang_cao}

\begin{figure}
	\centering
	\begin{subfigure}{.5\textwidth}
		\centering
		\includegraphics[width=.3\linewidth]{duong_suc_dien_tich_diem_duong}
		\caption{Đường sức của 1 điện tích điểm dương.}
	\end{subfigure}%
	\begin{subfigure}{.5\textwidth}
		\centering
		\includegraphics[width=.3\linewidth]{duong_suc_dien_tich_diem_am}
		\caption{Đường sức của 1 điện tích điểm âm.}
	\end{subfigure}
	\caption{Đường sức của 1 điện tích điểm, \cite[Hình 3.3, p. 15]{SGK_Vat_Ly_11_nang_cao}.}
	\label{fig:duong suc cua 1 dien tich diem}
\end{figure}

\begin{figure}
	\centering
	\begin{subfigure}{.5\textwidth}
		\centering
		\includegraphics[width=.4\linewidth]{duong_suc_he_2_dien_tich_diem_duong}
		\caption{2 điện tích điểm dương.}
	\end{subfigure}%
	\begin{subfigure}{.5\textwidth}
		\centering
		\includegraphics[width=.4\linewidth]{duong_suc_he_2_dien_tich_diem_trai_dau}
		\caption{2 điện tích trái dấu.}
	\end{subfigure}
	\caption{Đường sức của hệ 2 điện tích điểm, \cite[Hình 3.4, p. 15]{SGK_Vat_Ly_11_nang_cao}.}
	\label{fig:duong suc cua he 2 dien tich diem}
\end{figure}

\subsubsection{Các tính chất của đường sức điện}
``Các đường sức điện có 1 số tính chất sau đây:
\begin{itemize}
	\item \textit{Tại mỗi điểm trong điện trường, ta có thể vẽ được 1 \& chỉ 1 đường sức điện đi qua.}
	\item \textit{Các đường sức điện là các đường cong không kín. Nó xuất phát từ các điện tích dương \& tận cùng ở các điện tích âm (hoặc ở vô cực)}. Trong trường hợp chỉ có 1 điện tích, thì các đường sức xuất phát từ điện tích dương ra vô cực, hoặc từ vô cực đến điện tích âm (Fig. \ref{fig:duong suc cua 1 dien tich diem}).
	\item \textit{Nơi nào cường độ điện trường lớn hơn thì các đường sức điện ở đó được vẽ mau hơn (dày hơn), nơi nào cường độ điện trường nhỏ hơn thì các đường sức điện ở đó được vẽ thưa hơn}. E.g., trên các Figs. \ref{fig:duong suc cua 1 dien tich diem}--\ref{fig:duong suc cua he 2 dien tich diem}, ở nơi gần điện tích, các đường sức điện mau hơn nơi xa điện tích.'' --  \cite[p. 15]{SGK_Vat_Ly_11_nang_cao}
\end{itemize}

\subsubsection{Điện phổ}
``Dùng 1 loại bột cách điện rắc vào dầu cách điện \& khuấy đều. Sau đó đặt 1 quả cầu nhỏ nhiễm điện vào trong dầu. Gõ nhẹ vào khay dầu thì các hạt bột sẽ sắp xếp thành các ``đường hạt bột''. Ta gọi hệ các ``đường hạt bột'' đó là \textit{điện phổ} của quả cầu nhiễm điện. Điện phổ cho phép ta hình dung dạng \& sự phân bố các đường sức điện. Đường sức điện vẽ trong các Figs. \ref{fig:duong suc cua 1 dien tich diem}--\ref{fig:duong suc cua he 2 dien tich diem} tương ứng với các điện phổ ở \cite[Hình 3.5: \textsf{Điện phổ của 1 quả cầu nhiễm điện} \& Hình 3.6: \textsf{Điện phổ của 2 quả cầu nhiễm điện cùng \& trái dấu}, p. 16]{SGK_Vat_Ly_11_nang_cao}

\subsection{Điện trường đều}

\begin{dinhnghia}[Điện trường đều]
	1 điện trường mà vector cường độ điện trường tại mọi điểm đều bằng nhau gọi là \emph{điện trường đều}.
\end{dinhnghia}
Theo tính chất của đường sức, ta suy ra các đường sức của điện trường đều là các đường thẳng song song \& cách đều nhau. \cite[Hình 3.7: \textsf{Điện phổ của điện trường ở giữa 2 tấm kim loại phẳng, rộng, song song, mang điện tích trái dấu, có độ lớn bằng nhau}, p. 16]{SGK_Vat_Ly_11_nang_cao} cho biết điện phổ của 2 tấm kim loại phẳng, rộng, song song, mang điện tích trái dấu, có độ lớn bằng nhau. Ở rìa của 2 tấm kim loại, các ``đường hạt bột'' là các đường cong, còn ở giữa 2 tấm, các ``đường hạt bột'' song song \& cách đều nhau. Dựa vào điện phổ, ta có thể nói điện trường giữa 2 tấm kim loại là \textit{điện trường đều}. Đường sức của điện trường này được vẽ trên Fig. \ref{fig:duong suc dien truong deu}.'' -- \cite[p. 16]{SGK_Vat_Ly_11_nang_cao}

\begin{figure}[H]
	\centering
	\includegraphics[scale=0.15]{duong_suc_dien_truong_deu}
	\caption{Các đường sức ở giữa 2 tấm kim loại phẳng, rộng, song song, mang điện tích trái dấu, có độ lớn bằng nhau. Các đường sức này song song với nhau \& cách đều nhau, \cite[Hình 3.8, p. 16]{SGK_Vat_Ly_11_nang_cao}.}
	\label{fig:duong suc dien truong deu}
\end{figure}

\subsection{Điện trường của 1 điện tích điểm}
``2 điện tích điểm $q,Q$ đặt cách nhau 1 khoảng $r$ trong chân không thì lực Coulomb tác dụng lên điện tích $q$ được viết dưới dạng:
\begin{align*}
	F = 9\cdot 10^9\frac{|qQ|}{r^2}.
\end{align*}
Từ công thức \eqref{cuong do dien truong} ta suy ra cường độ điện trường của điện tích điểm $Q$ tại 1 điểm là:
\begin{align}
	\label{cuong do dien truong 2}
	E = 9\cdot 10^9\frac{|Q|}{r^2},
\end{align}
$r$ là khoảng cách từ điểm khảo sát đến điện tích $Q$. Nếu $Q > 0$ thì cường độ điện trường hướng ra xa điện tích $Q$ (Fig. \ref{fig:chieu cua vector cuong do dien truong cua dien tich duong}), nếu $Q < 0$ thì cường độ điện trường hướng về phía điện tích $Q$ (Fig. {fig:chieu cua vector cuong do dien truong cua dien tich am}).'' -- \cite[pp. 16--17]{SGK_Vat_Ly_11_nang_cao}

\begin{figure}
	\centering
	\begin{subfigure}{.5\textwidth}
		\centering
		\includegraphics[width=.4\linewidth]{chieu_cua_vector_cuong_do_dien_truong_cua_dien_tich_diem_duong}
		\caption{$Q > 0$.}
		\label{fig:chieu cua vector cuong do dien truong cua dien tich duong}
	\end{subfigure}%
	\begin{subfigure}{.5\textwidth}
		\centering
		\includegraphics[width=.4\linewidth]{chieu_cua_vector_cuong_do_dien_truong_cua_dien_tich_diem_am}
		\caption{$Q < 0$.}
		\label{fig:chieu cua vector cuong do dien truong cua dien tich am}
	\end{subfigure}
	\caption{Chiều của vector cường độ điện trường của điện tích điểm phụ thuộc vào dấu của điện tích, \cite[Hình 3.9, p. 17]{SGK_Vat_Ly_11_nang_cao}.}
	\label{fig:chieu cua vector cuong do dien truong cua dien tich}
\end{figure}

\subsection{Nguyên lý chồng chất điện trường}

\begin{nguyenly}[Nguyên lý chồng chất điện trường]
	Giả sử ta có hệ $n$ điện tích điểm $\{Q_i\}_{i=1}^n$. Gọi cường độ điện trường của hệ ở 1 điểm nào đó là $\overrightarrow{E}$. Cường độ điện trường chỉ của điện tích $Q_i$ là $\overrightarrow{E_i}$ tại điểm đang xét, $i = 1,\ldots, n$. Khi đó ta có:
	\begin{align}
		\label{nguyen ly chong chat dien truong}
		\overrightarrow{E} = \sum_{i=1}^n \overrightarrow{E_i} = \overrightarrow{E_1} + \cdots + \overrightarrow{E_n}.
	\end{align}
\end{nguyenly}

\subsection{Tương tác gần \& tương tác xa}
``Tương tác giữa 2 vật không tiếp xúc với nhau được thực hiện bằng cách nào? Có 2 cách giải  đáp câu hỏi đó. Cách giải đáp thứ nhất cho rằng 2 vật không tiếp xúc với nhau vẫn có thể tương tác với nhau. Quan điểm này gọi là \textit{quan điểm tương tác xa}. Định luật vạn vật hấp dẫn \& định luật Coulomb thể hiện quan điểm đó.

Cách giải đáp thứ 2 cho rằng, có tương tác hấp dẫn giữa 2 vật không tiếp xúc với nhau là vì vật này được đặt trong trường hấp dẫn của vật kia, lực hấp dẫn tác dụng lên vật $B$ là do trường hấp dẫn của vật $A$ tại điểm đặt vật $B$ gây ra. Tương tự như vậy, nếu có 2 điện tích $A$ \& $B$, thì có lực điện tác dụng lên điện tích $B$ là vì $B$ được đặt trong điện trường của điện tích $A$. Điện trường của điện tích $A$ là thực thể vật lý truyền lực điện từ điện tích $A$ đến điện tích $B$. Quan điểm này gọi là \textit{quan điểm tương tác gần}. Theo quan điểm tương tác xa thì tốc độ truyền tương tác là vô hạn. Điều đó trái với thực tế. Còn theo quan điểm tương tác gần thì tốc độ truyền tương tác là hữu hạn. Nhiều sự kiện thực nghiệm đã chứng tỏ quan điểm tương tác gần là phù hợp với thực tế.'' -- \cite[p. 18]{SGK_Vat_Ly_11_nang_cao}

%------------------------------------------------------------------------------%

\section{Công của Lực Điện. Hiệu Điện Thế}

\begin{cauhoi}
	``Công của trọng lực được biểu diễn qua hiệu thế năng hấp dẫn. Còn công của lực điện có thể biểu diễn qua đại lượng nào?'' -- \cite[p. 19]{SGK_Vat_Ly_11_nang_cao}
\end{cauhoi}

\subsection{Công của lực điện}

%------------------------------------------------------------------------------%

\section{Bài Tập về Lực Coulomb \& Điện Trường}

%------------------------------------------------------------------------------%

\section{Vật Dẫn \& Điện Môi Trong Điện Trường}

%------------------------------------------------------------------------------%

\section{Tụ Điện}

%------------------------------------------------------------------------------%

\section{Năng Lượng Điện Trường}

%------------------------------------------------------------------------------%

\section{Bài Tập về Tụ Điện}

%------------------------------------------------------------------------------%

\section{Máy Sao Chụp Quang Học (Photocopy)}

%------------------------------------------------------------------------------%

\section{Tóm Tắt Chương 1}

%------------------------------------------------------------------------------%

\chapter{Dòng Điện Không Đổi}

\section{Dòng Điện Không Đổi. Nguồn Điện}

%------------------------------------------------------------------------------%

\section{Pin \& Acquy}

%------------------------------------------------------------------------------%

\section{Điện Năng \& Công Suất Điện. Định Luật Jun--Len-xơ}

%------------------------------------------------------------------------------%

\section{Định Luật Ôm Đối với Toàn Mạch}

%------------------------------------------------------------------------------%

\section{Định Luật Ôm Đối với Các Loại Mạch Điện. Mắc Các Nguồn Điện Thành Bộ}

%------------------------------------------------------------------------------%

\section{Bài Tập về Định Luật Ôm \& Công Suất Điện}

%------------------------------------------------------------------------------%

\section{Điện Tâm Đồ}

%------------------------------------------------------------------------------%

\section{Thực Hành: Đo Suất Điện Động \& Điện Trở Trong của Nguồn Điện}

%------------------------------------------------------------------------------%

\section{Tóm Tắt Chương 2}

%------------------------------------------------------------------------------%

\chapter{Dòng Diện Trong Các Môi Trường}

\section{Dòng Điện Trong Kim Loại}

%------------------------------------------------------------------------------%

\section{Hiện Tượng Nhiệt Điện. Hiện Tượng Siêu Dẫn}

%------------------------------------------------------------------------------%

\section{Dòng Điện Trong Chất Điện Phân. Định Luật Faraday}

%------------------------------------------------------------------------------%

\section{Bài Tập về Dòng Điện Trong Kim Loại \& Chất Điện Phân}

%------------------------------------------------------------------------------%

\section{Dòng Điện Trong Chân Không}

%------------------------------------------------------------------------------%

\section{Dòng Điện Trong Chất Khí}

%------------------------------------------------------------------------------%

\section{Dòng Điện Trong Chất Bán Dẫn}

%------------------------------------------------------------------------------%

\section{Linh Kiện Bán Dẫn}

%------------------------------------------------------------------------------%

\section{Thực Hành: Khảo Sát Đặc Tính Chỉnh Lưu của Diot Bán Dẫn \& Đặc Tính Khuếch Đại của Tranzito}

%------------------------------------------------------------------------------%

\section{Tóm Tắt Chương 3}

%------------------------------------------------------------------------------%

\chapter{Từ Trường}

\section{Từ Trường}

%------------------------------------------------------------------------------%

\section{Phương \& Chiều của Lực Từ Tác Dụng Lên Dòng Điện}

%------------------------------------------------------------------------------%

\section{Cảm Ứng Từ. Định Luật Ampe}

%------------------------------------------------------------------------------%

\section{Từ Trường của 1 Số Dòng Điện Có Dạng Đơn Giản}

%------------------------------------------------------------------------------%

\section{Bài Tập về Từ Trường}

%------------------------------------------------------------------------------%

\section{Tương Tác Giữa 2 Dòng Điện Thẳng Song Song. Định Nghĩa Đơn Vị Ampe}

%------------------------------------------------------------------------------%

\section{Lực Lo-ren-xơ}

%------------------------------------------------------------------------------%

\section{Khung Dây có Dòng Điện Đặt trong Từ Trường}

%------------------------------------------------------------------------------%

\section{Sự Từ Hóa Các Chất. Sắt Từ}

%------------------------------------------------------------------------------%

\section{Từ Trường Trái Đất}

%------------------------------------------------------------------------------%

\section{Bài Tập về Lực Từ}

%------------------------------------------------------------------------------%

\section{Từ Trường \& Máy Gia Tốc}

%------------------------------------------------------------------------------%

\section{Thực Hành: Xác Định Thành Phần Năm Ngang của Từ Trường Trái Đất}

%------------------------------------------------------------------------------%

\section{Tóm Tắt Chương 4}

%------------------------------------------------------------------------------%

\chapter{Cảm Ứng Điện Từ}

\section{Hiện Tượng Cảm Ứng Điện Từ. Suất Điện Động Cảm Ứng}

%------------------------------------------------------------------------------%

\section{Suất Điện Động Cảm Ứng Tron 1 Đoạn Dây Dẫn Chuyển Động}

%------------------------------------------------------------------------------%

\section{Dòng Điện Fu-cô}

%------------------------------------------------------------------------------%

\section{Hiện Tượng Tự Cảm}

%------------------------------------------------------------------------------%

\section{Năng Lượng Từ Trường}

%------------------------------------------------------------------------------%

\section{Bài Tập về Cảm Ứng Điện Từ}

%------------------------------------------------------------------------------%

\section{1 Số Mốc Thời Gian Đáng Lưu Ý Trong Lĩnh Vực Điện Tử}

%------------------------------------------------------------------------------%

\section{Tóm Tắt Chương 5}

%------------------------------------------------------------------------------%

\part{Quang Hình Học}

\chapter{Khúc Xạ Ánh Sáng}

\section{Khúc Xạ Ánh Sáng}

%------------------------------------------------------------------------------%

\section{Phản Xạ Toàn Phần}

%------------------------------------------------------------------------------%

\section{Bài Tập về Khúc Xạ Ánh Sáng \& Phản Xạ Toàn Phần}

%------------------------------------------------------------------------------%

\section{Bài Đọc Thêm. Hiện Tượng Ảo Ảnh}

%------------------------------------------------------------------------------%

\section{Tóm Tắt Chương 6}

%------------------------------------------------------------------------------%

\chapter{Mắt. Các Dụng Cụ Quang}

\section{Lăng Kính}

%------------------------------------------------------------------------------%

\section{Thấu Kính Mỏng}

%------------------------------------------------------------------------------%

\section{Bài Tập về Lăng Kính \& Thấu Kính Mỏng}

%------------------------------------------------------------------------------%

\section{Mắt}

%------------------------------------------------------------------------------%

\section{Các Tật của Mắt \& Cách Khắc Phục}

%------------------------------------------------------------------------------%

\section{Kính Lúp}

%------------------------------------------------------------------------------%

\section{Kính Hiển Vi}

%------------------------------------------------------------------------------%

\section{Kính Thiên Văn}

%------------------------------------------------------------------------------%

\section{Bài Tập về Dụng Cụ Quang}

%------------------------------------------------------------------------------%

\section{Thực Hành: Xác Định Chiết Suất của Nước \& Tiêu Cự của Thấu Kính Phân Kỳ}

%------------------------------------------------------------------------------%

\section{Tóm Tắt Chương 7}

%------------------------------------------------------------------------------%

\printbibliography[heading=bibintoc]
	
\end{document}