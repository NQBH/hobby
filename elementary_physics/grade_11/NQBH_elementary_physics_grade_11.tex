\documentclass[oneside]{book}
\usepackage[backend=biber,natbib=true,style=authoryear]{biblatex}
\addbibresource{/home/hong/1_NQBH/reference/bib.bib}
\usepackage[utf8]{vietnam}
\usepackage{tocloft}
\renewcommand{\cftsecleader}{\cftdotfill{\cftdotsep}}
\usepackage[colorlinks=true,linkcolor=blue,urlcolor=red,citecolor=magenta]{hyperref}
\usepackage{amsmath,amssymb,amsthm,mathtools,float,graphicx,algpseudocode,algorithm,tcolorbox,tikz,tkz-tab,subcaption}
\usepackage[inline]{enumitem}
\allowdisplaybreaks
\numberwithin{equation}{section}
\newtheorem{assumption}{Assumption}[section]
\newtheorem{nhanxet}{Nhận xét}[section]
\newtheorem{conjecture}{Conjecture}[section]
\newtheorem{corollary}{Corollary}[section]
\newtheorem{hequa}{Hệ quả}[section]
\newtheorem{definition}{Definition}[section]
\newtheorem{dinhnghia}{Định nghĩa}[section]
\newtheorem{dinhluat}{Định luật}[section]
\newtheorem{example}{Example}[section]
\newtheorem{vidu}{Ví dụ}[section]
\newtheorem{lemma}{Lemma}[section]
\newtheorem{notation}{Notation}[section]
\newtheorem{principle}{Principle}[section]
\newtheorem{problem}{Problem}[section]
\newtheorem{baitoan}{Bài toán}[section]
\newtheorem{proposition}{Proposition}[section]
\newtheorem{menhde}{Mệnh đề}[section]
\newtheorem{nguyenly}{Nguyên lý}[section]
\newtheorem{question}{Question}[section]
\newtheorem{cauhoi}{Câu hỏi}[section]
\newtheorem{remark}{Remark}[section]
\newtheorem{luuy}{Lưu ý}[section]
\newtheorem{theorem}{Theorem}[section]
\newtheorem{dinhly}{Định lý}[section]
\usepackage[left=0.5in,right=0.5in,top=1.5cm,bottom=1.5cm]{geometry}
\usepackage{fancyhdr}
\pagestyle{fancy}
\fancyhf{}
\lhead{\small \textsc{Sect.} ~\thesection}
\rhead{\small \nouppercase{\leftmark}}
\renewcommand{\sectionmark}[1]{\markboth{#1}{}}
\cfoot{\thepage}
\def\labelitemii{$\circ$}

\makeatletter
\let\old@endpart\@endpart
\renewcommand\@endpart[1][]{%
	\begin{quote}#1\end{quote}%
	\old@endpart}
\makeatother

\title{Some Topics in Elementary Physics\texttt{/}Grade 11}
\author{Nguyễn Quản Bá Hồng\footnote{Independent Researcher, Ben Tre City, Vietnam\\e-mail: \texttt{nguyenquanbahong@gmail.com}; website: \url{https://nqbh.github.io}.}}
\date{\today}

\begin{document}
\frontmatter
\maketitle
\setcounter{secnumdepth}{4}
\setcounter{tocdepth}{3}
\tableofcontents
\newpage

%------------------------------------------------------------------------------%

\mainmatter
\part{Điện Học -- Điện Từ Học}
[``Phần Điện học -- Điện từ học đề cập đến các hiện tượng liên quan đến tương tác giữa các điện tích đứng yên \& chuyển động, gọi chung là \textit{hiện tượng điện từ} \& các quy luật chi phối các hiện tượng này. Các hiện tượng điện từ rất phổ biến trong tự nhiên, rất phong phú \& đa dạng. Chúng được ứng dụng rộng rãi trong khoa học \& kỹ thuật, cũng như trong cuộc sống.'' -- \cite{SGK_Vat_Ly_11_nang_cao}, p. 3]

\chapter{Điện Tích -- Điện Trường}

\begin{quotation}
	\textbf{Nội dung.} \textit{Định luật tương tác giữa các điện tích điểm (định luật Coulomb), điện trường, cường độ điện trường của điện tích điểm, hiệu điện thế, điện thế \& công của lực điện, năng lượng điện trường, tụ điện, ghép tụ điện}.
\end{quotation}

\section{Điện Tích. Định Luật Coulomb}
\textbf{Nội dung.} \textit{1 số khái niệm mở đầu về điện tích (điện tích dương, điện tích âm, sự nhiễm điện của các vật) \& về định luật tương tác giữa 2 điện tích}.

\subsection{2 loại điện tích. Sự nhiễm điện của các vật}

\subsubsection{2 loại điện tích}
``Có 2 loại điện tích: điện tích dương, điện tích âm. Các điện tích cùng dấu thì đẩy nhau, các điện tích khác dấu thì hút nhau. Đơn vị điện tích là coulomb\footnote{\textsc{Charles Coulomb} (1736--1806), nhà vật lý người Pháp. Có thể đọc thêm \href{https://vi.wikipedia.org/wiki/Charles-Augustin_de_Coulomb}{Wikipedia\texttt{/}Charles-Augustin de Coulomb
} \& \href{https://en.wikipedia.org/wiki/Charles-Augustin_de_Coulomb}{Charles-Augustin de Coulomb}.}, ký hiệu là C. Điện tích của electron là điện tích âm \& có độ lớn $e = 1,6\cdot 10^{-19}$ C. 1 điện tích $e = 1,6\cdot 10^{-19}$ C được gọi là \textit{điện tích nguyên tố}. Thí nghiệm đã chứng tỏ rằng, trong tự nhiên không có hạt nào có điện tích nhỏ hơn điện tích nguyên tố. Độ lớn của điện tích 1 hạt bao giờ cũng bằng 1 số nguyên lần điện tích nguyên tố.

Dựa vào sự tương tác giữa các điện tích cùng dấu người ta chế tạo ra điện nghiệm.

\begin{figure}[H]
	\centering
	\includegraphics[scale=0.15]{dien_nghiem}
	\caption{Điện nghiệm. 1. Bình thủy tinh; 2. Nút cách điện; 3. Nút kim loại; 4. Thanh kim loại; 5. 2 lá kim loại nhẹ. \cite[Hình 1.1, p. 6]{SGK_Vat_Ly_11_nang_cao}}
\end{figure}
Điện nghiệm dùng để phát hiện điện tích ở 1 vật. Khi 1 vật nhiễm điên chạm vào núm kim loại, thì điện tích truyền đến 2 lá kim loại (nhiễm điện do tiếp xúc). Do đó, 2 lá kim loại đẩy nhau \& xòe ra.'' -- \cite[p. 6]{SGK_Vat_Ly_11_nang_cao}

\subsubsection{Sự nhiễm điện của các vật}

\paragraph{Nhiễm điện do cọ xát.} ``Sau khi cọ xát vào lụa, thanh thủy tinh có thể hút được các mẩu giấy vụn (\cite[Hình 1.2: \textsf{Thanh thủy tin nhiễm điện hút các mẩu giấy}, p. 6]{SGK_Vat_Ly_11_nang_cao}). Người ta nói thanh thủy tinh được \textit{nhiễm điện do cọ xát}.'' -- \cite[p. 6]{SGK_Vat_Ly_11_nang_cao}

\paragraph{Nhiễm điện do tiếp xúc.} ``Cho thanh kim loại không nhiễm điện chạm vào quả cầu đã nhiễm điện thì thanh kim loại nhiễm điện cùng dấu với điện tích của quả cầu (Fig. \ref{fig:nhiem_dien_do_tiep_xuc}). Người ta nói thanh kim loại được \textit{nhiễm điện do tiếp xúc}. Đưa thanh kim loại ra xa quả cầu thì thanh kim loại vẫn nhiễm điện.'' -- \cite[p. 7]{SGK_Vat_Ly_11_nang_cao}

\begin{figure}[H]
	\centering
	\includegraphics[scale=0.15]{nhiem_dien_do_tiep_xuc}
	\caption{Nhiễm điện do tiếp xúc, \cite[Hình 1.3, p. 7]{SGK_Vat_Ly_11_nang_cao}.}
	\label{fig:nhiem_dien_do_tiep_xuc}
\end{figure}

\paragraph{Nhiễm điện do hưởng ứng.} ``Đưa thanh kim loại không nhiễm điện đến gần quả cầu đã nhiễm điện nhưng không chạm vào quả cầu, thì 2 đầu thanh kim loại đươc nhiễm điện. Đầu gần quả cầu hơn nhiễm điện trái dấu với điện tích của quả cầu, đầu xa hơn nhiễm điện cùng dấu (Fig. \ref{fig:nhiem_dien_do_huong_ung}). Đưa thanh kim loại ra xa quả cầu thì thanh kim loại trở về trạng thái không nhiễm điện như lúc đầu.'' ``1 vật được nhiễm điện cũng gọi là vật được tích điện.'' -- \cite[p. 7]{SGK_Vat_Ly_11_nang_cao}

\begin{figure}[H]
	\centering
	\includegraphics[scale=0.15]{nhiem_dien_do_huong_ung}
	\caption{Nhiễm điện do hưởng ứng, \cite[Hình 1.4, p. 7]{SGK_Vat_Ly_11_nang_cao}.}
	\label{fig:nhiem_dien_do_huong_ung}
\end{figure}

\subsection{Định luật Coulomb}
``Coulomb đã dùng chiếc cân xoắn (Fig. \ref{fig:can_xoan_Coulomb}) để khảo sát lực tương tác giữa 2 quả cầu nhiễm điện tích có kích thước nhỏ so với khoảng cách giữa chúng. Các vật nhiễm điện có kích thước nhỏ như vậy gọi là các \textit{điện tích điểm}.

\begin{figure}[H]
	\centering
	\includegraphics[scale=0.15]{can_xoan_Coulomb}
	\caption{Cân xoắn Coulomb, \cite[Hình 1.5, p. 7]{SGK_Vat_Ly_11_nang_cao}. Khoảng cách giữa 2 quả cầu $A,B$ được điều chỉnh nhờ chiếc núm xoay $C$ của cân. Độ xoắn của sợi dây treo cho phép ta xác định lực tương tác giữa 2 quả cầu.}
	\label{fig:can_xoan_Coulomb}
\end{figure}
Năm 1785, Coulomb tổng kết các kết quả thí nghiệm của mình \& nêu thành định luật sau đây gọi là \textit{định luật Coulomb}:

\begin{dinhluat}[Định luật Coulomb]
	\label{dinh luat: Coulomb}
	Độ lớn của lực tương tác giữa 2 điện tích điểm tỷ lệ thuận với tích các độ lớn của 2 điện tích đó \& tỷ lệ nghịch với bình phương khoảng cách giữa chúng. Phương của lực tương tác giữa 2 điện tích điểm là đường thẳng nối 2 điện tích điểm đó. 2 điện tích cùng dấu thì đẩy nhau, 2 điện tích trái dấu thì hút nhau (Fig. \ref{fig:luc_tuong_tac_giua_2_dien_tich_diem}).
\end{dinhluat}

\begin{figure}[H]
	\centering
	\includegraphics[scale=0.15]{luc_tuong_tac_giua_2_dien_tich_diem}
	\caption{Phương \& chiều của lực tương tác giữa 2 điện tích điểm, \cite[Hình 1.6, p. 7]{SGK_Vat_Ly_11_nang_cao}.}
	\label{fig:luc_tuong_tac_giua_2_dien_tich_diem}
\end{figure}
Lực tương tác giữa 2 điện tích gọi là \textit{lực điện}, hay cũng gọi là \textit{lực Coulomb}.'' -- \cite[p. 7]{SGK_Vat_Ly_11_nang_cao}

``Công thức tính độ lớn của lực tương tác giữa 2 điện tích điểm:
\begin{align}
	\label{luc tuong tac giua 2 dien tich diem}
	F = k\frac{|q_1q_2|}{r^2},
\end{align}
$r$ là khoảng cách giữa 2 điện tích điểm $q_1,q_2$; $k$ là hệ số tỷ lệ phụ thuộc vào hệ đơn vị. Trong hệ SI, $k = 9\cdot 10^9\ \frac{\rm N\cdot m^2}{\rm C^2}$.'' -- \cite[p. 8]{SGK_Vat_Ly_11_nang_cao}

``Lực tương tác giữa 2 điện tích điểm đứng yên (lực Coulomb): \textit{Điểm đặt}: điện tích. \textit{Phương}: đường thẳng nối 2 điện tích. \textit{Chiều}: lực đẩy nếu $q_1q_2 > 0$, lực hút nếu $q_1q_2 < 0$. \textit{Độ lớn}: tỷ lệ thuận với tích các độ lớn điện tích, tỷ lệ nghịch với bình phương khoảng cách giữa các điện tích. $F = \frac{k}{\varepsilon}\frac{|q_1q_2|}{r^2}$ ($F = F_{12} = F_{21}$) với $q$: coulomb (C), $r$: mét (m), $F$: Newton (N). $k = \frac{1}{4\pi\varepsilon_0} = 9\cdot 10^9\ {\rm Nm^2C^{-2}}$ ($\varepsilon_0 =$ \textit{hằng số điện}), $\varepsilon =$ \textit{hằng số điện môi} của môi trường ($\varepsilon\ge 1$) (chân không: $\varepsilon = 1$; không khí: $\varepsilon\approx 1$). Định luật Coulomb chỉ áp dụng được cho:
\begin{enumerate*}
	\item[$\bullet$] Các \textit{điện tích điểm};
	\item[$\bullet$] Các điện tích \textit{phân bố đều} trên những vật dẫn \textit{hình cầu} (coi như điện tích điểm ở tâm).'' -- \cite[p. 5--6]{Giai_Toan_Vat_Ly_11_tap_1}
\end{enumerate*}

\subsection{Lực tương tác của các điện tích trong điện môi (chất cách điện)}
``Thí nghiệm chứng tỏ rằng, lực tương tác giữa các điện tích điểm đặt trong điện môi đồng tính, chiếm đầy không gian xung quanh điện tích, giảm đi $\varepsilon$ lần so với khi chúng được đặt trong chân không.
\begin{align}
	\label{luc tuong tac cua cac dien tich trong dien moi}
	F = k\frac{|q_1q_2|}{\varepsilon r^2}.
\end{align}
Đại lượng $\varepsilon$ chỉ phụ thuộc vào tính chất của điện môi mà không phụ thuộc vào độ lớn các điện tích \& khoảng cách giữa các điện tích. $\varepsilon$ được gọi là \textit{hằng số điện môi}.

Người ta quy ước hằng số điện môi của chân không bằng $1$. Trong bảng \ref{tab:hang so dien moi}, ta chú ý hằng số điện môi của không khí gần bằng $1$. Thí nghiệm Coulomb được tiến hành trong không khí, nhưng vì hằng số điện môi của không khí gần bằng $1$ nên kết quả của thí nghiệm cũng được coi là đúng cả trong chân không.'' -- \cite[p. 8]{SGK_Vat_Ly_11_nang_cao}

\begin{table}[H]
	\centering
	\begin{tabular}{|c|c|}
		\hline
		\textbf{Chất} & \textbf{Hằng số điện môi} \\
		\hline
		Thủy tinh & $5\div 10$ \\
		\hline
		Sứ & $5.5$ \\
		\hline
		Êbônit & $2.7$ \\
		\hline
		Cao su & $2.3$ \\
		\hline
		Nước nguyên chất & $81.0$ \\
		\hline
		Dầu hỏa & $2.1$ \\
		\hline
		Không khí & $1.000594$ \\
		\hline
	\end{tabular}
	\caption{Hằng số điện môi của 1 số chất, \cite[Bảng 1.1, p. 8]{SGK_Vat_Ly_11_nang_cao}.}
	\label{tab:hang so dien moi}
\end{table}

\subsection{Máy lọc bụi}
``Sơ đồ của máy lọc bụi được trình bày trên \cite[Hình 1.8: \textsf{Sơ đồ máy lọc bụi}, p. 9]{SGK_Vat_Ly_11_nang_cao}. Không khí có nhiều bụi được quạt vào máy qua lớp lọc bụi thông thường. Tại đây, các hạt bụi có kích thước lớn bị gạt lại. Dòng không khí có lẫn các hạt bụi kích thước nhỏ vẫn bay lên. 2 lưới 1 \& 2 thực chất là 2 điện cực: lưới 1 là điện cực dương, lưới 2 là điện cực âm. Khi bay qua lưới 1 các hạt bụi nhiễm điện dương. Do đó, khi gặp lưới 2 nhiễm điện âm, các hạt bụt bị hút vào lưới. Vì vậy, khi qua lưới 2, không khí đã được lọc sạch bụi. Sau đó có thể cho không khí đi qua lớp lọc bằng than để khử mùi. Bằng cách này có thể loc đến $95\%$ bụi trong không khí. Máy lọc bụi là 1 ứng dụng của lực tương tác giữa các điện tích. Ngoài ra, lực tương tác giữa các điện tích còn có nhiều ứng dụng khác trong công nghiệp cũng như trong đời sống. E.g., kỹ thuật sơn tĩnh điện là 1 trong những ứng dụng đó. Muốn sơn vỏ xe ô tô, người ta làm cho sơn \& vỏ xe nhiễm điện trái dấu nhau. Khi sơn được phun vào vỏ xe, thì các hạt sơn nhỏ li ti sẽ bị hút \& bám chặt vào mặt vỏ xe.'' -- \cite[p. 9]{SGK_Vat_Ly_11_nang_cao}

%------------------------------------------------------------------------------%

\section{Thuyết Electron. Định Luật Bảo Toàn Điện Tích}

\subsection{Thuyết electron}
``Thuyết dựa vào sự có mặt của electron \& chuyển động của chúng để giải thích 1 số hiện tượng điện từ gọi là \textit{thuyết electron}. Thuyết electron trong phạm vi giải thích tính dẫn điện hay cách điện \& sự nhiễm điện của các vật gồm 1 số nội dung chính như sau:
\begin{itemize}
	\item Bình thường tổng đại số tất cả các điện tích trong nguyên tử bằng không, nguyên tử trung hòa về điện (\cite[Hình 2.1: \textsf{Mô hình đơn giản của nguyên tử liti}, p. 10]{SGK_Vat_Ly_11_nang_cao}).
	
	Nếu nguyên tử bị mất đi 1 số electron thì tổng đại số các điện tích trong nguyên tử là 1 số dương, nó là 1 \textit{ion dương}. Ngược lại, nếu nguyên tử nhận thêm 1 số electron thì nó là \textit{ion âm} (\cite[Hình 2.2: \textsf{Mô hình đơn giản của nguyên tử liti. (a) ion dương liti; (b) ion âm liti}, p. 10]{SGK_Vat_Ly_11_nang_cao}).
	\item Khối lượng của electron rất nhỏ nên độ linh động của electron rất lớn. Vì vậy, do 1 số điều kiện nào đó (cọ xát, tiếp xúc, nung nóng, $\ldots$) 1 số electron có thể bứt ra khỏi nguyên tử, di chuyển trong vật hay di chuyển từ vật này sang vật khác. Electron di chuyển từ vật này sang vật khác làm cho các vật \textit{nhiễm điện}. Vật nhiễm điện âm là vật thừa electron, vật nhiễm điện dương là vật thiếu electron.'' -- \cite[p. 10]{SGK_Vat_Ly_11_nang_cao}
\end{itemize}
``Nhiều trường hợp lẽ ra phải nói ``hạt mang điện'' hay ``vật mang điện'' thì người ta lại quen nói gọn là ``điện tích''. Ngoài ra, thuật ngữ ``điện tích'' nhiều khi được dùng với ý nghĩa là điện lượng.'' -- \cite[p. 10]{SGK_Vat_Ly_11_nang_cao}

\subsection{Vật (chất) dẫn điện \& vật (chất) cách điện}
``Xét về tính dẫn điện của môi trường, người ta phân biệt vật dẫn điện (vật dẫn) với vật cách điện (điện môi). \textit{Vật dẫn điện} là những vật có nhiều hạt mang điện có thể di chuyển được trong những khoảng lớn hơn nhiều lần kích thước phân tử của vật. Những hạt đó gọi là các \textit{điện tích tự do}. Kim loại có nhiều electron tự do, các dung dịch muối, axit, bazơ có nhiều ion tự do. Chúng là những chất dẫn điện. Những vật có chứa rất ít điện tích tự do là những \textit{vật điện môi}. Thủy tinh, nước nguyên chất, không khí khô, $\ldots$ có rất ít điện tích tự do. Chúng là những điện môi.'' -- \cite[pp. 10--11]{SGK_Vat_Ly_11_nang_cao}

\subsection{Giải thích 3 hiện tượng nhiễm điện}

\subsubsection{Nhiễm điện do cọ xát}
``Nếu có những điểm tiếp xúc chặt chẽ giữa thanh thủy tinh \& mảnh lụa, thì ở những điểm đó có 1 số electron từ thủy tinh di chuyển sang lụa. Khi thanh thủy tinh cọ xát với lụa thì số điểm tiếp xúc chặt chẽ tăng lên rất lớn. Do đó số electron di chuyển từ thủy tinh sang lụa cũng tăng lên. Vì vậy, thanh thủy tinh nhiễm điện dương, mảnh lụa nhiễm điện âm (\cite[Hình 2.3: \textsf{Nhiễm điện do cọ xát}, p. 11]{SGK_Vat_Ly_11_nang_cao}).'' -- \cite[p. 11]{SGK_Vat_Ly_11_nang_cao}

\subsubsection{Nhiễm điện do tiếp xúc}
``Khi thanh kim loại trung hòa điện tiếp xúc với quả cầu nhiễm điện âm, thì 1 phần trong số electron thừa ở quả cầu di chuyển sang thanh kim loại. Vì thế thanh kim loại cũng thừa electron. Do đó, thanh kim loại nhiễm điện âm (\cite[Hình 2.4: \textsf{Nhiễm điện do tiếp xúc}, p.11]{SGK_Vat_Ly_11_nang_cao}). Ngược lại, nếu thanh kim loại trung hòa điện tiếp xúc với quả cầu nhiễm điện dương, thì 1 số electron tự do từ thanh kim loại sẽ di chuyển sang quả cầu. Vì thế thanh kim loại trở thành thiếu electron. Do đó, thanh kim loại nhiễm điện dương.'' -- \cite[p. 11]{SGK_Vat_Ly_11_nang_cao}

\subsubsection{Nhiễm điện do hưởng ứng}
``Thanh kim loại trung hòa điện đặt gần quả cầu nhiễm điện âm, thì các electron tự do trong thanh kim loại bị đẩy ra xa quả cầu. Do đó, đầu thanh kim loại xa quả cầu thừa electron, nên nhiễm điện âm. Đầu thanh kim loại gần quả cầu thiếu electron, nên nhiễm điện dương (\cite[Hình 2.5: \textsf{Nhiễm điện do hưởng ứng}, p.11]{SGK_Vat_Ly_11_nang_cao}). Thanh kim loại đặt gần quả cầu nhiễm điện dương, thì electron tự do trong thanh kim loại bị hút lại gần quả cầu. Do đó, đầu thanh gần quả cầu thừa electron nên nhiễm điện âm, còn đầu kia thiếu electron nên nhiễm điện dương. Vậy thực chất của sự nhiễm điện do hưởng ứng là sự phân bố lại điện tích trong thanh kim loại.'' -- \cite[pp. 11--12]{SGK_Vat_Ly_11_nang_cao}

\subsection{Định luật bảo toàn điện tích}

\begin{dinhluat}[Định luật bảo toàn điện tích]
	\label{dinh luat: bao toan dien tich}
	Ở 1 hệ vật cô lập về điện, i.e., hệ không trao đổi điện tích với các hệ khác, thì tổng đại số các điện tích trong hệ là 1 hằng số.
\end{dinhluat}
``1 vật nào đó trong hệ được nhiễm điện không có nghĩa là điện tích được sinh ra mà là các định tích âm \& dương được tách ra \& được phân bố lại trong nội bộ hệ vật. Cho đến nay, định luật bảo toàn điện tích đã được kiểm nghiệm trong nhiều điều kiện khác nhau, nhưng người ta chưa gặp 1 trường hợp nào cho thấy định luật này không được thỏa mãn.'' -- \cite[p. 12]{SGK_Vat_Ly_11_nang_cao}. ``Trong 1 hệ kín, tổng các điện tích của hệ bảo toàn: $\sum_i q_i = \mbox{const}$.'' -- \cite[p. 5]{Giai_Toan_Vat_Ly_11_tap_1}

\subsection{Problems}

\begin{baitoan}[\cite{Giai_Toan_Vat_Ly_11_tap_1}, Bài toán 1, p. 6]
	``Xác định các đại lượng liên quan đến lực tương tác giữa 2 điện tích điểm đứng yên.''
\end{baitoan}
``Áp dụng công thức $F = \frac{k}{\varepsilon}\frac{|q_1q_2|}{r^2}$ để suy ra giá trị của đại lượng cần xác định. 1 số hiện tượng cần để ý:
\begin{enumerate}
	\item[$\bullet$] Khi cho 2 quả cầu nhỏ dẫn điện như nhau, đã nhiễm điện tiếp xúc nhau \& sau đó tách rời nhau thì tổng điện tích chia đều cho mỗi quả cầu.
	\item[$\bullet$] Hiện tượng cũng xảy ra tương tự khi nối 2 quả cầu như trên bằng dây dẫn mảnh rồi cắt bỏ dây nối.
	\item[$\bullet$] Khi chạm tay vào 1 quả cầu nhỏ dẫn điện đã tích điện thì quả cầu mất điện tích \& trở thành trung hòa.'' -- \cite[p. 7]{Giai_Toan_Vat_Ly_11_tap_1}
\end{enumerate}

\begin{baitoan}[\cite{Giai_Toan_Vat_Ly_11_tap_1}, \textbf{1.1.}, p. 7]
	2 quả cầu kim loại giống nhau, mang các điện tích $q_1,q_2$, đặt trong không khí, cách nhau 1 đoạn $R = 20$cm. Chúng hút nhau bằng lực $F = 3.6\cdot 10^{-4}{\rm N}$. Cho 2 quả cầu tiếp xúc nhau rồi lại đưa về khoảng cách cũ, chúng đẩy nhau bằng lực $F' = 2.025\cdot 10^{-4}{\rm N}$. Tính $q_1,q_2$.
\end{baitoan}
Không cho giá trị cụ thể, 1 tổng quát của bài toán trên:

\begin{baitoan}
	2 quả cầu kim loại giống nhau, mang các điện tích $q_1,q_2$, đặt trong không khí, cách nhau 1 đoạn $R\ {\rm m}$, $R > 0$. Chúng hút nhau bằng lực $F\ {\rm N}$, $F > 0$ . Cho 2 quả cầu tiếp xúc nhau rồi lại đưa về khoảng cách cũ, chúng đẩy nhau bằng lực $F'\ {\rm N}$, $F' > 0$. Tính $q_1,q_2$ (theo $R,F,F'$ đã cho).
\end{baitoan}

\begin{proof}[Giải]
	Ban đầu, $F$ là lực hút, $q_1$ \& $q_2$ trái dấu: $q_1q_2 < 0$. Sử dụng định luật Coulomb \ref{dinh luat: Coulomb}, \eqref{luc tuong tac giua 2 dien tich diem} cho $F = k\frac{|q_1q_2|}{R^2} = -k\frac{q_1q_2}{R^2}$, hay $q_1q_2 = -\frac{FR^2}{k}\ {\rm C}^2$. Cho 2 quả cầu tiếp xúc, điện tích trên các quả cầu được phân bố lại. Vì các quả cầu giống nhau nên các điện tích của chúng bằng nhau: $q_1' = q_2'$. Theo định luật bảo toàn điện tích \ref{dinh luat: bao toan dien tich}: $q_1' = q_2' = \frac{q_1 + q_2}{2}$. Vậy $F' = k\frac{|q_1'q_2'|}{R^2} = k\frac{(q_1 + q_2)^2}{4R^2}$, hay $(q_1 + q_2)^2 = \frac{4F'R^2}{k}\ {\rm C^2}$, hay $|q_1 + q_2| = 2R\sqrt{\frac{F'}{k}}$ C. Vì $q_1,q_2$ trái dấu nên chưa biết được dấu của tổng $q_1 + q_2$. Kết hợp những điều thu được cho hệ phương trình:
	\begin{equation*}
		\left\{\begin{split}
			|q_1 + q_2| &= 2R\sqrt{\frac{F'}{k}}\ ({\rm C}),\\
			q_1q_2 &= -\frac{FR^2}{k}\ ({\rm C}^2).
		\end{split}\right.
	\end{equation*}
	Xét 2 trường hợp sau tương ứng với dấu của $q_1 + q_2$:
	\begin{itemize}
		\item \textit{Trường hợp 1: $q_1 + q_2 = 2R\sqrt{\frac{F'}{k}}$ C.} Khi đó, theo định lý Vi\`ete, $q_1,q_2$ là 2 nghiệm của phương trình bậc 2: $x^2 - 2R\sqrt{\frac{F'}{k}}x - \frac{FR^2}{k} = 0$. Biệt thức rút gọn $\Delta' = \frac{R^2(F + F')}{k} > 0$ (vì $F,F',k > 0$), phương trình có 2 nghiệm thực là $q_{1,2} = R\left(\sqrt{\frac{F'}{k}}\pm\sqrt{\frac{F + F'}{k}}\right)$.
		\item \textit{Trường hợp 2: $q_1 + q_2 = -2R\sqrt{\frac{F'}{k}}$ C.} Tương tự như trường hợp 1, $q_1,q_2$ là 2 nghiệm của phương trình bậc 2: $x^2 + 2R\sqrt{\frac{F'}{k}}x - \frac{FR^2}{k} = 0$. Biệt thức rút gọn $\Delta' = \frac{R^2(F + F')}{k} > 0$ (vì $F,F',k > 0$), phương trình có 2 nghiệm thực là $q_{1,2} = -R\left(\sqrt{\frac{F'}{k}}\pm\sqrt{\frac{F + F'}{k}}\right)$.
	\end{itemize}
	Vậy $(q_1,q_2)$ có thể bằng 1 trong 4 cặp giá trị:
	
	$\left(R\left(\sqrt{\frac{F'}{k}} + \sqrt{\frac{F + F'}{k}}\right),R\left(\sqrt{\frac{F'}{k}} - \sqrt{\frac{F + F'}{k}}\right)\right)$, $\left(R\left(\sqrt{\frac{F'}{k}} - \sqrt{\frac{F + F'}{k}}\right),R\left(\sqrt{\frac{F'}{k}} + \sqrt{\frac{F + F'}{k}}\right)\right)$,
	
	$\left(-R\left(\sqrt{\frac{F'}{k}} + \sqrt{\frac{F + F'}{k}}\right),-R\left(\sqrt{\frac{F'}{k}} - \sqrt{\frac{F + F'}{k}}\right)\right)$, $\left(-R\left(\sqrt{\frac{F'}{k}} - \sqrt{\frac{F + F'}{k}}\right),-R\left(\sqrt{\frac{F'}{k}} + \sqrt{\frac{F + F'}{k}}\right)\right)$.
\end{proof}
Với $R = 0.2$ m, $F = 3.6\cdot 10^{-4}$N, $F' = 2.025\cdot 10^{-4}$N, bài toán trên trở thành \cite[\textbf{1.1}, p. 7]{Giai_Toan_Vat_Ly_11_tap_1}. Thay số vào kết quả vừa thu được, ta được $(q_1,q_2)$ có thể là 1 trong 4 cặp giá trị sau: $(\pm8\cdot 10^{-8},\mp2\cdot 10^{-8})$, $(\pm2\cdot 10^{-8},\mp8\cdot 10^{-8})$.

\begin{baitoan}[\cite{Giai_Toan_Vat_Ly_11_tap_1}, \textbf{1.2.}, p. 9]
	2 điện tích điểm đặt trong không khí, cách nhau khoảng $R = 20\ {\rm cm}$. Lực tương tác tĩnh điện giữa chúng có 1 giá trị nào đó. Khi đặt trong dầu, ở cùng khoảng cách, lực tương tác tĩnh điện giữa chúng giảm $4$ lần. Hỏi khi đặt trong dầu, khoảng cách giữa các điện tích phải là bao nhiêu để lực tương tác giữa chúng bằng lực tương tác ban đầu trong không khí.
\end{baitoan}
Không cho giá trị cụ thể, 1 tổng quát của bài toán trên:

\begin{baitoan}
	2 điện tích điểm đặt trong không khí, cách nhau khoảng $R\ {\rm m}$. Lực tương tác tĩnh điện giữa chúng có 1 giá trị nào đó. Khi đặt trong dầu, ở cùng khoảng cách, lực tương tác tĩnh điện giữa chúng giảm $n$ lần. Hỏi khi đặt trong dầu, khoảng cách giữa các điện tích phải là bao nhiêu để lực tương tác giữa chúng bằng lực tương tác ban đầu trong không khí.
\end{baitoan}

\begin{proof}[Giải]
	Đặt $F,F'$ lần lượt là độ lớn của lực tương tác tĩnh điện giữa 2 điện tích $q_1,q_2$ khi chúng được đặt cách nhau khoảng $R$ trong không khí \& dầu. Sử dụng định luật Coulomb \ref{dinh luat: Coulomb}, \eqref{luc tuong tac giua 2 dien tich diem} cho $F = k\frac{|q_1q_2|}{r^2}$, $F' = k\frac{|q_1q_2|}{\varepsilon r^2}$. Suy ra $\frac{F'}{F} = \frac{1}{\varepsilon}$, kết hợp với giả thiết, suy ra $\varepsilon = n$. Đặt $r'$ là khoảng cách trong dầu của 2 điện tích để lực tương tác tĩnh điện vẫn như trong không khí lúc chúng cách nhau khoảng $r$, $F = k\frac{|q_1q_2|}{r^2} = k\frac{|q_1q_2|}{\varepsilon(r')^2}$, suy ra $r' = \frac{r}{\sqrt{\varepsilon}}$.
\end{proof}

%------------------------------------------------------------------------------%

\section{Điện Trường}

\subsection{Điện trường}

\subsubsection{Khái niệm điện trường}
``1 vật tác dụng lực hấp dẫn lên các vật khác ở gần nó vì xung quanh vật đó có trường hấp dẫn. Ở đây ta cũng có hiện tượng tương tự. \textit{1 điện tích tác dụng lực điện lên các điện tích khác ở gần nó. Ta nói, xung quanh điện tích có điện trường}. Các điện tích tương tác được với nhau là vì điện trường của điện tích này tác dụng lên điện tích kia. Hiện nay, khoa học chứng tỏ những điều trên là đúng.'' ``Con cá mập đầu búa có thể nhận biết được điện trường.'' -- \cite[p. 13]{SGK_Vat_Ly_11_nang_cao}

\subsubsection{Tính chất cơ bản của điện trường}
``\textit{Tính chất cơ bản của điện trường là nó tác dụng lực điện lên điện tích đặt trong nó}. 1 vật có kích thước nhỏ, mang 1 điện tích nhỏ, được dùng để phát hiện lực điện tác dụng lên nó gọi là \textit{điện tích thử}. Người ta dùng điện tích thử để nhận biết điện trường.'' `Trong \cite[Chap. 1]{SGK_Vat_Ly_11_nang_cao}, ta chỉ xét điện trường của các điện tích đứng yên đối với nhau, i.e., \textit{điện trường tĩnh}, gọi tắt là \textit{điện trường}.'' -- \cite[p. 13]{SGK_Vat_Ly_11_nang_cao}

\subsection{Cường độ điện trường}
``Giả sử ta có 1 số điện tích thử $q_1,q_2,q_3,\ldots$ Đặt lần lượt các điện tích này tại 1 điểm nhất định trong điện trường \& xác định các lực $\overrightarrow{F_1},\overrightarrow{F_2},\overrightarrow{F_3},\ldots$ tác dụng lên chúng. Thí nghiệm cho biết các lực $\overrightarrow{F_1},\overrightarrow{F_2},\overrightarrow{F_3}$, có độ lớn khác nhau, nhưng các thương số dạng $\frac{F}{|q|}$ thì bằng nhau. Nếu để ý đến cả chiều của các lực tác dụng lên các điện tích thử thì các thương dạng $\frac{\overrightarrow{F}}{q}$ cũng không đổi, i.e., $\frac{\overrightarrow{F_i}}{q_i} = \mbox{const}$, $i\in\mathbb{N}$. Làm thí nghiệm ở các điểm khác nhau thì các thương $\frac{\overrightarrow{F}}{q}$ là khác nhau. Thương $\frac{\overrightarrow{F}}{q}$ đặc trưng cho điện trường ở điểm đang xét về mặt tác dụng lực gọi là \textit{cường độ điện trường} \& ký hiệu là $\overrightarrow{E}$.
\begin{align}
	\label{cuong do dien truong}
	\boxed{\overrightarrow{E} = \frac{\overrightarrow{F}}{q}.}
\end{align}
Trong trường hợp đã biết cường độ điện trường, thì từ công thức \eqref{cuong do dien truong} suy ra:
\begin{align}
	\label{cuong do dien truong 1}
	\overrightarrow{F} = q\overrightarrow{E}.
\end{align}
Từ \eqref{cuong do dien truong 1} ta thấy nếu $q > 0$ thì $\overrightarrow{F}$ cùng chiều với $\overrightarrow{E}$ (Fig. \ref{fig:chieu_luc_dien}), ngược lại nếu $q < 0$ thì $\overrightarrow{F}$ ngược chiều với $\overrightarrow{E}$ (Fig. \ref{fig:chieu_luc_dien}).

\begin{figure}[H]
	\centering
	\includegraphics[scale=0.15]{chieu_luc_dien}
	\caption{Chiều của lực điện tác dụng lên điện tích, \cite[Hình 3.1, p. 14]{SGK_Vat_Ly_11_nang_cao}.}
	\label{fig:chieu_luc_dien}
\end{figure}
Trong hệ SI, đơn vị cường độ điện trường có thể là Newton trên Coulomb, nhưng thường dùng đơn vị Von trên mét, ký hiệu là V\texttt{/}m.'' ``Cường độ điện trường ($\overrightarrow{E}$) là đại lượng vector, nhưng nhiều khi người ta cũng gọi độ lớn của $\overrightarrow{E}$, ký hiệu $E\coloneqq| \overrightarrow{E}|$, là cường độ điện trường.'' -- \cite[p. 14]{SGK_Vat_Ly_11_nang_cao}

\subsection{Đường sức điện}

\subsubsection{Định nghĩa}
``Có nhiều cách mô tả điện trường. Cách mô tả có tính trực quan rõ rệt là dùng cách vẽ các đường sức điện.

\begin{dinhnghia}[Đường sức điện]
	\emph{Đường sức điện} là đường được vẽ trong điện trường sao cho tiếp tuyến tại bất kỳ điểm nào trên đường cũng trùng với phương của vector cường độ điện trường tại điểm đó.
\end{dinhnghia}
Tuy nhiên, trong thực tế người ta thường quy định cho đường sức 1 chiều đi sao cho chiều của đường sức \& chiều của vector cường độ điện trường tại các điểm trên đường là trùng nhau. Khi đó, ta hiểu các đường sức là các đường có chiều xác định (Fig. \ref{fig:duong_suc_dien})

\begin{figure}[H]
	\centering
	\includegraphics[scale=0.15]{duong_suc_dien}
	\caption{Đường sức điện \& vector cường độ điện trường, \cite[Hình 3.2, p. 15]{SGK_Vat_Ly_11_nang_cao}.}
	\label{fig:duong_suc_dien}
\end{figure}
Các đường sức điện của 1 điện tích điểm \& của hệ 2 điện tích điểm được trình bày trên Figs. \ref{fig:duong_suc_dien_tich_diem}--\ref{fig:duong_suc_he_2_dien_tich_diem}.'' --  \cite[pp. 14--15]{SGK_Vat_Ly_11_nang_cao}

\begin{figure}[H]
	\centering
	\begin{subfigure}{.5\textwidth}
		\centering
		\includegraphics[width=.3\linewidth]{duong_suc_dien_tich_diem_duong}
		\caption{Đường sức của 1 điện tích điểm dương.}
	\end{subfigure}%
	\begin{subfigure}{.5\textwidth}
		\centering
		\includegraphics[width=.3\linewidth]{duong_suc_dien_tich_diem_am}
		\caption{Đường sức của 1 điện tích điểm âm.}
	\end{subfigure}
	\caption{Đường sức của 1 điện tích điểm, \cite[Hình 3.3, p. 15]{SGK_Vat_Ly_11_nang_cao}.}
	\label{fig:duong_suc_dien_tich_diem}
\end{figure}

\begin{figure}[H]
	\centering
	\begin{subfigure}{.5\textwidth}
		\centering
		\includegraphics[width=.4\linewidth]{duong_suc_he_2_dien_tich_diem_duong}
		\caption{2 điện tích điểm dương.}
	\end{subfigure}%
	\begin{subfigure}{.5\textwidth}
		\centering
		\includegraphics[width=.4\linewidth]{duong_suc_he_2_dien_tich_diem_trai_dau}
		\caption{2 điện tích trái dấu.}
	\end{subfigure}
	\caption{Đường sức của hệ 2 điện tích điểm, \cite[Hình 3.4, p. 15]{SGK_Vat_Ly_11_nang_cao}.}
	\label{fig:duong_suc_he_2_dien_tich_diem}
\end{figure}

\subsubsection{Các tính chất của đường sức điện}
``Các đường sức điện có 1 số tính chất sau đây:
\begin{itemize}
	\item \textit{Tại mỗi điểm trong điện trường, ta có thể vẽ được 1 \& chỉ 1 đường sức điện đi qua.}
	\item \textit{Các đường sức điện là các đường cong không kín. Nó xuất phát từ các điện tích dương \& tận cùng ở các điện tích âm (hoặc ở vô cực)}. Trong trường hợp chỉ có 1 điện tích, thì các đường sức xuất phát từ điện tích dương ra vô cực, hoặc từ vô cực đến điện tích âm (Fig. \ref{fig:duong_suc_dien_tich_diem}).
	\item \textit{Nơi nào cường độ điện trường lớn hơn thì các đường sức điện ở đó được vẽ mau hơn (dày hơn), nơi nào cường độ điện trường nhỏ hơn thì các đường sức điện ở đó được vẽ thưa hơn}. E.g., trên các Figs. \ref{fig:duong_suc_dien_tich_diem}--\ref{fig:duong_suc_he_2_dien_tich_diem}, ở nơi gần điện tích, các đường sức điện mau hơn nơi xa điện tích.'' --  \cite[p. 15]{SGK_Vat_Ly_11_nang_cao}
\end{itemize}

\subsubsection{Điện phổ}
``Dùng 1 loại bột cách điện rắc vào dầu cách điện \& khuấy đều. Sau đó đặt 1 quả cầu nhỏ nhiễm điện vào trong dầu. Gõ nhẹ vào khay dầu thì các hạt bột sẽ sắp xếp thành các ``đường hạt bột''. Ta gọi hệ các ``đường hạt bột'' đó là \textit{điện phổ} của quả cầu nhiễm điện. Điện phổ cho phép ta hình dung dạng \& sự phân bố các đường sức điện. Đường sức điện vẽ trong các Figs. \ref{fig:duong_suc_dien_tich_diem}--\ref{fig:duong_suc_he_2_dien_tich_diem} tương ứng với các điện phổ ở \cite[Hình 3.5: \textsf{Điện phổ của 1 quả cầu nhiễm điện} \& Hình 3.6: \textsf{Điện phổ của 2 quả cầu nhiễm điện cùng \& trái dấu}, p. 16]{SGK_Vat_Ly_11_nang_cao}

\subsection{Điện trường đều}

\begin{dinhnghia}[Điện trường đều]
	1 điện trường mà vector cường độ điện trường tại mọi điểm đều bằng nhau gọi là \emph{điện trường đều}.
\end{dinhnghia}
Theo tính chất của đường sức, ta suy ra các đường sức của điện trường đều là các đường thẳng song song \& cách đều nhau. \cite[Hình 3.7: \textsf{Điện phổ của điện trường ở giữa 2 tấm kim loại phẳng, rộng, song song, mang điện tích trái dấu, có độ lớn bằng nhau}, p. 16]{SGK_Vat_Ly_11_nang_cao} cho biết điện phổ của 2 tấm kim loại phẳng, rộng, song song, mang điện tích trái dấu, có độ lớn bằng nhau. Ở rìa của 2 tấm kim loại, các ``đường hạt bột'' là các đường cong, còn ở giữa 2 tấm, các ``đường hạt bột'' song song \& cách đều nhau. Dựa vào điện phổ, ta có thể nói điện trường giữa 2 tấm kim loại là \textit{điện trường đều}. Đường sức của điện trường này được vẽ trên Fig. \ref{fig:duong_suc_dien_truong_deu}.'' -- \cite[p. 16]{SGK_Vat_Ly_11_nang_cao}

\begin{figure}[H]
	\centering
	\includegraphics[scale=0.15]{duong_suc_dien_truong_deu}
	\caption{Các đường sức ở giữa 2 tấm kim loại phẳng, rộng, song song, mang điện tích trái dấu, có độ lớn bằng nhau. Các đường sức này song song với nhau \& cách đều nhau, \cite[Hình 3.8, p. 16]{SGK_Vat_Ly_11_nang_cao}.}
	\label{fig:duong_suc_dien_truong_deu}
\end{figure}

\subsection{Điện trường của 1 điện tích điểm}
``2 điện tích điểm $q,Q$ đặt cách nhau 1 khoảng $r$ trong chân không thì lực Coulomb tác dụng lên điện tích $q$ được viết dưới dạng:
\begin{align*}
	F = 9\cdot 10^9\frac{|qQ|}{r^2}.
\end{align*}
Từ công thức \eqref{cuong do dien truong} ta suy ra cường độ điện trường của điện tích điểm $Q$ tại 1 điểm là:
\begin{align}
	\label{cuong do dien truong 2}
	E = 9\cdot 10^9\frac{|Q|}{r^2},
\end{align}
$r$ là khoảng cách từ điểm khảo sát đến điện tích $Q$. Nếu $Q > 0$ thì cường độ điện trường hướng ra xa điện tích $Q$ (Fig. \ref{fig:chieu_cua_vector_cuong_do_dien_truong_cua_dien_tich_diem_duong}), nếu $Q < 0$ thì cường độ điện trường hướng về phía điện tích $Q$ (Fig. \ref{fig:chieu_cua_vector_cuong_do_dien_truong_cua_dien_tich_diem_am}).'' -- \cite[pp. 16--17]{SGK_Vat_Ly_11_nang_cao}
	
\begin{figure}[H]
	\centering
	\begin{subfigure}{.5\textwidth}
		\centering
		\includegraphics[width=.35\linewidth]{chieu_cua_vector_cuong_do_dien_truong_cua_dien_tich_diem_duong}
		\caption{$Q > 0$.}
		\label{fig:chieu_cua_vector_cuong_do_dien_truong_cua_dien_tich_diem_duong}
	\end{subfigure}%
	\begin{subfigure}{.5\textwidth}
		\centering
		\includegraphics[width=.35\linewidth]{chieu_cua_vector_cuong_do_dien_truong_cua_dien_tich_diem_am}
		\caption{$Q < 0$.}
		\label{fig:chieu_cua_vector_cuong_do_dien_truong_cua_dien_tich_diem_am}
	\end{subfigure}
	\caption{Chiều của vector cường độ điện trường của điện tích điểm phụ thuộc vào dấu của điện tích, \cite[Hình 3.9, p. 17]{SGK_Vat_Ly_11_nang_cao}.}
	\label{fig:chieu_cua_vector_cuong_do_dien_truong_cua_dien_tich_diem}
\end{figure}

\subsection{Nguyên lý chồng chất điện trường}

\begin{nguyenly}[Nguyên lý chồng chất điện trường]
	Giả sử ta có hệ $n$ điện tích điểm $\{Q_i\}_{i=1}^n$. Gọi cường độ điện trường của hệ ở 1 điểm nào đó là $\overrightarrow{E}$. Cường độ điện trường chỉ của điện tích $Q_i$ là $\overrightarrow{E_i}$ tại điểm đang xét, $i = 1,\ldots, n$. Khi đó ta có:
	\begin{align}
		\label{nguyen ly chong chat dien truong}
		\overrightarrow{E} = \sum_{i=1}^n \overrightarrow{E_i} = \overrightarrow{E_1} + \cdots + \overrightarrow{E_n}.
	\end{align}
\end{nguyenly}

\subsection{Tương tác gần \& tương tác xa}
``Tương tác giữa 2 vật không tiếp xúc với nhau được thực hiện bằng cách nào? Có 2 cách giải  đáp câu hỏi đó. Cách giải đáp thứ nhất cho rằng 2 vật không tiếp xúc với nhau vẫn có thể tương tác với nhau. Quan điểm này gọi là \textit{quan điểm tương tác xa}. Định luật vạn vật hấp dẫn \& định luật Coulomb thể hiện quan điểm đó.

Cách giải đáp thứ 2 cho rằng, có tương tác hấp dẫn giữa 2 vật không tiếp xúc với nhau là vì vật này được đặt trong trường hấp dẫn của vật kia, lực hấp dẫn tác dụng lên vật $B$ là do trường hấp dẫn của vật $A$ tại điểm đặt vật $B$ gây ra. Tương tự như vậy, nếu có 2 điện tích $A$ \& $B$, thì có lực điện tác dụng lên điện tích $B$ là vì $B$ được đặt trong điện trường của điện tích $A$. Điện trường của điện tích $A$ là thực thể vật lý truyền lực điện từ điện tích $A$ đến điện tích $B$. Quan điểm này gọi là \textit{quan điểm tương tác gần}. Theo quan điểm tương tác xa thì tốc độ truyền tương tác là vô hạn. Điều đó trái với thực tế. Còn theo quan điểm tương tác gần thì tốc độ truyền tương tác là hữu hạn. Nhiều sự kiện thực nghiệm đã chứng tỏ quan điểm tương tác gần là phù hợp với thực tế.'' -- \cite[p. 18]{SGK_Vat_Ly_11_nang_cao}

%------------------------------------------------------------------------------%

\section{Công của Lực Điện. Hiệu Điện Thế}

\begin{cauhoi}
	``Công của trọng lực được biểu diễn qua hiệu thế năng hấp dẫn. Còn công của lực điện có thể biểu diễn qua đại lượng nào?'' -- \cite[p. 19]{SGK_Vat_Ly_11_nang_cao}
\end{cauhoi}

\subsection{Công của lực điện}
``Ta xét công của lực điện tác dụng lên 1 điện tích $q$ chuyển động từ $M$ đến $N$ trong điện trường đều, e.g. điện trường giữa 2 tấm kim loại rộng, song song, mang điện tích trái dấu có độ lớn bằng nhau. Giả sử $q > 0$ \& đường đi của điện tích $q$ là đoạn đường cong $MN$ (Fig. \ref{fig:cong_luc_dien}).

\begin{figure}[H]
	\centering
	\includegraphics[scale=0.15]{cong_luc_dien}
	\caption{Công của lực điện tác dụng lên 1 điện tích $q$ chuyển động từ $M$ đến $N$ trong điện trường đều giữa 2 tấm kim loại rộng, song song, mang điện tích trái dấu có độ lớn bằng nhau, \cite[Hình 4.1, p. 19]{SGK_Vat_Ly_11_nang_cao}.}
	\label{fig:cong_luc_dien}
\end{figure}
Để tính công của lực điện trên đoạn đường cong $MN$, ta chia $MN$ thành nhiều đoạn nhỏ, công của lực điện tác dụng lên $q$ bằng tổng các công trên các đoạn nhỏ đó. Vì $q > 0$ nên lực điện tác dụng lên $q$ có chiều hướng từ tấm mang điện tích dương sang tấm mang điện tích âm. Coi rằng đoạn đường cong $MN$ được chia thành nhiều đoạn nhỏ sao cho mỗi đoạn nhỏ đó có thể coi là đoạn thẳng\footnote{Quá trình phân 1 đường cong ra thành những ``đoạn thẳng đủ nhỏ'' này thường được dùng trong vi phân \& tích phân, xem chương trình Toán 11 \& Toán 12 \& tài liệu của tác giả: \href{https://github.com/NQBH/hobby/blob/master/elementary_mathematics/grade_11/NQBH_elementary_mathematics_grade_11.pdf}{GitHub\texttt{/}NQBH\texttt{/}hobby\texttt{/}elementary mathematics\texttt{/}grade 11\texttt{/}lecture} \& \href{https://github.com/NQBH/hobby/blob/master/elementary_mathematics/grade_12/NQBH_elementary_mathematics_grade_12.pdf}{GitHub\texttt{/}NQBH\texttt{/}hobby\texttt{/}elementary mathematics\texttt{/}grade 12\texttt{/}lecture}.}. Khi đó công thức tính công trên 1 đoạn nhỏ nào đó, e.g., đoạn $PQ$, là:
\begin{align*}
	\Delta A_{PQ} = qE\cdot PQ\cdot\cos\alpha = qE\cdot\overline{P'Q'},
\end{align*}
ở đây $\overline{P'Q'}$ là hình chiếu của $PQ$ lên trục $Ox$; quy ước vẽ trục $Ox$ có chiều trùng với chiều của đường sức. Công trên toàn đoạn $MN$ bằng:
\begin{align}
	A_{MN} = \sum \Delta A = qE\left(\overline{M'R'} + \cdots + \overline{P'Q'} + \cdots + \overline{S'N'}\right) = qE\cdot\overline{M'N'}.
\end{align}
Kết quả trên đây được rút ra từ giả thiết $q > 0$. Tuy nhiên, nếu $q < 0$ ta cũng rút ra được công thức như trên. Do đó có thể viết:
\begin{align}
	\label{cong cua luc dien}
	A_{MN} = qE\cdot\overline{M'N'},
\end{align}
$M',N'$ là hình chiếu của 2 điểm $M,N$ lên trục $Ox$; $\overline{M'N'}$ là độ dài đại số của đoạn $M'N'$; còn $q$ có dấu tùy ý. Từ  \eqref{cong cua luc dien} ta có nhận xét là, công của lực điện tác dụng lên điện tích $q$ không phụ thuộc vào dạng của đoạn đường đi $MN$ mà chỉ phụ thuộc vào vị trí của 2 điểm $M,N$, i.e., của điểm đầu \& điểm cuối của đường đi. Người ta đã chứng minh nhận xét trên đây cũng đúng cả trong trường hợp điện trường không đều.

\begin{dinhluat}
	Công của lực điện tác dụng lên 1 điện tích không phụ thuộc dạng đường đi của điện tích mà chỉ phụ thuộc vào vị trí điểm đầu \& điểm cuối của đường đi trong điện trường.
\end{dinhluat}
Do đó, người ta nói \textit{điện trường tĩnh là 1 trường thế năng} (tương tự như trường hấp dẫn).'' ``Trong \textit{Cơ học}\texttt{/}Mechanics\footnote{\textbf{mechanics} [n] \textbf{1.} [uncountable] the science of movement \& force; \textbf{2.} [plural] \textbf{mechanics of something} the way something works or is done.} ta cũng đã rút ra kết luận là \textit{công của lực hấp dẫn không phụ thuộc vào dạng đường đi của vật mà chỉ phụ thuộc vào điểm đầu \& điểm cuối của đường đi}.'' -- \cite[pp. 19--20]{SGK_Vat_Ly_11_nang_cao}

\subsection{Khái niệm hiệu điện thế}

\subsubsection{Công của lực điện \& hiệu thế năng của điện tích}
``Công của trọng lực \& công của lực điện cùng có 1 đặc tính quan trọng là những công này không phụ thuộc dạng đường đi của vật mà chỉ phụ thuộc vị trí điểm đầu \& điểm cuối của đường đi. Ta đã biết, công của trọng lực được biểu diễn qua hiệu thế năng tại vị trí đầu \& cuối đường đi của vật đó (\cite[\S35, pp. 164--166]{SGK_Vat_Ly_10_nang_cao}). Ở đây, ta cũng coi 1 điện tích $q$ ở trong điện trường thì có thế năng, \& công của lực điện khi điện tích $q$ di chuyển từ điểm $M$ đến điểm $N$ cũng được biểu diễn qua hiệu của các thế năng của điện tích $q$ tại 2 điểm đó: $A_{MN} = W_M - W_N$.'' -- \cite[p. 20]{SGK_Vat_Ly_11_nang_cao}

\subsubsection{Hiệu điện thế, điện thế}
``Hiệu thế năng của vật trong trọng trường tỷ lệ với khối lượng $m$ của vật. Ở đây, ta cũng coi hiệu thế năng của điện tích $q$ trong điện trường tỷ lệ với điện tích $q$, i.e., có thể biểu diễn $A_{MN}$ dưới dạng sau:
\begin{align}
	\label{hieu the nang}
	A_{MN} = q(V_M - V_N),
\end{align}
$(V_M - V_N)$ được gọi là \textit{hiệu điện thế} (hay \textit{điện áp}) giữa 2 điểm $M,N$ \& ký hiệu là $U_{MN}$. Từ \eqref{hieu the nang} rút ra công thức sau đây được coi là công thức định nghĩa hiệu điện thế:
\begin{align}
	\label{hieu dien the}
	U_{MN} = V_M - V_N = \frac{A_{MN}}{q}.
\end{align}
\textit{Hiệu điện thế giữa 2 điểm trong điện trường là đại lượng đặc trưng cho khả năng thực hiện công của điện trường khi có 1 điện tích di chuyển giữa 2 điểm đó}. Các đại lượng $V_M,V_N$ được gọi là \textit{điện thế của điện trường} tại điểm $M,N$ tương ứng. Điện thế của điện trường phụ thuộc vào cách chọn mốc tính điện thế. Thường người ta chọn điện thế ở xa vô cực làm mốc. Cũng có khi người ta chọn điện thế ở mặt đất làm mốc (i.e., coi điện thế ở mặt đất bằng $0$). Vì vậy, khi nói tới điện thế tại 1 điểm $A$ nào đó thì thực chất đó là hiệu điện thế $V_A - V_B$, trong đó $V_B$ là điện thế được chọn làm mốc, i.e., $V_B = 0$.

Trong hệ SI, đơn vị điện thế \& hiệu điện thế là vôn ký hiệu là V. Từ công thức \eqref{hieu the nang} suy ra, nếu $U_{MN} = 1$ V, $q = 1$ C thì $A_{MN} = 1$ J. Vậy vôn là hiệu điện thế giữa 2 điểm $M,N$ mà khi 1 điện tích dương 1 C di chuyển từ điểm $M$ đến điểm $N$ thì lực điện sẽ thực hiện 1 công dương là $1$ J. Để đo hiệu điện thế giữa 2 vật, người ta dùng \textit{tĩnh điện kế} (Fig. \ref{fig:tinh_dien_ke}). Trong kỹ thuật, hiệu điện thế gọi là \textit{điện áp}.'' ``Muốn đo hiệu điện thế giữa 2 vật, ta nối 1 vật với cần của tĩnh điện kế, vật kia với vỏ. Độ lệch của kim cho biết hiệu điện thế giữa 2 vật đó.'' -- \cite[p. 21]{SGK_Vat_Ly_11_nang_cao}

\begin{figure}[H]
	\centering
	\includegraphics[scale=0.15]{tinh_dien_ke}
	\caption{Tĩnh điện kế: 1. Kim của tĩnh điện kế; 2. Trục quay của kim; 3. Thanh kim loại, gọi là \textit{cần của tĩnh điện kế}; 4. Vỏ tĩnh điện kế bằng kim loại, \cite[Hình 4.2, p. 21]{SGK_Vat_Ly_11_nang_cao}.}
	\label{fig:tinh_dien_ke}
\end{figure}

\subsection{Liên hệ giữa cường độ điện trường \& hiệu điện thế}
``So sánh 2 công thức \eqref{hieu the nang} \& \eqref{hieu dien the} ta rút ra:
\begin{align}
	\label{lien he giua cuong do dien truong & hieu dien the doi voi dien truong deu}
	E = \frac{U_{MN}}{\overline{M'N'}}.
\end{align}
Đó là công thức biểu thị mối liên hệ giữa cường độ điện trường \& hiệu điện thế đối với điện trường đều. Các điểm $M,N,M',N'$ được chỉ rõ trên Fig. \ref{fig:cong_luc_dien}. Từ \eqref{lien he giua cuong do dien truong & hieu dien the doi voi dien truong deu} ta hiểu tại sao đơn vị cường độ điện trường là vôn trên mét. Mối liên hệ giữa cường độ điện trường \& hiệu điện thế thường được viết dưới dạng đơn giản như sau:
\begin{align}
	\label{lien he giua cuong do dien truong & hieu dien the}
	E = \frac{U}{d},
\end{align}
$d$ là \textit{khoảng cách} giữa 2 điểm $M',N'$, i.e., $d\coloneqq = M'N' = |\overrightarrow{M'N'}$.'' -- \cite[pp. 21--22]{SGK_Vat_Ly_11_nang_cao} Từ công thức \eqref{lien he giua cuong do dien truong & hieu dien the}, điện thế giảm theo chiều của đường sức.

\subsection{Thí nghiệm Millikan}
``Thí nghiệm Millikan (Robert Andrews Millikan, 1868--1953, nhà vật lý người Mỹ, giải Nobel năm 1925) nhằm xác định điện tích nhỏ nhất trong tự nhiên. Sơ đồ thí nghiệm Millikan được trình bày trên Fig. \ref{fig:so_do_thi_nghiem_Millikan}.

\begin{figure}[H]
	\centering
	\includegraphics[scale=0.15]{so_do_thi_nghiem_Millikan}
	\caption{Sơ đồ thí nghiệm Millikan: 1, 2. 2 tấm kim loại; 3. Máy phun; 4. Lỗ nhỏ; 5. Đèn chiếu sáng; 6. Kính quan sát, \cite[Hình 4.6, p. 24]{SGK_Vat_Ly_11_nang_cao}.}
	\label{fig:so_do_thi_nghiem_Millikan}
\end{figure}
Trong hình đó 1 \& 2 là 2 tấm kim loại đường kính khoảng 20 cm. 2 tấm được đặt nằm ngang \& cách nhau chừng 2 cm. Dùng máy phun, phun vào lỗ nhỏ ở tấm kim loại 1 những hạt dầu có kích thước rất nhỏ (vào cỡ 1 $\rm\mu m$). Do cọ xát với miệng vòi phun mà 1 số hạt dầu được nhiễm điện. Qua lỗ nhỏ có 1 số hạt dầu rời vào bên trong khoảng không gian giữa 2 tấm kim loại. Dùng kính quan sát các hạt đó trong khoảng thời gian chừng vài ba giờ. Đầu tiên, khi 2 tấm 1 \& 2 chưa nối với nguồn, ta thấy những hạt dầu rơi xuống với tốc độ lớn dần. Sau đó tốc độ rơi của chúng không đổi. Đó là lúc lực ma sát cân bằng với lực hấp dẫn. Ở đây lực ma sát tỷ lệ với tốc độ rơi của hạt. Ký hiệu tốc độ không đổi này là $v$ thì $f_{\rm ms} = kv$. Ta có hệ thức sau: $mg = f_{\rm ms} = kv$  ($k$ là 1 hệ số tỷ lệ). Bây giờ nối tấm 1 với cực dương \& tấm 2 với cực âm của 1 nguồn điện. Khi đó có 1 số hạt không rơi xuống mà lại chuyển động từ dưới lên trên, đó là những hạt nhiễm điện âm. Khi những hạt này đạt đến tốc độ không đổi $v_1$, ta có thể viết công thức sau:
\begin{align}
	\label{SGK Vat ly 11 (1) p. 24}
	q\frac{U}{d} = mg + kv_1,
\end{align}
trong đó $q$ là điện tích của hạt dầu, $U$ là hiệu điện thế giữa 2 tấm kim loại, $d$ là khoảng cách giữa 2 tấm đó.

Ion hóa không khí trong khoảng không gian giữa 2 tấm kim loại (bằng tia $X$, tia phóng xạ, $\ldots$) thì có 1 số hạt thay đổi tốc độ đột ngột do chúng nhậm thêm hạt mang điện từ không khí. Khi những hạt nhận thêm điện tích này đạt đến tốc độ không đổi $v_2$, ta có thể viết hệ thức sau:
\begin{align}
	\label{SGK Vat ly 11 (2) p. 24}
	(q + q_n)\frac{U}{d} = mg + kv_2,
\end{align}
ở đây $q_n$ là điện tích mà hạt dầu nhận thêm được. Từ \eqref{SGK Vat ly 11 (1) p. 24} \& \eqref{SGK Vat ly 11 (2) p. 24} ta rút ra:
\begin{align*}
	\frac{U}{d}q_n = k(v_2 - v_1),
\end{align*}
$d$ \& $U$ là những đại lượng có thể đo đươc, $v_2$ \& $v_1$ có thể xác định được bằng kính quan sát. Còn $k$ phải xác định bằng những phương pháp riêng xuất phát từ hệ thức $mg = kv$ (ở đây không nói đến). Từ đó ta tìm được $q_n$.

Trong khoảng thời gian 1909--1913, Millikan \& các cộng sự của ông đã đo điện tích của chừng vài nghìn hạt. Ông nhận thấy rằng, không có hạt nào có điện tích nhỏ hơn $1.6\cdot 10^{-19}$ C \& điện tích của các hạt đều bằng 1 số nguyên lần $1.6\cdot 10^{-19}$ C. Từ đó ông rút ra kết luận rằng, trong tự nhiên tồn tại điện tích nguyên tố ($1.6\cdot 10^{-19}$ C).'' -- \cite[pp. 23--24]{SGK_Vat_Ly_11_nang_cao}

%------------------------------------------------------------------------------%

\section{Bài Tập về Lực Coulomb \& Điện Trường}

\begin{baitoan}[\cite{SGK_Vat_Ly_11_nang_cao}, \textbf{1.}, p. 25]
	Cho 2 điện tích dương $q_1 = 2\ {\rm nC}$ \& $q_2 = 18\ {\rm nC}$ đặt cố định trong không khí \& cách nhau $10\ {\rm cm}$. Đặt thêm 1 điện tích thứ 3 $q_0$ tại 1 điểm trên đường thẳng nối 2 điện tích $q_1,q_2$ sao cho $q_0$ nằm cân bằng. Tìm:
	\begin{enumerate*}
		\item[(a)] Vị trí đặt $q_0$.
		\item[(b)] Dấu \& độ lớn của $q_0$.
	\end{enumerate*}
\end{baitoan}
Không cho giá trị cụ thể, 1 tổng quát của bài toán trên:
\begin{baitoan}
	Cho 2 điện tích dương $q_1\ {\rm C}$ \& $q_2\ {\rm C}$ đặt cố định trong không khí \& cách nhau $a\ {\rm m}$. Đặt thêm 1 điện tích thứ 3 $q_0$ tại 1 điểm trên đường thẳng nối 2 điện tích $q_1,q_2$ sao cho $q_0$ nằm cân bằng. Tìm:
	\begin{enumerate*}
		\item[(a)] Vị trí đặt $q_0$.
		\item[(b)] Dấu \& độ lớn của $q_0$.
	\end{enumerate*}
\end{baitoan}

\begin{figure}[H]
	\centering
	\begin{subfigure}{.5\textwidth}
		\centering
		\includegraphics[width=.45\linewidth]{3_dien_tich_thang_hang_a}
		\caption{$q_0 < 0$.}
	\end{subfigure}%
	\begin{subfigure}{.5\textwidth}
		\centering
		\includegraphics[width=.45\linewidth]{3_dien_tich_thang_hang_b}
		\caption{$q_0 > 0$.}
	\end{subfigure}
	\caption{\cite[Hình 5.1, p. 5.1]{SGK_Vat_Ly_11_nang_cao}.}
	\label{fig:3_dien_tich_thang_hang}
\end{figure}

\begin{proof}[Giải]
	\begin{enumerate*}
		\item[(a)] Giả sử $q_1$ \& $q_2$ được đặt như trên Fig. \ref{fig:3_dien_tich_thang_hang}. Để $q_0$ nằm cân bằng thì $q_0$ phải nằm bên trong 2 điện tích $q_1,q_2$. Gọi khoảng cách giữa $q_0$ \& $q_1$ là $x$. Gọi độ lớn của lực Coulomb mà $q_1,q_2$ tác dụng lên $q_0$ là $F_1,F_2$ tương ứng, ta có:
		\begin{enumerate*}
			\item[$\bullet$] Nếu $q_0 < 0$, $F_1 = k\frac{q_1|q_0|}{x^2}$, $F_2 = k\frac{q_2|q_0|}{(a - x)^2}$.
			\item[$\bullet$] Nếu $q_0 > 0$, $F_1 = k\frac{q_1q_0}{x^2}$, $F_2 = k\frac{q_2q_0}{(a - x)^2}$.
		\end{enumerate*}
		Muốn $q_0$ nằm cân bằng thì $F_1 = F_2$. Trong cả 2 trường hợp đều có thể rút ra: $\frac{q_1}{x^2} = \frac{q_2}{(a - x)^2}$, hay $q_1(a - x)^2 = q_2x^2$, vì $q_1 > 0$, $q_2 > 0$, $a > x$, phương trình bậc 2 trở thành $\sqrt{q_1}(a - x) = \sqrt{q_2}x$, hay $x = \frac{a\sqrt{q_1}}{\sqrt{q_1} + \sqrt{q_2}}$.
		\item[(b)] ``Kết quả tìm được trên đây không phụ thuộc vào dấu \& độ lớn của điện tích $q_0$. Vì vậy, dấu \& độ lớn của $q_0$ là tùy ý. Tuy nhiên, ta có thể thấy tính cân bằng của $q_0$ trong 2 trường hợp $q_0 > 0$ \& $q_0 < 0$ là khác nhau.'' -- \cite[p. 25]{SGK_Vat_Ly_11_nang_cao}
	\end{enumerate*}
\end{proof}

\begin{baitoan}[\cite{SGK_Vat_Ly_11_nang_cao}, \textbf{2.}, p. 25]
	Có 2 điện tích điểm $q_1 = 0.5\ {\rm nC}$ \& $q_2 = -0.5\ {\rm nC}$ đặt cách nhau 1 đoạn $a = 6\ {\rm cm}$ trong không khí. Xác định cường độ điện trường $\overrightarrow{E}$ tại điểm $M$ cách đều 2 điện tích $q_1,q_2$ \& cách đường thẳng nối $q_1,q_2$ 1 đoạn $l = 4\ {\rm cm}$.
\end{baitoan}

\begin{proof}[Đáp số]
	$E = 2160$ V\texttt{/}m.
\end{proof}
Không cho giá trị cụ thể, 1 tổng quát của bài toán trên:

\begin{baitoan}
	Có 2 điện tích điểm $q_1\ {\rm C}$ \& $q_2\ {\rm C}$, với $q_1 > 0$, $q_2 = -q_1$, đặt cách nhau 1 đoạn $a\ {\rm m}$ trong không khí. Xác định cường độ điện trường $\overrightarrow{E}$ tại điểm $M$ cách đều 2 điện tích $q_1,q_2$ \& cách đường thẳng nối $q_1,q_2$ 1 đoạn $l\ {\rm m}$.
\end{baitoan}

\begin{figure}[H]
	\centering
	\includegraphics[scale=0.15]{2_dien_tich_tam_giac_can}
	\caption{\cite[Hình 5.2, p. 26]{SGK_Vat_Ly_11_nang_cao}.}
	\label{fig:2_dien_tich_tam_giac_can}
\end{figure}

\begin{proof}[Giải]
	Đặt $r\coloneqq d(M,q_1) = d(M,q_2)$ là khoảng cách từ $M$ đến mỗi điện tích $q_1,q_2$.\footnote{Trong Toán học \& Vật lý, b$d$ thường được dùng để ký hiệu \textit{khoảng cách} do được lấy từ chữ cái đầu của từ \textit{distance}.} ``Gọi cường độ điện trường do điện tích $q_1,q_2$ gây ra tại $M$ là $\overrightarrow{E_1},\overrightarrow{E_2}$. Vì độ lớn của 2 điện tích $q_1,q_2$ bằng nhau \& điểm $M$ cách đều 2 điện tích đó nên $E_1 = E_2$. Các vector $\overrightarrow{E_1},\overrightarrow{E_2}$ được vẽ trên Fig. \ref{fig:2_dien_tich_tam_giac_can}. Theo công thức xác định cường độ điện trường của 1 điện tích điểm, $E_1 = E_2 = 9\cdot 10^9\frac{q_1}{r^2}$, $r^2 = l^2 + \left(\frac{a}{2}\right)^2$. Ta có $\overrightarrow{E} = \overrightarrow{E_1} + \overrightarrow{E_2}$. Vì $E_1 = E_2$ nên vector $\overrightarrow{E}\parallel$ đường thẳng nối $q_1,q_2$ \& có chiều như Fig. \ref{fig:2_dien_tich_tam_giac_can}. Từ Fig. \ref{fig:2_dien_tich_tam_giac_can}, ta suy ra: $E = 2E_1\cos\alpha$ trong đó $\cos\alpha = \frac{a}{2\sqrt{l^2 + \left(\frac{a}{2}\right)^2}}$. Do đó,
	\begin{align*}
		E = 2\cdot 9\cdot 10^9\frac{q_1}{r^2}\frac{a}{2\sqrt{l^2 + \left(\frac{a}{2}\right)^2}} = \frac{9\cdot 10^9q_1a}{\left[l^2 + \left(\frac{a}{2}\right)^2\right]^{3/2}} = \frac{8\cdot 9\cdot 10^9q_1a}{(4l^2 + a^2)^{3/2}} = \frac{7.2\cdot 10^{10}q_1a}{(4l^2 + a^2)^{3/2}}.
	\end{align*}
\end{proof}

\begin{luuy}
	Nếu điểm $M$ tạo với điểm đặt 2 điện tích $q_1,q_2$ 1 tam giác đều, thì $l = a\sin\frac{\pi}{3} = \frac{a\sqrt{3}}{2}$, kết quả của lời giải trên trở thành $E = \frac{9\cdot 10^9q_1a}{\left[\left(\frac{a\sqrt{3}}{2}\right)^2 + \left(\frac{a}{2}\right)^2\right]^{3/2}} = \frac{9\cdot 10^9q_1}{a^2}$.
\end{luuy}
Tổng quát hơn nữa, trong bài toán, thay vì tam giác cân, xét 1 tam giác tùy ý, \& 2 điện tích điểm không nhất thiết phải có độ lớn bằng nhau:

\begin{baitoan}
	Có 2 điện tích điểm $q_1\ {\rm C}$ \& $q_2\ {\rm C}$, đặt cách nhau 1 đoạn $a\ {\rm m}$ trong không khí. Xác định cường độ điện trường $\overrightarrow{E}$ tại điểm $M$ cách 2 điện tích $q_1,q_2$ 2 khoảng lần lượt bằng $b$ \& $c$.
\end{baitoan}
Bài toán này liên quan đến bài toán giải tam giác khi biết độ dài 3 cạnh của tam giác đó. Có thể xem các bài toán về \textit{giải tam giác} ở \cite[\S2, pp. 71--78]{SGK_Toan_10_Canh_Dieu_tap_1} \& tài liệu của tác giả cho chương trình Toán lớp 10: \href{https://github.com/NQBH/hobby/blob/master/elementary_mathematics/grade_10/NQBH_elementary_mathematics_grade_10.pdf}{GitHub\texttt{/}NQBH\texttt{/}hobby\texttt{/}elementary mathematics\texttt{/}grade 10\texttt{/}lecture}.

\begin{baitoan}[\cite{SGK_Vat_Ly_11_nang_cao}, \textbf{3.}, p. 26]
	Có 2 tấm kim loại (1), (2) rộng, nằm ngang song song với nhau \& cách nhau $d = 10\ {\rm cm}$ (Fig. \ref{fig:2_tam_kim_loai_song_song}).
	
	\begin{figure}[H]
		\centering
		\includegraphics[scale=0.15]{2_tam_kim_loai_song_song}
		\caption{\cite[Hình 5.3, p. 26]{SGK_Vat_Ly_11_nang_cao}.}
		\label{fig:2_tam_kim_loai_song_song}
	\end{figure}
	Tấm (1) mang điện dương, tấm (2) mang điện âm, điện tích trên 2 tấm có độ lớn bằng nhau. Bên trong 2 tấm kim loại có 1 hạt bụi khối lượng $m = 2\cdot 10^{-9}\ {\rm g}$ mang điện tích $q = -0.06\ {\rm pC}$ bị vướng ở điểm $O$ (nằm yên tại $O$). $O$ cách tấm kim loại (2) $1.6\ {\rm cm}$ \& cách mép trái 2 tấm kim loại $10\ {\rm cm}$. Lúc $t = 0$, ta truyền cho hạt bụi 1 vận tốc $v = 25\ {\rm cm\emph{\texttt{/}}s}$ theo phương nằm ngang. Sau đó ít lâu hạt bụi đi đến $M$, $M$ cách tấm kim loại (1) $2\ {\rm cm}$ \& cách mép trái 2 tấm kim loại $14\ {\rm cm}$.
	\begin{enumerate*}
		\item[(a)] Hỏi hiệu điện thế giữa 2 tấm kim loại bằng bao nhiêu?
		\item[(b)] Tính công của lực điện trong di chuyển nói trên của hạt bụi.
	\end{enumerate*}
	Coi rằng quỹ đạo của hạt bụi nằm trong mặt phẳng hình vẽ.
	\begin{figure}[H]
		\centering
		\includegraphics[scale=0.15]{quy_dao_hat_bui}
		\caption{\cite[Hình 5.4, p. 26]{SGK_Vat_Ly_11_nang_cao}.}
		\label{fig:quy_dao_hat_bui}
	\end{figure}
	Điện trường bên trong 2 tấm kim loại là điện trường đều. Lấy $g = 10{\rm m\emph{\texttt{/}}s^2}$.
\end{baitoan}
Không cho giá trị cụ thể, 1 tổng quát của bài toán trên:

\begin{baitoan}[\cite{SGK_Vat_Ly_11_nang_cao}, \textbf{3.}, p. 26]
	Có 2 tấm kim loại (1), (2) rộng, nằm ngang song song với nhau \& cách nhau $d\ {\rm m}$ (Fig. \ref{fig:2_tam_kim_loai_song_song}). Tấm (1) mang điện dương, tấm (2) mang điện âm, điện tích trên 2 tấm có độ lớn bằng nhau. Bên trong 2 tấm kim loại có 1 hạt bụi khối lượng $m\ {\rm g}$ mang điện tích $q\ {\rm pC}$ bị vướng ở điểm $O$ (nằm yên tại $O$). $O$ cách tấm kim loại (2) $d_1\ {\rm m}$ \& cách mép trái 2 tấm kim loại $d_2\ {\rm m}$. Lúc $t = 0$, ta truyền cho hạt bụi 1 vận tốc $v\ {\rm m\emph{\texttt{/}}s}$ theo phương nằm ngang. Sau đó ít lâu hạt bụi đi đến $M$, $M$ cách tấm kim loại (1) $d_3\ {\rm m}$ \& cách mép trái 2 tấm kim loại $d_4\ {\rm m}$.
	\begin{enumerate*}
		\item[(a)] Hỏi hiệu điện thế giữa 2 tấm kim loại bằng bao nhiêu?
		\item[(b)] Tính công của lực điện trong di chuyển nói trên của hạt bụi.
	\end{enumerate*}
	Coi rằng quỹ đạo của hạt bụi nằm trong mặt phẳng hình vẽ. Điện trường bên trong 2 tấm kim loại là điện trường đều. Lấy $g = 10{\rm m\emph{\texttt{/}}s^2}$.
\end{baitoan}

\begin{proof}[Giải]
	\begin{enumerate*}
		\item[(a)] Xem \cite[p. 27]{SGK_Vat_Ly_11_nang_cao}.
	\end{enumerate*}
\end{proof}

%------------------------------------------------------------------------------%

\section{Vật Dẫn \& Điện Môi Trong Điện Trường}
\textbf{Nội dung.} ``\textit{Những tính chất của vật dẫn \& điện môi khi chúng được đặt trong điện trường}. Chú ý rằng 1 vật dẫn tích điện cũng có thể coi là vật dẫn đặt trong điện trường.'' -- \cite[p. 28]{SGK_Vat_Ly_11_nang_cao}

\subsection{Vật dẫn trong điện trường}

\subsubsection{Trạng thái cân bằng điện}
``1 vật dẫn có thể được tích điện bằng hưởng ứng, bằng cọ xát hay bằng tiếp xúc. Dù được tích điện bằng cách nào, thì lúc đầu của quá trình tích điện cũng có sự di chuyển các điện  tích tự do \& tạo thành dòng điện trong vật dẫn. Tuy nhiên, dòng điện chỉ tồn tại trong khoảng thời gian rất ngắn. Khi trong vật dẫn không còn dòng điện nữa, người ta nói \textit{vật dẫn cân bằng tĩnh điện}, hay cũng nói tắt là \textit{cân bằng điện}. Say này, khi nói \textit{vật dẫn trong điện trường}, ta hiểu ngầm rằng chỉ nói đến trường hợp vật dẫn cân bằng điện trong điện trường.'' -- \cite[p. 28]{SGK_Vat_Ly_11_nang_cao}

\subsubsection{Điện trường trong vật dẫn tích điện}
``Thí nghiệm chứng tỏ rằng \textit{bên trong vật dẫn, điện trường bằng 0}. Điều đó cũng dễ hiểu, vì trong vật dẫn đã có sẵn các điện tích tự do nên nếu điện trường khác 0 thì nó sẽ tác dụng lực lên các điện tích tự do \& gây ra dòng điện. Điều này trái với giả thiết là vật ở trạng thái cân bằng điện. Trong phần rỗng của vật dẫn, điện trường cũng bằng 0 nếu ở phần này không có điện tích. Điện trường bên trong vật dẫn rỗng bằng 0 nên người ta dùng các vật dẫn rỗng làm các \textit{màn chắn điện}. Để các dụng cụ hay các máy móc chính xác không bị ảnh hưởng bởi điện trường ngoài, người ta đặt chúng trong những chiếc hộp kim loại. \textit{Cường độ điện trường tại 1 điểm trên mặt ngoài vật dẫn vuông góc với mặt vật} (Fig. \ref{fig:cuong_do_dien_truong_tren_mat_vat_dan}).

\begin{figure}[H]
	\centering
	\includegraphics[scale=0.15]{cuong_do_dien_truong_tren_mat_vat_dan}
	\caption{Trên mặt vật dẫn, vector cường độ điện trường vuông góc với mặt vật, \cite[Hình 6.1, p. 28]{SGK_Vat_Ly_11_nang_cao}.}
	\label{fig:cuong_do_dien_truong_tren_mat_vat_dan}
\end{figure}
Vì nếu cường độ điện trường không vuông góc với mặt vật dẫn thì sẽ có 1 thành phần tiếp tuyến với mặt vật. Thành phần này tác dụng lực lên các điện tích tự do \& gây ra dòng điện trên mặt vật.'' -- \cite[p. 28]{SGK_Vat_Ly_11_nang_cao}

\subsubsection{Điện thế của vật dẫn tích điện}
\begin{itemize}
	\item ``\textit{Điện thế trên mặt ngoài vật dẫn}. Thí nghiệm: Sơ đồ của thí nghiệm như Fig. \ref{fig:thi_nghiem_dien_the_mat_ngoai_vat_dan}.
	
	\begin{figure}[H]
		\centering
		\includegraphics[scale=0.15]{thi_nghiem_dien_the_mat_ngoai_vat_dan}
		\caption{Thí nghiệm về điện thế ở mặt ngoài của vật dẫn. 1. Vật dẫn nhiễm điện; 2. Quả cầu thử bằng kim loại; 3. Tay cầm bằng nhựa; 4. Tĩnh điện kế; 5. Dây nối quả cầu \& tĩnh điện kế, \cite[Hình 6.2, p. 29]{SGK_Vat_Ly_11_nang_cao}.}
		\label{fig:thi_nghiem_dien_the_mat_ngoai_vat_dan}
	\end{figure}
	Nối núm kim loại của cần tĩnh điện kế với quả cầu thử. Di chuyển quả cầu thử đến nhiều điểm khác nhau trên mặt vật nhiễm điện. Tại mọi điểm (kể cả những điểm ở phần lõm của vật) góc lệch của kim tĩnh điện kế đều như nhau.
	
	Thí nghiệm chứng tỏ \textit{điện thế tại mọi điểm trên mặt ngoài vật dẫn có giá trị bằng nhau}.
	\item \textit{Điện thế bên trong vật dẫn}. Vì điện trường trong vật dẫn bằng 0, nên từ mối liên hệ giữa $E$ \& $U$ có thể suy ra rằng điện thế tại mọi điểm bên trong vật dẫn phải bằng nhau \& bằng điện thế trên mặt ngoài của vật. Vậy vật dẫn là \textit{vật đẳng thế}.'' -- \cite[p. 29]{SGK_Vat_Ly_11_nang_cao}
\end{itemize}

\subsubsection{Sự phân bố điện tích ở vật dẫn tích điện}
\begin{itemize}
	\item ``\textit{Sự phân bố điện tích ở mặt ngoài của vật dẫn}. Thí nghiệm: Sơ đồ thí nghiệm như ở Fig. \ref{fig:thi_nghiem_phan_bo_dien_tich_mat_ngoai_vat_dan}.
	
	\begin{figure}[H]
		\centering
		\includegraphics[scale=0.15]{thi_nghiem_phan_bo_dien_tich_mat_ngoai_vat_dan}
		\caption{Thí nghiệm về sự phân bố điện tích ở mặt ngoài vật dẫn. 1. Quả cầu kim loại nhiễm điện; 2. Quả cầu thử bằng kim loại; 3. Tay cầm bằng nhựa; 4. Điện nghiệm, \cite[Hình 6.3, p. 29]{SGK_Vat_Ly_11_nang_cao}.}
		\label{fig:thi_nghiem_phan_bo_dien_tich_mat_ngoai_vat_dan}
	\end{figure}
	Cho quả cầu thử tiếp xúc mới mặt ngoài quả cầu kim loại 1. Sau đó đưa quả cầu thử chạm vào núm kim loại của điện nghiệm, ta thấy 2 lá kim loại xòe ra. Nhưng nếu cho quả cầu thử tiếp xúc với mặt trong của quả cầu 1 thì 2 lá kim loại không xòe ra. Thí nghiệm chứng tỏ, \textit{ở 1 vật dẫn rỗng nhiễm điện, thì điện tích chỉ phân bố ở mặt ngoài của vật}. Với vật dẫn đặc, điện tích cũng chỉ phân bố ở mặt ngoài của vật.
	\item \textit{Sự phân bố điện tích trên vật trong trường hợp mà mặt ngoài có chỗ lồi, chỗ lõm}. Thí nghiệm: Sơ đồ thí nghiệm như ở Fig. \ref{fig:thi_nghiem_phan_bo_dien_tich_vat_dan_loi_lom}.
	
	\begin{figure}[H]
		\centering
		\includegraphics[scale=0.15]{thi_nghiem_phan_bo_dien_tich_vat_dan_loi_lom}
		\caption{Thí nghiệm về sự phân bố điện tích ở vật dẫn trong trường hợp mặt ngoài có chỗ lồi, chỗ lõm, \cite[Hình 6.4, p. 30]{SGK_Vat_Ly_11_nang_cao}.}
		\label{fig:thi_nghiem_phan_bo_dien_tich_vat_dan_loi_lom}
	\end{figure}
	Cho quả cầu thử chạm với mặt vật dẫn nhiễm điện. Sau đó đưa quả cầu đến chạm với núm kim loại của điện nghiệm. Cứ sau mỗi lần thử như trên, ta khử điện tích ở quả cầu \& ở điện nghiệm rồi mới làm phép thử tiếp theo. Quả cầu thử có kích thước nhỏ để cho sau mỗi lần thử, điện tích của vật nhiễm điện thay đổi không đáng kể. Góc xòe của 2 lá điện nghiệm ở Hình 6.4b lớn nhất. Ở Hình 6.4c 2 lá kim loại hầu như không xòe ra.
	
	Từ thí nghiệm, ta rút ra kết luận: \textit{Ở những chỗ lồi của mặt vật dẫn, điện tích tập trung nhiều hơn; ở những chỗ mũi nhọn điện tích tập trung nhiều nhất; ở chỗ lõm hầu như không có điện tích}. Điện tích phân bố bên mặt ngoài của vật dẫn không đều, nên cường độ điện trường ở gần mặt ngoài của vật cũng khác nhau. Nơi nào điện tích tập trung nhiều hơn, điện trường ở đó mạnh hơn, đặc biệt ở gần các mũi nhọn điện trường rất mạnh. Nếu mũi nhọn đặt trong không khí, thì 1 số hạt mang điện có sẵn trong không khí ở gần mũi nhọn được tăng tốc \& làm cho không khí ở đó bị ion hóa. Các hạt mang điện trái dấu với điện tích của mũi nhọn bị hút vào mũi nhọn làm cho điện tích của mũi nhọn giảm nhanh. Điều này được áp dụng trong cột chống sét.''  -- \cite[pp. 29--30]{SGK_Vat_Ly_11_nang_cao}
\end{itemize}

\subsection{Điện môi trong điện trường}
``Khi đặt 1 vật điện môi trong điện trường thì hạt nhân \& electron trong các nguyên tử của vật đó chịu tác dụng lực của điện trường. Các electron xê dịch ngược chiều điện trường. Còn hạt nhân thì hầu như không xê dịch. Kết quả là mỗi nguyên tử như được kéo dãn ra 1 chút \& chia thành 2 đầu mang điện tích trái dấu nhau. Người ta nói \textit{điện môi bị phân cực}.

Khi mẩu điện môi được đặt trong điện trường đều, e.g., đặt trong điện trường bên trong 2 tấm kim loại phẳng rộng, song song tích điện trái dấu \& bằng nhau, thì do sự phân cực mà các mặt ngoài của điện môi trở thành các mặt nhiễm điện như trên Fig. \ref{fig:phan_cuc_dien_moi}.

\begin{figure}[H]
	\centering
	\includegraphics[scale=0.15]{phan_cuc_dien_moi}
	\caption{Sự phân cực của điện môi, \cite[Hình 6.5, p. 30]{SGK_Vat_Ly_11_nang_cao}.}
	\label{fig:phan_cuc_dien_moi}
\end{figure}
Các mặt nhiễm điện của điện môi làm xuất hiện điện trường phụ. Điện trường phụ ngược chiều với điện trường ngoài làm cho điện trường bên trong điện môi giảm. Điện trường giảm kéo theo lực điện tác dụng lên điện tích trong điện môi cũng giảm. Đó là điều ta đã nói đến ở Bài 1.'' -- \cite[p. 30]{SGK_Vat_Ly_11_nang_cao}

%------------------------------------------------------------------------------%

\section{Tụ Điện}

%------------------------------------------------------------------------------%

\section{Năng Lượng Điện Trường}

%------------------------------------------------------------------------------%

\section{Bài Tập về Tụ Điện}

%------------------------------------------------------------------------------%

\section{Máy Sao Chụp Quang Học (Photocopy)}

%------------------------------------------------------------------------------%

\section{Tóm Tắt Chương 1}

%------------------------------------------------------------------------------%

\chapter{Dòng Điện Không Đổi}

\section{Dòng Điện Không Đổi. Nguồn Điện}

%------------------------------------------------------------------------------%

\section{Pin \& Acquy}

%------------------------------------------------------------------------------%

\section{Điện Năng \& Công Suất Điện. Định Luật Jun--Len-xơ}

%------------------------------------------------------------------------------%

\section{Định Luật Ôm Đối với Toàn Mạch}

%------------------------------------------------------------------------------%

\section{Định Luật Ôm Đối với Các Loại Mạch Điện. Mắc Các Nguồn Điện Thành Bộ}

%------------------------------------------------------------------------------%

\section{Bài Tập về Định Luật Ôm \& Công Suất Điện}

%------------------------------------------------------------------------------%

\section{Điện Tâm Đồ}

%------------------------------------------------------------------------------%

\section{Thực Hành: Đo Suất Điện Động \& Điện Trở Trong của Nguồn Điện}

%------------------------------------------------------------------------------%

\section{Tóm Tắt Chương 2}

%------------------------------------------------------------------------------%

\chapter{Dòng Diện Trong Các Môi Trường}

\section{Dòng Điện Trong Kim Loại}

%------------------------------------------------------------------------------%

\section{Hiện Tượng Nhiệt Điện. Hiện Tượng Siêu Dẫn}

%------------------------------------------------------------------------------%

\section{Dòng Điện Trong Chất Điện Phân. Định Luật Faraday}

%------------------------------------------------------------------------------%

\section{Bài Tập về Dòng Điện Trong Kim Loại \& Chất Điện Phân}

%------------------------------------------------------------------------------%

\section{Dòng Điện Trong Chân Không}

%------------------------------------------------------------------------------%

\section{Dòng Điện Trong Chất Khí}

%------------------------------------------------------------------------------%

\section{Dòng Điện Trong Chất Bán Dẫn}

%------------------------------------------------------------------------------%

\section{Linh Kiện Bán Dẫn}

%------------------------------------------------------------------------------%

\section{Thực Hành: Khảo Sát Đặc Tính Chỉnh Lưu của Diot Bán Dẫn \& Đặc Tính Khuếch Đại của Tranzito}

%------------------------------------------------------------------------------%

\section{Tóm Tắt Chương 3}

%------------------------------------------------------------------------------%

\chapter{Từ Trường}

\section{Từ Trường}

%------------------------------------------------------------------------------%

\section{Phương \& Chiều của Lực Từ Tác Dụng Lên Dòng Điện}

%------------------------------------------------------------------------------%

\section{Cảm Ứng Từ. Định Luật Ampe}

%------------------------------------------------------------------------------%

\section{Từ Trường của 1 Số Dòng Điện Có Dạng Đơn Giản}

%------------------------------------------------------------------------------%

\section{Bài Tập về Từ Trường}

%------------------------------------------------------------------------------%

\section{Tương Tác Giữa 2 Dòng Điện Thẳng Song Song. Định Nghĩa Đơn Vị Ampe}

%------------------------------------------------------------------------------%

\section{Lực Lo-ren-xơ}

%------------------------------------------------------------------------------%

\section{Khung Dây có Dòng Điện Đặt trong Từ Trường}

%------------------------------------------------------------------------------%

\section{Sự Từ Hóa Các Chất. Sắt Từ}

%------------------------------------------------------------------------------%

\section{Từ Trường Trái Đất}

%------------------------------------------------------------------------------%

\section{Bài Tập về Lực Từ}

%------------------------------------------------------------------------------%

\section{Từ Trường \& Máy Gia Tốc}

%------------------------------------------------------------------------------%

\section{Thực Hành: Xác Định Thành Phần Năm Ngang của Từ Trường Trái Đất}

%------------------------------------------------------------------------------%

\section{Tóm Tắt Chương 4}

%------------------------------------------------------------------------------%

\chapter{Cảm Ứng Điện Từ}

\section{Hiện Tượng Cảm Ứng Điện Từ. Suất Điện Động Cảm Ứng}

%------------------------------------------------------------------------------%

\section{Suất Điện Động Cảm Ứng Tron 1 Đoạn Dây Dẫn Chuyển Động}

%------------------------------------------------------------------------------%

\section{Dòng Điện Fu-cô}

%------------------------------------------------------------------------------%

\section{Hiện Tượng Tự Cảm}

%------------------------------------------------------------------------------%

\section{Năng Lượng Từ Trường}

%------------------------------------------------------------------------------%

\section{Bài Tập về Cảm Ứng Điện Từ}

%------------------------------------------------------------------------------%

\section{1 Số Mốc Thời Gian Đáng Lưu Ý Trong Lĩnh Vực Điện Tử}

%------------------------------------------------------------------------------%

\section{Tóm Tắt Chương 5}

%------------------------------------------------------------------------------%

\part{Quang Hình Học}

\chapter{Khúc Xạ Ánh Sáng}

\section{Khúc Xạ Ánh Sáng}

%------------------------------------------------------------------------------%

\section{Phản Xạ Toàn Phần}

%------------------------------------------------------------------------------%

\section{Bài Tập về Khúc Xạ Ánh Sáng \& Phản Xạ Toàn Phần}

%------------------------------------------------------------------------------%

\section{Bài Đọc Thêm. Hiện Tượng Ảo Ảnh}

%------------------------------------------------------------------------------%

\section{Tóm Tắt Chương 6}

%------------------------------------------------------------------------------%

\chapter{Mắt. Các Dụng Cụ Quang}

\section{Lăng Kính}

%------------------------------------------------------------------------------%

\section{Thấu Kính Mỏng}

%------------------------------------------------------------------------------%

\section{Bài Tập về Lăng Kính \& Thấu Kính Mỏng}

%------------------------------------------------------------------------------%

\section{Mắt}

%------------------------------------------------------------------------------%

\section{Các Tật của Mắt \& Cách Khắc Phục}

%------------------------------------------------------------------------------%

\section{Kính Lúp}

%------------------------------------------------------------------------------%

\section{Kính Hiển Vi}

%------------------------------------------------------------------------------%

\section{Kính Thiên Văn}

%------------------------------------------------------------------------------%

\section{Bài Tập về Dụng Cụ Quang}

%------------------------------------------------------------------------------%

\section{Thực Hành: Xác Định Chiết Suất của Nước \& Tiêu Cự của Thấu Kính Phân Kỳ}

%------------------------------------------------------------------------------%

\section{Tóm Tắt Chương 7}

%------------------------------------------------------------------------------%

\printbibliography[heading=bibintoc]
	
\end{document}