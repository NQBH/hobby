\documentclass[oneside]{book}
\usepackage[backend=biber,natbib=true,style=authoryear]{biblatex}
\addbibresource{/home/hong/1_NQBH/reference/bib.bib}
\usepackage[utf8]{vietnam}
\usepackage{tocloft}
\renewcommand{\cftsecleader}{\cftdotfill{\cftdotsep}}
\usepackage[colorlinks=true,linkcolor=blue,urlcolor=red,citecolor=magenta]{hyperref}
\usepackage{amsmath,amssymb,amsthm,mathtools,float,graphicx,algpseudocode,algorithm,tcolorbox,tikz,tkz-tab}
\usepackage[inline]{enumitem}
\allowdisplaybreaks
\numberwithin{equation}{section}
\newtheorem{assumption}{Assumption}[section]
\newtheorem{nhanxet}{Nhận xét}[section]
\newtheorem{conjecture}{Conjecture}[section]
\newtheorem{corollary}{Corollary}[section]
\newtheorem{hequa}{Hệ quả}[section]
\newtheorem{definition}{Definition}[section]
\newtheorem{dinhnghia}{Định nghĩa}[section]
\newtheorem{example}{Example}[section]
\newtheorem{vidu}{Ví dụ}[section]
\newtheorem{lemma}{Lemma}[section]
\newtheorem{notation}{Notation}[section]
\newtheorem{principle}{Principle}[section]
\newtheorem{problem}{Problem}[section]
\newtheorem{baitoan}{Bài toán}[section]
\newtheorem{proposition}{Proposition}[section]
\newtheorem{menhde}{Mệnh đề}[section]
\newtheorem{question}{Question}[section]
\newtheorem{cauhoi}{Câu hỏi}[section]
\newtheorem{remark}{Remark}[section]
\newtheorem{luuy}{Lưu ý}[section]
\newtheorem{theorem}{Theorem}[section]
\newtheorem{dinhly}{Định lý}[section]
\usepackage[left=0.5in,right=0.5in,top=1.5cm,bottom=1.5cm]{geometry}
\usepackage{fancyhdr}
\pagestyle{fancy}
\fancyhf{}
\lhead{\small \textsc{Sect.} ~\thesection}
\rhead{\small \nouppercase{\leftmark}}
\renewcommand{\sectionmark}[1]{\markboth{#1}{}}
\cfoot{\thepage}
\def\labelitemii{$\circ$}

\title{Some Topics in Elementary Physics\texttt{/}Grade 11}
\author{Nguyễn Quản Bá Hồng\footnote{Independent Researcher, Ben Tre City, Vietnam\\e-mail: \texttt{nguyenquanbahong@gmail.com}; website: \url{https://nqbh.github.io}.}}
\date{\today}

\begin{document}
\frontmatter
\maketitle
\setcounter{secnumdepth}{4}
\setcounter{tocdepth}{3}
\tableofcontents
\newpage

%------------------------------------------------------------------------------%

\mainmatter

\part{Điện Học -- Điện Từ Học}

\chapter{Điện Tích -- Điện Trường}

\section{Điện Tích. Định Luật Colomb}

\section{Thuyết Electron. Định Luật Bảo Toàn Điện Tích}

\section{Điện Trường}

\section{Công của Lực Điện. Hiệu Điện Thế}

\section{Bài Tập về Lực Colomb \& Điện Trường}

\section{Vật Dẫn \& Điện Môi Trong Điện Trường}

\section{Tụ Điện}

\section{Năng Lượng Điện Trường}

\section{Bài Tập về Tụ Điện}

\section{Máy Sao Chụp Quang Học (Photocopy)}

\section{Tóm Tắt Chương 1}

%------------------------------------------------------------------------------%

\chapter{Dòng Điện Không Đổi}

\section{Dòng Điện Không Đổi. Nguồn Điện}

\section{Pin \& Acquy}

\section{Điện Năng \& Công Suất Điện. Định Luật Jun--Len-xơ}

\section{Định Luật Ôm Đối với Toàn Mạch}

\section{Định Luật Ôm Đối với Các Loại Mạch Điện. Mắc Các Nguồn Điện Thành Bộ}

\section{Bài Tập về Định Luật Ôm \& Công Suất Điện}

\section{Điện Tâm Đồ}

\section{Thực Hành: Đo Suất Điện Động \& Điện Trở Trong của Nguồn Điện}

\section{Tóm Tắt Chương 2}

%------------------------------------------------------------------------------%

\chapter{Dòng Diện Trong Các Môi Trường}

\section{Dòng Điện Trong Kim Loại}

\section{Hiện Tượng Nhiệt Điện. Hiện Tượng Siêu Dẫn}

\section{Dòng Điện Trong Chất Điện Phân. Định Luật Faraday}

\section{Bài Tập về Dòng Điện Trong Kim Loại \& Chất Điện Phân}

\section{Dòng Điện Trong Chân Không}

\section{Dòng Điện Trong Chất Khí}

\section{Dòng Điện Trong Chất Bán Dẫn}

\section{Linh Kiện Bán Dẫn}

\section{Thực Hành: Khảo Sát Đặc Tính Chỉnh Lưu của Diot Bán Dẫn \& Đặc Tính Khuếch Đại của Tranzito}

\section{Tóm Tắt Chương 3}

%------------------------------------------------------------------------------%

\chapter{Từ Trường}

\section{Từ Trường}

\section{Phương \& Chiều của Lực Từ Tác Dụng Lên Dòng Điện}

\section{Cảm Ứng Từ. Định Luật Ampe}

\section{Từ Trường của 1 Số Dòng Điện Có Dạng Đơn Giản}

\section{Bài Tập về Từ Trường}

\section{Tương Tác Giữa 2 Dòng Điện Thẳng Song Song. Định Nghĩa Đơn Vị Ampe}

\section{Lực Lo-ren-xơ}

\section{Khung Dây có Dòng Điện Đặt trong Từ Trường}

\section{Sự Từ Hóa Các Chất. Sắt Từ}

\section{Từ Trường Trái Đất}

\section{Bài Tập về Lực Từ}

\section{Từ Trường \& Máy Gia Tốc}

\section{Thực Hành: Xác Định Thành Phần Năm Ngang của Từ Trường Trái Đất}

\section{Tóm Tắt Chương 4}

%------------------------------------------------------------------------------%

\chapter{Cảm Ứng Điện Từ}

\section{Hiện Tượng Cảm Ứng Điện Từ. Suất Điện Động Cảm Ứng}

\section{Suất Điện Động Cảm Ứng Tron 1 Đoạn Dây Dẫn Chuyển Động}

\section{Dòng Điện Fu-cô}

\section{Hiện Tượng Tự Cảm}

\section{Năng Lượng Từ Trường}

\section{Bài Tập về Cảm Ứng Điện Từ}

\section{1 Số Mốc Thời Gian Đáng Lưu Ý Trong Lĩnh Vực Điện Tử}

\section{Tóm Tắt Chương 5}

%------------------------------------------------------------------------------%

\part{Quang Hình Học}

\chapter{Khúc Xạ Ánh Sáng}

\section{Khúc Xạ Ánh Sáng}

\section{Phản Xạ Toàn Phần}

\section{Bài Tập về Khúc Xạ Ánh Sáng \& Phản Xạ Toàn Phần}

\section{Bài Đọc Thêm. Hiện Tượng Ảo Ảnh}

\section{Tóm Tắt Chương 6}

%------------------------------------------------------------------------------%

\chapter{Mắt. Các Dụng Cụ Quang}

\section{Lăng Kính}

\section{Thấu Kính Mỏng}

\section{Bài Tập về Lăng Kính \& Thấu Kính Mỏng}

\section{Mắt}

\section{Các Tật của Mắt \& Cách Khắc Phục}

\section{Kính Lúp}

\section{Kính Hiển Vi}

\section{Kính Thiên Văn}

\section{Bài Tập về Dụng Cụ Quang}

\section{Thực Hành: Xác Định Chiết Suất của Nước \& Tiêu Cự của Thấu Kính Phân Kỳ}

\section{Tóm Tắt Chương 7}

%------------------------------------------------------------------------------%

\printbibliography[heading=bibintoc]
	
\end{document}