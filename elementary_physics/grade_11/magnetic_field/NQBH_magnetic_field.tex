\documentclass{article}
\usepackage[backend=biber,natbib=true,style=authoryear]{biblatex}
\addbibresource{/home/nqbh/reference/bib.bib}
\usepackage[utf8]{vietnam}
\usepackage{tocloft}
\renewcommand{\cftsecleader}{\cftdotfill{\cftdotsep}}
\usepackage[colorlinks=true,linkcolor=blue,urlcolor=red,citecolor=magenta]{hyperref}
\usepackage{amsmath,amssymb,amsthm,mathtools,float,graphicx,algpseudocode,algorithm,tcolorbox}
\usepackage[inline]{enumitem}
\allowdisplaybreaks
\numberwithin{equation}{section}
\newtheorem{assumption}{Assumption}[section]
\newtheorem{baitoan}{Bài toán}
\newtheorem{cauhoi}{Câu hỏi}[section]
\newtheorem{conjecture}{Conjecture}[section]
\newtheorem{corollary}{Corollary}[section]
\newtheorem{dangtoan}{Dạng toán}[section]
\newtheorem{definition}{Definition}[section]
\newtheorem{dinhly}{Định lý}[section]
\newtheorem{dinhnghia}{Định nghĩa}[section]
\newtheorem{example}{Example}[section]
\newtheorem{ghichu}{Ghi chú}[section]
\newtheorem{hequa}{Hệ quả}[section]
\newtheorem{hypothesis}{Hypothesis}[section]
\newtheorem{lemma}{Lemma}[section]
\newtheorem{luuy}{Lưu ý}[section]
\newtheorem{nhanxet}{Nhận xét}[section]
\newtheorem{notation}{Notation}[section]
\newtheorem{note}{Note}[section]
\newtheorem{principle}{Principle}[section]
\newtheorem{problem}{Problem}[section]
\newtheorem{proposition}{Proposition}[section]
\newtheorem{question}{Question}[section]
\newtheorem{remark}{Remark}[section]
\newtheorem{theorem}{Theorem}[section]
\newtheorem{vidu}{Ví dụ}[section]
\usepackage[left=0.5in,right=0.5in,top=1.5cm,bottom=1.5cm]{geometry}
\usepackage{fancyhdr}
\pagestyle{fancy}
\fancyhf{}
\lhead{\small Sect.~\thesection}
\rhead{\small\nouppercase{\leftmark}}
\renewcommand{\subsectionmark}[1]{\markboth{#1}{}}
\cfoot{\thepage}
\def\labelitemii{$\circ$}

\title{Magnetic Field -- Từ Trường}
\author{Nguyễn Quản Bá Hồng\footnote{Independent Researcher, Ben Tre City, Vietnam\\e-mail: \texttt{nguyenquanbahong@gmail.com}; website: \url{https://nqbh.github.io}.}}
\date{\today}

\begin{document}
\maketitle
\begin{abstract}
	\textsc{[en]} This text is a collection of problems, from easy to advanced, about magnetic field. This text is also a supplementary material for my lecture note on Elementary Physics grade 11, which is stored \& downloadable at the following link: \href{https://github.com/NQBH/hobby/blob/master/elementary_physics/grade_11/NQBH_elementary_physics_grade_11.pdf}{GitHub\texttt{/}NQBH\texttt{/}hobby\texttt{/}elementary physics\texttt{/}grade 11\texttt{/}lecture}\footnote{\textsc{url}: \url{https://github.com/NQBH/hobby/blob/master/elementary_physics/grade_11/NQBH_elementary_physics_grade_11.pdf}.}. The latest version of this text has been stored \& downloadable at the following link: \href{https://github.com/NQBH/hobby/blob/master/elementary_physics/grade_11/magnetic_field/NQBH_magnetic_field.pdf}{GitHub\texttt{/}NQBH\texttt{/}hobby\texttt{/}elementary physics\texttt{/}grade 11\texttt{/}magnetic field}\footnote{\textsc{url}: \url{https://github.com/NQBH/hobby/blob/master/elementary_physics/grade_11/magnetic_field/NQBH_magnetic_field.pdf}.}.
	\vspace{2mm}
	
	\textsc{[vi]} Tài liệu này là 1 bộ sưu tập các bài tập chọn lọc từ cơ bản đến nâng cao về từ trường. Tài liệu này là phần bài tập bổ sung cho tài liệu chính -- bài giảng \href{https://github.com/NQBH/hobby/blob/master/elementary_physics/grade_11/NQBH_elementary_physics_grade_11.pdf}{GitHub\texttt{/}NQBH\texttt{/}hobby\texttt{/}elementary physics\texttt{/}grade 11\texttt{/}lecture} của tác giả viết cho Vật Lý Sơ Cấp lớp 11. Phiên bản mới nhất của tài liệu này được lưu trữ \& có thể tải xuống ở link sau: \href{https://github.com/NQBH/hobby/blob/master/elementary_physics/grade_11/magnetic_field/NQBH_magnetic_field.pdf}{GitHub\texttt{/}NQBH\texttt{/}hobby\texttt{/}elementary physics\texttt{/}grade 11\texttt{/}magnetic field}.
\end{abstract}
\tableofcontents
\newpage

%------------------------------------------------------------------------------%

\section{Từ Trường -- Magnetic Field}

%------------------------------------------------------------------------------%

\section{Phương \& Chiều của Lực Từ Tác Dụng Lên Dòng Điện}

%------------------------------------------------------------------------------%

\section{Cảm Ứng Từ. Định Luật Amp\`ere}

%------------------------------------------------------------------------------%

\section{Từ Trường của 1 Số Dòng Điện Có Dạng Đơn Giản}

%------------------------------------------------------------------------------%

\section{Tương Tác Giữa 2 Dòng Điện Thẳng Song Song. Định Nghĩa Đơn Vị Ampe}

%------------------------------------------------------------------------------%

\section{Lực Lorentz}

%------------------------------------------------------------------------------%

\section{Khung Dây Có Dòng Điện Đặt Trong Từ Trường}

%------------------------------------------------------------------------------%

\section{Sự Từ Hóa Các Chất. Sắt Từ}

%------------------------------------------------------------------------------%

\section{Từ Trường Trái Đất}

%------------------------------------------------------------------------------%

\section{Từ Trường \& Máy Gia Tốc}

%------------------------------------------------------------------------------%

\printbibliography[heading=bibintoc]
	
\end{document}