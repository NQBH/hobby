\documentclass{article}
\usepackage[backend=biber,natbib=true,style=authoryear]{biblatex}
\addbibresource{/home/hong/1_NQBH/reference/bib.bib}
\usepackage[utf8]{vietnam}
\usepackage{tocloft}
\renewcommand{\cftsecleader}{\cftdotfill{\cftdotsep}}
\usepackage[colorlinks=true,linkcolor=blue,urlcolor=red,citecolor=magenta]{hyperref}
\usepackage{amsmath,amssymb,amsthm,mathtools,float,graphicx,algpseudocode,algorithm,tcolorbox}
\usepackage[inline]{enumitem}
\allowdisplaybreaks
\numberwithin{equation}{section}
\newtheorem{assumption}{Assumption}[section]
\newtheorem{conjecture}{Conjecture}[section]
\newtheorem{corollary}{Corollary}[section]
\newtheorem{hequa}{Hệ quả}[section]
\newtheorem{definition}{Definition}[section]
\newtheorem{dinhnghia}{Định nghĩa}[section]
\newtheorem{example}{Example}[section]
\newtheorem{vidu}{Ví dụ}[section]
\newtheorem{lemma}{Lemma}[section]
\newtheorem{notation}{Notation}[section]
\newtheorem{principle}{Principle}[section]
\newtheorem{problem}{Problem}[section]
\newtheorem{baitoan}{Bài toán}[section]
\newtheorem{proposition}{Proposition}[section]
\newtheorem{question}{Question}[section]
\newtheorem{cauhoi}{Câu hỏi}[section]
\newtheorem{remark}{Remark}[section]
\newtheorem{luuy}{Lưu ý}[section]
\newtheorem{theorem}{Theorem}[section]
\newtheorem{dinhly}{Định lý}[section]
\usepackage[left=0.5in,right=0.5in,top=1.5cm,bottom=1.5cm]{geometry}
\usepackage{fancyhdr}
\pagestyle{fancy}
\fancyhf{}
\lhead{\small \textsc{Sect.} ~\thesection}
\rhead{\small \nouppercase{\leftmark}}
\renewcommand{\sectionmark}[1]{\markboth{#1}{}}
\cfoot{\thepage}
\def\labelitemii{$\circ$}

\title{Problems in Elementary Physics\texttt{/}Grade 11}
\author{Nguyễn Quản Bá Hồng\footnote{Independent Researcher, Ben Tre City, Vietnam\\e-mail: \texttt{nguyenquanbahong@gmail.com}; website: \url{https://nqbh.github.io}.}}
\date{\today}

\begin{document}
\maketitle
\begin{abstract}
	1 bộ sưu tập các bài toán chọn lọc từ cơ bản đến nâng cao cho Vật lý sơ cấp lớp 11. Tài liệu này là phần bài tập bổ sung cho tài liệu chính \href{https://github.com/NQBH/hobby/blob/master/elementary_physics/grade_11/NQBH_elementary_physics_grade_11.pdf}{GitHub\texttt{/}NQBH\texttt{/}hobby\texttt{/}elementary physics\texttt{/}grade 11\texttt{/}lecture}\footnote{Explicitly, \url{https://github.com/NQBH/hobby/blob/master/elementary_physics/grade_11/NQBH_elementary_physics_grade_11.pdf}.} của tác giả viết cho Toán lớp 6. Phiên bản mới nhất của tài liệu này được lưu trữ ở link sau: \href{https://github.com/NQBH/hobby/blob/master/elementary_physics/grade_11/problem/NQBH_elementary_physics_grade_11_problem.pdf}{GitHub\texttt{/}NQBH\texttt{/}hobby\texttt{/}elementary physics\texttt{/}grade 11\texttt{/}problem}\footnote{Explicitly, \url{https://github.com/NQBH/hobby/blob/master/elementary_physics/grade_11/problem/NQBH_elementary_physics_grade_11_problem.pdf}.}.
\end{abstract}
\tableofcontents
\newpage

\section{Điện Học -- Điện Từ Học}

\subsection{Lực tương tác tĩnh điện}

\begin{baitoan}[\cite{Giai_Toan_Vat_Ly_11_tap_1}, Bài toán 1, p. 6]
	``Xác định các đại lượng liên quan đến lực tương tác giữa 2 điện tích điểm đứng yên.''
\end{baitoan}
``Áp dụng công thức $F = \frac{k}{\varepsilon}\frac{|q_1q_2|}{r^2}$ để suy ra giá trị của đại lượng cần xác định. 1 số hiện tượng cần để ý:
\begin{enumerate}
	\item[$\bullet$] Khi cho 2 quả cầu nhỏ dẫn điện như nhau, đã nhiễm điện tiếp xúc nhau \& sau đó tách rời nhau thì tổng điện tích chia đều cho mỗi quả cầu.
	\item[$\bullet$] Hiện tượng cũng xảy ra tương tự khi nối 2 quả cầu như trên bằng dây dẫn mảnh rồi cắt bỏ dây nối.
	\item[$\bullet$] Khi chạm tay vào 1 quả cầu nhỏ dẫn điện đã tích điện thì quả cầu mất điện tích \& trở thành trung hòa.'' -- \cite[p. 7]{Giai_Toan_Vat_Ly_11_tap_1}
\end{enumerate}

\begin{baitoan}[\cite{Giai_Toan_Vat_Ly_11_tap_1}, \textbf{1.1.}, p. 7]
	2 quả cầu kim loại giống nhau, mang các điện tích $q_1,q_2$, đặt trong không khí, cách nhau 1 đoạn $R = 20$cm. Chúng hút nhau bằng lực $F = 3.6\cdot 10^{-4}{\rm N}$. Cho 2 quả cầu tiếp xúc nhau rồi lại đưa về khoảng cách cũ, chúng đẩy nhau bằng lực $F' = 2.025\cdot 10^{-4}{\rm N}$. Tính $q_1,q_2$.
\end{baitoan}
Không cho giá trị cụ thể, 1 tổng quát của bài toán trên:

\begin{baitoan}
	2 quả cầu kim loại giống nhau, mang các điện tích $q_1,q_2$, đặt trong không khí, cách nhau 1 đoạn $R\ {\rm m}$, $R > 0$. Chúng hút nhau bằng lực $F\ {\rm N}$, $F > 0$ . Cho 2 quả cầu tiếp xúc nhau rồi lại đưa về khoảng cách cũ, chúng đẩy nhau bằng lực $F'\ {\rm N}$, $F' > 0$. Tính $q_1,q_2$ (theo $R,F,F'$ đã cho).
\end{baitoan}

\begin{baitoan}[\cite{Giai_Toan_Vat_Ly_11_tap_1}, \textbf{1.2.}, p. 9]
	2 điện tích điểm đặt trong không khí, cách nhau khoảng $R = 20\ {\rm cm}$. Lực tương tác tĩnh điện giữa chúng có 1 giá trị nào đó. Khi đặt trong dầu, ở cùng khoảng cách, lực tương tác tĩnh điện giữa chúng giảm $4$ lần. Hỏi khi đặt trong dầu, khoảng cách giữa các điện tích phải là bao nhiêu để lực tương tác giữa chúng bằng lực tương tác ban đầu trong không khí.
\end{baitoan}
Không cho giá trị cụ thể, 1 tổng quát của bài toán trên:

\begin{baitoan}
	2 điện tích điểm đặt trong không khí, cách nhau khoảng $R\ {\rm m}$. Lực tương tác tĩnh điện giữa chúng có 1 giá trị nào đó. Khi đặt trong dầu, ở cùng khoảng cách, lực tương tác tĩnh điện giữa chúng giảm $n$ lần. Hỏi khi đặt trong dầu, khoảng cách giữa các điện tích phải là bao nhiêu để lực tương tác giữa chúng bằng lực tương tác ban đầu trong không khí.
\end{baitoan}

\begin{baitoan}[\cite{Giai_Toan_Vat_Ly_11_tap_1}, \textbf{1.3.}, p. 10]
	2 điện tích điểm bằng nhau đặt trong chân không, cách nhau đoạn $R = 4\ {\rm cm}$. Lực đẩy tĩnh điện giữa chúng là $F = 10^{-5}\ {\rm N}$.
	\begin{enumerate*}
		 \item[(a)] Tìm độ lớn mỗi điện tích.
		 \item[(b)] Tìm khoảng cách $R_1$ giữa chúng để lực đẩy tĩnh điện là $F_1 = 2.5\cdot 10^{-6}\ {\rm N}$.
	\end{enumerate*}
\end{baitoan}
Không cho giá trị cụ thể, 1 tổng quát của bài toán trên:

\begin{baitoan}
	2 điện tích điểm bằng nhau đặt trong chân không, cách nhau đoạn $R\ {\rm m}$. Lực đẩy tĩnh điện giữa chúng là $F\ {\rm N}$.
	\begin{enumerate*}
		\item[(a)] Tìm độ lớn mỗi điện tích.
		\item[(b)] Tìm khoảng cách $R_1$ giữa chúng để lực đẩy tĩnh điện là $F_1\ {\rm N}$.
	\end{enumerate*}
\end{baitoan}

\begin{baitoan}[\cite{Giai_Toan_Vat_Ly_11_tap_1}, \textbf{1.4.}, p. 10]
	2 hạt bụi trong không khí ở cách nhau 1 đoạn $R = 3\ {\rm cm}$, mỗi hạt mang điện tích $q = -9.6\cdot 10^{-13}\ {\rm C}$.
	\begin{enumerate*}
		\item[(a)] Tính lực tĩnh điện giữa 2 hạt.
		\item[(b)] Tính số electron dư trong mỗi hạt bụi, biết điện tích mỗi electron là $e = 1.6\cdot 10^{-19}\ {\rm C}$.
	\end{enumerate*}
\end{baitoan}
Không cho giá trị cụ thể, 1 tổng quát của bài toán trên:

\begin{baitoan}
	2 hạt bụi trong không khí ở cách nhau 1 đoạn $R\ {\rm m}$, mỗi hạt mang điện tích $q\ {\rm C}$.
	\begin{enumerate*}
		\item[(a)] Tính lực tĩnh điện giữa 2 hạt.
		\item[(b)] Tính số electron dư trong mỗi hạt bụi, biết điện tích mỗi electron là $e = 1.6\cdot 10^{-19}\ {\rm C}$.
	\end{enumerate*}
\end{baitoan}

\begin{baitoan}[\cite{Giai_Toan_Vat_Ly_11_tap_1}, \textbf{1.5.}, p. 11]
	Mỗi proton có khối lượng $m = 1.67\cdot 10^{-27}\ {\rm kg}$, điện tích $q = 1.6\cdot 10^{-19}\ {\rm C}$. Hỏi lực đẩy Coulomb giữa 2 proton lớn hơn lực hấp dẫn giữa chúng bao nhiêu lần?
\end{baitoan}

\begin{proof}[Hint]
	Lực hấp dẫn $F = G\frac{m_1m_2}{r^2}$, $G = 6.67\cdot 10^{-11}$ (SI).
\end{proof}

\begin{baitoan}[\cite{Giai_Toan_Vat_Ly_11_tap_1}, \textbf{1.6.}, p. 11]
	2 vật nhỏ giống nhau, mỗi vật thừa 1 electron. Tìm khối lượng mỗi vật để lực tĩnh điện bằng lực hấp dẫn.
\end{baitoan}
1 tổng quát của bài toán trên:

\begin{baitoan}
	2 vật nhỏ giống nhau, mỗi vật thừa $n\in\mathbb{N}^\star$ electron. Tìm khối lượng mỗi vật (theo $n$) để lực tĩnh điện bằng lực hấp dẫn.
\end{baitoan}

\begin{baitoan}[\cite{Giai_Toan_Vat_Ly_11_tap_1}, \textbf{1.7.}, p. 11]
	Electron quay quanh hạt nhân nguyên tử hydro theo quỹ đạo tròn với bán kính $R = 5\cdot 10{-11}\ {\rm m}$.
	\begin{enumerate*}
		\item[(a)] Tính độ lớn lực hướng tâm đặt lên electron.
		\item[(b)] Tính vận tốc \& tần số chuyển động của electron.
	\end{enumerate*}
	Coi electron \& hạt nhân trong nguyên tử hydro tương tác theo luật tĩnh điện.
\end{baitoan}

\begin{proof}[Hint]
	Trong chuyển động tròn đều: $F = ma = m\frac{v^2}{R}$, $v = 2\pi Rn$, $n$ là tần số chuyển động. 
\end{proof}

\begin{baitoan}[\cite{Giai_Toan_Vat_Ly_11_tap_1}, \textbf{1.8.}, p. 11]
	2 vật nhỏ mang điện tích đặt trong không khí cách nhau đoạn $R = 1\ {\rm m}$, đẩy nhau bằng lực $F = 1.8\ {\rm N}$. Điện tích tổng cộng của 2 vật là $Q = 3\cdot 10^{-5}\ {\rm C}$. Tính điện tích mỗi vật.
\end{baitoan}
Không cho giá trị cụ thể, 1 tổng quát của bài toán trên:

\begin{baitoan}
	2 vật nhỏ mang điện tích đặt trong không khí cách nhau đoạn $R\ {\rm m}$, đẩy nhau bằng lực $F\ {\rm N}$. Điện tích tổng cộng của 2 vật là $Q\ {\rm C}$. Tính điện tích mỗi vật.
\end{baitoan}

\begin{baitoan}[\cite{Giai_Toan_Vat_Ly_11_tap_1}, \textbf{1.8.}, p. 11]
	2 quả cầu kim loại nhỏ như nhau mang các điện tích $q_1,q_2$ đặt trong không khí cách nhau $R = 2\ {\rm cm}$, đẩy nhau bằng lực $F = 2.7\cdot 10^{-4}\ {\rm N}$. Cho 2 quả cầu tiếp xúc nhau rồi lại đưa về vị trí cũ, chúng đẩy nhau bằng lực $F' = 3.6\cdot 10^{-4}\ {\rm N}$. Tính $q_1,q_2$.
\end{baitoan}
Không cho giá trị cụ thể, 1 tổng quát của bài toán trên:

\begin{baitoan}
	2 quả cầu kim loại nhỏ như nhau mang các điện tích $q_1,q_2$ đặt trong không khí cách nhau $R\ {\rm m}$, đẩy nhau bằng lực $F\ {\rm N}$. Cho 2 quả cầu tiếp xúc nhau rồi lại đưa về vị trí cũ, chúng đẩy nhau bằng lực $F'\ {\rm N}$. Tính $q_1,q_2$.
\end{baitoan}

\subsection{Tìm lực tổng hợp tác dụng lên 1 điện tích}

%------------------------------------------------------------------------------%

\section{Quang Hình Học}

%------------------------------------------------------------------------------%

\section{Solutions}

%------------------------------------------------------------------------------%

Tài liệu: \cite{SGK_Hoa_Hoc_11_co_ban, SGK_Hoa_Hoc_11_nang_cao}.

\printbibliography[heading=bibintoc]
	
\end{document}