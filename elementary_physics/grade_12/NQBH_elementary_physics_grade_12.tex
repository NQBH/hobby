\documentclass{article}
\usepackage[backend=biber,natbib=true,style=authoryear]{biblatex}
\addbibresource{/home/hong/1_NQBH/reference/bib.bib}
\usepackage[utf8]{vietnam}
\usepackage{tocloft}
\renewcommand{\cftsecleader}{\cftdotfill{\cftdotsep}}
\usepackage[colorlinks=true,linkcolor=blue,urlcolor=red,citecolor=magenta]{hyperref}
\usepackage{amsmath,amssymb,amsthm,mathtools,float,graphicx,algpseudocode,algorithm,tcolorbox,tikz,tkz-tab,subcaption}
\DeclareMathOperator{\arccot}{arccot}
\usepackage[inline]{enumitem}
\allowdisplaybreaks
\numberwithin{equation}{section}
\newtheorem{assumption}{Assumption}[section]
\newtheorem{nhanxet}{Nhận xét}[section]
\newtheorem{conjecture}{Conjecture}[section]
\newtheorem{corollary}{Corollary}[section]
\newtheorem{hequa}{Hệ quả}[section]
\newtheorem{definition}{Definition}[section]
\newtheorem{dinhnghia}{Định nghĩa}[section]
\newtheorem{example}{Example}[section]
\newtheorem{vidu}{Ví dụ}[section]
\newtheorem{lemma}{Lemma}[section]
\newtheorem{notation}{Notation}[section]
\newtheorem{principle}{Principle}[section]
\newtheorem{problem}{Problem}[section]
\newtheorem{baitoan}{Bài toán}[section]
\newtheorem{proposition}{Proposition}[section]
\newtheorem{menhde}{Mệnh đề}[section]
\newtheorem{question}{Question}[section]
\newtheorem{cauhoi}{Câu hỏi}[section]
\newtheorem{quytac}{Quy tắc}
\newtheorem{remark}{Remark}[section]
\newtheorem{luuy}{Lưu ý}[section]
\newtheorem{theorem}{Theorem}[section]
\newtheorem{tiende}{Tiên đề}[section]
\newtheorem{dinhly}{Định lý}[section]
\usepackage[left=0.5in,right=0.5in,top=1.5cm,bottom=1.5cm]{geometry}
\usepackage{fancyhdr}
\pagestyle{fancy}
\fancyhf{}
\lhead{\small \textsc{Sect.} ~\thesection}
\rhead{\small \nouppercase{\leftmark}}
\renewcommand{\sectionmark}[1]{\markboth{#1}{}}
\cfoot{\thepage}
\def\labelitemii{$\circ$}

\title{Some Topics in Elementary Physics\texttt{/}Grade 12}
\author{Nguyễn Quản Bá Hồng\footnote{Independent Researcher, Ben Tre City, Vietnam\\e-mail: \texttt{nguyenquanbahong@gmail.com}; website: \url{https://nqbh.github.io}.}}
\date{\today}

\begin{document}
\maketitle
\begin{abstract}
	
\end{abstract}
\setcounter{secnumdepth}{4}
\setcounter{tocdepth}{3}
\tableofcontents
\newpage

%------------------------------------------------------------------------------%

\section{Động Lực Học Vật Rắn}

\subsection{Chuyển Động Quay của Vật Rắn Quanh 1 Trục Cố Định}

%------------------------------------------------------------------------------%

\subsection{Phương Trình Động Lực Học của Vật Rắn Quay Quanh 1 Trục Cố Định}

%------------------------------------------------------------------------------%

\subsection{Momen Động Lượng. Định Luật Bảo Toàn Momen Động Lượng}

%------------------------------------------------------------------------------%

\subsection{Động Năng của Vật Rắn Quay Quanh 1 Trục Cố Định}

%------------------------------------------------------------------------------%

\subsection{Bài Tập về Động Lực Học Vật Rắn}

%------------------------------------------------------------------------------%

\subsection{Tóm Tắt Chương I}

%------------------------------------------------------------------------------%

\section{Dao Động Cơ}

\subsection{Dao Động Điều Hòa}

%------------------------------------------------------------------------------%

\subsection{Con Lắc Đơn. Con Lắc Vật Lý}

%------------------------------------------------------------------------------%

\subsection{Năng Lượng trong Dao Động Điều Hòa}

%------------------------------------------------------------------------------%

\subsection{Bài Tập về Dao Động Điều Hòa}

%------------------------------------------------------------------------------%

\subsection{Dao Động Tắt Dần \& Dao Động Duy Trì}

%------------------------------------------------------------------------------%

\subsection{Dao Động Cưỡng Bức. Cộng Hưởng}

%------------------------------------------------------------------------------%

\subsection{Tổng Hợp Dao Động}

%------------------------------------------------------------------------------%

\subsection{Thực Hành: Xác Định Chu Kỳ Dao Động của Con Lắc Đơn hoặc Con Lắc Lò Xo \& Gia Tốc Trọng Trường}

%------------------------------------------------------------------------------%

\subsection{Tóm Tắt Chương II}

%------------------------------------------------------------------------------%

\section{Sóng Cơ}

\subsection{Sóng Cơ. Phương Trình Sóng}

%------------------------------------------------------------------------------%

\subsection{Phản Xạ Sóng. Sóng Dừng}

%------------------------------------------------------------------------------%

\subsection{Giao Thoa Sóng}

%------------------------------------------------------------------------------%

\subsection{Sóng Âm. Nguồn Nhạc Âm}

%------------------------------------------------------------------------------%

\subsection{Hiệu Ứng Doppler}

%------------------------------------------------------------------------------%

\subsection{Bài Tập về Sóng Cơ}

%------------------------------------------------------------------------------%

\subsection{Thực Hành: Xác Định Tốc Độ Truyền Âm}

%------------------------------------------------------------------------------%

\subsubsection{Tóm Tắt Chương III}

%------------------------------------------------------------------------------%

\section{Dao Động \& Sóng Điện Từ}

\subsection{Dao Động Điện Từ}

%------------------------------------------------------------------------------%

\subsection{Bài Tập về Dao Động Điện Từ}

%------------------------------------------------------------------------------%

\subsection{Điện Từ Trường}

%------------------------------------------------------------------------------%

\subsection{Sóng Điện Từ}

%------------------------------------------------------------------------------%

\subsection{Truyền Thông bằng Sóng Điện Từ}

%------------------------------------------------------------------------------%

\subsection{Bộ Dao Động Thạch Anh (Quartz)}

%------------------------------------------------------------------------------%

\subsection{Tóm Tắt chương IV}

%------------------------------------------------------------------------------%

\section{Dòng Điện Xoay Chiều}

\subsection{Dòng Điện Xoay Chiều. Mạch Điện Xoay Chiều Chỉ Có Điện Trở Thuần}

%------------------------------------------------------------------------------%

\subsection{Mạch Điện Xoay Chiều Chỉ Có Tụ Điện, Cuộn Cảm}

%------------------------------------------------------------------------------%

\subsection{Mạch Có $R,L,C$ Mắc Nối Tiếp. Cộng Hưởng Điện}

%------------------------------------------------------------------------------%

\subsection{Công Suất của Dòng Điện Xoay Chiều. Hệ Số Công Suất}

%------------------------------------------------------------------------------%

\subsection{Máy Phát Điện Xoay Chiều}

%------------------------------------------------------------------------------%

\subsection{Động Cơ Không Đồng Bộ 3 Pha}

%------------------------------------------------------------------------------%

\subsection{Máy Biến Áp. Truyền Tải Điện Năng}

%------------------------------------------------------------------------------%

\subsection{Bài Tập về Dòng Điện Xoay Chiều}

%------------------------------------------------------------------------------%

\subsection{Sản Xuất Điện}

%------------------------------------------------------------------------------%

\subsection{Thực Hành: Khảo Sát Đoạn Mạch Điện Xoay Chiều có $R,L,C$ Mắc Nối Tiếp}

%------------------------------------------------------------------------------%

\subsection{Tóm Tắt Chương V}

%------------------------------------------------------------------------------%

\section{Sóng Ánh Sáng}

\subsection{Tán Sắc Ánh Sáng}

%------------------------------------------------------------------------------%

\subsection{Nhiễu Xạ Ánh Sáng. Giao Thoa Ánh Sáng}

%------------------------------------------------------------------------------%

\subsection{Khoảng Vân. Bước Sóng \& Màu Sắc Ánh Sáng}

%------------------------------------------------------------------------------%

\subsection{Bài Tập về Giao Thoa Ánh Sáng}

%------------------------------------------------------------------------------%

\subsection{Máy Quang Phổ. Các Loại Quang Phổ}

%------------------------------------------------------------------------------%

\subsection{Tia Hồng Ngoại. Tia Tử Ngoại}

%------------------------------------------------------------------------------%

\subsection{Tia X. Thuyết Điện Từ Ánh Sáng. Thang Sóng Điện Từ}

%------------------------------------------------------------------------------%

\subsection{Cầu Vồng}

%------------------------------------------------------------------------------%

\subsection{Thực Hành: Xác Định Bước Sóng Ánh Sáng}

%------------------------------------------------------------------------------%

\subsection{Tóm Tắt Chương VI}

%------------------------------------------------------------------------------%

\section{Lượng Tử Ánh Sáng}

\subsection{Hiện Tượng Quang Điện Ngoài. Các Định Luật Quang Điện}

%------------------------------------------------------------------------------%

\subsection{Thuyết Lượng Tử Ánh Sáng. Lưỡng Tính Sóng - Hạt của Ánh Sáng}

%------------------------------------------------------------------------------%

\subsection{Bài Tập về Hiện Tượng Quang Điện}

%------------------------------------------------------------------------------%

\subsection{Hiện Tượng Quang Điện Trong. Quang Điện Trở \& Pin Quang Điện}

%------------------------------------------------------------------------------%

\subsection{Mẫu Nguyên Tử Bo \& Quang Phổ Vạch của Nguyên Tử Hydro}

%------------------------------------------------------------------------------%

\subsection{Hấp Thụ \& Phản Xạ Lọc Lựa Ánh Sáng. Màu Sắc Các Vật}

%------------------------------------------------------------------------------%

\subsection{Sự Phát Quang. Sơ Lược về Laze}

%------------------------------------------------------------------------------%

\subsection{Cấu Tạo \& Hoạt Động của Laze}

%------------------------------------------------------------------------------%

\subsection{Tóm Tắt Chương VII}

%------------------------------------------------------------------------------%

\section{Sơ Lược về Thuyết Tương Đối Hẹp}

\subsection{Thuyết Tương Đối Hẹp}

%------------------------------------------------------------------------------%

\subsection{Hệ Thức Einstein Giữa Khối Lượng \& Năng Lượng}

%------------------------------------------------------------------------------%

\subsection{Tóm Tắt Chương VIII}

%------------------------------------------------------------------------------%

\section{Hạt Nhân Nguyên Tử}

\subsection{Cấu Tạo của Hạt Nhân Nguyên tử. Độ Hụt Khối}

%------------------------------------------------------------------------------%

\subsection{Phóng Xạ}

%------------------------------------------------------------------------------%

\subsection{Phản Ứng Hạt Nhân}

%------------------------------------------------------------------------------%

\subsection{Bài Tập về Phóng Xạ \& Phản Ứng Hạt Nhân}

%------------------------------------------------------------------------------%

\subsection{Phản Ứng Phân Hạch}

%------------------------------------------------------------------------------%

\subsection{Phản Ứng Nhiệt Hạch}

%------------------------------------------------------------------------------%

\subsection{Tóm Tắt Chương IX}

%------------------------------------------------------------------------------%

\section{Từ Vi Mô đến Vĩ Mô}

\subsection{Các Hạt Sơ Cấp}

%------------------------------------------------------------------------------%

\subsection{Mặt Trời. Hệ Mặt Trời}

%------------------------------------------------------------------------------%

\subsection{Sao. Thiên Hà}

%------------------------------------------------------------------------------%

\subsection{Thuyết Big Bang}

%------------------------------------------------------------------------------%

\subsection{Liệu Có -- Hoặc Đã Từng Có -- Sự Sóng trên Hỏa Tinh hay không?}

\subsection{Tóm Tắt Chương X}

%------------------------------------------------------------------------------%

\printbibliography[heading=bibintoc]
	
\end{document}