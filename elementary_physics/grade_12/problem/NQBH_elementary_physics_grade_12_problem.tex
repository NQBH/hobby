\documentclass{article}
\usepackage[backend=biber,natbib=true,style=authoryear]{biblatex}
\addbibresource{/home/hong/1_NQBH/reference/bib.bib}
\usepackage[utf8]{vietnam}
\usepackage{tocloft}
\renewcommand{\cftsecleader}{\cftdotfill{\cftdotsep}}
\usepackage[colorlinks=true,linkcolor=blue,urlcolor=red,citecolor=magenta]{hyperref}
\usepackage{amsmath,amssymb,amsthm,mathtools,float,graphicx,algpseudocode,algorithm,tcolorbox}
\usepackage[inline]{enumitem}
\allowdisplaybreaks
\numberwithin{equation}{section}
\newtheorem{assumption}{Assumption}[section]
\newtheorem{conjecture}{Conjecture}[section]
\newtheorem{corollary}{Corollary}[section]
\newtheorem{hequa}{Hệ quả}[section]
\newtheorem{definition}{Definition}[section]
\newtheorem{dinhnghia}{Định nghĩa}[section]
\newtheorem{example}{Example}[section]
\newtheorem{vidu}{Ví dụ}[section]
\newtheorem{lemma}{Lemma}[section]
\newtheorem{notation}{Notation}[section]
\newtheorem{principle}{Principle}[section]
\newtheorem{problem}{Problem}[section]
\newtheorem{baitoan}{Bài toán}[section]
\newtheorem{proposition}{Proposition}[section]
\newtheorem{question}{Question}[section]
\newtheorem{cauhoi}{Câu hỏi}[section]
\newtheorem{remark}{Remark}[section]
\newtheorem{luuy}{Lưu ý}[section]
\newtheorem{theorem}{Theorem}[section]
\newtheorem{dinhly}{Định lý}[section]
\usepackage[left=0.5in,right=0.5in,top=1.5cm,bottom=1.5cm]{geometry}
\usepackage{fancyhdr}
\pagestyle{fancy}
\fancyhf{}
\lhead{\small Sect.~\thesection}
\rhead{\small \nouppercase{\leftmark}}
\renewcommand{\sectionmark}[1]{\markboth{#1}{}}
\cfoot{\thepage}
\def\labelitemii{$\circ$}

\title{Some Topics in Elementary Physics\texttt{/}Grade 12}
\author{Nguyễn Quản Bá Hồng\footnote{Independent Researcher, Ben Tre City, Vietnam\\e-mail: \texttt{nguyenquanbahong@gmail.com}; website: \url{https://nqbh.github.io}.}}
\date{\today}

\begin{document}
\maketitle
\begin{abstract}
	1 bộ sưu tập các bài toán vật lý chọn lọc từ cơ bản đến nâng cao cho Vật lý sơ cấp lớp 12. Tài liệu này là phần bài tập bổ sung cho tài liệu chính \href{https://github.com/NQBH/hobby/blob/master/elementary_physics/grade_12/NQBH_elementary_physics_grade_12.pdf}{GitHub\texttt{/}NQBH\texttt{/}hobby\texttt{/}elementary physics\texttt{/}grade 12\texttt{/}lecture}\footnote{Explicitly, \url{https://github.com/NQBH/hobby/blob/master/elementary_physics/grade_12/NQBH_elementary_physics_grade_12.pdf}.} của tác giả viết cho Toán lớp 6. Phiên bản mới nhất của tài liệu này được lưu trữ ở link sau: \href{https://github.com/NQBH/hobby/blob/master/elementary_physics/grade_12/problem/NQBH_elementary_physics_grade_12_problem.pdf}{GitHub\texttt{/}NQBH\texttt{/}hobby\texttt{/}elementary physics\texttt{/}grade 12\texttt{/}problem}\footnote{Explicitly, \url{https://github.com/NQBH/hobby/blob/master/elementary_physics/grade_12/problem/NQBH_elementary_physics_grade_12_problem.pdf}.}.
\end{abstract}
\tableofcontents
\newpage

%------------------------------------------------------------------------------%

%------------------------------------------------------------------------------%

\section{Động Lực Học Vật Rắn}

\subsection{Chuyển Động Quay của Vật Rắn Quanh 1 Trục Cố Định}

%------------------------------------------------------------------------------%

\subsection{Phương Trình Động Lực Học của Vật Rắn Quay Quanh 1 Trục Cố Định}

%------------------------------------------------------------------------------%

\subsection{Momen Động Lượng. Định Luật Bảo Toàn Momen Động Lượng}

%------------------------------------------------------------------------------%

\subsection{Động Năng của Vật Rắn Quay Quanh 1 Trục Cố Định}

%------------------------------------------------------------------------------%

\subsection{Bài Tập về Động Lực Học Vật Rắn}

%------------------------------------------------------------------------------%

\subsection{Tóm Tắt Chương I}

%------------------------------------------------------------------------------%

\newpage
\section{Dao Động Cơ}

\subsection{Dao Động Điều Hòa}

\subsubsection{Bài Tập Lý Thuyết}

\begin{baitoan}
	Chuyển động nào dưới đây không phải là dao động?
	\begin{enumerate*}
		\item[{\rm\sf A.}] Chuyển động của quả lắc đồng hồ.
		\item[{\rm\sf B.}] Chuyển động của đầu kim đồng hồ.
		\item[{\rm\sf C.}] Chuyển động của con lắc lò xo.
		\item[{\rm\sf D.}] Chuyển động của cái võng.
	\end{enumerate*}
\end{baitoan}

\begin{baitoan}
	Tìm phát biểu sai về chu kỳ của vật dao động điều hòa.
	\begin{enumerate*}
		\item[{\rm\sf A.}] Chu kỳ là khoảng thời gian ngắn nhất để ly độ \& vận tốc của vật trở lại độ lớn như cũ.
		\item[{\rm\sf B.}] Chu kỳ là khoảng thời gian vật thực hiện được $1$ dao động toàn phần.
		\item[{\rm\sf C.}] Thời gian vật đi hết chiều dài quỹ đạo là $\frac{1}{2}$ chu kỳ.
		\item[{\rm\sf D.}] Thời gian ngắn nhất mà vật đi từ vị trí cân bằng đến vị trí biên là $\frac{1}{4}$ chu kỳ.
	\end{enumerate*}
\end{baitoan}

\begin{baitoan}
	Tìm phát biểu sai về ly độ, vận tốc, \& gia tốc của vật dao động điều hòa.
	\begin{enumerate*}
		\item[{\rm\sf A.}] Khi vật qua vị trí cân bằng thì vận tốc \& gia tốc đều có độ lớn cực đại.
		\item[{\rm\sf B.}] Khi vật qua vị trí cân bằng thì vận tốc có độ lớn cực đại \& ly độ bằng $0$.
		\item[{\rm\sf C.}] Khi vật ở biên thì vận tốc bằng $0$ \& gia tốc có độ lớn cực đại.
		\item[{\rm\sf D.}] Khi vật ở biên thì vận tốc bằng $0$ \& ly độ có độ lớn cực đại.
	\end{enumerate*}
\end{baitoan}

\begin{baitoan}
	Tìm phát biểu đúng về vận tốc \& gia tốc của vật dao động điều hòa.
	\begin{enumerate*}
		\item[{\rm\sf A.}] Vận tốc có độ lớn cực đại ở vị trí biên, gia tốc có độ lớn cực đại ở vị trí cân bằng.
		\item[{\rm\sf B.}] Vận tốc \& gia tốc có độ lớn cực đại ở vị trí biên.
		\item[{\rm\sf C.}] Vận tốc \& gia tốc có độ lớn cực đại ở vị trí cân bằng.
		\item[{\rm\sf D.}] Vận tốc có độ lớn cực đại ở vị trí cân bằng, gia tốc có độ lớn cực đại ở vị trí biên.
	\end{enumerate*}
\end{baitoan}

\begin{baitoan}
	1 vật dao động điều hòa, khi ở vị trí biên thì:
	\begin{enumerate*}
		\item[{\rm\sf A.}] Vận tốc \& gia tốc bằng $O$.
		\item[{\rm\sf B.}] Vận tốc có độ lớn cực đại \& gia tốc bằng $0$.
		\item[{\rm\sf C.}] Vận tốc bằng $0$ \& gia tốc có độ lớn cực đại.
		\item[{\rm\sf D.}] Vận tốc \& gia tốc có độ lớn cực đại.
	\end{enumerate*}
\end{baitoan}

\begin{baitoan}
	Tìm phát biểu sai đối với $1$ vật dao động điều hòa.
	\begin{enumerate*}
		\item[{\rm\sf A.}] Đồ thị của ly độ, vận tốc, \& gia tốc của vật đều có dạng hình sin.
		\item[{\rm\sf B.}] Ly độ, vận tốc, \& gia tốc của vật biến thiee điều hòa cùng tần số.
		\item[{\rm\sf C.}] Ly độ là đạo hàm bậc nhất của vận tốc theo thời gian.
		\item[{\rm\sf D.}] Gia tốc là đạo hàm bậc nhất của vận tốc theo thời gian.
	\end{enumerate*}
\end{baitoan}

\begin{baitoan}
	Trong dao động điều hòa, ly độ \& gia tốc biến thiên điều hòa
	\begin{enumerate*}
		\item[{\rm\sf A.}] cùng pha với nhau.
		\item[{\rm\sf B.}] ngược pha với nhau.
		\item[{\rm\sf C.}] lệch pha nhau $\frac{\pi}{2}$.
		\item[{\rm\sf D.}] lệch pha nhau $\frac{\pi}{4}$.
	\end{enumerate*}
\end{baitoan}

\begin{baitoan}
	Trong dao động điều hòa, vận tốc biến thiên điều hòa
	\begin{enumerate*}
		\item[{\rm\sf A.}] trễ pha $\frac{\pi}{2}$ so với ly độ.
		\item[{\rm\sf B.}] sớm pha $\frac{\pi}{2}$ so với ly độ.
		\item[{\rm\sf C.}] ngược pha với ly độ.
		\item[{\rm\sf D.}] cùng pha với ly độ.
	\end{enumerate*}
\end{baitoan}

\begin{baitoan}
	Nếu bỏ qua ma sát thì cơ năng của $1$ vật dao động điều hòa không đổi \& tỷ lệ với
	\begin{enumerate*}
		\item[{\rm\sf A.}] bình phương tần số.
		\item[{\rm\sf B.}] bình phương biên độ.
		\item[{\rm\sf C.}] bình phương tần số góc.
		\item[{\rm\sf D.}] bình phương chu kỳ.
	\end{enumerate*}
\end{baitoan}

\begin{baitoan}
	Chọn câu sai.
	\begin{enumerate*}
		\item[{\rm\sf A.}] Vận tốc không đổi chiều \& có độ lớn cực đại khi vật dao động điều hòa đi qua vị trí cân bằng.
		\item[{\rm\sf B.}] Vận tốc, gia tốc của vật dao động điều hòa biến thiên theo định luật  dạng sin hay côsin đối với thời gian.
		\item[{\rm\sf C.}] Khi vật dao động điều hòa ở vị trí biên thì động năng của vật cực đại, còn thế năng bằng $0$.
		\item[{\rm\sf D.}] Khi vật dao động điều hòa đi qua vị trí cân bằng thì gia tốc bằng $0$, vận tốc có độ lớn cực đại.
	\end{enumerate*}
\end{baitoan}

\begin{baitoan}
	Chọn câu sai.
	\begin{enumerate*}
		\item[{\rm\sf A.}] Pha dao động là đại lượng xác định vị trí \& chiều chuyển động của vật tại thời điểm $t$.
		\item[{\rm\sf B.}] Tần số góc của dao động điều hòa tương ứng với tốc độ góc của chuyển động tròn đều.
		\item[{\rm\sf C.}] Biên độ dao động là $1$ hằng số dương.
		\item[{\rm\sf D.}] Chu kỳ dao động là khoảng thời gian ngắn nhất để vật dao động điều hòa trở lại ly độ cũ. 
	\end{enumerate*}
\end{baitoan}

\begin{baitoan}
	Chọn câu sai. Với vật dao động điều hòa:
	\begin{enumerate*}
		\item[{\rm\sf A.}] Chu kỳ dao động không phụ thuộc vào biên độ dao động.
		\item[{\rm\sf B.}] Khi vật đi từ vị trí cân bằng ra $2$ biên thì vận tốc \& gia tốc luôn cùng dấu.
		\item[{\rm\sf C.}] Gia tốc của vật luôn hướng về vị trí cân bằng \& có độ lớn tỷ lệ với độ lớn của ly độ.
		\item[{\rm\sf D.}] Biên độ dao động của vật phụ thuộc vào cách kích thích ban đầu cho vật dao động.
	\end{enumerate*}
\end{baitoan}

%------------------------------------------------------------------------------%

\subsubsection{Bài Tập Toán Lý}

\begin{baitoan}
	Gọi $x$ là ly độ, $\omega$ là tần số góc thì gia tốc trong dao động điều hòa được xác định bởi biểu thức:
	\begin{enumerate*}
		\item[{\rm\sf A.}] $a = x\omega^2$;
		\item[{\rm\sf B.}] $a = \omega x^2$;
		\item[{\rm\sf C.}] $a = -x\omega^2$;
		\item[{\rm\sf D.}] $a = -\omega x^2$.
	\end{enumerate*}
\end{baitoan}

\begin{baitoan}[\cite{SGK_Vat_Ly_12_nang_cao}, Câu hỏi 2, p. 34]
	Xét 3 đại lượng đặc trưng $A,\varphi,\omega$ cho dao động điều hòa của $1$ con lắc lò xo đã cho. Những đại lượng nào có thể có những giá trị khác nhau, tùy thuộc cách kích thích dao động? Đại lượng nào chỉ có $1$ giá trị xác định đối với con lắc lò xo đã cho?
\end{baitoan}

\begin{baitoan}[\cite{SGK_Vat_Ly_12_nang_cao}, Câu hỏi 3, p. 34]
	Nói rõ về thứ nguyên của các đại lượng $A,\varphi,\omega$.
\end{baitoan}
Tốc độ của chất điểm dao động điều hòa cực đại khi ly độ bằng $0$ vì ``ở vị trí cân bằng $x = 0$ thì vận tốc $v$ có độ lớn cực đại bằng $\omega A$.'' -- \cite[p. 32]{SGK_Vat_Ly_12_nang_cao}

\begin{baitoan}[\cite{SGK_Vat_Ly_12_nang_cao}, \textbf{4.}, p. 34]
	\begin{enumerate*}
		\item[(a)] Thử lại rằng $x = A_1\cos\omega t + A_2\sin\omega t$, trong đó $A_1$ \& $A_2$ là 2 hằng số bất kỳ, cũng là nghiệm của phương trình $x'' + \omega^2x = 0$.
		\item[(b)] Chứng tỏ rằng, nếu chọn $A_1$ \& $A_2$ trong biểu thức ở vế phải của $x = A_1\cos\omega t + A_2\sin\omega t$ như sau: $A_1 = A\cos\varphi$, $A_2 = -A\sin\varphi$ thì biểu thức ấy trùng với biểu thức ở vế phải của $x = A\cos(\omega t + \varphi)$.
	\end{enumerate*}
\end{baitoan}

\begin{baitoan}[1 tổng quát của \cite{SGK_Vat_Ly_12_nang_cao}, \textbf{5.}, p. 34]
	Phương trình dao động của $1$ vật là: $x = A\cos(\omega t + \varphi)$ (m).
	\begin{enumerate*}
		\item[(a)] Xác định biên độ, tần số góc, chu kỳ, \& tần số của dao động.
		\item[(b)] Xác định pha của dao động tại thời điểm $t = t_0$ s, từ đó suy ra ly độ tại thời điểm ấy.
		\item[(c)] Vẽ vector quay biểu diễn dao động vào thời điểm $t = 0$.
	\end{enumerate*}
\end{baitoan}

\begin{baitoan}[1 tổng quát của \cite{SGK_Vat_Ly_12_nang_cao}, \textbf{6.}, p. 34]
	1 vật dao động điều hòa với biên độ $A$ m \& chu kỳ $T$ s.
	\begin{enumerate*}
		\item[(a)] Viết phương trình dao động của vật, chọn góc thời gian là lúc nó đi qua vị trí cân bằng theo chiều dương.
		\item[(b)] Tính ly độ của vật tại thời điểm $t = t_0$ s.
	\end{enumerate*}
\end{baitoan}

\begin{baitoan}[1 tổng quát của \cite{SGK_Vat_Ly_12_nang_cao}, \textbf{7.}, p. 34]
	1 vật nặng treo vào $1$ lò xo làm cho lò xo dãn ra $a$ m. Cho vật dao động. Tìm chu kỳ dao động ấy.
\end{baitoan}

\begin{baitoan}[1 tổng quát của \cite{Giai_Toan_Trac_Nghiem_Vat_Ly_12_tap_1}, \textbf{4.1}, p. 48]
	Cho các phương trình chuyển động sau đây, trong đó $A,A_i > 0$:
	\begin{enumerate*}
		\item[$\bullet$] $x_1 = -A\cos(\omega t + \varphi)$.
		\item[$\bullet$] $x_2 = A\sin(\omega t + \varphi)$.
		\item[$\bullet$] $x_3 = -A\sin(\omega t + \varphi)$.
		\item[$\bullet$] $x_4 = A\cos^2(\omega t + \varphi)$.
		\item[$\bullet$] $x_5 = A_1\cos(\omega t + \varphi) + A_2\sin(\omega t + \varphi)$.
		\item[$\bullet$] $x_6 = A_1\cos(\omega t + \varphi) - A_2\sin(\omega t + \varphi)$.
		\item[$\bullet$] $x_7 = A_1\sin(\omega t + \varphi) - A_2\cos(\omega t + \varphi)$.
		\item[$\bullet$] $x_8 = \cos^2(\omega t + \varphi) - \sin^2(\omega t + \varphi) = 2\cos^2(\omega t + \varphi) - 1 = 1 - 2\sin^2(\omega t + \varphi)$.
		\item[$\bullet$] $x_9 = 2\sin(\omega t + \varphi)\cos(\omega t + \varphi)$.
	\end{enumerate*}
	$\bullet$ $x_{10} = 4\cos^3(\omega t + \varphi) - 3\cos(\omega t + \varphi) = \cos^3(\omega t + \varphi) - 3\sin^2(\omega t + \varphi)\cos(\omega t + \varphi) = 2\cos^3(\omega t + \varphi) - \cos(\omega t + \varphi) - 2\sin^2(\omega t + \varphi)\cos(\omega t + \varphi) = \cos(\omega t + \varphi) - 4\sin^2(\omega t + \varphi)\cos(\omega t + \varphi)$. $\bullet$ $x_{11} = 3\sin(\omega t + \varphi) - 4\sin^3(\omega t + \varphi) = 3\sin(\omega t + \varphi)\cos^2(\omega t + \varphi) - \sin^3(\omega t + \varphi) = 4\sin(\omega t + \varphi)\cos^2(\omega t + \varphi) - \sin(\omega t + \varphi) = \sin(\omega t + \varphi) - 2\sin^3(\omega t + \varphi) + 2\sin(\omega t + \varphi)\cos^2(\omega t + \varphi)$. Với mỗi phương trình, chứng tỏ chuyển động là dao động điều hòa. Xác định biên độ, tần số, \& pha ban đầu.
\end{baitoan}

\begin{proof}[Hint]
	``Thực hiện biến đổi (nếu cần) để đưa phương trình ly độ về dạng tổng quát: $x = A\cos(\omega t + \varphi) + B$, $A > 0$. Từ đó kết luận về chuyển động \& so sánh để suy ra các đại lượng cần tìm.'' -- \cite[p. 48]{Giai_Toan_Trac_Nghiem_Vat_Ly_12_tap_1}
\end{proof}

\begin{baitoan}[\cite{Giai_Toan_Trac_Nghiem_Vat_Ly_12_tap_1}, \textbf{4.2}, p. 49]
	Vật nhỏ có khối lượng $m = 1$ \emph{kg} treo vào lò xo nhẹ có độ cứng $k = 400$ \emph{N\texttt{/}m}. Bỏ qua mọi lực cản. Lập phương trình dao động của vật trong mỗi trường hợp sau:
	\begin{enumerate*}
		\item[(a)] Dời vật tới ly độ $x_0 = 5$ \emph{cm} \& buông tự do. Chọn lúc buông làm gốc thời gian ($t = 0$).
		\item[(b)] Truyền cho vật đang ở vị trí cân bằng vận tốc $v_0 = 1$ \emph{m\texttt{/}s}. Chọn lúc truyền vận tốc làm gốc thời gian ($t = 0$).
		\item[(c)] Dời vật tới ly độ $x_0 = -4$ \emph{cm} \& truyền vận tốc $v_0 = -80$ \emph{cm\texttt{/}s} theo phương của trục lò xo. Chọn lúc truyền vận tốc làm gốc thời gian ($t = 0$).
	\end{enumerate*}
\end{baitoan}

\begin{baitoan}[1 mở rộng của \cite{Giai_Toan_Trac_Nghiem_Vat_Ly_12_tap_1}, \textbf{4.2}, p. 49]
	Vật nhỏ có khối lượng $m$ \emph{kg} treo vào lò xo nhẹ có độ cứng $k$ \emph{N\texttt{/}m}. Bỏ qua mọi lực cản. Lập phương trình dao động của vật trong mỗi trường hợp sau:
	\begin{enumerate*}
		\item[(a)] Dời vật tới ly độ $x_0$ \emph{m} \& buông tự do. Chọn lúc buông làm gốc thời gian ($t = 0$).
		\item[(b)] Truyền cho vật đang ở vị trí cân bằng vận tốc $v_0$ \emph{m\texttt{/}s}. Chọn lúc truyền vận tốc làm gốc thời gian ($t = 0$).
		\item[(c)] Dời vật tới ly độ $x_0$ \emph{m} \& truyền vận tốc $v_0$ \emph{m\texttt{/}s} theo phương của trục lò xo. Chọn lúc truyền vận tốc làm gốc thời gian ($t = 0$).
	\end{enumerate*}
\end{baitoan}

\begin{baitoan}
	1 chất điểm dao động điều hòa với phương trình $x_1 = A_1\cos\omega t$ có cơ năng $W_1$. Khi chất điểm này dao động với phương trình $x_2 = A_2\cos\left(\omega t + \frac{\pi}{3}\right)$ thì có cơ năng $W_2 = 4W_1$. Tính cơ năng của chất điểm dao động với phương trình $x_1 + x_2$.
\end{baitoan}
1 tổng quát của bài toán này:

\begin{baitoan}
	1 chất điểm dao động điều hòa với phương trình $x_1 = A_1\cos\omega t$ có cơ năng $W_1$. Khi chất điểm này dao động với phương trình $x_2 = A_2\cos\left(\omega t + \varphi\right)$ thì có cơ năng $W_2 = \alpha W_1$. Tính cơ năng của chất điểm dao động với phương trình $ax_1 + bx_2$.
\end{baitoan}

\begin{baitoan}
	Cho con lắc lò xo dao động trên trần thang máy, khi thang máy đứng yên thì con lắc dao động với chu kỳ $T = 0.4$s \& biên độ $A = 5$cm. Khi con lắc qua vị trí lò xo không biến dạng theo chiều từ trên xuống thì cho thang máy chuyển động nhanh dần đều lên với gia tốc $a = 5{\rm m\texttt{/}s^2}$. Tìm biên độ sau đó của con lắc.
	\begin{enumerate*}
		\item[{\rm\sf A.}] $5$cm;
		\item[{\rm\sf B.}] $5\sqrt{3}$cm; 
		\item[{\rm\sf C.}] $3\sqrt{5}$cm;
		\item[{\rm\sf D.}] $7$cm.
	\end{enumerate*}
\end{baitoan}
1 tổng quát của bài toán này:
\begin{baitoan}
	Cho con lắc lò xo dao động trên trần thang máy, khi thang máy đứng yên thì con lắc dao động với chu kỳ $T$s \& biên độ $A$cm. Khi con lắc qua vị trí lò xo không biến dạng theo chiều từ trên xuống thì cho thang máy chuyển động nhanh dần đều lên với gia tốc $a{\rm m\texttt{/}s^2}$. Tìm biên độ sau đó của con lắc. 
\end{baitoan}

\begin{baitoan}
	1 vật thực hiện $3$ dao động điều hòa có phương trình $x_1 = 10\sin(100\pi t + \varphi)$cm, $x_2 = 5\cos(100\pi t + \varphi)$cm, \& $x_3 = A\cos(100\pi t + \varphi)$cm. Biết $x_1^2 + x_2^2 + x_3^3 = 100$. Tìm $A$?
\end{baitoan}

\begin{baitoan}
	1 con lắc đơn có khối lượng của quả cầu $m = 0.2$kg, chiều dài của dây treo $l = 0.4$m, treo vào $1$ điểm cố định tại nơi có gia tốc trọng trường $g = 10{\rm m\texttt{/}s^2}$. Kéo vật khỏi vị trí cân bằng sao cho dây treo hợp với phương thẳng đứng $1$ góc $0.1$rad, rồi truyền cho vật $1$ vận tốc $0.15{\rm m\texttt{/}s}$ theo phương vuông góc với dây treo về vị trí cân bằng. Sau khi vật được truyền vận tốc xem như con lắc dao động điều hòa. Lực căng của dây treo khi vật nặng qua vị trí $s = \frac{S_0}{2}$, $S_0$ là biên độ dài.
	\begin{enumerate*}
		\item[{\rm\sf A.}] $1.01$N;
		\item[{\rm\sf B.}] $2.02$N; 
		\item[{\rm\sf C.}] $3.03$N;
		\item[{\rm\sf D.}] $4.04$N.
	\end{enumerate*}
\end{baitoan}

\begin{baitoan}
	1 vật thực hiện đồng thời $3$ giao động điều hòa cùng tần số $x_1,x_2,x_3$. Với $x_{12} = x_1 + x_2$, $x_{23} = x_2 + x_3$, $x_{13} = x_1 + x_3$, $x = x_1 + x_2 + x_3$. Biết $x_{12} = 6\cos\left(\pi t + \frac{\pi}{6}\right)$, $x_{23} = 6\cos\left(\pi t + \frac{2\pi}{3}\right)$, $x_{13} = 6\sqrt{2}\cos\left(\pi t + \frac{5\pi}{12}\right)$. Tìm $x$ biết $x^2 = x_1^2 + x_3^2$.
\end{baitoan}

\begin{baitoan}
	1 con lắc lò xo nằm ngang có độ cứng là $k$, vật nối vào lò xo có khối lượng $m = 0.1$kg kích thích để con lắc dao động điều hòa với $W = 0.02$J, khoảng thời gian ngắn nhất vật đi giữa $2$ vị trí có cùng tốc độ $v_0 = 10\pi{\rm cm\texttt{/}s} < v_{\max}$ là $\frac{1}{6}$s. Gọi $Q$ là điểm cố định của lò xo, khoảng thời gian ngắn nhất giữa $2$ lần liên tiếp $Q$ chịu lực tác dụng lúc kéo lò xo có độ lớn $0.2$N là:
	\begin{enumerate*}
		\item[{\rm\sf A.}] $0.5$s;
		\item[{\rm\sf B.}] $\frac{1}{6}$s; 
		\item[{\rm\sf C.}] $0.25$s;
		\item[{\rm\sf D.}] $\frac{1}{3}$s.
	\end{enumerate*}
\end{baitoan}

\begin{baitoan}
	1 vật có khối lượng $m_1 = 1.25$kg mắc vào lò xo nhẹ có độ cứng $k = 200$\emph{N\texttt{/}m}, đầu kia của lò xo gắn chặt vào tường. Vật \& lò xo đặt trên mặt phẳng nằm ngang có ma sát không đáng kể. Đặt vật thứ $2$ có khối lượng $m_2 = 3.75$kg sát với vật thứ nhất rồi đẩy chậm cả $2$ vật cho lò xo nén lại $8$cm. Khi thả nhẹ chúng ra, lò xo đẩy $2$ vật chuyển động về $1$ phía, Lấy $\pi^2 = 10$, khi lò xo giãn cực đại lần đầu tiên thì $2$ vật cách xa nhau $1$ đoạn là:
	\begin{enumerate*}
		\item[{\rm\sf A.}] $4\pi - 8$cm;
		\item[{\rm\sf B.}] $16$cm;
		\item[{\rm\sf C.}] $2\pi - 4$cm;
		\item[{\rm\sf D.}] $4\pi - 4$cm.
	\end{enumerate*}
\end{baitoan}

%------------------------------------------------------------------------------%






%------------------------------------------------------------------------------%

\subsection{Con Lắc Đơn. Con Lắc Vật Lý}

%------------------------------------------------------------------------------%

\subsection{Năng Lượng trong Dao Động Điều Hòa}

%------------------------------------------------------------------------------%

\subsection{Bài Tập về Dao Động Điều Hòa}

%------------------------------------------------------------------------------%

\subsection{Dao Động Tắt Dần \& Dao Động Duy Trì}

%------------------------------------------------------------------------------%

\subsection{Dao Động Cưỡng Bức. Cộng Hưởng}

%------------------------------------------------------------------------------%

\subsection{Tổng Hợp Dao Động}

%------------------------------------------------------------------------------%

\subsection{Thực Hành: Xác Định Chu Kỳ Dao Động của Con Lắc Đơn hoặc Con Lắc Lò Xo \& Gia Tốc Trọng Trường}

%------------------------------------------------------------------------------%

\subsection{Tóm Tắt Chương II}

%------------------------------------------------------------------------------%

\section{Sóng Cơ}

\subsection{Sóng Cơ. Phương Trình Sóng}

%------------------------------------------------------------------------------%

\subsection{Phản Xạ Sóng. Sóng Dừng}

%------------------------------------------------------------------------------%

\subsection{Giao Thoa Sóng}

%------------------------------------------------------------------------------%

\subsection{Sóng Âm. Nguồn Nhạc Âm}

%------------------------------------------------------------------------------%

\subsection{Hiệu Ứng Doppler}

%------------------------------------------------------------------------------%

\subsection{Bài Tập về Sóng Cơ}

%------------------------------------------------------------------------------%

\subsection{Thực Hành: Xác Định Tốc Độ Truyền Âm}

%------------------------------------------------------------------------------%

\subsubsection{Tóm Tắt Chương III}

%------------------------------------------------------------------------------%

\section{Dao Động \& Sóng Điện Từ}

\subsection{Dao Động Điện Từ}

%------------------------------------------------------------------------------%

\subsection{Bài Tập về Dao Động Điện Từ}

%------------------------------------------------------------------------------%

\subsection{Điện Từ Trường}

%------------------------------------------------------------------------------%

\subsection{Sóng Điện Từ}

%------------------------------------------------------------------------------%

\subsection{Truyền Thông bằng Sóng Điện Từ}

%------------------------------------------------------------------------------%

\subsection{Bộ Dao Động Thạch Anh (Quartz)}

%------------------------------------------------------------------------------%

\subsection{Tóm Tắt chương IV}

%------------------------------------------------------------------------------%

\section{Dòng Điện Xoay Chiều}

\subsection{Dòng Điện Xoay Chiều. Mạch Điện Xoay Chiều Chỉ Có Điện Trở Thuần}

%------------------------------------------------------------------------------%

\subsection{Mạch Điện Xoay Chiều Chỉ Có Tụ Điện, Cuộn Cảm}

%------------------------------------------------------------------------------%

\subsection{Mạch Có $R,L,C$ Mắc Nối Tiếp. Cộng Hưởng Điện}

%------------------------------------------------------------------------------%

\subsection{Công Suất của Dòng Điện Xoay Chiều. Hệ Số Công Suất}

%------------------------------------------------------------------------------%

\subsection{Máy Phát Điện Xoay Chiều}

%------------------------------------------------------------------------------%

\subsection{Động Cơ Không Đồng Bộ 3 Pha}

%------------------------------------------------------------------------------%

\subsection{Máy Biến Áp. Truyền Tải Điện Năng}

%------------------------------------------------------------------------------%

\subsection{Bài Tập về Dòng Điện Xoay Chiều}

%------------------------------------------------------------------------------%

\subsection{Sản Xuất Điện}

%------------------------------------------------------------------------------%

\subsection{Thực Hành: Khảo Sát Đoạn Mạch Điện Xoay Chiều có $R,L,C$ Mắc Nối Tiếp}

%------------------------------------------------------------------------------%

\subsection{Tóm Tắt Chương V}

%------------------------------------------------------------------------------%

\section{Sóng Ánh Sáng}

\subsection{Tán Sắc Ánh Sáng}

%------------------------------------------------------------------------------%

\subsection{Nhiễu Xạ Ánh Sáng. Giao Thoa Ánh Sáng}

%------------------------------------------------------------------------------%

\subsection{Khoảng Vân. Bước Sóng \& Màu Sắc Ánh Sáng}

%------------------------------------------------------------------------------%

\subsection{Bài Tập về Giao Thoa Ánh Sáng}

%------------------------------------------------------------------------------%

\subsection{Máy Quang Phổ. Các Loại Quang Phổ}

%------------------------------------------------------------------------------%

\subsection{Tia Hồng Ngoại. Tia Tử Ngoại}

%------------------------------------------------------------------------------%

\subsection{Tia X. Thuyết Điện Từ Ánh Sáng. Thang Sóng Điện Từ}

%------------------------------------------------------------------------------%

\subsection{Cầu Vồng}

%------------------------------------------------------------------------------%

\subsection{Thực Hành: Xác Định Bước Sóng Ánh Sáng}

%------------------------------------------------------------------------------%

\subsection{Tóm Tắt Chương VI}

%------------------------------------------------------------------------------%

\section{Lượng Tử Ánh Sáng}

\subsection{Hiện Tượng Quang Điện Ngoài. Các Định Luật Quang Điện}

%------------------------------------------------------------------------------%

\subsection{Thuyết Lượng Tử Ánh Sáng. Lưỡng Tính Sóng - Hạt của Ánh Sáng}

%------------------------------------------------------------------------------%

\subsection{Bài Tập về Hiện Tượng Quang Điện}

%------------------------------------------------------------------------------%

\subsection{Hiện Tượng Quang Điện Trong. Quang Điện Trở \& Pin Quang Điện}

%------------------------------------------------------------------------------%

\subsection{Mẫu Nguyên Tử Bo \& Quang Phổ Vạch của Nguyên Tử Hydro}

%------------------------------------------------------------------------------%

\subsection{Hấp Thụ \& Phản Xạ Lọc Lựa Ánh Sáng. Màu Sắc Các Vật}

%------------------------------------------------------------------------------%

\subsection{Sự Phát Quang. Sơ Lược về Laze}

%------------------------------------------------------------------------------%

\subsection{Cấu Tạo \& Hoạt Động của Laze}

%------------------------------------------------------------------------------%

\subsection{Tóm Tắt Chương VII}

%------------------------------------------------------------------------------%

\section{Sơ Lược về Thuyết Tương Đối Hẹp}

\subsection{Thuyết Tương Đối Hẹp}

%------------------------------------------------------------------------------%

\subsection{Hệ Thức Einstein Giữa Khối Lượng \& Năng Lượng}

%------------------------------------------------------------------------------%

\subsection{Tóm Tắt Chương VIII}

%------------------------------------------------------------------------------%

\section{Hạt Nhân Nguyên Tử}

\subsection{Cấu Tạo của Hạt Nhân Nguyên tử. Độ Hụt Khối}

%------------------------------------------------------------------------------%

\subsection{Phóng Xạ}

%------------------------------------------------------------------------------%

\subsection{Phản Ứng Hạt Nhân}

%------------------------------------------------------------------------------%

\subsection{Bài Tập về Phóng Xạ \& Phản Ứng Hạt Nhân}

%------------------------------------------------------------------------------%

\subsection{Phản Ứng Phân Hạch}

%------------------------------------------------------------------------------%

\subsection{Phản Ứng Nhiệt Hạch}

%------------------------------------------------------------------------------%

\subsection{Tóm Tắt Chương IX}

%------------------------------------------------------------------------------%

\section{Từ Vi Mô đến Vĩ Mô}

\subsection{Các Hạt Sơ Cấp}

%------------------------------------------------------------------------------%

\subsection{Mặt Trời. Hệ Mặt Trời}

%------------------------------------------------------------------------------%

\subsection{Sao. Thiên Hà}

%------------------------------------------------------------------------------%

\subsection{Thuyết Big Bang}

%------------------------------------------------------------------------------%

\subsection{Liệu Có -- Hoặc Đã Từng Có -- Sự Sóng trên Hỏa Tinh hay không?}

\subsection{Tóm Tắt Chương X}

%------------------------------------------------------------------------------%

\newpage
\printbibliography[heading=bibintoc]
	
\end{document}