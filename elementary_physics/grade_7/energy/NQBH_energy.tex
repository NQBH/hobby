\documentclass{article}
\usepackage[backend=biber,natbib=true,style=authoryear,maxbibnames=10]{biblatex}
\addbibresource{/home/nqbh/reference/bib.bib}
\usepackage[utf8]{vietnam}
\usepackage{tocloft}
\renewcommand{\cftsecleader}{\cftdotfill{\cftdotsep}}
\usepackage[colorlinks=true,linkcolor=blue,urlcolor=red,citecolor=magenta]{hyperref}
\usepackage{amsmath,amssymb,amsthm,float,graphicx,mathtools,tikz,tipa}
\usepackage[version=4]{mhchem}
\allowdisplaybreaks
\newtheorem{assumption}{Assumption}
\newtheorem{baitoan}{Bài toán}
\newtheorem{cauhoi}{Câu hỏi}
\newtheorem{conjecture}{Conjecture}
\newtheorem{corollary}{Corollary}
\newtheorem{dangtoan}{Dạng toán}
\newtheorem{definition}{Definition}
\newtheorem{dinhly}{Định lý}
\newtheorem{dinhnghia}{Định nghĩa}
\newtheorem{example}{Example}
\newtheorem{ghichu}{Ghi chú}
\newtheorem{hequa}{Hệ quả}
\newtheorem{hypothesis}{Hypothesis}
\newtheorem{lemma}{Lemma}
\newtheorem{luuy}{Lưu ý}
\newtheorem{nhanxet}{Nhận xét}
\newtheorem{notation}{Notation}
\newtheorem{note}{Note}
\newtheorem{principle}{Principle}
\newtheorem{problem}{Problem}
\newtheorem{proposition}{Proposition}
\newtheorem{question}{Question}
\newtheorem{remark}{Remark}
\newtheorem{theorem}{Theorem}
\newtheorem{vidu}{Ví dụ}
\usepackage[left=1cm,right=1cm,top=5mm,bottom=5mm,footskip=4mm]{geometry}

\title{Energy \& Transformation -- Năng Lượng \& Sự Biến Đổi}
\author{Nguyễn Quản Bá Hồng\footnote{Independent Researcher, Ben Tre City, Vietnam\\e-mail: \texttt{nguyenquanbahong@gmail.com}; website: \url{https://nqbh.github.io}.}}
\date{\today}

\begin{document}
\maketitle
\begin{abstract}
	\textsc{[en]} This text is a collection of problems, from easy to advanced, about energy. This text is also a supplementary material for my lecture note on Elementary Physics, which is stored \& downloadable at the following link: \href{https://github.com/NQBH/hobby/blob/master/elementary_physics/grade_8/NQBH_elementary_physics_grade_8.pdf}{GitHub\texttt{/}NQBH\texttt{/}hobby\texttt{/}elementary physics\texttt{/}grade 8\texttt{/}lecture}\footnote{\textsc{url}: \url{https://github.com/NQBH/hobby/blob/master/elementary_physics/grade_8/NQBH_elementary_physics_grade_8.pdf}.}. The latest version of this text has been stored \& downloadable at the following link:\\\href{https://github.com/NQBH/hobby/blob/master/elementary_physics/energy/NQBH_energy.pdf}{GitHub\texttt{/}NQBH\texttt{/}hobby\texttt{/}elementary physics\texttt{/}grade 8\texttt{/}energy}\footnote{\textsc{url}: \url{https://github.com/NQBH/hobby/blob/master/elementary_physics/energy/NQBH_energy.pdf}.}.
	\vspace{2mm}
	
	\textsc{[vi]} Tài liệu này là 1 bộ sưu tập các bài tập chọn lọc từ cơ bản đến nâng cao về nguyên tử, nguyên tố hóa học, \& hợp chất hóa học. Tài liệu này là phần bài tập bổ sung cho tài liệu chính -- bài giảng \href{https://github.com/NQBH/hobby/blob/master/elementary_physics/grade_8/NQBH_elementary_physics_grade_8.pdf}{GitHub\texttt{/}NQBH\texttt{/}hobby\texttt{/}elementary physics\texttt{/}grade 8\texttt{/}lecture} của tác giả viết cho Vật Lý Sơ Cấp. Phiên bản mới nhất của tài liệu này được lưu trữ \& có thể tải xuống ở link sau: \href{https://github.com/NQBH/hobby/blob/master/elementary_physics/grade_8/real/NQBH_real.pdf}{GitHub\texttt{/}NQBH\texttt{/}hobby\texttt{/}elementary physics\texttt{/}grade 8\texttt{/}energy}.
\end{abstract}
\setcounter{secnumdepth}{4}
\setcounter{tocdepth}{3}
\tableofcontents
\newpage

%------------------------------------------------------------------------------%

\begin{center}\LARGE
	Chủ đề: Speed -- Tốc Độ.
\end{center}

\section{Tốc Độ của Chuyển Động}
\textsf{\textbf{Nội dung.} Ý nghĩa vật lý của tốc độ, tốc độ qua quãng đường vật đi được trong khoảng thời gian tương ứng, tốc độ bằng quãng đường vật đi chia thời gian đi quãng đường đó, 1 số đơn vị đo tốc độ thường dùng, cách đo tốc độ bằng đồng hồ bấm giây \& cổng quang điện trong dụng cụ thực hành ở nhà trường, thiết bị ``bắn tốc độ'' trong kiểm tra tốc độ của các phương tiện giao thông.}

\begin{baitoan}[\cite{SGK_KHTN_7_Canh_Dieu}, p. 47]
	Trong 1 buổi tập luyện, vận động viên A bơi được quãng đường \emph{48 m} trong \emph{32 s}, vận động viên B bơi được quãng đường \emph{46.5 m} trong \emph{30 s}. Trong 2 vận động viên này, vận động viên nào bơi nhanh hơn?
\end{baitoan}

\subsection{Khái niệm tốc độ}
Để so sánh vật này chuyển động nhanh hay chậm hơn so với vật kia, ta cần so sánh độ dài quãng đường mà mỗi vật đi được trong cùng 1 khoảng thời gian xác định.

\begin{vidu}
	Nếu trong $1$ giờ, ô tô đi được \emph{50 km}, xe máy đi được \emph{30 km}, thì ô tô đi nhanh hơn xe máy. Trong trường hợp này, ta nói ô tô \emph{có tốc độ lớn hơn} xe máy.
\end{vidu}
Vật nào có tốc độ lớn hơn, vật đó chuyển động nhanh hơn. Tốc độ được tính bằng quãng đường vật đi trong 1 khoảng thời gian xác định. Khoảng thời gian xác định có thể là 1 giây (second, abbr.: s), 1 phút (minute, abbr.: min), 1 giờ (hour, abbr.: h), 1 ngày (day, abbr.: d), $\ldots$

Nếu biết quãng đường vật đi \& thời gian vật đi hết quãng đường đó thì tốc độ được xác định như sau: $\mbox{tốc độ} = \frac{\mbox{quãng đường}}{\mbox{thời gian}}$. Ký hiệu quãng đường vật đi là $s$, thời gian vật đi hết quãng đường đó là $t$ (time, abbr.: $t$), tốc độ (speed, trong khi vận tốc là velocity, abbr.: v) của vật được tính là: $v = \frac{s}{t}$.

\begin{baitoan}[\cite{SGK_KHTN_7_Canh_Dieu}, 1, p. 47]
	Từ kinh nghiệm thực tế, thảo luận về việc làm thế nào để biết vật chuyển động nhanh hay chậm.
\end{baitoan}

\begin{baitoan}[\cite{SGK_KHTN_7_Canh_Dieu}, 1, p. 47]
	Bảng dưới đây cho biết quãng đường \& thời gian đi hết quãng đường đó của 4 xe A, B, C, D. Xe nào đi nhanh nhất, xe nào đi chậm nhất? Sắp theo theo thứ tự nhanh dần.
	\begin{table}[H]
		\centering
		\begin{tabular}{|c|c|c|}
			\hline
			Xe & Quãng đường (km) & Thời gian (phút) \\
			\hline
			A & 80 & 50 \\
			\hline
			B & 72 & 50 \\
			\hline
			C & 80 & 40 \\
			\hline
			D & 99 & 45 \\
			\hline
		\end{tabular}
	\end{table}
\end{baitoan}

\subsection{Đơn vị đo tốc độ}
Nếu đơn vị đo quãng đường là mét, đơn vị đo thời gian là giây thì đơn vị đo tốc độ là mét\texttt{/}giây, ký hiệu: m\texttt{/}s.

\begin{vidu}
	Trong $1$ giây xe đi được quãng đường là $10$ mét, tốc độ của xe là \emph{10 m\texttt{/}s}.
\end{vidu}
Có nhiều đơn vị đo khác nhau của tốc độ, tùy từng trường hợp mà ta chọn đơn vị đo thích hợp, e.g., khi đo tốc độ của con sên, dùng đơn vị cm\texttt{/}s sẽ thuận tiện hơn đơn vị m\texttt{/}s. Ngoài m\texttt{/}s, đơn vị đo tốc độ của chuyển động thường dùng là kilomet\texttt{/}giờ, ký hiệu km\texttt{/}h.

\begin{baitoan}[\cite{SGK_KHTN_7_Canh_Dieu}, 2, p. 48]
	Kể tên các đơn vị đo tốc độ.
\end{baitoan}

\begin{baitoan}[\cite{SGK_KHTN_7_Canh_Dieu}, p. 48]
	1 vận động viên chạy trên quãng đường dài \emph{1 km}. Người đó đi \& về hết \emph{400 s}. Tính tốc độ của vận động viên.
\end{baitoan}

\begin{proof}[Giải]
	Quãng đường vận động viên chạy là: $s = s_{\footnotesize\mbox{lượt đi}} + s_{\footnotesize\mbox{lượt về}} về = 1$ km $+ 1$ km $= 2$ km $= 2000$ m. Ta có $v = \frac{s}{t} = \frac{\rm2000m}{\rm400s} = 5$m\texttt{/}s. Vậy tốc độ của vận động viên là $5$ m\texttt{/}s.
\end{proof}

\begin{baitoan}[\cite{SGK_KHTN_7_Canh_Dieu}, 2, p. 48]
	1 ô tô đi được bao xa trong thời gian \emph{0.75 h} với tốc độ \emph{88 km\texttt{/}h}?
\end{baitoan}

\begin{baitoan}[\cite{SGK_KHTN_7_Canh_Dieu}, 3, p. 48]
	Bảng dưới đây cho biết thời gian đi \emph{1000 m} của 1 số vật chuyển động. Tính tốc độ của các vật đó.
	\begin{table}[H]
		\centering
		\begin{tabular}{|c|c|}
			\hline
			Vật chuyển động & Thời gian (s) \\
			\hline
			Xe đua & 10 \\
			\hline
			Máy bay chở khách & 4 \\
			\hline
			Tên lửa bay vào vũ trụ & 0.1 \\
			\hline
		\end{tabular}
	\end{table}
\end{baitoan}

\subsection{Cách đo tốc độ bằng dụng cụ thực hành ở nhà trường}
Muốn đo được tốc độ của 1 vật đi trên 1 quãng đường nào đó, ta phải đo được chiều dài quãng đường \& thời gian vật đi hết quãng đường đó.

Ở nhà trường, có thể đo chiều dài quãng đường bằng các dụng cụ đo chiều dài như thước mét, thước dây, $\ldots$ Thời gian vật đi có thể được đo bằng đồng hồ bấm giây hoặc đồng hồ đo thời gian hiện số \& cổng quang điện.

\begin{baitoan}[\cite{SGK_KHTN_7_Canh_Dieu}, 3, p. 48]
	Có những cách nào để đo tốc độ của 1 vật trong phòng thí nghiệm?
\end{baitoan}

\subsubsection{Cách đo tốc độ bằng đồng hồ bấm giây}
Có thể dùng đồng hồ bấm giây để đo khoảng thời gian vật đi trên quãng đường AB. Bấm đồng hồ đo khi vật ở A \& bấm dừng đồng hồ khi vật ở B. Đồng hồ bấm giây sẽ cho biết khoảng thời gian vật đi từ A đến B.

Đo quãng đường từ A đến B bằng dụng cụ đo chiều dài. Lấy chiều dài quãng đường AB chia cho khoảng thời gian đo bởi đồng hồ bấm giây. Kết quả thu được chính là tốc độ của vật.

\begin{baitoan}[\cite{SGK_KHTN_7_Canh_Dieu}, 4, p. 49]
	2 người cùng đo thời gian của 1 chuyển động bằng đồng hồ bấm giây nhưng lại cho kết qảu lệch nhau. Giải thích. Từ đó thảo luận về ưu điểm \& hạn chế của phương pháp đo tốc độ dùng đồng hồ bấm giây.
\end{baitoan}

\subsubsection{Cách đo tốc độ bằng đồng hồ đo thời gian hiện số \& cổng quang điện}
Để đo tốc độ của 1 xe đi từ vị trí A đến vị trí B, ta tiến hành:
\begin{itemize}
	\item Cố định cổng quang điện 1 ở vị trí A \& cổng quang điện 2 ở vị trí B. Khoảng cách giữa A \& B được đọc ở thước đo gắn với giá đỡ. Thời gian xe đi từ A đến B được đọc ở đồng hồ đo thời gian hiện số.
	\item Tốc độ của xe được tính bằng tỷ số khoảng cách giữa 2 cổng quang điện \& thời gian xe đi từ A đến B.
\end{itemize}

\begin{baitoan}[\cite{SGK_KHTN_7_Canh_Dieu}, p. 49]
	Đánh giá ưu điểm của phương pháp đo tốc độ bằng đồng hồ đo thời gian hiện số so với phương pháp đo tốc độ bằng đồng hồ bấm giây.
\end{baitoan}

\subsection{Cách đo tốc độ bằng thiết bị ``bắn tốc độ''}
Để kiểm tra tốc độ của các phương tiện giao thông đường bộ, người ta dùng thiết bị ``bắn tốc độ'' (súng ``bắn tốc độ''). Thiết bị này đo tốc độ của xe đang chuyển động như sau:

Súng phát tia sáng tới xe, bộ phận xử lý tín hiệu của súng xác định thời gian từ lúc phát tia sáng tới lúc nhận lại tia phản xạ từ xe về súng. Lấy khoảng thời gian đó nhân với tốc độ ánh sáng rồi chia cho 2 để tính ra khoảng cách từ xe tới súng.

Súng tiếp tục phát tia sáng lần thứ 2, tia sáng tới xe \& trở lại bộ phận thu giống như lần trước. Khoảng cách giữa xe \& súng được tính tương tự như trên.

Hiệu khoảng cách từ xe tới súng giữa 2 lần bắn chính là quãng đường xe đi giữa 2 lần bắn.

Bộ phận xử lý của súng tính ra tốc độ của xe bằng cách chia quãng đường này cho khoảng thời gian giữa 2 lần bắn (được lập trình sẵn trong súng).

\noindent\fbox{%
	\parbox{\textwidth}{%
		\noindent\textsf{\textbf{Kiến thức cốt lõi.}} \fbox{\bf 1} Tốc độ cho ta biết 1 vật chuyển động nhanh hay chậm. \fbox{\bf 2} Tốc độ được tính bằng thương số giữa quãng đường vật đi \& thời gian đi quãng đường đó. \fbox{\bf 3} Đơn vị đo tốc độ thường là m\texttt{/}s \& km\texttt{/}h. \fbox{\bf 4} Trong phòng thí nghiệm, để đo tốc độ có thể dùng đồng hồ bấm giây hoặc cổng quang điện kết hợp với đồng hồ đo thời gian hiện số. \fbox{\bf 5} Thiết bị ``bắn tốc độ'' thường được dùng để xác định tốc độ của các phương tiện giao thông đường bộ.
	}%
}

\begin{baitoan}[\cite{SGK_Toan_6_Canh_Dieu_tap_2}, 2, p. 63]
	Trong không khí, ánh sáng chuyển động với vận tốc khoảng \emph{300000 km\texttt{/}s}, còn âm thanh lan truyền với vận tốc khoảng \emph{343.2 m\texttt{/}s}. Tính tỷ số của vận tốc ánh sáng \& vận tốc âm thanh.
\end{baitoan}

%------------------------------------------------------------------------------%

\section{Đồ Thị Quãng Đường--Thời Gian}

%------------------------------------------------------------------------------%

\begin{center}\LARGE
	Sound -- Âm Thanh
\end{center}

\section{Sự Truyền Âm}

%------------------------------------------------------------------------------%

\section{Biên Độ, Tần Số, Độ To, \& Độ Cao của Âm}

%------------------------------------------------------------------------------%

\section{Phản Xạ Âm}

%------------------------------------------------------------------------------%

\begin{center}\LARGE
	Light -- Ánh Sáng
\end{center}

\section{Ánh Sáng, Tia Sáng}

%------------------------------------------------------------------------------%

\section{Sự Phản Xạ Ánh Sáng}

%------------------------------------------------------------------------------%

\begin{center}\LARGE
	Magnetism -- Tính Chất Từ của Chất
\end{center}

\section{Nam Châm}

%------------------------------------------------------------------------------%

\section{Từ Trường}

%------------------------------------------------------------------------------%

\section{Từ Trường Trái Đất}

%------------------------------------------------------------------------------%

\printbibliography[heading=bibintoc]
	
\end{document}