\documentclass{article}
\usepackage[backend=biber,natbib=true,style=authoryear]{biblatex}
\addbibresource{/home/hong/1_NQBH/reference/bib.bib}
\usepackage[utf8]{vietnam}
\usepackage{tocloft}
\renewcommand{\cftsecleader}{\cftdotfill{\cftdotsep}}
\usepackage[colorlinks=true,linkcolor=blue,urlcolor=red,citecolor=magenta]{hyperref}
\usepackage{amsmath,amssymb,amsthm,mathtools,float,graphicx,algpseudocode,algorithm,tcolorbox,tikz,tkz-tab,subcaption}
\DeclareMathOperator{\arccot}{arccot}
\usepackage[inline]{enumitem}
\allowdisplaybreaks
\numberwithin{equation}{section}
\newtheorem{assumption}{Assumption}[section]
\newtheorem{nhanxet}{Nhận xét}[section]
\newtheorem{conjecture}{Conjecture}[section]
\newtheorem{corollary}{Corollary}[section]
\newtheorem{hequa}{Hệ quả}[section]
\newtheorem{definition}{Definition}[section]
\newtheorem{dinhnghia}{Định nghĩa}[section]
\newtheorem{example}{Example}[section]
\newtheorem{vidu}{Ví dụ}[section]
\newtheorem{lemma}{Lemma}[section]
\newtheorem{notation}{Notation}[section]
\newtheorem{principle}{Principle}[section]
\newtheorem{problem}{Problem}[section]
\newtheorem{baitoan}{Bài toán}[section]
\newtheorem{proposition}{Proposition}[section]
\newtheorem{menhde}{Mệnh đề}[section]
\newtheorem{question}{Question}[section]
\newtheorem{cauhoi}{Câu hỏi}[section]
\newtheorem{quytac}{Quy tắc}
\newtheorem{remark}{Remark}[section]
\newtheorem{luuy}{Lưu ý}[section]
\newtheorem{theorem}{Theorem}[section]
\newtheorem{tiende}{Tiên đề}[section]
\newtheorem{dinhly}{Định lý}[section]
\usepackage[left=0.5in,right=0.5in,top=1.5cm,bottom=1.5cm]{geometry}
\usepackage{fancyhdr}
\pagestyle{fancy}
\fancyhf{}
\lhead{\small Subsect.~\thesubsection}
\rhead{\small \nouppercase{\leftmark}}
\renewcommand{\sectionmark}[1]{\markboth{#1}{}}
\cfoot{\thepage}
\def\labelitemii{$\circ$}

\title{Some Topics in Elementary Physics\texttt{/}Grade 8}
\author{Nguyễn Quản Bá Hồng\footnote{Independent Researcher, Ben Tre City, Vietnam\\e-mail: \texttt{nguyenquanbahong@gmail.com}; website: \url{https://nqbh.github.io}.}}
\date{\today}

\begin{document}
\maketitle
\begin{abstract}
	
\end{abstract}
\setcounter{secnumdepth}{4}
\setcounter{tocdepth}{3}
\tableofcontents
\newpage

%------------------------------------------------------------------------------%

\section{Cơ Học -- Mechanics}
\textsf{\textbf{Nội dung.} Chuyển động, đứng yên; chuyển động đều\texttt{/}không đều; quan hệ của lực với vận tốc; quán tính; áp suất; sự khác nhau giữa áp suất gây ra bởi chất rắn, chất lỏng, \& áp suất khí quyển; lực đẩy Archimedes, hiện tượng vật nổi\texttt{/}chìm; công cơ học; công suất; cơ năng, động năng, thế năng; bảo toàn \& chuyển hóa cơ năng.}

\subsection{Chuyển Động Cơ Học}

\subsubsection{Cách để biết 1 vật chuyển động\texttt{/}đứng yên}
``Trong Vật lý học, để nhận biết 1 vật chuyển động hay đứng yên người ta dựa vào vị trí của vật đó so với vật khác được chọn làm mốc (\textit{vật mốc}). Có thể chọn bất kỳ 1 vật nào làm vật mốc. Thường người ta chọn Trái Đất \& những vật gắn với Trái Đất như nhà cửa, cây cối, cột cây số, $\ldots$ làm vật mốc.'' ``Nếu không nói tới vật mốc thì hiểu ngầm vật mốc là Trái Đất hoặc những vật gắn với Trái Đất.

\begin{dinhnghia}[Chuyển động cơ học]
	Khi vị trí của vật so với vật mốc thay đổi theo thời gian thì vật chuyển động so với vật mốc. Chuyển động này gọi là \emph{chuyển động cơ học} (gọi tắt là \emph{chuyển động}).'' -- \cite[p. 4]{SGK_Vat_Ly_8}
\end{dinhnghia}

\subsubsection{Tính tương đối của chuyển động \& đứng yên}
``1 vật được coi là chuyển động hay đứng yên phụ thuộc vào việc chọn vật làm mốc. \textit{Chuyển động hay đứng yên có tính tương đối}.'' -- \cite[p. 5]{SGK_Vat_Ly_8}

\subsubsection{1 số chuyển động thường gặp}

\begin{dinhnghia}[Quỹ đạo của chuyển động]
	 ``Đường mà vật chuyển động vạch ra gọi là \emph{quỹ đạo của chuyển động}.
\end{dinhnghia}
Tùy theo hình dạng của quỹ đạo, người ta phân biệt chuyển động thẳng \& chuyển động cong. Chuyển động tròn là 1 chuyển động cong đặc biệt.'' -- \cite[p. 6]{SGK_Vat_Ly_8}
\vspace{2mm}

\noindent\textbf{Tóm tắt kiến thức.}
\begin{enumerate*}
	\item[$\bullet$] ``Sự thay đổi vị trí của 1 vật theo thời gian so với vật khác gọi là \textit{chuyển động cơ học}.
	\item[$\bullet$] Chuyển động \& đứng yên có tính tương đối tùy thuộc vào vật được chọn làm vật mốc. Người ta thường chọn những vật gắn với Trái Đất làm vật mốc.
	\item[$\bullet$] Các dạng chuyển động cơ học thường gặp là chuyển động thẳng\texttt{/}cong.'' -- \cite[p. 7]{SGK_Vat_Ly_8}
\end{enumerate*}

``Vì đầu van xe đạp vừa chuyển động tròn xung quanh trục bánh xe, vừa cùng với xe đạp chuyển động thẳng trên đường. Do đó, đối với người đứng bên đường thì chuyển động của đầu van xe đạp khá phức tạp \& có dạng như Fig. \ref{fig:quy_dao_chuyen_dong_dau_van_xe_dap}. Như vậy, việc chọn vật nào làm mốc không những quyết định nhiều tính chất khác nữa của chuyển động.'' -- \cite[p. 7]{SGK_Vat_Ly_8}

\begin{figure}[h]
	\centering
	\includegraphics[scale=0.15]{quy_dao_chuyen_dong_dau_van_xe_dap}
	\caption{Quỹ đạo chuyển động của đầu van xe đạp, \cite[Hình 1.5, p. 7]{SGK_Vat_Ly_8}.}
	\label{fig:quy_dao_chuyen_dong_dau_van_xe_dap}
\end{figure}

%------------------------------------------------------------------------------%

\subsection{Vận Tốc -- Velocity}

\subsubsection{Khái niệm vận tốc}

\begin{dinhnghia}[Vận tốc]
	``Quãng đường chạy được trong $1$s được gọi là \emph{vận tốc}.'' -- \cite[p. 8]{SGK_Vat_Ly_8}
\end{dinhnghia}

\subsubsection{Công thức tính vận tốc}
``$v = \frac{s}{t}$, trong đó: $v$: vận tốc, $s$: quãng đường đi được, $t$: thời gian để đi hết quãng đường đó.'' -- \cite[p. 9]{SGK_Vat_Ly_8}

\subsubsection{Đơn vị vận tốc}
``Đơn vị vận tốc phụ thuộc vào đơn vị độ dài \& đơn vị thời gian. Đơn vị hợp pháp của vận tốc là m\texttt{/}s \& km\texttt{/}h. Độ lớn của vận tốc được đo bằng dụng cụ gọi là \textit{tốc kế} (còn gọi là \textit{đồng hồ vận tốc}).'' -- \cite[p. 9]{SGK_Vat_Ly_8}
\vspace{2mm}

\noindent\textbf{Tóm tắt kiến thức.}
\begin{enumerate*}
	\item[$\bullet$] ``Độ lớn của vận tốc cho biết mức độ nhanh hay chậm của chuyển động \& được xác định bằng độ dài quãng đường đi được trong 1 đơn vị thời gian.
	\item[$\bullet$] Công thức tính vận tốc: $v = \frac{s}{t}$, trong đó: $s$: độ dài quãng đường đi được, $t$: thời gian để đi hết quãng đường đó.
	\item[$\bullet$] Đơn vị vận tốc phụ thuộc vào đơn vị độ dài \& đơn vị thời gian. Đơn vị hợp pháp của vận tốc là m\texttt{/}s \& km\texttt{/}h.'' -- \cite[p. 10]{SGK_Vat_Ly_8}
\end{enumerate*}

``Trong hàng hải, người ta thường dùng ``nút'' làm đơn vị đo vận tốc. Nút là vận tốc của 1 chuyển động mỗi giờ đi được 1 hải lý. Biết độ dài của hải lý là $1.852$km ta dễ dàng tính được nút ra km\texttt{/}h \& m\texttt{/}s: 1 nút $= 1.852$km\texttt{/}h $= 0.514$m\texttt{/}s. Các tàu thủy có lắp cánh ngầm lướt trên sóng rất nhanh nhưng cũng không mấy tàu vượt qua được vận tốc $30$ nút.

Vận tốc ánh sáng là $300000$km\texttt{/}s. Trong vũ trụ, khoảng cách giữa các thiên thể rất lớn, vì vậy trong thiên văn người ta hay biểu thị những khoảng cách đó bằng ``năm ánh sáng''. Năm ánh sáng là quãng đường ánh sáng truyền đi trong thời gian 1 năm. 1 năm ánh sáng ứng với khoảng cách bằng: $3\cdot 10^5\cdot 365\cdot 24\cdot 3600 = 9.4608\cdot 10^{12}$km. Trong thiên văn, người ta lấy tròn 1 năm ánh sáng bằng $10^{16}$m ($10$ triệu tỷ mét). Khoảng cách từ Trái Đất tới ngôi sao gần nhất cũng lên tới $4.3$ năm ánh sáng.'' -- \cite[p. 10]{SGK_Vat_Ly_8}

%------------------------------------------------------------------------------%

\subsection{Chuyển Động Đều -- Chuyển Động Không Đều}

\subsubsection{Định nghĩa}

\begin{dinhnghia}[Chuyển động đều\texttt{/}không đều]
	``\emph{Chuyển động đều} là chuyển động mà vận tốc có độ lớn không thay đổi theo thời gian. \emph{Chuyển động không đều} là chuyển động mà vận tốc có độ lớn thay đổi theo thời gian.'' -- \cite[p. 11]{SGK_Vat_Ly_8}
\end{dinhnghia}

\subsubsection{Vận tốc trung bình của chuyển động không đều}
``Trên từng quãng đường, trung bình mỗi giây trục bánh xe lăn được bao nhiêu m thì ta nói \textit{vận tốc trung bình} của trục bánh xe trên mỗi quãng đường đó là bấy nhiêu m\texttt{/}s.'' -- \cite[p. 12]{SGK_Vat_Ly_8}
\vspace{2mm}

\noindent\textbf{Tóm tắt kiến thức.}
\begin{enumerate*}
	\item[$\bullet$] ``\textit{Chuyển động đều} là chuyển động mà vận tốc có độ lớn không thay đổi theo thời gian.
	\item[$\bullet$] \textit{Chuyển động không đều} là chuyển động mà vận tốc có độ lớn thay đổi theo thời gian.
	\item[$\bullet$] \textit{Vận tốc trung bình} của 1 chuyển động không đều trên 1 quãng đường được tính bằng công thức: $v_{\rm tb} = \frac{s}{t}$, trong đó: $s$: quãng đường đi được, $t$: thời gian để đi hết quãng đường đó.'' -- \cite[p. 13]{SGK_Vat_Ly_8}
\end{enumerate*}

``\textbf{1 số vận tốc trung bình.} Con sên: $0.0014$m\texttt{/}s ($0.005$km\texttt{/}h), con rùa: $0.055$m\texttt{/}s ($0.2$km\texttt{/}h), người đi bộ: $1.5$m\texttt{/}s ($5.4$km\texttt{/}h), người đi xe đạp: $4$m\texttt{/}s ($14.4$km\texttt{/}h), tàu hỏa: $15$m\texttt{/}s ($54$km\texttt{/}h), ôtô du lịch: $15$m\texttt{/}s ($54$km\texttt{/}h), máy bay dân dụng phản lực: $200$m\texttt{/}s ($720$km\texttt{/}h), vận tốc âm thanh trong không khí: $340$m\texttt{/}s, vận tốc ánh sáng trong không khí: $3\cdot 10^8$m\texttt{/}s.'' -- \cite[p. 14]{SGK_Vat_Ly_8}

%------------------------------------------------------------------------------%

\subsection{Biểu Diễn Lực}

%------------------------------------------------------------------------------%

\subsection{Sự Cân Bằng Lực -- Quán Tính}

%------------------------------------------------------------------------------%

\subsection{Lực Ma Sát}

%------------------------------------------------------------------------------%

\subsection{Áp Suất}

%------------------------------------------------------------------------------%

\subsection{Áp Suất Chất Lỏng -- Bình Thông Nhau}

%------------------------------------------------------------------------------%

\subsection{Áp Suất Khí Quyển}

%------------------------------------------------------------------------------%

\subsection{Lực Đẩy Archimedes}

%------------------------------------------------------------------------------%

\subsection{Thực Hành: Nghiệm Lại Lực Đẩy Asimet}

%------------------------------------------------------------------------------%

\subsection{Sự Nổi}

%------------------------------------------------------------------------------%

\subsection{Công Cơ Học}

%------------------------------------------------------------------------------%

\subsection{Định Luật về Công}

%------------------------------------------------------------------------------%

\subsection{Công Suất}

%------------------------------------------------------------------------------%

\subsection{Cơ Năng}

%------------------------------------------------------------------------------%

\subsection{Sự Chuyển Hóa \& Bảo Toàn Cơ Năng}

%------------------------------------------------------------------------------%

\section{Nhiệt Học}

\subsection{Cách Các Chất Được Cấu Tạo}

%------------------------------------------------------------------------------%

\subsection{Nguyên Tử, Phân Tử Chuyển Động Hay Đứng Yên?}

%------------------------------------------------------------------------------%

\subsection{Nhiệt Năng}

%------------------------------------------------------------------------------%

\subsection{Dẫn Nhiệt}

%------------------------------------------------------------------------------%

\subsection{Đối Lưu -- Bức Xạ Nhiệt}

%------------------------------------------------------------------------------%

\subsection{Công Thức Tính Nhiệt Lượng}

%------------------------------------------------------------------------------%

\subsection{Phương Trình Cân Bằng Nhiệt}

%------------------------------------------------------------------------------%

\subsection{Năng Suất Tỏa Nhiệt của Nhiên Liệu}

%------------------------------------------------------------------------------%

\subsection{Sự Bảo Toàn Năng Lượng trong Các Hiện Tượng Cơ \& Nhiệt}

%------------------------------------------------------------------------------%

\subsection{Động Cơ Nhiệt}

%------------------------------------------------------------------------------%

\printbibliography[heading=bibintoc]
	
\end{document}