\documentclass{article}
\usepackage[backend=biber,natbib=true,style=authoryear]{biblatex}
\addbibresource{/home/hong/1_NQBH/reference/bib.bib}
\usepackage[utf8]{vietnam}
\usepackage{tocloft}
\renewcommand{\cftsecleader}{\cftdotfill{\cftdotsep}}
\usepackage[colorlinks=true,linkcolor=blue,urlcolor=red,citecolor=magenta]{hyperref}
\usepackage{amsmath,amssymb,amsthm,mathtools,float,graphicx,algpseudocode,algorithm,tcolorbox}
\usepackage[inline]{enumitem}
\allowdisplaybreaks
\numberwithin{equation}{section}
\newtheorem{assumption}{Assumption}[section]
\newtheorem{conjecture}{Conjecture}[section]
\newtheorem{corollary}{Corollary}[section]
\newtheorem{hequa}{Hệ quả}[section]
\newtheorem{definition}{Definition}[section]
\newtheorem{dinhnghia}{Định nghĩa}[section]
\newtheorem{example}{Example}[section]
\newtheorem{vidu}{Ví dụ}[section]
\newtheorem{lemma}{Lemma}[section]
\newtheorem{notation}{Notation}[section]
\newtheorem{principle}{Principle}[section]
\newtheorem{problem}{Problem}[section]
\newtheorem{baitoan}{Bài toán}[section]
\newtheorem{proposition}{Proposition}[section]
\newtheorem{question}{Question}[section]
\newtheorem{cauhoi}{Câu hỏi}[section]
\newtheorem{remark}{Remark}[section]
\newtheorem{luuy}{Lưu ý}[section]
\newtheorem{theorem}{Theorem}[section]
\newtheorem{dinhly}{Định lý}[section]
\usepackage[left=0.5in,right=0.5in,top=1.5cm,bottom=1.5cm]{geometry}
\usepackage{fancyhdr}
\pagestyle{fancy}
\fancyhf{}
\lhead{\small Subsect.~\thesubsection}
\rhead{\small\nouppercase{\leftmark}}
\renewcommand{\subsectionmark}[1]{\markboth{#1}{}}
\cfoot{\thepage}
\def\labelitemii{$\circ$}

\title{Problems in Elementary Physics\texttt{/}Grade 8}
\author{Nguyễn Quản Bá Hồng\footnote{Independent Researcher, Ben Tre City, Vietnam\\e-mail: \texttt{nguyenquanbahong@gmail.com}; website: \url{https://nqbh.github.io}.}}
\date{\today}

\begin{document}
\maketitle
\begin{abstract}
	1 bộ sưu tập các bài toán chọn lọc từ cơ bản đến nâng cao cho Vật Lý sơ cấp lớp 8. Tài liệu này là phần bài tập bổ sung cho tài liệu chính \href{https://github.com/NQBH/hobby/blob/master/elementary_physics/grade_8/NQBH_elementary_physics_grade_8.pdf}{GitHub\texttt{/}NQBH\texttt{/}hobby\texttt{/}elementary physics\texttt{/}grade 8\texttt{/}lecture}\footnote{\textsc{url}: \url{https://github.com/NQBH/hobby/blob/master/elementary_physics/grade_8/NQBH_elementary_physics_grade_8.pdf}.} của tác giả viết cho Toán lớp 8. Phiên bản mới nhất của tài liệu này được lưu trữ ở link sau: \href{https://github.com/NQBH/hobby/blob/master/elementary_physics/grade_8/problem/NQBH_elementary_physics_grade_8_problem.pdf}{GitHub\texttt{/}NQBH\texttt{/}hobby\texttt{/}elementary physics\texttt{/}grade 8\texttt{/}problem}\footnote{\textsc{url}: \url{https://github.com/NQBH/hobby/blob/master/elementary_physics/grade_8/problem/NQBH_elementary_physics_grade_8_problem.pdf}.}.
\end{abstract}
\tableofcontents
\newpage

%------------------------------------------------------------------------------%

\section{Cơ Học}

\subsection{Chuyển Động Cơ Học}

\begin{baitoan}[\cite{Van2022}, \textbf{1.1}, p. 9]
	1 ống bằng thép dài $25$\emph{m}. Khi 1 học sinh dùng 1 búa gõ vào 1 đầu ống thì 1 học sinh khác đặt tai ở đầu kia của ống nghe thấy $2$ tiếng gõ; tiếng nọ cách tiếng kia $0.055$\emph{s}.
	\begin{enumerate*}
		\item[(a)] Giải thích tại sao gõ $1$ tiếng mà lại nghe thấy $2$ tiếng.
		\item[(b)] Tìm vận tốc âm thanh trong thép biết vận tốc âm thanh trong không khí là $333$\emph{m\texttt{/}s} \& âm truyền trong thép nhanh hơn trong không khí.
	\end{enumerate*}
\end{baitoan}

\begin{baitoan}[\cite{Van2022}, \textbf{1.2}, p. 9]
	Để đo độ sâu của biển người ta dùng máy phát siêu âm theo nguyên tắc như sau: tia siêu âm được phát thẳng đứng từ máy phát đặt trên mặt biển khi gặp đáy biển sẽ dội lại máy thu đặt liền với máy phát. Căn cứ vào thời gian từ lúc phát siêu âm tới lúc thu được siêu âm người ta sẽ tìm được độ sâu của biển.
	\begin{enumerate*}
		\item[(a)] Tìm chiều sâu của hố Marian (Thái Bình Dương) biết sau khi phát siêu âm đi $73.55$\emph{s} thì máy thu nhận được tia siêu âm trở lại. Cho biết vận tốc siêu âm trong nước biển là $300$\emph{m\texttt{/}s}.
		\item[(b)] Giả sử tại khu vực này có 1 tàu bị nạn chìm xuống với vận tốc $0.5$\emph{m\texttt{/}s} thì bao nhiêu lâu sau tàu chìm tới đáy biển?
	\end{enumerate*}
\end{baitoan}

\begin{baitoan}[\cite{Van2022}, \textbf{1.3}, p. 9]
	1 khán giả ngồi trong nhà nghe ca sĩ hát trực tiếp, còn 1 thính giả ở cách xa nhà hát 1 khoảng cách $l = 7500$\emph{km} nghe ca sĩ đó hát qua máy thu thanh (đặt sát tai). Cho biết micro đặt ngay cạnh ca sĩ, vận tốc của âm là $v = 340$\emph{m\texttt{/}s}, của sóng vô tuyến điện là $c = 3\cdot 10^8$\emph{m\texttt{/}s}.
	\begin{enumerate*}
		\item[(a)] Hỏi khán giả trong nhà hát phải ngồi cách ca sĩ bao nhiêu mét để nghe được đồng thời với thính giả ngoài nhà hát?
		\item[(b)] Hỏi thính giả ngoài nhà hát phải ngồi cách ca sĩ bao nhiêu mét để nghe được đồng thời với 1 khán giả thứ 2 ngồi cách ca sĩ 1 khoảng cách $d = 30$\emph{m}?
	\end{enumerate*}
\end{baitoan}

\begin{baitoan}[\cite{Van2022}, \textbf{1.4}, p. 9]
	1 khẩu pháo chống tăng bắn thẳng vào xe tăng. Pháo thủ thấy xe tăng tung lên sau $0.6$\emph{s} kể từ lúc bắn \& nghe thấy tiếng nổ sau $2.1$\emph{s} kể từ lúc bắn.
	\begin{enumerate*}
		\item[(a)] Tìm khoảng cách từ súng tới xe tăng, cho biết vận tốc của âm $330$\emph{m\texttt{/}s}.
		\item[(b)] Tìm vận tốc của vỏ đạn.
	\end{enumerate*}
\end{baitoan}

\begin{baitoan}[\cite{Van2022}, \textbf{1.5}, p. 9]
	Lúc 7:00 sáng, 1 mô tô đi từ Sài Gòn đến Biên Hòa cách nhau $30$\emph{km}. Lúc 7:20, mô tô còn cách Biên Hòa $10$\emph{km}.
	\begin{enumerate*}
		\item[(a)] Tính vận tốc của mô tô.
		\item[(b)] Nếu mô tô đi liên tục không nghỉ thì sẽ đến Biên Hòa lúc mấy giờ?
	\end{enumerate*}
\end{baitoan}

\begin{baitoan}[\cite{Van2022}, \textbf{1.6}, p. 9]
	1 người đi xe đạp xuống 1 cái dốc dài $100$\emph{m}. Trong $25$\emph{m} đầu, người ấy đi hết $10$\emph{s}; quãng đường còn lại đi mất $15$\emph{s}. Tính vận tốc trung bình ứng với từng đoạn dốc \& cả dốc.
\end{baitoan}

\begin{baitoan}[\cite{Van2022}, \textbf{1.7}, p. 9]
	1 ô tô vượt qua 1 đoạn đường dốc gồm $2$ đoạn: lên dốc \& xuống dốc. Biết thời gian lên dốc bằng $\frac{1}{2}$ thời gian xuống dốc, vận tốc trung bình khi xuống dốc gấp $2$ lần vận tốc trung bình khi lên dốc. Tính vận tốc trung bình trên cả đoạn đường dốc của ô tô. Biết vận tốc trung bình khi lên dốc là $30$\emph{km\texttt{/}h}.
\end{baitoan}

\begin{baitoan}[\cite{Van2022}, \textbf{1.8}, pp. 9--10]
	Trên đoạn đường dốc gồm $3$ đoạn: lên dốc, đường bằng, \& xuống dốc. Khi lên dốc mất thời gian $30$\emph{ph}, trên đoạn đường bằng phẳng xe chuyển động đều với vận tốc $60$\emph{km\texttt{/}h} mất thời gian $10$\emph{ph}, đoạn xuống dốc mất thời gian $10$\emph{ph}. Biết vận tốc trung bình khi lên dốc bằng $\frac{1}{2}$ vận tốc trên đoạn đường bằng phẳng, vận tốc khi xuống dốc gấp $\frac{3}{2}$ vận tốc trên đoạn đường bằng. Tính chiều dài cả dốc trên.
\end{baitoan}

\begin{baitoan}[\cite{Van2022}, \textbf{1.9}, p. 10]
	1 người đi xe đạp, nửa đầu quãng đường có vận tốc $v_1 = 12$\emph{km\texttt{/}h}, nửa sau quãng đường có vận tốc $v_2$ không đổi. Biết vận tốc trung bình trên cả quãng đường là $v = 8$\emph{km\texttt{/}h}, tính $v_2$.
\end{baitoan}

\begin{baitoan}[\cite{Van2022}, \textbf{1.10}, p. 10]
	1 chuyển động trong nửa đầu quãng đường, chuyển động có vận tốc không đổi $v_1$, trong nửa quãng đường còn lại có vận tốc $v_2$. Tính vận tốc trung bình của nó trên toàn bộ quãng đường. Chứng tỏ: vận tốc trung bình này không lớn hơn trung bình cộng của 2 vận tốc $v_1$ \& $v_2$.
\end{baitoan}

\begin{baitoan}[\cite{Van2022}, \textbf{1.11}, p. 10]
	1 chuyển động trong nửa thời gian chuyển động với vận tốc $v_1$, quãng đường còn lại chuyển động với vận tốc $v_2$. Tính vận tốc trung bình của nó trên cả quãng đường. So sánh vận tốc trung bình trên cả quãng đường trong 2 bài toán trước.
\end{baitoan}

\begin{baitoan}[\cite{Van2022}, \textbf{1.12}, p. 10]
	1 ô tô chuyển động trên nửa đầu đoạn đường với vận tốc $60$\emph{km\texttt{/}h}. Phần còn lại, nó chuyển động với vận tốc $15$\emph{km\texttt{/}h} trong nửa thời gian đầu \& $45$\emph{km\texttt{/}h} trong nửa thời gian sau. Tìm vận tốc trung bình của ô tô trên cả đoạn đường.
\end{baitoan}

\begin{baitoan}[\cite{Van2022}, \textbf{1.13}, p. 10]
	1 người đi từ $A$ đến $B$. $\frac{1}{3}$ quãng đường đầu người đó đi với vận tốc $v_1$, $\frac{2}{3}$ thời gian còn lại đi với vận tốc $v_2$. Quãng đường cuối cùng đi với vận tốc $v_3$. Tính vận tốc trung bình của người đó trên cả quãng đường.
\end{baitoan}

\begin{baitoan}[\cite{Van2022}, \textbf{1.14}, p. 10]
	1 ca nô chạy giữa 2 bến sông cách nhau $90$\emph{km}. Vận tốc cano đối với nước là $25$\emph{km\texttt{/}h} \& vận tốc nước chảy là $1.39$\emph{m\texttt{/}s}.
	\begin{enumerate*}
		\item[(a)] Tìm thời gian ca nô đi ngược dòng từ bến nọ tới bến kia.
		\item[(b)] Giả sử không nghỉ lại ở bến tới, tìm thời gian ca nô đi \& về.
	\end{enumerate*}
\end{baitoan}

\begin{baitoan}[\cite{Van2022}, \textbf{1.15}, p. 10]
	1 chiếc thuyền khi xuôi dòng mất thời gian $t_1$, ngược dòng mất thời gian $t_2$. Hỏi nếu thuyền trôi theo dòng nước trên quãng đường trên sẽ mất thời gian bao nhiêu?
\end{baitoan}

\begin{baitoan}[\cite{Van2022}, \textbf{1.16}, p. 10]
	1 thuyền đi từ A đến B (cách nhau $6$\emph{km}) mất thời gian $1$\emph{h} rồi lại đi từ B trở về A mất $\rm 1h30ph$. Biết vận tốc của thuyền so với nước \& vận tốc nước so với bờ không đổi.
	\begin{enumerate*}
		\item[(a)] Nước chảy theo chiều nào?
		\item[(b)] Vận tốc thuyền so với nước \& vận tốc nước so với bờ.
		\item[(c)] Muốn thời gian đi từ B trở về A cùng là $1$\emph{h} thì vận tốc của thuyền so với nước phải là bao nhiêu?
	\end{enumerate*}
\end{baitoan}

\begin{baitoan}[\cite{Van2022}, \textbf{1.17}, p. 10]
	1 người đi xe đạp từ A đến B có chiều dài $24$\emph{km}. Nếu đi liên tục không nghỉ thì sau $2$\emph{h} người đó sẽ đến B. Nhưng khi đi được $30$\emph{ph}, người đó dừng lại $15$\emph{ph} rồi mới đi tiếp. Hỏi ở quãng đường sau người đó phải đi vận tốc bao nhiêu để đến B kịp lúc?
\end{baitoan}

\begin{baitoan}[\cite{Van2022}, \textbf{1.18}, p. 10]
	1 người đi mô tô trên quãng đường dài $60$\emph{km}. Lúc đầu, người này dự định đi với vận tốc $30$\emph{km\texttt{/}h}. Nhưng sau $\frac{1}{4}$ quãng đường đi, người này muốn đến nơi sớm hơn $30$\emph{ph}. Hỏi ở quãng đường sau người đó phải đi với vận tốc bao nhiêu?
\end{baitoan}

\begin{baitoan}[\cite{Van2022}, \textbf{1.19}, pp. 10--11]
	1 người đi xe đạp từ A đến B với vận tốc $v_1 = 12$\emph{km\texttt{/}h}. Nếu người đó tăng vận tốc lên $3$\emph{km\texttt{/}h} thì đến nơi sớm hơn $1$\emph{h}.
	\begin{enumerate*}
		\item[(a)] Tìm quãng đường $AB$ \& thời gian dự định đi từ A đến B.
		\item[(b)] Ban đầu người đó đi với vận tốc $v_1 = 12$\emph{km\texttt{/}h} được 1 quãng đường $s_1$ thì xe bị hư phải sửa chữa mất $15$\emph{ph}. Do đó trong quãng đường con lại người ấy đi với vận tốc $v_2 = 15$\emph{km\texttt{/}h} thì đến nơi vẫn sớm hơn dự định $30$\emph{ph}. Tìm quãng đường $s_1$.
	\end{enumerate*}
\end{baitoan}

\begin{baitoan}[\cite{Van2022}, \textbf{1.20}, p. 11]
	1 người đi xe đạp từ A đến B với dự định mất $t = 4$\emph{h}. Do nửa quãng đường sau người ấy tăng vận tốc thêm $3$\emph{km\texttt{/}h} nên đến sớm hơn dự định $20$\emph{ph}.
	\begin{enumerate*}
		\item[(a)] Tính vận tốc dự định \& quãng đường $AB$.
		\item[(b)] Nếu sau khi đi được $1$\emph{h}, do có việc người ấy phải ghé lại mất $30$\emph{ph}. Hỏi đoạn đường còn lại người ấy phải đi với vận tốc bao nhiêu để đến nơi như dự định?
	\end{enumerate*}
\end{baitoan}

\begin{baitoan}[\cite{Van2022}, \textbf{1.21}, p. 11, TS PTNK ĐHQG TPHCM 2001]
	Minh \& Nam đứng ở 2 điểm $M,N$ cách nhau $750$\emph{m} trên 1 bãi sông. Khoảng cách từ $M$ đến sông $150$\emph{m}, từ $N$ đến sông $600$\emph{m}. Tính thời gian ngắn nhất để Minh chạy ra sông múc 1 thùng nước mang đến chỗ Nam. Cho biết đoạn sông thẳng, vận tốc chạy của Minh không đổi $v = 2$\emph{m\texttt{/}s}; bỏ qua thời gian múc nước.
\end{baitoan}

\begin{baitoan}[\cite{Van2022}, \textbf{1.22}, p. 11, TS PTNK ĐHQG TPHCM 2001]
	1 viên bi được thả lăn từ đỉnh 1 cái dốc xuống chân dốc. Bi đi xuống nhanh dần \& quãng đường mà bi đi được trong giây thứ $i$ là: $s_i = 4i - 2$ \emph{(m)}, $i = 1,\ldots,n$.
	\begin{enumerate*}
		\item[(a)] Tính quãng đường mà bi đi được: trong giây thứ $2$, sau $2$\emph{s}.
		\item[(b)] Chứng minh: quãng đường tổng cộng mà bi đi được sau $n$ giây ($i,n\in\mathbb{N}$) là $L_n = 2n^2$ \emph{(m)}.
	\end{enumerate*}
\end{baitoan}

\begin{baitoan}[\cite{Van2022}, \textbf{1.23}, p. 11, TS PTNK ĐHQG TPHCM 2003]
	\begin{enumerate*}
		\item[(a)] 2 đĩa mỏng, đồng trục, đặt cách nhau $L = 0.5$\emph{m} đang quay đều cùng với trục. 1 viên đạn bay song song với trục, xuyên qua cả 2 đĩa, vận tốc $v$ của nó hầu như không thay đổi trên đoạn đường ngắn này. Khi dựng các đường kính đi qua vết đạn trên 2 đĩa, người ta thấy chúng tạo với nhau 1 góc $12^\circ$. Biết tốc độ quay của trục $n = 1600$\emph{vòng\texttt{/}phút}. Tính $v$.
		\item[(b)] Vận tốc của 1 vật chuyển động thẳng bằng $v_0$ trong khoảng thời gian $0$ đến $t_0$ \& bằng $v_0 + a(t - t_0)$ ở các thời điểm $t > t_0$ với $a$ là 1 số dương không đổi cho trước. Tìm quãng đường vật đi được sau thời gian $t > t_0$ theo $v_0,t_0,t$, \& $a$.
	\end{enumerate*}
\end{baitoan}

\begin{baitoan}[\cite{Van2022}, \textbf{1.24}, p. 11]
	1 học sinh đi từ nhà đến trường, sau khi đi được $\frac{1}{4}$ quãng đường thì chợt nhớ mình quên 1 quyển sách nên vội trở về \& đi ngay đến trường thì trễ mất $15$\emph{ph}.
	\begin{enumerate*}
		\item[(a)] Tính vận tốc chuyển động của em học sinh, biết quãng đường từ nhà tới trường là $s = 6$\emph{km}. Bỏ qua thời gian lên xuống xe khi về nhà.
		\item[(b)] Để đến trường đúng thời gian dự định thì quay về \& đi lần 2, phải đi với vận tốc bao nhiêu?
	\end{enumerate*}
\end{baitoan}

\begin{baitoan}[\cite{Van2022}, \textbf{1.25}, p. 11]
	1 thuyền máy dự định đi xuôi dòng từ A đến B rồi lại quay về. Biết vận tốc của thuyền so với nước yên lặng là $15$\emph{km\texttt{/}h}, vận tốc của nước so với bờ là $3$\emph{km\texttt{/}h}, $AB$ dài $18$\emph{km}.
	\begin{enumerate*}
		\item[(a)] Tính thời gian chuyển động của thuyền.
		\item[(b)] Tuy nhiên, trên đường quay về $A$, thuyền bị hỏng máy \& sau $24$\emph{ph} thì sửa xong. Tính thời gian chuyển động của thuyền.
	\end{enumerate*}
\end{baitoan}

\begin{baitoan}[\cite{Van2022}, \textbf{1.26}, p. 12]
	1 chiếc xuồng máy chuyển động xuôi dòng nước giữa 2 bến sông cách nhau $100$\emph{km}. Khi cách đích $10$\emph{km} thì xuồng bị hỏng máy.
	\begin{enumerate*}
		\item[(a)] Tính thời gian xuồng máy đi hết đoạn đường đó biết vận tốc của xuồng đối với nước là $35$\emph{km\texttt{/}h} \& của nước là $5$\emph{km\texttt{/}h}. Thời gian sửa mất $12$\emph{ph}, sau khi sửa vẫn đi với vận tốc như cũ.
		\item[(b)] Nếu xuồng không phải sửa thì về đến nơi mất bao lâu?
	\end{enumerate*}
\end{baitoan}

\begin{baitoan}[\cite{Van2022}, \textbf{1.27}, p. 12]
	1 thuyền đánh cá chuyển động ngược dòng nước làm rớt 1 cái phao. Do không phát hiện kịp, thuyền tiếp tục chuyển động thêm $30$\emph{ph} nữa thì mới quay lại \& gặp phao tại nơi cách chỗ làm rớt $5$\emph{km}. Tìm vận tốc của dòng nước, biết vận tốc của thuyền đối với nước không đổi.
\end{baitoan}

\begin{baitoan}[\cite{Van2022}, \textbf{1.28}, p. 12]
	1 chiếc bè bằng gỗ trôi trên sông. Khi cách 1 bến phà $15$\emph{km} thì bị 1 ca nô chạy cùng chiều vượt qua. Sau khi vượt qua bè được $45$\emph{ph} thì ca nô quay lại \& gặp bè ở 1 nơi chỉ còn cách bến phà $6$\emph{km}. Tìm vận tốc nước chảy.
\end{baitoan}

\begin{baitoan}[\cite{Van2022}, \textbf{1.29}, p. 12, TS PTNK ĐHQG TPHCM 1997]
	Ca nô đang ngược dòng qua điểm $A$ thì gặp 1 bè gỗ trôi xuôi. Ca nô đi tiếp $40$\emph{ph}, do hỏng máy nên bị trôi theo dòng nước. Sau $10$\emph{ph} sửa xong máy, ca nô quay lại đuổi theo bè \& gặp bè tại $B$. Cho biết $AB = 4.5$\emph{km}, công suất của ca nô không đổi trong suốt quá trình chuyển động. Tính vận tốc dòng nước.
\end{baitoan}

\begin{baitoan}[\cite{Van2022}, \textbf{1.30}, p. 12]
	Long có việc phải ra bưu điện. Long có thể đi bộ với vận tốc $5$\emph{km\texttt{/}h} hoặc cũng có thể chờ $20$\emph{ph} thì sẽ có xe bus dừng trước cửa nhà \& đi xe bus ra bưu điện với vận tốc $30$\emph{km\texttt{/}h}. Long nên chọn cách nào để đến bưu điện sớm hơn. (Biện luận theo khoảng cách từ nhà đến bưu điện).
\end{baitoan}

\begin{baitoan}[\cite{Van2022}, \textbf{1.31}, p. 12]
	Ông Bình định đi xe máy từ nhà đến cơ quan, nhưng xe không nổ được máy, nên đành đi bộ. Ở nhà, con ông sửa được xe, liền lấy xe đuổi theo để đèo ông đi tiếp. Nhờ đó, thời gian tổng cộng để ông đến cơ quan chỉ bằng $\frac{1}{2}$ thời gian nếu ông phải đi bộ suốt quãng đường, nhưng cũng vẫn gấp $3$ thời gian nếu ông đi xe máy ngay từ nhà. Hỏi ông đã đi bộ được mấy phần quãng đường thì con ông đuổi kịp?
\end{baitoan}

\begin{baitoan}[\cite{Van2022}, \textbf{1.32}, p. 12]
	Tâm đi thăm 1 người bạn cách nhà mình $22$\emph{km} bằng xe đạp. Chú Tâm bảo Tâm chờ $10$\emph{ph} \& dùng xe mô tô đèo Tâm với vận tốc $40$\emph{km\texttt{/}h}. Sau khi đi được $15$\emph{ph} xe hư phải chờ sửa xe trong $30$\emph{ph}. Sau đó chú Tâm \& Tâm tiếp tục đi với vận tốc là $10$\emph{m\texttt{/}s}. Tâm đến nhà bạn sớm hơn dự định đi xe đạp là $25$\emph{ph}. Hỏi nếu đi xe đạp thì Tâm phải đi với vận tốc bao nhiêu?
\end{baitoan}

\begin{baitoan}[\cite{Van2022}, \textbf{1.33}, p. 12]
	Hàng ngày, bố Lâm đạp xe từ nhà tới trường đón con, bao giờ ông cũng đến trường đúng lúc Lâm ra tới cổng trường. 1 hôm, Lâm tan học sớm hơn thường lệ $45$\emph{ph}, em đi bộ về luôn nên giữa đường gặp bố đang đạp xe đến đón. Bố liền đèo em về nhà sớm hơn được $30$\emph{ph} so với mọi hôm.
	\begin{enumerate*}
		\item[(a)] Lâm đã đi bộ trong bao lâu?
		\item[(b)] So sánh vận tốc của xe đạp với vận tốc đi bộ của Lâm.
	\end{enumerate*}
\end{baitoan}

\begin{baitoan}[\cite{Van2022}, \textbf{1.34}, p. 12]
	2 anh em Bình, An muốn đến thăm bà ở cách nhà mình $12$\emph{km}, mà chỉ có 1 chieec xe đạp không đèo được. Vận tốc của Bình khi đi bộ \& khi đi xe đạp lần lượt là $4$\emph{km\texttt{/}h} \& $12$	\emph{km\texttt{/}h}, còn của An là $5$\emph{km\texttt{/}h} \& $10$\emph{km\texttt{/}h}. Hỏi 2 anh em có thể thay nhau dùng xe như thế nào để xuất phát cùng 1 lúc \& đến nơi cũng cùng 1 lúc? (Xe có thể dựng bên đường \& thời gian lên hoặc xuống xe không đáng kể.) Mỗi người chỉ đi xe đạp 1 lần.
\end{baitoan}

\begin{baitoan}[\cite{Van2022}, \textbf{1.35}, p. 12]
	2 xe ô tô chuyển động đều ngược chiều nhau từ 2 địa điểm cách nhau $150$\emph{km}. Hỏi sau bao nhiêu lâu thì chúng gặp nhau biết rằng vận tốc xe thứ nhất là $60$\emph{km\texttt{/}h} \& vận tốc xe thứ 2 là $40$\emph{km\texttt{/}h}?
\end{baitoan}

\begin{baitoan}[\cite{Van2022}, \textbf{1.36}, p. 13]
	1 ô tô chuyển động đều với vận tốc $60$\emph{km\texttt{/}h} đuổi theo 1 xe khách cách nó $50$\emph{km}. Biết xe khách có vận tốc là $40$\emph{km\texttt{/}h}. Hỏi bao lâu sau thì ô tô đuổi kịp xe khách?
\end{baitoan}

\begin{baitoan}[\cite{Van2022}, \textbf{1.37}, p. 13]
	2 người chuyển động đều khởi hành cùng 1 lúc. Người thứ nhất khởi hành từ A với vận tốc $v_1$. Người thứ 2 khởi hành từ B với vận tốc $v_2$ ($v_2 < v_1$). $AB$ dài $20$\emph{km}. Nếu 2 người đi ngược chiều nhau thì sau $12$\emph{ph} thì gặp nhau. Nếu 2 người đi cùng chiều nhau thì sau $1$\emph{h} người thứ nhất đuổi kịp người thứ 2. Tính vận tốc của mỗi người.
\end{baitoan}

\begin{baitoan}[\cite{Van2022}, \textbf{1.38}, p. 13]
	Trên 1 đường thẳng, có 2 xe chuyển động đều với vận tốc không đổi. Xe 1 chuyển động với vận tốc $35$\emph{km\texttt{/}h}. Nếu đi ngược chiều nhau thì sau $30$\emph{ph}, khoảng cách giữa 2 xe giảm $25$\emph{km}. Nếu đi cùng chiều nhau thì sau bao lâu khoảng cách giữa chúng thay đổi $5$\emph{km}? Có nhận xét gì?
\end{baitoan}

\begin{baitoan}[\cite{Van2022}, \textbf{1.39}, p. 13]
	Lúc 7:00, 1 người đi bộ khởi hành từ A đi về B với vận tốc $v_1 = 4$\emph{km\texttt{/}h}. Lúc 9:00, 1 người đi xe đạp cũng xuất phát từ A đi về B với vận tốc $v_2 = 12$\emph{km\texttt{/}h}.
	\begin{enumerate*}
		\item[(a)] 2 người gặp nhau lúc mấy giờ? Nơi gặp nhau cách A bao nhiêu?
		\item[(b)] Lúc mấy giờ, 2 người đó cách nhau $2$\emph{km}.
	\end{enumerate*}
\end{baitoan}

\begin{baitoan}[\cite{Van2022}, \textbf{1.40}, p. 13]
	Xe thứ nhất khởi hành từ A chuyển động đều đến B với vận tốc $36$\emph{km\texttt{/}h}. Nửa giờ sau, xe thứ 2 chuyển động đều từ B đến A với vận tốc $5$\emph{m\texttt{/}s}. Biết quãng đường từ A đến B dài $72$\emph{km}. Hỏi sau bao lâu kể từ lúc xe 2 khởi hành thì:
	\begin{enumerate*}
		\item[(a)] 2 xe gặp nhau.
		\item[(b)] 2 xe cách nhau $13.5$\emph{km}.
	\end{enumerate*}
\end{baitoan}

\begin{baitoan}[\cite{Van2022}, \textbf{1.41}, p. 13]
	An \& Bình cùng đi từ A đến B ($AB = 6$\emph{km}). An đi với vận tốc $v_1 = 12$\emph{km\texttt{/}h}, Bình khởi hành sau An $15$\emph{ph} \& đến nơi sau An $30$\emph{ph}.
	\begin{enumerate*}
		\item[(a)] Tìm vận tốc của Bình.
		\item[(b)] Để đến nơi cùng lúc với An, Bình phải đi với vận tốc bao nhiêu?
	\end{enumerate*}
\end{baitoan}

\begin{baitoan}[\cite{Van2022}, \textbf{1.42}, p. 13]
	2 xe cùng khởi hành từ 1 nơi \& cùng đi quãng đường $60$\emph{km}. Xe 1 đi với vận tốc $30$\emph{km\texttt{/}h}, đi liên tục không nghỉ \& đến nơi sớm hơn xe 2 $30$\emph{ph}. Xe 2 khởi hành sớm hơn $1$\emph{h}, nhưng nghỉ giữa đường $45$\emph{ph}.
	\begin{enumerate*}
		\item[(a)] Vận tốc của xe 2?
		\item[(b)] Muốn đến nơi cùng lúc với xe 1, xe 2 phải đi với vận tốc bao nhiêu?
	\end{enumerate*}
\end{baitoan}

\begin{baitoan}[\cite{Van2022}, \textbf{1.43}, p. 13, TS PT chuyên Lý ĐHQG Hà Nội 2003]
	3 người đi xe đạp từ A đến B với các vận tốc không đổi. Người thứ nhất \& người thứ 2 xuất phát cùng 1 lúc với các vận tốc tương ứng là $v_1 = 10$\emph{km\texttt{/}h} \& $v_2 = 12$\emph{km\texttt{/}h}. Người thứ 3 xuất phát sau 2 người nói trên $30$\emph{ph}. Khoảng thời gian giữa 2 lần gặp của người thứ 3 với 2 người đi trước là $\Delta t = 1$\emph{h}. Tìm vận tốc của người thứ 3.
\end{baitoan}

\begin{baitoan}[\cite{Van2022}, \textbf{1.44}, p. 13]
	1 người đi xe đạp (với vận tốc $8$\emph{km\texttt{/}h}) \& 1 người đi bộ (với vận tốc $4$\emph{km\texttt{/}h}) khởi hành cùng 1 lúc ở cùng 1 nơi \& chuyển động ngược chiều nhau. Sau khi đi được $30$\emph{ph}, người đi xe đạp dừng lại, nghỉ $30$\emph{ph} rồi quay trở lại đuổi theo người đi bộ (với vận tốc như cũ). Hỏi kể từ lúc cùng khởi hành, sau bao lâu người đi xe đạp đuổi kịp người đi bộ?
\end{baitoan}

\begin{baitoan}[\cite{Van2022}, \textbf{1.45}, p. 13]
	Cùng 1 lúc, co 2 người cùng khởi hành từ A để đi trên quãng đường $ABC$ (với $AB = 2BC$). Người thứ nhất đi quãng đường $AB$ với vận tốc $12$\emph{km\texttt{/}h}, quãng đường $BC$ với vận tốc $4$\emph{km\texttt{/}h}. Người thứ 2 đi quãng đường $AB$ với vận tốc $4$\emph{km\texttt{/}h}, quãng $BC$ với vận tốc $12$\emph{km\texttt{/}h}. Người nọ đến trước người kia $30$\emph{ph}. Ai đến nơi sớm hơn? Tính chiều dài quãng đường $ABC$.
\end{baitoan}

\begin{baitoan}[\cite{Van2022}, \textbf{1.46}, p. 13]
	Trên 1 đường thẳng, có 2 xe A, B chuyển động cùng chiều với vận tốc $v_1,v_2$. Tính vận tốc $v_3$ của xe C để:
	\begin{enumerate*}
		\item[(a)] Xe C luôn luôn ở chính giữa 2 xe A, B.
		\item[(b)] Xe C cách xe A $2$ lần khoảng cách đến xe B.
		\item[(c)] Xe C cách xe A $n$ lần khoảng cách đến xe B, với $n\in\mathbb{N}^\star$.
	\end{enumerate*}
\end{baitoan}

\begin{baitoan}[\cite{Van2022}, \textbf{1.47}, p. 13]
	Lúc 6:00 1 người đi xe đạp xuất phát từ a đi về B với vận tốc $v_1 = 12$\emph{km\texttt{/}h}. Sau đó $2$\emph{h}, 1 người đi bộ từ B về A với vận tốc $v_2 = 4$\emph{km\texttt{/}h}. Biết $AB = 48$\emph{km}.
	\begin{enumerate*}
		\item[(a)] 2 người gặp nhau lúc mấy giờ? Nơi gặp nhau cách A bao nhiêu \emph{km}?
		\item[(b)] Nếu người đi xe đạp sau khi đi được $2$\emph{h} rồi nghỉ ngơi $1$\emph{h} thì 2 người gặp nhau lúc mấy giờ? Nơi gặp cách A bao nhiêu \emph{km}?
	\end{enumerate*}
\end{baitoan}

\begin{baitoan}[\cite{Van2022}, \textbf{1.48}, p. 13]
	Hàng ngày, ô tô thứ I xuất phát từ A lúc 6:00 đi về B, ô tô thứ II xuất phát từ B đi về A lúc 7:00 \& 2 xe gặp nhau lúc 9:00. 1 hôm, ô tô thứ I xuất phát từ A vào lúc 8:00 còn ô tô thứ II vẫn khởi hành lúc 7:00 nên 2 xe gặp nhau lúc 9:48. Hỏi hằng ngày ô tô thứ I sẽ đế B \& ô tô thứ 2 sẽ đến A lúc mấy giờ? Cho vận tốc của mỗi xe không đổi.
\end{baitoan}

\begin{baitoan}[\cite{Van2022}, \textbf{1.49}, p. 13]
	1 người đi bộ khởi hành từ C đi đến B với vận tốc $v_1 = 5$\emph{km\texttt{/}h}. Sau khi đi được $2$\emph{h}, người ấy ngồi nghỉ $30$\emph{ph} rồi đi tiếp về B. 1 người khác đi xe đạp khởi hành từ A ($AC > CB$ \& $C$ nằm giữa $AB$) cũng đi về B với vận tốc $v_2 = 15$\emph{km\texttt{/}h} nhưng khởi hành sau người đi bộ $1$\emph{h}.
	\begin{enumerate*}
		\item[(a)] Tính quãng đường $AC$ \& $AB$, biết cả 2 người đến B cùng lúc \& khi người đi bộ bắt đầu ngồi nghỉ thì người đi xe đạp đã đi được $\frac{3}{4}$ quãng đường $AC$.
		\item Để gặp người đi bộ tại chỗ ngồi nghỉ, người đi xe đạp phải đi với vận tốc bao nhiêu?
	\end{enumerate*}
\end{baitoan}

\begin{baitoan}[\cite{Van2022}, \textbf{1.50}, p. 13]
	2 xe đạp cùng xuất phát từ 1 điểm trên vòng đua hình tròn bán kính $200$\emph{m}.
	\begin{enumerate*}
		\item[(a)] Hỏi bao nhiêu lâu sau thì chúng gặp nhau biết vận tốc của 2 xe là $30$\emph{km\texttt{/}h} \& $32$\emph{km\texttt{/}h}?
		\item[(b)] Trong $2$\emph{h} đuổi nhau như vậy, 2 xe đạp gặp nhau mấy lần?
	\end{enumerate*}
\end{baitoan}

\begin{baitoan}[\cite{Van2022}, \textbf{1.51}, p. 13]
	1 chiếc thuyền xuôi dòng từ A đến B, rồi ngược dòng từ B đến A hết $\rm 2h30ph$.
	\begin{enumerate*}
		\item[(a)] Tính khoảng cách $AB$, biết vận tốc thuyền khi xuôi dòng là $v_1 = 18$\emph{km\texttt{/}h}; khi ngược dòng là $v_2 = 12$\emph{km\texttt{/}h}.
		\item[(b)] Trước khi thuyền khởi hành $t_3 = 30$\emph{ph}, có 1 chiếc bè trôi theo dòng nước qua $A$. Tìm thời điểm các lần thuyền \& bè gặp nhau; khoảng cách từ nơi gặp nhau đến A?
	\end{enumerate*}
\end{baitoan}

\begin{baitoan}[\cite{Van2022}, \textbf{1.52}, p. 13]
	2 địa điểm A \& B cách nhau $72$\emph{km}. Cùng lúc, 1 ô tô đi từ A \& 1 người đi xe đạp từ B ngược chiều nhau \& gặp nhau sau $\rm 1h12ph$. Sau đó, ô tô tiếp tục về B rồi quay lại với vận tốc cũ \& gặp lại người đi xe đạp sau $48$\emph{ph} kể từ lần gặp trước.
	\begin{enumerate*}
		\item[(a)] Tính vận tốc của xe ô tô \& xe đạp.
		\item[(b)] Nếu ô tô tiếp tục đi về A rồi quay lại thì sẽ gặp người đi xe đạp sau bao lâu (kể từ lần gặp thứ 2)?
	\end{enumerate*}
\end{baitoan}

\begin{baitoan}[\cite{Van2022}, \textbf{1.53}, p. 13]
	Giang \& Huệ cùng đứng 1 nơi trên 1 chiếc cầu $AB$ cách đầu cầu $50$\emph{m}. Lúc Tâm vừa đến 1 nơi cách đầu cầu A 1 quãng đúng bằng chiều dài chiếc cầu thì Giang \& Huệ bắt đầu đi 2 hướng ngược nhau. Giang đi phía Tâm \& Tâm gặp Giang ở đầu cầu A, gặp Huệ ở đầu cầu B. Biết vận tốc của Giang bằng $\frac{1}{2}$ vận tốc của Huệ. Tìm chiều dài $l$ của chiếc cầu.
\end{baitoan}

\begin{baitoan}[\cite{Van2022}, \textbf{1.54}, p. 13]
	An \& Bình cùng đứng ở giữa 1 chiếc cầu. Khi gặp Long đang đi xe đạp về phía đầu cầu A, cách A đúng bằng chiều dài cầu thì 2 bạn chia tay, đi về 2 phía. An đi về phía A với vận tốc $6$\emph{km\texttt{/}h} \& gặp Long sau thời gian $t_1 = 3$\emph{ph} tại A. Sau đó 2 bạn đèo nhau cùng đuổi theo Bình \& gặp bạn tại đầu cầu B sau khi họ gặp nhau là $t_2 = 3.75$\emph{ph}. Biết vận tốc của An gấp $1.5$ lần vận tốc của Bình.
	\begin{enumerate*}
		\item[(a)] Tính chiều dài chiếc cầu, vận tốc của người đi xe đạp.
		\item[(b)] Nếu 2 bạn vẫn ngồi giữa cầu thì sẽ gặp Long sau bao lâu?
	\end{enumerate*}
\end{baitoan}

\begin{baitoan}[\cite{Van2022}, \textbf{1.55}, p. 15]
	3 người cùng khởi hành từ A lúc 8:00 để đến B ($AB = s = 8$\emph{km}). Do chỉ có 1 xe đạp nên người thứ nhất chở người thứ 2 đến B với vận tốc $v_1 = 16$\emph{km\texttt{/}h}, rồi quay lại đón người thứ 3. Trong lúc đó người thứ 3 đi bộ đến B với vận tốc $v_2 = 4$\emph{km\texttt{/}h}.
	\begin{enumerate*}
		\item[(a)] Người thứ 3 đến B lúc mấy giờ? Quãng đường phải đi bộ là bao nhiêu \emph{km}?
		\item[(b)] Để đến B chậm nhất lúc $9$\emph{h}, người thứ nhất bỏ người thứ 2 tại điểm nào đó rồi quay lại đón người thứ 3. Tìm quãng đường đi bộ của người thứ 3 \& thứ 2. (Vận tốc đi bộ của người thứ 2 vẫn bằng người thứ 3.) Người thứ 2 đến B lúc mấy giờ?
	\end{enumerate*}
\end{baitoan}

\begin{baitoan}[\cite{Van2022}, \textbf{1.56}, p. 15]
	Người thứ nhất khởi hành từ A đến B với vận tốc $8$\emph{km\texttt{/}h}. Cùng lúc đó người thứ 2 \& thứ 3 cùng khởi hành từ B về A với vận tốc lần lượt là $4$\emph{km\texttt{/}h} \& $15$\emph{km\texttt{/}h}. Khi người thứ 3 gặp người thứ nhất thì lập tức quay lại chuyển động về phía người thứ 2. Khi gặp người thứ 2 cũng lập tức quay lại chuyển động về phía người thứ nhất \& quá trình cứ tiếp diễn cho đến lúc 3 người ở cùng 1 nơi. Hỏi kể từ lúc khởi hành cho đến khi 3 người ở cùng 1 nơi thì người thứ 3 đã đi được quãng đường bằng bao nhiêu? Biết chiều dài quãng đường $AB$ là $48$\emph{km}.
\end{baitoan}

\begin{baitoan}[\cite{Van2022}, \textbf{1.59}, pp. 15--16]
	1 người đi bộ khởi hành từ A với vận tốc $v_1 = 5$\emph{km\texttt{/}h} ($AB = 20$\emph{km}). Người này cứ đi $1$\emph{h} lại nghỉ $30$\emph{ph}.
	\begin{enumerate*}
		\item[(a)] Hỏi sau bao lâu thì người đó đến B. Đã nghỉ mấy lần? Đi được mấy đoạn?
		\item[(b)] 1 người khác đi xe đạp từ B về A với vận tốc $v_2 = 20$\emph{km\texttt{/}h}. Sau khi đến A lại quay về B với vận tốc cũ, rồi lại tiếp tục đi. Sau khi người đi bộ đến B, người đi xe đạp cũng nghỉ tại B. Hỏi:
		\begin{enumerate*}
			\item[$\bullet$] Họ gặp nhau mấy lần?
			\item[$\bullet$] Các lần gặp nhau có gì đặc biệt?
			\item[$\bullet$] Thử tìm vị trí \& thời điểm họ gặp nhau?
		\end{enumerate*}
	\end{enumerate*}
\end{baitoan}

%------------------------------------------------------------------------------%

\section{Nhiệt Học}

%------------------------------------------------------------------------------%

\section{Quang Học}

%------------------------------------------------------------------------------%

\section{Điện Học}

%------------------------------------------------------------------------------%

\printbibliography[heading=bibintoc]
	
\end{document}