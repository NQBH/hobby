\documentclass{article}
\usepackage[backend=biber,natbib=true,style=authoryear]{biblatex}
\addbibresource{/home/nqbh/reference/bib.bib}
\usepackage[utf8]{vietnam}
\usepackage{tocloft}
\renewcommand{\cftsecleader}{\cftdotfill{\cftdotsep}}
\usepackage[colorlinks=true,linkcolor=blue,urlcolor=red,citecolor=magenta]{hyperref}
\usepackage{amsmath,amssymb,amsthm,mathtools,float,graphicx,algpseudocode,algorithm,tcolorbox}
\usepackage[inline]{enumitem}
\allowdisplaybreaks
\numberwithin{equation}{section}
\newtheorem{assumption}{Assumption}[section]
\newtheorem{baitoan}{Bài toán}
\newtheorem{cauhoi}{Câu hỏi}[section]
\newtheorem{conjecture}{Conjecture}[section]
\newtheorem{corollary}{Corollary}[section]
\newtheorem{dangtoan}{Dạng toán}[section]
\newtheorem{definition}{Definition}[section]
\newtheorem{dinhly}{Định lý}[section]
\newtheorem{dinhnghia}{Định nghĩa}[section]
\newtheorem{example}{Example}[section]
\newtheorem{ghichu}{Ghi chú}[section]
\newtheorem{hequa}{Hệ quả}[section]
\newtheorem{hypothesis}{Hypothesis}[section]
\newtheorem{lemma}{Lemma}[section]
\newtheorem{luuy}{Lưu ý}[section]
\newtheorem{nhanxet}{Nhận xét}[section]
\newtheorem{notation}{Notation}[section]
\newtheorem{note}{Note}[section]
\newtheorem{principle}{Principle}[section]
\newtheorem{problem}{Problem}[section]
\newtheorem{proposition}{Proposition}[section]
\newtheorem{question}{Question}[section]
\newtheorem{remark}{Remark}[section]
\newtheorem{theorem}{Theorem}[section]
\newtheorem{vidu}{Ví dụ}[section]
\usepackage[left=0.5in,right=0.5in,top=1.5cm,bottom=1.5cm]{geometry}
\usepackage{fancyhdr}
\pagestyle{fancy}
\fancyhf{}
\lhead{\small Sect.~\thesection}
\rhead{\small\nouppercase{\leftmark}}
\renewcommand{\subsectionmark}[1]{\markboth{#1}{}}
\cfoot{\thepage}
\def\labelitemii{$\circ$}

\title{Thermodynamics -- Nhiệt Học}
\author{Nguyễn Quản Bá Hồng\footnote{Independent Researcher, Ben Tre City, Vietnam\\e-mail: \texttt{nguyenquanbahong@gmail.com}; website: \url{https://nqbh.github.io}.}}
\date{\today}

\begin{document}
\maketitle
\begin{abstract}
	\textsc{[en]} This text is a collection of problems, from easy to advanced, about algebraic expression. This text is also a supplementary material for my lecture note on Elementary Mathematics grade 8, which is stored \& downloadable at the following link: \href{https://github.com/NQBH/hobby/blob/master/elementary_physics/grade_8/NQBH_elementary_physics_grade_8.pdf}{GitHub\texttt{/}NQBH\texttt{/}hobby\texttt{/}elementary physics\texttt{/}grade 8\texttt{/}lecture}\footnote{\textsc{url}: \url{https://github.com/NQBH/hobby/blob/master/elementary_physics/grade_8/NQBH_elementary_physics_grade_8.pdf}.}. The latest version of this text has been stored \& downloadable at the following link: \href{https://github.com/NQBH/hobby/blob/master/elementary_physics/grade_8/algebraic_expression/NQBH_algebraic_expression.pdf}{GitHub\texttt{/}NQBH\texttt{/}hobby\texttt{/}elementary physics\texttt{/}grade 8\texttt{/}algebraic expression}\footnote{\textsc{url}: \url{https://github.com/NQBH/hobby/blob/master/elementary_physics/grade_8/algebraic_expression/NQBH_algebraic_expression.pdf}.}.
	\vspace{2mm}
	
	\textsc{[vi]} Tài liệu này là 1 bộ sưu tập các bài tập chọn lọc từ cơ bản đến nâng cao về biểu thức đại số. Tài liệu này là phần bài tập bổ sung cho tài liệu chính -- bài giảng \href{https://github.com/NQBH/hobby/blob/master/elementary_physics/grade_8/NQBH_elementary_physics_grade_8.pdf}{GitHub\texttt{/}NQBH\texttt{/}hobby\texttt{/}elementary physics\texttt{/}grade 8\texttt{/}lecture} của tác giả viết cho Toán Sơ Cấp lớp 8. Phiên bản mới nhất của tài liệu này được lưu trữ \& có thể tải xuống ở link sau: \href{https://github.com/NQBH/hobby/blob/master/elementary_physics/grade_8/algebraic_expression/NQBH_algebraic_expression.pdf}{GitHub\texttt{/}NQBH\texttt{/}hobby\texttt{/}elementary physics\texttt{/}grade 8\texttt{/}algebraic expression}.
\end{abstract}
\tableofcontents
\newpage

%------------------------------------------------------------------------------%

\section{Cấu Tạo của Các Chất}

\begin{baitoan}[\cite{SBT_Vat_Ly_8}, \textbf{19.1.}, p. 50]
	Tại sao quả bóng bay dù được buộc chặt để lâu ngày vẫn bị xẹp? {\sf A.} Vì khi mới thổi, không khí từ miệng vào bóng còn nóng, sau đó lạnh dần nên co lại. {\sf B.} Vì cao su là chất đàn hồi nên sau khi bị thổi căng nó tự động co lại. {\sf C.} Vì không khí nhẹ nên có thể chui qua chỗ buộc ra ngoài. {\sf D.} Vì giữa các phân tử của chất làm vỏ bóng có khoảng cách nên các phân tử không khí có thể qua đó thoát ra ngoài.
\end{baitoan}

\begin{baitoan}[\cite{SBT_Vat_Ly_8}, \textbf{19.2.}, p. 50]
	Khi đổ $50{\rm cm}^3$ rượu vào $50{\rm cm}^3$ nước, ta thu được 1 hỗn hợp rượu -- nước có thể tích: {\sf A.} $= 100{\rm cm}^3$. {\sf B.} $> 100{\rm cm}^3$. {\sf C.} $< 100{\rm cm}^3$. {\sf D.} $\le100{\rm cm}^3$.
\end{baitoan}

\begin{baitoan}[\cite{SBT_Vat_Ly_8}, \textbf{19.3.}, p. 50]
	Mô tả 1 hiện tượng chứng tỏ các chất được cấu tạo từ các hạt riêng biệt, giữa chúng có khoảng cách.
\end{baitoan}

\begin{baitoan}[\cite{SBT_Vat_Ly_8}, \textbf{19.4.}, p. 50]
	Tại sao các chất trông đều có vẻ như liền 1 khối mặc dù chúng đều được cấu tạo từ các hạt riêng biệt?
\end{baitoan}

\begin{baitoan}[\cite{SBT_Vat_Ly_8}, \textbf{19.5.}, p. 50]
	Lấy 1 cốc nước đầy \& 1 thìa con muối tinh. Cho muối dần dần vào nước cho đến khi hết thìa muối ta thấy nước vẫn không tràn ra ngoài. Giải thích.
\end{baitoan}

\begin{baitoan}[\cite{SBT_Vat_Ly_8}, \textbf{19.6.}, p. 50]
	Kích thước của $1$ phân tử hydro vào $\approx0.00000023$\emph{mm}. Tính độ dài của 1 chuỗi gồm $1$ triệu phân tử này đứng nối tiếp nhau.
\end{baitoan}

\begin{baitoan}[\cite{SBT_Vat_Ly_8}, \textbf{19.7.}, p. 51]
	Cách đây $\approx300$ năm, 1 nhà bác học người Ý đã làm thí nghiệm để kiểm tra xem có nén được nước hay không. Ông đổ đầy nước vào 1 bình cầu bằng bạc hàn thật kín rồi lấy búa nện thật mạnh lên bình cầu. Nếu nước nén được thì bình phải bẹp. Nhưng ông đã thu được kết quả bất ngờ. Sau khi nện búa thật mạnh, ông thấy nước thấm qua thành bình ra ngoài trong khi bình vẫn nguyên vẹn. Giải thích.
\end{baitoan}

\begin{baitoan}[\cite{SBT_Vat_Ly_8}, \textbf{19.8.}, p. 51]
	Khi dùng piston né khí trong 1 xi lanh (tiếng Pháp: \emph{cylindre}) kín thì: {\sf A.} kích thước mỗi phân tử khí giảm. {\sf B.} khoảng cách giữa các phân tử khí giảm. {\sf C.} khối lượng mỗi phân tử khí giảm. {\sf D.} số phân tử khí giảm.
\end{baitoan}

\begin{baitoan}[\cite{SBT_Vat_Ly_8}, \textbf{19.9.}, p. 51]
	Khi nhiệt độ của 1 miếng đồng tăng thì: {\sf A.} thể tích của mỗi nguyên tử đồng tăng. {\sf B.} khoảng cách giữa các nguyên tử đồng tăng. {\sf C.} số nguyên tử đồng tăng. {\sf D.} cả 3 đều sai.
\end{baitoan}

\begin{baitoan}[\cite{SBT_Vat_Ly_8}, \textbf{19.10.}, p. 51]
	Biết khối lượng riêng của hơi nước bao giờ cũng nhỏ hơn khối lượng riêng của nước. Hỏi câu này sau đây so sánh các phân tử nước trong hơi nước \& các phân tử nước trong nước là đúng? {\sf A.} Các phân tử trong hơi nước có cùng kích thước với các phân tử trong nước, nhưng khoảng cách giữa các phân tử trong hơi nước lớn hơn. {\sf B.} Các phân tử trong hơi nước có kích thước \& khoảng cách lớn hơn các phân tử trong nước. {\sf C.} Các phân tử trong hơi nước có kích thước \& khoảng cách bằng các phân tử trong nước. {\sf D.} Các phân tử trong hơi nước có cùng kích thước với các phân tử trong nước, nhưng khoảng cách giữa các phân tử trong hơi nước nhỏ hơn.
\end{baitoan}

\begin{baitoan}[\cite{SBT_Vat_Ly_8}, \textbf{19.11.}, p. 51]
	Các nguyên tử trong 1 miếng sắt có tính chất nào sau đây: {\sf A.} Khi nhiệt độ tăng thì nở ra. {\sf B.} Khi nhiệt độ giảm thì co lại. {\sf C.} Đứng rất gần nhau. {\sf D.} Đứng rất xa nhau.
\end{baitoan}

\begin{baitoan}[\cite{SBT_Vat_Ly_8}, \textbf{19.12.}, p. 51]
	Tại sao khi muối dưa, muối có thể thấm vào lá dưa \& cọng dưa?
\end{baitoan}

\begin{baitoan}[\cite{SBT_Vat_Ly_8}, \textbf{19.13.}, p. 51]
	Nếu bơm không khí vào 1 quả bóng bay thì dù có buộc chặt không khí vẫn thoát được ra ngoài, còn nếu bơm không khí vào 1 quả cầu bằng kim loại rồi hàn kín thì hầu như không khí không thể thoát được ra ngoài. Tại sao?
\end{baitoan}

\begin{baitoan}[\cite{SBT_Vat_Ly_8}, \textbf{19.14.}, p. 52]
	Tại sao săm xe đạp sau khi được bơm căng, mặc dù đã vặn van thật chặt, nhưng để lâu ngày vẫn bị xẹp? {\sf A.} Vì lúc bơm, không khí vào săm còn nóng, sau đó không khí nguội dần, co lại, làm săm bị xẹp. {\sf B.} Vì săm xe làm bằng cao su là chất đàn hồi, nên sau khi giãn ra thì tự động co lại làm cho săm để lâu ngày bị xẹp. {\sf C.} Vì giữa các phân tử cao su dùng làm săm có khoảng cách nên các phân tử không khí có thể thoát ra ngoài làm săm xẹp dần. {\sf D.} Vì cao su dùng làm săm đẩy các phân tử không khí lại gần nhau nên săm bị xẹp.
\end{baitoan}

%------------------------------------------------------------------------------%

\section{Nguyên Tử, Phân Tử Chuyển Động\texttt{/}Đứng Yên?}

\begin{baitoan}[\cite{SBT_Vat_Ly_8}, \textbf{20.1.}, p. 53]
	Trong các hiện tượng sau, hiện tượng nào không phải do chuyển động không ngừng của các nguyên tử, phân tử gây ra? {\sf A.} Sự khuếch tán của đồng sunfat vào nước. {\sf B.} Quả bóng bay dù được buộc thật chặt vẫn xẹp dần theo thời gian. {\sf C.} Sự tạo thành gió. {\sf D.} Đường tan vào nước.
\end{baitoan}

\begin{baitoan}[\cite{SBT_Vat_Ly_8}, \textbf{20.2.}, p. 53]
	Khi các nguyên tử, phân tử cấu tạo nên vật chuyển động nhanh lên thì đại lượng nào sau đây tăng lên? {\sf A.} Khối lượng của vật. {\sf B.} Trọng lượng của vật. {\sf C.} Cả khối lượng lẫn trọng lượng của vật. {\sf D.} Nhiệt độ của vật.
\end{baitoan}

\begin{baitoan}[\cite{SBT_Vat_Ly_8}, \textbf{20.3.}, p. 53]
	Tại sao đường tan vào nước nóng nhanh hơn tan vào nước lạnh?
\end{baitoan}

\begin{baitoan}[\cite{SBT_Vat_Ly_8}, \textbf{20.4.}, p. 53]
	Mở lọ nước hoa trong lớp học. Sau vài giây cả lớp đều ngửi thấy mùi nước hoa. Giải thích.
\end{baitoan}

\begin{baitoan}[\cite{SBT_Vat_Ly_8}, \textbf{20.5.}, p. 53]
	Nhỏ 1 giọt mực vào 1 cốc nước. Dù không khuấy cũng chỉ sau 1 thời gian ngắn toàn bộ nước trong cốc đã có màu mực? Tại sao? Nếu tăng nhiệt độ của nước thì hiện tượng trên xảy ra nhanh lên hay chậm đi? Tại sao?
\end{baitoan}

\begin{baitoan}[\cite{SBT_Vat_Ly_8}, \textbf{20.6.}, p. 53]
	Nhúng đầu 1 băng giấy hẹp vào dung dịch phenolphthalein rồi đặt vào 1 ống nghiệm. Đậy ống nghiệm bằng 1 tờ bìa cứng có dán 1 ít bông tẩm dung dịch amoniac. Khoảng nửa phút sau ta thấy đầu dưới của băng giấy ngả sang màu hồng mặc dù hơi amoniac nhẹ hơn không khí. Giải thích.
\end{baitoan}

\begin{baitoan}[\cite{SBT_Vat_Ly_8}, \textbf{20.7.}, p. 53]
	Nguyên tử, phân tử không có tính chất nào sau đây? {\sf A.} Chuyển động không ngừng. {\sf B.} Giữa chúng có khoảng cách. {\sf C.} Nở ra khi nhiệt độ tăng, co lại khi nhiệt độ giảm. {\sf D.} Chuyển động càng nhanh khi nhiệt độ càng cao.
\end{baitoan}

\begin{baitoan}[\cite{SBT_Vat_Ly_8}, \textbf{20.8.}, p. 54]
	Trong thí nghiệm của Brown các hạt phấn hoa chuyển động hỗn độn không ngừng vì: {\sf A.} giữa chúng có khoảng cách. {\sf B.} chúng là các phân tử. {\sf C.} các phân tử nước chuyển động không ngừng, va chạm vào chúng từ mọi phía. {\sf D.} chúng là các thực thể sống.
\end{baitoan}

\begin{baitoan}[\cite{SBT_Vat_Ly_8}, \textbf{20.9.}, p. 54]
	Hiện tượng khuếch tán giữa 2 chất lỏng xác định xảy ra nhanh hay chậm phụ thuộc vào: {\sf A.} nhiệt độ chất lỏng. {\sf B.} khối lượng chất lỏng. {\sf C.} trọng lượng chất lỏng. {\sf D.} thể  tích chất lỏng.
\end{baitoan}

\begin{baitoan}[\cite{SBT_Vat_Ly_8}, \textbf{20.10.}, p. 54]
	Tính chất nào sau đây không phải của phân tử chất khí? {\sf A.} Chuyển động không ngừng. {\sf B.} Chuyển động càng chậm thì nhiệt độ của khí càng thấp. {\sf C.} Chuyển động càng nhanh thì nhiệt độ của khí càng cao. {\sf D.} Chuyển động không hỗn độn.
\end{baitoan}

\begin{baitoan}[\cite{SBT_Vat_Ly_8}, \textbf{20.11.}, p. 54]
	Đối với không khí trong 1 lớp học thì khi nhiệt độ tăng: {\sf A.} kích thước các phân tử không khí tăng. {\sf B.} vận tốc các phân tử không khí tăng. {\sf C.} khối lượng không khí trong phòng tăng. {\sf D.} thể tích không khí trong phòng tăng.
\end{baitoan}

\begin{baitoan}[\cite{SBT_Vat_Ly_8}, \textbf{20.12.}, p. 54]
	Vật rắn có hình dạng xác định vì phân tử cấu tạo nên vật rắn: {\sf A.} không chuyển động. {\sf B.} đứng sát nhau. {\sf C.} chuyển động với vận tốc nhỏ không đáng kể. {\sf D.} chuyển động quanh 1 vị trí xác định.
\end{baitoan}

\begin{baitoan}[\cite{SBT_Vat_Ly_8}, \textbf{20.13.}, pp. 54--55]
	Khi tăng nhiệt độ của khí đựng trong 1 bình khí làm bằng inva (1 chất hầu như không nở vì nhiệt) thì: {\sf A.} khoảng cách giữa các phân tử khí tăng. {\sf B.} khoảng cách giữa các phân tử khí giảm. {\sf C.} vận tốc của các phân tử khí tăng. {\sf D.} vận tốc của các phân tử khí giảm.
\end{baitoan}

\begin{baitoan}[\cite{SBT_Vat_Ly_8}, \textbf{20.14.}, p. 55]
	Hiện tượng khuếch tán xảy ra chỉ vì: {\sf A.} giữa các phân tử có khoảng cách. {\sf B.} các phân tử chuyển động không ngừng. {\sf C.} các phân tử chuyển động không ngừng \& giữa chúng có khoảng cách. {\sf D.} Cả 3 phương án trên đều đúng.
\end{baitoan}

\begin{baitoan}[\cite{SBT_Vat_Ly_8}, \textbf{20.15.}, p. 55]
	Bỏ 1 cục đường phèn vào trong 1 cốc đựng nước. Đường chìm xuống đáy cốc. 1 lúc sau, nếm nước ở trên vẫn thấy ngọt. Tại sao?
\end{baitoan}

\begin{baitoan}[\cite{SBT_Vat_Ly_8}, \textbf{20.16.}, p. 55]
	Người ta mài thật nhẵn bề mặt của 1 miếng đồng \& 1 miếng nhôm rồi ép chặt chúng vào nhau. Sau 1 thời gian, quan sát thấy ở bề mặt của miếng nhôm có đồng, ở bề mặt của miếng đồng có nhôm. Giải thích.
\end{baitoan}

\begin{baitoan}[\cite{SBT_Vat_Ly_8}, \textbf{20.18.}, p. 55]
	Tại sao đun nóng chất khí đựng trong 1 bình kín thì thể tích của chất khí có thể coi như không đổi, còn áp suất chất khí tác dụng lên thành bình lại tăng?
\end{baitoan}

%------------------------------------------------------------------------------%

\section{Nhiệt Năng}

\begin{baitoan}[\cite{SBT_Vat_Ly_8}, \textbf{21.1.}, p. 57]
	Khi chuyển động nhiệt của các phân tử cấu tạo nên vật nhanh lên thì đại lượng nào sau đây của vật không tăng? {\sf A.} Nhiệt độ. {\sf B.} Nhiệt năng. {\sf C.} Khối lượng. {\sf D.} Thể tích.
\end{baitoan}

\begin{baitoan}[\cite{SBT_Vat_Ly_8}, \textbf{21.2.}, p. 57]
	Nhỏ 1 giọt nước đang sôi vào 1 cốc đựng nước ấm thì nhiệt năng của giọt nước \& của nước trong cốc thay đổi như thế nào? {\sf A.} Nhiệt năng của giọt nước tăng, của nước trong cốc giảm. {\sf B.} Nhiệt năng của giọt nước giảm, của nước trong cốc tăng. {\sf C.} Nhiệt năng của giọt nước \& của nước trong cốc đều giảm. {\sf D.} Nhiệt năng của giọt nước \& của nước trong cốc đều tăng.
\end{baitoan}

\begin{baitoan}[\cite{SBT_Vat_Ly_8}, \textbf{21.3.}, p. 57]
	1 viên đạn đang bay trên cao có những dạng năng lượng nào?
\end{baitoan}

\begin{baitoan}[\cite{SBT_Vat_Ly_8}, \textbf{21.4.}, p. 57]
	Đun nóng 1 ống nghiệm nút kín có đựng nước. Nước trong ống nghiệm nóng dần, tới 1 lúc nào đó hơi nước trong ống làm bật nút lên. Trong thí nghiệm trên, khi nào thì có truyền nhiệt, khi nào thì có thực hiện công? 
\end{baitoan}

\begin{baitoan}[\cite{SBT_Vat_Ly_8}, \textbf{21.5.}, p. 57]
	Khi để bầu nhiệt kế vào luồng khí phun mạnh ra từ 1 quả bóng thì mực thủy ngân trong nhiệt kế dâng lên hay tụt xuống? Tại sao?
\end{baitoan}

\begin{baitoan}[\cite{SBT_Vat_Ly_8}, \textbf{21.6.}, p. 57]
	1 chai thủy tinh được đậy kín bằng 1 nút cao su nối với 1 bơm tay. Khi bơm không khí vào chai, ta thấy tới 1 lúc nào đó nút cao su bật ra, đồng thời trong chai xuất hiện sương mù do những giọt nước rất nhỏ tạo thành. Giải thích.
\end{baitoan}

\begin{baitoan}[\cite{SBT_Vat_Ly_8}, \textbf{21.7.}, p. 58]
	\emph{Đ\texttt{/}S?} {\sf A.} Nhiệt năng của 1 vật là 1 dạng năng lượng. {\sf B.} Nhiệt năng của 1 vật là tổng động năng \& thế năng của vật. {\sf C.} Nhiệt năng của 1 vật là năng lượng vật lúc nào cũng có. {\sf D.} Nhiệt năng của 1 vật là tổng động năng của các phân tử cấu tạo nên vật.
\end{baitoan}

\begin{baitoan}[\cite{SBT_Vat_Ly_8}, \textbf{21.8.}, p. 58]
	Nhiệt lượng là: {\sf A.} 1 dạng năng lượng có đơn vị là jun. {\sf B.} đại lượng chỉ xuất hiện trong sự thực hiện công. {\sf C.} phần nhiệt năng mà vật nhận thêm hay mất bớt trong sự truyền nhiệt. {\sf D.} đại lượng tăng khi nhiệt độ của vật tăng, giảm khi nhiệt độ của vật giảm.
\end{baitoan}

\begin{baitoan}[\cite{SBT_Vat_Ly_8}, \textbf{21.9.}, p. 58]
	Nhiệt năng của 1 vật: {\sf A.} chỉ có thể thay đổi bằng truyền nhiệt. {\sf B.} chỉ có thể thay đổi bằng thực hiện công. {\sf C.} chỉ có thể thay đổi bằng cả thực hiện công \& truyền nhiệt. {\sf D.} có thể thay đổi bằng thực hiện công hoặc truyền nhiệt, hoặc bằng cả thực hiện công \& truyền nhiệt.
\end{baitoan}

\begin{baitoan}[\cite{SBT_Vat_Ly_8}, \textbf{21.10.}, p. 58]
	Các nguyên tử, phân tử cấu tạo nên vật chuyển động càng nhanh thì: {\sf A.} động năng của vật càng lớn. {\sf B.} thế năng của vật càng lớn. {\sf C.} cơ năng của vật càng lớn. {\sf D.} nhiệt năng của vật càng lớn.
\end{baitoan}

\begin{baitoan}[\cite{SBT_Vat_Ly_8}, \textbf{21.11.}, p. 58]
	Nhiệt năng của vật tăng khi: {\sf A.} vật truyền nhiệt cho vật khác. {\sf B.} vật thực hiện công lên vật khác. {\sf C.} chuyển động nhiệt của các phần tử cấu tạo nên vật nhanh lên. {\sf D.} chuyển động của vật nhanh lên.
\end{baitoan}

\begin{baitoan}[\cite{SBT_Vat_Ly_8}, \textbf{21.12.}, pp. 58--59]
	Đại lượng nào sau đây của vật rắn không thay đổi, khi chuyển động nhiệt của các phân tử cấu tạo nên vật thay đổi? {\sf A.} Nhiệt độ của vật. {\sf B.} Khối lượng của vật. {\sf C.} Nhiệt năng của vật. {\sf D.} Thể tích của vật.
\end{baitoan}

\begin{baitoan}[\cite{SBT_Vat_Ly_8}, \textbf{21.13.}, p. 59]
	Người ta có thể nhận ra sự thay đổi nhiệt năng của 1 vật rắn dựa vào sự thay đổi: {\sf A.} khối lượng của vật. {\sf B.} khối lượng riêng của vật. {\sf C.} nhiệt độ của vật. {\sf D.} vận tốc của các phần tử cấu tạo nên vật.
\end{baitoan}

\begin{baitoan}[\cite{SBT_Vat_Ly_8}, \textbf{21.14.}, p. 59]
	Ở giữa 1 ống thủy tinh được hàn kín 2 đầu có 1 giọt thủy ngân. Dùn đèn cồn hơ nóng nửa ống bên phải thì giọt thủy ngân dịch chuyển vế phía bên trái ống. Cho biết nhiệt năng của khí trong nửa ống bên phải đã thay đổi bằng những quá trình nào?
\end{baitoan}

\begin{baitoan}[\cite{SBT_Vat_Ly_8}, \textbf{21.15.}, p. 59]
	Giải thích sự thay đổi nhiệt năng trong các trường hợp sau: (a) Khi đun nước, nước nóng lên. (b) Khi cưa, cả lưỡi cưa \& gỗ đều nóng lên. (c) Khi tiếp tục đun nước đang sôi, nhiệt độ của nước không tăng.
\end{baitoan}

\begin{baitoan}[\cite{SBT_Vat_Ly_8}, \textbf{21.16.}, p. 59]
	Gạo đang nấu trong nồi \& gạo đang xát đều nóng lên. Hỏi về mặt thay đổi nhiệt năng thì có gì giống nhau, khác nhau trong 2 hiện tượng trên?
\end{baitoan}

\begin{baitoan}[\cite{SBT_Vat_Ly_8}, \textbf{21.17.}, p. 59]
	So sánh 2 quá trình thực hiện công \& truyền nhiệt.
\end{baitoan}

\begin{baitoan}[\cite{SBT_Vat_Ly_8}, \textbf{21.18.}, p. 59]
	1 học sinh nói: ``1 giọt nước ở nhiệt độ $60^\circ{\rm C}$ có nhiệt năng lớn hơn nước trong 1 cốc nước ở nhiệt độ $30^\circ{\rm C}$''. \emph{Đ\texttt{/}S?} Tại sao? Phải nói thế nào mới đúng?
\end{baitoan}

\begin{baitoan}[\cite{SBT_Vat_Ly_8}, \textbf{21.19.}, p. 59]
	Ở giữa 1 ống thủy tinh được hàn kín có 1 giọt thủy ngân. Người ta quay lộn ngược ống nhiều lần. Hỏi nhiệt độ của giọt thủy ngân có tăng lên hay không? Tại sao?
\end{baitoan}

%------------------------------------------------------------------------------%

\section{Dẫn Nhiệt}

\begin{baitoan}[\cite{SBT_Vat_Ly_8}, \textbf{22.1.}, p. 60]
	Trong các cách sắp xếp vật liệu dẫn nhiệt từ tốt đến kém sau, cách nào đúng? {\sf A.} Đồng, nước, thủy tinh, không khí. {\sf B.} Đồng, thủy tinh, nước, không khí. {\sf C.} Thủy tinh, đồng, nước, không khí. {\sf D.} Không khí, nước, thủy tinh, đồng.
\end{baitoan}

\begin{baitoan}[\cite{SBT_Vat_Ly_8}, \textbf{22.2.}, p. 60]
	Trong sự dẫn nhiệt, nhiệt tự truyền: {\sf A.} từ vật có nhiệt năng lớn hơn sang vật có nhiệt năng nhỏ hơn. {\sf B.} từ vật có khối lượng lớn hơn sang vật có khối lượng nhỏ hơn. {\sf C.} từ vật có nhiệt độ cao hơn sang vật có nhiệt độ thấp hơn. {\sf D.} Cả 3 đều đúng.
\end{baitoan}

\begin{baitoan}[\cite{SBT_Vat_Ly_8}, \textbf{22.3.}, p. 60]
	Tại sao khi rót nước sôi vào cốc thủy tinh thì cốc dày dễ bị vỡ hơn cốc mỏng? Muốn cốc khỏi bị vỡ khi rót nước sôi vào thì làm thế nào?
\end{baitoan}

\begin{baitoan}[\cite{SBT_Vat_Ly_8}, \textbf{22.4.}, p. 60]
	Đun nước bằng ấm nhôm \& bằng ấm đất trên cùng 1 bếp lửa thì nước trong ấm nào sẽ chóng sôi hơn?
\end{baitoan}

\begin{baitoan}[\cite{SBT_Vat_Ly_8}, \textbf{22.5.}, p. 60]
	Tại sao về mùa lạnh khi sờ vào miếng đồng ta cảm thấy lạnh hơn khi sờ vào miếng gỗ? Có phải vì nhiệt độ của đồng thấp hơn của gỗ không?
\end{baitoan}

\begin{baitoan}[\cite{SBT_Vat_Ly_8}, \textbf{22.6.}, p. 60]
	1 hòn bi chuyển động nhanh va chạm vào 1 hòn bi chuyển động chậm hơn sẽ truyền 1 phần động năng của nó cho hòn bi này \& chuyển động chậm đi trong khi hòn bi chuyển động chậm hơn sẽ chuyển động nhanh lên. Hiện tượng này tương tự như hiện tượng truyền nhiệt năng giữa các phân tử trong sự dẫn nhiệt. Dùng sự tương tự này để giải thích hiện tượng xảy ra khi thả 1 miếng đồng được nung nóng vào 1 cốc nước lạnh.
\end{baitoan}

\begin{baitoan}[\cite{SBT_Vat_Ly_8}, \textbf{22.7.}, p. 60]
	Dẫn nhiệt là hình thức truyền nhiệt chủ yếu của: {\sf A.} chất rắn. {\sf B.} chất khí \& chất lỏng. {\sf C.} chất khí. {\sf D.} chất lỏng.
\end{baitoan}

\begin{baitoan}[\cite{SBT_Vat_Ly_8}, \textbf{22.8.}, pp. 60--61]
	Bản chất của sự dẫn nhiệt là: {\sf A.} sự truyền nhiệt độ từ vật này đến vật khác. {\sf B.} sự truyền nhiệt năng từ vật này đến vật khác. {\sf C.} sự thực hiện công từ vật này lên vật khác. {\sf D.} sự truyền động năng của các nguyên tử, phân tử này sang các nguyên tử, phân tử khác.
\end{baitoan}

\begin{baitoan}[\cite{SBT_Vat_Ly_8}, \textbf{22.9.}, p. 61]
	Sự dẫn nhiệt chỉ có thể xảy ra giữa 2 vật rắn khi: {\sf A.} 2 vật có nhiệt năng khác nhau. {\sf B.} 2 vật có nhiệt năng khác nhau, tiếp xúc nhau. {\sf C.} 2 vật có nhiệt độ khác nhau. {\sf D.} 2 vật có nhiệt độ khác nhau, tiếp xúc nhau.
\end{baitoan}

\begin{baitoan}[\cite{SBT_Vat_Ly_8}, \textbf{22.10.}, p. 61]
	Để giữ nước đá lâu chảy, người ta thường để nước đá vào các hộp xốp kín vì: {\sf A.} hộp xốp kín nên dẫn nhiệt kém. {\sf B.} trong xốp có các khoảng không khí nên dẫn nhiệt kém. {\sf C.} trong xốp có các khoảng chân không nên dẫn nhiệt kém. {\sf D.} Vì cả 3 lý do trên.
\end{baitoan}

\begin{baitoan}[\cite{SBT_Vat_Ly_8}, \textbf{22.11.}, p. 61]
	Về mùa hè ở 1 số nước châu Phi rất nóng, người ta thường mặc quần áo trùm kín cả người; còn ở nước ta về mùa hè người ta lại thường mặc quần áo ngắn. Tại sao?
\end{baitoan}

\begin{baitoan}[\cite{SBT_Vat_Ly_8}, \textbf{22.12.}, p. 61]
	Tại sao vào mùa hè, không khí trong nhà mái tôn nóng hơn trong nhà mái tranh; còn về mùa đông, không khí trong nhà mái tôn lại lạnh hơn trong nhà mái tranh.
\end{baitoan}

\begin{baitoan}[\cite{SBT_Vat_Ly_8}, \textbf{22.13.}, p. 61]
	Tại sao muốn giữ cho nước chè nóng lâu, người ta thường để ấm vào giỏ có chèn bông, trấu hoặc mùn cưa?
\end{baitoan}

\begin{baitoan}[\cite{SBT_Vat_Ly_8}, \textbf{22.14.}, p. 61]
	Thiết kế 1 thí nghiệm dùng để so sánh độ dẫn nhiệt của cát \& của mùn cưa với các dụng cụ sau đây: cát, mùn cưa, 2 ống nghiệm, 2 nhiệt kế, 1 cốc đựng nước nóng.
\end{baitoan}

\begin{baitoan}[\cite{SBT_Vat_Ly_8}, \textbf{22.15.}, p. 61]
	Có 2 ấm đun nước khối lượng bằng nhau, 1 làm bằng nhôm, 1 bằng đồng. (a) Nếu đun cùng 1 lượng nước bằng 2 ấm này trên những bếp tỏa nhiệt như nhau thì nước ở ấm nào sôi trước? Tại sao? (b) Nếu sau khi nước sôi, ta tắt lửa đi, thì nước ở ấm nào nguội nhanh hơn? Tại sao?
\end{baitoan}

%------------------------------------------------------------------------------%

\section{Đối Lưu -- Bức Xạ Nhiệt}

%------------------------------------------------------------------------------%

\section{Công Thức Tính Nhiệt Lượng}

%------------------------------------------------------------------------------%

\section{Phương Trình Cân Bằng Nhiệt}

%------------------------------------------------------------------------------%

\section{Năng Suất Tỏa Nhiệt của Nhiên Liệu}

%------------------------------------------------------------------------------%

\section{Sự Bảo Toàn Năng Lượng Trong Các Hiện Tượng Cơ \& Nhiệt}

%------------------------------------------------------------------------------%

\section{Động Cơ Nhiệt}

%------------------------------------------------------------------------------%

\section{Miscellaneous}

%------------------------------------------------------------------------------%

\printbibliography[heading=bibintoc]
	
\end{document}