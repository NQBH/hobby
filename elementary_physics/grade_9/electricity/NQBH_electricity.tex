\documentclass{article}
\usepackage[backend=biber,natbib=true,style=authoryear,maxbibnames=10]{biblatex}
\addbibresource{/home/nqbh/reference/bib.bib}
\usepackage[utf8]{vietnam}
\usepackage{tocloft}
\renewcommand{\cftsecleader}{\cftdotfill{\cftdotsep}}
\usepackage[colorlinks=true,linkcolor=blue,urlcolor=red,citecolor=magenta]{hyperref}
\usepackage{amsmath,amssymb,amsthm,float,graphicx,mathtools,tikz,tipa}
\usepackage[version=4]{mhchem}
\allowdisplaybreaks
\newtheorem{assumption}{Assumption}
\newtheorem{baitoan}{Bài toán}
\newtheorem{cauhoi}{Câu hỏi}
\newtheorem{conjecture}{Conjecture}
\newtheorem{corollary}{Corollary}
\newtheorem{dangtoan}{Dạng toán}
\newtheorem{definition}{Definition}
\newtheorem{dinhly}{Định lý}
\newtheorem{dinhnghia}{Định nghĩa}
\newtheorem{example}{Example}
\newtheorem{ghichu}{Ghi chú}
\newtheorem{hequa}{Hệ quả}
\newtheorem{hypothesis}{Hypothesis}
\newtheorem{lemma}{Lemma}
\newtheorem{luuy}{Lưu ý}
\newtheorem{nhanxet}{Nhận xét}
\newtheorem{notation}{Notation}
\newtheorem{note}{Note}
\newtheorem{principle}{Principle}
\newtheorem{problem}{Problem}
\newtheorem{proposition}{Proposition}
\newtheorem{question}{Question}
\newtheorem{remark}{Remark}
\newtheorem{theorem}{Theorem}
\newtheorem{vidu}{Ví dụ}
\usepackage[left=1cm,right=1cm,top=5mm,bottom=5mm,footskip=4mm]{geometry}

\title{Electricity -- Điện Học}
\author{Nguyễn Quản Bá Hồng\footnote{Independent Researcher, Ben Tre City, Vietnam\\e-mail: \texttt{nguyenquanbahong@gmail.com}; website: \url{https://nqbh.github.io}.}}
\date{\today}

\begin{document}
\maketitle
\begin{abstract}
	\textsc{[en]} This text is a collection of problems, from easy to advanced, about electricity. This text is also a supplementary material for my lecture note on Elementary Physics, which is stored \& downloadable at the following link: \href{https://github.com/NQBH/hobby/blob/master/elementary_physics/grade_9/NQBH_elementary_physics_grade_9.pdf}{GitHub\texttt{/}NQBH\texttt{/}hobby\texttt{/}elementary physics\texttt{/}grade 9\texttt{/}lecture}\footnote{\textsc{url}: \url{https://github.com/NQBH/hobby/blob/master/elementary_physics/grade_9/NQBH_elementary_physics_grade_9.pdf}.}. The latest version of this text has been stored \& downloadable at the following link:\\\href{https://github.com/NQBH/hobby/blob/master/elementary_physics/electricity/NQBH_electricity.pdf}{GitHub\texttt{/}NQBH\texttt{/}hobby\texttt{/}elementary physics\texttt{/}grade 9\texttt{/}electricity}\footnote{\textsc{url}: \url{https://github.com/NQBH/hobby/blob/master/elementary_physics/electricity/NQBH_electricity.pdf}.}.
	\vspace{2mm}
	
	\textsc{[vi]} Tài liệu này là 1 bộ sưu tập các bài tập chọn lọc từ cơ bản đến nâng cao về nguyên tử, nguyên tố hóa học, \& hợp chất hóa học. Tài liệu này là phần bài tập bổ sung cho tài liệu chính -- bài giảng \href{https://github.com/NQBH/hobby/blob/master/elementary_physics/grade_9/NQBH_elementary_physics_grade_9.pdf}{GitHub\texttt{/}NQBH\texttt{/}hobby\texttt{/}elementary physics\texttt{/}grade 9\texttt{/}lecture} của tác giả viết cho Vật Lý Sơ Cấp. Phiên bản mới nhất của tài liệu này được lưu trữ \& có thể tải xuống ở link sau: \href{https://github.com/NQBH/hobby/blob/master/elementary_physics/grade_9/real/NQBH_real.pdf}{GitHub\texttt{/}NQBH\texttt{/}hobby\texttt{/}elementary physics\texttt{/}grade 9\texttt{/}electricity}.
\end{abstract}
\setcounter{secnumdepth}{4}
\setcounter{tocdepth}{3}
\tableofcontents
\newpage

%------------------------------------------------------------------------------%

\noindent\fbox{%
	\parbox{\textwidth}{%
		\noindent\textsf{\textbf{Kiến thức cốt lõi.}} \fbox{\bf 1} \textit{Định luật Ohm cho đoạn mạch}: $I = \frac{U}{R}$ với $I$: cường độ dòng điện (A), $U$: hiệu điện thế (V), $R$: điện thế (W). \fbox{\bf 2} Công thức điện trở: $r = \rho\frac{l}{S}$ với $l$: \textit{chiều dài} dây dẫn (m), $S$: tiết diện dây dẫn ($\rm m^2$), $r$: điện trở suất (Wm). \fbox{\bf 3} \textit{Định luật Ohm cho đoạn mạch có các điện trở mắc nối tiếp}: Cường độ dòng điện trong đoạn mạch nối tiếp: $I = I_i$, $\forall i = 1,2,\ldots,n$, i.e., $I = I_1 = I_2 = \cdots = I_n$. Hiệu điện thế trong đoạn mạch nối tiếp: $U = \sum_{i=1}^n U_i = U_1 + U_2 + \cdots + U_n$. Điện trở toàn phần\texttt{/}tương đương của đoạn mạch nối tiếp: $R = \sum_{i=1}^n R_i = R_1 + R_2 + \cdots + R_n$. \fbox{\bf 4} \textit{Định luật Ohm cho đoạn mạch có các điện trở mắc song song}: Cường độ dòng điện trong mạch chính bằng tổng các cường độ dòng điện trong các đoạn mạch rẽ: $I = \sum_{i=1}^n I_i = I_1 + I_2 + \cdots + I_n$. Hiệu điện thế của đoạn mạch song song bằng hiệu điện thế của mỗi đoạn mạch rẽ: $U = U_i$, $\forall i = 1,2,\ldots,n$, i.e., $U = U_1 = U_2 = \cdots = U_n$. Điện trở tương đương của đoạn mạch song song: $\frac{1}{R} = \sum_{i=1}^n \frac{1}{R_i} = \frac{1}{R_1} + \frac{1}{R_2} + \cdots + \frac{1}{R_n}$. Nếu chỉ có 2 điện trở $R_1,R_2$ mắc song song: $\frac{1}{R} = \frac{1}{R_1} + \frac{1}{R_2}$ hay $R = \frac{R_1R_2}{R_1 + R_2}$. \fbox{\bf 5} \textit{Điện năng, công, \& công suất của dòng điện}: Công của dòng điện: $A = UIt$. Trong đoạn mạch chỉ có điện trở: $A = UIt = RI^2t = \frac{U^2}{R}t$. \textit{Công suất} có số đo bằng công thực hiện được trong 1 s: $P = \frac{A}{t} = UI$. Trong đoạn mạch chỉ có điện trở: $P = UI = RI^2 = \frac{U^2}{R}$. \textit{Định luật Joule--Lenz}: $Q = UIt = \frac{U^2}{R}t = RI^2t$, $P = UI = \frac{U^2}{R} = RI^2$. Khi có cân bằng nhiệt thì $Q_{\tiny\mbox{tỏa}} = Q_{\tiny\mbox{thu}}$ với $Q_{\tiny\mbox{thu}}$ có thể tính $Q_{\tiny\mbox{thu}} = mc(t_2 - t_1)$ \& $Q_{\tiny\mbox{tỏa}}$ tính theo định luật Joule--Lenz. \textit{Hiệu suất sử dụng} là: $H = \frac{Q_{\tiny\mbox{hữu ích}}}{Q_{\tiny\mbox{toàn phần}}}\cdot100\%$ hay $H = \frac{P_{\tiny\mbox{hữu ích}}}{P_{\tiny\mbox{toàn phần}}}\cdot100\%$.
	}%
}

\section{Miscellaneous}

\begin{baitoan}[\cite{Van_500_BT_Vat_Ly_THCS}, 4.1., p. 138]
	1 vật A mang điện tích hút 1 quả cầu kim loại nhỏ treo bằng sợi tơ. Từ đó ta có thể suy ra quả cầu kim loại mang điện tích âm được không? (a) Với điều kiện nào thì quả cầu \& vật tích điện cùng dấu lại hút nhau? (b) Với điều kiện nào thì quả cầu \& vật tích điện trái dấu lại hút nhau?
\end{baitoan}

\begin{baitoan}[\cite{Van_500_BT_Vat_Ly_THCS}, 4.2., p. 138]
	Đặt 1 quả cầu trung hòa điện được treo bằng dây tơ mảnh vào chính giữa 2 bản kim loại tích điện trái dấu nhau. Biết quả cầu không thể chạm các bản. Quả cầu có đứng yên không nếu: (a) 2 bản có điện tích bằng nhau? (b) 1 bản có điện tích lớn hơn?
\end{baitoan}

\begin{baitoan}[\cite{Van_500_BT_Vat_Ly_THCS}, 4.3., p. 138]
	Dùng hình vẽ để giải thích tại sao 2 quả cầu kim loại nhiễm điện trái dấu lại hút nhau bằng 1 lực lớn hơn lực đẩy khi chúng được nhiễm điện cùng dấu, trong cùng những điều kiện như nhau về vị trí \& độ lớn của các quả cầu.
\end{baitoan}

\begin{baitoan}[\cite{Van_500_BT_Vat_Ly_THCS}, 4.4., p. 138]
	Treo 2 quả cầu nhỏ bằng nhau trên 2 sợi tơ mảnh. 1 quả mang điện, còn quả kia không mang điện. Trong tay bạn không có 1 vật dụng gì, bạn có thể xác định được quả cầu nào mang điện không? Giải thích.
\end{baitoan}

\begin{baitoan}[\cite{Van_500_BT_Vat_Ly_THCS}, 4.6., p. 138]
	Người ta đặt nhẹ 1 cái kim khâu sao cho nó nổi trong 1 cốc nước. Cái kim sẽ dịch chuyển như thế nào nếu ta đưa đũa êbônit đã nhiễm điện tới gần nó.
\end{baitoan}

\begin{baitoan}[\cite{Van_500_BT_Vat_Ly_THCS}, 4.7., pp. 138--139, TS PTNK ĐHQG TpHCM 1999]
	(a) Trong thí nghiệm thứ nhất người ta cho vật A nhiễm điện chạm vào quả cầu của điện nghiệm B sau đó đưa A ra xa. Trong thí nghiệm thứ 2 người ta cho vật C nhiễm điện lại gần quả cầu của điện nghiệm D sau đó đưa C ra xa. Mô tả hiện tượng xảy ra trong 2 thí nghiệm \& giải thích vì sao có sự khác nhau trong 2 lần thí nghiệm đó. (b) Cho 2 quả cầu kim loại có đế cách điện: quả A nhiễm điện, quả B không nhiễm điện. Trình bày cách làm cho 2 lá nhôm của điện nghiệm C xòe ra, không cụp lại khi đưa A \& B ra xa C mà điện tích của A vẫn không bị giảm.
\end{baitoan}

\begin{baitoan}[\cite{Van_500_BT_Vat_Ly_THCS}, 4.9., p. 139]
	Có $3$ bóng đèn $\rm\mbox{Đ}_1,\mbox{Đ}_2,\mbox{Đ}_3$ cùng loại, 1 số dây dẫn điện, 1 nguồn điện, \& 1 khóa k. Vẽ các sơ đồ mạch điện thỏa mãn 2 điều kiện: (a) k đóng, 3 đèn đều sáng. (b) k mở, chỉ có $2$ đèn $\rm\mbox{Đ}_1,\mbox{Đ}_2$ sáng, đèn $\rm\mbox{Đ}_3$ không sáng.
\end{baitoan}

\begin{baitoan}[\cite{Van_500_BT_Vat_Ly_THCS}, 4.10., p. 139]
	Có $3$ bóng đèn $\rm\mbox{Đ}_1,\mbox{Đ}_2,\mbox{Đ}_3$, 1 số dây dẫn điện \& $1$ nguồn điện. Vẽ các sơ đồ mạch điện mà khi tháo bớt $1$ bóng đèn ra thì $2$ bóng còn lại vẫn có thể sáng. Chỉ rõ bóng nào được tháo ra trong từng sơ đồ.
\end{baitoan}

\begin{baitoan}[\cite{Van_500_BT_Vat_Ly_THCS}, 4.11., p. 139]
	Trong thời gian $2$ phút, số electron tự do đã dịch chuyển qua tiết diện thẳng của vật dẫn là $37.5\cdot10^{19}$ electron. Hỏi: (a) Điện lượng đã chuyển qua tiết diện thẳng của vật dẫn trên bằng bao nhiêu? (b) Cường độ dòng điện qua vật dẫn bằng bao nhiêu? (c) Để cường độ dòng điện qua vật dẫn tăng gấp đôi thì trong thời gian $3$ phút, điện lượng chuyển qua vật dẫn bằng bao nhiêu?
\end{baitoan}

\begin{baitoan}[\cite{Van_500_BT_Vat_Ly_THCS}, 4.12., p. 139]
	Trong phân nửa thời gian, điện lượng chuyển qua tiết diện thẳng của đoạn mạch thứ nhất bằng $\frac{2}{3}$ điện lượng chuyển qua tiết diện thẳng của đoạn mạch thứ 2. Tính điện lượng chuyển qua tiết diện thẳng của đoạn mạch thứ 2 trong thời gian $5$ phút. Biết cường độ dòng điện qua đoạn mạch thứ nhất là $\frac{4}{3}$ \emph{A}.
\end{baitoan}

\begin{baitoan}[\cite{Van_500_BT_Vat_Ly_THCS}, 4.13., p. 139]
	1 dây dẫn dài \emph{100 m}, tiết diện $\rm0.28\ mm^2$ đặt giữa 2 điểm có hiệu điện thế là \emph{12 V} thì cường độ dòng điện qua dây dẫn là \emph{1.2 A}. Hỏi nếu thay dây dẫn trên bằng 1 dây dẫn khác cùng chất với dây dẫn trên, dài \emph{25 m}, điện trở $2.8\ \Omega$ thì dây dẫn này có tiết diện là bao nhiêu? Cường độ dòng điện qua nó là bao nhiêu?
\end{baitoan}

\begin{baitoan}[\cite{Van_500_BT_Vat_Ly_THCS}, 4.14., p. 139]
	1 dây đồng có điện trở $R$. Kéo giãn đều cho độ dài của dây tăng lên gấp đôi (nhưng thể tích của dây không đổi). Hỏi điện trở của dây sau khi được kéo?
\end{baitoan}

\begin{baitoan}[\cite{Van_500_BT_Vat_Ly_THCS}, 4.15., p. 139]
	1 dây dẫn bằng đồng, dài \emph{1 km}, tiết diện đều, có điện trở là $2\ \Omega$. Tính khối lượng của đồng dùng làm dây dẫn này. Biết điện trở suất \& khối lượng riêng của đồng là $\rm1.7\cdot10^{-8}\ \Omega m$ \& $8.9\cdot10^3$ \emph{kg\texttt{/}$\rm m^3$}.
\end{baitoan}

%------------------------------------------------------------------------------%

\printbibliography[heading=bibintoc]
	
\end{document}