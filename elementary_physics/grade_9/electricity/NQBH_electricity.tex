\documentclass{article}
\usepackage[backend=biber,natbib=true,style=alphabetic,maxbibnames=50]{biblatex}
\addbibresource{/home/nqbh/reference/bib.bib}
\usepackage[utf8]{vietnam}
\usepackage{tocloft}
\renewcommand{\cftsecleader}{\cftdotfill{\cftdotsep}}
\usepackage[colorlinks=true,linkcolor=blue,urlcolor=red,citecolor=magenta]{hyperref}
\usepackage{amsmath,amssymb,amsthm,float,graphicx,mathtools,tikz,tipa}
\usepackage[version=4]{mhchem}
\allowdisplaybreaks
\newtheorem{assumption}{Assumption}
\newtheorem{baitoan}{Bài toán}
\newtheorem{cauhoi}{Câu hỏi}
\newtheorem{conjecture}{Conjecture}
\newtheorem{corollary}{Corollary}
\newtheorem{dangtoan}{Dạng toán}
\newtheorem{definition}{Definition}
\newtheorem{dinhly}{Định lý}
\newtheorem{dinhnghia}{Định nghĩa}
\newtheorem{example}{Example}
\newtheorem{ghichu}{Ghi chú}
\newtheorem{hequa}{Hệ quả}
\newtheorem{hypothesis}{Hypothesis}
\newtheorem{lemma}{Lemma}
\newtheorem{luuy}{Lưu ý}
\newtheorem{nhanxet}{Nhận xét}
\newtheorem{notation}{Notation}
\newtheorem{note}{Note}
\newtheorem{principle}{Principle}
\newtheorem{problem}{Problem}
\newtheorem{proposition}{Proposition}
\newtheorem{question}{Question}
\newtheorem{remark}{Remark}
\newtheorem{theorem}{Theorem}
\newtheorem{vidu}{Ví dụ}
\usepackage[left=1cm,right=1cm,top=5mm,bottom=5mm,footskip=4mm]{geometry}

\title{Electricity -- Điện Học}
\author{Nguyễn Quản Bá Hồng\footnote{Independent Researcher, Ben Tre City, Vietnam\\e-mail: \texttt{nguyenquanbahong@gmail.com}; website: \url{https://nqbh.github.io}.}}
\date{\today}

\begin{document}
\maketitle
\begin{abstract}
	\textsf{[en]} This text is a collection of problems, from easy to advanced, about electricity. This text is also a supplementary material for my lecture note on Elementary Physics, which is stored \& downloadable at the following link: \href{https://github.com/NQBH/hobby/blob/master/elementary_physics/grade_9/NQBH_elementary_physics_grade_9.pdf}{GitHub\texttt{/}NQBH\texttt{/}hobby\texttt{/}elementary physics\texttt{/}grade 9\texttt{/}lecture}\footnote{\textsc{url}: \url{https://github.com/NQBH/hobby/blob/master/elementary_physics/grade_9/NQBH_elementary_physics_grade_9.pdf}.}. The latest version of this text has been stored \& downloadable at the following link:\\\href{https://github.com/NQBH/hobby/blob/master/elementary_physics/electricity/NQBH_electricity.pdf}{GitHub\texttt{/}NQBH\texttt{/}hobby\texttt{/}elementary physics\texttt{/}grade 9\texttt{/}electricity}\footnote{\textsc{url}: \url{https://github.com/NQBH/hobby/blob/master/elementary_physics/electricity/NQBH_electricity.pdf}.}.
	\vspace{2mm}
	
	\textsf{[vi]} Tài liệu này là 1 bộ sưu tập các bài tập chọn lọc từ cơ bản đến nâng cao về nguyên tử, nguyên tố hóa học, \& hợp chất hóa học. Tài liệu này là phần bài tập bổ sung cho tài liệu chính -- bài giảng \href{https://github.com/NQBH/hobby/blob/master/elementary_physics/grade_9/NQBH_elementary_physics_grade_9.pdf}{GitHub\texttt{/}NQBH\texttt{/}hobby\texttt{/}elementary physics\texttt{/}grade 9\texttt{/}lecture} của tác giả viết cho Vật Lý Sơ Cấp. Phiên bản mới nhất của tài liệu này được lưu trữ \& có thể tải xuống ở link sau: \href{https://github.com/NQBH/hobby/blob/master/elementary_physics/grade_9/real/NQBH_real.pdf}{GitHub\texttt{/}NQBH\texttt{/}hobby\texttt{/}elementary physics\texttt{/}grade 9\texttt{/}electricity}.
\end{abstract}
\setcounter{secnumdepth}{4}
\setcounter{tocdepth}{3}
\tableofcontents

%------------------------------------------------------------------------------%

\section{Lý Thuyết Dòng Điện Không Đổi}
See, e.g., \cite[Chap. 1]{SGK_Vat_Ly_9}, \cite[Chủ đề I, pp. 5--8]{Hoe_Vat_Ly_9}.

\subsection{Cường độ dòng điện}

\begin{dinhnghia}[Cường độ dòng điện]
	Nếu trong thời gian $t$, có 1 lượng điện tích $q$ chuyển qua tiết diện của dây dẫn thì đại lượng $I = \dfrac{q}{t}$ được gọi là \emph{cường độ dòng điện}.
\end{dinhnghia}
Đơn vị của cường độ dòng điện: Khi $q$ đo bằng đơn vị coulomb (C), $t$ đo bằng đơn vị giây (s), thì đơn vị của cường độ dòng điện là ampere (A). Trong chương trình Vật lý Trung học cơ sở, chỉ xét dòng điện có cường độ $I =$ const không đổi theo thời gian.

\subsection{Hiệu điện thế. Điện thế}
Hiệu điện thế giữa 2 điểm $A$ \& $B$ (ký hiệu là $U_{AB}$) được xác định bằng công của dòng điện làm chuyển dời 1 đơn vị điện tích từ điểm $A$ đến $B$. Điện thế tại 1 điểm $A$ ($V_A$) được xác định bằng công của dòng điện làm chuyển dời 1 đơn vị điện tích từ $A$ đến vô cùng $\infty$. \fbox{$U_{AB} = V_A - V_B$}.

\subsection{Ohm law -- Định luật Ohm}
Cường độ dòng điện chạy qua 1 dây dẫn tỷ lệ thuận với hiệu điện thế giữa 2 đầu dây: \fbox{$I = aU$} ($\star$), trong đó $a =$ const là hằng số, được gọi là \emph{độ dẫn điện của dây}. Đại lượng nghịch đảo của $a$ là $R$ với $R = \frac{1}{a}$ gọi là \textit{điện trở} (i.e., khả năng cản trở dòng điện) của dây. Đưa $R$ vào ($\star$) ta có định luật Ohm \fbox{$I = \dfrac{U}{R}$} ($\star\star$). Cả 2 công thức ($\star$), ($\star\star$) đều biểu thị định luật Ohm: ``\textit{Cường độ dòng điện chạy qua 1 đoạn mạch tỷ lệ thuận với hiệu điện thế giữa 2 đầu đoạn mạch, tỷ lệ nghịch với điện trở của nó.}''. Đơn vị của điện trở là Ohm, ký hiệu là $\Omega$.

\begin{luuy}
	$R$ là điện trở của 1 dây kim loại, có thể là phi kim, hoặc của 1 chất điện phân, chất khí.
\end{luuy}

\subsection{Sự phụ thuộc của điện trở vào kích thước, hình dạng, \& bản chất dây dẫn}
\fbox{$R = \rho\dfrac{l}{S}$}, trong đó $l$: chiều dài của dây dẫn (hình trụ), $S$: tiết diện thẳng của dây, $\rho$: điện trở suất của vật liệu làm dây dẫn. Nếu đơn vị của $R$ là Ohm, của $l$ là m, của $S$ là $\rm m^2$, thì đơn vị của $\rho$ là $\rm\Omega\cdot m$, đọc là Ohm nhân mét.

\subsection{Sự phụ thuộc của điện trở vào nhiệt độ}
\fbox{$R = R_0(1 + \alpha t)$}, trong đó $R$: điện trở ở $t^\circ$C, $R_0$: điện trở ở $0^\circ$C, $\alpha$: \textit{hệ số nhiệt điện trở}, phụ thuộc vào bản chát của vật liệu làm điện trở.
\begin{itemize}
	\item Với kim loại, hợp kim thì $\alpha > 0$, $R$ tăng theo nhiệt độ.
	\item Với chất điện phân \& 1 số phi kim thì $\alpha < 0$, điện trở giảm khi nhiệt độ tăng lên.
\end{itemize}

\subsection{Định luật Ohm cho đoạn mạch ghép nối tiếp}

\subsection{Định luật Ohm cho đoạn mạch ghép song song}

\subsection{Công \& công suất của dòng điện}
Với điện trở thuần (dòng điện chạy qua nó chỉ gây ra tác dụng nhiệt), công của dòng điện biến hoàn toàn thành nhiệt:
\begin{align*}
	\boxed{Q = A = UIt = I^2Rt = \frac{U^2}{R}t,}
\end{align*}
trong đó $A,Q$ đo bằng đon vị Jun (J). Công suất của dòng điện:
\begin{align*}
	\boxed{\mathcal{P} = \frac{A}{t} = UI = I^2R = \frac{U^2}{R},}
\end{align*}
đơn vị của $\mathcal{P}$ là W (đọc là oát).

\begin{luuy}
	Trong dân dụng, người ta còn dùng đơn vị công của dòng điện là kilooát$\cdot$giờ, ký hiệu là \emph{kW$\cdot$h}. Đơn vị này còn gọi là ``1 số điện''. \emph{1 kW$\cdot$h $= 1000$ W$\cdot3600$ s $= 3600000$ J}.
\end{luuy}
\noindent\fbox{%
	\parbox{\textwidth}{%
		\noindent\textsf{\textbf{Kiến thức cốt lõi.}} \fbox{\bf 1} \textit{Định luật Ohm cho đoạn mạch}: $I = \frac{U}{R}$ với $I$: cường độ dòng điện (A), $U$: hiệu điện thế (V), $R$: điện thế (W). \fbox{\bf 2} Công thức điện trở: $r = \rho\frac{l}{S}$ với $l$: \textit{chiều dài} dây dẫn (m), $S$: tiết diện dây dẫn ($\rm m^2$), $r$: điện trở suất (Wm). \fbox{\bf 3} \textit{Định luật Ohm cho đoạn mạch có các điện trở mắc nối tiếp}: Cường độ dòng điện trong đoạn mạch nối tiếp: $I = I_i$, $\forall i = 1,2,\ldots,n$, i.e., $I = I_1 = I_2 = \cdots = I_n$. Hiệu điện thế trong đoạn mạch nối tiếp: $U = \sum_{i=1}^n U_i = U_1 + U_2 + \cdots + U_n$. Điện trở toàn phần\texttt{/}tương đương của đoạn mạch nối tiếp: $R = \sum_{i=1}^n R_i = R_1 + R_2 + \cdots + R_n$. \fbox{\bf 4} \textit{Định luật Ohm cho đoạn mạch có các điện trở mắc song song}: Cường độ dòng điện trong mạch chính bằng tổng các cường độ dòng điện trong các đoạn mạch rẽ: $I = \sum_{i=1}^n I_i = I_1 + I_2 + \cdots + I_n$. Hiệu điện thế của đoạn mạch song song bằng hiệu điện thế của mỗi đoạn mạch rẽ: $U = U_i$, $\forall i = 1,2,\ldots,n$, i.e., $U = U_1 = U_2 = \cdots = U_n$. Điện trở tương đương của đoạn mạch song song: $\frac{1}{R} = \sum_{i=1}^n \frac{1}{R_i} = \frac{1}{R_1} + \frac{1}{R_2} + \cdots + \frac{1}{R_n}$. Nếu chỉ có 2 điện trở $R_1,R_2$ mắc song song: $\frac{1}{R} = \frac{1}{R_1} + \frac{1}{R_2}$ hay $R = \frac{R_1R_2}{R_1 + R_2}$. \fbox{\bf 5} \textit{Điện năng, công, \& công suất của dòng điện}: Công của dòng điện: $A = UIt$. Trong đoạn mạch chỉ có điện trở: $A = UIt = RI^2t = \frac{U^2}{R}t$. \textit{Công suất} có số đo bằng công thực hiện được trong 1 s: $P = \frac{A}{t} = UI$. Trong đoạn mạch chỉ có điện trở: $P = UI = RI^2 = \frac{U^2}{R}$. \textit{Định luật Joule--Lenz}: $Q = UIt = \frac{U^2}{R}t = RI^2t$, $P = UI = \frac{U^2}{R} = RI^2$. Khi có cân bằng nhiệt thì $Q_{\footnotesize\mbox{tỏa}} = Q_{\footnotesize\mbox{thu}}$ với $Q_{\footnotesize\mbox{thu}}$ có thể tính $Q_{\footnotesize\mbox{thu}} = mc(t_2 - t_1)$ \& $Q_{\footnotesize\mbox{tỏa}}$ tính theo định luật Joule--Lenz. \textit{Hiệu suất sử dụng} là: $H = \dfrac{Q_{\footnotesize\mbox{hữu ích}}}{Q_{\footnotesize\mbox{toàn phần}}}\cdot100\%$ hay $H = \dfrac{P_{\footnotesize\mbox{hữu ích}}}{P_{\footnotesize\mbox{toàn phần}}}\cdot100\%$.
	}%
}

\section{Miscellaneous}

%------------------------------------------------------------------------------%

\printbibliography[heading=bibintoc]
	
\end{document}