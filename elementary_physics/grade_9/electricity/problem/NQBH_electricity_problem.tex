\documentclass{article}
\usepackage[backend=biber,natbib=true,style=authoryear,maxbibnames=10]{biblatex}
\addbibresource{/home/nqbh/reference/bib.bib}
\usepackage[utf8]{vietnam}
\usepackage{tocloft}
\renewcommand{\cftsecleader}{\cftdotfill{\cftdotsep}}
\usepackage[colorlinks=true,linkcolor=blue,urlcolor=red,citecolor=magenta]{hyperref}
\usepackage{amsmath,amssymb,amsthm,float,graphicx,mathtools}
\allowdisplaybreaks
\newtheorem{assumption}{Assumption}
\newtheorem{baitoan}{Bài toán}
\newtheorem{cauhoi}{Câu hỏi}
\newtheorem{conjecture}{Conjecture}
\newtheorem{corollary}{Corollary}
\newtheorem{dangtoan}{Dạng toán}
\newtheorem{definition}{Definition}
\newtheorem{dinhly}{Định lý}
\newtheorem{dinhnghia}{Định nghĩa}
\newtheorem{example}{Example}
\newtheorem{ghichu}{Ghi chú}
\newtheorem{hequa}{Hệ quả}
\newtheorem{hypothesis}{Hypothesis}
\newtheorem{lemma}{Lemma}
\newtheorem{luuy}{Lưu ý}
\newtheorem{nhanxet}{Nhận xét}
\newtheorem{notation}{Notation}
\newtheorem{note}{Note}
\newtheorem{principle}{Principle}
\newtheorem{problem}{Problem}
\newtheorem{proposition}{Proposition}
\newtheorem{question}{Question}
\newtheorem{remark}{Remark}
\newtheorem{theorem}{Theorem}
\newtheorem{vidu}{Ví dụ}
\usepackage[left=1cm,right=1cm,top=5mm,bottom=5mm,footskip=4mm]{geometry}
\def\labelitemii{$\circ$}
\DeclareRobustCommand{\divby}{%
	\mathrel{\vbox{\baselineskip.65ex\lineskiplimit0pt\hbox{.}\hbox{.}\hbox{.}}}%
}

\title{Problem: Electricity -- Điện Học}
\author{Nguyễn Quản Bá Hồng\footnote{Independent Researcher, Ben Tre City, Vietnam\\e-mail: \texttt{nguyenquanbahong@gmail.com}; website: \url{https://nqbh.github.io}.}}
\date{\today}

\begin{document}
\maketitle
\begin{abstract}
	
\end{abstract}
\tableofcontents
\newpage

%------------------------------------------------------------------------------%

\section{Sự Phụ Thuộc của Cường Độ Dòng Điện vào Hiện Điện Thế giữa 2 Đầu Dây Dẫn}

%------------------------------------------------------------------------------%

\section{Điện Trở của Dây Dẫn -- Định Luật Ohm}

%------------------------------------------------------------------------------%

\section{Đoạn Mạch Nối Tiếp}

%------------------------------------------------------------------------------%

\section{Đoạn Mạch Song Song}

%------------------------------------------------------------------------------%

\section{Bài Tập Vận Dụng Định Luật Ohm}

%------------------------------------------------------------------------------%

\section{Sự Phụ Thuộc của Điện Trở vào Chiều Dài Dây Dẫn}

%------------------------------------------------------------------------------%

\section{Sự Phụ Thuộc của Điện Trở vào Tiết Diện Dây Dẫn}

%------------------------------------------------------------------------------%

\section{Sự Phụ Thuộc của Điện Trở vào Vật Liệu Làm Dây Dẫn}

%------------------------------------------------------------------------------%

\section{Biến Trở -- Điện Trở Dùng Trong Kỹ Thuật}

%------------------------------------------------------------------------------%

\section{Bài tập Vận Dụng Định Luật Ohm \& Công Thức Tính Điện Trở của Dây Dẫn}

%------------------------------------------------------------------------------%

\section{Công Suất Điện}

%------------------------------------------------------------------------------%

\section{Điện Năng -- Công của Dòng Điện}

%------------------------------------------------------------------------------%

\section{Bài Tập về Công Suất Điện \& Điện Năng Sử Dụng}

%------------------------------------------------------------------------------%

\section{Định Luật Jule-Lenz \& Bài Tập Vận Dụng}

%------------------------------------------------------------------------------%

\section{Sử Dụng An Toàn \& Tiết Kiệm Điện}

%------------------------------------------------------------------------------%

\printbibliography[heading=bibintoc]
	
\end{document}