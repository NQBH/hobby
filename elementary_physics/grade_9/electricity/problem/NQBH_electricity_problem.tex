\documentclass{article}
\usepackage[backend=biber,natbib=true,style=alphabetic,maxbibnames=50]{biblatex}
\addbibresource{/home/nqbh/reference/bib.bib}
\usepackage[utf8]{vietnam}
\usepackage{tocloft}
\renewcommand{\cftsecleader}{\cftdotfill{\cftdotsep}}
\usepackage[colorlinks=true,linkcolor=blue,urlcolor=red,citecolor=magenta]{hyperref}
\usepackage{amsmath,amssymb,amsthm,float,graphicx,mathtools}
\allowdisplaybreaks
\newtheorem{assumption}{Assumption}
\newtheorem{baitoan}{Bài toán}
\newtheorem{cauhoi}{Câu hỏi}
\newtheorem{conjecture}{Conjecture}
\newtheorem{corollary}{Corollary}
\newtheorem{dangtoan}{Dạng toán}
\newtheorem{definition}{Definition}
\newtheorem{dinhly}{Định lý}
\newtheorem{dinhnghia}{Định nghĩa}
\newtheorem{example}{Example}
\newtheorem{ghichu}{Ghi chú}
\newtheorem{hequa}{Hệ quả}
\newtheorem{hypothesis}{Hypothesis}
\newtheorem{lemma}{Lemma}
\newtheorem{luuy}{Lưu ý}
\newtheorem{nhanxet}{Nhận xét}
\newtheorem{notation}{Notation}
\newtheorem{note}{Note}
\newtheorem{principle}{Principle}
\newtheorem{problem}{Problem}
\newtheorem{proposition}{Proposition}
\newtheorem{question}{Question}
\newtheorem{quyuoc}{Quy ước}
\newtheorem{remark}{Remark}
\newtheorem{theorem}{Theorem}
\newtheorem{vidu}{Ví dụ}
\usepackage[left=1cm,right=1cm,top=5mm,bottom=5mm,footskip=4mm]{geometry}
\def\labelitemii{$\circ$}
\DeclareRobustCommand{\divby}{%
	\mathrel{\vbox{\baselineskip.65ex\lineskiplimit0pt\hbox{.}\hbox{.}\hbox{.}}}%
}

\title{Problem: Electricity -- Điện Học}
\author{Nguyễn Quản Bá Hồng\footnote{Independent Researcher, Ben Tre City, Vietnam\\e-mail: \texttt{nguyenquanbahong@gmail.com}; website: \url{https://nqbh.github.io}.}}
\date{\today}

\begin{document}
\maketitle
\begin{abstract}
	
\end{abstract}
\tableofcontents
\newpage

%------------------------------------------------------------------------------%

\section{Sự Phụ Thuộc của Cường Độ Dòng Điện vào Hiện Điện Thế giữa 2 Đầu Dây Dẫn}

\begin{baitoan}[\cite{SBT_Vat_Ly_9}, 1.1., p. 4]
	Khi đặt vào 2 đầu dây dẫn 1 hiệu điện thế \emph{12 V} thì cường độ dòng điện chạy qua nó là \emph{0.5 A}. Nếu hiệu điện thế đặt vào 2 đầu dây dẫn đó tăng lên đến \emph{36 V} thì cường độ dòng điện chạy qua nó là bao nhiêu?
\end{baitoan}

\begin{baitoan}[\cite{SBT_Vat_Ly_9}, 1.2., p. 4]
	Cường độ dòng điện chạy qua 1 dây dẫn là \emph{1.5 A} khi nó được mắc vào hiệu điện thế \emph{12 V}. Muốn dòng điện chạy qua dây dẫn đó tăng thêm \emph{0.5 A} thì hiệu điện thế phải là bao nhiêu?
\end{baitoan}

\begin{baitoan}[\cite{SBT_Vat_Ly_9}, 1.3., p. 4]
	1 dây dẫn được mắc vào hiệu điện thế \emph{6 V} thì cường độ dòng điện chạy qua nó là \emph{0.3 A}. 1 bạn học sinh nói: Nếu giảm hiệu điện thế đặt vào 2 đầu dây dẫn đi \emph{2 V} thì dòng điện chạy qua dây khi đó có cường độ là \emph{0.15 A}. \emph{Đ\texttt{/}S?} Vì sao?
\end{baitoan}

\begin{baitoan}[\cite{SBT_Vat_Ly_9}, 1.4., p. 4]
	Khi đặt hiệu điện thế \emph{12 V} vào 2 đầu 1 dây dẫn thì dòng điện chạy qua nó có cường độ \emph{6 mA}. Muốn dòng điện chạy qua dây dẫn đó có cường độ giảm đi \emph{4 mA} thì hiệu điện thế là: {\sf A.} \emph{3 V}. {\sf B.} \emph{8 V}. {\sf C.} \emph{5 V}. {\sf D.} \emph{4 V}.
\end{baitoan}

\begin{baitoan}[\cite{SBT_Vat_Ly_9}, 1.5., p. 4]
	Cường độ dòng điện chạy qua 1 dây dẫn phụ thuộc như thế nào vào hiệu điện thế giữa 2 đầu dây dẫn đó? {\sf A.} Không thay đổi khi thay đổi hiệu điện thế. {\sf B.} Tỷ lệ nghịch với hiệu điện thế. {\sf C.} Tỷ lệ thuận với hiệu điện thế. {\sf D.} Giảm khi tăng hiệu điện thế.
\end{baitoan}

\begin{baitoan}[\cite{SBT_Vat_Ly_9}, 1.6., p. 5]
	Nếu tăng hiệu điện thế giữa 2 đầu 1 dây dẫn lên $4$ lần thì cường độ dòng điện chạy qua dây dẫn này thay đổi như thế nào? {\sf A.} Tăng $4$ lần. {\sf B.} Giảm $4$ lần. {\sf C.} Tăng $2$ lần. {\sf D.} Giảm $2$ lần.
\end{baitoan}

\begin{baitoan}[\cite{SBT_Vat_Ly_9}, 1.7., p. 5]
	Đồ thị nào dưới đây biểu diễn sự phụ thuộc của cường độ dòng điện chạy qua 1 dây dẫn vào hiệu điện thế giữa 2 đầu dây dẫn đó?
	\begin{figure}[H]
		\centering
		\includegraphics[scale=0.3]{SBT_1.1}
	\end{figure}
\end{baitoan}

\begin{baitoan}[\cite{SBT_Vat_Ly_9}, 1.8., p. 5]
	Dòng điện đi qua 1 dây dẫn có cường độ $I_1$ khi hiệu điện thế giữa 2 đầu dây là \emph{12 V}. Để dòng điện này có cường độ $I_2$ nhỏ hơn $I_1$ 1 lượng là $0.6I_1$ thì phải đặt giữa 2 đầu dây này 1 hiệu điện thế là bao nhiêu?
\end{baitoan}

\begin{baitoan}[\cite{SBT_Vat_Ly_9}, 1.9., p. 5]
	Ta đã biết: để tăng tác dụng của dòng điện, e.g., để đèn sáng hơn, thì phải tăng cường độ dòng điện chạy qua bóng đèn đó. Thế nhưng trên thực tế thì người ta lại tăng hiệu điện thế đặt vào 2 đầu bóng đèn. Giải thích.
\end{baitoan}

\begin{baitoan}[\cite{SBT_Vat_Ly_9}, 1.10., p. 5]
	Cường độ dòng điện đi qua 1 dây dẫn là $I_1$ khi hiệu điện thế giữa 2 đầu dây dẫn này là $U_1 = 7.2$ \emph{V}. Dòng điện đi qua dây dẫn này sẽ có cường độ $I_2$ lớp hơn $I_1$ bao nhiêu lần nếu hiệu điện thế giữa 2 đầu có nó tăng thêm \emph{10.8 V}?
\end{baitoan}

\begin{baitoan}[\cite{SBT_Vat_Ly_9}, 1.11., p. 5]
	Khi đặt 1 hiệu điện thế \emph{10 V} giữa 2 đầu 1 dây dẫn thì dòng điện đi qua nó có cường độ là \emph{1.25 A}. Hỏi phải giảm hiệu điện thế giữa 2 đầu dây này đi 1 lượng là bao nhiêu để dòng điện đi qua dây chỉ còn là \emph{0.75 A}?
\end{baitoan}

%------------------------------------------------------------------------------%

\section{Điện Trở của Dây Dẫn -- Định Luật Ohm}

\begin{quyuoc}[Điện trở của thiết bị điện]
	Điện trở của ampe kế, dây nối, công tắc K rất nhỏ $\approx0$, được coi là $= 0$, còn điện trở của vôn kế là vô cùng lớn, i.e., $R = +\infty$.
\end{quyuoc}

\begin{baitoan}[\cite{SBT_Vat_Ly_9}, 2.1., p. 6]
	Hình sau vẽ đồ thị biểu diễn sự phụ thuộc của cường độ dòng điện vào hiệu điện thế của 3 dây dẫn khác nhau.
	\begin{figure}[H]
		\centering
		\includegraphics[scale=0.3]{SBT_2.1}
	\end{figure}
	\noindent(a) Từ đồ thị, xác định giá trị cường độ dòng điện chạy qua mỗi dây dẫn khi hiệu điện thế đặt giữa 2 đầu dây dẫn là \emph{3 V}. (b) Dây dẫn nào có điện trở lớn nhất? Nhỏ nhất? Giải thích bằng 3 cách khác nhau.
\end{baitoan}

\begin{baitoan}[\cite{SBT_Vat_Ly_9}, 2.2., p. 6]
	Cho điện trở $R = 15\ \Omega$. (a) Khi mắc điện trở này vào hiệu điện thế \emph{6 V} thì dòng điện chạy qua nó có cường độ bao nhiêu? (b) Muốn cường độ dòng điện chạy qua điện trở tăng thêm \emph{0.3 A} so với trường hợp trên thì hiệu điện thế đặt vào 2 đầu điện trở khi đó là bao nhiêu?
\end{baitoan}

\begin{baitoan}[\cite{SBT_Vat_Ly_9}, 2.3., p. 6]
	Làm thí nghiệm khảo sát sự phụ thuộc của cường độ dòng điện vào hiệu điện thế đặt giữa 2 đầu vật dẫn bằng kim loại, người ta thu được bảng số liệu:
	\begin{table}[H]
		\centering
		\begin{tabular}{|c|c|c|c|c|c|c|c|}
			\hline
			$U$ (V) & 0 & 1.5 & 3 & 4.5 & 6 & 7.5 & 9 \\
			\hline
			$I$ (A) & 0 & 0.31 & 0.61 & 0.9 & 1.29 & 1.49 & 1.78 \\
			\hline
		\end{tabular}
	\end{table}
	\noindent(a) Vẽ đồ thị biểu diễn sự phụ thuộc của $I$ vào $U$. (b) Dựa vào đồ thị ở (a), tính điện trở của vật dẫn nếu bỏ qua các sai số trong phép đo.
\end{baitoan}

\begin{baitoan}[\cite{SBT_Vat_Ly_9}, 2.4., p. 7]
	Cho mạch điện có sơ đồ:
	\begin{figure}[H]
		\centering
		\includegraphics[scale=0.3]{SBT_2.2}
	\end{figure}
	\noindent điện trở $R_1 = 10\ \Omega$, hiệu điện thế giữa 2 đầu đoạn mạch là $U_{\rm MN} = 12$ \emph{V}. (a) Tính cường độ dòng điện $I_1$ chạy qua $R_1$. (b) Giữ nguyên $U_{\rm MN} = 12$ \emph{V}, thay điện trở $R_1$ bằng điện trở $R_2$, khi đó ampe kế chỉ giá trị $I_2 = \frac{1}{2}I_1$. Tính điện trở $R_2$.
\end{baitoan}

\begin{baitoan}[\cite{SBT_Vat_Ly_9}, 2.5., p. 7]
	Điện trở của 1 dây dẫn nhất định có mối quan hệ phụ thuộc nào sau đây? {\sf A.} Tỷ lệ thuận với hiệu điện thế đặt vào 2 đầu dây dẫn. {\sf B.} Tỷ lệ nghịch với cường độ dòng điện chạy qua dây dẫn. {\sf C.} Không phụ thuộc vào hiệu điện thế đặt vào 2 đầu dây dẫn. {\sf D.} Giảm khi cường độ dòng điện chạy qua dây dẫn giảm.
\end{baitoan}

\begin{baitoan}[\cite{SBT_Vat_Ly_9}, 2.6., p. 7]
	Khi đặt 1 hiệu điện thế $U$ vào 2 đầu 1 điện trở $R$ thì dòng điện chạy qua nó có cường độ là $I$. Hệ thức nào dưới đây biểu thị định luật Ohm? {\sf A.} $U = \frac{I}{R}$. {\sf B.} $I = \frac{U}{R}$. {\sf C.} $I = \frac{R}{U}$. {\sf D.} $R = \frac{U}{I}$.
\end{baitoan}

\begin{baitoan}[\cite{SBT_Vat_Ly_9}, 2.7., p. 7]
	 Đơn vị nào dưới dây là đơn vị đo điện trở? {\sf A.} Ohm $\Omega$. {\sf B.} Watz W. {\sf C.} Ampe A. {\sf D.} Volt V.
\end{baitoan}

\begin{baitoan}[\cite{SBT_Vat_Ly_9}, 2.8., p. 7]
	Trong thí nghiệm khảo sát định luật Ohm, có thể làm thay đổi đại lượng nào trong số các đại lượng gồm hiệu điện thế, cường độ dòng điện, điện trở dây dẫn? {\sf A.} Chỉ thay đổi hiệu điện thế. {\sf B.} Chỉ thay đổi cường độ dòng điện. {\sf C.} Chỉ thay đổi điện trở dây dẫn. {\sf D.} Cả 3 đại lượng trên.
\end{baitoan}

\begin{baitoan}[\cite{SBT_Vat_Ly_9}, 2.9., p. 8]
	Dựa vào công thức $R = \frac{U}{I}$ có học sinh phát biểu như sau: ``Điện trở của dây dẫn tỷ lệ thuận với hiệu điện thế giữa 2 đầu dây \& tỷ lệ nghịch với cường độ dòng điện chạy qua dây.'' \emph{Đ\texttt{/}S?} Vì sao?
\end{baitoan}

\begin{baitoan}[\cite{SBT_Vat_Ly_9}, 2.10., p. 8]
	Đặt hiệu điện thế \emph{6 V} vào 2 đầu 1 điện trở thì dòng điện đi qua điện trở có cường độ \emph{0.15 A}. (a) Tính trị số của điện trở này. (b) Nếu tăng hiệu điện thế đặt vào 2 đầu điện trở này lên thành \emph{8 V} thì trị số của điện trở này có thay đổi không? Trị số của nó khi đó là bao nhiêu? Dòng điện đi qua nó khi đó có cường độ là bao nhiêu?
\end{baitoan}

\begin{baitoan}[\cite{SBT_Vat_Ly_9}, 2.11., p. 8]
	Giữa 2 đầu 1 điện trở $R_1 = 20\ \Omega$ có 1 hiệu điện thế là $U = 3.2$ \emph{V}. (a) Tính cường độ dòng điện $I_1$ đi qua điện trở này khi đó. (b) Giữ nguyên hiệu điện thế $U$ đã cho, thay điện trở $R_1$ bằng điện trở $R_2$ sao cho dòng điện đi qua $R_2$ có cường độ $I_2 = 0.8I_1$. Tính $R_2$.
\end{baitoan}

\begin{baitoan}[\cite{SBT_Vat_Ly_9}, 2.12., p. 8]
	Trên hình sau có vẽ đồ thị biểu diễn sự phụ thuộc của cường độ dòng điện vào hiệu điện thế đối với 2 điện trở $R_1,R_2$:
	\begin{figure}[H]
		\centering
		\includegraphics[scale=0.3]{SBT_2.3}
	\end{figure}
	\noindent(a) Từ đồ thị, tính trị số các điện trở $R_1,R_2$. (b) Tính cường độ dòng điện $I_1,I_2$ tương ứng đi qua mỗi điện trở khi lần lượt đặt hiệu điện thế $U = 1.8$ \emph{V} vào 2 đầu mỗi điện trở đó.
\end{baitoan}

%------------------------------------------------------------------------------%

\section{Đoạn Mạch Nối Tiếp}

%------------------------------------------------------------------------------%

\section{Đoạn Mạch Song Song}

%------------------------------------------------------------------------------%

\section{Bài Tập Vận Dụng Định Luật Ohm}

%------------------------------------------------------------------------------%

\section{Sự Phụ Thuộc của Điện Trở vào Chiều Dài Dây Dẫn}

%------------------------------------------------------------------------------%

\section{Sự Phụ Thuộc của Điện Trở vào Tiết Diện Dây Dẫn}

%------------------------------------------------------------------------------%

\section{Sự Phụ Thuộc của Điện Trở vào Vật Liệu Làm Dây Dẫn}

%------------------------------------------------------------------------------%

\section{Biến Trở -- Điện Trở Dùng Trong Kỹ Thuật}

%------------------------------------------------------------------------------%

\section{Bài tập Vận Dụng Định Luật Ohm \& Công Thức Tính Điện Trở của Dây Dẫn}

%------------------------------------------------------------------------------%

\section{Công Suất Điện}

%------------------------------------------------------------------------------%

\section{Điện Năng -- Công của Dòng Điện}

%------------------------------------------------------------------------------%

\section{Bài Tập về Công Suất Điện \& Điện Năng Sử Dụng}

%------------------------------------------------------------------------------%

\section{Định Luật Jule-Lenz \& Bài Tập Vận Dụng}

%------------------------------------------------------------------------------%

\section{Sử Dụng An Toàn \& Tiết Kiệm Điện}

%------------------------------------------------------------------------------%

\printbibliography[heading=bibintoc]
	
\end{document}