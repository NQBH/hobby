\documentclass{article}
\usepackage[backend=biber,natbib=true,style=alphabetic,maxbibnames=50]{biblatex}
\addbibresource{/home/nqbh/reference/bib.bib}
\usepackage[utf8]{vietnam}
\usepackage{tocloft}
\renewcommand{\cftsecleader}{\cftdotfill{\cftdotsep}}
\usepackage[colorlinks=true,linkcolor=blue,urlcolor=red,citecolor=magenta]{hyperref}
\usepackage{amsmath,amssymb,amsthm,float,graphicx,mathtools}
\allowdisplaybreaks
\newtheorem{assumption}{Assumption}
\newtheorem{baitoan}{Bài toán}
\newtheorem{cauhoi}{Câu hỏi}
\newtheorem{conjecture}{Conjecture}
\newtheorem{corollary}{Corollary}
\newtheorem{dangtoan}{Dạng toán}
\newtheorem{definition}{Definition}
\newtheorem{dinhly}{Định lý}
\newtheorem{dinhnghia}{Định nghĩa}
\newtheorem{example}{Example}
\newtheorem{ghichu}{Ghi chú}
\newtheorem{hequa}{Hệ quả}
\newtheorem{hypothesis}{Hypothesis}
\newtheorem{lemma}{Lemma}
\newtheorem{luuy}{Lưu ý}
\newtheorem{nhanxet}{Nhận xét}
\newtheorem{notation}{Notation}
\newtheorem{note}{Note}
\newtheorem{principle}{Principle}
\newtheorem{problem}{Problem}
\newtheorem{proposition}{Proposition}
\newtheorem{question}{Question}
\newtheorem{remark}{Remark}
\newtheorem{theorem}{Theorem}
\newtheorem{vidu}{Ví dụ}
\usepackage[left=1cm,right=1cm,top=5mm,bottom=5mm,footskip=4mm]{geometry}
\def\labelitemii{$\circ$}
\DeclareRobustCommand{\divby}{%
	\mathrel{\vbox{\baselineskip.65ex\lineskiplimit0pt\hbox{.}\hbox{.}\hbox{.}}}%
}

\title{Problem: Electricity -- Điện Học}
\author{Nguyễn Quản Bá Hồng\footnote{Independent Researcher, Ben Tre City, Vietnam\\e-mail: \texttt{nguyenquanbahong@gmail.com}; website: \url{https://nqbh.github.io}.}}
\date{\today}

\begin{document}
\maketitle
\begin{abstract}
	
\end{abstract}
\tableofcontents
\newpage

%------------------------------------------------------------------------------%

\section{Sự Phụ Thuộc của Cường Độ Dòng Điện vào Hiện Điện Thế giữa 2 Đầu Dây Dẫn}

\begin{baitoan}[\cite{SBT_Vat_Ly_9}, 1.1., p. 4]
	Khi đặt vào 2 đầu dây dẫn 1 hiệu điện thế \emph{12 V} thì cường độ dòng điện chạy qua nó là \emph{0.5 A}. Nếu hiệu điện thế đặt vào 2 đầu dây dẫn đó tăng lên đến \emph{36 V} thì cường độ dòng điện chạy qua nó là bao nhiêu?
\end{baitoan}

\begin{baitoan}[\cite{SBT_Vat_Ly_9}, 1.2., p. 4]
	Cường độ dòng điện chạy qua 1 dây dẫn là \emph{1.5 A} khi nó được mắc vào hiệu điện thế \emph{12 V}. Muốn dòng điện chạy qua dây dẫn đó tăng thêm \emph{0.5 A} thì hiệu điện thế phải là bao nhiêu?
\end{baitoan}

\begin{baitoan}[\cite{SBT_Vat_Ly_9}, 1.3., p. 4]
	1 dây dẫn được mắc vào hiệu điện thế \emph{6 V} thì cường độ dòng điện chạy qua nó là \emph{0.3 A}. 1 bạn học sinh nói: Nếu giảm hiệu điện thế đặt vào 2 đầu dây dẫn đi \emph{2 V} thì dòng điện chạy qua dây khi đó có cường độ là \emph{0.15 A}. \emph{Đ\texttt{/}S}? Vì sao?
\end{baitoan}

\begin{baitoan}[\cite{SBT_Vat_Ly_9}, 1.4., p. 4]
	Khi đặt hiệu điện thế \emph{12 V} vào 2 đầu 1 dây dẫn thì dòng điện chạy qua nó có cường độ \emph{6 mA}. Muốn dòng điện chạy qua dây dẫn đó có cường độ giảm đi \emph{4 mA} thì hiệu điện thế là: {\sf A.} \emph{3 V}. {\sf B.} \emph{8 V}. {\sf C.} \emph{5 V}. {\sf D.} \emph{4 V}.
\end{baitoan}

\begin{baitoan}[\cite{SBT_Vat_Ly_9}, 1.5., p. 4]
	Cường độ dòng điện chạy qua 1 dây dẫn phụ thuộc như thế nào vào hiệu điện thế giữa 2 đầu dây dẫn đó? {\sf A.} Không thay đổi khi thay đổi hiệu điện thế. {\sf B.} Tỷ lệ nghịch với hiệu điện thế. {\sf C.} Tỷ lệ thuận với hiệu điện thế. {\sf D.} Giảm khi tăng hiệu điện thế.
\end{baitoan}

\begin{baitoan}[\cite{SBT_Vat_Ly_9}, 1.6., p. 5]
	Nếu tăng hiệu điện thế giữa 2 đầu 1 dây dẫn lên $4$ lần thì cường độ dòng điện chạy qua dây dẫn này thay đổi như thế nào? {\sf A.} Tăng $4$ lần. {\sf B.} Giảm $4$ lần. {\sf C.} Tăng $2$ lần. {\sf D.} Giảm $2$ lần.
\end{baitoan}

\begin{baitoan}[\cite{SBT_Vat_Ly_9}, 1.7., p. 5]
	Đồ thị nào dưới đây biểu diễn sự phụ thuộc của cường độ dòng điện chạy qua 1 dây dẫn vào hiệu điện thế giữa 2 đầu dây dẫn đó?
	\begin{figure}[H]
		\centering
		\includegraphics[scale=0.3]{IU_graph}
	\end{figure}
\end{baitoan}

\begin{baitoan}[\cite{SBT_Vat_Ly_9}, 1.8., p. 5]
	Dòng điện đi qua 1 dây dẫn có cường độ $I_1$ khi hiệu điện thế giữa 2 đầu dây là \emph{12 V}. Để dòng điện này có cường độ $I_2$ nhỏ hơn $I_1$ 1 lượng là $0.6I_1$ thì phải đặt giữa 2 đầu dây này 1 hiệu điện thế là bao nhiêu?
\end{baitoan}

\begin{baitoan}[\cite{SBT_Vat_Ly_9}, 1.9., p. 5]
	Ta đã biết: để tăng tác dụng của dòng điện, e.g., để đèn sáng hơn, thì phải tăng cường độ dòng điện chạy qua bóng đèn đó. Thế nhưng trên thực tế thì người ta lại tăng hiệu điện thế đặt vào 2 đầu bóng đèn. Giải thích.
\end{baitoan}

\begin{baitoan}[\cite{SBT_Vat_Ly_9}, 1.10., p. 5]
	Cường độ dòng điện đi qua 1 dây dẫn là $I_1$ khi hiệu điện thế giữa 2 đầu dây dẫn này là $U_1 = 7.2$ \emph{V}. Dòng điện đi qua dây dẫn này sẽ có cường độ $I_2$ lớp hơn $I_1$ bao nhiêu lần nếu hiệu điện thế giữa 2 đầu có nó tăng thêm \emph{10.8 V}?
\end{baitoan}

\begin{baitoan}[\cite{SBT_Vat_Ly_9}, 1.11., p. 5]
	Khi đặt 1 hiệu điện thế \emph{10 V} giữa 2 đầu 1 dây dẫn thì dòng điện đi qua nó có cường độ là \emph{1.25 A}. Hỏi phải giảm hiệu điện thế giữa 2 đầu dây này đi 1 lượng là bao nhiêu để dòng điện đi qua dây chỉ còn là \emph{0.75 A}?
\end{baitoan}

%------------------------------------------------------------------------------%

\section{Điện Trở của Dây Dẫn -- Định Luật Ohm}

%------------------------------------------------------------------------------%

\section{Đoạn Mạch Nối Tiếp}

%------------------------------------------------------------------------------%

\section{Đoạn Mạch Song Song}

%------------------------------------------------------------------------------%

\section{Bài Tập Vận Dụng Định Luật Ohm}

%------------------------------------------------------------------------------%

\section{Sự Phụ Thuộc của Điện Trở vào Chiều Dài Dây Dẫn}

%------------------------------------------------------------------------------%

\section{Sự Phụ Thuộc của Điện Trở vào Tiết Diện Dây Dẫn}

%------------------------------------------------------------------------------%

\section{Sự Phụ Thuộc của Điện Trở vào Vật Liệu Làm Dây Dẫn}

%------------------------------------------------------------------------------%

\section{Biến Trở -- Điện Trở Dùng Trong Kỹ Thuật}

%------------------------------------------------------------------------------%

\section{Bài tập Vận Dụng Định Luật Ohm \& Công Thức Tính Điện Trở của Dây Dẫn}

%------------------------------------------------------------------------------%

\section{Công Suất Điện}

%------------------------------------------------------------------------------%

\section{Điện Năng -- Công của Dòng Điện}

%------------------------------------------------------------------------------%

\section{Bài Tập về Công Suất Điện \& Điện Năng Sử Dụng}

%------------------------------------------------------------------------------%

\section{Định Luật Jule-Lenz \& Bài Tập Vận Dụng}

%------------------------------------------------------------------------------%

\section{Sử Dụng An Toàn \& Tiết Kiệm Điện}

%------------------------------------------------------------------------------%

\printbibliography[heading=bibintoc]
	
\end{document}