\documentclass{article}
\usepackage[backend=biber,natbib=true,style=alphabetic,maxbibnames=50]{biblatex}
\addbibresource{/home/nqbh/reference/bib.bib}
\usepackage[utf8]{vietnam}
\usepackage{tocloft}
\renewcommand{\cftsecleader}{\cftdotfill{\cftdotsep}}
\usepackage[colorlinks=true,linkcolor=blue,urlcolor=red,citecolor=magenta]{hyperref}
\usepackage{amsmath,amssymb,amsthm,float,graphicx,mathtools}
\allowdisplaybreaks
\newtheorem{assumption}{Assumption}
\newtheorem{baitoan}{Bài toán}
\newtheorem{cauhoi}{Câu hỏi}
\newtheorem{conjecture}{Conjecture}
\newtheorem{corollary}{Corollary}
\newtheorem{dangtoan}{Dạng toán}
\newtheorem{definition}{Definition}
\newtheorem{dinhly}{Định lý}
\newtheorem{dinhnghia}{Định nghĩa}
\newtheorem{example}{Example}
\newtheorem{ghichu}{Ghi chú}
\newtheorem{hequa}{Hệ quả}
\newtheorem{hypothesis}{Hypothesis}
\newtheorem{lemma}{Lemma}
\newtheorem{luuy}{Lưu ý}
\newtheorem{nhanxet}{Nhận xét}
\newtheorem{notation}{Notation}
\newtheorem{note}{Note}
\newtheorem{principle}{Principle}
\newtheorem{problem}{Problem}
\newtheorem{proposition}{Proposition}
\newtheorem{question}{Question}
\newtheorem{quyuoc}{Quy ước}
\newtheorem{remark}{Remark}
\newtheorem{theorem}{Theorem}
\newtheorem{vidu}{Ví dụ}
\usepackage[left=1cm,right=1cm,top=5mm,bottom=5mm,footskip=4mm]{geometry}
\def\labelitemii{$\circ$}
\DeclareRobustCommand{\divby}{%
	\mathrel{\vbox{\baselineskip.65ex\lineskiplimit0pt\hbox{.}\hbox{.}\hbox{.}}}%
}

\title{Problem: Electricity -- Bài Tập: Điện Học}
\author{Nguyễn Quản Bá Hồng\footnote{Independent Researcher, Ben Tre City, Vietnam\\e-mail: \texttt{nguyenquanbahong@gmail.com}; website: \url{https://nqbh.github.io}.}}
\date{\today}

\begin{document}
\maketitle
\begin{abstract}
	
\end{abstract}
\tableofcontents
\newpage

%------------------------------------------------------------------------------%

\section{Sự Phụ Thuộc của Cường Độ Dòng Điện vào Hiện Điện Thế giữa 2 Đầu Dây Dẫn}

\begin{baitoan}[\cite{SBT_Vat_Ly_9}, 1.1., p. 4]
	Khi đặt vào 2 đầu dây dẫn 1 hiệu điện thế $U_1 = 12$ \emph{V} thì cường độ dòng điện chạy qua nó là $I_1 = 0.5$ \emph{A}. Nếu hiệu điện thế đặt vào 2 đầu dây dẫn đó tăng lên đến $U_2 = 36$ \emph{V} thì cường độ dòng điện chạy qua nó là bao nhiêu?
\end{baitoan}

\begin{proof}[Giải]
	\textsf{Tóm tắt.} $U_1 = 12$ V, $I_1 = 0.5$ A, $U_2 = 36$ V, $I_2 =$ ?. \textsf{Giải.} Vì cường độ dòng điện chạy qua một đoạn dây dẫn tỉ lệ thuận với hiệu điện thế đặt vào hai đầu dây dẫn, ta có: $\frac{U_1}{I_1} = \frac{U_2}{I_2}\Leftrightarrow I_2 = \frac{I_1U_2}{U_1} = \frac{36\cdot0.5}{12} = 1.5$ A. Vậy cường độ dòng điện chạy qua dây dẫn khi $U = 36$ V là $I = 1.5$ A.
\end{proof}

\begin{baitoan}[Mở rộng \cite{SBT_Vat_Ly_9}, 1.1., p. 4]
	Khi đặt vào 2 đầu dây dẫn 1 hiệu điện thế $U_1$ \emph{V} thì cường độ dòng điện chạy qua nó là $I_1$ \emph{A}. Giả sử hiệu điện thế đặt vào 2 đầu dây dẫn đó tăng thêm 1 lượng $\Delta U$ \emph{V}, tính cường độ dòng điện chạy qua nó theo $I_1,U_1,\Delta U$.
\end{baitoan}

\begin{proof}[Giải]
	$\frac{U_1}{I_1} = \frac{U_2}{I_2}\Leftrightarrow I_2 = \frac{I_1U_2}{U_1} = \frac{I_1(U_1 + \Delta U)}{U_1}$ A.
\end{proof}

\begin{baitoan}[\cite{SBT_Vat_Ly_9}, 1.2., p. 4]
	Cường độ dòng điện chạy qua 1 dây dẫn là \emph{1.5 A} khi nó được mắc vào hiệu điện thế \emph{12 V}. Muốn dòng điện chạy qua dây dẫn đó tăng thêm \emph{0.5 A} thì hiệu điện thế phải là bao nhiêu?
\end{baitoan}

\begin{proof}[Giải]
	\textsf{Tóm tắt.} $I_1 = 1.5$ A, $U_1 = 12$ V. $I_2 = I_1 + 0.5$, $U_2$ ? \textsf{Giải.} $\frac{U_1}{I_1} = \frac{U_2}{I_2}\Leftrightarrow U_2 = \frac{U_1I_2}{I_1} = \frac{12(1.5 + 0.5)}{1.5} = 16$ V.
\end{proof}

\begin{baitoan}[\cite{SBT_Vat_Ly_9}, 1.3., p. 4]
	1 dây dẫn được mắc vào hiệu điện thế \emph{6 V} thì cường độ dòng điện chạy qua nó là \emph{0.3 A}. 1 bạn học sinh nói: Nếu giảm hiệu điện thế đặt vào 2 đầu dây dẫn đi \emph{2 V} thì dòng điện chạy qua dây khi đó có cường độ là \emph{0.15 A}. \emph{Đ\texttt{/}S?} Vì sao?
\end{baitoan}

\begin{proof}[Giải]
	\textsf{Tóm tắt.} $U_1 = 6$ V, $I_1 = 0.3$ A, $U_2 = U_1 - 2$ V, $I_2$? \textsf{Giải.} S vì $\frac{U_1}{I_1} = \frac{U_2}{I_2}\Leftrightarrow I_2 = \frac{U_2I_1}{U_1} = \frac{(6 - 2)\cdot0.3}{6} = 0.2$ A.
\end{proof}

\begin{baitoan}[\cite{SBT_Vat_Ly_9}, 1.4., p. 4]
	Khi đặt hiệu điện thế \emph{12 V} vào 2 đầu 1 dây dẫn thì dòng điện chạy qua nó có cường độ \emph{6 mA}. Muốn dòng điện chạy qua dây dẫn đó có cường độ giảm đi \emph{4 mA} thì hiệu điện thế phải bằng bao nhiêu?
\end{baitoan}

\begin{proof}[Giải]
	\textsf{Tóm tắt.} $U_1 = 12$ v, $I_1 = 6$ mA, $I_2 = I_1 - 4$ mA, $U_2$? \textsf{Giải.} $\frac{U_1}{I_1} = \frac{U_2}{I_2}\Leftrightarrow U_2 = \frac{U_1I_2}{I_1} = \frac{12(6 - 4)}{6} = 4$ V.
\end{proof}

\begin{baitoan}[\cite{SBT_Vat_Ly_9}, 1.5., p. 4]
	Cường độ dòng điện chạy qua 1 dây dẫn phụ thuộc như thế nào vào hiệu điện thế giữa 2 đầu dây dẫn đó? {\sf A.} Không thay đổi khi thay đổi hiệu điện thế. {\sf B.} Tỷ lệ nghịch với hiệu điện thế. {\sf C.} Tỷ lệ thuận với hiệu điện thế. {\sf D.} Giảm khi tăng hiệu điện thế.\hfill{\sf Ans: C.}
\end{baitoan}

\begin{baitoan}[\cite{SBT_Vat_Ly_9}, 1.6., p. 5]
	Nếu tăng hiệu điện thế giữa 2 đầu 1 dây dẫn lên $4$ lần thì cường độ dòng điện chạy qua dây dẫn này thay đổi như thế nào? {\sf A.} Tăng $4$ lần. {\sf B.} Giảm $4$ lần. {\sf C.} Tăng $2$ lần. {\sf D.} Giảm $2$ lần.\hfill{\sf Ans: A.}
\end{baitoan}

\begin{luuy}
	Với $x\in\mathbb{R}$, $x > 0$, bất kỳ, nếu tăng hiệu điện thế giữa 2 đầu 1 dây dẫn lên $x$ lần (tương ứng, giảm $x$ lần) thì cường độ dòng điện chạy qua dây dẫn này cũng tăng lên $x$ lần (tương ứng, giảm $x$ lần): $\frac{U_1}{I_1} = \frac{xU_1}{xI_1}$, $\forall x\in\mathbb{R}$, $x > 0$.
\end{luuy}

\begin{baitoan}[\cite{SBT_Vat_Ly_9}, 1.7., p. 5]
	Đồ thị nào dưới đây biểu diễn sự phụ thuộc của cường độ dòng điện chạy qua 1 dây dẫn vào hiệu điện thế giữa 2 đầu dây dẫn đó?\hfill{\sf Ans: B.}
	\begin{figure}[H]
		\centering
		\includegraphics[scale=0.25]{SBT_1.1}
	\end{figure}
\end{baitoan}

\begin{proof}[Giải]
	{\sf B.} Đồ thị biểu diễn sự phụ thuộc của cường độ dòng điện vào hiệu điện thế giữa hai đầu dây dẫn là một đường thẳng đi qua gốc tọa độ.
\end{proof}

\begin{baitoan}[\cite{SBT_Vat_Ly_9}, 1.8., p. 5]
	Dòng điện đi qua 1 dây dẫn có cường độ $I_1$ khi hiệu điện thế giữa 2 đầu dây là \emph{12 V}. Để dòng điện này có cường độ $I_2$ nhỏ hơn $I_1$ 1 lượng là $0.6I_1$ thì phải đặt giữa 2 đầu dây này 1 hiệu điện thế là bao nhiêu?
\end{baitoan}

\begin{proof}[Giải]
	\textsf{Tóm tắt.} $U_1 = 12$ V, $I_2 = I_1 - 0.6I_1 = 0.4I_1$, $U_2$? \textsf{Giải.} $\frac{U_1}{I_1} = \frac{U_2}{I_2}\Leftrightarrow U_2 = \frac{U_1I_2}{I_1} = \frac{12\cdot0.4I_1}{I_1} = 12\cdot0.4 = 4.8$ V.
\end{proof}

\begin{baitoan}[\cite{SBT_Vat_Ly_9}, 1.9., p. 5]
	Ta đã biết: để tăng tác dụng của dòng điện, e.g., để đèn sáng hơn, thì phải tăng cường độ dòng điện chạy qua bóng đèn đó. Thế nhưng trên thực tế thì người ta lại tăng hiệu điện thế đặt vào 2 đầu bóng đèn. Giải thích.
\end{baitoan}

\begin{proof}[Giải]
	Vì cường độ dòng điện phụ thuộc vào hiệu điện thế. Cụ thể, cường độ dòng điện chạy qua dây dẫn tỉ lệ thuận với hiệu điện thế đặt vào hai đầu dây dẫn đó . Vì vậy, nếu tăng hiệu điện thế thì cường độ dòng điện tăng \& bóng đèn sẽ sáng hơn. Hơn nữa, tăng hiệu điện thế cũng dễ dàng \& ít tốn kém hơn so với tăng cường độ dòng điện.
\end{proof}

\begin{baitoan}[\cite{SBT_Vat_Ly_9}, 1.10., p. 5]
	Cường độ dòng điện đi qua 1 dây dẫn là $I_1$ khi hiệu điện thế giữa 2 đầu dây dẫn này là $U_1 = 7.2$ \emph{V}. Dòng điện đi qua dây dẫn này sẽ có cường độ $I_2$ lớp hơn $I_1$ bao nhiêu lần nếu hiệu điện thế giữa 2 đầu có nó tăng thêm \emph{10.8 V}?
\end{baitoan}

\begin{proof}[Giải]
	\textsf{Tóm tắt.} $U_1 = 7.2$ V, $U_2 = U_1 = 10.8$ V, $\frac{I_2}{I_1}$? \textsf{Giải.} $\frac{I_2}{I_1} = \frac{U_2}{U_1} = \frac{7.2 + 10.8}{7.2} = 2.5\Rightarrow I_2 = 2.5I_1$. Vậy $I_2$ gấp 2.5 lần $I_1$.
\end{proof}

\begin{baitoan}[\cite{SBT_Vat_Ly_9}, 1.11., p. 5]
	Khi đặt 1 hiệu điện thế \emph{10 V} giữa 2 đầu 1 dây dẫn thì dòng điện đi qua nó có cường độ là \emph{1.25 A}. Hỏi phải giảm hiệu điện thế giữa 2 đầu dây này đi 1 lượng là bao nhiêu để dòng điện đi qua dây chỉ còn là \emph{0.75 A}?
\end{baitoan}

\begin{proof}[Giải]
	\textsf{Tóm tắt.} $U_1 = 10$ V, $I_1 = 1.25$ A, $I_2 = 0.75$ A, $U_1 - U_2$? \textsf{Giải.} Áp dụng tính chất của dãy tỷ số bằng nhau: $\frac{U_1}{I_1} = \frac{U_2}{I_2} = \frac{U_1 - U_2}{I_1 - I_2}\Rightarrow U_1 - U_2 = \frac{U_1(I_1 - I_2)}{I_1} = \frac{10(1.25 - 0.75)}{1.25} = 4$ V. Vậy phải giảm hiệu điện thế giữa 2 đầu dây dẫn này đi 4 V để dòng điện đi qua dây là 0.75 A.
\end{proof}

\begin{luuy}
	Áp dụng tính chất dãy tỷ số bằng nhau cho tỷ lệ thức $\frac{U_1}{I_1} = \frac{U_2}{I_2}$:
	\begin{align*}
		\frac{U_1}{I_1} = \frac{U_2}{I_2} = \frac{aU_1 + bU_2}{aI_1 + bI_2},\ \forall a,b\in\mathbb{R},\,aI_1 + bI_2\ne0.
	\end{align*}
	Nói riêng, khi $a = -1$, $b = 1$, đẳng thức trên trở thành: $\frac{U_1}{I_1} = \frac{U_2}{I_2} = \frac{U_2 - U_1}{I_2 - I_1} = \frac{\Delta U}{\Delta I}$, với $I_2\ne I_1$, i.e., tỷ số giữa hiệu điện thế \& cường độ dòng điện bằng tỷ số của độ thay đổi của chúng. Tổng quát hơn, ta có thể áp dụng tính chất dãy tỷ số bằng nhau cho dãy tỷ lệ thức $\frac{U_1}{I_1} = \frac{U_2}{I_2} = \cdots = \frac{U_n}{I_n} = k$, i.e., $\frac{U_i}{I_i} = k\in\mathbb{R}$, $k > 0$, $\forall i = 1,2,\ldots,n$:
	\begin{align*}
		\frac{U_i}{I_i} &= \frac{\sum_{k=1}^n a_kU_k}{\sum_{k=1}^n a_kI_k} = k,\ \forall a_i,b_i\in\mathbb{R},\,\sum_{k=1}^n a_kI_k\ne0,\ \forall i = 1,2,\ldots,n,\mbox{ i.e.,}\\		
		\frac{U_1}{I_1} = \frac{U_2}{I_2} = \cdots = \frac{U_n}{I_n} &= \frac{a_1U_1 + a_2U_2 + \cdots + a_nU_n}{a_1I_1 + a_2I_2 + \cdots + a_nI_n} = k,\ \forall a_i,b_i\in\mathbb{R},\,a_1I_1 + a_2I_2 + \cdots + a_nI_n\ne0,\ \forall i = 1,2,\ldots,n.
	\end{align*}
\end{luuy}

%------------------------------------------------------------------------------%

\section{Điện Trở của Dây Dẫn -- Định Luật Ohm}

\begin{quyuoc}[Điện trở của thiết bị điện]
	Điện trở của ampe kế, dây nối, công tắc K rất nhỏ $\approx0$, được coi là $= 0$, còn điện trở của vôn kế là vô cùng lớn, i.e., $R = +\infty$.
\end{quyuoc}

\begin{baitoan}[\cite{SBT_Vat_Ly_9}, 2.1., p. 6]
	Hình sau vẽ đồ thị biểu diễn sự phụ thuộc của cường độ dòng điện vào hiệu điện thế của 3 dây dẫn khác nhau.
	\begin{figure}[H]
		\centering
		\includegraphics[scale=0.25]{SBT_2.1}
	\end{figure}
	\noindent(a) Từ đồ thị, xác định giá trị cường độ dòng điện chạy qua mỗi dây dẫn khi hiệu điện thế đặt giữa 2 đầu dây dẫn là \emph{3 V}. (b) Dây dẫn nào có điện trở lớn nhất? Nhỏ nhất? Giải thích bằng 3 cách khác nhau.
\end{baitoan}

\begin{baitoan}[\cite{SBT_Vat_Ly_9}, 2.2., p. 6]
	Cho điện trở $R = 15\ \Omega$. (a) Khi mắc điện trở này vào hiệu điện thế \emph{6 V} thì dòng điện chạy qua nó có cường độ bao nhiêu? (b) Muốn cường độ dòng điện chạy qua điện trở tăng thêm \emph{0.3 A} so với trường hợp trên thì hiệu điện thế đặt vào 2 đầu điện trở khi đó là bao nhiêu?
\end{baitoan}

\begin{baitoan}[\cite{SBT_Vat_Ly_9}, 2.3., p. 6]
	Làm thí nghiệm khảo sát sự phụ thuộc của cường độ dòng điện vào hiệu điện thế đặt giữa 2 đầu vật dẫn bằng kim loại, người ta thu được bảng số liệu:
	\begin{table}[H]
		\centering
		\begin{tabular}{|c|c|c|c|c|c|c|c|}
			\hline
			$U$ (V) & 0 & 1.5 & 3 & 4.5 & 6 & 7.5 & 9 \\
			\hline
			$I$ (A) & 0 & 0.31 & 0.61 & 0.9 & 1.29 & 1.49 & 1.78 \\
			\hline
		\end{tabular}
	\end{table}
	\noindent(a) Vẽ đồ thị biểu diễn sự phụ thuộc của $I$ vào $U$. (b) Dựa vào đồ thị ở (a), tính điện trở của vật dẫn nếu bỏ qua các sai số trong phép đo.
\end{baitoan}

\begin{baitoan}[\cite{SBT_Vat_Ly_9}, 2.4., p. 7]
	Cho mạch điện có sơ đồ:
	\begin{figure}[H]
		\centering
		\includegraphics[scale=0.25]{SBT_2.2}
	\end{figure}
	\noindent điện trở $R_1 = 10\ \Omega$, hiệu điện thế giữa 2 đầu đoạn mạch là $U_{\rm MN} = 12$ \emph{V}. (a) Tính cường độ dòng điện $I_1$ chạy qua $R_1$. (b) Giữ nguyên $U_{\rm MN} = 12$ \emph{V}, thay điện trở $R_1$ bằng điện trở $R_2$, khi đó ampe kế chỉ giá trị $I_2 = \frac{1}{2}I_1$. Tính điện trở $R_2$.
\end{baitoan}

\begin{baitoan}[\cite{SBT_Vat_Ly_9}, 2.5., p. 7]
	Điện trở của 1 dây dẫn nhất định có mối quan hệ phụ thuộc nào sau đây? {\sf A.} Tỷ lệ thuận với hiệu điện thế đặt vào 2 đầu dây dẫn. {\sf B.} Tỷ lệ nghịch với cường độ dòng điện chạy qua dây dẫn. {\sf C.} Không phụ thuộc vào hiệu điện thế đặt vào 2 đầu dây dẫn. {\sf D.} Giảm khi cường độ dòng điện chạy qua dây dẫn giảm.
\end{baitoan}

\begin{baitoan}[\cite{SBT_Vat_Ly_9}, 2.6., p. 7]
	Khi đặt 1 hiệu điện thế $U$ vào 2 đầu 1 điện trở $R$ thì dòng điện chạy qua nó có cường độ là $I$. Hệ thức nào dưới đây biểu thị định luật Ohm? {\sf A.} $U = \frac{I}{R}$. {\sf B.} $I = \frac{U}{R}$. {\sf C.} $I = \frac{R}{U}$. {\sf D.} $R = \frac{U}{I}$.
\end{baitoan}

\begin{baitoan}[\cite{SBT_Vat_Ly_9}, 2.7., p. 7]
	 Đơn vị nào dưới dây là đơn vị đo điện trở? {\sf A.} Ohm $\Omega$. {\sf B.} Watz W. {\sf C.} Ampe A. {\sf D.} Volt V.
\end{baitoan}

\begin{baitoan}[\cite{SBT_Vat_Ly_9}, 2.8., p. 7]
	Trong thí nghiệm khảo sát định luật Ohm, có thể làm thay đổi đại lượng nào trong số các đại lượng gồm hiệu điện thế, cường độ dòng điện, điện trở dây dẫn? {\sf A.} Chỉ thay đổi hiệu điện thế. {\sf B.} Chỉ thay đổi cường độ dòng điện. {\sf C.} Chỉ thay đổi điện trở dây dẫn. {\sf D.} Cả 3 đại lượng trên.
\end{baitoan}

\begin{baitoan}[\cite{SBT_Vat_Ly_9}, 2.9., p. 8]
	Dựa vào công thức $R = \frac{U}{I}$ có học sinh phát biểu như sau: ``Điện trở của dây dẫn tỷ lệ thuận với hiệu điện thế giữa 2 đầu dây \& tỷ lệ nghịch với cường độ dòng điện chạy qua dây.'' \emph{Đ\texttt{/}S?} Vì sao?
\end{baitoan}

\begin{baitoan}[\cite{SBT_Vat_Ly_9}, 2.10., p. 8]
	Đặt hiệu điện thế \emph{6 V} vào 2 đầu 1 điện trở thì dòng điện đi qua điện trở có cường độ \emph{0.15 A}. (a) Tính trị số của điện trở này. (b) Nếu tăng hiệu điện thế đặt vào 2 đầu điện trở này lên thành \emph{8 V} thì trị số của điện trở này có thay đổi không? Trị số của nó khi đó là bao nhiêu? Dòng điện đi qua nó khi đó có cường độ là bao nhiêu?
\end{baitoan}

\begin{baitoan}[\cite{SBT_Vat_Ly_9}, 2.11., p. 8]
	Giữa 2 đầu 1 điện trở $R_1 = 20\ \Omega$ có 1 hiệu điện thế là $U = 3.2$ \emph{V}. (a) Tính cường độ dòng điện $I_1$ đi qua điện trở này khi đó. (b) Giữ nguyên hiệu điện thế $U$ đã cho, thay điện trở $R_1$ bằng điện trở $R_2$ sao cho dòng điện đi qua $R_2$ có cường độ $I_2 = 0.8I_1$. Tính $R_2$.
\end{baitoan}

\begin{baitoan}[\cite{SBT_Vat_Ly_9}, 2.12., p. 8]
	Trên hình sau có vẽ đồ thị biểu diễn sự phụ thuộc của cường độ dòng điện vào hiệu điện thế đối với 2 điện trở $R_1,R_2$:
	\begin{figure}[H]
		\centering
		\includegraphics[scale=0.25]{SBT_2.3}
	\end{figure}
	\noindent(a) Từ đồ thị, tính trị số các điện trở $R_1,R_2$. (b) Tính cường độ dòng điện $I_1,I_2$ tương ứng đi qua mỗi điện trở khi lần lượt đặt hiệu điện thế $U = 1.8$ \emph{V} vào 2 đầu mỗi điện trở đó.
\end{baitoan}

%------------------------------------------------------------------------------%

\section{Đoạn Mạch Nối Tiếp}

\begin{baitoan}[\cite{SBT_Vat_Ly_9}, 4.1., p. 9]
	2 điện trở $R_1,R_2$ \& ampe kế được mắc nối tiếp với nhau vào 2 điểm A, B. (a) Vẽ sơ đồ mạch điện này. (b) Cho $R_1 = 5\ \Omega$, $R_2 = 10\ \Omega$, ampe kế chỉ \emph{0.2 A}. Tính hiệu điện thế của đoạn mạch AB theo 2 cách.
\end{baitoan}

\begin{baitoan}[\cite{SBT_Vat_Ly_9}, 4.2., p. 9]
	1 điện trở $10\ \Omega$ được mắc vào hiệu điện thế \emph{12 V}. (a) Tính cường độ dòng điện chạy qua điện trở đó. (b) Muốn kiểm tra kết quả tính ở trên, ta có thể dùng ampe kế để đo. Muốn ampe kế chỉ đúng giá trị cường độ dòng điện đã tính được phải có điều kiện gì đối với ampe kế? Vì sao?
\end{baitoan}

\begin{baitoan}[\cite{SBT_Vat_Ly_9}, 4.3., p. 9]
	Cho mạch điện có sơ đồ:
	\begin{figure}[H]
		\centering
		\includegraphics[scale=0.25]{SBT_4.1}
	\end{figure}
	\noindent trong đó điện trở $R_1 = 10\ \Omega$, $R_2 = 20\ \Omega$, hiệu điện thế giữa 2 đầu đoạn mạch \emph{AB} bằng \emph{12 V}. (a) Số chỉ của vôn kế \& ampe kế là bao nhiêu? (b) Chỉ với 2 điện trở này, nêu 2 cách làm tăng cường độ dòng điện trong mạch lên gấp $3$ lần (có thể thay đổi $U_{\rm AB}$.
\end{baitoan}

\begin{baitoan}[\cite{SBT_Vat_Ly_9}, 4.4., p. 9]
	Cho mạch điện có sơ đồ:
	\begin{figure}[H]
		\centering
		\includegraphics[scale=0.25]{SBT_4.2}
	\end{figure}
	\noindent trong đó điện trở $R_1 = 5\ \Omega$, $R_2 = 15\ \Omega$, vôn kế chỉ \emph{3 V}. (a) Số chỉ của ampe kế là bao nhiêu? (b) Tính hiệu điện thế giữa 2 đầu \emph{AB} của đoạn mạch.
\end{baitoan}

\begin{baitoan}[\cite{SBT_Vat_Ly_9}, 4.5., p. 10]
	3 điện trở có các giá trị là $10\ \Omega,20\ \Omega, 30 \ \Omega$. Có thể mắc các điện trở này như thế nào vào mạch có hiệu điện thế \emph{12 V} để dòng điện trong mạch có cường độ \emph{0.4 A}? Vẽ sơ đồ các cách mắc đó.
\end{baitoan}

\begin{baitoan}[\cite{SBT_Vat_Ly_9}, 4.6., p. 10]
	Cho 2 điện trở, $R_1 = 20\ \Omega$ chịu được dòng điện có cường độ tối đa \emph{2 A} \& $R_2 = 40\ \Omega$ chịu được dòng điện có cường độ tối đa \emph{1.5 A}. Tính hiệu điện thế tối đa có thể đặt vào 2 đầu đoạn mạch gồm $R_1$ nối tiếp $R_2$.
\end{baitoan}

\begin{baitoan}[\cite{SBT_Vat_Ly_9}, 4.7., p. 10]
	3 điện trở $R_1 = 5\ \Omega$, $R_2 = 10\ \Omega$, $R_3 = 15\ \Omega$ được mắc nối tiếp nhau vào hiệu điện thế \emph{12 V}. (a) Tính điện trở tương đương của đoạn mạch. (b) Tính hiệu điện thế giữa 2 đầu mỗi điện trở.
\end{baitoan}

\begin{baitoan}[\cite{SBT_Vat_Ly_9}, 4.8., p. 10]
	Đặt hiệu điện thế $U = 12$ \emph{V} vào 2 đầu đoạn mạch gồm điện trở $R_1 = 40\ \Omega$ \& $R_2 = 80\ \Omega$ mắc nối tiếp. Tính cường độ dòng điện chạy qua đoạn mạch này.
\end{baitoan}

\begin{baitoan}[\cite{SBT_Vat_Ly_9}, 4.9., p. 10]
	1 đoạn mạch gồm 2 điện trở $R_1$ \& $R_2 = 1.5R_1$ mắc nối tiếp với nhau. Cho dòng điện chạy qua đoạn mạch này thì thấy hiệu điện thế giữa 2 đầu điện trở $R_1$ là \emph{3 V}. Tính hiệu điện thế giữa 2 đầu đoạn mạch.
\end{baitoan}

\begin{baitoan}[\cite{SBT_Vat_Ly_9}, 4.10., p. 10]
	Phát biểu nào sau đây không đúng đối với đoạn mạch gồm các điện trở mắc nối tiếp? {\sf A.} Cường độ dòng điện là như nhau tại mọi vị trí của đoạn mạch. {\sf B.} Hiệu điện thế giữa 2 đầu đoạn mạch bằng tổng các hiệu điện thế giữa 2 đầu mỗi điện trở mắc trong đoạn mạch. {\sf C.} Hiệu điện thế giữa 2 đầu đoạn mạch bằng hiệu điện thế giữa 2 đầu mỗi điện trở mắc trong đoạn mạch. {\sf D.} Hiệu điện thế giữa 2 đầu mỗi điện trở mắc trong đoạn mạch tỷ lệ thuận với điện trở đó.
\end{baitoan}

\begin{baitoan}[\cite{SBT_Vat_Ly_9}, 4.11., p. 10]
	Đoạn mạch gồm các điện trở mắc nối tiếp là đoạn mạch không có đặc điểm nào sau đây? {\sf A.} Đoạn mạch có các điểm nối chung của nhiều điện trở. {\sf B.} Đoạn mạch có các điểm nối chung chỉ của 2 điện trở. {\sf C.} Dòng điện chạy qua các điện trở của đoạn mạch có cùng cường độ. {\sf D.} Đoạn mạch gồm các điện trở mắc liên tiếp với nhau \& không có mạch rẽ.
\end{baitoan}

\begin{baitoan}[\cite{SBT_Vat_Ly_9}, 4.12., p. 10]
	Đặt 1 hiệu điện thế $U_{\rm AB}$ vào 2 đầu đoạn mạch gồm 2 điện trở $R_1,R_2$ mắc nối tiếp. Hiệu điện thế giữa 2 đầu mỗi điện trở tương ứng là $U_1,U_2$. Hệ thức nào sau đây là không đúng? {\sf A.} $R_{\rm AB} = R_1 + R_2$. {\sf B.} $I_{\rm AB} = I_1 = I_2$. {\sf C.} $\frac{U_1}{U_2} = \frac{R_2}{R_1}$. {\sf D.} $U_{\rm AB} = U_1 + U_2$.
\end{baitoan}

\begin{baitoan}[\cite{SBT_Vat_Ly_9}, 4.13., p. 10]
	Đặt 1 hiệu điện thế $U$ vào 2 đầu 1 đoạn mạch có sơ đồ:
	\begin{figure}[H]
		\centering
		\includegraphics[scale=0.25]{SBT_4.3}
	\end{figure}
	\noindent trong đó các điện trở $R_1 = 3\ \Omega$, $R_2 = 6\ \Omega$. Hỏi số chỉ của ampe kế khi công tắc K đóng lớn hơn hay nhỏ hơn bao nhiêu lần so với khi công tắc K mở? {\sf A.} Nhỏ hơn $2$ lần. {\sf B.} Lớn hơn $2$ lần. {\sf C.} Nhỏ hơn $3$ lần. {\sf D.} Lớn hơn $3$ lần.
\end{baitoan}

\begin{baitoan}[\cite{SBT_Vat_Ly_9}, 4.14., p. 10]
	Đặt 1 hiệu điện thế $U = 6$ \emph{V} vào 2 đầu đoạn mạch gồm 3 điện trở $R_1 = 3\ \Omega$, $R_2 = 5\ \Omega$, \& $R_3 = 7\ \Omega$ mắc nối tiếp. (a) Tính cường độ dòng điện chạy qua mỗi điện trở của đoạn mạch này. (b) Tính hiệu điện thế giữa 2 đầu điện trở của 3 điện trở đã cho.
\end{baitoan}

\begin{baitoan}[\cite{SBT_Vat_Ly_9}, 4.15., p. 11]
	Đặt 1 hiệu điện thế $U$ vào 2 đầu đoạn mạch có sơ đồ:
	\begin{figure}[H]
		\centering
		\includegraphics[scale=0.25]{SBT_4.4}
	\end{figure}
	\noindent trong đó điện trở $R_1 = 4\ \Omega$, $R_2 = 5\ \Omega$. (a) Cho biết số chỉ của ampe kế khi công tắc K mở \& khi K đong hơn kém nhau $3$ lần. Tính điện trở $R_3$. (b) Cho biết $U = 5.4$ \emph{V}. Số chỉ của ampe kế khi công tắc K mở là bao nhiêu?
\end{baitoan}

\begin{baitoan}[\cite{SBT_Vat_Ly_9}, 4.16., p. 11]
	Đặt 1 hiệu điện thế $U$ vào 2 đầu 1 đoạn mạch có sơ đồ:
	\begin{figure}[H]
		\centering
		\includegraphics[scale=0.25]{SBT_4.5}
	\end{figure}
	\noindent Khi đóng công tắc K vào vị trí 1 thì ampe kế có số chỉ $I_1 = 1$, khi chuyển công tắc này sang vị trí số 2 thì ampe kế có số chỉ là $I_2 = \frac{1}{3}I$, còn khi chuyển K sang vị trí 3 thì ampe kế có số chỉ $I_3 = \frac{1}{8}I$. Cho biết $R_1 = 3\ \Omega$, tính $R_2,R_3$.
\end{baitoan}

%------------------------------------------------------------------------------%

\section{Đoạn Mạch Song Song}

%------------------------------------------------------------------------------%

\section{Bài Tập Vận Dụng Định Luật Ohm}

%------------------------------------------------------------------------------%

\section{Sự Phụ Thuộc của Điện Trở vào Chiều Dài Dây Dẫn}

%------------------------------------------------------------------------------%

\section{Sự Phụ Thuộc của Điện Trở vào Tiết Diện Dây Dẫn}

%------------------------------------------------------------------------------%

\section{Sự Phụ Thuộc của Điện Trở vào Vật Liệu Làm Dây Dẫn}

%------------------------------------------------------------------------------%

\section{Biến Trở -- Điện Trở Dùng Trong Kỹ Thuật}

%------------------------------------------------------------------------------%

\section{Bài tập Vận Dụng Định Luật Ohm \& Công Thức Tính Điện Trở của Dây Dẫn}

%------------------------------------------------------------------------------%

\section{Công Suất Điện}

%------------------------------------------------------------------------------%

\section{Điện Năng -- Công của Dòng Điện}

%------------------------------------------------------------------------------%

\section{Bài Tập về Công Suất Điện \& Điện Năng Sử Dụng}

%------------------------------------------------------------------------------%

\section{Định Luật Jule-Lenz \& Bài Tập Vận Dụng}

%------------------------------------------------------------------------------%

\section{Sử Dụng An Toàn \& Tiết Kiệm Điện}

%------------------------------------------------------------------------------%

\section{Miscellaneous}

%------------------------------------------------------------------------------%

\printbibliography[heading=bibintoc]
	
\end{document}