\documentclass{article}
\usepackage[backend=biber,natbib=true,style=alphabetic,maxbibnames=50]{biblatex}
\addbibresource{/home/nqbh/reference/bib.bib}
\usepackage{tocloft}
\renewcommand{\cftsecleader}{\cftdotfill{\cftdotsep}}
\usepackage[colorlinks=true,linkcolor=blue,urlcolor=red,citecolor=magenta]{hyperref}
\usepackage{amsmath,amssymb,amsthm,caption,float,graphicx,mathtools,subcaption}
\allowdisplaybreaks
\numberwithin{equation}{section}
\newtheorem{assumption}{Assumption}[section]
\newtheorem{conjecture}{Conjecture}[section]
\newtheorem{corollary}{Corollary}[section]
\newtheorem{definition}{Definition}[section]
\newtheorem{example}{Example}[section]
\newtheorem{lemma}{Lemma}[section]
\newtheorem{notation}{Notation}[section]
\newtheorem{principle}{Principle}[section]
\newtheorem{problem}{Problem}[section]
\newtheorem{proposition}{Proposition}[section]
\newtheorem{question}{Question}[section]
\newtheorem{remark}{Remark}[section]
\newtheorem{theorem}{Theorem}[section]
\usepackage[left=1cm,right=1cm,top=5mm,bottom=5mm,footskip=4mm]{geometry}
\def\labelitemii{$\circ$}

\title{The Farlex Grammar Book: Complete English Grammar Rules: Examples, Exceptions, Exercises, \& Everything You Need to Master Proper Grammar}
\author{Peter Herring}
\date{\today}

\begin{document}
\maketitle
\tableofcontents

%------------------------------------------------------------------------------%

\section*{Preface}
``Grammar is without a doubt 1 of the most daunting aspects of the English language, an area riddled with complexities, inconsistencies, \& contradictions. It has also been in a state of flux for pretty much its entire existence. For native speakers of English, as well as for those learning it as a new language, grammar presents a very serious challenge to speaking \& writing both accurately \& effectively.

Having a single, reliable, go-to reference guide should therefore be indispensable to those trying to learn, improve, or perfect their speech or writing. This book is that guide: a clear, unambiguous, \& comprehensive source of information that covers all the relevant topics of English grammar, while still being easy to understand \& enjoyable to read.

Every topic in the book has been broken down into basic units. Each unit can be read \& understood in its own right, but throughout the book you will find cross-references to other sections \& chapters to help make it clear how all the pieces fit together. If you're having trouble understanding something, try going back (or forward) to other related topics in the book.

Finally, it must be mentioned that, because English is such a flexible, inconsistent language, the ``rules'' that are often bandied about are usually not rules at all, but rather guides that reflect how the language is used. Accordingly, the guidelines contained within this book are just that -- guidelines. They are not intended to provide constrictive or proscriptive rules that confine everyone to a particular way of speaking or writing.

Learning how the English language works will enhance your engagement with speech \& writing every day, from the books you read, to the e-mails you write, to the conversations you have with friends \& strangers alike.

As such, mastering grammar is not an exercise that is confined to the classroom. While it is certainly important to learn the structures, styles, \& rules that shape the language, the key to truly learning English is to read \& listen to the way people write \& speak every day, from the most well-known authors to the people you talk to on the bus. Take the information you find in this book \& carry it with you into the world. - P. Herring'' -- \cite[p. 9]{Herring2016}

%------------------------------------------------------------------------------%

\section{English Grammar}
``\textit{Grammar} refers to the way words are used, classified, \& structured together to form coherent written or spoken communication.

This guide takes a traditional approach to teaching English grammar, breaking the topic into 3 fundamental elements: \textit{Parts of Speech, Inflection}, \& \textit{Syntax}. Each of these is a discrete, individual part, but they are all intrinsically linked together in meaning.'' -- \cite[p. 11]{Herring2016}

\subsection{Parts of Speech}
``In the 1st part of the guide, we will look at the basic components of English -- words. The \textit{parts of speech} are the categories to which different words are assigned, based on their meaning, structure, \& function in a sentence.

We'll look in great detail at the 7 main parts of speech -- \textit{nouns, pronouns, verbs, adjectives, adverbs, prepositions}, \& \textit{conjunctions} -- as well as other categories of words that don't easily fit in with the rest, such as \textit{particles, determiners}, \& \textit{gerunds}.

By understanding the parts of speech, we can better understand how (\& why) we structure words together to form sentences.'' -- \cite[p. 11]{Herring2016}

\subsection{Inflection}
``Although the parts of speech provide the building blocks for English, another very important element is \textit{inflection}, the process by which words are \textit{changed} in form to create new, specific meanings.

There are 2 main categories of inflection: \textit{conjugation} \& \textit{declension}. Conjugation refers to the inflection of verbs, while declension refers to the inflection of nouns, pronouns, adjectives, \& adverbs. Whenever we change a verb from the \textit{present tense} to the \textit{past tense}, e.g., we are using \textit{conjugation}. Likewise, when we make a noun \textit{plural} to show that there is more than 1 of it, we are using \textit{declension}.'' -- \cite[p. 11]{Herring2016}

\subsection{Syntax}
``The 3rd \& final part of the guide will focus on \textit{syntax}, the rules \& patterns that govern how we \textit{structure} sentences. The grammatical structures that constitute syntax can be thought of as a hierarchy, with sentences at the top as the largest cohesive unit in the language \& words (the parts of speech) at the bottom.

We'll begin the 3rd part by looking at the basic structural units present in all sentences -- \textit{subjects \& predicates} -- \& progressively move on to larger classes of structures, discussing \textit{modifiers, phrases}, \& \textit{clauses}. Finally, we will end by looking at the different structures \& categories of \textit{sentences} themselves.'' -- \cite[pp. 11--12]{Herring2016}

\subsection{Using the 3 parts together}
``The best way to approach this guide is to think of it as a cross-reference of itself; when you see a term or concept in 1 section that you're unfamiliar with, check the other sections to find a more thorough explanation. Neither parts of speech nor inflection nor syntax exists as truly separate units; it's equally important to examine \& learn about the different \underline{kinds} of words, how they can \textit{change} to create new meaning, \& the guidelines by which they are \textit{structured} into sentences.

When we learn to use all 3 parts together, we gain a much fuller understanding of how to make our speech \& writing not only proper, but natural \& effective.'' -- \cite[p. 12]{Herring2016}

\section{Parts of Speech}

\begin{definition}
	The {\rm parts of speech} are the primary categories of words according to their function in a sentence.
\end{definition}
English has 7 main parts of speech. We'll look at a brief overview of each below; continue on to their individual chapters to learn more about them.

\subsection{Nouns}

%------------------------------------------------------------------------------%

\printbibliography[heading=bibintoc]
	
\end{document}