\documentclass{article}
\usepackage[backend=biber,natbib=true,style=authoryear]{biblatex}
\addbibresource{/home/hong/1_NQBH/reference/bib.bib}
\usepackage[utf8]{vietnam}
\usepackage{tocloft}
\renewcommand{\cftsecleader}{\cftdotfill{\cftdotsep}}
\usepackage[colorlinks=true,linkcolor=blue,urlcolor=red,citecolor=magenta]{hyperref}
\usepackage{amsmath,amssymb,amsthm,mathtools,float,graphicx,algpseudocode,algorithm,tcolorbox,tikz,tkz-tab,subcaption}
\DeclareMathOperator{\arccot}{arccot}
\usepackage[inline]{enumitem}
\allowdisplaybreaks
\numberwithin{equation}{section}
\newtheorem{assumption}{Assumption}[section]
\newtheorem{nhanxet}{Nhận xét}[section]
\newtheorem{conjecture}{Conjecture}[section]
\newtheorem{corollary}{Corollary}[section]
\newtheorem{hequa}{Hệ quả}[section]
\newtheorem{definition}{Definition}[section]
\newtheorem{dinhnghia}{Định nghĩa}[section]
\newtheorem{example}{Example}[section]
\newtheorem{vidu}{Ví dụ}[section]
\newtheorem{lemma}{Lemma}[section]
\newtheorem{notation}{Notation}[section]
\newtheorem{principle}{Principle}[section]
\newtheorem{problem}{Problem}[section]
\newtheorem{baitoan}{Bài toán}[section]
\newtheorem{proposition}{Proposition}[section]
\newtheorem{menhde}{Mệnh đề}[section]
\newtheorem{question}{Question}[section]
\newtheorem{cauhoi}{Câu hỏi}[section]
\newtheorem{quytac}{Quy tắc}
\newtheorem{remark}{Remark}[section]
\newtheorem{luuy}{Lưu ý}[section]
\newtheorem{theorem}{Theorem}[section]
\newtheorem{tiende}{Tiên đề}[section]
\newtheorem{dinhly}{Định lý}[section]
\usepackage[left=0.5in,right=0.5in,top=1.5cm,bottom=1.5cm]{geometry}
\usepackage{fancyhdr}
\pagestyle{fancy}
\fancyhf{}
\lhead{\small Subsect.~\thesubsection}
\rhead{\small\nouppercase{\leftmark}}
\renewcommand{\subsectionmark}[1]{\markboth{#1}{}}
\cfoot{\thepage}
\def\labelitemii{$\circ$}

\title{Some Topics in Elementary Natural Science\texttt{/}Grades 6 \& 7}
\author{Nguyễn Quản Bá Hồng\footnote{Independent Researcher, Ben Tre City, Vietnam\\e-mail: \texttt{nguyenquanbahong@gmail.com}; website: \url{https://nqbh.github.io}.}}
\date{\today}

\begin{document}
\maketitle
\begin{abstract}
	
\end{abstract}
\setcounter{secnumdepth}{4}
\setcounter{tocdepth}{3}
\tableofcontents
\newpage

%------------------------------------------------------------------------------%

\section{Chất \& Sự Biến Đổi của Chất}

\subsection{Nguyên Tử}

%------------------------------------------------------------------------------%

\subsection{Nguyên Tố Hóa Học}

%------------------------------------------------------------------------------%

\subsection{Sơ Lược về Bảng Tuần Hoàn Các Nguyên Tố Hóa Học}

%------------------------------------------------------------------------------%

\subsection{Phân Tử, Đơn Chất, Hợp Chất}

%------------------------------------------------------------------------------%

\subsection{Giới Thiệu về Liên Kết Hóa Học}

%------------------------------------------------------------------------------%

\subsection{Hóa Trị, Công Thức Hóa Học}

%------------------------------------------------------------------------------%

\section{Năng Lượng \& Sự Biến Đổi}

\subsection{Tốc Độ của Chuyển Động}

%------------------------------------------------------------------------------%

\subsection{Đồ Thị Quãng Đường -- Thời Gian}

%------------------------------------------------------------------------------%

\subsection{Sự Truyền Âm}

%------------------------------------------------------------------------------%

\subsection{Biên Độ, Tần Số, Độ To \& Độ Cao của Âm}

%------------------------------------------------------------------------------%

\subsection{Phản Xạ Âm}

%------------------------------------------------------------------------------%

\subsection{Ánh Sáng, Tia Sáng}

%------------------------------------------------------------------------------%

\subsection{Sự Phản Xạ Ánh Sáng}

%------------------------------------------------------------------------------%

\subsection{Nam Châm}

%------------------------------------------------------------------------------%

\subsection{Từ Trường}

%------------------------------------------------------------------------------%

\subsection{Từ Trường Trái Đất}

%------------------------------------------------------------------------------%

\section{Vật Sống}

\subsection{Vai Trò của Trao Đổi Chất \& Chuyển Hóa Năng Lượng ở Sinh Vật}

%------------------------------------------------------------------------------%

\subsection{Quang Hợp ở Thực Vật}

%------------------------------------------------------------------------------%

\subsection{Các Yếu Tố Ảnh Hưởng Đến Quang Hợp}

%------------------------------------------------------------------------------%

\subsection{Thực Hành về Quang Hợp ở Cây Xanh}

%------------------------------------------------------------------------------%

\subsection{Hô Hấp Tế Bào}

%------------------------------------------------------------------------------%

\subsection{Các Yếu Tố Ảnh Hưởng Đến Hô Hấp Tế Bào}

%------------------------------------------------------------------------------%

\subsection{Trao Đổi Khí ở Sinh Vật}

%------------------------------------------------------------------------------%

\subsection{Vai Trò của Nước \& Các Chất Dinh Dưỡng Đối với Cơ Thể Sinh Vật}

%------------------------------------------------------------------------------%

\subsection{Trao Đổi Nước \& Các Chất Dinh Dưỡng ở Thực Vật}

%------------------------------------------------------------------------------%

\subsection{Trao Đổi Nước \& Các Chất Dinh Dưỡng ở Động Vật}

%------------------------------------------------------------------------------%

\subsection{Khái Quát về Cảm Ứng \& Cảm Ứng ở Thực Vật}

%------------------------------------------------------------------------------%

\subsection{Tập Tính ở Động Vật}

%------------------------------------------------------------------------------%

\subsection{Khái Quát về Sinh Trưởng \& Phát Triển ở Sinh Vật}

%------------------------------------------------------------------------------%

\subsection{Sinh Trưởng \& Phát Triển ở Thực Vật}

%------------------------------------------------------------------------------%

\subsection{Sinh Trưởng \& Phát Triển ở Động Vật}

%------------------------------------------------------------------------------%

\subsection{Khái Quát về Sinh Sinh \& Sinh Sản Vô Tính ở Sinh Vật}

%------------------------------------------------------------------------------%

\subsection{Sinh Sản Hữu Tính ở Sinh Vật}

%------------------------------------------------------------------------------%

\subsection{Các Yếu Tố Ảnh Hưởng đến Sinh Sản \& Điều Khiển Sinh Sản ở Sinh Vật}

%------------------------------------------------------------------------------%

\subsection{Sự Thống Nhất về Cấu Trúc \& Các Hoạt Động Sống trong Cơ Thể Sinh Vật}

%------------------------------------------------------------------------------%

\printbibliography[heading=bibintoc]
	
\end{document}