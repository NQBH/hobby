\documentclass{article}
\usepackage[backend=biber,natbib=true,style=authoryear]{biblatex}
\addbibresource{/home/hong/1_NQBH/reference/bib.bib}
\usepackage[utf8]{vietnam}
\usepackage{tocloft}
\renewcommand{\cftsecleader}{\cftdotfill{\cftdotsep}}
\usepackage[colorlinks=true,linkcolor=blue,urlcolor=red,citecolor=magenta]{hyperref}
\usepackage{amsmath,amssymb,amsthm,mathtools,float,graphicx,algpseudocode,algorithm,tcolorbox,tikz,tkz-tab,subcaption}
\DeclareMathOperator{\arccot}{arccot}
\usepackage[inline]{enumitem}
\allowdisplaybreaks
\numberwithin{equation}{section}
\newtheorem{assumption}{Assumption}[section]
\newtheorem{nhanxet}{Nhận xét}[section]
\newtheorem{conjecture}{Conjecture}[section]
\newtheorem{corollary}{Corollary}[section]
\newtheorem{hequa}{Hệ quả}[section]
\newtheorem{definition}{Definition}[section]
\newtheorem{dinhnghia}{Định nghĩa}[section]
\newtheorem{example}{Example}[section]
\newtheorem{vidu}{Ví dụ}[section]
\newtheorem{lemma}{Lemma}[section]
\newtheorem{notation}{Notation}[section]
\newtheorem{principle}{Principle}[section]
\newtheorem{problem}{Problem}[section]
\newtheorem{baitoan}{Bài toán}[section]
\newtheorem{proposition}{Proposition}[section]
\newtheorem{menhde}{Mệnh đề}[section]
\newtheorem{question}{Question}[section]
\newtheorem{cauhoi}{Câu hỏi}[section]
\newtheorem{quytac}{Quy tắc}
\newtheorem{remark}{Remark}[section]
\newtheorem{luuy}{Lưu ý}[section]
\newtheorem{theorem}{Theorem}[section]
\newtheorem{tiende}{Tiên đề}[section]
\newtheorem{dinhly}{Định lý}[section]
\usepackage[left=0.5in,right=0.5in,top=1.5cm,bottom=1.5cm]{geometry}
\usepackage{fancyhdr}
\pagestyle{fancy}
\fancyhf{}
\lhead{\small Subsect.~\thesubsection}
\rhead{\small\nouppercase{\leftmark}}
\renewcommand{\subsectionmark}[1]{\markboth{#1}{}}
\cfoot{\thepage}
\def\labelitemii{$\circ$}

\title{Some Topics in Elementary Natural Science\texttt{/}Grades 6 \& 7}
\author{Nguyễn Quản Bá Hồng\footnote{Independent Researcher, Ben Tre City, Vietnam\\e-mail: \texttt{nguyenquanbahong@gmail.com}; website: \url{https://nqbh.github.io}.}}
\date{\today}

\begin{document}
\maketitle
\begin{abstract}
	
\end{abstract}
\setcounter{secnumdepth}{4}
\setcounter{tocdepth}{3}
\tableofcontents
\newpage

%------------------------------------------------------------------------------%

\section{Chất \& Sự Biến Đổi của Chất}

\subsection{Nguyên Tử}
\textsf{\textbf{Nội dung.} Mô hình nguyên tử của Rutherford--Bohr -- mô hình sắp xếp electron trong lớp vỏ nguyên tử, khối lượng của 1 nguyên tử theo đơn vị quốc tế amu (đơn vị khối lượng nguyên tử).}

``Khoảng năm 440 trước Công nguyên, nhà triết học Hy Lạp, Democritus cho rằng nếu chia nhỏ nhiều lần 1 đồng tiền vàng cho đến khi ``không thể phân chia được nữa'', thì sẽ được 1 hạt gọi là \textit{nguyên tử}. ``Nguyên tử'' trong tiếng Hy Lạp là atomos, i.e., ``không chia nhỏ hơn được nữa''.'' -- \cite[p. 10]{SGK_KHTN_7_Canh_Dieu}

\subsubsection{Khái niệm nguyên tử}
``Các nhà khoa học hiện nay đã tìm thấy hàng chục triệu chất khác nhau. Tuy nhiên, khi phân tích các chất đó, người ta thấy mọi chất đều được cấu tạo từ những \textit{hạt cực kỳ nhỏ bé, không mang điện}. Những hạt đó được gọi là \textit{nguyên tử}. E.g., đồng tiền vàng được cấu tạo từ các nguyên tử vàng (gold). Khí oxygen được cấu tạo từ các nguyên tử oxygen. Kim cương, than chì đều được cấu tạo từ các nguyên tử carbon. Nước được tạo nên từ các nguyên tử hydrogen \& oxygen. Đường ăn được tạo nên từ các nguyên tử carbon, oxygen, \& hydrogen.'' ``\textit{Nguyên tử nhỏ cỡ nào?} Có thể coi nguyên tử như những quả cầu cực nhỏ. Đường kính của nguyên tử nhỏ hơn đường kính của sợi tóc khoảng $100000$--$500000$ lần. Vì thế, không thể quan sát nguyên tử bằng mắt hoặc các kính hiển vi thông thường. Đường kính sợi tóc là $0.1$mm.'' -- \cite[p. 10]{SGK_KHTN_7_Canh_Dieu}

\subsubsection{Cấu tạo nguyên tử}
``Nguyên tử được coi như 1 quả cầu, gồm vỏ nguyên tử \& hạt nhân nguyên tử.'' -- \cite[p. 11]{SGK_KHTN_7_Canh_Dieu}

\paragraph{Vỏ nguyên tử.} ``Vỏ nguyên tử được tạo bởi 1 hay nhiều electron chuyển động xung quanh hạt nhân. Electron ký hiệu là e, mang điện tích âm \& có giá trị bằng 1 điện tích nguyên tố (1 điện tích nguyên tố $= 1.602\cdot 10^{-19}$ coulomb (C)), được viết đơn giản là $-1$.'' -- \cite[p. 11]{SGK_KHTN_7_Canh_Dieu}

\paragraph{Hạt nhân nguyên tử.} ``Hạt nhân nguyên tử được tạo bởi các proton \& neutron.
\begin{enumerate*}
	\item[$\bullet$] Proton ký hiệu là p, mang điện tích dương \& có giá trị bằng 1 điện tích nguyên tố, được viết là $+1$. Điện tích của proton bằng điện tích của electron về độ lớn nhưng khác dấu.
	\item[$\bullet$] Neutron ký hiệu là n, không mang điện.
\end{enumerate*}
Điện tích của hạt nhân nguyên tử bằng tổng điện tích của các proton. Số đơn vị điện tích hạt nhân bằng số proton. E.g.: nguyên tử nitrogen (nitơ) có $7$ proton nên nitrogen có điện tích hạt nhân là $+7$, số đơn vị điện tích hạt nhân là $7$. Trong nguyên tử, số electron bằng số proton. E.g.: nguyên tử helium gồm hạt nhân có $2$ proton, $2$ neutron, \& vỏ nguyên tử có $2$ electron. Kích thước của hạt nhân rất nhỏ so với kích thước của nguyên tử. Nếu coi hạt nhân là quả bóng có đường kính là $10$cm thì nguyên tử sẽ là quả cầu khổng lồ với đường kính là $1$km (lớp gấp $10000$ lần kích thước hạt nhân nguyên tử).'' -- \cite[p. 11]{SGK_KHTN_7_Canh_Dieu}

``Aluminium là kim loại có nhiều ứng dụng trong thực tiễn, được dùng làm dây dẫn điện, chế tạo các thiết bị, máy móc trong công nghiệp \& nhiều đồ dùng sinh hoạt. Tổng số hạt trong hạt nhân nguyên tử aluminium là $27$, số đơn vị điện tích hạt nhân là $13$.'' ``Helium có 2 proton, mỗi proton có điện tích $+1$, tổng số điện tích: $+2$; có 2 electron, mỗi electron có điện tích $-1$, tổng số điện tích: $-2$. Tổng điện tích trong nguyên tử helium bằng $0$. Ta nói nguyên tử không mang điện hay trung hòa về điện.'' -- \cite[p. 12]{SGK_KHTN_7_Canh_Dieu}

\subsubsection{Sự chuyển động của electron trong nguyên tử}
``Theo mô hình của Rutherford--Bohr trong nguyên tử, các electron chuyển động trên những quỹ đạo xác định xung quanh hạt nhân, như các hành tinh quay quanh Mặt Trời. Trong nguyên tử, các electron được xếp thành từng lớp. Các electron được sắp xếp lần lượt vào các lớp theo chiều từ gần hạt nhân ra ngoài. Mỗi lớp có số electron tối đa xác định, như lớp thứ nhất có tối đa $2$ electron, lớp thứ 2 có tối đa $8$ electron, $\ldots$. E.g., nguyên tử oxygen có $8$ electron, được phân bố thành $2$ lớp electron, lớp thứ nhất có 2 electron, lớp thứ 2 có $6$ electron. Ta nói nguyên tử oxygen có $6$ electron ở lớp ngoài cùng.''

``Ernest Rutherford (1871--1937), nhà vật lý người New Zealand, đã đưa ra mô hình hành tinh nguyên tử để giải thích cấu tạo nguyên tử. Năm 1911, ông đã khám phá ra hầu hết các nguyên tử có cấu tạo rỗng, gồm hạt nhân ở giữa tích điện dương \& vỏ nguyên tử gồm các electron tích điện âm. Mô hình hành tinh nguyên tử của Rutherford chưa mô tả được sự phân bố electron trong vỏ nguyên tử. Sau đó, nhà vật lý người Đan Mạch, Niels Bohr đã đề xuất 1 mô hình mới chỉ rõ các electron được sắp xếp trên các lớp khác nhau.'' -- \cite[p. 12]{SGK_KHTN_7_Canh_Dieu}

``Trong số các nguyên tử đã biết hiện nay, nguyên tử có kích thước lớn nhất là francium, có chứa $7$ lớp electron. Nguyên tử helium có kích thước nhỏ nhất với $1$ lớp electron.'' -- \cite[p. 13]{SGK_KHTN_7_Canh_Dieu}

\subsubsection{Khối lượng nguyên tử}
``Nguyên tử có khối lượng rất nhỏ. 1 gam của bất kỳ chất nào cũng chứa tới hàng tỷ tỷ nguyên tử. Do vậy, để biểu thị khối lượng của nguyên tử, người ta dùng đơn vị khối lượng nguyên tử, ký hiệu là amu (atomic mass unit). $1$amu $= 1.6605\cdot 10^{-24}$g. Khối lượng của 1 nguyên tử bằng tổng khối lượng của proton, neutron, \& electron trong nguyên tử đó. Proton \& neutron đều có khối lượng xấp xỉ bằng $1$amu. Khối lượng của electron là $0.00055$amu, nhỏ hơn nhiều lần so với khối lượng của proton \& neutron nên có thể coi khối lượng nguyên tử bằng khối lượng hạt nhân. E.g.: Nguyên tử hydrogen chỉ có $1$ proton, nên khối lượng nguyên tử hydrogen là $1$amu. Nguyên tử oxygen có $8$ proton \& $8$ neutron, nên khối lượng nguyên tử oxygen là: $8\cdot 1 + 8\cdot 1 = 16$amu.'' -- \cite[p. 13]{SGK_KHTN_7_Canh_Dieu}

``Ruột của bút chì thường được làm từ than chì \& đất sét. Than chì được cấu tạo từ các nguyên tử carbon.''
\vspace{2mm}

\noindent\textsc{Tóm tắt kiến thức.}
``\begin{enumerate*}
	\item[$\bullet$] Nguyên tử là những hạt cực kỳ nhỏ bé, không mang điện, cấu tạo nên chất. Cấu tạo nguyên tử gồm hạt nhân \& vỏ nguyên tử.
	\item[$\bullet$] Hạt nhân của nguyên tử mang điện tích dương, được tạo bởi các proton \& neutron. Vỏ nguyên tử gồm 1 hay nhiều electron mang điện tích âm.
	\item[$\bullet$] Theo mô hình Rutherford--Bohr, trong nguyên tử, electron phân bố trên các lớp electron \& chuyển động quanh hạt nhân nguyên tử trên những quỹ đạo xác định.
	\item[$\bullet$] Khối lượng nguyên tử được coi bằng tổng khối lượng của proton \& neutron có trong nguyên tử, được tính bằng đơn vị amu.'' -- \cite[p. 14]{SGK_KHTN_7_Canh_Dieu}
\end{enumerate*}

%------------------------------------------------------------------------------%

\subsection{Nguyên Tố Hóa Học}

%------------------------------------------------------------------------------%

\subsection{Sơ Lược về Bảng Tuần Hoàn Các Nguyên Tố Hóa Học}

%------------------------------------------------------------------------------%

\subsection{Phân Tử, Đơn Chất, Hợp Chất}

%------------------------------------------------------------------------------%

\subsection{Giới Thiệu về Liên Kết Hóa Học}

%------------------------------------------------------------------------------%

\subsection{Hóa Trị, Công Thức Hóa Học}

%------------------------------------------------------------------------------%

\section{Năng Lượng \& Sự Biến Đổi}

\subsection{Tốc Độ của Chuyển Động}

%------------------------------------------------------------------------------%

\subsection{Đồ Thị Quãng Đường -- Thời Gian}

%------------------------------------------------------------------------------%

\subsection{Sự Truyền Âm}

%------------------------------------------------------------------------------%

\subsection{Biên Độ, Tần Số, Độ To \& Độ Cao của Âm}

%------------------------------------------------------------------------------%

\subsection{Phản Xạ Âm}

%------------------------------------------------------------------------------%

\subsection{Ánh Sáng, Tia Sáng}

%------------------------------------------------------------------------------%

\subsection{Sự Phản Xạ Ánh Sáng}

%------------------------------------------------------------------------------%

\subsection{Nam Châm}

%------------------------------------------------------------------------------%

\subsection{Từ Trường}

%------------------------------------------------------------------------------%

\subsection{Từ Trường Trái Đất}

%------------------------------------------------------------------------------%

\section{Vật Sống}

\subsection{Vai Trò của Trao Đổi Chất \& Chuyển Hóa Năng Lượng ở Sinh Vật}

%------------------------------------------------------------------------------%

\subsection{Quang Hợp ở Thực Vật}

%------------------------------------------------------------------------------%

\subsection{Các Yếu Tố Ảnh Hưởng Đến Quang Hợp}

%------------------------------------------------------------------------------%

\subsection{Thực Hành về Quang Hợp ở Cây Xanh}

%------------------------------------------------------------------------------%

\subsection{Hô Hấp Tế Bào}

%------------------------------------------------------------------------------%

\subsection{Các Yếu Tố Ảnh Hưởng Đến Hô Hấp Tế Bào}

%------------------------------------------------------------------------------%

\subsection{Trao Đổi Khí ở Sinh Vật}

%------------------------------------------------------------------------------%

\subsection{Vai Trò của Nước \& Các Chất Dinh Dưỡng Đối với Cơ Thể Sinh Vật}

%------------------------------------------------------------------------------%

\subsection{Trao Đổi Nước \& Các Chất Dinh Dưỡng ở Thực Vật}

%------------------------------------------------------------------------------%

\subsection{Trao Đổi Nước \& Các Chất Dinh Dưỡng ở Động Vật}

%------------------------------------------------------------------------------%

\subsection{Khái Quát về Cảm Ứng \& Cảm Ứng ở Thực Vật}

%------------------------------------------------------------------------------%

\subsection{Tập Tính ở Động Vật}

%------------------------------------------------------------------------------%

\subsection{Khái Quát về Sinh Trưởng \& Phát Triển ở Sinh Vật}

%------------------------------------------------------------------------------%

\subsection{Sinh Trưởng \& Phát Triển ở Thực Vật}

%------------------------------------------------------------------------------%

\subsection{Sinh Trưởng \& Phát Triển ở Động Vật}

%------------------------------------------------------------------------------%

\subsection{Khái Quát về Sinh Sinh \& Sinh Sản Vô Tính ở Sinh Vật}

%------------------------------------------------------------------------------%

\subsection{Sinh Sản Hữu Tính ở Sinh Vật}

%------------------------------------------------------------------------------%

\subsection{Các Yếu Tố Ảnh Hưởng đến Sinh Sản \& Điều Khiển Sinh Sản ở Sinh Vật}

%------------------------------------------------------------------------------%

\subsection{Sự Thống Nhất về Cấu Trúc \& Các Hoạt Động Sống trong Cơ Thể Sinh Vật}

%------------------------------------------------------------------------------%

\printbibliography[heading=bibintoc]
	
\end{document}