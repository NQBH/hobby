\documentclass{article}
\usepackage[backend=biber,natbib=true,style=authoryear]{biblatex}
\addbibresource{/home/hong/1_NQBH/reference/bib.bib}
\usepackage[utf8]{vietnam}
\usepackage{tocloft}
\renewcommand{\cftsecleader}{\cftdotfill{\cftdotsep}}
\usepackage[colorlinks=true,linkcolor=blue,urlcolor=red,citecolor=magenta]{hyperref}
\usepackage{amsmath,amssymb,amsthm,mathtools,float,graphicx,algpseudocode,algorithm,tcolorbox,tikz,tkz-tab,subcaption}
\DeclareMathOperator{\arccot}{arccot}
\usepackage[inline]{enumitem}
\allowdisplaybreaks
\numberwithin{equation}{section}
\newtheorem{assumption}{Assumption}[section]
\newtheorem{nhanxet}{Nhận xét}[section]
\newtheorem{conjecture}{Conjecture}[section]
\newtheorem{corollary}{Corollary}[section]
\newtheorem{hequa}{Hệ quả}[section]
\newtheorem{definition}{Definition}[section]
\newtheorem{dinhnghia}{Định nghĩa}[section]
\newtheorem{example}{Example}[section]
\newtheorem{vidu}{Ví dụ}[section]
\newtheorem{lemma}{Lemma}[section]
\newtheorem{notation}{Notation}[section]
\newtheorem{principle}{Principle}[section]
\newtheorem{problem}{Problem}[section]
\newtheorem{baitoan}{Bài toán}[section]
\newtheorem{proposition}{Proposition}[section]
\newtheorem{menhde}{Mệnh đề}[section]
\newtheorem{question}{Question}[section]
\newtheorem{cauhoi}{Câu hỏi}[section]
\newtheorem{quytac}{Quy tắc}
\newtheorem{remark}{Remark}[section]
\newtheorem{luuy}{Lưu ý}[section]
\newtheorem{theorem}{Theorem}[section]
\newtheorem{tiende}{Tiên đề}[section]
\newtheorem{dinhly}{Định lý}[section]
\usepackage[left=0.5in,right=0.5in,top=1.5cm,bottom=1.5cm]{geometry}
\usepackage{fancyhdr}
\pagestyle{fancy}
\fancyhf{}
\lhead{\small Subsect.~\thesubsection}
\rhead{\small\nouppercase{\leftmark}}
\renewcommand{\subsectionmark}[1]{\markboth{#1}{}}
\cfoot{\thepage}
\def\labelitemii{$\circ$}

\title{Some Topics in Elementary Natural Science\texttt{/}Grades 6 \& 7}
\author{Nguyễn Quản Bá Hồng\footnote{Independent Researcher, Ben Tre City, Vietnam\\e-mail: \texttt{nguyenquanbahong@gmail.com}; website: \url{https://nqbh.github.io}.}}
\date{\today}

\begin{document}
\maketitle
\begin{abstract}
	
\end{abstract}
\setcounter{secnumdepth}{4}
\setcounter{tocdepth}{3}
\tableofcontents
\newpage

%------------------------------------------------------------------------------%

\section{Chất \& Sự Biến Đổi của Chất}

\subsection{Nguyên Tử}
\textsf{\textbf{Nội dung.} Mô hình nguyên tử của Rutherford--Bohr -- mô hình sắp xếp electron trong lớp vỏ nguyên tử, khối lượng của 1 nguyên tử theo đơn vị quốc tế amu (đơn vị khối lượng nguyên tử).}

``Khoảng năm 440 trước Công nguyên, nhà triết học Hy Lạp, Democritus cho rằng nếu chia nhỏ nhiều lần 1 đồng tiền vàng cho đến khi ``không thể phân chia được nữa'', thì sẽ được 1 hạt gọi là \textit{nguyên tử}. ``Nguyên tử'' trong tiếng Hy Lạp là atomos, i.e., ``không chia nhỏ hơn được nữa''.'' -- \cite[p. 10]{SGK_KHTN_7_Canh_Dieu}

\subsubsection{Khái niệm nguyên tử}
``Các nhà khoa học hiện nay đã tìm thấy hàng chục triệu chất khác nhau. Tuy nhiên, khi phân tích các chất đó, người ta thấy mọi chất đều được cấu tạo từ những \textit{hạt cực kỳ nhỏ bé, không mang điện}. Những hạt đó được gọi là \textit{nguyên tử}. E.g., đồng tiền vàng được cấu tạo từ các nguyên tử vàng (gold). Khí oxygen được cấu tạo từ các nguyên tử oxygen. Kim cương, than chì đều được cấu tạo từ các nguyên tử carbon. Nước được tạo nên từ các nguyên tử hydrogen \& oxygen. Đường ăn được tạo nên từ các nguyên tử carbon, oxygen, \& hydrogen.'' ``\textit{Nguyên tử nhỏ cỡ nào?} Có thể coi nguyên tử như những quả cầu cực nhỏ. Đường kính của nguyên tử nhỏ hơn đường kính của sợi tóc khoảng $100000$--$500000$ lần. Vì thế, không thể quan sát nguyên tử bằng mắt hoặc các kính hiển vi thông thường. Đường kính sợi tóc là $0.1$mm.'' -- \cite[p. 10]{SGK_KHTN_7_Canh_Dieu}

\subsubsection{Cấu tạo nguyên tử}
``Nguyên tử được coi như 1 quả cầu, gồm vỏ nguyên tử \& hạt nhân nguyên tử.'' -- \cite[p. 11]{SGK_KHTN_7_Canh_Dieu}

\paragraph{Vỏ nguyên tử.} ``Vỏ nguyên tử được tạo bởi 1 hay nhiều electron chuyển động xung quanh hạt nhân. Electron ký hiệu là e, mang điện tích âm \& có giá trị bằng 1 điện tích nguyên tố (1 điện tích nguyên tố $= 1.602\cdot 10^{-19}$ coulomb (C)), được viết đơn giản là $-1$.'' -- \cite[p. 11]{SGK_KHTN_7_Canh_Dieu}

\paragraph{Hạt nhân nguyên tử.} ``Hạt nhân nguyên tử được tạo bởi các proton \& neutron.
\begin{enumerate*}
	\item[$\bullet$] Proton ký hiệu là p, mang điện tích dương \& có giá trị bằng 1 điện tích nguyên tố, được viết là $+1$. Điện tích của proton bằng điện tích của electron về độ lớn nhưng khác dấu.
	\item[$\bullet$] Neutron ký hiệu là n, không mang điện.
\end{enumerate*}
Điện tích của hạt nhân nguyên tử bằng tổng điện tích của các proton. Số đơn vị điện tích hạt nhân bằng số proton. E.g.: nguyên tử nitrogen (nitơ) có $7$ proton nên nitrogen có điện tích hạt nhân là $+7$, số đơn vị điện tích hạt nhân là $7$. Trong nguyên tử, số electron bằng số proton. E.g.: nguyên tử helium gồm hạt nhân có $2$ proton, $2$ neutron, \& vỏ nguyên tử có $2$ electron. Kích thước của hạt nhân rất nhỏ so với kích thước của nguyên tử. Nếu coi hạt nhân là quả bóng có đường kính là $10$cm thì nguyên tử sẽ là quả cầu khổng lồ với đường kính là $1$km (lớp gấp $10000$ lần kích thước hạt nhân nguyên tử).'' -- \cite[p. 11]{SGK_KHTN_7_Canh_Dieu}

``Aluminium là kim loại có nhiều ứng dụng trong thực tiễn, được dùng làm dây dẫn điện, chế tạo các thiết bị, máy móc trong công nghiệp \& nhiều đồ dùng sinh hoạt. Tổng số hạt trong hạt nhân nguyên tử aluminium là $27$, số đơn vị điện tích hạt nhân là $13$.'' ``Helium có 2 proton, mỗi proton có điện tích $+1$, tổng số điện tích: $+2$; có 2 electron, mỗi electron có điện tích $-1$, tổng số điện tích: $-2$. Tổng điện tích trong nguyên tử helium bằng $0$. Ta nói nguyên tử không mang điện hay trung hòa về điện.'' -- \cite[p. 12]{SGK_KHTN_7_Canh_Dieu}

\subsubsection{Sự chuyển động của electron trong nguyên tử}
``Theo mô hình của Rutherford--Bohr trong nguyên tử, các electron chuyển động trên những quỹ đạo xác định xung quanh hạt nhân, như các hành tinh quay quanh Mặt Trời. Trong nguyên tử, các electron được xếp thành từng lớp. Các electron được sắp xếp lần lượt vào các lớp theo chiều từ gần hạt nhân ra ngoài. Mỗi lớp có số electron tối đa xác định, như lớp thứ nhất có tối đa $2$ electron, lớp thứ 2 có tối đa $8$ electron, $\ldots$. E.g., nguyên tử oxygen có $8$ electron, được phân bố thành $2$ lớp electron, lớp thứ nhất có 2 electron, lớp thứ 2 có $6$ electron. Ta nói nguyên tử oxygen có $6$ electron ở lớp ngoài cùng.''

``Ernest Rutherford (1871--1937), nhà vật lý người New Zealand, đã đưa ra mô hình hành tinh nguyên tử để giải thích cấu tạo nguyên tử. Năm 1911, ông đã khám phá ra hầu hết các nguyên tử có cấu tạo rỗng, gồm hạt nhân ở giữa tích điện dương \& vỏ nguyên tử gồm các electron tích điện âm. Mô hình hành tinh nguyên tử của Rutherford chưa mô tả được sự phân bố electron trong vỏ nguyên tử. Sau đó, nhà vật lý người Đan Mạch, Niels Bohr đã đề xuất 1 mô hình mới chỉ rõ các electron được sắp xếp trên các lớp khác nhau.'' -- \cite[p. 12]{SGK_KHTN_7_Canh_Dieu}

``Trong số các nguyên tử đã biết hiện nay, nguyên tử có kích thước lớn nhất là francium, có chứa $7$ lớp electron. Nguyên tử helium có kích thước nhỏ nhất với $1$ lớp electron.'' -- \cite[p. 13]{SGK_KHTN_7_Canh_Dieu}

\subsubsection{Khối lượng nguyên tử}
``Nguyên tử có khối lượng rất nhỏ. 1 gam của bất kỳ chất nào cũng chứa tới hàng tỷ tỷ nguyên tử. Do vậy, để biểu thị khối lượng của nguyên tử, người ta dùng đơn vị khối lượng nguyên tử, ký hiệu là amu (atomic mass unit). $1$amu $= 1.6605\cdot 10^{-24}$g. Khối lượng của 1 nguyên tử bằng tổng khối lượng của proton, neutron, \& electron trong nguyên tử đó. Proton \& neutron đều có khối lượng xấp xỉ bằng $1$amu. Khối lượng của electron là $0.00055$amu, nhỏ hơn nhiều lần so với khối lượng của proton \& neutron nên có thể coi khối lượng nguyên tử bằng khối lượng hạt nhân. E.g.: Nguyên tử hydrogen chỉ có $1$ proton, nên khối lượng nguyên tử hydrogen là $1$amu. Nguyên tử oxygen có $8$ proton \& $8$ neutron, nên khối lượng nguyên tử oxygen là: $8\cdot 1 + 8\cdot 1 = 16$amu.'' -- \cite[p. 13]{SGK_KHTN_7_Canh_Dieu}

``Ruột của bút chì thường được làm từ than chì \& đất sét. Than chì được cấu tạo từ các nguyên tử carbon.''
\vspace{2mm}

\noindent\textbf{Tóm tắt kiến thức.}
``\begin{enumerate*}
	\item[$\bullet$] Nguyên tử là những hạt cực kỳ nhỏ bé, không mang điện, cấu tạo nên chất. Cấu tạo nguyên tử gồm hạt nhân \& vỏ nguyên tử.
	\item[$\bullet$] Hạt nhân của nguyên tử mang điện tích dương, được tạo bởi các proton \& neutron. Vỏ nguyên tử gồm 1 hay nhiều electron mang điện tích âm.
	\item[$\bullet$] Theo mô hình Rutherford--Bohr, trong nguyên tử, electron phân bố trên các lớp electron \& chuyển động quanh hạt nhân nguyên tử trên những quỹ đạo xác định.
	\item[$\bullet$] Khối lượng nguyên tử được coi bằng tổng khối lượng của proton \& neutron có trong nguyên tử, được tính bằng đơn vị amu.'' -- \cite[p. 14]{SGK_KHTN_7_Canh_Dieu}
\end{enumerate*}

%------------------------------------------------------------------------------%

\subsection{Nguyên Tố Hóa Học}
\textsf{\textbf{Nội dung.} Nguyên tố hóa học, ký hiệu nguyên tố hóa học.}

``Trên nhãn của 1 loại thuốc phòng bệnh loãng xương, giảm đau xương khớp có ghi các từ ``calcium'', ``magnesium'', ``zinc''. Đó là tên của 3 nguyên tố hóa học có trong thành phần thuốc để bổ sung cho cơ thể.''

\subsubsection{Khái niệm nguyên tố hóa học}

\begin{dinhnghia}[Nguyên tố hóa học]
	``\emph{Nguyên tố hóa học} là tập hợp những nguyên tử có cùng số proton trong hạt nhân.''
\end{dinhnghia}
``1 nguyên tố hóa học được đặc trưng bởi số proton trong nguyên tử. Các nguyên tử của cùng nguyên tố hóa học đều có tính chất hóa học giống nhau. Cho đến nay, Liên minh Quốc tế về Hóa học thuần túy \& Hóa học ứng dụng (IUPAC) đã công bố tìm thấy $118$ nguyên tố hóa học, trong đó trên $90$ nguyên tố có trong tự nhiên, số còn lại do con người tổng hợp được, gọi là các \textit{nguyên tố nhân tạo}. Hiện tại, các nhà khoa học vẫn đang tiếp tục nghiên cứu để tìm ra những nguyên tố hóa học mới.'' \textit{Những nguyên tố hóa học trong cơ thể con người}: ``Các chất trong cơ thể chúng ta được tạo thành từ khoảng $25$ nguyên tố hóa học, nhưng chủ yếu là các nguyên tố: oxygen, carbon, hydrogen, phosphorus, calcium, nitrogen. Trong đó, nguyên tố calcium có nhiều trong xương \& men răng. Nguyên tố iron (sắt) là thành phần quan trọng của hồng cầu trong máu.'' -- \cite[p. 15]{SGK_KHTN_7_Canh_Dieu}

\subsubsection{Tên nguyên tố hóa học}
``Mỗi nguyên tố hóa học đều có tên gọi riêng. Việc đặt tên nguyên tố dựa vào nhiều cách khác nhau như: liên quan đến tính chất \& ứng dụng của nguyên tố; theo tên các nhà khoa học hoặc theo tên các địa danh. E.g.: Tên nguyên tố carbon (thành phần chính của than) bắt nguồn từ tiếng Latin, ``carbo'' nghĩa là than. Tên nguyên tố hydrogen bắt nguồn từ tiếng Pháp, ``hydrogène'' nghĩa là sinh ra nước. Tên nguyên tố mendelevi bắt nguồn từ tên nhà hóa học người Nga Dimitri Ivanovich Mendeleev. Tên nguyên tố poloni bắt nguồn từ tên đất nước Ba Lan (Poland).'' ``Có 13 nguyên tố hóa học đã quen dùng trong đời sống của người Việt Nam là: vàng (gold), bạc (silver), đồng (copper), chì (lead), sắt (iron), nhôm (aluminium), kẽm (zinc), lưu huỳnh (sulfur), thiếc (tin), nitơ (nitrogen), natri (sodium), kali (potassium), \& thủy ngân (mercury). Vì vậy, trong thực tế, các nguyên tố này được dùng cả 2 tên tiếng Việt \& tên tiếng Anh để tiện tra cứu.'' -- \cite[p. 16]{SGK_KHTN_7_Canh_Dieu}

\subsubsection{Ký hiệu hóa học}
``Mỗi nguyên tố hóa học được biểu diễn bằng 1 ký hiệu riêng, được gọi là ký hiệu hóa học của nguyên tố. Ký hiệu hóa học của 1 nguyên tố được biểu diễn bằng 1 hoặc 2 chữ cái trong tên nguyên tố. Chữ cái đầu tiên được viết ở dạng chữ in hoa, chữ cái thứ 2 (nếu có) ở dạng chữ thường. E.g.: ký hiệu hóa học của nguyên tố hydrogen là H, của nguyên tố oxygen là O, của nguyên tố carbon là C, của nguyên tố chlorine là Cl, $\ldots$''  -- \cite[pp. 16--17]{SGK_KHTN_7_Canh_Dieu}

``Trong 1 số trường hợp, ký hiệu hóa học của nguyên tố không tương ứng với tên theo IUPAC. E.g.: Ký hiệu nguyên tố potassium là K, bắt nguồn từ tên Latin: kalium. Ký hiệu nguyên tố copper là Cu, bắt nguồn từ tên Latin: cuprum.'' ``Tên gọi \& ký hiệu của 1 số nguyên tố hóa học:
\begin{enumerate*}
	\item Hydrogen H.
	\item Helium He.
	\item Lithium Li.
	\item Beryllium Be.
	\item Boron B.
	\item Carbon C.
	\item Nitrogen (Nitơ) N.
	\item Oxygen O.
	\item Fluorine F.
	\item Neon Ne.
	\item Sodium (Natri) Na.
	\item Magnesiumm Mg.
	\item Aluminium (Nhôm) Al.
	\item Silicon Si.
	\item Phosphorus P.
	\item Sulfur (Lưu huỳnh) S.
	\item Chlorine Cl.
	\item Argon Ar.
	\item Potassium (Kali) K.
	\item Calcium Ca.
\end{enumerate*}
'' -- \cite[p. 17]{SGK_KHTN_7_Canh_Dieu}

``Calcium là 1 nguyên tố hóa học có nhiều trong xương \& răng, giúp cho xương \& răng chắc khỏe. Ngoài ra, calcium còn cần cho quá trình hoạt động của thần kinh, cơ, tim, chuyển hóa của tế bào \& quá trình đông máu. Thực phẩm \& thuốc bổ chứa nguyên tố calcium giúp phòng ngừa bệnh loãng xương ở tuổi già \& hỗ trợ quá trình phát triển chiều cao của trẻ em.'' -- \cite[p. 18]{SGK_KHTN_7_Canh_Dieu}
\vspace{2mm}

\noindent\textbf{Tóm tắt kiến thức.}
``\begin{enumerate*}
	\item[$\bullet$] \textit{Nguyên tố hóa học} là tập hợp những nguyên tử có cùng số proton trong hạt nhân.
	\item Mỗi nguyên tố hóa học có tên gọi \& ký hiệu hóa học riêng.
	\item \textit{Ký hiệu hóa học của nguyên tố} được biểu diễn bằng 1 hoặc 2 chữ cái trong tên nguyên tố; trong đó, chữ cái đầu tiên được viết ở dạng chữ in hoa, chữ cái thứ 2 (nếu có) được viết ở dạng chữ thường.'' -- \cite[p. 18]{SGK_KHTN_7_Canh_Dieu}
\end{enumerate*}

%------------------------------------------------------------------------------%

\subsection{Sơ Lược về Bảng Tuần Hoàn Các Nguyên Tố Hóa Học}
\textsf{\textbf{Nội dung.} Nguyên tắc xây dựng bảng tuần hoàn các nguyên tố hóa học; cấu tạo bảng tuần hoàn: ô, nhóm, chu kỳ; sử dụng được bảng tuần hoàn để chỉ ra các nhóm nguyên tố\texttt{/}nguyên tố kim loại, các nhóm nguyên tố\texttt{/}nguyên tố phi kim, nhóm nguyên tố khí hiếm trong bảng tuần hoàn.}

\subsubsection{Nguyên tắc sắp xếp các nguyên tố hóa học trong bảng tuần hoàn}
``Các nguyên tố hóa học được xếp theo quy luật trong 1 bảng, gọi là \textit{bảng tuần hoàn các nguyên tố hóa học} (gọi tắt là \textit{bảng tuần hoàn}). Bảng tuần hoàn hiện nay có $118$ nguyên tố hóa học \& được sắp xếp theo nguyên tắc sau:
\begin{enumerate*}
	\item[$\bullet$] Các nguyên tố hóa học được xếp theo chiều tăng dần của điện tích hạt nhân nguyên tử.
	\item[$\bullet$] Các nguyên tố trong cùng 1 cột có tính chất hóa học tương tự nhau.
\end{enumerate*}
Năm 1869, nhà bác học Nga D. I. Mendeleev (1834--1907), đã tiến hành nghiên cứu việc phân loại các nguyên tố hóa học. Ông đã phát hiện ra sự thay đổi tuần hoàn tính chất các nguyên tố theo khối lượng nguyên tử của chúng \& sắp xếp $63$ nguyên tố hóa học đã biết vào bảng theo chiều tăng dần của khối lượng nguyên tử. Việc tìm ra bảng tuần hoàn là 1 trong những phát hiện xuất sắc nhất trong ngành hóa học.'' -- \cite[p. 20]{SGK_KHTN_7_Canh_Dieu}

\subsubsection{Cấu tạo bảng tuần hoàn}
``Bảng tuần hoàn gồm các ô được sắp xếp thành các hàng \& các cột.'' -- \cite[p. 20]{SGK_KHTN_7_Canh_Dieu}

\paragraph{Ô nguyên tố.} ``Mỗi nguyên tố hóa học được xếp vào 1 ô của bảng tuần hoàn, gọi là \textit{ô nguyên tố}. Ô nguyên tố cho biết: số hiệu nguyên tử, ký hiệu hóa học, tên nguyên tố, \& khối lượng nguyên tử của nguyên tố đó. Số hiệu nguyên tử (ký hiệu là Z) bằng số đơn vị điện tích hạt nhân (bằng số proton \& bằng số electron trong nguyên tử của nguyên tố) \& cùng là số thứ tự của nguyên tố trong bảng tuần hoàn.'' -- \cite[p. 20]{SGK_KHTN_7_Canh_Dieu}

\paragraph{Chu kỳ.} ``Chu kỳ gồm các nguyên tố mà nguyên tử của chúng có cùng số lớp electron \& được xếp thành hàng theo chiều tăng dần của điện tích hạt nhân. Số thứ tự của chu kỳ bằng số lớp electron trong nguyên tử của các nguyên tố trong chu kỳ đó. Bảng tuần hoàn hiện nay gồm $7$ chu kỳ, được đánh số từ $1$ đến $7$.'' ``Chu kỳ 1 gồm 2 nguyên tố là H \& He. Nguyên tử của các nguyên tố này có 1 lớp electron. Điện tích hạt nhân tăng từ H là $+1$ đến He là $+2$. Chu kỳ 2 gồm $8$ nguyên tố từ Li đến Ne. Nguyên tử của các nguyên tố này có $2$ lớp electron. Điện tích hạt nhân tăng dần từ Li là $+3$ đến Ne là $+10$. Chu kỳ 3 gồm $8$ nguyên tố từ Na đến Ar. Nguyên tử của các nguyên tố này có $3$ lớp electron. Điện tích hạt nhân tăng dần từ Na là $+11$ đến Ar là $+18$. Trong 1 chu kỳ, khi đi từ trái sang phải theo chiều tăng dần của điện tích hạt nhân: mở đầu chu kỳ là 1 kim loại điển hình (trừ chu kỳ 1), cuối chu kỳ là 1 phi kim điển hình \& kết thúc chu kỳ là 1 khí hiếm. E.g.: Trong chu kỳ 3, mở đầu chu kỳ là nguyên tố sodium (Na), là 1 kim loại điển hình; cuối chu kỳ là nguyên tố chlorine (Cl), là 1 phi kim điển hình \& kết thúc chu kỳ là nguyên tố khí hiếm argon (Ar).'' -- \cite[pp. 21--22]{SGK_KHTN_7_Canh_Dieu}

\paragraph{Nhóm.} ``Nhóm gồm các nguyên tố có tính chất hóa học tương tự nhau, được xếp thành cột theo chiều tăng dần của điện tích hạt nhân. Bảng tuần hoàn gồm $18$ cột, trong đó có $8$ cột là nhóm A \& $10$ cột là nhóm B (còn gọi là \textit{nhóm các nguyên tố kim loại chuyển tiếp}). Nhóm A được đánh số thứ tự bằng số La Mã lần lượt từ nhóm IA đến VIIIA. Số thứ tự của nhóm A bằng số electron lớp ngoài cùng trong nguyên tử của nguyên tố thuộc nhóm đó.'' -- \cite[p. 22]{SGK_KHTN_7_Canh_Dieu}

``Nhóm IA gồm các nguyên tố kim loại hoạt động mạnh (kim loại điển hình), trừ hydrogen (H). Nguyên tử của chúng đều có $1$ electron ở lớp ngoài cùng. Điện tích hạt nhân của các nguyên tử kim loại trong nhóm IA tăng dần từ Li ($+3$) đến Fr ($+87$). Nhóm VIIA gồm các nguyên tố phi kim hoạt động mạnh (phi kim điển hình), trừ tennessine (Ts). Nguyên tử của chúng đều có $7$ electron ở lớp ngoài cùng. Điện tích hạt nhân của các nguyên tử phi kim trong nhóm VIIA tăng dần từ F ($+9$) đến At ($+85$). Nhóm VIIIA gồm các nguyên tố khí hiếm. Nguyên tử của chúng đều có $8$ electron ở lớp ngoài cùng (trừ helium). Điện tích hạt nhân tăng từ He ($+2$) đến Og ($+118$).'' -- \cite[p. 23]{SGK_KHTN_7_Canh_Dieu}


%------------------------------------------------------------------------------%

\subsection{Phân Tử, Đơn Chất, Hợp Chất}

%------------------------------------------------------------------------------%

\subsection{Giới Thiệu về Liên Kết Hóa Học}

%------------------------------------------------------------------------------%

\subsection{Hóa Trị, Công Thức Hóa Học}

%------------------------------------------------------------------------------%

\section{Năng Lượng \& Sự Biến Đổi}

\subsection{Tốc Độ của Chuyển Động}

%------------------------------------------------------------------------------%

\subsection{Đồ Thị Quãng Đường -- Thời Gian}

%------------------------------------------------------------------------------%

\subsection{Sự Truyền Âm}

%------------------------------------------------------------------------------%

\subsection{Biên Độ, Tần Số, Độ To \& Độ Cao của Âm}

%------------------------------------------------------------------------------%

\subsection{Phản Xạ Âm}

%------------------------------------------------------------------------------%

\subsection{Ánh Sáng, Tia Sáng}

%------------------------------------------------------------------------------%

\subsection{Sự Phản Xạ Ánh Sáng}

%------------------------------------------------------------------------------%

\subsection{Nam Châm}

%------------------------------------------------------------------------------%

\subsection{Từ Trường}

%------------------------------------------------------------------------------%

\subsection{Từ Trường Trái Đất}

%------------------------------------------------------------------------------%

\section{Vật Sống}

\subsection{Vai Trò của Trao Đổi Chất \& Chuyển Hóa Năng Lượng ở Sinh Vật}

%------------------------------------------------------------------------------%

\subsection{Quang Hợp ở Thực Vật}

%------------------------------------------------------------------------------%

\subsection{Các Yếu Tố Ảnh Hưởng Đến Quang Hợp}

%------------------------------------------------------------------------------%

\subsection{Thực Hành về Quang Hợp ở Cây Xanh}

%------------------------------------------------------------------------------%

\subsection{Hô Hấp Tế Bào}

%------------------------------------------------------------------------------%

\subsection{Các Yếu Tố Ảnh Hưởng Đến Hô Hấp Tế Bào}

%------------------------------------------------------------------------------%

\subsection{Trao Đổi Khí ở Sinh Vật}

%------------------------------------------------------------------------------%

\subsection{Vai Trò của Nước \& Các Chất Dinh Dưỡng Đối với Cơ Thể Sinh Vật}

%------------------------------------------------------------------------------%

\subsection{Trao Đổi Nước \& Các Chất Dinh Dưỡng ở Thực Vật}

%------------------------------------------------------------------------------%

\subsection{Trao Đổi Nước \& Các Chất Dinh Dưỡng ở Động Vật}

%------------------------------------------------------------------------------%

\subsection{Khái Quát về Cảm Ứng \& Cảm Ứng ở Thực Vật}

%------------------------------------------------------------------------------%

\subsection{Tập Tính ở Động Vật}

%------------------------------------------------------------------------------%

\subsection{Khái Quát về Sinh Trưởng \& Phát Triển ở Sinh Vật}

%------------------------------------------------------------------------------%

\subsection{Sinh Trưởng \& Phát Triển ở Thực Vật}

%------------------------------------------------------------------------------%

\subsection{Sinh Trưởng \& Phát Triển ở Động Vật}

%------------------------------------------------------------------------------%

\subsection{Khái Quát về Sinh Sinh \& Sinh Sản Vô Tính ở Sinh Vật}

%------------------------------------------------------------------------------%

\subsection{Sinh Sản Hữu Tính ở Sinh Vật}

%------------------------------------------------------------------------------%

\subsection{Các Yếu Tố Ảnh Hưởng đến Sinh Sản \& Điều Khiển Sinh Sản ở Sinh Vật}

%------------------------------------------------------------------------------%

\subsection{Sự Thống Nhất về Cấu Trúc \& Các Hoạt Động Sống trong Cơ Thể Sinh Vật}

%------------------------------------------------------------------------------%

\printbibliography[heading=bibintoc]
	
\end{document}