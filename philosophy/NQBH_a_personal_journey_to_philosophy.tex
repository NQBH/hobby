\documentclass[oneside]{book}
\usepackage[backend=biber,natbib=true,style=authoryear]{biblatex}
\addbibresource{/home/hong/1_NQBH/reference/bib.bib}
\usepackage[vietnamese,english]{babel}
\usepackage{tocloft}
\renewcommand{\cftsecleader}{\cftdotfill{\cftdotsep}}
\usepackage[colorlinks=true,linkcolor=blue,urlcolor=red,citecolor=magenta]{hyperref}
\usepackage{algorithm,algpseudocode,amsmath,amssymb,amsthm,float,graphicx,mathtools}
\allowdisplaybreaks
\numberwithin{equation}{section}
\newtheorem{assumption}{Assumption}[chapter]
\newtheorem{conjecture}{Conjecture}[chapter]
\newtheorem{corollary}{Corollary}[chapter]
\newtheorem{definition}{Definition}[chapter]
\newtheorem{example}{Example}[chapter]
\newtheorem{lemma}{Lemma}[chapter]
\newtheorem{notation}{Notation}[chapter]
\newtheorem{principle}{Principle}[chapter]
\newtheorem{problem}{Problem}[chapter]
\newtheorem{proposition}{Proposition}[chapter]
\newtheorem{question}{Question}[chapter]
\newtheorem{remark}{Remark}[chapter]
\newtheorem{theorem}{Theorem}[chapter]
\usepackage[left=0.5in,right=0.5in,top=1.5cm,bottom=1.5cm]{geometry}
\usepackage{fancyhdr}
\pagestyle{fancy}
\fancyhf{}
\lhead{\small \textsc{Sect.} ~\thesection}
\rhead{\small \nouppercase{\leftmark}}
\renewcommand{\sectionmark}[1]{\markboth{#1}{}}
\cfoot{\thepage}
\def\labelitemii{$\circ$}

\title{A Personal Journey to Philosophy}
\author{\selectlanguage{vietnamese} Nguyễn Quản Bá Hồng}
\date{\today}

\begin{document}
\maketitle
\tableofcontents

%------------------------------------------------------------------------------%

\chapter*{Foreword}

A \textit{personal} journey to philosophy -- the hardest subject I have ever faced to \& fought against. A collection of quotes from different resources, e.g., philosophical books, websites, forums, and Facebook philosophical pages, etc., and some \textit{personal} (again) thoughts about them.

%------------------------------------------------------------------------------%

\chapter*{Basic Terminologies}
\begin{itemize}
	\item \textbf{philosophy} [n] \textbf{1.} [uncountable] the study of the nature \& meaning of the universe \& of human life; \textbf{natural philosophy} is an old term for the study of the physical world, which developed into the natural sciences; \textbf{2.} [countable] a particular set or system of beliefs resulting from the search for knowledge about life \& the universe; \textbf{3.} [countable] a set of beliefs or an attitude to life that guides somebody's behavior.
\end{itemize}

%------------------------------------------------------------------------------%

\chapter{Jordan B. Peterson. \textit{12 Rules for Life: An Antidote to Chaos}}

\section{Introduction}
``\textit{12 Rules for Life: An Antidote to Chaos} is a 2018 \href{https://en.wikipedia.org/wiki/Self-help_book}{self-help book} by the Canadian clinical\footnote{\textbf{clinical} [a] [only before noun] connected with the examination \& treatment of patients \& their illnesses.} psychologist\footnote{\textbf{psychologist} [n] a scientist who studies psychology.} \href{https://en.wikipedia.org/wiki/Jordan_Peterson}{Jordan Peterson}. It provides life advice through essays in abstract ethical\footnote{\textbf{ethical} [a] \textbf{1.} connected with beliefs \& principles about what is right \& wrong; \textbf{2.} morally correct or acceptable.} principles, psychology, mythology\footnote{\textbf{mythology} [n] [uncountable, countable] \textbf{1.} ancient myths in general; the ancient myths of a particular culture, society, etc.; \textbf{2.} \textbf{mythology (of something)} ideas that many people think are true but are in fact false.}, religion\footnote{\textbf{religion} [n] \textbf{1.} [uncountable] the belief in the existence of a god or gods, \& the activities that are connected with the worship of them; \textbf{2.} [countable] 1 of the systems of belief that are based on the belief in the existence of a particular god or gods.}, \& personal anecdotes\footnote{\textbf{anecdote} [n] [countable, uncountable] \textbf{1.} \textbf{anecdote (about somebody\texttt{/}something)} a short, interesting or funny story about a real person or event; \textbf{2.} a personal account of an event, especially one that is considered as possibly not true or accurate.}.''[$\ldots$] ``The book is written in a more accessible style than his previous academic book, \href{https://en.wikipedia.org/wiki/Maps_of_Meaning:_The_Architecture_of_Belief}{Maps of Meaning: The Artchitecture of Belief} (1999). A sequel, \href{https://en.wikipedia.org/wiki/Beyond_Order}{Beyond Order: 12 More Rules for Life}, was published in Mar 2021.

\subsection{Overview}

\paragraph*{Background.} ``Peterson's interest in writing the book grew out of a personal hobby of answering questions posted on \href{https://en.wikipedia.org/wiki/Quora}{Quora}; 1 such question being
\begin{question}
	\fbox{``What are the most valuable things everyone should know?'',}
\end{question}
to which his answer comprised 42 rules. The early vision \& promotion of the book aimed to include all rules, with the title ``42''. Peterson stated that it ``isn't only written for other people. It's warning to me.'''' -- \href{https://en.wikipedia.org/wiki/12_Rules_for_Life#Background}{Wikipedia\texttt{/}12 Rules for Life\texttt{/}overview\texttt{/}background}

\paragraph*{12 Rules.} ``The book is divided into chapters with each title representing 1 of the following 12 specific rules for life as explained through an essay.
\begin{enumerate}
	\item ``Stand up straight with your shoulders back.''
	\item ``Treat yourself like you are someone you are responsible for helping.''
	\item ``Make friends with people who want the best for you.''
	\item ``Compare yourself to who you were yesterday, not to who someone else is today.''
	\item ``Do not let your children do anything that makes you dislike them.''
	\item ``Set your house in perfect order before you criticize the world.''
	\item ``Pursue what is meaningful (not what is expedient\footnote{\textbf{expedient} [n] an action that is useful or necessary for a particular purpose, but not always fair or right.}).''
	\item ``Tell the truth -- or, at least, don't lie.''
	\item ``Assume that the person you are listening to might know something you don't.''
	\item ``Be precise in your speech.''
	\item ``Do not bother children when they are skate-boarding.''
	\item ``Pet a cat when you encounter\footnote{\textbf{encounter} [v] \textbf{1.} \textbf{encounter something} to experience something, especially something unpleasant or difficult, while you are trying to do something else, \textsc{synonym}: \textbf{run into something}; \textbf{2.} \textbf{encounter something\texttt{/}somebody} to discover or experience something, or meet somebody, especially something\texttt{/}somebody new, unusual or unexpected, \textsc{synonym}: \textbf{come across somebody\texttt{/}something}; [n] a meeting, especially one that is sudden or unexpected.} one on the street.'''' -- \href{https://en.wikipedia.org/wiki/12_Rules_for_Life#12_Rules}{Wikipedia\texttt{/}12 Rules for Life\texttt{/}overview\texttt{/}content}
\end{enumerate} 

\paragraph*{Content.} ``The book's central idea is that ``\fbox{suffering is built into the structure of \href{https://en.wikipedia.org/wiki/Being}{being}}'' \& although it can be unbearable\footnote{\textbf{unbearable} [a] too painful, annoying or unpleasant to deal with or accept, \textsc{synonym}: \textbf{intolerable}, \textsc{opposite}: \textbf{bearable}.}, people have a choice either to withdraw\footnote{\textbf{withdraw} [v] \textbf{1.} [transitive, intransitive] (used especially about armed forces) to make people leave a place; to leave a place; \textbf{2.} [intransitive] \textbf{withdraw (to something)} to leave a room; to go away from other people; \textbf{3.} [transitive] to move something back, out or away from something; \textbf{4.} [transitive] to take money out of a bank account or financial institution; \textbf{5.} [intransitive] to stop taking part in something; \textbf{6.} [intransitive] to stop wanting to speak to, or be with, other people; \textbf{7.} [transitive] to no longer provide or offer something; to no longer make something available; \textbf{8.} [transitive] \textbf{withdraw something} to say that you no longer agree with what you said before.}, which is a ``suicidal\footnote{\textbf{suicidal} [a] (of people) very unhappy or depressed \& feeling that they want to kill themselves; (of behavior) showing this.} gesture\footnote{\textbf{gesture} [n] \textbf{1.} [countable, uncountable] \textbf{gesture (of something)} something that you do or say to show a particular feeling or intention; \textbf{2.} [countable, uncountable] a movement that you make with your hands, your head or your face to show a particular meaning.}'', or to face \& transcend\footnote{\textbf{transcend} [v] \textbf{transcend something} to be or go beyond the usual limits of something.} it. Living in a world of chaos \& order,\fbox{ everyone has ``darkness''} that can \fbox{``turn them into the monsters they're capable of being''} to satisfy their \fbox{dark impulses\footnote{\textbf{impulse} [n] \textbf{1.} [countable, usually singular, uncountable] a sudden strong wish or need to do something, without stopping to think about the results; \textbf{2.} [countable, usually singular] something that causes somebody\texttt{/}something to do something or to develop \& make progress; \textbf{3.} [countable] a brief electric current, e.g. one that travels from a nerve to a muscle; \textbf{4.} [countable] (\textit{physics}) the change in momentum of an object due to a force.} in the right situations}. Scientific experiments like the \href{https://en.wikipedia.org/wiki/Inattentional_blindness#Invisible_Gorilla_Test}{Invisible Gorilla Test} show that perception\footnote{\textbf{perception} [n] \textbf{1.} [uncountable, countable] an idea, a belief or an image you have as a result of how you see or understand something; \textbf{2.} [uncountable] the way you notice things or the ability to notice things with the senses; in biology, \textbf{perception} refers to the processes in the nervous system by which a living thing becomes aware of events \& things outside itself; \textbf{3.} [uncountable] the ability to understand the true nature of something, \textsc{synonym}: \textbf{insight}.} is adjusted to aims, \& it is \fbox{better to seek \href{https://en.wikipedia.org/wiki/Meaning_(psychology)}{meaning} rather than happiness}. Peterson notes:
\begin{quotation}
	``It's all very well to think the meaning of life is happiness, but what happens when you're unhappy? Happiness is a great side effect. When it comes, accept it gratefully\footnote{\textbf{grateful} [a] \textbf{1.} feeling or showing thanks because somebody has done something kind for you or has done as you asked; \textbf{2.} used to make a request, especially in a letter or in a formal situation.}. But it's fleeting\footnote{\textbf{fleeting} [a] [usually before noun] lasting only a short time, \textsc{synonym}: \textbf{brief}.} \& unpredictable\footnote{\textbf{unpredictable} [a] that cannot be predicted because it changes a lot or depends on too many different things, \textsc{opposite}: \textbf{predictable}.}. It's not something to aim at -- because it's not an aim. \& if happiness is the purpose of life, what happens when you're unhappy? Then you're a failure.''
\end{quotation}
The book advances the idea that \fbox{people are born with an instinct\footnote{\textbf{instinct} [n] [uncountable, countable] a natural tendency for people \& animals to behave in a particular way, using the knowledge \& abilities that they were born with rather than thought or training.} for ethics \& meaning}, \& should take responsibility\footnote{\textbf{responsibility} [n] \textbf{1.} [uncountable, countable] a duty to deal with or take care of somebody\texttt{/}something, so that you may be blamed if something goes wrong; \textbf{2.} [uncountable] \textbf{responsibility (for something)} blame for something bad that has happened; \textbf{3.} [countable, uncountable] a moral duty to behave well with regard to somebody\texttt{/}something.} to search for meaning above their own interests (Rule 7, ``Pursue what is meaningful, not what is expedient''). Such thinking is reflected both in contemporary\footnote{\textbf{contemporary} [a] \textbf{1.} belonging to the present time, \textsc{synonym}: \textbf{modern}; \textbf{2.} (especially of people \& society) belonging to the same time as somebody\texttt{/}something else; [n] a person or thing living or existing at the same time as somebody\texttt{/}something else, especially somebody who is about the same age as somebody else.} stories e.g. \href{https://en.wikipedia.org/wiki/Pinocchio_(1940_film)}{Pinocchio}, \href{https://en.wikipedia.org/wiki/The_Lion_King}{The Lion King}, \& \href{https://en.wikipedia.org/wiki/Harry_Potter}{Harry Potter}, \& in ancient stories from the \href{https://en.wikipedia.org/wiki/Bible}{Bible}. To ``stand up straight with your shoulders back'' (Rule 1) is to ``accept the terrible responsibility of life'', to make self-sacrifice\footnote{\textbf{self-sacrifice} [n] [uncountable] (\textit{approving}) the act of not allowing yourself to have or do something in order to help other people.}, because the individual must rise above \href{https://en.wikipedia.org/wiki/Victimisation}{victimization}\footnote{\textbf{victimize} [v] [often passive] \textbf{victimize somebody} to make somebody suffer unfairly because you do not like them, their opinions or something that they have done.} \& ``conduct his or her life in a manner that requires the rejection\footnote{\textbf{rejection} [n] [uncountable, countable] \textbf{1.} the act of refusing to accept or consider something; \textbf{2.} the act of refusing to accept somebody for a job or position; \textbf{3.} the decision not to use, sell, publish, etc. something because its quality is not good enough; \textbf{4.} \textbf{rejection (of something)} an occasion when somebody's body does not accept a new organ after a transplant operation, by producing substances that attack the organ; \textbf{5.} the act of failing to give a person or an animal enough care or affection.} of immediate gratification\footnote{\textbf{gratification} [n] [uncountable, countable] (\textit{formal}) the state of feeling pleasure when something goes well for you or when your desires are satisfied; something that gives you pleasure, \textsc{synonym}: \textbf{satisfaction}.}, of natural \& perverse\footnote{\textbf{perverse} [a] showing a deliberate \& determined desire to behave in a way that most people think is wrong, unacceptable or unreasonable.} desires alike.'' The comparison to \href{https://en.wikipedia.org/wiki/Neurology}{neurological}\footnote{\textbf{neurological} [a] relating to nerves or to the science of neurology.} structures \& behavior of \href{https://en.wikipedia.org/wiki/Lobsters}{lobsters} is used as a natural example to the formation\footnote{\textbf{formation} [n] \textbf{1.} [uncountable] the action of forming something; the process of being formed; \textbf{2.} [countable] a thing that has been formed, especially in a particular place or in a particular way; \textbf{3.} [countable, uncountable] a particular arrangement or pattern of people or things.} of \href{https://en.wikipedia.org/wiki/Hierarchy}{social hierarchies}\footnote{\textbf{hierarchy} [n] \textbf{1.} [countable, uncountable] a system, especially in a society or an organization, in which people are organized into different levels of importance from highest to lowest; \textbf{2.} [countable] a system that ideas or beliefs can be arranged into.}.

The other parts of the work explore \& criticize the state of young men; the upbringing\footnote{\textbf{upbringing} [n] [singular, uncountable] the way in which a child is cared for \& taught how to behave while it is growing up.} that ignores \href{https://en.wikipedia.org/wiki/Sex_differences_in_humans}{sex differences} between boys \& girls (criticism of \href{https://en.wikipedia.org/wiki/Overprotective}{over-protection} \& \href{https://en.wikipedia.org/wiki/Tabula_rasa}{tabula rasa} model in \href{https://en.wikipedia.org/wiki/Social_science}{social sciences}); male-female \href{https://en.wikipedia.org/wiki/Interpersonal_relationship}{interpersonal relationships}; \href{https://en.wikipedia.org/wiki/School_shooting}{school shootings}; religion \& moral \href{https://en.wikipedia.org/wiki/Nihilism}{nihilism}\footnote{\textbf{nihilism} [n] [uncountable] (\textit{philosophy}) the belief that life has no meaning or purpose \& that religious \& moral principles have no value.}; \href{https://en.wikipedia.org/wiki/Relativism}{relativism}\footnote{\textbf{relativism} [n] [uncountable] the belief that truth is not always \& generally valid, but can be judged only in relation to other things, e.g. your personal situation.}; \& lack of respect for the values that built \href{https://en.wikipedia.org/wiki/Western_world}{Western society}.

In the last chapter, Peterson outlines the ways in which one can cope with the most tragic\footnote{\textbf{tragic} [a] \textbf{1.} making you feel very sad, usually because somebody has died or suffered a lot; \textbf{2.} [usually before noun] connected with tragedy ($=$ the style of literature).} events, events that are often \fbox{out of one's control}. In it, he describes his own personal struggle upon discovering that his daughter, Mikhaila, had a rare bone disease. The chapter is a meditation\footnote{\textbf{meditation} [n] \textbf{1.} [uncountable] the practice of thinking deeply, usually in silence, especially for religious reasons or in order to make your mind calm; \textbf{2.} [countable, usually plural] \textbf{meditation (on something)} serious thoughts on a particular subject that somebody writes down or speaks.} on how to maintain\footnote{\textbf{maintain} [v] \textbf{1.} \textbf{maintain something} to cause or enable a condition or situation to continue, \textsc{synonym}: \textbf{preserve}; \textbf{2.} \textbf{maintain something} to keep something at the same level or rate; \textbf{3.} to state strongly that something is true, even when some other people may not believe it; \textbf{4.} \textbf{maintain somebody\texttt{/}something} to support somebody\texttt{/}something over a long period of time by providing money, paying for food, etc.; \textbf{5.} \textbf{maintain something} to keep a building, machine, etc. in good condition by checking or repairing it regularly; \textbf{6.} \textbf{maintain a record} to write something down as a record \& keep adding the most recent information, \textsc{synonym}: \textbf{keep}.} a watchful\footnote{\textbf{watchful} [a] paying attention to what is happening in case of danger, accidents, etc.} eye on, and cherish\footnote{\textbf{cherish} [v] (\textit{formal}) \textbf{1.} \textbf{cherish somebody\texttt{/}something} to love somebody\texttt{/}something very much \& want to protect them or it; \textbf{2.} \textbf{cherish something} to keep an idea, a hope or a pleasant feeling in your mind for a long time.}, life's small redeemable\footnote{\textbf{redeemable} [a] \textbf{redeemable (against something)} that can be exchanged for money or goods.} qualities (i.e., ``pet a cat when you encounter one''). It also outlines a practical way to deal with hardship\footnote{\textbf{hardship} [n] [uncountable, countable] a situation that is difficult \& unpleasant because you do not have enough money, food, clothes, etc.}: to shorten one's temporal\footnote{\textbf{temporal} [a] \textbf{1.} connected with or limited by time; \textbf{2.} connected with the real physical world, not spiritual matters; \textbf{3.} (\textit{anatomy}) near the temples at the side of the head.} scope of responsibility (e.g., focusing on the next minute rather than the next 3 months).

Canadian psychiatrist and psychoanalyst \href{https://en.wikipedia.org/wiki/Norman_Doidge}{Norman Doidge} wrote \cite{Peterson2018}'s foreword.'' -- \href{https://en.wikipedia.org/wiki/12_Rules_for_Life#Content}{Wikipedia\texttt{/}12 Rules for Life\texttt{/}overview\texttt{/}content}

\footnote{\textbf{antidote} [n] \textbf{1.} \textbf{antidote (to something)} a substance that controls the effects of a poison or disease; \textbf{2.} \textbf{antidote (to something)} anything that takes away the effects of something unpleasant.} \footnote{\textbf{chaos} [n] [uncountable] a state of complete confusion \& lack of order; in physics, \textbf{chaos} is the property of a complex system whose behavior is so unpredictable that it appears random, especially because small changes in conditions can have very large effects; \textbf{chaos theory} is the branch of mathematics that deals with these complex systems.}
\begin{quotation}
	``The most influential public intellectual\footnote{\textbf{intellectual} [a] [usually before noun] connected with or using a person's ability to think in a logical way \& understand things, \textsc{synonym}: \textbf{mental}; [n] a person who is well educated \& enjoys activities in which they have to think seriously about things.} in the Western world right now.'' -- New York Times
\end{quotation}
``Rules? More rules? Really? Isn't life complicated\footnote{\textbf{complicated} [a] \textbf{1.} made of many different things or parts that are connected; difficult to understand, \textsc{synonym}: \textbf{complex}, \textsc{opposite}: \textbf{uncomplicated}; \textbf{2.} (of a medical condition) involving complications, \textsc{opposite}: \textbf{uncomplicated}.} enough, restricting enough, without abstract rules that don't take our unique, individual situations into account? \& given  that our brains are plastic, \& all develop differently based on our life experiences, why even expect that a few rules might be helpful to us all?'' -- \cite[Foreword]{Peterson2018}

%------------------------------------------------------------------------------%

\chapter{Miscellaneous}

\section{Young, Dumb, \& Broke}
Watch \& listen \href{https://www.youtube.com/watch?v=IPfJnp1guPc}{Youtube\texttt{/}Khalid\texttt{/}Young Dumb \& Broke}.

\section{Existential Crisis}

\section{Meaning of Life?}

\section{Art of Balancing in Life?}

%------------------------------------------------------------------------------%

\begin{thebibliography}{99}
	\bibitem[NQBH\texttt{/}psychology]{NQBH/psychology} \selectlanguage{vietnamese} Nguyễn Quản Bá Hồng. \href{https://github.com/NQBH/hobby/blob/master/psychology/NQBH_a_personal_journey_to_psychology.pdf}{\textit{A Personal Journey to Psychology: The Way I Perceive}}. March 2022--now.
\end{thebibliography}

\printbibliography[heading=bibintoc]
	
\end{document}