\documentclass[oneside]{book}
\usepackage[backend=biber,natbib=true,style=authoryear]{biblatex}
\addbibresource{/home/hong/1_NQBH/reference/bib.bib}
\usepackage[vietnamese,english]{babel}
\usepackage{tocloft}
\renewcommand{\cftsecleader}{\cftdotfill{\cftdotsep}}
\usepackage[colorlinks=true,linkcolor=blue,urlcolor=red,citecolor=magenta]{hyperref}
\usepackage{amsmath,amssymb,amsthm,mathtools,float,graphicx}
\allowdisplaybreaks
\numberwithin{equation}{section}
\newtheorem{assumption}{Assumption}[chapter]
\newtheorem{conjecture}{Conjecture}[chapter]
\newtheorem{corollary}{Corollary}[chapter]
\newtheorem{definition}{Definition}[chapter]
\newtheorem{example}{Example}[chapter]
\newtheorem{lemma}{Lemma}[chapter]
\newtheorem{notation}{Notation}[chapter]
\newtheorem{principle}{Principle}[chapter]
\newtheorem{problem}{Problem}[chapter]
\newtheorem{proposition}{Proposition}[chapter]
\newtheorem{question}{Question}[chapter]
\newtheorem{remark}{Remark}[chapter]
\newtheorem{theorem}{Theorem}[chapter]
\usepackage[left=0.5in,right=0.5in,top=1.5cm,bottom=1.5cm]{geometry}
\usepackage{fancyhdr}
\pagestyle{fancy}
\fancyhf{}
\lhead{\small \textsc{Sect.} ~\thesection}
\rhead{\small \nouppercase{\leftmark}}
\renewcommand{\sectionmark}[1]{\markboth{#1}{}}
\cfoot{\thepage}
\def\labelitemii{$\circ$}

\title{A Personal Journey to Philosophy}
\author{\selectlanguage{vietnamese} Nguyễn Quản Bá Hồng\footnote{Independent Researcher, Ben Tre City, Vietnam\\e-mail: \texttt{nguyenquanbahong@gmail.com}}}
\date{\today}

\begin{document}
\maketitle
\selectlanguage{english}
\setcounter{secnumdepth}{4}
\setcounter{tocdepth}{4}
\tableofcontents

%------------------------------------------------------------------------------%

\chapter*{Foreword}

A \textit{personal} journey to philosophy -- the hardest subject I have ever faced to \& fought against. A collection of quotes from different resources, e.g., philosophical books, websites, forums, \& Facebook philosophical pages, etc., \& some \textit{personal} (again) thoughts about them.

%------------------------------------------------------------------------------%

\chapter*{Basic Terminologies}
\begin{itemize}
	\item \textbf{philosophy} [n] \textbf{1.} [uncountable] the study of the nature \& meaning of the universe \& of human life; \textbf{natural philosophy} is an old term for the study of the physical world, which developed into the natural sciences; \textbf{2.} [countable] a particular set or system of beliefs resulting from the search for knowledge about life \& the universe; \textbf{3.} [countable] a set of beliefs or an attitude to life that guides somebody's behavior.
\end{itemize}

%------------------------------------------------------------------------------%

\chapter{Wikipedia's}

\section{\href{https://en.wikipedia.org/wiki/Jeet_Kune_Do}{Wikipedia\texttt{/}Jeet Kune Do}}
\textbf{Jeet Kune Do.} The \textbf{Jeet Kune Do} Emblem The \href{https://en.wikipedia.org/wiki/Taijitu}{Taijitu} represents the concepts of \href{https://en.wikipedia.org/wiki/Yin_and_yang}{yin \& yang}. The \href{https://en.wikipedia.org/wiki/Chinese_character}{Chinese characters} indicate: ``Using no way as way'' \& ``Having no limitation as limitation''. This slogan incarnates the self-recursive behavior of many Sinitic languages, which also appears incorporated into the practice of the material art. Also, the arrows represent the endless interaction between yang \& yin.
\begin{itemize}
	\item \textbf{Also known as.} JKD, Jun Fan Jeet Kune Do
	\item \textbf{Focus.} \href{https://en.wikipedia.org/wiki/Hybrid_martial_arts}{Hybrid}
	\item \textbf{Creator.} \href{https://en.wikipedia.org/wiki/Bruce_Lee}{Bruce Lee}
	\item \textbf{Famous practitioners.} \href{https://en.wikipedia.org/wiki/Jeet_Kune_Do#Notable_practitioners}{Wikipedia\texttt{/}Jeet Kune Do\texttt{/}notable practioners}
	\item \textbf{Parenthood.} \textit{Jun Fan \href{https://en.wikipedia.org/wiki/Gung_Fu}{Gung Fu}}: \href{https://en.wikipedia.org/wiki/Wing_Chun}{Wing Chun}, \href{https://en.wikipedia.org/wiki/Boxing}{Boxing}, \href{https://en.wikipedia.org/wiki/Fencing}{Fencing}, \href{https://en.wikipedia.org/wiki/Escrima}{Escrima}, \href{https://en.wikipedia.org/wiki/Judo}{Judo}, \href{https://en.wikipedia.org/wiki/Jujutsu}{Jujutsu}, \href{https://en.wikipedia.org/wiki/Savate}{Savate}, \href{https://en.wikipedia.org/wiki/Taekwondo#1946:_Traditional_Taekwondo}{traditional Taekwondo}, \href{https://en.wikipedia.org/wiki/Tai_Chi}{Tai Chi}, \href{https://en.wikipedia.org/wiki/Catch_Wrestling}{Catch Wrestling}
	\item \textbf{Descendant arts.} \href{https://en.wikipedia.org/wiki/Jesse_Glover}{Non-classical Gung Fu}, \href{https://en.wikipedia.org/wiki/James_W._DeMile}{Wing Chun Do}, \href{https://en.wikipedia.org/wiki/Emerson_Combat_Systems}{Emerson Combat Systems}, \href{https://en.wikipedia.org/wiki/Leo_Fong}{Wei Kuen Do}, \href{https://en.wikipedia.org/wiki/Mixed_Martial_Arts}{Mixed Martial Arts} (modern)
	\item \textbf{Literal meaning.} ``Way of the Intercepting Fist''
\end{itemize}
``\textit{Jeet Kune Do} is an \href{https://en.wikipedia.org/wiki/Hybrid_martial_art}{eclectic martial arts philosophy}\footnote{\textbf{electic} [a] (\textit{formal}) not following 1 style or set of ideas but choosing from or using a wide variety.} heavily influenced \& adapted by the personal philosophy \& experiences of \href{https://en.wikipedia.org/wiki/Martial_artist}{martial artist} \href{https://en.wikipedia.org/wiki/Bruce_Lee}{Bruce Lee}.'' -- \href{https://en.wikipedia.org/wiki/Jeet_Kune_Do}{Wikipedia\texttt{/}Jeet Kune Do}

\subsection{Overview \& philosophy}
``See also: \href{https://en.wikipedia.org/wiki/Bruce_Lee}{Wikipedia\texttt{/}Bruce Lee}. Jeet Kune Do was conceived by \href{https://en.wikipedia.org/wiki/Bruce_Lee}{Bruce Lee}, based on his experiences in unarmed fighting \& self-defense. Originally, Lee had studied \& researched various forms of \href{https://en.wikipedia.org/wiki/Martial_arts}{martial arts} \& would formalize a system named \textbf{Jun Fan Gung Fu} circa 1962. However, around 1964, following his encounter with \href{https://en.wikipedia.org/wiki/Wong_Jack-man}{Wong Jack-man}, Lee came to realize the error of binding oneself to a systematized martial art \& denounced the Jun Fan Gung Fu. Following this, Lee began to passionately work on research \& practice in order to refine his way of practicing material arts. In 1965, he outlined the basic concept of Jeet Kune Do.

Not wanting to create another style that would share the limitations that all styles had, he instead described the process which he used to create it:
\begin{quotation}
	``I have not invented a ``new style,'' composite, modified or otherwise that is set within distinct form as apart from ``this'' method or ``that'' method. On the contrary, I hope to free my followers from clinging to styles, patterns, or molds. Remember that Jeet Kune Do is merely a name used, a mirror in which to see ``ourselves'' $\ldots$ Jeet Kune Do is not an organized institution that one can be a member of. Either you understand or you don't, \& that is that. There is no mystery about my style. \fbox{My movements are simple, direct, \& non-classical.} The extraordinary part of it lies in its \fbox{simplicity}. Every movement in Jeet Kune Do is being so of itself. There is nothing artificial about it. I always believe that \fbox{the easy way is the right way}. Jeet Kune Do is simply the direct expression of one's feelings with the minimum of movements \& energy. The closer to the true way of Kung Fu, the less wastage of expression there is. Finally, a Jeet Kune Do man who says Jeet Kune Do is exclusively Jeet Kune Do is simply not with it. He is still hung up on his \fbox{self-closing resistance}, in this case, anchored down to a reactionary pattern, \& naturally is still bound by another modified pattern \& can move within its limits. He has not digested the simple fact that truth exists outside all molds; pattern \& awareness is never exclusive. Again let me remind you Jeet Kune Do is just a name used, a boat to get one across, \& once across it is to be discarded \& not to be carried on one's back.'' -- Bruce Lee''
\end{quotation}
\textsf{Fig. \href{https://en.wikipedia.org/wiki/Bruce_Lee}{Bruce Lee} with \href{https://en.wikipedia.org/wiki/Wing_Chun}{Wing Chun} grandmaster \href{https://en.wikipedia.org/wiki/Ip_Man}{Ip Man}}.

``Lee stated his concept does not add more \& more things on top of each other to form a system, but rather selects the best thereof. The \href{https://en.wikipedia.org/wiki/Metaphor}{metaphor} lee borrowed from \href{https://en.wikipedia.org/wiki/Zen}{Chan} \href{https://en.wikipedia.org/wiki/Buddhism}{Buddhism} was of constantly filling a cup with water, \& then emptying it, used for describing Lee's philosophy of ``casting off what is useless''. Lee considered traditional form-based martial arts, that placed emphasis on pre-arranged patterns, \href{https://en.wikipedia.org/wiki/Kata}{forms} \& techniques to be restrictive \& at worst, ineffective in dealing with chaotic self-defense situations. Lee believed that \fbox{real combat was \href{https://en.wikipedia.org/wiki/Aliveness_(martial_arts)}{alive \& dynamic}}.

Jeet June Do was conceived to be dynamic, to enable its practitioners to adapt to the constant changes \& fluctuations of live combat. He believed these decisions should be made within the context of ``real combat'' \&\texttt{/}or ``all-out sparring'' \& that it was only in this environment that a practitioner could actually deem a technique worthy of adoption.'' -- \href{https://en.wikipedia.org/wiki/Jeet_Kune_Do#Overview_and_philosophy}{Wikipedia\texttt{/}Jeet Kune Do\texttt{/}overview \& philosophy}

\subsection{Principles}
``Unlike more \href{https://en.wikipedia.org/wiki/Traditional_martial_arts}{traditional martial arts}, Jeet Kune Do is not fixed or patterned \& is a philosophy with guiding ideas. Named for the \href{https://en.wikipedia.org/wiki/Fencing}{Fencing} concept of \href{https://en.wikipedia.org/wiki/Interception}{interception} or attacking when one's opponent is about to attack, Jeet Kune Do's practitioners believe in minimal effort with maximal effect \& extreme speed.

The following are \href{https://en.wikipedia.org/wiki/Principle}{principles} that Lee incorporated into Jeet Kune Do. He felt that universal combat truths were \href{https://en.wikipedia.org/wiki/Self-evident}{self-evident}, \& would lead to combat success if followed. Familiarity with each of the ``4 ranges of combat'', in particular, is thought to be instrumental in becoming a ``total'' martial artist.

JKD believes the best defense is a strong offense, hence the principle of an ``intercepting fist''. For someone attack another hand-to-hand, the attacker must approach the target. This provides an opportunity for the targeted person to ``intercept'' the attacking movement. The principle of interception may be applied to more than intercepting the actual physical attack; \href{https://en.wikipedia.org/wiki/Nonverbal_communication}{non-verbal cues} (subtle movements of which opponent may be unaware) may also be perceived or ``intercepted'', \& thus used to one's advantage. The ``5 ways of attack'', categories that help JKD practitioners organize their fighting repertoire, comprise the offensive teachings of JKD. The concepts of ``Stop hits \& stop kicks,'' \& ``Simultaneous parrying \& punching,'' based on the concept of single fluid motions that attack while defending (in systems such as \href{https://en.wikipedia.org/wiki/%C3%89p%C3%A9e}{\'ep\'ee} fencing \& \href{https://en.wikipedia.org/wiki/Wing_Chun}{Wing Chun}), compose JKD's defensive teachings. These were modified for unarmed combat \& implemented into the JKD framework by Lee to complement the principle of interception.'' -- \href{https://en.wikipedia.org/wiki/Jeet_Kune_Do#Principles}{Wikipedia\texttt{/}Jeet Kune Do\texttt{/}principles}

\subsubsection{Stance}
``Seen in many of his film fight scenes such as in the \href{https://en.wikipedia.org/wiki/Way_of_the_Dragon}{\textit{Way of the Dragon}} where he fought against \href{https://en.wikipedia.org/wiki/Chuck_Norris}{Chuck Norris}, Bruce Lee fought in a side \href{https://en.wikipedia.org/wiki/Southpaw_stance}{southpaw} \href{https://en.wikipedia.org/wiki/Horse_stance}{southpaw horse stance}. His \href{https://en.wikipedia.org/wiki/Jab}{jabs} \& crosses came from his right hand \& followed up with a lot of sidekicks. Instead of a common \textit{check} seen in \href{https://en.wikipedia.org/wiki/Muay_thai}{muay thai}, Bruce uses an \textit{oblique leg kick} to block a potential kick. This technique is called the \textit{jeet tek} (``stop kick'' or ``intercepting kick''). He adopted other defensive concepts found in many other systems such as slipping \& rolling from \href{https://en.wikipedia.org/wiki/Boxing}{Western boxing} \& \href{https://en.wikipedia.org/wiki/Forearm}{forearm} blocks found in Eastern martial arts such as \href{https://en.wikipedia.org/wiki/Kung_Fu}{Kung Fu}.'' -- \href{https://en.wikipedia.org/wiki/Jeet_Kune_Do#Stance}{Wikipedia\texttt{/}Jeet Kune Do\texttt{/}principles\texttt{/}stance}

\subsubsection{Footwork}
``Lee's nimble \& agile skipping-like \href{https://en.wikipedia.org/wiki/Footwork_(martial_arts)}{footwork} is seen in his movies. This technique was adopted from \href{https://en.wikipedia.org/wiki/Muhammad_Ali}{Muhammad Ali}'s footwork in his \href{https://en.wikipedia.org/wiki/Boxing_career_of_Muhammad_Ali}{boxing stance}. This footwork can be achieved from practice using a \href{https://en.wikipedia.org/wiki/Skipping_rope}{jump rope} as jumping rope imitates this nimble, jumpy action that is a quick way to maneuver your way around \& away from an enemy's strikes. The footwork also has its influences from \href{https://en.wikipedia.org/wiki/Fencing}{fencing}.'' -- \href{https://en.wikipedia.org/wiki/Jeet_Kune_Do#Footwork}{Wikipedia\texttt{/}Jeet Kune Do\texttt{/}principles\texttt{/}footwork}

\subsubsection{Straight lead}
``Lee felt that the straight lead was the most integral part of Jeet Kune Do punching, saying, ``The leading straight punch is the backbone of all punching in Jeet Kune Do.'' The straight lead is not a power strike but a strike formulated for speed. It is believed that \href{https://web.archive.org/web/20170513180657/http://jkdjeetkunedo.blogspot.co.uk/2009/11/jkd-leading-straight-punch.html}{the straight lead} should always be held loosely with a slight motion, as this adds to its speed \& makes it more difficult to see \& block. The strike is believed to be not only the fastest punch in JKD, but also the most accurate. The speed is attributed to the fact that the fist is held out slightly making it closer to the target \& its accuracy is gained from the punch being thrown straight forward from one's centerline. The lead should be held \& thrown loosely \& easily, tightening only upon impact, adding to one's punch. The punch can be thrown from multiple angles \& levels.'' -- \href{https://en.wikipedia.org/wiki/Jeet_Kune_Do#Straight_lead}{Wikipedia\texttt{/}Jeet Kune Do\texttt{/}principles\texttt{/}straight lead}

\subsubsection{Non-telegraphed punch}
``Lee believed that explosive attacks, without telegraphing signs of intent, were most effective. He argued that the attacks should catch the opponent off-guard, throwing them off balance \& leaving them unable to defend against subsequent attacks. ``The concept behind this is that when you initiate your punch without any forewarning, such as tensing your shoulders or moving your foot or body, the opponent will not have enough time to react,'' Lee wrote. The key is that one must keep one's body \& arms loose, weaving one's arms slightly \& only becoming tense upon impact. Lee wanted no wind-up movements or ``get ready poses'' to prelude any JKD attacks. He explained that any twitches or slight movements before striking should be avoided as they will give the opponent signs or hints as to what is being planned \& then they will be able to strike 1st while one is preparing an attack. Consequently, the non-telegraphed movement is believed to be an essential part of Jeet Kune Do philosophy.'' -- \href{https://en.wikipedia.org/wiki/Jeet_Kune_Do#Non-telegraphed_punch}{Wikipedia\texttt{/}Jeet Kune Do\texttt{/}principles\texttt{/}non-telegraphed punch}

\subsubsection{``Be like water''}
``Lee emphasized that every situation, in fighting or in everyday life, is varied. To obtain victory, therefore, it is believed essential not to be rigid, but to be fluid \& adaptable to any situation. Lee compared it to being like water, saying ``Empty your mind, be formless, shapeless, like water. If you put water into a cup, it becomes the cup. You put water into a bottle \& it becomes the bottle. You put it in a teapot it becomes the teapot. Now water can flow, or it can crash. Be water, my friend.'' His theory behind this was that one must be able to function in any scenario one is thrown into \& should react accordingly. One should know when to speed up or slow down, when to expand \& when to contract, \& when to remain flowing \& when to crash. It is the awareness that both life \& fighting can be shapeless \& ever-changing that allows one to be able to adapt to those changes instantaneously \& bring forth the appropriate solution. Lee did not believe in styles \& felt that every person \& situation is different \& not everyone fits into a mold; one must remain flexible in order to obtain new knowledge \& victory in both life \& combat. It is believed that one must never become stagnant in the mind or method, always evolving \& moving towards improving oneself.'' -- \href{https://en.wikipedia.org/wiki/Jeet_Kune_Do#%22Be_like_water%22}{Wikipedia\texttt{/}Jeet Kune Do\texttt{/}principles\texttt{/}``be like water''}

\subsubsection{Economy of motion}
``Jeet Kune Do seeks to be econoical in time \& movement, teaching that the simplest things work best, as in Wing Chun. The economy of motion is the principle by which JKD practitioners achive:
\begin{itemize}
	\item \textit{Efficiency}: An attack that reaches its target in the least time, with maximum force
	\item \textit{Directness}: Doing what comes naturally in a disciplined way
	\item \textit{Simplicity}: Thinking in an uncomplicated manner; without ornamentation
\end{itemize}
This is meant to help a practitioner conserve both energy \& time, 2 crucial components in a physical confrontation. Maximized force seeks to end the battle quickly due to the amount of damage inflicted upon the opponent. Rapidity aims to reach the target before the opponent can react, which is half-beat faster tiing, as taught in Wing Chun \& Western boxing. Learned techniques are utilized in JKD to apply these principles to a variety of situations.

\paragraph{Stop hits.} ``When the distance is wide, the attacking opponent requires some sort of preparation. Therefore, attack him on his preparation of attack. To reach me, you must move to me. Your attack offers me an opportunity to intercept you.'' This means intercepting an opponent's attack with an attack of one's own instead of simply blocking it. It is for this concept Jeet Kune Do is named. JKD practitioners believe that this is the most difficult defensive skill to develop. This strategy is a feature of some traditional Chinese martial arts as \href{https://en.wikipedia.org/wiki/Wing_Chun}{Wing Chun}, as well as an essential component of European \'ep\'ee Fencing. Stop hits \& kicks utilize the principle of economy of motion by combining attack \& defense into 1 movement, thus minimizing the ``time'' element.

\paragraph{Simultaneous parrying \& punching.} When confronting an incoming attack, the attack is \href{https://en.wikipedia.org/wiki/Parried}{parried}or deflected, \& a \href{https://en.wikipedia.org/wiki/Counterattack}{counterattack} is delivered simultaneously. This is not as advanced as a stop hit but more effective than blocking \& counterattacking in sequence. Practiced in some Chinese martial arts such as Wing Chun, it is also known in \href{https://en.wikipedia.org/wiki/Krav_Maga}{Krav Maga} as ``bursting''. Simultaneous parrying \& punching utilize the principle of economy of motion by combining attack \& defense into 1 movement, thus minimizing the ``time'' element \& maximizing the ``energy'' element. Efficiency is gained by utilizing a parry rather than a block. By definition, a ``block'' stops an attack, whereas a parry merely re-directs it. Redirection has 2 advantages, it requires less energy to execute \& utilizes an opponent's energy against him by creating an imbalance. Efficiency is gained in that an opponent has less time to react to an incoming attack, since he is still withdrawing from his attack.

\paragraph{Low kicks.} JKD practitioners believe they should direct their kicks, as in Wing Chun, to their opponent's shins, \href{https://en.wikipedia.org/wiki/Knees}{knees}, \href{https://en.wikipedia.org/wiki/Thighs}{thighs}, \& \href{https://en.wikipedia.org/wiki/Midsection}{midsection}. These targets are the closest to the feet, provide more stability \& are more difficult to defend against. Maintaining low kicks utilizes the principle of economy of motion by reducing the distance a kick must travel, thus minimizing the ``time'' element. However, as with all other JKD principles nothing is set in stone. In a typical JKD style, if a target of opportunity presents itself, even a target above the waist, one could take advantage \& not be hampered by this principle.'' -- \href{https://en.wikipedia.org/wiki/Jeet_Kune_Do#Economy_of_motion}{Wikipedia\texttt{/}Jeet Kune Do\texttt{/}principles\texttt{/}economy of motion}

\subsubsection{3 ranges of combat}
``Long. Medium. Close. Jeet Kune Do students train in each of the aforementioned ranges equally. According to Lee, this range of training serves to differentiate JKD from other martial arts. He stated that most but not all traditional martial arts systems specialize in training at 1 or 2 ranges. His theories have been especially influential \& substantiated in the field of mixed martial arts, as the \href{https://en.wikipedia.org/wiki/Mixed_martial_arts#Phases_of_combat}{MMA Phases of Combat} are essential the same concept as the JKD combat ranges.

As a historic note, the ranges in JKD have evolved over time. Initially the ranges were categorized as short or close, medium, \& long range. These terms proved ambiguous \& some instructors eventually evolved into their more descriptive forms, although there is a lot of disagreement on whether or not this is correct. Many believe that the 3 ranges as described above are correct as distance to a target doesn't dictate what `tools' can be used. E.g., in close range, one can still kick, in addition to punching, grappling, trapping etc. To rename `close range' the trapping or even grappling range is conditioning the practitioner in believing that is all that should be done in that particular range. So for this reason many still prefer these original 3 categories.'' -- \href{https://en.wikipedia.org/wiki/Jeet_Kune_Do#Three_ranges_of_combat}{Wikipedia\texttt{/}Jeet Kune Do\texttt{/}principles\texttt{/}3 ranges of combat}

\subsubsection{5 ways of attack}
``JKD's original 5 ways of attack are:
\begin{enumerate}
	\item Simple Angular Attack or Simple Direct Attack (SDA or SAA)
	\item Attack By Combination (ABC)
	\item Progressive Indirect Attack (PIA)
	\item Immobilization Attacks (IA)
	\item Attack By Drawing (ABD)'' -- \href{https://en.wikipedia.org/wiki/Jeet_Kune_Do#Five_ways_of_attack}{Wikipedia\texttt{/}Jeet Kune Do\texttt{/}principles\texttt{/}5 ways of attack}
\end{enumerate}

\subsubsection{Centerline}
\textsf{Fig. The \href{https://en.wikipedia.org/wiki/Wing_Chun}{Wing Chun} centerline.} \textsf{Fig. Punching from the \href{https://en.wikipedia.org/wiki/Wing_Chun}{Wing Chun} centerline.}

``The centerline is an imaginary line drawn vertically along the center of a standing human body, \& refers to the space directly in front of that body. If one draws an \href{https://en.wikipedia.org/wiki/Isosceles_triangle}{isosceles triangle} on the floor, for which one's body forms the base, \& one's arms form the equal legs of the triangle, then $h$ (the height of the triangle) is the centerline. The Wing Chun concept is to exploit, control \& dominate an opponent's centerline. All attacks, defenses, \& footwork are designed to guard one's own centerline while entering the opponent's centerline space. Lee incorporated this theory into JKD from his \href{https://en.wikipedia.org/wiki/Sifu}{Sifu} \href{https://en.wikipedia.org/wiki/Ip_Man}{Ip Man}'s \href{https://en.wikipedia.org/wiki/Wing_Chun}{Wing Chun}.

The 3 guidelines for the centerline are:
\begin{itemize}
	\item The one who controls the centerline will control the fight.
	\item Protect \& maintain your own centerline while you control \& exploit your opponent's.
	\item Control the centerline by occupying it.
\end{itemize}
This notion is closely related to maintaining control of the center squares in the strategic game \href{https://en.wikipedia.org/wiki/Chess}{chess}. The concept is naturally present in \href{https://en.wikipedia.org/wiki/Xiangqi}{xiangqi} (Chinese chess), where an ``X'' is drawn on the \href{https://en.wikipedia.org/wiki/Game_board}{game board}, in front of both players' general \& advisors.'' -- \href{https://en.wikipedia.org/wiki/Jeet_Kune_Do#Centerline}{Wikipedia\texttt{/}Jeet Kune Do\texttt{/}principles\texttt{/}centerline}

\textsf{Fig. The centerline can be expressed as the height of a triangle.}

\textsf{Fig. An animation of \href{https://en.wikipedia.org/wiki/Linkage_(mechanical)}{mechanical linkage} to the shoulders of the triangle illustrates the importance of guarding the centerline.}

\subsection{Combat realism}
``1 of the premises that Lee incorporated in Jeet Kune Do was ``combat realism.'' He insisted that martial arts techniques should be incorporated based upon their effectiveness in real combat situations. This would differentiate it from other systems where there was an emphasis on ``flowery technique'', as Lee would put it. He claimed that flashy ``flowery techniques'' would arguably ``look good'' but were often not practical or would prove ineffective in street survival \& \href{https://en.wikipedia.org/wiki/Self-defense}{self-defense} situations. This premise would differentiate JKD from other ``sport''-oriented martial arts systems that were geared towards ``tournament'' or ``point systems'' (traditional martial art). Lee felt that these systems were ``artificial'' \& fooled their practitioners into a false sense of true martial skill. He felt that because these systems incorporated too many rule sets that would ultimately handicap a practitioner in self-defense situations \& that these approaches to martial arts become a ``game of tag'' leading to bad habits such as pulling punches \& other attacks; this would again lead to negative consequences in real-world situations.

Another aspect of realistic martial arts training fundamental to JKD is what Lee referred to as ``\href{https://en.wikipedia.org/wiki/Aliveness_(martial_arts)}{Aliveness}''. This is the concept of training techniques with an unwilling assistant who offers resistance. He made a reference to this concept in his famous quote ``Boards don't hit back!''. Because of this perspective of realism \& aliveness, Lee utilized \href{https://en.wikipedia.org/wiki/Safety_gear}{safety gear} from various other contact sports to allow him to spar with opponents ``full out''. This approach to training allowed practitioners to come as close as possible to real combat situations with a high degree of safety.'' -- \href{https://en.wikipedia.org/wiki/Jeet_Kune_Do#Combat_realism}{Wikipedia\texttt{/}Jeet Kune Do\texttt{/}combat realism}

\subsection{Conditioning}
``To keep up with the demand of Jeet Kune Do combat, the practitioner must condition his body. Some exercises Lee did included \textit{Da Sam Sing} or \textit{Gak Sam Sing} which is a traditional method of \href{https://en.wikipedia.org/wiki/Forearm}{forearm} conditioning practiced in Classical Kung Fu. He also did exercises simulating a fight against a 4-limbed human using the traditional \href{https://en.wikipedia.org/wiki/Mu_ren_zhuang}{Mook Yan Yong} (Cantonese) used in Wing Chun.

Bruce Lee was an avid follower of wrestler \href{https://en.wikipedia.org/wiki/Great_Gama}{Great Gama}'s training routine. He read articles about him \& how he employed his exercises to build his legendary strength for \href{https://en.wikipedia.org/wiki/Wrestling}{wrestling}, quickly incorporating them into his own routine. The training routines Lee used included isometrics as well as ``\href{https://en.wikipedia.org/wiki/Push-up}{the cat stretch}'', ``the squat'' (known as ``baithak''), \& also known as the ``deep-knee bend.'''' -- \href{https://en.wikipedia.org/wiki/Jeet_Kune_Do#Conditioning}{Wikipedia\texttt{/}Jeet Kune Do\texttt{/}conditioning}

\subsection{Notable practitioners}
``For practitioners of Jeet Kune Do, see \href{https://en.wikipedia.org/wiki/Category:Jeet_Kune_Do_practitioners}{Wikipedia\texttt{/}Category: Jeet Kune Do practitioners}.'' See a list at \href{https://en.wikipedia.org/wiki/Jeet_Kune_Do#Notable_practitioners}{Wikipedia\texttt{/}Jeet Kune Do\texttt{/}notable practitioners}

%------------------------------------------------------------------------------%

\chapter{Jordan B. Peterson. \textit{12 Rules for Life: An Antidote to Chaos}}

\section*{Introduction}
``\textit{12 Rules for Life: An Antidote\footnote{\textbf{antidote} [n] \textbf{1.} \textbf{antidote (to something)} a substance that controls the effects of a poison or disease; \textbf{2.} \textbf{antidote (to something)} anything that takes away the effects of something unpleasant.} to Chaos\footnote{\textbf{chaos} [n] [uncountable] a state of complete confusion \& lack of order; in physics, \textbf{chaos} is the property of a complex system whose behavior is so unpredictable that it appears random, especially because small changes in conditions can have very large effects; \textbf{chaos theory} is the branch of mathematics that deals with these complex systems.}} is a 2018 \href{https://en.wikipedia.org/wiki/Self-help_book}{self-help book} by the Canadian clinical\footnote{\textbf{clinical} [a] [only before noun] connected with the examination \& treatment of patients \& their illnesses.} psychologist\footnote{\textbf{psychologist} [n] a scientist who studies psychology.} \href{https://en.wikipedia.org/wiki/Jordan_Peterson}{Jordan Peterson}. It provides life advice through essays in abstract ethical\footnote{\textbf{ethical} [a] \textbf{1.} connected with beliefs \& principles about what is right \& wrong; \textbf{2.} morally correct or acceptable.} principles, psychology, mythology\footnote{\textbf{mythology} [n] [uncountable, countable] \textbf{1.} ancient myths in general; the ancient myths of a particular culture, society, etc.; \textbf{2.} \textbf{mythology (of something)} ideas that many people think are true but are in fact false.}, religion\footnote{\textbf{religion} [n] \textbf{1.} [uncountable] the belief in the existence of a god or gods, \& the activities that are connected with the worship of them; \textbf{2.} [countable] 1 of the systems of belief that are based on the belief in the existence of a particular god or gods.}, \& personal anecdotes\footnote{\textbf{anecdote} [n] [countable, uncountable] \textbf{1.} \textbf{anecdote (about somebody\texttt{/}something)} a short, interesting or funny story about a real person or event; \textbf{2.} a personal account of an event, especially one that is considered as possibly not true or accurate.}.''[$\ldots$] ``The book is written in a more accessible style than his previous academic book, \href{https://en.wikipedia.org/wiki/Maps_of_Meaning:_The_Architecture_of_Belief}{Maps of Meaning: The Artchitecture of Belief} (1999). A sequel, \href{https://en.wikipedia.org/wiki/Beyond_Order}{Beyond Order: 12 More Rules for Life}, was published in Mar 2021.'' -- \href{https://en.wikipedia.org/wiki/12_Rules_for_Life}{Wikipedia\texttt{/}12 Rules for Life}

\subsection*{Overview}

\paragraph*{Background.} ``Peterson's interest in writing the book grew out of a personal hobby of answering questions posted on \href{https://en.wikipedia.org/wiki/Quora}{Quora}; 1 such question being
\begin{question}
	\fbox{``What are the most valuable things everyone should know?'',}
\end{question}
to which his answer comprised 42 rules. The early vision \& promotion of the book aimed to include all rules, with the title ``42''. Peterson stated that it ``isn't only written for other people. It's warning to me.'''' -- \href{https://en.wikipedia.org/wiki/12_Rules_for_Life#Background}{Wikipedia\texttt{/}12 Rules for Life\texttt{/}overview\texttt{/}background}

\paragraph*{12 Rules.} ``The book is divided into chapters with each title representing 1 of the following 12 specific rules for life as explained through an essay.
\begin{enumerate}
	\item ``Stand up straight with your shoulders back.''
	\item ``Treat yourself like you are someone you are responsible for helping.''
	\item ``Make friends with people who want the best for you.''
	\item ``Compare yourself to who you were yesterday, not to who someone else is today.''
	\item ``Do not let your children do anything that makes you dislike them.''
	\item ``Set your house in perfect order before you criticize the world.''
	\item ``Pursue what is meaningful (not what is expedient\footnote{\textbf{expedient} [n] an action that is useful or necessary for a particular purpose, but not always fair or right.}).''
	\item ``Tell the truth -- or, at least, don't lie.''
	\item ``Assume that the person you are listening to might know something you don't.''
	\item ``Be precise in your speech.''
	\item ``Do not bother children when they are skate-boarding.''
	\item ``Pet a cat when you encounter\footnote{\textbf{encounter} [v] \textbf{1.} \textbf{encounter something} to experience something, especially something unpleasant or difficult, while you are trying to do something else, \textsc{synonym}: \textbf{run into something}; \textbf{2.} \textbf{encounter something\texttt{/}somebody} to discover or experience something, or meet somebody, especially something\texttt{/}somebody new, unusual or unexpected, \textsc{synonym}: \textbf{come across somebody\texttt{/}something}; [n] a meeting, especially one that is sudden or unexpected.} one on the street.'''' -- \href{https://en.wikipedia.org/wiki/12_Rules_for_Life#12_Rules}{Wikipedia\texttt{/}12 Rules for Life\texttt{/}overview\texttt{/}content}
\end{enumerate} 

\paragraph*{Content.} ``The book's central idea is that ``\fbox{suffering is built into the structure of \href{https://en.wikipedia.org/wiki/Being}{being}}'' \& although it can be unbearable\footnote{\textbf{unbearable} [a] too painful, annoying or unpleasant to deal with or accept, \textsc{synonym}: \textbf{intolerable}, \textsc{opposite}: \textbf{bearable}.}, people have a choice either to withdraw\footnote{\textbf{withdraw} [v] \textbf{1.} [transitive, intransitive] (used especially about armed forces) to make people leave a place; to leave a place; \textbf{2.} [intransitive] \textbf{withdraw (to something)} to leave a room; to go away from other people; \textbf{3.} [transitive] to move something back, out or away from something; \textbf{4.} [transitive] to take money out of a bank account or financial institution; \textbf{5.} [intransitive] to stop taking part in something; \textbf{6.} [intransitive] to stop wanting to speak to, or be with, other people; \textbf{7.} [transitive] to no longer provide or offer something; to no longer make something available; \textbf{8.} [transitive] \textbf{withdraw something} to say that you no longer agree with what you said before.}, which is a ``suicidal\footnote{\textbf{suicidal} [a] (of people) very unhappy or depressed \& feeling that they want to kill themselves; (of behavior) showing this.} gesture\footnote{\textbf{gesture} [n] \textbf{1.} [countable, uncountable] \textbf{gesture (of something)} something that you do or say to show a particular feeling or intention; \textbf{2.} [countable, uncountable] a movement that you make with your hands, your head or your face to show a particular meaning.}'', or to face \& transcend\footnote{\textbf{transcend} [v] \textbf{transcend something} to be or go beyond the usual limits of something.} it. Living in a world of chaos \& order,\fbox{ everyone has ``darkness''} that can \fbox{``turn them into the monsters they're capable of being''} to satisfy their \fbox{dark impulses\footnote{\textbf{impulse} [n] \textbf{1.} [countable, usually singular, uncountable] a sudden strong wish or need to do something, without stopping to think about the results; \textbf{2.} [countable, usually singular] something that causes somebody\texttt{/}something to do something or to develop \& make progress; \textbf{3.} [countable] a brief electric current, e.g. one that travels from a nerve to a muscle; \textbf{4.} [countable] (\textit{physics}) the change in momentum of an object due to a force.} in the right situations}. Scientific experiments like the \href{https://en.wikipedia.org/wiki/Inattentional_blindness#Invisible_Gorilla_Test}{Invisible Gorilla Test} show that perception\footnote{\textbf{perception} [n] \textbf{1.} [uncountable, countable] an idea, a belief or an image you have as a result of how you see or understand something; \textbf{2.} [uncountable] the way you notice things or the ability to notice things with the senses; in biology, \textbf{perception} refers to the processes in the nervous system by which a living thing becomes aware of events \& things outside itself; \textbf{3.} [uncountable] the ability to understand the true nature of something, \textsc{synonym}: \textbf{insight}.} is adjusted to aims, \& it is \fbox{better to seek \href{https://en.wikipedia.org/wiki/Meaning_(psychology)}{meaning} rather than happiness}. Peterson notes:
\begin{quotation}
	``It's all very well to think the meaning of life is happiness, but what happens when you're unhappy? Happiness is a great side effect. When it comes, accept it gratefully\footnote{\textbf{grateful} [a] \textbf{1.} feeling or showing thanks because somebody has done something kind for you or has done as you asked; \textbf{2.} used to make a request, especially in a letter or in a formal situation.}. But it's fleeting\footnote{\textbf{fleeting} [a] [usually before noun] lasting only a short time, \textsc{synonym}: \textbf{brief}.} \& unpredictable\footnote{\textbf{unpredictable} [a] that cannot be predicted because it changes a lot or depends on too many different things, \textsc{opposite}: \textbf{predictable}.}. It's not something to aim at -- because it's not an aim. \& if happiness is the purpose of life, what happens when you're unhappy? Then you're a failure.''
\end{quotation}
The book advances the idea that \fbox{people are born with an instinct\footnote{\textbf{instinct} [n] [uncountable, countable] a natural tendency for people \& animals to behave in a particular way, using the knowledge \& abilities that they were born with rather than thought or training.} for ethics \& meaning}, \& should take responsibility\footnote{\textbf{responsibility} [n] \textbf{1.} [uncountable, countable] a duty to deal with or take care of somebody\texttt{/}something, so that you may be blamed if something goes wrong; \textbf{2.} [uncountable] \textbf{responsibility (for something)} blame for something bad that has happened; \textbf{3.} [countable, uncountable] a moral duty to behave well with regard to somebody\texttt{/}something.} to search for meaning above their own interests (Rule 7, ``Pursue what is meaningful, not what is expedient''). Such thinking is reflected both in contemporary\footnote{\textbf{contemporary} [a] \textbf{1.} belonging to the present time, \textsc{synonym}: \textbf{modern}; \textbf{2.} (especially of people \& society) belonging to the same time as somebody\texttt{/}something else; [n] a person or thing living or existing at the same time as somebody\texttt{/}something else, especially somebody who is about the same age as somebody else.} stories e.g. \href{https://en.wikipedia.org/wiki/Pinocchio_(1940_film)}{Pinocchio}, \href{https://en.wikipedia.org/wiki/The_Lion_King}{The Lion King}, \& \href{https://en.wikipedia.org/wiki/Harry_Potter}{Harry Potter}, \& in ancient stories from the \href{https://en.wikipedia.org/wiki/Bible}{Bible}. To ``stand up straight with your shoulders back'' (Rule 1) is to ``accept the terrible responsibility of life'', to make self-sacrifice\footnote{\textbf{self-sacrifice} [n] [uncountable] (\textit{approving}) the act of not allowing yourself to have or do something in order to help other people.}, because the individual must rise above \href{https://en.wikipedia.org/wiki/Victimisation}{victimization}\footnote{\textbf{victimize} [v] [often passive] \textbf{victimize somebody} to make somebody suffer unfairly because you do not like them, their opinions or something that they have done.} \& ``conduct his or her life in a manner that requires the rejection\footnote{\textbf{rejection} [n] [uncountable, countable] \textbf{1.} the act of refusing to accept or consider something; \textbf{2.} the act of refusing to accept somebody for a job or position; \textbf{3.} the decision not to use, sell, publish, etc. something because its quality is not good enough; \textbf{4.} \textbf{rejection (of something)} an occasion when somebody's body does not accept a new organ after a transplant operation, by producing substances that attack the organ; \textbf{5.} the act of failing to give a person or an animal enough care or affection.} of immediate gratification\footnote{\textbf{gratification} [n] [uncountable, countable] (\textit{formal}) the state of feeling pleasure when something goes well for you or when your desires are satisfied; something that gives you pleasure, \textsc{synonym}: \textbf{satisfaction}.}, of natural \& perverse\footnote{\textbf{perverse} [a] showing a deliberate \& determined desire to behave in a way that most people think is wrong, unacceptable or unreasonable.} desires alike.'' The comparison to \href{https://en.wikipedia.org/wiki/Neurology}{neurological}\footnote{\textbf{neurological} [a] relating to nerves or to the science of neurology.} structures \& behavior of \href{https://en.wikipedia.org/wiki/Lobsters}{lobsters} is used as a natural example to the formation\footnote{\textbf{formation} [n] \textbf{1.} [uncountable] the action of forming something; the process of being formed; \textbf{2.} [countable] a thing that has been formed, especially in a particular place or in a particular way; \textbf{3.} [countable, uncountable] a particular arrangement or pattern of people or things.} of \href{https://en.wikipedia.org/wiki/Hierarchy}{social hierarchies}\footnote{\textbf{hierarchy} [n] \textbf{1.} [countable, uncountable] a system, especially in a society or an organization, in which people are organized into different levels of importance from highest to lowest; \textbf{2.} [countable] a system that ideas or beliefs can be arranged into.}.

The other parts of the work explore \& criticize the state of young men; the upbringing\footnote{\textbf{upbringing} [n] [singular, uncountable] the way in which a child is cared for \& taught how to behave while it is growing up.} that ignores \href{https://en.wikipedia.org/wiki/Sex_differences_in_humans}{sex differences} between boys \& girls (criticism of \href{https://en.wikipedia.org/wiki/Overprotective}{over-protection} \& \href{https://en.wikipedia.org/wiki/Tabula_rasa}{tabula rasa} model in \href{https://en.wikipedia.org/wiki/Social_science}{social sciences}); male-female \href{https://en.wikipedia.org/wiki/Interpersonal_relationship}{interpersonal relationships}; \href{https://en.wikipedia.org/wiki/School_shooting}{school shootings}; religion \& moral \href{https://en.wikipedia.org/wiki/Nihilism}{nihilism}\footnote{\textbf{nihilism} [n] [uncountable] (\textit{philosophy}) the belief that life has no meaning or purpose \& that religious \& moral principles have no value.}; \href{https://en.wikipedia.org/wiki/Relativism}{relativism}\footnote{\textbf{relativism} [n] [uncountable] the belief that truth is not always \& generally valid, but can be judged only in relation to other things, e.g. your personal situation.}; \& lack of respect for the values that built \href{https://en.wikipedia.org/wiki/Western_world}{Western society}.

In the last chapter, Peterson outlines the ways in which one can cope with the most tragic\footnote{\textbf{tragic} [a] \textbf{1.} making you feel very sad, usually because somebody has died or suffered a lot; \textbf{2.} [usually before noun] connected with tragedy ($=$ the style of literature).} events, events that are often \fbox{out of one's control}. In it, he describes his own personal struggle upon discovering that his daughter, Mikhaila, had a rare bone disease. The chapter is a meditation\footnote{\textbf{meditation} [n] \textbf{1.} [uncountable] the practice of thinking deeply, usually in silence, especially for religious reasons or in order to make your mind calm; \textbf{2.} [countable, usually plural] \textbf{meditation (on something)} serious thoughts on a particular subject that somebody writes down or speaks.} on how to maintain\footnote{\textbf{maintain} [v] \textbf{1.} \textbf{maintain something} to cause or enable a condition or situation to continue, \textsc{synonym}: \textbf{preserve}; \textbf{2.} \textbf{maintain something} to keep something at the same level or rate; \textbf{3.} to state strongly that something is true, even when some other people may not believe it; \textbf{4.} \textbf{maintain somebody\texttt{/}something} to support somebody\texttt{/}something over a long period of time by providing money, paying for food, etc.; \textbf{5.} \textbf{maintain something} to keep a building, machine, etc. in good condition by checking or repairing it regularly; \textbf{6.} \textbf{maintain a record} to write something down as a record \& keep adding the most recent information, \textsc{synonym}: \textbf{keep}.} a watchful\footnote{\textbf{watchful} [a] paying attention to what is happening in case of danger, accidents, etc.} eye on, \& cherish\footnote{\textbf{cherish} [v] (\textit{formal}) \textbf{1.} \textbf{cherish somebody\texttt{/}something} to love somebody\texttt{/}something very much \& want to protect them or it; \textbf{2.} \textbf{cherish something} to keep an idea, a hope or a pleasant feeling in your mind for a long time.}, life's small redeemable\footnote{\textbf{redeemable} [a] \textbf{redeemable (against something)} that can be exchanged for money or goods.} qualities (i.e., ``pet a cat when you encounter one''). It also outlines a practical way to deal with hardship\footnote{\textbf{hardship} [n] [uncountable, countable] a situation that is difficult \& unpleasant because you do not have enough money, food, clothes, etc.}: to shorten one's temporal\footnote{\textbf{temporal} [a] \textbf{1.} connected with or limited by time; \textbf{2.} connected with the real physical world, not spiritual matters; \textbf{3.} (\textit{anatomy}) near the temples at the side of the head.} scope of responsibility (e.g., focusing on the next minute rather than the next 3 months).

Canadian psychiatrist \& psychoanalyst \href{https://en.wikipedia.org/wiki/Norman_Doidge}{Norman Doidge} wrote \cite{Peterson2018}'s foreword.'' -- \href{https://en.wikipedia.org/wiki/12_Rules_for_Life#Content}{Wikipedia\texttt{/}12 Rules for Life\texttt{/}overview\texttt{/}content}

\begin{quotation}
	``The most influential public intellectual\footnote{\textbf{intellectual} [a] [usually before noun] connected with or using a person's ability to think in a logical way \& understand things, \textsc{synonym}: \textbf{mental}; [n] a person who is well educated \& enjoys activities in which they have to think seriously about things.} in the Western world right now.'' -- New York Times
\end{quotation}

\section*{Foreword}
``Rules? More rules? Really? Isn't life complicated\footnote{\textbf{complicated} [a] \textbf{1.} made of many different things or parts that are connected; difficult to understand, \textsc{synonym}: \textbf{complex}, \textsc{opposite}: \textbf{uncomplicated}; \textbf{2.} (of a medical condition) involving complications, \textsc{opposite}: \textbf{uncomplicated}.} enough, restricting enough, without abstract rules that don't take our unique, individual situations into account? \& given that our brains are plastic\footnote{\textbf{plastic} [n] \textbf{1.} [uncountable, countable, usually plural] a light strong material that is produced by chemical processes \& can be formed into shapes when heated. There are many different types of plastic, used to make different objects \& fabrics; \textbf{2. (plastics)} [uncountable] the science of making plastics; [a] \textbf{1.} made of plastic; \textbf{2.} (of a material or substance) easily formed into different shapes; \textbf{3.} (\textit{biology}) (of a living thing) able to adapt to change or variety in the environment.}, \& all develop differently based on our life experiences, why even expect that a few rules might be helpful to us all?

People don't clamor\footnote{\textbf{clamor} [v] \textbf{1.} [intransitive, transitive] (\textit{formal}) to demand something loudly; \textbf{2.} [intransitive] (of many people) to shout loudly, especially in a confused way; [n] (\textit{formal}) \textbf{1.} [singular] a loud noise, especially on that is made by a lot of people or animals; \textbf{2.} [uncountable, countable] \textbf{clamor (for something)} a demand for something made by a lot of people.} for rules, even in the Bible $\ldots$ as when Moses comes down the mountain, after a long absence\footnote{\textbf{absence} [n] \textbf{1.} [uncountable] the fact of somebody\texttt{/}something not existing or not being available, \textsc{synonym}: \textbf{lack}, \textsc{opposite}: \textbf{presence}; \textbf{2.} [uncountable, countable] the fact of somebody being away from a place where they are usually expected to be; the occasion or period of time when somebody is away.}, bearing the tablets\footnote{\textbf{tablet} [n] \textbf{1.} (\textit{especially British English}) a small round solid piece of medicine that you swallow, \textsc{synonym}: \textbf{pill}; \textbf{2.} a flat piece of stone, etc. with words or symbols on it; \textbf{3.} (also \textbf{tablet computer}) (\textit{trademark} in the UK) a small, light, flat computer that can be used without a keyboard or mouse, by touching the screen.} inscribed\footnote{\textbf{inscribe} [v] \textbf{1.} [often passive] to write or cut words, your name, etc. onto something; \textbf{2.} [often passive] \textbf{inscribe something $+$ adv.\texttt{/}prep.} to make something present in, on, etc. something.} with 10 commandments\footnote{\textbf{commandment} [n] a law given by God, especially any of \textbf{the Ten Commandments} given to the Jews in the Bible.}, \& finds the Children of Israel in revelry\footnote{\textbf{revelry} [n] [uncountable] noisy fun, usually involving a lot of eating \& drinking, \textsc{synonym}: \textbf{festivity, merrymaking}.}. They'd been Pharaoh's slaves \& subject to his tyrannical\footnote{\textbf{tyrannical} [a] using power or authority over people in an unfair \& cruel way.} regulations\footnote{\textbf{regulation} [n] \textbf{1.} [countable, usually plural] an official rule made by a government or some other authority; \textbf{2.} [uncountable] the act of controlling something by means of rules; \textbf{3.} [uncountable] the act of controlling how a machine or system operates or how something behaves.} for 400 years, \& after that Moses subjected them to the harsh\footnote{\textbf{harsh} [a] \textbf{1.} very strict; \textbf{2.} (of weather or living conditions) very difficult \& unpleasant to live in.} desert\footnote{\textbf{desert} [n] [uncountable, countable] a large area of land that has very little water \& very few plants growing on it. Many desert areas are covered by sand; [v] \textbf{1.} [transitive, often passive] \textbf{desert somebody} to leave somebody without help or support, \textsc{synonym}: \textbf{abandon}; \textbf{2.} [transitive, often passive] \textbf{desert something} to go away from a place \& leave it empty, \textsc{synonym}: \textbf{abandon}; \textbf{3.} [intransitive, transitive] \textbf{desert (something)} to leave the armed forces without permission; \textbf{4.} [transitive] \textbf{desert (something) 9for something} to stop using, buying or supporting something.} wilderness\footnote{\textbf{wilderness} [n] [usually singular] a large area of land that has never been developed or used for growing crops because it is difficult to live there.} for another 40 years, to purify\footnote{\textbf{purify} [v] \textbf{1.} \textbf{purify something} to make something pure by removing anything that is bad, unpleasant or not wanted; \textbf{2.} [often passive] (\textit{specialist}) to separate a pure form of a substance from a mixture that contains it; to remove the impurities from a substance; \textbf{3.} \textbf{purify somebody\texttt{/}yourself} to make somebody\texttt{/}yourself pure by removing evil, especially in a ceremony.} them of their slavishness. Now, free at last, they are unbridled\footnote{\textbf{unbridled} [a] [usually before noun] (\textit{literary}) lacking control \& therefore extreme.}, \& have lost all control as they dance wildly around an idol, a golden calf\footnote{\textbf{calf} [n] \textbf{1.} [countable] the back part of the leg between the ankle \& the knee; \textbf{2.} [countable] a young cow; \textbf{3.} [countable] a young animal of some other type such as a young elephant or whale; \textbf{4.} [countable] (also \textbf{calfskin}) soft thin leather made from the skin of calves, used especially for making shoes \& clothing.}, displaying all manner of corporeal\footnote{\textbf{corporeal} [a] (\textit{formal}) \textbf{1.} that can be touched; physical rather than spiritual; \textbf{2.} of or for the body.} corruption\footnote{\textbf{corruption} [n] \textbf{1.} [uncountable] dishonest or illegal behavior, especially of people in authority; \textbf{2.} [uncountable] \textbf{corruption (of something)} the act or effect of making somebody change from moral to immoral standards of behavior; \textbf{3.} [countable, usually singular] \textbf{corruption of something} the form of a word or phrase that has become changed from its original form in some way; \textbf{4.} [uncountable] (\textit{computing}) the process by which mistakes are introduced into a computer file, etc. with the result that the data in it is no longer correct.}.

``I've got some good news $\ldots$ \& I've got some bad news,'' the lawgiver yells to them. ``Which do you want 1st?''

``The good news!'' the hedonists\footnote{\textbf{hedonist} [n] a person who believes that pleasure is the most important thing in life.} reply.

``I got Him from 15 commandments down to 10!''

``Hallelujah!'' cries the unruly\footnote{\textbf{unruly} [a] difficult to control or manage, \textsc{synonym}: \textbf{disorderly}.} crowd. ``\& the bad?''

``Adultery\footnote{\textbf{adultery} [n] [uncountable] sex between a married person \& somebody who is not their husband or wife.} is still in.''

So rules there will be -- but, please, not too many. We are ambivalent\footnote{\textbf{ambivalent} [a] having or showing both good \& bad feelings about somebody\texttt{/}something.} about rules, even when we know they are good for us. If we are spirited souls, if we have character, rules seem restrictive, an affront\footnote{\textbf{affront} [n] [usually singular] \textbf{affront (to somebody\texttt{/}something)} a remark or an action that offends somebody\texttt{/}something, \textsc{synonym}: \textbf{insult}; [v] [usually passive] (\textit{formal}) to say or do something that offends somebody, \textsc{synonym}: \textbf{insult}.} to our sense of agency\footnote{\textbf{agency} [n] \textbf{1.} [countable] a business or an organization that provides a particular service especially on behalf of other businesses or organizations; \textbf{2.} [countable] (\textit{especially North American English}) a government department that provides a particular service; \textbf{3.} [uncountable, countable] a person or thing that acts to produce a particular result; action that produces a particular result.} \& our pride in working out our own lives. \fbox{Why should we be judged according to another's rule?}

\& judged we are. After all, God didn't give Moses ``The Ten Suggestions,'' he gave Commandments; \& if I'm a free agent, my 1st reaction to a command might just be that nobody, not even God, tells me what to do, even if it's good for me. But the story of the golden calf also reminds us that \fbox{without rules we quickly becomes slaves to our passions} -- \& there's nothing freeing about that.

\& the story suggests something more: unchaperoned\footnote{\textbf{unchaperoned} [a] unaccompanied or unsupervised.}, \& left to our own untutored\footnote{\textbf{untutored} [a] (\textit{formal}) not having been formally taught about something.} judgment, we are quick to aim low \& worship qualities that are beneath\footnote{\textbf{beneath} [prep] \textbf{1.} in or to a lower position than somebody\texttt{/}something; under somebody\texttt{/}something; \textbf{2.} behind an appearance or feeling; \textbf{3.} not good enough for somebody; [adv] \textbf{1.} in or to a lower position; \textbf{2.} hidden behind an appearance or feeling.} us -- in this case, an artificial\footnote{\textbf{artificial} [a] \textbf{1.} made or produced by humans to copy something natural, rather than occurring naturally; \textbf{2.} created by people; not happening naturally.} animal that brings out our own animal instincts\footnote{\textbf{instinct} [n] [uncountable, countable] a natural tendency for people \& animals to behave in a particular way, using the knowledge \& abilities that they were born with rather than thought or training.} in a completely unregulated\footnote{\textbf{unregulated} [a] not controlled by laws or official rules.} way. The old Hebrew story makes it clear how the ancients felt about our prospects\footnote{\textbf{prospect} [n] \textbf{1.} [uncountable, singular] the possibility that something will happen; \textbf{2.} [singular] an idea of what might or will happen in the future; \textbf{3.} \textbf{(prospects)} [plural] the chances of being successful.} for civilized\footnote{\textbf{civilized} [a] \textbf{1.} well-organized socially with a very developed culture \& way of life; \textbf{2.} having laws \& customs that are fair \& morally acceptable.} behavior in the absence of rules that seek to elevate\footnote{\textbf{elevate} [v] \textbf{1.} \textbf{elevate something} (\textit{specialist}) to make the level of something increase; \textbf{2.} \textbf{elevate something} \textit{specialist} to lift something up or put something in a higher position; \textbf{3.} \textbf{elevate somebody\texttt{/}something (to\texttt{/}into something)} to give somebody\texttt{/}something a higher position or rank; \textbf{4.} \textbf{elevate something} to improve a person's mood, so  that they feel happy.} our gaze\footnote{\textbf{gaze} [n] [usually singular] a long steady look at somebody\texttt{/}something; [v] [intransitive] \textbf{$+$ adv.\texttt{/}prep.} to look steadily at somebody\texttt{/}something for a long time, either because you are very interested or surprised, or because you are thinking or something else.} \& raise our standards.

1 neat\footnote{\textbf{neat} [a] \textbf{1.} in good order; carefully done or arranged; \textbf{2.} simple but clever; \textbf{3.} containing or made out of just 1 substance; not mixed with anything else.} thing about the Bible story is that it doesn't simply list its rules, as lawyers or legislators\footnote{\textbf{legislator} [n] a member of a group of people that has the power or make laws.} or administrators\footnote{\textbf{administrator} [n] \textbf{1.} a person whose job is to organize the work of a business, school or other organization; \textbf{2.} (\textit{British English, law}) a person officially chosen to manage the financial affairs of a business that cannot pay its debts.} might; it embeds\footnote{\textbf{embed} [v] [usually passive] \textbf{1.} to make something a fixed \& important part of something else, that is difficult to change or remove; \textbf{2.} \textbf{embed something (in something)} to fix something firmly into a substance or solid object; \textbf{3.} \textbf{embed something (in something)} to make images, sound, software, etc. part of a computer program; \textbf{4.} \textbf{embed something} (\textit{linguistics}) to place a sentence inside another sentence.} them in a dramatic\footnote{\textbf{dramatic} [a] \textbf{1.} (of a change or an event) sudden, very great \& often surprising; \textbf{2.} exciting \& impressive; \textbf{3.} [usually before noun] connected with the theater or plays.} tale\footnote{\textbf{tale} [n] \textbf{1.} a story created using the imagination, especially one that is full of action \& adventure; \textbf{2.} an exciting spoken description of an event, which may not be completely true.} that illustrates why we need them, thereby making them easier to understand. Similarly, in this book Prof. Peterson doesn't just propose\footnote{\textbf{propose} [v] \textbf{1.} to suggest a plan or an idea for people to consider \& decide on; \textbf{2.} to suggest an explanation of something for people to consider.} his 12 rules, he tells stories, too, bringing to bear\footnote{\textbf{bear} [v] \textbf{1.} \textbf{bear something} to have something as a characteristic or feature; to be connected with something; \textbf{2.} \textbf{bear something} to have a particular mark, word or symbol that can be seen; \textbf{3.} \textbf{bear something} to have a particular name; \textbf{4.} \textbf{bear something} to take responsibility for something difficult; to be affected by or deal with something unpleasant. If somebody \textbf{cannot bear} something, they feel unable to deal with it or accept it: \textit{Her jealous husband could not bear the possibility of his wife talking to another man.} The short form `can't\texttt{/}couldn't bear' is not suitable in academic writing, unless you are quoting.; \textbf{5.} to have a feeling, especially a negative feeling; \textbf{6.} \textbf{bear (doing) something} to be suitable for something; to be worth doing. If something \textbf{does not bear close inspection}, it will be found to be unacceptable when carefully examined: \textit{This claim does not bear close inspection.} If something \textbf{does not bear comparison} with something else, it is not nearly as good: \textit{Her later work does not bear comparison with her earlier novels.}; \textbf{7.} \textbf{bear somebody\texttt{/}something} (\textit{formal}) to carry or hold somebody\texttt{/}something; \textbf{8.} (\textit{formal}) to give birth to a child; \textbf{9.} \textbf{bear something} (\textit{formal}) to produce flowers or fruit.} his knowledge of many fields as he illustrates \& explains why the best rules do not ultimately\footnote{\textbf{ultimately} [adv] \textbf{1.} in the end, finally; \textbf{2.} at the most basic \& important level, \textsc{synonym}: \textbf{basically, essentially}.} restrict us but instead facilitate\footnote{\textbf{facilitate} [v] \textbf{facilitate something} to make an action or a process possible or easier.} our goals \& make for fuller, freer lives.

The 1st time I [\textsc{Norman Doidge}] met \textsc{Jordan Peterson} was on Sep 12, 2004, at the home of 2 mutual friends, TV producer Wodek Szemberg \& medical internist\footnote{\textbf{internist} [n] (\textit{North American English}) a doctor who is a specialist in the treatment of diseases of the organs inside the body \& who does not usually do medical operations.} Estera Bekier. It was Wodek's birthday party. Wodek \& Estera are Polish \'emigr\'es who grew up within the Soviet empire\footnote{\textbf{empire} [n] \textbf{1.} a group of countries or states that are controlled by 1 ruler or government; \textbf{2.} a group of commercial organizations controlled by 1 person or company.}, where it was understood that many topics were off limits, \& that casually\footnote{\textbf{casual} [a] \textbf{1.} [usually before noun] without paying attention to detail; \textbf{2.} [usually before noun] not showing much care or thought; \textbf{3.} [usually before noun] (of a relationship) lasting only a short time \& without deep affection; \textbf{4.} [usually before noun] (\textit{British English}) (of work) not permanent; not regular; \textbf{5.} not formal; \textbf{6.} [only before noun] happening by chance; doing something by chance.} questioning certain social arrangements \& philosophical ideas (not to mention the regime\footnote{\textbf{regime} [n] \textbf{1.} a government, especially one that has not been elected in a fair way; \textbf{2.} a method or system of organizing or managing something; \textbf{3.} the conditions under which a natural, scientific or industrial process occurs; \textbf{4.} $=$ \textbf{regimen}.

\textbf{regimen} [n] (also \textbf{regime}) a course of medical treatment \& sometimes changes to diet \& behavior that somebody has to follow in order to recover from or control an illness.} itself) could mean big trouble.

But now, host\footnote{\textbf{host} [n] \textbf{1.} (\textit{biology}) an animal or a plant on which another animal or plant lives \& feeds; \textbf{2.} a country, a city or an organization that arranges \& holds a special event; \textbf{3.} a country that provides homes \& work for people who come from another country; \textbf{4.} a country where a company that is based in another country does business; \textbf{5.} \textbf{host of something} a large number of people or things; \textbf{6.} the main computer in a network that controls or supplies information to other computers that are connected to it; [v] \textbf{1.} \textbf{host something} to organize an event to which others are invited \& make all the arrangements for them; \textbf{2.} \textbf{host something} to store a website on a computer connected to the Internet, usually in return for payment.} \& hostess\footnote{\textbf{hostess} [n] \textbf{1.} a woman who invites guests to a meal, a party, etc.; a woman who has people staying at her home; \textbf{2.} a woman who is employed to welcome \& entertain people at a nightclub; \textbf{3.} a woman who introduces \& talks to guests on a television or radio show, \textsc{synonym}: \textbf{comp\`ere}; \textbf{4.} (\textit{North American English}) a woman who welcomes the customers in a restaurant.} luxuriated\footnote{\textbf{luxuriate in} [phrasal verb] \textbf{luxuriate in something} to relax while enjoying something very pleasant.} in easygoing\footnote{\textbf{easygoing} [a] relaxed \& happy to accept things without worrying or getting angry.}, honest\footnote{\textbf{honest} [a] \textbf{1.} always telling the truth, \& never stealing or deceiving people, \textsc{opposite}: \textbf{dishonest}; \textbf{2.} not hiding the truth about something.} talk, by having elegant\footnote{\textbf{elegant} [a] \textbf{1.} (of people or their behavior) attractive \& showing a good sense of style; \textbf{2.} (of clothes, places \& things) attractive \& designed well; \textbf{3.} (of a plan or an idea) clever but simple.} parties devoted to the pleasure\footnote{\textbf{pleasure} [n] \textbf{1.} [uncountable] a state of feeling or being happy or satisfied; the activity of enjoying yourself, \textsc{synonym}: \textbf{enjoyment}; \textbf{2.} [countable] a thing that makes you happy or satisfied.} of saying what you \textit{really} thought \& hearing others do the same, in an uninhibited\footnote{\textbf{uninhibited} [a] behaving or expressing yourself freely without worrying about what other people think, \textsc{synonym}: \textbf{unrestrained}, \textsc{opposite}: \textbf{inhibited}.} give-\&-take. Here, the rule was ``Speak your mind.'' If the conversation turned to politics\footnote{\textbf{politics} [n] \textbf{1.} [uncountable $+$ singular or plural verb] the activities involved in getting \& using power in public life, \& being able to influence decisions that effect a country or society; \textbf{2.} [uncountable $+$ singular or plural verb] the activities of governments concerning the political relations between states; \textbf{3.} [uncountable $+$ singular or plural verb] matters concerned with getting or using power within a particular group of organization; \textbf{4.} [plural] a person's political views or beliefs; \textbf{5.} [uncountable] $=$ \textbf{political science}; \textbf{6.} [singular] \textbf{politics (of something)} a system of political beliefs; a state of political affairs; \textbf{7.} [singular, uncountable $+$ singular or plural verb] \textbf{politics (of something)} the principles connected with a particular area of activity or interest, especially when concerned with power \& status.}, people of different political\footnote{\textbf{political} [a] \textbf{1.} connected with the state, government or public affairs; \textbf{2.} connected with the different groups working in politics, especially their policies \& the competition between them; \textbf{3.} (of people) interested in or active in politics; \textbf{4.} concerned with the competition for power within an organization, rather than with matters of principle.} persuasions\footnote{\textbf{persuasion} [n] \textbf{1.} [uncountable] the act of persuading somebody to do something or to believe something; \textbf{2.} [countable, uncountable] a particular set of beliefs, especially about religion or politics.} spoke to each other -- indeed, looked forward to it -- in a manner that is increasingly rare. Sometimes Wodek's own opinions, or truths, exploded out of him, as did his laugh. Then he'd hug whoever had made him laugh or provoked\footnote{\textbf{provoke} [v] \textbf{1.} \textbf{provoke something} to cause a particular reaction or have a particular effect; \textbf{2.} to say or do something in order to produce a strong reaction from somebody, usually anger.} him to speak his mind with greater intensity\footnote{\textbf{intensity} [n] \textbf{1.} [uncountable, singular] \textbf{intensity (of something)} the state or quality of being strong or intense; \textbf{2.} [uncountable, countable] the strength of something, e.g. light, that can be measured.} than even he might have intended. This was the best part of the parties, \& this frankness\footnote{\textbf{frank} [a] \textbf{1.} (\textbf{franker, frankest}) (\textbf{more frank} is also common) honest \& direct in what you say, sometimes in a way that other people might not like; \textbf{2.} (\textit{medical}) that cannot be confused with something else; obvious.}, \& his warm embraces\footnote{\textbf{embrace} [v] \textbf{1.} \textbf{embrace something} to accept an idea, a proposal, a set of beliefs, etc., especially when it is done with enthusiasm; \textbf{2.} \textbf{embrace something} to include something; \textbf{3.} \textbf{embrace somebody} to put your arms around somebody as a sign of love or friendship; [n] [countable, uncountable].}, made it worth provoking him. Meanwhile, Estera's voice lilted\footnote{\textbf{lilt} [n] [singular] \textbf{1.} the pleasant way in which a person's voice rises \& falls; \textbf{2.} a regular rising \& falling pattern in music, with a strong rhythm.} across the room on a very precise path towards its intended listener. \fbox{Truth explosions didn't make the atmosphere any less easygoing for the company} -- they made for more truth explosions! -- liberating\footnote{\textbf{liberate} [v] \textbf{1.} to free a country or a person from the control of somebody\texttt{/}something else; \textbf{2.} \textbf{liberate somebody\texttt{/}something (from something)} to free somebody\texttt{/}something from something that limits their ability to do things or enjoy life; \textbf{3.} (\textit{chemistry, physics}) to release gas, energy, etc. as a result of a chemical reaction or physical process.} us, \& more laughs, \& making the whole evening more pleasant, because with de-repressing\footnote{\textbf{repress} [v] \textbf{1.} \textbf{repress something} to try not to have or show an emotion, a thought, etc. In Freudian psychology, \textbf{repress} has a particular meaning, which is to stop yourself having particular thoughts or feelings so completely that they become or remain unconscious; \textbf{2.} [often passive] \textbf{repress somebody\texttt{/}something} to use political \&\texttt{/}or military force to control a group of people \& restrict their freedom, \textsc{synonym}: \textbf{put something down, suppress}; \textbf{3.} \textbf{repress something} (\textit{biology}) to prevent a gene from being expressed.} Eastern Europeans like the Szemberg-Bekiers, you always knew with what \& with whom you were dealing, \& that frankness was enlivening\footnote{\textbf{enliven} [v] (\textit{formal}) \textbf{enliven something} to make something more interesting or more fun.}. Honor\'e de Balzac, the novelist\footnote{\textbf{novelist} [n] a person who writes novels.}, once described the balls \& parties in his native France, observing that what appeared to be a single party was always really 2. In the 1st hours, the gathering was suffused\footnote{\textbf{suffuse} [v] [often passive] (\textit{literary}) \textbf{suffuse somebody\texttt{/}something (with something)} (especially of a color, light or feeling) to spread all over or through somebody\texttt{/}something.} with bored people posing\footnote{\textbf{pose} [v] \textbf{1.} [transitive] \textbf{pose something} to create a problem that has to be dealt with; \textbf{2.} [transitive] \textbf{pose something} to ask a question, especially one that needs serious thought, \textsc{synonym}: \textbf{raise}; \textbf{3.} [intransitive] \textbf{pose as somebody\texttt{/}something} to pretend to be somebody\texttt{/}something that you are not; \textbf{4.} [intransitive] \textbf{pose (for somebody\texttt{/}something)} to sit or stand in a particular position in order to be painted, drawn or photographed.} \& posturing\footnote{\textbf{posturing} [n] [uncountable, countable] (\textit{disapproving}) behavior that is not natural or sincere but is intended to attract attention or to have a particular effect.}, \& attendees who came to meet perhaps 1 special person who would confirm them in their beauty \& status. Then, only in the very late hours, after most of the guests had left, would the 2nd party, the real party, begin. Here the conversation was shared by each person present, \& open-hearted\footnote{\textbf{open-hearted} [a] kind \& friendly.} laughter replaced the starchy\footnote{\textbf{starchy} [a] \textbf{1.} (of food) containing a lot of starch; \textbf{2.} (\textit{informal, disapproving}) (of a person or their behavior) very formal; not friendly or relaxed.} airs. At Estera \& Wodek's parties, this kind of wee-hours-of-the-morning disclosure\footnote{\textbf{disclosure} [n] \textbf{1.} [uncountable] \textbf{disclosure (of something) (to somebody)} the act of making something known or public that was previously secret or private, \textsc{synonym}: \textbf{revelation}; \textbf{2.} [countable] \textbf{disclosure} (about somebody\texttt{/}something) information or a fact that is made known or public that was previously secret or private, \textsc{synonym}: \textbf{revelation}.} \& intimacy\footnote{\textbf{intimate} [a] \textbf{1.} (of a link between things) very close; \textbf{2.} (of people) having a close \& friendly relationship; \textbf{3.} sexual; \textbf{4.} private \& personal, often in a sexual way; \textbf{5.} (of a place or situation) encouraging close, friendly relationships; \textbf{6.} (of knowledge) very detailed \& thorough.

\textbf{intimacy} [n] [uncountable, countable, usually plural].} often began as soon as we entered the room.

Wodek is a silver-haired, lion-maned hunter, always on the lookout for potential public intellectuals, who knows how to spot people who can \textit{really} talk in front of a TV  camera \& who look authentic\footnote{\textbf{authentic} [a] \textbf{1.} known to be real \& genuine \& not a copy, \textsc{synonym}: \textbf{genuine}; \textbf{2.} true \& accurate; based on fact; \textbf{3.} made to be exactly like the original.} because they are (the camera picks up on that). He often invites such people to these salons\footnote{\textbf{salon} [n] \textbf{1.} a shop that gives customers hair or beauty treatment or that sells expensive clothes; \textbf{2.} (\textit{old-fashioned}) a room in a large house used for entertaining guests; \textbf{3.} (in the past) a regular meeting of writers, artists \& other guests at the house of a famous or important person.}. That day Wodek brought a psychology professor, from my own University of Toronto, who fit the bill: intellect \& emotion in tandem\footnote{\textbf{tandem} [n] \textbf{in tandem (with somebody\texttt{/}something)} [idiom] a thing that works or happens in tandem with something else works together with it or happens at the same time as it.}. Wodek was the 1st to put \textsc{Jordan Peterson} in front of a camera, \& thought of him as a teacher in search of students -- because he was always ready to explain. \& it helped that he liked the camera \& that camera liked him back.

That afternoon there was a large table set outside in the Szemberg-Beliers' garden; around it was gathered the usual \fbox{collection of lips \& ears}, \& loquacious\footnote{\textbf{loquacious} [a] (\textit{formal}) talking a lot, \textsc{synonym}: \textbf{talkative}.} virtuosos\footnote{\textbf{virtuoso} [n] (plural \textbf{virtuosos, virtuosi}) a person who shows very great skill at doing something, especially playing a musical instrument; [a] [only before noun] showing extremely great skill.}. We seemed, however, to be plagued\footnote{\textbf{plague} [v] \textbf{1.} \textbf{plague somebody\texttt{/}something (with something)} to cause pain or trouble to somebody\texttt{/}something over a period of time, \textsc{synonym}: \textbf{trouble}; \textbf{2.} \textbf{plague somebody (with something)} to annoy somebody or create problems, especially by asking for something, demanding attention, etc., \textsc{synonym}: \textbf{hound}; [n] \textbf{1.} (also \textbf{the plague}) (also \textbf{bubonic plague}) [uncountable] a disease spread by rats that causes a high temperature, swellings ($=$ areas that are larger \& rounder than usual) on the body \& usually death; \textbf{2.} [countable] any disease that spreads quickly \& kills a lot of people, \textsc{synonym}: \textbf{epidemic}; \textbf{3.} [countable] \textbf{plague of something} large numbers of an animal or insect that come into an area \& cause great damage.} by a buzzing\footnote{\textbf{buzz} [v] \textbf{1.} [intransitive] (of a bee) to make a continuous low sound; \textbf{2.} [intransitive] to make a sound like a bee buzzing; \textbf{3.} [intransitive] to be full of excitement, activity, etc.; \textbf{4.} [intransitive, transitive] \textbf{buzz (something) (for somebody\texttt{/}something)} to call somebody to come by pressing a buzzer; \textbf{5.} [transitive] \textbf{buzz somebody\texttt{/}something} (\textit{informal}) to fly very close to somebody\texttt{/}something, especially as a warning or threat; [n] \textbf{1.} [countable, usually singular] (also \textbf{buzzing} [uncountable, singular]) a continuous sound like the one that a bee, a buzzer or other electronic device makes; \textbf{2.} [singular] the sound of people talking, especially in an excited way; \textbf{3.} [singular, uncountable] (\textit{informal}) a strong feeling of pleasure, excitement or achievement; \textbf{4.} \textbf{the buzz} [singular] (\textit{informal}) news that people tell each other that may or may not be true, \textsc{synonym}: \textbf{rumor}.} paparazzi\footnote{\textbf{paparazzo} [n] (also \textbf{pap}) (plural \textbf{paparazzi}) [usually plural] a photographer who follows famous people around in order to get interesting photographs of them to sell to a newspaper.} of bees, \& here was this new fellow\footnote{\textbf{fellow} [n] \textbf{1.} [usually plural] a person that you work with or that is like you; a thing that is similar to the one mentioned; \textbf{2.} (\textit{British English}) a senior member of some colleges or universities; \textbf{3.} a member of an academic or professional organization; \textbf{4.} (\textit{especially North American English}) a graduate student who holds a fellowship; [a] [only before noun] used  to describe somebody who is the same as you in some way, or in the same situation.} at the table, with an Albertan\footnote{\textbf{Alberta} [n] a province in western Canada, east of British Columbia \& west of Saskatchewan. The capital is Edmonton.} accent\footnote{\textbf{accent} [n] \textbf{1.} a way of pronouncing the words of a language that shows which country, area or social class a person comes from; \textbf{2.} the emphasis that you should give to part of a word when saying it, \textsc{synonym}: \textbf{stress}; \textbf{3.} a mark on a letter to show that it should be pronounced in a particular way; \textbf{4.} [singular] \textbf{accent (on something)} a special importance that is given to something, \textsc{synonym}: \textbf{emphasis}.}, in cowboy boots, who was ignoring them, \& kept on talking. He kept talking while the rest of us were playing musical chairs to keep away from the pests\footnote{\textbf{pest} [n] an insect or animal that destroys plants, food, etc.}, yet also trying to remain at the table because this new addition to our gatherings was so interesting.

He had this odd habit of speaking about the deepest questions to whoever was at this table -- most of them new acquaintances\footnote{\textbf{acquaintance} [n] \textbf{1.} [countable] a person that you know but who is not a close friend; \textbf{2.} [uncountable, countable] \textbf{acquaintance (with somebody)} (\textit{formal}) slight friendship; \textbf{3.} [uncountable, countable] \textbf{acquaintance with something} (\textit{formal}) knowledge of something.} -- as though he were just making small talk. Or, if he did do small talk, the interval between ``How do you know Wodek \& Estera?'' or ``I was a beekeeper once, so I'm used to them'' \& more serious topics would be nanoseconds\footnote{\textbf{nanosecond} [n] (abbr. \textbf{ns}) $10^{-3}$ second.}.

One might hear such questions discussed at parties where professors \& professionals\footnote{\textbf{professional} [n] a person who does a job that needs special training \& a high level of education.} gather, but usually the conversation would remain between 2 specialists\footnote{\textbf{specialist} [n] \textbf{1.} a doctor who has specialized in a particular area of medicine; \textbf{2.} \textbf{specialist (in something)} a person who is an expert in a particular area of work or study; [a] [only before noun] \textbf{1.} connected with a doctor who has specialized in a particular area of medicine; \textbf{2.} having or involving detailed knowledge of a particular topic or area of study.} in the topic, off in a corner, or if shared with the whole group it was often not without someone preening\footnote{\textbf{preen} [v] \textbf{1.} [transitive, intransitive] \textbf{preen (yourself)} (\textit{usually disapproving}) to spend a lot of time making yourself look attractive \& then admiring your appearance; \textbf{2.} [transitive] \textbf{preen yourself (on something)} (\textit{usually disapproving}) to feel very pleased with yourself about something \& show other people how pleased you are; \textbf{3.} [intransitive, transitive] \textbf{preen (itself)} (of a bird) to clean itself or make its feathers smooth with its beak.}. But this Peterson, though erudite\footnote{\textbf{erudite} [a] (\textit{formal, approving}) having or showing great knowledge that is gained from academic study, \textsc{synonym}: \textbf{learned}.}, didn't come across as a pedant\footnote{\textbf{pedant} [n] (\textit{disapproving}) a person who is too concerned with small details or rules especially when learning or teaching.}. He had the enthusiasm of a kid who had just learned something new \& had to share it. He seemed to be assuming, as a child would -- before learning how dulled\footnote{\textbf{dull} [v] \textsf{pain} \textbf{1.} [transitive, intransitive] \textbf{dull (something)} to make a pain or an emotion weaker or less severe; to become weaker or less severe; \textsf{person} \textbf{2.} [transitive] \textbf{dull somebody} to make a person slower or less lively; \textsf{colors\texttt{/}sounds} \textbf{3.} [intransitive, transitive] to become less bright, clean or sharp; to make something less bright, clean or sharp; [a] \textsf{boring} \textbf{1.} not interesting or exciting, \textsc{synonym}: \textbf{dreary}; \textsf{light\texttt{/}colors} \textbf{2.} not bright or shiny; \textsf{weather} \textbf{3.} not bright, with a lot of clouds, \textsc{synonym}: \textbf{overcast}; \textsf{sounds} \textbf{4.} not clear or cloud; \textsf{pain} \textbf{5.} not very severe, but continuous; \textsf{person} \textbf{6.} slow in understanding, \textsc{synonym}: \textbf{stupid}; \textsf{trade} \textbf{7.} (\textit{especially North American English}) not busy; slow.} adults can become -- that if he thought something was interesting, then so might others. There was something boyish\footnote{\textbf{boyish} [a] (\textit{approving}) looking or behaving like a boy, in a way that is attractive.} in the cowboy, in his broaching\footnote{\textbf{broach} [v] \textbf{broach something (to\texttt{/}with somebody)} to begin talking about a subject that is difficult to discuss, especially because it is embarrassing or because people disagree about it.} of subjects as though we had all grown up together in the same small town, or family, \& had all been thinking about the very same problems of human existence\footnote{\textbf{existence} [n] \textbf{1.} [uncountable, countable, usually singular] the state or fact of happening or being found in a particular place, time or situation; the state of being alive; \textbf{2.} [uncountable] \textbf{existence (of something)} the fact of being real; \textbf{3.} [countable, usually singular] a way of living, especially when this is difficult.} all along.

Peterson wasn't really an ``eccentric''\footnote{\textbf{eccentric} [a] considered by other people to be strange or unusual; [n] a person who is considered by other people to be strange or unusual.}; he had sufficient conventional\footnote{\textbf{conventional} [a] \textbf{1.} [usually before noun] based on what is generally believed; following the way something is usually done; \textbf{2.} (\textit{often disapproving}) tending to follow what is done or considered acceptable by society in general; normal \& ordinary, \& perhaps not very interesting, \textsc{opposite}: \textbf{unconventional}; \textbf{3.} [usually before noun] (especially of weapons) not nuclear; \textbf{4.} (of literature, art or the theater) using a traditional style or method.} chops\footnote{\textbf{chop} [v] \textbf{1.} to cut something into pieces with a sharp tool such as a knife; \textbf{2.} [usually passive] (\textit{informal}) to suddenly stop providing or allowing something; to suddenly reduce something by a large amount, \textsc{synonym}: \textbf{cut}; \textbf{3.} \textbf{chop somebody\texttt{/}something} to hit somebody\texttt{/}something downwards with a quick, short movement; [n] \textbf{1.} [countable] a thick slide of meat with a bone attached to it, especially from a pig or sheep; \textbf{2.} [countable, usually singular] an act of cutting something in a quick movement downwards using an axe or a knife; \textbf{3.} [countable] an act of hitting somebody\texttt{/}something with the side of your hand in a quick movement downwards; \textbf{4.} \textbf{chops} [plural] (\textit{informal}) the part of a person's or an animal's face around the mouth; \textbf{5.} \textbf{chops} [plural] the technical skill of an actor or a jazz or rock musician.}, had been a Harvard professor, was a gentleman\footnote{\textbf{gentleman} [n] (plural \textbf{gentlemen}) \textbf{1.} (\textit{formal}) a polite or formal way of referring to a man; \textbf{2.} (in the past) a man from a high social class, especially one who did not need to work.} (as cowboys can be) though he did say \textit{damn} \& \textit{bloody} a lot, in a rural\footnote{\textbf{rural} [a] [usually before noun] connected with or like the countryside.} 1950s sort of way. But everyone listened, with fascination\footnote{\textbf{fascination} [n] \textbf{1.} [countable, usually singular] a very strong attraction, that makes something very interesting; \textbf{2.} [uncountable, singular] the state of being very attracted to \& interested in somebody\texttt{/}something.} on their faces, because he was in fact addressing questions of concern to everyone at the table.

There was something freeing about being with a person so learned\footnote{\textbf{learned} [a] [usually before noun] \textbf{1.} developed by training or experience; not existing at birth; \textbf{2.} having a lot of knowledge because you have studied \& read a lot; \textbf{3.} connected with or for leraned people; showing deep knowledge; \textsc{synonym}: \textbf{scholarly}.} yet speaking in such an unedited way. His thinking was motoric; it seemed he needed to think \textit{aloud}, to use his motor\footnote{\textbf{motor} [n] \textbf{1.} a device that uses electricity, petrol, etc. to produce movement \& makes a machine, a vehicle, a boat, etc. work; \textbf{2.} a source of power, energy or movement; \textbf{3.} (\textit{British English, old-fashioned humorous}) a car; [a] [only before noun] \textbf{1.} having an engine; using the power of an engine; \textbf{2.} (\textit{especially British English}) connected with vehicles that have engines; \textbf{3.} (\textit{specialist}) connected with movement of the body that is produced by muscles; connected with the nerves that control movement; [v] [intransitive] (\textit{British English, old-fashioned}) \textbf{$+$ adv.\texttt{/}prep.} to travel by car, especially for pleasure.} cortex\footnote{\textbf{cortex} [n] (plural \textbf{cortices}) (\textit{anatomy}) the outer layer of an organ in the body, especially the brain.} to think, but that motor also had to run fast to work properly. To get to liftoff\footnote{\textbf{liftoff} [n] [countable, uncountable] the act of a rocket or helicopter leaving the ground \& rising into the air.}. Not quite manic\footnote{\textbf{manic} [a] \textbf{1.} (\textit{informal}) full of activity, excitement \& stress; behaving in a busy, excited, anxious way, \textsc{synonym}: \textbf{hectic}; \textbf{2.} (\textit{psychology}) connected with mania.}, but his idling\footnote{\textbf{idle} [v] \textbf{1.} [transitive, intransitive] to spend time doing nothing important; \textbf{2.} [intransitive] (of an engine) to run slowly while the vehicle is not moving, \textsc{synonym}: \textbf{tick over}; \textbf{3.} [transitive] \textbf{idle somebody\texttt{/}something} (\textit{North American English}) to close a factory, etc. or stop providing work for the workers, especially temporarily.} speed revved\footnote{\textbf{revved} [v] [transitive, intransitive] \textbf{rev (something) (up)} when you rev an engine or it revs, it runs quickly; [n] (\textit{informal}) a complete turn of an engine, used when talking about an engine's speed, \textsc{synonym}: \textbf{revolution}.} high. Spirited thoughts were tumbling\footnote{\textbf{tumble} [v] \textbf{1.} [intransitive, transitive] \textbf{tumble (somebody\texttt{/}something) $+$ adv.\texttt{/}prep.} to fall downwards, often hitting the ground several times, but usually without serious injury; to make somebody\texttt{/}something fall in this way; \textbf{2.} [intransitive] \textbf{tumble (down)} to fall suddenly \& in a dramatic way; \textbf{3.} [intransitive] to fall rapidly in value or amount; \textbf{4.} [intransitive] \textbf{$+$ adv.\texttt{/}prep.} to move or fall somewhere in a relaxed or noisy way, or with a lack of control; \textbf{5.} [intransitive] to perform acrobatics on the floor, especially somersaults ($=$ a jump in which you turn over completely in the air); [n] \textbf{1.} [countable, usually singular] a sudden fall; \textbf{2.} [singular] \textbf{tumble (of something)} an untidy group of things.} out. But unlike many academics\footnote{\textbf{academic} [a] \textbf{1.} [usually before noun] connected with education, especially studying in schools \& universities, \textsc{synonym}: \textbf{educational}; \textbf{2.} [usually before noun] involving a lot of reading \& studying rather than practical or technical skills; \textbf{3.} not connected to a real or practical situation \& therefore not important; [n] a person who teaches \&\texttt{/}or does research at a university or college.} who take the floor \& hold it, if someone challenged or corrected him he really seemed to \textit{like} it. He didn't rear up\footnote{\textbf{rear} [a] [only before noun] at or near the back of something; [n] \textbf{1.} (usually \textbf{the rear}) [singular] the back part of something; \textbf{2.} (also \textbf{rear end}) [countable, usually singular] (\textit{informal}) the part of the body that you sit on, \textsc{synonym}: \textbf{backside, bottom}; [v] \textbf{1.} [transitive] \textbf{rear somebody\texttt{/}something} [often passive] to care for young children or animals until they are fully grown, \textsc{synonym}: \textbf{bring up, raise}; \textbf{2.} [transitive] \textbf{rear something} to keep \& breed ($=$ produce young from) animals or birds, e.g. on a farm; \textbf{3.} [intransitive] \textbf{rear (up)} (of an animal, especially a horse) to raise itself on its back legs, with the front legs in the air; \textbf{4.} [intransitive] \textbf{rear (up)} (of something large) to seem to lean over you, especially in a way that makes you feel frightened.} \& neigh\footnote{\textbf{neigh} [v] [intransitive] when a horse neighs it makes a long high sound; [n] a long high sound made by a horse.}. He'd say, in a kind of folksy\footnote{\textbf{folksy} [a] (also \textbf{folky}) \textbf{1.} (\textit{especially North American English}) simple, friendly \& informal; \textbf{2.} (\textit{sometimes disapproving}) done or made in a traditional style that is supposed to be typical of simple customs in the past, but sometimes in a false or artificial way.} way, ``Yeah,'' \& bow his head involuntarily\footnote{\textbf{involuntary} [a] \textbf{1.} happening without the person concerned wanting it to; \textbf{2.} an involuntary movement, etc. is made suddenly, without you intending it or being able to control it, \textsc{opposite}: \textbf{voluntary}.}, wag\footnote{\textbf{wag} [v] \textbf{1.} [transitive, intransitive] \textbf{wag (something)} if a dog wags its tail, or its tail wags, its tail moves from side to side several times; \textbf{2.} [transitive] \textbf{wag something} to shake your finger or your head from side to side or up \& down, often because you do not approve of something; \textbf{3.} [transitive] \textbf{wag something} (\textit{Australian English, New Zealand English}) to stay away from school without permission; [n] \textbf{1.} (\textit{especially British English, old-fashioned}) a person who enjoys making jokes, \textsc{synonym}: \textbf{joker}; \textbf{2.} a wagging movement.} it if he had overlooked\footnote{\textbf{overlook} [v] \textbf{1.} \textbf{overlook something} to fail to see or notice something, \textsc{synonym}: \textbf{miss}; \textbf{2.} \textbf{overlook something} if a building, etc. overlooks a place, you can see that place from the building; \textbf{3.} \textbf{overlook somebody (for something)} to not consider somebody for a job or position, even though they might be suitable.} something, laughing at himself for overgeneralizing\footnote{\textbf{overgeneralize} [v] [intransitive] to make a statement that is not accurate because it is too general.}. He appreciated being shown another side of an issue, \& it became clear that thinking through a problem was, for him, a dialogic process.

One could not but be struck by another unusual thing about him: for an \fbox{egghead}\footnote{\textbf{egghead} [n] (\textit{informal, disapproving or humorous}) a person who is very intelligent \& is only interested in studying} Peterson was extremely practical\footnote{\textbf{practical} [a] \textbf{1.} connected with real situations rather than with ideas or theories; \textbf{2.} (of an idea, a method or a course of action) right or sensible; possible \& likely to be successful, \textsc{synonym}: \textbf{feasible, workable}, \textsc{opposite}: \textbf{impractical}; \textbf{3.} (of things) useful or suitable for a particular purpose, \textsc{opposite}: \textbf{impractical}; \textbf{4.} (of a person) sensible \& realistic in the way they approach a problem or situation; \textbf{for (all) practical purposes} [idiom] used to say that something is so nearly true that it can be considered to be so; [n] (\textit{British English, informal}) a lesson or an exam in science or technology in which students have to do or make things, not just read or write about them.}. His examples were filled with applications to everyday life: business management, how to make furniture (he made much of his own), designing a simple house, making a room beautiful (now an internet meme) or in another, specific case related to education, creating an online writing project that kept minority students from dropping out of school by getting them to do a kind of psychoanalytic\footnote{\textbf{psychoanalysis} [n] (also \textbf{analysis}) [uncountable] a method of treating mental illness by investigating the influence of the unconscious mind, by getting somebody to talk about their fears, past experiences, dreams, etc.} exercise on themselves, in which they would free-associate\footnote{\textbf{free association} [n] [uncountable] \textbf{1.} the mental process by which 1 word or image may suggest another without any obvious connection; \textbf{2.} a method of treating a patient by asking them to use the mental process of free association.} about their past, present \& future (now known as the Self-Authoring Program).

I was always especially fond of mid-Western, Prairie\footnote{\textbf{prairie} [n] [countable, uncountable] a flat, wide area of land in North America \& Canada, without many trees \& originally covered with grass.} types who come from a farm (where they learned all about nature\footnote{\textbf{nature} [n] \textbf{1.} (often \textbf{Nature}) [uncountable] all the plants, animals \& things that exist in the universe that are not made by people. You cannot use `the nature' when you are referring to the natural world.; \textbf{2.} (often \textbf{Nature}) [uncountable] the way that things happen in the physical world when it is not controlled by people; \textbf{3.} [singular] the basic character or qualities of something; \textbf{4.} [singular] a type or kind of something; \textbf{5.} [uncountable, countable] the usual way that a person or an animal behaves that is part of their character.}), or from a very small town, \& who have worked with their hands to make things, spent long periods outside in the harsh elements, \& are often self-educated \& go to university against the odds\footnote{\textbf{odds} [n] [plural] \textbf{1.} (usually \textbf{the odds}) the degree to which something is likely to happen; \textbf{2.} greater advantage; the state of being greater in strength, power or resources.}. I found them quite unlike their sophisticated\footnote{\textbf{sophisticated} [a] \textbf{1.} (of things, systems, methods or ideas) clever \& complicated; \textbf{2.} [usually before noun] able to deal with complicated ideas; \textbf{3.} knowing a lot about the modern world \& about things that people consider to be socially important.} but somewhat denatured urban\footnote{\textbf{urban} [a] [usually before noun] connected with a town or city.} counterparts\footnote{\textbf{counterpart} [n] a person or thing that has the same position or function as somebody\texttt{/}something else in a different place or situation.}, for whom higher education was pre-ordained\footnote{\textbf{preordained} [a] (\textit{formal}) already decided or planned by God or by fate, \textsc{synonym}: \textbf{predestined}.}, \& for that reason sometimes taken for granted, or thought of not as an end in itself by simply as a life stage in the service of career advancement\footnote{\textbf{advancement} [n] \textbf{1.} [uncountable, countable] the process of helping something to make progress or succeed; the progress that is made; \textbf{2.} [uncountable] progress in a job or social class.}. These Westerners were different: self-made, unentitled\footnote{\textbf{entitled} [a] (\textit{usually disapproving}) feeling that you have a right to the good things in life without necessarily having to work for them.}, hands on, neighborly\footnote{\textbf{neighborly} [a] \textbf{1.} involving people, countries, etc. that live or are located near each other; \textbf{2.} friendly \& helpful, \textsc{synonym}: \textbf{kind}.} \& less precious\footnote{\textbf{precious} [a] \textbf{1.} rare \& worth a lot of money; \textbf{2.} valuable or important \& not to be wasted; \textbf{3.} loved or valued very much, \textsc{synonym}: \textbf{treasured}; \textbf{4.} [only before noun] (\textit{informal}) used to show you are angry that another person thinks something is very important; \textbf{5.} (\textit{disapproving}) (especially of people \& their behavior) very formal, exaggerated \& not natural in what you say \& do, \textsc{synonym}: \textbf{affected}.} than many of their big-city peers, who increasingly spend their lives indoors\footnote{\textbf{indoors} [v] inside or into a building, \textsc{opposite}: \textbf{outdoors}.}, manipulating symbols on computers. This cowboy psychologist seemed to care about a thought only if it might, in some way, be helpful to someone.

We became friends. As a psychiatrist\footnote{\textbf{psychiatrist} [n] a doctor who studies \& treats mental illnesses.} \& psychoanalyst\footnote{\textbf{psychoanalyst} [n] (also \textbf{analyst}) a person who treats patients using psychoanalysis.} who loves literature\footnote{\textbf{literature} [n] \textbf{1.} [uncountable] pieces of writing that are considered to be works of art, especially novels, plays \& poems (in contrast to technical books \& newspaper, magazines, etc.); \textbf{2.} [uncountable, countable] pieces of writing or printed information on a particular subject.}, I was drawn to him because here was a clinician\footnote{\textbf{clinician} [n] a doctor, psychologist, etc. who has direct contact with patients.} who also had given himself a great books education, \& who not only loved soulful\footnote{\textbf{soulful} [a] expressing deep feelings, especially feelings of love or being sad.} Russian novels, philosophy\footnote{\textbf{philosophy} [n] (\textbf{philosophies}) \textbf{1.} [uncountable] the study of the nature \& meaning of the universe \& of human life. \textbf{Natural philosophy} is an old term for the study of the physical world, which developed into the natural sciences. The term may still be used in the study of the history of science.; \textbf{2.} [countable] a particular set or system of beliefs resulting from the search for knowledge about life \& the universe; \textbf{3.} [countable] a set of beliefs or an attitude to life that guides somebody's behavior.} \& ancient\footnote{\textbf{ancient} [a] \textbf{1.} belonging to a period of history that is thousands of years in the past, \textsc{opposite}: \textbf{modern}; \textbf{2.} very old; having existed for a very long time; \textbf{3.} (\textbf{the ancients}) [n] [plural] the people who lived in the ancient times, especially the Egyptians, Greeks \& Romans.} mythology\footnote{\textbf{mythology} [n] (plural \textbf{mythologies}) [uncountable, countable] \textbf{1.} ancient myths in general; the ancient myths of a particular culture, society, etc.; \textbf{2.} \textbf{mythology (of something)} ideas that many people think are true but are in fact false.}, but who also seemed to treat them as his most treasured inheritance\footnote{\textbf{inheritance} [n] \textbf{1.} [uncountable, countable, usually singular] the process of receiving something such as a medical condition, physical characteristic or quality from parents, etc.; the condition, characteristic, etc. that is received; \textbf{2.} [countable, uncountable] the money or property that you receive from somebody when they die; the fact of receiving something when somebody dies; \textbf{3.} [countable, usually singular] a situation or tradition that you receive from a former owner or period of time.}. But he also did illuminating\footnote{\textbf{illuminate} [v] \textbf{1.} \textbf{illuminate something} to make something clearer or easier to understand, \textsc{synonym}: \textbf{clarify}; \textbf{2.} \textbf{illuminate something} to shine light on something.}\,\footnote{\textbf{illuminating} [a] helping to make something clear or easier to understand.} statistical\footnote{\textbf{statistical} [a] connected with statistics.} research on personality\footnote{\textbf{personality} [n] (plural \textbf{personalities}) \textbf{1.} [countable, uncountable] the various aspects of a person's character that combine to make them different from other people; \textbf{2.} [uncountable] the qualities of a person's character that make them interesting \& attractive; \textbf{3.} [countable] a famous person, especially one who works in entertainment or sport, \textsc{synonym}: \textbf{celebrity}; \textbf{4.} [countable] a person whose strong character makes them easy to notice; \textbf{5.} [uncountable] the qualities of a place or thing that make it interesting \& different, \textsc{synonym}: \textbf{character}.} \& temperament\footnote{\textbf{temperament} [n] \textbf{1.} [countable, uncountable] a person's or an animal's nature as shown in the way they behave or react to situations or people; \textbf{2.} [uncountable] the fact of tending to get emotional \& excited very easily \& behave in an unreasonable way.}, \& had studied neuroscience\footnote{\textbf{neuroscience} [n] [uncountable] the science that deals with the structure \& function of the brain \& the nervous system.}. Though trained as a behaviorist\footnote{\textbf{behaviorist} [n] (\textit{US English} \textbf{behaviorist}) (\textit{psychology}) a scientist who studies or accepts the theory of behaviorism.}, he was powerfully\footnote{\textbf{powerfully} [adv] in a way that has a strong effect or people's feelings or thoughts.} drawn to psychoanalysis with its focus on dreams\footnote{\textbf{dream} [n] \textbf{1.} a series of images, events \& feelings that happen in your mind while you are sleeping; \textbf{2.} \textbf{dream (of something\texttt{/}doing something)} a wish to have, do or be something, especially one that seems difficult to achieve; [v] \textbf{1.} [intransitive, transitive] to imagine \& think about something that you would like to happen; \textbf{2.} [intransitive, transitive] to experience a series of images, events \& feelings in your mind while you are sleeping.}, archetypes\footnote{\textbf{archetypes} [n] (\textit{formal}) the most typical or perfect example of a particular kind of person or thing.}, the persistence\footnote{\textbf{persistence} [n] [uncountable] \textbf{1.} \textbf{persistence (of something)} the state of continuing to exist for a long period of time; \textbf{2.} the fact of continuing to do something despite difficulties or opposition.} of childhood conflicts\footnote{\textbf{conflict} [n] [countable, uncountable] \textbf{1.} a situation in which people, groups or countries are involved in a serious disagreement or argument; \textbf{2.} a violent situation or period of fighting between countries or groups of people; \textbf{3.} a situation in which there are opposing ideas, opinions, feelings or wishes; \textbf{conflict of interest(s)} [idiom] \textbf{1.} a situation in which somebody has a role or responsibility that may prevent them from treating another role or responsibility equally \& fairly; \textbf{2.} a situation in which somebody's aims or needs are in opposition to the aims or needs of another person or group.} in the adult, \& the role of defences\footnote{\textbf{defence} [n] (US \textbf{defense}) \textbf{1.} [countable, uncountable] support for somebody\texttt{/}something that has been criticized, \textsc{opposite}: \textbf{attack}; \textbf{2.} [uncountable, countable] the action of protecting somebody\texttt{/}something from attack, \textsc{opposite}: \textbf{attack}; \textbf{3.} [countable, uncountable] something that provides protection against attack from enemies, the weather, illness, etc.; \textbf{4.} [uncountable] military measures or resources for protecting a country from attack; \textbf{5.} [countable] a set of facts or arguments presented in court to support a person who has been accuse of committing a crime, or who is being sued; \textbf{6.} (\textbf{the defence}) [singular $+$ singular or plural verb] the lawyer or lawyers whose job is to represent in court a person who has been accused of committing a crime, or who is being sued.} \& rationalization\footnote{\textbf{rationalize} [v] (\textit{British English also} \textbf{rationalise}) \textbf{1.} [transitive, intransitive] \textbf{rationalize (something)} to find or try to find a logical reason to explain why somebody thinks or behaves in a particular way; \textbf{2.} [transitive] \textbf{rationalize something} to make changes to a business, system, etc. in order to make it more efficient, especially by spending less money.

\textbf{rationalization} [n] (\textit{British English also} \textbf{rationalisation}) [uncountable, countable].} in everyday life. He was also an outlier\footnote{\textbf{outlier} [n] \textbf{1.} a person or thing that is different from or in a position away from others in the group; \textbf{2.} (\textit{statistics}) a data point on a graph or in a set of results that is very much bigger or smaller than the next nearest data point.} in being the only member of the research-oriented Department of Psychology at the University of Toronto who also kept a clinical\footnote{\textbf{clinical} [a] [only before noun] connected with the examination \& treatment of patients \& their illnesses.} practice.

On my visits, our conversations began with banter\footnote{\textbf{banter} [n] [uncountable] friendly remarks \& jokes; [v] [intransitive] \textbf{banter (with somebody)} to joke with somebody.} \& laughter -- that was the small-town Peterson from the Alberta hinterland\footnote{\textbf{hinterland} [n] the areas of a country that are away from the coast, from the banks of a large river or from the main cities.} -- his teenage years right out of the movie FUBAR -- welcoming you into his home. The house had been gutted\footnote{\textbf{gutted} [a] [not before noun] (\textit{British English, informal}) extremely sad or disappointed.} by Tammy, his wife, \& himself, \& turned into perhaps the most fascinating \& shocking middle-class\footnote{\textbf{middle-class} [a] connected with the middle social class.} home I had seen. They had art, some carved\footnote{\textbf{carve} [v] \textbf{1.} [transitive, intransitive] to make objects, patterns, etc. by cutting away material from a piece of wood or stone, or another hard material; \textbf{2.} [transitive] \textbf{carve something (on something)} to write something on a surface by cutting into it; \textbf{3.} [transitive, intransitive] to cut a large piece of cooked meat into smaller pieces for eating; \textbf{4.} [transitive, no passive] to work hard in order to have a successful career, reputation, etc.} masks\footnote{\textbf{mask} [n] \textbf{1.} a covering for part or all of the face, worn to protect it or hide it; \textbf{2.} an object that fits over somebody's face \& that is connected to a container of oxygen, used for helping them to breathe; \textbf{3.} [usually singular] a manner or an expression that hides somebody's true character or feelings; [v] \textbf{mask something} to hide a fact or feeling so that it cannot be easily seen or noticed.}, \& abstract\footnote{\textbf{abstract} [a] \textbf{1.} existing in thought or as an idea but not as a physical thing; \textbf{2.} based on general ideas \& not on any particular real person, thing or situation; \textbf{3.} representing an idea, a quality or a state rather than a physical object; \textbf{4.} (of art) not representing people or things in a realistic way, but expressing the artist's ideas about them using shapes, colors \& textures; [n] a short piece of writing containing the main ideas of a research article, book or speech, \textsc{synonym}: \textbf{summary}; \textbf{in the abstract} [idioms] in a general way, without referring to a particular real person, thing or situation; [v] \textbf{1.} [transitive] \textbf{abstract something (from something)} to remove something from somewhere, \textsc{synonym}: \textbf{extract}; \textbf{2.} [transitive, intransitive] \textbf{abstract (something) (from something)} to think about something generally or separately from something else.} portraits\footnote{\textbf{portrait} [n] \textbf{1.} \textbf{portrait (of somebody\texttt{/}something)} a painting, drawing or photograph of a person, especially of the head \& shoulders; \textbf{2.} \textbf{portrait (of somebody\texttt{/}something)} a detailed description of somebody\texttt{/}something, \textsc{synonym}: \textbf{depiction}.}, but they were overwhelmed\footnote{\textbf{overwhelm} [v] [often passive] \textbf{1.} to be so bad or so great that a person, organization or system cannot deal with it; to give too much of a thing to a person or thing; \textbf{2.} to have such a strong emotional effect on somebody that it is difficult for them to resist or know how to react, \textsc{synonym}: \textbf{overcome}; \textbf{3.} \textbf{overwhelm somebody} to defeat somebody completely.} by a huge collection of original Socialist\footnote{\textbf{socialist} [a] [usually before noun] supporting socialism; [n].} Realist\footnote{\textbf{realist} [n] \textbf{1.} (\textit{politics}) a person who believes that the subject matter of politics is political power, not matters of principle; \textbf{2.} a writer, artist, etc. whose work represents things as they are in real life; \textbf{3.} (\textit{philosophy}) a person who believes that reality exists independently of how people view it; [a].} paintings of Lenin \& the early Communists\footnote{\textbf{communist} [n] \textbf{1.} a person who believes in or supports communism; \textbf{2.} (\textbf{Communist}) a member of a communist party; [a] (\textbf{Communist}) connected with communism.} commissioned\footnote{\textbf{commission} [n] \textsf{official group} \textbf{1.} (often \textbf{Commission}) [countable] an official group of people who have been given responsibility to control something, or to find out about something, usually for the government; \textsf{money} \textbf{2.} [uncountable, countable] an amount of money that is paid to somebody for selling goods \& that increases with the amount of goods that are sold; \textbf{3.} [uncountable, singular] an amount of money that is charged by a bank, etc. for providing a particular service; \textsf{for art\texttt{/}music, etc.} \textbf{4.} [countable] a formal request to somebody to design or make a piece of work such as a building or a painting; the fact of making such a request; \textsf{in armed forces} \textbf{5.} [countable] the position of an officer in the armed forces, typically with the rank of lieutenant or higher; \textsf{of crime} \textbf{6.} [uncountable] (\textit{formal}) the act of doing something wrong or illegal; [v] \textsf{piece of art\texttt{/}music, etc.} \textbf{1.} to officially ask somebody to write, make or create something or to do a task for you; \textsf{in armed forces} \textbf{2.} [usually passive] to choose somebody as an officer in 1 of the armed forces.} by the USSR\footnote{\textbf{USSR} [abbr] (the former) Union of Soviet Socialist Republics.}. Not long after the Soviet Union fell, \& most of the world breathed a sigh\footnote{\textbf{sigh} [v] \textbf{1.} [intransitive] to take \& then let out a long deep breath that can be heard, to show that you are disappointed, sad, tired, etc.; \textbf{2.} [transitive] \textbf{$+$ speech} to say something with a sigh; \textbf{3.} [intransitive] (\textit{literary}) (especially of the wind) to make a long sound like a sigh; [n] an act or the sound of sighing.} of relief\footnote{\textbf{relief} [n] \textbf{1.} [uncountable, singular] the feeling of happiness that you have when something unpleasant stops or does not happen; \textbf{2.} [uncountable] the act of removing or reducing pain, anxiety, etc.; \textbf{3.} [uncountable] food, money, medicine, etc. that is given to help people in places where there has been a war or natural disaster, \textsc{synonym}: \textbf{aid}; \textbf{4.} [uncountable] financial help given by the government to people who need it; \textbf{5.} [uncountable, singular] something that is interesting or enjoyable that replaces something boring, difficult or unpleasant for a short period of time; \textbf{6.} [uncountable] the quality of a particular situation, problem, etc. that makes it easier to notice than before; \textbf{7.} [uncountable] (\textit{geography}) difference in height from the surrounding land; \textbf{8.} [uncountable, countable] a way of decorating wood, stone, etc. by cutting designs into the surface of it so that some parts stick out more than others; a design that is made in this way; \textbf{9.} [countable $+$ singular or plural verb] a person or group of people that replaces others who have been on duty; \textbf{10.} [singular] \textbf{relief of $\ldots$} the act of freeing a town, etc. from an enemy army that has surrounded it.}, Peterson began purchasing this propaganda\footnote{\textbf{propaganda} [n] [uncountable] (\textit{usually disapproving}) ideas or statements that may be false or give a false impression \& that are used in order to gain support for a political leader, party, etc.} for a song online. Paintings lionizing\footnote{\textbf{lionize} [v] (\textit{British English also} \textbf{lionise}) (\textit{formal}) \textbf{lionize somebody} to treat somebody as a famous or important person.} the Soviet revolutionary\footnote{\textbf{revolutionary} [a] \textbf{1.} [usually before noun] connected with political revolution; \textbf{2.} involving a great or complete change; [n] (plural \textbf{revolutionaries}) a person who starts or supports a revolution, especially a political one.} spirit\footnote{\textbf{spirit} [n] \textbf{1.} [uncountable, countable] the part of a person that includes their mind, feelings \& character rather than their body; \textbf{2.} [singular, uncountable] an attidue or way of thinking; \textbf{3.} [uncountable, singular] loyal feelings towards a group, team or society; \textbf{4.} [singular] \textbf{spirit (of something)} the typical or most important quality or mood of something; \textbf{5.} [uncountable] \textbf{spirit (of something)} the real or intended meaning or purpose of something; \textbf{6.} [uncountable] courage, determination or energy; \textbf{7.} [countable] \textbf{spirit (of somebody)} the part of a person that many people believe still exists after their body is dead; \textbf{8.} [countable] an imaginary creature with magic powers; \textbf{9.} [countable, usually plural] (\textit{especially British English}) a strong alcoholic drink.} completely filled every single wall, the ceilings, even the bathrooms. The paintings were not there because Jordan had any totalitarian\footnote{\textbf{totalitarian} [a] (\textit{disapproving}) (of a country or system of government) in which there is only 1 political party, which has complete power \& control over the people.} sympathies\footnote{\textbf{sympathy} [n] (plural \textbf{sympathies}) \textbf{1.} [uncountable, countable, usually plural] \textbf{sympathy (for somebody)} the feeling of being sorry for somebody; showing that you understand \& care about somebody's problems; \textbf{2.} [countable, usually plural, uncountable] the act of showing support for or approval of an idea, a cause, an organization, etc..}, but because he wanted to remind himself of something he knew he \& everyone would rather forget: that hundreds of millions were murdered\footnote{\textbf{murder} [v] \textbf{murder somebody} to kill somebody deliberately \& illegally.} in the name of \fbox{utopia}\footnote{\textbf{utopia} [n] (also \textbf{Utopia}) [countable, uncountable] an imaginary place or state in which everything is perfect.}.

It took getting used to, this semi-haunted house ``decorated'' by a delusion\footnote{\textbf{delusion} [n] \textbf{1.} [countable] a false belief or opinion about yourself or your situation, especially as a sign of mental illness; \textbf{2.} [uncountable] the act of believing or making yourself believe something that is not true.} that had practically\footnote{\textbf{practically} [adv] \textbf{1.} almost; very nearly, \textsc{synonym}: \textbf{virtually}' \textbf{2.} in a realistic or sensible way; in real situations.} destroyed\footnote{\textbf{destroy} [v] \textbf{destroy something} to damage something so badly that is no longer exists or can no longer be used.} mankind\footnote{\textbf{mankind} [n] [uncountable] all humans, considered as 1 large group; the human race.}. But it was eased\footnote{\textbf{ease} [v] \textbf{1.} [intransitive, transitive] to become less unpleasant, painful, severe, etc.; to make something less unpleasant, etc.; \textbf{2.} [transitive] \textbf{ease something} to make something easier, \textsc{synonym}: \textbf{facilitate}; \textbf{3.} [transitive] \textbf{ease somebody\texttt{/}something $+$ adv.\texttt{/}prep.} to slowly \& carefully make somebody\texttt{/}something reach a particular state or condition; \textbf{4.} [intransitive, transitive] to become lower in price or value; to make something lower in price or value.} by his wonderful \& unique\footnote{\textbf{unique} [a] \textbf{1.} being the only one of their\texttt{/}its kind; different from everyone or everything else. In general English, \textbf{unique} is sometimes used after a word such as `very' or `rather', to suggest that something is very or rather unusual or special: \textit{This is a very unique case}. This use is best avoided in academic writing; \textbf{2.} \textbf{unique to somebody\texttt{/}something} involving 1 particular person, place or thing.} spouse\footnote{\textbf{spouse} [n] (\textit{formal} or \textit{law}) a husband or wife.}, Tammy, who was all in, who embraced \& encouraged\footnote{\textbf{encourage} [v] \textbf{1.} to make something more likely to happen or develop, \textsc{opposite}: \textbf{discourage}; \textbf{2.} to persuade somebody to do something by making it easier for them \& making them believe it is a good thing to do, \textsc{opposite}: \textbf{discourage}; \textbf{3.} \textbf{encourage somebody} to give somebody support or hope, \textsc{opposite}: \textbf{discourage}.} this unusual need for expression! These paintings provided a visitor with the 1st window onto the full extent of Jordan's concern about our \fbox{human capacity for evil in the name of good}\footnote{\textbf{capacity} [n] (plural \textbf{capacities}) \textbf{1.} [countable, uncountable] the ability to understand or to do something; \textbf{2.} [uncountable, countable, usually singular] the number of things or people that a container or space can hold; \textbf{3.} [singular, uncountable] the quantity that a factory, machine, etc. can produce; \textbf{4.} [countable, usually singular] the official position or function that somebody has, \textsc{synonym}: \textbf{role}; \textbf{5.}}, \& the psychological\footnote{\textbf{psychological} [a] \textbf{1.} [usually before noun] connected with a person's mind \& the way it works; \textbf{2.} [only before noun] connected with the study of psychology.} mystery\footnote{\textbf{mystery} [n] (plural \textbf{mysteries}) \textbf{1.} [countable] something that is difficult or impossible to understand or to explain; \textbf{2.} [uncountable] the quality of being difficult or impossible to understand or to explain, especially when this makes somebody\texttt{/}something seem interesting \& exciting; \textbf{3.} [countable] a story, film or play in which crimes \& strange events are only explained at the end; \textbf{4.} (\textbf{mysteries}) [plural] \textbf{mystery (of something)} the skills or knowledge needed for a particular activity \& regarded as too difficult to understand for whose without such skills or knowledge; \textbf{5.} [countable] \textbf{mystery (of somebody\texttt{/}something)} a religious belief that cannot be explained or proved in a scientific way.} of self-deception\footnote{\textbf{self-deception} [n] [uncountable] the act of making yourself believe something that you know is not true.} (how can a person deceive\footnote{\textbf{deceive} [v] [transitive] \textbf{1.} \textbf{deceive somebody} to deliberately make somebody believe something that is not true; \textbf{2.} \textbf{deceive somebody\texttt{/}something} (of a thing) to make somebody have a false idea about somebody\texttt{/}something.} himself \& get away with\footnote{\textbf{get away with (doing) something} [idiom] to do something wrong \& not be punished for it.} it?) -- an interest we share. \& then there were also the hours we'd spend discussing what I might call a lesser problem (lesser because rarer), the\\\fbox{human capacity for evil for the sake of evil}, the joy some people take in destroying others, captured famously by the 17th-century English poet John Milton in \textit{Paradise Lost}.

\& so we'd chat \& have our tea in his kitchen-underworld, walled by this odd\footnote{\textbf{odd} [a] \textbf{1.} (no comparative or superlative) (of numbers) that cannot be divided exactly by the number 2, \textsc{opposite}: \textbf{even}; \textbf{2.} strange or unusual; \textbf{3.} (\textbf{the odd}) [only before noun] (no comparative or superlative) happening or appearing occasionally; not every regular or frequent, \textsc{synonym}: \textbf{occasional}; \textbf{4.} [only before noun] (no comparative or superlative) of no particular type of size; various; \textbf{5.} [only before noun] available; that somebody can use, \textsc{synonym}: \textbf{spare}; \textbf{6.} (no comparative or superlative; usually placed immediately after a number) (\textit{informal}) approximately or a little more than the number mentioned.} art collection, a visual\footnote{\textbf{visual} [a] of or connected with seeing or sight.} marker\footnote{\textbf{marker} [n] \textbf{1.} a feature or sign that shows that something exists or what it is like; \textbf{2.} (\textit{biochemistry}) 1 of 2 or more forms of a gene that can be used to identify a chromosome or the location of other genes.} of his earnest\footnote{\textbf{earnest} [a] very serious \& sincere; \textbf{in earnest} [idiom] more seriously \& with more force or effort than before; very serious \& sincere about what you are saying \& about your intentions; in a way that shows that you are serious.} quest\footnote{\textbf{quest} [n] a long or difficult search for something, especially for a quality such as knowledge or truth.} to move beyond\footnote{\textbf{beyond} [prep] \textbf{1.} on or to the further side of something; \textbf{2.} more developed than something; reaching further than something; \textbf{3.} used to say that something is not impossible; \textbf{4.} more than a particular amount; \textbf{5.} later than a particular time; \textbf{6.} too far or too advanced for somebody\texttt{/}something; [adv] \textbf{1.} \textbf{(\&) beyond} on the other side; further on; \textbf{2.} \textbf{(\&) beyond} afterwards or later.} simplistic\footnote{\textbf{simplistic} [a] (\textit{disapproving}) treating complicated issues \& problems as if they were much simpler than they really are.} ideology\footnote{\textbf{ideology} [n] (plural \textbf{ideologies}) [countable, uncountable] (\textit{sometimes disapproving}) a set of ideas \& beliefs that an economic or political system is based on, or that influences the way a person or group behaves. The term \textbf{ideology} is sometimes used in a disapproving way to suggest a set of beliefs that are too fixed or not realistic or fair.}, left or right, \& not repeat mistakes of the past. After a while, there was nothing peculiar\footnote{\textbf{peculiar} [a] belonging to or connected with 1 particular place, situation, person, etc., \& not others.} about taking tea in the kitchen, discussing family issues, one's latest reading, with those ominous\footnote{\textbf{ominous} [a] suggesting that something bad is going to happen in the future, \textsc{synonym}: \textbf{foreboding}.} pictures hovering. It was just living in the world as it was, or in some places, is.

In Jordan's 1st \& only book before this one, \textit{Maps of Meaning}, he shares his profound\footnote{\textbf{profound} [a] \textbf{1.} very great; felt or experienced very strongly; \textbf{2.} showing great knowledge or understanding; \textbf{3.} needing a lot of study or thought; \textbf{4.} (\textit{medical}) very serious; complete.} insights\footnote{\textbf{insight} [n] \textbf{1.} [countable, uncountable] an understanding of a particular situation or thing; \textbf{2.} [uncountable] the ability to see \& understand the truth about the people or situations.} into universal\footnote{\textbf{universal} [a] \textbf{1.} done by or involving all the people in the world or in a particular grooup; \textbf{2.} true or right at all times \& in all places.} themes\footnote{\textbf{theme} [n] the subject of a talk, piece of writing, exhibition, etc.; an idea that keeps returning in a piece of research or a work of art or literature.} of world mythology, \& explains how all cultures have created stories to help us grapple\footnote{\textbf{grapple} [v] \textbf{1.} [intransitive, transitive] to take a strong hold of somebody\texttt{/}something \& struggle with them; \textbf{2.} [intransitive] to try hard to find a solution to a problem.} with, \& ultimately map, the chaos into which we are thrown at birth; this chaos is everything that is unknown to us, \& any unexplored\footnote{\textbf{unexplored} [a] \textbf{1.} (of a country or an area of land) that no one has investigated or put on a map; that has not been explored; \textbf{2.} (of an area of activity or thought) that has not yet been examined or discussed thoroughly.} territory\footnote{\textbf{territory} [n] (plural \textbf{territories}) \textbf{1.} [uncountable, countable] land that is under the control of a particular country or ruler; \textbf{2.} [countable, uncountable] an area that an animal or group of animals considers as its own \& defends against others who try to enter it; \textbf{3.} [uncountable, countable] an area of knowledge, activity or experience; \textbf{4.} [countable] an area of a town, country, etc. that somebody has particular rights in or responsibility for in their work or another activity; \textbf{5.} [uncountable] a particular type of land; \textbf{6.} (\textbf{Territory}) [countable] a country or an area that is part of the US, Australia or Canada but is not a state or province.} that we must traverse\footnote{\textbf{traverse} [v] (\textit{formal}) \textbf{traverse something} to cross an area of land or water; [n] (\textit{specialist}) (in mountain climbing) an act of moving sideways across a steep slope, not climbing up or down it; a place where this is possible or necessary.}, be it in the world outside or th psyche\footnote{\textbf{psyche} [n] the mind; your deepest feelings \& attitudes.} within.

Combining evolution\footnote{\textbf{evolution} [n] [uncountable] \textbf{1.} (\textit{biology}) the gradual development of living things over many years as they adapt to changes in their environment; \textbf{2.} the gradual development of something.}, the neuroscience of emotion\footnote{\textbf{emotion} [n] \textbf{1.} [countable, uncountable] a strong feeling such as love, fear or anger; these feelings considered together; \textbf{2.} [uncountable] the part of a person's nature that consists of feelings rather than thought or knowledge.}, some of the best of Jung, some of Freud, much of the great works of \textsc{Nietzsche, Dostoevsky, Solzhenitsyn, Eliade, Neumann, Piaget, Frye \& Frankl}, \textit{Maps of Meaning}, published nearly 2 decades ago, shows Jordan's wide-ranging approach to understanding how human beings \& the human brain deal with the archetypal\footnote{\textbf{archetypal} [a] having all the important qualities that make somebody\texttt{/}something a typical example of a particular kind of person or thing.} situation that arises whenever we, in our daily lives, must face something we do not understand. The brilliance\footnote{\textbf{brilliance} [n] [uncountable] \textbf{1.} the quality of being extremely impressive, intelligent or skillful; \textbf{2.} (\textit{formal}) (of light or colors) the quality of being very bright.} of the book is in his demonstration\footnote{\textbf{demonstration} [n] \textbf{1.} [countable, uncountable] \textbf{demonstration (of something)} an act of giving proof or evidence for something; \textbf{2.} [countable] a public meeting or march at which people show that they are protesting against or supporting somebody\texttt{/}something; \textbf{3.} [countable] an act of showing or explaining how something works or is done.} of how rooted\footnote{\textbf{rooted} [a] \textbf{1.} \textbf{rooted in something} developing from or being strongly influenced by something; \textbf{2.} \textbf{rooted in something} fixed in 1 place; not moving or changing.} this situation is in evolution, our DNA, our brains \& our most ancient stories. \& he shows that these stories have survived\footnote{\textbf{survive} [v] \textbf{1.} [intransitive] to continue to live or exist; \textbf{2.} [transitive] to continue to live or exist despite a dangerous event or time; \textbf{3.} [transitive] \textbf{survive somebody\texttt{/}something} to live or exist longer than somebody\texttt{/}something.} because they still provide guidance\footnote{\textbf{guidance} [n] [uncountable] \textbf{1.} help or advice that is given to somebody, especially by somebody in authority; \textbf{2.} the process of controlling the direction or position of something using special equipment.} in dealing with uncertainty\footnote{\textbf{uncertainty} [n] (plural \textbf{uncertainties}) \textbf{1.} [uncountable] the state of not knowing or of not being known exactly; the state of being uncertain; \textbf{2.} [countable, usually plural] something that you cannot be sure about; a situation that causes you to be uncertain.}, \& the unavoidable\footnote{\textbf{unavoidable} [a] impossible to avoid or prevent, \textsc{synonym}: \textbf{inevitable}, \textsc{opposite}: \textbf{avoidable}.} unknown\footnote{\textbf{unknown} [a] \textbf{1.} not known or identified; \textbf{2.} (of people) not famous or well known; \textbf{3.} never happening or existing; [n] \textbf{1.} (\textbf{the unknown}) [singular] places or things that are not known about; \textbf{2.} [countable] a person who is not well known; \textbf{3.} [countable] a fact or an influence that is not known; \textbf{4.} [countable] (\textit{mathematics}) a quantity that does not have a known value.}.

1 of the many virtues\footnote{\textbf{virtue} [n] \textbf{1.} [countable, uncountable] \textbf{virtue (of something)} an attractive or useful quality, \textsc{synonym}: \textbf{advantage}; \textbf{2.} [countable] a particular good quality or habit; \textbf{3.} [uncountable] behavior or attitudes that show high moral standards; \textbf{by\texttt{/}in virtue of (doing) something} (\textit{formal}) because or as a result of something.} of the book you are reading now is that it provides an entry point into \textit{Maps of Meaning}, which is a \fbox{highly complex work} because Jordan was working out his approach to psychology as he wrote it. But it was foundational\footnote{\textbf{foundation} [n] \textbf{1.} [countable, uncountable] a principle, an idea or a fact that something is based on \& that it grows from; \textbf{2.} [countable] an organization that is established to provide money for a particular purpose, e.g. for scientific research or charity; \textbf{3.} [uncountable] the act of starting a new institution or organization, \textsc{synonym}: \textbf{establishment}; \textbf{4.} [countable, usually plural] a layer of stone, concrete, etc. that forms the solid underground base of a building; \textbf{5.} [uncountable] a skin-colored cream that is put on the face under other make-up.}, because no matter how different our genes or life experiences may be, or how differently our plastic brains are wired\footnote{\textbf{wired} [a] \textbf{1.} connected to a device or computer network by wires; \textbf{2.} (of a glass, material, etc.) containing wires that make it strong or stiff; \textbf{3.} (\textit{informal}) excited or nervous; not relaxed; \textbf{4.} (\textit{informal, especially North American English}) under the influence of alcohol or an illegal drug.} by our experience, we all have to deal with the unknown, \& we all attempt to move from chaos to order. \& this is why many of the rules in this book, being based on \textit{Maps of Meaning}, have an element of universality\footnote{\textbf{universality} [n] \textbf{1.} the fact of being done by or involving all the people in the world or in a particular group; \textbf{2.} the fact of being true or right at all times \& in all places.} to them.

\textit{Maps of Meaning} was sparked\footnote{\textbf{spark} [v] to cause something to start or develop, especially suddenly; [n] \textbf{1.} a small flash of light produced by an electric current; \textbf{2.} a very small burning piece of material that is produced by something that is burning or by hitting 2 hard substances together; \textbf{3.} an action or event that causes something important to develop, especially trouble or violence; \textbf{4.} [usually singular] \textbf{spark of something} a small amount of a particular quality or feeling.} by Jordan's agonized\footnote{\textbf{agonized} [a] (\textit{British English also} \textbf{agonised}) suffering or expressing severe pain or worry.} awareness\footnote{\textbf{awareness} [n] [uncountable, singular] \textbf{1.} the fact of knowing that something is true or exists; \textbf{2.} concern or interest in a particular situation or development.}, as a teenager growing up in the midst\footnote{\textbf{midst} [n] [singular] used after a preposition (\textit{formal}) the middle part of something, \textsc{synonym}: \textbf{middle}.} of the Cold War, that much of mankind seemed on the verge\footnote{\textbf{on the verge of (doing) something} [idiom] very near to the moment when something happens or when somebody does something.} of blowing up the planet to defend their various\footnote{\textbf{various} [a] several different.} identities\footnote{\textbf{identity} [n] (plural \textbf{identities}) \textbf{1.} [countable, uncountable] the characteristics that make a person or thing who or what they are \& make them different from others; \textbf{2.} [countable, uncountable] (abbr. \textbf{ID}) \textbf{identity (of somebody\texttt{/}something)} the fact of being who or what a person or thing is; \textbf{3.} [uncountable] the state of being the same as somebody\texttt{/}something; the feeling of having a close association or connection with somebody\texttt{/}something; \textbf{4.} [countable] (\textit{mathematics}) an equation that is true for all possible values of the letters in the equation, e.g., $(x + 1)^2 = x^2 + 2x + 1$.}. He felt he had to understand how it could be that people would sacrifice\footnote{\textbf{sacrifice} [n] \textbf{1.} [countable, uncountable] the fact of giving up something important or valuable to you in order to get or do something that seems more important; something that you give up in this way; \textbf{2.} [countable, uncountable] the act of offering something to a god, especially an animal that has been killed in a special way; an animal, etc. that is offered in this way; [v] \textbf{1.} [transitive] to give up something that is important or valuable to you in order to get or do something that seems more important for yourself or for another person; \textbf{2.} [transitive, intransitive] to kill an animal or a person \& offer\texttt{/}them to a god, in order to please the god.} everything for an ``identity,'' whatever that was. \& he felt he had to understand the ideologies\footnote{\textbf{ideology} [n] (plural \textbf{ideologies}) [countable, uncountable] (\textit{sometimes disapproving}) a set of ideas \& beliefs that an economic or political system is based on, or that influences the way a person or group behaves. The term \textbf{ideology} is sometimes used in a disapproving way to suggest a set of beliefs that are too fixed or not realistic or fair.} that drove totalitarian regimes to a variant\footnote{\textbf{variant} [n] a thing that is a slightly different form or type of something else, \textsc{synonym}: \textbf{variation}.} of that same behavior: killing their own citizens\footnote{\textbf{citizen} [n] \textbf{1.} a person who has the legal right to belong to a particular country; \textbf{2.} a person who lives in a particular place.}. In \textit{Maps of Meaning}, \& again in this book, 1 of the matters he cautions\footnote{\textbf{caution} [n] [uncountable] \textbf{1.} care that you take in order to avoid mistakes or danger; \textbf{2.} a warning or a piece of advice about a possible danger or risk.} readers to be most wary\footnote{\textbf{wary} [a] (\textbf{warier}, no superlative) careful when dealing with somebody\texttt{/}something because you think that there may be a danger or problem, \textsc{synonym}: \textbf{cautious}.} of is ideology, no matter who is peddling\footnote{\textbf{peddle} [v] \textbf{1.} \textbf{peddle something} to try to sell goods by going from house to house or from place to place; \textbf{2.} \textbf{peddle something} (\textit{usually disapproving}) to spread an idea or story in order to get people to accept it.} it or to what end.

Ideologies are simple ideas, disguised\footnote{\textbf{disguise} [v] \textbf{1.} to hide the true nature of something so that it cannot be recognized, \textsc{synonym}: \textbf{conceal}; \textbf{2.} \textbf{disguise somebody\texttt{/}yourself (as somebody\texttt{/}something)} to change your appearance so that people cannot recognize you.} as science or philosophy, that purport\footnote{\textbf{purport} [v] \textbf{purport to be\texttt{/}have\texttt{/}do something} (\textit{formal}) to claim to be, have or do something, when this may not be true.} to explain\footnote{\textbf{explain} [v] \textbf{1.} [transitive, intransitive] to tell somebody about something in a way that makes it easy to understand; \textbf{2.} [intransitive, transitive] to give a reason for something; to be a reason for something.} the complexity\footnote{\textbf{complexity} [n] \textbf{1.} [uncountable] the state of being formed of many parts; the state of being difficult to understand; \textbf{2.} (\textbf{complexities}) [plural] \textbf{complexity of something} the features of a problem or situation that are difficult to understand.} of the world \& offer remedies\footnote{\textbf{remedy} [n] (plural \textbf{remedies}) \textbf{1.} a way of dealing with or improving an unpleasant or difficult situation, \textsc{synonym}: \textbf{solution}; \textbf{2.} a treatment or medicine to cure a disease or to reduce pain that is not very serious; \textbf{3.} (\textit{law}) a way of dealing with a problem, using the processes of the law, \textsc{synonym}: \textbf{redress}; [v] \textbf{remedy something} to correct or improve something.} that will perfect\footnote{\textbf{perfect} [a] \textbf{1.} having everything that is necessary; complete \& without faults or weaknesses; \textbf{2.} completely exact or accurate; \textsc{synonym}: \textbf{exact}; \textbf{3.} as good as it is possible to be; \textbf{4.} very good of its kind; \textbf{5.} \textbf{perfect for somebody\texttt{/}something} exactly right for somebody\texttt{/}something, \textsc{synonym}: \textbf{ideal}; \textbf{6.} (\textit{grammar}) connected with the form of a verb that in English consists of part of the verb \textit{have} with the past participle of the main verb, used to express actions completed by the present or a particular point in the past or future; \textbf{in an ideal\texttt{/}a perfect world} [idiom] used to say that something is what you would like to happen or what should happen, but you know it cannot; [v] \textbf{perfect something} to develop something so that it becomes perfect or as good as possible; [n] (\textbf{the perfect}) (also \textbf{the perfect tense}) [singular] (\textit{grammar}) the form of a verb that expresses actions completed by the present or a particular point in the past or future, formed in English with part of the verb \textit{have} \& the past participle of the main verb.} it. Ideologues\footnote{\textbf{ideologue} [n] (also \textbf{ideologist}) (\textit{formal, sometimes disapproving}) a person whose actions are influenced by belief in a set of principles ($=$ by an ideology).} are people who pretend\footnote{\textbf{pretend} [v] \textbf{1.} to behave in a particular way, in order to make other people believe something that is not true; \textbf{2.} (usually used in negative sentences \& questions) to claim to be, do or have something, especially when this is not true.} they know how to ``make the world a better place'' before they've taken care of their own chaos within. (The warrior\footnote{\textbf{warrior} [n] (\textit{formal}) (especially in the past) a brave or experienced solider or fighter.} identity that their ideology gives them covers over that chaos.) That's hubris\footnote{\textbf{hubris} [n] [uncountable] (\textit{literary}) the fact of being too proud. In literature, a character with this quality ignores warnings \& laws \& this usually results in their downfall \& death.}, of course, \& 1 of the most important themes of this book, is ``set your house in order'' 1st, \& Jordan provides practical advice on how to do this.

Ideologies are substitutes\footnote{\textbf{substitute} [v] [intransitive, transitive] to take the place of somebody\texttt{/}something else; to use somebody\texttt{/}something instead of somebody\texttt{/}something else; [n] a person or thing that you use or have instead of the usual one.} for true knowledge, \& ideologues are always dangerous when they come to power\footnote{\textbf{power} [n] \textbf{1.} [uncountable] the ability to control people or things; \textbf{2.} [uncountable] political control of a country or an area; \textbf{3.} [uncountable] (\textbf{powers} [plural]) (in people) the ability or opportunity to do something or to act in a particular way; \textbf{4.} [uncountable, countable, usually plural] the right or authority of a person or group to do something; \textbf{5.} [countable] a country with a lot of influence in world affairs; \textbf{6.} [uncountable] (in compounds) strength or influence in a particular area of activity; \textbf{7.} [uncountable] the influence of a particular thing or group within society; \textbf{8.} [uncountable] \textbf{power (of something\texttt{/}somebody)} the physical strength of something\texttt{/}somebody; \textbf{9.} [uncountable] \textbf{power (of something)} the quality of being effective or having a strong effect on people's feelings or thoughts; \textbf{10.} [uncountable] energy that can be collected \& used to operate a machine, to make electricity, etc.; \textbf{11.} [countable, uncountable] \textbf{power (of something)} (\textit{mathematics}) the number of times that an amount is to be multiplied by itself.}, because a simple-minded\footnote{\textbf{simple-minded} [a] (\textit{disapproving}) not intelligent; not understanding how complicated things are.} I-know-it-all approach is no match for the \fbox{complexity of existence}. Furthermore, when their social contraptions\footnote{\textbf{contraption} [n] a machine or piece of equipment that looks strange or complicated \& possibly does not work well.} fail to fly, ideologues blame\footnote{\textbf{blame} [v] to think or say that somebody\texttt{/}something is responsible for something bad; \textbf{be to blame (for something)} [idiom] to be responsible for something bad; [n] [uncountable] responsibility for doing something badly or wrongly; saying that somebody\texttt{/}something is responsible for something.} not themselves but all who see through the simplifications\footnote{\textbf{simplification} [n] \textbf{1.} [uncountable] \textbf{simplification (of something)} the process of making something less complicated, or easier to do or understand; \textbf{2.} [countable] a change that makes a problem, statement, system, etc. less complicated or easier to understand or do.}. Another great U of T professor, Lewis Feuer, in his book \textit{Ideology \& the Ideologists}, observed that ideologies retool\footnote{\textbf{retool} [v] \textbf{1.} [transitive, intransitive] \textbf{retool (something)} to replace or change the machines or equipment in a factory so that it can produce new or better goods; \textbf{2.} [transitive] \textbf{retool something} (\textit{North American English, informal}) to organize something in a new or different way.} the very religious\footnote{\textbf{religious} [a] \textbf{1.} [only before noun] connected with religion or with a particular religion; \textbf{2.} (of a person) believing strongly in the existence of a god or gods.} stories they purport to have supplanted\footnote{\textbf{supplant} [v] (\textit{formal}) \textbf{supplant somebody\texttt{/}something} to take the place of somebody\texttt{/}something (especially somebody\texttt{/}something older or less modern), \textsc{synonym}: \textbf{replace}.}, but eliminate\footnote{\textbf{eliminate} [v] \textbf{1.} to remove or get rid of something\texttt{/}somebody; \textbf{2.} \textbf{eliminate somebody} to kill somebody, especially an enemy or opponent; \textbf{3.} \textbf{eliminate something} (\textit{mathematics}) to remove a variable from an equation, typically by substituting another which is shown by another equation to have the same value; \textbf{4.} \textbf{eliminate something} (\textit{chemistry}) to produce a simple substance such as water in addition to a more complex substance as a result of a chemical reaction involving larger organic molecules.} the narrative\footnote{\textbf{narrative} [n] \textbf{1.} [countable] a description of events, especially in a novel, \textsc{synonym}: \textbf{story}; \textbf{2.} [uncountable] the act, process or skill of telling a story; \textbf{3.} [uncountable] the part of a work of literature that is narrated, as opposed to dialogue; \textbf{4.} [countable] a way of presenting a particular situation or process so that it makes clear or follows a set of aims or values; [a] [only before noun] connected with the act, process or skill of telling a story.} \& psychological richness\footnote{\textbf{richness} [n] [uncountable] the state of being rich in something, such as a variety of types or interesting qualities.}. Communism borrowed from the story of the Children of Israel in Egypt, with an enslaved\footnote{\textbf{enslave} [v] [usually passive] \textbf{1.} \textbf{enslave somebody} to make somebody a slave; \textbf{2.} \textbf{enslave somebody\texttt{/}something (to something)} [usually passive] to make somebody\texttt{/}something completely depending on something so that they cannot manage without it.} class, rich persecutors\footnote{\textbf{persecutor} [n] a person who treats another person or group of people in a cruel \& unfair way.}, a leader, like Lenin, who goes abroad, lives among the enslavers, \& then leads the enslaved to the promised\footnote{\textbf{promise} [n] \textbf{1.} [countable] a statement in which you say that you will definitely do something, or that something will definitely happen; \textbf{2.} [uncountable] the quality of being likely to be excellent or successful, \textsc{synonym}: \textbf{potential}; \textbf{3.} [uncountable, singular] a sign or a reason for hope that something may happen, especially something good; [v] \textbf{1.} [intransitive, transitive] to tell somebody that you will definitely do something, or that something will definitely happen; \textbf{2.} [transitive] to make something seem likely to happen; to show signs of something.} land (the utopia; the dictatorship\footnote{\textbf{dictatorship} [n] \textbf{1.} [countable, uncountable] government by a dictator; \textbf{2.} [countable] a country that is ruled by a dictator.} of the proletariat\footnote{\textbf{the proletariat} [n] [singular $+$ singular or plural verb] (\textit{specialist}) (used especially when talking about the past) the class of ordinary people who earn money by working, especially those who do not own any property.}).

To understand ideology, Jordan read extensively\footnote{\textbf{extensively} [adv] \textbf{1.} in a way that includes or deals with a wide range of information; \textbf{2.} in a way that covers a large area; \textbf{3.} to a great extent; in a wide range of ways.} about not only the Soviet gulag\footnote{\textbf{Gulag} [n] \textbf{1.} \textbf{the Gulag} [singular] a system of prison labor camps in the Soviet Union from 1930--1955, where many people died; \textbf{2.} \textbf{gulag} [countable] any political labor camp.}, but also the Holocaust\footnote{\textbf{holocaust} [n] \textbf{1.} [countable] a situation in which many things are destroyed \& many people killed, especially because of a war or a fire; \textbf{2.} \textbf{the Holocaust} [singular] the killing of millions of Jews by the German Nazi government in the period 1941--5.} \& the rise of Nazism\footnote{\textbf{Nazism} [n] [uncountable] the policies \& beliefs of the National Socialist party which controlled Germany from 1933--1945.}. I had never before met a person, born Christian \& of my generation, who was so utterly\footnote{\textbf{utter} [a] [only before noun] used to emphasize how complete something is, \textsc{synonym}: \textbf{total}.

\textbf{utterly} [adv].} tormented\footnote{\textbf{torment} [n] [uncountable, countable] (\textit{formal}) extreme pain, especially mental pain; a person or thing that causes this, \textsc{synonym}: \textbf{anguish}.} by what happened in Europe to the Jews, \& who had worked so hard to understand how it could have occurred. I too had studied this in depth. My own father survived Auschwitz\footnote{See, e.g., \href{https://en.wikipedia.org/wiki/Auschwitz_concentration_camp}{Wikipedia\texttt{/}Auschwitz concentration camp}.}. My grandmother was middle-aged when she stood face to face with Dr. Josef Mengele, the Nazi physician who conducted unspeakably\footnote{\textbf{unspeakably} [adv] (\textit{literary, usually disapproving}) in a way that cannot be described in words, usually because it is so bad, \textsc{synonym}: \textbf{indescribably}.} cruel\footnote{\textbf{cruel} [a] (\textbf{crueler, cruelest}) \textbf{1.} having a desire to cause pain \& suffering, \textsc{opposite}: \textbf{kind}; \textbf{2.} causing pain or suffering, \textsc{synonym}: \textbf{harsh}.} experiments on his victims, \& she survived Auschwitz by disobeying his order to join the line with the elderly, the grey \& the weak, \& instead slipping into a line with younger people. She avoided the gas chambers\footnote{\textbf{chamber} [n] \textbf{1.} [countable] a space inside the body, a plant, a machine, etc. which is separated from the rest; \textbf{2.} [countable] (in compounds) a room used for the particular purpose that is mentioned; \textbf{3.} [countable] a large room in a public building that is used for formal meetings; \textbf{4.} [countable $+$ singular or plural verb] 1 of the parts of a parliament; the people who belong to that part.} a 2nd time by trading food for hair dye\footnote{\textbf{dye} [v] to change the color of something, especially by using a special liquid or substance; [n] [countable, uncountable] a substance that is used to change the color of things such as cloth or hair.} so she wouldn't be murdered for looking too old. My grandfather, her husband, survived the Mauthausen concentration\footnote{\textbf{concentration} [n] \textbf{1.} [countable, uncountable] the amount of a substance in a liquid or in another substance; \textbf{2.} [countable] \textbf{concentration (of something)} a lot of something in 1 place; \textbf{3.} [uncountable] the process of people directing effort \& attention on a particular thing; \textbf{4.} [uncountable] the ability to direct all your effect \& attention on 1 thing, without thinking of other things.} camp\footnote{\textbf{concentration camp} [n] a type of prison, often consisting of a number of buildings inside a fence, where political prisoners, etc. are kept in extremely bad conditions.}, but choked\footnote{\textbf{choke} [v] \textbf{1.} [intransitive, transitive] to be unable to breathe because the passage to your lungs is blocked or you cannot get enough air; to make somebody unable to breathe; \textbf{2.} [transitive] \textbf{choke somebody} to make somebody stop breathing by pressing their throat, especially with your fingers, \textsc{synonym}: \textbf{strangle}; \textbf{3.} [intransitive, transitive] to be unable to speak normally especially because of strong emotion; to make somebody feel too emotional to speak normally; \textbf{4.} [transitive, usually passive] to block or fill a passage, space, etc. so that movement is difficult; \textbf{5.} [intransitive] (\textit{informal}) to fail at something, e.g. because you are nervous.} to death on the 1st piece of solid\footnote{\textbf{solid} [a] [usually before noun] \textbf{1.} not in the form of a liquid or gas; \textbf{2.} hard or firm, with a surface that does not move when pressed; \textbf{3.} having no holes or empty spaces inside; \textbf{4.} having a strong basis; reliable; \textbf{5.} (\textit{specialist}) having a shape with length; width \& height; \textbf{6.} [only before noun] made completely of the material mentioned; \textbf{7.} (of a line or color) without spaces; [n] \textbf{1.} [countable] a substance that is not a liquid or a gas; \textbf{2.} [countable] (\textit{geometry}) a shape that has length, width \& height; \textbf{3.} (\textbf{solids}) [plural] food that is not liquid.} food he was given, just before liberation\footnote{\textbf{liberation} [n] \textbf{1.} [uncountable] the act of freeing a country or a person from the control of somebody else; \textbf{2.} [uncountable] the act of freeing somebody from something that limits their ability to do things or enjoy life; freedom from these limits; \textbf{3.} [uncountable, singular] \textbf{liberation (of something)} (\textit{chemistry, physics}) the release of gas, energy, etc. as a result of a chemical reaction or physical process.} day. I relate\footnote{\textbf{relate} [v] \textbf{1.} to show or make a connection between 2 or more things, \textsc{synonym}: \textbf{connect}; \textbf{2.} to give a spoken or written report of something; to tell a story; \textbf{relate to something\texttt{/}somebody} [phrasal verb] \textbf{1.} to be connected with somebody\texttt{/}something; to refer to something\texttt{/}somebody; \textbf{2.} to be able to understand \& have sympathy with somebody\texttt{/}something.} this, because years after we became friends, when Jordan would take a classical liberal\footnote{\textbf{liberal} [a] \textbf{1.} willing to understand \& respect other people's behavior \& opinions, especially when they are different from your own; believing people should be able to choose how they behave; \textbf{2.} wanting or allowing a lot of political \& economic freedom \& supporting gradual social, political or religious change; \textbf{3.} (\textbf{Liberal}) connected with the British Liberal Party in the past, or of a Liberal Party in another country; \textbf{4.} (of education) concerned with increasing somebody's general knowledge \& experience rather than particular skills; \textbf{5.} not completely accurate or exact; [n] \textbf{1.} a person who supports political, social \& religious change; \textbf{2.} a person who understands \& respects other people's opinions \& behavior, especially when they are different from their own; \textbf{3.} (\textbf{Liberal}) a member of the British Liberal Party in the past, or of a Liberal Party in another country.} stand for free speech, he would be accused\footnote{\textbf{accuse} [v] \textbf{1.} to say formally that somebody has committed a crime so there can be a trial in court; \textbf{2.} to claim that somebody has done something wrong.} by left-wing\footnote{\textbf{left-wing} [a] strongly supporting the ideas of socialism.} extremists\footnote{\textbf{extremist} [n] (\textit{usually disapproving}) a person whose opinions, especially about religion or politics, are extreme, \& who may do things that are violent, illegal, etc. for what they believe; [a] [usually before noun].} as being a right-wing\footnote{\textbf{right-wing} [a] strongly supporting the capitalist system, \textsc{opposite}: \textbf{left-wing}.} bigot\footnote{\textbf{bigot} [n] a person who has very strong, unreasonable beliefs or opinions about race, religion or politics \& who will not listen to or accept the opinions of anyone who disagrees.}.

Let me say, with all the moderation\footnote{\textbf{moderation} [n] [uncountable] the quality of being reasonable \& not extreme.} I can summon\footnote{\textbf{summon} [v] \textbf{1.} to order somebody to appear in court; \textbf{2.} \textbf{summon somebody (to something) (to do something)} to order somebody to come to you; \textbf{3.} \textbf{summon something} to arrange an official meeting, \textsc{synonym}: \textbf{convene}; \textbf{4.} \textbf{summon something} to call for or try to obtain something; \textbf{5.} \textbf{summon something (up)} to make an effort to produce a particular quality in yourself, especially when you find it difficult; \textbf{summon something up} [phrasal verb] to make a feeling, an idea, a memory, etc. come into your mind, \textsc{synonym}: \textbf{evoke}.}: \textit{at best}, those accusers\footnote{\textbf{accuser} [n] a person who says that somebody has done something wrong or is guilty of something.} have simply not done their due\footnote{\textbf{due} [a] \textbf{1.} [not before noun] caused by somebody\texttt{/}something; \textbf{2.} [only before noun] suitable or right in the circumstances; \textbf{3.} [not before noun] arranged or expected; \textbf{4.} [not usually before noun] when a sum of money is due, it must be paid immediately; \textbf{5.} [not before noun] \textbf{due (to somebody)} owed to somebody as a debt, because it is their right or because they have done something to deserve it; [n] \textbf{1.} (\textbf{your\texttt{/}somebody's\texttt{/}something's due}) [uncountable] something that should be given to somebody\texttt{/}something by right; \textbf{2.} (\textbf{dues} [plural] charges, e.g. to be a member of a club.)} diligence\footnote{\textbf{diligence} [n] [uncountable] (\textit{formal}) careful work or great effort.}. I have; with a family history such as mine, one develops not only radar\footnote{\textbf{radar} [n] [uncountable] a system that users radio waves to find the position \& movement of objects, e.g. planes \& ships, when they cannot be seen.}, but underwater\footnote{\textbf{underwater} [adv] below the surface or water.} sonar\footnote{\textbf{sonar} [n] [uncountable] equipment or a system of finding objects underwater using sound waves.} for right-wing bigotry\footnote{\textbf{bigotry} [n] [uncountable] the state of feeling, or the act of expressing, strong, unreasonable beliefs or opinions.}; but even more important, one learns to recognize the kind of person with the comprehension\footnote{\textbf{comprehension} [n] [uncountable] the ability to understand.}, tools, good will \& courage\footnote{\textbf{courage} [n] [uncountable] the ability to do something dangerous, or to face pain or opposition, without showing fear, \textsc{synonym}: \textbf{bravery}.} to combat\footnote{\textbf{combat} [n] [uncountable, countable] fighting or a fight, especially during a time of war; [v] \textbf{combat something} to stop something unpleasant or harmful from happening or from getting worse.} it, \& \textsc{Jordan Peterson} is \textit{that} person.

My own dissatisfaction\footnote{\textbf{dissatisfaction} [n] [uncountable, countable] a feeling that you are not pleased or satisfied, because something is not as good as you expected, \textsc{opposite}: \textbf{satisfaction}.} with modern political science's attempts to understand the rise of Nazism, totalitarianism\footnote{\textbf{totalitarianism} [n] [uncountable] (\textit{disapproving}) the principles \& practices of a political system in which there is only 1 party, which has complete power \& control over the people.} \& prejudice\footnote{\textbf{prejudice} [n] [uncountable, countable] an unreasonable dislike of a person, group, etc. especially when it is based on their race, religion, sex, etc.; \textbf{without prejudice (to something)} [idiom] (\textit{law}) without affecting any other legal matter.} was a major\footnote{\textbf{major} [a] \textbf{1.} [usually before noun] large, important or serious, \textsc{opposite}: \textbf{minor}; \textbf{2.} [only before noun] greater or more important; main; \textsc{synonym}: \textbf{main}; [n] (\textit{North American English}) \textbf{1.} the main subject or course of a student at college or university; \textbf{2.} a student studying a particular subject as the main part of their course.} factor\footnote{\textbf{factor} [n] \textbf{1.} 1 of several things that cause or affect something; \textbf{2.} \textbf{by a factor of something} the amount by which something increases or decreases. The \textbf{factor} is the number you multiply or divide by to show the amount of the increase or decrease; \textbf{3.} (\textit{mathematics}) a number that divides into another number exactly; \textbf{4.} (also \textbf{factor of production} \textit{economics}) any of the resources that are used to produce goods \& services. The main factors of production are land, labor \& capital; \textbf{5.} (\textit{biology}) a substance that has a function in a particular biological process, e.g. growth or blood clotting; [v] \textbf{factor something in $|$ factor something into something} to include a particular fact or situation when you are thinking about or planning something.} in my decision to supplement\footnote{\textbf{supplement} [v] to add something to something in order to improve it or make it more complete; [n] \textbf{1.} a thing that is added to something else to improve or complete it; \textbf{2.} \textbf{supplement (to something)} a book or a section at the end of a book or online that gives extra information or deals with a special subject.} my studies of political science with the study of the unconscious\footnote{\textbf{unconscious} [a] \textbf{1.} in a state like sleep because of an injury or illness, \& not able to use your senses, \textsc{opposite}: \textbf{conscious}; \textbf{2.} (of feelings, thoughts, etc.) existing or happening without you realizing or being aware; not deliberate or controlled, \textsc{opposite}: \textbf{conscious}; \textbf{3.} \textbf{unconscious somebody\texttt{/}something} not aware of somebody\texttt{/}something; not noticing something; not conscious, \textsc{opposite}: \textbf{conscious}.}\,\footnote{\textbf{the unconscious} [n] [singular] (\textit{psychology}) the part of a person's mind with thoughts \& feelings that they are not aware of \& cannot control but which can sometimes be understood by studying their behavior or dreams.}, projection\footnote{\textbf{projection} [n] \textbf{1.} [countable] an estimate or forecast of a future situation based on what is happening now, \textsc{synonym}: \textbf{forecast}; \textbf{2.} [uncountable, countable] \textbf{projection (of something) (on\texttt{/}onto something)} the act of putting an image of something onto a surface; an image that is shown in this way; \textbf{3.} [countable] a method for representing a solid shape or object on a flat surface; \textbf{4.} [countable] \textbf{projection ($+$ adv.\texttt{/}prep.)} something that sticks out from a surface; \textbf{5.} [uncountable, countable] \textbf{projection (of something)} the act of giving a form \& structure to thoughts \& feelings; the form \& structure given to thoughts \& feelings; \textbf{6.} [uncountable] the act of imagining that somebody else has the same feelings, thoughts \& reactions as you.}, psychoanalysis, the regressive\footnote{\textbf{regressive} [a] \textbf{1.} becoming or making something less advanced; \textbf{2.} (of a tax) having less effect on the rich than on the poor.} potential\footnote{\textbf{potential} [a] [only before noun] that can develop into something or be developed in the future, \textsc{synonym}: \textbf{possible}; [n] \textbf{1.} [uncountable] the possibility of something happening or being developed or used; \textbf{2.} [uncountable] qualities that exist \& can be developed, \textsc{synonym}: \textbf{promise}; \textbf{3.} [uncountable, countable] (\textit{physics}) the difference in voltage between 2 points in an electric field or circuit.} of group psychology, psychiatry\footnote{\textbf{psychiatry} [n] [uncountable] the study \& treatment of mental illness.} \& the brain. Jordan switched out of political science for similar reasons. With these important parallel\footnote{\textbf{parallel} [a] \textbf{1.} 2 or more lines that are parallel to each other are the same distance apart at every point; \textbf{2.} very similar; taking place at the same time; [n] \textbf{1.} [countable, uncountable] a person, a situation or an event that is very similar to another, especially one in a different place or time, \textsc{synonym}: \textbf{equivalent}; \textbf{2.} [countable, usually plural] \textbf{parallel between A \& B} a comparison between 2 things; \textbf{in parallel (with something)} with \& at the same time as something else; [v] \textbf{1.} \textbf{parallel something} to be similar to something; to happen at the time as something; \textbf{2.} to be the same distance apart from something at every point.} interests, we didn't always agree on ``the answers'' (thank God), but we almost \fbox{always agreed on the questions}.

Our friendship wasn't all doom\footnote{\textbf{doom} [n] [uncountable] death or destruction; any terrible event that you cannot avoid; [v] [usually passive] to make somebody\texttt{/}something certain to fail, suffer, die, etc.} \& gloom\footnote{\textbf{gloom} [n] \textbf{1.} [uncountable, singular] a feeling of being sad \& without hope, \textsc{synonym}: \textbf{depression}; \textbf{2.} [uncountable] (\textit{literary}) almost total darkness.}\,\footnote{\textbf{doom \& gloom $|$ gloom \& doom} [idiom] a general feeling of having lost all hope, \& of pessimism ($=$ expecting things to go badly).}. I have made a habit\footnote{\textbf{habit} [n] \textbf{1.} [countable, uncountable] something that you do often \& almost without thinking about it, especially something that is difficult to change or stop; a person's usual behavior; \textbf{2.} [countable] a typical way of behaving that something has; the fact that something tends to happen in a particular way.} of attending my fellow professors' classes at our university, \& so attended his, which were always packed, \& I saw what now millions have seem online: a brilliant, often dazzling\footnote{\textbf{dazzling} [a] \textbf{1.} (of light) so bright that you cannot see for a short time, \textsc{synonym}: \textbf{blinding}; \textbf{2.} impressing somebody very much, \textsc{synonym}: \textbf{brilliant}.} public speaker who was at his best riffing\footnote{\textbf{riff} [v] \textbf{1.} to play a short repeated pattern of notes in popular music or jazz; \textbf{2.} \textbf{riff (on something)} to perform a monologue ($=$ long speech by 1 person) on a particular subject, especially a funny one that you make up as you are speaking; [n] \textbf{1.} a short repeated pattern of notes in popular music or jazz; \textbf{2.} \textbf{riff (on something)} a monologue ($=$ long speech by 1 person) on a particular subject, especially a funny one that you make up as you are speaking.} like a jazz\footnote{\textbf{jazz} [n] [uncountable] a type of music with strong rhythms, in which the players often improvise ($=$ make up the music as they are playing), originally created by African American musicians at the beginning of the 20th century; [v] \textbf{jazz up} [phrasal verb].} artist\footnote{\textbf{artist} [n] \textbf{1.} a person who creates works of art, especially paintings or drawings; \textbf{2.} a person who performs for a profession, such as a singer, a dancer or an actor.}; at times he resembled\footnote{\textbf{resemble} [v] [no passive] (not used in the progressive tenses) \textbf{resemble somebody\texttt{/}something} to look like or be similar to another person or thing.} an ardent\footnote{\textbf{ardent} [a] [usually before noun] very enthusiastic \& showing strong feelings about something\texttt{/}somebody, \textsc{synonym}: \textbf{passionate}.} Prairie preacher\footnote{\textbf{preacher} [n] a person, often a member of the clergy, who gives religious talks \& often performs religious ceremonies, e.g. in a church.} (not in evangelizing\footnote{\textbf{evangelize} [v] (\textit{British English also} \textbf{evangelise}) [transitive, intransitive] \textbf{evangelize (somebody\texttt{/}something)} to try to persuade people to become Christians.}, but in his passion, in his ability to tell stories that convey\footnote{\textbf{convey} [v] \textbf{1.} to communicate information, a message, an idea or a feeling; \textbf{2.} to take, carry or transport somebody\texttt{/}something from 1 place to another; \textbf{3.} (\textit{law}) to change the legal owner of a property or piece of land, \textsc{synonym}: \textbf{transfer}.} the life-stakes\footnote{\textbf{stake} [n] \textbf{1.} [countable] a share of a business that somebody owns because they have invested money in it, \textsc{synonym}: \textbf{holding}; \textbf{2.} [singular] \textbf{stalk in something} a part in something that is important to you \& that you want to be successful; \textbf{3.} (\textbf{stakes}) [plural] something that you risk losing when you are involved in an activity that can succeed or fail; \textbf{4.} [countable] a wooden or metal post that is pointed at 1 end \& pushed into the ground in order to support something, mark a particular place, etc.; \textbf{5.} (\textbf{the stake}) [singular] (in the past) a wooden post that somebody could be tied to before being burnt to death as a punishment; [v] \textbf{1.} \textbf{stake something on (doing) something} to risk money or something important on the result of something; \textbf{2.} to state your opinion or position on something very clearly.} that go with believing or disbelieving\footnote{\textbf{disbelieving} [a] showing that you do not believe that something is true or that somebody is telling the truth.} various ideas). Then he'd just as easily switch to do a breathtakingly\footnote{\textbf{breathtakingly} [adv] in a way that is very exciting, impressive or surprising.} systematic\footnote{\textbf{systematic} [a] \textbf{1.} done according to a system or plan, in a thorough, efficient or determined way; \textbf{2.} (of an error) happening in the same way all through a process or set of results; caused by the system that is used.} summary\footnote{\textbf{summary} [n] (plural \textbf{summaries}) a short statement that gives only the main points of something, not the details; [a] [only before noun] \textbf{1.} giving only the main points of something, not the details; \textbf{2.} done immediately, without paying attention to the normal process that should be followed.} of a series of scientific\footnote{\textbf{scientific} [a] [usually before noun] \textbf{1.} involving science; connected with science; \textbf{2.} done in a careful \& organized way, \textsc{synonym}: \textbf{methodical}.} studies. He was a master at helping students become more reflective\footnote{\textbf{reflective} [a] \textbf{1.} thinking carefully about things, especially about your work or studies; \textbf{2.} \textbf{reflective of something} typical of a particular situation or thing; showing the state or nature of something; \textbf{3.} reflective surfaces send back light or heat.}, \& take themselves \& their futures seriously\footnote{\textbf{seriously} [adv] \textbf{1.} to a degree that is important \& worrying; \textbf{2.} carefully \& sincerely; \textbf{take somebody\texttt{/}something seriously} [idiom] to think that somebody\texttt{/}something is important \& deserves attention \& respect.}. He taught them to respect many of the greatest books ever written. He gave vivid\footnote{\textbf{vivid} [a] \textbf{1.} (of memories, a description, etc.) producing very clear pictures in your mind, \textsc{synonym}: \textbf{graphic}; \textbf{2.} (of light, colors, etc.) very bright.} examples from clinical\footnote{\textbf{clinical} [a] [only before noun] connected with the examination \& treatment of patients \& their illnesses.} practice, was (appropriately\footnote{\textbf{appropriately} [adv] in a way that is suitable, acceptable or correct for the particular circumstances.}) self-revealing\footnote{\textbf{revealing} [a] \textbf{1.} giving you interesting information that you did not know before; \textbf{2.} (of clothes) allowing more of somebody's body to be seen than usual.}, even of his own vulnerabilities\footnote{\textbf{vulnerability} [n] [uncountable] \textbf{vulnerability (of somebody\texttt{/}something) (to something)} the fact of being weak \& easily hurt physically or emotionally.}, \& made fascinating links between evolution, the brain \& religious\footnote{\textbf{religious} [a] \textbf{1.} [only before noun] connected with religion or with a particular religion; \textbf{2.} (of a person) believing strongly in the existence of a god or gods.} stories. In a world where students are taught to see evolution \& religion as simply opposed (by thinkers like Richard Dawkins), Jordan showed his students how evolution, of all things, helps to explain the profound psychological appeal\footnote{\textbf{appeal} [n] \textbf{1.} [countable, uncountable] a formal request to a court or to somebody in authority for a judgment or a decision to be changed; \textbf{2.} [uncountable] a quality that makes somebody\texttt{/}something attractive or interesting; \textbf{3.} [countable] \textbf{appeal (for something)} an urgent request for money, help or information; [v] \textbf{1.} [intransitive] to make a formal request to a court or to somebody in authority for a judgment or a decision to be changed. In North American English, the \textbf{appeal (something) (to somebody\texttt{/}something)} is usually used, without a preposition.; \textbf{2.} [intransitive] \textbf{appeal to somebody} to attract or interest somebody; \textbf{3.} [intransitive] to make a serious \& urgent request; \textbf{4.} [intransitive] \textbf{appeal to something} to try to persuade somebody to do something by suggesting that it is a fair, reasonable or honest thing to do.} \& wisdom\footnote{\textbf{wisdom} [n] \textbf{1.} [uncountable, singular] the ability to make sensible decisions \& give good advice, because of the experience \& knowledge that you have; \textbf{2.} [uncountable, countable] the knowledge \& experience that develops within a particular society or group of people. \textbf{(The) conventional\texttt{/}received wisdom} is what most people believe to be true. \textbf{Common, popular} \& \textbf{traditional} are also used in this way.; \textbf{3.} [singular] \textbf{the wisdom of (doing) something} how sensible something is.} of many ancient stories, from Gilgamesh to the life of the Buddha, Egyptian mythology \& the Bible. He showed, e.g., how stories about journeying voluntarily into the unknown -- \fbox{the hero's quest} -- mirror\footnote{\textbf{mirror} [n] \textbf{1.} a piece of special glass that reflects images \& light; \textbf{2.} [usually singular] \textbf{mirror of something} a thing that shows what something else is like. To \textbf{hold a mirror up to something} is to examine it or show what it is like.; [v] to have features that are similar to something else, especially in a way that clearly shows what the other thing is like, \textsc{synonym}: \textbf{reflect}.} universal\footnote{\textbf{universal} [a] \textbf{1.} done by or involving all the people in the world or in a particular group; \textbf{2.} true or right at all times \& in all places.} tasks for which the brain evolved\footnote{\textbf{evolve} [v] \textbf{1.} [intransitive, transitive] to develop gradually, especially from a simple to a more complicated form; to develop something in this way; \textbf{2.} [intransitive, transitive] (\textit{biology}) (of living things) to develop over time, often many generations, into forms that are better adapted to survive changes in their environment.}. He respected the stories, was not reductionist\footnote{\textbf{reductionist} [n] (\textit{formal, often disapproving}) a person who believes that complicated things can be explained by considering them as a combination of simple parts; [a] \textbf{reductionist} [a] (\textit{formal, often disapproving}) showing the belief that complicated things can be explained by considering them as a combination of simple parts.}, \& never claimed to exhaust\footnote{\textbf{exhaust} [n] \textbf{1.} [uncountable] waste gases that come out of a vehicle, an engine or a machine; \textbf{2.} [countable] the system in a vehicle through which exhaust gases come out; [v] \textbf{1.} to make somebody feel very tired, \textsc{synonym}: \textbf{wear out}; \textbf{2.} \textbf{exhaust something} to use all of something so that there is none left; \textbf{3.} \textbf{exhaust something} to talk about or study a subject until there is nothing else to say about it.} their wisdom. If he discussed a topic such as prejudice, or its emotional\footnote{\textbf{emotional} [a] \textbf{1.} [usually before noun] connected with people's feelings; \textbf{2.} causing people to feel strong emotions, \textsc{synonym}: \textbf{emotive}.} relatives\footnote{\textbf{relative} [a] \textbf{1.} considered \& judged by being compared with something else; \textbf{2.} [only before noun] existing or having a particular quality only when compared with something else, \textsc{synonym}: \textbf{comparative}; \textbf{3.} (\textit{grammar}) referring to an earlier noun, sentence or part of a sentence; \textbf{relative to somebody\texttt{/}something} [idiom] \textbf{1.} in comparison with somebody\texttt{/}something; \textbf{2.} in relation to somebody\texttt{/}something; \textbf{3.} about or concerning somebody\texttt{/}something; [n] \textbf{1.} a person who is in the same family as somebody else, \textsc{synonym}: \textbf{relation}; \textbf{2.} a type of animal or plant that belongs to the same group as something else.} fear\footnote{\textbf{fear} [n] [uncountable, countable] the bad feeling that you have when you are in danger, when something bad might happen, or when a particular thing frightens you; \textbf{for fear of something\texttt{/}of doing something, for fear that $\ldots$} [idiom] to avoid the danger of something happening; [v] \textbf{1.} to be frightened of somebody\texttt{/}something or frightened of doing something; \textbf{2.} to feel that something bad might have happened or might happen in the future; \textbf{fear for somebody\texttt{/}something} [phrasal verb] to be worried about somebody\texttt{/}something.} \& disgust\footnote{\textbf{disgust} [n] [uncountable] a strong feeling of dislike for somebody\texttt{/}something that you feel is unacceptable, or for something that looks, smells, etc. unpleasant; [v] \textbf{disgust somebody} if something disgusts you, it makes you feel shocked \& almost sick because it is so unpleasant.}, or the differences between the sexes on average, he was able to show how these traits\footnote{\textbf{trait} [n] a particular quality or characteristic, especially in somebody's personality. In biology, a \textbf{trait} is a characteristic in a person or animal that depends on the genes passed down from the parents.} evolved \& why they survived.

Above all, he alerted\footnote{\textbf{alert} [a] \textbf{1.} \textbf{alert to something} aware of something, especially a problem or danger; \textbf{2.} able to think quickly; quick to notice things; [v] \textbf{1.} to warn somebody about a dangerous or urgent situation; \textbf{2.} \textbf{alert somebody to something} to make somebody aware of something.} his students to topics rarely\footnote{\textbf{rarely} [adv] not often} discussed in university, such as the simple fact that all the ancients, from Buddha\footnote{\textbf{Buddha} [n] \textbf{1.} (also \textbf{the Buddha}) [singular] a title given to Siddhartha Gautama, the person on whose teachings the Buddhist religion is based; \textbf{2.} [countable] a statue or picture of the Buddha; \textbf{3.} [countable] a person who has achieved enlightenment ($=$ spiritual knowledge) in Buddhism.} to the biblical\footnote{\textbf{biblical} [a] (also \textbf{Biblical}) \textbf{1.} connected with the Bible; in the Bible; \textbf{2.} very great; on a large scale; \textbf{know somebody in the biblical sense} [idiom] (\textit{humorous}) to have had sex with somebody.} authors, knew what every slightly\footnote{\textbf{slightly} [adv] a little.} worn-out\footnote{\textbf{worn out} [a] \textbf{1.} (of a thing) badly damaged \&\texttt{/}or no longer useful because it has been used a lot; \textbf{2.} [not usually before noun] (of a person) looking or feeling very tired, especially as a result of hard work or physical exercise.} adult knows, that \fbox{life is suffering}. If you are suffering, or someone close to you is, that's sad. But alas\footnote{\textbf{alas} [exclamation] (\textit{old use or literary}) used to show you are sad or sorry.}, it's not particularly special. We don't suffer only because ``politicians\footnote{\textbf{politician} [n] a person whose job is concerned with politics, especially as an elected member of parliament, etc.} are dimwitted\footnote{\textbf{dim-witted} [a] (\textit{informal}) stupid.},'' or ``the system is corrupt\footnote{\textbf{corrupt} [a] \textbf{1.} (of people) willing to use their power to do dishonest or illegal things in return for money or to get an advantage; \textbf{2.} (of behavior) dishonest or immoral; [v] \textbf{1.} \textbf{corrupt somebody} to have a bad effect on somebody \& make them behave in an immoral or dishonest way; \textbf{2.} [often passive] \textbf{corrupt something} to change the original form of something, so that it is damaged or spoiled in some way; \textbf{3.} },'' or because you \& I, like almost everyone else, can \textit{legitimately}\footnote{\textbf{legitimately} [adv] \textbf{1.} in a way that can be defended with a fair \& acceptable reason; \textbf{2.} in a way that is allowed according to the law or rules.} describe ourselves, in some way, as a victim\footnote{\textbf{victim} [n] \textbf{1.} a person who has been injured or killed as the result of a crime, disease, accident, etc.; \textbf{2.} a person, organization, etc. that has suffered because of a difficult situation, or because of the attitudes or actions of other people; \textbf{3.} an animal or person that is killed \& offered to a god; \textbf{fall victim (to something)} [idiom] to be injured, killed, damaged or destroyed by something.} of something or someone. It is because we are born human that we are guaranteed\footnote{\textbf{guarantee} [n] \textbf{1.} a firm promise that something will be one or that something will happen, \textsc{synonym}: \textbf{assurance}; \textbf{2.} something that makes something else certain to happen; \textbf{3.} a written promise given by a company that something you buy will be replaced or repaired without payment if it goes wrong within a particular period; \textbf{4.} a written promise to pay back money that somebody else owes, or do something that somebody else promised to do, if they cannot do it themselves; [v] \textbf{1.} to promise to do or keep something; to promise something will happen or exist; \textbf{2.} to make something certain to happen; \textbf{3.} to agree to be legally responsible for something or for doing something, especially for paying back money that somebody else owes if they cannot pay it back themselves; \textbf{be guaranteed to do something} [idiom] to be certain to have a particular result.} a good dose\footnote{\textbf{dose} [n] \textbf{1.} an amount of a medicine or a drug that is taken, or recommended to be taken; \textbf{2.} \textbf{dose (of something)} an amount of radiation that is given at 1 time, or over a period of time; [v] \textbf{dose somebody\texttt{/}something (with something)} to give a person or animal a medicine or drug.} of suffering\footnote{\textbf{suffering} [n] \textbf{1.} [uncountable] physical or mental pain; \textbf{2.} (\textbf{sufferings}) [plural] \textbf{suffering (of somebody)} feelings of pain \& unhappiness.}. \& chances are, if you or someone you love is not suffering now, they will be within 5 years, unless you are freakishly\footnote{\textbf{freakishly} [adv] in a way that is very strange, unusual or unexpected.} lucky. Rearing\footnote{\textbf{rear} [v] \textbf{1.} \textbf{rear somebody\texttt{/}something} [often passive] to care for young children or animals until they are fully grown, \textsc{synonym}: \textbf{raise}; \textbf{2.} \textbf{rear something} to breed or keep animals or birds, e.g. on a farm; \textbf{something rears its head} (of something unpleasant) [idiom] to appear or happen; [n] (usually \textbf{the rear}) [singular] the back part of something; [a] at the near the back of something.} kids is hard, work is hard, aging, sickness \& death are hard, \& Jordan emphasized that doing all that totally\footnote{\textbf{totally} [adv] (used to emphasize the following word or phrase) completely.} on your own, without the benefit\footnote{\textbf{benefit} [n] \textbf{1.} [countable, uncountable] a helpful \& useful effect that something has; an advantage that something provides; \textbf{2.} [uncountable, countable] (\textit{British English}) money provided by the government to people who need financial help because they are unemployed, sick, etc.; [v] \textbf{1.} [intransitive] to be in a better position because of something; \textbf{2.} [transitive] \textbf{benefit somebody\texttt{/}something} to be useful or provide an advantage to somebody\texttt{/}something.} of a loving relationship, or wisdom, or the psychological insights of the greatest psychologists, only makes it harder. He wasn't scaring the students; in fact, they found this frank talk reassuring\footnote{\textbf{reassuring} [a] making you feel less worried or uncertain about something.}, because in the depths\footnote{\textbf{depth} [n] \textbf{1.} [countable, uncountable] the distance from the top or surface to the bottom of something; how deep something is; \textbf{2.} [uncountable] \textbf{depth (of something)} the fact of having or providing a lot of information or knowledge; \textbf{3.} [uncountable] \textbf{depth (of something)} the fact of being very important or serious; \textbf{4.} [uncountable] the quality in an image that makes it appear not to be flat; \textbf{the depths of something} [idiom] \textbf{1.} the deepest part of something; \textbf{2.} the most serious or extreme part of something; \textbf{in depth} [idiom] in a detailed \& thorough way.} of their psyches\footnote{\textbf{psyche} [n] the mind; your deepest feelings \& attitudes.}, most of them knew what he said was true, even if there was never a forum\footnote{\textbf{forum} [n] (plural \textbf{forums, fora}) \textbf{1.} a place where people can exchange opinions \& ideas on a particular issue; a meeting organized for this purpose; \textbf{2.} an Internet group or website for discussing a particular issue; \textbf{3.} (in ancient Rome) a public place where meetings were held.} to discuss it -- perhaps because the adults in their lives had become so naively\footnote{\textbf{naive} [a] (also \textbf{na\"ive}) \textbf{1.} (\textit{disapproving}) lacking experience of life, knowledge or good judgment; \textbf{2.} (\textit{approving}) (of people \& their behavior) simple \& lacking experience in life.}\,\footnote{\textbf{naively} [adv] (also \textbf{na\"ively}) \textbf{1.} (\textit{disapproving}) in a way that shows you lack knowledge, good judgment or experience of life \& are willing to believe that people always tell you the truth; \textbf{2.} (\textit{art}) in a style which is deliberately very simple, often uses bright colors \& is similar to that produced by a child.} overprotective\footnote{\textbf{overprotective} [a] (\textit{disapproving}) too anxious to protect somebody from being hurt, in a way that limits their freedom.} that they deluded\footnote{\textbf{delude} [v] to make somebody believe something that is not true, \textsc{synonym}: \textbf{deceive}.} themselves into thinking that not talking about suffering would in some way magically\footnote{\textbf{magical} [a] containing magic; used in magic.}\,\footnote{\textbf{magically} [adv] \textbf{1.} in a way that cannot easily be explained \& seems to involve the use of magic; \textbf{2.} in a very beautiful or pleasant way, \textsc{synonym}: \textbf{enchantingly}; \textbf{3.} by magic; using magic.} protect\footnote{\textbf{protect} [v] \textbf{1.} [transitive, intransitive] to keep somebody\texttt{/}something safe from harm or injury; \textbf{2.} [transitive, usually passive] to introduce laws that make it illegal to kill, harm or damage a particular animal, area of land, building, etc.; \textbf{3.} [transitive] to help an industry in your own country by taxing goods from other countries so that there is less competition; \textbf{4.} [transitive, intransitive] to provide somebody\texttt{/}something with insurance against fire, injury, damage, etc.} their children from it.

Here he would relate the myth\footnote{\textbf{myth} [n] [countable, uncountable] \textbf{1.} a story from ancient times, especially one that was told to explain natural events or to describe the early history of a people; this type of story, \textsc{synonym}: \textbf{legend}; \textbf{2.} something that many people believe but that does not exist or is false, \textsc{synonym}: \textbf{fallacy}.} of the hero\footnote{\textbf{hero} [n] (plural \textbf{heroes}) \textbf{1.} the main male character of a story, who usually has good qualities; \textbf{2.} a person, especially a man, who is admired by many person for doing something brave or good.}, a cross-cultural\footnote{\textbf{cross-cultural} [a] involving 2 or more different countries or cultures.} theme explored psychoanalytically \footnote{\textbf{psychoanalytically} [adv] in a way that uses or relates to psychoanalysis ($=$ treatment of mental health problems by encouraging somebody to talk about past experiences \& feelings, in order to understand fears \& feeling that they were not aware of).} by \textsc{Otto Rank}, who noted, following Freud, that hero myths are similar in many cultures\footnote{\textbf{culture} [n] \textbf{1.} [uncountable] the customs, beliefs, art, way of life or social organization of a particular country or group; \textbf{2.} [countable] a country or group with its own customs \& beliefs, art, way of life \& social organization; \textbf{3.} [countable, uncountable] the typical beliefs, attitudes \& behavior that people in a particular group or organization share; \textbf{4.} [uncountable] \textbf{culture (of something)} activities such as literature, music, art \& film, thought as a group; \textbf{5.} [uncountable] the process of growing cells or bacteria in an artificial substance for medical or scientific study; the substance in which they are grown; \textbf{6.} [countable] a group of cells or bacteria grown for medical or scientific study.}, a theme that was picked up by \textsc{Carl Jung, Joseph Campbell \& Erich Neumann}, among others. Where Freud made great contributions\footnote{\textbf{contribution} [n] \textbf{1.} [usually singular] the part played by a person or thing in achieving, improving or causing something; \textbf{2.} a sum of money that is given to a person or an organization in order to help pay for something, \textsc{synonym}: \textbf{donation}; \textbf{contribution (to something)} an item that forms part of a book, magazine, broadcast, discussion, etc.; \textbf{4.} a sum of money that you pay regularly to your employer or the government in order to pay for benefits such as health insurance or a pension.} in explaining neuroses\footnote{\textbf{neurosis} [n] [countable, uncountable] (plural \textbf{neuroses}) \textbf{1.} (\textit{psychology}) a mental health condition in which a person has strong feelings of fear or worry; \textbf{2.} any strong fear or worry, \textsc{synonym}: \textbf{anxiety}.} by, among other things, focusing on understanding what we might call a failed-hero story (that of Oedipus), Jordan focused on triumphant\footnote{\textbf{triumphant} [a] very successful; showing great happiness about a victory or success.} heroes. In all these triumph\footnote{\textbf{triumph} [n] \textbf{1.} [countable, uncountable] a great success, achievement or victory; \textbf{2.} [uncountable] the state of having achieved a great success or victory; the feeling of happiness that you get from this; [v] [intransitive] to defeat somebody\texttt{/}something; to be successful.} stories, the hero has to go into the unknown, into an unexplored territory, \& deal with a new great challenge \& take great risks. In the process, something of himself has to die, or be given up, so he can be reborn\footnote{\textbf{reborn} [v] \textbf{be reborn} used only in the passive without \textit{by}, \textbf{1.} to become active or popular again; \textbf{2.} to be born again; [a] [usually before noun] \textbf{1.} having become active again; \textbf{2.} having experienced a complete spiritual change.} \& meet the challenge. This requires courage\footnote{\textbf{courage} [n] [uncountable] the ability to do something dangerous, or to face pain or opposition, without showing fear, \textsc{synonym}: \textbf{bravery}.}, something rarely discussed in a psychology class or textbook\footnote{\textbf{textbook} [n] (\textit{North American English also} \textbf{text}) a book that teaches a particular subject \& that is used especially in schools \& colleges.}. During his recent public stand for free speech \& against what I call ``forced speech'' (because it involves a government forcing citizens to voice political views), the stakes were very high, he had much to lose, \& knew it. Nonetheless, I saw him (\& Tammy, for that matter) not only display such courage, but also continue to live by many of the rules in this book, some of which can be very demanding\footnote{\textbf{demanding} [a] \textbf{1.} (of a task) needing a lot of skill, care or effort; \textbf{2.} (of a person) expecting a lot of work or attention from others; not easily satisfied.}.

I saw him grow, from the remarkable\footnote{\textbf{remarkable} [a] unusual or surprising in a way that causes people to take notice, \textsc{opposite}: \textbf{unremarkable}.} person he was, into someone even more able \& assured -- through living by these rules. In fact, it was the process of writing this book, \& developing these rules, that led him to take the stand he did against forced or compelled\footnote{\textbf{compel} [v] \textbf{1.} to force somebody to do something; \textbf{2.} \textbf{compel something} to make something happen through the use of force or pressure; \textbf{3.} compel something (not used in the progressive tenses) to cause a particular reaction.} speech. \& that is why, during those events, he started posting some of his thoughts about life \& these rules on the Internet. Now, over 100 million YouTube hits later, we know they have struck\footnote{\textbf{strike} [v] \textbf{1.} [transitive] \textbf{strike somebody\texttt{/}something} to hit somebody\texttt{/}something hard or with force; \textbf{2.} [transitive] \textbf{strike somebody\texttt{/}something} to hit somebody\texttt{/}something with your hand or a weapon; \textbf{3.} [intransitive, transitive] to attack somebody\texttt{/}something, especially suddenly; \textbf{4.} [intransitive, transitive] to happen suddenly \& have a harmful or damaging effect on somebody\texttt{/}something; \textbf{5.} [intransitive, transitive] (of lighting) to hit \& hurt or damage somebody\texttt{/}something on the ground; \textbf{6.} [transitive] \textbf{strike something} (of light) to fall on a surface; \textbf{7.} [transitive, often passive] to cause somebody to notice or be interested; to make a particular impression on somebody; \textbf{8.} [intransitive] to refuse to work, because of a disagreement over pay or conditions.} a chord\footnote{\textbf{chord} [n] \textbf{1.} (\textit{music}) 3 or more notes played together; \textbf{2.} (\textit{mathematics}) a straight line that joins 2 points on a curve; \textbf{strike\texttt{/}touch a chord (with somebody)} [idiom] to say or do something that makes people feel sympathy or enthusiasm.}.

Given our distaste\footnote{\textbf{distaste} [n] [uncountable, singular] a feeling that something is unpleasant or offensive.} for rules, how do we explain the extraordinary\footnote{\textbf{extraordinary} [a] \textbf{1.} unexpected, surprising or strange; \textbf{2.} not normal or ordinary; greater or better than usual; \textbf{3.} [only before noun] (of a meeting, etc.) arranged for a special purpose \& happening in addition to what normally or regularly happens.} response to his lectures\footnote{\textbf{lecture} [n] a talk that is given to a group of people to teach them about a particular subject, often as part of a university or college course; [v] [intransitive] \textbf{lecture (in\texttt{/}on something) (to somebody)} to give a talk or a series of talks to a group of people on a particular subject, especially as a way of teaching in a university or college.}, which give rules? In Jordan's case, it was of course his charisma\footnote{\textbf{charisma} [n] [uncountable] the powerful personal quality that some people have to attract \& impress other people.} \& a rare willingness\footnote{\textbf{willingness} [n] [uncountable, singular] \textbf{willingness (of somebody) to do something} the fact of being willing to do something.} to stand for a principle that got him a wide hearing online initially\footnote{\textbf{initially} [adv] at the beginning.}; views of his 1st YouTube statements quickly numbered in the hundreds of thousands. But people have kept listening because what he is saying meets a deep \& unarticulated\footnote{\textbf{articulated} [a] (of a vehicle) with 2 or more sections joined together in a way that makes it easier to turn corners.} need. \& that is because alongside\footnote{\textbf{alongside} [prep] \textbf{1.} next to or at the side of something; \textbf{2.} together with something\texttt{/}somebody; at the same time as something\texttt{/}somebody.} our wish to be free of rules, we all search for structure\footnote{\textbf{structure} [n] \textbf{1.} [uncountable, countable] the way in which the parts of something are connected together, arranged or organized; a particular arrangement of parts; \textbf{2.} [countable] a thing that is made of several parts arranegd in a particular way, e.g. a building; \textbf{3.} [uncountable, countable] the state of being well organized or planned with all the parts linked together; a careful plan; [v] [often passive] to arrange or organize something into a system or pattern.}.

The hunger\footnote{\textbf{hunger} [n] \textbf{1.} [uncountable] the state of not having enough food to eat, especially when this causes illness or death; \textbf{2.} [uncountable] the feeling caused by a need to eat; \textbf{3.} [singular] \textbf{hunger (for something)} (\textit{formal}) a strong desire for something.} among many younger people for rules, or at least guidelines\footnote{\textbf{guideline} [n] \textbf{1.} [usually plural] a rule or instruction that is given by an official organization telling you how to do something; \textbf{2.} something that can be used to help you decide or form an opinion about something.}, is greater today for good reason. In the West at least, millennials\footnote{\textbf{millennial} [n] [usually plural] a person who was born between the early 1980s \& the late 1990s; a member of Generation Y.} are living through a unique historical situation. They are, I believe, the 1st generation to have been so thoroughly\footnote{\textbf{thoroughly} [adv] \textbf{1.} very; very much; completely; \textbf{2.} carefully \& with great attention to detail.} taught 2 seemingly\footnote{\textbf{seemingly} [adv] in a way that appears to be true but may in fact not be, \textsc{synonym}: \textbf{apparently}.} contradictory\footnote{\textbf{contradictory} [a] containing or showing a contradiction.} ideas about morality\footnote{\textbf{morality} [n] (plural \textbf{moralities}) \textbf{1.} [uncountable] principles concerning right \& wrong or good \& bad behavior; \textbf{2.} [uncountable] the degree to which something is right or wrong, or good or bad, according to moral principles; \textbf{3.} [uncountable, countable] a system of moral principles followed by a particular group of people.}, simultaneously\footnote{\textbf{simultaneously} [adv] at the same time as something else.} -- at their schools, colleges \& universities, by many in my own generation. This contradiction\footnote{\textbf{contradiction} [n] \textbf{1.} [countable, uncountable] a lack of agreement between facts, opinions or actions; \textbf{2.} [uncountable, countable] the act of saying that something that somebody else has said is wrong or not true; an example of this; \textbf{a contradiction in terms} [idiom] a statement containing 2 words or phrases that contradict each other's meaning.} has left them at times disoriented\footnote{\textbf{disorientated} [a] (also \textbf{disoriented}) \textbf{1.} unable to recognize where you are or where you should go; \textbf{2.} feeling confused \& unable to think clearly.} \& uncertain\footnote{\textbf{uncertain} [a] \textbf{1.} [not before noun] feeling doubt about something; not sure, \textsc{opposite}: \textbf{certain}; \textbf{2.} likely to change, especially in a negative or unpleasant way; \textbf{3.} not definite or decided; not known exactly, \textsc{synonym}: \textbf{unclear}; \textbf{4.} not confident; \textbf{in no uncertain terms} [idiom] clearly \& strongly.}, without guidance\footnote{\textbf{guidance} [n] [uncountable] \textbf{1.} help or advice that is given to somebody, especially by somebody in authority; \textbf{2.} the process of controlling the direction or position of something using special equipment.} \&, more tragically\footnote{\textbf{tragically} [adv] in a way that makes you feel very sad, usually because somebody has died or suffered a lot.}, deprived\footnote{\textbf{deprived} [a] without enough food, education, \& all the things that are necessary for people to live a happy \& comfortable life.} of\footnote{\textbf{deprive of} [phrasal verb] \textbf{deprive somebody\texttt{/}something of something} to prevent somebody from having or doing something, especially something important.} riches they don't even know exist.

The 1st idea or teaching is that \fbox{morality is relative}, at best a personal ``value judgment.'' \textit{Relative} means that there is no absolute right or wrong in anything; instead, morality \& the rules associated with it are just a matter of personal opinion or happenstance\footnote{\textbf{happenstance} [n] [uncountable, countable] (\textit{especially North American English}) chance, especially when it results in something good.}, ``relative to'' or ``related to'' a particular framework\footnote{\textbf{framework} [n] \textbf{1.} a set of beliefs, ideas or principles that is used as the basis for examining or understand something; \textbf{2.} a system of rules, laws or agreements that controls the way that something works in business, politics or society.}, such as one's ethnicity\footnote{\textbf{ethnicity} [n] (plural \textbf{ethnicities}) [uncountable, countable] the fact or state of belonging to a social group that has a shared national or cultural tradition.}, one's upbringing\footnote{\textbf{upbringing} [n] [singular, uncountable] the way in which a child is cared for \& taught how to behave while it is growing up.}, or the culture or historical\footnote{\textbf{historical} [a] [usually before noun] \textbf{1.} connected with the past; \textbf{2.} connected with the study of history; \textbf{3.} (of a book or film) about people \& events in the past.} moment one is born into. It's nothing but an accident\footnote{\textbf{accident} [n] \textbf{1.} [countable] an unpleasant event, especially in a vehicle, that happens unexpected \& causes injury or damage; \textbf{2.} [countable, uncountable] something that happens by chance; \textbf{by accident} [idiom] in a way that is not planned or organized, \textsc{opposite}: \textbf{deliberately, on purpose}.} of birth. According to this argument\footnote{\textbf{argument} [n] \textbf{1.} [countable, uncountable] a reason or set of reasons that somebody uses to show that something is true or correct; \textbf{2.} [countable, uncountable] \textbf{argument (with somebody)} (about\texttt{/}over something) a discussion in which 2 or more people disagree; \textbf{3.} [countable] \textbf{argument of a function} (\textit{mathematics}) any of the independent variables that the value of a function depends on; \textbf{4.} [countable] (\textit{mathematics}) the angle formed by the line between a complex number \& the origin, \& the real, positive axis; \textbf{for the sake of argument} [idiom] for the purpose of having a discussion.} (now a creed\footnote{\textbf{creed} [n] [countable, uncountable] a set of principles or religious beliefs; a statement of these principles or beliefs.}), history teaches that religions, tribes\footnote{\textbf{tribe} \textbf{1.} a social group in a traditional society consisting of families or communities with the same culture, language, religion, etc. \& usually with a particular leader; \textbf{2.} (\textit{biology}) a group of related animals or plants that is larger than a genus \& smaller than a family.}, nations\footnote{\textbf{nation} [n] \textbf{1.} [countable] a country considered as a group of people with the same language, culture \& history, who live in a particular area under 1 government; \textbf{2.} [singular] all the people in a country, \textsc{synonym}: \textbf{population}.} \& ethnic\footnote{\textbf{ethnic} [a] connected with or belonging to a race or people that shares a cultural tradition.} groups tend to disagree\footnote{\textbf{disagree} [v] \textbf{1.} [intransitive] to have or express a different opinion from somebody else, \textsc{opposite}: \textbf{agree}; \textbf{2.} [intransitive] (of 2 results or reports) to give different information about the same thing, \textsc{synonym}: \textbf{conflict}, \textsc{opposite}: \textbf{agree}; \textbf{disagree with something} [phrasal verb] to disapprove of something.} about fundamental\footnote{\textbf{fundamental} [a] \textbf{1.} serious \& very important; affecting the most central \& important parts of something, \textsc{synonym}: \textbf{basic}; \textbf{2.} forming the necessary basis of something, \textsc{synonym}: \textbf{essential}.} matters, \& always have. Today, the postmodernist\footnote{\textbf{postmodernist} [a] [usually before noun] in the style of postmodernism; [n] an artist, architect or writer who works in the style of postmodernism.}\,\footnote{\textbf{postmodernism} [n] [uncountable] an attitude or approach to something, such as a particular subject, that is a reaction against the accepted modern way of thinking about it. \textbf{Postmodernism} has influenced many fields including art, architecture, literature \& cultural \& social studies. A \textbf{postmodernist} aesthetic deliberately mixes features from traditional \& modern styles \& different artistic media; it tends to show a distrust of general theories \& encourage critical engagement with a particular subject.} left makes the additional claim that 1 group's morality is \textit{nothing but} its attempt to exercise power over another group. So, the decent\footnote{\textbf{decent} [a] \textbf{1.} of a good enough standard or quality; \textbf{2.} (of people or behavior) honest \& fair; treating people with respect; \textbf{3.} acceptable according to the moral or social rules of a particular group.} thing to do -- once it becomes apparent\footnote{\textbf{apparent} [a] \textbf{1.} [not usually before noun] easy to see or understand, \textsc{synonym}: \textbf{obvious}; \textbf{2.} [usually before noun] that seems to be true, but may not be so, \textsc{synonym}: seeming.} how arbitrary\footnote{\textbf{arbitrary} [a] \textbf{1.} (of a decision, rule, system, etc.) not seeming to be based on reason, \& sometimes seeming unfair; \textbf{2.} using power or authority without restriction \& without considering other people; \textbf{3.} (\textit{mathematics}) (of a quantity) of a value that is not stated.} your, \& your society's, ``moral values'' are -- is to show tolerance\footnote{\textbf{tolerance} [n] \textbf{1.} [uncountable] willingness to accept or tolerate somebody\texttt{/}something, especially opinions or behavior that you may not agree with, or people who are not like you; \textbf{2.} [countable, uncountable] the ability to suffer something, especially pain, difficult conditions, etc. without being harmed; \textbf{3.} [countable, uncountable] \textbf{tolerance (of something)} (\textit{specialist}) the amount by which the measurements of a manufactured object may be allowed to vary without causing problems.} for people who think differently\footnote{\textbf{differently} [adv] \textbf{1.} in various different ways; \textbf{2.} in a different way from somebody\texttt{/}something else; \textbf{put\texttt{/}stated differently} [idiom] in other words; used to introduced an explanation of something.}, \& who come from different (diverse\footnote{\textbf{diverse} [a] very different from each other; containing people or things of various kinds.}) backgrounds\footnote{\textbf{background} [n] \textbf{1.} [countable, uncountable] the details of a person's family, education \& experience; \textbf{2.} [countable, usually singular, uncountable] the present circumstances or past events that help to explain an event or situation; information about these; \textbf{3.} [singular] a position in which people are not paying attention to somebody\texttt{/}something or not as much attention as they are paying to somebody\texttt{/}something else; \textbf{4.} [countable, usually singular, uncountable] the part of a picture, photograph or view behind the main objects or people.}. The emphasis on tolerance is so paramount\footnote{\textbf{paramount} [a] more important than anything else.} that for many people 1 of the worst character flaws\footnote{\textbf{flaw} [n] \textbf{1.} \textbf{flaw (in something)} a mistake or weakness in something that means that it is not correct or does not work correctly, \textsc{synonym}: \textbf{defect, fault}; \textbf{2.} \textbf{flaw (in\texttt{/}of somebody\texttt{/}something)} a weakness in somebody's character.} a person can have is to be ``judgmental\footnote{\textbf{judgemental} [a] (also \textbf{judgmental} \textit{North American English, British English}) \textbf{1.} (\textit{disapproving}) judging people \& criticizing them too quickly; \textbf{2.} connected with the process of judging things.}.''\footnote{``The yin\texttt{/}yang symbol is the 2nd part of the more comprehensive 5-part \textit{tajitu}, a diagram representing both the original absolute unity \& its division into the multiplicity of the observed world. This is discussed in more detail in Rule 2, below, as well as elsewhere in the book.''}\,\footnote{\textbf{comprehensive} [a] \textbf{1.} including all, or almost all, the items or information that may be concerned, \textsc{synonym}: \textbf{complete, full}; \textbf{2.} (\textit{British English}) (of education) designed for students of all abilities in the same school.} \&, since we don't know right from wrong, or what is good, just about \textit{the most inappropriate\footnote{\textbf{inappropriate} [a] not suitable or appropriate in a particular situation.} thing an adult can do is give a young person advice} about \fbox{how to live}.

\& so a generation \footnote{\textbf{generation} [n] \textbf{1.} [countable $+$ singular or plural verb] all the people who were born at about the same time; \textbf{2.} [countable] the average time in which children grow up, become adults \& have children of their own (usually considered to be about 30 years); \textbf{3.} [countable, uncountable] a single stage in the history of a family, a \textbf{1st-generation} American, etc. is a person whose family has lived in America, etc. for 1 generation. A \textbf{2nd-generation} American, etc. is a person whose family has lived in America, etc. for 2 generations.} has been raised\footnote{\textbf{raise} [v] \textbf{1.} \textbf{raise something} to mention something for people to discuss or somebody to deal with; \textbf{2.} \textbf{raise something} to cause or produce a feeling or reaction; to make a problem appear; \textbf{3.} to increase the amount or level of something, \textsc{opposite}: \textbf{lower}; \textbf{4.} \textbf{raise something} to collect or bring money or people together; \textbf{5.} to care for a child or young animal until it is able to take care of itself; \textbf{6.} \textbf{raise something} to breed particular animals; to grow particular crops; \textbf{7.} \textbf{raise something} to lift or move something to a higher level, \textsc{opposite}: \textbf{lower}; \textbf{8.} \textbf{raise somebody (from something)} to make somebody who has died come to life again; \textbf{raise your voice (about\texttt{/}against something)} [idiom] to clearly express your opinion about something; [n] (\textit{North American English}) $=$ \textbf{rise}.}\,\footnote{\textbf{rise} [n] \textbf{1.} [countable] an increase in an amount, a number or a level. Note that you use a \textbf{rise in something} to talk about the thing that rises, \& a \textbf{rise of something} to talk about how large or small the rise is, \textsc{opposite}: \textbf{fall}; \textbf{2.} [singular] \textbf{rise (of somebody\texttt{/}something)} the process of becoming more important, successful or powerful; \textbf{3.} [countable] (\textit{British English}) (\textit{North American English} \textbf{raise}) an increase in the money you are paid for the work you do; \textbf{4.} [singular] an upward movement; \textbf{give rise to something} [idiom] to cause something to happen or exist; [v] \textbf{1.} [intransitive] to increase in amount or number; \textbf{2.} [intransitive] to come or go upwards; to reach a higher level or position; \textbf{3.} [intransitive] to become more successful, important or powerful; \textbf{4.} [intransitive] to begin to fight against a ruler, government or army that controls you; \textbf{5.} [intransitive] (of the sun or moon) to appear above the horizon, \textsc{opposite}: \textbf{set}; \textbf{6.} [intransitive] \textbf{$+$ adv.\texttt{/}prep.} (of land or mountains) to slope upwards from or be visible above the surroundings; \textbf{rise to the challenge (of something)} [idiom] to be successful in dealing with a new or difficult task or situation; \textbf{rise to something} [phrasal verb] to show that you are able to deal with an unexpected situation or problem.} untutored in what was once called, aptly\footnote{\textbf{aptly} [adv] in a way that is suitable or appropriate in the circumstances.}, ``practical wisdom,'' which guided previous generations. Millennials, often told they have received the finest\footnote{\textbf{fine} [a] (\textbf{finer, finest}) \textbf{1.} [usually before noun] difficult to see or describe, \textsc{synonym}: \textbf{subtle}; \textbf{2.} very small; \textbf{3.} made of very small grains, \textsc{opposite}: \textbf{coarse}; \textbf{4.} very thing or narrow; \textbf{5.} [usually before noun] of high quality; good; \textbf{6.} (\textit{especially British English}) (of weather) bright \& not raining; \textbf{7.} [usually before noun] pleasing to look at; \textbf{8.} [usually before noun] attractive \& delicate; \textbf{9.} sounding important \& impressive but unlikely to have any effect; [n] a sum of money that must be paid as an official punishment for breaking a law or rule; [v] [often passive] to make somebody pay money as an official punishment for breaking a law or rule.} education available anywhere, have actually suffered a form of serious intellectual\footnote{\textbf{intellectual} [a] [usually before noun] connected with or using a person's ability to think in a logical way \& understand things, \textsc{synonym}: \textbf{mental}; [n] a person who is well educated \& enjoys activities in which they have to think seriously about things.} \& more neglect\footnote{\textbf{neglect} [v] \textbf{1.} \textbf{neglect somebody\texttt{/}something} to fail to take care of somebody\texttt{/}something; \textbf{2.} \textbf{neglect something} to not give enough attention to something; \textbf{3.} \textbf{neglect something} to ignore something because it is not important, especially in a scientific experiment, \textsc{synonym}: \textbf{disregard}; \textbf{4.} \textbf{neglect to do something} to fail or forget to do something that you ought to do, \textsc{synonym}: \textbf{omit}; [n] [uncountable] the fact of not giving enough care or attention to somebody\texttt{/}something; the state of not receiving enough care or attention.}. The relativists\footnote{\textbf{relativist} [n] (\textit{formal}) a person who believes in relativism ($=$ the belief that truth \& right \& wrong cannot be judged generally, but can be judged only in relation to other things, such as your personal situation); [a] (\textit{formal}) supporting or connected with relativism ($=$ the belief that truth \& right \& wrong cannot be judged generally, but only in relation to other things, such as your personal situation).} of my generation \& Jordan's, many of whom became their professors, chose to devalue\footnote{\textbf{devalue} [v] \textbf{1.} [transitive, intransitive] \textbf{devalue (something) (against something)} to reduce the official value of the money of 1 country when it is exchanged for the money of another country; \textbf{2.} [transitive] \textbf{devalue somebody\texttt{/}something} to give less or not enough value or importance to somebody\texttt{/}something.} thousands of years of human knowledge about how to acquire\footnote{\textbf{acquire} [v] \textbf{1.} \textbf{acquire something} to learn or develop a skill, habit or quality; \textbf{2.} \textbf{acquire something} to obtain something by buying or being given it; \textbf{3.} \textbf{acquire something} to come to have a particular reputation.} virtue, dismissing\footnote{\textbf{dismiss} [v] \textbf{1.} to officially remove somebody from their job, especially because of bad work or bad behavior, \textsc{synonym}: \textbf{fire}; \textbf{2.} to decide that somebody\texttt{/}something is not important \& not worth thinking or talking about; \textbf{3.} \textbf{dismiss something} to put thoughts or feelings out of your mind; \textbf{4.} \textbf{dismiss something} (\textit{law}) to say that a trial or legal case should not continue, often because there is not enough evidence.} it as pass\'e, ``not relevant'' or even ``oppressive\footnote{\textbf{oppressive} [a] treating people in a cruel \& unfair way \& not giving them the same freedom, rights, etc. as other people.}.'' They were so successful at it that the very word ``virtue'' sounds out of date, \& someone using it appears anachronistically\footnote{\textbf{anachronistic} [a] \textbf{1.} used to describe a person, a custom or an idea that seems old-fashioned \& does not belong to the present; \textbf{2.} used to describe something that is placed, e.g. in a book or play, in the wrong period of history.} moralistic\footnote{\textbf{moralistic} [a] (\textit{usually disapproving}) having or showing very fixed ideas about what is right \& wrong, especially when this causes you to judge other people's behavior.} \& self-righteous\footnote{\textbf{self-righteous} [a] (\textit{disapproving}) feeling or behaving as if what you say or do is always morally right, \& other people are wrong, \textsc{synonym}: \textbf{sanctimonious}.}.

\fbox{The study of virtue is not quite the same as the study of morals} (right \& wrong, good \& evil). Aristotle defined the virtues simply as the ways of behaving that are most conductive\footnote{\textbf{conductive} [a] (\textit{physics}) able to conduct electricity, heat, etc.} to happiness\footnote{\textbf{happiness} [n] [uncountable] the quality or state of being happy.} in life. Vice\footnote{\textbf{vice} [n] \textbf{1.} [uncountable] criminal activities that involve sex or drugs; \textbf{2.} [uncountable, countable] behavior that is evil or immortal; a quality in somebody's character that is evil or immoral; \textbf{3.} (\textit{especially British English}) (\textit{North American English usually} \textbf{vise}) [countable] a tool with 2 mental blocks that can be moved together by turning a screw. The vice is used to hold an object in place while work is done on it.} was defined as the ways of behaving least conductive to happiness. He observed that the virtues always aim\footnote{\textbf{aim} [n] the purpose of doing something; what somebody is trying to achieve; \textbf{take aim at somebody\texttt{/}something} to direct your criticism at somebody\texttt{/}something; [v] \textbf{1.} [transitive] \textbf{be aimed at (doing) something} to have the intention of achieving something; \textbf{2.} [intransitive, transitive] to try or plan to achieve something; \textbf{3.} [transitive, usually passive] \textbf{aim something at somebody} to say or do something that is intended to influence or affect a particular person or group.} for balance\footnote{\textbf{balance} [n] \textbf{1.} [singular, uncountable] a situation in which all parts exist in equal or appropriate amounts; \textbf{2.} [countable, usually singular] the amount of money in a bank account; the amount of a bill that remains after part has been paid; \textbf{3.} [uncountable] the ability to keep steady with an equal amount of weight on each side of the body; [v] \textbf{1.} [transitive, often passive, intransitive] to be equal in importance or amount to something else that has the opposite effect, \textsc{synonym}: \textbf{offset}; \textbf{2.} [transitive] \textbf{balance A with\texttt{/}\& B} to give equal importance to 2 different things or parts of something; \textbf{3.} [transitive, often passive] \textbf{balance A against B} to compare the importance of 2 different things; \textbf{4.} [transitive] \textbf{balance something} (\textit{finance}) to show or make sure that in an account the total money spent is equal to the total money received; \textbf{5.} [intransitive, transitive] \textbf{balance (something) (on something)} to put your body or something else into a position where it is steady \& does not fall.} \& avoid the extremes\footnote{\textbf{extreme} [a] \textbf{1.} not ordinary or usual; serious or severe, \textsc{synonym}: \textbf{exceptional}; \textbf{2.} [usually before noun] very great in degree; \textbf{3.} (of people, political organizations, opinions, etc.) far from what most people consider to be normal, reasonable or acceptable, \textsc{opposite}: \textbf{moderate}; \textbf{4.} [only before noun] as far as possible from the center, the beginning or in the direction mentioned, \textsc{synonym}: \textbf{far}; [n] \textbf{1.} a feeling, situation, way of behaving, etc. that is as different as possible from another or is opposite to it; \textbf{2.} the greatest or highest degree of something.} of the vices. Aristotle studied the virtues \& the vices in his \textit{Nicomachean Ethics}. It was a book based on experience \& observation, not conjecture\footnote{\textbf{conjecture} [n] (\textit{formal}) \textbf{1.} [countable] an opinion or idea that is not based on defiite knowledge \& is formed by guessing, \textsc{synonym}: \textbf{guess}; \textbf{2.} [uncountable] the act of forming an opinion or idea that is not based on definite knowledge; [v] [intransitive, transitive] (\textit{formal}) to form an opinion about something even though you do not have much information on it, \textsc{synonym}: \textbf{guess}.}, about the kind of happiness that was possible for human beings. Cultivating\footnote{\textbf{cultivate} [v] \textbf{1.} \textbf{cultivate something} to prepare \& use land for growing plants or crops; \textbf{2.} \textbf{cultivate something} to grow plants or crops, \textsc{synonym}: \textbf{grow}; \textbf{3.} \textbf{cultivate something} (\textit{biology}) to grow or keep living cells, etc. in grow; \textbf{4.} \textbf{cultivate somebody\texttt{/}something} (\textit{sometimes disapproving}) to try to get somebody's friendship or support, often because you want something in return; \textbf{5.} \textbf{cultivate something} to develop an attitude, a way of talking or behaving, etc.} \textit{judgment} about the difference between virtue \& vice is the beginning of wisdom, something that can \fbox{never be out of date}. 

By contrast, our modern relativism\footnote{\textbf{relativism} [n] [uncountable] the belief that truth is not always \& generally valid, but can be judged only in relation to other things, such as your personal situation.} begins by asserting\footnote{\textbf{assert} [v] \textbf{1.} to state clearly \& firmly that something is true; \textbf{2.} to make other people recognize your right or authority to do something, by behaving firmly \& confidently; \textbf{3.} \textbf{assert yourself (as something)} to behave in a confident \& determined way so that other people pay attention to your opinions; \textbf{4.} \textbf{assert itself} to start to have an effect.} that making judgments about how to live is impossible, because there is no \textit{real} good, \& no \textit{true} virtue (as these too are relative). Thus relativism's closest approximation to ``virtue'' is ``tolerance.'' Only tolerance will provide social cohesion\footnote{\textbf{cohesion} [n] [uncountable] \textbf{1.} the act of state of keeping together, \textsc{synonym}: \textbf{unity}; \textbf{2.} (\textit{physics, chemistry}) the force causing molecules of the same substance to stick together.} between different groups, \& save us from harming each other. On Facebook \& other forms of social media, therefore, you signal\footnote{\textbf{signal} [n] \textbf{1.} a series of electrical waves that carry sounds, pictures or messages, e.g. to a radio, television or mobile phone; \textbf{2.} an event, action or fact that shows that something exists or is likely to happen, \textsc{synonym}: \textbf{indication}; \textbf{3.} a movement or sound that you make to give somebody information, instructions or a warning, \textsc{synonym}: \textbf{sign}; \textbf{4.} a piece of equipment that uses different colored lights to tell drivers to go slower, stop, etc., used especially on railways \& roads; [v] \textbf{1.} [transitive] to be a sign that something exists or is likely to happen, \textsc{synonym}: \textbf{indicate}; \textbf{2.} [transitive] to show something such as a feeling or opinion through your actions or attitude; \textbf{3.} [intransitive, transitive] to make a movement or sound to give somebody a message, an instruction or a warning.} your so-called virtue, telling everyone how tolerant\footnote{\textbf{tolerant} [a] \textbf{1.} able to accept what other people say or do even if you do not agree with it; \textbf{2.} (of plants, animals, machines or systems) able to survive or operate in difficult conditions.}, open \& compassionate\footnote{\textbf{compassionate} [a] feeling or showing sympathy for people or animals who are suffering.} you are, \& wait for likes to accumulate\footnote{\textbf{accumulate} [v] \textbf{1.} [transitive] \textbf{accumulate something} to gradually get more \& more of something over a period of time; \textbf{2.} [intransitive] to gradually increase in number or quantity over a period of time, \textsc{synonym}: \textbf{build up}.}. (Leave aside that telling people you're virtuous\footnote{\textbf{virtuous} [a] behaving in a very good \& moral way.} isn't a virtue, it's self-promotion\footnote{\textbf{self-promotion} [n] [uncountable] (\textit{disapproving}) the activity of making people notice you \& your abilities, especially in a way that annoys other people.}. Virtue signaling is not virtue. Virtue signaling is, quite possibly, our commonest\footnote{\textbf{common} [a] (\textbf{commoner, commonest}) (\textbf{more common} \& \textbf{most common} are more frequent) \textbf{1.} happening often; existing in large numbers or in many places, \textsc{opposite}: \textbf{rare, uncommon}; \textbf{2.} [usually before noun] shared by or belonging to 2 or more people, groups of things, or by the people or things in a group; \textbf{3.} [only before noun] not unusual or special, \textsc{synonym}: \textbf{ordinary}; [n].} vice.)

Intolerance\footnote{\textbf{intolerance} [n] [uncountable, countable] \textbf{1.} (\textit{disapproving}) the fact of not being willing to accept ideas or ways of behaving that are different from your own; \textbf{2.} (\textit{specialist}) the fact of not being able to eat particular foods, use particular medicines, etc. without becoming ill, \textsc{opposite}: \textbf{tolerance}.} of others' views (no matter how ignorant\footnote{\textbf{ignorant} [a] lacking knowledge or information about something; not educated.} or incoherent\footnote{\textbf{incoherent} [a] \textbf{1.} not logical or well organized, \textsc{opposite}: \textbf{coherent}; \textbf{2.} (of spoken or written language) not clear \& hard to understand; \textsc{opposite}: \textbf{coherent}; \textbf{3.} (\textit{physics}) (of waves) not in phase with each other, \textsc{opposite}: \textbf{coherent}.} they may be) is not simply wrong; in a world where there is no right or wrong, it is worse: it is a sign you are embarrassingly\footnote{\textbf{embarrassingly} [adv] \textbf{1.} in a way that makes you feel shy, uncomfortable or ashamed; \textbf{2.} in a way that makes somebody\texttt{/}something look bad, stupid, dishonest, etc.} unsophisticated\footnote{\textbf{unsophisticated} [a] \textbf{1.} not having or showing much experience of the world \& social situations; \textbf{2.} simple \& basic; not complicated, \textsc{synonym}: \textbf{crude}, \textsc{opposite}: \textbf{sophisticated}.} or, possibly, dangerous\footnote{\textbf{dangerous} [a] likely to injure, harm or kill somebody, or to damage or destroy something.}.

But it turns out that many people cannot tolerate the vacuum\footnote{\textbf{vacuum} [n] \textbf{1.} a space that is completely empty of all substances, including all air or other gas; \textbf{2.} [usually singular] a situation in which somebody\texttt{/}something is missing or lacking; \textbf{in a vacuum} [idiom] existing separately from other people, events, etc. when there should be a connection.} -- the chaos -- which is inherent\footnote{\textbf{inherent} [a] that is a permanent, basic or typical feature somebody\texttt{/}something, \textsc{synonym}: \textbf{intrinsic}.} in life, but made worse by this moral relativism; they cannot live without a moral compass, without an ideal at which to aim in their lives. (For relativists, ideals\footnote{\textbf{ideal} [a] \textbf{1.} perfect; most suitable; \textbf{2.} [only before noun] the best that can be imagined, but not likely to become real; \textbf{in an ideal\texttt{/}a perfect world} [idiom] used to say that something is what you would like to happen or what should happen, but you know it cannot; [n] \textbf{1.} \textbf{ideal (of somebody\texttt{/}something)} an idea or a standard that seems perfect \& worth trying to achieve; \textbf{2.} [usually singular] \textbf{ideal (of something)} a person or thing considered as perfect.} are values too, \& like all values, they are merely\footnote{\textbf{merely} [adv] used meaning `only' or `simply' to emphasize a fact or something that you are saying.} ``relative'' \& hardly\footnote{\textbf{hardly} [adv] \textbf{1.} used to suggest that something is not likely or not reasonable; \textbf{2.} almost no; almost not; almost none; \textbf{3.} used especially after `can' or `could' \& before the main verb, to emphasize that it is difficult to do something.} worth sacrificing for.) So, right alongside\footnote{\textbf{alongside} [prep] \textbf{1.} next to or at the side of something; \textbf{2.} together with something\texttt{/}somebody; at the same time as something\texttt{/}somebody.} relativism, we find the spread\footnote{\textbf{spread} [v] \textbf{1.} [intransitive, transitive] to affect or be known or used by more \& more people; to make something do this; \textbf{2.} [intransitive, transitive] to be in a number of different places; to cause something to be in a number of different places; \textbf{3.} [intransitive, transitive] to cover a larger \& larger area; to make something cover a larger \& larger area; \textbf{4.} [transitive] to separate something into parts \& divide them between different times or different people; \textbf{5.} [transitive] to distribute something in a particular way; \textbf{6.} [transitive] \textbf{spread something (out)} to open something that has been folded so that it covers a larger area than before; \textbf{7.} [transitive] to put a layer of a substance onto the surface of something; [n] \textbf{1.} [uncountable] \textbf{spread (of something)} an increase in the amount or number of something that there is, or in the area that is affected by something; \textbf{2.} [countable, usually singular] a range or variety of people or things; \textbf{3.} [uncountable] \textbf{spread (of something)} the area that something exists in or happens in; \textbf{4.} [countable] \textbf{spread (between A \& B)} (\textit{finance}) the difference between 2 rates or prices.} of nihilism\footnote{\textbf{nihilism} [n] [uncountable] (\textit{philosophy}) the belief that lief has no meaning or purpose \& that religious \& moral principles have no value.} \& despair\footnote{\textbf{despair} [n] [uncountable] the feeling of having lost all hope; [v] [intransitive] to stop having any hope that a situation will change or improve.}, \& also the opposite of moral relativism: the blind\footnote{\textbf{blind} [a] (\textbf{blinder, blindest}) \textbf{1.} not able to see; \textbf{2.} (\textbf{the blind}) [n] [plural] people who are blind; \textbf{3.} \textbf{blind to something} not noticing or realizing something; \textbf{4.} [usually before noun] (of strong feelings) seeming to be unreasonable, \& accepted without question; \textbf{5.} [usually before noun] (of a situation or an event) that cannot be controlled by reason; \textbf{6.} (of a test or experiment) in which the people taking the test do not know, e.g., which drug, substance, etc., they have been given. A \textbf{double-blind} test is one in which neither the participants nor the researchers know which drug, substance, etc. each participant has been given; [v] \textbf{1.} [often passive] \textbf{blind somebody} to make somebody unable to see, permanently or for a short time; \textbf{2.} to make somebody no longer able to think clearly or behave in a sensible way; \textbf{3.} \textbf{blind somebody\texttt{/}something} to make somebody who is taking part in an experiment or interview unaware of what is being tested or measured.} certainty\footnote{\textbf{certainty} [n] (plural \textbf{certainties}) \textbf{1.} [uncountable] the strong belief that something is true; \textbf{2.} [countable] something that you know is completely true or reliable; an event that is definitely going to happen; \textbf{3.} [uncountable] the quality of being definitely true or reliable.} offered by ideologies that claim to have an answer for everything.

\& so we arrive at the 2nd teaching that millennials have been bombarded\footnote{\textbf{bombard} [v] \textbf{1.} \textbf{bombard somebody\texttt{/}something (with something)} to attack a place by firing large guns at it or dropping bombs on it continuously; \textbf{2.} \textbf{bombard somebody\texttt{/}something (with something)} to attack somebody with a lot of questions, criticisms, etc. or by giving them too much information.} with. They sign up for a humanities\footnote{\textbf{humanity} [n] \textbf{1.} [uncountable] people in general; \textbf{2.} [uncountable] \textbf{humanity (of somebody)} the state of being a person rather than a god, an animal or a machine; \textbf{3.} [uncountable] the quality of being kind to people \& animals by making sure that they do not suffer more than is necessary; the quality of being humane; \textbf{4.} \textbf{((the) humanities)} [plural] the subject of study that are connected with human culture, especially literature, history, art, music \& philosophy.} course, to study greatest books ever written. But they're not assigned the books; instead they are given ideological\footnote{\textbf{ideological} [a] (\textit{sometimes disapproving}) connected with an ideology.} attacks on them, based on some appalling\footnote{\textbf{appalling} [a] \textbf{1.} (\textit{North American English, formal or British English}) extremely bad, especially from a moral point of view, \textsc{synonym}: \textbf{shocking}; \textbf{2.} (\textit{informal}) very bad; of very poor quality.} simplification\footnote{\textbf{simplification} [n] \textbf{1.} [uncountable] \textbf{simplification (of something)} the process of making something less complicated, or easier to do or understand; \textbf{2.} [countable] a change that makes a problem, statement, system, etc. less complicated or easier to understand or do}. Where the relativist is filled with uncertainty, the ideologue is the very opposite. He or she is hyper-judgmental\footnote{\textbf{judgemental} [a] (\textit{especially British English}) (also \textbf{judgmental} \textit{especially in North American English}) \textbf{1.} (\textit{disapproving}) judging people \& criticizing them too quickly; \textbf{2.} (\textit{formal}) connected with the process of judging things.} \& censorious\footnote{\textbf{censorious} [a] (\textit{formal}) tending to criticize people or things a lot, \textsc{synonym}: \textbf{critical}.}, always know what's wrong about others, \& what to do about it. Sometimes it seems the only people willing to give advice in a relativistic society are those with the least to offer.

\textit{Modern} moral relativism has many sources. As we in the West learned more history, we understood that different epochs\footnote{\textbf{epoch} [n] \textbf{1.} a period of time in history, especially one during which important events or changes happen, \textsc{synonym}: \textbf{era}; \textbf{2.} (\textit{earth sciences}) a length of time that is a division of a period.} had different moral\footnote{\textbf{moral} [a] \textbf{1.} [only before noun] concerned with principles of right \& wrong behavior; \textbf{2.} [only before noun] based on a sense of what is right \& fair, not on legal rights or duties, \textsc{synonym}: \textbf{ethical}; \textbf{3.} following the standards of behavior considered acceptable \& right by most people, \textsc{synonym}: \textbf{good, honorable}; \textbf{4.} [only before noun] able to understand the difference between right \& wrong; [n] \textbf{1.} (\textbf{morals}) [plural] standards or principles of good behavior, especially in matters of sexual relationships; \textbf{2.} [countable] \textbf{moral of something} a practical lesson that a story, an event or an experience teaches you.} codes\footnote{\textbf{code} [n] \textbf{1.} [countable] a series of letters, numbers or symbols that are used to identify, sort or represent something; \textbf{2.} [countable, uncountable] (often in compounds) a system of words, letters, numbers or symbols that represent a message or record information secretly; \textbf{3.} [uncountable] a word, phrase or symbol that is used to represent an idea in an indirect way; \textbf{4.} [uncountable] (\textit{computing}) a system of computer programming instructions; \textbf{5.} [countable] a set of moral principles or rules of behavior that are generally accepted by society or a social group; \textbf{6.} [countable] a system of laws or rules that state how people in an institution or a country should behave; \textbf{7.} [countable] (\textit{biology}) $=$ genetic code; [v] \textbf{1.} [transitive, often passive] \textbf{code something} to write or print words, letters, numbers, etc. on something so that you know what it is, what group it belongs to, etc.; \textbf{2.} [transitive, often passive] \textbf{code something} to put a message into code so that it can be understood by only a few people; \textbf{3.} [transitive, often passive] \textbf{code something (as something)} 9\textit{computing} to write a computer program by putting 1 system of numbers, words \& symbols into another system, \textsc{synonym}: \textbf{encode}; \textbf{4.} [transitive, usually passive] \textbf{be coded (into somebody\texttt{/}something)} (\textit{biology}) to be contained in a person's genetic code; \textbf{5.} [intransitive] \textbf{code for something} (\textit{biology}) to be the genetic code for something.}. As we traveled the seas \& explored the globe\footnote{\textbf{globe} [n] \textbf{1.} (\textbf{the globe}) [singular] the world (used especially to emphasize its size); \textbf{2.} [countable] an object shaped like a ball with a map of the world on its surface.}, we learned of far-flung\footnote{\textbf{far-flung} [a] [usually before noun] (\textit{literary}) \textbf{1.} a long distance away; \textbf{2.} spread over a wide area.} tribes\footnote{\textbf{tribe} [n] \textbf{1.} a social group in a traditional society consisting of families or communities with the same culture, language, religion, etc. \& usually with a particular leader; \textbf{2.} (\textit{biology}) a group of related animals or plants that is larger than a genus \& smaller than a family.} on different continents\footnote{\textbf{continent} [n] \textbf{1.} [countable] 1 of the 7 main continuous land masses of the earth (Africa, Asia, Australia, Antarctica, Europe \& North \& South America); \textbf{3.} (\textbf{the Continent}) [singular] (\textit{British English}) the main part of the continent of Europe, not including Britain or Ireland.} whose different moral codes made sense relative to, or within the framework of, their societies\footnote{\textbf{society} [n] (plural \textbf{societies}) \textbf{1.} [uncountable, countable] people in general, living together in communities; a particular community of people who share the same customs, laws, etc.; \textbf{2.} [countable] a group of people who join together for a particular purpose. The written abbreviation \textbf{Soc.} is used in the names of particular societies.; \textbf{3.} [uncountable] \textbf{society (of somebody)} the state of being with other people, \textsc{synonym}: \textbf{company}.}. Science played a role, too, by attacking the religious view of the world, \& thus undermining the religious grounds for ethics\footnote{\textbf{ethic} [n] \textbf{1.} (\textbf{ethics}) [plural] moral principles that control or influence a person's behavior; \textbf{2.} [singular] a system of moral principles or rules of behavior; \textbf{3.} (\textbf{ethics}) [uncountable] the branch of philosophy that deals with moral principles.} \& rules. Materialist\footnote{\textbf{materialist} [n] \textbf{1.} a person who believes that money, possessions \& physical comforts are more important than spiritual values in life; \textbf{2.} a person who believes in the philosophy of materialism.} social science implied that we could divide the world into facts (which were subjective\footnote{\textbf{subjective} [a] \textbf{1.} based on a particular person's beliefs or opinions, rather than on facts or evidence that everyone can recognize, \textsc{opposite}: \textbf{objective}; \textbf{2.} [usually before noun] (of ideas, feelings or experiences) existing in somebody's mind rather than in the real world, \textsc{opposite}: \textbf{objective}.} \& personal\footnote{\textbf{personal} [a] \textbf{1.} [only before noun] your own; not belonging to or connected with anyone else; \textbf{2.} [only before noun] connected with individual people people, especially their feelings, characters \& relationships; \textbf{3.} not connected with a person's job or official position; \textbf{4.} [only before noun] done by a particular person rather than by somebody who is acting for them; \textbf{5.} [only before noun] made or done for a particular person rather than for a large group of people or people in general; \textbf{6.} [only before noun] connected with a person's body; \textbf{7.} connected with a particular person's character, appearance or private life in a way that is offensive.}). Then we cold 1st agree on the facts, \&, maybe, 1 day, develop a scientific code of ethics (which has yet to arrive). Moreover, by implying that values had a lesser reality\footnote{\textbf{reality} [n] (plural \textbf{realities}) \textbf{1.} [uncountable] the true situation \& the problems that actually exist in the world, especially in contrast to how people would like it to be; \textbf{2.} [countable] a thing that is actually experienced or seen, in contrast to what people might imagine; \textbf{3.} [uncountable] \textbf{reality television\texttt{/}TV\texttt{/}shows\texttt{/}series\texttt{/}contestants} television\texttt{/}shows, etc. that use real people (not actors) in real situations, presented as entertainment; \textbf{in reality} [idiom] used to say that a situation is different from what has just been said or from what people believe.} than facts, science contributed in yet another way to moral relativism, for it treated ``value'' as secondary\footnote{\textbf{secondary} [a] \textbf{1.} less important than something else; \textbf{2.} happening as a result of something else; \textbf{3.} [only before noun] (of writing) based on other books, etc, not on direct research or observation; \textbf{4.} [only before noun] connected with the education of children aged around 11--18; \textbf{5.} (\textit{chemistry}) (of an organic compound) having its functional group located on a carbon atom which is bonded to 2 other carbon atoms; containing a nitrogen atom bonded to 2 carbon atoms.}. (But the idea that we can easily separate facts \& values was \& remains naive; to some extent, one's values determine what one will pay attention to, \& what will count as a fact.)

The idea that different societies had different rules \& morals was known to the ancient world too, \& it is interesting to compare its response to this realization\footnote{\textbf{realization} [n] (\textit{British English also} \textbf{realisation}) \textbf{1.} [uncountable, singular] \textbf{realization (that) $\ldots$} the process of becoming aware of something, \textsc{synonym}: \textbf{awareness}; \textbf{2.} [uncountable] \textbf{realization (of something)} the process of achieving a particular aim, etc., \textsc{synonym}: \textbf{achievement}; \textbf{3.} [uncountable, countable] \textbf{realization (of something)} (\textit{formal}) the act of producing something in an actual or physical form; the thing that is produced.} with the modern response (relativism, nihilism \& ideology). When the ancient Greeks sailed to India \& elsewhere, they too discovered that rules, morals \& customs\footnote{\textbf{customs} [n] [plural] \textbf{1.} (\textbf{Customs}) the government department that collects taxes on goods bought \& sold \& on goods bought into the country, \& that checks what is brought in. American English uses a singular verb with \textbf{customs} in this meaning.; \textbf{2.} the taxes that must be paid to the government when goods are brought in from other countries; \textbf{3.} the place at a port or an airport where your bags are checked as you come into a country.} differed from place to place, \& saw that the explanation for what was right \& wrong was often rooted\footnote{\textbf{rooted} [a] \textbf{1.} \textbf{rooted in something} developing from or being strongly influenced by something; \textbf{2.} \textbf{rooted in something} fixed in 1 place; not moving or changing.} in some ancestral\footnote{\textbf{ancestral} [a] connected with or belonging to earlier members of a family, race of people or species.} authority\footnote{\textbf{authority} [n] (plural \textbf{authorities}) \textbf{1.} [uncountable] the power to give orders to people or to say how things should be done; \textbf{2.} [uncountable] official permission or the right to do something; \textbf{3.} [countable] an organization that has the power to make decisions or that has a particular area of responsibility in a country or region; \textbf{4.} [uncountable] the power to influence people because they respect your knowledge or official position; \textbf{5.} [countable] \textbf{authority (on something)} a person with special knowledge, \textsc{synonym}: \textbf{specialist}.}. The Greek response was not despair, but a new invention\footnote{\textbf{invention} [n] \textbf{1.} [countable] something that has been created or designed that has not existed before; \textbf{2.} [uncountable] \textbf{invention of something} the act of creating or designing something that has not existed before; \textbf{3.} [countable, uncountable] the act of saying or describing something, \& pretending that is true, especially in order to deceive people; something that is said or described in thi sway; \textbf{4.} [uncountable] the ability to have new \& interesting ideas.}: philosophy.

Socrates, reacting to the uncertainty bred\footnote{\textbf{breed} [v] \textbf{1.} [intransitive] (of animals) to have sex \& produce young; \textbf{2.} [transitive] to keep animals or plants in order to produce young ones in a controlled way; \textbf{3.} [transitive] \textbf{breed something} to be the cause of something; [n] \textbf{1.} a type of animal with a particular appearance that makes it different from others of the same species \& that is the result of having been developed in a controlled way; \textbf{2.} [usually singular] a type of person.} by awareness of these conflicting moral codes, decided that instead of becoming a nihilist\footnote{\textbf{nihilist} [n] a person who believes in nihilism.}, a relativist or an ideologue\footnote{\textbf{ideologue} [n] (also \textbf{ideologist}) (\textit{formal, sometimes disapproving}) a person whose actions are influenced by belief in a set of principles ($=$ by an ideology).}, he would devote his life to the search for wisdom that could reason about these differences, i.e., he helped invent philosophy. He spent his life asking perplexing\footnote{\textbf{perflexing} [a] making you confused or worried because you do not understand something, \textsc{synonym}: \textbf{puzzling}.}, foundational questions, such as ``What is virtue?'' \& ``How can one live the good life?'' \& ``What is justice?'' \& he looked at different approaches\footnote{\textbf{approach} [n] \textbf{1.} [countable] a way of doing or thinking about something such as a problem or task; \textbf{2.} [singular] movement nearer to somebody\texttt{/}something in distance or time; \textbf{3.} [countable] \textbf{approach (to somebody\texttt{/}something)} the act of speaking to somebody about something, especially when making an offer or a request; \textbf{4.} [countable] a path, sea passage, etc. that leads to a particular place; \textbf{5.} [singular] \textbf{approach to something} a thing that is like something else that is mentioned; [v] \textbf{1.} [transitive] to start dealing with a problem or task or considering a topic or situation in a particular way; \textbf{2.} [transitive] \textbf{approach something} to come close to something in quantity or quality; \textbf{3.} [intransitive, transitive] to move near to somebody\texttt{/}something in distance or time; \textbf{4.} [transitive] to speak to somebody about something, especially to offer to do something or to ask them for something.}, asking which seemed most coherent\footnote{\textbf{coherent} [a] \textbf{1.} (of an argument, theory, statement or policy) logical \& well organized; easy to understand \& clear, \textsc{opposite}: \textbf{incoherent}; \textbf{2.} (of a person) able to talk \& express yourself clearly; showing this, \textsc{opposite}: \textbf{incoherent}; \textbf{3.} made up of different parts that fit or work well together; \textbf{4.} (\textit{physics}) (of waves) in phase with each other, \textsc{opposite}: \textbf{incoherent}.} \& most in accord\footnote{\textbf{accord} [v] (\textit{formal}) to give somebody\texttt{/}something authority, status or a particular type of treatment, \textsc{synonym}: \textbf{grant}; \textbf{accord with something} [phrasal verb] to agree with or match something; [n] a formal agreement between 2 or more organizations or countries; \textbf{in accord (with something\texttt{/}somebody)} in agreement with; \textbf{of your own accord} without being asked, forced or helped.} with human nature. These are the kinds of questions that I believe animate\footnote{\textbf{animate} [v] \textbf{1.} \textbf{animate something} to make something more lively or full of energy; \textbf{2.} [usually passive] to make models, toys, images, etc. seem to move in a film, either by rapidly showing slightly different pictures of them in a series, one after another, or by using computer techniques to create moving images; [a] (\textit{formal}) living; having life, \textsc{opposite}: \textbf{inanimate}.} this book.

For the ancients, the discovery\footnote{\textbf{discovery} [n] (plural \textbf{discoveries}) \textbf{1.} [countable, uncountable] an act or the process of finding somebody\texttt{/}something, or learning about something that was not known about before; \textbf{2.} [countable] a thing, fact or person that is found or learned about for the 1st time.} that different people have different ideas about how, practically\footnote{\textbf{practically} [adv] \textbf{1.} almost; very nearly, \textsc{synonym}: \textbf{virtually}; \textbf{2.} in a realistic or sensible way; in real situations.}, to live, did not paralyze\footnote{\textbf{paralyze} [v] (\textit{British English}) (\textit{North American English} \textbf{paralyze}) [often passive] \textbf{1.} \textbf{paralyse somebody} to make somebody unable to feel or more all part of their body; \textbf{2.} \textbf{paralyze something} to prevent something from functioning normally.} them; it deepened\footnote{\textbf{deepen} [v] \textbf{1.} [intransitive, transitive] (of a feeling or connection) to become stronger; to make a feeling or connection stronger; \textbf{2.} [intransitive, transitive] to become worse; to make something worse; \textbf{3.} [intransitive, transitive] to become greater in size; to make something greater in size; \textbf{4.} [transitive] \textbf{deepen something} to improve your knowledge or understanding of something; \textbf{5.} [intransitive, transitive] to become deeper; to make something deeper.} their understanding\footnote{\textbf{understanding} [n] \textbf{1.} [uncountable, countable, usually singular] the fact or state of knowing or realizing something, e.g. what somebody\texttt{/}something is like, how or why people do things, how something happens or why something is important; \textbf{2.} [uncountable] kindness \& sympathy, often towards somebody who has different views or who has behaved badly; \textbf{3.} [countable, usually singular] an agreement, often not written in a contract, that people will help each other or that something will happen in a particular way; \textbf{4.} [uncountable, countable] \textbf{understanding (of something) (is that $\ldots$)} the particular way in which somebody understands something.} of humanity \& led to some of the most satisfying conversations human beings have ever had, about \fbox{how life might be lived}.

Likewise\footnote{\textbf{likewise} [adv] \textbf{1.} the same; in a similar way; \textbf{2.} also.}, Aristotle. Instead of despairing about the differences in moral codes, Aristotle argued that though specific rules, laws \& customs differed from place to place, what does not differ is that in all places human beings, by their nature, have a proclivity\footnote{\textbf{proclivity} [n] (\textit{formal}) (plural \textbf{proclivities}) \textbf{proclivity (for something\texttt{/}for doing something)} a natural desire or need that makes you tend to do something, often something bad, \textsc{synonym}: \textbf{propensity}.} to make rules, laws \& customs. To put this in modern terms, it seems that all human beings are, by some kind of biological\footnote{\textbf{biological} [a] \textbf{1.} connected with the processes that take place within living things; \textbf{2.} connected with the science of biology; \textbf{3.} a child's biological parents are their natural parents, not the people who adopted him\texttt{/}her.} endowment\footnote{\textbf{endowment} [n] (\textit{formal}) \textbf{1.} [countable, uncountable] \textbf{endowment (of something)} money that is given to a school, a college or another institution to provide it with an income; the act of giving this money; \textbf{2.} [countable, usually plural] a quality or an ability that somebody is born with; \textbf{3.} [uncountable, countable] the resources that a country or an area has.}, so ineradicably\footnote{\textbf{ineradicable} [a] (\textit{formal}) (of a quality or situation) that cannot be removed or changed.} concerned with morality that we create a structure of laws \& rules wherever we are. The idea that human life can be free of moral concerns is a fantasy\footnote{\textbf{fantasy} [n] (plural \textbf{fantasies}) \textbf{1.} [countable] an idea, image or situation that a person imagines, but that is not real or is not likely to happen; \textbf{2.} [uncountable] the act of imagining things; a person's imagination.}.

We are rule generators\footnote{\textbf{generator} [n] \textbf{1.} a machine for producing electricity; \textbf{2.} (\textit{British English}) a company that produces electricity to sell to the public; \textbf{3.} a machine, an organization, etc. that produces something.}. \& given that we are moral animals, what must be the effect of our simplistic\footnote{\textbf{simplistic} [a] (\textit{disapproving}) treating complicated issues \& problems as if they were much simpler than they really are.} modern relativism upon us? It means we are hobbling\footnote{\textbf{hobble} [v] \textbf{1.} [intransitive] (\textbf{$+$ adv.\texttt{/}prep.}) to walk with difficulty, especially because your feet or legs hurt, \textsc{synonym}: \textbf{limp}; \textbf{2.} [transitive] \textbf{hobble something} to tie together 2 legs of a horse or other animal in order to stop it from running away; \textbf{3.} [transitive] \textbf{hobble something} to make it more difficult for somebody to do something or for something to happen.} ourselves by pretending to be something we are not. It is a mask, but a strange one, for it mostly deceives\footnote{\textbf{deceive} [v] [transitive] \textbf{1.} \textbf{deceive somebody} to deliberately make somebody believe something that is not true; \textbf{2.} \textbf{deceive somebody\texttt{/}something} (of a thing) to make somebody have a false idea about somebody\texttt{/}something.} the one who wears it. \textit{Scccccratccch} the most clever postmodern-relativist professor's Mercedes with a key, \& you will see how fast the mask of relativism (with its pretense\footnote{\textbf{pretence} [n] (\textit{British English}) (\textit{North American English} \textbf{pretense}) \textbf{1.} [uncountable, countable, usually singular] (\textit{formal}) a claim that you have a particular quality or skill; \textbf{2.} [uncountable, singular] \textbf{pretence (of something)} the act of behaving in a particular way, in order to make other people believe something that is not true.} that there can neither right nor wrong) \& the cloak\footnote{\textbf{cloak} [n] \textbf{1.} [countable] a type of coat that has no arms, fastens at the neck \& hangs loosely from the shoulders, worn especially in the past; \textbf{2.} [singular] (\textit{literary}) a thing that hides or covers somebody\texttt{/}something; [v] [often passive] (\textit{literary}) to cover or hide something.} of radial\footnote{\textbf{radial} [a] having a pattern of lines that go out from a central points towards the edge of a circle.} tolerance come off.

Because we do not yet have an ethics based on modern science, Jordan is not trying to develop his rules by wiping\footnote{\textbf{wipe} [v] \textbf{1.} to rub something against a surface, in order to remove dirt or liquid from it; to rub a surface with a cloth, etc. in order to clean it; \textbf{2.} to remove dirt, liquid, etc. from something by using a cloth, your hand, etc.; \textbf{3.} to remove information, sound, images, etc. from a computer, video, etc., \textsc{synonym}: \textbf{erase}; \textbf{4.} to deliberately forget an experience because it was unpleasant or embarrassing, \textsc{synonym}: \textbf{erase}.} the slate\footnote{\textbf{slate} [n] \textbf{1.} [uncountable] a type of dark grey stone that splits easily into thin flat layers; \textbf{2.} [countable] a small thin piece of slate, used for covering roofs; \textbf{3.} [countable] (\textit{North American English}) a list of the candidates in an election; \textbf{4.} [countable] a small sheet of slate in a wooden frame, used in the past in schools for children to write on; \textbf{a clean slate\texttt{/}sheet} [idiom] a record of your work or behavior that does not show any mistakes or bad things that you have done; \textbf{wipe the slate clean} [idiom] to agree to forget about past mistakes or arguments \& start again with a relationship; [v] \textbf{slate somebody\texttt{/}something (for something)} (\textit{British English}) to criticize somebody\texttt{/}something, especially in a newspaper; \textbf{2.} [usually passive] to plan that something will happen at a particular time in the future; \textbf{3.} [usually passive] (\textit{especially North American English, informal}) to suggest or choose somebody for a job, position, etc.} clean -- by dismissing\footnote{\textbf{dismiss} [v] \textbf{1.} to officially remove somebody from their job, especially because of bad work or bad behavior, \textsc{synonym}: \textbf{fire}; \textbf{2.} to decide that somebody\texttt{/}something is not important \& not worth thinking or talking about; \textbf{3.} \textbf{dismiss something} to put thoughts or feelings out of your mind; \textbf{4.} \textbf{dismiss something} (\textit{law}) to say that a trial or legal case should not continue, often because there is not enough evidence.} thousands of years of wisdom as mere\footnote{\textbf{mere} [a] [only before noun] \textbf{1.} used to say that the fact that a particular thing is present in a situation is enough to have an influence on that situation; \textbf{2.} used when you want to emphasize how small or unimportant somebody\texttt{/}something is.} superstition\footnote{\textbf{superstition} [n] [uncountable, countable] the belief that particular events happen in a way that cannot be explained by reason or science; the belief that particular events bring good or bad luck.} \& ignoring our greatest moral achievements\footnote{\textbf{achievement} [n] \textbf{1.} [countable] a thing that somebody has done successfully, especially using their own effort \& skill; \textbf{2.} [uncountable] the fact or process of achieving something; \textbf{3.} [uncountable] a child's or student's progress in a course of learning, especially as measured by standard tests.}. Far better to integrate\footnote{\textbf{integrate} [v] \textbf{1.} [transitive] to combine 2 or more things so that they work together; \textbf{2.} [intransitive, transitive] to become or make somebody become accepted as a member of a social group, especially when they come from a different culture; \textbf{3.} [transitive] \textbf{integrate something} (\textit{mathematics}) to find the integral of something.} the best of what we are now learning with the books human beings saw fit to preserve\footnote{\textbf{preserve} [v] \textbf{1.} \textbf{preserve something} to keep a particular quality or feature; \textbf{2.} to keep something safe from harm, in good condition or in its original state; \textbf{3.} to prevent something from decaying, by treating it in a particular way; [n] [singular] an activity, job or interest that is thought to be suitable for 1 particular person or group of people.} over millennia\footnote{\textbf{millennium} [n] (plural \textbf{millennia} or \textbf{millenniums}) \textbf{1.} a period of 1000 years, especially as calculated before or after the birth of Christ; \textbf{2.} (\textbf{the millennium}) the time when 1 period of 1000 years ends \& another begins.}, \& with the stories that have survived, against all odds, time's tendency\footnote{\textbf{tendency} [n] (plural \textbf{tendencies}) \textbf{1.} [countable] if somebody\texttt{/}something has a particular tendency, they are likely to behave or act in a particular way; \textbf{2.} [countable] a new custom that is starting to develop, \textsc{synonym}: \textbf{trend}; \textbf{3.} [countable $+$ singular or plural verb] (\textit{British English}) a group within a larger political group, whose views are more extreme than those of the rest of the group.} to obliterate\footnote{\textbf{obliterate} [v] [often passive] \textbf{obliterate something} to remove all signs of something, either by destroying or covering it completely.}.

He is doing what reasonable guides have always done: he makes no claim that human wisdom begins with himself, but, rather, turns 1st to his own guides. \& although the topics in this book are serious, Jordan often has great fun addressing them with a light touch, as the chapter headings convey. He makes no claim to be exhaustive\footnote{\textbf{exhaustive} [a] including everything possible; very thorough or complete.}, \& sometimes the chapters consist of wide-ranging\footnote{\textbf{wide-ranging} [a] including or dealing with a large number of different subjects or areas.} discussions of our psychology as he understands it.

So why not call this book of ``guidelines,'' a far more relaxed\footnote{\textbf{relaxed} [a] \textbf{1.} (of a person) calm \& not anxious or worried; \textbf{2.} \textbf{relaxed (about something)} not caring too much about making people follow rules; \textbf{3.} (of a place or situation) calm \& informal.}, user-friendly\footnote{\textbf{user-friendly} [a] easy for people who are not experts to use or understand.} \& less rigid\footnote{\textbf{rigid} [a] \textbf{1.} (of an object or substance) stiff \& difficult to move or bend; \textbf{2.} (of rules, methods, etc.) very strict \& difficult to change or adapt, \textsc{synonym}: \textbf{inflexible}; \textbf{3.} \textbf{rigid (about something\texttt{/}doing something)} (of a person or organization) not willing to change or adapt ideas or behavior, \textsc{synonym}: \textbf{inflexible}.} sounding\footnote{\textbf{sound} [n] \textbf{1.} [countable] something that can be heard; \textbf{2.} [uncountable] continuous movements (called vibrations) that travel through air or water \& can be heard when they reach a person's or an animal's ear; \textbf{3.} [uncountable] what you can hear coming from a television, radio, etc., or as part of a film; [v] (not usually used in the progressive tenses) \textbf{1.} \textit{linking verb} to give a particular impression when heard or read about. In spoken English, people often use \textbf{like} instead of \textbf{as if} or \textbf{as though} in this meaning. This is not correct in academic English. \textbf{Like} can be used before a noun phrase (\textit{an approaching vehicle}) but not before a clause.; \textbf{2.} (\textbf{-sounding}) (in adjectives) giving the impression of being something; \textbf{3.} [intransitive, transitive] to give a signal such as warning by making a sound; \textbf{4.} [transitive] to express a particular opinion about a situation or idea; [a] (\textbf{sounder, soundest}) \textbf{1.} sensible; that can be relied on \& that will probably give good results; \textbf{2.} in good condition; not damaged or hurt; \textbf{3.} [only before noun] good \& thorough.} term than ``rules''?

Because these really are rules. \& the foremost\footnote{\textbf{foremost} [a] the most important or famous; in a position at the front; [adv] more than anything else.} rule is that \fbox{you must take responsibility for your own life}. Period\footnote{\textbf{period} [n] \textbf{1.} a particular length of time; \textbf{2.} a length of time in the life of a particular person, the history of a particular country, etc.; \textbf{3.} (\textit{earth sciences}) a length of time that is a division of an era. A period is divided into epochs.; \textbf{4.} \textbf{period (of something)} (\textit{physics}) the length of time it takes to reach the same point in a cycle each time; \textbf{5.} \textbf{period (of something)} any of the parts that a day is divided into at a school or college for a lesson or other activity; \textbf{6.} (\textit{chemistry}) a set of elements that occupy a horizontal row in the periodic table; \textbf{7.} the flow of blood each month from the body of a woman who is not pregnant; \textbf{8.} (\textit{North American English}) $=$ \textbf{full stop}.}.

One might think that a generation that has heard endlessly\footnote{\textbf{endlessly} [adv] in a way that continues for a long time \& seems to have no end.}, from their more ideological teachers, about the rights, rights, rights that belong to them, would object to being told that they would do better to focus instead on taking responsibility. Yet this generation, many of whom were raised in small families by hyper-protective\footnote{\textbf{protective} [a] \textbf{1.} [only before noun] providing or intended to provide protection; \textbf{2.} \textbf{protective (of somebody\texttt{/}something)} having or showing a wish to protect somebody\texttt{/}something; \textbf{3.} intended to give an advantage to your own country's industry.} parents, on soft-surface playgrounds, \&  then taught in universities with ``safe spaces'' where they don't have to hear things they don't want to -- schooled to be risk-averse\footnote{\textbf{risk-averse} [a] not willing to do something if it is possible that something bad could happen as a result.} -- has among it, now, millions who feel stultified\footnote{\textbf{stultify} [v] (\textit{formal}) \textbf{stultify somebody\texttt{/}something} to make somebody feel very bored \& unable to think of new ideas.} by this underestimation\footnote{\textbf{underestimate} [v] \textbf{1.} to think or guess that the amount, cost, size or importance of something is smaller or less than it really is, \textsc{opposite}: \textbf{overestimate}; \textbf{2.} \textbf{underestimate somebody\texttt{/}something} to not realize how good, strong, determined, etc., \textsc{opposite}: \textbf{overestimate}; \textbf{underestimate} [n]; \textbf{underestimation} [n] \textbf{underestimation (of something)}.} of their potential resilience\footnote{\textbf{resilience} [n] (also \textit{less frequent} \textbf{resiliency}) [uncountable] \textbf{1.} the ability of people or things to recover quickly after something unpleasant, such as shock or an injury; \textbf{2.} the ability of a substance to return to its original shape after it has been bent, stretched or pressed.} \& who have embraced Jordan's message that each individual has ultimate\footnote{\textbf{ultimate} [a] [only before noun] \textbf{1.} happening at the end of a process, \textsc{synonym}: \textbf{final}; \textbf{2.} most extreme; best; worst, greatest, most important, etc.; \textbf{3.} from which something originally comes, \textsc{synonym}: \textbf{fundamental}.} responsibility to bear\footnote{\textbf{bear} [v] \textbf{1.} \textbf{bear something} to have something as a characteristic or feature; to be connected with something; \textbf{2.} \textbf{bear something} to have a particular mark, word or symbol that can be seen; \textbf{3.} \textbf{bear something} to have a particular name; \textbf{4.} \textbf{bear something} to take responsibility for something difficult; to be affected by or deal with something unpleasant. If somebody \textbf{cannot bear} something, they feel unable to deal with it or accept it. The short form `can't\texttt{/}couldn't bear' is not suitable in academic writing, unless you are quoting.; \textbf{5.} to have a feeling, especially a negative feeling; \textbf{6.} \textbf{bear (doing) something} to be suitable for something; to be worth doing. If something \textbf{does not bear close inspection}, it will be found to be unacceptable when carefully examined. If something \textbf{does not bear comparison} with something else, it is not nearly as good.; \textbf{7.} \textbf{bear somebody\texttt{/}something} (\textit{formal}) to carry or hold somebody; \textbf{8.} (\textit{formal}) to give birth to a child; \textbf{9.} \textbf{bear something} (\textit{formal}) to produce flowers or fruit.}; that if one wants to live a full life, one 1st sets one's own house in order; \& only then can one sensibly\footnote{\textbf{sensible} [a] \textbf{1.} (of actions, plans, decisions, etc.) done or chosen with good judgment based on reason \& experience rather than emotion;  practical; \textbf{2.} (of people) able to make good judgments based on reason \& experience rather than emotion.}\,\footnote{\textbf{sensibly} [adv] \textbf{1.} in a way that shows the ability to make good judgments based on reason \& experience rather than emotion; \textbf{2.} in clothes that are useful rather than fashionable.} aim to take on bigger responsibilities\footnote{\textbf{responsibility} [n] (plural \textbf{responsibilities}) \textbf{1.} [uncountable, countable] a duty to deal with or take care of somebody\texttt{/}something, so that you may be blamed if something goes wrong; \textbf{2.} [uncountable] \textbf{responsibility (for something)} blame for something bad that has happened; \textbf{3.} [countable, uncountable] a moral duty to behave well with regard to somebody\texttt{/}something; \textbf{on your own responsibility} [idiom] without official permission \& being willing to take the blame if something goes wrong.}. The extent of this reaction\footnote{\textbf{reaction} [n] \textbf{1.} [countable, uncountable] what you do, say or think as a result of something that has happened; \textbf{2.} [countable] (\textit{chemistry}) a chemical change produced by 2 or more substances acting on each other; \textbf{3.} [countable, uncountable] (\textit{medical}) a response by the body, usually a bad one, to something such as a drug or a chemical substance; \textbf{4.} [uncountable, countable] (\textit{physics}) a force shown by something in response to another force, which is of equal strength \& acts in the opposite direction; \textbf{5.} [countable, usually singular] \textbf{reaction (against something)} a change in people's attitudes or behavior caused by strong disapproval of other very different attitudes; \textbf{6.} [uncountable] opposition to social or political progress or change; \textbf{7.} (\textbf{reactions}) [plural] the ability to move quickly in response to something, especially if in danger.} has often moved both of us to the brink\footnote{\textbf{brink} [n] [singular] \textbf{1.} \textbf{the brink (of something)} if you are on the brink of something, you are almost in a very new, dangerous or exciting situation; \textbf{2.} (\textit{literary}) the extreme edge of land, e.g. at the top of a cliff or by a river.} of tears\footnote{\textbf{tear} [v] \textbf{1.} [transitive, intransitive] to damage something by pulling it apart or into pieces or by cutting it on something sharp; to become damaged in this way; \textbf{2.} [transitive] \textbf{tear something $+$ adv.\texttt{/}prep.} to remove something from something else by pulling it violently; \textbf{3.} (\textbf{-torn}) (in adjectives) very badly affected or damaged by something; \textbf{tear somebody\texttt{/}something apart, to pieces, etc.} [idiom] to destroy or defeat somebody\texttt{/}something completely; [n] \textbf{tear (in something)} damage or a hole in something made by tearing; [n] [usually plural] a drop of liquid that comes out of your eye when you cry.}.

Sometimes these rules are demanding. They require you to undertake an incremental\footnote{\textbf{incremental} [a] \textbf{1.} happening in regular stages; \textbf{2.} increasing by regular amounts.} process that over time will stretch you to a new limit. That requires, as I've said, venturing\footnote{\textbf{venture} [n] a business project or activity, especially one that involving taking risks, \textsc{synonym}: \textbf{undertaking}; [v] \textbf{1.} [intransitive] \textbf{$+$ adv.\texttt{/}prep.} to go somewhere or do something even though it involves risks; \textbf{2.} [transitive, intransitive] (\textit{formal}) to say or do something in a careful way, especially because it might upset or offend somebody.} into the unknown. Stretching yourself beyond the boundaries of your current self requires carefully choosing \& then pursuing ideals: ideals that are up there, above you, superior to you -- \& that you can't always be sure you will reach.

But if it's uncertain that our ideals are attainable\footnote{\textbf{attainable} [a] that you can achieve, \textsc{synonym}: \textbf{achievable}.}, why do we bother\footnote{\textbf{bother} [v] \textbf{1.} [intransitive, transitive] (often used in negative sentences \& questions) to spend time \&\texttt{/}or energy doing something; \textbf{2.} [transitive] to annoy, worry or upset somebody; to cause somebody trouble or pain; \textbf{3.} [transitive] to interrupt somebody; to talk to somebody when they do not want to talk to you; [n] \textbf{1.} [uncountable] trouble or difficult; \textbf{2.} \textbf{a bother} [singular] an annoying situation, thing or person, \textsc{synonym}: \textbf{nuisance}; [exclamation] (\textit{British English, informal}) used to express the fact that you are annoyed about something\texttt{/}somebody.} reaching in the 1st place? Because if you don't reach for them, \fbox{it is certain you will never feel that you life has meaning}.

\& perhaps because, as unfamiliar\footnote{\textbf{unfamiliar} [a] \textbf{1.} that you do not know or recognize, \textsc{opposite}: \textbf{familiar}; \textbf{2.} \textbf{unfamiliar with something} not having any knowledge or experience of something, \textsc{opposite}: \textbf{familiar}.} \& strange\footnote{\textbf{strange} [a] \textbf{stranger, strangest} \textbf{1.} unusual or surprising, especially in a way that is difficult to understand or explain; \textbf{2.} not familiar because you have not visited, seen or experienced it before.} as it sounds, in the deepest part of our psyche, \fbox{we all want to be judged}.'' -- \cite[Foreword by Dr. \textsc{Norman Doidge}, MD, is the author of \textit{The Brain That Changes Itself}, pp. 5--19]{Peterson2018}

\section*{Overture}
\footnote{\textbf{overture} [n] \textbf{1.} a piece of music written as an introduction to an opera or a ballet; \textbf{2.} [usually plural] \textbf{overture (to somebody)} a suggestion or an action by which somebody tries to make friends, start a business relationship, have discussions, etc. with somebody else.} ``This book has a short history \& a long history. We'll begin with the short history.

In 2012, I started contributing to a website called Quora. On Quora, anyone can ask a question, of any sort -- \& anyone can answer. Readers upvote\footnote{\textbf{upvote} [v] \textbf{upvote (something)} to show that you agree with an online article or comment by using a particular icon, \textsc{opposite}: \textbf{downvote}; [n] an act of showing that you agree with an online article or comment by using a particular icon, \textsc{opposite}: \textbf{downvote}.} those answers they like, \& downvote\footnote{\textbf{downvote} [v] [transitive, intransitive] \textbf{downvote (something)} to show that you disagree with an online article or comment by using a particular icon, \textsc{opposite}: \textbf{upvote}; [n] an act of showing that you disagree with an online article or comment by using a particular icon, \textsc{opposite}: \textbf{upvote}.} those they don't. In this manner, the most useful answers rise to the top, while the others sink\footnote{\textbf{sink} [v] \textbf{1.} [intransitive] to go down below the surface or towards the bottom of a liquid or soft substance; \textbf{2.} [transitive] \textbf{sink something} to damage a boat or ship so that it goes below the surface of the sea, etc.; \textbf{3.} [intransitive] (of an object) to move slowly downwards; \textbf{4.} [intransitive] \textbf{sink (to something)} to decrease in amount, volume, strength, etc.; \textbf{sink in $|$ sink into something} [phrasal verb] to go down into another substance through the surface; \textbf{sink into something} [phrasal verb] to go gradually into a less active, happy or pleasant state; \textbf{sink something into something} [idiom] to spend a lot of money on a business or an activity, e.g. in order to make money from it in the future; [n] \textbf{1.} a large open container that has taps to supply water \& that you use for washing dishes in; \textbf{2.} (\textit{specialist}) a body or process which acts to absorb or remove energy or a particular component from a system, \textsc{opposite}: \textbf{source}.} into oblivion\footnote{\textbf{oblivion} [n] [uncountable] \textbf{1.} a state in which you are not aware of what is happening around you, usually because you are unconscious or asleep; \textbf{2.} the state in which somebody\texttt{/}something has been forgotten \& is no longer famous or important, \textsc{synonym}: \textbf{obscurity}; \textbf{3.} a state in which something has been completely destroyed.}. I was curious about the site. I liked its \fbox{free-for-all nature}. The discussion was often compelling\footnote{\textbf{compelling} [a] \textbf{1.} that makes you think it is true or valid; \textbf{2.} making you pay attention through being so interesting \& exciting; \textbf{3.} that cannot be resisted.}, \& it was interesting to see the diverse\footnote{\textbf{diverse} [a] very different from each other; containing people or things of various kinds.} range of opinions\footnote{\textbf{opinion} [n] \textbf{1.} [countable] someone's feelings or thoughts about somebody\texttt{/}something, rather than a fact, \textsc{synonym}: \textbf{view}; \textbf{2.} [uncountable] the beliefs or views of a group of people; \textbf{3.} [countable] advice from a professional person.} generated\footnote{\textbf{generate} [v] \textbf{1.} \textbf{generate something} to create feelings, opinions or situations; \textbf{2.} \textbf{generate something} to produce a physical effect; \textbf{3.} \textbf{generate something} to produce something by performing a particular operation, e.g. using a computer; \textbf{4.} \textbf{generate something} to make money or create work; to increase business; \textbf{5.} \textbf{generate something} to produce energy, especially electricity.} by the same question.

When I was taking a break (or avoiding work), I often turned to Quora, looking for questions to engage\footnote{\textbf{engage} [v] \textbf{1.} \textbf{engage somebody\texttt{/}something} to succeed in attracting \& keeping somebody's attention \& interest; \textbf{2.} to employ somebody to do a particular job; \textbf{engage in something $|$ be engaged in something} [phrasal verb] to take part in an activity; \textbf{engage with something\texttt{/}somebody} [phrasal verb] to become involved with \& try to understand something\texttt{/}somebody.} with. I considered, \& eventually answered, such questions as ``What's the difference between being happy \& being content\footnote{\textbf{content} [n] \textbf{1.} (\textbf{contents}) [plural] \textbf{content (of something)} the things that are contained in something; \textbf{2.} (\textbf{contents}) [plural] the different sections that are contained in a book, magazine, journal or website; a list of these sections; \textbf{3.} [singular] the subject matter of a book, speech, programme, etc.; \textbf{4.} [singular] (following a noun or an adjective) the amount of a substance that is contained in something else; \textbf{5.} [uncountable] the information or other material contained on a website, CD-ROM, etc.; [a] [not before noun] satisfied \& happy with what you have; willing to do or accept something; [v] \textbf{content yourself with something} to accept \& be satisfied with something \& not try to have or do something better.}?'', ``What things get better as you age?'' \& ``What makes life more meaningful\footnote{\textbf{meaningful} [a] \textbf{1.} serious, useful or important; \textbf{2.} clearly showing the information that is required.}?''

Quora tells you how many people have viewed your answer \& how many upvotes you received. Thus, you can determine your reach, \& see what people think of your ideas. Only a small minority of those who view an answer upvote it. As of Jul 2017, as I write this -- \& 5 years after I addressed ``What makes life more meaningful?''\footnote{See \href{https://www.quora.com/What-makes-life-more-meaningful}{Quora\texttt{/}What makes life more meaningful?}.} -- my answer to that question has received a relatively small audience (14,000 views, \& 133 upvotes), while my response to the question about aging has been viewed by 7,200 people \& received 36 upvotes. Not exactly home runs\footnote{\textbf{home run} [n] (\textit{also North American English, informal} \textbf{homer}) (in baseball) a hit that allows the person hitting the ball to run around all the bases without stopping.}. However, it's to be expected. On such sites, most answers receive\footnote{\textbf{receive} [v] \textbf{1.} to get or accept something that is sent or given to you; \textbf{2.} to experience, suffer or be given a particular type of attention or treatment; \textbf{3.} [usually passive] to react to something new, in a particular way; \textbf{4.} to change broadcast signals into sounds or pictures on a television or other equipment; \textbf{5.} \textbf{receive somebody} to welcome or entertain a visitor; \textbf{6.} \textbf{receive somebody (into something)} (\textit{formal}) to officially recognize \& accept somebody as a member of a group.} very little attention, while a tiny\footnote{\textbf{tiny} [a] (\textbf{tinier, tiniest}) very small in size or amount.} minority\footnote{\textbf{minority} [n] (plural \textbf{minorities}) \textbf{1.} [singular $+$ singular or plural verb] the smaller part of a group; less than half of the people or things in a large group, \textsc{opposite}: \textbf{majority}; \textbf{2.} [countable] a group within a community or country that is different because of race, religion, culture or language; \textbf{3.} [singular] (in a parliament, committee, etc.) the people who did not win enough votes to have a clear victory; the votes of these people, \textsc{opposite}: \textbf{majority}; \textbf{4.} [uncountable] (\textit{law}) the state of being under the age at which somebody is legally an adult; \textbf{be in a\texttt{/}the minority} [idiom] to form less than half of a large group, \textsc{opposite}: \textbf{be in the\texttt{/}a majority}.} become disproportionately\footnote{\textbf{disproportionate} [a] too large or too small when compared with something else.}\,\footnote{\textbf{disproportionately} [adv] in a way that is too large or too small when compared with something else.} popular\footnote{\textbf{popular} [a] \textbf{1.} liked or admired by many people or by a particular person or group, \textsc{opposite}: \textbf{unpopular}; \textbf{2.} [only before noun] (\textit{sometimes disapproving}) made for the tastes \& knowledge of ordinary people; \textbf{3.} [only before noun] (of an idea, belief or opinion) shared by most or many people; \textbf{4.} [only before noun] (of political activity) done by the ordinary people of a country rather than limited to politicians or political parties; \textbf{contrary to popular belief} [idiom] opposite to what most people believe.}.

Soon after, I answered another question: \fbox{``What are the most valuable things everyone should know?''} I wrote a list of rules, or maxims\footnote{\textbf{maxim} [n] a well-known phrase that expresses something that is usually true or that people think is a rule for sensible behavior.}; some dead serious, some tongue-in-cheek\footnote{\textbf{tongue-in-cheek} [a] not intended seriously; done or said as a joke; [adv] not seriously; as a joke.} -- ``Be grateful in spite of your suffering,'' ``Do not do things that you hate,'' ``Do not hide things in the fog\footnote{\textbf{fog} [n] [uncountable, countable] \textbf{1.} a thick cloud of very small drops of water in the air close to the land or sea, that is very difficult to see through it; \textbf{2.} a state in which things are not clear \& seem difficult to understand; [v] \textbf{1.} [intransitive, transitive] \textbf{fog (something) (up)} if a glass surface \textbf{fogs} or \textbf{is fogged} up, it becomes covered in steam or small drops of water so that you cannot see through; \textbf{2.} [transitive] \textbf{fog something} to make somebody\texttt{/}something confused or less clear.},'' \& so on. The Quora readers appeared pleased with this list. They commented on \& shared it. They said such things as ``I'm definitely printing this list out \& keeping it as a reference. Simply phenomenal\footnote{\textbf{phenomenal} [a] \textbf{1.} very great or impressive, \textsc{synonym}: \textbf{extraordinary}; \textbf{2.} that can be felt through the senses or through immediate experience.},'' \& ``You win Quora. We can just close the site now.'' Students at the University of Toronto, where I teach, came up to me \& told me how much they liked it. To date, my answer to ``What are the most valuable things $\ldots$''\footnote{See \href{https://www.quora.com/What-are-the-most-valuable-things-everyone-should-know}{Quora\texttt{/}What are the most valuable things everyone should know?}.} has been viewed by a hundred \& 20,000 people \& been upvoted 2300 times. Only a few hundred of the roughly 600,000 questions on Quora have cracked\footnote{\textbf{crack} [n] a line on the surface of something where it has broken but not split into separate parts; [v] \textbf{1.} [intransitive, transitive] to break without dividing into separate parts; to break something in this way; \textbf{2.} [intransitive] to no longer be able to function normally because of pressure; \textbf{3.} [transitive] \textbf{crack something} to find the solution to a problem, etc.; \textbf{crack down on somebody\texttt{/}something} [phrasal verb] to try harder to prevent an illegal activity \& deal more severely with those who are caught doing it.} the 2000-upvote barrier\footnote{\textbf{barrier} [n] \textbf{1.} a problem, rule or situation that prevents somebody from doing something, or that makes something impossible; \textbf{2.} something that exists between 1 thing or person \& another \& keeps them separate.}. My procrastination-induced\footnote{\textbf{procrastination} [n] [uncountable] (\textit{formal, disapproving}) the act of delaying something that you should do, usually because you do not want to do it.} musings\footnote{\textbf{musing} [n] [uncountable, countable, usually plural] a period of thinking carefully about something or telling people your thoughts about it.} hit\footnote{\textbf{hit} [v] \textbf{1.} to bring your hand, or an object you are holding, against somebody\texttt{/}something quickly \& with force; \textbf{2.} \textbf{hit something\texttt{/}somebody} to come against something\texttt{/}somebody with force, especially causing damage or injury; \textbf{3.} \textbf{hit something (on\texttt{/}against something)} to come against something with force with a part of your body; \textbf{4.} [often passive] \textbf{hit somebody\texttt{/}something} (of a bullet, bomb, etc. or a person using them) to reach \& touch a person or thing suddenly \& with force; \textbf{5.} \textbf{hit somebody\texttt{/}something} to have a bad effect on somebody\texttt{/}something; \textbf{6.} \textbf{hit somebody} to reach a particular level; \textbf{7.} \textbf{hit something} (\textit{rather informal}) to experience something difficult or unpleasant; \textbf{hit\texttt{/}touch a (raw\texttt{/sensitive}) nerve} [idiom] to mention a subject that makes somebody feel angry, upset or embarrassed; \textbf{hit on\texttt{/}upon something} [phrasal verb] [no passive] (\textit{rather informal}) to think of a good idea suddenly or by chance; [n] \textbf{1.} a person or thing that is very popular; \textbf{2.} a visit by somebody to a particular website; a result of a search on a computer; e.g. on the Internet; \textbf{3.} an occasion when something is damaged by something, especially by something that has been thrown or fired at it.} a nerve\footnote{\textbf{nerve} [n] \textbf{1.} [countable] any of the long threads that carry messages between the brain \& parts of the body, enabling you to move, feel pain, etc.; \textbf{2.} (nerves) [plural] feelings of anxiety, \textsc{synonym}: \textbf{anxiety}; \textbf{3.} [uncountable] the courage to do something difficult or dangerous.}. I had written 99.9 percentile\footnote{\textbf{percentile} [n] \textbf{percentile (of something)} (\textit{statistics}) 1 of the 100 equal groups that a larger population can be divided into, according to their place on a scale measuring a particular value.} answer.

It was not obvious\footnote{\textbf{obvious} [a] \textbf{1.} easy to see or understand, \textsc{synonym}: \textbf{clear}; \textbf{2.} that most people would think of or agree to.} to me when I wrote the list of rules for living that it was going to perform\footnote{\textbf{perform} [v] \textbf{1.} [transitive] \textbf{perform something} to do something, such as a piece of work, task or duty, \textsc{synonym}: \textbf{carry something out}; \textbf{2.} [intransitive] \textbf{$+$ adv.\texttt{/}prep.} to work or function well or badly; \textbf{3.} [transitive, intransitive] \textbf{perform (something)} to entertain an audience by playing a piece of music, acting in a play, etc.} so well. I had put a fair bit of care into all the 60 or so answers I submitted\footnote{\textbf{submit} [v] \textbf{1.} [transitive] to give a proposal, application or other document to somebody in authority so that they can consider or judge it; \textbf{2.} [intransitive, transitive] to accept the authority, control or greater strength of somebody\texttt{/}something; to agree to something because of this, \textsc{synonym}: \textbf{give it to something\texttt{/}somebody, yield}; \textbf{3.} [transitive] \textbf{submit that $\ldots$} (\textit{law} or \textit{formal}) to say or suggest something.} in the few months surrounding\footnote{\textbf{surrounding} [a] [only before noun] \textbf{1.} that is near or around something; \textbf{2.} that is closely connected with something\texttt{/}somebody.} that post\footnote{\textbf{post} [n] \textbf{1.} [countable] a job, especially an important one in a large organization, \textsc{synonym}: \textbf{position}; \textbf{2.} (\textit{especially North American English}) $=$ \textbf{posting}; \textbf{3.} (British English) (also \textbf{mail} \textit{North American English, British English}) [uncountable] the official system used for sending \& delivering letters \& packages; letters \& packages that are sent \& delivered; \textbf{4.} (also \textbf{posting}) [countable] a piece of writing that forms part of a blog; a message sent to a discussion group on the Internet; [v] \textbf{1.} [transitive, intransitive] to put information or pictures on a website; \textbf{2.} (\textit{British English}) (\textit{North American English} \textbf{mail}) [transitive] \textbf{post something (to somebody)} to send a letter, etc. to somebody by post; \textbf{3.} [transitive] \textbf{post something $+$ adv.\texttt{/}prep.} to put a notice, etc. in a public place so that people can see it, \textsc{synonym}: \textbf{display}; \textbf{4.} [transitive, usually passive] to send somebody to a place for a period of time as part of their job; \textbf{5.} [transitive] \textbf{post somebody $+$ adv.\texttt{/}prep.} to put somebody, especially a soldier, in a particular place so that they can guard a building or area; [prep] after.}. Nonetheless\footnote{\textbf{nonetheless} [adv] despite this fact, \textsc{synonym}: \textbf{nevertheless}.}, \fbox{Quora provides market research at its finest}. The respondents\footnote{\textbf{respondent} [n] \textbf{1.} \textbf{respondent (to something)} a person who answers questions, especially in a survey; \textbf{2.} (\textit{law}) a person who is accused of something.} are anonymous\footnote{\textbf{anonymous} [a] \textbf{1.} (of a person) with a name that is not known or that is not made public; \textbf{2.} written, given, made, etc. by somebody who does not want their name to be known or made public.}. They're disinterested\footnote{\textbf{disinterested} [a] not influenced by personal feelings or by the chance of getting some advantage for yourself, \textsc{synonym}: \textbf{impartial, objective, unbiased}.}, in the best sense. Their opinions are spontaneous\footnote{\textbf{spontaneous} [a] \textbf{1.} happening naturally, without being made to happen; \textbf{2.} not planned but done because you suddenly want to do it.} \& unbiased\footnote{\textbf{unbiased} [a] fair \& not influenced by your own or somebody else's opinions or wishes, \textsc{synonym}: \textbf{impartial}, \textsc{opposite}: \textbf{biased}.}. So, I paid attention to the results, \& thought about the reasons for that answer's disproportionate success. Perhaps I struck the right balance\footnote{\textbf{balance} [n] \textbf{1.} [singular, uncountable] a situation in which all parts exist in equal or appropriate amounts; \textbf{2.} [countable, usually singular] the amount of money in a bank account; the amount of a bill that remains after parts has been paid; \textbf{3.} [uncountable] the ability to keep steady with an equal amount of weight on each side of the body; [v] \textbf{1.} [transitive, often passive, intransitive] to be equal in importance or amount to something else that has been opposite effect, \textsc{synonym}: \textbf{offset}; \textbf{2.} [transitive] \textbf{balance A with\texttt{/}\& B} to give equal importance to 2 different things or parts of something; \textbf{3.} [transitive, often passive] \textbf{balance A against B} to compare the importance of 2 different things; \textbf{4.} [transitive] \textbf{balance something} (\textit{finance}) to show or make sure that in an account the total money spent is equal to the total money received; \textbf{5.} [intransitive, transitive] \textbf{balance (something) (on something)} to put your body or something else into a position where it is steady \& does not fall.} between the familiar\footnote{\textbf{familiar} [a] \textbf{1.} \textbf{familiar with something} knowing something well, \textsc{opposite}: \textbf{unfamiliar}; \textbf{2.} well known to you; often seen or heard \& therefore easy to recognize, \textsc{opposite}: \textbf{unfamiliar}.} \& the unfamiliar while formulating the rules. Perhaps people were drawn to the structure that such rules apply. \fbox{Perhaps people just like lists.}

A few months earlier, in March of 2012, I had received an email from a literary agent. She had heard me speak on CBC radio during a show entitled \textit{Just Say No to Happiness}, where I had criticized the idea that happiness was the proper goal for life. Over the previous decades I had read more than my share of dark books about the 20th century, focusing particularly on Nazi Germany \& the Soviet Union. Aleksandr Solzhenitsyn, the great documenter of the slave-labor-camp horrors of the latter, once wrote that the ``pitiful\footnote{\textbf{pitiful} [a] \textbf{1.} deserving pity or causing you to feel pity, \textsc{synonym}: \textbf{pathetic}; \textbf{2.} not deserving respect, \textsc{synonym}: \textbf{poor}.} ideology'' holding that ``human beings are created for happiness'' was an ideology ``done in by the 1st blow\footnote{\textbf{blow} [v] \textbf{1.} [intransitive, transitive] to send our air from the mouth; \textbf{2.} [intransitive] when the wind or a current of air blows, it is moving; \textbf{3.} [intransitive, transitive] to be moved by the wind, somebody's breath, etc.; to move something in this way; \textbf{blow up} [phrasal verb] to explode; to be destroyed by an explosion; \textbf{blow something up} [phrasal verb] \textbf{1.} to destroy something by an explosion; \textbf{2.} to fill something with air or gas so that it becomes firm; \textbf{3.} to make a photograph bigger, \textsc{synonym}: \textbf{enlarge}; [n] \textbf{1.} a sudden event that has damaging effects on somebody\texttt{/}something, causing sadness or disappointment; \textbf{2.} \textbf{blow (to something)} a hard hit with the hand, a weapon, etc.} of the work assigner's\footnote{\textbf{assign} [v] \textbf{1.} to give somebody something that they can use, or some work or a duty, \textsc{synonym}: \textbf{allocate}; \textbf{2.} to say that somebody\texttt{/}something is responsible for something; \textbf{3.} to say that something has a a particular value or function, or happens at a particular time or place; \textbf{4.} to choose somebody for a particular task, position or purpose; \textbf{5.} [usually passive] \textbf{assign somebody to somebody\texttt{/}something} to send a person to work or live under the authority of somebody or in a particular group or place; \textbf{6.} \textbf{assign something to somebody} (\textit{law}) to say that your property or rights now belong to somebody else.} cudgel\footnote{\textbf{cudgel} [n] a short thick stick that is used as a weapon; [v] \textbf{cudgel somebody} to hit somebody with a cudgel.}.''\footnote{Solzhenitsyn, A.I. (1975). \textit{The Gulag Archipelago 1918--1956: An experiment in literary investigation} (Vol. 2). (T.P. Whitney, Trans.). New York: Harper \& Row, p. 626.} In a crisis\footnote{\textbf{crisis} [n] (plural \textbf{crises}) [countable, uncountable] a time of great danger, difficulty or confusion when problems must be solved or important decisions must be made.}, the inevitable suffering that life entails\footnote{\textbf{entail} [v] \textbf{1.} to have something as a necessary part of a process or plan, \textsc{synonym}: \textbf{involve}; \textbf{2.} to have something as a necessary result, according to the laws of logic.} can rapidly\footnote{\textbf{rapidly} [adv] in a short period of time or at a fast rate.} make a mockery\footnote{\textbf{mockery} [n] (plural \textbf{mockeries}) \textbf{1.} [uncountable, countable] comments or actiosn that are intended to make somebody\texttt{/}something seem silly, \textsc{synonym}: \textbf{ridicule, scorn}; \textbf{2.} [countable, usually singular] (\textit{disapproving}) an action, a decision, etc. that is a failure \& that is not as it is supposed to be, \textsc{synonym}: \textbf{travesty}; \textbf{made a mockery of something} [idiom] to make something seem silly or without effect.} of the idea that happiness is the proper pursuit\footnote{\textbf{pursuit} [n] \textbf{1.} [uncountable] the act of trying to find, obtain or achieve something; \textbf{2.} [countable] an activity, especially one that you do because you enjoy it; \textbf{3.} [uncountable] the act of following or trying to catch somebody.} of the individual\footnote{\textbf{individual} [n] \textbf{1.} a person considered separately rather than as part of a group; \textbf{2.} a single member of a group or class; \textbf{3.} a person who is very different from others \& has lots of new \& interesting ideas; [a] \textbf{1.} [only before noun] considered separately rather than as part of a group; \textbf{2.} [only before noun] of or for a particular person; \textbf{3.} [only before noun] designed for use by 1 person; \textbf{4.} characteristic of a particular person or thing; \textbf{5.} (\textit{usually approving}) having an unusual character, \textsc{synonym}: \textbf{distinctive, original}.}. On the radio show, I suggested, instead, that \fbox{a deeper meaning was required}. I noted that the nature of such meaning was constantly re-presented\footnote{\textbf{re-present} [v] \textbf{re-present something} to give, show or send something again, especially a cheque, bill, etc. that has not been paid.} in the great stories of the past, \& that it had more to do with developing character in the face of suffering than with happiness. This is part of the long history of the present work.

From 1985 until 1999 I worked for about 3 hours a day on the only other book I have ever published: \textit{Map of Meaning: The Architecture\footnote{\textbf{architecture} [n] \textbf{1.} [uncountable] the design or style of a building or buildings; the art \& study of designing buildings; \textbf{2.} [uncountable, countable] \textbf{architecture (of something)} the structure or design of something; \textbf{3.} [uncountable, countable] (\textit{computing}) the structure \& logical organization of a computer system.} of Belief\footnote{\textbf{belief} [n] \textbf{1.} [uncountable] a strong feeling that something\texttt{/}somebody exists or is true; confidence that something\texttt{/}somebody is good or right; \textbf{2.} [countable, usually plural] something that you believe, often as part of your religion; \textbf{3.} [singular, uncountable] an opinion about something; something that you think is true; \textbf{contrary to popular belief} [idiom] opposite to popular belief.}}. During that time, \& in the years since, I also taught a course on the material in that book, 1st at Harvard, \& now at the University of Toronto. In 2013, observing the rise of YouTube, \& because of the popularity of some work I had done with TVO, a Canadian public TV station, I decided to film my university \& public lectures \& place them online. They attracted an increasingly large audience\footnote{\textbf{audience} [n] \textbf{1.} [countable $+$ singular or plural verb] the people who are watching or listening to a play, concert, somebody speaking, etc.; \textbf{2.} [countable] a number of people or a particular group of people who watch, read or listen to the same thing; \textbf{3.} [countable] \textbf{audience with somebody} a formal meeting with an important person.} -- more than a million views by Apr 2016. The number of views has risen very dramatically since then (up to 18 million as I write this), but that is in part because I became embroiled\footnote{\textbf{embroil} [v] [often passive] (\textit{formal}) to involve somebody\texttt{/}yourself in an argument or a difficult situation.} in a political controversy\footnote{\textbf{controversy} [n] (plural \textbf{controversies}) [uncountable, countable] public discussion \& argument abotu something that many people strongly disagree about, disapprove of, or are shocked by.} that drew an inordinate\footnote{\textbf{inordinate} [a] (\textit{formal}) far more than is usual or expected, \textsc{synonym}: \textbf{excessive}.} amount of attention.

That's another story. Maybe even another book.

I proposed\footnote{\textbf{propose} [v] \textbf{1.} to suggest a plan or an idea for people to consider \& decide on; \textbf{2.} to suggest an explanation of something for people to consider.} in \textit{Maps of Meaning} that the great myths \& religious stories of the past, particularly those derived from an earlier, oral\footnote{\textbf{oral} [a] \textbf{1.} [usually before noun] spoken rather than written, \textsc{opposite}: \textbf{written}; \textbf{2.} [only before noun] connected with the mouth.} tradition, were \textit{moral} in their intent\footnote{\textbf{intent} [n] [uncountable] (\textit{formal} or \textit{law}) what you intend to do, \textsc{synonym}: \textbf{intention}; \textbf{to all intents \& purposes} [idiom] (\textit{British English}) (\textit{North American English} \textbf{for all intents \& purposes}) in the effects that something has, if not officially; almost completely.}, rather than descriptive\footnote{\textbf{descriptive} [a] \textbf{1.} describing what something is like, rather than saying what it should be like or what category it belongs to; \textbf{2.} saying or showing clearly what something is like; giving a clear account of something.}. Thus, they did not concern themselves with what the world was, as a scientist\footnote{\textbf{scientist} [n] a person who studies 1 or more of the natural sciences.} might have it, but with now a human being should act. I suggested that our ancestors\footnote{\textbf{ancestor} [n] \textbf{1.} \textbf{ancestor (of somebody)} a person in your family who lived a long time ago; \textbf{2.} \textbf{ancestor (of something)} an animal or plant that lived or grew in the past which a modern animal or plant has developed from; \textbf{3.} \textbf{ancestor (of something)} an early form of something which later became more developed.} portrayed\footnote{\textbf{portray} [v] \textbf{1.} to show somebody\texttt{/}something in a picture or film; to describe somebody\texttt{/}something in a piece of writing, \textsc{synonym}: \textbf{depict}; \textbf{2.} to describe or show somebody\texttt{/}something in a particular way, especially when this does not give a complete or accurate impression of what they are like, \textsc{synonym}: \textbf{represent}; \textbf{3.} \textbf{portray somebody\texttt{/}something} to act a particular role in a film or play, \textsc{synonym}: \textbf{play}.} the world as a stage\footnote{\textbf{stage} [n] \textbf{1.} [countable] a point, period or step in a process or in the development of something; \textbf{2.} [countable] a raised area where actors, dancers, speakers, etc. perform; \textbf{3.} (often \textbf{the stage}) [singular] the theater \& the world of acting as a form of entertainment; \textbf{4.} [singular] an area of activity where important things happen, especially in politics; \textbf{5.} [countable] the part of a microscope on which you put the object you are looking at; \textbf{set the stage for something} [idiom] to make it possible for something to happen; to make something likely to happen; [v] \textbf{1.} \textbf{stage something} to organize \& present a play or an event for people to see; \textbf{2.} \textbf{stage something} to organize \& take part in action that needs careful planning, especially as a public protest; \textbf{3.} \textbf{stage somebody\texttt{/}something} (\textit{medical}) to say how far a disease, especially cancer, has progressed in a patient.} -- a drama\footnote{\textbf{drama} [n] \textbf{1.} [countable] a play for the theater, television or radio, \textsc{synonym}: \textbf{play}; \textbf{2.} [uncountable] plays considered as a form of literature; \textbf{3.} [uncountable] the fact of being exciting, \textsc{synonym}: \textbf{excitement}; \textbf{4.} [countable] an exciting event.} -- instead of a place of objects. I described how I had come to believe that the constituent\footnote{\textbf{constituent} [n] \textbf{1.} 1 of the parts of something that combine to form the whole; \textbf{2.} a person who lives in a constituency \& can vote in elections; [a] [only before noun] forming or helping to make a whole.} elements\footnote{\textbf{element} [n] \textbf{1.} [countable] a necessary or typical part of something; \textbf{2.} [countable] a simple chemical substance that consists of atoms of only 1 type \& cannot be split by chemical means into a simpler substance. Gold, oxygen \& carbon are all elements; \textbf{3.} [countable, usually singular] \textbf{element of risk, truth, surprise, etc.} a small amount of a quality or feeling; \textbf{4.} [countable, usually plural] \textbf{element (of something)} a group of people who form a part of a larger group of society; \textbf{5.} [countable] (\textit{mathematics}) a member of a set of group; \textbf{6.} [countable] the part of a piece of electrical equipment that gives out heat; \textbf{7.} [countable] 1 of the 4 substances (earth, air, fire \& water) which people used to believe everything else was made of; \textbf{8.} (\textbf{the elements}) [plural] the weather, especially bad weather; \textbf{in your element} [idiom] doing what you are good at \& enjoy.} of the world as drama were order or chaos, \& not material\footnote{\textbf{material} [n] \textbf{1.} [countable, uncountable] a substance from which a thing is or can be made; a substance with a particular quality; \textbf{2.} [uncountable] information or ideas used in books or other work; \textbf{3.} [countable, usually plural, uncountable] things that are needed in order to do a particular activity, \textsc{synonym}: \textbf{resource}; \textbf{4.} [uncountable, countable] cloth used for making clothes, etc., \textsc{synonym}: \textbf{cloth, fabric}; [a] \textbf{1.} [only before noun] connected with money \& possessions rather than with the needs of the mind or spirit, \textsc{opposite}: \textbf{spiritual}; \textbf{2.} [only before noun] connected with the physical world rather than with the mind or spirit, \textsc{opposite}: \textbf{spiritual}; \textbf{3.} important \& needing to be considered. In law, \textbf{material} is used to describe evidence or facts that are important, especially when these facts might have an effect on the result of a case.} things.

Order\footnote{\textbf{order} [n] \textbf{1.} [uncountable, countable] the way in which people or things are placed or arranged in relation to each other; \textbf{2.} [uncountable] the state in which everything is in the right place or something is as it should be, \textsc{opposite}: \textbf{disorder}; \textbf{3.} [uncountable] the state that exists when people obey laws, rules or authority; \textbf{4.} [countable] something that somebody is told too do by somebody in authority; \textbf{5.} [countable] a written instruction by a court or judge; \textbf{6.} [countable, uncountable] a request to make or supply goods; \textbf{7.} [countable, usually singular] the way that a society, the world, etc. is arranged, with its system of rules \& customs; \textbf{8.} [singular] a particular quality or degree; \textbf{9.} [countable] \textbf{order (of something)} (\textit{biology}) a group into which animals, plants, etc. that are related are divided, smaller than a class \& larger than a family; [v] \textbf{1.} to use your position of authority to tell somebody to do something or say that something must happen; \textbf{2.} \textbf{order something (from somebody\texttt{/}something)} to ask for goods to be made or supplied; to ask for a service to be provided; \textbf{3.} \textbf{order something} to organize or arrange something.} is where the people around you act according to well-understood social norms, \& remain\footnote{\textbf{remain} [v] (not usually used in the progressive tenses) \textbf{1.} \textit{linking verb} to continue to be something; to be still in the same state or condition; \textbf{2.} [intransitive] \textbf{remain (of something)} to still be present after the other parts have been removed or used; to continue to exist; \textbf{3.} [intransitive] to still need to be done, said or dealt with; \textbf{4.} [intransitive] \textbf{$+$ adv.\texttt{/}prep.} to stay in the same place; to not leave.} predictable\footnote{\textbf{predictable} [a] if something is predictable, you know in advance that it will happen or what it will be like.} \& cooperative\footnote{\textbf{cooperative} [a] (\textit{British English also} \textbf{co-operative}) \textbf{1.} [usually before noun] involving working together with others towards a shared aim; \textbf{2.} helpful by doing what you are asked to do; \textbf{3.} [usually before noun] (of a business) owned \& run by the people involved, with the profits shared by them; [n] (\textit{British English also} \textbf{co-operative}) a farm, business or other organization which is owned \& run jointly by its members, who share the profits or benefits.}. It;s the world of social structure, explored territory\footnote{\textbf{territory} [n] (plural \textbf{territories}) \textbf{1.} [uncountable, countable] land that is under the control of a particular country or ruler; \textbf{2.} [countable, uncountable] an area that an animal or group of animals considers as its own \& defends against others who try to enter it; \textbf{3.} [uncountable, countable] an area of knowledge, activity or experience; \textbf{4.} [countable] an area of a town, country, etc. that somebody has a particular rights in or responsibility for in their work or another activity; \textbf{5.} [uncountable] a particular type of land; \textbf{6.} (\textbf{Territory}) [countable] a country or an area that is part of the US, Australia or Canada but is not a state or province.}, \& familiarity\footnote{\textbf{familiarity} [n] \textbf{1.} [uncountable, singular] \textbf{familiarity with something} the state of knowing somebody\texttt{/}something well; the state of recognizing somebody\texttt{/}something; \textbf{2.} [uncountable] the fact of being well known to you.}. The state of Order is typically\footnote{\textbf{typically} [adv] \textbf{1.} used to say that something usually happens in the way that you are stating; \textbf{2.} in a way that shows the usual qualities or features of a particular type of person, thing or group.} portrayed\footnote{\textbf{portray} [v] \textbf{1.} to show somebody\texttt{/}something in a picture or film; to describe somebody\texttt{/}something in a piece of writing, \textsc{synonym}: \textbf{depict}; \textbf{2.} to describe or show somebody\texttt{/}something in a particular way, especially when this does not give a complete or accurate impression of what they are like, \textsc{synonym}: \textbf{represent}; \textbf{3.} \textbf{portray somebody\texttt{/}something} to act a particular role in a film or play, \textsc{synonym}: \textbf{play}.}, symbolically\footnote{\textbf{symbolically} [adv] as a symbol; in a way that involves or uses symbols.} -- imaginatively\footnote{\textbf{imaginatively} [adv] in a way that shows new \& exciting ideas, \textsc{synonym}: \textbf{inventively}.} -- as masculine\footnote{\textbf{masculine} [a] \textbf{1.} having the qualities or appearance considered to be typical of men; connected with or like men; \textbf{2.} (in some languages) belonging to a class of nouns, pronouns or adjectives that have masculine gender, not feminine or neuter.}. It's the Wise King \& the Tyrant\footnote{\textbf{tyrant} [n] a person who has complete power in a country \& uses it in a cruel \& unfair way, \textsc{synonym}: \textbf{dictator}.}, forever bound\footnote{\textbf{bound} [a] [not before noun] \textbf{1.} \textbf{bound to do\texttt{/}be something} certain or likely to happen, or to do or be something; \textbf{2.} forced to do something by law, duty or a particular situation; \textbf{3.} (in compounds) prevented from working normally by the conditions mentioned; \textbf{4.} (also in compounds) traveling, or ready to travel, in a particular direction or to a particular place; \textbf{bound together (by\texttt{/}in something)} [idiom] closely connected; \textbf{bound up in something} [idiom] \textbf{1.} very busy with something; very interested or involved in something; \textbf{2.} (also \textbf{bound up with something}) closely connected with something; [n] \textbf{1.} (\textbf{bounds}) [plural] the accepted or furthest limits of something; \textbf{2.} [countable] \textbf{bound (of something)} (\textit{specialist}) a limiting value, line or plane; \textbf{out of bounds} [idiom] not reasonable or acceptable; \textbf{out of bounds (to\texttt{/}for somebody)} (\textit{especially British English}) [idiom] outside the limits of where somebody is allowed to be; [v] [usually passive] \textbf{1.} \textbf{bound something} to form the edge or limit of an area, object or quantity; \textbf{2.} \textbf{bound something} to limit something; past tense, past participle of \textbf{bind}.} together, as society is simultaneously\footnote{\textbf{simultaneously} [adv] at the same time as something else.} structure \& oppression\footnote{\textbf{oppression} [n] [uncountable] cruel \& unfair treatment of people, especially by not giving them the same freedom, rights, etc. as other people.}.

Chaos, by contrast\footnote{\textbf{contrast} [n] \textbf{1.} [countable, uncountable] a difference between 2 or more people or things that you can see clearly when they are compared or put close together; the fact of comparing 2 or more things in order to show the differences between them; \textbf{2.} [countable, usually singular] a person or thing that is clearly different from somebody\texttt{/}something else; \textbf{3.} [uncountable] the amount of difference between light \& dark in a photograph of the picture on a screen; \textbf{4.} [countable, uncountable] differences in color or in light \& dark, used in photographs \& paintings to create a special effect; [v] \textbf{1.} [transitive, often passive] to compare 2 things in order to show the differences between them; \textbf{2.} [intransitive] to show a clear difference when close together or when compared.}, is where -- or when -- something unexpected happens. Chaos emerges\footnote{\textbf{emerge} [v] \textbf{1.} [intransitive, transitive] (of facts or ideas) to become known; \textbf{2.} [intransitive] to start to exist or appear; \textbf{3.} [intransitive] \textbf{emerge (from something) (into something)} to come out of a dark or hidden place; \textbf{4.} [intransitive] \textbf{emerge (from something)} to survive a difficult situation or experience.}, in trivial\footnote{\textbf{trivial} [a] \textbf{1.} not important, serious or valuable; not worth considering. \textbf{Trivial} is often used with a negative, to show that something is important, serious or valuable, \& needs attention., \textsc{opposite}: \textbf{non-trivial}; \textbf{2.} (\textit{mathematics}) used to describe the solution given when the value of each variable in the question is zero or their sum equals an identity, \textsc{opposite}: \textbf{non-trivial}.} form, when you tell a joke at a party with people you think you know \& a silent \& embarrassing chill falls over the gathering. Chaos is what emerges more catastrophically\footnote{\textbf{catastrophically} [adv] \textbf{1.} in a way that causes a lot of problems or makes people suffer; \textbf{2.} very badly, \textsc{synonym}: \textbf{disastrously}.} when you suddenly find yourself without employment\footnote{\textbf{employment} [n] [uncountable] \textbf{1.} work, especially when it is done to earn money; the state of being employed; \textbf{2.} the situation in which people have work; the number of people who have work in a country or area, \textsc{opposite}: \textbf{unemployment}; \textbf{3.} \textbf{employment (of somebody)} the act of employing somebody; \textbf{4.} \textbf{employment (of something)} the use of something.}, or are betrayed\footnote{\textbf{betray} [v] \textbf{1.} to fail to support somebody\texttt{/}something, by not doing what somebody trusted you to do or by not doing what is right; \textbf{2.} [often passive] to give information about somebody\texttt{/}something to an enemy; \textbf{3.} \textbf{betray something} to make somebody aware of a piece of information, a feeling, etc., often without meaning to.} by a lover\footnote{\textbf{lover} [n] \textbf{1.} a partner in a sexual or romantic relationship outside marriage; \textbf{2.} (often in compounds) a person who likes or enjoys a particular thing.}. As the antithesis\footnote{\textbf{antithesis} [n] [usually singular] (plural \textbf{antitheses}) (\textit{formal}) \textbf{1.} the opposite of something; \textbf{2.} a contrast between 2 things.} of symbolically masculine order, it's presented imaginatively as feminine\footnote{\textbf{feminine} [a] \textbf{1.} having the qualities or appearance considered to be typical of women; connected with women; \textbf{2.} (in some languages) belonging to a class of nouns, pronouns or adjective that have feminine gender, not masculine or neuter.}. It's the new \& unpredictable suddenly\footnote{\textbf{suddenly} [adv] quickly \& unexpected, \textsc{opposite}: \textbf{gradually}.} emerging in the midst of the commonplace\footnote{\textbf{commonplace} [a] done very often, or existing in many places, \& therefore not unusual; [n] \textbf{1.} [usually singular] an event, etc. that happens very often \& is not usual; \textbf{2.} a remark, etc. that is not new or interesting.} familiar. It's \fbox{Creation \& Destruction}, the source of new things \& the destination\footnote{\textbf{destination} [n] a place to which somebody\texttt{/}something is going or being sent.} of the dead (as nature, as opposed to culture, is simultaneously birth \& demise\footnote{\textbf{demise} [n] [singular] \textbf{1.} \textbf{demise (of something)} the end or failure or an institution, an idea, a company, etc.; \textbf{2.} (\textit{formal} or \textit{humorous}) death}.).

Order \& chaos are the yang\footnote{\textbf{yang} [n] [uncountable] (\textit{from Chinese}) (in Chinese philosophy) the bright active male principle of the universe.} \& yin\footnote{\textbf{yin} [n] [uncountable] (from Chinese) (in Chinese philosophy) the dark, not active, female principle or the universe.} of the famous Taoist\footnote{\textbf{Taoist} [n] a person who follows the Chinese philosophy that is based on the writings of Lao-tzu.} symbol: 2 serpents\footnote{\textbf{serpent} [n] (\textit{literary}) a snake, especially a large one.}, head to tail. Order is the white, masculine serpent; Chaos, its black, feminine counterpart. The black dot in the white -- \& the white in the black -- indicate\footnote{\textbf{indicate} [v] \textbf{1.} to show that something is true or exist; \textbf{2.} to be a sign of something; to show that something is possible or likely, \textsc{synonym}: \textbf{suggest}; \textbf{3.} \textbf{indicate something} to represent information without using words; \textbf{4.} to give information in writing; \textbf{5.} [usually passive] to suggest something as a necessary or recommended course of action; \textbf{6.} to mention something, especially in an indirect or brief way; \textbf{7.} \textbf{indicate something} (of an instrument for measuring things) to show a particular measurement.} the possibility\footnote{\textbf{possibility} [n] (plural \textbf{possibilities}) \textbf{1.} [uncountable, countable] the fact that something might exist, happen, or be true, but is not certain; \textbf{2.} [countable, usually plural] 1 of the different things that you can do in a particular situation.} of transformation\footnote{\textbf{transformation} [n] \textbf{1.} [countable] a complete change in somebody\texttt{/}something. In ecology, \textbf{transformation} is the process of changing inorganic matter into organic matter \& the other way round.; \textbf{2.} [countable] (\textit{mathematics}) a process by which an expression is changed by replacing 1 set of variables with another or a shape is changed following a particular rule; \textbf{3.} [uncountable] (\textit{biology}) the genetic alteration of a cell, by introducing DNA not naturally found in the cell.}: just when things seem secure\footnote{\textbf{secure} [v] \textbf{1.} to obtain or achieve something, especially when this means using a lot of effort; \textbf{2.} \textbf{secure something on\texttt{/}against something} to legally agree to give somebody property or goods that are worth the same amount as the money that you have borrowed from them, if you are unable to pay the money back; \textbf{3.} \textbf{secure something (against something)} to protect something so that it is safe \& difficult to attack or damage; \textbf{4.} \textbf{secure something (to something)} to attach or fix something firmly; [a] \textbf{1.} safe from being attacked, harmed or damaged; protected \&\texttt{/}or made stronger so that it is difficult for people to enter or leave, or to take something, \textsc{opposite}: \textbf{insecure}; \textbf{2.} likely to continue or be successful for a long time, \textsc{synonym}: \textbf{safe}, \textsc{opposite}: \textbf{insecure}; \textbf{3.} feeling happy \& confident about yourself or a particular situation, so that you do not need to worry, \textsc{opposite}: \textbf{insecure}; \textbf{4.} fixed or attached firmly.}, the unknown can loom\footnote{\textbf{loom} [v] [intransitive] to appear important or frightening \& likely to happen soon; \textbf{loom large} [idiom] to be worrying or frightening \& seem hard to avoid.}, unexpected \& large. Conversely\footnote{\textbf{conversely} [adv] in a way that is the opposite of something.}, just when everything seems lost, new order can emerge from catastrophe\footnote{\textbf{catastrophe} [n] \textbf{1.} a sudden very serious event that causes great suffering for many people, \textsc{synonym}: \textbf{disaster}; \textbf{2.} an event that has a very bad effect on somebody\texttt{/}something or makes very serious difficulties.} \& chaos.

For the Taoists, \fbox{meaning is to be found on the border between the ever-entwined}\footnote{\textbf{entwine} [v] [usually passive] \textbf{1.} to twist or wind something around something else; \textbf{2.} \textbf{be entwined (with something)} to be very closely involved or connected with something.} pair. To walk that border is to stay on the path of life, the divine\footnote{\textbf{divine} [a] [usually before noun] coming from or connected with God or a god; being a god.} Way.

\& that's \fbox{much better than happiness.}

The literary agent I referred to listened to the CBC radio broadcast\footnote{\textbf{broadcast} [v] \textbf{1.} [transitive, intransitive] to send out programmes on television or radio; \textbf{2.} [transitive] \textbf{broadcast something} to tell a lot of people about something; [n] [countable, uncountable] a radio or television programme; the sending out of a radio or television programme.} where I discussed such issues. It left her asking herself deeper questions. She emailed me, asking if I had considered writing a book for a general audience. I had previously attempted to produce a more accessible\footnote{\textbf{accessible} [a] \textbf{1.} that can be reached, entered, used or obtained; \textbf{2.} easy to understand.} version\footnote{\textbf{version} [n] \textbf{1.} a form of something that is slightly different from an earlier form or from other forms of the same thing; \textbf{2.} \textbf{version (of something)} a film, play, piece of music, etc. that is based on a particular piece of work but is in a different form, style or language; \textbf{3.} \textbf{version (of something)} a description of an event from the position of a particular person or group of people.} of \textit{Maps of Meaning}, which is a very dense\footnote{\textbf{dense} [a] (\textbf{dense, densest}) \textbf{1.} containing a lot of people, things, plants, etc. with little space between them; \textbf{2.} (\textit{specialist}) (of a substance) heavy in relation to its size; \textbf{3.} (of clouds, smoke, etc.) thick \& difficult to see through.} book. But I found that the spirit was neither in me during that attempt nor in the resultant\footnote{\textbf{resultant} [a] [only before noun] caused by the things that has just been mentioned.} manuscript\footnote{\textbf{manuscript} [n] (abbreviation \textbf{MS}) \textbf{1.} [countable, uncountable] a copy of a book, piece of music, etc. before it has been printed; \textbf{2.} [countable] a book, document or piece of music written by hand rather than typed or printed, especially a very old one.}. I think this was because I was imitating\footnote{\textbf{imitate} [v] \textbf{imitate somebody\texttt{/}something} to copy somebody\texttt{/}something.} my former\footnote{\textbf{former} [a] [only before noun] \textbf{1.} that used to exist in earlier times; \textbf{2.} that used to have  a particular position or status in the past; \textbf{3.} (\textbf{the former $\ldots$}) used to refer to the 1st of 2 things or people mentioned.} self\footnote{\textbf{self} [n] \textbf{1.} [countable] the type of person you are, especially the way you normally behave, look or feel; \textbf{2.} [uncountable] (\textbf{the self} [singular]) a person's personality or character that makes them different from other people; \textbf{3.} [uncountable] used to refer to somebody as the one affected by their own actions.}, \& my previous book, instead of occupying\footnote{\textbf{occupy} [v] \textbf{1.} \textbf{occupy something} to fill or use a space, area or amount of time, \textsc{synonym}: \textbf{take up something}; \textbf{2.} \textbf{occupy something} to live or work in a room, house or building; \textbf{3.} \textbf{occupy something} to enter a place in a large group \& take control of it, especially by military force; \textbf{4.} \textbf{occupy something} to have an official job or position, \textsc{synonym}: \textbf{hold}; \textbf{5.} \textbf{occupy something} to be in or at a particular position in a system, \textsc{synonym}: hold; \textbf{6.} to fill your time or keep you busy doing something.} the place between order \& chaos \& producing something new. I suggested that she watch 4 of the lectures I had done for a TVO program called \textit{Big Ideas} on my YouTube channel. I thought if she did that we could have a more informed\footnote{\textbf{informed} [a] \textbf{1.} having or showing a lot of knowledge about a particular subject or situation; \textbf{2.} (of a decision or choice) based on an understanding of the facts.} \& thorough\footnote{\textbf{thorough} [a] \textbf{1.} done completely; with great attention to detail; \textbf{2.} [not usually before noun] (of a person) doing things very carefully \& with great attention to detail.} discussion about what kind of topics I might address in a more publicly\footnote{\textbf{publicly} [adv] \textbf{1.} in a way that can be seen, heard or known by people in general, \textsc{opposite}: \textbf{privately}; \textbf{2.} by the state or government, rather than by a private company or individual, \textsc{opposite}: \textbf{privately}; \textbf{3.} in a way that affects or concerns ordinary people in society in general; \textbf{4.} on a stock exchange.} accessible\footnote{\textbf{accessible} [a] \textbf{1.} that can be reached, entered, used or obtained; \textbf{2.} easy to understand.} book.

She contacted me a few weeks later, after watching all 4 lectures \& discussing them with a colleague\footnote{\textbf{colleague} [n] a person that you work with, especially in a profession or business.}. Her interest had been further heightened\footnote{\textbf{heighten} [v] \textbf{heighten something} to make a feeling or an effect greater or stronger, \textsc{synonym}: \textbf{intensify}.}, as had her commitment\footnote{\textbf{commitment} [n] \textbf{1.} [singular, uncountable] a strong belief in a cause or activity \& a promise to support it; \textbf{2.} [countable, uncountable] a promise to do something or to behave in a particular way; \textbf{3.} [uncountable] the willingness to work hard \& give your energy \& time to a job or an activity; \textbf{4.} [countable] (used in compounds) a thing that you have promised or agreed to do, or that you have to do; \textbf{5.} [countable, uncountable] agreeing to use money, time or people in order to achieve something.} to the project\footnote{\textbf{project} [n] \textbf{1.} a planned piece of work that is designed to find information about something, to produce something new or to improve something; \textbf{2.} a piece of research done by a school or college student; \textbf{3.} \textbf{project (of something\texttt{/}of doing something)} a set of aims, ideas or activities that somebody is interested or wants to bring to people's attention; [v] \textbf{1.} [transitive, usually passive] to estimate what the size, cost or amount of something will be in the future, based on what is happening now, \textsc{synonym}: \textbf{forecast}; \textbf{2.} [transitive, usually passive] \textbf{be projected (for something)} to plan an activity, a project, etc. for a time in the future; \textbf{3.} [transitive] to present somebody\texttt{/}something\texttt{/}yourself to other people in a particular way, especially one that gives a good impression; \textbf{4.} [intransitive] \textbf{$+$ adv.\texttt{/}prep.} to stick out beyond an edge or surface; \textbf{5.} [transitive] \textbf{project something\texttt{/}somebody ($+$ adv.\texttt{/}prep.)} to throw something or make something more forward or away; \textbf{6.} [transitive] \textbf{project something} (on\texttt{/}onto something) to make light, an image, etc. fall onto a flat surface or screen; \textbf{project something onto somebody} [idiom] to imagine that other people have the same feelings, problems, etc. as you, especially when this is not true.}. That was promising\footnote{\textbf{promising} [a] showing signs of future success.} -- \& unexpected. I'm always surprised when people respond\footnote{\textbf{respond} [v] \textbf{1.} [intransitive] to do something as a reaction to something that somebody has said or done, \textsc{synonym}: \textbf{react}; \textbf{2.} [intransitive] \textbf{respond (to something)} to improve as a result of a particular kind of treatment; \textbf{3.} [intransitive, transitive] to give a spoken or written answer to somebody\texttt{/}something, \textsc{synonym}: \textbf{reply}.} positively\footnote{\textbf{positively} [adv] \textbf{1.} in a way that is good or useful, \textsc{opposite}: \textbf{negatively}; \textbf{2.} in a way that shows you are considering what is good in somebody\texttt{/}something, or are feeling confidence or hope, \textsc{opposite}: \textbf{negatively}; \textbf{3.} in a way that shows you approve or agree, or that involves giving the answer yes, \textsc{opposite}: \textbf{negatively}; \textbf{4.} in such a way that, then 1 thing increases, another thing also increase, \textsc{opposite}: \textbf{negatively}; \textbf{5.} in a way that leaves no possibility of doubt; \textbf{6.} in a way that contains or produces the type of electricity that is carried by an proton.} to what I am saying, given its seriousness\footnote{\textbf{seriousness} [n] [uncountable, singular] \textbf{seriousness (of something)} the state of being serious.} \& strange nature. I'm amazed I have been allowed (even encouraged) to teach what I taught 1st in Boston \& now in Toronto. I've always thought that if people really noticed what I was teaching there would be Hell to pay. You can decide for yourself what truth there might be in that concern after reading this book. :)

She suggested that I write a guide of sorts to what a person needs ``to live well'' -- whatever that might mean. I thought immediately about my Quora list. I had in the meantime\footnote{\textbf{in the meantime} [idiom] in the period of time between 2 times or 2 events; between now \& a future event.} written some further thoughts about of the rules I had posted. People had responded positively towards those new ideas, as well. It seemed to me, therefore, that there might be a nice fit between the Quora list \& my new agent's ideas. So, I sent her the list. She liked it.

At about the same time, a friend \& former student of mine -- the novelist \& screenwriter\footnote{\textbf{screenwriter} [n] a person who writes screenplays.} Gregg Hurwitz -- was considering a new book, which would become the bestselling\footnote{\textbf{bestselling} [a] [only before noun] (of a product, usually a book) bought by large numbers of people.} thriller\footnote{\textbf{thriller} [n] a book, play or film with an exciting story, especially one about crime or spying.} \textit{Orphan X}. He likes the rules, too. He had Mia, the book's female lead, post a selection of them, 1 by 1, on her fridge, at points in the story where they seemed apropos\footnote{\textbf{apropos} [prep] (also \textbf{apropos of}) in connection with or related to somebody\texttt{/}something.}. That was another piece of evidence supporting my supposition\footnote{\textbf{supposition} [n] (\textit{formal}) \textbf{1.} [countable] an idea that you think is true although you may not be able to prove it, \textsc{synonym}: \textbf{assumption}; \textbf{2.} [uncountable] the act of believing or claiming that something is true even though it cannot be proved.} of their attractiveness\footnote{\textbf{attractiveness} [n] [uncountable] \textbf{1.} the fact of being pleasant to look at, especially in a sexual way; the fact of making an animal interested in a sexual way; \textbf{2.} the fact of having features or qualities that make something seem interesting \& worth having, \textsc{synonym}: \textbf{appeal}.}. I suggested to my agent that I write a brief\footnote{\textbf{brief} [a] (\textbf{briefer, briefest}) \textbf{1.} using few words; \textbf{2.} lasting only a short time, \textsc{synonym}: \textbf{short}, \textsc{opposite}: \textbf{lengthy}; \textbf{in brief} [idiom] in a few words, without details; [n] \textbf{1.} (\textit{British English}) the instructions that a person is explaining what their job is \& what their duties are; \textbf{2.} (\textit{law}) a document giving the facts about a legal case; [v] to give somebody information about something so that they are prepared to deal with it.} chapter on each of the rules. She agreed, so I wrote a book proposal\footnote{\textbf{proposal} [n] \textbf{1.} a formal suggestion or plan; \textbf{2.} \textbf{proposal (that $\ldots$)} an explanation suggested for people to consider.} suggesting as much. When I started writing the actual\footnote{\textbf{actual} [a] [only before noun] \textbf{1.} existing in fact; real; \textbf{2.} used to emphasize the most important part of something.} chapters, however, they weren't at all brief. I had much more to say about each rule than I originally\footnote{\textbf{originally} [adv] used to described the situation that existed at the beginning of a particular period or activity, especially before something was changed.} envisioned\footnote{\textbf{envision} [v] \textbf{1.} \textbf{envision something} (\textit{formal}) to imagine what a situation will be like in the future, especially a situation you intend to work towards; \textbf{2.} (\textit{especially North American English}) (\textit{British English usually} \textbf{envisage}) to imagine what will happen in the future.}.

This was partly because I had spent a very long time researching my 1st book: studying history, mythology, neuroscience, psychoanalysis, child psychology, poetry, \& large sections of the Bible. I read \& perhaps even understood much of Milton's \textit{Paradise Lost}, Goethe's \textit{Faust} \& Dante's \textit{Inferno}. I integrated all of that, for better or worse, trying to address a perplexing\footnote{\textbf{perplexing} [a] making you confused or worried because you do not understand something, \textsc{synonym}: \textbf{puzzling}.} problem: the reason or reasons for the nuclear\footnote{\textbf{nuclear} [a] [usually before noun] \textbf{1.} of the nucleus ($=$ central part) of an atom; \textbf{2.} using, producing or resulting from energy that is produced by splitting the nucleus of atoms; \textbf{3.} connected with weapons that use energy produced by splitting atoms; \textbf{4.} (\textit{biology}) of the nucleus ($=$ central part) of a cell.} standoff\footnote{\textbf{standoff} [n] \textbf{standoff (between A \& B)} a situation in which no agreement can be reached, \textsc{synonym}: \textbf{deadlock}.} of the Cold War. I couldn't understand how belief systems could be so important to people that they were willing to risk the destruction\footnote{\textbf{destruction} [n] [uncountable, countable] the act of destroying something; the process of being destroyed.} of the world to protect them. I came to realize that shared belief systems made people intelligible\footnote{\textbf{intelligible} [a] that can be easily understood, \textsc{synonym}: \textbf{understandable}.} to one another -- \& that the systems weren't just about belief.

People who live by the same code are rendered\footnote{\textbf{render} [v] \textbf{1.} \textbf{render somebody\texttt{/}something $+$ adj.} to cause somebody\texttt{/}something to be in a particular state or condition, \textsc{synonym}: \textbf{make}; \textbf{2.} \textbf{render something (to somebody\texttt{/}something)} to give somebody something, especially in return for something or because it is expected; \textbf{3.} \textbf{render something} (\textit{formal}) to announce something, especially when it is done officially.} mutually\footnote{\textbf{mutually} [adv] done equally by 2 or more people or things. If 2 ideas, states or things are \textbf{mutually exclusive} or \textbf{mutually incompatible}, they cannot both be true or exist at the same time or be used together.} predictable\footnote{\textbf{predictable} [a] if something is predictable, you know in advance that it will happen or what it will be like.} to one another. They act in keeping with each other's expectations\footnote{\textbf{expectation} [n] \textbf{1.} [countable, usually plural, uncountable] the belief that something will happen or is likely to happen; \textbf{2.} [countable, usually plural] a belief about the particular way something should happen or how somebody should behave.} \& desires\footnote{\textbf{desire} [n] \textbf{1.} [countable, uncountable] a strong wish to have or do something; \textbf{2.} [uncountable] \textbf{desire (for somebody)} a strong wish to have sex with somebody; [v] (not used in the progressive tenses) (\textit{formal}) to want something.}. They can cooperate\footnote{\textbf{cooperate} [v] (\textit{British English also} \textbf{co-operate}) \textbf{1.} [intransitive] to work together with somebody in order to achieve something; \textbf{2.} [intransitive] to be helpful by doing what somebody asks you to do.}. They can even compete\footnote{\textbf{compete} [v] \textbf{1.} [intransitive] to try to be more successful than others. If somebody\texttt{/}something \textbf{cannot compete} with\texttt{/}against somebody\texttt{/}something else, they are not as successful.; \textbf{2.} [intransitive] to try to get something or do something, rather than letting somebody\texttt{/}something else get it or do it; \textbf{3.} [intransitive] \textbf{compete (with somebody\texttt{/}something)} to oppose somebody\texttt{/}soemthing; \textbf{4.} [intransitive] to take part in an election, sports event or other contest.} peacefully\footnote{\textbf{peacefully} [adv] \textbf{1.} in a way that does not involve a war, violence or argument, \textsc{synonym}: \textbf{peaceably}; \textbf{2.} in a quiet \& calm way; in a way that shows that you are not worried or disturbed in any way; \textsc{synonym}: \textbf{tranquilly}; \textbf{3.} in a way that shows that you are trying to create peace or to live in peace \& that you do not like violence or argument, \textsc{synonym}: \textbf{peaceably}.}, because everyone knows what to expect from everyone else. A shared belief system, partly psychological, partly acted out, simplifies everyone -- in their own eyes, \& in the eyes of others. Shared beliefs simplify the world, as well, because people who know what to expect from one another can act together to tame\footnote{\textbf{tame} [a] (comparative \textbf{tamer}, superlative \textbf{tamest}) \textbf{1.} (of animals, birds, etc.) not afraid of people, \& used to living with them, \textsc{opposite}: \textbf{wild}; \textbf{2.} (\textit{informal}) not interesting or exciting; \textbf{3.} (\textit{informal}) (of a person) willing to do what other people ask; [v] \textbf{1.} \textbf{tame something} to make an animal, bird, etc. not afraid of people \& used to living with them; \textbf{2.} \textbf{tame something} to make an emotion, an organization, a situation, etc., less powerful or easier to control.} the world. There is perhaps nothing more important than the maintenance\footnote{\textbf{maintenance} [n] [uncountable] \textbf{1.} the act of keeping something in good condition by checking or repairing it regularly; \textbf{2.} \textbf{maintenance (of something)} the act of making a condition or situation continue; \textbf{3.} \textbf{maintenance (of something)} the act of keeping something at the same level or rate; \textbf{4.} (\textit{British English}) money that somebody must pay regularly to their former wife, husband or partner, especially when they have had children together.} of this organization\footnote{\textbf{organization} [n] (\textit{British English also} \textbf{organisation}) \textbf{1.} [countable] an organized group of people with a particular purpose, such as a business or government department; \textbf{2.} [uncountable] the way in which the different parts of something are arranged, \textsc{synonym}: \textbf{structure}; \textbf{3.} [uncountable] the act of making arrangements or preparations for something, \textsc{synonym}: \textbf{planning}; \textbf{4.} [uncountable] the quality of being arranged in a neat, careful \& logical way; the ability to plan your work or life well \& in an efficient way.} -- this simplification. If it's threatened\footnote{\textbf{threaten} [v] \textbf{1.} [transitive] to say that you will cause trouble, hurt somebody, etc. if you do not get what you want; \textbf{2.} [transitive] to be a danger of something; to be likely to harm something, \textsc{synonym}: \textbf{endanger}; \textbf{3.} [intransitive] to seem likely to happen or cause something unpleasant.}, the great ship of state\footnote{\textbf{state} [n] \textbf{1.} [countable] the mental, emotional or physical condition that a person or thing is in. In physics \& chemistry, the \textbf{state} of a substance is whether it is a solid, liquid or gas.; \textbf{2.} (\textbf{State}) [countable] a country considered as an organized political community controlled by 1 government; \textbf{3.} (\textbf{State}) [countable] (abbr. \textbf{St.}) \textbf{state (of something)} an organized political community forming part of a country; \textbf{4.} (\textbf{the State}) [singular, uncountable] the government of a country; \textbf{5.} [uncountable] the formal ceremonies connected with high levels of government or with kings \& queens; \textbf{a state of affairs} [idiom] a situation; \textbf{state of the art} [idiom] the most modern or advanced techniques or methods in a particular field; [a] (also \textbf{State}) [only before noun] \textbf{1.} provided by, controlled by or belonging to the government of a country; \textbf{2.} connected with the leader of a country attending an official ceremony; \textbf{3.} connected with a particular state of a country, especially in the US; [v] \textbf{1.} to formally write or say something, especially in a carefully \& clear way; \textbf{2.} [usually passive] to fix or announce the details of something, especially on a written document; \textbf{put\texttt{/}stated differently} [idiom] in other words; used to introduce an explanation of something.} rocks.

It isn't precisely\footnote{\textbf{precisely} [adv] \textbf{1.} exactly; \textbf{2.} accurately; carefully; \textbf{3.} used to emphasize that something is very true or obvious; \textbf{more precisely} [idiom] used to show that you are giving more detailed \& accurate information about something you have just mentioned.} that people will fight for what they believe. They will fight, instead, to maintain \textit{the match between} what they believe, what they expect, \& what they desire. They will fight to maintain the match between what they expect \& how everyone is acting. It is precisely the maintenance of that match that enables everyone to live together peacefully, predictably \& productively\footnote{\textbf{productively} [adv] \textbf{1.} in a way that does a lot or achieves a lot; \textbf{2.} in an efficient way that produces large quantities of goods or crops.}. It reduces uncertainty \& the chaotic\footnote{\textbf{chaotic} [a] without any order; in a completely confused state.} mix of intolerable\footnote{\textbf{intolerable} [a] so bad or difficult that you cannot tolerate it; completely unacceptable, \textsc{synonym}: \textbf{unbearable}.} emotions that uncertainty inevitably\footnote{\textbf{inevitably} [adv] as is certain to happen.} produces.

Imagine someone betrayed by a trusted lover. The sacred\footnote{\textbf{sacred} [a] \textbf{1.} connected with God or a god \& thought to deserve special respect, \textsc{synonym}: \textbf{holy}; \textbf{2.} very important \& treated with great respect.} social contract obtaining between the 2 has been violated\footnote{\textbf{violate} [v] \textbf{1.} \textbf{violate something} to go against or refuse to obey a law, an agreement, etc.; \textbf{2.} \textbf{violate something} to not treat something with respect.}. \fbox{Actions speak louder than words}, \& an act of betrayal\footnote{\textbf{betrayal} [n] [uncountable, countable] the act of betraying somebody\texttt{/}something or the fact of being betrayed.} disrupts\footnote{\textbf{disrupt} [v] \textbf{1.} \textbf{disrupt something} to make it difficult for something to continue in the normal way; \textbf{2.} \textbf{disrupt something} (\textit{business}) to cause significant change in an industry or market by means of innovation ($=$ new ideas or methods).} the fragile\footnote{\textbf{fragile} [a] \textbf{1.} easily broken or damaged; \textbf{2.} weak \& uncertain; easy to destroy or harm or spoilt; \textbf{3.} thin or light \& often beautiful; \textbf{4.} not strong \& likely to become ill.} \& carefully negotiated\footnote{\textbf{negotiate} [v] \textbf{1.} [intransitive] to try to reach an argument by formal discussion; \textbf{2.} [transitive] to arrange or agree something by formal discussion; \textbf{3.} [transitive] \textbf{negotiate something ($+$ adv.\texttt{/}prep.)} to successfully get over or past a difficult part on a path or route; \textbf{4.} [transitive] \textbf{negotiate something ($+$ adv.\texttt{/}prep.)} to successfully solve a problem that is preventing you from achieving something.} peace\footnote{\textbf{peace} [n] \textbf{1.} [uncountable, singular] a situation or a period of time in which there is no war in a country or an area; \textbf{2.} [uncountable] a situation in which there is no public violence or disorder; \textbf{3.} [uncountable] the state of being calm or quiet; mental or emotional calm; \textbf{4.} [uncountable] \textbf{peace (with somebody)} the state of living in friendship with somebody without arguing.} of an intimate\footnote{\textbf{intimate} [a] \textbf{1.} (of a link between things) very close; \textbf{2.} (of people) having a close \& friendly relationship; \textbf{3.} sexual; \textbf{4.} private \& personal, often in a sexual way; \textbf{5.} (of a place or situation) encouraging close, friendly relationships; \textbf{6.} (of knowledge) very detailed \& thorough.} relationship\footnote{\textbf{relationship} [n] \textbf{1.} [countable] the way in which 2 people, groups or countries behave towards each other or deal with each other; \textbf{2.} [countable, uncountable] the way in which 2 or more people or things are connected, \textsc{synonym}: \textbf{relation}; \textbf{3.} [countable] a loving \&\texttt{/}or sexual friendship between 2 people; \textbf{4.} [countable, uncountable] the way in which a person is related to somebody else in a family.}. In the aftermath\footnote{\textbf{aftermath} [n] [usually singular] the situation that exists as a result of an important (\& usually unpleasant) event, especially a war, an accident, etc.} of disloyalty\footnote{\textbf{disloyalty} [n] [uncountable] \textbf{disloyalty (to somebody\texttt{/}something)} the fact of not showing support for your friends, family, country, etc, \textsc{opposite}: \textbf{loyalty}.}, people are seized\footnote{\textbf{seize} [v] \textbf{1.} \textbf{seize something} to be quick to take advantage of something such as chance or an opportunity; \textbf{2.} to take control of a place or situation, often suddenly \& violently; \textbf{3.} \textbf{seize something} (of the police, etc.) to take possession of something by legal right; \textbf{4.} \textbf{seize somebody} to arrest or capture somebody; \textbf{5.} \textbf{seize somebody\texttt{/}something (from somebody)} to take hold of somebody\texttt{/}something suddenly \& using force; \textbf{seize on\texttt{/}upon something} [phrasal verb] to suddenly show a lot of interest in something, especially because you can use it to your advantage.} by terrible\footnote{\textbf{terrible} [a] \textbf{1.} causing great harm or injury; very serious; \textbf{2.} [only before noun] (\textit{rather informal}) used to show the great extent or degree of something bad.} emotions: disgust\footnote{\textbf{disgust} [n] [uncountable] a strong feeling of dislike for somebody\texttt{/}something that you feel is unacceptable, or for something that looks, smells, etc. unpleasant; [v] \textbf{disgust somebody} if something \textbf{disgusts} you, it makes you feel shocked \& almost sick because it is so unpleasant.}, contempt\footnote{\textbf{contempt} [n] [uncountable] \textbf{1.} the feeling that somebody\texttt{/}something is without value \& deserves no respect at all; \textbf{2.} \textbf{concept for something} a lack of worry or fear about rules, danger, etc.; \textbf{3.} (also \textbf{contempt of court}) the crime of refusing to obey or show respect for a court or a judge.} (for self \& traitor\footnote{\textbf{traitor} [n] \textbf{traitor (to somebody\texttt{/}something)} a person who betrays their friends, their country, etc. y giving away secrets about them, by lying to or about them or by doing other things that will harm them.}), guilt\footnote{\textbf{guilt} [n] [uncountable] \textbf{1.} the fact that somebody has done something illegal, \textsc{opposite}: \textbf{innocence}; \textbf{2.} the unhappy feelings caused by knowing or thinking that you have done something wrong.}, anxiety\footnote{\textbf{anxiety} [n] (plural \textbf{anxieties}) \textbf{1.} [uncountable] the state of feeling nervous that something bad is going to happen; a fear about something; \textbf{2.} [uncountable] \textbf{anxiety to do something} a strong feeling of wanting to do something or of wanting something to happen.}, rage\footnote{\textbf{rage} [n] [uncountable, countable] a feeling of violent anger that is difficult to control; [v] \textbf{1.} [intransitive] (of a storm, a battle, an argument, etc.) to continue in a violent way; \textbf{2.} [intransitive] \textbf{rage (at\texttt{/}against\texttt{/}about somebody\texttt{/}something)} to show that you are very angry about something or with somebody, especially by shouting.} \& dread\footnote{\textbf{dread} [v] to be very afraid of something; to fear that something bad is going to happen; [n] \textbf{1.} [uncountable, countable, usually singular] a feeling of great fear about something that might or will happen in the future; a thing that causes this feeling; \textbf{2.} \textbf{dreads} [plural] (\textit{informal}) \textbf{dreadlocks} ($=$ hair that is twisted into long thick pieces that hang down from the head, worn especially by Rastafarians); [a] (\textit{formal}) \textbf{dreaded} ($=$ causing fear).}. Conflict is inevitable, sometimes with deadly\footnote{\textbf{deadly} [a] (\textbf{deadlier, deadliest}) (\textbf{More deadly} \& \textbf{deadliest} are the usual forms. You can also use \textbf{most deadly}.) causing or likely to cause death, \textsc{synonym}: \textbf{lethal}.} results. Shared belief systems -- shared systems of agreed-upon conduct\footnote{\textbf{conduct} [v] \textbf{1.} \textbf{conduct something} to organize \&\texttt{/}or do a particular activity; \textbf{2.} \textbf{conduct something} (of a substance) to allow heat or electricity to pass along or through it; \textbf{3.} \textbf{conduct yourself $+$ adv.\texttt{/}prep.} (\textit{formal}) to behave in a particular way; [n] [uncountable] (\textit{formal}) \textbf{1.} a person's behavior; \textbf{2.} \textbf{conduct of something} the way in which a business or an activity is organized \& managed.} \& expectation\footnote{\textbf{expectation} [n] \textbf{1.} [countable, usually plural, uncountable] the belief that something will happen or is likely to happen; \textbf{2.} [countable, usually plural] a belief about the particular way something should happen or how somebody should behave.} -- regulate\footnote{\textbf{regulate} [v] \textbf{1.} to control the rate of a machine or process so that it works in the correct way; to control how somebody\texttt{/}something behaves; \textbf{2.} to control something by means of rules.} \& control\footnote{\textbf{control} [n] \textbf{1.} [uncountable] the power to direct how a company, a country, etc. is run or to influence a process or a course of events; \textbf{2.} [uncountable, countable] (often in compounds) the act of restricting, limiting or managing something; a method of doing this; \textbf{3.} [uncountable] the ability to manage your emotions or actions; \textbf{4.} [countable] (often in compounds) a person, group or thing used as a standard of comparison for checking the results of a survey or an experiment; an experiment whose result is known; used for checking working methods; \textbf{5.} [uncountable, countable] a place where checks are made; the people who make these checks; \textbf{6.} [countable, usually plural] the switches \& buttons, etc. that you use to operate a machine or a vehicle; \textbf{7.} [uncountable] (also \textbf{control key}) [singular] (on a computer keyboard) a key that you press when you want to perform a particular operation; [v] \textbf{1.} to have power over a person, company, country, process, etc. so that you are able to decide what they must do or how it is run; \textbf{2.} \textbf{control something} to limit the number, level or strength of something, usually something negative; \textbf{3.} to make something, such as a machine or system, work in a particular way, \textsc{synonym}: \textbf{regulate}; \textbf{4.} \textbf{control something\texttt{/}yourself} to manage to make yourself remain calm, even though you are upset or angry; \textbf{control for something} [phrasal verb] to consider factors which are not important in your research but which many influence the results of an experiment or survey.} all those powerful\footnote{\textbf{powerful} [a] \textbf{1.} (of people, organizations or groups) able to control \& influence people \& events; \textsc{synonym}: \textbf{influential}; \textbf{2.} having great power or force; very effective; \textbf{3.} having a strong effect on people's feelings or thoughts.} forces\footnote{\textbf{force} [n] \textbf{1.} [countable] a person or thing that has a lot of power or influence; \textbf{2.} [uncountable] power or influence that somebody\texttt{/}something has. \textbf{Legal force} or \textbf{the force of the law} is the power or authority of the law.; \textbf{3.} [countable, uncountable] (\textit{physics}) an effect that causes things to move, change direction or change shape; \textbf{4.} [uncountable] violent physical action used to obtain or achieve something; \textbf{5.} [countable $+$ singular or plural verb, usually plural] soldiers or others whose job is to fight or to protect people; \textbf{6.} [countable $+$ singular or plural verb] a group of people who have been organized for a particular purpose; \textbf{7.} [uncountable] the physical strength of something as it hits something else; \textbf{8.} [countable, usually singular] a unit for measuring the strengths of the wind; [v] \textbf{1.} [often passive] to make somebody do something that they do not want to go, or go somewhere that they do not want to go; \textbf{2.} to make something happen, especially before people are ready; \textbf{3.} \textbf{force something $+$ adv.\texttt{/}prep.} to make something move in a particular direction.}. It's no wonder that people will fight to protect something that saves them from being possessed\footnote{\textbf{possess} [v] (not used in the progressive tenses) \textbf{1.} \textbf{possess something} to have or own something; \textbf{2.} to have a particular quality or feature.} by emotions of chaos \& terror\footnote{\textbf{terror} [n] \textbf{1.} [uncountable] a feeling or extreme fear; \textbf{2.} [uncountable] violent action or the threat of violent action that is intended to cause fear, usually for political purposes, \textsc{synonym}: \textbf{terrorism}; \textbf{3.} [countable] a person, situation or thing that makes you very afraid.} (\& after that from degeneration\footnote{\textbf{degeneration} [n] [uncountable, singular] \textbf{degeneration (of something)} the progress of becoming worse or less acceptable in quality or condition.} into strife\footnote{\textbf{strife} [n] [uncountable] (\textit{formal} or \textit{literary}) angry or violent disagreement between people or groups of people, \textsc{synonym}: \textbf{conflict}.} \& combat)\footnote{\textbf{combat} [n] [uncountable, countable] fighting or a fight, especially during a time of war; [v] \textbf{combat something} to stop something unpleasant or harmful from happening or from  getting worse.}.

There's more to it, too. A shared cultural system stabilizes\footnote{\textbf{stabilize} [v] (\textit{British English also} \textbf{stabilise}) \textbf{1.} [intransitive, transitive] to become firmly established \& not likely to change; to make something do this; \textbf{2.} [transitive] \textbf{stabilize something} to make something firm or steady so that it is not likely to move or fall over; \textbf{3.} [intransitive, transitive] (of a patient or their medical condition) to stop getting any worse after an injury or operation; to make a patient or their condition do this.} human interaction\footnote{\textbf{interaction} [n] [uncountable, countable] \textbf{1.} the effect that 2 things have on each other; \textbf{2.} the way that people communicate with each other, especially while they work or spend time with them.}, but is also a system of value -- a hierarchy\footnote{\textbf{hierarchy} [n] (plural \textbf{hierarchies}) \textbf{1.} [countable, uncountable] a system, especially in a society or organization, in which people are organized into different levels of important from highest to lowest; \textbf{2.} [countable] a system that ideas or beliefs can be arranged into.} of value, where some things are given priority\footnote{\textbf{priority} [n] (plural \textbf{priorities}) \textbf{1.} [countable] something that you think is more important than other things \& should be dealt with 1st; \textbf{2.} [uncountable] the condition of being considered or treated as more important than other things or people, \textsc{synonym}: \textbf{precedence}.} \& importance\footnote{\textbf{importance} [n] [uncountable] the quality of being important.} \& others are not. In the absence of such a system of value, people simply cannot act. In fact, they can't even perceive\footnote{\textbf{perceive} [v] \textbf{1.} to notice or become aware of something, \textsc{synonym}: \textbf{notice}; \textbf{2.} to be aware of or experience something using the senses; \textbf{3.} [often passive] to understand or think of somebody\texttt{/}something in a particular way; to believe that a particular thing is true, \textsc{synonym}: \textbf{see}.}, because both action \& perception\footnote{\textbf{perception} [n] \textbf{1.} [uncountable, countable] an idea, a belief or an image you have as a result of how you see or understand something; \textbf{2.} [uncountable] the way you notice things or the ability to notice things with the senses. In biology, \textbf{perception} refers to the processes in the nervous system by which a living thing becomes aware of events \& things outside itself.; \textbf{3.} [uncountable] the ability of understand the true nature of something, \textsc{synonym}: \textbf{insight}.} require\footnote{\textbf{require} [v] (not usually used in the progressive tenses) \textbf{1.} to need something; to depend on somebody\texttt{/}something; \textbf{2.} [often passive] to make somebody do or have something, especially because it is necessary according to a particular law or set of rules.} a goal\footnote{\textbf{goal} [n] something that you hope to achieve, \textsc{synonym}: \textbf{aim}.}, \& a valid\footnote{\textbf{valid} [a] \textbf{1.} based on what is logical or true, \textsc{opposite}: \textbf{invalid}; \textbf{2.} that is legally or officially acceptable, \textsc{opposite}: \textbf{invalid}.} goal is, by necessity\footnote{\textbf{necessary} [n] \textbf{1.} [uncountable] the fact that something must happen or be done; the need for something; \textbf{2.} [countable] necessity (of something) a thing that you must have \& cannot manage without; \textbf{3.} [countable, usually singular] a situation that must happen \& that cannot be avoided.}, something valued. We experience much of our positive emotion in relation to goals. We are not happy, technically\footnote{\textbf{technically} [adv] \textbf{1.} in a way that is connected with the use of science or technology; in a way that involves the use of machines; \textbf{2.} in a way that involves the skills \& processes needed for a particular activity; \textbf{3.} according to the exact meaning or facts, \textsc{synonym}: \textbf{strictly, strictly speaking}.} speaking, unless we see ourselves progressing\footnote{\textbf{progress} [n] \textbf{1.} [uncountable] the process of improving or developing, or of getting nearer to achieving or completing something; \textbf{2.} \textbf{progress (of somebody\texttt{/}something) ($+$ adv.\texttt{/}prep.)} movement forwards or towards a place; \textbf{in progress} [idiom] happening at this time; [v] \textbf{1.} [intransitive] to develop over a period of time to a better or more advanced state; to make progress, \textsc{synonym}: \textbf{advance}; \textbf{2.} [intransitive] to go forward in time, \textsc{synonym}: \textbf{go on}; \textbf{3.} [intransitive] \textbf{$+$ adv.\texttt{/}prep.} to move forward; \textbf{4.} [transitive] \textbf{progress something} to cause a task, project, etc. to make progress; \textbf{progress to something} [idiom] to move on from doing 1 thing to doing something else.} -- \& \fbox{the very idea of progression implies value}\footnote{\textbf{progression} [n] \textbf{1.} [uncountable, countable] the process of developing gradually from 1 stage or state to another; \textbf{2.} [countable] \textbf{progression (of something)} a number of things that come in a series.}. Worse yet is the fact that the meaning of life without positive value is not simply neutral\footnote{\textbf{neutral} [a] \textbf{1.} not supporting or helping either side in a disagreement, competition, etc., \textsc{synonym}: \textbf{impartial}; \textbf{2.} not belonging to or supporting any of the countries that are involved in a war; \textbf{3.} neither acid or alkaline; \textbf{4.} (abbr. \textbf{N}) having neither a positive nor a negative electrical charge; \textbf{5.} having no effect on other things; having their positive nor negative characteristics; \textbf{6.} deliberately not expressing any strong feeling; \textbf{7.} (of colors) not very bright or strong, such as grey or light brown.}. Because we are vulnerable\footnote{\textbf{vulnerable} [a] \textbf{vulnerable (to somebody\texttt{/}something)} weak \& easily hurt physically or emotionally.} \& mortal\footnote{\textbf{mortal} [a] \textbf{1.} that cannot live for ever \& must die; \textbf{2.} causing death or likely to cause death; very serious; \textbf{3.} lasting until death.}, pain\footnote{\textbf{pain} [n] [uncountable, countable] \textbf{1.} the feelings that somebody has in their body when they have been hurt or when they are ill; \textbf{2.} \textbf{pain (of something)} mental or emotional suffering; \textbf{on pain of something} [idiom] with the threat of having something done to you as a punishment if you do not obey.} \& anxiety are an integral\footnote{\textbf{integral} [a] \textbf{1.} being an essential part of something; \textbf{2.} [only before noun] (\textit{mathematics}) connected with an integer; involving only integers; [n] \textbf{integral (of something)} (\textit{mathematics}) an operation within calculus used to determine the area under a graph.} part of human existence. We must have something to set against the suffering that is intrinsic\footnote{\textbf{intrinsic} [a] belonging to or part of the real nature of something\texttt{/}somebody; forming an essential part of something.} to Being\footnote{\textbf{being} [n] \textbf{1.} [countable] a real or imaginary living creature; \textbf{2.} [uncountable] existence.}\footnote{``I use the term Being (with a capital ``B'') in part because of my exposure to the ideas of the 20th-century German philosopher Martin Heidegger. Heidegger tried to distinguish between reality, as conceived objectively, \& the totality of human experience (which is his ``Being''). Being (with a capital ``B'') is what each of us experiences, subjectively, personally \& individually, as well as what we each experience jointly with others. As such, it includes emotions, drives, dreams, visions \& revelations, as well as our private thoughts \& perceptions. Being is also, finally, something that is brought into existence by action, so its nature is to an indeterminate degree a consequence of our decisions \& choices -- something shaped by our hypothetically free will. Construed in this manner, Being is
\begin{enumerate}
	\item not something easily \& directly reducible to the material \& objective \&
	\item something that most definitely requires its  own term, as Heidegger labored for decades to indicate.''
\end{enumerate}}\,\footnote{\textbf{exposure} [n] \textbf{1.} [uncountable, countable] \textbf{exposure (to something)} the state of being in a place or situation where there is no protection from something harmful or unpleasant; \textbf{2.} [uncountable] \textbf{exposure (of something)} the fact of being discussed or mentioned on television, in newspapers, etc., \textsc{synonym}: \textbf{publicity}; \textbf{3.} [uncountable] \textbf{exposure (of something)} the state of having the true facts about somebody\texttt{/}something told after they have been hidden because they are bad, immortal or illegal; \textbf{4.} [uncountable] a medical condition caused by being out in very cold weather for too long without protection; \textbf{5.} [uncountable] \textbf{exposure (of something)} the act of showing something that is usually hidden.}\,\footnote{\textbf{philosopher} [n] a person who studies or writes about philosophy.}\,\footnote{\textbf{distinguish} [v] \textbf{1.} [intransitive, transitive] to recognize or show the difference between 2 people or things, \textsc{synonym}: \textbf{differentiate}; \textbf{2.} [transitive] (not used in the progressive tenses) to be a characteristic that makes 2 people, animals or things different, \textsc{synonym}: \textbf{differentiate}; \textbf{3.} [transitive] \textbf{distinguish A (from B)} to make something different or seem different from other similar things, \textsc{synonym}: \textbf{differentiate}; \textbf{4.} [transitive] to do something so well that people notice \& admire you; \textbf{5.} [transitive] (not used in the progressive tenses) \textbf{distinguish something} to be able to see or hear something, \textsc{synonym}: \textbf{make somebody\texttt{/}something out}.}\,\footnote{\textbf{reality} [n] (plural \textbf{realities}) \textbf{1.} [uncountable] the true situation \& the problems that actually exist in the world, especially in contrast to how people would like it to be; \textbf{2.} [countable] a thing that is actually experienced or seen, in contrast to what people might imagine; \textbf{3.} [uncountable] \textbf{reality television\texttt{/}TV\texttt{/}shows\texttt{/}series\texttt{/}contestants} television\texttt{/}shows, etc. that use real people (not actors) in real situations, presented as entertainment; \textbf{in reality} [idiom] used to say that a situation is different from what just been said or from what people believe.}\,\footnote{\textbf{conceive} [v] \textbf{1.} [transitive] to form an idea or plan in your mind; \textbf{2.} [transitive, intransitive] to think of something in a particular way; to imagine something; \textbf{3.} [intransitive, transitive] (of a woman) to become pregnant.}\,\footnote{\textbf{objectively} [adv] using facts \& not influenced by personal feelings or beliefs.}\,\footnote{\textbf{totality} [n] [countable, uncountable] the state of being complete or whole; the whole amount or number.}\,\footnote{\textbf{experience} [n] \textbf{1.} [uncountable] the knowledge \& skills that you have gained through doing something for a period of time; the process of gaining this; \textbf{2.} [uncountable] the things that have happened to you that affect the way you think \& behave; \textbf{3.} [countable] an event or activity that affects you in some way; \textbf{4.} (\textbf{the $\ldots$ experience}) [singular] events or knowledge shared by all the members of a particular group in society, that affects the way they think \& behave; [v] \textbf{1.} \textbf{experience something} to have a particular situation affect you or happen to you; \textbf{2.} \textbf{experience something} to have a particular emotion or physical feeling.}\,\footnote{\textbf{subjectively} [adv] \textbf{1.} in a way that is based on a person's own ideas, opinions or feelings rather than the facts, \textsc{opposite}: \textbf{objectively}; \textbf{2.} in a way that is based on what is in somebody's mind rather than on facts that can be proved, \textsc{opposite}: \textbf{objectively}.}\,\footnote{\textbf{personally} [adv] \textbf{1.} by a particular person rather than by somebody acting for them; \textbf{2.} in a way that is connected with 1 particular person rather than a group of people, \textsc{synonym}: \textbf{individually}; \textbf{3.} in a way that is connected with somebody's personal life rather than with their job or official position; \textbf{4.} with the personal presence or action of the individual mentioned; \textbf{5.} used to show that you are giving your own opinion about something; \textbf{6.} in a way that is intended to be offensive; \textbf{take something personally} [idiom] to believe that a remark or action is directed against you \& be upset or offended by it.}\,\footnote{\textbf{individually} [adv] separately, rather than as a group.}\,\footnote{\textbf{jointly} [adv] in a way that involves 2 or more people, groups or things together.}. \texttt{[stop translating here $\to$ read faster]} We must have the meaning inherent in a profound system of value or the horror of existence rapidly becomes paramount. Then, nihilism beckons, with its hopelessness \& despair.

\fbox{So: no value, no meaning.} Between value systems, however, there is the possibility of conflict. We are thus eternally caught between the most diamantine rock \& the hardest of places: loss of group-centered belief renders life chaotic, miserable, intolerable; presence of group-centered belief makes conflict with other groups inevitable. In the West, we have been withdrawing from our tradition-, religion- \& even nation-centered cultures, partly to decrease the danger of group conflict. But we are increasingly falling prey to the desperation of meaningless, \& that is not improvement at all.

While writing \textit{Maps of Meaning}, I was (also) driven by the realization that we can no longer afford conflict -- certainly not on the scale of the world conflagrations of the 20th century. Our technologies of destruction have become too powerful. The potential consequences of war are literally apocalyptic. But we cannot simply abandon our systems of value, our beliefs, our cultures, either. I agonized over this apparently intractable problem for months. Was there a 3rd way, invisible to me? I dreamt 1 night during this period that I was suspended in mid-air, clinging to a chandelier, many stories above the ground, directly under the dome of a massive cathedral. The people on the floor below were distant \& tiny. There was a great expanse between me \& any wall -- \& even the peak of the dome itself.

I have learned to pay attention to dreams, not least because of my training as a clinical psychologist. Dreams shed light on the dim places where reason itself has yet to voyage. I have studied Christianity a fair bit, too (more than other religious traditions, although I am always trying to redress this lack). Like others, therefore, I must \& do draw more from what I do know than from what I do not. I knew that cathedrals were constructed in the shape of a cross, \& that the point under the dome was the center of the cross. I knew that the cross was simultaneously, the point of greatest suffering, the point of death \& transformation, \& the symbolic center of the world. That was not somewhere I wanted to be. I managed to get down, out of the heights -- out of the symbolic sky -- back to safe, familiar, anonymous ground. I don't know how. Then, still in my dream, I returned to my bedroom \& my bed \& tried to return to sleep \& the peace of unconsciousness. As I relaxed, however, I could feel my body transported. A great wind was dissolving me, preparing to propel me back to the cathedral, to place me once again at that central point. There was no escape. It was a true nightmare. I forced myself awake. The curtains behind me were blowing in over my pillows. Half asleep, I looked at the foot of the bed. I saw the great cathedral doors. I shook myself completely awake \& they disappeared.

My dream placed me at the center of Being itself, \& there was no escape. It took me months to understand what this meant. During this time, I came to a more complete, personal realization of what the great stories of the past continually insist upon: the center is occupied by the individual. The center is marked by the cross, as X marks the spot. Existence at that cross is suffering \& transformation -- \& that fact, above all, needs to be voluntarily accepted. It is possible to transcend slavish adherence to the group \& its doctrines \&, simultaneously, to avoid the pitfalls of its opposite extreme, nihilism. It is possible, instead, to find sufficient meaning in individual consciousness \& experience.

How could the world be freed from the terrible dilemma of conflict, on the 1 hand, \& psychological \& social dissolution, on the other? The answer was this: through the elevation \& development of the individual, \& through the willingness of everyone to shoulder the burden of Being \& to take the heroic path. We must each adopt as much responsibility as possible for individual life, society \& the world. We must each tell the truth \& repair what is in disrepair \& break down \& recreate what is old \& outdated. It is in this manner that we can \& must reduce the suffering that poisons the world. It's asking a lot. It's asking for everything. But the alternative -- the horror of authoritarian belief, the chaos of the collapsed state, the tragic catastrophe of the unbridled natural world, the existential angst \& weakness of the purposeless individual -- is clearly worse.

I have been thinking \& lecturing about such ideas for decades. I have built up a large corpus of stories \& concepts pertaining to them. I am not for a moment claiming, however, that I am entirely correct or complete in my thinking. Being is far more complicated than 1 person can know, \& I don't have the whole story. I'm simply offering the best I can manage.

In any case, the consequence of all that previous research \& thinking was the new essays which eventually became this book. My initial idea was to write a short essay on all 40 of the answers I had provided to Quora. That proposal was accepted by Penguin Random House Canada. While writing, however, I cut the essay number to 25 \& then to 16 \& then finally, to the current 12. I've been editing that remainder, with the help \& care of my official editor (\& with the vicious \& horribly accurate criticism of Hurwitz, mentioned previously) for the past 3 years.

It took a long time to settle on a little: \textit{12 Rules for Life: An Antidote to Chaos}. Why did that one rise up above all others? 1st \& foremost, because of its simplicity. It indicates clearly that people need ordering principles, \& that chaos otherwise beckons. We require rules, standards, values -- alone \& together. We're pack animals, beasts of burden. We must bear a load, to \fbox{justify our miserable existence}. We require routine \& tradition. That's order. Order can become excessive, \& that's not good, but chaos can swamp us, so we drown -- \& that is also not good. We need to stay on the straight \& narrow path. Each of the 12 rules of this book -- \& their accompanying essays -- therefore provide a guide to being there. ``There'' is the dividing line between order \& chaos. That's where we are simultaneously stable enough, exploring enough, transforming enough, repairing enough, \& cooperating enough. It's there we find the meaning that justifies lief \& its inevitable suffering. Perhaps, if we lived properly, we would be able to tolerate the weight of our own self-consciousness. Perhaps, if we lived properly, we could withstand the knowledge of our own fragility \& mortality, without the sense of aggrieved victimhood that produces, 1st, resentment, then envy, \& then the desire for vengeance \& destruction. Perhaps, if we lived properly, we wouldn't have to turn to totalitarian certainly to shield ourselves from the knowledge of our own insufficiency \& ignorance. Perhaps we could come to avoid those pathways to Hell -- \& we have seen in the terrible 20th century just how real Hell can be.

I hope that these rules \& their accompanying essays will help people understand what they already know: that the soul of the individual eternally hungers for the heroism of genuine being, \& that the willingness to take on that responsibility is identical to the decision to live a meaningful lie.

\fbox{If we each live properly, we will collectively flourish.}

Best wishes to you all, as you proceed through these pages.
\begin{flushright}
	Dr. \textsc{Jordan B. Peterson}
	
	Clinical Psychologist \& Professor of Psychology
\end{flushright}
'' -- \cite[pp. 20--28]{Peterson2018}

\section{Rule 1\texttt{/}Stand up straight with your shoulders back}

\subsection{Lobsters -- \& territory}
``if you are like most people, you don't often think about lobsters\footnote{If you want to do some serious thinking about lobsters, this is a good place to start: Corson, T. (2005). \textit{``The secret life of lobsters: How fishermen \& scientists are unraveling the mysteries of our favorite crustacean}. New York: Harper Perennial.''} -- unless you're eating one. However, these interesting \& delicious crustaceans are very much worth considering. Their nervous systems are comparatively simple, with large, easily observable neurons, the magic cells of the brain. Because of this, scientists have been able to map the neutral circuitry of lobsters very accurately. This has helped us understand the structure \& function of the brain \& behavior of more complex animals, including human beings. Lobsters have more in common with you than you might think (particularly when you are feeling crabby -- ha ha).

Lobsters live on the ocean floor. They need a home base down there, a range within which they hunt for prey \& scavenge around for stray edible bits \& pieces of whatever rains down from the continual chaos of carnage \& death far above. They want somewhere secure, where the hunting \& the gathering is  good. They want a home.

This can present a problem, since there are many lobsters. What if 2 of them occupy the same territory, at the bottom of the ocean, at the same time, \& both want to live there? What if there are hundreds of lobsters, all trying to make a living \& raise a family, in the same crowded patch of sand \& refuse?

Other creatures have this problem, too. When songbirds come north in the spring, e.g., they engage in ferocious territorial disputes. The songs they sing, so peaceful \& beautiful to human ears, are siren calls \& cries of domination. A brilliantly musical bird is a small warrior proclaiming his sovereignty. Take the wren, e.g., a small, feisty, insect-eating songbird common in North America. A newly arrived wren wants a sheltered place to build a nest, away from the wind \& rain. He wants it close to food, \& attractive to potential mates. He also wants to convince competitors for that space to keep their distance.'' -- \cite[pp. 31--32]{Peterson2018}

\subsection{Birds -- \& Territory}
``My dad \& I designed a house for a wren family when I was 10 years gold. It looked like a Conestoga wagon, \& had a front entrance about the size of a quarter. This made it a good house for wrens, who are tiny, \& not so good for other, larger birds, who couldn't get in. My elderly neighbor had a birdhouse, too, which we build for her at the same time, from an old rubber boot. It had an opening large enough for a bird the size of a robin. She was looking forward to the day it was occupied.

A wren soon discovered our birdhouse, \& made himself at home there. We could hear his lengthy, trilling song, repeated over \& over, during the early spring. Once he'd built his nest in the covered wagon, however, our new avian tenant started carrying small sticks to our neighbor's nearby boot. He packed it so full that no other bird, large or small, could possibly get in. Our neighbor was not pleased by this pre-emptive strike, but there was thing to be done about it. ``If we take it down,'' said my dad, ``clean it up, \& put it back in the tree, the wren will just pack it full of sticks again.'' Wrens are small, \& they're cute, but they're merciless.

I had broken my leg skiing the previous winter -- 1st time down the hill -- \& had received some money from a school insurance policy designed to reward unfortunate, clumsy children. I purchased a cassette recorder 9a high-tech novelty at the time) with the proceeds. My dad suggested that i sit on the back lawn, record the wren's song, play it back, \& watch what happened. So, I went out into the bright spring sunlight \& taped a few minutes of the wren laying furious claim to his territory with song. Then I let him hear his own voice. That little bird, $\frac{1}{3}$ the size of a sparrow, began to dive-bomb me \& my cassette recorder, swooping back \& forth, inches from the speaker. We saw a lot of that sort of behavior, even in the absence of the tape recorder. If a larger bird ever dared to sit \& rest in any of the trees near our birdhouse there was a good chance he would get knocked off his perch by a kamikaze\footnote{\textbf{kamikaze} [a] [only before noun] (\textit{from Japanese}) used to describe the way soldiers attack the enemy, knowing that they too will be killed, \textsc{synonym}: \textbf{suicidal}.} wren.

Now, wrens \& lobsters are very different. Lobsters do not fly, sing or perch in trees. Wrens have feathers, not hard shells. Wrens can't breathe underwater, \& are seldom served with butter. However, they are also similar in important ways. Both are obsessed with status \& position, e.g., like a great many creatures. The Norwegian zoologist \& comparative psychologist Thorlief Schjelderup-Ebbe observed (back in 1921) that even common barnyard chickens, establish a ``pecking order.''\footnote{Schjelderup-Ebbe, \& T. (1935). \textit{Social behavior of birds}. Clark University Press. Retrieved from \url{http://psycnet.apa.org/psycinfo/1935-19907-007}; see also Price, J. S., \& Sloman, L. (1987). ``Depression as yielding behavior: An animal model based on Schjelderup-Ebbe's pecking order.'' \textit{Ethology \& Sociobiology}, 8, 85--98.}

The determination of Who's Who in the chicken world has important implications for each individual bird's survival, particularly in times of scarcity. The birds that always have priority access to whatever food is sprinkled out in the yard in the morning are the celebrity chickens. After them come the 2nd-stringers, the hangers-on \& wannabes. Then the 3rd-rate chickens have their turn, \& so on, down to the bedraggled, partially-feathered \& badly-pecked wretches who occupy the lowest, untouchable stratum of the chicken hierarchy.

Chickens, like suburbanites, live communally. Songbirds, such as wrens, do not, but they still inhabit a dominance hierarchy. It's just spread out over more territory. The wiliest, strongest, healthiest \& most fortunate birds occupy prime territory, \& defend it. Because of this, they are more likely to attract high-quality mates, \& to hatch chicks who survive \& thrive. Protection from wind, rain \& predators, as well as easy access to superior food, makes for a much less stressed existence. Territory matters, \& there is little difference between territorial rights \& social status. It is often a matter of life \& death.

If a contagious avian disease sweeps through a neighborhood of well-stratified songbirds, it is the least dominant \& most stressed birds, occupying the lowest rungs of the bird world, who are most likely to sicken \& die.\footnote{Sapolsky, R. M. (2004). ``Social status and health in humans and other animals.'' \textit{Annual Review of Anthropology}, 33, 393--418.} This is equally true of human neighborhoods, when bird flu viruses \& other illnesses sweep across the planet. The poor \& stressed always die 1st, \& in greater numbers. They are also much more susceptible to noninfectious diseases, such as cancer, diabetes \& heart disease. When the aristocracy catches a cold, as it is said, the working class dies of pneumonia.

Because territory matters, \& because the best locales are always in short supply, territory-seeking among animals produces conflict. Conflict, in turn, produces another problem: how to win or lose without the disagreeing parties incurring too great a cost. This latter point is particularly important. Imagine that 2 birds engage in a squabble about a desirable nesting area. The interaction can easily degenerate into outright physical combat. Under such circumstances, 1 bird, usually the largest, will eventually win -- but even the victor may be hurt by the fight. That means a 3rd bird, an undamaged, canny bystandard, can move in, opportunistically, \& defeat the now-crippled victor. That is not at all a good deal for the 1st 2 birds.'' -- \cite[pp. 32--34]{Peterson2018}

\subsection{Conflict -- \& Territory}
``Over the millennia, animals who must co-habit with others in the same territories have in consequence learned many tricks to establish dominance, while risking the least amount of possible damage. A defeated wolf, e.g., will roll over on its back, exposing its throat to the victor, who will not then design to tear it out. The now-dominant wolf may still require a future hunting partner, after all, even one as pathetic as his now-defeated foe. Bearded dragons, remarkable social lizards, wave their front legs peaceably at one another to indicate their wish for continued social harmony. Dolphins produce specialized sound pulses while hunting \& during other times of high excitement to reduce potential conflict among dominant \& subordinate group members. Such behavior is endemic in the community of living things.

Lobsters, scuttling around on the ocean floor, are no exception. If you catch a few dozen, \& transport them to a new location, you can observe their status-forming rituals \& techniques. Each lobster will 1st begin to explore the new territory, partly to map its details, \& partly to find a good place for shelter. Lobsters learn a lot about where they live, \& they remember what they learn. If you startle one near its nest, it will quickly zip back \& hide there. If you startle it some distance away, however, it will immediately dart towards the nearest suitable shelter, previously identified \& now remembered.

A lobster needs a safe hiding place to rest, free from predators \& the forces of nature. Furthermore, as lobsters grow, they moult, or shed their shells, which leaves them soft \& vulnerable for extended periods of time. A burrow under a rock makes a good lobster home, particularly if it is located where shells \& other detritus can be dragged into place to cover the entrance, once the lobster is snugly ensconced inside. However, there may be only a small number of high-quality shelters or hiding places in each new territory. They are scarce \& valuable. Other lobsters continually seek them out.

This means that lobsters often encounter one another when out exploring. Researchers have demonstrated that even a lobster raised in isolation knows what to do when such a think happens.\footnote{Kravitz, E.A. (2000). ``Serotonin and aggression: Insights gained from a lobster model system and speculations on the role of amine neurons in a complex behavior.'' Journal of Comparative Physiology, 186, 221--238.} It has complex defensive \& aggressive behaviors built right into its nervous system. It begins to dance around, like a boxer, opening \& raising its claws, moving backward, forward, \& side to side, mirroring its opponent, waving its opened claws back \& forth. At the same time, it employs special jets under its eyes to direct streams of liquid at its opponent. The liquid spray contains a mix of chemicals that tell the other about its size sex, health, \& mood.

Sometimes 1 lobster can tell immediately from the display of claw size that it is much smaller than its opponent, \& will back down without a fight. The chemical information exchanged in the spray can have the same effect, convincing a less healthy or less aggressive lobster to retreat. That's dispute resolution Level 1.\footnote{Huber, R., \& Kravitz, E. A. (1995). ``A quantitative analysis of agonistic behavior in juvenile American lobsters (Homarus americanus L.)''. \textit{Brain, Behavior \& Evolution}, 46, 72--83.} If 2 lobsters are very close in size \& apparent ability, however, or if the exchange of liquid has been insufficiently informative, they will proceed to dispute resolution Level 2. With antennae whipping madly \& claws folded downward, one will advance, \& the other retreat. Then the defender will advance, \& the aggressor retreat. After a couple of rounds of this behavior, the more nervous of the lobsters may feel that continuing is not in his best interest. He will flick his tail reflexively, dart backwards, \& vanish, to try his luck elsewhere. If neither blinks, however, the lobsters move to Level 3, which involves genuine combat.

This time, the now enraged lobsters come at each other viciously, with their claws extended, to grapple. Each tries to flip the other on its back. A successfully flipped lobster will conclude that its opponent is capable of inflicting serious damage. It generally gives up \& leaves (although it harbors intense resentment \& gossips endlessly about the victor behind its back). If neither can overturn the other -- or if one will not quit despite being flipped -- the lobsters move to Level 4. Doing so involves extreme risk, \& is not something to be engaged in without forethought: one or both lobsters will emerge damaged from the ensuing fray, perhaps fatally.

The animals advance on each other, with increasing speed. Their claws are open, so they can grab a leg, or antenna, or an eye-stalk, or anything else exposed \& vulnerable. Once a body part has been successfully grabbed, the grabber will tail-flick backwards, sharply, with claw clamped firmly shut, \& try to tear it off. Disputes that have escalated to this point typically create a clear winner \& loser. The loser is unlikely to survive, particularly if he or she remains in the territory occupied by the winner, now a mortal enemy.

In the aftermath of a losing battle, regardless of how aggressively a lobster has behaved, it becomes unwilling to fight further, even against another, previously defeated opponent. A vanquished competitor loses confidence, sometimes for days. Sometimes the defeat can have even more severe consequences. If a dominant lobster is badly defeated, its brain basically dissolves. Then it grows a new, subordinate's brain -- one more appropriate to its new, lowly position.\footnote{Yeh S-R, Fricke RA, Edwards DH (1996) ``The effect of social experience on serotonergic modulation of the escape circuit of crayfish.'' Science, 271, 366--369.} Its original brain just isn't sophisticated to manage the transformation from king to bottom dog without virtually complete dissolution \& regrowth. Anyone who has experienced a painful transformation after a serious defeat in romance or career may feel some sense of kinship with the once successful crustacean.'' -- \cite[pp. 34--36]{Peterson2018}

\subsection{The Neurochemistry of Defeat \& Victory}
``A lobster loser's brain chemistry differs importantly from that of a lobster winner. This is reflected in their relative postures. Whether a lobster is confident or cringing depends on the ratio of 2 chemicals that modulate communication between lobster neurons: serotonin \& octapamine. Winning increases the ratio of the former to the latter.

A lobster with high levels of serotonin \& low levels of octopamine is a cocky, strutting sort of shellfish, much less likely to back down when challenged. This is because serotonin helps regulate postural flexion. A flexed lobster extends its appendages so that it can look tall \& dangerous, like Clint Eastwood in a spaghetti Western. When a lobster that has just lost a battle is exposed to serotonin, it will stretch itself out, advance even on former victors, \& fight longer \& harder.\footnote{Huber, R., Smith, K., Delago, A., Isaksson, K., \& Kravitz, E. A. (1997). ``Serotonin and aggressive motivation in crustaceans: Altering the decision to retreat.'' \textit{Proceedings of the National Academy of Sciences of the United States of America}, 94, 5939--42.} The drugs prescribed to depressed human beings, which are selective serotonin reuptake inhibitors, have much the same chemical \& behavioral effect. In 1 of the more staggering demonstrations of the evolutionary continuity of life on Earth, Prozac even cheers up lobsters.\footnote{Antonsen, B. L., \& Paul, D. H. (1997). ```Serotonin and octopamine elicit stereotypical agonistic behaviors in the squat lobster Munida quadrispina (Anomura, Galatheidae).'' \textit{Journal of Comparative Physiology A: Sensory, Neural, and Behavioral Physiology}, 181, 501--510.}

High serotonin\texttt{/}low octopamine characterizes the victor. The opposite neurochemical configuration, a high ratio of octopamine to serotonin, produces a defeated-looking, scrunched-up, inhibited, drooping, skulking sort of lobster, very likely to hang around street corners, \& to vanish at the 1st hint of trouble. Serotonin \& octopamine also regulate the tail-flick reflex, which serves to propel a lobster rapidly backwards when it needs to escape. Less provocation is necessary to trigger that reflex in a defeated lobster. You can see an echo of that in the heightened startle reflex characteristic of the soldier or battered child with post-traumatic stress disorder.'' -- \cite[pp. 36--37]{Peterson2018}

\subsection{The Principle of Unequal Distribution}
``$\ldots$''-- \cite[pp. 37--38]{Peterson2018}

\section{Rule 2\texttt{/}Treat yourself like someone you are responsible for helping}

\section{Rule 3\texttt{/}Make friends with people who want the best for you}

\section{Rule 4\texttt{/}Compare yourself to who you were yesterday, not to who someone else is today}

\section{Rule 5\texttt{/}Do not let your children do anything that makes you dislike them}

\section{Rule 6\texttt{/}Set your house in perfect order before you criticize the world}

\section{Rule 7\texttt{/}Pursue what is meaningful (not what is expedient)}

\section{Rule 8\texttt{/}Tell the truth -- or, at least, don't lie}

\section{Rule 9\texttt{/}Assume that the person you are listening to might know something you don't}

\section{Rule 10\texttt{/}Be precise in your speech}

\section{Rule 11\texttt{/}Do not bother children when they are skateboarding}

\section{Rule 12\texttt{/}Pet a cat when you encounter one on the street}

%------------------------------------------------------------------------------%

\chapter{Miscellaneous}

\section{Young, Dumb, \& Broke}
Watch \& listen \href{https://www.youtube.com/watch?v=IPfJnp1guPc}{Youtube\texttt{/}Khalid\texttt{/}Young Dumb \& Broke}.

\section{Existential Crisis}

\section{Meaning of Life?}

\section{Art of Balancing in Life?}

%------------------------------------------------------------------------------%

\selectlanguage{vietnamese}

\begin{thebibliography}{99}
	\bibitem[NQBH\texttt{/}psychology]{NQBH/psychology} \selectlanguage{vietnamese} Nguyễn Quản Bá Hồng. \href{https://github.com/NQBH/hobby/blob/master/psychology/NQBH_a_personal_journey_to_psychology.pdf}{\textit{A Personal Journey to Psychology: The Way I Perceive}}. March 2022--now.
\end{thebibliography}

\selectlanguage{english}

\printbibliography[heading=bibintoc]
	
\end{document}