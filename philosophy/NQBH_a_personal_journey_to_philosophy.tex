\documentclass[oneside]{book}
\usepackage[backend=biber,natbib=true,style=authoryear]{biblatex}
\addbibresource{/home/hong/1_NQBH/reference/bib.bib}
\usepackage[vietnamese,english]{babel}
\usepackage{tocloft}
\renewcommand{\cftsecleader}{\cftdotfill{\cftdotsep}}
\usepackage[colorlinks=true,linkcolor=blue,urlcolor=red,citecolor=magenta]{hyperref}
\usepackage{algorithm,algpseudocode,amsmath,amssymb,amsthm,float,graphicx,mathtools}
\allowdisplaybreaks
\numberwithin{equation}{section}
\newtheorem{assumption}{Assumption}[chapter]
\newtheorem{conjecture}{Conjecture}[chapter]
\newtheorem{corollary}{Corollary}[chapter]
\newtheorem{definition}{Definition}[chapter]
\newtheorem{example}{Example}[chapter]
\newtheorem{lemma}{Lemma}[chapter]
\newtheorem{notation}{Notation}[chapter]
\newtheorem{principle}{Principle}[chapter]
\newtheorem{problem}{Problem}[chapter]
\newtheorem{proposition}{Proposition}[chapter]
\newtheorem{question}{Question}[chapter]
\newtheorem{remark}{Remark}[chapter]
\newtheorem{theorem}{Theorem}[chapter]
\usepackage[left=0.5in,right=0.5in,top=1.5cm,bottom=1.5cm]{geometry}
\usepackage{fancyhdr}
\pagestyle{fancy}
\fancyhf{}
\lhead{\small \textsc{Sect.} ~\thesection}
\rhead{\small \nouppercase{\leftmark}}
\renewcommand{\sectionmark}[1]{\markboth{#1}{}}
\cfoot{\thepage}
\def\labelitemii{$\circ$}

\title{A Personal Journey to Philosophy}
\author{\selectlanguage{vietnamese} Nguyễn Quản Bá Hồng}
\date{\today}

\begin{document}
\maketitle
\tableofcontents

%------------------------------------------------------------------------------%

\chapter*{Foreword}

A \textit{personal} journey to philosophy -- the hardest subject I have ever face to \& fight against. A collection of quotes from different resources, e.g., philosophical books, websites, forums, and Facebook philosophical pages, etc., and some \textit{personal} (again) thoughts about them.

%------------------------------------------------------------------------------%

\chapter{Jordan B. Peterson. 12 Rules for Life: An Antidote to Chaos}

\cite{Peterson2018}\footnote{\textbf{antidote} [n] \textbf{1.} \textbf{antidote (to something)} a substance that controls the effects of a poison or disease; \textbf{2.} \textbf{antidote (to something)} anything that takes away the effects of something unpleasant.} \footnote{\textbf{chaos} [n] [uncountable] a state of complete confusion \& lack of order; in physics, \textbf{chaos} is the property of a complex system whose behavior is so unpredictable that it appears random, especially because small changes in conditions can have very large effects; \textbf{chaos theory} is the branch of mathematics that deals with these complex systems.}
\begin{quotation}
	``The most influential public intellectual in the Western world right now.'' -- New York Times
\end{quotation}

``Rules? More rules? Really? Isn't life complicated enough, restricting enough, without abstract rules that don't take our unique, individual situations into account? \& given  that our brains are plastic, \& all develop differently based on our life experiences, why even expect that a few rules might be helpful to us all?'' -- \cite[Foreword]{Peterson2018}

%------------------------------------------------------------------------------%

\chapter{Miscellaneous}

\section{Young, Dumb, \& Broke}
Watch \& listen \href{https://www.youtube.com/watch?v=IPfJnp1guPc}{Youtube\texttt{/}Khalid\texttt{/}Young Dumb \& Broke}.

\section{Existential Crisis}

\section{Meaning of Life?}

\section{Art of Balancing in Life?}

%------------------------------------------------------------------------------%

\begin{thebibliography}{99}
	\bibitem[NQBH\texttt{/}psychology]{NQBH/psychology} \selectlanguage{vietnamese} Nguyễn Quản Bá Hồng. \href{https://github.com/NQBH/hobby/blob/master/psychology/NQBH_a_personal_journey_to_psychology.pdf}{\textit{A Personal Journey to Psychology: The Way I Perceive}}. March 2022--now.
\end{thebibliography}

\printbibliography[heading=bibintoc]
	
\end{document}