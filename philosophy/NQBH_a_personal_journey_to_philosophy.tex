\documentclass[oneside]{book}
\usepackage[backend=biber,natbib=true,style=authoryear]{biblatex}
\addbibresource{/home/hong/1_NQBH/reference/bib.bib}
\usepackage[english,vietnamese]{babel}
\usepackage{tocloft}
\renewcommand{\cftsecleader}{\cftdotfill{\cftdotsep}}
\usepackage[colorlinks=true,linkcolor=blue,urlcolor=red,citecolor=magenta]{hyperref}
\usepackage{algorithm,algpseudocode,amsmath,amssymb,amsthm,float,graphicx,mathtools}
\allowdisplaybreaks
\numberwithin{equation}{section}
\newtheorem{assumption}{Assumption}[chapter]
\newtheorem{conjecture}{Conjecture}[chapter]
\newtheorem{corollary}{Corollary}[chapter]
\newtheorem{definition}{Definition}[chapter]
\newtheorem{example}{Example}[chapter]
\newtheorem{lemma}{Lemma}[chapter]
\newtheorem{notation}{Notation}[chapter]
\newtheorem{principle}{Principle}[chapter]
\newtheorem{problem}{Problem}[chapter]
\newtheorem{proposition}{Proposition}[chapter]
\newtheorem{question}{Question}[chapter]
\newtheorem{remark}{Remark}[chapter]
\newtheorem{theorem}{Theorem}[chapter]
\usepackage[left=0.5in,right=0.5in,top=1.5cm,bottom=1.5cm]{geometry}
\usepackage{fancyhdr}
\pagestyle{fancy}
\fancyhf{}
\lhead{\small \textsc{Sect.} ~\thesection}
\rhead{\small \nouppercase{\leftmark}}
\renewcommand{\sectionmark}[1]{\markboth{#1}{}}
\cfoot{\thepage}
\def\labelitemii{$\circ$}

\title{A Personal Journey to Philosophy}
\author{Nguyễn Quản Bá Hồng}
\date{\today}

\begin{document}
\maketitle
\selectlanguage{english}
\tableofcontents

%------------------------------------------------------------------------------%

\chapter*{Foreword}

A \textit{personal} journey to philosophy -- the hardest subject I have ever faced to \& fought against. A collection of quotes from different resources, e.g., philosophical books, websites, forums, and Facebook philosophical pages, etc., and some \textit{personal} (again) thoughts about them.

%------------------------------------------------------------------------------%

\chapter*{Basic Terminologies}
\begin{itemize}
	\item \textbf{philosophy} [n] \textbf{1.} [uncountable] the study of the nature \& meaning of the universe \& of human life; \textbf{natural philosophy} is an old term for the study of the physical world, which developed into the natural sciences; \textbf{2.} [countable] a particular set or system of beliefs resulting from the search for knowledge about life \& the universe; \textbf{3.} [countable] a set of beliefs or an attitude to life that guides somebody's behavior.
\end{itemize}

%------------------------------------------------------------------------------%

\chapter{Jordan B. Peterson. \textit{12 Rules for Life: An Antidote to Chaos}}

\section*{Introduction}
``\textit{12 Rules for Life: An Antidote\footnote{\textbf{antidote} [n] \textbf{1.} \textbf{antidote (to something)} a substance that controls the effects of a poison or disease; \textbf{2.} \textbf{antidote (to something)} anything that takes away the effects of something unpleasant.} to Chaos\footnote{\textbf{chaos} [n] [uncountable] a state of complete confusion \& lack of order; in physics, \textbf{chaos} is the property of a complex system whose behavior is so unpredictable that it appears random, especially because small changes in conditions can have very large effects; \textbf{chaos theory} is the branch of mathematics that deals with these complex systems.}} is a 2018 \href{https://en.wikipedia.org/wiki/Self-help_book}{self-help book} by the Canadian clinical\footnote{\textbf{clinical} [a] [only before noun] connected with the examination \& treatment of patients \& their illnesses.} psychologist\footnote{\textbf{psychologist} [n] a scientist who studies psychology.} \href{https://en.wikipedia.org/wiki/Jordan_Peterson}{Jordan Peterson}. It provides life advice through essays in abstract ethical\footnote{\textbf{ethical} [a] \textbf{1.} connected with beliefs \& principles about what is right \& wrong; \textbf{2.} morally correct or acceptable.} principles, psychology, mythology\footnote{\textbf{mythology} [n] [uncountable, countable] \textbf{1.} ancient myths in general; the ancient myths of a particular culture, society, etc.; \textbf{2.} \textbf{mythology (of something)} ideas that many people think are true but are in fact false.}, religion\footnote{\textbf{religion} [n] \textbf{1.} [uncountable] the belief in the existence of a god or gods, \& the activities that are connected with the worship of them; \textbf{2.} [countable] 1 of the systems of belief that are based on the belief in the existence of a particular god or gods.}, \& personal anecdotes\footnote{\textbf{anecdote} [n] [countable, uncountable] \textbf{1.} \textbf{anecdote (about somebody\texttt{/}something)} a short, interesting or funny story about a real person or event; \textbf{2.} a personal account of an event, especially one that is considered as possibly not true or accurate.}.''[$\ldots$] ``The book is written in a more accessible style than his previous academic book, \href{https://en.wikipedia.org/wiki/Maps_of_Meaning:_The_Architecture_of_Belief}{Maps of Meaning: The Artchitecture of Belief} (1999). A sequel, \href{https://en.wikipedia.org/wiki/Beyond_Order}{Beyond Order: 12 More Rules for Life}, was published in Mar 2021.'' -- \href{https://en.wikipedia.org/wiki/12_Rules_for_Life}{Wikipedia\texttt{/}12 Rules for Life}

\subsection*{Overview}

\paragraph*{Background.} ``Peterson's interest in writing the book grew out of a personal hobby of answering questions posted on \href{https://en.wikipedia.org/wiki/Quora}{Quora}; 1 such question being
\begin{question}
	\fbox{``What are the most valuable things everyone should know?'',}
\end{question}
to which his answer comprised 42 rules. The early vision \& promotion of the book aimed to include all rules, with the title ``42''. Peterson stated that it ``isn't only written for other people. It's warning to me.'''' -- \href{https://en.wikipedia.org/wiki/12_Rules_for_Life#Background}{Wikipedia\texttt{/}12 Rules for Life\texttt{/}overview\texttt{/}background}

\paragraph*{12 Rules.} ``The book is divided into chapters with each title representing 1 of the following 12 specific rules for life as explained through an essay.
\begin{enumerate}
	\item ``Stand up straight with your shoulders back.''
	\item ``Treat yourself like you are someone you are responsible for helping.''
	\item ``Make friends with people who want the best for you.''
	\item ``Compare yourself to who you were yesterday, not to who someone else is today.''
	\item ``Do not let your children do anything that makes you dislike them.''
	\item ``Set your house in perfect order before you criticize the world.''
	\item ``Pursue what is meaningful (not what is expedient\footnote{\textbf{expedient} [n] an action that is useful or necessary for a particular purpose, but not always fair or right.}).''
	\item ``Tell the truth -- or, at least, don't lie.''
	\item ``Assume that the person you are listening to might know something you don't.''
	\item ``Be precise in your speech.''
	\item ``Do not bother children when they are skate-boarding.''
	\item ``Pet a cat when you encounter\footnote{\textbf{encounter} [v] \textbf{1.} \textbf{encounter something} to experience something, especially something unpleasant or difficult, while you are trying to do something else, \textsc{synonym}: \textbf{run into something}; \textbf{2.} \textbf{encounter something\texttt{/}somebody} to discover or experience something, or meet somebody, especially something\texttt{/}somebody new, unusual or unexpected, \textsc{synonym}: \textbf{come across somebody\texttt{/}something}; [n] a meeting, especially one that is sudden or unexpected.} one on the street.'''' -- \href{https://en.wikipedia.org/wiki/12_Rules_for_Life#12_Rules}{Wikipedia\texttt{/}12 Rules for Life\texttt{/}overview\texttt{/}content}
\end{enumerate} 

\paragraph*{Content.} ``The book's central idea is that ``\fbox{suffering is built into the structure of \href{https://en.wikipedia.org/wiki/Being}{being}}'' \& although it can be unbearable\footnote{\textbf{unbearable} [a] too painful, annoying or unpleasant to deal with or accept, \textsc{synonym}: \textbf{intolerable}, \textsc{opposite}: \textbf{bearable}.}, people have a choice either to withdraw\footnote{\textbf{withdraw} [v] \textbf{1.} [transitive, intransitive] (used especially about armed forces) to make people leave a place; to leave a place; \textbf{2.} [intransitive] \textbf{withdraw (to something)} to leave a room; to go away from other people; \textbf{3.} [transitive] to move something back, out or away from something; \textbf{4.} [transitive] to take money out of a bank account or financial institution; \textbf{5.} [intransitive] to stop taking part in something; \textbf{6.} [intransitive] to stop wanting to speak to, or be with, other people; \textbf{7.} [transitive] to no longer provide or offer something; to no longer make something available; \textbf{8.} [transitive] \textbf{withdraw something} to say that you no longer agree with what you said before.}, which is a ``suicidal\footnote{\textbf{suicidal} [a] (of people) very unhappy or depressed \& feeling that they want to kill themselves; (of behavior) showing this.} gesture\footnote{\textbf{gesture} [n] \textbf{1.} [countable, uncountable] \textbf{gesture (of something)} something that you do or say to show a particular feeling or intention; \textbf{2.} [countable, uncountable] a movement that you make with your hands, your head or your face to show a particular meaning.}'', or to face \& transcend\footnote{\textbf{transcend} [v] \textbf{transcend something} to be or go beyond the usual limits of something.} it. Living in a world of chaos \& order,\fbox{ everyone has ``darkness''} that can \fbox{``turn them into the monsters they're capable of being''} to satisfy their \fbox{dark impulses\footnote{\textbf{impulse} [n] \textbf{1.} [countable, usually singular, uncountable] a sudden strong wish or need to do something, without stopping to think about the results; \textbf{2.} [countable, usually singular] something that causes somebody\texttt{/}something to do something or to develop \& make progress; \textbf{3.} [countable] a brief electric current, e.g. one that travels from a nerve to a muscle; \textbf{4.} [countable] (\textit{physics}) the change in momentum of an object due to a force.} in the right situations}. Scientific experiments like the \href{https://en.wikipedia.org/wiki/Inattentional_blindness#Invisible_Gorilla_Test}{Invisible Gorilla Test} show that perception\footnote{\textbf{perception} [n] \textbf{1.} [uncountable, countable] an idea, a belief or an image you have as a result of how you see or understand something; \textbf{2.} [uncountable] the way you notice things or the ability to notice things with the senses; in biology, \textbf{perception} refers to the processes in the nervous system by which a living thing becomes aware of events \& things outside itself; \textbf{3.} [uncountable] the ability to understand the true nature of something, \textsc{synonym}: \textbf{insight}.} is adjusted to aims, \& it is \fbox{better to seek \href{https://en.wikipedia.org/wiki/Meaning_(psychology)}{meaning} rather than happiness}. Peterson notes:
\begin{quotation}
	``It's all very well to think the meaning of life is happiness, but what happens when you're unhappy? Happiness is a great side effect. When it comes, accept it gratefully\footnote{\textbf{grateful} [a] \textbf{1.} feeling or showing thanks because somebody has done something kind for you or has done as you asked; \textbf{2.} used to make a request, especially in a letter or in a formal situation.}. But it's fleeting\footnote{\textbf{fleeting} [a] [usually before noun] lasting only a short time, \textsc{synonym}: \textbf{brief}.} \& unpredictable\footnote{\textbf{unpredictable} [a] that cannot be predicted because it changes a lot or depends on too many different things, \textsc{opposite}: \textbf{predictable}.}. It's not something to aim at -- because it's not an aim. \& if happiness is the purpose of life, what happens when you're unhappy? Then you're a failure.''
\end{quotation}
The book advances the idea that \fbox{people are born with an instinct\footnote{\textbf{instinct} [n] [uncountable, countable] a natural tendency for people \& animals to behave in a particular way, using the knowledge \& abilities that they were born with rather than thought or training.} for ethics \& meaning}, \& should take responsibility\footnote{\textbf{responsibility} [n] \textbf{1.} [uncountable, countable] a duty to deal with or take care of somebody\texttt{/}something, so that you may be blamed if something goes wrong; \textbf{2.} [uncountable] \textbf{responsibility (for something)} blame for something bad that has happened; \textbf{3.} [countable, uncountable] a moral duty to behave well with regard to somebody\texttt{/}something.} to search for meaning above their own interests (Rule 7, ``Pursue what is meaningful, not what is expedient''). Such thinking is reflected both in contemporary\footnote{\textbf{contemporary} [a] \textbf{1.} belonging to the present time, \textsc{synonym}: \textbf{modern}; \textbf{2.} (especially of people \& society) belonging to the same time as somebody\texttt{/}something else; [n] a person or thing living or existing at the same time as somebody\texttt{/}something else, especially somebody who is about the same age as somebody else.} stories e.g. \href{https://en.wikipedia.org/wiki/Pinocchio_(1940_film)}{Pinocchio}, \href{https://en.wikipedia.org/wiki/The_Lion_King}{The Lion King}, \& \href{https://en.wikipedia.org/wiki/Harry_Potter}{Harry Potter}, \& in ancient stories from the \href{https://en.wikipedia.org/wiki/Bible}{Bible}. To ``stand up straight with your shoulders back'' (Rule 1) is to ``accept the terrible responsibility of life'', to make self-sacrifice\footnote{\textbf{self-sacrifice} [n] [uncountable] (\textit{approving}) the act of not allowing yourself to have or do something in order to help other people.}, because the individual must rise above \href{https://en.wikipedia.org/wiki/Victimisation}{victimization}\footnote{\textbf{victimize} [v] [often passive] \textbf{victimize somebody} to make somebody suffer unfairly because you do not like them, their opinions or something that they have done.} \& ``conduct his or her life in a manner that requires the rejection\footnote{\textbf{rejection} [n] [uncountable, countable] \textbf{1.} the act of refusing to accept or consider something; \textbf{2.} the act of refusing to accept somebody for a job or position; \textbf{3.} the decision not to use, sell, publish, etc. something because its quality is not good enough; \textbf{4.} \textbf{rejection (of something)} an occasion when somebody's body does not accept a new organ after a transplant operation, by producing substances that attack the organ; \textbf{5.} the act of failing to give a person or an animal enough care or affection.} of immediate gratification\footnote{\textbf{gratification} [n] [uncountable, countable] (\textit{formal}) the state of feeling pleasure when something goes well for you or when your desires are satisfied; something that gives you pleasure, \textsc{synonym}: \textbf{satisfaction}.}, of natural \& perverse\footnote{\textbf{perverse} [a] showing a deliberate \& determined desire to behave in a way that most people think is wrong, unacceptable or unreasonable.} desires alike.'' The comparison to \href{https://en.wikipedia.org/wiki/Neurology}{neurological}\footnote{\textbf{neurological} [a] relating to nerves or to the science of neurology.} structures \& behavior of \href{https://en.wikipedia.org/wiki/Lobsters}{lobsters} is used as a natural example to the formation\footnote{\textbf{formation} [n] \textbf{1.} [uncountable] the action of forming something; the process of being formed; \textbf{2.} [countable] a thing that has been formed, especially in a particular place or in a particular way; \textbf{3.} [countable, uncountable] a particular arrangement or pattern of people or things.} of \href{https://en.wikipedia.org/wiki/Hierarchy}{social hierarchies}\footnote{\textbf{hierarchy} [n] \textbf{1.} [countable, uncountable] a system, especially in a society or an organization, in which people are organized into different levels of importance from highest to lowest; \textbf{2.} [countable] a system that ideas or beliefs can be arranged into.}.

The other parts of the work explore \& criticize the state of young men; the upbringing\footnote{\textbf{upbringing} [n] [singular, uncountable] the way in which a child is cared for \& taught how to behave while it is growing up.} that ignores \href{https://en.wikipedia.org/wiki/Sex_differences_in_humans}{sex differences} between boys \& girls (criticism of \href{https://en.wikipedia.org/wiki/Overprotective}{over-protection} \& \href{https://en.wikipedia.org/wiki/Tabula_rasa}{tabula rasa} model in \href{https://en.wikipedia.org/wiki/Social_science}{social sciences}); male-female \href{https://en.wikipedia.org/wiki/Interpersonal_relationship}{interpersonal relationships}; \href{https://en.wikipedia.org/wiki/School_shooting}{school shootings}; religion \& moral \href{https://en.wikipedia.org/wiki/Nihilism}{nihilism}\footnote{\textbf{nihilism} [n] [uncountable] (\textit{philosophy}) the belief that life has no meaning or purpose \& that religious \& moral principles have no value.}; \href{https://en.wikipedia.org/wiki/Relativism}{relativism}\footnote{\textbf{relativism} [n] [uncountable] the belief that truth is not always \& generally valid, but can be judged only in relation to other things, e.g. your personal situation.}; \& lack of respect for the values that built \href{https://en.wikipedia.org/wiki/Western_world}{Western society}.

In the last chapter, Peterson outlines the ways in which one can cope with the most tragic\footnote{\textbf{tragic} [a] \textbf{1.} making you feel very sad, usually because somebody has died or suffered a lot; \textbf{2.} [usually before noun] connected with tragedy ($=$ the style of literature).} events, events that are often \fbox{out of one's control}. In it, he describes his own personal struggle upon discovering that his daughter, Mikhaila, had a rare bone disease. The chapter is a meditation\footnote{\textbf{meditation} [n] \textbf{1.} [uncountable] the practice of thinking deeply, usually in silence, especially for religious reasons or in order to make your mind calm; \textbf{2.} [countable, usually plural] \textbf{meditation (on something)} serious thoughts on a particular subject that somebody writes down or speaks.} on how to maintain\footnote{\textbf{maintain} [v] \textbf{1.} \textbf{maintain something} to cause or enable a condition or situation to continue, \textsc{synonym}: \textbf{preserve}; \textbf{2.} \textbf{maintain something} to keep something at the same level or rate; \textbf{3.} to state strongly that something is true, even when some other people may not believe it; \textbf{4.} \textbf{maintain somebody\texttt{/}something} to support somebody\texttt{/}something over a long period of time by providing money, paying for food, etc.; \textbf{5.} \textbf{maintain something} to keep a building, machine, etc. in good condition by checking or repairing it regularly; \textbf{6.} \textbf{maintain a record} to write something down as a record \& keep adding the most recent information, \textsc{synonym}: \textbf{keep}.} a watchful\footnote{\textbf{watchful} [a] paying attention to what is happening in case of danger, accidents, etc.} eye on, and cherish\footnote{\textbf{cherish} [v] (\textit{formal}) \textbf{1.} \textbf{cherish somebody\texttt{/}something} to love somebody\texttt{/}something very much \& want to protect them or it; \textbf{2.} \textbf{cherish something} to keep an idea, a hope or a pleasant feeling in your mind for a long time.}, life's small redeemable\footnote{\textbf{redeemable} [a] \textbf{redeemable (against something)} that can be exchanged for money or goods.} qualities (i.e., ``pet a cat when you encounter one''). It also outlines a practical way to deal with hardship\footnote{\textbf{hardship} [n] [uncountable, countable] a situation that is difficult \& unpleasant because you do not have enough money, food, clothes, etc.}: to shorten one's temporal\footnote{\textbf{temporal} [a] \textbf{1.} connected with or limited by time; \textbf{2.} connected with the real physical world, not spiritual matters; \textbf{3.} (\textit{anatomy}) near the temples at the side of the head.} scope of responsibility (e.g., focusing on the next minute rather than the next 3 months).

Canadian psychiatrist and psychoanalyst \href{https://en.wikipedia.org/wiki/Norman_Doidge}{Norman Doidge} wrote \cite{Peterson2018}'s foreword.'' -- \href{https://en.wikipedia.org/wiki/12_Rules_for_Life#Content}{Wikipedia\texttt{/}12 Rules for Life\texttt{/}overview\texttt{/}content}

\begin{quotation}
	``The most influential public intellectual\footnote{\textbf{intellectual} [a] [usually before noun] connected with or using a person's ability to think in a logical way \& understand things, \textsc{synonym}: \textbf{mental}; [n] a person who is well educated \& enjoys activities in which they have to think seriously about things.} in the Western world right now.'' -- New York Times
\end{quotation}

\section*{Foreword}
``Rules? More rules? Really? Isn't life complicated\footnote{\textbf{complicated} [a] \textbf{1.} made of many different things or parts that are connected; difficult to understand, \textsc{synonym}: \textbf{complex}, \textsc{opposite}: \textbf{uncomplicated}; \textbf{2.} (of a medical condition) involving complications, \textsc{opposite}: \textbf{uncomplicated}.} enough, restricting enough, without abstract rules that don't take our unique, individual situations into account? \& given that our brains are plastic\footnote{\textbf{plastic} [n] \textbf{1.} [uncountable, countable, usually plural] a light strong material that is produced by chemical processes \& can be formed into shapes when heated. There are many different types of plastic, used to make different objects \& fabrics; \textbf{2. (plastics)} [uncountable] the science of making plastics; [a] \textbf{1.} made of plastic; \textbf{2.} (of a material or substance) easily formed into different shapes; \textbf{3.} (\textit{biology}) (of a living thing) able to adapt to change or variety in the environment.}, \& all develop differently based on our life experiences, why even expect that a few rules might be helpful to us all?

People don't clamor\footnote{\textbf{clamor} [v] \textbf{1.} [intransitive, transitive] (\textit{formal}) to demand something loudly; \textbf{2.} [intransitive] (of many people) to shout loudly, especially in a confused way; [n] (\textit{formal}) \textbf{1.} [singular] a loud noise, especially on that is made by a lot of people or animals; \textbf{2.} [uncountable, countable] \textbf{clamor (for something)} a demand for something made by a lot of people.} for rules, even in the Bible $\ldots$ as when Moses comes down the mountain, after a long absence\footnote{\textbf{absence} [n] \textbf{1.} [uncountable] the fact of somebody\texttt{/}something not existing or not being available, \textsc{synonym}: \textbf{lack}, \textsc{opposite}: \textbf{presence}; \textbf{2.} [uncountable, countable] the fact of somebody being away from a place where they are usually expected to be; the occasion or period of time when somebody is away.}, bearing the tablets\footnote{\textbf{tablet} [n] \textbf{1.} (\textit{especially British English}) a small round solid piece of medicine that you swallow, \textsc{synonym}: \textbf{pill}; \textbf{2.} a flat piece of stone, etc. with words or symbols on it; \textbf{3.} (also \textbf{tablet computer}) (\textit{trademark} in the UK) a small, light, flat computer that can be used without a keyboard or mouse, by touching the screen.} inscribed\footnote{\textbf{inscribe} [v] \textbf{1.} [often passive] to write or cut words, your name, etc. onto something; \textbf{2.} [often passive] \textbf{inscribe something $+$ adv.\texttt{/}prep.} to make something present in, on, etc. something.} with 10 commandments\footnote{\textbf{commandment} [n] a law given by God, especially any of \textbf{the Ten Commandments} given to the Jews in the Bible.}, \& finds the Children of Israel in revelry\footnote{\textbf{revelry} [n] [uncountable] noisy fun, usually involving a lot of eating \& drinking, \textsc{synonym}: \textbf{festivity, merrymaking}.}. They'd been Pharaoh's slaves \& subject to his tyrannical\footnote{\textbf{tyrannical} [a] using power or authority over people in an unfair \& cruel way.} regulations\footnote{\textbf{regulation} [n] \textbf{1.} [countable, usually plural] an official rule made by a government or some other authority; \textbf{2.} [uncountable] the act of controlling something by means of rules; \textbf{3.} [uncountable] the act of controlling how a machine or system operates or how something behaves.} for 400 years, \& after that Moses subjected them to the harsh\footnote{\textbf{harsh} [a] \textbf{1.} very strict; \textbf{2.} (of weather or living conditions) very difficult \& unpleasant to live in.} desert\footnote{\textbf{desert} [n] [uncountable, countable] a large area of land that has very little water \& very few plants growing on it. Many desert areas are covered by sand; [v] \textbf{1.} [transitive, often passive] \textbf{desert somebody} to leave somebody without help or support, \textsc{synonym}: \textbf{abandon}; \textbf{2.} [transitive, often passive] \textbf{desert something} to go away from a place \& leave it empty, \textsc{synonym}: \textbf{abandon}; \textbf{3.} [intransitive, transitive] \textbf{desert (something)} to leave the armed forces without permission; \textbf{4.} [transitive] \textbf{desert (something) 9for something} to stop using, buying or supporting something.} wilderness\footnote{\textbf{wilderness} [n] [usually singular] a large area of land that has never been developed or used for growing crops because it is difficult to live there.} for another 40 years, to purify\footnote{\textbf{purify} [v] \textbf{1.} \textbf{purify something} to make something pure by removing anything that is bad, unpleasant or not wanted; \textbf{2.} [often passive] (\textit{specialist}) to separate a pure form of a substance from a mixture that contains it; to remove the impurities from a substance; \textbf{3.} \textbf{purify somebody\texttt{/}yourself} to make somebody\texttt{/}yourself pure by removing evil, especially in a ceremony.} them of their slavishness. Now, free at last, they are unbridled\footnote{\textbf{unbridled} [a] [usually before noun] (\textit{literary}) lacking control \& therefore extreme.}, \& have lost all control as they dance wildly around an idol, a golden calf\footnote{\textbf{calf} [n] \textbf{1.} [countable] the back part of the leg between the ankle \& the knee; \textbf{2.} [countable] a young cow; \textbf{3.} [countable] a young animal of some other type such as a young elephant or whale; \textbf{4.} [countable] (also \textbf{calfskin}) soft thin leather made from the skin of calves, used especially for making shoes \& clothing.}, displaying all manner of corporeal\footnote{\textbf{corporeal} [a] (\textit{formal}) \textbf{1.} that can be touched; physical rather than spiritual; \textbf{2.} of or for the body.} corruption\footnote{\textbf{corruption} [n] \textbf{1.} [uncountable] dishonest or illegal behavior, especially of people in authority; \textbf{2.} [uncountable] \textbf{corruption (of something)} the act or effect of making somebody change from moral to immoral standards of behavior; \textbf{3.} [countable, usually singular] \textbf{corruption of something} the form of a word or phrase that has become changed from its original form in some way; \textbf{4.} [uncountable] (\textit{computing}) the process by which mistakes are introduced into a computer file, etc. with the result that the data in it is no longer correct.}.

``I've got some good news $\ldots$ \& I've got some bad news,'' the lawgiver yells to them. ``Which do you want 1st?''

``The good news!'' the hedonists\footnote{\textbf{hedonist} [n] a person who believes that pleasure is the most important thing in life.} reply.

``I got Him from 15 commandments down to 10!''

``Hallelujah!'' cries the unruly\footnote{\textbf{unruly} [a] difficult to control or manage, \textsc{synonym}: \textbf{disorderly}.} crowd. ``\& the bad?''

``Adultery\footnote{\textbf{adultery} [n] [uncountable] sex between a married person \& somebody who is not their husband or wife.} is still in.''

So rules there will be -- but, please, not too many. We are ambivalent\footnote{\textbf{ambivalent} [a] having or showing both good \& bad feelings about somebody\texttt{/}something.} about rules, even when we know they are good for us. If we are spirited souls, if we have character, rules seem restrictive, an affront\footnote{\textbf{affront} [n] [usually singular] \textbf{affront (to somebody\texttt{/}something)} a remark or an action that offends somebody\texttt{/}something, \textsc{synonym}: \textbf{insult}; [v] [usually passive] (\textit{formal}) to say or do something that offends somebody, \textsc{synonym}: \textbf{insult}.} to our sense of agency\footnote{\textbf{agency} [n] \textbf{1.} [countable] a business or an organization that provides a particular service especially on behalf of other businesses or organizations; \textbf{2.} [countable] (\textit{especially North American English}) a government department that provides a particular service; \textbf{3.} [uncountable, countable] a person or thing that acts to produce a particular result; action that produces a particular result.} \& our pride in working out our own lives. \fbox{Why should we be judged according to another's rule?}

\& judged we are. After all, God didn't give Moses ``The Ten Suggestions,'' he gave Commandments; \& if I'm a free agent, my 1st reaction to a command might just be that nobody, not even God, tells me what to do, even if it's good for me. But the story of the golden calf also reminds us that \fbox{without rules we quickly becomes slaves to our passions} -- \& there's nothing freeing about that.

\& the story suggests something more: unchaperoned\footnote{\textbf{unchaperoned} [a] unaccompanied or unsupervised.}, \& left to our own untutored\footnote{\textbf{untutored} [a] (\textit{formal}) not having been formally taught about something.} judgment, we are quick to aim low \& worship qualities that are beneath\footnote{\textbf{beneath} [prep] \textbf{1.} in or to a lower position than somebody\texttt{/}something; under somebody\texttt{/}something; \textbf{2.} behind an appearance or feeling; \textbf{3.} not good enough for somebody; [adv] \textbf{1.} in or to a lower position; \textbf{2.} hidden behind an appearance or feeling.} us -- in this case, an artificial\footnote{\textbf{artificial} [a] \textbf{1.} made or produced by humans to copy something natural, rather than occurring naturally; \textbf{2.} created by people; not happening naturally.} animal that brings out our own animal instincts\footnote{\textbf{instinct} [n] [uncountable, countable] a natural tendency for people \& animals to behave in a particular way, using the knowledge \& abilities that they were born with rather than thought or training.} in a completely unregulated\footnote{\textbf{unregulated} [a] not controlled by laws or official rules.} way. The old Hebrew story makes it clear how the ancients felt about our prospects\footnote{\textbf{prospect} [n] \textbf{1.} [uncountable, singular] the possibility that something will happen; \textbf{2.} [singular] an idea of what might or will happen in the future; \textbf{3.} \textbf{(prospects)} [plural] the chances of being successful.} for civilized\footnote{\textbf{civilized} [a] \textbf{1.} well-organized socially with a very developed culture \& way of life; \textbf{2.} having laws \& customs that are fair \& morally acceptable.} behavior in the absence of rules that seek to elevate\footnote{\textbf{elevate} [v] \textbf{1.} \textbf{elevate something} (\textit{specialist}) to make the level of something increase; \textbf{2.} \textbf{elevate something} \textit{specialist} to lift something up or put something in a higher position; \textbf{3.} \textbf{elevate somebody\texttt{/}something (to\texttt{/}into something)} to give somebody\texttt{/}something a higher position or rank; \textbf{4.} \textbf{elevate something} to improve a person's mood, so  that they feel happy.} our gaze\footnote{\textbf{gaze} [n] [usually singular] a long steady look at somebody\texttt{/}something; [v] [intransitive] \textbf{$+$ adv.\texttt{/}prep.} to look steadily at somebody\texttt{/}something for a long time, either because you are very interested or surprised, or because you are thinking or something else.} \& raise our standards.

1 neat\footnote{\textbf{neat} [a] \textbf{1.} in good order; carefully done or arranged; \textbf{2.} simple but clever; \textbf{3.} containing or made out of just 1 substance; not mixed with anything else.} thing about the Bible story is that it doesn't simply list its rules, as lawyers or legislators\footnote{\textbf{legislator} [n] a member of a group of people that has the power or make laws.} or administrators\footnote{\textbf{administrator} [n] \textbf{1.} a person whose job is to organize the work of a business, school or other organization; \textbf{2.} (\textit{British English, law}) a person officially chosen to manage the financial affairs of a business that cannot pay its debts.} might; it embeds\footnote{\textbf{embed} [v] [usually passive] \textbf{1.} to make something a fixed \& important part of something else, that is difficult to change or remove; \textbf{2.} \textbf{embed something (in something)} to fix something firmly into a substance or solid object; \textbf{3.} \textbf{embed something (in something)} to make images, sound, software, etc. part of a computer program; \textbf{4.} \textbf{embed something} (\textit{linguistics}) to place a sentence inside another sentence.} them in a dramatic\footnote{\textbf{dramatic} [a] \textbf{1.} (of a change or an event) sudden, very great \& often surprising; \textbf{2.} exciting \& impressive; \textbf{3.} [usually before noun] connected with the theater or plays.} tale\footnote{\textbf{tale} [n] \textbf{1.} a story created using the imagination, especially one that is full of action \& adventure; \textbf{2.} an exciting spoken description of an event, which may not be completely true.} that illustrates why we need them, thereby making them easier to understand. Similarly, in this book Prof. Peterson doesn't just propose\footnote{\textbf{propose} [v] \textbf{1.} to suggest a plan or an idea for people to consider \& decide on; \textbf{2.} to suggest an explanation of something for people to consider.} his 12 rules, he tells stories, too, bringing to bear\footnote{\textbf{bear} [v] \textbf{1.} \textbf{bear something} to have something as a characteristic or feature; to be connected with something; \textbf{2.} \textbf{bear something} to have a particular mark, word or symbol that can be seen; \textbf{3.} \textbf{bear something} to have a particular name; \textbf{4.} \textbf{bear something} to take responsibility for something difficult; to be affected by or deal with something unpleasant. If somebody \textbf{cannot bear} something, they feel unable to deal with it or accept it: \textit{Her jealous husband could not bear the possibility of his wife talking to another man.} The short form `can't\texttt{/}couldn't bear' is not suitable in academic writing, unless you are quoting.; \textbf{5.} to have a feeling, especially a negative feeling; \textbf{6.} \textbf{bear (doing) something} to be suitable for something; to be worth doing. If something \textbf{does not bear close inspection}, it will be found to be unacceptable when carefully examined: \textit{This claim does not bear close inspection.} If something \textbf{does not bear comparison} with something else, it is not nearly as good: \textit{Her later work does not bear comparison with her earlier novels.}; \textbf{7.} \textbf{bear somebody\texttt{/}something} (\textit{formal}) to carry or hold somebody\texttt{/}something; \textbf{8.} (\textit{formal}) to give birth to a child; \textbf{9.} \textbf{bear something} (\textit{formal}) to produce flowers or fruit.} his knowledge of many fields as he illustrates \& explains why the best rules do not ultimately\footnote{\textbf{ultimately} [adv] \textbf{1.} in the end, finally; \textbf{2.} at the most basic \& important level, \textsc{synonym}: \textbf{basically, essentially}.} restrict us but instead facilitate\footnote{\textbf{facilitate} [v] \textbf{facilitate something} to make an action or a process possible or easier.} our goals \& make for fuller, freer lives.

The 1st time I [Norman Doidge] met Jordan Peterson was on Sep 12, 2004, at the home of 2 mutual friends, TV producer Wodek Szemberg \& medical internist\footnote{\textbf{internist} [n] (\textit{North American English}) a doctor who is a specialist in the treatment of diseases of the organs inside the body \& who does not usually do medical operations.} Estera Bekier. It was Wodek's birthday party. Wodek \& Estera are Polish \'emigr\'es who grew up within the Soviet empire\footnote{\textbf{empire} [n] \textbf{1.} a group of countries or states that are controlled by 1 ruler or government; \textbf{2.} a group of commercial organizations controlled by 1 person or company.}, where it was understood that many topics were off limits, \& that casually\footnote{\textbf{casual} [a] \textbf{1.} [usually before noun] without paying attention to detail; \textbf{2.} [usually before noun] not showing much care or thought; \textbf{3.} [usually before noun] (of a relationship) lasting only a short time \& without deep affection; \textbf{4.} [usually before noun] (\textit{British English}) (of work) not permanent; not regular; \textbf{5.} not formal; \textbf{6.} [only before noun] happening by chance; doing something by chance.} questioning certain social arrangements \& philosophical ideas (not to mention the regime\footnote{\textbf{regime} [n] \textbf{1.} a government, especially one that has not been elected in a fair way; \textbf{2.} a method or system of organizing or managing something; \textbf{3.} the conditions under which a natural, scientific or industrial process occurs; \textbf{4.} $=$ \textbf{regimen}.

\textbf{regimen} [n] (also \textbf{regime}) a course of medical treatment \& sometimes changes to diet \& behavior that somebody has to follow in order to recover from or control an illness.} itself) could mean big trouble.

But now, host\footnote{\textbf{host} [n] \textbf{1.} (\textit{biology}) an animal or a plant on which another animal or plant lives \& feeds; \textbf{2.} a country, a city or an organization that arranges \& holds a special event; \textbf{3.} a country that provides homes \& work for people who come from another country; \textbf{4.} a country where a company that is based in another country does business; \textbf{5.} \textbf{host of something} a large number of people or things; \textbf{6.} the main computer in a network that controls or supplies information to other computers that are connected to it; [v] \textbf{1.} \textbf{host something} to organize an event to which others are invited \& make all the arrangements for them; \textbf{2.} \textbf{host something} to store a website on a computer connected to the Internet, usually in return for payment.} \& hostess\footnote{\textbf{hostess} [n] \textbf{1.} a woman who invites guests to a meal, a party, etc.; a woman who has people staying at her home; \textbf{2.} a woman who is employed to welcome \& entertain people at a nightclub; \textbf{3.} a woman who introduces \& talks to guests on a television or radio show, \textsc{synonym}: \textbf{comp\`ere}; \textbf{4.} (\textit{North American English}) a woman who welcomes the customers in a restaurant.} luxuriated\footnote{\textbf{luxuriate in} [phrasal verb] \textbf{luxuriate in something} to relax while enjoying something very pleasant.} in easygoing\footnote{\textbf{easygoing} [a] relaxed \& happy to accept things without worrying or getting angry.}, honest\footnote{\textbf{honest} [a] \textbf{1.} always telling the truth, \& never stealing or deceiving people, \textsc{opposite}: \textbf{dishonest}; \textbf{2.} not hiding the truth about something.} talk, by having elegant\footnote{\textbf{elegant} [a] \textbf{1.} (of people or their behavior) attractive \& showing a good sense of style; \textbf{2.} (of clothes, places \& things) attractive \& designed well; \textbf{3.} (of a plan or an idea) clever but simple.} parties devoted to the pleasure\footnote{\textbf{pleasure} [n] \textbf{1.} [uncountable] a state of feeling or being happy or satisfied; the activity of enjoying yourself, \textsc{synonym}: \textbf{enjoyment}; \textbf{2.} [countable] a thing that makes you happy or satisfied.} of saying what you \textit{really} thought \& hearing others do the same, in an uninhibited\footnote{\textbf{uninhibited} [a] behaving or expressing yourself freely without worrying about what other people think, \textsc{synonym}: \textbf{unrestrained}, \textsc{opposite}: \textbf{inhibited}.} give-\&-take. Here, the rule was ``Speak your mind.'' If the conversation turned to politics\footnote{\textbf{politics} [n] \textbf{1.} [uncountable $+$ singular or plural verb] the activities involved in getting \& using power in public life, \& being able to influence decisions that effect a country or society; \textbf{2.} [uncountable $+$ singular or plural verb] the activities of governments concerning the political relations between states; \textbf{3.} [uncountable $+$ singular or plural verb] matters concerned with getting or using power within a particular group of organization; \textbf{4.} [plural] a person's political views or beliefs; \textbf{5.} [uncountable] $=$ \textbf{political science}; \textbf{6.} [singular] \textbf{politics (of something)} a system of political beliefs; a state of political affairs; \textbf{7.} [singular, uncountable $+$ singular or plural verb] \textbf{politics (of something)} the principles connected with a particular area of activity or interest, especially when concerned with power \& status.}, people of different political\footnote{\textbf{political} [a] \textbf{1.} connected with the state, government or public affairs; \textbf{2.} connected with the different groups working in politics, especially their policies \& the competition between them; \textbf{3.} (of people) interested in or active in politics; \textbf{4.} concerned with the competition for power within an organization, rather than with matters of principle.} persuasions\footnote{\textbf{persuasion} [n] \textbf{1.} [uncountable] the act of persuading somebody to do something or to believe something; \textbf{2.} [countable, uncountable] a particular set of beliefs, especially about religion or politics.} spoke to each other -- indeed, looked forward to it -- in a manner that is increasingly rare. Sometimes Wodek's own opinions, or truths, exploded out of him, as did his laugh. Then he'd hug whoever had made him laugh or provoked\footnote{\textbf{provoke} [v] \textbf{1.} \textbf{provoke something} to cause a particular reaction or have a particular effect; \textbf{2.} to say or do something in order to produce a strong reaction from somebody, usually anger.} him to speak his mind with greater intensity\footnote{\textbf{intensity} [n] \textbf{1.} [uncountable, singular] \textbf{intensity (of something)} the state or quality of being strong or intense; \textbf{2.} [uncountable, countable] the strength of something, e.g. light, that can be measured.} than even he might have intended. This was the best part of the parties, \& this frankness\footnote{\textbf{frank} [a] \textbf{1.} (\textbf{franker, frankest}) (\textbf{more frank} is also common) honest \& direct in what you say, sometimes in a way that other people might not like; \textbf{2.} (\textit{medical}) that cannot be confused with something else; obvious.}, \& his warm embraces\footnote{\textbf{embrace} [v] \textbf{1.} \textbf{embrace something} to accept an idea, a proposal, a set of beliefs, etc., especially when it is done with enthusiasm; \textbf{2.} \textbf{embrace something} to include something; \textbf{3.} \textbf{embrace somebody} to put your arms around somebody as a sign of love or friendship; [n] [countable, uncountable].}, made it worth provoking him. Meanwhile, Estera's voice lilted\footnote{\textbf{lilt} [n] [singular] \textbf{1.} the pleasant way in which a person's voice rises \& falls; \textbf{2.} a regular rising \& falling pattern in music, with a strong rhythm.} across the room on a very precise path towards its intended listener. \fbox{Truth explosions didn't make the atmosphere any less easygoing for the company} -- they made for more truth explosions! -- liberating\footnote{\textbf{liberate} [v] \textbf{1.} to free a country or a person from the control of somebody\texttt{/}something else; \textbf{2.} \textbf{liberate somebody\texttt{/}something (from something)} to free somebody\texttt{/}something from something that limits their ability to do things or enjoy life; \textbf{3.} (\textit{chemistry, physics}) to release gas, energy, etc. as a result of a chemical reaction or physical process.} us, \& more laughs, \& making the whole evening more pleasant, because with de-repressing\footnote{\textbf{repress} [v] \textbf{1.} \textbf{repress something} to try not to have or show an emotion, a thought, etc. In Freudian psychology, \textbf{repress} has a particular meaning, which is to stop yourself having particular thoughts or feelings so completely that they become or remain unconscious; \textbf{2.} [often passive] \textbf{repress somebody\texttt{/}something} to use political \&\texttt{/}or military force to control a group of people \& restrict their freedom, \textsc{synonym}: \textbf{put something down, suppress}; \textbf{3.} \textbf{repress something} (\textit{biology}) to prevent a gene from being expressed.} Eastern Europeans like the Szemberg-Bekiers, you always knew with what \& with whom you were dealing, \& that frankness was enlivening\footnote{\textbf{enliven} [v] (\textit{formal}) \textbf{enliven something} to make something more interesting or more fun.}. Honor\'e de Balzac, the novelist\footnote{\textbf{novelist} [n] a person who writes novels.}, once described the balls \& parties in his native France, observing that what appeared to be a single party was always really 2. In the 1st hours, the gathering was suffused\footnote{\textbf{suffuse} [v] [often passive] (\textit{literary}) \textbf{suffuse somebody\texttt{/}something (with something)} (especially of a color, light or feeling) to spread all over or through somebody\texttt{/}something.} with bored people posing\footnote{\textbf{pose} [v] \textbf{1.} [transitive] \textbf{pose something} to create a problem that has to be dealt with; \textbf{2.} [transitive] \textbf{pose something} to ask a question, especially one that needs serious thought, \textsc{synonym}: \textbf{raise}; \textbf{3.} [intransitive] \textbf{pose as somebody\texttt{/}something} to pretend to be somebody\texttt{/}something that you are not; \textbf{4.} [intransitive] \textbf{pose (for somebody\texttt{/}something)} to sit or stand in a particular position in order to be painted, drawn or photographed.} \& posturing\footnote{\textbf{posturing} [n] [uncountable, countable] (\textit{disapproving}) behavior that is not natural or sincere but is intended to attract attention or to have a particular effect.}, \& attendees who came to meet perhaps 1 special person who would confirm them in their beauty \& status. Then, only in the very late hours, after most of the guests had left, would the 2nd party, the real party, begin. Here the conversation was shared by each person present, \& open-hearted\footnote{\textbf{open-hearted} [a] kind \& friendly.} laughter replaced the starchy\footnote{\textbf{starchy} [a] \textbf{1.} (of food) containing a lot of starch; \textbf{2.} (\textit{informal, disapproving}) (of a person or their behavior) very formal; not friendly or relaxed.} airs. At Estera \& Wodek's parties, this kind of wee-hours-of-the-morning disclosure\footnote{\textbf{disclosure} [n] \textbf{1.} [uncountable] \textbf{disclosure (of something) (to somebody)} the act of making something known or public that was previously secret or private, \textsc{synonym}: \textbf{revelation}; \textbf{2.} [countable] \textbf{disclosure} (about somebody\texttt{/}something) information or a fact that is made known or public that was previously secret or private, \textsc{synonym}: \textbf{revelation}.} \& intimacy\footnote{\textbf{intimate} [a] \textbf{1.} (of a link between things) very close; \textbf{2.} (of people) having a close \& friendly relationship; \textbf{3.} sexual; \textbf{4.} private \& personal, often in a sexual way; \textbf{5.} (of a place or situation) encouraging close, friendly relationships; \textbf{6.} (of knowledge) very detailed \& thorough.

\textbf{intimacy} [n] [uncountable, countable, usually plural].} often began as soon as we entered the room.

Wodek is a silver-haired, lion-maned hunter, always on the lookout for potential public intellectuals, who knows how to spot people who can \textit{really} talk in front of a TV  camera \& who look authentic\footnote{\textbf{authentic} [a] \textbf{1.} known to be real \& genuine \& not a copy, \textsc{synonym}: \textbf{genuine}; \textbf{2.} true \& accurate; based on fact; \textbf{3.} made to be exactly like the original.} because they are (the camera picks up on that). He often invites such people to these salons\footnote{\textbf{salon} [n] \textbf{1.} a shop that gives customers hair or beauty treatment or that sells expensive clothes; \textbf{2.} (\textit{old-fashioned}) a room in a large house used for entertaining guests; \textbf{3.} (in the past) a regular meeting of writers, artists \& other guests at the house of a famous or important person.}. That day Wodek brought a psychology professor, from my own University of Toronto, who fit the bill: intellect \& emotion in tandem\footnote{\textbf{tandem} [n] \textbf{in tandem (with somebody\texttt{/}something)} [idiom] a thing that works or happens in tandem with something else works together with it or happens at the same time as it.}. Wodek was the 1st to put Jordan Peterson in front of a camera, \& thought of him as a teacher in search of students -- because he was always ready to explain. \& it helped that he liked the camera \& that camera liked him back.

That afternoon there was a large table set outside in the Szemberg-Beliers' garden; around it was gathered the usual \fbox{collection of lips \& ears}, \& loquacious\footnote{\textbf{loquacious} [a] (\textit{formal}) talking a lot, \textsc{synonym}: \textbf{talkative}.} virtuosos\footnote{\textbf{virtuoso} [n] (plural \textbf{virtuosos, virtuosi}) a person who shows very great skill at doing something, especially playing a musical instrument; [a] [only before noun] showing extremely great skill.}. We seemed, however, to be plagued\footnote{\textbf{plague} [v] \textbf{1.} \textbf{plague somebody\texttt{/}something (with something)} to cause pain or trouble to somebody\texttt{/}something over a period of time, \textsc{synonym}: \textbf{trouble}; \textbf{2.} \textbf{plague somebody (with something)} to annoy somebody or create problems, especially by asking for something, demanding attention, etc., \textsc{synonym}: \textbf{hound}; [n] \textbf{1.} (also \textbf{the plague}) (also \textbf{bubonic plague}) [uncountable] a disease spread by rats that causes a high temperature, swellings ($=$ areas that are larger \& rounder than usual) on the body \& usually death; \textbf{2.} [countable] any disease that spreads quickly \& kills a lot of people, \textsc{synonym}: \textbf{epidemic}; \textbf{3.} [countable] \textbf{plague of something} large numbers of an animal or insect that come into an area \& cause great damage.} by a buzzing\footnote{\textbf{buzz} [v] \textbf{1.} [intransitive] (of a bee) to make a continuous low sound; \textbf{2.} [intransitive] to make a sound like a bee buzzing; \textbf{3.} [intransitive] to be full of excitement, activity, etc.; \textbf{4.} [intransitive, transitive] \textbf{buzz (something) (for somebody\texttt{/}something)} to call somebody to come by pressing a buzzer; \textbf{5.} [transitive] \textbf{buzz somebody\texttt{/}something} (\textit{informal}) to fly very close to somebody\texttt{/}something, especially as a warning or threat; [n] \textbf{1.} [countable, usually singular] (also \textbf{buzzing} [uncountable, singular]) a continuous sound like the one that a bee, a buzzer or other electronic device makes; \textbf{2.} [singular] the sound of people talking, especially in an excited way; \textbf{3.} [singular, uncountable] (\textit{informal}) a strong feeling of pleasure, excitement or achievement; \textbf{4.} \textbf{the buzz} [singular] (\textit{informal}) news that people tell each other that may or may not be true, \textsc{synonym}: \textbf{rumor}.} paparazzi\footnote{\textbf{paparazzo} [n] (also \textbf{pap}) (plural \textbf{paparazzi}) [usually plural] a photographer who follows famous people around in order to get interesting photographs of them to sell to a newspaper.} of bees, \& here was this new fellow\footnote{\textbf{fellow} [n] \textbf{1.} [usually plural] a person that you work with or that is like you; a thing that is similar to the one mentioned; \textbf{2.} (\textit{British English}) a senior member of some colleges or universities; \textbf{3.} a member of an academic or professional organization; \textbf{4.} (\textit{especially North American English}) a graduate student who holds a fellowship; [a] [only before noun] used  to describe somebody who is the same as you in some way, or in the same situation.} at the table, with an Albertan\footnote{\textbf{Alberta} [n] a province in western Canada, east of British Columbia \& west of Saskatchewan. The capital is Edmonton.} accent\footnote{\textbf{accent} [n] \textbf{1.} a way of pronouncing the words of a language that shows which country, area or social class a person comes from; \textbf{2.} the emphasis that you should give to part of a word when saying it, \textsc{synonym}: \textbf{stress}; \textbf{3.} a mark on a letter to show that it should be pronounced in a particular way; \textbf{4.} [singular] \textbf{accent (on something)} a special importance that is given to something, \textsc{synonym}: \textbf{emphasis}.}, in cowboy boots, who was ignoring them, \& kept on talking. He kept talking while the rest of us were playing musical chairs to keep away from the pests\footnote{\textbf{pest} [n] an insect or animal that destroys plants, food, etc.}, yet also trying to remain at the table because this new addition to our gatherings was so interesting.

He had this odd habit of speaking about the deepest questions to whoever was at this table -- most of them new acquaintances\footnote{\textbf{acquaintance} [n] \textbf{1.} [countable] a person that you know but who is not a close friend; \textbf{2.} [uncountable, countable] \textbf{acquaintance (with somebody)} (\textit{formal}) slight friendship; \textbf{3.} [uncountable, countable] \textbf{acquaintance with something} (\textit{formal}) knowledge of something.} -- as though he were just making small talk. Or, if he did do small talk, the interval between ``How do you know Wodek \& Estera?'' or ``I was a beekeeper once, so I'm used to them'' \& more serious topics would be nanoseconds\footnote{\textbf{nanosecond} [n] (abbr. \textbf{ns}) $10^{-3}$ second.}.

One might hear such questions discussed at parties where professors \& professionals\footnote{\textbf{professional} [n] a person who does a job that needs special training \& a high level of education.} gather, but usually the conversation would remain between 2 specialists\footnote{\textbf{specialist} [n] \textbf{1.} a doctor who has specialized in a particular area of medicine; \textbf{2.} \textbf{specialist (in something)} a person who is an expert in a particular area of work or study; [a] [only before noun] \textbf{1.} connected with a doctor who has specialized in a particular area of medicine; \textbf{2.} having or involving detailed knowledge of a particular topic or area of study.} in the topic, off in a corner, or if shared with the whole group it was often not without someone preening\footnote{\textbf{preen} [v] \textbf{1.} [transitive, intransitive] \textbf{preen (yourself)} (\textit{usually disapproving}) to spend a lot of time making yourself look attractive \& then admiring your appearance; \textbf{2.} [transitive] \textbf{preen yourself (on something)} (\textit{usually disapproving}) to feel very pleased with yourself about something \& show other people how pleased you are; \textbf{3.} [intransitive, transitive] \textbf{preen (itself)} (of a bird) to clean itself or make its feathers smooth with its beak.}. But this Peterson, though erudite\footnote{\textbf{erudite} [a] (\textit{formal, approving}) having or showing great knowledge that is gained from academic study, \textsc{synonym}: \textbf{learned}.}, didn't come across as a pedant\footnote{\textbf{pedant} [n] (\textit{disapproving}) a person who is too concerned with small details or rules especially when learning or teaching.}. He had the enthusiasm of a kid who had just learned something new \& had to share it. He seemed to be assuming, as a child would -- before learning how dulled\footnote{\textbf{dull} [v] \textsf{pain} \textbf{1.} [transitive, intransitive] \textbf{dull (something)} to make a pain or an emotion weaker or less severe; to become weaker or less severe; \textsf{person} \textbf{2.} [transitive] \textbf{dull somebody} to make a person slower or less lively; \textsf{colors\texttt{/}sounds} \textbf{3.} [intransitive, transitive] to become less bright, clean or sharp; to make something less bright, clean or sharp; [a] \textsf{boring} \textbf{1.} not interesting or exciting, \textsc{synonym}: \textbf{dreary}; \textsf{light\texttt{/}colors} \textbf{2.} not bright or shiny; \textsf{weather} \textbf{3.} not bright, with a lot of clouds, \textsc{synonym}: \textbf{overcast}; \textsf{sounds} \textbf{4.} not clear or cloud; \textsf{pain} \textbf{5.} not very severe, but continuous; \textsf{person} \textbf{6.} slow in understanding, \textsc{synonym}: \textbf{stupid}; \textsf{trade} \textbf{7.} (\textit{especially North American English}) not busy; slow.} adults can become -- that if he thought something was interesting, then so might others. There was something boyish\footnote{\textbf{boyish} [a] (\textit{approving}) looking or behaving like a boy, in a way that is attractive.} in the cowboy, in his broaching\footnote{\textbf{broach} [v] \textbf{broach something (to\texttt{/}with somebody)} to begin talking about a subject that is difficult to discuss, especially because it is embarrassing or because people disagree about it.} of subjects as though we had all grown up together in the same small town, or family, \& had all been thinking about the very same problems of human existence\footnote{\textbf{existence} [n] \textbf{1.} [uncountable, countable, usually singular] the state or fact of happening or being found in a particular place, time or situation; the state of being alive; \textbf{2.} [uncountable] \textbf{existence (of something)} the fact of being real; \textbf{3.} [countable, usually singular] a way of living, especially when this is difficult.} all along.

Peterson wasn't really an ``eccentric''\footnote{\textbf{eccentric} [a] considered by other people to be strange or unusual; [n] a person who is considered by other people to be strange or unusual.}; he had sufficient conventional\footnote{\textbf{conventional} [a] \textbf{1.} [usually before noun] based on what is generally believed; following the way something is usually done; \textbf{2.} (\textit{often disapproving}) tending to follow what is done or considered acceptable by society in general; normal \& ordinary, \& perhaps not very interesting, \textsc{opposite}: \textbf{unconventional}; \textbf{3.} [usually before noun] (especially of weapons) not nuclear; \textbf{4.} (of literature, art or the theater) using a traditional style or method.} chops\footnote{\textbf{chop} [v] \textbf{1.} to cut something into pieces with a sharp tool such as a knife; \textbf{2.} [usually passive] (\textit{informal}) to suddenly stop providing or allowing something; to suddenly reduce something by a large amount, \textsc{synonym}: \textbf{cut}; \textbf{3.} \textbf{chop somebody\texttt{/}something} to hit somebody\texttt{/}something downwards with a quick, short movement; [n] \textbf{1.} [countable] a thick slide of meat with a bone attached to it, especially from a pig or sheep; \textbf{2.} [countable, usually singular] an act of cutting something in a quick movement downwards using an axe or a knife; \textbf{3.} [countable] an act of hitting somebody\texttt{/}something with the side of your hand in a quick movement downwards; \textbf{4.} \textbf{chops} [plural] (\textit{informal}) the part of a person's or an animal's face around the mouth; \textbf{5.} \textbf{chops} [plural] the technical skill of an actor or a jazz or rock musician.}, had been a Harvard professor, was a gentleman\footnote{\textbf{gentleman} [n] (plural \textbf{gentlemen}) \textbf{1.} (\textit{formal}) a polite or formal way of referring to a man; \textbf{2.} (in the past) a man from a high social class, especially one who did not need to work.} (as cowboys can be) though he did say \textit{damn} \& \textit{bloody} a lot, in a rural\footnote{\textbf{rural} [a] [usually before noun] connected with or like the countryside.} 1950s sort of way. But everyone listened, with fascination\footnote{\textbf{fascination} [n] \textbf{1.} [countable, usually singular] a very strong attraction, that makes something very interesting; \textbf{2.} [uncountable, singular] the state of being very attracted to \& interested in somebody\texttt{/}something.} on their faces, because he was in fact addressing questions of concern to everyone at the table.

There was something freeing about being with a person so learned\footnote{\textbf{learned} [a] [usually before noun] \textbf{1.} developed by training or experience; not existing at birth; \textbf{2.} having a lot of knowledge because you have studied \& read a lot; \textbf{3.} connected with or for leraned people; showing deep knowledge; \textsc{synonym}: \textbf{scholarly}.} yet speaking in such an unedited way. His thinking was motoric; it seemed he needed to think \textit{aloud}, to use his motor cortex to think, but that motor also had to run fast to work properly. To get to liftoff. \texttt{pause at p. 8/403 due to concern of dictionaries: a lot of words are unavailable in Oxford dictionaries, both learner \& academic versions.}

'' -- \cite[Foreword]{Peterson2018}

%------------------------------------------------------------------------------%

\chapter{Miscellaneous}

\section{Young, Dumb, \& Broke}
Watch \& listen \href{https://www.youtube.com/watch?v=IPfJnp1guPc}{Youtube\texttt{/}Khalid\texttt{/}Young Dumb \& Broke}.

\section{Existential Crisis}

\section{Meaning of Life?}

\section{Art of Balancing in Life?}

%------------------------------------------------------------------------------%

\selectlanguage{vietnamese}

\begin{thebibliography}{99}
	\bibitem[NQBH\texttt{/}psychology]{NQBH/psychology} \selectlanguage{vietnamese} Nguyễn Quản Bá Hồng. \href{https://github.com/NQBH/hobby/blob/master/psychology/NQBH_a_personal_journey_to_psychology.pdf}{\textit{A Personal Journey to Psychology: The Way I Perceive}}. March 2022--now.
\end{thebibliography}

\selectlanguage{english}

\printbibliography[heading=bibintoc]
	
\end{document}