\documentclass{article}
\usepackage[backend=biber,natbib=true,style=authoryear]{biblatex}
\addbibresource{/home/nqbh/reference/bib.bib}
\usepackage{tocloft}
\renewcommand{\cftsecleader}{\cftdotfill{\cftdotsep}}
\usepackage[colorlinks=true,linkcolor=blue,urlcolor=red,citecolor=magenta]{hyperref}
\usepackage{algorithm,algpseudocode,amsmath,amssymb,amsthm,float,graphicx,mathtools}
\allowdisplaybreaks
\numberwithin{equation}{section}
\newtheorem{assumption}{Assumption}[section]
\newtheorem{conjecture}{Conjecture}[section]
\newtheorem{corollary}{Corollary}[section]
\newtheorem{definition}{Definition}[section]
\newtheorem{example}{Example}[section]
\newtheorem{lemma}{Lemma}[section]
\newtheorem{notation}{Notation}[section]
\newtheorem{principle}{Principle}[section]
\newtheorem{problem}{Problem}[section]
\newtheorem{proposition}{Proposition}[section]
\newtheorem{question}{Question}[section]
\newtheorem{remark}{Remark}[section]
\newtheorem{theorem}{Theorem}[section]
\usepackage[left=0.5in,right=0.5in,top=1.5cm,bottom=1.5cm]{geometry}
\usepackage{fancyhdr}
\pagestyle{fancy}
\fancyhf{}
\lhead{\small Sect.~\thesection}
\rhead{\small\nouppercase{\leftmark}}
\renewcommand{\sectionmark}[1]{\markboth{#1}{}}
\cfoot{\thepage}
\def\labelitemii{$\circ$}

\title{12 Rules for Life: An Antidote to Chaos}
\author{Jordan B. Peterson}
\date{\today}

\begin{document}
\maketitle
\tableofcontents
\vspace{5mm}
\begin{quotation}
	\textit{``The most influential public intellectual in the Western world right now.''} -- New York Times
\end{quotation}

%------------------------------------------------------------------------------%

\section{Foreword by Noman Doidge}
``Rules? More rules? Really? Isn't life complicated enough, restricting enough, without abstract rules that don't take our unique, individual situations into account? \& given that our brains are plastic, \& all develop differently based on our life experiences, why even expect that a few rules might be helpful to us all?

People don't clamor for rules, even in the Bible $\ldots$ as when Moses comes down the mountain, after a long absence, bearing the tablets inscribed with 10 commandments, \& finds the Children of Israel in revelry. They'd been Pharaoh's slaves \& subject to his tyrannical regulations for 400 years, \& after that Moses subjected them to the harsh desert wilderness for another 40 years, to purify them of their slavishness. Now, free at last, they are unbridled, \& have lost all control as they dance wildly around an idol, a golden calf, displaying all manner of corporeal corruption.

``I've got some good news $\ldots$ \& I've got some bad news,'' the lawgiver yells to them. ``Which do you want 1st?''

``The good news!'' the hedonists reply.

``I got Him from 15 commandments down to 10!''

``Hallelujah!'' cries the unruly crowd. ``\& the bad?''

``Adultery is still in.''

So rules there will be -- but, please, not too many. We are ambivalent about rules, even when we know they are good for us. If we are spirited souls, if we have character, rules seem restrictive, an affront to our sense of agency \& our pride in working out our own lives. Why should we be judged according to another's rule?

\& judged we are. After all, God didn't give Moses ``The 10 Suggestions,'' he gave Commandments; \& if I'm a free agent, my 1st reaction to a command might just be that nobody, not even God, tells me what to do, even if it's good for me. But the story of the golden calf also reminds us that without rules we quickly become slaves to our passions -- \& there's nothing freeing about that.

\& the story suggests something more: unchaperoned, \& left to our own untutored judgment, we are quick to aim low \& worship qualities that are beneath us -- in this case, an artificial animal that brings out our own animal instincts in a completely unregulated way. The old Hebrew story makes it clear how the ancients felt about our prospects for civilized behavior in the absence of rules that seek to elevate our gaze \& raise our standards.

1 neat thing about the Bible story is that it doesn't simply list its rules, as lawyers or legislators or administrators might; it embeds them in a dramatic tale that illustrates why we need them, thereby making them easier to understand. Similarly, in this book Prof. Peterson doesn't just propose his 12 rules, he tells stories, too, bringing to bear his knowledge of many fields as he illustrates \& explains why the best rules do not ultimately restrict us but instead facilitate our goals \& make for fuller, freer lives. p. 6

'' -- \cite[pp. 5--]{Peterson2018}



%------------------------------------------------------------------------------%

\section{Overture}

%------------------------------------------------------------------------------%

\section{Rule 1: Stand Up Straight with Your Shoulders Back}

%------------------------------------------------------------------------------%

\section{Rule 2: Treat Yourself Like Someone You Are Responsible for Helping}

%------------------------------------------------------------------------------%

\section{Rule 3: Make Friends with People Who Want The Best for You}

%------------------------------------------------------------------------------%

\section{Rule 4: Compare Yourself to Who You Were Yesterday, Not to Who Someone Else Is Today}

%------------------------------------------------------------------------------%

\section{Rule 5: Do Not Let Your Children Do Anything That Makes You Dislike Them}

%------------------------------------------------------------------------------%

\section{Rule 6: Set Your House In Perfect Order Before You Criticize The World}

%------------------------------------------------------------------------------%

\section{Rule 7: Pursue What Is Meaningful (Not What Is Expedient)}

\subsection{Get While The Getting's Good}
``Life is suffering. That's clear. There is no more basic, irrefutable truth. It's basically what God tells Adam \& Eve, immediately before he kicks them out of Paradise.
\begin{quotation}
	Unto the woman he said, I will greatly multiply thy sorrow \& thy conception; in sorrow thou shalt bring 4th children; \& thy desire shall be to thy husband, \& he shall rule over thee.
	
	\& unto Adam he said, Because thou hast hearkened unto the voice of thy wife, \& hast eaten of the tree, of which I commanded thee, saying, Thou shalt not eat of it: cursed is the ground for thy sake; in sorrow shalt thou eat of it all the days of thy life;
	
	Thorns also \& thistles shall it bring 4th to thee; \& thou shalt eat the herb of the field;
	
	By the sweat of your brow you will eat your food until you return to the ground, since from it you were taken; for dust you are \& to dust you will return.'' (Genesis 3:16-19. KJV)
\end{quotation}
What in the world should be done about that?

The simplest, most obvious, \& most direct answer? Pursue pleasure. Follow your impulses. Live for the moment. Do what's expedient. Lie, cheat, steal, deceive, manipulate -- but don't get caught. In an ultimately meaningless universe, what possible difference could it make? \& this is by no means a new idea. The fact of life's tragedy \& the suffering that is part of it has been used to justify the pursuit of immediate selfish gratification for a very long time.

'' -- \cite[pp. 183--]{Peterson2018}



%------------------------------------------------------------------------------%

\section{Rule 8: Tell The Truth -- Or, At Least, Don't Lie}

%------------------------------------------------------------------------------%

\section{Rule 9: Assume That The Person You Are Listening to Might Know Something You Don't}

%------------------------------------------------------------------------------%

\section{Rule 10: Be Precise In Your Speech}

%------------------------------------------------------------------------------%

\section{Rule 11: Do Not Bother Children When They Are Skateboarding}

%------------------------------------------------------------------------------%

\section{Rule 12: Pet A Cat When You Encounter One On The Street}

%------------------------------------------------------------------------------%

\section{Coda}

%------------------------------------------------------------------------------%

\section{Endnotes}

%------------------------------------------------------------------------------%

\printbibliography[heading=bibintoc]
	
\end{document}