\documentclass{article}
\usepackage[backend=biber,natbib=true,style=authoryear]{biblatex}
\addbibresource{/home/nqbh/reference/bib.bib}
\usepackage{tocloft}
\renewcommand{\cftsecleader}{\cftdotfill{\cftdotsep}}
\usepackage[colorlinks=true,linkcolor=blue,urlcolor=red,citecolor=magenta]{hyperref}
\usepackage{algorithm,algpseudocode,amsmath,amssymb,amsthm,float,graphicx,mathtools}
\allowdisplaybreaks
\numberwithin{equation}{section}
\newtheorem{assumption}{Assumption}[section]
\newtheorem{conjecture}{Conjecture}[section]
\newtheorem{corollary}{Corollary}[section]
\newtheorem{definition}{Definition}[section]
\newtheorem{example}{Example}[section]
\newtheorem{lemma}{Lemma}[section]
\newtheorem{notation}{Notation}[section]
\newtheorem{principle}{Principle}[section]
\newtheorem{problem}{Problem}[section]
\newtheorem{proposition}{Proposition}[section]
\newtheorem{question}{Question}[section]
\newtheorem{remark}{Remark}[section]
\newtheorem{theorem}{Theorem}[section]
\usepackage[left=0.5in,right=0.5in,top=1.5cm,bottom=1.5cm]{geometry}
\usepackage{fancyhdr}
\pagestyle{fancy}
\fancyhf{}
\lhead{\small Sect.~\thesection}
\rhead{\small\nouppercase{\leftmark}}
\renewcommand{\sectionmark}[1]{\markboth{#1}{}}
\cfoot{\thepage}
\def\labelitemii{$\circ$}

\title{12 Rules for Life: An Antidote to Chaos}
\author{Jordan B. Peterson}
\date{\today}

\begin{document}
\maketitle
\tableofcontents
\vspace{5mm}
\begin{quotation}
	\textit{``The most influential public intellectual in the Western world right now.''} -- New York Times
\end{quotation}

%------------------------------------------------------------------------------%

\section{Foreword by Noman Doidge}
``Rules? More rules? Really? Isn't life complicated enough, restricting enough, without abstract rules that don't take our unique, individual situations into account? \& given that our brains are plastic, \& all develop differently based on our life experiences, why even expect that a few rules might be helpful to us all?

People don't clamor for rules, even in the Bible $\ldots$ as when Moses comes down the mountain, after a long absence, bearing the tablets inscribed with 10 commandments, \& finds the Children of Israel in revelry. They'd been Pharaoh's slaves \& subject to his tyrannical regulations for 400 years, \& after that Moses subjected them to the harsh desert wilderness for another 40 years, to purify them of their slavishness. Now, free at last, they are unbridled, \& have lost all control as they dance wildly around an idol, a golden calf, displaying all manner of corporeal corruption.

``I've got some good news $\ldots$ \& I've got some bad news,'' the lawgiver yells to them. ``Which do you want 1st?''

``The good news!'' the hedonists reply.

``I got Him from 15 commandments down to 10!''

``Hallelujah!'' cries the unruly crowd. ``\& the bad?''

``Adultery is still in.''

So rules there will be -- but, please, not too many. We are ambivalent about rules, even when we know they are good for us. If we are spirited souls, if we have character, rules seem restrictive, an affront to our sense of agency \& our pride in working out our own lives. Why should we be judged according to another's rule?

\& judged we are. After all, God didn't give Moses ``The 10 Suggestions,'' he gave Commandments; \& if I'm a free agent, my 1st reaction to a command might just be that nobody, not even God, tells me what to do, even if it's good for me. But the story of the golden calf also reminds us that without rules we quickly become slaves to our passions -- \& there's nothing freeing about that.

\& the story suggests something more: unchaperoned, \& left to our own untutored judgment, we are quick to aim low \& worship qualities that are beneath us -- in this case, an artificial animal that brings out our own animal instincts in a completely unregulated way. The old Hebrew story makes it clear how the ancients felt about our prospects for civilized behavior in the absence of rules that seek to elevate our gaze \& raise our standards.

1 neat thing about the Bible story is that it doesn't simply list its rules, as lawyers or legislators or administrators might; it embeds them in a dramatic tale that illustrates why we need them, thereby making them easier to understand. Similarly, in this book Prof. Peterson doesn't just propose his 12 rules, he tells stories, too, bringing to bear his knowledge of many fields as he illustrates \& explains why the best rules do not ultimately restrict us but instead facilitate our goals \& make for fuller, freer lives. p. 6

'' -- \cite[pp. 5--]{Peterson2018}



%------------------------------------------------------------------------------%

\section{Overture}

%------------------------------------------------------------------------------%

\section{Rule 1: Stand Up Straight with Your Shoulders Back}

%------------------------------------------------------------------------------%

\section{Rule 2: Treat Yourself Like Someone You Are Responsible for Helping}

%------------------------------------------------------------------------------%

\section{Rule 3: Make Friends with People Who Want The Best for You}

%------------------------------------------------------------------------------%

\section{Rule 4: Compare Yourself to Who You Were Yesterday, Not to Who Someone Else Is Today}

%------------------------------------------------------------------------------%

\section{Rule 5: Do Not Let Your Children Do Anything That Makes You Dislike Them}

%------------------------------------------------------------------------------%

\section{Rule 6: Set Your House In Perfect Order Before You Criticize The World}

%------------------------------------------------------------------------------%

\section{Rule 7: Pursue What Is Meaningful (Not What Is Expedient)}

\subsection{Get While The Getting's Good}
``Life is suffering. That's clear. There is no more basic, irrefutable truth. It's basically what God tells Adam \& Eve, immediately before he kicks them out of Paradise.
\begin{quotation}
	Unto the woman he said, I will greatly multiply thy sorrow \& thy conception; in sorrow thou shalt bring 4th children; \& thy desire shall be to thy husband, \& he shall rule over thee.
	
	\& unto Adam he said, Because thou hast hearkened unto the voice of thy wife, \& hast eaten of the tree, of which I commanded thee, saying, Thou shalt not eat of it: cursed is the ground for thy sake; in sorrow shalt thou eat of it all the days of thy life;
	
	Thorns also \& thistles shall it bring 4th to thee; \& thou shalt eat the herb of the field;
	
	By the sweat of your brow you will eat your food until you return to the ground, since from it you were taken; for dust you are \& to dust you will return.'' (Genesis 3:16-19. KJV)
\end{quotation}
What in the world should be done about that?

The simplest, most obvious, \& most direct answer? Pursue pleasure. Follow your impulses. Live for the moment. Do what's expedient. Lie, cheat, steal, deceive, manipulate -- but don't get caught. In an ultimately meaningless universe, what possible difference could it make? \& this is by no means a new idea. The fact of life's tragedy \& the suffering that is part of it has been used to justify the pursuit of immediate selfish gratification for a very long time.
\begin{quotation}
	Short \& sorrowful is our life, \& there is no remedy when a man comes to his end, \& no one has been known to return from Hades.
	
	Because we were born by mere chance, \& hereafter we shall be as though we had never been; because the breath in our nostrils is smoke, \& reason is a spark kindled by the beating of our hearts.
	
	When it is extinguished, the body will turn to ashes, \& the spirit will dissolve like empty air. Our name will be forgotten in time \& no one will remember our works; our life will pass away like the traces of a cloud, \& be scattered like mist that is chased by the rays of the sun \& overcome by its heat.
	
	For our allotted time is the passing of a shadow, \& there is no return from our death, because it is sealed up \& no one turns back.
	
	Come, therefore, let us enjoy the good things that exist, \& make use of the creation to the full as in youth.
	
	Let us take our fill of costly wine \& perfumes, \& let no flower of spring pass by us.
	
	Let us crown ourselves with rosebuds before they wither.
	
	Let none of us fail to share in our revelry, everywhere let us leave signs of enjoyment, because this is our portion, \& this our lot.
	
	Let us oppress the righteous poor man; let us not spare the widow nor regard the gray hairs of the aged.
	
	But let our might be our law of right, for what is weak proves itself to be useless.
	
	(Wisdom 2:1-11, RSV).
\end{quotation}
The pleasure of expediency may be fleeing, but it's pleasure, nonetheless, \& that's something to stack up against the terror \& pain of existence. Every man for himself, \& the devil take the hindmost, as the old proverb has it. Why not simply take everything you can get, whenever the opportunity arises? Why not determine to live in that manner?

Or is there an alternative, more powerful \& more compelling?

Our ancestors worked out very sophisticated answers to such questions, but we still don't understand them very well. This is because they are in large part still implicit -- manifest primarily in ritual \& myth \&, as of yet, incompletely articulated. We act them out \& represent them in stories, but we're not yet wise enough to formulate them explicitly. We're still chimps in a troupe, or wolves in a pack. we know how to behave. We know who's who, \& why. We've learned that through experience. Our knowledge has been shaped by our interaction with others. We've established predictable routines \& patterns of behavior -- but we don't really understand them, or know where they originated. They've evolved over great expanses of time. No one was formulating them explicitly (at least not in the dimmest reaches of the past), even though we've been telling each other how to act forever. 1 day, however, not so long ago, we woke up. We were already doing, but we started \textit{noticing} what we were doing. We started using our bodies as devices to represent their own actions. We started imitating \& dramatizing. We invented ritual. We started acting out our own experiences. Then we started to tell stories. We coded our observations of our own drama in these stories. In this manner, the information that was 1st only embedded in our behavior became represented in our stories. But we didn't \& still don't understand what it all means.

The Biblical narrative of Paradise \& the Fall is 1 such story, fabricated by our collective imagination, working over the centuries. It provides a profound account of the nature of Being, \& points the way to a mode of conceptualization \& action well-matched to that nature. In the Garden of Eden, prior to the dawn of self-consciousness -- so goes the story -- human beings were sinless. Our primordial parents, Adam \& Eve, walked with God. Then, tempted by the snake, the 1st couple ate from the tree of the knowledge of good \& evil, discovered Death \& vulnerability, \& turned away from God. Mankind was exiled from Paradise, \& began its effortful mortal existence. The idea of sacrifice enters soon afterward, beginning with the account of Cain \& Abel, \& developing through the Abrahamic adventures \& the Exodus: After much contemplation, struggling humanity learns that God's favor could be gained, \& his wrath averted, through proper sacrifice -- \&, also, that bloody murder might be motivated among those unwilling or unable to succeed in this manner.'' -- \cite[pp. 183--185]{Peterson2018}

\subsection{The Delay of Gratification}

%------------------------------------------------------------------------------%

\section{Rule 8: Tell The Truth -- Or, At Least, Don't Lie}

%------------------------------------------------------------------------------%

\section{Rule 9: Assume That The Person You Are Listening to Might Know Something You Don't}

%------------------------------------------------------------------------------%

\section{Rule 10: Be Precise In Your Speech}

%------------------------------------------------------------------------------%

\section{Rule 11: Do Not Bother Children When They Are Skateboarding}

%------------------------------------------------------------------------------%

\section{Rule 12: Pet A Cat When You Encounter One On The Street}

%------------------------------------------------------------------------------%

\section{Coda}

%------------------------------------------------------------------------------%

\section{Endnotes}

%------------------------------------------------------------------------------%

\printbibliography[heading=bibintoc]
	
\end{document}