\documentclass{article}
\usepackage[backend=biber,natbib=true,style=authoryear]{biblatex}
\addbibresource{/home/hong/1_NQBH/reference/bib.bib}
\usepackage[utf8]{vietnam}
\usepackage{tocloft}
\renewcommand{\cftsecleader}{\cftdotfill{\cftdotsep}}
\usepackage[colorlinks=true,linkcolor=blue,urlcolor=red,citecolor=magenta]{hyperref}
\usepackage{amsmath,amssymb,amsthm,mathtools,float,graphicx,algpseudocode,algorithm,tcolorbox}
\usepackage[inline]{enumitem}
\allowdisplaybreaks
\numberwithin{equation}{section}
\newtheorem{assumption}{Assumption}[section]
\newtheorem{conjecture}{Conjecture}[section]
\newtheorem{corollary}{Corollary}[section]
\newtheorem{hequa}{Hệ quả}[section]
\newtheorem{definition}{Definition}[section]
\newtheorem{dinhnghia}{Định nghĩa}[section]
\newtheorem{example}{Example}[section]
\newtheorem{vidu}{Ví dụ}[section]
\newtheorem{lemma}{Lemma}[section]
\newtheorem{notation}{Notation}[section]
\newtheorem{principle}{Principle}[section]
\newtheorem{problem}{Problem}[section]
\newtheorem{baitoan}{Bài toán}[section]
\newtheorem{proposition}{Proposition}[section]
\newtheorem{question}{Question}[section]
\newtheorem{cauhoi}{Câu hỏi}[section]
\newtheorem{remark}{Remark}[section]
\newtheorem{luuy}{Lưu ý}[section]
\newtheorem{theorem}{Theorem}[section]
\newtheorem{dinhly}{Định lý}[section]
\usepackage[left=0.5in,right=0.5in,top=1.5cm,bottom=1.5cm]{geometry}
\usepackage{fancyhdr}
\pagestyle{fancy}
\fancyhf{}
\lhead{\small Sect.~\thesection}
\rhead{\small \nouppercase{\leftmark}}
\renewcommand{\sectionmark}[1]{\markboth{#1}{}}
\cfoot{\thepage}
\def\labelitemii{$\circ$}

\title{Problems in Elementary Programming}
\author{Nguyễn Quản Bá Hồng\footnote{Independent Researcher, Ben Tre City, Vietnam\\e-mail: \texttt{nguyenquanbahong@gmail.com}; website: \url{https://nqbh.github.io}.}}
\date{\today}

\begin{document}
\maketitle
\begin{abstract}
	1 bộ sưu tập các bài toán chọn lọc từ cơ bản đến nâng cao cho Lập Trình. Tài liệu này là phần bài tập bổ sung cho tài liệu chính \href{https://github.com/NQBH/hobby/blob/master/elementary_mathematics/grade_6/NQBH_elementary_mathematics_grade_6.pdf}{GitHub\texttt{/}NQBH\texttt{/}hobby\texttt{/}elementary mathematics\texttt{/}grade 6\texttt{/}lecture}\footnote{\textsc{url}: \url{https://github.com/NQBH/hobby/blob/master/elementary_mathematics/grade_6/NQBH_elementary_mathematics_grade_6.pdf}.} của tác giả viết cho Toán lớp 6. Phiên bản mới nhất của tài liệu này được lưu trữ ở link sau: \href{https://github.com/NQBH/hobby/blob/master/elementary_mathematics/grade_6/problem/NQBH_elementary_mathematics_grade_6_problem.pdf}{GitHub\texttt{/}NQBH\texttt{/}hobby\texttt{/}elementary mathematics\texttt{/}grade 6\texttt{/}problem}\footnote{\textsc{url}: \url{https://github.com/NQBH/hobby/blob/master/elementary_mathematics/grade_6/problem/NQBH_elementary_mathematics_grade_6_problem.pdf}.}.
\end{abstract}
\tableofcontents
\newpage

%------------------------------------------------------------------------------%

\textit{Programming languages}: Pascal, C\texttt{/}C++, Python.

\section{Basic Problems}

\begin{baitoan}[Giải phương trình bậc nhất $ax + b$]
	Viết thuật toán \& chương trình để biện luận 
\end{baitoan}


%------------------------------------------------------------------------------%

\section{Problems on Loops}

\begin{baitoan}[Dãy Fibonacci]
	Biết dãy Fibonacci $(F_n)_{n=0}^\infty$ được xác định bằng công thức truy hồi như sau: $F_0 = 0$, $F_1 = 1$, $F_{n} = F_{n-1} + F_{n-2}$, $\forall n\in\mathbb{N}$, $n\ge 2$. Viết thuật toán \& chương trình in ra, với $n\in\mathbb{N}$ bất kỳ được nhập từ bàn phím:
	\begin{enumerate*}
		\item[(a)] Số Fibonacci thứ $n$: $F_n$.
		\item[(b)] $n$ số Fibonacci đầu tiên.
	\end{enumerate*}
\end{baitoan}

\begin{baitoan}[Số nguyên tố vs. hợp số]
	Viết thuật toán \& chương trình để kiểm tra 1 số $n\in\mathbb{N}$ là số nguyên tố hay hợp số.
\end{baitoan}

\begin{baitoan}[Giai thừa]
	Viết thuật toán \& chương trình tính $n!$ với $n\in\mathbb{N}$ bất kỳ được nhập từ bàn phím.
\end{baitoan}

\begin{baitoan}[Chuyển đổi giữa các hệ cơ số]
	Viết thuật toán \& chương trình để chuyển đổi $n\in\mathbb{Z}$ sang hệ cơ số $b\in[2,32]\cap\mathbb{N}$ bất kỳ được nhập từ bàn phím.
\end{baitoan}

%------------------------------------------------------------------------------%

\printbibliography[heading=bibintoc]
	
\end{document}