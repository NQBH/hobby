\documentclass{article}
\usepackage[backend=biber,natbib=true,style=authoryear]{biblatex}
\addbibresource{/home/nqbh/reference/bib.bib}
\usepackage{tocloft}
\renewcommand{\cftsecleader}{\cftdotfill{\cftdotsep}}
\usepackage[colorlinks=true,linkcolor=blue,urlcolor=red,citecolor=magenta]{hyperref}
\usepackage{algorithm,algpseudocode,amsmath,amssymb,amsthm,float,graphicx,mathtools}
\allowdisplaybreaks
\numberwithin{equation}{section}
\newtheorem{assumption}{Assumption}[section]
\newtheorem{conjecture}{Conjecture}[section]
\newtheorem{corollary}{Corollary}[section]
\newtheorem{definition}{Definition}[section]
\newtheorem{example}{Example}[section]
\newtheorem{lemma}{Lemma}[section]
\newtheorem{notation}{Notation}[section]
\newtheorem{principle}{Principle}[section]
\newtheorem{problem}{Problem}[section]
\newtheorem{proposition}{Proposition}[section]
\newtheorem{question}{Question}[section]
\newtheorem{remark}{Remark}[section]
\newtheorem{theorem}{Theorem}[section]
\usepackage[left=1cm,right=1cm,top=5mm,bottom=5mm,footskip=4mm]{geometry}
\def\labelitemii{$\circ$}

\title{The Science of Living}
\author{Alfred Adler}
\date{\today}

\begin{document}
\maketitle
\tableofcontents
\vspace{5mm}
\textbf{The Science of Living.} ``Originally published in 1930 \textit{The Science of Living} looks at Individual Psychology as a science. Adler discusses the various elements of Individual Psychology \& its application to everyday life: including the inferiority complex, the superiority complex \& other social aspects, such as, love \& marriage, sex \& sexuality, children \& their education. This is an important book in the history of psychoanalysis \& Alderian therapy.''

\section*{A Note on the Author \& His Work}
``\textsc{Dr. Alfred Adler}'s work in psychology, while it is scientific \& general in method, is essentially the study of the separate personalities we are, \& is therefore called Individual Psychology. Concrete, particular, unique human beings are the subjects of this psychology, \& it can only be truly learned from the men, women \& children we meet.

The supreme importance of this contribution to modern psychology is due to the manner in which it reveals how all the activities of the soul are drawn together into the service of the individual, how all his faculties \& strivings are related to 1 end. We are enabled by this to enter into the ideals, the difficulties, the efforts \& discouragements of our fellow-men, in such a way that we may obtain a whole \& living picture of each as a personality. In this co-ordinating idea, something like finality is achieved, though we must understand it as finality of foundation. There has never before been a method so rigorous \& yet adaptable for following the fluctuations of that most fluid, variable \& elusive of all realities, the individual human soul.

Since Adler regards not only science but even intelligence itself as the result of the communal efforts of humanity, we shall find his consciousness of his own unique contribution more than usually tempered by recognition of his collaborators, both past \& contemporary. It will therefore be useful to consider Adler's relation to the movement called Psycho-analysis, \& 1st of all to recall, however briefly, the philosophic impulses which inspired the psycho-analytic movement as a whole.

The conception of the Unconscious as vital memory -- biological memory -- is a common to modern psychology as a whole. But Freud, from the 1st a specialist in hysteria, took the memories of success or failure in the sexual life, as of the 1st -- \& almost the only -- importance. Jung, a psychiatrist of genius, has tried to widen this distressingly narrow view, by seeking to reveal the super-individual or racial memories which, he believes, have as much power as the sexual \& a higher kind of value for life.

It was left to Alfred Adler, a physician of wide \& general experience, to unite the conception of the Unconscious more firmly with biological reality. A man of the original school of psycho-analysts, he had done much work by that method of analyzing memories out of their coagulated emotional state into clearness \& objectivity. But he showed that the whole scheme of memory is different in every individual. Individuals do not form their unconscious memories all around the same central motive -- not all around sexuality, for instance. In every individual we find an individual way of selecting its experiences from all possible experience. What is the principle of that selectivity? Adler has answered that it is, fundamentally, the organic consciousness of a \textit{need}, of some specific inferiority which has to be compensated. It is as though every soul had consciousness of its whole physical reality, \& were concentrated, with sleepless insistence, upon achieving compensation for the defects in it.

Thus the whole life of the small man, for instance, would be interpretable as a struggle to achieve immediate greatness in some way, \& that of a deaf man to obtain a compensation for not hearing. It is not so simple as that, of course, for a system of defects may give rise to a constellation of guiding ideas, \& also in human life we have to deal with imaginary inferiorities \& fantastic strivings, but even here the principle is the same.

The sexual life, far from controlling all activities, fits perfectly into the frame of those more important strivings, for it is pre-eminently under the control of emotion, \& emotion is moulded by the entire vital history. Thus a Freudian analysis gives a true account of the sexual \textit{consequences} of a given life-line, but it is a true \textit{diagnosis} only in that sense.

Psychology becomes now for the 1st time rooted in biology. The tendencies of the soul, \& the mind's development, are seen to be controlled from the 1st by the effort to compensate for organic defects or for positions of inferiority. Everything that is exceptional or individual in the disposition of an organic being originates in this way. The principle is common to man \& animal, probably even to the vegetable kingdom also; \& the special endowments of species are to be taken as arising from experience of defects \& inferiorities in relation to their environment, which has been successfully compensated by activity, growth \& structure.

There is nothing new in the idea of compensation as a biological principle, for it has been long known that the body will over-develop certain parts in compensation for the injury of others. If 1 kidney ceases to function, e.g., the other develops abnormally until it does the work of both; if the heart springs a leak in a valve, the whole organ grows larger to allow for its loss of efficiency, \& when nervous tissue is destroyed, adjacent tissue of another kind endeavors to take on the nerve-function. The compensatory developments of the whole organism to meet the exigencies of any special work or exertion are too numerous \& well known to need illustration. But it is Dr. Adler who has 1st transferred this principle bodily to psychology as a fundamental idea, \& demonstrated the part it plays in the soul \& intellect.

Adler recommends the study of Individual Psychology not only to doctors, but generally to laymen \& especially to teachers. Culture in psychology has become a general necessity, \& must be firmly advocated in the teeth of popular opposition to it, which is founded upon the notion that modern psychology requires an unhealthy concentration of the mind upon cases of disease \& misery. It is true that the literature of psychoanalysis has revealed the most central \& the most universal evils in modern society. But it is not now a question of contemplating our errors, it is necessary that we should learn by them. We have been trying to live as though the soul of man were not a reality, as though we could build up a civilized life in defiance of psychic truths. What Adler proposes is not the universal study of psycho-pathology, but the practical reform of society \& culture in accordance with a positive \& scientific psychology to which he has contributed the 1st principles. But this is impossible if we are too much afraid of the truth. The clearer consciousness of right aims in life, which is indispensable to us, cannot be gained without a deeper understanding also of the mistakes in which we are involved. We may not desire to know ugly facts, but the more truly we are aware of life, the more clearly we perceive the real errors which frustrate it, much as the concentration of a light gives definition to the shadows.

A positive psychology, useful for human life, cannot be derived from the psychic phenomena alone, still less from pathological manifestations. It requires also a regulative principle, \& Adler has not shrunk from this necessity, by recognizing, as if it were of absolute metaphysical validity, the logic of our communal life in the world.

Recognizing this principle, we must proceed to estimate the psychology of the individual in relation too it. The way in which an individual's inner life is related to the communal being is distinguishable in 3 ``life-attitudes,'' as they are called -- his general reactions to society, to work \& to love.

By their feeling towards society as a whole -- to any other \& to tell others -- man \& women may know how much social courage they possess. The feeling of inferiority is always manifested in a sense of fear or uncertainty in the presence of society, whether its outward expression is 1 of timidity or defiance, reserve or over-anxiety. All feelings of innate suspicion or hostility, of an undefined caution \& desire for some concealment, when such feelings affect the individual in social relations generally, evince the same tendency to withdraw from reality, which inhibits self-affirmation. The ideal, or rather normal, attitude to society is an unstrained \& unconsidered assumption of human equality unchanged by any inequalities of position. Social courage depends upon this feeling of secure membership of the human family, a feeling which depends upon the harmony of one's own life. By the tone of his feeling towards his neighbors, his township \& nation \& to other nationalities, \& even by his reactions when he reads of all these things in his newspaper, a man may infer how securely his own soul is grounded in itself.

The attitude towards work is closely dependent upon this self-security in society. In the occupation by which a man earns his share in social goods \& privileges, he has to face the logic of social needs. If he has too great a sense of weakness or division from society, it will make him unable to believe that his worth will ever be recognized, \& he will not even work for recognition: instead, he will play for safety, \& work for money or advantage only, suppressing his own valuation of what is the truest service he can render. He will always be afraid to supply or demand the best, for fear it may not pay. Or he may be always seeking for some quiet backwater of the economic life, where he can do something just as he likes himself, without proper consideration of either usefulness or profit. In both cases it is not only society that suffers by not getting the best service: the individual who has not attained his proper social significance is also deeply dissatisfied. The modern world is full of men, both successful \& unsuccessful in a worldly sense, who are in open conflict with their occupation. They do not believe in it, \& they blame social \& economic conditions with some real justice; but it is also a fact that they have often had too little courage to fight for the best value in their economic function. They were afraid to claim the right to give what they genuinely believed in, or else they felt disdainful of the service society really needed of them. Hence they pursued their gain in an individualistic or even furtive spirit. We must, of course, recognize that so much is wrong in the organization of society, that, besides the possibility of making mistakes of judgment, the individual who is determined to render real social service has often to face heavy opposition. But it is precisely that sense of struggle to give his best which the individual needs no less than society benefits by it. One cannot love a vocation which does not afford some experience of victory over difficulties, \& not merely of compromise with them.

It is the 3rd of these life-attitudes -- the attitude to love -- which determines the course of the erotic life.

'' -- \cite[pp. 9--]{Adler2013}

%------------------------------------------------------------------------------%

\section{The Science of Living}

%------------------------------------------------------------------------------%

\section{The Inferiority Complex}

%------------------------------------------------------------------------------%

\section{The Superiority Complex}

%------------------------------------------------------------------------------%

\section{The Style of Life}

%------------------------------------------------------------------------------%

\section{Old Remembrances}

%------------------------------------------------------------------------------%

\section{Attitudes \& Movements}

%------------------------------------------------------------------------------%

\section{Dreams \& Their Interpretation}

%------------------------------------------------------------------------------%

\section{Problem Children \& Their Education}

%------------------------------------------------------------------------------%

\section{Social Problems \& Social Adjustment}

%------------------------------------------------------------------------------%

\section{Social Feeling, Common Sense \& the Inferiority Complex}

%------------------------------------------------------------------------------%

\section{Love \& Marriage}

%------------------------------------------------------------------------------%

\section{Sexuality \& Sex Problems}

%------------------------------------------------------------------------------%

\section{Conclusion}

%------------------------------------------------------------------------------%

\printbibliography[heading=bibintoc]
	
\end{document}