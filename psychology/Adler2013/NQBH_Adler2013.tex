\documentclass{article}
\usepackage[backend=biber,natbib=true,style=authoryear]{biblatex}
\addbibresource{/home/nqbh/reference/bib.bib}
\usepackage{tocloft}
\renewcommand{\cftsecleader}{\cftdotfill{\cftdotsep}}
\usepackage[colorlinks=true,linkcolor=blue,urlcolor=red,citecolor=magenta]{hyperref}
\usepackage{algorithm,algpseudocode,amsmath,amssymb,amsthm,float,graphicx,mathtools}
\allowdisplaybreaks
\numberwithin{equation}{section}
\newtheorem{assumption}{Assumption}[section]
\newtheorem{conjecture}{Conjecture}[section]
\newtheorem{corollary}{Corollary}[section]
\newtheorem{definition}{Definition}[section]
\newtheorem{example}{Example}[section]
\newtheorem{lemma}{Lemma}[section]
\newtheorem{notation}{Notation}[section]
\newtheorem{principle}{Principle}[section]
\newtheorem{problem}{Problem}[section]
\newtheorem{proposition}{Proposition}[section]
\newtheorem{question}{Question}[section]
\newtheorem{remark}{Remark}[section]
\newtheorem{theorem}{Theorem}[section]
\usepackage[left=1cm,right=1cm,top=5mm,bottom=5mm,footskip=4mm]{geometry}
\def\labelitemii{$\circ$}

\title{The Science of Living}
\author{Alfred Adler}
\date{\today}

\begin{document}
\maketitle
\tableofcontents
\vspace{5mm}
\textbf{The Science of Living.} ``Originally published in 1930 \textit{The Science of Living} looks at Individual Psychology as a science. Adler discusses the various elements of Individual Psychology \& its application to everyday life: including the inferiority complex, the superiority complex \& other social aspects, such as, love \& marriage, sex \& sexuality, children \& their education. This is an important book in the history of psychoanalysis \& Alderian therapy.''

\section*{A Note on the Author \& His Work}
``\textsc{Dr. Alfred Adler}'s work in psychology, while it is scientific \& general in method, is essentially the study of the separate personalities we are, \& is therefore called Individual Psychology. Concrete, particular, unique human beings are the subjects of this psychology, \& it can only be truly learned from the men, women \& children we meet.

The supreme importance of this contribution to modern psychology is due to the manner in which it reveals how all the activities of the soul are drawn together into the service of the individual, how all his faculties \& strivings are related to 1 end. We are enabled by this to enter into the ideals, the difficulties, the efforts \& discouragements of our fellow-men, in such a way that we may obtain a whole \& living picture of each as a personality. In this co-ordinating idea, something like finality is achieved, though we must understand it as finality of foundation. There has never before been a method so rigorous \& yet adaptable for following the fluctuations of that most fluid, variable \& elusive of all realities, the individual human soul.

Since Adler regards not only science but even intelligence itself as the result of the communal efforts of humanity, we shall find his consciousness of his own unique contribution more than usually tempered by recognition of his collaborators, both past \& contemporary. It will therefore be useful to consider Adler's relation to the movement called Psycho-analysis, \& 1st of all to recall, however briefly, the philosophic impulses which inspired the psycho-analytic movement as a whole.

The conception of the Unconscious as vital memory -- biological memory -- is a common to modern psychology as a whole. But Freud, from the 1st a specialist in hysteria, took the memories of success or failure in the sexual life, as of the 1st -- \& almost the only -- importance. Jung, a psychiatrist of genius, has tried to widen this distressingly narrow view, by seeking to reveal the super-individual or racial memories which, he believes, have as much power as the sexual \& a higher kind of value for life.

It was left to Alfred Adler, a physician of wide \& general experience, to unite the conception of the Unconscious more firmly with biological reality. A man of the original school of psycho-analysts, he had done much work by that method of analyzing memories out of their coagulated emotional state into clearness \& objectivity. But he showed that the whole scheme of memory is different in every individual. Individuals do not form their unconscious memories all around the same central motive -- not all around sexuality, for instance. In every individual we find an individual way of selecting its experiences from all possible experience. What is the principle of that selectivity? Adler has answered that it is, fundamentally, the organic consciousness of a \textit{need}, of some specific inferiority which has to be compensated. It is as though every soul had consciousness of its whole physical reality, \& were concentrated, with sleepless insistence, upon achieving compensation for the defects in it.

Thus the whole life of the small man, for instance, would be interpretable as a struggle to achieve immediate greatness in some way, \& that of a deaf man to obtain a compensation for not hearing. It is not so simple as that, of course, for a system of defects may give rise to a constellation of guiding ideas, \& also in human life we have to deal with imaginary inferiorities \& fantastic strivings, but even here the principle is the same.

The sexual life, far from controlling all activities, fits perfectly into the frame of those more important strivings, for it is pre-eminently under the control of emotion, \& emotion is moulded by the entire vital history. Thus a Freudian analysis gives a true account of the sexual \textit{consequences} of a given life-line, but it is a true \textit{diagnosis} only in that sense.

Psychology becomes now for the 1st time rooted in biology. The tendencies of the soul, \& the mind's development, are seen to be controlled from the 1st by the effort to compensate for organic defects or for positions of inferiority. Everything that is exceptional or individual in the disposition of an organic being originates in this way. The principle is common to man \& animal, probably even to the vegetable kingdom also; \& the special endowments of species are to be taken as arising from experience of defects \& inferiorities in relation to their environment, which has been successfully compensated by activity, growth \& structure.

There is nothing new in the idea of compensation as a biological principle, for it has been long known that the body will over-develop certain parts in compensation for the injury of others. If 1 kidney ceases to function, e.g., the other develops abnormally until it does the work of both; if the heart springs a leak in a valve, the whole organ grows larger to allow for its loss of efficiency, \& when nervous tissue is destroyed, adjacent tissue of another kind endeavors to take on the nerve-function. The compensatory developments of the whole organism to meet the exigencies of any special work or exertion are too numerous \& well known to need illustration. But it is Dr. Adler who has 1st transferred this principle bodily to psychology as a fundamental idea, \& demonstrated the part it plays in the soul \& intellect.

Adler recommends the study of Individual Psychology not only to doctors, but generally to laymen \& especially to teachers. Culture in psychology has become a general necessity, \& must be firmly advocated in the teeth of popular opposition to it, which is founded upon the notion that modern psychology requires an unhealthy concentration of the mind upon cases of disease \& misery. It is true that the literature of psychoanalysis has revealed the most central \& the most universal evils in modern society. But it is not now a question of contemplating our errors, it is necessary that we should learn by them. We have been trying to live as though the soul of man were not a reality, as though we could build up a civilized life in defiance of psychic truths. What Adler proposes is not the universal study of psycho-pathology, but the practical reform of society \& culture in accordance with a positive \& scientific psychology to which he has contributed the 1st principles. But this is impossible if we are too much afraid of the truth. The clearer consciousness of right aims in life, which is indispensable to us, cannot be gained without a deeper understanding also of the mistakes in which we are involved. We may not desire to know ugly facts, but the more truly we are aware of life, the more clearly we perceive the real errors which frustrate it, much as the concentration of a light gives definition to the shadows.

A positive psychology, useful for human life, cannot be derived from the psychic phenomena alone, still less from pathological manifestations. It requires also a regulative principle, \& Adler has not shrunk from this necessity, by recognizing, as if it were of absolute metaphysical validity, the logic of our communal life in the world.

Recognizing this principle, we must proceed to estimate the psychology of the individual in relation too it. The way in which an individual's inner life is related to the communal being is distinguishable in 3 ``life-attitudes,'' as they are called -- his general reactions to society, to work \& to love.

By their feeling towards society as a whole -- to any other \& to tell others -- man \& women may know how much social courage they possess. The feeling of inferiority is always manifested in a sense of fear or uncertainty in the presence of society, whether its outward expression is 1 of timidity or defiance, reserve or over-anxiety. All feelings of innate suspicion or hostility, of an undefined caution \& desire for some concealment, when such feelings affect the individual in social relations generally, evince the same tendency to withdraw from reality, which inhibits self-affirmation. The ideal, or rather normal, attitude to society is an unstrained \& unconsidered assumption of human equality unchanged by any inequalities of position. Social courage depends upon this feeling of secure membership of the human family, a feeling which depends upon the harmony of one's own life. By the tone of his feeling towards his neighbors, his township \& nation \& to other nationalities, \& even by his reactions when he reads of all these things in his newspaper, a man may infer how securely his own soul is grounded in itself.

The attitude towards work is closely dependent upon this self-security in society. In the occupation by which a man earns his share in social goods \& privileges, he has to face the logic of social needs. If he has too great a sense of weakness or division from society, it will make him unable to believe that his worth will ever be recognized, \& he will not even work for recognition: instead, he will play for safety, \& work for money or advantage only, suppressing his own valuation of what is the truest service he can render. He will always be afraid to supply or demand the best, for fear it may not pay. Or he may be always seeking for some quiet backwater of the economic life, where he can do something just as he likes himself, without proper consideration of either usefulness or profit. In both cases it is not only society that suffers by not getting the best service: the individual who has not attained his proper social significance is also deeply dissatisfied. The modern world is full of men, both successful \& unsuccessful in a worldly sense, who are in open conflict with their occupation. They do not believe in it, \& they blame social \& economic conditions with some real justice; but it is also a fact that they have often had too little courage to fight for the best value in their economic function. They were afraid to claim the right to give what they genuinely believed in, or else they felt disdainful of the service society really needed of them. Hence they pursued their gain in an individualistic or even furtive spirit. We must, of course, recognize that so much is wrong in the organization of society, that, besides the possibility of making mistakes of judgment, the individual who is determined to render real social service has often to face heavy opposition. But it is precisely that sense of struggle to give his best which the individual needs no less than society benefits by it. One cannot love a vocation which does not afford some experience of victory over difficulties, \& not merely of compromise with them.

It is the 3rd of these life-attitudes -- the attitude to love -- which determines the course of the erotic life. Where the 2 previous life-attitudes, to society \& to work, have been rightly adjusted, this last comes right by itself. Where it is distorted \& wrong it cannot be improved by itself apart from the others. Although we can think how to improve the social relations \& the occupations, a concentration of thought upon the individual sex-problem is almost sure to make it worse. For this is far more the sphere of results than of course. A soul that is defeated in ordinary social life, or thwarted in its occupation, acts in the sex-life as though it were trying to obtain compensation for the kinds of expression of which it fails in their proper spheres. This is actually the best way in which we can understand all sexual vagaries, whether they isolate the individual, degrade the sexual partner or in any way distort the instinct. The friendships of an individual also are integral with the love-life as a whole; not, as the 1st psycho-analysts imagined, because friendship is a sublimation of sexual attraction, but the other way about. Sexual compulsion -- sex as an insubordinate psychic factor -- is an abnormal substitute for the vitalizing intimacy of useful friendships, \& homosexuality is always the consequence of incapability for love.

The meaning \& value which we give to sensations are also united closely with the erotic life, as many good poets have testified. The quality of our feeling for Nature, our response to the beauty of sea \& land, \& to significances of form \& sound \& color, as well as our confidence in scenes of storm \& gloom, are all involved with our integrity as lovers. The aesthetic life, with all it means to art \& culture, is thus ultimately derived, through individuals, from social courage \& intelligent usefulness.

We ought not to regard the communal feeling as something to be created with difficulty. It is as natural \& inherent as egoism itself, \& indeed as a principle of life it has priority. We have not to create, but only to liberate, it where it is repressed. It is the saving principle of life as we experience it. If anyone thinks that the services of 'busmen, railwaymen \& milkmen would be rendered as well as they are without the presence of very much instinctive communal feeling he must be suspected of a highly neurotic scheme of apperception. What inhibits it is, to speak bluntly, the enormous vanity of the human soul, which is, moreover, so subtle that no professional psychologist before Adler had been able to demonstrate it, though a few artists had divined its omnipresence. All unsuspected as it often is, the ambition of many a minor journalist or shop-assistant, to say nothing of the great ones of the world, would be enough to bring about the fall of an archangel. Every feeling of inferiority that has embittered his contact with life has fed the imagination of greatness with another god-like assumption until, in many cases, the fantasy has become so inflated as to demand not even supremacy in this world for its appeasement, but the creation of a new world altogether, \& to be the god of it. This revelation of the depth of human nature is verified, not so strikingly from the study of cases of practical ambition, however Napoleonic, as from those of passive resistance, procrastination, \& malingering, for it is these which show most clearly that an individual who feels painfully unable to dominate the real world will refuse to co-operate with it, at whatever disadvantage to himself, partly in order to tyrannize over a narrower sphere, \& partly even from an irrational feeling that the real world, without his divine assistance, will some day crumble \& shrink to his own diminished measure.\footnote{In case this should seem an exaggeration, we may recall the fact that nearly all the narrowest kind of sects, religious or secular, have a belief in world-catastrophe: the world from which they have withdrawn, \& which they despair of converting, is to be brought to destruction, \& only a remnant will survive, who will be of their own persuasion.}

The question is thus raised, how should we act, knowing this tendency to inordinate vanity in the human soul, \& that we dare not merely add to that vanity by assuming ourselves to be miraculous exceptions? Adler's reply is that we should preserve a certain attitude to all our experience, which he calls the attitude of ``half-\&-half.'' Our conception of normal behavior should be to allow the world or society, or the person with whom we are confronted, to be somehow in the right equally with ourselves. We should not depreciate either ourselves or our environment; but, assuming that each is 1-half in the right, affirm the reality of ourselves \& others equally. This applies not only to contacts with other souls, but to our mental reactions towards rainy weather, holidays or comforts that we cannot afford, \& even to the omnibus we have just missed.

Rightly understood, this is not an ideal of difficult \& distasteful humility. It is in reality a tremendous assumption of worth, to claim exactly equal reality \& omnipotence with the whole of the rest of creation, in whatever particular manifestation we may be meeting with it. To claim less than this is a false humility, for what results from any contact we make does in fact depend for half its reality upon the way in which we make it. The individual should affirm his part in everything which occurs to him, as his own half of it.

This is often a particularly difficult counsel to keep in relation to the occupation. In their business, people face more naked realities than are usually allowed to appear in social life; \& it is often almost impossible to allow equal validity to one's own aims \& to the conditions of a disorganized world. To do so, means the admission that conditions, just such as they are, are one's real problem -- \&, indeed, one's proper sphere of action. The division of labor, logical \& useful as it is in itself, has given opportunity for human megalomania to create entirely false inequalities, distinctions \& injustices, so that we live in an economic disorder which will hardly hold together. To such crazy conditions, the best of men often find it difficult to oppose themselves with perseverance, equally grating its reality \& working for its reform. They are tempted to acquiesce in disorder by some inner subterfuge, or to devote themselves to superficial remedies which evade the real problem; \& sometimes they treat their work-life as an unavoidable contamination by things inherently squalid, quite unaware that such an attitude makes them conceited, haughty \&, in a profound sense, unscrupulous. It occurs to very few that the right way would be to make alliance on human grounds with others in the same predicament \& profession, to assert its proper dignity as a social service \& improve it; but this is the only way in which the individual can really be reconciled with his economic function. Many of those who complain most about the conditions prevailing in their work are doing nothing whatever to reorganize it as a function of human life, \& never think of attacking the anarchic individualism which is its ruin. We derive it from Individual Psychology, as a categorical imperative, that every man's duty is to work to make his profession, whatever it may be, into a brotherhood, a friendship, a social unity with a powerful morale of co-operation, \& that if a man does not want to do this his own psychological state is precarious. It is true that now, in many professions, the task that this presents is terribly difficult. It is all the more essential that the effort should be made towards integration. For a man's work will never liberate the forces of his psyche unless he is striving, in a large sense, to make it the expression of his whole being, \& his idea of his profession must be not only an executive in which he has independence of action, but also a legislative in which he has some authority of direction. In a man's business life that half-\&-half valuation leads equally to recognition of reality \& to struggle with it by the only realistic method, which is necessarily co-operative.

The pedagogic principles of Individual Psychology, infallible as far as they go, are useless without this practical work of social organization. What has been written above of an individual's duty in his occupation applis in a large sense to his entire social function. A person's function includes active membership of his nation \& of humanity, to say nothing of his family. There is a certain parliament which rises for no vacation, \& to whose decisions all elected assemblies must in the end defer. It meets in schools, markets, \& everywhere on sea \& land, for it is the Parliament of Man, in which every word or look exchanged, whether of courtesy or recrimination, of wisdom or folly, has its measure of importance in the affairs of the race. It is everyone's interest to make this wide assembly more united \& its discussion more intelligible, for none of us has any real human existence except by reflection from it. When its conclaves are peaceful, all our lives are heightened in tone, health \& wealth accrues \& arts \& education flourish; when its conversation is reversed \& suspicious, work fails, men starve \& children languish. In the heat of its dissensions we perish by the million. All its decrees, by which we live or die, \& grow or decay, are rooted in our individual attitudes towards man, woman \& child in every relation of life.

When we face, objectively, this fact of the relation of all souls \& their mutual responsibility, what are we to think of the inner confusion of the neurotic? Is it not simply a narrowing of the sphere of interest, an over-concentration upon certain personal or subjective gains? The neurotic soul is the result of treating the rest of humanity as though its life \& aims were altogether of less importance than one's own, \& thus losing interest in any larger life. Paradoxically, it often happens that a neurotic has very large schemes of saving himself \& others. He is intelligent enough to try to compensate his real sense of isolation \& impotence in the human assembly, by a fantasy of exaggerated importance \& beneficent activity. He may want to reform education, to abolish war, to establish universal brotherhood or create a new culture, \& even plans or joins societies with these aims. He is defeated in such aims, of course, by the unreality of his contact with others \& with life as a whole. It is as though he had taken a standpoint outside of life altogether \& were trying to direct it by some unexplained magic.

Modern city life especially, with its intellectualism, gives unlimited scope for the neurotic thus to compensate his real unsociability with imaginary messianism, \& the result is the disintegration of a people full of saviors who are not on speaking terms.

What is needed, of course, is something very different. It is not that the individual should renounce messianism; for it is a fact that a share of responsibility for the whole future of the race is his alone. It is only necessary that he should take a reasonable view of his power to save society, correctly viewed from his own standpoint: he must become able to regard his immediate personal relations \& his occupations as if \textit{they} were of world-importance, for in fact they are so, being the only world-meaning an individual has. When they are chaotic or wrong, it is because we do not, in day-to-day experience, treat them as things of universal meaning. We sometimes treat them as important, no doubt, but generally in a personal sense only.

This tendency of the modern soul, to narrow the sphere of interest, both practically \& ideally, is most difficult to subdue, because it is reinforced by the scheme of apperception. For that reason an individual alone cannot do it, excepting only in rare cases. He needs conference with other minds, \& an entirely new kind of conference. A resolution to treat one's immediate surroundings \& daily activities as if they were the supreme significance of life brings an individual immediately into conflict with internal resistances of his own, \& often with external difficulties also, which he cannot at once understand \& which no others could rightly estimate unless they were making the same experiment. Hence, the practice of Individual Psychology demands that its students should submit themselves to mutual scrutiny, each one to be estimated by the others as a whole personality. This practice, striking at the root of the false individualism which is the basis of all neurosis, is naturally very difficult to initiate. Upon its success, however, depends the whole future of psycho-analysis as an influence in life at large, outside of clinics \& consulting rooms.

In Vienna the work of such groups has already made itself felt in education. The co-operation it has established between teachers \& medical practitioners has revolutionized the work of certain schools, \& established an equality between teachers \& pupils \& between pupils themselves, which has cured many children of criminal tendencies, dullness \& laziness. Abolition of competition \& the cultivation of encouragement have been found to liberate the energy of both pupils \& teachers. These changes are already affecting the surrounding family life, which comes into question immediately the child is psychologically considered. Education, though naturally the 1st, is not the only sphere of life which out to be invaded by the activity of such groups. Business \& political circles, which experience the deadlock of modern life most acutely, need to be vitalized with knowledge of human nature, which they have forgotten how to recognize.

It is for this work of releasing a new energy for daily life \& its reformation, that Alfred Adler has founded the International Society for Individual Psychology. The culture of human behavior which this work has begun already to propagate might well be mistaken for an almost platitudinous ethics, but for 2 things -- its practical results, \& the background of scientific method out of which it is appearing. In his realistic grasp of the social nature of the individual's problem \& his inexorable demonstration of the unity of health \& harmonious behavior, Adler resembles no one so much as the great Chinese thinkers. If the occidental world is not too far gone to make use of his service, he may well come to be known as the Confucius of the West. -- \textsc{Phillipe Mairet}'' -- \cite[pp. 9--30]{Adler2013}

%------------------------------------------------------------------------------%

\section{The Science of Living}
``Only a science which is directly related to life, said the great philosopher William James, is really a science. It might also be said that in a science which is directly related to life theory \& practice become almost inseparable. The science of life, precisely because it models itself directly on the movement of life, becomes a science of living. These considerations apply with special force to the science of Individual Psychology. Individual Psychology tries to see individual lives as a whole \& regards each single reaction, each movement \& impulse as an articulated part of an individual attitude towards life. Such a science is of necessity oriented in a practical sense, for which the aid of knowledge we can correct \& alter our attitudes. Individual Psychology is thus \textit{prophetic} in a double sense: not only does it predict what will happen, but, like the prophet Jonah, it predicts what \textit{will} happen in order that it should \textit{not} happen.

The science of Individual Psychology developed out of the effort to understand that mysterious creative power of life -- that power which expresses itself in the desire to develop, to strive \& to achieve -- \& even to compensate for defeats in 1 direction by striving for success in another. This power is \textit{teleological} -- it expresses itself in the striving after a goal, \& in this striving every bodily \& psychic movement is made to co-operate. It is thus absurd to study bodily movements \& mental conditions abstractly without relation to an individual whole. It is absurd, e.g., that in criminal psychology we should pay so much more attention to the crime than to the criminal. It is the criminal, not the crime that counts, \& no matter how much we contemplate the criminal act we shall never understand its criminality unless we see it as an episode in the life of a particular individual. The same outward act may be criminal in 1 case \& not criminal in another. The important thing is to understand the individual context -- the goal of an individual's life which marks the line of direction for all his acts \& movements. This goal enables us to understand the hidden meaning behind the various separate acts -- we see them as parts of a whole. Vice versa when we study the parts -- provided we study them as parts of a whole -- we get a better sense of the whole.

In the author's own case the interest in psychology developed out of the practice of medicine. The practice of medicine provided the teleological or purposive viewpoint which is necessary for the understanding of psychological facts. In medicine we see all organs striving to develop towards definite goals. They have definite forms which they achieve upon maturity. Moreover, in cases where there are organic defects we always find nature making special efforts to overcome the deficiency, or else to compensate for it by developing another organ to take over the functions of the defective one. Life always seeks to continue, \& the life force never yields to external obstacles without a struggle.

Now the movement of the psyche is analogous to the movement of organic life. In each mind there is the conception of a goal or ideal to get beyond the present state, \& to overcome the present deficiencies \& difficulties by postulating a concrete aim for the future. By means of this concrete aim or goal the individual can think \& feel himself superior to the difficulties of the present because he has in mind his success of the future. Without the sense of a goal individual activity would cease to have anything meaning.

All evidence points to the fact that the fixing of this goal -- giving it a concrete form -- must take place early in life, during the formative period of childhood. A kind of prototype or model of a matured personality begins to develop at this time. We can imagine how the process takes place. A child, being weak, feels inferior \& finds itself in a situation which it cannot bear. Hence it strives to develop, \& it strives to develop along a line of direction fixed by the goal which it chooses for itself. The material used for development at this stage is less important than the goal which decides the line of direction. How this goal is fixed it is difficult to say, but it is obvious that such a goal exists \& that it dominates the child's every movement. Little is indeed understood about powers, impulses, reasons, abilities or disabilities at this early period. As yet there is really no key, for the direction is definitely established only after the child has fixed its goal. Only when we see the direction in which a life is tending can we guess what steps will be taken in the future.

When the prototype -- that early personality which embodies the goal -- is formed, the line of direction is established \& the individual becomes definitely oriented. It is this fact which enables us to predict what will happen later in life. The individual's apperceptions are from then on bound to fall into a groove established by the line of direction. The child will not perceive given situations as they actually exist, but according to a personal scheme of apperception -- that is to say, he will perceive situations under the prejudice of his own interests.

'' -- \cite[pp. 31--30]{Adler2013}

%------------------------------------------------------------------------------%

\section{The Inferiority Complex}

%------------------------------------------------------------------------------%

\section{The Superiority Complex}

%------------------------------------------------------------------------------%

\section{The Style of Life}

%------------------------------------------------------------------------------%

\section{Old Remembrances}

%------------------------------------------------------------------------------%

\section{Attitudes \& Movements}

%------------------------------------------------------------------------------%

\section{Dreams \& Their Interpretation}

%------------------------------------------------------------------------------%

\section{Problem Children \& Their Education}

%------------------------------------------------------------------------------%

\section{Social Problems \& Social Adjustment}

%------------------------------------------------------------------------------%

\section{Social Feeling, Common Sense \& the Inferiority Complex}

%------------------------------------------------------------------------------%

\section{Love \& Marriage}

%------------------------------------------------------------------------------%

\section{Sexuality \& Sex Problems}

%------------------------------------------------------------------------------%

\section{Conclusion}

%------------------------------------------------------------------------------%

\printbibliography[heading=bibintoc]
	
\end{document}