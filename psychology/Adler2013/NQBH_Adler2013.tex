\documentclass{article}
\usepackage[backend=biber,natbib=true,style=authoryear]{biblatex}
\addbibresource{/home/nqbh/reference/bib.bib}
\usepackage{tocloft}
\renewcommand{\cftsecleader}{\cftdotfill{\cftdotsep}}
\usepackage[colorlinks=true,linkcolor=blue,urlcolor=red,citecolor=magenta]{hyperref}
\usepackage{algorithm,algpseudocode,amsmath,amssymb,amsthm,float,graphicx,mathtools}
\allowdisplaybreaks
\numberwithin{equation}{section}
\newtheorem{assumption}{Assumption}[section]
\newtheorem{conjecture}{Conjecture}[section]
\newtheorem{corollary}{Corollary}[section]
\newtheorem{definition}{Definition}[section]
\newtheorem{example}{Example}[section]
\newtheorem{lemma}{Lemma}[section]
\newtheorem{notation}{Notation}[section]
\newtheorem{principle}{Principle}[section]
\newtheorem{problem}{Problem}[section]
\newtheorem{proposition}{Proposition}[section]
\newtheorem{question}{Question}[section]
\newtheorem{remark}{Remark}[section]
\newtheorem{theorem}{Theorem}[section]
\usepackage[left=1cm,right=1cm,top=5mm,bottom=5mm,footskip=4mm]{geometry}
\def\labelitemii{$\circ$}

\title{The Science of Living}
\author{Alfred Adler}
\date{\today}

\begin{document}
\maketitle
\tableofcontents
\vspace{5mm}
\textbf{The Science of Living.} ``Originally published in 1930 \textit{The Science of Living} looks at Individual Psychology as a science. Adler discusses the various elements of Individual Psychology \& its application to everyday life: including the inferiority complex, the superiority complex \& other social aspects, such as, love \& marriage, sex \& sexuality, children \& their education. This is an important book in the history of psychoanalysis \& Alderian therapy.''

\section*{A Note on the Author \& His Work}
``\textsc{Dr. Alfred Adler}'s work in psychology, while it is scientific \& general in method, is essentially the study of the separate personalities we are, \& is therefore called Individual Psychology. Concrete, particular, unique human beings are the subjects of this psychology, \& it can only be truly learned from the men, women \& children we meet.

The supreme importance of this contribution to modern psychology is due to the manner in which it reveals how all the activities of the soul are drawn together into the service of the individual, how all his faculties \& strivings are related to 1 end. We are enabled by this to enter into the ideals, the difficulties, the efforts \& discouragements of our fellow-men, in such a way that we may obtain a whole \& living picture of each as a personality. In this co-ordinating idea, something like finality is achieved, though we must understand it as finality of foundation. There has never before been a method so rigorous \& yet adaptable for following the fluctuations of that most fluid, variable \& elusive of all realities, the individual human soul.

Since Adler regards not only science but even intelligence itself as the result of the communal efforts of humanity, we shall find his consciousness of his own unique contribution more than usually tempered by recognition of his collaborators, both past \& contemporary. It will therefore be useful to consider Adler's relation to the movement called Psycho-analysis, \& 1st of all to recall, however briefly, the philosophic impulses which inspired the psycho-analytic movement as a whole.

The conception of the Unconscious as vital memory -- biological memory -- is a common to modern psychology as a whole. But Freud, from the 1st a specialist in hysteria, took the memories of success or failure in the sexual life, as of the 1st -- \& almost the only -- importance. Jung, a psychiatrist of genius, has tried to widen this distressingly narrow view, by seeking to reveal the super-individual or racial memories which, he believes, have as much power as the sexual \& a higher kind of value for life.

It was left to Alfred Adler, a physician of wide \& general experience, to unite the conception of the Unconscious more firmly with biological reality. A man of the original school of psycho-analysts, he had done much work by that method of analyzing memories out of their coagulated emotional state into clearness \& objectivity. But he showed that the whole scheme of memory is different in every individual. Individuals do not form their unconscious memories all around the same central motive -- not all around sexuality, e.g., . In every individual we find an individual way of selecting its experiences from all possible experience. What is the principle of that selectivity? Adler has answered that it is, fundamentally, the organic consciousness of a \textit{need}, of some specific inferiority which has to be compensated. It is as though every soul had consciousness of its whole physical reality, \& were concentrated, with sleepless insistence, upon achieving compensation for the defects in it.

Thus the whole life of the small man, e.g., , would be interpretable as a struggle to achieve immediate greatness in some way, \& that of a deaf man to obtain a compensation for not hearing. It is not so simple as that, of course, for a system of defects may give rise to a constellation of guiding ideas, \& also in human life we have to deal with imaginary inferiorities \& fantastic strivings, but even here the principle is the same.

The sexual life, far from controlling all activities, fits perfectly into the frame of those more important strivings, for it is pre-eminently under the control of emotion, \& emotion is moulded by the entire vital history. Thus a Freudian analysis gives a true account of the sexual \textit{consequences} of a given life-line, but it is a true \textit{diagnosis} only in that sense.

Psychology becomes now for the 1st time rooted in biology. The tendencies of the soul, \& the mind's development, are seen to be controlled from the 1st by the effort to compensate for organic defects or for positions of inferiority. Everything that is exceptional or individual in the disposition of an organic being originates in this way. The principle is common to man \& animal, probably even to the vegetable kingdom also; \& the special endowments of species are to be taken as arising from experience of defects \& inferiorities in relation to their environment, which has been successfully compensated by activity, growth \& structure.

There is nothing new in the idea of compensation as a biological principle, for it has been long known that the body will over-develop certain parts in compensation for the injury of others. If 1 kidney ceases to function, e.g., the other develops abnormally until it does the work of both; if the heart springs a leak in a valve, the whole organ grows larger to allow for its loss of efficiency, \& when nervous tissue is destroyed, adjacent tissue of another kind endeavors to take on the nerve-function. The compensatory developments of the whole organism to meet the exigencies of any special work or exertion are too numerous \& well known to need illustration. But it is Dr. Adler who has 1st transferred this principle bodily to psychology as a fundamental idea, \& demonstrated the part it plays in the soul \& intellect.

Adler recommends the study of Individual Psychology not only to doctors, but generally to laymen \& especially to teachers. Culture in psychology has become a general necessity, \& must be firmly advocated in the teeth of popular opposition to it, which is founded upon the notion that modern psychology requires an unhealthy concentration of the mind upon cases of disease \& misery. It is true that the literature of psychoanalysis has revealed the most central \& the most universal evils in modern society. But it is not now a question of contemplating our errors, it is necessary that we should learn by them. We have been trying to live as though the soul of man were not a reality, as though we could build up a civilized life in defiance of psychic truths. What Adler proposes is not the universal study of psycho-pathology, but the practical reform of society \& culture in accordance with a positive \& scientific psychology to which he has contributed the 1st principles. But this is impossible if we are too much afraid of the truth. The clearer consciousness of right aims in life, which is indispensable to us, cannot be gained without a deeper understanding also of the mistakes in which we are involved. We may not desire to know ugly facts, but the more truly we are aware of life, the more clearly we perceive the real errors which frustrate it, much as the concentration of a light gives definition to the shadows.

A positive psychology, useful for human life, cannot be derived from the psychic phenomena alone, still less from pathological manifestations. It requires also a regulative principle, \& Adler has not shrunk from this necessity, by recognizing, as if it were of absolute metaphysical validity, the logic of our communal life in the world.

Recognizing this principle, we must proceed to estimate the psychology of the individual in relation too it. The way in which an individual's inner life is related to the communal being is distinguishable in 3 ``life-attitudes,'' as they are called -- his general reactions to society, to work \& to love.

By their feeling towards society as a whole -- to any other \& to tell others -- man \& women may know how much social courage they possess. The feeling of inferiority is always manifested in a sense of fear or uncertainty in the presence of society, whether its outward expression is 1 of timidity or defiance, reserve or over-anxiety. All feelings of innate suspicion or hostility, of an undefined caution \& desire for some concealment, when such feelings affect the individual in social relations generally, evince the same tendency to withdraw from reality, which inhibits self-affirmation. The ideal, or rather normal, attitude to society is an unstrained \& unconsidered assumption of human equality unchanged by any inequalities of position. Social courage depends upon this feeling of secure membership of the human family, a feeling which depends upon the harmony of one's own life. By the tone of his feeling towards his neighbors, his township \& nation \& to other nationalities, \& even by his reactions when he reads of all these things in his newspaper, a man may infer how securely his own soul is grounded in itself.

The attitude towards work is closely dependent upon this self-security in society. In the occupation by which a man earns his share in social goods \& privileges, he has to face the logic of social needs. If he has too great a sense of weakness or division from society, it will make him unable to believe that his worth will ever be recognized, \& he will not even work for recognition: instead, he will play for safety, \& work for money or advantage only, suppressing his own valuation of what is the truest service he can render. He will always be afraid to supply or demand the best, for fear it may not pay. Or he may be always seeking for some quiet backwater of the economic life, where he can do something just as he likes himself, without proper consideration of either usefulness or profit. In both cases it is not only society that suffers by not getting the best service: the individual who has not attained his proper social significance is also deeply dissatisfied. The modern world is full of men, both successful \& unsuccessful in a worldly sense, who are in open conflict with their occupation. They do not believe in it, \& they blame social \& economic conditions with some real justice; but it is also a fact that they have often had too little courage to fight for the best value in their economic function. They were afraid to claim the right to give what they genuinely believed in, or else they felt disdainful of the service society really needed of them. Hence they pursued their gain in an individualistic or even furtive spirit. We must, of course, recognize that so much is wrong in the organization of society, that, besides the possibility of making mistakes of judgment, the individual who is determined to render real social service has often to face heavy opposition. But it is precisely that sense of struggle to give his best which the individual needs no less than society benefits by it. One cannot love a vocation which does not afford some experience of victory over difficulties, \& not merely of compromise with them.

It is the 3rd of these life-attitudes -- the attitude to love -- which determines the course of the erotic life. Where the 2 previous life-attitudes, to society \& to work, have been rightly adjusted, this last comes right by itself. Where it is distorted \& wrong it cannot be improved by itself apart from the others. Although we can think how to improve the social relations \& the occupations, a concentration of thought upon the individual sex-problem is almost sure to make it worse. For this is far more the sphere of results than of course. A soul that is defeated in ordinary social life, or thwarted in its occupation, acts in the sex-life as though it were trying to obtain compensation for the kinds of expression of which it fails in their proper spheres. This is actually the best way in which we can understand all sexual vagaries, whether they isolate the individual, degrade the sexual partner or in any way distort the instinct. The friendships of an individual also are integral with the love-life as a whole; not, as the 1st psycho-analysts imagined, because friendship is a sublimation of sexual attraction, but the other way about. Sexual compulsion -- sex as an insubordinate psychic factor -- is an abnormal substitute for the vitalizing intimacy of useful friendships, \& homosexuality is always the consequence of incapability for love.

The meaning \& value which we give to sensations are also united closely with the erotic life, as many good poets have testified. The quality of our feeling for Nature, our response to the beauty of sea \& land, \& to significances of form \& sound \& color, as well as our confidence in scenes of storm \& gloom, are all involved with our integrity as lovers. The aesthetic life, with all it means to art \& culture, is thus ultimately derived, through individuals, from social courage \& intelligent usefulness.

We ought not to regard the communal feeling as something to be created with difficulty. It is as natural \& inherent as egoism itself, \& indeed as a principle of life it has priority. We have not to create, but only to liberate, it where it is repressed. It is the saving principle of life as we experience it. If anyone thinks that the services of 'busmen, railwaymen \& milkmen would be rendered as well as they are without the presence of very much instinctive communal feeling he must be suspected of a highly neurotic scheme of apperception. What inhibits it is, to speak bluntly, the enormous vanity of the human soul, which is, moreover, so subtle that no professional psychologist before Adler had been able to demonstrate it, though a few artists had divined its omnipresence. All unsuspected as it often is, the ambition of many a minor journalist or shop-assistant, to say nothing of the great ones of the world, would be enough to bring about the fall of an archangel. Every feeling of inferiority that has embittered his contact with life has fed the imagination of greatness with another god-like assumption until, in many cases, the fantasy has become so inflated as to demand not even supremacy in this world for its appeasement, but the creation of a new world altogether, \& to be the god of it. This revelation of the depth of human nature is verified, not so strikingly from the study of cases of practical ambition, however Napoleonic, as from those of passive resistance, procrastination, \& malingering, for it is these which show most clearly that an individual who feels painfully unable to dominate the real world will refuse to co-operate with it, at whatever disadvantage to himself, partly in order to tyrannize over a narrower sphere, \& partly even from an irrational feeling that the real world, without his divine assistance, will some day crumble \& shrink to his own diminished measure.\footnote{In case this should seem an exaggeration, we may recall the fact that nearly all the narrowest kind of sects, religious or secular, have a belief in world-catastrophe: the world from which they have withdrawn, \& which they despair of converting, is to be brought to destruction, \& only a remnant will survive, who will be of their own persuasion.}

The question is thus raised, how should we act, knowing this tendency to inordinate vanity in the human soul, \& that we dare not merely add to that vanity by assuming ourselves to be miraculous exceptions? Adler's reply is that we should preserve a certain attitude to all our experience, which he calls the attitude of ``half-\&-half.'' Our conception of normal behavior should be to allow the world or society, or the person with whom we are confronted, to be somehow in the right equally with ourselves. We should not depreciate either ourselves or our environment; but, assuming that each is 1-half in the right, affirm the reality of ourselves \& others equally. This applies not only to contacts with other souls, but to our mental reactions towards rainy weather, holidays or comforts that we cannot afford, \& even to the omnibus we have just missed.

Rightly understood, this is not an ideal of difficult \& distasteful humility. It is in reality a tremendous assumption of worth, to claim exactly equal reality \& omnipotence with the whole of the rest of creation, in whatever particular manifestation we may be meeting with it. To claim less than this is a false humility, for what results from any contact we make does in fact depend for half its reality upon the way in which we make it. The individual should affirm his part in everything which occurs to him, as his own half of it.

This is often a particularly difficult counsel to keep in relation to the occupation. In their business, people face more naked realities than are usually allowed to appear in social life; \& it is often almost impossible to allow equal validity to one's own aims \& to the conditions of a disorganized world. To do so, means the admission that conditions, just such as they are, are one's real problem -- \&, indeed, one's proper sphere of action. The division of labor, logical \& useful as it is in itself, has given opportunity for human megalomania to create entirely false inequalities, distinctions \& injustices, so that we live in an economic disorder which will hardly hold together. To such crazy conditions, the best of men often find it difficult to oppose themselves with perseverance, equally grating its reality \& working for its reform. They are tempted to acquiesce in disorder by some inner subterfuge, or to devote themselves to superficial remedies which evade the real problem; \& sometimes they treat their work-life as an unavoidable contamination by things inherently squalid, quite unaware that such an attitude makes them conceited, haughty \&, in a profound sense, unscrupulous. It occurs to very few that the right way would be to make alliance on human grounds with others in the same predicament \& profession, to assert its proper dignity as a social service \& improve it; but this is the only way in which the individual can really be reconciled with his economic function. Many of those who complain most about the conditions prevailing in their work are doing nothing whatever to reorganize it as a function of human life, \& never think of attacking the anarchic individualism which is its ruin. We derive it from Individual Psychology, as a categorical imperative, that every man's duty is to work to make his profession, whatever it may be, into a brotherhood, a friendship, a social unity with a powerful morale of co-operation, \& that if a man does not want to do this his own psychological state is precarious. It is true that now, in many professions, the task that this presents is terribly difficult. It is all the more essential that the effort should be made towards integration. For a man's work will never liberate the forces of his psyche unless he is striving, in a large sense, to make it the expression of his whole being, \& his idea of his profession must be not only an executive in which he has independence of action, but also a legislative in which he has some authority of direction. In a man's business life that half-\&-half valuation leads equally to recognition of reality \& to struggle with it by the only realistic method, which is necessarily co-operative.

The pedagogic principles of Individual Psychology, infallible as far as they go, are useless without this practical work of social organization. What has been written above of an individual's duty in his occupation applis in a large sense to his entire social function. A person's function includes active membership of his nation \& of humanity, to say nothing of his family. There is a certain parliament which rises for no vacation, \& to whose decisions all elected assemblies must in the end defer. It meets in schools, markets, \& everywhere on sea \& land, for it is the Parliament of Man, in which every word or look exchanged, whether of courtesy or recrimination, of wisdom or folly, has its measure of importance in the affairs of the race. It is everyone's interest to make this wide assembly more united \& its discussion more intelligible, for none of us has any real human existence except by reflection from it. When its conclaves are peaceful, all our lives are heightened in tone, health \& wealth accrues \& arts \& education flourish; when its conversation is reversed \& suspicious, work fails, men starve \& children languish. In the heat of its dissensions we perish by the million. All its decrees, by which we live or die, \& grow or decay, are rooted in our individual attitudes towards man, woman \& child in every relation of life.

When we face, objectively, this fact of the relation of all souls \& their mutual responsibility, what are we to think of the inner confusion of the neurotic? Is it not simply a narrowing of the sphere of interest, an over-concentration upon certain personal or subjective gains? The neurotic soul is the result of treating the rest of humanity as though its life \& aims were altogether of less importance than one's own, \& thus losing interest in any larger life. Paradoxically, it often happens that a neurotic has very large schemes of saving himself \& others. He is intelligent enough to try to compensate his real sense of isolation \& impotence in the human assembly, by a fantasy of exaggerated importance \& beneficent activity. He may want to reform education, to abolish war, to establish universal brotherhood or create a new culture, \& even plans or joins societies with these aims. He is defeated in such aims, of course, by the unreality of his contact with others \& with life as a whole. It is as though he had taken a standpoint outside of life altogether \& were trying to direct it by some unexplained magic.

Modern city life especially, with its intellectualism, gives unlimited scope for the neurotic thus to compensate his real unsociability with imaginary messianism, \& the result is the disintegration of a people full of saviors who are not on speaking terms.

What is needed, of course, is something very different. It is not that the individual should renounce messianism; for it is a fact that a share of responsibility for the whole future of the race is his alone. It is only necessary that he should take a reasonable view of his power to save society, correctly viewed from his own standpoint: he must become able to regard his immediate personal relations \& his occupations as if \textit{they} were of world-importance, for in fact they are so, being the only world-meaning an individual has. When they are chaotic or wrong, it is because we do not, in day-to-day experience, treat them as things of universal meaning. We sometimes treat them as important, no doubt, but generally in a personal sense only.

This tendency of the modern soul, to narrow the sphere of interest, both practically \& ideally, is most difficult to subdue, because it is reinforced by the scheme of apperception. For that reason an individual alone cannot do it, excepting only in rare cases. He needs conference with other minds, \& an entirely new kind of conference. A resolution to treat one's immediate surroundings \& daily activities as if they were the supreme significance of life brings an individual immediately into conflict with internal resistances of his own, \& often with external difficulties also, which he cannot at once understand \& which no others could rightly estimate unless they were making the same experiment. Hence, the practice of Individual Psychology demands that its students should submit themselves to mutual scrutiny, each one to be estimated by the others as a whole personality. This practice, striking at the root of the false individualism which is the basis of all neurosis, is naturally very difficult to initiate. Upon its success, however, depends the whole future of psycho-analysis as an influence in life at large, outside of clinics \& consulting rooms.

In Vienna the work of such groups has already made itself felt in education. The co-operation it has established between teachers \& medical practitioners has revolutionized the work of certain schools, \& established an equality between teachers \& pupils \& between pupils themselves, which has cured many children of criminal tendencies, dullness \& laziness. Abolition of competition \& the cultivation of encouragement have been found to liberate the energy of both pupils \& teachers. These changes are already affecting the surrounding family life, which comes into question immediately the child is psychologically considered. Education, though naturally the 1st, is not the only sphere of life which out to be invaded by the activity of such groups. Business \& political circles, which experience the deadlock of modern life most acutely, need to be vitalized with knowledge of human nature, which they have forgotten how to recognize.

It is for this work of releasing a new energy for daily life \& its reformation, that Alfred Adler has founded the International Society for Individual Psychology. The culture of human behavior which this work has begun already to propagate might well be mistaken for an almost platitudinous ethics, but for 2 things -- its practical results, \& the background of scientific method out of which it is appearing. In his realistic grasp of the social nature of the individual's problem \& his inexorable demonstration of the unity of health \& harmonious behavior, Adler resembles no one so much as the great Chinese thinkers. If the occidental world is not too far gone to make use of his service, he may well come to be known as the Confucius of the West. -- \textsc{Phillipe Mairet}'' -- \cite[pp. 9--30]{Adler2013}

%------------------------------------------------------------------------------%

\section{The Science of Living}
``Only a science which is directly related to life, said the great philosopher William James, is really a science. It might also be said that in a science which is directly related to life theory \& practice become almost inseparable. The science of life, precisely because it models itself directly on the movement of life, becomes a science of living. These considerations apply with special force to the science of Individual Psychology. Individual Psychology tries to see individual lives as a whole \& regards each single reaction, each movement \& impulse as an articulated part of an individual attitude towards life. Such a science is of necessity oriented in a practical sense, for which the aid of knowledge we can correct \& alter our attitudes. Individual Psychology is thus \textit{prophetic} in a double sense: not only does it predict what will happen, but, like the prophet Jonah, it predicts what \textit{will} happen in order that it should \textit{not} happen.

The science of Individual Psychology developed out of the effort to understand that mysterious creative power of life -- that power which expresses itself in the desire to develop, to strive \& to achieve -- \& even to compensate for defeats in 1 direction by striving for success in another. This power is \textit{teleological} -- it expresses itself in the striving after a goal, \& in this striving every bodily \& psychic movement is made to co-operate. It is thus absurd to study bodily movements \& mental conditions abstractly without relation to an individual whole. It is absurd, e.g., that in criminal psychology we should pay so much more attention to the crime than to the criminal. It is the criminal, not the crime that counts, \& no matter how much we contemplate the criminal act we shall never understand its criminality unless we see it as an episode in the life of a particular individual. The same outward act may be criminal in 1 case \& not criminal in another. The important thing is to understand the individual context -- the goal of an individual's life which marks the line of direction for all his acts \& movements. This goal enables us to understand the hidden meaning behind the various separate acts -- we see them as parts of a whole. Vice versa when we study the parts -- provided we study them as parts of a whole -- we get a better sense of the whole.

In the author's own case the interest in psychology developed out of the practice of medicine. The practice of medicine provided the teleological or purposive viewpoint which is necessary for the understanding of psychological facts. In medicine we see all organs striving to develop towards definite goals. They have definite forms which they achieve upon maturity. Moreover, in cases where there are organic defects we always find nature making special efforts to overcome the deficiency, or else to compensate for it by developing another organ to take over the functions of the defective one. Life always seeks to continue, \& the life force never yields to external obstacles without a struggle.

Now the movement of the psyche is analogous to the movement of organic life. In each mind there is the conception of a goal or ideal to get beyond the present state, \& to overcome the present deficiencies \& difficulties by postulating a concrete aim for the future. By means of this concrete aim or goal the individual can think \& feel himself superior to the difficulties of the present because he has in mind his success of the future. Without the sense of a goal individual activity would cease to have anything meaning.

All evidence points to the fact that the fixing of this goal -- giving it a concrete form -- must take place early in life, during the formative period of childhood. A kind of prototype or model of a matured personality begins to develop at this time. We can imagine how the process takes place. A child, being weak, feels inferior \& finds itself in a situation which it cannot bear. Hence it strives to develop, \& it strives to develop along a line of direction fixed by the goal which it chooses for itself. The material used for development at this stage is less important than the goal which decides the line of direction. How this goal is fixed it is difficult to say, but it is obvious that such a goal exists \& that it dominates the child's every movement. Little is indeed understood about powers, impulses, reasons, abilities or disabilities at this early period. As yet there is really no key, for the direction is definitely established only after the child has fixed its goal. Only when we see the direction in which a life is tending can we guess what steps will be taken in the future.

When the prototype -- that early personality which embodies the goal -- is formed, the line of direction is established \& the individual becomes definitely oriented. It is this fact which enables us to predict what will happen later in life. The individual's apperceptions are from then on bound to fall into a groove established by the line of direction. The child will not perceive given situations as they actually exist, but according to a personal scheme of apperception -- that is to say, he will perceive situations under the prejudice of his own interests.

An interesting fact that has been discovered in this connection is that children with organic defects connect all their experiences with the function of the defective organ. E.g., a child having stomach trouble shows an abnormal interest in eating, while one with defective eyesight is more preoccupied with things visible. This preoccupation is in keeping with the private scheme of apperception which we have said characterizes all persons. It might be suggested, therefore, that in order to find out where a child's interest lies we need only to ascertain which organ is defective. But things do not work out quite so simply. The child does not experience the fact of organ inferiority in the way that an external observer sees it, but as modified by its own scheme of apperception. Hence while the fact of organ inferiority counts as an element in the child's scheme of appreciation, the external observation of the inferiority does not necessarily give the cue to the scheme of apperception.

The child is steeped in a scheme of relativity, \& in this he is indeed like the rest of us -- none of us is blessed with the knowledge of the absolute truth. Even our science is not blessed with absolute truth. It is based on common sense, which is to say that it is ever changing \& that it is content gradually to replace big mistakes by smaller ones. We all make mistakes, but the important thing is that we can correct them. Such correction is easier at the time of the formation of the prototype. \& when we do not correct them at that time, we may correct the mistakes later on by recalling the whole situation of that period. Thus if we are confronted with the task of treating a neurotic patient, our problem is to discover, not the ordinary mistakes he makes in later life, but the very fundamental mistakes made early in his life in the course of the constitution of his prototype. If we discover these mistakes, it is possible to correct them by appropriate treatment.

In the light of Individual Psychology the problem of inheritance thus decreases in importance. It is not what one has inherited that is important, but what one does with his inheritance in the early years -- that is to say, the prototype that is built up in the childhood environment. Heredity is of course responsible for inherited organic defects, but our problem there is simply to relieve the particular difficulty \& place the child in a favorable situation. As a matter of fact we have even a great advantage here, inasmuch as when we see the defect we know how to act accordingly. Oftentimes a healthy child without any inherited defects may fare worse through malnutrition or through any of the many errors in upbringing.

In the case of children born with imperfect organs it is the psychological situation which is all-important. Because these children are placed in a more difficult situation they show marked indications of an exaggerated feeling of inferiority. At the time the prototype is being formed they are already more interested in themselves than in others, \& they tend to continue that way later on in life. Organic inferiority is not the only cause of mistakes in the prototype: other situations may also cause the same mistakes -- the situations of pampered \& hated children, e.g. We shall have occasion later on to describe these situations more in detail \& to present actual case histories illustrating the 3 situations which are particularly unfavorable, that of children with imperfect organs, that of petted children, \& that of hated children. For the present it is sufficient to note that these children grow up handicapped \& that they constantly fear attacks inasmuch as they have grown up in an environment in which they never learned independence.

It is necessary to understand the social interest from the very upset since it is the most important part of our education, of our treatment \& of our cure. Only such persons as are courageous, self-confident \& at home in the world can benefit both by the difficulties \& by the advantages of life. They are never afraid. They know that there are difficulties, but they also know that they can overcome them. They are prepared for all the problems of life, which are invariably social problems. From a human standpoint it is necessary to be prepared for social behavior. The 3 types of children we have mentioned develop a prototype with a lesser degree of social interest. They have not the mental attitude which is conducive to the accomplishment of what is necessary in life or to the solution of its difficulties. Feeling defeated, the prototype has a mistaken attitude towards the problems of life \& tends to develop the personality on the useless side of life. On the other hand our task in treating such patients is to develop behavior on the useful side \& to establish in general a useful attitude towards life \& society.

Lack of social interest is equivalent to being oriented towards the useless side of life. The individuals who lack social interest are those who make up the groups of problem children, criminals, insane persons, \& drunkards. Our problem in their case is to find means to influence them to go back to the useful side of life \& to make them interested in others. In this way it may be said that our so-called Individual Psychology is actually a social psychology.

After the social interest, our next task is to find out the difficulties that confront the individual in his development. This task is somewhat more confusing at 1st glance, but it is in reality not very complicated. We know that every petted child becomes a hated child. Our civilization is such that neither society nor the family wishes to continue the pampering process indefinitely. A pampered child is very soon confronted with life's problems. In school he finds himself in a new social institution, with a new social problem. He does not want to write or play with his fellows, for his experience has not prepared him for the communal life of the school. In fact his experiences as lived through at the prototype stage make him afraid of such situations \& make him look for more pampering. Now the characteristics of such an individual are not inherited -- far from it -- for we can deduce them from a knowledge of the nature of his prototype \& his goal. Because he has the particular characteristics conducive to his moving in the direction of his goal, it is not possible for him to have characteristics that would tend in any other direction.

The next step in the science of living lies in the study of the feelings. Not only does the axis line, the line of direction posited by the goal, affect individual characteristics, physical movements, expressions \& general outward symptoms, but it dominates the life of the feelings as well. It is a remarkable thing that individuals always try to justify their attitudes by feelings. Thus if a man wants to do good work, we will find this idea magnified \& dominating his whole emotional life. We can conclude that the feelings always agree with the individual's viewpoint of his task: they strengthen the individual in his bent for activity. We always do that which we would do even without the feelings, \& the feelings are simply an accompaniment to our acts.

We can see this fact quite clearly in dreams, the discovery of whose purpose was perhaps 1 of the latest achievements of Individual Psychology. Every dream has of course a purpose, although this was never clearly understood until now. The purpose of a dream -- expressed in general \& not specific terms -- is to create a certain movement of feeling or emotion, which movement of emotion in turn furthers the movement of the dream. It is an interesting commentary on the old idea that a dream is always a deception. We dream in the way that we would like to behave. Dreams are an emotional rehearsal of plans \& attitudes for waking behavior -- a rehearsal, however, in which the actual play may never come off. In this sense dreams are deceptive -- the emotional imagination gives us the thrill of action without the action.

This characteristic of dreams is also found in our waking life. We always have a strong inclination to deceive ourselves emotionally -- we always want to persuade ourselves to go the way of our prototypes as they were formed in the 4th or 5th year of life.

The analysis of the prototype is next in order in our scheme of science. As we have said, at 4 or 5 the prototype is already built up, \& so we have to look for impressions made n the child before or at that time. These impressions can be quite varied, far more varied than we imagine from a normal adult's point of view. 1 of the most common influences on a child's mind is the feeling of suppression brought about by a father's or mother's excessive punishment or abuse. This influence makes the child strive for release, \& sometimes this is expressed in an attitude of psychological exclusion. Thus we find that some girls having high-tempered fathers have prototypes that exclude men because they are high-tempered. Or boys suppressed by severe mothers may exclude women. This excluding attitude may of course be variously expressed: e.g., the child may become bashful, or on the other hand, he may become perverted sexually (which is simply another way of excluding women). Such perversions are not inherited, but arise from the environment surrounding the child in these years.

The early mistakes of the child are costly. \& despite this fact the child receives little guidance. Parents do not know or will not confess to the child the results of their experiences, \& the child must thus follow his own line.

Curiously enough we will find that no 2 children, even those born in the same family, grow up in the same situation. Even within the same family the atmosphere that surrounds each individual child is quite particular. Thus the 1st child has notoriously a different set of a circumstances from the other children. The 1st child is at 1st alone \& is thus the center of attention. Once the 2nd child is born, he finds himself dethroned \& he does not like the change of situation. In fact it is quite a tragedy in his life that he has been in power \& is so no longer. This sense of tragedy goes into the formation of his prototype \& will crop out in his adult characteristics. As a matter of fact case histories show that such children always suffer downfall.

Another intra-family difference of environment is to be found in the different treatments accorded to boys \& to girls. The usual case is for boys to be overvalued \& the girls to be treated as if they could not accomplish anything. These girls will grow up always hesitating \& in doubt. Throughout life they will hesitate too much, always remaining under the impression that only men are really able to accomplish anything.

The position of the 2nd child is also characteristic \& individual. He is in an entirely different position from that of the 1st child, inasmuch as for him there is always a pace-maker, moving along parallel with him. Usually the 2nd child overcomes his pace-maker, \& if we look for the cause we shall find simply that the older child was annoyed by having such a competitor \& that the annoyance in the end affected his position in the family. The older child becomes frightened by the competition \& does not do so well. He sinks more \& more in the estimation of his parents, who begin to appreciate the 2nd child. On the other hand the 2nd child is always confronted by the pace-maker, \& he is thus always in a race. All his characteristics will reflect this peculiar position in the family constellation. He shows rebellion \& does not recognize power or authority.

History \& legend recount numerous incidents of powerful youngest children. Joseph is a case in point: he wanted to overcome all the others. The fact that a younger brother was born into the family unknown to him years after he left home obviously does not alter the situation. His position was that of the youngest. We find also the same description in all the fairy tales, in which the youngest child plays the leading role. We can see how these characteristics actually originate in early childhood \& cannot be changed until the insight of the individual has increased. In order to readjust a child you must make him understand what happened in his 1st childhood. He must be made to understand that his prototype is erroneously influencing all the situations in his life.

A valuable tool for understanding the prototype \& hence the nature of the individual is the study of old remembrances. All our knowledge \& observation force us to the conclusion that our remembrances belong to the prototype. An illustration will make our point clear. Consider a child of the 1st type, one with imperfect organs -- with a weak stomach, let us say. If he remembers having seen something or heard something it will probably in some way concern eatables. Or take a child that is left-handed: his left-handedness will likewise affect his viewpoint. A person may tell you about his mother who pampered him, or about the birth of a younger child. He may tell you how he was beaten, if he had a high-tempered father, or how he was attacked if he was a hated child at school. All such indications are very valuable provided we learn the art of reading their significance.

The art of understanding old remembrances involves a very high power of sympathy, a power to identify oneself with the child in his childhood situation. It is only by such power of sympathy that we are able to understand the intimate significance in a child's life of the advent of a younger child in the family, or the impression made on a child's mind by the abuse of a high-tempered father.

\& while we are on the subject it cannot be overemphasized that nothing is gained by punishing, admonishing \& preaching. Nothing is accomplished when neither the child nor the adult knows on which point the change has to be made. When the child does not understand, he becomes slyer \& more cowardly. His prototype, however, cannot be changed by such punishment \& preaching. It cannot be changed by mere experience of life, for the experience of life is already in accordance with the individual's personal scheme of apperception. It is only when we get at the basic personality that we accomplish any changes.

If we observe a family with badly developed children, we shall see that though they all seem to be intelligent (in the sense that if you ask a question they give the right answer), yet when we look for symptoms \& expressions, they have a great feeling of inferiority. Intelligence of course is not necessarily common sense. The children have an entirely personal -- what we might term, a private -- mental attitude of the sort that one finds among neurotic persons. In a compulsion neurosis, e.g., the patient realizes the futility of always counting windows but cannot stop. One interested in useful things would never act this way. Private understanding \& language are also characteristic of the insane. The insane never speak in the language of common sense, which represents the height of social interest.

If we contrast the judgment of common sense with private judgment, we shall find that the judgment of common sense is usually nearly right. By common sense we distinguish between good \& bad, \& while in a complicated situation we usually make mistakes, the mistakes tend to correct themselves through the very movement of common sense. But those who are always looking out for their own private interests cannot distinguish between right \& wrong as readily as others. In fact they rather betray their inability, inasmuch as all their movements are transparent to the observer.

Consider e.g. the commission of crimes. If we inquire about the intelligence, the understanding \& the motive of a criminal, we shall find that the criminal always looks upon his crimes as both clever \& heroic. He believes that he has achieved a goal of superiority -- namely, that he has become more clever than the police \& is able to overcome others. He is thus a hero in his own mind, \& does not see that his actions indicate something quite different, something very far from heroic. His lack of social interest, which puts his activity on the useless side of life, is connected with a lack of courage, with cowardice, but he does not know this. Those who turn to the useless side of things are often afraid of darkness \& isolation; they wish to be with others. This is cowardice \& should be labeled as such. Indeed the best way to stop crime would be to convince everybody that crime is nothing but an expression of cowardice.

It is well known that some criminals when they approach the age of 30 will take a job, marry \& become good citizens in later life. What happens? Consider a burglar. How can a 30-year old burglar compete with a 20-year old burglar? The latter is cleverer \& more powerful. Moreover, at the age of 30 the criminal is forced to live differently from the way he lived before. As a result the profession of crime no longer pays the criminal \& he finds it convenient to retire.

Another fact to be borne in mind in connection with criminals is that if we increase the punishments, so far from frightening the individual criminal, we merely help to increase his belief that he is a hero. We must not forget that the criminal lives in a self-centered world, a world in which one will never find true courage, self-confidence, communal sense, or understanding of common values. It is not possible for such persons to join a society. Neurotics seldom start a club, \& it is an impossible feat for persons suffering from agoraphobia or for insane persons. Problem children or persons who commit suicide never make friends -- a fact for which the reason is never given. There is a reason, however: they never make friends because the early life took a self-centered direction. Their prototypes were oriented towards false goals \& followed lines of direction on the useless side of life.

Let us now consider the program which Individual Psychology offers for the education \& training of neurotic persons -- neurotic children, criminals, \& persons who are drunkards \& wish to escape by such means from the useful side of life.

In order to understand easily \& quickly what is wrong, we begin by asking at what time the trouble originated. Usually the blame is laid on some new situation. But this is a mistake, for before this actual occurrence, our patient -- so we shall find upon investigation -- had not been well prepared for the situation. So long as he was in a favorable situation the mistakes of his prototype were not apparent, for each new situation is in the nature of an experiment to which he reacts according to the scheme of apperception created by his prototype. His responses are not mere reactions, they are creative \& consistent with his goal, which is dominant throughout his life. Experience taught us early in our studies of Individual Psychology that we might exclude the importance of inheritance, as well as the importance of an isolated part. We see that the prototype answers experiences in accordance with its own scheme of apperception. \& it is this scheme of apperception that we must work upon in order to produce any results.

This sums up the approach of Individual Psychology which has been developed in the last 25 years. As one may see, Individual Psychology has traveled a long way in a new direction. There are many psychologies \& psychiatries in existence. 1 psychologist takes 1 direction, another another direction, \& no one believes that the others are right. Perhaps the reader, too, should not rely on belief \& faith. Let him compare. He will see that we cannot agree with what is called ``drive'' psychology (McDougall represents this tendency best in America), because in their ``drives'' too big a place is set aside for inherited tendencies. Similarly we cannot agree with the ``conditioning'' \& ``reactions'' of Behaviorism. It is useless to construct the fate \& character of an individual out of ``drives'' \& ``reactions'' unless we understand the goal to which such movements are directed. Neither of these psychologies thinks in terms of individual goals.

It is true that when the word ``goal'' is mentioned, the reader is likely to have a hazy impression. The idea needs to be concretized. Now in the last analysis to have a goal is to aspire to be like God. But to be like God is of course the ultimate goal -- the goal of goals, if we may use the term. Educators should be cautious in attempting to educate themselves \& their children to be like God. As a matter of fact we find that the child in his development substitutes a more concrete \& immediate goal. Children look for the strongest person in their environment \& make him their model or their goal. It may be the father, or perhaps the mother, for we find that even a boy may be influenced to imitate his mother if she seems the strongest person. Later on they want to be coachmen because they believe the coachman is the strongest person.

When children 1st conceive such a goal they behave, feel \& dress like the coachman \& take on all the characteristics consistent with the goal. But let the policeman lift a finger, \& the coachman becomes nothing $\ldots$ Later on the ideal may become the doctor or the teacher. For the teacher can punish the child \& thus he arouses his respect as a strong person.

The child has a choice of concrete symbols in selecting his goal, \& we find that the goal he chooses is really an index of his social interests. A boy, asked what he wanted to be in later life, said, ``I want to be a hangman.'' This displays a lack of social interest. The boy wished to be the master of life \& death -- a role which belongs to God. He wished to be more powerful than society, \& he was thus headed for the useless life. The goal of being a doctor is also fashioned around the God-like desire of being master of life \& death, but here the goal is realized through social service.'' -- \cite[pp. 31--55]{Adler2013}

%------------------------------------------------------------------------------%

\section{The Inferiority Complex}
``The use of the terms ``consciousness'' \& ``unconsciousness'' to designate distinctive factors is incorrect in the practice of Individual Psychology. Consciousness \& unconsciousness move together in the same direction \& are not contradictions, as is so often believed. What is more, there is no definite line of demarcation between them. It is merely a question of discovering the purpose of their joint movement. It is impossible to decide on what is conscious \& what is not until the whole connection has been obtained. This connection is revealed in the prototype, that pattern of life which we analyzed in the last chapter.

A case history will serve to illustrate the intimate connection between conscious \& unconscious life. A married man, 40 years old, suffered from 1 anxiety -- a desire to jump out of the window. He was always struggling against this desire, but aside from this he was quite well. He had friends, a good position, \& lived with his wife happily. His case is inexplicable except in terms of the collaboration of consciousness \& unconsciousness. Consciously he had the feeling that he must jump out of a window. Nonetheless he lived on, \& in fact he never even attempted to jump out of a window. The reason for this is that there was another side to his life, a side in which a struggle against his desire to commit suicide played an important part. As a result of the collaboration of this unconscious side of his being with his consciousness, he came out victorious. In fact in his ``style of life'' -- to use a term about which we shall have more to say in a later chapter -- he was a conqueror who had attained the goal of superiority. The reader might ask how could this man feel superior when he had this conscious tendency \textit{to commit suicide?} The answer is that there was something in his being that was fighting his battle against his suicidal tendency. It is his success in this battle that made him a conqueror \& a superior being. Objectively his struggle for superiority was conditioned by his own weakness, as is very often the case with persons who in 1 way or another feel inferior. But the important thing is that in his own private battle his striving for superiority, his striving to live \& to conquer, came out ahead of his sense of inferiority \& desire to die -- \& this despite the fact that the latter was expressed in his conscious life \& the former in his unconscious life.

Let us see if the development of this man's prototype bears out our theory. Let us analyze his childhood remembrances. At an early age, we learn, he had trouble at school. He did not like other boys \& wanted to run away from them. Nonetheless he collected all his powers to stay \& face them. In other words we can already perceive an effort on his part to overcome his weakness. He faced his problem \& conquered.

If we analyze our patient's character, we shall see that his 1 aim in life was to overcome fear \& anxiety. In this aim his conscious ideas cooperated with his unconscious ones \& formed a unity. Now a person who does not see the human being as a unity might believe that this patient was not superior \& was not successful. He might think him to be only an ambitious person, one who wanted to struggle \& fight but who was at bottom a coward. Such a view would be erroneous, however, since it would not take into consideration all the facts in the case \& interpret them with reference to the unity of a human life. Our whole psychology, our whole understanding or striving to understand individuals would be futile \& useless if we could not be sure that the human being is a unity. If we presupposed 2 sides without relation to one another it would be impossible to see life as a complete entity.

In addition to regarding an individual's life as a unity, we must also take it together with its context of social relations. Thus children when 1st born are weak, \& their weakness makes it necessary for other persons to care for them. Now the style or the pattern of a child's life cannot be understood without reference to the persons who look after him \& who make up for his inferiority. The child has interlocking relations with the mother \& family which could never be understood if we confined our analysis to the periphery of the child's physical being in space. The individuality of the child cuts across his physical individuality, it involves a whole context of social relations.

What applies to the child applies also, to a certain extent, to men as a whole. The weakness which is responsible for the child's living in a family group is paralleled by the weakness which drives men to live in society. All persons feel inadequate in certain situations. They feel overwhelmed by the difficulties of life \& are incapable of meeting them single-handed. Hence 1 of the strongest tendencies in man has been to form groups in order that he may live as a member of a society \& not as an isolated individual. This social life has without doubt been a great help to him in overcoming his feeling of inadequacy \& inferiority. We know that this is the case with animals, where the weaker species always live in groups in order that their combined powers might help to meet their individual needs. Thus a herd of buffaloes can defend themselves against wolves. 1 buffalo alone would find this impossible, but in a group they stick their heads together \& fight with their feet until they are saved. On the other hand, gorillas, lions \& tigers can live isolated because nature has given them the means of self-protection. A human being has not their great strength, their claws, nor their teeth, \& so cannot live apart. Thus we find that the beginning of social life lies in the weakness of the individual.

Because of this fact we cannot expect to find that the abilities \& faculties of all human begins in society are equal. But a society that is rightly adjusted will not be behindhand in supporting the abilities of the individuals who compose it. This is an important point to grasp, since otherwise we would be led to suppose that individuals have to be judged entirely on their inherited abilities. As a matter of fact an individual who might be deficient in certain faculties if he lived in an isolated condition could well compensate for his lacks in a rightly organized society.

Let us suppose that our individual insufficiencies are inherited. It then becomes the aim of psychology to train people to live well with others, in order to help decrease the effect of their natural disabilities. The history of social progress tells the story of how men co-operated in order to overcome deficiencies \& lacks. Everybody knows that language is  a social invention, but few people realize that individual deficiency was the mother o that invention. This truth, however, is illustrated in the early behavior of children. When their desires are not being satisfied, they want to gain attention \& they try to do so by some sort of language. But if a child should not need to gain attention, he would not try to speak at all. This is the case in the 1st few months, when the child's mother supplies everything that the child wishes before it speaks. There are cases on record of children who did not speak until 6 years of age because it was never necessary for them to do so. The same truth is illustrated in the case of a particular child of deaf \& dumb parents. When he fell \& hurt himself he cried, but he cried without noise. He knew that noise would be useless as his parents could not hear him. Therefore he made the appearance of crying in order to gain the attention of his parents, but it was noiseless.

We see therefore that we must always look at the whole social context of the facts we study. We must look at the social environment in order to understand the particular ``goal of superiority'' an individual chooses. We must look at the social situation, too, in order to understand a particular maladjustment. Thus many persons are mal-adjusted because they find it impossible to make the normal contact with others by means of language. The stammerer is a case in point. If we examine the stammerer we shall see that since the beginning of his life he was never socially well adjusted. He did not want to join in activities, \& he did not want friends or comrades. His language development needed association with others, but he did not want to associate. Therefore his stammering continued. There are really 2 tendencies in stammerers -- one to associate with others, \& another that makes them seek isolation for themselves.

Later in life, among adult persons not living a social life, we find that they cannot speak in public \& have a tendency to stage fright. This is because they regard their audiences as enemies. They have a feeling of inferiority when confronted by a seemingly hostile \& dominating audience. The fact is that only when a person trusts himself \& his audience can he speak well, \& only then will he not have stage fright.

The feeling of inferiority \& the problem of social training are thus intimately connected. Just as the feeling of inferiority arises from a social maladjustment, so social training is the basic method by which we can all overcome our feelings of inferiority.

There is a direct connection between social training \& common sense. When we say that people solve their difficulties by common sense, we have in mind the pooled intelligence of the social group. On the other hand, as we indicated in the last chapter, persons who act with a private language \& a private understanding manifest an abnormality. The insane, the neurotics \& the criminals are of this type. We find that certain things are not interesting to them -- people, institutions, the social norms make no appeal to them. \& yet it is through these things that the road to their salvation lies.

In working with such persons our task is to make social facts appeal to them. Nervous persons always feel justified if they show good will. But more than good will is needed. We must teach them that it is what they actually accomplish, what they actually give, that matters in society.

While the feeling of inferiority \& the striving for superiority are universal, it would be a mistake to regard this fact as indicating that all men are equal. Inferiority \& superiority are the general conditions which govern the behavior of men, but besides these conditions there are differences in bodily strength, in health, \& in environment. For that reason different mistakes are made by individuals in the same given conditions. If we examine children we shall see that there is no one absolutely fixed \& right manner for them to respond. They respond in their individual ways. They strive towards a better style of life, but they all strive in their own way, making their own mistakes \& their own type of approximations to success.

Let us analyze some of the variations \& peculiarities of individuals. Let us take, e.g., left-handed children. There are children who may never known that they are left-handed because they have been so carefully trained in the use of the right hand. At 1st they are clumsy \& imperfect with the right hand, \& they are scolded, criticized \& derided. It is an error to deride, but both hands should be trained. A left-handed child can be recognized in the cradle because his left hand moves more than his right. In later life he may feel that he is burdened because of the imperfection of his right hand. On the other hand, he often develops a greatest interest in his right hand \& arm, which interest is manifested, e.g., in drawing, writing, etc. In fact it is not surprising to find that later in life such a child is better trained than a normal child. Because he has had to get interested, he has gotten up earlier, so to speak, \& thus his imperfection has led him to more careful training. This is often a great advantage in developing artistic talent \& ability. A child in such a position is usually ambitious \& fights to overcome his limitations. Sometimes, however, if the struggle is a serious one, he may become envious or jealous of others \& thus develop a greater feeling of inferiority which is more difficult to overcome than in normal cases. Through constant struggling a child may become a fighting child or a fighting adult, always striving with the fixed idea in mind that he ought not to be clumsy \& deficient. Such an individual is more burdened than others.

Children strive, make mistakes, \& develop in various ways according to the prototypes they formed in the 1st 4 or 5 years of life. The goal of each is different. 1 child may want to be a painter, while another may wish himself out of this world where he is a misfit. We may know how he can overcome his imperfection, but he does not know it, \& all too often the facts are not explained to him in the right way.

Many children have imperfect eyes, ears, lungs or stomachs, \& we find their interest stimulated in the direction of the imperfection. A curious instance of this is revealed in the case of a man who suffered from attacks of asthma only when he came home at night from the office. He was a man of 45, married, \& with a good position. He was asked why the attacks always occurred after he came home from the office. He explained, ``You see, my wife is very materialistic \& I am idealistic, hence we do not agree. When I come home I would like to be quiet, to enjoy myself at home, but my wife wants to go into society \& so she complains about remaining at home. Whereupon I get into a bad temper \& start to suffocate.''

Why did this man suffocate: why did he not vomit? The fact is he was only being true to his prototype. It seems that as a child he had to be bandaged for some weakness \& this tight binding affected his breathing \& made him very uncomfortable. He had a maid servant, however, who liked him \& would sit beside him \& console him. All her interest was in him \& not in herself. She thus gave him the impression that he would always be amused \& consoled. When he was 4 years old the nurse went away to a wedding \& he accompanied her to the station crying very bitterly. After the nurse had left he said to his mother, ``The world has no more interest for me now that my nurse has gone away.''

Hence we see him in manhood as in the years of his prototype, looking for an ideal person who would always amuse him \& console him \& be interested in him alone. The trouble was not too little air but the fact that he was not being amused \& consoled at all times. Naturally, to find a person who will always amuse you is not easy. He always wanted to rule the whole situation \& to a certain degree it helped him when he succeeded. Thus when he took to suffocating, his wife stopped wanting to go to the theater or into society. He had then obtained his ``goal of superiority.''

Consciously this man was always right \& proper, but in his mind he has the desire to be the conqueror. He wanted to make his wife what he called idealistic instead of materialistic. We should suspect such a man of motives different from those on the surface $\ldots$

We often see children with imperfect eyes take more of an interest in visual things. They develop a keen faculty in this way. We see Gustav Freitag, a great poet who had poor, astigmatic eyes, accomplishing much. Poets  \& painter often have trouble with their eyes. But this in itself often creates greater interest. Freitag said about himself: ``Because my eyes were different from those of other people, it seems that I was compelled to use \& train my fantasy. I do not know that this has helped me to be a great writer, but in any case as a result of my eyesight it has come about that I can see better in fantasy than others in reality.''

If we examine the personalities of geniuses we shall often find poor eyes or some other deficiency. In the history of all ages even the gods have had some deficiency such as blindness in 1 or both eyes. The fact that there are geniuses who though nearly blind are yet able to understand better than others the differences in lines, shadows \& colors shows what can be done with afflicted children if their problems are properly understood.

Some people are more interested in eatables than others. Because of this they are always discussing what they can \& what they cannot eat. Usually such persons have had a hard time at the beginning of life in the matter of eating \& so have developed more interest in it than others. They had probably been told constantly by a watchful mother what they could \& could not eat. Such persons have to train to overcome the imperfections of their stomachs, \& they become vitally interested in what they will have for lunch, dinner or breakfast. As a result of their constant thought about eating they sometimes develop the art of cookery or become experts on questions of diet.

At times, however, a weakness of the stomach or the intestines causes people to look for a substitute for eating. Sometimes this substitute is money, \& such persons become miserly or great money-making bankers. They often strive extremely hard to collect money, training themselves for this purpose day \& night. They never stop thinking of their business, -- a fact which may sometimes give them a great advantage over others in similar walks of life. \& it is interesting to note that we often hear of rich men suffering from stomach trouble.

Let us remind ourselves at this point of the connection frequently made between body \& mind. A given defect does not always lead to the same result. There is no necessary cause \& effect relation between a physical imperfection \& a bad style of life. For the physical imperfection we can often give good treatment in the form of right nutrition \& thereby partly obviate the physical situation. But it is not the physical defect which causes the bad results: it is the patient's attitude which is responsible. That is why for the individual psychologist mere physical defects or exclusive physical causality does not exist, but only mistaken attitudes towards physical situations. Also that is why the individual psychologist seeks to foster a striving against the feeling of inferiority during the development of the prototype.

Sometimes we see a person impatient because he cannot wait to overcome difficulties. Whenever we see persons constantly in motion, with strong tempers \& passions, we can always conclude that they are persons with a great feeling of inferiority. A person who knows he can overcome his difficulties will not be impatient. On the other hand he may not always accomplish what is necessary. Arrogant, impertinent, fighting children also indicate a great feeling of inferiority. It is our task in their case to look for the reasons -- for the difficulties they have -- in order to prescribe the treatment. We should never criticize or \textit{punish} mistakes in the style of life of the prototype.

We can recognize these prototype traits among children in very peculiar ways -- in their unusual interests, in their scheming \& striving to surpass others, \& in building toward the goal of superiority. There is a type that does not trust himself in movement \& expression. He prefers to exclude others as far as possible. He prefers not to go where he is confronted with new situations but to stay in the little circle in which he feels sure. In school, in life, in society, in marriage he does the same. He is always hoping to accomplish much in his little place in order to arrive at a goal of superiority. We find this trait among many human beings. They all forget that to accomplish results, one must be prepared to meet all situations. Everything must be faced. If one eliminates certain situations \& certain persons, one has only private intelligence to justify oneself, \& this is not enough. One needs all the renovating winds of social contact \& common sense.

If a philosopher wants to accomplish his work, he cannot always go to lunch or dinner with others, for he needs to be alone for long periods of time in order to collect his ideas \& use the right method. But later on he must grow through contact with society. This contact is an important part of his development. \& so when we meet with such a person we must remember his 2 requirements. We must remember, too, that he can be useful or useless \& should therefore look carefully for the difference between useful \& useless behavior.

The key to the entire social process is to be found in the fact that persons are always striving to find a situation in which they excel. Thus children who have a great feeling of inferiority want to exclude stronger children \& play with weaker children whom they can rule \& domineer. This is an abnormal \& pathological expression of the feeling of inferiority, for it is important to realize that it is not the sense of inferiority which matters but the degree \& character of it.

The abnormal feeling of inferiority has acquired the name of ``inferiority complex.'' But complex is not the correct word for this feeling of inferiority that permeates the whole personality. It is more than a complex, it is almost a disease whose ravages vary under different circumstances. Thus we sometimes do not notice the feeling of inferiority when a person is on his job because he feels sure of his work. On the other hand he may not be sure of himself in society or in his relations with the opposite sex, \& in  this way we are able to discover his true psychological situation.

We notice mistakes in a greater degree in a tense or difficult situation. It is in the difficult or new situation that the prototype appears rightly, \& in fact the difficult situation is nearly always the new one. That is why, as we said in the 1st chapter, the expression of the degree of social interest appears in a new social situation.

If we put a child to school we may observe his social interest there just as in general social life. We can see whether he mixes with his fellows or avoids them. If we see hyperactive, sly, clever children, we must look into their minds to find the reasons. \& if we see some go forward only conditionally or hesitatingly, we must be on the lookout for the same characteristics to be revealed later on in society, life \& marriage.

We always meet persons who say, ``I would do this in this way,'' ``I would take that job,'' ``I would fight that man, $\ldots$ but $\ldots$!'' All such statements are a sign of a great feeling of inferiority, \& in fact if we read them this way we get a new light on certain emotions, such as doubt. We recognize that a person in doubt usually remains in doubt \& accomplishes nothing. However, when a person says ``I won't,'' he will probably act accordingly.

The psychologist, if he looks closely can often see contradictions in men. Such contradictions may be considered as a sign of a feeling of inferiority. But we must also observe the movements of a person who constitutes our problem on hand. Thus, his approach, his way of meeting people, may be poor, \& we must observe if he comes towards persons with a hesitating step \& bodily attitude. This hesitation will often be expressed in other situations of life. There are many persons who take 1 step forward \& 1 backward -- a sign of a great feeling of inferiority.

Our whole task is to train such persons away from their hesitating attitude. The proper treatment for such persons is to encourage them -- never to discourage them. We must make them understand that they are capable of facing difficulties \& solving the problems of life. This is the only way to build self-confidence, \& this is the only way the feeling of inferiority should be treated.'' -- \cite[pp. 56--77]{Adler2013}

%------------------------------------------------------------------------------%

\section{The Superiority Complex}

%------------------------------------------------------------------------------%

\section{The Style of Life}

%------------------------------------------------------------------------------%

\section{Old Remembrances}

%------------------------------------------------------------------------------%

\section{Attitudes \& Movements}

%------------------------------------------------------------------------------%

\section{Dreams \& Their Interpretation}

%------------------------------------------------------------------------------%

\section{Problem Children \& Their Education}

%------------------------------------------------------------------------------%

\section{Social Problems \& Social Adjustment}

%------------------------------------------------------------------------------%

\section{Social Feeling, Common Sense \& the Inferiority Complex}

%------------------------------------------------------------------------------%

\section{Love \& Marriage}

%------------------------------------------------------------------------------%

\section{Sexuality \& Sex Problems}

%------------------------------------------------------------------------------%

\section{Conclusion}

%------------------------------------------------------------------------------%

\printbibliography[heading=bibintoc]
	
\end{document}