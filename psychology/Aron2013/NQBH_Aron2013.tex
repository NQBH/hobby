\documentclass{article}
\usepackage[backend=biber,natbib=true,style=authoryear]{biblatex}
\addbibresource{/home/nqbh/reference/bib.bib}
\usepackage{tocloft}
\renewcommand{\cftsecleader}{\cftdotfill{\cftdotsep}}
\usepackage[colorlinks=true,linkcolor=blue,urlcolor=red,citecolor=magenta]{hyperref}
\usepackage{algorithm,algpseudocode,amsmath,amssymb,amsthm,float,graphicx,mathtools}
\allowdisplaybreaks
\numberwithin{equation}{section}
\newtheorem{assumption}{Assumption}[section]
\newtheorem{conjecture}{Conjecture}[section]
\newtheorem{corollary}{Corollary}[section]
\newtheorem{definition}{Definition}[section]
\newtheorem{example}{Example}[section]
\newtheorem{lemma}{Lemma}[section]
\newtheorem{notation}{Notation}[section]
\newtheorem{principle}{Principle}[section]
\newtheorem{problem}{Problem}[section]
\newtheorem{proposition}{Proposition}[section]
\newtheorem{question}{Question}[section]
\newtheorem{remark}{Remark}[section]
\newtheorem{theorem}{Theorem}[section]
\usepackage[left=1cm,right=1cm,top=5mm,bottom=5mm,footskip=4mm]{geometry}
\def\labelitemii{$\circ$}

\title{The Highly Sensitive Person: How to Thrive When the World Overwhelms You}
\author{Elaine N. Aron}
\date{\today}

\begin{document}
\maketitle
\tableofcontents

\begin{quotation}
	``Engaging, perceptive $\ldots$ suggests new paths for making sensitivity a blessing, not a handicap. A must-read.'' -- Philip G. Zimbardo, author of \textit{Shyness}
\end{quotation}

\section*{What readers are saying about Elaine Aron \& \textit{The Highly Sensitive Person} $\ldots$}

\begin{quotation}
	``I have just finished \textit{The Highly Sensitive Person} \& I can't thank you enough for writing such a wonderful book -- you put into clear, understandable words what I have always known about myself. As I read your book, I felt for the 1st time in my life that someone truly understood what it was like to go through life as a highly sensitive individual $\ldots$ Your book was the 1st that I have ever read that not only validated the traits of highly sensitive individuals but cast them as necessary for our society.'' -- M. C., Rockaway, NJ
	
	``1st, let me express my deep gratitude to you if I can. I have just finished reading your book $\ldots$ You have truly given me hope for a new life at the age of 52. I hardly know how to express the comfort \& joy I have received from you $\ldots$ Once again, thank you, thank you, thank you!'' -- J. M., New York, NY
	
	``I cannot thank you enough for the inner peace your book has given me!'' -- S. P., Sacramento, CA
	
	``This book has opened my eyes to the fact that I am not alone in my sensitivity \textit{\&} that it is OK to be this way $\ldots$ I've always felt that there was something wrong with me $\ldots$ It has given me tremendous insight $\ldots$ So thank you for your research \& your words of encouragement. They've both been a blessing.'' -- M. G., Belle River, Ontario (Canada)
	
	``I am writing to express my gratitude to Elaine Aron for her book, \textit{The Highly Sensitive Person}. I laughed \& cried, I felt known. I felt affirmed. It is not only `OK' to be highly sensitive, it is a gift. Thank you.'' -- L. H., Findlay, OH
	
	``Thank you for writing such a wonderful book.'' -- R. P., Norwalk, CA
	
	``$\ldots$ it really helped me understand myself a lot better.'' -- E. S., Westerville, OH
	
	``I can't remember the last time I sat down \& read a book from cover to cover in 1 day. It has really made me feel like a part of a larger group, \& not quite so weird after all $\ldots$ I am looking forward to reading this book again.'' -- K. J., San Francisco, CA
	
	``I loved the book!'' -- S. R., Springfield, MA
	
	``I just finished reading Elaine N. Aron's excellent book $\ldots$ The descriptions fit me perfectly! It was inspiring, informative, \& emotional.'' -- R. D., San Francisco, CA
	
	``I find Dr. Aron's book immensely valuable.'' -- L. J. W., Provo, UT
	
	``I have been trying to find out who I am \& what I can do. Many of the situations described in the book I find fit my situation $\ldots$ I wish I could send [it to] everyone I know \& have known.'' -- C. M., Riverside, CA
	
	``I just read your book \& it is \textit{extraordinary}! Absolutely the best \& most helpful of many I've read $\ldots$ You have done tremendous work \& I am so deeply touched by much of what you say.'' -- S. S., New York, NY
	
	``This book, \textit{The Highly Sensitive Person}, was a revelation to me.'' -- A. A., Tustin, CA
	
	``Your book $\ldots$ has helped me so much.'' -- A. B., Lethbridge, Alberta (Canada)
	
	``\textit{The Highly Sensitive Person} was a true revelation to me \& to several others I recommended the book to.'' -- D. R., Irvine, CA
	
	``Elaine Aron's book, \textit{The Highly Sensitive Person}, is the 1st ever to really speak to \textit{me}!'' -- M. J., Houston, TX
	
	``I have enjoyed reading your book, \textit{The Highly Sensitive Person}, \& find the information \& insights extremely valuable.'' -- M. F., Mountain View, CA
\end{quotation}

\begin{quotation}
	To Irene Bernadicou Pettit, Ph.D. -- being both poet \& peasant, she knew how to plant this seed \& tend it until it blossomed.
	
	To Art, who especially loves the flowers -- 1 more love we share.
\end{quotation}

\section*{Acknowledgments}
``I especially want to acknowledge all the highly sensitive person I interviewed. You were the 1st to come forward \& talk about what you had known very privately about yourself for a long time, changing yourselves from isolated individuals to a group to be respected. My thanks also to those who have come to my courses or seen me for a consultation or in psychotherapy. Every word of this book reflects what you all have taught me.

My many student research assistants -- too many to name -- also earn a big thanks, as do Barbara Kouts, my agent, \& Bruce Shostak, my editor at Carol, for their effort to see that this book reached all of you. Barbara found a publisher with vision; Bruce brought the manuscript into good shape, reining me in at all the right places but otherwise letting me run with it as I saw it.

It's harder to find words for my husband, Art. But here are some: Friend, colleague, supporter, beloved-thanks, with all my love.'' -- \cite[p. 6]{Aron2013}

\begin{quotation}
	``I believed in aristocracy, though -- if that is the right word, \& if a democrat may use it. Not an aristocracy of power $\ldots$ but $\ldots$ of the sensitive, the considerate $\ldots$ Its members are to be found in all nations \& classes, \& all through the ages, \& there is a secret understanding between them when they meet. They represent the true human tradition, the 1 permanent victory of our queer race over cruelty \& chaos. Thousands of them perish in obscurity, a few are great names. They are sensitive for others as well as themselves, they are considerate without being fussy, their pluck is not swankiness but the power to endure $\ldots$'' -- E. M. Forster, ``What I Believe,'' in \textit{2 Cheers for Democracy} 
\end{quotation}

%------------------------------------------------------------------------------%

\section*{Author's Note, 2012}
``In 1998, 3 years after this book was 1st published, I wrote a new preface for it titled ``A Celebration.'' It was an invitation for all of us to feel good about how many people had discovered they were highly sensitive \& found the book useful, \& that the idea was catching on in the scientific world. Now we can celebrate about 50 times more of the same. \textit{The Highly Sensitive Person} has been translated into 14 languages, from Swedish, Spanish, \& Korean to Hebrew, French, \& Hungarian. There have been articles about high sensitivity in many prominent media throughout  the world. In the U.S., that has included a feature in \textit{Psychology Today}, a shorter discussion in \textit{Time}, \& many women's \& health magazines such as \textit{O Magazine} as well as numerous health websites. There are ``HSP Gatherings'' \& courses on the subject in the United States \& Europe, plus YouTube videos, books, magazines, newsletters, \& websites \& all sorts of services exclusively highly sensitive persons -- most good \& some, well, not as good. Tens of thousands subscribe to my own newsletter, \textit{Comfort Zone}, at \url{hsperson.com}, where there are now hundreds of newsletter articles archived covering every aspect of being highly sensitive. We have come a long way.'' -- \cite[p. 10]{Aron2013}

\subsection*{3 Revisions, Right Here}
``Given that this book was written at the very beginning of a minor revolution, I have thought I should revise it. But when I look it over, there's not much I would change. It does the job well, with 3 exceptions. 1st, \& most important, I wanted to add the expanded scientific research. That's vital because it helps us all to trust that this trait is real, that what is in this book is real. This preface will update you on the research.

2nd, there is now a simple, comprehensive description of the trait, ``DOES,'' that expresses its facets nicely. \textit{D} is for depth of processing. Our fundamental characteristic is that we observe \& reflect before we act. We process everything more, whether we are conscious of it or not. \textit{O} is for being easily overstimulated, because if you are going to pay more attention to everything, you are bound to tire sooner. \textit{E} is for giving emphasis to our emotional reactions \& having strong empathy which among other things helps us notice \& learn. \textit{S} is for being sensitive to all the subtleties around us. I will say more about these when I discuss the research.

3rd, a smaller point can be taken care of right now -- the discussion in the book of antidepressants, which focused on Prozac. Medications for treating depression have proliferated since 1996, as have the pros \& cons about them. Do they damage the rest of the body? Are they just placebos for most people, making them feel good to the same degree as if they had been given a sugar pill? But what about many suicides they have surely prevented? Haven't they also improved the lives of people close to those who are no longer depressed? The arguments on both sides are still there, both worth understanding. Thankfully these are now all on the Internet somewhere (but stick to reading about scientific research -- skip the horror stories, on either side). So my basic advice is the same: Become very well informed; then decide for yourself. To form an opinion before you ever become depressed is preferable, because under certain circumstances highly sensitive people are genetically more susceptible to depression, \& it is a difficult decision when you are in the thick of it.

At this point, if you are not interested in the research on sensitivity you can stop reading or just skim. Perhaps you are the type who understands this trait intuitively or ``from the heart,'' with no need for the intellect. However, I imagine that you sometimes find that you have to satisfy others' skepticism or even hostility about your suggestion that you are highly sensitive \& you might like some toolds for handing such times, which research findings can provide.'' -- \cite[p. 10]{Aron2013}

\subsection*{The Research Since 1996}
``Not only has science verified so much of what's in this book (some of which was only based on my observations at the time), but the findings have gone far beyond what we knew when I wrote it. I have tried to keep what follows interesting, but with enough detail to satisfy those who really want to know. You can find the full methodology \& results by reading the articles themselves. I published a good summary of the theory \& research in 2012 \& a current list of studies can always be found at \url{www.hsperson.com}. \textit{Sensory processing sensitivity} is the scientific name I have given the trait (not at all the same as Sensory Processing Disorder or Sensory Integration Disorder, which, alas, was given a similar name). I should add that concepts very much like sensitivity are being studied by other researchers. If you are interested in this work, you can look up terms such as Biological Sensitivity to Context (Thomas Boyce, Bruce Ellis, \& others), Differential Susceptibility (Jay Belsky, Michael Pluess, etc), \& Orienting Sensitivity (D. Evans \& Mary Rothbart, etc.) \& find even more research, all done since \textit{The Highly Sensitive Person} was written.'' -- \cite[p. 11]{Aron2013}

\subsubsection*{The 1st Research}
``The very 1st published studies we did (myself \& my husband, who is unusually good at designing research) generated the Highly Sensitive Person (HSP) Scale in this book. This research was also intended to demonstrate that high sensitivity is not the same as introversion or ``neuroticism'' (professional jargon for a tendency to be depressed or excessively anxious). We were right; the trait was not the same. But it was strongly associated with neuroticism. I had a hunch why, \& our 2nd series of studies, published in 2005, verified it: HSPs with a troubled childhood are more at risk of becoming depressed, anxious, \& shy than non-sensitive people with a similar childhood; but those with good-enough childhoods were no more at risk than others. There was even some indication -- \& more since -- that they are better off than nonsensitive people with good childhoods, as if they are more affected by any environment. A later study by Miriam Liss \& others found the same result, mainly for depression. Remember this is ``on the average.'' Some sensitive people with good childhoods may still be depressed \& some of with poor childhoods will not be. Further, many other things besides childhood difficulties affect us. The level of stress one lives under is surely 1 large factor.

This interaction of the trait \& one's childhood environment explains the relatively strong association between neuroticism or negative feelings \& high sensitivity that we found in the 1st study. Roughly half of the questions on the HSP Scale tap negative feelings -- ``I am made uncomfortable $\ldots$'' ``I get rattled $\ldots$'' ``I am annoyed $\ldots$'' \& so forth. Since many HSPs have had difficult childhoods, often because no one understood their innate temperament, their persistent bad feelings due to the trait could cause them to feel even more uncomfortable, rattled, or annoyed in situations that bother all sensitive persons to some degree. This would have added to the overlap of high sensitivity \& neuroticism for a reason that has nothing to do with the trait itself. When we use the scale now, we have various ways of asking people how much negative emotion they feel generally \& take that into account statistically.

Unfortunately, quite a few clinical studies of the relationship between being highly sensitive \&, e.g., being anxious, stressed, or having communication phobias have not taken the role of ``nurture'' into account, making it seem that all HSPs have these  problems. Hence I will not describe that research here.'' -- \cite[pp. 11--12]{Aron2013}

\subsubsection*{Serotonin \& HSPs}
``This finding about the additional impact on HSPs of their childhood, good \& bad, adds a nice footnote to something I said in this book, in the chapter on doctors \& medications. I cited a study by Stephen Suomi about a minority of rhesus monkeys who are born with a trait that was originally called ``up tight'' because they were more affected by being raised under stressful conditions. Not only did they appear more depressed \& anxious, but like depressed humans, they had less serotonin available in their brains, what antidepressants correct. Serotonin is a chemical used in at least 17 places in the brain in order to move around information. as it turned out, these vulnerable monkeys had a genetic variation that results in lower levels of serotonin generally, \& these levels are further reduced by stress. Sensitive humans have the same genetic variation. Interestingly, the variation is only found in 2 primate species, humans \& rhesus monkeys, \& both are highly social \& able to adapt to a wide range of environments. Perhaps the highly sensitive members of a group are better able to notice the subtleties, such as which new foods can be safely eaten \& which dangers to avoid, allowing them to survive better in a new place.

There are many, many genetic variations in all of us -- hair, eye, \& skin color, e.g., or special abilities or certain phobias. Some of these variations appear to serve little purpose; others are useful or not (or even a disadvantage) depending on the environment. If you live where there are many poisonous snakes, having an innate fear of them could be an advantage, but perhaps become a problem if you want to be a science teacher.

Anyway, since I wrote the book \& explained about those monkeys, research done in Denmark by Cecilie Licht \& others suggests that HSPs have the same genetic variation. For years, research had only looked for low serotonin's association with depression, \& the results were highly inconsistent, probably because in some studies they had inadvertently included too many sensitive people with good childhoods for depression to show up.

There had to be some positive reason for so many people having what should be an evolutionary disadvantage, a ``tendency to depression.'' Now new research demonstrates that this genetic variation causing lower serotonin to be available in the brain also bestows benefits, such as improved memory of learned material, better decision making, \& better overall mental functioning, plus gaining even more positive mental health than others from positive life experiences. The same mental benefits are also found in rhesus monkeys with  the same genetic variation. Perhaps the best vindication for HSPs tired of being seen as weaklings or sick is a study by Suomi finding that rhesus monkeys with this trait, if raised by skilled mothers, were more likely to show ``developmental precocity'' resilience to stress, \& be leaders of their social groups.

In the same vein, a growing body of research by others suggests that some individuals are especially sensitive \& therefore more susceptible to their environment -- e.g., as children they are more affected by parenting, by teachers, \& by helpful interventions. What is the underlying trait that leads to this ``for better \textit{\&} for worse'' outcome for us?'' -- \cite[pp. 12--14]{Aron2013}

\subsection*{What Makes Us So Different?}
``As I wrote in this book, many species -- we now know it's over 100, so far, including fruit flies \& some fish species -- have a minority of individuals that are highly sensitive. Although obviously the trait leads to different behaviors depending on whether you are a fruit fly fish, bird, dog, deer, monkey or human, a general description of it would be that the minority who have inherited it have adopted a survival strategy of pausing to check, observe, \& reflect on or process what has been noticed before choosing an action. Slowness to act, however, is not the hallmark of the trait. When sensitive individuals see right away that their situation is like a past one, thanks to having learned so thoroughly from thinking it over, they can react to a danger or opportunity faster than others. For this reason, the most basic aspect of the trait -- the depth of processing -- has been difficult to observe. Without knowing about it, when someone paused before acting, others could only guess what was happening inside that person. Often HSPs were thought to be inhibited, shy, fearful, or introverted (in fact, 30\% of HSPs are actually extraverts, \& many introverts are not HSPs). Some HSPs accepted those labels, having no other explanation for their hesitancy. Indeed, feeling different \& flawed, some of us found the label ``shy, or fearful of social judgment'' self-fulfilling, as I describe in Chap. 5. Others knew they were different, but hid  it \& adapted, acting like the non-sensitive majority.

Understanding why we evolved as we did tells us much more about ourselves than I knew when I wrote this book. At that time I thought our sensitivity had evolved because the trait served the larger group, as sensitive individuals can sense a danger or opportunity that the others miss, while these others serve by doing something about it once they are alerted. This may still be partly true, but that may only be a side effect of the trait. The current explanation comes from a computer model done by biologists in the Netherlands. Max Wolf \& his colleagues were curious about how sensitivity might evolve, so they set up a situation using a computer program in order to exclude all other factors. Then they varied just a few things at a time \& watched to see what happened when they ran out the various possible situations \& strategies. They wanted to see if being highly responsive could be a successful enough trait to remain in a population -- traits that make us unsuccessful at life don't last long.

The sensitive strategy was tested by setting up the scenario in which they varied how much an individual, learning from Situation A, by being more sensitive to everything that happened there, was more successful in Situation B because of having that information (they also had to vary the amount of benefit that came with being successful in Situation B). The other extreme scenario was such that learning from Situation A provided no help in Situation B because the 2 had nothing to do with each other. The question was, under what conditions would you see the evolution of 2 types of individuals, one using the strategy of learning from experience \& one not? It turned out that there only had to be a small benefit for the 2 strategies to emerge, hence explaining why the 2 would exist in real people.

You might think that being sensitive is always an advantage, but many times it is not. Indeed, sensitivity only serves the individual if he or she is in the minority. If everyone were sensitive it would be no advantage, as when, if everyone knows a short cut \& uses it, there are so many making use of the information that it benefits no one. In short, sensitivity, or responsibility as these biologists also called it, involves paying more attention to details than others do, then using that knowledge to make better predictions in the future. Sometimes you are better off doing so, but other times your extra attention \& effort have no pay-off.

Sensitivity does have its costs, as you know. It really can be a waste of energy if what is happening now has nothing to do with your past experiences. Further, when a past experience was very bad, an HSP can overgeneralize \& avoid or feel anxious in too many situations, just because the new ones resemble in some small way the past bad one. The biggest cost to us of being highly sensitive, however, is that our nervous system can become overloaded. Everyone has a limit as to how much information or stimulation can be taken in before getting overloaded, overstimulated, overaroused, overwhelmed, \& just \textit{over}! We simply reach that point sooner than others. Fortunately, as soon as we get some downtime we recover nicely.'' -- \cite[pp. 14--16]{Aron2013}

\subsection*{It's Really in Our Genes}
``When I wrote the book, I said sensitivity is innate. I knew it had been found from birth in children, \& in animals where the genetics had been identified, you can selectively breed animals to be more sensitive. But I had no genetic research using the HSP Scale on which to base that claim. Now it exists. I already mentioned 1 study that found scores on the test were related to a variation in a gene known to affect the availability of serotonin in the brain. Chen \& his associates, working in China, took a different approach. Rather than looking at a specific gene with known properties, they looked at all of the gene variations (98 in all) affecting the amount of dopamine, another chemical necessary for the transmission of information, available in certain areas of the brain. They found the HSP Scale associated with 10 variations on 7 different dopamine-controlling genes. Although everyone agrees that much of our personality is inherited, no researchers had found genes as strongly associated as this when they studied the standard personality traits, such as introversion, conscientiousness, or agreeableness. These researchers in China looked at high sensitivity instead, believing it to be more ``deeply rooted in the nervous system.''

Interestingly, it was combinations of the genetic variations that predicted the trait, \& the function of those variations are mostly unknown, so the genetics of personality will be very complicated to figure out. Also, for some reason, getting the same results again using the same methods is notoriously difficult with genetic studies; we will need to see more studies like these to be sure. Nevertheless, I feel even more confident that this is an inherited trait.'' -- \cite[pp. 16--17]{Aron2013}

\subsection*{We Do Exist As a Distinct Set of People}
``Although I said in this book that usually you are either highly sensitive or not, I had no direct evidence for that point either. I assumed it because Jerome Kagan of Harvard found it true for the trait of inhibitedness in children, \& that seemed to be an understandable misnomer for sensitivity, given that it was based on observing children who do ot rush into a room full of complicated, strange toys, but pause to look at it 1st. But many scientists thought sensitivity must be more like height, with most people in the middle. For the doctoral thesis at the University of Bielefeld in Germany, Franziska Borries did a particular statistical analysis that distinguishes between categories \& dimensions in a study of over 900 people who took the HSP Scale. She found that being highly sensitive is indeed a category, not a dimension. Mostly, you either are or you are not.

It's difficult to know the exact percentage in any given population, as there will always be reasons why there might be more or less than the average of 15--20\%. Plus, many factors affect how a person scores, so that some people will score in the middle for other reasons. Perhaps some people just rate everything lower than others, or some may be distracted on the day they take the scale, or whatever. Also, men tend to score lower even though we know just as many are born with the trait. Somehow taking the test seems to affect men differently. Still, most people are not in the middle, but either have the trait or do not.'' -- \cite[pp. 17--18]{Aron2013}

\subsection*{DOES Describes It}
``When I wrote \textit{Psychotherapy \& the Highly Sensitive Person} in 2011 (to help therapists understand us better, \& especially that our trait is not an illness or flaw), I created the acronym I already mentioned in order to help therapists assess for this trait. I've come to like it as a way of describing both us \& the research about us.'' -- \cite[p. 18]{Aron2013}

\subsubsection*{D is for Depth of Processing}
``At the foundation of the trait of high sensitivity is the tendency to process information more deeply. When people are given a phone number \& have no way to write it down, they will probably try to process it in some way so as to remember it, such as by repeating it many times, thinking of patterns or meanings in the digits, or noticing the numbers' similarity to something else. If you don't process it in some way you know you will forget it. HSPs simply process everything more, relating \& comparing what they notice to their past experience with other similar things. They do it whether they are aware of it or not. When we decide without knowing how we came to that decision, we call this intuition, \& HSPs have good (but not infallible!) intuition. When you make a decision consciously, you may notice that you are slower than others because you think over all the options so carefully. That's depth of processing, too.

Studies supporting the depth of processing aspect of the trait have compared the brain activation of sensitive \& nonsensitive people doing various perceptual tasks. Research by Jadzia Jagiellowicz found that the highly sensitive use more of those parts of the brain associated with ``deeper'' processing of information, especially on tasks that involve noticing subtleties. In another study, by ourselves \& others, sensitive \& nonsensitive people were given perceptual tasks that were already known to be difficult (require more brain activation or effort), depending on the culture a person is from. The nonsensitive people showed the usual difficulty, but the highly sensitive subjects' brains apparently did no have this difficulty, regardless of their culture. It was as if they found it natural to look beyond their cultural expectations to how things ``really are.''

Research by Bianca Acevedo \& her associates has shown more brain activation in HSPs than others in an area called the \textit{insula}, a part of the brain that integrates moment-to-moment knowledge of inner states \& emotions, bodily position, \& outer events. Some have called it the seat of consciousness. If we are more aware of what is going around inside \& outside, this would be exactly the result one would expect.'' -- \cite[pp. 18--19]{Aron2013}

\subsubsection*{O is for Overstimulation}
``If you are going to notice every little thing in a situation, \& if the situation is complicated (many things to remember), intense (noisy, cluttered, etc.), or goes on too long (a 2-hour commute), it seems obvious that you will also tend to wear out sooner from having to process so much. Others, not noticing as much as you have (or any of it), will not tire as quickly. They may even think it quite strange that you find it too much to sightsee all day \& go to a nightclub in the evening. They might talk blithely on when you need them to be quiet a moment so that you can have sometime just to think, or they might enjoy an ``energetic'' restaurant or a party when you can hardly bear the noise. Indeed this is often the behavior we \& others have noticed most -- that HSPs are easily stressed by overstimulation (including social stimulation), or having learned their lesson, that they avoid intense situations more than others do.

A recent study by Friederike Gerstenberg in Germany compared sensitive \& nonsensitive people on a task of deciding whether or not a T turned in various ways was hidden among a great many Ls turned various ways on a computer screen. HSPs were faster \& more accurate, but also more stressed than others after doing the task. Was it the perceptual effort or the emotional effect of being in the experiment? Whatever the reason, they were feeling stressed. Just as we say a piece of metal shows stress when it is overloaded, so do we.

High sensitivity however, is not mainly about being distressed by high levels of stimuli, as some have suggested, although that naturally happens when too much comes at us. Be careful not to mix up being an HSP with some problem condition: Sensory discomfort can by itself be a sign of disorder due to problems with sensory processing rather than having unusually good sensory processing. E.g., sometimes persons with autistic spectrum disorders complain of sensory overload, but at other times they underreact. Their problem seems to be a difficulty recognizing where to focus attention \& what to ignore. When speaking with someone, they may find the person's face no more important to look at than the pattern on the floor or the type of light-bulbs in the room. Naturally they can complain intensely about being overwhelmed by stimulation. They may even be more aware of subtleties, but in social situations, especially they more often notice something irrelevant, whereas HSPs would be paying more attention to subtle facial expressions, at least when not overaroused.'' -- \cite[pp. 19--21]{Aron2013}

\subsubsection*{E is for Emotional Reactivity}
``A series of studies done by Jadzia Jagiellowicz found that HSPs particularly react more than non-HSPs to pictures with a ``positive valence.'' (Data from surveys \& experiments had already found some evidence that HSPs react more to both positive \& negative experiences.) This was even more the case if they had had a good childhood. In her studies of the brain, this reaction to positive pictures was not only in the areas associated with the initial experience of strong emotions, but also in ``higher'' areas of thinking \& perceiving, i.e., in some of the same areas of those found in the depth-of-processing brain studies. This stronger reaction to positive pictures being even more enhanced by a good childhood fits with a new concept suggested by Michael Pluess \& Jay Belsky, the idea of ``vantage sensitivity'' which they created in order to highlight the specific potential for sensitive people to benefit from positive circumstances \& interventions.

\textit{E} is also for \textit{empathy}. In another study by Bianca Acevedo, sensitive \& nonsensitive persons looked at photos of both strangers \& loved ones expressing happiness, sadness, or a neutral feeling. In all situations, when there was emotion in the photo, sensitive persons showed increased activation in the insula but also more activity in their \textit{mirror neuron} system, especially when looking at the happy faces of loved ones. The brain's mirror neurons were only discovered in the last 20 years or so. When we watch someone else do something or feel something, this clump of neurons fires in the same way as some of the neurons in the person we are observing. As an example, the same neurons fire, to varying degrees, whether we are kicking a soccer ball, see someone else kicking a soccer ball, hear the sound of someone kicking a soccer ball, or hear or say the word ``kick.''

Not only do these amazing neurons help us learn through imitation, but in conjunction with the other areas of the brain that were especially active for HSPs, they help us know others' intentions \& how they feel. Hence they are largely responsible for the universal human capacity for empathy. We do not just have an idea of how someone else feels; we actually feel that way ourselves to some extent. This is very familiar to sensitive people. Anyone's sad face tended to generate more activity in these mirror neurons in HSPs than others. When seeing photos of their loved ones being unhappy, sensitive persons also showed more activation in areas suggesting they wanted to do something, to act, even more than in areas involving empathy (perhaps we learn to cool down our intense empathy in order to help). But overall, brain activation indicating empathy was stronger in HSPs than non-HSPs when looking at photos of faces showing strong emotion of any type.

There is a common misunderstanding that emotions cause us to think illogically. But recent scientific thinking, reviewed by psychologist Roy Baumeister \& his colleagues, has placed emotion at the center of wisdom. 1 reason is that most emotion is felt after an event, which apparently serves to help us remember what happened \& learn from it. The more upset we are by a mistake, the more we think about it \& will be able to avoid it the next time. The more delighted we are by a success, the more we think \& talk about it \& how we did it, causing us to be more likely to be able to repeat it.

Other studies discussed by Baumeister, which explore the contribution of emotion to clear thinking, find that unless people have some emotional reason to learn something, they do not learn it very well or at all. This is 1 reason why it is easier to learn a foreign language in the country where it is spoken -- we are highly motivated to find our way, converse when spoken to, \& generally not seem foolish. From this point of view, it would seem almost impossible for a highly sensitive person to process things deeply without having stronger emotional reactions to motivate them. \& remember, when HSPs react more, it is as much or more to positive emotions, such as curiosity, anticipation of success (using that short cut others don't know about), a pleasant desire for something, satisfaction, joy, contentedness. It may be that everyone reacts strongly to negative situations, but HSPs seem to have evolved so that we especially relish a good outcome \& figure out more than others do how to make it happen. I imagine that we can plan an especially good birthday celebration, anticipating the happiness it will bring.'' -- \cite[pp. 21--23]{Aron2013}

\subsubsection*{S is for Sensing the Subtle}
``Most of the studies already cited required perceiving subtleties. This is often what is most noticeable to us personally, the little things we notice that others miss. Given that, \& because I called the trait high sensitivity, many have thought this is the heart of the trait. (To correct this confusion \& emphasize the role of processing, we used ``sensory \textit{processing} sensitivity'' as its more formal scientific designation.) However, this trait is not so much about extraordinary senses -- after all, there are sensitive people who have poor eyesight or hearing. True, some sensitive people report that 1 or more senses are very acute, but even in these cases it could be that they process the sensory information more carefully rather than having something unusual about their eyes, nose, skin, taste buds, or ears. Again, the brain areas that are more active when sensitive people perceive are those that do the more complex processing of sensory information: not so much the areas that recognize alphabet letters by their shape or even that read words, but the areas that catch the subtle meaning of words.

On the 1 hand, our awareness of subtleties is useful in an infinite number of ways, from simple pleasure in life to strategizing our response based on our awareness of others' nonverbal cues (that they may have no idea they are giving off) about their mood or trustworthiness. On the other hand, of course, when we are worn out we may be the least aware of anything, subtle or gross, except our own need for a break. This brings us to an important point.'' -- \cite[p. 23]{Aron2013}

\subsection*{Every Highly Sensitive Person Is Different, \& Different at Different Times}
``DOES is a wonderful general guideline for understanding high sensitivity but it is not infallible. Depending on how we are feeling, we may not be reflecting on our behavior or noticing subtleties even as much as the non-HSPs around us. We also differ from each other. People have other traits, different histories, \& are just different. In our enthusiasm to identify ourselves as a group -- even as a misunderstood minority -- we do not want to forget that we are not identical by any means. In particular, we are not all, or all the time, aware, conscientious, wonderful people!

Take \textit{O for easily overstimulated}. 2 sensitive people may behave quite differently when being bothered by loud noise or rude, upsetting behavior by others. One may rarely complain or be visibly bothered by such things because this person avoids such situations or quietly exits them. He or she will not, e.g., stay in a job if noise, rudeness, or other annoyances are present. If this HSP cannot escape the problems, he or she quietly tolerates them until they can be corrected. Other HSPs, usually with a more stressful past, will feel more victimized \& upset, \& at the same time be less able to place themselves in the right environments \& avoid the wrong ones. Maybe they feel they have to please others or prove something. In the workplace, they may not quit a job until a crisis occurs so that everyone working there knows about their ``over'' sensitivity.

A study done by Bhavini Shrivastava of HSPs in an information technology firm in India found that they felt more stressed than others by their work environment, but were actually seen as more productive than others by their managers. If we assume that those HSPs whose performance had suffered from stress had already quit or been let go, the remaining HSPs (who were older \& longer on the job) apparently were quietly adapting, perhaps with special considerations from their supervisors, \& contributing their depth of processing \& awareness of subtleties to their company. So we see 2 (or more) types of HSPs -- able to manage or not, due to other facets of their personality. Or in other instances, 2 (or more types) of situations: a little stressful, s.t. HSPs in that situation seem like strong people who find ways to adapt that others miss; or hopelessly stressful, s.t. they cannot adapt \& seem weak.'' -- \cite[pp. 24--25]{Aron2013}

\subsection*{Final Thoughts}
``Studying high sensitivity has been an amazing journey for me. It began with a simple curiosity about something someone else said about me. I did some interviews of people who thought they might be highly sensitive just to see what it was, with no further research plans \& definitely no intention of writing a book for the public. Then, as I like to put it, I found I was walking down a street \& a parade began to form behind me, a parade of people who were highly sensitive \& had never heard the term before.

Over \& over I am asked, ``How could you discover a new trait?'' The answer is that sensitivity is not new but just difficult to observe by watching how people behave, which is usually how psychology proceeds. Hence psychologists \& people in general were coming up with names for the trait that were close but not precise, such as \textit{shyness} \& \textit{introversion}. We make it especially hard for others to observe our trait because we are so responsive to our environments that we can be something like chameleons when around others, doing whatever it takes to fit in. I happened to be in the position to be both a curious scientist \& a highly sensitive person, who could know this experience from the inside. Still, as I said in the original preface, even for me to focus on my own sensitivity required someone else to comment on it in me 1st, after I had an ``over'' reaction to a medical procedure.

When we are visible, the most obvious thing we do is ``over'' react compared to others -- the \textit{O} of being overstimulated \& the \textit{E} of stronger emotional reactions. But then we are a minority, so of course we are above average here \& not reacting as most people do. It's the more noticeable \textit{O} \& \textit{E} that have made it seem to ourselves \& others that we have a flaw. Further, those HSPs with a troubled past have less control over their reactions, \& hence the trait becomes associated with people having difficulties. The few observable things we do that would indicate \textit{D} \& \textit{S}, depth of processing \& awareness of subtleties, can easily be overlooked or misunderstood. E.g., if we are seen taking our time before entering a situation or making a decision, that can seem, again, to be different, a potential problem, \& therefore a flaw. It is easy to overlook how good those decisions can be when finally made. Further, this sort of slowness can be caused by many things besides sensitivity, such as fear or even low intelligence. It's what's going on inside, out of sight, that most clearly sorts the highly sensitive minority from others. Thank goodness for these new ways of doing brain research that show these differences \& for all of you who have stepped forward \& said, yes, that's what goes on inside of me, too.

So let's celebrate! Maybe with a parade!'' -- \cite[pp. 25--26]{Aron2013}

%------------------------------------------------------------------------------%

\section*{Preface}
``Cry baby!''

``Scaredy-cat!''

``Don't be a spoilsport!''

Echoes from the past? \& how about this well-meaning warning: ``You're just too sensitive for your own good.''

If you were like me, you heard a lot of that, \& it made you feel there must be something very different about you. I was convinced that I had a fatal flaw that I had to hide \& that doomed me to a 2nd-rate life. I thought there was something wrong with me.

In fact, there is something very right with you \& me. If you answered true to 12 or more of the questions on the self-test at the beginning of this book, or if the detailed description in Chap. 1 seems to fit you (really the best test), then you are a very special type of human being, a highly sensitive person -- which hereafter we'll call an HSP. \& this book is just for you.

\textit{Having a sensitive nervous system is normal, a basically neutral trait}. You probably inherited it. It occurs in about 15--20\% of the population. It means you are aware of subtleties in your surroundings, a great advantage in many situations. It also means you are more easily overwhelmed when you have been out in a highly stimulating environment for too long, bombarded by sights \& sounds until you are exhausted in a nervous-system sort of way. Thus, being sensitive has both advantages \& disadvantages.

In our culture, however, possessing this trait is not considered ideal \& that fact probably has had a major impact on you. Well-meaning parents \& teachers probably tried to help you ``overcome'' it, as if it were a defect. Other children were not always as nice about it. As an adult, it has probably been harder to find the right career \& relationships \& generally to feel self-worth \& self-confidence.'' -- \cite[p. 27]{Aron2013}

\subsection*{What This Book Offers You}
``This book provides basic, detailed information you need about your trait, data that exist nowhere else. It is the product of 5 years of research, in-depth interviews, clinical experience, courses \& individual consultations with hundreds of HSPs, \& careful reading between he lines of what psychology has already learned about the trait but does not realize it knows. In the 1st 3 chapters you will learn all the basic facts about your trait \& how to handle overstimulation \& overarousal of your nervous system.

Next, this book considers the impact of your sensitivity on your personal history, career, relationships, \& inner life. It focuses on the advantages you may not have thought of, plus it gives advice about typical problems some HSPs face, such as shyness or difficulty finding the right sort of work.

It is quite a journey we'll take. Most of the HSPs I've helped with the information that is in this book have told me that it has dramatically changed their lives -- \& they've told me to tell you that.'' -- \cite[p. 28]{Aron2013}

\subsection*{A Word to the Sensitive-But-Less-So}
``1st, if yu have picked up this book because you're the parent, spouse, or friend of an HSP, then you're especially welcome here. Your relationship with your HSP will be greatly improved.

2nd, a telephone survey of 300 randomly selected individuals of all ages found that while 20\% were extremely or quite sensitive, another 22\% were moderately sensitive. Those of you who fall into this moderately sensitive category will also benefit from this book.

By the way, 42\% said they were not sensitive at all -- which suggests why the highly sensitive can feel so completely out of step with a large part of the world. \& naturally, it's that segment of the population that's always turning up the radio or honking their horns.

Further, it is safe to say that everyone can become highly sensitive at times -- e.g., after a month alone in a mountain cabin. \& everyone becomes more sensitive as they age. Indeed, most people, whether they admit it or not, probably have a highly sensitive facet that comes to the fore in certain situations.'' -- \cite[p. 29]{Aron2013}

\subsection*{\& Some Things to Say to Non-HSPs}
``Sometimes non-HSPs feel excluded \& hurt by the idea that we are different from them \& maybe sound like we think we are somehow better. They say, ``Do you mean I'm not sensitive?'' 1 problem is that ``sensitive'' also means being understanding \& aware. Both HSPs \& non-HSPs can have these qualities, which are optimized when we are feeling good \& alert to the subtle. When very calm, HSPs may even enjoy the advantage of picking up more delicate nuances. When overaroused, however, a frequent state for HSPs, we are anything but understanding or sensitive. Instead, we are overwhelmed, frazzled, \& need to be alone. By contrast, your non-HSP friends are actually more understanding of others in highly chaotic situations.

I thought long \& hard about what to call this trait. I knew I didn't want to repeat the mistake of confusing it with introversion, shyness, inhibitedness, \& a host of other misnomers laid on us by other psychologists. None of them captures the neutral, much less the positive, aspects of the trait. ``Sensitivity'' does express the neutral fact of greater receptivity to stimulation. So it seemed to be time to make up for the bias against HSPs by using a term that might be taken in our favor.

On the other hand, being ``highly sensitive'' is anything but positive to some. While sitting in my quiet house writing this, at a time when no one is talking about the trait, I'll go on record: This book will generate more than its share of hurtful jokes \& comments about HSPs. There is tremendous collective psychological energy around the idea of being sensitive -- almost as much as around gender issues, with which sensitivity is often confused. (There are as many male as female babies born sensitive; but men are not supposed to possess the trait \& women are. Both genders pay a high price for that confusion.) So just be prepared for that energy. Protect both your sensitivity \& your newly budding understanding of it by not talking about it at all when that seems most prudent.

Mostly, enjoy knowing that there are also many like-minded people out there. We have not been in touch before. But we are now, \& both we \& our society will be the better for it. In Chaps. 1, 5, \& 10, I will comment at some length on the HSP's important social function.'' -- \cite[pp. 29--30]{Aron2013}

\subsection*{What You Need}
``I have found that HSPs benefit a fourfold approach, which the chapters in this book will follow.
\begin{enumerate}
	\item \textit{Self-knowledge.} You have to understand what it means to be an HSP. Thoroughly. \& how it fits with your other traits \& how your society's negative attitude has affected you. Then you need to know your sensitive body very well. No more ignoring your body because it seems too uncooperative or weak.
	\item \textit{Reframing.} You must actively reframe much of your past in the light of knowing you came into the world highly sensitive. So many of your ``failures'' were inevitable because neither you nor your parents \& teachers, friends \& colleagues, understood you. Reframing how you experienced your past can lead to solid self-esteem, \& self-esteem is especially important for HSPs, for it decreases our overarousal in new (\& therefore highly stimulating) situations.
	
	Reframing is not automatic, however. That is why I include ``activities'' at the end of each chapter that often involve it.
	\item \textit{Healing.} If you have not yet done so, you must begin to heal the deeper wounds. You were very sensitive as a child; family \& school problems, childhood illnesses, \& the like all affected you more than others. Furthermore, you were different from other kids \& almost surely suffered for that.
	
	HSPs especially, sensing the intense feelings that must arise, may hold back from the inner work necessary to heal the wounds from the past. Caution \& slowness are justified. But you will cheat yourself if you delay.
	\item \textit{Help With Feeling Okay When Out in the World \& Learning When to Be Less Out.} You can be, should be, \& need to be involved in the world. It truly needs you. But you have to be skilled at avoiding overdoing or underdoing it. This book, free of the confusing messages from a less sensitive culture, is about discovering that way.
\end{enumerate}
I will also teach you about your trait's effect on your close relationships. \& I'll discuss psychotherapy \& HSPs -- which HSPs should be in therapy \& why, what kind, with whom, \& especially how therapy differs for HSPs. Then I'll consider HSPs \& medical care, including plenty of information on medications like Prozac, often taken by HSPs. At the end of this book we will savor our rich inner life.'' -- \cite[pp. 30--31]{Aron2013}

\subsection*{About Myself}
``I am a research psychologist, university professor, psychotherapist, \& published novelist. What matters most, however, is that I am an HSP like you. I am definitely not writing from on high, aiming down to help you, poor soul, overcome your ``syndrome.'' I know personally about \textit{our} trait, its assets \& its challenges.

As a child, at home, I hid from the chaos in my family. At school I avoided sports, games, \& kids in general. What a mixture of relief \& humiliation when my strategy succeeded \& I was totally ignored.

In junior high school an extrovert took me under her wing. In high school that relationship continued, plus I studied most of the time. In college my life became far more difficult. After many stops \& starts, including a 4-year marriage undertaken too young, I finally graduated Phi Beta Kappa from the University of California at Berkeley. But I spent my share of time crying in rest rooms, thinking I was going crazy. (My research has found that retreating like this, often to cry, is typical of HSPs.)

In my 1st try at graduate school I was provided with an office, to which I also retreated \& cried, trying to regain some calm. Because of such reactions, I stopped my studies with a master's degree, even though I was highly encouraged to continue for a doctorate. It took 25 years for me to gain the information about my trait that made it possible to understand my reactions \& so complete that doctorate.

When I was 23, I met my current husband \& settled down into a very protected life of writing \& rearing a son. I was simultaneously delighted \& ashamed of not being ``out there.'' I was vaguely aware of my lost opportunities to learn, to enjoy more public recognition of my abilities, to be more connected with all kinds of people. But from bitter experienced I thought I had no choice.

Some arousing events, however, cannot be avoided. I had to undergo a medical procedure from which I assumed I would recover in a few weeks. Instead, for months my body seemed to resound with physical \& emotional reactions. I was being forced to face once again that mysterious ``fatal flaw'' of mine that made me so different. So I tried some psychotherapy. \& got lucky. After listening to me for a few sessions, my therapist said, ``But of course you were upset; you are a very highly sensitive person.''

What is this, I thought, some excuse? She said she had never thought much about it, but from her experience it seemed that there were real differences in people's tolerance for stimulation \& also their openness to the deeper significance of an experience, good \& bad. To her, such sensitivity was hardly a sign of a mental flaw or disorder. At least she hoped not, for she was highly sensitive herself. I recall her grin. ``As are most of the people who strike me as really worth knowing.''

I spent several years in therapy, none of it wasted, working through various issues from my childhood. But the central theme became the impact of this trait. There was my sense of being flawed. There was the willingness of others to protect me in return for enjoying my imagination, empathy, creativity, \& insight, which I myself hardly appreciated. \& there was my resulting isolation from the world. But as I gained insight, I was able to reenter the world. I take great pleasure now in being part of things, a professional, \& sharing the special gifts of my sensitivity.'' -- \cite[pp. 31--33]{Aron2013}

\subsection*{The Research Behind This Book}
``As knowledge about my trait changed my life, I decided to read more about it, but there was almost nothing available. I thought the closest topic might be introversion. The psychiatrist Carl Jung wrote very wisely on the subject, calling it a tendency to turn inward. The work of Jung, himself an HSP, has been a major help to me, but the more scientific work on introversion was focused on introverts not being sociable, \& it was that idea which made me wonder if introversion \& sensitivity were being wrongly equated.

With so little information to go on, I decided to put a notice in a newsletter that went to the staff of the university where I was teaching at the time. I asked to interview anyone who felt they were highly sensitive to stimulation, introverted, or quick to react emotionally. Soon I had more volunteers than I needed.

Next, the local paper did a story on the research. Even though there was nothing said in the article about how to reach me, over a hundred people phoned \& wrote me, thanking me, wanting help, or just wanting to say, ``Me, too.'' 2 years later, people were still contacting me. (HSPs sometimes think things over for a while before making their move!)

Based on the interviews (40 for 2--3 hours each), I designed a questionnaire that I have distributed to thousands all over North America. \& I directed a random-dialing telephone survey of 300 people as well. The point that matters for you is that everything in this book is based on solid research, my own or that of others. Or I am speaking from my repeated observations of HSPs, from my courses, conversations, individual consultations, \& psychotherapy with them. These opportunities to explore the personal lives of HSPs have numbered in the thousands. Even so, I will say ``probably'' \& ``maybe'' more than you are used to in books for the general reader, but I think HSPs appreciate that.

Deciding to do all of this research, writing, \& teaching has made me a kind of pioneer. But that, too, is part of being an HSP. We are often the 1st ones to see what needs to be done. As our confidence in our virtues grows, perhaps more \& more of us will speak up -- in our sensitive way.'' -- \cite[p. 33]{Aron2013}

\subsection*{Instructions to the Reader}
\begin{enumerate}
	\item ``Again, I address the reader as an HSP, but this book is written equally for someone seeking to understand HSPs, whether as a friend, relative, advisor, employer, educator, or health professional.
	\item This book involves seeing yourself as having a trait common to many. I.e., it labels you. The advantages are that you can feel normal \& benefit from the experience \& research of others. But any label misses your uniqueness. HSPs are each utterly different, even with their common trait. Please remind yourself of that as you proceed.
	\item While you are reading this book, you will probably see everything in your life in light of being highly sensitive. That is to be expected. In fact, it is exactly the idea. Total immersion helps with learning any new language, including a new way of talking about yourself. If others feel a little concerned, left out, or annoyed, ask for their patience. There will come a day when the concept will settle in \& you'll be talking about it less.
	\item This book includes some activities which I have found useful for HSPs. But I'm not going to say that you must do them if you want to gain anything from this book. Trust your HSP intuition \& do what feels right.
	\item Any of the activities could bring up strong feelings. If that happens, I do urge you to seek professional help. If you are now in therapy, this book should fit well with your work there. The ideas here might even shorten the time you will need therapy as you envision a new ideal self -- not the culture's ideal but your own, someone you can be \& maybe already are. But remember that this book does not substitute for a good therapist when things get intense or confusing.
\end{enumerate}
This is an exciting moment for me as I imagine you turning the page \& entering into this new world of mine, of yours, of \textit{ours}. After thinking for so long that you might be the only one, it is nice to have company, isn't it?'' -- \cite[p. 34]{Aron2013}

%------------------------------------------------------------------------------%

\section*{\textit{Are You Highly Sensitive?} A Self-Test}
``Answer each question according to the way you feel. Answer true if it is at least somewhat true for you. Answer false if it is not very true or not at all true for you.
\begin{enumerate}
	\item I seem to be aware of subtleties in my environment.
	\item Other people's moods affect me.
	\item I tend to be very sensitive to pain.
	\item I find myself needing to withdraw during busy days, into bed or into a darkened room or any place where I can have some privacy \& relief from stimulation.
	\item I am particularly sensitive to the effects of caffeine.
	\item I am easily overwhelmed by things like bright lights, strong smells, coarse fabrics, or sirens close by.
	\item I have a rich, complex inner life.
	\item I am made uncomfortable by loud noises.
	\item I am deeply moved by the arts or music.
	\item I am conscientious.
	\item I startle easily.
	\item I get rattled when I have a lot to do in a short amount of time.
	\item When people are uncomfortable in a physical environment I tend to know what needs to be done to make it more comfortable (like changing the lighting or the seating).
	\item I am annoyed when people try to get me to do too many things at once.
	\item I try hard to avoid making mistakes or forgetting things.
	\item I make it a point to avoid violent movies \& TV shows.
	\item I become unpleasantly aroused when a lot is going on around me.
	\item Being very hungry creates a strong reaction in me, disrupting my concentration or mood.
	\item Changes in my life shake me up.
	\item I notice \& enjoy delicate or fine scents, tastes, sounds, works of art.
	\item I make it a high priority to arrange my life to avoid upsetting or overwhelming situations.
	\item When I must compete or be observed while performing a task, I become so nervous or shaky that I do much worse than I would otherwise.
	\item When I was a child, my parents or teachers seemed to see me as sensitive or shy.
\end{enumerate}

\subsection*{Scoring Yourself}
If you answered true to 12 or more of the questions, you're probably highly sensitive.

But frankly, no psychological test is so accurate that you should base your life on it. If only 1 or 2 questions are true of you but they are extremely true, you might also be justified in calling yourself highly sensitive.

Read on, \& if you recognize yourself in the in-depth description of a highly sensitive person in Chap. I, consider yourself one. The rest of this book will help you understand yourself better \& learn to thrive in today's not-so-sensitive world.'' -- \cite[pp. 35--36]{Aron2013}

%------------------------------------------------------------------------------%

\section{The Facts About Being Highly Sensitive: A (Wrong) Sense of Being Flawed}
``In this chapter you will learn the basic facts about your trait \& how it makes you different from others. You will also discover the rest of your inherited personality \& have your eyes opened about your culture's view of you. But first you should meet Kristen.'' -- \cite[p. 37]{Aron2013}

\subsection{She Thought She Was Crazy}
``Kristen was the 23rd interview of my research on HSPs. She was an intelligent, clear-eyed college student. But soon into our interview her voice began to tremble.

``I'm sorry,'' she whispered. ``But I really signed up to see you because you're a psychologist \& I had to talk to someone who could tell me --'' Her voice broke. ``Am I \textit{crazy}?'' I studied her with sympathy. She was obviously feeling desperate, but nothing she had said so far had given me any sense of mental illness. But then, I was already listening differently to people like Kristen.

She tried again, as if afraid to give me time to answer. ``I feel so different. I always did. I don't mean -- I mean, my family was great. My childhood was almost idyllic until I had to go to school. Although Mom says I was always a grumpy baby.''

She took a breath. I said something reassuring, \& she plunged on. ``But in nursery school I was afraid of everything. Even music time. When they would pass out the pots \& pans to pound, I would put my hands over my ears \& cry.''

She looked away, her eyes glistening with tears now, too. ``In elementary school I was always the teacher's pet. Yet they'd say I was `spacey.'''

Her ``spaciness'' prompted a distressing series of medical \& psychological tests. 1st for mental retardation. As a result, she was enrolled in a program for the \textit{gifted}, which did not surprise me.

Still the message was ``Something is wrong with this child.'' Her hearing was tested. Normal. In 4th grade she had a brain scan on the theory that her inwardness was due to petit mal seizures. Her brain was normal.

The final diagnosis? She had ``trouble screening out stimuli.'' But the result was a child who believed she was defective.'' -- \cite[pp. 37--38]{Aron2013}

\subsection{Special But Deeply Misunderstood}
``The diagnosis was right as far as it went. HSPs do take in a lot -- all the subtleties others miss. But what seems ordinary to others, like loud music or crowds, can be highly stimulating \& thus stressful for HSPs.

Most people ignore sirens, glaring lights, strange odors, clutter \& chaos. HSPs are disturbed by them.

Most people's feet may be tired at the end of a day in a mall or a museum, but they're ready for more when you suggest an evening party. HSPs need solitude after such a day. They feel jangled, overaroused.

Most people walk into a room \& perhaps notice the furniture, the people -- that's about it. HSPs can be instantly aware, whether they wish to be or not, of the mood, the friendships \& enmities, the freshness or staleness of the air, the personality of the one who arranged the flowers.

If you are an HSP, however, it is hard to grasp that you have some remarkable ability. How do you compare inner experiences? Not easily. Mostly you notice that you seem unable to tolerate as much as other people. You forget that you belong to a group that has often demonstrated great creativity, insight, passion, \& caring -- all highly valued by society.

We are a package deal, however. Our trait of sensitivity means we will also be cautious, inward, needing extra time alone. Because people without the trait (the majority) do not understand that, they see us as timid, shy, weak, or that greatest sin of all, unsociable. Fearing these labels, we try to be like others. But that leads to our becoming overaroused \& distressed. Then \textit{that} gets us labeled neurotic or crazy, 1st by others \& then by ourselves.'' -- \cite[pp. 38--39]{Aron2013}

\subsection{Kristen's Dangerous year}
``Sooner or later everyone encounters stressful life experiences, but HSPs react more to such stimulation. If you see this reaction as part of some basic flaw, you intensify the stress already present in any life crisis. Next come feelings of hopelessness \& worthlessness.

Kristen, e.g., had such a crisis the year she started college. She had attended a low-key private high school \& had never been away from home. Suddenly she was living among strangers, fighting in crowds for courses \& books, \& always overstimulated. Next she feel in love, fast \& hard (as HSPs can do). Shortly after, she went to Japan to meet her boyfriend's family, an event she already had good reason to fear. It was while she was in Japan that, in her words, she ``flipped out.''

Kristen had never thought of herself as an anxious person, but suddenly, in Japan, she was overcome by fears \& could not sleep. Then she became depressed. Frightened by her own emotions, her self-confidence plummeted. Her young boyfriend could not cope with her ``craziness'' \& wanted to end the relationship. By then she had returned to school, but feared she was going to fail at that, too. Kristen was on the edge.

She looked up at me after sobbing out the last of her story. ``Then I heard about this research, about being sensitive, \& I thought, Could that be me? But it isn't, I know. Is it?''

I told her that of course I could not be sure from such a brief conversation, but I believed that, yes, her sensitivity in combination with all these stresses might well explain her state of mind. \& so I had the privilege of explaining Kristen to herself -- an explanation obviously long overdue.'' -- \cite[pp. 39--40]{Aron2013}

\subsection{Defining High Sensitivity -- 2 Facts to Remember}
\textbf{Fact 1: Everyone, HSP or not, feels best when neither too bored nor too aroused.} ``An individual will perform best on any kind of task, whether engaging in a conversation or playing in the Super Bowl, if his or her nervous system is moderately alert \& aroused. Too little arousal \& one is dull, ineffective. To change that underaroused physical state, we drink some coffee, turn on the radio, call a friend, strike up a conversation with a total stranger, change careers -- anything!

At the other extreme, too much arousal of the nervous system \& anyone will become distressed, clumsy, \& confused. We cannot think; the body is not coordinated; we feel out of control. Again, we have many ways to correct the situation. Sometimes we rest. Or mentally shut down. Some of us drink alcohol or take a Valium.

The best amount of arousal falls somewhere in the middle. That there is a need \& desire for an ``optimal level of arousal'' is, in fact, 1 of the most solid findings of psychology. It is true for everyone, even infants. They hate to feel bored or overwhelmed.

\textbf{Fact 2: People differ considerably in how much their nervous system is aroused in the same situation, under the same stimulation.} The difference is largely inherited, \& is very real \& normal. In fact, it can be observed in all higher animals -- mice, cats, dogs, horses, monkeys, humans. Within a species, the percentage that is very sensitive to stimulation is usually about the same, around 15--20\%. Just as some within a species are a little bigger in size than others, some are a little more sensitive. In fact, through careful breeding of animals, mating the sensitive ones to each other can create a sensitive strain in just a few generations. In short, among inborn traits of temperament, this one creates the most dramatic, observable differences.'' -- \cite[p. 40]{Aron2013}

\subsection{The Good News \& the No-So-Good}
``What this difference in arousability means is that you notice levels of stimulation that go unobserved by others. This is true whether we are talking about subtle sounds, sights, or physical sensations like pain. It is not that your hearing, vision, or other senses are more acute (plenty of HSPs wear glasses). The difference seems to lie somewhere on the way to the brain or in the brain, in a more careful processing of information. We reflect more on everything. \& we sort things into finer distinctions. Like those machines that grade fruit by size -- we sort into 10 sizes while others sort into 2 or 3.

This greater awareness of the subtle tends to make you more intuitive, which simply means picking up \& working through information in a semiconscious or unconscious way. The result is that you often ``just know'' without realizing how. Furthermore, this deeper processing of subtle details causes you to consider the past or future more. You ``just know'' how things got to be the way they are or how they are going to turn out. This is that ``6th sense'' people talk about. It can be wrong, of course, just as your eyes \& ears can be wrong, but your intuition is right often enough that HSPs tend to be visionaries, highly intuitive artists, or inventors, as well as more conscientious, cautious, \& wise people.

The downside of the trait shows up at more intense levels of stimulation. What is \textit{moderately} arousing for most people is highly arousing for HSPs. What is \textit{highly} arousing for most people causes an HSP to become very frazzled indeed, until they reach a shutdown point called ``transmarginal inhibition.'' Transmarginal inhibition was 1st discussed around the turn of the century by the Russian physiologist Ivan Pavlov, who was convinced that the most basic inherited difference among people was how soon they reach this shutdown point \& that the quick-to-shut-down have a fundamentally different type of nervous system.

No one likes being overaroused, HSP or not. A person feels out of control, \& the whole body warns that it is in trouble. Overarousal often means failing to perform at one's best. Of course, it can also mean danger. An extra dread of overarousal may even be built into all of us. Since a newborn cannot run or fight or even recognize danger, it is best if it howls at anything new, anything arousing at all, so that grown-ups can come \& rescue it.

Like the fire department, we HSPs mostly respond to false alarms. But if our sensitivity saves a life even once, it is a trait that has a genetic payoff. So, yes, when our trait leads to overarousal, it is a nuisance. But it is part of a package deal with many advantages.'' -- \cite[pp. 41--42]{Aron2013}

\subsection{More About Stimulation}
``Stimulation is anything that wakes up the nervous system, gets its attention, makes the nerves fire off another around of the little electrical charges that they carry. We usually think of stimulation as coming from outside, but of course it can come from our body (such as pain, muscle tension, hunger, thirst, or sexual feelings) or as memories, fantasies, thoughts, or plans.

Stimulation can vary in intensity (like the loudness of a noise) or in duration. It can be more stimulating because it is novel, as when one is startled by a honk or shout, or in its complexity, as when one is at a party \& hearing 4 conversations at once plus music.

Often we can get used to stimulation. But sometimes we think we have \& aren't being bothered, but suddenly feel exhausted \& realize why: We have been putting up with something at a conscious level while it was actually wearing us down. Even a moderate \& familiar stimulation, like a day at work, can cause an HSP to need quiet by evening. At that point, 1 more ``small'' stimulation can be the last straw.

Stimulation is even more complicated because the same stimulus can have different meanings for different people. A crowded shopping mall at Christmastime may remind 1 person of happy family shopping excursions \& create a warm holiday spirit. But another person may have been forced to go shopping with others, tried to buy gifts without enough money \& no idea of what to purchase, had unhappy memories of past holidays, \& so suffers intensely in malls at Christmas.

\textbf{VALUING YOUR SENSITIVITY: Think back to 1 or more times that your sensitivity has saved you or someone else from suffering, great loss, or even death. (In my own case, I \& all my family would be dead if I had not awakened at the 1st flicker of firelight in the ceiling of an old wooden house in which we were living.)}

1 general rule is that when we have no control over stimulation, it is more upsetting, even more so if we feel we are someone's victim. While music played by ourselves may be pleasant, heard from the neighbor's stereo, it can be annoying, \& if we have previously asked them to turn it down, it becomes a hostile invasion. This book may even increase your annoyance a bit as you begin to appreciate that you are a minority whose rights to have less stimulation are generally ignored.

Obviously it would help if we were enlightened \& detached from all of these associations so that nothing could arouse us. No wonder so many HSPs become interested in spiritual paths.'' -- \cite[pp. 42--43]{Aron2013}

\subsection{Is Arousal Really Different From Anxiety \& Fear?}
``It is important not to confuse arousal with fear. Fear creates arousal, but so do many other emotions, including joy, curiosity, or anger. But we can also be overaroused by semiconscious thoughts or low levels of excitement that create no obvious emotion. Often we are not aware of what is arousing us, such as the newness of a situation or noise or the many things our eyes are seeing.

Actually, there are several ways to \textit{be} aroused \& still other ways to \textit{feel} aroused, \& they differ from time to time \& from person to person. Arousal may appear as blushing, trembling, heart pounding, hands shaking, foggy thinking, stomach churning, muscles tensing, \& hands or other parts of the body perspiring. Often people in such situations are not aware of some or all of these reactions as they occur. On the other hand, some people say they feel aroused, but that arousal shows up very little in any of these ways. Still, the term does describe something that all these experiences \& physical states share. Like the word ``stress,'' arousal is a word that really communicates something we all know about, even if that something varies a lot. \& of course stress is closely related to arousal: Our response to stress is to become aroused.

Once we do notice arousal, we want to name it \& know its source in order to recognize danger. \& often we think that our arousal is due to fear. We do not realize that our heart may be pounding from the sheer effort of processing extra stimulation. Or other people assume we are afraid, given our obvious arousal, so we assume it, too. Then, deciding we must be afraid, we become even more aroused. \& we avoid the situation in the future when staying in it \& getting used to it might have calmed us down. We will discuss again the importance of not confusing fear and arousal in Chap. 5 when we talk about ``shyness.'''' -- \cite[pp. 43--44]{Aron2013}

\subsection{Your Trait Really Does Make You Special}
``There are many fruits growing from the trait of sensitivity. Your mind works differently. Please remember that what follows is \textit{on the average}; nobody has all these traits. But compared to non-HSPs, most of us are:
\begin{itemize}
	\item Better at spotting errors \& avoiding making errors.
	\item Highly conscientious.
	\item Able to concentrate deeply. (\textit{But we do best without distractions.})
	\item Especially good at tasks requiring vigilance, accuracy, speed, \& the detection of minor differences.
	\item Able to process material to deeper levels of what psychologists call ``semantic memory.'' Often thinking about our own thinking.\footnote{NQBH: Superthinking.}
	\item Able to learn without being aware we have learned.
	\item Deeply affected by other people's moods \& emotions.
\end{itemize}
Of course, there are many exceptions, especially to our being conscientious. \& we don't want to be self-righteous about this; plenty of harm can be done in the name of trying to do good. Indeed, all of these fruits have their bruised spots. We are so skilled, but alas, when being watched, timed, or evaluated, we often cannot display our competence. Our deeper processing may make it seem that at 1st we are not catching on, but with time we understand \& remember more than others. This may be why HSPs learn languages better (although arousal may make one less fluent than others when speaking).

By the way, thinking more than others about our own thoughts is not self-centeredness. It means that if asked what's on our mind, we are less likely to mention being aware of the world around us, \& more likely to mention our inner reflections or musings. But we are no less likely to mention thinking about other people.

Our bodies are different too. Most of us have nervous systems that make us:
\begin{itemize}
	\item Specialists in fine motor movements.
	\item Good at holding still.
	\item ``Morning people.'' (\textit{Here there are many exceptions.})
	\item More affected by stimulants like caffeine unless we are very used to them.
	\item More ``right-brained'' (less linear, more creative in a synthesizing way).
	\item More sensitive to things in the air. (\textit{Yes, that means more hay fever \& skin rashes.})
\end{itemize}
Overall, again, our nervous systems seem designed to react to subtle experiences, which also makes us slower to recover when we must react to intense stimuli.

But HSPs are not in a more aroused state all the time. We are not ``chronically aroused'' in day-to-day life or when asleep. We are just more aroused by new or prolonged stimulation. (Being an HSP is \textit{not} the same as being ``neurotic'' -- i.e., constantly anxious for no apparent reason).'' -- \cite[pp. 44--45]{Aron2013}

\subsection{How to Think About Your Differences}
``I hope  that by now you are seeing your trait in positive terms. But I really suggest trying to view it as neutral. It becomes an advantage or disadvantage only when you enter a particular situation. Since the trait exists in all higher animals, it must have value in many circumstances. My hunch is that it survives in a certain percentage of all higher animals because it is useful to have at least a few around who are always watching for subtle signs. 15--20\% seems about the right proportion to have always on the alert for danger, new foods, the needs of the young \& sick, \& the habits of other animals.

Of course, it is also good to have quite a few in a group who are not so alert to all the dangers \& consequences of every action. They will rush out without a whole lot of thought to explore every new thing or fight for the group or territory. Every society needs both. \& maybe there is a need for more of the \textit{less} sensitive because more of them tend to get killed! This is all speculation, of course.

Another hunch of mine, however, is that the human race benefits more from HSPs than do other species. HSPs do more of that which makes humans different from other animals: We imagine possibilities. We humans, \& HSPs especially, are acutely aware of the past \& future. On top of that, if necessity is the mother of invention, HSPs must spend far more time trying to invent solutions to human problems just because they are more sensitive to hunger, cold, insecurity, exhaustion, \& illness.

Sometimes people with our trait are said to be less happy or less capable of happiness. Of course, we can seem unhappy \& moody, at least to non-HSPs, because we spend so much time thinking about things like the meaning of life \& death \& how complicated everything is -- not black-\&-white thoughts at all. Since most non-HSPs do not seem to enjoy thinking about such things, they assume we must be unhappy doing all that pondering. \& we certainly don't get any happier having them tell us we are unhappy (by \textit{their} definition of happy) \& that we are a problem for them because we seem unhappy. All those accusations could make \textit{anyone} unhappy.

The point is best made by Aristotle, who supposedly asked, ``Would you rather be a happy pig or an unhappy human?'' HSPs prefer the good feeling of being very conscious, very human, even if what we are conscious of is not always cause for rejoicing.

The point, however, is not that non-HSPs are pigs! I \textit{know} someone is going to say I am trying to make an elite out of us. But that would last about 5 minutes with most HSPs, who would soon feel guilty for feeling superior. I'm just out to encourage us enough to make more of us feel like equals.'' -- \cite[pp. 45--46]{Aron2013}

\subsection{Heredity \& Environment}
``Some of you may be wondering if you really inherited this trait, especially if you remember a time when your sensitivity seemed to begin or greatly increase.

In most cases, sensitivity is inherited. The evidence for this is strong, mainly from studies of identical twins who were raised apart but grew up behaving similarly, which always suggests that behavior is at least partly genetically determined.

On the other hand, it is not always true that both separated twins show the trait, even if they are identical. E.g., each twin will also tend to develop a personality quite like the mother raising that twin, even though she is not the biological mother. The fact is, there are probably no inherited traits that cannot also be enhanced, decreased, or entirely produced or eliminated by enough of certain kinds of life experiences. E.g., a child under stress at home or at school only needs to be born with a slight tendency to be sensitive \& he or she will withdraw. Which may explain why children who have older brothers \& sisters are more likely to be HSPs -- \& that would have nothing to do with genes. Similarly, studies of baby monkeys traumatized by separation from their mothers have found that these monkeys in adulthood behave much like monkeys born innately sensitive.

Circumstances can also force the trait to disappear. Many children born very sensitive are pushed hard by parents, schools, or friends to be bolder. Living in a noisy or crowded environment, growing up in a large family, or being made to be more physically active may sometimes reduce sensitivity, just as sensitive animals that are handled a great deal will sometimes lose some of their natural caution, at least with certain people or in specific situations. That the underlying trait is entirely gone, however, seems unlikely.'' -- \cite[pp. 46--47]{Aron2013}

\subsection{What About You?}
``It is difficult to know for any particular adult whether you inherited the trait or developed it during your life. The best evidence, though hardly perfect, is whether your parents remember you as sensitive from the time you were born. If it is easy to do so, ask them, or whoever was your caretaker, to tell you all about what you were like in the 1st 6 months of life.

Probably you will learn more if you do \textit{not} begin by asking if you were sensitive. Just ask what you were like as a baby. Often the stories about you will tell it all. After a while, ask about some typical signs of highly sensitive babies. Were you difficult about change -- about being undressed \& put into water at bath time, about trying new foods, about noise? Did you have colic often? Were you slow to fall asleep, hard to keep asleep, or a short sleeper, especially when you were overtired?

Remember, if your parents had no experience with other babies, they may not have noticed anything unusual at that age because they had no one to compare you to. Also, given all the blaming of parents for their children's every difficulty, your parents may need to convince you \& themselves that all was perfect in your childhood. If you want, you can reassure them that you know they did their best \& that all babies pose a few problems but that you wonder which problems you presented.

You might also let them see the questionnaire at the front of this book. Ask them if they or anyone else in your family has this trait. Especially if you find relatives with it on both sides, the odds are very good your trait is inherited.

But what if it wasn't or you aren't sure? It probably does not matter at all. What \textit{does} is that it is \textit{your} trait now. So do not struggle too long over the question. The next topic is far more important.'' -- \cite[pp. 47--48]{Aron2013}

\subsection{Learning About Our Culture -- What You Don't Realize WILL Hurt You}
``You \& I are learning to see our trait as a neutral thing -- use--ful in some situations, not in others -- but our culture definitely does not see it, or any trait, as neutral. The anthropologist Margaret Mead explained it well. Although a culture's newborns will show a broad range of inherited temperaments, only a narrow band of these, a certain type, will be the ideal.

'' -- \cite[pp. 47--48]{Aron2013}

%------------------------------------------------------------------------------%

\section{Digging Deeper}

%------------------------------------------------------------------------------%

\section{General Health \& Lifestyle for HSPs}

%------------------------------------------------------------------------------%

\section{Reframing Your Childhood \& Adolescence}

%------------------------------------------------------------------------------%

\section{Social Relationships}

%------------------------------------------------------------------------------%

\section{Thriving at Work}

%------------------------------------------------------------------------------%

\section{Close Relationships}

%------------------------------------------------------------------------------%

\section{Healing the Deeper Wounds}

%------------------------------------------------------------------------------%

\section{Medics, Medications, \& HSPs}

%------------------------------------------------------------------------------%

\section{Soul \& Spirit}

%------------------------------------------------------------------------------%

\section{Tips for Health-Care Professionals Working With Highly Sensitive People}

%------------------------------------------------------------------------------%

\section{Tips for Teachers Working With Highly Sensitive Students}

%------------------------------------------------------------------------------%

\section{Tips for Employers of Highly Sensitive People}

%------------------------------------------------------------------------------%

\section{Notes}

%------------------------------------------------------------------------------%

\section{About the Author}

%------------------------------------------------------------------------------%

\printbibliography[heading=bibintoc]
	
\end{document}