\documentclass{article}
\usepackage[backend=biber,natbib=true,style=alphabetic]{biblatex}
\addbibresource{/home/nqbh/reference/bib.bib}
\usepackage{tocloft}
\renewcommand{\cftsecleader}{\cftdotfill{\cftdotsep}}
\usepackage[colorlinks=true,linkcolor=blue,urlcolor=red,citecolor=magenta]{hyperref}
\usepackage{algorithm,algpseudocode,amsmath,amssymb,amsthm,float,graphicx,mathtools}
\usepackage{enumitem}
\setlist{leftmargin=4mm}
\allowdisplaybreaks
\numberwithin{equation}{section}
\newtheorem{assumption}{Assumption}[section]
\newtheorem{conjecture}{Conjecture}[section]
\newtheorem{corollary}{Corollary}[section]
\newtheorem{definition}{Definition}[section]
\newtheorem{example}{Example}[section]
\newtheorem{lemma}{Lemma}[section]
\newtheorem{notation}{Notation}[section]
\newtheorem{principle}{Principle}[section]
\newtheorem{problem}{Problem}[section]
\newtheorem{proposition}{Proposition}[section]
\newtheorem{question}{Question}[section]
\newtheorem{remark}{Remark}[section]
\newtheorem{theorem}{Theorem}[section]
\usepackage[left=1cm,right=1cm,top=5mm,bottom=5mm,footskip=4mm]{geometry}
\def\labelitemii{$\circ$}

\title{The Highly Sensitive Person: How to Thrive When the World Overwhelms You}
\author{Elaine N. Aron}
\date{\today}

\begin{document}
\maketitle
\tableofcontents

\begin{quotation}
	``Engaging, perceptive $\ldots$ suggests new paths for making sensitivity a blessing, not a handicap. A must-read.'' -- Philip G. Zimbardo, author of \textit{Shyness}
\end{quotation}

\section*{What readers are saying about Elaine Aron \& \textit{The Highly Sensitive Person} $\ldots$}

\begin{quotation}
	``I have just finished \textit{The Highly Sensitive Person} \& I can't thank you enough for writing such a wonderful book -- you put into clear, understandable words what I have always known about myself. As I read your book, I felt for the 1st time in my life that someone truly understood what it was like to go through life as a highly sensitive individual $\ldots$ Your book was the 1st that I have ever read that not only validated the traits of highly sensitive individuals but cast them as necessary for our society.'' -- M. C., Rockaway, NJ
	
	``1st, let me express my deep gratitude to you if I can. I have just finished reading your book $\ldots$ You have truly given me hope for a new life at the age of 52. I hardly know how to express the comfort \& joy I have received from you $\ldots$ Once again, thank you, thank you, thank you!'' -- J. M., New York, NY
	
	``I cannot thank you enough for the inner peace your book has given me!'' -- S. P., Sacramento, CA
	
	``This book has opened my eyes to the fact that I am not alone in my sensitivity \textit{\&} that it is OK to be this way $\ldots$ I've always felt that there was something wrong with me $\ldots$ It has given me tremendous insight $\ldots$ So thank you for your research \& your words of encouragement. They've both been a blessing.'' -- M. G., Belle River, Ontario (Canada)
	
	``I am writing to express my gratitude to Elaine Aron for her book, \textit{The Highly Sensitive Person}. I laughed \& cried, I felt known. I felt affirmed. It is not only `OK' to be highly sensitive, it is a gift. Thank you.'' -- L. H., Findlay, OH
	
	``Thank you for writing such a wonderful book.'' -- R. P., Norwalk, CA
	
	``$\ldots$ it really helped me understand myself a lot better.'' -- E. S., Westerville, OH
	
	``I can't remember the last time I sat down \& read a book from cover to cover in 1 day. It has really made me feel like a part of a larger group, \& not quite so weird after all $\ldots$ I am looking forward to reading this book again.'' -- K. J., San Francisco, CA
	
	``I loved the book!'' -- S. R., Springfield, MA
	
	``I just finished reading Elaine N. Aron's excellent book $\ldots$ The descriptions fit me perfectly! It was inspiring, informative, \& emotional.'' -- R. D., San Francisco, CA
	
	``I find Dr. Aron's book immensely valuable.'' -- L. J. W., Provo, UT
	
	``I have been trying to find out who I am \& what I can do. Many of the situations described in the book I find fit my situation $\ldots$ I wish I could send [it to] everyone I know \& have known.'' -- C. M., Riverside, CA
	
	``I just read your book \& it is \textit{extraordinary}! Absolutely the best \& most helpful of many I've read $\ldots$ You have done tremendous work \& I am so deeply touched by much of what you say.'' -- S. S., New York, NY
	
	``This book, \textit{The Highly Sensitive Person}, was a revelation to me.'' -- A. A., Tustin, CA
	
	``Your book $\ldots$ has helped me so much.'' -- A. B., Lethbridge, Alberta (Canada)
	
	``\textit{The Highly Sensitive Person} was a true revelation to me \& to several others I recommended the book to.'' -- D. R., Irvine, CA
	
	``Elaine Aron's book, \textit{The Highly Sensitive Person}, is the 1st ever to really speak to \textit{me}!'' -- M. J., Houston, TX
	
	``I have enjoyed reading your book, \textit{The Highly Sensitive Person}, \& find the information \& insights extremely valuable.'' -- M. F., Mountain View, CA
\end{quotation}

\begin{quotation}
	To Irene Bernadicou Pettit, Ph.D. -- being both poet \& peasant, she knew how to plant this seed \& tend it until it blossomed.
	
	To Art, who especially loves the flowers -- 1 more love we share.
\end{quotation}

\section*{Acknowledgments}
``I especially want to acknowledge all the highly sensitive person I interviewed. You were the 1st to come forward \& talk about what you had known very privately about yourself for a long time, changing yourselves from isolated individuals to a group to be respected. My thanks also to those who have come to my courses or seen me for a consultation or in psychotherapy. Every word of this book reflects what you all have taught me.

My many student research assistants -- too many to name -- also earn a big thanks, as do Barbara Kouts, my agent, \& Bruce Shostak, my editor at Carol, for their effort to see that this book reached all of you. Barbara found a publisher with vision; Bruce brought the manuscript into good shape, reining me in at all the right places but otherwise letting me run with it as I saw it.

It's harder to find words for my husband, Art. But here are some: Friend, colleague, supporter, beloved-thanks, with all my love.'' -- \cite[p. 6]{Aron2013}

\begin{quotation}
	``I believed in aristocracy, though -- if that is the right word, \& if a democrat may use it. Not an aristocracy of power $\ldots$ but $\ldots$ of the sensitive, the considerate $\ldots$ Its members are to be found in all nations \& classes, \& all through the ages, \& there is a secret understanding between them when they meet. They represent the true human tradition, the 1 permanent victory of our queer race over cruelty \& chaos. Thousands of them perish in obscurity, a few are great names. They are sensitive for others as well as themselves, they are considerate without being fussy, their pluck is not swankiness but the power to endure $\ldots$'' -- E. M. Forster, ``What I Believe,'' in \textit{2 Cheers for Democracy} 
\end{quotation}

%------------------------------------------------------------------------------%

\section*{Author's Note, 2012}
``In 1998, 3 years after this book was 1st published, I wrote a new preface for it titled ``A Celebration.'' It was an invitation for all of us to feel good about how many people had discovered they were highly sensitive \& found the book useful, \& that the idea was catching on in the scientific world. Now we can celebrate about 50 times more of the same. \textit{The Highly Sensitive Person} has been translated into 14 languages, from Swedish, Spanish, \& Korean to Hebrew, French, \& Hungarian. There have been articles about high sensitivity in many prominent media throughout  the world. In the U.S., that has included a feature in \textit{Psychology Today}, a shorter discussion in \textit{Time}, \& many women's \& health magazines such as \textit{O Magazine} as well as numerous health websites. There are ``HSP Gatherings'' \& courses on the subject in the United States \& Europe, plus YouTube videos, books, magazines, newsletters, \& websites \& all sorts of services exclusively highly sensitive persons -- most good \& some, well, not as good. Tens of thousands subscribe to my own newsletter, \textit{Comfort Zone}, at \url{hsperson.com}, where there are now hundreds of newsletter articles archived covering every aspect of being highly sensitive. We have come a long way.'' -- \cite[p. 10]{Aron2013}

\subsection*{3 Revisions, Right Here}
``Given that this book was written at the very beginning of a minor revolution, I have thought I should revise it. But when I look it over, there's not much I would change. It does the job well, with 3 exceptions. 1st, \& most important, I wanted to add the expanded scientific research. That's vital because it helps us all to trust that this trait is real, that what is in this book is real. This preface will update you on the research.

2nd, there is now a simple, comprehensive description of the trait, ``DOES,'' that expresses its facets nicely. \textit{D} is for depth of processing. Our fundamental characteristic is that we observe \& reflect before we act. We process everything more, whether we are conscious of it or not. \textit{O} is for being easily overstimulated, because if you are going to pay more attention to everything, you are bound to tire sooner. \textit{E} is for giving emphasis to our emotional reactions \& having strong empathy which among other things helps us notice \& learn. \textit{S} is for being sensitive to all the subtleties around us. I will say more about these when I discuss the research.

3rd, a smaller point can be taken care of right now -- the discussion in the book of antidepressants, which focused on Prozac. Medications for treating depression have proliferated since 1996, as have the pros \& cons about them. Do they damage the rest of the body? Are they just placebos for most people, making them feel good to the same degree as if they had been given a sugar pill? But what about many suicides they have surely prevented? Haven't they also improved the lives of people close to those who are no longer depressed? The arguments on both sides are still there, both worth understanding. Thankfully these are now all on the Internet somewhere (but stick to reading about scientific research -- skip the horror stories, on either side). So my basic advice is the same: Become very well informed; then decide for yourself. To form an opinion before you ever become depressed is preferable, because under certain circumstances highly sensitive people are genetically more susceptible to depression, \& it is a difficult decision when you are in the thick of it.

At this point, if you are not interested in the research on sensitivity you can stop reading or just skim. Perhaps you are the type who understands this trait intuitively or ``from the heart,'' with no need for the intellect. However, I imagine that you sometimes find that you have to satisfy others' skepticism or even hostility about your suggestion that you are highly sensitive \& you might like some toolds for handing such times, which research findings can provide.'' -- \cite[p. 10]{Aron2013}

\subsection*{The Research Since 1996}
``Not only has science verified so much of what's in this book (some of which was only based on my observations at the time), but the findings have gone far beyond what we knew when I wrote it. I have tried to keep what follows interesting, but with enough detail to satisfy those who really want to know. You can find the full methodology \& results by reading the articles themselves. I published a good summary of the theory \& research in 2012 \& a current list of studies can always be found at \url{www.hsperson.com}. \textit{Sensory processing sensitivity} is the scientific name I have given the trait (not at all the same as Sensory Processing Disorder or Sensory Integration Disorder, which, alas, was given a similar name). I should add that concepts very much like sensitivity are being studied by other researchers. If you are interested in this work, you can look up terms such as Biological Sensitivity to Context (Thomas Boyce, Bruce Ellis, \& others), Differential Susceptibility (Jay Belsky, Michael Pluess, etc), \& Orienting Sensitivity (D. Evans \& Mary Rothbart, etc.) \& find even more research, all done since \textit{The Highly Sensitive Person} was written.'' -- \cite[p. 11]{Aron2013}

\subsubsection*{The 1st Research}
``The very 1st published studies we did (myself \& my husband, who is unusually good at designing research) generated the Highly Sensitive Person (HSP) Scale in this book. This research was also intended to demonstrate that high sensitivity is not the same as introversion or ``neuroticism'' (professional jargon for a tendency to be depressed or excessively anxious). We were right; the trait was not the same. But it was strongly associated with neuroticism. I had a hunch why, \& our 2nd series of studies, published in 2005, verified it: HSPs with a troubled childhood are more at risk of becoming depressed, anxious, \& shy than non-sensitive people with a similar childhood; but those with good-enough childhoods were no more at risk than others. There was even some indication -- \& more since -- that they are better off than nonsensitive people with good childhoods, as if they are more affected by any environment. A later study by Miriam Liss \& others found the same result, mainly for depression. Remember this is ``on the average.'' Some sensitive people with good childhoods may still be depressed \& some of with poor childhoods will not be. Further, many other things besides childhood difficulties affect us. The level of stress one lives under is surely 1 large factor.

This interaction of the trait \& one's childhood environment explains the relatively strong association between neuroticism or negative feelings \& high sensitivity that we found in the 1st study. Roughly half of the questions on the HSP Scale tap negative feelings -- ``I am made uncomfortable $\ldots$'' ``I get rattled $\ldots$'' ``I am annoyed $\ldots$'' \& so forth. Since many HSPs have had difficult childhoods, often because no one understood their innate temperament, their persistent bad feelings due to the trait could cause them to feel even more uncomfortable, rattled, or annoyed in situations that bother all sensitive persons to some degree. This would have added to the overlap of high sensitivity \& neuroticism for a reason that has nothing to do with the trait itself. When we use the scale now, we have various ways of asking people how much negative emotion they feel generally \& take that into account statistically.

Unfortunately, quite a few clinical studies of the relationship between being highly sensitive \&, e.g., being anxious, stressed, or having communication phobias have not taken the role of ``nurture'' into account, making it seem that all HSPs have these  problems. Hence I will not describe that research here.'' -- \cite[pp. 11--12]{Aron2013}

\subsubsection*{Serotonin \& HSPs}
``This finding about the additional impact on HSPs of their childhood, good \& bad, adds a nice footnote to something I said in this book, in the chapter on doctors \& medications. I cited a study by Stephen Suomi about a minority of rhesus monkeys who are born with a trait that was originally called ``up tight'' because they were more affected by being raised under stressful conditions. Not only did they appear more depressed \& anxious, but like depressed humans, they had less serotonin available in their brains, what antidepressants correct. Serotonin is a chemical used in at least 17 places in the brain in order to move around information. as it turned out, these vulnerable monkeys had a genetic variation that results in lower levels of serotonin generally, \& these levels are further reduced by stress. Sensitive humans have the same genetic variation. Interestingly, the variation is only found in 2 primate species, humans \& rhesus monkeys, \& both are highly social \& able to adapt to a wide range of environments. Perhaps the highly sensitive members of a group are better able to notice the subtleties, such as which new foods can be safely eaten \& which dangers to avoid, allowing them to survive better in a new place.

There are many, many genetic variations in all of us -- hair, eye, \& skin color, e.g., or special abilities or certain phobias. Some of these variations appear to serve little purpose; others are useful or not (or even a disadvantage) depending on the environment. If you live where there are many poisonous snakes, having an innate fear of them could be an advantage, but perhaps become a problem if you want to be a science teacher.

Anyway, since I wrote the book \& explained about those monkeys, research done in Denmark by Cecilie Licht \& others suggests that HSPs have the same genetic variation. For years, research had only looked for low serotonin's association with depression, \& the results were highly inconsistent, probably because in some studies they had inadvertently included too many sensitive people with good childhoods for depression to show up.

There had to be some positive reason for so many people having what should be an evolutionary disadvantage, a ``tendency to depression.'' Now new research demonstrates that this genetic variation causing lower serotonin to be available in the brain also bestows benefits, such as improved memory of learned material, better decision making, \& better overall mental functioning, plus gaining even more positive mental health than others from positive life experiences. The same mental benefits are also found in rhesus monkeys with  the same genetic variation. Perhaps the best vindication for HSPs tired of being seen as weaklings or sick is a study by Suomi finding that rhesus monkeys with this trait, if raised by skilled mothers, were more likely to show ``developmental precocity'' resilience to stress, \& be leaders of their social groups.

In the same vein, a growing body of research by others suggests that some individuals are especially sensitive \& therefore more susceptible to their environment -- e.g., as children they are more affected by parenting, by teachers, \& by helpful interventions. What is the underlying trait that leads to this ``for better \textit{\&} for worse'' outcome for us?'' -- \cite[pp. 12--14]{Aron2013}

\subsection*{What Makes Us So Different?}
``As I wrote in this book, many species -- we now know it's over 100, so far, including fruit flies \& some fish species -- have a minority of individuals that are highly sensitive. Although obviously the trait leads to different behaviors depending on whether you are a fruit fly fish, bird, dog, deer, monkey or human, a general description of it would be that the minority who have inherited it have adopted a survival strategy of pausing to check, observe, \& reflect on or process what has been noticed before choosing an action. Slowness to act, however, is not the hallmark of the trait. When sensitive individuals see right away that their situation is like a past one, thanks to having learned so thoroughly from thinking it over, they can react to a danger or opportunity faster than others. For this reason, the most basic aspect of the trait -- the depth of processing -- has been difficult to observe. Without knowing about it, when someone paused before acting, others could only guess what was happening inside that person. Often HSPs were thought to be inhibited, shy, fearful, or introverted (in fact, 30\% of HSPs are actually extraverts, \& many introverts are not HSPs). Some HSPs accepted those labels, having no other explanation for their hesitancy. Indeed, feeling different \& flawed, some of us found the label ``shy, or fearful of social judgment'' self-fulfilling, as I describe in Chap. 5. Others knew they were different, but hid  it \& adapted, acting like the non-sensitive majority.

Understanding why we evolved as we did tells us much more about ourselves than I knew when I wrote this book. At that time I thought our sensitivity had evolved because the trait served the larger group, as sensitive individuals can sense a danger or opportunity that the others miss, while these others serve by doing something about it once they are alerted. This may still be partly true, but that may only be a side effect of the trait. The current explanation comes from a computer model done by biologists in the Netherlands. Max Wolf \& his colleagues were curious about how sensitivity might evolve, so they set up a situation using a computer program in order to exclude all other factors. Then they varied just a few things at a time \& watched to see what happened when they ran out the various possible situations \& strategies. They wanted to see if being highly responsive could be a successful enough trait to remain in a population -- traits that make us unsuccessful at life don't last long.

The sensitive strategy was tested by setting up the scenario in which they varied how much an individual, learning from Situation A, by being more sensitive to everything that happened there, was more successful in Situation B because of having that information (they also had to vary the amount of benefit that came with being successful in Situation B). The other extreme scenario was such that learning from Situation A provided no help in Situation B because the 2 had nothing to do with each other. The question was, under what conditions would you see the evolution of 2 types of individuals, one using the strategy of learning from experience \& one not? It turned out that there only had to be a small benefit for the 2 strategies to emerge, hence explaining why the 2 would exist in real people.

You might think that being sensitive is always an advantage, but many times it is not. Indeed, sensitivity only serves the individual if he or she is in the minority. If everyone were sensitive it would be no advantage, as when, if everyone knows a short cut \& uses it, there are so many making use of the information that it benefits no one. In short, sensitivity, or responsibility as these biologists also called it, involves paying more attention to details than others do, then using that knowledge to make better predictions in the future. Sometimes you are better off doing so, but other times your extra attention \& effort have no pay-off.

Sensitivity does have its costs, as you know. It really can be a waste of energy if what is happening now has nothing to do with your past experiences. Further, when a past experience was very bad, an HSP can overgeneralize \& avoid or feel anxious in too many situations, just because the new ones resemble in some small way the past bad one. The biggest cost to us of being highly sensitive, however, is that our nervous system can become overloaded. Everyone has a limit as to how much information or stimulation can be taken in before getting overloaded, overstimulated, overaroused, overwhelmed, \& just \textit{over}! We simply reach that point sooner than others. Fortunately, as soon as we get some downtime we recover nicely.'' -- \cite[pp. 14--16]{Aron2013}

\subsection*{It's Really in Our Genes}
``When I wrote the book, I said sensitivity is innate. I knew it had been found from birth in children, \& in animals where the genetics had been identified, you can selectively breed animals to be more sensitive. But I had no genetic research using the HSP Scale on which to base that claim. Now it exists. I already mentioned 1 study that found scores on the test were related to a variation in a gene known to affect the availability of serotonin in the brain. Chen \& his associates, working in China, took a different approach. Rather than looking at a specific gene with known properties, they looked at all of the gene variations (98 in all) affecting the amount of dopamine, another chemical necessary for the transmission of information, available in certain areas of the brain. They found the HSP Scale associated with 10 variations on 7 different dopamine-controlling genes. Although everyone agrees that much of our personality is inherited, no researchers had found genes as strongly associated as this when they studied the standard personality traits, such as introversion, conscientiousness, or agreeableness. These researchers in China looked at high sensitivity instead, believing it to be more ``deeply rooted in the nervous system.''

Interestingly, it was combinations of the genetic variations that predicted the trait, \& the function of those variations are mostly unknown, so the genetics of personality will be very complicated to figure out. Also, for some reason, getting the same results again using the same methods is notoriously difficult with genetic studies; we will need to see more studies like these to be sure. Nevertheless, I feel even more confident that this is an inherited trait.'' -- \cite[pp. 16--17]{Aron2013}

\subsection*{We Do Exist As a Distinct Set of People}
``Although I said in this book that usually you are either highly sensitive or not, I had no direct evidence for that point either. I assumed it because Jerome Kagan of Harvard found it true for the trait of inhibitedness in children, \& that seemed to be an understandable misnomer for sensitivity, given that it was based on observing children who do ot rush into a room full of complicated, strange toys, but pause to look at it 1st. But many scientists thought sensitivity must be more like height, with most people in the middle. For the doctoral thesis at the University of Bielefeld in Germany, Franziska Borries did a particular statistical analysis that distinguishes between categories \& dimensions in a study of over 900 people who took the HSP Scale. She found that being highly sensitive is indeed a category, not a dimension. Mostly, you either are or you are not.

It's difficult to know the exact percentage in any given population, as there will always be reasons why there might be more or less than the average of 15--20\%. Plus, many factors affect how a person scores, so that some people will score in the middle for other reasons. Perhaps some people just rate everything lower than others, or some may be distracted on the day they take the scale, or whatever. Also, men tend to score lower even though we know just as many are born with the trait. Somehow taking the test seems to affect men differently. Still, most people are not in the middle, but either have the trait or do not.'' -- \cite[pp. 17--18]{Aron2013}

\subsection*{DOES Describes It}
``When I wrote \textit{Psychotherapy \& the Highly Sensitive Person} in 2011 (to help therapists understand us better, \& especially that our trait is not an illness or flaw), I created the acronym I already mentioned in order to help therapists assess for this trait. I've come to like it as a way of describing both us \& the research about us.'' -- \cite[p. 18]{Aron2013}

\subsubsection*{D is for Depth of Processing}
``At the foundation of the trait of high sensitivity is the tendency to process information more deeply. When people are given a phone number \& have no way to write it down, they will probably try to process it in some way so as to remember it, such as by repeating it many times, thinking of patterns or meanings in the digits, or noticing the numbers' similarity to something else. If you don't process it in some way you know you will forget it. HSPs simply process everything more, relating \& comparing what they notice to their past experience with other similar things. They do it whether they are aware of it or not. When we decide without knowing how we came to that decision, we call this intuition, \& HSPs have good (but not infallible!) intuition. When you make a decision consciously, you may notice that you are slower than others because you think over all the options so carefully. That's depth of processing, too.

Studies supporting the depth of processing aspect of the trait have compared the brain activation of sensitive \& nonsensitive people doing various perceptual tasks. Research by Jadzia Jagiellowicz found that the highly sensitive use more of those parts of the brain associated with ``deeper'' processing of information, especially on tasks that involve noticing subtleties. In another study, by ourselves \& others, sensitive \& nonsensitive people were given perceptual tasks that were already known to be difficult (require more brain activation or effort), depending on the culture a person is from. The nonsensitive people showed the usual difficulty, but the highly sensitive subjects' brains apparently did no have this difficulty, regardless of their culture. It was as if they found it natural to look beyond their cultural expectations to how things ``really are.''

Research by Bianca Acevedo \& her associates has shown more brain activation in HSPs than others in an area called the \textit{insula}, a part of the brain that integrates moment-to-moment knowledge of inner states \& emotions, bodily position, \& outer events. Some have called it the seat of consciousness. If we are more aware of what is going around inside \& outside, this would be exactly the result one would expect.'' -- \cite[pp. 18--19]{Aron2013}

\subsubsection*{O is for Overstimulation}
``If you are going to notice every little thing in a situation, \& if the situation is complicated (many things to remember), intense (noisy, cluttered, etc.), or goes on too long (a 2-hour commute), it seems obvious that you will also tend to wear out sooner from having to process so much. Others, not noticing as much as you have (or any of it), will not tire as quickly. They may even think it quite strange that you find it too much to sightsee all day \& go to a nightclub in the evening. They might talk blithely on when you need them to be quiet a moment so that you can have sometime just to think, or they might enjoy an ``energetic'' restaurant or a party when you can hardly bear the noise. Indeed this is often the behavior we \& others have noticed most -- that HSPs are easily stressed by overstimulation (including social stimulation), or having learned their lesson, that they avoid intense situations more than others do.

A recent study by Friederike Gerstenberg in Germany compared sensitive \& nonsensitive people on a task of deciding whether or not a T turned in various ways was hidden among a great many Ls turned various ways on a computer screen. HSPs were faster \& more accurate, but also more stressed than others after doing the task. Was it the perceptual effort or the emotional effect of being in the experiment? Whatever the reason, they were feeling stressed. Just as we say a piece of metal shows stress when it is overloaded, so do we.

High sensitivity however, is not mainly about being distressed by high levels of stimuli, as some have suggested, although that naturally happens when too much comes at us. Be careful not to mix up being an HSP with some problem condition: Sensory discomfort can by itself be a sign of disorder due to problems with sensory processing rather than having unusually good sensory processing. E.g., sometimes persons with autistic spectrum disorders complain of sensory overload, but at other times they underreact. Their problem seems to be a difficulty recognizing where to focus attention \& what to ignore. When speaking with someone, they may find the person's face no more important to look at than the pattern on the floor or the type of light-bulbs in the room. Naturally they can complain intensely about being overwhelmed by stimulation. They may even be more aware of subtleties, but in social situations, especially they more often notice something irrelevant, whereas HSPs would be paying more attention to subtle facial expressions, at least when not overaroused.'' -- \cite[pp. 19--21]{Aron2013}

\subsubsection*{E is for Emotional Reactivity}
``A series of studies done by Jadzia Jagiellowicz found that HSPs particularly react more than non-HSPs to pictures with a ``positive valence.'' (Data from surveys \& experiments had already found some evidence that HSPs react more to both positive \& negative experiences.) This was even more the case if they had had a good childhood. In her studies of the brain, this reaction to positive pictures was not only in the areas associated with the initial experience of strong emotions, but also in ``higher'' areas of thinking \& perceiving, i.e., in some of the same areas of those found in the depth-of-processing brain studies. This stronger reaction to positive pictures being even more enhanced by a good childhood fits with a new concept suggested by Michael Pluess \& Jay Belsky, the idea of ``vantage sensitivity'' which they created in order to highlight the specific potential for sensitive people to benefit from positive circumstances \& interventions.

\textit{E} is also for \textit{empathy}. In another study by Bianca Acevedo, sensitive \& nonsensitive persons looked at photos of both strangers \& loved ones expressing happiness, sadness, or a neutral feeling. In all situations, when there was emotion in the photo, sensitive persons showed increased activation in the insula but also more activity in their \textit{mirror neuron} system, especially when looking at the happy faces of loved ones. The brain's mirror neurons were only discovered in the last 20 years or so. When we watch someone else do something or feel something, this clump of neurons fires in the same way as some of the neurons in the person we are observing. As an example, the same neurons fire, to varying degrees, whether we are kicking a soccer ball, see someone else kicking a soccer ball, hear the sound of someone kicking a soccer ball, or hear or say the word ``kick.''

Not only do these amazing neurons help us learn through imitation, but in conjunction with the other areas of the brain that were especially active for HSPs, they help us know others' intentions \& how they feel. Hence they are largely responsible for the universal human capacity for empathy. We do not just have an idea of how someone else feels; we actually feel that way ourselves to some extent. This is very familiar to sensitive people. Anyone's sad face tended to generate more activity in these mirror neurons in HSPs than others. When seeing photos of their loved ones being unhappy, sensitive persons also showed more activation in areas suggesting they wanted to do something, to act, even more than in areas involving empathy (perhaps we learn to cool down our intense empathy in order to help). But overall, brain activation indicating empathy was stronger in HSPs than non-HSPs when looking at photos of faces showing strong emotion of any type.

There is a common misunderstanding that emotions cause us to think illogically. But recent scientific thinking, reviewed by psychologist Roy Baumeister \& his colleagues, has placed emotion at the center of wisdom. 1 reason is that most emotion is felt after an event, which apparently serves to help us remember what happened \& learn from it. The more upset we are by a mistake, the more we think about it \& will be able to avoid it the next time. The more delighted we are by a success, the more we think \& talk about it \& how we did it, causing us to be more likely to be able to repeat it.

Other studies discussed by Baumeister, which explore the contribution of emotion to clear thinking, find that unless people have some emotional reason to learn something, they do not learn it very well or at all. This is 1 reason why it is easier to learn a foreign language in the country where it is spoken -- we are highly motivated to find our way, converse when spoken to, \& generally not seem foolish. From this point of view, it would seem almost impossible for a highly sensitive person to process things deeply without having stronger emotional reactions to motivate them. \& remember, when HSPs react more, it is as much or more to positive emotions, such as curiosity, anticipation of success (using that short cut others don't know about), a pleasant desire for something, satisfaction, joy, contentedness. It may be that everyone reacts strongly to negative situations, but HSPs seem to have evolved so that we especially relish a good outcome \& figure out more than others do how to make it happen. I imagine that we can plan an especially good birthday celebration, anticipating the happiness it will bring.'' -- \cite[pp. 21--23]{Aron2013}

\subsubsection*{S is for Sensing the Subtle}
``Most of the studies already cited required perceiving subtleties. This is often what is most noticeable to us personally, the little things we notice that others miss. Given that, \& because I called the trait high sensitivity, many have thought this is the heart of the trait. (To correct this confusion \& emphasize the role of processing, we used ``sensory \textit{processing} sensitivity'' as its more formal scientific designation.) However, this trait is not so much about extraordinary senses -- after all, there are sensitive people who have poor eyesight or hearing. True, some sensitive people report that 1 or more senses are very acute, but even in these cases it could be that they process the sensory information more carefully rather than having something unusual about their eyes, nose, skin, taste buds, or ears. Again, the brain areas that are more active when sensitive people perceive are those that do the more complex processing of sensory information: not so much the areas that recognize alphabet letters by their shape or even that read words, but the areas that catch the subtle meaning of words.

On the 1 hand, our awareness of subtleties is useful in an infinite number of ways, from simple pleasure in life to strategizing our response based on our awareness of others' nonverbal cues (that they may have no idea they are giving off) about their mood or trustworthiness. On the other hand, of course, when we are worn out we may be the least aware of anything, subtle or gross, except our own need for a break. This brings us to an important point.'' -- \cite[p. 23]{Aron2013}

\subsection*{Every Highly Sensitive Person Is Different, \& Different at Different Times}
``DOES is a wonderful general guideline for understanding high sensitivity but it is not infallible. Depending on how we are feeling, we may not be reflecting on our behavior or noticing subtleties even as much as the non-HSPs around us. We also differ from each other. People have other traits, different histories, \& are just different. In our enthusiasm to identify ourselves as a group -- even as a misunderstood minority -- we do not want to forget that we are not identical by any means. In particular, we are not all, or all the time, aware, conscientious, wonderful people!

Take \textit{O for easily overstimulated}. 2 sensitive people may behave quite differently when being bothered by loud noise or rude, upsetting behavior by others. One may rarely complain or be visibly bothered by such things because this person avoids such situations or quietly exits them. He or she will not, e.g., stay in a job if noise, rudeness, or other annoyances are present. If this HSP cannot escape the problems, he or she quietly tolerates them until they can be corrected. Other HSPs, usually with a more stressful past, will feel more victimized \& upset, \& at the same time be less able to place themselves in the right environments \& avoid the wrong ones. Maybe they feel they have to please others or prove something. In the workplace, they may not quit a job until a crisis occurs so that everyone working there knows about their ``over'' sensitivity.

A study done by Bhavini Shrivastava of HSPs in an information technology firm in India found that they felt more stressed than others by their work environment, but were actually seen as more productive than others by their managers. If we assume that those HSPs whose performance had suffered from stress had already quit or been let go, the remaining HSPs (who were older \& longer on the job) apparently were quietly adapting, perhaps with special considerations from their supervisors, \& contributing their depth of processing \& awareness of subtleties to their company. So we see 2 (or more) types of HSPs -- able to manage or not, due to other facets of their personality. Or in other instances, 2 (or more types) of situations: a little stressful, s.t. HSPs in that situation seem like strong people who find ways to adapt that others miss; or hopelessly stressful, s.t. they cannot adapt \& seem weak.'' -- \cite[pp. 24--25]{Aron2013}

\subsection*{Final Thoughts}
``Studying high sensitivity has been an amazing journey for me. It began with a simple curiosity about something someone else said about me. I did some interviews of people who thought they might be highly sensitive just to see what it was, with no further research plans \& definitely no intention of writing a book for the public. Then, as I like to put it, I found I was walking down a street \& a parade began to form behind me, a parade of people who were highly sensitive \& had never heard the term before.

Over \& over I am asked, ``How could you discover a new trait?'' The answer is that sensitivity is not new but just difficult to observe by watching how people behave, which is usually how psychology proceeds. Hence psychologists \& people in general were coming up with names for the trait that were close but not precise, such as \textit{shyness} \& \textit{introversion}. We make it especially hard for others to observe our trait because we are so responsive to our environments that we can be something like chameleons when around others, doing whatever it takes to fit in. I happened to be in the position to be both a curious scientist \& a highly sensitive person, who could know this experience from the inside. Still, as I said in the original preface, even for me to focus on my own sensitivity required someone else to comment on it in me 1st, after I had an ``over'' reaction to a medical procedure.

When we are visible, the most obvious thing we do is ``over'' react compared to others -- the \textit{O} of being overstimulated \& the \textit{E} of stronger emotional reactions. But then we are a minority, so of course we are above average here \& not reacting as most people do. It's the more noticeable \textit{O} \& \textit{E} that have made it seem to ourselves \& others that we have a flaw. Further, those HSPs with a troubled past have less control over their reactions, \& hence the trait becomes associated with people having difficulties. The few observable things we do that would indicate \textit{D} \& \textit{S}, depth of processing \& awareness of subtleties, can easily be overlooked or misunderstood. E.g., if we are seen taking our time before entering a situation or making a decision, that can seem, again, to be different, a potential problem, \& therefore a flaw. It is easy to overlook how good those decisions can be when finally made. Further, this sort of slowness can be caused by many things besides sensitivity, such as fear or even low intelligence. It's what's going on inside, out of sight, that most clearly sorts the highly sensitive minority from others. Thank goodness for these new ways of doing brain research that show these differences \& for all of you who have stepped forward \& said, yes, that's what goes on inside of me, too.

So let's celebrate! Maybe with a parade!'' -- \cite[pp. 25--26]{Aron2013}

%------------------------------------------------------------------------------%

\section*{Preface}
``Cry baby!''

``Scaredy-cat!''

``Don't be a spoilsport!''

Echoes from the past? \& how about this well-meaning warning: ``You're just too sensitive for your own good.''

If you were like me, you heard a lot of that, \& it made you feel there must be something very different about you. I was convinced that I had a fatal flaw that I had to hide \& that doomed me to a 2nd-rate life. I thought there was something wrong with me.

In fact, there is something very right with you \& me. If you answered true to 12 or more of the questions on the self-test at the beginning of this book, or if the detailed description in Chap. 1 seems to fit you (really the best test), then you are a very special type of human being, a highly sensitive person -- which hereafter we'll call an HSP. \& this book is just for you.

\textit{Having a sensitive nervous system is normal, a basically neutral trait}. You probably inherited it. It occurs in about 15--20\% of the population. It means you are aware of subtleties in your surroundings, a great advantage in many situations. It also means you are more easily overwhelmed when you have been out in a highly stimulating environment for too long, bombarded by sights \& sounds until you are exhausted in a nervous-system sort of way. Thus, being sensitive has both advantages \& disadvantages.

In our culture, however, possessing this trait is not considered ideal \& that fact probably has had a major impact on you. Well-meaning parents \& teachers probably tried to help you ``overcome'' it, as if it were a defect. Other children were not always as nice about it. As an adult, it has probably been harder to find the right career \& relationships \& generally to feel self-worth \& self-confidence.'' -- \cite[p. 27]{Aron2013}

\subsection*{What This Book Offers You}
``This book provides basic, detailed information you need about your trait, data that exist nowhere else. It is the product of 5 years of research, in-depth interviews, clinical experience, courses \& individual consultations with hundreds of HSPs, \& careful reading between he lines of what psychology has already learned about the trait but does not realize it knows. In the 1st 3 chapters you will learn all the basic facts about your trait \& how to handle overstimulation \& overarousal of your nervous system.

Next, this book considers the impact of your sensitivity on your personal history, career, relationships, \& inner life. It focuses on the advantages you may not have thought of, plus it gives advice about typical problems some HSPs face, such as shyness or difficulty finding the right sort of work.

It is quite a journey we'll take. Most of the HSPs I've helped with the information that is in this book have told me that it has dramatically changed their lives -- \& they've told me to tell you that.'' -- \cite[p. 28]{Aron2013}

\subsection*{A Word to the Sensitive-But-Less-So}
``1st, if yu have picked up this book because you're the parent, spouse, or friend of an HSP, then you're especially welcome here. Your relationship with your HSP will be greatly improved.

2nd, a telephone survey of 300 randomly selected individuals of all ages found that while 20\% were extremely or quite sensitive, another 22\% were moderately sensitive. Those of you who fall into this moderately sensitive category will also benefit from this book.

By the way, 42\% said they were not sensitive at all -- which suggests why the highly sensitive can feel so completely out of step with a large part of the world. \& naturally, it's that segment of the population that's always turning up the radio or honking their horns.

Further, it is safe to say that everyone can become highly sensitive at times -- e.g., after a month alone in a mountain cabin. \& everyone becomes more sensitive as they age. Indeed, most people, whether they admit it or not, probably have a highly sensitive facet that comes to the fore in certain situations.'' -- \cite[p. 29]{Aron2013}

\subsection*{\& Some Things to Say to Non-HSPs}
``Sometimes non-HSPs feel excluded \& hurt by the idea that we are different from them \& maybe sound like we think we are somehow better. They say, ``Do you mean I'm not sensitive?'' 1 problem is that ``sensitive'' also means being understanding \& aware. Both HSPs \& non-HSPs can have these qualities, which are optimized when we are feeling good \& alert to the subtle. When very calm, HSPs may even enjoy the advantage of picking up more delicate nuances. When overaroused, however, a frequent state for HSPs, we are anything but understanding or sensitive. Instead, we are overwhelmed, frazzled, \& need to be alone. By contrast, your non-HSP friends are actually more understanding of others in highly chaotic situations.

I thought long \& hard about what to call this trait. I knew I didn't want to repeat the mistake of confusing it with introversion, shyness, inhibitedness, \& a host of other misnomers laid on us by other psychologists. None of them captures the neutral, much less the positive, aspects of the trait. ``Sensitivity'' does express the neutral fact of greater receptivity to stimulation. So it seemed to be time to make up for the bias against HSPs by using a term that might be taken in our favor.

On the other hand, being ``highly sensitive'' is anything but positive to some. While sitting in my quiet house writing this, at a time when no one is talking about the trait, I'll go on record: This book will generate more than its share of hurtful jokes \& comments about HSPs. There is tremendous collective psychological energy around the idea of being sensitive -- almost as much as around gender issues, with which sensitivity is often confused. (There are as many male as female babies born sensitive; but men are not supposed to possess the trait \& women are. Both genders pay a high price for that confusion.) So just be prepared for that energy. Protect both your sensitivity \& your newly budding understanding of it by not talking about it at all when that seems most prudent.

Mostly, enjoy knowing that there are also many like-minded people out there. We have not been in touch before. But we are now, \& both we \& our society will be the better for it. In Chaps. 1, 5, \& 10, I will comment at some length on the HSP's important social function.'' -- \cite[pp. 29--30]{Aron2013}

\subsection*{What You Need}
``I have found that HSPs benefit a fourfold approach, which the chapters in this book will follow.
\begin{enumerate}
	\item \textit{Self-knowledge.} You have to understand what it means to be an HSP. Thoroughly. \& how it fits with your other traits \& how your society's negative attitude has affected you. Then you need to know your sensitive body very well. No more ignoring your body because it seems too uncooperative or weak.
	\item \textit{Reframing.} You must actively reframe much of your past in the light of knowing you came into the world highly sensitive. So many of your ``failures'' were inevitable because neither you nor your parents \& teachers, friends \& colleagues, understood you. Reframing how you experienced your past can lead to solid self-esteem, \& self-esteem is especially important for HSPs, for it decreases our overarousal in new (\& therefore highly stimulating) situations.
	
	Reframing is not automatic, however. That is why I include ``activities'' at the end of each chapter that often involve it.
	\item \textit{Healing.} If you have not yet done so, you must begin to heal the deeper wounds. You were very sensitive as a child; family \& school problems, childhood illnesses, \& the like all affected you more than others. Furthermore, you were different from other kids \& almost surely suffered for that.
	
	HSPs especially, sensing the intense feelings that must arise, may hold back from the inner work necessary to heal the wounds from the past. Caution \& slowness are justified. But you will cheat yourself if you delay.
	\item \textit{Help With Feeling Okay When Out in the World \& Learning When to Be Less Out.} You can be, should be, \& need to be involved in the world. It truly needs you. But you have to be skilled at avoiding overdoing or underdoing it. This book, free of the confusing messages from a less sensitive culture, is about discovering that way.
\end{enumerate}
I will also teach you about your trait's effect on your close relationships. \& I'll discuss psychotherapy \& HSPs -- which HSPs should be in therapy \& why, what kind, with whom, \& especially how therapy differs for HSPs. Then I'll consider HSPs \& medical care, including plenty of information on medications like Prozac, often taken by HSPs. At the end of this book we will savor our rich inner life.'' -- \cite[pp. 30--31]{Aron2013}

\subsection*{About Myself}
``I am a research psychologist, university professor, psychotherapist, \& published novelist. What matters most, however, is that I am an HSP like you. I am definitely not writing from on high, aiming down to help you, poor soul, overcome your ``syndrome.'' I know personally about \textit{our} trait, its assets \& its challenges.

As a child, at home, I hid from the chaos in my family. At school I avoided sports, games, \& kids in general. What a mixture of relief \& humiliation when my strategy succeeded \& I was totally ignored.

In junior high school an extrovert took me under her wing. In high school that relationship continued, plus I studied most of the time. In college my life became far more difficult. After many stops \& starts, including a 4-year marriage undertaken too young, I finally graduated Phi Beta Kappa from the University of California at Berkeley. But I spent my share of time crying in rest rooms, thinking I was going crazy. (My research has found that retreating like this, often to cry, is typical of HSPs.)

In my 1st try at graduate school I was provided with an office, to which I also retreated \& cried, trying to regain some calm. Because of such reactions, I stopped my studies with a master's degree, even though I was highly encouraged to continue for a doctorate. It took 25 years for me to gain the information about my trait that made it possible to understand my reactions \& so complete that doctorate.

When I was 23, I met my current husband \& settled down into a very protected life of writing \& rearing a son. I was simultaneously delighted \& ashamed of not being ``out there.'' I was vaguely aware of my lost opportunities to learn, to enjoy more public recognition of my abilities, to be more connected with all kinds of people. But from bitter experienced I thought I had no choice.

Some arousing events, however, cannot be avoided. I had to undergo a medical procedure from which I assumed I would recover in a few weeks. Instead, for months my body seemed to resound with physical \& emotional reactions. I was being forced to face once again that mysterious ``fatal flaw'' of mine that made me so different. So I tried some psychotherapy. \& got lucky. After listening to me for a few sessions, my therapist said, ``But of course you were upset; you are a very highly sensitive person.''

What is this, I thought, some excuse? She said she had never thought much about it, but from her experience it seemed that there were real differences in people's tolerance for stimulation \& also their openness to the deeper significance of an experience, good \& bad. To her, such sensitivity was hardly a sign of a mental flaw or disorder. At least she hoped not, for she was highly sensitive herself. I recall her grin. ``As are most of the people who strike me as really worth knowing.''

I spent several years in therapy, none of it wasted, working through various issues from my childhood. But the central theme became the impact of this trait. There was my sense of being flawed. There was the willingness of others to protect me in return for enjoying my imagination, empathy, creativity, \& insight, which I myself hardly appreciated. \& there was my resulting isolation from the world. But as I gained insight, I was able to reenter the world. I take great pleasure now in being part of things, a professional, \& sharing the special gifts of my sensitivity.'' -- \cite[pp. 31--33]{Aron2013}

\subsection*{The Research Behind This Book}
``As knowledge about my trait changed my life, I decided to read more about it, but there was almost nothing available. I thought the closest topic might be introversion. The psychiatrist Carl Jung wrote very wisely on the subject, calling it a tendency to turn inward. The work of Jung, himself an HSP, has been a major help to me, but the more scientific work on introversion was focused on introverts not being sociable, \& it was that idea which made me wonder if introversion \& sensitivity were being wrongly equated.

With so little information to go on, I decided to put a notice in a newsletter that went to the staff of the university where I was teaching at the time. I asked to interview anyone who felt they were highly sensitive to stimulation, introverted, or quick to react emotionally. Soon I had more volunteers than I needed.

Next, the local paper did a story on the research. Even though there was nothing said in the article about how to reach me, over a hundred people phoned \& wrote me, thanking me, wanting help, or just wanting to say, ``Me, too.'' 2 years later, people were still contacting me. (HSPs sometimes think things over for a while before making their move!)

Based on the interviews (40 for 2--3 hours each), I designed a questionnaire that I have distributed to thousands all over North America. \& I directed a random-dialing telephone survey of 300 people as well. The point that matters for you is that everything in this book is based on solid research, my own or that of others. Or I am speaking from my repeated observations of HSPs, from my courses, conversations, individual consultations, \& psychotherapy with them. These opportunities to explore the personal lives of HSPs have numbered in the thousands. Even so, I will say ``probably'' \& ``maybe'' more than you are used to in books for the general reader, but I think HSPs appreciate that.

Deciding to do all of this research, writing, \& teaching has made me a kind of pioneer. But that, too, is part of being an HSP. We are often the 1st ones to see what needs to be done. As our confidence in our virtues grows, perhaps more \& more of us will speak up -- in our sensitive way.'' -- \cite[p. 33]{Aron2013}

\subsection*{Instructions to the Reader}
\begin{enumerate}
	\item ``Again, I address the reader as an HSP, but this book is written equally for someone seeking to understand HSPs, whether as a friend, relative, advisor, employer, educator, or health professional.
	\item This book involves seeing yourself as having a trait common to many. I.e., it labels you. The advantages are that you can feel normal \& benefit from the experience \& research of others. But any label misses your uniqueness. HSPs are each utterly different, even with their common trait. Please remind yourself of that as you proceed.
	\item While you are reading this book, you will probably see everything in your life in light of being highly sensitive. That is to be expected. In fact, it is exactly the idea. Total immersion helps with learning any new language, including a new way of talking about yourself. If others feel a little concerned, left out, or annoyed, ask for their patience. There will come a day when the concept will settle in \& you'll be talking about it less.
	\item This book includes some activities which I have found useful for HSPs. But I'm not going to say that you must do them if you want to gain anything from this book. Trust your HSP intuition \& do what feels right.
	\item Any of the activities could bring up strong feelings. If that happens, I do urge you to seek professional help. If you are now in therapy, this book should fit well with your work there. The ideas here might even shorten the time you will need therapy as you envision a new ideal self -- not the culture's ideal but your own, someone you can be \& maybe already are. But remember that this book does not substitute for a good therapist when things get intense or confusing.
\end{enumerate}
This is an exciting moment for me as I imagine you turning the page \& entering into this new world of mine, of yours, of \textit{ours}. After thinking for so long that you might be the only one, it is nice to have company, isn't it?'' -- \cite[p. 34]{Aron2013}

%------------------------------------------------------------------------------%

\section*{\textit{Are You Highly Sensitive?} A Self-Test}
``Answer each question according to the way you feel. Answer true if it is at least somewhat true for you. Answer false if it is not very true or not at all true for you.
\begin{enumerate}
	\item I seem to be aware of subtleties in my environment.
	\item Other people's moods affect me.
	\item I tend to be very sensitive to pain.
	\item I find myself needing to withdraw during busy days, into bed or into a darkened room or any place where I can have some privacy \& relief from stimulation.
	\item I am particularly sensitive to the effects of caffeine.
	\item I am easily overwhelmed by things like bright lights, strong smells, coarse fabrics, or sirens close by.
	\item I have a rich, complex inner life.
	\item I am made uncomfortable by loud noises.
	\item I am deeply moved by the arts or music.
	\item I am conscientious.
	\item I startle easily.
	\item I get rattled when I have a lot to do in a short amount of time.
	\item When people are uncomfortable in a physical environment I tend to know what needs to be done to make it more comfortable (like changing the lighting or the seating).
	\item I am annoyed when people try to get me to do too many things at once.
	\item I try hard to avoid making mistakes or forgetting things.
	\item I make it a point to avoid violent movies \& TV shows.
	\item I become unpleasantly aroused when a lot is going on around me.
	\item Being very hungry creates a strong reaction in me, disrupting my concentration or mood.
	\item Changes in my life shake me up.
	\item I notice \& enjoy delicate or fine scents, tastes, sounds, works of art.
	\item I make it a high priority to arrange my life to avoid upsetting or overwhelming situations.
	\item When I must compete or be observed while performing a task, I become so nervous or shaky that I do much worse than I would otherwise.
	\item When I was a child, my parents or teachers seemed to see me as sensitive or shy.
\end{enumerate}

\subsection*{Scoring Yourself}
If you answered true to 12 or more of the questions, you're probably highly sensitive.

But frankly, no psychological test is so accurate that you should base your life on it. If only 1 or 2 questions are true of you but they are extremely true, you might also be justified in calling yourself highly sensitive.

Read on, \& if you recognize yourself in the in-depth description of a highly sensitive person in Chap. I, consider yourself one. The rest of this book will help you understand yourself better \& learn to thrive in today's not-so-sensitive world.'' -- \cite[pp. 35--36]{Aron2013}

%------------------------------------------------------------------------------%

\section{The Facts About Being Highly Sensitive: A (Wrong) Sense of Being Flawed}
``In this chapter you will learn the basic facts about your trait \& how it makes you different from others. You will also discover the rest of your inherited personality \& have your eyes opened about your culture's view of you. But first you should meet Kristen.'' -- \cite[p. 37]{Aron2013}

\subsection{She Thought She Was Crazy}
``Kristen was the 23rd interview of my research on HSPs. She was an intelligent, clear-eyed college student. But soon into our interview her voice began to tremble.

``I'm sorry,'' she whispered. ``But I really signed up to see you because you're a psychologist \& I had to talk to someone who could tell me --'' Her voice broke. ``Am I \textit{crazy}?'' I studied her with sympathy. She was obviously feeling desperate, but nothing she had said so far had given me any sense of mental illness. But then, I was already listening differently to people like Kristen.

She tried again, as if afraid to give me time to answer. ``I feel so different. I always did. I don't mean -- I mean, my family was great. My childhood was almost idyllic until I had to go to school. Although Mom says I was always a grumpy baby.''

She took a breath. I said something reassuring, \& she plunged on. ``But in nursery school I was afraid of everything. Even music time. When they would pass out the pots \& pans to pound, I would put my hands over my ears \& cry.''

She looked away, her eyes glistening with tears now, too. ``In elementary school I was always the teacher's pet. Yet they'd say I was `spacey.'''

Her ``spaciness'' prompted a distressing series of medical \& psychological tests. 1st for mental retardation. As a result, she was enrolled in a program for the \textit{gifted}, which did not surprise me.

Still the message was ``Something is wrong with this child.'' Her hearing was tested. Normal. In 4th grade she had a brain scan on the theory that her inwardness was due to petit mal seizures. Her brain was normal.

The final diagnosis? She had ``trouble screening out stimuli.'' But the result was a child who believed she was defective.'' -- \cite[pp. 37--38]{Aron2013}

\subsection{Special But Deeply Misunderstood}
``The diagnosis was right as far as it went. HSPs do take in a lot -- all the subtleties others miss. But what seems ordinary to others, like loud music or crowds, can be highly stimulating \& thus stressful for HSPs.

Most people ignore sirens, glaring lights, strange odors, clutter \& chaos. HSPs are disturbed by them.

Most people's feet may be tired at the end of a day in a mall or a museum, but they're ready for more when you suggest an evening party. HSPs need solitude after such a day. They feel jangled, overaroused.

Most people walk into a room \& perhaps notice the furniture, the people -- that's about it. HSPs can be instantly aware, whether they wish to be or not, of the mood, the friendships \& enmities, the freshness or staleness of the air, the personality of the one who arranged the flowers.

If you are an HSP, however, it is hard to grasp that you have some remarkable ability. How do you compare inner experiences? Not easily. Mostly you notice that you seem unable to tolerate as much as other people. You forget that you belong to a group that has often demonstrated great creativity, insight, passion, \& caring -- all highly valued by society.

We are a package deal, however. Our trait of sensitivity means we will also be cautious, inward, needing extra time alone. Because people without the trait (the majority) do not understand that, they see us as timid, shy, weak, or that greatest sin of all, unsociable. Fearing these labels, we try to be like others. But that leads to our becoming overaroused \& distressed. Then \textit{that} gets us labeled neurotic or crazy, 1st by others \& then by ourselves.'' -- \cite[pp. 38--39]{Aron2013}

\subsection{Kristen's Dangerous year}
``Sooner or later everyone encounters stressful life experiences, but HSPs react more to such stimulation. If you see this reaction as part of some basic flaw, you intensify the stress already present in any life crisis. Next come feelings of hopelessness \& worthlessness.

Kristen, e.g., had such a crisis the year she started college. She had attended a low-key private high school \& had never been away from home. Suddenly she was living among strangers, fighting in crowds for courses \& books, \& always overstimulated. Next she feel in love, fast \& hard (as HSPs can do). Shortly after, she went to Japan to meet her boyfriend's family, an event she already had good reason to fear. It was while she was in Japan that, in her words, she ``flipped out.''

Kristen had never thought of herself as an anxious person, but suddenly, in Japan, she was overcome by fears \& could not sleep. Then she became depressed. Frightened by her own emotions, her self-confidence plummeted. Her young boyfriend could not cope with her ``craziness'' \& wanted to end the relationship. By then she had returned to school, but feared she was going to fail at that, too. Kristen was on the edge.

She looked up at me after sobbing out the last of her story. ``Then I heard about this research, about being sensitive, \& I thought, Could that be me? But it isn't, I know. Is it?''

I told her that of course I could not be sure from such a brief conversation, but I believed that, yes, her sensitivity in combination with all these stresses might well explain her state of mind. \& so I had the privilege of explaining Kristen to herself -- an explanation obviously long overdue.'' -- \cite[pp. 39--40]{Aron2013}

\subsection{Defining High Sensitivity -- 2 Facts to Remember}
\textbf{Fact 1: Everyone, HSP or not, feels best when neither too bored nor too aroused.} ``An individual will perform best on any kind of task, whether engaging in a conversation or playing in the Super Bowl, if his or her nervous system is moderately alert \& aroused. Too little arousal \& one is dull, ineffective. To change that underaroused physical state, we drink some coffee, turn on the radio, call a friend, strike up a conversation with a total stranger, change careers -- anything!

At the other extreme, too much arousal of the nervous system \& anyone will become distressed, clumsy, \& confused. We cannot think; the body is not coordinated; we feel out of control. Again, we have many ways to correct the situation. Sometimes we rest. Or mentally shut down. Some of us drink alcohol or take a Valium.

The best amount of arousal falls somewhere in the middle. That there is a need \& desire for an ``optimal level of arousal'' is, in fact, 1 of the most solid findings of psychology. It is true for everyone, even infants. They hate to feel bored or overwhelmed.

\textbf{Fact 2: People differ considerably in how much their nervous system is aroused in the same situation, under the same stimulation.} The difference is largely inherited, \& is very real \& normal. In fact, it can be observed in all higher animals -- mice, cats, dogs, horses, monkeys, humans. Within a species, the percentage that is very sensitive to stimulation is usually about the same, around 15--20\%. Just as some within a species are a little bigger in size than others, some are a little more sensitive. In fact, through careful breeding of animals, mating the sensitive ones to each other can create a sensitive strain in just a few generations. In short, among inborn traits of temperament, this one creates the most dramatic, observable differences.'' -- \cite[p. 40]{Aron2013}

\subsection{The Good News \& the No-So-Good}
``What this difference in arousability means is that you notice levels of stimulation that go unobserved by others. This is true whether we are talking about subtle sounds, sights, or physical sensations like pain. It is not that your hearing, vision, or other senses are more acute (plenty of HSPs wear glasses). The difference seems to lie somewhere on the way to the brain or in the brain, in a more careful processing of information. We reflect more on everything. \& we sort things into finer distinctions. Like those machines that grade fruit by size -- we sort into 10 sizes while others sort into 2 or 3.

This greater awareness of the subtle tends to make you more intuitive, which simply means picking up \& working through information in a semiconscious or unconscious way. The result is that you often ``just know'' without realizing how. Furthermore, this deeper processing of subtle details causes you to consider the past or future more. You ``just know'' how things got to be the way they are or how they are going to turn out. This is that ``6th sense'' people talk about. It can be wrong, of course, just as your eyes \& ears can be wrong, but your intuition is right often enough that HSPs tend to be visionaries, highly intuitive artists, or inventors, as well as more conscientious, cautious, \& wise people.

The downside of the trait shows up at more intense levels of stimulation. What is \textit{moderately} arousing for most people is highly arousing for HSPs. What is \textit{highly} arousing for most people causes an HSP to become very frazzled indeed, until they reach a shutdown point called ``transmarginal inhibition.'' Transmarginal inhibition was 1st discussed around the turn of the century by the Russian physiologist Ivan Pavlov, who was convinced that the most basic inherited difference among people was how soon they reach this shutdown point \& that the quick-to-shut-down have a fundamentally different type of nervous system.

No one likes being overaroused, HSP or not. A person feels out of control, \& the whole body warns that it is in trouble. Overarousal often means failing to perform at one's best. Of course, it can also mean danger. An extra dread of overarousal may even be built into all of us. Since a newborn cannot run or fight or even recognize danger, it is best if it howls at anything new, anything arousing at all, so that grown-ups can come \& rescue it.

Like the fire department, we HSPs mostly respond to false alarms. But if our sensitivity saves a life even once, it is a trait that has a genetic payoff. So, yes, when our trait leads to overarousal, it is a nuisance. But it is part of a package deal with many advantages.'' -- \cite[pp. 41--42]{Aron2013}

\subsection{More About Stimulation}
``Stimulation is anything that wakes up the nervous system, gets its attention, makes the nerves fire off another around of the little electrical charges that they carry. We usually think of stimulation as coming from outside, but of course it can come from our body (such as pain, muscle tension, hunger, thirst, or sexual feelings) or as memories, fantasies, thoughts, or plans.

Stimulation can vary in intensity (like the loudness of a noise) or in duration. It can be more stimulating because it is novel, as when one is startled by a honk or shout, or in its complexity, as when one is at a party \& hearing 4 conversations at once plus music.

Often we can get used to stimulation. But sometimes we think we have \& aren't being bothered, but suddenly feel exhausted \& realize why: We have been putting up with something at a conscious level while it was actually wearing us down. Even a moderate \& familiar stimulation, like a day at work, can cause an HSP to need quiet by evening. At that point, 1 more ``small'' stimulation can be the last straw.

Stimulation is even more complicated because the same stimulus can have different meanings for different people. A crowded shopping mall at Christmastime may remind 1 person of happy family shopping excursions \& create a warm holiday spirit. But another person may have been forced to go shopping with others, tried to buy gifts without enough money \& no idea of what to purchase, had unhappy memories of past holidays, \& so suffers intensely in malls at Christmas.

\textbf{VALUING YOUR SENSITIVITY: Think back to 1 or more times that your sensitivity has saved you or someone else from suffering, great loss, or even death. (In my own case, I \& all my family would be dead if I had not awakened at the 1st flicker of firelight in the ceiling of an old wooden house in which we were living.)}

1 general rule is that when we have no control over stimulation, it is more upsetting, even more so if we feel we are someone's victim. While music played by ourselves may be pleasant, heard from the neighbor's stereo, it can be annoying, \& if we have previously asked them to turn it down, it becomes a hostile invasion. This book may even increase your annoyance a bit as you begin to appreciate that you are a minority whose rights to have less stimulation are generally ignored.

Obviously it would help if we were enlightened \& detached from all of these associations so that nothing could arouse us. No wonder so many HSPs become interested in spiritual paths.'' -- \cite[pp. 42--43]{Aron2013}

\subsection{Is Arousal Really Different From Anxiety \& Fear?}
``It is important not to confuse arousal with fear. Fear creates arousal, but so do many other emotions, including joy, curiosity, or anger. But we can also be overaroused by semiconscious thoughts or low levels of excitement that create no obvious emotion. Often we are not aware of what is arousing us, such as the newness of a situation or noise or the many things our eyes are seeing.

Actually, there are several ways to \textit{be} aroused \& still other ways to \textit{feel} aroused, \& they differ from time to time \& from person to person. Arousal may appear as blushing, trembling, heart pounding, hands shaking, foggy thinking, stomach churning, muscles tensing, \& hands or other parts of the body perspiring. Often people in such situations are not aware of some or all of these reactions as they occur. On the other hand, some people say they feel aroused, but that arousal shows up very little in any of these ways. Still, the term does describe something that all these experiences \& physical states share. Like the word ``stress,'' arousal is a word that really communicates something we all know about, even if that something varies a lot. \& of course stress is closely related to arousal: Our response to stress is to become aroused.

Once we do notice arousal, we want to name it \& know its source in order to recognize danger. \& often we think that our arousal is due to fear. We do not realize that our heart may be pounding from the sheer effort of processing extra stimulation. Or other people assume we are afraid, given our obvious arousal, so we assume it, too. Then, deciding we must be afraid, we become even more aroused. \& we avoid the situation in the future when staying in it \& getting used to it might have calmed us down. We will discuss again the importance of not confusing fear \& arousal in Chap. 5 when we talk about ``shyness.'''' -- \cite[pp. 43--44]{Aron2013}

\subsection{Your Trait Really Does Make You Special}
``There are many fruits growing from the trait of sensitivity. Your mind works differently. Please remember that what follows is \textit{on the average}; nobody has all these traits. But compared to non-HSPs, most of us are:
\begin{itemize}
	\item Better at spotting errors \& avoiding making errors.
	\item Highly conscientious.
	\item Able to concentrate deeply. (\textit{But we do best without distractions.})
	\item Especially good at tasks requiring vigilance, accuracy, speed, \& the detection of minor differences.
	\item Able to process material to deeper levels of what psychologists call ``semantic memory.'' Often thinking about our own thinking.\footnote{NQBH: Superthinking.}
	\item Able to learn without being aware we have learned.
	\item Deeply affected by other people's moods \& emotions.
\end{itemize}
Of course, there are many exceptions, especially to our being conscientious. \& we don't want to be self-righteous about this; plenty of harm can be done in the name of trying to do good. Indeed, all of these fruits have their bruised spots. We are so skilled, but alas, when being watched, timed, or evaluated, we often cannot display our competence. Our deeper processing may make it seem that at 1st we are not catching on, but with time we understand \& remember more than others. This may be why HSPs learn languages better (although arousal may make one less fluent than others when speaking).

By the way, thinking more than others about our own thoughts is not self-centeredness. It means that if asked what's on our mind, we are less likely to mention being aware of the world around us, \& more likely to mention our inner reflections or musings. But we are no less likely to mention thinking about other people.

Our bodies are different too. Most of us have nervous systems that make us:
\begin{itemize}
	\item Specialists in fine motor movements.
	\item Good at holding still.
	\item ``Morning people.'' (\textit{Here there are many exceptions.})
	\item More affected by stimulants like caffeine unless we are very used to them.
	\item More ``right-brained'' (less linear, more creative in a synthesizing way).
	\item More sensitive to things in the air. (\textit{Yes, that means more hay fever \& skin rashes.})
\end{itemize}
Overall, again, our nervous systems seem designed to react to subtle experiences, which also makes us slower to recover when we must react to intense stimuli.

But HSPs are not in a more aroused state all the time. We are not ``chronically aroused'' in day-to-day life or when asleep. We are just more aroused by new or prolonged stimulation. (Being an HSP is \textit{not} the same as being ``neurotic'' -- i.e., constantly anxious for no apparent reason).'' -- \cite[pp. 44--45]{Aron2013}

\subsection{How to Think About Your Differences}
``I hope  that by now you are seeing your trait in positive terms. But I really suggest trying to view it as neutral. It becomes an advantage or disadvantage only when you enter a particular situation. Since the trait exists in all higher animals, it must have value in many circumstances. My hunch is that it survives in a certain percentage of all higher animals because it is useful to have at least a few around who are always watching for subtle signs. 15--20\% seems about the right proportion to have always on the alert for danger, new foods, the needs of the young \& sick, \& the habits of other animals.

Of course, it is also good to have quite a few in a group who are not so alert to all the dangers \& consequences of every action. They will rush out without a whole lot of thought to explore every new thing or fight for the group or territory. Every society needs both. \& maybe there is a need for more of the \textit{less} sensitive because more of them tend to get killed! This is all speculation, of course.

Another hunch of mine, however, is that the human race benefits more from HSPs than do other species. HSPs do more of that which makes humans different from other animals: We imagine possibilities. We humans, \& HSPs especially, are acutely aware of the past \& future. On top of that, if necessity is the mother of invention, HSPs must spend far more time trying to invent solutions to human problems just because they are more sensitive to hunger, cold, insecurity, exhaustion, \& illness.

Sometimes people with our trait are said to be less happy or less capable of happiness. Of course, we can seem unhappy \& moody, at least to non-HSPs, because we spend so much time thinking about things like the meaning of life \& death \& how complicated everything is -- not black-\&-white thoughts at all. Since most non-HSPs do not seem to enjoy thinking about such things, they assume we must be unhappy doing all that pondering. \& we certainly don't get any happier having them tell us we are unhappy (by \textit{their} definition of happy) \& that we are a problem for them because we seem unhappy. All those accusations could make \textit{anyone} unhappy.

The point is best made by Aristotle, who supposedly asked, ``Would you rather be a happy pig or an unhappy human?'' HSPs prefer the good feeling of being very conscious, very human, even if what we are conscious of is not always cause for rejoicing.

The point, however, is not that non-HSPs are pigs! I \textit{know} someone is going to say I am trying to make an elite out of us. But that would last about 5 minutes with most HSPs, who would soon feel guilty for feeling superior. I'm just out to encourage us enough to make more of us feel like equals.'' -- \cite[pp. 45--46]{Aron2013}

\subsection{Heredity \& Environment}
``Some of you may be wondering if you really inherited this trait, especially if you remember a time when your sensitivity seemed to begin or greatly increase.

In most cases, sensitivity is inherited. The evidence for this is strong, mainly from studies of identical twins who were raised apart but grew up behaving similarly, which always suggests that behavior is at least partly genetically determined.

On the other hand, it is not always true that both separated twins show the trait, even if they are identical. E.g., each twin will also tend to develop a personality quite like the mother raising that twin, even though she is not the biological mother. The fact is, there are probably no inherited traits that cannot also be enhanced, decreased, or entirely produced or eliminated by enough of certain kinds of life experiences. E.g., a child under stress at home or at school only needs to be born with a slight tendency to be sensitive \& he or she will withdraw. Which may explain why children who have older brothers \& sisters are more likely to be HSPs -- \& that would have nothing to do with genes. Similarly, studies of baby monkeys traumatized by separation from their mothers have found that these monkeys in adulthood behave much like monkeys born innately sensitive.

Circumstances can also force the trait to disappear. Many children born very sensitive are pushed hard by parents, schools, or friends to be bolder. Living in a noisy or crowded environment, growing up in a large family, or being made to be more physically active may sometimes reduce sensitivity, just as sensitive animals that are handled a great deal will sometimes lose some of their natural caution, at least with certain people or in specific situations. That the underlying trait is entirely gone, however, seems unlikely.'' -- \cite[pp. 46--47]{Aron2013}

\subsection{What About You?}
``It is difficult to know for any particular adult whether you inherited the trait or developed it during your life. The best evidence, though hardly perfect, is whether your parents remember you as sensitive from the time you were born. If it is easy to do so, ask them, or whoever was your caretaker, to tell you all about what you were like in the 1st 6 months of life.

Probably you will learn more if you do \textit{not} begin by asking if you were sensitive. Just ask what you were like as a baby. Often the stories about you will tell it all. After a while, ask about some typical signs of highly sensitive babies. Were you difficult about change -- about being undressed \& put into water at bath time, about trying new foods, about noise? Did you have colic often? Were you slow to fall asleep, hard to keep asleep, or a short sleeper, especially when you were overtired?

Remember, if your parents had no experience with other babies, they may not have noticed anything unusual at that age because they had no one to compare you to. Also, given all the blaming of parents for their children's every difficulty, your parents may need to convince you \& themselves that all was perfect in your childhood. If you want, you can reassure them that you know they did their best \& that all babies pose a few problems but that you wonder which problems you presented.

You might also let them see the questionnaire at the front of this book. Ask them if they or anyone else in your family has this trait. Especially if you find relatives with it on both sides, the odds are very good your trait is inherited.

But what if it wasn't or you aren't sure? It probably does not matter at all. What \textit{does} is that it is \textit{your} trait now. So do not struggle too long over the question. The next topic is far more important.'' -- \cite[pp. 47--48]{Aron2013}

\subsection{Learning About Our Culture -- What You Don't Realize WILL Hurt You}
``You \& I are learning to see our trait as a neutral thing -- use--ful in some situations, not in others -- but our culture definitely does not see it, or any trait, as neutral. The anthropologist Margaret Mead explained it well. Although a culture's newborns will show a broad range of inherited temperaments, only a narrow band of these, a certain type, will be the ideal. The ideal personality is embodied, in Mead's words, in ``every thread of the social fabric -- in the care of the young child, the games the children play, the songs the people sing, the political organization, the religious observance, the art \& the philosophy.'' Other traits are ignored, discouraged, or if all else fails, ridiculed.

What is the ideal in our culture? Movies, advertisements, the design of public spaces, all tell us we should be as tough as the Terminator, as stoic as Clint Eastwood, as outgoing as Goldie Hawn. We should be pleasantly stimulated by bright lights, noise, a gang of cheerful fellows hanging out in a bar. If we are feeling overwhelmed \& sensitive, we can always take a painkiller.

If you remember only 1 thing from this book, it should be the following research study. Xinyin Chen \& Kenneth Rubin of the University of Waterloo in Ontario, Canada, \& Yuerong Sun of Shanghai Teachers University compared 480 schoolchildren in Shanghai to 296 in Canada to see what traits made children most popular. In China ``shy'' \& ``sensitive'' children were among those most chosen by others to be friends or playmates. (In Mandarin, the word for shy or quiet means good or well-behaved; sensitive can be translated as ``having understanding,'' a term of praise.) In Canada, shy \& sensitive children were among the least chosen. Chances are, this is the kind of attitude you faced growing up.

Think about the impact on you of not being the ideal for your culture. It has to affect you -- not only how others have treated you but how you have come to treat yourself.

\textbf{Shedding the Majority's Rule.}
\begin{enumerate}
	\item \textit{What was your parents' attitude toward your sensitivity?} Did they want you to keep it or lose it? Did they think of it as an inconvenience, as shyness, unmanliness, cowardice, a sign of artistic ability, cute? What about your other relatives, your friends, your teachers?
	\item \textit{Think about the media, especially in childhood}. Who were your role models \& idols? Did they seem like HSPs? Or were they people you now see you could never be like?
	\item \textit{Consider your resulting attitude}. How has it affected your career, romantic relationships, recreational activities, \& friendships?
	\item \textit{How are you as an HSP being treated now by the media?} Think about positive \& negative images of HSPs. Which predominate? (Note that when someone is a victim in a movie or book, he or she is often portrayed as by nature sensitive, vulnerable, overaroused. This is good for dramatic effect, because the victim is visibly shaken \& upset, but bad for HSPs, because ``victim'' comes to be equated with sensitivity.)
	\item \textit{Think about how HSPs have contributed to society}. Look for examples you know personally or have read about. Abraham Lincoln is probably a place to start.
	\item \textit{Think about your own contribution to society}. Whatever you are doing -- sculpting, raising children, studying physics, voting -- you tend to reflect deeply on the issues, attend to the details, have a vision of the future, \& attempt to be conscientious.'' -- \cite[pp. 47--50]{Aron2013}
\end{enumerate}

\subsection{Psychology's Bias}
``Psychological research is gaining valuable insights about people, \& much of this book is based on those findings. But psychology is not perfect. It can only reflect the biases of the culture from which it comes. I could give example after example of research in psychology that reflects a bias that people I call HSPs are less happy \& less mentally healthy, even less creative \& intelligent (the 1st two are definitely not true). However, I will save these examples for reeducating my colleagues. Just be careful about accepting labels for yourself, such as ``inhibited,'' ``introverted,'' or ``shy.'' As we move on, you'll understand why each of these \textit{mis}labels you. In general, they miss the essence of the trait \& give it a negative tone. E.g., research has found that most people, quite wrongly, associate introversion with poor mental health. When HSPs identify with these labels, their confidence drops lower, \& their arousal increases in situations in which people thus labeled are expected to be awkward.

It helps to know that in cultures in which the trait is more valued, such as Japan, Sweden, \& China, the research takes on a different tone. E.g., Japanese psychologists seem to expect their sensitive subjects to perform better, \& they do. When studying stress, Japanese psychologists see more flaws in the way that the nonsensitive cope. There is no point in blaming our culture's psychology or its well-meaning researchers, however. They are doing their best.'' -- \cite[pp. 50--51]{Aron2013}

\subsection{Royal Advisors \& Warrior Kings}
``For better \& worse, the world is increasingly under the control of aggressive cultures -- those that like to look outward, to expand, to compete \& win. This is because, when cultures come in contact, the more aggressive ones naturally tend to take over.

How did we get into this situation? For most of the world, it began on the steppes of Asia, where the Indo-European culture was born. Those horse-riding nomads survived by expanding their herds of horses \& cattle, mainly by stealing the herds \& lands of others. They entered Europe about 7000 years ago, reaching the Middle East \& South Asia a little later. Before their arrival there was little or no warfare, slavery, monarchy, or domination of 1 class by another. The newcomers made serfs or slaves out of the people they found, the ones without horses, built walled cities where there had been peaceful settlements, \& set out to expand into larger kingdoms or empires through war or trade.

The most long-lasting, happy Indo-European cultures have always used 2 classes to govern themselves -- the warrior-kings balanced by their royal or priestly advisors. \& Indo-European cultures have done well for themselves. Half of the world speaks an Indo-European language, which means they cannot help but think in an Indo-European way. Expansion, freedom, \& fame are good. Those are the values of the warrior-kings.

For aggressive societies to survive, however, they always need that priest-judge-advisor class as well. This class balances the kings \& warriors (as the U.S. Supreme Court balances the president \& his armed forces). It is a more thoughtful group, often acting to check the impulses of the warrior-kings. Since the advisor class often proves right, its members are respected as counselors, historians, teachers, scholars, \& the upholders of justice. They have the foresight, e.g., to look out for the well-being of those common folks on whom the society depends, those who grow the food \& raise the children. They warn against hasty wars \& bad use of the land.

In short, a strong royal advisor class insists on stopping \& thinking. \& it tries, I think with growing success in modern times, to direct the wonderful, expansive energy of their society away from aggression \& domination. Better to use that energy for creative inventions, exploration, \& protection of the planet \& the powerless.

HSPs tend to fill that advisor role. We are the writers, historians, philosophers, judges, artists, researchers, theologians, therapists, teachers, parents, \& plain conscientious citizens. What we bring to any of these roles is a tendency to think about all the possible effects of an idea. Often we have to make ourselves unpopular by stopping the majority from rushing ahead. Thus, to perform our role well, we have to feel very good about ourselves. We have to ignore all the messages from the warriors that we are not as good as they are. The warriors have their bold style, which has its value. But we, too, have our style \& our own important contribution to make.'' -- \cite[pp. 51--52]{Aron2013}

\subsection{The Case of Charles}
``Charles was 1 of the few HSPs I interviewed who had known he had sensitive his whole life \& always saw it as a good thing. His unusual childhood \& its consequences are a fine demonstration of the importance of self-esteem \& of the effect of one's culture.

Charles is happily married for the 2nd time \& enjoys a well-paid \& admirable academic career of service \& scholarship. In this leisure time he is a pianist of exceptional talent. \& he has a deep sense that these gifts are more than sufficient to give meaning to his life. After hearing all of this at the outset of our interview, I was, naturally, curious about his background.

Here is Charles's 1st memory. (I always ask this in my interviews -- even if inaccurate, what is recalled usually sets the tone or provides the theme of the whole life.) He is standing on a sidewalk at the back of a crowd that is admiring a window filed with Christmas decorations. He cries out, ``Everyone away, I want to see.'' They laugh \& let him come to the front.

What confidence! This courage to speak up so boldly surely began at home.

Charles's parents were delighted by his sensitivity. In their circle of friends -- their artistic, intellectual subculture -- sensitivity was associated with particular intelligence, good breeding, \& fine tastes. Rather than being upset that he studied so much instead of playing games with other boys, his parents encouraged him to read even more. To them, Charles was the ideal son.

With this background, Charles believed in himself. He knew he had absorbed excellent aesthetic tastes \& moral values at an early age. He did not see himself as flawed in any way. He did eventually realize he was unusual, part of a minority, but his entire subculture was unusual, \& it had taught him to see that subculture as superior, not inferior. He had always felt confident among strangers, even when he was enrolled in the best preparatory schools, followed by an Ivy League university, \& then took a position as a professor.

When I asked Charles if he saw any advantages of the trait, he had no trouble reciting many. E.g., he was certain it contributed to his musical ability. It had also helped him deepen his self-awareness during several years of psychoanalysis.

As for the disadvantages of the trait \& his way of making peace with them, noise bothers him a great deal, so he lives in a quiet neighborhood, surrounding himself with lovely \& subtle sounds, including a fountain in his backyard \& good music. He has deep emotions that can lead to occasional depression, but he explores \& resolves his feelings. He knows he takes things too hard but tries to allow for that.

His experience of overarousal is mainly of an intense physical response, the aftermath of which can prevent him from sleeping. But usually he can handle it in the moment through self-control, by ``comporting myself a certain way.'' When matters at work overwhelm him, he leaves as soon as he is not needed \& ``walks it out'' or plays the piano. He deliberately avoided a business career because of his sensitivity. When he was promoted to an academic position that stressed him too much, he changed positions as soon as he could.

Charles has organized his life around his trait, maintaining an optimal level of arousal without feeling in any way flawed for doing so. When I asked, as I usually do, what advice he would give others, he said, ``Spend enough time putting yourself out there in the world -- your sensitivity is not something to be feared.'' -- \cite[pp. 52--54]{Aron2013}

\subsection{A Reason for Great Pride}
``This 1st chapter may have been very stimulating! All sorts of strong, confusing feelings could be arising in you by now. I know from experience, however, that as you read \& work through this book, those feelings will become increasingly clear \& positive.

To sum it up again, you pick up on the subtleties that others miss \& so naturally you also arrive quickly at the level of arousal past which you are no longer comfortable. That 1st fact about you could not be true without the 2nd being true as well. It's a package deal, \& a very good package.

It's also important that you keep in mind that this book is about both your personal innate physical trait \& also about your frequently unappreciated social importance. You were born to be among the advisors \& thinkers, the spiritual \& moral leaders of your society. There is every reason for pride.'' -- \cite[p. 54]{Aron2013}

\subsection{Working With What You Have Learned: Reframing Your Reactions to Change}
``At the end of some chapters I will ask you to ``reframe'' your experiences in the light of what you now know. Reframing is a term from cognitive psychotherapy which simply means seeing something in a new way, in a new context, with a new frame around it.

Your 1st reframing task is to think about 3 major changes in your life that you remember well. HSPs usually respond to change with resistance. Or we try to throw ourselves into it, but we still suffer from it. We just don't ``do'' change well, even good changes. That can be the most maddening. When my novel was published \& I had to go to England to promote it, I was finally living a fantasy I had cherished for years. Of course, I got sick \& hardly enjoyed a minute of the trip. At the time, I thought I must be neurotically robbing myself of my big moment. Now, understanding this trait, I see that the trip was just too exciting.

My new understanding of that experience is exactly what I mean by reframing. So now it is your turn. Think of 3 major changes or surprises in your life. Choose one -- a loss or ending -- that seemed bad at the time. Choose one that seems as if it should have been neutral, just a major change. \& one that was good, something to celebrate or something done for you \& meant to be kind. Now follow these steps for each.
\begin{enumerate}
	\item \textit{Think about your response to the change \& how you have always viewed it}. Did you feel you responded ``wrong'' or not as others would have? Or for too long? Did you decide you were no good in some way? Did you try to hide your upset from others? Or did others find out \& tell you that you were being ``too much''?
	
	Here's an example of a negative change. Josh is 30 now, but for more than 20 years he has carried a sense of shame from when, in the middle of 3rd grade, he had to go to a new elementary school. He had been well enough liked at his old school for his drawing ability, his sense of humor, his funny choices of clothes \& such. At the new school these same qualities made him the target of bullying \& teasing. He acted as if he didn't care, but deep inside he felt awful. Even at 30, in the back of his mind he wondered if he hadn't deserved to be so ``unpopular.'' Maybe he really was odd \& a ``weakling.'' Or else why hadn't he defended himself better? Maybe it was all true.
	\item \textit{Consider your response in the light of what you know now about how your body automatically operates}. In the case of Josh I would say that he was highly aroused during those 1st weeks at the new school. It must have been difficult to think up clever kid stuff to say, to succeed in the games \& classroom tasks by which other children judge a new student. The bullies saw him as an easy target who could make them appear tougher. The others were afraid to defend him. He lost confidence \& felt flawed, not likable. This intensified his arousal when he tried anything new while others were around. He could never seem relaxed \& normal. It was a painful time but nothing to be ashamed of.
	\item \textit{Think if there's anything that needs to be done now}. I especially recommend sharing your new view of the situation with someone else -- provided they will appreciate it. Perhaps it could even be someone who was present at the time who could help you continue to fit details into the picture. I also advocate writing down your old \& new views of the experience \& keeping them around for a while as a reminder.'' -- \cite[pp. 54--56]{Aron2013}
\end{enumerate}

%------------------------------------------------------------------------------%

\section{Digging Deeper}
\textbf{Understanding Your Trait for All That It Is.} ``Now let's rearrange your mental furniture \& make it impossible for you to doubt the reality of your trait. This is important, for the trait has been discussed so little in the field of psychology. We'll look at a case history as well as scientific evidence, most of it from studying children's temperaments, which makes it all the more fitting that the case history is a tale of 2 children.'' -- \cite[p. 57]{Aron2013}

\subsection{Observing Rob \& Rebecca}
``About the time I began studying high sensitivity, a close friend gave birth to twins -- a boy, Rob, \& a girl, Rebecca. From the 1st day one could sense a difference between them, \& I understood exactly what it was. The scientist in me was delighted. Not only would I watch a highly sensitive child growing up, but Rob came with his own ``control group,'' or comparison, his sister, Rebecca, born into exactly the same environment.

A particular benefit of knowing Rob from birth was that it dispelled any doubts I had about the trait being inheritable. While it is true that he \& his sister were also treated differently from the start, at 1st that was largely because of his sensitivity, a difference he brought into the world. (Being different genders, Rob \& Rebecca are fraternal twins, not identical, which means that their genes are no more similar than are those of any brother's \& sister's.)

To add frosting to this psychologist's cake, the genders associated with sensitivity were switched. The boy, Rob, was the sensitive one; the girl, Rebecca, was not. The stereotypes were also reversed in that Rob was smaller than Rebecca.

As you read about Rob, don't be surprised if you experience an emotional response. The whole point of my description is that some of it may also apply to you. Thus, vague memories, or feelings from before you can remember what the feelings were attached to, may return. Be easy about such feelings. Just observe them. In fact, it might be helpful to write them down. It will be useful information as you read \& work through the next few chapters.'' -- \cite[pp. 57--58]{Aron2013}

\subsection{Sleep Troubles}
``In the 1st few days after Rob \& Rebecca were born, the differences in temperament were greatest when the infants were tired. Rebecca would fall asleep easily \& not wake up. Especially as a result of some change -- visitors, travel -- Rob would stay awake \& cry. Which would mean that Mom or Dad would have to walk, rock, sing, or pat him, trying to bring him to a peaceful state.

With a slightly older sensitive child, current advice is to put the child to bed \& let the quiet \& dark gradually temper the overstimulation that is the true cause of the crying. HSPs know all about being ``too tired to sleep.'' They are actually too \textit{frazzled} to sleep.

Leaving a newborn to scream for an hour, however, is more than most parents can bear, probably because it is not really very wise to do so. A newborn is usually best soothed by motion. In Rob's case, his parents finally found that an electric swing induced sleep best.

Then came the problem of his remaining asleep. There are always points in anyone's sleep cycle that make it very easy or difficult to be awakened, but sensitive children seem to have fewer periods of deep, imperturbable sleep. \& once awake, they have greater difficulty going back to sleep. (Remember, this was probably also true of you, whether you remember or not.) My own solution, with our highly sensitive child, was to use blankets to cover his crib. In his little tent all was quiet \& cozy, especially if we were laying him down in an unfamiliar place. Sometimes sensitive children really force their parents to be both empathetic \& creative.'' -- \cite[pp. 58--59]{Aron2013}

\subsection{1 Night, 2 Kids}
``When Rob \& Rebecca were almost 3, their little brother was born. My husband \& I visited for the night \& slept in the bed of their parents, who were at the hospital. We had been warned that Rob might wake up at least once, frightened by a bad dream. (He had many more of them than his sister -- HSPs often do.)

As expected, at 5 in the morning Rob wandered in, crying softly. When he saw the wrong people in his parents' bed, his sleepy moans became screams.

I have no idea what his mind envisioned. Perhaps ``Danger! Mother is gone! Horrible beings have taken her place!''

Most parents agree that everything gets easier once a child can understand words. This is so much more true with a highly sensitive child, caught up in his own imagination. The trick was to slip some quick, soothing words of mine in between his sobs.

Fortunately, Rob has a great sense of humor. So I reminded him of a recent evening when I had baby-sat \& served the 2 of them cookies as ``appetizers,'' \textit{before} dinner.

He gulped \& stared, then smiled. \& somewhere in his brain, I moved from the category of Monster Who Has Taken Mother to Silly Elaine.

I asked him if he wanted to join us, but I knew he would choose his own bed. Soon he was back there, sleeping soundly.

In the morning Rebecca came in. When she saw that her parents were gone, she smiled \& said, ``Hi, Elaine. Hi, Art,'' \& walked out. That is the difference in the non-HSP.

It is painful to imagine what would have happened if I had been the sort to have shouted at Rob to shut up \& get back to bed. He probably would have done just that, feeling abandoned in a dangerous world. But he would not have slept. His intuitive mind would have elaborated on the experience for hours, including probably deciding he was somehow to blame. With sensitive children, physical blows or traumas aren't required to make them afraid of the dark.'' -- \cite[p. 59]{Aron2013}

\subsection{Rounding Out Our Picture of Rob}
``By day, when the twins went out with their parents during that 1st year, the mariachi band at the Mexican restaurant fascinated Rebecca; it made Rob cry. In their 2nd year, Rebecca was delighted by ocean waves, haircuts, \& merry-go-rounds; Rob was afraid of them, at least at 1st, just as he was on the 1st day of nursery school \& with the stimulation accompanying each birthday \& holiday. Furthermore, Rob developed fears -- of pinecones, of figures printed on his bedspread, of shallows on the wall. The fears were strange \& unrealistic to us, but they were certainly real to him.

In short, Rob's childhood has been a little difficult for him \& for his caring, stable, competent parents. Actually, unfair as it is, the difficult aspects of any temperament are displayed more when the home environment is sound. Otherwise, in order to survive, an infant will do whatever he or she must to adapt to the caretakers, with temperament going underground to resurface in some other way later, perhaps in stress-related physical symptoms. But Rob is free to be who he is, so his sensitivity is out there for all to see. He can express his feelings, \& as a result he can learn what does \& does not work.

E.g., during his 1st 4 years, when Rob was overwhelmed, he would often burst into angry tears. At these times, his parents would patiently help him contain his feelings. \& with every month he seemed better able to not become overwhelmed. When watching a movie with scary or sad sequences, e.g., he learned to tell himself what his parents would say: ``It's just a movie,'' or ``Yeah, but I know it ends okay.'' Or he would close his eyes \& cover his ears or leave the room for a little while.

Probably because he is more cautious, he has been slower to learn some physical skills. With other boys he is less comfortable with wilder, rougher play. But he wants to be like them \& tries, so he is accepted. \& thanks to careful attention to his adjustment, thus far he likes school a great deal.

There are some other points about Rob that are not surprising, given his trait: He has an extraordinary imagination. He is drawn to everything artistic, especially music (true for many HSPs). He is funny \& a great ham when he feels at home with his audience. Sine he was 3 he was ``thought like a lawyer,'' quick to notice fine points \& make subtle distinctions. He is concerned about the suffering of others \& polite, kind, \& considerate -- except, perhaps, when he is overcome by too much stimulation. His sister, meanwhile, has her own numerous virtues. One is that she is a steady sort, the anchor in her brother's life.

What makes Rob \& Rebecca so very different from each other? What makes you answer yes to so many items on the self-test at the beginning of this book when most people would not?'' -- \cite[pp. 59--61]{Aron2013}

\subsection{You Are Truly a Different Breed}
``Jerome Kagan, a psychologist at Harvard, has devoted much of his career to the study of this trait. For him it is as observable a difference as hair or eye color. Of course, he calls it other names -- inhibitedness, shyness, or timidity in children -- \& I cannot agree with his terms. But I understand that from the outside, \& especially in a laboratory setting, the children he studies do seem mainly inhibited, shy, or timid. Just remember as I discuss Kagan that sensitivity is the real trait \& that a child standing still \& observing others may be quite \textit{un}inhibited inside in his or her processing of all the nuances of what is being seen.

Kagan has been following the development of 22 children with the trait. He is also studying 19 who seemed to be very ``uninhibited.'' According to their parents, as infants the ``inhibited'' children has had more allergies, insomnia, colic, \& constipation than the average child. As young children, seen in the laboratory for the 1st time, their heartbeat rates are generally higher \& under stress show less change. (Heart rate can't change much if it is already high.) Also when under stress, their pupils dilate sooner, \& their vocal cords are more tense, making their voice change to a higher pitch. (Many HSPs are relieved to know why their voice can become so strange sounding when they are aroused.)

The body fluids (blood, urine, saliva) of sensitive children show indications of high levels of norepinephrine present in their brains, especially after the children are exposed to various forms of stress in the laboratory. Norepinephrine is associated with arousal; in fact, it is the brain's version of adrenaline. Sensitive children's body fluids also contain more cortisol, both when under stress \& when at home. Cortisol is the hormone present when one is in a more or less constants state of arousal or wariness. Remember cortisol; it comes up again.

Kagan then studied infants to see which ones would grow into ``inhibited'' children. He found that about 20\% of all babies are ``highly reactive'' when exposed to various stimuli: They pump \& flex their limbs vigorously, arch their backs as if irritated or trying to get away, \& frequently cry. A year later, $\frac{2}{3}$ of the study's reactive babies were ``inhibited'' children \& showed high levels of fear in new situations. Only 10\% showed low levels. So the trait is roughly observable from birth, as was the case with Rob.

All of this suggests what I have already said -- that sensitive children come with a built-in tendency to react more strongly to external stimuli. But Kagan \& others are discovering the details that make that so. E.g., Kagan found that babies who later showed this trait also had cooler foreheads on the right side of their head, which indicates greater activity on the right side of the brain. (The blood is drawn away from the surface toward the activity.) Other studies have also found that many HSPs have more activity in the right hemisphere of the brain, especially those who stay sensitive from birth into childhood -- i.e., were clearly born that way.

Kagan's conclusion is that persons with the trait of sensitivity or inhibitedness are a special breed. They are genetically quite different, although still utterly human, just as bloodhounds \& border collies are quite different, although both are still definitely dogs.

My own research also points to the idea of a distinct genetic ``breed'' of sensitive people. In my telephone survey of 300 randomly selected people, I found both a distinct group \& also a continuum. On a scale of 1--5, about 20\% felt they were ``extremely'' or ``quite a bit'' sensitive. An additional 27\% said ``moderately.'' Together, those 3 categories seemed like a continuum. But then there was a sharp break. A measly 8\% were ``not.'' \& a whopping 42\% said they were \textit{``not at all''} sensitive, as if we were asking Laplanders about coconuts.

My sense of HSPs from meeting them is that they are indeed a distinct group, separate from the nonsensitive. Yet among them there is also a wide range in sensitivity. This may be due to there being several different causes of the trait, leading to different kinds, or ``flavors,'' of sensitivity, some of them stronger than others, or to some people being born with 2 kinds, 3 kinds, \& so on. \& there are so many ways that humans can increase or decrease their sensitivity through experience or conscious choice. All of these effects could cause a blurring of the boundary of what is still basically a separate group.

There is no denying the sense that Rob \& Rebecca are 2 different sorts of humans. You are, too. Your differences are very real.'' -- \cite[pp. 61--63]{Aron2013}

\subsection{The Brain's 2 Systems}
``A number of researchers think that there are 2 systems in the brain \& that is the balance of these 2 that creates sensitivity. 1 system, the ``behavioral activation'' (or ``approach,'' or ``facilitation,'' system) is hooked up to the parts of the brain that take in messages from the senses \& send out orders to the limbs to get moving. This system is designed to move us toward things, especially new ones. It is probably meant to keep us eagerly searching for the good things in life, like fresh food \& companionship, all of which we need for survival. When the activation system is operating, we are curious, bold, \& impulsive. The other system is called the ``behavioral inhibition'' (or ``withdrawal,'' or ``avoidance,'' system). (You can already tell by the names which is the ``good'' once according to our culture.) This system is said to move us away from things, making us attentive to dangers. It makes us alert, cautious, \& watchful for signs. Not surprisingly, this system is hooked up to all the parts of the brain Kagan noted to be more active in his ``inhibited'' children.

But what does this system really do? It takes in everything about a situation \& then automatically compares the present to what has been normal \& usual in the past \& what should be expected in the future. If there is a mismatch, the system makes us stop \& wait until we understand the new circumstance. To me this is a very significant part of being intelligent. So I prefer to give it a more positive name: the automatic pause-to-check system.

But now consider how one might have a more active pause-to-check system. Imagine Rob \& Rebecca coming to school 1 morning. Rebecca sees the same classroom, teacher, \& children as were there yesterday. She runs off to play. Rob notices that the teacher is in a bad mood, 1 of the children is looking angry, \& some bags are in the corner that were not there before. Rob hesitates \& may decide that there is reason for caution. So sensitivity -- the subtle processing of sensory information -- is the real difference once again. Notice how psychology has described the 2 systems as having opposing purposes. How like the opposition I described in the last chapter between the warrior-king class \& the royal-advisor class.

This 2-system explanation of sensitivity also suggests 2 different types of HSPs. Some might have only an average-strength pause-to-check system but an activation system that is even weaker. This kind of HSP might be very calm, quiet, \& content with a simple life. It's as if the royal advisors are monks who rule the whole country{\tt/}person. Another kind of HSP could potentially have an even stronger pause-to-check system but an activation system that is also very strong -- just not quite as strong. This kind of HSP would be both very curious \& very cautious, easily bored yet easily overaroused. The optimal level of arousal is a narrow range. One could say there is a constant power struggle between the advisor \& the impulsive, expansive warrior within the person.

I think Rob is this type. Other young children, however, are described as so quiet \& uncurious that they are in danger of being ignored \& neglected.

What type are you? Does your pause-to-check{\tt/}advisor system rule alone, thanks to a quiet activator{\tt/}warrior-king system? I.e., is it easy for you to be content with a quiet life? Or are the 2 branches that govern you in constant conflict? I.e., do you always want to be trying new things even if you know that afterward you will be exhausted?'' -- \cite[pp. 63--65]{Aron2013}

\subsection{You Are More Than Genes \& Systems}
``Let's not forget that you are a complicated being. Certain investigators, such as Mary Rothbart of the University of Oregon, are adamant that temperament is quite a different matter when you study adult humans, who can reason, make choices, \& exert willpower to follow through on their choices. Rothbart believes that if psychologists study children \& animals too much, they will overlook the role of human thinking \& a lifetime's experience.

Let's go over your development, \& Rob's, as Rothbart sees it, \& how being sensitive would differ at each stage.

At birth, an infant's only reaction is negative -- irritability, discomfort. Sensitive babies like you \& Rob were mainly different in being more irritable \& uncomfortable -- what Kagan called ``highly reactive.''

At about 2 months the behavioral-activation system becomes functional. Now you showed an interest in new things in case they might satisfy your needs. Along with that came a new feeling -- anger \& frustration when you did not get what you wanted. So positive emotions \& anger were possible, \& how much you felt them depended on the strength of your activation system. Rob, having both systems strong, became an easily angered baby. But sensitive babies with a low activation system would be placid \& ``good'' at this age.

At 6 months your superior automatic pause-to-check system came on line. You could compare present experiences with those of the past, \& if the present ones were upsetting, as those in the past, you would experience fear. But again, you saw more subtle differences in each experience. For you there was more that was unfamiliar \& possibly frightening.

At this point, 6 months, every experience becomes very important for HSPs. One can see how a few bad experiences when approaching new things could turn the pause-to-check system into a pause-\&-do-nothing system, a true inhibition system. The best way to avoid bad things would seem to be avoiding everything. \&, of course, the more the world is avoided, the newer everything will seem. Imagine how frightening the world could have seemed to you.

Finally, around 10 months, you began to develop the ability to shift your attention, to decide how to experience something, or to stop a behavior. Only at this point could you start to handle conflicts between the 2 systems. A conflict would be \textit{I want to try that, but it seems so strange}. (At 10 months we might not use those words, but that would be the idea.) But now you could make some choices about which emotion to obey. One could almost see Rob doing it: \textit{Okay, it's unfamiliar, but I'll go ahead, anyway}.

You probably had favorite methods of overriding the pause-to-check system if it slowed you too long or often. 1 way might have been to imitate those with less of it. You just went ahead \& got some good things, too, like them, in spite of your caution. Another might have been the recategorizing of the stimulation to make it familiar. The growling wolf in the move ``is just a big dog.'' But most of your help probably came from others who wanted you to feel safe, not afraid.

Social help with fears involves yet another system that Rothbart believes is highly developed in adult humans. It also arrives at about 10 months. With it, a child begins to connect with others, to enjoy them. If these social experiences are positive \& supportive, another physiological system develops for which humans are biologically prepared. One could call this the loving system. It creates endorphins, the ``good feeling'' neurochemicals.

How much could could you overcome your fears by trusting others to help? Who was around whom you could rely on? Did you act as if \textit{Mother is here so I'll try?} Did you learn to imitate her calming words \& deeds, applying them to yourself? ``Don't be afraid, it'll be okay.'' I have seen Rob using all of these methods.

Now you might spend a moment thinking about yourself \& your childhood, \& we will do more of this in the next 2 chapters. I know you don't really remember, but judging from what facts you have, what was that 1st year \textit{probably} like? How does your thinking \& self-control affect your sensitivity now? Are there times when you can control your arousal? Who taught you to do so? Who were your role models? Do you think you were taught to control your cautiousness too much so that you dare to do more than your body can handle? Or does it seem that your lesson was that the world is unsafe \& overarousal is uncontrollable?'' -- \cite[pp. 65--67]{Aron2013}

\subsection{How Trust Becomes Mistrust \& the Unfamiliar Becomes Dangerous}
``Most researchers on temperament have studied short-term arousal. It's easy to study, for it's quite apparent from the higher levels of heartbeat, respiration, perspiration, pupil dilation, \& adrenaline.

There is another system of arousal, however, that is governed more by hormones. It goes into action just as quickly, but the effect of its main product, cortisol, is most noticeable after 10 to 20 minutes. An important point is that when cortisol is present, the short-term arousal response is also even more likely. I.e., this long-term type of arousal makes us even more excitable, more sensitive, than before.

Most of the effects of cortisol occur over hours or even days. There are mainly measured in the blood, saliva, or urine, so studying long-term arousal is less convenient. But psychologist Megan Gunnar of the University of Minnesota thought that the whole point of the pause-to-check system might be to protect the individual from this unhealthy, unpleasant, long-term arousal.

Research shows that when people 1st encounter something new \& potentially threatening, the short-term response always comes 1st. Meanwhile, we start to consider our resources. What are our abilities? What have we learned about this sort of situation from past experiences? Who is around who might help out? If we think we or those with us can cope with the situation, we stop seeing it as a threat. The short-term alert dies out, \& the long-term alarm never goes off.

Gunnar demonstrated this process in an interesting experiment. She set up a threatening situation much like those Kagan uses to identify ``inhibited'' children. But 1st, the 9-month-old babies were separated from their mothers for a half hour. Half were left with a very attentive baby-sitter who responded to all of the child's moods. The other half were left with a baby-sitter who was inattentive \& unresponsive unless the child actually fussed or cried. Next, while alone with the baby-sitter, each 9-month-old was exposed to something startlingly new.

What is so important here is that only the highly sensitive babies with the inattentive baby-sitters showed more cortisol in their saliva. It was as if those with the attentive sitter felt they had a resource \& had no need to make a long-term stress response.

Suppose the caretaker is your own mother? Psychologists observing babies with their mothers have discovered certain signs that tell them if a child feels ``securely attached.'' A secure child feels safe to explore, \& new experiences are not usually seen as a threat. Other signs indicate that a child is ``insecurely attached.'' The mothers of these children are either too protective or too neglectful (or even dangerous). (We will discuss ``attachment'' more in Chaps. 3--4). Research on sensitive children facing a novel, startling situation in the company of their mothers has found that these children do show their usual, strong short-term response. But if a sensitive child is securely attached to Mom, there is no long-term cortisol effect from the stress. Without secure attachment, however, a startling experience will produce long-term arousal.

One can see why it is important that young HSPs (\& older ones, too) stay out in the world, trying things rather than retreating. But their feelings about their caretakers have to be secure \& their experiences have to be successful or their reasons not to approach will only be proved true. \& all of this gets started before you can even talk!

Many intelligent, sensitive parents provide all the needed experiences almost automatically. Rob's parents are constantly praising his successes \& encouraging him to test his fears to see if they are realistic while offering help if he needs it. With time, his idea of the world will be that it is not as frightening as his nervous system was telling him it was during that 1st year or 2. His creative traits \& intuitive abilities, all the advantages of being sensitive, will flourish. The difficult areas will fade.

When parents do nothing special to help a sensitive child feel safe, whether the child becomes truly ``inhibited'' probably depends on the relative strength of the activation \& pause-to-check systems. But remember that some parents \& environments can make matters much worse. Certainly repeated frightening experiences will strongly reinforce caution, especially experiences of failing to be calmed or helped, of being punished for active exploring, \& of having others who should be helpful become dangerous instead.

Another important point is that the more cortisol in an infant's body, the less the child will sleep, \& the less sleep, the more cortisol. In the daytime, the more cortisol, the more fear, the more fear, the more cortisol. Uninterrupted sleep at night \& timely naps all reduce cortisol in infants. \& remember, lower cortisol also means fewer short-term alarms. It was easy to see that this was a constant problem with Rob. It may have been for you, too.

Furthermore, if sleep problems beginning in infancy are not controlled, they may last into adulthood \& make a highly sensitive person almost unbearably sensitive. So get your help!'' -- \cite[pp. 67--70]{Aron2013}

\subsection{Into the Depths}
``There is another aspect of your trait that is harder to capture in studies or observations -- except when strange fears \& nightmares visit the sensitive child (or adult). To understand this very real aspect of the trait, one leaves the laboratory \& enters the consulting room of the depth psychologist.

Depth psychologists place great emphasis on the unconscious \& the experiences imbedded there, repressed or simply preverbal, that continue to govern our adult life. It is not surprising that highly sensitive children, \& adults, too, have a hard time with sleep \& report more vivid, alarming, ``archetypal'' dreams. With the coming of darkness, subtle sounds \& shapes begin to rule the imagination, \& HSPs sense them more. There are also the unfamiliar experiences of the day -- some only half-noticed, some totally repressed. All of them swirl in the mind just as we are relaxing the conscious mind so that we can fall asleep.

Falling asleep, staying asleep, \& going back to sleep when awakened require an ability to soothe oneself, to feel safe in the world.

The only depth psychologist to write explicitly about sensitivity was 1 of the founders of depth work, Carl Jung, \& what he said was important -- \& exceptionally positive, for a change.

Way back when psychotherapy began with Sigmund Freud, there was controversy about how much innate temperament shaped personality, including emotional problems. Before Freud, the medical establishment had emphasized inherited constitutional differences. Freud tried to prove that ``neurosis'' (his specialty) was caused by traumas, especially upsetting sexual experiences. Carl Jung, Freud's follower for a long time, split with him finally on the issue of the centrality of sexuality. Jung decided that the fundamental difference was an inherited greater sensitivity. He believed that when highly sensitive patients had experienced a trauma, sexual or otherwise, they had been unusually affected \& so developed a neurosis. Note that Jung was saying that sensitive people are not traumatized in childhood are not inherently neurotic. One thinks of Gunnar's finding that the sensitive child with a secure attachment to his or her mother does not feel threatened by new experiences. Indeed, Jung thought very highly of sensitive people -- but then he was one himself.

That Jung wrote about HSPs is a little-known fact. (I did not know this when I began my work on the trait.) E.g., he said that ``a certain innate sensitiveness produces a special prehistory, a special way of experiencing infantile events'' \& that ``events bound up with powerful impressions can never pass off without leaving some trace on sensitive people.'' Later, Jung began to describe introverted \& intuitive types in similar ways, but even more positively. He said they had to be more self-protective -- what he meant by being introverted. But he also said that they were ``educators \& promoters of culture $\ldots$ their life teaches the other possibility, the interior life which is so painfully wanting in our civilization.''

Such people, Jung said, are naturally more influenced by their unconscious, which gives them information of the ``utmost importance,'' a ``prophetic foresight.'' To Jung, the unconscious contains important wisdom to be learned. A life lived in deep communication with the unconscious is far more influential \& personally satisfying.

But such a life is also potentially more difficult, especially if in childhood there were too many disturbing experiences without a secure attachment. As you saw from Gunnar's research \& as you will see in Chap. 8, Jung was exactly right.'' -- \cite[pp. 70--72]{Aron2013}

\subsection{So It's Real \& It's Okay}
``Rob, Jerome Kagan, Megan gunnar, \& Carl Jug should have you well convinced now that your trait is utterly real. You are different. In the next chapter, you will consider how you may need to live differently from others if you are going to be in healthy harmony with your quite different, highly sensitive body.

By now you may be seeing a somewhat dark picture too -- one of fear, timidity, inhibitedness, \& distressed overarousal. Only Jung spoke of the trait's advantages, but even then it was in terms of our connection to the depths \& darkness of the psyche. But remember that this sort of negativity is, once again, largely a sign of our culture's bias. Preferring toughness, the culture sees our trait as something difficult to live with, something to be cured. Do not forget that HSPs differ mainly in their sensitive processing of subtle stimuli. This is your most basic quality. That is a positive \& accurate way to understand your trait.

\textbf{Working \textit{With} What You \textit{Have Learned: Your Deeper Response}.} This is something to do right now, just as you have finished reading this chapter. Your intellect has taken in some ideas, but your emotions may be having some deeper reactions to what you have been reading.

To reach these deeper reactions, you need to reach the deeper parts of the body, of your emotions, of the more fundamental, instinctual sort of consciousness that Jung called the \textit{unconscious}. This is when the ignored or forgotten parts of yourself dwell, areas that may be threatened or relieved or excited or saddened by what you are learning.

Read all of what is here: then proceed. Begin by breathing very consciously from the center of your body, from your abdomen. Make certain that your diaphragm stays involved -- at 1st blow out through your mouth fairly hard, as if blowing up a balloon. Your belly will tighten as you do this. Then, when you inhale, the breath will be taken in from the level of of your stomach, very automatically. Your breathing in should be automatic \& easy. Only your breathing out should be extended. That, too, can become less forceful \& no longer out through your mouth once you are settled into breathing from your center, your belly, \& not from high up in your chest.

Once settled, you need to create a safe space within your imagination where anything at all is welcome. Invite any feeling to enter awareness there. It might be a bodily feeling -- an ache in the back, a tension in your throat, an unsettled stomach. Let the sensation grow \& let it tell you what it is there to show you. You also might see a fleeting image. Or hear a voice. Or observe an emotion. Or a series of these -- a physical feeling might become an image. Or a voice might express an emotion you begin to feel.

Notice all that you can in this quiet state. If feelings need to be expressed -- if you need to laugh, cry, or rage -- try to let yourself do that a little.

Then, as you emerge from that state, think about what happened. Note what stirred the feelings you had -- what it was in what you read, what it was in what you thought or remembered while you read. How were your feelings related to being sensitive?

Afterward, put into words some of what you have learned -- think about it for yourself, tell someone else, or write it down. Indeed, keeping a journal of your feelings while you read this book will be very helpful.'' -- \cite[pp. 72--73]{Aron2013}

%------------------------------------------------------------------------------%

\section{General Health \& Lifestyle for HSPs}
\textbf{Loving \& Learning: From Your Infant{\tt/}Body Self.} ``In this chapter you'll learn to appreciate your highly sensitive body's needs. Since this is often surprising difficult for HSPs, I have learned to approach it through a metaphor -- treating the body as you would an infant. It is such a good metaphor, as you will see, that it may not be one at all.'' -- \cite[p. 74]{Aron2013}

\subsection{6 Weeks of Age: How It May Have Been}
``A storm threatens. The sky turns metallic. The march of clouds across the sky breaks apart. Pieces of sky fly off in different directions. The wind picks up force, in silence $\ldots$ The world is disintegrating. Something is about to happen. Uneasiness grows. It spreads from the center \& turns into pain.

The above is a moment of growing hunger as experienced by a hypothetical 6-week-old infant called Joey, as imagined by developmental psychologist Daniel Stern in his charming book \textit{Diary of a Baby}. A tremendous amount of recent research on infancy informs Joey's diary. E.g., it is now thought that infants cannot separate inner from outer stimulation or sort out the different senses or the present from a remembered experience that has just happened. Nor do they have a sense of themselves as the one who is experiencing it all, the one to whom it is happening.

Given all of the above, Stern found that weather is a good analogy for an infant's experience. Things just happen, varying mostly in intensity. Intensity is all that disturbs, by creating a storm of overarousal. HSPs take note: Overarousal is the 1st \& most basic distressing experience of life; our 1st lessons about overarousal begin at birth.

Here is how Stern imagines Joey feeling after he has nursed \& eased his hunger:
\begin{quotation}
	All is remade. A changed world is waking. The storm has passed. The winds are quiet. The sky is softened. Running lines \& flowing volumes appear. They trace a harmony \&, like shifting light, make everything come alive.
\end{quotation}
Stern sees infants as having the same needs as adults for a moderate level of arousal:
\begin{quotation}
	A baby's nervous system is prepared to evaluate immediately the intensity of $\ldots$ anything accessible to 1 of his senses. How intensely he feels about something is probably the 1st clue he has available to tell him whether to approach it or to stay away $\ldots$ if something is moderately intense $\ldots$ he is spellbound. That just-tolerable intensity arouses him $\ldots$ It increases his animation, activates his whole being.
\end{quotation}
In other words, it is no fun to be bored. On the other hand, the infant{\tt/}body self is born with an instinct to stay away from whatever is highly intense, to avoid the state of overarousal. For some, however, it's harder to do.'' -- \cite[pp. 74--75]{Aron2013}

\subsection{6 Weeks \& Highly Sensitive}
``Now I will try my own hand at this new literary genre of infant diary, with the experience of an imaginary, highly sensitive infant, Jesse.

The wind has been blowing incessantly, sometimes gusting into a howling gale, sometimes falling to an edgy, exhausting moan. For a seeming eternity clouds have swirled in random patterns of blinding light \& glowering dark. Now an ominous dusk is descending, \& for a moment the wind seems to ebb with the light. 

But the darkness is disorienting in itself, \& the howling wind begins to shift directions indecisively, as it might in the region of tornadoes. Indeed, out of this rising chaos the veerings do take a shape, gaining energy from one another, until a cyclonic fury emerges. A hellish hurricane is happening in deepest night.

There is some place or time where this awfulness stops, but there is no way to find that haven, for this weather has neither up nor down, east nor west -- only round \& round toward the fearful center.

I imagined the above happening after Jesse had gone with his mother \& 2 sisters to the shopping mall, riding in his car seat, then a stroller, then back home in the car seat. It was a Saturday, \& the mall was jammed. On the way home his 2 sisters had a fight about which radio station to listen to, each of them turning the volume louder. There was considerable traffic, many stops \& starts. They returned home late, long after Jesse's usual nap time. When offered a chance to nurse, he only cried \& fussed, too overwhelmed to attend to his vaguer sense of hunger. So his mother tried putting him down to sleep. That is when the hurricane finally hit.

We should not forget that Jesse was hungry, too. Hunger is yet another stimulus, from inside. Besides arousing one further, it produces a diminution of the biochemical substances necessary for the usual, calmer functioning of the nervous system. My research indicates that hunger has an especially strong effect on HSPs. As one put it, ``Sometimes when I'm tired it's like I regress to this age where I can almost hear myself saying `I \textit{must} have my milk \& cookies, \textit{right now}.''' Yet once overaroused, we may not even notice hunger. Taking good care of a highly sensitive body is like taking care of an infant.'' -- \cite[pp. 75--76]{Aron2013}

\subsection{Why the Infant{\tt/}Body Self?}
``Think of what the infant \& the body have in common. 1st, both are wonderfully content \& cooperative when they are not overstimulated, worn out, \& hungry. 2nd, when babies \& sensitive bodies really are exhausted, both are largely helpless to correct things on their own. The baby-you relied on a caretaker to set limits \& satisfy your simple, basic needs, \& your body relies on you to do it now.

Both also cannot use words to explain their troubles; they can only give louder \& louder signals for help or develop a symptom so serious it cannot be ignored. The wise caretaker knows that much woe is avoided by responding to the infant{\tt/}body at the 1st sign of distress.

Finally, as we noted in the last chapter, caretakers who think newborn babies or bodies can be spoiled \& should be ``left to cry'' are wrong. Research demonstrates that if a small infant's crying is responded to promptly (except at those times when responding just adds to the overstimulation), that infant will cry less, not more, when older.

This infant{\tt/}body self is an expert on sensitivity. She has been sensitive from the day she was born. She knows what was hardest then, what is hard now. He knows what you lacked, what you learned from your parents \& other caretakers about how to treat him, what he needs now, \& how you can take care of him in the future. By starting here, we make use of the old adage ``Well begun is half-done.'''' -- \cite[pp. 76--77]{Aron2013}

\subsection{You \& Your Caretaker}
``About half or a little more than half of all infants are raised by adequate parents, \& thus become what is called ``securely attached'' children. The term is taken from biology. All newborn primates hang on to Mom, \& most moms want their infants to hang on tight, securely.

As the infant gets a little older, when feeling safe he or she can begin to explore \& try to do things independently. The mother will feel pleased about that -- watchful \& ready if there is trouble but otherwise glad that her little one is growing up. But there will still be a kind of invisible attachment. The moment there is danger, their bodies will reunite \& become attached again. Secure.

Now \& then, for various reasons usually having to do with how the mother or father was raised, a primary caretaker may give 1 of 2 other messages, creating an insecure attachment. One is that the world is so awful, or the caretaker is so preoccupied or vulnerable, that the infant must hang on very, very tight. The child does not dare to explore very much. Maybe the caretaker does not want exploring or would leave the infant behind if he or she did not hang on. These babies are said to be anxious about, or preoccupied with, their attachment to their caretaker.

The other message an infant may receive is that the caretaker is dangerous \& ought to be avoided or values more highly a child who is minimal trouble \& very independent. Perhaps the caretaker is too stressed to care for a child. \& there are those who at times, in anger or desperation, even want the infant to disappear or die. In that case the infant will do best not to be attached at all. Such infants are said to be avoidant. When separated from their mothers or fathers, they seem quite indifferent. (Sometimes, of course, a child is securely attached to 1 parent \& not to the other.)

From our 1st attachment experiences we tend to develop a rather enduring mental idea of what to expect from someone we are close to \& depend on. While that may seem to make for rigidity \& lost opportunities, meeting your 1st caretaker's desires about how you attached was important for your survival. Even when it ceases to be a matter of survival, the program is still there \& very conservative. Sticking to whichever plan works -- to be secure, anxious, or avoidant -- protects against making dangerous mistakes.'' -- \cite[pp. 77--78]{Aron2013}

\subsection{Attachment \& the Highly Sensitive Body}
``Remember in the last chapter the highly sensitive children who did not have long-term arousal in unfamiliar situations? They were the ones with responsive caretakers or mothers with whom they had secure relationships. This suggests that you HSPs who grew up feeling securely attached knew that you had good resources \& could handle overstimulation fairly well. Eventually, you learned to do for yourself what your good caretakers had been doing for you.

Meanwhile, your body was learning not to respond as if threatened by each new experience. \& in the absence of a response, the body did not experience distressing, long-term arousal. You found that your body was a friend to trust. At the same time, you were learning that you had a special body, a sensitive nervous system. But you could handle things by learning when to push yourself a little, when to take your time, when to back off entirely, when to rest \& try later.

Like the remainder of the population, however, about half of you had parents who were less than ideal. It is painful to think about, but we'll take up this issue slowly, returning to it several times later. But you do need to face what you may have missed. Having an inadequate parent had to have more of an impact because you were sensitive. You needed understanding, not special problems.

Those of you with an insecure childhood also need to face it so that you can be more patient with yourself. Most important, you need to know what was not done so you can be a different sort of parent to your infant{\tt/}body. Chances are that you are not taking good care of yourself -- either neglecting your body or being too overprotective \& fussy. It is almost surely because you are treating your body as your not-so-great 1st caretaker once took care of you{\tt/}it (including overreacting in the opposite way to that experience).

So let's see exactly what a good caretaker \& not-so-good caretaker of an infant{\tt/}body is like. We start with the care of the newborn -- or with your body at those times now in your life when it feels as tiny \& helpless as a newborn's. A good description of what is needed comes from the psychologist Ruthellen Josselson:
\begin{quotation}
	Enfolded in arms, we have a barrier between ourselves \& whatever might be hurtful or overwhelming in the world. In arms, we have an extra layer of protection from the world. We sense that buffer even though we may be unclear what part of it comes from ourselves \& what from outside.
	
	A good-enough mother, in her holding function, manages things so that her baby is not overstimulated. She senses how much stimulation is welcomed \& can be tolerated. An adequate holding environment leaves the baby free to develop in a state of being; the infant does not always have to react. In the state of optimal holding, the self can come into existence free of external intrusion.
	
	When holding is not adequate, when the infant{\tt/}body is intruded upon or neglected -- or worse, abused -- stimulation is too intense for the infant{\tt/}body self. Its only recourse is to stop being conscious \& present, thereby developing a habit of ``dissociating'' as a defense. Overstimulation at this age also interrupts self-development. All energy must be directed toward keeping the world from intruding. The whole world is dangerous.
\end{quotation}
Now let's consider a little later age, when you were ready to explore if you felt safe. This equates with those times now when your body is ready to explore \& be out in the world if it feels safe. At this stage an overprotective caretaker probably becomes a greater problem fro a sensitive infant{\tt/}body than a neglectful one. During infancy or when we are feeling very delicate, constant intruding \& checking on the infant{\tt/}body are sources of overstimulation \& worry. At this stage anxious overprotection inhibits exploring \& independence. An infant{\tt/}body constantly watched cannot function freely \& confidently.

E.g., just a little time feeling hunger \& crying or feeling cold \& fussing helps an infant{\tt/}body know his or her own wants. If the caretaker is feeding the infant{\tt/}body before it is even hungry, it loses contact with its instincts. \& if the infant{\tt/}body is kept from exploring, it does not get used to the world. The caretaker{\tt/}you is reinforcing the impression that the world is threatening \& the infant{\tt/}body cannot survive out there. There are no opportunities to avoid, manage, or endure overarousal. Everything remains unfamiliar \& overarousing. In terms of the previous chapter, the infant{\tt/}body does not have enough successful approach experiences to balance the strong, inherited pause-to-check system that can take over \& become too inhibiting.

If this is your style with your infant{\tt/}body, you may want to think back to its source. Perhaps you had an overprotective, needy caretaker who really wanted a child very dependent \& never able to leave. Or the caretaker's own sense of strength or self-worth has bolstered by being stronger \& so needed. If your caretaker had several children, being the most sensitive made you ideal for these purposes. Note that there were probably many times, too, that this sort of caretaker really was not available, whatever you were told -- such a caretaker was tuned into her or his needs, not yours.

The point of all this si that how others took care of you as an infant{\tt/}body has very much shaped how you take care of your infant{\tt/}body now. Their attitude toward your sensitivity has shaped your attitude toward it. Think about it. Who else could have taught as deep a lesson? Their care for you \& their attitude toward your body directly affects your health, happiness, longevity, \& contributions to the world. So unless this section of the chapter is distressing you, stop \& take some time to think about your infant{\tt/}body's 1st caretaker \& the similarities between that early caretaking \& how you care for yourself now.

If you do feel distressed, take a break. If you think you might need some professional (or perhaps nonprofessional) emotional support \& company while you look at your insecure attachment \& its affects on you now, get that help.'' -- \cite[pp. 78--81]{Aron2013}

\subsection{Out Too Much, In Too Much}
``Just as there are 2 kinds of problem caretakers -- underprotective \& overprotective -- there are 2 general ways that HSPs fail to care properly for their bodies. You may push yourself out too \textit{much} -- overstimulate yourself with too much work, risk taking, or exploring. Or you may keep yourself in too \textit{much} -- overprotecting yourself when you really long to be out in the world like others.

\textbf{Your infant{\tt/}body's 1st caretaker \& the one who cares for it now}
\begin{quotation}
	Thinking about what you know about your 1st 2 years, make a list of the sorts of words or phrases that your parents might have used to describe you as a baby. Or you can ask them. Some examples:

	A joy. Fussy. Difficult. No trouble. Never slept. Sickly. Angry. Easily tired. Smiled a lot. Difficult to feed. Beautiful. Can't recall anything about your infancy. Walked early. Most reared by a series of caretakers. Rarely left with baby-sitters or at a child care center. Fearful. Shy. Happiest alone. Always into things.
	
	Watch for the phrase that was almost your ``middle name'' -- the one they would put on your gravestone if given half a chance. (Mine was ``She never caused anybody any trouble.'') Watch for the phrases that stir up emotion, confusion, conflict in you. Or the phrases that seem too strongly emphasized, so much so that the opposite is even more true if you think about it. An example would be an asthmatic child being described as ``no trouble.''
	
	Now, think about the parallels between how your caretakers viewed your infant{\tt/}body \& how you do now. Which of their descriptions of you are really true for you? Which were really their worries \& conflicts that you could shed now? E.g., ``sickly.'' Do you still see yourself as sickly? Were you \& are you now really more sickly than others? (If so, do learn the details of your childhood illnesses -- your body remembers \& deserves your sympathy.)
	
	Or how about ``walked early.'' Are achievements \& milestones how people earned attention in your family? If your body fails to achieve to your satisfaction, can you love it, anyway?
\end{quotation}
By ``too much,'' I mean more than you would really like, more than feels good, more than your body can handle. Never mind what others have told you is ``too much.'' Some of you may be people who, at least for a period of your lives, truly belong in or out almost all the time. It feels right. Rather, I am referring to the situation where you sense you are overdoing it 1 way or the other \& would like to change but cannot. Furthermore, I do not mean to imply that those who were anxiously attached, with overprotective or inconsistent caretakers, are always overprotective of the infant{\tt/}body self. It's not that simple. 1st, our minds are s.t. we can as easily overreact or compensate \& do the opposite. Or, more likely, we'll swing back \& forth between the 2 extremes or apply them in different areas of life (e.g., overdo at work, protect too much in intimate relationships; neglect mental health but overattend to physical health). Finally, you may have overcome all of this \& be treating your body just fine.

On the other hand, you who were securely attached may be wondering why you are struggling with these same 2 extremes. But our circumstances, culture, subculture or work culture, friends, \& our own other traits can all also make us go too far either way.

If you are unsure about which you do, review the box ``Are You Too Out? Too In?'''' -- \cite[pp. 81--83]{Aron2013}

\subsection{The Problem of Being In Too Much}
``Some HSPs, perhaps all of us at times, get sidelined because of thinking that there is no way an HSP can be out in the world \& survive. One feels too different, too vulnerable, perhaps too flawed.

I heartily agree that you will not be able to be involved in the world in the style of the nonsensitive, bolder sorts of folks you may be comparing yourself to. But there are many HSPs who have found a way to be successful on their terms, in the world, doing something useful \& enjoyable, with plenty of time for staying home \& having a rich, peaceful inner life, too.

\noindent\textbf{Are You Too Out? Too In?}

For each statement, put a 3 for very true, 2 for somewhat \textit{true} or \textit{equally true} \& \textit{not true}, depending on the situation, or 1 for \textit{hardly ever true}.
\begin{enumerate}
	\item I often experience the brief effects of being overaroused, overstimulated, or stressed-things like blushing, heart pounding, or my breath becoming more rapid or shallow, my stomach tensing, my hands sweating or trembling, or suddenly feeling on the verge of tears or panic.
	\item I am bothered by the long-term effects of arousal -- the sense of distress or anxiety, upset digestion or loss of appetite, or not being able to fall, or stay, asleep.
	\item I try to face situations that make me overaroused.
	\item In a given week, I stay home more than I go out. (Take the time to figure this out carefully, adding up only the available hours, \textit{excluding} sleep \& a couple of hours for dressing, undressing, bathing, etc.)
	\item In a given week, I spend more time along than with others. (Figure as above.)
	\item I push myself to do things I fear.
	\item I go out even when I don't feel like it.
	\item People tell me I work too much.
	\item When I notice I have overdone it physically, mentally, or emotionally, I immediately stop \& rest \& do whatever else I know I need to do for myself.
	\item I add things to my body -- coffee, alcohol, medications, \& the like -- to keep myself at the right level of arousal.
	\item I get sleepy in a dark theater \&{\tt/}or during a lecture unless I'm quite interested.
	\item I wake in the middle of the night or very early in the morning \& can't go back to sleep.
	\item I don't take time to eat well or to exercise regularly.
\end{enumerate}
Add up your answers to all of the questions, \textit{excluding} 4, 5, \& 9. Then add up 4, 5, \& 9, \& subtract these from the total of the others. The most ``out'' score possible would be 27. Most ``in'' would be a 1. A moderate score would be 14. Especially if you scored 10 or under, do reflect on what I have to say about \textit{The Problem of Being In Too Much}. If you scored over 20, better take to heart what I have written in the next section, \textit{On Being Too Out in the World}.

It may help to consider your behavior from the viewpoint of your infant{\tt/}body. If it wants to try new things but is afraid, you need to help it, not reinforce the fear. Otherwise, you are telling it that it really is all wrong about its desires, that it is not fit to survive out there. That is a crippling message to give a child. You'll want to think long \& hard about who gave you this feeling in childhood, \& why, rather than helping you get out \& learn to do things your way.

As you reparent your body, the 1st thing to realize is that the more it avoids stimulation, the more arousing the remaining stimulation becomes. A teacher of medication once told the story of a man who wanted nothing to do with the stress of life, so he retreated to a cave to meditate day \& night for the rest of his life. But soon he came out again, driven to overwhelming distress by the sound of the dripping of water in his cave. The moral is that, at least to some extent, the stresses will always be there, for we bring our sensitivity with us. What we need is a new way of living with the stressors.

2nd, it is often the case that the more your body acts -- looks out the window, goes bowling, travels, speaks in public -- the less difficult \& arousing it becomes. This is called \textit{habituation}. If it is a skill, you also become better at it. E.g., traveling alone in a foreign country can seem utterly overwhelming to an HSP. \& you may always choose to avoid some aspects of it. But the more you do it, the easier it becomes \& the more you know about what you do \& do not like.

The way to come to tolerate \& then enjoy being involved in the world is by being in the world.

I do not say any of this lightly, however. I was someone who mostly avoided the world until midlife, when I was more or less forced to change by powerful inner events. Since then I have had to face some fear, overarousal, \& discomfort almost every day. This is serious business \& isn't fun. But it really can be done. \& it feels wonderful to be out there, succeeding, calling out to the world, ``Look at me! I can do it, too!'''' -- \cite[pp. 83--85]{Aron2013}

\subsection{On Being Too Out in the World}
``If the root of being in too much is a belief that the infant-body is defective, the root of being out too much is equally negative. It suggests that you love the child so little that you are willing to neglect \& abuse it. Where did you get \textit{that} attitude?

It doesn't necessarily all come from parents. Our culture has an idea of competition in the pursuit of excellence that can make anyone not striving for the top feel like a worthless, nonproductive bystander. This applies not only to one's career but even to one's leisure. Are you fit enough, are you progressing in your hobby, are you competent as a cook or gardener? \& family life -- is your marriage intimate enough, your sexual life optimal, have you done all that you can do to raise excellent children?

The infant{\tt/}body rebels under all this pressure, signaling its distress. In response, we find ways to toughen it or to medicate it into silence. So the chronic stress-related symptoms arise, like digestion problems, muscle tension, constant fatigue, insomnia, migraine headaches; or a weak immune system makes us more susceptible to the flu \& to colds.

Stopping the abuse 1st requires admitting it is just that. It also helps to find what part of you is the abuser. The part that has bought into society's script of perfection? That needs to outdo a brother or sister? That has to prove that you really are not flawed or ``too sensitive''? That wants to win your parents' love or even just a glance your way for once? That needs to prove you're as gifted as they think? Or that the world cannot survive without you? Or that you can control everything \& are perfect \& immortal? There is often some arrogance in there somewhere, even if it is the arrogance of another about you.

There is 1 other reason HSPs drive their bodies too hard, \& that is their intuition, which gives some of them a steady stream of creative ideas. They want to express them all.

Guess what? You cannot. You will have to pick \& choose. Doing anything else is arrogance again \& cruel abuse of your body.

I had a dream about this once -- about headless, glowing, unstoppable beings out to get me -- which in the morning brought to mind Disney's animation of \textit{The Sorcerer's Apprentice}. Mickey Mouse plays the apprentice \& uses sorcery to bring to life a broom to do the chore his master wants done: the filling of a cistern. This is not just laziness -- Mickey is too arrogant to do something so lowly, working slowly within the limits of his own body. But Mickey has started something he cannot end. When the water is flooding the room \& the broom still will not stop, Mickey chops it up, \& soon hundreds of headless brooms are carrying water, drowning Mickey in the fulfillment of his own bright ideas.

That is the lively revenge you can expect from your body when you treat it like a lifeless broom, all in the service of too many bright ideas.

The choice of Mickey to be the apprentice was a good one -- he is usually so representative of the average guy in our culture -- upbeat, energetic. That energy has its good side; it promotes the belief that as individuals \& as a people we can do anything if we work hard enough \& are clever enough. Anyone can be president or rich \& famous. But the ``shadow'' or dangerous side of that virtue (all virtues have a shadow) is that it makes life an inhuman competition.'' -- \cite[pp. 85--87]{Aron2013}

\subsection{The Balancing Act}
``How much you are out in the world or how much you avoid it must be answered individually \& will change with time. I realize, too, that for most people a lack of time \& money made the balancing act very difficult. We are forced to make choices \& set priorities, but being very conscientious, HSPs often put themselves last. Or at least we give ourselves no more time off or opportunity to learn new skills than anyone else. In fact, however, we need more.

If you are in too much, the evidence is clear that you \& your subtle sensitivity are needed in the world. If you are out too much, the evidence is equally clear that you will perform any responsibility far better if you obtain adequate rest \& recreation.

Here is the wise advice of 1 HSP I interviewed:
\begin{quotation}
	You need to learn all about this sensitivity. It will be an obstacle or an excuse only if you allow it to be. For myself, when I am too withdrawn, I would like to stay home for the rest of my life. But it is self-destructive. So I go out to meet the rest of the world, then come back to incorporate them. Creative people need time without people. But they can't go too long. When you retreat, you lose your sense of reality, your adaptability.
	
	Getting older can also take you out of touch with reality, cause you to lose your flexibility. You need to stay out there more as you age. But as you age, grace develops, too. Your basic traits become stronger, especially if you develop all of yourself, not just your sensitivity.
	
	Be in tune with your body. It is a great gift you can use, this sensitivity to your body. It can guide you; \& your opening to it will make it better. Of course, sensitive people want to \textit{shut} the doors to the world \& to their bodies. They become fearful. You can't do that. Self-expression is the better way.'' -- \cite[pp. 87--88]{Aron2013}
\end{quotation}

\subsection{Rest}
``Infants need a lot of rest, don't they? So do highly sensitive bodies. We need all kinds of rest.

1st, we need sleep. If you have trouble sleeping, make this your 1st priority. Research on chronic sleep loss has found that when people are allowed to sleep as much as they need, it can take 2 weeks for them to reach the point where they show no signs of sleep deprivation (dropping off to sleep abnormally quickly or in any darkened room). If you are showing signs of ``sleep debt,'' you need to plan some vacation time periodically that allows you to do nothing but sleep as much as you want. You will be surprised by how much that will be.

HSPs do worse than others working night shifts or mixed shifts, \& they recover more slowly from jet lag. Sorry, but it goes with the territory. Better not to plan, or at least not to plan to enjoy, brief trips across many time zones.

If insomnia is a problem, you can find plenty of advice on that in other sources. There are even centers for treating it. But here are some points that may apply especially to HSPs. 1st, respect your natural rhythms \& retire when you 1st become sleep. For a morning person, that means going to bed early in the evening. For a night person, the ones with the more difficult problem, it means sleeping late as often as possible.

Sleep researchers tend to advise people to associate their bed only with sleeping \& to get up if they cannot sleep. But I find HSPs sometimes do better if they promise themselves to stay in bed for 9 hours with their eyes closed without worrying if they are actually sleeping. Since 80\% of sensory stimulation comes in through the eyes, just resting with your eyes closed gives you quite a break.

The problem with staying in bed while awake, however, is that some people begin to worry or otherwise overarouse themselves with their thoughts \& imaginings. If this happens, it might be better to read. Or get up \& think through the issue on your mind, write down your ideas or solutions, then go back to bed. Sleep problems are 1 of those many areas where we each are unique \& must find what works for ourselves.

We need other kinds of rest, too, however. HSPs tend to be very conscientious \& perfectionistic. We cannot ``play'' until all the details of our work are done. The details are like little needles of arousal poking us. But that can make it difficult to relax \& have some fun. The infant{\tt/}body wants play, \& play creates endorphins \& all the other good changes that undo stress. If you are depressed, overly emotional in other ways, not sleeping, or showing other signs of being out of balance, force yourself to plan more play.

But what is fun? Be careful not to let the non-HSPs in your world define that for you. For many HSPs, fun is reading a good book or gardening a little bit, at their own pace, or a quiet meal at home, prepared \& eaten slowly. In particular, squeezing in a dozen activities by noon may not be your idea of fun at all. Or it may be okay in the morning but not by afternoon. So always plan a way to bail out. If you are with someone, be sure to warn them ahead of time so that they will not feel insulted or hurt when you drop out.

Finally, when planning a vacation, consider the cost in terms of airline tickets or deposits if you decide you want to come home early or stop traveling \& stay in 1 place. Then be mentally prepared ahead of time to pay that cost.

Besides sleep \& recreation, HSPs also need plenty of ``downtime'' just for unwinding \& thinking over the day. Sometimes we can do this while performing our daily tasks -- driving, washing dishes, gardening. But if you have found ways to eliminate some of those tasks, you still need that downtime. Take it.

Yet another form of rest, perhaps the most essential, is ``transcendence'' -- rising above it all, usually in the form of meditation, contemplation, or prayer. At least some of your transcendent time should be aimed at taking you out of all ordinary thinking, into pure consciousness, pure being, pure unity, or oneness with God. Even if your transcendence falls short of this, when you return you will have a bigger, fresher perspective on your life.

Sleep takes you out of your narrow state of mind, too, of course, but the brain is in a different state when asleep. Indeed, it is in a different state for each type of activity -- sleep, play, meditation, prayer, yoga -- so a mixture is good. But do include some meditation that has the goal of experiencing pure consciousness \& involves no physical activity \& no concentration or effort. This state is undoubtedly the one that provides the most deep rest while the mind is still alert. Research on Transcendental Meditation, which does create this state, has very consistently found that meditators show less of the distressing long-term arousal described in the previous chapter. (Cortisol in meditators' blood decreases.) It is as if their meditations give them some of the needed feeling of security \& inner resources.

Of course, you also want to pay attention to what you eat \& to get enough exercise. But that is a very individual matter, \& there are plenty of other books to advise you on that. Do learn about the foods that tend to calm the body or take the edge off, helping you to sleep. \& get enough of the vitamins \& minerals -- e.g., magnesium -- that affect stress \& overarousal.

If you are used to caffeine, it probably does nothing special to arouse you unless you drink a little more than usual. It is a powerful drug for HSPs, however. Be careful about using it just now \& then, thinking it will improve your performance the way it does for those around you. E.g., if you are a morning person \& do not usually drink caffeine \& then drink it 1 morning before an important exam or interview, it could actually make you perform worse by overarousing you.'' -- \cite[pp. 88--90]{Aron2013}

\subsection{Strategies for Overarousal}
``A good caretaker develops many strategies for soothing his or her infant. Some are more psychological; some, more physical. Either approach will change the other. Choose according to your intuition. Any approach requires taking action -- getting up, going to the infant, doing something.

E.g., you walk into New York's Pennsylvania Station, are overwhelmed, \& begin to feel afraid. Psychologically or physically you need to do something to keep the infant{\tt/}body from getting upset. In this case it might be a good idea to work through the fear \& upset psychologically: This is not a noisy hell filled with dangerous strangers. It is just a larger version of many train stations you have dealt with, overflowing with normal people trying to get where they want to go, with plenty who would help you if you asked.

Here are some other psychological methods useful in handling overarousal:
\begin{itemize}
	\item Reframe the situation.
	\item Repeat a phrase, prayer, or mantra that, through daily practice, you have come to associate with deep inner calm.
	\item Witness your overarousal.
	\item Love the situation.
	\item Love your overarousal.
\end{itemize}
In \textit{reframing}, notice what is familiar \& friendly, what you have successfully dealt with that is similar. When \textit{repeating a mantra or prayer}, if your mind races back to what is overarousing it, it is important not to get discouraged \& stop. You will still be calmer than you would be without it.

When \textit{witnessing}, imagine standing to 1 side, watching yourself, perhaps talking about yourself with a comforting imaginary figure. ``There's Ann again, so overwhelmed she's falling to pieces. I really feel for her. When she's like this, of course, she can't see beyond right now. Tomorrow, when she's rested, she'll be all excited again about her work. She just has to take some rest now no matter what seems to need to be done. Once she's rested, it will go smoothly.''

\textit{Loving the situation} sounds pretty flippant, but it's important. An expanded, loving mind, one that is open to the whole universe, is the opposite of a tightly constricted, overaroused mind. \& if you cannot love the situation, it is vitally important \& even more essential that you \textit{love yourself in your state} of not being able to love the situation.

Finally, do not forget the power of music to change your mood. (Why do you think armies have bands \& buglers?) But beware that most HSPs are strongly affected by music, so the right choice is essential. When you are already aroused, you do not want to stir yourself up more with emotional pieces or something associated with important memories (the music most people, being underaroused, cannot get enough of). Sobbing violins are out at such times. \&, of course, since any music increases stimulation, use it only when it seems to soothe you. Its purpose is to distract you. Sometimes you need to be distracted; at other times, you need to attend carefully.

But since we are dealing with the body, it can be an equally good idea to try a physical approach.

Here's a list of some purely physical strategies:
\begin{itemize}
	\item Get out of the situation!
	\item Close your eyes to shut out some of the stimulation.
	\item Take frequent breaks.
	\item Go out-of-doors.
	\item Use water to take the stress away.
	\item Take a walk.
	\item Calm your breathing.
	\item Adjust your posture to be more relaxed \& confident.
	\item Move!
	\item Smile softly.
\end{itemize}
It's amazing how often we forget to take action simply to get out of a situation. Or take a break. Or take the situation -- task, discussion, argument -- out of doors. Many HSPs find nature deeply soothing.

Water helps in many ways. When overaroused, keep drinking it -- a big glass of it once an hour. Walk beside some water, look at it, listen to it. Get into some if you can, for a bath or a swim. Hot tubs \& hot springs are popular for good reasons.

Walking is also 1 of those basic comforts. The familiar rhythm is soothing. So is the rhythm of slow breathing, especially from your stomach. Exhale slowly with a little extra effort, as if blowing out a candle. You will automatically inhale from your stomach. Or merely attend to your breathing -- this old friend will settle you down.

The mind often imitates the body. E.g., you may notice that you have been walking around leaning forward slightly, as if rushing toward the future. Balance yourself over your center instead. Or your shoulders may be rounded, your head down, as if under a burden. Straighten up, throw off the burden.

Tucking your head between raised shoulders may be your favorite position both in sleep \& while awake, an unconscious attempt at self-protection from blows of stimulation \& waves of overarousal. Instead, uncurl. When standing, raise your head, pull your shoulders back, center your upper body over your torso \& feet so that the weight feels most effortlessly balanced. Feel the solid ground through your feet. Bend your knees a little \& breathe deeply from your stomach. Feel your body's strong center.

Try to create the moves as well as the posture of someone calm, in command. Lean back, relax. Or get up \& move toward what appeals to you. Get the ``approach system'' on line. Or move like someone angry, disdainful. Shake a fist. Scowl. Gather your things \& prepare to walk out. Your mind will imitate your body.

It is crucial to hold yourself \& move in the manner you want to feel. Overaroused HSPs tend to substitute ``freeze'' for the ``fight or flight'' response. Relaxed posture \& free movement can break that numb tension. Or stop moving if you just being frantic or jittery.

Smiling? Maybe it is a smile just for yourself. Why you are smiling does not matter.'' -- \cite[pp. 90--93]{Aron2013}

\subsection{The Containers in Your Life}
``Another way to understand all of this advice is to remember how we began this chapter, by appreciating that your infant{\tt/}body's earliest \& still most basic need is to be held \& protected from overstimulation. On that strong basis, you can go out \& explore, feeling secure about that safe harbor of the good caretaker's arms.

If you think about it, your life is filled with such safe containers. Some are concrete -- your home, car, office, neighborhood, a cottage or cabin, a certain valley or hilltop, a forest or bit of shoreline, certain clothing, or certain beloved public places, such as a church or library.

Some of the most important containers are the precious people in your life: spouse, parent, child, brother or sister, grandparent, close friend, spiritual guide, or therapist. Then there are the even less tangible containers: your work, memories of good times, certain people you cannot be with anymore but who live on in memory, your deepest beliefs \& philosophy of life, inner worlds of prayer or meditation.

The physical containers may seem the most reliable \& valuable, especially to the infant{\tt/}body self. It is the intangible ones, however, that are really the most reliable. There are so many accounts of people who maintained their sanity by retreating into such containers while under extreme stress or danger. Whatever happened, nothing \& no one could take from them their private love, faith, creative thinking, mental practice, or spiritual exercise. Part of maturing into wisdom is transferring more \& more of your sense of security from the tangible to the intangible containers.

Perhaps the greatest maturity is our ability to conceive the whole universe as our container, our body as a microcosm of that universe, with no boundaries. That is more or less enlightenment. But most of us will need more finite containers for a while, even if we are beginning to learn to make do with intangible ones in a pinch. Indeed, as long as we are in bodies, enlightened or not, we need some bit of tangible safety, or at least a sense of sameness.

Above all, if you do lose a container (or worse, several), accept that you will feel especially vulnerable \& overwhelmed until you can adjust.'' -- \cite[pp. 93--94]{Aron2013}

\subsection{Boundaries}
``Boundaries are obviously an idea closely related to containers. Boundaries should be flexible, letting in what you want \& keeping out what you don't want. You want to avoid shutting everyone out all of the time indiscriminately. \& you want to control any urges to merge with others. It would be nice, but it just doesn't work for long. You lose all of your autonomy.

Many HSPs tell me that a major problem for them is poor boundaries -- getting involved in situations that are not really their business or their problem, letting too many people distress them, saying more than they wanted, getting mired in other people's messes, becoming too intimate too fast or with the wrong people.

There's 1 essential rule here: \textit{Boundaries take practice!} Make good boundaries your goal. They are your right, your responsibility, your greatest source of dignity. But do not become too distressed when you slip up. Just notice how much better you are getting at it.

Besides all the other reasons to have good boundaries, you can use them to keep out stimulation when you have had all that you can take. I have met a few HSPs (one in particular raised in an overcrowded urban housing project) who can, at will, shut out almost all stimulation in their environment. Quite a handy skill. ``At will'' is important, however. I am not referring to involuntary dissociation or ``spacing out.'' I am talking about choosing to shut out the voices \& other sounds around you, or at least decreasing their impact on you.

So, want to practice? Go sit by a radio. Imagine yourself with some kind of boundary around yourself that keeps out what you don't want -- maybe it is light, energy, or the presence of a trusted protector. Then turn on the radio but keep out the radio's message. You will probably still hear the words, but refuse to let them in. After a while, turn the radio off \& think about what you experienced. Could you give yourself permission to shut out the broadcasting? Could you feel that boundary? If not, practice it again someday. It will improve.'' -- \cite[pp. 94--95]{Aron2013}

\subsection{The Infant{\tt/}Body's Message}

\begin{enumerate}
	\item ``Please don't make me handle more than I can. I am helpless when you do this, \& I hurt all over. Please, please, protect me.
	\item I was born this way \& can't change. I know you sometimes think something awful must have made me this way, or at least made me ``worse,'' but that ought to give you even more sympathy for me. Because either way I can't help it. Either way, \textit{don't blame me for how I am}.
	\item What I am is wonderful -- I let you sense \& feel so much more deeply. I am really 1 of the best things about you.
	\item Check in on me often \& take care of me right at that moment if you possibly can. Then, when you can't, I can trust that you are at least trying \& I won't have long to wait.
	\item If you must make me wait for my rest, please ask me nicely if it's okay. I'm only more miserable \& troublesome if you get angry \& try to force me.
	\item Don't listen to all the people who say you spoil me. You know me. You decide. Yes, sometimes I might do better left alone to cry myself to sleep. But trust your intuition. Sometimes you \textit{know} I am too upset to be left alone. I do need a pretty attentive, regular routine, but I'm not easily spoiled.
	\item When I'm exhausted, I need sleep. Even when I seem totally wide awake. A regular schedule \& a calm routine before bed are important to me. Otherwise, I will lie awake in bed all stirred up for hours. I need a lot of time in bed, even if I'm lying awake. I may need it in the middle of the day, too. Please let me have it.
	\item Get to know me better. E.g., noisy restaurants seem silly to me -- how can anybody eat in them? I have a lot of feelings about such things.
	\item Keep my toys simple \& my life uncomplicated. Don't take me to more than 1 party in a week.
	\item I might get used to anything in time, but \textit{I don't do well with a lot of sudden change}. Please plan for that, even if the others with you can take it \& you don't want to be a drag. Let \textit{me} go slow.
	\item But I don't want you to coddle me. I especially don't want you to think of me as sick or weak. I'm wonderfully clever \& strong, in my way. I certainly don't want you hovering over me, worried about me all day. Or making a lot of excuses for me. I don't want to be seen as a nuisance, to you or to others. Above all, I count on you, the grown-up, to figure out how to do all of this.
	\item Please don't ignore me. Love me!
	\item \& like me. As I am.'' -- \cite[pp. 95--96]{Aron2013}
\end{enumerate}

\subsection{Working With What You Have Learned: Receiving Your 1st Advice From Your Infant{\tt/}Body Self}
``Pick a time when you are not rushed \& will not be interrupted, when you are feeling solid \& in the mood for self-exploration. What follows can bring up strong feelings, so if you start to feel overwhelmed, take it slow or just stop. What follows can also just be difficult to do because of resistance that cause the mind to wander, the body to become uncomfortable or sleepy. If that happens, it's natural \& fine. Try a little of this on several occasions \& appreciate whatever happens.

\noindent\textbf{Part I.} Read all of these instructions 1st, so that, as much as possible, you can do without looking back at them as you proceed.
\begin{enumerate}
	\item Curl up like a baby or lie on your stomach or back -- find the position you think was yours.
	\item Shift from thinking in your head to feeling emotionally from your body, as a baby does. To help with that, for a full 3 minutes breathe very consciously from the center of your body, your stomach.
	\item After the breathing, become yourself as an infant. You think you cannot remember, but your body will. Start with an image of weather, like the example at the start of this chapter. Is it mostly fair or stormy?
	
	Or begin with your earliest conscious memory, even if it was from a little later age. It is all right to be an infant with a slightly older child's understanding. E.g., this slightly older child may be certain that it is better not to cry for help. Alone is best.
	\item Be especially aware that you are a \textit{highly sensitive} infant.
	\item Be aware of what you most need.
\end{enumerate}
\textbf{Part II.} Now or at a later time $\ldots$ Again, read all of the instructions 1st, so that you will not need to refer back to them so much that it's distracting.
\begin{enumerate}
	\item Imagine a very beautiful baby who is about 6 weeks old. Really tiny. Admire the sweetness, the delicateness. Notice that you would do almost anything to protect this child.
	\item Now realize that this wonderful baby is your infant{\tt/}body self. Even if this baby is more like some infant you have recently seen, this is your imagination's baby.
	\item Now watch as you start to whimper \& fuss. Something is the matter. Ask this baby, ``What can I do for you?'' \& listen well. This is your infant{\tt/}body speaking.
	
	Do not worry that you are ``just making this up.'' Of course, you are, but your infant{\tt/}body self will be involved somewhere in the ``making up.''
	\item Answer back, start a dialogue. If you foresee difficulties meeting this infant's needs, talk about it. If you are sorry for something, apologize. If you get angry or sad, that's something also good to know about your relationship with the baby.
	\item Do not hesitate to do either part of this exercise again or to do it differently. E.g., next time just open your mind to the infant{\tt/}body self, at whatever age \& in whatever setting she or he wants to appear.'' -- \cite[pp. 96--98]{Aron2013}
\end{enumerate}

%------------------------------------------------------------------------------%

\section{Reframing Your Childhood \& Adolescence}

\subsection{Learning to Parent Yourself}
``In this chapter we'll begin to rethink your childhood. As you read about typical experiences of sensitive children, memories of yourself as a child will return. But you'll see them freshly through the lens of your new knowledge about your trait.

These experiences matter. Like a plant, the kind of seed that goes into the ground -- your innate temperament -- is only part of the story. The quality of soil, water, \& sun also deeply affects the grown plant that is now you. If the growing conditions are very poor, the leaves, flowers, \& seeds barely appear. Likewise, as a child, you did not expose your sensitivity if your survival required different behavior.

When I began my research, I discovered ``2 kinds'' of HSPs. Some reported difficulty with depression \& anxiety; some reported very little of these feelings. The separatedness of the 2 groups was quite clear. Later, I discovered that the depressed \& anxious HSPs almost all had troubled childhoods. Non-HSPs with troubled childhoods do not show nearly as much depression \& anxiety. But neither do HSPs with healthy childhoods. It is important that we \& the public not confuse high sensitivity with ``neuroticism,'' which includes certain types of intense anxiety, depression, overattachment, or avoidance of intimacy, \& are usually due to a troubled childhood. True, some of us were dealt both hands in life -- high sensitivity {\it\&} neuroticism -- but the 2 things are not at all the same. This confusion of sensitivity with neuroticism \& the effects of childhood trauma is 1 reason for some of the negative stereotypes of HSPs (that we are by nature always anxious, depressed, \& so on). So let's all work to keep it straight.

It is easy to understand why a troubled childhood might affect HSPs more than non-HSPs.

HSPs are prone to see all the details, all the implications, of a threatening experience. But it is easy to underestimate the impact of childhood, since so much that is important occurs before we can remember. Moreover, some of what is important was just too distressing \& deliberately forgotten. If someone caring for you became angry or dangerous, the conscious mind buried that information as too awful to acknowledge, even while your unconscious developed a deeply mistrustful attitude.

The good news is that we can work on any negative effects. I have seen HSPs who have done just that \& been freed of much of their depression \& anxiety. But it takes time.

Even if your childhood was wonderful, however, it was probably difficult being highly sensitive. You felt different. \& your parents \& teachers, even if excellent in most respects, did not know how to handle a sensitive child. There simply was not much information on the trait, \& there was so much tension about making you ``normal,'' like the ideal.

A final point to remember: Sensitive boyhoods \& girlhoods are quite different from each other. In this chapter, therefore, I will pause often to point out how your experience probably differed with your gender.'' -- \cite[pp. 99--100]{Aron2013}

\subsection{Marsha, a Wisely Avoidant Little Girl}
``Marsha, an HSP in her 60s, saw me in psychotherapy for several years, hoping to understand some of her ``compulsions.'' In her 40s she had become a poet \& photographer \& at 60 her work was gaining considerable respect.

While some of her story is distressing, her parents basically did their best. \& she has dealt well with her past \& continues to learn from it, both inwardly \& through her art. I think if you were to ask today if she is happy, she would say yes. But what matters most is her steady growth in wisdom.

Marsha was the youngest of 6 children born to immigrant parents struggling to make ends meet in a small midwestern town. Marsha's older sisters recall their mother sobbing at the news of each of her pregnancies. Marsha's aunts recall her mother, their sister, as deeply depressed. But Marsha has no memory of her mother ever being slowed down by grief, depression, fatigue, or hopelessness. She was an impeccable German housekeeper \& devout churchgoer. Similarly, Marsha's father ``worked, ate, slept.''

The children did not feel unloved. Their parents simply had no time, energy, or money for affection, conversation, vacations, helping with homework, imparting wisdom, or even giving gifts. This brood of 6 chicks, as Marsha sometimes described herself \& the others, mostly raised themselves.

Of the 3 styles of attachment that you read about in the previous chapter -- secure, anxious, \& avoidant -- Marsha's early childhood required her to be avoidant. She had to be a child who did not need anyone, who caused as little trouble as possible.

\subsubsection{Little Marsha, HSP, in Bed With the Big Beasts}
During the 1st 2 years of Marsha's life, the household sleeping arrangements placed her in the same bed with 3 older brothers. Alas, they used their baby sister to experiment sexually, in the way that unsupervised children sometimes do. After 2 years, she was moved to her sisters' rooms. All she recalls is that ``finally I felt a little bit safe at night.'' But she remained the target of cruel, overt sexual harassment from 1 of her older brothers until she was 12.

Marsha's parents never noticed all of this, \& she believed that if she told on her brothers, her father would kill them. Killing was a part of life. It seemed that it could happen. Marsha recalled being stunned by the regular beheading of chickens in the backyard \& the casual, callous attitude toward this necessity of life. So there is extra meaning in her seeing the children of her family as a brood of chicks.

Besides the sexual torment, her brothers loved to tease \& frighten Marsha, as if she were their personal toy. More than once they caused her to faint from fear. (HSPs make such good targets because we react so strongly.) However, no cloud lacks its silver. As their special plaything, she was also taken places \& tasted freedoms that girls usually missed in those days. Her brothers, who possessed a tough independence that she preferred to her mother's \& sisters' passivity, were her role models -- \& this was in some ways a valuable experience for a sensitive girl.

For Marsha, the best attachment was with an older sister, but that sister died when Marsha was 13. Marsha recalls lying on her parents' bed, staring into space, waiting for news of her sister. She had been told that if her parents did not call in an hour, it would mean that her sister had died. When the clock struck the hour, Marsha recalls picking up a book \& going back to reading. Here was yet another lesson in not attaching.

\subsubsection{Marsha as a Tiny Fairy, Marsha in the Chicken Coop}
``Marsha's 1st memory is of lying naked in the sunlight watching dust motes, awed by the beauty -- a memory of her sensitivity as a source of joy. It has been that way all her life, especially now that it can be expressed in her art.

Notice that no other person is in her 1st memory. In a similar way, her poetry \& photography tend to be about things, not people. There are often images of houses -- with closed windows \& doors. The haunting emptiness of some of her work speaks to the private experiences of all of us, especially those whose early childhoods taught us to avoid closeness.

In 1 photo, produced during her therapy, chickens are in the foreground, in clear focus. (Recall the significance of chickens for Marsha.) The chicken wire \& door frame of a jail-like children coop are fainter. Faintest of all, in the darkened door of the coop, is the ghostly image of a group of ragged children. Another important image of her art came from a dream about a bright, angry little fairy who lived in a secret garden \& would allow no one in.

Marsha has used food, alcohol, \& various drugs compulsively -- in amounts that bordered on excessive. But she was too smart to go over the edge, having a very practical streak \& an IQ of over 135. In 1 dream she was wheeling a starving, angry infant through a banquet hall filled with food, but it wanted none of it. We discovered that the baby was starved, in a greedy, desperate way, for love \& attention. Like hungry chickens, when we cannot be fed what we need, we feed ourselves what we can find.'' -- \cite[pp. 100--103]{Aron2013}

\subsection{HSPs \& Attachment}
``In the previous chapters we learned about the importance of your attachment to your caretaker, usually your mother. An insecure attachment style will persist throughout life unless one has an unusually secure one with someone in adulthood, such as a male or during long-term psychotherapy. Alas, non-therapy relationships sometimes can't withstand the task of undoing childhood-based insecurity (the avoidance of intimacy or the compulsion to merge \& fear of being abandoned). Also, while you go out into the world unconsciously seeking that long-desired security, without extensive experience with what you seek, you often repeat the same old mistakes, choosing again \& again the same familiar sort of person who makes you feel insecure.

While I found a slight tendency for more HSPs than non-HSPs to show 1 of the nonsecure attachment styles as adults, that does not mean the trait creates the situation. It probably reflects the way a sensitive child is more aware of the subtle cues in any relationship.

As an HSP, some of your most important lessons about the other was whether to expect help with overarousal or an extra dose of it. Every day was a lesson.

In his \textit{Diary of a Baby} (described in Chap. 2), Stern gives the example of a ``face duet'' between mother \& the imaginary Joey. Mother coos \& brings her face close, then withdraws. Joey smiles, laughs, encourages the game. But eventually it becomes too intense. At these moments of overarousal, Stern's imaginary Joey breaks eye contact \& looks away, in effect stopping the arousal. To describe this face duet, Stern again uses the weather analogy, the mother being the wind which plays over the child. So when Joey is overwhelmed, Stern imagines this diary entry:
\begin{quotation}
	Her next gust is rushing towards me, whipping up space \& sound. It is upon me. It strikes me. I try to meet its force, to run with it, but it jolts me through \& through. I quake. My body stalls. I hesitate. Then I veer off. I turn my back to her wind. \& I coast into quiet water, all alone.
\end{quotation}
This should all be familiar to you now -- Joey is trying for that optimal level of arousal described in Chap. 1. Those taking care of babies usually sense this. When a baby is restless \& bored, they invent games like the face duet or something more arousing, like making weird faces or reaching slowly toward the child while saying, ``I'm gonna get you.'' The squeals of delight are a great reward to the adult. \& there may be a sense that being pushed to the limit is good for the child's confidence \& flexibility. When the child shows distress, however, most adults stop.

Now consider our imaginary, highly sensitive Jesse. The face duet is probably not much different except that it is a little quieter \& briefer. Jesse's mother will have adjusted her playing in order to keep Jesse within his comfort range.

But what about those times when others get their hands on Jesse? Suppose his older sister or Grandpa makes the duet a little more intense? What if, when Jesse averts his gaze, his way of taking a time-out, his sister moves so close that they are face-to-face again. Or she turns Jesse's face back.

Maybe Jesse closes his eyes.

Maybe his sister puts her mouth by Jesse's ear \& screams.

Maybe Grandpa takes him \& tickles him or tosses him up in the air a few times.

Jesse has lost all control over his arousal level. \& each howl of Jesse's brings a fresh rationalization: ``He loves it, he loves it -- he's just scared a little.'''' -- \cite[pp. 103--104]{Aron2013}

\subsection{The Confusing Issue -- Do You ``Love It''?}
``Have you imagined yourself in Jesse's place? What a confusing situation. The source of your arousal is utterly out of your control. Your intuition tells you that the other, usually so helpful, is now anything but help. Yet the other is laughing, having fun, expecting you to.

Here is a reason why you may find it hard even now to know what you do \& do not like, separate from what others like to do to you or with you or think you should like.

I remember once watching 2 dog owners taking their small pups into the surf \& throwing them out into deep water. The dogs swam desperately to their owners' waiting arms even though it meant that the treatment would be repeated. Not only was it probably the only alternative to drowning, but these arms were the ones that provided all the safety \& food the pups had ever known. So they wagged their tails wildly, \& I suppose their owners believed that they wanted \& thought the pups loved the ``game.'' Maybe even the pups were unsure after a while.

Then there was the HSP whose earliest memory was of being the imaginary ``dough'' in a family--reunion skit of ``patty-cake, patty-cake.'' In spite of crying \& pleading with her parents, this 2-year-old was passed in a circle from stranger to stranger. Reliving the long-repressed feelings that went with this memory, she realized that it (\& other situations she had probably repressed completely) left her with a sense of helpless terror about being picked up, about being controlled physically in any way, \& about her parents not protecting her.

The bottom line is that in those 1st years you either learned to trust the other, \& the outer world generally, or you didn't. If you did, your sensitivity remained, but you were rarely threatened into distressing long-term arousal. You knew how to handle it; it seemed under your control. If you asked others to stop doing something, they did. You knew you could trust them to help you rather than overburden you. On the other hand, 1 way that chronic shyness, anxiety, or social avoidance can begin is if your early experiences did not build that trust. It is not inborn but learned.

This either-or effect is not rigid -- you probably learned to trust in some situations more than others. But it is also true that in the 1st 2 years the child adapts an overall strategy or mental representation of the world which can be quite enduring.'' -- \cite[pp. 104--105]{Aron2013}

\subsection{HSPs with Good Childhoods}
``There are, by the way, some reasons to expect many HSPs to have had unusually good childhoods. Gwynn Mettetal is a psychologist at Indiana University studying how best to help parents of the ``temperamentally at risk.'' She notes that most parents try hard to understand their children \& raise them correctly. A sensitive child's realization of these good intentions can provide a stronger than usual feeling of being loved.

Parents of a highly sensitive child often develop an especially intimate bond with their child. The communication is more subtle, \& the triumphs in the world are more significant. ``Look, Mom -- I scored a goal!'' takes on all new meaning to parents \& coaches when the soccer-player is an HSP. \& since the trait is inborn, there is a good chance that 1 or both of your parents understood you very well.

Research at the University of California Medical School in San Francisco found that children who were ``highly sensitive to stress'' had more injuries \& illnesses if they were under stress but actually had fewer when not under stress. Since stress is greatly influenced by the security of a child's attachment \& family life, I think it's safe to assume that highly sensitive children enjoying a secure attachment style also enjoy unusually good health. Isn't that interesting to know?

Finally, even if your parents practiced benign, neglect, you may have received enough love \& been allowed sufficient space to grow up fine on your own. Perhaps imaginary figures, characters in books, or nature itself calmed \& supported you enough; your trait may have made you happier than other children with this solitude. Or your intuition \& many good attributes may have brought you into other, healthier close relationships, with a relative or teacher. Even a little time with the right person can make all the difference.

If your family was unusually difficult, you should also be aware that your trait may have protected you from being quite as involved in or confused by the chaos as another child might have been. \& when you start to heal, your intuition will help you in that process. Those who study attachment find that most of the time we impart to our children the same experience we had, but there are definitely exceptions, \& they are the adults who have healed their worst childhood hurts. If you make the admittedly painful effort, you can eb 1 of those, too. We will return to this in Chap. 8.'' -- \cite[pp. 106--107]{Aron2013}

\subsection{New Fears Out in the World}
``As you approached school age, there were new tasks \& new ways your sensitivity could help or hinder you. Like Rob in Chap. 2, your exposure to the big, wide world would have further stimulated your imagination, produced a heightened awareness of everything that escaped others, \& given you great joy \& appreciation of the smallest beauties in life. As your sensitivity encountered a bigger world, it also probably gave rise to new ``unreasonable'' fears \& phobias.

Fears can increase at this age for many reasons. 1st, there is simple conditioning: Whatever was around when you were overaroused became associated with overarousal \& so became something more to be feared. 2nd, you may have realized just how much was going to be expected of you, how little your hesitations would be understood. 3rd, your sensitivity tuned ``antenna'' picked up on all the feelings in others, even those emotions they wanted to hide from you or themselves. Since some of those feelings were frightening (given that your survival depended on these people), you may have repressed your knowledge of them. But your fear remained \& expressed itself as more ``unreasonable'' fear.

Finally, being sensitive to the discomfort, disapproval, or anger of others probably made you quick to follow every rule as perfectly as possible, afraid to make a mistake. Being so good all the time, however, meant ignoring many of your normal human feelings -- irritation, frustration, selfishness, rage. Since you were so eager to please, others could ignore your needs when, in fact, yours were often greater than theirs. This would only fuel your anger. But such feelings may have been so frightening that you buried them. The fear of being breaking out would become yet another source of ``unreasonable'' fears \& nightmares.

Finally, for many of you, the patience your parents showed about your sensitivity in the 1st 3 years now ran thin. They had hoped you would outgrow it. But as it came time to send you to school, they knew the world was not going to treat you gently. They may have begun to blame themselves for overprotecting you, to launch a campaign of pushing you harder. Maybe they even sought professional help, giving you an even stronger message that something was wrong with you. All of this also could have added to your anxiety at this age.'' -- \cite[pp. 107--108]{Aron2013}

\subsection{The Problem of Sensitive Little Boys}
``There appear to be just as many males as females born as HSPs. But then your culture gets hold of you. Cultures have strong ideas about how little men \& women ought to behave.

The issue is so important to us that it is almost funny. A colleague told me about this informal social psychological experiment: A new baby was left in a park with an attendant who, when asked by passerby, would claim to have agreed to sit with the child for a few moments \& did not know if it was a boy or girl. Everyone stopping to admire the infant was quite distressed at not being able to know the child's gender. Some even offered to undress the child to find out. Other studies explain why gender matters so much: people tend to treat baby boys \& girls quite differently.

It is fascinating how extensively gender is confused with sensitivity. Men should not be sensitive, women should be. \& it all begins at home. Research shows that little boys who are ``shy'' are not liked as well by their mothers, which, according to the researchers, ``can be interpreted as a consequence of the value-system of the mother.'' What a start on life. Shy boys get negative reactions from others, too, especially when the boy is also mild-tempered at home.'' -- \cite[p. 108]{Aron2013}

\subsection{Sensitive Little Girls -- Mothers' Special Companions}
``In contrast to shy boys, girls seen as shy get along well with their mothers; they are the good ones. The problem here is that sensitive girls can be overprotected. In the sensitive daughter a mother may find the child she dreams of, the one who will not, should not, \& cannot leave home -- all of which dampens the sensitive little girl's natural urge to explore \& overcome her fears.

Girls at every age show more negative effects (including withdrawal from the world) from any negative attitudes their mother has toward them -- criticism, rejection, coldness. This is probably far more true for sensitive girls. Moreover, fathers often forget to help their daughters overcome their fears. Finally, overall, little girls are more affected by both parents, for better or worse.

Having read all of this, it is time to think about the ways in which you need to be a different kind of parent to yourself now. To begin, do the self-assessment of ``How You Cope With Threats of Overarousal.'''' -- \cite[p. 109]{Aron2013}

\subsection{Being a Different Kind of Parent to Yourself}
``Some situations are overstimulating because they are too intense or long. The child in you cannot bear the fireworks, cannot take another hour at the carnival. Reading the previous chapter should have helped you to take your infant{\tt/}body seriously when it has had enough. But sometimes it is doing fine but afraid of what is coming, of the very of seeing the fireworks or riding the Ferris wheel. When new situations produce overstimulation because they are unfamiliar, \& unfamiliar things in the past have turned out to be upsetting, then naturally we reject everything new without trying it. That means we can miss out on a lot.

In order to be willing to try new things, you need plenty of experiences in facing new situations \& doing okay. For an HSP, doing well in new circumstances is never automatic. Parents who understand their highly sensitive kids develop a ``step-by-step'' strategy. Then the children themselves eventually learn to apply it to themselves. If your parents did not teach you step by step, it is time you learned to teach yourself this style of meeting the unfamiliar.
\begin{quotation}
	\textbf{How you cope with threats of overarousal}
	
	Do not hesitate to check several of the statements below, even if they seem inconsistent. Just check the items that apply to you, making each answer independent of the one before.
	
	\textit{When I am afraid of trying something new or am on the verge of being overstimulated or overaroused, I usually}
	\begin{itemize}
		\item Try to escape the situation.
		\item Look for ways to control the stimulation.
		\item Expect to be able to endure it in some way.
		\item Feel a rising sense of fear that everything may go wrong now.
		\item Seek out someone I can trust to help me or at least keep that person in mind.
		\item Get away from everyone so that at least no one can add to the problem.
		\item Try to be with others -- friends, family, a group I know well -- or go to church, take a class, get out in public somewhere.
		\item Vow to try even harder to avoid it \& everything like it no matter how much I miss out on.
		\item Complain, get angry, do whatever I must to get those responsible to stop distressing me.
		\item Focus on calming down \& trying to take things 1 step at a time.
	\end{itemize}
	Your own methods? $\ldots\ldots\ldots$
	
	All of these have their place -- even fear, which can mobilize us to act. But some are obviously better suited to some situations than others, so flexibility is the key. If you use fewer than 3, you should look at the list again \& think about adopting more.
	
	Who taught you these methods? What might have happened to prevent you from using more of them? Recognizing these childhood sources of your coping responses can help you see what is still useful, what is no longer necessary.
\end{quotation}
I have adapted here some of the advice on the ``shy child'' from Alicia Lieberman's \textit{Emotional Life of the Toddler}, to be used by grownups when we feel afraid to enter new situations:
\begin{enumerate}
	\item Just as a parent does not send a toddler into a new situation alone, do not do that to yourself. Take someone else along.
	\item Just as a parent begins by talking about the situation with the child, talk to the fearful part of yourself. Focus on what is familiar \& safe.
	\item Just as a parent keeps the promise that the child can leave if he or she becomes too upset, allow yourself to go home if you need to.
	\item Just as a parent is confident the child will be okay after a while, expect the part of yourself that is afraid to be okay after some time to adjust to all the unfamiliar stimulation.
	\item Just as a parent is careful not to respond to a child's fear with more concern than is justified by the situation, if the part that is fearful needs help, respond with no more anxiety than the braver part of you thinks is justified.
\end{enumerate}
Remember, too, that overarousal can be mistaken for anxiety. A good parent to yourself might say, ``There sure is a lot going on here; it makes your heart pound with excitement, doesn't it?'''' -- \cite[pp. 109--111]{Aron2013}

\subsection{Weighing ``Special Needs'' Against the Risk of Lasting Discouragement}
``Perhaps the most difficult task is deciding how much to protect yourself, how much to gently push yourself forward. It is the problem all parents of sensitive children face. You probably know how to put pressure on yourself; you do it just the way your parents, teachers, \& friends did. Few HSPs escape the pressure to be a good sport, normal, or pleasing to others, \& even when those others are long gone, you keep on trying to please them. You imitate their failure to accept your special need to be buffered. In the terms of the last chapter, you tend to be ``out'' too much.

Or you may have imitated overprotection, which may have been nothing but a failure to help you were both afraid \& eager to try something within your ability. In that case, you may be ``in'' too much.

How discouraging to watch your friends enjoying something you are too afraid to try. Do not underestimate such discouragement. It can be just as present in adulthood as you see friends taking on careers, travel, moves, \& relationships that you would fear. Yet deep inside you also know you have the same or more talent, desire, \& potential.

Envy can wake us up to 1 of 2 truths: We want something \& better do something about it while we still can, or we want something \& just cannot have it. As you saw in Chap. 2, in Rotherbart's description of how we develop, adult humans are capable of directing attention, using willpower, \& deciding to overcome a fear. If your envy is strong \& you decide you want to do something, you probably can.

Another, equally important part of growing up is no longer pretending we will be able to do absolutely everything. Life is short \& filled with limits \& responsibilities. We each get a piece of the ``good'' to enjoy, just as we each contribute a piece of that good to the world. But none of us can have it all for ourselves or do it all for others.

I have noticed that not all HSPs feel discouraged by not being able to do everything their peers do. They have little envy. They appreciate their trait \& know it gives them much that others lack. I think the discouragement, like the failure to buffer ourselves, comes from attitudes learned in early childhood.'' -- \cite[pp. 111--112]{Aron2013}

\subsection{It's Never Too Late to Overcome Discouragement}
``While it is wise to accept what we cannot change about ourselves, it is also good to remember that we are never too old to replace discouragement with bits \& pieces of confidence \& hope.

As a child, I had a special sensitivity to falling, which spiraled into overarousal \& a loss of coordination whenever I was up high or relying on my own balance. Thus, I never insisted on being taught things like riding a bicycle, roller skating, or ice skating -- which I think only relieved my mother. Thus, I have always been more of an envious observer than a participant in physical activities, but there have been glowing exceptions, such as what happened at the end of a summer-solstice celebration I attended in California, on a ranch in the foothills of the Sierras.

The women at the event were of all ages. But in the evening, when they had found a swing, they became a group of young girls. The swing was on a long rope \& swept out over a slope. In the twilight, it was like flying to the stars. Or so they said. Everyone had tried it except me.

When the others had wandered indoors, I stayed, looking at the swing \& feeling that old shame of being the scaredy-cat, even though probably no one had noticed.

Then a woman much younger than I appeared \& offered to show me how to use the swing. I said no, I didn't want to. But she ignored that. She promised she would never push me harder than I wanted. \& she held out the swing.

It took some time. But somehow I felt safe with her, \& I built up the courage to swing out toward the stars like the others.

I never saw that young woman again, but I will always be grateful not only for the experience but for the respect '\& understanding she showed as she taught me how -- 1 gentle swing at a time.'' -- \cite[pp. 112--113]{Aron2013}

\subsection{Your School Years}
``Marsha's memories of her school years were typical of HSPs. She excelled in school \& was even a kind of leader when it came to plans \& ideas. She was also bored. Her restless imagination led her to read books during lessons. Still, she was ``usually the smartest.''

At the same time that she was bored, the overstimulation of school always bothered her. She recalls best just the noise. It did not frighten her, but especially if the teacher left the room, the noise was unbearable. The racket at home, 8 people in a tiny house, also made her miserable. In good weather she hid in trees or under the porch \& read books. In bad weather she just learned to tune everything out while she read.

At school, however, overarousal can be harder to avoid. 1 day the teacher read aloud some newspaper accounts of the horrible tortures of certain prisoners of war. Marsha passed out.

When you started school, like Marsha, you encountered the wider world. The 1st shock may have been the separation from home. But even if you were prepared for that by going to preschool, your senses could never be prepared for the long, noisy day in the average primary-grade classroom. At best, your teachers maintained a range of stimulation that worked for the average child's optimal level of arousal. For you it was almost always excessive.

Probably at 1st you dealt with school by withdrawing \& just observing. I recall well my son's 1st day in school. He went to the corner \& stared as if dumbstruck. But silent watching is not ``normal.'' The teacher says, ``The others are playing -- why don't you?'' Rather than displeasing the teacher or being seen as odd, maybe you overcame your reluctance. Or maybe you simply could not. In which case, more \& more attention came your way -- just what you did not need.

Jens Asendorpf of the Max Plank Institute for Psychology in Munich has written on how normal it is for some children to prefer to play alone. At home parents usually sense that doing so is just part of their child's personality. But at school things are different. By 2nd grade, playing alone causes a child to be rejected by other children \& an object of concern for teachers.

For some of you, all this overarousal \& shame led to poor classroom performance. Most of you, however, being fond of reading \& quiet study, excelled in. To handle that, perhaps you found a close friend to play with. \& perhaps you had the reputation of being the one who thought up the best games, wrote the best stories, \& painted the best pictuers.

Indeed, if you had entered school with confidence about yourself \& your trait, as did Charles in Chap. 1, you may have been a real leader. If not, as 1 sensitive friend of mine, a physicist, said, ``Can you think of anyone really great who had an easy time of it in school?'''' -- \cite[pp. 113--114]{Aron2013}

\subsection{Schoolboys, Schoolgirls}
``In my research I found that by school age most male HSPs were introverts. This makes sense, since a sensitive boy is not ``normal.'' They had to be careful in groups or with strangers to see how they were going to be treated.

Sensitive girls, like sensitive boys, often rely on 1 or 2 friends throughout their school years. But some of them are fairly extroverted. Unlike the boys, if they display some overarousal or emotion, they are doing what is expected of them. It may even help them to be accepted by the other girls.

The negative side of this permission to be emotional, however, can be that a sensitive girl is never forced to put on the armor that sensitive boys have to don to survive. Girls may have little practice in emotional control \& feel helpless in the face of emotional overarousal. Or they may use their emotions to manipulate others, including to protect themselves from overarousal. ``If we have to play that game again, I'm going to cry.'' The straightforward self-assertion needed in adulthood is not expected or wanted from them.'' -- \cite[pp. 114--115]{Aron2013}

\subsection{Giftedness}
``If you were labeled gifted, your childhood may have been easier. Your sensitivity was understood as part of a larger trait that was more socially accepted. There existed better advice to teachers \& parents concerning gifted children. E.g., 1 researcher reminds parents that such children cannot be expected to blend well with their peers. Parents will not produce a spoiled freak if they give their child special treatment \& extra opportunities. Parents \& teachers are firmly told to allow gifted children to just be who they are. This is good advice for children with \textit{all} traits that miss the average \& ideal, but giftedness is valued enough to permit deviation from the norm.

There is some good \& bad in everything, however. Parents or teachers may have pressured you. Your self-worth may have been entirely contingent upon your achievements. Meanwhile, if you were not with gifted peers, you would be lonely \& possibly rejected. There are now some better guidelines for raising gifted children. I have adapted them for reparenting your gifted self.'' -- \cite[p. 115]{Aron2013}

\subsection{Reparenting Your ``Gifted'' Self}

\begin{enumerate}
	\item ``Appreciate yourself for being, not doing.
	\item Praise yourself for taking risks \& learning something new rather than for your successes; it will help you cope with failure.
	\item Try not to constantly compare yourself to others; it invites excessive competition.
	\item Give yourself opportunities to interact with other gifted people.
	\item Do not overschedule yourself. Allow time to think, to daydream.
	\item Keep your expectations realistic.
	\item Do not hide your abilities.
	\item Be your own advocate. Support your right to be yourself.
	\item Accept it when you have narrow interests. Or broad ones.
\end{enumerate}
About this last point -- maybe you just want to study neutrinos \& nothing else. Or maybe you just want to read, travel, study, or talk until you figure out the meaning of human life on this planet. It takes both types to make a world. (Besides, you will probably change at another stage in life.) More will be said about gifted in adults (a neglected topic) in Chap. 6.'''' -- \cite[pp. 115--116]{Aron2013}

\subsection{The Highly Sensitive Adolescent}
``Adolescence is a difficult time for anyone. But my research has found that on the average HSPs report their high school years as the most difficult of all. There are mind-boggling biological changes \& the rapid addition of 1 adult responsibility after another: driving, vocational or college choices, the proper use of alcohol \& drugs, potential parenthood, being trusted with children in jobs as baby-sitters or camp counselors, \& little things like keeping track of ID, money, \& keys. Then there is the big one, the awakening of sexual feelings \& the painful self-consciousness that it brings. Sensitive youths seem bound to feel uneasy with the sexual roles of victim or aggressor that the media imply they are expected to play.

It is also possible, however, to displace energy or anxiety onto sex because the real source of the anxiety is harder to face. Think of the pressure to make choices that will determine your entire life, with no idea of the outcome; the expectation that you'll leave the home you've always known \& do so gladly or at least resolutely; the fear that now your ``fatal flaw'' will reveal itself fully as you fail to make the expected transition to independent living.

It is not surprising that many sensitive adolescents meet the crisis by destroying their budding self so they will not have to watch it fail to bloom ``right.'' \& there are plenty of ways to self-destruct: marrying or having a baby in a way that imprisons one in a narrow, prescribed role; abusing drugs or alcohol; becoming physically or mentally incapacitated; joining a cult or organization that offers security \& answers; or suicide. It is not that all of these behaviors are caused by being sensitive (or that the self, tough plant that it is, will not survive some of them \& be a late bloomer). But these escapes, available to all adolescents, are used by some HSPs as well.

Of course, for many the duties of adulthood are postponed by going to college. (Then there may be graduate school, a postdoctorate, an internship.) Or one finds another way to assume life's duties very gradually. Delaying, as opposed to avoidance, is a fine tactic, another form of the method of learning I call step-by-step. Never feel bad about using it for a while.

Maybe you delayed leaving home. You lived with parents for a few years, worked for them for a while, or moved in with hometown high school friends. Becoming a functioning adult step-by-step really works fine. Suddenly 1 day you are an adult, doing it all, \& you never noticed how you got there.

Sometimes, however, we take too big a step. College can be that for some HSPs. I have known so many HSPs who dropped out after the 1st term (or after their 1st return home, often at Christmas). Neither they nor their parents nor their counselors understand the real problem, overstimulation from a whole new life -- new people, new ideas, new life plans, plus living in a noisy dorm \& staying up all night talking or partying, plus probably experimenting with sex, drugs, \& alcohol (or nursing your friends through the aftereffects of \textit{their} experiments).

Even when the sensitive student would rather withdraw \& rest, there is that pressure to do what others do, be normal, keep up, make friends, satisfy everyone's expectations. Whatever trouble you had in college should be reframed. It was not some personal failure.

Not surprisingly, a good home life helps all adolescents a great deal, even at the time for leaving the nest. The enduring influence of the home is especially strong for HSPs. BY adolescence your family had taught you a great deal about how you can \& should behave in the real world.'' -- \cite[pp. 116--118]{Aron2013}

\subsection{When Sensitive Boys \& Girls Become Men \& Women}
``As highly sensitive adolescents become adults, the differences between the genders increase. Like tiny variations in direction at the start of a trip, differences in upbringing can cause sensitive men \& women to arrive at very different destinations.

In general, men have higher self-esteem than women. When parents appreciate their sensitive boy, as in the case of Charles in Chap. 1, then, as an adult, he will have great self-confidence. At the other extreme, I found many highly sensitive men who were filled with self-loathing -- not surprising, given the rejections they had experienced.

A study of men who had been shy since childhood (I assume that most were HSPs) found that they married an average of 3 years later than other men, had their 1st child 4 years later, \& began a stable career 3 years later, which in turn tended to lead to lower professional achievement. This could reflect cultural prejudice against shy men or lower self-confidence. It could also indicate the kind of caution \& delaying that is healthy for an HSP or the valuing of other things beside family \& career -- perhaps spiritual or artistic goals. At any rate, if you have been slow to take these steps, you have plenty of company.

In contrast, the same study found that shy women went through the traditional stages of life right on time. A shy woman was far less likely to have worked at all or to have continued working when she married, as if availing herself of the patriarchal tradition of going from her father's house to her husband's without having to learn to support herself.

Yet in high school these same women tended to have a ``quiet independence, an interest in things intellectual, a high aspiration level, \& an inner-directedness.'' One can only imagine the tension in these women's lives created by that ``quiet independence,'' the need to follow their inner direction, \& their sense that the only safe, quiet oasis for them was a traditional marriage.

Many of the women I interviewed felt that their 1st marriages were mistakes, attempts to deal with their sensitivity through adding another person to their life or by assuming a safe role. I do not know if their divorce rate is any higher, but their reasons might be different from those of other women. It seems that they are eventually forced both to face the world alone \& to find outlets for their strong intuition, creativity, \& other talents. If their 1st marriages did not allow room for that growth, it became a stepping-stone from home to greater independence when they were finally ready.

Marsha was certainly 1 such woman. She married young \& waited until her 40s to develop the creative \& intellectual abilities so evident in hr school years. For Marsha (\& about $\frac{1}{3}$ of the women I interviewed) there may have been more to this hesitancy about the world than simple sensitivity. These women had had upsetting sexual experiences -- Marsha with her brothers. Even without overt sexual abuse, all young women are known to experience a descent into low self-esteem at puberty, probably as they realize their role as sexual objects. The highly sensitive girl will sense all the implications even more \& make self-protection a high priority. Some overeat to become unattractive, some overstudy or overtrain so they have no free time, some pick 1 boy early \& hang on to him for protection.

Marsha reported that her leadership \& classroom brilliance ended in junior high school, as soon as her breasts developed (with above-average fullness). Suddenly she was drawing constant attention from boys. She wore an overcoat to school in all weather \& became as inconspicuous as she could. Besides, as she put it, the leaders now were the ``dumb, giggly boy chasers.'' She could not or would not be 1 of them.

She was often accosted by boys, anyway. 1 day a pair chased her \& stole a kiss. She went home horrified, stepped into the house, \& saw an actual or visionary rat -- she never knew which -- hurtling down the stairs at her. For years after that, whenever she kissed a boy, she saw that rat.

At 16 she fell in love for the 1st time, but broke it off when they seemed to be getting too close. She remained a virgin until 23, when she was date-raped. After that she gave herself to anyone who persisted -- ``except the boys I really loved.'' Then came an abusive marriage, the long wait for the courage to divorce that man, \& the beginning of her artistic career.

In sum, once again there is the gender difference in how sensitivity is manifested. As sensitive boys become men, they must fall out of step with other men in the timing \& the nature of their lives. Being sensitive is not ``normal'' for men. Meanwhile, for women sensitivity is expected. Sensitive girls find it all too easy to take the path of traditional values without 1st learning how to be in the world.'' -- \cite[pp. 118--120]{Aron2013}

\subsection{Growing Up's Bottom Line: We Grow Into a Highly Social World}
``We are at the end of a chapter but possibly the start of a life's work: learning to see your childhood in the light of your trait \& reparenting yourself when necessary.

Looking back, you will notice how much this chapter about growing up highly sensitive has been concerned with you \& your relationships with others -- with parents, relatives, peers, teachers, strangers, friends, dating partners, spouses. Humans are very social animals, even we HSPs! It seems to be time to turn to the HSP's social life \& to this world which keeps coming up, this state of mind called ``shy.'''' -- \cite[p. 120]{Aron2013}

\subsection{Working With What You Have Learn: Reframing Your Childhood}
``The heart of this chapter \& perhaps of this book is the reframing of your life in terms of your sensitivity. It is the task of seeing your failures, hurts, shyness, embarrassing moments, \& all the rest in a new way, one that is both more coolly accurate \& warmly compassionate.

List the major events you remember from childhood \& adolescence, the memories that shaped who you are. These might be single moments -- a school play or the day your parents told you they were divorcing. Or they might be whole categories -- the 1st day of school each year or being sent to camp each summer. Some memories will be negative, even traumatic \& tragic. Being bullied or teased. Some will have been positive but still perhaps overwhelming: Christmas morning, family vacations, successes, honors.

Choose 1 event \& go through the steps for reframing introduced in Chap. 1:
\begin{enumerate}
	\item \textit{Think about your response to the event \& how you have always viewed it}. Did you feel you responded ``wrong'' or not like others would have? Or for too long? Did you decide you were no good in some way? Did you try to hide your upset from others? Or did others find out \& tell you that you were being ``too much''?
	\item \textit{Consider your response in the light of what you know now about how your body automatically operates}. Or imagine me, an author, explaining it to you.
	\item \textit{Think if there's anything that needs to be done about it now}. If it seems right, share your new view of the situation with someone else. Perhaps it could even be someone who was present at the time who could help you continue to fit details into the picture. Or write down your old \& new view of the experience \& keep it around for a while as a reminder.
\end{enumerate}
If this is helpful, reframe another major childhood event in a few days, until you have gone through the list. Do not rush the process. Allow a few days for each. A major event deserves time to be digested.'' -- \cite[pp. 120--121]{Aron2013}

%------------------------------------------------------------------------------%

\section{Social Relationships}

\subsection{The Slide Into ``Shy''}
````You're too shy.'' Did you hear that often? You will think about it differently after reading this chapter, which discusses where shyness usually happens most: in your nonintimate social relationships. (The close ones are discussed in Chap. 7). Many of you are gifted socially -- that's a fact. But since there is no point in fixing something that's not broken, I'll focus here on a problem that typically needs fixing -- what others call ``shyness,'' social ``avoidance,'' social ``phobia.'' But we will approach it, \& a few other common issues for HSPs, in a very different way.

Again, by focusing on problems I don't mean to imply that HSPs necessarily have a difficult social life. But even the president of the United States \& the queen of England must sometimes worry about what others are thinking about them. So you probably worry about that, too, sometimes. \& worry makes us overaroused, our special Achilles heel.

Also, we often are told, ``Don't worry; no one is judging you.'' But being sensitive, you may be noticing that people really are watching \& judging; people usually do. The nonsensitive are often happily oblivious of it. So your task in life is much harder: to \textit{know} about those glances, those silent judgments, \& still not let them affect you too much. It's not easy.'' -- \cite[p. 122]{Aron2013}

\subsection{If You Have Always Considered Yourself Shy}
``Most people confuse sensitivity with shyness. That is why you heard ``You're too shy.'' People say a certain dog, cat, or horse was born ``shy'' when it really has a sensitive nervous system (unless it has been abused; then it would be more accurate to say it's ``afraid''). Shyness is the fear others are not going to like or approve of us. That makes it a response to a situation. It is a certain \textit{state}, not an always-present trait. Shyness, even chronic shyness, is not inherited. Sensitivity is. \& while chronic shyness does develop more in HSPs, it needn't. I have met many HSPs who are almost never shy.

If you often feel shy, there is a good explanation for how you or anyone else probably got that way, non-HSPs included. Sometime in your past you entered a social situation (usually overstimulating to begin with) \& felt that you failed. Others said you did something wrong or did not seem to like you, or you failed to meet your own standards in the situation. Maybe you were already overaroused, having used your excellent imagination to envision all that might go wrong.

Usually 1 failure is not enough to make anyone chronically shy, although it can happen. Usually it happens that the next time you were in the same situation, you were more aroused because you feared a repeat of the previous time. \& being more aroused made failure more likely. By the 3rd time, you were being very brave, but you were also impossibly aroused. You couldn't think of anything to say, you acted inferior \& were treated that way, \& so forth. You can see how this pattern could repeat \& repeat into downward spiral. It can also spread to other situations that are even a little similar, like all situations with people present!

HSPs, being more easily aroused, are more likely to enter that spiral. But you were not born shy, just sensitive.'' -- \cite[pp. 122--123]{Aron2013}

\subsection{Ridding ``Shyness'' From Your Self-Concept}
``There are 3 problems with accepting the label ``shy.'' 1st, it is totally inaccurate. It misses the real you, your sensitivity to subtlety \& your difficulty with overarousal. Remember, overarousal is not always due to fear. Thinking it is fear can make you feel shy when you are not, as we will see.

This confusion of your trait with the state of mind called shyness is natural, given that 75\% of the population (at least in the United States) are very socially outgoing. When they see that you look overaroused, they do not realize that it could be due to too much stimulation. That is not \textit{their} experience. They think you must be afraid of being rejected. You're shy. You fear rejection. Why else would you not be socializing?

Sometimes you \textit{are} afraid of rejection. Why not? Your style is not the cultural ideal, after all. But as an HSP, sometimes you just don't want the extra arousal. When others are treating you as if you're shy \& afraid, it can be hard to realize that you've simply chosen to be alone, at least at 1st. You are the one rejecting. You are not being rejected. (Besides not understanding because they were born needing more arousal than you to be comfortable, non-HSPs also can project their own fear of rejection onto you -- i.e., attributing to you something they do not want to admit in themselves.)

If you are spending less time in crowds or meeting strangers, when you do have to be in such situations, you are almost bound to be less skilled. It is not your specialty. But again, to assume you are shy or afraid is inaccurate. When people set out to help you, they are usually starting from the wrong premise. E.g., they think you lack confidence \& reassure you that you are likable. In effect, however, that is telling you there is something the matter with you -- low self-esteem. Not knowing your underlying trait, they give you the wrong reason for your being less sociable \& cannot give you the many real reasons you should feel fine about yourself.'' -- \cite[pp. 123--124]{Aron2013}

\subsection{Calling Yourself Shy Is Negative}
``Unfortunately, the term shy has some very negative connotations. It does not have to; shy can also be equated with words such as discreet, self-controlled, thoughtful, \& sensitive. But studies have shown that most people on 1st meeting those I would call HSPs considered them shy \& equated that with anxious, awkward, fearful, inhibited, \& timid. Even mental health professionals have rated them, more often than not, this way \& also as lower on intellectual competence, achievement, \& mental health, which, in fact, bear no association with shyness. Only people who knew the shy people well, such as their spouses, chose the positive terms. Another study found that the tests used by psychologists to measure shyness are replete with the same negative terms. Maybe that would be all right if the tests were of a state of mind, but they're often used to identify ``shy people,'' who then bear a negative label. Beware of the hidden prejudice behind the word shy.'' -- \cite[p. 125]{Aron2013}

\subsection{Calling Yourself Shy is Self-Fulfilling}
``A rather charming psychological experiment involving shyness, done at Stanford University by Susan Brodt \& Philip Zimbardo, demonstrates why you need to know that you are not shy but just an HSP who can become overaroused.

Brodt \& Zimbardo found women students who said they were extremely ``shy,'' especially with men, \& others who were not ``shy,'' to serve as a comparison group. In the study, which supposedly concerned the effects of loud noise, each woman spent time with a young man. The man, who was unaware of whether or not the woman was ``shy,'' had been instructed to converse with each woman in the same style. The interesting twist was that some of the shy women were fooled into thinking that their overarousal -- their pounding heart \& racing pulse -- was due to the loud noise.

The result was that those ``shy'' women who believed their overarousal had been caused by a loud noise talked just as much as the nonshy women. They even took charge, controlling the topic of conversation just as much as the nonshy women did. The other group of shy women, who had nothing else on which to blame their arousal, talked much less \& allowed the man to control the conversation much more. After the experiment, the young man was asked to guess which women were shy. He could not distinguish nonshy women from shy women who had been led to believe their arousal was due to the noise.

These shy women became less shy by assuming that there was no \textit{social} reason for their overarousal. They also said they did not feel shy \& truly enjoyed the experience. Indeed, when asked if they would prefer to be alone next time if they were again a participant in a ``noise bombardment experiment,'' $\frac{2}{3}$ said they would prefer not to be, compared to only 14\% of the other shy women \& 25\% of the nonshy. Apparently these shy women had an especially fine time just because they though their overorousal was caused by something besides shyness.

Remember this experiment the next time you feel overaroused in a social situation. Your heart may be pounding for any number of reasons having nothing to do with the people you are with. There may be too much noise, or you may be worrying about something else you are only half aware of that has nothing to do with the person you are with. So go ahead, ignore the other causes (if you can), \& have a good time.

I have given you 3 strong reasons not to call yourself shy anymore. It is inaccurate, negative, \& self-fulfilling. \& do not let others label you with it, either. Let's say it is your civic duty to eradicate this social prejudice. Not only is it unfair, but as discussed in Chap. 1, it is dangerous because it helps to silence the thoughtful voices of HSPs by reducing their self-confidence.'' -- \cite[pp. 125--126]{Aron2013}

\subsection{How to Think About Your ``Social Discomfort''}
``Social discomfort (the term I prefer to ``shy'') is almost always due to overarousal, which makes you act, speak, or appear not very socially skilled. Or it is the dread that you will become overaroused. You dread doing something awkward, not being able to think of what to say. But the dread itself is usually enough now to create the overarousal, once in the situation.

Remember, discomfort is temporary, \& it gives you choices. Suppose you are uncomfortably cold. You can tolerate it. You can find a more congenial environment. You can create some heat -- build a fire, turn up the thermostat -- or ask those in charge to do it. You can put on a coat. The 1 thing you should not do is blame yourself for being inherently more susceptible to a cold environment.

The same is true of a temporary social discomfort due to overarousal. You can put up with it, leave the situation, charge the social atmosphere or ask others to, or do something else to make you more comfortable, like put on your ``persona'' (I'll discuss this later).

In all cases, you are consciously ridding yourself of the discomfort. So forget the idea that you are inherently uncomfortable in social situations.'' -- \cite[pp. 126--127]{Aron2013}

\subsection{5 Ways to Handle Overarousal in Social Situations}

\begin{enumerate}
	\item ``Remember that overarousal is not necessarily fear.
	\item Find other HSPs to talk to, 1 on 1.
	\item Use your arousal-reducing skills.
	\item Develop a good ``persona'' \& consciously use it.
	\item Explain your trait to others.
\end{enumerate}
Never underestimate the power of simply acknowledging to yourself that you are overaroused, possibly by something having nothing to do with the people you are with. If you are judged for that, it is not the real you but the one temporarily flustered by overarousal. If \& when they know the calm you, the you who is subtly aware, they will be favorably impressed. You know that is true because you have close friends who admire you.

When I went back to graduate school at midlife, on the very 1st day, in the 1st hour, in the breakfast room, I dropped a full glass of milk all over myself \& the floor \& several others in the vicinity. No one had bumped me. I just hit it against something. It happened in full view of all of my future fellow students \& faculty, the people I most wanted to impress.

The pure shock added to my already almost unbearable overarousal. But thanks only to the research I was doing on HSPs like you \& me, I knew all about why I had done it. My body's inability to even carry milk was predictable. The day was difficult, but I did not let the spilled milk add to my social discomfort.

As the day went by, I found other HSPs, \& that helped a great deal. We were all spilling milk, so to speak. In the average social situation there ought to be about 20\% who are HSPs \& another 30\% who feel moderately sensitive. Studies of shyness find that on an anonymous questionnaire 40\% call themselves shy. In a roomful of people, the odds are that there is at least 1 person with your trait or who is feeling social discomfort. Catch their eyes after you stumble, literally or metaphorically, \& notice the look of deep sympathy. You have an instant friend.

Meanwhile, use all the points suggested in Chap. 3 to reduce your arousal. Take breaks. Go for a walk. Breathe deeply. More in some way. Consider your options. Maybe it's time to go. Maybe there's a better place to position yourself, by an open window, an aisle, or the door. Think in terms of containers -- who or what quiet, familiar presence could hold you right now?

There were times on that 1st day of graduate school when I feared that the faculty would think something was seriously wrong with me. With the average non-HSP, being this overaroused could only mean serious conflicts \& instability. So I used all my tricks -- walking, meditating, driving off campus at lunchtime, calling home for some comfort. \& it worked well enough.

We often think our overarousal is more noticeable to others than it really is. You know that much of social life is 1 ``persona'' meeting another, with neither person looking too far beneath the surface. By behaving in a predictable way, talking the way others do even when you don't feel like it, no one will hassle you or draw the wrong conclusion that you're arrogant, aloof, plotting, \& so forth. E.g., research finds that ``shy'' students tend to see themselves as doing their best socially, but their roommates tend to think they're just not trying enough. That may be the fault of the culture for not understanding HSPs, but until we change it, you may want to make your life a little easier by acting a little more like everyone else does. Put on your persona; the term comes from the Greek word for mask. Behind the mask you can be whoever you want.

On the other hand, sometimes the best tactic is to explain your overarousal. I often do this when speaking or teaching in front of a group of strangers. I tell them that I know I sound a little strained but in a few minutes I'll be fine. In a group, explaining your trait may lead to a more intimate conversation about everyone's social discomfort, make it possible for you to go off alone without feeling guilty, or free you to take a break without being left out when you return. Perhaps there is someone who could lessen the stimulation you're experiencing -- ad--just the lighting or volume or let you pass when introductions are made.

Once you mention being highly sensitive, you'll arouse 1 of 2 stereotypes, depending on your choice of words. 1 stereotype, frankly, is of a passive victim, someone weak \& troubled. The other is of a gifted, deep, powerful presence in the room. It takes practice bringing up the positive stereotype through the words you choose to explain your needs. We'll work on that in Chap. 6.

When I have to be with a group of people for a whole day or weekend, I often explain that I need plenty of time alone. Often others do, too. But even if I'm the only one going to my room early \& taking long walks alone, I've learned not to generate sympathy or pity but to leave behind an air of mystery. Members of the ``royal advisor'' class must consider these matters. Be a little cagey about your HSP ``PR.'''' -- \cite[pp. 127--129]{Aron2013}

\subsection{People, Arousal, \& Introversion}
``Thus far we have attacked the ``problem'' by getting rid of the label shy \& understanding what is going on as familiar overarousal. It is equally important that you appreciate that there's more than 1 right way to be social.

Your way of being social arises from a basic fact: For most of us, the majority of the arousing stimulation in the external world is created by other people, whether at home, at work, or in public. We're \textit{all} social beings who enjoy \& must depend on others. But many HSPs avoid people who come in the overstimulating packages -- the strangers, the big parties, the crowds. For most HSPs, this is a smart strategy. In a highly stimulating, demanding world, everyone has to establish priorities.

Of course, no one can be an expert at dealing with the situations they choose to avoid. But most of you can manage or learn to. Just managing is an acceptable, smart way to save your energy for whatever else matters to you.

It's also true that some HSPs avoid strangers, parties, \& other group situations because of having been rejected by peers \& groups in the past. Because they didn't fit our culture's ideal of being outgoing, they've been judged harshly \& avoid people they cannot be sure of. That seems reasonable, although sad, \& is nothing to be ashamed of.

In all, 70\% of HSPs tend to be socially ``introverted.'' That does not mean you dislike people. It means you prefer to have a few close relationships rather than a large circle of friends \& don't usually enjoy large parties or crowds. But even the most introverted person is sometimes an extravert \& enjoys a stranger or a crowd. Even the most extraverted is sometimes an introvert.

Introverts are still social beings. In fact, their well-being is more affected by their social relationships than is the well-being of extraverts. Introverts just go for quality, not quantity.

(If you're not enjoying a sense of emotional well-being, however, a close relationship with someone does not always solve that problem. Many people, in fact, cannot have a good, close relationship until they develop a greater sense of well-being through some healing work in psychotherapy, in the broadest sense, as described in Chap. 8.)'' -- \cite[pp. 129--130]{Aron2013}

\subsection{The Extraverted HSP}
``I want to emphasize that being an HSP is not the same as being socially introverted. In my studies I've found that 30\% of us are socially extraverted. As an extravert, you have large circles of friends \& tend to enjoy groups \& strangers. Perhaps you were raised in a big, sociable, loving family or safe neighborhood \& learned to see people as sources of safety rather than reasons to be on guard.

You still find other sources of arousal difficult, however, like a long work day or being in the city too much. When overaroused, you avoid socializing. (Extraverted non-HSPs actually relax better with people around.) While most of our attention here is on the habitually introverted, extraverts will probably find it useful, too.'' -- \cite[p. 130]{Aron2013}

\subsection{Appreciating the Introverted Style}
``Avril Thorne, now of the University of California at Santa Cruz, sat down to watch how introverts actually interact. Using tests to identify highly introverted \& highly extraverted women college students, she paired them either with the same sort of person or their opposite \& videotaped the conversations.

The highly introverted women were serious \& focused. They talked more about problems \& were more cautious. They tended to listen, to interview, to give advice; they seemed to be concentrating on the other in a deep way.

In contrast, the highly extraverted women did more ``pleasure'' talk, sought more agreement, looked for similarities in background \& experience, \& paid more compliments. They were upbeat \& expansive \& liked being paired with either type, as if their main pleasure were in the talking.

When the extraverted were with someone who was highly introverted, they liked not having to be so cheerful. \& the introverted found conversing with the extraverted ``a breath of fresh air.'' The picture we gain from Thorne is that each type contributes something to this world that is \textit{equally} important. But given the undervaluing of the introverted style, it will be good to spend some time focusing on the virtues of the introverted.'' -- \cite[p. 131]{Aron2013}

\subsection{Carl Jung on the Introverted Style}
``Carl Jung saw introversion as a basic division among humans, causing the major battles of philosophy \& psychology, most of which boil down to conflicts over whether the outer facts or the inner understanding of those facts are more important in comprehending any situation or subject.

Jung saw the 2 as attitudes toward life, which in most people alternate, like breathing in \& out. But a few are more consistently in or out. Furthermore, to him the 2 attitudes had nothing directly to do with being sociable or not. To be introverted is simply to turn inward, towards the subject, the self, rather than outward toward the object. Introversion arises from a need \& preference to protect the inner, ``subjective'' aspect of life, to value it more, \& in particular not to allow it to be overwhelmed by the ``objective'' world. 

One cannot emphasize enough the importance of introverts in Jung's view.

They are living evidence that this rich \& varied world with its overflowing \& intoxicating life is not purely external, but also exists within $\ldots$ Their life teaches more than their words $\ldots$ Their lives teach the other possibility, the interior life which is so painfully wanting in our civilization.

Jung knew the prejudice in Western culture against the introverted. He could tolerate it when it came from the extraverted. But he felt that the introverted who undervalue themselves are truly doing the world a disservice.'' -- \cite[pp. 131--132]{Aron2013}

\subsection{It Takes All Kinds}
``Sometimes we do need just to enjoy the world out there as it is \& be glad for those who help us, the extraverted who can make even total strangers feel connected. Sometimes we need an inner anchor -- i.e., those who are introverted \& give their full attention to the deepest nuances of private experience. Life is not just about the motives we have both seen \& the restaurants we have both tried. Sometimes discussing the subtler questions is essential for the soul.

Linda Silverman, an expert on gifted children, found that the brighter the child, the more likely he or she will be introverted. Introverted are exceptionally creative even with something as simple as the number of unusual responses to a Rorschach inkblot test. They are also more flexible in a sense, in that sometimes they \textit{must} do what extraverts do all the time, meet strangers \& go to parties. But some extraverted people can avoid being introverted, turning inward, for years at a time. This greater versatility on the part of some introverts is especially important later in life, when one begins to develop what one has lacked up until midlife. Later in life, too, self-reflection becomes more important for everyone. In short, introverts may mature more gracefully.

So you're in good company. Ignore the barbs about ``lightening up.'' Enjoy the levity of others \& allow yourself your own specialty. If you are not good at chitchat, be proud of your silence. Equally important, when your mood changes \& your extraverted self appears, let it be as clumsy or silly as it needs to be. We are all awkward doing our nonspecialty. Your possess 1 piece of the ``good.'' It would only be arrogance to think any of us should have it all.'' -- \cite[pp. 132--133]{Aron2013}

\subsection{Making Friends}
``Introverts prefer close relationships for many reasons. Intimates can understand \& support each other best. A good friend or partner can also upset you more, but that forces inner growth, which is often a high priority for HSPs. \&, given your intuition, you probably like to talk about complicated things like philosophy, feelings, \& struggles. That is hard to do with a stranger or at a party. Finally, introverts possess traits that can make them good at close relationships; with intimates they can experience social success.

The extraverted are right, however, when they say that ``a stranger is just a friend I haven't met yet.'' All your closest friends were once strangers. As those relationships change (or even end), you'll always want to meet new potential close friends. Do you might think back over how you met your best friends.'' -- \cite[p. 133]{Aron2013}

\subsection{The Persona \& Good Manners}
``Especially if you are unusually introverted, remember that in most social situations you at least need to meet minimal social expectations. HSPs can reduce all rules of etiquette to a 4-word rule: Minimize the other's overarousal. (Or to 2 words: Be kind.) Dead silence, since it is not expected, can arouse another person. But so can being too outgoing, which is often the extravert's mistake. The goal is just to say something pleasant \& unsurprising.

Yes, this may bore some people who are nonsensitive \& enjoy lots of stimulation. But you want your short-term arousal when meeting a new person to settle down even if it is not a problem for the other. Then later on you can be as creative \& surprising as you want. (But at that point you are taking calculated risks, \& any successes can be counted as extra credit points.)
\begin{quotation}
	\textbf{How you met your best friends}
	
	Write down the names of your best friends, 1 to a page of paper. Then answer the following questions about the beginning of each friendship.
	
	Did circumstances force you to talk?
	
	Did the other take the initiative?
	
	Was there anything unusual about how you were feeling?
	
	Were you being especially extraverted that day?
	
	How were you dressed or feeling about your appearance?
	
	Where were you? At school, at work, on vacation, at a party?
	
	What was the situation? Who introduced you? Or were you thrown together by chance? Or did 1 of you happen to speak to the other about something? What happened?
	
	What were the 1st moments \& hours \& days like?
	
	When \& how did you know this would be a friendship?
	
	Now look for commonalities among these. E.g., you may not like parties, but that was the setting for meeting 2 of your best friends. Are any of these common features, like going to school or working with others, absent from your life at present? Is there anything you want to do about what you have learned? Vow to go to 1 party a month? (Or to avoid parties from now on -- they turned out not to be a source of friends, after all.)
\end{quotation}
Now you need a more advanced course on personas, or social roles. A good persona obviously involves good manners \& predictable, nonarousing behavior. But it can be a bit more specialized, according to your needs. A banker wants to have a solid, practical persona. If there's an artist within, that is kept private. Artists, on the other hand, will do well to keep their banker sensibilities hidden from public view. Students are smarts to appear a little humble; teachers need to appear authoritative.

The idea of a persona goes against North American culture's admiration of openness \& authenticity. Europeans have a far better grasp of the value of not saying everything that one is thinking. Yet there are people who identify too much with their persona. We all know the type. Having nothing else underneath, it's hard to say they're being dishonest or unauthentic. But it's rare for an HSP to overidentify with a persona.

If you still think I'm asking you to be insincere, think of it as choosing the appropriate level of openness for that place \& time. Take the example of when you've barely met someone who wants a friendship with you \& you have decided not to pursue it. You probably don't reject the other's lunch invitation by saying, ``I have come to realize that I don't want to be a close friend of yours.'' You say something about having a very busy schedule right now.

This response is honest at a certain level -- if you had infinite time you might actually pursue such a relationship at least a step further. To tell the other person what put them low on your list of priorities is not, in my experience, morally correct. The best persona \& good manners involve this sort of face-saving, compassionate level of honesty, especially with those you don't know particularly well.'' -- \cite[pp. 133--135]{Aron2013}

\subsection{Learning More Social Skills}
``There are 2 kinds of information about social skills, whether it comes packaged as a book, tape, article, lecture, or course. 1 kind comes from the experts on extraversion, social skills, sales, personnel management, \& etiquette. These folks are often witty \& upbeat. They talk about learning, not curing, so they do not lower your self-esteem by implying that you have a serious problem. If you turn to these pros, just understand that your goal is not to be exactly like them but to learn a few tips. Watch for titles like \textit{How to Win Over a Crowd} \& \textit{What to Say in Every Possible Awkward Moment}. (I made these up; new ones are constantly appearing.)

The other kind of information comes from psychologists trying to help people with shyness. Their style is 1st to make you worried so you'll be motivated, then to take you step by step through some pretty sophisticated, well-researched methods of changing your behaviors. This approach can be very effective, but also has some problems for HSPs, although it may seem more suited to you. Talk about ``curing'' your shyness or ``conquering your syndrome'' cannot help but make you feel flawed, \& it overlooks the positive side of your inherited trait.

Whatever advice you read or hear, remember that you do not have to accept how the extraverted $\frac{3}{4}$ of the population defines social skills -- working the room, always having a good comeback, never allowing ``awkward'' silences. You have your own skills -- talking seriously, listening well, allowing silences in which deeper thoughts can develop.

It is also probably true that you already know much of what is covered by these experts. So I've taken the main points \& put them into a short test to show you what you do know \& teach you some of the rest.

\textbf{Do you know the latest on overcoming social discomfort?} Answer true{\tt/}false. If you were right on a dozen or more, sorry to have bored you. You should write your own book. Otherwise, these answers provide much of what you need to know.
\begin{enumerate}
	\item It helps to try to control negative ``self-talk,'' such as ``He probably won't like me'' or ``I'll probably fail as I always do.''
	
	True. ``Negative self-talk'' keeps you aroused \& makes it hard to listen to the other person.
	\item When people feel shy, it is obvious to those around them.
	
	False. You, an HSP, may notice shyness in others, but most people don't.
	\item You need to expect some rejections \& not take it personally.
	
	True. People can reject you for all sorts of reasons having nothing to do with you. If it upsets you, feel that for a moment. Then try to let it go.
	\item It helps to have a plan for overcoming your social discomfort -- e.g., trying to meet 1 new person a week.
	
	True. Decide to take so many specific, gradual steps per day or week no matter how nervous the 1st steps make you.
	\item When formulating your plan, the bigger the steps you take, the faster you will achieve your goal.
	
	False. Big steps would be best if you could take those steps. But since you're a little afraid, \& also afraid you will fail, you must promise the fearful part of yourself that you'll not go too fast, even as you're firm that the fear is going to be overcome eventually.
	\item It is best not to rehearse what you'll say with a new person or in a new situation; it will make you sound stiff \& unspontaneous.
	
	False. The more you rehearse, the less nervous you'll be -- which means you'll be more, not less, relaxed \& spontaneous.
	\item Be careful about body language; the less it conveys, the better.
	
	False. Body language is always conveying something. A stiff, still body can be interpreted many ways, but most of them won't be positive. Better to let your body move \& show some interest, caring, enthusiasm, or sheer liveliness.
	\item When trying to get a conversation started or to continue, ask questions that are a little bit personal \& that cannot be answered with 1 or 2 words.
	
	True. It's okay to pry a little. Most people love to talk about themselves \& will like your interest \& slight boldness.
	\item A way to show you are listening is to sit back with your arms \& legs crossed, keep your face still, \& don't meet the other person's eyes.
	
	False. Stand or sit as close as is appropriate \& comfortable, lean forward, uncross your arms \& legs, \& make frequent eye contact. If eye contact is too arousing, it's okay to look at the other person's nose or ear -- people cannot tell the difference. Smile \& use other facial expressions (being careful, of course, not to convey more interest than you want to).
	\item Never touch another person.
	
	False. Depending on the situation, of course, a brief touch on the shoulder, arm, or hand, especially at parting, just conveys warmth.
	\item Don't read the newspaper before going out where you will meet people -- it will just upset you.
	
	False. In general, a glance at the paper will give you some ideas for conversations \& connect you up with the world. Just avoid the depressing stories.
	\item Self-disclosure is not important to conversation as long as you are talking about something interesting.
	
	False. Self-disclosure is important if your goal is to feel some connection \& not just pass the time. This doesn't mean you have to reveal deep secrets. Too much self-disclosure too soon will create overarousal, besides seeming inappropriate. Be sure to ask for the other's opinion, too, of course.
	\item Good listeners repeat back what they've heard, reflecting the other's feelings, \& then respond with their own feelings, not ideas.
	
	True. E.g., someone says they are excited about a new project. You can say, ``Wow, I hear how excited you are. That must feel great.'' By taking the time to reflect that \textit{feeling} before asking about the specifics of the project, you display 1 of your greatest assets, your sensitivity to feelings. You also encourage the other person to reveal more of his or her inner life, which you would probably prefer to talk about, anyway.
	\item Don't tell other people interesting details about yourself; it will only make them jealous.
	
	False. You don't want to gloat, of course. But everyone wants to be talking with someone worthwhile. Take the time to write out some of the best or most interesting things about yourself \& think of how you might slip them into conversations. Not ``I moved here because I like the mountains'' but ``I moved here because I am starting a mountain-climbing school'' or ``I especially like mountain backdrops for my photographs of rare birds of prey.''
	\item To deepen a conversation or make it more interesting to both of you, sometimes it works to share your own flaws or problems.
	
	True -- \textit{with} some cautions. When 1st meeting someone, you don't want to reveal too many needs for flaws. You don't want to seem self-effacing in a submissive way or unaware of appropriate behavior. But there is also something nice in admitting to your human nature if you can convey that you still feel good about yourself. (My favorite line from Captain Picard of \textit{Star Strek: The Next Generation} is, ``I have made some \textit{fine} mistakes in my life.'' It is so humble, wise, \& self-confident, all at once.) Certainly if the other person has revealed something painful or embarrassing, it will deepen the conversation considerably if you do the same.
	\item Try not to disagree with the other person.
	
	False. Most people enjoy a \textit{little} conflict. Moreover, maybe the point of the conflict is important to you or reveals something you should know about the other person.
	\item When a conversation is making you feel that you want to spend more time with the other person, it's best to say no.
	
	True. Of course, take your time to be sure of what you feel \& be ready for an occasional rejection.
\end{enumerate}
Based on Jonathan Cheek, \textit{Conquering Shyness} (New York: Dell, 1989) \& M. McKay, M. Dewis, \& P. Fanning, \textit{Messages: The Communication Book} (Oakland, Calif.: New Harbinger Press, 1983.)'' -- \cite[pp. 135--137]{Aron2013}

\subsection{Don't Feel Bad If You Know What to Do But Don't Always Do It}
``Gretchen Hill, a psychologist at the University of Kansas, questioned shy \& nonshy people about what was the appropriate behavior in 25 social situations. She found that the shy people knew equally well what was expected of them but said they were not capable of doing it. She hints that shy people lack self-confidence -- the usual inner flaw attributed to us. So we are told to be more confident. Which we can't, of course. So we've failed again. But maybe we're sometimes justified in our lack of confidence, with so many experiences of being too aroused to behave appropriately. Naturally, some of us expect not to be able to do what we know to be socially correct. I think that simply telling ourselves to be more confident rarely helps. Stick to the 2fold approach of this chapter: Work on the overarousal, appreciate your introverted style.

Another reason for not being able to put into practice what you know about social skills is that old patterns from childhood may be taking over \& need to be faced. Or some feelings command your attention. 1 sure sign? You keep saying things like ``I don't know why I did that -- I knew better -- that just wasn't like me.'' Or, ``After all my efforts, nothing is working.'''' -- \cite[pp. 137--138]{Aron2013}

\subsection{The Case of Paula}
``Paula was definitely born highly sensitive. Her parents had commented on her ``shyness'' from birth. She was always aware of having a greater sensitivity to sound \& confusion than her friends. In her 30s, when I interviewed her, she was extremely capable in her profession, which involved organizing major events from behind the scenes. But she stood no chance of advancing because of her terror of public speaking \& of people in general, which kept her from managing anything but the smallest team of coworkers. In fact, Paula had organized her life around the few times when her job demanded that she convene staff meetings. For these she needed to exercise for hours \& performs various rituals in order to prepare herself emotionally.

Paula had read all the books on overcoming such fears \& had used her considerable willpower to fight the feelings. But she realized that her fear was unusual, so she tried some longer, more intense therapy. There she found some reasons for the fear \& started working through it.

As Paula was growing up, her father was a ``rage-aholic'' (He is now an alcoholic as well.) He had always been a smart man, analytic, helpful with his children's homework. Indeed, he was very involved with all of them \& actually a little less cruel to Paula than he was to her brothers. But some of that attention may have been sexual, Paula was beginning to discover, \& was certainly confusing. At any rate, his rage affected her the most.

Paula's mother was very nervous about other people \& their opinions \& was highly dependent on her iron-willed husband. She was also something of a martyr, building her life around their children. Yet she also disliked everything about raising them. Her explicit horror stories of childbirth \& lack of fondness for babies make it seem likely that Paula's 1st attachment was anything but secure. Later, her mother made Paula her confidante, telling her far more than a child could handle, including a whole catalog of reasons to dislike sex. Indeed, both parents told her all about their feelings for each other, including their sexual intimacies.

Given this background, Paula's ``fear of public speaking'' was more akin to a basic distrust of other people. She was born sensitive \& therefore easily overaroused, yes. But she was also insecurely attached as a child, which makes it far harder for a child to face threatening situations with confidence. Indeed, her mother felt \& taught a general irrational fear ot (rather than confidence in) people. Finally, Paula's early attempts at speaking her own mind had been met with rage from her father.

Perhaps a final reason for her fear of speaking in public was that she came to feel she knew too much -- about her father's possibly incestuous feelings for her \& both of her parents' private lives.

These are not easy issues to resolve, but they can be brought into one's consciousness \& worked on with a competent therapist. The voices afraid to speak are finally freer. Specific training in social skills might still be needed afterward, but at that point they should really work.'' -- \cite[pp. 138--139]{Aron2013}

\subsection{Basic Social Advice for HSPs}
``Here are some suggestions about some situations that often cause HSPs social discomfort.

\textit{When you just have to chitchat.} Decide whether you would rather talk or listen. If you want to listen, most people will be glad to talk. Ask a few specific questions. Or just ask, ``So what do you do when you aren't at parties?'' (Or conferences, weddings, concerts, etc.)

If you want to talk (which puts you in control \& keeps you from being bored), plan ahead to plant the topic you enjoy \& can go on \& on about. Like, ``Bad weather, isn't it? At least it makes it easier to stay in \& work on my writing project.'' Of course, the other person will ask what you are writing. Or, ``Bad weather -- I couldn't train today.'' Or, ``Bad weather -- my snakes hate it.''

\textit{Remembering names.} You may forget a person's name because you were distracted \& overaroused when you both 1st met. If you hear a name, try to make it a habit of using it in your next sentence. ``Arnold, how nice to meet you.'' Then use it again within 2 minutes. Thinking back afterward about who you met might make it stick even longer. But trouble with names just goes with the territory.

\textit{Having to make a request.} The small ones, e.g., for information, ought to be easy. But sometimes we put them on our list of things to do \& they just sit there, seeming big \& difficult. If possible, making the requests the moment you realize you need to. Or make them in a bunch, when you are feeling in an outgoing mood. For slightly more important requests, make them small again. Think about how quickly it will be over \& how little trouble it will be for the person you ask. For even more important requests, make a list of what you want to cover. Begin with being sure you're talking to the right person for your purpose. Making an important request should be rehearsed with someone, having the other person respond to you in every possible way. This does not make it much easier. But you will feel more prepared.

\textit{Selling.} Frankly, it's not a usual HSP career. But even if you do not sell a commercial product, there are many times in life when we want to sell an idea, ourselves for a job, or maybe our creative work. \& what if you believe something could truly help a person or the world at large? In its gentlest form, yours probably, selling is simply sharing with others what you know about something. Once they understand what you think is its value, you can let them make up their own mind.

When money is to be exchanged, HSPs often feel guilty that they're taking ``so much'' or anything at all. (\& if we feel flawed, ``What am I really worth, anyway?'') Usually we cannot \& should not give ourselves or our products away. We need money to continue to make available what we're offering. People understand that, just as you do when you purchase something.

\textit{Making a complaint.} This can be difficult for an HSP even if it's legitimate. But it is worth practicing; the assertiveness is empowering for those who often feel put down just for being who they are (too young, too old, too fat, too dark-skinned, too sensitive, etc.)

You must be ready, however, for the other's response. Anger is the most arousing emotion for good reasons; it is meant to mobilize us to fight. It is arousing whether it is yours, theirs, or even that of someone you're just observing from a distance.

\textit{Being in a small group.} Groups, classes, \& committees can be a complicated business for HSPs. We often pick up on a great deal that others do not notice. But our desire not to add to our arousal level may keep us quiet. Eventually, though, someone will ask what \textit{you} think. This is an awkward moment but an important one for the group. Habitually quiet HSPs often fail to allow for the fact that the silent person gains more \& more influence with time. Besides wanting to give you a chance to speak, the group may unconsciously be worried. Are you in the group or out? Are you sitting there judging them? Are you-unhappy \& about to leave? If you left, they would be left with these fears, which is why quiet members eventually get so much attention. It may be out of politeness, too, but the fear is always there as well. If you do not join in with just the right enthusiasm, you will receive considerable attention. Then others may well find that their best defense is rejecting you before you reject them. If you do not believe me, try remaining silent in a new group just once \& you'll see it all unfold.

Given this energy that always goes to the silent one, if you want to be quieter than others, you need to reassure them that you're not rejecting them or planning to leave the group. Tell them you feel part of the group just by listening. Tell them your positive feelings about the group, if you have any. Tell them you'll speak up when you're ready. Or ask them to ask you again.

You can also decide if you want to explain your sensitivity. But it means you'll have a label that will tend to become self-fulfilling.

\textit{Public speaking or performing.} This is a natural for HSPs -- yes, it is. (I leave you to think of all the reasons why it is \textit{harder} for us.) 1st, we often feel we have something important to say that others have missed. When others are grateful for our contribution, we feel rewarded, \& the next time is easier. 2nd, we \textit{prepare}. In some situations, as when we go back to check if we turned off the toaster, we can seem ``compulsive'' to people not as determined to ourselves to prevent all unnecessary surprises (like a burned house). But anyone would be a fool not to ``overprepare'' for the extra arousal due to an audience. Having prepared best, we succeed most. (Those are 2 reasons why all the books on shyness can cite so many politicians, performers, \& comics who ``conquered their shyness, so you can, too.'')

The key, again, is to prepare, prepare, prepare. You probably are not afraid to read out loud, so until you feel more comfortable, prepare exactly what you want to say \& read it. If doing so is a bit unusual for the situation, explain confidently some good reasons for reading. Then do it with authority.

Reading well also requires preparation \& practice. Be sure you use emphasis \& can meet any time limits so that you can read slowly.

Later, you can graduate to notes. In a large group I always make notes before I raise my hand to speak or ask a question just in case my mind goes blank when I'm called on. (I do the same in any situation that makes me overaroused, including doctors' offices.)

Above all, practice as much as possible in front of an audience, replicating, as much as possible, the performance situation. Use the same room at the same time of day, wear the clothes you'll wear, have the sound system set up, \& so forth so that there will be fewer new elements to the situation. It is the greatest secret to getting the arousal under control. Once you do, you might just enjoy yourself up there.

I overcame my fear of public speaking by teaching -- a good beginning for an HSP. You are giving, you are needed, so your conscientious side takes over. The audience is not expecting entertainment, so anything you can do to make the occasion enjoyable will be gratefully received. \& you will find you have real insights once you become bold enough to express them.

Students can sometimes be callous, though. I was lucky to begin at a college where quiet politeness \& open expressions of gratitude were the norm. If you can establish the same norms, it will help all of you in your classroom. Some of your students are also afraid to speak up. You can all learn together.

What if others are watching you? Are they really? Maybe you have created an inner audience that you fear. You can carry such an audience around \& ``project'' it (see it where it is not, or at least is not to the degree you imagine).

If others really are watching, can you ask them not to watch? Can you refuse to be watched? Or can you take any pleasure in their watching?

Here is the story of my only belly-dancing lesson. Learning any physical skill in a group is almost impossible for me because the overarousal from being watched wrecks my coordination. I soon fall behind the others \& perform even worse.

This time, however, I played a new role. I was the lovable \& endearing (that part was important) absent-minded woman professor whose head is always in the clouds \& has entirely forgotten where she left her body. She has been set down in this hilarious situation to try to learn to belly dance, \& everyone enjoys the lesson more by watching her struggles.

The result was that I knew they were watching me, but it was all right. They laughed, but I heard it as loving. Any progress I made was given inordinate praise \& recognition. For me, it worked.

Next time you feel watched, try meeting the stares \& labeling yourself for them with something you can enjoy. ``We poets are never very good with adding'' or ``There's something about being a natural-born mechanic that makes it hard to draw pictures that don't look like the inside of a broken engine.''

Sometimes a situation is awkward by anyone's standards. So you turn red \& survive it. It is part of being human. It does not happen that often. Once I was standing in a line at a formal event, \& my 3-year-old son accidentally pulled my skirt off. Do you have one to top that? Sharing stories afterward is about all we can do.'' -- \cite[pp. 139--144]{Aron2013}

\subsection{Working With What You Have Learned: Reframing Your Shy Moments}
``Think of 3 occasions when you felt social discomfort. If possible, choose 3 rather different situations \& ones that you recall in some detail. Reframe them, one at a time, in terms of the 2 main points of this chapter: Shyness is not your trait -- it is a state anyone can feel. (2) The introverted social style is every bit as valuable as the extraverted one.
\begin{enumerate}
	\item \textit{Think about your response to the event \& how you have always viewed it}. Maybe you felt ``shy'' at recent party. It was a Friday night after a hard day at work. Dragged along by people at your office, you were hoping to meet someone who would become a real friend. But the others went off, \& you ended up in a corner, feeling conspicuous because you were not talking to anyone. So you left early \& spent the rest of the night assessing your whole personality, your entire life, \& feeling rotten.
	\item \textit{Consider your response in the light of what you now know about how your nervous system automatically operates}. Or imagine me explaining it to you: ``Hey, give yourself a break! The crowded, noisy room after a busy day, being left alone by your friends, your past experiences with these sorts of parties -- it was an avalanche waiting for a cuckoo clock. You like to be introverted. Sure, go to parties, but they should be small ones where you know people. Otherwise, pick out someone who appears as sensitive \& deeply interesting as yourself \& take off together as soon as possible. That's the HSP way to party. You are not shy or unlovable. You will definitely meet interesting people \& have close relationships -- you just have to pick \& choose your situations.''
	\item \textit{Is there anything you want to do about this now?} Perhaps there is a friend you could call \& arrange to spend some time with, your way.'' -- \cite[p. 144]{Aron2013}
\end{enumerate}

%------------------------------------------------------------------------------%

\section{Thriving at Work}

\subsection{Follow Your Bliss \& Let Your Light Shine Through}
``Of all the topics I cover in my seminars, vocations, making a living, \& getting along at work are the most urgent concerns of many HSPs, which makes sense in some ways, since we don't thrive on long hours, stress, \& overstimulating work environments. But much of our difficulty at work, I believe, is our not appreciating our role, style, \& potential contribution. This chapter therefore deals 1st with your place in society \& your vocation's place in your inner life. Impractical as these may sound, they actually have great practical significance. Once you understand your true vocation, your own intuition will begin to solve your specific vocational problems. (No book can do that as well for you because none can address your unique situation.)'' -- \cite[p. 147]{Aron2013}

\subsection{``Vocation'' Is Not ``Vacation'' Misspelled}
``A vocation, or calling, originally referred to being called to the religious life. Otherwise, in Western culture one did as is done in most cultures: what one's parents did. In the Middle Ages one was a noble, serf, artisan, \& so forth. Because in Christian Indo-European countries the ``priestly royal advisor'' class that I spoke of in Chap. 1 was officially celibate, no one was born into that class. It was the only job to which one had to be called.

With the Renaissance \& te rise of the middle class in the cities, people were freer to choose their work. But it is a very recent idea that there is 1 right job for each person. (It came at about the time of another idea, that there is 1 right person for each of us to marry.) At the same time, the number of possible vocations has increased greatly, as has the importance \& difficulty of matching the right person to the right job.'' -- \cite[p. 147]{Aron2013}

\subsection{The Vocation of All HSPs}
``As I said in Chap. 1, the more aggressive cultures in the world, all Western societies included, stem from an original social organization that divided people into 2 classes, the impulsive \& though warriors \& kings on the 1 hand \& the more thoughtful, learned priests, judges, \& royal advisors on the other. I also said that the balance of these 2 classes is important to the survival of such cultures \& that most HSPs naturally gravitate to the royal advisor class.

Speaking now of vocation, I don't mean that all HSPs become scholars, theologians, psychotherapists, consultants, or judges, although these are classic royal-advisor-class careers. Whatever our career, we are likely to pursue it less like a warrior, more like a priest or royal counselor -- thoughtfully, in all senses. Without HSPs in positions at the top in a society or organization, the warrior types tend to make impulsive decisions that lack intuition, use power \& force abusively, \& fail to take into account history \& future trends. That's no insult to them; it is just their nature. (This was the whole point of Merlin's role in the King Arthur legends; similar figures are in most Indo-European epics.)

1 practical implication of belonging to the advisory class is that an HSP can hardly ever have enough education \& experience. (I add experience because sometimes HSPs pursue education at the cost of experience.) The greater the variety of our experiences, \textit{within the range of what is reasonable for us} (hang gliding is not required), the wiser our counsel.

The education of HSPs is also important in order to validate our quieter, subtler style. I believe we need to stay well represented in our traditional professions -- teaching, medicine, law, the arts, science, counseling, religion -- which are increasingly becoming the domain of non-HSPs. That means that these social needs are being met in the warrior style, with expansion \& profit the only concern.

Our ``priestly'' influence has declined partly due to our having lost self-respect. At the same time, the professions themselves are losing respect without our quieter, more dignified contribution.

None of which is meant to imply some terrible plot by the less sensitive. As the world becomes more difficult \& stimulating, it is natural for the non-HSPs to thrive, at least at 1st. But they will not thrive long without us.'' -- \cite[pp. 148--149]{Aron2013}

\subsection{Vocation, Individuation, \& the HSP}
``Now, what about your particular vocation? Following the thinking of Carl Jung, I see each life as an \textit{individuation} process, 1 of discovering the particular question you were put on earth to answer. This question may have been left unfinished by an ancestor, although you must proceed with it in the manner of your own generation. But the question is not easy, or it would not take a lifetime. What matters is that working through it deeply satisfies the soul.

This individuation process is what the scholar of mythology Joseph Campbell referred to when he would exhort students struggling with their vocation to ``follow your bliss.'' He always made it clear that he didn't mean doing whatever is easy or fun at the moment; he meant engaging in work that feels right, that calls you. To have such work (\& if we are very fortunate, to be paid for it, too) is 1 of life's greatest blessings.

The individuation process requires enormous sensitivity \& intuition in order to know when you are working on the right question in the right way. As an HSP, you are built for this as a racing yacht is designed to catch the wind. I.e., the HSP's vocation in the larger sense is being careful to pursue well his or her vocation in the personal sense.'' -- \cite[p. 149]{Aron2013}

\subsection{Jobs \& Vocation}
``But then there is the problem of who will pay HSPs for pursuing our bliss. I usually agree with what Jung always insisted: It is a big mistake to financially support our type. If an HSP is not forced to be practical, he will lose all touch with the rest of the world. She will become an empty windbag no one listens to. But how can one make money \& still follow a calling?

1 way is to seek the point where the path directed by our greatest bliss crosses that directed by the world's greatest need -- i.e., what it is willing to pay for. At this intersection you will earn money for doing what you love.

Actually, the relation of a person's vocation to his or her paying job can be quite varied \& will change over a lifetime. Sometimes your job is just the way to make money; the vocation is pursued in your spare time. A fine example is Einstein's developing the theory of relativity while he was a clerk in a patent office, happy to have mindless work so he could be free to think about what mattered to him. At other times, we can find or create a job that fulfills our vocation, \& the pay will be at least adequate. There may be many possible jobs that do that, or the job that will serve the purpose will change as experience grows \& the vocation deepens.'' -- \cite[pp. 149--150]{Aron2013}

\subsection{Vocation \& the Liberated HSP}
``Individuation is, above all, about being able to hear your inner voice or voices through all the inner \& outer noise. Some of us get caught up in demands from others. These may be real responsibilities or may be the common ideas of what makes for success -- money, prestige, security. Then there are the pressure others can bring to bear on us because we are so unwilling to displease anyone.

Eventually, many, if not most, HSPs are probably forced into what I call ``liberation,'' even if it doesn't happen until the 2nd half of life. They tune in to the inner question \& the inner voices rather than the questions others are asking them to answer.
\begin{quotation}
	\textbf{Reframing the critical points in your vocational \& work history}
	
	Now might be a good time for you to pause \& do some reframing, as you did in previous chapters. Make a list of your major vocational steps or job changes. Write down how you've always understood those events. Perhaps your parents wanted you to be a doctor but you knew it was not for you. Having no better explanation, maybe you accepted the idea that you were ``too soft'' or ``lacked motivation.'' Now write down what you understand in light of your trait. In this case, that most HSPs are utterly unsuited for the inhuman grind required, unfortunately, by most medical schools.
	
	Does your new understanding suggest something you need to do? In the example, this new understanding of medical school might need to be discussed with one's parents if they still insist on their negative views. Or it might mean finding a medical school that is more humane or studying a related subject, such as physiology or acupuncture, that allows for a different style of professional education.	
\end{quotation}
Being so eager to please, we're not easy to liberate. We're too aware of what others need. Yet our intuition also picks up on the inner question that must be answered. These 2 strong, conflicting currents may buffet us for years. Don't worry if your progress toward liberation is slow, for it's almost inevitable.

I don't, however, want to develop some idealized image of a certain kid of HSP you must become. That is precisely \textit{not} liberation. It is finding who you are, not what you think someone else wants you to become.'' -- \cite[pp. 150--151]{Aron2013}

\subsection{Knowing Your Own Vocation}
``Some of you may be struggling with discovering your vocation \& feeling a little frustrated that your intuition is not helping you more. Alas, intuition can also stand in your way because it makes you aware of too many inner voices speaking for too many different possibilities. Yes, it would be desirable just to serve others, thinking little of my material gain. But that rules out a lifestyle with time to pursue the finer things in life. \& both exclude the actualizing of my artistic gifts. \& I have always admired the quiet life, centered in family. Or should it be centered in the spiritual? But that is so up in the air when I admire a life close to the earth. Perhaps I would be happiest working for ecological causes. But then, the needs of humans are so great.

All the voices are strong. Which one is right? If you're flooded with such voices, you will probably have trouble with decisions of all sorts; very intuitive people usually do. But you'll need to develop your decision-making skills for whatever vocation you choose. So start now paring down the choices to 2 or 3. Maybe make a rational list of the pros \& cons. Or pretend you have made up your mind definitely 1 way \& live with that for a day or 2.

Another problem for HSPs who are very intuition \&{\tt/}or introverted is that we may not be well informed about the \textit{facts}. We let our hunches guide us. We don't like to \textit{ask}. But gathering concrete information from real people is part of the individuation process of introverted or intuition people especially.

If you feel you ``just cannot,'' you are revealing the 3rd obstacle to knowing your vocation: low self-confidence. Deep inside you probably know what you really want to do. Of course you may have selected something you cannot possibly succeed at in order to avoid moving along \& doing the possible. But maybe you're still confused about what you can \& cannot do.

As an HSP you may have great difficulty with certain tasks which, by your culture's standards, are crucial to success in most vocations -- perhaps public speaking or performance, perhaps tolerating noise, meetings, networking, office politics, travel. But now you know the specific cause of your difficulty with these \& can explore ways around the overarousal they create. So there's really very little that you can't do if you find a way to do it in your own style.

Low self-confidence is very understandable in HSPs however. Many of you have felt flawed. You may have tried so hard to please others that you've been little more than a bridge on other peoples' paths \& have been treated just like that, like something underfoot. However, how will you feel going to your grave without having tried?

You say you're afraid of failing. Which inner voice says that? A wise one that protects you? Or a critical one that paralyzes you? For the sake of getting going, assume the voice is right \& you're fail. Forget about the people who tried \& succeeded, the theme of so many movies. I know people who have tried \& failed. Many of them. They may be out megabucks \& megatime, but they're still happier for having tried. Now they're moving on to other goals, wiser for what they learned about themselves \& the world. \& really, since no effort amounts to a total failure, they're much more confident about themselves than when they were sitting on the sidelines.

Finally, in finding your vocation, do use the excellent books \& services on vocational choice. Just keep your sensitivity always in your awareness as an important factor which most vocational counselors don't address.'' -- \cite[pp. 151--153]{Aron2013}

\subsection{What Other HSPs Are Doing}
``Perhaps it would help to hear about the kinds of careers other HSPs have chosen. Of course, we bring our own flair to everything. In my telephone survey, I found, e.g., that not many HSPs were salespersons, but one was -- of 5 wines. Another sold real estate, saying that she used her intuition to match people with homes.

One can imagine other HSPs shaping other jobs -- almost any job -- into something quiet, thoughtful, \& conscientious, as when HSPs said they were teachers, hair stylists, mortgage brokers, pilots, flight attendants, professors, actors, early childhood educators, secretaries, doctors, nurses, insurance agents, professional athletes, cooks, \& consultants.

Other jobs seemed obviously suited to HSPs: cabinetmaker, pet groomer, psychotherapist, minister, heavy-equipment operator (noisy but no people), farmer, writer, artist (lots of these), X-ray technician, meteorologist, tree trimmer, scientist, medical transcriber, editor, scholar in the humanities, accountant, \& electrician.

While some research has found that so-called shy people make less money, I certainly found plenty of HSPs in positions that sounded well paid -- administrators, managers, bankers. Maybe other studies found that their so-called shy respondents were poorly paid because of a quirk in their data similar to mine: Twice as many HSPs as non-HSPs in my study called themselves homemaker, housewife, or full-time parent. (Not all were women.) If you counted them as not earning money, this would certainly lower their income average as a group. But, of course, these people add income to their family by performing services which, if paid for, would be expensive.

HSP ``homemakers'' find a good niche for themselves, provided they can ignore the culture's undervaluing of their work. In fact, the culture benefits greatly. Research on parenting, e.g., continually finds the elusive quality of ``sensitivity'' to be the key in raising children well.'' -- \cite[pp. 153--154]{Aron2013}

\subsection{Turning Vocation Into a Job That Pays}
``There are good books written just on the topic of turning what you love into what pays you a salary, so as usual I will focus on the aspects especially relevant to us. To make a paying job out of your true vocation often requires creating an entirely new service or profession, \& that may mean starting your own business or creating a new job where you already work. That can seem daunting unless you remember to do it in the style of an HSP.

1st, throw out the image of everyone getting their work done through networking, knowing the right people, \& the like. Some networking is always needed, but there are ways that are effective enough \& far more pleasant for an HSP -- letters, e-mail, staying in touch with 1 person who stays in touch with many, taking out to lunch \& ``debriefing'' your extraverted colleague who goes to every conference.

2nd, you need to trust some of your advantages. With your intuition, you can study trends \& perceive needs or markets before others. If you are excited about something, there is a good chance others are, or could be, once they heard your reasons. If your interest is not too unusual, it should fit into existing jobs. If it is very unusual, you are probably the leading expert, \& someone somewhere will need you soon, especially once you share your vision.

Years ago an HSP with a passion for film \& video took a job as a librarian \& convinced her university that they should have a state-of-the-art film \& video department. She saw that these media would be the cutting edge in education, especially continuing education of the public. Everyone sees that now, \& her film \& video library is the finest in the country.

Self-employment (or being granted full autonomy within a larger organization) is a logical route for HSPs. You control the hours, the stimulation, the kinds of people you will deal with, \& there are no hassles with supervisors or coworkers. \&, unlike many small or 1st-time entrepreneurs, you will probably be conscientious about research \& planning before you take any risks.

You will have to watch, however, for certain tendencies. If you are a typical HSP, you can be a worry-prone perfectionist. You may be the most hard-driving manager you have ever worked for. You also may have to overcome a certain lack of focus. If your creativity \& intuition give you a million ideas, at some point, early, you will have to let most of them go, \& you will have to make all kinds of difficult decisions.

If you are an introvert also, you will have to make an extra effort to stay in touch with your public or market. You can always bring in an extravert as a partner or assistant. Indeed, having partners or hiring others to absorb all sorts of excess stimulation is a good idea. But with them as a buffer between you \& the world, your intuition will not receive direct input unless you plan for some real contact with those you serve.'' -- \cite[pp. 154--155]{Aron2013}

\subsection{Art as Vocation}
``Almost all HSPs have an artistic side they enjoy expressing. Or they deeply appreciate some form of art. But some of you will pursue the arts as your vocation or even your livelihood. Almost all studies of the personalities of prominent artists insist that sensitivity is central. Unfortunately, that sensitivity is also linked with mental illness.

The difficulty, I believe, is that normally we artists work alone, refining our craft \& our subtle creative vision. But withdrawal of any kind increases sensitivity -- that is part of why one withdraws. So we are extrasensitive when the time comes to show our work, perform it, explain it, sell it, read reviews of it, \& accept rejection or acclaim. Then there's the sense of loss \& confusion when a major work is done or a performance is over. The stream of ideas surging up from the unconscious no longer has an outlet. Artists are more skilled at encouraging \& expressing that force than understanding its sources or its impact if acted upon.

It is not surprising that artists turn to drugs, alcohol, \& medications to control their arousal or to recontact their inner self. But the long-term effect is a body further off balance. Moreover, it is part of the myth or archetype of the artist that any psychological help will destroy creativity by making the artist too normal.

But a highly sensitive artist in particular had better think deeply about the mythology surrounding the artist. The troubled, intense artist is 1 of the most romantic figures in our culture, now that saints, outlaws, \& explorers are on the wane. I recall a creative-writing teacher once listing nearly every famous author on the blackboard \& asking us what they had in common. The answer was attempted suicide. I'm not sure the class saw it as a tragedy so much as a romantic aspect of their chosen career. But as a psychologist as well as an artist, I saw a deadly serious situation. How often the value of artists' works has increased once they were declared insane or had committed suicide. While the life of the artistic hero-adventurer especially calls to the young HSP, it can also be a trap quite unconsciously laid by those with mundane lives who allow no time for the artist within \& want someone else to be the artist for them, displaying all the craziness they repress in themselves. Much of the suffering of sensitive artists could be prevented by understanding the impact of this alternating of the low stimulation of creative isolation with the increased stimulation of public exposure which I have described. But I am not sure that this understanding will be widely applied until the myth of the unstable artist \& the need for it have also been understood.'' -- \cite[pp. 155--156]{Aron2013}

\subsection{Service to Others as a Vocation}
``HSPs tend to be enormously aware of the suffering of others. Often their intuition gives them a clearer picture of what needs to be done. Thus, many HSPs choose vocations of service. \& many ``burn out.''

But to be helpful to others you do not have to work at a job that burns you out. Many HSPs insist on working on the front lines, so to speak, receiving the most stimulation. They could feel guilty staying behind, sending others out to do what to them seems so onerous. But by now I think you can see that some people are, in fact, perfectly suited to the front lines \& love it there. So why not let them satisfy their urges? People are also needed behind the front lines, developing strategy from a lookout above the battlefield.

To put it another way, some people like to cook, \& some like to wash dishes. For years I could not let others take on the chore of cleaning up after I had enjoyed cooking, 1 of my favorite pastimes. Then I finally, truly, heard someone when he insisted he \textit{really liked} to clean up -- \& detested cooking.

1 summer I toured Greenpeace's \textit{Rainbow Warrior} \& listened to some of the crew's adventures, such as their having been dropped in front of the bow of huge whaling-factory ships or having been in the sights of torpedoes \& machine guns for days at a time. For all my love of whales, I would be more trouble than help in those sorts of circumstances. But I knew I could support the effort in other ways.

In short, you do not have to take the job that will create excessive stress \& overarousal. Someone else will take it \& flourish in it. You do not have to work long hours. Indeed, it may be your duty to work shorter ones. It may be best not to advertise it, but keeping yourself healthy \& in your right range of arousal is the 1st condition for helping others.'' -- \cite[p. 157]{Aron2013}

\subsection{A Lesson From Greg}
``Greg was a highly sensitive schoolteacher who was much loved \& respected by his students \& colleagues. Yet he came to me to discuss why he was quitting the only profession he had ever wanted, expecting me to verify that teaching was no profession for HSPs. I agreed it was a tough one. But I also think good, sensitive teachers are essential to every sort of happiness \& progress, for individuals \& society. I could not bear to see such a gem leave the field.

Thinking about it with me, he agreed that teaching was a very logical vocation for a sensitive, caring person. Teaching jobs ought to be designed for them, but in fact the pressures are making it hard for HSPs to stay in teaching. His task, he realized, was to change the job description. Indeed, it was his ethical duty. He would do far more good by refusing to overwork himself than by quitting.

Beginning the very next day, Greg never worked after 4 in the afternoon. It required applying much of his creativity just to find the right shortcuts. Many were not ideal \& truly distressed his conscientious soul. He felt he had to hide his new work habits from his colleagues \& principal, although in time they caught on. (The principal approved, seeing that Greg was doing the essential tasks well \& feeling happier.) Some of his colleagues copied him; some envied \& resented him but could not change their ways. 10 years later, Greg is still a highly successful teacher, \& a happy, healthy one.

It is true that even when exhausted you still are providing something to those you serve. But you are out of touch with your deepest strengths, role-modeling self-destructive behavior, martyring yourself, \& giving others cause for guilt. \& in the end you will want to quit, like Greg, or be forced to by your body.'' -- \cite[pp. 157--158]{Aron2013}

\subsection{HSPs \& Social Responsibility}
``None of the above is meant to take more HSPs out of the battle for social justice \& environmental sanity. On the contrary, we need to be out there, but in our way. Perhaps some of what goes wrong in government \& politics is not so much a product of the Left or Right but the lack of enough HSPs making everyone pause to check the consequences. We have abdicated, leaving things to the more impulsive, aggressive sorts, who do happen to thrive on running for political office \& then on running everything else.

The Romans had a great general named Cincinnatus. The legend is that he had wanted to live quietly on his farm but was persuaded twice to return to public life to save his people from military disasters. The world needs to coax more such folks into public positions. But if they don't coax us, we had better volunteer now \& then.'' -- \cite[p. 159]{Aron2013}

\subsection{HSPs in the World of Business}
``The business world is undoubtedly undervaluing its HSPs. People who are gifted \& intuitive yet conscientious \& determined not to make mistakes ought to be treasured employees. But we are less likely to fit into the business world when the metaphors for achievement are warfare, pioneering, \& expansion.

Business can also be seen as a work of art requiring an artist, a task of prophecy requiring a visionary, a social responsibility requiring a judge, a job of growing requiring skills like those of a farmer or parent, a challenge of educating the public requiring skills like those of a teacher, \& the like.

Companies do vary. Be alert to corporate culture when you take a position or have the chance to influence the corporate culture you are in. Listen to what is said but also use your intuition. Who is admired, rewarded, \& promoted? Those who foster toughness, competitiveness, \& insensitivity? Creativity \& vision? Harmony \& morale? Service to the customer? Quality control? HSPs should feel at home in varying degrees in all but the 1st.'' -- \cite[p. 159]{Aron2013}

\subsection{The Gifted HSP in the Workplace}
``In my opinion, all HSPs are gifted because of their trait itself. But some are unusually so. Indeed, 1 reason for the idea of ``liberated'' HSPs was the seemingly odd mixture of traits emerging from study after study of gifted adults: impulsivity, curiosity, the strong need for independence, a high energy level, along with introversion, intuitiveness, emotional sensitivity, \& nonconformity.

Giftedness in the workplace, however, is tricky to handle. 1st, your originality can become a particular problem when you must offer your ideas in a group situation. Many organizations stress group problem solving just because it brings out the ideas in people like you, which are then tempered by others. The difficulty arises when everyone proposes ideas \& yours seem so obviously better to you. Yet the others just do not seem to get it. When you go along with the group, you feel untrue to yourself \& are unable to commit to the group's results. When you do not, you feel alienated \& misunderstood. A good manager or supervisor knows these dynamics \& will protect a gifted employee. Otherwise, you may want to offer your giftedness elsewhere.

2nd, you can be intensely excited about your work \& ideas. In your excitement you may seem to others to take big risks. To you the risks are not great because the outcome is clear. But you're not infallible, \& others may take particular pleasure in your failures, even if they're rare. Furthermore, those not understanding this intensity will say you work all the time \& probably resent it -- you make them look bad. But for you, work is play. \textit{Not} to work would be work. If this is you, you may have to keep your long hours a secret, known only to your supervisor.

Or better, skip the long hours. Try treating even the most positive excitement as a state of overarousal \& strive to balance work with recreation. Your work will benefit.

Another result of your intensity is that your restless mind may drive you onward to other projects before you have completed the details of the last, \& others may harvest what you planted. Unless you plan around this, which is not usually your style, the result will have to be accepted.

A 3rd aspect of giftedness, emotional sensitivity, can draw you into others' complicated private lives. In the workplace especially this is not a good idea. You want to have some professional boundaries. Especially at work, you need to spend more time with the less sensitive, who can be a great balance to you, \& you to them. Develop outside of work the more intense sorts of relationships that offer you the emotional depth you seek.

Also outside of work should be the relationships that offer the safe harbor from the emotional storms created by your sensitivity. Don't look for that among your colleagues, \& especially not from your supervisors. You're just too much for them to handle, \& they may decide there's ``something wrong with you.''

A 4th trait of the gifted, intuition, can seem almost magical to others. They don't see what you see -- this contrast between the surface \& ``what's really going on.'' So as with your unusual ideas, you must decide whether to be honest or go along with things as others see them \& feel secretly a bit alienated.

Finally, your giftedness may give you a certain charisma. Others may hope you will guide them rather than their having to guide themselves. It is a flattering temptation, but you can only end up seeming to have stolen their freedom, which in a sense you would have.

From your side, you may find that others seem to have little to offer in return. Initial sharing may be followed by a sense of disappointment. But giving up on others leads to more alienation, whereas in fact you need others.

1 solution to all of this is not to insist that your gifts all be expressed at work. Express yourself through private projects \& arts, schemes for future or parallel self-employment, \& through life itself.

In other words, expand your use of your giftedness beyond producing the most noticed ideas at work. Use it to attain greater self-insight \& to gain wisdom about human beings in groups \& organizations. When that is your goal, sitting back \& observing are okay. So is participating as an ordinary person sometimes, not a gifted one, \& seeing how that feels.

Finally, stay in good contact with many kinds of other people, at work \& elsewhere, accepting that no 1 person can relate to all of you. Indeed, accepting the loneliness that goes with giftedness may be the most freeing, empowering step of all. But also accept its opposite, that there's no need to feel isolated, for everyone is gifted in some way. \& then there's the opposite truth: No one, including yourself, is special in the sense of being exempted from the universals of aging \& death.'' -- \cite[pp. 159--162]{Aron2013}

\subsection{Seeing That Your Trait Is Properly Valued}
``I hope by now that you can imagine the many ways in which being an HSP can be an asset in your work, whether you are self-employed or working for another. But I've found that it takes considerable work before HSPs can undo past negative ideas about their trait \& truly value it. You cannot possibly convince anyone else of its value if you're not convinced yourself. So please do the following, without fail.

List every asset that might possibly belong to an HSP. Follow the rules of brainstorming \& accept all ideas without criticism. Don't worry if non-HSPs have some of the same assets. It's enough if we have them more or also. \& use every strategy: logical deduction from the basic trait; thinking about your growing image of the typical HSP; considering the HSPs you know \& admire; thinking about yourself; looking through this book. Your list should be \textit{long}. It's very long when HSPs do it as a group -- with my pushing them. So keep at it until yours is substantial.

Now do 2 things: Write a little speech you might use during an interview \& also a more formal letter, \& in both of them express some of your assets, embedding your trait of sensitivity among them in a way that quietly educates your employer.

Here is part of a possible script (which would be a little informal for a letter):
\begin{quotation}
	\& besides my 10 years of experience with young children, I have considerable knowledge of graphic arts \& practical experience with layout. In all of this, I am aware of the unique contribution of my personality \& temperament -- I am 1 of those persons who is extremely conscientious, thorough, \& concerned about doing a good job.
	
	At the same time, I think I have a pretty amazing imagination. I've always been seen as highly creative (along with earning excellent grades in school \& having a high IQ). My intuition about my work has always been 1 of my greatest strengths, including being able to spot possible potential trouble or errors.
	
	Yet I am not one to cause a fuss. I like to keep things calm around me. Indeed, you should know I work best when I'm feeling calm, when things around me are quiet. So most people find me especially comfortable to work with, although I myself am just as happy working alone as with a few others. My independence in that regard, my ability to work well alone \& on my own, has always been another 1 of my strength $\ldots$'' -- \cite[pp. 162--163]{Aron2013}
\end{quotation}

\subsection{Training}
``Training situations can be very overarousing because you tend to perform worse when being observed or when overaroused in any other way -- e.g., being given too much information at once, having too many people around talking or straining to learn, imagining all the dire consequences of failing to remember something.

If possible, try to train yourself. Take home the instruction manuals or stay after hours \& work on your own. Or arrange to be trained 1-on-1, preferably by someone who puts you at ease. Ask to be shown a step, then to be left to practice it alone. Next, allow someone else to watch you who is not a supervisor, someone who doesn't make you so nervous.'' -- \cite[p. 163]{Aron2013}

\subsection{Being Physically Comfortable on the Job}
``Because you're more sensitive, you don't need extra discomfort or stress around you. A situation may have been deemed safe but still be stressful for you. Likewise, others may have no problem with fluorescent lights, low levels of machine noise, or chemical odors, but you do. This is a very individual matter, even among HSPs.

If you do have to complain, think realistically about what you're up against. If you still want to go ahead, mention the efforts you've made on your own to solve the situation. Emphasize your productivity \& accomplishments but that you can do still better when this problem is resolved (if that's realistic).'' -- \cite[p. 163]{Aron2013}

\subsection{Advancing in an Organization}
``Research on ``shy'' people claims that they tend to be paid too little \& to work below their competence level. I suspect this is true of many HSPs, although it's sometimes our choice. But if you want to be advancing \& are not or if layoffs are being considered \& you don't want to be among them, you have to pay attention to strategy.

Often HSPs don't like to ``play politics.'' But that in itself can make us subject to suspicion. We're so easily misperceived in all sorts of ways, especially if we spend less time with others in our workplace or do not share our thoughts with them. We can seem aloof, arrogant, odd. If we're also not pushy, we can seem uninterested or weak. Often these are utterly unwarranted projections. But you have to be watchful for these dynamics \& plan to defuse the projections.

When it is appropriate, casually (or formally) let others know your good feelings about them \& the organization. You may think your positive feelings are obvious, but they may not be if you are low key \& the others are not very aware. Consider whether you also need to talk more openly about what you think you contribute, where you would like to see yourself eventually in the organization, \& how long you are willing to wait to see it happen.

Meanwhile, be sure that you will not be taken for granted when the next promotions are handed out by writing down once a week all your latest contributions to your organization, plus any achievements elsewhere in your profession or your life. Be very detailed. At least you'll be aware of them \& more likely to mention them, but if possible, show a summary of these achievements to your supervisor at your next review.

If you resist doing this task or a month from now find you still haven't done it, think deeply about why. Does it feel like bragging? Then consider the possibility that you do your organization \& supervisor a major disservice by not reminding them of your value. Sooner or later you'll feel dissatisfied \& want to move on, or you'll be lured away by the competition, or you'll be laid off while someone less competent is kept. Do you wish others would notice your value without your having to remind them? That is a common desire stemming from childhood that is seldom fulfilled in this world.

Or, are you in fact accomplishing very little? Do you care? Maybe you need to keep a record of the accomplishments that do matter to you -- trails biked, books read, conversations had with friends. If something besides work takes most of your energy, it may be what you most enjoy. Is there any way to be paid for doing that? \& if a responsibility such as children or an aging parent is taking up your time, feel pride in meeting that responsibility. List this as an accomplishment, too, even though it cannot be shared with most employers.

Finally, if you are not advancing or feel ``someone is out to get you,'' it's quite possible that you're just not savvy enough.'' -- \cite[pp. 164--165]{Aron2013}

\subsection{Bette Meets Machiavelli}
``Bette was an HSP who saw me in psychotherapy. 1 of the issues she often brought to me was her frustration at work. Therapists can never know for certain what's going on in situations we only hear 1 side of. But it sounded as if Bette was doing a fine job at work but was never promoted.

Then, at 1 review, she was criticized for the very sorts of behaviors that seemed to us would be valued by most supervisors. Very reluctantly, Bette began to wonder if her supervisor was ``out to get her.'' The supervisor had a distressed personal life, \& Bette had been warned by the last supervisor she might ``stab her in the back.''

Most of the other employees got along all right with the new supervisor, but Bette's intuition told her they were going out of their way to placate their boss because they feared her. Being much older, Bette had just seen her as immature but not a threat. But Bette was also dedicated \& conscientious. She often received praise from visitors implying that she was the most competent of those they had met in her department. She thought she had nothing to fear, but she had overlooked her supervisor's envy. But then Bette didn't like to think anything negative of anyone.

Eventually, Bette took the step of asking someone in personnel to let her see her life (an appropriate move in this organization) \& found that her supervisor had been putting in notes about her that were simply untrue, while positive information Bette had asked to be included was missing.

Bette finally had to admit she was in a power struggle with her supervisor, but she didn't know what to do. In particular, she said over \& over that she didn't want to stoop to thinking like this nemesis.

The important issue to me became helping Bette see why she had been targeted. Indeed, she admitted it wasn't the 1st time in her work history. I suspected that in this case it was because, untrue as it was, she seemed aloof, superior, \& therefore threatening to an insecure younger person. But underlying that was Bette's failure, even refusal, to see the conflict coming.

Here \& in other work settings in the past, Bette made herself an easy target by preferring to be ``separate from the herd.'' Like the many HSPs who are introverted, she preferred to go to work, do her job well, \& go home without adding to her stimulation by socializing. She often told me, ``I don't enjoy gossiping like the others.'' 1 effect of this style was that it kept her too uninformed abut what was going on at an informal level. She needed to haul out her persona \& do some chatting simply to protect herself, to know what was up, to ``have some friends at court.'' A 2nd effect was that, in a sense, she was rejecting the others, or so they felt. At any rate, they felt no urge to come to her aid, \& thus the supervisor had known she would be safe to act against Bette.

Another understandable mistake that Bette made, so typical of an HSP, was to be totally unaware of her supervisor's ``shadow'' or less desirable aspects. In fact, Bette tended to idealize supervisors. She expected only kindness \& protection from someone in charge. When she didn't receive it, in this case she went to the person over her supervisor for help. But she thought it ``only right'' to let her supervisor know what she was doing! The supervisor, of course, beat her to it, turning the higher-up against Bette. Another overidealized authority had behaved, predictably, like a mortal.

When I asked Bette to be more savvy, more ``political,'' at 1st she felt I was asking her to dirty herself. But I knew such purity had to be casting a long shadow, \& eventually she encountered in her dreams an angry, fenced-in goat, then a tough little ``street fighter,'' \& finally a rather sophisticated businesswoman. In getting to know these dream figures, each added something to Bette that she had in fact possessed but hadn't used \& was vehemently repressing as unacceptable. They taught her how to be at least a little suspicious of everyone, especially those she was idealizing (including me).

As she advanced in her self-reflection -- much of it obviously requiring considerable courage \& intelligence -- Bette admitted that she had deep doubts about the motivations of everyone. But she was always trying to suppress these suspicious as 1 more unsavory aspect of herself. By becoming aware of them \& checking them out, she found she could trust some people more, not less, \& her own less conflicted intuitions most of all. You will have a chance to meet your own inner power broker at the end of this chapter.'' -- \cite[pp. 165--167]{Aron2013}

\subsection{Regrets -- Evitable \& Inevitable}
``It is hard to face all the things we're not going to get to do in this lifetime. But that's part of being mortal. How wonderful if we can make even a little progress on the question life has asked us. It is even more wonderful if we find a way to be paid while doing this. \& nearly a miracle if we're able to work on that in the company of others, in harmony \& mutual appreciation. If these are your blessings, do appreciate them. If you have not achieved them yet, I hope you now have a sense of how you might.

On the other hand, you may be having to come to terms with a vocation that was often blocked by other responsibilities or by your culture's failure to appreciate you. If you can reach a place of peace about this, then you may well be the wisest of us all.'' -- \cite[pp. 167--168]{Aron2013}

\subsection{Working With What You Have Learned: Meeting Your Machiavelli}
``Machiavelli, a Renaissance advisor to Italian princes, wrote with brutal frankness about how to get ahead \& stay ahead. His name is associated, perhaps too much, with manipulating, lying, betraying, \& all the rest of the conniving that goes on ``at court.'' I do not recommend that you become Machiavelli, but I do assert that the more his qualities repulse you, the more you need to be aware of their lurking in yourself \& others. The more you claim to know nothing about such things, the more you will be troubled by secret conniving in yourself or others.

In short, somewhere inside you there is a Machiavelli. Yes, he is a ruthless manipulator; but no prince, especially a kind one, would stay in power long without at least 1 advisor with as remorseless a point of view as that of the enemies a prince will surely have. The trick is to listen well but keep Machiavelli in his place.

Maybe you already know this part of yourself. But give that aspect flesh. Try to imagine how he or she looks, what he or she says, his or her name. (It probably will not be Machiavelli.) \& then have a chat. Let him or her tell you all about the organization where you work. Ask who's doing what to get ahead \& who's out to get you. Ask what you could do to get ahead. Let that voice speak for a while.

Later, being very careful to keep your values \& good character intact, think about what you learned. E.g., were you told that someone is using unfair tactics \& hurting you \& the organization in the process? Is this inner voice being paranoid, or is it something you have known but didn't want to admit? Are there any wise moves you can make to counter these or at least protect yourself?'' -- \cite[p. 168]{Aron2013}

%------------------------------------------------------------------------------%

\section{Close Relationships}

\subsection{The Challenge of Sensitive Love}
``This chapter is a love story. It begins with how HSPs fall in love \& into loving friendship. Then it helps with the rewarding work of keeping that love alive, HSP style.''  -- \cite[p. 169]{Aron2013}

\subsection{HSP Intimacy -- So Many Ways We Do It}
``Cora is 64, a homemaker \& author of children's books. She has been married once, to her ``only sexual partner,'' \& informed me firmly that she is ``very content with this aspect of my life.'' Dick, her husband, is ``anything but an HSP.'' But each enjoys what the other brings to the marriage, especially now that the rough spots are worked out. E.g., over the years she learned to resist his wanting her to share his pleasure in adventure movies, downhill skiing, \& attending Superbowls. He goes with friends.

Mark, in his 50s, is a professor \& poet, an expert on T.S. Eliot. He is unmarried \& lives in Sweden, where he teaches English literature. Friendships are central to Mark's life. He has become skilled at finding those few souls in the world like himself \& cultivating deep relationships with them. I suspect they consider themselves very fortunate.

As for romance, Mark remembers intense crushes even as a child. As an adult, his relationships have been ``rare but overwhelming. 2 are always there. Painful. There is no end, although the door is closed.'' But then I recall his tone becoming wry. ``But I have a rich fantasy life.''

Ann also recalls being intensely in love as a child. ``There was always someone; it was a quest, a search.'' She married at 20 \& had 3 children in 7 years. There was never enough money, \& as the tensions mounted, so did her husband's abusiveness. After he hit her hard a few times, she knew she had to leave, had to grow up \& somehow support herself.

Over the years there were other men in Ann's life, but she never married again. At 50, she says her quest for the ``magical other'' has ended at last. Indeed, when I asked her if there were special ways she had organized her life to accommodate her sensitivity, her 1st response was ``I finally got men out of my life, so I'm not tried by \textit{that} anymore.'' Close friendships with women, however, \& close ties to her children \& sisters give Ann great happiness.

Kristen, the student wet met in Chap. 1, was yet another with intense crushes throughout her childhood. ``Each year I would pick out one. But as I got older \& it became more serious, especially when I was actually with them, I wanted them to leave me alone. Then there was the one I went to Japan for. He was so important to me, but that's ending, thank goodness. Now that I'm 20, I'm not so into boys. I want to figure out who I am 1st.'' Kristen, so worried about her sanity, is definitely sounding very sane.

Lily, 30, spent a promiscuous youth in rebellion against her strict Chinese mother. But 2 years before, when Lily's health failed due to her wild life, she finally realized she was miserable. During our interview she even began to wonder whether she had chosen this overstimulating life in order to distance herself from a family she had seen as boring \& lacking American vigor. At any rate, when she regained her health, she entered a relationship with a man she saw as even more sensitive than herself. At 1st, they were merely friends; like her family, he seemed boring. But something gentle \& thoughtful grew between them. They moved in with each other, but she didn't rush into marriage.

Lynn is in her 20s \& recently married Craig, with whom she shares a common spiritual path \& deep, new love. But 1 issue between them is how much sex they want. In keeping with the spiritual tradition he was part of, which she embraced upon meeting him, Craig was abstaining from sexuality. At the time of our interview, he had changed his mind, \& she was the one who wanted to follow this tradition \& abstain. The compromise that pleases them both thus far has been lovemaking that is ``infrequent'' (once or twice a month) but ``very special.''

These examples illustrate the richly diverse manner in which HSPs fulfill their very human desire to be close to others. Although I have no large-scale statistical data yet to confirm this, it is my impression from my interviews that HSPs vary more than others in the kinds of arrangements they work out in this area, choosing being single more often than the general population, or firmer monogamy, or close relationships with friends or family members rather than romance. True, this marching to a different love song may be due to HSPs' different personal histories \& needs. But then necessary is the mother of invention.

With all this diversity, we HSPs still have some common issues to consider regarding our close relationships, all arising from our special ability to perceive the subtle \& out greater tendency to become overaroused.'' -- \cite[pp. 169--171]{Aron2013}

\subsection{HSPs \& Falling in Love}
``In respect to falling in love, my research suggests that we HSPs do fall in love harder than others. That can be good. E.g., research shows that falling in love tends to increase anyone's sense of competence \& the sheer breadth of the person's self-concept. When in love, one feels bigger, better. On the other hand, it is good to know some of the reasons we fall in love harder that have little or nothing to do with the other -- just in case there are times when we would rather not.

Before we begin, however, write down what happened to you on 1 or more occasions when you feel deeply in love. Then you can watch to see if any of what I described was operating in your case.

I realize that some HSPs never seem to fall in love. (They usually have the avoidant-attachment style I have described earlier.) But saying I will never love is like saying it will never rain in the desert. Anyone who knows the desert will tell you that when it does rain, watch out. So if you think you never fall in love hard, you might just read on, anyway -- in case of rain.'' -- \cite[pp. 171--172]{Aron2013}

\subsection{When It Is Too Intense}
``Before turning to the kind of powerful falling in love or friendship that can lead to a wonderful relationship, you might be interested in the rarer but more notorious case of overwhelming, impossible love. It can happen to anyone, but it seems to happen a little more often to HSPs. \& since it is often a miserable experience for both parties, some information may be helpful should you stumble into such a situation.

This sort of love is usually unrequired. The failure to be loved back can be the very cause of the intensity. If a real relationship could develop, the absurd idealization would cool as one came to know the beloved better, warts \& all. But the intensity can also stop the relationship. Extremely intense love is often rejected by the beloved just because it is so demanding \& unrealistic. The one being loved often feels smothered \& not really loved at all in the sense that his or her feelings are being considered. Indeed, it can seem as if the lover may abandon everything for the dream of perfect happiness which the other alone can fulfill.

How does such love happen? There is no 1 answer, but some strong possibilities. Carl Jung held that the habitually introverted (most HSPs) turn their energy inward to protect their treasured inner life from being overwhelmed by the outer world. But Jung pointed out that the more successfully introverted you are, the more pressure builds in the unconscious to compensate for the inward turning. It is as if the house becomes filled with bored (but probably gifted) kids who eventually find their way out the back door. This pent-up energy often lands on 1 person (or place or thing), which becomes all-important to the poor upended introvert. You have fallen intensely in love, \& it really has less to do with the other person \& more to do with how long you have delayed reaching out.

Many movies \& novels have captured this kind of love. The classic film example might be \textit{The Blue Angel}, about a professor falling in love with a dance-hall girl. The book classic might be Hermann Hesse's \textit{Steppenwolf}, about a highly introverted older man meeting a provocative young dancer \& her passionate, sensuous crowd. In both cases, the protagonists are hopelessly drawn into a world of love, sex, drugs, jealousy, \& violence -- all the stimulation \& stuff of the senses which their intuitive, introverted self had once rejected \& knew nothing about handling. But women experience it, too, as in some of the novels of Jane Austen or Charlotte Bronte, in which controlled, introverted, bookish women are swept away by love.

No matter how introverted, you are a social being. You cannot escape your need \& spontaneous desire to connect with others even if your conflicting urge to protect yourself is very strong. Fortunately, once you have been out there a bit \& in love a few times, you will realize that no one is that perfect. As they say, there are always other fish in the sea. The best protection against falling in love too intensely is being more in the world, not less. Once you reach a balance, you may even find that certain people actually help you stay calm \& secure. So since you are going to be soaked someday, anyway, you might as well dive in with the rest of us now.

Look back at your own story of falling in love or friendship. Did it follow a long period of isolation?'' -- \cite[pp. 172--173]{Aron2013}

\subsection{Human \& Divine Love}
``Another way to fall in love hard is to project one's spiritual yearnings onto another person. Again, mistaking your human beloved for a divine beloved would be corrected if you could live with that person for a while. But when we cannot, the projection can be surprisingly persistent.

The source of such love has to be something pretty big, \& I think it is. As Jungians would put it, we each possess an inner helpmate who is meant to lead us to the deepest inner realms. But we may not know that inner helpmate very well, or more often, we mistakenly project him or her onto others in our desperate desire to find that one we need so much. We want that helpmate to be real, \& of course, while things can be very real that are entirely inner, that is an idea that can be hard to learn.

Jungian tradition holds that for a man this inner helpmate is usually a feminine soul or anima figure \& for a women it is usually a masculine spiritually guide or animus. So when we fall in love, we are often really falling in love with that inner anima or animus who will take us where we long to go, to paradise. We see the anima or animus in flesh-\&-blood people with whom we hope to share an earthly, sensual paradise (usually including a tropical cruise or a weekend of skiing at Vale--ad--vertisers are happy to help us project these archetypes onto the outer world). Don't get me wrong. Flesh \& blood \& sensuality are all great. They just aren't going to substitute for the inner figure or the inner goal. But you can see what a confusion divine love can make when 2 mortals set out to love each other in a human way.

But maybe the confusion is okay, for a time, at some point in one's life. As the novelist Charles Williams wrote, ``Unless devotion is given to the thing which must prove false in the end, the thing that is true in the end cannot enter.'''' -- \cite[pp. 173--174]{Aron2013}

\subsection{Overwhelming Love \& Insecure Attachment}
``As we already have discussed, HSPs' relationships to everyone \& everything are greatly affected by the nature of their childhood attachments to their 1st caretakers. Since only about 50--60\% of the population enjoyed a secure attachment in childhood (a shocking statistic, really), those of you HSPs who tend to be very cautious about close relationships (avoidant), or very intense in them (anxious-ambivalent), can still consider yourselves quite normal. But your responses to relationship are powerful because there is so much unfinished business in that department.

Often those with insecure attachment styles try very hard to avoid love in order not to be hurt. Or maybe it just seems like a waste of time \& you try not to think about why you see it differently than most of the world. Yet no matter how hard you try, someday you may find yourself trying again to get it right. Someone appears, \& it seems safe enough to risk an attachment. Or there is something about the other person that reminds you of some safe person who passed too briefly through your life. Or something inside is getting desperate enough to take another chance. Suddenly you attach, as Ellen did.

Although Ellen had never felt as close to her husband as he would have liked, she had thought she was fairly happily married at the time that she finished her 1st large sculpture. But after the yearlong project was completed \& shipped out, she found herself feeling oddly empty. She rarely shared such feelings with anyone, but 1 day she slipped into talking about it with an older, stout woman who wore her long gray hair in a bun.

Until that conversation, Ellen had never noticed the woman, who was considered something of an eccentric in Ellen's community. But the older woman happened to have been trained as a counselor \& knew how to listen empathically. The next day, Ellen found herself thinking about the woman all the time. She wanted to be with her again. The woman was flattered to have such a glamorous artist as a friend, \& the relationship bloomed.

But for Ellen it was more than friendship. It was a strangely desperate need. To her own amazement, it soon became sexualized for both of them, \& Ellen's marriage grew turbulent. For the sake of her husband \& children, she decided she wanted to break off the relationship, but she could not. It was utterly impossible.

After a year of stormy scenes among the 3 of them, Ellen began to find intolerable faults in the other woman -- mainly, a violent temper. The relationship ended, \& Ellen's marriage survived. But she never understood what had happened to her until years later in psychotherapy.

In the course of exploring her early childhood, Ellen learned from her older sister that their busy mother had had little time or inclination for babies. Ellen had been raised by a series of baby-sitters. Ellen could remember one, a Mrs. North, who later was her 1st Sunday school teacher. Mrs. North had been extraordinarily kind \& warm; indeed, little Ellen had thought Mrs. North was God. \& Mrs. North had been a stout, homely woman who wore her gray hair in a bun.

Ellen had grown up unconsciously programmed. 1st, she was programmed to avoid attaching to anyone, since her caretakers had changed so often. But at a deeper level she was programmed to watch for someone like Mrs. North \& then to risk everything to be secure once again, as she had been for a few hours each day in infancy with the actual Mrs. North.

We all go out programmed in some way: to please \& cling to the 1st kind person who promises to love \& protect us; to find the perfect parent \& worship that person totally; to be extremely careful of attaching to anyone; to attach to someone just like the person who did not want us the 1st time (to see if we can change them this time) or who insisted we never grow up; or just to find another safe harbor like the one we enjoyed as children.

Look back over your love history. Can you make sense of it in terms of your early attachment? Did you bring to it intense needs left over from childhood? To have some of those needs left over is to be supplied with the normal ``glue'' of adult closeness. But we can ask only so much from a fellow adult. Anyone who really wants an adult with a child's needs (e.g., a need to never have the other out of sight) has something unresolved going on from the past, too. Psychotherapy is about the only place where one can wake up to what was lost, mourn the rest, \& learn to control the overwhelming feelings.

But what about normal romantic love, which temporarily makes life so wonderfully nonnormal?'' -- \cite[pp. 174--176]{Aron2013}

\subsection{The 2 Ingredients for Mutual Love}
``In studying hundreds of accounts of falling in love (\& friendship) written by people of all ages, my husband (a social psychologist with whom I have conducted considerable research on close relationships) \& I found 2 themes to be most common. 1st, obviously the person falling in love liked certain things about the other very much. But also Cupid's arrow usually pierced their armor only at the moment when they found out that the other person liked them.

These 2 factors -- liking certain things about the other \& finding out the other person likes you -- give me an image of a world in which people walk around admiring each other, just waiting for someone else to confess their love. This image is important for HSPs to keep in mind because 1 of the most arousing moments in one's life is either confessing or receiving a declaration of affection. But if we want to be close to someone, we must do it! We must endure all the risks of getting closer \& being close, including speaking up. Cyrano de Bergerac learned that lesson, \& so did Capt. John Smith.'' -- \cite[p. 177]{Aron2013}

\subsection{How Arousal Can Make Anyone Fall In Love}
``A man meets an attractive woman on a flimsy suspension bridge swaying in the wind high over a mountain gorge. Or he meets the same woman on a sturdy wooden bridge a foot above a rivulet. In which place is the man more likely to be romantically attracted to the woman? According to the results of an experiment done by my husband \& a colleague (which is now famous in social psychology), there will be far more falling in love on the suspension bridge. Other research has found that we are more likely to be romantically attracted to someone else if we are aroused in any way, even from running in place or listening to a tape of a comedy monologue.

There are several theories about why arousal of any kind can lead to attraction if someone appropriate is at hand. 1 reason might be that we always try to attribute arousal to something, \& if we can, we would especially like to attribute it to feeling attraction. Or it may be that high but tolerable levels of arousal are associated in our minds with self-expansion \& excitement, \& these in turn are associated with being attracted to someone. This discovery has interesting implications for HSPs. If we are more easily aroused than others, we will on average be more likely to fall in love (\& perhaps harder as well) when we're with someone who's attractive.

Look back at your own love history. Did you go through an arousing experience before or while meeting someone you've loved? For that matter, after going through some ordeal, have you ever felt strongly attached to the people who went through it with you? Or to doctors, therapists, family members, or friends who've helped you deal with a crisis or with pain? Think of all the friendships formed in high school \& college, while everyone is experiencing so many new, intensely arousing situations. Now you understand why.'' -- \cite[pp. 177--178]{Aron2013}

\subsection{2 Other Reasons HSPs Are More Prone to Love}
``Another source of falling in love can be doubts about one's own self-worth. E.g., 1 study found that women students whose self-esteem had been lowered (by something they were told during the experiment) were more attracted to a potential male partner than those whose self-esteem had not been compromised. Similarly, people are especially likely to fall in love after a breakup.

As I have emphasized, HSPs are prone to low self-esteem because they are not their culture's ideal. So sometimes they consider themselves lucky if someone wants them at all. But love on this basis can backfire. Later, you may realize that the person you fell in love with was very much your inferior or simply not your type.

Look back at your own love history. Has low self-esteem played a role?

The main solution, of course, is to build up your self-esteem by reframing your life in terms of your sensitivity, doing some inner work on whatever else lowered your confidence, \& getting out in the world on your terms \& proving to yourself that you're okay. You'll be surprised how many people will love you deeply just \textit{because} of your sensitivity.

Then there is the very human tendency to enter or persist in a close relationship out of sheer fear of being alone, overaroused, or faced with new or frightening situations. I think this is a major reason why research finds that $\frac{1}{3}$ of college students fall in love during their 1st year away from home. We're all social animals, feeling safer in each other's company. But you don't want to put up with just anyone out of fear of being alone. The other will sense it eventually \& be hurt or take advantage of you. You both deserve better.

Look back over your love history. Did you fall in love out of fear of being alone? I believe that HSPs ought to feel that they can survive at least for a while without a close, romantic relationship. Otherwise, we are not free to wait for a person we really like.

If you cannot live alone yet, it's nothing to be ashamed of. Most likely something damaged your trust in the world, or someone wanted you not to develop that trust. But if it's practical, do try living on your own. Should it seem too difficult, work it through with a therapist to support \& coach you -- someone who will not abuse or abandon you \& who has no interest in the outcome except seeing you self-sufficient.

Nor do you have to be \textit{totally} alone. There are some great other comforts available, like good friends, loyal family members, the roommate who happens to be home \& ready to go to a movie, big-hearted dogs, \& cuddly cats.'' -- \cite[pp. 178--179]{Aron2013}

\subsection{Deepening a Friendship}

%------------------------------------------------------------------------------%

\section{Healing the Deeper Wounds}

%------------------------------------------------------------------------------%

\section{Medics, Medications, \& HSPs}

%------------------------------------------------------------------------------%

\section{Soul \& Spirit}

%------------------------------------------------------------------------------%

\section{Tips for Health-Care Professionals Working With Highly Sensitive People}

%------------------------------------------------------------------------------%

\section{Tips for Teachers Working With Highly Sensitive Students}

%------------------------------------------------------------------------------%

\section{Tips for Employers of Highly Sensitive People}

%------------------------------------------------------------------------------%

\section{Notes}

%------------------------------------------------------------------------------%

\section{About the Author}

%------------------------------------------------------------------------------%

\printbibliography[heading=bibintoc]
	
\end{document}