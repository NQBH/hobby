\documentclass{article}
\usepackage[backend=biber,natbib=true,style=authoryear]{biblatex}
\addbibresource{/home/nqbh/reference/bib.bib}
\usepackage{tocloft}
\renewcommand{\cftsecleader}{\cftdotfill{\cftdotsep}}
\usepackage[colorlinks=true,linkcolor=blue,urlcolor=red,citecolor=magenta]{hyperref}
\usepackage{algorithm,algpseudocode,amsmath,amssymb,amsthm,float,graphicx,mathtools}
\allowdisplaybreaks
\numberwithin{equation}{section}
\newtheorem{assumption}{Assumption}[section]
\newtheorem{conjecture}{Conjecture}[section]
\newtheorem{corollary}{Corollary}[section]
\newtheorem{definition}{Definition}[section]
\newtheorem{example}{Example}[section]
\newtheorem{lemma}{Lemma}[section]
\newtheorem{notation}{Notation}[section]
\newtheorem{principle}{Principle}[section]
\newtheorem{problem}{Problem}[section]
\newtheorem{proposition}{Proposition}[section]
\newtheorem{question}{Question}[section]
\newtheorem{remark}{Remark}[section]
\newtheorem{theorem}{Theorem}[section]
\usepackage[left=0.5in,right=0.5in,top=1.5cm,bottom=1.5cm]{geometry}
\usepackage{fancyhdr}
\pagestyle{fancy}
\fancyhf{}
\lhead{\small Sect.~\thesection}
\rhead{\small\nouppercase{\leftmark}}
\renewcommand{\sectionmark}[1]{\markboth{#1}{}}
\cfoot{\thepage}
\def\labelitemii{$\circ$}

\title{Quiet: The Power of Introverts in a World That Can't Stop Talking}
\author{Susan Cain}
\date{\today}

\begin{document}
\maketitle
\tableofcontents

%------------------------------------------------------------------------------%

\section*{More Advance Noise for \textit{Quiet}}

\begin{quotation}
	``An intriguing \& potentially life-altering examination of the human psyche that is sure to benefit both introverts \& extroverts alike.'' -- Kirkus Reviews (starred review)
	
	``Gentle is powerful $\ldots$ Solitude is socially productive $\ldots$ These important counterintuitive ideas are among the many reasons to take \textit{Quiet} to a quiet corner \& absorb its brilliant, thought-provoking message.'' -- Rosabeth Moss Kanter, professor at Harvard Business School, author of \textit{Confidence} \& \textit{SuperCorp}
	
	``All informative, well-researched book on the power of quietness \& the virtues of having a rich inner life. It dispels the myth that you have to be extroverted to be happy \& successful.'' -- Judith Orloff, M.D., author of \textit{Emotional Freedom}
	
	``In this engaging \& beautiful written book, Susan Cain makes a powerful case for the wisdom of introspection. She also warns us ably about the downside to our culture's noisiness, including all that it risks drowning out. About the din, Susan's own voice remains a compelling presence -- thoughtful, generous, calm, \& eloquent. \textit{Quiet} deserves a very large readership.'' -- Christopher Lane, author of \textit{Shyness: How Normal Behavior Became a Sickness}
	
	``Susan Cain's quest to understand introversion, a beautifully wrought journey from the lab bench to the motivational speaker's hall, offers convincing evidence for valuing substance over style, steak over sizzle, \& qualities that are, in America, often derided. This book is brilliant, profound, full of feeling \& brimming with insights.'' -- Sheri Fink, M.D., author of \textit{War Hospital}
	
	``Brilliant, illuminating, empowering! \textit{Quiet} gives not only a voice, but a path to homecoming for so may who've walked through the better part of their lives thinking the way they engage with the world is something in need of fixing.'' -- Jonathan Fields, authors of \textit{Uncertainty: Turning Fear \& Doubt into Fuel for Brilliance}
	
	``Shatters misconceptions $\ldots$ Cain consistently holds the reader's interest by presenting individual profiles $\ldots$ \& reporting on the latest studies. Her diligence, research, \& passion for this important topic has richly paid off.'' -- Adam M. Grant, Ph.D., associate professor of management, the Wharton School of Business
	
	``Once in a blue moon, a book comes along that gives us startling new insights. \textit{Quiet} is that book: it's part page-turner, part cutting-edge science. The implications for business are especially valuable: \textit{Quiet} offers tips on how introverts can lead effectively, give winning speeches, avoid burnout, \& choose the right roles. This charming, gracefully written, thoroughly researched book is simply masterful.'' -- Publishers Weekly
	
	``\textit{Quiet} elevates the conversation about introverts in our outwardly oriented society to new heights. I think that many introverts will discover that, even though they didn't know it, they have been waiting for this book all their lives.'' -- Adam S. McHugh, author of \textit{Introverts in the Church}
	
	``Susan Cain's \textit{Quiet} is wonderfully informative about the culture of the extravert ideal \& the psychology of a sensitive temperament, \& she is helpful perceptive about how introverts can make the most of their personality preferences in all aspects of life. Society needs introverts, so everyone can benefit from the insights in this important book.'' -- Jonathan M. Cheek, professor of psychology at Wellesley College, co-editor of \textit{Shyness: Perspectives on Research \& Treatment}
	
	``A brilliant, important, \& personally affecting book. Cain shows that, for all its virtue, America's Extrovert Ideal takes up way too much oxygen. Cain herself is the perfect person to make this case -- with winning grace \& clarity she shows us what it looks like to think outside the group.'' -- Christine Kenneally, author of \textit{The 1st Word}
	
	``What Susan Cain understands -- \& readers of this fascinating volume will soon appreciate -- is something that psychology \& our fast-moving \& fasttalking society have been all too slow to realize: Not only is there really nothing wrong with being quiet, reflective, shy, \& introverted, but there are distinct advantages to being this way.'' -- Jay Belsky, Robert M. \& Natalie Reid Dorn Professor, Human \& Community Development, University of California, Davis
	
	``Author Susan Cain exemplifies her own quiet power in this exquisitely written \& highly readable page-turner. She brings important research \& poignant, personal examples into the light, greatly deepening our understanding of the introvert experience.'' -- Jennifer B. Kahnweiler, Ph.D., author of \textit{The Introverted Leader}
	
	\textit{``A species in which everyone was General Patton would not succeed, any more than would a race in which everyone was Vincent van Gogh. I prefer to think that the planet needs athletes, philosophers, sex symbols, painters, scientists; it needs the warmhearted, the hardhearted, the coldhearted, \& the weakhearted. It needs those who can devote their lives to studying how many droplets of water are secreted by the salivary glands of dogs under which circumstances, \& it needs those who can capture the passing impression of cherry blossoms in a 14-syllable poem or devote 25 pages to the dissection of a small boy's feelings as he lies in bed in the dark waiting for his mother to kiss him goodnight $\ldots$ Indeed the presence of outstanding strengths presupposes that energy needed in other areas has been channeled away from them.''} -- Allen Shawn
\end{quotation}

%------------------------------------------------------------------------------%

\section*{Author's Note}
``I have been working on this book officially since 2005, \& unofficially for my entire adult life. I have spoken \& written to hundreds, perhaps thousands, of people about the topics covered inside, \& have read as many books, scholarly papers, magazine articles, chat-room discussions, \& blog posts. Some of these I mention in the book; others informed almost every sentence I wrote. \textit{Quiet} stands on many shoulders, especially the scholars \& researchers whose work taught me so much. In a perfect world, I would have named every 1 of my sources, mentors, \& interviewees. But for the sake of readability, some names appear only in the Notes or Acknowledgments.

For similar reasons, I did not use ellipses or brackets in certain quotations but made sure that the extra or missing words did not change the speaker's or writer's meaning. If you would like to quote these written sources from the original, the citations directing you to the full quotations appear in the Notes.

I've changed the names \& identifying details of some of the people whose stories I tell, \& in the stories of my own work as a lawyer \& consultant. To protect the privacy of the participants in Charles di Cagno's public speaking workshop, who did not plan to be included in a book when they signed up for the class, the story of my 1st evening in class is a composite based on several sessions; so is the story of Greg \& Emily, which is based on many interviews with similar couples. Subject to the limitations of memory, all other stories are recounted as they happened or were told to me. I did not fact-check the stories people told me about themselves, but only included those I believed to be true.'' -- \cite[p. 12]{Cain2013}

%------------------------------------------------------------------------------%

\section*{Introduction: The North \& South of Temperament}
``Montgomery, Alabama. Dec 1, 1955. Early evening. A public bus pulls to a stop \& a sensibly dressed woman in her 40s get son. She carries herself erectly, despite having spent the day bent over an ironing board in a dingy basement tailor shop at the Montgomery Fair department store. Her feet are swollen, her shoulders ache. She sits in the 1st row of the Colored section \& watches quietly as the bus fills with riders. Until the driver orders her to give her seat to a white passenger.

The woman utters a single word that ignites 1 of the most important civil rights protests of the 20th century, 1 word that helps America find its better self.

The word is ``No.''

The driver threatens to have her arrested.

``You may do that,'' says Rosa Parks.

A police officer arrives. He asks Parks why she won't move.

``Why do you all push us around?'' she answers simply.

``I don't know,'' he says. ``But the law is the law, \& you're under arrest.''

On the afternoon of her trial \& conviction for disorderly conduct, the Montgomery Improvement Association holds a rally for Parks at the Holt Street Baptist Church, in the poorest section of town. 5000 gather to support Parks's lonely act of courage. They squeeze inside the church until its pews can hold no more. The rest wait patiently outside, listening through loudspeakers. The Reverend Martin Luther King Jr. addresses the crowd. ``There comes a time that people get tired of being trampled over by the iron feet of oppression,'' he tells them. ``There comes a time when people get tired of being pushed out of the glittering sunlight of life's July \& left standing amidst the piercing chill of an Alpine November.''

He praises Parks's bravery \& hugs her. She stands silently, her mere presence enough to galvanize the crowd. The association launches a city-wide bus boycott that lasts 381 days. The people trudge miles to work. They carpool with strangers. They change the course of American history.

I had always imagined Rosa Parks as a stately woman with a bold temperament, someone who could easily stand up to a busload of glowering passengers. But when she died in 2005 at the age of 92, the flood of obituaries recalled her as soft-spoken, sweet, \& small in stature. They said she was ``timid \& shy'' but had ``the courage of a lion.'' They were full of phrases like ``radical humility'' \& ``quiet fortitude.'' What does it mean to be quiet \textit{\&} have fortitude? These descriptions asked implicitly. How could you be shy \textit{\&} courageous?

Parks herself seemed aware of this paradox, calling her autobiography \textit{Quiet Strength} -- a title that challenges us to question our assumptions. Why \textit{shouldn't} quiet be strong? \& what else can quiet do that we don't give it credit for?

Our lives are shaped us profoundly by personality as by gender or race. \& the single most important aspect of personality -- the ``north \& south of temperament,'' as 1 scientist puts it -- is where we fall on the introvert-extrovert spectrum. Our place on this continuum influences our choice of friends \& mates, \& how we make conversation, resolve differences, \& show love. It affects the careers we choose \& whether or not we succeed at them. It governs how likely we are to exercise, commit adultery, function well without sleep, learn from our mistakes, place big bets in the stock market, delay gratification, be a good leader, \& ask ``what if.''\footnote{Answer key: exercise: extroverts; commit adultery: extroverts; function well without sleep: introverts; learn from our mistakes: introverts; place big bets: extroverts; delay gratification: introverts; be a good leader: in some cases introverts, in other cases extroverts, depending on the type of leadership called for; ask ``what if'': introverts.} It's reflected in our brain pathways, neurotransmitters, \& remote corners of our nervous systems. Today introversion \& extroversion are 2 of the most exhaustively researched subjects in personality psychology, arousing the curiosity of hundreds of scientists.

These researchers have made exciting discoveries aided by the latest technology, but they're part of a long \& storied tradition. Poets \& philosophers have been thinking about introverts \& extroverts since the dawn of recorded time. Both personalities types appear in the Bible \& in the writings of Greek \& Roman physicians, \& some evolutionary psychologists say that the history of these types reaches back even farther than that: the animal kingdom also boasts ``introverts'' \& ``extroverts,'' as we'll see, from fruit ies to pumpkinseed sh to rhesus monkeys. As with other
complementary pairings -- masculinity \& femininity, East \& West, liberal \& conservative -- humanity would be unrecognizable, \& vastly diminished, without both personality styles.

Take the partnership of Rosa Parks \& Martin Luther King Jr.: a formidable orator refusing to give up his seat on a segregated bus wouldn't have had the same effect as a modest woman who'd clearly prefer to keep silent but for the exigencies of the situation. \& Parks didn't have the stuff to thrill a crowd if she'd tried to stand up \& announce that she had a dream. But with King's help, she didn't
have to.

Yet today we make room for a remarkably narrow range of personality styles. We're told that to be great is to be bold, to be happy is to be sociable. We see ourselves as a nation of extroverts -- which means that we've lost sight of who we really are. Depending on which study you consult, $\frac{1}{3}$--$\frac{1}{2}$ of Americans are introverts -- in other words, \textit{1 out of every 2 or 3 people you know}. (Given that the United States is among the most extroverted of nations, the number must be at least as high in other parts of the world.) If you're not an introvert yourself, you are surely raising, managing, married to, or coupled with one.

If these statistics surprise you, that's probably because so many people pretend to be extroverts. Closet introverts pass undetected on playgrounds, in high school locker rooms, \& in the corridors of corporate America. Some fool even themselves, until some life event -- a layoff, an empty nest, an inheritance that frees them to spend time as they like -- jolts them into taking stock of their true natures. You have only to raise the subject of this book with your friends \& acquaintances to find that the most unlikely people consider themselves introverts.

It makes sense that so many introverts hide even from themselves. We live with a value system that I call the Extrovert Ideal -- the omnipresent belief that the ideal self is gregarious, alpha, \& comfortable in the spotlight. The archetypal extrovert prefers action to contemplation, risk-taking to heed-taking, certainty to doubt. He favors quick decisions, even at the risk of being wrong. She works well in teams \& socializes in groups. We live to think that we value individuality, but all too often we admire 1 \textit{type} of individual -- the kind who's comfortable ``putting himself out there.'' Sure, we allow technologically gifted loners who launch companies in garages to have any personality they please, but they are the exceptions, not the rule, \& our tolerance extends mainly to those who get fabulously wealthy or hold the promise of doing so.

Introversion -- along with its cousins sensitivity, seriousness, \& shyness -- is now a 2nd-class personality trait, somewhere between a disappointment \& a pathology. Introverts living under the Extrovert Ideal are like woman in a man's world, discounted because of a trait that goes to the core of who they are. Extroversion is an enormously appealing personality style, but we've turned it into an oppressive standard to which most of us feel we must conform.

The Extrovert Ideal has been documented in many studies, though this research has never been grouped under a single name. Talkative people, e.g., are rated as smarter, better-looking, more interesting, \& more desirable as friends. Velocity of speech counts as well as volume: we rank fast talkers as more competent \& likable than slow ones.The same dynamics apply in groups, where research shows that the voluble are considered smarter than the reticent -- even though there's zero correlation between the gift of gab \& good ideas. Even the word \textit{introvert} is stigmatized -- 1 informal study, by psychologist Laurie Helgoe, found that introverts described their own physical appearance in vivid language (``green blue eyes,'' ``exotic,'' ``high cheekbones''), but when asked to describe generic introverts they drew a bland \& distasteful picture (``ungainly,'' ``neutral colors,'' ``skin problems'').

But we make a grave mistake to embrace the Extrovert Ideal so unthinkingly. Some of our greatest ideas, art, \& inventions -- from the theory of evolution to van Gogh's sunflowers to the personal computer -- came from quiet \& cerebral people who knew how to tune in to their inner worlds \& the treasures to be found there. Without introverts, the world would be devoid of:
\begin{quotation}
	the theory of gravity, the theory of relativity, W. B. Yeats's ``The 2nd Coming'', Chopin's nocturnes, Proust's \textit{In Search of Lost Time}, Peter Pan, Orwell's \textit{Nineteen Eighty-Four} \& \textit{Animal Farm}, The Cat in the Hat, Charlie Brown, \textit{Schindler's List, E.T.}, \& \textit{Close Encounters of the 3rd Kind}, Google, Harry Potter\footnote{Sir Isaac Newton, Albert Einstein, W. B. Yeats, Fr\'ed\'eric Chopin, Marcel Proust, J. M. Barrie, George Orwell, Theodor Geisel (Dr. Seuss), Charles Schulz, Steven Spielberg, Larry Page, J. K. Rowling.}
\end{quotation}
As the science journalist Winifred Gallagher writes: ``The glory of the disposition that stops to consider stimuli rather than rushing to engage with them is its long association with intellectual \& artistic achievement. Neither $E = mc^2$ nor \textit{Paradise Lost} was dashed off by a party animal.'' Even in less obviously introverted occupations, like finance, politics, \& activism, some of the greatest leaps forward were made by introverts. In this book we'll see how figures like Eleanor Roosevelt, Al Gore, Warren Bu ett, Gandhi -- \& Rosa Parks -- achieved what they did not in spite of but \textit{because of} their introversion.

Yes, as \textit{Quiet} will explore, many of the most important institutions of contemporary life are designed for those who enjoy group projects \& high levels of stimulation. As children, our classroom desks are increasingly arranged in pods, the better to foster group learning, \& research suggests that the vast majority of teachers believe that the ideal student is an extrovert. We watch TV shows whose protagonists are not the ``children next door,'' like the Cindy Bradys \& Beaver Cleavers of yesteryear, but rock stars \& webcast hostesses with outsized personalities, like Hannah Montana \& Carly Shay of \textit{iCarly}. Even Sid the Science Kid, a PBS-sponsored role model for the preschool set, kicks off each school day by performing dance moves with his pals. (``Check out my moves! I'm a rock star!'')

As adults, many of us work for organizations that insist we work in teams, in offices without walls, for supervisors who value ``people skills'' above all. To advance our careers, we're expected to promote ourselves unabashedly. The scientists whose research gets funded often have confident, perhaps overconfident, personalities. The artists whose work adorns the walls of contemporary museums strike impressive poses at gallery openings. The authors whose books get published -- once accepted as a reclusive breed -- are now vetted by publicists to make sure they're talk-show ready. (You wouldn't be reading this book if I hadn't convinced my publisher that I was enough of a pseudo-extrovert to promote it.)

If you're an introvert, you also know that the bias against quiet can cause deep psychic pain. As a child you might have overheard your parents apologize for your shyness. (``Why can't you be more like the Kennedy boys?'' the Camelot-besotted parents of 1 man I interviewed repeatedly asked him.) Or at school you might have been prodded to come ``out of your shell'' -- that noxious expression which fails to appreciate that some animals naturally carry shelter everywhere they go, \& that some humans are just the same. ``All the comments from childhood still ring in my ears, that I was lazy, stupid, slow, boring,'' writes a member of an e-mail list called Introvert Retreat. ``By the time I was old enough to figure out that I was simply introverted, it was a part of my being, the assumption that there is something inherently wrong with me. I wish I could find that little vestige of doubt \& remove it.

Now that you're an adult, you might still feel a pang of guilt when you decline a dinner invitation in favor of a good book. Or maybe you like to eat alone in restaurants \& could do without the pitying looks from fellow diners. Or you're told that you're ``in your head too much,'' a phrase that's often deployed against the quiet \& cerebral.

Of course, there's another word for such people: thinkers.

I have seen firsthand how difficult it is for introverts to take stock of their own talents, \& how powerful it is when finally they do. For more than 10 years I trained people of all stripe -- corporate lawyers \& college students, hedge-fund managers \& married couples -- in negotiation skills. Of course, we covered the basics: how to prepare for a negotiation, when to make the 1st offer, \& what to do when the other person says ``take it or leave it.'' But I also helped clients figure out their natural personalities \& how to make the most of them.

My very 1st client was a young woman named Laura. She was a Wall Street lawyer, but a quiet \& daydreamy one who dreaded the spotlight \& disliked aggression. She had managed somehow to make it through the crucible of Harvard Law School -- a place where classes are conducted in huge, gladiatorial amphitheaters, \& where she once got so nervous that she threw up on the way to class. Now that she was in the real world, she wasn't sure she could represent her clients as forcefully as they expected.

For the 1st 3 years on the job, Laura was so junior that she never had to test this premise. But 1 day the senior lawyer she'd been working with went on vacation, leaving her in charge of an important negotiation. The client was a South American manufacturing company that was about to default on a bank loan \& hoped to renegotiate its terms; a syndicate of bankers that owned the endangered loan sat on the other side of the negotiating table.

Laura would have preferred to hide under said table, but she was accustomed to fighting such impulses. Gamely but nervously, she took her spot in the lead chair, flanked by her clients: general counsel on 1 side \& senior financial officer on the other. These happened to be Laura's favorite clients: gracious \& soft-spoken, very different from the master-of-the-universe types her firm usually represented. In the past, Laura had taken the general counsel to a Yankees game \& the financial officer shopping for a handbag for her sister. But now these cozy outings -- just the kind of socializing Laura enjoyed -- seemed a world away. Across the table sat 9 disgruntled investment bankers in tailored suits \& expensive shoes, accompanied by their lawyer, a square-jawed woman with a hearty manner. Clearly not the self-doubting type, this woman launched into an impressive speech on how Laura's clients would be lucky simply to accept the bankers' terms. It was, she said, a very magnanimous offer.

Everyone waited for Laura to reply, but she couldn't think of anything to say. So she just sat there. Blinking. All eyes on her. Her clients shifting uneasily in their seats. Her thoughts running in a familiar loop: \textit{I'm too quiet for this kind of thing, too unassuming, too cerebral}. She imagined the person who would be better equipped to save the day: someone bold, smooth, ready to pound the table. In middle school this person, unlike Laura, would have been called ``outgoing,'' the highest accolade her 7th-grade classmates knew, higher even than ``pretty,'' for a girl, or ``athletic,'' for a guy. Laura promised herself that she only had to make it through the day. Tomorrow she would go look for another career.

Then she remembered what I'd told her again \& again: she was an introvert, \& as such she had unique powers in negotiation -- perhaps less obvious but no less formidable. She'd probably prepared more than everyone else. She had a quiet but firm speaking style. She rarely spoke without thinking. Being mild-mannered, she could take strong, even aggressive, positions while coming across as perfectly reasonable. And she tended to ask questions -- lots of them -- \& actually listen to the answers, which, no matter what your personality, is crucial to strong negotiation.

So Laura finally started doing what came naturally.

``Let's go back a step. What are your numbers based on?'' she asked.

``What if we structured the loan this way, do you think it might work?''

``That way?''

``Some other way?''

At 1st her questions were tentative. She picked up steam as she went along, posing them more forcefully \& making it clear that she'd done her homework \& wouldn't concede the facts. But she also stayed true to her own style, never raising her voice or losing her decorum. Every time the bankers made an assertion that seemed unbudgeable, Laura tried to be constructive. ``Are you saying that's the only way to go? What if we took a different approach?''

Eventually her simple queries shifted the mood in the room, just as the negotiation textbooks say they will. The bankers stopped speechifying \& dominance-posing, activities for which Laura felt hopelessly ill-equipped, \& they started having an actual conversation.

More discussion. Still no agreement. 1 of the bankers revved up again, throwing his papers down \& storming out of the room. Laura ignored this display, mostly because she didn't know what else to do. Later on someone told her that at that pivotal moment she'd played a good game of something called ``negotiation jujitsu''; but she knew that she was just doing what you learn to do naturally as a quiet person in a loudmouth world.

Finally the 2 sides struck a deal. The bankers left the building, Laura's favorite clients headed for the airport, \& Laura went home, curled up with a book, \& tried to forget the day's tensions.

But the next morning, the lead lawyer for the bankers -- the vigorous woman with the strong jaw -- called to offer her a job. ``I've never seen anyone so nice \& so tough at the same time,'' she said. \& the day after that, the lead banker called Laura, asking if \textit{her}
law firm would represent \textit{his} company in the future. ``We need someone who can help us put deals together without letting ego get in the way,'' he said.

By sticking to her own gentle way of doing things, Laura had reeled in new business for her firm \& a job offer for herself. Raising her voice \& pounding the table was unnecessary.

Today Laura understands that her introversion is an essential part of who she is, \& she embraces her reflective nature. The loop inside her head that accused her of being too quiet \& unassuming plays much less often. Laura knows that she can hold her own when she needs to.

What exactly do I mean when I say that Laura is an \textit{introvert}? When I started writing this book, the 1st thing I wanted to find out was precisely how researchers define introversion \& extroversion. I knew that in 1921 the influential psychologist Carl Jung had published a bombshell of a book, \textit{Psychological Types}, popularizing the terms \textit{introvert} \& \textit{extrovert} as the central building blocks of personality. Introverts are drawn to the inner world of thought \& feeling, said Jung, extroverts to the external life of people \& activities. Introverts focus on the meaning they make of the events swirling around them; extroverts plunge into the events themselves. Introverts recharge their batteries by being alone; extroverts need to recharge when they don't socialize enough. If you've ever taken a Myers-Briggs personality test, which is based on Jung's thinking \& used by the majority of universities \& Fortune 100 companies, then you may already be familiar with these ideas.

'' -- \cite[pp. 13--]{Cain2013}

%------------------------------------------------------------------------------%

\begin{center}\LARGE\sf
	\textbf{Part I: The Extrovert Ideal}
\end{center}

\section{The Rise of The ``Mighty Likable Fellow'': How Extroversion Became the Cultural Ideal}

%------------------------------------------------------------------------------%

\section{The Myth of Charismatic Leadership: The Culture of Personality, 100 Years Later}

%------------------------------------------------------------------------------%

\section{When Collaboration Kills Creativity: The Rise of the New Groupthink \& the Power of Working Alone}

%------------------------------------------------------------------------------%

\begin{center}\LARGE\sf
	\textbf{Part II: Your Biology, Your Self?}
\end{center}

\section{Is Temperament Destiny?: Nature, Nurture, \& the Orchid Hypothesis}

%------------------------------------------------------------------------------%

\section{Beyond Temperament: The Role of Free Will (\& the Secret of Public Speaking for Introverts)}

%------------------------------------------------------------------------------%

\section{``Franklin Was a Politician, But Eleanor Spoke Out of Conscience'': Why Cool Is Overrated}

%------------------------------------------------------------------------------%

\section{Why Did Wall Street Crash \& Warren Buffett Prosper?: How Introverts \& Extroverts Think (\& Process Dopamine) Differently}

%------------------------------------------------------------------------------%

\begin{center}\LARGE\sf
	\textbf{Part III: Do All Cultures Have An Extrovert Ideal?}
\end{center}

\section{Soft Power: Asian-Americans \& the Extrovert Ideal}

%------------------------------------------------------------------------------%

\begin{center}\LARGE\sf
	\textbf{Part IV: How to Love, How to Work}
\end{center}

\section{When Should You Act More Extroverted Than You Really Are?}

%------------------------------------------------------------------------------%

\section{The Communication Gap: How to Talk to Members of the Opposite Type}

%------------------------------------------------------------------------------%

\section{On Cobblers \& Generals: How to Cultivate Quiet Kids in a World That Can't Hear Them}

%------------------------------------------------------------------------------%

\section{Conclusion: Wonderland}

%------------------------------------------------------------------------------%

\section{A Note on the Dedication}

%------------------------------------------------------------------------------%

\section{A Note on the Words \textit{Introvert} \& \textit{Extrovert}}

%------------------------------------------------------------------------------%

\printbibliography[heading=bibintoc]
	
\end{document}