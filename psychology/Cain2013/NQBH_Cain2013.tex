\documentclass{article}
\usepackage[backend=biber,natbib=true,style=authoryear]{biblatex}
\addbibresource{/home/nqbh/reference/bib.bib}
\usepackage{tocloft}
\renewcommand{\cftsecleader}{\cftdotfill{\cftdotsep}}
\usepackage[colorlinks=true,linkcolor=blue,urlcolor=red,citecolor=magenta]{hyperref}
\usepackage{algorithm,algpseudocode,amsmath,amssymb,amsthm,float,graphicx,mathtools}
\allowdisplaybreaks
\numberwithin{equation}{section}
\newtheorem{assumption}{Assumption}[section]
\newtheorem{conjecture}{Conjecture}[section]
\newtheorem{corollary}{Corollary}[section]
\newtheorem{definition}{Definition}[section]
\newtheorem{example}{Example}[section]
\newtheorem{lemma}{Lemma}[section]
\newtheorem{notation}{Notation}[section]
\newtheorem{principle}{Principle}[section]
\newtheorem{problem}{Problem}[section]
\newtheorem{proposition}{Proposition}[section]
\newtheorem{question}{Question}[section]
\newtheorem{remark}{Remark}[section]
\newtheorem{theorem}{Theorem}[section]
\usepackage[left=0.5in,right=0.5in,top=1.5cm,bottom=1.5cm]{geometry}
\usepackage{fancyhdr}
\pagestyle{fancy}
\fancyhf{}
\lhead{\small Sect.~\thesection}
\rhead{\small\nouppercase{\leftmark}}
\renewcommand{\sectionmark}[1]{\markboth{#1}{}}
\cfoot{\thepage}
\def\labelitemii{$\circ$}

\title{Quiet: The Power of Introverts in a World That Can't Stop Talking}
\author{Susan Cain}
\date{\today}

\begin{document}
\maketitle
\tableofcontents

%------------------------------------------------------------------------------%

\section*{More Advance Noise for \textit{Quiet}}

\begin{quotation}
	``An intriguing \& potentially life-altering examination of the human psyche that is sure to benefit both introverts \& extroverts alike.'' -- Kirkus Reviews (starred review)
	
	``Gentle is powerful $\ldots$ Solitude is socially productive $\ldots$ These important counterintuitive ideas are among the many reasons to take \textit{Quiet} to a quiet corner \& absorb its brilliant, thought-provoking message.'' -- Rosabeth Moss Kanter, professor at Harvard Business School, author of \textit{Confidence} \& \textit{SuperCorp}
	
	``All informative, well-researched book on the power of quietness \& the virtues of having a rich inner life. It dispels the myth that you have to be extroverted to be happy \& successful.'' -- Judith Orloff, M.D., author of \textit{Emotional Freedom}
	
	``In this engaging \& beautiful written book, Susan Cain makes a powerful case for the wisdom of introspection. She also warns us ably about the downside to our culture's noisiness, including all that it risks drowning out. About the din, Susan's own voice remains a compelling presence -- thoughtful, generous, calm, \& eloquent. \textit{Quiet} deserves a very large readership.'' -- Christopher Lane, author of \textit{Shyness: How Normal Behavior Became a Sickness}
	
	``Susan Cain's quest to understand introversion, a beautifully wrought journey from the lab bench to the motivational speaker's hall, offers convincing evidence for valuing substance over style, steak over sizzle, \& qualities that are, in America, often derided. This book is brilliant, profound, full of feeling \& brimming with insights.'' -- Sheri Fink, M.D., author of \textit{War Hospital}
	
	``Brilliant, illuminating, empowering! \textit{Quiet} gives not only a voice, but a path to homecoming for so may who've walked through the better part of their lives thinking the way they engage with the world is something in need of fixing.'' -- Jonathan Fields, authors of \textit{Uncertainty: Turning Fear \& Doubt into Fuel for Brilliance}
	
	``Shatters misconceptions $\ldots$ Cain consistently holds the reader's interest by presenting individual profiles $\ldots$ \& reporting on the latest studies. Her diligence, research, \& passion for this important topic has richly paid off.'' -- Adam M. Grant, Ph.D., associate professor of management, the Wharton School of Business
	
	``Once in a blue moon, a book comes along that gives us startling new insights. \textit{Quiet} is that book: it's part page-turner, part cutting-edge science. The implications for business are especially valuable: \textit{Quiet} offers tips on how introverts can lead effectively, give winning speeches, avoid burnout, \& choose the right roles. This charming, gracefully written, thoroughly researched book is simply masterful.'' -- Publishers Weekly
	
	``\textit{Quiet} elevates the conversation about introverts in our outwardly oriented society to new heights. I think that many introverts will discover that, even though they didn't know it, they have been waiting for this book all their lives.'' -- Adam S. McHugh, author of \textit{Introverts in the Church}
	
	``Susan Cain's \textit{Quiet} is wonderfully informative about the culture of the extravert ideal \& the psychology of a sensitive temperament, \& she is helpful perceptive about how introverts can make the most of their personality preferences in all aspects of life. Society needs introverts, so everyone can benefit from the insights in this important book.'' -- Jonathan M. Cheek, professor of psychology at Wellesley College, co-editor of \textit{Shyness: Perspectives on Research \& Treatment}
	
	``A brilliant, important, \& personally affecting book. Cain shows that, for all its virtue, America's Extrovert Ideal takes up way too much oxygen. Cain herself is the perfect person to make this case -- with winning grace \& clarity she shows us what it looks like to think outside the group.'' -- Christine Kenneally, author of \textit{The 1st Word}
	
	``What Susan Cain understands -- \& readers of this fascinating volume will soon appreciate -- is something that psychology \& our fast-moving \& fasttalking society have been all too slow to realize: Not only is there really nothing wrong with being quiet, reflective, shy, \& introverted, but there are distinct advantages to being this way.'' -- Jay Belsky, Robert M. \& Natalie Reid Dorn Professor, Human \& Community Development, University of California, Davis
	
	``Author Susan Cain exemplifies her own quiet power in this exquisitely written \& highly readable page-turner. She brings important research \& poignant, personal examples into the light, greatly deepening our understanding of the introvert experience.'' -- Jennifer B. Kahnweiler, Ph.D., author of \textit{The Introverted Leader}
	
	\textit{``A species in which everyone was General Patton would not succeed, any more than would a race in which everyone was Vincent van Gogh. I prefer to think that the planet needs athletes, philosophers, sex symbols, painters, scientists; it needs the warmhearted, the hardhearted, the coldhearted, \& the weakhearted. It needs those who can devote their lives to studying how many droplets of water are secreted by the salivary glands of dogs under which circumstances, \& it needs those who can capture the passing impression of cherry blossoms in a 14-syllable poem or devote 25 pages to the dissection of a small boy's feelings as he lies in bed in the dark waiting for his mother to kiss him goodnight $\ldots$ Indeed the presence of outstanding strengths presupposes that energy needed in other areas has been channeled away from them.''} -- Allen Shawn
\end{quotation}

%------------------------------------------------------------------------------%

\section*{Author's Note}
``I have been working on this book officially since 2005, \& unofficially for my entire adult life. I have spoken \& written to hundreds, perhaps thousands, of people about the topics covered inside, \& have read as many books, scholarly papers, magazine articles, chat-room discussions, \& blog posts. Some of these I mention in the book; others informed almost every sentence I wrote. \textit{Quiet} stands on many shoulders, especially the scholars \& researchers whose work taught me so much. In a perfect world, I would have named every 1 of my sources, mentors, \& interviewees. But for the sake of readability, some names appear only in the Notes or Acknowledgments.

For similar reasons, I did not use ellipses or brackets in certain quotations but made sure that the extra or missing words did not change the speaker's or writer's meaning. If you would like to quote these written sources from the original, the citations directing you to the full quotations appear in the Notes.

I've changed the names \& identifying details of some of the people whose stories I tell, \& in the stories of my own work as a lawyer \& consultant. To protect the privacy of the participants in Charles di Cagno's public speaking workshop, who did not plan to be included in a book when they signed up for the class, the story of my 1st evening in class is a composite based on several sessions; so is the story of Greg \& Emily, which is based on many interviews with similar couples. Subject to the limitations of memory, all other stories are recounted as they happened or were told to me. I did not fact-check the stories people told me about themselves, but only included those I believed to be true.'' -- \cite[p. 12]{Cain2013}

%------------------------------------------------------------------------------%

\section*{Introduction: The North \& South of Temperament}
``Montgomery, Alabama. Dec 1, 1955. Early evening. A public bus pulls to a stop \& a sensibly dressed woman in her 40s get son. She carries herself erectly, despite having spent the day bent over an ironing board in a dingy basement tailor shop at the Montgomery Fair department store. Her feet are swollen, her shoulders ache. She sits in the 1st row of the Colored section \& watches quietly as the bus fills with riders. Until the driver orders her to give her seat to a white passenger.

The woman utters a single word that ignites 1 of the most important civil rights protests of the 20th century, 1 word that helps America find its better self.

The word is ``No.''

The driver threatens to have her arrested.

``You may do that,'' says Rosa Parks.

A police officer arrives. He asks Parks why she won't move.

``Why do you all push us around?'' she answers simply.

``I don't know,'' he says. ``But the law is the law, \& you're under arrest.''

On the afternoon of her trial \& conviction for disorderly conduct, the Montgomery Improvement Association holds a rally for Parks at the Holt Street Baptist Church, in the poorest section of town. 5000 gather to support Parks's lonely act of courage. They squeeze inside the church until its pews can hold no more. The rest wait patiently outside, listening through loudspeakers. The Reverend Martin Luther King Jr. addresses the crowd. ``There comes a time that people get tired of being trampled over by the iron feet of oppression,'' he tells them. ``There comes a time when people get tired of being pushed out of the glittering sunlight of life's July \& left standing amidst the piercing chill of an Alpine November.''

He praises Parks's bravery \& hugs her. She stands silently, her mere presence enough to galvanize the crowd. The association launches a city-wide bus boycott that lasts 381 days. The people trudge miles to work. They carpool with strangers. They change the course of American history.

I had always imagined Rosa Parks as a stately woman with a bold temperament, someone who could easily stand up to a busload of glowering passengers. But when she died in 2005 at the age of 92, the flood of obituaries recalled her as soft-spoken, sweet, \& small in stature. They said she was ``timid \& shy'' but had ``the courage of a lion.'' They were full of phrases like ``radical humility'' \& ``quiet fortitude.'' What does it mean to be quiet \textit{\&} have fortitude? These descriptions asked implicitly. How could you be shy \textit{\&} courageous?

Parks herself seemed aware of this paradox, calling her autobiography \textit{Quiet Strength} -- a title that challenges us to question our assumptions. Why \textit{shouldn't} quiet be strong? \& what else can quiet do that we don't give it credit for?

Our lives are shaped us profoundly by personality as by gender or race. \& the single most important aspect of personality -- the ``north \& south of temperament,'' as 1 scientist puts it -- is where we fall on the introvert-extrovert spectrum. Our place on this continuum influences our choice of friends \& mates, \& how we make conversation, resolve differences, \& show love. It affects the careers we choose \& whether or not we succeed at them. It governs how likely we are to exercise, commit adultery, function well without sleep, learn from our mistakes, place big bets in the stock market, delay gratification, be a good leader, \& ask ``what if.''\footnote{Answer key: exercise: extroverts; commit adultery: extroverts; function well without sleep: introverts; learn from our mistakes: introverts; place big bets: extroverts; delay gratification: introverts; be a good leader: in some cases introverts, in other cases extroverts, depending on the type of leadership called for; ask ``what if'': introverts.} It's reflected in our brain pathways, neurotransmitters, \& remote corners of our nervous systems. Today introversion \& extroversion are 2 of the most exhaustively researched subjects in personality psychology, arousing the curiosity of hundreds of scientists.

'' -- \cite[pp. 13--]{Cain2013}

%------------------------------------------------------------------------------%

\begin{center}\LARGE\sf
	\textbf{Part I: The Extrovert Ideal}
\end{center}

\section{The Rise of The ``Mighty Likable Fellow'': How Extroversion Became the Cultural Ideal}

%------------------------------------------------------------------------------%

\section{The Myth of Charismatic Leadership: The Culture of Personality, 100 Years Later}

%------------------------------------------------------------------------------%

\section{When Collaboration Kills Creativity: The Rise of the New Groupthink \& the Power of Working Alone}

%------------------------------------------------------------------------------%

\begin{center}\LARGE\sf
	\textbf{Part II: Your Biology, Your Self?}
\end{center}

\section{Is Temperament Destiny?: Nature, Nurture, \& the Orchid Hypothesis}

%------------------------------------------------------------------------------%

\section{Beyond Temperament: The Role of Free Will (\& the Secret of Public Speaking for Introverts)}

%------------------------------------------------------------------------------%

\section{``Franklin Was a Politician, But Eleanor Spoke Out of Conscience'': Why Cool Is Overrated}

%------------------------------------------------------------------------------%

\section{Why Did Wall Street Crash \& Warren Buffett Prosper?: How Introverts \& Extroverts Think (\& Process Dopamine) Differently}

%------------------------------------------------------------------------------%

\begin{center}\LARGE\sf
	\textbf{Part III: Do All Cultures Have An Extrovert Ideal?}
\end{center}

\section{Soft Power: Asian-Americans \& the Extrovert Ideal}

%------------------------------------------------------------------------------%

\begin{center}\LARGE\sf
	\textbf{Part IV: How to Love, How to Work}
\end{center}

\section{When Should You Act More Extroverted Than You Really Are?}

%------------------------------------------------------------------------------%

\section{The Communication Gap: How to Talk to Members of the Opposite Type}

%------------------------------------------------------------------------------%

\section{On Cobblers \& Generals: How to Cultivate Quiet Kids in a World That Can't Hear Them}

%------------------------------------------------------------------------------%

\section{Conclusion: Wonderland}

%------------------------------------------------------------------------------%

\section{A Note on the Dedication}

%------------------------------------------------------------------------------%

\section{A Note on the Words \textit{Introvert} \& \textit{Extrovert}}

%------------------------------------------------------------------------------%

\printbibliography[heading=bibintoc]
	
\end{document}