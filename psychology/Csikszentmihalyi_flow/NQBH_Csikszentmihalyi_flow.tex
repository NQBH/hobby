\documentclass{article}
\usepackage[backend=biber,natbib=true,style=alphabetic,maxbibnames=50]{biblatex}
\addbibresource{/home/nqbh/reference/bib.bib}
\usepackage{tocloft}
\renewcommand{\cftsecleader}{\cftdotfill{\cftdotsep}}
\usepackage[colorlinks=true,linkcolor=blue,urlcolor=red,citecolor=magenta]{hyperref}
\usepackage{algorithm,algpseudocode,amsmath,amssymb,amsthm,float,graphicx,mathtools}
\allowdisplaybreaks
\numberwithin{equation}{section}
\newtheorem{assumption}{Assumption}[section]
\newtheorem{conjecture}{Conjecture}[section]
\newtheorem{corollary}{Corollary}[section]
\newtheorem{definition}{Definition}[section]
\newtheorem{example}{Example}[section]
\newtheorem{lemma}{Lemma}[section]
\newtheorem{notation}{Notation}[section]
\newtheorem{principle}{Principle}[section]
\newtheorem{problem}{Problem}[section]
\newtheorem{proposition}{Proposition}[section]
\newtheorem{question}{Question}[section]
\newtheorem{remark}{Remark}[section]
\newtheorem{theorem}{Theorem}[section]
\usepackage[left=1cm,right=1cm,top=5mm,bottom=5mm,footskip=4mm]{geometry}
\def\labelitemii{$\circ$}

\title{Flow: The Psychology of Optimal Experience}
\author{Mihaly Csikszentmihalyi}
\date{\today}

\begin{document}
\maketitle
\tableofcontents

%------------------------------------------------------------------------------%

\section{Quotes}
See \cite{Csikszentmihalyi2008}.
\begin{enumerate}
	\item ``Control of consciousness determines the quality of life.''
	\item ``To overcome the anxieties \& depressions of contemporary life, individuals must become independent of the social environment to the degree that they no longer respond exclusively in terms of its rewards \& punishments. To achieve such autonomy, a person has to learn to provide rewards to herself. She has to develop the ability to find enjoyment \& purpose regardless of external circumstances.''
	\item ``Most enjoyable activities are not natural; they demand an effort that initially one is reluctant to make. But once the interaction starts to provide feedback to the person's skills, it usually begins to be intrinsically rewarding.''
	\item ``The best moments in our lives, are not the passive, receptive, relaxing times -- although such experiences can also be enjoyable, if we have worked hard to attain them. The best moments usually occur when a person's body or mind is stretched to its limits in a voluntary effort to accomplish something difficult \& worthwhile.
	
	Optimal experience is thus something that we make happen. For a child, it could be placing with trembling fingers the last block on a tower she has built, higher than any she has built so far; for a swimmer, it could be trying to beat his own record; fora violinist, mastering an intricate musical passage. For each person there are thousands of opportunities, challenges to expand ourselves.''
	\item ``$\ldots$ It is when we act freely, for the sake of the action itself rather than for ulterior motives, that we learn to become more than what we were.''
	\item ``Of all the virtues we can learn no trait is more useful, more essential for survival, \& more likely to improve the quality of life than the ability to transform adversity into an enjoyable challenge.''
	\item ``Few things are sadder than encountering a person who knows exactly what he should do, yet cannot muster enough energy to do it. ``He who desires but acts not,'' wrote Blake with his accustomed vigor, ``Breeds pestilence.''''
	\item ``$\ldots$ success, like happiness, cannot be pursued; it must ensue $\ldots$ as the unintended side-effect of one's personal dedication to a course greater than oneself.''
	\item ``The mystique of rock climbing is climbing; you get to the top of a rock glad it's over but really wish it would go on forever. The justification of climbing is climbing, like the justification of poetry is writing; you don't conquer anything except things in yourself $\ldots$ The act of writing justifies poetry. Climbing is the same: recognizing that you are a flow. The purpose of the flow is to keep on flowing, not looking for a peak or utopia but staying in the flow. It is not a moving up but a continuous flowing; you move up to keep the flow going. There is no possible reason for climbing except the climbing itself; it is a self-communication.''
	\item ``On the job people feel skillful \& challenged, \& therefore feel more happy, strong, creative, \& satisfied. In their free time people feel that there is generally not much to do \& their skills are not being used, \& therefore they tend to feel more sad, weak, dull, \& dissatisfied. Yet they would like to work less \& spend more time in leisure.
	
	What does this contradictory pattern mean? There are several possible explanations, but 1 conclusion seems inevitable: when it comes to work, people do not heed the evidence of their senses. They disregard the quality of immediate experience, \& base their motivation instead on the strongly rooted cultural stereotype of what work is supposed to be like. They think of it as an imposition, a constraint, an infringement of their freedom, \& therefore something to be avoided as much as possible.''
	\item ``People who learn to control inner experience will be able to determine the quality of their lives, which is as close as any of us can come to being happy.''
	\item ``The psychic entropy peculiar to the human condition involves seeing more to do than one can actually accomplish \& feeling able to accomplish more than what conditions allow.''
	\item ``Few things are sadder than encountering a person who knows exactly he should do, yet cannot muster enough energy to do it.''
	\item ``It is not the skills we actually have that determine how we feel but the ones we think we have.''
	\item ``Attention is like energy in that without it no work can be done, \& in doing work is dissipated. We create ourselves by how we use this energy. Memories, thoughts \& feelings are all shaped by how use it. \& it is an energy under control, to do with as we please; hence attention is our most important tool in the task of improving the quality of experience.''
	\item ``It is when we act freely, for the sake of the action itself rather than for ulterior motives, that we learn to become more than what we were. When we choose a goal \& invest ourselves in it to the limits of concentration, whatever we do will be enjoyable. \& once we have tasted this joy, we will redouble our efforts to taste it again. This is the way the self grows.''
	\item ``it's a wise parent who allows her children to give up the things of childhood in their own time.''
	\item ``The universe is not hostile, nor yet is it friendly,'' in the words of J. H. Holmes. ``It is simply indifferent.''
	\item ``Control over consciousness is not simply a cognitive skill. At least as much as intelligence, it requires the commitment of emotions \& will. It is not enough to know how to do it; one must do it, consistently, in the same way as athletes or musicians who must keep practicing what they know in theory.''
	\item ``It's exhilarating to come closer \& closer to self-discipline.''
	\item ``writing gives the mind a disciplined means of expression.''
	\item ``Most of us become so rigidly fixed in the ruts carved out by genetic programming \& social conditioning that we ignore the options of choosing any other course of action. Living exclusively by genetic \& social instructions is fine as long as everything goes well. But the moment biological or social goals are frustrated -- which in the long run is inevitable -- a person must formulate new goals, \& create a new flow activity for himself, or else he will always waste his energies in inner turmoil.''
	\item ``A person can make himself happy, or miserable, regardless of what is actually happening ``outside,'' just by changing the contents of consciousness. We all know individuals who can transform hopeless situations into challenges to be overcome, just through the force of their personalities. This ability to persevere despite obstacles \& setbacks is the quality people most admire in others, \& justly so; it is probably the most important trait not only for succeeding in life, but for enjoying it as well.''
	\item ``A person who has achieved control over psychic energy \& has invested it in consciously chosen goals cannot help but grow into a more complex being. By stretching skills, by reaching toward higher challenges, such a person becomes an increasingly extraordinary individual.''
\end{enumerate}

%------------------------------------------------------------------------------%

\printbibliography[heading=bibintoc]
	
\end{document}