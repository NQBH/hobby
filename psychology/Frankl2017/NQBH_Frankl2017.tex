\documentclass{article}
\usepackage[backend=biber,natbib=true,style=authoryear]{biblatex}
\addbibresource{/home/nqbh/reference/bib.bib}
\usepackage{tocloft}
\renewcommand{\cftsecleader}{\cftdotfill{\cftdotsep}}
\usepackage[colorlinks=true,linkcolor=blue,urlcolor=red,citecolor=magenta]{hyperref}
\usepackage{algorithm,algpseudocode,amsmath,amssymb,amsthm,float,graphicx,mathtools}
\allowdisplaybreaks
\numberwithin{equation}{section}
\newtheorem{assumption}{Assumption}[section]
\newtheorem{conjecture}{Conjecture}[section]
\newtheorem{corollary}{Corollary}[section]
\newtheorem{definition}{Definition}[section]
\newtheorem{example}{Example}[section]
\newtheorem{lemma}{Lemma}[section]
\newtheorem{notation}{Notation}[section]
\newtheorem{principle}{Principle}[section]
\newtheorem{problem}{Problem}[section]
\newtheorem{proposition}{Proposition}[section]
\newtheorem{question}{Question}[section]
\newtheorem{remark}{Remark}[section]
\newtheorem{theorem}{Theorem}[section]
\usepackage[left=0.5in,right=0.5in,top=1.5cm,bottom=1.5cm]{geometry}
\usepackage{fancyhdr}
\pagestyle{fancy}
\fancyhf{}
\lhead{\small Sect.~\thesection}
\rhead{\small\nouppercase{\leftmark}}
\renewcommand{\sectionmark}[1]{\markboth{#1}{}}
\cfoot{\thepage}
\def\labelitemii{$\circ$}

\title{Man's Search for Meaning}
\author{Viktor Emil Frankl}
\date{\today}

\begin{document}
\maketitle
\tableofcontents
\vspace{5mm}

%------------------------------------------------------------------------------%

\section*{Foreword by John Boyne}
``When I was 15 years old, my English teacher came into school on the last day before our summer holidays \& read out a list of about 40 books. ``I want you to read some of these over the break,'' he told us. ``I won't test you on them afterwards. I won't ask you what you read or tell you to write any book reports. But each one is a powerful work of literature that has meant a lot to me over the years, \& if you give them a chance, if you let these stories \& voices into your lives, then they might make you view the world in a different way. \& they might change you.''

I was unfamiliar with most of the titles, for although this was a point in my development where I was beginning to discover adult literature in a more serious way, I was new to almost all of it. However, being a teenager obsessed with both reading \& writing, the list excited me, \& I took to my bike \& cycled to my local bookshop the next morning ready to begin. It took a long time to decide where this reading journey should start, but eventually I settled on Primo Levi's \textit{If This Is a Man}.

The book had a profound effect on me. It may seem strange now, but growing up in Ireland in the mid-1980s, we didn't study the Holocaust in school, \& so, outside of war films that I'd seen on television \& the occasional World War II novel aimed at younger readers that I'd borrowed from the library, this was my introduction to a subject that would grow to fascinate \& horrify me in equal parts over the years to come. A subject that would become an intrinsic part of my own story, even though I had been born more than a quarter century after the last of the death camps had been liberated.

As often happens with reading, 1 book that summer led me to another, \& another after that, \& soon I abandoned my teacher's list \& allowed the writers \& the stories to dictate where I should go next. I found myself working my way through much of Primo Levi's autobiographical writings, as well as the wroks of Anne Frank \& Elie Wiesel, alongside some history books \&, memorably, a biography of Hitler. \& then, just as the summer was drawing to a close \& I was being measured up for the next year's school uniform, I discovered \textit{Man's Search for Meaning} by Viktor E. Frankl, a book whose title intimidated me but that felt like the natural continuation of my education as I attempted to understand the most terrible period of history, a time when it seemed that the human race displayed just how cruel it could be.

Frankl's book, like Levi's, like Wiesel's, like Anne Frank's, affected me greatly, both when I read it in 1986 \& again, 30 years later, when I re-read it in preparation for writing this foreword. There's a clarity to his analysis of how prisoners felt in this most dehumanizing of experiences that is both surprising in its lack of bitterness \& illuminating in its lucidity: The feelings of shock, dismay, \& fear that victims felt having been pulled from their homes \& brought to unfamiliar \& frightening places. The cycle that the mind must have gone through as it adjusted to living within the electrified fences of death. The basic human urge that must have forced a person to do all he or she could simply to stay alive. \& then the effect on the devastated psyche for those victims who survived the experience \& found themselves liberated into a changed world, who would take decades to come to terms with the things that had taken place.

Frankl even makes us wonder about the word \textit{survived}. Did anyone actually \textit{survive} the camps? I'm not so sure that they did, the experience proving so overwhelming, the memories so brutal, the grief for lost loved ones so intense that it must have weighed on the consciousness like an insupportable burden.

Each memoir written by a victim of the camps is different \& surprising, but for me, 1 of the most unexpected elements of Frankl's book is his central belief that every moment of our existence -- happy, sad, generous, or cruel -- gives us our experience of life \& should be accepted as such. There's a stoicism to this concept that takes time to understand -- indeed, I think one would have had to experience what Frankl experienced to comprehend it fully -- but it gave the author the courage to survive on a daily basis \& eventually to use his experiences to join that extraordinary group of men \& women who were willing to plunge fearlessly into their darkest days to share their experiences with the world without self-pity or acrimony but with nothing more than a clear desire that we should understand, \& through understanding, prevent such things from taking place again.

Frankl was ahead of his time in his controversial belief that freedom of choice remains an inherent part of the captive's life, that under the most extreme of situations, to lose hope is to surrender entirely to the darkness. His analysis of the different types of person -- whether guards or prisoners -- is also startling, \& there is real daring in his clinical examination of both that must have proved shocking \& perhaps even unpleasant to some when the memoir was 1st published in 1946.

For me, however, the most fascinating element of the book is his exploration of the life of the victim -- \& I prefer the word \textit{victim} to \textit{prisoner}, for \textit{prisoner} suggests rightful incarceration while \textit{victim} is clear on the innocence of the abused party -- after his or her liberation. It's impossible for someone of my generation \& background to imagine the feelings of relief, confusion, \& anger that must have jostled for dominion in the mind. Impossible for any of us, anymore, I think. The emotions \& the personality must have been so shattered that to take pleasure in any part of life afterward would have proved a substantial challenge. When you have seen the worst of human behavior, after all, how do you continue to live among people? This was something that each victim had to come to terms with \& perhaps some were more successful than others. For Frankl, there was only 1 way to hope: \textit{write about it}.

Viktor Frankl died when I was just a child, but as a novelist who has made an attempt at understanding the Holocaust through my own fiction, I have been fortunate to be part of that last group of writers to have had the privilege of meeting survivors of the camps through my travels over the years. The single most humbling experience of my professional lief has been standing in community centers, halls, theaters, \& on festival stages around the world while audience members have risen to recount their own memories. Talking to them afterward, feeling honored to be in their presence \& a little unworthy to have written about a subject using only my imagination when their true stories are so much more powerful \& authentic than my own, is something I will always cherish. \& in those conversations the name of Viktor Frankl \& \textit{Man's Search for Meaning} has come up time \& again, alongside those other classic works that help keep the memories of that time alive. \& it always will. That is his legacy.

Why do we keep writing about it? is a question that comes up time \& again. I think it's because although all powerful works of fiction, nonfiction, \& memoir inspire discussion, passion, \& even criticism, the joy of literature, as opposed too often to the practice of politics or religion, is that it embraces differing opinions; it encourages debate, \& it allows us to have heated conversations with our closest friends \& dearest loved ones. \& through it all no one gets hurt, no one gets taken away from their homes, \& no one gets killed. This is something that Viktor Frankl must have understood, for \textit{Man's Search for Meaning} is such a book. One to read, to cherish, to debate, \& one that will ultimately keep the memories of the victims alive.

John Boyne\footnote{John Boyne is the author of several notes for adults \& 5 novels for younger readers, including \textit{The Boy in the Striped Pajamas}.}'' -- \cite[pp. 6--8]{Frankl2017}

%------------------------------------------------------------------------------%

\section*{Preface to the 1992 Edition}
``This book has now lived to see nearly 100 printings in English -- in addition to having been published in 21 other languages. \& the English editions alone have sold more than 3 million copies.

These are the dry facts, \& they may well be the reason why reporters of American newspapers \& particularly of American TV stations more often than not start their interviews, after listing these facts, by exclaiming: ``Dr. Frankl, your book has become a true bestseller -- how do you feel about such a success?'' Whereupon I react by reporting that in the 1st place I do not at all see in the bestseller status of my book an achievement \& accomplishment on my part but rather an expression of the misery of our time: if hundreds of thousands of people reach out for a book whose very title promises to deal with the question of a meaning to life, it must be a question that burns under their fingernails.

To be sure, something else may have contributed to the impact of the book: its 2nd, theoretical part (``Logotherapy in a Nutshell'') boils down, as it were, to the lesson one may distill from the 1st part, the autobiographical account (``Experiences in a Concentration Camp''), whereas Part I serves as the existential validation of my theories. Thus, both parts mutually support their credibility.

I had none of this in mind when I wrote the book in 1945. \& I did so within 9 successive days \& with the firm determination that the book should be published anonymously. In fact, the 1st printing of the original German version does not show my name on the cover, though at the last moment, just before the book's initial publication, I did finally give in to my friends who had urged me to let it be published with my name at least on the title page. At 1st, however, it had been written with the absolute conviction that, as an anonymous opus, it could never earn its author literary fame. I had wanted simply to convey to the reader by way of a concrete example that life holds a potential meaning under any conditions, even the most miserable ones. \& I though that if the point were demonstrated in a situation as extreme as that in a concentration camp, my book might gain a hearing. I therefore felt responsible for writing down what I had gone through, for I thought it might be helpful to people who are prone to despair.

\& so it is both strange \& remarkable to me that -- among some dozens of books I have authored -- precisely this one, which I had intended to be published anonymously so that it could never built up any reputation on the part of the author, did become a success. Again \& again I therefore admonish my students both in Europe \& in America: ``Don't aim at success -- the more you aim at it \& make it a target, the more you are going to miss it. For success, like happiness, cannot be pursued; it must ensue, \& it only does so as the unintended side-effect of one's dedication to cause a greater than oneself or as the by-product of one's surrender to a person other than oneself. Happiness must happen, \& the same holds for success: you have to let it happen by not caring about it. I want you to listen to what your conscience commands you to do \& go on to carry it out to the best of your knowledge. Then you will live to see that in the long run -- in the long run, I say! -- success will follow you precisely because you had \textit{forgotten} to think of it.''

The reader may ask me why I did not try to escape what was in store for me after Hitler had occupied Austria. Let me answer by recalling the following story. Shortly before the United States entered World War II, I received an invitation to come to the American Consulate in Vienna to pick up my immigration visa. My old parents were overjoyed because they expected that I would soon be allowed to leave Austria. I suddenly hesitated, however. The question beset me: could I really afford to leave my parents alone to face their fate, to be sent, sooner or later, to a concentration camp, or even to a so-called extermination camp? Where did my responsibility lie? Should I foster my brain child, logotherapy, by emigrating to fertile soil where I could write my books? Or should I concentrate on my duties as a real child, the child of my parents who had to do whatever he could to protect them? I pondered the problem this way \& that but could not arrive at a solution; this was the type of dilemma that made one wish for ``a hint from Heaven,'' as the phrase goes.

It was then that I noticed a piece of marble lying on a table at home. When I asked my father about it, he explained that he had found it on the site where the National Socialists had burned down the largest Viennese synagogue. He had taken the piece home because it was a part of the tablets on which the 10 Commandments were inscribed. One gilded Hebrew letter was engraved on the piece; my father explained that this letter stood for 1 of the Commandments. Eagerly I asked, ``Which one is it?'' He answered, ``Honor thy father \& thy mother that thy days may be long upon the land.'' At that moment I decided to stay with my father \& my mother upon the land, \& to let the American visa lapse.

-- Viktor E. Frankl, \textit{Vienna, 1992}'' -- \cite[pp. 10--11]{Frankl2017}

%------------------------------------------------------------------------------%

\section{Experiences in a Concentration Camp}

%------------------------------------------------------------------------------%

\section{Logotherapy in a Nutshell (Abridged)}

%------------------------------------------------------------------------------%

\section{Afterword by William J. Winslate}
``On Jan 27, 2006, the 61st anniversary of the liberation of the Auschwitz death camp, where 1.5 million people died, nations around the world observed the 1st International Holocaust Remembrance Day. A few months later, they might well have celebrated the anniversary of 1 of the most abiding pieces of writing from that horrendous time. 1st published in German in 1946 as \textit{A Psychologist Experiences the Concentration Camp} \& later called \textit{Say Yes to Life in Spite of Everything}, subsequent editions were supplemented by an introduction to logotherapy \& a postscript on tragic optimism, or how to remain optimistic in the face of pain, guilt, \& death. The English translation, 1st published in 1959, was called \textit{Man's Search for Meaning}.

Viktor Frankl's book has now sold more than 12 million copies in a total of 24 languages. A 1991 Library of Congress\texttt{/}Book-of-the-Month-Club survey asking readers to name a ``book that made a difference in your life'' found \textit{Man's Search for Meaning} among the 10 most influential books in America. It has inspired religious \& philosophical thinkers, mental health professionals, teachers, students, \& general readers from all walks of life. It is routinely assigned to college, graduate, \& high school students in psychology, philosophy, history, literature, Holocaust studies, religion, \& theology. What accounts for its pervasive influence \& enduring value?

Viktor Frankl's life spanned nearly all of the 20th century, from his birth in 1905 to his death in 1997. At the age of 3 he decided to become a physician. In his autobiographical reflections, he recalls that s a youth he would ``think for some minutes about the meaning of life. Particularly about the meaning of the coming day \& its meaning for \textit{me}.''

As a teenager Frankl was fascinated by philosophy, experimental psychology, \& psychoanalysis. To supplement his high school classes, he attended adult-education classes \& began a correspondence with Sigmund Freud that led Freud to submit a manuscript of Frankl's to the \textit{International Journal of Psychoanalysis}. The article was accepted \& later published. That same year, at age 16, Frankl attended an adult-education workshop on philosophy. The instructor, recognizing Frankl's precocious intellect, invited him to give a lecture on the meaning of life. Frankl told the audience that ``It is we ourselves who must answer the questions that life asks of us, \& to these questions we can respond only by being responsible for our existence.'' This belief became the cornerstone of Frankl's personal life \& professional identity.

Under the influence of Freud's ideas, Frankl decided while he was still in high school to become a psychiatrist. Inspired in part by a fellow student who told him he had a gift for helping others, Frankl had begun to realize that he had a talent not only for diagnosing psychological problems, but also for discovering what motivates people.

Frankl's 1st counseling job was entirely his own -- he founded Vienna's 1st private youth counseling program \& worked with troubled youths. From 1930--1937 he worked as a psychiatrist at the University Clinic in Vienna, caring for suicidal patients. He sought to help his patients find a way to make their lives meaningful even in the face of depression or mental illness. By 1939 he was head of the department of neurology at Rothschild Hospital, the only Jewish hospital in Vienna.

In the early years of the war, Frankl's work at Rothschild gave him \& his family some degree of protection from the threat of deportation. When the hospital was closed down by the National Sociolist government, however, Frankl realized that they were at grave risk of being sent to a concentration camp. In 1942 the American consulate in Vienna informed him that he was eligible for a U.S. immigration visa. Although an escape from Austria would have enabled him to complete his book on logotherapy, he decided to let his visa lapse: he felt he should stay in Vienna for the sake of his aging parents. In Sep 1942, Frankl \& his family were arrested \& deported. Frankl spent the next 3 years at 4 different concentration camps -- Theresienstadt, Auschwitz-Birkenau, Kaufering, and T\"urkheim, part of the Dachau complex.

It is important to note that Frankl's imprisonment was not the only impetus for \textit{Man's Search for Meaning}. Before his deportation, he had already begun to formulate an argument that the quest for meaning is the key to mental health \& human flourishing. As a prisoner, he was suddenly forced to assess whether his own life still had any meaning. His survival was a combined result of his will to live, his instinct for self-preservation, some generous acts of human decency, \& shrewdness; of course, it also depended on blind luck, such as where he happened to be imprisoned, the whims of the guards, \& arbitrary decisions about where to line up \& who to trust or believe. However, something more was needed to overcome the deprivations \& degradations of the camps. Frankl drew constantly upon uniquely human capacities such as inborn optimism, humor, psychological detachment, brief moments of solitude, inner freedom, \& a steely resolve not to give up or commit suicide. He realized that he must try to live for the future, \& he drew strength from loving thoughts of his wife \& his deep desire to finish his book on logotherapy. He also found meaning in glimpses of beauty in nature \& art. Most important, he realized that, no matter what happened, he retained the freedom to choose how to respond to his suffering. He saw this not merely as an option but as his \& every person's responsibility to choose ``the way in which he bears his burden.''

Sometimes Frankl's ideas are inspirational, as when he explains how dying patients \& quadriplegics come to terms with their fate. Others are aspirational, as when he asserts that a person finds meaning by ``striving \& struggling for a worthwhile goal, a freely chosen task.'' He shows how existential frustration provoked \& motivated an unhappy diplomat to seek a new, more satisfying career. Frankl also uses moral exhortation, however, to call attention to ``the gap between what one is \& what one should become'' \& the idea that ``man is responsible \& must actualize the potential meaning of his life.'' He sees freedom \& responsibility as 2 sides of the same coin. When he spoke to American audiences, Frankl was fond of saying, ``I recommend that the Statue of Liberty on the East Coast be supplemented by a Statue of Responsibility on the West Coast.'' To achieve personal meaning, he says, one must transcend subjective pleasures by doing something that ``points, \& is directed, to something, or someone, other than oneself $\ldots$ by giving himself to a cause to serve or another person to love.'' Frankl himself chose to focus on his parents by staying in Vienna when he could have had safe passage to America. While he was in the same concentration camp as his father, Frankl managed to obtain morphine to ease his father's pain \& stayed by his side during his dying days.

Even when confronted by loss \& sadness, Frankl's optimism, his constant affirmation of \& exuberance about life, led him to insist that hope \& positive energy can turn challenges into triumphs. In \textit{Man's Search for Meaning}, he hastens to add that suffering is not \textit{necessary} to find meaning, only that ``meaning is possible in spite of suffering.'' Indeed, he goes on to say that ``to suffer unnecessary is masochistic rather than heroic.''

I 1st read \textit{Man's Search for Meaning} as a philosophy professor in the mid-1960s. The book was brought to my attention by a Norwegian philosopher who had himself been incarcerated in a Nazi concentration camp. My colleague remarked how strongly he agreed with Frankl about the importance of nourishing one's inner freedom, embracing the value of beauty in nature, art, poetry, \& literature, \& feeling love for family \& friends. But other personal choices, activities, relationships, hobbies, \& even simple pleasures can also give meaning to life. Why, then, do some people find themselves feeling so empty? Frankl's wisdom here is worth emphasizing: it is a question of the \textit{attitude} one takes toward life's challenges \& opportunities, both large \& small. A positive attitude enables a person to endure suffering \& disappointment as well as enhance enjoyment \& satisfaction. A negative attitude intensifies pain \& deepens disappointments; it undermines \& diminishes pleasure, happiness, \& satisfaction; it may even lead to depression or physical illness.

My friend \& former colleague Norman Cousins was a tireless advocate for the value of positive emotions in promoting health, \& he warned of the danger that negative emotions may jeopardize it. Although some critics attacked Cousins's views as simplistic, subsequent research in psychoneuroimmunology has supported the ways in which positive emotions, expectations, \& attitudes enhance our immune system. This research also reinforces Frankl's belief that one's approach to everything from life-threatening challenges to everyday situations helps to share the meaning of our lives. The simple truth that Frankl so ardently promoted has profound significance for anyone who listens.

The choices humans make should be active rather than passive. In making personal choices we affirm our autonomy. ``A human being is not 1 thing among others; \textit{things} determine each other,'' Frankl writes, ``but \textit{man} is ultimately self determining. What he becomes -- within the limits of endowment \& environment -- he has made out of himself.'' E.g., the darkness of despair threatened to overwhelm a young Israeli soldier who had lost both his legs in the Yom Kippur War. He was drowning in depression \& contemplating suicide. 1 day a friend noticed that his outlook had changed to hopeful serenity. The soldier attributed his transformation to reading \textit{Man's Search for Meaning}. When he was  told about the soldier, Frankl wondered whether ``there may be such a thing as autobibliotherapy -- healing through reading.''

Frankl's comment hints at the reasons why \textit{Man's Search for Meaning} has such a powerful impact on many readers. Persons facing existential challenges or crises may seek advice or guidance from family, friends, therapists, or religious counselors. Sometimes such advice is helpful; sometimes it is not. Persons facing difficult choices may not fully appreciate how much their own attitude interferes with the decision they need to make or the action they need to take. Frankl offers readers who are searching for answers to life's dilemmas a critical mandate: he does not tell people \textit{what} to do, but \textit{why} they must do it.

After his liberation in 1945 from the T\"urkheim camp, where he had nearly died of typhus, Frankl discovered that he was utterly alone. On the 1st day of his return to Vienna in Aug 1945, Frankl learned that his pregnant wife, Tilly, had died of sickness or starvation in the Bergen-Belsen concentration camp. Sadly, his parents \& brother had all died in the camps. Overcoming his losses \& inevitable depression, he remained in Vienna to resume his career as a psychiatrist -- an unusual choice when so many others, especially Jewish psychoanalysts \& psychiatrists, had emigrated to other countries. Several factors may have contributed to this decision: Frankl felt an intense connection to Vienna, especially to psychiatric patients who needed his help in the postwar period. He also believed strongly in reconciliation rather than revenge; he once remarked, ``I do not forget any good deed done to me, \& I do not carry a grudge for a bad one.'' Notably, he renounced the idea of collective guilt. Frankl was able to accept that his Viennese colleagues \& neighbors may have known about or even participated in his persecution, \& he did not condemn them for failing to join the resistance or die heroic deaths. Instead, he was deeply committed to the idea that even a vile Nazi criminal or a seemingly hopeless madman has the potential to transcend evil or insanity by making responsible choices.

He threw himself passionately into his work. In 1946 he reconstructed \& revised the book that was destroyed when he was 1st deported (The \textit{Doctor \& the Soul}), \& that same year -- in only 9 days -- he wrote \textit{Man's Search for Meaning}. He hoped to cure through his writings the personal alienation \& culture malaise that plagued many individuals who felt an ``inner emptiness'' or a ``void within themselves.'' Perhaps this flurry of professional activity helped Frankl to restore meaning to his own life.

2 years later he married Eleonore Schwindt, who, like his 1st wife, was a nurse. Unlike Tilly, who was Jewish, Elly was Catholic. Although this may have been mere coincidence, it was characteristic of Viktor Frankl to accept individuals regardless of their religious beliefs or secular convictions. His deep commitment to the uniqueness \& dignity of each individual was illustrated by his admiration for Freud \& Adler even though he disagreed with their philosophical \& psychological theories. He also valued his personal relationships with philosophers as radically different as Martin Heidegger, a reformed Nazi sympathizer, Karl Jaspers, an advocate of collective guilt, \& Gabriel Marcel, a Catholic philosopher \& writer. As a psychiatrist, Frankl avoided direct reference to his personal religious beliefs. He was fond of saying that the aim of psychiatry was the healing of the soul, leaving to religion the salvation of the soul.

He remained head of the neurology department at the Vienna Policlinic Hospital for 25 years \& wrote more than 30 books for both professionals \& general readers. He lectured widely in Europe, the Americas, Australia, Asia, \& Africa; held professorships at Harvard, Stanford, \& the University of Pittsburgh; \& was Distinguished Professor of Logotherapy at the U.S. International University in San Diego. He met with politicians, world leaders such as Pope Paul VI, philosophers, students, teachers, \& numerous individuals who had read \& been inspired by his books. Even in his 90s, Frankl continued to engage in dialogue with visitors from all over the world \& to respond personally to some of the hundreds of letters he received every week. 29 universities awarded him honorary degrees, \& the American Psychiatric Association honored him with the Oskar Pfister Award.

Frankl is credited with establishing logotherapy as a psychiatric technique that uses existential analysis to help patients resolve their emotional conflicts. He stimulated many therapists to look beyond patients' past or present problems to help them choose productive futures by making personal choices \& taking responsibility for them. Several generations of therapists were inspired by his humanistic insights, which gained influence as a result of Frankl's prolific writing, provocative lectures, \& engaging personality. He encouraged others to use existential analysis creatively rather than to establish an official doctrine. He argued thta therapists should focus on the specific needs of individual patients rather than extrapolate from abstract theories.

Despite a demanding schedule, Frankl also found time to take flying lessons \& pursue his lifelong passion for mountain climbing. He joked that in contrast to Freud's  \& Adler's ``depth psychology,'' which emphasizes delving into an individual's past \& his or her unconscious instincts \& desires, he practiced ``height psychology,'' which focuses on a person's future \& his or her conscious decisions \& actions. His approach to psychotherapy stressed the importance of helping people to reach new heights of personal meaning through self transcendence: the application of positive effort, technique, acceptance of limitations, \& wise decisions. His goal was to provoke people into realizing that they could \& should exercise their capacity for choice to achieve their own goals. Writing about tragic optimism, he cautioned us that ``the world is in a bad state, but everything will become still worse unless each of us does his best.''

Frankl was once asked to express in 1 sentence the meaning of his own life. He wrote the response on paper \& asked his students to guess what he had written. After some moments of quiet reflection, a student surprised Frankl by saying, ``The meaning of your life is to help others find the meaning of theirs.'' 

``That was it, exactly,'' Frankl said. ``Those are the very words I had written.'' -- William J. Winslade

William J. Winslade is a philosopher, lawyer, \& psychoanalyst who teaches psychiatry, medical ethics, \& medical jurisprudence at the University of Texas Medical Branch in Galveston \& the University of Houston Law Center.'' -- \cite[pp. 108--116]{Frankl2017}

%------------------------------------------------------------------------------%

\section{Selected Writings}

%------------------------------------------------------------------------------%

\section{Chronology of Viktor Frankl's Life \& of the Holocaust}

%------------------------------------------------------------------------------%

\printbibliography[heading=bibintoc]
	
\end{document}