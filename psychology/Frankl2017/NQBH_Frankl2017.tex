\documentclass{article}
\usepackage[backend=biber,natbib=true,style=authoryear]{biblatex}
\addbibresource{/home/nqbh/reference/bib.bib}
\usepackage{tocloft}
\renewcommand{\cftsecleader}{\cftdotfill{\cftdotsep}}
\usepackage[colorlinks=true,linkcolor=blue,urlcolor=red,citecolor=magenta]{hyperref}
\usepackage{algorithm,algpseudocode,amsmath,amssymb,amsthm,float,graphicx,mathtools}
\allowdisplaybreaks
\numberwithin{equation}{section}
\newtheorem{assumption}{Assumption}[section]
\newtheorem{conjecture}{Conjecture}[section]
\newtheorem{corollary}{Corollary}[section]
\newtheorem{definition}{Definition}[section]
\newtheorem{example}{Example}[section]
\newtheorem{lemma}{Lemma}[section]
\newtheorem{notation}{Notation}[section]
\newtheorem{principle}{Principle}[section]
\newtheorem{problem}{Problem}[section]
\newtheorem{proposition}{Proposition}[section]
\newtheorem{question}{Question}[section]
\newtheorem{remark}{Remark}[section]
\newtheorem{theorem}{Theorem}[section]
\usepackage[left=0.5in,right=0.5in,top=1.5cm,bottom=1.5cm]{geometry}
\usepackage{fancyhdr}
\pagestyle{fancy}
\fancyhf{}
\lhead{\small Sect.~\thesection}
\rhead{\small\nouppercase{\leftmark}}
\renewcommand{\sectionmark}[1]{\markboth{#1}{}}
\cfoot{\thepage}
\def\labelitemii{$\circ$}

\title{Man's Search for Meaning}
\author{Viktor Emil Frankl}
\date{\today}

\begin{document}
\maketitle
\tableofcontents
\vspace{5mm}

%------------------------------------------------------------------------------%

\section*{Foreword by John Boyne}
``When I was 15 years old, my English teacher came into school on the last day before our summer holidays \& read out a list of about 40 books. ``I want you to read some of these over the break,'' he told us. ``I won't test you on them afterwards. I won't ask you what you read or tell you to write any book reports. But each one is a powerful work of literature that has meant a lot to me over the years, \& if you give them a chance, if you let these stories \& voices into your lives, then they might make you view the world in a different way. \& they might change you.''

I was unfamiliar with most of the titles, for although this was a point in my development where I was beginning to discover adult literature in a more serious way, I was new to almost all of it. However, being a teenager obsessed with both reading \& writing, the list excited me, \& I took to my bike \& cycled to my local bookshop the next morning ready to begin. It took a long time to decide where this reading journey should start, but eventually I settled on Primo Levi's \textit{If This Is a Man}.

The book had a profound effect on me. It may seem strange now, but growing up in Ireland in the mid-1980s, we didn't study the Holocaust in school, \& so, outside of war films that I'd seen on television \& the occasional World War II novel aimed at younger readers that I'd borrowed from the library, this was my introduction to a subject that would grow to fascinate \& horrify me in equal parts over the years to come. A subject that would become an intrinsic part of my own story, even though I had been born more than a quarter century after the last of the death camps had been liberated.

As often happens with reading, 1 book that summer led me to another, \& another after that, \& soon I abandoned my teacher's list \& allowed the writers \& the stories to dictate where I should go next. I found myself working my way through much of Primo Levi's autobiographical writings, as well as the wroks of Anne Frank \& Elie Wiesel, alongside some history books \&, memorably, a biography of Hitler. \& then, just as the summer was drawing to a close \& I was being measured up for the next year's school uniform, I discovered \textit{Man's Search for Meaning} by Viktor E. Frankl, a book whose title intimidated me but that felt like the natural continuation of my education as I attempted to understand the most terrible period of history, a time when it seemed that the human race displayed just how cruel it could be.

Frankl's book, like Levi's, like Wiesel's, like Anne Frank's, affected me greatly, both when I read it in 1986 \& again, 30 years later, when I re-read it in preparation for writing this foreword. There's a clarity to his analysis of how prisoners felt in this most dehumanizing of experiences that is both surprising in its lack of bitterness \& illuminating in its lucidity: The feelings of shock, dismay, \& fear that victims felt having been pulled from their homes \& brought to unfamiliar \& frightening places. The cycle that the mind must have gone through as it adjusted to living within the electrified fences of death. The basic human urge that must have forced a person to do all he or she could simply to stay alive. \& then the effect on the devastated psyche for those victims who survived the experience \& found themselves liberated into a changed world, who would take decades to come to terms with the things that had taken place.

Frankl even makes us wonder about the word \textit{survived}. Did anyone actually \textit{survive} the camps? I'm not so sure that they did, the experience proving so overwhelming, the memories so brutal, the grief for lost loved ones so intense that it must have weighed on the consciousness like an insupportable burden.

Each memoir written by a victim of the camps is different \& surprising, but for me, 1 of the most unexpected elements of Frankl's book is his central belief that every moment of our existence -- happy, sad, generous, or cruel -- gives us our experience of life \& should be accepted as such. There's a stoicism to this concept that takes time to understand -- indeed, I think one would have had to experience what Frankl experienced to comprehend it fully -- but it gave the author the courage to survive on a daily basis \& eventually to use his experiences to join that extraordinary group of men \& women who were willing to plunge fearlessly into their darkest days to share their experiences with the world without self-pity or acrimony but with nothing more than a clear desire that we should understand, \& through understanding, prevent such things from taking place again.

Frankl was ahead of his time in his controversial belief that freedom of choice remains an inherent part of the captive's life, that under the most extreme of situations, to lose hope is to surrender entirely to the darkness. His analysis of the different types of person -- whether guards or prisoners -- is also startling, \& there is real daring in his clinical examination of both that must have proved shocking \& perhaps even unpleasant to some when the memoir was 1st published in 1946.

For me, however, the most fascinating element of the book is his exploration of the life of the victim -- \& I prefer the word \textit{victim} to \textit{prisoner}, for \textit{prisoner} suggests rightful incarceration while \textit{victim} is clear on the innocence of the abused party -- after his or her liberation. It's impossible for someone of my generation \& background to imagine the feelings of relief, confusion, \& anger that must have jostled for dominion in the mind. Impossible for any of us, anymore, I think. The emotions \& the personality must have been so shattered that to take pleasure in any part of life afterward would have proved a substantial challenge. When you have seen the worst of human behavior, after all, how do you continue to live among people? This was something that each victim had to come to terms with \& perhaps some were more successful than others. For Frankl, there was only 1 way to hope: \textit{write about it}.

Viktor Frankl died when I was just a child, but as a novelist who has made an attempt at understanding the Holocaust through my own fiction, I have been fortunate to be part of that last group of writers to have had the privilege of meeting survivors of the camps through my travels over the years. The single most humbling experience of my professional lief has been standing in community centers, halls, theaters, \& on festival stages around the world while audience members have risen to recount their own memories. Talking to them afterward, feeling honored to be in their presence \& a little unworthy to have written about a subject using only my imagination when their true stories are so much more powerful \& authentic than my own, is something I will always cherish. \& in those conversations the name of Viktor Frankl \& \textit{Man's Search for Meaning} has come up time \& again, alongside those other classic works that help keep the memories of that time alive. \& it always will. That is his legacy.

Why do we keep writing about it? is a question that comes up time \& again. I think it's because although all powerful works of fiction, nonfiction, \& memoir inspire discussion, passion, \& even criticism, the joy of literature, as opposed too often to the practice of politics or religion, is that it embraces differing opinions; it encourages debate, \& it allows us to have heated conversations with our closest friends \& dearest loved ones. \& through it all no one gets hurt, no one gets taken away from their homes, \& no one gets killed. This is something that Viktor Frankl must have understood, for \textit{Man's Search for Meaning} is such a book. One to read, to cherish, to debate, \& one that will ultimately keep the memories of the victims alive.

John Boyne\footnote{John Boyne is the author of several notes for adults \& 5 novels for younger readers, including \textit{The Boy in the Striped Pajamas}.}'' -- \cite[pp. 6--8]{Frankl2017}

%------------------------------------------------------------------------------%

\section*{Preface to the 1992 Edition}

%------------------------------------------------------------------------------%

\section{Experiences in a Concentration Camp}

%------------------------------------------------------------------------------%

\section{Logotherapy in a Nutshell (Abridged)}

%------------------------------------------------------------------------------%

\section{Afterword by William J. Winslate}

%------------------------------------------------------------------------------%

\section{Selected Writings}

%------------------------------------------------------------------------------%

\section{Chronology of Viktor Frankl's Life \& of the Holocaust}

%------------------------------------------------------------------------------%

\printbibliography[heading=bibintoc]
	
\end{document}