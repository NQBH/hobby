\documentclass{article}
\usepackage[backend=biber,natbib=true,style=authoryear]{biblatex}
\addbibresource{/home/nqbh/reference/bib.bib}
\usepackage{tocloft}
\renewcommand{\cftsecleader}{\cftdotfill{\cftdotsep}}
\usepackage[colorlinks=true,linkcolor=blue,urlcolor=red,citecolor=magenta]{hyperref}
\usepackage{algorithm,algpseudocode,amsmath,amssymb,amsthm,float,graphicx,mathtools}
\allowdisplaybreaks
\numberwithin{equation}{section}
\newtheorem{assumption}{Assumption}[section]
\newtheorem{conjecture}{Conjecture}[section]
\newtheorem{corollary}{Corollary}[section]
\newtheorem{definition}{Definition}[section]
\newtheorem{example}{Example}[section]
\newtheorem{lemma}{Lemma}[section]
\newtheorem{notation}{Notation}[section]
\newtheorem{principle}{Principle}[section]
\newtheorem{problem}{Problem}[section]
\newtheorem{proposition}{Proposition}[section]
\newtheorem{question}{Question}[section]
\newtheorem{remark}{Remark}[section]
\newtheorem{theorem}{Theorem}[section]
\usepackage[left=0.5in,right=0.5in,top=1.5cm,bottom=1.5cm]{geometry}
\usepackage{fancyhdr}
\pagestyle{fancy}
\fancyhf{}
\lhead{\small Sect.~\thesection}
\rhead{\small\nouppercase{\leftmark}}
\renewcommand{\sectionmark}[1]{\markboth{#1}{}}
\cfoot{\thepage}
\def\labelitemii{$\circ$}

\title{Who Are You, Really? The Surprising Puzzle of Personality}
\author{Brina R. Little}
\date{\today}

\begin{document}
\maketitle
\tableofcontents

%------------------------------------------------------------------------------%

\section*{Introduction}
``Who are you? It's a nosy question, I know, \& perhaps even an uncomfortable one. If I asked you that question over a beer at a bar, you'd probably bolt for the door. But once you realized I was merely an inquisitive psychologist, I suspect you'd have a list of personality traits at the ready. ``I'm an extravert.'' you might say proudly. Or ``I'm a nurturer,'' or ``I'm a worrier,'' or ``I am the 5th least narcissistic person on earth.'' Each of us has a sense of the basic traits that define us.

Next, if I asked you \textit{why} you are that way, you'd probably also have some answers already in your quiver. ``Because I'm from the west coast,'' you might say. Or ``Because I'm an oldest child,'' or ``Because my dad was a drinker,'' or ``Because the Great Recession hit while I was in high school.'' You'd have good reason to make those connections. It's clear that outside influences -- your home life, the community where you grew up, the political milieu into which you were born -- have shaped your life \& the way you behave.

\& that's it, you might think, it's settled. You are who you are because of your inherent nature coupled with the external forces that have influenced you throughout your life. It isn't really that complex, is it? You've spent enough years getting to know yourself that you should have the picture of your personality put together by now. Right?

You'd better settle in, because your exploration of you is just getting started.

You see, genetics \& experiences aren't everything. There is a 3rd force that also determines your personality. \& when it comes to this force, our usual assumptions have it backward; it's not \textit{who you are} that explains \textit{what you do}, it's the other way around. That, in fact, is the very idea I'm about to present to you. It is an important new way of looking at personality, \& it is what I've spent the better part of a half-century researching \& understanding.

Your life \& your identity derive from more than just your inborn traits \& your circumstances; they are borne of your aspirations \& commitments, your dreams \& your everyday doings. These defining activities are, in 2 words, your \textit{personal projects}. Personal projects can range from the seemingly trivial pursuits of particular Thursdays to the overarching quest of your life. They include endeavors small \& large, from the intimate to the professional, from the mundane to the existential. They range from ``taking out the garbage'' to ``taking out my political opponent.'' These personal projects, for better or worse, are shaped in part by both our biological traits \& our social contexts. But they transcend each. Because unlike nature \& nurture, they are 1 feature of human life that is not given to us by heredity or society but is generated from within.

You might already be wondering how much your activities could really affect something that seems as stable as your personality \& sense of self. The answer is perhaps more than you might imagine. Personal projects are central not only to who you think you are but also to how well you are doing in life -- whether you are flourishing or floundering, or like most of us, just muddling through as best you can. Your personal projects, in short, are key to your prospects. Learn to understand them \& their impact, \& you learn to guide your life in the direction you want it to go.

In these pages we'll look closely at your personality in terms of how your life has gone \& how it is going now. But we'll also be concerned with how it might yet go in the future. This is where your personal projects come in: Once you can clearly identify your personal projects \& their power, you'll also see the degrees of freedom or spaces for movement that are open to you in determining your own course. My own personal project with this book is to help you see \& steer your life -- \& to do this before kids with scrapes, cats with furballs, or friends who really need to talk right now divert you from plotting your future self.

As I want to make this personal for both of us, let's start with my own account of how I came to study human personality. It was an unusually hot September afternoon in 1965 when I cautiously tapped on the office door of Prof. Theodore R. Sarbin. Sarbin was an eminent scholar of psychology at the University of California, Berkeley. I was a 2nd-year graduate student eager to join his research group. The door swung open \& a voice intoned loudly, ``WHO ARE YOU?'' I inferred from Sarbin's stentorian voice \& the way he drew out the ``o'' in ``who'' that this was more than a desire to know my name. He wanted me to declare my identity! Or what role I was playing, what self I was enacting at that very moment. So I said, in a self-mocking, elevated tone: ``A seeker after truth.'' Sarbin rolled his eyes, snickered, \& said, ``Oh no, not another one.''

A more honest answer to Sarbin's question would have been less grandiose but more complex \& interesting. I could have described the personality traits that I'd known were mine since childhood: introverted, curious, \& affable. I could have described my roles in relation to other people \& the world: a psychology student, a passionate dabbler in piano, \& a Kennedy supporter still aching from the assassination. But that was a lot to spurt out in a professor's doorway. Besides, even that would not have contained an entirely accurate picture. Because at the precise moment in my life, I was undergoing a radical change spurred by the extraordinary political events unfolding around me, which I will get to shortly.

But 1st some context: Psychology at the time was still grappling with whether biological or social forces were more powerful, more consequential, in shaping our personalities -- what, back then, we called the nature-nurture debate.

``I am, in essence, my brain, Prof. Sarbin,'' I could have said, aligning myself with the believers in nature, or biological determinism. Indeed, the opportunity to explore the biological basis of behavior was the reason I had chosen to go to Berkeley in the 1st place. Prior to grad school I had been a research assistant in a neuropsychology laboratory, \& when I applied to grad school, I was convinced that what shapes our personalities is primarily genetic \& neuropsychological -- what I call biogenic influences. I believed that the study of the brain would be the best route to understanding who we \textit{really} are.

Or I could have sworn my allegiance to the nurture camp. I was a short, skinny kid from the west coast of Canada, the son of a whimsical Irish father \& a nurturing English mother, \& raised in a whimsically nurturing environment. These sociogenic influences aligned with Sarbin's view of what shapes our behavior: He saw individuals as the products of social \& cultural forces that provide codes, roles, \& scripts for how to live.

All the time, there was also a new twist developing around the nature-nurture debate. An interdisciplinary team of psychologists \& anatomists at Berkeley had demonstrated that by enriching the external living environments of rodents, they could directly change the animals' brain structure \& biochemistry.\footnote{This was a collaborative project between Mark Rosenzweig \& David Krech of the Department of Psychology \& Marian Diamond of the Department of Anatomy at the University of California, Berkeley. See Rosenzweig, Krech, Bennett, \& Diamond (1962).} Animals reared with social stimulation \& complex, exploratory objects in their cages (``friends \& toys,'' as the researchers put it) literally had heavier brains \& more complex neural circuitry. This was groundbreaking \& controversial stuff, carrying potential implications for improving the quality of life for mice \& men (\& wolves \& women). Yes, biological influences were necessary for a full understanding of behavior, but they were not fixed \& immutable; change was possible.

By now, of course, psychology has moved way beyond the old nature-nurture debate of my student days. We now know that these influences are interpenetrating. It is possible to nurture our natures -- that was, after all, the lesson we were taught by those little rodents with friends \& toys in Berkeley.\footnote{The emerging field of social genomics has exciting implications for understanding health \& flourishing. E.g., Steve Cole \& his colleagues at UCLA have demonstrated that how genes express themselves is linked to how much loneliness is experienced by participants in his studies. See Cole (2009).}

But as I would come to understand, these answers to ``Who are you?'' simply don't provide the best insights for understanding our true natures. What I have been exploring since that fateful knock on Sarbin's door is how our singular, idiosyncratic pursuits -- our personal objects -- not only rival the biological \& social explanations for who we are, but transform the way we think about each of them. These projects are even powerful enough drivers to make us act out of character, redefining our very personalities. I've experienced this myself.

As I hinted earlier, I was in the midst of a fundamental personal transformation when I entered Sarbin's office. I had arrived in Berkeley a year earlier in Sep 1964, the very week that the Free Speech Movement (FSM) began on campus. The university administration had just banned tables from the area students were using to recruit volunteers for freedom rides in the American South. The policy sparked student demonstrations, sit-ins, \& teach-ins. Protestors claimed that the massive, distinguished university -- a self-proclaimed multiversity -- was in thrall to its Nobel laureates \& industrial contracts \& had little concern for its students.

The FSM captured my imagination, \& its impact was palpable. It was a call to action -- to get involved in projects that became deeply personal, even self-defining. Suddenly the introspective psychology student in me, one who would rather sing about revolution than start it, felt driven to speak out to overcome injustice. This was new, \& it shook my sense of identity to its roots. What's more, that shift propelled me not only to think \& feel new things but to act in new ways. This pursuit that I had chosen was, almost invisibly, reshaping the person I was. Projects like ``sitting in'' or ``going to the demonstration'' or ``seeking justice'' were now commitments -- acts of meaning with consequences for the person I was becoming.

Which takes us back to the question I originally asked you -- the same one that Sarbin startled me with that sweltering day when I knocked on his door: ``Who are you?'' Understanding yourself as simply the product of biogenic forces prodding you or sociogenic forces shaping you is unduly limiting. I want to convince you that you are also shaped by the personal projects that draw from both your biology \& your culture \& can, as we will see, transform both. Such projects may cause you to stretch yourself in new directions, to create a sense of meaning in your life. This new way of thinking about yourself will allow us to ask: Who are you, \textit{really}? \& equipped with that self-knowledge, you can then understand how you're doing -- \& begin actively navigating your future.'' -- \cite[pp. 6--11]{Little2017}

%------------------------------------------------------------------------------%

\section{Scanning Your Personality: The Big Picture}

%------------------------------------------------------------------------------%

\section{Personal Projects: The Doings of Personality}

%------------------------------------------------------------------------------%

\section{Personal Contexts: The Social Ecology of Project Pursuit}

%------------------------------------------------------------------------------%

\section{The Myth of Authenticity: The Challenge of Being Oneself}

%------------------------------------------------------------------------------%

\section{Well-Doing: The Sustainable Pursuit of Core Projects}

%------------------------------------------------------------------------------%

\printbibliography[heading=bibintoc]
	
\end{document}