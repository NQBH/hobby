\documentclass{article}
\usepackage[backend=biber,natbib=true,style=authoryear]{biblatex}
\addbibresource{/home/nqbh/reference/bib.bib}
\usepackage{tocloft}
\renewcommand{\cftsecleader}{\cftdotfill{\cftdotsep}}
\usepackage[colorlinks=true,linkcolor=blue,urlcolor=red,citecolor=magenta]{hyperref}
\usepackage{algorithm,algpseudocode,amsmath,amssymb,amsthm,float,graphicx,mathtools}
\allowdisplaybreaks
\numberwithin{equation}{section}
\newtheorem{assumption}{Assumption}[section]
\newtheorem{conjecture}{Conjecture}[section]
\newtheorem{corollary}{Corollary}[section]
\newtheorem{definition}{Definition}[section]
\newtheorem{example}{Example}[section]
\newtheorem{lemma}{Lemma}[section]
\newtheorem{notation}{Notation}[section]
\newtheorem{principle}{Principle}[section]
\newtheorem{problem}{Problem}[section]
\newtheorem{proposition}{Proposition}[section]
\newtheorem{question}{Question}[section]
\newtheorem{remark}{Remark}[section]
\newtheorem{theorem}{Theorem}[section]
\usepackage[left=0.5in,right=0.5in,top=1.5cm,bottom=1.5cm]{geometry}
\usepackage{fancyhdr}
\pagestyle{fancy}
\fancyhf{}
\lhead{\small Sect.~\thesection}
\rhead{\small\nouppercase{\leftmark}}
\renewcommand{\sectionmark}[1]{\markboth{#1}{}}
\cfoot{\thepage}
\def\labelitemii{$\circ$}

\title{Who Are You, Really? The Surprising Puzzle of Personality}
\author{Brina R. Little}
\date{\today}

\begin{document}
\maketitle
\tableofcontents

%------------------------------------------------------------------------------%

\section*{Introduction}
``Who are you? It's a nosy question, I know, \& perhaps even an uncomfortable one. If I asked you that question over a beer at a bar, you'd probably bolt for the door. But once you realized I was merely an inquisitive psychologist, I suspect you'd have a list of personality traits at the ready. ``I'm an extravert.'' you might say proudly. Or ``I'm a nurturer,'' or ``I'm a worrier,'' or ``I am the 5th least narcissistic person on earth.'' Each of us has a sense of the basic traits that define us.

Next, if I asked you \textit{why} you are that way, you'd probably also have some answers already in your quiver. ``Because I'm from the west coast,'' you might say. Or ``Because I'm an oldest child,'' or ``Because my dad was a drinker,'' or ``Because the Great Recession hit while I was in high school.'' You'd have good reason to make those connections. It's clear that outside influences -- your home life, the community where you grew up, the political milieu into which you were born -- have shaped your life \& the way you behave.

\& that's it, you might think, it's settled. You are who you are because of your inherent nature coupled with the external forces that have influenced you throughout your life. It isn't really that complex, is it? You've spent enough years getting to know yourself that you should have the picture of your personality put together by now. Right?

You'd better settle in, because your exploration of you is just getting started.

You see, genetics \& experiences aren't everything. There is a 3rd force that also determines your personality. \& when it comes to this force, our usual assumptions have it backward; it's not \textit{who you are} that explains \textit{what you do}, it's the other way around. That, in fact, is the very idea I'm about to present to you. It is an important new way of looking at personality, \& it is what I've spent the better part of a half-century researching \& understanding.

Your life \& your identity derive from more than just your inborn traits \& your circumstances; they are borne of your aspirations \& commitments, your dreams \& your everyday doings. These defining activities are, in 2 words, your \textit{personal projects}. Personal projects can range from the seemingly trivial pursuits of particular Thursdays to the overarching quest of your life. They include endeavors small \& large, from the intimate to the professional, from the mundane to the existential. They range from ``taking out the garbage'' to ``taking out my political opponent.'' These personal projects, for better or worse, are shaped in part by both our biological traits \& our social contexts. But they transcend each. Because unlike nature \& nurture, they are 1 feature of human life that is not given to us by heredity or society but is generated from within.

You might already be wondering how much your activities could really affect something that seems as stable as your personality \& sense of self. The answer is perhaps more than you might imagine. Personal projects are central not only to who you think you are but also to how well you are doing in life -- whether you are flourishing or floundering, or like most of us, just muddling through as best you can. Your personal projects, in short, are key to your prospects. Learn to understand them \& their impact, \& you learn to guide your life in the direction you want it to go.

In these pages we'll look closely at your personality in terms of how your life has gone \& how it is going now. But we'll also be concerned with how it might yet go in the future. This is where your personal projects come in: Once you can clearly identify your personal projects \& their power, you'll also see the degrees of freedom or spaces for movement that are open to you in determining your own course. My own personal project with this book is to help you see \& steer your life -- \& to do this before kids with scrapes, cats with furballs, or friends who really need to talk right now divert you from plotting your future self.

As I want to make this personal for both of us, let's start with my own account of how I came to study human personality. It was an unusually hot September afternoon in 1965 when I cautiously tapped on the office door of Prof. Theodore R. Sarbin. Sarbin was an eminent scholar of psychology at the University of California, Berkeley. I was a 2nd-year graduate student eager to join his research group. The door swung open \& a voice intoned loudly, ``WHO ARE YOU?'' I inferred from Sarbin's stentorian voice \& the way he drew out the ``o'' in ``who'' that this was more than a desire to know my name. He wanted me to declare my identity! Or what role I was playing, what self I was enacting at that very moment. So I said, in a self-mocking, elevated tone: ``A seeker after truth.'' Sarbin rolled his eyes, snickered, \& said, ``Oh no, not another one.''

A more honest answer to Sarbin's question would have been less grandiose but more complex \& interesting. I could have described the personality traits that I'd known were mine since childhood: introverted, curious, \& affable. I could have described my roles in relation to other people \& the world: a psychology student, a passionate dabbler in piano, \& a Kennedy supporter still aching from the assassination. But that was a lot to spurt out in a professor's doorway. Besides, even that would not have contained an entirely accurate picture. Because at the precise moment in my life, I was undergoing a radical change spurred by the extraordinary political events unfolding around me, which I will get to shortly.

But 1st some context: Psychology at the time was still grappling with whether biological or social forces were more powerful, more consequential, in shaping our personalities -- what, back then, we called the nature-nurture debate.

``I am, in essence, my brain, Prof. Sarbin,'' I could have said, aligning myself with the believers in nature, or biological determinism. Indeed, the opportunity to explore the biological basis of behavior was the reason I had chosen to go to Berkeley in the 1st place. Prior to grad school I had been a research assistant in a neuropsychology laboratory, \& when I applied to grad school, I was convinced that what shapes our personalities is primarily genetic \& neuropsychological -- what I call biogenic influences. I believed that the study of the brain would be the best route to understanding who we \textit{really} are.

Or I could have sworn my allegiance to the nurture camp. I was a short, skinny kid from the west coast of Canada, the son of a whimsical Irish father \& a nurturing English mother, \& raised in a whimsically nurturing environment. These sociogenic influences aligned with Sarbin's view of what shapes our behavior: He saw individuals as the products of social \& cultural forces that provide codes, roles, \& scripts for how to live.

All the time, there was also a new twist developing around the nature-nurture debate. An interdisciplinary team of psychologists \& anatomists at Berkeley had demonstrated that by enriching the external living environments of rodents, they could directly change the animals' brain structure \& biochemistry.\footnote{This was a collaborative project between Mark Rosenzweig \& David Krech of the Department of Psychology \& Marian Diamond of the Department of Anatomy at the University of California, Berkeley. See Rosenzweig, Krech, Bennett, \& Diamond (1962).} Animals reared with social stimulation \& complex, exploratory objects in their cages (``friends \& toys,'' as the researchers put it) literally had heavier brains \& more complex neural circuitry. This was groundbreaking \& controversial stuff, carrying potential implications for improving the quality of life for mice \& men (\& wolves \& women). Yes, biological influences were necessary for a full understanding of behavior, but they were not fixed \& immutable; change was possible.

By now, of course, psychology has moved way beyond the old nature-nurture debate of my student days. We now know that these influences are interpenetrating. It is possible to nurture our natures -- that was, after all, the lesson we were taught by those little rodents with friends \& toys in Berkeley.\footnote{The emerging field of social genomics has exciting implications for understanding health \& flourishing. E.g., Steve Cole \& his colleagues at UCLA have demonstrated that how genes express themselves is linked to how much loneliness is experienced by participants in his studies. See Cole (2009).}

But as I would come to understand, these answers to ``Who are you?'' simply don't provide the best insights for understanding our true natures. What I have been exploring since that fateful knock on Sarbin's door is how our singular, idiosyncratic pursuits -- our personal objects -- not only rival the biological \& social explanations for who we are, but transform the way we think about each of them. These projects are even powerful enough drivers to make us act out of character, redefining our very personalities. I've experienced this myself.

As I hinted earlier, I was in the midst of a fundamental personal transformation when I entered Sarbin's office. I had arrived in Berkeley a year earlier in Sep 1964, the very week that the Free Speech Movement (FSM) began on campus. The university administration had just banned tables from the area students were using to recruit volunteers for freedom rides in the American South. The policy sparked student demonstrations, sit-ins, \& teach-ins. Protestors claimed that the massive, distinguished university -- a self-proclaimed multiversity -- was in thrall to its Nobel laureates \& industrial contracts \& had little concern for its students.

The FSM captured my imagination, \& its impact was palpable. It was a call to action -- to get involved in projects that became deeply personal, even self-defining. Suddenly the introspective psychology student in me, one who would rather sing about revolution than start it, felt driven to speak out to overcome injustice. This was new, \& it shook my sense of identity to its roots. What's more, that shift propelled me not only to think \& feel new things but to act in new ways. This pursuit that I had chosen was, almost invisibly, reshaping the person I was. Projects like ``sitting in'' or ``going to the demonstration'' or ``seeking justice'' were now commitments -- acts of meaning with consequences for the person I was becoming.

Which takes us back to the question I originally asked you -- the same one that Sarbin startled me with that sweltering day when I knocked on his door: ``Who are you?'' Understanding yourself as simply the product of biogenic forces prodding you or sociogenic forces shaping you is unduly limiting. I want to convince you that you are also shaped by the personal projects that draw from both your biology \& your culture \& can, as we will see, transform both. Such projects may cause you to stretch yourself in new directions, to create a sense of meaning in your life. This new way of thinking about yourself will allow us to ask: Who are you, \textit{really}? \& equipped with that self-knowledge, you can then understand how you're doing -- \& begin actively navigating your future.'' -- \cite[pp. 6--11]{Little2017}

%------------------------------------------------------------------------------%

\section{Scanning Your Personality: The Big Picture}
``So, how you are doing? Are you happy? Are you accomplishing the things that matter to you? Are you living up to your capabilities? Are you able to love \& be loved? Are you physically well? Is there some laughter in your life? If you answer yes to all such questions, we might say that you are \textit{flourishing}. If you answer with an emphatic ``No!'' or even an eyeball-rolling ``Seriously, get real,'' you might be better described as \textit{floundering}. \& in between these extremes, we might find you in the middle, doing reasonably well considering the circumstances.

Biogenic traits deeply influence whether you flourish or flounder. You may be temperamentally predisposed to viewing your life positively \& optimistically, even though the objective reality that you confront might be rather bleak. Or despite living in a relatively safe, nurturing, \& prosperous environment, you may see your life as half empty, or utterly miserable. The forces of nature \& nurture that provide answers to ``who are you?'' are also key to answering the question ``How are you doing?'' The relation between these biogenic \& sociogenic influences can be simply graphed as:
\begin{center}
	\fbox{Biogenic Sources} $\to$ \fbox{Flourishing} $\leftarrow$ \fbox{Sociogenic Sources}.
\end{center}
Whether you are flourishing or floundering, in other words, is partly determined by the combination of biogenic \& sociogenic sources that impinge on you during the course of your life. These aren't the only influences, but we need to understand how they work before we explore how your personal projects empower you to deliberately design who \& how you are. So let us begin with a brief tour through the inner biogenic \& the outer sociogenic forces that shape your personality.'' -- \cite[p. 12]{Little2017}

\subsection{Personal Zoom: Scanning the Inner You}
``Imagine a microscope that dips under your skin \& zooms down to reveal your tissues, organelles, cell nuclei, chromosomes, \& genes. It darts up to your brain \& homes in on a single neuron firing a squirt of neurotransmitters \& the explosion of activity in associated cells. It then zooms out to focus on the physical body reading this book wondering about who it is \& how it's doing. This ``it'' is the biogenic you.

Within personality psychology, those who study the biogenic perspective explore how your relatively stable personality traits influence your qualify of life. These stable traits correspond to differences in brain structure \& function -- those microscopic events we just saw when zooming in on the inner you. These biogenic features can be assessed by measures of electrical activity in various regions of your brain or through analysis, which can now be done for roughly \$200. In \textit{My Beautiful Genome}, the Danish science writer Lone Frank relates the fascinating account of her quest to examine aspects of her personal genome \& its links to her health \& personality. She discovered that she had a gene variant that predisposed her to negative emotionality \& what she most agreeably describes as her ``own miserably low score in agreeableness.''\footnote{See Lone Frank (2011).}

Some of these biogenic personality traits will incline you toward being happy or healthy or accomplished or, conversely, will explain why you despair over life's various hiccups. Let's say your life is flourishing right now -- you are happy, healthy, \& successful, certainly compared to your mopey best friend, but maybe even in an absolute sense. This may be due to your having biogenic features of temperament \& personality that dispose you to adopt a positive outlook. Even when life sucks, your stable dispositions make you resilient \& buoyant. You continue to grow \& prosper. Indeed, you may have \textit{pronoia}, the delusional belief that other people are plotting your well-being or saying good things about you behind your back.\footnote{The term \textit{pronoia} was 1st coined by the sociologist Fred Goldner \& is meant to depict the characteristics that are the polar opposite to those of paranoia. See Goldner (1982).} Your friend's stable traits, in comparison, may not be conductive to flourishing at all. She is angry \& defiant \& unsatisfied, \& according to her mother, she was like this from birth. She is temperamentally disposed to being ill-disposed. She flounders.'' -- \cite[pp. 12--14]{Little2017}

\subsection{The Big 5: The Original You}
``Did you know that it is virtually impossible for you to lick the outside of your own elbow? \& did you know, strange as it may sound, that how you responded to that piece of information -- whether \& how you attempted the pursuit -- might provide a hint about the stable traits you are born with \& that form the bedrock of your personality? Let me explain: While there are thousands of ways we might distinguish people on the basis of their traits, personality psychologists have reached a consensus that people vary from one another along 5 basic dimensions: the Big 5 traits. The Big Five have major consequences for how our lives play out.\footnote{There is now a substantial body of research on the Big Five traits. See especially the review by Ozer \& Benet-Mart\'inez (2006) that explores the practical consequences of traits for education, marriage, health, \& work. Daniel Nettle (2007) has written an excellent introduction for the general reader. See also Little (2014), especially Chap. 2.} If you would like to get a quick assessment of where you stand on these major traits, the Appendix provides some questions that can guide your own self-assessment.

The 5 dimensions spell out an acronym -- OCEAN (or CANOE if you prefer): Open to Experience (vs. Closed), Conscientious (vs. Casual), Extraverted (vs. Introverted), Agreeable (vs. Disagreeable), Neurotic (vs. Stable). Each of these traits has a strong biogenic base, \& researchers in personality neuroscience are now identifying the neural structures \& pathways underlying them.\footnote{An especially promising analysis of the neuropsychological basis of the Big Five traits appears in the work of Colin DeYoung \& his colleagues. See DeYoung (2010).} Because the same dimensions emerge in virtually all countries, cultures, \& linguistic groups, these can be regarded as universal dimensions of personality. This doesn't mean that all humans are the same -- far from it. Rather, it means that everywhere we go, individuals differ from one another along these dimensions. Also, these 5 traits do not have rigid boundaries; individuals are aligned with each trait on a spectrum, with most of them piled up in the middle of the range \& fewer appearing at the extremes. Here is a short overview of each one.'' -- \cite[pp. 14--16]{Little2017}

\subsection{Open to Experience}
``Those who are high in openness to experience are easily attracted to new ventures \& show alacrity in exploring alternative ways of doing things. Those low in openness prefer the tried \& true \& would, unlike their more open friends, be very comfortable using a phrase like ``tried \& true.'' A landmark study at the Institute of Personality Assessment \& Research at Berkeley (now the Institute of Personality \& Social Research) revealed that openness to experience was the defining feature of individuals who are exceptionally creative.\footnote{See, e.g., MacKinnon (1962) \& Chap. 8 in Little (2014). For a detailed exposition of how the assessment process was carried out, see Serraino (2016).} In an intriguing study by 1 of the prime developers of the Big 5, open individuals were found to experience aesthetic chills or piloerections -- their hair stood up -- when exposed to music or art that moved them.\footnote{See McCrae (2007).}

So did you try to lick your elbow? I suspect that if you are game for new experiences, you would have had a go at it. If you are low on openness, you were more likely to just keep reading.'' -- \cite[pp. 16--17]{Little2017}

\subsection{Conscientiousness}
``Individuals who score high on conscientiousness are particularly likely to satisfy traditional definitions of success. They perform better in academic pursuits \& in measures of occupational achievement than those who are low in conscientiousness. It should be noted, though, that these successes are most frequently found in courses \& careers that stress conventional problem solving, while thos who are high in openness excel at tasks that involve coming up with original ways of doing things. Highly conscientious people are punctual \& persevering; they can focus intently on the activities in front of them. This laser focus, however, might work better in some fields than others. E.g., Robert \& Janice Hogan, pioneers in the study of personality \& organizations, devised a study in which jazz musicians rated their fellow musicians on how good they were as performers. Those who score high in conscientiousness were rated by their peers as \textit{less} effective. Perhaps this is because musicians who intensely concentrate on their playing may be inhibiting the spontaneity crucial to improvisational jazz.\footnote{See Hogan \& Hogan (1993).}

Conscientious adults are likely to avoid drugs, stay clear of dangerous activities, \& stick to health \& fitness regimens. As such, they are healthier \& live longer compared with their less conscientious peers. \& the difference in well-being isn't minor: Lack of conscientiousness has been shown to be as important as having heart disease in predicting early death. Conscientious individuals also invest more in work \& family roles that reward \& increase conscientiousness.

How about their elbows \& the implicit invitation to lick them? When I've asked people to do this in groups, the conscientious ones are less likely to lick. Instead, I think they write a note to themselves to check it out when they get home. Those who are exceptionally conscientious, I suspect, will have already googled ``licking own elbow'' to see if it really is impossible to accomplish.'' -- \cite[pp. 17--18]{Little2017}

\subsection{Extraversion}
``Extraverts are highly responsive to potential rewards in their environments. They quickly spot \& move toward the positive stimulation that they crave in order to accomplish their daily tasks \& projects efficiently. This trait, too, has biogenic roots. We've seen the evidence in the fact that extraverts, relative to introverts, perform better on cognitive tasks involving anagrams or short-term memory when their brains are aroused by chemical stimulants such as caffeine. Conversely, they do worse if they ingest a sedative such as alcohol.

Extraverts' musical preferences turn to the loud, pulsing, \& energizing. Partly because of their need for stimulation \& their focus on the potential for reward rather than punishment, extraverts are more likely to have brushes with authority like getting traffic tickets or, earlier in their lives, being sent, repeatedly, to their rooms.\footnote{There is a vast research literature on extraversion \& its effects on performance, motivation, \& risk-taking. An authoritative \& comprehensive review has been carried out by Wilt \& Revelle (2008).}

1 of the most stimulating situations for extraverts is social interaction, \& they engage with such encounters readily. Indeed, among the most intensely stimulating social activities are sexual ones -- extraverts have been shown to have greater frequency \& diversity of sexual experience than more introverted individuals. Introverts can take some comfort, however, in the fact that there is a quality\texttt{/}quantity trade-off in various types of task performance: Extraverts opt for quantity over quality \& introverts the reverse. My introverted students tell me these results so apply to their sexual performance. I am open-minded but data-free on this speculation.\footnote{The study of frequency of intercourse was originally reported in Giese \& Schmidt (1968). I have been unable to find a more recent study, so I would caution the reader that these results were obtained from unmarried, heterosexual German university students who had reported being sexually active. And it was the sixties.}

When it comes to the question of licking elbows, I strongly suspect that extraverted readers will have attempted to lick their own elbows. They may have also successfully licked the elbow of the person sitting next to them.'' -- \cite[p. 18]{Little2017}

\subsection{Agreeableness}
``Highly agreeable individuals are particularly effective in groups where they can be relied on to smooth over conflict \& build alliances. Relative to less agreeable people, they are very trusting \& for this reason might be seen by others as na\"ive. Highly agreeable people score high on a measure of person-orientation, which is associated with empathy, altruism, \& an interaction style that conveys warmth \& expressiveness. They also attend to be the expensive cues of other individuals, \& this contributes to their ability to empathize.\footnote{Little (1976) provides research on person-orientation, warmth, \& expressiveness.}

Those scoring low on agreeableness are cynical \& distrustful of others. They display patterns of hostility that raise their risk for health issues, especially cardiovascular problems. In this respect, they resemble the so-called coronary-prone, or Type A, personality, who is typically time-pressured \& hard-driving. It is now recognized that the disagreeableness \& hostility underlying Type A behavior is the real risk factor for cardiac problems, rather than their drive to succeed.

How about agreeableness \& elbow licking? Highly agreeable people are, well, agreeable, so it is likely that they agreed to play along. Those scoring very low on agreeableness, however, probably weren't game. They may have stopped reading altogether \& gone outside to yell at their neighbors' kids.'' -- \cite[pp. 18--19]{Little2017}

\subsection{Neuroticism}
``The term \textit{neuroticism} has a pejorative tone, \& though there are some places where neuroticism is frequently valued \& sought after (New York City comes to mind), it is generally regarded unfavorably. Those who score high on the neuroticism scale of the Big 5 are disposed to anxiety, depression, \& vulnerability. This does not mean that they aer clinically depressed or phobic; they simply experience negative emotions that interfere with their quality of life. Just as extraverts tend to seek out potential rewards when exploring their environments, neurotic individuals are acutely attuned to potential punishment. Not surprisingly, when we look at which of the Big 5 traits best predict whether a person will be happy, stable extraverts are the most happy \& neurotic introverts the least.

Is there anything positive about neuroticism? In some respects, neurotic individuals are highly sensitive people who, like the canaries in the mine, can detect things that less sensitive people simply don't register -- changes in the environment, disturbances in routines, \& whiffs of danger from unexpected sources. This is not conducive to relaxed \& easy living. But writers \& artists \& others who are astute observers of life are often found to have a neurotic disposition. In the evolutionary provenance of human personality, I suspect that stable extraverts were the 1st to discover prey, \& we all benefited from eating what they caught. To survive, however, we also needed the neurotic introverts who were especially likely to discover predators. We should be equally grateful to them for decreasing our chances of being sniffed out, hunted down, \& eaten.\footnote{Elaine Aron has written perceptively about highly sensitive individuals who demonstrate some, but not all, of the characteristics of neurotics. Although more frequently associated with introversion, about 30\% of highly sensitive individuals are extraverted. See Aron (1996).}

If you're neurotic, perhaps you have been agonizing over the elbow-licking question for some time \& worrying that your ability to rise to small challenges has once again been disproved. But I certainly hope not. Sensitivity is often underrated. \& from an evolutionary perspective, we really owe you a lot.

What are the implications here for understanding who you are? There is evidence that there is a genetic base to each of the Big 5 dimensions of personality. These essential traits form our 1st natures. Yet that does not mean that 1st natures \& the luck of the genetic lottery are the sole determinants of our paths in life.'' -- \cite[pp. 19--20]{Little2017}

\subsection{The Outer Sociogenic You}
``Now let's take a macroscopic view of you. Imagine a camera zooming out \& away from where you are. 1st, we see an image of you reading this book, then other people in your living room, on the train, or at the caf\'e where you are reading come into view. Then we zoom out to catch images of your city, region, country, \& eventually the whole earth. These images reveal the complex web of situations, settings, places, \& contexts where you \& others engage in your daily pursuits.

This imaginary camera has an added feature: It can scan your cyber world, social networks, \& virtual spaces -- last week's e-mails, yesterday's selfies (even the ones you deleted), \& your whole browsing history for the last 3 years (gulp). \& right in the center of this vast interconnected web of social \& cultural practices \& people is a creature that other people know \& call by your name. This is the sociogenic you.

Devotees of the sociogenic perspective explore the situations you confront in your daily life \& the larger contexts in which they happen. If biogenic forces shape your 1st nature, then sociogenic forces sculpt your 2nd nature. From this perspective, who you are \& how you are doing do not hinge on your stable traits but on the recurring circumstances of your life. You are molded by the nurturing \& opportunities that you're given, the norms you're imbued with, \& the ways other people expect you to be. Psychologists adopting this viewpoint wish to understand your roles in life, your social networks, \& the prevailing economic \& political systems that govern what you do.

E.g., in the latest World Health Organization survey of happiness in 156 countries, you might have anticipated that the happiest nations would be those with palm trees, turquoise waters, \& drinks with little umbrellas in them. But these are the happiest countries in 2017: Norway, Denmark, Iceland, Switzerland, Finland, The Netherlands, Canada. Not many palm trees here, although to be fair, the next 2 on the list are New Zealand \& Australia, and they have palm trees galore. But what is common to all of these happiest places is that they are relatively peaceful and prosperous countries. \& crucially, they allow sufficient freedom to pursue individual desires \& offer support systems like welfare \& medical assistance for when things go wrong. If you are fortunate enough to live in 1 of these countries, your chances of happiness \& a bountiful life are favorable. (Of course, the price you pay for flourishing might be freezing your butt off for half the year.)'' -- \cite[pp. 20--21]{Little2017}

\subsection{But Wait: 2 Caveats}
``What if we were to stop here? Is this all you are? What if we concluded that you are a biogenic creature with a brain that shapes you \& a sociogenic self with a culture that molds you, \& that these forces determine whether you flourish? If we left it at that, we would be making 2 fundamental mistakes.

1st, contemporary research has increasingly revealed that nature \& nurture, far from being separate features of the human condition, are intimately linked. Brain plasticity, the capacity to change neural functioning through experience \& training, at least temporarily, is now widely acknowledged. Remember those little mammals exploring their enriched cages in Berkeley \& gaining brain weight as a result? Those findings, radical \& controversial at the time, are now commonplace. Indeed, today there is widespread adoption of programs that are based on the assumption that optimal functioning is malleable. In other words, your social contexts can affect your biology. \& the reverse is also true -- biogenic personality traits can directly influence the social contexts of your life. Infants who are temperamentally easygoing \& affable raise their parents very differently than those who are cranky \& tense. Individuals who are naturally disposed to seeking out social stimulation will create very different social contexts for themselves than will those naturally closed to others.

In short, there are vital links between the biogenic \& sociogenic factors that shape our lives. So a more accurate depiction of their link with flourishing would be this: \fbox{Flourishing} $\leftarrow$ \fbox{Biogenic} $\leftrightarrows$ \fbox{Sociogenic} $\rightarrow$ \fbox{Flourishing}

Yet there is a 2nd, more fundamental mistake with looking at yourself as simply a biogenic creature or a sociogenic self, even if you assume these identities can interact \& influence each other. Both assume that you are a passive recipient of forces that play on you -- that you are not an agent of your own development but a pawn moved by the power of genes or environment or both.

The implications of this view are profound. If who you are \& what you do are simply mechanical consequences of forces beyond your control, then you lose the capacity for responsible action or for self-change. To think of yourself in this restrictive fashion shortchanges you -- it decreases your degree of freedom.

Certainly, it is both enlightening \& intriguing to learn about the genetic influences on your personality or the shaping of your life through dominant social institutions. They are necessary for a full account of what it means to be human. But they are not sufficient. To reflect accurately on your essential personality or your options in life or what potential selves you wish to explore, we need a more expansive view of what kind of creature you truly are. We need to change the camera angle.'' -- \cite[pp. 21--23]{Little2017}

\subsection{3rd Nature}
``If we want to understand you fully, we'd start with your 1st nature, pinpointing your biogennic trait such as where you stand on the Big 5 dimension scale. Then we would identify your 2nd nature, the sociogenic influences that supply the roles \& scripts through which you engage with your world. But there is a 3rd nature that shapes us in powerful ways. This is your \textit{idiogenic} self, derived from the Greek \textit{idio-}, meaning personal or particular to oneself.

To see the origins of this idea, we need to zoom back in my own history, to a serendipitous encounter with a book that changed not just my work but my understanding of who I am -- \& who you are.'' -- \cite[p. 23]{Little2017}

\subsection{Personal Constructs: Your Goggles for Viewing the World}
``A month or so before heading off to graduate school, I was searching my college library for a book on brain anatomy. As I reached up to where that book should have been, I pulled down a misshelved tome by George A. Kelly titled \textit{The Psychology of Personal Constructs}. I had heard some favorable things about Kelly in 1 of my classes, so I thought I should skim a few pages. Several hours later, completely engrossed \& aching from having squatted so long on the library floor, I had 1 of those epiphanies that make life so interesting. Although I still wanted to study neuropsychology, it would have to wait until I could explore Kelly's theory of personality in real depth.

The essential idea behind personal construct theory is this: All individuals are essentially scientists erecting \& testing hypotheses about the world \& revising them in the light of their experience. Those hypotheses are called \textit{personal constructs}, \& they are the conceptual goggles through which we view the world.\footnote{The best source for reading about personal construct theory is still Kelly (1955).}

The critical word here is \textit{personal}. These constructs are individually significant to you \& expressed in your own words. Constructs are typically communicated as short phrases that compare \& contrast different people, objects, or events. E.g., you may size up other people as nice versus mean, blunt versus nuanced, bright versus stupid, or high energy versus low energy. These ways of construing individuals mean a lot to you \& have enabled you to negotiate most of your daily encounters. At the same time, your coworker or neighbor are probably working with their own set of go-to descriptors.

Kelly's theory was considered radical when he published his work in the mid-20th century. The prevailing views of personality at the time were grounded in psychoanalysis \& behaviorism, each of which, taking different tacks, regarded humans as passive creatures. But Kelly's prototypical human -- you, e.g. -- is not driven by unconscious biogenic forces or buffeted about by sociogenic reinforcements. You are inquisitive, prospective, \& exploratory. \& to understand you, we need to know the personal constructs through which you interpret objects, events, other people, \& yourself.

1 of the interesting things about personal constructs is that they're always in flux. The goggles through which you viewed life in April may no longer be helpful to you in May. As a lay scientist you revise your predictions about the world, you test new ideas, \& in the process, consolidate a new set of personal constructs that works for you. You are driven by your own explorations, your active attempts to make sense of everything around you. These attempts are idiosyncratic, singular, \& deeply personal. They are idiogenic.

So if I want to understand you, I need to put on your goggles \& see the world through your personal constructs. If I wish to understand your personality \& whether you feel your life is going well, I need to look at your world through the lenses that you have created. How are you flourishing or floundering in those aspects of your life that are \textit{personally significant} to you?

The implications of this perspective fascinated me. I saw it clearly: To best understand human personality \& our capacity to flourish, we need to explore not 2 but \textit{3} sources of self: the biogenic, the sociogenic, \& the idiogenic. The interplay among them can be graphed like this: \textsf{Graph of relations of Biogenic, Sociogenic, Idiogenic, Flourishing}.'' -- \cite[pp. 23--25]{Little2017}

\subsection{Another Epiphany: On the Road}
``2 years after stumbling across Kelly's book, I had an opportunity to dig deeper into personal construct theory. As luck would have it, Kelly came out to the West Coast to reach a course on personality psychology at Stanford. I eagerly signed up, \& each day drove the 56 minutes along El Camino Real from Berkeley to Palo Alto, ready to have my personal constructs challenged. The assignment that Kelly set each of us was not for the faint of heart: create a new theory of personality.

After the very 1st lecture, I knocked cautiously on Kelly's office door. He didn't answer with a ``WHO \textit{ARE} YOU?'' but I told him anyway. I said that I believed in the precepts of personal construct theory but wanted to challenge him on 2 different issues. 1st, jazz: I raised the question of whether there was room in his theory for \textit{passive} pleasures such as listening to music. Kelly's eyes lit up \& he began to talk animatedly about \textit{playing} jazz -- how what we identify as a jazz musician's distinctive style is an interpretation of music through their personal constructs. He didn't pick up on my question about \textit{listening} to jazz rather than playing it, but his answer intrigued me nonetheless. I pressed on with a 2nd issue. I told Kelly about my experience with the Free Speech Movement, which I saw as an affirmation of his model of human beings as active determinants of their lives. But I worried that by using personal constructs as ways of studying the person, we weren't paying sufficient attention to the contexts of our lives -- the situations, the institutions, the political climate of the moment. Kelly encouraged me to explore it further, \& I left his office buzzing with excitement.

On the road back to Berkeley that night, I plotted out some possibilities for a revised theory of personality. I was just north of South San Francisco when it occurred to me that the drive itself, the journey I was on, was more than just the elaboration of my personal constructs. Something was missing. I pulled off the highway, too distracted by the idea taking form to keep driving. What I realized there \& then was that what I was pursuing at that moment was a personal \textit{project}. I began to consider the implications of humans pursuing personal projects in their lives -- everyday pursuits that are trivial or transformative, singular or communal, brief encounters or enduring commitments. The concept of personal projects allowed me to bring together both the inner maps that personal constructs provide \& the outer ecology of possibilities, like the off-ramps, cul-de-sacs, \& open highways that formed the route I was taking.

Your relatively fixed traits set some limits on the destinations that your projects might explore. Your social \& culture environments will open up some paths \& shut down others. \& the way you construe the journey -- the way you define, describe, \& judge your own projects -- will be central to whether you keep exploring, turn back or, alas, crash \& burn. In short, project quests involve the interplay of all 3 aspects of our personality -- the biogenic, sociogenic, \& idiogenic -- \& their success is essential for flourishing.\footnote{I have given a more detailed treatment of the influence of personal construct theory on my own research elsewhere. See chapter 1 in Little, Salmela-Aro, and Phillips (2007).}'' -- \cite[pp. 25--26]{Little2017}

%------------------------------------------------------------------------------%

\section{Personal Projects: The Doings of Personality}
``There are 2 ways in which you can think about your personality. The 1st is in terms of the personality attributes you \textit{have} -- like the traits of neuroticism, extraversion, or conscientiousness. The 2nd is in terms of what you are \textit{doing} -- your personal projects: ``get over my social anxiety,'' ``deliver an awesome pitch in my sales meeting,'' or ``stop procrastinating.'' The study of personal projects, the ``doings'' of daily lives, provides us with a different perspective \& greater scope to reflect on our lives than the study of our ``havings'' alone.\footnote{There is now an extensive research literature on Personal Projects Analysis. See Little (1983, 1998, 1999) \& especially Chaps. 9--10 of Little (2014).}

By now you've caught a glimpse of that most counterintuitive argument with which we began: that what you do affects who you are. This is because personal projects are all about the future -- they point us forward, guiding us along routes that might be short \& jerky or long \& smooth. By tracing their route, we can map the most intimate of terrains: ourselves. Most thrilling is that we can learn to adjust our trajectories, riding over the rough patches \& extending the smooth stretches to make our endeavors more effective. In this way, projects help define us -- they shape our capacity for a flourishing life. Because in an important sense, as go your projects, so goes your life.\footnote{The prospective nature of human beings has been featured in a fascinating new book called Homo Prospectus. See Seligman, Railton, Baumeister, \& Sripada (2016).}

An important clarification: Personal projects are not limited to formal projects that are required of us, such as getting Mom into a good nursing home, although sometimes we pursue them out of a sense of duty. They are also, crucially, acts we gladly choose. Toddlers are pursuing projects when they toddle, \& so are lovers when they love. I am certain that our cat has a project when she stalks, pounces, \& sits atop our other cat, purring.

Personal projects can also be fairly trivial pursuits, like taking the dog (or the cat that thinks it’s a dog)\footnote{The question of what is a trivial pursuit is no trivial matter. ``Walking the dog'' is a more or less straightforward project for most of us, unless we don't have a dog. But if you are in a wheelchair, the dog is rambunctious, and the pavement is uneven, this is no trivial pursuit. A fair evaluation of another person's project needs to take both their aspirations \& personal contexts into account. See Little (1989) for more details on this matter.} for a walk. But they can also represent the highest reaches of human aspiration \& acts of courage. When Rosa Parks chose not to move to the back of the bus, it was a small but courageous act with big consequences. The care you take in getting Mom into the nursing home is more than duty, as it breaks your heart to see what Alzheimer's has taken from her.

\textit{Personal projects are extended sets of personally salient action in context}. Let's parse this definition. \textit{Personal}: Personal projects are framed through the idiosyncratic lens of  the project pursuer. We can't simply watch you build a tree house, or train for a marathon, \& surmise what the pursuit of your personal project says about you. In order to truly understand, we need to ask you the question crucial to revealing your idiogenic nature: What do you \textit{think} you're doing? In other words, what does this mean to you? The answer is often surprising.

Somebody observing you right now might infer that you are reading a book. But you know better. You are barely registering the words in front of you because your personal project is actually ``appearing to be independent \& self-confident so no one will ask why I am alone on this cruise ship.'' Your reading behavior is actually a decoy that helps you cope with your rapidly changing answer to the questions ``Who are you?'' (a recovering romantic) \& ``How are you doing?'' (not bad considering the breakup). Let's continue unpacking our definition. \textit{Extended}: A project is not a momentary act but typically a sequence of actions that are extended in space \& time (from seconds to decades). In contrast with the stable traits that are freeze-frame shots of your personality, personal projects are moving pictures; their full meaning is not apparent until the entire sequence comes into view.

\textit{Set}: A project encompasses a series of actions that are considered interrelated by the project pursuer. Let's say you are not actually reading this book to hide from your cruise companions, but rather to glean lessons that will help in your pursuit of making new friends on the boat. Besides reading this book, you might read other books, take personality quizzes, seek advice from people who know you well, \& join in on the ship's game night. But the woman in a floppy hat sitting in the lounge chair beside you could never know, just by observing you reading a book, what personal meaning it has for you.

Switch this around for a minute \& think of how you see others. You might observe the actions of another person who shows up whenever you go to a certain bar, shares the same dentist as you, \& is often seen outside your house. Depending on the person who is engaged in that set of actions, their personal project may be ``being a partner to my wife,'' ``flirting with my neighbor,'' or ``stalking.'' Sets of action may be innocent \& compassionate. They may also be contemptible. It depends on the project underlying them.

\textit{Personally salient}: By salient I mean that a person identifies a course of action as standing out from all the other possibilities that might be pursued. A personal project enacts 1 possibility among all those dizzying alternative possibilities. It is deliberate. \& that is because it means something particular to you, the pursuer.

\textit{Action}: A project is not a passive response to external forces but an intentional sequence of behavior. A blink is not an action if it occurs as a reflexive response to a puff of air. But a wink is an action that has consequences. Notice how subtle the difference is between action \& nonaction: Even a blink can be an action if it is performed by an optometrist showing you what to do after receiving eye drops.

\textit{In context}: Personal projects are enacted in physical, social, cultural, \& temporal contexts, \& these contexts, as we've discovered, can stimulate, potentiate, inhibit, or block project pursuit.'' -- \cite[pp. 27--31]{Little2017}

\subsection{Personal Projects: A Deeper Dive}
``Take a look at your own personal projects. For about 10 minutes, list the various personal projects you are pursuing at present. Don't agonize over what to write down. At any 1 time, we usually have a mixture of trivial pursuits \& magnificent obsessions. People usually list around 15 projects in 10 minutes, but when we remove time constraints we have seen lists ranging from 1 project to hundreds.\footnote{For details on the actual methods through which we do a detailed analysis of a personal project system, see Little \& Gee (2007).}

The content of your personal projects can be very revealing. The 1st time I administered this type of assessment, which I call Personal Projects Analysis (PPA), to university students, these were the 1st 2 personal projects generated by a male student: ``Clarify my philosophical beliefs,'' \& ``Get laid.''

Both the lofty \& lusty pursuits of this undergraduate seemed to capture a certain reality of students' academic \& social lives at the time.

We have now studied the personal projects of thousands of individuals \& have identified several major types of content. Here are the most frequent types of projects that adults engage in (with the most frequent 1st), together with some examples:
\begin{itemize}
	\item Occupational\texttt{/}Work: Make sure department budget is done.
	\item Interpersonal: Have dinner with the woman in the floppy hat.
	\item Maintenance: Get more bloody ink cartridges.
	\item Recreational: Take cruising holiday.
	\item Health\texttt{/}Body: Lose 15 pounds.
	\item Intrapersonal: Try to deal with my sadness.
\end{itemize}
Sometimes the mere listing of a person's projects gives a hint of a life story unfolding. You may have detected one in the list above. A splendid example of this is a short story in \textit{The Guardian} by Jennifer Egan.\footnote{See Egan (2011).} It is simply called ``To Do,'' \& the 1st few entries are: 1. Mow lawn. 2. Get rid of that fucking hose. 3. Wash windows. 4. Spay cat. 5. Dye hair. 6. Do tarot cards. 7. Pick up kids. 8. Drop off kids at Mom's. 9. Buy wig. 10. See if small removable portion of fence can be cut QUIETLY $\ldots$. 11. Send warning letter. a. Newspaper cutouts? b. Get kids to write it? c. Write with left hand? d. Be vague. ``Certain unpleasant things''. 12. Mail letter. a. Or drop it off while wearing wig. 13. Renew meds.

The project ``to-do'' list continues in this fashion \& is strikingly evocative, blending everyday parental pursuits with deadly pursuits, underscoring how one's projects can speak volumes about one's personality.'' -- \cite[pp. 31--32]{Little2017}

\subsection{Self Projects}
``Although they are comparatively infrequent in project lists, those we call \textit{intra}personal projects are especially interesting \& important. They are projects focused on the self, such as ``try to be less socially anxious,'' or ``become a better listener.'' Are these sorts of endeavors good or bad for us? The answer is complicated.

On the downside, such projects are known to be linked with feelings of depression \& vulnerability.\footnote{See the work of Katariina Salmela-Aro, who initiated research on this topic (Salmela-Aro, 1992).} If you have projects of this sort, you may find that you get into a kind of ruminative loop where you can't make progress. You overthink the change you feel you need to make \& over-scrutinize your (lack of) progress. On the upside, however, we also have evidence that engaging in intrapersonal projects can be associated with aspects of creativity \& openness to experience.\footnote{See Little (1989).} Why, on the one hand, is an intrapersonal project associated with negative emotions \& vulnerability, \& on the other, is seen as a creative adventure? It is likely due to the \textit{origin} of the self-focused project.

Did you include a self-change project on your list? Who told you to listen better, or to get on top of your social anxiety? Who told you to get your values straightened out, you degenerate, before you find yourself sucked into a destructive lifestyle that you can't escape? If it is a parent, boss, or lover who generates the project, it is more likely to create negative emotions than if you yourself were the initiator. There is no a significant body of research demonstrating that ``intrinsically regulated'' project pursuit will be more successful \& lead to greater well-being than ``externally regulated'' pursuit.\footnote{} If you included self projects on your list, then ask yourself who instigated it? If they spring from your own vision of a possible self, you are likely to feel better while pursuing them. \& those projects are ultimately more likely to succeed. Those initiated by others might be willingly undertaken. But if they are forced or coerced, they may be nonstarters.'' -- \cite[pp. 32--33]{Little2017}

\subsection{``Trying'' Projects}
``How else can you improve your project experience \& your odds of eventual success? Even the way in which you \textit{phrase} your personal projects has implications for how successful they will be. Some people engage in ambivalently framed projects like ``try to be more sensitive to my idiotic coworkers'' or ``maybe apologize to my sister for calling her boyfriend creepy.'' Canadian psychologist Neil Chambers has shown that such ``trying'' individuals are less likely to accomplish their projects than those who phrase their pursuits more directly.\footnote{} Were the projects to be rephrased as ``\textit{be} more sensitive'' \& simply ``apologize,'' they would be more likely to benefit your idiotic coworkers \& your sister (not to mention yourself).'' -- \cite[p. 33]{Little2017}

\subsection{How's It Going? Appraisals of Personal Projects}

%------------------------------------------------------------------------------%

\section{Personal Contexts: The Social Ecology of Project Pursuit}

%------------------------------------------------------------------------------%

\section{The Myth of Authenticity: The Challenge of Being Oneself}

%------------------------------------------------------------------------------%

\section{Well-Doing: The Sustainable Pursuit of Core Projects}

%------------------------------------------------------------------------------%

\printbibliography[heading=bibintoc]
	
\end{document}