\documentclass{article}
\usepackage[backend=biber,natbib=true,style=authoryear]{biblatex}
\addbibresource{/home/nqbh/reference/bib.bib}
\usepackage{tocloft}
\renewcommand{\cftsecleader}{\cftdotfill{\cftdotsep}}
\usepackage[colorlinks=true,linkcolor=blue,urlcolor=red,citecolor=magenta]{hyperref}
\usepackage{algorithm,algpseudocode,amsmath,amssymb,amsthm,float,graphicx,mathtools}
\allowdisplaybreaks
\numberwithin{equation}{section}
\newtheorem{assumption}{Assumption}[section]
\newtheorem{conjecture}{Conjecture}[section]
\newtheorem{corollary}{Corollary}[section]
\newtheorem{definition}{Definition}[section]
\newtheorem{example}{Example}[section]
\newtheorem{lemma}{Lemma}[section]
\newtheorem{notation}{Notation}[section]
\newtheorem{principle}{Principle}[section]
\newtheorem{problem}{Problem}[section]
\newtheorem{proposition}{Proposition}[section]
\newtheorem{question}{Question}[section]
\newtheorem{remark}{Remark}[section]
\newtheorem{theorem}{Theorem}[section]
\usepackage[left=0.5in,right=0.5in,top=1.5cm,bottom=1.5cm]{geometry}
\usepackage{fancyhdr}
\pagestyle{fancy}
\fancyhf{}
\lhead{\small Sect.~\thesection}
\rhead{\small\nouppercase{\leftmark}}
\renewcommand{\sectionmark}[1]{\markboth{#1}{}}
\cfoot{\thepage}
\def\labelitemii{$\circ$}

\title{Psychopath Free: Recovering From Emotionally Abusive Relationships with Narcissists, Sociopaths, \& Other Toxic People}
\author{Jackson MacKenzie}
\date{\today}

\begin{document}
\maketitle
\tableofcontents
\vspace{5mm}

%------------------------------------------------------------------------------%

\textit{Connect.} \url{psychopathfree.com}, \url{facebook.com/psychopathfree}, \url{twitter.com/psychopathfree}.

\begin{quotation}
	``No hurt survives for long without our help, she said \& then she kissed me \& sent me out to play again for the rest of my life.'' -- Brian Andreas, \textit{Story People}
\end{quotation}

\section{Introduction}

\subsection{An Adventure}
``Finding yourself involved with a psychopath is an adventure, that's fore sure. It will open your eyes to human nature, our broken society, \&, perhaps most important of all, your own spirit. It's a dark journey that will throw you into spells of depression, rage, \& loneliness. It will unravel your deepest insecurities, leaving you with a lingering emptiness that haunts your every breath.

But ultimately, it will heal you.

You will become stronger than you could ever imagine. You will understand who you are truly meant to be. \& in the end, you will be glad it happened.

No one ever believes me about that last part. At least, not at 1st. But I promise you, it's an adventure worth taking. One that will change your life forever.

\textit{So what is a psychopath? How about a narcissist or a sociopath?} They're manipulative people -- completely devoid of empathy -- who intentionally cause harm to others without any sense of remorse or responsibility. \& despite some differences between each disorder, the bottom line is that their relationship cycles can be predicted like clockwork: Idealize, Devalue, Discard.

Years ago, this cycle had me thinking I'd never be happy again. Falling in love had somehow wiped out my entire sense of self. Instead of being joyful \& trusting, I had become an unrecognizable mess of insecurities \& anxiety.

But life is a lot of fun these days -- mostly just running around outside in my bathing suit \& eating pizza. \& this is all thanks to a lucky Google search that lead me to psychopathy, which led me to the friends who saved my life, which led us to cofound a tiny online recovery community, which now reaches millions of survivors every month!

At \url{PsychopathFree.com}, we see new members join every single day, always with a seemingly hopeless \& all-too-familiar tale. Left feeling lost \& broken, they wonder if they will ever find happiness again.

1 year later, that person is nowhere to be found.

In his or her place, there is a beautiful stranger who stands tall \& helps others out from the shadows. A stranger who takes pride in their own greatest qualities: empathy, compassion, \& kindness. A stranger who speaks of self-respect \& boundaries. A stranger who practices introspection in order to better conquer their own demons.

So what happened in that year?

Well, a lot of good stuff. So much that I had to write a book. I might be biased -- actually, I definitely am -- but I think \url{PsychopathFree.com} has 1 of the coolest healing process out there. We believe in education, open dialogue, validation, \& self-discovery. We have a uniquely inspiring user base, full of resilient values \& honest friendships.

Yes, friendships. Because this journey is personal, but it's also remarkably universal. Whether it be a whirlwind romance, a scheming coworker, an abusive family member, or a life-consuming affair, a relationship with a psychopath is always the same. Your mind is left spinning. You feel worthless \& lost. You become numb to the things that once made you happy.

I cannot fix a toxic relationship (because toxic people cannot change), but I can give you a new place to start. \& I can promise that you will feel joy again. You will learn to trust your intuition. You will walk this world with the wisdom of a survivor \& the gentle wonder of a dreamer.

But 1st, you'll need to forget everything you thought you knew about people. Understanding psychopathy requires letting go of your basic emotional instincts. Remember, these are people who prey on forgiveness. They thrive on your need for closure. They manipulate compassion \& exploit sympathy.

Since the dawn of time, psychopaths have waged psychological warfare on others -- humiliating \& shaming kind, unsuspecting victims -- people who never asked for it; people who aren't even aware of the war until it's over.

But this is all about to change.

So say farewell to love triangles, cryptic letters, self-doubt, \& manufactured anxiety. You will never again find yourself desperately awaiting a text from the person you love. You will never again censor your spirit for fear of losing the ``perfect'' relationship. You will never again be told to stop overanalyzing something that urgently needs analysis. You are no longer a pawn in the mind games of a psychopath. You are free.

\& now it's time for your adventure.

Love, Jackson.'' -- \cite[pp. 7--9]{MacKenzie2015}

%------------------------------------------------------------------------------%

\section{Spotting Toxic People}
``Your strengthened intuition is the greatest defense against a manipulative person. It is a skill that can never be exploited -- \& once learned, it will serve you a lifetime.'' -- \cite[p. 10]{MacKenzie2015}

\subsection{30 Red Flags}
``There are a lot of phenomenal studies on the traits \& characteristics of psychopaths. A quick Internet search will lead you to them. The red flags in this book are intended to supplement these resources.

So what's different about this list? Well, for one, it's specifically about relationships. But it's also about you. Each point requires introspection \& self-awareness. Because if you want to spot toxic people, you cannot focus entirely on their behavior -- that's only half the battle. You must also come to recognize the looming red flags in your own heart. Then you will be ready for anything.
\begin{itemize}
	\item \textbf{Gaslighting \& crazy-making.} The blatantly deny their own manipulative behavior \& ignore evidence when confronted with it. They become dismissive \& critical if you attempt to disprove their fabrications with facts. Instead of them actually addressing their inappropriate behavior, somehow it always becomes \textit{your} fault for being ``sensitive'' \& ``crazy.'' Toxic people condition you to believe that the problem isn't the abuse itself, but instead your reactions to their abuse.
	\item \textbf{Cannot put themselves in your shoes, or anyone else's, for that matter.} You find yourself desperately trying to explain how they might feel if you were treating them this way, \& they just stare at you blankly. You slowly learn not to communicate your feelings with them, because you're usually met with silence or annoyance.
	\item \textbf{The ultimate hypocrite.} ``Do as I say, not as I do.'' They have extremely high expectations for fidelity, respect, \& adoration. After the idealization phrase, they will give none of this back to you. They will cheat, lie, criticize, \& manipulate. But you are expected to remain perfect, otherwise you will promptly be replaced \& deemed unstable.
	\item \textbf{Pathological lying \& excuses.} There is always an excuse for everything, even things that don't require excusing. They make up lies faster than you can question them. They constantly blame others -- it is never their fault. They spend more time rationalizing their behavior than improving it. Even when caught in a lie, they express no remorse or embarrassment. Oftentimes, it almost seems as if they \textit{wanted} you to catch them.
	\item \textbf{Focuses on your mistakes \& ignores their own.} If they're 2 hours late, don't forget that you were once 5 minutes late to your 1st date. If you point out their inappropriate behavior, they will always be quick to turn the conversation back to you. You might begin to adopt perfectionist qualities, very aware that any mistake can \& will be used against you.
	\item \textbf{You find yourself explaining the basic elements of human respect to a full-grown man or woman.} Normal people understand fundamental concepts like honesty \& kindness. Psychopaths often appear to be childlike \& innocent, but don't let this mask fool you. No adult should need to be told how he or she is making other people fee.
	\item \textbf{Selfishness \& a crippling thirst for attention.} They drain the energy from you \& consume your entire life. Their demand for adoration is insatiable. You thought you were the only one who could make them happy, but now you feel that anyone with a beating pulse could fit the role. However, the truth is: no one can fill the void of a psychopath's soul.
	\item \textbf{Accuses you of feeling emotions that they are intentionally provoking.} They call you jealous after blatantly flirting with an ex -- often done over social networking for the entire world to see. They call you needy after intentionally ignoring you for days on end. They use your manufactured reactions to garner sympathy from other targets, trying to prove how ``hysterical'' you've become. You probably once considered yourself to be exceptionally easygoing person, but an encounter with a psychopath will (temporarily) turn that notion upside down.
	\item \textbf{You find yourself playing detective.} It's never happened in any other relationship, but suddenly you're investigating the person you once trusted unconditionally. If they're active on Facebook, you start scrolling back years on their posts \& albums. Same with their ex. You're seeking answers to a feeling you can't quite explain.
	\item \textbf{You are the only one who sees their true colors.} No matter what they do, they always seem to have a fan club cheering for them. The psychopath uses these people for money, resources, \& attention -- but the fan club won't notice, because this person strategically distracts  them with shallow praise. Psychopaths are able to maintain superficial friendships far longer than relationships.
	\item \textbf{You hear that any fight could be your last.} Normal couples argue to resolve issues, but psychopaths make it clear that negative conversations will jeopardize the relationship, especially ones regarding their behavior. Any of your attempts to improve communication will typically result in the silent treatment. You apologize \& forgive quickly, otherwise you know they'll lose interest in you.
	\item \textbf{Slowly \& steadily erodes your boundaries.} They criticize you with a condescending, joking sort of attitude. They smirk when you try to express yourself. Teasing becomes the primary mode of communication in your relationship. They subtly belittle your intelligence \& abilities. If you point this out, they call you sensitive \& crazy. You might begin to feel resentful \& upset, but you learn to push away those feelings in favor of maintaining the peace.
	\item \textbf{They withhold attention \& undermine your self-esteem.} After once showering you with nonstop attention \& admiration, they suddenly seem completely bored by you. They treat you with silence \& become very annoyed that you're interested in continuing the passionate relationship that \textit{they} created. You begin to feel like a chore to them.
	\item \textbf{They expect you to read their mind.} If they stop communicating with you for several days, it's your fault for not knowing about the plans they never told you about. They will always be an excuse that makes them out to be the victim to go along with this. They make important decisions about the relationship \& they inform everyone except you.
	\item \textbf{You feel on edge around this person, but you still want them to like you.} You find yourself writing off most of their questionable behavior as accidental or insensitive, because you're in constant competition with others for their attention \& praise. They don't seem to care when you leave their side -- they can just as easily move on to the next source of energy.
	\item \textbf{An unusual number of ``crazy'' people in their past.} Any ex-partner or friend who did not come crawling back to them will likely be labeled jealous, bipolar, an alcoholic, or some other nasty smear. Make no mistake: they will speak about you the same way to their next target.
	\item \textbf{Provokes jealousy \& rivalries while maintaining their cover of innocence.} They once directed all of their attention to you, which makes it especially confusing when they begin to withdraw \& focus on other people. They do things that constantly make you doubt your place in their heart. If they're active on social media, they'll bait previously denounced exes with old songs, photos, \& inside jokes. They attend to the ``competition's'' activity \& ignore yours.
	\item \textbf{Idealization, love-bombing, \& flattery.} When you 1st meet, things move extremely fast. They tell you how much they have in common with you -- how perfect you are for them. Like a chameleon, they mirror your hopes, dreams, \& insecurities in order to form an immediate bond of trust \& excitement. They constantly initiate communication \& seem to be fascinated with you on every level. If you have a Facebook page, they might plaster it with songs, compliments, poems, \& inside jokes.
	\item \textbf{Compares you to everyone else in their life.} They compare you to ex-lovers, friends, family members, \& your eventual replacement. When idealizing, they make you feel special by telling you how much better you are than these people. When devaluing, they use these comparisons to make you feel jealous \& inferior.
	\item \textbf{The qualities they once claimed to admire about you suddenly become glaring faults.} At 1st, they appeal to your deepest vanities \& vulnerabilities, observing \& mimicking exactly what they think you want to hear. But after you're hooked, they start to use these things against you. You spend more \& more time trying to prove yourself worthy to the very same person who once said you were perfect.
	\item \textbf{Cracks in their mask.} There are fleeting moments when the charming, cute, innocent persona is replaced by something else entirely. You see a side to them that never came out during the idealization phase, \& it is a side that's cold, inconsiderate, \& manipulative. You start to notice that their personality just doesn't add up -- that the person you fell in love with doesn't actually seem to exist.
	\item \textbf{Easily bored.} They are constantly surrounded by other people, stimulated \& praised at all times. They can't tolerate being alone for an extended period of time. They become quickly uninterested by anything that doesn't directly impact them in a positive or thrilling way. At 1st, you might think they're exciting \& worldly, \& you feel inferior for preferring familiarity \& consistency.
	\item \textbf{Triangulation.} They surround themselves with former lovers, potential mates, \& anyone else who provides them with added attention. This includes people that the psychopath may have previously denounced \& declared you superior to. This makes you feel confused \& creates the perception that the psychopath is in high demand at all times.
	\item \textbf{Covert abuse.} From an early age, most of us were taught to identify physical mistreatment \& blatant verbal insults, but with psychopaths, the abuse is not so obvious. You likely won't even understand that you were in an abusive relationship until long after it's over. Through personalized idealization \& subtle devaluation, a psychopath can effectively erode the identity of \textit{any} chosen target. From an outsider's perspective, you will appear to have ``lost it,'' while the psychopath calmly walks away, completely unscathed.
	\item \textbf{Pity plays \& sympathy stories.} Their bad behavior always has sob-story roots. They claim to behave this way because of an abusive ex, an abusive parent, or an abusive cat. They sat that all they've ever wanted is some peace \& quiet. They say they hate drama -- \& yet there's more drama surrounding them than anyone you've ever known.
	\item \textbf{The mean \& sweet cycle.} Sometimes they shower you with attention, sometimes they ignore you, sometimes they criticize you. They treat you differently in public than they do behind closed doors. They could be talking about marriage 1 day \& breaking up the next. You never know where you stand with them. As my morning-coffee friend Rydia wrote: ``They put forth as little effort as possible \& then step it up when you try to disengage.''
	\item \textbf{This person becomes your entire life.} You're spending more of your time with them \& their friends, \& less time with your own support network. They're all you think \& talk about anymore. You isolate yourself in order to make sure you're available for them. You cancel plans \& eagerly wait by the phone for their next communication. For some reason, the relationship seems to involve a lot of sacrifices on your end, but very few on theirs.
	\item \textbf{Arrogance.} Despite the humble, sweet image they presented in the early stages, you start to notice an unmistakable air of superiority about them. They talk down to you as if you are intellectually deficient \& emotionally unstable. They have no shame when it comes to flaunting new targets after the breakup, ensuring that you see how happy they are without you.
	\item \textbf{Backstabbing gossip that changes on a whim.} They plant little seeds of poison, whispering about everyone, idealizing them to their face, \& then complaining about them behind their backs. You find yourself disliking or resenting people you've never even met. For some reason, you might even feel special for being the one he or she complains to. But once the relationship turns sour, they'll turn back to everyone they once insulted to you, lamenting about how crazy you've become.
	\item \textbf{Your feelings.} Your natural love \& compassion has transformed into overwhelming panic \& anxiety. You apologize \& cry more than you ever have in your life. You barely sleep, \& you wake up every morning feeling anxious \& unhinged. You have no idea what happened to your old relaxed, fun, easygoing self. After a run-in with a psychopath, you will feel insane, exhausted, drained, shocked, \& empty. You tear apart your entire life -- spending money, ending friendships, \& searching for some sort of reason behind it all.
\end{itemize}
You will find that normal, loving people do not raise any of these flags. After an encounter with a psychopath, most survivors face the struggle of hypervigilance: Who can really be trusted? Your gauge will swing back \& forth for a while, like a volatile pendulum. You will wonder if you've gone absolutely mad -- wanting to believe the best in an old friend or a new date, but feeling sick to your stomach when you actually spend time with them because you're waiting for the manipulative behavior to start.

Developing your intuition is a personal process, but I would leave you with this: the world is mostly full of good people, \& you don't want to miss out on that because you've been hurt. Spend some time getting in touch with your feelings. Keep tweaking until you find a comfortable balance between awareness \& trust. Look within \& understand why you felt the way you did when you were with your abusive partner \& how you felt before you met them. You will discover that many old relationships may need revisiting. \& as you begin to abandon toxic patterns, healthier ones will inevitably appear in their place.

To quote a longtime member \& friend, Phoenix, you will stop asking ``Do they like me?'' \& start asking ``Do I like them?'''' -- \cite[pp. 10--17]{MacKenzie2015}

\subsection{What Is Normal?}
``If your ``soul mate'' went from fascinated to bored in the blink of an eye, this is not normal. If you were called jealous \& crazy by someone who actively cheated on you, this is not normal. If you were desperately waiting by your phone for texts they once initiated on a minute-by-minute basis, this is not normal. If all of their exes were ``bipolar'' or ``madly in love'' with them, this is not normal. Psychopaths are parasitic, emotionally stunted, \& incapable of change. Once this individual is gone from your life, you will find that everything begins to make sense again. The chaos dissipates \& your sanity returns. Things will be normal once again.'' -- \cite[p. 17]{MacKenzie2015}

\subsection{Beware the Vultures}
``You are taking the 1st steps to recover from a toxic relationship with a psychopath. That's great! The work you'll be doing will not only free you from the grasp of your abuser, but it will also enable you to reclaim yourself -- the self that was trampled on, beaten down, \& transformed into a shell of who you once were. I know it may be difficult to face some of the truths we'll be exploring, but it's also empowering, as you'll see how much you've survived, how strong you really are.

As you begin this work, I strongly encourage you to seek out a recovery professional or a healing community. You'll need the support \&, at times, an encouraging reminder that you're on the right path.

I'd like to extend a special warning to those of you who are new to recovery. After psychopathic abuse, you're going to be extremely raw \& vulnerable. As you start to put the pieces together, you'll feel devastated, miserable, \& angry. It's overwhelming.

You're probably used to repressing your emotions \& dealing with things on your own. But this time, everything is out in the open. You're dependent like a newborn child, seeking out someone -- anyone -- to understand what you're going through.

In general, it's important to be open with your emotions. But at your most insecure moments, you may unknowingly open the floodgates for more abuse.

It's no mystery that survivors seem to attract more pathological people like magnets. As you frantically share your story, you latch on to the quickest \& most sympathetic ear -- anyone who claims to understand you. The problem is, these people do not always have your best interests at heart.

Those willing to listen to your psychopathic story for hours on end are, unfortunately, not likely to be people who are truly invested in your recovery. They are most likely ``vultures.''

Vultures often seem exceptionally kind \& warm at 1st. They want to fix you \& absorb your problems. They are fascinated by your struggles. But sooner or later, you will find yourself lost in another nightmare. They begin drowning you in unsolicited advice. They need constant praise \& attention. You are never allowed to disagree with them. They feel off drama \& an insatiable need to be appreciated by others.

You will find that they lash out as you become happier. They perceive your progress as a threat to their control. They want to keep you in a perpetual state of dependency. They do not want you to seek help from anyone except them.

Whether these people are pathological or not, you don't need this toxic garbage after what you've been through.

I would strongly urge all survivors to avoid seeking out new friendships \& relationships for at least a few months. You must get to the point where you no longer need -- or want -- to talk about your abuser anymore.

When you do need help, stick to professional therapy or recovery communities \& services. These people know what you've been through, \& you're going to find that all of them are willing to help -- with no strings attached.

I understand the temptation to go out \& meet new people. You're looking to start rebuilding your life. You want to surround yourself with kinder \& more genuine friends.

\& you will.

But real friends won't be acting as your therapist, \& they definitely won't be rambling on about their ability to empathize \& care. Their actions should speak louder than their words.

It takes a long time to start building healthier relationships. It takes breaking old habits, forming new ones, developing your intuition, \& finally coming to understand what it is that you want from this world.

So be on the lookout for vultures. In the writing world, there's a universal rule called ``show -- don't tell.'' This rule also applies to people. If you encounter someone who's constantly telling you who they are, how much they want to help you, how they will make things right for you, take a step back \& look at their actual behavior. Manipulative people are always ``telling'' because they have nothing good to show. Their inappropriate \& dishonest actions never actually match up with their promising words, causing an overwhelming cognitive dissonance in the people who trust them.

You will find the decent, humble human beings aren't trying to tell you who they are \& what they can do for you. They simply show it through consistent love \& kindness. You never need to question them, because their intentions are always pure. Vultures, on the other hand, are really acting out of self-interest; they want to be praised \& adored. In an argument, a ``teller'' will frequently remind you of how well they treat you, even after blatantly hurting you. A ``show-er'' will simply share their point of view without trying to twist the conversation in their favor. Avoid those who tell you how nice they are, how generous they are, how successful they are, how honest they are, \& how important they are. Instead, search for the quiet ones who show these qualities everyday through their actions.'' -- \cite[pp. 17--19]{MacKenzie2015}

\subsection{The Constant}
``You know about psychopaths. You've got the red flags. So now the big question: \textit{Are you really involved with a psychopath?}

Well, barring any major scientific advancements, you really can't know for certain whether or not someone has a conscience. In fact, I don't think there's any approach that will allow you to spot a psychopath with 100\% confidence.

Fortunately, there's a different way to keep yourself safe. \& this one involves looking within. It will work with anyone, anywhere, anytime. It's a question with answers -- lots of them.

``How are you feeling today?''

Seriously, I'm asking you. Because most people might respond with a vague ``fine'' \& follow up with a casual comment about their weekend, a promotion at work, or their favorite television show.

But what about you? Perhaps you're feeling empty? Broken? Hopeless? Maybe you woke up with that constant aching in your heart, eating away at your soul like a cancer. You spend the day trying to keep your thoughts free from painful topics -- only to find that your mind keeps racing right back to them. Memories that once brought you so much joy now make you feel sick. You oscillate between anger \& depression because you are unable to decide which one hurts less.

Those are answers.

So when you feel those things after a relationship, \textit{does it really matter if your ex was a psychopath, a sociopath, a narcissist, or a garden-variety jerk?} The label doesn't make your feelings any more or less valid. Your feelings are absolutes. They will endure, no matter which word you settle upon.

\& here's what you know from those feelings: someone uprooted your life, introducing a new kind of anxiety that you've never felt before. They've introduced you to a whole range of horrible emotions that make each day seem unbearable. During the relationship, you may have felt constantly on edge \& unhinged, worried that any mistake could mark the end of your dream. Maybe you found yourself desperately comparing yourself to other people, trying to win back your rightful place by your partner's side.

So I ask you again, \textit{does it matter if they are a psychopath by definition?}

You already have everything you need to know -- from your own feelings. You felt horrible around them, right? So during the relationship, why wasn't that enough to confirm that they should have no place in your life?

Because you were groomed \& idealized. You were tricked into falling in love -- the strongest of all human bonds -- so that your feelings could be more easily manipulated.

Toxic people condition us to ignore our intuition, \& we must learn to trust it again. Instead of judging outwardly, we need to perceive inwardly. When we start focusing on our own feelings, this is where the heading begins. \& if you are anything like me, we can agree on this simple truth: good people make you feel good \& bad people make you feel bad.

Everything else falls into place from there.

Don't listen to the folks who say your feelings should be totally independent of the world around you. If you've got an open heart, that's impossible. As human beings, we have this incredible gift -- the ability to make another person feel wonderful. With a word, a gesture, or a quiet smile. It's what makes the world beautiful. Some people would call this love.

But you experienced an abuser, someone who manipulated this gift in order to cause pain. \& now you want to know how to avoid people like this so it'll never happen again. You're worried that you've become hypervigilant -- untrusting of everyone \& everything around you. You feel that you need a little something extra. Something beyond your intuition.

So this is where I'd like to introduce the idea of a Constant. Your Constant will comfort \& protect you throughout this book, \& for the rest of your life.

Think of someone you love. Someone who consistently inspires \& never disappoints. It could be anyone -- your mom, a close friend, your children, your cat, a deceased relative. Really, anyone. You might feel that you have no Constant. Of course you do; you can even dream one up. Imagine a higher power in your mind -- one that brings peace to your heart. Colorful, glowing, \& full of life. Embodying all of the qualities you admire most: empathy, compassion, kindness. A gentle spirit who will always keep you safe. \& voil\`a, you have a Constant.

So now that you've got a Constant in mind (tangible or imagined), I have some questions. Does your Constant make you feel unhinged? Anxious? Jealous? Does your heart rise up into your throat when they speak to  you? When you're away from your Constant, do you spend hours analyzing their behavior \& defending yourself from hypothetical arguments?

Of course not.

So why is that? Why can one dismissive person make you doubt everything good going on in your life? What's the difference between your Constant \& the people who make you feel like garbage?

If you can't answer these questions quite yet, you're not alone. \& that's the beauty of it all. You do not need to understand why you don't like being around a person. You have a Constant, \& that's all you need to know for now. Self-respect comes later.

Your Constant is a private reminder that you are not crazy, even when it feels like you're taking on the entire world. With time, you will begin to filter out the people who make you feel bad. You realize that you do not need to put up with negativity when there is a Constant who brings out the best in you.

Once you become more comfortable with the idea, you'll be ready to ask the most important question of all: Shouldn't I feel this same kind of peace with everyone in my life?

Absolutely. So let's get started.'' -- \cite[pp. 19--22]{MacKenzie2015}

%------------------------------------------------------------------------------%

\section{The Manufactured Soul Mate}
``Perhaps most insidious of all the psychopath's evils: their relationship cycle, during which they gleefully \& systematically wipe out the identity of an unsuspecting victim. Cold \& calculated emotional rape.'' -- \cite[p. 23]{MacKenzie2015}

\subsection{Personalized Grooming}
``The psychopath trains you to become the perfect partner. In a matter of weeks, they take over your entire life, consuming your mind \& body with unrivaled pleasure. Ultimately, you are to become their newest source of endless adoration \& praise -- but 1st, you must fall in love. Then your heart will be open to their every suggestion. There are 3 key components to this process: idealization, indirect persuasion, \& testing the waters.'' -- \cite[p. 24]{MacKenzie2015}

\subsubsection{Idealization}
``The idealization phase in a psychopathic relationship will be unlike anything you've ever experienced. You will be swept off your feet, lost in a passionate fantasy with someone who excites you on every level: emotionally, spiritually, \& sexually. They will be the 1st thing on your mind when you wake up in the morning, waiting for their cheerful, funny texts to start your day. You will quickly find yourself planning a future with them -- forgetting about the dull realities of life. None of that matters anymore. They're the person you want to spend the rest of your life with.

While all of this is going on in your heart, their thoughts are occupied by something else entirely: ``Good. It's working.''

Psychopaths never truly feel the things they display. They're observing you, mirroring your every emotion, \& pretending to ride this high with you.

Because the higher you rise, the lower you'll fall.

Idealization is the 1st step in the psychopath's grooming process. Also known as \textit{love-bombing}, it quickly breaks down your guard, unlocks your heart, \& modifies your brain chemicals to become addicted to the pleasure centers firing away. The excessive flattery \& compliments play on your deepest vanities \& insecurities -- qualities you likely don't even know you possess.

They will feed you constant praise \& attention through your phone, Facebook Timeline, \& email inbox. Within a matter of weeks, the 2 of you will have your own set of inside jokes, pet names, \& cute songs. Looking back, you'll see how insane the whole thing was. But when you're in the middle of it, you can't even imagine life without them.

\textit{So how did they do it?}

Aside from gifts \& poems, the psychopath uses a variety of brainwashing techniques to win you over. They will emphasize 6 major points during the idealize process:

\paragraph{1. We have So Much in Common.} \textit{We see the world the same way. We have the same sense of humor. We're both so empathetic, constantly helping out our friends \& family members. We are perfect for each other.}

The psychopath repeatedly drills these points home, oftentimes even going so far as to say: ``We're practically the same person.'' During the grooming phase, psychopaths observe \& mimic. They steal qualities from their victims, \& almost seem to become a ``better'' version of their target's personality -- co-opting of all of the cheerful positives, without any of the burdensome emotions that come along with them. But this is all an act. These amplified, mirrored qualities are nothing more than a facade. Psychopaths don't truly feel or understand any of the things they imitate. They are able to offer a superficial \& flattering copy of their victim's personality. Nothing more. None of the depth, compassion, \& empathy that come along with being human. Like everything else they have to offer, their copycat personalities are hollow \& empty.

The psychopath will spend most of the idealization phase listening to you \& excitingly responding that he or she feels the same way as you do. You will eventually come to think that they're the only person you'll ever meet who's so similar to you. \& you're right. Because it is flat-out impossible (\& creepy) for 2 people to be identical in every way.

Normal people have differences. It's what makes life interesting. But psychopaths can skip this complication because they don't have an identity. They do not have a sense of self. They don't have life experiences that shape their needs, insecurities, \& fantasies. Instead, they steal yours. Like a chameleon, they will mimic every part of your personality to become your perfect match.

\paragraph{2. We Have the Same Hopes \& Dreams.} The psychopath will consume your present life, but they will also take over your future. In order to raise the stakes in the relationship, they will make many long-term promises. This ensures that you are highly invested in the relationship. \textit{After all, who wants to stick around for a romance that has no future?}

The psychopath takes this a step further, quickly discussing major life events like marriage \& moving in together. These are decisions that typically take years to arrive in a healthy relationship. But you don't need all that time. You already know you'll be spending the rest of your life with them. If you've always dreamed of a family \& kids, they will fit that role perfectly. If you want to start a business, they will be your right-hand man or woman. If you're in an unhappy marriage, they will have a plan ready to replace your spouse. (What you might not notice until later is that these plans always seem to involve some sort of sacrifice on your end -- never theirs.)

\paragraph{3. We Share the Same Insecurities.} The psychopath will never actually mention your vulnerabilities, but they can sniff them out in a second. Then they will mirror your insecurities to drive up your sympathy -- so that you attempt to heal their problems with the same care you might hope to receive yourself.

As an empathetic person, you are naturally drawn to offer comfort to people who are hurting or vulnerable. This inclination to comfort increases when you also recognize someone else's insecurities as your own. You see someone feeling inferior, \& you believe that you know how to make them feel better.

The psychopath seems to genuinely adore all of your efforts. They compare you to their exes, idealizing you above everyone else. They praise your caring nature, which makes you want to do more for them. You feel that all of your efforts are appreciated, \& you want to do even more to prove how much you care. You see their insecurities \& perceive them as genuine, open, vulnerable, \& sympathetic -- someone you want to help. Psychopaths see insecurities in a very different way -- as a tool for manipulation \& control.

\paragraph{4. You Are Beautiful.} The psychopath is obsessed with the way you look. You will never meet another human being who comments so frequently on your clothes, your hair, your skin, your pictures, or whatever other superficial quality they choose to focus on that day. At 1st, these words feel like compliments. They can't believe how beautiful you are -- they don't even feel worthy of being your partner. They say they walk around the park \& can't find a person more attractive than you (how this is a compliment, I'm not quite sure).

Going along with the idea of insecurities, you begin to return all of this flattery. You want to make sure they feel adequate -- that they understood how attractive you think they are. \& that's what they're aiming for. By showering you with compliments, they know they can expect the adoration to rebound shortly. Suddenly they become very comfortable sharing photos of themselves with you. Your relationship becomes an unending exchange of praise \& approval.

You begin to place your self-esteem in their words, because they are so reliably positive. You can actually feel yourself glowing. Your body goes through changes as your confidence rises with their every word. You spend more \& more time improving your appearance to keep them impressed.

\paragraph{5. I've Never Felt This Way in My Life.} This is where the comparisons begin. They hold you in high regard, far above all of their other relationships. They explain -- in detail -- every 1 of the reasons you are better than their exes. They can't remember the last time they've been this happy.

You will constantly hear sweeping declarations like ``I can't believe how lucky I am.'' Statements like these play on your innate desire to make others happy. They convince you that you're providing them with a special sort of joy, something that they cannot find in anyone else. This becomes a point of pride for you -- knowing that you are the one they want, despite all of their other admirers.

The psychopath will refer to you as ``perfect'' \& ``flawless,'' which becomes an overwhelming source of cognitive dissonance when the words inevitably change to ``crazy'' \& ``jealous.'' As you work through these memories, remember that their compliments were always shallow \& calculated. They use these tactics with everyone. For each target, the idealization phase will be different. However, 1 thing remains true throughout each relationship: they really have ``never felt this way'' in their life. Psychopaths do not actually feel the love \& happiness that they so frequently proclaim. They oscillate between contempt, envy, \& boredom. Nothing more.

\paragraph{6. We Are Soul Mates.} Psychopaths love the idea of soul mates. It implies something different from love. It implies that there are higher powers at work. That you are meant to be together. It means that they consume your entire being -- mind \& body. It creates a psychic bond that lasts long after the relationship has ended.

Perhaps there is a small part in all of us that longs for a soul mate -- the perfect person to complete our lives, someone with whom we can share everything, a lover \& a best friend.

\& there is nothing wrong with this. I cannot stress the point enough. Psychopaths will manipulate your dreams \& fantasies, but that does not invalidate your dreams \& fantasies \& make them weaknesses.

After being discarded by a psychopath, many survivors denounce everything about their past life, raising a permanent guard to protect themselves from more abuse. Please don't do this.

If you believe in soul mates, you will find a real one. You will meet a man or woman who is full of gentle compassion \& kindness. You will never question your heart because of them. Your love will blossom on its own, without all of the manufactured intensity. The psychopath was not your soul mate, \& they never will be. To be your soul mate, they would -- of course -- need to have a soul.

After reading the above list, you may feel angry with yourself for falling for this duplicity. ``How could I be so stupid?'' you might ask. But please, don't beat yourself up. You weren't targeted because you were stupid. On the contrary, you were chosen because of so many good qualities you possess. A psychopath's perfect target is idealistic, forgiving, generous, \& romantic. Most targets are very selective about their partners, often feeling lonely \& frustrated by the dating scene. So when the psychopath comes along to mirror all of your greatest fantasies, you pour your entire heart \& soul into the relationship. You'll invest everything you can -- emotionally, financially, \& physically. You quickly feel comfortable opening up because the psychopath grooms you to believe you've found ``the one.'' This forms an immediate bond of trust \& familiarity.

However, when the psychopath begins the devaluing process, you'll attempt to absorb all of the blame in the relationship, in order to restore the perfect memory you had of the person who once claimed to be your soul mate. This is why psychopathy awareness is so important. Without the missing puzzle piece, it is only logical to assume that this ``soul mate'' existed at some point, \& might return again with enough love \& care. But once we understand psychology, we come to realize that this person never existed at all. It was a mirror image -- a carbon copy -- of everything we wanted in a partner. When psychopaths lose this element of surprise, their pool of victims diminishes significantly.'' -- \cite[pp. 24--29]{MacKenzie2015}

\subsubsection{Indirect Persuasion}
``After they've idealized you, they're ready to begin conditioning your behavior. Using indirect persuasion, psychopaths are able to make subtle suggestions that will ultimately be accepted by their victims. They maintain an illusion of innocence, since most people won't believe ``they made me feel these things.''

1 method they use involves the way they offer compliments. They will insult their exes as a way to flatter their target, but what they are really doing is grooming their target. E.g., by saying ``my ex always used to do this, but you never do that,'' they are \textit{telling} you to behave a certain way. This is not a compliment -- it's a warning that if you repeat any of the ex's alleged behavior, you'll be discarded as well. The ex likely didn't even do any of these things. It's just a way for the psychopath to indirectly tell you how they expect you to behave. Here are some of the most common examples: ``My ex \& I always fought. We never fight.'' ``My ex always needed to talk on the phone. You're not needy or demanding.'' ``My ex would always nag me about getting a job. You're so much more understanding.''

Let me say it again: These are not compliments. They are expectations. The psychopath has come up with a checklist of human traits \& emotions that bother them, \& now they're planting the idea in your mind: don't express these things, or else.

Now, when you fight, you will try to end it as quickly \& pleasantly as possible so you're not like their ex. When you haven't heard from them in 3 days, you won't call because you don't want to be like their ex. When they're sitting on their rear end, unemployed for 6 months, you won't say anything because you don't want to be like their ex.

Any deviation from this plan, \& you will receive the silent treatment or a sharp comment about your changed behavior -- a reminder that the idealization could end at any time.

This is why most survivors feel so much anger after the abuse has ended. You've been shoving aside your own intuition \& needs in order to be ``nice.'' You think you've been giving them some sort of special treatment that no one else can provide. \& then suddenly they go running back to the very same people they used to complain about. Meanwhile, you've been repressing the urge to tell them to get a job, or call more often, or just be a good partner. You pushed all of that away because you thought it was the only way to stay with them, to stay on their good side.

Just remember, normal, empathetic people do not make such comparisons about the people they love. \& they certainly don't keep a tally for everyone involved to see publicly. When you're truly in love, you don't need to convince yourself \& others that this experience is better than all of your past experiences. Likewise, if you're falling out of love, you don't need to convince yourself \& others that this experience is worse than all of your past experiences.

But psychopaths do this. Every single time. Because it's a strategically ambiguous way to influence your behavior.'' -- \cite[pp. 30--31]{MacKenzie2015}

\subsubsection{No Support}
``Psychopaths provide shallow praise \& flattery only in order to gain trust. When you actually need emotional support, they will typically offer an empty response -- or they will completely ignore you. With time, this conditions you not to bother them with your feelings, even when you need a partner the most, especially during times of tragedy or illness. You will begin to notice that you are never allowed to express anything but positive praise for them. \& even then, they will become bored soon enough \& move on to the next target. Unable to empathize with pain \& suffering, psychopaths cannot provide compassion during difficult times. This is why their ``support'' will always feel hollow \& mechanical at best.'' -- \cite[p. 31]{MacKenzie2015}

\subsubsection{The ``Crazy'' Ex}
``Psychopaths talk about their exes a lot -- more than any healthy individual with a new romantic partner should. After 1st making you feel like the only person in the world, they quickly try to evoke your pity by sharing stories about their nasty ex who's so very jealous of you \& your passionate new relationship. Because these stories are completely invented, they can \& absolutely will change on a whim. 1 day their ex is bipolar, the next day they're great friends, \& then finally the ex is crazy \& hysterical. \& before long, you will become the ``crazy ex'' used to lure in a new victim.

\textit{But what do all of these labels really mean? What purpose do they serve?}'' -- \cite[p. 32]{MacKenzie2015}

\subsubsection{``My Ex Is Bipolar''}
``Name-calling someone ``bipolar'' is like name-calling someone ``diabetic.'' Bipolar disorder is a crippling illness with a specific set of symptoms that are a bit more complicated than ``mood swings that I happen to dislike.'' Also called manic-depressive illness, it is characterized by unusual mood shifts, with reoccurring episodes of mania \& depression. While bipolar disorder is real, how likely is it that their ex was bipolar? More likely, it is an insult they throw around to evoke your sympathy. It shouldn't be surprising, then, that when your relationship ends you'll likely carry this label as well.

If you suddenly became ``bipolar'' after a relationship with someone, \& you've never been bipolar before, then you might want to think twice before accepting the diagnosis -- especially if that diagnosis came from your ex.

The thing about ``bipolar'' is that it's actually a perfect label for the psychopath's ideal victim. If you're naturally cheerful \& optimistic, these traits become your ``mania.'' Then your valid reactions to your partner's abuse become the ``depression.'' During the idealization phase, when the psychopath was charming \& mirroring your entire personality, you were walking on sunshine. Life was amazing. But when they began criticizing you \& cheating on you, so you became upset \& cried. They gave you the silent treatment, all the while dangling new \& former loves in your face. Did this upset you? Excellent. Voil\`a, you're bipolar!

It horrifies me to think about the number of victims who falsely diagnose themselves based on volatile emotions that were intentionally provoked by someone else. Most survivors find that it takes 1--2 years for their moods to fully restabilize. Until that point, please be very reserved about deciding what's wrong with you.

Note: Millions of adults truly do suffer from bipolar disorder. If you're genuinely concerned about your mental heath, please seek the opinion of a professional, not an ex-partner whose behavior drove you to a book called \textit{Psychopath Free}.'' -- \cite[p. 32]{MacKenzie2015}

\subsubsection{``My Ex Is Crazy \& Hysterical''}
``\& it's definitely not worth thinking about how they came to be that way, right?

Seriously, though, let's think about it. This insult implies 1 of 2 things:
\begin{itemize}
	\item \textbf{Their ex was always crazy \& hysterical, \& for some reason, they still decided to date that person.} Seems unhealthy, no?
	\item \textbf{Something changed during the relationship to make the ex that way.} What exactly could it be? Did they just snap 1 day, for absolutely no reason at all? Or did it maybe have something to do with the constant triangulation, lying, manipulating, \& criticizing? If someone tells you how ``crazy'' their ex is, you should take a step back \& \textit{really} rethink that one.
\end{itemize}
This characterization serves another purpose; it informs you about what is considered ``acceptable'' behavior. ``Crazy'' \& ``hysterical'' are words of invalidation, minimization, \& dismissal. They imply that the reactions this person displayed were over-the-top. You will be wary of acting this way, too. This strategy encourages you to stop reacting, \& thereby to stop standing up for yourself. By making you question your own sanity, the psychopath is able to take the spotlight away from their own abusive behavior.'' -- \cite[p. 33]{MacKenzie2015}

\subsubsection{``My Ex Is Bitter''}
``What the heck? Seriously, what does this even mean? It's like punching someone in the face \& then saying ``you're bitter.'' Well, yeah, that person is bitter because you punched them in the face. Does saying ``you're bitter'' somehow make the bitterness inappropriate?

Again, it's about minimization \& dismissal. After their abusive behavior, lying, \& mind games, the psychopath expects their victim to simply shut up or grovel. That's it. Any signs of anger or disbelief are equated with bitterness. The psychopath will then commiserate with their new partner about their ex being childish \& holding grudges, neglecting to mention all of the details that suggest why that person might be bitter in the 1st place.'' -- \cite[p. 34]{MacKenzie2015}

\subsubsection{``My Ex Is Jealous of Us \& Still in Love with Me''}
``1st of all, who brags about something like this? It's so off-putting, \& even if it's true, this sort of pigheaded arrogance should be avoided in any sort of romantic endeavor.

Digging deeper, we should also examine \textit{why} this person is jealous \& still in love with them. Psychopaths typically flaunt their new victims for the whole world to see mere days after their previous relationship ended. You know what that does? Gasp, it creates jealousy.

Psychopaths manufacture toxic, desperate love. \& the thing about this sort of idealized \& then devalued passion is that it's long-lasting \& obsessive. Psychopaths groom others to spend every waking moment thinking about them, \& then they tear it all away without a moment's notice. Because psychopaths are eternally bored \& incapable of human bonding, this transition is quite easy for them. But to a normal, healthy individual it's devastating. You send desperate texts in an attempt to fix everything, unaware that they're using these frantic communications as ``proof'' of your insanity to garner sympathy from their next victim. It leaves you with a broken heart, crippling insecurities, a need to defend yourself, feelings of inferiority, \& a million unanswered questions. This is why it takes so long to get over a psychopath.

These claims of jealousy also serve to make their new target feel special -- as if they are the chosen one among the psychopath's many admirers. The psychopath will gladly string obedient exes along to make themselves seem in high demand at all times.'' -- \cite[p. 34]{MacKenzie2015}

\subsubsection{``But My Ex Really Was Awful!''}
``Everyone has horror stories about their exes. That's perfectly normal. What's not normal is when an ex's name comes up so frequently in a new relationship that you begin to feel like they're actually a part of your relationship. It's also not normal to trash an ex \& then hang out with them on a daily basis. Trust your intuition, \& remember that psychopaths always use exes as tools for manipulation \& persuasion.

The bottom line is this: anyone who speaks so regularly \& so negatively about their ex is -- at best -- not at all ready for a romantic relationship. But at worst, this person is manipulating your every thought, pitting you against people you've never even met. \& you can be assured that they'll soon be speaking the exact same way about you to every other pawn in their never-ending game of chess.'' -- \cite[p. 35]{MacKenzie2015}

\subsubsection{Testing the Waters}
``Once they have you programmed, psychopaths will begin experimenting with their newfound control to see how far they can push you. A useful victim will not talk back, \& they certainly won't defend themselves if the situation calls for it. If the idealization phase worked as planned, you should be more invested in maintaining the passion than standing up for yourself.

During this period, you will see tiny glimpses of the psychopath's darker side. They may teasingly call you a ``whore'' in the bedroom to see how you react. If you're married to someone else, they might casually joke about your spouse's ignorance about the whole situation. They will begin making subtle digs about your intelligence, abilities, \& dreams.

These are all tests, \& unfortunately if you're reading this book, it means you passed them. If you react in a negative way, the psychopath will assure you that they were obviously joking. As they test the waters, you will begin to feel more \& more oversensitive. You've always considered yourself to be an exceptionally easygoing person, but now you're questioning that. You stop mentioning your concerns, optimistically hoping to keep things perfect.

They use these subtle digs in combination with flattery, ensuring that the addictive brain chemicals continue to fire even when you're feeling upset. This trick slowly trains your mind to ignore your intuition, in favor of the high you feel when you're with them.

If you look back at the early stages of your relationship, you will likely remember small warning signs that you tried to ignore -- signs that just didn't fit in with the whole ``nice person'' act. Maybe they bragged a little bit too much about how much their ex still wants them. Or perhaps they ``forgot'' to call when they promised they would, contacting you hours later than planned. They probably stopping paying for dates, letting you pick up the tab. So what di you do? You brushed it all aside. You forgave them quickly \& moved forward because you were determined to be different -- the partner who could keep them happy \& absorb anything, no matter the cost.

\& that's when the grooming is complete.'' -- \cite[pp. 35--36]{MacKenzie2015}

\subsection{Identity Erosion}
``The psychopath strips you of your dignity by taking back everything they once pretended to feel during the idealization period. They make a mockery of your dreams, subtly suggesting that you may not be the one for them after all -- but nonetheless stringing you along for the added attention. After grooming you to be dependent \& compliant, they use this power to manufacture desperation \& desire. In a whirlwind of overwhelming emotions, your fantasy gradually shifts into an inconceivable nightmare.'' -- \cite[p. 37]{MacKenzie2015}

\subsubsection{Destroying Your Boundaries}
``Emotional abusers condition their victims to feel ashamed, inadequate, \& unstable. This is because they are cowards, incapable of healthy relationships with strong \& self-respecting individuals. Oftentimes, they choose targets who are unusually successful \& idealistic, because these people have more to lose. But abusers cannot control someone with such qualities, \& so they break down the target's self-esteem through belittling, teasing, \& manufactured jealousy. The target may have perfectionist tendencies, striving to meet the abuser's impossible standards. This results in a strange dynamic where the abuser is idealized, despite being lazy, dishonest, \& unfaithful, while the victim is devalued, despite putting more effort into this relationship than ever before.

Like sandpaper, the psychopath will wear away at your self-esteem through a calculated ``mean \& sweet'' cycle. Slowly, your standards will fall so low that you become grateful for utterly mediocre treatment. Like a frog in boiling water, you won't even realize what's happened until it's far too late. Your friends \& family will wonder what happened to the man or woman who used to be so strong \& energetic. You will frantically excuse your partner's behavior, unable to acknowledge the painful truth behind your relationship: something has changed.

You spend hours waiting by the phone, hoping for that morning text message or a promised phone call. You cancel your plans for the day just to make sure you'll be available for them. You begin to initiate contact more often, brushing aside the nagging sensation that they don't want to talk with you -- that they're simply ``putting up'' with you. You find yourself filling their Facebook wall with compliments \& cute jokes, trying to reestablish the perfect dream from the beginning of your relationship. But their responses now feel hollow at best.

You invent romantic stories \& exaggerate their positive aspects to anyone who will listen. By convincing others that they are a wonderful person, you can continue to live the lie yourself. Throughout the worst of the relationship, your friends \& family will likely know them as the ``perfect'' partner you described. After the relationship ends, it will be confusing \& awkward to explain what really happened. Your stories will seem implausible, \& your friends will wonder why you didn't speak up sooner. They will not understand that you didn't even know you were in an abusive relationship.

While you're struggling with all of this unexpected anxiety, the psychopath is able to push your boundaries even further. You're in a vulnerable place now, because you're willing to put up with mostly anything -- so long as they're paying attention to you.

Their opinions about your appearance become much more critical than before. Suddenly they begin to notice every little part of your body, commenting freely on your supposed inadequacies. You may even develop an eating disorder, failing to take care of yourself in an effort to keep them interested. Psychopaths are fascinated by body-image issues, \& will reward your unhealthy habits with the occasional compliment to keep you striving for perfection. Since your self-worth is invested entirely in their oscillating opinions, your moods will become unstable \& volatile.

They will also begin to humiliate you in front of friends -- no longer limited to belittling you behind closed doors. But it will always be done under a guise of humorous intention. You will be hurt to see that others seem to take your partner's side \& laugh, despite the way they're making you feel. A psychopath doesn't care when they take a joke too far, \& they will dismiss any concerns you might have, accusing you of being hypersensitive. You begin to go along with it, playing the role of a crazy, unintelligent partner whose only purpose is to entertain your love. With time, you will come to believe this facade.

All the while, they will sprinkle intermittent reminders of the idealization phase. If you reach a breaking point, they will always be ready to swoop back in with promises of unlimited love \& affection. Although they will never take the blame for their behavior, these superficial distractions will be enough to convince you that they're still the person you fell in love with. \& nothing else matters.'' -- \cite[pp. 37--39]{MacKenzie2015}

\subsubsection{Manufactured Emotions}
``During a relationship with a psychopath, you are likely to experience a range of emotions that you've never felt before: extreme jealousy, neediness, rage, anxiety, \& paranoia. After every outburst, you constantly think to yourself, ``If only I hadn't behaved that way, then maybe they'd be happier with me.''

Think again.

Those were not your emotions. I repeat: those were not your emotions. They were carefully manufactured by the psychopath in order to make you question your own good nature. Victims are often prone to believe that they can understand, forgive, \& absorb all of the problems in a relationship. Essentially, they checkmate themselves by constantly trying to rationalize the abuser's completely irrational behavior.'' -- \cite[p. 39]{MacKenzie2015}

\subsubsection{The Serial Provoker}
``Serial provokers are experts at seeking out flexible, easygoing people. They exploit this quality by constantly provoking their target with covert jabs, minimization, veiled humor, \& patronizing. The target will attempt to avoid conflict by remaining pleasant, choosing to forgive \& excuse this behavior in favor of maintaining harmony. But the serial provoker will continue to aggravate the target until they finally snap. Once this occurs, the provoker will sit back, feign surprise, \& marvel at how passive-aggressive, angry, \& volatile the target is. The target will immediately feel bad, apologize, \& absorb the blame. They are essentially shamed for rightfully losing their patience \& behaving the way the serial provoker behaves every single day. The difference is, the target feels remorse -- the serial provoker does not. The target is expected to remain calm \& peaceful no matter what, while the serial provoker feels entitled to do whatever they please.

E.g., you probably didn't consider yourself to be a jealous person before you met the psychopath. You might have been taken pride in being remarkably relaxed \& open-minded. The psychopath recognizes this \& seeks to exploit it. During the grooming phase, they draw you in by flattering you about those traits -- they just can't believe how perfect you are. The 2 of you never fight. There's never any drama. You're so easygoing compared to their crazy, evil ex.

But behind the scenes, something else is going on. Psychopaths become bored very easily, \& the idealization is only fun until they have you hooked. Once that happens, these strengths of yours become vulnerabilities that they use against you. They begin to inject as much drama into the relationship as they possibly can, throwing you into impossible situations \& then judging you for reacting to them.

Most people would agree that jealousy is toxic in a relationship. But there's a huge difference between true jealousy \& the psychopath's manufactured jealousy.

Take the following 2 (exaggerated) conversations:

\textbf{Case 1.}
\begin{quotation}
	\textsc{Boyfriend}: Hey, my old high school friend is coming into town if you'd like to meet her.
	
	\textsc{Girlfriend}: No! Why do you need other female friends? You have me.
\end{quotation}
In this case, the girlfriend truly seems to have some jealousy issues that need to be addressed. Assuming the boyfriend hasn't abused her in the past, this is an inappropriate display of jealousy.

\textbf{Case 2.}
\begin{quotation}
	\textsc{Boyfriend}: My ex is coming into town. You know, the crazy abusive one who's still completely obsessed with me.
	
	\textsc{Girlfriend}: Oh, I'm sorry to hear that!
	
	\textsc{Boyfriend}: We're probably going to meet up later for drinks. She always hits on me when she drinks.
	
	\textsc{Girlfriend}: I'm confused. Could we talk about this in person/
	
	\textsc{Boyfriend}: You have a problem with it?
	
	\textsc{Girlfriend}: Nope! No problem. I guess I was just a little confused since you said she abused you. But I hope things go well! It's nice when exes are able to be friends.
	
	\textsc{Boyfriend}: Wow, you're so jealous sometimes.
	
	\textsc{Girlfriend}: I'm sorry. I'm not trying to be jealous. I was just confused at 1st.
	
	\textsc{Boyfriend}: Your jealousy is ruining our relationship \& creating so much unnecessary drama.
	
	\textsc{Girlfriend}: I'm sorry! We don't have to talk about it in person. I really didn't mean to come across that way.
	
	\textsc{Boyfriend}: It's fine, I forgive you. We'll just have to work through your jealousy issues.
\end{quotation}
In this case, the psychopath did 3 things:
\begin{itemize}
	\item Put his girlfriend in an impossible situation that would make any human being jealous, especially after talking about how much his ex loves him.
	\item Accused her of being jealous, even though she tried to respond reasonably.
	\item Played ``good cop'' by offering to forgive her for a problem that he created in the 1st place. This places him in his favorite role of teacher vs. student.
\end{itemize}
The longer this abuse occurs, the more she'll begin to wonder if she actually has a jealously problem.

\& it's not just limited to jealousy. Perhaps you began to feel needy \& clingy during the relationship with the psychopath. Again, it's all manufactured. Who was the one responsible for initiating constant conversation \& attention in the 1st place? It was them. Once they're bored, they will start to lash out at you for trying to continue practices that they initiated.

Again, most people would agree that neediness is toxic in a relationship. But there's a huge difference between true neediness \& the psychopath's manufactured neediness.

\textbf{Case 1.}
\begin{quotation}
	\textsc{Boyfriend}: Hey, I won't be around tonight because my grandmother wants to get dinner. Sorry!
	
	\textsc{Girlfriend}: Oh my God, I haven't seen you in 3 hours. This is getting ridiculous. You better text me the entire time.
\end{quotation}
In this case, the boyfriend truly seems to have some neediness issues that need to be addressed. Assuming the girlfriend hasn't abused him in the past, this is an inappropriate display of neediness.

\textbf{Case 2.}
\begin{quotation}
	\textsc{Boyfriend}: Hi, I haven't heard from you in 3 days. Just want to make sure you're doing okay.
	
	\textsc{Girlfriend}: God, I have a life outside of you, you know.
	
	\textsc{Boyfriend}: I know, I was just sort of confused because I'm used to hearing from you each morning.
	
	\textsc{Girlfriend}: You're so needy. I have important things to do \& I can't just drop everything to text you.
	
	\textsc{Boyfriend}: I'm sorry, I didn't mean to sound needy. It was the 1st text I've sent in 3 days.
	
	\textsc{Girlfriend}: I can't deal with this. I've never met someone so needy in my life.
	
	\textsc{Boyfriend}: I'm really sorry! I won't bother you again.
	
	\textsc{Girlfriend}: It's fine. I forgive you. We'll just have to work through your neediness issues.
\end{quotation}
Once again, the psychopath did 3 things:
\begin{itemize}
	\item Put her boyfriend in an impossible situation that would make any human being needy, especially after the constant attention in the idealization phase.
	\item Accused him of being needy, even though he tried to respond reasonably.
	\item Played ``good cop'' by offering to forgive him for a problem that she created in the 1st place. This places her in her favorite role of teacher vs. student.
\end{itemize}
The longer this abuse occurs, the more he'll begin to wonder if he's actually a needy person.

I could go on like this for paranoia, anger, hysteria -- \& every other nasty emotion you may have felt in your toxic relationship. When you're in the grip of these emotions, it is natural to wonder what's wrong with you. You might even think you're going crazy. Well, psychopaths want you to believe you're crazy because it makes you seem more unstable to the rest of the world. But you'll find that once they're gone from your life, everything starts to make sense again. If you went from normal to ``crazy'' to normal again, that's not crazy. That's someone provoking you.

You must understand that in loving, healthy relationships, no one would ever put you in these situations in the 1st place. Your boundaries were put to the test, \& you did the absolute best you could, given the circumstances. In the future, you should never allow someone to tell you who you are or how you feel.'' -- \cite[pp. 40--43]{MacKenzie2015}

\subsubsection{Word Salad}
``When they're feeling threatened or bored, the psychopath will often use what's called \textit{word salad} in an attempt to keep your mind occupied. Basically, it's a conversation from hell. They aren't actually saying anything at all; they're just talking at you. Before you can even respond to 1 outrageous statement, they're already on to the next. You'll be left with your head spinning. Study the warning signs, \& disengage before any damage can be done:

\paragraph{1. Circular Conversations.} You'll think you worked something out, only to begin discussing it again in 2 minutes. \& it's as if you never even said a word the 1st time around. They'll begin reciting all of the same tired garbage, ignoring any legitimate arguments you may have provided moments ago. If something is going to be resolved, it will be on their terms. With psychopaths, the same issues will come up over \& over again: \textit{Why are they so friendly with their ex again? Why are they suddenly not paying any attention to you? Why do they sound so eager to get off the phone?} \& every time you bring up these issues, it's as if you never even had the argument in the past. You get sucked back in, only to feel crazy \& high maintenance when they decide, ``I'm sick of always arguing about this.'' It's a merry-go-round.

\paragraph{2. Bringing Up Your Past Wrongdoings \& Ignoring Their Own.} If you point out something nasty they're doing -- like ignoring you or cheating -- they'll mention something totally unrelated from the past that you've done wrong. Did you used to drink too much? Well then, their cheating isn't really all that bad compared to your drinking problem. Were you late to your 1st date 2 years ago? Well the, you can't complain about them ignoring you for 3 days straight. \& God forbid you bring up any of their wrongdoings. Then you are a bitter lunatic with a list of grievances.

\paragraph{3. Condescending \& Patronizing Tone.} Throughout the entire conversation they will have this calm, cool demeanor. It's almost as if they're mocking you, gauging your reactions to see how much further they can push. When you finally react emotionally, that's when they'll tell you to calm down, raise their eyebrows, smirk, or feign disappointment. The whole point of word salad is to make you unhinged, \& thereby give them the upper hand. Because remember, conversations are competitions -- just like anything else with a psychopath.

\paragraph{4. Accusing You of Doing Things That They Themselves Are Doing.} In heated arguments, psychopaths have no shame. They will begin labeling you with their own horrible qualities. It goes beyond projection, because most people project unknowingly. Psychopaths know they are smearing you with their own flaws, \& they are seeking a reaction. \textit{After all, how can you not react to such blatant hypocrisy?}

\paragraph{Multiple Personas.} Through the course of a word-salad conversation, you're likely to experience a variety of their personalities. It's sort of like good cop, bad cop, demented cop, stalker cop, scary cop, baby cop. If you're pulling away, sick of their abuse \& lies, they will evoke a glimpse of the idealization phase. A little torture to lure you back in with empty promises. If that doesn't work, suddenly they'll start insulting the things they once idealized. You'll be left wondering who you're even talking to, because their personas are imploding as they struggle to regain control. Our beloved administrator, Victoria, summed this up perfectly: ``The devil itself was unleashed in a desperate fit of fury after being recognized: twisting, turning, writhing, spewing, flattering, sparkling, vomiting.''

\paragraph{6. The Eternal Victim.} Somehow, their cheating \& lying will always lead back to a conversation about their abusive past or a crazy ex. You will end up feeling bad for them, even when they've done something horribly wrong. You will use it as an opportunity to bond with them over their supposed complex feelings. \& once they've successfully diverted your attention, everything will go back to the way it was. No bonding or deep spiritual connection whatsoever. Psychopaths cry ``abuse'' -- but in the end, you are the only one being abused.

\paragraph{7. You Begin Explaining Basic Human Emotions.} You find yourself explaining things like ``empathy'' \& ``feelings'' \& ``being nice.'' Normal adults do not need to be taught the golden rules from kindergarten. You are not the 1st person who has attempted to see the good in them, \& you will not be the last. You think to yourself, ``If they can just understand why I'm hurt, then they'll stop doing it.'' But they won't. They wouldn't have hurt you in the 1st place if they were a decent human being. The worst part is, they pretended to be decent when you 1st met -- sucking you in with this sweet, caring persona. They know how to be kind \& good, but they find it boring.

\paragraph{8. Excuses.} Everyone messes up every now \& then, but the psychopath recites excuses more often than they actually follow through with promises. Their actions never match up with their words. You are disappointed so frequently that you feel relieved when they do something halfway decent -- they condition you to become grateful for mediocre treatment.

\paragraph{9. ``What in the World Just Happened?''.} These conversations leave you drained. You will be left with an actual headache. You will spend hours, even days, obsessing over the argument. You'll feel as if you exhausted all of your emotional energy to accomplish absolutely nothing. You will have a million preplanned arguments in your head, ready to respond to all the unaddressed points that you couldn't keep up with. You will feel the need to defend yourself. You'll try to come up with a diplomatic solution that evenly distributes the blame, \& therefore gives you both a chance to apologize \& make up. But in the end, you'll find that you're the only one apologizing.'' -- \cite[pp. 43--46]{MacKenzie2015}

\subsubsection{Gaslighting \& Projection}
``Gaslighting is when the psychopath intentionally distorts reality -- often with trivial lies \& wrongdoings -- to bring about a reaction \& then deny that it ever took place. Like most victims, you are probably exceptionally easygoing \& will hold off on reacting for as long as possible. But inevitably, you're going to feel frustrated enough to finally speak up, \& that's when the psychopath will either rewrite history or reject that the incident ever even occurred. You may start to doubt your own sanity as the psychopath slowly erodes your grasp on reality.'' -- \cite[p. 46]{MacKenzie2015}

\subsubsection{Enemy Number 1}
``Psychopaths carefully select targets \& copy their personalities in order to create an immediate bond of trust \& familiarity. But because they are only acting, they cannot play the role perfectly forever. So inevitably, their targets begin to notice small cracks in the mask -- uncanny, inexplicable moments that simply don't seem to match up with the person they're pretending to be. Any target who dares point out these inconsistencies instantly becomes enemy number 1. Instead of admitting fault \& deceit, the psychopath will attempt to drive the target insane. Through mind games, triangulation, gaslighting, \& silence, they innocently encourage the target to self-destruct. Most targets don't even know they're dealing with a pathological personality at the time. They might only have gently observed that ``something feels different between us.'' But to a psychopath, this is the greatest insult of all -- a challenge to their credibility as a normal, healthy human being. \& so they must smear the target as ``crazy'' before quickly moving along to another victim who will better appreciate their world of lies.

Gaslighting isn't usually blatant abuse. It could be as simple as them saying they're going to do 1 thing, \& then doing something entirely different. E.g., they may tell you they're on their way to the gym, but then they go to a restaurant with friends instead. \textit{What's the point of a lie like that?} Why not just tell you they're going out to dinner?

Earlier on in the relationship, you might have innocently asked what happened to their gym plans \& they would inevitably start to make up strange, pointless excuses. But as the abuse worsens, they'll likely deny that they ever even said they were going to the gym. You'll find yourself disagreeing with them \& repeating the entire conversation, only to discover how petty you sound \& how much you're annoying your partner.

\& that's the thing about conversations that come from gaslighting: they \textit{do} sound petty. Who wants to argue about gym plans turning into a dinner out on the town? \textit{Who cares?} In any normal relationship, you wouldn't even bat an eye. But with psychopaths, these needless lies happen on a regular basis, \& you find that you're always getting sucked into ridiculous, pointless conversations that make you seem like an obsessed detective.

Speaking of detectives, if you present evidence of the truth -- like a text message or an email -- the psychopath will punish you with the silent treatment \& turn the entire conversation around on you for being paranoid \& crazy. You slowly learn that you're becoming a nuisance \& that open communication is unofficially prohibited in your relationship.'' -- \cite[pp. 46--47]{MacKenzie2015}

\subsubsection{The Black Hole}
``Psychopaths always see themselves as victims, no matter how horribly they've treated someone else. Nothing is ever their fault -- they've always been wronged in 1 way or another. To them, the problem is not their lying, cheating, stealing, \& abuse. The problem is that you started to notice all of those things. \textit{Why couldn't you just remain happy with the idealization phase?} How dare you betray them by standing up for yourself? Encounters with these people are like drowning in a black hole, because no matter how much they hurt you, it'll still be your fault.

Our invaluable administrator, Smitten Kitten, explains this mind-boggling process clearly \& coherently:

``Psychopaths project \& blame you for their own behavior. They accuse you of being negative when they are the most negative people in the world. They gaslight you into believing that your normal reactions to their abuse are the problem -- not the abuse itself. When you feel angry \& hurt because of their silent treatment, broken promises, lying, or cheating, there is something wrong with you. When you call them out on their dishonest behavior, you're the abnormal one who is too sensitive, too critical, \& always focusing on the negative.''

``This is all part of the brainwashing process. Acting inappropriately, unacceptably, downright abusively -- \& then trying to turn it around to make it your fault. They intentionally cause pain you don't deserve, all the while denying they've done anything to begin with. \& on top of that, they try to make it your fault -- so that you blame yourself for something that supposedly didn't even happen.''

``Yes, reread that. That's how illogical it is.''

``It's their parting `gift' to dump all of the blame on you for the looming failure of the relationship. Problem is, it never had a chance to begin with.''

``If only you had maintained the glowing optimism \& na\"{i}vet\'e you felt during the love-bombing stage -- throughout all of their subsequent lies \& abuse. Then everything would have been fine. If only you hadn't questioned the contradictions \& lies you recalled from letters that they later denied sending. Yes, if only you had stayed compliant \& quiet, in spite of the overwhelming evidence staring you in the face -- evidence they planted, just to test you. Then it would all be fine.''

``But even then, they would become bored \& disappointed that you hadn't caught on or challenged them. So they'd invent something to accuse you of, in order to justify their abuse \& create drama. No matter what you do, it's always a lose-lose situation with a psychopath. They want you to believe you're the loser when really, it's them.'' -- \cite[p. 48]{MacKenzie2015}

\subsubsection{Sexual Manipulation}
``Sex with the psychopath seemed perfect at 1st. They knew exactly where to touch you, what to say, \& when to do the right things. You were perfectly compatible in the bedroom, right?

Well, sort of.

Like everything else, the psychopath also mirrored your deepest sexual desires. That's why it felt so incredibly passionate \& flawless when you were together -- \& that's why it feels like rape during the identity erosion. Because the psychopath does not, in fact, share your most intimate fantasies. Instead, they've been observing \& tailoring their behavior to match yours. It's shocking when you realize this, because you come to understand that they never felt the emotional \& spiritual pleasure that you felt. While you were at your most vulnerable, they were simply watching \& learning.

You find yourself in a desperate situation, needing their sexual approval \& flattery to feel attractive. They use this to control you. They pull away in order to make you seem desperate, needy, \& slutty. In the idealization phase, they couldn't get enough of you. But once they have you hooked, they begin to play mind games. They withhold sex, redefining it as a privilege that they hold the key to.

When you're lying next to them in bed, you can practically feel them waiting for you to make the next move. They're ready to mock you -- to make you feel unnatural \& sex-crazed. They will laugh at you, insulting you with jokes that aren't even remotely funny. The passionate sex you remember has been replaced by a game. A competition.

They will make you feel ugly by announcing that their sex drive is lower than ever -- that they haven't even had sexual thoughts in weeks. The implication is clear: they haven't thought about you in weeks.

\& then, when the triangulation begins, you find it impossible to believe that they could have such a great sex life with anyone else. How could they? You seemed like soul mates when you made love. They liked all of the same things as you. But remember, it was manufactured. If you loved something in the bedroom, the psychopath quickly picked up on that in the grooming phase. They'll pick up on something else entirely for their next victim.

You unknowingly formed a bond with a con artist. Your consent was based on a lie. so many survivors blame themselves because they couldn't get past the sexual addiction, keeping them bonded to their abuser. But it's not your fault. You were tricked into feeling an overwhelmingly strong attachment during the grooming phase. \& then they manipulated that -- toying with the toxic addition firing through your body.

You will reclaim your sexual freedom -- I promise. We have an open \& honest dialogue about sex at \url{PsychopathFree.com}. It is a hugely important part of the psychopathic relationship cycle, \& more importantly, it plays an essential role in your own healing process. Recovery is a joint effort of the mind \& body.'' -- \cite[pp. 48--50]{MacKenzie2015}

\subsubsection{The Transitional Target}
``This section comes directly from a conversation I had with a dear friend, so it may feel a bit personal at times. I've done my best to edit it to apply to a broader audience. It's a special note for anyone who felt unusually disposable compared to the psychopath's other targets.

Psychopaths are always on the prowl for their next target but as the psychopath transitions from 1 ``stable'' relationship to the next, they need something (someone) to fill the void in between: a temporary target to dispose of as soon as they find something else. If you are selected as a transitional target, you may find your treatment a bit different from what I've already described. Although the psychopath likely doesn't even have the next target scouted out yet, they already know they're not staying with you so they want to move things along quickly. Because of this, you tend to get ripped out of the idealize phase much faster than most. Additionally, the idealize phase is lazy: no money, no actions, no special treatment. Just words. You were showered with a lot of words, which sucked you in, because you wanted to believe the words so badly. But, as you're starting to see, their actions never match up with their words.

But to you, the relationship means everything -- it's attention \& appreciation you've never experienced before. It appeals to your deepest dreams of making someone else happy, after all of their alleged pain \& sadness. It seems like they know you so well. You finally found a soul mate after so much loneliness \& frustration.

Unfortunately, none of this is true. To the psychopath, you are merely a diversion. Psychopaths are especially indifferent with transitional targets, not really caring 1 way or another -- leaving you with the feeling that they're being insensitive. You tend to fill in their abuse with your own love, in hopes that you can restore the brief idealize phase. Targets often experience cognitive dissonance, trying to project their own reasoning into an unreasonable person. But the psychopath's behavior is neither accidental nor unintentional.

\& then comes the most heartbreaking moment: they discard you \& go running off with another person whom they suddenly seem ready to settle down with. They move in together, post pictures, pay for things, \& live the life you always dreamed of. It's the ultimate insult when you were not given any of that special treatment. Basically, as soon as they got their quick fix of power \& control over you, they felt re-energized \& ready to scout out their next great adventure.

Statistically, most victims in an abusive relationship with a psychopath return to their abusers 7 times before they finally realize the treatment is unacceptable \& leave for good. Not so, for the transitional target. Here's how the 2 types of the relationships play out:

Typical relationship with a psychopath:
\begin{quotation}\it
	Idealize, devalue, idealize, devalue, idealize, devalue (repeat) $\to$ Finally a breaking point.
\end{quotation}
This is what transitional targets get:
\begin{quotation}\it
	Mediocre idealize phase with huge promises that make you feel amazing $\to$ Sudden discard out of nowhere.
\end{quotation}
This leaves the transitional target with zero closure because you can't even look back at the cycle of violence that exists for most people. Not that you would want to by any means, but you're basically just left in limbo after the abandonment, from such a ridiculous high to a devastating low, with no time or perspective to realize what just happened. It's emotional torture. You are left only with an intense love \& a horrible discard.

Psychopaths use their mind games on every target -- it's always the same. The difference is, transitional targets never experience that ``full'' idealize phase, with some time \& stability for things to blow up. This is because the psychopath never intended for the transitional target to become a stable part of their life to begin with. You were perfect for what they wanted at the time: attention \& admiration.

But they also recognized that you were emotionally intelligent \& uniquely perceptive. The fact that you're reading this book is not some sort of accident -- you're a truth seeker, determined to find out what just happened to you.

Psychopaths settle for targets who don't truly see their nasty behavior. If you're reading this now, that means the psychopath could never settle for you, because over the course of months, years, or decades, you saw through the facade. They need someone who won't catch on. Ever.

So yes, on 1 hand, the eventual long-term victim is useful, because they won't call out the psychopath's lying \& cheating. But on the other hand, the psychopath silently resents these people for not seeing through their facade. Strange, right? They pretend to give off this perfect, happy image with their ``settlement,'' but they much prefer the thrill of someone more empathetic -- someone who truly feels the torture of their mind games. But the psychopath can rarely have 1 of those people permanently, so instead he or she uses them during transition periods, like a quick high before the settlement. (Every once in a while, though, they will end up spending years with a highly empathetic person. I know several of those people from the forum. They are usually locked into a psychopathic relationship because they have children with the psychopath. These dynamics often seem to result in horrific discards.)

Many survivors tell stories of love-bombing that lacked the actual courtship seen with the psychopath's next target or previous ex. You never got to spend as much time with them as the others, right? Instead, you experienced a much shorter fling, cut off abruptly in the middle of the idealize phase with some unbelievably vicious identity erosion. \& then suddenly they've settled down with another partner, leaving you wondering how they're able to spend years with that person, when they could barely handle a few months with you.

\& that's the point -- psychopaths typically can't last long with empathetic people (except for cases with children \& long-term manipulation), because you tend to absorb their poison. Yes, they get the high of sweeping you off your feet \& making you a perfect servant to their mind games. But the downside is, eventually you subconsciously spit that poison right back in their face. You don't want to ruin the idealize phase, but you find yourself unable to stop pointing out their lies \& changed behavior.

Transitional targets \& truth-seeking targets figured them out, all the way down to their nasty core. Psychopaths would never admit it, but they'll always have a bitter respect for people who can see them for what they really are. \& at the same time, they'll also strongly resent those who can't -- even though that's all they can get in the long run. This is why they always lose \& reinvent the rules of the game, to convince themselves that their choice is correct.

But the fact is, psychopaths settle. They always do. \& that's why they needed to destroy you before settling: to convince themselves that they're not losing anything special. So why not make you self-destruct -- even hurt yourself? Perfect, the nagging doubts are finally gone.

\& from your perspective, this is why you're left so resentful \& angry. You did so much, encouraged by their fake appreciation that kept you going. \textit{At the very least, why couldn't they be a jerk from the start so you'd know not to waste your time \& emotions?} Instead, they used words \& promises to brainwash you into giving, giving, \& more giving. So when they not only don't appreciate it -- but actually destroy you -- you're left feeling broken \& empty. \& then you see them running off with someone else, paying for things, settling down a bit $\ldots$ It makes you think ``Hey, maybe they are capable of a relationship after all. Maybe the problem was me.''

No, the problem was not you. \& it never will be.'' -- \cite[pp. 50--54]{MacKenzie2015}

\subsubsection{Putting You on the Defensive}
``If you're dealing with a psychopath, it's a given that they will make unfounded accusations about you at some point -- especially if you're starting to put together the red flags in their behavior. These insults have a very specific purpose: to put you on the defensive.

\textit{Why?}

It's actually a lot simpler than you might think. People who defend themselves seem guilty by default. Whether or not they're guilty doesn't matter. Once they start to defend themselves, opinions \& assumptions are made. \textit{Is it unfair?} You bet. But it's how people often react. We've seen lives destroyed because of this phenomenon -- a man falsely accused of rape will continue to be thought of as a rapist, even after he's been proven innocent. The truth doesn't matter. No one trusts him anymore.

So the psychopath says all sorts of ridiculous things, \& you're suddenly defending yourself from accusations you've never even dreamed of. \textit{How could you not?} Your name is being smeared -- if not to others, then to the partner who supposedly loves you. So you get caught up in trying to prove them wrong, \& that's where the calculated self-destruction begins.

The psychopath can sit back, relax, \& enjoy the show. They can calmly point to the hysterical victim \& say, ``Jeez, that poor, crazy person $\ldots$'' Essentially, they provoke your anger, \& then calmly use it to prove their own point.

You may try to expose their lies: ``They're the liar, here's the proof!'' or ``They cheated on me, here's the proof!'' or ``They've done the same thing with 10 other partners, here's the proof!'' The problem is, nobody cares about the proof. They just see you desperately trying to defend yourself, \& because you're defending yourself, you seem like the guilty one.

Here's the most important thing to remember: defending yourself will only make things worse. Sometimes less is more, \& this is 1 of those times. You think you have a perfect response to their ridiculous defamation? Yes, the psychopath is counting on that. In fact, they've carefully crafted their insults to make sure of it. They attack the things you value most, because those are the things in life you will defend most passionately.

\& make no mistake -- it's intentional.

The easiest way for them to suck you in is to accuse you of doing things that they themselves did. It's almost too easy for you to point out the hypocrisy. \& that's the point -- yes, it's too easy. Because it's a trap. If you have a perfect retort to their garbage, there's a reason for that. Do not fall for it. They want you on the defensive, trying to prove yourself to everyone. Once you've taken the bait, their job is done.'' -- \cite[pp. 54--55]{MacKenzie2015}

\subsubsection{Boredom}
``When you're having a bad day, sometimes you just need to be on your own to recharge \& figure things out. Maybe you like to use your alone time to create, write, cook, imagine, meditate, paint, or dream. Or maybe you're just looking to take a well-deserved nap. My point is, whether you're an introvert or extrovert, occasional quiet time seems to be a universal human need.

But this is not the case with psychopaths.

Alone time may be 1 of the few things that truly agitates these people, who are otherwise seamlessly cool \& calm. Without a conscience, there's not very much to think about on their own. \& without people spoonfeeding them adoration \& attention, the boredom sets in quickly.

Psychopaths are always bored, \& they constantly seek out stimuli to distract themselves from this nagging condition. They cannot tolerate being alone for any extended period of time. Healthy human beings learn to enjoy quiet time \& introspection -- this is how we discover some of the most important things about ourselves. Psychopaths, on the other hand, have nothing to discover. They spend their free time mirroring others \& copying desirable traits. Empathy allows us to experience imagination \& creativity, 2 of the most wonderful human qualities that psychopaths can only mimic.

In various forums devoted to self-proclaimed sociopaths, there are frequent discussions about how to cope with the overwhelming boredom that consumes their daily life. Not surprisingly, most of the answers have to do with sex, alcohol, drugs, \& -- of course -- manipulating others.

Relationships provide the psychopath with the most stable \& reliable form of boredom relief, because once they have you hooked, they can reach out to you at any time for compliments, attention, \& praise. \& once they have other targets readily available, they're safe to start emotionally abusing you, which is far more interesting than pretending to love you. Watching you scurry around like a rat in a maze provides them with an entertaining distraction from their otherwise insufferably boring life. The idealization phase is merely a by-product of this boredom -- a necessary step in grooming you so that they can abuse you for as long as possible.

At times in the relationship, you may have felt exhausted because it seemed like you never had any alone time with your partner after the honeymoon period. It was always spending time with them \& their friends, from 1 busy plan to the next. Psychopaths also have a tendency to behave like innocent children, surrounding themselves with maternal \& paternal types who constantly want to be there for them, offering support \& providing for them at every turn.

These people are all kept around to alleviate the psychopath's boredom. The more targets the better. At 1st, the psychopath might have complained about them to you. But after you're locked into the relationship, the psychopath will oscillate between choosing you \& choosing others in order to keep everyone on their toes. No matter who they choose for the day, 1 thing is certain: they will not spend time on their own. It's far better to provoke hurt feelings in you \& watch your reactions than to simply sit alone in their room for an hour.

Eventually, the most hurtful boredom comes when it seems that they have suddenly lost all interest in you. Everything you do seems to bore them. As you struggle to regain their attention, you find that the qualities they once admired in you have apparently become glaring faults. Nothing you do seems to bring back the intense focus they once aimed at you -- the time when you were the \textit{only} one who could cure all of that boredom.

Even after they begin treating you terribly, a part of you may wish that you could win back your place as ``Boredom Relief \#1.'' This is completely normal, because once you realize that the idealization phase is gone forever, the next best option is to frantically ensure that you at least remain 1 of their many sources of entertainment.

Yes, this is how warped our standards become.'' -- \cite[pp. 55--57]{MacKenzie2015}

\subsubsection{Covert Gossiping}
``The psychopath claims to hate ``drama,'' but you will slowly come to find that there's more drama surrounding their life than anyone you've ever known. Of course, according to them, none of this is actually their fault. People just love them too much, treat them badly, \& always seem to go crazy around them. How unfair for them.

But, as we're beginning to discover, that's not really the case.

The psychopath is constantly provoking drama, rivalries, \& competitions. What separates them from everyday drama queens is their ability to appear innocent in all of it. They make subtle suggestions, then sit back \& watch as others go down in flames for them. This is where the ``covert'' part comes into play.

They plant little seeds of poison, whispering to everyone, idealizing them to their face, \& then insulting them behind their backs. ``Insulting'' doesn't really even capture the subtlety of a psychopath's gossip. Instead of overtly trashing people, psychopaths paint themselves as victims. Someone is always wronging them in 1 way or another. So instead of being a backstabbing gossiper, they come across as a sympathetic victim of everyone else's bad behavior. At some point, you might have even felt special being the one they chose to complain to (again, that's how screwed up our standards become).

But then you end up on the receiving end of it.

Suddenly \textit{you're} the villain -- the one causing them all this pain \& torment. This process begins in earnest once the relationship starts to unravel. They begin running back to the very same people they once complained about to you, using them to lament about how crazy you've become. This generates a lot of sympathy, which is the perfect way to transition to their next target without anyone judging their blatant infidelity.

During the relationship, you probably found yourself disliking \& resenting people you'd never even met. Could that have anything to do with the psychopath's constant suggestions that these people were all in love with them, wanted them, abused them, or were jealous of you? Over time, this builds up so much negativity \& envy -- more than you'd experience in any healthy relationship. \& the sad part is, this same negativity is also felt toward \textit{you} by everyone else.'' -- \cite[pp. 57--58]{MacKenzie2015}

\subsubsection{The Manipulator Test}
``If you're looking for a way to discern manipulators from empathetic people, pay attention to the way they speak about others in relation to you. Kind, decent people will go out of their way to make sure you know their friends \& family members really like you. Manipulators, on the other hand, will always seek to triangulate. They provoke rivalries \& jealousy by manufacturing competitions. They whisper in your ear that their friend, an ex, or a family member is very jealous of you -- or maybe that they said something nasty about you. Make no mistake, they're whispering the exact same things about you to those very same people. So ask yourself, does this person create harmony, or do they engineer chaos?

Sure, friends \& exes might smile to your face, just like you smile to theirs. But deep down, everyone under the psychopath's spell is harboring this growing resentment toward one another. \& it's not for any legitimate reason -- it's because you've been turned against each other. You're all just movable pawns in the game the psychopath plays for attention \& drama. They keep each target far enough apart that you can't compare notes, but close enough that you're always on edge \& unsure of where you stand.

As impossible as it might seem while you're still involved with a psychopath, all of these people are probably perfectly nice human beings. Like you, they've just been poisoned \& brainwashed to believe the worst about everyone else. Most empathetic partners are excited to befriend their partner's friends \& make a positive impression. This was probably true at the beginning of the relationship, but as the triangulation \& gossip worsened, you started to experience more \& more negative emotions. \& perhaps you blamed yourself, believing that you were becoming an unrecognizable mess of jealousy \& resentment.

This is what happens when you enter the psychopath's reality -- all of their gossip \& lies start to distort your own reality. Because here are the 2 realities you must choose from:
\begin{itemize}
	\item The psychopath is normal. Everyone else is jealous, ill-intentioned, \& self-serving.
	\item The psychopath is jealous, ill-intentioned, \& self-serving. Everyone else is normal.
\end{itemize}
During the idealization \& love-bombing, we favor option 1. It all feels absolutely incredible, \& so we begin to define a new reality around our ``soul mate.'' They are wonderful, everyone else is wonderful. Life is good! But then the covert gossip begins, \& that's when our reality starts to shift. In order for us to continue living in the dream reality, where the psychopath is honest \& good, all of their gossip must also be true. All of these other people must be jealous, ill-intentioned, \& self-serving. Because if that's not true, then that means the psychopath is a liar, gossiper, \& manipulator.

The stronger this reality becomes, the harder it is to break away from it. It really does start to feel like all of these people are against you -- \& that's how the psychopath wants you to feel. Because then your entire sense of happiness is dependent on them. \& what's more, maintaining a false reality requires that you make a lot of excuses \& explanations that keep you on the defensive \& in denial. This is an extremely effective distraction (\& destabilizer) from the frightening truth that your idealizer is actually your enemy.

During the breakup \& recovery, we're ripped away from option 1 -- which was really just a fantasy -- \& faced with reality. We feel devastated, empty, \& hopeless. Without that fantasy, we've lost \textit{everything}. We've lost the most important, wonderful, perfect partner in the entire world. \& on top of that, everyone else is untrustworthy \& toxic. We have nobody \& nothing.

But then we start to put the pieces together \& experience small moments of trust with the people around us. This is where a Constant can truly change someone's life. We see how we feel around someone who treats us well. We notice the freedom in spending time with someone who is not judging, triangulating, or spreading lies. Eventually, we start to place the negativity where it really belongs, instead of shifting it onto the rest of the world.

\& finally, everything falls into place.

Reality number 2 strengthens. The 1st option doesn't even make sense anymore. It was never you \& your partner against the world -- it was only your partner against you. You begin to find peace \& feel compassion toward people you once disliked. The cognitive dissonance lessons as you spend more time in the proper reality, away from the psychopath's web of lies. Your natural mercy \& empathy return to you. Manufactured paranoia transforms into genuine trust, \& at long last, the distractions are gone so that you can turn your focus to the \textit{real} problem.'' -- \cite[pp. 59--61]{MacKenzie2015}

\subsubsection{Torture by Triangulation}

\subsection{The Grand Finale}

%------------------------------------------------------------------------------%

\section{The Path to Recovery}

\subsection{Why Does It Take So Long?}

\subsection{The Stages of Grief -- Part I}

\subsection{The Stages of Grief -- Part II}

%------------------------------------------------------------------------------%

\section{Freedom}

\subsection{Looking Back, Moving Forward}

\subsection{Introspection \& Insecurities}

\subsection{Self-Respect}

\subsection{30 Signs of Strength}

\subsection{Spirituality \& Love}

\subsection{The Fool \& the World}

\subsection{A Bigger Picture}

\subsection{Afterword: The Constant: Revisited}

%------------------------------------------------------------------------------%

\printbibliography[heading=bibintoc]
	
\end{document}