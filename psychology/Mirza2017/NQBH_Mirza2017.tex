\documentclass{article}
\usepackage[backend=biber,natbib=true,style=authoryear]{biblatex}
\addbibresource{/home/nqbh/reference/bib.bib}
\usepackage{tocloft}
\renewcommand{\cftsecleader}{\cftdotfill{\cftdotsep}}
\usepackage[colorlinks=true,linkcolor=blue,urlcolor=red,citecolor=magenta]{hyperref}
\usepackage{algorithm,algpseudocode,amsmath,amssymb,amsthm,float,graphicx,mathtools}
\allowdisplaybreaks
\numberwithin{equation}{section}
\newtheorem{assumption}{Assumption}[section]
\newtheorem{conjecture}{Conjecture}[section]
\newtheorem{corollary}{Corollary}[section]
\newtheorem{definition}{Definition}[section]
\newtheorem{example}{Example}[section]
\newtheorem{lemma}{Lemma}[section]
\newtheorem{notation}{Notation}[section]
\newtheorem{principle}{Principle}[section]
\newtheorem{problem}{Problem}[section]
\newtheorem{proposition}{Proposition}[section]
\newtheorem{question}{Question}[section]
\newtheorem{remark}{Remark}[section]
\newtheorem{theorem}{Theorem}[section]
\usepackage[left=0.5in,right=0.5in,top=1.5cm,bottom=1.5cm]{geometry}
\usepackage{fancyhdr}
\pagestyle{fancy}
\fancyhf{}
\lhead{\small Sect.~\thesection}
\rhead{\small\nouppercase{\leftmark}}
\renewcommand{\sectionmark}[1]{\markboth{#1}{}}
\cfoot{\thepage}
\def\labelitemii{$\circ$}

\title{The Covert Passive Aggressive Narcissist}
\author{Debbie Mirza}
\date{\today}

\begin{document}
\maketitle
\tableofcontents

%------------------------------------------------------------------------------%

\section*{Foreword}
``3 years ago, my reality was imploding all around me like a scene from the movie \textit{Inception}. Everything I had believed to be true was suddenly collapsing. I was forced to face the truth. Until that moment, I couldn't quite put my finger on exactly what the problem was. I was mostly in denial about it, but I think I knew something was wrong for a while.

As 1 of my friends mentioned the word narcissist, she pointed out, ``He's like the ones from the past, he's just the more sophisticated model.''

The heart-breaking dissolution of that relationship sent me on a journey to Peru where I met several more characters like him, mostly in my work life. A year later \& totally devastated, I reluctantly went back to the original covert abuser in my life to get on my feet again.

Back where it all began, 1st the complex-PTSD breakdown happened \& then I broke through. Observing my mother with new awareness, I was finally able to see where that pattern of relationships came from. In facing my wound \& working on healing, I created a new sense of purpose \& the work I do now to help people self-heal after narcissistic abuse.

Recently, when 1 of my clients told me she was writing a book on covert narcissists, I was so excited that I offered to write the foreword. I'm grateful that Debbie dedicated herself to write this book on such an important, nuanced topic within the genre of narcissistic abuse.

This book is meeting a great need because unfortunately there is not enough information available for people who have been through the more covert forms of abuse. It's incredibly sophisticated \& stealth, so it's often missed by mental health professionals who were not trained to recognize it.

Debbie writes,
\begin{quotation}\it
	``You think you're on the right track after discovering narcissistic personality disorder, but then you read things that are not completely what you experienced.''
	
	``Coverts do have a grandiose sense of self, are preoccupied with fantasies of power, require excessive admiration, but they hide all these attributes so people will like \& trust them.''
\end{quotation}
She \textit{gets} it. The author clearly knows this war from the front lines.

After years of talking with people in my personal \& professional life about covert narcissists, I've come to believe that in order to really understand the nature of the covert narcissist, you have to have lived it. Whether it's a spouse, significant other, friendship, boss, co-worker, neighbor or family member, the patterns are nearly identical \& only someone who has been inside that nightmare can really know what the experiences is like. Even then it's hard to describe.

I believe the covert types are by far the most dangerous because of their ability to fly stealth, undetected by normal radar. They leverage their intelligence through a meticulous choice of words \& silence in order to manipulate others. The smarter the narcissist, the more dangerous because the cloak of invisibility is so high-tech.

The overt type of abuser is much more obvious because they lack the intelligence to manipulate that cleverly, so they resort to aggression \& violence as their primary weapons.

Plausible deniability is the covert narcissist's greatest weapon in their arsenal of gaslighting tools.

With a covert narcissist everything on the surface looks normal \& often lovely for months, years, even decades. They know how to say all the right things, exactly the things that you personally want to hear. They can mirror empathy, concern, \& tears better than most Hollywood actors. However, underneath the surface the feeling is off. It's so subtle that you could easily miss it or dismiss it.

After a relationship with a covert narcissist, you feel like you can't trust your perception of reality because no one else can see what you see. Most people adore covert narcissists because of how very careful they are in choosing who they unmask around \& how much effort they put into optics \& public perception.

When you ask for advice from friends \& even professionals, you might only hear people giving the benefit of the doubt to the covert narcissist or worse yet telling you that you're being paranoid, overreacting or some other way of blaming the victim. Unfortunately, asking advice from people who don't understand it can feel invalidating, lead you to more setbacks \& possibly even encourage you to stay in a dangerous situation.

It's terrifying when every part of your intuition is telling you something is really wrong but the covert narcissist \& everyone else are telling you that things are just fine, \& implying that maybe you are the problem.

Survivors of covert narcissists need to know that they're not crazy. This is the primary concern I hear from clients who were with covert types of abusers. Reading this book will give you that validation.

The recovery of self-trust after abuse by a covert narcissist can take some time. Be patient with yourself in this process. You have been through a severe relational trauma \& while you can't see your wounds, they are very real.

This book will help you to understand what happened in your relationship with a cover narcissist through an inventory of their typical characteristics \& behaviors, in addition to survivors' stories of interactions with them. You'll likely have many aha moments when you connect the dots to similar experiences that happened to you.

I know this book is going to help a lot of people make sense of the insensible.

Big hug to you!

Meredith Miller, Coach \& Author. Mexico City, Mexico'' -- \cite[pp. 6--8]{Mirza2017}

%------------------------------------------------------------------------------%

\section*{Preface}

\begin{quotation}\it
	``Are you in a romantic relationship or coming out of one that feels incredibly confusing \& is making you feel like you're going crazy?
	
	Does your mother appear amazing to everyone else, but growing up you felt alone, found it hard to have your own identity, \& you felt like things were always your fault?
	
	Did you feel like you walked on eggshells growing up with your dad \& find it hard to connect with him, but people have always told you know lucky you are to have a dad like yours?
	
	Do you have a boss or coworker that everyone thinks is great, but after years of working with him\texttt{/}her, you find yourself feeling a lot of anxiety, never feeling good enough, \& questioning your own sanity?
	
	Has someone told you your loved one might be a narcissist? You've done some research but are confused because the person you are wondering about doesn't come across as a self-absorbed, arrogant over the top person that fits the description of a narcissist?
\end{quotation}
If any of these scenarios resonate, this could mean you are dealing with a covert narcissist. This is the hardest type of narcissist to diagnose because they are so disguised, so covert.

Covert narcissism is the worst \& most insidious form of narcissism because the abuse is so hidden. Most people don't even realize they are being abused when they are in these relationships. The life inside them is slowly depleted over time without them recognizing that this is a result of years of devaluing tactics by the narcissist. Their self-worth is beaten down. There are no visible scars, but the impact these people have on you is profound. You have been emotionally \& psychologically abused \& you are often the only one to see this side of them while everyone around them thinks they are great. This furthers your confusion \& minimizes your pain.

When I say ``worst form,'' I do not want to minimize anyone else's trauma when dealing with an overt narcissist or any other type of psychologically abusive personality. Abuse is abuse -- it is horrific \& always undeserved. My heart goes out to anyone who has \& is experiencing anything that harms \& devalues them.

1 reason covert narcissists are so damaging is because of cognitive dissonance. This is when you have 2 competing thoughts in your mind. You love your mom, spouse, boyfriend, or girlfriend \& thought they loved you the same. Yet when you look back, their behaviors are making you question your beliefs about them. As you research you begin to wonder, \textit{``Could this person really have been controlling \& manipulating me for years, \& I didn't see it $\ldots$ or were things really my fault \& I'm just overdramatizing my experience?''} You have a solid belief that has been built up over years that this is a good person who cares about you, \& at the same time, they are being incredibly cruel \& controlling. The cognitive dissonance is dizzying \& crazy making.

The overt types of narcissists are obvious, in your-face-face kind of people. They will let others know how great they are. When their mask comes off others around them roll their eyes \& say, \textit{``Oh, yeah, he's terrible.''}

On the other hand, covert narcissists are well liked. They are charming \& kind. They appear humble \& empathetic. They can be good listeners \& appear to really care. You can feel incredibly loved by them. They simultaneously make you feel terrible about yourself. They use cloaked tactics that you don't see for years.

It is common for people to be in romantic relationships with covert narcissists for over 10, 20, 30, 40-plus years not recognizing the abuse they have endured for decades.

This is especially devastating when it is a family member. Sometimes you are the only one who sees it when your siblings still think their dad or mom is amazing \& blame you for a plethora of things. You feel like you are going crazy \& you start minimizing the abuse yourself. If no one else sees it, you think it must be you that is the problem.

This type of abuse does not look as messy as it really is. It is so invisible. It is hard to put your finger on what is wrong.

If you relate to any of this, you are not alone, \& you can trust yourself.

I went through years of confusion \& cognitive dissonance myself. I have had several covert narcissists in my life. They have been incredibly confusing \& crazy-making relationships.

Years ago I searched for answers to help with my own confusion. I read a lot of books on narcissism but could not find any on the covert type. After years of piecing together information from various sources, I decided to write the book I had needed \& couldn't find so other survivors would have the information they needed in 1 place.

In preparation for this book, I interviewed over 100 survivors. I did in-depth research on the topic because I wanted to make sure this book would be accurate, comprehensive, \& incredibly helpful for you. You deserve that.

As I met more \& more people who have been through this type of relationship, my heart was affected tremendously. Witnessing their pain, their wounded hearts, \& their strength was humbling \& brought out a fierceness in me that made me want to make this the most helpful book I possibly could.

In the following pages, I will explain the traits of a covert narcissist. I share lots of stories from people I've interviewed to illustrate the traits. All the names have been changed \& details altered, so their identities are protected.

I facilitate a support group in my area \& have seen how important stories are. When I begin the meetings, I often ask what they are hoping to get out of the evening. Most people say, \textit{``Stories! I need to hear stories so I know I'm not crazy.''} You will hear plenty of stories in this book to help you recognize things you have experienced \& help validate the truth of what you have been through.

I also spend a lot of time talking about healing. If you have read this far, my hunch is you have probably been through or are going through a tremendously difficult \& crazy-making experience with a covert narcissist. You deserve to find clarity \& ultimately heal the wounds this relationship has caused.

Being with a covert narcissist can take you far away from the person you really are. My hope is this book will help bring you back to your stunning self.

May you find all the answers you are looking for \& come to a place of freedom \& peace. That may not feel possible right now, but trust me, it is.

With so much love,

Debbie Mirza'' -- \cite[pp. 9--12]{Mirza2017}

%------------------------------------------------------------------------------%

\section*{Introduction}

\begin{quotation}\it
	``Is your husband a narcissist?''
	
	``No! I would never use that word to describe him. He's the nicest guy. Everybody loves him. You would love him if you met him.''
	
	``Yes, that's what they are like.'' The divorce attorney saw Amy's confused face, walked closer to her \& said with concern, ``I am seeing a lot of classic signs, Amy. I suggest when you get home, you get a hold of as much information as you can about narcissists because you need to know what you're dealing with.''
\end{quotation}
Amy left the appointment in a daze. \textit{Narcissist?} That was the last word she would have used to describe her husband of over 30 years. She had always seem him as kind, someone she respected. Their relationship wasn't perfect, but most of the time she would have described it as a good marriage. She felt lucky to be with someone so easy going. However, his behavior over the past year had been vastly different than the man she thought she knew. So many things didn't make sense. The way he was treating her was so hurtful, disturbing, \& utterly confusing. Then suddenly, he was done with her. The ending of the marriage was shocking \& incredibly confusing.

When she 1st met her husband, she felt lucky to have met such a great guy. He was kind. He talked about his feelings. He listened to her, asked her questions about herself, really wanted to know her. Her family \& friends loved him, felt so happy she had met someone like him.

It was remarkable how alike they were. It all felt so easy. The 1st year they dated was pure bliss. Then things became difficult, but that was because of outside circumstances, she always believed. They worked through things. Their communication was great, she thought. They had some issues but always talked about things. She considered him her best friend.

Recently, though, he had been treating her in ways she had never experienced. This man she had seen as kind \& loving had become incredibly cruel \& aggressive toward her. He was continuously telling her all the things he believed was wrong with her \& blaming her for making it impossible for their marriage to work. All of this seemed to come out of nowhere. After 30 years, he moved out \& made sure she knew how much happier he was without her. It didn't seem to bother him at all that this was ending. He wanted out, \& according to him, it was all her fault.

She decided to take her attorney's advice, reading books as well as articles on narcissism, hoping this might be the piece that would finally make sense of this confusing puzzle.

When she read the descriptions of narcissists, she kept thinking, \textit{``that doesn't sound like him.''} Each book described someone who was flashy, drove expensive cars, liked to show off their fancy homes, people who were aggressive, annoying, obviously self-centered. She read stories of gaslighting that seemed really extreme to her. At the same time, amist the grandiose images, there were some things that did sound like her husband.

She read the basic traits of lack of empathy, rage, not a strong sense of self, controlling, manipulative, selfish. She began to feel like her eyes were opening to thins she had not seen previously.

Even though she believed her marriage was good, Amy had spent years doubting her self, believing she was to blame for issues that did arise in the marriage. Now, the way she was being treated felt wrong, even though she still wondered if the things he was saying about her were true. He seemed so confident \& sounded so rational. His words to her were cruel, but they were also mixed with loving words that made it even more confusing. When she was in conversations with him, her body felt confused, muddled, \& even nauseous at times. It was hard for her to think clearly. She felt run over \& talked down to by him. The words that came out of him were demeaning. He would ``teach'' her about life \& how she needed to be.

She found herself calling close friends \& family \& asking them questions like,
\begin{quotation}\it
	``Am I controlling \& manipulative like he's saying I am \& I just don't see it? Am I inconsiderate? Maybe I have been selfish? I can't think clearly. I can't even see what is true about me anymore?''
\end{quotation}
Feeling so confused, she needed reminders of who she really was because she felt like she was going crazy. While she was feeling incredibly emotional \& unstable, he was calm \& rational, which made her question herself even more.

Amy become a voracious reader \& researcher of narcissism. She also went to a therapist that was an expert on the subject. The therapist asked odd questions like:
\begin{quotation}\it
	``Did your husband forget things a lot, like when you would ask him to pick up some apples while he was at the store?''
	
	``Yes! About 70\% of the time.'' That's a trait?
	
	When he would go to a coffee shop I would ask if he could get me some water while he was in there. He would happily say `Sure!' About 7 out of 10 times he would come back to the car with no water \& say, `Oh! I'm so sorry. I totally forgot.' He seemed like he felt bad each time. I would always tell him it was okay. I felt frustrated \& confused because it happened so often, but I didn't feel like it was ok for me to be upset because it was an honest mistake.'' Is this common with narcissists?''
	
	``Yes. Very. Did he ever go back \& get you water?'' the therapist asked Amy.''
	
	``No. Never. I never thought about that.''
	
	``What about your birthdays? What were those like with him?''
	
	``They were awful. But the thing is I can't tell you exactly why.'' Amy's face scrunched up in confusion. ``He wasn't mean to me. He always bought me gifts. Sometimes took me out to dinner. For some reason, though, I ended up crying on my birthdays \& apologizing to him for something. I don't even remember why now. Maybe not being appreciative enough? He would buy me things, but many times they were things I wouldn't want. Then he would tell me a long story about how he found this gift \& all the thought \& effort that went into it, \& I would feel like I needed to have a big reaction even though it was something I never would have wanted. Then I'd feel bad because I was being shallow \& not grateful. Birthdays were always disappointing, \& I was glad when they were over. They wore me out for some reason. It never felt like he enjoyed celebrating me, treating me. He always seemed irritated that he had to do things for me. Sometimes he would spend a lot o money \& get me something grandiose. It actually stressed me out because of the amount of money he used. I didn't feel like I could say anything because of all the trouble he had gone to.''
	
	``When he gave you big things or made grandiose gestures, were other people around to see?''
	
	``Well, when I think about it, yes. Always $\ldots$ that's interesting. I never thought about that. I don't remember any private moments where he would give me something special that showed how well he knew me \& how much he loved me. I didn't feel loved on my birthdays when I look back.''
\end{quotation}
With each question, Amy realized she must learn more.

1 day, after telling her story to another therapist who had a lot of experience with narcissistic abuse, Amy heard a term she had not seen in her research, \& that changed everything for her. The therapist said, \textit{``It sounds like your husband is a Covert Passive-Aggressive Narcissist. Those are the hardest to recognize.''}

Amy felt chills go up her spine. \textit{``Please tell me more about that,''} she told the therapist. Everything began to make sense for the 1st time.

Hearing covert passive aggressive in front of the word narcissist gave her the missing piece she needed for her quest to understand what was happening. This sent her on a journey that would change her life forever, \& ultimately bring her the clarity \& healing she so desperately needed \& deserved. Amy now leads men \& women through different healing modalities in the mountains of Peru \& feels tremendously fulfilled \& happy. Years ago, when she 1st discovered the truth of her marriage, she never would have imagined that she would someday feel so free \& happy. She now has a glow about her that is inspiring \& gives others hope. She knows who she is \& has learned to trust herself implicitly.

Amy was 1 of the women I interviewed as part of my research for this book. Her story reflects what I heard from so many that have experienced the crazy-making relationship with a covert narcissist.

If you are reading this, I imagine you might relate to part of Amy's story, maybe even a lot of it. You may be on your own search, trying to make sense of a very confusing person in your life. This book is for you, to give you clarity, strength, \& understanding. It will educate you as well as give you hope.

The word ``narcissist'' is thrown around a lot \& grossly misused. \textit{``He's so narcissistic! Oh, yeah, I was with a narcissist too!''} People often use this word to describe someone who is selfish \& arrogant. The true definition goes much deeper \& unfortunately because it is used so carelessly, it diminishes the painful reality of victims of narcissistic people. Someone who has experienced a true narcissist would never toss the word around so lightly. It is important to know what it really means.

We tend to label people a lot, \& that can be a destructive thing. In this case, the label is important. When victims are looking for answers, \& they finally discover their partner or parent or co-worker might be a covert narcissist, everything begins to make sense. It is incredibly helpful in understanding \& beginning to heal.

I was talking about healing \& restoration to 1 survivor I interviewed. Through tears, she looked at me. With desperation \& shakiness in her voice, she asked, \textit{``Do you think it's even possible?''}

It is for this woman \& so many others like her that I have written this book. Profound healing \& freedom are absolutely possible. I, along with so many others, am proof of this. There is hope. The healing you will experience is profound \& will bring you to a very strong place inside yourself.

I also interviewed men who have experienced this abuse too. Both genders are affected by this destructive personality disorder.

Most people I talked to struggled to describe the relationship. This is common. There was a perpetual confused look on each face. It can be difficult to explain because the abuse is so hidden \& subtle. They weren't yelled at or physically abused. There are no visible scars. Yet the impact it makes on the psyche is profound.

Like the people I interviewed, I have also experienced covert narcissists, several actually. I know what it's like to be subtly abused for a very long time without recognizing it. I also know what it's like trying to find information on the covert type. You think you are on the right track after discovering narcissistic personality disorder, but then you read things that are not completely what you experienced. You read about aggressive behavior, physical abuse, dramatic stories of deception, \& you think maybe you are off track. Your story doesn't appear that bad compared to what you are reading, which then diminishes your own pain \& adds to your confusion. That's why I felt it was so important for me to write this book \& put everything I had learned into 1 place.

Many people who go to therapy to get help because they are depressed, low on energy, experiencing low self-esteem, feeling a lot of anxiety, \& confusion usually have no idea that the cause of their issue is an abusive relationship, whether that is with a romantic partner, a parent, or a boss at work.

Some victims become re-traumatized by a therapist or friend that doesn't understand. Most therapists are not educated on the covert type of narcissism. Only the overt type is taught in higher education, so most understandably don't recognize the signs \& traits. I talked to 1 woman who was in a covert relationship \& went to therapy for 10 years. She tried a few different therapists because no one seemed to be able to help her with her depression, anxiety, \& lack of energy. They didn't recognize she was in an abusive relationship. Finally, after 10 years, she tried another therapist who after 15 minutes told her she was in an abusive relationship. The others couldn't see it. Neither could she.

This is such a common story. She lived for years thinking something was wrong with her. She was being subtly manipulated \& devalued at home without seeing it. Her body was reacting. She was slowly dying inside \& couldn't figure out why. Thank goodness for the therapist who understood covert narcissism \& recognized the signs.

When the relationship ultimately ends in a breakup or divorce, victims have a very difficult time understanding what just happened. When a relationship with a covert narcissist ends, it is sudden \& painful. It can look like a ``normal'' divorce, but it is not even close.

Well-meaning friends \& family will wonder why it is so hard for you to get past the partner, why you have no desire to date anyone, why it is taking you so long to recover \& get back to the way you used to be. Breakups are a part of life, but this type of break up is a whole other animal. The only people that will fully understand what you are going through are those that have gone through it themselves.

You might be wondering whether you are on the right track when you picked up this book. You may wonder if you are being overly dramatic \& looking for someone to blame by thinking your ex or parent or co-worker might be a narcissist. Here's the thing. You're smart. I am sure the narcissist in your life has given you the opposite message about yourself, but the truth is some of the smartest people I've met are people who have been in relationships with covert narcissists.

A helpful thing to notice while you are trying to find answers is the fact that men \& women who are with healthy people don't enter words into online search engines such as toxic relationships; energy vampires; mean spouses; confusing relationships; hidden abuse; subtle abuse; manipulation; narcissism; covert narcissism; sociopaths. The same is true for people who are going through a divorce or a break up where they just realized they weren't a good match, or they fell out of love, or they find themselves wanting other things. If you are searching for answers because you feel utterly confused, you are on the right track because you're smart. If your body feels weak \& flustered around someone, it knows something is not right. If you feel like you are going crazy, you are not alone; this is a common feeling among survivors.

Trust your gut, your intuition, how your body feels. There is nothing wrong with you. You know more than you probably give yourself credit. You are a brilliant individual that has been beaten down, lied to, \& manipulated, so you naturally have a lot of self-doubts. That is normal \& completely understandable.

What you have been through is not a small thing. There are several types of narcissists. The covert type is 1 of the most destructive to your heart, psyche, \& physical body because it is so hidden, unrecognizable, \& you are usually the only one that sees it. People who know the narcissist in your life probably think they are 1 of the nicest people they've ever met \& often wish they could be as lucky as you to have a mom, husband, dad, wife, boyfriend, boss, or friend like you do. They feel the same way you did, maybe for a long time, about the covert narcissist in your life. They have experienced the same illusion, just not identified the truth.

When I was in the middle of my own confusion, searching for information \& understanding my eyes began to open. The more they opened, the more overwhelming grief \& anger I felt. With time, education, \& support, this awakening turned into a growing strength \& hope inside me. This will happen for you, too. Reading this book is going to be incredibly helpful for you as you begin to awaken to the truth of what you have been through.

Here is the truth. If you have lived with a covert narcissist, you have been held down for a long time. You have experienced the illusion of love, not the real thing. You have been lied to, manipulated, \& controlled. You have not been heard or valued. You were devalued \& brutally discarded by someone who said they cared about you, but in fact only cared about themselves. You have experienced a crazy-making relationship that is difficult to describe. Your self-confidence, your zest for life, your adventurous spirit, the light inside you has slowly dimmed; there is part of you that may not want to be here anymore but is cared to say that out loud or to anyone else. I understand. I've been there. This is incredibly common among survivors.

Here is the good news. You have begun a journey that will bring you to the truth you are seeking, the truth of what you have been through, \& the realization of how stunning \& valuable you actually are. With time, you will have clarity \& feel strength \& freedom that may be hard to comprehend right now, but trust me it's possible. You will experience love (the real thing this time), \& you will cherish every moment of it because of what you've been through. Your light will come back brighter than it's ever been. You will be able to love people \& help others in ways you couldn't have before. You will be free. Life will actually feel enjoyable, \& you'll be glad you're here. I promise you this is all possible.

I wrote this book for you, putting all the valuable information I have learned into 1 book to make it easier. I spent years researching for my own understanding \& healing. For this book, I decided to dive even deeper by reading more books, finding additional articles, listening to more YouTube videos, \& interviewing over 100 people around the world who have experienced a covert narcissistic wife, husband, mother, father, sister, brother, boss, boyfriend, girlfriend, or friend. I wanted to make this the most comprehensive, thorough, \& accurate book I could for you. The interviews were fascinating. Even though our stories were different, \& some of the relationship types were different, I found myself feeling like I was looking into a mirror when each person would tell me their story over Skype or across the table at a restaurant. These brave people furthered my motivation to get all this information down in 1 place.

I will be using stories from the people I've interviewed, but I have changed their names \& specific information to protect their identity. I will not be sharing who the narcissists are in my life. I also interviewed therapists \& life coaches who specialize in this area. They were extraordinarily helpful, \& I will share what I learned from them as well.

At the end of the book, there will be a list of helpful resources for you to do further study if you'd like.

This can feel like a lonely road since oftentimes you are the only one who ever sees this side of the narcissist. To give you a sense of how not alone you are, here is 1 statistic:
\begin{quotation}\it
	The World Narcissistic Abuse Awareness Day website \url{www.wnaad.com} estimates there are ``over 158 million people in the USA alone that are narcissistically abused by a person with either narcissistic personality disorder or anti-social personality disorder.'' The two have similar traits.
\end{quotation}
This is a massive problem that seems to be growing. 1 therapist I interviewed said she didn't know what was happening, but every person that has walked through her door in the past couple years is dealing with narcissistic abuse. Her appointments are constantly booked.

You are definitely not alone.

Meeting the people I've interviewed has made me discover a world out there I didn't know existed. When survivors find each other, there is an immediate connection, a feeling of safety, of understanding. We find ourselves enthusiastically nodding with relief when we hear each other's stories.

My intention for this book is to provide you with a tremendous amount of information as well as talking about different things you can do to heal.

I will be using 3 different terms when referring to a person who has experienced covert narcissistic abuse \& would like to explain my thoughts behind them.

If you have been the recipient of this crazy-making behavior, you were a target, a victim, \& are a survivor. The word ``victim'' can bring up reactions in people because we are warned against having a ``victim mentality.'' The truth is you were a victim. This doesn't mean that this needs to be a cloud that follows you the rest of your life. You were a person who was harmed. This is the direct definition of a victim. 

You were also a target, as hard as that is to believe \& accept. Covert narcissists seek out certain types of people. They look for people who are kind, authentic, self-reflective, nurturing, loving, \& caring people with a conscience. They look for energy supplies. Without these attributes, the narcissist has no use for you, \& their manipulative tactics wouldn't work for you.

You are also a survivor. You experienced subtle, manipulative abuse \& you are still here. Many people come out of these relationships not wanting to be here anymore after years of being emotionally beaten down. So the fact that you are still here, getting up every morning is something to be recognized \& commended. You are a survivor. You are stronger than you know.

My hope is this book will bring you the clarity \& understanding you want \& need.

Welcome to the beginning of your freedom.'' -- \cite[pp. 13--23]{Mirza2017}

%------------------------------------------------------------------------------%

\section{What is a Covert Passive-Aggressive Narcissist?}
``There are several types of narcissists. Some are classified as overt, covert, somatic, cerebral, parasitic, \& boomerang. If you search for types of narcissists on the Internet, you will find countless articles listing many types \& subtypes. All narcissists have the same core traits. The official list of these traits is found in the Diagnostic \& Statistical Manual of Mental Disorders, 4th Edition (DSM-IV) by which psychiatric diagnoses are categorized. Mental health professionals use this manual when diagnosing patients.

A patient must have at least 5 of the following traits to be diagnosed as having Narcissistic Personality Disorder according to the DSM-IV:

A pervasive pattern of grandiosity (in fantasy or behavior), need for admiration, \& lack of empathy, beginning by early adulthood \& present in a variety of contexts, as indicated by 5 (or more) of the following:
\begin{itemize}
	\item Has a grandiose sense of self-importance (e.g., exaggerates achievements \& talents, expects to be recognized as superior without commensurate achievements).
	\item Is preoccupied with fantasies of unlimited success, power, brilliance, beauty, or ideal love.
	\item Believes that he or she is ``special'' \& unique \& can only be understood by, or should associate with, other special or high-status people (or institutions).
	\item Requires excessive admiration.
	\item Has a sense of entitlement, i.e., unreasonable expectations of especially favorable treatment or automatic compliance with his or her expectations.
\end{itemize}
What are your thoughts after reading this? You might look at these \& see how they perfectly fit the person in your life who you think may be a narcissist. Perhaps you might feel confused because this list doesn't exactly sound like this person you're trying to figure out. Maybe 1 or 2 traits match. If it doesn't quite match or only matches a few, the confusing person in your life either isn't a narcissist, or they might be a covert narcissist.

The word \textit{covert} is defined in the Merriam-Webster dictionary as ``not openly shown.'' \textit{Passive aggressive} is defined as ``displaying behavior characterized by the expression of negative feelings, resentment, \& aggression in an unassertive passive way.''

All the narcissistic traits are true of overts \& coverts. The difference is the covert narcissist hides their dark attributes because they want people to like them. Their reputation is extremely important to them. Overt narcissists are usually annoying people. Most people don't like them. They are showy. They love to talk about their achievements \& accomplishments. It is obvious they are all about themselves. The overt narcissist is the type of person that will go on \& on about how great they are, how much they've accomplished \& achieved while people in the room listening are rolling their eyes.

Overt narcissists tend to have shorter marriages \& romantic relationships. It is common for people to be married to coverts for decades \& not know they are married to 1 for most of the relationship. It is also common for people to be in dating relationships with covert narcissists that go on for years. Children of covertly narcissistic parents often do not realize the truth about their mom or dad until their 30s.

Here are some examples of how the traits from the DSM-IV look in a covert narcissist. These people can be pastors, spiritual leaders, therapists, \& heads of non-profit organizations. They can be politicians who are charming, look you right in the eye, \& really seem to care. Coverts do have a grandiose sense of self, are preoccupied with fantasies of power, require excessive admiration, but they hide all these attributes so people will like \& trust them. They know if they are obvious about their self-absorbed traits, people won't like them. They believe they are ``special'' \& entitled, but they know it would turn people off to let that out. They know they must appear humble to be liked \& revered. They know how to play people, how to charm them. They are master manipulators.

They don't have empathy but have learned how to act empathetically. They will look you in the eyes, making you feel special \& heard, make sounds \& give looks that tell you they care, but they really don't. They mirror your emotions, so it seems like they have empathy. They have observed \& learned how to appear to care. They thrive off the attention of others. People that think or act as if they are amazing are their energy supply. They have people around them that adore them, respect them, reverse them, see them as special \& almost perfect, \& in some cases seem to worship them.

``Holy Hell'' is a documentary that is a great example of a covert narcissist that led a cult with a large following who stayed with him for over 20 years. The people that followed him aren't stupid. They are smart, kind, talented, tender people who were exploited, used, \& convinced by a covert narcissist (CN) that appeared to love \& care about them.

After living with a CN for a long time, cult deprogramming would actually be more beneficial than regular therapy with a therapist who does not understand this disorder. The effects of ending a relationship with a CN are similar to the effects of coming out of a cult. There is a lot of deprogramming that needs to happen in order to heal \& see clearly. It is gut-wrenching in the 1st stages. If you watch the interviews at the end of ``Holy Hell,'' you will be amazed how much you will relate to what the people are thinking \& feeling when they finally leave the cult leader. Even though you haven't come out of a cult, it is a profoundly similar experience.

Covert narcissists are likable to the outside world; they appear to giving, humble, \& kind. Image is the most important thing to them. These people are law-abiding citizens. They usually have well paying steady careers. They are not outwardly aggressive. You could know them for years \& never see this side of them. This can change during the discard phase, which I'll discuss in a later chapter. It is usually only the person that gets to know them intimately that sees the destructive traits. The rest of the world sees the fa\c{c}ade, the ``nice guy.'' Many therapists don't see through the mask \& indeed are often impressed with how kind \& aware they are. CN's seem to get worse around middle age; they rarely change because narcissists blame others because they usually don't think they have a problem.

They are generally successful \& charming. Everyone loves them on a surface level. They tend to not have long-lasting friendships with people that really know them deeply. They may have friends that have known them for years but don't really know them. Yet they are rarely without a partner. After they discard you, they usually move on quickly to another source, another target that will think they are so lucky to have found such a ``nice guy'' just like you did in the beginning.

Many times daughters will consider their CN mom to be their best friend until later in life. It is a devastating realization when they recognize that the person who they thought loved them the most has actually been using them for years. They don't know what to believe anymore. This new awareness at the same time helps validate mixed messages they received growing up.

Covert  narcissists are often doctors, lawyers, military officers, pilots, motivational speakers, pastors, actors, professors, spiritual leaders, \& therapists. They will have careers that are impressive to people \& seek positions of authority. There can be exceptions to this, but this is quite common.

The covert type is the most insidious type of narcissist because you don't know what's coming at you. It is all about them, but they know how to appear like it isn't. E.g., they despite taking care of you when you are sick of recovering from surgery or an injury. They won't tell you that, but you feel it. They let you know through passive-aggressive ways. To family \& friends, they will tell stories of how much they feel for you \& appear to be taking exceptional care of you. They will come across as humble \& will be sure to paint a picture of being a great caretaker. People around you will think how lucky you are to have someone so tender \& loving by your side. The CN might even do things that look like they are taking care of you, but you will feel their resentment of you, finding yourself feeling alone \& unsupported even though they are doing things that appear to be helpful.

An overt type might yell, call your names, \& put you down by saying you're lazy \& leave you to fend for yourself. You will feel like a covert thinks you are lazy, but they won't actually say it. You will feel how much they hate taking care of you, but they won't tell you that. They might word things in a way that gives you that message without directly putting you down. They will give you subtle messages that make you question yourself. You think you're just being too sensitive, reading into things, after all, they didn't actually tell you they think you're lazy. You will find yourself feeling badly for taking up their time, for inconveniencing them, \& often end up apologizing for something. A CN will somehow manipulate things so the attention comes back on them, \& you won't even notice it happening.

They will do things that are unkind to you, but somehow you end up apologizing. It's not uncommon to feel like things are your fault. They aren't doing anything wrong; you convince yourself. When you are with a CN, you learn to ignore your gut feelings, your instincts, \& over time believe the narcissist more than yourself. You will come to realize that the CN has slowly programmed you to see things the way they want you to see them, gave you messages about yourself they want you to believe so they could keep controlling \& manipulating you into continuing to be their ``supply.''

It is common for a survivor to have a really hard time explaining what they have been through, because in their mind it doesn't sound that bad, \& they fear people will think something is wrong with them. A phrase I hear often in local support groups is ``crazy-making.'' Whenever 1 person uses that term, the room erupts in enthusiastic nodding of heads, with smiles of empathy \& relief that we are not the only ones that feel this way.

Many share how alone they feel \& misunderstood by others who have not experienced this type of hidden abuse. A survivor will often start a sentence by saying, \textit{``I know this might not seem that bad, \& I'm embarrassed even saying it, but $\ldots$''} After she's done sharing, the whole room is filled with other survivors saying, \textit{``I totally get that!''}

When you 1st begin to realize a person you have loved \& fully believed loved you is a covert narcissist, it is so hard to believe because you have been them in such a different light for so long. It is a struggle for the brain to reconcile the man or woman you thought existed with the one that is now treating you with such anger \& hostility. This is called \textit{cognitive dissonance}, having 2 competing thoughts in your mind at the same time. This is part of the confusing \& crazy-making feeling you might be experiencing. It is painful \& exhausting.

A covert narcissist can appear to be a loving partner for a long time. Their behavior often becomes more aggressive at the end of the relationship. This is when the narcissist traits listed in the DSM-IV become more obvious; the sense of entitlement, superiority, \& arrogant attitudes become more pronounced. They will still be covert with others, but the survivor will see \& experience more of the overt traits coming to the surface. Their mask cracks when you, the survivor, begin trusting yourself. The stronger you become, the less they can control \& manipulate you. When this happens, they no longer need you. You are no longer supplying them. This is when you feel their rage more than ever. This is when their behavior turns aggressive, cruel, \& shocking.

After victims have left these relationships, they look back \& realize how depressed they were for the majority of the relationship, how alone they felt, how they blamed their constant fatigue, health issues, \& sadness on other things, not realizing the toll the relationship was taking on their body \& spirit. As a result, it is common for victims to experience many healthy issues while they are in these toxic relationships. This is because covert narcissists slowly break your spirit over time without you seeing it, \& you end up feeling emotionally like you were the problem, which results in physical manifestations of various ailments.

The covert narcissist is the common title for this personality disorder. I named this book \textit{The Covert Passive-Aggressive Narcissist} because covert \& passive-aggressive behavior are both parts of being a CN. \textit{``They are both indirect ways to aggress, but they are most definitely not the same thing. Passive-aggression is, as the term implies, aggressing through passivity. In contrast, covert aggression is very active. When someone is being covertly aggressive, they are using calculating, underhanded means to get what they want or manipulate the response of others while keeping their aggressive intentions under cover.''} In \textit{Sheep's Clothing: Understanding and Dealing With Manipulative People} by George Simon.

It is a lot when you are 1st discovering things you haven't seen for years, unfolding a picture that does not match what you thought was true. Take deep breaths as you go through this book to help your body as it is taking in new thoughts. We will dive in deeper to what the behavior of a covert narcissist looks like \& how this affects you; then we will talk about healing. That's where you get to exhale. Be extraordinarily kind to yourself throughout this discovery process. You deserve tenderness more than ever.

A big part of healing is educating yourself.

So, let's talk about the common pattern of love-bombing, devaluing, \& discarding that are the typical stages with covert narcissists.'' -- \cite[pp. 24--31]{Mirza2017}

%------------------------------------------------------------------------------%

\section{The 3 Phases: Love Bombing, Devaluing, \& the Discard}
``There is a relationship cycle that is typical of narcissistic abuse. It generally follows a pattern that includes 3 stages. The 1st stage is often referred to as love bombing (or the idealization stage), followed by devaluing, \& finally the discard.

Having these described as stages can give the impression they are in sequential order. In some ways they are, \& in other ways, the 1st 2 are experienced intermittently throughout the relationship until the discard. The cycle of all 3 can also repeat numerous times. The combination of the stages creates a dizzying whirlwind of emotion \& confusion.

I will describe the stages in the context of a romantic relationship. But no matter what type of role this person has in your life, you will be able to see these 3 phrases explained \& illustrated.'' -- \cite[p. 32]{Mirza2017}

\subsection{Love Bombing\texttt{/}Idealization Phase}
``Love Bombing happens at the very beginning. This is where the groundwork is laid for you to fully trust \& believe in this person for years to come. Because of your initial experience with them, you end up seeing everything they do through the lens of the solid view that this is someone who is a good person, someone who cares about you, \& someone you can trust with your heart.

The idealization phase usually lasts between 6 months \& a year. This is generally the case, but not always.

Here are some descriptions people I interviewed gave me of the covert narcissists in their life during the love bombing phase:
\begin{itemize}
	\item He was so kind.
	\item I felt so lucky to find him.
	\item He was different. He talked about his feelings.
	\item He asked me lots of questions about myself. He really wanted to know me. He seemed to really care.
	\item He was kind of shy.
	\item We were so much alike!
	\item He opened up to me about his abusive childhood. He was really honest \& vulnerable.
	\item She was beautiful. Out of my league. I felt so lucky that she liked me.
	\item She was fun.
	\item He felt like my soul-mate; like I had known him for a long time.
	\item He was interesting, intriguing.
	\item He was confident. He seemed to have his life together.
	\item He was great with kids.
	\item I felt lucky to be with her.
	\item He seemed tender.
	\item I felt safe with him.
	\item He was a really good listener.
	\item He was humble, kind, sensitive, \& easy to connect with.
	\item He could get along with anyone. It was remarkable.
	\item She had everything I was looking for.
	\item He was spiritual, open, \& philosophical.
	\item He was soft, which was so nice after experiencing a lot of anger in other relationships. I felt like I was going into this relationship with my eyes wide open.
	\item We talked about everything. The communication was great!
	\item I didn't know men like this existed!
	\item My friends \& family were so happy for me that I had found such a great guy.
\end{itemize}
Many told me they felt so at east with the CN in the beginning stage. \textit{``Although it feels amazing at 1st, this idealization is actually responsible for most of the damage when the relationship comes crashing down. They set a trap, \& it's a trap no unsuspecting victim could hope to escape from.''} \textit{Psychopath Free} by Jackson MacKenzie.

It is very common for targets to say, \textit{``We seemed so much alike.''} This is because the covert narcissist mirrors you in the beginning, in a sense becomes you. They are observing you during this period. They will ride the wave of emotion you are feeling, so it feels like they are just as excited about this relationship as you are. This can carry on for a while. Many survivors look back \& realize that the excitement they felt, the energy of the relationship they so believed in, actually only came from them. They were the only source of life \& were under the illusion that it went both ways because the CN was mirroring their emotions.

CNs are often chameleons that become whoever they are around. They don't have a strong sense of self. They pick up what a person wants, \& they become that. Because of this, people are impressed with how well they can seem to relate to all types of people.

I spoke with 1 woman who would watch her narcissistic mom observe other people's insecurities \& shower them with compliments \& praise in those areas. The ``targets'' felt loved, seen, heard. Her mom didn't care about these people. She only wanted to look good \& be impressive. She was using them for the attention \& admiration she received from them. They were her energy supply.

Similarly, if a target is spiritually minded, it is common for them to feel like they have found their soul mate when they meet \& date a CN. The connection feels like home. The CN mimics the same zeal for spirituality as the target genuinely has, which feels amazing to the victim. They are on the same page it seems. This is an illusion the CN presents.

It is common for CNs to test targets, to see if they are someone who will be a supply for a long time. Some of the women I talked to told me after 6 months to a year of dating, their CN started to have doubts about them. When the negativity started, the CN's target kicked into fighting for this relationship because she\texttt{/}he believed in the connection so much. This made them perfect for the CN because they now proved they would stick with them through anything. This is the type of person they want, the type of person the CN will groom.

When Sara dated Timothy, after a year of bliss there began to be issues in the relationship. They both agreed this relationship was worth fighting for, so they decided to go to therapy to figure things out. Sara thought it was amazing he was open to doing that. Most of the guys she had dated before would have never agreed to counseling. She believed it was 1 more impressive thing she had discovered about him.

What she didn't know was 1 of the worst things you can do is take a CN to therapy, especially in the beginning. Here is why. It's like a training ground for them. When the counselor tells them what they are doing wrong, how they are hurting you, it shows them which part of their mask is cracking. They learn what you want, \& what they need to do to impress you as well as others. They do what the therapists suggest, impressing the target \& the therapist. Their heart isn't in it, but they act like it is. The therapy sessions make you feel even more love \& respect for them, once again sealing their image as the perfect mate, ensuring your love \& loyalty for a very long time.

They learn your vulnerabilities \& insecurities. CNs make sure to build you up \& compliment you in these areas. It can feel like they are part of your healing. They will later use what they learn about you to trigger you, manipulate you, control you, \& would you in ways that feel like the biggest betrayal you have ever felt.

They also hook you with sympathy in the beginning. when Sara \& Timothy were in therapy, he shared that he had never felt like anyone had really wanted to get to know him, everything about him. He expressed how much he longed to have someone take the time to pursue \& love every part of him. This tugged at Sara's heartstrings. She is a caring woman, full of empathy. These are common traits of targets. She determined she would live her life doing just that. She would whole-heartedly get to know everything about him \& love him like he had never experienced. She would give him all the attention he was craving, getting to know him in the way he longed to be known.

She did that for over 25 years. When he would do things that weren't kind or respectful, she would see him as a wounded man that never got the love he needed \& would excuse his behavior over \& over because of it. He used her sympathy to control \& manipulate her for decades. She would never have tolerated a lot of his behavior if it hadn't been for the groundwork laid in the love bombing phase. Later, during the discard, Timothy told Sara it was clear to him that she never loved him.

After a survivor has experienced the discard phase \& discovers they have been living with a CN for years, they feel embarrassed. \textit{How did I not see this? How did I live with this for so long \& be okay with it? What is wrong with me?}

It is important to know these are master manipulators that could fool just about anyone. People who haven't experienced this will never fully understand. When others hear the stories, they can wonder why the survivor stayed for so long. It all begins with the love bombing stage. It lays the foundation \& sets everything in motion.

I also want you to know that all the survivors I interviewed were intelligent people. Many of them were aware of psychological concepts. Some are in the mental health care field themselves. They are tender \& have a tremendous amount of empathy. Many of them are also highly intuitive \& aware of toxic behavior. They pick up when something is off with others. These are not na\"ive people. You can be super smart as well as highly aware \& still be fooled by a CN.

Don't feel bad about yourself if you are a survivor. You've bad enough of someone else making you feel bad about yourself. You are smart. You are strong. You got involved with someone who used your beautiful traits against you. That is not your fault. Millions of people are taken in by CNs. Be extraordinarily kind to yourself.

All the kind words \& actions from the CN during the love bombing phase -- all the attentiveness, the open communication, the compliments, the ease of it all -- sets you up so that when the subtle devaluing begins, you don't even notice.'' -- \cite[pp. 32--36]{Mirza2017}

\subsection{The Devaluing Phase}
``The word ``de-value'' says it all. At the beginning of a relationship with a covert narcissist, you feel incredibly valued. Then you begin to experience little things, statements they make, looks they give that begin to demean \& devalue you. It is all very subtle. Over a long period of time, you are given the message by someone you love \& trust that you have to value, no matter what you do, no matter how kind you are, no matter how much you do for them, you will never ever be enough for them. The cold, hard truth is you do not matter to them, \& unfortunately, the message you end up receiving is that you do not matter, period.

The confusing thing is, while you are being devalued, you are also experiencing kindness. You receive beautiful love letters, affection, \& loving gestures. You continue to believe this is a good relationship, \& your partner loves you. You ell everyone around you how lucky you are to have the partner you do because you sincerely believe that. Your friends tell you they wish their husband\texttt{/}wife\texttt{/}partner were more like yours. However, though you are saying all of these things, you don't notice your self-image \& self-worth slowly declining over time.

Through the years, you notice your health isn't great, you feel a low level of depression, you aren't that happy, but you contribute these things to other things in life or blame yourself. They way your CN spouse treats you goes unnoticed because it has become your normal. You don't notice the consistent devaluing because it is so subtle, so the connection of how you are feeling in life is not seen as being a result of the trauma of living with an abuser.

Susan thought she had almost the perfect marriage. There were issues in their sex life she could never figure out, but everything else seemed great. They looked like the ideal couple to those around them. Like all victims, the discard phase was incredibly confusing for Susan as well as excruciatingly painful. Her husband of 18 years was suddenly done with her, telling her how happy he was without her, blaming her for all kinds of things. He told her how unhappy he had been the whole time they were married \& listed all the ways it was her fault. She was blindsided \& looking for answers as to what in the world just happened.

As she learned more about covert narcissism, Susan decided to read through her journals hoping she might see things that would help her make sense of what was happening. She wondered if she would see things in her writings that she wasn't remembering or hadn't noticed. \textit{Had there been devaluing?} She didn't remember any. From her recollection, there hadn't been anything wrong with the marriage except for their sex life, which she had blamed herself for anyway. But when she began reading from the time they met until the end of the relationship, she couldn't believe all she saw with more educated eyes. She was stunned when she read a part of her journal that was written while they were dating that said, \textit{``I'm getting married soon, \& I don't know why, but I have this strange fear that I will be taken advantage of \& not even see it.''} Her body knew from the very beginning. But like so many other times, she didn't trust it. She explained it away \& for the next 18 years, she made countless excuses for his subtly demeaning behavior.

She also couldn't believe how many stories she came across in her journal showing how he had devalued her, sabotaging every vacation, birthday, \& most holidays, times that meant something to her. All the stories started about a year after dating. A few months before they were married, she read about a trip he had gone on with a friend. She didn't hear from him for 10 days. Silence. No contact. No explanation. She had written in her journal how it felt like everything was on his time, what worked for him. She had also forgotten that she planned the entire honeymoon without his help \& most of the wedding. He wouldn't show up for appointments when he had promised he would. Throughout the marriage, she saw how he did so many subtle acts to make her feel like she was too opinionated, too strong, too loud. He never acknowledged the mother she was, the wife she was, all she did for him \& their family. He did not acknowledge any of her accomplishments. She never felt like she did enough to satisfy him. She was stunned as she read her own writings finding story after story of the ways she was devalued for a very long time \& didn't see it.

It can be hard for people to understand why someone like Susan would stay with a man that would treat her that way. The love bombing phase is incredibly powerful to the psyche. The devaluing stage is mixed with many loving acts. That's the incredibly confusing part.

At the same time he was devaluing Susan, he was also telling her how beautiful she was, giving her heartfelt cards. They were enjoying times together, laughing about things, going on road trips, talking about their dreams together, bonding over movies, books, \& their mutual love of scuba diving. As the years passed, Susan found herself feeling worse about who she was. She was tired a lot, started experiencing allergies she had never had previously, she felt drained; like she didn't have the life inside her she used to have. She never associated these things with her marriage; she thought she just needed to figure out how to find more fulfilling work in her life. Maybe she needed to change how she eats she thought, exercise more. She had gained weight over the years. There were a lot of things to which she attributed the way she felt, but never to her husband.

As she read her journals, she also realized how alone she felt for the majority of the marriage. She would write about how lucky she is to have a husband like she did. However, when she really looked at it, she, in fact, didn't have a partner who really cared about her \& wanted the best for her.

This is such a difficult thing to grasp for victims of covert abuse. They convince themselves that the love they feel for their partner is also how their partner feels for them when in fact this is no true, \& never was. Covert narcissists are not capable of real love. It was an illusion. That is an incredibly painful \& disillusioning realization.

The devaluing stage comes on very subtly. They don't call when they said they would. They don't show up for appointments that they said they would be at. Little acts that always have excuses that give the target the message that they don't matter. They will do things like invite you to dinner; then when you arrive, it feels like they don't really want you there. You feel confused. They use the silent treatment to make you wonder if you are doing something wrong. A CN will control you through their moods, through looks they give you, through statements they make that may not seem like put-downs on the surface, but make you feel bad about yourself. They will say nothing is wrong when it feels to you like something is wrong. During the devaluing phase, the victims are programmed to not trust themselves.

They will also devalue you by letting you think something is your fault when it is actually their issue. This is called \textit{projection}. They project what is true about them onto you \& you end up taking the blame without even noticing. The emotional needs of the victim are not of importance to the CN. Only the CN's desires, needs, or priorities matter to them.

It is also quite common for the victim to become responsible for everything. E.g., CN's don't like to help around the house. They will, but the target will feel their anger \& irritation. After time, the victim learns it's just easier to do things on his or her own \& to not ask for their help. The CN does not want to give in the relationship, only receive.

The mixed messages you get from a CN wreak havoc on your heart, mind, \& body. They love bomb you \& devalue you interchangeably for years. It is hard to make sense of because you have a solid belief that this person loves you \& wants the best for you. The devaluing is often so subtle you don't notice yourself slowly declining inside as time goes by. Your self-worth \& confidence are diminished as well as your physical body feeling more tired than normal. You slowly forget the free spirit you used to be \& attribute things you are feeling to circumstances outside your relationship.

Targets receive lots of messages about themselves from the CN. Some they say right to your face while others you receive from the CN's actions, looks they give, deafening silence, the quiet rage emanating from them.

Real love never has mixed messages, \& when the final discard happens, the truth comes out about the CN that had been disguised for years.'' -- \cite[pp. 36--40]{Mirza2017}

\subsection{The Discard}
``This phase feels like the most confusing painful betrayal you've ever felt in your life. The person you have loved for years \& who you believed loved you back is now saying the cruelest things you would have never imagined possible. They are treating you like a child, teaching you, punishing you, \& telling you how you should be behaving. Every vulnerable thing you shared over the years with them is now being used to wound you in the most devastating way. They are lashing out at you with what feels like a fire hose of insults. Sometimes they are calm \& sound rational. Other times they are rattling off a slew of words that make absolutely no sense but are delivering them like they are the most normal thinking on the planet. They are also mixing in words of love \& affection. Then in the next breath, they are telling you how vile you are \& how done with you they are. You have no idea who this person is. This is nowhere near who you thought you had been living with all these years. You are left reeling.

They paint a false reality \& say things about you that simply aren't true, but you question yourself wondering if they are right, because they sound so confident, act like they know more than you, \& you feel like you can't think straight. They twist your words \& confuse you with strange thinking. This leaves you questioning \& doubting yourself constantly. You feel weak, confused, \& incredibly fearful about your future. You feel very alone.

This is the time when most survivors hear someone tell them their spouse\texttt{/}partner sounds like or might be a narcissist. Sometimes this happens in therapy, in a meeting with a divorce attorney, during a search on the Internet when you are trying to figure out what is happening, or in talking to a friend. You find yourself spending all your free time watching YouTube videos, reading books \& articles, seeking answers, \& trying to make sense of this dizzying treatment you are experiencing.

You also find yourself spying on your spouse\texttt{/}partner when you never have considered doing this earlier in the relationship. You are reading their emails, checking bank statements, wondering if they are having an affair. You are asking them questions, \& they are not giving you answers that make any sense. In fact, they are usually taking your words \& turning them against you to show you all the things that are ``wrong'' with you. They are defensive \& angry. Then they are calm \& void of feeling.

This is a crazy-making time for so many reasons, 1 of which is the vast difference in how you feel vs. how the CN feels. You are devastated. You are crying, curled up in the fetal position. They are done. They move out quickly. You are trying to find answers. They are not. You are deeply sad. They are letting you know they are the happiest they've ever been now that they aren't with you.

Many times they initiate the discard during a time that is special to you, or in a place that means a lot to you. They like to sabotage dates \& places that mean something to you. Bill told me his wife of 26 years told him she was done being married to him on his birthday. After years of telling him how handsome he was, how much she loved him, how amazing he was, she told him, \textit{``I have never fully trusted you! Most women should have never lasted this long with you. I can't believe you haven't gotten therapy for your issues after all these years!''} She kept going with a long list of other shocking \& devastating words.

Similarly, when Karen's husband told her he wasn't sure if he wanted to be married to her anymore out of the blue after 17 years of being together, they were at her family's lake house where she had always felt the most safe \& at peace growing up.

For many victims, the discard phase is the end of the relationship. Others experience a confusing cycle of breaking ups \& getting back together several times over years of dating a covert narcissist. I spoke with targets that had been in relationships like these for over 10 years.

During the discard phase, you feel very low about yourself. The CN paints a picture of you that is not accurate, but they make you feel like it is. There are enough grains of truth mixed in with bizarre distortions of reality that make you wonder if they are right about you. It is an incredibly confusing time. The CN also becomes more aggressive with their words \& actions than you've seen before, but once again, you are the only one that sees this side of them.

The discard phase is sudden \& harsh. When they are done, they move on quickly \& usually go right to another target. This is a stark contrast to how you feel. You are falling apart. You never expected this seemingly good marriage to ever end. While you are devastated at the thought of a relationship ending with someone you called your lover \& best friend, the CN is not falling apart. They are bizarrely calm \& rational. You don't see them feeling sad. You feel their rage as it is directed at you, but they are not experiencing devastation \& sadness, which makes you feel even crazier.

Sam \& his CN partner, Adam, had been together for over 15 years. They were both spiritual thinkers. They had met at a yoga retreat \& bonded over their similar thinking on life. After years of passivity, seeming so laid back \& easy going, Adam became verbally abusive during the discard phase. There would be days when Sam would receive lists from his partner about everything that has wrong with him \& would need to change for this relationship to work. Then the next day Adam would ask Sam if he wanted a foot massage, \& would calmly tell him he was just flowing with the Universe, trusting the process. When Adam moved out of the house, he would lambaste Sam with abusive emails; then a few hours later would send an email thanking Sam for how well he had loved him all those years. A CN's behavior in this place can be very manic. You don't know what you are going to get from moment to moment.

This may happen while you are still living with your CN partner. \textit{``I wish I could get rid of this anger \& resentment I feel toward you,''} Don said to Emily with concern on his face, then seconds later in a fit of rage, he yelled at her for ruining every friendship he had ever had in his life! Later, he brought home dinner for her \& their kids \& asked if she'd like him to make her favorite tea at bedtime.

Blind-sided, shell-shocked, dazed, \& confused are some other ways to describe this time. They cut you off quickly \& heartlessly. The ironic thing is the CN usually initiates the end of the relationship, but it is often the survivor that actually files for divorce. The CN wants to be liked, to be seen as the victim, not the one who destroyed a family. They want people to feel sorry for them \& see you as the one to blame. How they look to others is their top priority.

The CN will blame you for just about anything \& everything. Back to the last example, Emily counted over 30 things Don had told her she would need to change about herself for the marriage to work. The thing you start noticing when you become aware of the issues with the CN is that most of what they say about you is actually a projection of what is true of them.

After reading about this phase, you may identify that you are experiencing the discard phase right now. For many survivors, this is the time of avid research on the subject of narcissism. So many others are experiencing what you are at this moment. It's empowering \& helpful to realize this. You are in the company of some amazing people just like yourself that are also looking for answers/

You may be feeling shocked, fully of anxiety, alone, depressed. You may be having suicidal thoughts, your body may feel like it's deteriorating, \& it's really hard to focus at a time when you are making big decisions. You may be feeling reactive \& impulsive. It's probably been a while since you were able to get a good night sleep. This is all so common.

You have been through a lot \& are still in the thick of it. You will get through this. You are in the right place, \& will someday see things clearly. Breathe. Reach out to friends \& family who love you. Keep learning. Join a support group. Give yourself permission to have times of falling apart. Hire a good attorney who knows about narcissism if this is a divorce situation for you. Above all know that you deserve kindness \& respect.

I will talk more in-depth about what it is like to divorce a CN in a later chapter \& also more about their traits \& ways they manipulate \& control. Before I do that I want to talk about you -- the target, the victim, \& the survivor.

Let's talk about beautiful \& valuable you.'' -- \cite[pp. 40--45]{Mirza2017}

%------------------------------------------------------------------------------%

\section{Traits of Targets}
``Covert narcissists will seek out a certain type of person for intimacy. They know what traits someone needs to have to be able to control \& manipulate them. E.g., they wouldn't be able to use their emotions \& comments to manipulate if the person didn't have empathy, compassion, \& a nurturing heart. They wouldn't be able to convince someone to take the blame for something that isn't their fault if the person didn't have the trait of being self-reflective.

As I interviewed survivors \& listened to victims of narcissistic abuse in local support groups, I found it fascinating to see so many traits in common. Here is what I found. These people are smart. They are responsible. They are the ones who hold things together, often the heart of the family. They do almost everything in the home \& as parents. These are people on which you can rely. They are loyal, faithful women \& men.

They are often the dreamers, the optimists of the world, seeing the best in others. They are loving, kind, \& pure-hearted. They trust the word of others because they are trustworthy \& it is difficult for them to believe that someone they love would lie to them since it is not in them to behave that way. Lying, controlling, \& manipulating to against how they are intrinsically made. They were made to love \& be loved; that is when they thrive.

They are honest \& real. They do not pose. They do not pretend to be someone they are not. What you see is what you get with them, \& it is incredibly refreshing.

They tend to be flexible, easy going. Planning an event with them is a delight. They are easy working partners \& generous with their time. I was just at a support group where we discussed branching to create a women's only group. A few of us were discussing what we needed to do to make the change. We talked about how we would get the money to pay for the dues \& who would coordinate certain aspects, essentially all the details of what needed to be done to make this transition. It was the easiest conversation in the world. 1 lady offered to pay for all of it. Others said they'd be happy to split the cost. No one was stressed. No one was in a hurry. No one was out for themselves. Everyone was generous, not wanting the action steps to fall on 1 person's shoulder. It was peaceful, loving, productive, \& easy.

Meeting other survivors has been like finding gold in this world. These are people that are not interested in drama. They love peace \& harmony. They are self-reflective women \& men who are interested in growing \& bettering themselves. They look to see where they can improve. They don't blame others; they take responsibility for their own behavior. If they are feeling hurt or frustrated in a relationship, they will say, \textit{``I'm feeling this $\ldots$''} or \textit{``When you said that I felt this $\ldots$''}

They won't tell you what they think is wrong with you. This great quality, however, is used against them by a CN. CNs will blame them for things, \& because they are people who look at themselves, they will be open to seeing if they are to blame. Often, the victim finds himself or herself thinking, \textit{``There must be some truth to it. This person knows me. They live with me, so there might be something I need to change here.''} They won't recognize how badly they are being treated \& take way more responsibility than necessary.

CNs are skilled at phrasing things so you don't notice their cruel behavior. Emily's counselor told her, \textit{``Your husband has this amazing ability to start out a conversation appearing so humble \& kind. You become so impressed you don't notice how he blames you by the end. You take that on \& feel shame. You end up apologizing for something that isn't even your fault, but he sets you up to feel like it is.''}

Targets are trusting people. They are nurturing, forgiving, \& compassionate. They are full of empathy, \& this, sadly, is 1 of the most exploited traits by a CN. Your empathy is a big reason they chose you. They will prey on this treasured part of you by using displays of emotion \& manipulative comments in addition to actions to control \& punish you.

1 women with whom I talked said her brother passed away recently. She was very close to him. When the anniversary of his death came, her husband suddenly sunk into a deep depression. Because he did this, she didn't feel she was allowed to be sad. She turned her attention on him to make sure he was okay. CNs do subtle things like this to control you, to keep the attention on them, \& to sabotage times that are sacred to you.

Another example of how a CN will turn the focus back on them is demonstrated by Wendy's dramatic story. Wendy's eyes had been opened to something she experienced years ago where she now sees that her husband exploited her sensitive heart. Her care \& concern for him brought her to a place of feeling bad; she ended up taking care of him after he had been cruel \& dishonoring to her. She told me the story of how this happened. Her husband was on the phone with his mom. When they were don talking, he told Wendy that his mom had said Wendy was controlling \& manipulative. They had been married for about 10 years at the time. Here \& there her husband would go through times of questioning Wendy, telling her things his family would say about her to see what she thought. This time after starting to defend herself, Wendy stopped \& said,
\begin{quotation}\it
	``You know what, I have been defending myself to you for 10 years. If you don't know who I am by now, I don't know what to tell you. You're just going to have to decide for yourself what you think is true about me.''
\end{quotation}
He responded with silence, left the room \& went out to their backyard to think. After a few minutes he came back \& suddenly fell to the ground in agonizing pain. He said his back hurt \& he couldn't get up. He stayed on the floor at night. She felt horrible. She brought him containers to urinate in when he needed them throughout the evening. She felt like she was the cause of all the stress he felt. If she didn't exist, he wouldn't be feeling this. There would be no issues between him \& his mom. In the morning he asked her to call the ambulance since he was still in great pain. They wheeled him out, gave him some shots, \& he came home pain-free. The pain never came back. Quite a quick recovery for such excruciating chronic pain over a 2-day period $\ldots$ This is dramatic, but not unusual for a CN. Many survivors have told me stories of CNs faking injuries \& illnesses, some for years. They will go to great lengths to make you feel bad \& turn the attention back on them. He was cruel to tell her what his mom said. As long as you are feeling bad, shame, they have you. You are under their control. He should have defended Wendy to his mom. He should have never told her about the phone conversation, knowing it would just hurt her. That is what love looks like. The real thing, not the illusion you get used to with a CN. This tactic works on targets because they are tenderhearted, sensitive, caring, \& trusting people.

Another thing I noticed about targets I interviewed \& observed is they are smart individuals. They have been manipulated to think they are not, but these are people that are very intelligent. Most of them could have their Masters in Psychology by now after their research \& experience. Many of them are now helping others who are going through their own pain. These are beautiful souls. They love solutions where everyone wins. They are team players who will have your back \& be your biggest fan. I have experienced such love \& encouragement from each \& every 1 of the ones I've met. I feel grateful to know them. They are great listeners, independent, hard-working individuals, although most of them have been told they are lazy by their CN, which is a common put down \& the farthest thing from the truth.

During more than 1 interview, I found myself sitting across from a woman watching her body physically shake as she told her story. My own tears would come \& go. I have a tender heart \& a fierce one, like many survivors. These people are the cream of the crop. They are the ones we are so grateful to have on this planet. They are the ones who bring life \& love wherever they go. For me to see each 1 of these treasured beings have their spirits crushed, their souls wearied, \& their hearts wounded over \& over year after year, it did something to me. I felt angry. Really angry. This is wrong. These beautiful women in front of me have been leveled, confused, \& emotionally raped for years. Their pain was palpable. The more I heard, the more motivated I became to write this book. These survivors did not deserve any of this. 1 person caused their bright light to almost go out, \& that's not okay.

Targets are strong people. They are fiercer than the CN in their life could ever imagine. Every survivor has the capability to open their eyes \& see the truth of what they have been through, then become a force to be reckoned with. They will be a force for good, of course, because that is who they are.

That is who you are.

Now it's time to educate you, dear powerful one, on the traits \& manipulative tactics of a covert narcissist, so you can finally see clearly \& make your way home.'' -- \cite[pp. 46--50]{Mirza2017}

%------------------------------------------------------------------------------%

\section{Traits of a Covert Narcissist}
``The DSM-IV gives us a list of core traits of narcissists. However, there are other traits I have found to be common in covert narcissists.

Before I list them, it is important to note that there is a spectrum with narcissists. On a scale from 1--10, you may have a CN in your life that is lower on the spectrum. They may exhibit a few of these traits. It's also possible that you may know a CN that fits almost all, if not all of them. Narcissists can also be a combination of covert \& overt.'' -- \cite[p. 51]{Mirza2017}

\subsection{They Do Not Have a Strong Sense of Self}
``They don't have a solid identity, a knowing of who they are. If you think of certain people in your life that you know well, you would probably have a lot to say to describe them. They are distinguishable. With CNs, the description seems more generic. \textit{``They are really nice,''} or \textit{``They are easy to get along with,''} are common descriptions, but it rarely goes further than that. There is a feeling of, \textit{``Who are they really?''} If you look at CNs, you will notice a hollow feeling about them, almost vacant. They can feel like a shell of a person.

They can often be chameleons, becoming the people they are around.

It's common for a survivor to talk about the beginning stages of their relationship \& say how amazingly similar they were. Then, after the discard when the CN begins dating someone else, they begin to become just like their new target.'' -- \cite[p. 51]{Mirza2017}

\subsection{Silent Rage}
``CNs have a lot of rage inside them. They may not yell, or get violent, but you can feel their quiet rage. They mask it around others, but when you live with them, it can feel like being next to a dormant volcano that could erupt at any moment. Their rage controls the climate of the home \& keeps people feeling like they are walking on eggshells. This is 1 way they maintain control of people close to them.

This type of action immediately resonated for Mary. Over the years, she would ask her husband, \textit{``Are you okay? You seem angry.''} He would respond by saying calmly, \textit{``No. Just tired.''} Then he would stay quiet while Mary would wonder who to trust: her own instincts or his word. Her body could feel his anger, but why would he lie? She chose to trust him \& ignore her own feelings. This is part of gaslighting (will discuss further in the next chapter) \& other manipulative tools used to get the victim to slowly, after time, believe the CN over their own inner guidance.

Many victims told me they felt responsible for their CN spouse's or parent's anger. They get the unspoken (\& sometimes spoken) message they are the cause for the CN's rage. This could not be further from the truth, but it is what the CN wants the victim to believe.'' -- \cite[pp. 51--52]{Mirza2017}

\subsection{Lying}
``CNs are incredibly good at hiding \& masking their lies. Many victims find it hard to believe they ever heard lies from their CN. CNs rarely seem like people who would lie. They come across as dependable \& trustworthy. Targets in relationships with CNs often feel they have open communication about everything. However, toward the end of the relationship, they start noticing things that make them wonder.

During the discard phase, the targets find themselves spying on their partners, which is something they would have never dreamed of doing before investigating this behavior. Reading texts \& emails late at night, they begin to uncover things their CN has not shared \& sometimes catching them in outright lies.

If a target brings any of this to the CN's attention, the CN will find a way to turn it around \& make something the target's fault. Since the unwavering trust has been built over time even the most bizarre excuse from the CN makes the victim still consider their explanation. It's so hard for them to imagine their partner would lie after years of seeing them as trustworthy.

Toward the end of her marriage, Valerie started feeling strange about a friendship her husband, Jack, was having with a guy he had just met. She was happy for him, but at the same time it seemed \& felt odd. E.g., they would go on late night walks together. Sometimes he wouldn't get back home until 3 o'clock in the morning. They would send each other pictures of themselves. Jack would get really giddy every time he got to see his new friend. There were days he would ask Valerie if she could leave the house so he could have some quiet time to himself. She would \& later found out he used that time to have lunch with his friend.

During this time, Jack was pulling more \& more away from Valerie \& their kids. She was concerned \& mentioned it to him. She got the courage in 1 conversation to ask if he had romantic feelings for his new friend. She told him it was okay if he did; in fact, it would help make sense of things in their marriage. The dormant volcano exploded, hurling accusations at her, saying she had ruined every male friendship he had ever had \& now she was ruining this one.

Weeks went by, \& Valerie came across an email from the friend to her husband that said, \textit{``When are you coming over? My dick isn't going to suck itself.''} She showed Jack the email. He brushed it off saying, \textit{``Oh, that was just us messing with you because we were angry at what you were insinuating.''}

To the outside world, it was obvious what was occurring, but Valerie had been with Jack for long \& had never known him to lie. She accepted his explanation. She had gotten so used to accepting things being her fault that she didn't recognize what was right in front of her or acknowledge that even if he was telling the truth, that was an incredibly cruel, immature, \& disrespectful act. He also never once addressed her feelings. They did not matter to him. She didn't notice this because she had lived with subtle messages that she did not matter \& was not worthy of respect for a very long time.

When a CN lies \& you confront them about it, they will not acknowledge your feelings as would happen in a healthy relationship. They will never put themselves in your shoes. Instead, they will deflect, so the negative attention gets turned around on you \& off of them. They will blame you for their bad behavior $\ldots$ \textit{``you made me do it, you drove me to it, this is your fault $\ldots$''}'' -- \cite[pp. 52--54]{Mirza2017}

\subsection{Hoovering}
``Hoovering is a technique used by some CNs to make it hard for victims to move on. It is how they pull their targets back into their sphere of influence.

I spoke with several men \& women who were in relationships that would have a long cycle of breaking ups \& getting back together. This was the case with Rose. As soon as she would pull back from the relationship, the CN would pursue her more. He would apologize, explain away his hurtful behavior in terms that made sense to her \& even triggered her compassionate heart. He would be loving, tender, saying everything she was longing to hear.

He was romantic \& thoughtful. In her mind, she felt like she was seeing the real man return, the one with whom she fell in love \& believed with all her heart. She felt loved \& relieved; she took him back. Things were great for a while -- until the cycle began again. He would start to be distant, pulling away, acting in ways that confused her, doing passive aggressive things that devalued her; she would pull away, rethink things, \& he would come back as prince charming again $\ldots$

CNs observe you, groom you, they know exactly what you need to hear to get you back. It's all about control for them.'' -- \cite[p. 54]{Mirza2017}

\subsection{Constant Criticism}
``CNs will constantly criticize \& judge you. Because they are covert, this will be done in ways that are not always obvious. This enables them to control you as your self-worth slowly declines. Over time, this works on your sense of value. You end up seeing yourself as being not lovable, not wanted, \& either too much or not enough.

They will judge you \& put you down for the strangest things. Emma said her mom would make underhanded comments to her siblings often about how Emma doesn't like to stand for long periods of time. In another example, Marcy's ex-spouse would give her ``construction criticism'' because he was ``concerned'' about the way she dressed \& presented herself.

The constant criticism chips away at your sense of who you are, resulting in you believing the CN is superior \& knows more than you, creating an unhealthy dependence on him or her. This greatly weakens you over time \& makes you vulnerable to their manipulation tactics.

Their criticism of you increases \& becomes more blatant when you begin to stand up for yourself.'' -- \cite[pp. 54--55]{Mirza2017}

\subsection{Jealousy}
``Sam's CN parner never said he was jealous of him, but Sam would feel like things were different when he started experiencing more success \& happiness in his life. He started noticing that his partner wouldn't have big reactions when Sam would tell him exciting things that were happening in his life. His partner would become more withdrawn \& disengage from him.

Sally's mom would sabotage times Sally had with her friends. She was so excited to meet a group of girls after they moved to a new town. 1 night, she had a sleepover, \& her mom made it awkward \& uncomfortable. Then she would ground Sally for unreasonable things to keep her away from her new friends.

Many survivors have a hard time explaining how their CN was jealous of them. It's more of a feeling they gathered when they lived with them. Narcissists are deeply unhappy people. They get jealous of you when you are experiencing life \& happiness. They do not want you to be happy \& strong as those feelings threaten their ability to control you.'' -- \cite[p. 55]{Mirza2017}

\subsection{They Project Their Own Issues on to You}
``If you think about all the things the CN in your life has told you that are wrong with you, my hunch is if you really look at the list most of the statements are actually true of them. CNs don't acknowledge their own issues. Instead, they project them on to you. This means you end up feeling guilt \& shame for things that are not even your issues.

Amy put some of her \& her husband's joint savings into a real estate business venture. She spent a lot of time \& effort on this project for over a year. At 1st, things were going well for her, then the market crashed. She lost the money she had used to fund the project \& had to take more money out of their account to cover the loss. She was devastated. This was the 1st time she had done anything like that. It took courage, research, \& many hours of labor on her part. The work she did was amazing, but in her eyes, she failed. She felt she had failed her family. She cried on \& off for weeks feeling so guilty she had done this. Her husband didn't yell at her, didn't say hurtful things; instead, he said nothing. He was punishing her through silence. Without saying a word, he was saying volumes. She never took a risk like that again.

A healthy, loving, empathetic man would have said something like,
\begin{quotation}\it
	``So it didn't work out this time. You did an amazing job. I'm so impressed with what you were able to accomplish! You've learned a lot. You know what to do \& not do now, so go out there \& find another investment. I know this one will be a success. You'll do great.''
\end{quotation}
This response would have changed everything for her -- to be seen, believed in, \& respected. Instead she sobbed, felt so much guilt, \& experienced shame; clearly, the message from her CN was that she made terrible decisions, wasn't good with money, \& should never think about doing anything like that again.

She was able to see later that the truth was her husband was actually not good with money \& had made terrible decisions, losing money himself. He was projecting his own issue with money on to her. He had manipulated her for years, leading her to believe she was terrible with money, so she didn't notice that this was actually more his issue than hers. The way he went about giving her this message was so covert she took the shame \& blame for years without seeing the truth.

Toward the end of these relationships, the CN will be more direct about what they think of you. Here are some common things I have heard survivors say they have been told by their CN: \textit{You are controlling, manipulative, inconsiderate; you don't care about my feelings; you are lazy; it's all about you; I can't trust you; you only did that so you would look good to others.} Notice a pattern here? These are all traits of narcissists. They don't take responsibility for their own behavior. Instead, they project what is true about them on to you.

Another way they do this is by sharing their ``fears'' with you. \textit{``I'm afraid you are going to cheat on me.''} Later, you find out they had an affair. \textit{``I have this fear that you are manipulating me.''} They are the one manipulating you. CNs will manipulate you by putting the focus on you. They seem completely sincere, so you don't notice these are actually their issues.

They will say things like, \textit{``I am taking responsibility for my part of this, but you aren't taking any.''} This makes you wonder if this is true. You question yourself. \textit{Maybe I'm not. Am I?} They throw curve balls at you all the time in order to deflect the truth that they are lying. They do it in a way that gets you spinning \& self-doubting, so you don't notice the lie. They are kings \& queens of saying things that aren't true, but they say them with such surety \& confidence, it's hard to doubt them. No wonder you end up doubting yourself with all the smoke \& mirrors to keep you confused \& under their control.

There is also something else that happens, especially if you are empathic in a relationship with a CN. You find yourself feeling things that you later notice are gone after you don't live with them anymore.

Megan had a lot of self-hatred while she was married to a CN. She would take pictures of her body to be a reminder of how much she hated the way she looked. Megan would look at the pictures \& feel disgust as well as anger at herself for gaining weight over the years. This was a way of punishing herself. After 15 years of marriage, the CN left. Soon after he did, she noticed her self-hatred was gone. She is the type of person who easily picks up on what others are feeling \& wondered if she had actually been feeling the CN's self-hatred, which had nothing to do with her. He never let on how much he hated himself, but subconsciously projected his issue onto her. He knew about the pictures, but never took the time to hold her, to tell her how beautiful she was. He never felt sorrow for her pain, never felt empathy. He allowed her to believe the lies of self-hatred.'' -- \cite[pp. 55--58]{Mirza2017}

\subsection{Their Words Don't Match Their Actions}
``When Bonnie was in mediation trying to agree on a parenting plan with her CN husband Charles, he kept telling his attorney \& the mediator how much he wanted the best for his kids, how much he cared about them, \& how concerned he was with the way Bonnie parented them. His attorney would tell the mediator compassionately, \textit{``He's just trying to do the right thing.''} Then he asked for the least amount of time commitment possible to see his kids. Bonnie told me he sees his kids an average of 1 meal a week \& continues to tell her how terrible a parent she is. A CN's hypocrisy can last for years after the marriage has ended. Soon after our conversation, Bonnie called me to tell me she had just received an email from her ex-husband after being divorced for 5 years. Still continuing to only see their kids 1 meal a week, the message said he \& his girlfriend were working really hard to undo the terrible job Bonnie is doing as a parent.

CNs are such convincing people that it is easy to just listen to their words \& not notice their actions or lack thereof. This is especially true in the area of intimacy where they know they have a tremendous amount of power to manipulate.

When Ann was married to Tim, their sex life was confusing, which is very common in relationships with CNs. She had absolutely no sex drive or desire to be with him \& couldn't figure out why (her body knew she wasn't safe with him). She felt terrible that she was depriving him \& he, of course, would let her know often how hard this was for him. Thinking something was wrong with her, she tried therapy, hormones, dietary changes, \& read sex books over the years they were together. Nothing changed. She felt so much guilt \& shame. Her husband was very passive about it, waiting for her to fix herself. At the end of the marriage, he yelled at her by saying she had never tried \& how much he had. This wasn't true, but she felt even more shame hearing his words, somehow believing that he had tried when in fact he had done nothing, putting it all on her. He had spent their years together letting her take the blame, not admitting his own sexual issues, not caring about her feelings, \& not helping in any way. His lack of actions went unnoticed, \& his words damaged what she thought was true about her. It turned out the problems were actually true about him since he had unresolved sexual issues he projected onto her. A CN's words are a way of distracting you so you don't notice how their actions don't match.'' -- \cite[pp. 58--59]{Mirza2017}

\subsection{They Are Emotionally Disconnected}
``It is difficult to have a real connection with a CN. This can be a trait that is hard to put in words because it's something you feel inside you when you are around someone but have a difficult time explaining it. Something just feels off about them, but you can't put your finger on it.

There is almost a robotic feeling about CNs like they are scripted. They are so used to being chameleons, posing as someone they are not, it's like their real self is unattainable. Thinking about it another way, healthy people feel \& express their thoughts \& emotions in a genuine way. CNs tell you what they think you want to hear in order to achieve their agenda.

You think they are connecting with you when both of you are talking about feelings, but when you really look at your history, you were the one carrying the connection. Because they don't have a strong sense of self, they are not able to connect with anyone on a deep level. They will have times of opening up, sharing what they are feeling, but it feels different than when you talk to someone who is connected with who they are, someone who is genuine \& real with no agenda.

CNs will cry or rage during times that seem over the top. Annie's partner would get irate when his sports team would lose on TV. His mood would be affected for the next couple days. But when Annie almost lost her life, he seemed indifferent, annoyed that he had to take care of her during her recovery from a car accident. It's like their emotions are displaced because they are so disconnected with who they are. This characteristic makes them dangerous individuals because they have the capacity to hurt you \& others without feeling remorse.

The fundamental part of human connection is to be able to feel with each other. Without that, there is no possibility of a relationship with real depth \& authentic love. This makes it very confusing when they act like they care, but in fact actually don't. Because they are emotionally disconnected, they are unable to experience deep intimacy \& will always leave you feeling void of real connection.'' -- \cite[pp. 59--60]{Mirza2017}

\subsection{Flying Monkeys}
``This is a phrase used to describe people in the narcissist's life that act on their behalf. They are the CN's biggest fans that have a solid belief the narcissist is the victim, \& you are to blame for a multitude of things.

These people add to the already overwhelming hurt caused by a CN. They will do things like smear your name to others, send you scathing emails telling you how you are to blame for everything, testify against you in court, \& sabotage you in any way they can. A CN brings you to a place where you doubt yourself; the flying monkeys add to this since they are also telling you things that are ``wrong'' with you as well. This compounds the self-doubt, confusion, \& crazy-making behavior you already have from the narcissist.

Flying monkeys are the narcissist's enablers, their loyal team of supporters. Many times these people do not know what they are doing. They believe whole-heartedly in the CN just like you did for years. They are being fed very convincing lies just like you were. CNs are master manipulators, \& their flying monkeys are often additional victims of their deception.'' -- \cite[pp. 60--61]{Mirza2017}

\subsection{They Take Credit for Your Ideas}
``Samantha didn't think much of it at 1st when she would whisper a funny comment into her husband's ears, \& he would immediately repeat what she said to people around them enjoying the laughter \& attention. She enjoyed providing him with an experience that felt good to him. She enjoyed that he thought her comment was funny enough to share. But year after year of him doing this \& never giving her the credit felt odd.

This is also a very common trait seen in the work place between co-workers. It adds to a CN's main foal of putting as much attention \& accolades on themselves while giving the target the message they are unimportant, only there for the CN, \& not worthy of praise or attention.'' -- \cite[p. 61]{Mirza2017}

\subsection{They Withhold Praise \& Recognition}
``When Annie was researching \& learning more about cover narcissism, she found herself recalling events in her marriage that she had forgotten about \& now saw them in a different light. 1 day while clearing out her garage, she came across a painting she had created years ago. She looked at it \& realized her husband had never said a word about it. Her friends \& family had raved about it, saying she should do an art show. They were amazed at her talent. Annie began to think through all of her accomplishments over the 19-year marriage \& realized that not once had her husband said, \textit{``Wow. That's amazing! I'm so impressed!''}

On the other hand, she recalled all the times she had given him praise, complimented him, \& told him how impressed she was at things he had done. She had always supported \& encouraged him.

When you study narcissism, you start noticing abuse behavior you had missed. What often goes unnoticed is what wasn't there.

It was an eye-opening realization to Jen that not 1 time in 25 years did her CN husband ever recognize \& acknowledge the great job she had done as a mom to their children, all the times she volunteered to help people in need, \& the way she had always taken care of things to ease his load. Instead, she felt like whatever she did was never enough.

Since interviewing people who have experienced the targeting of covert narcissism, I decided to also talk to women \& men who are married to healthy, normal spouses. When you are so used to the illusion of love it is helpful as you are healing to see what the real thing looks like. It was always the same story I heard from them. So much praise. Liz has been married for 25 years to her husband \& said anytime she has accomplished anything or helped him out in any way he always praises her, thanks her, \& appreciates her. He lets her know often how grateful he is for her. This does not happen in a relationship with a covert narcissist. You get the message you don't matter.

Connie had moved every couple years during her 15-year marriage to Dave because of his job as a sports coach. She had been a stay at home mom taking care of their 3 kids through all the moves. She had been flexible \& adapted to change well, making the best of things. Dave was gone a lot during their years together, so she took care of things at home while he was away, so he was able to move up in his career without worrying about things at home. He never once told her how much he appreciated all she did, recognizing that she could have pursued her own career. After their divorce, he told her he never needed her. She had nothing to do with helping him get where he was in his career, that he could have just hired a nanny. Another cruel \& heartless message from the narcissist -- you are easily replaceable.

CNs are not interested in building you up as a person, in seeing you happy, in praising you or recognizing all you do. A relationship with a CN must be all about them. When it stops being that way, they have no more use for you \& move onto their next target.'' -- \cite[pp. 61--62]{Mirza2017}

\subsection{They Sabotage Birthdays, Holidays, Vacations, \& Meaningful Dates}
``Whenever I ask survivors what their birthdays \& holidays were like with their CN, the answer is always the same. Across the board, whether the abuser was a parent or a romantic partner, they all describe these special dates as being terrible.
\begin{quotation}\it
	``I can't even tell you what was wrong each birthday. I just remember calling my best friend almost every year crying. I never felt like I had a reason to cry. He always did something, gave me a gift, but it felt like he dreaded that day \& never enjoyed celebrating me. I would usually end up apologizing for something. It felt like something was always my fault.''
	
	``My mom always made a big deal of my birthday, but I hated that day. It was all about her. The parties were lavish. Everyone was always so impressed by them, but I would hear my mom complaining about how much work she had put into it. She would talk about the homemade cake she made \& how expensive everything was \& how hard I was to shop for. It was exhausting, \& I felt like I was the cause of her stress. I would try to help, but it was never enough.''
	
	``Christmas always felt like a struggle. He became depressed every year. I felt bad because he would tell me how he missed the magic of his childhood Christmases \& he felt sad every year that he didn't feel that same feeling. I would try to make it beautiful \& magical \& special, but his mood never changed. Then he would say things to me that gave me the message I was doing things wrong. It was exhausting.''
\end{quotation}
CNs are very passive. They put the responsibility on you to make sure they are happy \& blame you when they're not.
\begin{quotation}\it
	``He was a nightmare on every vacation. I did everything, planned everything, tried to make it enjoyable for all of us, but no matter what, he would be moody, irritable, \& grumpy the whole time. He would sulk \& complain about the littlest things. We would be in the most beautiful places, \& he would find something that would make him angry. I somehow felt like it was my fault, feeling responsible for his unhappiness.''
\end{quotation}
Narcissists do not like to celebrate you. They do not like it when you are happy. They want the attention on them, so they sabotage days \& events that are special to you like Mothers Day, Fathers Day, Holidays, Birthdays, even anniversaries of people you were close to you that have passed away. They do this through their moods, by complaining, making you feel bad about something, anything. They make it about them in all kinds of ways.

There are so many examples of this dampening effect. In 1 example, Jeanine loved to travel. It was the 1 thing in her life that gave her the most joy. She came to life every time she was able to explore new places, visit friends, \& have new experiences. 1 day, her husband sat her down \& broke the news to her. He told her he had just talked to a psychic about her \& the psychic said she was traveling too much.

They will go to bizarre measures to sabotage you \& keep you from enjoying your life. When you start doing things you want to do, things that delight you they lose control of you \& don't have the power to keep you down. They use whatever means they have to punish you \& keep you contained.'' -- \cite[pp. 63--64]{Mirza2017}

\subsection{They Belittle You \& ``Teach You Lessons''}
``CNs will belittle you in ways that are indirect \& sometimes not noticeable. There is an overall message from them that they know more than you \& you are not doing it right. This gets more aggressive during the discard phase.

This can look like advice they give you or ``constructive criticism.'' This can be especially cloaked when they are parents. They come across like they are just trying to help guide you, but you leave feeling disempowered \& scared of life, believing you don't have what it takes to figure things out. You get the subtle message you are doing things wrong, but it comes in the form of ``concern for you.'' You feel the life go out of you \& you don't know why.

During the discard phase, Jeanine's husband moved out of the house, closed their bank accounts without telling her, \& met her to discuss how money was going to work from that point forward. She was a stay at home mom taking care of 3 kids, completely dependent on his income. He told her how much he would send her each month. She told him that was less than half of what she usually needs for her \& the kids. He told her the same psychic that had previously said she traveled too much also told him that Jeanine had come to this planet to learn how to work hard \& not to expect others to support her. So, he was ``helping her'' learn her lesson by keeping the money from her.

Similarly, Mary remembers times her CN would become super nice to people when she was being direct \& bold with them. He was letting her know indirectly that her behavior was too much as he tried to counteract her strength. He would also gently put his hand on her back to steer her away from them, to stop her from talking. This furthered the message that it was not okay for her to stand up to people. He was more concerned about what the other people thought than he was about standing by \& supporting her. He was ``teaching'' her how to behave.

The way CNs belittle spouses who have stuck by them for years, been faithful to them, \& loved them is appalling, disturbing, \& inhumane. Survivors often feel like prisoners in their own home during the later stages of the relationship. They are told what they should \& should not be doing \& treated like children who need guidance. It is so upsetting to see these good-hearted people in front of me breaking down as they tell me story after story of demeaning treatment they have received by someone who they thought loved them.'' -- \cite[pp. 64--65]{Mirza2017}

\subsection{They Are Self-focused \& Emotionally Immature}
``It's been remarkable to hear so many stories of the self-centered \& emotionally immature behaviors of CN parents. Here are 3 examples that may resonate with your own experiences.

Bill's wife had issues with alcohol abuse. She had sought treatment for it \& learned different tools to help herself. She would get frustrated with her teenage kids when they didn't understand her process \& wanted them to be more enthusiastic about her healing. \textit{``They need to understand what I'm going through!''} she would tell Bill. She favored 1 child because she listened to her problems \& alienated the other ones who didn't want to hear.

Catherine's husband loved the movie Phenomenon. He was upset at his teenage daughter because he kept trying to get her to watch it with him \& she kept turning him down. He wanted her to get into his world, to see what he was passionate about. She never felt like watching the movie with him. He was very upset \& asked Catherine to talk to their daughter about this for him.

Jen's husband barely spent time with their kids after the divorce. His son kept trying to get together with him, but each time his dad turned him down with some excuse. 1 day, their son won free tickets to a movie. He was so excited \& asked his dad if he could take him. He told him he couldn't because he was going to see the Dodgers play with Sara (his new target). Their son told him he was frustrated \& disappointed that he didn't see him very much. Exasperated his dad told their son,
\begin{quotation}\it
	``Look; you have to understand I have a lot of people I need to take care of. I'm an uncle, a boyfriend, a brother, a son, \& I have things to do. \& aren't you excited for me that I get to see The Dodgers?!''
\end{quotation}
Everything truly is about them, \& the ways in which they show this to their kids is appalling.'' -- \cite[pp. 65--66]{Mirza2017}

\subsection{There Are Always Strings Attached}
``When they do something nice for you, it doesn't feel like unconditional love, like they just enjoy treating you because it makes you happy. It feels like there are strings attached \& you will need to pay them back later in some way.

The same is true of gifts they give you. It never feels like they had so much fun looking for you \& feel so much delight in giving you gifts. It feels like it was a drudgery to them, \& you better know how much trouble it caused them, \& you better feel bad, \& you better give back to them in whatever says they demand later -- \& you better take care of their feelings after they went to so much trouble.

This applies to times they show loving acts to you, like holding the door open, rubbing your feet, listening to you, doing nice things. It all feels like it will come back to bite you \& be used against you in some way if you don't cater to them. It doesn't feel like clean love. It doesn't feel like it comes from someone who cherishes you \& enjoys loving you.

In a normal relationship, there is a natural back \& forth that happens. With CNs, the rules are different. It is all about them in every area. This is a 1-way relationship.'' -- \cite[pp. 66--67]{Mirza2017}

\subsection{They Use People}
``CNs use people to get what they want. You can feel it when they act like they are listening to someone. It doesn't feel genuine. They don't feel fully engaged; like they are really there with the person.

Sometimes they talk to people to gain information from them, sometimes to get sympathy, other times to help them get where they want to be in their career. Another motivation is to get people on their side as is the case with their flying monkeys.

If you want to see covert narcissism in action, watch the television series House of Cards. I spoke to 1 target that woke up to the fact that her boyfriend was a CN by watching a scene in a movie. Film can be a helpful medium. At the end of S1.E3, Frank Underwood (played by Kevin Spacey) is manipulating a couple who have just lost their child. He comes across like he cares about them tremendously, but in fact, he is just trying to further his political agenda. Toward the end of his manipulative conversation with them where he is trying to gain their trust, he looks at the camera to reveal what he is really thinking \& says, \textit{``What you have to understand about my people is that they are a noble people. Humility is their form of pride. It is their strength. It is their weakness. \& if you can humble yourself before them, they will do anything you ask $\ldots$''}

He didn't care about their hearts, their grief. He was out for himself, \& he was using innocent people to get there. That is what CNs do. Their motives are not pure. Their lack of empathy results in a lack of conscience. People are in their lives to be used, not loved.'' -- \cite[p. 67]{Mirza2017}

\subsection{They Are Dizzying Conversationalists}
``Once their mask has begun to crack, \& their deplorable behavior starts to be more pronounced, you experience conversations with them that leave you feeling confused, muddled, \& exhausted. You find yourself questioning reality \& your own sanity.

They throw a conglomeration of words your way that makes you feel jumbled \& shaken inside. You feel like your forehead is in a perpetually confused position. You feel like screaming, wondering what in the world is happening. At the same time, they have the appearance of a normal human being that is making complete \& total sense.

Before you realize they are a narcissist, you see them as a normal person with empathy, someone who doesn't manipulate. You trust their words are coming from a place of love. You give them the benefit of the doubt, \& essentially project your own good qualities onto them. So when they hurl a bunch of words at you with statements that wound, confuse, \& sound right as well as totally wrong at the same time, it is extraordinarily crazy-making.'' -- \cite[p. 68]{Mirza2017}

\subsection{They Create Drama}
``CNs get energy from drama. They create it when it doesn't need to be there. They are not interested in creating \& promoting harmony or peace. They like to do things to keep you rattled, trying to get you to become unglued. This is done through initiating gossip, planting seeds in someone else's ears to direct them to think differently about you. Some CNs will even reach out to your friends or family to try to convince them that you are at fault, unstable, a liar \& manipulator, extending their projections to others hoping to drain you of your emotional support. They will send you a rage-filled email or text out of the blue to get a rise out of you. They will have 1 of their flying monkeys make a passive aggressive comment on a post you made online. They will subscribe to your YouTube Channel or follow you on Instagram or SnapChat to let you know they are still watching you. It's downright creepy.

Some will take you to court, mediation, or an arbitrator as much as they possibly can. It's strange how much time \& energy they spend on trying to make your life as miserable as possible. This will still happen to some survivors years after a separation or divorce. The ironic thing is many of them will talk about how they hate drama. They will put others down for being dramatic, even telling others how dramatic you are. Be aware that this is yet another projection.

They also have a remarkable way of acting completely innocent as they bring pain to others through causing confusing \& dissension.'' -- \cite[pp. 68--69]{Mirza2017}

\subsection{They Don't Make Love; They Take It}
``I have yet to hear a survivor say sex with a narcissist felt like love. Not all sexual experiences look the same among victims, but they all have the same theme. It's all about the CN. Sex is supposed to be a beautiful bonding experience where both people feel loved \& cherished. It is a way of expressing your love for your partner, a chance to give each other pleasure, enjoying each other. Just like a relationship with a CN, sex with them is a 1-sided experience.

There is so much to say about this topic that I have dedicated an entire chapter to this subject. I spoke with so many women \& men who found this area of their relationship to be so confusing \& damaging. It is important to talk about it since there is so much shame involved. Most survivors stay quiet because of the embarrassment they feel.'' -- \cite[p. 69]{Mirza2017}

\subsection{They Are Not Protective}
``A CN cares more about what others think than protecting you. Some men \& women have that beautiful protective quality for their loved ones. This is absent with a CN.

If someone criticizes you, a CN won't come to your defense. They will either stay quiet or in passive-aggressive ways suggest to you the criticism might be correct. When you defend yourself, the CN will tell you things like, \textit{``You are not open to people's opinions,'' ``you are stubborn,'' ``you can't handle the truth,''} etc.

Many people with whom I spoke who had CN parents or spouses said they felt emotionally unprotected by these people, they felt alone in these relationships. Often a CN will stay quiet when someone hurts you, which makes you question yourself.

Sherry had an experience where someone was verbally abusive to her. She was understandably upset. Her CN husband was there \& did nothing to defend her. Instead, they went out to dinner with a group of people that included the man who had been verbally abusive. Sherry felt completely uncomfortable. Her CN knew this \& talked to the other people at dinner, including the abuser, as if nothing had happened. Without saying it, her husband gave her clear messages that she was not worthy of respect \& kindness, her feelings didn't matter, \& she was alone in this relationship.'' -- \cite[pp. 69--70]{Mirza2017}

\subsection{They Create Stories in Their Head}
``CNs will tell you stories that have no basis in reality that will boggle your mind. They will create these tales \& accuse you of things that are not even close to the truth. They will presume to know exactly what you are thinking \& your motives behind your actions. Because we think of them as being normal for so long \& loving us as we loved them, it is a strange thing to witness your spouse or parent not knowing who you are, not having an accurate picture of someone they have lived with for years, sometimes decades.

Torrey was excited when she saw how far she had come as she looked at her ex CN spouse at a local event. She had felt a lot of anxiety leading up to the day knowing she would see him. She hadn't seen him for years since trying to have no contact \& had always felt weak, flustered, \& angry in his presence since the divorce. At 1st, she was rattled at the site of him, but as time passed, she found herself actually feeling love for him. After all the cruelty she had experienced, she was surprised \& pleased to feel love. She had done a lot of healing work on herself \& felt relieved to see herself not being as affected as she had been in the past. When he came up to her she decided to go with this feeling, so she hugged him \& said: \textit{``I love you.''} It came from a place of healing, recognizing that she really had loved him \& the higher part of her felt a pure love. She knew she could never be with him \& needed to continue no contact overall. However, in that moment, she decided to just let him be on his own journey \& allow herself to feel that divine ever-present love insider of her. Later he wrote her \& told her how disgusted he was at her for putting on a show, for acting kind in order to impress people around them. He accused her of being fake \& wanting attention. Do you see the projection there? CNs wholeheartedly believe the stories they create in their minds \& leave you perpetually blurting out \textit{``What?!''}'' -- \cite[pp. 70--71]{Mirza2017}

\subsection{They Have No Desire to Know You}
``Pretending to want to get to know you is part of grooming you to be their supply. It is not genuine. It is an act of manipulation. As time passes, this becomes more evident. They are not interested in who you are, what you think, \& what you feel. This is not a normal, healthy person.'' -- \cite[p. 71]{Mirza2017}

\subsection{They Have No Interest in Making This a Great Relationship}
``These are not people in relationships that fight for them or put much work into them at all. Most survivors say they were the ones planning dates, initiating communication, \& trying to nurture the relationship. If there are issues in the marriage or partnership, they are not the ones to try to find solutions \& work through things to come back together. CNs have no interest in putting effort into relationships.'' -- \cite[pp. 71--72]{Mirza2017}

\subsection{Control \& Manipulation}
``Covert narcissists control \& devalue victims through very subtle manipulation tactics over a long period of time. The impact this has on you is devastating. With each year that you are with the CN, you find yourself feeling less energy, less excitement for life, less confidence, less joy while you are in that relationship. You feel like you're existing, but not fully alive. You feel yourself slowly declining, but aren't sure why. The life in you has been drained. It's like the story of the frog. If you put a frog in boiling water, it will die a quick \& painful death. If you put the frog in lukewarm water \& slowly turn up the heat over a long period of time, the frog will eventually die without noticing what is happening. This is what it is like to live with a CN. Your essence, your spirit, the light inside you slowly drains out of you without you noticing. You feel depressed \& unmotivated, but you attribute how you are feeling to other things, often blaming yourself for things that are not actually your fault.

We are all aware of the term manipulation, but don't often recognize it when it is happening to us because we don't know what the different tactics look like that are being used to confuse \& control us. In the next chapter, I will go into detailed explanations of different ways covert narcissists manipulate so you can see what this looks like in real life \& be able to recognize their passive-aggressive behavior.'' -- \cite[p.72]{Mirza2017}

%------------------------------------------------------------------------------%

\section{Control \& Manipulation Tactics}

%------------------------------------------------------------------------------%

\section{Covert Narcissistic Parents}

%------------------------------------------------------------------------------%

\section{In the Workplace}

%------------------------------------------------------------------------------%

\section{Sex with a Covert Narcissist}

%------------------------------------------------------------------------------%

\section{Divorcing a Covert Narcissist}

%------------------------------------------------------------------------------%

\section{Why Do They Emotionally \& Psychologically Abuse?}

%------------------------------------------------------------------------------%

\section{The Most Dangerous Trait of All}

%------------------------------------------------------------------------------%

\section{Your Body Knew: Common Illnesses}

%------------------------------------------------------------------------------%

\section{What Survivors Feel}

%------------------------------------------------------------------------------%

\section{The Road to Healing \& Restoration}

%------------------------------------------------------------------------------%

\section{Traits of a Covert Narcissist Checklist}

%------------------------------------------------------------------------------%

\section{Educational Resources}

%------------------------------------------------------------------------------%

\section{Articles \& Books Quoted}

%------------------------------------------------------------------------------%

\printbibliography[heading=bibintoc]
	
\end{document}