\documentclass[oneside]{book}
\usepackage[backend=biber,natbib=true,style=authoryear]{biblatex}
\addbibresource{/home/hong/1_NQBH/reference/bib.bib}
\usepackage[vietnamese,english]{babel}
\usepackage{tocloft}
\renewcommand{\cftsecleader}{\cftdotfill{\cftdotsep}}
\usepackage[colorlinks=true,linkcolor=blue,urlcolor=red,citecolor=magenta]{hyperref}
\usepackage{amsmath,amssymb,amsthm,mathtools,float,graphicx}
\allowdisplaybreaks
\numberwithin{equation}{section}
\newtheorem{assumption}{Assumption}[chapter]
\newtheorem{conjecture}{Conjecture}[chapter]
\newtheorem{corollary}{Corollary}[chapter]
\newtheorem{definition}{Definition}[chapter]
\newtheorem{example}{Example}[chapter]
\newtheorem{lemma}{Lemma}[chapter]
\newtheorem{notation}{Notation}[chapter]
\newtheorem{principle}{Principle}[chapter]
\newtheorem{problem}{Problem}[chapter]
\newtheorem{proposition}{Proposition}[chapter]
\newtheorem{question}{Question}[chapter]
\newtheorem{remark}{Remark}[chapter]
\newtheorem{theorem}{Theorem}[chapter]
\usepackage[left=0.5in,right=0.5in,top=1.5cm,bottom=1.5cm]{geometry}
\usepackage{fancyhdr}
\pagestyle{fancy}
\fancyhf{}
\lhead{\small \textsc{Sect.} ~\thesection}
\rhead{\small \nouppercase{\leftmark}}
\renewcommand{\sectionmark}[1]{\markboth{#1}{}}
\cfoot{\thepage}
\def\labelitemii{$\circ$}

\title{A Personal Journey to Psychology: The Way I Perceive}
\author{\selectlanguage{vietnamese} Nguyễn Quản Bá Hồng\footnote{Independent Researcher, Ben Tre City, Vietnam\\e-mail: \texttt{nguyenquanbahong@gmail.com}}}
\date{\today}

\begin{document}
\maketitle
A \textit{personal} journey to psychology. A collection of quotes from different resources, e.g., psychological books, websites, forums, \& Facebook psychological pages, etc., \& some \textit{personal} (again) thoughts about them.
\tableofcontents

%------------------------------------------------------------------------------%

\chapter{Wikipedia's}
\selectlanguage{vietnamese}

\section{\href{https://en.wikipedia.org/wiki/Psychology}{Wikipedia\texttt{/}Psychology}}
``\textit{Psychology} is the \href{https://en.wikipedia.org/wiki/Science}{scientific} study of \href{https://en.wikipedia.org/wiki/Mind}{mind} \& \href{https://en.wikipedia.org/wiki/Behavior}{behavior}. Psychology includes the study of \href{https://en.wikipedia.org/wiki/Consciousness}{conscious} \& \href{https://en.wikipedia.org/wiki/Unconscious_mind}{unconscious} phenomena, including \href{https://en.wikipedia.org/wiki/Feeling}{feelings} \& \href{https://en.wikipedia.org/wiki/Thought}{thoughts}. It is an academic discipline of immense scope, crossing the boundaries between the \href{https://en.wikipedia.org/wiki/Natural_science}{natural} \& \href{https://en.wikipedia.org/wiki/Social_science}{social sciences}. Psychologists seek an understanding of the \href{https://en.wikipedia.org/wiki/Emergence}{emergent} properties of \href{https://en.wikipedia.org/wiki/Brain}{brains}, linking the discipline to \href{https://en.wikipedia.org/wiki/Neuroscience}{neuroscience}. As social scientists, psychologists aim to understand the behavior of individuals \& groups. $\Psi$ (or \href{https://en.wikipedia.org/wiki/Psi_(Greek)}{psi}) is a \href{https://en.wikipedia.org/wiki/Greek_alphabet}{Greek letter} which is commonly associated with the science of psychology.

A professional practitioner or researcher involved in the discipline is called a \href{https://en.wikipedia.org/wiki/Psychologist}{psychologist}. Some psychologists can also be classified as \href{https://en.wikipedia.org/wiki/Behavioural_sciences}{behavioral} or \href{https://en.wikipedia.org/wiki/Cognitive_science}{cognitive scientists}. Some psychologists attempt to understand the role of mental functions in individual \& \href{https://en.wikipedia.org/wiki/Social_behavior}{social behavior}. Other explore the \href{https://en.wikipedia.org/wiki/Physiology}{physiological} \& \href{https://en.wikipedia.org/wiki/Nervous_system}{neurobiological} processes that underline cognitive functions \& behaviors.

Psychologists are involved in research on \href{https://en.wikipedia.org/wiki/Perception}{perception}, \href{https://en.wikipedia.org/wiki/Cognition}{cognition}, \href{https://en.wikipedia.org/wiki/Attention}{attention}, \href{https://en.wikipedia.org/wiki/Emotion}{emotion}, \href{https://en.wikipedia.org/wiki/Intelligence}{intelligence}, \href{https://en.wikipedia.org/wiki/Phenomenology_(psychology)}{subjective experiences}, \href{https://en.wikipedia.org/wiki/Motivation}{motivation}, \href{https://en.wikipedia.org/wiki/Human_brain#Function}{brain functioning}, \& \href{https://en.wikipedia.org/wiki/Personality_psychology}{personality}. Psychologists' interests extend to \href{https://en.wikipedia.org/wiki/Interpersonal_relationship}{impersonal relationships}, \href{https://en.wikipedia.org/wiki/Psychological_resilience}{psychological resilience}, \href{https://en.wikipedia.org/wiki/Family_resilience}{family resilience}, \& other areas within \href{https://en.wikipedia.org/wiki/Social_psychology}{social psychology}. They also consider the unconscious mind. Research psychologists employ \href{https://en.wikipedia.org/wiki/Empirical_research}{empirical mehods} to infer \href{https://en.wikipedia.org/wiki/Causality}{causal} \& \href{https://en.wikipedia.org/wiki/Correlation}{correlational} relationships between psychological \href{https://en.wikipedia.org/wiki/Dependent_and_independent_variables}{variables}. Some, but not all, \href{https://en.wikipedia.org/wiki/Clinical_psychology}{clinical} \& \href{https://en.wikipedia.org/wiki/Counseling_psychology}{counseling} psychologists rely on \href{https://en.wikipedia.org/wiki/Hermeneutics#Psychology_and_cognitive_science}{symbolic interpretation}.

While psychological knowledge is often applied to the assessment \& treatment of mental health problems, it is also directed towards understanding \& solving problems in several spheres of human activity. By many accounts, psychology ultimately aims to benefit society. Many psychologists are involved in some kind of therapeutic role, practicing \href{https://en.wikipedia.org/wiki/Psychotherapy}{psychotherapy} in clinical, counseling, or \href{https://en.wikipedia.org/wiki/School_psychology}{school} settings. Other psychologists conduct scientific research on a wide range of topics related to mental processes \& behavior. Typically the latter group of psychologists work in academic settings (e.g., universities, medical schools, or hospitals). Another group of psychologists is employed in \href{https://en.wikipedia.org/wiki/Industrial_and_organizational_psychology}{industrial \& organizational} settings. Yet others are involved in work on \href{https://en.wikipedia.org/wiki/Developmental_psychology}{human development}, aging, \href{https://en.wikipedia.org/wiki/Sports_psychology}{sports}, health, \href{https://en.wikipedia.org/wiki/Forensic_psychology}{forensic science}, \href{https://en.wikipedia.org/wiki/Educational_psychology}{education}, \& the \href{https://en.wikipedia.org/wiki/Media_psychology}{media}.'' -- \href{https://en.wikipedia.org/wiki/Psychology}{Wikipedia\texttt{/}psychology}

\subsection{Etymology \& Definitions}
``The word \href{https://en.wiktionary.org/wiki/psychology}{\textit{psychology}} derives from the Greek word \href{https://en.wikipedia.org/wiki/Psyche_(psychology)}{psyche}, for spirit or \href{https://en.wikipedia.org/wiki/Soul_(spirit)}{soul}. The latter part of the word ``psychology'' derives from $-\lambda{\rm o}\gamma\acute{\i}\alpha$ \href{https://en.wiktionary.org/wiki/-logia}{-logia}, which refers to ``study'' or ``research''. The \href{https://en.wikipedia.org/wiki/Latin}{Latin} word \textit{psychologia} was 1st used by the \href{https://en.wikipedia.org/wiki/Croatia}{croatian} \href{https://en.wikipedia.org/wiki/Humanism}{humanist} \& \href{https://en.wikipedia.org/wiki/Croatian_latinistic_literature}{Latinist} \href{https://en.wikipedia.org/wiki/Marko_Maruli%C4%87}{Marko Maruli\'c} in his book, \href{https://en.wikipedia.org/wiki/Psichiologia_de_ratione_animae_humanae}{\textit{Psichiologia de ratione animae humanae}} (\textit{Psychology, on the Nature of the Human Soul}) in the late 15th century or early 16th century. The earliest known reference to the word \textit{psychology} in English was by \href{https://en.wikipedia.org/wiki/Steven_Blankaart}{Steven Blankaart} in 1694 in \textit{The Physical Dictionary}. The dictionary refers to ``Anatomy, which treats the Body, \& Psychology, which treats of the Soul.''

In 1890, \href{https://en.wikipedia.org/wiki/William_James}{William James} defined \textit{psychology} as ``the science of mental life, both of its phenomena \& their conditions.'' This definition enjoyed widespread currency for decades. However, this meaning was contested, notably by radical \href{https://en.wikipedia.org/wiki/Behaviorism}{behaviorists} such as \href{https://en.wikipedia.org/wiki/John_B._Watson}{John B. Watson}, who in 1913 asserted that the discipline is a ``natural science,'' the theoretical goal of which ``is the prediction \& control of behavior.'' Since James defined ``psychology'', the term more strongly implicates scientific \href{https://en.wikipedia.org/wiki/Experiment}{experimentation}. \href{https://en.wikipedia.org/wiki/Folk_psychology}{Folk psychology} refers to \href{https://en.wikipedia.org/wiki/Laity}{ordinary people}'s, as contrasted with psychology professionals', understanding of the mental states \& behaviors of people.'' -- \href{https://en.wikipedia.org/wiki/Psychology#Etymology_and_definitions}{Wikipedia\texttt{/}psychology\texttt{/}etymology \& definitions}

\subsection{History}
``Main article: \href{https://en.wikipedia.org/wiki/History_of_psychology}{Wikipedia\texttt{/}history of psychology}. The ancient civilizations of Egypt, Greece, China, India, \& Persia all engaged in the philosophical study of psychology. In Ancient Egypt the \href{https://en.wikipedia.org/wiki/Ebers_papyrus}{Ebers Papyrus} mentioned \href{https://en.wikipedia.org/wiki/Clinical_depression}{depression} \& thought disorders. Historians note that Greek philosophers, including \href{https://en.wikipedia.org/wiki/Thales}{Thales}, \href{https://en.wikipedia.org/wiki/Plato}{Plato}, \& \href{https://en.wikipedia.org/wiki/Aristotle}{Aristotle} (especially in his \href{https://en.wikipedia.org/wiki/On_the_Soul}{De Anima} treatise), addressed the \fbox{workings of the mind}. As early as the 4th century BC, the Greek physician \href{https://en.wikipedia.org/wiki/Hippocrates}{Hippocrates} theorized that \href{https://en.wikipedia.org/wiki/Mental_disorder}{mental disorders} had physical rather than supernatural causes. In 387 BCE, Plato suggested that the brain is where mental processes take place, \& in 335 BCE Aristotle suggested that it was the heart.

In China, psychological understanding grew from the philosophical works of \href{https://en.wikipedia.org/wiki/Laozi}{Laozi} \& \href{https://en.wikipedia.org/wiki/Confucius}{Confucius}, \& later from the doctrines of \href{https://en.wikipedia.org/wiki/Buddhism}{Buddhism}. This body of knowledge involves insights drawn from introspection \& observation, as well as techniques for focused thinking \& acting. It frames the universe in term of a division of physical reality \& mental reality as well as the interaction between the physical \& the mental. Chinese philosophy also emphasized purifying the mind in order to increase virtue \& power. An ancient text known as \href{https://en.wikipedia.org/wiki/Huangdi_Neijing}{The Yellow Emperor's Classic of Internal Medicine} identifies the brain as the nexus of wisdom \& sensation, includes theories of personality based on \href{https://en.wikipedia.org/wiki/Yin_and_yang}{yin-yang} balance, \& analyzes mental disorder in terms of physiological \& social disequilibria. Chinese scholarship that focused on the brain advanced during the \href{https://en.wikipedia.org/wiki/Qing_Dynasty}{Qing Dynasty} with the work of Western-educated Fang Yizhi (1611--1671), \href{https://en.wikipedia.org/wiki/Liu_Zhi_(scholar)}{Liu Zhi} (1660--1730), \& Wang Qingren (1768--1831). Wang Qingren emphasized the importance of the brain as the center of the nervous system, linked mental disorder with brain diseases, investigated the causes of dreams \& \href{https://en.wikipedia.org/wiki/Insomnia}{insomnia}, \& advanced a theory of \href{https://en.wikipedia.org/wiki/Lateralization_of_brain_function}{hemispheric lateralization} in brain function.

Influenced by \href{https://en.wikipedia.org/wiki/Hinduism}{Hinduism}, \href{https://en.wikipedia.org/wiki/Indian_philosophy}{Indian philosophy} explored distinctions in types of awareness. A central idea of the \href{https://en.wikipedia.org/wiki/Upanishads}{\textit{Upanishads}} \& other \href{https://en.wikipedia.org/wiki/Vedic_period}{Vedic} texts that formed the foundations of Hinduism was the distinction between a person's transient mundane self \& their \href{https://en.wikipedia.org/wiki/%C4%80tman_(Hinduism)}{eternal, unchanging soul}. Divergent Hindu doctrines \& \href{https://en.wikipedia.org/wiki/Buddhism}{Buddhism} have challenged this hierarchy of selves, but have all emphasized the importance of reaching higher awareness. \href{https://en.wikipedia.org/wiki/Yoga}{Yoga} encompasses a range of techniques used in pursuit of this goal. \href{https://en.wikipedia.org/wiki/Theosophy}{Theosophy}, a religion established by \href{https://en.wikipedia.org/wiki/Russian_Americans}{Russian--American} philosopher \href{https://en.wikipedia.org/wiki/Helena_Blavatsky}{Helena Blavatsky}, drew inspiration from these doctrines during her time in \href{https://en.wikipedia.org/wiki/British_Raj}{British India}.

Psychology was of interest to \href{https://en.wikipedia.org/wiki/Age_of_Enlightenment}{Enlightenment thinkers} in Europe. In Germany, \href{https://en.wikipedia.org/wiki/Gottfried_Wilhelm_Leibniz}{Gottfried Wilhelm Leibniz} (1646--1716) applied his principles of calculus to the mind, arguing that mental activity took place on an indivisible continuum. He suggested that the difference between conscious \& unconscious awareness is only a matter of degree. \href{https://en.wikipedia.org/wiki/Christian_Wolff_(philosopher)}{Christian Wolff} identified psychology as its own science, writing \textit{Psychologia Empirica} in 1732 \& \textit{Psychologia Rationalis} in 1734. \href{https://en.wikipedia.org/wiki/Immanuel_Kant}{Immanuel Kant} advanced the idea of \href{https://en.wikipedia.org/wiki/Anthropology}{anthropology} as a discipline, with psychology an important subdivision. Kant, however, explicitly rejected the idea of an \href{https://en.wikipedia.org/wiki/Experimental_psychology}{experimental psychology}, writing that ``the empirical doctrine of the soul can also never approach chemistry even as a systematic art of analysis or experimental doctrine, for in it the manifold of inner observation can be separated only by mere division in thought, \& cannot then be held separate \& recombined at will (but still less does another thinking subject suffer himself to be experimented upon to suit our purpose), \& even observation by itself already changes \& displaces the state of the observed object.'' In 1783, Ferdinand Ueberwasser (1752--1812) designated himself \textit{Professor of Empirical Psychology \& Logic} \& gave lectures on scientific psychology, though these developments were soon overshadowed by the \href{https://en.wikipedia.org/wiki/Napoleonic_Wars}{Napoleonic Wars}. At the end of the Napoleonic era, Prussian authorities discontinued the Old University of M\"unster. Having consulted philosophers \href{https://en.wikipedia.org/wiki/Georg_Friedrich_Wilhelm_Hegel}{Hegel} \& \href{https://en.wikipedia.org/wiki/Johann_Friedrich_Herbart}{Herbart}, however, in 1825 \href{https://en.wikipedia.org/wiki/Prussia}{the Prussian state} established psychology as a mandatory discipline in its rapidly expanding \& highly influential \href{https://en.wikipedia.org/wiki/Prussian_education_system}{educational system}. However, this discipline did not yet embrace experimentation. In England, early psychology involved \href{https://en.wikipedia.org/wiki/Phrenology}{phrenology} \& the response to social problems including alcoholism, violence, \& the country's crowded ``lunatic'' asylums.'' -- \href{https://en.wikipedia.org/wiki/Psychology#History}{Wikipedia\texttt{/}psychology\texttt{/}history}

\subsubsection{Beginning of experimental psychology}
\textsf{Fig. \href{https://en.wikipedia.org/wiki/Wilhelm_Wundt}{Wilhelm Wundt} (seated) with colleagues in his psychological laboratory, the 1st of its kind.}

``Philosopher \href{https://en.wikipedia.org/wiki/John_Stuart_Mill}{John Stuart Mill} believed that the human mind was open to scientific investigation, even if the science is in some ways inexact. Mill proposed a ``mental \href{https://en.wikipedia.org/wiki/Chemistry}{chemistry}'' in which elementary thoughts could combine into ideas of greater complexity. \href{https://en.wikipedia.org/wiki/Gustav_Fechner}{Gustave Fechner} began conducting \href{https://en.wikipedia.org/wiki/Psychophysics}{psychophysics} research in \href{https://en.wikipedia.org/wiki/Leipzig}{Leipzig} in the 1830s. He articulated the principle that human perception of a stimulus varies \href{https://en.wikipedia.org/wiki/Logarithmically}{logarithmically} according to its intensity. The principle became known as the \href{https://en.wikipedia.org/wiki/Weber%E2%80%93Fechner_law}{Weber--Fechner law}. Fechner's 1860 \textit{Elements of Psychophysics} challenged Kant's negative view with regard to conducting quantitative research on the mind. Fechner's achievement was to show that ``mental processes could not only be given numerical magnitudes, but also that these could be measured by experimental methods.'' In Heidelberg, \href{https://en.wikipedia.org/wiki/Hermann_von_Helmholtz}{Hermann von Helmholtz} conducted parallel research on sensory perception, \& trained physiologist \href{https://en.wikipedia.org/wiki/Wilhelm_Wundt}{Wilhelm Wundt}. Wundt, in turn, came to Leipzig University, where he established th psychological \href{https://en.wikipedia.org/wiki/Laboratory}{laboratory} that brought experimental psychology to the world. Wundt focused on breaking down mental processes into the most basic components, motivated in part by an analogy to recent advances in chemistry, \& its successful investigation of the elements \& structure of materials. \href{https://en.wikipedia.org/wiki/Paul_Flechsig}{Paul Flechsig} \& \href{https://en.wikipedia.org/wiki/Emil_Kraepelin}{Emil Kraepelin} soon created another influential laboratory at Leipzig, a psychology-related lab, that focused more on experimental psychiatry.

The German psychologist \href{https://en.wikipedia.org/wiki/Hermann_Ebbinghaus}{Hermann Ebbinghaus}, a researcher at the \href{https://en.wikipedia.org/wiki/University_of_Berlin}{University of berlin}, was another 19th-century contributor to the field. He pioneered the experimental study of memory \& developed quantitative models of learning \& forgetting. In the early 20th century, \href{https://en.wikipedia.org/wiki/Wolfgang_Kohler}{Wolfgang Kohler}, \href{https://en.wikipedia.org/wiki/Max_Wertheimer}{Max Wertheimer}, \& \href{https://en.wikipedia.org/wiki/Kurt_Koffka}{Kurt Koffka} co-founded the school of \href{https://en.wikipedia.org/wiki/Gestalt_psychology}{Gestalt psychology} (not to be confused with the \href{https://en.wikipedia.org/wiki/Gestalt_therapy}{Gestalt therapy} of \href{https://en.wikipedia.org/wiki/Fritz_Perls}{Fritz Perls}). The approach of Gestalt psychology is based upon the idea that individuals experience things as unified wholes. Rather than \href{https://en.wikipedia.org/wiki/Reductionism}{reducing} thoughts \& behavior into smaller component elements, as in structuralism, the Gestaltists maintained that whole of experience is important, \& differs from the sum of its parts.

Psychologists in Germany, Denmark, Austria, England, \& the United States soon followed Wundt in setting up laboratories. \href{https://en.wikipedia.org/wiki/G._Stanley_Hall}{G. Stanley Hall}, an American who studied with Wundt, founded a psychology lab that became internationally influential. The lab was located at \href{https://en.wikipedia.org/wiki/Johns_Hopkins_University}{Johns Hopkins University}. Hall, in turn, trained \href{https://en.wikipedia.org/wiki/Y%C5%ABjir%C5%8D_Motora}{Yujiro Motora}, who brought experimental psychology, emphasizing psychophysics, to the \href{https://en.wikipedia.org/wiki/Imperial_University_of_Tokyo}{Imperial University of Tokyo}. Wundt's assistant, \href{https://en.wikipedia.org/wiki/Hugo_M%C3%BCnsterberg}{Hugo M\"unsterberg}, taught psychology at Harvard to students such as \href{https://en.wikipedia.org/wiki/Narendra_Nath_Sen_Gupta}{Narendra Nath Sen Gupta} -- who, in 1905, founded a psychology department \& laboratory at the \href{https://en.wikipedia.org/wiki/University_of_Calcutta}{University of Calcutta}. Wundt's students \href{https://en.wikipedia.org/wiki/Walter_Dill_Scott}{Walter Dill Scott}, \href{https://en.wikipedia.org/wiki/Lightner_Witmer}{Lightner Witmer}, \& \href{https://en.wikipedia.org/wiki/James_McKeen_Cattell}{James McKeen Cattell} worked on developing tests of mental ability. Cattell, who also studied with \href{https://en.wikipedia.org/wiki/Eugenics}{eugenicist} \href{https://en.wikipedia.org/wiki/Francis_Galton}{Francis Galton}, went on to found the \href{https://en.wikipedia.org/wiki/Psychological_Corporation}{Psychological Corporation}. Witmer focused on the mental testing of children; Scott, on employee selection.

Another student of Wundt, the Englishman \href{https://en.wikipedia.org/wiki/Edward_Titchener}{Edward Titchener}, created the psychology program at \href{https://en.wikipedia.org/wiki/Cornell_University}{Cornell University} \& advanced ``\href{https://en.wikipedia.org/wiki/Structuralism_(psychology)}{structuralist}'' psychology. The idea behind structuralism was to analyze \& classify different aspects of the mind, primarily through the method of \href{https://en.wikipedia.org/wiki/Introspection}{introspection}. William James, \href{https://en.wikipedia.org/wiki/John_Dewey}{John Dewey}, \& \href{https://en.wikipedia.org/wiki/Harvey_Carr}{Harvey Carr} advanced the idea of \href{https://en.wikipedia.org/wiki/Functional_psychology}{functionalism}, an expansive approach to psychology that underlined the Darwinian idea of a behavior's usefulness to the individual. In 1890, James wrote an influential book, \href{https://en.wikipedia.org/wiki/The_Principles_of_Psychology}{The Principles of Psychology}, which expanded on the structuralism. He memorably described ``\href{https://en.wikipedia.org/wiki/Stream_of_consciousness_(psychology)}{stream of consciousness}.'' James's ideas interested many American students in the emerging discipline. Dewey integrated psychology with societal concerns, most notably by promoting \href{https://en.wikipedia.org/wiki/Progressive_education}{progressive education}, inculcating moral values in children, \& assimilating immigrants.

A different strain of experimentalism, with a greater connection to physiology, emerged in South America, under the leadership of Horacio G. Pi\~nero at the \href{https://en.wikipedia.org/wiki/University_of_Buenos_Aires}{University of Buenos Aires}. In Russia, too, researchers placed greater emphasis on the biological basis for psychology, beginning with \href{https://en.wikipedia.org/wiki/Ivan_Sechenov}{Ivan Sechenov}'s 1873 essay, ``Who Is to Develop Psychology \& How?'' Sechenov advanced the idea of brain \href{https://en.wikipedia.org/wiki/Reflexes}{reflexes} \& aggressively promoted a \href{https://en.wikipedia.org/wiki/Determinism}{deterministic} view of human behavior. The Russian-Soviet \href{https://en.wikipedia.org/wiki/Physiologist}{physiologist} \href{https://en.wikipedia.org/wiki/Ivan_Pavlov}{Ivan Pavlov} discovered in dogs a learning process that was later termed ``\href{https://en.wikipedia.org/wiki/Classical_conditioning}{classical conditioning}'' \& applied the process to human beings.

\textsf{Fig. 1 of the dogs used in Pavlov's experiment with a surgically implanted \href{https://en.wikipedia.org/wiki/Cannula}{cannula} to measure \href{https://en.wikipedia.org/wiki/Saliva}{salivation}, \href{https://en.wikipedia.org/wiki/Taxidermy}{preserved} in the Pavlov Museum in \href{https://en.wikipedia.org/wiki/Ryazan}{Ryazan}, Russia.}'' -- \href{https://en.wikipedia.org/wiki/Psychology#Beginning_of_experimental_psychology}{Wikipedia\texttt{/}history\texttt{/}beginning of experimental psychology}

\subsubsection{Consolidation \& funding}
``1 of the earliest psychology societies was \textit{La Soci\'et\'e de Psychologie Physiologique} in France, which lasted from 1885 to 1893. The 1st meeting of the International Congress of Psychology sponsored by the \href{https://en.wikipedia.org/wiki/International_Union_of_Psychological_Science}{International Union of Psychological Science} took place in Paris, in Aug 1889, amidst \href{https://en.wikipedia.org/wiki/Exposition_Universelle_(1889)}{the World
s Fair} celebrating the centennial of the French Revolution. William James was 1 of 3 Americans among the 400 attendees. The \href{https://en.wikipedia.org/wiki/American_Psychological_Association}{American Psychological Association} (APA) was founded soon after, in 1892. The International Congress continued to be held at different locations in Europe \& with wide international participation. The 6th Congress, held in Geneva in 1909, included presentations in Russian, Chinese, \& Japanese, as well as \href{https://en.wikipedia.org/wiki/Esperanto}{Esperanto}. After a hiatus for World War I, the 7th Congress met in Oxford, with substantially greater participation from the war-victorious Anglo-Americans. In 1929, the Congress took place at Yale University in New Haven, Connecticut, attended by hundreds of members of the APA. Tokyo Imperial University led the way in bringing new psychology to the East. New ideas about psychology diffused from Japan into China.

American psychology gained status upon the U.S.'s entry into World War I. A standing committee headed by \href{https://en.wikipedia.org/wiki/Robert_Yerkes}{Robert Yerkes} administered mental tests (``\href{https://en.wikipedia.org/wiki/Army_Alpha}{Army Alpha}'' \& ``\href{https://en.wikipedia.org/wiki/Army_Beta}{Army Beta}'') to almost 1.8 million soldiers. Subsequently, the \href{https://en.wikipedia.org/wiki/Rockefeller_family}{Rockefeller family}, via the \href{https://en.wikipedia.org/wiki/Social_Science_Research_Council}{Social Science Research Council}, began to provide funding for behavioral research. Rockefeller charities funded the National Committee on Mental Hygiene, which disseminated the concept of mental illness \& lobbied for applying ideas from psychology to child rearing. Through the Bureau of Social Hygiene \& later funding of \href{https://en.wikipedia.org/wiki/Alfred_Kinsey}{Alfred Kinsey}, Rockefeller foundations helped established research on sexuality in the U.S. Under the influence of the Carnegie-funded \href{https://en.wikipedia.org/wiki/Eugenics_Record_Office}{Eugenics Record Office}, the Draper-funded \href{https://en.wikipedia.org/wiki/Pioneer_Fund}{Pioneer Fund}, \& other institutions, the \href{https://en.wikipedia.org/wiki/Eugenics_in_the_United_States}{eugenics movement} also influenced American psychology. In the 1910s \& 1920s, eugenics became a standard topic in psychology classes. In contrast to the US, in the UK psychology was met with antagonism by the scientific \& medical establishments, \& up until 1939, there were only 6 psychology chairs in universities in England.

During World War II \& the Cold War, the U.S. military \& intelligence agencies established themselves as leading funders of psychology by way of the armed forces \& in the new \href{https://en.wikipedia.org/wiki/Office_of_Strategic_Services}{Office of Strategic Services} intelligence agency. University of Michigan psychologist Dorwin Cartwright reported that university researchers began large-scale propaganda research in 1939--1941. He observed that ``the last few months of the war saw a social psychologist become chiefly responsible for determining the week-by-week-propaganda policy for the United States Government.'' Cartwright also wrote that psychologists had significant roles in managing the domestic economy. The Army rolled out its new \href{https://en.wikipedia.org/wiki/Army_General_Classification_Test}{General Classification Test} to assess the ability of millions of soldiers. The Army also engaged in large-scaled psychological research of \href{https://en.wikipedia.org/wiki/Samuel_A._Stouffer#Studies_in_Social_Psychology_in_World_War_II:_The_American_Soldier}{troop morale \& mental health}. In the 1950s, the \href{https://en.wikipedia.org/wiki/Rockefeller_Foundation}{Rockefeller Foundation} \& \href{https://en.wikipedia.org/wiki/Ford_Foundation}{Ford Foundation} collaborated with the \href{https://en.wikipedia.org/wiki/Central_Intelligence_Agency}{Central Intelligence Agency} (CIA) to fund research on \href{https://en.wikipedia.org/wiki/Psychological_warfare}{psychological warfare}. In 1965, public controversy called attention to the Army's \href{https://en.wikipedia.org/wiki/Project_Camelot}{Project Camelot}, the ``Manhattan Project'' of social science, an effort which enlisted psychologists \& anthropologists to analyze the plans \& policies of foreign countries for strategic purposes.

In Germany after World War I, psychology held institutional power through the military, which was subsequently expanded along with the rest of the military during \href{https://en.wikipedia.org/wiki/Nazi_Germany}{Nazi Germany}. Under the direction of \href{https://en.wikipedia.org/wiki/Hermann_G%C3%B6ring}{Hermann G\"oring}'s cousin \href{https://en.wikipedia.org/wiki/Matthias_G%C3%B6ring}{Matthias G\"oring}, the \href{https://en.wikipedia.org/wiki/Berlin_Psychoanalytic_Institute}{Berlin Psychoanalytic Institute} was renamed the G\"oring Institute. \href{https://en.wikipedia.org/wiki/Freudian_psychoanalysis}{Freudian psychoanalysts} were expelled \& persecuted under the anti-Jewish policies of the \href{https://en.wikipedia.org/wiki/Nazi_Party}{Nazi Party}, \& all psychologists had to distance themselves from \href{https://en.wikipedia.org/wiki/Sigmund_Freud}{Freud} \& \href{https://en.wikipedia.org/wiki/Alfred_Adler}{Adler}, founders of \href{https://en.wikipedia.org/wiki/Psychoanalysis}{psychoanalysis} who were also Jewish. The G\"oring Institute was well-financed throughout the war with a mandate to create a ``New German Psychotherapy.'' This psychotherapy aimed to align suitable Germans with the overall goals of the Reich. As described by 1 physician, ``Despite the importance of analysis, spiritual guidance \& the active cooperation of the patient represent the best way to overcome individual mental problems \& to subordinate them to the requirements of the \href{https://en.wikipedia.org/wiki/Volk}{Volk} \& the \href{https://en.wikipedia.org/wiki/Gemeinschaft_and_Gesellschaft}{Gemeinschaft}.'' Psychologists were to provide \textit{Seelenf\"uhrung} [lit., soul guidance], the leadership of the mind, to integrate people into the new vision of a German community. \href{https://en.wikipedia.org/wiki/Harald_Schultz-Hencke}{Harald Schultz-Hencke} melded psychology with the Nazi theory of biology \& racial origins, criticizing psychoanalysis as a study of the weak \& deformed. \href{https://en.wikipedia.org/wiki/Johannes_Heinrich_Schultz}{Johannes Heinrich Schultz}, a German psychologist recognized for developing the technique of \href{https://en.wikipedia.org/wiki/Autogenic_training}{autogenic training}, prominently advocated sterilization \& euthanasia of men considered genetically undesirable, \& devised techniques for facilitating this process.

After the war, new institutions were created although some psychologists, because of their Nazi affiliation, were discredited. \href{https://en.wikipedia.org/wiki/Alexander_Mitscherlich_(psychologist)}{Alexander Mitscherlich} founded a prominent applied psychoanalysis journal called \textit{Psyche}. With funding from the Rockefeller Foundation, Mitscherlich established the 1st clinical psychosomatic medicine division at Heidelberg University. In 1970, psychology was integrated into the required studies of medical students.

After the \href{https://en.wikipedia.org/wiki/Russian_Revolution}{Russian Revolution}, the \href{https://en.wikipedia.org/wiki/Bolsheviks}{Bolsheviks} promoted psychology as a way to engineer the ``New Man'' of socialism. Consequently, university psychology departments trained large numbers of students in psychology. At the completion of training, positions were made available for those students at schools, workplaces, cultural institutions, \& in the military. The Russian state emphasized \href{https://en.wikipedia.org/wiki/Pedology_(children_study)}{pedology} \& the study of child development. \href{https://en.wikipedia.org/wiki/Lev_Vygotsky}{Lev Vygotsky} became prominent in the field of child development. The Bolsheviks also promoted \href{https://en.wikipedia.org/wiki/Free_love}{free love} \& embraced the doctrine of psychoanalysis as an antidote to sexual repression. Although pedology \& intelligence testing fell out of favor in 1936, psychology maintained its privileged position as an instrument of the Soviet Union. \href{https://en.wikipedia.org/wiki/Stalinist_purges}{Stalinist purges} took a heavy toll \& instilled a climate of fear in the profession, as elsewhere in Soviet society. Following World War II, Jewish psychologists past \& present, including \href{https://en.wikipedia.org/wiki/Lev_Vygotsky}{Lev Vygotsky}, \href{https://en.wikipedia.org/wiki/Alexander_Luria}{A. R. Luria}, \& Aron Zalkind, were denounced; Ivan Pavlov (posthumously) \& Stalin himself were celebrated as heroes of Soviet psychology. Soviet academics experienced a degree of liberalization during the \href{https://en.wikipedia.org/wiki/Khrushchev_Thaw}{Khrushchev Thaw}. The topics of cybernetics, linguistics, \& genetics became acceptable again. The new field of \href{https://en.wikipedia.org/wiki/Engineering_psychology}{engineering psychology} emerged. The field involved the study of the mental aspects of complex jobs (such as pilot \& cosmonaut). Interdisciplinary studies became popular \& scholars such as \href{https://en.wikipedia.org/wiki/Georgy_Shchedrovitsky}{Georgy Shchedrovitsky} developed systems theory approaches to human behavior.

20th-century Chinese psychology originally modeled itself on U.S. psychology, with translations from American authors like William James, the establishment of university psychology departments \& journals, \& the establishment of groups including the Chinese Association of Psychological Testin (1930) \& the \href{https://en.wikipedia.org/wiki/Chinese_Psychological_Society}{Chinese Psychological Society} (1937). Chinese psychologists were encouraged to focus on education \& language learning. Chinese psychologists were drawn to the idea that \fbox{education would enable modernization}. John Dewey, who lectured to Chinese audiences between 1919 \& 1921, had a significant influence on psychology in China. Chancellor \href{https://en.wikipedia.org/wiki/Cai_Yuanpei}{T'sai Yuan-p'ei} introduced him at \href{https://en.wikipedia.org/wiki/Peking_University}{Peking University} as a greater thinker than Confucius. \href{https://en.wikipedia.org/wiki/Zing-Yang_Kuo}{Kuo Zing-yang} who received a PhD at the University of California, Berkeley, became President of \href{https://en.wikipedia.org/wiki/Zhejiang_University}{Zhejiang University} \& popularized \href{https://en.wikipedia.org/wiki/Behaviorism}{behaviorism}. After the \href{https://en.wikipedia.org/wiki/Chinese_Communist_Party}{Chinese Communist Party} gained control of the country, the Stalinist Soviet Union became the major influence, with \href{https://en.wikipedia.org/wiki/Marxism%E2%80%93Leninism}{Marxism--Leninism} the leading social doctrine \& Pavlovian conditioning the approved means of behavior change. Chinese psychologists elaborated on Lenin's model of a ``reflective'' consciousness, envisioning an ``active consciousness'' (\href{https://en.wikipedia.org/wiki/Pinyin}{pinyin}: \textit{tzu-chueh neng-tung-li}) able to transcend material conditions through hard work \& ideological struggle. They developed a concept of ``recognition'' (pinyin: \textit{jen-shih}) which referred to the interface between individual perceptions \& the socially accepted worldview; failure to correspond with party doctrine was ``incorrect recognition.'' Psychology education was centralized under the \href{https://en.wikipedia.org/wiki/Chinese_Academy_of_Sciences}{Chinese Academy of Sciences}, supervised by the \href{https://en.wikipedia.org/wiki/State_Council_of_the_People%27s_Republic_of_China}{State Council}. In 1951, the academy created a Psychology Research Office, which in 1956 became the Institute of Psychology. Because most leading psychologists were educated in the United States, the 1st concern of the academy was the re-education of these psychologists in the Soviet doctrines. Child psychology \& pedagogy for the purpose of a nationally cohesive education remained a central goal of the discipline.'' -- \href{https://en.wikipedia.org/wiki/Psychology#Consolidation_and_funding}{Wikipedia\texttt{/}history\texttt{/}consolidation \& funding}

\subsection{Disciplinary Organization}

\subsubsection{Institutions}
``See also: \href{https://en.wikipedia.org/wiki/List_of_psychology_organizations}{Wikipedia\texttt{/}list of psychology organizations}. In 1920, \href{https://en.wikipedia.org/wiki/%C3%89douard_Clapar%C3%A8de}{\'Edouard Clapar\`ede} \& \href{https://en.wikipedia.org/wiki/Pierre_Bovet}{Pierre Bovet} created a new applied psychology organization called the International Congress of Psychotechnics Applied to Vocational Guidance, later called the International Congress of Psychotechnics \& then the \href{https://en.wikipedia.org/wiki/International_Association_of_Applied_Psychology}{International Association of Applied Psychology}. The IAAP is considered the oldest international psychology association. Today, at least 65 international groups deal with specialized aspects of psychology. In response to male predominance in the field, female psychologists in the U.S. formed the National Council of Women Psychologists in 1941. This organization became the International Council of Women Psychologists after World War II \& the International Council of Psychologists in 1959. Several associations including the \href{https://en.wikipedia.org/wiki/Association_of_Black_Psychologists}{Association of Black Psychologists} \& the Asian American Psychological Association have arisen to promote the inclusion of non-European racial groups in the profession.

The \href{https://en.wikipedia.org/wiki/International_Union_of_Psychological_Science}{International Union of Psychological Science} (IUPsyS) is the world federation of national psychological societies. The IUPsyS was founded in 1951 under the auspices of the \href{https://en.wikipedia.org/wiki/UNESCO}{United Nations Educational, Cultural \& Scientific Organization (UNESCO)}. Psychology departments have since proliferated around the world, based primarily on the Euro-American model. Since 1966, the Union has published the \textit{International Journal of Psychology}. IAAP \& IUPsyS agreed in 1976 each to hold a congress every 4 years, on a staggered basis.

IUPsyS recognizes 66 national psychology associations \& at least 15 others exist. The American Psychological Association is the oldest \& largest. Its membership has increased from 5,000 in 1945 to 100,000 in the present day. The APA includes \href{https://en.wikipedia.org/wiki/Divisions_of_the_American_Psychological_Association}{54 divisions}, which since 1960 have steadily proliferated to include more specialties. Some of these divisions, such as the \href{https://en.wikipedia.org/wiki/Society_for_the_Psychological_Study_of_Social_Issues}{Society for the Psychological Study of Social Issues} \& the \href{https://en.wikipedia.org/wiki/American_Psychology%E2%80%93Law_Society}{American Psychology--Law Society}, began as autonomous groups.

The \href{https://en.wikipedia.org/wiki/Interamerican_Psychological_Society}{Interamerican Psychological Society}, founded in 1951, aspires to promote psychology across the Western Hemisphere. It holds the Interamerican Congress of Psychology \& has had 1,000 members in year 2000. The European Federation of Professional Psychology Associations, founded in 1981, represents 30 national associations with a total of 100,000 individual members. At least 30 other international organizations represent psychologists in different regions.

In some places, governments legally regulate who can provide psychological services or represent themselves as a ``psychologist.'' The APA defines a psychologist as someone with a doctoral degree in psychology.'' -- \href{https://en.wikipedia.org/wiki/Psychology#Institutions}{Wikipedia\texttt{/}psychology\texttt{/}disciplinary organization\texttt{/}institutions}

\subsubsection{Boundaries}
``Early practitioners of experimental psychology distinguished themselves from \href{https://en.wikipedia.org/wiki/Parapsychology}{parapsychology}, which in the late 19th century enjoyed popularity (including the interest of scholars such as William James). Some people considered parapsychology to be part of ``psychology.'' Parapsychology, hypnotism, \& \href{https://en.wikipedia.org/wiki/Psychic}{psychism} were major topics at the early International Congresses. But students of these fields were eventually ostracized, \& more or less banished from the Congress in 1900--1905. Parapsychology persisted for a time at Imperial University in Japan, with publications such as \textit{Clairvoyance \& Thoughtography} by Tomokichi Fukurai, but it was mostly shunned by 1913.

As a discipline, psychology has long sought to fend off accusations that it is a ``soft'' science. Philosopher of science \href{https://en.wikipedia.org/wiki/Thomas_Kuhn}{Thomas Kuhn}'s 1962 critique implied psychology overall was in a pre-paradigm state, lacking agreement on the type of overarching theory found in mature sciences such as chemistry \& physics. Because some areas of psychology rely on research methods such as surveys \& questionnaires, critics asserted that psychology is not an objective science. Skeptics have suggested that personality, thinking, \& emotion cannot be directly measured \& are often inferred from subjective self-reports, which may be problematic. Experimental psychologists have devised a variety of ways to indirectly measure these elusive phenomenological entities.

Divisions still exist within the field, with some psychologists more oriented towards the unique experiences of individual humans, which cannot be understood only as data points within a larger population. Critics inside \& outside the field have argued that mainstream psychology has become increasingly dominated by a ``cult of empiricism,'' which limits the scope of research because investigators restrict themselves to methods derived from the physical sciences. Feminist critiques have argued that claims to scientific objectivity obscure the values \& agenda of (historically) mostly male researchers. Jean Grimshaw, e.g., argues that mainstream psychological research has advanced a \href{https://en.wikipedia.org/wiki/Patriarchal}{patriarchal} agenda through its efforts to control behavior.'' -- \href{https://en.wikipedia.org/wiki/Psychology#Boundaries}{Wikipedia\texttt{/}psychology\texttt{/}disciplinary organization\texttt{/}boundaries}

\subsection{Major Schools of Thought}

\subsubsection{Biological}
\textsf{Fig. False-color representations of \href{https://en.wikipedia.org/wiki/White_matter}{cerebral fiber} pathways affected, per Van Horn et al.}

``Main article: \href{https://en.wikipedia.org/wiki/Cognitive_neuroscience}{Wikipedia\texttt{/}cognitive neuroscience}. Psychologists generally consider biology the substrate of thought \& feeling, \& therefore an important area of study. Behavioral neuroscience, also known as \textit{biological psychology}, involves the application of biological principles to the study of physiological \& genetic mechanisms underlying behavior in humans \& other animals. The allied field of \href{https://en.wikipedia.org/wiki/Comparative_psychology}{comparative psychology} is the scientific study of the behavior \& mental processes of non-human animals. A leading question in behavioral neuroscience has been whether \& how mental functions are \href{https://en.wikipedia.org/wiki/Functional_specialization_(brain)}{localized in the brain}. From \href{https://en.wikipedia.org/wiki/Phineas_Gage}{Phineas Gage} to \href{https://en.wikipedia.org/wiki/Henry_Molaison}{H.M.} \& \href{https://en.wikipedia.org/wiki/Clive_Wearing}{Clive Wearing}, individual people with mental deficits traceable to physical brain damage have inspired new discoveries in this area. Modern behavioral neuroscience could be said to originate in the 1870s, when in France \href{https://en.wikipedia.org/wiki/Paul_Broca}{Paul Broca} traced production of speech to the left frontal gyrus, thereby also demonstrating hemispheric lateralization of brain function. Soon after, \href{https://en.wikipedia.org/wiki/Carl_Wernicke}{Carl Wernicke} identified a related area necessary for the understanding of speech.

The contemporary field of \href{https://en.wikipedia.org/wiki/Behavioral_neuroscience}{behavioral neuroscience} focuses on the physical basis of behavior. Behavioral neuroscientists use animal models, often relying on rats, to study the neural, genetic, \& cellular mechanisms that underlie behaviors involved in learning, memory, \& fear responses. \href{https://en.wikipedia.org/wiki/Cognitive_neuroscience}{Cognitive neuroscientists}, by using neural imaging tools, investigate the neural correlates of psychological processes in humans. \href{https://en.wikipedia.org/wiki/Neuropsychology}{Neuropsychologists} conduct psychological assessments to determine how an individual's behavior \& cognition are related to the brain. The \href{https://en.wikipedia.org/wiki/Biopsychosocial_model}{biopsychosocial model} is a cross-disciplinary, holistic model that concerns the ways in which interrelationships of biological, psychological, \& socio-environmental factors affect health \& behavior.

\href{https://en.wikipedia.org/wiki/Evolutionary_psychology}{Evolutionary psychology} approaches thought \& behavior from a modern \href{https://en.wikipedia.org/wiki/Evolution}{evolutionary} perspective. This perspective suggests that psychological adaptations evolved to solve recurrent problems in human ancestral environments. Evolutionary psychologists attempt to find out how human psychological traits are evolved adaptations, the results of \href{https://en.wikipedia.org/wiki/Natural_selection}{natural selection} or \href{https://en.wikipedia.org/wiki/Sexual_selection}{sexual selection} over the course of human evolution.

The history of the biological foundations of psychology includes evidence of racism. The idea of white supremacy \& indeed the modern concept of race itself arose during the process of world conquest by Europeans. \href{https://en.wikipedia.org/wiki/Carl_von_Linnaeus}{Carl von Linnaeus}'s 4-fold classification of humans classifies Europeans as intelligent \& severe, Americans as contented \& free, Asians as ritualistic, \& Africans as lazy \& capricious. Race was also used to justify the construction of socially specific mental disorders such as \href{https://en.wikipedia.org/wiki/Drapetomania}{drapetomania} \& \href{https://en.wikipedia.org/wiki/Dysaesthesia_aethiopica}{\textit{dysaesthesia aethiopica}} -- the behavior of uncooperative African slaves. After the creation of experimental psychology, ``ethnical psychology'' emerged as a subdiscipline, based on the assumption that studying primitive races would provide an important link between animal behavior \& the psychology of more evolved humans.'' -- \href{https://en.wikipedia.org/wiki/Psychology#Biological}{Wikipedia\texttt{/}major schools of thought\texttt{/}biological}

\subsubsection{Behaviorist}
\textsf{Fig. Skinner's \href{https://en.wikipedia.org/wiki/Teaching_machine}{teaching machine}, a mechanical invention to automate the task of \href{https://en.wikipedia.org/wiki/Programmed_instruction}{programmed instruction}.}

``Main article: \href{https://en.wikipedia.org/wiki/Behaviorism}{Wikipedia\texttt{/}behaviorism}, \href{https://en.wikipedia.org/wiki/Psychological_behaviorism}{Psychological behaviorism}, \& \href{https://en.wikipedia.org/wiki/Radical_behaviorism}{Radical behaviorism}. A tenet of behavior research is that a large part of both human \& lower-animal behavior is learned. A principle associated with behavioral research is that the mechanisms involved in learning apply to humans \& non-human animals. Behavioral researchers have developed a treatment known as \href{https://en.wikipedia.org/wiki/Behavior_modification}{behavior modification}, which is used to help individuals replace undesirable behaviors with desirable ones.

Early behavioral researchers studied stimulus-response pairings, now known as \href{https://en.wikipedia.org/wiki/Classical_conditioning}{classical conditioning}. They demonstrated that when a biologically potent stimulus (e.g., food that elicits salivation) is paired with a previously neutral stimulus (e.g., a bell) over several learning trials, the neutral stimulus by itself can come to elicit the response the biologically potent stimulus elicits. \href{https://en.wikipedia.org/wiki/Ivan_Pavlov}{Ivan Pavlov} -- known best for inducing dogs to salivate in the presence of a stimulus previously linked with food -- became a leading figure in the Soviet Union \& inspired followers to use his methods on humans. In the United States, \href{https://en.wikipedia.org/wiki/Edward_Lee_Thorndike}{Edward Lee Thorndike} initiated ``\href{https://en.wikipedia.org/wiki/Connectionism}{connectionist}'' studied by trapping animals in ``puzzle boxes'' \& rewarding them for escaping. Thorndike wrote in 1911, ``There can be no moral warrant for studying man's nature unless the study will enable us to control his acts.'' From 1910 to 1913 the American Psychological Association went through a sea change of opinion, away from \href{https://en.wikipedia.org/wiki/Mentalism_(psychology)}{mentalism} \& towards ``behavioralism.'' In 1913, John B. Watson coined the term behaviorism for this school of thought. Watson's famous \href{https://en.wikipedia.org/wiki/Little_Albert_experiment}{Little Albert experiment} [\textsf{Video. The film of the Little Albert experiment.}] in 1920 was at 1st thought to demonstrate that repeated use of upsetting loud noises could instill \href{https://en.wikipedia.org/wiki/Phobia}{phobias} (aversions to other stimuli) in an infant human, although such a conclusion was likely an exaggeration. \href{https://en.wikipedia.org/wiki/Karl_Lashley}{Karl Lashley}, a close collaborator with Watson, examined biological manifestations of learning in the brain.

\href{https://en.wikipedia.org/wiki/Clark_L._Hull}{Clark L. Hull}, \href{https://en.wikipedia.org/wiki/Edwin_Guthrie}{Edwin Guthrie}, \& others did much to help behaviorism become a widely used paradigm. A new method of ``instrumental'' or ``\href{https://en.wikipedia.org/wiki/Operant_conditioning}{operant}'' conditioning added the concepts of \href{https://en.wikipedia.org/wiki/Reinforcement}{reinforcement} \& \href{https://en.wikipedia.org/wiki/Punishment}{punishment} to the model of behavior change. \href{https://en.wikipedia.org/wiki/Radical_behaviorism}{Radical behaviorists} avoided discussing the inner workings of the mind, especially the unconscious mind, which they considered impossible to assess scientifically. Operant conditioning was 1st described by Miller \& Kanorski \& popularized in the U.S. by \href{https://en.wikipedia.org/wiki/B.F._Skinner}{B.F. Skinner}, who emerged as a leading intellectual of the behaviorist movement.

\href{https://en.wikipedia.org/wiki/Noam_Chomsky}{Noam Chomsky} published an influential critique of radical behaviorism on the grounds that behaviorist principles could not adequately explain the complex mental process of \href{https://en.wikipedia.org/wiki/Language_acquisition}{language acquisition} \& language use. The review, which was scathing, did much to reduce the status of behaviorism within psychology. \href{https://en.wikipedia.org/wiki/Martin_Seligman}{Martin Seligman} \& his colleagues discovered that they could condition in dogs a state of ``\href{https://en.wikipedia.org/wiki/Learned_helplessness}{learned helplessness}'', which was not predicted by the behaviorist approach to psychology. \href{https://en.wikipedia.org/wiki/Edward_C._Tolman}{Edward C. Tolman} advanced a hybrid ``cognitive behavioral'' model, most notably with his 1948 publication discussing the \href{https://en.wikipedia.org/wiki/Cognitive_map}{cognitive maps} used by rats to guess at the location of food at the end of a maze. Skinner's behaviorism did not die, in part because it generated successful practical applications.

The \href{https://en.wikipedia.org/wiki/Association_for_Behavior_Analysis_International}{Association for Behavior Analysis International} was founded in 1974 \& by 2003 had members from 42 countries. The field has gained a foothold in Latin America \& Japan. \href{https://en.wikipedia.org/wiki/Applied_behavior_analysis}{Applied behavior analysis} is the term used for the application of the principles of operant conditioning to change socially significant behavior (it supersedes the term, ``behavior modification'').'' -- \href{https://en.wikipedia.org/wiki/Psychology#Behaviorist}{Wikipedia\texttt{/}major schools of thought\texttt{/}behaviorist}

\subsubsection{Cognitive}
``Main article: \href{https://en.wikipedia.org/wiki/Cognitive_psychology}{Wikipedia\texttt{/}cognitive psychology}. Cognitive psychology involves the study of \href{https://en.wikipedia.org/wiki/Mental_process}{mental processes}, including \href{https://en.wikipedia.org/wiki/Perception}{perception}, \href{https://en.wikipedia.org/wiki/Attention}{attention}, language comprehension \& production, \href{https://en.wikipedia.org/wiki/Memory}{memory}, \& problem solving. Researchers in the field of cognitive psychology are sometimes called \href{https://en.wikipedia.org/wiki/Cognitivism_(psychology)}{cognitivists}. They rely on an \href{https://en.wikipedia.org/wiki/Information_processing}{information processing} model of mental functioning. Cognitivist research is informed by \href{https://en.wikipedia.org/wiki/Functionalism_(philosophy_of_mind)}{functionalism} \& experimental psychology.

Starting in the 1950s, the experimental techniques developed by Wundt, James, Ebbinghaus, \& others re-emerged as experimental psychology became increasingly cognitivist \&, eventually, constituted a part of the wider, interdisciplinary \href{https://en.wikipedia.org/wiki/Cognitive_science}{cognitive science}. Some called this development the \href{https://en.wikipedia.org/wiki/Cognitive_revolution}{cognitive revolution} because it rejected the anti-mentalist dogma of behaviorism as well as the strictures of psychoanalysis.

\textsf{Fig. The Stroop effect is the fact that naming the color of the 1st set of words is easier \& quicker than the 2nd. Fig. \href{https://en.wikipedia.org/wiki/Baddeley's_model_of_working_memory}{Baddeley's model of working memory}.}

\href{https://en.wikipedia.org/wiki/Albert_Bandura}{Albert Bandura} helped along the transition in psychology from behaviorism to cognitive psychology. Bandura \& other \href{https://en.wikipedia.org/wiki/Social_learning_theory}{social learning theorists} advanced the idea of vicarious learning. In other words, they advanced the view that a child can learn by observing his or her social environment \& not necessarily from having been reinforced for enacting a behavior, although they did not rule out the influence of reinforcement on learning a behavior.

\textsf{Fig. The \href{https://en.wikipedia.org/wiki/Muller-Lyer_illusion}{M\"uller--Lyer illusion}. Psychologists make inferences about mental processes from shared phenomena such as optical illusions.}

Technological advances also renewed interest in mental states \& mental representations. English neuroscientist \href{https://en.wikipedia.org/wiki/Charles_Sherrington}{Charles Sherrington} \& Canadian psychologist \href{https://en.wikipedia.org/wiki/Donald_O._Hebb}{Donald O. Hebb} used experimental methods to link psychological phenomena to the structure \& function of the brain. The rise of computer science, cybernetics, \& artificial intelligence underlined the value of comparing information processing in humans \& machines.

A popular \& representative topic in this area is \href{https://en.wikipedia.org/wiki/Cognitive_bias}{cognitive bias}, or irrational thought. Psychologists (\& economists) have classified \& described a \href{https://en.wikipedia.org/wiki/List_of_cognitive_biases}{sizeable catalogue of biases} which recur frequently in human thought. The \href{https://en.wikipedia.org/wiki/Availability_heuristic}{availability heuristic}, e.g., is the tendency to overestimate the importance of something which happens to come readily to mind.

Elements of behaviorism \& cognitive psychology were synthesized to form \href{https://en.wikipedia.org/wiki/Cognitive_behavioral_therapy}{cognitive behaviorial therapy}, a form of psychotherapy modified from techniques developed by American psychologist \href{https://en.wikipedia.org/wiki/Albert_Ellis_(psychologist)}{Albert Ellis} \& American psychiatrist \href{https://en.wikipedia.org/wiki/Aaron_Beck}{Aaron T. Beck}.

On a broader level, cognitive science is an interdisciplinary enterprise involving cognitive psychologists, cognitive neuroscientists, linguists, \& researchers in artificial intelligence, human--computer interaction, \& \href{https://en.wikipedia.org/wiki/Computational_neuroscience}{computational neuroscience}. The discipline of cognitive science covers cognitive psychology as well as philosophy of mind, computer science, \& neuroscience. Computer simulations are sometimes used to model phenomena of interest.'' -- \href{https://en.wikipedia.org/wiki/Psychology#Cognitive}{Wikipedia\texttt{/}major schools of thought\texttt{/}cognitive}

\subsubsection{Social}
``Main article: \href{https://en.wikipedia.org/wiki/Social_psychology}{Wikipedia\texttt{/}social psychology}. See also: \href{https://en.wikipedia.org/wiki/Social_psychology_(sociology)}{Wikipedia\texttt{/}social psychology (sociology)}. Social psychology is concerned with how \href{https://en.wikipedia.org/wiki/Behavior}{behaviors}, \href{https://en.wikipedia.org/wiki/Thought}{thoguhts}, \href{https://en.wikipedia.org/wiki/Feeling}{feelings}, \& the social environment influence human interactions. Social psychologists study such topics as the influence of others on an individual's behavior (e.g., \href{https://en.wikipedia.org/wiki/Conformity_(psychology)}{conformity}, \href{https://en.wikipedia.org/wiki/Persuasion}{persuasion}) \& the formation of beliefs, \href{https://en.wikipedia.org/wiki/Attitude_(psychology)}{attitudes}, \& \href{https://en.wikipedia.org/wiki/Stereotype}{stereotypes} about other people. \href{https://en.wikipedia.org/wiki/Social_cognition}{Social cognition} fuses elements of social \& cognitive psychology for the purpose of understanding how people process, remember, or distort social information. The study of \href{https://en.wikipedia.org/wiki/Group_dynamics}{group dynamics} involves research on the nature of leadership, organizational communication, \& related phenomena. In recent years, social psychologists have become interested in \href{https://en.wikipedia.org/wiki/Implicit_Association_Test}{implicit} measures, \href{https://en.wikipedia.org/wiki/Mediation_(statistics)}{medicational} models, \& the interaction of person \& social factors in accounting for behavior. Some concepts that \href{https://en.wikipedia.org/wiki/Sociology}{sociologists} have applied to the study of psychiatric disorders, concepts such as the social role, sick role, social class, life events, culture, migration, \& \href{https://en.wikipedia.org/wiki/Total_institution}{total institution}, have influenced social psychologists.'' -- \href{https://en.wikipedia.org/wiki/Psychology#Social}{Wikipedia\texttt{/}major schools of thought\texttt{/}social}

\subsubsection{Psychoanalytic}
\textsf{Fig. Group photo 1909 in front of \href{https://en.wikipedia.org/wiki/Clark_University}{Clark University}. Front row: Sigmund Freud, G. Stanley Hall, Carl Jung; back row: \href{https://en.wikipedia.org/wiki/Abraham_A._Brill}{Abraham A. Brill}, \href{https://en.wikipedia.org/wiki/Ernest_Jones}{Ernest Jones}, \href{https://en.wikipedia.org/wiki/Sandor_Ferenczi}{S\'andor Ferenczi}.}

``Main articles: \href{https://en.wikipedia.org/wiki/Psychodynamics}{Wikipedia\texttt{/}psychodynamics} \& \href{https://en.wikipedia.org/wiki/Psychoanalysis}{Wikipedia\texttt{/}psychoanalysis}. Psychoanalysis refers to the theories \& therapeutic techniques applied to the unconscious mind \& its impact on everyday life. These theories \& techniques inform treatments for mental disorders. Psychoanalysis originated in the 1890s, most prominently with the work of \href{https://en.wikipedia.org/wiki/Sigmund_Freud}{Sigmund Freud}. Freud's psychoanalytic theory was largely based on interpretive methods, \href{https://en.wikipedia.org/wiki/Introspection}{introspection}, \& clinical observation. It became very well known, largely because it tackled subjects such as \href{https://en.wikipedia.org/wiki/Human_sexuality}{sexuality}, \href{https://en.wikipedia.org/wiki/Psychological_repression}{repression}, \& the unconscious. Freud pioneered the methods of \href{https://en.wikipedia.org/wiki/Free_association_(psychology)}{free association} \& \href{https://en.wikipedia.org/wiki/Dream_interpretation}{dream interpretation}.

Psychoanalytic theory is not monolithic. Other well-known psychoanalytic thinkers who diverged from Freud include \href{https://en.wikipedia.org/wiki/Alfred_Adler}{Alfred Adler}, \href{https://en.wikipedia.org/wiki/Carl_Jung}{Carl Jung}, \href{https://en.wikipedia.org/wiki/Erik_Erikson}{Erik Erikson}, \href{https://en.wikipedia.org/wiki/Melanie_Klein}{Melanie Klein}, \href{https://en.wikipedia.org/wiki/Donald_Winnicott}{D. W. Winnicott}, \href{https://en.wikipedia.org/wiki/Karen_Horney}{Karen Horney}, \href{https://en.wikipedia.org/wiki/Erich_Fromm}{Erich Fromm}, \href{https://en.wikipedia.org/wiki/John_Bowlby}{John Bowlby}, Freud's daughter \href{https://en.wikipedia.org/wiki/Anna_Freud}{Anna Freud}, \& \href{https://en.wikipedia.org/wiki/Harry_Stack_Sullivan}{Harry Stack Sullivan}. These individuals ensured that psychoanalysis would evolve into diverse schools of thought. Among these schools are \href{https://en.wikipedia.org/wiki/Ego_psychology}{ego psychology}, \href{https://en.wikipedia.org/wiki/Object_relations}{object relations}, \& \href{https://en.wikipedia.org/wiki/Interpersonal_psychoanalysis}{interpersonal}, \href{https://en.wikipedia.org/wiki/Jacques_Lacan}{Lacanian}, \& \href{https://en.wikipedia.org/wiki/Relational_psychoanalysis}{relational psychoanalysis}.

Psychologists such as \href{https://en.wikipedia.org/wiki/Hans_Eysenck}{Hans Eysenck} \& philosophers including \href{https://en.wikipedia.org/wiki/Karl_Popper}{Karl Popper} sharply criticized psychoanalysis. Popper argued that psychoanalysis had been misrepresented as a scientific discipline, whereas Eysenck advanced the view that psychoanalytic tenets had been contradicted by \href{https://en.wikipedia.org/wiki/Experiment}{experimental} data. By the end of the 20th century, psychology departments in \href{https://en.wikipedia.org/wiki/Higher_education_in_the_United_States}{American universities} mostly had marginalized Freudian theory, dismissing it as a ``desiccated \& dead'' historical artifact. Researchers such as \href{https://en.wikipedia.org/wiki/Ant%C3%B3nio_Dam%C3%A1sio}{Ant\'onio Dam\'asio}, \href{https://en.wikipedia.org/wiki/Oliver_Sacks}{Oliver Sacks}, \& \href{https://en.wikipedia.org/wiki/Joseph_LeDoux}{Joseph LeDoux}; \& individuals in the emerging field of \href{https://en.wikipedia.org/wiki/Neuro-psychoanalysis}{neuro-psychoanalysis} have defended some of Freud's ideas on scientific grounds.'' -- \href{https://en.wikipedia.org/wiki/Psychology#Psychoanalytic}{Wikipedia\texttt{/}major schools of thought\texttt{/}psychoanalytic}

\subsubsection{Existential-humanistic}
\textsf{Fig. Psychologist Abraham Maslow in 1943 posited that humans have a hierarchy of needs, \& it makes sense to fulfill the basic needs 1st (food, water, etc.) before higher-order needs can be met.}

``Main articles: \href{https://en.wikipedia.org/wiki/Existential_psychology}{Wikipedia\texttt{/}existential psychology} \& \href{https://en.wikipedia.org/wiki/Humanistic_psychology}{Wikipedia\texttt{/}humanistic psychology}. \href{https://en.wikipedia.org/wiki/Humanistic_psychology}{Humanistic psychology}, which has been influenced by existentialism \& phenomenology, stresses \href{https://en.wikipedia.org/wiki/Free_will}{free will} \& \href{https://en.wikipedia.org/wiki/Self-actualization}{self-actualization}. It emerged in the 1950s as a movement within academic psychology, in reaction to both behaviorism \& psychoanalysis. The humanistic approach seeks to view the whole person, not just fragmented parts of the personality or isolated cognitions. Humanistic psychology also focuses on personal growth, \href{https://en.wikipedia.org/wiki/Self-concept}{self-identity}, death, aloneness, \& freedom. It emphasizes subjective meaning, the rejection of determinism, \& concern for positive growth rather than pathology. Some founders of the humanistic school of thought were American psychologists \href{https://en.wikipedia.org/wiki/Abraham_Maslow}{Abraham Maslow}, who formulated a \href{https://en.wikipedia.org/wiki/Maslow%27s_hierarchy_of_needs}{hierarchy of human needs}, \& \href{https://en.wikipedia.org/wiki/Carl_Rogers}{Carl Rogers}, who created \& developed \href{https://en.wikipedia.org/wiki/Client-centered_therapy}{client-centered therapy}.

Later, \href{https://en.wikipedia.org/wiki/Positive_psychology}{positive psychology} opened up humanistic themes to scientific study. Positive psychology is the study of factors which contribute to human happiness \& well-being, focusing more on people who are currently healthy. In 2010, \textit{Clinical Psychological Review} published a special issue devoted to positive psychological interventions, such as \href{https://en.wikipedia.org/wiki/Gratitude_journal}{gratitude journaling} \& the physical expression of gratitude. It is, however, far from clear that positive psychology is effective is making people happier. Positive psychological interventions have been limited in scope, but their effects are thought to be somewhat better than \href{https://en.wikipedia.org/wiki/Placebo}{placebo} effects. The evidence, however, is far from clear that interventions based on positive psychology increase human happiness or resilience.

The \textit{American Association for Humanistic Psychology}, formed in 1963, declared:
\begin{quotation}
	Humanistic psychology is primarily an orientation toward the whole of psychology rather than a distinct area or school. It stands for respect for the worth of persons, respect for differences of approach, open-mindedness as to acceptable methods, \& interest in exploration of new aspects of human behavior. As a ``3rd force'' in contemporary psychology, it is concerned with topics having little place in existing theories \& systems: e.g., love, creativity, self, growth, organism, basic need-gratification, self-actualization, higher values, being, becoming, spontaneity, play, humor, affection, naturalness, warmth, ego-transcendence, objectivity, autonomy, responsibility, meaning, fair-play, transcendental experience, peak experience, courage, \& related concepts.
\end{quotation}
Existential psychology emphasizes the need to understand a client's total orientation towards the world. Existential psychology is opposed to reductionism, behaviorism, \& other methods that objectify the individual. In the 1950s \& 1960s, influenced by philosophers \href{https://en.wikipedia.org/wiki/S%C3%B8ren_Kierkegaard}{S\o ren Kierkegaard} \& \href{https://en.wikipedia.org/wiki/Martin_Heidegger}{Martin Heidegger}, psychoanalytically trained American psychologist \href{https://en.wikipedia.org/wiki/Rollo_May}{Rollo May} helped to develop existential psychology. \href{https://en.wikipedia.org/wiki/Existential_therapy}{Existential psychotheorapy}, which follows from existential psychology, is a therapeutic approach that is based on the idea that a person's inner conflict arises from that individual's confrontation with the givens of existence. Swiss psychoanalyst \href{https://en.wikipedia.org/wiki/Ludwig_Binswanger}{Ludwig Binswanger} \& American psychologist \href{https://en.wikipedia.org/wiki/George_Kelly_(psychologist)}{George Kelly} may also be said to belong to the existential school. Existential psychologists tend to differ from more ``humanistic'' psychologists in the former's relatively neutral view of human nature \& relatively positive assessment of anxiety. Existential psychologists emphasized the humanistic themes o death, free will, \& meaning, suggesting that meaning can be shaped by myths \& narratives; meaning can be deepened by the acceptance of free will, which is requisite to living an \href{https://en.wikipedia.org/wiki/Authenticity_(philosophy)}{authentic} life, albeit often with anxiety with regard to death.

Austrian existential psychiatrist \& \href{https://en.wikipedia.org/wiki/Holocaust}{Holocaust} survivor \href{https://en.wikipedia.org/wiki/Viktor_Frankl}{Viktor Frankl} drew evidence of meaning's therapeutic power from reflections upon his own \href{https://en.wikipedia.org/wiki/Internment}{internment}. He created a variation of existential psychotherapy called \href{https://en.wikipedia.org/wiki/Logotherapy}{logotherapy}, a type of \href{https://en.wikipedia.org/wiki/Existentialism}{existentialist} analysis that focuses on a \textit{will to meaning} (in one's life), as opposed to Adler's \href{https://en.wikipedia.org/wiki/Nietzsche}{Nietzschean} doctrine of \href{https://en.wikipedia.org/wiki/Will_to_power}{\textit{will to power}} or Freud's \href{https://en.wikipedia.org/wiki/Pleasure_principle_(psychology)}{will to pleasure}.'' -- \href{https://en.wikipedia.org/wiki/Psychology#Existential-humanistic}{Wikipedia\texttt{/}major schools of thought\texttt{/}existential-humanistic}

\subsection{Themes}

\subsubsection{Personality}
``Main article: \href{https://en.wikipedia.org/wiki/Personality_psychology}{Wikipedia\texttt{/}personality psychology}. Personality psychology is concerned with enduring patterns of behavior, thought, \& emotion. Theories of personality vary across different psychological schools of thought. Each theory carries different assumptions about such features as the role of the unconscious \& the importance of childhood experience. According to Freud, personality is based on the dynamic interactions of the \href{https://en.wikipedia.org/wiki/Id,_ego,_and_super-ego}{id, ego, \& super-ego}. By contrast, \href{https://en.wikipedia.org/wiki/Trait_theorist}{trait theorists} have developed taxonomies of personality constructs in describing personality in terms of key traits. Trait theorists have often employed statistical data-reduction methods, such as \href{https://en.wikipedia.org/wiki/Factor_analysis}{factor analysis}. Although the number of proposed traits has varied widely, \href{https://en.wikipedia.org/wiki/Hans_Eysenck}{Hans Eysenck}'s early biologically-based model suggests at least 3 major trait constructs are necessary to describe human personality, \href{https://en.wikipedia.org/wiki/Extraversion_and_introversion}{extraversion--introversion}, \href{https://en.wikipedia.org/wiki/Neuroticism}{neuroticism}-stability, \& \href{https://en.wikipedia.org/wiki/Psychoticism}{psychoticism}-normality. \href{https://en.wikipedia.org/wiki/Raymond_Cattell}{Raymond Cattell} empirically derived a theory of \href{https://en.wikipedia.org/wiki/16_personality_factors}{16 personality factors} at the primary-factor level \& up to 8 broader second-stratum factors. Since 1980s, the \href{https://en.wikipedia.org/wiki/Big_Five_personality_traits}{Big 5} (\href{https://en.wikipedia.org/wiki/Openness_to_experience}{openness to experience}, \href{https://en.wikipedia.org/wiki/Conscientiousness}{conscientiousness}, \href{https://en.wikipedia.org/wiki/Extraversion_and_introversion}{extraversion}, \href{https://en.wikipedia.org/wiki/Agreeableness}{agreeableeness}, \& \href{https://en.wikipedia.org/wiki/Neuroticism}{neuroticism}) emerged as an important trait theory of personality. Dimensional models of personality are receiving increasing support, \& a version of dimensional assessment has been included in the \href{https://en.wikipedia.org/wiki/DSM-V}{DSM-V}. However, despite a plethora of research into the various versions of the ``Big 5'' personality dimensions, it appears necessary to move on from static conceptualizations of personality structure to a more dynamic orientation, acknowledging that personality constructs are subject to learning \& change over the lifespan.

An early example of personality assessment was the \href{https://en.wikipedia.org/wiki/Woodworth_Personal_Data_Sheet}{Woodworth Personal Data Sheet}, constructed during World War I. The popular, although psychometrically inadequate, \href{https://en.wikipedia.org/wiki/Myers%E2%80%93Briggs_Type_Indicator}{Myers--Briggs Type Indicator} was developed to assess individuals' ``personality types'' according to the \href{https://en.wikipedia.org/wiki/Psychological_Types}{personality theories of Carl Jung}. The \href{https://en.wikipedia.org/wiki/Minnesota_Multiphasic_Personality_Inventory}{Minnesota Multiphasic Personality Inventory} (MMPI), despite its name, is more a dimensional measure o f psychopathology than a personality measure. \href{https://en.wikipedia.org/wiki/California_Psychological_Inventory}{California Psychological Inventory} contains 20 personality scales (e.g., independence, tolerance). The \href{https://en.wikipedia.org/wiki/International_Personality_Item_Pool}{International Personality Item Pool}, which is in the public domain, has become a source of scales that can be used personality assessment.'' -- \href{https://en.wikipedia.org/wiki/Psychology#Personality}{Wikipedia\texttt{/}psychology\texttt{/}themes\texttt{/}personality}

\subsubsection{Unconscious mind}
``See also: \href{https://en.wikipedia.org/wiki/Unconscious_mind#Psychology}{Wikipedia\texttt{/}unconscious mind\texttt{/}psychology}. Study of the unconscious mind, a part of the psyche outside the individual's awareness but that is believed to influence conscious thought \& behavior, was a hallmark of early psychology. In 1 of the 1st psychology experiments conducted in the United States, \href{https://en.wikipedia.org/wiki/C.S._Peirce}{C.S. Peirce} \& \href{https://en.wikipedia.org/wiki/Joseph_Jastrow}{Joseph Jastrow} found in 1884 that research subjects could choose the minutely heavier of 2 weights even if consciously uncertain of the difference. Freud popularized the concept of the unconscious mind, particularly when he referred to an uncensored intrusion of unconscious thought into one's speech (a \href{https://en.wikipedia.org/wiki/Freudian_slip}{Freudian slip}) or to his efforts \href{https://en.wikipedia.org/wiki/The_Interpretation_of_Dreams}{to interpret dreams}. His 1901 book \href{https://en.wikipedia.org/wiki/The_Psychopathology_of_Everyday_Life}{\textit{The Psychopathology of Everyday Life}} catalogues hundreds of everyday events that Freud explains in terms of unconscious influence. \href{https://en.wikipedia.org/wiki/Pierre_Janet}{Pierre Janet} advanced the idea of a subconscious mind, which could contain autonomous mental elements unavailable to the direct scrutiny of the subject.

The concept of unconscious processes has remained important in psychology. Cognitive psychologists have used a ``filter'' model of attention. According to the model, much information processing takes place below the threshold of consciousness, \& only certain stimuli, limited by their nature \& number, make their way through the filter. Much research has shown that subconscious \href{https://en.wikipedia.org/wiki/Priming_(psychology)}{\textit{priming}} of certain ideas can covertly influence thoughts \& behavior. Because of the unreliability of self-reporting, a major hurdle in his type of research involves demonstrating that a subject's conscious mind has not perceived a target stimulus. For this reason, some psychologists prefer to distinguish between \href{https://en.wikipedia.org/wiki/Implicit_memory}{\textit{implicit}} \& \href{https://en.wikipedia.org/wiki/Explicit_memory}{\textit{explicit}} memory. In another approach, one can also describe a \href{https://en.wikipedia.org/wiki/Subliminal_stimulus}{subliminal stimulus} as meeting an \textit{objective} but not a \textit{subjective} threshold.

The \href{https://en.wikipedia.org/wiki/Automaticity}{automaticity} model of \href{https://en.wikipedia.org/wiki/John_Bargh}{John Bargh} \& others involves the ideas of automatically \& unconscious processing in our understanding of \href{https://en.wikipedia.org/wiki/Social_behavior}{social behavior}, although there has been dispute with regard to replication. Some experimental data suggest that the \href{https://en.wikipedia.org/wiki/Neuroscience_of_free_will}{brain begins to consider taking actions} before the mind becomes aware of them. The influence of unconscious forces on people's choices bears on the philosophical question of free will. John Bargh, \href{https://en.wikipedia.org/wiki/Daniel_Wegner}{Daniel Wegner}, \& \href{https://en.wikipedia.org/wiki/Illusion_of_control}{Ellen Langer describe free will as an illusion}.'' -- \href{https://en.wikipedia.org/wiki/Psychology#Unconscious_mind}{Wikipedia\texttt{/}psychology\texttt{/}themes\texttt{/}unconscious mind}

\subsubsection{Motivation}
``Main article: \href{https://en.wikipedia.org/wiki/Motivation}{Wikipedia\texttt{/}motivation}. Some psychologist study motivation or the subject of why people or lower animals initiate a behavior at a particular time. It also involves the study of why humans \& lower animals continue or terminate a behavior. Psychologists such as William James initially used the term \textit{motivation} to refer to intention, in a sense similar to the concept of \href{https://en.wikipedia.org/wiki/Will_(philosophy)}{\textit{will}} in European philosophy. With the steady rise of Darwinian \& Freudian thinking, instinct also came to be seen as a primary source of motivation. According to \href{https://en.wikipedia.org/wiki/Drive_theory}{drive theory}, the forces of instinct combine into a single source of energy which exerts a constant influence. Psychoanalysis, like biology, regarded these forces as demands originating in the nervous system. Psychoanalysts believed that these forces, especially the sexual instincts, could become entangled \& transmuted within the psyche. Classical psychoanalysis conceives of a struggle between the pleasure principle \& the \href{https://en.wikipedia.org/wiki/Reality_principle}{reality principle}, roughly corresponding to id \& ego. Later, in \href{https://en.wikipedia.org/wiki/Beyond_the_Pleasure_Principle}{\textit{Beyond the Pleasure Principle}}, Freud introduced the concept of the \href{https://en.wikipedia.org/wiki/Death_drive}{death drive}, a compulsion towards aggression, destruction, \& \href{https://en.wikipedia.org/wiki/Repetition_compulsion}{psychic repetition of traumatic events}. Meanwhile, behaviorist researchers used simple dichotomous models (pleasure\texttt{/}pain, reward\texttt{/}punishment) \& well-established principles such as the idea that a thirsty creature will take pleasure in drinking. \href{https://en.wikipedia.org/wiki/Clark_Hull}{Clark Hull} formalized the latter idea with his \href{https://en.wikipedia.org/wiki/Drive_reduction_theory_(learning_theory)}{drive reduction} model.

Hunger, thirst, fear, sexual desire, \& thermoregulation constitute fundamental motivations in animals. Humans seem to exhibit a more complex set of motivations -- though theoretically these could be explained as resulting from desires for belonging, positive self-image, self-consistency, truth, love, \& control.

Motivation can be modulated or manipulated in many different ways. Researchers have found that \href{https://en.wikipedia.org/wiki/Eating}{eating}, e.g., depends not only on the organism's fundamental need for \href{https://en.wikipedia.org/wiki/Homeostasis}{homeostatis} -- an important factor causing the experience of hunger -- but also on circadian rhythms, food availability, food palatability, \& cost. Abstract motivations are also malleable, as evidenced by such phenomena as \textit{goal contagion}: the adoption of goals, sometimes unconsciously, based on inferences about the goals of others. Vohs \& \href{https://en.wikipedia.org/wiki/Roy_Baumeister}{Baumeister} suggest that contrary to the need-desire-fulfillment cycle of animal instincts, human motivations sometimes obey a ``getting begets wanting'' rule: the more you get a reward such as self-esteem, love, drugs, or money, the more you want it. They suggest that this principle can even apply to food, drink, sex, \& sleep.'' -- \href{https://en.wikipedia.org/wiki/Psychology#Motivation}{Wikipedia\texttt{/}psychology\texttt{/}themes\texttt{/}motivation}

\subsubsection{Development psychology}
\textsf{Fig. Developmental psychologists would engage a child with a book \& then make observations based on how the child interacts with the object.}

``Main article: \href{https://en.wikipedia.org/wiki/Developmental_psychology}{Wikipedia\texttt{/}developmental psychology}. Developmental psychology refers to the scientific study of how \& why the thought processes, emotions, \& behaviors of humans change over the course of their lives. Some credit Charles Darwin with conducting the 1st systematic study within the rubric of developmental psychology, having published in 1877 a short paper detailing the development of innate forms of communication based on his observations of his infant son. The main origins of the discipline, however, are fund in the work of \href{https://en.wikipedia.org/wiki/Jean_Piaget}{Jean Piaget}. Like Piaget, developmental psychologists originally focused primarily on the development of cognition from infancy to adolescence. Later, development psychology extended itself to the study cognition over the life span. In addition to studying cognition, developmental psychologists have also come to focus on affective, behavioral, moral, social, \& neural development.

Developmental psychologists who study children use a number of research methods. E.g., they make observations of children in natural settings such as preschools \& engage them in experimental tasks. Such tasks often resemble specially designed games \& activities that are both enjoyable for the child \& scientifically useful. Developmental researchers have even devised clever methods to study the mental processes of infants. In addition to studying children, development psychologists also study aging \& processes throughout the life span, including old age. These psychologists draw on the full range of psychological theories to inform their research.'' -- \href{https://en.wikipedia.org/wiki/Psychology#Development_psychology}{Wikipedia\texttt{/}psychology\texttt{/}themes\texttt{/}development psychology}

\subsubsection{Genes \& environment}
``Main article: \href{https://en.wikipedia.org/wiki/Behavioral_genetics}{Wikipedia\texttt{/}behavioral genetics}. All researched psychological traits are influenced by both \href{https://en.wikipedia.org/wiki/Genes}{genes} \& \href{https://en.wikipedia.org/wiki/Social_environment}{environment}, to varying degrees. These 2 sources of influence are often confounded in observational research of individuals \& families. An example of this confounding can be shown in the transmission of \href{https://en.wikipedia.org/wiki/Depression_(mood)}{depression} from a depressed mother to her offspring. A theory based on environmental transmission would hold that an offspring, by virtue of his or her having a problematic rearing environment managed by a depressed mother, is at risk for developing depression. On the other hand, a hereditarian theory would hold that depression risk in an offspring is influenced to some extent by genes passed to the child from the mother. Genes \& environment in these simple transmission models are completely confounded. A depressed mother may both carry genes that contribute to depression in her offspring \& also create a rearing environment that increases the risk of depression in her child.

\href{https://en.wikipedia.org/wiki/Behavioral_genetics}{Behavioral genetics} researchers have employed methodologies that help to disentangle this confound \& understand the nature \& origins of individual differences in behavior. Traditionally the research has involved \href{https://en.wikipedia.org/wiki/Twin_studies}{twin studies} \& \href{https://en.wikipedia.org/wiki/Adoption_study}{adoption studies}, 2 designs where genetic \& environmental influences can be partially un-confounded. More recently, gene-focused research has contributed to understanding genetic contributions to the development of psychological traits.

The availability of \href{https://en.wikipedia.org/wiki/Microarray}{microarray} \href{https://en.wikipedia.org/wiki/Molecular_genetics}{molecular genetic} or \href{https://en.wikipedia.org/wiki/Genome_sequencing}{genome sequencing} technologies allows researchers to measure participant DNA variation directly, \& test whether individual genetic variants within genes are associated with psychological traits \& \href{https://en.wikipedia.org/wiki/Psychopathology}{psychopathology} through methods including \href{https://en.wikipedia.org/wiki/Genome-wide_association_studies}{genome-wide association studies}. 1 goal of such research is similar to that in \href{https://en.wikipedia.org/wiki/Positional_cloning}{positional cloning} \& its success in \href{https://en.wikipedia.org/wiki/Huntington%27s}{Huntington's}: once a causal gene is discovered biological research can be conducted to understand how that gene influences the phenotype. 1 major result of genetic association studies is the general finding that psychological traits \& psychopathology, as well as complex medical diseases, are highly \href{https://en.wikipedia.org/wiki/Polygenic}{polygenic}, where a large number (on the order of hundreds to thousands) of genetic variants, each of small effect, contribute to individual differences in the behavioral trait oor propensity to the disorder. Active research continues to work toward understanding the genetic \& environmental bases of behavior \& their interaction.'' -- \href{https://en.wikipedia.org/wiki/Psychology#Genes_and_environment}{Wikipedia\texttt{/}psychology\texttt{/}themes\texttt{/}genes \& environment}

\subsection{Applications}

\subsubsection{Psychological testing}

\subsubsection{Mental health care}

\subsubsection{Education}

\subsubsection{Work}

\subsubsection{Military \& intelligence}

\subsubsection{Health, well-being, \& social change}

\paragraph{Social change.}

\paragraph{Medical applications.}

\paragraph{Worker health, safety \& wellbeing.}

\paragraph{Occupational health psychology.}

\subsection{Research Methods}

\subsubsection{Controlled experiments}

\subsubsection{Other types of studies}

\subsubsection{Direct brain observation\texttt{/}manipulation}

\subsubsection{Computer simulation}

\subsubsection{Animal studies}

\subsubsection{Qualitative research}

\subsubsection{Program evaluation}

\subsection{Contemporary issues in methodology \& practice}

\subsubsection{Meta science}

\paragraph{Confirmation bias.}

\paragraph{Replication.}

\paragraph{Misuse of statistics.}

\subsubsection{WEIRD bias}

\subsubsection{Unscientific mental health training}

\subsection{Ethics}

\subsubsection{Humans}

\subsubsection{Other animals}

\subsection{References}




%------------------------------------------------------------------------------%

\selectlanguage{english}
\chapter{\href{https://nesslabs.com/}{Ness Labs}}
\selectlanguage{vietnamese}

\section{\href{https://nesslabs.com/}{Ness Labs}}
\textbf{Slogan.} ``\textit{Make the most of your mind.} Build a lab for your mind with neuroscience-based\footnote{\textbf{neuroscience} [n] [uncountable] the science that deals with the structure \& function of the brain \& the nervous system.} content\footnote{\textbf{content} [n] \textbf{1.} (\textbf{content}) [plural] \textbf{content (of something)} the things that are contained in something; \textbf{2.} (\textbf{contents}) [plural] the different sections that are contained in a book, magazine, journal or website; a list of these sections; \textbf{3.} [singular] the subject matter of a book, speech, programme, etc.; \textbf{4.} [singular] (following a noun or an adjective) the amount of a substance that is contained in something else; \textbf{5.} [uncountable] the information or other material contained on a website, CD-ROM, etc.; [a] [not before noun] satisfied \& happy with what you have; willing to do or accept something; [v] \textbf{content yourself with something} to accept \& be satisfied with something \& not try to have or do something better.} \& conversations\footnote{\textbf{conversation} [n] [countable, uncountable] an informal talk involving a small group of people or only 2; the activity of talking in this way.}. Join a community\footnote{\textbf{community} [n] (plural \textbf{communities}) \textbf{1.} (often \textbf{the community}) [singular] all the people who live in a particular area, country, etc. when considered as a group; \textbf{2.} [countable] (used in compounds) a group of people who share the same religion, race, job, etc.; \textbf{3.} [uncountable] (\textit{approving}) the feeling or sharing things \& belonging to a group in the place where you live; \textbf{4.} [countable] (\textit{biology}) a group of plants \& animals growing or living in the same place or environment; \textbf{the global\texttt{/}international community} [idiom] the countries of the world, considered as a group.} of curious\footnote{\textbf{curious} [a] \textbf{1.} having a strong desire to know about something; \textbf{2.} strange \& unusual.} humans who want to achieve more without sacrificing their mental health\footnote{\textbf{mental health} [n] [uncountable] \textbf{1.} the state of health of somebody's mind; \textbf{2.} the system for treating people with mental health problems.}. 1 weekly email with mindful\footnote{\textbf{mindful} [a] \textbf{1.} [not before noun] (\textit{formal}) remembering somebody\texttt{/}something \& considering them or it when you do something, \textsc{synonym}: \textbf{conscious}; \textbf{2.} concentrating on the present moment, especially as a technique to help you relax.} productivity \& creativity\footnote{\textbf{creativity} [n] [uncountable] the ability to produce something new, using skill \& imagination.} tips.''

\begin{quotation}
	\textit{``When learning is purposeful\footnote{\textbf{purposeful} [a] having a useful purpose; acting with a clear aim \& with determination.}, creativity blossoms\footnote{\textbf{blossom} [n] [countable, uncountable] a flower or a mass of flowers, especially on a fruit tree or bush; [v] \textbf{1.} [intransitive] (of a tree or bush) to produce blossom; \textbf{2.} [intransitive] to become more healthy, confident or successful.}. When creativity blossoms, thinking emanates\footnote{\textbf{emanate} [v] \textbf{emanate from something} to come from something or somewhere, \textsc{synonym}: \textbf{issue from something}.}. When thinking emanates, knowledge is fully lit\footnote{\textbf{lit} past tense, past participle of \textbf{light}.}.''} -- A.P.J. Abdul Kalam (1931--2015), Aerospace Scientist
	
	\textit{``The consistency\footnote{\textbf{consistency} [n] (plural \textbf{consistencies}) \textbf{1.} [uncountable] (\textit{often approving}) the quality of always behaving in the same way or of having the same opinions or standards; the quality of being consistent; \textbf{2.} [countable, uncountable] the consistency of a mixture or a substance, especially a liquid, is how thick, firm or smooth it is.} \& thoughtfulness\footnote{\textbf{thoughtfulness} [n] [uncountable] \textbf{1.} the quality of being quiet, because you are thinking; \textbf{2.} \textbf{thoughtfulness (for somebody)} (\textit{approving}) the quality of thinking about \& caring for other people, \textsc{synonym}: \textbf{consideration, kindness}; \textbf{3.} careful thought that is put into doing something.} of Ness Labs inspires\footnote{\textbf{inspire} [v] \textbf{1.} to make somebody feel confident \ excited about doing something; \textbf{2.} [usually passive] to give somebody the idea for something; to be the reason why somebody does something; \textbf{3.} to make somebody have a particular feeling or emotion.} me to question the ordinary\footnote{\textbf{ordinary} [a] not unusual or different in any way.} \& iterate\footnote{\textbf{iterate} [v] [intransitive] to repeat a mathematical or computing process or set of instructions again \& again, each time applying it to the result of the previous stage.} towards\footnote{\textbf{towards} [prep] (also \textbf{toward} \textit{especially in North American English}) \textbf{1.} in the direction of somebody\texttt{/}something; \textbf{2.} aiming to achieve something; moving closer to achieving something; \textbf{3.} close or closer to a point in time; \textbf{4.} in relation to somebody\texttt{/}something.} being a better version of myself.''} -- Steph Smith, Founder, Integral Labs
	
	\textit{``Anne-Laure is skilled\footnote{\textbf{skilled} [a] \textbf{1.} having enough ability, experience \& knowledge to be able to do something well, \textsc{synonym}: \textbf{expert}; \textbf{2.} having special experience or training in doing a particular job, \textsc{opposite}: \textbf{unskilled}; \textbf{3.} (of a job) needing special abilities or training, \textsc{opposite}: \textbf{unskilled}.} at researching\footnote{\textbf{research} [n] [uncountable] careful study of a subject, especially in order to discover new facts or information about it. The plural form \textbf{researches} is also sometimes used in British English, but is much less frequent.; [v] \textbf{1.} [transitive, intransitive] to study something carefully \& try to discover new facts about it; \textbf{2.} [transitive] to collect information for an article, a book, etc.} complex\footnote{\textbf{complex} [a] \textbf{1.} made of many different things or parts that are connected, \textsc{synonym}: \textbf{complicated}; \textbf{2.} difficult to understand or deal with; [n] \textbf{1.} \textbf{complex of something} a large number of things that are connected, often in a way that is confusing or difficult to understand; \textbf{2.} a group of buildings of a similar type together in 1 place; \textbf{3.} (\textit{chemistry}) an ion or molecule in which 1 or more groups are bonded to a metal atom by shared pairs of electrons provided by atoms in the group.} topics\footnote{\textbf{topic} [n] a particular subject that is studied, written about or discussed.}, \& condensing\footnote{\textbf{condense} [v] \textbf{1.} [intransitive, transitive] to change from a gas into a liquid; to make a gas change into a liquid; \textbf{2.} [intransitive, transitive] to fill a smaller amount of space; to put something into a smaller amount of space; \textbf{3.} [transitive] to put something such as a piece of writing into fewer words; to put a lot of information into a small space.} her findings\footnote{\textbf{finding} [n] \textbf{1.} [usually plural] information that is discovered as the result of research into something; \textbf{2.} (\textit{law}) a decision made by the judge or jury in a court case.} into a digestible\footnote{\textbf{digestible} [a] \textbf{1.} (of food) easy to digest, \textsc{opposite}: \textbf{indigestible}; \textbf{2.} (of information) easy to understand, \textsc{opposite}: \textbf{indigestible}.} format\footnote{\textbf{format} [n] [countable, uncountable] \textbf{1.} the general arrangement, plan or design of something; \textbf{2.} a particular way in which data is processed, stored or displayed; the form in which information or recordings are made available; [v] \textbf{1.} \textbf{format something} to prepare a computer disk so that data can be recorded on it; \textbf{2.} \textbf{format something (to do something)} to arrange text, etc. in a particular way on a page or screen.} that both entertains\footnote{\textbf{entertain} [v] \textbf{1.} [transitive, intransitive] to interest \& be enjoyed by somebody; \textbf{2.} [transitive] (not used in the progressive tenses) \textbf{entertain something} to consider an idea, a hope, a feeling, etc.; \textbf{3.} [intransitive, transitive] to invite people to eat or drink with you as your guests, especially in your home.} \& makes you smarter.''} -- Leandro, Co-Founder, Unubo
	
	\textit{``This was the resource\footnote{\textbf{resource} [n] \textbf{1.} [countable, usually plural] a supply of something that a country, an organization or a person has \& can use; \textbf{2.} [countable] something that can be used to help achieve an aim, especially as a part of work or study; \textbf{3.} (\textbf{resources}) [plural] personal qualities that help you deal with a situation.} I didn't know I needed -- SO badly. Bite-sized\footnote{\textbf{bite-sized} [a] (also \textbf{bite-size}) [usually before noun] \textbf{1.} small enough to put into the mouth \& eat; \textbf{2.} (\textit{informal}) very small or short.} but in-depth\footnote{\textbf{in-depth} [a] [usually before noun] very thorough \& detailed.} insights\footnote{\textbf{insight} [n] \textbf{1.} [countable, uncountable] an understanding of a particular situation or thing; \textbf{2.} [uncountable] the ability to see \& understand the truth about people or situations.} into my brain. Anne-Laure's writing has changed the way I approach work.''} -- Kelly Miller, Director, BPA
\end{quotation}

\section{\href{https://nesslabs.com/taker-giver-matcher}{Ness Labs\texttt{/}Are you a taker, a giver, or a matcher?}}
``Some people only help when it benefits\footnote{\textbf{benefit} [n] \textbf{1.} [countable, uncountable] a helpful \& useful effect that something has; an advantage that something provides; \textbf{2.} [uncountable, countable] (\textit{British English}) money provided by the government to people who need financial help because they are unemployed, sick, etc.; \textbf{give somebody the benefit of the doubt} [idiom] to accept that somebody has told the truth or has not done something wrong because you cannot prove that they have not told the truth\texttt{/}have done something wrong; [v] \textbf{1.} [intransitive] to be in a better position because of something; \textbf{2.} [transitive] \textbf{benefit somebody\texttt{/}something} to be useful or provide an advantage to somebody\texttt{/}something.} themselves, other foster\footnote{\textbf{foster} [v] \textbf{1.} \textbf{foster something} to encourage something to develop, \textsc{synonym}: \textbf{promote}; \textbf{2.} \textbf{foster somebody} (\textit{especially British English}) to take another person's child into your home for a period of time, without becoming the child's legal parent; [a] [only before noun] used with some nouns in connection with the fostering of a child.} transactional\footnote{\textbf{transactional} [a] \textbf{1.} relating to the process of buying or selling; \textbf{2.} relating to communication between people.} relationships, while yet others are generous\footnote{\textbf{generous} [a] (\textit{approving}) \textbf{1.} giving or willing to give time, money, etc. freely; given freely; \textbf{2.} more than is necessary; large; \textbf{3.} kind in the way you treat people; willing to see what is good about somebody\texttt{/}something.} with their time \& energy\footnote{\textbf{energy} [n] \textbf{1.} [uncountable, countable] the ability of matter or radiation to perform work because of its mass, movement, electrical charge, etc.; \textbf{2.} [uncountable] a source of power that can be used by somebody\texttt{/}something, e.g. to provide light \& heat, or to work machines; \textbf{3.} [uncountable] the effort needed to do work or other physical or mental activities; \textbf{4.} (\textbf{energies}) [plural] the physical \& mental effort that you use to do something.}, without asking for anything in return\footnote{\textbf{return} [v] \textbf{1.} [intransitive] \textbf{return (to $\ldots$) (from $\ldots$)} to come or go back from 1 place to another; \textbf{2.} [transitive] to bring, give, put or send something\texttt{/}somebody back to a particular person or place; \textbf{3.} [intransitive] to come back again, \textsc{synonym}: \textbf{reappear}; \textbf{4.} [intransitive] \textbf{return (to something)} to start discussing a subject you were discussing earlier, or doing an activity you were doing earlier; \textbf{5.} [intransitive, transitive] to go back, or to make something go back, to a previous state; \textbf{6.} [transitive] \textbf{return something} to do something or give something to somebody because they have done or given the same to you 1st; \textbf{7.} [transitive] \textbf{return something} to give or produce something such as a response, a result, a particular amount of money, etc.; \textbf{8.} [transitive, often passive] \textbf{return somebody (to something) $|$ return somebody (as something)} (\textit{British English}) to elect somebody to a political position; \textbf{9.} [transitive] \textbf{return a verdict} to give a decision about something in court; [n] \textbf{1.} [singular] the action of arriving in or coming back to a place that you were in before; \textbf{2.} [singular, uncountable] the action of giving, putting or sending something\texttt{/}somebody back; \textbf{3.} [singular] \textbf{return (of something)} the situation when a feeling or state that has not been experienced for some time starts again, \textsc{synonym}: \textbf{reappearance}; \textbf{4.} [singular] \textbf{return to something} the action of going back to an activity that you used to do, or to a situation that you used to be in; \textbf{5.} [uncountable, countable, usually plural] \textbf{return (on something)} the amount of profit that you get from something, \textsc{synonym}: \textbf{earnings, yield}; \textbf{6.} [countable] an official report or statement that gives particular information to the government or another body; \textbf{in return (for something)} [idiom] as an exchagne or a reward for something; as a response to something.}. Whether in their personal or professional\footnote{\textbf{professional} [a] \textbf{1.} [only before noun] connected with a job that needs special training or skill, especially one that needs a high level of education; \textbf{2.} (of people) having a job that needs special training \& a high level of education; \textbf{3.} showing that somebody is well trained \& extremely skilled, \textsc{synonym}: \textbf{competent}; \textbf{4.} suitable or appropriate for somebody working in a particular profession; \textbf{5.} doing something as a paid job rather than just for pleasure; [n] a person who does a job that needs special training \& a high level of education.} relationships, takers\footnote{\textbf{taker} [n] \textbf{1.} [usually plural] a person who is willing to accept something that is being offered; \textbf{2.} (often in compounds) a person who takes something.}, givers\footnote{\textbf{giver} [n] a person or an organization that gives something, especially money.}, \& matchers achieve different outcomes\footnote{\textbf{outcome} [n] the result or effect of an action or event.}. Surprisingly\footnote{\textbf{surprisingly} [adv] in a way that causes surprise.}, givers display\footnote{\textbf{display} [v] \textbf{1.} [transitive] to put something in a place where people can see it easily; to show something to people, \textsc{synonym}: \textbf{exhibit}; \textbf{2.} [transitive] \textbf{display something} to show signs of something, especially a quality, characteristic or feeling; \textbf{3.} [transitive] \textbf{display something} (of a computer, notice, table, etc.) to show information; \textbf{4.} [intransitive] (of male birds \& animals) to show a special pattern of behavior that is intended to attract a female bird or animal; [n] \textbf{1.} [countable] an arrangement of things in a public place to give information or entertain people or advertise something for sale. Things that are \textbf{on display} are put in a place where people can look at them.; \textbf{2.} [countable, uncountable] \textbf{display of something} behavior that shows a particular quality, feeling or ability; \textbf{3.} [uncountable] \textbf{display of something} the act of placing something in a public place for people to see; \textbf{4.} [countable] \textbf{display (of something)} an act of performing a skill or of showing something happening, in order to entertain; \textbf{5.} [countable, uncountable] \textbf{display (of something)} a special pattern of behavior that a male bird or animal shows in order to attract a female bird or animal.} the most radically\footnote{\textbf{radically} [adv] completely; to a very great extent.} distinctive\footnote{\textbf{distinctive} [a] having a quality or characteristic that makes something different \& easily noticed, \textsc{synonym}: \textbf{characteristic}.} results. \textit{Are you a taker, a giver, or a matcher?} \textit{\& how can you shift\footnote{\textbf{shift} [n] \textbf{1.} [countable] a change in position or direction; \textbf{2.} [countable] a period of time worked by a group of workers who start work as another group finishes; \textbf{3.} [uncountable] the system on a keyboard that allows capital letters or a different set of characters to be typed; the key that operates this system; [v] \textbf{1.} [transitive] \textbf{shift something (away from\texttt{/}from A) (to\texttt{/}towards B)} to change the attention, direction or focus of something; \textbf{2.} [intransitive] (of the emphasis or direction of something) to change from 1 state or position to another; \textbf{3.} [intransitive, transitive] to move from 1 position or place to another; to move something in this way; \textbf{shift your ground} [idiom] (\textit{usually disapproving}) to change your opinion about a subject, especially during a discussion.} your reciprocity\footnote{\textbf{reciprocity} [n] [uncountable] a situation in which 2 people, countries, etc. provide the same help or advantages to each other.} style\footnote{\textbf{style} [n] \textbf{1.} [countable, uncountable] the particular way in which something is done; \textbf{2.} [countable, uncountable] the features of a book, painting, building, etc. that make it typical of a particular author, artist, historical period, etc.; \textbf{3.} [countable] a particular design of something, e.g. clothes; \textbf{4.} [uncountable] the quality of being elegant or fashionable \& made to a high standard; \textbf{5.} [uncountable] the correct use of language; \textbf{6.} (in adjectives) having the type of style mentioned; \textbf{7.} [countable] (\textit{biology}) the long thin part of a flower that carries the stigma.} to have a positive impact\footnote{\textbf{impact} [n] [countable, usually singular, uncountable] \textbf{1.} the powerful effect that something has on somebody\texttt{/}something; \textbf{2.} the act of 1 object hitting another; the force with which this happens; [v] [transitive, intransitive] to have an effect on something.} on your work, your relationships, \& the world in general\footnote{\textbf{general} [a] \textbf{1.} affecting or including all or most people, places or things; \textbf{2.} [usually before noun] normal; usual; true in most cases; \textbf{3.} including the most important aspects of something; not exact or detailed, \textsc{synonym}: \textbf{broad}, \textsc{opposite}: \textbf{specific}; \textbf{4.} \textbf{the general direction\texttt{/}area} used to describe the approximate, but not exact, direction or area mentioned; \textbf{5.} not limited to a particular subject, use or activity; \textbf{6.} not limited to 1 part or aspect of a person or thing; \textbf{7.} [only before noun] highest in rank. In some titles, \textbf{General} comes after the noun.; \textbf{as a general rule} [idiom] usually; \textbf{of general interest} [idiom] of interest to most people; [n] (abbr., \textbf{Gen.}) an officer of very high rank in the army or the US air force; the commander of an army; \textbf{in general} [idiom] \textbf{1.} usually; mainly; \textbf{2.} as a whole.}?}

\subsection{Takers, Givers, Matchers}
In his book \href{https://amzn.to/32suu4A}{Give \& Take}, psychologist\footnote{\textbf{psychologist} [n] a scientist who studies psychology.} \& Wharton's top-rated\footnote{\textbf{top-rated} [a] [only before noun] most popular with the public.} professor \textsc{Adam Grant} divides\footnote{\textbf{divide} [v] \textbf{1.} [transitive, usually passive, intransitive] to separate into parts or groups; to make something separate into parts or groups; \textbf{2.} [transitive] \textbf{divide something (up) between\texttt{/}among somebody} to give a share of something to each of a number of different people or organizations, \textsc{synonym}: \textbf{share}; \textbf{3.} [transitive] to be the real or imaginary line or barrier that separates 2 areas, things or people, \textsc{synonym}: \textbf{separate}; \textbf{4.} [transitive] \textbf{divide something (between A \& B)} to use different parts of your time or energy for different activities; \textbf{5.} [transitive] to cause 2 or more people to disagree, \textsc{synonym}: \textbf{split}; \textbf{6.} [transitive] \textbf{divide something by something} to calculate something by finding out how many times 1 number or amount is contained in another; \textbf{divide \& rule} [idiom] to keep control over people by making them disagree with \& fight each other, therefore not giving them the chance to join together \& oppose you; [n] [usually singular] \textbf{1.} a difference between 2 groups of people that separates them from each other; a difference between 2 sets of ideas or areas of activity; \textbf{2.} \textbf{divide (between A \& B)} (\textit{especially North American English}) a line of high land that separates 2 valleys or systems of rivers, \textsc{synonym}: \textbf{watershed}; \textbf{bridge the gap\texttt{/}divide (between A \& B)} [idiom] to reduce or get rid of the differences that exist between 2 things or groups of people.} people into 3 groups: takers, givers, \& matchers. He explains: ``Whereas takers strive\footnote{\textbf{strive} [v] [intransitive] to try very hard to achieve something.} to get as much as possible from others \& matchers aim\footnote{\textbf{aim} [n] the purpose of doing something; what somebody is trying to achieve; \textbf{take aim at somebody\texttt{/}something} [idiom] to direct your criticism at somebody\texttt{/}something; [v] \textbf{1.} [transitive] \textbf{be aimed at (doing) something} to have the intention of achieving something; \textbf{2.} [intransitive, transitive] to try or plan to achieve something; \textbf{3.} [transitive, usually passive] \textbf{aim something at somebody} to say or do something that is intended to influence or affect a particular person or group.} to trade\footnote{\textbf{trade} [n] \textbf{1.} [uncountable] the activity of buying \& selling or of exchanging goods or services between people or countries. \textbf{Fair trade} is trade between companies in developed countries \& producers in developing countries in which fair prices are paid to the producers.; \textbf{2.} [countable] a particular type of business; \textbf{3.} (\textbf{the trade}) [singular $+$ singular or plural verb] the people or companies that are connected with a particular area of business; \textbf{4.} [countable, uncountable] a job, especially one that involves working with your hands \& that requires special training \& skills; \textbf{5.} [uncountable, singular] the amount of goods or services that are sold, \textsc{synonym}: \textbf{business}; [v] \textbf{1.} [intransitive, transitive] to buy \& sell goods \& services. In economics, \textbf{trade} is usually refer to 1 country or economy exchanging goods or services with another.; \textbf{2.} [intransitive] to exist \& operate as a business or company; \textbf{3.} [intransitive, transitive] to be bought \& sold, or to buy \& sell something, on a stock exchange or other financial institution; \textbf{4.} [transitive] to exchange something that you have for something else.} evenly\footnote{\textbf{evenly} [adv] \textbf{1.} in a smooth or regular way; \textbf{2.} with equal amounts for each person or in each place.}, givers are the rare\footnote{\textbf{rare} [a] (\textbf{rarer, rarest}) \textbf{1.} not done, seen, happening, etc. very often; \textbf{2.} existing only in small numbers \& therefore valuable or interesting.} breed\footnote{\textbf{breed} [v] \textbf{1.} [intransitive] (of animals) to have sex \& produce young; \textbf{2.} [transitive] to keep animals or plants in order to produce young ones in a controlled way; \textbf{3.} [transitive] \textbf{breed something} to be the cause of something; [n] \textbf{1.} a type of animal with a particular appearance that makes it different from others of the same species \& that is the result of having been developed in a controlled way; \textbf{2.} [usually singular] a type of person.} of people who contribute\footnote{\textbf{contribute} [v] \textbf{1.} [intransitive] \textbf{contribute (to something)} to be 1 of the causes of something; \textbf{2.} [intransitive, transitive] to help to improve or achieve something, especially by adding new ideas; \textbf{3.} [transitive, intransitive] to give something, especially money or goods, to help somebody\texttt{/}something; \textbf{4.} [transitive, intransitive] to write something for a newspaper, magazine, website, or a radio or television programme; to speak during a meeting or conversation, especially to give your opinion.} to others without expecting anything in return.''
\begin{itemize}
	\item \textbf{Takers.} Takers are self-focused\footnote{\textbf{focused} [a] (also \textbf{focussed}) with your attention directed to what you want to do; with very clear aims.} \& only help others strategically\footnote{\textbf{strategically} [adv] \textbf{1.} in a way that is connected with achieving a particular purpose or gaining an advantage; \textbf{2.} in a way that is connected with gaining an advantage in a war or other military situation.}, when the benefits to themselves outweigh\footnote{\textbf{outweigh} [v] \textbf{outweigh something} to be greater or more important than something.} the personal costs. In the words of Adam Grant: ``Takers have a distinctive\footnote{\textbf{distinctive} [a] having a quality or characteristic that makes something different \& easily noticed, \textsc{synonym}: \textbf{characteristic}.} signature\footnote{\textbf{signature} [n] \textbf{1.} [countable] your name as you usually write it, e.g. at the end of a letter; \textbf{2.} [uncountable] the act of signing something; \textbf{3.} [countable] a particular quality that makes something different from other similar things \& makes it easy to recognize.}: they like to get more than they give. They tilt\footnote{\textbf{tilt} [v] \textbf{1.} [intransitive, transitive] to move into a position with 1 side or end higher than the other; to make something move in this way, \textsc{synonym}: \textbf{tip}; \textbf{2.} [transitive, intransitive] to influence a situation so that 1 particular opinion, person, etc. is preferred or more likely to succeed than another; to change in this way; [n] [singular, uncountable] a position in which 1 end or side of something is higher than the other.} reciprocity\footnote{\textbf{reciprocity} [n] [uncountable] a situation in which 2 people, countries, etc. provide the same help or advantages to each other.} in their own favor\footnote{\textbf{favour} [n] (\textit{US} \textbf{favor}) \textbf{1.} [countable] a thing that you do to help somebody; \textbf{2.} [uncountable] approval or support for somebody\texttt{/}something; \textbf{find favor (with somebody\texttt{/}something)} [idiom] to become accepted \& popular; \textbf{in favor (of somebody\texttt{/}something)} [idiom] \textbf{1.} supporting \& agreeing with something\texttt{/}somebody; \textbf{2.} likely to produce a particular result, often in an unfair way; \textbf{3.} in exchange for another thing (because the other thing is better or you want it more); \textbf{in somebody's favor} [idiom] \textbf{1.} if something is in somebody's favor, it gives them an advantage or helps them; \textbf{2.} a decision or judgment that is in somebody's favor benefits that person or says that they were right; [v] \textbf{1.} to prefer 1 thing to another, especially a particular system, plan or way of doing something; \textbf{2.} to treat somebody\texttt{/}something better than others, especially in an unfair way; \textbf{3.} \textbf{favor something} to provide suitable conditions for something; to make it easier for something to happen.}, putting their own interests ahead of other's needs.''
	\item \textbf{Givers.} On the other hand, givers will help whenever\footnote{\textbf{whenever} [conjunction] \textbf{1.} every time that; \textbf{2.} at any time that; on any occasion that.} the benefits to others exceed\footnote{\textbf{exceed} [v] \textbf{1.} \textbf{exceed something} to be greater than a particular number or amount; \textbf{2.} \textbf{exceed something} to go beyond what the law, an order or a rule says you are allowed to do; \textbf{3.} \textbf{exceed something} to be better than something, \textsc{synonym}: \textbf{surpass}.} the personal costs. As Adam Grant explains: ``In the workplace\footnote{\textbf{workplace} [n] (often \textbf{the workplace}) [singular] a place where people work, such as an office or factory.}, givers are a relatively\footnote{\textbf{relatively} [adv] to a fairly large degree, especially in comparison with something else; \textbf{relatively speaking} [idiom] used when you are comparing something with all similar things.} rare breed. They tilt reciprocity in the other direction, preferring to give more than they get. Whereas takers tend to be self-focused, evaluating what other people can offer them, givers are other-focused, paying more attention to what other people need from them.''
	\item \textbf{Matchers.} Finally, matchers strive to preserve\footnote{\textbf{preserve} [v] \textbf{1.} \textbf{preserve something} to keep a particular quality or feature; \textbf{2.} to keep something safe from harm, in good condition or in its original state; \textbf{3.} to prevent something from decaying, by treating it in a particular way; [n] [singular] an activity, job or interest that is thought to be suitable for 1 particular person or group of people.} an equal balance\footnote{\textbf{balance} [n] \textbf{1.} [singular, uncountable] a situation in which all parts exist in equal or appropriate amounts; \textbf{2.} [countable, usually singular] the amount of money in a bank account; the amount of a bill that remains after part has been paid; \textbf{3.} [uncountable] the ability to keep steady with an equal amount of weight on each side of the body; \textbf{strike a balance (between A \& B)} [idiom] to manage to find a way of being fair to 2 opposing things; to find an acceptable position which is between 2 things; [v] \textbf{1.} [transitive, often passive, intransitive] to be equal in important or amount to something else that has the opposite effect, \textsc{synonym}: \textbf{offset}; \textbf{2.} [transitive] \textbf{balance A with\texttt{/}\& B} to give equal importance to 2 different things or parts of something; \textbf{3.} [transitive, often passive] \textbf{balance a against B} to compare the importance of 2 different things; \textbf{4.} [transitive] \textbf{balance something} (\textit{finance}) to show or make sure that in an account the total money spent is equal to the total money received; \textbf{5.} [intransitive, transitive] \textbf{balance (something) (on something)} to put your body or something else into a position where it is steady \& does not fall.} between giving \& getting. ``Matchers operate\footnote{\textbf{operate} [v] \textbf{1.} [intransitive] to work, happen or exist, especially in a particular way or place at a particular time, \textsc{synonym}: \textbf{function}; \textbf{2.} [transitive] \textbf{operate something} to use or control a system, process or machine; \textbf{3.} [intransitive] \textbf{operate (on somebody\texttt{/}something)} to cut open somebody's body in order to remove or repair a damaged part.} on the principle\footnote{\textbf{principle} [n] \textbf{1.} [countable] a law, rule or theory that something is based on; \textbf{2.} [singular] a general or scientific law that explains how something works or why something happens; \textbf{3.} [countable] a belief that is accepted as a reason for acting or thinking in a particular way; \textbf{4.} [countable, usually plural, uncountable] a moral rule or a strong belief that influences your actions; \textbf{in principle} [idiom] \textbf{1.} if something can be done in principle, there is no good reason why it should not be done although it has not yet been done \& there may be some difficulties; \textbf{2.} in general but not in detail.} of fairness\footnote{\textbf{fairness} [n] [uncountable] the quality of treating people equally or according to the law or rules.}: when they help others, they protect\footnote{\textbf{protect} [v] \textbf{1.} [transitive, intransitive] to keep somebody\texttt{/}something safe from harm or injury; \textbf{2.} [transitive, usually passive] to introduce laws that make it illegal to kill, harm or damage a particular animal, area of land, building, etc.; \textbf{3.} [transitive] to help an industry in your own country by taxing goods from other countries so that there is less competition; \textbf{4.} [transitive, intransitive] to provide somebody\texttt{/}something with insurance against fire, injury, damage, etc.} themselves by seeking\footnote{\textbf{seek} [v] \textbf{1.} [transitive] to ask for something from somebody, such as help or support; \textbf{2.} [transitive, intransitive] to try to obtain or achieve something; \textbf{3.} [intransitive] \textbf{seek to do something} to try to do something, \textsc{synonym}: \textbf{attempt}; \textbf{4.} (\textbf{-seeking}) (in adjectives \& nouns) looking for or trying to get  the thing mentioned; the activity of doing this; \textbf{seek your fortune} [idiom] (\textit{literary}) to try to find a way to become rich, especially by going to another place; \textbf{seek somebody\texttt{/}something out} [phrasal verb] too look for \& find somebody\texttt{/}something, especially when this means using a lot of effort.} reciprocity. If you're a matcher, you believe in tit for tat\footnote{\textbf{tit for tat} [n] [uncountable] a situation in which you do something bad to somebody because they have done the same to you.}, \& your relationships are governed\footnote{\textbf{govern} [v] \textbf{1.} [transitive, intransitive] \textbf{govern (something)} to control a country or its people \& be responsible for introducing new laws \& for organizing public services \& the economy; \textbf{2.} [transitive, often passive] \textbf{govern something} to control or influence how something happens or functions; to control or influence somebody's actions or behavior.} by even\footnote{\textbf{even} [adv] \textbf{1.} used to emphasized something unexpected or surprising; \textbf{2.} used when you are comparing things, to make the comparison stronger; \textbf{3.} used to introduce a more exact description of somebody\texttt{/}something; \textbf{even as} [idiom] just at the same time as somebody does something or as something else happens; \textbf{even if} [idiom] despite the possibility, fact or belief that; no matter whether; \textbf{even now\texttt{/}then} [idiom] \textbf{1.} despite what has\texttt{/}had happened; \textbf{2.} at this or that exact moment; \textbf{even so} [idiom] despite that; [a] \textbf{1.} equal in number, amount or value; shared equally, \textsc{opposite}: \textbf{uneven}; \textbf{2.} that can be divided exactly by 2, \textsc{opposite}: \textbf{odd}; \textbf{break even} [idiom] to complete a piece of business without either losing money or making a profit; \textbf{have an even chance (of doing something)} [idiom] to be equally likely to do or not do something.} exchanges\footnote{\textbf{exchange} [n] \textbf{1.} [countable, uncountable] an act of giving something to somebody or doing something for somebody \& receiving something in return; \textbf{2.} [countable] a conversation or an argument; \textbf{3.} [uncountable] the process of changing the money of 1 country into that of another; \textbf{4.} [countable] an arrangement when 2 people or groups from different countries visit each other's homes or do each other's jobs for a short time.} of favors.''
\end{itemize}
Of course, most people are not locked\footnote{\textbf{lock} [v] \textbf{1.} [transitive, intransitive] \textbf{lock (something)} to fasten something with a lock; to be fastened with a lock; \textbf{2.} [transitive] \textbf{lock something $+$ adv.\texttt{/}prep} to put something in a safe place \& lock it; \textbf{3.} [intransitive, transitive] to become fixed in 1 position \& unable to move; to make something become fixed in this way; \textbf{4.} [transitive] (\textbf{be locked in\texttt{/}into something}) to be involved in a difficult situation, an argument, a disagreement, etc.; \textbf{lock somebody\texttt{/}yourself in ($\ldots$)} [phrasal verb] to prevent somebody from leaving a place by locking the door; \textbf{lock somebody up} [phrasal verb] (\textit{rather informal}) to put somebody in prison; \textbf{lock something up} [phrasal verb] \textbf{1.} to put money into an investment that you cannot easily turn into cash; \textbf{2.} (\textbf{be locked up in something}) to be in a place where it cannot easily be obtained.} in 1 reciprocity style. ``Giving, taking, \& matching are 3 fundamental\footnote{\textbf{fundamental} [a] \textbf{1.} serious \& very important; affecting the most central \& important parts of something, \textsc{synonym}: \textbf{basic}; \textbf{2.} forming the necessary basis of something, \textsc{synonym}: \textbf{essential}.} styles of social\footnote{\textbf{social} [a] \textbf{1.} [only before noun] connected with society \& the way it is organized; \textbf{2.} [only before noun] connected with activities in which people meet each other for pleasure; \textbf{3.} [only before noun] connected with a person's position in society; \textbf{4.} [only before noun] (\textit{ecology}) (of animals) living naturally in groups, rather than alone.} interaction\footnote{\textbf{interaction} [n] [uncountable, countable] \textbf{1.} the effect that 2 things have on each other; \textbf{2.} the way that people communicate with each other, especially while they work or spend time with them.}, but the lines between them aren't hard \& fast. You might find that you shift from 1 reciprocity style to another as you travel across different work roles \& relationships.'' E.g., you may be a giver when mentoring\footnote{\textbf{mentor} [n] \textbf{1.} an experienced person who advises \& helps somebody with less experience over a period of time; \textbf{2.} an experienced person in a company, university, etc. who trains \& advises new employees or students.}\,\footnote{\textbf{mentoring} [n] [uncountable] the practice of helping \& advising a less experienced person over a period of time, especially as part of a formal programme in a company, university, etc.} a less-experienced\footnote{\textbf{experienced} [a] \textbf{1.} having knowledge or skill in a particular job or activity; \textbf{2.} having knowledge as a result of doing something for a long time, or having had a lot of different experiences.} colleague, act as a taker when negotiating\footnote{\textbf{negotiate} [v] \textbf{1.} [intransitive] to try to reach an agreement by formal discussion; \textbf{2.} [transitive] to arrange or agree something by formal discussion; \textbf{3.} [transitive] \textbf{negotiate something ($+$ adv.\texttt{/}prep.)} to successfully get over or past a difficult part on a path or route; \textbf{4.} [transitive] \textbf{negotiate something ($+$ adv.\texttt{/}prep.)} to successfully solve a problem that is preventing you from achieving something.} your salary\footnote{\textbf{salary} [n] (plural \textbf{salaries}) money that employees receive for doing their job, especially professional employees or people working in an office, usually paid every month.}, \& be a matcher when exchanging productivity\footnote{\textbf{productivity} [n] [uncountable] the rate at which a worker, a company or country produces goods; the amount produced, compared with how much time, work \& money is needed to produce them.} tips\footnote{\textbf{tip} [n] \textbf{1.} the thin pointed end of something; \textbf{2.} a small piece of advice about something practical, \textsc{synonym}: \textbf{hint}; \textbf{3.} a small amount of extra money that you give to somebody, e.g. somebody who serves you in a restaurant; \textbf{the tip of the iceberg} [idiom] only a small part of a much larger problem; [v] \textbf{1.} [intransitive, transitive] to move so that 1 end or side is higher than the other; to move something into this position, \textsc{synonym}: \textbf{tilt}; \textbf{2.} [transitive] \textbf{tip something $+$ adv.\texttt{/}prep.} to make something come out of a container by holding the container at the angle; \textbf{3.} [intransitive, transitive] to develop in a particular direction; to make something develop in a particular direction; \textbf{tip the balance\texttt{/}scales (in favor of, against, etc. somebody\texttt{/}something)} to give somebody\texttt{/}something enough of an advantage or disadvantage, so that the result of something is affected.} with a friend.

Instead of an automatic\footnote{\textbf{automatic} [a] \textbf{1.} (of a machine or device) having controls that work without needing a person to operate them; \textbf{2.} done or happening without thinking, \textsc{synonym}: \textbf{instinctive}; \textbf{3.} always happening as a result of a particular action or situation.} behavior\footnote{\textbf{behavior} [n] \textbf{1.} [uncountable, countable] the way that somebody\texttt{/}something functions or reacts in a particular situation; \textbf{2.} [uncountable] the way that somebody behaves, especially towards other people.}, choosing how we engage\footnote{\textbf{engage} [v] \textbf{1.} \textbf{engage somebody\texttt{/}something} to succeed in attracting \& keeping somebody's attention \& interest; \textbf{2.} to employ somebody to do a particular job; \textbf{engage in something $|$ be engaged in something} [phrasal verb] to take part in an activity; \textbf{engage with something\texttt{/}somebody} [phrasal verb] to become involved with \& try to understand something\texttt{/}somebody.} with friends \& colleagues can be a conscious\footnote{\textbf{conscious} [a] \textbf{1.} [not before noun] aware of something; noticing something, \textsc{opposite}: \textbf{unconscious}; \textbf{2.} able to use your senses \& mental powers to understand what is happening, \textsc{opposite}: \textbf{unconscious}; \textbf{3.} (of actions, feelings, etc.) deliberate or controlled, \textsc{opposite}: \textbf{unconscious}; \textbf{4.} being particularly interested in something.} choice. Adam Grant explains: ``Every time we interact\footnote{\textbf{interact} [v] \textbf{1.} [intransitive] if 1 thing interacts with another, or if 2 things interact, 1 thing has an effect on the other, or the 2 things have an effect on each other; \textbf{2.} [intransitive] \textbf{interact (with somebody)} to communicate with somebody, especially while you work or spend time with them.} with another person at work, we have a choice to make: do we try to claim as much value as we can, or contribute value without worrying about what we receive in return?''

\subsection{The Impact of Giving}
\textit{Does being a giver pay\footnote{\textbf{pay} [v] \textbf{1.} [intransitive, transitive] to give somebody money for work, goods, services, etc.; \textbf{2.} [intransitive] (of a business, etc.) to produce a profit; \textbf{3.} [intransitive, transitive] to result in some advantage for somebody; \textbf{4.} [intransitive, transitive] to suffer or accept a disadvantage because of your beliefs or actions; \textbf{5.} [transitive] \textbf{pay attention\texttt{/}heed\texttt{/}regard\texttt{/}tribute\texttt{/}homage\texttt{/}respect (to somebody\texttt{/}something)} to give attention, etc. to somebody\texttt{/}something; \textbf{6.} [transitive] \textbf{pay a visit (to somebody\texttt{/}something) $|$ pay (somebody\texttt{/}something) a visit} to visit somebody\texttt{/}something; \textbf{pay off} [phrasal verb] (of a plan or an action) to bring benefits or good results; \textbf{pay something off} [phrasal verb] to finish paying money owed for something.} off?} It seems giving does have a positive impact at an organizational\footnote{\textbf{organizational} [a] (\textit{British English also} \textbf{organisational}) \textbf{1.} connected with an organization or with organizations in general; \textbf{2.} connected with the ability to arrange or organize things well.} level. Nathan P. Podsakoff \& his team at the University of Arizona conducted\footnote{\textbf{conduct} [v] \textbf{1.} \textbf{conduct something} to organize \&\texttt{/}or do a particular activity; \textbf{2.} \textbf{conduct something} (of a substance) to allow heat or electricity to pass along or through it; \textbf{3.} \textbf{conduct yourself $+$ adv.\texttt{/}prep.} (\textit{formal}) to behave in a particular way; [n] [uncountable] (\textit{formal}) \textbf{1.} a person's behavior; \textbf{2.} \textbf{conduct of something} the way in which business or an activity is organized \& managed.} a meta-analysis\footnote{\textbf{meta-analysis} [n] [countable, uncountable] (plurla \textbf{meta-analyses}) research that combines the results of a number of related studies.} [\href{https://www.researchgate.net/publication/200824574_Individual-_and_Organizational-Level_Consequences_of_Organizational_Citizenship_Behaviors_A_Meta-Analysis_Article}{Nathan P. Podsakoff, Steven W. Whiting, Philip Podsakoff, Brian D. Blume. \textit{Individual- \& Organizational-Level Consequence of Organizational Citizenship Behaviors: A Meta-Analysis}}] across 38 studies covering more than 3,500 business units, \& found that companies with a culture\footnote{\textbf{culture} [n] \textbf{1.} [uncountable] the customs, beliefs, art, way of lief or social organization of a particular country or group; \textbf{2.} [countable] a country or group with its own customs \& beliefs, art, way of life \& social organization; \textbf{3.} [countable, uncountable] the typical beliefs, attitudes \& behavior that people in a particular group or organization share; \textbf{4.} [uncountable] \textbf{culture (of something)} activities such as literature, music, art \& film, thought of as a group; \textbf{5.} [uncountable] the process of growing cells or bacteria in an artificial substance or medical or scientific study; the substance in which they are grown; \textbf{6.} [countable] a group of cells or bacteria grown for medical or scientific study; [v] \textbf{culture something} to keep cells or bacteria in conditions that are suitable for growth, for medical or scientific study.} of generosity\footnote{\textbf{generosity} [n] [uncountable] the quality of being kind \& generous.} \& giving -- which they call ``Organizational Citizenship\footnote{\textbf{citizenship} [n] [uncountable] \textbf{1.} the legal right to belong to a particular country; \textbf{2.} the state of being a citizen \& accepting the responsibilities of it.} Behaviors'' -- are more likely\footnote{\textbf{likely} [a] (\textbf{likelier, likeliest}) (\textbf{more likely} \& \textbf{most likely} are the usual forms.) \textbf{1.} that can be expected, \textsc{synonym}: \textbf{probable}; \textbf{2.} if somebody is likely to do something, or something is likely to happen, they will probably do it or it will probably happen, \textsc{opposite}: \textbf{unlikely}; \textbf{3.} seeming suitable for a purpose; [adv] probably.} to have higher productivity\footnote{\textbf{productivity} [n] [uncountable] the rate at which a worker, a company or country produces goods; the amount produced, compared with how much time, work \& money is needed to produce them.}, efficiency\footnote{\textbf{efficiency} [n] \textbf{1.} [uncountable] the quality of doing something well with no waste of time or money; \textbf{2.} [uncountable, countable] (\textit{specialist}) the relationship between the amount of energy that goes into a machine or an engine, \& the amount that it produces; \textbf{3.} (\textbf{efficiencies}) [plural] ways of wasting less time \& money or of saving time or money.}, customer satisfaction\footnote{\textbf{satisfaction} [n] \textbf{1.} [uncountable, countable] the good feeling that you have when you have achieved something or when something that you wanted to happen does happen; something that gives you this feeling, \textsc{opposite}: \textbf{dissatisfaction}; \textbf{2.} [uncountable, singular] \textbf{satisfaction (of something)} the act of satisfying a need or desire; \textbf{3.} [uncountable] \textbf{satisfaction (of something)} (\textit{formal}) an acceptable way of dealing with a complaint, a debt, an injury, etc.; \textbf{to somebody's satisfaction} [idiom] \textbf{1.} if you do something to somebody's satisfaction, they are pleased with it; \textbf{2.} if you prove something to somebody's satisfaction, they believe or accept it.}, as well as reduced costs.

But you may want to ask about the individual\footnote{\textbf{individual} [n] \textbf{1.} a person considered separately rather than as part of a group; \textbf{2.} a single member of a group or class; \textbf{3.} a person who is very different from others \& has lots of new \& interesting ideas; [a] \textbf{1.} [only before noun] considered separately rather than as part of a group; \textbf{2.} [only before noun] of or for a particular person; \textbf{3.} [only before noun] designed for use by 1 person; \textbf{4.} characteristic of a particular person or thing; \textbf{5.} (\textit{usually approving}) having an unusual character, \textsc{synonym}: \textbf{distinctive, original}.} impact of being a giver. The answer is pretty surprising. Givers are most likely to occupy\footnote{\textbf{occupy} [v] \textbf{1.} \textbf{occupy something} to fill or use a space, area or amount of time, \textsc{synonym}: \textbf{take up something}; \textbf{2.} \textbf{occupy something} to live or work in a room, house or building; \textbf{3.} \textbf{occupy something} to enter a place in a large group \& take control of it, especially by military force; \textbf{4.} \textbf{occupy something} to have an official job or position, \textsc{synonym}: \textbf{hold}; \textbf{5.} \textbf{occupy something} to be in or at a particular position in a system, \textsc{synonym}: \textbf{hold}; \textbf{6.} to fill your time or keep you busy doing something.} \textit{both the lowest \& highest levels} of an organization\footnote{\textbf{organization} [n] (\textit{British English also} \textbf{organisation}) \textbf{1.} [countable] an organized group of people with a particular purpose, such as a business or government department; \textbf{2.} [uncountable] the way in which the different parts of something are arranged, \textsc{synonym}: \textbf{structure}; \textbf{3.} [uncountable] the act of making arrangements or preparations for something, \textsc{synonym}: \textbf{planning}; \textbf{4.} [uncountable] the quality of being arranged in a neat, careful \& logical way; the ability to plan your work or life well \& in an efficient way.}. ``The worst performers\footnote{\textbf{performer} [n] \textbf{1.} a person or thing that behaves or works in the way mentioned; \textbf{2.} a person who performs for an audience in a show or concert.} \& the best performers are givers; takers \& matchers are more likely to land\footnote{\textbf{land} [n] \textbf{1.} [uncountable] the part of the earth's surface that is not covered by water; \textbf{2.} [uncountable] (\textbf{lands} [plural]) the area of ground that somebody owns, especially when you think of it as property that can be bought or sold; \textbf{3.} [uncountable] (\textbf{lands} [plural]) an area of ground, especially of a particular type or used for a particular purpose, \textsc{synonym}: \textbf{terrain}; \textbf{4.} [countable] a country or state; \textbf{5.} (\textbf{the land}) [uncountable] used to refer to country areas \& the way of life in the countryside, or to ground or soil used for farming; [v] [intransitive, transitive] to arrive on land or another surface; to put somebody\texttt{/}something on land or another surface.} in the middle. ($\ldots$) Givers dominate\footnote{\textbf{dominate} [v] \textbf{1.} [transitive, intransitive] \textbf{dominate (something\texttt{/}somebody)} to control or have a lot of influence over something\texttt{/}somebody, especially in a negative way; \textbf{2.} [transitive] \textbf{dominate something} to be the most important or obvious feature of something; \textbf{3.} [transitive, intransitive] \textbf{dominate (something)} to be the largest, highest or most common thing in a place.} the bottom \& the top of the success ladder\footnote{\textbf{ladder} [n] \textbf{1.} [usually singular] a series of stages by which you can make progress in your life or career; \textbf{2.} a piece of equipment for climbing up \& down something such as the side of a building, consisting of 2 lengths of wood or metal that are joined together by steps.}. Across\footnote{\textbf{across} [prep] \textbf{1.} from 1 side to the other side of something; \textbf{2.} on the other side of something; \textbf{3.} on or over a part of the body; \textbf{4.} in every part of a place, group of people, etc., \textsc{synonym}: \textbf{throughout}; [adv] from 1 side to the other side; \textbf{across from somebody\texttt{/}something} [idiom] opposite somebody\texttt{/}something.} occupations\footnote{\textbf{occupation} [n] \textbf{1.} [countable] a job or profession; \textbf{2.} [uncountable] the act of moving into a country, town, etc. \& taking control of it using military force; the period of time during which a country, town, etc. is controlled in this way; \textbf{3.} [uncountable] the act of living in or using a building, room or piece of land; \textbf{4.} [countable] a way of spending time, especially when you are not working.}, if you examine\footnote{\textbf{examine} [v] \textbf{1.} to consider or study an idea or subject very carefully; \textbf{2.} to look at somebody\texttt{/}something closely, to see if there is anything wrong or to find the cause of a problem; \textbf{3.} \textbf{examine somebody} to give somebody a test to see how much they know about a subject or what they can do.} the link\footnote{\textbf{link} [v] [often passive] \textbf{1.} to make a physical or electronic connection between 1 object, machine or place \& another, \textsc{synonym}: \textbf{connect}; \textbf{2.} to make or have a connection with somebody\texttt{/}something, especially where 1 thing affects the other; \textbf{3.} to state that there is a connection or relationship between 2 things or people, \textsc{synonym}: \textbf{associate}; \textbf{link up (with somebody\texttt{/}something)} [phrasal verb] to join or become joined with somebody\texttt{/}something; [n] \textbf{1.} a connection between 2 or more people or things, especially where one affects the other; \textbf{2.} a relationship between 2 or more people, countries or organizations; \textbf{3.} a means of traveling or communicating between 2 places; \textbf{4.} (\textit{computing}) a place in an electronic document that is connected to another electronic document or to another part of the same document; \textbf{a link in the chain} [idiom] 1 of the stages in a process or a line of argument; \textbf{the weak link (in the chain)} [idiom] the point at which a system or an organization is most likely to fail.} between reciprocity styles \& success, the givers are more likely to become champs\footnote{\textbf{champ} [v] [intransitive, transitive] \textbf{champ (something)} (especially of horses) to bite or eat something noisily; \textbf{champing at the bit} [idiom] (\textit{informal}) impatient to do or start doing something; [n] an informal way of referring to a champion, often used in newspapers.} -- not only chumps\footnote{\textbf{chump} [n] (\textit{old-fashioned, informal}) a stupid person.}.''

As you can see, givers are more rare than takers \& matchers, \& have dramatically\footnote{\textbf{dramatically} [adv] \textbf{1.} in a very sudden or extreme way; to a very great degree; \textbf{2.} in a way that is exciting or impressive; \textbf{3.} using the style of a play in telling a story or giving an account of an event.} different performance\footnote{\textbf{performance} [n] \textbf{1.} [uncountable, countable] how well or badly you do something; how well or badly something works; \textbf{2.} [uncountable, singular] \textbf{performance of something} the action or process of performing a task or function; \textbf{3.} [countable] \textbf{performance (of something)} an act of presenting a play, concert or some other form of entertainment; \textbf{4.} [countable] an act of performing a song, a piece of music, or a role in a play or film.} results. While low-performing givers say yes to everything at the expense of their own work, which has a negative impact on their time management, project delivery, communication, \& execution in general, smart givers take into account what is best for the organization, not only what is best for the person asking for help. As a result, they are highly valued \& manage to both be helpful to their colleagues while positively impacting their organization.

In addition, givers may \href{https://nesslabs.com/how-to-build-a-support-group}{get more support} from fellow colleagues on their way up to success. ``There's something distinctive that happens when givers succeed: \fbox{it spreads \& cascades}. When takers win, there's usually someone else who loses. Research shows that people tend to envy successful takers \& look for ways to knock them down a notch. In contrast, when givers ($\ldots$) win, people are rooting for them \& supporting them, rather than gunning for them. Givers succeed in a way that creates a ripple effect, enhancing the success of people around them.''

In essence, successful givers generate win-win-win situations, where they succeed, their colleagues are elevated, \& the company performs better. Since givers can end up either at the lowest or the highest levels of performance, how can you make sure you are 1 of the most successful givers?

\subsection{How to Be A Smart Giver}
If your goal is moderate success, you can decide to act like a taker or a matcher. But if you want to be part of the top performing members of your organization, or to have a positive impact on the world \& foster win-win-win relationships with people around you, you may want to try to become a smart giver.
\begin{itemize}
	\item \textbf{Change your mindset.} Consider the lens through which you are viewing your job \& your relationships with friends \& family. For your professional context, ask yourself who exactly is affected by your work? How do your choices impact the experience of colleagues \textit{\&} customers? How can you align your decisions so when you win, everyone wins? Instead of being self-focused like a taker or transactional like a matcher, think of an expanding pie where everyone can benefit from your success.
	\item \textbf{Help wisely.} A problem low-performance givers face is the \href{https://nesslabs.com/focused-mind}{lack of focus} on the way they give. Tracking your impact does not mean you need to become a taker \& only help when it benefits you, nor that you need to become a matcher \& only help when you receive equal value in return. Rather, it means that you need to make sure you are helping achieve goals that are beneficial in general, not only to the person you are helping. Ask yourself: is this good for the company, for the customers, for the team? In a personal context, ask: is this good for our group of friends, our family, or our relationship in general? If the answer is no, try to brainstorm a better solution.
	\item \textbf{Track your impact.} From time to time, \href{https://nesslabs.com/weekly-review}{block some time for self-reflection} to look back at past times you have helped, \& what the outcome was. In the end, who benefitted from your help? Was it just 1 person, who may have been a taker? Or did your help have a wider positive impact, which justifies the time \& energy you spent to provide your support? If you feel like your impact wasn't as positive as you expected, try to think of the factors at play, \& how you can be wiser next time you are asked for help so your involvement can be as beneficial as possible.
\end{itemize}
These strategies can be helpful for anyone, but especially for low-performing givers who are spending too much time \& energy on providing scattered support which negatively impact their own work \& relationships. Wherever you are on the spectrum of reciprocity styles, remember that it is a choice: you can practice wise generosity to become a smart giver \& create a positive ripple effect around yourself.'' -- \href{https://nesslabs.com/author/annelaure}{Anne-Laure Le Cunff}


\textbf{Quick notes.} Dr. Who -- a disagreeable giver?

Peterson take \& giver.

%------------------------------------------------------------------------------%

\selectlanguage{english}
\chapter{Miscellaneous}
\selectlanguage{vietnamese}

\begin{itemize}
	\item \textbf{psychiatrist} [n] a doctor who studies \& treats mental illnesses.
	\item \textbf{psychoanalyst} [n] (also \textbf{analyst}) a person who treats patients using psychoanalysis.
\end{itemize}

\section{An Untrained \&\texttt{/}Thus (?) Failed Eidetiker: The Way I Remember}

\begin{remark}
	At the beginning, I am not so sure that this concept should be mentioned here, in the subject of psychology. But when I recalled back some pieces of my memory, I realize how serious \& devastated\footnote{\textsc{vi}: Hồi ức của 1 kẻ có trí nhớ điện tử. Cần phân biệt với bộ phim \href{https://www.imdb.com/title/tt0353969/}{IMDb\texttt{/}Memories of Murder} (2003), original title: Salinui chueok, i.e., Hồi ức kẻ sát nhân.} this ability has affected the development of my personality \& psychology in various aspects.
\end{remark}

\begin{definition}[\href{https://en.wikipedia.org/wiki/Eidetic_memory}{Wikipedia\texttt{/}eidetic memory}]
	``\emph{Eidetic memory} (more commonly called \emph{photographic memory} or \emph{total recall}) is the ability to recall an image from \href{https://en.wikipedia.org/wiki/Memory}{memory} with high precision for a brief period after seeing it only once, \& without using a \href{https://en.wikipedia.org/wiki/Mnemonic_device}{mnemonic device}.''
\end{definition}

\begin{remark}[\href{https://en.wikipedia.org/wiki/Eidetic_memory}{Wikipedia\texttt{/}eidetic memory}]
	``Although the terms \emph{eidetic memory} \& \emph{photographic memory} are popularly used interchangeably, they are also distinguished, with \emph{eidetic memory} referring to the ability to see an object for a few minutes after it is no longer present \& \emph{photographic memory} referring to the ability to recall pages of text or numbers, or similar, in great detail. When the concepts are distinguished, eidetic memory is reported to occur in a small number of children \& generally not found in adults, while true photographic memory has never been demonstrated to exist.'' \footnote{The word eidetic comes from the Greek word \textit{eidos} meaning ``visible form''.}
\end{remark}

\begin{question}
	Eidetic memory: A gift or a curse?
\end{question}

\subsection{Eidetic vs. Photographic}
From \href{https://en.wikipedia.org/wiki/Eidetic_memory#Eidetic_vs._photographic}{Wikipedia\texttt{/}eidetic memory\texttt{/}eidetic vs. photographic}:

``The terms \textit{eidetic memory} \& \textit{photographic memory} are commonly used interchangeably, but they are also distinguished. Scholar Annette Kujawski Taylor stated,
\begin{quotation}
	``In eidetic memory, a person has an almost faithful mental image snapshot or photograph of an event in their memory. However, eidetic memory is not limited to visual aspects of memory \& includes auditory memories as well as various sensory aspects across a range of stimuli associated with a visual image.''
\end{quotation}
Author Andrew Hudmon commented:
\begin{quotation}
	``Examples of people with a photographic-like memory are rare. Eidetic imagery is the ability to remember an image in so much detail, clarity, \& accuracy that it is as though the image were still being perceived. It is not perfect, as it is subject to distortions \& additions (like episodic memory) \& vocalization interferes with the memory.''
\end{quotation}
``Eidetikers'', as those who possess this ability are called, report a vivid \href{https://en.wikipedia.org/wiki/Afterimage}{after image} that lingers in the visual field with their eyes appearing to scan across the image as it is described. Contrary to ordinary mental imagery, eidetic images are externally projected, experienced as ``out there'' rather than in the mind. Vividness \& stability of the image begin to fade within minutes after the removal of the visual stimulus.

\href{https://en.wikipedia.org/wiki/Scott_Lilienfeld}{Lilienfeld} et al. stated,
\begin{quotation}
	``People with eidetic memory can supposedly hold a visual image in their mind with such clarity that they can describe it perfectly or almost perfectly $\ldots$, just as we can describe the details of a painting immediately in front of us with \fbox{near perfect accuracy}.''
\end{quotation}
By contrast, photographic memory may be defined as the ability to recall pages of text, numbers, or similar, in great detail, without the visualization that comes with eidetic memory. It may be described as the ability to briefly look at a page of information \& then recite it perfectly from memory. This type of ability--absolute recall of all events in a lifetime--has never been proven to exist.''\footnote{This appeared in the movie \href{https://www.imdb.com/title/tt0119217/}{Good Will Hunting} (1997) mentioned in the quotes section.}

\subsection{Prevalence}
From \href{https://en.wikipedia.org/wiki/Eidetic_memory#Prevalence}{Wikipedia\texttt{/}eidetic memory\texttt{/}prevalence}:

``\fbox{Eidetic memory is typically found only in young children, as it is virtually nonexistent in adults.} Hudmon stated, \textit{``Children possess far more capacity for eidetic imagery than adults, suggesting that a developmental change (e.g., acquiring language skills) may disrupt the potential for eidetic imagery.''}''

``It has been hypothesized that language acquisition \& verbal skills allow older children to think more abstractly \& thus rely less on \href{https://en.wikipedia.org/wiki/Visual_memory}{visual memory} systems. Extensive research has failed to demonstrate consistent correlations between the presence of eidetic imagery \& any cognitive, intellectual, neurological, or emotional measure.''

``A few adults have had phenomenal memories (not necessarily of images), but their abilities are also unconnected with their intelligence levels \& tend to be highly specialized. In extreme cases, like those of \href{https://en.wikipedia.org/wiki/Solomon_Shereshevsky}{Solomon Shereshevsky} \& \href{https://en.wikipedia.org/wiki/Kim_Peek}{Kim Peek}, memory skills can reportedly hinder social skills. Shereshevsky was \fbox{a trained \href{https://en.wikipedia.org/wiki/Mnemonist}{mnemonist}, not an eidetic memorizer}, \& there are no studies that confirm whether Kim Peek had true eidetic memory.''

\subsection{Skepticism}
From \href{https://en.wikipedia.org/wiki/Eidetic_memory#Skepticism}{Wikipedia\texttt{/}eidetic memory\texttt{/}skepticism}: [$\ldots$]

``Lilienfeld et al. stated:
\begin{quotation}
	``Some psychologists believe that eidetic memory reflects an unusually long persistence of the iconic image in some lucky people''. [$\ldots$] ``More recent evidence raises questions about whether any memories are truly photographic (Rothen, Meier \& Ward, 2012). Eidetikers' memories are clearly remarkable, but they are rarely perfect. Their memories often contain minor errors, including information that was not present in the original visual stimulus. So even eidetic memory often appears to be reconstructive''.
\end{quotation}
\href{https://en.wikipedia.org/wiki/Skeptical_movement}{Scientific skeptic} author \href{https://en.wikipedia.org/wiki/Brian_Dunning_(author)}{Brian Dunning} reviewed the literature on the subject of both eidetic \& photographic memory in 2016 \& concluded that there is ``a lack of compelling evidence that eidetic memory exists at all among healthy adults, \& no evidence that photographic memory exists. But there's a common theme running through many of these research papers, \& that's that the difference between ordinary memory \& \href{https://en.wikipedia.org/wiki/Exceptional_memory}{exceptional memory} appears to be one of degree.''''

\subsection{Trained Mnemonists}
From \href{https://en.wikipedia.org/wiki/Eidetic_memory#Trained_mnemonists}{Wikipedia\texttt{/}eidetic memory\texttt{/}trained mnemonists}:

``To constitute photographic or eidetic memory, the visual recall must persist without the use of mnemonics, expert talent, or other cognitive strategies. Various cases have been reported that rely on such skills \& are erroneously attributed to photographic memory.''

\begin{example}
	``An example of extraordinary memory abilities being ascribed to eidetic memory comes from the popular interpretations of \href{https://en.wikipedia.org/wiki/Adriaan_de_Groot}{Adriaan de Groot}'s classic experiments into the ability of \href{https://en.wikipedia.org/wiki/Chess}{chess} \href{https://en.wikipedia.org/wiki/Grandmaster_(chess)}{grandmaster} to memorize complex positions of chess pieces on a chessboard. Initially, Initially, it was found that these experts could recall surprising amounts of information, far more than nonexperts, suggesting eidetic skills. However, when the experts were presented with arrangements of chess pieces that could never occur in a game, their recall was no better than that of the nonexperts, suggesting that they had \fbox{developed an ability to organize certain types of information, rather than possessing innate eidetic ability}.
\end{example}
Individuals identified as having a condition known as \href{https://en.wikipedia.org/wiki/Hyperthymesia}{hyperthymesia} are able to remember very  intricate details of their own personal lives, but the ability seems not to extend to other, non-autobiographical information. They may have vivid recollections such as who they were with, what they were wearing, \& how they were feeling on a specific date many years in the past. Patients under study, e.g., \href{https://en.wikipedia.org/wiki/Jill_Price}{Jill Price}, show brain scans that resemble those with \href{https://en.wikipedia.org/wiki/Obsessive-compulsive_disorder}{obsessive-compulsive disorder}. In fact, Price's unusual autobiographical memory has been attributed as a byproduct of compulsively making journal \& diary entries. Hyperthymestic patients may additionally have depression\footnote{NQBH: a connection between eidetic memory \& depression.} stemming from the inability to forget unpleasant memories \& experiences from the past.\footnote{Exactly my case.} It is a misconception that hyperthymesia suggests any eidetic ability.\footnote{It seems to me that I possess both of these curses, although the latter is less obvious when I grow up: My memory is less sharp \& more messy (somehow the capacity of my memory seems to expand).}

Each year at the \href{https://en.wikipedia.org/wiki/World_Memory_Championships}{World Memory Championships}, the world's best memorizers compete for prizes. None of the world's best competitive memorizers has a photographic memory, \& no one with claimed eidetic or photographic memory has ever won the championship.''

\subsection{Notable Claims}
From \href{https://en.wikipedia.org/wiki/Eidetic_memory#Notable_claims}{Wikipedia\texttt{/}eidetic memory\texttt{/}notable claims}:

``Main article: \href{https://en.wikipedia.org/wiki/List_of_people_claimed_to_possess_an_eidetic_memory}{Wikipedia\texttt{/}List of people claimed to possess an eidetic memory}.

There are a number of individuals whose extraordinary memory has been labeled ``eidetic'', but it is not established conclusively whether they use \href{https://en.wikipedia.org/wiki/Mnemonic}{mnemonics} \& other, non-eidetic memory-enhancement.

\begin{example}
	`Nadia', who began \fbox{drawing realistically} at the age of 3, is \fbox{autistic} \& has been closely studied. During her childhood she produced highly precocious, repetitive drawings from memory, remarkable for being in perspective (which children tend not to achieve until at least adolescence) at the age of 3, which showed different perspectives on an image she was looking at. E.g., when at the age of three she was obsessed with horses after seeing a horse in a story book she generated numbers of images of what a horse should look like in any posture. She could draw other animals, objects, \& parts of human bodies accurately, but represented human faces as jumbled forms.'' \footnote{Cf. my untrained drawing ability compared to a trained adult when I was a boy.}
\end{example}

\begin{example}
	Others have not been thoroughly tested, though savant \href{https://en.wikipedia.org/wiki/Stephen_Wiltshire}{Stephen Wiltshire} can look at a subject once \& then produce, often before an audience, an accurate \& detailed drawing of it, \& has drawn entire cities from memory, based on single, brief helicopter rides; his 6-meter drawing of 305 square miles of New York City is based on a single 20-minute helicopter ride.
\end{example}

\begin{example}
	Another less thoroughly investigated instance is the art of \href{https://en.wikipedia.org/wiki/Winnie_Bamara}{Winnie Bamara}, an Australian indigenous artist of the 1950s.
\end{example}

\begin{question}
	Connection\emph{\texttt{/}}Correlation between eidetic memory \& gifted drawing ability?
\end{question}

\subsection{Quotes on Eidetic Memory}
\begin{itemize}
	\item In the movie \href{https://www.imdb.com/title/tt0289765/}{Red Dragon} (2002), I like the following conversation:
	\begin{quotation}
		Dr. Hannibal Lecter: \textit{``That's fascinating. You know I'd always suspected as much, you are an eidetiker.''}
		
		Will Graham: \textit{``I'm not psychic.''}
		
		Dr. Hannibal Lecter: \textit{``No, no, no, this is different; more akin to artistic imagination. You're able to assume the emotional point-of-view of other people, even those that scare or sicken you. It's a troubling gift, I should think.''}
	\end{quotation}
	\item In the movie \href{https://www.imdb.com/title/tt0119217/}{Good Will Hunting} (1997):
	\begin{quotation}
		\textit{``Do you have a photographic memory?''} [$\ldots$]
	\end{quotation}
\end{itemize}

\section{Psychology \& Scientists\texttt{/}Mathematicians}
``According to \href{https://en.wikipedia.org/wiki/Herman_Goldstine}{Herman Goldstine}, the mathematician \href{https://en.wikipedia.org/wiki/John_von_Neumann}{John von Neumann} was able to recall from memory every book he had ever read.'' -- \href{https://en.wikipedia.org/wiki/Eidetic_memory#Prevalence}{Wikipedia\texttt{/}eidetic memory\texttt{/}prevalence}

\section{Psychology \& Music}
Han Zimmer's masterpieces: $\ldots$

\section{Introversity \&\texttt{/}vs. Extroversity}

\section{Depression: The Unphysical Cancer}
Well, it will take me a really really long long time to beat this shit.

\section{Monomaniac: A Social Loser or A Lonely Wolf?}
Monomaniac - Kẻ độc hành.

\section{Rich Dad, Poor Dad}
I just realize: If I cannot teach my son to become a man, a real man, then I should not have him. ``Like father, like son''. If I cannot help my son get out of the life circle\footnote{\textsc{vi}: vòng lặp lẩn quẩn của cuộc đời.} of poor \& stupidity, then why should I have one?

\section{Undisputed Truth}
Mike Tyson's  autobiography \cite{Tyson_Sloman2013}:
\begin{quotation}
	``This book is dedicated to all the outcasts -- Everyone who has ever been mesmerized, marginalized, tranquilized, beaten down, \& falsely accused. \& incapable of receiving love.'' -- \cite[Dedication]{Tyson_Sloman2013}
\end{quotation}


\section{Miscellaneous}
Ask myself before doing anything literally:

\begin{question}[Decision question]
	Should I do it or not? If yes, why? If no, why?
\end{question}

\begin{question}[Self-study questions]
	What? Why? \& How?
\end{question}

\begin{question}
	What is the best status or feeling in life?
\end{question}
This question lies in the borderline between the fields of psychology \& philosophy. Should I move it to \cite{NQBH/philosophy}?

\begin{proof}[NQBH's personal answer]
	Concentration \& contributions.
\end{proof}

\begin{quotation}
	``He [G. H. Hardy] was, as I [C. P. Snow] later discovered, shy \& self-conscious\footnote{\textbf{self-conscious} [a] \textbf{1.} \textbf{self-conscious (about sth)} nervous\texttt{/}embarrassed about your appearance or what other people think of you; \textbf{2.} \textit{(often disapproving)} done in a way that shows you are aware of the effect that is being produced, \textit{opposite}: \textbf{unselfconscious}.} in all formal actions, \& had a dread of introductions. He just put his head down as it were in a butt of acknowledgment, \& without any preamble whatever began: $\ldots$'' [$\ldots$] ``I [C. P. Snow] half-guessed that he [G. H. Hardy] had a horror of persons, then prevalent in academic society, who devotedly studied the literature but had never played the game.'' [$\ldots$] ``He appeared to find the reply partially reassuring\footnote{\textbf{reassuring} [a] making you feel less worried or uncertain about something.}, \& went on to more tactical questions.'' [$\ldots$] ``As I had plenty of opportunities to realize in the future, Hardy had no faith in intuitions\footnote{\textbf{intuition} [n] \textbf{1.} [uncountable] the ability to know something by using your feelings rather than considering the facts; \textbf{2.} [countable] \textbf{intuition (that $\ldots$)} an idea or a strong feeling that something is true although you cannot explain why. \textsc{vi}: trực giác.} or impressions, his own or anyone else's. The only way to assess someone's knowledge, in Hardy's view, was to examine him. That went for mathematics, literature, philosophy, politics, anything you like. If the man had bluffed \& then wilted under the questions, that was his lookout. \fbox{1st things came 1st, in that brilliant \& concentrated mind.}'' [$\ldots$] ``Nothing else mattered. In the end he [G. H. Hardy] smiled with immense charm, with child-like openness, \& said that Fenner's (the university cricket ground) next season might be bearable after all, with the prospect of some reasonable conversation.'' -- \cite[Foreword, pp. 10--11]{Hardy1992}
	
	``I [C. P. Snow] don't know what the moral is. But it was a major piece of luck for me. This was intellectually the most valuable friendship of my life. His mind, as I have just mentioned, was brilliant \& concentrated: so much so that by his side anyone else's seemed a little muddy, a little pedestrian \& confused. He wasn't a great genius, as Einstein \& Rutherford were. He said, with his usual clarity\footnote{\textbf{clarity} [n] [uncountable] \textbf{1.} the quality of being expressed clearly; \textbf{2.} the ability to think about or understand something clearly; \textbf{3.} if a picture, substance or sound has clarity, you can see or hear it very clearly, or see through it easily.}, that if the word meant anything he was not a genius at all. At his best, he said, he was for a short time the 5th best pure mathematician in the world. Since this character was as beautiful \& candid\footnote{\textbf{candid} [a] \textbf{1.} saying what you think openly \& honestly; not hiding your thoughts; \textbf{2.} a \textbf{candid} photograph is one that is taken without the person in it knowing that they are being photographed.} as his mind, he always made the point that his friend \& collaborator Littlewood was an appreciably more powerful mathematician than he was, \& that his prot\'eg\'e\footnote{\textbf{prot\'eg\'e} [n] (feminine \textbf{prot\'eg\'ee}) \textit{(from French)} a young person who is helped in their career \& personal development by a more experienced person.} Ramanujan really had natural genius in the sense (though not to the extent, \& nothing like so effectively) that the greatest mathematicians had it.
	
	People sometimes thought he was under-rating himself, when he spoke of these friends. It is true that he was magnanimous\footnote{\textbf{magnanimous} [a] \textit{(formal)} kind, generous \& forgiving, especially towards an enemy or competitor.}, as far from envy as a man can be: but I think one mistakes his quality if one doesn't accept his judgment. I prefer to believe in his own statement in \textit{A Mathematician's Apology}, at the same time \fbox{so proud \& so humble}:
	\begin{quotation}
		`I still say to myself when I am depressed \& find myself forced to listen to pompous \& tiresome people, ``Well, I have done 1 thing you could never have done, \& that is to have collaborated with Littlewood \& Ramanujan on something like equal terms.'''
	\end{quotation}
	In any case, his precise ranking must be left to the historians of mathematics (though it will be an almost impossible job, since so much of his best work was done in collaboration). There is something else, thought, at which he was \fbox{clearly superior} to Einstein or Rutherford or any other great genius: \& that is at turning any work of the intellect\footnote{\textbf{intellect} [n] \textbf{1.} [uncountable, countable] the ability to think in a logical way \& understand things, especially at an advanced level; your mind; \textbf{2.} [countable] a very intelligent person.}, major or minor or sheer play, into a work of art. It was \fbox{that gift above all}, I think, which made him, almost without realizing it, purvey\footnote{\textbf{purvey} [v] \textit{(formal)} \textbf{purvey something} to supply food, services or information to people.} such intellectual delight\footnote{\textbf{delight} [n] \textbf{1.} [uncountable, singular] a feeling of great pleasure, \textsc{synonym}: \textbf{joy}; \textbf{2.} [countable] something that gives you great pleasure, \textsc{synonym}: \textbf{joy}.}. When \textit{A Mathematician's Apology} was 1st published, Graham Greene in a review wrote that along with Henry James's notebooks, this was the best account of what it was like to be a \fbox{\textit{creative artist}}\footnote{NQBH: a creative artist wannabe.}. Thinking about the effect Hardy had on all those round him, I believe that is the clue.'' -- \cite[Foreword, pp. 12--13]{Hardy1992}
\end{quotation}


%------------------------------------------------------------------------------%

\selectlanguage{english}
\begin{thebibliography}{99}
	\selectlanguage{vietnamese}
	\bibitem[NQBH\texttt{/}philosophy]{NQBH/philosophy} Nguyễn Quản Bá Hồng. \href{https://github.com/NQBH/hobby/blob/master/philosophy/NQBH_a_personal_journey_to_philosophy.pdf}{\textit{A Personal Journey to Philosophy}}. Mar 2022--now.
	
	\bibitem[Wikipedia]{Wikipedia} \href{https://en.wikipedia.org/wiki/Main_Page}{Wikipedia.org}.
	\begin{itemize}
		\item \href{https://en.wikipedia.org/wiki/Eidetic_memory}{Wikipedia\texttt{/}eidetic memory}
	\end{itemize}
\end{thebibliography}

\printbibliography[heading=bibintoc]
	
\end{document}