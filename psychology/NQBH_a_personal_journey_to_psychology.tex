\documentclass{article}
%\usepackage[backend=biber,natbib=true,style=authoryear]{biblatex}
\usepackage[utf8]{vietnam}
\usepackage{tocloft}
\renewcommand{\cftsecleader}{\cftdotfill{\cftdotsep}}
\usepackage[colorlinks=true,linkcolor=blue,urlcolor=red,citecolor=magenta]{hyperref}
\usepackage{amsmath,amssymb,amsthm,mathtools,float,graphicx}
\allowdisplaybreaks
\numberwithin{equation}{section}
\newtheorem{assumption}{Assumption}[section]
\newtheorem{lemma}{Lemma}[section]
\newtheorem{corollary}{Corollary}[section]
\newtheorem{definition}{Definition}[section]
\newtheorem{proposition}{Proposition}[section]
\newtheorem{theorem}{Theorem}[section]
\newtheorem{notation}{Notation}[section]
\newtheorem{remark}{Remark}[section]
\newtheorem{example}{Example}[section]
\newtheorem{question}{Question}[section]
\newtheorem{problem}{Problem}[section]
\newtheorem{conjecture}{Conjecture}[section]
\usepackage[left=0.5in,right=0.5in,top=1.5cm,bottom=1.5cm]{geometry}
\usepackage{fancyhdr}
\pagestyle{fancy}
\fancyhf{}
\lhead{\small \textsc{Sect.} ~\thesection}
\rhead{\small \nouppercase{\leftmark}}
\renewcommand{\sectionmark}[1]{\markboth{#1}{}}
\cfoot{\thepage}
\def\labelitemii{$\circ$}

\title{A Personal Journey to Psychology: The Way I Perceive}
\author{Nguyễn Quản Bá Hồng}
\date{\today}

\begin{document}
\maketitle
\begin{abstract}
	A \textit{personal} journey to psychology. A collection of quotes from different resources, e.g., psychological books, websites, forums, and Facebook psychological pages, etc., and some \textit{personal} (again) thoughts about them.
\end{abstract}
\tableofcontents

%------------------------------------------------------------------------------%

\section{An Untrained \&\texttt{/}Thus (?) Failed Eidetiker}

\begin{remark}
	At the beginning, I am not so sure that this concept should be mentioned here, in the subject of psychology. But when I recalled back some pieces of my memory, I realize how serious and devastated this ability has affected the development of my personality and psychology in some aspects.
\end{remark}

\begin{definition}[\href{https://en.wikipedia.org/wiki/Eidetic_memory}{Wikipedia\texttt{/}eidetic memory}]
	``\emph{Eidetic memory} (more commonly called \emph{photographic memory} or \emph{total recall}) is the ability to recall an image from \href{https://en.wikipedia.org/wiki/Memory}{memory} with high precision for a brief period after seeing it only once, and without using a \href{https://en.wikipedia.org/wiki/Mnemonic_device}{mnemonic device}.''
\end{definition}

\begin{remark}[\href{https://en.wikipedia.org/wiki/Eidetic_memory}{Wikipedia\texttt{/}eidetic memory}]
	``Although the terms \emph{eidetic memory} and \emph{photographic memory} are popularly used interchangeably, they are also distinguished, with \emph{eidetic memory} referring to the ability to see an object for a few minutes after it is no longer present and \emph{photographic memory} referring to the ability to recall pages of text or numbers, or similar, in great detail. When the concepts are distinguished, eidetic memory is reported to occur in a small number of children and generally not found in adults, while true photographic memory has never been demonstrated to exist.'' \footnote{The word eidetic comes from the Greek word \textit{eidos} meaning ``visible form''.}
\end{remark}

\begin{question}
	Eidetic memory: A gift or a curse?
\end{question}

\subsection{Eidetic vs. Photographic}
From \href{https://en.wikipedia.org/wiki/Eidetic_memory#Eidetic_vs._photographic}{Wikipedia\texttt{/}eidetic memory\texttt{/}eidetic vs. photographic}:

``The terms \textit{eidetic memory} and \textit{photographic memory} are commonly used interchangeably, but they are also distinguished. Scholar Annette Kujawski Taylor stated,
\begin{quotation}
	``In eidetic memory, a person has an almost faithful mental image snapshot or photograph of an event in their memory. However, eidetic memory is not limited to visual aspects of memory and includes auditory memories as well as various sensory aspects across a range of stimuli associated with a visual image.''
\end{quotation}
Author Andrew Hudmon commented:
\begin{quotation}
	``Examples of people with a photographic-like memory are rare. Eidetic imagery is the ability to remember an image in so much detail, clarity, and accuracy that it is as though the image were still being perceived. It is not perfect, as it is subject to distortions and additions (like episodic memory) and vocalization interferes with the memory.''
\end{quotation}
``Eidetikers'', as those who possess this ability are called, report a vivid \href{https://en.wikipedia.org/wiki/Afterimage}{after image} that lingers in the visual field with their eyes appearing to scan across the image as it is described. Contrary to ordinary mental imagery, eidetic images are externally projected, experienced as ``out there'' rather than in the mind. Vividness and stability of the image begin to fade within minutes after the removal of the visual stimulus.

\href{https://en.wikipedia.org/wiki/Scott_Lilienfeld}{Lilienfeld} et al. stated,
\begin{quotation}
	``People with eidetic memory can supposedly hold a visual image in their mind with such clarity that they can describe it perfectly or almost perfectly $\ldots$, just as we can describe the details of a painting immediately in front of us with \fbox{near perfect accuracy}.''
\end{quotation}
By contrast, photographic memory may be defined as the ability to recall pages of text, numbers, or similar, in great detail, without the visualization that comes with eidetic memory. It may be described as the ability to briefly look at a page of information and then recite it perfectly from memory. This type of ability--absolute recall of all events in a lifetime--has never been proven to exist.''\footnote{This appeared in the movie \href{https://www.imdb.com/title/tt0119217/}{Good Will Hunting} (1997) mentioned in the quotes section.}

\subsection{Prevalence}
From \href{https://en.wikipedia.org/wiki/Eidetic_memory#Prevalence}{Wikipedia\texttt{/}eidetic memory\texttt{/}prevalence}:

``\fbox{Eidetic memory is typically found only in young children, as it is virtually nonexistent in adults.} Hudmon stated, \textit{``Children possess far more capacity for eidetic imagery than adults, suggesting that a developmental change (e.g., acquiring language skills) may disrupt the potential for eidetic imagery.''}''

``It has been hypothesized that language acquisition and verbal skills allow older children to think more abstractly and thus rely less on \href{https://en.wikipedia.org/wiki/Visual_memory}{visual memory} systems. Extensive research has failed to demonstrate consistent correlations between the presence of eidetic imagery and any cognitive, intellectual, neurological, or emotional measure.''

``A few adults have had phenomenal memories (not necessarily of images), but their abilities are also unconnected with their intelligence levels and tend to be highly specialized. In extreme cases, like those of \href{https://en.wikipedia.org/wiki/Solomon_Shereshevsky}{Solomon Shereshevsky} and \href{https://en.wikipedia.org/wiki/Kim_Peek}{Kim Peek}, memory skills can reportedly hinder social skills. Shereshevsky was \fbox{a trained \href{https://en.wikipedia.org/wiki/Mnemonist}{mnemonist}, not an eidetic memorizer}, and there are no studies that confirm whether Kim Peek had true eidetic memory.''

\subsection{Skepticism}
From \href{https://en.wikipedia.org/wiki/Eidetic_memory#Skepticism}{Wikipedia\texttt{/}eidetic memory\texttt{/}skepticism}: [$\ldots$]

``Lilienfeld et al. stated:
\begin{quotation}
	``Some psychologists believe that eidetic memory reflects an unusually long persistence of the iconic image in some lucky people''. [$\ldots$] ``More recent evidence raises questions about whether any memories are truly photographic (Rothen, Meier \& Ward, 2012). Eidetikers' memories are clearly remarkable, but they are rarely perfect. Their memories often contain minor errors, including information that was not present in the original visual stimulus. So even eidetic memory often appears to be reconstructive''.
\end{quotation}
\href{https://en.wikipedia.org/wiki/Skeptical_movement}{Scientific skeptic} author \href{https://en.wikipedia.org/wiki/Brian_Dunning_(author)}{Brian Dunning} reviewed the literature on the subject of both eidetic and photographic memory in 2016 and concluded that there is ``a lack of compelling evidence that eidetic memory exists at all among healthy adults, and no evidence that photographic memory exists. But there's a common theme running through many of these research papers, and that's that the difference between ordinary memory and \href{https://en.wikipedia.org/wiki/Exceptional_memory}{exceptional memory} appears to be one of degree.''''

\subsection{Trained Mnemonists}
From \href{https://en.wikipedia.org/wiki/Eidetic_memory#Trained_mnemonists}{Wikipedia\texttt{/}eidetic memory\texttt{/}trained mnemonists}:

``To constitute photographic or eidetic memory, the visual recall must persist without the use of mnemonics, expert talent, or other cognitive strategies. Various cases have been reported that rely on such skills and are erroneously attributed to photographic memory.''

\begin{example}
	``An example of extraordinary memory abilities being ascribed to eidetic memory comes from the popular interpretations of \href{https://en.wikipedia.org/wiki/Adriaan_de_Groot}{Adriaan de Groot}'s classic experiments into the ability of \href{https://en.wikipedia.org/wiki/Chess}{chess} \href{https://en.wikipedia.org/wiki/Grandmaster_(chess)}{grandmaster} to memorize complex positions of chess pieces on a chessboard. Initially, Initially, it was found that these experts could recall surprising amounts of information, far more than nonexperts, suggesting eidetic skills. However, when the experts were presented with arrangements of chess pieces that could never occur in a game, their recall was no better than that of the nonexperts, suggesting that they had \fbox{developed an ability to organize certain types of information, rather than possessing innate eidetic ability}.
\end{example}
Individuals identified as having a condition known as \href{https://en.wikipedia.org/wiki/Hyperthymesia}{hyperthymesia} are able to remember very  intricate details of their own personal lives, but the ability seems not to extend to other, non-autobiographical information. They may have vivid recollections such as who they were with, what they were wearing, and how they were feeling on a specific date many years in the past. Patients under study, e.g., \href{https://en.wikipedia.org/wiki/Jill_Price}{Jill Price}, show brain scans that resemble those with \href{https://en.wikipedia.org/wiki/Obsessive-compulsive_disorder}{obsessive-compulsive disorder}. In fact, Price's unusual autobiographical memory has been attributed as a byproduct of compulsively making journal and diary entries. Hyperthymestic patients may additionally have depression\footnote{NQBH: a connection between eidetic memory and depression.} stemming from the inability to forget unpleasant memories and experiences from the past.\footnote{Exactly my case.} It is a misconception that hyperthymesia suggests any eidetic ability.\footnote{It seems to me that I possess both of these curses, although the latter is less obvious when I grow up: My memory is less sharp and more messy (somehow the capacity of my memory seems to expand).}

Each year at the \href{https://en.wikipedia.org/wiki/World_Memory_Championships}{World Memory Championships}, the world's best memorizers compete for prizes. None of the world's best competitive memorizers has a photographic memory, and no one with claimed eidetic or photographic memory has ever won the championship.''

\subsection{Notable Claims}
From \href{https://en.wikipedia.org/wiki/Eidetic_memory#Notable_claims}{Wikipedia\texttt{/}eidetic memory\texttt{/}notable claims}:

``Main article: \href{https://en.wikipedia.org/wiki/List_of_people_claimed_to_possess_an_eidetic_memory}{List of people claimed to possess an eidetic memory}.

There are a number of individuals whose extraordinary memory has been labeled ``eidetic'', but it is not established conclusively whether they use \href{https://en.wikipedia.org/wiki/Mnemonic}{mnemonics} and other, non-eidetic memory-enhancement.

\begin{example}
	`Nadia', who began \fbox{drawing realistically} at the age of 3, is \fbox{autistic} and has been closely studied. During her childhood she produced highly precocious, repetitive drawings from memory, remarkable for being in perspective (which children tend not to achieve until at least adolescence) at the age of 3, which showed different perspectives on an image she was looking at. E.g., when at the age of three she was obsessed with horses after seeing a horse in a story book she generated numbers of images of what a horse should look like in any posture. She could draw other animals, objects, and parts of human bodies accurately, but represented human faces as jumbled forms.'' \footnote{Cf. my untrained drawing ability compared to a trained adult when I was a boy.}
\end{example}

\begin{example}
	Others have not been thoroughly tested, though savant \href{https://en.wikipedia.org/wiki/Stephen_Wiltshire}{Stephen Wiltshire} can look at a subject once and then produce, often before an audience, an accurate and detailed drawing of it, and has drawn entire cities from memory, based on single, brief helicopter rides; his 6-meter drawing of 305 square miles of New York City is based on a single 20-minute helicopter ride.
\end{example}

\begin{example}
	Another less thoroughly investigated instance is the art of \href{https://en.wikipedia.org/wiki/Winnie_Bamara}{Winnie Bamara}, an Australian indigenous artist of the 1950s.
\end{example}

\begin{question}
	Connection\emph{\texttt{/}}Correlation between eidetic memory and gifted drawing ability?
\end{question}

\subsection{Quotes on Eidetic Memory}
\begin{itemize}
	\item In the movie \href{https://www.imdb.com/title/tt0289765/}{Red Dragon} (2002), I like the following conversation:
	\begin{quotation}
		Dr. Hannibal Lecter: \textit{``That's fascinating. You know I'd always suspected as much, you are an eidetiker.''}
		
		Will Graham: \textit{``I'm not psychic.''}
		
		Dr. Hannibal Lecter: \textit{``No, no, no, this is different; more akin to artistic imagination. You're able to assume the emotional point-of-view of other people, even those that scare or sicken you. It's a troubling gift, I should think.''}
	\end{quotation}
	\item In the movie \href{https://www.imdb.com/title/tt0119217/}{Good Will Hunting} (1997):
	\begin{quotation}
		\textit{``Do you have a photographic memory?''} [$\ldots$]
	\end{quotation}
\end{itemize}

\section{Psychology \& Scientists\texttt{/}Mathematicians}
``According to \href{https://en.wikipedia.org/wiki/Herman_Goldstine}{Herman Goldstine}, the mathematician \href{https://en.wikipedia.org/wiki/John_von_Neumann}{John von Neumann} was able to recall from memory every book he had ever read.'' -- \href{https://en.wikipedia.org/wiki/Eidetic_memory#Prevalence}{Wikipedia\texttt{/}eidetic memory\texttt{/}prevalence}

\section{Psychology \& Music}
Han Zimmer's masterpieces: $\ldots$

\section{Depression: The Unphysical Cancer}
Well, it will take me a really really long long time to beat this shit.

\section{Monomaniac: A Socially Loser or A Lonely Wolf?}
Monomaniac - Kẻ độc hành.

\section{Miscellaneous}
Ask myself before doing anything literally:

\begin{question}[Decision question]
	Should I do it or not? If yes, why? If no, why?
\end{question}

\begin{question}[Self-study questions]
	What? Why? \& How?
\end{question}

\begin{question}
	What is the best status or feeling in life?
\end{question}
This question lies in the borderline between the fields of psychology and philosophy. Should I move it to \cite{NQBH/philosophy}?

\begin{proof}[Answer]
	Concentration and contributions.
\end{proof}

%------------------------------------------------------------------------------%

\begin{thebibliography}{99}
	\bibitem[NQBH\texttt{/}philosophy]{NQBH/philosophy} Nguyễn Quản Bá Hồng. \href{https://github.com/NQBH/hobby/blob/master/philosophy/NQBH_a_personal_journey_to_philosophy.pdf}{\textit{A Personal Journey to Philosophy}}. Mar 2022--now.
	
	\bibitem[Wikipedia]{Wikipedia} \href{https://en.wikipedia.org/wiki/Main_Page}{Wikipedia.org}.
	\begin{itemize}
		\item \href{https://en.wikipedia.org/wiki/Eidetic_memory}{Wikipedia\texttt{/}eidetic memory}
	\end{itemize}
\end{thebibliography}
	
\end{document}