\documentclass{article}
\usepackage[backend=biber,natbib=true,style=authoryear]{biblatex}
\addbibresource{/home/hong/1_NQBH/reference/bib.bib}
\usepackage[utf8]{vietnam}
\usepackage{tocloft}
\renewcommand{\cftsecleader}{\cftdotfill{\cftdotsep}}
\usepackage[colorlinks=true,linkcolor=blue,urlcolor=red,citecolor=magenta]{hyperref}
\usepackage{amsmath,amssymb,amsthm,mathtools,float,graphicx}
\allowdisplaybreaks
\numberwithin{equation}{section}
\newtheorem{assumption}{Assumption}[section]
\newtheorem{conjecture}{Conjecture}[section]
\newtheorem{corollary}{Corollary}[section]
\newtheorem{definition}{Definition}[section]
\newtheorem{example}{Example}[section]
\newtheorem{lemma}{Lemma}[section]
\newtheorem{notation}{Notation}[section]
\newtheorem{principle}{Principle}[section]
\newtheorem{problem}{Problem}[section]
\newtheorem{proposition}{Proposition}[section]
\newtheorem{question}{Question}[section]
\newtheorem{remark}{Remark}[section]
\newtheorem{theorem}{Theorem}[section]
\usepackage[left=0.5in,right=0.5in,top=1.5cm,bottom=1.5cm]{geometry}
\usepackage{fancyhdr}
\pagestyle{fancy}
\fancyhf{}
\lhead{\small \textsc{Sect.} ~\thesection}
\rhead{\small \nouppercase{\leftmark}}
\renewcommand{\sectionmark}[1]{\markboth{#1}{}}
\cfoot{\thepage}
\def\labelitemii{$\circ$}

\title{A Personal Journey to Psychology: The Way I Perceive}
\author{Nguyễn Quản Bá Hồng}
\date{\today}

\begin{document}
\maketitle
\begin{abstract}
	A \textit{personal} journey to psychology. A collection of quotes from different resources, e.g., psychological books, websites, forums, and Facebook psychological pages, etc., and some \textit{personal} (again) thoughts about them.
\end{abstract}
\tableofcontents

%------------------------------------------------------------------------------%

\section*{Foreword\texttt{/}Lời Dẫn}
\begin{question}
	Why this text? \& Why psychology?
\end{question}
Mình từ nhỏ, cũng như nhiều đứa trẻ khác, thích tìm tòi về Khoa học, đặc biệt là Toán học, rồi tiếp đến là Vật lý \& Hóa học. Mình học khá đều các môn này. Nhưng khi bắt đầu tham gia các cuộc thi các cấp, thì mình chỉ học mỗi môn Toán thay vì giỏi nhiều \& đều môn như cấp 2, \& nực cười thay khi thâm chí trong cái môn mình yêu thích nhất là Toán, mình chỉ tập trung cày bừa 1 vài chuyên đề yêu thích như Bất Đẳng Thức \& Số Học, không chú tâm nhiều đến các chủ đề khác. Kết quả thì hiển nhiên mà khó đạt được giải cao. Sau này lên Đại học, thì mình mới có cái laptop cá nhân đầu tiên vào cuối năm nhất. Mình thích tìm hiểu về Toán \& Máy tính. Tiếc là chỉ biết tập trung vào những thứ đó mà mình đã tự hạn hẹp kiến thức của bản thân \& thui chột nhiều khả năng khác mà mình có thể phát triển bằng việc tự học, \& đương nhiên: Tự học là vua của mọi loại kỹ năng.

Hết Đại học, mình học Master ở Pháp. Sau khi tiếp xúc với nhiều thứ mới, nhiều người mới \& thú vị, thì mình bị trầm cảm nặng suốt 1 năm. Mình bắt đầu tỉnh ngộ: mình đã quá chú tâm vào việc cày Toán 1 cách máy móc \& kết quả là mình đã phát triển tâm lý \& nhận thức chậm hơn những người khác. Rồi mình nhận ra tầm quan trọng của những thứ khác để phát triển bản thân, đặc biệt là các kiến thức Tâm lý.

Người nào giúp mình 1 lần, mình sẽ nhớ mãi, biết ơn, \& cố gắng trả ơn cho họ.

[\ldots]

Mỗi người sẽ kể 1 câu chuyện, tùy theo cách nhìn nhận, ghi nhớ của họ, \& thường là sẽ bị móm méo để có lợi cho chính bản thân họ, đôi khi để đổ lỗi, hạ thấp, thậm chí là triệt hạ người khác.

Đời là 1 chuỗi vòng lặp vô tận nhưng không lối thoát thì chưa chắc. \texttt{insert Nguyễn Ngọc Tư quotes}

Bạn cảm thấy  vui\texttt{/}sung sướng khi người khác thất bại? Thì bạn là 1 kẻ tồi (asshole).

\section{An Untrained \&\texttt{/}Thus (?) Failed Eidetiker: The Way I Remember}

\begin{remark}
	At the beginning, I am not so sure that this concept should be mentioned here, in the subject of psychology. But when I recalled back some pieces of my memory, I realize how serious and devastated this ability has affected the development of my personality and psychology in various aspects.
\end{remark}

\begin{definition}[\href{https://en.wikipedia.org/wiki/Eidetic_memory}{Wikipedia\texttt{/}eidetic memory}]
	``\emph{Eidetic memory} (more commonly called \emph{photographic memory} or \emph{total recall}) is the ability to recall an image from \href{https://en.wikipedia.org/wiki/Memory}{memory} with high precision for a brief period after seeing it only once, and without using a \href{https://en.wikipedia.org/wiki/Mnemonic_device}{mnemonic device}.''
\end{definition}

\begin{remark}[\href{https://en.wikipedia.org/wiki/Eidetic_memory}{Wikipedia\texttt{/}eidetic memory}]
	``Although the terms \emph{eidetic memory} and \emph{photographic memory} are popularly used interchangeably, they are also distinguished, with \emph{eidetic memory} referring to the ability to see an object for a few minutes after it is no longer present and \emph{photographic memory} referring to the ability to recall pages of text or numbers, or similar, in great detail. When the concepts are distinguished, eidetic memory is reported to occur in a small number of children and generally not found in adults, while true photographic memory has never been demonstrated to exist.'' \footnote{The word eidetic comes from the Greek word \textit{eidos} meaning ``visible form''.}
\end{remark}

\begin{question}
	Eidetic memory: A gift or a curse?
\end{question}

\subsection{Eidetic vs. Photographic}
From \href{https://en.wikipedia.org/wiki/Eidetic_memory#Eidetic_vs._photographic}{Wikipedia\texttt{/}eidetic memory\texttt{/}eidetic vs. photographic}:

``The terms \textit{eidetic memory} and \textit{photographic memory} are commonly used interchangeably, but they are also distinguished. Scholar Annette Kujawski Taylor stated,
\begin{quotation}
	``In eidetic memory, a person has an almost faithful mental image snapshot or photograph of an event in their memory. However, eidetic memory is not limited to visual aspects of memory and includes auditory memories as well as various sensory aspects across a range of stimuli associated with a visual image.''
\end{quotation}
Author Andrew Hudmon commented:
\begin{quotation}
	``Examples of people with a photographic-like memory are rare. Eidetic imagery is the ability to remember an image in so much detail, clarity, and accuracy that it is as though the image were still being perceived. It is not perfect, as it is subject to distortions and additions (like episodic memory) and vocalization interferes with the memory.''
\end{quotation}
``Eidetikers'', as those who possess this ability are called, report a vivid \href{https://en.wikipedia.org/wiki/Afterimage}{after image} that lingers in the visual field with their eyes appearing to scan across the image as it is described. Contrary to ordinary mental imagery, eidetic images are externally projected, experienced as ``out there'' rather than in the mind. Vividness and stability of the image begin to fade within minutes after the removal of the visual stimulus.

\href{https://en.wikipedia.org/wiki/Scott_Lilienfeld}{Lilienfeld} et al. stated,
\begin{quotation}
	``People with eidetic memory can supposedly hold a visual image in their mind with such clarity that they can describe it perfectly or almost perfectly $\ldots$, just as we can describe the details of a painting immediately in front of us with \fbox{near perfect accuracy}.''
\end{quotation}
By contrast, photographic memory may be defined as the ability to recall pages of text, numbers, or similar, in great detail, without the visualization that comes with eidetic memory. It may be described as the ability to briefly look at a page of information and then recite it perfectly from memory. This type of ability--absolute recall of all events in a lifetime--has never been proven to exist.''\footnote{This appeared in the movie \href{https://www.imdb.com/title/tt0119217/}{Good Will Hunting} (1997) mentioned in the quotes section.}

\subsection{Prevalence}
From \href{https://en.wikipedia.org/wiki/Eidetic_memory#Prevalence}{Wikipedia\texttt{/}eidetic memory\texttt{/}prevalence}:

``\fbox{Eidetic memory is typically found only in young children, as it is virtually nonexistent in adults.} Hudmon stated, \textit{``Children possess far more capacity for eidetic imagery than adults, suggesting that a developmental change (e.g., acquiring language skills) may disrupt the potential for eidetic imagery.''}''

``It has been hypothesized that language acquisition and verbal skills allow older children to think more abstractly and thus rely less on \href{https://en.wikipedia.org/wiki/Visual_memory}{visual memory} systems. Extensive research has failed to demonstrate consistent correlations between the presence of eidetic imagery and any cognitive, intellectual, neurological, or emotional measure.''

``A few adults have had phenomenal memories (not necessarily of images), but their abilities are also unconnected with their intelligence levels and tend to be highly specialized. In extreme cases, like those of \href{https://en.wikipedia.org/wiki/Solomon_Shereshevsky}{Solomon Shereshevsky} and \href{https://en.wikipedia.org/wiki/Kim_Peek}{Kim Peek}, memory skills can reportedly hinder social skills. Shereshevsky was \fbox{a trained \href{https://en.wikipedia.org/wiki/Mnemonist}{mnemonist}, not an eidetic memorizer}, and there are no studies that confirm whether Kim Peek had true eidetic memory.''

\subsection{Skepticism}
From \href{https://en.wikipedia.org/wiki/Eidetic_memory#Skepticism}{Wikipedia\texttt{/}eidetic memory\texttt{/}skepticism}: [$\ldots$]

``Lilienfeld et al. stated:
\begin{quotation}
	``Some psychologists believe that eidetic memory reflects an unusually long persistence of the iconic image in some lucky people''. [$\ldots$] ``More recent evidence raises questions about whether any memories are truly photographic (Rothen, Meier \& Ward, 2012). Eidetikers' memories are clearly remarkable, but they are rarely perfect. Their memories often contain minor errors, including information that was not present in the original visual stimulus. So even eidetic memory often appears to be reconstructive''.
\end{quotation}
\href{https://en.wikipedia.org/wiki/Skeptical_movement}{Scientific skeptic} author \href{https://en.wikipedia.org/wiki/Brian_Dunning_(author)}{Brian Dunning} reviewed the literature on the subject of both eidetic and photographic memory in 2016 and concluded that there is ``a lack of compelling evidence that eidetic memory exists at all among healthy adults, and no evidence that photographic memory exists. But there's a common theme running through many of these research papers, and that's that the difference between ordinary memory and \href{https://en.wikipedia.org/wiki/Exceptional_memory}{exceptional memory} appears to be one of degree.''''

\subsection{Trained Mnemonists}
From \href{https://en.wikipedia.org/wiki/Eidetic_memory#Trained_mnemonists}{Wikipedia\texttt{/}eidetic memory\texttt{/}trained mnemonists}:

``To constitute photographic or eidetic memory, the visual recall must persist without the use of mnemonics, expert talent, or other cognitive strategies. Various cases have been reported that rely on such skills and are erroneously attributed to photographic memory.''

\begin{example}
	``An example of extraordinary memory abilities being ascribed to eidetic memory comes from the popular interpretations of \href{https://en.wikipedia.org/wiki/Adriaan_de_Groot}{Adriaan de Groot}'s classic experiments into the ability of \href{https://en.wikipedia.org/wiki/Chess}{chess} \href{https://en.wikipedia.org/wiki/Grandmaster_(chess)}{grandmaster} to memorize complex positions of chess pieces on a chessboard. Initially, Initially, it was found that these experts could recall surprising amounts of information, far more than nonexperts, suggesting eidetic skills. However, when the experts were presented with arrangements of chess pieces that could never occur in a game, their recall was no better than that of the nonexperts, suggesting that they had \fbox{developed an ability to organize certain types of information, rather than possessing innate eidetic ability}.
\end{example}
Individuals identified as having a condition known as \href{https://en.wikipedia.org/wiki/Hyperthymesia}{hyperthymesia} are able to remember very  intricate details of their own personal lives, but the ability seems not to extend to other, non-autobiographical information. They may have vivid recollections such as who they were with, what they were wearing, and how they were feeling on a specific date many years in the past. Patients under study, e.g., \href{https://en.wikipedia.org/wiki/Jill_Price}{Jill Price}, show brain scans that resemble those with \href{https://en.wikipedia.org/wiki/Obsessive-compulsive_disorder}{obsessive-compulsive disorder}. In fact, Price's unusual autobiographical memory has been attributed as a byproduct of compulsively making journal and diary entries. Hyperthymestic patients may additionally have depression\footnote{NQBH: a connection between eidetic memory and depression.} stemming from the inability to forget unpleasant memories and experiences from the past.\footnote{Exactly my case.} It is a misconception that hyperthymesia suggests any eidetic ability.\footnote{It seems to me that I possess both of these curses, although the latter is less obvious when I grow up: My memory is less sharp and more messy (somehow the capacity of my memory seems to expand).}

Each year at the \href{https://en.wikipedia.org/wiki/World_Memory_Championships}{World Memory Championships}, the world's best memorizers compete for prizes. None of the world's best competitive memorizers has a photographic memory, and no one with claimed eidetic or photographic memory has ever won the championship.''

\subsection{Notable Claims}
From \href{https://en.wikipedia.org/wiki/Eidetic_memory#Notable_claims}{Wikipedia\texttt{/}eidetic memory\texttt{/}notable claims}:

``Main article: \href{https://en.wikipedia.org/wiki/List_of_people_claimed_to_possess_an_eidetic_memory}{List of people claimed to possess an eidetic memory}.

There are a number of individuals whose extraordinary memory has been labeled ``eidetic'', but it is not established conclusively whether they use \href{https://en.wikipedia.org/wiki/Mnemonic}{mnemonics} and other, non-eidetic memory-enhancement.

\begin{example}
	`Nadia', who began \fbox{drawing realistically} at the age of 3, is \fbox{autistic} and has been closely studied. During her childhood she produced highly precocious, repetitive drawings from memory, remarkable for being in perspective (which children tend not to achieve until at least adolescence) at the age of 3, which showed different perspectives on an image she was looking at. E.g., when at the age of three she was obsessed with horses after seeing a horse in a story book she generated numbers of images of what a horse should look like in any posture. She could draw other animals, objects, and parts of human bodies accurately, but represented human faces as jumbled forms.'' \footnote{Cf. my untrained drawing ability compared to a trained adult when I was a boy.}
\end{example}

\begin{example}
	Others have not been thoroughly tested, though savant \href{https://en.wikipedia.org/wiki/Stephen_Wiltshire}{Stephen Wiltshire} can look at a subject once and then produce, often before an audience, an accurate and detailed drawing of it, and has drawn entire cities from memory, based on single, brief helicopter rides; his 6-meter drawing of 305 square miles of New York City is based on a single 20-minute helicopter ride.
\end{example}

\begin{example}
	Another less thoroughly investigated instance is the art of \href{https://en.wikipedia.org/wiki/Winnie_Bamara}{Winnie Bamara}, an Australian indigenous artist of the 1950s.
\end{example}

\begin{question}
	Connection\emph{\texttt{/}}Correlation between eidetic memory and gifted drawing ability?
\end{question}

\subsection{Quotes on Eidetic Memory}
\begin{itemize}
	\item In the movie \href{https://www.imdb.com/title/tt0289765/}{Red Dragon} (2002), I like the following conversation:
	\begin{quotation}
		Dr. Hannibal Lecter: \textit{``That's fascinating. You know I'd always suspected as much, you are an eidetiker.''}
		
		Will Graham: \textit{``I'm not psychic.''}
		
		Dr. Hannibal Lecter: \textit{``No, no, no, this is different; more akin to artistic imagination. You're able to assume the emotional point-of-view of other people, even those that scare or sicken you. It's a troubling gift, I should think.''}
	\end{quotation}
	\item In the movie \href{https://www.imdb.com/title/tt0119217/}{Good Will Hunting} (1997):
	\begin{quotation}
		\textit{``Do you have a photographic memory?''} [$\ldots$]
	\end{quotation}
\end{itemize}

\section{Psychology \& Scientists\texttt{/}Mathematicians}
``According to \href{https://en.wikipedia.org/wiki/Herman_Goldstine}{Herman Goldstine}, the mathematician \href{https://en.wikipedia.org/wiki/John_von_Neumann}{John von Neumann} was able to recall from memory every book he had ever read.'' -- \href{https://en.wikipedia.org/wiki/Eidetic_memory#Prevalence}{Wikipedia\texttt{/}eidetic memory\texttt{/}prevalence}

\section{Psychology \& Music}
Han Zimmer's masterpieces: $\ldots$

\section{Introversity \&\texttt{/}vs. Extroversity}

\section{Depression: The Unphysical Cancer}
Well, it will take me a really really long long time to beat this shit.

\section{Monomaniac: A Social Loser or A Lonely Wolf?}
Monomaniac - Kẻ độc hành.

\section{Rich Dad, Poor Dad}
I just realize: If I cannot teach my son to become a man, a real man, then I should not have him. Like father, like son. If I cannot help my son get out of the life circle\footnote{\textsc{vi}: vòng lặp lẩn quẩn của cuộc đời.} of poor \& stupidity, then why should I have one?

\section{Undisputed Truth}
Mike Tyson's  autobiography \cite{Tyson_Sloman2013}:
\begin{quotation}
	``This book is dedicated to all the outcasts -- Everyone who has ever been mesmerized, marginalized, tranquilized, beaten down, \& falsely accused. \& incapable of receiving love.'' -- \cite[Dedication]{Tyson_Sloman2013}
\end{quotation}


\section{Miscellaneous}
Ask myself before doing anything literally:

\begin{question}[Decision question]
	Should I do it or not? If yes, why? If no, why?
\end{question}

\begin{question}[Self-study questions]
	What? Why? \& How?
\end{question}

\begin{question}
	What is the best status or feeling in life?
\end{question}
This question lies in the borderline between the fields of psychology and philosophy. Should I move it to \cite{NQBH/philosophy}?

\begin{proof}[NQBH's personal answer]
	Concentration and contributions.
\end{proof}

\begin{quotation}
	``He [G. H. Hardy] was, as I [C. P. Snow] later discovered, shy \& self-conscious\footnote{\textbf{self-conscious} [a] \textbf{1.} \textbf{self-conscious (about sth)} nervous\texttt{/}embarrassed about your appearance or what other people think of you; \textbf{2.} \textit{(often disapproving)} done in a way that shows you are aware of the effect that is being produced, \textit{opposite}: \textbf{unselfconscious}.} in all formal actions, \& had a dread of introductions. He just put his head down as it were in a butt of acknowledgment, \& without any preamble whatever began: $\ldots$'' [$\ldots$] ``I [C. P. Snow] half-guessed that he [G. H. Hardy] had a horror of persons, then prevalent in academic society, who devotedly studied the literature but had never played the game.'' [$\ldots$] ``He appeared to find the reply partially reassuring\footnote{\textbf{reassuring} [a] making you feel less worried or uncertain about something.}, \& went on to more tactical questions.'' [$\ldots$] ``As I had plenty of opportunities to realize in the future, Hardy had no faith in intuitions\footnote{\textbf{intuition} [n] \textbf{1.} [uncountable] the ability to know something by using your feelings rather than considering the facts; \textbf{2.} [countable] \textbf{intuition (that $\ldots$)} an idea or a strong feeling that something is true although you cannot explain why. \textsc{vi}: trực giác.} or impressions, his own or anyone else's. The only way to assess someone's knowledge, in Hardy's view, was to examine him. That went for mathematics, literature, philosophy, politics, anything you like. If the man had bluffed \& then wilted under the questions, that was his lookout. \fbox{1st things came 1st, in that brilliant \& concentrated mind.}'' [$\ldots$] ``Nothing else mattered. In the end he [G. H. Hardy] smiled with immense charm, with child-like openness, \& said that Fenner's (the university cricket ground) next season might be bearable after all, with the prospect of some reasonable conversation.'' -- \cite[Foreword, pp. 10--11]{Hardy1992}
	
	``I [C. P. Snow] don't know what the moral is. But it was a major piece of luck for me. This was intellectually the most valuable friendship of my life. His mind, as I have just mentioned, was brilliant \& concentrated: so much so that by his side anyone else's seemed a little muddy, a little pedestrian \& confused. He wasn't a great genius, as Einstein \& Rutherford were. He said, with his usual clarity\footnote{\textbf{clarity} [n] [uncountable] \textbf{1.} the quality of being expressed clearly; \textbf{2.} the ability to think about or understand something clearly; \textbf{3.} if a picture, substance or sound has clarity, you can see or hear it very clearly, or see through it easily.}, that if the word meant anything he was not a genius at all. At his best, he said, he was for a short time the 5th best pure mathematician in the world. Since this character was as beautiful \& candid\footnote{\textbf{candid} [a] \textbf{1.} saying what you think openly \& honestly; not hiding your thoughts; \textbf{2.} a \textbf{candid} photograph is one that is taken without the person in it knowing that they are being photographed.} as his mind, he always made the point that his friend \& collaborator Littlewood was an appreciably more powerful mathematician than he was, \& that his prot\'eg\'e\footnote{\textbf{prot\'eg\'e} [n] (feminine \textbf{prot\'eg\'ee}) \textit{(from French)} a young person who is helped in their career \& personal development by a more experienced person.} Ramanujan really had natural genius in the sense (though not to the extent, \& nothing like so effectively) that the greatest mathematicians had it.
	
	People sometimes thought he was under-rating himself, when he spoke of these friends. It is true that he was magnanimous\footnote{\textbf{magnanimous} [a] \textit{(formal)} kind, generous \& forgiving, especially towards an enemy or competitor.}, as far from envy as a man can be: but I think one mistakes his quality if one doesn't accept his judgment. I prefer to believe in his own statement in \textit{A Mathematician's Apology}, at the same time \fbox{so proud \& so humble}:
	\begin{quotation}
		`I still say to myself when I am depressed \& find myself forced to listen to pompous \& tiresome people, ``Well, I have done 1 thing you could never have done, \& that is to have collaborated with Littlewood \& Ramanujan on something like equal terms.'''
	\end{quotation}
	In any case, his precise ranking must be left to the historians of mathematics (though it will be an almost impossible job, since so much of his best work was done in collaboration). There is something else, thought, at which he was \fbox{clearly superior} to Einstein or Rutherford or any other great genius: \& that is at turning any work of the intellect\footnote{\textbf{intellect} [n] \textbf{1.} [uncountable, countable] the ability to think in a logical way \& understand things, especially at an advanced level; your mind; \textbf{2.} [countable] a very intelligent person.}, major or minor or sheer play, into a work of art. It was \fbox{that gift above all}, I think, which made him, almost without realizing it, purvey\footnote{\textbf{purvey} [v] \textit{(formal)} \textbf{purvey something} to supply food, services or information to people.} such intellectual delight\footnote{\textbf{delight} [n] \textbf{1.} [uncountable, singular] a feeling of great pleasure, \textsc{synonym}: \textbf{joy}; \textbf{2.} [countable] something that gives you great pleasure, \textsc{synonym}: \textbf{joy}.}. When \textit{A Mathematician's Apology} was 1st published, Graham Greene in a review wrote that along with Henry James's notebooks, this was the best account of what it was like to be a \fbox{\textit{creative artist}}\footnote{NQBH: a creative artist wannabe.}. Thinking about the effect Hardy had on all those round him, I believe that is the clue.'' -- \cite[Foreword, pp. 12--13]{Hardy1992}
\end{quotation}


%------------------------------------------------------------------------------%

\begin{thebibliography}{99}
	\bibitem[NQBH\texttt{/}philosophy]{NQBH/philosophy} Nguyễn Quản Bá Hồng. \href{https://github.com/NQBH/hobby/blob/master/philosophy/NQBH_a_personal_journey_to_philosophy.pdf}{\textit{A Personal Journey to Philosophy}}. Mar 2022--now.
	
	\bibitem[Wikipedia]{Wikipedia} \href{https://en.wikipedia.org/wiki/Main_Page}{Wikipedia.org}.
	\begin{itemize}
		\item \href{https://en.wikipedia.org/wiki/Eidetic_memory}{Wikipedia\texttt{/}eidetic memory}
	\end{itemize}
\end{thebibliography}

\printbibliography[heading=bibintoc]
	
\end{document}