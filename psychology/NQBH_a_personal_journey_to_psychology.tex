\documentclass[oneside]{book}
\usepackage[backend=biber,natbib=true,style=authoryear]{biblatex}
\addbibresource{/home/nqbh/reference/bib.bib}
\usepackage[vietnamese,english]{babel}
\usepackage{tocloft}
\renewcommand{\cftsecleader}{\cftdotfill{\cftdotsep}}
\usepackage[colorlinks=true,linkcolor=blue,urlcolor=red,citecolor=magenta]{hyperref}
\usepackage{amsmath,amssymb,amsthm,mathtools,float,graphicx}
\allowdisplaybreaks
\numberwithin{equation}{section}
\newtheorem{assumption}{Assumption}[chapter]
\newtheorem{conjecture}{Conjecture}[chapter]
\newtheorem{corollary}{Corollary}[chapter]
\newtheorem{definition}{Definition}[chapter]
\newtheorem{example}{Example}[chapter]
\newtheorem{lemma}{Lemma}[chapter]
\newtheorem{notation}{Notation}[chapter]
\newtheorem{principle}{Principle}[chapter]
\newtheorem{problem}{Problem}[chapter]
\newtheorem{proposition}{Proposition}[chapter]
\newtheorem{question}{Question}[chapter]
\newtheorem{remark}{Remark}[chapter]
\newtheorem{theorem}{Theorem}[chapter]
\usepackage[left=0.5in,right=0.5in,top=1.5cm,bottom=1.5cm]{geometry}
\usepackage{fancyhdr}
\pagestyle{fancy}
\fancyhf{}
\lhead{\small \textsc{Sect.}~\thesection}
\rhead{\small\nouppercase{\leftmark}}
\renewcommand{\sectionmark}[1]{\markboth{#1}{}}
\cfoot{\thepage}
\def\labelitemii{$\circ$}

\title{A Personal Journey to Psychology: The Way I Perceive}
\author{\selectlanguage{vietnamese} Nguyễn Quản Bá Hồng\footnote{Independent Researcher, Ben Tre City, Vietnam\\e-mail: \texttt{nguyenquanbahong@gmail.com}}}
\date{\today}

\begin{document}
\maketitle
\setcounter{tocdepth}{3}
\setcounter{secnumdepth}{3}
\tableofcontents
\newpage

A \textit{personal} journey to psychology. A collection of quotes from different resources, e.g., psychological books, websites, forums, \& Facebook psychological pages, etc., \& some \textit{personal} (again) thoughts about them.

%------------------------------------------------------------------------------%

\chapter{Wikipedia's}
\selectlanguage{vietnamese}

\section{\href{https://en.wikipedia.org/wiki/First_impression_(psychology)}{Wikipedia\texttt{/}1st Impression (Psychology)}}
``In \href{https://en.wikipedia.org/wiki/Psychology}{psychology}, a \textit{1st impression} is the event when 1 person 1st encounters another person \& forms a \href{https://en.wikipedia.org/wiki/Mental_image}{mental image} of that person. Impression accuracy varies depending on the observer \& the target (person, object, scene, etc.) being observed. 1st impressions are based on a wide range of characteristics: age, \href{https://en.wikipedia.org/wiki/Race_(human_categorization)}{race}, \href{https://en.wikipedia.org/wiki/Culture}{culture}, \href{https://en.wikipedia.org/wiki/Language}{language}, \href{https://en.wikipedia.org/wiki/Gender}{gender}, \href{https://en.wikipedia.org/wiki/Physical_appearance}{physical appearance}, \href{https://en.wikipedia.org/wiki/Accent_(sociolinguistics)}{accent}, \href{https://en.wikipedia.org/wiki/Posture_(psychology)}{posture}, \href{https://en.wikipedia.org/wiki/Voice}{voice}, number of people present, economic status, \& \href{https://en.wikipedia.org/wiki/Time}{time} allowed to process. The 1st impressions individuals give to other could greatly influence how they are treated \& viewed in many contexts of everyday life.'' -- \href{https://en.wikipedia.org/wiki/First_impression_(psychology)}{Wikipedia\texttt{/}1st impression (psychology)}

\subsection{Speed \& accuracy}
``It takes just $\frac{1}{10}$ of a second for us to judge someone \& make a 1st impression. Research finds that the more time participants are afforded to form the impression, the more confidence in impressions they report. Not only are people quick to form 1st impressions, they are also fairly accurate when the target presents themselves genuinely. People are generally not good at perceiving feigned emotions or detecting lies. Research participants who reported forming accurate impressions of specific targets did tend to have more accurate perceptions of specific targets that aligned with other's reports of the target. Individuals are also fairly reliable at understanding the 1st impression that they will project to others.

The rate at which different qualities are detected in 1st impressions may be linked to what has been important survival from an evolutionary perspective. E.g., trustworthiness \& attractiveness were the 2 traits most quickly detected \& evaluated in a study of human faces. People are fairly good at assessing personality traits of others in general, but there appears to be a difference in 1st impression judgments between older \& younger adults. Older adults judged young adult target photos as healthier, more trustworthy, \& less hostile, but more aggressive, than younger adults did of the same photos. Older adults could have a lower response to negative cues due to a slower processing speed, causing them to see facial features on young adults as more positive than younger adults do.'' -- \href{https://en.wikipedia.org/wiki/First_impression_(psychology)#Speed_and_accuracy}{Wikipedia\texttt{/}1st impression (psychology)\texttt{/}speed \& accuracy}

\subsection{Number of observers}
``One's 1st impressions are affected by whether they're alone or with any number of people. Joint experiences are more globally processed (see \href{https://en.wikipedia.org/wiki/Global_precedence}{global precedence} for more on processing), as in collectivist cultures. Global processing emphasizes 1st impressions more because the collective 1st impression tends to remain stable over time. Solo experiences tend to facilitate local processing, causing the viewer to take a more critical look at the target. Thus, individuals are more likely to have negative 1st impressions than groups of 2 or more viewers of the same target. At the same time, individuals are more likely to experience an upward trend over the course of a series of impressions, e.g. individual viewers will like the final episode of a TV season more than the 1st even if it is really the same quality.

When viewing pieces of art in an experiment, participants in a solo context rated art in an improving sequence significantly higher than when the targets are presented in a declining sequence. When viewing the art in a joint context, participants evaluated the 1st \& last pieces similarly in both kinds of sequence. Simply priming viewers to feel like they were in solo or joint contexts or to process analytically or holistically was enough to produce the same viewing effects.'' -- \href{https://en.wikipedia.org/wiki/First_impression_(psychology)#Number_of_observers}{Wikipedia\texttt{/}1st impression (psychology)\texttt{/}number of observers}

\subsection{Cultural influences}

\subsubsection{Individualisms vs. collectivism}
``Similar to the number of viewers present, \href{https://en.wikipedia.org/wiki/Collectivism_and_individualism}{collectivism} vs. \href{https://en.wikipedia.org/wiki/Individualism}{individualism} can influence impression formation. Collectivists are at ease as long as their impressions are largely in alignment with the larger group's impressions. When a collectivist wants to change their impression, they may be compelled to change the views of all group members. However, this could be challenging for collectivists, who tend to be less confrontational than individualists. Individualists are willing to change their own views at will \& are generally more comfortable with uncertainty, which makes them naturally more willing to change their impressions.'' -- \href{https://en.wikipedia.org/wiki/First_impression_(psychology)#Individualism_versus_collectivism}{Wikipedia\texttt{/}1st impression (psychology)\texttt{/}cultural influences\texttt{/}individualism vs. collectivism}

\subsubsection{Influence of media richness}
``There is no research regarding if national culture mediates the relationship between media richness \& bias in impression formation. Some studies that manipulated media richness have found that information presented in  text form yields similar impressions (measured by reported appraisal scores) among cultures, while other studies found that richer forms of information such as videos reduce cross-cultural bias more effectively. The latter findings support \href{https://en.wikipedia.org/wiki/Media_Richness_Theory}{Media Richness Theory}.'' -- \href{https://en.wikipedia.org/wiki/First_impression_(psychology)#Influence_of_media_richness}{Wikipedia\texttt{/}1st impression (psychology)\texttt{/}cultural influences\texttt{/}influence of media richness}

\subsubsection{Accents \& speech}
``Accents \& unique speech patterns can influence how people are perceived by those to whom they are speaking. E.g., when hypothetically interviewing an applicant with a Midwestern U.S. accent, Colombian accent, or French accent, Midwestern U.S. participants evaluated the U.S. accent as significantly more positive than the applicant with the French accent due to perceived similarity to themselves. The evaluation of the applicant with the Colombian accent did not, however, differ significantly from the other 2. 1st impressions can be heavily influenced by a similarity-attraction hypothesis where others are immediately put into ``similar'' or ``dissimilar'' categories from the viewer \& judged accordingly.'' -- \href{https://en.wikipedia.org/wiki/First_impression_(psychology)#Accents_and_speech}{Wikipedia\texttt{/}1st impression (psychology)\texttt{/}cultural influences\texttt{/}accents \& speech}

\subsubsection{Physical characteristics \& personality}
``Although populations from different cultures can be quick to view others as dissimilar, there are several 1st impression characteristics that are universal across cultures. When comparing trait impressions of faces among U.S. \& the culturally isolated Tsimane' people of Bolivia, there was between-culture agreement when ascribing certain physical features to descriptive traits such as attractiveness, intelligence, health, \& warmth. Both cultures also show a strong attractiveness halo when forming impressions, meaning that those seen as attractive were also rated as more competent, sociable, intelligent, \& healthy.'' -- \href{https://en.wikipedia.org/wiki/First_impression_(psychology)#Physical_characteristics_and_personality}{Wikipedia\texttt{/}1st impression (psychology)\texttt{/}cultural influences\texttt{/}physical characteristics \& personality}

\subsection{Physical appearance}

\subsubsection{Faces \& features}
``Physical appearance gives us clear clues as to a person's personality without them ever having to speak or move. Women tend to be better than men at judging nonverbal behavior. After viewing pictures of people in a neutral position \& in a self-chosen posed position, observers were accurate at judging the target's levels of \href{https://en.wikipedia.org/wiki/Extraversion}{extraversion}, emotional stability, openness, \href{https://en.wikipedia.org/wiki/Self-esteem}{self-esteem}, \& religiosity. The combined impression of physical characteristics, body posture, facial expression, \& clothing choices lets observers form accurate images of a target's personality, so long as the person observed is presenting themselves genuinely. However, there is some conflicting data in this field. Other evidence suggests that people sometimes rely too much on appearances cues over actual information. When provided with descriptive information about a target, participants still reply on physical appearance cues when making judgments about others' personalities \& capabilities. Participants struggle to look past physical appearance cues even when they know information contrary to their initial judgment. Physical cues are also used to make judgments about political candidates based on extremely brief exposures to their pictures. Perceived competence level of a candidate measured from 1st impressions of facial features can directly predict voting results.

The ``beautiful is good'' effect is a very present phenomenon when dealing with 1st impressions of others. Targets who are attractive are rated more positively \& as possessing more unique characteristics than those who are unattractive. Beauty is also found to be somewhat subjective so that even targets who are not universally attractive can receive the benefit of this effect if the observer is attracted to them.

In a 2014 study, a group at the \href{https://en.wikipedia.org/wiki/University_of_York}{University of York} reported that impressions of the traits of approachability, youthfulness\texttt{/}attractiveness \& dominance can be formed from measurable characteristics such as the shape of \& the spacing around the eyes, nose, \& mouth. It was found that 1st impressions of social traits, such as trustworthiness or dominance, are reliably perceived in faces. Physical facial features were objectively measured from feature positions \& colors. A neural network was used to model the dimensions of approachabbility, youthful-attractiveness, \& dominance. 58\% of the variance in raters' impressions was accounted for by this linear model.'' -- \href{https://en.wikipedia.org/wiki/First_impression_(psychology)#Faces_and_features}{Wikipedia\texttt{/}1st impression (psychology)\texttt{/}physical apperance\texttt{/}faces \& features}

\subsubsection{Apparel \& cosmetics}
``Cosmetic use is also an important cue for forming impressions, particularly of women. Those wearing heavy makeup are seen as significantly more feminine than those wearing moderate makeup or no makeup \& those wearing heavy or moderate makeup are seen as more attractive than those wearing no makeup. While a woman wearing no makeup is perceived as being more moral than the other 2 conditions, there is no difference between experimental conditions when judging personality or personal temperament.

1st impression formation can be influenced by the use of cognitive short hands such as \href{https://en.wikipedia.org/wiki/Stereotypes}{stereotypes} \& \href{https://en.wikipedia.org/wiki/Representative_heuristic}{representative heuristics}. When asked to rate the \href{https://en.wikipedia.org/wiki/Socioeconomic_status}{socioeconomic status} (SES) \& degree of interest in friendship with African American \& Caucasian female models wearing either a \href{https://en.wikipedia.org/wiki/Kmart_(United_States)}{K-Mart}, \href{https://en.wikipedia.org/wiki/Abercrombie_%26_Fitch}{Abercrombie \& Fitch}, or non-logoed sweatshirt, Caucasian models were rated more favorably than the African American models. Abercrombie \& Fitch wearers were rated as higher SES than the other sweatshirts. Participants wanted to be friends with the Caucasian model most when she was wearing a plain sweatshirt \& the African American model most when she was wearing either the plain or K-Mart sweatshirt. It is unclear why the plain sweatshirt was most associated with friendship, but the general results suggest that mismatching class \& race reduced the model's friendship appeal.'' -- \href{https://en.wikipedia.org/wiki/First_impression_(psychology)#Apparel_and_cosmetics}{Wikipedia\texttt{/}1st impression (psychology)\texttt{/}physical apperance\texttt{/}apparel \& cosmetics}

\subsection{Specific contexts}

\subsubsection{Online}
``Online profiles \& communication channels such as email provide fewer cues than in-person interactions, which makes targets more difficult to understand. When research participants were asked to evaluate a person's facial attractiveness \& perceived ambition based on an online dating profile, amount of time permitted for processing \& reporting an evaluation of the target produced a difference in impression formation. Spontaneous evaluations relied on physical attractiveness almost exclusively, whereas deliberate evaluations weighed both types of information. Although deliberate evaluations used the information provided on both physical attractiveness \& ambition of each target, the particular impact of each kind of information appeared to depend on the consistency between the 2. A significant effect of attractiveness on deliberate evaluations was found only when perceived ambition was consistent with the perceived level of attractiveness. The consistency found in profiles seemed to particularly influence deliberate evaluations.

In a study of online impressions, participants who were socially expressive \& disclosed a lot about themselves both on their webpages \& in person were better liked than those who were less open. Social expressivity includes liveliness in voice, smiling, etc.'' -- \href{https://en.wikipedia.org/wiki/First_impression_(psychology)#Online}{Wikipedia\texttt{/}1st impression (psychology)\texttt{/}specific contexts\texttt{/}online}

\subsubsection{Dating \& sexuality}
``Upon seeing photographs of straight, gay, \& bisexual people, participants correctly identified gay vs. straight males \& females at above-chance levels based solely on seeing a picture of their face, however, bisexual targets were only identified at chance. The findings suggest a straight-non straight dichotomy in the categorization of sexual orientation.

The more time participants are allowed to make some judgment about a person, the more they will weigh information beyond physical appearance. Specific manipulations include identifying men as gay vs. straight \& people as trustworthy or not. In a study of the interaction between ratings of people in speed dating \& the form of media used to present them, impression accuracy in a speed dating task was not significantly different when a potential date was presented in person vs. in a video. However, impressions of dates made via video were to be much more negative than those made in person. An additional study that looked at characterization of a romantic partner suggested that people are more likely to rely on ``gut reactions'' when meeting in person, but there isn't sufficient information for this kind of evaluation when viewing someone online.'' -- \href{https://en.wikipedia.org/wiki/First_impression_(psychology)#Dating_and_sexuality}{Wikipedia\texttt{/}1st impression (psychology)\texttt{/}specific contexts\texttt{/}dating \& sexuality}

\subsubsection{Professional}
``Non-verbal behaviors are particularly important to forming 1st impressions when meeting a business acquaintance. Specifically, components of social expressivity, such as smiling, eyebrow position, emotional expression, \& eye contact are emphasized. Straightening one's posture, leaning in slightly, \& giving a firm handshake promotes favorable impression formation in the American business context. Other impression management tactics in the business world include researching the organization \& interviewers beforehand, preparing specific questions for the interviewer, showing confidence, \& dressing appropriately.

A qualitative review of previous literature looking at self-report data suggests that men \& women use \href{https://en.wikipedia.org/wiki/Impression_management}{impression management} tactics in the corporate world that are consistent with stereotypical gender roles when presenting themselves to others. This research proposes that women are put in a double bind where those who portray themselves as more communal \& submissive are overlooked for leadership positions \& women who try to utilize male tactics (such as being more aggressive) receive negative consequences for violating normative gender roles. To change this dynamic the authors suggest that managerial positions should be re-advertised to highlight the feminine qualities needed for a position \& staff training should involve a segment accentuating gender issues in the office to make everyone aware of possible discrimination.

Data collected from interviews with physicians distinguishes between 1st impressions \& intuition \& contributes to understanding the occurrence of gut feelings in the medical field. Gut feelings go beyond 1st impressions: Physicians expressed feeling doubtful about their initial impressions as they gathered more data from their patients. More experienced physicians reported more instances of gut feelings than those less experienced, but the quality of the intuition was related to the quality of feedback received during the data collection process in general. Emotional engagement enhanced learning just as it does in 1st impressions.'' -- \href{https://en.wikipedia.org/wiki/First_impression_(psychology)#Professional}{Wikipedia\texttt{/}1st impression (psychology)\texttt{/}specific contexts\texttt{/}professional}

\subsection{Neuroscience}
``1st impressions are formed within milliseconds of seeing a target. When intentionally forming a 1st impression, encoding relies on the dorsomedial \href{https://en.wikipedia.org/wiki/Prefrontal_cortex}{prefrontal cortext} (dmPFC). Readings from \href{https://en.wikipedia.org/wiki/FMRI}{fMRIs} of research participants show that processing of diagnostic information (e.g. distinguishing features) engaged the dmPFC more than processing neutral information.

Participants generally formed more negative impressions of the faces that showed a negative emotion compared to neutral faces. results suggest that the dmPFC \& \href{https://en.wikipedia.org/wiki/Amygdala} together play a large role in negative impression formation. When forming immediate impressions based on emotion, the stimulus can bypass the \href{https://en.wikipedia.org/wiki/Neo-cortex}{neo-cortex} by way of the ``\href{https://en.wikipedia.org/wiki/Amygdala_hijack}{amygdala hijack}.'' -- \href{https://en.wikipedia.org/wiki/First_impression_(psychology)#Neuroscience}{Wikipedia\texttt{/}1st impression (psychology)\texttt{/}neuroscience}

\subsubsection{Familiarity}
``Research indicates that people are efficient evaluators when forming impressions based on existing biases. The \href{https://en.wikipedia.org/wiki/Posterior_cingulate_cortex}{posterior cingulate cortex} (PCC), amygdala, \& the thalamus sort relevant vs. irrelevant information according to these biases. The dmPFC is also involved in the impression formation process, especially with person-descriptive information.

FMRI results show activation of the fusiform cortex, \href{https://en.wikipedia.org/wiki/Posterior_cingulate_gyrus}{posterior cingulate gyrus}, \& amygdala when individuals are asked to identify previously seen faces that were encoded as either ``friends'' or ``foes.'' Additionally, the caudate \& \href{https://en.wikipedia.org/wiki/Anterior_cingulate_cortex}{anterior cingulate cortex} are more activated when looking at faces of ``foes'' vs. ``friends.'' This research suggests that quick 1st impressions of hostility or support from unknown people can lead to long-term effects on memory that will later be associated with that person.'' -- \href{https://en.wikipedia.org/wiki/First_impression_(psychology)#Familiarity}{Wikipedia\texttt{/}1st impression (psychology)\texttt{/}neuroscience\texttt{/}familiarity}

\subsubsection{Alcohol \& impressions}
``\href{https://en.wikipedia.org/wiki/Alcohol_(drug)}{Alcohol} consumption \& belief of consumption influenced \href{https://en.wikipedia.org/wiki/Emotion}{emotion} detection in 10 second clips. Participants who thought they had consumed an alcoholic beverage rated 1 facial expression (approximately 3\% of the facial expressions they saw) more in each clip as happy compared to the control group. Thus, impression formation may be affected by even the perception of alcohol consumption.'' -- \href{https://en.wikipedia.org/wiki/First_impression_(psychology)#Alcohol_and_Impressions}{Wikipedia\texttt{/}1st impression (psychology)\texttt{/}alcohol \& impressions}

\subsubsection{Cross-cultural differences}
``There appears to be cross-cultural similarities in brain responses to 1st impression formations. In a \href{https://en.wikipedia.org/wiki/Mock_election}{mock election} both American \& Japanese individuals voted for the candidate that elicited a stronger response in their bilateral amygdala than those who did not, regardless of the candidate's culture. Individuals also showed a stronger response to cultural outgroup faces than cultural \href{https://en.wikipedia.org/wiki/Ingroup}{ingroup} faces because the amygdala is presumably more sensitive to novel stimuli. However, this finding was unrelated to actual voting decisions.'' -- \href{https://en.wikipedia.org/wiki/First_impression_(psychology)#Cross-cultural_differences}{Wikipedia\texttt{/}1st impression (psychology)\texttt{/}cross-cultural differences}

\subsection{Stability}
``\fbox{Once formed, 1st impressions tend to be stable.} A review of the literature on the accuracy \& impact of 1st impressions on rater-based assessments found that raters' 1st impressions are highly correlated with later scores, but it is unclear exactly why. 1 study tested stability by asking participants to form impressions people based purely on \href{https://en.wikipedia.org/wiki/Photographs}{photographs}. Participants' opinions of the people in photographs did not significantly differ after interacting with that person a month later. 1 potential reason for this stability is that one's 1st impressions could serve as a guide for their next steps, such as what questions are asked \& how raters go about scoring. More research needs to be done on the stability of 1st impressions to fully understand how 1st impressions guide subsequent treatment, \href{https://en.wikipedia.org/wiki/Self-fulfilling_prophecies}{self-fulfilling prophecies}, \& the \href{https://en.wikipedia.org/wiki/Halo_effect}{halo effect}. Assessment tools can influence impressions too, e.g. if a question provides only a dichotomous ``yes'' or ``no'' response or if a rater uses a \href{https://en.wikipedia.org/wiki/Scale_(ratio)}{scale (ratio)}. Although this study was conducted with the intention of improving rating methods in medical education, the literature review was sufficiently broad enough to generalize.'' -- \href{https://en.wikipedia.org/wiki/First_impression_(psychology)#Stability}{Wikipedia\texttt{/}1st impression (psychology)\texttt{/}stability}

%------------------------------------------------------------------------------%

\section{\href{https://en.wikipedia.org/wiki/Aggression}{Wikipedia\texttt{/}Aggression}}
\textsf{Fig. 2 \href{https://en.wikipedia.org/wiki/Warthog}{warthogs} preparing to fight.}

``\textit{Aggression} is overt or covert, often harmful, social interaction with the intention of inflicting damage or other harm upon another individual; although it can be channeled into creative \& practical outlets for some. It may occur either reactively or without provocation. In humans, aggression can be caused by various triggers, from \href{https://en.wikipedia.org/wiki/Frustration}{frustration} due to blocked goals to feeling disrespected. Human aggression can be classified into direct \& indirect aggression; whilst the former is characterized by physical or verbal behavior intended to cause harm to someone, the latter is characterized by behavior intended to harm the social relations of an individual or group.

In definitions commonly used in the \href{https://en.wikipedia.org/wiki/Social_sciences}{social sciences} \& \href{https://en.wikipedia.org/wiki/Behavioral_science}{behavioral sciences}, aggression is an action or response by an individual that delivers something unpleasant to another person. Some definitions include that the individual must intend to harm another person.

In an interdisciplinary perspective, aggression is regarded as ``an ensemble of mechanism formed during the course of evolution in order to assert oneself, relatives or friends against others, to gain or to defend resources (ultimate causes) by harmful damaging means $\ldots$ These mechanisms are often motivated by emotions like fear, frustration, anger, feelings of stress, dominance or pleasure (proximate causes) $\ldots$ Sometimes aggressive behavior serves as a stress relief or a subjective feeling or power.'' \href{https://en.wikipedia.org/wiki/Predation}{Predatory} or defensive behavior between members of different species may not be considered aggression in the same sense.

Aggression can take a variety of forms, which may be expressed physically, or communicated \href{https://en.wikipedia.org/wiki/Verbal_aggressiveness}{verbally} or non-verbally: including anti-predator aggression, defensive aggression (fear-included), predatory aggression, dominance aggression, inter-male aggression, resident-intruder aggression, maternal aggression, species-specific aggression, sex-related aggression, territorial aggression, isolation-induced aggression, irritable aggression, \& brain-stimulation-induced aggression (hypothalamus). There are 2 subtypes of human aggression:
\begin{enumerate}
	\item controlled-instrumental subtype (purposeful or goal-oriented); \&
	\item reactive-impulsive subtype (often elicits uncontrollable actions that are inappropriate or undesirable).
\end{enumerate}
Aggression differs from what is commonly called \href{https://en.wikipedia.org/wiki/Assertiveness}{assertiveness}, although the terms are often used interchangeably among laypeople (as in phrases such as ``an aggressive salesperson'').'' -- \href{https://en.wikipedia.org/wiki/Aggression}{Wikipedia\texttt{/}aggression}

\subsection{Overview}
``\href{https://en.wikipedia.org/wiki/John_Dollard}{Dollard} et al. (1939) proposed that \fbox{aggression was due to \href{https://en.wikipedia.org/wiki/Frustration}{frustration}}, which was described as an unpleasant emotion resulting from any interference with achieving a rewarding goal. \href{https://en.wikipedia.org/wiki/Leonard\_Berkowitz}{Berkowitz} extended this \href{https://en.wikipedia.org/wiki/Frustration%E2%80%93aggression_hypothesis}{frustration--aggression hypothesis} \& proposed that it is not so much the frustration as the unpleasant emotion that evokes aggressive tendencies, \& that all aversive events produce negative \href{https://en.wikipedia.org/wiki/Affect_(psychology)}{affect} \& thereby aggressive tendencies, as well as \href{https://en.wikipedia.org/wiki/Fear}{fear} tendencies. Besides \href{https://en.wikipedia.org/wiki/Classical_conditioning}{conditioned} stimulie, Archer categorized aggression-evoking (as well as fear-evoking) stimuli into 3 groups; namely, \href{https://en.wikipedia.org/wiki/Pain}{pain}, \href{https://en.wikipedia.org/wiki/Novelty}{novelty}, \& frustration, although he also described ``\href{https://en.wikipedia.org/wiki/Looming}{looming},'' which refers to an object rapidly moving towards the visual sensors of a subject, \& can be categorized as ``\href{https://en.wikipedia.org/wiki/Intensity_(physics)}{intensity}.''

Aggression can have adaptive benefits or negative effects. Aggressive behavior is an individual or collective social interaction that is a hostile \href{https://en.wikipedia.org/wiki/Behavior}{behavior} with the intention of inflicting damage or harm. 2 broad categories of aggression are commonly distinguished. One includes \href{https://en.wikipedia.org/wiki/Affective}{affect} (emotional) \& hostile, reactive, or \href{https://en.wikipedia.org/wiki/Revenge}{retaliatory} aggression that is a response to provocation, \& the other includes instrumental, goal-oriented or \href{https://en.wikipedia.org/wiki/Predatory}{predatory}, in which aggression is used as a means to achieve a goal. An example of hostile aggression would be a person who punches someone who insulted him or her. An instrumental form of aggression would be \href{https://en.wikipedia.org/wiki/Armed_robbery}{armed robbery}. Research on \href{https://en.wikipedia.org/wiki/Violence}{violence} from a range of disciplines lend some support to a distinction between affective \& predatory aggression. However, some researchers question the usefulness of a hostile versus instrumental distinction in humans, despite its ubiquity in research, because most real-life cases involve mixed motives \&  interacting causes.

A number of classifications \& dimensions of aggression have been suggested. These depend on such things as whether the aggression is verbal or physical; whether or not it involves \href{https://en.wikipedia.org/wiki/Relational_aggression}{relational aggression} such as covert bullying \& social manipulation; whether harm to others is intended or not; whether it is carried out actively or expressed passively; \& whether the aggression is aimed directly or indirectly. Classification may also encompass aggression-related emotions (e.g. \href{https://en.wikipedia.org/wiki/Anger}{anger}) \& mental states (e.g. \href{https://en.wikipedia.org/wiki/Impulsivity}{impulsivity}, \href{https://en.wikipedia.org/wiki/Hostility}{hostility}). Aggression may occur in response to non-social as well as social factors, \& can have a close relationship with stress coping style. Aggression may be \href{https://en.wikipedia.org/wiki/Threat_display}{displayed} in order to \href{https://en.wikipedia.org/wiki/Intimidate}{intimidate}.

The operative definition of aggression may be affected by \href{https://en.wikipedia.org/wiki/Morality}{moral} or \href{https://en.wikipedia.org/wiki/Politics}{political} views. Examples are the axiomatic moral view called the \href{https://en.wikipedia.org/wiki/Non-aggression_principle}{non-aggression principle} \& the political rules governing the behavior of 1 country toward another. Likewise in competitive \href{https://en.wikipedia.org/wiki/Sports}{sports}, or in the \href{https://en.wikipedia.org/wiki/Workplace}{workplace}, some forms of aggression may be sanctioned \& others not (see \href{https://en.wikipedia.org/wiki/Workplace_aggression}{Workplace aggression}). Aggressive behaviors are associated with adjustment problems \& several psychopathological symptoms such as \href{https://en.wikipedia.org/wiki/Antisocial_Personality_Disorder}{Antisocial Personality Disorder}, \href{https://en.wikipedia.org/wiki/Borderline_Personality_Disorder}{Borderline Personality Disorder}, \& \href{https://en.wikipedia.org/wiki/Intermittent_Explosive_Disorder}{Intermittent Explosive Disorder}.

Biological approaches conceptualize aggression as an internal energy released by external stimuli, a product of evolution through natural selection, part of genetics, a product of hormonal fluctuations. Psychological approaches conceptualize aggression as a destructive instinct, a response to frustration, an affect excited by a negative stimulus, a result of observed learning of society \& diversified reinforcement, a resultant of variables that affect personal \& situational environments.'' -- \href{https://en.wikipedia.org/wiki/Aggression#Overview}{Wikipedia\texttt{/}aggression\texttt{/}overview}

\subsection{Etymology}
``The term aggression comes from the \href{https://en.wikipedia.org/wiki/Latin}{Latin} word \textit{aggressio}, meaning attack. The Latin was itself a joining of \textit{ad-} \& \textit{gradi-}, which meant step at. The 1st known use dates back to 1611, in the sense of un unprovoked attack. A psychological sense of ``hostile or destructive behavior'' dates back to a 1912 English translation of \href{https://en.wikipedia.org/wiki/Sigmund_Freud}{Sigmund Freud}'s writing. \href{https://en.wikipedia.org/wiki/Alfred_Adler}{Alfred Adler} theorized about an ``aggressive drive'' in 1908. \href{https://en.wikipedia.org/wiki/Parenting}{Child raising} experts began to refer to aggression, rather than anger, from the 1930s.'' -- \href{https://en.wikipedia.org/wiki/Aggression#Etymology}{Wikipedia\texttt{/}aggression\texttt{/}etymology}

\subsection{Ethology}
\textsf{Male \href{https://en.wikipedia.org/wiki/Elephant_seal}{element seals} fighting.}

``\href{https://en.wikipedia.org/wiki/Ethology}{Ethologists} study aggression as it relates to the interaction \& \href{https://en.wikipedia.org/wiki/Evolution}{evolution} of animals in natural settings. In such settings aggression can involve bodily contact such as biting, hitting or pushing, but most conflicts are settled by threat displays \& intimidating thrusts that cause no physical harm. This form of aggression may include the display of body size, antlers, claws or teeth; stereotyped signals including facial expressions; vocalizations such as bird song; the release of chemicals; \& changes in coloration. The term \href{https://en.wikipedia.org/wiki/Agonistic_behaviour}{agonistic behavior} is sometimes used to refer to these forms of behavior.

Most ethologists believe that aggression confers biological advantages. Aggression may help an animal secure \href{https://en.wikipedia.org/wiki/Territory_(animal)}{territory}, including resources such as food \& water. Aggression between males often occurs to secure mating opportunities, \& results in selection of the healthier\texttt{/}more vigorous animal. Aggression may also occur for self-protection or to protect offspring. Aggression between groups of animals may also confer advantage; e.g., hostile behavior may force a population of animals into a new territory, where the need to adapt to a new environment may lead to an increase in genetic flexibility.'' -- \href{https://en.wikipedia.org/wiki/Aggression#Ethology}{Wikipedia\texttt{/}aggression\texttt{/}ethology}

\subsubsection{Between species \& groups}
``The most apparent type of \href{https://en.wikipedia.org/wiki/Interspecific}{interspecific} aggression is that observed in the interaction between a \href{https://en.wikipedia.org/wiki/Predator}{predator} \& its \href{https://en.wikipedia.org/wiki/Prey}{prey}. However, according to many researchers, \href{https://en.wikipedia.org/wiki/Predation}{predation} is not aggression. A cat does not hiss or arch its back when pursuing a rat, \& the active areas in its \href{https://en.wikipedia.org/wiki/Hypothalamus}{hypothalamus} resemble those that reflect hunger rather than those that reflect aggression. However, others refer to this behavior as predatory aggression, \& point out cases that resemble hostile behavior, such as mouse-killing by rats. In \href{https://en.wikipedia.org/wiki/Aggressive_mimicry}{agggression mimicry} a predator has the appearance of a harmless organism or object attractive to the prey; when the prey approaches, the predator attacks.

An animal defending against a predator may engage in either ``\href{https://en.wikipedia.org/wiki/Fight-or-flight_response}{fight or flight}'' or ``\href{https://en.wikipedia.org/wiki/Tend_and_befriend}{tend \& befriend}'' in response to predator attack or threat of attack, depending on its estimate of the predator's strength relative to its own. Alternative defenses include a range of \href{https://en.wikipedia.org/wiki/Antipredator_adaptation}{antipredator adaptations}, including \href{https://en.wikipedia.org/wiki/Alarm_signal}{alarm signals}. An example of an alarm signal is nerol, a chemical which is found in the mandibular glands of \href{https://en.wikipedia.org/wiki/Trigona_fulviventris}{\textit{Trigona fulviventris}} individuals. Release of nerol by T. fulviventris individuals in the nest has been shown to decrease the number o individuals leaving the nest by 50\%, as well as increasing aggressive behaviors like biting. Alarm signals like nerol can also act as attraction signals; in T. fulviventris, individuals that have been captured by a predator may release nerol to attract nestmates, who will proceed to attack or bite the predator.

Aggression between groups is determined partly by willingness to fight, which depends on a number of factors including numerical advantage, distance from home territories, how often the groups encounter each other, competitive abilities, differences in body size, \& whose territory is being invaded. Also, an individual is more likely to become aggressive if other aggressive group members are nearby. 1 particular phenomenon -- the formation of coordinated coalitions that raid neighboring territories to kill \href{https://en.wikipedia.org/wiki/Conspecific}{conspecifics} -- has only been documented in 2 species in the animal kingdom: \href{https://en.wikipedia.org/wiki/Common_chimpanzee}{`common' chimpanzees} \& \href{https://en.wikipedia.org/wiki/Humans}{humans}.'' -- \href{https://en.wikipedia.org/wiki/Aggression#Between_species_and_groups}{Wikipedia\texttt{/}aggression\texttt{/}ethology\texttt{/}between species \& groups}

\subsubsection{Within a group}
``Aggression between conspecifics in a group typically involves access to resources \& breeding opportunities. 1 of its most common functions is to establish a \href{https://en.wikipedia.org/wiki/Dominance_hierarchy}{dominance hierarchy}. This occurs in many species by aggressive encounters between contending males when they are 1st together in a common environment. Usually the more aggressive animals become the more dominant. In test situations, most of the conspecific aggression ceases about 24 hours after the group of animals is brought together. Aggression has been defined from this viewpoint as ``behavior which is intended to increase the social dominance of the organism relative to the dominance position of other organisms''. Losing confrontations may be called \href{https://en.wikipedia.org/wiki/Social_defeat}{social defeat}, \& winning or losing is associated with a range of practical \& psychological consequences.

Conflicts between animals occur in many contexts, such as between potential mating partners, between parents \& offspring, between siblings \& between competitors for resources. Group-living animals may dispute over the direction of travel or the allocation of time to joint activities. Various factors limit the escalation of aggression, including communicative displays, conventions, \& routines. In addition, following aggressive incidents, various forms of \href{https://en.wikipedia.org/wiki/Conflict_resolution}{conflict resolution} have been observed in mammalian species, particularly in gregarious primates. These can mitigate or repair possible adverse consequences, especially for the recipient of aggression who may become vulnerable to attacks by other members of a group. Conciliatory acts vary by species \& may involve specific gestures or simply more proximity \& interaction between the individuals involved. However, conflicts over food are rarely followed by post conflicts reunions, even though they are the most frequent type in foraging primates.

Other questions that have been considered in the study of primate aggression, including in humans, is how aggression affects the organization of a group, what costs are incurred by aggression, \& why some primates avoid aggressive behavior. E.g., \href{https://en.wikipedia.org/wiki/Bonobo}{bonobo chimpanzee} groups are known for low levels of aggression within a partially \href{https://en.wikipedia.org/wiki/Matriarchal}{matriarchal} society. \href{https://en.wikipedia.org/wiki/Captivity_(animal)}{Captive} animals including primates may show abnormal levels of social aggression \& self-harm that are related to aspects of the physical or social environment; this depends on the species \& individual factors such as gender, age \& background (e.g. raised wild or captive).'' -- \href{https://en.wikipedia.org/wiki/Aggression#Within_a_group}{Wikipedia\texttt{/}aggression\texttt{/}ethology\texttt{/}within a group}

\subsubsection{Aggression, fear \& curiosity}
``Within ethology, it has long been recognized that there is a relation between aggression, \href{https://en.wikipedia.org/wiki/Fear}{fear}, \& \href{https://en.wikipedia.org/wiki/Curiosity}{curiosity}. A \href{https://en.wikipedia.org/wiki/Cognitive_science}{cognitive} approach to this relationship puts aggression in the broader context of \href{https://en.wikipedia.org/wiki/Cognitive_dissonance}{inconsistency reduction}, \& proposes that aggressive behavior is caused by an inconsistency between a desired, or expected, situation \& the actually perceived situation (e.g., ``\href{https://en.wikipedia.org/wiki/Frustration%E2%80%93aggression_hypothesis}{frustration}''), \& functions to forcefully manipulate the perception into matching the expected situation. In this approach, when the inconsistency between perception \& expectancy is small, learning as a result of curiosity reduces inconsistency by updating expectancy to match perception. If the inconsistency is larger, fear or aggressive behavior may be employed to alter the perception in order to make it match expectancy, depending on the size of the inconsistency as well as the specific context. Uninhibited fear results in fleeing, thereby removing the inconsistent stimulus from the perceptual field \& resolving the inconsistency. In some cases thwarted escape may trigger aggressive behavior in an attempt to remove the thwarting stimulus.'' -- \href{https://en.wikipedia.org/wiki/Aggression#Aggression,_fear_and_curiosity}{Wikipedia\texttt{/}aggression\texttt{/}ethology\texttt{/}aggression, fear \& curiosity}

\subsection{Evolutionary explanations}
``Like many behaviors, aggression can be examined in terms of its ability to help an animal itself survive \& reproduced, or alternatively to risk survival \& reproduction. This \href{https://en.wikipedia.org/wiki/Cost-benefit_analysis}{cost-benefit analysis} can be looked at in terms of \href{https://en.wikipedia.org/wiki/Evolution}{evolution}. However, there are profound differences in the extent of acceptance of a biological or evolutionary basis for human aggression.

According to the \href{https://en.wikipedia.org/wiki/Male_warrior_hypothesis}{male warrior hypothesis}, intergroup aggression represents an opportunity for men to gain access to mates, territory, resources \& increased status. As such, conflicts may have created selection evolutionary pressures for psychological mechanisms in men to initiate intergroup aggression.'' -- \href{https://en.wikipedia.org/wiki/Aggression#Evolutionary_explanations}{Wikipedia\texttt{/}aggression\texttt{/}evolutionary explanations}

\subsubsection{Violence \& conflict}
``Aggression can involve \href{https://en.wikipedia.org/wiki/Violence}{violence} that may be \href{https://en.wikipedia.org/wiki/Adaptation}{adaptive} under certain circumstances in terms of \href{https://en.wikipedia.org/wiki/Natural_selection}{natural selection}. This is most obviously the case in terms of attacking prey to obtain food, or in anti-predatory defense. It may also be the case in competition between members of the same species or subgroup, if the average reward (e.g. status, access to resources, protection of self or kin) outweighs average costs (e.g. injury, exclusion from the group, death). There are some hypotheses of specific adaptations for violence in humans under certain circumstances, including for \href{https://en.wikipedia.org/wiki/Homicide}{homicide}, but it is often unclear what behaviors may have been selected for \& what may have been a byproduct, as in the case of collective violence.

Although aggressive encounters are ubiquitous in the animal kingdom, with often high stakes, most encounters that involve aggression may be resolved through posturing, or displaying \& trial of strength. \href{https://en.wikipedia.org/wiki/Game_theory}{Game theory} is used to understand how such behaviors might spread by \href{https://en.wikipedia.org/wiki/Natural_selection}{natural selection} within a population, \& potentially become `Evolutionary Stable Strategies'. An initial model of resolution of conflicts is the \href{https://en.wikipedia.org/wiki/Hawk-dove_game}{hawk-dove game}. Others include the \href{https://en.wikipedia.org/wiki/Risk_management}{Sequential assessment model} \& the \href{https://en.wikipedia.org/wiki/Attrition_warfare}{Energetic war of attrition}. These try to understand not just 1-off encounters but protracted stand-offs, \& mainly differ in the criteria by which an individual decides to give up rather than risk loss \& harm in physical conflict (such as through estimates of \href{https://en.wikipedia.org/wiki/Resource_holding_potential}{resource holding potential}).'' -- \href{https://en.wikipedia.org/wiki/Aggression#Violence_and_conflict}{Wikipedia\texttt{/}aggression\texttt{/}evolutionary explanations\texttt{/}violence \& conflict}

\subsubsection{Gender}

\paragraph{General.} ``Gender plays an important role in human aggression. There are multiple theories that seek to explain findings that males \& females of the same species can have differing aggressive behaviors. One review concluded that male aggression tended to produce pain or physical injury whereas female aggression tended towards psychological or social harm.

In general, \href{https://en.wikipedia.org/wiki/Sexual_dimorphism}{sex dimorphism} can be attributed to greater \href{https://en.wikipedia.org/wiki/Intraspecific_competition}{instraspecific competition} in 1 sex, either between rivals for access to mates \&\texttt{/}or to be \href{https://en.wikipedia.org/wiki/Mate_choice}{chosen by mates}. This may stem from the other gender being constrained by providing greater \href{https://en.wikipedia.org/wiki/Parental_investment}{parental investment}, in terms of factors such as \href{https://en.wikipedia.org/wiki/Gamete}{gamete} production, \href{https://en.wikipedia.org/wiki/Gestation}{gestation}, \href{https://en.wikipedia.org/wiki/Lactation}{lactation}, or upbringing of young. Although there is much variation in species, generally the more physically aggressive sex is the male, particularly in mammals. In species where parental care by both sexes is required, there tends to be less of a difference. When the female can leave the male to care for the offspring, then females may be the larger \& more physically aggressive. Competitiveness despite parental investment has also been observed in some species. A related factor is the rate at which males \& females are able to mate again after producing offspring, \& the basic principles of \href{https://en.wikipedia.org/wiki/Sexual_selection}{sexual selection} are also influenced by ecological factors affecting the ways or extent to which 1 sex can compete for the other. The role of such factors in human evolution is controversial.

The pattern of male \& female aggression is argued to be consistent with evolved sexually-selected behavioral differences, while alternative or complementary views emphasize conventional \href{https://en.wikipedia.org/wiki/Gender_role}{social roles} stemming from physical evolved differences. Aggression in women may have evolved to be, on average, less physically dangerous \& more covert or \href{https://en.wikipedia.org/wiki/Indirect_aggression}{indirect}. However, there are critiques for using animal behavior to explain human behavior. Especially in the application of evolutionary explanations to contemporary human behavior, including differences between the genders.

According to the 2015 \href{https://en.wikipedia.org/wiki/International_Encyclopedia_of_the_Social_%26_Behavioral_Sciences}{\textit{International Encyclopedia of the Social \& Behavioral Sciences}}, sex differences in aggression is 1 of the most robust \& oldest findings in psychology. Past meta-analyses in the encyclopedia found males regardless of age engaged in more physical \& verbal aggression while small effect for females engaging in more indirect aggression such as rumor spreading or gossiping. It also found males tend to engage in more unprovoked aggression in higher frequency than females. This analysis also conforms with the \textit{Oxford Handbook of Evolutionary Psychology} which reviewed past analysis which found men to use more verbal \& physical aggression with the difference being greater in the physical type. There are more recent findings that show that differences in male \& female aggression appear at about 2 years of age, though the differences in aggression are more consistent in middle-aged children \& adolescence. Tremblay, Japel \& P\'erusse (1999) asserted that physically aggressive behavior such as kicking, biting \& hitting are age-typical expressions of innate \& spontaneous reactions to biological drives such as anger, hunger, \& affiliation. Girls' \href{https://en.wikipedia.org/wiki/Relational_aggression}{relational aggression}, meaning non-physical or indirect, tends to increase after age 2 while physical aggression decreases. There was no significant difference in aggression between males \& females before 2 years of age. A possible explanation for this could be that girls develop language skills more quickly than boys, \& therefore have better ways of verbalizing their wants \& needs. They are more likely to use communication when trying to retrieve a toy with the words ``Ask nicely'' or ``Say please.''

According to the journal of \textit{Aggressive Behavior}, an analysis across 9 countries found boys reported more in the use of physical aggression. At the same time no consistent sex differences emerged within relational aggression. It has been found that girls are more likely than boys to use reactive aggression \& then retract, but boys are more likely to increase rather than to retract their aggression after their 1st reaction. Studies show girls' aggressive tactics included \href{https://en.wikipedia.org/wiki/Gossip}{gossip}, \href{https://en.wikipedia.org/wiki/Social_rejection}{ostracism}, breaking confidences, \& criticism of a victim's clothing, appearance, or personality, whereas boys engage in aggression that involves a direct physical \&\texttt{/}or verbal assault. This could be due to the fact that girls' frontal lobes develop earlier than boys, allowing them to self-restrain.

1 factor that shows insignificant differences between male \& female aggression in sports. In sports, the rate of aggression in both contact \& non-contact sports is relatively equal. Since the establishment of Title IX, female sports have increased in competitiveness \& importance, which could contribute to the evening of aggression \& the ``need to win'' attitude between both genders. Among sex differences found in adult sports were that females have a higher scale of indirect hostility while men have a higher scale of assault. Another difference found is that men have up to 20 times higher levels of \href{https://en.wikipedia.org/wiki/Testosterone}{testosterone} than women. 

\paragraph{In intimate relationships.} Some studies suggest that romantic involvement in adolescence decreases aggression in males \& females, but decreases at a higher rate in females. Females will seem more desirable to their mate if they fit in with society \& females that are aggressive do not usually fit well in society, they can often be viewed as antisocial. Female aggression is not considered the norm in society \& going against the norm can sometimes prevent 1 from getting a mate. However, studies have shown that an increasing number of women are getting arrested for domestic violence charges. In many states, women now account for a quarter to $\frac{1}{3}$ of all domestic violence arrests, up from $< 10$\% a decade ago. The new statistics reflect a reality documented in research: women are perpetrators as well as victims of family violence. However, another equally possible explanation is a case of improved diagnostics: it has become more acceptable for men to report female domestic violence to the authorities while at the same time actual female domestic violence has not increased at all. This can be the case when men have become less ashamed of reporting female violence against them, therefore an increasing number of women are arrested, although the actual number of violent women remains the same.

In addition, males in competitive sports are often advised by their coaches not to be in intimate relationships based on the premises that they become more docile \& less aggressive during an athletic event. The circumstances in which males \& females experience aggression are also different. A study showed that social anxiety \& stress was positively correlated with aggression in males, meaning as stress \& social anxiety increases so does aggression. Furthermore, a male with higher social skills has a lower rate of aggressive behavior than a male with lower social skills. In females, higher rates of aggression were only correlated with higher rates of stress. Other than biological factors that contribute to aggression there are physical factors as well.

\paragraph{Physiological factors.} Regarding sexual dimorphism, humans fall into an intermediate group with moderate sex differences in body size but relatively large \href{https://en.wikipedia.org/wiki/Testes}{testes}. This is a typical pattern of primates where several males \& females live together in a group \& the male faces an intermediate number of challenges from other males compared to exclusive \href{https://en.wikipedia.org/wiki/Polygyny}{polygyny} \& \href{https://en.wikipedia.org/wiki/Monogamy}{monogamy} but frequent \href{https://en.wikipedia.org/wiki/Sperm_competition}{sperm competition}.

\href{https://en.wikipedia.org/wiki/Evolutionary_psychology}{Evolutionary psychology} \& \href{https://en.wikipedia.org/wiki/Sociobiology}{sociobiology} have also discussed \& produced theories for some specific forms of male aggression such as \href{https://en.wikipedia.org/wiki/Sociobiological_theories_of_rape}{sociobiological theories of rape} \& theories regarding the \href{https://en.wikipedia.org/wiki/Cinderella_effect}{Cinderella effect}. Another evolutionary theory explaining gender differences in aggression is the \href{https://en.wikipedia.org/wiki/Male_Warrior_hypothesis}{Male Warrior hypothesis}, which explains that males have psychologically evolved for intergroup aggression in order to gain access to mates, resources, territory \& status.'' -- \href{https://en.wikipedia.org/wiki/Aggression#Gender}{Wikipedia\texttt{/}aggression\texttt{/}evolutionary explanations\texttt{/}gender}

\subsection{Physiology}

\subsubsection{Brain pathways}

\subsubsection{Testosterone}

\subsubsection{Dehydroepiandrosterone}

\subsubsection{Glucocorticoids}

\subsubsection{Pheromones}

\subsection{Genetics}

\subsection{Society \& culture}

\subsubsection{Culture}

\subsubsection{Fear-induced aggression}

\subsubsection{Children}

\subsubsection{Gender}

\subsubsection{Situational factors}

\subsubsection{As a positive adaptation theory}

%------------------------------------------------------------------------------%

\section{\href{https://en.wikipedia.org/wiki/Assertiveness}{Wikipedia\texttt{/}Assertiveness}}
``\textit{Assertiveness} is the quality of being self-assured \& confident without being aggressive to defend a right point of view or a relevant statement. In the field of \href{https://en.wikipedia.org/wiki/Psychology}{psychology} \& \href{https://en.wikipedia.org/wiki/Psychotherapy}{psychotherapy}, it is a skill that can be learned \& a mode of communication. \href{https://en.wikipedia.org/wiki/Dorland%27s_Medical_Dictionary}{\textit{Dorland's Medical Dictionary}} defines assertiveness as:
\begin{quotation}
	``a form of behavior characterized by a confident declaration or affirmation of a statement without need of proof; this affirms the person's rights or point of view without either aggressively threatening the rights of another (assuming a position of dominance) or submissively permitting another to ignore or deny one's rights or point of view.''
\end{quotation}
It is considered a critical life skill \& recommended for children to develop. Assertiveness is a communication skill that can be taught \& the skills of assertive communication effectively learned.

\fbox{Assertiveness is a method of critical thinking}, where an individual speaks up in defense of their views or in light of erroneous information. Assertive people are able to be outspoken \& analyze information \& point out areas of information lacking substance, details or evidence. Assertiveness supports creative thinking \& effective communication.

The level of assertiveness demonstrated in any human community is a factor of social \& cultural practices at the time of inquiry. These factors can evolve with time \& may vary from a community to another one. E.g., nowadays, in the Western world, there are global public discussions about controversial topics such as drug addiction, rape \& sexual abuse of women \& children, which were not openly discussed in 1940.

Within families, children are not always encouraged to develop assertiveness skills \& must usually accept \& obey rulings by their parents. Today, however, outspoken children can legally input to decisions about their lives through legal emancipation prematurely \& may attain the rights of an adult between the ages of 14 \& 16.

During the 2nd half of the 20th century, assertiveness was increasingly singled out as a behavioral skill taught by many \href{https://en.wikipedia.org/wiki/Personal_development}{personal development} experts, \href{https://en.wikipedia.org/wiki/Behavior_therapist}{behavior therapists}, \& \href{https://en.wikipedia.org/wiki/Cognitive_behavioral_therapist}{cognitive behavioral therapists}. Assertiveness is often linked to \href{https://en.wikipedia.org/wiki/Self-esteem}{self-esteem}. The term \& concept was popularized to the general public by books such as \textit{Your Perfect Right: A Guide to Assertive Behavior} (1970) by Robert Eating.'' -- \href{https://en.wikipedia.org/wiki/Assertiveness}{Wikipedia\texttt{/}assertiveness}

\subsection{Training}
``\href{https://en.wikipedia.org/wiki/Joseph_Wolpe}{Joseph Wolpe} initially explored the use of assertiveness as a means of ``\href{https://en.wikipedia.org/wiki/Reciprocal_inhibition}{reciprocal inhibition}'' of anxiety, in this 1958 book on treating nerosis; \& it has since been commonly employed as an intervention in \href{https://en.wikipedia.org/wiki/Behavior_therapy}{behavior therapy}. Assertiveness Training (``AT'') was introduced by \href{https://en.wikipedia.org/wiki/Andrew_Salter}{Andrew Salter} (1961) \& popularized by Joseph Wolpe. Wolpe's belief was that a person could not be both assertive \& anxious at the same time, \& thus being assertive would inhibit anxiety. The goals of assertiveness training include:
\begin{itemize}
	\item increased awareness of personal rights
	\item differentiation between non-assertiveness \& assertiveness
	\item differentiation between \href{https://en.wikipedia.org/wiki/Passive%E2%80%93aggressiveness}{passive--aggressiveness} \& \href{https://en.wikipedia.org/wiki/Aggressiveness}{aggressiveness}
	\item learning both verbal \& non-verbal assertiveness skills.
\end{itemize}
As a communication style \& strategy, assertiveness is thus distinguished from both aggression \& \href{https://en.wikipedia.org/wiki/Passivity_(behavior)}{passivity}. How people deal with \href{https://en.wikipedia.org/wiki/Personal_boundaries}{personal boundaries}, including their own \& those of other people, helps to distinguish between these 3 concepts. Passive communicators are not likely to try to influence anyone else because they fear social conflict. Because of this fear, passive communicators do not defend their own personal boundaries or ideas, \& thus allow aggressive people to \href{https://en.wikipedia.org/wiki/Abuse}{abuse} or \href{https://en.wikipedia.org/wiki/Psychological_manipulation}{manipulate} them. Additionally, they often hold in negative feelings such as anger because they allow this domination to happen. Aggressive people do not \href{https://en.wikipedia.org/wiki/Respect}{respect} the personal boundaries of others \& thus are liable to harm others by influencing them through personal attacks often taking the form of embarrassment. A person communicates assertively by clearly stating his or her thoughts \&\texttt{/}or feelings in a nonaggressive manner, often in an effort to influence others; doing so in a way that respects the personal boundaries of the other person, or people, involved \& avoids negative confrontation. Assertive people are also willing to defend themselves against aggressive people.'' -- \href{https://en.wikipedia.org/wiki/Assertiveness#Training}{Wikipedia\texttt{/}assertiveness\texttt{/}training}

\subsection{Communication}
``Assertive communication involves respect for the boundaries of oneself \& others. It also presumes an interest in the fulfillment of needs \& wants through \href{https://en.wikipedia.org/wiki/Cooperation}{cooperation}.

According to the textbook \textit{Cognitive Behavior Therapy} (2008), ``Assertive communication of personal opinions, needs, \& boundaries has been $\ldots$ conceptualized as the behavioral middle ground, lying between ineffective passive \& aggressive responses''. Such communication ``emphasizes expressing feelings forthrightly, but in a way that will not spiral into aggression''.

If others' actions threaten one's boundaries, one communicates this to prevent escalation.

In contrast, ``aggressive communication'' judges, threatens, lies, breaks confidences, stonewalls, \& violates others' boundaries.

At the opposite end of the dialectic is ``passive communication''. Victims may passively permit others to violate their boundaries. At a later time, they may come back \& attack with a sense of impunity or righteous indignation.

Assertive communication attempts to transcend these extremes by appealing to the shard interest of all parties; it ``focuses on the issue, not the person''. Aggressive \&\texttt{/}or passive communication, on the other hand, may mark a relationship's end, \& reduced self-respect.'' -- \href{https://en.wikipedia.org/wiki/Assertiveness#Communication}{Wikipedia\texttt{/}assertiveness\texttt{/}communication}

\subsection{Characteristics}
``Assertive people tend to have the following characteristics:
\begin{itemize}
	\item They feel free to express their feelings, thoughts, \& desires.
	\item They are ``also able to initiate \& maintain comfortable relationships with [other] people''.
	\item They know their rights.
	\item They have control over their anger. This does not mean that they repress this feeling; it means that they control anger \& talk about it in a reasoning manner.
	\item ``Assertive people $\ldots$ are willing to compromise with others, rather than always wanting their own way $\ldots$ \& tend to have good self-esteem''.'' -- \href{https://en.wikipedia.org/wiki/Assertiveness#Characteristics}{Wikipedia\texttt{/}assertiveness\texttt{/}characteristics}
\end{itemize}

\subsection{Techniques}
``Techniques of assertiveness can vary widely. Manuel Smith, in his 1975 book \textit{When I Say No, I Feel Guilty}, offered some of the following behaviors:

\subsubsection{Broken record}
The ``broken record'' technique consists of simply repeating your requests or your refusals every time you are met with resistance. The term comes from \href{https://en.wikipedia.org/wiki/Vinyl_record}{vinyl records}, the surface of which when scratched would lead the needle of a \href{https://en.wikipedia.org/wiki/Record_player}{record player} to loop over the same few seconds of the recording indefinitely. ``As with a broken record, the key to this approach is repetition $\ldots$ where your partner will not take no for an answer.''

A disadvantage with this technique is that when resistance continues, your requests may lose power every time you have to repeat them. If the requests are repeated too often, it can backfire on the authority of your words. In these cases, it is necessary to have some sanctions on hand.''

\subsubsection{Fogging}
``Fogging consists of finding some limited truth to agree with in what an antagonist is saying. More specially, one can \textit{agree in part} or \textit{agree in principle}.''

\subsubsection{Negative inquiry}
``Negative inquiry consists of requesting further, more specific criticism.''

\subsubsection{Negative assertion}
``Negative assertion is agreement with criticism without letting up demand.''

\subsubsection{I-statements}
``\href{https://en.wikipedia.org/wiki/I-statements}{I-statements} can be used to voice one's feelings \& wishes from a personal position without expressing a judgment about the other person or blaming one's feelings on them.'''' -- \href{https://en.wikipedia.org/wiki/Assertiveness#Techniques}{Wikipedia\texttt{/}assertiveness\texttt{/}techniques}

\subsection{Applications}
``Several research studies have identified assertiveness training as a useful tool in the prevention of alcohol-use disorders. Psychological skills in general including assertiveness \& \href{https://en.wikipedia.org/wiki/Social_skills}{social skills} have been posed as intervention for a variety of disorders with some empirical support.

In connection with \href{https://en.wikipedia.org/wiki/Gender_theory}{gender theory}, ``\href{https://en.wikipedia.org/wiki/Deborah_Tannen}{Tannen} argues that men \& women would both benefit from learning to use the others' style. $\ldots$ So, women would benefit from assertiveness training just as men might benefit from sensitivity training''.'' -- \href{https://en.wikipedia.org/wiki/Assertiveness#Applications}{Wikipedia\texttt{/}assertiveness\texttt{/}applications}

\subsection{Challenges}
``Assertiveness may be practiced in an unbalanced way, especially by  those new to the process: ``[One] problem with the concept of assertiveness is that it is both complex \& situation-specific. $\ldots$ Behaviors that are assertive in 1 circumstance may not be so in another''. More particularly, while ``unassertiveness courts 1 set of problems, over-assertiveness creates another.'' Assertiveness manuals recognize that ``many people, when trying out assertive behavior for the 1st time, find that they go too far \& become aggressive.''

In the late 1970's \& early 1980's, in the heyday of assertiveness training, some so-called assertiveness training techniques were distorted \& ``people were told to do some pretty obnoxious things in the name of assertiveness. Like blankly repeating some request over \& over until you got your way''. Divorced from respect for the rights of others, so-called assertiveness techniques could be psychological tools that might be readily abused: The line between repeatedly demanding with sanctions (``broken record'') versus coercive \href{https://en.wikipedia.org/wiki/Nagging}{nagging}, \href{https://en.wikipedia.org/wiki/Emotional_blackmail}{emotional blackmail}, or \href{https://en.wikipedia.org/wiki/Bullying}{bullying}, could be a fine one, \& the caricature of assertiveness training as ``training in how to get your own way $\ldots$ or how to become as aggressive as the next person'' was perpetuated.'' -- \href{https://en.wikipedia.org/wiki/Assertiveness#Challenges}{Wikipedia\texttt{/}assertiveness\texttt{/}challenges}

%------------------------------------------------------------------------------%

\section{\href{https://en.wikipedia.org/wiki/Critical_thinking}{Wikipedia\texttt{/}Critical Thinking}}
``\textit{Critical thinking} is the analysis of available \href{https://en.wikipedia.org/wiki/Fact}{facts}, \href{https://en.wikipedia.org/wiki/Evidence}{evidence}, \href{https://en.wikipedia.org/wiki/Observation}{observations}, \& \href{https://en.wikipedia.org/wiki/Argument}{arguments} to form a judgment. The subject is complex; several different \href{https://en.wikipedia.org/wiki/Critical_thinking#Definitions}{definitions} exist, which generally include the \href{https://en.wikipedia.org/wiki/Rational}{rational}, \href{https://en.wikipedia.org/wiki/Skepticism}{skeptical}, \& \href{https://en.wikipedia.org/wiki/Unbiased}{unbiased} analysis or evaluation of factual \href{https://en.wikipedia.org/wiki/Evidence}{evidence}. Critical thinking is \href{https://en.wikipedia.org/wiki/Self-directedness}{self-directed}, \href{https://en.wikipedia.org/wiki/Discipline}{self-disciplined}, self-\href{https://en.wikipedia.org/wiki/Monitoring_(medicine)}{monitored}, \& self-\href{https://en.wikipedia.org/wiki/Corrective_feedback}{corrective} thinking. It presupposes assent to rigorous standards of \href{https://en.wikipedia.org/wiki/Excellence}{excellence} \& mindful command of their use. It entails effective communication \& problem-solving abilities as well as a commitment to overcome native \href{https://en.wikipedia.org/wiki/Egocentrism}{egocentrism} \& \href{https://en.wikipedia.org/wiki/Sociocentrism}{sociocentrism}.

\subsection{History}
\textsf{Fig. Sculpture of Socrates.}

``The earliest records of critical thinking are the teachings of \href{https://en.wikipedia.org/wiki/Socrates}{Socrates} recorded by \href{https://en.wikipedia.org/wiki/Plato}{Plato}. These included a part in Plato's early dialogues, where Socrates engages with 1 or more interlocutors on the issue of ethics such as question whether it was right for Socrates to escape from prison. The philosopher considered \& reflected on  this question \& came to the conclusion that escape violates all the things that he holds higher than himself: the laws of Athens \& the guiding voice that Socrates claims to hear.

Socrates established the fact that one cannot depend upon those in ``authority'' to have sound knowledge \& insight. He demonstrated that persons may have power \& high position \& yet be deeply confused \& irrational. Socrates maintained that for an individual to have a good life or to have one that is worth living, he must be a \fbox{critical questioner} \& possess an \fbox{interrogative soul}. He established the importance of asking deep questions that probe profoundly into thinking before we accept ideas as worthy of belief.

Socrates established the importance of ``seeking evidence, close examining reasoning \& assumptions, analyzing basic concepts, \& tracing out implications not only of what is said but of what is done as well''. His method of questioning is now known as ``\href{https://en.wikipedia.org/wiki/Socratic_questioning}{Socratic questioning}'' \& is the best known critical thinking teaching strategy. In his mode of questioning, Socrates highlighted the need for thinking for clarity \& logical consistency. He asked people questions to reveal their irrational thinking or lack of reliable knowledge. Socrates demonstrated that \fbox{having authority does not ensure accurate knowledge}. He established the method of questioning beliefs, closely inspecting assumptions \& relying on evidence \& sound rationale. Plato recorded Socrates' teachings \& carried on the tradition of critical thinking. Aristotle \& subsequent Greek skeptics refined Socrates' teachings, using systematic thinking \& asking questions to ascertain the true nature of reality beyond the way things appear from a glance.

Socrates set the agenda for the tradition of critical thinking, namely, to reflectively question common beliefs \& explanations, carefully distinguishing beliefs that are reasonable \& logical from those that -- however appealing to our native egocentrism, however much they serve our vested interests, however comfortable or comforting they may be -- lack adequate evidence or rational foundation to warrant belief.

Critical thinking was described by Richard W. Paul as a movement in 2 waves (1994). The ``1st wave'' of critical thinking is often referred to as a `critical analysis' that is clear, \fbox{\href{https://en.wikipedia.org/wiki/Rationality}{rational} thinking involving \href{https://en.wikipedia.org/wiki/Critique}{critique}}. Its details vary amongst those who \href{https://en.wikipedia.org/wiki/Operational_definition}{define} it. According to Barry K. Beyer (1995), critical thinking means making clear, reasoned judgments. During the process of critical thinking, ideas should be reasoned, well thought out, \& judged. The U.S. National Council for Excellence in Critical Thinking defines critical thinking as the ``intellectually disciplined process of actively \& skillfully conceptualizing, applying, analyzing, synthesizing, or evaluating information gathered on, or generated by, observation, experience, reflection, reasoning, or communication, as a guide to belief \& action.'''' -- \href{https://en.wikipedia.org/wiki/Critical_thinking#History}{Wikipedia\texttt{/}critical thinking\texttt{/}history}

\subsection{Etymology \& origin of critical thinking}
``In the term \textit{critical thinking}, the word \href{https://en.wiktionary.org/wiki/critical}{\textit{critical}}, (Grk. $\kappa\rho\iota\tau\iota\kappa\acute{o}\varsigma =$ \textit{kritikos} $=$ ``critic'') derives from the word \textit{critic} \& implies a \href{https://en.wikipedia.org/wiki/Critique}{critique}; it identifies the intellectual capacity \& the means ``of judging'', ``of judgment'', ``for judging'', \& of being ``able to discern.'' The intellectual roots of critical thinking are as ancient as its etymology, traceable, ultimately, to the \href{https://en.wikipedia.org/wiki/Teachings_and_philosophy_of_Swami_Vivekananda}{teahcing} practice \& vision of \href{https://en.wikipedia.org/wiki/Socrates}{Socrates} 2,500 years ago who discovered by a method of probing questioning that people could not rationally justify their confident claims to \href{https://en.wikipedia.org/wiki/Knowledge}{knowledge}.'' -- \href{https://en.wikipedia.org/wiki/Critical_thinking#Etymology_and_origin_of_critical_thinking}{Wikipedia\texttt{/}critical thinking\texttt{/}etymology \& origin of critical thinking}

\subsection{Definitions}
``Traditionally, critical thinking has been variously defined as follows:
\begin{itemize}
	\item ``The intellectually disciplined process of actively \& skillfully conceptualizing, applying, analyzing, synthesizing, \&\texttt{/}or evaluating information gathered from, or generated by, observation, experience, reflection, reasoning, or communication, as a guide to belief \& action.''
	\item ``Disciplined thinking that is clear, rational, open-minded, \& informed by evidence.''
	\item ``Purposeful, \href{https://en.wikipedia.org/wiki/Self-control}{self-regulatory} judgment which results in interpretation, analysis, evaluation, \& inference, as well as explanation of the evidential, conceptual, methodological, criteriological, or contextual considerations upon which that judgment is based''
	\item ``Includes a commitment to using reason in the formulation of our beliefs''
	\item The skill \& propensity to engage in an activity with reflective scepticism (McPeck, 1981)
	\item Thinking about one's thinking in a manner designed to organize \& clarify, raise the efficiency of, \& recognize errors \& biases in one's own thinking. Critical thinking is not `hard' thinking nor is it directed at solving problems (other than `improving' one's own thinking). Critical thinking is inward-directed with the intent of maximizing the \href{https://en.wikipedia.org/wiki/Rationality}{rationality} of the thinker. One does not use critical thinking to solve problems -- one uses critical thinking to improve one's process of thinking.
	\item ``An appraisal based on careful analytical evaluation''
	\item ``Critical thinking is a type of thinking pattern that requires people to be reflective, \& pay attention to decision-making which guides their beliefs \& actions. Critical thinking allows people to deduct with more logic, to process sophisticated information \& look at various sides of an issue so they can produce more solid conclusions.''
	\item Critical thinking has 7 critical features: being inquisitive \& curious, being open-minded to different sides, being able to think systematically, being analytical, being persistent to truth, being confident about critical thinking itself, \& lastly, being mature.
	\item Although critical thinking could be defined in several different ways, there is a general agreement in its key component -- the desire to reach for a satisfactory result, \&  this should be achieved by rational thinking \& result-driven manner. Halpern thinks that critical thinking 1stly involves learned abilities such as problem-soling, calculation \& successful probability application. It also includes a tendency to engage the thinking process. In recent times, Stanovich believed that modern IQ testing could hardly measure the ability of critical thinking.
	\item ``Critical thinking is essentially a questioning, challenging approach to knowledge \& perceived wisdom. It involves ideas \& information from an objective position \& then questioning this information in the light of our own values, attitudes \& personal philosophy.''
\end{itemize}
Contemporary critical thinking scholars have expanded these traditional definitions to include qualities, concepts, \& processes such as creativity, imagination, discovery, reflection, empathy, connecting knowing, feminist theory, subjectivity, ambiguity, \& inconclusiveness. Some definitions of critical thinking exclude these subjective practices.
\begin{enumerate}
	\item According to Ennis, ``Critical thinking is the intellectually disciplined process of actively \& skillfully conceptualizing, applying, analyzing, synthesizing, \&\texttt{/}or evaluating information gathered from, or generated by, observation, experience, reflection, reasoning, or communication, as a guide to belief \& action.'' This definition Ennis provided is highly agreed by Harvey Siegel, Peter Facione, \& Deanna Kuhn.
	\item According to Ennis' definition, critical thinking requires a lot of attention \& brain function. When a critical thinking approach is applied to education, it helps the student's brain function better \& understand texts differently.
	\item Different fields of study may require different types of critical thinking. Critical thinking provides more angles \& perspectives upon the same material.'' -- \href{https://en.wikipedia.org/wiki/Critical_thinking#Definitions}{Wikipedia\texttt{/}critical thinking\texttt{/}definitions}
\end{enumerate}

\subsection{Logic \& rationality}
``Main article: \href{https://en.wikipedia.org/wiki/Logic_and_rationality}{Wikipedia\texttt{/}logic \& rationality}. The study of logical argumentation is relevant to the study of critical thinking. Logic is concerned with the analysis of arguments, including the appraisal of their correctness or incorrectness. In the field of \href{https://en.wikipedia.org/wiki/Epistemology}{epistemology}, critical thinking is considered to be logically correct thinking, which allows for differentiation between logically true \& logically false statements.

In ``1st wave'' logical thinking, the \href{https://en.wikipedia.org/wiki/Thought}{thinker} is removed from the train of thought, \& the analysis of connections between concepts or points in thought is ostensibly free of any bias. In his essay \textit{Beyond Logicism \& Critical Thinking} \href{https://en.wikipedia.org/wiki/Kerry_S._Walters}{Kerry S. Walters} describes this ideology thus: ``A logistic approach to critical thinking conveys the message to students that thinking is legitimate only when it conforms to the procedures of informal (\&, to a lesser extent, formal) logic \& that the good thinker necessarily aims for styles of examination \& appraisal that are analytical, abstract, universal, \& objective. This model of thinking has become so entrenched in conventional academic wisdom that many educators accept it as canon.'' Such principles are concomitant with the increasing dependence on a \textit{quantitative} understanding of the world.

In the `2nd wave' of critical thinking, authors consciously moved away from the logocentric mode of critical thinking characteristic of the `1st wave'. Although many scholars began to take a less exclusive view of what constitutes critical thinking, rationality \& logic remain widely accepted as essential bases for critical thinking. Walters argues that exclusive logicism in the 1st wave sense is based on ``the unwarranted assumption that good thinking is reducible to logical thinking''.'' -- \href{https://en.wikipedia.org/wiki/Critical_thinking#Logic_and_rationality}{Wikipedia\texttt{/}critical thinking\texttt{/}logic \& rationality}

\subsubsection{Deduction, abduction \& induction}
\textsf{Fig. \href{https://en.wikipedia.org/wiki/Argument}{Argument} terminology used in \href{https://en.wikipedia.org/wiki/Logic}{logic}.}

``Main article: \href{https://en.wikipedia.org/wiki/Logical_reasoning}{Wikipedia\texttt{/}logical reasoning}. There are 3 types of \href{https://en.wikipedia.org/wiki/Logical_reasoning}{logical reasoning}. Informally, 2 kinds of logical reasoning can be distinguished in addition to formal \href{https://en.wikipedia.org/wiki/Deductive_reasoning}{deduction}, which are \href{https://en.wikipedia.org/wiki/Inductive_reasoning}{induction} \& \href{https://en.wikipedia.org/wiki/Abductive_reasoning}{abduction}.

\paragraph{Deduction.} \href{https://en.wikipedia.org/wiki/Deductive_reasoning}{Deduction} is the conclusion drawn from the structure of an argument's \href{https://en.wikipedia.org/wiki/Premises}{permises}, by use of \href{https://en.wikipedia.org/wiki/Rules_of_inference}{rules of inference} formally those of \href{https://en.wikipedia.org/wiki/Propositional_calculus#Basic_and_derived_argument_forms}{propositonal calculus}. E.g.: $X$ is human \& all humans have a face, so $X$ has a face.

\paragraph{Induction.} \href{https://en.wikipedia.org/wiki/Inductive_reasoning}{induction} is drawing a conclusion from a pattern that is guaranteed by the strictness of the structure to which it applies. E.g.: The sum of even integers is even. Let $x,y,z\in\mathbb{Z}$ then $2x,2y,2z$ are even by definition. $2x + 2y = 2(x + y) = 2z$, which is even; so summing 2 even numbers results in an even number.

\paragraph{Abduction.} \href{https://en.wikipedia.org/wiki/Abductive_reasoning}{Abduction} is drawing a conclusion using a \href{https://en.wikipedia.org/wiki/Heuristic}{heuristic} that is likely, but not inevitable given some foreknowledge. E.g.: I observe sheep in a field, \& they appear white from my viewing angle, so sheep are white. Contrast with the deductive statement: Some sheep are white on at least 1 side.'' -\href{https://en.wikipedia.org/wiki/Critical_thinking#Deduction,_abduction_and_induction}{Wikipedia\texttt{/}critical thinking\texttt{/}logic \& rationality\texttt{/}deduction, abduction \& induction}

\subsubsection{Critical thinking \& rationality}
``\href{https://en.wikipedia.org/wiki/Kerry_S._Walters}{Kerry S. Walters}, an emeritus philosophy professor from \href{https://en.wikipedia.org/wiki/Gettysburg_College}{Gettysburg College}, argues that rationality demands more than just logical or traditional methods or problem solving \& analysis or what he calls the ``calculus of justification'' but also considers ``\href{https://en.wikipedia.org/wiki/Cognitive}{cognitive} acts such as \href{https://en.wikipedia.org/wiki/Imagination}{imagination}, conceptual creativity, \href{https://en.wikipedia.org/wiki/Intuition}{intuition} \& insight'' (p. 63). These ``functions'' are focused on discovery, on more abstract processes instead of linear, rules-based approaches to problem-solving. The linear \& non-sequential mind must both be engaged in the \href{https://en.wikipedia.org/wiki/Rational}{rational} \href{https://en.wikipedia.org/wiki/Mind}{mind}.

The ability to critically analyze an argument -- to dissect structure \& components, thesis \& reasons -- is essential. But so is the ability to be flexible \& consider non-traditional alternatives \& perspectives. These complementary functions are what allow for critical thinking to be a practice encompassing imagination \& intuition in cooperation with traditional modes of deductive inquiry.'' - \href{https://en.wikipedia.org/wiki/Critical_thinking#Critical_thinking_and_rationality}{Wikipedia\texttt{/}critical thinking\texttt{/}logic \& rationality\texttt{/}critical thinking \& rationality}

\subsection{Functions}
``The list of core critical thinking skills includes observation, interpretation, analysis, inference, evaluation, explanation, \& \href{https://en.wikipedia.org/wiki/Metacognition}{metacognition}. According to Reynolds (2011), an individual or group engaged in a strong way of critical thinking gives due consideration to establish e.g.:
\begin{itemize}
	\item Evidence through reality
	\item Context skills to isolate the problem from context
	\item Relevant criteria for making the judgment well
	\item Applicable methods or techniques for forming the judgment
	\item Applicable theoretical constructs for understanding the problem \& the question at hand
\end{itemize}
In addition to possessing strong critical-thinking skills, one must be disposed to engage problems \& decisions using those skills. Critical thinking employs not only \href{https://en.wikipedia.org/wiki/Logic}{logic} but broad \href{https://en.wikipedia.org/wiki/Intellect}{intellectual} criteria such as clarity, \href{https://en.wikipedia.org/wiki/Credibility}{credibility}, \href{https://en.wikipedia.org/wiki/Accuracy}{accuracy}, precision, \href{https://en.wikipedia.org/wiki/Relevance}{relevance}, depth, \href{https://en.wikipedia.org/wiki/Breadth}{breadth}, significance, \& fairness.

Critical thinking calls for the ability to:
\begin{itemize}
	\item Recognize problems, to find workable means for meeting those problems
	\item Understand the importance of prioritization \& order of precedence in problem-solving
	\item Gather \& marshal pertinent (relevant) information
	\item Recognize \href{https://en.wikipedia.org/wiki/Unstated_assumption}{unstated assumptions} \& values
	\item Comprehend \& use \href{https://en.wikipedia.org/wiki/Language}{language} with accuracy, clarity, \& \href{https://en.wiktionary.org/wiki/discernment}{discernment}
	\item Interpret data, to appraise evidence \& evaluate arguments
	\item Recognize the existence (or non-existence) of logical relationships between propositions
	\item Draw warranted conclusions \& generalizations
	\item Put to test the conclusions \& generalizations at which one arrives
	\item Reconstruct one's patterns of beliefs on the basis of wider experience
	\item Render accurate judgments about specific things \& qualities in everyday life
\end{itemize}
In sum: ``A persistent effort to examine any belief or supposed form of knowledge in the light of the evidence that supports or refutes it \& the further conclusions to which it tends.'' -- \href{https://en.wikipedia.org/wiki/Critical_thinking#Functions}{Wikipedia\texttt{/}critical thinking\texttt{/}functions}

\subsection{Habits or traits of the mind}
``The habits of \href{https://en.wikipedia.org/wiki/Mind}{mind} that characterize a person strongly disposed toward critical thinking include a desire to follow reason \& evidence wherever they may lead, a systematic approach to problem solving, \href{https://en.wikipedia.org/wiki/Inquisitiveness}{inquisitiveness}, even-handedness, \& confidence in \href{https://en.wikipedia.org/wiki/Reasoning}{reasoning}.

According to a definition analysis by Kompf \& Bond (2001), critical thinking involves problem solving, decision making, \href{https://en.wikipedia.org/wiki/Metacognition}{metacognition}, rationality, rational thinking, \href{https://en.wikipedia.org/wiki/Reasoning}{reasoning}, \href{https://en.wikipedia.org/wiki/Knowledge}{knowledge}, \href{https://en.wikipedia.org/wiki/Intelligence}{intelligence} \& also a moral component such as reflective thinking. Critical thinkers therefore need to have reached a level of maturity in their development, possess a certain attitude as well as a set of taught skills.

There is a postulation by some writers that the tendencies from habits of mind should be thought as virtues to demonstrate the characteristics of a critical thinker. These \href{https://en.wikipedia.org/wiki/Intellectual_virtue}{intellectual virtues} are ethical qualities that encourage motivation to think in particular ways towards specific circumstances. However, these virtues have also been criticized by skeptics, who argue that there is lacking evidence for this specific mental basis that are causative to critical thinking.'' -- \href{https://en.wikipedia.org/wiki/Critical_thinking#Habits_or_traits_of_the_mind}{Wikipedia\texttt{/}critical thinking\texttt{/}habits or traits of the mind}

\subsection{Research in critical thinking}
``Edwawrd M. Glaser proposed that the ability to think critically involves 3 elements:
\begin{enumerate}
	\item An attitude of being disposed to consider in a thoughtful way the problems \& subjects that come within the range of one's experiences
	\item Knowledge of the methods of logical \href{https://en.wikipedia.org/wiki/Inquiry}{inquiry} \& reasoning
	\item Some skill in applying those methods.
\end{enumerate}
Educational programs aimed at developing critical thinking in children \& adult learners, individually or in group problem solving \& decision making contexts, continue to address the same 3 central elements.

The Critical Thinking project at Human Science Lab, \href{https://en.wikipedia.org/wiki/London}{London}, is involved in the scientific study of all major \href{https://en.wikipedia.org/wiki/Educational_system}{educational systems} in prevalence today to assess how the systems are working to promote or impede critical thinking.

Contemporary cognitive psychology regards human reasoning as a complex process that is both reactive \& reflective. This presents a problem which is detailed as a division of a critical mind in juxtaposition to sensory data \& memory.

The psychological theory disposes of the absolute nature of the rational mind, in reference to conditions, abstract problems \& discursive limitations. Where the relationship between critical thinking skills \& critical thinking dispositions is an empirical question, the ability to attain causal domination exists, for which Socrates was known to be largely disposed against as the practice of \href{https://en.wikipedia.org/wiki/Sophist}{Sophistry}. Accounting for a measure of ``critical thinking dispositions'' is the California Measure of Mental Motivation \& the California Critical Thinking Dispositions Inventory. The Critical Thinking Toolkit is an alternative measure that examines student beliefs \& attitudes about critical thinking.'' -- \href{https://en.wikipedia.org/wiki/Critical_thinking#Research_in_critical_thinking}{Wikipedia\texttt{/}critical thinking\texttt{/}research in critical thinking}

\subsection{Education}
``\href{https://en.wikipedia.org/wiki/John_Dewey}{John Dewey} is 1 of many educational leaders who recognized that a curriculum aimed at building thinking skills would benefit the individual learner, the community, \& the entire democracy.

Critical thinking is significant in the learning process of \href{https://en.wikipedia.org/wiki/Internalization}{internalization}, in the construction of basic ideas, principles, \& theories inherent in content. \& critical thinking is significant in the learning process of application, whereby those ideas, principles, \& theories are implemented effectively as they become relevant in learners' lives.

Each discipline adapts its use of critical thinking concepts \& principles. The core concepts are always there, but they are embedded in subject-specific content. For students to learn content, intellectual engagement is crucial. All students must do their \fbox{own thinking}, their \fbox{own construction of knowledge}. Good teachers recognize this \& therefore focus on the questions, readings, activities that stimulate the mind to take ownership of key concepts \& principles underlying the subject.

Historically, the teaching of critical thinking focused only on logical procedures such as formal \& informal logic. This emphasized to students that \fbox{good thinking is equivalent to logical thinking}. However, a 2nd wave of critical thinking, urges educators to value conventional techniques, meanwhile expanding what it means to be a critical thinker. In 1994, Kerry Walters compiled a conglomeration of sources surpassing this logical restriction to include many different authors' research regarding connected knowing, empathy, gender-sensitive ideals, collaboration, world views, intellectual autonomy, morality \& enlightenment. These concepts invite students to incorporate their own perspectives \& experiences into their thinking.

In the English \& Welsh school systems, \textit{Critical Thinking} is offered as a subject that 16- to 18-year-olds can take as an \href{https://en.wikipedia.org/wiki/Advanced_Level_(UK)}{A-Level}. Under the \href{https://en.wikipedia.org/wiki/OCR_(exam_board)}{OCR} \href{https://en.wikipedia.org/wiki/Exam_board}{exam board}, students can sit 2 exam papers for the AS: ``Credibility of Evidence'' \& ``Assessing \& Developing Argument''. The full Advanced \href{https://en.wikipedia.org/wiki/General_Certificate_of_Education}{GCE} is now available: in addition to the 2 AS units, candidates sit the 2 papers ``Resolution of Dilemmas'' \& ``Critical Reasoning''. The A-level tests candidates on their ability to think critically about, \& analyze, arguments on their deductive or inductive validity, as well as producing their own arguments. It also tests their ability to analyze certain related topics such as credibility \& ethical decision-making. However, due to its comparative lack of subject content, many universities do not accept it as a main A-level for admissions. Nevertheless, the AS is often useful in developing reasoning skills, \& the full Advanced GCE is useful for degree courses in politics, philosophy, history or \href{https://en.wikipedia.org/wiki/Theology}{theology}, providing the skills required for critical analysis that are useful, e.g., in biblical study.

There used to also be an \href{https://en.wikipedia.org/wiki/Advanced_Extension_Award}{Advanced Extension Award} offered in Critical Thinking in the UK, open to any A-level student regardless of whether they have a Critical Thinking A-level. \href{https://en.wikipedia.org/wiki/Cambridge_International_Examinations}{Cambridge International Examinations} have an A-level in Thinking Skills.

From 2008, \href{https://en.wikipedia.org/wiki/Assessment_and_Qualifications_Alliance}{Assessment \& Qualifications Alliance} has also been offering an A-level Critical Thinking specification. \href{https://en.wikipedia.org/wiki/OCR_(exam_board)}{OCR} \href{https://en.wikipedia.org/wiki/Exam_board}{exam board} have also modified theirs for 2008. Many examinations for university entrance set by universities, on top of A-level examinations, also include a critical thinking component, such as the \href{https://en.wikipedia.org/wiki/LNAT}{LNAT}, the \href{https://en.wikipedia.org/wiki/UKCAT}{UKCAT}, the \href{https://en.wikipedia.org/wiki/BioMedical_Admissions_Test}{BioMedical Admissions Test} \& the \href{https://en.wikipedia.org/wiki/Thinking_Skills_Assessment}{Thinking Skills Assessment}.

In \href{https://en.wikipedia.org/wiki/Qatar}{Qatar}, critical thinking was offered by \href{https://en.wikipedia.org/wiki/AL-Bairaq}{AL-Bairaq} -- an outreach, non-traditional educational program that targets high school students \& focuses on a curriculum based on \href{https://en.wikipedia.org/wiki/STEM_fields}{STEM fields}. The idea behind \href{https://en.wikipedia.org/wiki/AL-Bairaq}{AL-Bairaq} is to offer high school students the opportunity to connect with the research environment in the Center for Advanced Materials (CAM) at Qatar University. Faculty members train \& mentor the students \& help develop \& enhance their critical thinking, problem-solving, \& teamwork skills.'' -- \href{https://en.wikipedia.org/wiki/Critical_thinking#Education}{Wikipedia\texttt{/}critical thinking\texttt{/}education}

\subsubsection{Effectiveness}
``In 1995, a meta-analysis of the literature on teaching effectiveness in higher \href{https://en.wikipedia.org/wiki/Education}{education} was undertaken. The study noted concerns from higher \href{https://en.wikipedia.org/wiki/Education}{education}, \href{https://en.wikipedia.org/wiki/Politician}{politicians}, \& \href{https://en.wikipedia.org/wiki/Business}{business} that higher education was failing to meet society's requirements for well-educated citizens. It concluded that although faculty may aspire to develop students' \href{https://en.wikipedia.org/wiki/Thinking}{thinking} skills, in practice they have tended to aim at facts \& concepts utilizing lowest levels of \href{https://en.wikipedia.org/wiki/Cognition}{cognition}, rather than developing intellect or \href{https://en.wikipedia.org/wiki/Values}{values}.

In a more recent meta-analysis, researchers reviewed 341 quasi- or true-experimental studies, all of which used some form of standardized critical thinking measure to assess the outcome variable. The authors describe the various methodological approaches \& attempt to categorize the differing assessment tools, which include standardized tests (\& 2nd-source measures), tests developed by teachers, tests developed by researchers, \& tests developed by teachers who also serve the role as the \href{https://en.wikipedia.org/wiki/Researcher}{researcher}. The results emphasized the need for exposing students to real-world problems \& the importance of encouraging open dialogue within a supportive environment. Effective strategies for teaching critical thinking are thought to be possible in a wide variety of \href{https://en.wikipedia.org/wiki/Educational}{educational} settings. 1 attempt to assess the \href{https://en.wikipedia.org/wiki/Humanities}{humanities}' role in teaching critical thinking \& reducing belief in \href{https://en.wikipedia.org/wiki/Pseudoscience}{pseudoscientific} claims was made at \href{https://en.wikipedia.org/wiki/North_Carolina_State_University}{North Carolina State University}. Some success was noted \& the researchers emphasized the value of the humanities in providing the skills to evaluate current events \& qualitative data in context.

\href{https://en.wikipedia.org/wiki/Scott_Lilienfeld}{Scott Lilienfeld} notes that there is some evidence to suggest that basic critical thinking skills might be successfully taught to children at a younger age than previously thought.'' -- \href{https://en.wikipedia.org/wiki/Critical_thinking#Effectiveness}{Wikipedia\texttt{/}critical thinking\texttt{/}education\texttt{/}effectiveness}

\subsection{Importance in academics}
``Critical thinking is an important element of all professional fields \& academic disciplines (by referencing their respective sets of permissible questions, evidence sources, criteria, etc.). Within the framework of \href{https://en.wikipedia.org/wiki/Scientific_skepticism}{scientific skepticism}, the process of critical thinking involves the careful acquisition \& interpretation of information \& use of it to reach a \href{https://en.wikipedia.org/wiki/Theory_of_justification}{well-justified} conclusion. The concepts \& principles of critical thinking can be applied to any context or case but only by reflecting upon the nature of that application. Critical thinking forms, therefore, a system of related, \& overlapping, modes of thought such as anthropological thinking, sociological thinking, historical thinking, political thinking, \href{https://en.wikipedia.org/wiki/Psychological}{psychological} thinking, philosophical thinking, mathematical thinking, chemical thinking, biological thinking, ecological thinking, legal thinking, ethical thinking, musical thinking, thinking like a painter, sculptor, engineer, business person, etc. In other words, though critical thinking principles are universal, their application to disciplines requires a process of reflective \href{https://en.wikipedia.org/wiki/Contextualism}{contextualization}. Psychology offerings, e.g., have included courses such as Critical Thinking about the \href{https://en.wikipedia.org/wiki/Paranormal}{Paranormal}, in which students are subjected to a series of \href{https://en.wikipedia.org/wiki/Cold_reading}{cold readings} \& tested on their belief of the ``psychic'', who is eventually announced to be a fake.

Critical thinking is considered important in the academic fields for enabling one to analyze, evaluate, explain, \& restructure thinking, thereby ensuring the act of thinking without false belief. However, even with knowledge of the methods of logical inquiry \& reasoning, mistakes occur, \& due to a thinker's inability to apply the methodology consistently, \& because of overruling character traits such as \href{https://en.wikipedia.org/wiki/Egocentrism}{egocentrism}. Critical thinking includes identification of \href{https://en.wikipedia.org/wiki/Prejudice}{prejudice}, \href{https://en.wikipedia.org/wiki/Bias}{bias}, propaganda, self-deception, distortion, \href{https://en.wikipedia.org/wiki/Misinformation}{misinformation}, etc. Given research in \href{https://en.wikipedia.org/wiki/Cognitive_psychology}{cognitive psychology}, some \href{https://en.wikipedia.org/wiki/Education}{educators} believe that schools should focus on teaching their students critical thinking \href{https://en.wikipedia.org/wiki/Skill}{skills} \& cultivation of intellectual traits.

Critical thinking skills can be used to help nurses during the assessment process. Through the use of critical thinking, nurses can question, evaluate, \& reconstruct the nursing care process by challenging the established theory \& practice. Critical thinking skills can help nurses problem solve, reflect, \& make a conclusive decision about the current situation they face. Critical thinking creates ``new possibilities for the development of the nursing knowledge''. Due to the sociocultural, environmental, \& political issues that are affecting healthcare delivery, it would be helpful to embody new techniques in nursing. Nurses can also engage their critical thinking skills through the Socratic method of dialogue \& reflection. This practice standard is even part of some regulatory organizations such as the College of Nurses of Ontario's Professional Standards for Continuing Competencies (2006). It requires nurses to engage in \href{https://en.wikipedia.org/wiki/Reflective_Practice}{Reflective Practice} \& keep records of this continued professional development for possible review by the college.

Critical thinking is also considered important for \href{https://en.wikipedia.org/wiki/Human_rights_education}{human rights education} for \href{https://en.wikipedia.org/wiki/Toleration}{toleration}. The \href{https://en.wikipedia.org/wiki/International_Day_for_Tolerance}{Declaration of Principlse on Tolerance} adopted by \href{https://en.wikipedia.org/wiki/UNESCO}{UNESCO} in 1995 affirms that ``education for tolerance could aim at countering factors that lead to fear \& exclusion of others, \& could help young people to develop capacities for independent judgment, \textit{critical thinking} \& ethical \href{https://en.wikipedia.org/wiki/Reasoning}{reasoning}''.'' -- \href{https://en.wikipedia.org/wiki/Critical_thinking#Importance_in_academics}{Wikipedia\texttt{/}critical thinking\texttt{/}importance in academics}

\subsection{Online communication}
``The advent \& rising popularity of online courses have prompted some to ask if computer-mediated communication (CMC) promotes, hinders, or has no effect on the amount \& quality of critical thinking in a course (relative to face-to-face communication). There is some evidence to suggest a 4th, more nuanced possibility: that CMC may promote some aspects of critical thinking but hinder others. E.g., Guiller et al. (2008) found that, relative to face-to-face discourse, online discourse featured more justifications, while face-to-face discourse featured more instances of students expanding on what others had said. The increase in justifications may be due to the asynchronous nature of online discussions, while the increase in expanding comments may be due to the spontaneity of `real-time' discussion. Newman et al. (1995) showed similar differential effects. They found that while CMC boasted more important statements \& linking of ideas, it lacked novelty. The authors suggests that this may be due to difficulties participating in a brainstorming-style activity in an asynchronous environment. Rather, the asynchrony may promote users to put forth ``considered, thought out contributions''.

Researchers assessing critical thinking in online discussion forums often employ a technique called Content Analysis, where the text of online discourse (or the transcription of face-to-face discourse) is systematically coded for different kinds of statements relating to critical thinking. E.g., a statement might be coded as ``Discuss ambiguities to clear them up'' or ``Welcoming outside knowledge'' as positive indicators of critical thinking. Conversely, statements reflecting poor critical thinking may be labeled as ``Sticking to prejudice or assumptions'' or ``Squashing attempts to bring in outside knowledge''. The frequency of these codes in CMC \& face-to-face discourse can be compared to draw conclusions about the quality of critical thinking.

Searching for evidence of critical thinking in discourse has roots in a definition of critical thinking put forth by Kuhn (1991), which emphasizes the social nature of discussion \& knowledge construction. There is limited research on the role of social experience in critical thinking development, but there is some evidence to suggest it is an important factor. E.g., research has shown that 3- to 4-year-old children can discern, to some extent, the differential creditability \& expertise of individuals. Further evidence for the impact of social experience on the development of critical thinking skills comes from work that found that 6- to 7-year-olds from China have similar levels of skepticism to 10- \& 11-year-olds in the United States. If the development of critical thinking skills was solely due to masturation, it is unlikely we would see such dramatic differences across cultures.'' -- \href{https://en.wikipedia.org/wiki/Critical_thinking#Online_communication}{Wikipedia\texttt{/}critical thinking\texttt{/}online communication}

%------------------------------------------------------------------------------%

\section{\href{https://en.wikipedia.org/wiki/Psychology}{Wikipedia\texttt{/}Psychology}}
``\textit{Psychology} is the \href{https://en.wikipedia.org/wiki/Science}{scientific} study of \href{https://en.wikipedia.org/wiki/Mind}{mind} \& \href{https://en.wikipedia.org/wiki/Behavior}{behavior}. Psychology includes the study of \href{https://en.wikipedia.org/wiki/Consciousness}{conscious} \& \href{https://en.wikipedia.org/wiki/Unconscious_mind}{unconscious} phenomena, including \href{https://en.wikipedia.org/wiki/Feeling}{feelings} \& \href{https://en.wikipedia.org/wiki/Thought}{thoughts}. It is an academic discipline of immense scope, crossing the boundaries between the \href{https://en.wikipedia.org/wiki/Natural_science}{natural} \& \href{https://en.wikipedia.org/wiki/Social_science}{social sciences}. Psychologists seek an understanding of the \href{https://en.wikipedia.org/wiki/Emergence}{emergent} properties of \href{https://en.wikipedia.org/wiki/Brain}{brains}, linking the discipline to \href{https://en.wikipedia.org/wiki/Neuroscience}{neuroscience}. As social scientists, psychologists aim to understand the behavior of individuals \& groups. $\Psi$ (or \href{https://en.wikipedia.org/wiki/Psi_(Greek)}{psi}) is a \href{https://en.wikipedia.org/wiki/Greek_alphabet}{Greek letter} which is commonly associated with the science of psychology.

A professional practitioner or researcher involved in the discipline is called a \href{https://en.wikipedia.org/wiki/Psychologist}{psychologist}. Some psychologists can also be classified as \href{https://en.wikipedia.org/wiki/Behavioural_sciences}{behavioral} or \href{https://en.wikipedia.org/wiki/Cognitive_science}{cognitive scientists}. Some psychologists attempt to understand the role of mental functions in individual \& \href{https://en.wikipedia.org/wiki/Social_behavior}{social behavior}. Other explore the \href{https://en.wikipedia.org/wiki/Physiology}{physiological} \& \href{https://en.wikipedia.org/wiki/Nervous_system}{neurobiological} processes that underline cognitive functions \& behaviors.

Psychologists are involved in research on \href{https://en.wikipedia.org/wiki/Perception}{perception}, \href{https://en.wikipedia.org/wiki/Cognition}{cognition}, \href{https://en.wikipedia.org/wiki/Attention}{attention}, \href{https://en.wikipedia.org/wiki/Emotion}{emotion}, \href{https://en.wikipedia.org/wiki/Intelligence}{intelligence}, \href{https://en.wikipedia.org/wiki/Phenomenology_(psychology)}{subjective experiences}, \href{https://en.wikipedia.org/wiki/Motivation}{motivation}, \href{https://en.wikipedia.org/wiki/Human_brain#Function}{brain functioning}, \& \href{https://en.wikipedia.org/wiki/Personality_psychology}{personality}. Psychologists' interests extend to \href{https://en.wikipedia.org/wiki/Interpersonal_relationship}{impersonal relationships}, \href{https://en.wikipedia.org/wiki/Psychological_resilience}{psychological resilience}, \href{https://en.wikipedia.org/wiki/Family_resilience}{family resilience}, \& other areas within \href{https://en.wikipedia.org/wiki/Social_psychology}{social psychology}. They also consider the unconscious mind. Research psychologists employ \href{https://en.wikipedia.org/wiki/Empirical_research}{empirical mehods} to infer \href{https://en.wikipedia.org/wiki/Causality}{causal} \& \href{https://en.wikipedia.org/wiki/Correlation}{correlational} relationships between psychological \href{https://en.wikipedia.org/wiki/Dependent_and_independent_variables}{variables}. Some, but not all, \href{https://en.wikipedia.org/wiki/Clinical_psychology}{clinical} \& \href{https://en.wikipedia.org/wiki/Counseling_psychology}{counseling} psychologists rely on \href{https://en.wikipedia.org/wiki/Hermeneutics#Psychology_and_cognitive_science}{symbolic interpretation}.

While psychological knowledge is often applied to the assessment \& treatment of mental health problems, it is also directed towards understanding \& solving problems in several spheres of human activity. By many accounts, psychology ultimately aims to benefit society. Many psychologists are involved in some kind of therapeutic role, practicing \href{https://en.wikipedia.org/wiki/Psychotherapy}{psychotherapy} in clinical, counseling, or \href{https://en.wikipedia.org/wiki/School_psychology}{school} settings. Other psychologists conduct scientific research on a wide range of topics related to mental processes \& behavior. Typically the latter group of psychologists work in academic settings (e.g., universities, medical schools, or hospitals). Another group of psychologists is employed in \href{https://en.wikipedia.org/wiki/Industrial_and_organizational_psychology}{industrial \& organizational} settings. Yet others are involved in work on \href{https://en.wikipedia.org/wiki/Developmental_psychology}{human development}, aging, \href{https://en.wikipedia.org/wiki/Sports_psychology}{sports}, health, \href{https://en.wikipedia.org/wiki/Forensic_psychology}{forensic science}, \href{https://en.wikipedia.org/wiki/Educational_psychology}{education}, \& the \href{https://en.wikipedia.org/wiki/Media_psychology}{media}.'' -- \href{https://en.wikipedia.org/wiki/Psychology}{Wikipedia\texttt{/}psychology}

\subsection{Etymology \& Definitions}
``The word \href{https://en.wiktionary.org/wiki/psychology}{\textit{psychology}} derives from the Greek word \href{https://en.wikipedia.org/wiki/Psyche_(psychology)}{psyche}, for spirit or \href{https://en.wikipedia.org/wiki/Soul_(spirit)}{soul}. The latter part of the word ``psychology'' derives from $-\lambda{\rm o}\gamma\acute{\i}\alpha$ \href{https://en.wiktionary.org/wiki/-logia}{-logia}, which refers to ``study'' or ``research''. The \href{https://en.wikipedia.org/wiki/Latin}{Latin} word \textit{psychologia} was 1st used by the \href{https://en.wikipedia.org/wiki/Croatia}{croatian} \href{https://en.wikipedia.org/wiki/Humanism}{humanist} \& \href{https://en.wikipedia.org/wiki/Croatian_latinistic_literature}{Latinist} \href{https://en.wikipedia.org/wiki/Marko_Maruli%C4%87}{Marko Maruli\'c} in his book, \href{https://en.wikipedia.org/wiki/Psichiologia_de_ratione_animae_humanae}{\textit{Psichiologia de ratione animae humanae}} (\textit{Psychology, on the Nature of the Human Soul}) in the late 15th century or early 16th century. The earliest known reference to the word \textit{psychology} in English was by \href{https://en.wikipedia.org/wiki/Steven_Blankaart}{Steven Blankaart} in 1694 in \textit{The Physical Dictionary}. The dictionary refers to ``Anatomy, which treats the Body, \& Psychology, which treats of the Soul.''

In 1890, \href{https://en.wikipedia.org/wiki/William_James}{William James} defined \textit{psychology} as ``the science of mental life, both of its phenomena \& their conditions.'' This definition enjoyed widespread currency for decades. However, this meaning was contested, notably by radical \href{https://en.wikipedia.org/wiki/Behaviorism}{behaviorists} such as \href{https://en.wikipedia.org/wiki/John_B._Watson}{John B. Watson}, who in 1913 asserted that the discipline is a ``natural science,'' the theoretical goal of which ``is the prediction \& control of behavior.'' Since James defined ``psychology'', the term more strongly implicates scientific \href{https://en.wikipedia.org/wiki/Experiment}{experimentation}. \href{https://en.wikipedia.org/wiki/Folk_psychology}{Folk psychology} refers to \href{https://en.wikipedia.org/wiki/Laity}{ordinary people}'s, as contrasted with psychology professionals', understanding of the mental states \& behaviors of people.'' -- \href{https://en.wikipedia.org/wiki/Psychology#Etymology_and_definitions}{Wikipedia\texttt{/}psychology\texttt{/}etymology \& definitions}

\subsection{History}
``Main article: \href{https://en.wikipedia.org/wiki/History_of_psychology}{Wikipedia\texttt{/}history of psychology}. The ancient civilizations of Egypt, Greece, China, India, \& Persia all engaged in the philosophical study of psychology. In Ancient Egypt the \href{https://en.wikipedia.org/wiki/Ebers_papyrus}{Ebers Papyrus} mentioned \href{https://en.wikipedia.org/wiki/Clinical_depression}{depression} \& thought disorders. Historians note that Greek philosophers, including \href{https://en.wikipedia.org/wiki/Thales}{Thales}, \href{https://en.wikipedia.org/wiki/Plato}{Plato}, \& \href{https://en.wikipedia.org/wiki/Aristotle}{Aristotle} (especially in his \href{https://en.wikipedia.org/wiki/On_the_Soul}{De Anima} treatise), addressed the \fbox{workings of the mind}. As early as the 4th century BC, the Greek physician \href{https://en.wikipedia.org/wiki/Hippocrates}{Hippocrates} theorized that \href{https://en.wikipedia.org/wiki/Mental_disorder}{mental disorders} had physical rather than supernatural causes. In 387 BCE, Plato suggested that the brain is where mental processes take place, \& in 335 BCE Aristotle suggested that it was the heart.

In China, psychological understanding grew from the philosophical works of \href{https://en.wikipedia.org/wiki/Laozi}{Laozi} \& \href{https://en.wikipedia.org/wiki/Confucius}{Confucius}, \& later from the doctrines of \href{https://en.wikipedia.org/wiki/Buddhism}{Buddhism}. This body of knowledge involves insights drawn from introspection \& observation, as well as techniques for focused thinking \& acting. It frames the universe in term of a division of physical reality \& mental reality as well as the interaction between the physical \& the mental. Chinese philosophy also emphasized purifying the mind in order to increase virtue \& power. An ancient text known as \href{https://en.wikipedia.org/wiki/Huangdi_Neijing}{The Yellow Emperor's Classic of Internal Medicine} identifies the brain as the nexus of wisdom \& sensation, includes theories of personality based on \href{https://en.wikipedia.org/wiki/Yin_and_yang}{yin-yang} balance, \& analyzes mental disorder in terms of physiological \& social disequilibria. Chinese scholarship that focused on the brain advanced during the \href{https://en.wikipedia.org/wiki/Qing_Dynasty}{Qing Dynasty} with the work of Western-educated Fang Yizhi (1611--1671), \href{https://en.wikipedia.org/wiki/Liu_Zhi_(scholar)}{Liu Zhi} (1660--1730), \& Wang Qingren (1768--1831). Wang Qingren emphasized the importance of the brain as the center of the nervous system, linked mental disorder with brain diseases, investigated the causes of dreams \& \href{https://en.wikipedia.org/wiki/Insomnia}{insomnia}, \& advanced a theory of \href{https://en.wikipedia.org/wiki/Lateralization_of_brain_function}{hemispheric lateralization} in brain function.

Influenced by \href{https://en.wikipedia.org/wiki/Hinduism}{Hinduism}, \href{https://en.wikipedia.org/wiki/Indian_philosophy}{Indian philosophy} explored distinctions in types of awareness. A central idea of the \href{https://en.wikipedia.org/wiki/Upanishads}{\textit{Upanishads}} \& other \href{https://en.wikipedia.org/wiki/Vedic_period}{Vedic} texts that formed the foundations of Hinduism was the distinction between a person's transient mundane self \& their \href{https://en.wikipedia.org/wiki/%C4%80tman_(Hinduism)}{eternal, unchanging soul}. Divergent Hindu doctrines \& \href{https://en.wikipedia.org/wiki/Buddhism}{Buddhism} have challenged this hierarchy of selves, but have all emphasized the importance of reaching higher awareness. \href{https://en.wikipedia.org/wiki/Yoga}{Yoga} encompasses a range of techniques used in pursuit of this goal. \href{https://en.wikipedia.org/wiki/Theosophy}{Theosophy}, a religion established by \href{https://en.wikipedia.org/wiki/Russian_Americans}{Russian--American} philosopher \href{https://en.wikipedia.org/wiki/Helena_Blavatsky}{Helena Blavatsky}, drew inspiration from these doctrines during her time in \href{https://en.wikipedia.org/wiki/British_Raj}{British India}.

Psychology was of interest to \href{https://en.wikipedia.org/wiki/Age_of_Enlightenment}{Enlightenment thinkers} in Europe. In Germany, \href{https://en.wikipedia.org/wiki/Gottfried_Wilhelm_Leibniz}{Gottfried Wilhelm Leibniz} (1646--1716) applied his principles of calculus to the mind, arguing that mental activity took place on an indivisible continuum. He suggested that the difference between conscious \& unconscious awareness is only a matter of degree. \href{https://en.wikipedia.org/wiki/Christian_Wolff_(philosopher)}{Christian Wolff} identified psychology as its own science, writing \textit{Psychologia Empirica} in 1732 \& \textit{Psychologia Rationalis} in 1734. \href{https://en.wikipedia.org/wiki/Immanuel_Kant}{Immanuel Kant} advanced the idea of \href{https://en.wikipedia.org/wiki/Anthropology}{anthropology} as a discipline, with psychology an important subdivision. Kant, however, explicitly rejected the idea of an \href{https://en.wikipedia.org/wiki/Experimental_psychology}{experimental psychology}, writing that ``the empirical doctrine of the soul can also never approach chemistry even as a systematic art of analysis or experimental doctrine, for in it the manifold of inner observation can be separated only by mere division in thought, \& cannot then be held separate \& recombined at will (but still less does another thinking subject suffer himself to be experimented upon to suit our purpose), \& even observation by itself already changes \& displaces the state of the observed object.'' In 1783, Ferdinand Ueberwasser (1752--1812) designated himself \textit{Professor of Empirical Psychology \& Logic} \& gave lectures on scientific psychology, though these developments were soon overshadowed by the \href{https://en.wikipedia.org/wiki/Napoleonic_Wars}{Napoleonic Wars}. At the end of the Napoleonic era, Prussian authorities discontinued the Old University of M\"unster. Having consulted philosophers \href{https://en.wikipedia.org/wiki/Georg_Friedrich_Wilhelm_Hegel}{Hegel} \& \href{https://en.wikipedia.org/wiki/Johann_Friedrich_Herbart}{Herbart}, however, in 1825 \href{https://en.wikipedia.org/wiki/Prussia}{the Prussian state} established psychology as a mandatory discipline in its rapidly expanding \& highly influential \href{https://en.wikipedia.org/wiki/Prussian_education_system}{educational system}. However, this discipline did not yet embrace experimentation. In England, early psychology involved \href{https://en.wikipedia.org/wiki/Phrenology}{phrenology} \& the response to social problems including alcoholism, violence, \& the country's crowded ``lunatic'' asylums.'' -- \href{https://en.wikipedia.org/wiki/Psychology#History}{Wikipedia\texttt{/}psychology\texttt{/}history}

\subsubsection{Beginning of experimental psychology}
\textsf{Fig. \href{https://en.wikipedia.org/wiki/Wilhelm_Wundt}{Wilhelm Wundt} (seated) with colleagues in his psychological laboratory, the 1st of its kind.}

``Philosopher \href{https://en.wikipedia.org/wiki/John_Stuart_Mill}{John Stuart Mill} believed that the human mind was open to scientific investigation, even if the science is in some ways inexact. Mill proposed a ``mental \href{https://en.wikipedia.org/wiki/Chemistry}{chemistry}'' in which elementary thoughts could combine into ideas of greater complexity. \href{https://en.wikipedia.org/wiki/Gustav_Fechner}{Gustave Fechner} began conducting \href{https://en.wikipedia.org/wiki/Psychophysics}{psychophysics} research in \href{https://en.wikipedia.org/wiki/Leipzig}{Leipzig} in the 1830s. He articulated the principle that human perception of a stimulus varies \href{https://en.wikipedia.org/wiki/Logarithmically}{logarithmically} according to its intensity. The principle became known as the \href{https://en.wikipedia.org/wiki/Weber%E2%80%93Fechner_law}{Weber--Fechner law}. Fechner's 1860 \textit{Elements of Psychophysics} challenged Kant's negative view with regard to conducting quantitative research on the mind. Fechner's achievement was to show that ``mental processes could not only be given numerical magnitudes, but also that these could be measured by experimental methods.'' In Heidelberg, \href{https://en.wikipedia.org/wiki/Hermann_von_Helmholtz}{Hermann von Helmholtz} conducted parallel research on sensory perception, \& trained physiologist \href{https://en.wikipedia.org/wiki/Wilhelm_Wundt}{Wilhelm Wundt}. Wundt, in turn, came to Leipzig University, where he established th psychological \href{https://en.wikipedia.org/wiki/Laboratory}{laboratory} that brought experimental psychology to the world. Wundt focused on breaking down mental processes into the most basic components, motivated in part by an analogy to recent advances in chemistry, \& its successful investigation of the elements \& structure of materials. \href{https://en.wikipedia.org/wiki/Paul_Flechsig}{Paul Flechsig} \& \href{https://en.wikipedia.org/wiki/Emil_Kraepelin}{Emil Kraepelin} soon created another influential laboratory at Leipzig, a psychology-related lab, that focused more on experimental psychiatry.

The German psychologist \href{https://en.wikipedia.org/wiki/Hermann_Ebbinghaus}{Hermann Ebbinghaus}, a researcher at the \href{https://en.wikipedia.org/wiki/University_of_Berlin}{University of berlin}, was another 19th-century contributor to the field. He pioneered the experimental study of memory \& developed quantitative models of learning \& forgetting. In the early 20th century, \href{https://en.wikipedia.org/wiki/Wolfgang_Kohler}{Wolfgang Kohler}, \href{https://en.wikipedia.org/wiki/Max_Wertheimer}{Max Wertheimer}, \& \href{https://en.wikipedia.org/wiki/Kurt_Koffka}{Kurt Koffka} co-founded the school of \href{https://en.wikipedia.org/wiki/Gestalt_psychology}{Gestalt psychology} (not to be confused with the \href{https://en.wikipedia.org/wiki/Gestalt_therapy}{Gestalt therapy} of \href{https://en.wikipedia.org/wiki/Fritz_Perls}{Fritz Perls}). The approach of Gestalt psychology is based upon the idea that individuals experience things as unified wholes. Rather than \href{https://en.wikipedia.org/wiki/Reductionism}{reducing} thoughts \& behavior into smaller component elements, as in structuralism, the Gestaltists maintained that whole of experience is important, \& differs from the sum of its parts.

Psychologists in Germany, Denmark, Austria, England, \& the United States soon followed Wundt in setting up laboratories. \href{https://en.wikipedia.org/wiki/G._Stanley_Hall}{G. Stanley Hall}, an American who studied with Wundt, founded a psychology lab that became internationally influential. The lab was located at \href{https://en.wikipedia.org/wiki/Johns_Hopkins_University}{Johns Hopkins University}. Hall, in turn, trained \href{https://en.wikipedia.org/wiki/Y%C5%ABjir%C5%8D_Motora}{Yujiro Motora}, who brought experimental psychology, emphasizing psychophysics, to the \href{https://en.wikipedia.org/wiki/Imperial_University_of_Tokyo}{Imperial University of Tokyo}. Wundt's assistant, \href{https://en.wikipedia.org/wiki/Hugo_M%C3%BCnsterberg}{Hugo M\"unsterberg}, taught psychology at Harvard to students such as \href{https://en.wikipedia.org/wiki/Narendra_Nath_Sen_Gupta}{Narendra Nath Sen Gupta} -- who, in 1905, founded a psychology department \& laboratory at the \href{https://en.wikipedia.org/wiki/University_of_Calcutta}{University of Calcutta}. Wundt's students \href{https://en.wikipedia.org/wiki/Walter_Dill_Scott}{Walter Dill Scott}, \href{https://en.wikipedia.org/wiki/Lightner_Witmer}{Lightner Witmer}, \& \href{https://en.wikipedia.org/wiki/James_McKeen_Cattell}{James McKeen Cattell} worked on developing tests of mental ability. Cattell, who also studied with \href{https://en.wikipedia.org/wiki/Eugenics}{eugenicist} \href{https://en.wikipedia.org/wiki/Francis_Galton}{Francis Galton}, went on to found the \href{https://en.wikipedia.org/wiki/Psychological_Corporation}{Psychological Corporation}. Witmer focused on the mental testing of children; Scott, on employee selection.

Another student of Wundt, the Englishman \href{https://en.wikipedia.org/wiki/Edward_Titchener}{Edward Titchener}, created the psychology program at \href{https://en.wikipedia.org/wiki/Cornell_University}{Cornell University} \& advanced ``\href{https://en.wikipedia.org/wiki/Structuralism_(psychology)}{structuralist}'' psychology. The idea behind structuralism was to analyze \& classify different aspects of the mind, primarily through the method of \href{https://en.wikipedia.org/wiki/Introspection}{introspection}. William James, \href{https://en.wikipedia.org/wiki/John_Dewey}{John Dewey}, \& \href{https://en.wikipedia.org/wiki/Harvey_Carr}{Harvey Carr} advanced the idea of \href{https://en.wikipedia.org/wiki/Functional_psychology}{functionalism}, an expansive approach to psychology that underlined the Darwinian idea of a behavior's usefulness to the individual. In 1890, James wrote an influential book, \href{https://en.wikipedia.org/wiki/The_Principles_of_Psychology}{The Principles of Psychology}, which expanded on the structuralism. He memorably described ``\href{https://en.wikipedia.org/wiki/Stream_of_consciousness_(psychology)}{stream of consciousness}.'' James's ideas interested many American students in the emerging discipline. Dewey integrated psychology with societal concerns, most notably by promoting \href{https://en.wikipedia.org/wiki/Progressive_education}{progressive education}, inculcating moral values in children, \& assimilating immigrants.

A different strain of experimentalism, with a greater connection to physiology, emerged in South America, under the leadership of Horacio G. Pi\~nero at the \href{https://en.wikipedia.org/wiki/University_of_Buenos_Aires}{University of Buenos Aires}. In Russia, too, researchers placed greater emphasis on the biological basis for psychology, beginning with \href{https://en.wikipedia.org/wiki/Ivan_Sechenov}{Ivan Sechenov}'s 1873 essay, ``Who Is to Develop Psychology \& How?'' Sechenov advanced the idea of brain \href{https://en.wikipedia.org/wiki/Reflexes}{reflexes} \& aggressively promoted a \href{https://en.wikipedia.org/wiki/Determinism}{deterministic} view of human behavior. The Russian-Soviet \href{https://en.wikipedia.org/wiki/Physiologist}{physiologist} \href{https://en.wikipedia.org/wiki/Ivan_Pavlov}{Ivan Pavlov} discovered in dogs a learning process that was later termed ``\href{https://en.wikipedia.org/wiki/Classical_conditioning}{classical conditioning}'' \& applied the process to human beings.

\textsf{Fig. 1 of the dogs used in Pavlov's experiment with a surgically implanted \href{https://en.wikipedia.org/wiki/Cannula}{cannula} to measure \href{https://en.wikipedia.org/wiki/Saliva}{salivation}, \href{https://en.wikipedia.org/wiki/Taxidermy}{preserved} in the Pavlov Museum in \href{https://en.wikipedia.org/wiki/Ryazan}{Ryazan}, Russia.}'' -- \href{https://en.wikipedia.org/wiki/Psychology#Beginning_of_experimental_psychology}{Wikipedia\texttt{/}history\texttt{/}beginning of experimental psychology}

\subsubsection{Consolidation \& funding}
``1 of the earliest psychology societies was \textit{La Soci\'et\'e de Psychologie Physiologique} in France, which lasted from 1885 to 1893. The 1st meeting of the International Congress of Psychology sponsored by the \href{https://en.wikipedia.org/wiki/International_Union_of_Psychological_Science}{International Union of Psychological Science} took place in Paris, in Aug 1889, amidst \href{https://en.wikipedia.org/wiki/Exposition_Universelle_(1889)}{the World
s Fair} celebrating the centennial of the French Revolution. William James was 1 of 3 Americans among the 400 attendees. The \href{https://en.wikipedia.org/wiki/American_Psychological_Association}{American Psychological Association} (APA) was founded soon after, in 1892. The International Congress continued to be held at different locations in Europe \& with wide international participation. The 6th Congress, held in Geneva in 1909, included presentations in Russian, Chinese, \& Japanese, as well as \href{https://en.wikipedia.org/wiki/Esperanto}{Esperanto}. After a hiatus for World War I, the 7th Congress met in Oxford, with substantially greater participation from the war-victorious Anglo-Americans. In 1929, the Congress took place at Yale University in New Haven, Connecticut, attended by hundreds of members of the APA. Tokyo Imperial University led the way in bringing new psychology to the East. New ideas about psychology diffused from Japan into China.

American psychology gained status upon the U.S.'s entry into World War I. A standing committee headed by \href{https://en.wikipedia.org/wiki/Robert_Yerkes}{Robert Yerkes} administered mental tests (``\href{https://en.wikipedia.org/wiki/Army_Alpha}{Army Alpha}'' \& ``\href{https://en.wikipedia.org/wiki/Army_Beta}{Army Beta}'') to almost 1.8 million soldiers. Subsequently, the \href{https://en.wikipedia.org/wiki/Rockefeller_family}{Rockefeller family}, via the \href{https://en.wikipedia.org/wiki/Social_Science_Research_Council}{Social Science Research Council}, began to provide funding for behavioral research. Rockefeller charities funded the National Committee on Mental Hygiene, which disseminated the concept of mental illness \& lobbied for applying ideas from psychology to child rearing. Through the Bureau of Social Hygiene \& later funding of \href{https://en.wikipedia.org/wiki/Alfred_Kinsey}{Alfred Kinsey}, Rockefeller foundations helped established research on sexuality in the U.S. Under the influence of the Carnegie-funded \href{https://en.wikipedia.org/wiki/Eugenics_Record_Office}{Eugenics Record Office}, the Draper-funded \href{https://en.wikipedia.org/wiki/Pioneer_Fund}{Pioneer Fund}, \& other institutions, the \href{https://en.wikipedia.org/wiki/Eugenics_in_the_United_States}{eugenics movement} also influenced American psychology. In the 1910s \& 1920s, eugenics became a standard topic in psychology classes. In contrast to the US, in the UK psychology was met with antagonism by the scientific \& medical establishments, \& up until 1939, there were only 6 psychology chairs in universities in England.

During World War II \& the Cold War, the U.S. military \& intelligence agencies established themselves as leading funders of psychology by way of the armed forces \& in the new \href{https://en.wikipedia.org/wiki/Office_of_Strategic_Services}{Office of Strategic Services} intelligence agency. University of Michigan psychologist Dorwin Cartwright reported that university researchers began large-scale propaganda research in 1939--1941. He observed that ``the last few months of the war saw a social psychologist become chiefly responsible for determining the week-by-week-propaganda policy for the United States Government.'' Cartwright also wrote that psychologists had significant roles in managing the domestic economy. The Army rolled out its new \href{https://en.wikipedia.org/wiki/Army_General_Classification_Test}{General Classification Test} to assess the ability of millions of soldiers. The Army also engaged in large-scaled psychological research of \href{https://en.wikipedia.org/wiki/Samuel_A._Stouffer#Studies_in_Social_Psychology_in_World_War_II:_The_American_Soldier}{troop morale \& mental health}. In the 1950s, the \href{https://en.wikipedia.org/wiki/Rockefeller_Foundation}{Rockefeller Foundation} \& \href{https://en.wikipedia.org/wiki/Ford_Foundation}{Ford Foundation} collaborated with the \href{https://en.wikipedia.org/wiki/Central_Intelligence_Agency}{Central Intelligence Agency} (CIA) to fund research on \href{https://en.wikipedia.org/wiki/Psychological_warfare}{psychological warfare}. In 1965, public controversy called attention to the Army's \href{https://en.wikipedia.org/wiki/Project_Camelot}{Project Camelot}, the ``Manhattan Project'' of social science, an effort which enlisted psychologists \& anthropologists to analyze the plans \& policies of foreign countries for strategic purposes.

In Germany after World War I, psychology held institutional power through the military, which was subsequently expanded along with the rest of the military during \href{https://en.wikipedia.org/wiki/Nazi_Germany}{Nazi Germany}. Under the direction of \href{https://en.wikipedia.org/wiki/Hermann_G%C3%B6ring}{Hermann G\"oring}'s cousin \href{https://en.wikipedia.org/wiki/Matthias_G%C3%B6ring}{Matthias G\"oring}, the \href{https://en.wikipedia.org/wiki/Berlin_Psychoanalytic_Institute}{Berlin Psychoanalytic Institute} was renamed the G\"oring Institute. \href{https://en.wikipedia.org/wiki/Freudian_psychoanalysis}{Freudian psychoanalysts} were expelled \& persecuted under the anti-Jewish policies of the \href{https://en.wikipedia.org/wiki/Nazi_Party}{Nazi Party}, \& all psychologists had to distance themselves from \href{https://en.wikipedia.org/wiki/Sigmund_Freud}{Freud} \& \href{https://en.wikipedia.org/wiki/Alfred_Adler}{Adler}, founders of \href{https://en.wikipedia.org/wiki/Psychoanalysis}{psychoanalysis} who were also Jewish. The G\"oring Institute was well-financed throughout the war with a mandate to create a ``New German Psychotherapy.'' This psychotherapy aimed to align suitable Germans with the overall goals of the Reich. As described by 1 physician, ``Despite the importance of analysis, spiritual guidance \& the active cooperation of the patient represent the best way to overcome individual mental problems \& to subordinate them to the requirements of the \href{https://en.wikipedia.org/wiki/Volk}{Volk} \& the \href{https://en.wikipedia.org/wiki/Gemeinschaft_and_Gesellschaft}{Gemeinschaft}.'' Psychologists were to provide \textit{Seelenf\"uhrung} [lit., soul guidance], the leadership of the mind, to integrate people into the new vision of a German community. \href{https://en.wikipedia.org/wiki/Harald_Schultz-Hencke}{Harald Schultz-Hencke} melded psychology with the Nazi theory of biology \& racial origins, criticizing psychoanalysis as a study of the weak \& deformed. \href{https://en.wikipedia.org/wiki/Johannes_Heinrich_Schultz}{Johannes Heinrich Schultz}, a German psychologist recognized for developing the technique of \href{https://en.wikipedia.org/wiki/Autogenic_training}{autogenic training}, prominently advocated sterilization \& euthanasia of men considered genetically undesirable, \& devised techniques for facilitating this process.

After the war, new institutions were created although some psychologists, because of their Nazi affiliation, were discredited. \href{https://en.wikipedia.org/wiki/Alexander_Mitscherlich_(psychologist)}{Alexander Mitscherlich} founded a prominent applied psychoanalysis journal called \textit{Psyche}. With funding from the Rockefeller Foundation, Mitscherlich established the 1st clinical psychosomatic medicine division at Heidelberg University. In 1970, psychology was integrated into the required studies of medical students.

After the \href{https://en.wikipedia.org/wiki/Russian_Revolution}{Russian Revolution}, the \href{https://en.wikipedia.org/wiki/Bolsheviks}{Bolsheviks} promoted psychology as a way to engineer the ``New Man'' of socialism. Consequently, university psychology departments trained large numbers of students in psychology. At the completion of training, positions were made available for those students at schools, workplaces, cultural institutions, \& in the military. The Russian state emphasized \href{https://en.wikipedia.org/wiki/Pedology_(children_study)}{pedology} \& the study of child development. \href{https://en.wikipedia.org/wiki/Lev_Vygotsky}{Lev Vygotsky} became prominent in the field of child development. The Bolsheviks also promoted \href{https://en.wikipedia.org/wiki/Free_love}{free love} \& embraced the doctrine of psychoanalysis as an antidote to sexual repression. Although pedology \& intelligence testing fell out of favor in 1936, psychology maintained its privileged position as an instrument of the Soviet Union. \href{https://en.wikipedia.org/wiki/Stalinist_purges}{Stalinist purges} took a heavy toll \& instilled a climate of fear in the profession, as elsewhere in Soviet society. Following World War II, Jewish psychologists past \& present, including \href{https://en.wikipedia.org/wiki/Lev_Vygotsky}{Lev Vygotsky}, \href{https://en.wikipedia.org/wiki/Alexander_Luria}{A. R. Luria}, \& Aron Zalkind, were denounced; Ivan Pavlov (posthumously) \& Stalin himself were celebrated as heroes of Soviet psychology. Soviet academics experienced a degree of liberalization during the \href{https://en.wikipedia.org/wiki/Khrushchev_Thaw}{Khrushchev Thaw}. The topics of cybernetics, linguistics, \& genetics became acceptable again. The new field of \href{https://en.wikipedia.org/wiki/Engineering_psychology}{engineering psychology} emerged. The field involved the study of the mental aspects of complex jobs (such as pilot \& cosmonaut). Interdisciplinary studies became popular \& scholars such as \href{https://en.wikipedia.org/wiki/Georgy_Shchedrovitsky}{Georgy Shchedrovitsky} developed systems theory approaches to human behavior.

20th-century Chinese psychology originally modeled itself on U.S. psychology, with translations from American authors like William James, the establishment of university psychology departments \& journals, \& the establishment of groups including the Chinese Association of Psychological Testin (1930) \& the \href{https://en.wikipedia.org/wiki/Chinese_Psychological_Society}{Chinese Psychological Society} (1937). Chinese psychologists were encouraged to focus on education \& language learning. Chinese psychologists were drawn to the idea that \fbox{education would enable modernization}. John Dewey, who lectured to Chinese audiences between 1919 \& 1921, had a significant influence on psychology in China. Chancellor \href{https://en.wikipedia.org/wiki/Cai_Yuanpei}{T'sai Yuan-p'ei} introduced him at \href{https://en.wikipedia.org/wiki/Peking_University}{Peking University} as a greater thinker than Confucius. \href{https://en.wikipedia.org/wiki/Zing-Yang_Kuo}{Kuo Zing-yang} who received a PhD at the University of California, Berkeley, became President of \href{https://en.wikipedia.org/wiki/Zhejiang_University}{Zhejiang University} \& popularized \href{https://en.wikipedia.org/wiki/Behaviorism}{behaviorism}. After the \href{https://en.wikipedia.org/wiki/Chinese_Communist_Party}{Chinese Communist Party} gained control of the country, the Stalinist Soviet Union became the major influence, with \href{https://en.wikipedia.org/wiki/Marxism%E2%80%93Leninism}{Marxism--Leninism} the leading social doctrine \& Pavlovian conditioning the approved means of behavior change. Chinese psychologists elaborated on Lenin's model of a ``reflective'' consciousness, envisioning an ``active consciousness'' (\href{https://en.wikipedia.org/wiki/Pinyin}{pinyin}: \textit{tzu-chueh neng-tung-li}) able to transcend material conditions through hard work \& ideological struggle. They developed a concept of ``recognition'' (pinyin: \textit{jen-shih}) which referred to the interface between individual perceptions \& the socially accepted worldview; failure to correspond with party doctrine was ``incorrect recognition.'' Psychology education was centralized under the \href{https://en.wikipedia.org/wiki/Chinese_Academy_of_Sciences}{Chinese Academy of Sciences}, supervised by the \href{https://en.wikipedia.org/wiki/State_Council_of_the_People%27s_Republic_of_China}{State Council}. In 1951, the academy created a Psychology Research Office, which in 1956 became the Institute of Psychology. Because most leading psychologists were educated in the United States, the 1st concern of the academy was the re-education of these psychologists in the Soviet doctrines. Child psychology \& pedagogy for the purpose of a nationally cohesive education remained a central goal of the discipline.'' -- \href{https://en.wikipedia.org/wiki/Psychology#Consolidation_and_funding}{Wikipedia\texttt{/}history\texttt{/}consolidation \& funding}

\subsection{Disciplinary Organization}

\subsubsection{Institutions}
``See also: \href{https://en.wikipedia.org/wiki/List_of_psychology_organizations}{Wikipedia\texttt{/}list of psychology organizations}. In 1920, \href{https://en.wikipedia.org/wiki/%C3%89douard_Clapar%C3%A8de}{\'Edouard Clapar\`ede} \& \href{https://en.wikipedia.org/wiki/Pierre_Bovet}{Pierre Bovet} created a new applied psychology organization called the International Congress of Psychotechnics Applied to Vocational Guidance, later called the International Congress of Psychotechnics \& then the \href{https://en.wikipedia.org/wiki/International_Association_of_Applied_Psychology}{International Association of Applied Psychology}. The IAAP is considered the oldest international psychology association. Today, at least 65 international groups deal with specialized aspects of psychology. In response to male predominance in the field, female psychologists in the U.S. formed the National Council of Women Psychologists in 1941. This organization became the International Council of Women Psychologists after World War II \& the International Council of Psychologists in 1959. Several associations including the \href{https://en.wikipedia.org/wiki/Association_of_Black_Psychologists}{Association of Black Psychologists} \& the Asian American Psychological Association have arisen to promote the inclusion of non-European racial groups in the profession.

The \href{https://en.wikipedia.org/wiki/International_Union_of_Psychological_Science}{International Union of Psychological Science} (IUPsyS) is the world federation of national psychological societies. The IUPsyS was founded in 1951 under the auspices of the \href{https://en.wikipedia.org/wiki/UNESCO}{United Nations Educational, Cultural \& Scientific Organization (UNESCO)}. Psychology departments have since proliferated around the world, based primarily on the Euro-American model. Since 1966, the Union has published the \textit{International Journal of Psychology}. IAAP \& IUPsyS agreed in 1976 each to hold a congress every 4 years, on a staggered basis.

IUPsyS recognizes 66 national psychology associations \& at least 15 others exist. The American Psychological Association is the oldest \& largest. Its membership has increased from 5,000 in 1945 to 100,000 in the present day. The APA includes \href{https://en.wikipedia.org/wiki/Divisions_of_the_American_Psychological_Association}{54 divisions}, which since 1960 have steadily proliferated to include more specialties. Some of these divisions, such as the \href{https://en.wikipedia.org/wiki/Society_for_the_Psychological_Study_of_Social_Issues}{Society for the Psychological Study of Social Issues} \& the \href{https://en.wikipedia.org/wiki/American_Psychology%E2%80%93Law_Society}{American Psychology--Law Society}, began as autonomous groups.

The \href{https://en.wikipedia.org/wiki/Interamerican_Psychological_Society}{Interamerican Psychological Society}, founded in 1951, aspires to promote psychology across the Western Hemisphere. It holds the Interamerican Congress of Psychology \& has had 1,000 members in year 2000. The European Federation of Professional Psychology Associations, founded in 1981, represents 30 national associations with a total of 100,000 individual members. At least 30 other international organizations represent psychologists in different regions.

In some places, governments legally regulate who can provide psychological services or represent themselves as a ``psychologist.'' The APA defines a psychologist as someone with a doctoral degree in psychology.'' -- \href{https://en.wikipedia.org/wiki/Psychology#Institutions}{Wikipedia\texttt{/}psychology\texttt{/}disciplinary organization\texttt{/}institutions}

\subsubsection{Boundaries}
``Early practitioners of experimental psychology distinguished themselves from \href{https://en.wikipedia.org/wiki/Parapsychology}{parapsychology}, which in the late 19th century enjoyed popularity (including the interest of scholars such as William James). Some people considered parapsychology to be part of ``psychology.'' Parapsychology, hypnotism, \& \href{https://en.wikipedia.org/wiki/Psychic}{psychism} were major topics at the early International Congresses. But students of these fields were eventually ostracized, \& more or less banished from the Congress in 1900--1905. Parapsychology persisted for a time at Imperial University in Japan, with publications such as \textit{Clairvoyance \& Thoughtography} by Tomokichi Fukurai, but it was mostly shunned by 1913.

As a discipline, psychology has long sought to fend off accusations that it is a ``soft'' science. Philosopher of science \href{https://en.wikipedia.org/wiki/Thomas_Kuhn}{Thomas Kuhn}'s 1962 critique implied psychology overall was in a pre-paradigm state, lacking agreement on the type of overarching theory found in mature sciences such as chemistry \& physics. Because some areas of psychology rely on research methods such as surveys \& questionnaires, critics asserted that psychology is not an objective science. Skeptics have suggested that personality, thinking, \& emotion cannot be directly measured \& are often inferred from subjective self-reports, which may be problematic. Experimental psychologists have devised a variety of ways to indirectly measure these elusive phenomenological entities.

Divisions still exist within the field, with some psychologists more oriented towards the unique experiences of individual humans, which cannot be understood only as data points within a larger population. Critics inside \& outside the field have argued that mainstream psychology has become increasingly dominated by a ``cult of empiricism,'' which limits the scope of research because investigators restrict themselves to methods derived from the physical sciences. Feminist critiques have argued that claims to scientific objectivity obscure the values \& agenda of (historically) mostly male researchers. Jean Grimshaw, e.g., argues that mainstream psychological research has advanced a \href{https://en.wikipedia.org/wiki/Patriarchal}{patriarchal} agenda through its efforts to control behavior.'' -- \href{https://en.wikipedia.org/wiki/Psychology#Boundaries}{Wikipedia\texttt{/}psychology\texttt{/}disciplinary organization\texttt{/}boundaries}

\subsection{Major Schools of Thought}

\subsubsection{Biological}
\textsf{Fig. False-color representations of \href{https://en.wikipedia.org/wiki/White_matter}{cerebral fiber} pathways affected, per Van Horn et al.}

``Main article: \href{https://en.wikipedia.org/wiki/Cognitive_neuroscience}{Wikipedia\texttt{/}cognitive neuroscience}. Psychologists generally consider biology the substrate of thought \& feeling, \& therefore an important area of study. Behavioral neuroscience, also known as \textit{biological psychology}, involves the application of biological principles to the study of physiological \& genetic mechanisms underlying behavior in humans \& other animals. The allied field of \href{https://en.wikipedia.org/wiki/Comparative_psychology}{comparative psychology} is the scientific study of the behavior \& mental processes of non-human animals. A leading question in behavioral neuroscience has been whether \& how mental functions are \href{https://en.wikipedia.org/wiki/Functional_specialization_(brain)}{localized in the brain}. From \href{https://en.wikipedia.org/wiki/Phineas_Gage}{Phineas Gage} to \href{https://en.wikipedia.org/wiki/Henry_Molaison}{H.M.} \& \href{https://en.wikipedia.org/wiki/Clive_Wearing}{Clive Wearing}, individual people with mental deficits traceable to physical brain damage have inspired new discoveries in this area. Modern behavioral neuroscience could be said to originate in the 1870s, when in France \href{https://en.wikipedia.org/wiki/Paul_Broca}{Paul Broca} traced production of speech to the left frontal gyrus, thereby also demonstrating hemispheric lateralization of brain function. Soon after, \href{https://en.wikipedia.org/wiki/Carl_Wernicke}{Carl Wernicke} identified a related area necessary for the understanding of speech.

The contemporary field of \href{https://en.wikipedia.org/wiki/Behavioral_neuroscience}{behavioral neuroscience} focuses on the physical basis of behavior. Behavioral neuroscientists use animal models, often relying on rats, to study the neural, genetic, \& cellular mechanisms that underlie behaviors involved in learning, memory, \& fear responses. \href{https://en.wikipedia.org/wiki/Cognitive_neuroscience}{Cognitive neuroscientists}, by using neural imaging tools, investigate the neural correlates of psychological processes in humans. \href{https://en.wikipedia.org/wiki/Neuropsychology}{Neuropsychologists} conduct psychological assessments to determine how an individual's behavior \& cognition are related to the brain. The \href{https://en.wikipedia.org/wiki/Biopsychosocial_model}{biopsychosocial model} is a cross-disciplinary, holistic model that concerns the ways in which interrelationships of biological, psychological, \& socio-environmental factors affect health \& behavior.

\href{https://en.wikipedia.org/wiki/Evolutionary_psychology}{Evolutionary psychology} approaches thought \& behavior from a modern \href{https://en.wikipedia.org/wiki/Evolution}{evolutionary} perspective. This perspective suggests that psychological adaptations evolved to solve recurrent problems in human ancestral environments. Evolutionary psychologists attempt to find out how human psychological traits are evolved adaptations, the results of \href{https://en.wikipedia.org/wiki/Natural_selection}{natural selection} or \href{https://en.wikipedia.org/wiki/Sexual_selection}{sexual selection} over the course of human evolution.

The history of the biological foundations of psychology includes evidence of racism. The idea of white supremacy \& indeed the modern concept of race itself arose during the process of world conquest by Europeans. \href{https://en.wikipedia.org/wiki/Carl_von_Linnaeus}{Carl von Linnaeus}'s 4-fold classification of humans classifies Europeans as intelligent \& severe, Americans as contented \& free, Asians as ritualistic, \& Africans as lazy \& capricious. Race was also used to justify the construction of socially specific mental disorders such as \href{https://en.wikipedia.org/wiki/Drapetomania}{drapetomania} \& \href{https://en.wikipedia.org/wiki/Dysaesthesia_aethiopica}{\textit{dysaesthesia aethiopica}} -- the behavior of uncooperative African slaves. After the creation of experimental psychology, ``ethnical psychology'' emerged as a subdiscipline, based on the assumption that studying primitive races would provide an important link between animal behavior \& the psychology of more evolved humans.'' -- \href{https://en.wikipedia.org/wiki/Psychology#Biological}{Wikipedia\texttt{/}major schools of thought\texttt{/}biological}

\subsubsection{Behaviorist}
\textsf{Fig. Skinner's \href{https://en.wikipedia.org/wiki/Teaching_machine}{teaching machine}, a mechanical invention to automate the task of \href{https://en.wikipedia.org/wiki/Programmed_instruction}{programmed instruction}.}

``Main article: \href{https://en.wikipedia.org/wiki/Behaviorism}{Wikipedia\texttt{/}behaviorism}, \href{https://en.wikipedia.org/wiki/Psychological_behaviorism}{Psychological behaviorism}, \& \href{https://en.wikipedia.org/wiki/Radical_behaviorism}{Radical behaviorism}. A tenet of behavior research is that a large part of both human \& lower-animal behavior is learned. A principle associated with behavioral research is that the mechanisms involved in learning apply to humans \& non-human animals. Behavioral researchers have developed a treatment known as \href{https://en.wikipedia.org/wiki/Behavior_modification}{behavior modification}, which is used to help individuals replace undesirable behaviors with desirable ones.

Early behavioral researchers studied stimulus-response pairings, now known as \href{https://en.wikipedia.org/wiki/Classical_conditioning}{classical conditioning}. They demonstrated that when a biologically potent stimulus (e.g., food that elicits salivation) is paired with a previously neutral stimulus (e.g., a bell) over several learning trials, the neutral stimulus by itself can come to elicit the response the biologically potent stimulus elicits. \href{https://en.wikipedia.org/wiki/Ivan_Pavlov}{Ivan Pavlov} -- known best for inducing dogs to salivate in the presence of a stimulus previously linked with food -- became a leading figure in the Soviet Union \& inspired followers to use his methods on humans. In the United States, \href{https://en.wikipedia.org/wiki/Edward_Lee_Thorndike}{Edward Lee Thorndike} initiated ``\href{https://en.wikipedia.org/wiki/Connectionism}{connectionist}'' studied by trapping animals in ``puzzle boxes'' \& rewarding them for escaping. Thorndike wrote in 1911, ``There can be no moral warrant for studying man's nature unless the study will enable us to control his acts.'' From 1910 to 1913 the American Psychological Association went through a sea change of opinion, away from \href{https://en.wikipedia.org/wiki/Mentalism_(psychology)}{mentalism} \& towards ``behavioralism.'' In 1913, John B. Watson coined the term behaviorism for this school of thought. Watson's famous \href{https://en.wikipedia.org/wiki/Little_Albert_experiment}{Little Albert experiment} [\textsf{Video. The film of the Little Albert experiment.}] in 1920 was at 1st thought to demonstrate that repeated use of upsetting loud noises could instill \href{https://en.wikipedia.org/wiki/Phobia}{phobias} (aversions to other stimuli) in an infant human, although such a conclusion was likely an exaggeration. \href{https://en.wikipedia.org/wiki/Karl_Lashley}{Karl Lashley}, a close collaborator with Watson, examined biological manifestations of learning in the brain.

\href{https://en.wikipedia.org/wiki/Clark_L._Hull}{Clark L. Hull}, \href{https://en.wikipedia.org/wiki/Edwin_Guthrie}{Edwin Guthrie}, \& others did much to help behaviorism become a widely used paradigm. A new method of ``instrumental'' or ``\href{https://en.wikipedia.org/wiki/Operant_conditioning}{operant}'' conditioning added the concepts of \href{https://en.wikipedia.org/wiki/Reinforcement}{reinforcement} \& \href{https://en.wikipedia.org/wiki/Punishment}{punishment} to the model of behavior change. \href{https://en.wikipedia.org/wiki/Radical_behaviorism}{Radical behaviorists} avoided discussing the inner workings of the mind, especially the unconscious mind, which they considered impossible to assess scientifically. Operant conditioning was 1st described by Miller \& Kanorski \& popularized in the U.S. by \href{https://en.wikipedia.org/wiki/B.F._Skinner}{B.F. Skinner}, who emerged as a leading intellectual of the behaviorist movement.

\href{https://en.wikipedia.org/wiki/Noam_Chomsky}{Noam Chomsky} published an influential critique of radical behaviorism on the grounds that behaviorist principles could not adequately explain the complex mental process of \href{https://en.wikipedia.org/wiki/Language_acquisition}{language acquisition} \& language use. The review, which was scathing, did much to reduce the status of behaviorism within psychology. \href{https://en.wikipedia.org/wiki/Martin_Seligman}{Martin Seligman} \& his colleagues discovered that they could condition in dogs a state of ``\href{https://en.wikipedia.org/wiki/Learned_helplessness}{learned helplessness}'', which was not predicted by the behaviorist approach to psychology. \href{https://en.wikipedia.org/wiki/Edward_C._Tolman}{Edward C. Tolman} advanced a hybrid ``cognitive behavioral'' model, most notably with his 1948 publication discussing the \href{https://en.wikipedia.org/wiki/Cognitive_map}{cognitive maps} used by rats to guess at the location of food at the end of a maze. Skinner's behaviorism did not die, in part because it generated successful practical applications.

The \href{https://en.wikipedia.org/wiki/Association_for_Behavior_Analysis_International}{Association for Behavior Analysis International} was founded in 1974 \& by 2003 had members from 42 countries. The field has gained a foothold in Latin America \& Japan. \href{https://en.wikipedia.org/wiki/Applied_behavior_analysis}{Applied behavior analysis} is the term used for the application of the principles of operant conditioning to change socially significant behavior (it supersedes the term, ``behavior modification'').'' -- \href{https://en.wikipedia.org/wiki/Psychology#Behaviorist}{Wikipedia\texttt{/}major schools of thought\texttt{/}behaviorist}

\subsubsection{Cognitive}
``Main article: \href{https://en.wikipedia.org/wiki/Cognitive_psychology}{Wikipedia\texttt{/}cognitive psychology}. Cognitive psychology involves the study of \href{https://en.wikipedia.org/wiki/Mental_process}{mental processes}, including \href{https://en.wikipedia.org/wiki/Perception}{perception}, \href{https://en.wikipedia.org/wiki/Attention}{attention}, language comprehension \& production, \href{https://en.wikipedia.org/wiki/Memory}{memory}, \& problem solving. Researchers in the field of cognitive psychology are sometimes called \href{https://en.wikipedia.org/wiki/Cognitivism_(psychology)}{cognitivists}. They rely on an \href{https://en.wikipedia.org/wiki/Information_processing}{information processing} model of mental functioning. Cognitivist research is informed by \href{https://en.wikipedia.org/wiki/Functionalism_(philosophy_of_mind)}{functionalism} \& experimental psychology.

Starting in the 1950s, the experimental techniques developed by Wundt, James, Ebbinghaus, \& others re-emerged as experimental psychology became increasingly cognitivist \&, eventually, constituted a part of the wider, interdisciplinary \href{https://en.wikipedia.org/wiki/Cognitive_science}{cognitive science}. Some called this development the \href{https://en.wikipedia.org/wiki/Cognitive_revolution}{cognitive revolution} because it rejected the anti-mentalist dogma of behaviorism as well as the strictures of psychoanalysis.

\textsf{Fig. The Stroop effect is the fact that naming the color of the 1st set of words is easier \& quicker than the 2nd. Fig. \href{https://en.wikipedia.org/wiki/Baddeley's_model_of_working_memory}{Baddeley's model of working memory}.}

\href{https://en.wikipedia.org/wiki/Albert_Bandura}{Albert Bandura} helped along the transition in psychology from behaviorism to cognitive psychology. Bandura \& other \href{https://en.wikipedia.org/wiki/Social_learning_theory}{social learning theorists} advanced the idea of vicarious learning. In other words, they advanced the view that a child can learn by observing his or her social environment \& not necessarily from having been reinforced for enacting a behavior, although they did not rule out the influence of reinforcement on learning a behavior.

\textsf{Fig. The \href{https://en.wikipedia.org/wiki/Muller-Lyer_illusion}{M\"uller--Lyer illusion}. Psychologists make inferences about mental processes from shared phenomena such as optical illusions.}

Technological advances also renewed interest in mental states \& mental representations. English neuroscientist \href{https://en.wikipedia.org/wiki/Charles_Sherrington}{Charles Sherrington} \& Canadian psychologist \href{https://en.wikipedia.org/wiki/Donald_O._Hebb}{Donald O. Hebb} used experimental methods to link psychological phenomena to the structure \& function of the brain. The rise of computer science, cybernetics, \& artificial intelligence underlined the value of comparing information processing in humans \& machines.

A popular \& representative topic in this area is \href{https://en.wikipedia.org/wiki/Cognitive_bias}{cognitive bias}, or irrational thought. Psychologists (\& economists) have classified \& described a \href{https://en.wikipedia.org/wiki/List_of_cognitive_biases}{sizeable catalogue of biases} which recur frequently in human thought. The \href{https://en.wikipedia.org/wiki/Availability_heuristic}{availability heuristic}, e.g., is the tendency to overestimate the importance of something which happens to come readily to mind.

Elements of behaviorism \& cognitive psychology were synthesized to form \href{https://en.wikipedia.org/wiki/Cognitive_behavioral_therapy}{cognitive behaviorial therapy}, a form of psychotherapy modified from techniques developed by American psychologist \href{https://en.wikipedia.org/wiki/Albert_Ellis_(psychologist)}{Albert Ellis} \& American psychiatrist \href{https://en.wikipedia.org/wiki/Aaron_Beck}{Aaron T. Beck}.

On a broader level, cognitive science is an interdisciplinary enterprise involving cognitive psychologists, cognitive neuroscientists, linguists, \& researchers in artificial intelligence, human--computer interaction, \& \href{https://en.wikipedia.org/wiki/Computational_neuroscience}{computational neuroscience}. The discipline of cognitive science covers cognitive psychology as well as philosophy of mind, computer science, \& neuroscience. Computer simulations are sometimes used to model phenomena of interest.'' -- \href{https://en.wikipedia.org/wiki/Psychology#Cognitive}{Wikipedia\texttt{/}major schools of thought\texttt{/}cognitive}

\subsubsection{Social}
``Main article: \href{https://en.wikipedia.org/wiki/Social_psychology}{Wikipedia\texttt{/}social psychology}. See also: \href{https://en.wikipedia.org/wiki/Social_psychology_(sociology)}{Wikipedia\texttt{/}social psychology (sociology)}. Social psychology is concerned with how \href{https://en.wikipedia.org/wiki/Behavior}{behaviors}, \href{https://en.wikipedia.org/wiki/Thought}{thoguhts}, \href{https://en.wikipedia.org/wiki/Feeling}{feelings}, \& the social environment influence human interactions. Social psychologists study such topics as the influence of others on an individual's behavior (e.g., \href{https://en.wikipedia.org/wiki/Conformity_(psychology)}{conformity}, \href{https://en.wikipedia.org/wiki/Persuasion}{persuasion}) \& the formation of beliefs, \href{https://en.wikipedia.org/wiki/Attitude_(psychology)}{attitudes}, \& \href{https://en.wikipedia.org/wiki/Stereotype}{stereotypes} about other people. \href{https://en.wikipedia.org/wiki/Social_cognition}{Social cognition} fuses elements of social \& cognitive psychology for the purpose of understanding how people process, remember, or distort social information. The study of \href{https://en.wikipedia.org/wiki/Group_dynamics}{group dynamics} involves research on the nature of leadership, organizational communication, \& related phenomena. In recent years, social psychologists have become interested in \href{https://en.wikipedia.org/wiki/Implicit_Association_Test}{implicit} measures, \href{https://en.wikipedia.org/wiki/Mediation_(statistics)}{medicational} models, \& the interaction of person \& social factors in accounting for behavior. Some concepts that \href{https://en.wikipedia.org/wiki/Sociology}{sociologists} have applied to the study of psychiatric disorders, concepts such as the social role, sick role, social class, life events, culture, migration, \& \href{https://en.wikipedia.org/wiki/Total_institution}{total institution}, have influenced social psychologists.'' -- \href{https://en.wikipedia.org/wiki/Psychology#Social}{Wikipedia\texttt{/}major schools of thought\texttt{/}social}

\subsubsection{Psychoanalytic}
\textsf{Fig. Group photo 1909 in front of \href{https://en.wikipedia.org/wiki/Clark_University}{Clark University}. Front row: Sigmund Freud, G. Stanley Hall, Carl Jung; back row: \href{https://en.wikipedia.org/wiki/Abraham_A._Brill}{Abraham A. Brill}, \href{https://en.wikipedia.org/wiki/Ernest_Jones}{Ernest Jones}, \href{https://en.wikipedia.org/wiki/Sandor_Ferenczi}{S\'andor Ferenczi}.}

``Main articles: \href{https://en.wikipedia.org/wiki/Psychodynamics}{Wikipedia\texttt{/}psychodynamics} \& \href{https://en.wikipedia.org/wiki/Psychoanalysis}{Wikipedia\texttt{/}psychoanalysis}. Psychoanalysis refers to the theories \& therapeutic techniques applied to the unconscious mind \& its impact on everyday life. These theories \& techniques inform treatments for mental disorders. Psychoanalysis originated in the 1890s, most prominently with the work of \href{https://en.wikipedia.org/wiki/Sigmund_Freud}{Sigmund Freud}. Freud's psychoanalytic theory was largely based on interpretive methods, \href{https://en.wikipedia.org/wiki/Introspection}{introspection}, \& clinical observation. It became very well known, largely because it tackled subjects such as \href{https://en.wikipedia.org/wiki/Human_sexuality}{sexuality}, \href{https://en.wikipedia.org/wiki/Psychological_repression}{repression}, \& the unconscious. Freud pioneered the methods of \href{https://en.wikipedia.org/wiki/Free_association_(psychology)}{free association} \& \href{https://en.wikipedia.org/wiki/Dream_interpretation}{dream interpretation}.

Psychoanalytic theory is not monolithic. Other well-known psychoanalytic thinkers who diverged from Freud include \href{https://en.wikipedia.org/wiki/Alfred_Adler}{Alfred Adler}, \href{https://en.wikipedia.org/wiki/Carl_Jung}{Carl Jung}, \href{https://en.wikipedia.org/wiki/Erik_Erikson}{Erik Erikson}, \href{https://en.wikipedia.org/wiki/Melanie_Klein}{Melanie Klein}, \href{https://en.wikipedia.org/wiki/Donald_Winnicott}{D. W. Winnicott}, \href{https://en.wikipedia.org/wiki/Karen_Horney}{Karen Horney}, \href{https://en.wikipedia.org/wiki/Erich_Fromm}{Erich Fromm}, \href{https://en.wikipedia.org/wiki/John_Bowlby}{John Bowlby}, Freud's daughter \href{https://en.wikipedia.org/wiki/Anna_Freud}{Anna Freud}, \& \href{https://en.wikipedia.org/wiki/Harry_Stack_Sullivan}{Harry Stack Sullivan}. These individuals ensured that psychoanalysis would evolve into diverse schools of thought. Among these schools are \href{https://en.wikipedia.org/wiki/Ego_psychology}{ego psychology}, \href{https://en.wikipedia.org/wiki/Object_relations}{object relations}, \& \href{https://en.wikipedia.org/wiki/Interpersonal_psychoanalysis}{interpersonal}, \href{https://en.wikipedia.org/wiki/Jacques_Lacan}{Lacanian}, \& \href{https://en.wikipedia.org/wiki/Relational_psychoanalysis}{relational psychoanalysis}.

Psychologists such as \href{https://en.wikipedia.org/wiki/Hans_Eysenck}{Hans Eysenck} \& philosophers including \href{https://en.wikipedia.org/wiki/Karl_Popper}{Karl Popper} sharply criticized psychoanalysis. Popper argued that psychoanalysis had been misrepresented as a scientific discipline, whereas Eysenck advanced the view that psychoanalytic tenets had been contradicted by \href{https://en.wikipedia.org/wiki/Experiment}{experimental} data. By the end of the 20th century, psychology departments in \href{https://en.wikipedia.org/wiki/Higher_education_in_the_United_States}{American universities} mostly had marginalized Freudian theory, dismissing it as a ``desiccated \& dead'' historical artifact. Researchers such as \href{https://en.wikipedia.org/wiki/Ant%C3%B3nio_Dam%C3%A1sio}{Ant\'onio Dam\'asio}, \href{https://en.wikipedia.org/wiki/Oliver_Sacks}{Oliver Sacks}, \& \href{https://en.wikipedia.org/wiki/Joseph_LeDoux}{Joseph LeDoux}; \& individuals in the emerging field of \href{https://en.wikipedia.org/wiki/Neuro-psychoanalysis}{neuro-psychoanalysis} have defended some of Freud's ideas on scientific grounds.'' -- \href{https://en.wikipedia.org/wiki/Psychology#Psychoanalytic}{Wikipedia\texttt{/}major schools of thought\texttt{/}psychoanalytic}

\subsubsection{Existential-humanistic}
\textsf{Fig. Psychologist Abraham Maslow in 1943 posited that humans have a hierarchy of needs, \& it makes sense to fulfill the basic needs 1st (food, water, etc.) before higher-order needs can be met.}

``Main articles: \href{https://en.wikipedia.org/wiki/Existential_psychology}{Wikipedia\texttt{/}existential psychology} \& \href{https://en.wikipedia.org/wiki/Humanistic_psychology}{Wikipedia\texttt{/}humanistic psychology}. \href{https://en.wikipedia.org/wiki/Humanistic_psychology}{Humanistic psychology}, which has been influenced by existentialism \& phenomenology, stresses \href{https://en.wikipedia.org/wiki/Free_will}{free will} \& \href{https://en.wikipedia.org/wiki/Self-actualization}{self-actualization}. It emerged in the 1950s as a movement within academic psychology, in reaction to both behaviorism \& psychoanalysis. The humanistic approach seeks to view the whole person, not just fragmented parts of the personality or isolated cognitions. Humanistic psychology also focuses on personal growth, \href{https://en.wikipedia.org/wiki/Self-concept}{self-identity}, death, aloneness, \& freedom. It emphasizes subjective meaning, the rejection of determinism, \& concern for positive growth rather than pathology. Some founders of the humanistic school of thought were American psychologists \href{https://en.wikipedia.org/wiki/Abraham_Maslow}{Abraham Maslow}, who formulated a \href{https://en.wikipedia.org/wiki/Maslow%27s_hierarchy_of_needs}{hierarchy of human needs}, \& \href{https://en.wikipedia.org/wiki/Carl_Rogers}{Carl Rogers}, who created \& developed \href{https://en.wikipedia.org/wiki/Client-centered_therapy}{client-centered therapy}.

Later, \href{https://en.wikipedia.org/wiki/Positive_psychology}{positive psychology} opened up humanistic themes to scientific study. Positive psychology is the study of factors which contribute to human happiness \& well-being, focusing more on people who are currently healthy. In 2010, \textit{Clinical Psychological Review} published a special issue devoted to positive psychological interventions, such as \href{https://en.wikipedia.org/wiki/Gratitude_journal}{gratitude journaling} \& the physical expression of gratitude. It is, however, far from clear that positive psychology is effective is making people happier. Positive psychological interventions have been limited in scope, but their effects are thought to be somewhat better than \href{https://en.wikipedia.org/wiki/Placebo}{placebo} effects. The evidence, however, is far from clear that interventions based on positive psychology increase human happiness or resilience.

The \textit{American Association for Humanistic Psychology}, formed in 1963, declared:
\begin{quotation}
	Humanistic psychology is primarily an orientation toward the whole of psychology rather than a distinct area or school. It stands for respect for the worth of persons, respect for differences of approach, open-mindedness as to acceptable methods, \& interest in exploration of new aspects of human behavior. As a ``3rd force'' in contemporary psychology, it is concerned with topics having little place in existing theories \& systems: e.g., love, creativity, self, growth, organism, basic need-gratification, self-actualization, higher values, being, becoming, spontaneity, play, humor, affection, naturalness, warmth, ego-transcendence, objectivity, autonomy, responsibility, meaning, fair-play, transcendental experience, peak experience, courage, \& related concepts.
\end{quotation}
Existential psychology emphasizes the need to understand a client's total orientation towards the world. Existential psychology is opposed to reductionism, behaviorism, \& other methods that objectify the individual. In the 1950s \& 1960s, influenced by philosophers \href{https://en.wikipedia.org/wiki/S%C3%B8ren_Kierkegaard}{S\o ren Kierkegaard} \& \href{https://en.wikipedia.org/wiki/Martin_Heidegger}{Martin Heidegger}, psychoanalytically trained American psychologist \href{https://en.wikipedia.org/wiki/Rollo_May}{Rollo May} helped to develop existential psychology. \href{https://en.wikipedia.org/wiki/Existential_therapy}{Existential psychotheorapy}, which follows from existential psychology, is a therapeutic approach that is based on the idea that a person's inner conflict arises from that individual's confrontation with the givens of existence. Swiss psychoanalyst \href{https://en.wikipedia.org/wiki/Ludwig_Binswanger}{Ludwig Binswanger} \& American psychologist \href{https://en.wikipedia.org/wiki/George_Kelly_(psychologist)}{George Kelly} may also be said to belong to the existential school. Existential psychologists tend to differ from more ``humanistic'' psychologists in the former's relatively neutral view of human nature \& relatively positive assessment of anxiety. Existential psychologists emphasized the humanistic themes o death, free will, \& meaning, suggesting that meaning can be shaped by myths \& narratives; meaning can be deepened by the acceptance of free will, which is requisite to living an \href{https://en.wikipedia.org/wiki/Authenticity_(philosophy)}{authentic} life, albeit often with anxiety with regard to death.

Austrian existential psychiatrist \& \href{https://en.wikipedia.org/wiki/Holocaust}{Holocaust} survivor \href{https://en.wikipedia.org/wiki/Viktor_Frankl}{Viktor Frankl} drew evidence of meaning's therapeutic power from reflections upon his own \href{https://en.wikipedia.org/wiki/Internment}{internment}. He created a variation of existential psychotherapy called \href{https://en.wikipedia.org/wiki/Logotherapy}{logotherapy}, a type of \href{https://en.wikipedia.org/wiki/Existentialism}{existentialist} analysis that focuses on a \textit{will to meaning} (in one's life), as opposed to Adler's \href{https://en.wikipedia.org/wiki/Nietzsche}{Nietzschean} doctrine of \href{https://en.wikipedia.org/wiki/Will_to_power}{\textit{will to power}} or Freud's \href{https://en.wikipedia.org/wiki/Pleasure_principle_(psychology)}{will to pleasure}.'' -- \href{https://en.wikipedia.org/wiki/Psychology#Existential-humanistic}{Wikipedia\texttt{/}major schools of thought\texttt{/}existential-humanistic}

\subsection{Themes}

\subsubsection{Personality}
``Main article: \href{https://en.wikipedia.org/wiki/Personality_psychology}{Wikipedia\texttt{/}personality psychology}. Personality psychology is concerned with enduring patterns of behavior, thought, \& emotion. Theories of personality vary across different psychological schools of thought. Each theory carries different assumptions about such features as the role of the unconscious \& the importance of childhood experience. According to Freud, personality is based on the dynamic interactions of the \href{https://en.wikipedia.org/wiki/Id,_ego,_and_super-ego}{id, ego, \& super-ego}. By contrast, \href{https://en.wikipedia.org/wiki/Trait_theorist}{trait theorists} have developed taxonomies of personality constructs in describing personality in terms of key traits. Trait theorists have often employed statistical data-reduction methods, such as \href{https://en.wikipedia.org/wiki/Factor_analysis}{factor analysis}. Although the number of proposed traits has varied widely, \href{https://en.wikipedia.org/wiki/Hans_Eysenck}{Hans Eysenck}'s early biologically-based model suggests at least 3 major trait constructs are necessary to describe human personality, \href{https://en.wikipedia.org/wiki/Extraversion_and_introversion}{extraversion--introversion}, \href{https://en.wikipedia.org/wiki/Neuroticism}{neuroticism}-stability, \& \href{https://en.wikipedia.org/wiki/Psychoticism}{psychoticism}-normality. \href{https://en.wikipedia.org/wiki/Raymond_Cattell}{Raymond Cattell} empirically derived a theory of \href{https://en.wikipedia.org/wiki/16_personality_factors}{16 personality factors} at the primary-factor level \& up to 8 broader second-stratum factors. Since 1980s, the \href{https://en.wikipedia.org/wiki/Big_Five_personality_traits}{Big 5} (\href{https://en.wikipedia.org/wiki/Openness_to_experience}{openness to experience}, \href{https://en.wikipedia.org/wiki/Conscientiousness}{conscientiousness}, \href{https://en.wikipedia.org/wiki/Extraversion_and_introversion}{extraversion}, \href{https://en.wikipedia.org/wiki/Agreeableness}{agreeableeness}, \& \href{https://en.wikipedia.org/wiki/Neuroticism}{neuroticism}) emerged as an important trait theory of personality. Dimensional models of personality are receiving increasing support, \& a version of dimensional assessment has been included in the \href{https://en.wikipedia.org/wiki/DSM-V}{DSM-V}. However, despite a plethora of research into the various versions of the ``Big 5'' personality dimensions, it appears necessary to move on from static conceptualizations of personality structure to a more dynamic orientation, acknowledging that personality constructs are subject to learning \& change over the lifespan.

An early example of personality assessment was the \href{https://en.wikipedia.org/wiki/Woodworth_Personal_Data_Sheet}{Woodworth Personal Data Sheet}, constructed during World War I. The popular, although psychometrically inadequate, \href{https://en.wikipedia.org/wiki/Myers%E2%80%93Briggs_Type_Indicator}{Myers--Briggs Type Indicator} was developed to assess individuals' ``personality types'' according to the \href{https://en.wikipedia.org/wiki/Psychological_Types}{personality theories of Carl Jung}. The \href{https://en.wikipedia.org/wiki/Minnesota_Multiphasic_Personality_Inventory}{Minnesota Multiphasic Personality Inventory} (MMPI), despite its name, is more a dimensional measure o f psychopathology than a personality measure. \href{https://en.wikipedia.org/wiki/California_Psychological_Inventory}{California Psychological Inventory} contains 20 personality scales (e.g., independence, tolerance). The \href{https://en.wikipedia.org/wiki/International_Personality_Item_Pool}{International Personality Item Pool}, which is in the public domain, has become a source of scales that can be used personality assessment.'' -- \href{https://en.wikipedia.org/wiki/Psychology#Personality}{Wikipedia\texttt{/}psychology\texttt{/}themes\texttt{/}personality}

\subsubsection{Unconscious mind}
``See also: \href{https://en.wikipedia.org/wiki/Unconscious_mind#Psychology}{Wikipedia\texttt{/}unconscious mind\texttt{/}psychology}. Study of the unconscious mind, a part of the psyche outside the individual's awareness but that is believed to influence conscious thought \& behavior, was a hallmark of early psychology. In 1 of the 1st psychology experiments conducted in the United States, \href{https://en.wikipedia.org/wiki/C.S._Peirce}{C.S. Peirce} \& \href{https://en.wikipedia.org/wiki/Joseph_Jastrow}{Joseph Jastrow} found in 1884 that research subjects could choose the minutely heavier of 2 weights even if consciously uncertain of the difference. Freud popularized the concept of the unconscious mind, particularly when he referred to an uncensored intrusion of unconscious thought into one's speech (a \href{https://en.wikipedia.org/wiki/Freudian_slip}{Freudian slip}) or to his efforts \href{https://en.wikipedia.org/wiki/The_Interpretation_of_Dreams}{to interpret dreams}. His 1901 book \href{https://en.wikipedia.org/wiki/The_Psychopathology_of_Everyday_Life}{\textit{The Psychopathology of Everyday Life}} catalogues hundreds of everyday events that Freud explains in terms of unconscious influence. \href{https://en.wikipedia.org/wiki/Pierre_Janet}{Pierre Janet} advanced the idea of a subconscious mind, which could contain autonomous mental elements unavailable to the direct scrutiny of the subject.

The concept of unconscious processes has remained important in psychology. Cognitive psychologists have used a ``filter'' model of attention. According to the model, much information processing takes place below the threshold of consciousness, \& only certain stimuli, limited by their nature \& number, make their way through the filter. Much research has shown that subconscious \href{https://en.wikipedia.org/wiki/Priming_(psychology)}{\textit{priming}} of certain ideas can covertly influence thoughts \& behavior. Because of the unreliability of self-reporting, a major hurdle in his type of research involves demonstrating that a subject's conscious mind has not perceived a target stimulus. For this reason, some psychologists prefer to distinguish between \href{https://en.wikipedia.org/wiki/Implicit_memory}{\textit{implicit}} \& \href{https://en.wikipedia.org/wiki/Explicit_memory}{\textit{explicit}} memory. In another approach, one can also describe a \href{https://en.wikipedia.org/wiki/Subliminal_stimulus}{subliminal stimulus} as meeting an \textit{objective} but not a \textit{subjective} threshold.

The \href{https://en.wikipedia.org/wiki/Automaticity}{automaticity} model of \href{https://en.wikipedia.org/wiki/John_Bargh}{John Bargh} \& others involves the ideas of automatically \& unconscious processing in our understanding of \href{https://en.wikipedia.org/wiki/Social_behavior}{social behavior}, although there has been dispute with regard to replication. Some experimental data suggest that the \href{https://en.wikipedia.org/wiki/Neuroscience_of_free_will}{brain begins to consider taking actions} before the mind becomes aware of them. The influence of unconscious forces on people's choices bears on the philosophical question of free will. John Bargh, \href{https://en.wikipedia.org/wiki/Daniel_Wegner}{Daniel Wegner}, \& \href{https://en.wikipedia.org/wiki/Illusion_of_control}{Ellen Langer describe free will as an illusion}.'' -- \href{https://en.wikipedia.org/wiki/Psychology#Unconscious_mind}{Wikipedia\texttt{/}psychology\texttt{/}themes\texttt{/}unconscious mind}

\subsubsection{Motivation}
``Main article: \href{https://en.wikipedia.org/wiki/Motivation}{Wikipedia\texttt{/}motivation}. Some psychologist study motivation or the subject of why people or lower animals initiate a behavior at a particular time. It also involves the study of why humans \& lower animals continue or terminate a behavior. Psychologists such as William James initially used the term \textit{motivation} to refer to intention, in a sense similar to the concept of \href{https://en.wikipedia.org/wiki/Will_(philosophy)}{\textit{will}} in European philosophy. With the steady rise of Darwinian \& Freudian thinking, instinct also came to be seen as a primary source of motivation. According to \href{https://en.wikipedia.org/wiki/Drive_theory}{drive theory}, the forces of instinct combine into a single source of energy which exerts a constant influence. Psychoanalysis, like biology, regarded these forces as demands originating in the nervous system. Psychoanalysts believed that these forces, especially the sexual instincts, could become entangled \& transmuted within the psyche. Classical psychoanalysis conceives of a struggle between the pleasure principle \& the \href{https://en.wikipedia.org/wiki/Reality_principle}{reality principle}, roughly corresponding to id \& ego. Later, in \href{https://en.wikipedia.org/wiki/Beyond_the_Pleasure_Principle}{\textit{Beyond the Pleasure Principle}}, Freud introduced the concept of the \href{https://en.wikipedia.org/wiki/Death_drive}{death drive}, a compulsion towards aggression, destruction, \& \href{https://en.wikipedia.org/wiki/Repetition_compulsion}{psychic repetition of traumatic events}. Meanwhile, behaviorist researchers used simple dichotomous models (pleasure\texttt{/}pain, reward\texttt{/}punishment) \& well-established principles such as the idea that a thirsty creature will take pleasure in drinking. \href{https://en.wikipedia.org/wiki/Clark_Hull}{Clark Hull} formalized the latter idea with his \href{https://en.wikipedia.org/wiki/Drive_reduction_theory_(learning_theory)}{drive reduction} model.

Hunger, thirst, fear, sexual desire, \& thermoregulation constitute fundamental motivations in animals. Humans seem to exhibit a more complex set of motivations -- though theoretically these could be explained as resulting from desires for belonging, positive self-image, self-consistency, truth, love, \& control.

Motivation can be modulated or manipulated in many different ways. Researchers have found that \href{https://en.wikipedia.org/wiki/Eating}{eating}, e.g., depends not only on the organism's fundamental need for \href{https://en.wikipedia.org/wiki/Homeostasis}{homeostatis} -- an important factor causing the experience of hunger -- but also on circadian rhythms, food availability, food palatability, \& cost. Abstract motivations are also malleable, as evidenced by such phenomena as \textit{goal contagion}: the adoption of goals, sometimes unconsciously, based on inferences about the goals of others. Vohs \& \href{https://en.wikipedia.org/wiki/Roy_Baumeister}{Baumeister} suggest that contrary to the need-desire-fulfillment cycle of animal instincts, human motivations sometimes obey a ``getting begets wanting'' rule: the more you get a reward such as self-esteem, love, drugs, or money, the more you want it. They suggest that this principle can even apply to food, drink, sex, \& sleep.'' -- \href{https://en.wikipedia.org/wiki/Psychology#Motivation}{Wikipedia\texttt{/}psychology\texttt{/}themes\texttt{/}motivation}

\subsubsection{Development psychology}
\textsf{Fig. Developmental psychologists would engage a child with a book \& then make observations based on how the child interacts with the object.}

``Main article: \href{https://en.wikipedia.org/wiki/Developmental_psychology}{Wikipedia\texttt{/}developmental psychology}. Developmental psychology refers to the scientific study of how \& why the thought processes, emotions, \& behaviors of humans change over the course of their lives. Some credit Charles Darwin with conducting the 1st systematic study within the rubric of developmental psychology, having published in 1877 a short paper detailing the development of innate forms of communication based on his observations of his infant son. The main origins of the discipline, however, are fund in the work of \href{https://en.wikipedia.org/wiki/Jean_Piaget}{Jean Piaget}. Like Piaget, developmental psychologists originally focused primarily on the development of cognition from infancy to adolescence. Later, development psychology extended itself to the study cognition over the life span. In addition to studying cognition, developmental psychologists have also come to focus on affective, behavioral, moral, social, \& neural development.

Developmental psychologists who study children use a number of research methods. E.g., they make observations of children in natural settings such as preschools \& engage them in experimental tasks. Such tasks often resemble specially designed games \& activities that are both enjoyable for the child \& scientifically useful. Developmental researchers have even devised clever methods to study the mental processes of infants. In addition to studying children, development psychologists also study aging \& processes throughout the life span, including old age. These psychologists draw on the full range of psychological theories to inform their research.'' -- \href{https://en.wikipedia.org/wiki/Psychology#Development_psychology}{Wikipedia\texttt{/}psychology\texttt{/}themes\texttt{/}development psychology}

\subsubsection{Genes \& environment}
``Main article: \href{https://en.wikipedia.org/wiki/Behavioral_genetics}{Wikipedia\texttt{/}behavioral genetics}. All researched psychological traits are influenced by both \href{https://en.wikipedia.org/wiki/Genes}{genes} \& \href{https://en.wikipedia.org/wiki/Social_environment}{environment}, to varying degrees. These 2 sources of influence are often confounded in observational research of individuals \& families. An example of this confounding can be shown in the transmission of \href{https://en.wikipedia.org/wiki/Depression_(mood)}{depression} from a depressed mother to her offspring. A theory based on environmental transmission would hold that an offspring, by virtue of his or her having a problematic rearing environment managed by a depressed mother, is at risk for developing depression. On the other hand, a hereditarian theory would hold that depression risk in an offspring is influenced to some extent by genes passed to the child from the mother. Genes \& environment in these simple transmission models are completely confounded. A depressed mother may both carry genes that contribute to depression in her offspring \& also create a rearing environment that increases the risk of depression in her child.

\href{https://en.wikipedia.org/wiki/Behavioral_genetics}{Behavioral genetics} researchers have employed methodologies that help to disentangle this confound \& understand the nature \& origins of individual differences in behavior. Traditionally the research has involved \href{https://en.wikipedia.org/wiki/Twin_studies}{twin studies} \& \href{https://en.wikipedia.org/wiki/Adoption_study}{adoption studies}, 2 designs where genetic \& environmental influences can be partially un-confounded. More recently, gene-focused research has contributed to understanding genetic contributions to the development of psychological traits.

The availability of \href{https://en.wikipedia.org/wiki/Microarray}{microarray} \href{https://en.wikipedia.org/wiki/Molecular_genetics}{molecular genetic} or \href{https://en.wikipedia.org/wiki/Genome_sequencing}{genome sequencing} technologies allows researchers to measure participant DNA variation directly, \& test whether individual genetic variants within genes are associated with psychological traits \& \href{https://en.wikipedia.org/wiki/Psychopathology}{psychopathology} through methods including \href{https://en.wikipedia.org/wiki/Genome-wide_association_studies}{genome-wide association studies}. 1 goal of such research is similar to that in \href{https://en.wikipedia.org/wiki/Positional_cloning}{positional cloning} \& its success in \href{https://en.wikipedia.org/wiki/Huntington%27s}{Huntington's}: once a causal gene is discovered biological research can be conducted to understand how that gene influences the phenotype. 1 major result of genetic association studies is the general finding that psychological traits \& psychopathology, as well as complex medical diseases, are highly \href{https://en.wikipedia.org/wiki/Polygenic}{polygenic}, where a large number (on the order of hundreds to thousands) of genetic variants, each of small effect, contribute to individual differences in the behavioral trait oor propensity to the disorder. Active research continues to work toward understanding the genetic \& environmental bases of behavior \& their interaction.'' -- \href{https://en.wikipedia.org/wiki/Psychology#Genes_and_environment}{Wikipedia\texttt{/}psychology\texttt{/}themes\texttt{/}genes \& environment}

\subsection{Applications}
``Further information: \href{https://en.wikipedia.org/wiki/Outline_of_psychology}{Wikipedia\texttt{/}outline of psychology}, \href{https://en.wikipedia.org/wiki/List_of_psychology_disciplines}{Wikipedia\texttt{/}list of psychology disciplines}, \href{https://en.wikipedia.org/wiki/Applied_psychology}{Wikipedia\texttt{/}applied psychology}, \& \href{https://en.wikipedia.org/wiki/Subfields_of_psychology}{Wikipedia\texttt{/}subfields of psychology}. Psychology encompasses many subfields \& includes different approaches to the study of mental processes \& behavior.'' -- \href{https://en.wikipedia.org/wiki/Psychology#Applications}{Wikipedia\texttt{/}psychology\texttt{/}applications}

\subsubsection{Psychological testing}
``See also: \href{https://en.wikipedia.org/wiki/Psychometrics}{Wikipedia\texttt{/}psychometrics} \& \href{https://en.wikipedia.org/wiki/Social_statistics}{Wikipedia\texttt{/}social statistics}. Psychological testing has ancient origins, dating as far back as 2200 BC, in the \href{https://en.wikipedia.org/wiki/Imperial_examination}{examinations for the Chinese civil service}. Written exams began during the Han dynasty (202 BC -- AD 200). By 1370, the Chinese system required a stratified series of tests, involving essay writing \& knowledge of diverse topics. The system was ended in 1906. In Europe, mental assessment took a different approach, with theories of \href{https://en.wikipedia.org/wiki/Physiognomy}{physiognomy} -- judgment of character based on the face -- described by Aristotle in 4th century BC Greece. Physiognomy remained current through the Enlightenment, \& added the doctrine of phrenology: a study of mind \& intelligence based on simple assessment of neuroanatomy.

When experimental psychology came to Britain, Francis Galton was a leading practitioner. By virtue of his procedures for measuring reaction time \& sensation, he is considered an inventor of modern mental testing (also known as \href{https://en.wikipedia.org/wiki/Psychometrics}{\textit{psychometrics}}). James Mckeen Cattell, a student of Wundt \& Galton, brought the idea of psychological testing to the United States, \& in fact coined the term ``mental test''. In 1901, Cattell's student \href{https://en.wikipedia.org/wiki/Clark_Wissler}{Clark Wissler} published discouraging results, suggesting that mental testing of Columbia \& Barnard students failed to predict academic performance. In response to 1904 orders from the \href{https://en.wikipedia.org/wiki/Ministry_of_National_Education_(France)}{Minister of Public Instruction}, French psychologists \href{https://en.wikipedia.org/wiki/Alfred_Binet}{Alfred Binet} \& \href{https://en.wikipedia.org/wiki/Th%C3%A9odore_Simon}{Th\'eodore Simon} developed \& elaborated a new test of intelligence in 1905--1911. They used a range of questions diverse in their nature \& difficulty. Binet \& Simon introduced the concept of \href{https://en.wikipedia.org/wiki/Mental_age}{mental age} \& referred to the lowest scorers on their test as \href{https://en.wikipedia.org/wiki/Idiot}{\textit{idiots}}. \href{https://en.wikipedia.org/wiki/Henry_H._Goddard}{Henry H. Goddard} put the Binet--Simon scale to work \& introduced classifications of mental level such as \textit{imbecile} \& \textit{feebleminded}. In 1916, (after Binet's death), Stanford professor \href{https://en.wikipedia.org/wiki/Lewis_M._Terman}{Lewis M. Terman} modified the Binet--Simon scale (renamed the \href{https://en.wikipedia.org/wiki/Stanford%E2%80%93Binet_Intelligence_Scales}{Stanford--Binet scale}) \& introduced the \href{https://en.wikipedia.org/wiki/Intelligence_quotient}{intelligence quotient} as a score report. Based on his test findings, \& reflecting the racism common to that era, Terman concluded that intellectual disability ``represents the level of intelligence which is very, very common among Spanish--Indians \& Mexican families of the Southwest \& also among negroes. Their dullness seems to be racial.''

Following the Army Alpha \& Army Beta tests, which was developed by psychologist \href{https://en.wikipedia.org/wiki/Robert_Yerkes}{Robert Yerkes} in 1917 \& then used in World War I by industrial \& organizational psychologists for large-scale employee testing \& selection of military personnel. Mental testing also became popular in the U.S., where it was applied to schoolchildren. The federally created National Intelligence Test was administered to 7 million children in the 1920s. In 1926, the \href{https://en.wikipedia.org/wiki/College_Entrance_Examination_Board}{College Entrance Examination Board} created the \href{https://en.wikipedia.org/wiki/Scholastic_Aptitude_Test}{Scholastic Aptitude Test} to standardize college admissions. The results of intelligence tests were used to argue for segregated schools \& economic functions, including the preferential training of Black Americans for manual labor. These practices were criticized by Black intellectuals such a \href{https://en.wikipedia.org/wiki/Horace_Mann_Bond}{Horace Mann Bond} \& \href{https://en.wikipedia.org/wiki/Allison_Davis_(anthropologist)}{Allison Davis}. Eugenicists used mental testing to justify \& organize compulsory sterilization of individuals classified as mentally retarded (now referred to as \href{https://en.wikipedia.org/wiki/Intellectual_disability}{intellectual disability}). In the United states, tens of thousands of men \& women were sterilized. Setting a precedent that has never been overturned, the U.s. Supreme Court affirmed the constitutionality of this practice in the 1927 case \href{https://en.wikipedia.org/wiki/Buck_v._Bell}{Buck v. Bell}.

Today mental testing is a routine phenomenon for people of all ages in Western societies. Modern testing aspires to criteria including standardization of procedure, \href{https://en.wikipedia.org/wiki/Reliability_(psychometrics)}{consistency of results}, output of an interpretable score, statistical norms describing population outcomes, \&, ideally, \href{https://en.wikipedia.org/wiki/Test_validity}{effective prediction} of behavior \& life outcomes outside of testing situations. Developments in psychometrics include work on test \& scale \href{https://en.wikipedia.org/wiki/Reliability_(statistics)}{reliability} \& \href{https://en.wikipedia.org/wiki/Test_validity}{validity}. Developments in \href{https://en.wikipedia.org/wiki/Item-response_theory}{item-response theory}, \href{https://en.wikipedia.org/wiki/Structural_equation_modeling}{structural equation modeling}, \& bifactor analysis have helped in strengthening test \& scale construction.''

-- \href{https://en.wikipedia.org/wiki/Psychology#Psychological_testing}{Wikipedia\texttt{/}psychology\texttt{/}applications\texttt{/}psychological testing}

\subsubsection{Mental health care}
``See also: \href{https://en.wikipedia.org/wiki/Clinical_psychology}{Wikipedia\texttt{/}clinical psychology}. The provision of psychological health services is generally called \textit{clinical psychology} in the U.S. sometimes, however, members of the school psychology \& counseling psychology professions engage in practices that resemble that of clinical psychologists. Clinical psychologists typically include people who have graduated from doctoral programs in clinical psychology. In Canada, some of the members of the abovementioned groups usually fall within the larger category of \href{https://en.wikipedia.org/wiki/Professional_psychology}{professional psychology}. In Canada \& the U.S., practitioners get bachelor's degrees \& doctorates; doctoral students in clinical psychology usually spend 1 year in a predoctoral internship \& 1 year in postdoctoral internship. In Mexico \& most other Latin American \& European countries, psychologists do not get bachelor's \& doctoral degrees; instead, they take a 3-year professional course following high school. Clinical psychology is at present the largest specialization within psychology. It includes the study \& application of psychology for the purpose of understanding, preventing, \& relieving psychological distress, dysfunction, \&\texttt{/}or \href{https://en.wikipedia.org/wiki/Mental_illness}{mental illness}. Clinical psychologists also try to promote subjective well-being \& personal growth. Central to the practice of clinical psychology are psychological assessment \& psychotherapy although clinical psychologists may also engage in research, teaching, consultation, forensic testimony, \& program development \& administration.

Credit for the 1st psychology clinic in the United States typically goes to \href{https://en.wikipedia.org/wiki/Lightner_Witmer}{Lightner Witmer}, who established his practice in Philadelphia in 1896. Another modern psychotherapist was \href{https://en.wikipedia.org/wiki/Morton_Prince}{Morton Prince}, an early advocate for the establishment of psychology as a clinical \& academic discipline. In the 1st part of the 20th century, most mental health care in the United States was performed by psychiatrists, who are medical doctors. Psychology entered the field with its refinements of mental testing, which promised to improve the diagnosis of mental problems. For their part, some psychiatrists became interested in using \href{https://en.wikipedia.org/wiki/Psychoanalysis}{psychoanalysis} \& other forms of \href{https://en.wikipedia.org/wiki/Psychodynamic_psychotherapy}{psychodynamic psychotherapy} to understand \& treat the mentally ill.

Psychotherapy as conducted by psychiatrists blurred the distinction between psychiatry \& psychology, \& this trend continued with the rise of \href{https://en.wikipedia.org/wiki/Community_mental_health_service}{community mental health facilities}. Some in the clinical psychology community adopted \href{https://en.wikipedia.org/wiki/Behavioral_therapy}{behavioral therapy}, a thoroughly non-psychodynamic model that used behaviorist learning theory to change the actions of patients. A key aspect of behavior therapy is empirical evaluation of the treatment's effectiveness. In the 1970s, \href{https://en.wikipedia.org/wiki/Cognitive-behavior_therapy}{cognitive-behavior therapy} emerged with the work of \href{https://en.wikipedia.org/wiki/Albert_Ellis}{Albert Ellis} \& \href{https://en.wikipedia.org/wiki/Aaron_Beck}{Aaron Beck}. Although there are similarities between behavior therapy \& cognitive-behavior therapy, cognitive-behavior therapy required the application of cognitive constructs. Since the 1970s, the popularity of cognitive-behavior therapy among clinical psychologists increased. A key practice in behavioral \textit{\&} cognitive-behavioral therapy is exposing patients to things they fear, based on the premise that their responses (fear, panic, anxiety) can be deconditioned.

Mental health care today involves psychologists \& \href{https://en.wikipedia.org/wiki/Social_work}{social workers} in increasing numbers. In 1977, National Institute of Mental Health director \href{https://en.wikipedia.org/wiki/Bertram_S._Brown}{Bertram Brown} described this shift as a source of ``intense competition \& role confusion.'' Graduate programs issuing doctorates in clinical psychology emerged in the 1950s \& underwent rapid increase through the 1980s. The PhD degree is intended to train practitioners who could also conduct scientific research. The PsyD degree is more exclusively designed to train practitioners.

Some clinical psychologists focus on the clinical management of patients with brain injury. This subspecialty is known as \href{https://en.wikipedia.org/wiki/Clinical_neuropsychology}{clinical neuropsychology}. In many countries, clinical psychology is a regulated mental health profession. The emerging field of \textit{disaster psychology} (see \href{https://en.wikipedia.org/wiki/Crisis_intervention}{crisis intervention}) involves professionals who respond to large-scale traumatic events.

The work performed by clinical psychologists tends to be influenced by various therapeutic approaches, all of which involve a formal relationship between professional \& client (usually an individual, couple, family, or small group). Typically, these approaches encourage new ways of thinking, feeling, or behaving. 4 major theoretical perspectives are psychodynamic, cognitive behavioral, existential--humanistic, \& systems or family therapy. There has been a growing movement to integrate the various therapeutic approaches, especially with an increased understanding of issues regarding culture, gender, spirituality, \& sexual orientation. With the advent of more robust research findings regarding psychotherapy, there is evidence that most of the major therapies have equal effectiveness, with the key common element being a strong \href{https://en.wikipedia.org/wiki/Therapeutic_relationship}{therapeutic alliance}. Because of this, more training programs \& psychologists are now adopting an \href{https://en.wikipedia.org/wiki/Integrative_Psychotherapy}{eclectic therapeutic orientation}.

Diagnosis in clinical psychology usually follows the \textit{Diagnostic \& Statistical Manual of Mental Disorders} (DSM). The study of mental illnesses is called \href{https://en.wikipedia.org/wiki/Abnormal_psychology}{abnormal psychology}.'' -- \href{https://en.wikipedia.org/wiki/Psychology#Mental_health_care}{Wikipedia\texttt{/}psychology\texttt{/}applications\texttt{/}mental health care}

\subsubsection{Education}
\textsf{Fig. An example of an item from a cognitive abilities test used in educational psychology.}

``Main articles: \href{https://en.wikipedia.org/wiki/Educational_psychology}{Wikipedia\texttt{/}educational psychology} \& \href{https://en.wikipedia.org/wiki/School_psychology}{Wikipedia\texttt{/}school psychology}. \href{https://en.wikipedia.org/wiki/Educational_psychology}{Educational psychology} is the study of how humans learn in educational settings, the effectiveness of educational interventions, the psychology of teaching, \& the social psychology of \href{https://en.wikipedia.org/wiki/School}{schools} as organizations. Educational psychologists can be found in preschools, schools of all levels including post secondary institutions, community organizations \& learning centers, Government or private research films, \& independent or private consultant. The work of developmental psychologists such as Lev Vygotsky, \href{https://en.wikipedia.org/wiki/Jean_Piaget}{Jean Piaget}, \& \href{https://en.wikipedia.org/wiki/Jerome_Bruner}{Jerome Bruner} has been influential in creating teaching methods \& educational practices. Educational psychology is often included in teacher education programs in places such as North America, Australia, \& New Zealand.

School psychology combines principles from educational psychology \& clinical psychology to understand \& treat students with learning disabilities; to foster the intellectual growth of \href{https://en.wikipedia.org/wiki/Intellectual_giftedness}{gifted} students; to facilitate \href{https://en.wikipedia.org/wiki/Prosocial_behavior}{prosocial behaviors} in adolescents; \& otherwise to promote safe, supportive, \& effective learning environments. School psychologists are trained in educational \& behavioral assessment, intervention, prevention, \& consultation, \& many have extensive training in research.'' -- \href{https://en.wikipedia.org/wiki/Psychology#Education}{Wikipedia\texttt{/}psychology\texttt{/}applications\texttt{/}education}

\subsubsection{Work}
``See also: \href{https://en.wikipedia.org/wiki/Industrial_and_organizational_psychology}{Wikipedia\texttt{/}industrial \& organizational psychology} \& \href{https://en.wikipedia.org/wiki/Organizational_behavior}{Wikipedia\texttt{/}organizational behavior}. Industrial \& organizational (I\texttt{/}O) psychology involves research \& practices that apply psychological theories \& principles to organizations \& individuals' work-lives. In the field's beginnings, industrialists brought the nascent field of psychology to bear on the study of \href{https://en.wikipedia.org/wiki/Scientific_management}{scientific management} techniques for improving workplace efficiency. The field was at 1st called \textit{economic psychology} or \textit{business psychology}; later, \textit{industrial psychology, employment psychology}, or \textit{psychotechnology}. An influential early study examined workers at Western Electric's Hawthorne plant in Cicero, Illinois from 1924 to 1932. Western Electric experimented on factory workers to assess their responses to changes in illumination, breaks, food, \& wages. The researchers came to focus on workers' responses to observation itself, \& the term \href{https://en.wikipedia.org/wiki/Hawthorne_effect}{Hawthrone effect} is now used to describe the fact that people work harder when they think they're being watched. Although the Hawthorne research can be found in psychology textbooks, the research \& its findings, however, were weak at best.

The name industrial \& organizational psychology emerged in the 1960s. In 1973, it became enshrined in the name of the \href{https://en.wikipedia.org/wiki/Society_for_Industrial_and_Organizational_Psychology}{Society for Industrial \& Organizational Psychology}, Division 14 of the American Psychological Association. 1 goal of the discipline is to optimize human potential in the workplace. Personnel psychology is a subfield of I\texttt{/}O psychology. Personnel psychologists apply the methods \& principles of psychology in selecting \& evaluating workers. Another subfield, \href{https://en.wikipedia.org/wiki/Organizational_psychology}{organizational psychology}, examines the effects of work environments \& management styles on worker motivation, job satisfaction, \& productivity. Most I\texttt{/}O psychologists work outside of academia, for private \& public organizations \& as consultants. A psychology consultant working in business today might expect to provide executives with information \& ideas about their industry, their target markets, \& the organization of their company.

Organizational behavior (OB) is an allied field involved in the study of human behavior within organizations. 1 way to differentiate I\texttt{/}O psychology from OB is to note that I\texttt{/}O psychologists train in university psychology departments \& OB specialists, in business schools.'' -- \href{https://en.wikipedia.org/wiki/Psychology#Work}{Wikipedia\texttt{/}psychology\texttt{/}applications\texttt{/}work}

\subsubsection{Military \& intelligence}
``1 role for \href{https://en.wikipedia.org/wiki/Military_psychology}{psychologists in the military} has been to evaluate \& counsel soldiers \& other personnel. In the U.S., this function began during World War I, when Robert Yerkes established the School of Military Psychology at \href{https://en.wikipedia.org/wiki/Fort_Oglethorpe,_Georgia}{Fort Oglethorpe} in Georgia. The school provided psychological training for military staff. Today, U.S. Army psychologists perform psychological screening, clinical psychotherapy, \href{https://en.wikipedia.org/wiki/Suicide_prevention}{suicide prevention}, \& treatment for post-traumatic stress, as well as provide prevention-related services, e.g., smoking cessation. The United States Army's Mental Health Advisory Teams implement psychological interventions to help combat troops experiencing mental problems.

Psychologists may also work on a diverse set of campaigns known broadly as psychological warfare. Psychological warfare chiefly involves the use of propaganda to influence enemy soldiers \& civilians. This so-called \textit{black propaganda} is designed to seem as if it originates from a source other than the Army. The \href{https://en.wikipedia.org/wiki/CIA}{CIA}'s \href{https://en.wikipedia.org/wiki/MKULTRA}{MKULTRA} program involved more individualized efforts at \href{https://en.wikipedia.org/wiki/Mind_control}{mind control}, involving techniques such as hypnosis, torture, \& covert involuntary administration of \href{https://en.wikipedia.org/wiki/LSD}{LSD}. The U.S. military used the name \href{https://en.wikipedia.org/wiki/Psychological_Operations_(United_States)}{Psychological Operations} (PSYOP) until 2010, when these activities were reclassified as Military Information Support Operations (MISO), part of \href{https://en.wikipedia.org/wiki/Information_Operations_(United_States)}{Information Operations} (IO). Psychologists have sometimes been involved in assisting the interrogation \& torture of suspects, staining the records of the psychologists involved.'' -- \href{https://en.wikipedia.org/wiki/Psychology#Military_and_intelligence}{Wikipedia\texttt{/}psychology\texttt{/}applications\texttt{/}military \& intelligence}

\subsubsection{Health, well-being, \& social change}
``See also: \href{https://en.wikipedia.org/wiki/Health_psychology}{Wikipedia\texttt{/}health psychology}, \href{https://en.wikipedia.org/wiki/Social_issues}{Wikipedia\texttt{/}social issues}, \& \href{https://en.wikipedia.org/wiki/Occupational_health_psychology}{Wikipedia\texttt{/}occupational health psychology}.

\paragraph{Social change.}
An example of the contribution of psychologists to social change involves the research of \href{https://en.wikipedia.org/wiki/Kenneth_B._Clark}{Kenneth B. Clark} \& \href{https://en.wikipedia.org/wiki/Mamie_Phipps_Clark}{Mamie Phipps Clark}. These 2 African American psychologists studied segregation's adverse psychological impact on Black children. Their research findings played a role in the desegregation case \href{https://en.wikipedia.org/wiki/Brown_v._Board_of_Education}{Brown v. Board of Education} (1954).

The impact of psychology on social change includes the discipline's broad influence on teaching \& learning. Research has shown that compared to the ``whole word'' or ``whole language'' approach, the phonics approach to reading instruction is more efficacious.

\paragraph{Medical applications.} Medical facilities increasingly employ psychologists to perform various roles. 1 aspect of health psychology is the \href{https://en.wikipedia.org/wiki/Psychoeducation}{psychoeducation} of patients: instructing them in how to follow a medical regimen. Health psychologists can also educate doctors \& conduct research on patient compliance. Psychologists in the field of public health use a wide variety of interventions to influence human behavior. These range from public relations campaigns \& outreach to governmental laws \& policies. Psychologists study the composite influence of all these different tools in an effort to influence whole \href{https://en.wikipedia.org/wiki/Population}{populations} of people.

\paragraph{Worker health, safety \& wellbeing.} Psychologists work with organizations to apply findings from psychological research to improve the health \& well-being of employees. Some work as external consultants hired by organizations to solve specific problems, whereas others are full-time employees of the organization. Applications include conducting surveys to identify issues \& designing interventions to make work healthier. Some of the specific health areas include:
\begin{itemize}
	\item Accidents \& injuries: A major contribution is the concept of \href{https://en.wikipedia.org/wiki/Safety_climate}{safety climate}, which is employee shared perceptions of the behaviors that are encouraged (e.g., wearing safety gear) \& discouraged (not following safety rules) at work. Organizations with strong safety climates have fewer \href{https://en.wikipedia.org/wiki/Work_accidents}{work accidents} \& injuries.
	\item \href{v}{Cardiovascular disease}: Cardiovascular disease has been related to lack of \href{https://en.wikipedia.org/wiki/Job_control_(workplace)}{job control}.
	\item Mental health: Exposure to \href{https://en.wikipedia.org/wiki/Occupational_stress}{occupational stress} is associated with mental health disorder.
	\item \href{https://en.wikipedia.org/wiki/Musculoskeletal_disorder}{Musculoskeletal disorder}: These are injuries in bones, nerves \& tendons due to overexertion \& repetitive strain. They have been linked to job satisfaction \& workplace stress.
	\item Physical health symptoms: Occupational stress has been linked to physical symptoms such as digestive distress \& headache.
	\item \href{https://en.wikipedia.org/wiki/Workplace_violence}{Workplace violence}: Violence prevention climate is related to being physically assaulted \& psychologically mistreated at work.
\end{itemize}
Interventions that improve climates are a way to address accidents \& violence. Interventions that reduce stress at work or provide employees with tools to better manage it can help in areas where stress is an important component.

Industrial psychology became interested in worker fatique during World War I, when government ministers in Britain were concerned about the impact of fatique on workers in munitions factories but not other types of factories. In the U.K. some interest in worker \href{https://en.wikipedia.org/wiki/Well-being}{well-being} emerged with the efforts of \href{https://en.wikipedia.org/wiki/Charles_Samuel_Myers}{Charles Samuel Myers} \& his National Institute of Industrial Psychology (NIIP) during the inter-War years. In the U.S. during the mid-20th century industrial psychologist \href{https://en.wikipedia.org/wiki/Arthur_Kornhauser}{Arthur Kornhauser} pioneered the study of occupational mental health, linking industrial working conditions to mental health as well as the spillover of an unsatisfying job into a worker's personal life. Zickar accumulated evidence to show that ``no other industrial psychologist of his era was as devoted to advocating management \& labor practices that would improve the lives of working people.''

\paragraph{Occupational health psychology.}
As interest in the worker health expanded toward the end of the 20th century, the field of \href{https://en.wikipedia.org/wiki/Occupational_health_psychology}{occupational health psychology} (OHP) emerged. OHP is a branch of psychology that is interdisciplinary. OHP is concerned with the health \& safety of workers. OHP addresses topic areas such as the impact of occupational stressors on physical \& mental health, mistreatment of workers (e.g., bullying \& violence), work-family balance, the impact of \href{https://en.wikipedia.org/wiki/Involuntary_unemployment}{involuntary unemployment} on physical \& mental health, the influence of psychosocial factors on safety \& accidents, \& interventions designed to improve\texttt{/}protect worker health. OHP grew out of \href{https://en.wikipedia.org/wiki/Health_psychology}{health psychology}, \href{https://en.wikipedia.org/wiki/Industrial_and_organizational_psychology}{industrial \& organizational psychology}, \& \href{https://en.wikipedia.org/wiki/Occupational_medicine}{occupational medicine}. OHP has also been informed by disciplines outside psychology, including \href{https://en.wikipedia.org/wiki/Industrial_engineering}{industrial engineering}, \href{https://en.wikipedia.org/wiki/Sociology}{sociology}, \& \href{https://en.wikipedia.org/wiki/Economics}{economics}.'' -- \href{https://en.wikipedia.org/wiki/Psychology#Health,_well-being,_and_social_change}{Wikipedia\texttt{/}psychology\texttt{/}applications\texttt{/}health, well-being, \& social change}

\subsection{Research Methods}
``Main articles: \href{https://en.wikipedia.org/wiki/Psychological_research}{Wikipedia\texttt{/}psychological research} \& \href{https://en.wikipedia.org/wiki/List_of_psychological_research_methods}{Wikipedia\texttt{/}list of psychological research methods}. \href{https://en.wikipedia.org/wiki/Quantitative_psychological_research}{Quantitative psychological research} lends itself to the statistical testing of hypotheses. Although the field makes abundant use of randomized \& controlled experiments in laboratory settings, such research can only assess a limited range of short-term phenomena. Some psychologists rely on less rigorously controlled, but more \href{https://en.wikipedia.org/wiki/Ecological_validity}{ecologically valid}, \href{https://en.wikipedia.org/wiki/Field_experiments}{field experiments} as well. Other research psychologists rely on statistical methods to glean knowledge from population data. The statistical methods research psychologists employ include the \href{https://en.wikipedia.org/wiki/Pearson_product%E2%80%93moment_correlation_coefficient}{Pearson product--moment correlation coefficient}, the \href{https://en.wikipedia.org/wiki/Analysis_of_variance}{analysis of variance}, \href{https://en.wikipedia.org/wiki/Multiple_linear_regression}{multiple linear regression}, \href{https://en.wikipedia.org/wiki/Logistic_regression}{logical regression}, \href{https://en.wikipedia.org/wiki/Structural_equation_modeling}{structural equation modeling}, \& \href{https://en.wikipedia.org/wiki/Hierarchical_linear_modeling}{hierarchical linear modeling}. The \href{https://en.wikipedia.org/wiki/Psychometrics}{measurement} \& \href{https://en.wikipedia.org/wiki/Operational_definition}{operationalization} of important \href{https://en.wikipedia.org/wiki/Construct_(psychology)}{constructs} is an essential part of these research designs.

Although this type of psychological research is much less abundant than quantitative research, some psychologists conduct \href{https://en.wikipedia.org/wiki/Qualitative_research}{qualitative research}. This type of research can involve interviews, questionnaires, \& 1st-hand observation. While hypothesis testing is rare, virtually impossible, in qualitative research, qualitative studies can be helpful in theory \& hypothesis generation, interpreting seemingly contradictory quantitative findings, \& understanding why some interventions fail \& others succeed.'' -- \href{https://en.wikipedia.org/wiki/Psychology#Research_methods}{Wikipedia\texttt{/}psychology\texttt{/}research methods}

\subsubsection{Controlled experiments}
\textsf{Fig. Flowchart of 4 phrases (enrollment, intervention allocation, follow-up, \& data analysis) of a parallel randomized trial of 2 groups, modified from the \href{https://en.wikipedia.org/wiki/Consolidated_Standards_of_Reporting_Trials}{CONSORT 2010 Statement}.}

``Main article: \href{https://en.wikipedia.org/wiki/Experiment}{Wikipedia\texttt{/}experiment}. A \href{https://en.wikipedia.org/wiki/True_experiment}{true experiment} with \href{https://en.wikipedia.org/wiki/Randomized_controlled_trial}{random} assignment of research participants (sometimes called \textit{subjects}) to rival conditions allows researchers to make strong inferences about causal relationships. When there are large numbers of research participants, the random assignment (also called \textit{random allocation}) of those participants to rival conditions ensures that the individuals in those conditions will, on average, be similar on most characteristics, including characteristics that went unmeasured. In an experiment, the researcher alters 1 or more variables of influence, called \href{https://en.wikipedia.org/wiki/Independent_variable}{independent variables}, \& measures resulting changes in the factors of interest, called \href{https://en.wikipedia.org/wiki/Dependent_variable}{dependent variables}. Prototypical experimental research is conducted in a laboratory with a carefully controlled environment.

A \href{https://en.wikipedia.org/wiki/Quasi-experimental_design}{quasi-experiment} refers to a situation in which there are rival conditions under study but random assignment to the different conditions is not possible. Investigators must work with preexisting groups of people. Researchers can use common sense to consider how much the nonrandom assignment threatens the study's \href{https://en.wikipedia.org/wiki/Validity_(logic)}{validity}. E.g., in research on the best way to affect reading achievement in the 1st 3 grades of school, school administrators may not permit educational psychologists to randomly assign children to phonics \& whole language classrooms, in which case the psychologists must work with preexisting classroom assignments. Psychologists will compare the achievement of children attending phonics \& whole language classes \&, perhaps, statistically adjust for any initial differences in reading level.

\textsf{Fig. The experimenter (E) orders the teacher (T), the subject of the experiment, to give what the latter believes are painful electric shocks to a learner (L), who is actually an actor \& \href{https://en.wiktionary.org/wiki/confederate}{confederate}. The subject believes that for each wrong answer, the learner was receiving actual electric shocks, though in reality there were no such punishments. Being separated from the subject, the confederate set up a tape recorder integrated with the electro-shock generator, which played pre-recorded sounds for each shock level etc.}

Experimental researchers typically use a \href{https://en.wikipedia.org/wiki/Statistical_hypothesis_testing}{statistical hypothesis testing} model which involves making predictions before conducting the experiment, then assessing how well the data collected are consistent with the predictions. These predictions are likely to originate from 1 or more abstract scientific \href{https://en.wikipedia.org/wiki/Hypotheses}{hypotheses} about how the phenomenon under study actually works.'' -- \href{https://en.wikipedia.org/wiki/Psychology#Controlled_experiments}{Wikipedia\texttt{/}psychology\texttt{/}research methods\texttt{/}controlled experiments}

\subsubsection{Other types of studies}
``\href{https://en.wikipedia.org/wiki/Survey_methodology}{Surveys} are used in psychology for the purpose of measuring \href{https://en.wikipedia.org/wiki/Attitude_(psychology)}{attitudes} \& \href{https://en.wikipedia.org/wiki/Trait_theory}{traits}, monitoring changes in \href{https://en.wikipedia.org/wiki/Mood_(psychology)}{mood}, \& checking the validity of experimental manipulations (checking research participants' perception of the condition they were assigned to). Psychologists have commonly used paper-\&-pencil surveys. However, surveys are also conducted over the phone or through e-mail. Web-based surveys are increasingly used to conveniently reach many subjects.

\href{https://en.wikipedia.org/wiki/Observational_studies}{Observational studies} are commonly conducted in psychology. In \href{https://en.wikipedia.org/wiki/Cross-sectional_studies}{cross-sectional} observational studies, psychologists collect data at a single point in time. The goal of many cross-sectional studies is the assess the extent factors are correlated with each other. By contrast, in \href{https://en.wikipedia.org/wiki/Longitudinal_studies}{longitudinal studies} psychologists collect data on the same sample at 2 or more points in time. Sometimes the purpose of longitudinal research is to study trends across time such as the stability of traits or age-related changes in behavior. Because some studies involve endpoints that psychologists cannot ethically study from an experimental standpoint, such as identifying the causes of depression, they conduct longitudinal studies a large group of depression-free people, periodically assessing what is happening in the individuals' lives. In this way psychologists have an opportunity to test causal hypotheses regarding conditions that commonly arise in people's lives that put them at risk for depression. Problems that affect longitudinal studies include \href{https://en.wikipedia.org/wiki/Selection_bias#Attrition}{slection attrition}, the type of problem in which bias is introduced when a certain type of research participant disproportionately leaves a study.

\href{https://en.wikipedia.org/wiki/Exploratory_data_analysis}{Expoloratory data analysis} refers to a variety of practices that researchers use to reduce a great many variables to a small number overarching factors. In \href{https://en.wikipedia.org/wiki/Charles_Sanders_Peirce#Modes_of_inference}{Peirce's 3 modes of inference}, exploratory data analysis corresponds to \href{https://en.wikipedia.org/wiki/Abduction_(logic)}{abduction}. \href{https://en.wikipedia.org/wiki/Meta-analysis}{Meta-analysis} is the technique research psychologists use to integrate results from many studies of the same variables \& arriving at a grand average of the findings.'' -- \href{https://en.wikipedia.org/wiki/Psychology#Other_types_of_studies}{Wikipedia\texttt{/}psychology\texttt{/}research methods\texttt{/}other types of studies}

\subsubsection{Direct brain observation\texttt{/}manipulation}
\textsf{Fig. An EEG recording setup.}

``A classic \& popular tool used to relate mental \& neural activity is the \href{https://en.wikipedia.org/wiki/Electroencephalogram}{electroencephalogram} (EEG), a technique using amplified electrodes on a person's scalp to measure voltage changes in different parts of the brain. \href{https://en.wikipedia.org/wiki/Hans_Berger}{Hans Berger}, the 1st researcher to use EEG on an unopened skull, quickly found that brains exhibit signature ``brain waves'': electric oscillations which correspond to different states of consciousness. Researchers subsequently refined statistical methods for synthesizing the electrode data, \& identified unique brain wave patterns such as the \href{https://en.wikipedia.org/wiki/Delta_wave}{delta wave} observed during non-REM sleep.

Newer \href{https://en.wikipedia.org/wiki/Functional_neuroimaging}{functional neuroimaging} techniques include \href{https://en.wikipedia.org/wiki/Functional_magnetic_resonance_imaging}{functional magnetic resonance imaging} \& \href{https://en.wikipedia.org/wiki/Positron_emission_tomography}{position emission tomography}, both of which track the flow of blood through the brain. These technologies provide more localized information about activity in the brain \& create representations of the brain with widespread appeal. They also provide insight which avoids the classic problems of subjective self-reporting. It remains challenging to draw hard conclusions about where in the brain specific thoughts originate -- or even how usefully such localization corresponds with reality. However, neuroimaging has delivered unmistakable results showing the existence of correlations between mind \& brain. Some of these draw on a systemic \href{https://en.wikipedia.org/wiki/Neural_network}{neural network} model rather than a localized function model.

Interventions such as \href{https://en.wikipedia.org/wiki/Transcranial_magnetic_stimulation}{transcranial magnetic stimulation} \& drugs also provide information about brain-mind interactions. \href{https://en.wikipedia.org/wiki/Psychopharmacology}{Psychopharmacology} is the study of drug-induced mental effects.'' -- \href{https://en.wikipedia.org/wiki/Psychology#Direct_brain_observation/manipulaiton}{Wikipedia\texttt{/}psychology\texttt{/}research methods\texttt{/}direct brain observation\texttt{/}manipulation}

\subsubsection{Computer simulation}
\textsf{Fig. \href{https://en.wikipedia.org/wiki/Artificial_neural_network}{Artificial neural network} with 2 layers, an interconnected group of nodes, akin to the vast network of neurons in the human brain.}

``See also: \href{https://en.wikipedia.org/wiki/Computational_cognition}{Wikipedia\texttt{/}computational cognition}, \href{https://en.wikipedia.org/wiki/Graph_theory}{Wikipedia\texttt{/}graph theory}, \& \href{https://en.wikipedia.org/wiki/Network_theory}{Wikipedia\texttt{/}network theory}. Computational modeling is a tool used in \href{https://en.wikipedia.org/wiki/Mathematical_psychology}{mathematical psychology} \& cognitive psychology to simulate behavior. This method has several advantages. Since modern computers process information quickly, simulations can be run in a short time, allowing for high statistical power. Modeling also allows psychologists to visualize hypotheses about the functional organization of mental events that couldn't be directly observed in a human. Computational neuroscience uses mathematical models to simulate the brain. Another method is symbolic modeling, which represents many mental objects using variables \& rules. Other types of modeling include \href{https://en.wikipedia.org/wiki/Dynamic_systems}{dynamic systems} \& \href{https://en.wikipedia.org/wiki/Stochastic_process}{stochastic} modeling.'' -- \href{https://en.wikipedia.org/wiki/Psychology#Computer_simulation}{Wikipedia\texttt{/}psychology\texttt{/}research methods\texttt{/}computer simulation}

\subsubsection{Animal studies}
\textsf{Fig. A rat undergoing a \href{https://en.wikipedia.org/wiki/Morris_water_navigation_test}{Morris water navigation test} used in \href{https://en.wikipedia.org/wiki/Behavioral_neuroscience}{behavioral neuroscience} to study the role of the \href{https://en.wikipedia.org/wiki/Hippocampus}{hippocampus} in \href{https://en.wikipedia.org/wiki/Spatial_learning}{spatial learning} \& memory.}

``Animal experiments aid in investigating many aspects of human psychology, including perception, emotion, learning, memory, \& thought, to name a few. In the 1890s, Russian psychologist Ivan Pavlov famously used dogs to demonstrate classical conditioning. Non-human primates, cats, dogs, pigeons, \& rats \& other rodents are often used in psychological experiments. Ideally, controlled experiments introduce only 1 independent variable at a time, in order to ascertain its unique effects upon dependent variables. These conditions are approximated best in laboratory settings. In contrast, human environments \& genetic backgrounds vary so widely, \& depend upon so many factors, that it is difficult to control important \href{https://en.wikipedia.org/wiki/Variable_(research)}{variables} for human subjects. There are pitfalls, however, in generalizing findings from animal studies to humans through animal models.

Comparative psychology refers to the scientific study of the behavior \& mental processes of non-human animals, especially as these relate to the phylogenetic history, adaptive significance, \& development of behavior. Research in this area explores the behavior of many species, from insects to primates. It is closely related to other disciplines that study animal behavior such as \href{https://en.wikipedia.org/wiki/Ethology}{ethology}. Research in comparative psychology sometimes appears to shed light on human behavior, but some attempts to connect the 2 have been quite controversial, e.g. the \href{https://en.wikipedia.org/wiki/Sociobiology}{Sociobiology} of \href{https://en.wikipedia.org/wiki/E.O._Wilson}{E.O. Wilson}. Animal models are often used to study neural processes related to human behavior, e.g. in cognitive neuroscience.'' -- \href{https://en.wikipedia.org/wiki/Psychology#Animal_studies}{Wikipedia\texttt{/}psychology\texttt{/}research methods\texttt{/}animal studies}

\subsubsection{Qualitative research}
``Qualitative research is often designed to answer questions about the thoughts, feelings, \& behaviors of individuals. Qualitative research involving 1st-hand observation can help describe events as they occur, with the goal of capturing the richness of everyday behavior \& with the hope of discovering \& understanding phenomena that might have been missed if only more cursory examinations are made.

\href{https://en.wikipedia.org/wiki/Qualitative_psychological_research}{Qualitative psychological research} methods include interviews, 1st-hand observation, \& participant observation. Creswell (2003) identified 5 main possibilities for qualitative research, including narrative, phenomenology, \href{https://en.wikipedia.org/wiki/Ethnography}{ethnography}, \href{https://en.wikipedia.org/wiki/Case_study}{case study}, \& \href{https://en.wikipedia.org/wiki/Grounded_theory}{grounded theory}. Qualitative researchers sometimes aim to enrich our understanding of symbols, subjective experiences, or social structures. Sometimes \href{https://en.wikipedia.org/wiki/Hermeneutic}{hermeneutic} \& critical aims can give rise to quantitative research, as in \href{https://en.wikipedia.org/wiki/Erich_Fromm}{Erich Fromm}'s application of psychological \& sociological theories, in his book \href{https://en.wikipedia.org/wiki/Escape_from_Freedom}{\textit{Escape from Freedom}}, to understanding why many ordinary Germans supported Hitler.

\textsf{Fig. \href{https://en.wikipedia.org/wiki/Phineas_P._Gage}{Phineas P. Gage} survived an accident in which a large iron rod was driven completely through his head, destroying much of his brain's left frontal lobe, \& is remembered for that injury's reported effects on his personality \& behavior.}

Just as \href{https://en.wikipedia.org/wiki/Jane_Goodall}{Jane Goodall} studied chimpanzee social \& family life by careful observation of chimpanzee behavior in the field, psychologists conduct \href{https://en.wikipedia.org/wiki/Naturalistic_observation}{naturalistic observation} of ongoing human social, professional, \& family life. Sometimes the participants are aware they are being observed, \& other times the participants do not know they are being observed. Strict ethical guidelines must be followed when covert observation is being carried out.'' -- \href{https://en.wikipedia.org/wiki/Psychology#Qualitative_research}{Wikipedia\texttt{/}psychology\texttt{/}research methods\texttt{/}qualitative research}

\subsubsection{Program evaluation}
``\href{https://en.wikipedia.org/wiki/Program_evaluation}{Program evaluation} involves the systematic collection, analysis, \& application of information to answer questions about projects, policies \& programs, particularly about their effectiveness. In both the public \& private sectors, stakeholders often want to know the extent which the programs they are funding, implementing, voting for, receiving, or objecting to are producing the intended effects. While program evaluation 1st focuses on effectiveness, important considerations often include how much the program costs per participant, how the program could be improved, whether the program is worthwhile, whether there are better alternatives, if there are unintended outcomes, \& whether the program goals are appropriate \& useful.'' -- \href{https://en.wikipedia.org/wiki/Psychology#Program_evaluation}{Wikipedia\texttt{/}psychology\texttt{/}research methods\texttt{/}program evaluation}

\subsection{Contemporary issues in methodology \& practice}

\subsubsection{Meta science}
``Metascience involves the application of scientific methodology to study science itself. The field of \href{https://en.wikipedia.org/wiki/Metascience}{metascience} has revealed problems in psychological research. Some psychological research has suffered from \href{https://en.wikipedia.org/wiki/Bias}{bias}, problematic \href{https://en.wikipedia.org/wiki/Reproducibility}{reproducibility}, \& \href{https://en.wikipedia.org/wiki/Misuse_of_statistics}{misuse of statistics}. These findings have led to calls for reform from within \& from outside the scientific community.

\paragraph{Confirmation bias.} In 1959, statistician Theodore Sterling examined the results of psychological studies \& discovered that 97\% of them supported their initial hypotheses, implying possible \href{https://en.wikipedia.org/wiki/Publication_bias}{publication bias}. Similarly, Fanelli (2010) found that 91.5\% of psychiatry\texttt{/}psychology studies confirmed the effects they were looking for, \& concluded that the odds of this happening (a positive result) was around 5 times higher than in fields such as \href{https://en.wikipedia.org/wiki/Space_science}{space science} or \href{https://en.wikipedia.org/wiki/Geoscience}{geosciences}. Fanelli argued that this is because researchers in ``softer'' sciences have fewer constraints to their conscious \& unconscious biases.

\paragraph{Replication.} Further information: \href{https://en.wikipedia.org/wiki/Replication_crisis#In_psychology}{Wikipedia\texttt{/}replication crisis\texttt{/}in psychology}. A \href{https://en.wikipedia.org/wiki/Replication_crisis}{replication crisis} in psychology has emerged. Many notable findings in the field have not been replicated. Some researchers were even accused of publishing fraudulent results. Systematic efforts, including efforts by the \href{https://en.wikipedia.org/wiki/Reproducibility_Project}{Reproducibility Project} of the \href{https://en.wikipedia.org/wiki/Center_for_Open_Science}{Center for Open Science}, to assess the extent of the problem found that as many as $\frac{2}{3}$ of highly publicized findings in psychology failed to be replicated. Reproducibility has generally been stronger in cognitive psychology (in studies \& journals) than social psychology \& subfields of \href{https://en.wikipedia.org/wiki/Differential_psychology}{differential psychology}. Other subfields of psychology have also been implicated in the replication crisis, including clinical psychology, developmental psychology, \& a field closely related to psychology, \href{https://en.wikipedia.org/wiki/Educational_research}{educational research}.

Focus on the replication crisis has led to other renewed efforts in the discipline to re-test important findings. In response to concerns about publication bias \& \href{https://en.wikipedia.org/wiki/Data_dredging}{data dredging} (conducting a large number of statistical tests on a great many variables but restricting reporting to the results that were statistically significant), 295 psychology \& medical journals have adopted \href{https://en.wikipedia.org/wiki/Scholarly_peer_review#Result-blind_peer_review}{result-blind peer review} where studies are accepted not on the basis of their findings \& after the studies are completed, but before the studies are conducted \& upon the basis of the methodological rigor of their experimental designs \& the theoretical justifications for their proposed statistical analysis before data collection or analysis is conducted. In addition, large-scale collaborations among researchers working in multiple labs in different countries have taken place. The collaborators regularly make their data openly available for different researchers to assess. Allen \& Mehler estimated that 61\% of result-blind studies have yielded \href{https://en.wikipedia.org/wiki/Null_result}{null results}, in contrast to an estimated 5--20\% in traditional research.

\paragraph{Misuse of statistics.} Further information: \href{https://en.wikipedia.org/wiki/Misuse_of_statistics}{Wikipedia\texttt{/}misuse of statistics} \& \href{https://en.wikipedia.org/wiki/Misuse_of_p-values}{Wikipedia\texttt{/}misuse of $p$-values}. Some critics view \href{https://en.wikipedia.org/wiki/Statistical_hypothesis_testing#Criticism}{statistical hypothesis testing} as misplaced. Psychologist \& statistician \href{https://en.wikipedia.org/wiki/Jacob_Cohen_(statistician)}{Jacob Cohen} wrote in 1994 that psychologists routinely confuse statistical significance with practical importance, enthusiastically reporting great certainty in unimportant facts. Some psychologists have responded with an increased use of \href{https://en.wikipedia.org/wiki/Effect_size}{effect size} statistics, rather than sole reliance on $p$-values.'' -- \href{https://en.wikipedia.org/wiki/Psychology#Metascience}{Wikipedia\texttt{/}psychology\texttt{/}contemporary issues in methodology \& practice\texttt{/}metascience}

\subsubsection{WEIRD bias}
``See also: \href{https://en.wikipedia.org/wiki/Cultural_psychology}{Wikipedia\texttt{/}cultural psychology}, \href{https://en.wikipedia.org/wiki/Indigenous_psychology}{Wikipedia\texttt{/}indigenous psychology}, \href{https://en.wikipedia.org/wiki/Transnational_psychology}{Wikipedia\texttt{/}transnational psychology},\\\& \href{https://en.wikipedia.org/wiki/Cross-cultural_psychology}{Wikipedia\texttt{/}cross-cultural psychology}. In 2008, Arnett pointed out that most articles in American Psychological Association journals were about U.S. populations when U.S. citizens are only 5\% of the world's population. He complained that psychologists had no basis for assuming psychological processes to be universal \& generalizing research findings to the rest of the global population. In 2010, Henrich, Heine, \& Norenzayan reported a bias in conducting psychology studies with participants from ``\textit{WEIRD}'' (```Western, Educated, Industrialized, Rich, \& Democratic'') societies. Henrich et al. found that ``96\% of psychological samples come from countries with only 12\% of the world's populations'' (p. 63). The article gave examples of results that differ significantly between people from WEIRD \& tribal cultures, including the \href{https://en.wikipedia.org/wiki/M%C3%BCller-Lyer_illusion}{M\"uller--Lyer illusion}. Arnett (2008), \href{https://en.wikipedia.org/wiki/Elizabeth_Altmaier}{Altmaier} \& hall (2008) \& Morgan-Consoli et al. (2018) view the Western bias in research \& theory as a serious problem considering psychologists are increasingly applying psychological principles developed in WEIRD regions in their research, clinical work, \& consultation with populations around the world. In 2018, Rad, Martingano, \& Ginges showed that nearly a decade after Henrich et al.'s paper, over 80\% of the samples used in studies published in the journal \href{https://en.wikipedia.org/wiki/Psychological_Science}{Psychological Science} employed WEIRD samples. Moreover, their analysis showed that several studies did not fully disclose the origin of their samples; the authors offered a set of recommendations to editors \& reviewers to reduce WEIRD bias.'' -- \href{https://en.wikipedia.org/wiki/Psychology#WEIRD_bias}{Wikipedia\texttt{/}psychology\texttt{/}contemporary issues in methodology \& practice\texttt{/}WEIRD bias}

\subsubsection{Unscientific mental health training}
``Some observers perceive a gap between scientific theory \& its application -- in particular, the application of unsupported or unsound clinical practices. Critics say there has been an increase in the number of mental health training programs that do not instill scientific competence. Practices such as ``\href{https://en.wikipedia.org/wiki/Facilitated_communication}{facilitated communication} for infantile autism''; memory-recovery techniques including \href{https://en.wikipedia.org/wiki/Bodywork_(alternative_medicine)}{body work}; \& other therapies, such as \href{https://en.wikipedia.org/wiki/Rebirthing_(breathwork)}{rebirthing} \& \href{https://en.wikipedia.org/wiki/Reparenting}{reparenting}, may be dubious or even dangerous, despite their popularity. These practices, however, are outside the mainstream practices taught in clinical psychology doctoral programs.'' -- \href{https://en.wikipedia.org/wiki/Psychology#Unscientific_mental_health_training}{Wikipedia\texttt{/}psychology\texttt{/}contemporary issues in methodology \& practice\texttt{/}unscientific mental health training}

\subsection{Ethics}
``Ethical standards in the discipline have changed over time. Some famous past studies are today considered unethical \& in violation of \href{https://en.wikipedia.org/wiki/Guidelines_for_human_subject_research#APA_Ethics_Code}{established codes} (the Canadian Code of Conduct for Research Involving Humans, \& the \href{https://en.wikipedia.org/wiki/Belmont_Report}{Belmont Report}). The American Psychological Association has advanced a set of ethical principles \& a code of conduct for the profession.

The most important contemporary standards include informed \& voluntary consent. After World War II, the \href{https://en.wikipedia.org/wiki/Nuremberg_Code}{Nuremberg Code} was established because of Nazi abuses of experimental subjects. Later, most countries (\& scientific journals) adopted the \href{https://en.wikipedia.org/wiki/Declaration_of_Helsinki}{Declaration of Helsinki}. In the U.S., the \href{https://en.wikipedia.org/wiki/National_Institutes_of_Health}{National Institutes of Health} established the \href{https://en.wikipedia.org/wiki/Institutional_Review_Board}{Institutional Review Board} in 1966, \& in 1974 adopted the \href{https://en.wikipedia.org/wiki/National_Research_Act}{National Research Act} (HR 7724). All of these measures encouraged researchers to obtain informed consent from human participants in experimental studies. A number of influential but ethically dubious studies led to the establishment of this rule; such studies included the \href{https://en.wikipedia.org/wiki/Walter_E._Fernald_Developmental_Center#Nuclear_medicine_research_in_children}{MIT--Harvad Fernald School radioisotope studies}, the \href{https://en.wikipedia.org/wiki/Thalidomide_scandal}{Thalidomide tragedy}, the \href{https://en.wikipedia.org/wiki/Hepatitis#Willowbrook_State_School_experiments}{Willowbrook hepatitis study}, \& \href{https://en.wikipedia.org/wiki/Milgram_experiment}{Stanley Milgram' studies of obedience to authority}.'' -- \href{https://en.wikipedia.org/wiki/Psychology#Ethics}{Wikipedia\texttt{/}psychology\texttt{/}ethics}

\subsubsection{Humans}
``Universities have ethics committees dedicated to protecting the rights (e.g., voluntary nature of participation in the research, privacy) \& well-being (e.g., minimizing distress) of research participants. University ethics committees evaluate proposed research to ensure that researchers protect the rights \& well-being of participants; an investigator's research project cannot be conducted unless approved by such an ethics committee.

The ethics code of the American Psychological Association originated in 1951 as ``Ethical Standards of Psychologists''. This code has guided the formation of licensing laws in most American states. It has changed multiple times over the decades since its adoption. In 1989, the APA revised its policies on advertising \& referral fees to negotiate the end of an investigation by the Federal Trade Commission. The 1992 incarnation was the 1st to distinguish between ``aspirational'' ethical standards \& ``enforceable'' ones. Members of the public have a 5-year window to file ethics complaints about APA members with the APA ethics committee; members of the APA have a 3-year window.

Some of the ethical issues considered most important are the requirement to practice only within the area of competence, to maintain confidentiality with the patients, \& to avoid sexual relations with them. Another important principle is \href{https://en.wikipedia.org/wiki/Informed_consent}{informed consent}, the idea that a patient or research subject must understand \& freely choose a procedure they are undergoing. Some of the most common complaints against clinical psychologists include sexual misconduct.'' -- \href{https://en.wikipedia.org/wiki/Psychology#Humans}{Wikipedia\texttt{/}psychology\texttt{/}ethics\texttt{/}humans}

\subsubsection{Other animals}
``Research on other animals is also governed by university ethics committees. Research on nonhuman animals cannot proceed without permission of the ethics committee of the researcher's home institution. Current ethical guidelines state that using non-human animals for scientific purposes is only acceptable when the harm (physical or psychological) done to animals is outweighed by the benefits of the research. Keeping this in mind, psychologists can use certain research techniques on animals that could not be used on humans.
\begin{itemize}
	\item Comparative psychologist \href{https://en.wikipedia.org/wiki/Harry_Harlow}{Harry Harlow} drew moral condemnation for \href{https://en.wikipedia.org/wiki/Pit_of_despair}{isolation experiments} on rhesus macaque monkeys at the \href{https://en.wikipedia.org/wiki/University_of_Wisconsin--Madison}{University of Wisconsin--Madison} in the 1970s. The aim of the research was to produce an animal model of clinical depression. Harlow also devised what he called a ``rape rack'', to which the female isolates were tied in normal monkey mating posture. In 1974, American literary critic \href{https://en.wikipedia.org/wiki/Wayne_C._Booth}{Wayne c. Booth} wrote that, ``Harry Harlow \& his colleagues go on torturing their nonhuman primates decade after decade, invariably proving what we all knew in advance -- that social creatures can be destroyed by destroying their social ties.'' He writes that Harlow made no mention of the criticism of the morality of his work.'' -- \href{https://en.wikipedia.org/wiki/Psychology#Other_animals}{Wikipedia\texttt{/}psychology\texttt{/}ethics\texttt{/}other animals}
\end{itemize}
\selectlanguage{english}

%------------------------------------------------------------------------------%

\chapter{\cite{Foer2012}. Joshua Foer. Moonwalking with Einstein: The Art \& Science of Remembering Everything}

\section{The Smartest Man Is Hard to Find}

%------------------------------------------------------------------------------%

\section{The Man Who Remembered Too Much}

%------------------------------------------------------------------------------%

\section{The Expert Expert}

%------------------------------------------------------------------------------%

\section{The Most Forgetful Man in The World}

%------------------------------------------------------------------------------%

\section{The Memory Palace}

%------------------------------------------------------------------------------%

\section{How to Memorize a Poem}

%------------------------------------------------------------------------------%

\section{The End of Remembering}

%------------------------------------------------------------------------------%

\section{The OK Plateau}

%------------------------------------------------------------------------------%

\section{The Talented 10th}

%------------------------------------------------------------------------------%

\section{The Little Rain Man in All of Us}

%------------------------------------------------------------------------------%

\section{The U.S. Memory Championship}

%------------------------------------------------------------------------------%

\chapter{\cite{Grant2013, Grant2022}. Adam Grant. Give \& Take: A Revolutionary Approach to Success}

\section*{Praise for Adam Grant's Give \& Take}
\begin{quotation}
	``\textit{Give \& Take} just might be the most important book of this young century. As insightful \& entertaining as Malcom Gladwell at his best, this book has profound implications for how we manage our careers, deal with our friends \& relatives, raise our children, \& design our institutions. This gem is a joy to read, \& it shatters the myth that greed is the path to success.'' -- Robert Sutton, author of \textit{The No *sshole Rule} \& \textit{Good Boss, Bad Boss}\\
	
	``\textit{Give \& Take} is a truly exhilarating\footnote{\textbf{exhilarating} [a] very exciting \& great fun.} book---the rare work that will shatter your assumptions about how the world works \& keep your brain firing for weeks after you've turned the last page.'' -- Daniel H. Pink, author of \textit{Drive} \& \textit{A Whole New Mind}\\
	
	``\textit{Give \& Take} is brimming\footnote{\textbf{brim} [v] [intransitive] to be full of something; to fill something.} with life-changing insights. As brilliant as it is wise, this is not just a book---it's a new \& shining worldview. Adam Grant is 1 of the great social scientists of our time, \& his extraordinary new book is sure to be a bestseller.'' -- Susan Cain, author of \textit{Quiet}\\
	
	``\textit{Give \& Take} cuts through the clutter of clich\'es in the marketplace \& provides a refreshing new perspective on the art \& science of success. Adam Grant has crafted a unique, must-have toolkit for accomplishing goals through collaboration \& reciprocity.'' -- William P. Launder, executive chairman, The Est\'ee Lauder Companies Inc.\\
	
	``\textit{Give \& Take} is a pleasure to read, extraordinarily informative, \& will likely become 1 of the classic books on workplace leadership \& management. It has changed the way I see my personal \& professional relationships, \& has encouraged me to be a more thoughtful friend \& colleague.'' -- Jeff Ashby, NASA space shuttle commander\\
	
	``With \textit{Give \& Take}, Adam Grant has marshaled compelling evidence for a revolutionary way of thinking about personal success in business \& in life. Besides the fundamentally uplifting character of the case he makes, readers will be delighted by the truly engaging way he makes it. This is a must read.'' -- Robert Cialdini, author of \textit{Influence}\\
	
	``\textit{Give \& Take} is a brilliant, well-documented, \& motivating debunking of `good guys finish last'! I've noticed for years that generosity generates its own kind of equity, \& Grant's fascinating research \& engaging style have created not only a solid validation of that principle but also practical wisdom \& techniques for utilizing it more effectively. This is a super manifesto for getting meaningful things done, sustainably.'' -- David Allen, author of \textit{Getting Things Done}\\
	
	``Packed with cutting-edge research, concrete examples, \& deep insight, \textit{Give \& Take} offers extraordinarily thought-provoking---\& often surprising---conclusions about how our interactions with others drive our success \& happiness. This important \& compulsively readable book deserves to be a huge success.'' -- Gretchen Rubin, author of \textit{The Happiness Project} \& \textit{Happier at Home}\\
	
	``1 of the great secrets of life is that those who win most are often those who give most. In this elegant \& lucid book, filled with compelling evidence \& evocative examples, Adam Grant shows us why \& how this is so. Highly recommended!'' -- William Ury, coauthor of \textit{Getting to Yes} \& author of \textit{The Power of a Positive No}\\
	
	``\fbox{Good guys finish 1st}---\& Adam Grant knows why. \textit{Give \& Take} is the smart surprise you can't afford to miss.'' -- Daniel Gilbert, author of \textit{Stumbling on Happiness}\\
	
	``\textit{Give \& Take} is an enlightening read for leaders who aspire to create meaningful \& sustainable\footnote{\textbf{sustainable} [a] \textbf{1.} involving the use of natural products \& energy in a way that does not harm the environment, \textsc{opposite}: \textbf{unsustainable}; \textbf{2.} that can continue or be continued for a long time, \textsc{opposite}: \textbf{unsustainable}.} changes to their environments. Grant demonstrates how a generous orientation toward others can serve as a formula for producing successful leaders \& organizational performance. His writing is as engaging \& enjoyable as his style in the classroom.'' -- Kenneth Frazier, chairman, president, \& CEO, Merck \& Co., Inc.\\
	
	``In this riveting \& sparkling book, Adam Grant turns the conventional wisdom upside down about what it takes to win \& get ahead. With page-turning stories \& compelling studies, \textit{Give \& Take} reveals the surprising forces behind success \& the steps we can take to enhance our own.'' -- Laszlo Bock, senior vice president of people operations, Google\\
	
	``\textit{Give \& Take} dispels commonly held beliefs that equate givers with weakness \& takers with strength. Grant shows us the importance of nurturing \& encouraging prosocial behaviors.'' -- Dan Ariely, author of \textit{Predictably Irrational}\\
	
	``\textit{Give \& Take} defines a road to success marked by new ways of relating to colleagues \& customers as well as new ways of growing a business.'' -- Tony Hsieh, CEO, \url{Zappos.com} \& author of \textit{Delivering Happiness}\\
	
	``\textit{Give \& Take} will fundamentally change the way you think about success. Unfortunately in America, we have too often succumbed\footnote{\textbf{succumb} [v] (\textit{formal}) \textbf{1.} [intransitive] to not be able to fight an attack, a temptation, etc.; \textbf{2.} [intransitive] \textbf{succumb (to something)} to die from the effect of a disease or an injury.} to the worldview that if everyone behaved in their own narrow self-interest, \& would be fine. Adam Grant shows us with compelling research \& fascinating stories there is a better way.'' -- Lenny Mendonca, director, McKinsey \& Co.\\
	
	``Adam Grant, a rising star of positive psychology, seamlessly weaves together science \& stories of business success \& failure, convincing us that giving is, in the long run, the recipe for success in the corporate world. En route you will find yourself reexamining your own life. Read it yourself, then give copies to the people you care most about in this world.'' -- Martin Seligman, author of \textit{Learned Optimism} \& \textit{Flourish}\\
	
	``\textit{Give \& Take} presents a groundbreaking new perspective on success. Adam Grant offers a captivating window into innovative principles that drive effectiveness at every level of an organization \& can immediately to put into action. Along with being a fascinating read, this book holds the key to a more satisfied \& productive workplace, better customer relationships, \& higher profits.'' -- Chip Conley, founder, Joie de Vivre Hotels \& author of \textit{Peak} \& \textit{Emotional Equations}\\
	
	``\textit{Give \& Take} is a game changer. Reading Adam Grant's compelling book will change the way doctors doctor, managers manage, teachers teach, \& bosses boss. It will create a society in which people do better by being better. Read the book \& change the way you live \& work.'' -- Barry Schwartz, author of \textit{The Paradox of Choice} \& \textit{Practical Wisdom}\\
	
	``\textit{Give \& Take} is a new behavioral benchmark for doing business for better, providing an inspiring new perspective on how to succeed to the benefit of all. Adam Grant provides great support for the new paradigm of creating a win win for people, planet, \& profit with many fabulous insights \& wonderful stories to get you fully hooked \& infected with wanting to give more \& take less.'' -- Jochen Zeitz, former CEO \& chairman, PUMA\\
	
	``\textit{Give \& Take} is a real gift. Adam Grant delivers a triple treat: stories as good as a well-written novel, surprising insights drawn from rigorous science, \& advice on using those insights to catapult ourselves \& our organizations to success. I can't think of another book with more powerful implications for both business \textit{\&} life.'' -- Teresa Amabile, author of \textit{The Progress Principle}\\
	
	``Adam Grant has written a landmark book that examines what makes some extraordinarily successful people so great. By introducing us to highly impressive individuals, he proves that, contrary to popular belief, the best way to climb to the top of the ladder is to take others up there with you. \textit{Give \& Take} presents the road to success for the 21st century.'' -- Maria Eitel, founding CEO \& president, the Nike Foundation\\
	
	``In an era of business literature that drones on with the same-old, over-used platitudes\footnote{\textbf{platitude} [n] \textit{disapproving} a comment or statement that has been made very often before \& is therefore not interesting.}, Adam Grant forges into brilliant new territory. \textit{Give \& Take} helps readers understand how to maximize their effectiveness \textbf{\textit{\&}} help others simultaneously. It will serve as a new framework for both insight \& achievement. A must read!'' -- Josh Linkner, founder, ePrize, CEO, Detroit Venture Partners, \& author of \textit{Disciplined Dreaming}\\
	
	``What \textit{The No *sshole Rule} did for corporate culture, \textit{Give \& Take} does for each of us as individuals. Grant presents an evidence-based case for the counterintuitive link between generosity \& finishing 1st.'' -- Douglas Stone \& Sheila Heen, coauthors of \textit{Difficult Conversations}\\
	
	``Adam Grant is a wunderkind\footnote{\textbf{wunderkind} [n] (plural \textbf{wunderkinds} or \textbf{wunderkinder}) (\textit{from German, sometimes disapproving}) a person who is very successful at a young age.}. He has won every distinguished research award \& teaching award in his field, \& his work has changed the way that people see the world. If you want to be surprised---very pleasantly surprised---by what really drives success, then \textit{Give \& Take} is for you. If you want to make the world a better place, read this book. If you want to make you life better, read this book.'' -- Tai Ben-Shahar, author of \textit{Happier}\\
	
	``In 1 of the most engaging \& insightful books I've read in years, Adam Grant makes a persuasive argument for a counterintuitive approach to success. \textit{Give \& Take} is an instant classic that should be read by anyone who wants to be more productive---\& happier---in the office or at home.'' -- Noah Goldstein, author of \textit{Yes!}\\
	
	``\textit{Give \& Take} is sensational\footnote{\textbf{sensational} [a] \textbf{1.} causing great surprise, excitement or interest, \textbf{2.} (\textit{disapproving}) (of a newspaper, etc.) trying to get your interest by presenting facts or events as worse or more shocking than they really are.}, with fascinating insights on page after page. I learned much that I intend to incorporate into my lief immediately. The lessons will not only make you a better person, they will make you more capable of doing good for many people, including yourself.'' -- Rabbi Joseph Telushkin, author of \textit{Jewish Literacy} \& \textit{A Code of Jewish Ethics}\\
	
	``Adam grant is the 1st to define what has changed about relationships in a digital age---\& he backs it up with empirical evidence. In \textit{Give \& Take}, he brilliantly demonstrates that in our deeply interconnected world, the roots of sustainable success lie in creating success for those around you. It's 1 of those rare books that is both enlightening immensely practical. You'll want to read \& revisit it every year.'' -- Paul Saffo, managing director, Foresight \& member, World Economic Forum Council on Strategic Foresight
\end{quotation}
Grant, Adam M. \textit{Give \& take: the hidden social dynamics of success}.

\section{Good Returns: \textit{The Dangers \& Rewards of Giving More Than You Get}}
\begin{quotation}
	``The principle of give \& take; that is diplomacy\footnote{\textbf{diplomacy} [n] [uncountable] the activity of managing relations between different countries; the skill in doing this.}---give 1 \& take 10.'' -- Mark Twain, author \& humorist\footnote{Opening quote: Samuel L. Clemens (aka Mark Twain), ``At the Dinner to Joseph H. Choate, Nov 16, 1901,'' in \textit{Speechless at the Lotos Club}, ed. J. Elderkin, C. S. Lord, \& H. N. Fraser (New York: Lotus Club, 1911), 38.}
\end{quotation}
``\underline{On a sunny Saturday afternoon in Silicon Valley}\footnote{Story of David Hornik \& Danny Shader: Personal interviews with David Hornik (Jan 30 \& Mar 12, 2012) \& Danny Shader (Feb 13, 2012).}, 2 proud fathers stood on the sidelines of a soccer field. They were watching their young daughters play together, \& it was only a matter of time before they struck up a conversation about work. The taller of the 2 men was Danny Shader, a serial entrepreneur\footnote{\textbf{entrepreneur} [n] a person who makes money by starting or running businesses, especially when this involves taking financial risks.} who had spent time at Netscape, Motorola, \& Amazon. Intense\footnote{\textbf{intense} [a] \textbf{1.} very great; very strong, \textsc{synonym}: \textbf{extreme}; \textbf{2.} involving a lot of activity in a short period of time.}, dark-haired, \& capable of talking about business forever, Shader was in his late 30s by the time he launched his 1st company, \& he liked to call himself the ``old man of the Internet.'' He loved building companies, \& he was just getting his 4th start-up off the ground.

Shader had instantly\footnote{\textbf{instant} [a] [usually before noun] happening immediately, \textsc{synonym}: \textbf{immediate}; [n] [usually singular] \textbf{1.} a particular point in time, \textsc{synonym}: \textbf{moment}; \textbf{2.} a very short period of time, \textsc{synonym}: \textbf{moment}.} taken a liking to the other father, a man named David Hornik who invests in companies for a living. At 5'4', with dark hair, glasses, \& a goatee\footnote{\textbf{goatee} [n] a small pointed beard ($=$ hair growing on a man's face) that is grown only on the chin.}, Hornik is a man of eclectic\footnote{\textbf{eclectic} [a] (\textit{formal}) not following 1 style or set of ideas but choosing from or using a wide variety.} interests: he collects \textit{Alice in Wonderland} books, \& in college he created his own major in computer music. He went on to earn a master's in criminology\footnote{\textbf{criminology} [n] [uncountable] the scientific study of crime \& criminals.} \& a law degree, \& after burning the midnight oil\footnote{\textbf{burn the midnight oil} [idiom] to study or work until late at night.} at a law firm, he accepted a job offer to join a venture\footnote{\textbf{venture} [n] a business project or activity, especially one that involves taking risks, \textsc{synonym}: \textbf{undertaking}; [v] \textbf{1.} [intransitive] \textbf{$+$ adv.\texttt{/}prep.} to go somewhere or do something even thought it involves risks; \textbf{2.} [transitive, intransitive] (\textit{formal}) to say or do something in a careful way, especially because it might upset or offend somebody.} capital\footnote{\textbf{venture capital} [n] [uncountable] (\textit{business}) money that is invested in a new company to help it develop, which may involve a lot of risk.} firm\footnote{\textbf{firm} [n] a business or company, especially one involving a partnership of 2 or more people; [a] (\textbf{firmer, firmest}) \textbf{1.} fairly hard; not easy to press into a different shape, \textsc{opposite}: \textbf{soft}; \textbf{2.} [usually before noun] not likely to change; that you can rely on; \textbf{3.} strongly fixed in place, \textsc{opposite}: \textbf{unstable}; \textbf{4.} showing that you are strong \& in control of a situation; \textbf{5.} (of somebody's voice or hand movements) strong \& steady; \textbf{stand fast\texttt{/}firm} [idiom] \textbf{1.} to refuse to change your opinions; \textbf{2.} to refuse to move back.}, where he spent the next decade listening to pitches from entrepreneurs \& deciding whether or not to fund them.

During a break between soccer games, Shader turned to Hornik \& said, ``I'm working on something---do you want to see a pitch\footnote{\textbf{pitch} [n] \textsf{for sport} \textbf{1.} (\textit{British English}) (also \textbf{field} \textit{North American English, British English}); \textsf{of sound} \textbf{2.} [singular, uncountable] how high or low a sound is, especially a musical note; \textsf{degree\texttt{/}strength} \textbf{3.} [singular, uncountable] the degree or strength of a feeling or activity; the highest point of something; \textsf{to sell something} \textbf{4.} [countable, usually singular] talk or arguments used by a person trying to sell something or persuade people to do something; \textsf{in baseball} \textbf{5.} [countable] an act of throwing the ball; the way in which it is thrown; \textsf{black substance} \textbf{6.} [uncountable] a black sticky substance made from oil or coal, used on roofs or the wooden boards of a ship to stop water from coming through; \textsf{in street\texttt{/}market} \textbf{7.} [countable] (\textit{British English}) a place in a street or market where somebody sells things, or where somebody performs in order to entertain people outdoors; \textsf{camping} \textbf{8.} (\textit{British English}) (\textit{North American English} \textbf{campsite}) a place in a campsite where you can put up 1 tent or park 1 caravan, etc.; \textsf{of ship\texttt{/}aircraft} \textbf{9.} [uncountable] (\textit{specialist}) the movement of a ship up \& down in the water or of an aircraft in the air; \textsf{of roof} \textbf{10.} [singular, uncountable] (\textit{specialist}) the degree to which a roof slopes.}?'' Hornik specialized\footnote{\textbf{specialized} [a] (\textit{British English also} \textbf{specialised}) \textbf{1.} connected with a particular area of work or study; \textbf{2.} requiring or involving detailed \& particular knowledge or training; \textbf{3.} designed or developed for a particular purpose or area of knowledge.} in Internet companies, so he seemed like an ideal\footnote{\textbf{ideal} [a] \textbf{1.} perfect; most suitable; \textbf{2.} [only before noun] the best that can be imagined, but not likely to become real; \textbf{in an ideal\texttt{/}perfect world} [idiom] used to say that something is what you would like to happen or what should happen, but you know it cannot; [n] \textbf{ideal (of somebody\texttt{/}something)} an idea or a standard that seems perfect \& worth trying to achieve; \textbf{2.} [usually singular] \textbf{ideal (of something)} a person or thing considered as perfect.} investor\footnote{\textbf{investor} [n] a person or an organization that invests money in something.} to Shader. The interest was mutual\footnote{\textbf{mutual} [a] \textbf{1.} used to describe feelings that 2 or more people have for each other equally, or actions that affect 2 or more people or things equally; \textbf{2.} [only before noun] shared by 2 or more people.}. Most people who pitch ideas are 1st-time entrepreneurs, with no track record of success. In contrast, Shader was a blue-chip\footnote{\textbf{blue-chip} [a] [only before noun] (\textit{finance}) a \textbf{blue-chip} investment is thought to be safe \& likely to make a profit.} entrepreneur who had hit the jackpot not once, but twice. In 1999, his 1st start-up, \url{Accept.com}, was acquired by Amazon for \$175 million. In 2007, his next company, Good Technology, was acquired by Motorola for \$500 million. Given Shader's history, Hornik was eager\footnote{\textbf{eager} [a] very interested \& excited by something that is going to happen or about something that you want to do, \textsc{synonym}: \textbf{keen}.} to hear what he was up to next.

A few days after the soccer game, Shader drove to Hornik's office \& pitched his newest idea. Nearly a quarter of Americans have trouble making online purchases because they don't have a bank account or credit card, \& Shader was proposing an innovative solution to this problem. Hornik was 1 of the 1st venture capitalists\footnote{\textbf{capitalist} [a] (also \textit{less frequent} \textbf{capitalistic} BrE) based on the principles of capitalism; [n] \textbf{1.} a person who supports capitalism; \textbf{2.} a person who owns or controls a lot of wealth \& uses it to produce more wealth, e.g. by investing in trade \& industry.} to hear the pitch, \& right off the bat\footnote{\textbf{right off the bat} [idiom] (\textit{especially North American English, informal}) immediately; without delay.}, he loved it. Within a week, he put Shader in front of his partners \& offered him a term sheet: he wanted to fund Shader's company.

Although Hornik had moved fast, Shader was in a strong position. Given Shader's reputation\footnote{\textbf{reputation} [n] the opinion that people have about what somebody\texttt{/}something is like, based on what has happened in the past.}, \& the quality of his idea, Hornik knew plenty of investors\footnote{\textbf{investor} [n] a person or organization that invests money in something.} would be clamoring\footnote{\textbf{clamour} [v] (\textit{US English} \textbf{clamor})  \textbf{1.} [intransitive, transitive] (\textit{formal}) to demand something loudly; \textbf{2.} [intransitive] (of many people) to shout loudly, especially in a confused way; [n] (also \textbf{clamor}) (\textit{formal}) \textbf{1.} [singular] a loud noise, especially one that is made by a lot of people or animals; \textbf{2.} [uncountable, countable] \textbf{clamour (for something)} a demand for something made by a lot of people.} to work with Shader. ``You're rarely the only investor giving an entrepreneur a term sheet,'' Hornik explains. ``You're competing with the best venture capital firms in the country, \& trying to convince the entrepreneur to take your money instead of theirs.''

The best way for Hornik to land the investment\footnote{\textbf{investment} [n] \textbf{1.} [uncountable, countable] the action or process of investing money for profit; \textbf{2.} [countable] an amount of money that you invest; \textbf{3.} [countable] something that you invest money in; \textbf{4.} [uncountable, countable] \textbf{investment in something} the action or process of spending money on something in order to make it better or more successful; \textbf{5.} [uncountable, countable] the action of giving time, energy or effort to a task in order to achieve something.} was to set a deadline\footnote{\textbf{deadline} [n] a point in time by which something must be done.} for Shader to make his decision. If Hornik made a compelling\footnote{\textbf{compelling} [a] \textbf{1.} that makes you think it is true or valid; \textbf{2.} making you pay attention through beings so interesting \& exciting; \textbf{3.} that cannot be resisted.} offer with a short fuse\footnote{\textbf{fuse} [v] [intransitive, transitive] (of 2 things) to join together to form a single thing; to join 2 things in this way.}, Shader might sign it before he had the chance to pitch to other investors. This is what many venture capitalists do to stack\footnote{\textbf{stack} [n] \textbf{1.} a pile of something, usually neatly arranged; \textbf{2.} (\textit{computing}) a way of storing data in a computer in which only the most recently stored item can be retrieved ($=$ found or got back). \textbf{Stacks} are often called `last in, 1st out' data structures; [v] [transitive, often passive] \textbf{stack something ($+$ adv.\texttt{/}prep.)} to arrange objects neatly, especially in a pile.} the odds\footnote{\textbf{odds} [n] [plural] \textbf{1.} (usually \textbf{the odds}) the degree to which something is likely to happen; \textbf{2.} greater advantage; the state of being greater in strength, power or resources; \textbf{be at odds (with something)} [idiom] to be different from something, when the 2 things should be the same; \textbf{be ad odds (with somebody) (over\texttt{/}on something)} [idiom] to disagree with somebody about something.} in their favor\footnote{\textbf{favour} [n] (\textit{US} \textbf{favor}) \textbf{1.} [countable] a thing that you do to help somebody; \textbf{2.} [uncountable] approval or support for somebody\texttt{/}something; \textbf{find favor (with somebody\texttt{/}something)} [idiom] to become accepted \& popular; \textbf{in favor (of somebody\texttt{/}something)} [idiom] \textbf{1.} supporting \& agreeing with something\texttt{/}somebody; \textbf{2.} likely to produce a particular result, often in an unfair way; \textbf{3.} in exchange for another thing (because the other thing is better or you want it more); \textbf{in somebody's favor} [idiom] \textbf{1.} if something is in somebody's favor, it gives them an advantage or helps them; \textbf{2.} a decision or judgment that is in somebody's favor benefits that person or says that they were right; [v] \textbf{1.} to prefer 1 thing to another, especially a particular system, plan or way of doing something; \textbf{2.} to treat somebody\texttt{/}something better than others, especially in an unfair way.}.

But Hornik didn't give Shader a deadline. In fact, he practically\footnote{\textbf{practically} [adv] \textbf{1.} almost; very nearly, \textsc{synonym}: \textbf{virtually}; \textbf{2.} in a realistic or sensible way; in real situations.} invited Shader to shop his offer around to other investors. Hornik believed that entrepreneurs need time to evaluate\footnote{\textbf{evaluate} [v] to form an opinion of the amount, value or quality of something after thinking about it carefully, \textsc{synonym}: \textbf{assess}.} their options, so as a matter of principle\footnote{\textbf{principle} [n] \textbf{1.} [countable] a law, rule or theory that something is based on; \textbf{2.} [singular] a general or scientific law that explains how something works or why something happens; \textbf{3.} [countable] a belief that is accepted as a reason for acting or thinking in a particular way; \textbf{4.} [countable, usually plural, uncountable] a moral rule or a strong belief that influences your actions; \textbf{in principle} [idiom] \textbf{1.} if something can be done in principle, there is no good reason why it should not be done although it has not yet been done \& there may be some difficulties; \textbf{2.} in general but not in detail.}, he refused to present exploding\footnote{\textbf{explode} [v] \textbf{1.} [intransitive, transitive] to burst or make something burst loudly \& violently, causing damage, \textsc{synonym}: \textbf{blow up}; \textbf{2.} [intransitive] \textbf{explode (into something)} (of a situation) to suddenly become very violent or dangerous; \textbf{3.} [intransitive] to increase suddenly \& very quickly in number or amount.} offers. ``Take as much time as you need to make the right decision,'' he said. Although Hornik hoped Shader would conclude that the right decision was to sign with him, he put Shader's best interests ahead of his own, giving Shader space to explore\footnote{\textbf{explore} [v] \textbf{1.} [transitive] to examine something completely or carefully in order to find out more about it, \textsc{synonym}: \textbf{analyze}; \textbf{2.} [transitive, intransitive] to travel to or around an area or a country in order to learn about it.} other options.

Shader did just that: he spent the next few weeks pitching his idea to other investors. In the meantime\footnote{\textbf{in the meantime} [idiom] in the period of time between 2 times or 2 events; between now \& a future event.}, Hornik wanted to make sure he was still a strong contender\footnote{\textbf{contender} [n] a person or team with a chance of winning a competition.}, so he sent Shader his most valuable resource: a list of 40 references\footnote{\textbf{reference} [n] \textbf{1.} [countable, uncountable] a thing you say or write that mentions somebody\texttt{/}something else; the act of mentioning somebody\texttt{/}something; \textbf{2.} [countable] a mention of a source of information in a book, an article, etc; a source of information that is mentioned in this way; \textbf{3.} [uncountable] the act of looking at something for information; \textbf{4.} \textbf{$+$ noun} a measure, substance, set of values, etc. that is regarded as normal or typical \& used when making comparisons; \textbf{5.} [countable] (abbr., \textbf{ref.}) a number, word or symbol that shows where you can find a piece of information; [v] \textbf{1.} \textbf{reference something} (\textit{formal}) to refer to something; \textbf{2.} [usually passive] to provide a book, an article, etc. with references.} who could attest\footnote{\textbf{attest} [v] (\textit{formal}) \textbf{1.} [intransitive, transitive, usually passive] to show or state that something exists or is true, \textbf{bear witness\texttt{/}testimony to something};\textbf{2.} [transitive] \textbf{attest something} to make an official statement that something is true or genuine, \textsc{synonym}: \textbf{witness}.} to Hornik's caliber\footnote{\textbf{calibre} [n] (\textit{US English} \textbf{caliber}) \textbf{1.} [uncountable] the quality of something, especially a person's ability, \textsc{synonym}: \textbf{standard}; \textbf{2.} [countable] the measurement from 1 side of the inside of a tube or gun to the other; the measurement from 1 side of a bullet to the other.} as an investor. Hornik knew that entrepreneurs look for the same attributes\footnote{\textbf{attribute} [v] \textbf{1.} \textbf{attribute something to something} to say or believe that something is the result of a particular thing; \textbf{2.} \textbf{attribute something to somebody} to say or believe that somebody is responsible for doing something, especially for saying, writing or painting something; [v] a quality or feature of somebody\texttt{/}something.} in investors that we all seek in financial\footnote{\textbf{financial} [a] [usually before noun] connected with money \& finance.} advisers\footnote{\textbf{adviser} [n] (also \textbf{advisor}) a person who gives advice, especially somebody who knows a lot about a particular subject.}: competence\footnote{\textbf{competence} [n] \textbf{1.} (also \textit{less frequent} \textbf{competency}) [uncountable, countable] the ability to do something well, \textsc{opposite}: \textbf{incompetence}; \textbf{2.} [uncountable] \textbf{competence (of something)} the legal authority of a court or other institution or organization to deal with a particular matter; \textbf{3.} (also \textbf{competency}) [countable] \textbf{competence (to do something)} a skill needed in a particular job or for a particular task.} \& trustworthiness\footnote{\textbf{trustworthiness} [n] [uncountable] the quality of always being good, honest, sincere, etc. so that people can rely on you, \textsc{synonym}: \textbf{reliability}.}. When entrepreneurs sign with an investor, the investor joins their board of directors\footnote{\textbf{director} [n] \textbf{1.} 1 of a group of senior managers who are involved in running a company; \textbf{2.} a person who is in charge of a particular activity or department in a company, a college, etc.; \textbf{3.} a person in charge of a film or play who tells the actors \& staff what to do.} \& provides expert advice. Hornik's list of references reflected the blood, sweat\footnote{\textbf{sweat} [n] \textbf{1.} [uncountable] drops of liquid that appear on the surface of your skin when you are hot, ill or afraid; \textbf{2.} [countable] the state of being covered with sweat; [v] [intransitive] when you sweat, drops of liquid appear on the surface of your skin, e.g. when you are hot, ill or afraid.}, \& tears that he had devoted\footnote{\textbf{devote} [v] \textbf{devote yourself to somebody\texttt{/} something} to give most of your time, energy or attention to somebody\texttt{/}something, \textsc{synonym}: \textbf{dedicate}; \textbf{devote something to something} to give an amount of time, attention or resources to something.} to entrepreneurs over the course\footnote{\textbf{course} [n] \textbf{1.} [countable] a series of classes or lectures on a particular subject; \textbf{2.} [countable] (\textit{especially British English}) a period of study at a college or university that leads to an exam or a qualification; \textbf{3.} [singular] the way that something develops or should develop; \textbf{4.} (also \textbf{course of action}) [countable] a way of acting in or dealing with a particular situation; \textbf{5.} [countable, usually singular] the general direction in which somebody's ideas or actions are moving; \textbf{6.} [uncountable, countable, usually singular] a direction or route followed by a ship or an aircraft, or by another moving object; \textbf{7.} [countable] \textbf{course (of something)} a series of medical treatments; \textbf{during\texttt{/}in\texttt{/}over the course of $\ldots$} [idiom] during; \textbf{of course} [idiom] used to show that what you are saying is not surprising or is generally known or accepted.} of more than a decade in the venture business. He knew they would vouch for\footnote{\textbf{vouch for} [phrasal verb] \textbf{vouch for somebody\texttt{/}something} (\textit{formal}) to say that you believe that somebody will behave well \& that you will be responsible for their actions; \textbf{vouch for something} (\textit{formal}) to say that you believe that something is true or good because you have evidence for it, \textsc{synonym}: \textbf{confirm}.} his skill \& his character.

A few weeks later, Hornik's phone rang. It was Shader, ready to announce\footnote{\textbf{announce} [v] \textbf{1.} to make a formal public statement about a fact, event or intention; \textbf{2.} to say something in a loud \&\texttt{/}or serious way.} his decision.

``I'm sorry,'' Shader said, ``but I'm signing with another investor.''

The financial terms of the offer from Hornik \& the other investor were virtually\footnote{\textbf{virtually} [adv] \textbf{1.} almost or very nearly, so that any slight difference is not important; \textbf{2.} by the use of computer software that makes something appear to exist; \textbf{3.} by means of computers \& computer networks.} identical\footnote{\textbf{identical} [a] \textbf{1.} similar in every detail; \textbf{2.} (\textbf{the identical}) [only before noun] the same.}, so Hornik's list of 40 references should have given him an advantage. \& after speaking with the references, it was clear to Shader that Hornik was a great guy.

But it was this very same spirit of generosity\footnote{\textbf{generosity} [n] [uncountable] the quality of being kind \& generous.} that doomed\footnote{\textbf{doom} [n] [uncountable] death or destruction; any terrible event that you cannot avoid; \textbf{doom \& gloom $|$ gloom \& doom} [idiom] a general feeling of having lost all hope, \& of pessimism ($=$ expecting things to go badly); [v] [usually passive] to make somebody\texttt{/}something certain to fail, suffer, die, etc.} Hornik's case. Shader worried that Hornik would spend more time encouraging him than challenging him. Hornik might not be tough\footnote{\textbf{tough} [a] (\textbf{tougher, toughest}) \textbf{1.} (of a thing) not easily damaged; strong; \textbf{2.} (\textit{rather informal}) having or causing problems, \textsc{synonym}: \textbf{difficult}; \textbf{3.} (\textit{rather informal}) demanding that laws be obeyed, \& not accepting any reasons for not obeying them, \textsc{opposite}; \textbf{soft}; \textbf{4.} (\textit{rather informal}) (of a person) strong enough to deal successfully with difficult conditions or situations. \textbf{Tough} can sometimes suggest that somebody may be violent. The more formal word \textbf{resilient} does not suggest this.} enough to help Shader start a successful business, \& the other investor had a reputation for being a brilliant adviser who questioned \& pushed entrepreneurs. Shader walked away thinking, ``I should probably add somebody to the board who will challenge me more. Hornik is so affable\footnote{\textbf{affable} [a] pleasant, friendly \& easy to talk to, \textsc{synonym}: \textbf{genial}.} that I don't know what he'll be like in the boardroom\footnote{\textbf{boardroom} [n] a room in which the meetings of the board of a company ($=$ the group of people who control it) are held.}.'' When he called Hornik, he explained, ``My heart said to go with you, but my head said to go with them. I decided to go with my head instead of my heart.''

Hornik was devastated\footnote{\textbf{devastated} [a] extremely upset \& shocked.}, \& he began to 2nd-guess\footnote{\textbf{second-guess} [v] \textbf{1.} [transitive] \textbf{second-guess somebody\texttt{/}something\texttt{/}yourself} to guess what somebody will do before they do it; to guess how you will feel in the future; \textbf{2.} [transitive, intransitive] \textbf{second-guess (somebody\texttt{/}something)} (\textit{especially North American English}) to criticize somebody after a decision has been made; to criticize something after it has happened.} himself. ``Am I a dope\footnote{\textbf{dope} [n] \textbf{1.} [uncountable] (\textit{informal}) a drug that is used illegally for pleasure. In (\textit{British English}) \textbf{dope} usually means cannabis. In (\textit{North American English}) it can also refer to heroin.; \textbf{2.} [uncountable] a drug that is taken by a person or given to an animal to affect their performance in a race or sport; \textbf{3.} [countable] (\textit{informal}) a stupid person, \textsc{synonym}: \textbf{idiot}; \textbf{4.} [uncountable] \textbf{the dope (on somebody\texttt{/}something)} (\textit{informal}) information on somebody\texttt{/}something, especially details that are not generally known; [v] \textbf{1.} \textbf{dope somebody\texttt{/}something} to give a drug to a person or an animal in order to affect their performance in a race or sport; \textbf{2.} \textbf{dope somebody\texttt{/}something} to give somebody a drug, often in their food or drink, in order to make them unconscious; to put a drug in food, etc.; \textbf{3.} [usually passive] \textbf{dope somebody (up)} (\textit{informal}) if somebody is \textbf{doped} or \textbf{doped up}, they cannot think clearly or act normally because they are under the influence of drugs.}? If I had applied pressure to take the term sheet, maybe he would have taken it. But I've spent a decade building my reputation so this wouldn't happen. How did this happen?''

David Hornik learned his lesson the hard way: \fbox{good guys finish last}.

Or do they?

According to conventional\footnote{\textbf{conventional} [a] \textbf{1.} [usually before noun] based on what is generally believed; following the way something is usually done; \textbf{2.} (\textit{often disapproving}) tending to follow what is done or considered acceptable by society in general; normal \& ordinary, \& perhaps not very interesting, \textsc{opposite}: \textbf{unconventional}; \textbf{3.} [usually before noun] (especially of weapons) not nuclear; \textbf{4.} (of the literature, art or the theater) using a traditional style or method.} wisdom\footnote{\textbf{wisdom} [n] \textbf{1.} [uncountable, singular] the ability to make sensible decisions \& give good advice, because of the experience \& knowledge that you have; \textbf{2.} [uncountable, countable] the knowledge \& experience that develops within a particular society or group of people. \textbf{(The) conventional\texttt{/}received wisdom} is what most people believe to be true. \textbf{Common, popular} \& \textbf{traditional} are also used in this way.; \textbf{3.} [singular] \textbf{the wisdom of (doing) something} how sensible something is.}, highly successful people have 3 things in common: motivation\footnote{\textbf{motivation} [n] \textbf{1.} [countable] a reason or reasons for doing a particular activity or behaving in particular way; \textbf{2.} [uncountable, countable] desire or willingness to do something.}, ability\footnote{\textbf{ability} [n] (plural \textbf{abilities}) \textbf{1.} [singular] the fact that somebody\texttt{/}something is able to do something, \textsc{opposite}: \textbf{inability}; \textbf{2.} [uncountable, countable] a level of skill or intelligence.}, \& opportunity\footnote{\textbf{opportunity} [n] (plural \textbf{opportunities}) [countable, uncountable] a situation or time that makes it possible to do or achieve something, \textsc{synonym}: \textbf{chance}. As an uncountable noun, \textbf{opportunity} is used to talk about the existence or extent of opportunities; \textbf{at the earliest opportunity} [idiom] as soon as possible.}. If we want to succeed, we need a \fbox{combination of hard work, talent, \& luck}. The story of Danny Shader \& David Hornik highlights\footnote{\textbf{highlight} [v] \textbf{1.} to emphasize something, especially so that people give it more attention; \textbf{2.} \textbf{highlight something (in something)} to mark part of a text with a special colored pen, or to make an area on a computer screen, to emphasize it or make it easier to see; [n] \textbf{highlight (of something)} the best, most interesting or most exciting part of something.} a 4th ingredient\footnote{\textbf{ingredient} [n] \textbf{1.} 1 of the substances from which something is made; \textbf{2.} 1 of the things or qualities that are necessary to make something what it is.}, one that's critical\footnote{\textbf{critical} [a] \textbf{1.} extremely important, e.g. because a future situation will be affected by it, \textsc{synonym}: \textbf{crucial}; \textbf{2.} involving making fair, careful judgments about the good \& bad qualities of somebody\texttt{/}something, \textsc{opposite}: \textbf{uncritical}; \textbf{3.} challenging traditional ideas in the study of society, literature, etc.; \textbf{4.} (of a text) containing detailed notes \& analysis by an expert; \textbf{5.} \textbf{critical (of somebody\texttt{/}something)} expressing disapproval of somebody\texttt{/}something \& saying what you think is bad about them\texttt{/}it; \textbf{6.} used to describe a situation that is serious \& uncertain \& in which bad things could happen; \textbf{7.} [only before noun] according to the judgment of people whose job is to write or broadcast their opinions about art, music, plays, etc.} but often neglected\footnote{\textbf{neglect} [v] \textbf{1.} \textbf{neglect somebody\texttt{/}something} to fail to take care of somebody\texttt{/}something; \textbf{2.} \textbf{neglect something} to not give enough attention to something; \textbf{3.} \textbf{neglect something} to ignore something because it is not important, especially in a scientific experiment, \textsc{synonym}: \textbf{disregard}; \textbf{4.} \textbf{neglect to do something} to fail or forget to do something that you ought to do, \textsc{synonym}: \textbf{omit}; [n] [uncountable] the fact of not giving enough care or attention to somebody\texttt{/}something; the stae of not receiving enough care or attention.}\,\footnote{\textbf{neglected} [a] \textbf{1.} not receiving enough attention; \textbf{2.} not receiving enough care.}: success depends heavily on how we approach our interactions\footnote{\textbf{interaction} [n] [uncountable, countable] \textbf{1.} the effect that 2 things have on each other; \textbf{2.} the way that people communicate with each other, especially while they work or spend time with them.} with other people. Every time we interact\footnote{\textbf{interact} [v] \textbf{1.} [intransitive] if 1 thing interacts with another, or if 2 things interact, 1 thing has an effect on the other, or the 2 things have an effect on each other; \textbf{2.} [intransitive] \textbf{interact (with somebody)} to communicate with somebody, especially while you work or spend time with them.} with another person at work, we have a choice to make: \textit{do we try to claim as much value as we can, or contribute\footnote{\textbf{contribute} [v] \textbf{1.} [intransitive] \textbf{contribute (to something)} to be 1 of the causes of something; \textbf{2.} [intransitive, transitive] to help or improve or achieve something, especially by adding new ideas; \textbf{3.} [transitive, intransitive] to give something, especially money or goods, to help somebody\texttt{/}something; \textbf{4.} [transitive, intransitive] to write something for a newspaper, magazine, website, or a radio or television programme; to speak during a meeting or conversation, especially to give your opinion.} value without worrying about what we receive in return?}

As an organizational\footnote{\textbf{organizational} [a] (\textit{British English also} \textbf{organisational}) \textbf{1.} connected with an organization or with organizations in general; \textbf{2.} connected with the ability to arrange or organize things well.} psychologist \& Wharton professor, I've dedicated\footnote{\textbf{dedicate} [v] \textbf{1.} to give time \& effort to a particular activity or purpose because you think it is important, \textsc{synonym}: \textbf{devote}; \textbf{2.} \textbf{dedicate something to something} to use all or part of a piece of writing to discuss a particular subject; \textbf{3.} \textbf{dedicate something to somebody} to say at the beginning of a book, a piece of music or a performance that you are doing it for somebody, as a way of thanking them or showing respect; \textbf{4.} \textbf{dedicate something (to somebody\texttt{/}something)} to officially say that a building or an object has a special purpose, especially a religious one.}\,\footnote{\textbf{dedicated} [a] \textbf{1.} working hard at something because it is very important to you, \textsc{synonym}: \textbf{committed}; \textbf{2.} [only before noun] designed to do only 1 particular type of work; used for 1 particular purpose only.} more than 10 years of my professional\footnote{\textbf{professional} [a] \textbf{1.} [only before noun] connected with a job that needs special training or skill, especially one that needs a high level of education; \textbf{2.} (of people) having a job that needs special training \& a high level of education; \textbf{3.} showing that somebody is well trained \& extremely skilled, \textsc{synonym}: \textbf{competent}; \textbf{4.} suitable or appropriate for somebody working in a particular profession; \textbf{5.} doing something as a paid job rather than just for pleasure.} life to studying these choices at organizations\footnote{\textbf{organization} [n] (\textit{British English also} \textbf{organisation}) \textbf{1.} [countable] an organized group of people with a particular purpose, such as a business or government department; \textbf{2.} [uncountable] the way in which the different parts of something are arranged, \textsc{synonym}: \textbf{structure}; \textbf{3.} [uncountable] the act of making arrangements or preparations for something, \textsc{synonym}: \textbf{planning}; \textbf{4.} [uncountable] the quality of being arranged in a neat, careful \& logical way; the ability to plan your work or life well \& in an efficient way.} ranging from Google to the U.S. Air Force, \& it turns out that they have staggering\footnote{\textbf{staggering} [a] (\textit{rather informal}) so great, shocking or surprising that is difficult to believe, \textsc{synonym}: \textbf{astounding}.} consequences\footnote{\textbf{consequence} [n] \textbf{1.} [countable] (often plural) a result of something that has happened; \textbf{2.} [uncountable] importance; \textbf{in consequence (of something)} [idiom] as a result of something.} for success. Over the past 3 decades, in a series of groundbreaking\footnote{\textbf{groundbreaking} [a] [usually before noun] making new discoveries; using new methods.} studies, social scientists\footnote{\textbf{social scientist} [n] a person who studies social science.} have discovered\footnote{\textbf{discover} [v] \textbf{1.} to find some new information about something; \textbf{2.} \textbf{discover something} to be the 1st person to realize that a particular thing or place exists; \textbf{3.} \textbf{discover somebody\texttt{/}something} to find somebody\texttt{/}something that was hidden or that you did not expect to find.} that people differ\footnote{\textbf{differ} [v] \textbf{1.} [intransitive] to be different from somebody\texttt{/}something; \textbf{2.} [intransitive] to disagree with somebody; \textbf{somebody begs to differ} [idiom] used to say that somebody does not agree with something that has just been said.} dramatically\footnote{\textbf{dramatically} [adv] \textbf{1.} in a very sudden or extreme way; to a very great degree; \textbf{2.} in a way that is exciting or impressive; \textbf{3.} using the style of a play in telling a story or giving an account of an event.} in their \underline{preferences\footnote{\textbf{preference} [n] \textbf{1.} [countable, usually singular, uncountable] a greater interest in or desire for somebody\texttt{/}something than somebody\texttt{/}something else; \textbf{2.} [countable] a thing that is liked better or best; \textbf{give (a) preference to somebody\texttt{/}something} [idiom] to treat somebody\texttt{/}something in a way that gives them\texttt{/}it an advantage over other people or things; \textbf{in preference to somebody\texttt{/}something} [idiom] rather than somebody\texttt{/}something.} for reciprocity\footnote{\textbf{reciprocity} [n] [uncountable] a situation in which 2 people, countries, etc. provide the same help or advantages to each other.}}\footnote{Edward W. Miles, John D. Hatfield, \& Richard C. Huseman, ``The Equity Sensitivity Construct: Potential Implications for Worker Performance,'' \textit{Journal of Management} 15 (1989): 581--588.}---their desired\footnote{\textbf{desire} [n] \textbf{1.} [countable, uncountable] a strong wish to have or do something; \textbf{2.} [uncountable] \textbf{desire (for somebody)} a strong wish to have sex with somebody; [v] (not used in the progressive tenses) (\textit{formal}) to want something.} mix of taking \& giving. To shed\footnote{\textbf{shed} [v] \textbf{1.} \textbf{shed something} if an animal sheds its skin, or a plant sheds leaves, it loses them naturally; \textbf{2.} \textbf{shed something} to let something fall; to drop something; \textbf{3.} \textbf{shed something} to get rid of something that is no longer wanted; \textbf{4.} \textbf{shed blood} to kill or injure people, especially in a war; \textbf{cast\texttt{/}shed\texttt{/}throw light on something} [idiom] to help people to understand something, especially in a new way, by providing new explanations or information.} some light on these preferences, let me introduce you to 2 kinds of people who fall at opposite\footnote{\textbf{opposite} [a] \textbf{1.} [usually before noun] as differen as possible from something; involving 2 different extremes; \textbf{2.} [usually before noun] on the other side of something or facing something; \textbf{the opposite sex} [idiom] the other sex; \textbf{pull in different\texttt{/}opposite directions} [idiom] to have different aims that cannot be achieved together without causing problems; [n] \textbf{1.} (\textbf{the opposite}) [singular] the situation, idea or activity that is as different from another situation, etc. as it is possible to be, \textsc{synonym}: \textbf{the reverse}; \textbf{2.} (\textbf{opposites}) [plural] people, ideas or situations that are as different as possible from each other; \textbf{the exact opposite} [idiom] a person or thing that is as different as possible from somebody\texttt{/}something else; [prep] on the other side of a particular area from somebody\texttt{/}something, \& usually facing them.} ends of the reciprocity\footnote{\textbf{reciprocity} [n] [uncountable] a situation in which 2 people, countries, etc. provide the same help or advantages to each other.} spectrum\footnote{\textbf{spectrum} [n] (plural \textbf{spectra}) \textbf{1.} a pattern of colored bands formed when light is split into its constituent wavelengths, as seen, e.g., in a rainbow; \textbf{2.} (\textit{physics}) any signal ordered by 1 of its properties, such as its energy or mass; \textbf{3.} [usually singular] a complete or wide range of people or things.} at work. I call them \textit{takers \& givers}.

\textit{Takers} have a distinctive\footnote{\textbf{distinctive} [a] having a quality or characteristic that makes something different \& easily noticed, \textsc{synonym}: \textbf{characteristic}.} signature\footnote{\textbf{signature} [n] \textbf{1.} [countable] your name as you usually write it, e.g., at the end of a letter; \textbf{2.} [uncountable] the act of signing something; \textbf{3.} [countable] a particular quality that makes something different from other similar things \& makes it easy to recognize.}: \fbox{they like to get more than they give}. They tilt\footnote{\textbf{tilt} [v] \textbf{1.} [intransitive, transitive] to move into a position with 1 side or end higher than the other; to make something move in this way, \textsc{synonym}: \textbf{tip}; \textbf{2.} [transitive, intransitive] to influence a situation so that 1 particular opinion, person, etc. is preferred or more likely to succeed than another; to change in this way; [n] [singular, uncountable] a position in which 1 end or side of something is higher than the other.} reciprocity in their own favor putting their own interests ahead of others' needs. Takers believe that the world is a competitive\footnote{\textbf{competitive} [a] \textbf{1.} connected with competition, especially in the world of business; \textbf{2.} (used especially about the world of business) as good as, or better than, others; \textbf{3.} competing to be the best; \textbf{4.} (\textit{ecology}) connected with a situation in which animals, plants or other living things compete to get resources.}, dog-eat-dog\footnote{in a doggy style?} place. \fbox{They feel that to succeed, they need to be better than others.} To prove their competence\footnote{\textbf{competence} [n] \textbf{1.} (also \textit{less frequent} \textbf{competency}) [uncountable, countable] the ability to do something well, \textsc{opposite}: \textbf{incompetence}; \textbf{2.} [uncountable] \textbf{competence (of something)} the legal authority of a court or other institution or organization to deal with a particular matter; \textbf{3.} (also \textbf{competency}) [countable] \textbf{competence (to do something)} a skill needed in a particular job or for a particular task.}, they self-promote\footnote{\textbf{self-promotion} [n] [uncountable] (\textit{disapproving}) the activity of making people notice you \& your abilities, especially in a way that annoys other people.} \& make sure they get plenty\footnote{\textbf{plenty} [pronoun] a large amount; as much as or many as you need.} of credit\footnote{\textbf{credit} [n] \textbf{1.} [uncountable] praise or approval that is given to somebody because they are responsible for something; \textbf{2.} [uncountable] the act of stating who somebody is \& the work that they have done; \textbf{3.} [uncountable, countable] money that is borrowed from a bank or another financial institution; a loan; \textbf{4.} [uncountable] the status of being trusted to pay back money to somebody who lends it to you; \textbf{5.} [countable, uncountable] (\textit{specialist}) payment that somebody has a right to receive for a particular reason; \textbf{6.} [uncountable] an arrangement that you make, e.g. with a shop, to pay later for something that you buy; \textbf{7.} [uncountable] if you or your bank account are in credit, there is money in the account; \textbf{8.} [countable, uncountable] a unit of study at a college or university (or, in the US, also at a school); the fact of having successfully completed a unit of study; \textbf{to somebody's credit} [idiom] making somebody deserve praise or respect; [v] [usually passive] to believe to say that somebody\texttt{/}something is responsible for doing something, especially something good.} for their efforts\footnote{\textbf{effort} [n] \textbf{1.} [uncountable, countable] the physical or mental energy that you need to do something; something that takes a lot of energy; \textbf{2.} [countable] an attempt to do something, especially when it is difficult to do; \textbf{3.} [countable] (usually after a noun) a particular activity that a group of people organizes in order to achieve something.}. Garden-variety\footnote{\textbf{garden-variety} [a] (\textit{North American English}) (\textit{British English} \textbf{common or garden}) [only before noun] ordinary; with no special features.} takers aren't cruel\footnote{\textbf{cruel} [a] (\textbf{crueller, cruellest}) \textbf{1.} having a desire to cause pain \& suffering, \textsc{opposite}: \textbf{kind}; \textbf{2.} causing pain or suffering, \textsc{synonym}: \textbf{harsh}.} or cutthroat\footnote{\textbf{cutthroat} [a] [usually before noun] (of an activity) in which people compete with each other in aggressive \& unfair ways.}; they're just cautious\footnote{\textbf{cautious} [a] being careful about what you say or do, especially in order to avoid danger or mistakes; not taking any risks.} \& self-protective\footnote{\textbf{protective} [a] \textbf{1.} [only before noun] providing or intended to provide protection; \textbf{2.} \textbf{protective (of somebody\texttt{/}something)} having or showing a wish to protect somebody\texttt{/}something; \textbf{3.} intended to give an advantage to your own country's industry.}. ``If I don't look out for myself 1st,'' takers think, ``no one will.'' Had David Hornik been more of a taker, he would have given Danny Shader a deadline, putting his goal of landing the investment ahead of Shader's desire for a flexible\footnote{\textbf{flexible} [a] \textbf{1.} able to change to suit new conditions or situations, \textsc{opposite}: \textbf{inflexible}. In economics, \textbf{flexible} is used to describe prices, wages, exchange rates, etc. that are quick to change or react to change. \textsc{opposite}: \textbf{sticky}; \textbf{2.} able to bend easily.} timeline\footnote{\textbf{timeline} [n] a horizontal line that is used to represent time, with the past towards the left \& the future towards the right.}.

But Hornik is the opposite of a taker; he's a \textit{giver}. In the workplace\footnote{\textbf{workplace} [n] (often \textbf{the workplace}) [singular] a place where people work, such as an office or factory.}, givers are a relatively\footnote{\textbf{relatively} [adv] to a fairly large degree, especially in comparison with something else; \textbf{relatively speaking} [idiom] used when you are comparing something with all similar things.} rare\footnote{\textbf{rare} [a] (\textbf{rarer, rarest}) \textbf{1.} not done, seen, happening, etc. very often; \textbf{2.} existing only in small numbers \& therefore valuable or interesting.} breed\footnote{\textbf{breed} [v] \textbf{1.} [intransitive] (of animals) to have sex \& produce young; \textbf{2.} [transitive] to keep animals or plants in order to produce young ones in a controlled way; \textbf{3.} [transitive] \textbf{breed something} to be the causes of something; [n] \textbf{1.} a type of animal with a particular appearance that makes it different from others of the same species \& that is the result of having been developed in a controlled way; \textbf{2.} [usually singular] a type of person.}. They tilt reciprocity in the other direction, preferring to give more than they get. Whereas \fbox{takers tend to be self-focused}, evaluating what other people can offer them, \fbox{givers are other-focused}, paying more attention to what other people need from them. These preferences aren't about money: givers \& takers aren't distinguished\footnote{\textbf{distinguish} [v] \textbf{1.} [intransitive, transitive] to recognize or show the difference between 2 people or things, \textsc{synonym}: \textbf{differentiate}; \textbf{2.} [transitive] (not used in the progressive tenses) to be a characteristic that makes 2 people, animals or things different, \textsc{synonym}: \textbf{differentiate}; \textbf{3.} [transitive] \textbf{distinguish A (from B)} to make something different or seem different from other similar things, \textsc{synonym}: \textbf{differentiate}; \textbf{4.} [transitive] to do something so well that people notice \& admire you; \textbf{5.} [transitive] (not used in the progressive tenses) \textbf{distinguish  something} to be able to see or hear something, \textsc{synonym}: \textbf{make somebody\texttt{/}something out}.}\,\footnote{\textbf{distinguished} [a] very successful \& admired by other people.} by how much they donate\footnote{\textbf{donate} [v] \textbf{1.} [transitive, intransitive] to give money, food, clothes, etc. to somebody\texttt{/}something, especially a charity; \textbf{2.} [transitive, intransitive] to allow doctors to remove blood, a body organ, etc. in order to help somebody who needs it, or so that it can be used for research; \textbf{3.} [transitive] \textbf{donate something (to something)} (\textit{chemistry, physics}) to provide 1 or more electrons or a protons to another form, molecule or ion.} to charity\footnote{\textbf{charity} [n] (plural \textbf{charities}) \textbf{1.} [countable] an organization for helping people in need; \textbf{2.} [uncountable] charities considered as a group; \textbf{3.} [uncountable] money, food, help, etc. that is given to people who are in need; \textbf{4.} [uncountable] kindness \& sympathy towards other people, especially when you are judging them.} or the compensation\footnote{\textbf{compensation} [n] [uncountable] \textbf{1.} money that somebody receives because they have been hurt or have suffered loss or damage; \textbf{2.} something that reduces, balances or removes the negative effect of something; \textbf{3.} \textbf{compensation (for something\texttt{/}doing something)} a reward or benefit that somebody receives in return for doing something.} that they command\footnote{\textbf{command} [n] \textbf{1.} [uncountable] \textbf{command (of somebody\texttt{/}something)} control \& authority over a situation or a group of people; \textbf{2.} [singular, uncountable] \textbf{command (of something)} your knowledge of something; your ability to do or use something, especially a language; \textbf{3.} [countable] an order given to a person or an animal; \textbf{4.} [countable] an instruction causing a computer to perform a function; \textbf{at your command} [idiom] if you have a skill or an amount of something at your command, you are able to use it well \& completely; [v] \textbf{1.} [transitive] (of somebody in a position of authority) to tell somebody to do something, \textsc{synonym}: \textbf{order}; \textbf{2.} [transitive, intransitive] \textbf{command (somebody\texttt{/}something)} to be in charge of a group of people in the army, navy, etc.; \textbf{3.} [transitive, no passive] (not used in the progressive tenses) \textbf{command something} to deserve \& get something because of the special qualities you have; \textbf{4.} [transitive, no passive] (not used in the progressive tenses) \textbf{command something} to be in a strong enough position to have or get something; \textbf{5.} [transitive, no passive] (not used in the progressive tenses) \textbf{command something} to have something available for use; \textbf{6.} [transitive, no passive] (not used in the progressive tenses) \textbf{command something} to be in a position from where you can see or control something.} from their employers\footnote{\textbf{employer} [n] a person or company that pays people to work for them}. Rather\footnote{\textbf{rather} [adv] \textbf{1.} used to mean `fairly' or `to some degree', often when expressing slight criticism or surprise; \textbf{2.} used to correct something you have said, or to give more accurate information; \textbf{3.} used to introduce an idea that is different or opposite to the idea that you have stated previously; \textbf{would rather $\ldots$ (than $\ldots$)} [idiom] would prefer to.}, gives \& takers differ in their attitudes\footnote{\textbf{attitude} [n] [countable, uncountable] a way of thinking or feeling about somebody\texttt{/}something; the way of behaving towards somebody\texttt{/}something that shows how somebody thinks or feels.} \& actions\footnote{\textbf{action} [n] \textbf{1.} [uncountable] the fact or process of doing something, typically to achieve an aim; \textbf{2.} [countable] a thing that somebody does; \textbf{3.} [countable, uncountable] \textbf{action (for something) (against somebody)} a legal process to stop a person or company from doing something, or to make them pay for a mistake, etc.; \textbf{4.} [uncountable] fighting in a battle or war; \textbf{5.} [uncountable] the effect that 1 substance or chemical has on another; \textbf{in action} [idiom] if somebody\texttt{/}something is in action, they are doing the activity or work that is typical for them; \textbf{into action} [idiom] if you put an idea or a plan into action, you start making it happen or work.} toward other people. If you're a taker, you help others strategically\footnote{\textbf{strategically} [adv] \textbf{1.} in a way that is connected with achieving a particular purpose or gaining an advantage; \textbf{2.} in a way that is connected with gaining an advantage in a war or other military situation.}, when the benefits to \textit{you} outweigh\footnote{\textbf{outweigh} [v] \textbf{outweigh something} to be greater or more important than something.} the personal costs. If you're a giver, you might use a different cost-benefit\footnote{\textbf{cost-benefit} [n] [uncountable] the relationship between the cost of doing something \& the value of the benefit that results from it.} analysis\footnote{\textbf{analysis} [n] (plural \textbf{analyses}) \textbf{1.} [uncountable, countable] the detailed study or examination of something in order to understand more about it; the result of the study; \textbf{2.} [uncountable, countable] a careful examination of a substance in order to find out what it consists of; \textbf{3.} [uncountable] $=$ \textbf{psychoanalysis}.}: you help whenever\footnote{\textbf{whenever} [conjunction] \textbf{1.} every time that; \textbf{2.} at any time that; on any occasion that.} the benefits\footnote{\textbf{benefit} [n] \textbf{1.} [countable, uncountable] a helpful \& useful effect that something has; an advantage that something provides; \textbf{2.} [uncountable, countable] (\textit{British English}) money provided by the government to people who need financial help because they are unemployed, sick, etc.; \textbf{give somebody the benefit of the doubt} [idiom] to accept that somebody has told the truth or has not done something wrong because you cannot prove that they have not told the truth\texttt{/}have done something wrong; [v] \textbf{1.} [intransitive] to be in a better position because of something; \textbf{2.} [transitive] \textbf{benefit somebody\texttt{/}something} to be useful or provide an advantage to somebody\texttt{/}something.} to \textit{others} exceed\footnote{\textbf{exceed} [v] \textbf{1.} \textbf{exceed something} to be greater than a particular number or amount; \textbf{2.} \textbf{exceed something} to go beyond what the law, an order or a rule says you are allowed to do; \textbf{3.} \textbf{exceed something} to be better than something, \textsc{synonym}: \textbf{surpass}.} the personal costs. Alternatively\footnote{\textbf{alternatively} [adv] used to introduce a suggestion that is a 2nd choice or possibility.}, you might not think about personal costs at all, helping others without expecting anything in return. If you're a giver at work, you simply strive\footnote{\textbf{strive} [v] [intransitive] to try very hard to achieve something.} to be generous\footnote{\textbf{generous} [a] (\textit{approving}) \textbf{1.} giving or willing to give time, money, etc. freely; given freely; \textbf{2.} more than is necessary; large; \textbf{3.} kind in the way you treat people; willing to see what is good about somebody\texttt{/}something.} in sharing your time, energy, knowledge, skills, ideas, \& connections with other people who can benefit from them.

It's temping\footnote{\textbf{temp} [v] [intransitive] (\textit{informal}) to do a temporary job or a series of temporary jobs; [abbr.] (also \textbf{temp.} \textit{especially in North American English}) temperature.} to reserve\footnote{\textbf{reserve} [n] \textbf{1.} [countable, usually plural] a supply of something that is available to be used in the future or when it is needed; \textbf{2.} (\textit{North American English also} \textbf{preserve}) [countable] a piece of land that is a protected area for animals, plants, etc.; \textbf{3.} (\textbf{the reserve}) [singular] (\textbf{(the) reserves} [plural]) an extra military force, etc. thta is not part of a country's regular forces, but is available to be used when needed; \textbf{4.} [uncountable] a feeling that you do not want to accept or agree to something until you are quite sure that it is all right to do so; \textbf{5.} [uncountable] the quality that somebody has when they do not talk easily to other people about their ideas or feelings; \textbf{6.} (also \textbf{reserve price}) [countable] (\textit{British English}) the lowest price that somebody will accept for something; \textbf{7.} [countable] $=$ \textbf{reservation}; \textbf{in reserve} [idiom] available to be used in the future or when needed; [v] \textbf{1.} \textbf{reserve something} to have or keep a particular power; \textbf{2.} to keep something for somebody\texttt{/}something, so that it cannot be used by any other person or for any other reason.} the giver label for larger-than-life heroes such as Mother Teresa or Mahatma Gandhi, but being a giver doesn't require extraordinary\footnote{\textbf{extraordinary} [a] \textbf{1.} unexpected, surprising or strange; \textbf{2.} not normal or ordinary; greater or better than usual; \textbf{3.} [only before noun] (of a meeting, etc.) arranged for a special purpose \& happening in addition to what normally or regularly happens.} acts of sacrifice\footnote{\textbf{sacrifice} [n] \textbf{1.} [countable, uncountable] the fact of giving up something important or valuable to you in order to get or do something that seems more important; something that you give up in this way; \textbf{2.} [countable, uncountable] the act of offering something to a god, especially an animal that has been killed in a special way; an animal, etc. that is offered in this way; [v] \textbf{1.} [transitive] to give up something that is important or valuable to you in order to get or do something that seems more important for yourself or for another person; \textbf{2.} [transitive, intransitive] to kill an animal or a person \& offer it\texttt{/}them to a god, in order to please the god.}. It just involves\footnote{\textbf{involve} [v] \textbf{1.} if a situation, an event or an activity involves something, that thing is an important or necessary part or result of it, \textsc{synonym}: \textbf{mean}; \textbf{2.} if a situation, an event or an activity involves somebody\texttt{/}something, they take part in it or are affected by it; \textbf{3.} to make somebody take part in something; \textbf{4.} \textbf{involve somebody (in something)} to say or do something to show that somebody took part in something, especially a crime, \textsc{synonym}: \textbf{implicate}.} a focus on acting in the interests of others, such as by giving help, providing mentoring\footnote{\textbf{mentoring} [n] [uncountable] the practice of helping \& advising a less experienced person over a period of time, especially as part of a formal programme in a company, university, etc.}, sharing credit, or making connections for others. Outside the workplace, this type of behavior is quite common. According to research led by Yale psychologist Margaret Clark, \underline{most people act like givers in close relationships}\footnote{Margaret S. Clark \& Judson Mills, ``The Difference between Communal \& Exchange Relationships: What It Is \& Is Not,'' \textit{Personality \& Social Psychology Bulletin} 19 (1993): 684--691.} In marriages\footnote{\textbf{marriage} [n] \textbf{1.} [countable, uncountable] the legal relationship between a husband \& wife; the state of being married. A \textbf{mixed marriage} is one between partners of different races or religions.; \textbf{2.} [countable, uncountable] (in some places) the legal relationship between partners of the same sex; \textbf{3.} [countable] the ceremony in which 2 people become husband \& wife, for legal partners; \textbf{4.} [countable] \textbf{marriage of A \& B} a combination of or close relationship between 2 things; \textbf{by marriage} [idiom] when somebody is related to you by marriage, they are married to somebody in your family, or you are married to somebody in their family.} \& friendships\footnote{\textbf{friendship} [n] \textbf{1.} [countable] a relationship between friends; \textbf{2.} [uncountable] the feeling or relationship that friends have; the state of being friends.}, we contribute whenever we can without keeping score\footnote{\textbf{score} [n] the number of points somebody gets for correct or positive answers in a test; [v] \textbf{1.} [transitive, intransitive] to gain marks or points in a test; \textbf{2.} [transitive] \textbf{score something\texttt{/}somebody ($+$ adv.\texttt{/}prep.)} to give something\texttt{/}somebody a particular number of points; \textbf{3.} [transitive] \textbf{score something} to succeed; to have an advantage.}.

\fbox{But in the workplace, give \& take becomes more complicated.} Professionally\footnote{\textbf{professionally} [adv] \textbf{1.} in a way that is connected with a person's job or training; \textbf{2.} in a way that shows skill \& experience; \textbf{3.} by a person who has the right skills \& qualifications.}, few of us act purely\footnote{\textbf{purely} [adv] only; completely.} like givers or takers, adopting\footnote{\textbf{adopt} [v] \textbf{1.} [transitive] \textbf{adopt something} to start to use a particular method or to show a particular attitude towards somebody\texttt{/}something; \textbf{2.} [transitive] \textbf{adopt something} to formally accept a suggestion or policy by voting; \textbf{3.} [transitive, intransitive] \textbf{adopt (somebody)} to take somebody else's child into your family \& becomes its legal parent(s); \textbf{4.} [transitive] \textbf{adopt something} to choose a new name or custom \& begin to use it as your own; to choose \& move to a country as your permanent home; \textbf{5.} [transitive] \textbf{adopt something} to use a particular manner or way of speaking.} a 3rd style instead. We become \fbox{\textit{matchers}, striving to preserve an equal balance of giving \& getting}. Matchers operate\footnote{\textbf{operate} [v] \textbf{1.} [intransitive] to work, happen or exist, especially in a particular way or place or at a particular time, \textsc{synonym}: \textbf{function}; \textbf{2.} [transitive] \textbf{operate something} to use or control a system, process or machine; \textbf{3.} [intransitive] \textbf{operate (on somebody\texttt{/}something)} to cut open somebody's body in order to remove or repair a damaged part.} on the principle of fairness\footnote{\textbf{fairness} [n] [uncountable] the quality of treating people equally or according to the law or rules.}: when they help others, they protect\footnote{\textbf{protect} [v] \textbf{1.} [transitive, intransitive] to keep somebody\texttt{/}something safe from harm or injury; \textbf{2.} [transitive, usually passive] to introduce laws that make it illegal to kill, harm or damage a particular animal, area of land, building, etc.; \textbf{3.} [transitive] to help an industry in your own country by taxing goods from other countries so that there is less competition; \textbf{4.} [transitive, intransitive] to provide somebody\texttt{/}something with insurance against fire, injury, damage, etc.} themselves by seeking reciprocity. If you're a matcher, you believe in tit for tat\footnote{\textbf{tit for tat} [n] [uncountable] a situation in which you do something bad to somebody because they have done the same to you.}, \& your relationships\footnote{\textbf{relationship} [n] \textbf{1.} [countable] the way in which 2 people, groups or countries behave towards each other or deal with each other; \textbf{2.} [countable, uncountable] the way in which 2 or more people or things are connected, \textsc{synonym}: \textbf{relation}; \textbf{3.} [countable] a loving \&\texttt{/}or sexual friendship between 2 people; \textbf{4.} [countable, uncountable] the way in which a person is related to somebody else in a family.} are governed\footnote{\textbf{govern} [v] \textbf{1.} [transitive, intransitive] \textbf{govern (something)} to control a country or its people \& be responsible for introducing new laws \& for organizing public services \& the economy; \textbf{2.} [transitive, often passive] \textbf{govern something} to control or influence how something happens or functions; to control or influence somebody's actions or behavior.} by even\footnote{\textbf{even} [adv] \textbf{1.} used to emphasize something unexpected or surprising; \textbf{2.} used when you are comparing things, to make the comparison stronger; \textbf{3.} used to introduce a more exact description of somebody\texttt{/}something; \textbf{even as} [idiom] just at the same time as somebody does something or as something else happens; \textbf{even if} [idiom] despite the possibility, fact or belief that; no matter whether; \textbf{even now\texttt{/}then} [idiom] \textbf{1.} despite what has\texttt{/}had happened; \textbf{2.} at this or that exact moment; \textbf{even so} [idiom] despite that; [a] \textbf{1.} equal in number, amount or value; shared equally, \textsc{opposite}: \textbf{uneven}; \textbf{2.} that can be divided exactly by 2, \textsc{opposite}: \textbf{odd}; \textbf{have an even chance (of doing something)} [idiom] to be equally likely to do or not do something.} exchanges of favors.

Giving, taking, \& matching are 3 fundamental styles of social interaction, but the lines between them aren't hard \& fast. You might find that you shift from 1 reciprocity style to another as you travel across different work roles \& relationships.\footnote{Alan Fiske, an anthropologist at UCLA, finds that \underline{people engage in a mix of giving, taking, \& matching} [Alan P. Fiske, \textit{Structures of Social Life: The Four Elementary Forms of Human Relations} (New York: Free Press, 1991)] in every human culture -- from North to South America, Europe to Africa, \& Australia to Asia. While living with a West African tribal group in Burkina Faso called the Mossi, Fiske found people switching between giving, taking, \& matching. When it comes to land, the Mossie are givers. If you want to move into their village, they will automatically grant you land without expecting anything in return. But in the marketplace, the Mossi are more inclined toward taking, haggling aggressively for the best prices. \& when it comes to cultivating food, the Mossi are likely to be matchers: everyone is expected to make an equal contribution, \& meals are divided into 7 shares.} It wouldn't be surprising if you act like a taker when negotiating\footnote{\textbf{negotiate} [v] \textbf{1.} [intransitive] to try to reach an agreement by formal discussion; \textbf{2.} [transitive] to arrange or agree something by formal discussion; \textbf{3.} [transitive] \textbf{negotiate something ($+$ adv.\texttt{/}prep)} to successfully get over or past a difficult part on a path or route; \textbf{4.} [transitive] \textbf{negotiate something ($+$ adv.\texttt{/}prep.)} to successfully solve a problem that is preventing you from achieving something.} your salary, a giver when mentoring someone with less experience than you, \& a matcher when sharing expertise\footnote{\textbf{expertise} [n] [uncountable] special knowledge or skill in a particular subject, activity or job.} with a colleague. But evidence shows that at work, the vast majority of people develop a primary reciprocity style, which captures how they approach most of the people most of the time. \& this primary style can play as much of a role in our success as hard work, talent, \& luck.

In fact, \fbox{the patterns of success based on reciprocity styles are remarkably clear.} If I asked you to guess who's the most likely to end up at the bottom of the success ladder, what would you say -- takers, givers, or matchers?

Professionally, all 3 reciprocity styles have their own benefits \& drawbacks. But there's 1 style that proves more costly than the other 2. Based on David Hornik's story, you might predict that givers achieve the worst results -- \& you'd be right. Research demonstrates that givers sink\footnote{\textbf{sink} [v] \textbf{1.} [intransitive] to go down below the surface or towards the bottom of a liquid or soft substance; \textbf{2.} [transitive] \textbf{sink something} to damage a boat or ship so that it goes below the surface of the sea, etc.; \textbf{3.} [intransitive] (of an object) to move slowly downwards; \textbf{4.} \textbf{sink (to something)} to decrease in amount, volume, strength, etc.; [n] \textbf{1.} a large open container that has taps to supply water \& that you use for washing dishes in; \textbf{2.} (\textit{specialist}) a body or process which acts to absorb or remove energy or a particular component from a system, \textsc{opposite}: \textbf{source}.} to the bottom of the success ladder. Across a wide range of important occupations\footnote{\textbf{occupation} [n] \textbf{1.} [countable] a job or profession; \textbf{2.} [uncountable] the act of moving into a country, town, etc. \& taking control of it using military force; the period of time during which a country, town, etc. is controlled in this way; \textbf{3.} [uncountable] the act of living in or using a building, room or piece of land; \textbf{4.} [countable] a way of spending time, especially when you are not working.}, givers are at a disadvantage: they make others better off but sacrifice their own success in the process.

In the \underline{world of engineering}\footnote{Francis J. Flynn, ``How Much Should I Give \& How Often? The Effects of Generosity \& Frequency of Favor Exchange on Social Status \& Productivity,'' \textit{Academy of Management Journal} 46 (2003): 539--553.}, \fbox{the least productive \& effective engineers are givers}. In 1 study, when $> 160$ professional engineers in California rated one another on help given \& received, the least successful engineers were those who gave more than they received. These givers had the worst objective scores in their firm for the number of tasks, technical reports, \& drawings completed -- not to mention errors made, deadlines missed, \& money wasted. Going out of their way to help others prevented them from getting their own work done.

The same pattern emerges in medical school. In a study of $> 600$ \underline{medical students in Belgium}\footnote{Filip Lievens, Deniz S. Ones, \& Stephan Dilchert, ``Personality Scale Validities Increase Throughout Medical School,'' \textit{Journal of Applied Psychology} 94 (2009): 1514--1535.}, the students with the lowest grades had unusually high score on giver statements like ``I love to help others'' \& ``I anticipate\footnote{\textbf{anticipate} [v] \textbf{1.} to expect or predict something; \textbf{2.} to see what might happen in the future \& take action to prepare for it; \textbf{3.} \textbf{anticipate something} to think with pleasure \& excitement about something that is going to happen; \textbf{4.} \textbf{anticipate something} to come before \& influence something else that is similar; to be a sign of what is going to happen.} the needs of others.'' The givers went out of their way to help their peers study, sharing what they already knew at the expense of filling gaps in their own knowledge, \& it gave their peers a leg up\footnote{\textbf{leg-up} [idiom] (\textit{informal}) \textbf{1.} (\textit{especially British English}) an act of helping somebody to get on a horse, over a wall, etc. by allowing them to put their foot in your hands \& lifting them up; \textbf{2.} (\textit{especially British English}) an act of helping somebody to improve their situation; \textbf{have\texttt{/}get a leg-up on somebody} [idiom] (\textit{North American English, informal}) to have\texttt{/}get an advantage over somebody.} at test time. Salespeople\footnote{\textbf{salesperson} [n] (plural \textbf{salespeople}) a person whose job is to sell goods.} are no different. In a study I led of \underline{salespeople in North Carolina}\footnote{Adam M. Grant \& Dane Barnes, ``Predicting Sales Revenue'' (working paper, 2011).}, compared with takers \& matchers, givers brought in 2.5 times less annual sales revenue. They were so concerned about what was best for their customers that they weren't willing to sell aggressively.

Across occupations, it appears that givers are just too caring\footnote{\textbf{caring} [a] [usually before noun] \textbf{1.} (\textit{especially British English}) connected with work that involves looking after or helping other people; \textbf{2.} kind, helpful \& showing that you care about other people.}, too trusting\footnote{\textbf{trusting} [a] tending to believe that other people are good, honest, etc.; showing this.}, \& too willing\footnote{\textbf{willing} [a] \textbf{1.} happy or read to do something, without needing to be persuaded; not objecting to something, \textsc{opposite}: \textbf{unwilling}; \textbf{2.} done or given freely.} to sacrifice their own interests for the benefit of others. There's even evidence that compared with takers, on average, \underline{givers earn 14\% less money}\footnote{Timothy A. Judge, Beth A. Livingston, \& Charlice Hurst, ``Do Nice Guys--and Gals--Really Finish Last? The Joint Effects of Sex \& Agreeableness on Income,” \textit{Journal of Personality \& Social Psychology} 102 (2012): 390--407.}, have \underline{twice the risk of becoming victims of crimes}\footnote{Robert J. Homant, ``Risky Altruism as a Predictor of Criminal Victimization,'' \textit{Criminal Justice \& Behavior} 37 (2010): 1195--1216.}, \& are \underline{judged as 22\% less powerful \& dominant\footnote{\textbf{dominant} [a] \textbf{1.} stronger, \& having more power \& influence than other things or people, \textsc{synonym}: \textbf{predominant}; \textbf{2.} more common, easier to notice, or more important than other things, \textsc{synonym}: \textbf{predominant}; \textbf{3.} (\textit{ecology}) (of a type of plant or animal) more common in a place than other types of plant or animal; \textbf{4.} (\textit{biology}) connected with a characteristic that appears in an individual even if it only has 1 gene for this characteristic, passed on by only 1 of its parents.}}\footnote{Nir Halevy, Eileen Y. Chou, Taya R. Cohen, and Robert W. Livingston, ``Status Conferral in Intergroup Social Dilemmas: Behavioral Antecedents and Consequences of Prestige and Dominance,'' \textit{Journal of Personality and Social Psychology} 102 (2012): 351--366.}.

So if givers are most likely to land at the bottom of the success ladder, who's at the top -- takers or matchers?

Neither. When I took another look at the data, I discovered a surprising pattern: \textit{It's the givers again}.

As we've seen, the engineers with the lowest productivity are mostly givers. But when we look at the engineers with the highest productivity, the evidence shows that they're givers too. The California engineers with the best objective scores for quantity \& quality of results are those who consistently\footnote{\textbf{consistently} [adv] always in the same way; following the same pattern or standard.} give more to their colleagues than they get. The worst performers \& the best performers are givers; takers \& matchers are more likely to land in the middle.

This pattern holds up across the board. The Belgian medical students with the lowest grades have unusually high giver scores, but so do the students with the \textit{highest} grades. Over the course of medical school, being a giver accounts for 11\% higher grades. Even in sales, I found that the least productive salespeople had 25\% higher giver scores than average performers -- but so did the most productive salespeople. The top performers were givers, \& they averaged 50\% more annual revenue than the takers \& matchers. Givers dominate the bottom \textit{\&} the top of the success ladder. Across occupations, if you examine the link between reciprocity styles \& success, the givers are more likely to become champs -- not only chumps.

Guess which one David Hornik turns out to be?

After Danny Shader signed with the other investor, he had a gnawing\footnote{\textbf{gnawing} [a] [only before noun] making you feel worried over a period of time.} feeling. ``We just closed a big round. We should be celebrating. \textit{Why am I not happier?} I was excited about my investor, who's exceptionally\footnote{\textbf{exceptionally} [adv] \textbf{1.} used before an adjective or adverb to emphasize how strong or unusual the quality is; \textbf{2.} only in unusual circumstances.} bright \& talented, but I was missing the opportunity to work with Hornik.'' Shader wanted to find a way to engage Hornik, but there was a catch. To involve him, Shader \& his lead investor would have to sell more of the company, diluting\footnote{\textbf{dilute} [v] \textbf{1.} [often passive] to make a liquid weaker by adding water or another liquid; \textbf{2.} \textbf{dilute something} to make something weaker in force or value by changing it or adding something, \textsc{synonym}: \textbf{diminish}; \textbf{3.} (\textit{business}) to reduce the value of the shares that a particular shareholder owns by issuing more shares without increasing the company's assets; [a] \textbf{1.} (also \textbf{diluted}) (of a liquid or solution) made weaker by adding water or another liquid; \textbf{2.} (of light, color or radiation) weak.} their ownership.

Shader decided it was worth the cost to him personally. Before the financing closed, he invited Hornik to invest in his company. Hornik accepted the offer \& made an investment, earning some ownership of the company. He began coming to board meetings, \& Shader was impressed with Hornik's ability to push him to consider new directions. ``I got to see the other side of him,'' Shader says. ``It had just been overshadowed\footnote{\textbf{overshadow} [v] [often passive] \textbf{1.} \textbf{overshadow somebody\texttt{/}something} to make somebody\texttt{/}something less important, or successful; \textbf{2.} \textbf{overshadow something} to make an event less pleasant than it should be, \textsc{synonym}: \textbf{cloud}; \textbf{3.} \textbf{overshadow something} to throw a shadow over something.} by how affable\footnote{\textbf{affable} [a] pleasant, friendly \& easy to talk to, \textsc{synonym}: \textbf{genial}.} he is.'' Thanks in part to Hornik's advice, Shader's start-up has taken off\footnote{\textbf{take off} [phrasal verb] \textbf{1.} to become successful or popular very quickly or suddenly; \textbf{2.} (of an aircraft, etc.) to leave the ground \& begin to fly, \textsc{opposite}: \textbf{land}.}. It's called PayNearMe, \& it enables Americans who don't have a bank account or a credit card to make online purchases with a barcode\footnote{\textbf{barcode} [n] a pattenr of thick \& thin lines that is printed on things you buy. It contains information that a computer can read.} or a card, \& then pay cash for them at participating\footnote{\textbf{participate} [v] [intransitive] \textbf{participate (in something)} to take part in or become involved in an activity.} establishments\footnote{\textbf{establishment} [n] \textbf{1.} [uncountable] \textbf{establishment (of something)} the act of starting or creating something that is meant to last for a long time; \textbf{2.} (usually \textbf{the Establishment}) [singular $+$ singular or plural verb] (\textit{often disapproving}) the people in a society or a profession who have influence \& power \& who usually do not support change; \textbf{3.} [countable] (\textit{formal}) a business organization or public institution.}. Shader landed major partnerships with 7-Eleven \& Greyhound to provide these services, \& in the 1st year \& a half since launching, PayNearMe has been growing at $> 30$\% per month. As an investor, Hornik has a small share in this growth.

Hornik has also added Shader to his list of references, which is probably even more valuable than the deal itself. When entrepreneurs call to ask about Hornik, Shader tells them, ``You may be thinking he's just a nice guy, but he's a lot more than that. He's phenomenal\footnote{\textbf{phenomenal} [a] \textbf{1.} very great or impressive, \textsc{synonym}: \textbf{extraordinary}; \textbf{2.} that can be felt through the senses or through immediate experience.}: super-hardworking \& very courageous\footnote{\textbf{courageous} [a] showing courage.}. He can be both challenging\footnote{\textbf{challenging} [a] difficult in an interesting way that tests your ability.} \& supportive\footnote{\textbf{supportive} [a] giving help, encouragement or sympathy to somebody.} at the same time. \& he's incredibly\footnote{\textbf{incredibly} [adv] \textbf{1.} extremely, \textsc{synonym}: \textbf{unbelievably}; \textbf{2.} in a way that is very difficult to believe.} responsive\footnote{\textbf{responsive} [a] \textbf{1.} [not usually before noun] \textbf{responsive (to somebody\texttt{/}something)} reacting quickly \& in a positive way, \textsc{opposite}: \textbf{unresponsive}; \textbf{2.} reacting with interest or enthusiasm, \textsc{synonym}: \textbf{receptive}, \textsc{opposite}: \textbf{unresponsive}.}, which is 1 of the best characteristics you can have in an investor. He'll get back to you any hour -- day or night -- quickly, on anything that matters.''

The payoff\footnote{\textbf{payoff} [n] (\textit{informal}) \textbf{1.} a payment of money to someone so that they will not cause you any trouble, or to make them keep a secret, \textsc{synonym}: \textbf{bribe}; \textbf{2.} a payment of money to someone to persuade them to leave their job; \textbf{3.} an advantage or a reward for something you have done.} for Hornik was not limited to this single deal on PayNearMe. After seeing Hornik in action, Shader came to admire Hornik's commitment\footnote{\textbf{commitment} [n] \textbf{1.} [singular, uncountable] a strong belief in a cause or activity \& a promise to support it; \textbf{2.} [countable, uncountable] a promise to do something or to behave in a particular way; \textbf{3.} [uncountable] the willingness to work hard \& give your energy \& time to a job or an activity; \textbf{4.} [countable] (used in compounds) a thing that you have promised or agreed to do, or that you have to do; \textbf{5.} [countable, uncountable] agreeing to use money, time or people in order to achieve something.} to acting in the best interests of entrepreneurs, \& he began to set Hornik up with other investment opportunities. In 1 case, after meeting the CEO of a company called Rocket Lawyer, Shader recommended Hornik as an investor. Although the CEO already had a term sheet from another investor, Hornik ended up winning the investment.

Although he recognizes the downsides\footnote{\textbf{downside} [n] [singular] \textbf{downside (of something\texttt{/}doing something)} the disadvantages or less positive aspects of something.}, David Hornik believes that operating like a giver has been a driving force behind his success in venture capital. Hornik estimates that when most venture capitalists offer term sheets to entrepreneurs, they have a signing rate near 50\%: ``If you get half of the deals you offer, you're doing pretty well.'' Yet in 11 years as a venture capitalist, Hornik has offered 28 term sheets to entrepreneurs, \& 25 have accepted. Shader is 1 of just 3 people who have ever turned down an investment from Hornik. The other 89\% of the time entrepreneurs have taken Hornik's money. Thanks to his funding \& expert advice, these entrepreneurs have gone on to build a number of successful start-ups -- one was valued at more than \$3 billion on its 1st day of trading in 2012, \& others have been acquired by Google, Oracle, Ticketmaster, \& Monster.

Hornik's hard work \& talent, not to mention his luck at being on the right sideline\footnote{\textbf{sideline} [n] \textbf{1.} [countable] an activity that you do as well as your main job in order to earn extra money; \textbf{2.} \textbf{sidelines} [plural] the lines along the 2 long sides of a sports field, tennis court, etc. that mark the outer edges; the area just outside these; \textbf{on\texttt{/}from the sidelines} [idiom] watching something but not actually involved in it; [v] \textbf{1.} to prevent somebody from playing in a team, especially because of an injury; \textbf{2.} to prevent somebody from having an important part in something that other people are doing.} at his daughter's soccer game, played a big part in lining up the deal with Danny Shader. But it was his reciprocity style that ended up winning the day for him. Even better, he wasn't the only winner. Shader won too, as did the companies to which Shader later recommended Hornik. By operating as a giver, Hornik created value for himself while maximizing opportunities for value to flow outward for the benefit of others.\\

In this book, I want to persuade you that we underestimate the success of givers like David Hornik. Although we often stereotype\footnote{\textbf{stereotype} [n] \textbf{stereotype (of somebody\texttt{/}something)} a fixed idea or image that many people have of a particular type of person or thing, but which is often not really true.} givers as chumps \& doormats\footnote{\textbf{doormat} [n] \textbf{1.} a small piece of strong material near a door that people can clean their shoes on; \textbf{2.} (\textit{informal}) a person who allows other people to treat them badly but usually does not complain.}, they turn out to be surprisingly successful. To figure out why givers dominate the top of the success ladder, we'll examine startling\footnote{\textbf{startling} [a] \textbf{1.} extremely unusual \& surprising; \textbf{2.} (of a color) extremely bright.} studies \& stories that illuminate\footnote{\textbf{illuminate} [v] \textbf{1.} \textbf{illuminate something} to make something clearer or easier to understand, \textsc{synonym}: \textbf{clarify}; \textbf{2.} \textbf{illuminate something} to shine light on something.} how giving can be more powerful -- \& less dangerous -- than most people believe. Along the way, I'll introduce you to successful givers from many different walks of life, including consultants\footnote{\textbf{consultant} [n] \textbf{1.} a person who has a lot of knowledge about a particular subject \& is employed to give advice about it to other people; \textbf{2.} (\textit{British English}) a hospital doctor of the highest rank who is a specialist in a particular area of medicine.}, lawyers, doctors, engineers, salespeople, writers, entrepreneurs, accountants\footnote{\textbf{accountant} [n] a person whose job is to keep or check financial accounts.}, teachers, financial advisers, \& sports executives\footnote{\textbf{executive} [n] \textbf{1.} [countable] a person who has an important job as a manager of a company or an organization; \textbf{2.} (\textbf{the executive}) [singular $+$ singular or plural verb] the part of a government responsible for putting laws into effect; [a] [only before noun] \textbf{1.} connected with managing a business or an organization, \& with making plans \& decisions; \textbf{2.} having the power to put important laws \& decisions into effect.}. These givers reverse the popular plan of succeeding 1st \& giving back later, raising the possibility that those who give 1st are often best positioned for success later.

But we can't forget about those engineers \& salespeople at the bottom of the ladder. Some givers do become pushovers\footnote{\textbf{pushover} [n] (\textit{informal}) \textbf{1.} a thing that is easy to do or win; \textbf{2.} a person who is easy to persuade or influence.} \& doormats, \& I want to explore what separates the champs from the chumps. The answer is less about raw talent or aptitude\footnote{\textbf{aptitude} [n] [uncountable, countable] natural ability or skill at doing something, \textsc{synonym}: \textbf{talent}.}, \& more about the strategies givers use \& the choices they make. To explain how givers avoid the bottom of the success ladder, I'm going to debunk\footnote{\textbf{debunk} [v] \textbf{debunk something} to show that an idea, a belief, etc. is false; to show that something is not as good as people think it is.} 2 common myths\footnote{\textbf{myth} [n] [countable, uncountable] \textbf{1.} a story from ancient times, especially one that was told to explain natural events or to describe the early history of a people; this type of story, \textsc{synonym}: \textbf{legend}; \textbf{2.} something that many people believe but that does not exist or is false, \textsc{synonym}: \textbf{fallacy}.} about givers by showing you that they're not necessarily nice, \& they're not necessarily altruistic\footnote{\textbf{altruistic} [a] (\textit{formal}) caring about the needs \& happiness of other people \& being willing to do things to help them, even if it brings no advantage to yourself.}. We all have goals for our own individual achievements, \& it turns out that successful givers are every bit as ambitious\footnote{\textbf{ambitious} [a] \textbf{1.} determined to be successful, rich or powerful; \textbf{2.} needing a lot of effort, money or time to succeed.} as takers \& matchers. They simply have a different way of pursuing their goals.

This brings us to my 3rd aim, which is to reveal what's unique about the success of givers. Let me be clear that givers, takers, \& matchers all can -- \& do -- achieve success. But there's something distinctive that happens when givers succeed: it spreads \& cascades. When takers win, there's usually someone else who loses. Research shows that people tend to \underline{envy successful takers}\footnote{Eugene Kim and Theresa M. Glomb, ``Get Smarty Pants: Cognitive Ability, Personality, \& Victimization,'' \textit{Journal of Applied Psychology} 95 (2010): 889--901.} \& look for ways to knock them down a notch\footnote{\textbf{notch} [n] \textbf{1.} a level on a scale, often marking quality or achievement; \textbf{2.} a V-shape or a circle cut in an edge or a surface, sometimes used to keep a record of something; \textbf{3.} each of a series of holes, e.g. in a belt; [v] \textbf{1.} \textbf{notch something (up)} (\textit{informal}) to achieve something such as a win or a high score; \textbf{2.} \textbf{notch something} to make a small cut in the shape of a V in an edge or a surface.}. In contrast, when givers like David Hornik win, people are rooting for them \& supporting them, rather than gunning\footnote{\textbf{be gunning for} [phrasal verb] \textbf{be gunning for somebody} (\textit{informal}) to be looking for an opportunity to blame or attack somebody; \textbf{be gunning for something} to be competing for or trying hard to get something.} for them. Givers succeed in a way that creates a ripple\footnote{\textbf{ripple} [n] \textbf{1.} a small wave on the surface of a liquid, especially water in a lake, etc.; \textbf{2.} a thing that looks or moves like a small wave; \textbf{3.} [usually singular] \textbf{ripple of something} a sound that gradually becomes louder \& then quieter again; \textbf{4.} [usually singular] \textbf{ripple of something} a feeling that gradually spreads through a person or group of people; [v] \textbf{1.} [intransitive, transitive] to move or to take something move in very small waves; \textbf{2.} [intransitive] \textbf{$+$ adv.\texttt{/}prep.} (of a feeling, etc.) to spread through a person or a group of people like a wave.} effect\footnote{\textbf{ripple effect} [n] a situation in which an event or action has an effect on something, which then has an effect on something else.}, enhancing the success of people around them. You'll see that the difference lies in how giver success creates value, instead of just claiming it. As the venture capitalist Randy Komisar remarks, ``\underline{It's easier to win}\footnote{Personal interview with Randy Komisar (Mar 30, 2012).} if everybody wants you to win. If you don't make enemies out there, it's easier to succeed.''

But in some arenas\footnote{\textbf{arena} [n] \textbf{1.} an area of activity, especially one where there is a lot of discussion or argument; \textbf{2.} a place with a flat open area in the middle \& seats around it where people can watch sports \& entertainment.}, it seems that the costs of giving clearly outweigh the benefits. In politics, e.g., Mark Twain's opening quote suggests that diplomacy\footnote{\textbf{diplomacy} [n] [uncountable] the activity of managing relations between different countries; the skill in doing this.} involves taking 10 times as much as giving. ``\underline{Politics}\footnote{Bill Clinton, \textit{Giving: How Each of Us Can Change the World} (New York: Random House, 2007), ix.}'' writes former president Bill Clinton, ``is a `getting' business. You have to get support, contributions, \& votes, over \& over again.'' Takers should have an edge in lobbying\footnote{\textbf{lobby} [v] [transitive, intransitive] to try to influence a politician or the government \&, e.g., persuade them to support or oppose a change in the law; [n] (plural \textbf{lobbies}) [countable $+$ singular or plural verb] a group of people who try to influence politicians on a particular issue.} \& outmaneuvering\footnote{\textbf{outmanoeuvre} [v] (\textit{US English} \textbf{outmaneuver}) \textbf{outmanoeuvre somebody\texttt{/}something} to do better than an opponent by acting in a way that is cleverer or shows more skill.} their opponents in competitive elections\footnote{\textbf{election} [n] \textbf{1.} [countable, uncountable] the process of choosing a person or a group of people for a position, especially a political position, by voting; \textbf{2.} [uncountable] the fact of having been chosen by election.}, \& matchers may be well suited to the constant trading of favors that politics demands. What happens to givers in the world of politics?

\underline{Consider the political struggles of a hick\footnote{\textbf{hick} [a] (\textit{informal, especially North American English}) connected with people from the country who are considered to be stupid \& to have little experience of life; [n] a person from the country who is considered to be stupid \& to have little experience of life.} who went by the name Sampson}\footnote{My account of Abraham Lincoln's rise is based primarily on the riveting book by Doris Kearns Goodwin, \textit{Team of Rivals}: \textit{The Political Genius of Abraham Lincoln} (New York: Simon \& Schuster, 2006).}. He said his goal was to be the ``Clinton of Illinois,'' \& he set his sights on winning a seat in the Senate\footnote{\textbf{Senate} [n] (usually \textbf{the Senate}) [singular] \textbf{1.} 1 of the 2 groups of elected politicians who make laws in some countries, e.g. in the US, Australia, Canada \& France. The Senate is smaller than the other group but higher in rank. Many state parliaments in the US also have a Senate; \textbf{2.} (in ancient Rome) the most important council of the government; the building where the council met.}. Sampson was an unlikely candidate for political office, having spent his early years working on a farm. But Sampson had great ambition; he made his 1st run for a seat in the state legislature\footnote{\textbf{legislature} [n] a group of people who have the power to make \& change laws.} when he was just 23 years old. There were 15 candidates, \& only the top 4 won seats. Sampson made a lackluster\footnote{\textbf{lacklustre} [a] (\textit{US English} \textbf{lackluster}) \textbf{1.} not interesting or exciting; not bright; \textbf{2.} (of the hair or eyes) not bright or shining; dull.} showing, finishing 8th.

After losing that race, Sampson turned his eye to business, taking out a loan to start a small shop with a friend. The business failed, \& Sampson was unable to repay the loan, so his possessions were seized\footnote{\textbf{seize} [v] \textbf{1.} \textbf{seize something} to be quick to take advantage of something such as a chance or an opportunity; \textbf{2.} to take control of a place or situation, often suddenly \& violently; \textbf{3.} \textbf{seize something} (of the police, etc.) to take possession of something by legal right; \textbf{4.} \textbf{seize somebody} to arrest or capture somebody; \textbf{5.} \textbf{seize somebody\texttt{/}something (from somebody)} to take hold of somebody\texttt{/}something suddenly \& using force; \textbf{seize on\texttt{/}upon something} [phrasal verb] to suddenly show a lot of interest in something, especially because you can use it to your advantage.} by local authorities. Shortly thereafter\footnote{\textbf{thereafter} [adv] (\textit{formal}) after the time or event mentioned.}, his business partner died without assets\footnote{\textbf{asset} [n] \textbf{1.} a person or thing that is valuable or useful to somebody\texttt{/}something; \textbf{2.} [usually plural] a thing of value, especially property, that a person or company owns, which can be used or sold to pay debts.}, \& Sampson took on the debt. Sampson jokingly\footnote{\textbf{jokingly} [adv] in a way that is intended to be funny \& not serious.} called his liability\footnote{\textbf{liability} [n] (plural \textbf{liabilities}) \textbf{1.} [uncountable, singular] the state of being legally responsible for something; \textbf{2.} [countable, usually plural] the amount of money that a person or company owes; \textbf{3.} [countable, usually singular] a person or thing that a company, person, etc. has that can cause a lot of problems.} ``the national debt'': he owned 15 times his annual income. It would take him years, but he eventually paid back every cent.

After his business failed, Sampson made a 2nd run for the state legislature. Although he was only 25 years old, he finished 2nd, landing a seat. For his 1st legislative\footnote{\textbf{legislative} [a] [only before noun] connected with the act of making \& passing laws.} session, he had to borrow the money to buy his 1st suit. For the next 8 years, Sampson served in the state legislature, earning a law degree along the way. Eventually, at age 45, he was ready to pursue influence on the national stage. He made a bid\footnote{\textbf{bid} [n] \textbf{1.} \textbf{bid (for something)} an offer by a person or a company to pay a particular amount of money for something; \textbf{2.} \textbf{bid (for something)} (\textit{North American English also}) \textbf{bid (on something)} an offer to do work or provide a service for a particular price, in competition with other companies, \textsc{synonym}: \textbf{tender}; \textbf{3.} an effort to do something or to obtain something; [v] to offer to do work or provide a service for a particular price, in competition with other companies, etc., \textsc{synonym}: \textbf{tender}.} for the Senate.

p. 17/260

'' -- \cite[Chap. 1]{Grant2013}

\section{The Peacock \& the Panda: \textit{How Givers, Takers, \& Matchers Build Networks}}


\section{The Ripple Effect: \textit{Collaboration \& the Dynamics of Giving \& Taking Credit}}


\section{Finding the Diamond in the Rough: \textit{The Fact \& Fiction of Recognizing Potential}}


\section{The Power of Powerless Communication: \textit{How to Be Modest \& Influence People}}


\section{The Art of Motivation Maintenance: \textit{Why Some Givers Burn Out but Others Are On Fire}}


\section{Chump Change: \textit{Overcoming the Doormat Effect}}


\section{The Scrooge Shift: \textit{Why a Soccer Team, a Fingerprint, \& a Name Can Tilt Us in the Other Direction}}


\section{Out of the Shadows}

\section{Actions for Impact}

%------------------------------------------------------------------------------%

\selectlanguage{english}
\chapter{DK. How Psychology Works}

``Lying at the intersection of a number of disciplines\footnote{\textbf{discipline} [n] \textbf{1.} [countable] a subject of study, especially in a university; \textbf{2.} [uncountable] the practice of training people to obey rules \& behave well; \textbf{3.} [uncountable] the practice of training your mind or of controlling your behavior; [v] \textbf{1.} \textbf{discipline somebody (for something)} to punish somebody for something they have done; \textbf{2.} \textbf{discipline somebody} to train somebody, especially a child, to obey particular rules \& control the way they behave.}, including biology, sociology\footnote{\textbf{sociology} [n] [uncountable] the scientific study of the nature \& development of society \& social behavior.}, medicine\footnote{\textbf{medicine} [n] \textbf{1.} [uncountable] the study \& treatment of diseases \& injuries. In technical language, the term \textbf{medicine} is often considered not to include surgery ($=$ treatment that usually involves cutting open a person's body).; \textbf{2.} [uncountable, countable] \textbf{medicine (for something)} a substance, especially a liquid that you drink or swallow in order to cure an illness.}, anthropology\footnote{\textbf{anthropology} [n] [uncountable] \textbf{1.} (also \textbf{cultural anthropology} or \textbf{social anthropology}) the study of the human race by comparing human societies \& cultures \& how they have developed; \textbf{2.} (also \textbf{physical anthropology}) the study of the human race by examining how humans behave \& how their bodies work \& have changed during their development.}, \& artificial intelligence, psychology\footnote{\textbf{psychology} [n] \textbf{1.} [uncountable] the scientific study of the mind \& how it influences behavior; \textbf{2.} [uncountable, singular] the way that people think \& therefore behave in a particular situation.} has always fascinated\footnote{\textbf{fascinate} [v] [transitive, intransitive] to attract or interest somebody very much.}\,\footnote{\textbf{fascinated} [a] very interested.} people. \textit{How do psychologists interpret\footnote{\textbf{interpret} [v] \textbf{1.} to explain the meaning of something; \textbf{2.} to decide that something such as an action or situation has a particular meaning \& to understand it in this way.} human behavior\footnote{\textbf{behaviour} [n] (US \textbf{behavior}) \textbf{1.} [uncountable, countable] the way that somebody\texttt{/}something functions or reacts in a particular situation; \textbf{2.} [uncountable] the way that somebody behaves, especially towards other people.} to understand why we do what do? Why are there so many branches\footnote{\textbf{branch} [n] \textbf{1.} a part of a government or other large organization that deals with 1 particular aspect of its work, \textsc{synonym}: \textbf{department}; \textbf{2.} a local office or shop belonging to a company or organization; \textbf{3.} \textbf{branch (of something)} a division of an area of knowledge, or a group of languages; \textbf{4.} \textbf{branch (of something)} a group of members of a family who all have the same ancestors; \textbf{5.} \textbf{branch (of something)} a part of a tree that grows out from the trunk \& on which leaves, flowers \& fruit grow; \textbf{6.} \textbf{branch (of something)} a smaller or less important part of something that leads away from the main part; \textbf{root \& branch} [idiom] thorough \& complete; [v] [intransitive] to divide into 2 or more parts, especially smaller or less important parts.} \& approaches\footnote{\textbf{approach} [n] \textbf{1.} [countable] a way of doing or thinking about something such as a problem or task; \textbf{2.} [singular] movement nearer to somebody\texttt{/}something in distance or time; \textbf{3.} [countable] \textbf{approach (to somebody\texttt{/}something)} the act of speaking to somebody about something, especially when making an offer or a request; \textbf{4.} [countable] a path, sea passage, etc. that leads to a particular place; \textbf{5.} [singular] \textbf{approach to something} a thing that is like something else that is mentioned; [v] \textbf{1.} [transitive] to start dealing with a problem or task or considering a topic or situation in a particular way; \textbf{2.} [transitive] \textbf{approach something} to come close to something in quantity or quality; \textbf{3.} [intransitive, transitive] to move near to somebody\texttt{/}something in distance or time; \textbf{4.} [transitive] to speak to somebody about something, especially to offer to do something or to ask them for something.}, \& how do they work in a practical\footnote{\textbf{practical} [a] \textbf{1.} connected with real situations rather than with ideas or theories; \textbf{2.} (of an idea, a method or a course of action) right or sensible; possible \& likely to be successful, \textsc{synonym}: \textbf{feasible, workable}, \textsc{opposite}: \textbf{impractical}; \textbf{3.} (of things) useful or suitable for a particular purpose, \textsc{opposite}: \textbf{impractical}; \textbf{4.} (of a person) sensible \& realistic in the way they approach a problem or situation; \textbf{for (all) practical purposes} [idiom] used to say that something is so nearly true that it can be considered to be so; [n] (\textit{British English, informal}) a lesson or an exam in science or technology in which students have to do or make things, not just read or write about them.} sense\footnote{\textbf{sense} [n] \textbf{1.} [countable] the meaning that a word or phrase has; a way of understanding something; \textbf{2.} [countable] \textbf{sense of something} a feeling about something important; \textbf{3.} [singular] an understanding about something; an ability to judge something; \textbf{4.} [countable] 1 of the 5 powers (sight, hearing, smell, taste \& touch) that your body uses to get information about the world around you; \textbf{make sense} [idiom] \textbf{1.} to have a meaning that can be understood; \textbf{2.} to be a sensible thing to do; \textbf{make sense of something} [idiom] to understand something that is difficult or has no clear meaning; [v] (not used in the progressive senses) \textbf{1.} to become aware of something even though you cannot clearly see it, hear it, etc.; \textbf{2.} \textbf{sense something} to become aware of something by seeing it, hearing it, etc.; \textbf{3.} \textbf{sense something} (of a device) to discover \& record or react to something.} in our day-to-day\footnote{\textbf{day-to-day} [a] [only before noun] involving the usual events or tasks of each day; happening every day in a regular way.} lives? Is psychology an art or a science, or a fusion\footnote{\textbf{fusion} [n] \textbf{1.} [uncountable, singular] \textbf{fusion (of something)} the process or result of joining 2 or more things together to form one; \textbf{2.} (also \textbf{nuclear fusion}) [uncountable] (\textit{physics}) the act or process of combining the nuclei ($=$ central parts) of atoms to form a heavier nucleus, with energy being released; \textbf{3.} [uncountable] \textbf{fusion (of something)} a mixture of different styles or ideas.} of both?}

While theories come \& go out of fashion\footnote{\textbf{fashion} [n] \textbf{1.} [uncountable] the business of making or selling clothes in new \& different styles; \textbf{2.} [countable] a popular style of clothes, hair, etc. at a particular time or place; \textbf{3.} [uncountable] the state of being popular at a particular time or place; \textbf{4.} [countable] a popular way of behaving, doing an activity, etc.; \textbf{after a fashion} [idiom] to some extent, but not very well; \textbf{in (a) $\ldots$ fashion} [idiom] in a particular way; [v] \textbf{1.} \textbf{fashion something} to create or invent something; \textbf{2.} to make or shape something, especially with your hands; to use a particular material to make or shape an object.} -- \& new studies, experiments\footnote{\textbf{experiment} [n] \textbf{1.} [countable] a scientific test that is done in order to study what happens \& to gain new knowledge; \textbf{2.} [countable] a new activity, idea or method that you try out to see what happens or what effect it has; \textbf{3.} [uncountable] the process of testing something to study what happens or to see what effect it has; [v] \textbf{1.} [intransitive] to try or test new ideas or methods to find out what effect they have; \textbf{2.} [intransitive] to do a scientific experiment or experiments.}, \& research are conducted\footnote{\textbf{conduct} [v] \textbf{1.} \textbf{conduct something} to organize \&\texttt{/}or do a particular activity; \textbf{2.} \textbf{conduct something} (of a substance) to allow heat or electricity to pass along or through it; \textbf{3.} \textbf{conduct yourself $+$ adv.\texttt{/}prep.} (\textit{formal}) to behave in a particular way; [n] [uncountable] (\textit{formal}) \textbf{1.} a person's behavior; \textbf{2.} \textbf{conduct of something} the way in which a business or an activity is organized \& managed.} all the time -- the essence\footnote{\textbf{essence} [n] [uncountable] \textbf{essence (of something)} the most important quality or feature of something, that makes it what it is; \textbf{in essence} [idiom] in the most important \& basic ways, without considering things that are less important; \textbf{of the essence} [idiom] necessary \& very important.} of psychology is to explain\footnote{\textbf{explain} [v] \textbf{1.} [transitive, intransitive] to tell somebody about something in a way that makes it easy to understand; \textbf{2.} [intransitive, transitive] to give a reason for something; to be a reason for something; \textbf{explain something away} [phrasal verb] to give reasons why something is not important or is not your fault.} the behavior of individuals\footnote{\textbf{individual} [n] \textbf{1.} a person considered separately rather than as part of a group; \textbf{2.} a single member of a group or class; \textbf{3.} a person who is very different from others \& has lots of new \& interesting ideas; [a] \textbf{1.} [only before noun] considered separately rather than as part of a group; \textbf{2.} [only before noun] of or for a particular person; \textbf{3.} [only before noun] designed for use by 1 person; \textbf{4.} characteristic of a particular person or thing; \textbf{5.} (\textit{usually approving}) having an unusual character, \textsc{synonym}: \textbf{distinctive, original}.} based on the workings\footnote{\textbf{working} [a] [only before noun] \textbf{1.} connected with your job \& the time you spend doing it; \textbf{2.} having a job for which you are paid; \textbf{3.} (used especially when talking about the past) having a job that involves hard physical work rather than office work, studying, etc.; \textbf{4.} functioning or able to function; \textbf{5.} being used in order to do work; \textbf{6.} used as a basis for work, discussion, etc. but likely to be changed or improved in the future; \textbf{7.} if you have a working knowledge of something, you can use it at a basic level; \textbf{in working order} (especially of machines) [idiom] working well, not broken; [n] [usually plural] \textbf{working (of something)} the way in which a machine, a system, an organization, etc. works.} of the mind\footnote{\textbf{mind} [n] \textbf{1.} [countable, uncountable] the part of a person that makes them able to be aware of things, to think \& to feel; \textbf{2.} [countable] the ability to think \& reason; somebody's intelligence, \textsc{synonym}: \textbf{intellect}; [v] \textbf{1.} [transitive, no passive, intransitive] (used especially in questions or with negatives) to be upset or worried by something; \textbf{2.} (\textbf{(not) mind doing something}) [transitive] to be (not) willing to do something.}. In these often turbulent\footnote{\textbf{turbulent} [a] [usually before noun] \textbf{1.} (of air or water) changing direction suddenly \& violently; \textbf{2.} in which there is a lot of sudden change, confusion, disagreement \& sometimes violence.} \& uncertain\footnote{\textbf{uncertain} [a] \textbf{1.} [not before noun] feeling doubt about something; not sure, \textsc{opposite}: \textbf{certain}; \textbf{2.} likely to change, especially in a negative or unpleasant way; \textbf{3.} not definite or decided; not known exactly, \textsc{synonym}: \textbf{unclear}; \textbf{4.} not confident; \textbf{in no uncertain terms} [idiom] clearly \& strongly.} times, people are increasingly looking to psychology \& psychologists to help them make sense of why the powerful\footnote{\textbf{powerful} [a] \textbf{1.} (of people, organizations or groups) able to control \& influence people \& events, \textsc{synonym}: \textbf{influential}; \textbf{2.} having great power or force; very effective; \textbf{3.} having a strong effect on people's feelings or thoughts.} \& influential\footnote{\textbf{influential} [a] having a lot of influence on the way that somebody\texttt{/}something behaves or develops, or on the way that somebody thinks.} behave\footnote{\textbf{behave} [v] \textbf{1.} [intransitive] to function or react in a particular way; \textbf{2.} [intransitive] \textbf{$+$ adv.\texttt{/}prep. (towards somebody)} to act in a particular way, especially towards other people, \textsc{synonym}: \textbf{act}; \textbf{3.} (\textbf{-behaved}) (in adjectives) behaving in the way mentioned.} the way that they do, \& the resulting impact\footnote{\textbf{impact} [n] [countable, usually singular, uncountable] \textbf{1.} the powerful effect that something has on somebody\texttt{/}something; \textbf{2.} the act of 1 object hitting another; the force with which this happens; [v] [transitive, intransitive] to have an effect on something.} that might have on us. But psychology also has huge\footnote{\textbf{huge} [a] extremely large in size or amount; extremely great in degree; \textsc{synonym}: \textbf{enormous, vast}.} relevance\footnote{\textbf{relevance} [n] [uncountable] \textbf{1.} the fact of being valuable \& useful to people in their lives \& work; \textbf{2.} \textbf{relevance (of something) (to something\texttt{/}somebody)} the fact of being closely connected with the subject you are discussing or the situation you are thinking about.} to those much closer to us than politicians\footnote{\textbf{politician} [n] a person whose job is concerned with politics, especially as an elected member of parliament, etc.}, celebrities\footnote{\textbf{celebrity} [n] (plural \textbf{celebrities}) \textbf{1.} [countable] a famous person; \textbf{2.} [uncountable] the state of being famous, \textsc{synonym}: \textbf{fame}.}, or business\footnote{\textbf{business} [n] \textbf{1.} [uncountable] the activity of making, buying, selling or supplying goods or services for money; \textbf{2.} [countable] a commercial organization such as a company, shop or factory; \textbf{3.} [uncountable] work or another activity that is part of your job \& not done for pleasure or for any other reason; \textbf{4.} [uncountable] the amount of work done by a company, etc.; the rate, volume, value or quality of this work; \textbf{5.} [countable] a particular area of commercial activity; \textbf{6.} [uncountable] the fact of a person or people buying goods or services from a business; \textbf{7.} [uncountable] something that concerns a particular person or organization; \textbf{8.} [uncountable] important matters that need to be dealt with or discussed; \textbf{9.} [singular] (usually with an adjective) \textbf{business (of something\texttt{/}of doing something)} a situation or a series of events; \textbf{go about your business} [idiom] to do the things that you normally do; \textbf{out of business} [idiom] having stopped operating as a business because there is no more money or work available.} magnates\footnote{\textbf{magnate} [n] a person who is rich, powerful \& successful, especially in business.} -- it tells us a great deal about our own families, friends, partners, \& work colleagues. It also resonates\footnote{\textbf{resonate} [v] \textbf{1.} [intransitive] to remind somebody of something; to be similar to what somebody thinks or believes; \textbf{2.} [intransitive] (of a voice, an instrument, etc.) to make a deep, clear sound that continues for a long time; \textbf{3.} [intransitive] (\textit{physics}) to come to resonance.} a great deal in understanding our own minds, leading to a greater self-awareness\footnote{\textbf{self-awareness} [n] [uncountable] knowledge \& understanding of your own character.} of our own thoughts \& behaviors.

As well as offering us a basic understanding\footnote{\textbf{understanding} [n] \textbf{1.} [uncountable, countable, usually singular] the fact or state of knowing or realizing something, e.g. what somebody\texttt{/}something is like, how or why people do things, how something happens or why something is important; \textbf{2.} [uncountable] kindness \& sympathy, often towards somebody who has different views or who has behaved badly; \textbf{3.} [countable, usually singular] an agreement, often not written in a contract, that people will help each other or that something will happen in a particular way; \textbf{4.} [uncountable, countable] \textbf{understanding (of something) (is that $\ldots$)} the particular way in which somebody understand something.} of all the various theories, disorders\footnote{\textbf{disorder} [n] \textbf{1.} [countable, uncountable] an illness that causes the body or the mind to stop working correctly; \textbf{2.} [uncountable] violent behavior of large groups of people; \textbf{3.} [uncountable] confusion; a lack of order or organization, \textsc{opposite}: \textbf{order}.}, \& therapies\footnote{\textbf{therapy} [n] (plural \textbf{therapies}) \textbf{1.} [uncountable, countable] treatment for a physical problem or an illness; \textbf{2.} [uncountable] $=$ \textbf{psychotherapy}.} that form part of this ever-changing field of study, psychology plays a huge role in our everyday lives. Whether it is in education, the workplace\footnote{\textbf{workplace} [n] (often \textbf{the workplace}) [singular] a place where people work, such as an office or factory.}, sports, or our personal \& intimate\footnote{\textbf{intimate} [a] \textbf{1.} (of a link between things) very close; \textbf{2.} (of people) having a close \& friendly relationship; \textbf{3.} sexual; \textbf{4.} private \& personal, often in a sexual way; \textbf{5.} (of a place or situation) encouraging close, friendly relationships; \textbf{6.} (of knowledge) very detailed \& thorough.} relationships -- \& even the way that we spend our money or how we vote -- there is a branch of psychology that impacts every single 1 of us in our daily lives on a constant\footnote{\textbf{constant} [a] \textbf{1.} [only before noun] happening all the time or repeated; \textbf{2.} that does not change, \textsc{synonym}: \textbf{fixed}, \textsc{opposite}: \textbf{variable}; [n] \textbf{1.} a situation that does not change; \textbf{2.} (\textit{mathematics}) a quantity or measure that does not change its value, \textsc{opposite}: \textbf{variable}.} \& continued basis\footnote{\textbf{basis} [n] (plural \textbf{bases}) \textbf{1.} [singular] the reason why people take a particular action; \textbf{2.} [singular] \textbf{on a $\ldots$ basis} they way something is organized or arranged; \textbf{3.} [countable, usually singular, uncountable] the important facts, ideas or events that support something \& that it can develop from.}. 

\textit{How Psychology Works} considers\footnote{\textbf{consider} [v] \textbf{1.} [transitive] (often used in orders) to think about to give attention to something that you are discussing or studying; \textbf{2.} [transitive, often passive] to think of somebody\texttt{/}something in a particular way; \textbf{3.} [transitive, intransitive] to think about something carefully, especially in order to make a decision; \textbf{4.} [transitive] \textbf{consider somebody\texttt{/}something} to think about something, especially the needs \& feelings of other people, \& be influenced by this when making a decision or taking action; \textbf{all things considered} [idiom] thinking carefully about all the facts of a situation, especially the problems or difficulties.} all aspects\footnote{\textbf{aspect} [n] \textbf{1.} [countable] a particular feature of a situation, an idea or a process; a way in which something may be considered; \textbf{2.} [countable, usually singular] \textbf{aspect (of something)} (\textit{specialist}) a particular surface or side of an object or a part of the body; the direction in which something faces; \textbf{3.} [uncountable, countable] (\textit{grammar}) the form of a verb that shows, e.g., whether the action happens once or many times, is completed or is still continuing.} of psychology -- from theories to therapies, personal issues\footnote{\textbf{issue} [n] \textbf{1.} [countable] an important topic that people are discussing or arguing about; \textbf{2.} [countable] (often \textbf{issues} [plural]) a problem, concern or difficulty; \textbf{3.} [countable] 1 of a regular series of magazines or newspapers; \textbf{4.} [countable, uncountable] something that is supplied or made available for people to buy or use; the fact of supplying or making available things for people to buy or use; \textbf{5.} [uncountable] (\textit{law}) children of your own; [v] \textbf{1.} to make something known formally; to make something available publicly; \textbf{2.} [often passive] to give something to somebody, especially officially; \textbf{3.} \textbf{issue something} to start a legal process against somebody, especially by means of an official document; \textbf{4.} \textbf{issue something} to produce new stamps, coins, shares, etc. for sale to the public; \textbf{issue from something} [phrasal verb] (\textit{formal}) to come out of something.} to practical applications, all presented in an accessible\footnote{\textbf{accessible} [a] \textbf{1.} that can be reached, entered, used or obtained; \textbf{2.} easy to understand.}, stylish\footnote{\textbf{stylish} [a] (\textit{approving}) fashionable \& attractive, \textsc{synonym}: \textbf{classy}.}, \& beautifully simple way. I wish it had been around when I was a psychology student!'' -- Jo Hemmings (consultant editor), \cite[Foreword, p. 9]{DK2018}

\section{What Is Psychology?}
``There are many different approaches to psychology -- the scientific study of the human mind \& how individuals behave. All seeks the key to unlock\footnote{\textbf{unlock} [v] \textbf{1.} \textbf{unlock something} to open the lock of a door, window, etc., usually using a key, \textsc{opposite}: \textbf{lock}; \textbf{2.} \textbf{unlock something} to discover something \& let it be known; \textbf{3.} \textbf{unlock something} to enable a mobile phone to use any network rather than only particular one; \textbf{4.} \textbf{unlock something} to use a code or password to get access to more data or features on a computer or phone, or in a computer game.} people's thoughts, memories, \& emotions.'' -- \cite[Foreword, p. 10]{DK2018}

\subsection{The Development of Psychology}
``Most advances in psychology are recent, dating back about 150 years, but its origins\footnote{\textbf{origin} [n] \textbf{1.} [countable, uncountable] (\textbf{origins} [plural]) the point from which something starts; the cause of something; \textbf{2.} [countable, uncountable] (\textbf{origins} [plural]) a person's social \& family background; \textbf{3.} [countable] (\textit{mathematics}) a fixed point from which coordinates are measured.} lie with the philosophers\footnote{\textbf{philosopher} [n] a person who studies or writes about philosophy.} of ancient Greece \& Persia. Many approaches \& fields of study have been developed that give psychologists a toolkit\footnote{\textbf{toolkit} [n] \textbf{1.} a set of tools in a box or bag; \textbf{2.} (\textit{computing}) a set of software tools; \textbf{3.} the things that you need in order to achieve something.} to apply to the real world. As society\footnote{\textbf{society} [n] (plural \textbf{societies}) \textbf{1.} [uncountable, countable] people in general, living together in communities; a particular community of people who share the same customs, laws, etc.; \textbf{2.} [countable] a group of people who join together for a particular purpose. The written abbreviation \textbf{Soc.} is used in the names of particular societies.; \textbf{3.} [uncountable] \textbf{society (of something)} the state of being with other people, \textsc{synonym}: \textbf{company}.} has changed, new applications\footnote{\textbf{application} [n] \textbf{1.} [uncountable, countable] the use of something such as an idea, method, rule, etc.; a use that something has; \textbf{2.} [countable] a formal (often written) request to an organization or authority for something, such as a job or permission to do something, or to join a group; \textbf{3.} [countable] a program or piece of software designed to do a particular job; \textbf{4.} [countable, uncountable] \textbf{application (of something) (to something)} the use of something to produce a particular physical effect; \textbf{5.} [countable, uncountable] \textbf{application (of something)} the action of putting or spreading something onto a surface or object.} have also arisen to meet people's needs.''
\begin{itemize}
	\item \textbf{c.1550 BCE.} The Ebers Papyrus (Egyptian medical papyrus\footnote{\textbf{papyrus} [n] (plural \textbf{papyri}) \textbf{1.} [uncountable] a tall plant with thick stems that grows in water, especially in Africa; \textbf{2.} [uncountable] paper made from the stems of the papyrus plant, used in ancient Egypt for writing \& drawing on; \textbf{3.} [countable] a document or piece of paper made of papyrus.}) mentions depression\footnote{\textbf{depression} [n] \textbf{1.} [uncountable, countable] a medical condition in which somebody feels very sad \& anxious \& often has physical symptoms such as being unable to sleep; \textbf{2.} [uncountable, countable] a period when there is little economic activity \& many people are poor or without jobs; \textbf{3.} [countable] a part of a surface that is lower than the parts around it, \textsc{synonym}: \textbf{hollow}; \textbf{4.} [countable] (\textit{specialist}) a weather condition in which the pressure of the air becomes lower, often causing rain; \textbf{5.} [uncountable, countable] \textbf{depression of something} the action of pressing something; the fact of something becoming lower.}
\end{itemize}
\textsc{Ancient Greek philosophers.}
\begin{itemize}
	\item \textbf{470--370 BCE.} Democritus makes a distinction\footnote{\textbf{distinction} [n] \textbf{1.} [countable] a clear difference, especially between people or things that are similar or related; \textbf{2.} [uncountable] the division of people or things into different groups; \textbf{3.} [singular] \textbf{distinction of being\texttt{/}doing something} the quality of being something that is special; \textbf{4.} [uncountable, countable] a special mark, grade or award that is given to somebody especially a student, for excellent work.} between the intellect\footnote{\textbf{intellect} [n] [uncountable, countable] the ability to think in a logical way \& understand things, especially at an advanced level; your mind.} \& knowledge gained through the senses; Hippocrates introduces the principle of scientific medicine
	\item \textbf{387 BCE.} Plato suggests that \fbox{the brain is the seat of mental processes}.
	\item \textbf{350 BCE.} Aristotle writes on the soul\footnote{\textbf{soul} [n] \textbf{1.} [countable] the spiritual part of a person, believed to exist after death; \textbf{2.} [countable] a person or group's inner character, containing their particular beliefs, desires \& characteristics; \textbf{3.} [singular] the spiritual \& moral qualities of humans in general, \textsc{synonym}: \textbf{psyche}.} in \textit{De Anima}, \& he introduces the tabula rasa\footnote{\textbf{tabula rasa} [n] (plural \textbf{tabulae rasae}) (\textit{from Latin, formal}) \textbf{1.} a situation in which there are no fixed ideas about how something should develop; \textbf{2.} the human mind as it is at birth, with no ideas or thoughts in it.} (blank slate\footnote{\textbf{a blank canvas\texttt{/}slate} [idiom] a person or thing that hs the potential to be developed or changed in many different ways.}) concept\footnote{\textbf{concept} [n] an idea; a basic principle.} of the mind
	\item \textbf{c.300--30 BCE.} Zeno teaches stoicism\footnote{\textbf{stoicism} [n] [uncountable] (\textit{formal}) the fact of not complaining or showing what you are feeling when you are suffering.}, the inspiration\footnote{\textbf{inspiration} [n] \textbf{1.} [uncountable] the experience of being made to feel confident \& excited about doing something; \textbf{2.} [countable, usually singular] \textbf{inspiration (to somebody)} a person or thing that makes you feel confident \& excited about doing something; \textbf{3.} [uncountable, countable, usually singular] the idea of doing something or the reason for doing something; the person or thing that provides this.} for CBT (cognitive\footnote{\textbf{cognitive} [a] [usually before noun] (\textit{psychology}) connected with the mental processes of understanding.} behavioral\footnote{\textbf{behavioural} [a] (US \textbf{behavioral}) [usually before noun] involving or connected with behavior.} therapy\footnote{\textbf{therapy} [n] (plural \textbf{therapies}) \textbf{1.} [uncountable, countable] treatment for a physical problem or an illness; \textbf{2.} [uncountable] $=$ \textbf{psychotherapy}.}) in the 1960s
	\item \textbf{705 CE.} The 1st hospital for the mentally\footnote{\textbf{mentally} [adv] connected with or happening in the mind.} ill\footnote{\textbf{ill} [a] \textbf{1.} (\textit{especially British English}) (\textit{North American English usually} \textbf{sick}) [not usually before noun] suffering from an illness or disease; not feeling well; \textbf{2.} [usually before noun] bad or harmful; \textbf{ill at ease} [idiom] feeling embarrassed \& uncomfortable; [adv] \textbf{1.} (especially in compounds) badly or in an unpleasant way; \textbf{2.} badly; not in an acceptable way; \textbf{3.} only with difficulty; [n] [usually plural] (\textit{formal}) a problem or harmful thing.} is built in Baghdad (followed by hospitals in Cairo in 800 \& Damascus in 1270)
\end{itemize}
\textsc{Scholars of the early Muslim world.}
\begin{itemize}
	\item \textbf{850.} Ali ibn Sahl Rabban al Tabari develops the idea of clinical\footnote{\textbf{clinical} [a] [only before noun] connected with the examination \& treatment of patients \& their illnesses.} psychiatry\footnote{\textbf{psychiatry} [n] [uncountable] the study \& treatment of mental illness.} to treat mental patients
\end{itemize}
p. 12

\subsection{Psychoanalytical Theory}

\subsection{Behaviorist Approach}

\subsection{Humanism}

\subsection{Cognitive Psychology}

\subsection{Biological Psychology}

\subsection{How The Brain Works}

\subsection{How Memory Works}

\subsection{How Emotions Work}

\section{Psychological Disorders}

\subsection{Diagnosing Disorders}

\subsection{Depression}

\subsection{Bipolar Disorder}

\subsection{Perinatal Mental Illness}

\subsection{DMDD (Disruptive Mood Dysregulation Disorder)}

\subsection{SAD (Seasonal Affective Disorder)}

\subsection{Panic Disorder}

\subsection{Specific Phobias}

\subsection{Agoraphobia}

\subsection{Claustrophobia}

\subsection{GAD (Generalized Anxiety Disorder)}

\subsection{Social Anxiety Disorder}

\subsection{Separation Anxiety Disorder}

\subsection{Selective Mutism}

\subsection{OCD (Obsessive Compulsive Disorder)}

\section{Healing Therapies}

\section{Psychology in The Real World}

%------------------------------------------------------------------------------%

\selectlanguage{english}
\chapter{\href{https://nesslabs.com/}{Ness Labs}}
\selectlanguage{vietnamese}

\section{\href{https://nesslabs.com/}{Ness Labs}}
\textbf{Slogan.} ``\textit{Make the most of your mind.} Build a lab for your mind with neuroscience-based\footnote{\textbf{neuroscience} [n] [uncountable] the science that deals with the structure \& function of the brain \& the nervous system.} content\footnote{\textbf{content} [n] \textbf{1.} (\textbf{content}) [plural] \textbf{content (of something)} the things that are contained in something; \textbf{2.} (\textbf{contents}) [plural] the different sections that are contained in a book, magazine, journal or website; a list of these sections; \textbf{3.} [singular] the subject matter of a book, speech, programme, etc.; \textbf{4.} [singular] (following a noun or an adjective) the amount of a substance that is contained in something else; \textbf{5.} [uncountable] the information or other material contained on a website, CD-ROM, etc.; [a] [not before noun] satisfied \& happy with what you have; willing to do or accept something; [v] \textbf{content yourself with something} to accept \& be satisfied with something \& not try to have or do something better.} \& conversations\footnote{\textbf{conversation} [n] [countable, uncountable] an informal talk involving a small group of people or only 2; the activity of talking in this way.}. Join a community\footnote{\textbf{community} [n] (plural \textbf{communities}) \textbf{1.} (often \textbf{the community}) [singular] all the people who live in a particular area, country, etc. when considered as a group; \textbf{2.} [countable] (used in compounds) a group of people who share the same religion, race, job, etc.; \textbf{3.} [uncountable] (\textit{approving}) the feeling or sharing things \& belonging to a group in the place where you live; \textbf{4.} [countable] (\textit{biology}) a group of plants \& animals growing or living in the same place or environment; \textbf{the global\texttt{/}international community} [idiom] the countries of the world, considered as a group.} of curious\footnote{\textbf{curious} [a] \textbf{1.} having a strong desire to know about something; \textbf{2.} strange \& unusual.} humans who want to achieve more without sacrificing their mental health\footnote{\textbf{mental health} [n] [uncountable] \textbf{1.} the state of health of somebody's mind; \textbf{2.} the system for treating people with mental health problems.}. 1 weekly email with mindful\footnote{\textbf{mindful} [a] \textbf{1.} [not before noun] (\textit{formal}) remembering somebody\texttt{/}something \& considering them or it when you do something, \textsc{synonym}: \textbf{conscious}; \textbf{2.} concentrating on the present moment, especially as a technique to help you relax.} productivity \& creativity\footnote{\textbf{creativity} [n] [uncountable] the ability to produce something new, using skill \& imagination.} tips.''

\begin{quotation}
	\textit{``When learning is purposeful\footnote{\textbf{purposeful} [a] having a useful purpose; acting with a clear aim \& with determination.}, creativity blossoms\footnote{\textbf{blossom} [n] [countable, uncountable] a flower or a mass of flowers, especially on a fruit tree or bush; [v] \textbf{1.} [intransitive] (of a tree or bush) to produce blossom; \textbf{2.} [intransitive] to become more healthy, confident or successful.}. When creativity blossoms, thinking emanates\footnote{\textbf{emanate} [v] \textbf{emanate from something} to come from something or somewhere, \textsc{synonym}: \textbf{issue from something}.}. When thinking emanates, knowledge is fully lit\footnote{\textbf{lit} past tense, past participle of \textbf{light}.}.''} -- A.P.J. Abdul Kalam (1931--2015), Aerospace Scientist
	
	\textit{``The consistency\footnote{\textbf{consistency} [n] (plural \textbf{consistencies}) \textbf{1.} [uncountable] (\textit{often approving}) the quality of always behaving in the same way or of having the same opinions or standards; the quality of being consistent; \textbf{2.} [countable, uncountable] the consistency of a mixture or a substance, especially a liquid, is how thick, firm or smooth it is.} \& thoughtfulness\footnote{\textbf{thoughtfulness} [n] [uncountable] \textbf{1.} the quality of being quiet, because you are thinking; \textbf{2.} \textbf{thoughtfulness (for somebody)} (\textit{approving}) the quality of thinking about \& caring for other people, \textsc{synonym}: \textbf{consideration, kindness}; \textbf{3.} careful thought that is put into doing something.} of Ness Labs inspires\footnote{\textbf{inspire} [v] \textbf{1.} to make somebody feel confident \ excited about doing something; \textbf{2.} [usually passive] to give somebody the idea for something; to be the reason why somebody does something; \textbf{3.} to make somebody have a particular feeling or emotion.} me to question the ordinary\footnote{\textbf{ordinary} [a] not unusual or different in any way.} \& iterate\footnote{\textbf{iterate} [v] [intransitive] to repeat a mathematical or computing process or set of instructions again \& again, each time applying it to the result of the previous stage.} towards\footnote{\textbf{towards} [prep] (also \textbf{toward} \textit{especially in North American English}) \textbf{1.} in the direction of somebody\texttt{/}something; \textbf{2.} aiming to achieve something; moving closer to achieving something; \textbf{3.} close or closer to a point in time; \textbf{4.} in relation to somebody\texttt{/}something.} being a better version of myself.''} -- Steph Smith, Founder, Integral Labs
	
	\textit{``Anne-Laure is skilled\footnote{\textbf{skilled} [a] \textbf{1.} having enough ability, experience \& knowledge to be able to do something well, \textsc{synonym}: \textbf{expert}; \textbf{2.} having special experience or training in doing a particular job, \textsc{opposite}: \textbf{unskilled}; \textbf{3.} (of a job) needing special abilities or training, \textsc{opposite}: \textbf{unskilled}.} at researching\footnote{\textbf{research} [n] [uncountable] careful study of a subject, especially in order to discover new facts or information about it. The plural form \textbf{researches} is also sometimes used in British English, but is much less frequent.; [v] \textbf{1.} [transitive, intransitive] to study something carefully \& try to discover new facts about it; \textbf{2.} [transitive] to collect information for an article, a book, etc.} complex\footnote{\textbf{complex} [a] \textbf{1.} made of many different things or parts that are connected, \textsc{synonym}: \textbf{complicated}; \textbf{2.} difficult to understand or deal with; [n] \textbf{1.} \textbf{complex of something} a large number of things that are connected, often in a way that is confusing or difficult to understand; \textbf{2.} a group of buildings of a similar type together in 1 place; \textbf{3.} (\textit{chemistry}) an ion or molecule in which 1 or more groups are bonded to a metal atom by shared pairs of electrons provided by atoms in the group.} topics\footnote{\textbf{topic} [n] a particular subject that is studied, written about or discussed.}, \& condensing\footnote{\textbf{condense} [v] \textbf{1.} [intransitive, transitive] to change from a gas into a liquid; to make a gas change into a liquid; \textbf{2.} [intransitive, transitive] to fill a smaller amount of space; to put something into a smaller amount of space; \textbf{3.} [transitive] to put something such as a piece of writing into fewer words; to put a lot of information into a small space.} her findings\footnote{\textbf{finding} [n] \textbf{1.} [usually plural] information that is discovered as the result of research into something; \textbf{2.} (\textit{law}) a decision made by the judge or jury in a court case.} into a digestible\footnote{\textbf{digestible} [a] \textbf{1.} (of food) easy to digest, \textsc{opposite}: \textbf{indigestible}; \textbf{2.} (of information) easy to understand, \textsc{opposite}: \textbf{indigestible}.} format\footnote{\textbf{format} [n] [countable, uncountable] \textbf{1.} the general arrangement, plan or design of something; \textbf{2.} a particular way in which data is processed, stored or displayed; the form in which information or recordings are made available; [v] \textbf{1.} \textbf{format something} to prepare a computer disk so that data can be recorded on it; \textbf{2.} \textbf{format something (to do something)} to arrange text, etc. in a particular way on a page or screen.} that both entertains\footnote{\textbf{entertain} [v] \textbf{1.} [transitive, intransitive] to interest \& be enjoyed by somebody; \textbf{2.} [transitive] (not used in the progressive tenses) \textbf{entertain something} to consider an idea, a hope, a feeling, etc.; \textbf{3.} [intransitive, transitive] to invite people to eat or drink with you as your guests, especially in your home.} \& makes you smarter.''} -- Leandro, Co-Founder, Unubo
	
	\textit{``This was the resource\footnote{\textbf{resource} [n] \textbf{1.} [countable, usually plural] a supply of something that a country, an organization or a person has \& can use; \textbf{2.} [countable] something that can be used to help achieve an aim, especially as a part of work or study; \textbf{3.} (\textbf{resources}) [plural] personal qualities that help you deal with a situation.} I didn't know I needed -- SO badly. Bite-sized\footnote{\textbf{bite-sized} [a] (also \textbf{bite-size}) [usually before noun] \textbf{1.} small enough to put into the mouth \& eat; \textbf{2.} (\textit{informal}) very small or short.} but in-depth\footnote{\textbf{in-depth} [a] [usually before noun] very thorough \& detailed.} insights\footnote{\textbf{insight} [n] \textbf{1.} [countable, uncountable] an understanding of a particular situation or thing; \textbf{2.} [uncountable] the ability to see \& understand the truth about people or situations.} into my brain. Anne-Laure's writing has changed the way I approach work.''} -- Kelly Miller, Director, BPA
\end{quotation}

\section{\href{https://nesslabs.com/taker-giver-matcher}{Ness Labs\texttt{/}Are you a taker, a giver, or a matcher?}}
``Some people only help when it benefits\footnote{\textbf{benefit} [n] \textbf{1.} [countable, uncountable] a helpful \& useful effect that something has; an advantage that something provides; \textbf{2.} [uncountable, countable] (\textit{British English}) money provided by the government to people who need financial help because they are unemployed, sick, etc.; \textbf{give somebody the benefit of the doubt} [idiom] to accept that somebody has told the truth or has not done something wrong because you cannot prove that they have not told the truth\texttt{/}have done something wrong; [v] \textbf{1.} [intransitive] to be in a better position because of something; \textbf{2.} [transitive] \textbf{benefit somebody\texttt{/}something} to be useful or provide an advantage to somebody\texttt{/}something.} themselves, other foster\footnote{\textbf{foster} [v] \textbf{1.} \textbf{foster something} to encourage something to develop, \textsc{synonym}: \textbf{promote}; \textbf{2.} \textbf{foster somebody} (\textit{especially British English}) to take another person's child into your home for a period of time, without becoming the child's legal parent; [a] [only before noun] used with some nouns in connection with the fostering of a child.} transactional\footnote{\textbf{transactional} [a] \textbf{1.} relating to the process of buying or selling; \textbf{2.} relating to communication between people.} relationships, while yet others are generous\footnote{\textbf{generous} [a] (\textit{approving}) \textbf{1.} giving or willing to give time, money, etc. freely; given freely; \textbf{2.} more than is necessary; large; \textbf{3.} kind in the way you treat people; willing to see what is good about somebody\texttt{/}something.} with their time \& energy\footnote{\textbf{energy} [n] \textbf{1.} [uncountable, countable] the ability of matter or radiation to perform work because of its mass, movement, electrical charge, etc.; \textbf{2.} [uncountable] a source of power that can be used by somebody\texttt{/}something, e.g. to provide light \& heat, or to work machines; \textbf{3.} [uncountable] the effort needed to do work or other physical or mental activities; \textbf{4.} (\textbf{energies}) [plural] the physical \& mental effort that you use to do something.}, without asking for anything in return\footnote{\textbf{return} [v] \textbf{1.} [intransitive] \textbf{return (to $\ldots$) (from $\ldots$)} to come or go back from 1 place to another; \textbf{2.} [transitive] to bring, give, put or send something\texttt{/}somebody back to a particular person or place; \textbf{3.} [intransitive] to come back again, \textsc{synonym}: \textbf{reappear}; \textbf{4.} [intransitive] \textbf{return (to something)} to start discussing a subject you were discussing earlier, or doing an activity you were doing earlier; \textbf{5.} [intransitive, transitive] to go back, or to make something go back, to a previous state; \textbf{6.} [transitive] \textbf{return something} to do something or give something to somebody because they have done or given the same to you 1st; \textbf{7.} [transitive] \textbf{return something} to give or produce something such as a response, a result, a particular amount of money, etc.; \textbf{8.} [transitive, often passive] \textbf{return somebody (to something) $|$ return somebody (as something)} (\textit{British English}) to elect somebody to a political position; \textbf{9.} [transitive] \textbf{return a verdict} to give a decision about something in court; [n] \textbf{1.} [singular] the action of arriving in or coming back to a place that you were in before; \textbf{2.} [singular, uncountable] the action of giving, putting or sending something\texttt{/}somebody back; \textbf{3.} [singular] \textbf{return (of something)} the situation when a feeling or state that has not been experienced for some time starts again, \textsc{synonym}: \textbf{reappearance}; \textbf{4.} [singular] \textbf{return to something} the action of going back to an activity that you used to do, or to a situation that you used to be in; \textbf{5.} [uncountable, countable, usually plural] \textbf{return (on something)} the amount of profit that you get from something, \textsc{synonym}: \textbf{earnings, yield}; \textbf{6.} [countable] an official report or statement that gives particular information to the government or another body; \textbf{in return (for something)} [idiom] as an exchagne or a reward for something; as a response to something.}. Whether in their personal or professional\footnote{\textbf{professional} [a] \textbf{1.} [only before noun] connected with a job that needs special training or skill, especially one that needs a high level of education; \textbf{2.} (of people) having a job that needs special training \& a high level of education; \textbf{3.} showing that somebody is well trained \& extremely skilled, \textsc{synonym}: \textbf{competent}; \textbf{4.} suitable or appropriate for somebody working in a particular profession; \textbf{5.} doing something as a paid job rather than just for pleasure; [n] a person who does a job that needs special training \& a high level of education.} relationships, takers\footnote{\textbf{taker} [n] \textbf{1.} [usually plural] a person who is willing to accept something that is being offered; \textbf{2.} (often in compounds) a person who takes something.}, givers\footnote{\textbf{giver} [n] a person or an organization that gives something, especially money.}, \& matchers achieve different outcomes\footnote{\textbf{outcome} [n] the result or effect of an action or event.}. Surprisingly\footnote{\textbf{surprisingly} [adv] in a way that causes surprise.}, givers display\footnote{\textbf{display} [v] \textbf{1.} [transitive] to put something in a place where people can see it easily; to show something to people, \textsc{synonym}: \textbf{exhibit}; \textbf{2.} [transitive] \textbf{display something} to show signs of something, especially a quality, characteristic or feeling; \textbf{3.} [transitive] \textbf{display something} (of a computer, notice, table, etc.) to show information; \textbf{4.} [intransitive] (of male birds \& animals) to show a special pattern of behavior that is intended to attract a female bird or animal; [n] \textbf{1.} [countable] an arrangement of things in a public place to give information or entertain people or advertise something for sale. Things that are \textbf{on display} are put in a place where people can look at them.; \textbf{2.} [countable, uncountable] \textbf{display of something} behavior that shows a particular quality, feeling or ability; \textbf{3.} [uncountable] \textbf{display of something} the act of placing something in a public place for people to see; \textbf{4.} [countable] \textbf{display (of something)} an act of performing a skill or of showing something happening, in order to entertain; \textbf{5.} [countable, uncountable] \textbf{display (of something)} a special pattern of behavior that a male bird or animal shows in order to attract a female bird or animal.} the most radically\footnote{\textbf{radically} [adv] completely; to a very great extent.} distinctive\footnote{\textbf{distinctive} [a] having a quality or characteristic that makes something different \& easily noticed, \textsc{synonym}: \textbf{characteristic}.} results. \textit{Are you a taker, a giver, or a matcher?} \textit{\& how can you shift\footnote{\textbf{shift} [n] \textbf{1.} [countable] a change in position or direction; \textbf{2.} [countable] a period of time worked by a group of workers who start work as another group finishes; \textbf{3.} [uncountable] the system on a keyboard that allows capital letters or a different set of characters to be typed; the key that operates this system; [v] \textbf{1.} [transitive] \textbf{shift something (away from\texttt{/}from A) (to\texttt{/}towards B)} to change the attention, direction or focus of something; \textbf{2.} [intransitive] (of the emphasis or direction of something) to change from 1 state or position to another; \textbf{3.} [intransitive, transitive] to move from 1 position or place to another; to move something in this way; \textbf{shift your ground} [idiom] (\textit{usually disapproving}) to change your opinion about a subject, especially during a discussion.} your reciprocity\footnote{\textbf{reciprocity} [n] [uncountable] a situation in which 2 people, countries, etc. provide the same help or advantages to each other.} style\footnote{\textbf{style} [n] \textbf{1.} [countable, uncountable] the particular way in which something is done; \textbf{2.} [countable, uncountable] the features of a book, painting, building, etc. that make it typical of a particular author, artist, historical period, etc.; \textbf{3.} [countable] a particular design of something, e.g. clothes; \textbf{4.} [uncountable] the quality of being elegant or fashionable \& made to a high standard; \textbf{5.} [uncountable] the correct use of language; \textbf{6.} (in adjectives) having the type of style mentioned; \textbf{7.} [countable] (\textit{biology}) the long thin part of a flower that carries the stigma.} to have a positive impact\footnote{\textbf{impact} [n] [countable, usually singular, uncountable] \textbf{1.} the powerful effect that something has on somebody\texttt{/}something; \textbf{2.} the act of 1 object hitting another; the force with which this happens; [v] [transitive, intransitive] to have an effect on something.} on your work, your relationships, \& the world in general\footnote{\textbf{general} [a] \textbf{1.} affecting or including all or most people, places or things; \textbf{2.} [usually before noun] normal; usual; true in most cases; \textbf{3.} including the most important aspects of something; not exact or detailed, \textsc{synonym}: \textbf{broad}, \textsc{opposite}: \textbf{specific}; \textbf{4.} \textbf{the general direction\texttt{/}area} used to describe the approximate, but not exact, direction or area mentioned; \textbf{5.} not limited to a particular subject, use or activity; \textbf{6.} not limited to 1 part or aspect of a person or thing; \textbf{7.} [only before noun] highest in rank. In some titles, \textbf{General} comes after the noun.; \textbf{as a general rule} [idiom] usually; \textbf{of general interest} [idiom] of interest to most people; [n] (abbr., \textbf{Gen.}) an officer of very high rank in the army or the US air force; the commander of an army; \textbf{in general} [idiom] \textbf{1.} usually; mainly; \textbf{2.} as a whole.}?}

\subsection{Takers, Givers, Matchers}
In his book \href{https://amzn.to/32suu4A}{Give \& Take}, psychologist\footnote{\textbf{psychologist} [n] a scientist who studies psychology.} \& Wharton's top-rated\footnote{\textbf{top-rated} [a] [only before noun] most popular with the public.} professor \textsc{Adam Grant} divides\footnote{\textbf{divide} [v] \textbf{1.} [transitive, usually passive, intransitive] to separate into parts or groups; to make something separate into parts or groups; \textbf{2.} [transitive] \textbf{divide something (up) between\texttt{/}among somebody} to give a share of something to each of a number of different people or organizations, \textsc{synonym}: \textbf{share}; \textbf{3.} [transitive] to be the real or imaginary line or barrier that separates 2 areas, things or people, \textsc{synonym}: \textbf{separate}; \textbf{4.} [transitive] \textbf{divide something (between A \& B)} to use different parts of your time or energy for different activities; \textbf{5.} [transitive] to cause 2 or more people to disagree, \textsc{synonym}: \textbf{split}; \textbf{6.} [transitive] \textbf{divide something by something} to calculate something by finding out how many times 1 number or amount is contained in another; \textbf{divide \& rule} [idiom] to keep control over people by making them disagree with \& fight each other, therefore not giving them the chance to join together \& oppose you; [n] [usually singular] \textbf{1.} a difference between 2 groups of people that separates them from each other; a difference between 2 sets of ideas or areas of activity; \textbf{2.} \textbf{divide (between A \& B)} (\textit{especially North American English}) a line of high land that separates 2 valleys or systems of rivers, \textsc{synonym}: \textbf{watershed}; \textbf{bridge the gap\texttt{/}divide (between A \& B)} [idiom] to reduce or get rid of the differences that exist between 2 things or groups of people.} people into 3 groups: takers, givers, \& matchers. He explains: ``Whereas takers strive\footnote{\textbf{strive} [v] [intransitive] to try very hard to achieve something.} to get as much as possible from others \& matchers aim\footnote{\textbf{aim} [n] the purpose of doing something; what somebody is trying to achieve; \textbf{take aim at somebody\texttt{/}something} [idiom] to direct your criticism at somebody\texttt{/}something; [v] \textbf{1.} [transitive] \textbf{be aimed at (doing) something} to have the intention of achieving something; \textbf{2.} [intransitive, transitive] to try or plan to achieve something; \textbf{3.} [transitive, usually passive] \textbf{aim something at somebody} to say or do something that is intended to influence or affect a particular person or group.} to trade\footnote{\textbf{trade} [n] \textbf{1.} [uncountable] the activity of buying \& selling or of exchanging goods or services between people or countries. \textbf{Fair trade} is trade between companies in developed countries \& producers in developing countries in which fair prices are paid to the producers.; \textbf{2.} [countable] a particular type of business; \textbf{3.} (\textbf{the trade}) [singular $+$ singular or plural verb] the people or companies that are connected with a particular area of business; \textbf{4.} [countable, uncountable] a job, especially one that involves working with your hands \& that requires special training \& skills; \textbf{5.} [uncountable, singular] the amount of goods or services that are sold, \textsc{synonym}: \textbf{business}; [v] \textbf{1.} [intransitive, transitive] to buy \& sell goods \& services. In economics, \textbf{trade} is usually refer to 1 country or economy exchanging goods or services with another.; \textbf{2.} [intransitive] to exist \& operate as a business or company; \textbf{3.} [intransitive, transitive] to be bought \& sold, or to buy \& sell something, on a stock exchange or other financial institution; \textbf{4.} [transitive] to exchange something that you have for something else.} evenly\footnote{\textbf{evenly} [adv] \textbf{1.} in a smooth or regular way; \textbf{2.} with equal amounts for each person or in each place.}, givers are the rare\footnote{\textbf{rare} [a] (\textbf{rarer, rarest}) \textbf{1.} not done, seen, happening, etc. very often; \textbf{2.} existing only in small numbers \& therefore valuable or interesting.} breed\footnote{\textbf{breed} [v] \textbf{1.} [intransitive] (of animals) to have sex \& produce young; \textbf{2.} [transitive] to keep animals or plants in order to produce young ones in a controlled way; \textbf{3.} [transitive] \textbf{breed something} to be the cause of something; [n] \textbf{1.} a type of animal with a particular appearance that makes it different from others of the same species \& that is the result of having been developed in a controlled way; \textbf{2.} [usually singular] a type of person.} of people who contribute\footnote{\textbf{contribute} [v] \textbf{1.} [intransitive] \textbf{contribute (to something)} to be 1 of the causes of something; \textbf{2.} [intransitive, transitive] to help to improve or achieve something, especially by adding new ideas; \textbf{3.} [transitive, intransitive] to give something, especially money or goods, to help somebody\texttt{/}something; \textbf{4.} [transitive, intransitive] to write something for a newspaper, magazine, website, or a radio or television programme; to speak during a meeting or conversation, especially to give your opinion.} to others without expecting anything in return.''
\begin{itemize}
	\item \textbf{Takers.} Takers are self-focused\footnote{\textbf{focused} [a] (also \textbf{focussed}) with your attention directed to what you want to do; with very clear aims.} \& only help others strategically\footnote{\textbf{strategically} [adv] \textbf{1.} in a way that is connected with achieving a particular purpose or gaining an advantage; \textbf{2.} in a way that is connected with gaining an advantage in a war or other military situation.}, when the benefits to themselves outweigh\footnote{\textbf{outweigh} [v] \textbf{outweigh something} to be greater or more important than something.} the personal costs. In the words of Adam Grant: ``Takers have a distinctive\footnote{\textbf{distinctive} [a] having a quality or characteristic that makes something different \& easily noticed, \textsc{synonym}: \textbf{characteristic}.} signature\footnote{\textbf{signature} [n] \textbf{1.} [countable] your name as you usually write it, e.g. at the end of a letter; \textbf{2.} [uncountable] the act of signing something; \textbf{3.} [countable] a particular quality that makes something different from other similar things \& makes it easy to recognize.}: they like to get more than they give. They tilt\footnote{\textbf{tilt} [v] \textbf{1.} [intransitive, transitive] to move into a position with 1 side or end higher than the other; to make something move in this way, \textsc{synonym}: \textbf{tip}; \textbf{2.} [transitive, intransitive] to influence a situation so that 1 particular opinion, person, etc. is preferred or more likely to succeed than another; to change in this way; [n] [singular, uncountable] a position in which 1 end or side of something is higher than the other.} reciprocity\footnote{\textbf{reciprocity} [n] [uncountable] a situation in which 2 people, countries, etc. provide the same help or advantages to each other.} in their own favor\footnote{\textbf{favour} [n] (\textit{US} \textbf{favor}) \textbf{1.} [countable] a thing that you do to help somebody; \textbf{2.} [uncountable] approval or support for somebody\texttt{/}something; \textbf{find favor (with somebody\texttt{/}something)} [idiom] to become accepted \& popular; \textbf{in favor (of somebody\texttt{/}something)} [idiom] \textbf{1.} supporting \& agreeing with something\texttt{/}somebody; \textbf{2.} likely to produce a particular result, often in an unfair way; \textbf{3.} in exchange for another thing (because the other thing is better or you want it more); \textbf{in somebody's favor} [idiom] \textbf{1.} if something is in somebody's favor, it gives them an advantage or helps them; \textbf{2.} a decision or judgment that is in somebody's favor benefits that person or says that they were right; [v] \textbf{1.} to prefer 1 thing to another, especially a particular system, plan or way of doing something; \textbf{2.} to treat somebody\texttt{/}something better than others, especially in an unfair way; \textbf{3.} \textbf{favor something} to provide suitable conditions for something; to make it easier for something to happen.}, putting their own interests ahead of other's needs.''
	\item \textbf{Givers.} On the other hand, givers will help whenever\footnote{\textbf{whenever} [conjunction] \textbf{1.} every time that; \textbf{2.} at any time that; on any occasion that.} the benefits to others exceed\footnote{\textbf{exceed} [v] \textbf{1.} \textbf{exceed something} to be greater than a particular number or amount; \textbf{2.} \textbf{exceed something} to go beyond what the law, an order or a rule says you are allowed to do; \textbf{3.} \textbf{exceed something} to be better than something, \textsc{synonym}: \textbf{surpass}.} the personal costs. As Adam Grant explains: ``In the workplace\footnote{\textbf{workplace} [n] (often \textbf{the workplace}) [singular] a place where people work, such as an office or factory.}, givers are a relatively\footnote{\textbf{relatively} [adv] to a fairly large degree, especially in comparison with something else; \textbf{relatively speaking} [idiom] used when you are comparing something with all similar things.} rare breed. They tilt reciprocity in the other direction, preferring to give more than they get. Whereas takers tend to be self-focused, evaluating what other people can offer them, givers are other-focused, paying more attention to what other people need from them.''
	\item \textbf{Matchers.} Finally, matchers strive to preserve\footnote{\textbf{preserve} [v] \textbf{1.} \textbf{preserve something} to keep a particular quality or feature; \textbf{2.} to keep something safe from harm, in good condition or in its original state; \textbf{3.} to prevent something from decaying, by treating it in a particular way; [n] [singular] an activity, job or interest that is thought to be suitable for 1 particular person or group of people.} an equal balance\footnote{\textbf{balance} [n] \textbf{1.} [singular, uncountable] a situation in which all parts exist in equal or appropriate amounts; \textbf{2.} [countable, usually singular] the amount of money in a bank account; the amount of a bill that remains after part has been paid; \textbf{3.} [uncountable] the ability to keep steady with an equal amount of weight on each side of the body; \textbf{strike a balance (between A \& B)} [idiom] to manage to find a way of being fair to 2 opposing things; to find an acceptable position which is between 2 things; [v] \textbf{1.} [transitive, often passive, intransitive] to be equal in important or amount to something else that has the opposite effect, \textsc{synonym}: \textbf{offset}; \textbf{2.} [transitive] \textbf{balance A with\texttt{/}\& B} to give equal importance to 2 different things or parts of something; \textbf{3.} [transitive, often passive] \textbf{balance a against B} to compare the importance of 2 different things; \textbf{4.} [transitive] \textbf{balance something} (\textit{finance}) to show or make sure that in an account the total money spent is equal to the total money received; \textbf{5.} [intransitive, transitive] \textbf{balance (something) (on something)} to put your body or something else into a position where it is steady \& does not fall.} between giving \& getting. ``Matchers operate\footnote{\textbf{operate} [v] \textbf{1.} [intransitive] to work, happen or exist, especially in a particular way or place at a particular time, \textsc{synonym}: \textbf{function}; \textbf{2.} [transitive] \textbf{operate something} to use or control a system, process or machine; \textbf{3.} [intransitive] \textbf{operate (on somebody\texttt{/}something)} to cut open somebody's body in order to remove or repair a damaged part.} on the principle\footnote{\textbf{principle} [n] \textbf{1.} [countable] a law, rule or theory that something is based on; \textbf{2.} [singular] a general or scientific law that explains how something works or why something happens; \textbf{3.} [countable] a belief that is accepted as a reason for acting or thinking in a particular way; \textbf{4.} [countable, usually plural, uncountable] a moral rule or a strong belief that influences your actions; \textbf{in principle} [idiom] \textbf{1.} if something can be done in principle, there is no good reason why it should not be done although it has not yet been done \& there may be some difficulties; \textbf{2.} in general but not in detail.} of fairness\footnote{\textbf{fairness} [n] [uncountable] the quality of treating people equally or according to the law or rules.}: when they help others, they protect\footnote{\textbf{protect} [v] \textbf{1.} [transitive, intransitive] to keep somebody\texttt{/}something safe from harm or injury; \textbf{2.} [transitive, usually passive] to introduce laws that make it illegal to kill, harm or damage a particular animal, area of land, building, etc.; \textbf{3.} [transitive] to help an industry in your own country by taxing goods from other countries so that there is less competition; \textbf{4.} [transitive, intransitive] to provide somebody\texttt{/}something with insurance against fire, injury, damage, etc.} themselves by seeking\footnote{\textbf{seek} [v] \textbf{1.} [transitive] to ask for something from somebody, such as help or support; \textbf{2.} [transitive, intransitive] to try to obtain or achieve something; \textbf{3.} [intransitive] \textbf{seek to do something} to try to do something, \textsc{synonym}: \textbf{attempt}; \textbf{4.} (\textbf{-seeking}) (in adjectives \& nouns) looking for or trying to get  the thing mentioned; the activity of doing this; \textbf{seek your fortune} [idiom] (\textit{literary}) to try to find a way to become rich, especially by going to another place; \textbf{seek somebody\texttt{/}something out} [phrasal verb] too look for \& find somebody\texttt{/}something, especially when this means using a lot of effort.} reciprocity. If you're a matcher, you believe in tit for tat\footnote{\textbf{tit for tat} [n] [uncountable] a situation in which you do something bad to somebody because they have done the same to you.}, \& your relationships are governed\footnote{\textbf{govern} [v] \textbf{1.} [transitive, intransitive] \textbf{govern (something)} to control a country or its people \& be responsible for introducing new laws \& for organizing public services \& the economy; \textbf{2.} [transitive, often passive] \textbf{govern something} to control or influence how something happens or functions; to control or influence somebody's actions or behavior.} by even\footnote{\textbf{even} [adv] \textbf{1.} used to emphasized something unexpected or surprising; \textbf{2.} used when you are comparing things, to make the comparison stronger; \textbf{3.} used to introduce a more exact description of somebody\texttt{/}something; \textbf{even as} [idiom] just at the same time as somebody does something or as something else happens; \textbf{even if} [idiom] despite the possibility, fact or belief that; no matter whether; \textbf{even now\texttt{/}then} [idiom] \textbf{1.} despite what has\texttt{/}had happened; \textbf{2.} at this or that exact moment; \textbf{even so} [idiom] despite that; [a] \textbf{1.} equal in number, amount or value; shared equally, \textsc{opposite}: \textbf{uneven}; \textbf{2.} that can be divided exactly by 2, \textsc{opposite}: \textbf{odd}; \textbf{break even} [idiom] to complete a piece of business without either losing money or making a profit; \textbf{have an even chance (of doing something)} [idiom] to be equally likely to do or not do something.} exchanges\footnote{\textbf{exchange} [n] \textbf{1.} [countable, uncountable] an act of giving something to somebody or doing something for somebody \& receiving something in return; \textbf{2.} [countable] a conversation or an argument; \textbf{3.} [uncountable] the process of changing the money of 1 country into that of another; \textbf{4.} [countable] an arrangement when 2 people or groups from different countries visit each other's homes or do each other's jobs for a short time.} of favors.''
\end{itemize}
Of course, most people are not locked\footnote{\textbf{lock} [v] \textbf{1.} [transitive, intransitive] \textbf{lock (something)} to fasten something with a lock; to be fastened with a lock; \textbf{2.} [transitive] \textbf{lock something $+$ adv.\texttt{/}prep} to put something in a safe place \& lock it; \textbf{3.} [intransitive, transitive] to become fixed in 1 position \& unable to move; to make something become fixed in this way; \textbf{4.} [transitive] (\textbf{be locked in\texttt{/}into something}) to be involved in a difficult situation, an argument, a disagreement, etc.; \textbf{lock somebody\texttt{/}yourself in ($\ldots$)} [phrasal verb] to prevent somebody from leaving a place by locking the door; \textbf{lock somebody up} [phrasal verb] (\textit{rather informal}) to put somebody in prison; \textbf{lock something up} [phrasal verb] \textbf{1.} to put money into an investment that you cannot easily turn into cash; \textbf{2.} (\textbf{be locked up in something}) to be in a place where it cannot easily be obtained.} in 1 reciprocity style. ``Giving, taking, \& matching are 3 fundamental\footnote{\textbf{fundamental} [a] \textbf{1.} serious \& very important; affecting the most central \& important parts of something, \textsc{synonym}: \textbf{basic}; \textbf{2.} forming the necessary basis of something, \textsc{synonym}: \textbf{essential}.} styles of social\footnote{\textbf{social} [a] \textbf{1.} [only before noun] connected with society \& the way it is organized; \textbf{2.} [only before noun] connected with activities in which people meet each other for pleasure; \textbf{3.} [only before noun] connected with a person's position in society; \textbf{4.} [only before noun] (\textit{ecology}) (of animals) living naturally in groups, rather than alone.} interaction\footnote{\textbf{interaction} [n] [uncountable, countable] \textbf{1.} the effect that 2 things have on each other; \textbf{2.} the way that people communicate with each other, especially while they work or spend time with them.}, but the lines between them aren't hard \& fast. You might find that you shift from 1 reciprocity style to another as you travel across different work roles \& relationships.'' E.g., you may be a giver when mentoring\footnote{\textbf{mentor} [n] \textbf{1.} an experienced person who advises \& helps somebody with less experience over a period of time; \textbf{2.} an experienced person in a company, university, etc. who trains \& advises new employees or students.}\,\footnote{\textbf{mentoring} [n] [uncountable] the practice of helping \& advising a less experienced person over a period of time, especially as part of a formal programme in a company, university, etc.} a less-experienced\footnote{\textbf{experienced} [a] \textbf{1.} having knowledge or skill in a particular job or activity; \textbf{2.} having knowledge as a result of doing something for a long time, or having had a lot of different experiences.} colleague, act as a taker when negotiating\footnote{\textbf{negotiate} [v] \textbf{1.} [intransitive] to try to reach an agreement by formal discussion; \textbf{2.} [transitive] to arrange or agree something by formal discussion; \textbf{3.} [transitive] \textbf{negotiate something ($+$ adv.\texttt{/}prep.)} to successfully get over or past a difficult part on a path or route; \textbf{4.} [transitive] \textbf{negotiate something ($+$ adv.\texttt{/}prep.)} to successfully solve a problem that is preventing you from achieving something.} your salary\footnote{\textbf{salary} [n] (plural \textbf{salaries}) money that employees receive for doing their job, especially professional employees or people working in an office, usually paid every month.}, \& be a matcher when exchanging productivity\footnote{\textbf{productivity} [n] [uncountable] the rate at which a worker, a company or country produces goods; the amount produced, compared with how much time, work \& money is needed to produce them.} tips\footnote{\textbf{tip} [n] \textbf{1.} the thin pointed end of something; \textbf{2.} a small piece of advice about something practical, \textsc{synonym}: \textbf{hint}; \textbf{3.} a small amount of extra money that you give to somebody, e.g. somebody who serves you in a restaurant; \textbf{the tip of the iceberg} [idiom] only a small part of a much larger problem; [v] \textbf{1.} [intransitive, transitive] to move so that 1 end or side is higher than the other; to move something into this position, \textsc{synonym}: \textbf{tilt}; \textbf{2.} [transitive] \textbf{tip something $+$ adv.\texttt{/}prep.} to make something come out of a container by holding the container at the angle; \textbf{3.} [intransitive, transitive] to develop in a particular direction; to make something develop in a particular direction; \textbf{tip the balance\texttt{/}scales (in favor of, against, etc. somebody\texttt{/}something)} to give somebody\texttt{/}something enough of an advantage or disadvantage, so that the result of something is affected.} with a friend.

Instead of an automatic\footnote{\textbf{automatic} [a] \textbf{1.} (of a machine or device) having controls that work without needing a person to operate them; \textbf{2.} done or happening without thinking, \textsc{synonym}: \textbf{instinctive}; \textbf{3.} always happening as a result of a particular action or situation.} behavior\footnote{\textbf{behavior} [n] \textbf{1.} [uncountable, countable] the way that somebody\texttt{/}something functions or reacts in a particular situation; \textbf{2.} [uncountable] the way that somebody behaves, especially towards other people.}, choosing how we engage\footnote{\textbf{engage} [v] \textbf{1.} \textbf{engage somebody\texttt{/}something} to succeed in attracting \& keeping somebody's attention \& interest; \textbf{2.} to employ somebody to do a particular job; \textbf{engage in something $|$ be engaged in something} [phrasal verb] to take part in an activity; \textbf{engage with something\texttt{/}somebody} [phrasal verb] to become involved with \& try to understand something\texttt{/}somebody.} with friends \& colleagues can be a conscious\footnote{\textbf{conscious} [a] \textbf{1.} [not before noun] aware of something; noticing something, \textsc{opposite}: \textbf{unconscious}; \textbf{2.} able to use your senses \& mental powers to understand what is happening, \textsc{opposite}: \textbf{unconscious}; \textbf{3.} (of actions, feelings, etc.) deliberate or controlled, \textsc{opposite}: \textbf{unconscious}; \textbf{4.} being particularly interested in something.} choice. Adam Grant explains: ``Every time we interact\footnote{\textbf{interact} [v] \textbf{1.} [intransitive] if 1 thing interacts with another, or if 2 things interact, 1 thing has an effect on the other, or the 2 things have an effect on each other; \textbf{2.} [intransitive] \textbf{interact (with somebody)} to communicate with somebody, especially while you work or spend time with them.} with another person at work, we have a choice to make: do we try to claim as much value as we can, or contribute value without worrying about what we receive in return?''

\subsection{The Impact of Giving}
\textit{Does being a giver pay\footnote{\textbf{pay} [v] \textbf{1.} [intransitive, transitive] to give somebody money for work, goods, services, etc.; \textbf{2.} [intransitive] (of a business, etc.) to produce a profit; \textbf{3.} [intransitive, transitive] to result in some advantage for somebody; \textbf{4.} [intransitive, transitive] to suffer or accept a disadvantage because of your beliefs or actions; \textbf{5.} [transitive] \textbf{pay attention\texttt{/}heed\texttt{/}regard\texttt{/}tribute\texttt{/}homage\texttt{/}respect (to somebody\texttt{/}something)} to give attention, etc. to somebody\texttt{/}something; \textbf{6.} [transitive] \textbf{pay a visit (to somebody\texttt{/}something) $|$ pay (somebody\texttt{/}something) a visit} to visit somebody\texttt{/}something; \textbf{pay off} [phrasal verb] (of a plan or an action) to bring benefits or good results; \textbf{pay something off} [phrasal verb] to finish paying money owed for something.} off?} It seems giving does have a positive impact at an organizational\footnote{\textbf{organizational} [a] (\textit{British English also} \textbf{organisational}) \textbf{1.} connected with an organization or with organizations in general; \textbf{2.} connected with the ability to arrange or organize things well.} level. Nathan P. Podsakoff \& his team at the University of Arizona conducted\footnote{\textbf{conduct} [v] \textbf{1.} \textbf{conduct something} to organize \&\texttt{/}or do a particular activity; \textbf{2.} \textbf{conduct something} (of a substance) to allow heat or electricity to pass along or through it; \textbf{3.} \textbf{conduct yourself $+$ adv.\texttt{/}prep.} (\textit{formal}) to behave in a particular way; [n] [uncountable] (\textit{formal}) \textbf{1.} a person's behavior; \textbf{2.} \textbf{conduct of something} the way in which business or an activity is organized \& managed.} a meta-analysis\footnote{\textbf{meta-analysis} [n] [countable, uncountable] (plurla \textbf{meta-analyses}) research that combines the results of a number of related studies.} [\href{https://www.researchgate.net/publication/200824574_Individual-_and_Organizational-Level_Consequences_of_Organizational_Citizenship_Behaviors_A_Meta-Analysis_Article}{Nathan P. Podsakoff, Steven W. Whiting, Philip Podsakoff, Brian D. Blume. \textit{Individual- \& Organizational-Level Consequence of Organizational Citizenship Behaviors: A Meta-Analysis}}] across 38 studies covering more than 3,500 business units, \& found that companies with a culture\footnote{\textbf{culture} [n] \textbf{1.} [uncountable] the customs, beliefs, art, way of lief or social organization of a particular country or group; \textbf{2.} [countable] a country or group with its own customs \& beliefs, art, way of life \& social organization; \textbf{3.} [countable, uncountable] the typical beliefs, attitudes \& behavior that people in a particular group or organization share; \textbf{4.} [uncountable] \textbf{culture (of something)} activities such as literature, music, art \& film, thought of as a group; \textbf{5.} [uncountable] the process of growing cells or bacteria in an artificial substance or medical or scientific study; the substance in which they are grown; \textbf{6.} [countable] a group of cells or bacteria grown for medical or scientific study; [v] \textbf{culture something} to keep cells or bacteria in conditions that are suitable for growth, for medical or scientific study.} of generosity\footnote{\textbf{generosity} [n] [uncountable] the quality of being kind \& generous.} \& giving -- which they call ``Organizational Citizenship\footnote{\textbf{citizenship} [n] [uncountable] \textbf{1.} the legal right to belong to a particular country; \textbf{2.} the state of being a citizen \& accepting the responsibilities of it.} Behaviors'' -- are more likely\footnote{\textbf{likely} [a] (\textbf{likelier, likeliest}) (\textbf{more likely} \& \textbf{most likely} are the usual forms.) \textbf{1.} that can be expected, \textsc{synonym}: \textbf{probable}; \textbf{2.} if somebody is likely to do something, or something is likely to happen, they will probably do it or it will probably happen, \textsc{opposite}: \textbf{unlikely}; \textbf{3.} seeming suitable for a purpose; [adv] probably.} to have higher productivity\footnote{\textbf{productivity} [n] [uncountable] the rate at which a worker, a company or country produces goods; the amount produced, compared with how much time, work \& money is needed to produce them.}, efficiency\footnote{\textbf{efficiency} [n] \textbf{1.} [uncountable] the quality of doing something well with no waste of time or money; \textbf{2.} [uncountable, countable] (\textit{specialist}) the relationship between the amount of energy that goes into a machine or an engine, \& the amount that it produces; \textbf{3.} (\textbf{efficiencies}) [plural] ways of wasting less time \& money or of saving time or money.}, customer satisfaction\footnote{\textbf{satisfaction} [n] \textbf{1.} [uncountable, countable] the good feeling that you have when you have achieved something or when something that you wanted to happen does happen; something that gives you this feeling, \textsc{opposite}: \textbf{dissatisfaction}; \textbf{2.} [uncountable, singular] \textbf{satisfaction (of something)} the act of satisfying a need or desire; \textbf{3.} [uncountable] \textbf{satisfaction (of something)} (\textit{formal}) an acceptable way of dealing with a complaint, a debt, an injury, etc.; \textbf{to somebody's satisfaction} [idiom] \textbf{1.} if you do something to somebody's satisfaction, they are pleased with it; \textbf{2.} if you prove something to somebody's satisfaction, they believe or accept it.}, as well as reduced costs.

But you may want to ask about the individual\footnote{\textbf{individual} [n] \textbf{1.} a person considered separately rather than as part of a group; \textbf{2.} a single member of a group or class; \textbf{3.} a person who is very different from others \& has lots of new \& interesting ideas; [a] \textbf{1.} [only before noun] considered separately rather than as part of a group; \textbf{2.} [only before noun] of or for a particular person; \textbf{3.} [only before noun] designed for use by 1 person; \textbf{4.} characteristic of a particular person or thing; \textbf{5.} (\textit{usually approving}) having an unusual character, \textsc{synonym}: \textbf{distinctive, original}.} impact of being a giver. The answer is pretty surprising. Givers are most likely to occupy\footnote{\textbf{occupy} [v] \textbf{1.} \textbf{occupy something} to fill or use a space, area or amount of time, \textsc{synonym}: \textbf{take up something}; \textbf{2.} \textbf{occupy something} to live or work in a room, house or building; \textbf{3.} \textbf{occupy something} to enter a place in a large group \& take control of it, especially by military force; \textbf{4.} \textbf{occupy something} to have an official job or position, \textsc{synonym}: \textbf{hold}; \textbf{5.} \textbf{occupy something} to be in or at a particular position in a system, \textsc{synonym}: \textbf{hold}; \textbf{6.} to fill your time or keep you busy doing something.} \textit{both the lowest \& highest levels} of an organization\footnote{\textbf{organization} [n] (\textit{British English also} \textbf{organisation}) \textbf{1.} [countable] an organized group of people with a particular purpose, such as a business or government department; \textbf{2.} [uncountable] the way in which the different parts of something are arranged, \textsc{synonym}: \textbf{structure}; \textbf{3.} [uncountable] the act of making arrangements or preparations for something, \textsc{synonym}: \textbf{planning}; \textbf{4.} [uncountable] the quality of being arranged in a neat, careful \& logical way; the ability to plan your work or life well \& in an efficient way.}. ``The worst performers\footnote{\textbf{performer} [n] \textbf{1.} a person or thing that behaves or works in the way mentioned; \textbf{2.} a person who performs for an audience in a show or concert.} \& the best performers are givers; takers \& matchers are more likely to land\footnote{\textbf{land} [n] \textbf{1.} [uncountable] the part of the earth's surface that is not covered by water; \textbf{2.} [uncountable] (\textbf{lands} [plural]) the area of ground that somebody owns, especially when you think of it as property that can be bought or sold; \textbf{3.} [uncountable] (\textbf{lands} [plural]) an area of ground, especially of a particular type or used for a particular purpose, \textsc{synonym}: \textbf{terrain}; \textbf{4.} [countable] a country or state; \textbf{5.} (\textbf{the land}) [uncountable] used to refer to country areas \& the way of life in the countryside, or to ground or soil used for farming; [v] [intransitive, transitive] to arrive on land or another surface; to put somebody\texttt{/}something on land or another surface.} in the middle. ($\ldots$) Givers dominate\footnote{\textbf{dominate} [v] \textbf{1.} [transitive, intransitive] \textbf{dominate (something\texttt{/}somebody)} to control or have a lot of influence over something\texttt{/}somebody, especially in a negative way; \textbf{2.} [transitive] \textbf{dominate something} to be the most important or obvious feature of something; \textbf{3.} [transitive, intransitive] \textbf{dominate (something)} to be the largest, highest or most common thing in a place.} the bottom \& the top of the success ladder\footnote{\textbf{ladder} [n] \textbf{1.} [usually singular] a series of stages by which you can make progress in your life or career; \textbf{2.} a piece of equipment for climbing up \& down something such as the side of a building, consisting of 2 lengths of wood or metal that are joined together by steps.}. Across\footnote{\textbf{across} [prep] \textbf{1.} from 1 side to the other side of something; \textbf{2.} on the other side of something; \textbf{3.} on or over a part of the body; \textbf{4.} in every part of a place, group of people, etc., \textsc{synonym}: \textbf{throughout}; [adv] from 1 side to the other side; \textbf{across from somebody\texttt{/}something} [idiom] opposite somebody\texttt{/}something.} occupations\footnote{\textbf{occupation} [n] \textbf{1.} [countable] a job or profession; \textbf{2.} [uncountable] the act of moving into a country, town, etc. \& taking control of it using military force; the period of time during which a country, town, etc. is controlled in this way; \textbf{3.} [uncountable] the act of living in or using a building, room or piece of land; \textbf{4.} [countable] a way of spending time, especially when you are not working.}, if you examine\footnote{\textbf{examine} [v] \textbf{1.} to consider or study an idea or subject very carefully; \textbf{2.} to look at somebody\texttt{/}something closely, to see if there is anything wrong or to find the cause of a problem; \textbf{3.} \textbf{examine somebody} to give somebody a test to see how much they know about a subject or what they can do.} the link\footnote{\textbf{link} [v] [often passive] \textbf{1.} to make a physical or electronic connection between 1 object, machine or place \& another, \textsc{synonym}: \textbf{connect}; \textbf{2.} to make or have a connection with somebody\texttt{/}something, especially where 1 thing affects the other; \textbf{3.} to state that there is a connection or relationship between 2 things or people, \textsc{synonym}: \textbf{associate}; \textbf{link up (with somebody\texttt{/}something)} [phrasal verb] to join or become joined with somebody\texttt{/}something; [n] \textbf{1.} a connection between 2 or more people or things, especially where one affects the other; \textbf{2.} a relationship between 2 or more people, countries or organizations; \textbf{3.} a means of traveling or communicating between 2 places; \textbf{4.} (\textit{computing}) a place in an electronic document that is connected to another electronic document or to another part of the same document; \textbf{a link in the chain} [idiom] 1 of the stages in a process or a line of argument; \textbf{the weak link (in the chain)} [idiom] the point at which a system or an organization is most likely to fail.} between reciprocity styles \& success, the givers are more likely to become champs\footnote{\textbf{champ} [v] [intransitive, transitive] \textbf{champ (something)} (especially of horses) to bite or eat something noisily; \textbf{champing at the bit} [idiom] (\textit{informal}) impatient to do or start doing something; [n] an informal way of referring to a champion, often used in newspapers.} -- not only chumps\footnote{\textbf{chump} [n] (\textit{old-fashioned, informal}) a stupid person.}.''

As you can see, givers are more rare than takers \& matchers, \& have dramatically\footnote{\textbf{dramatically} [adv] \textbf{1.} in a very sudden or extreme way; to a very great degree; \textbf{2.} in a way that is exciting or impressive; \textbf{3.} using the style of a play in telling a story or giving an account of an event.} different performance\footnote{\textbf{performance} [n] \textbf{1.} [uncountable, countable] how well or badly you do something; how well or badly something works; \textbf{2.} [uncountable, singular] \textbf{performance of something} the action or process of performing a task or function; \textbf{3.} [countable] \textbf{performance (of something)} an act of presenting a play, concert or some other form of entertainment; \textbf{4.} [countable] an act of performing a song, a piece of music, or a role in a play or film.} results. While low-performing givers say yes to everything at the expense of their own work, which has a negative impact on their time management, project delivery, communication, \& execution in general, smart givers take into account what is best for the organization, not only what is best for the person asking for help. As a result, they are highly valued \& manage to both be helpful to their colleagues while positively impacting their organization.

In addition, givers may \href{https://nesslabs.com/how-to-build-a-support-group}{get more support} from fellow colleagues on their way up to success. ``There's something distinctive that happens when givers succeed: \fbox{it spreads \& cascades}. When takers win, there's usually someone else who loses. Research shows that people tend to envy successful takers \& look for ways to knock them down a notch. In contrast, when givers ($\ldots$) win, people are rooting for them \& supporting them, rather than gunning for them. Givers succeed in a way that creates a ripple effect, enhancing the success of people around them.''

In essence, successful givers generate win-win-win situations, where they succeed, their colleagues are elevated, \& the company performs better. Since givers can end up either at the lowest or the highest levels of performance, how can you make sure you are 1 of the most successful givers?

\subsection{How to Be A Smart Giver}
If your goal is moderate success, you can decide to act like a taker or a matcher. But if you want to be part of the top performing members of your organization, or to have a positive impact on the world \& foster win-win-win relationships with people around you, you may want to try to become a smart giver.
\begin{itemize}
	\item \textbf{Change your mindset.} Consider the lens through which you are viewing your job \& your relationships with friends \& family. For your professional context, ask yourself who exactly is affected by your work? How do your choices impact the experience of colleagues \textit{\&} customers? How can you align your decisions so when you win, everyone wins? Instead of being self-focused like a taker or transactional like a matcher, think of an expanding pie where everyone can benefit from your success.
	\item \textbf{Help wisely.} A problem low-performance givers face is the \href{https://nesslabs.com/focused-mind}{lack of focus} on the way they give. Tracking your impact does not mean you need to become a taker \& only help when it benefits you, nor that you need to become a matcher \& only help when you receive equal value in return. Rather, it means that you need to make sure you are helping achieve goals that are beneficial in general, not only to the person you are helping. Ask yourself: is this good for the company, for the customers, for the team? In a personal context, ask: is this good for our group of friends, our family, or our relationship in general? If the answer is no, try to brainstorm a better solution.
	\item \textbf{Track your impact.} From time to time, \href{https://nesslabs.com/weekly-review}{block some time for self-reflection} to look back at past times you have helped, \& what the outcome was. In the end, who benefitted from your help? Was it just 1 person, who may have been a taker? Or did your help have a wider positive impact, which justifies the time \& energy you spent to provide your support? If you feel like your impact wasn't as positive as you expected, try to think of the factors at play, \& how you can be wiser next time you are asked for help so your involvement can be as beneficial as possible.
\end{itemize}
These strategies can be helpful for anyone, but especially for low-performing givers who are spending too much time \& energy on providing scattered support which negatively impact their own work \& relationships. Wherever you are on the spectrum of reciprocity styles, remember that it is a choice: you can practice wise generosity to become a smart giver \& create a positive ripple effect around yourself.'' -- \href{https://nesslabs.com/author/annelaure}{Anne-Laure Le Cunff}


\textbf{Quick notes.} Dr. Who -- a disagreeable giver?

Peterson take \& giver.

%------------------------------------------------------------------------------%

\selectlanguage{english}
\chapter{\cite{Simon2010}. In Sheep's Clothing: Understanding \& Dealing with Manipulative People}

\begin{quotation}
	``[After reading \textit{In Sheep's Clothing}] I am beginning to reclaim\footnote{\textbf{reclaim} [v] \textbf{1.} to get something back or to ask to have it back after it has been lost, taken away, etc.; \textbf{2.} \textbf{reclaim something (from something)} to make land that is naturally too wet or too dry suitable to be built on, farmed, etc.; \textbf{3.} [usually passive] \textbf{reclaim something} if a piece of land is reclaimed by desert, forest, etc., it turns back into desert, etc. after being used for farming or building; \textbf{4.} \textbf{reclaim something (from something)} to obtain materials from waste products so that they can be used again; \textbf{5.} \textbf{reclaim somebody (from something)} to rescue somebody from a bad or criminal way of life.} my life, find my self-respect\footnote{\textbf{self-respect} [n] [uncountable] a feeling or pride in yourself that what you do, say, etc. is right \& good.} \& confidence\footnote{\textbf{confidence} [n] [uncountable] \textbf{1.} the feeling that you can trust, believe in \& be sure about the abilities or good qualities of somebody\texttt{/}something; \textbf{2.} a belief in your own ability to do things \& be successful; \textbf{3.} the feeling that you are certain about something; \textbf{4.} \textbf{(in) confidence} a feeling of trust that somebody will keep information private.}.'' -- Marc, Virginia\\
	
	``After having read several books on several different self-help topics, psychology books, psychiatry books, etc., \textsc{I must} recommend you buy this one, 1st. It cuts straight through the bs -- neatly\footnote{\textbf{neat} [a] (\textbf{neater, neatest}) \textbf{1.} in good order; carefully done or arranged; \textbf{2.} simple but clever; \textbf{3.} containing or made out of just 1 substance; not mixed with anything else.} \& cleanly. I have bought copies of this book for friends \& can't recommend it enough.'' -- E. Adams, Online Purchaser\\
	
	``Don't Be Bossed-Around Ever Again!!! $\ldots$ \textit{In Sheep's Clothing: Understanding \& Dealing with Manipulative People} by George K. Simon, Jr., Ph.D., is a godsend\footnote{\textbf{godsend} [n] [singular] something good that happens unexpected \& helps somebody\texttt{/}something when they need help.} to anyone who has ever questioned their own sanity\footnote{\textbf{sanity} [n] [uncountable] \textbf{1.} the state of having a healthy mind; \textbf{2.} the state of being sensible \& reasonable, \textsc{opposite}: \textbf{insanity}.} while in any kind of relationship with a controlling\footnote{\textbf{controlling} [a] [only before noun] having power over a company so that you are able to decide how it is run.} \& manipulative\footnote{\textbf{manipulative} [a] \textbf{1.} (\textit{disapproving}) skillful at controlling or influencing a person or situation, often in a dishonest way; \textbf{2.} connected with the ability to handle an object or part of the body skillfully.} person.'' -- The Aeolian Kid, Online Purchaser\\
	
	``Dr. Simon teaches the mechanics\footnote{\textbf{mechanics} [n] \textbf{1.} [uncountable] the science of  movement \& force; \textbf{2.} [plural] \textbf{mechanics of something} the way something works or is done.} of popular tactics\footnote{\textbf{tactic} [n] \textbf{1.} [countable, usually plural] the particular method you use to achieve something; \textbf{2.} (\textbf{tactics}) [plural] the art of moving soldiers \& military equipment around during a battle or war in order to use them in the most effective way.} used by manipulators\footnote{\textbf{manipulator} [n] (\textit{often disapproving}) a person who shows skill at influencing people or situations in order to get what they want.} \& how you can identify\footnote{\textbf{identify} [v] \textbf{1.} to find or discover somebody\texttt{/}something; \textbf{2.} to recognize somebody\texttt{/}something \& be able to say who or what they are; \textbf{3.} to make it possible to recognize who or what somebody\texttt{/}something is.} \& thwart\footnote{\textbf{thwart} [v] to prevent somebody from doing what they want to do, \textsc{synonym}: \textbf{frustrate}.} their attacks so that you control the outcome. This book helped me with a person that I have no choice but to see daily. After the end of every ``friendly'' conversation I felt depressed or insulted\footnote{\textbf{insult} [v] \textbf{insult somebody\texttt{/}something} to say or do something that offends somebody.} but could not figure out how this person was doing it. This book helped me to understand what was really happening. Dr. Simon's guidelines exposed\footnote{\textbf{expose} [v] \textbf{1.} \textbf{expose something} to show something that is usually hidden, \textsc{synonym}: \textbf{reveal}; \textbf{2.} to tell the true facts about a person or a situation, \& show them\texttt{/}it to be immoral, illegal, etc.; \textbf{3.} to allow light onto the film inside a camera when taking a photograph; \textbf{be exposed to something} [phrasal verb] to be in a place or situation where you are\texttt{/}it is not protected from something harmful or unpleasant; \textbf{expose somebody to something} [phrasal verb] to let somebody find out about something by giving them experience of it or showing them what it is like.}\,\footnote{\textbf{exposed} [a] \textbf{1.} (of a place) not protected from the weather by trees, buildings or high ground; \textbf{2.} (\textit{finance}) likely to experience financial losses; \textbf{3.} \textbf{exposed (to something)} (of a  person) not protected from attack or criticism.} this person \& [allowed me to take] control. Because this person knows [I] can no longer [be] controlled, I now have -- not a perfect relationship -- but one that's better than the alternative.'' -- A reader in Chicago\\
	
	``This book is like the secret decoder\footnote{\textbf{decoder} [n] a device that changes a message into a form that people can understand, e.g. by changing an electronic signal into sound \& pictures.} ring for the jumbled\footnote{\textbf{jumbled} [a] mixed together in a confused or untidy way.} mess that is a manipulator's modus operandi\footnote{\textbf{modus operandi} [n] [singular] (\textit{from Latin, formal}) (abbr., \textbf{MO}) a particular method of working.}. \textbf{Do yourself a favor \& get this book now.}'' -- Christy, Missouri\\
	
	``It's sad that there are people out there that make life so much harder than it should have to be for others. Being able to identify such people in your life (both at home \& at work) is very important \& can be of invaluable\footnote{\textbf{invaluable} [a] extremely useful, \textsc{synonym}: \textbf{valuable}.} help to 1) not go crazy oneself, \& 2) take corrective\footnote{\textbf{corrective}  [a] designed to improve or put right something that was wrong before; [n] \textbf{corrective (to something)} something that corrects something or that helps to give a more accurate or fairer view of somebody\texttt{/}something.} action. Dr. Simon's book is written with amazing clarity\footnote{\textbf{clarity} [n] [uncountable] \textbf{1.} the quality of being expressed clearly; \textbf{2.} the ability to think about or understand something clearly; \textbf{3.} if a picture, substance or sound has clarity, you can see or hear it very clearly, or see through it easily.}. \textbf{If you read only 1 book this year, read this one.}'' -- JA008, Online Purchaser\\
	
	``This is 1 of the best books I've ever read \& \textbf{I would recommend it to anyone.} It has redefined\footnote{\textbf{redefine} [v] to change the nature or limits of something; to make people consider something in a new way.} how I judge people \& helped me to become a stronger person. I used to be very naive \& unaware\footnote{\textbf{unaware} [a] [not before noun] not knowing or realizing that something is happening or that something exists.} of people's ulterior\footnote{\textbf{ulterior} [a] [only before noun] (of a reason for doing something) that somebody keeps hidden \& does not admit.} motives\footnote{\textbf{motive} [n] a reason for doing something.}, \& I have learned a tremendous\footnote{\textbf{tremendous} [a] (\textit{rather informal}) very great, \textsc{synonym}: \textbf{huge}.} amount from reading this book.'' -- S. Brescenti, Online Purchaser\\
	
	``This book makes it clear that evil is allowed free rein\footnote{\textbf{rein} [n] (\textbf{the reins}) [plural] \textbf{rein (of something)} the state of being in control or the leader of something; \textbf{give\texttt{/}allow somebody\texttt{/}something (a) free\texttt{/}full rein, give\texttt{/}allow (a) free\texttt{/}full rein to somebody\texttt{/}something} [idiom] to give somebody complete freedom of action; to allow something to be expressed freely.} because of our ignorance\footnote{\textbf{ignorance} [n] [uncountable] a lack of knowledge or information about something.} of its nature. Simon shows us what seemingly\footnote{\textbf{seemingly} [adv] in a way that appears to be true but may in fact not be, \textsc{synonym}: \textbf{apparently}.} mundane \footnote{\textbf{mundane} [a] (\textit{often disapproving}) not interesting or exciting, \textsc{synonym}: \textbf{dull, ordinary}.} interactions that leave us perplexed\footnote{\textbf{perplexed} [a] confused \& anxious because you are unable to understand something; showing this.} may really be about. According to him, master manipulators leave us drained \& confused\footnote{\textbf{confused} [a] \textbf{1.} unable to think clearly or to understand what is happening or what somebody is saying; \textbf{2.} not clear or easy to understand.} as we try to change them into the good person we want to believe they really are. I would add that \fbox{the manipulators are just plain evil} because evil requires lies, manipulation \& a weakening\footnote{\textbf{weaken} [v] \textbf{1.} [transitive, intransitive] \textbf{weaken (somebody\texttt{/}something)} to make somebody\texttt{/}something less strong or powerful; to become less strong or powerful, \textsc{opposite}: \textbf{strengthen}; \textbf{2.} [transitive, intransitive] \textbf{weaken (something)} to make something less physically strong; to become less physically strong, \textsc{opposite}: \textbf{strengthen}; \textbf{3.} [intransitive, transitive] to become less determined or certain about something; to make somebody less determined or certain, \textsc{opposite}: \textbf{strengthen}.} of the other's will through deception\footnote{\textbf{deception} [n] \textbf{1.} [uncountable] the act of deliberately making somebody believe something that is not true, \textsc{synonym}: \textbf{deceit}; \textbf{2.} [countable] something that you say or do that is intended to make somebody believe something that is not true.}. Simon shows you how to recognize\footnote{\textbf{recognize} [v] (\textit{British English also} \textbf{recognise}) (not used in the progressive tenses) \textbf{1.} \textbf{recognize somebody\texttt{/}something} to know who somebody is or what something is when you see or hear them, because you have seen or heard them before; \textbf{2.} to admit or to be aware that something exists or is true, \textsc{synonym}: \textbf{acknowledge}; \textbf{3.} to accept \& approve of somebody\texttt{/}something officially; \textbf{4.} to be thought of as very good or important by people in general; \textbf{5.} \textbf{recognized somebody\texttt{/}something} to give somebody official thanks for something that they have done or achieved.} the signs \& what you can do about it. Good people are responsible\footnote{\textbf{responsible} [a] \textbf{1.} [not before noun] having the job or duty of doing something or taking care of somebody\texttt{/}something, so that you may be blamed if something goes wrong; \textbf{2.} [not before noun] that can be blamed for something; \textbf{3.} [not before noun] \textbf{responsible (for something)} being the cause of something; \textbf{4.} \textbf{responsible to somebody\texttt{/}something} having to report to somebody\texttt{/}something with authority or in a higher position \& explain to them what you have done; \textbf{5.} (of people, actions or behavior) that can be trusted \& relied on, \textsc{opposite}: \textbf{irresponsible}; \textbf{6.} [usually before noun] (of a job or position) needing somebody who can be trusted \& relied on; involving important duties.} for informing\footnote{\textbf{inform} [v] \textbf{1.} [transitive, intransitive] to tell somebody about something, especially in an official way; \textbf{2.} [transitive] \textbf{inform something} to have an important influence on something; \textbf{inform on somebody} [phrasal verb] to give information to the police or somebody in authority about the illegal activities of somebody; \textbf{inform on something} [phrasal verb] to provide information about something.} \& protecting themselves from the manipulators in society. This book is a necessary start.'' -- Kaye, a reader in New York state\\
	
	``Pithy\footnote{\textbf{pithy} [a] (\textit{approving}) (\textbf{pithier, pithiest}) (of a comment, piece of writing, etc.) short but expressed well \& full of meaning.} \& often funny, George Simon takes the bluster\footnote{\textbf{bluster} [v] \textbf{1.} [transitive, intransitive] \textbf{bluster (something) $|$ $+$ speech} to talk in an aggressive or threatening way, but with little effect; \textbf{2.} [intransitive] (of the wind) to blow violently.} \& obfuscation\footnote{\textbf{obfuscation} [n] [uncountable, countable] (\textit{formal}) the act of making something less clear \& more difficult to understand, usually deliberately.} of overbearing\footnote{\textbf{overbearing} [a] (\textit{disapproving}) trying to control other people in an unpleasant way, \textsc{synonym}: \textbf{domineering}.}, weasely bosses, nasty\footnote{\textbf{nasty} [a] (\textbf{nastier, nastiest}) \textbf{1.} very bad or unpleasant; \textbf{2.} unkind; unpleasant, \textsc{synonym}: \textbf{mean}; \textbf{3.} dangerous or serious; \textbf{4.} offensive; in bad taste.} neighbors, \& obnoxious\footnote{\textbf{obnoxious} [a] extremely unpleasant, especially in a way that offends people, \textsc{synonym}: \textbf{offensive}.} coworkers \& boils\footnote{\textbf{boil} [v] [intransitive, transitive] when a liquid boils or when you boil it, it is heated to the point where it turns to steam or vapor; \textbf{boil something down (to\texttt{/}into something)} [phrasal verb] to make something, especially information, shorter by leaving out the parts that are not important; \textbf{boil down to something} [phrasal verb] (not used in the progressive tenses) (\textit{rather informal}) (of a situation, problem, etc.) to have something as a main or basic part.} it down to show you the simple psychological strategies being used to impose\footnote{\textbf{impose} [v] \textbf{1.} to introduce something such as a new law, tax or system; to order that a law or punishment be used; \textbf{2.} \textbf{impose something (on\texttt{/}upon somebody\texttt{/}something)} to make somebody accept or follow the same opinions or beliefs as your own; \textbf{3.} \textbf{impose something (on\texttt{/}upon somebody\texttt{/}something)} to give something that is difficult or unpleasant to somebody\texttt{/}something; \textbf{4.} \textbf{impose yourself (on\texttt{/}upon somebody\texttt{/}something)} to make somebody\texttt{/}something accept you or your ideas.} on your patience\footnote{\textbf{patience} [n] [uncountable] \textbf{1.} the ability to stay calm \& accept delay, problems or suffering without complaining; \textbf{2.} the ability to spend a lot of time doing something difficult that needs a lot of attention \& effort.}, good will, or even wallet. \textbf{I have recommended this book to everyone I know \& bought extra copies for my kids} when they went out into the work world. Highly Recommended!'' -- C. MacCallum, Online Purchaser
\end{quotation}

\section*{Preface}
``Whether it's the supervisor who claims to support you while thwarting every opportunity you have to get ahead, the co-worker who quietly undermines\footnote{\textbf{underline} [v] \textbf{1.} \textbf{undermine something} to make something, especially somebody's confidence or authority, gradually weaker or less effective; \textbf{2.} \textbf{undermine something} to make something weaker at the base, e.g. by digging under it.} you to gain the boss's favor, the spouse who professes\footnote{\textbf{profess} [v] (\textit{formal}) \textbf{1.} to claim that something is true or correct, especially when it is not; \textbf{2.} to state honestly that you have a particular belief or feeling, \textsc{synonym}: \textbf{declare}.} to love \& care about you but seems to control you life, or the child who always seems to know just which buttons to push in order to get their way, manipulative people are like the proverbial\footnote{\textbf{proverbial} [a] \textbf{1.} [only before noun] used to show that you are referring to a particular proverb or well-known phrase; \textbf{2.} [not usually before noun] well known \& talked about by a lot of people, \textsc{synonym}: \textbf{famous}.} wolf\footnote{\textbf{wolf} [n] (plural \textbf{wolves}) a large wild animal of the dog family, that lives \& hunts in groups; \textbf{a lone wolf} [idiom] a person who prefers to be alone; \textbf{a wolf in sheep's clothing} [idiom] a person who seems to be friendly or not likely to cause any harm but is really an enemy; [v] \textbf{wolf something (down)} to eat food very quickly, especially by putting a lot of it in your mouth at once, \textsc{synonym}: \textbf{gobble}.} in sheep's clothing. On the surface they can appear charming\footnote{\textbf{charming} [a] \textbf{1.} very pleasant or attractive; \textbf{2.} (\textit{ironic, informal}) used to show that you have a low opinion of somebody's behavior.} \& genial\footnote{\textbf{genial} [a] friendly \& cheerful, \textsc{synonym}: \textbf{affable}.}. But underneath\footnote{\textbf{underneath} [prep, adv] \textbf{1.} under or below something else, especially when it is hidden or covered by the thing on top; \textbf{2.} used to talk about somebody's real feelings or character, as opposed to the way they seem to be; [n] \textbf{the underneath} [singular] the lower surface or part of something.}, they can be ever so calculating \& ruthless\footnote{\textbf{ruthless} [a] (\textit{often disapproving}) (of people or their behavior) hard \& cruel; determined to get what you want \& not caring if you hurt other people.}. Cunning\footnote{\textbf{cunning} [a] \textbf{1.} (\textit{disapproving}) able to get what you want in a clever way, especially by tricking or cheating somebody, \textsc{synonym}: \textbf{crafty, wily}; \textbf{2.} clever \& showing skill, \textsc{synonym}: \textbf{ingenious}; [n] [uncountable] the ability to achieve something by tricking or cheating other people in a clever way, \textsc{synonym}: \textbf{craftiness}.} \& subtle\footnote{\textbf{subtle} [a] (\textbf{sutler, subtlest}) (\textbf{more subtle} is also common) \textbf{1.} (\textit{often approving}) (especially of a change or difference) not very obvious; not easy to notice; \textbf{2.} (of a person or their behavior) behaving in a clever way \& using indirect methods in order to achieve something; \textbf{3.} showing a good understanding of things that are not obvious to other people.}, they prey\footnote{\textbf{prey} [n] [uncountable, singular] an animal that is hunted, killed \& eaten by another, \textbf{be\texttt{/}fall prey to something} [idiom] to be harmed or affected by something bad; [v] \textbf{prey on\texttt{/}upon somebody\texttt{/}something} \textbf{1.} (of an animal or a bird) to hunt \& kill other animal for food; \textbf{2.} to harm or take advantage of somebody who is weaker than you.} on your weaknesses\footnote{\textbf{weakness} [n] \textbf{1.} [uncountable] \textbf{weakness (of somebody\texttt{/}something)} lack of strength, power or determination, \textsc{opposite}: \textbf{strength}; \textbf{2.} [countable] a weak point in an object, a system, somebody's character, etc., \textsc{opposite}: \textbf{strength}.} \& use clever\footnote{\textbf{clever} [a] (\textbf{cleverer, cleverest}) (you can also use \textbf{more clever} \& \textbf{most clever}) \textbf{1.} (\textit{especially British English}) quick at learning \& understanding things, \textsc{synonym}: \textbf{intelligent}; \textbf{2.} \textbf{clever (at something\texttt{/}doing something)} (\textit{especially British English}) skillful; \textbf{3.} showing intelligence or skill, e.g., in the design of an object, in an idea or somebody's actions.} tactics to gain advantage over you. They're the kind of people who fight hard for everything they want but do their best to conceal\footnote{\textbf{conceal} [v] to hide something.} their aggressive\footnote{\textbf{aggressive} [a] \textbf{1.} angry, \& behaving in a harmful or violent way; ready to attack; \textbf{2.} acting with force \& determination in order to succeed; \textbf{3.} (\textit{medical}) (of a disease or medical condition) quickly becoming more serious.} intentions\footnote{\textbf{intention} [n] [countable, uncountable] what you intend or plan to do; your aim.}. That's why I call them covert-aggressive\footnote{\textbf{covert} [a] secret or hidden, making it difficult to notice.} personalities\footnote{\textbf{personality} [n] (plural \textbf{personalities}) \textbf{1.} [countable, uncountable] the various aspects of a person's character that combine to make them different from other people; \textbf{2.} [uncountable] the qualities of a person's character that make them interesting \& attractive; \textbf{3.} [countable] a famous person, especially one who works in entertainment or sport, \textsc{synonym}: \textbf{celebrity}; \textbf{4.} [countable] a person whose strong character makes them easy to notice; \textbf{5.} [uncountable] the qualities of a place or thing that make it interesting \& different, \textsc{synonym}: \textbf{character}.}.

As a clinical psychologist in private practice, I began to focus on the problem of covert aggression\footnote{\textbf{aggression} [n] [uncountable] \textbf{1.} a violent attack or threats by 1 person or country against another person or by 1 country against another country; \textbf{2.} feelings of anger \& hate that may result in harmful or violent behavior.} over 20 years ago. I did so because the depression\footnote{\textbf{depression} [n] \textbf{1.} [uncountable, countable] a medical condition in which somebody feels very sad \& anxious \& often has physical symptoms such as being unable to sleep; \textbf{2.} [uncountable, countable] a period when there is little economic activity \& many people are poor or without jobs; \textbf{3.} [countable] a part of a surface that is lower than the parts around it, \textsc{synonym}: \textbf{hollow}; \textbf{4.} [countable] (\textit{specialist}) a weather condition in which the pressure of the air becomes lower, often causing rain; \textbf{5.} [uncountable, countable] \textbf{depression of something} the action of pressing something; the fact of something becoming lower.}, anxiety\footnote{\textbf{anxiety} [n] (plural \textbf{anxieties}) \textbf{1.} [uncountable] the state of feeling nervous that something bad is going to happen; a fear about something; \textbf{2.} [uncountable] \textbf{anxiety to do something} a strong feeling of wanting to do something or of wanting something to happen.}, \& feelings of insecurity\footnote{\textbf{insecurity} [n] (plural \textbf{insecurities}) [uncountable, countable] \textbf{1.} a feeling of lack of confidence, \textsc{opposite}: \textbf{security}; \textbf{2.} the fact that somebody\texttt{/}something is not safe or protected, \textsc{opposite}: \textbf{security}.} that initially led several of my patients to seek help eventually\footnote{\textbf{eventually} [adv] at the end of a period of time or a series of events. Use \textbf{finally} for the last in a list of things.} turned out to be in some way linked to their relationship with a manipulative person.

My 2nd objective is to explain precisely\footnote{\textbf{precisely} [adv] \textbf{1.} exactly; \textbf{2.} accurately; carefully; \textbf{3.} used to emphasize that something is very true or obvious; \textbf{more precisely} [idiom] used to show that you are giving more detailed \& accurate information about something you have just mentioned.} how covertly aggressive people manage to deceive\footnote{\textbf{deceive} [v] [transitive] \textbf{1.} \textbf{deceive somebody} to deliberately make somebody believe something that is not true; \textbf{2.} \textbf{deceive somebody\texttt{/}something} (of a thing) to make somebody have a false idea about somebody\texttt{/}something.}, manipulate\footnote{\textbf{manipulate} [v] \textbf{1.} \textbf{manipulate something} to change, correct or move text or data on a computer; \textbf{2.} \textbf{manipulate something} to change or present data in a way that deceives somebody; \textbf{3.} \textbf{manipulate something} to handle or control something in a skillful way; \textbf{4.} (\textit{disapproving}) to control or influence somebody\texttt{/}something, often in a dishonest way so that they do not realize it; \textbf{5.} \textbf{manipulate something} to examine or treat a part of the body by feeling or moving it with the hand.}, \& ``control'' others. Aggressive \& covertly aggressive people use a select group in interpersonal\footnote{\textbf{interpersonal} [a] [only before noun] connected with relationships between people.} maneuvers\footnote{\textbf{maneuver} (US) $=$ \textbf{manoeuvre} [n] \textbf{1.} [countable] a movement performed with care \& skill; \textbf{2.} [countable] a clever plan, action or movement that is used to give somebody an advantage; \textbf{3.} (\textbf{manoeuvres}) [plural] military exercises involving a large number of soldiers, ships, etc.; \textbf{freedom of\texttt{/}room for manoeuvre} [idiom] the change to change the way that something happens \& influence decisions that are made; [v] \textbf{1.} [intransitive, transitive] to move or turn skillfully or carefully; to move or turn something skillfully or carefully; \textbf{2.} [intransitive, transitive] to control or influence a situation or person in a skillful but sometimes dishonest way.}  or tactics to gain advantage over others. Becoming more familiar with these tactics really helps a person recognize manipulative behavior \textit{at the time it occurs}, \& makes it easier, therefore, to avoid being victimized\footnote{\textbf{victimize} [v] (\textit{British English also} \textbf{victimise}) [often passive] \textbf{victimize somebody} to make somebody suffer unfairly because you do not like them, their opinions or something that they have done.}. I'll also discuss the characteristics\footnote{\textbf{characteristic} [n] \textbf{characteristic (of something\texttt{/}somebody)} a typical feature or quality that something\texttt{/}somebody has; [a] very typical of something\texttt{/}somebody.} many of us possess that can make us unduly\footnote{\textbf{unduly} [adv] more than is reasonable or necessary; \textsc{synonym}: \textbf{excessively, unnecessarily}.} vulnerable\footnote{\textbf{vulnerable} [a] \textbf{vulnerable (to somebody\texttt{/}something)} weak \& easily hurt physically or emotionally.} to the tactics of manipulation. Knowing what aspects of your own character a manipulator is most likely to exploit\footnote{\textbf{exploit} [v] \textbf{1.} \textbf{exploit something} to use something well in order to gain as much from it as possible; \textbf{2.} to develop or use something for business or industry; \textbf{3.} \textbf{exploit somebody\texttt{/}something (for something)} (\textit{disapproving}) to treat a person or situation as an opportunity to gain an advantage for yourself; \textbf{4.} \textbf{exploit somebody} (\textit{disapproving}) to treat somebody unfairly by making them work \& not giving them much in return.} is another important step in avoiding victimization\footnote{\textbf{victimization} [n] (\textit{British English also} \textbf{victimisation}) [uncountable] the action of making somebody suffer unfairly because you do not like them, their opinions, or something that they have done.}.

My final objective is to outline\footnote{\textbf{outline} [v] \textbf{1.} to give a description of the main facts or points involved in something, \textsc{synonym}: \textbf{sketch}; \textbf{2.} [usually passive] \textbf{be outlined ($+$ adv.\texttt{/}prep.)} to show or mark the outer edge of something; [n] [countable, uncountable] \textbf{1.} a description of the main facts or points involved in something; \textbf{2.} the line that goes around the edge of something, showing its main shape but not the details.} the specific steps anyone can take to deal more effectively\footnote{\textbf{effectively} [adv] \textbf{1.} in a way that produces the intended result or a  successful result, \textsc{opposite}: \textbf{ineffectively}; \textbf{2.} used when you are saying what the facts of a situation are, \textsc{synonym}: \textbf{in effect}.} with aggressive \& covertly aggressive personalities. I'll present some general rules for redefining the rules of engagement\footnote{\textbf{engagement} [n] [uncountable] being involved with somebody\texttt{/}something in an attempt to understand them or achieve something.} with these kinds of individuals \& describe some specific tools of personal empowerment\footnote{\textbf{empowerment} [n] [uncountable] a positive feeling that you have some control over your life or the situation you are in.} that can help a person break the self-defeating\footnote{\textbf{self-defeating} [a] causing more problems \& difficulties instead of solving them; not achieving what you wanted to achieve but having an opposite effect.} cycle of trying to control their manipulator \& becoming depressed in the process. Using these tools makes it more likely that a 1-time victim\footnote{\textbf{victim} [n] \textbf{1.} a person who has been injured or killed as the result of a crime, disease, accident, etc.; \textbf{2.} a person, organization, etc. that has suffered because of a difficult situation, or because of the attitudes or actions of other people; \textbf{3.} an animal or person that is killed \& offered to a god; \textbf{fall victim (to something)} [idiom] to be injured, killed, damaged or destroyed by something.} will invest\footnote{\textbf{invest} [v] \textbf{1.} [intransitive, transitive] to put money into shares of a company, property or commercial project in the hope of making a profit; \textbf{2.} [intransitive, transitive] (of an organization or government, etc.) to spend money on something in order to make it better or more successful; \textbf{3.} [transitive] to spend time, energy or effort on something in the hope that it will be useful; \textbf{invest somebody\texttt{/}something with something} [phrasal verb] \textbf{1.} to make somebody\texttt{/}something seem to have a particular quality; \textbf{2.} to formally give a person or organization a rank, position or authority.} their energy\footnote{\textbf{energy} [n] \textbf{1.} [uncountable, countable] the ability of matter or radiation to perform work because of its mass, movement, electrical charge, etc.; \textbf{2.} [uncountable] a source of power that can be used by somebody\texttt{/}something, e.g. to provide light \& heat, or to work machines; \textbf{3.} [uncountable] the effort needed to do work or other physical or mental activities; \textbf{4.} (\textbf{energies}) [plural] the physical \& mental effort that you use to do something.} where they really have power -- in their own behavior\footnote{\textbf{behaviour} [n] (US \textbf{behavior}) \textbf{1.} [uncountable, countable] the way that somebody\texttt{/}something functions or reacts in a particular situation; \textbf{2.} [uncountable] the way that somebody behaves, especially towards other people.}. Knowing how to conduct yourself in a potentially\footnote{\textbf{potentially} [adv] possibly going to develop or be developed into something, especially something bad.} manipulative encounter\footnote{\textbf{encounter} [v] \textbf{1.} \textbf{encounter something} to experience something, especially something unpleasant or difficult, while you are trying to do something else, \textsc{synonym}: \textbf{run into something}; \textbf{2.} \textbf{encounter something\texttt{/}somebody} to discover or experience something, or meet somebody, especially something\texttt{/}somebody new, unusual or unexpected, \textsc{synonym}: \textbf{come across somebody\texttt{/}something}; [n] a meeting, especially one that is sudden or unexpected.} is crucial\footnote{\textbf{crucial} [a] extremely important, because it will affect other things, \textsc{synonym}: \textbf{critical, essential}.} to becoming less vulnerable to a manipulator's ploys\footnote{\textbf{ploy} [n] words or actions that are carefully planned to get an advantage over somebody else, \textsc{synonym}: \textbf{manoeuvre}.} \& asserting\footnote{\textbf{assert} [v] \textbf{1.} to state clearly \& firmly that something is true; \textbf{2.} to make other people recognize your right or authority to do something, by behaving firmly \& confidently; \textbf{3.} \textbf{assert yourself (as something)} to behave in a confident \& determined way so that other people pay attention to your opinions; \textbf{4.} \textbf{assert itself} to start to have an effect.} greater control over your own life.

I have attempted to write this book in a manner that is serious\footnote{\textbf{serious} [a] \textbf{1.} important \& worrying because of possible danger or risk; \textbf{2.} that must be treated as important \& thought about carefully; \textbf{3.} thinking about things in a careful \& sensible way; \textbf{4.} \textbf{serious about (doing) something} sincere about something.} \& substantial\footnote{\textbf{substantial} [a] large in amount, value or importance, \textsc{synonym}: \textbf{considerable}.} yet straightforward\footnote{\textbf{straightforward} [a] easy to do or to understand; not complicated, \textsc{synonym}: \textbf{easy}.} \& readily\footnote{\textbf{readily} [adv] \textbf{1.} quickly \& without difficulty, \textsc{synonym}: \textbf{freely}; \textbf{2.} in a way that shows that you do not object to something.} understandable\footnote{\textbf{understandable} [a] \textbf{1.} (of behavior, feelings or reactions) seeming normal \& reasonable in a particular situation, \textsc{synonym}: \textbf{natural}; \textbf{2.} (of languages,  documents, etc.) easy to understand, \textsc{synonym}: \textbf{comprehensible}.}. I have written it for the general public as well as the mental health professional, \& I hope both will find it useful. By adhering\footnote{\textbf{adhere} [v] [intransitive] \textbf{adhere (to something)} to stick firmly to something; \textbf{adhere to something} [phrasal verb] \textbf{1.} to believe in \& follow a particular principle or practice; \textbf{2.} to closely follow or represent a particular view.} to many traditional\footnote{\textbf{traditional} [a] \textbf{1.} following older methods \& ideas rather than modern or different ones, \textsc{synonym}: \textbf{conventional}; \textbf{2.} being part of the beliefs, customs or way of lief that have existed for a long time among a particular group of people.} assumptions\footnote{\textbf{assumption} [n] \textbf{1.} [countable] a belief or feeling that something is true or that something will happen, although there is no proof; \textbf{2.} [uncountable, singular] \textbf{assumption of something} the act of taking or beginning to have power or responsibility.}, labeling schemes\footnote{\textbf{scheme} [n] \textbf{1.} (\textit{British English}) a plan or system for doing or organizing something; \textbf{2.} a system for organizing information; \textbf{the ($\ldots$) scheme of things} [idiom] the way things seem to be organized in the world, in which everything has a place that relates to the general system.}, \& intervention\footnote{\textbf{intervention} [n] [uncountable, countable] \textbf{1.} the action of getting involved in a situation in order to improve it or stop it from gettin worse; \textbf{2.} action by a country to become involved in the affairs of another country when they have not been asked to do so; \textbf{3.} action taken to improve a medical condition or illness.} strategies\footnote{\textbf{strategy} [n] (plural \textbf{strategies}) \textbf{1.} [countable] a plan that is intended to achieve a particular purpose. In ecology, \textbf{strategies} are ways that have evolved ($=$ developed) in plants \& animals that enable them to survive \& be successful in their environment.; \textbf{2.} [uncountable] the process of planning something or putting a plan into operation in a skillful way; \textbf{3.} [uncountable, countable] the skill of planning the movements of armies in a battle or war; an example of doing this.}, therapists\footnote{\textbf{therapist} [n] \textbf{1.} (especially in compounds) a specialist who treats a particular type of illness or problem, or who uses a particular type of treatment; \textbf{2.} $=$ \textbf{psychotherapist}.} sometimes hold \& inadvertently\footnote{\textbf{inadvertently} [adv] by accident; without intending to, \textsc{synonym}: \textbf{unintentionally}.} reinforce\footnote{\textbf{reinforce} [v] \textbf{1.} \textbf{reinforce something} to make a feeling, idea, habit or tendency stronger; \textbf{2.} \textbf{reinforce something} to make a structure or material stronger, especially by adding another material to it; \textbf{3.} \textbf{reinforce something} to send more people or equipment in order to make an army, etc. stronger.} some of the same misconceptions\footnote{\textbf{misconception} [n] [countable, uncountable] a belief or an idea that is not based on correct information, or that is not understood by people.} that their patients harbor\footnote{\textbf{harbour} [n] (\textit{US English} \textbf{harbor}) [countable, uncountable] an area of water on the coast, protected from the open sea by strong walls, where ships can shelter; [v] \textbf{1.} \textbf{harbor somebody} to hide \& protect somebody who is hiding from the police; \textbf{2.} \textbf{harbor something} to keep feelings or thoughts, especially negative ones, in your mind for a long time; \textbf{3.} \textbf{harbor something} to contain something \& allow it to develop.} about the character \& behavior of manipulators that inevitably\footnote{\textbf{inevitably} [adv] as is certain to happen.} lead to continued victimization. I offer a new perspective\footnote{\textbf{perspective} [n] \textbf{1.} [countable] a particular attitude towards something; a way of thinking about something, \textsc{synonym}: \textbf{viewpoint}; \textbf{2.} [uncountable] the ability to think about problems \& decisions in a reasonable way without making them seem more serious or more important than they really are; \textbf{3.} [uncountable] the art of creating an effect of depth \& distance in a picture by representing people \& things that are far away as being smaller than those that are nearer the front.} in the hope of helping individuals \& therapists alike\footnote{\textbf{alike} [adv] \textbf{1.} in a very similar way; \textbf{2.} used after you have referred to 2 people or groups, to mean `both' or `equally'; [a] [not before noun] very similar.} avoid \textit{enabling} manipulative behavior.'' -- \cite[Preface]{Simon2010}

\section*{Author's Note on the Revised Edition}
``Since this book's 1st wide publication in 1996, I have received literally\footnote{\textbf{literally} [adv] \textbf{1.} in a literal way, \textsc{synonym}: \textbf{exactly}; \textbf{2.} used to emphasize the truth of something that may seem surprising. In more informal contexts, you may see \textbf{literally} used to emphasize a word or phrase, even though it is not actually true. This use is common, but is not considered correct in more formal \& academic contexts.} hundreds of calls, letters, \& emails, \& heard countless testimonials\footnote{\textbf{testimonial} [n] \textbf{1.} a formal written statement, often by a former employer, about somebody's abilities, qualities \& characters; a formal written statement about the quality of something; \textbf{2.} a thing that you give or do to show that you admire \& appreciate somebody.} \& comments at workshops\footnote{\textbf{workshop} [n] \textbf{1.} a period of discussion \& practical work on a particular subject, in which a group of people share their knowledge \& experience; \textbf{2.} a room or building in which things are made or repaired using tools or machinery.} from individuals whose lives were changed merely\footnote{\textbf{merely} [adv] used meaning `only' or `simply' to emphasize a fact or something that you are saying.} by being exposed to \& adopting\footnote{\textbf{adopt} [v] \textbf{1.} [transitive] \textbf{adopt something} to start to use a particular method or to show a particular attitude towards somebody\texttt{/}something; \textbf{2.} [transitive] \textbf{adopt something} to formally accept a suggestion or policy by voting; \textbf{3.} [transitive, intransitive] \textbf{adopt (somebody)} to take somebody else's child into your family \& become its legal parent(s); \textbf{4.} [transitive] \textbf{adopt something} to choose a new name or custom \& begin to use it as your own; to choose \& move to a country as your permanent home; \textbf{5.} [transitive] \textbf{adopt something} to use a particular manner or way of  speaking.} a new perspective on understanding human behavior. A common theme\footnote{\textbf{theme} [n] the subject of a talk, piece of writing, exhibition, etc.; an idea that keeps returning in a piece of researcher or a work of art or literature.} voiced by readers \& workshop attendees\footnote{\textbf{attendee} [n] a person who attends a meeting, etc.} is that once they dispelled\footnote{\textbf{dispel} [v] \textbf{dispel something} to get rid of a false belief or bad feeling.} old myths\footnote{\textbf{myth} [n] [countable, uncountable] \textbf{1.} a story from ancient times, especially one that was told to explain natural events or to describe the early history of a people; this type of story; \textsc{synonym}: \textbf{legend}; \textbf{2.} something that many people believe but that does not exist or is false, \textsc{synonym}: \textbf{fallacy}.} \& came to view problem behaviors in a different light, they could see clearly that what their intuition\footnote{\textbf{intuition} [n] \textbf{1.} [uncountable] the ability to know something by using your feelings rather than considering the facts; \textbf{2.} [countable] an idea or a strong feeling that something is true although it is not proved.} had told them all along was correct, \& thus felt validated\footnote{\textbf{validate} [v] \textbf{1.} \textbf{validate something} to prove that something is true or accurate; \textbf{2.} \textbf{validate something} to support or show the value of something.}. A similar phenomenon was held true for mental health professionals attending the many training seminars\footnote{\textbf{seminar} [n] \textbf{1.} a class at a university or college when a small group of students \& a teacher discuss or study a particular topic; \textbf{2.} a meeting for discussion or training.} I have given. Once they abandoned\footnote{\textbf{abandon} [v] \textbf{1.} to stop doing something, especially before it is finished; to stop planning to do something; \textbf{2.} to stop believing in something or supporting a party, cause, etc.; \textbf{3.} to leave somebody, especially somebody you are responsible fr, with no intention of returning; \textbf{4.} \textbf{abandon something} to leave a place or thing with no intention of returning, especially because of danger or economic problems.} their old notions\footnote{\textbf{notion} [n] an idea, a belief or an understanding of something.} about why their clients\footnote{\textbf{client} [n] \textbf{1.} a person who uses the services or advice of a professional person or organization; \textbf{2.} (\textit{computing}) a computer that is linked to a serve.} do the things they do, they were better able to help them \& their significant\footnote{\textbf{significant} [a] \textbf{1.} large or important enough to have an effect or to be noticed, \textsc{opposite}: \textbf{insignificant}; \textbf{2.} having a particular meaning; \textbf{3.} (\textit{statistics}) having statistical significance, \textsc{opposite}: \textbf{insignificant}.} others. I had already been doing workshops for 10 years before writing \textit{In Sheep's Clothing}. At that time, only a handful\footnote{\textbf{handful} [n] \textbf{1.} [singular] \textbf{handful (of somebody\texttt{/}something)} a small number of people or things; \textbf{2.} [countable] \textbf{handful (of something)} the amount of something that can be held in 1 hand.} of theorists\footnote{\textbf{theorist} [n] (also less frequent \textbf{theoretician}) a person who develops ideas about a particular subject in order to explain why things happen or exist.}, researchers, \& writers were recognizing the need for a new perspective on understanding \& dealing with disturbed\footnote{\textbf{disturbed} [a] \textbf{1.} having or caused by emotional \& mental problems; \textbf{2.} having had the normal arrangement or functioning changed; \textbf{3.} very upset about something that you disagree with or that has made you unhappy.} characters (e.g., Stanton Samenow, Samuel Yochelson, Robert Hare). What professionals today call the \textit{cognitive-behavioral}\footnote{\textbf{cognitive} [a] [usually before noun] (\textit{psychology}) connected with the mental processes of understanding.}\,\footnote{\textbf{cognitive behavioural therapy} [n] (\textit{US English} \textbf{cognitive behavioral therapy}) [uncountable] (abbr., \textbf{CBT}) a type of psychotherapy in which you are encouraged to change negative ways of thinking about yourself \& the world in order to change behavior patterns or treat conditions such as depression.} approach was in its infancy\footnote{\textbf{infancy} [n] [uncountable] \textbf{1.} the time when a child is a baby or very young; \textbf{2.} the early development of something.}. The early research on character disturbance\footnote{\textbf{disturbance} [n] \textbf{1.} [uncountable, countable, usually singular] the act of moving something; a change in the normal arrangement or state that something is in; \textbf{2.} [uncountable] \textbf{disturbance (to somebody\texttt{/}something)} the act of interrupting a peaceful situation; \textbf{3.} [uncountable, countable] a state in which somebody's mind or a function of the body is not working normally; \textbf{4.} [countable] a situation in which people behave violently in a public place.} inspired\footnote{\textbf{inspire} [v] \textbf{1.} to make somebody feel confident \& excited about doing something; \textbf{2.} [usually passive] to give somebody the idea for something; to be the reason why somebody does something; \textbf{3.} to make somebody have a particular feeling or emotion.} me \& helped me validate\footnote{\textbf{validate} [v] \textbf{1.} \textbf{validate something} to prove that something is true or accurate, \textsc{opposite}: \textbf{invalidate}; \textbf{2.} \textbf{validate something} to support or show the value of something.} my own observations\footnote{\textbf{observation} [n] \textbf{1.} [uncountable, countable] the act of watching somebody\texttt{/}something carefully for a period of time, especially to learn something; \textbf{2.} [uncountable] the ability to notice things, especially important details; \textbf{3.} [countable] \textbf{observation (about\texttt{/}on something)} a comment, especially based on something you have seen, heard or read, \textsc{synonym}: \textbf{remark}.}. Today an increasing number of professionals are recognizing the problem of character disturbance \& using cognitive-behavioral methods to diagnose\footnote{\textbf{diagnose} [v] [often passive] to say exactly what an illness or the cause of a problem is.} \& treat it.

We live in an age radically\footnote{\textbf{radically} [adv] completely; to a very great extent.} different from that in which the classical theories of psychology \& personality were developed. For the most part, truly\footnote{\textbf{truly} [adv] \textbf{1.} used to emphasize a particular quality; \textbf{2.} to the fullest degree; in the most complete way; \textbf{3.} in a way that is honest or genuine; \textbf{4.} used to emphasize that a particular description is accurate or correct.} pathological\footnote{\textbf{pathological} [a] \textbf{1.} (\textit{medical}) caused by, or connected with, disease or illness; \textbf{2.} (\textit{medical}) connected with the scientific study of the causes \& effects of diseases; \textbf{3.} not reasonable or sensible; impossible to control.} degrees of neurosis\footnote{\textbf{neurosis} [n] [countable, uncountable] (plural \textbf{neuroses}) \textbf{1.} (\textit{psychology}) a mental health condition in which a person has strong feelings of fear or worry; \textbf{2.} any strong fear or worry, \textsc{synonym}: \textbf{anxiety}.} are quite rare, \& problematic\footnote{\textbf{problematic} [a] (also less frequent \textbf{problematical}) difficult to deal with or understand; full of problems; not certain to be successful.} levels of character disturbance are increasingly\footnote{\textbf{increasingly} [adv] more \& more all the time.} commonplace\footnote{\textbf{commonplace} [a] done very often, or existing in many places, \& therefore not unusual; [n] \textbf{1.} [usually singular] an event, etc. that happens very often \& is not unusual; \textbf{2.} a remark, etc. that is not new or interesting.}. It's a pervasive\footnote{\textbf{pervasive} [a] (especially of something bad) spreading through all parts of a place or thing.} societal\footnote{\textbf{societal} [a] [only before noun] connected with society \& the way it is organized.} problem about which all of us would do well to expand our awareness\footnote{\textbf{awareness} [n] [uncountable, singular] \textbf{1.} the fact of knowing that something is true or exists; \textbf{2.} concern or interest in a particular situation or development.}. During the last 15 years, my experience working with disturbed characters of all types has grown immensely\footnote{\textbf{immensely} [adv] extremely; very much, \textsc{synonym}: \textbf{enormously}.}, as has the body of research. So, I have included in this edition an expanded discussion on the problem of character disturbance in general \& what sets the disturbed character apart from your garden-variety\footnote{\textbf{garden-variety} [a] (\textit{North American English}) (\textit{British English} \textbf{common or garden}) [only before noun] ordinary; with no special features.} neurotic\footnote{\textbf{neurotic} [a] \textbf{1.} (\textit{psychology}) caused by or having neurosis ($=$ a mental illness in which a person has strong feelings of fear \& worry); \textbf{2.} not behaving in a reasonable, calm way, because you are worried about something.}.

I am deeply grateful for the excellent word-of-mouth\footnote{\textbf{word of mouth} [idiom] the process by which people hear about something because they are told about it by other people \& not because they read about it or watch it on television.} support responsible for transforming a once small, independent work into a best seller  enjoying ever-increasing popularity even after 15 years. I sincerely\footnote{\textbf{sincerely} [adv] in a way that shows what you really feel or think about somebody\texttt{/}something.} hope this revised edition will provide you with all the information \& tools you need to better understand \& deal with the manipulative people in your life.'' \textsc{George K. Simon, Jr.}, Ph.D., Jan 2010 -- \cite[Author's note on the revised edition]{Simon2010}

\begin{center}\huge
	Part I: Understanding Manipulative Personalities
\end{center}

\section*{Introduction: Convert-Aggression: The Heart of Manipulation}

\subsection*{A Common Problem}
``Perhaps the following scenarios\footnote{\textbf{scenario} [n] (plural \textbf{scenarios}) \textbf{1.} a description of a possible series of events or situations; \textbf{2.} a description of what takes place in a film, play, novel, etc.} will sound familiar\footnote{\textbf{familiar} [a] \textbf{1.} \textbf{familiar with something} knowing something well, \textsc{opposite}: \textbf{unfamiliar}}. A wife tries to sort out her mixed feelings. She's mad at her husband for insisting their daughter make all A's. But she doubts she has the right to be mad. When she suggested that given her appraisal\footnote{\textbf{appraisal} [n] [countable, uncountable] \textbf{1.} \textbf{appraisal (of something)} a judgment of the value, performance or nature of somebody\texttt{/}something; \textbf{2.} (\textit{British English}) a meeting in which an employee discusses with their manager how well they have been doing their job; the system of holding such meetings.} of their daughter's abilities, he might be making unreasonable demands, his comeback\footnote{\textbf{comeback} [n] \textbf{1.} [usually singular] if a person in public life makes a comeback, they start doing something again that they had stopped doing, or they become popular again; \textbf{2.} if a thing makes a comeback, it becomes popular \& fashionable or successful again; \textbf{3.} \textbf{3.} (\textit{informal}) a quick reply to a critical remark, \textsc{synonym}: \textbf{retort}; \textbf{4.} a way of holding somebody responsible for something wrong that has been done to you.}, ``Shouldn't \textit{any} good parent want their child to do well \& succeed in life?'' made her feel like the insensitive\footnote{\textbf{insensitive} [a] \textbf{1.} not realizing or caring how other people feel, \& therefore likely to hurt or offend them, \textsc{opposite}: \textbf{sensitive}; \textbf{2.} not able to react to or feel something, \textsc{opposite}: \textbf{sensitive}.} one. In fact, whenever she confronts\footnote{\textbf{confront} [v] \textbf{1.} (of problems of a difficult situation) to appear \& need to be dealt with by somebody, \textsc{synonym}: \textbf{by}; \textbf{2.} \textbf{confront something} to deal with a problem or difficult situation, \textsc{synonym}: \textbf{face up to something}; \textbf{3.} \textbf{confront somebody} to face somebody so that they cannot avoid seeing \& hearing you, especially in an unfriendly or dangerous situation; \textbf{4.} \textbf{confront somebody with somebody\texttt{/}something} to make somebody face or deal with an unpleasant or difficult person or situation.} him, she somehow ends up feeling like the bad guy herself. When she suggested there might be more to her daughter's recent problems, \& that the family might do well to seek counseling\footnote{\textbf{counselling} [n] (\textit{US} \textbf{counseling}) [uncountable] professional advice about a problem, especially a personal problem.}, his retort\footnote{\textbf{retort} [v] to reply quickly to a comment, in an angry, offended or humorous way; [n] \textbf{1.} a quick, angry or humorous reply, \textsc{synonym}: \textbf{rejoinder, riposte}; \textbf{2.} a closed bottle with a long narrow bent spout that is used in a laboratory for heating chemicals.} ``Are you saying I'm psychiatrically disturbed\footnote{\textbf{disturbed} [a] \textbf{1.} having or caused by emotional \& mental problems; \textbf{2.} having had the normal arrangement or functioning changed; \textbf{3.} very upset about something that you disagree with or that has made you unhappy.}?'' made her feel guilty\footnote{\textbf{guilty} [a] (\textbf{guiltier, guiltiest}) (\textbf{more guilty} \& \textbf{most guilty} are more frequent.) \textbf{1.} having done something illegal, \textsc{opposite}: \textbf{innocent}; \textbf{2.} \textbf{guilty of (doing) something} having done something wrong, \textsc{opposite}: \textbf{innocent}; \textbf{3.} feeling ashamed because you have done something that you know is wrong or have not done something that you should have done.} for bringing up the issue. She often tries to assert\footnote{\textbf{assert} [v] \textbf{1.} to state clearly \& firmly that something is true; \textbf{2.} to make other people recognize your right or authority to do something, by behaving firmly \& confidently; \textbf{3.} \textbf{assert yourself (as something)} to behave in a confident \& determined way so that other people pay attention to your opinions; \textbf{4.} \textbf{assert itself} to start to have an effect.} her point of view, but always ends up giving-in to his. Sometimes, she thinks the problem is him, believing him to be selfish\footnote{\textbf{selfish} [a] caring only about yourself rather than about other people.}, demanding\footnote{\textbf{demanding} [a] \textbf{1.} (of a task) needing a lot of skill, care or effort; \textbf{2.} (of a person) expecting a lot of work or attention from others; not easily satisfied.}, intimidating\footnote{\textbf{intimidating} [a] frightening in a way that makes a person feel less confident.}, \& controlling\footnote{\textbf{controlling} [a] [only before noun] having power over a company so that you are able to decide how it is run.}. But this is a loyal\footnote{\textbf{loyal} [a] \textbf{loyal (to somebody\texttt{/}something)} not changing in your beliefs or support of somebody\texttt{/}something, \textsc{synonym}: \textbf{true}.} husband, good provider, \& a respected member of the community\footnote{\textbf{community} [n] (plural \textbf{communities}) \textbf{1.} (often \textbf{the community}) [singular] all the people who live in a particular area, country, etc. when considered as a group; \textbf{2.} [countable] (used in compounds) a group of people who share the same religion, race, job, etc.; \textbf{3.} [uncountable] (\textit{approving}) the feeling of sharing things \& belonging to a group in the place where you live; \textbf{4.} [countable] (\textit{biology}) a group of plants \& animals growing or living in the same place or environment; \textbf{the global\texttt{/}international community} [idiom] the countries of the world, considered as a group.}. By all rights she shouldn't resent\footnote{\textbf{resent} [v] to feel angry about something, especially because you feel it is unfair.} him. Yet, she does. So, she constantly\footnote{\textbf{constantly} [adv] all the time.} wonders\footnote{\textbf{wonder} [v] [transitive, intransitive] to think about something \& try to decide what is true, what will happen, what you should do, etc.; [n] \textbf{1.} [uncountable] a feeling of surprise \& admiration that you have when you see or experience something beautiful, unusual or unexpected; \textbf{2.} [countable] \textbf{wonder (of something)} something that fills you with surprise \& admiration; \textbf{(it is) no\texttt{/}little\texttt{/}small wonder (that) $\ldots$} [idiom] it is not surprising.} if there isn't something wrong with her.

A mother tries desperately\footnote{\textbf{desperate} [a] \textbf{1.} feeling or showing that you have little hope \& are ready to do anything without worrying about danger to yourself or others; \textbf{2.} [usually before noun] (of an action) giving little hope of success; tried when everything else has failed; \textbf{3.} (of a situation) extremely serious or dangerous.} to understand her daughter's behavior. No young girl, she thought, would threaten\footnote{\textbf{threaten} [v] \textbf{1.} [transitive] to say that you will cause trouble, hurt somebody, etc. if you do not get what you want; \textbf{2.} [transitive] to be a danger to something; to be likely to harm something, \textsc{synonym}: \textbf{endanger}; \textbf{3.} [intransitive] to seem likely to happen or cause something unpleasant.} to leave home, say things like ``Everybody hates me'' \& ``I wish I were never born,'' unless she were very insecure\footnote{\textbf{insecure} [a] \textbf{1.} not confident, especially about yourself or your abilities, \textsc{opposite}: \textbf{secure}; \textbf{2.} not safe or protected, \textsc{opposite}: \textbf{secure}.}, afraid\footnote{\textbf{afraid} [a] [not before noun] \textbf{1.} worried about what might happen; unwilling to do something because of this; \textbf{2.} feeling fear; frightened because you think that you might be hurt or suffer; \textbf{3.} \textbf{afraid for somebody\texttt{/}something} worried or frightened that something unpleasant, dangerous, etc. will happen to a particular person or thing.}, \& probably depressed. Part of her thinks her daughter is still the same child who used to hold her breath until she turned blue or threw tantrums\footnote{\textbf{tantrum} [n] a sudden short period of angry, unreasonable behavior, especially in a child.} whenever she didn't get her way. After all, it seems she only says \& does these things when she's about to be disciplined or she's trying to get something she wants. But a part of her is afraid to believe that. ``What if she really believes what she's saying?'' she wonders. ``What if I've really done something to hurt her \& I just don't realize it?'' she worries. She hates to feel ``bullied'' by her daughter's threats \& emotional displays, but she can't take the chance her daughter might really be hurting -- can she? Besides, children just don't act  this way unless they really feel insecure or threatened in some way underneath it all -- do they?''

\subsection*{The Heart of the Problem}
``Neither victim\footnote{\textbf{victim} [n] \textbf{1.} a person who has been injured or killed as the result of a crime, disease, accident, etc.; \textbf{2.} a person, organization, etc. that has suffered because of a difficult situation, or because of the attitudes or actions of other people; \textbf{3.} an animal or person that is killed \& offered to a god; \textbf{fall victim (to something)} [idiom] to be injured, killed, damaged or destroyed by something.} in the preceding scenarios trusted their ``gut\footnote{\textbf{gut} [n] \textbf{1.} [countable] the tube in the body through which food passes when it leaves the stomach, \textsc{synonym}: \textbf{intensive}; \textbf{2.} (\textbf{guts}) [plural] \textbf{gut (of something)} the organs in \& around the stomach, especially in an animal.}'' feelings\footnote{\textbf{feeling} [n] \textbf{1.} [countable] \textbf{feeling (of something)} something that you feel through the mind or the senses; \textbf{2.} [singular] the idea or belief that a particular thing is true or a particular situation is likely to happen; \textbf{3.} [countable, uncountable] an attitude or opinion about something; \textbf{4.} (\textbf{feelings}) [plural] a person's emotions rather than their thoughts or ideas; \textbf{5.} [uncountable] \textbf{feeling for somebody\texttt{/}something} the ability to understand somebody\texttt{/}something or to do something in a sensitive way; \textbf{6.} [plural, uncountable] \textbf{feeling (for somebody\texttt{/}something)} sympathy or love for somebody\texttt{/}something; \textbf{7.} [uncountable] strong emotion; \textbf{8.} [uncountable] the ability to feel physically, \textsc{synonym}: \textbf{sensation}; \textbf{9.} [singular] \textbf{feeling (of something)} the atmosphere of a place or situation.}. Unconsciously\footnote{\textbf{unconsciously} [adv] without being aware.}, they felt on the defensive\footnote{\textbf{defensive} [a] \textbf{1.} used or intended to protect somebody\texttt{/}something against attack; \textbf{2.} behaving in a way that shows that you feel that people are criticizing you in an unfair way.}, but consciously\footnote{\textbf{consciously} [adv] \textbf{1.} if somebody does something consciously, they are aware of doing it, \textsc{opposite}: \textbf{unconsciously}; \textbf{2.} deliberately.} they had trouble seeing their manipulator as merely a person on the offensive\footnote{\textbf{offensive} [a] \textbf{1.} rude in a way that causes you to feel upset; \textbf{2.} [only before noun] connected with the act of attacking somebody\texttt{/}something; \textbf{3.} extremely unpleasant; [n] \textbf{1.} a military operation in which large numbers of soldiers, etc. attack another country; \textbf{2.} a series of actions aimed at achieving something in a way that attracts a lot of attention.}. On 1 hand, they felt like the other person was trying to get the better of them. On the other, they found no objective\footnote{\textbf{objective} [n] \textbf{1.} something that you are trying to achieve, \textsc{synonym}: \textbf{goal, target}; \textbf{2.} (also \textbf{objective lens} specialist) the lens in a telescope or microscope that is nearest to the object being looked at; [a] \textbf{1.} not influenced by personal feelings or opinions; considering only facts, \textsc{synonym}: \textbf{impartial, unbiased}; \textbf{2.} (\textit{philosophy}) existing outside the mind; based on facts that can be proved.} evidence\footnote{\textbf{evidence} [n] \textbf{1.} [uncountable, countable] the facts, signs or objects that make you believe that something is true; \textbf{2.} [uncountable] the information that is used in court to try to prove something; \textbf{(be) in evidence} [idiom] present \& clearly seen; [v] [usually passive] to prove or show something; to be evidence of something.} at the time to back-up their gut-level hunch\footnote{\textbf{hunch} [v] [intransitive, transitive] to bend the top part of your body forward \& raise your shoulders \& back; [n] a feeling that something is true even though you do not have any evidence to prove it.}. They \fbox{ended up feeling crazy}.

They're not crazy. The fact is, \fbox{people fight almost all the time}. \& manipulative people are expert\footnote{\textbf{expert} [n] a person with special knowledge, skill or training in something; [a] \textbf{1.} done or provided by somebody with special knowledge or skill in a particular area; \textbf{2.} having special knowledge, skill or training in something.} at fighting in subtle\footnote{\textbf{subtle} [a] (\textbf{subtler, subtlest}) (\textbf{more subtle} is also common.) \textbf{1.} (\textit{often approving}) (especially of a change or difference) not very obvious; not easy to notice; \textbf{2.} (of a person or their behavior) behaving in a clever way \& using indirect methods in order to achieve something; \textbf{3.} showing a good understanding of things that are not obvious to other people.} \& almost undetectable\footnote{\textbf{undetectable} [a] impossible to see or find.} ways. Most of the time, when they're trying to take advantage or gain the upper hand, you don't even know you're in a fight until you're well on your way to losing. When you're being manipulated, chances are someone is fighting with you for position, advantage, or gain, but in a way that's difficult to readily\footnote{\textbf{readily} [adv] \textbf{1.} quickly \& without difficulty, \textsc{synonym}: \textbf{freely}; \textbf{2.} in a way that shows that you do not object to something, \textsc{synonym}: \textbf{willingly}.} see. Covert-aggression is at the heart of most manipulation.''

\subsection*{The Nature of Human Aggression}
``Our instinct to fight is a close cousin of our survival\footnote{\textbf{survival} [n] \textbf{1.} [uncountable] the state of continuing to live or exist, often despite difficulty or danger; \textbf{2.} [countable] \textbf{survival (of something)} something that has continued to exist from an earlier time, \textsc{synonym}: \textbf{relic}; \textbf{the survival of the fittest} [idiom] the idea that only the people or things that are best adapted to their surroundings will continue to exist.} instinct\footnote{\textbf{instinct} [n] [uncountable, countable] a natural tendency for people \& animals to behave in a particular way, using the knowledge \& abilities that they were born with rather than thought or training.}.\footnote{Storr, A., \textit{Human Destructiveness}, (Ballantine, 1991), pp. 7--17.} Most everyone ``fights'' to survive\footnote{\textbf{survive} [v] \textbf{1.} [intransitive] to continue to live or exist; \textbf{2.} [transitive] to continue to live or exist despite a dangerous event or time; \textbf{3.} [transitive] \textbf{survive somebody\texttt{/}something} to live or exist longer than somebody\texttt{/}something.} \& prosper\footnote{\textbf{prosper} [v] [intransitive] to develop in a successful way; to be successful, especially in making money, \textsc{synonym}: \textbf{thrive}.}, \& \textit{most} of the fighting we do is neither physically\footnote{\textbf{physically} [adv] \textbf{1.} in a way that is connected with a person's body rather than their mind; \textbf{2.} in a way that is connected with things that actually exist or are present \& can be seen, felt, etc. rather than things that only exist in a person's mind; \textbf{3.} according to the laws of nature or what is likely.} violent\footnote{\textbf{violent} [a] \textbf{1.} using or involving physical force intended to hurt, damage or kill somebody\texttt{/}something; \textbf{2.} involving great force; \textbf{3.} showing or involving very strong emotion.} nor inherently\footnote{\textbf{inherent} [a] that is a permanent, basic or typical feature of somebody\texttt{/}something, \textsc{synonym}: \textbf{intrinsic}.} destructive\footnote{\textbf{destructive} [a] causing destruction or damage.}. Some theorists have suggested that only when this most basic instinct is severely\footnote{\textbf{severely} [adv] \textbf{1.} very badly or seriously; \textbf{2.} in a very strict way.} frustrated\footnote{\textbf{frustrated} [a] \textbf{1.} feeling impatient \& slightly angry because you cannot do or achieve what you want; \textbf{2.} (of an emotion) having no effect; not being satisfied.} does our aggressive\footnote{\textbf{aggressive} [a] \textbf{1.} angry, \& behaving in a harmful or violent way; ready to attack; \textbf{2.} acting with force \& determination in order to succeed; \textbf{3.} (\textit{medical}) (of a disease or medical condition) quickly becoming more serious.} drive have the potential\footnote{\textbf{potential} [a] [only before noun] that can develop into something or be developed in the future, \textsc{synonym}: \textbf{possible}; [n] \textbf{1.} [uncountable] the possibility of something happening or being developed or used; \textbf{2.} [uncountable] qualities that exist \& can be developed, \textsc{synonym}: \textbf{promise}; \textbf{3.} [uncountable, countable] (\textit{physics}) the difference in voltage between 2 points in an electric field or circuit.} to be expressed violently\footnote{\textbf{violently} [adv] \textbf{1.} in a way that involves physical violence; \textbf{2.} in a way that shows or involves very strong emotion; \textbf{3.} with great force.}.\footnote{Storr, A., Human Destructiveness, (Ballantine, 1991), p. 21.} Others have suggested that some rare individuals seem to be predisposed\footnote{\textbf{predispose} [v] \textbf{1.} [transitive, intransitive] to make it likely that somebody will suffer from a particular illness or condition; \textbf{2.} [transitive] to influence somebody so that they are likely to think or behave in a particular way.} to aggression -- even violent aggression, despite the most benign\footnote{\textbf{benign} [a] \textbf{1.} (\textit{medical}) not dangerous; not likely to cause death, \textsc{opposite}: \textbf{malignant}; \textbf{2.} not hurting anyone; not harmful.} circumstances. But whether extraordinary\footnote{\textbf{extraordinary} [a] \textbf{1.} unexpected, surprising or strange; \textbf{2.} not normal or ordinary; greater or better than usual; \textbf{3.} [only before noun] (of a meeting, etc.) arranged for a special purpose \& happening in addition to what normally or regularly happens.} stressors, genetic\footnote{\textbf{genetic} [a] connected with genes or genetics.} predispositions\footnote{\textbf{predisposition} [n] [countable, uncountable] a condition that makes somebody\texttt{/}something likely to behave in a particular way or to suffer from a particular disease.}, reinforced\footnote{\textbf{reinforce} [v] \textbf{1.} \textbf{reinforce something} to make a feeling, idea, habit or tendency stronger; \textbf{2.} \textbf{reinforce something} to make a structure or material stronger, especially by adding another material to it; \textbf{3.} \textbf{reinforce something} to send more people or equipment in order to make an army, etc. stronger.} learning patterns\footnote{\textbf{pattern} [n] \textbf{1.} the regular way in which something happens or is done; \textbf{2.} a regular arrangement of lines, shapes, colors, etc. found in similar objects or as a design on material, etc. In science, \textbf{pattern formation} is the scientific study of patterns in nature.; \textbf{3.} [usually singular] \textbf{pattern (for something)} an example for others to copy; [v] \textbf{pattern something} (\textit{specialist}) to give a clear or regular form to something in nature or society.}, or some combination\footnote{\textbf{combination} [n] \textbf{1.} [countable] \textbf{combination (of something)} 2 or more things joined or mixed together to form a single unit; \textbf{2.} [uncountable] the act of joining or mixing together 2 or more things to form a single unit.} of these are at the root of violent aggression, most theorists agree that aggression per se\footnote{\textbf{per se} [adv] (\textit{from Latin}) used meaning `by itself' to show that you are referring to something on its own, rather than in connection with other things.} \& destructive violence are not synonymous\footnote{\textbf{synonymous} [a] \textbf{1.} \textbf{synonymous (with something)} so closely connected with something that the 2 things appear to be the same; \textbf{2.} (of words or expressions) having the same, or nearly the same, meaning.}. In this book, the term aggression will refer to the forceful\footnote{\textbf{forceful} [a] \textbf{1.} (of people) expressing opinion firmly \& clearly in a way that persuades other people to believe them, \textsc{synonym}: \textbf{assertive}; \textbf{2.} (of opinions, etc.) expressed firmly \& clearly so that other people believe them; \textbf{3.} using force; \textbf{4.} (of action) strong \& effective.} energy we all expend\footnote{\textbf{expend} [v] to use or spend time, money, energy, etc.} in our daily bids\footnote{\textbf{bid} [n] \textbf{1.} \textbf{bid (for something)} an offer by a person or a company to pay a particular amount of money for something; \textbf{2.} \textbf{bid (for something)} (\textit{North American English also}) \textbf{bid (on something)} an offer to do work or provide a service for a particular price, in competition with other companies, \textsc{synonym}: \textbf{tender}; \textbf{3.} an effort to do something or to obtain something; [v] [intransitive] to offer to do work or provide a service for a particular price, in competition with other companies, etc., \textsc{synonym}: \textbf{tender}.} to survive, advance ourselves, secure\footnote{\textbf{secure} [v] \textbf{1.} to obtain or achieve something, especially when this means using a lot of effort; \textbf{2.} \textbf{secure something on\texttt{/}against something} to legally agree to give somebody property or goods that are worth the same amount as the money that you have borrowed from them, if you are unable to pay the money back; \textbf{3.} \textbf{secure something (against something)} to protect something so that it is safe \& difficult to attack or damage; \textbf{4.} \textbf{secure something (to something)} to attack or fix something firmly; [a] \textbf{1.} safe from being attack, harmed or damaged; protected \&\texttt{/}or made stronger so that it is difficult for people to enter or leave, or to take something, \textsc{opposite}: \textbf{insecure}; \textbf{2.} likely to continue or be successful for a long time, \textsc{synonym}: \textbf{safe}, \textsc{opposite}: \textbf{insecure}; \textbf{3.} feeling happy \& confident about yourself or a particular situation, so that you do not need to worry, \textsc{opposite}: \textbf{insecure}; \textbf{4.} fixed or attached firmly.} things we believe will bring us some kind of pleasure\footnote{\textbf{pleasure} [n] \textbf{1.} [uncountable] a state of feeling or being happy or satisfied; the activity of enjoying yourself, \textsc{synonym}: \textbf{enjoyment}; \textbf{2.} [countable] a thing that makes you happy or satisfied.}, \& remove obstacles\footnote{\textbf{obstacle} [n] \textbf{1.} a situation, event or fact that makes it difficult for you to do or achieve something; \textbf{2.} an object that is in your way \& that makes it difficult for you to move forward.} to those ends.

People do a lot more fighting in their daily lives than we have ever been willing to acknowledge\footnote{\textbf{acknowledge} [v] \textbf{1.} to accept that something is true or exists; \textbf{2.} to accept that somebody\texttt{/}something has a particular quality, importance or status, \textsc{synonym}: \textbf{recognize}; \textbf{3.} \textbf{acknowledge somebody\texttt{/}something} to publicly express thanks for help or inspiration; \textbf{4.} \textbf{acknowledge something} to tell somebody that you have received something that they sent to you.}. The urge\footnote{\textbf{urge} [v] \textbf{1.} to advise or try hard to persuade somebody to do something; \textbf{2.} \textbf{urge something (on\texttt{/}upon somebody)} to recommend something strongly; [n] \textbf{urge (to do something)} a strong desire to do something.} to fight is fundamental \& instinctual\footnote{\textbf{instinctual} [a] (\textit{biology or psychology}) based on instinct ($=$ a natural quality that makes somebody\texttt{/}something behave in a particular way); not learned.}. Anyone who denies\footnote{\textbf{deny} [v] \textbf{1.} to say that something is not true; \textbf{2.} \textbf{deny something} to refuse to admit or accept something; \textbf{3.} to refuse to allow somebody to have something that they want or ask for; \textbf{4.} to refuse to let yourself have something that you would like to have, especially for moral or religious reasons; \textbf{there is no denying something\texttt{/}that $\ldots$} [idiom] used to say that it is impossible to refuse to accept that something is true.} the instinctual nature of aggression has either never witnessed 2 toddlers\footnote{\textbf{toddler} [n] a child who has only recently learnt to walk.} struggling for possession of the same toy, or has somehow forgotten this archetypal\footnote{\textbf{archetypal} [a] having all the important qualities that make somebody\texttt{/}something a typical example of a particular kind of person or thing.} scene. Fighting is a big part of our culture, also. From the fierce\footnote{\textbf{fierce} [a] (\textbf{fiercer, fiercest}) \textbf{1.} showing strong feelings or emotions or a lot of activity, often in a way that is violent; \textbf{2.} (especially of people or animals) angry \& aggressive in a way that is frightening.} partisan\footnote{\textbf{partisan} [a] (\textit{often disapproving}) showing too much support for 1 person, group or idea, especially without considering it carefully; [n] \textbf{1.} \textbf{partisan (of somebody\texttt{/}something)} a person who strongly supports a particular leader, group or idea; \textbf{2.} a member of an armed group that is fighting secretly against enemy soldiers who have taken control of its country.} wrangling\footnote{\textbf{wrangling} [n] [uncountable, countable] the process of conducting a complicated argument with somebody over a long period of time.} that characterizes\footnote{\textbf{characterize} [v] (\textit{British English also} \textbf{characterise}) \textbf{1.} [usually passive] \textbf{characterize something} to be the most typical or most obvious quality or feature of something, \textsc{synonym}: \textbf{typify}; \textbf{2.} [usually passive] \textbf{characterize something} to be the feature or quality that makes something different from similar things, \textsc{synonym}: \textbf{distinguish}; \textbf{3.} [often passive] \textbf{characterize somebody\texttt{/}something (as something)} to describe something\texttt{/}somebody in a particular way.} representative\footnote{\textbf{representative} [n] \textbf{1.} a person who has been chosen to act, speak or vote for somebody\texttt{/}something; a person who has been chosen to attend or take part in an event on behalf of somebody. In the US, the House of Representatives is the lower house of Congress, whose members are called \textbf{Representatives}.; \textbf{2.} \textbf{representative (of something)} a person, animal or thing that is a typical example of a particular group; \textbf{3.} a person who works for a company \& travels around selling its products. In general \& Business English, the less formal \textbf{rep} is used; [a] \textbf{1.} \textbf{representative (of somebody\texttt{/}something)} typical of a particular group; that can be used as an example; \textbf{2.} (\textit{specialist}) (of a sample or piece of work) containing or including examples of all the different types of people or things in a large group; \textbf{3.} (of a system of government, etc.) consisting of people who have been chosen to speak or vote for other people.} government, to the competitive corporate\footnote{\textbf{corporate} [a] [only before noun] \textbf{1.} connected with large business companies or a particular large company; \textbf{2.} forming a single unit as a corporation.} environment, to the adversarial\footnote{\textbf{adversarial} [a] (especially of political or legal systems) involving people who are in opposition \& who criticize or attack each other.} system of our judicial\footnote{\textbf{judicial} [usually before noun] connected with a court, judge or legal judgment.} system, much fighting is woven\footnote{\textbf{weave} [v] \textbf{1.} [transitive, intransitive] to make cloth by crossing threads or strips across, over \& under each other by hand or by machine; \textbf{2.} [transitive] to put facts, events, details, etc. together to make a story or a closely connected whole.} into our societal fabric\footnote{\textbf{fabric} [n] \textbf{1.} [uncountable, countable] cloth used for making clothes, covering furniture, etc., \textsc{synonym}: \textbf{material}; \textbf{2.} [singular] the basic structure of a society, an organization, etc. that enables it to function successfully; \textbf{3.} [uncountable] the basic structure of a building, such as the walls, floor \& roof.}. We sue\footnote{\textbf{sue} [v] \textbf{1.} [transitive, intransitive] \textbf{sue (somebody) (for something)} to make a claim against a person, company, etc. in court about something that they have said or done to harm you; \textbf{2.} [intransitive] \textbf{sue for something} (\textit{formal}) to formally ask for something.} on another, divorce\footnote{\textbf{divorce} [v] \textbf{1.} [transitive, intransitive] to end your marriage to somebody legally; \textbf{2.} [transitive, often passive] \textbf{divorce something from something} to separate an idea, a subject, etc. from something else; [n] \textbf{1.} [countable, uncountable] the legal ending of a marriage; \textbf{2.} [countable, usually singular] \textbf{divorce (between A \& B)} the ending of a relationship between 2 things, \textsc{synonym}: \textbf{separation}.} each other, battle\footnote{\textbf{battle} [n] \textbf{1.} [countable, uncountable] a fight between armies, ships or planes, especially during a war; \textbf{2.} [countable] a competition, an argument or a struggle between people or groups of people trying to win power or control; \textbf{3.} [countable, usually singular] a determined effort that somebody makes to solve a difficult problem or succeed in a difficult situation; [v] [intransitive, transitive] to try very hard to deal with something unpleasant or achieve something difficult, \textsc{synonym}: \textbf{fight}.} with one another over our children, compete\footnote{\textbf{compete} [v] \textbf{1.} [intransitive] to try to be more successful than others. If somebody\texttt{/}something \textbf{cannot compete} with\texttt{/}against somebody\texttt{/}something else, they are not as successful.; \textbf{2.} [intransitive] to try to get something or do something, rather than letting somebody\texttt{/}something else get it or do it; \textbf{3.} [intransitive] \textbf{compete (with somebody\texttt{/}something)} to oppose somebody\texttt{/}something; \textbf{4.} [intransitive] to take part in an election, sports event or other contest.} for jobs, \& struggle\footnote{\textbf{struggle} [n] \textbf{1.} [countable] a hard fight in which people try to obtain or achieve something, especially something that somebody else does not want them to have, \textsc{synonym}: \textbf{battle}; \textbf{2.} [singular] \textbf{struggle to do something} something that is difficult for you to do or achieve, \textsc{synonym}: \textbf{effort}; \textbf{3.} [countable] a physical fight between 2 people or groups, especially when 1 of them is trying to escape, or to get something from the other, \textsc{synonym}: \textbf{fight}; [v] \textbf{1.} [intransitive] to try very hard to do something when it is difficult or when there are a lot of problems; \textbf{2.} [intransitive] to fight against somebody\texttt{/}something in order to prevent a bad situation or result; \textbf{3.} [intransitive] to compete or argue with somebody, especially in order to get something.} with each other to advance certain goals, values, beliefs \& ideals\footnote{\textbf{ideal} [a] \textbf{1.} perfect; most suitable; \textbf{2.} [only before noun] the best that can be imagined, but not likely to become real; \textbf{in an ideal\texttt{/}a perfect world} [idiom] used to say that something is what you would like to happen or what should happen, but you know it cannot; [n] \textbf{1.} \textbf{ideal (of somebody\texttt{/}something)} an idea or a standard that seems perfect \& worth trying to achieve; \textbf{2.} [usually singular] \textbf{ideal (of something)} a person or thing considered as perfect.}. The psychodynamic theorist Alfred Adler noted many years ago that we also forcefully\footnote{\textbf{forceful} [a] \textbf{1.} (of people) expressing opinions firmly \& clearly in a way that persuades other people to believe them, \textsc{synonym}: \textbf{assertive}; \textbf{2.} (of opinions, etc.) expressed firmly \& clearly so that other people believe them; \textbf{3.} using force; \textbf{4.} (of action) strong \& effective.} strive\footnote{\textbf{strive} [v] [intransitive] to try very hard to achieve something.} to assert a sense of social superiority\footnote{\textbf{superiority} [n] [uncountable] \textbf{1.} the state or quality of being better, more skillful, more powerful, greater, etc. than others, \textsc{opposite}: \textbf{inferiority}; \textbf{2.} behavior that shows that you think you are better than other people.}.\footnote{Adler, A., \textit{Understanding Human Nature}, (Fawcett World Library, 1954), p. 178.} Fighting for personal \& social advantage, we jockey\footnote{\textbf{jockey} [n] a person who rides horses in races, especially as a job.} with one another for power, prestige\footnote{\textbf{prestige} [n] [uncountable] respect \& admiration that somebody\texttt{/}something receives, becaues people consider them\texttt{/}it to have great importance, quality or value.}, \& a secure social ``niche\footnote{\textbf{niche} [n] \textbf{1.} (\textit{business}) an opportunity to sell a particular product to a particular group of people; \textbf{2.} (\textit{biology}) a position or role taken by a kind of living thing within its community. Different living things may fill the same niche in different places.; \textbf{3.} a comfortable or suitable role, job or way of life; \textbf{4.} \textbf{niche (in something)} a small hollow place, e.g. in a rock or the side of a hill; a small hollow place in a wall to contain a statue, etc.}.'' Indeed, we do so much fighting in so many aspects of our lives I think it fair to say that when human beings aren't making some kind of love, they're likely to be waging\footnote{\textbf{wage} [n] [singular] (\textbf{wages} [plural]) a regular amount of money that you earn, usually every week, for work or services. In economics, \textbf{wages} are the price of providing labor, paid to the laborer. Wages are 1 part of \textbf{income} which also includes other sources of payments, such as interest on money that has been invested; [v] to begin \& continue a war or struggle.} some kind of war.

Fighting is not inherently wrong or harmful. Fighting openly\footnote{\textbf{openly} [adv] without hiding any information, opinions or feelings.} \& fairly\footnote{\textbf{fairly} [adv] \textbf{1.} (before adjectives \& adverbs) quite but not very; \textbf{2.} in a fair way; in a way that treats people equally \& according to the rules or law.} for our legitimate\footnote{\textbf{legitimate} [a] \textbf{1.} for which there is a fair \& acceptable reason, \textsc{synonym}: \textbf{valid, justifiable}; \textbf{2.} allowed \& acceptable according to the law, \textsc{synonym}: \textbf{legal}, \textsc{opposite}: \textbf{illegitimate}; \textbf{3.} (of a child) born when its parents are legally married to each other, \textsc{opposite}: \textbf{illegitimate}; \textbf{4.} having the legal right to have somebody's title when that person dies.} needs is often necessary \& constructive\footnote{\textbf{constructive} [a] having a useful \& helpful effect rather than being negative or with no purpose.}. When we fight for what we truly need while respecting the rights \& needs of others \& taking care not to needlessly\footnote{\textbf{needlessly} [adv] in a way that is not necessary because it could have been avoided.} injure\footnote{\textbf{injure} [v] \textbf{1.} \textbf{injure somebody\texttt{/}something\texttt{/}yourself} to harm yourself or somebody else physically, especially in an accident; to harm somebody mentally; \textbf{2.} \textbf{injure something} to damage somebody's reputation, pride, etc.} them, our behavior is best labeled \textit{assertive}\footnote{\textbf{assertive} [a] (\textit{often approving}) expressing opinions or desires strongly \& with confidence, so that people take notice.}, \& assertive behavior is 1 of the most healthy \& necessary human behaviors. It's wonderful when we learn to assert ourselves in the pursuit\footnote{\textbf{pursuit} [n] \textbf{1.} [uncountable] the act of trying to find, obtain or achieve something; \textbf{2.} [countable] an activity, especially one that you do because you enjoy it; \textbf{3.} [uncountable] the act of following or trying to catch somebody.} of personal needs, overcome\footnote{\textbf{overcome} [v] \textbf{1.} \textbf{overcome something} to succeed in dealing with or controlling a problem that has been preventing you from achieving something; \textbf{2.} [usually passive] \textbf{(be) overcome by something} to be extremely strongly affected by something, \textsc{synonym}: \textbf{overwhelm}; \textbf{3.} \textbf{overcome somebody\texttt{/}something} to defeat somebody.} unhealthy\footnote{\textbf{unhealthy} [a] \textbf{1.} harmful to your health; likely to make you ill, \textsc{opposite}: \textbf{healthy}; \textbf{2.} not having good health; showing a lack of good health, \textsc{opposite}: \textbf{healthy}; \textbf{3.} (of somebody's attitude or behavior) not normal \& likely to be harmful, \textsc{opposite}: \textbf{healthy}.} dependency\footnote{\textbf{dependency} [n] (plural \textbf{dependencies}) \textbf{1.} [uncountable] the state of relying on somebody\texttt{/}something for something, especially when this is nor normal or necessary. In economics \& geography, the \textbf{dependence ratio} is the number of children under the age of 18 \& adults over the age of 64 (i.e. people not in the labor force) per hundred workers.; \textbf{2.} [countable] a country or area that is controlled by another country; \textbf{3.} $=$ \textbf{dependence}.} \& become self-sufficient\footnote{\textbf{self-sufficient} [a] \textbf{self-sufficient (in something)} able to do or produce everything that you need without the help of other people.} \& capable\footnote{\textbf{capable} [a] \textbf{1.} having the ability or qualities necessary for doing something. Note that \textbf{capable} is never followed by an infinitive. \textsc{opposite}: \textbf{incapable}; \textbf{2.} having the ability to do things well, \textsc{synonym}: \textbf{skilled, competent}.}. But when we fight unnecessarily, or with little concern about how others are being affected\footnote{\textbf{affect} [v] \textbf{1.} [often passive] to make a difference to somebody\texttt{/}something or to what somebody thinks or does; \textbf{2.} \textbf{affect something\texttt{/}somebody} (of a disease) to attack a part of the body; to make somebody become ill; \textbf{3.} \textbf{be affected by something} if you are affected by an event, it makes you feel very sad; [n] [uncountable] (\textit{psychology}) desire or emotion, especially in relation to how they influence your thoughts or behavior.}, our behavior is most appropriately\footnote{\textbf{appropriately} [adv] in a way that is suitable, acceptable or correct for the particular circumstances.} labeled \textit{aggressive}. In a civilized\footnote{\textbf{civilized} [a] (\textit{British English also} \textbf{civilised}) \textbf{1.} well-organized socially with a very developed culture \& way of life; \textbf{2.} having laws \& customs that are fair \& morally acceptable.} world, undisciplined\footnote{\textbf{undisciplined} [a] not having enough control or organization; behaving badly, \textsc{opposite}: \textbf{disciplined}.} fighting (aggression) is almost always a problem. The fact that we are an aggressive species\footnote{\textbf{species} [n] (plural \textbf{species}) \textbf{1.} a group into which animals, plants, etc. that are able to breed with each other \& produce healthy young are divided, smaller than a genus \& identified by a Latin name; \textbf{2.} (\textit{chemistry, physics}) a particular kind of atom, molecule, ion or particle.} doesn't make us inherently flawed\footnote{\textbf{flawed} [a] having a flaw; damaged or spoiled.} or ``evil,'' either. Adopting a perspective advanced largely by Carl Jung,\footnote{Jung, C. G., 1953, Collected Works of, Vol. 7, p.25. H. Read, M. Fordham \& G. Adler, eds. New York: Pantheon.} I would assert that the evil that sometimes arises from a person's aggressive behavior necessarily stems from his or her failure to ``own'' \& discipline this most basic human instinct.''

\subsection*{2 Important Types of Aggression}
``2 of the most fundamental types of fighting (others, such as reactive\footnote{\textbf{reactive} [a] \textbf{1.} (\textit{chemistry}) tending to show chemical change when mixed with another substance; \textbf{2.} showing a reaction or response.} vs. predatory\footnote{\textbf{predatory} [a] \textbf{1.} (\textit{specialist}) (of animals) living by killing \& eating other animals; \textbf{2.} (of people or behavior) treating somebody badly in order to gain control of something.} or instrumental\footnote{\textbf{instrumental} [a] \textbf{1.} \textbf{instrumental (in something\texttt{/}in doing something)} important in making something happen; connected with something's function as a way of making something happen; \textbf{2.} connected with a scientific instrument or measuring device; \textbf{3.} made by or for musical instruments.} aggression) will be discussed later are \textit{overt}\footnote{\textbf{overt} [a] [usually before noun] done in an open way \& not secretly.} \& \textit{covert}\footnote{\textbf{covert} [a] secret or hidden, making it difficult to notice.} aggression. When you're determined\footnote{\textbf{determined} [a] \textbf{1.} [not before noun] \textbf{determined to do something} if you are determined to do something, you have made a decision to do it \& you will not let anyone prevent you; \textbf{2.} [only before noun] showing determination to do something.} to have your way or gain advantage \& you're open, direct, \& obvious in your manner of fighting, your behavior is best labeled overtly aggressive. When you're out to ``win,'' get your way, dominate\footnote{\textbf{dominate} [v] \textbf{1.} [transitive, intransitive] \textbf{dominate (something\texttt{/}somebody)} to control or have a lot of influence over something\texttt{/}somebody, especially in a negative way; \textbf{2.} [transitive] \textbf{dominate something} to be the most important or obvious feature of something; \textbf{3.} [transitive, intransitive] \textbf{dominate (something)} to be the largest, highest or most common thing in a place.}, or control, but are subtle, underhanded\footnote{\textbf{underhand} [a] (\textit{also less frequent} \textbf{underhanded}) (\textit{disapproving}) secret \& dishonest.}, or deceptive\footnote{\textbf{deceptive} [a] likely to make you believe something that is not true, \textsc{synonym}: \textbf{misleading}.} enough to hide your true intentions, your behavior is most appropriately labeled covertly aggressive. Concealing\footnote{\textbf{conceal} [v] to hide something.} overt displays\footnote{\textbf{display} [v] \textbf{1.} [transitive] to put something in a place where people can see it easily; to show something to people, \textsc{synonym}: \textbf{exhibit}; \textbf{2.} [transitive] \textbf{display something} to show signs of something, especially a quality, characteristic or feeling; \textbf{3.} [transitive] \textbf{display something} (of a computer, notice, table, etc.) to show information; \textbf{4.} [intransitive] (of male birds \& animals) to show a special pattern of behavior that is intended to attract a female bird or animal; [n] \textbf{1.} [countable] an arrangement of things in a public place to give information or entertain people or advertise something for sale. Things that are \textbf{on display} are put in a place where people can look at them.; \textbf{2.} [countable, uncountable] \textbf{display of something} behavior that shows a particular quality, feeling or ability; \textbf{3.} [uncountable] \textbf{display of something} the act of placing something in a public place for people to see; \textbf{4.} [countable] \textbf{display (of something)} an act of performing a skill or of showing something happening, in order to entertain; \textbf{5.} [countable, uncountable] \textbf{display (of something)} a special pattern of behavior that a male bird or animal shows in order to attract a female bird or animal.} of aggression while simultaneously intimidating\footnote{\textbf{intimidate} [v] \textbf{intimidate somebody (into something\texttt{/}into doing something)} to frighten or threaten somebody so that they will do what you want.}\,\footnote{\textbf{intimidating} [a] frightening in a way that makes a person feel less confident.} others into backing-off\footnote{\textbf{back off} [phrasal verb] \textbf{back off} \textbf{1.} to move backwards in order to get away from somebody\texttt{/}something, frightening or unpleasant; \textbf{2.} to stop threatening, criticizing or annoying somebody; \textbf{back off (from something)} to choose not to take action, in order to avoid a difficult situation.}, backing-down\footnote{\textbf{back down} [phrasal verb] \textbf{back down (on\texttt{/}from something)} (\textit{North American English also} \textbf{back off}) to take back a demand, an opinion, etc. that other people are strongly opposed to; to admit defeat.}, or giving-in\footnote{\textbf{give in} [phrasal verb] \textbf{give in (to somebody\texttt{/}something)} \textbf{1.} to admit that you have been defeated by somebody\texttt{/}something; \textbf{2.} to agree to do something that you do not want to do; \textbf{give something in (to somebody)} (\textit{British English}) (also \textbf{hand something $\leftrightarrow$ into (to somebody)} \textit{British English, North American English}) to hand over something to somebody in authority.} is a very powerful manipulative maneuver. That's why \textit{covert-aggression is most often the vehicle\footnote{\textbf{vehicle} [n] \textbf{1.} a thing that is used for transporting people or goods from 1 place to another, such as a car or truck; \textbf{2.} something that can be used to express your ideas or feelings as a way of achieving something; \textbf{3.} a privately controlled company through which an individual or organization conducts a particular kind of business, especially investment.} for interpersonal manipulation}.''

\subsection*{Covert \& Passive-Aggression}
``I often hear people say someone is being ``passive-aggressive'' when they're really trying to describe covertly aggressive behavior. Covert \& passive-aggression are both indirect\footnote{\textbf{indirect} [a] [usually before noun] \textbf{1.} not directly caused by or connected with something. \textbf{Indirect costs} are the costs not directly connected with making a product or providing a service, e.g. the cost of training, heating or rent. \textbf{Indirect taxes\texttt{/}taxation} are\texttt{/}is paid as an amount added to the price of goods \& services rather than being paid on income or profits. \textsc{opposite}: \textbf{direct}; \textbf{2.} not done directly; done on somebody's behalf by somebody else, \textsc{opposite}: \textbf{direct}; \textbf{3.} suggesting something without clearly showing it, \textsc{opposite}: \textbf{direct}; \textbf{4.} avoiding saying something in a clear \& obvious way, \textsc{opposite}: \textbf{direct}; \textbf{5.} (of a route) not going in a straight line; not following the shortest way, \textsc{opposite}: \textbf{direct}.} ways to aggress but they're most definitely\footnote{\textbf{definitely} [adv] \textbf{1.} without doubt; \textbf{2.} in a way that is certain or that shows that you are certain.} not the same thing. Passive-aggression is, as the term implies, aggressing through passivity\footnote{\textbf{passivity} [n] [uncountable] the state of accepting what happens without reacting or trying to fight against it.}. Examples of passive-aggression\footnote{\textbf{passive-aggressive} [a] being angry without expressing your anger openly, but resisting people in authority by refusing to do what they want or to accept responsibility for your actions.}  are playing the game of emotional ``get-back'' with someone by resisting cooperation with them, giving them the ``silent treatment,'' pouting or whining, not so accidentally ``forgetting'' something they wanted you to do because you're angry \& didn't really feel like obliging them, etc. 

'' -- \cite[Introduction]{Simon2010}

\section{Aggressive \& Covert-Aggressive Personalities}

\section{The Determination to Win}

\section{The Unbridled Quest for Power}

\section{The Penchant for Deception \& Seduction}

\section{Fighting Dirty}

\section{The Impaired Conscience}

\section{Abusive, Manipulative Relationships}

\section{The Manipulative Child}

\begin{center}\huge
	Part II: Dealing Effectively with Manipulative People
\end{center}

\section{Recognizing the Tactics of Manipulation \& Control}

\section{Redefining the Terms of Engagement}

\section*{Epilogue: Undisciplined Aggression in a Permissive Society}

\section*{Endnotes}

%------------------------------------------------------------------------------%

\selectlanguage{english}
\chapter{\href{https://www.criticalthinking.org/}{The Foundation for Critical Thinking}}

``The Foundation is a non-profit organization that seeks to promote essential change in education \& society through the cultivation of fairminded critical thinking -- thinking which embodies intellectual empathy, intellectual humility, intellectual perseverance, intellectual integrity \& intellectual responsibility.''

\begin{quotation}
	``I found that I was fitted for nothing so well as for the study of Truth $\ldots$ with desire to seek, patience to doubt, fondness to meditate, slowness to assert, readiness to consider, carefulness to dispose \& set in order $\ldots$ being a man that neither affects what is new nor admires what is old, \& that hates every kind of imposture.'' -- Francis Bacon (1605)
\end{quotation}

\section{\href{https://www.criticalthinking.org/pages/critical-thinking-where-to-begin/796}{The Foundation for Critical Thinking\texttt{/}Critical Thinking: Where to Begin}}
``Many of our resources, publications, \& materials are applicable to all professions \& across all domains of thought. We do, however, recognize that the depth \& breadth of content we offer may be daunting. We have therefore created the following pages as starting points by your studies.
\begin{itemize}
	\item \href{https://www.criticalthinking.org/starting/higher_ed.cfm}{For College \& University Faculty}
	\item \href{https://www.criticalthinking.org/starting/college_student.cfm}{For College \& University Students}
	\item \href{https://www.criticalthinking.org/starting/High_School.cfm}{For High School Teachers}
	\item \href{https://www.criticalthinking.org/starting/Jr_High.cfm}{For Jr. High School Teachers}
	\item \href{https://www.criticalthinking.org/starting/elem_Grades4-6.cfm}{For Elementary Teachers (Grades 4--6)}
	\item \href{https://www.criticalthinking.org/starting/elementary.cfm}{For Elementary Teachers (Kindergarten -- 3rd Grade)}
	\item \href{https://www.criticalthinking.org/starting/science_engineer.cfm}{For Science \& Engineering Instruction}
	\item \href{https://www.criticalthinking.org/starting/business.cfm}{For Business \& Professional Development}
	\item \href{https://www.criticalthinking.org/starting/nurse_health.cfm}{For Nursing \& Health Care}
	\item \href{https://www.criticalthinking.org/starting/home_school.cfm}{For Home Schooling \& Home Study}
\end{itemize}
If you are new to critical thinking or wish to deepen your concept of it, we recommend you review the content below \& bookmark this page for future reference.''

\subsection{Our Conception of Critical Thinking $\ldots$}
``There are many ways to articulate the concept of critical thinking, yet every substantive conception must contain certain core elements. Consider these brief conceptualizations of critical thinking $\ldots$
\begin{quotation}
	``Critical thinking is the intellectually disciplined process of actively \& skillfully conceptualizing, applying, analyzing, synthesizing, \&\texttt{/}or evaluating information gathered from, or generated by, observation, experience, reflection, reasoning, or communication, as a guide to belief \& action. In its exemplary form, it is based on universal intellectual values that transcend subject matter divisions: clarity, accuracy, precision, consistency, relevance, sound evidence, good reasons, depth, breadth, \& fairness $\ldots$'' -- A statement by Michael Scriven \& Richard Paul, presented at the 8th Annual International Conference on Critical Thinking \& Education Reform, 1987, \href{https://www.criticalthinking.org/aboutCT/define_critical_thinking.cfm}{a more complete version}
\end{quotation}

\begin{quotation}
	``Critical thinking is self-guided, self-disciplined thinking which attempts to reason at the highest level of quality in a fairminded way. People who think critically attempt, with consistent \& conscious effort, to live rationally, reasonably, \& empathically. They are keenly aware of the inherently flawed nature of human thinking when left unchecked. They strive to diminish the power of their egocentric \& sociocentric tendencies. They use the intellectual tools that critical thinking offers -- concepts \& principles that enable them to analyze, assess, \& improve thinking. They work diligently to develop the intellectual virtues of intellectual integrity, intellectual humility, intellectual civility, intellectual empathy, intellectual sense of justtice \& confidence in reason. They realize that no matter how skilled they are as thinkers, they can always improve their reasoning abilities \& they will at times fall prey to mistakes in reasoning, human irrationality, prejudices, biases, distortions, uncritically accepted social rules \& taboos, self-interest, \& vested interest.
	
	They strive to improve the world in whatever ways they can \& contribute to a more rational, civilized society. At the same time, they recognize the complexities often inherent in doing so. They strive never to think simplistically about complicated issues \& always to consider the rights \& needs of relevant others. They recognize the complexities in developing as thinkers, \& commit themselves to life-long practice toward self-improvement. They embody the Socratic principle: \textit{The unexamined life is not worth living}, because they realize that many unexamined lives together result in an uncritical, unjust, dangerous world.'' -- Linda Elder, Sep 2007''
\end{quotation}

\subsection{Why Critical Thinking?}

\paragraph*{The Problem.} ``Everyone thinks; it is our nature to do so. But much of our thinking, left to itself, is biased, distorted, partial, uninformed, or down-right prejudiced. Yet the quality of our lives \& that of what we produce, make, or build depends precisely on the quality of our thought. Shoddy thinking is costly, both in money \& in quality of life. Excellence in thought, however, must be systematically cultivated.''

\paragraph*{A Brief Definition.} ``Critical thinking is the art of analyzing \& evaluating thinking with a view to improving it.''

\paragraph*{The Result.} ``A well-cultivated critical thinker:
\begin{itemize}
	\item raises vital questions \& problems, formulating them clearly \& precisely;
	\item gathers \& assesses relevant information, using abstract ideas to interpret it effectively;
	\item comes to well-reasoned conclusions \& solutions, testing them against relevant criteria \& standards;
	\item thinks openmindedly within alternative systems of thought, recognizing \& assessing, as need be, their assumptions, implications, \& practical consequences; \&
	\item communicates effectively with others in figuring out solutions to complex problems.
\end{itemize}
Critical thinking is, in short, self-directed, self-disciplined, self-monitored, \& self-corrective thinking. It requires rigorous standards of excellence \& mindful command of their use. It entails effective communication \& problem-solving abilities, \& a commitment to overcoming our native egocentrism \& sociocentrism.'' Read \href{https://www.criticalthinking.org/aboutCT/ourConceptCT.cfm}{The Foundation for Critical Thinking's concept of critical thinking}.

\subsection{The Essential Dimensions of Critical Thinking}
``Our conception of critical thinking is based on the \href{https://www.criticalthinking.org/professionalDev/the-state-ct-today.cfm}{substantive approach} developed by Dr. Richard Paul \& his colleagues at the Center \& Foundation for Critical Thinking over multiple decades. It is relevant to every subject, discipline, \& profession, \& to reasoning through the problems of everyday life. It entails 5 essential dimensions of critical thinking:
\begin{enumerate}
	\item The analysis of thought.
	\item The assessment of thought.
	\item The dispositions of thought.
	\item The skills \& abilities of thought.
	\item The obstacles or barriers to critical thought.
\end{enumerate}
At the left is an overview of the 1st 3 dimensions. In sum, the elements or structures of thought enable us to ``take our thinking apart'' \& analyze it. The intellectual standards are used to assess \& evaluate the elements. The intellectual traits are dispositions of mind embodied by the fairminded critical thinker. To cultivate the mind, we need command of these essential dimensions, \& we need to consistently apply them as we think through the many problems \& issues in our lives.''

\textsf{Fig. Critical Thinkers Routinely Apply Intellectual Standards to The Elements of Reasoning In Order to Develop Intellectual Traits: \textbf{The Standards} [Clarity, Accuracy, Relevance, Logicalness, Breadth, Precision, Significance, Completeness, Fairness, Depth] $\to$(Must be applied to)$\to$ \textbf{The Elements} [Purposes, Questions, Points of view, Information, Inferences, Concepts, Implications, Assumptions] $\to$(As we learn to develop)$\to$ \textbf{Intellectual Traits} [Intellectual Humility, Intellectual Autonomy, Intellectual Integrity, Intellectual Courage, Intellectual Perseverance, Confidence in Reason, Intellectual Empathy, Fairmindedness].}

\subsection{The Elements of Reasoning \& Intellectual Standards}
``To learn more about the elements of thought \& how to apply the intellectual standards, check out our interactive model. simply click on the link \href{javascript:void(window.open('https://www.criticalthinking.org/CTmodel/CTModel1.cfm','CTModel','resizable=yes,location=no,menubar=yes,scrollbars=yes,status=no,toolbar=no,fullscreen=no,dependent=no,width=840,height=680,left=10,top=10'))}{Open the ``Elements \& Standards'' Online Learning Model} \& use your mouse to explore each concept.''

\subsubsection{Why the Analysis of Thinking Is Important}
``If you want to think well, you must understand at least the rudiments of thought, the most basic structures out of which all thinking is made. You must learn how to take thinking apart.''

\subsubsection{Analyzing the Logic of a Subject}
``When we understand the elements of reasoning, we realize that all subjects, all disciplines, have a fundamental logic defined by the structures of thought embedded within them. Therefore, to lay bare a subject's most fundamental logic, we should begin with these questions:
\begin{itemize}
	\item What is the main purpose or goal of studying this subject? What are people in this field trying to accomplish?
	\item What kinds of questions do they ask? What kinds of problems do they try to solve?
	\item What sorts of information or data do they gather?
	\item What types of inferences do they usually draw? What types of judgments do they typically make? (Judgments about $\ldots$)
	\item How do they go about gathering information in ways that are distinctive to this field?
	\item What are the most basic ideas, concepts, or theories in this field?
	\item What do professionals in this field take for granted or assume?
	\item How should studying this field affect my view of the world?
	\item What viewpoint is fostered in this field?
	\item What implications follow from studying this discipline? How are the products of this field used in everyday life? How might they be used in ways they are not currently?''
\end{itemize}

\subsection{Going Deeper $\ldots$}
``While most critical thinking concepts are intuitive, to integrate \& apply these concepts consistently \& rationally takes concerted effort, study, \& reflection. Just as professional athletes or musicians must practice to master their sport or art, so \fbox{too must thinkers practice to master their minds}. We invite you to return to our website often \& explore the resources available to assist you in developing \& cultivating your thinking.''

\subsubsection{The Critical Thinking Bookstore}
``\href{https://www.criticalthinking.org/bookstore/index.cfm}{Our online bookstore} houses numerous \href{https://www.criticalthinking.org/store/catalogs/books/214}{books \& teacher's manuals}, \href{https://www.criticalthinking.org/store/catalogs/thinkers-guides/224}{Thinker's Guides}, \href{https://www.criticalthinking.org/store/catalogs/videos/215}{videos}, \& other \href{https://www.criticalthinking.org/store/catalogs/materials/217}{educational materials}.''

\subsubsection{Learn from our Fellows \& Scholars}
``Watch our \href{https://www.criticalthinking.org/conference/index.cfm}{Event Calendar}, which provides an overview of all upcoming conferences \& academies hosted by the Foundation for Critical Thinking. Clicking an entry on the Event Calendar will bring up that event's details, \& the option to register.

For those interested in online learning, the Foundation offers \href{https://www.criticalthinking.org/pages/online-courses-for-instructors/574/}{accredited online courses in critical thinking} for both educators \& the general public, as well as an \href{https://www.criticalthinking.org/pages/online-critical-thinking-basic-concepts-test/679}{online test for evaluating basic comprehension of critical thinking concepts}. We are in the process of developing more online learning tools \& tests to offer the community.''

\subsubsection{Utilizing this Website}
``This website contain large amounts \href{https://www.criticalthinking.org/research/index.cfm}{research} \& an \href{https://www.criticalthinking.org/pages/index-of-articles/1021/}{online library of articles}, both of which are freely available to the public. We also invite you to become a member of the \href{https://www.criticalthinking.org/courses/OL_CT_Membership.cfm}{Critical Thinking Community}, where you will gain access to more tools \& materials.''

%------------------------------------------------------------------------------%

\section{\href{https://www.criticalthinking.org/pages/science-and-engineering/803}{The Foundation for Critical Thinking\texttt{/}Science \& Engineering}}
``While there are numerous resources on our website applicable to Science \& Engineering instruction, the following resources are among the most relevant to incorporating critical thinking concepts into the Science or Engineering classroom.''

\subsection{Engineering Instruction}
``Engineering increasingly attends to systems of systems, where the product of the engineer's intellect exhibits complex interactions with other systems, markets, technologies, the environment, \& society. Additionally, the workplace demands that the individual engineer continually develop, mastering new learning \& deal with increasing complexities. The thinking skills of our students \& young engineers provide the foundation for that growth, while in school \& in the workplace. When we explicitly target their thinking skills, we provide them leverage for learning both in class \& on the job.

``Critical Thinking'' can be an educational buzz-phrase which we presume implicit in rigorous programs. Or, substantively expressed, critical thinking becomes a ``system opening system,'' a lever for both cracking open both new domains \& intensifying insight into the web of connections that characterize engineering work. Generalizable critical thinking \textit{skills} \& \textit{dispositions} should guide professional reasoning through complex engineering questions \& issues, whether technological, commercial, environmental, ethical, or social.

Yet our students do not naturally think using the tools of critical thinking; they do not intuit the important questions they should be asking of themselves, teachers, colleagues, customers, or vendors, to either guide their understanding or refine their thinking. It is therefore essential that we foster, through engineering instruction, the skills, abilities \& traits of the disciplined mind.''

\subsection{Science Instruction}

\subsubsection{Complimentary Articles on Critical Thinking}
``The following pages on our website contain articles which, though not exclusively on the topic of science instruction, are none the less valuable \& applicable to any educational environment \& are therefore recommended reading for any science educator.
\begin{itemize}
	\item \href{https://www.criticalthinking.org/pages/critical-thinking-in-every-domain-of-knowledge-and-belief/698}{Critical Thinking in Every Domain of Knowledge \& Belief}
	\item \href{https://www.criticalthinking.org/articles/becoming-a-critic.cfm}{Becoming a Critic of Your Thinking}
	\item \href{https://www.criticalthinking.org/articles/ct-development-a-stage-theory.cfm}{Critical Thinking: A Stage Theory}
	\item \href{https://www.criticalthinking.org/articles/ct-identifying-targets.cfm}{Critical Thinking: Identifying the Targets}
	\item \href{https://www.criticalthinking.org/articles/glossary.cfm}{Glossary of Critical Thinking Terms}
	\item \href{https://www.criticalthinking.org/articles/helping-students-assess-their-thinking.cfm}{The Analysis \& Assessment of Thinking}
	\item \href{https://www.criticalthinking.org/articles/thinking-some-purpose.cfm}{The Role of Questions in Teaching, Thinking \& Learning}
	\item \href{https://www.criticalthinking.org/articles/universal-intellectual-standards.cfm}{Universal Intellectual Standards}
	\item \href{https://www.criticalthinking.org/articles/TRK12-using-intellectual-standards.cfm}{Using Intellectual Standards to Assess Student Reasoning}
	\item \href{https://www.criticalthinking.org/articles/distinguishing-inert-information.cfm}{Distinguishing Between Inert Information, Activated Ignorance, Activated Knowledge}
	\item \href{https://www.criticalthinking.org/articles/ct-distinguishing-inferencs.cfm}{Distinguished Between Interfaces \& Assumptions}
	\item \href{https://www.criticalthinking.org/articles/thinking-with-concepts.cfm}{Thinking with Concepts}
	\item \href{https://www.criticalthinking.org/articles/valuable-intellectual-traits.cfm}{Valuable Intellectual Traits}
\end{itemize}

%------------------------------------------------------------------------------%

\selectlanguage{english}
\chapter{Miscellaneous}
\selectlanguage{vietnamese}

\begin{itemize}
	\item \textbf{psychiatrist} [n] a doctor who studies \& treats mental illnesses.
	\item \textbf{psychoanalyst} [n] (also \textbf{analyst}) a person who treats patients using psychoanalysis.
\end{itemize}

\section{An Untrained \&\texttt{/}Thus (?) Failed Eidetiker: The Way I Remember}

\begin{remark}
	At the beginning, I am not so sure that this concept should be mentioned here, in the subject of psychology. But when I recalled back some pieces of my memory, I realize how serious \& devastated\footnote{\textsc{vi}: Hồi ức của 1 kẻ có trí nhớ điện tử. Cần phân biệt với bộ phim \href{https://www.imdb.com/title/tt0353969/}{IMDb\texttt{/}Memories of Murder} (2003), original title: Salinui chueok, i.e., Hồi ức kẻ sát nhân.} this ability has affected the development of my personality \& psychology in various aspects.
\end{remark}

\begin{definition}[\href{https://en.wikipedia.org/wiki/Eidetic_memory}{Wikipedia\texttt{/}eidetic memory}]
	``\emph{Eidetic memory} (more commonly called \emph{photographic memory} or \emph{total recall}) is the ability to recall an image from \href{https://en.wikipedia.org/wiki/Memory}{memory} with high precision for a brief period after seeing it only once, \& without using a \href{https://en.wikipedia.org/wiki/Mnemonic_device}{mnemonic device}.''
\end{definition}

\begin{remark}[\href{https://en.wikipedia.org/wiki/Eidetic_memory}{Wikipedia\texttt{/}eidetic memory}]
	``Although the terms \emph{eidetic memory} \& \emph{photographic memory} are popularly used interchangeably, they are also distinguished, with \emph{eidetic memory} referring to the ability to see an object for a few minutes after it is no longer present \& \emph{photographic memory} referring to the ability to recall pages of text or numbers, or similar, in great detail. When the concepts are distinguished, eidetic memory is reported to occur in a small number of children \& generally not found in adults, while true photographic memory has never been demonstrated to exist.'' \footnote{The word eidetic comes from the Greek word \textit{eidos} meaning ``visible form''.}
\end{remark}

\begin{question}
	Eidetic memory: A gift or a curse?
\end{question}

\subsection{Eidetic vs. Photographic}
From \href{https://en.wikipedia.org/wiki/Eidetic_memory#Eidetic_vs._photographic}{Wikipedia\texttt{/}eidetic memory\texttt{/}eidetic vs. photographic}:

``The terms \textit{eidetic memory} \& \textit{photographic memory} are commonly used interchangeably, but they are also distinguished. Scholar Annette Kujawski Taylor stated,
\begin{quotation}
	``In eidetic memory, a person has an almost faithful mental image snapshot or photograph of an event in their memory. However, eidetic memory is not limited to visual aspects of memory \& includes auditory memories as well as various sensory aspects across a range of stimuli associated with a visual image.''
\end{quotation}
Author Andrew Hudmon commented:
\begin{quotation}
	``Examples of people with a photographic-like memory are rare. Eidetic imagery is the ability to remember an image in so much detail, clarity, \& accuracy that it is as though the image were still being perceived. It is not perfect, as it is subject to distortions \& additions (like episodic memory) \& vocalization interferes with the memory.''
\end{quotation}
``Eidetikers'', as those who possess this ability are called, report a vivid \href{https://en.wikipedia.org/wiki/Afterimage}{after image} that lingers in the visual field with their eyes appearing to scan across the image as it is described. Contrary to ordinary mental imagery, eidetic images are externally projected, experienced as ``out there'' rather than in the mind. Vividness \& stability of the image begin to fade within minutes after the removal of the visual stimulus.

\href{https://en.wikipedia.org/wiki/Scott_Lilienfeld}{Lilienfeld} et al. stated,
\begin{quotation}
	``People with eidetic memory can supposedly hold a visual image in their mind with such clarity that they can describe it perfectly or almost perfectly $\ldots$, just as we can describe the details of a painting immediately in front of us with \fbox{near perfect accuracy}.''
\end{quotation}
By contrast, photographic memory may be defined as the ability to recall pages of text, numbers, or similar, in great detail, without the visualization that comes with eidetic memory. It may be described as the ability to briefly look at a page of information \& then recite it perfectly from memory. This type of ability--absolute recall of all events in a lifetime--has never been proven to exist.''\footnote{This appeared in the movie \href{https://www.imdb.com/title/tt0119217/}{Good Will Hunting} (1997) mentioned in the quotes section.}

\subsection{Prevalence}
From \href{https://en.wikipedia.org/wiki/Eidetic_memory#Prevalence}{Wikipedia\texttt{/}eidetic memory\texttt{/}prevalence}:

``\fbox{Eidetic memory is typically found only in young children, as it is virtually nonexistent in adults.} Hudmon stated, \textit{``Children possess far more capacity for eidetic imagery than adults, suggesting that a developmental change (e.g., acquiring language skills) may disrupt the potential for eidetic imagery.''}''

``It has been hypothesized that language acquisition \& verbal skills allow older children to think more abstractly \& thus rely less on \href{https://en.wikipedia.org/wiki/Visual_memory}{visual memory} systems. Extensive research has failed to demonstrate consistent correlations between the presence of eidetic imagery \& any cognitive, intellectual, neurological, or emotional measure.''

``A few adults have had phenomenal memories (not necessarily of images), but their abilities are also unconnected with their intelligence levels \& tend to be highly specialized. In extreme cases, like those of \href{https://en.wikipedia.org/wiki/Solomon_Shereshevsky}{Solomon Shereshevsky} \& \href{https://en.wikipedia.org/wiki/Kim_Peek}{Kim Peek}, memory skills can reportedly hinder social skills. Shereshevsky was \fbox{a trained \href{https://en.wikipedia.org/wiki/Mnemonist}{mnemonist}, not an eidetic memorizer}, \& there are no studies that confirm whether Kim Peek had true eidetic memory.''

\subsection{Skepticism}
From \href{https://en.wikipedia.org/wiki/Eidetic_memory#Skepticism}{Wikipedia\texttt{/}eidetic memory\texttt{/}skepticism}: [$\ldots$]

``Lilienfeld et al. stated:
\begin{quotation}
	``Some psychologists believe that eidetic memory reflects an unusually long persistence of the iconic image in some lucky people''. [$\ldots$] ``More recent evidence raises questions about whether any memories are truly photographic (Rothen, Meier \& Ward, 2012). Eidetikers' memories are clearly remarkable, but they are rarely perfect. Their memories often contain minor errors, including information that was not present in the original visual stimulus. So even eidetic memory often appears to be reconstructive''.
\end{quotation}
\href{https://en.wikipedia.org/wiki/Skeptical_movement}{Scientific skeptic} author \href{https://en.wikipedia.org/wiki/Brian_Dunning_(author)}{Brian Dunning} reviewed the literature on the subject of both eidetic \& photographic memory in 2016 \& concluded that there is ``a lack of compelling evidence that eidetic memory exists at all among healthy adults, \& no evidence that photographic memory exists. But there's a common theme running through many of these research papers, \& that's that the difference between ordinary memory \& \href{https://en.wikipedia.org/wiki/Exceptional_memory}{exceptional memory} appears to be one of degree.''''

\subsection{Trained Mnemonists}
From \href{https://en.wikipedia.org/wiki/Eidetic_memory#Trained_mnemonists}{Wikipedia\texttt{/}eidetic memory\texttt{/}trained mnemonists}:

``To constitute photographic or eidetic memory, the visual recall must persist without the use of mnemonics, expert talent, or other cognitive strategies. Various cases have been reported that rely on such skills \& are erroneously attributed to photographic memory.''

\begin{example}
	``An example of extraordinary memory abilities being ascribed to eidetic memory comes from the popular interpretations of \href{https://en.wikipedia.org/wiki/Adriaan_de_Groot}{Adriaan de Groot}'s classic experiments into the ability of \href{https://en.wikipedia.org/wiki/Chess}{chess} \href{https://en.wikipedia.org/wiki/Grandmaster_(chess)}{grandmaster} to memorize complex positions of chess pieces on a chessboard. Initially, Initially, it was found that these experts could recall surprising amounts of information, far more than nonexperts, suggesting eidetic skills. However, when the experts were presented with arrangements of chess pieces that could never occur in a game, their recall was no better than that of the nonexperts, suggesting that they had \fbox{developed an ability to organize certain types of information, rather than possessing innate eidetic ability}.
\end{example}
Individuals identified as having a condition known as \href{https://en.wikipedia.org/wiki/Hyperthymesia}{hyperthymesia} are able to remember very  intricate details of their own personal lives, but the ability seems not to extend to other, non-autobiographical information. They may have vivid recollections such as who they were with, what they were wearing, \& how they were feeling on a specific date many years in the past. Patients under study, e.g., \href{https://en.wikipedia.org/wiki/Jill_Price}{Jill Price}, show brain scans that resemble those with \href{https://en.wikipedia.org/wiki/Obsessive-compulsive_disorder}{obsessive-compulsive disorder}. In fact, Price's unusual autobiographical memory has been attributed as a byproduct of compulsively making journal \& diary entries. Hyperthymestic patients may additionally have depression\footnote{NQBH: a connection between eidetic memory \& depression.} stemming from the inability to forget unpleasant memories \& experiences from the past.\footnote{Exactly my case.} It is a misconception that hyperthymesia suggests any eidetic ability.\footnote{It seems to me that I possess both of these curses, although the latter is less obvious when I grow up: My memory is less sharp \& more messy (somehow the capacity of my memory seems to expand).}

Each year at the \href{https://en.wikipedia.org/wiki/World_Memory_Championships}{World Memory Championships}, the world's best memorizers compete for prizes. None of the world's best competitive memorizers has a photographic memory, \& no one with claimed eidetic or photographic memory has ever won the championship.''

\subsection{Notable Claims}
From \href{https://en.wikipedia.org/wiki/Eidetic_memory#Notable_claims}{Wikipedia\texttt{/}eidetic memory\texttt{/}notable claims}:

``Main article: \href{https://en.wikipedia.org/wiki/List_of_people_claimed_to_possess_an_eidetic_memory}{Wikipedia\texttt{/}List of people claimed to possess an eidetic memory}.

There are a number of individuals whose extraordinary memory has been labeled ``eidetic'', but it is not established conclusively whether they use \href{https://en.wikipedia.org/wiki/Mnemonic}{mnemonics} \& other, non-eidetic memory-enhancement.

\begin{example}
	`Nadia', who began \fbox{drawing realistically} at the age of 3, is \fbox{autistic} \& has been closely studied. During her childhood she produced highly precocious, repetitive drawings from memory, remarkable for being in perspective (which children tend not to achieve until at least adolescence) at the age of 3, which showed different perspectives on an image she was looking at. E.g., when at the age of three she was obsessed with horses after seeing a horse in a story book she generated numbers of images of what a horse should look like in any posture. She could draw other animals, objects, \& parts of human bodies accurately, but represented human faces as jumbled forms.'' \footnote{Cf. my untrained drawing ability compared to a trained adult when I was a boy.}
\end{example}

\begin{example}
	Others have not been thoroughly tested, though savant \href{https://en.wikipedia.org/wiki/Stephen_Wiltshire}{Stephen Wiltshire} can look at a subject once \& then produce, often before an audience, an accurate \& detailed drawing of it, \& has drawn entire cities from memory, based on single, brief helicopter rides; his 6-meter drawing of 305 square miles of New York City is based on a single 20-minute helicopter ride.
\end{example}

\begin{example}
	Another less thoroughly investigated instance is the art of \href{https://en.wikipedia.org/wiki/Winnie_Bamara}{Winnie Bamara}, an Australian indigenous artist of the 1950s.
\end{example}

\begin{question}
	Connection\emph{\texttt{/}}Correlation between eidetic memory \& gifted drawing ability?
\end{question}

\subsection{Quotes on Eidetic Memory}
\begin{itemize}
	\item In the movie \href{https://www.imdb.com/title/tt0289765/}{Red Dragon} (2002), I like the following conversation:
	\begin{quotation}
		Dr. Hannibal Lecter: \textit{``That's fascinating. You know I'd always suspected as much, you are an eidetiker.''}
		
		Will Graham: \textit{``I'm not psychic.''}
		
		Dr. Hannibal Lecter: \textit{``No, no, no, this is different; more akin to artistic imagination. You're able to assume the emotional point-of-view of other people, even those that scare or sicken you. It's a troubling gift, I should think.''}
	\end{quotation}
	\item In the movie \href{https://www.imdb.com/title/tt0119217/}{Good Will Hunting} (1997):
	\begin{quotation}
		\textit{``Do you have a photographic memory?''} [$\ldots$]
	\end{quotation}
\end{itemize}

\section{Psychology \& Scientists\texttt{/}Mathematicians}
``According to \href{https://en.wikipedia.org/wiki/Herman_Goldstine}{Herman Goldstine}, the mathematician \href{https://en.wikipedia.org/wiki/John_von_Neumann}{John von Neumann} was able to recall from memory every book he had ever read.'' -- \href{https://en.wikipedia.org/wiki/Eidetic_memory#Prevalence}{Wikipedia\texttt{/}eidetic memory\texttt{/}prevalence}

\section{Psychology \& Music}
Han Zimmer's masterpieces: $\ldots$

\section{Introversity \&\texttt{/}vs. Extroversity}

\section{Depression: The Unphysical Cancer}
Well, it will take me a really really long long time to beat this shit.

\section{Monomaniac: A Social Loser or A Lonely Wolf?}
Monomaniac - Kẻ độc hành.

\section{Rich Dad, Poor Dad}
I just realize: If I cannot teach my son to become a man, a real man, then I should not have him. ``Like father, like son''. If I cannot help my son get out of the life circle\footnote{\textsc{vi}: vòng lặp lẩn quẩn của cuộc đời.} of poor \& stupidity, then why should I have one?

\section{Undisputed Truth}
Mike Tyson's  autobiography \cite{Tyson_Sloman2013}:
\begin{quotation}
	``This book is dedicated to all the outcasts -- Everyone who has ever been mesmerized, marginalized, tranquilized, beaten down, \& falsely accused. \& incapable of receiving love.'' -- \cite[Dedication]{Tyson_Sloman2013}
\end{quotation}


\section{Miscellaneous}
Ask myself before doing anything literally:

\begin{question}[Decision question]
	Should I do it or not? If yes, why? If no, why?
\end{question}

\begin{question}[Self-study questions]
	What? Why? \& How?
\end{question}

\begin{question}
	What is the best status or feeling in life?
\end{question}
This question lies in the borderline between the fields of psychology \& philosophy. Should I move it to \cite{NQBH/philosophy}?

\begin{proof}[NQBH's personal answer]
	Concentration \& contributions.
\end{proof}

\begin{quotation}
	``He [G. H. Hardy] was, as I [C. P. Snow] later discovered, shy \& self-conscious\footnote{\textbf{self-conscious} [a] \textbf{1.} \textbf{self-conscious (about sth)} nervous\texttt{/}embarrassed about your appearance or what other people think of you; \textbf{2.} \textit{(often disapproving)} done in a way that shows you are aware of the effect that is being produced, \textit{opposite}: \textbf{unselfconscious}.} in all formal actions, \& had a dread of introductions. He just put his head down as it were in a butt of acknowledgment, \& without any preamble whatever began: $\ldots$'' [$\ldots$] ``I [C. P. Snow] half-guessed that he [G. H. Hardy] had a horror of persons, then prevalent in academic society, who devotedly studied the literature but had never played the game.'' [$\ldots$] ``He appeared to find the reply partially reassuring\footnote{\textbf{reassuring} [a] making you feel less worried or uncertain about something.}, \& went on to more tactical questions.'' [$\ldots$] ``As I had plenty of opportunities to realize in the future, Hardy had no faith in intuitions\footnote{\textbf{intuition} [n] \textbf{1.} [uncountable] the ability to know something by using your feelings rather than considering the facts; \textbf{2.} [countable] \textbf{intuition (that $\ldots$)} an idea or a strong feeling that something is true although you cannot explain why. \textsc{vi}: trực giác.} or impressions, his own or anyone else's. The only way to assess someone's knowledge, in Hardy's view, was to examine him. That went for mathematics, literature, philosophy, politics, anything you like. If the man had bluffed \& then wilted under the questions, that was his lookout. \fbox{1st things came 1st, in that brilliant \& concentrated mind.}'' [$\ldots$] ``Nothing else mattered. In the end he [G. H. Hardy] smiled with immense charm, with child-like openness, \& said that Fenner's (the university cricket ground) next season might be bearable after all, with the prospect of some reasonable conversation.'' -- \cite[Foreword, pp. 10--11]{Hardy1992}
	
	``I [C. P. Snow] don't know what the moral is. But it was a major piece of luck for me. This was intellectually the most valuable friendship of my life. His mind, as I have just mentioned, was brilliant \& concentrated: so much so that by his side anyone else's seemed a little muddy, a little pedestrian \& confused. He wasn't a great genius, as Einstein \& Rutherford were. He said, with his usual clarity\footnote{\textbf{clarity} [n] [uncountable] \textbf{1.} the quality of being expressed clearly; \textbf{2.} the ability to think about or understand something clearly; \textbf{3.} if a picture, substance or sound has clarity, you can see or hear it very clearly, or see through it easily.}, that if the word meant anything he was not a genius at all. At his best, he said, he was for a short time the 5th best pure mathematician in the world. Since this character was as beautiful \& candid\footnote{\textbf{candid} [a] \textbf{1.} saying what you think openly \& honestly; not hiding your thoughts; \textbf{2.} a \textbf{candid} photograph is one that is taken without the person in it knowing that they are being photographed.} as his mind, he always made the point that his friend \& collaborator Littlewood was an appreciably more powerful mathematician than he was, \& that his prot\'eg\'e\footnote{\textbf{prot\'eg\'e} [n] (feminine \textbf{prot\'eg\'ee}) \textit{(from French)} a young person who is helped in their career \& personal development by a more experienced person.} Ramanujan really had natural genius in the sense (though not to the extent, \& nothing like so effectively) that the greatest mathematicians had it.
	
	People sometimes thought he was under-rating himself, when he spoke of these friends. It is true that he was magnanimous\footnote{\textbf{magnanimous} [a] \textit{(formal)} kind, generous \& forgiving, especially towards an enemy or competitor.}, as far from envy as a man can be: but I think one mistakes his quality if one doesn't accept his judgment. I prefer to believe in his own statement in \textit{A Mathematician's Apology}, at the same time \fbox{so proud \& so humble}:
	\begin{quotation}
		`I still say to myself when I am depressed \& find myself forced to listen to pompous \& tiresome people, ``Well, I have done 1 thing you could never have done, \& that is to have collaborated with Littlewood \& Ramanujan on something like equal terms.'''
	\end{quotation}
	In any case, his precise ranking must be left to the historians of mathematics (though it will be an almost impossible job, since so much of his best work was done in collaboration). There is something else, thought, at which he was \fbox{clearly superior} to Einstein or Rutherford or any other great genius: \& that is at turning any work of the intellect\footnote{\textbf{intellect} [n] \textbf{1.} [uncountable, countable] the ability to think in a logical way \& understand things, especially at an advanced level; your mind; \textbf{2.} [countable] a very intelligent person.}, major or minor or sheer play, into a work of art. It was \fbox{that gift above all}, I think, which made him, almost without realizing it, purvey\footnote{\textbf{purvey} [v] \textit{(formal)} \textbf{purvey something} to supply food, services or information to people.} such intellectual delight\footnote{\textbf{delight} [n] \textbf{1.} [uncountable, singular] a feeling of great pleasure, \textsc{synonym}: \textbf{joy}; \textbf{2.} [countable] something that gives you great pleasure, \textsc{synonym}: \textbf{joy}.}. When \textit{A Mathematician's Apology} was 1st published, Graham Greene in a review wrote that along with Henry James's notebooks, this was the best account of what it was like to be a \fbox{\textit{creative artist}}\footnote{NQBH: a creative artist wannabe.}. Thinking about the effect Hardy had on all those round him, I believe that is the clue.'' -- \cite[Foreword, pp. 12--13]{Hardy1992}
\end{quotation}

%------------------------------------------------------------------------------%

\printbibliography[heading=bibintoc]
	
\end{document}