\documentclass{article}
\usepackage[backend=biber,natbib=true,style=authoryear]{biblatex}
\addbibresource{/home/nqbh/reference/bib.bib}
\usepackage{tocloft}
\renewcommand{\cftsecleader}{\cftdotfill{\cftdotsep}}
\usepackage[colorlinks=true,linkcolor=blue,urlcolor=red,citecolor=magenta]{hyperref}
\usepackage{algorithm,algpseudocode,amsmath,amssymb,amsthm,float,graphicx,mathtools}
\allowdisplaybreaks
\numberwithin{equation}{section}
\newtheorem{assumption}{Assumption}[section]
\newtheorem{conjecture}{Conjecture}[section]
\newtheorem{corollary}{Corollary}[section]
\newtheorem{definition}{Definition}[section]
\newtheorem{example}{Example}[section]
\newtheorem{lemma}{Lemma}[section]
\newtheorem{notation}{Notation}[section]
\newtheorem{principle}{Principle}[section]
\newtheorem{problem}{Problem}[section]
\newtheorem{proposition}{Proposition}[section]
\newtheorem{question}{Question}[section]
\newtheorem{remark}{Remark}[section]
\newtheorem{theorem}{Theorem}[section]
\usepackage[left=0.5in,right=0.5in,top=1.5cm,bottom=1.5cm]{geometry}
\usepackage{fancyhdr}
\pagestyle{fancy}
\fancyhf{}
\lhead{\small Sect.~\thesection}
\rhead{\small\nouppercase{\leftmark}}
\renewcommand{\sectionmark}[1]{\markboth{#1}{}}
\cfoot{\thepage}
\def\labelitemii{$\circ$}

\title{In Sheep's Clothing: Understanding \& Dealing with Manipulative People}
\author{George Simon, Jr.}
\date{\today}

\begin{document}
\maketitle
\tableofcontents
\vspace{5mm}
\begin{quotation}
	``[After reading \textit{In Sheep's Clothing}] I am beginning to reclaim my life, find my self-respect \& confidence.'' -- Marc, Virginia
	
	``After having read several books on several different self-help topics, psychology books, psychiatry books, etc., I MUST recommend you buy this one, 1st. It cuts straight through the bs -- neatly \& cleanly. I have bought copies of this book for friends \& can't recommend it enough.'' -- E. Adams, Online Purchaser
	
	``Don't Be Bossed-Around Ever Again!!! $\ldots$ \textit{In Sheep's Clothing: Understanding \& Dealing with Manipulative People} by George K. Simon, Jr., Ph.D., is a godsend to anyone who has ever questioned their own sanity while in any kind of relationship with a controlling \& manipulative person.'' -- The Aeolian Kid, Online Purchaser
	
	``Dr. Simon teaches the mechanics of popular tactics used by manipulators \& how you can identify \& thwart their attacks so that you control the outcome. This book helped me with a person that I have no choice but to see daily. After the end of every ``friendly'' conversation I felt depressed or insulted but could not figure out how this person was doing it. This book helped me to understand what was really happening. Dr. Simon's guidelines exposed this person \& [allowed me to take] control. Because this person knows [I] can be longer [be] controlled, I now have -- not a perfect relationship -- but one that's better than the alternative.'' -- A reader in Chicago
	
	``This book is like the secret decoder ring for the jumbled mess that is a manipulator's modus operandi. \textit{Do yourself a favor \& get this book now}.'' -- Christy, Missouri
	
	``It's sad that there are people out there that make life so much harder than it should have to be for others. Being able to identify such people in your life (both at home \& at work) is very important \& can be of invaluable help to (i) not go crazy oneself, \& (ii) take corrective action. Dr. Simon's book is written with amazing clarity. \textit{If you read only 1 book this year, read this one}.'' -- JA008, Online Purchaser
	
	``This is 1 of the best books I've ever read \& I would recommend it to anyone. It has redefined how I judge people \& helped me to become a stronger person. I used to be very naive \& unaware of people's ulterior motives, \& I have learned a tremendous amount from reading this book.'' -- S. Brescenti, Online Purchaser
	
	``This book makes it clear that evil is allowed free rein because of our ignorance of its nature. Simon shows us what seemingly mundane interactions that leave us perplexed may really be about. According to him, master manipulators leave us drained \& confused as we try to change them into the good person we want to believe they really are. I would add that the manipulators are just plain evil because evil requires lies, manipulation \& a weakening of the other's will through deception. Simon shows you how to recognize the signs \& what you can do about it. Good people are responsible for informing \& protecting themselves from the manipulators in society. This book is a necessary start.'' -- Kaye, a reader in New York state
	
	``Pithy \& often funny, George Simon takes the bluster \& obfuscation of overbearing, weasely bosses, nasty neighbors, \& obnoxious coworkers \& boils it down to show you the simple psychological strategies being used to imposed on your patience, good will, or even wallet. \textit{I have recommended this book to everyone I know \& bought extra copies for my kids} when they went out into the work world. Highly Recommended!'' -- C. MacCallum, Online Purchaser
\end{quotation}

\section{Acknowledgments}
``I am deeply grateful to my wife, Dr. Sherry Simon for her unfailing love, faith, understanding, patience, \& support. She is responsible for the title of this book \& was a valuable resource in helping me clarify my thoughts during its writing. I wish to thank Dr. Bruce Carruth for his critique of the original manuscript \& suggestions for making it more readable. I am deeply indebted to the work of Dr. Theodore Millon. His comprehensive approach to understanding personality has not only influenced my thinking on the subject but also proved invaluable in my efforts to help people change. I owe a supreme debt to the many individuals willing to share with me their experiences with manipulative people. They taught me much \& enriched my life. This book, in large measure, is a tribute to their courage \& support. I am most appreciative of the validation, support, \& enriching input consistently afforded me by workshop attendees. They have helped me clarity, refine, \& enhance 1 of my principal missions in life. Words cannot express the gratitude I have for the thousands of readers who have kept this book in the active lists of online booksellers \& retail outlets for over 15 years. The many emails, blog posts, \& letters readers have sent helped me make necessary updates \& changes to this Revised Edition. I have attempted to honor the considerable feedback I continue to receive by expanding the discussion of key concepts as well as introducing important new content for this newly revised edition. Finally, I want to thank Roger Armbrust \& Ted Parkhurst of Parkhurst Brothers, Inc., Publishers. Ted encouraged me at the outset \& was there when I needed him; Roger's grace \& presence have only benefited my work and readers.'' -- \cite[p. 7]{Simon2010}

%------------------------------------------------------------------------------%

\section*{Preface}
``Whether it's the supervisor who claims to support you while thwarting every opportunity you have to get ahead, the co-worker who quietly undermines you to gain the boss's favor, the spouse who professes to love \& care about you but seems to control your life, or the child who always seems to know just which buttons to push in order to get their way, manipulative people are like the proverbial wolf in sheep's clothing. On the surface they can appear charming \& genial. But underneath, they can be ever so calculating \& ruthless. Cunning \& stable, they prey on your weaknesses \& use clever tactics to gain advantage over you. They're the kind of people who fight hard for everything they want but do their best to conceal their aggressive intentions. That's why I call them \textit{covert-aggressive personalities}.8

As a clinical psychologist in private practice, I began to focus on the problem of covert aggression over 20 years ago. I did so because the depression, anxiety, \& feelings of insecurity that initially led several of my patients to seek help eventually turned out to be in some way linked to their relationship with a manipulative person. I've counseled not only the victims of covert-aggression, but also manipulators themselves experiencing distress because their usual ways of getting their needs met \& controlling other weren't working anymore. My work has given me an appreciation for how widespread problem of manipulative behavior is \& the unique emotional stress it can bring to a relationship.

The scope of the problem of covert-aggression seems self-evident. Most of us know at least 1 manipulative person. \& hardly a day goes by that we don't read in the newspaper or hear a broadcast about someone who managed to exploit or ``con'' many before fate shed some light on their true character. There's the tele-evangelist who preached love, honesty, \& decency while cheating on his wife \& fleecing his flock, the politician, sworn to ``public service,'' caught lining his pockets, or the spiritual ``guru'' who even managed to convince most of his followers that he was God incarnate while sexually exploiting their children \& subtly terrorizing those who challenged him. The world, it seems full of manipulators.

Although the extreme wolves in sheep's clothing that make headlines grab our attention \& pique our curiosity about what makes such people ``tick,'' most of the covertly aggressive people we are likely to encounter are not these larger-than-life characters. Rather, they are the subtly underhanded, backstabbing, deceptive, \& conniving individuals we may work with, associate with, or possibly even live with. \& they can make life miserable. They cause us grief because we find it so hard to truly understand them \& even harder to deal with them effectively.

When victims of covert-aggression 1st seek help for their emotional distress, they usually have little insight into why they feel so bad. They only know that they feel confused, anxious, or depressed. Gradually, however, they relate how dealing with a certain person in their lives makes them feel crazy. They don't really trust them but can't pinpoint why. They get mad at them but for some reason end up feeling guilty themselves. They confront them about their behavior, only to wind up on the defensive. They get frustrated because they find themselves frequently giving in when they really wanted to stand ground, saying ``yes'' when they mean to say ``no,'' \& becoming depressed because nothing they try seems to make things better. In the end, dealing with this person always leaves them feeling confused, exploited \& abused. After exploring the issues in therapy for a while, they eventually come to realize how much of their unhappiness is the direct result of their constant but fruitless attempts to understand, deal with, or control their manipulator's behavior.

Despite the fact that many of my patients are intelligent, resourceful individuals with a fair understanding of traditional psychological principles, most of the ways they tried to understand \& cope with their manipulator's behavior weren't getting them anywhere, \& some of the things they tried only seemed to make matters worse. Moreover, none of the ways that I initially tried to help made any real difference. Having an eclectic training background, I tried all sorts of different therapies \& strategies, all of which seemed to help the victims feel a little better, but none seemed to empower them enough to really change the nature of their relationship with their manipulator. Even more disconcerting was the fact that none of the approaches I tried was effective at all with the manipulators. Realizing that something must be fundamentally wrong with the traditional approaches to understanding \& dealing with manipulative people, I began to carefully study the problem in the hope of developing a practical, more effective approach.

In this book I would like to introduce you to a new way of understanding the character of manipulative people. I believe the perspective I will offer describes manipulators \& labels their behavior more accurately than many other approaches. I'll explain what covert-aggression is \& why I believe it's at the heart of most interpersonal manipulation. I'll focus some needed attention on dimensions of personality that are too often ignored by traditional perspectives. The framework I will be advancing challenges some of the more common assumptions we make about why people act the way they do \& explains why some of the most widely-held beliefs about human nature tend to set us up for victimization by manipulators.

I have 3 objectives to fulfill in this book. My 1st is to fully acquaint you the nature of disturbed characters as well as the distinctive character of the covertly aggressive personality. I'll discuss the characteristics of aggressive personality types in general \& outline the unique characteristics of the covert-aggressive personality. I'll present several vignettes, based on real cases \& situations, that will help you get the ``flavor'' of this personality type as well as illustrate how manipulative people operate. Being able to recognize a wolf in sheep's clothing \& knowing what to expect from this kind of person is the 1st step in avoiding being victimized by them.

My 2nd objective is to explain precisely how covertly aggressive people managed to deceive, manipulate, \& ``control'' others. Aggressive \& covertly aggressive people use a select group of interpersonal maneuvers or tactics to gain advantage over others. Becoming more familiar with these tactics really helps a person recognize manipulative behavior \textit{at the time it occurs}, \& makes it easier, therefore, to avoid being victimized. I'll also discuss the characteristics many of us possess that can make us unduly vulnerable to the tactics of manipulation. Knowing what aspects of your own character a manipulator is most likely to exploit is another important step in avoiding victimization.

My final objective is to outline the specific steps anyone can take to deal more effectively with aggressive \& covertly aggressive personalities. I'll present some general rules for redefining the rules of engagement with these kinds of individuals \& describe some specific tools of personal empowerment that can help a person break the self-defeating cycle of trying to control their manipulator \& becoming depressed in the process. Using these tools makes it more likely that a 1-time victim will invest their energy where they really have power -- in their own behavior. Knowing how to conduct yourself in a potentially manipulative encounter is crucial to becoming less vulnerable to a manipulator's ploys \& asserting greater control over your own life.

I have attempted to write this book in a manner that is serious \& substantial yet straightforward \& readily understandable. I have written it for the general public as well as the mental health professional, \& I hope both will find it useful. By adhering to many traditional assumptions, labeling schemes, \& intervention strategies, therapists sometimes hold \& inadvertently reinforce some of the same misconceptions that their patients harbor about the character \& behavior of manipulators that inevitably lead to continued victimization. I offer a new perspective in the hope of helping individuals \& therapists alike avoid \textit{enabling} manipulative behavior.'' -- \cite[pp. 8--12]{Simon2010}

%------------------------------------------------------------------------------%

\section*{Author's Note on the Revised Edition}
``Since this book's 1st wide publication in 1996, I have received literally hundreds of calls, letters, \& emails, \& heard countless testimonials \& comments at workshops from individuals whose lives were changed merely by being exposed to \& adopting a new perspective on understanding human behavior. A common theme voiced by readers \& workshop attendees is that once they dispelled old myths \& came to view problem behaviors in a different light, they could see clearly that what their intuition had told them all along was correct, \& thus felt validated. A similar phenomenon has held true for mental health professionals attending the many training seminars I have given. Once they abandoned their old notions about why their clients do the things they do, they were better able to help them \& their significant others. I had already been doing workshops for 10 years before writing \textit{In Sheep's Clothing}. At that time, only a handful of theorists, researchers, \& writers were recognizing the need for a new perspective on understanding \& dealing with disturbed characters (e.g., Stanton Samenow, Samuel Yochelson, Robert Hare). What professionals today call the \textit{cognitive-behavioral} approach was in its infancy. The early research on character disturbance inspired me \& helped me validate my own observations. Today an increasing number of professionals are recognizing the problem of character disturbance \& using cognitive-behavioral methods to diagnose \& treat it.

We live in an age radically different from that in which the classical theories of psychology \& personality were developed. For the most part, truly pathological degrees of neurosis are quite rare, \& problematic levels of character disturbance are increasingly commonplace. It's a pervasive societal problem about which all of us would do well to expand our awareness. During the last 15 years, my experience working with disturbed characters of all types has grown immensely, as has the body of research. So, I have included in this edition an expanded discussion on the problem of character disturbance in general \& what sets the disturbed character apart from your garden-variety neurotic.

I am deeply grateful for the excellent word-of-mouth support responsible for transforming a once small, independent work into a best seller enjoying ever-increasing popularity even after 15 years. I sincerely hoped this revised edition will provide you with all the information \& tools you need to better understand \& deal with the manipulative people in your life.

\textit{George K. Simon, Jr., Ph.D.}, Jan 2010'' -- \cite[p. 14]{Simon2010}

%------------------------------------------------------------------------------%

\begin{center}\Large\sf
	\textbf{Part I: Understanding Manipulative Personalities}
\end{center}

\section*{Introduction: Covert-Aggression: The Heart of Manipulation}

\subsection{A Common Problem}
``Perhaps the following scenarios will sound familiar. A wife tries to sort out her mixed feelings. She's mad at her husband for insisting their daughter make all A's. But she doubts she has the right to be mad. When she suggested that given her appraisal of their daughter's abilities, he might be making unreasonable demands, his comeback, ``Shouldn't \textit{any} good parent want their child to do well \& succeed in life?'' made her feel like the insensitive one. In fact, whenever she confronts him, she somehow ends up feeling like the bad guy herself. When she suggested there might be more to her daughter's recent problems, \& that the family might do well to seek counseling, his retort ``Are you saying I'm psychiatrically disturbed?'' made her feel guilty for bringing up the issue. She often tries to assert her point of view, but always ends up giving-in to his. Sometimes, she thinks the problem is him, believing him to be selfish, demanding, intimidating, \& controlling. But this is a loyal husband, good provider, \& a respected member of the community. By all rights she shouldn't resent him. Yet, she does. So, she constantly wonders if there isn't something wrong with her.

A mother tries desperately to understand her daughter's behavior. No young girl, she thought, would threaten to leave home, say things like ``Everybody hates me'' \& ``I wish I were never born,'' unless she were very insecure, afraid, \& probably depressed. Part of her thinks her daughter is still the same child who used to hold her breath until she turned blue or threw tantrums whenever she didn't get her way. After all, it seems she only says \& does these things when she's about to be disciplined or she's trying to get something she wants. But a part of her is afraid to believe that. ``What if she really believes what she's saying?'' she wonders. ``What if I've really done something to hurt her \& I just don't realize it?'' she worries. She hates to feel ``bullied'' by her daughter's threats \& emotional displays, but she can't take the chance her daughter might really be hurting -- can she? Besides, children just don't act this way unless they really feel insecure or threatened in some way underneath it all -- do they?'' -- \cite[pp. 17--18]{Simon2010}

\subsection{The Heart of the Problem}
``Neither victim in the preceding scenarios trusted their ``gut'' feelings. Unconsciously, they felt on the defensive, but consciously they had trouble seeing their manipulator as merely a person on the offensive. On 1 hand, they felt like the other person was trying to get the better of them. On the other, they found no objective evidence at the time to back-up their gut-level hunch. They ended up feeling crazy.

They're not crazy. The fact is, people fight almost all the time. \& manipulative people are expert at fighting in subtle \& most undetectable ways. Most of the time, when they're trying to take advantage or gain the upper hand, you don't even know you're in a fight until you're well on your way to losing. When you're being manipulated, chances are someone is fighting with you for position, advantage, or gain, but in a way that's difficult to readily see. Covert-aggression is at the heart of most manipulation.'' -- \cite[p. 18]{Simon2010}

\subsection{The Nature of Human Aggression}
``Our instinct to fight is a close cousin of our survival instinct.\footnote{Storr, A., \textit{Human Destructiveness}, (Ballantine, 1991), pp. 7-17.} Most everyone ``fights'' to survive \& prosper, \& \textit{most} of the fighting we do is neither physically violent nor inherently destructive. Some theorists have suggested that only when this most basic instinct is severely frustrated does our aggressive drive have the potential to be expressed violently.\footnote{Storr, A., \textit{Human Destructiveness}, (Ballantine, 1991), p. 21.} Others have suggested that some rare individuals seem to be predisposed to aggression -- even violent aggression, despite the most benign circumstances. But whether extraordinary stressors, genetic predispositions, reinforced learning patterns, or some combination of these are at the root of violent aggression, most theorists agree that aggression per se \& destructive violence are not synonymous. In this book, the term aggression will refer to the forceful energy we all expend in our daily bids to survive, advance ourselves, secure things we believe will bring us some kind of pleasure, \& remove obstacles to those ends.

People do a lot more fighting in their daily lives than we have ever been willing to acknowledge. The urge to fight is fundamental \& instinctual. Anyone who denies the instinctual nature of aggression has either never witnessed 2 toddlers struggling for possession of the same toy, or has somehow forgotten this archetypal scene. Fighting is a big part of our culture, also. From the fierce partisan wrangling that characterizes representative government, to the competitive corporate environment, to the adversarial system of our judicial system, much fighting is woven into our societal fabric. We sue one another, divorce each other, battle with one another over our children, compete for jobs, \& struggle with each other to advance certain goals, values, beliefs \& ideals. The psychodynamic theorist Alfred Adler noted many years ago that we also forcefully strive to assert a sense of social superiority.\footnote{Adler, A., \textit{Understanding Human Nature}, (Fawcett World Library, 1954), p. 178.} Fighting for personal \& social advantage, we jockey with one another for power, prestige, \& a secure social ``niche.'' Indeed, we do so much fighting in so many aspects of our lives I think it fair to say that when human beings aren't making some kind of love, they're likely to be waging some kind of war.

Fighting is not inherently wrong or harmful. Fighting openly \& fairly for our legitimate needs is often necessary \& constructive. When we fight for what we truly need while respecting the rights \& needs of others \& taking care not to needlessly injure them, our behavior is best labeled \textit{assertive}, \& assertive behavior is 1 of the most healthy \& necessary human behaviors. It's wonderful when we learn to assert ourselves in the pursuit of personal needs, overcome unhealthy dependency \& become self-sufficient \& capable. But when we fight unnecessarily, or with little concern about how others are being affected, our behavior is most appropriately labeled \textit{aggressive}. In a civilized world, undisciplined fighting (aggression) is almost always a problem. The fact that we are an aggressive species doesn't make us inherently flawed or ``evil,'' either. Adopting a perspective advanced largely by Carl Jung,\footnote{Jung, C. G., 1953, \textit{Collected Works of}, Vol. 7, p.25. H. Read, M. Fordham and G. Adler, eds. New York: Pantheon.} I would assert that the evil that sometimes arises from a person's aggressive behavior necessarily stems from his or her failure to ``own'' \& discipline this most basic human instinct.'' -- \cite[pp. 18--20]{Simon2010}

\subsection{2 Important Types of Aggression}
``2 of the most fundamental types of fighting (others, such as reactive vs. predatory or instrumental aggression) will be discussed later are \textit{overt} \& \textit{covert} aggression. When you're determined to have your way or gain advantage \& you're open, direct, \& obvious in your manner of fighting, your behavior is best labeled overtly aggressive. When you're out to ``win,'' get your way, dominate, or control, but are subtle, underhanded, or deceptive enough to hide your true intentions, your behavior is most appropriately labeled covertly aggressive. Concealing overt displays of aggression while simultaneously intimidating others into backing-off, backing-down, or giving-in is a very powerful manipulative maneuver. That's why \textit{covert-aggression is most often the vehicle for interpersonal manipulation}.'' -- \cite[p. 20]{Simon2010}

\subsection{Covert \& Passive-Aggression}
``I often hear people say something is being ```passive-aggressive'' when they're really trying to describe covertly aggressive behavior. Covert \& passive-aggression are both indirect ways to aggress but they're most definitely not the same thing. Passive-aggression is, as the term implies, aggressing though passivity. Examples of passive-aggression are playing the game of emotional ``get-back'' with someone by resisting cooperation with them, giving them the ``silent treatment,'' pouting or whining, not so accidentally ``forgetting'' something they wanted you to do because you're angry \& didn't really feel like obliging them, etc. In contrast, covert aggression is very \textit{active}, albeit veiled, aggression. When someone is being covertly aggressive, they're using calculating, underhanded means to get what they want or manipulate the response of others while keeping their aggressive intentions under cover.'' -- \cite[pp. 20--21]{Simon2010}

\subsection{Acts of Covert-Aggression vs. Covert-Aggressive Personalities}
``Most of us have engaged in some sort of covertly aggressive behavior from time to time but that doesn't necessarily make someone a covert-aggressive or manipulative personality. An individual's personality can be defined by the way he or she habitually perceives, relates to \& interacts with others \& the world at large.\footnote{Millon, T. \textit{Disorders of Personality}, (Wiley-Interscience, 1981), p. 4.} It's the distinctive interactive ``style'' or relatively engrained way a person prefers to deal with a wide variety of situations \& to get the things they want in life. Certain personalities can be ever so ruthless in their interpersonal conduct while concealing their aggressive character or perhaps even projecting a convincing, superficial charm. These covert-aggressive personalities can have their way with you \& look good in the process. They vary in their degree of ruthlessness \& character pathology. But because the more extreme examples can teach us much about the process of manipulation in general, this book will pay special attention to some of the more seriously disturbed covert-aggressive personalities.'' -- \cite[pp. 21--22]{Simon2010}

\subsection{The Process of Victimization}
``For a long time, I wondered why manipulation victims have a hard time seeing what really goes on in manipulative interactions. At 1st, I was tempted to fault them. But I've learned that they get hoodwinked for some very good reasons:
\begin{enumerate}
	\item A manipulator's aggression is not obvious. We might intuitively sense that they're trying to overcome us, gain power, or have their way, \& find ourselves \textit{unconsciously} intimidated. But because we can't point to clear, objective evidence they're aggressing against us, we can't readily validate our gut-level feelings.
	\item The tactics that manipulators frequently use are powerful deception techniques that make it hard to recognize them as clever ploys. They can make it seem like the person using them is hurting, caring, defending, or almost anything but fighting for advantage over us. Their explanations always make just enough sense to make another doubt his or her gut hunch that they're being taken advantage of or abused. Their tactics not only make it hard for a person to consciously \& objectively know their manipulator is fighting to overcome, but also simultaneously keep the victim unconsciously on the defensive. This makes the tactics highly effective psychological 1-2 punches. It's hard to think clearly when someone has you emotionally unnerved, so you're less likely to recognize the tactics for what they really are.
	\item All of us have weaknesses \& insecurities that a clever manipulator might exploit. Sometimes, we're aware of these weaknesses \& how someone might use them to take advantage of us. E.g., I hear parents say things like: ``Yeah, I know I have a big guilt button.'' But at the time their manipulative child is busily pushing that button, they can easily forget what's really going on. Besides, sometimes we're unaware of our biggest vulnerabilities. Manipulators often know us better than we know ourselves. They know what buttons to push, when to do so \& how hard to press. Our lack of self-awareness can easily set us up to be exploited.
	\item What our intuition tells us a manipulator is really like challenges everything we've been taught to believe about human nature. We've been inundated with a psychology that has us viewing people with problems, at least to some degree, as afraid, insecure or ``hung-up.'' So, while our gut tells us we're dealing with a ruthless conniver, our head tells us they must be really frightened, wounded, or self-doubting ``underneath.'' What's more, most of us generally hate to think of ourselves as callous \& insensitive people. We hesitate to make harsh or negative judgments about others. We want to give them the benefit of the doubt \& believe they don't really harbor the malevolent intentions we suspect. We're more apt to doubt \& blame ourselves for daring to believe what our gut tells us about our manipulator's character.'' -- \cite[pp. 22--23]{Simon2010}
\end{enumerate}

\subsection{Recognizing Aggressive Agendas}
``Accepting how fundamental it is for people to fight for the things they want \& becoming more aware of the subtle, underhanded ways people can \& do fight in so many of their daily endeavors \& relationships can be very consciousness-expanding. Learning to recognize an aggressive move when somebody makes one \& learning how to handle oneself in any of life's many battles has turned out to be the most empowering experience for the manipulation victims with whom I've worked. It's how they eventually freed themselves from their manipulator's dominance \& control \& gained a much-needed boost to their own sense of self-esteem.

Recognizing the inherent aggression in manipulative behavior \& becoming more aware of the slick, surreptitious ways that manipulative people prefer to aggress against us is extremely important. Not recognizing \& accurately labeling their subtly aggressive moves causes most people to misinterpret the behavior of manipulators \&, therefore, fail to respond to them in an appropriate fashion. Recognizing when \& how manipulators are fighting with you is fundamental to fairing well in any kind of encounter with them.

Unfortunately, mental health professionals \& lay persons alike often fail to recognize the aggressive agendas \& actions of others for what they really are. This is largely because we've been pre-programmed to believe that people only exhibit problem behaviors when they're ``troubled'' inside or anxious about something. We've also been taught that people aggress only when they're attacked in some way. So, even when our gut tells us that somebody is attacking us \& for no good reason, or merely trying to overpower us, we don't readily accept the notions. We usually start to wonder what's bothering the person so badly ``underneath it all'' that's making them act in such a disturbing way. We may even wonder what we may have said or done that ``threatened'' them. We may try to analyze the situation to death instead of simply responding to the attack. We almost never think that the person is simply fighting to get something they want, to have their way with us, or gain the upper hand. \&, when we view them as primarily hurting in some way, we strive to understand as opposed to taking care of ourselves.

Not only do we often have trouble recognizing the ways people aggress, but we also have difficulty discerning the distinctly aggressive character of some personalities. The legacy of Sigmund Freud's work has a lot to do with this. Freud's theories (\& the theories of others who expanded on his work) heavily influenced the field of psychology \& related social sciences for a long time. The basic tenets of these classical (psychodynamic) theories \& their hallmark construct, \textit{neurosis}, have become fairly well etched in the public consciousness, \& many psychodynamic terms have intruded into common parlance. These theories also tend to view \textit{everyone}, at least to some degree, as \textit{neurotic}. Neurotic individuals are overly inhibited people who suffer unreasonable \& excessive anxiety (i.e., non-specific fear), guilt, \& shame when it comes to acting on their basic instincts or trying to gratify their basic wants \& needs. The malignant impact of over-generalizing Freud's observations about a small group of overly inhibited individuals into a broad set of assumptions about the causes of psychological ill-heath in everyone cannot be overstated.\footnote{Torrey, F., \textit{Freudian Fraud}, (Harper Collins, 1992), p. 257.} But these theories have so permeated our thinking about human nature, \& especially our theories of personality, that when most of us try to analyze someone's character, we automatically start thinking in terms of what fears might be ``hanging them up,'' what kinds of ``defenses'' they use \& what kinds of psychologically ``threatening'' situations they may be trying to ``avoid.'''' -- \cite[pp. 23--25]{Simon2010}

\subsection{The Need for a New Psychological Perspective}
``Classical theories of personality were developed during an extremely repressive time. If there were a motto for the Victorian era, it would be: ``Don't even think about it!'' In such times, one would expect neurosis to be more prevalent. Freud treated individuals who were so riddled with excessive shame \& guilt about their primal urges that some went ``hysterically'' blind so they wouldn't run the risk of consciously laying lustful eyes on the objects of their desire. Times have certainly changed. Today's social climate is far more permissive. If there were a motto for our time, it would be as the once popular TV commercial exhorted: ``Just do it!'' Many of the problems coming to the attention of mental health professionals these days are less the result of an individual's unreasonable fears \& inhibitions \& more the result of the deficient self-restraint a person has exercised over his\texttt{/}her basic instincts. More simply, therapists are increasingly being asked to treat individuals suffering from too little as opposed to too much neurosis (i.e., individuals with some type of \textit{character disturbance}). As a result, classical theories of personality \& their accompanying prescription for helping troubled persons achieve greater psychological health have proved to be of limited value when working with many of today's disturbed characters.

Some mental health professionals may need to overcome significant biases in order to better recognize \& deal with aggressive or covertly aggressive behavior. Therapists who tend to see any kind of aggression not as a problem in itself but as a ``symptom'' of an underlying inadequacy, insecurity or unconscious fear, may focus so intently on their patient's supposed ``inner conflict'' that they overlook the aggressive behaviors most responsible for problems. Therapists whose training overly indoctrinated them in the theory of neurosis may ``frame'' the problems presented them incorrectly. They may, e.g., assume that a person who all their life has aggressively pursued independence, resisted allegiance to others, \& taken what they could from relationships without feeling obliged to give something back must necessarily be ``compensating'' for a ``fear'' of intimacy. In other words, they will view a hardened, abusive fighter as a terrified runner, thus misperceiving the core reality of the situation.

It's neither appropriate nor helpful to over-generalize the characteristics of neurotic personalities in the attempt to describe \& understand all personalities. We need to stop trying to define \textit{every} type of personality by their greatest fears of the principal ways they ``defend'' themselves. We need a completely different theoretical framework if we are to truly understand, deal with, \& treat the kinds of people who fight too much as opposed to those who cower or ``run'' too much. I will present just such a framework in Chap. 1. I will introduce several aggressive personality types whose psychological makeup differs radically from those of the more neurotic personalities. It is within this framework that you will be better able to understand the nature of disturbed characters in general as well as the distinctive character of the manipulative people I call covert-aggressive personalities. I hope to present this new perspective not only in a style readily digestible by the lay reader trying to understand \& cope with a difficult situation but also in a manner that should prove useful to mental health professionals attempting therapeutic interventions.'' -- \cite[pp. 25--26]{Simon2010}

%------------------------------------------------------------------------------%

\section{Aggressive \& Covert-Aggressive Personalities}
``Understanding the true character of manipulative people is the 1st step in dealing more effectively with them. In order to know what they're really like, we have to view them within an appropriate context. In this chapter, I hope to present a framework for understanding personality \& character that will help you distinguish manipulators from other personality types \& give you an increased ability to identify a wolf in sheep's clothing when you encounter one.'' -- \cite[p. 27]{Simon2010}

\subsection{Personality}
``The term personality derives from the Latin word ``persona,'' which means ``mask.'' In the ancient theater, when actors were only men, \& when the art of conveying emotions through dramatic techniques had not fully evolved, female characters \& various emotions were portrayed through the use of masks. Classical theorists, who conceptualized personality as the social fa\c{c}ade or ``mask'' a person wore to hide the ``true self,'' adopted the term. The classical definition of personality, however, has proven to be quite limiting.

Personality can also be defined as the unique manner that a person develops of perceiving, relating to \& interacting with others \& the world at large.\footnote{Millon, T. \textit{Disorders of Personality}, (Wiley-Interscience, 1981), p. 4.} Within this model of personality, biology plays a part (e.g., genetic, hormonal influences, brain biochemistry), as does temperament, \& of course, the nature of a person's environment \& what he or she has learned from past experiences are big influences, also. All of these factors dynamically interact \& contribute to the distinctive ``style''\footnote{Millon, T. \textit{Disorders of Personality}, (Wiley-Interscience, 1981), p. 4.} a person develops over time in dealing with others \& coping with life's stressors in general. A person's interpersonal interactive ``style'' or personality appears a largely stable characteristic that doesn't moderate much with time \& generalizes across a wide variety of situations.'' -- \cite[pp. 27--28]{Simon2010}

\subsection{Character}
``Everyone's unique style of relating to others has social, ethical \& moral ramifications. The aspect of someone's personality that reflects how they accept \& fulfill their social responsibilities \& how they conduct themselves with others has sometime been referred to as \textit{character}.\footnote{Millon, T. \textit{Disorders of Personality}, (Wiley-Interscience, 1981), p. 6.} Some use the terms character \& personality synonymously. But in this book, the term character will refer to those aspects of an individual's personality that reflect the extent to which they have developed personal integrity \& a commitment to responsible social conduct. Persons of sound character temper their instinctual drives, moderate important aspects of their conduct, \& especially, discipline their aggressive tendencies in the service of the greater social good.'' -- \cite[p. 28]{Simon2010}

\subsection{Some Basic Personality Types}
``Volumes of clinical literature have been written on the various personality types. A discussion of all of the personality types is beyond the scope of this book. However, I find it particularly useful to distinguish between 2 basic dimensions of personality that occupy positions on opposite ends of a continuum that reflects how an individual deals with the challenges of life.

As goal-directed creatures, we all invest considerable time \& energy trying to get the things we think will help us to prosper or bring us some kind of pleasure. Running into obstacles or barriers to what we want is the essence of human conflict. Now, there are fundamentally 2 things a person can do when running up against an obstacle to something they want. They can be so overwhelmed or intimidated by the resistance they encounter or so unsure of their ability to deal with it effectively, that they fearfully retreat. Alternatively, they can directly challenge the obstacle. If they are confident enough in their fighting ability \& tenacious enough in their temperament, they might try to forcefully remove or overcome whatever stands between them \& the object of their desire.

Submissive personalities \textit{habitually} \& \textit{excessively} retreat from potential conflicts. They doubt their abilities \& are excessively afraid to take a stand. Because they ``run'' from challenges too often, they deny themselves opportunities to experience success. This pattern makes it hard for them to develop a sense of personal competence \& achieve self-reliance. Some personality theorists describe these individuals as \textit{passive-dependent}\footnote{Millon, T. \textit{Disorders of Personality}, (Wiley-Interscience, 1981), p. 91.} because their passivity largely leads them to become overly dependent upon others to do their fighting for them. Feeling inadequate, they all too readily submit to the will of those they view as more powerful or more capable than themselves.

In contrast, aggressive personalities are overly prone to fight in any potential conflict. Their main objective in life is ``winning'' \& they pursue this objective with considerable passion. They forcefully strive to overcome, crush, or remove any barriers to what they want. They seek power ambitiously \& use it unreservedly \& unscrupulously when they get it. They always strive to be ``on top'' \& in control. They accept challenges readily. Whether their faith in their ability to handle themselves in most conflicts is well-founded or not, they tend to be overly self-reliant or emotionally \textit{independent}.'' -- \cite[pp. 28--29]{Simon2010}

\subsection{Neurotic \& Character-Disordered Personalities}
``There are 2 other important dimensions of personality that represent opposite ends on a different continuum. Personalities who are excessively uncertain about how to cope \& excessively anxious when they attempt to secure their basic needs have often been called \textit{neurotic}. The inner emotional turmoil a neurotic personality experiences most often arises from ``conflicts'' between their basic instinctual drives \& their qualms of conscience. As a general rule, therefore, Scott Peck's point in \textit{The Road Less Traveled} that neurotics suffer from too much conscience is correct.\footnote{Peck, M. S., \textit{The Road Less Traveled}, (Simon \& Schuster, 1978), pp. 35-36.} These individuals are too afraid to seek satisfaction of their needs because they're overly riddled with guilt or shame when they do. In contrast, \textit{character-disordered} personalities lack self-restraint when it comes to acting upon their primal urges. They're \textbf{\textit{not bothered enough}} by what they do. Again, as Peck points out, they're the kind of people who have too little conscience.\footnote{Peck, M. S., \textit{The Road Less Traveled}, (Simon \& Schuster, 1978), pp. 35-36.} It is not possible to characterize every individual as simply neurotic or character disordered, but everyone falls somewhere along the continuum between mostly neurotic \& mostly character disordered. Nonetheless, it's very helpful to make the distinction about whether a person is primarily neurotic vs. disturbed in character.

Freud postulated that civilization is the cause of neurosis. He noted that the principal ways people bring pain \& hardship into the lives of others involve acts of sex or aggression \& that society often condemns indiscriminate sexual or aggressive conduct. He theorized, therefore, that persons who internalize societal prohibitions, though transformed from savages, pay a price for their self-restraint in the form of neurosis. From another point of view, however, one could say that the willingness of most persons to restrain (or even worry about) their sexual \& aggressive urges is what makes civilization possible. Rare is the person who ``owns'' \& \textit{freely} disciplines their basic instincts \&, therefore, in the manner Carl Jung suggested is possible,\footnote{Jung, C. G., 1953,\textit{ Collected Works of}, Vol. 14, p. 168. H. Read, M. Fordham and G. Adler, eds. New York: Pantheon.} \textit{transcends} all neurosis. For the most part, therefore, it's our capacity for neurosis that keeps us civilized. Neurosis is a very functional phenomenon then, in moderation. In today's permissive social climate, it is much less common that an individual's neurosis has become so extreme that therapeutic intervention is necessary, \& moderately neurotic individuals are the backbone of our society.

In a civilized society character-disordered individuals are more problematic than neurotics. Neurotics mainly cause problems for themselves because they let their excessive \& unwarranted fears stifle their own success. \& this happens only in those relatively rare cases when neurosis is excessive. Contrarily, character-disordered personalities, unencumbered by qualms of conscience, passionately pursue their personal goals with indifference to -- \& often at the expense of -- the rights \& needs of others,\footnote{Millon, T. \textit{Modern Psychopathology}, (W. B. Saunders, 1969), p. 261.}, \& cause all sorts of problems for others \& society at large. A common saying among professionals is that if a person is making himself miserable, he's probably neurotic, \& if he's making everyone else miserable, he's probably character-disordered. Among the various personality types, submissive personalities are among the most neurotic \& the aggressive personalities are among the most character-disordered.

Very contrasting characteristics define mostly neurotic versus mostly character-disordered individuals. These differences are crucial to remember, whether you're a person in a problematic relationship or a therapist trying to understand \& remedy an unhealthy situation.'' -- \cite[pp. 29--31]{Simon2010}

\subsection{Neurotic Personality}

\begin{itemize}
	\item ``For neurotics, anxiety plays a major role in the development of their personality \& fuels their ``symptoms'' of distress.
	\item Neurotics have a well-developed, or perhaps even overactive conscience or superego.
	\item Neurotics have an excessive capacity for guilt \& or shame. This increases anxiety \& causes much of their distress.
	\item Neurotics employ defense mechanisms to help reduce anxiety \& protect themselves from unbearable emotional pain.
	\item Fear of social rejection prompts neurotics to mask their true self \& present a false fa\c{c}ade to others.
	\item The psychological ``symptoms'' of distress neurotics experience are ego-dystonic (i.e., experienced as unwanted \& undesirable). For this reason, neurotics often voluntarily seek help to alleviate their distress.
	\item Emotional conflicts underlie the symptoms reported by neurotics \& are the appropriate focus of therapy.
	\item Neurotics often have damaged or deficient self-esteem.
	\item Neurotics are hypersensitive to adverse consequence \& social rejection.
	\item Inner emotional conflicts that cause anxiety for neurotics \& the defense mechanisms they use to reduce this anxiety are largely unconscious.
	\item Because the root of problems is often unconscious, neurotics need \& often benefit from the increased self-awareness that traditional, \textit{insight-oriented} therapy approaches offer.'' -- \cite[pp. 31--32]{Simon2010}
\end{itemize}

\subsection{Disordered Character}

\begin{itemize}
	\item Anxiety plays a minor role in the problems experienced by the character-disordered individual (CDO). CDOs lack sufficient apprehension \& anxiety related to their dysfunctional behavior pattern.
	\item The extremely disordered character may have no conscience at all. Most CDOs have consciences that are significantly underdeveloped.
	\item CDOs have diminished capacities for experiencing genuine shame or guilt.
	\item What may appear a defense mechanism to some is more likely a power tactic used to manipulate others \& resist making concessions to societal demands.
	\item CDO individuals may try to manage your impression of them, but in basic personality, they are who they are.
	\item Problematic aspects of personality are ego-syntonic (i.e., CDOs like who they are \& are comfortable with their behavior patterns, even though who they are \& how they act might bother others a lot). They rarely seek help on their own but are usually pressured by others.
	\item Erroneous thinking patterns\texttt{/}attitudes underlie the problem behaviors CDOs display.
	\item CDOs most often have inflated self-esteem. Their inflated self-image is not a compensation for underlying feelings of inadequacy.
	\item CDOs are undeterred by adverse consequence or societal condemnation.
	\item The CDO's problematic behavior patterns may be habitual \& automatic, but they are conscious \& deliberate.
	\item The disordered character has plenty of insight \& awareness but despite it, resists changing his\texttt{/}her attitudes \& core beliefs. CDOs don't need any more insight. What they need \& can benefit from are limits, confrontation, \& most especially, correction. Cognitive-behavioral therapeutic approaches are the most appropriate.
\end{itemize}
As outlined, on almost every dimension, disturbed characters are very different from neurotic individuals. Most especially, disordered characters don't think the way most of us do. In recent years, researchers have come to realize the importance of recognizing the fact. How we think, what we believe, \& the attitudes we've developed largely determine how we will act. That's partly why current research indicates that cognitive-behavioral therapy (confronting erroneous thinking patterns \& reinforcing a person's willingness to change their thinking \& behavior patterns) is the treatment of choice for disturbed characters.

Research into the distorted thinking patterns of disordered characters began several years ago \& focused on the thinking patterns of criminals. Over the years, researchers have come to understand that problematic patterns of thinking are common to all types of disordered characters, I have adopted, modified, \& added to many of the known problematic patterns of thinking \& offer a brief summary of some of the more important ones:
\begin{itemize}
	\item \textbf{Self-Focused (self-centered) thinking.} Disordered characters are always thinking of themselves. They don't think about what others need or how their behavior might impact others. This kind of thinking leads to attitudes of selfishness \& disregard for social obligation.
	\item \textbf{Possessive thinking.} This is thinking of people as possessions to do with as I please or whose role it is to please me. Disturbed characters also tend to see others as objects (objectification) as opposed to individuals with dignity, worth, rights \& needs. This kind of thinking leads to attitudes of ownership, entitlement \& dehumanization.
	\item \textbf{Extreme (all-or-none) thinking.} The disordered character tends to think that if he can't have everything he wants, he won't accept anything. If he's not on top, he sees himself at the bottom. If someone doesn't agree with everything he says, he thinks they don't value his opinions at all. This kind of thinking keeps him from any sense of balance or moderation \& promotes an uncompromising attitude.
	\item \textbf{Egomaniacal thinking.} The disordered character so overvalues himself that he thinks that he is entitled to whatever he wants. He tends to think that things are owed him, as opposed to accepting that he needs to earn the things he desires. This kind of thinking promotes attitudes of superiority, arrogance, \& entitlement.
	\item \textbf{Shameless thinking.} A healthy sense of shame is lacking in the disturbed character. He tends not to care how his behavior reflects on him as a character. He may be embarrassed if someone exposes his true character, but embarrassment at being uncovered is not the same as feeling shameful about reprehensible conduct. Shameless thinking fosters an attitude of brazenness.
	\item \textbf{Quick \& easy thinking.} The disturbed character always wants things the easy way. He hates to put forth effort or accept obligation. He gets far more joy out of ``conning'' people. This way of thinking promotes an attitude of disdain for labor \& effort.
	\item \textbf{Guiltless thinking.} Never thinking of the rightness or wrongness of a behavior before he acts, the disturbed character takes whatever he wants, no matter what societal norm is violated. This kind of thinking fosters an attitude of irresponsibility \& anti-sociality.'' -- \cite[pp. 32--34]{Simon2010}
\end{itemize}

\subsection{Aggressive Personalities \& Aggressive Personality Subtypes}

%------------------------------------------------------------------------------%

\section{The Determination to Win}

%------------------------------------------------------------------------------%

\section{The Unbridled Quest for Power}

%------------------------------------------------------------------------------%

\section{The Penchant for Deception \& Seduction}

%------------------------------------------------------------------------------%

\section{Fighting Dirty}

%------------------------------------------------------------------------------%

\section{The Impaired Conscience}

%------------------------------------------------------------------------------%

\section{Abusive, Manipulative Relationships}

%------------------------------------------------------------------------------%

\section{The Manipulative Child}

%------------------------------------------------------------------------------%

\begin{center}\Large\bf
	Part II: Dealing Effectively with Manipulative People
\end{center}

\section{Recognizing the Tactics of Manipulation \& Control}

%------------------------------------------------------------------------------%

\section{Redefining the Terms of Engagement}

%------------------------------------------------------------------------------%

\section{Epilogue: Undisciplined Aggression in a Permissive Society}

%------------------------------------------------------------------------------%

\section{Endnotes}

%------------------------------------------------------------------------------%

\printbibliography[heading=bibintoc]
	
\end{document}