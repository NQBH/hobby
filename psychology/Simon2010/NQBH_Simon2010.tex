\documentclass{article}
\usepackage[backend=biber,natbib=true,style=authoryear]{biblatex}
\addbibresource{/home/nqbh/reference/bib.bib}
\usepackage{tocloft}
\renewcommand{\cftsecleader}{\cftdotfill{\cftdotsep}}
\usepackage[colorlinks=true,linkcolor=blue,urlcolor=red,citecolor=magenta]{hyperref}
\usepackage{algorithm,algpseudocode,amsmath,amssymb,amsthm,float,graphicx,mathtools}
\allowdisplaybreaks
\numberwithin{equation}{section}
\newtheorem{assumption}{Assumption}[section]
\newtheorem{conjecture}{Conjecture}[section]
\newtheorem{corollary}{Corollary}[section]
\newtheorem{definition}{Definition}[section]
\newtheorem{example}{Example}[section]
\newtheorem{lemma}{Lemma}[section]
\newtheorem{notation}{Notation}[section]
\newtheorem{principle}{Principle}[section]
\newtheorem{problem}{Problem}[section]
\newtheorem{proposition}{Proposition}[section]
\newtheorem{question}{Question}[section]
\newtheorem{remark}{Remark}[section]
\newtheorem{theorem}{Theorem}[section]
\usepackage[left=0.5in,right=0.5in,top=1.5cm,bottom=1.5cm]{geometry}
\usepackage{fancyhdr}
\pagestyle{fancy}
\fancyhf{}
\lhead{\small Sect.~\thesection}
\rhead{\small\nouppercase{\leftmark}}
\renewcommand{\sectionmark}[1]{\markboth{#1}{}}
\cfoot{\thepage}
\def\labelitemii{$\circ$}

\title{In Sheep's Clothing: Understanding \& Dealing with Manipulative People}
\author{George Simon, Jr.}
\date{\today}

\begin{document}
\maketitle
\tableofcontents
\vspace{5mm}
\begin{quotation}
	``[After reading \textit{In Sheep's Clothing}] I am beginning to reclaim my life, find my self-respect \& confidence.'' -- Marc, Virginia
	
	``After having read several books on several different self-help topics, psychology books, psychiatry books, etc., I MUST recommend you buy this one, 1st. It cuts straight through the bs -- neatly \& cleanly. I have bought copies of this book for friends \& can't recommend it enough.'' -- E. Adams, Online Purchaser
	
	``Don't Be Bossed-Around Ever Again!!! $\ldots$ \textit{In Sheep's Clothing: Understanding \& Dealing with Manipulative People} by George K. Simon, Jr., Ph.D., is a godsend to anyone who has ever questioned their own sanity while in any kind of relationship with a controlling \& manipulative person.'' -- The Aeolian Kid, Online Purchaser
	
	``Dr. Simon teaches the mechanics of popular tactics used by manipulators \& how you can identify \& thwart their attacks so that you control the outcome. This book helped me with a person that I have no choice but to see daily. After the end of every ``friendly'' conversation I felt depressed or insulted but could not figure out how this person was doing it. This book helped me to understand what was really happening. Dr. Simon's guidelines exposed this person \& [allowed me to take] control. Because this person knows [I] can be longer [be] controlled, I now have -- not a perfect relationship -- but one that's better than the alternative.'' -- A reader in Chicago
	
	``This book is like the secret decoder ring for the jumbled mess that is a manipulator's modus operandi. \textit{Do yourself a favor \& get this book now}.'' -- Christy, Missouri
	
	``It's sad that there are people out there that make life so much harder than it should have to be for others. Being able to identify such people in your life (both at home \& at work) is very important \& can be of invaluable help to (i) not go crazy oneself, \& (ii) take corrective action. Dr. Simon's book is written with amazing clarity. \textit{If you read only 1 book this year, read this one}.'' -- JA008, Online Purchaser
	
	``This is 1 of the best books I've ever read \& I would recommend it to anyone. It has redefined how I judge people \& helped me to become a stronger person. I used to be very naive \& unaware of people's ulterior motives, \& I have learned a tremendous amount from reading this book.'' -- S. Brescenti, Online Purchaser
	
	``This book makes it clear that evil is allowed free rein because of our ignorance of its nature. Simon shows us what seemingly mundane interactions that leave us perplexed may really be about. According to him, master manipulators leave us drained \& confused as we try to change them into the good person we want to believe they really are. I would add that the manipulators are just plain evil because evil requires lies, manipulation \& a weakening of the other's will through deception. Simon shows you how to recognize the signs \& what you can do about it. Good people are responsible for informing \& protecting themselves from the manipulators in society. This book is a necessary start.'' -- Kaye, a reader in New York state
	
	``Pithy \& often funny, George Simon takes the bluster \& obfuscation of overbearing, weasely bosses, nasty neighbors, \& obnoxious coworkers \& boils it down to show you the simple psychological strategies being used to imposed on your patience, good will, or even wallet. \textit{I have recommended this book to everyone I know \& bought extra copies for my kids} when they went out into the work world. Highly Recommended!'' -- C. MacCallum, Online Purchaser
\end{quotation}

\section{Acknowledgments}
``I am deeply grateful to my wife, Dr. Sherry Simon for her unfailing love, faith, understanding, patience, \& support. She is responsible for the title of this book \& was a valuable resource in helping me clarify my thoughts during its writing. I wish to thank Dr. Bruce Carruth for his critique of the original manuscript \& suggestions for making it more readable. I am deeply indebted to the work of Dr. Theodore Millon. His comprehensive approach to understanding personality has not only influenced my thinking on the subject but also proved invaluable in my efforts to help people change. I owe a supreme debt to the many individuals willing to share with me their experiences with manipulative people. They taught me much \& enriched my life. This book, in large measure, is a tribute to their courage \& support. I am most appreciative of the validation, support, \& enriching input consistently afforded me by workshop attendees. They have helped me clarity, refine, \& enhance 1 of my principal missions in life. Words cannot express the gratitude I have for the thousands of readers who have kept this book in the active lists of online booksellers \& retail outlets for over 15 years. The many emails, blog posts, \& letters readers have sent helped me make necessary updates \& changes to this Revised Edition. I have attempted to honor the considerable feedback I continue to receive by expanding the discussion of key concepts as well as introducing important new content for this newly revised edition. Finally, I want to thank Roger Armbrust \& Ted Parkhurst of Parkhurst Brothers, Inc., Publishers. Ted encouraged me at the outset \& was there when I needed him; Roger's grace \& presence have only benefited my work and readers.'' -- \cite[p. 7]{Simon2010}

%------------------------------------------------------------------------------%

\section*{Preface}
``Whether it's the supervisor who claims to support you while thwarting every opportunity you have to get ahead, the co-worker who quietly undermines you to gain the boss's favor, the spouse who professes to love \& care about you but seems to control your life, or the child who always seems to know just which buttons to push in order to get their way, manipulative people are like the proverbial wolf in sheep's clothing. On the surface they can appear charming \& genial. But underneath, they can be ever so calculating \& ruthless. Cunning \& stable, they prey on your weaknesses \& use clever tactics to gain advantage over you. They're the kind of people who fight hard for everything they want but do their best to conceal their aggressive intentions. That's why I call them \textit{covert-aggressive personalities}.8

As a clinical psychologist in private practice, I began to focus on the problem of covert aggression over 20 years ago. I did so because the depression, anxiety, \& feelings of insecurity that initially led several of my patients to seek help eventually turned out to be in some way linked to their relationship with a manipulative person. I've counseled not only the victims of covert-aggression, but also manipulators themselves experiencing distress because their usual ways of getting their needs met \& controlling other weren't working anymore. My work has given me an appreciation for how widespread problem of manipulative behavior is \& the unique emotional stress it can bring to a relationship.

The scope of the problem of covert-aggression seems self-evident. Most of us know at least 1 manipulative person. \& hardly a day goes by that we don't read in the newspaper or hear a broadcast about someone who managed to exploit or ``con'' many before fate shed some light on their true character. There's the tele-evangelist who preached love, honesty, \& decency while cheating on his wife \& fleecing his flock, the politician, sworn to ``public service,'' caught lining his pockets, or the spiritual ``guru'' who even managed to convince most of his followers that he was God incarnate while sexually exploiting their children \& subtly terrorizing those who challenged him. The world, it seems full of manipulators.

Although the extreme wolves in sheep's clothing that make headlines grab our attention \& pique our curiosity about what makes such people ``tick,'' most of the covertly aggressive people we are likely to encounter are not these larger-than-life characters. Rather, they are the subtly underhanded, backstabbing, deceptive, \& conniving individuals we may work with, associate with, or possibly even live with. \& they can make life miserable. They cause us grief because we find it so hard to truly understand them \& even harder to deal with them effectively.

When victims of covert-aggression 1st seek help for their emotional distress, they usually have little insight into why they feel so bad. They only know that they feel confused, anxious, or depressed. Gradually, however, they relate how dealing with a certain person in their lives makes them feel crazy. They don't really trust them but can't pinpoint why. They get mad at them but for some reason end up feeling guilty themselves. They confront them about their behavior, only to wind up on the defensive. They get frustrated because they find themselves frequently giving in when they really wanted to stand ground, saying ``yes'' when they mean to say ``no,'' \& becoming depressed because nothing they try seems to make things better. In the end, dealing with this person always leaves them feeling confused, exploited \& abused. After exploring the issues in therapy for a while, they eventually come to realize how much of their unhappiness is the direct result of their constant but fruitless attempts to understand, deal with, or control their manipulator's behavior.

Despite the fact that many of my patients are intelligent, resourceful individuals with a fair understanding of traditional psychological principles, most of the ways they tried to understand \& cope with their manipulator's behavior weren't getting them anywhere, \& some of the things they tried only seemed to make matters worse. Moreover, none of the ways that I initially tried to help made any real difference. Having an eclectic training background, I tried all sorts of different therapies \& strategies, all of which seemed to help the victims feel a little better, but none seemed to empower them enough to really change the nature of their relationship with their manipulator. Even more disconcerting was the fact that none of the approaches I tried was effective at all with the manipulators. Realizing that something must be fundamentally wrong with the traditional approaches to understanding \& dealing with manipulative people, I began to carefully study the problem in the hope of developing a practical, more effective approach.

In this book I would like to introduce you to a new way of understanding the character of manipulative people. I believe the perspective I will offer describes manipulators \& labels their behavior more accurately than many other approaches. I'll explain what covert-aggression is \& why I believe it's at the heart of most interpersonal manipulation. I'll focus some needed attention on dimensions of personality that are too often ignored by traditional perspectives. The framework I will be advancing challenges some of the more common assumptions we make about why people act the way they do \& explains why some of the most widely-held beliefs about human nature tend to set us up for victimization by manipulators.

I have 3 objectives to fulfill in this book. My 1st is to fully acquaint you the nature of disturbed characters as well as the distinctive character of the covertly aggressive personality. I'll discuss the characteristics of aggressive personality types in general \& outline the unique characteristics of the covert-aggressive personality. I'll present several vignettes, based on real cases \& situations, that will help you get the ``flavor'' of this personality type as well as illustrate how manipulative people operate. Being able to recognize a wolf in sheep's clothing \& knowing what to expect from this kind of person is the 1st step in avoiding being victimized by them.

My 2nd objective is to explain precisely how covertly aggressive people managed to deceive, manipulate, \& ``control'' others. Aggressive \& covertly aggressive people use a select group of interpersonal maneuvers or tactics to gain advantage over others. Becoming more familiar with these tactics really helps a person recognize manipulative behavior \textit{at the time it occurs}, \& makes it easier, therefore, to avoid being victimized. I'll also discuss the characteristics many of us possess that can make us unduly vulnerable to the tactics of manipulation. Knowing what aspects of your own character a manipulator is most likely to exploit is another important step in avoiding victimization.

My final objective is to outline the specific steps anyone can take to deal more effectively with aggressive \& covertly aggressive personalities. I'll present some general rules for redefining the rules of engagement with these kinds of individuals \& describe some specific tools of personal empowerment that can help a person break the self-defeating cycle of trying to control their manipulator \& becoming depressed in the process. Using these tools makes it more likely that a 1-time victim will invest their energy where they really have power -- in their own behavior. Knowing how to conduct yourself in a potentially manipulative encounter is crucial to becoming less vulnerable to a manipulator's ploys \& asserting greater control over your own life.

I have attempted to write this book in a manner that is serious \& substantial yet straightforward \& readily understandable. I have written it for the general public as well as the mental health professional, \& I hope both will find it useful. By adhering to many traditional assumptions, labeling schemes, \& intervention strategies, therapists sometimes hold \& inadvertently reinforce some of the same misconceptions that their patients harbor about the character \& behavior of manipulators that inevitably lead to continued victimization. I offer a new perspective in the hope of helping individuals \& therapists alike avoid \textit{enabling} manipulative behavior.'' -- \cite[pp. 8--12]{Simon2010}

%------------------------------------------------------------------------------%

\section*{Author's Note on the Revised Edition}
``Since this book's 1st wide publication in 1996, I have received literally hundreds of calls, letters, \& emails, \& heard countless testimonials \& comments at workshops from individuals whose lives were changed merely by being exposed to \& adopting a new perspective on understanding human behavior. A common theme voiced by readers \& workshop attendees is that once they dispelled old myths \& came to view problem behaviors in a different light, they could see clearly that what their intuition had told them all along was correct, \& thus felt validated. A similar phenomenon has held true for mental health professionals attending the many training seminars I have given. Once they abandoned their old notions about why their clients do the things they do, they were better able to help them \& their significant others. I had already been doing workshops for 10 years before writing \textit{In Sheep's Clothing}. At that time, only a handful of theorists, researchers, \& writers were recognizing the need for a new perspective on understanding \& dealing with disturbed characters (e.g., Stanton Samenow, Samuel Yochelson, Robert Hare). What professionals today call the \textit{cognitive-behavioral} approach was in its infancy. The early research on character disturbance inspired me \& helped me validate my own observations. Today an increasing number of professionals are recognizing the problem of character disturbance \& using cognitive-behavioral methods to diagnose \& treat it.

We live in an age radically different from that in which the classical theories of psychology \& personality were developed. For the most part, truly pathological degrees of neurosis are quite rare, \& problematic levels of character disturbance are increasingly commonplace. It's a pervasive societal problem about which all of us would do well to expand our awareness. During the last 15 years, my experience working with disturbed characters of all types has grown immensely, as has the body of research. So, I have included in this edition an expanded discussion on the problem of character disturbance in general \& what sets the disturbed character apart from your garden-variety neurotic.

I am deeply grateful for the excellent word-of-mouth support responsible for transforming a once small, independent work into a best seller enjoying ever-increasing popularity even after 15 years. I sincerely hoped this revised edition will provide you with all the information \& tools you need to better understand \& deal with the manipulative people in your life.

\textit{George K. Simon, Jr., Ph.D.}, Jan 2010'' -- \cite[p. 14]{Simon2010}

%------------------------------------------------------------------------------%

\begin{center}\LARGE\sf
	\textbf{Part I: Understanding Manipulative Personalities}
\end{center}

\section*{Introduction: Covert-Aggression: The Heart of Manipulation}

\subsection{A Common Problem}
``Perhaps the following scenarios will sound familiar. A wife tries to sort out her mixed feelings. She's mad at her husband for insisting their daughter make all A's. But she doubts she has the right to be mad. When she suggested that given her appraisal of their daughter's abilities, he might be making unreasonable demands, his comeback, ``Shouldn't \textit{any} good parent want their child to do well \& succeed in life?'' made her feel like the insensitive one. In fact, whenever she confronts him, she somehow ends up feeling like the bad guy herself. When she suggested there might be more to her daughter's recent problems, \& that the family might do well to seek counseling, his retort ``Are you saying I'm psychiatrically disturbed?'' made her feel guilty for bringing up the issue. She often tries to assert her point of view, but always ends up giving-in to his. Sometimes, she thinks the problem is him, believing him to be selfish, demanding, intimidating, \& controlling. But this is a loyal husband, good provider, \& a respected member of the community. By all rights she shouldn't resent him. Yet, she does. So, she constantly wonders if there isn't something wrong with her.

A mother tries desperately to understand her daughter's behavior. No young girl, she thought, would threaten to leave home, say things like ``Everybody hates me'' \& ``I wish I were never born,'' unless she were very insecure, afraid, \& probably depressed. Part of her thinks her daughter is still the same child who used to hold her breath until she turned blue or threw tantrums whenever she didn't get her way. After all, it seems she only says \& does these things when she's about to be disciplined or she's trying to get something she wants. But a part of her is afraid to believe that. ``What if she really believes what she's saying?'' she wonders. ``What if I've really done something to hurt her \& I just don't realize it?'' she worries. She hates to feel ``bullied'' by her daughter's threats \& emotional displays, but she can't take the chance her daughter might really be hurting -- can she? Besides, children just don't act this way unless they really feel insecure or threatened in some way underneath it all -- do they?'' -- \cite[pp. 17--18]{Simon2010}

\subsection{The Heart of the Problem}
``Neither victim in the preceding scenarios trusted their ``gut'' feelings. Unconsciously, they felt on the defensive, but consciously they had trouble seeing their manipulator as merely a person on the offensive. On 1 hand, they felt like the other person was trying to get the better of them. On the other, they found no objective evidence at the time to back-up their gut-level hunch. They ended up feeling crazy.

They're not crazy. The fact is, people fight almost all the time. \& manipulative people are expert at fighting in subtle \& most undetectable ways. Most of the time, when they're trying to take advantage or gain the upper hand, you don't even know you're in a fight until you're well on your way to losing. When you're being manipulated, chances are someone is fighting with you for position, advantage, or gain, but in a way that's difficult to readily see. Covert-aggression is at the heart of most manipulation.'' -- \cite[p. 18]{Simon2010}

\subsection{The Nature of Human Aggression}
``Our instinct to fight is a close cousin of our survival instinct.\footnote{Storr, A., \textit{Human Destructiveness}, (Ballantine, 1991), pp. 7-17.} Most everyone ``fights'' to survive \& prosper, \& \textit{most} of the fighting we do is neither physically violent nor inherently destructive. Some theorists have suggested that only when this most basic instinct is severely frustrated does our aggressive drive have the potential to be expressed violently.\footnote{Storr, A., \textit{Human Destructiveness}, (Ballantine, 1991), p. 21.} Others have suggested that some rare individuals seem to be predisposed to aggression -- even violent aggression, despite the most benign circumstances. But whether extraordinary stressors, genetic predispositions, reinforced learning patterns, or some combination of these are at the root of violent aggression, most theorists agree that aggression per se \& destructive violence are not synonymous. In this book, the term aggression will refer to the forceful energy we all expend in our daily bids to survive, advance ourselves, secure things we believe will bring us some kind of pleasure, \& remove obstacles to those ends.

People do a lot more fighting in their daily lives than we have ever been willing to acknowledge. The urge to fight is fundamental \& instinctual. Anyone who denies the instinctual nature of aggression has either never witnessed 2 toddlers struggling for possession of the same toy, or has somehow forgotten this archetypal scene. Fighting is a big part of our culture, also. From the fierce partisan wrangling that characterizes representative government, to the competitive corporate environment, to the adversarial system of our judicial system, much fighting is woven into our societal fabric. We sue one another, divorce each other, battle with one another over our children, compete for jobs, \& struggle with each other to advance certain goals, values, beliefs \& ideals. The psychodynamic theorist Alfred Adler noted many years ago that we also forcefully strive to assert a sense of social superiority.\footnote{Adler, A., \textit{Understanding Human Nature}, (Fawcett World Library, 1954), p. 178.} Fighting for personal \& social advantage, we jockey with one another for power, prestige, \& a secure social ``niche.'' Indeed, we do so much fighting in so many aspects of our lives I think it fair to say that when human beings aren't making some kind of love, they're likely to be waging some kind of war.

Fighting is not inherently wrong or harmful. Fighting openly \& fairly for our legitimate needs is often necessary \& constructive. When we fight for what we truly need while respecting the rights \& needs of others \& taking care not to needlessly injure them, our behavior is best labeled \textit{assertive}, \& assertive behavior is 1 of the most healthy \& necessary human behaviors. It's wonderful when we learn to assert ourselves in the pursuit of personal needs, overcome unhealthy dependency \& become self-sufficient \& capable. But when we fight unnecessarily, or with little concern about how others are being affected, our behavior is most appropriately labeled \textit{aggressive}. In a civilized world, undisciplined fighting (aggression) is almost always a problem. The fact that we are an aggressive species doesn't make us inherently flawed or ``evil,'' either. Adopting a perspective advanced largely by Carl Jung,\footnote{Jung, C. G., 1953, \textit{Collected Works of}, Vol. 7, p.25. H. Read, M. Fordham and G. Adler, eds. New York: Pantheon.} I would assert that the evil that sometimes arises from a person's aggressive behavior necessarily stems from his or her failure to ``own'' \& discipline this most basic human instinct.'' -- \cite[pp. 18--20]{Simon2010}

\subsection{2 Important Types of Aggression}
``2 of the most fundamental types of fighting (others, such as reactive vs. predatory or instrumental aggression) will be discussed later are \textit{overt} \& \textit{covert} aggression. When you're determined to have your way or gain advantage \& you're open, direct, \& obvious in your manner of fighting, your behavior is best labeled overtly aggressive. When you're out to ``win,'' get your way, dominate, or control, but are subtle, underhanded, or deceptive enough to hide your true intentions, your behavior is most appropriately labeled covertly aggressive. Concealing overt displays of aggression while simultaneously intimidating others into backing-off, backing-down, or giving-in is a very powerful manipulative maneuver. That's why \textit{covert-aggression is most often the vehicle for interpersonal manipulation}.'' -- \cite[p. 20]{Simon2010}

\subsection{Covert \& Passive-Aggression}
``I often hear people say something is being ```passive-aggressive'' when they're really trying to describe covertly aggressive behavior. Covert \& passive-aggression are both indirect ways to aggress but they're most definitely not the same thing. Passive-aggression is, as the term implies, aggressing though passivity. Examples of passive-aggression are playing the game of emotional ``get-back'' with someone by resisting cooperation with them, giving them the ``silent treatment,'' pouting or whining, not so accidentally ``forgetting'' something they wanted you to do because you're angry \& didn't really feel like obliging them, etc. In contrast, covert aggression is very \textit{active}, albeit veiled, aggression. When someone is being covertly aggressive, they're using calculating, underhanded means to get what they want or manipulate the response of others while keeping their aggressive intentions under cover.'' -- \cite[pp. 20--21]{Simon2010}

\subsection{Acts of Covert-Aggression vs. Covert-Aggressive Personalities}
``Most of us have engaged in some sort of covertly aggressive behavior from time to time but that doesn't necessarily make someone a covert-aggressive or manipulative personality. An individual's personality can be defined by the way he or she habitually perceives, relates to \& interacts with others \& the world at large.\footnote{Millon, T. \textit{Disorders of Personality}, (Wiley-Interscience, 1981), p. 4.} It's the distinctive interactive ``style'' or relatively engrained way a person prefers to deal with a wide variety of situations \& to get the things they want in life. Certain personalities can be ever so ruthless in their interpersonal conduct while concealing their aggressive character or perhaps even projecting a convincing, superficial charm. These covert-aggressive personalities can have their way with you \& look good in the process. They vary in their degree of ruthlessness \& character pathology. But because the more extreme examples can teach us much about the process of manipulation in general, this book will pay special attention to some of the more seriously disturbed covert-aggressive personalities.'' -- \cite[pp. 21--22]{Simon2010}

\subsection{The Process of Victimization}
``For a long time, I wondered why manipulation victims have a hard time seeing what really goes on in manipulative interactions. At 1st, I was tempted to fault them. But I've learned that they get hoodwinked for some very good reasons:
\begin{enumerate}
	\item A manipulator's aggression is not obvious. We might intuitively sense that they're trying to overcome us, gain power, or have their way, \& find ourselves \textit{unconsciously} intimidated. But because we can't point to clear, objective evidence they're aggressing against us, we can't readily validate our gut-level feelings.
	\item The tactics that manipulators frequently use are powerful deception techniques that make it hard to recognize them as clever ploys. They can make it seem like the person using them is hurting, caring, defending, or almost anything but fighting for advantage over us. Their explanations always make just enough sense to make another doubt his or her gut hunch that they're being taken advantage of or abused. Their tactics not only make it hard for a person to consciously \& objectively know their manipulator is fighting to overcome, but also simultaneously keep the victim unconsciously on the defensive. This makes the tactics highly effective psychological 1-2 punches. It's hard to think clearly when someone has you emotionally unnerved, so you're less likely to recognize the tactics for what they really are.
	\item All of us have weaknesses \& insecurities that a clever manipulator might exploit. Sometimes, we're aware of these weaknesses \& how someone might use them to take advantage of us. E.g., I hear parents say things like: ``Yeah, I know I have a big guilt button.'' But at the time their manipulative child is busily pushing that button, they can easily forget what's really going on. Besides, sometimes we're unaware of our biggest vulnerabilities. Manipulators often know us better than we know ourselves. They know what buttons to push, when to do so \& how hard to press. Our lack of self-awareness can easily set us up to be exploited.
	\item What our intuition tells us a manipulator is really like challenges everything we've been taught to believe about human nature. We've been inundated with a psychology that has us viewing people with problems, at least to some degree, as afraid, insecure or ``hung-up.'' So, while our gut tells us we're dealing with a ruthless conniver, our head tells us they must be really frightened, wounded, or self-doubting ``underneath.'' What's more, most of us generally hate to think of ourselves as callous \& insensitive people. We hesitate to make harsh or negative judgments about others. We want to give them the benefit of the doubt \& believe they don't really harbor the malevolent intentions we suspect. We're more apt to doubt \& blame ourselves for daring to believe what our gut tells us about our manipulator's character.'' -- \cite[pp. 22--23]{Simon2010}
\end{enumerate}

\subsection{Recognizing Aggressive Agendas}
``Accepting how fundamental it is for people to fight for the things they want \& becoming more aware of the subtle, underhanded ways people can \& do fight in so many of their daily endeavors \& relationships can be very consciousness-expanding. Learning to recognize an aggressive move when somebody makes one \& learning how to handle oneself in any of life's many battles has turned out to be the most empowering experience for the manipulation victims with whom I've worked. It's how they eventually freed themselves from their manipulator's dominance \& control \& gained a much-needed boost to their own sense of self-esteem.

Recognizing the inherent aggression in manipulative behavior \& becoming more aware of the slick, surreptitious ways that manipulative people prefer to aggress against us is extremely important. Not recognizing \& accurately labeling their subtly aggressive moves causes most people to misinterpret the behavior of manipulators \&, therefore, fail to respond to them in an appropriate fashion. Recognizing when \& how manipulators are fighting with you is fundamental to fairing well in any kind of encounter with them.

Unfortunately, mental health professionals \& lay persons alike often fail to recognize the aggressive agendas \& actions of others for what they really are. This is largely because we've been pre-programmed to believe that people only exhibit problem behaviors when they're ``troubled'' inside or anxious about something. We've also been taught that people aggress only when they're attacked in some way. So, even when our gut tells us that somebody is attacking us \& for no good reason, or merely trying to overpower us, we don't readily accept the notions. We usually start to wonder what's bothering the person so badly ``underneath it all'' that's making them act in such a disturbing way. We may even wonder what we may have said or done that ``threatened'' them. We may try to analyze the situation to death instead of simply responding to the attack. We almost never think that the person is simply fighting to get something they want, to have their way with us, or gain the upper hand. \&, when we view them as primarily hurting in some way, we strive to understand as opposed to taking care of ourselves.

Not only do we often have trouble recognizing the ways people aggress, but we also have difficulty discerning the distinctly aggressive character of some personalities. The legacy of Sigmund Freud's work has a lot to do with this. Freud's theories (\& the theories of others who expanded on his work) heavily influenced the field of psychology \& related social sciences for a long time. The basic tenets of these classical (psychodynamic) theories \& their hallmark construct, \textit{neurosis}, have become fairly well etched in the public consciousness, \& many psychodynamic terms have intruded into common parlance. These theories also tend to view \textit{everyone}, at least to some degree, as \textit{neurotic}. Neurotic individuals are overly inhibited people who suffer unreasonable \& excessive anxiety (i.e., non-specific fear), guilt, \& shame when it comes to acting on their basic instincts or trying to gratify their basic wants \& needs. The malignant impact of over-generalizing Freud's observations about a small group of overly inhibited individuals into a broad set of assumptions about the causes of psychological ill-heath in everyone cannot be overstated.\footnote{Torrey, F., \textit{Freudian Fraud}, (Harper Collins, 1992), p. 257.} But these theories have so permeated our thinking about human nature, \& especially our theories of personality, that when most of us try to analyze someone's character, we automatically start thinking in terms of what fears might be ``hanging them up,'' what kinds of ``defenses'' they use \& what kinds of psychologically ``threatening'' situations they may be trying to ``avoid.'''' -- \cite[pp. 23--25]{Simon2010}

\subsection{The Need for a New Psychological Perspective}
``Classical theories of personality were developed during an extremely repressive time. If there were a motto for the Victorian era, it would be: ``Don't even think about it!'' In such times, one would expect neurosis to be more prevalent. Freud treated individuals who were so riddled with excessive shame \& guilt about their primal urges that some went ``hysterically'' blind so they wouldn't run the risk of consciously laying lustful eyes on the objects of their desire. Times have certainly changed. Today's social climate is far more permissive. If there were a motto for our time, it would be as the once popular TV commercial exhorted: ``Just do it!'' Many of the problems coming to the attention of mental health professionals these days are less the result of an individual's unreasonable fears \& inhibitions \& more the result of the deficient self-restraint a person has exercised over his\texttt{/}her basic instincts. More simply, therapists are increasingly being asked to treat individuals suffering from too little as opposed to too much neurosis (i.e., individuals with some type of \textit{character disturbance}). As a result, classical theories of personality \& their accompanying prescription for helping troubled persons achieve greater psychological health have proved to be of limited value when working with many of today's disturbed characters.

Some mental health professionals may need to overcome significant biases in order to better recognize \& deal with aggressive or covertly aggressive behavior. Therapists who tend to see any kind of aggression not as a problem in itself but as a ``symptom'' of an underlying inadequacy, insecurity or unconscious fear, may focus so intently on their patient's supposed ``inner conflict'' that they overlook the aggressive behaviors most responsible for problems. Therapists whose training overly indoctrinated them in the theory of neurosis may ``frame'' the problems presented them incorrectly. They may, e.g., assume that a person who all their life has aggressively pursued independence, resisted allegiance to others, \& taken what they could from relationships without feeling obliged to give something back must necessarily be ``compensating'' for a ``fear'' of intimacy. In other words, they will view a hardened, abusive fighter as a terrified runner, thus misperceiving the core reality of the situation.

It's neither appropriate nor helpful to over-generalize the characteristics of neurotic personalities in the attempt to describe \& understand all personalities. We need to stop trying to define \textit{every} type of personality by their greatest fears of the principal ways they ``defend'' themselves. We need a completely different theoretical framework if we are to truly understand, deal with, \& treat the kinds of people who fight too much as opposed to those who cower or ``run'' too much. I will present just such a framework in Chap. 1. I will introduce several aggressive personality types whose psychological makeup differs radically from those of the more neurotic personalities. It is within this framework that you will be better able to understand the nature of disturbed characters in general as well as the distinctive character of the manipulative people I call covert-aggressive personalities. I hope to present this new perspective not only in a style readily digestible by the lay reader trying to understand \& cope with a difficult situation but also in a manner that should prove useful to mental health professionals attempting therapeutic interventions.'' -- \cite[pp. 25--26]{Simon2010}

%------------------------------------------------------------------------------%

\section{Aggressive \& Covert-Aggressive Personalities}
``Understanding the true character of manipulative people is the 1st step in dealing more effectively with them. In order to know what they're really like, we have to view them within an appropriate context. In this chapter, I hope to present a framework for understanding personality \& character that will help you distinguish manipulators from other personality types \& give you an increased ability to identify a wolf in sheep's clothing when you encounter one.'' -- \cite[p. 27]{Simon2010}

\subsection{Personality}
``The term personality derives from the Latin word ``persona,'' which means ``mask.'' In the ancient theater, when actors were only men, \& when the art of conveying emotions through dramatic techniques had not fully evolved, female characters \& various emotions were portrayed through the use of masks. Classical theorists, who conceptualized personality as the social fa\c{c}ade or ``mask'' a person wore to hide the ``true self,'' adopted the term. The classical definition of personality, however, has proven to be quite limiting.

Personality can also be defined as the unique manner that a person develops of perceiving, relating to \& interacting with others \& the world at large.\footnote{Millon, T. \textit{Disorders of Personality}, (Wiley-Interscience, 1981), p. 4.} Within this model of personality, biology plays a part (e.g., genetic, hormonal influences, brain biochemistry), as does temperament, \& of course, the nature of a person's environment \& what he or she has learned from past experiences are big influences, also. All of these factors dynamically interact \& contribute to the distinctive ``style''\footnote{Millon, T. \textit{Disorders of Personality}, (Wiley-Interscience, 1981), p. 4.} a person develops over time in dealing with others \& coping with life's stressors in general. A person's interpersonal interactive ``style'' or personality appears a largely stable characteristic that doesn't moderate much with time \& generalizes across a wide variety of situations.'' -- \cite[pp. 27--28]{Simon2010}

\subsection{Character}
``Everyone's unique style of relating to others has social, ethical \& moral ramifications. The aspect of someone's personality that reflects how they accept \& fulfill their social responsibilities \& how they conduct themselves with others has sometime been referred to as \textit{character}.\footnote{Millon, T. \textit{Disorders of Personality}, (Wiley-Interscience, 1981), p. 6.} Some use the terms character \& personality synonymously. But in this book, the term character will refer to those aspects of an individual's personality that reflect the extent to which they have developed personal integrity \& a commitment to responsible social conduct. Persons of sound character temper their instinctual drives, moderate important aspects of their conduct, \& especially, discipline their aggressive tendencies in the service of the greater social good.'' -- \cite[p. 28]{Simon2010}

\subsection{Some Basic Personality Types}
``Volumes of clinical literature have been written on the various personality types. A discussion of all of the personality types is beyond the scope of this book. However, I find it particularly useful to distinguish between 2 basic dimensions of personality that occupy positions on opposite ends of a continuum that reflects how an individual deals with the challenges of life.

As goal-directed creatures, we all invest considerable time \& energy trying to get the things we think will help us to prosper or bring us some kind of pleasure. Running into obstacles or barriers to what we want is the essence of human conflict. Now, there are fundamentally 2 things a person can do when running up against an obstacle to something they want. They can be so overwhelmed or intimidated by the resistance they encounter or so unsure of their ability to deal with it effectively, that they fearfully retreat. Alternatively, they can directly challenge the obstacle. If they are confident enough in their fighting ability \& tenacious enough in their temperament, they might try to forcefully remove or overcome whatever stands between them \& the object of their desire.

Submissive personalities \textit{habitually} \& \textit{excessively} retreat from potential conflicts. They doubt their abilities \& are excessively afraid to take a stand. Because they ``run'' from challenges too often, they deny themselves opportunities to experience success. This pattern makes it hard for them to develop a sense of personal competence \& achieve self-reliance. Some personality theorists describe these individuals as \textit{passive-dependent}\footnote{Millon, T. \textit{Disorders of Personality}, (Wiley-Interscience, 1981), p. 91.} because their passivity largely leads them to become overly dependent upon others to do their fighting for them. Feeling inadequate, they all too readily submit to the will of those they view as more powerful or more capable than themselves.

In contrast, aggressive personalities are overly prone to fight in any potential conflict. Their main objective in life is ``winning'' \& they pursue this objective with considerable passion. They forcefully strive to overcome, crush, or remove any barriers to what they want. They seek power ambitiously \& use it unreservedly \& unscrupulously when they get it. They always strive to be ``on top'' \& in control. They accept challenges readily. Whether their faith in their ability to handle themselves in most conflicts is well-founded or not, they tend to be overly self-reliant or emotionally \textit{independent}.'' -- \cite[pp. 28--29]{Simon2010}

\subsection{Neurotic \& Character-Disordered Personalities}
``There are 2 other important dimensions of personality that represent opposite ends on a different continuum. Personalities who are excessively uncertain about how to cope \& excessively anxious when they attempt to secure their basic needs have often been called \textit{neurotic}. The inner emotional turmoil a neurotic personality experiences most often arises from ``conflicts'' between their basic instinctual drives \& their qualms of conscience. As a general rule, therefore, Scott Peck's point in \textit{The Road Less Traveled} that neurotics suffer from too much conscience is correct.\footnote{Peck, M. S., \textit{The Road Less Traveled}, (Simon \& Schuster, 1978), pp. 35-36.} These individuals are too afraid to seek satisfaction of their needs because they're overly riddled with guilt or shame when they do. In contrast, \textit{character-disordered} personalities lack self-restraint when it comes to acting upon their primal urges. They're \textbf{\textit{not bothered enough}} by what they do. Again, as Peck points out, they're the kind of people who have too little conscience.\footnote{Peck, M. S., \textit{The Road Less Traveled}, (Simon \& Schuster, 1978), pp. 35-36.} It is not possible to characterize every individual as simply neurotic or character disordered, but everyone falls somewhere along the continuum between mostly neurotic \& mostly character disordered. Nonetheless, it's very helpful to make the distinction about whether a person is primarily neurotic vs. disturbed in character.

Freud postulated that civilization is the cause of neurosis. He noted that the principal ways people bring pain \& hardship into the lives of others involve acts of sex or aggression \& that society often condemns indiscriminate sexual or aggressive conduct. He theorized, therefore, that persons who internalize societal prohibitions, though transformed from savages, pay a price for their self-restraint in the form of neurosis. From another point of view, however, one could say that the willingness of most persons to restrain (or even worry about) their sexual \& aggressive urges is what makes civilization possible. Rare is the person who ``owns'' \& \textit{freely} disciplines their basic instincts \&, therefore, in the manner Carl Jung suggested is possible,\footnote{Jung, C. G., 1953,\textit{ Collected Works of}, Vol. 14, p. 168. H. Read, M. Fordham and G. Adler, eds. New York: Pantheon.} \textit{transcends} all neurosis. For the most part, therefore, it's our capacity for neurosis that keeps us civilized. Neurosis is a very functional phenomenon then, in moderation. In today's permissive social climate, it is much less common that an individual's neurosis has become so extreme that therapeutic intervention is necessary, \& moderately neurotic individuals are the backbone of our society.

In a civilized society character-disordered individuals are more problematic than neurotics. Neurotics mainly cause problems for themselves because they let their excessive \& unwarranted fears stifle their own success. \& this happens only in those relatively rare cases when neurosis is excessive. Contrarily, character-disordered personalities, unencumbered by qualms of conscience, passionately pursue their personal goals with indifference to -- \& often at the expense of -- the rights \& needs of others,\footnote{Millon, T. \textit{Modern Psychopathology}, (W. B. Saunders, 1969), p. 261.}, \& cause all sorts of problems for others \& society at large. A common saying among professionals is that if a person is making himself miserable, he's probably neurotic, \& if he's making everyone else miserable, he's probably character-disordered. Among the various personality types, submissive personalities are among the most neurotic \& the aggressive personalities are among the most character-disordered.

Very contrasting characteristics define mostly neurotic versus mostly character-disordered individuals. These differences are crucial to remember, whether you're a person in a problematic relationship or a therapist trying to understand \& remedy an unhealthy situation.'' -- \cite[pp. 29--31]{Simon2010}

\subsection{Neurotic Personality}

\begin{itemize}
	\item ``For neurotics, anxiety plays a major role in the development of their personality \& fuels their ``symptoms'' of distress.
	\item Neurotics have a well-developed, or perhaps even overactive conscience or superego.
	\item Neurotics have an excessive capacity for guilt \& or shame. This increases anxiety \& causes much of their distress.
	\item Neurotics employ defense mechanisms to help reduce anxiety \& protect themselves from unbearable emotional pain.
	\item Fear of social rejection prompts neurotics to mask their true self \& present a false fa\c{c}ade to others.
	\item The psychological ``symptoms'' of distress neurotics experience are ego-dystonic (i.e., experienced as unwanted \& undesirable). For this reason, neurotics often voluntarily seek help to alleviate their distress.
	\item Emotional conflicts underlie the symptoms reported by neurotics \& are the appropriate focus of therapy.
	\item Neurotics often have damaged or deficient self-esteem.
	\item Neurotics are hypersensitive to adverse consequence \& social rejection.
	\item Inner emotional conflicts that cause anxiety for neurotics \& the defense mechanisms they use to reduce this anxiety are largely unconscious.
	\item Because the root of problems is often unconscious, neurotics need \& often benefit from the increased self-awareness that traditional, \textit{insight-oriented} therapy approaches offer.'' -- \cite[pp. 31--32]{Simon2010}
\end{itemize}

\subsection{Disordered Character}

\begin{itemize}
	\item Anxiety plays a minor role in the problems experienced by the character-disordered individual (CDO). CDOs lack sufficient apprehension \& anxiety related to their dysfunctional behavior pattern.
	\item The extremely disordered character may have no conscience at all. Most CDOs have consciences that are significantly underdeveloped.
	\item CDOs have diminished capacities for experiencing genuine shame or guilt.
	\item What may appear a defense mechanism to some is more likely a power tactic used to manipulate others \& resist making concessions to societal demands.
	\item CDO individuals may try to manage your impression of them, but in basic personality, they are who they are.
	\item Problematic aspects of personality are ego-syntonic (i.e., CDOs like who they are \& are comfortable with their behavior patterns, even though who they are \& how they act might bother others a lot). They rarely seek help on their own but are usually pressured by others.
	\item Erroneous thinking patterns\texttt{/}attitudes underlie the problem behaviors CDOs display.
	\item CDOs most often have inflated self-esteem. Their inflated self-image is not a compensation for underlying feelings of inadequacy.
	\item CDOs are undeterred by adverse consequence or societal condemnation.
	\item The CDO's problematic behavior patterns may be habitual \& automatic, but they are conscious \& deliberate.
	\item The disordered character has plenty of insight \& awareness but despite it, resists changing his\texttt{/}her attitudes \& core beliefs. CDOs don't need any more insight. What they need \& can benefit from are limits, confrontation, \& most especially, correction. Cognitive-behavioral therapeutic approaches are the most appropriate.
\end{itemize}
As outlined, on almost every dimension, disturbed characters are very different from neurotic individuals. Most especially, disordered characters don't think the way most of us do. In recent years, researchers have come to realize the importance of recognizing the fact. How we think, what we believe, \& the attitudes we've developed largely determine how we will act. That's partly why current research indicates that cognitive-behavioral therapy (confronting erroneous thinking patterns \& reinforcing a person's willingness to change their thinking \& behavior patterns) is the treatment of choice for disturbed characters.

Research into the distorted thinking patterns of disordered characters began several years ago \& focused on the thinking patterns of criminals. Over the years, researchers have come to understand that problematic patterns of thinking are common to all types of disordered characters, I have adopted, modified, \& added to many of the known problematic patterns of thinking \& offer a brief summary of some of the more important ones:
\begin{itemize}
	\item \textbf{Self-Focused (self-centered) thinking.} Disordered characters are always thinking of themselves. They don't think about what others need or how their behavior might impact others. This kind of thinking leads to attitudes of selfishness \& disregard for social obligation.
	\item \textbf{Possessive thinking.} This is thinking of people as possessions to do with as I please or whose role it is to please me. Disturbed characters also tend to see others as objects (objectification) as opposed to individuals with dignity, worth, rights \& needs. This kind of thinking leads to attitudes of ownership, entitlement \& dehumanization.
	\item \textbf{Extreme (all-or-none) thinking.} The disordered character tends to think that if he can't have everything he wants, he won't accept anything. If he's not on top, he sees himself at the bottom. If someone doesn't agree with everything he says, he thinks they don't value his opinions at all. This kind of thinking keeps him from any sense of balance or moderation \& promotes an uncompromising attitude.
	\item \textbf{Egomaniacal thinking.} The disordered character so overvalues himself that he thinks that he is entitled to whatever he wants. He tends to think that things are owed him, as opposed to accepting that he needs to earn the things he desires. This kind of thinking promotes attitudes of superiority, arrogance, \& entitlement.
	\item \textbf{Shameless thinking.} A healthy sense of shame is lacking in the disturbed character. He tends not to care how his behavior reflects on him as a character. He may be embarrassed if someone exposes his true character, but embarrassment at being uncovered is not the same as feeling shameful about reprehensible conduct. Shameless thinking fosters an attitude of brazenness.
	\item \textbf{Quick \& easy thinking.} The disturbed character always wants things the easy way. He hates to put forth effort or accept obligation. He gets far more joy out of ``conning'' people. This way of thinking promotes an attitude of disdain for labor \& effort.
	\item \textbf{Guiltless thinking.} Never thinking of the rightness or wrongness of a behavior before he acts, the disturbed character takes whatever he wants, no matter what societal norm is violated. This kind of thinking fosters an attitude of irresponsibility \& anti-sociality.'' -- \cite[pp. 32--34]{Simon2010}
\end{itemize}

\subsection{Aggressive Personalities \& Aggressive Personality Subtypes}
``The personality theorist Theodore Million conceptualizes aggressive personalities as actively-independent\footnote{Millon, T. \textit{Disorders of Personality}, (Wiley-Interscience, 1981) p. 91.} in the way they interact with others \& deal with the world at large. He points out that these individuals actively take charge of getting their needs met \& resist depending on the support of others. He also suggests that there are 2 kinds of actively-independent personality, one able to conform his conduct well enough to function in society, \& the other unable to abide by the rule of law.\footnote{Millon, T. \textit{Disorders of Personality}, (Wiley-Interscience, 1981), p. 182.} I do not agree that the label ``aggressive'' best describes the interpersonal style of every subtype of actively independent personality. A person can adopt a style of actively taking care of himself without being truly aggressive about it. Such is the case with the assertive personality, which I regard as the healthiest of all personalities. But I wholeheartedly agree that there are many more types of aggressive personalities than career criminals \& it is unfortunate that the official psychiatric nomenclature only recognizes a small subtype of the active-independent personality, the \textit{antisocial} personality, as psychologically disordered.

Unlike th assertive personality, aggressive personalities pursue their interpersonal agendas with a degree of ruthlessness that bespeaks their disregard for the rights \& needs of others. Their core characteristics include a predisposition to meet life's challenges head-on \& with a steadfast determination to ``win,'' a feisty temperament \& mind-set, a maladaptive lack of fearfulness \& inhibitory control, a persistent desire to be in the dominant position, \& a particular kind of disdain \& disregard for those perceived as weaker. They are ``fighters'' to the core.

Aggressive personalities also share most of the characteristics of narcissists. In fact, some see this personality type as merely an aggressive variant of the narcissistic personality. Aggressive personalities are notoriously self-confident \& self-absorbed. Their wants, their agendas, their plans, etc. are all that matter to them. \& anyone or anything standing in their way must be rendered incapable of thwarting their goals.

Drawing from Millon's formulations about actively-independent personalities, some of the research on Type ``A'' (aggressive) personalities,\footnote{Keegan, D., Sinha, B. N., Merriman, J. E., \& Shipley, C. \textit{Type A Behavior Pattern}. Canadian Journal of Psychiatry, 1979, 24, 724-730.} emerging research on some of the most severely aggressive personalities, \& years of clinical experience working with disturbed characters of all sorts, I find it useful to categorize 5 basic aggressive personality types: the unbridled-aggressive, channeled-aggressive, sadistic, predatory (psychopathic) \& covert-aggressive personalities. Although they have much in common with one another, each of these aggressive personality types has some clearly unique defining characteristics. Some are more dangerous than others \& some are more difficult to understand than others. But all of the aggressive personalities pose considerable challenges to those who have to work for them, live with them, or labor under their influence.

\textbf{Unbridled-Aggressive} personalities are openly hostile, frequently violent \& often criminal in their behavior. These are the people we commonly label \textit{antisocial}. They tend to be easily angered, lack adaptive fearfulness or cautiousness, are impulsive, reckless, \& risk-taking, \& are overly prone to violate the rights of others. Many spend a good deal of their lives incarcerated because they simply won't conform, even when it's in their best interest. Traditional thinking on these personalities has always been that they are the way they are because they grew up in circumstances that made them mistrust authority \& others \& were too scarred from abuse \& neglect to adequately ``bond'' to others. My experience over the years has convinced me that some of these overtly aggressive personalities have indeed been fueled in their hostility by an inordinate mistrust of others. An even smaller number appear to be biologically predisposed to extreme vigilance \& suspiciousness (i.e., have some paranoid personality traits as well). But my experience has taught me that most unbridled aggressive personalities are not so much driven by mistrust \& suspicion, but rather an excessive readiness to aggress, even when unnecessary, unprompted, or fueled by anger. They will aggress without hesitation or regard to consequence either to themselves or others. \& a fair number of these individuals do not have abuse, neglect, or disadvantage in their backgrounds. Indeed, some were the beneficiaries of the best circumstances. So, many of our traditional assumptions about these personalities are being re-evaluated. 1 researcher has noted that about the only reliable common factor he could find among all of the various ``criminal personalities'' he had worked with was that they all seemed to enjoy engaging in illicit activity.

\textbf{Channeled-Aggressives} are overtly aggressive personalities who generally confine their aggression to socially acceptable outlets such as business, sports, law enforcement, the legal profession \& the military. These people are often rewarded for being tough, headstrong, \& competitive. They may openly talk about ``burying'' the competition or ``crushing'' their opponents. They don't usually cross the line into truly antisocial behavior but it really shouldn't surprise anyone when they do. That's because their social conformity is often more a matter of practicality rather than a true submission to a set of principles or higher authority. So, they'll break the rules \& inflict undue harm on others when they feel justified in so doing, or when they think they can get away with it.

The \textbf{Sadistic Aggressive} personality is another overtly aggressive personality subtype. Like all other aggressive personalities, they seek positions of power \& dominance over others. But these individuals gain particular satisfaction from seeing their victims squirm \& grovel in positions of vulnerability. For the other aggressive personality types, inflicting pain or injury on anyone standing in the way of something they want are seen as merely hazards of the fight. \textit{Most of the aggressive personalities don't set out to hurt, they set out to win}. The way they see it, if someone has to get hurt for them to have their way, then so be it. The sadist, however, \textit{enjoys} making people grovel \& suffer. Like the other aggressive personalities, sadists wants to dominate \& control, but they particularly enjoy doing that by humiliating \& denigrating their victims.

The \textbf{Predatory Aggressive} is the most dangerous of the aggressive personalities (also referred to by some as the psychopath or sociopath). There is perhaps no more learned expert on this topic than Robert Hare, whose book \textit{Without Conscience} is a chilling but very readable \& valuable primer on the subject. Fortunately, as a group, psychopaths are relatively uncommon. However, in my career I have encountered \& dealt with a fair number of them. These characters are radically different from most people. Their lack of conscience is unnerving. They tend to see themselves as superior creatures for whom the inferior, common man is rightful \textit{prey}. They are the \textit{most extreme manipulators} or con artists who thrive on exploiting \& abusing others. They can be charming \& disarming. As high skilled predators, they study the vulnerabilities of their prey carefully \& are capable of the most heinous acts of victimization with no sense of remorse or regret. Fortunately, most manipulators aren't psychopaths.

The various aggressive personalities have certain characteristic in common. They are all excessively prone to seek a position of power \& dominance over others. They are all relatively uninhibited by the threat of punishment or pangs of conscience. They also tend to view things \& to think in ways that distort reality of circumstances, prevent them from accepting \& exercising responsibility over their behavior, \& ``justify'' their overly aggressive stance. Their distorted, erroneous patterns of thinking have been the subject of much recent research.\footnote{Samenow, S. \textit{Inside the Criminal Mind}, (Random House, 1984).} Because the various aggressive personality types have so much in common, it's not unusual for 1 subtype to possess some of the characteristics of another. So, predominantly antisocial personalities may have some sadistic as well as covert-aggressive features \& covert-aggressives may have some antisocial tendencies, etc.

As mentioned earlier, all of the aggressive personalities have many characteristics in common with narcissistic personalities. Both display ego-inflation \& attitudes of entitlement. Both are exploitive in their interpersonal relationships. Both are emotionally independent personalities. I.e., they rely on themselves to get what they need. Millon describes narcissists as passive-independent personalities\footnote{Millon, T. \textit{Modern Psychopathology}, (W. B. Saunders, 1969), p. 260.} because they think so much of themselves that they believe that they just don't need anybody else to get along in life. They don't necessarily have to \textit{do} anything to demonstrate their competence \& superiority. They're always convinced of it. \& while narcissists are so self-centered \& absorbed that they might passively disregard the rights \& needs of others, the aggressive personalities, by contrast, actively engage in behaviors designed to secure \& maintain their independence \& actively trample upon the rights of others to secure their goals \& maintain a position of dominance over others.'' -- \cite[pp. 35--39]{Simon2010}

\subsection{The Covert-Aggressive Personality}
``As an aggressive personality subtype, one might expect covert-aggressives to share some of characteristics of narcissists as well as the other aggressive personalities. But covert-aggressives have many unique attributes that make them a truly distinct type of aggressive personality. These personalities are mostly distinguished from the other aggressive personality types by the way that they fight. They fight for what they want \& seek power over others in subtle, cunning \& underhanded ways. On balance, they are much more character disordered than neurotic. To the degree they might have some neurosis, they deceive themselves about their true character \& their covertly-aggressive conduct. To the degree they are character disordered, the more they actively attempt to deceive only their intended victims.

The covert-aggressive's dislike of appearing overtly aggressive is as practical as it is face-saving. Manipulators know that if they're above-board in their aggression, they'll encounter resistance. Having learned that 1 of the best ways to ``overcome'' an obstacle is to ``go around'' it, they're adept at fighting unscrupulously yet surreptitiously.

Some personality theorists have proposed that the cardinal quality of the covert-aggressive or manipulative personality is that the derive an inordinate sense of exhilaration from pulling the wool over the eyes of their victims.\footnote{Bursten, B. \textit{The Manipulative Personality}, Archives of General Psychiatry, 1972, p. 318.} But I believe their man agenda is the same as that of the other aggressive personalities. They just want to win \& have found covert ways of fighting to be the most effective way to meet their objective. I have found these to be their major attributes:
\begin{enumerate}
	\item Covert-aggressives always want to have their way or to ``win.'' For them, as with all aggressive personalities, every life situation is a challenge to be met, a battle to be won.
	\item Covert-aggressives seek power \& dominance over others. They always want to be 1-up \& in control. They use an arsenal of subtle but effective power tactics to gain \& keep the advantage in their interpersonal relations. They use certain tactics that make it more likely that others will go on the defensive, retreat, or concede while simultaneously concealing their aggressive intent.
	\item Covert-aggressives can be deceptively civil, charming \& seductive. They know how to ``look good'' \& how to win you over by ``melting'' your resistance. They know what to say \& do to get you to abandon any intuition mistrust \& give them what they want.
	\item Covert-aggressives can also be unscrupulous, underhanded, \& vindictive fighters. They know how to capitalize on any weakness \& will intensify their aggression if they notice you faltering. They know how to catch you unaware \& unprepared. \& if they think you're thwarted or gotten the better of them, they'll try to get you back. For them, the battle is never over until they think they've won.
	\item Covert-aggressives have uniquely impaired consciences. Like all aggressive personalities, they lack internal ``brakes.'' They know right from wrong, but won't let that stand in the way of getting what they want. To them, the ends \textit{always} justify the means. So, they deceive themselves \& others about what they're really doing.
	\item Covert-aggressives are abusive \& exploitive in their interpersonal relations. They view people as pawns in the game (contest) of life. Detesting weakness, they take advantage of every frailty they find in their ``opponents.'' 
\end{enumerate}
As is the case with any other type of personality, covert-aggressives vary in their degree of psychopathology. The most seriously disturbed covert aggressives go far beyond just being manipulative in their interpersonal style. Severely disturbed covert aggressives are capable of masking a considerable degree of ruthlessness \& power-thirstiness under a deceptively civil \& even alluring social fa\c{c}ade. Some may even be psychopathic. Jim Jones \& David Koresh are good examples. But even though a covert-aggressive personality can be a lot more than just a manipulator, habitual manipulators are most always covert-aggressive personalities.'' -- \cite[pp. 39--41]{Simon2010}

\subsection{Distinguish Covert-Aggressives From Passive-Aggressive \& Other Personality Types}
``Just as passive \& covert-aggression are very different behaviors, passive-aggressive \& covert-aggressive personalities are very different from one another. Millon describes the passive-aggressive or ``negativistic'' personality as one who is actively ambivalent about whether to adopt a primarily independent or dependent style of coping.\footnote{Millon, T. \textit{Modern Psychopathology}, (W. B. Saunders, 1969), p. 287.} These individuals want to take charge of their own life, but fear they lack the capability to do so effectively. Their ambivalence about whether to primarily fend for themselves or lean on others puts them \& those in relationships with them in a real bind. They chronically crave \& solicit support \& nurturance from others. But because they also resent being in positions of dependence \& submission, they often try to gain some sense of personal power by resisting cooperation with the very people from whom they solicit support. Waffling on a decision, they might complain that you decide. When you do, they hesitate to go along. In an argument with you, they may get fed up \& want to disengage. But afraid that if they truly disengage they might be emotionally abandoned, they'll stay \& ``pout'' until you plead with them to tell you what's wrong. Life with passive-aggressive personalities can be very difficult because there often seems to be no way to please them. Although he frequently fails to distinguish passive from covert-aggression, Scott Wetzler characterizes the passive-aggressive personality \& what life is like with such individuals quite well in his book \textit{Living with the Passive-Aggressive Man}.\footnote{Wetzler, S. \textit{Living with the Passive-Aggressive} Man, (Simon \& Schuster, 1992).}

Passive-aggressive patients in therapy are legend. They may ``whine'' \& complain about the lack of support they're getting from the therapist. But as soon as the therapist tries to give them something, they inevitably start ``bucking'' the therapist's suggestions with ``yes $\ldots$, but'' statements \& other subtle forms of passive-resistance. Most therapists can readily distinguish these actively ``ambivalent'' personalities who are driven by a hypersensitivity to shame from the more cunning, calculating manipulators I call covert-aggressive. But sometimes, unfamiliar with the more accurate term, \& wanting to highlight the subtle aggression manipulators display, therapists often misuse the label ``passive-aggressive'' to describe manipulators. Covert-aggressive personalities are not the same as obsessive-compulsive personalities. We all know perfectionistic, meticulous \& highly organized people. When they are reviewing our tax returns of performing brain surgery, we value these attributes quite highly. Yes, \textit{some} compulsive people can be forceful, authoritarian, domineering \& controlling. But that's because these kinds of people are also covertly aggressive. A person can use their purported commitment to principles \& standards as a vehicle for wielding power \& dominance over others. Obsessive-compulsive people who are also covertly aggressive are the kind of people who attempt to shove their own standards down everyone else's throats.

Covert-aggressive personalities are not identical to narcissistic personalities, although they almost always have narcissistic characteristics. People who think too much of themselves don't necessarily attempt to manipulate others. Narcissists can passively disregard the needs of others because of how absorbed they are with themselves. Some self-centered people, however, actively disregard the needs others \& intentionally victimize \& abuse them. Recognizing this, some writers have distinguished the benign from the malignant narcissist. But I think the difference between the kind of person who is too self-absorbed to be inattentive to the rights \& needs of others \& the kind of person who habitually exploits \& victimizes is that the latter, in addition to being narcissistic, is distinctly aggressive. So, egotists who cleverly exploit \& manipulative others are not just narcissistic, they're also covertly aggressive personalities.

Most covert-aggressive personalities are not antisocial. Because they have a disregard for the rights \& needs of others, have very impaired consciences, actively strive to gain advantage over others, \& try to get away with just about anything short of blatant crime or overt aggression, it's temping to label them antisocial. Indeed, some antisocial individuals use manipulation as part of their overall modus operandi. However, manipulators don't violate major social norms, lead lives of crime, or violently aggress against others, although they are capable of these things. Several attempts have been made to accurately describe the calculating, underhanded, controlling interpersonal style of manipulative people. They've been called all sorts of things from sociopathic to malignantly narcissistic, \& even, as Scott Peck suggests, ``evil.''\footnote{Peck, M.S., \textit{People of the Lie}, (Simon \& Schuster, 1983).} Sensing the subtly aggressive character of their behavior, many have called them passive-aggressive. But non of these labels accurately defines the core characteristic of manipulative personalities. It's important to recognize that for the most part, manipulation involves covert-aggression \& habitual manipulators are covert-aggressive personalities.

It's also important to remember that a manipulative person may have other personality characteristics in addition to their covert-aggressive propensities. So, in addition to being manipulative, they may have narcissistic, obsessive-compulsive, antisocial or other tendencies. But as a friend of mine once remarked, ``It may have short ears \& it may have long ears; it may have a lot of hair \& it may have no hair at all; it may be brown or it may be gray; but if it's big \& has tusks \& a trunk, it's always an elephant.'' As long as the person you're dealing with possesses the core attributes outlined earlier, no matter what else they are, they're a covert-aggressive personality.

Because the predatory aggressive or psychopathic personality is so adept at manipulation, some might tend to view the covert-aggressive personality as a milder variant of the psychopath. This is a fair perspective. Psychopaths are the most dangerous, cunning, \& manipulative of the aggressive personalities. Fortunately, however, they are also the most uncommon. The manipulative personalities described in this book are much more common \&, although they can wreak a good deal of havoc in the lives of their victims, they are not as dangerous as psychopaths.'' -- \cite[pp. 41--44]{Simon2010}

\subsection{How Someone Becomes a Covert-Aggressive Personality}
``How any aggressive personality gets to be the way they are varies. I have seen individuals whose early lives were so full of abuse \& neglect that they had to become strong ``fighters'' just to survive. I've also met plenty of individuals who seemed to have fought too much all of their lives despite growing up in the most nurturing \& supportive environments. These persons seem to have ``bucked'' the process of socialization from early on \& their character development appears to have been heavily influenced at every stage by their excessive combativeness. But regardless of whether nature or nurture is the stronger influence, somehow in their childhood development, most covert-aggressive personalities seem to have over-learned some, \& failed to learn other essential lessons about managing their aggression. Judging from the histories with which I am familiar, covertly aggressive personalities typically exhibit the following learning failures:
\begin{enumerate}
	\item They never learned when fighting is really necessary \& just. To them, daily living is a battle \& anything that stands in the way of something they want is the ``enemy.'' Obsessed with ``winning,'' they're far too willing \& too ready to fight.
	\item They never allowed themselves to learn that ``winning'' in the long-run is often characterized by a willingness to give ground, concede, or submit in the short-run. They failed to recognize those times when it's best to acquiesce. Their total aversion to submission prevents them from making the little concessions in life that often lead to ``victory'' later on.
	\item They never learned how to fight constructively or fairly. They might have learned to mistrust their ability to win a fair fight. Perhaps they were never willing to run the risk of losing. Sometimes, it's just because the found covert fighting to be so effective. Whatever the case, somehow they overlearned how to ``win'' (at least, in the short-run) by fighting underhandedly \& surreptitiously.
	\item Because they detest submission, they never allowed themselves to learn the potentially constructive benefits of admitting defeat. I think this dynamic is at the heart of the apparent failure of all aggressive (\& character-disordered) personalities to learn what we want them to learn from past experience. Truly learning (i.e. internalizing) a lesson in life always involves submitting oneself to a higher authority, power, or moral principle. The reason aggressive personalities don't change is because they don't submit.
	\item They never learned to get beyond their childish selfishness \& self-centeredness. They failed to realize that they're not necessarily entitled to go after something just because they want it. To them, the entire world is their oyster. Having become skilled at getting their way through manipulation, they come to think of themselves as invincible. This further inflates their already grandiose self-image.
	\item They never learned genuine respect or empathy for the vulnerabilities of others. To them, everyone else's weakness is simply their advantage. Having only disdain for weakness, especially emotional weakness, they over-learned how to find \& push their victims' emotional ``buttons.'''' -- \cite[pp. 44--46]{Simon2010}
\end{enumerate}

\subsection{Fertile Ground for Covert-Aggression}
``Some professions, social institutions \& fields of endeavor provide great opportunities for covert-aggressive personalities to exploit others. Politics, law enforcement \& religion are some prime examples. I am not implying that all politicians, law enforcement professionals or religious leaders are manipulative personalities. However, covert power-seekers that they are, manipulators cannot help but gravitate toward \& exploit the excellent opportunities for self-advancement \& the wielding of considerable power under the guise of service available in such endeavors. The tele-evangelists, cult leaders, political extremists, Sunday night TV ``success'' peddlers \& militant social activists who have been exposed in the headlines lately are no different in their overall modus operandi from the covert-aggressives we encounter in everyday life. They're just more extreme cases. The more cunning \& skilled at using the tactics of manipulation a covert-aggressive is, the easier it is for them to rise to a position of substantial power \& influence.'' -- \cite[p. 46]{Simon2010}

\subsection{Understanding \& Dealing with Manipulative People}
``It's easy to fall victim to the covert-aggressive's ploys. Anyone wanting to avoid victimization will need to:
\begin{enumerate}
	\item Get intimately acquainted with the character of these wolves in sheep's clothing. Get to know what they really want \& how they operate. Know them so intimately that you can always spot one when you encounter one. The stories in the following chapters are written in a genre that will hopefully make it easier for you to get the ``flavor'' of the covert-aggressive personality.
	\item Become acquainted with the favorite tactics covertly aggressive people use to manipulate \& control others. We not only need to know what covert-aggressives are like, but also what kinds of behaviors we should expect from them. In general, we can expect them to do whatever it takes to ``win,'' but knowing their most common ``tactics'' well \& recognizing when they are being used is most helpful in avoiding victimization.
	\item Become aware of the fears \& insecurities most of us possess that increase our vulnerability to the covert-aggressive's ploys. Knowing your own weaknesses can be your foremost strength in dealing more effectively with a manipulator.
	\item Learn what changes you can make in your own behavior to reduce your vulnerability to victimization \& exploitation. Using techniques such those presented in Chap. 10 can radically change the nature of your interactions with others \& empower you to deal more effectively with those who would otherwise manipulate \& control you.
\end{enumerate}
The stories in the next few chapters are designed to help you become more intimately acquainted with the character of manipulative people. Each chapter highlights 1 of the distinguishing characteristics of covertly aggressive personalities. In each story, I'll attempt to highlight the manipulators' main agendas, what power tactics they employ to advance them, \& the weaknesses they exploit in their victims.'' -- \cite[pp. 46--47]{Simon2010}

%------------------------------------------------------------------------------%

\section{The Determination to Win}
``The primary characteristic of covert-aggressive personalities is that they value winning over everything. Determined, cunning \& sometimes ruthless, they use a variety of manipulative tactics, not only to get what they want, but also to avoid seeing themselves or being seen by others as the kind of people they really are. The story of Joe \& Mary Blake will give you an idea of how much pain can enter the lives of members of a family in which 1 person, under the guise of care \& concern, is too determined to have his way.'' -- \cite[p. 48]{Simon2010}

\subsection{The Father Who Wanted A's}

%------------------------------------------------------------------------------%

\section{The Unbridled Quest for Power}
``Nothing is more important to any aggressive personality than gaining power \& achieving a position of dominance over others. In real estate, there is the old adage that 3 things are important: location, location, \& location. For any aggressive personality, only 3 things matter: position, position, \& position! Now, we all want some sense of power in our lives. That's not unhealthy. But how ambitiously we pursue it, how we go about preserving it, \& how we use it when we have it says a lot about the kind of person we are. Covert-aggressives are ruthlessly ambitious people but they're careful not to be perceived that way. The following story is about a man of the cloth who lies to himself \& his family about the real master he serves.'' -- \cite[p. 54]{Simon2010}

%------------------------------------------------------------------------------%

\section{The Penchant for Deception \& Seduction}
``Dealing with covert-aggressive personalities is like getting whiplash. Often, you don't really know what's hit you until long after the damage is done. If you've been involved in some way with 1 of these smooth operators, you know how charming \& disarming they can be. They are the masters of deception \& seduction. They'll show you what you want to see \& tell you what you want to hear. The following story is an example of a man who knows well how to charm \& beguile anyone while retaining the capacity to cut out their heart.'' -- \cite[p. 59]{Simon2010}

\subsection{The Story of Don \& Al}

%------------------------------------------------------------------------------%

\section{Fighting Dirty}
``Some says that it's dog-eat-dog in the business world \& one has to claw one's way to the top. But there's a difference between the fair competition that breeds excellence \& the crafty, underhanded maneuvering that sometimes wreaks havoc in the workplace. Having to work with a covertly aggressive co-worker can be a significant source of occupational stress.

The following is the story of a woman who never fights openly or fairly for what she wants. Neither her drive, ambition, nor her desire for power \& position are problems in themselves. Properly managed, these are desirable traits in anyone trying to get ahead in their organization \& help their co-workers achieve excellence. The really disturbing thing about her is the devious way she goes about getting what she wants.'' -- \cite[p. 63]{Simon2010}

\subsection{The Most Dedicated Woman in the Company}

%------------------------------------------------------------------------------%

\section{The Impaired Conscience}
``Aggressive personalities don't like anyone pushing them to do what they don't want to do or stopping them from doing what they want to do. ``No'' is never an answer they accept. Because they so actively resist any constraints on their behavior or desires, they have trouble forming a healthy conscience.

Conscience can be conceptualized as a self-imposed barrier to an unchecked pursuit of personal goals. It's a person's internal set of ``brakes.'' Aggressive personalities resist society's exhortation to install these brakes. They tend to \textit{fight he socialization process} early on. If they're not too aggressively predisposed, \& if they can see some benefit in self-restraint, they might internalize some inhibitions. But generally, any conscience they do form is likely to be significantly impaired. This is the heart of conscience development: \textbf{\textit{Internalization} of a societal prohibition is the definitive act of \textit{submission}.} Because all of the aggressive personalities \textit{detest} \& \textit{resist} submission, they necessarily develop impaired consciences.

The conscience of covert-aggressives is uniquely impaired in several ways. By refraining from overt acts of hostility towards others, they manage to convince themselves \& others they're not the ruthless people they are. They may observe the letter of a law but violate its spirit with ease. They may exhibit behavioral constraint when it's in their best interest, but they resist truly submitting themselves to any higher authority or set of principles. Many people have asked me if I'm really sure that covert-aggressives are as calculating \& conniving as I describe them. ``Maybe they just can't help it,'' they tell me or ``they must do these things unconsciously.'' While some covert-aggressives are to some extent neurotic \& therefore prone to deceiving themselves about their aggressive intentions, most of the covert-aggressives I've encountered have been primarily character disordered, striving primarily to conceal their true intentions \& aggressive agendas from others. They may behave with civility \& propriety when they're closely scrutinized or vulnerable. But when they believe they're immune from detection or retribution, it's an entirely different story. The following case is an example.'' -- \cite[pp. 69--70]{Simon2010}

\subsection{The Story of Mary Jane}

%------------------------------------------------------------------------------%

\section{Abusive, Manipulative Relationships}
``Covert-aggressives use a variety of ploys to keep their partners in a subordinate position in relationships. Of course, it takes 2 people to make a relationship work \& each party must assume responsibility for their own behavior. But covert-aggressives are often so expert at exploiting the weaknesses \& emotional insecurities of others that almost anyone can be duped. Persons in abusive relationships with covert-aggressives are often initially seduced by their smooth-talking, outwardly charming ways. By the time they realize their partner's true character, they've usually put a significant emotional investment into trying to make the relationship work. This makes it very hard to simply walk away.'' -- \cite[p. 75]{Simon2010}

\subsection{The Woman Who Couldn't Walk Away}

%------------------------------------------------------------------------------%

\section{The Manipulative Child}
``For many years professionals have focused on how children's fears \& insecurities influence their personality development. But they haven't given much attention to how children learn to discipline \& channel their aggressive instincts. It seems that when it comes to examining \& dealing with the truth about why \& how children fight, \& how the degree of their aggressiveness shapes their personalities, professionals have exhibited a major case of denial.

Children naturally fight for what they want. Early in their social development they fight openly \& often physically. For most children, this strategy proves unsuccessful \& invites substantial social sanction. If their parents are skilled enough at discipline, their social environments benign enough, \& if the children themselves are malleable enough, most children learn to modulate their overtly aggressive tendencies \& will explore other strategies for winning life's battles. Along the way, many will discover the emotional ``buttons'' their parents \& others possess that, when pressed, prompt them to back down or give ground in a conflict. They also learn the things that they can say or do (or fail to say or do) that will keep their ``opponents'' in the dark, off balance or on the defensive. These children then learn to fight covertly.

As the result of many social factors (permissiveness, indulgence, abuse, neglect, \& lack of accountability), it seems that there is an increasing number of overly aggressive \& covertly aggressive (manipulative) children these days. My perspective may be biased because about half of my work in the early years was with emotionally \& behaviorally disturbed children, adolescents \& their families. However, I'm constantly impressed by the number of cases I see in which a child has managed to gain inordinate power in the family as a result of learning all too well the tactics of manipulation. The following story is based on 1 of these cases.'' -- \cite[pp. 83--84]{Simon2010}

\subsection{Amanda the Tyrannical Child}

%------------------------------------------------------------------------------%

\begin{center}\LARGE\sf
	\textbf{Part II: Dealing Effectively with Manipulative People}
\end{center}

\section{Recognizing the Tactics of Manipulation \& Control}

\subsection{Defense Mechanisms \& Offensive Tactics}
``Almost everyone is familiar with the term \textit{defense mechanism}. Genuine defense mechanisms are the almost reflexive mental behaviors we sometimes employ to shield ourselves from the ``threat'' of some type of emotional pain. More specifically, ego defense mechanisms are mental behaviors people might use to ``defend'' their self-images from anxiety associated with societal ``invitations'' to feel ashamed or guilty about something. There are many different kinds of ego defenses, several of which are well known \& have made their way into common discourse.

The use of defense mechanisms is 1 of the cardinal tenets of traditional or psychodynamic approaches to understanding human behavior. In fact, these approaches have always tended to distinguish the various personality types, at least in part, by the types of ego defenses they are believed to most commonly use. As discussed briefly earlier, there are some characteristics of traditional approaches to understanding human behavior \& personality that do not really help us understand the disturbed character. Traditional approaches assert that people necessarily experience guilt, shame, \& anxiety when they do something wrong. They also claim that people defend themselves against ``threats'' to their self-image by using the automatic behaviors we call defense mechanisms. Finally, they maintain that people do so \textit{unconsciously}.

Traditional models of human behavior \& personality are not helpful when it comes to understanding the character disturbed individual. When disturbed characters engage in certain behaviors, some of which we have often called defense mechanisms, they don't do so primarily to protect against against emotional pain, guilt or shame. Nor do they do so to keep a feared event from happening. Rather, disturbed characters engage in these behaviors \textit{primarily} to ensure that some desired event does indeed happen, to manipulate \& control others, \& to solidify their resistance to accepting or internalizing social norms. They use them as vehicles to keep doing what society says we shouldn't do \&, as a result, they don't develop a healthy sense of guilt or shame. Furthermore, for the most part they engage in these behaviors \textit{consciously} even though habitual use prompts them to be employed nearly reflexively. So, many of the behaviors we have traditionally thought of as defense mechanisms more rightfully should be thought of as responsibility-avoidance behaviors \& \textit{tactics of manipulation \& control} when they are employed by disturbed characters.

Let's take the mechanisms of denial, e.g. Almost everyone has heard someone say something like: ``Sure, he has a problem, but he's \textit{in denial} about it.'' Most of the time, this term is misused. The true defense mechanism of denial is a \textit{psychological state} unconsciously employed to protect a person from unbearable emotional pain. Take the case of Agnes, an elderly woman still in relatively good health who has just been told by doctors at the hospital that the stroke her husband of 40 years has just suffered is critical \& means he likely won't recover. Paul has been her lover \& beloved partner for most of her adult life \& she is not prepared to lose him. She faces the prospect of being alone \& without his steadfast support. Life without him, she thinks, would be unbearable. So, despite the fact the brainwave charts are flat, she stays by his side, day after day, holding his hand, talking to him, \& insisting to those who tell her otherwise that she knows he'll make it -- he always has. This woman is ``in denial.'' She is not intentionally doing so, but unconsciously she is protecting herself against the sudden \& unbearable experience of the intense grief she will experience when reality eventually sets in. Over time, when she is more psychologically prepared to suffer the trauma, her denial mechanism will break down. When it finally does, she will be without the protection that kept her from the experience of pain, \& what will burst forth is an avalanche of emotion.

Contrast the aforementioned scenario with the case of Jeff, a character-disturbed adolescent called out by his junior high hall monitor for bullying an underclassman by shoving his books on the floor. ``What?'' he retorts. ``I didn't do anything!'' He is denying the behavior, but is he in a psychological state of denial? No! The classical perspective suggests: (1) underneath the pretense, he feels bad about what he did, (2) to \textit{defend} himself against unbearable feelings of shame \&\texttt{/}or guilt he simply can't admit to himself or anyone else what he did; \& (3) he consciously has no idea what he's doing. These are dangerous presuppositions, but ones that laypersons \& many professionals frequently make. They are also assumptions that, when it comes to the disturbed character, are \textit{completely erroneous}. The more accurate perspective is that Jeff is fairly lacking in guilt, shame, or anxiety about his behavior, which is why he so unhesitatingly committed the acts in the 1st place. What is also likely is that he hasn't made the commitment to deal with people in a non-aggressive way. Although other people aren't comfortable with his ways, he is. Because he has likely been chastised many times before for his problem behaviors, he's well aware that others view it as unacceptable. However, he's not prepared to submit himself to the standard of conduct others want him to adopt. He is also very aware of the likely consequences the hall monitor has in store for him. He may not want to face those consequences just as much as he doesn't want to change his style. So, his best bet is to try \& convince the hall monitor that she is in error, that she didn't see what she thought she saw, that she has him judged all wrong, that she should back off. In short, when Jeff is denying, he's \textit{not defending} in any way, he's mainly \textit{fighting}. He's not in a psychological state, he's \textit{employing a tactic}, \& he's very aware of what he's doing. The tactic he's using is often called denial, but it's really just a simple case of \textit{lying}. He's lying for the reasons people commonly lie: to get out of trouble. Proof positive could come when the hall monitor calls 2 or 3 other witnesses in front of him \& they all verify what the monitor saw. Jeff may then say something like ``Okay, okay. Maybe I shoved him a little. But he had it coming. He's been bugging me all week.'' Now, the traditionalists would say he's ``come out of his denial.'' But unlike Helen, we don't see what we usually see when someone truly comes out of such a psychological state. \textit{We don't see pain}. We don't see Jeff break down with grief. Instead, we see him making only a half-hearted admission \& he continues to adamantly fight submission to the principle we want him to adopt. We see neither signs of shame nor guilt. We see only signs of defiance.

A most important thing to remember about Jeff's behavior is that although he lied quickly, automatically, \& likely out of longstanding habit, he didn't do so unconsciously. \textit{H knew what he was doing}. Acting innocent \& denying something horrible so vehemently that your ``accuser'' begins to doubt the legitimacy of their complaint, is, from Jeff's abundant experience, an effective combat tool. It has gotten him out of trouble before, \& he hopes it will work again. Remember, behaviors that are habitual \& automatic are not the same thing as behaviors that are unconscious.

All character-disordered individuals, especially aggressive personalities, use a variety of mental behaviors \& interpersonal maneuvers to help ensure they get what they want. The behaviors soon to be enumerated in this chapter simultaneously accomplish several things that can lead to victimization. 1stly, they help conceal the aggressive intent of the person using them. 2ndly, their use frequently puts others on the defensive. 3rdly, their habitual use reinforces the user's dysfunctional but preferred way of dealing with the world. They obstruct any chance that the aggressor will accept \& submit to an important social principle at stake, \& thus change their ways. Lastly, because most people don't know how to correctly interpret the behaviors, they are effective tools to exploit, manipulate, abuse, \& control others. If you're 1 of those persons more familiar with traditional psychological models, you may tend to view a person using 1 of these behaviors as being ``on the defensive.'' But viewing someone who's in the act of aggressing as being defensive in any sense is a major set-up for victimization. Recognizing that when a person uses the behaviors soon to be described is primarily a person on the offensive mentally prepares you for the decisive action you might need to take to avoid being run over.

It's not possible to list all the tactics a good manipulator is capable of using to hoodwink or gain advantage over others. But the automatic mental behaviors \& interpersonal maneuvers enumerated below are some of the more popular weapons in the arsenal of disturbed characters in general, aggressive personalities in particular, \& especially covert-aggressives. It is important to remember that when people display these behaviors, they are at that very moment \textit{fighting}. They are fighting against the values of standards of conduct they know others want them to adopt or internalize. They are also fighting to overcome resistance in others \& to have their way.

Covert-aggressive individuals are especially adept at using these tactics to conceal their aggressive intentions while simultaneously throwing their opponents on the defensive. When people are on the defensive, their thoughts tend to become more confused, they tend to engage in more self-doubt, \& they feel the urge to retreat. Using these tactics increases the chances manipulators will get their way \& gain advantage over their victims. Sometimes, a tactic is used in isolation. More often, however, a skilled manipulator will throw so many of them at you at once that you might not really realize how badly you've been manipulated until it's too late.

\textbf{Minimization} -- This tactic is a unique kind of denial coupled with rationalization. When using this maneuver, the aggressor is attempting to assert that his behavior isn't really as harmful or irresponsible as someone else may be claiming. It's the aggressor's attempt to make a molehill out of a mountain. The use of minimization clearly illustrates the difference between the neurotic individual \& the disturbed character. Neurotics frequently make mountains out of molehills, or ``catastrophize.'' The disturbed character frequently trivializes the nature of his wrongdoing. Manipulators do this to make a person who might confront them feel they've been overly harsh in their criticism or unjust in their appraisal of a situation.

In the story of Janice \& Bill, Bill \textit{minimized} his substance use problem by insisting he didn't have much of a drinking problem \& asserting that binges occurred \textit{only} when he was very stressed or feeling unsupported by Janice. Janice initially bought into this minimization, saying to herself that because his drinking wasn't always unbearable, his substance use pattern wasn't that serious.

I've encountered hundreds of examples over the years of aggressive personalities of all types minimizing the nature \& impact of their aggressive conduct. ``Maybe I touched her once, but I didn't hit her.'' ``I pushed her a little, but I didn't leave any marks,'' they might say. They frequently use 2 ``4-letter words'' I forbid in therapy: \textit{just} \& \textit{only}. The story is always the same. What they mean to do is convince me that I would be wrong to conclude that their behavior was really as wrong as they know I suspect. Minimization is not primarily the way they make themselves feel better about what they did, it's primarily the way they try to manipulate by impression of them. They don't want me to see them as a person who behaves like a thug. Remember, they are most often comfortable with their aggressive personality style, so their primary objective is to get me to believe that there's nothing wrong with the kind of person they are.

\textbf{Lying.} It's hard to tell when a person is lying at the time they're doing it. Fortunately, there are time when the truth will out because circumstances don't bear out somebody's story. But there are times when you don't know you've been deceived until it's too late. 1 way to minimize the chances that someone will put one over on you is to remember that because aggressive personalities of all types will generally stop at nothing to get what they want, you can expect them to lie \& cheat. Another thing to remember is that manipulators -- covert-aggressive personalities that they are -- are prone to lie in subtle, covert ways. Someone was well aware of the many ways there are to lie when they suggested that court oaths charge a person to tell ``the truth, the whole truth, \& nothing but the truth.'' Manipulators \& other disturbed characters have refined lying to nearly an art form.

It's very important to remember that disturbed characters of all sorts lie frequently -- sometimes just for sport -- \& lie readily, even when the truth would easily suffice. \textbf{Lying by omission} is a very subtle form of lying that manipulators are. So is lying by \textit{distortion}. Manipulators will withhold a significant amount of the truth from you or distort essential elements the truth to keep you in the dark. I have treated individuals who have lied most egregiously by reciting a litany of true facts! How does someone lie by saying only true things? They do so by leaving out facts essential to knowing the bigger picture or ``whole story.''

1 of the most subtle forms of distortion is being deliberately vague. This is a favorite tactic of manipulators. They will carefully craft their stories so that you form the impression that you've been given information \textit{but leave out essential details that would have otherwise made it possible for you to know the larger truth}.

In the story of Al \& Don, Al didn't tell the whole truth when Don inquired about the safety of his job. It was a smooth, calculated \textit{omission} \& a damaging lie. He was deliberately vague about the company's plans. He may have even considered that Don would eventually learn the whole truth, but only after it was too late to thwart his plan.

\textbf{Denial} -- 

'' -- \cite[pp. 91--]{Simon2010}

%------------------------------------------------------------------------------%

\section{Redefining the Terms of Engagement}

%------------------------------------------------------------------------------%

\section{Epilogue: Undisciplined Aggression in a Permissive Society}

%------------------------------------------------------------------------------%

\section{Endnotes}

%------------------------------------------------------------------------------%

\printbibliography[heading=bibintoc]
	
\end{document}