\documentclass{article}
\usepackage[backend=biber,natbib=true,style=authoryear]{biblatex}
\addbibresource{/home/nqbh/reference/bib.bib}
\usepackage{tocloft}
\renewcommand{\cftsecleader}{\cftdotfill{\cftdotsep}}
\usepackage[colorlinks=true,linkcolor=blue,urlcolor=red,citecolor=magenta]{hyperref}
\usepackage{algorithm,algpseudocode,amsmath,amssymb,amsthm,float,graphicx,mathtools}
\allowdisplaybreaks
\numberwithin{equation}{section}
\newtheorem{assumption}{Assumption}[section]
\newtheorem{conjecture}{Conjecture}[section]
\newtheorem{corollary}{Corollary}[section]
\newtheorem{definition}{Definition}[section]
\newtheorem{example}{Example}[section]
\newtheorem{lemma}{Lemma}[section]
\newtheorem{notation}{Notation}[section]
\newtheorem{principle}{Principle}[section]
\newtheorem{problem}{Problem}[section]
\newtheorem{proposition}{Proposition}[section]
\newtheorem{question}{Question}[section]
\newtheorem{remark}{Remark}[section]
\newtheorem{theorem}{Theorem}[section]
\usepackage[left=0.5in,right=0.5in,top=1.5cm,bottom=1.5cm]{geometry}
\usepackage{fancyhdr}
\pagestyle{fancy}
\fancyhf{}
\lhead{\small Sect.~\thesection}
\rhead{\small\nouppercase{\leftmark}}
\renewcommand{\sectionmark}[1]{\markboth{#1}{}}
\cfoot{\thepage}
\def\labelitemii{$\circ$}

\title{In Sheep's Clothing: Understanding \& Dealing with Manipulative People}
\author{George Simon, Jr.}
\date{\today}

\begin{document}
\maketitle
\tableofcontents
\vspace{5mm}
\begin{quotation}
	``[After reading \textit{In Sheep's Clothing}] I am beginning to reclaim my life, find my self-respect \& confidence.'' -- Marc, Virginia
	
	``After having read several books on several different self-help topics, psychology books, psychiatry books, etc., I MUST recommend you buy this one, 1st. It cuts straight through the bs -- neatly \& cleanly. I have bought copies of this book for friends \& can't recommend it enough.'' -- E. Adams, Online Purchaser
	
	``Don't Be Bossed-Around Ever Again!!! $\ldots$ \textit{In Sheep's Clothing: Understanding \& Dealing with Manipulative People} by George K. Simon, Jr., Ph.D., is a godsend to anyone who has ever questioned their own sanity while in any kind of relationship with a controlling \& manipulative person.'' -- The Aeolian Kid, Online Purchaser
	
	``Dr. Simon teaches the mechanics of popular tactics used by manipulators \& how you can identify \& thwart their attacks so that you control the outcome. This book helped me with a person that I have no choice but to see daily. After the end of every ``friendly'' conversation I felt depressed or insulted but could not figure out how this person was doing it. This book helped me to understand what was really happening. Dr. Simon's guidelines exposed this person \& [allowed me to take] control. Because this person knows [I] can be longer [be] controlled, I now have -- not a perfect relationship -- but one that's better than the alternative.'' -- A reader in Chicago
	
	``This book is like the secret decoder ring for the jumbled mess that is a manipulator's modus operandi. \textit{Do yourself a favor \& get this book now}.'' -- Christy, Missouri
	
	``It's sad that there are people out there that make life so much harder than it should have to be for others. Being able to identify such people in your life (both at home \& at work) is very important \& can be of invaluable help to (i) not go crazy oneself, \& (ii) take corrective action. Dr. Simon's book is written with amazing clarity. \textit{If you read only 1 book this year, read this one}.'' -- JA008, Online Purchaser
	
	``This is 1 of the best books I've ever read \& I would recommend it to anyone. It has redefined how I judge people \& helped me to become a stronger person. I used to be very naive \& unaware of people's ulterior motives, \& I have learned a tremendous amount from reading this book.'' -- S. Brescenti, Online Purchaser
	
	``This book makes it clear that evil is allowed free rein because of our ignorance of its nature. Simon shows us what seemingly mundane interactions that leave us perplexed may really be about. According to him, master manipulators leave us drained \& confused as we try to change them into the good person we want to believe they really are. I would add that the manipulators are just plain evil because evil requires lies, manipulation \& a weakening of the other's will through deception. Simon shows you how to recognize the signs \& what you can do about it. Good people are responsible for informing \& protecting themselves from the manipulators in society. This book is a necessary start.'' -- Kaye, a reader in New York state
	
	``Pithy \& often funny, George Simon takes the bluster \& obfuscation of overbearing, weasely bosses, nasty neighbors, \& obnoxious coworkers \& boils it down to show you the simple psychological strategies being used to imposed on your patience, good will, or even wallet. \textit{I have recommended this book to everyone I know \& bought extra copies for my kids} when they went out into the work world. Highly Recommended!'' -- C. MacCallum, Online Purchaser
\end{quotation}

\section{Acknowledgments}
``I am deeply grateful to my wife, Dr. Sherry Simon for her unfailing love, faith, understanding, patience, \& support. She is responsible for the title of this book \& was a valuable resource in helping me clarify my thoughts during its writing. I wish to thank Dr. Bruce Carruth for his critique of the original manuscript \& suggestions for making it more readable. I am deeply indebted to the work of Dr. Theodore Millon. His comprehensive approach to understanding personality has not only influenced my thinking on the subject but also proved invaluable in my efforts to help people change. I owe a supreme debt to the many individuals willing to share with me their experiences with manipulative people. They taught me much \& enriched my life. This book, in large measure, is a tribute to their courage \& support. I am most appreciative of the validation, support, \& enriching input consistently afforded me by workshop attendees. They have helped me clarity, refine, \& enhance 1 of my principal missions in life. Words cannot express the gratitude I have for the thousands of readers who have kept this book in the active lists of online booksellers \& retail outlets for over 15 years. The many emails, blog posts, \& letters readers have sent helped me make necessary updates \& changes to this Revised Edition. I have attempted to honor the considerable feedback I continue to receive by expanding the discussion of key concepts as well as introducing important new content for this newly revised edition. Finally, I want to thank Roger Armbrust \& Ted Parkhurst of Parkhurst Brothers, Inc., Publishers. Ted encouraged me at the outset \& was there when I needed him; Roger's grace \& presence have only benefited my work and readers.'' -- \cite[p. 7]{Simon2010}

%------------------------------------------------------------------------------%

\section*{Preface}
``Whether it's the supervisor who claims to support you while thwarting every opportunity you have to get ahead, the co-worker who quietly undermines you to gain the boss's favor, the spouse who professes to love \& care about you but seems to control your life, or the child who always seems to know just which buttons to push in order to get their way, manipulative people are like the proverbial wolf in sheep's clothing. On the surface they can appear charming \& genial. But underneath, they can be ever so calculating \& ruthless. Cunning \& stable, they prey on your weaknesses \& use clever tactics to gain advantage over you. They're the kind of people who fight hard for everything they want but do their best to conceal their aggressive intentions. That's why I call them \textit{covert-aggressive personalities}.8

As a clinical psychologist in private practice, I began to focus on the problem of covert aggression over 20 years ago. I did so because the depression, anxiety, \& feelings of insecurity that initially led several of my patients to seek help eventually turned out to be in some way linked to their relationship with a manipulative person. I've counseled not only the victims of covert-aggression, but also manipulators themselves experiencing distress because their usual ways of getting their needs met \& controlling other weren't working anymore. My work has given me an appreciation for how widespread problem of manipulative behavior is \& the unique emotional stress it can bring to a relationship.

The scope of the problem of covert-aggression seems self-evident. Most of us know at least 1 manipulative person. \& hardly a day goes by that we don't read in the newspaper or hear a broadcast about someone who managed to exploit or ``con'' many before fate shed some light on their true character. There's the tele-evangelist who preached love, honesty, \& decency while cheating on his wife \& fleecing his flock, the politician, sworn to ``public service,'' caught lining his pockets, or the spiritual ``guru'' who even managed to convince most of his followers that he was God incarnate while sexually exploiting their children \& subtly terrorizing those who challenged him. The world, it seems full of manipulators.

Although the extreme wolves in sheep's clothing that make headlines grab our attention \& pique our curiosity about what makes such people ``tick,'' most of the covertly aggressive people we are likely to encounter are not these larger-than-life characters. Rather, they are the subtly underhanded, backstabbing, deceptive, \& conniving individuals we may work with, associate with, or possibly even live with. \& they can make life miserable. They cause us grief because we find it so hard to truly understand them \& even harder to deal with them effectively.

When victims of covert-aggression 1st seek help for their emotional distress, they usually have little insight into why they feel so bad. They only know that they feel confused, anxious, or depressed. Gradually, however, they relate how dealing with a certain person in their lives makes them feel crazy. They don't really trust them but can't pinpoint why. They get mad at them but for some reason end up feeling guilty themselves. They confront them about their behavior, only to wind up on the defensive. They get frustrated because they find themselves frequently giving in when they really wanted to stand ground, saying ``yes'' when they mean to say ``no,'' \& becoming depressed because nothing they try seems to make things better. In the end, dealing with this person always leaves them feeling confused, exploited \& abused. After exploring the issues in therapy for a while, they eventually come to realize how much of their unhappiness is the direct result of their constant but fruitless attempts to understand, deal with, or control their manipulator's behavior.

Despite the fact that many of my patients are intelligent, resourceful individuals with a fair understanding of traditional psychological principles, most of the ways they tried to understand \& cope with their manipulator's behavior weren't getting them anywhere, \& some of the things they tried only seemed to make matters worse. Moreover, none of the ways that I initially tried to help made any real difference. Having an eclectic training background, I tried all sorts of different therapies \& strategies, all of which seemed to help the victims feel a little better, but none seemed to empower them enough to really change the nature of their relationship with their manipulator. Even more disconcerting was the fact that none of the approaches I tried was effective at all with the manipulators. Realizing that something must be fundamentally wrong with the traditional approaches to understanding \& dealing with manipulative people, I began to carefully study the problem in the hope of developing a practical, more effective approach.

In this book I would like to introduce you to a new way of understanding the character of manipulative people. I believe the perspective I will offer describes manipulators \& labels their behavior more accurately than many other approaches. I'll explain what covert-aggression is \& why I believe it's at the heart of most interpersonal manipulation. I'll focus some needed attention on dimensions of personality that are too often ignored by traditional perspectives. The framework I will be advancing challenges some of the more common assumptions we make about why people act the way they do \& explains why some of the most widely-held beliefs about human nature tend to set us up for victimization by manipulators.

I have 3 objectives to fulfill in this book. My 1st is to fully acquaint you the nature of disturbed characters as well as the distinctive character of the covertly aggressive personality. I'll discuss the characteristics of aggressive personality types in general \& outline the unique characteristics of the covert-aggressive personality. I'll present several vignettes, based on real cases \& situations, that will help you get the ``flavor'' of this personality type as well as illustrate how manipulative people operate. Being able to recognize a wolf in sheep's clothing \& knowing what to expect from this kind of person is the 1st step in avoiding being victimized by them.

My 2nd objective is to explain precisely how covertly aggressive people managed to deceive, manipulate, \& ``control'' others. Aggressive \& covertly aggressive people use a select group of interpersonal maneuvers or tactics to gain advantage over others. Becoming more familiar with these tactics really helps a person recognize manipulative behavior \textit{at the time it occurs}, \& makes it easier, therefore, to avoid being victimized. I'll also discuss the characteristics many of us possess that can make us unduly vulnerable to the tactics of manipulation. Knowing what aspects of your own character a manipulator is most likely to exploit is another important step in avoiding victimization.

My final objective is to outline the specific steps anyone can take to deal more effectively with aggressive \& covertly aggressive personalities. I'll present some general rules for redefining the rules of engagement with these kinds of individuals \& describe some specific tools of personal empowerment that can help a person break the self-defeating cycle of trying to control their manipulator \& becoming depressed in the process. Using these tools makes it more likely that a 1-time victim will invest their energy where they really have power -- in their own behavior. Knowing how to conduct yourself in a potentially manipulative encounter is crucial to becoming less vulnerable to a manipulator's ploys \& asserting greater control over your own life.

I have attempted to write this book in a manner that is serious \& substantial yet straightforward \& readily understandable. I have written it for the general public as well as the mental health professional, \& I hope both will find it useful. By adhering to many traditional assumptions, labeling schemes, \& intervention strategies, therapists sometimes hold \& inadvertently reinforce some of the same misconceptions that their patients harbor about the character \& behavior of manipulators that inevitably lead to continued victimization. I offer a new perspective in the hope of helping individuals \& therapists alike avoid \textit{enabling} manipulative behavior.'' -- \cite[pp. 8--12]{Simon2010}

%------------------------------------------------------------------------------%

\section*{Author's Note on the Revised Edition}
``Since this book's 1st wide publication in 1996, I have received literally hundreds of calls, letters, \& emails, \& heard countless testimonials \& comments at workshops from individuals whose lives were changed merely by being exposed to \& adopting a new perspective on understanding human behavior. A common theme voiced by readers \& workshop attendees is that once they dispelled old myths \& came to view problem behaviors in a different light, they could see clearly that what their intuition had told them all along was correct, \& thus felt validated. A similar phenomenon has held true for mental health professionals attending the many training seminars I have given. Once they abandoned their old notions about why their clients do the things they do, they were better able to help them \& their significant others. I had already been doing workshops for 10 years before writing \textit{In Sheep's Clothing}. At that time, only a handful of theorists, researchers, \& writers were recognizing the need for a new perspective on understanding \& dealing with disturbed characters (e.g., Stanton Samenow, Samuel Yochelson, Robert Hare). What professionals today call the \textit{cognitive-behavioral} approach was in its infancy. The early research on character disturbance inspired me \& helped me validate my own observations. Today an increasing number of professionals are recognizing the problem of character disturbance \& using cognitive-behavioral methods to diagnose \& treat it.

We live in an age radically different from that in which the classical theories of psychology \& personality were developed. For the most part, truly pathological degrees of neurosis are quite rare, \& problematic levels of character disturbance are increasingly commonplace. It's a pervasive societal problem about which all of us would do well to expand our awareness. During the last 15 years, my experience working with disturbed characters of all types has grown immensely, as has the body of research. So, I have included in this edition an expanded discussion on the problem of character disturbance in general \& what sets the disturbed character apart from your garden-variety neurotic.

I am deeply grateful for the excellent word-of-mouth support responsible for transforming a once small, independent work into a best seller enjoying ever-increasing popularity even after 15 years. I sincerely hoped this revised edition will provide you with all the information \& tools you need to better understand \& deal with the manipulative people in your life.

\textit{George K. Simon, Jr., Ph.D.}, Jan 2010'' -- \cite[p. 14]{Simon2010}

%------------------------------------------------------------------------------%

\begin{center}\Large\sf
	\textbf{Part I: Understanding Manipulative Personalities}
\end{center}

\section*{Introduction: Covert-Aggression: The Heart of Manipulation}

\subsection{A Common Problem}
``Perhaps the following scenarios will sound familiar. A wife tries to sort out her mixed feelings. She's mad at her husband for insisting their daughter make all A's. But she doubts she has the right to be mad. When she suggested that given her appraisal of their daughter's abilities, he might be making unreasonable demands, his comeback, ``Shouldn't \textit{any} good parent want their child to do well \& succeed in life?'' made her feel like the insensitive one. In fact, whenever she confronts him, she somehow ends up feeling like the bad guy herself. When she suggested there might be more to her daughter's recent problems, \& that the family might do well to seek counseling, his retort ``Are you saying I'm psychiatrically disturbed?'' made her feel guilty for bringing up the issue. She often tries to assert her point of view, but always ends up giving-in to his. Sometimes, she thinks the problem is him, believing him to be selfish, demanding, intimidating, \& controlling. But this is a loyal husband, good provider, \& a respected member of the community. By all rights she shouldn't resent him. Yet, she does. So, she constantly wonders if there isn't something wrong with her.

A mother tries desperately to understand her daughter's behavior. No young girl, she thought, would threaten to leave home, say things like ``Everybody hates me'' \& ``I wish I were never born,'' unless she were very insecure, afraid, \& probably depressed. Part of her thinks her daughter is still the same child who used to hold her breath until she turned blue or threw tantrums whenever she didn't get her way. After all, it seems she only says \& does these things when she's about to be disciplined or she's trying to get something she wants. But a part of her is afraid to believe that. ``What if she really believes what she's saying?'' she wonders. ``What if I've really done something to hurt her \& I just don't realize it?'' she worries. She hates to feel ``bullied'' by her daughter's threats \& emotional displays, but she can't take the chance her daughter might really be hurting -- can she? Besides, children just don't act this way unless they really feel insecure or threatened in some way underneath it all -- do they?'' -- \cite[p. 17--18]{Simon2010}

\subsection{The Heart of the Problem}

%------------------------------------------------------------------------------%

\section{Aggressive \& Covert-Aggressive Personalities}

%------------------------------------------------------------------------------%

\section{The Determination to Win}

%------------------------------------------------------------------------------%

\section{The Unbridled Quest for Power}

%------------------------------------------------------------------------------%

\section{The Penchant for Deception \& Seduction}

%------------------------------------------------------------------------------%

\section{Fighting Dirty}

%------------------------------------------------------------------------------%

\section{The Impaired Conscience}

%------------------------------------------------------------------------------%

\section{Abusive, Manipulative Relationships}

%------------------------------------------------------------------------------%

\section{The Manipulative Child}

%------------------------------------------------------------------------------%

\begin{center}\Large\bf
	Part II: Dealing Effectively with Manipulative People
\end{center}

\section{Recognizing the Tactics of Manipulation \& Control}

%------------------------------------------------------------------------------%

\section{Redefining the Terms of Engagement}

%------------------------------------------------------------------------------%

\section{Epilogue: Undisciplined Aggression in a Permissive Society}

%------------------------------------------------------------------------------%

\section{Endnotes}

%------------------------------------------------------------------------------%

\printbibliography[heading=bibintoc]
	
\end{document}