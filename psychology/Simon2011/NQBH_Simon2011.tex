\documentclass{article}
\usepackage[backend=biber,natbib=true,style=alphabetic,maxbibnames=50]{biblatex}
\addbibresource{/home/nqbh/reference/bib.bib}
\usepackage{tocloft}
\renewcommand{\cftsecleader}{\cftdotfill{\cftdotsep}}
\usepackage[colorlinks=true,linkcolor=blue,urlcolor=red,citecolor=magenta]{hyperref}
\usepackage{amsmath,amssymb,amsthm,caption,float,graphicx,mathtools,subcaption}
\usepackage{enumitem}
\setlist{leftmargin=0mm}
\allowdisplaybreaks
\numberwithin{equation}{section}
\newtheorem{assumption}{Assumption}[section]
\newtheorem{conjecture}{Conjecture}[section]
\newtheorem{corollary}{Corollary}[section]
\newtheorem{definition}{Definition}[section]
\newtheorem{example}{Example}[section]
\newtheorem{lemma}{Lemma}[section]
\newtheorem{notation}{Notation}[section]
\newtheorem{principle}{Principle}[section]
\newtheorem{problem}{Problem}[section]
\newtheorem{proposition}{Proposition}[section]
\newtheorem{question}{Question}[section]
\newtheorem{remark}{Remark}[section]
\newtheorem{theorem}{Theorem}[section]
\usepackage[left=1cm,right=1cm,top=5mm,bottom=5mm,footskip=4mm]{geometry}
\def\labelitemii{$\circ$}

\title{Character Disturbance: The Phenomenon of Our Age}
\author{George K. Simon Jr.}
\date{\today}

\begin{document}
\maketitle
\tableofcontents
\begin{quotation}
	``\textit{For Sherry}, whose heart is truer than any I know, \& who inspires me daily to be a better person. \& for all those silent but committed souls of noble character, upon whom the very survival of freedom depends.'' -- \cite{Simon2011}
\end{quotation}

%------------------------------------------------------------------------------%

\section*{Preface}
``Imagine you recently read a newspaper article about a young girl who suddenly \& inexplicably lost her eyesight. Her frantic parents took her from doctor to doctor, specialist to specialist, \& clinic to clinic, yet no one could find the reason for her blindness. In desperation, 1 day they decided to take her to a psychologist. After months of traditional psychoanalysis, the therapist revealed that the child's blindness resulted from severe emotional trauma. It seems that several months before, while riding on the school bus, this young lady just happened to glance at a boy seated with some friends across the aisle. She thought to herself: ``This guy is really cute.'' Before long, she also began thinking things like: ``I wonder what it would be like to kiss him.'' But almost immediately after having these thoughts, she started to feel badly. She began fretting about what kind of horrible person she must be to entertain such ``impure'' thoughts. She worried that they could only lead to other impure urges \& temptations, \& even perhaps some impure action on her part. Eventually, she became consumed with guilt \& shame. She remembered times in the past when she had looked at boys, \& how hard it was to resist impure thoughts \& urges. Surely worse would follow, she feared, if she didn't keep herself in check. Shortly after this incident, she lost her vision.

When she 1st saw the psychologist, this troubled little girl didn't even remember the bus incident. She certainly didn't remember what she was thinking or feeling at the time. She'd even forgotten how deeply the incident unnerved her, \& how she dealt with her anxiety over the situation. The psychologist helped her see that she had ``repeated'' her memoires as a way of easing the intensity of her emotional pain. Her lengthy analysis eventually helped her not only recover her memory of that fateful day's events, but also enabled her to reconnect with her conflicted emotions. She came to realize this: She was so deeply distressed by what she thought were her unforgivable, impure desires that she actually believed it was better not to see at all than risk having such thoughts about boys again.

Once she had confessed her sins to the doctor \& he did not condemn her, the young girl slowly began to feel better. She took heart in the notion that he appeared to accept her just as she was. She slowly began to feel that she wasn't such a horrible person after all merely for the kinds of thoughts she sometimes had about boys. In time, she came to believe that the level of her fear, guilt, \& shame was excessive \& unwarranted, given the nature of the situation. No longer believing that simply having thoughts about kissing boys was as evil as she once did, she allowed herself to see again.

Now, I would pose to you, the reader, the same question I ask of professionals \& non-professionals alike at every 1 of the hundreds of workshops I've given over the past 25 years. How many scenarios similar to the one I described have you read or heard about in the last year? How about the past 5 years? How about the past, 10--20--30 years? It should come as no surprise that the answer I get is always the same: \textbf{\textit{zero}}. What should shock you, however (\& always gets the attention of my audience), is this: \textit{Almost all of the principles of classical-psychology paradigms stemmed from various theorists' attempts to explain this \& similar phenomena} (sometimes referred to by adherents to these theories as ``hysterical'' blindness).

You see, in Sigmund Freud's day, \textit{some} individuals actually suffered from such strange maladies. It's important to note, however, that these extreme psychological illnesses were \textit{never} widespread phenomena. But in the intensely socially repressive Victorian era, cases appeared of persons experiencing extraordinary levels of guilt or shame merely for being tempted to act on a primal human instinct, \& displaying pathological symptoms as a result. If there were a motto or saying that might best describe the ``zeitgeist'' (i.e. social or cultural milieu) of that time it would be: ``Don't even \textit{think} about it!'' So, some individuals were quite unnecessarily consumed with excessive guilt \& shame about their most basic human urges. Freud treated some of these individuals, most of whom were women, typically subjected to more oppression than men. He eventually coined the term ``neurosis'' to describe the internal struggle he believed went on between a person's instinctual urges (i.e. \textit{id}) \& their conscience (i.e. \textit{superego}), the excessive \textit{anxiety} or nervous tension that often accompanies this internal war, \& the unusual psychological \textit{symptoms} a person can develop when attempting to mitigate (i.e. through the use of marginally effective \textit{defense mechanisms}) the intense emotional pain associated with these inner conflicts. He then developed a set of theories \& constructs that appeared to adequately explain how his patients developed their bizarre maladies. In the process, however, \textit{he also came to believe that he had discovered universal, fundamental principles that explained personality formation \& the entire spectrum of human psychological functioning}.

Many of the constructs \& terms 1st articulated by Freud, \& several other ``psychodynamic'' theorists who followed him, found their way into common parlance over the years. For a significant period of time, these tenets also gained widespread acceptance by mental health professionals. \& many of the principles of what I refer to in this book as ``traditional psychology'' still enjoy a fair degree of acceptance, not only among professionals but also among lay persons. This is true despite the fact that, in recent years, several of the most important assumptions \& doctrines have been proven completely false or significantly flawed.

Not only are some of traditional psychology's central tenets of questionable validity, but also times have changed dramatically since the days of Sigmund Freud. If a motto or saying best befits our modern era's zeitgeist, it would be much like the once popular commercial that urged: ``Just \textit{do} it!'' As a result, highly pathological levels of neurosis (as exemplified by the girl who went blind simply because she ``lusted'' after a boy she found attractive) have all but completely disappeared, especially in industrialized free societies. Instead of being dominated by individuals overly riddled with unfounded guilt \& shame (i.e., ``hung-up'' as children of the `60s used to say), modern western culture has produced increasing numbers of individuals who aren't ``hung-up'' enough about the things they let themselves do.

So, today we are facing a near epidemic of what some theorists refer to as \textit{character disturbance}. Neurosis is still with us, but for the most part at \textit{functional} as opposed to pathological levels. I.e., most people today experience just enough apprehension \& internal turmoil when it comes to simply acting on their primal urges, that they don't in fact ``just do it.'' Instead, they experience sufficient anticipatory guilt or shame to restrain their impulses \& conform their conduct to more socially acceptable standards. So, one can easily say that their neurosis is \textit{functional}. It's largely what makes society work.

Freud used to say that civilization is the cause of neurosis, \& given the climate of his time, it's easy to see how he came to that conclusion. In a sense, he had a point; but his observation was more than a bit narrow-sighted. True, the prohibitions any society imposes on the unrestrained expression of primal urges can give rise to a fair degree of anxiety in some of us. \&, a brutally oppressive culture can breed excessive degrees of neurosis. But in large measure, \textit{it's most people's \textbf{capacity} to become unnerved when contemplating acting like an animal (i.e. their capacity to be ``neurotic'' to some degree) that makes civilization itself possible}. So, it's precisely because significant numbers of people still get a little ``hung up'' when they contemplate punching someone's lights out, or have some apprehension \& qualms of conscience when they blink about taking something that doesn't belong to them, that there's any degree of civilization left at all.

Today many types of professionals span a wide variety of disciplines that deal with mental health issues \& personal problems of 1 variety or another. Most of these professionals have \textit{never} encountered -- let alone treated -- a case of ``hysterical blindness,'' pseudo-paralysis, or any similar phenomenon. In fact, it's becoming increasingly rare for professionals to encounter a case of neurosis at a highly pathological level of intensity. Therapists rarely deal with problems that stem from a conscience so overactive or oppressive that it causes a person to develop bizarre or severely debilitating psychosomatic or other pathological symptoms. Instead, mental health clinicians in all disciplines increasingly find themselves intervening with individuals whose problems are related to their dysfunctional \textit{attitudes} \& \textit{thinking patterns}, their shallow, self-centered relationships, their moral immaturity \& social irresponsibility, \& their habitual, \textit{dysfunctional behavior patterns}. All of these stem from an \textit{underdeveloped conscience} \& reflect significant \textbf{\textit{deficiencies or disturbances of character}}.

This wouldn't be so much of a problem if it weren't for the fact that many mental health professionals not only are trained primarily in classical theories of human behavior, but also cling to beliefs about human nature \& the underpinnings of psycho-social dysfunction that originally emanated from these theories. For this reason, they often attempt to use the tenets \& the principles that flow from more traditional paradigms to guide them in their efforts to solve today's very different kinds of psychological problems. In short, they attempt to understand \& treat character disturbance with approaches \& methods originally designed to treat extreme levels of neurosis.

Retaining outdated notions about why people do the things they do can put anyone -- lay persons \& mental health professionals alike -- at a great disadvantage when it comes to understanding \& dealing with the disturbed character. Even some of our more modern frameworks for understanding human behavior are inadequate to address the phenomenon. The problem is compounded by this fact: It's not ``politically correct'' to consider people's emotional \& behavioral problems as stemming from or reflecting deficiencies in their character. So, sometimes problems actually rooted in character pathology might be framed as almost anything else (e.g., an ``addiction,'' ADHD, Bipolar Disorder, a ``chemical imbalance,'' etc.). Their symptoms can be somewhat managed with medication or other forms of treatment. Then, if at least some degree of change is observed, that mere fact validates the perspective that disease of some sort -- as opposed to character -- caused the problem. Sometimes, professionals actually do recognize personality or character disturbances, but regard them as unchangeable or untreatable. So, they target issues other than character concerns in therapy. At other times a professional's over-immersion in traditional paradigms will prompt them to view everyone -- even the most severely character-disturbed individual -- as neurotic, at least to some degree, \& then attempt to treat their supported neurosis. But the reality is this: \textbf{\textit{character disturbance is 1 of the most pressing psychological realities of our age; it's becoming increasingly prevalent; \& it's an entirely different phenomenon from neurosis, requiring a different perspective to adequately understand \& treat}.}

Understanding character disturbance requires viewing human beings, \& the reasons they do some of the things they do, in a very different light. Further, character disturbance simply can't be dealt with effectively using traditional approaches. The tools \& techniques that have proven effective in treating character disturbance are radically different from those originally developed to treat neurosis.

Over 14 years ago, I wrote \textit{In Sheep's Clothing: Understanding \& Dealing with Manipulative People} to help both therapists \& average folks understand what certain people are really like, \& how they manage to manipulate \& control others. The book has grown increasingly popular every year since its 1st printing, has been translated into several foreign languages, \& consistently draws highly laudatory reviews from professionals as well as lay persons. The comments are generally of a similar nature. Readers report that once they cast off old notions about the nature \& behavior of the manipulative persons in their lives, ``got it'' w.r.t. what really makes such individuals ``tick,'' \& adopted not only a new perspective for understanding such persons but also a new set of rules \& principles to guide their relationships with them, their circumstances dramatically improved. This kind of feedback has been more than edifying for me, \& inspired me to write this book. \textit{In Sheep's Clothing} was about 1 type of disturbed character. This book is about all of the various disturbed character types you are likely to encounter in your life. \textit{In Sheep's Clothing} briefly delved into the general topic of character disturbance. This book takes a much deeper look at this significant \& disturbing phenomenon, its growing prevalence, \& some of the socio-cultural features of the current era responsible for promoting it.

My primary purpose in writing this book is to help you: (1) cast off faulty assumptions that can place you at a distinct disadvantage, \& (2) understand the real reasons disturbed characters behave the way they do. It's important to recognize that disturbed characters differ dramatically from neurotics on almost every imaginable dimension of interpersonal functioning. They don't hold the same values, believe the same things, harbor the same attitudes, think the same way, or behave in the same manner as neurotics. This book will help you understand why you might have had such a difficult time dealing with the disturbed characters in your life, what makes these individuals so different, \& why aid you might have sought, hoping to deal with problems involving them, proved inadequate.

As with \textit{In Sheep's Clothing}, I've written this book in a concise manner, easy to read \& understand. It should be equally helpful to the layperson as well as the professional who is primarily versed in or aligned with traditional perspectives. I have deliberately not included mounds of difficult-to-understand scientific research data \& have attempted to translate sophisticated \& highly technical material into simpler, yet reasonably accurate language. My intention was not to craft an authoritative \& comprehensive textbook-style treatise on human nature, personality, or psychopathology. Rather, I present principles \& perspectives derived from years of experience working with disturbed characters \& their victims, illustrate those principles with real-life examples, \& support the most important contentions \& perspectives with relevant research findings when appropriate or necessary.

My secondary purpose in writing this book is to expose character disturbance for the significant social problem that it is, \& address some of the socio-cultural influences responsible for it. I am not the 1st to sound the alarm on the issue, \& I hope I will not be the last. In their book on character lengths \& virtues, the eminent psychology pioneers \& researchers Christopher Peterson \& Martin E.P. Seligman address the issue directly:

``After a detour through the hedonism of the 1960's, the narcissism of the 1970's, the materialism of the 1980's \& the apathy of the 1990's, most everyone today seems to believe that character is important after all \& that the United States is facing a character crisis on many fronts, from the playground to the classroom to the sports arena to the Hollywood screen to business corporations to politics.''\footnote{Peterson, C., \& Seligman, M., \textit{Character Strengths \& Values: A Handbook \& Classification}, (American Psychological Association \& Oxford University Press, 2004), p. 5.}

But no problem can be resolved until it's fully acknowledged \& adequately defined. That's the 1st step. The 2nd step is to take a serious look at the characteristics of our social milieu that contribute to the problem. This book will do both.

I have been working with disturbed characters for over 25 years now. Yet I know that much of what I assert in this book is likely to be controversial. When I 1st started doing workshops on manipulators \& other problem characters, especially with professionals, several attendees were uncomfortable with my perspective. Some even walked out! That's because a lot of what I had to say challenged some longstanding notions \& deeply-held beliefs about why people experience psychological problems in the 1st place, \& how professionals must assist them in achieving mental \& emotional health. But time \& recent research has validated much about the perspective I introduced those many years ago. These days, it's very common for me to see many heads in the audience frequently nod in approval as I articulate the principles I've adopted to understand \& deal with disturbed characters. It's also common for individuals who have already been to 1 of my workshops to come back again several times for ``refreshers'' on the tenets I will outline in this book. 1 of the more frequent comments I get from professionals involves how much more satisfying working with disturbed characters has become after they adopted some of these principles.

But old notions \& biases aren't easily shed, so I still expect some controversy \& debate. The same was initially true for \textit{In Sheep's Clothing}. But I am confident that -- if you approach the concepts I will introduce with an open mind, allow my disturbed-character descriptions to \textit{validate your experience}, accept the reality that our very different times have spawned problems the major helping professionals have only recently begun to face by developing sound paradigms, \& at least try out some of the principles \& tools I will outline -- you'll be better prepared to understand \& deal with the individuals in your life who display significant character pathology.

There are 2 principal reasons I felt it imperative to write this book at this time. 1st, I have come to a deep realization about how serious the problem of character disturbance has become, \& the degree to which it undermines the very foundations of our free society. For the past 15 years, I have been working with some of the most severely disturbed characters, most of whom have been incarcerated at some point in their lives, \& several of whom have been incarcerated multiple times. You should know that the United States has a higher percentage of its population incarcerated than any industrialized country, \& that percentage is increasing. You should also know that contrary to popular belief, it generally takes some very serious \&\texttt{/}or chronic misbehavior to end up incarcerated. Yet, prisons \& jails are overcrowded, \& the judicial system is overburdened. Although very few may be incarcerated who don't necessarily need to be, many others lead lives of wanton major social norm violations for which they rightfully could have been sanctioned many times, but weren't because of the scarcity of space \& resources. I've also worked with many individuals who, although they engaged in no major societal transgressions \& do not have criminal records, nonetheless harbor significant deficiencies of character. I've increasingly counseled individuals who didn't begin functioning in a truly responsible manner until they were in their late 40s, 50s, or even 60s. (I recently attended a discussion group in which a group member discussed his problematic relationship with a brother who, even at age 67, continues to exercise little control over his impulses \& engages in deeply irresponsible behavior!)

The greatness of American society has deteriorated dramatically in recent years because of our culturally-spawned character crisis. There is a saying often attributed to Alexis De Tocqueville: ``America is great because she is good; \& if America eve ceases to be good, she will cease to be great.'' De Tocqueville never actually said this, yet the saying contains a powerful truth. The overall character of a country can be no better than the collective character of its constituents. I grew up in an era of unprecedented American greatness, in significant measure spawned by the distinctive character of what Tom Brokaw rightly called ``the greatest generation.'' I have now lived long enough to witness what I regard as the tarnishing of the best American ideals; \& although I am no longer shocked by it, I am nonetheless outraged. I'm mostly outraged because I still see our unique brands of free expression \& free enterprise as the last, best hope for improving the human condition. \textbf{\textit{freedom \& character are inextricably interdependent}}. John Adams adamantly asserted that ``no government'' deliberately designed to be limited in power \& scope would be ``capable of contending with human passions unbridled by morality \& religion.'' He further noted that the American ``constitution was made \textit{only} for a religious \& moral people,'' \& that ``it is \textit{wholly inadequate} for the government'' of people ``of any other'' character.\footnote{Adams, J., Address to the Military, Oct 1798.}

In the absence of sound individual conscience, society is naturally tempted to legislate its way out of the problems that ensue, imposing rules \& restrictions that inevitably increase the role, size \& scope of government, \& limit individual freedoms. While it may seem like a necessary evil, such attempts to legislate social responsibility \textit{never} really fix the problem; they only set the stage for both economic \& cultural deterioration. So, it bothers me greatly that the populace's complacency \& denial, with a fair amount of aiding \& abetting by some mental health professionals, in recent times have all allowed this character crisis to surreptitiously \& steadily spin out of control, to fray our culture's once distinctly noble fabric, to erode the integrity of our national character, \& to hasten our country's fall from greatness.

Sometimes I wonder just how many fast-talking politicians, greedy executives, \& crafty religious ``leaders'' -- finally caught in their lies \& indiscretions, failed relationships, or major corporate scandals -- it will take for us to realize how much character \& integrity really matter. For a long time, it seems like we had resigned ourselves to the notion that people simply can't be any better or do any better than we've come to expect. But the 2nd major reason I felt compelled to write this book now is this: I know from years of experience that the problem of character disturbance is not at all hopeless. There are some straightforward things we can do to address the cultural contributors to the character crisis. There are interventions we can initiate at the individual level to encourage a person to develop the character they need to function responsibly in society, \& to form healthy, happy, intimate relationships. From a therapist's perspective, the interventions necessary are not of the same sort that one would use to deal with neurosis. Nor are they busy, given the challenging nature of the problem. But \textit{there is hope} for dealing effectively with the disturbed character, especially once you know \& accept the approach that's necessary to deal with the problem.

When I was developing \textit{In Sheep's Clothing}, the cognitive-behavioral therapy (CBT) revolution was in its infancy. Today, it has become more widely accepted. Nonetheless, many individuals who outwardly assert that they understand \& accept the tenets of cognitive-behavioral therapy still have a hard divorcing themselves from outdated \& ineffective paradigms. Furthermore, although they might readily focus on a person's distorted cognitions, some clinicians have an aversion to aligning themselves with the most important aspect of the cognitive-behavioral model: focusing on \& modifying \textit{behavior}. As a result, they often don't fully \& faithfully implement the principles of CBT (doing largely C instead of C{\bf B}T). This is a genuine shame because CBT has demonstrated its superiority to other forms of intervention when it comes to treating persons of disturbed character. The principles advanced in this book are very much in harmony with the principles of CBT.

We are all in this vast human experiment together. Like it or not, ours is an extremely interconnected \& interdependent world. My personal mission for the last several years has been to call attention to a significant social problem, \& to inspire people to address \& overcome it. This book is the culmination of my most recent efforts toward that end.

At a fundamental level, we are all savages seeking the survival of the fittest. But we also have within us the power \& capacity to elevate ourselves \& inspire others to a much higher plane of functioning. Doing so, however, requires the development of \textit{character}.'' -- \cite[pp. 8--19]{Simon2011}

%------------------------------------------------------------------------------%

\section*{Introduction}

\subsection*{Personality \& Character}
``This is a book about character, \& the disturbances of character it seeks to examine differ in significant ways from the various personality disturbances we'll discuss. Therefore, it's important that we clearly distinguish \& define these 2 terms. It's common to hear \textit{personality} \& \textit{character} used in a manner suggesting that they are virtually synonymous. Even professionals frequently employ the terms interchangeably. Although the 2 words are related, I think it not only important but also helpful to draw a firm distinction between these 2 concepts. A close look at the terms' origins \& the evolutions of their meanings will help clarify the working definitions we'll employ throughout this book.'' -- \cite[p. 20]{Simon2011}

\subsection*{Traditional Conceptualizations of Personality}
``The word ``personality'' derives from the Latin word \textit{persona}, which means ``mask.'' In the ancient theater, males played all the roles, including those of female characters. Also, the art of dramatizing situations \& conveying emotion was not as evolved as it is today. So actors used masks of various types not only to denote gender, but also to depict \& emphasize various emotional states. Classical psychology theories borrowed the term ``persona'' because they generally conceptualized \textit{personality} as the social ``mask'' a person wears to conceal the more authentic or ``true'' self.

Classical theories of personality also conceptualize all individuals as fundamentally struggling with \textit{fears} of various kinds, especially fears of social rejection, condemnation, or abandonment. These theories regard basic human needs, desires \& emotions as universal, \& potential environmental ``threats'' to them as ever-present. So, the theories postulate that (1) to a greater or lesser extent, everyone struggles with fears that those basic wants \& needs will be thwarted in some way; \& (2) people fear that they might encounter disappointment \& disapproval if they ever reveal their true wishes \& desires. Classical theories also propose that people engage in certain stylized but unconscious ways of ``defending'' themselves against a potentially hostile \& rejecting world, \& in the process end up estranged from their more authentic selves. The classical model, then, sees personality as an \textit{unconsciously constructed fa\c{c}ade} -- a mask that hides the genuine person behind a veritable wall of defenses. This conceptualization still has some value when it comes to understanding personalities who are best thought of as \textit{neurotic} (more about this later).

Traditional theories of personality also postulate that a person wouldn't need to hide his or her authentic self behind the fa\c{c}ade of personality if our environments weren't so hostile, cold, or rejecting. In other words, everyone would be ``authentic,'' healthy, \& without defensive armor if the world didn't force so much emotional trauma into our lives. From traditional perspectives, it's the slings \& arrows of life that have the greatest influence in shaping personality. Traditional theories view everyone as \textit{essentially the same} underneath their structure of defenses. Furthermore, the types of defenses a person most likely will employ are seen as logical growth -- responses to the kind \& severity of traumatic experience to which they've been subjected, especially during their formative years. Within traditional frameworks, the various personality types are defined primarily by the cluster of defenses they have learned to employ.'' -- \cite[pp. 20--21]{Simon2011}

\subsection*{A More Contemporary Conceptualization of Personality}
``More recent conceptualizations, most eloquently described by the eminent theorist \& researcher Theodore Millon, generally define personality as an individual's distinctive \& relatively engrained ``style'' of interrelating.\footnote{Millon, T., \textit{Disorders of Personality}, (Wiley Interscience, 1981), p. 4.} Such conceptualizations of personality often incorporate a multidimensional perspective: They recognize a person's constitutional predispositions (i.e. hereditary, hormonal, temperamental, \& other biological factors), environmental factors (e.g., early learning, significant life events, peer \& role-model influences, degree of nurturance available, etc.), \& a \textit{dynamic interaction} between what the world teaches a person \& how they are predisposed to respond to events. Multidimensional personality theorists believe all the aforementioned factors contribute to a person's distinctive manner of perceiving, relating to, \& interacting with others \& the world at large.

The multidimensional perspective of personality better explains how some individuals can develop very dysfunctional personalities, despite being reared in relatively benign, supportive environments; whereas others can develop remarkable \& admirable character despite experiencing the most egregious circumstances. If you recall, traditional theories assume this: Underneath, we're all the same, \& we'd be perfectly healthy \& authentic individuals if it weren't for the fact that we grew up in a world full of pain, rejection, \& emotional trauma. Within the multidimensional framework, both the environment \& an individual's innate predispositions play roles. The weight various factors might carry in shaping personality varies considerably from person to person. The choices a person makes about how to best cope with life's challenges also play a role. In short, the multidimensional model allows for the increasing scientific evidence that we are not all the same. Each of us has a unique collection of traits. Some developed as a result of learning, some we were simply endowed with by nature, \& some developed as a result of the dynamic interaction between our innate predispositions, the environment in which we were raised, \& the choices we've made. \& the model also allows for the fact that, once our ``preferred'' ways of thinking \& behaving congeal \& become ingrained, they fairly much define who we are as individuals. Personality, therefore, is seen -- not so much as a false-face or a pretense -- but rather as a stable set of traits, preferred thinking \& behavior patterns that define our unique \textit{style} of interaction over a wide variety of situations, \& for most of our lifetime.

So, the multidimensional conceptualizations of personality are more comprehensive than the classical definitions. They also appear more accurate \& useful when we try to understand the thinking \& behavior patterns of individuals best described as \textit{character-disturbed} rather than \textit{neurotic}. This will become increasingly evident to you when we take a more in-depth look at the nature of character disturbance \& its vast differences from neurosis.'' -- \cite[pp. 21--23]{Simon2011}

\subsection*{Character}
``The word ``character'' derives from both Old French \& Greek words meaning to engrave or furrow a \textit{distinctive mark}. The word has been used to denote the most distinguishing traits of overall personality that uniquely define or ``mark'' an individual as a social being. Most especially, the term commonly reflects an individual's positive personality aspects -- those \textit{socially desirable qualities} \& virtues such as self-control, ethics, loyalty, \& fortitude.

As mentioned before, it's not uncommon for professionals as well as lay persons to use the terms ``character'' \& ``personality'' interchangeably. It's also not uncommon for folks to speak of character as if it were synonymous with \textit{strength} of character. Many also erroneously equate the terms \textit{personality disorder} \& \textit{character disorder} (I'll have more to say about this later). But, again, this is a book about character \& the social \& psychological consequences of significant disturbances or deficiencies of character. So it's of paramount importance that we highlight the key differences between someone who possesses all of the various traits \& quirks of a certain \textit{personality} \& someone whose dominant personality features reflect significant deficiencies or defects in their character. Therefore, in this book, the term ``character'' will refer to those \textit{distinct aspects of personality} that reflect the presence \& strength of a person's virtues, personal ethics, social conscientiousness, \& depth of commitment to respect-worthy \& meritorious social conduct.'' -- \cite[p. 23]{Simon2011}

%------------------------------------------------------------------------------%

\section{Neurosis \& Character Disturbance}
``Before we can even begin a meaningful discussion about the differences between neurosis \& character disturbance, it's important to understand the strengths \& limitations of any scientific or philosophical metaphor. We simply can't have a discourse about the important matters of life without theories, definitions, \& constructs. But in the end, absolute truth is illusive. Some might say it's unknowable. We have to settle for ``metaphors'' that, indicated by evidence, approximate the incomprehensible truth as closely as possible.

Some metaphors appear more ideally suited to understanding \& dealing with various aspects of reality. E.g., Sir Isaac Newton tried to explain the phenomenon we know as gravity, as well as planets' motion \& orbits, with the ``law'' that every material object attracts every other in proportion to their mass, \& in inverse proportion to their distance from one another. In a competing ``metaphor'' for explaining the same phenomenon, Einstein proposed that what appears as 1 body attracting another is really the distortion or warping of ``space-time'' (a concept so complex it's far beyond the scope of this book); \& that all clusters of matter (including particles of light) simply follow the resulting ``curvature'' of distorted space-time. A landmark test of Einstein's theory during a solar eclipse demonstrated convincingly that his metaphor provides a more accurate description of gravitation. Newton's metaphor, however, still works very well when you're trying to plot a course for a rocket to the moon, or determine an orbital path around the earth. So, there are still situations in which the Newtonian metaphor is useful, despite its limitations. \& neither Einstein's metaphor nor Newton's gives us the whole truth about gravity. We still don't really know exactly what it is or how it works.

Social science metaphors are no less subject to limitation or appropriate application than any other scientific metaphor. \& any metaphor can become problematic when we over-generalize it, or stretch it to fit or explain phenomena it was never really meant to explain. Most branches of science have come to grips with this. All encompassing explanations of reality such as string theory, the theory of everything, or the unified field theory are still illusive. So, for the most part, scientists apply various theories in the areas they appear to suitably explain. But in the social sciences, we've been slow to recognize \& deal with limits to some of our traditional explanatory models.

Traditional psychodynamic metaphors still apply \& have value when you're trying to understand or deal with the phenomenon we call ``neurosis.'' But the evidence is mounting that such metaphors are inadequate at best, \& potentially counterproductive or even harmful at worst, when we're trying to understand or deal with character disturbance. The constructs of neurosis \& character disturbance are themselves only metaphors. Each attempts to describe a psychological reality we cannot adequately or completely define; but we have to attempt to give it some structure if we're to understand it at all. The terms themselves are not reality; they're metaphorical ways we attempt to describe some aspects of reality. \& each way has its strengths \& limitations, as well as areas of applicability \& areas of poor fit.

I find it useful to conceptualize neurosis \& character disturbance as constructs lying at different ends of a continuum. A representation of this concept is provided in the following diagram: Neurosis $\longleftrightarrow$ Self-Actualization Altruism, Character Disturbance.

At 1st glance, the representation depicted above might appear a bit confusing. It's important to realize, however, that very few individuals have ever walked our planet who could rightfully be considered so socially \& morally evolved (e.g., Jesus, the Buddha) that they're considered truly altruistic. By altruistic, I mean that their commitment to selfless, humanistic behavior is not rooted in pangs of guilt for doing otherwise, or in an unacknowledged selfish desire for admiration or immortal reputation. Altruists, by definition, are individuals who \textit{freely} \& completely commit themselves to advancing the greater good. They are not neurotic because they have no driving desire to avoid guilt or shame for doing otherwise. Also, they're not out for personal glory or to be revered by society. By definition, altruists simply, freely, \& nobly choose to subordinate their own selfish desires for the good of all.

Fully \textit{self-actualized} (altruistic) individuals are extremely rare indeed; some assert they don't really exist. That's because most people restrain their baser instincts out of a sense of guilt \&\texttt{/}or shame for doing otherwise. It's theoretically possible to be without neurosis \& not be character disordered, but the absence of neurosis usually results in some degree of character disturbance. That's because without potential pangs of guilt or shame influencing our decisions, most of us would act in socially irresponsible ways. So, character disturbance \& neurosis are at opposite poles of a continuum, with \textit{each individual personality falling somewhere along the continuum}. \&, as you might expect, personalities rightfully classified as more neurotic will differ significantly from those rightfully classified as more character-disturbed on several dimensions.

The framework advanced here ultimately views character disturbance as a crucial \textit{dimension} of personality. It varies in degree \& results from a number of shaping influences that affect an individual's personality development. E.g, there are people who started out in life with some degree of neurosis; but extraordinary pain \& trauma ``hardened'' their hearts, \textit{solidifying} their preferred styles of coping to such a degree that their neurosis eventually became relegated to their souls' deepest recesses. In such individuals, any neurosis they harbor can usually only re-emerge when they are truly ``broken down'' by sufficiently humiliating or spirit-fracturing life events, abject failures of their coping style, or sufficiently intensive \& confrontational therapeutic techniques. But some individuals were never very neurotic to start with. Either they didn't experience the usual neurosis-fostering \& civilizing experiences like most of us, or their innate predispositions prevented those typical civilizing influences from impacting them as such influences impact most of us.'' -- \cite[pp. 24--27]{Simon2011}

\subsection{Disturbance vs. Disorder}
``Throughout this book you will see the following terms: ``personality disturbance,'' ``personality disorder,'' ``character disturbance,'' \& ``character disorder.'' It's important to recognize that not all problematic aspects of personality or character rise to the level of a true disorder. For a disturbance of personality \&\texttt{/}or character to be considered a disorder, it must be of such intensity, inflexibility, \& intractability that it impairs adaptive functioning in a wide variety of situations. I've included a more in-depth discussion on this in the next chapter.'' -- \cite[p. 27]{Simon2011}

\subsection{Key Neurotic vs. Disturbed Character Differences}
``Character-disturbed personalities differ from neurotic personalities on almost any dimension of interpersonal functioning imaginable. The differences greatly affect how they function in interpersonal relationships, as well as how they respond to therapeutic interventions. It's virtually impossible to list all of the ways disturbed characters \& neurotics differ, but I think it's helpful to examine some of the more significant ones (summarized in the chart on p. 60):

\subsubsection{Anxiety}
Disturbed characters are significantly different from neurotics w.r.t. to their levels \& quality of anxiety. Anxiety is the primal emotion (i.e. fear response) we experience when we feel threatened in some way. It has always been thought to play the most important role in both creating \& maintaining neurosis. When our fear is rooted in a specific, identifiable circumstances -- such as being in a room filled with a lot of people, having to take a test, or coming face to face with a snake -- we call it a \textit{phobia}. When our apprehension does not appear connected to a specific thing or circumstance, is unidentifiable, unknown, or unconscious, we call it \textit{anxiety}.
	
Psychodynamic theorists of all persuasions give pre-eminence to anxiety's role in all forms of neurosis. Individuals who are overly anxious, excessively apprehensive, inordinately fretful, or too easily unnerved can suffer a host of maladies either directly caused or exacerbated by their anxiety. Further, therapists have traditionally thought neurotic ``symptoms'' -- whether stress-related ulcers, tension headaches, avoidance of crowds or open places (i.e. agoraphobia), obsessive worry, desperate actions to prevent abandonment, etc. -- to be rooted in anxiety.

It's also noteworthy that, whether a fear is conscious \& specific (i.e. a phobia) or unconscious or unidentified, research indicates that there is a common essential factor in getting rid of it: exposure. When phobic clients undergo treatment with behavioral or cognitive-behavioral methods, they are encouraged to systematically come into increasing contact with the very situations they fear, thus gaining an increased sense of personal empowerment, \& reducing their apprehension levels. In traditional psychotherapy, clients gradually come into conscious contact with, \& therefore ``face,'' their previously unconscious fears in the safe, supportive, accepting atmosphere promoted by the therapist; \& their fears are eventually reduced.

Anxiety is minimally present or plays a negligible role in the disturbed character's problems. In some cases, it's absent altogether. Character-disordered individuals are notoriously nonchalant about the things that upset most other people. Some, especially the aggressive personalities (more about them later) appear to \textit{lack adaptive levels} of fearfulness.\footnote{Millon, T., \textit{Disorders of Personality}, (Wiley Interscience, 1981), p. 198.} They don't get apprehensive enough about their circumstances or their conduct. They're not unnerved enough at the prospect of conflict, \& they readily leap into risky situations when others would hesitate. For the most part, disordered characters don't do dysfunctional things because some past trauma has them too ``hung-up'' to do otherwise. Instead, they do them because, unlike neurotics, they lack the capacity to get hung-up enough to think twice about their behavior, inhibit their impulses, or restrain their conduct. A little of the neurotic's typical apprehension would go a long way toward helping the disturbed character be more cautious or hesitant when it comes to frequently doing the  things that cause problems.

For several reasons that I have never fully understood, traditionally-minded therapists, as well as relatively neurotic individuals, appear determined to ascribe fears \& insecurities to disordered characters that simply don't exist. They will frequently misinterpret the behavior \& motivations of character-disordered individuals \& frame their behaviors inaccurately. E.g., some disordered characters have a \textit{passion} for novelty \& a \textit{craving} for excitement. So they constantly seek shallow, intense, short-lived, \& high-risk sexual involvements \& other interpersonal entanglements. But this characteristic thrill-seeking behavior is often framed as a ``fear'' of intimacy or commitment. This mistake is made because it's difficult for neurotic individuals (or traditionally-minded therapists for that matter) to imagine why a person wouldn't necessarily prefer a stable, intimate relationship over multiple risky encounters, unless they were in some way apprehensive about the prospects of engaging in something more substantive. This kind of thinking also reflects a long-held, but never proven, tenet of classical psychology: Everyone will naturally gravitate toward the healthiest life choices unless they are hung-up by unconscious fears that stem from past emotional trauma. It's also possible that some therapists are so married to their traditional metaphors about the nature of human behavior that, even when they encounter a ``square peg,'' they still try to fit it into the proverbial round hole.

\subsubsection{Conscience}
The neurotic individual is basically a person with a very well-developed conscience or superego. Sometimes that conscience can be overly active to the point of being oppressive. Neurotics have a huge sense of right \& wrong, \& always want to do what they think is the most correct. They can sometimes set impossibly high standards, engendering a significant amount of stress. Neurotics are also prone to judge themselves overly harshly when they fail to meet their own expectations. They take on inordinate burdens, carrying the world's weight on their shoulders. When something goes wrong, they quickly ask themselves what more  \textit{they} can do to make the situation better.
	

The disordered character's conscience is remarkably under-developed \& impaired. Disturbed characters don't hear that little voice that urges most of us to do right, or admonishes us when we're contemplating doing wrong. Or if they do hear it, they can easily ignore it, silence it, or put it in a ``lock box'' (i.e. \textit{compartmentalize it}). As opposed to persons with a sound conscience, they don't push themselves to take on unattractive burdens \& responsibilities; \& they don't hold themselves back when they want something they really shouldn't have.

In the most sense disturbances of character, conscience is not simply weak, underdeveloped, or flawed, but \textit{absent altogether}. The little voice most of us have simply isn't there, \& the capacity to even form a conscience is grossly deficient. Robert Hare aptly named his book about the most severely disordered character, the psychopath, \textit{Without Conscience}.\footnote{Hare, R., Without Conscience: \textit{The Disturbing World of the Psychopaths Among Us}, (Guilford Press, 1991).}

It's hard to imagine individuals with no conscience at all. \& as mentioned earlier, our traditional psychology metaphors have conditioned us to believe that underneath it all, we're all the same. Most folks find it unimaginable that some people are simply not ``normal.'' That's why psychopaths are able to prey on the unsuspecting so effectively. Their victims dupe themselves with their inability to accept that the predators they've been dealing with are heartless \& devoid of normal human empathy. It's very important to realize \& accept that not everyone is the same. Not everyone has an active, mature conscience. All the disordered characters have deficiencies of conscience. They vary from those whose conscience is markedly immature to those with no conscience at all. Such individuals can do others great harm with absolutely no compunction. Those who fail to recognize this are extremely vulnerable.

Lacking a mature conscience, possessing a diminished capacity to experience distress in the face of injury to others, \& deficient in empathy, many disturbed characters don't experience genuine remorse for their hurtful acts, whether acts of commission or omission. They might have some after-the-fact \& practical regret for a behavior, especially if it results in some clear \& appreciable cost to them. But because they don't have a healthy conscience, there's nothing to keep them from acting on their destructive impulses in the 1st place. So any practical regret for their own behavior is usually too little \& comes far too late.

\subsubsection{Shame \& Guilt}
Because they are persons of conscience, neurotics can experience high (\& sometimes toxic) levels of both shame \& guilt. Shame is the emotional state we experience when we feel badly about \textit{who we are}. Guilt is the emotional state we experience when we feel badly about \textit{what we've done}. Neurotics tend to judge themselves harshly, so they're quick to feel ashamed when they fail to measure up to their own high standards \& the self-image they try to maintain. They're also quick to feel guilty when they think they've fallen short, done something poorly, or caused injury to someone else.
	
Most neurotics' levels of shame \& guilt are adaptive to some degree. I have encountered a few neurotics who experienced such toxic levels of shame \& guilt growing up that it led them to develop truly pathological symptoms. But even these individuals rarely experienced the extreme levels of guilt or shame necessary to produce the kinds of bizarre psychological phenomena Freud used to treat. Most neurotics are not sick from unreasonable or extreme levels of guilt or shame. But they are hypersensitive to these feelings. They are quick to feel badly about themselves when they've done something that reflects negatively on their character; \& they're too quick to beat themselves up emotionally when they think they've committed unpardonable sins.

Disturbed characters lack sufficient pangs of guilt or shame when they do things that are harmful or hurtful. Such feelings can only emanate from a well-developed conscience, which, as we discussed earlier, they lack. so, shamelessness \& guiltlessness are 2 of the disturbed character's most distinguishing features. They don't feel badly enough about the kind of person they are when they repeated do things that negatively affect or injure others. They also don't feel badly enough about the harmful things they do, at least not badly enough to keep from doing them over \& over again. If they were able to experience enough guilt or shame, they might refrain from doing socially harmful things in the 1st place, or from doing the same wrongful acts repeatedly.

The plethora of books dealing with shame \& guilt that dominated the self-help \& ``recovery'' market of the `60s, `70s, \& even `80s, was largely written \textbf{\textit{by, for, \& about}} neurotics. Shame \& blame were the names of their game, \& most of those books blamed toxic levels of guilt \&\texttt{/}or shame for a wide variety of psychological problems that damaged a person's self-esteem. These books largely made us believe there was no such thing as good shame or healthy guilt. Some authors \& theorists later relented regarding guilt, acknowledging that at least some measure of guilt is necessary to keep us civilized. But even today, the dominant opinion about shame (\& supported by empirical research), casts it as a bad thing, period. The general consensus seems to be this: While it's a relatively good thing to feel badly about something harmful you've \textit{done}, feeling badly about oneself -- about whom one literally is -- is \textit{never} a good thing.

But after working for many years with disturbed characters, I quickly came to question the validity of this premise.

Most of us experience some genuine self-disgust with  the \textit{kind of person} we might find ourselves becoming when we engage in behaviors disordered characters display. This is precisely what prompts us to change our ways, \& restore a self-image we can live with. I've known many individuals who made significant changes in their characters. But when they did so, it was not only because they regretted their irresponsible behaviors, but also because they became unsettled enough with the person they had allowed themselves to become (i.e. become too ashamed of themselves) that they decided to change course. So, it appears that one must have the capacity to experience both shame \& guilt in order to forge a sound character. As is usually the case, however, it's a matter of degree. When individuals experience toxic \& unwarranted levels of either guilt or shame, there can indeed be a negative impact on psychological health.

Some professionals (\& non-professionals) take issue with positions I outlined above. They insist that even disordered characters actually do feel guilt \& shame, but that they effectively utilize -- or perhaps over-utilize -- certain ``defense mechanisms'' such as ``denial'' \& ``projection'' to assuage any emotional pain. This is largely because they still adhere to the tenets of classical psychology (i.e. that \textit{everyone} is to a greater or lesser degree neurotic, if they are not in fact psychotic, \& that all individuals are fundamentally the same at a deeper psychological level). So, it's impossible for them to imagine how anyone could behave so shamelessly unless they were, in fact, defending themselves against real pain \textit{underneath} it all. I'll address the pitfalls of the all-too-common tendency to over-extend \& over-generalize classical psychology's metaphors in more depth later in the book. But it's important to remember that -- although a disordered character's failure to acknowledge or ``come clean'' about wrongdoing can be viewed as a ``state of denial'' prompted by feelings of shame \& guilt -- such a perspective is often inaccurate. \& it's potentially dangerous when trying to deduce the true nature of their problems \& degree of character pathology.

Being embarrassed at being uncovered or found out is not the same as feeling genuine shame. Shame is 1 of those mechanisms that \textit{make a person think twice} about doing something wrong in the 1st place. Moreover, a person who truly feels ashamed is certainly not likely to do the same things over \& over again with no compunction. Character-disturbed people will sometimes claim they didn't come clean with themselves or others, or didn't seek help for their problematic conduct, because they were too ashamed to do so. This is often simply a lie they tell. They know a neurotic person is likely to find such an explanation more acceptable than the truth about their lack of motivation to change.

Individuals overly invested in human behavior's classical explanations also tend to misinterpret the careless, reckless, impulsive propensities of criminal behavior. They might assume criminal individuals had a subconscious desire (arising out of pangs of conscience) to be stopped when their seemingly thoughtless actions lead them to be easily apprehended. There has never been any empirical support for this notion, but that has not kept some from adhering to it. As I have stated several times already: Using metaphors that once appropriately described some neurotic behaviors to explain a disordered character's behavior almost always puts a person at a disadvantage when trying to understand or intervene in troublesome situations. A key thing to remember: Whereas neurotics have a propensity to experience guilt \& shame too readily \& with too great an intensity, the exact opposite is true for the disturbed character.

\subsubsection{Level of awareness}
The distress neurotic individuals experience generally stems from emotional conflicts  that are mostly \textit{unconscious}. A woman experiences an unexplained ``funk.'' She doesn't know it's related to her suppressed feelings of grief \& loss re-surfacing near the anniversary of her mother's death. If she \textit{did} know it, she might not even need to see a therapist to help her sort out why she was suddenly feeling so blue. A man with an ulcer is unaware that his obsessive worry over losing his job, in turn fueled by his deep-seated mistrust of authority figures, arises out of his experience with his abusive father. If he \textit{were} aware of it, he might never have needed to knock on the counselor's door. Neurotics are often in considerable emotional turmoil; but the deeper roots of their distress, \& sometimes even the very nature of their emotions, are often unknown to them.
	
The problems the disordered character experiences might be so ingrained that they fairly ``automatically'' occur. However, it's important to remember that the disordered character is \textit{fully conscious} of his problem behaviors. He not only knows exactly what he's doing, but also is fully aware of his \textit{motivations} for doing it.

Lying is 1 of the more common \& problematic behaviors of the disordered character. Sometimes this lying is done so ``automatically'' that the disturbed character finds himself lying without thinking much about it, \& even when the truth would have done just fine. That doesn't mean he doesn't know he's lying. He knows. He just does it so often \& so readily that he doesn't give it a second thought.

In my classical training I was taught never to ask clients ``why'' they did something because it would likely ``throw them on the \textit{defensive}'' \& they would be ``afraid'' to disclose. So, in my early work with disturbed characters, even though I was very interested in the motivations for their behaviors, I did not ask them directly why they did what they did. A fair amount of the time, when I did broach motivation issues, they would reply with something like: ``To tell you the truth, doctor, I really don't know,'' or ``That's what I'm in therapy to find out.'' This, of course, would reinforce the old notion that 1 of the major tasks for therapy would be to ``uncover'' the unconscious underpinnings of the conduct, so they could eventually ``see'' \& understand the reasons for their behavior \& ``work through'' their ``issues.''

What a surprise it was for me to learn that disordered characters are most often keenly aware, not only of their actions, but also of their motivations for them. This has proven to be true despite the fact that they might use the manipulation \& impression-management tactics of ``playing dumb,'' ``feigning ignorance,'' or ``feigning innocence'' to skirt responsibility. I eventually learned (as have other colleagues \& researchers in the area of character disturbance) that most of the time ``I don't know'' doesn't really mean the disturbed character is oblivious about his actions. It almost always means something else. It can mean:
\begin{itemize}
	\item ``I never really think about it that much.''
	\item ``I don't like to think about it.''
	\item ``I don't want to talk to you about it.''
	\item ``I know very well why I did it, but certainly don't want you to know. That would put you in a position of equal advantage with me, or possibly even give you an advantage over me -- having my number, so to speak -- \& I won't be able to manipulate you as easily or manage your impression of me.''
	\item ``I hope you'll buy the notion that I'm basically a good person whose intentions were benign. That I simply made un unwitting mistake, oblivious about the harm I caused; \& that I am willing to increase my awareness with your guidance.'' (This kind of implied message is the epitome of effective \textit{manipulation \& impression management}, \& it's amazing how many times it is successfully ``sold.'')
\end{itemize}
So, ``I don't know'' can mean any of the above \& a whole host of other things. But in the case of the disturbed character, it \textbf{\textit{rarely}}, truly means ``I don't know.'' So I no longer accept it for an answer. \& once I politely but firmly stopped taking it for an acceptable response, I immediately started getting explanations that made much more sense. Most importantly, the games of manipulation \& impression management my character-disordered clients tried to engage me in diminished dramatically.

Realizing how hoodwinked I had been by accepting their initial non-explanations for behavior helped me become dramatically aware of how expert disordered characters generally are on the subject of neurosis, as well as the mindsets of many mental health professionals, especially those steeped in traditional paradigms. They know very well how neurotics tend to think. They know the attitudes neurotics hold, \& the naiveties that make them vulnerable to tactics of manipulation \& impression management. They often know the neurotics in their lives better than those neurotics know themselves. They're also frequently quite ``couch broken'' (i.e. have made the rounds to many professionals \& become very familiar with psychological concepts, terms, \& paradigms); \& they know what to say that might be easily believed, especially if it's not scrutinized carefully. So, they can manipulate even a seasoned professional.

These days, I focus very little attention on the underpinnings of behavior, even though it's sometimes very helpful to know what all the nefarious motivations might be. I primarily assume that the disordered character is very aware -- not only of his behavior but also the reasons for it. So I direct attention mostly to the behavior itself with an emphasis on changing it.

\subsubsection{The role of feelings}
When a neurotic person seeks counseling for 1 reason to another, you can safely assume some emotional issues need to be attended to or resolved. Perhaps those feelings have been long repressed. Perhaps those feelings are very mixed \& conflicted. In any case, helping persons establish deeper contact with their feelings \& sort through their troubled emotions represents the hallmark of traditional psychotherapy. Regardless of the neurotic individual's problems, traditional psychotherapy almost always devotes considerable focus to the person's \textit{feelings}.
	
The disturbed character's problems with functioning well in a social context are not so much a consequence of the way he feels, but the way he \textit{thinks}. It's his ill-gotten \textit{attitudes}, erroneous \& distorted thinking patterns, \& dysfunctional core beliefs that lie at the root of problems. Awareness of this fact has in recent years spurred a revolution in therapeutic approaches.

The term \textit{cognitive-behavioral therapy} refers to an orientation founded on the principle of an inextricable relationship between a person's core beliefs, attitudes, \& thinking patterns \& that person's behavior. E.g., say a man's core beliefs include the view that any woman is naturally inferior to him, is designed by nature to be submissive, is a rightful personal possession if involved in a serious relationship with him, \& has value only as a sexual object or toy. One would not be particularly surprised to learn that this man had a history of abusive conduct with his wife or girlfriend. \textit{How we think in large measure determines how we will act}. When dealing with disordered characters, their kinds of problematic thinking become the bigger issue to address, as opposed to how they are feeling.

1 of the ways that folks become embroiled in abusive or exploitive relationships is by falling prey to concerns about the way their character-disordered partner is feeling. E.g., they might focus all too much on why their partner seems angry all the time, wondering what they might have done to engender such ire. They almost never consider that the brandishing of anger is sometimes a tactic that character-disturbed individuals use to manipulate \& control others, as opposed to a genuine feeling. What's more, they don't consider that it's their abuser's attitudes toward them, their value system, \& their distorted manner of thinking about things that constitutes the main problem, \& might be the precipitant of any unwarranted anger in the 1st place. If they manage to lure their partner into counseling, \& the counselor is of the traditional mindset \& primarily focuses on ``feelings,'' things are not likely to get much better. They \textit{may possibly get worse} as the non-character-disordered party bares his or her soul in the counseling process \& exposes even more of his or her vulnerability to the abusive party.

When I 1st began using more appropriate methods to counsel character-disordered persons, a frequent comment I heard from clients was, ``I don't think you really care.'' This kind of comment really put me on the defensive (as it would any good neurotic) initially. After all, I was a therapist, \& I was supposed to be a warm, empathic person dedicated to helping alleviate human misery. It took me awhile to realize that a comment like that was often a counter-therapeutic \& manipulative tactic on the disturbed characters' part. They wanted to get me to align with their point of view on things, to justify their conduct, \& most especially to see if they could convince me that they deserved just as much sympathy as those they had brought great pain. Of course I \textit{cared}. But I also came to know that if I were to \textit{care properly} \& \textit{therapeutically} for the disordered characters I was working with, I would have to confront the real impediments to their inner psychological health \& healthy interpersonal functioning. That meant calling them on their tactics \& letting them know that the genuineness of my ``caring'' would become all too evident in time. That would be through my unwavering dedication to challenging \& helping them correct the distorted thinking patterns that had made a mess of their lives \& the lives of others for too many years already. The door was always open to more traditional feelings-based counseling after character issues were resolved. Many did indeed walk through that door when the time came. That made me a believer in the notion that, when it comes to dealing with significant disturbances of character, focusing on feelings is not an exercise in ``caring,'' but rather a perfect example of \textit{enabling}.

In a later chapter, I will outline several of the major thinking errors disordered characters have in common, \& the dysfunctional social attitudes those erroneous ways of thinking breed. Knowing disturbed characters' frequent dysfunctional thinking patterns is crucial to understanding their problematic behavior, \& learning how to confront \& deal with them effectively.

\subsubsection{The role of defense mechanisms}
Neurotics are thought to use a variety of intra-psychic \textit{mechanisms} to \textit{defend} themselves against the experience of emotional pain, \& especially to alleviate anxiety associated with conflicts between their primal urges \& their consciences. Almost everyone has heard of these classic ``defense mechanisms.'' Such mechanisms, by definition, operate \textit{unconsciously}. I.e., the person doesn't deliberately engage in an action, but rather the unconscious mind employs the mechanism so the conscious mind never has to experience the pain in the 1st place. Here's the reason neurotic individuals develop problematic symptoms: These unconscious tools of anxiety mitigation, though powerful, are neither adequate, nor are they always fully adaptive ways to mitigate emotional pain. Many times, the symptoms the neurotic individual brings to the therapist's attention result from residual anxiety -- or the emotional pain left over after ineffective use of 1 or more of the typical defenses. At other times, people might seek help because their defenses have become increasingly inadequate or have begun to break down, letting their emotional pain underneath rise to the surface. In those cases, it's the emotional pain that brings someone into treatment. All successful therapies for neurosis depend upon building a trusting rapport with the therapist, so that an atmosphere of safety \& encouragement allows clients to relax their defenses \& reveal their inner pain or emotional conflicts.

Disordered characters engage in certain behaviors so ``automatic'' that it's tempting to think they do them unconsciously. On the surface, these behaviors often so resemble \textit{defense mechanisms} that they can be easily misinterpreted as such, especially by individuals overly immersed in traditional paradigms. However, on closer inspection, many of these behaviors are more accurately regarded as \textit{tactics} of \textit{manipulation, impression-management}, \& \textit{responsibility-resistance} (much more about this later).

In workshops, I always illustrate the contrast between a true ``defense'' mechanism \& a tactic of manipulation \& responsibility-avoidance using the concept of ``denial.'' 1 of the 10 most commonly misused terms in mental health (more about this later), denial can indeed be an unconscious defense mechanism. Let's take the example of a woman who has been married to the same man for 40 years. She has just rushed him to the hospital because, while they were out in the yard working, he began having trouble speaking \& looked in some distress. The doctors then tell her that he has suffered a stroke, is now virtually brain-dead, \& will not recover. Yet, every day she is by his bedside, holding his hand \& talking to him. The nurses tell her that he cannot hear, but she talks to him anyway. The doctors tell her he will not recover, but she only replies, ``I know he'll pull through, he's such a strong man.'' This woman is in a unique \textit{psychological state} -- the state of \textit{denial}. She can hardly believe what has happened. Not long ago she was in the yard with her darling, enjoying 1 of their favorite activities. The day before, they were at a friend's home for a get-together. He seemed the picture of happiness \& health. He didn't even seem that sick when she brought him to the hospital. Now -- in a blink of an eye -- they're telling her he's gone. This is far more emotional pain than she can bear just yet. She's not ready to accept that her partner of 40 years won't be coming home with her. She's not quite ready to face a life without him. So, her unconscious mind has provided her with an effective (albeit most likely temporary) \textit{defense} against the pain. Eventually, as she becomes better able to accept the distressing reality, her denial will break down. When it does, the pain it served to contain will gush forth \& she will grieve.

Now, let's take another example of so-called ``denial.'' Joe, the class bully, strolls up to 1 of his unsuspecting classmates \& engages in 1 of his favorite mischievous pastimes -- pushing the books out of her arms \& spilling them on the floor. It just so happens that the hall monitor catches the event \& sternly hollers: ``Joe!'' to which Joe, spreading his arms wide open \& with a look of great shock, surprise, \& innocence on his face retorts: ``Whaaaat?'' Does Joe really not understand the reality of what was happened? Does he actually think he didn't do what the hall monitor saw him do? Is he in some kind of altered psychological state? Is his possible altered state brought about by more emotional pain than he could possibly stand to bear? Is he so consumed with shame \&\texttt{/}or guilt for what he's done that he simply can't allow himself to believe he actually did such a horrible thing? More than likely, none of the aforementioned possibilities is correct. Joe is probably more concerned that he has another detention hall coming, which means another note to his parents, \& possibly even a suspension. So, he's got 1 long-shot \textit{tactic} to try. He'll do his best to make the hall monitor believe she didn't really see what she thought she saw. The hallway was crowded. Maybe it was someone else. Maybe it was just an ``accident.'' If he \textit{acts} surprised, innocent, \& righteously indignant enough, maybe, just maybe, she'll begin to doubt herself. He hopes that, unlike him, she might be just \textit{neurotic} enough (i.e. has an overactive conscience \& excessive sense of guilt or shame) to think she might have misjudged the situation. Maybe she'll even berate herself for jumping to conclusions or for causing a possibly innocent person unwarranted emotional pain. This tactic might have worked before. Maybe it will work again.

This preceding example is based on a real case. It is noteworthy that when ``Joe'' realized that he simply couldn't manipulate the hall monitor, he reluctantly stopped ``denying,'' saying: ``Well, maybe I did do it, but she had it coming because she's always bad-mouthing me to her friends.'' Now, we could engage in some discussion about the other tactics Joe is using to continue the game of manipulation \& impression management, but the most important thing to recognize is this: Unlike what happens in the case of real psychological denial as the defense mechanism, in Joe's case we don't see an outpouring of anguish \& grief when the denial ends. The reason is simple. Joe was never ``in denial'' (the psychological state) per se in the 1st place. He was simply \textit{lying}. He eventually stopped lying because it wasn't getting him anywhere. He moved on from the tactic of lying to excuse-making \& playing the victim, also effective tactics of manipulation \& impression-management.

I can't stress this enough: The ``denial'' of the unfortunate elderly woman mentioned above is nothing like the ``denial'' of Joe the school bully. One is truly an ego \textit{defense mechanism}; the other a \textit{manipulation \& responsibility-resistance tactic}. One is an unconscious mechanism of protection from deep emotional pain; the other is a deliberate, calculated lie. Yet many professionals (as well as lay persons) use the same term to describe these very different behaviors. \& because they presume there is only 1 type of denial, whenever their clients display behavior like Joe's, they regard it as a sign of shame-based, unconscious denial. I can't count the number of times clinicians have spoken to me of clients who are still ``in denial'' about 1 problem behavior or another, when what they were really describing was a client still ``lying \& manipulating'' as part of the game of impression management \& responsibility-resistance. As a result, they end up wasting precious time \& missing the mark in multiple sessions designed to help their clients ``come out of their denial.''

It's important for both lay persons \& clinicians not only to know the difference between a true defense mechanism \& a tactic of responsibility-avoidance \& interpersonal manipulation, but also to know how to appropriately respond to these very different types of behaviors. I presented the principal tactics manipulators use \& how to respond to them in \textit{In Sheep's Clothing}.\footnote{Simon, G., \textit{In Sheep's Clothing: Understanding \& Dealing with Manipulative People}, (A.J. Christopher \& Co., 1996).} In a later chapter, we'll take a more in-depth look at those tactics, as well as other responsibility-avoidance \& impression-management maneuvers used by the various disordered characters.

\subsubsection{The genuineness of \textit{style}}
Some of the more prominent traditionally-oriented theorists have conceptualized the neurotic personality as essentially a fraud. The neurotic's true self was thought to be hidden behind a social fa\c{c}ade. So, a particularly gregarious person might be perceived in reality as quite shy \& interpersonally anxious ``underneath'' the social face he or she presents, \& to ``compensate'' for this social insecurity with feigned sociability. Similarly, bullies have been perceived as cowards underneath their brash exterior, \& haughty individuals viewed as compensating for low-self esteem. In short, their outward presentation is an unconsciously constructed ``front'' to mask their inner insecurities. Classical theorists also believed they could essentially define the principal personality types by the defense mechanisms they typically used to protect their true selves. Even relatively recently, this kind of conceptualization was articulated in David Shapiro's landmark work \textit{Neurotic Styles}.\footnote{Shapiro, D., \textit{Neurotic Styles}, (Basic Books, 1999).} He eloquently describes the various personality ``styles,'' but still views them as essentially an expression of a person's neurosis.

Have you ever noticed the consistency in these traditional notions about a very different kind of reality lying underneath the fa\c{c}ade? They always involve an outward appearance that's not very appealing, \& a more pitiable reality underneath. In other words, traditional notions about personality tend to view egomaniacs as really having low self-esteem underneath, bullies being scared little kids underneath, \& abusers being traumatized victims underneath it all, etc. But you've probably never heard a devotee of classical perspectives claim that a shy person is really a ravenous animal underneath it all, wanting to jump the bones of everyone they meet. Or that a particularly sensitive person is really a vicious monster with a heart of stone underneath. I think we're too quick to align with psychology metaphors that have outlined their usefulness for this reason: Most of us still don't like to face the unpleasant things in life, \& want to explain them away with a perspective that makes the unnerving more palatable. After all, genuine \textit{denial} is 1 of the things neurotics do best!

While it might be true to some extent that a neurotic is quite different under the exterior, with the disturbed character, \textbf{\textit{what you see is what you get}}, unless a deliberate con game is being played. W.r.t. primary personality traits, there is no pretense. Disordered characters are who they are, as unfortunate as that may be, \& often to the core.

I once worked in a residential treatment program that specialized in young persons already displaying significant disturbances of character \& conduct. 1 day, a young man was admitted who, within minutes of arrival, began listing on a notepad program improvements he thought the staff needed to make. He wanted to present this list to the facility administrator \& demanded an audience to discuss matters. Steeped in traditional psychological theories, the head nurse -- when this young man's treatment plan was fashioned -- recalled his haughtiness, \& proposed a 1st treatment goal of increasing his sense of self-esteem. She assumed, as is common to neurotics \& devotees of traditional perspectives, that his pompous attitude simply \textit{must} have been a \textit{compensation} for underlying feelings of inferiority. But in time, it became quite apparent that this young man in fact had no feelings of inferiority. Rather, he possessed only a deeply-rooted sense of superiority \& \textit{entitlement} common to individuals over-indulged \& over-valued all their lives -- people who end up as deeply disturbed characters.

\subsubsection{Self-esteem}
The discussion above alludes to another very key difference between neurotics \& disordered characters. Neurotics have significantly damaged self-images, typically arising out of a deflated sense of self-worth or self-esteem. The neurotic often feels defective in some way, \& therefore not truly lovable. Sometimes this impaired self-esteem can lead to a profound sense of inadequacy \& serious weaknesses of character. Sometimes it prompts the neurotic to try hard to please others, \& to do their very best to ``earn'' approval. If not excessive, this tendency can provide a critical incentive to behave in socially acceptable or responsible ways. In other words, when people think their self-worth largely depends on how clearly they demonstrate their value to society, they can be motivated to conduct themselves in a manner that increases the likelihood of benefiting the greater good.

Disordered characters have an inflated sense of self-worth. They see themselves as superior to others. As a result, they often feel \textit{entitled} to use \& exploit others as they see fit. As noted earlier, their ego-inflation is not rooted in underlying feelings of insecurity or inferiority. It's not a pretense. They really do think they're something special \& above the common throng. \&, their sense of superiority fuels their attitudes of entitlement. Later in this book, we'll take a deeper look at 1 type of character disorder, the psychopath, in which this disturbing trait is present to its most pathological extreme.

In a later chapter, I'll also be introducing a critical distinction between the concepts of self-esteem \& self-respect. This distinction will hopefully not only give some additional clarity to self-esteem's definition, but also will help explain why there's often such confusion \& misunderstanding about how \& why some people's self-image gets bent out of shape.

\subsubsection{Response to adverse consequence}
Neurotics try very hard to effect positive outcomes, \& they easily become anxious \& upset when things go badly. They are \textit{hypersensitive to adverse consequence}. A co-occurring trait accompanying this hypersensitivity is the tendency to make \textit{internal attributions} about the reasons for problems. When a neurotic worker doesn't get the ``good job'' comment she craves from her boss, she might well beat herself up with self-criticism. When the neurotic therapist doesn't see positive change in her therapy group, she might well berate herself as a sub-standard counselor. Neurotics always want things to be right with the world, \& they get internally unnerved when things go wrong. They want the world to be at peace, for everyone to love them, \& for everyone's dreams to come true. When things go badly, they take it hard. They're unnerved by the cruelty \& unfairness of life, \& feeling inordinately responsible, they take it upon themselves to make things better. When things go poorly, they chide themselves, question themselves, conduct an internal debate about how they might do thing differently, \& try all the harder to make things right. Even when it's fairly impossible to blame themselves for a painful happenstance (e.g., a natural disaster or catastrophe), they still question \& prod themselves about what they can or should do to make the situation better. \& they're deeply affected by any unintended adverse consequences of their own actions, \& strive to make amends.

Disordered characters are largely unaffected \& \textit{undeterred by adverse consequence}. They have a characteristic imperturbability when it comes to dealing with the negative fallout of their behavior. They are typically not unnerved by situations that would make the neurotic upset. More importantly, they remain undeterred about their basic way of handling things. They might even be strengthened in their resolve to keep doing just as they have been doing, despite the objective fact that their way is clearly not working.

A co-occurring character trait is the disordered character's tendency to make \textit{external attributions} whenever anything bad does happen. They see others \& circumstances outside themselves as the source of problems. So, even if they've lost another job, had another marriage fall apart, run into legal difficulties, or even lost their freedom (if they have become incarcerated), they take it in stride. They never blame themselves, \& keep on behaving the same way they've always behaved, despite where it's gotten them. They even pride themselves in the notion that they will remain who they are, doing things as they've always done, despite the hardships in their lives, including those often resulting from their own behavior.

\subsubsection{Level of internal discomfort}
Neurotics are also different from character disorders on another very key dimension. In large measure, they experience the signs \& symptoms of their neuroses as unpleasant \& unwanted. Perhaps an individual has been worrying to the point that he's developed an ulcer, \& is now in frequent gastric distress. Perhaps this circumstance has slowed him down at work, making him less productive. He still drives hard like always, but his frequent bouts of pain have stymied his usual effectiveness. It's very likely that he does not like the person he has become as a result. Clinicians say that it is \textit{ego-dystonic} (unpalatable to his image of himself) for him not to be performing at his best. What's more, he doesn't like his symptoms either (i.e. the pain of his ulcer). As a result, he will likely take the initiative to get some help. He wants the problem gone, to be out of pain, to be at his best again; \& he'll probably do what he has to do to make things better again. If he seeks help from a therapist, he'll even be open to changing some things about himself so he'll be a better person \& have fewer disturbing symptoms. He'll likely be motivated to at least try what the therapist suggests, to regard the therapist's guidance as valuable as he begins to gain relief, \& to remain in treatment until he has overcome his problem \& doesn't need therapy services anymore.

Disturbed characters also display telltale signs \& symptoms of their disorder. Lying, conning, manipulating, defaulting on social obligations, etc., are several of the disordered character's defining features. The negative attitudes they hold, the distorted way they tend to think, \& the irresponsible ways they tend to behave are likely to be greatly upsetting to others. But these things are what clinicians call \textit{ego-syntonic} to the disordered character. I.e., the disordered character doesn't see anything wrong or disturbing about them. Moreover, he is not upset by the kind of person he has become as a result of these characteristics. He likes who he is \& how he operates. Others may complain that he tends to use \& exploit people. His answer might be that gullible or weaker people deserve to be taken advantage of. As he sees it, if others have a problem with him, it's because \textit{they} are all screwed up. When someone points out 1 of his most disturbing characteristics, he might retort: ``You got a problem with that?'' Disturbed characters also don't seek help in the manner neurotics do. Rather, they're more dragged into counseling by distraught neurotics who are in some kind of relationship with them. Disturbed characters are rarely in the kind of inner distress that prompts most people to seek \& appreciate guidance or counseling on their own.

\subsubsection{Needs in treatment}
Another important difference between neurotics \& disordered characters is what the need most from any kind of therapeutic experience. This is, perhaps, 1 of the most important points I need to make for the benefit of therapists. Because neurotics are struggling with inadequacies, insecurities, \& emotional conflicts, they both crave \& need positive regard as well as supportive guidance. In short, they need \textit{help}. Also, because many of their unresolved issues are mostly rooted in the unconscious, neurotics both need \& benefit from \textit{insight}. They literally don't know what they're doing to perpetuate their difficulties. So they benefit greatly from listening to their counselors \textit{interpret} the ``dynamics'' of their circumstances, thus shedding ``new light'' on their situation. Because they were largely unable to come to such insights on their own, they not only seek help but also appreciate it when they get it in the forms of new insights, emotional support, \& guidance.

In contrast to insight-deficient neurotics, disordered characters are already keenly aware of the ways their thinking \& behavior cause problems. There isn't 1 thing anyone can say or bring to their attention that they haven't heard a thousand times before from a variety of sources, or experienced in a variety of circumstances. In many of my workshops, I introduce the trite little saying: ``They already \textit{see}; they simply \textit{disagree}.'' They're just not disturbed enough by their way of doing things; or they may have been successful enough getting their way by doing those things, so they're resolved not to change that modus operandi. Because they do the things they do so automatically \& habitually, it can seem like they're unaware. But doing things that are 2nd nature to a person is not the same as doing things in which the motivations are truly unknown or unconscious. When it comes to the behaviors that cause problems in the lives of others, disordered characters \textit{know what they're doing} as well as their motivations for doing it. But they're so comfortable with their way of doing things, \& do them so habitually, that they don't give their behavior a second thought. So, what they really need within the context of any relationship (whether it be a therapeutic relationship with a counselor or any other relationship) is not so much \textit{help} \& \textit{insight} as benign yet firm \textit{confrontation, limit-setting}, \& most especially, \textit{correction}. I frame the things they need the most in any interpersonal encounter as ``corrective emotional \& behavioral experience.'' By this I mean they need an encounter which directly confronts \& challenges their dysfunctional beliefs, destructive attitudes, \& distorted ways of thinking; \& which stymies their typical attempts at manipulation \& impression management. This is done by setting firm limits on their maladaptive behavior, \& \textit{structuring the terms of engagement} in a manner that prompts them to try out alternative, more pro-social ways of interrelating, which can then be reinforced. Doing this resolutely but without hostility or other negative emotion is a genuine art.

Some therapists say it's impossible to effectively treat disturbances of character. This is a truly sad misconception. Disturbed characters can be treated, but it's virtually impossible to treat them effectively with the methods most therapists learned to treat neurosis. It's like a physician trying to do delicate brain surgery with a dentist's appliances. It doesn't matter how well-trained the physician is, or how carefully she conducts the surgery. If the tools \& implements she uses are those made for orthodontics, the outcome will likely be bleak. The outlook would be even bleaker if the physician held the belief that all human maladies arise from different forms of tooth disease (analogous to the absurd but still all-too-commonly-held notion that \textit{all} maladaptive human behavior or personality dysfunction are forms of neurosis). Character disturbance can in fact be treated, but because it's such a very different phenomenon from neurosis, the endeavor requires an approach radically different from those developed to treat neurosis. Even when using the proper tools, treating disturbed characters is a particularly challenging \& difficult endeavor. But it's truly an impossible task when therapists insist upon viewing \& intervening with the disturbed character as they would a neurotic.

Here's another reason clinicians have long believed that personality \& character disturbances couldn't be treated: They rarely tried addressing core character issues directly in therapy. Rather than focus on the dysfunctional coping ``style'' that begot an individual's problems \& symptoms in the 1st place, therapists often gave attention to the problems \& symptoms themselves. To address character issues directly -- instead of focusing on problems communicating, what a person's memories of childhood are, or how they were parented, etc. -- a therapist would need to confront, e.g., how a person' inflated self-image, fueled by their egocentric thinking \& attitudes of entitlement, leads them to chronically exploit \& demean people in relationships, \& how it causes other problems. It's impossible to ameliorate a condition you ignore. Giving a person's dysfunctional personality style center stage in therapy is essential to helping change it.

The last big reason some clinicians think it's impossible to effectively treat disturbed characters is because of their own over-immersion in insight-oriented techniques. They spend inordinate time trying to get their clients to ``see'' the folly of their ways. They might even have the vanity to think that, if they only find the right way to frame things, or make their case in a more eloquent, convincing, or empathic way than any therapist before them, they can make their client finally ``get it.'' They waste a lot of time \& energy on this. \& when the effort fails, rather than question the appropriateness of adopting the insight-oriented approach, they ascribe the failure to the seriousness \& intractability to their client's disturbance.

When I 1st began treating disturbed characters, cognitive-behavioral therapy (the paradigm of choice) was still in its infancy. But clinicians are increasingly coming to appreciate its value, especially in dealing with such problems. \& using the tools arising out of cognitive-behavioral paradigms is just the beginning. Treating character disturbance is relatively new territory, \& the tools \& techniques needed to address it are still very much in the developmental stage.

\subsubsection{Impact of symptoms \& behavioral ``style''}
Neurotics generally develop ``symptoms'' (e.g., stress-exacerbated ulcers, phobias, etc.) that are \textit{self-distressing} \& \textit{self-defeating}. Neurotics make themselves miserable \& stymie their own well-being as a result of their insecurities \& hang-ups. Their way of coping with stress (i.e. their coping \textit{style}) is inadequate \& negatively impacts their own personal development. By contrast, disordered characters' symptoms (e.g., problematic attitudes, thinking patterns, antisocial behaviors, etc.), for the most part negatively impact others. The way the disordered character behaves makes everyone else's life difficult. Their methods \& tactics might be self-defeating in the long-run, but they're certainly intended to be -- \& for a time often are --ruthlessly self-advancing, usually to everyone else's detriment. So profound truth lies in this old adage among mental health professionals: If clients are miserable, they're probably at least to some degree neurotic; \& if they're making someone else miserable, they're probably at least to some degree character disordered.

The chart on the next page outlines the significant differences between neurotics \& disturbed characters:'' -- \cite[pp. 27--53]{Simon2011}
\begin{table}[H]
	\centering
	\begin{tabular}{|p{.47\textwidth}|p{.49\textwidth}|}
		\hline
		\textbf{Neurotic} & \textbf{Character Disorder} \\
		\hline
		Anxiety is a major factor in symptoms\texttt{/}self-presentation. & Anxiety plays minor role or is problematically lacking. \\
		\hline
		Conscience is very well-developed, overactive. & Conscience is underdeveloped or lacking. \\
		\hline
		Excessive guilt\texttt{/}shame. & Insufficient guilt\texttt{/}shame. \\
		\hline
		Problems arise from \textit{unconscious} conflicts. & Problem behaviors are habitual but deliberate. \\
		\hline
		Conflicted emotions at root problems. & Problematic thinking patterns, attitudes, \& behaviors create difficulties. \\
		\hline
		Use defense mechanisms to mediate anxiety\texttt{/}emotional pain. & Use behaviors \& tactics to shirk responsibility. \\
		\hline
		Authentic self hidden behind defenses. & Self-presentation is authentic but problematic. \\
		\hline
		Damaged self-esteem. & Inflated sense of self-worth. \\
		\hline
		Hypersensitive to adverse consequence. & Undeterred by adverse consequence. \\
		\hline
		Needs\texttt{/}benefits from insight. & Has awareness. Needs correction. \\
		\hline
		Ego-dystonic symptoms \& coping patterns. & Ego-syntonic symptoms \& coping patterns. \\
		\hline
		Self-defeating coping patterns. & Coping patterns meant to be self-advancing \& victimize others. \\
		\hline
	\end{tabular}
\end{table}

%------------------------------------------------------------------------------%

\section{Major Disturbances of Personality \& Character}
``Personality traits of 1 variety or another help define every 1 of us as a unique individual. But at times, such traits are of a quality, intensity, intractability, or cluster in such a manner that they cause significant problems in everyday functioning. Then that personality is said to be \textit{disturbed}, or in the more extreme cases, \textit{disordered}. The official psychiatric diagnostic manual employs some very stringent guidelines to determine whether a person qualifies for a personality disorder diagnosis. Further, the manual attempts to be as objective as possible, using only observable behavioral criteria to delineate the various personality disorders.

This book's purpose is \textit{not} to provide definite diagnostic criteria about distinct disorders of personality per se. It's a book about character, \& specifically about disturbances of character. It is meant to be as accurately descriptive as possible, but not technically diagnostic in nature as we illustrate the major personality types \& disturbances of character. The various types depicted differ in the degree they represent a problematic level of interpersonal \& social functioning.

Several personality types have been generally recognized by researchers \& professionals for many years. More traditionally-oriented theorists still regard the various personality styles as different manifestations of neurosis.\footnote{Shapiro, D., \textit{Neurotic Styles}, (Basic Books, 1999), p. 1.} Theorists who share this perspective view the various outward behavioral manifestations of personality as a set of compensations for, or false representations of, a much different \& unconscious underlying reality. More contemporary thinkers view at least some personality patterns as genuine manifestations of an individual's conscious, but habitual \& preferred way, of relating to the world.\footnote{}Millon, T., \textit{Personality Disorders in Modern Life}, (Wiley, 2000), p. 2.

Now, a major question arises about how to decide to what degree a person's preferred way of coping represents a ``neurotic'' style or a disturbance of character, especially when, on the surface, it's often difficult to tell the difference. Biological science recognizes the difference between phenotype (the outward appearance of an organism) \& genotype (it's genetic makeup). It just so happens that some creatures are genetically very different from one another, but look identical \& even behave in a similar manner. If you're a biologist or genetic scientist, you can resolve the issue by typing the DNA. But judging personality is a bit more complex.

We must take into account 2 important factors when trying to decide whether we're dealing with a relatively neurotic personality or a person of genuinely disturbed character. 1st, some personality types are more often associated with varying degrees of neurosis, \& others with varying degrees of character disturbance. So, once you know the major personality types (as we'll be exploring shortly), you can make some preliminary judgments about the kind of person you're \textit{probably} dealing with. But to move beyond probability to a greater sense of certainty, you have to look at the cluster of characteristics, presented in Chap. 1, that distinguish character disturbance from neurosis. E.g., if you encounter an egoistic, pretentious person, you could entertain either the notion that they are a character-disturbed individual, or a neurotic whose pretentiousness is a fa\c{c}ade masking underlying insecurity. But if this person chronically displays egocentric thinking \& other thinking errors, reveals attitudes of entitlement, uses various tactics of responsibility-avoidance, routinely exploits others, persists in the behavior pattern despite adverse consequence, already shows a high level of insight, \& -- despite bringing repeated pain into the lives of others is completely comfortable with the kind of person he or she is, etc. -- you can be lot surer you're dealing with a character-disturbed narcissist. Keep the aforementioned guidelines in mind as we take a look at the major personality disturbances, with special emphasis on those types most often associated with significant deficiencies of character.

The figures below outline some essential dimensions of the more common interpersonal relating ``styles'' or personality types. They are a greatly simplified representation of the principal styles \& dimensions of functioning that Millon outlines in several of his works.\footnote{Millon, T., \textit{Personality Disorders in Modern Life}, (Wiley, 2000), pp. 60--61.} 2 of the represented domains of interpersonal functioning involve (1) whether a person primarily finds satisfaction of emotional needs in external sources (i.e. is emotionally dependent) vs. internal sources (i.e. is emotionally independent), \& (2) whether the interpersonal style of relating has been primarily shaped \& maintained by what the person actively does (i.e. the ``active'' dimension) as opposed to what he or she fails to do (i.e. the ``passive'' dimension). The aes depict varying degrees along a continuum w.r.t. these 2 dimensions. Most individuals achieve a healthy balance on all these dimensions. Some get ``stuck'' in a state of ambivalence, never really resolving the developmental task of solidifying a balanced style of relating. \&, of course, the personality traits of most individuals are neither so extreme nor inflexible that they cause interpersonal dysfunction. Those personalities whose relating styles significantly impair their ability to function adaptively are considered \textit{disordered}.'' -- \cite[pp. 54--56]{Simon2011}
\begin{figure}[H]
	\centering
	\begin{subfigure}[b]{0.49\textwidth}
		\centering
		\includegraphics[scale=0.55]{Simon2011_3}
	\end{subfigure}
	\begin{subfigure}[b]{0.49\textwidth}
		\centering
		\includegraphics[scale=0.55]{Simon2011_4}
	\end{subfigure}
\end{figure}

\subsection{Predominately Neurotic Personalities}

\subsubsection{The Deferential Pattern}
``Some predisposed individuals look externally to satisfy their emotional needs. They don't find within themselves either the resources or the confidence to tackle life's challenges, \& are overly reliant on others to provide necessary stimulation \& support. They are notoriously \textit{passive}, non-assertive, accommodating, \& acquiescent. \&, because they habitually fail to act in their own behalf, they deny themselves the opportunities for the potential successes they need to build self-confidence. They might occasionally take a chance \& venture out, but if they meet with significant obstacles or resistance, they quickly retreat \& are reluctant to try again. This then perpetuates their pattern of non-assertion.

\textit{Passive-dependent} (or simply, \textit{dependent}) is the label some clinicians \& researchers have given to those personalities. They are all too willing to concede defeat in the face of challenge, \& to turn their lives over to the care of someone else they view as more powerful, capable, \& more resilient than themselves. Because they are so emotionally dependent upon others \& lack the skill to function autonomously, they can be remarkably \textit{submissive} \& \textit{deferential} in their style of relating interpersonally.

The dependent personality is driven by several fears, namely the fears of abandonment, failure, \& even success. Failure signals to them that their self-doubts are justified, \& reinforces their perceived need of others. Success begets fears of separation from familiar sources of support. At a very deep level, these individuals equate being ``on their own'' being being ``alone,'' \& this creates intense \& deep-seated anxiety.

The dependent, interpersonal style of relating is both begotten \& maintained primarily by what these individuals do not do (hence the \textit{passive} component of passive-dependency). Typically, they don't assert themselves or act autonomously \& independently. They experience considerable anxiety w.r.t. their personal safety \& well-being, especially when they are not firmly tied to a reliable source of emotional support. Their anxieties about self-assertion \& the potential loss of support systems fuel considerable neurosis. When they say ``no,'' they too readily feel guilty. When they're tempted to challenge an oppressive situation or partner, their fears of potential abandonment kicks in. So instead of standing up for themselves \& becoming more independent, they acquiesce \& remain emotionally dependent.

The core characteristics of the Deferential Personality are:
\begin{itemize}
	\item Over-reliance on external sources of emotional gratification \& support.
	\item Excessive readiness to capitulate or submit in interpersonal encounters or when facing the challenges of daily living.
	\item A tendency to affiliate with those viewed as more powerful or capable than themselves.
	\item Apprehension, anxiety, \& other symptoms of distress when faced with potential losses of support.
\end{itemize}
There are several factors that have been advanced as possible contributors to the development of the deferential ``style'':

Possible Constitutional (biological, temperamental) Factors:
\begin{itemize}
	\item These individuals tend to have relatively pacific, retreating temperaments.
	\item They tend to have high needs for safety \& protection.
	\item They tend to be highly responsive to external reward.
\end{itemize}
Possible Learning Factors:
\begin{itemize}
	\item These individuals might have over-learned that powerful others will nurture \& protect them, possibly even better than they have learned to protect themselves.
	\item They appear to lack experience in fending for themselves emotionally, behaviorally, \& occupationally.
	\item They appear to have failed to adequately discriminate between functioning autonomously (i.e. being ``on their own'') \& being emotionally abandoned (i.e. totally ``alone'').
	\item On balance, individuals with this personality type are generally much more neurotic than they are character disturbed. If there is a dimension of their personality one could regard as character deficiency, it would be their \textit{strength} of character. These individuals are frequently seen as ``weak'' \& ineffective. They are often the archetypal ``doormats'' in relationships. They lack the necessary confidence, resoluteness, \& persistence necessary to fend for themselves, \& for others to be able to depend on them.
	\item Millon\footnote{Millon, T., \textit{Personality Disorders in Modern Life}, (Wiley, 2000), pp. 210--213.} suggests that there are some common variations of this personality type, depending upon which personality traits dominate. He proposes that the underdeveloped capacity of some dependent personalities to face life's challenges \& meet its responsibilities begets a pattern of immaturity \& inadequacy (i.e., the ``inadequate'' personality variant). The disquieted or avoidant dependent anticipates danger \& potential abandonment, unless closely aligned with trustworthy, supportive others. The overly selfless dependent cares so little about self that any sense of personal identity or worth becomes obscured or absorbed by another viewed as stronger or more powerful. The accommodating variation is overly agreeable, compliant, \& subservient, catering to the needs of others in exchange for a sense of being valued \& cared for.'' -- \cite[pp. 56--59]{Simon2011}
\end{itemize}

\subsubsection{The Histrionic Pattern}
``Another personality type also depends upon external sources for satisfying emotional needs. But individuals with this personality type are very \textit{active} in pursuing those sources of support \& stimulation. They are expert at securing the involvement of others in their lives. They have a flair for the dramatic, \& a repertoire of highly seductive \& superficially appealing behaviors they employ to solicit attention \& lure others into relationships. This is the historic personality type.

Some histrionics have a marked tendency to be overly reactive \& theatrical. Heightened emotionality is part of their constitutional makeup. To some degree, however, their antics are often superficial \& manipulative. They tend to over-dramatize \& to experience the secondary gains of securing attention from others. This often leads them to form relationships that are shallow, unsubstantial, \& unstable, although they can often be quite intense.

Some histrionic personalities tend to be rather vain \& self-focused, not only seeking to be the center of attention, but becoming quite unhappy when others are not doting on them. Some are preoccupied with physical beauty \& other ``accidental'' but desirable human attributes. Others can be quite manipulative when it comes to securing the attention \& involvement they seek from others.

On balance, histrionic personalities tend to be a bit more neurotic as opposed to character-disordered. But because of some of the traits just mentioned, they are not as far toward the neurotic end of the spectrum as some of the other personalities we'll be discussing; \& certainly not as far toward that end as their passive-dependent counterparts. Vanity, superficiality, \& excessive self-focus naturally reflect poorly on anyone's character. Exactly where a particular histrionic personality lies on the character-disorder vs. neurotic continuum can vary considerably. It depends on the various other traits that might be dominant in their personality (e.g., the craving for novelty \& excitement, the tendency to be overly emotional, reactive, sensation-seeking, erratic) as well the other personality traits that might co-exist with their dominant coping style.

The core characteristics of the histrionic personality are:
\begin{itemize}
	\item Over-reliance on external sources of emotional support, gratification, \& stimulation.
	\item Active, often dramatic attempts to secure desired attention \& involvement with others.
	\item Interpersonal gregariousness.
	\item Displays of intense \& occasionally superficial emotionality.
	\item Tendency to form highly emotionally-charged but relatively shallow relationships.
\end{itemize}
Possible Constitutional Factors:
\begin{itemize}
	\item These individuals tend to crave novelty \& to be excitement-seeking.
	\item They tend to have high levels of emotionality \& reactivity.
	\item They appear highly responsive to external sources of stimulation \& reward.
\end{itemize}
Possible Learning Factors:
\begin{itemize}
	\item These personalities might have over-learned that others can be seduced or manipulated into providing attention, support, \& gratification of emotional needs.
	\item They may have failed to learned that others have value that goes deeper than the accidental attributes they possess \& the excitement or stimulation they can bring to a relationship.'' -- \cite[pp. 59--61]{Simon2011}
\end{itemize}

\subsubsection{The Antisocial Pattern}
``Antisocial personalities are individuals who simply don't connect or engage with others as most of us do. What's more, they don't experience any pressing urge to do so. They have been given all sorts of clinical labels in the past such as ``schizoid'' or ``detached.'' In common parlance, they have been frequently but \textit{erroneously} labeled ``anti-social'' by individuals attempting to describe their idiosyncratic aloofness or \textit{asociality}. They are often described as ``loners'' or social isolates who don't appear to enjoy or desire the same kinds of social connection \& involvement that give most individuals' lives meaning \& richness.

Many asocial personalities appear predisposed t their style of relating as the result of biologically-based characteristics (e.g., mild autistic traits, lack of ability to respond to external stimulation \& reward, \& impoverished capacity for emotional responsiveness \& expression) as opposed to the environmental factors that often contribute to dysfunctional personality development. Some researchers suggest a dimension of human functioning \& a ``Spectrum'' of conditions exist that include schizoid personalities as well as the disorders of Asperger's Syndrome \& Autism.

Most of the difficulties these individuals experience for functioning adaptively do not appear to arise out of neurotic conflicts or deficiencies of integrity \& morality. So it's not really useful to assign them a place on the continuum of neurosis vs. character disturbance. Naturally, however, if other traits associated with either neurosis or character disturbance are also present, it can further complicate the problems such personalities experience in relating to others.

Millon\footnote{Millon, T., \textit{Personality Disorders in Modern Life}, (Wiley, 2000), pp. 315--317.} suggests some major variations of this personality, each of which is characterized by the predominance of 1 or more of their typical traits. He notes that some asocial personalities are remarkably remote \& live an almost hermit-like existence. Others appear to be extremely introversive, living in their own world, detached from others \& things around them. Some are lethargic \& energy-depressed, \& appear to have a fairly chronic anhedonia (i.e. inability to experience pleasure or joy). Others are primarily characterized by their emotional aloofness \&\texttt{/}or constriction.

The core characteristics of the Asocial Personality are:
\begin{itemize}
	\item A marked pattern of social detachment.
	\item Diminished capacity to experience pleasure in typical human social activities.
	\item Emotional constriction.
	\item Diminished capacity to react \& respond to others.
\end{itemize}
Possible Constitutional Factors:
\begin{itemize}
	\item Diminished capacity to be affected by external reward.
	\item Emotional imperturbability \& constriction.
	\item Intrinsic lethargy \& psychomotor retardation.
	\item Social detachment.
\end{itemize}
Possible Learning Factors:
\begin{itemize}
	\item It does not appear that learning failures, environmental trauma, or response to over-learning issues play significant roles in the development of this personality pattern. However, schizoid individuals generally don't experience the same types of social engagement, encouragement, reward, etc., that most of us do. So, the relative absence of such social reinforcements might play a role in the perpetuation of their interactive style.'' -- \cite[pp. 61--62]{Simon2011}
\end{itemize}

\subsubsection{The Avoidant Pattern}
``Some individuals actually want to connect with others but experience inordinate apprehension about doing so. As a result, they typically ``avoid'' potential hurtful or disappointing intimate involvements. Such individuals have been often labeled ``avoidant'' personalities.

A few avoidant personalities are so hypersensitive to perceived rejection or disappointment that they misjudge the intentions \& actions of others. So they end up denying themselves reasonable opportunities for intimacy \& support. Others tend to over-react to circumstances in which they allowed themselves to be vulnerable, \& to erroneously perceive that they were mistreated, ignored, or abused. Some avoidant personalities display a marked negativity \& pessimism. Others have a characteristic but less than paranoid level of mistrust. Still others experience a fair degree of persistent apprehensiveness, especially in situations where intimate involvement with or trust of others is at stake.

Avoidant personalities will form close attachments when they perceive they've received unusual \& unquestionable re-assurance that they will not be disappointed, criticized, or rejected. Even then, however, they are likely to continually test the loyalty of those with whom they wish to bound. When they sense they are safe, they often remain involved \& loyal to the end.

Because they are so fearful of rejection or disapproval, avoidant personalities will often shy away from occupational endeavors, or other enterprises that expose them to the social spotlight, inviting the risk of being negatively evaluated. Chronically fearing to venture out, some avoidant personalities develop a marked sense of inferiority \& incompetence. By persistently not taking risks, they only perpetuate their sense of personal inadequacy.

The core aspect of their personality involves a strong desire for meaningful involvements, yet anticipation of rejection, abandonment, \& mistreatment. So avoidant personalities experience a considerable degree of chronic anxiety. Their inner turmoil about whether they can safely satisfy their basic need for affiliation begets many approach-avoidance conflicts. On balance, therefore, they are much more neurotic than they are character-disordered.

Avoidant Personalities display the following core characteristics:
\begin{itemize}
	\item Apprehensive about intimate involvements.
	\item Preoccupied with approval \& loyalty, \& hypersensitive to perceived rejection, disappointment or betrayal.
	\item Avoid intimate relationship unless given strong guarantees of acceptance, \& avoid ventures that might result in disapproval.
\end{itemize}
Possible Constitutional Factors:
\begin{itemize}
	\item Hypersensitivity to rejection\texttt{/}disapproval.
	\item Excessive anxiety, especially social anxiety.
	\item Innate shyness.
\end{itemize}
Possible Learning Factors:
\begin{itemize}
	\item Social immaturity may have led to high levels of social rejection, mockery, \& isolation.
	\item Early bonding experiences might have led to initial intimacy followed by rejection \& abandonment.
	\item Chronic avoidance of risk-taking often leads to self-perceptions of ineptness \& incapability.'' -- \cite[pp. 62--64]{Simon2011}
\end{itemize}

\subsubsection{The Obsessive-Compulsive Pattern}
``These individuals are distinctively \& intensely ambivalent about 1 of the most crucial dimensions of interpersonal functioning: emotional independence vs. dependence. They want to function in an independent way, to chart their own course, \& set their own rules. But they also fear potentially losing the approval, support, \& reinforcement they desire from others. So, they keep their inner urges to rebel \& defy in close check, leading lives of conformity, \& rigid adherence to principles \& expectations. Their deep-seated ambivalence is perpetuated by what they will not let themselves do, namely from time to time cut loose \& act with relative indifference to the expectations of others.

Obsessive-Compulsives are the folks who are proverbially ``wound too tightly.'' These days, it's become fashionable once again to call them ``anal'' personalities. This gives some recognition \& credence to Freud's notion that they developed their personalities because they gained too much satisfaction, did not get sufficient satisfaction, or experienced too much trauma exercising their sphincter muscles during toilet-training, \& as a result became obsessed with ``control'' issues. The validity of this notion (especially as a general characteristic of all such personalities), however, has never been clearly demonstrated.

There are several minor variations of this personality, some whose cardinal attributes is their high level of conscientiousness. This can easily lead to work addiction, some who tend to be miserly \& no-giving, \& some who tend to be so concerned about the rules that they lack imagination \& appreciation for human emotional needs.

Obsessive-compulsive personalities are among the most neurotic of all the personality types. They suffer considerable, chronic anxiety, because underneath it all they so want to break free of their self-imposed chains, yet greatly fear to do so. They never fully mastered the developmental task outlined by Erikson of initiative vs. guilt.\footnote{Millon, T., \textit{Disorders of Personality}, (Wiley Interscience, 1981), p. 246.} Here's the main aspect of their personality that reflects negatively on their character: Their tendency to be so preoccupied with their obsessions \& compulsions that they don't fully appreciate the negative impact on others of their apparent cold \& controlling ways.

These personalities have an overly developed sense of guilt for doing things they think others will disapprove of. Overly guilt-sensitive would describe their core psychological dynamic. Their excessive desire to avoid pangs of guilt is what drives their obsessive \& compulsive behavior. They never want others to be able to convict them of wrongdoing.

Some Obsessive-Compulsives have certain other traits that make them slightly different in their overall modus operandi. Some are conscientious to a fault, harder on themselves than on anyone else, \& prone to doubting whether they can ever measure up to their own standards. Some tend to be miserly \& unforgiving \& prone to hoarding. Others tend to revel in bureaucracy \& find security in rules \& regulations. Others are overly puritanical \& dogmatic, tightly controlled morally, prudish, \& judgmental.

Obsessive-Compulsive personalities appear to be endowed with a high capacity to experience both fear \& anger. They both reduce fear \& channel anger by maintaining rigid control. Even though they are among the most neurotic personality types, some O-Cs evidence a degree of character disturbance. The thing that makes the big difference revolves around how their penchant for control is expressed. The more neurotic obsessive-compulsives cause themselves no end of grief because of the unreasonable demands they place upon themselves. Those who frequently attempt to control others \& use tactics to get others to do their bidding, in total disregard for the emotional toll it can take, have additional traits in their personality (which will be discussed later) that represent a degree of character disturbance.

The core characteristics of the Obsessive-Compulsive Personality are:
\begin{itemize}
	\item Over-conscientiousness regarding rules, propriety, etc.
	\item Perfectionism \& orderliness.
	\item Hesitance to surrender control.
	\item Rigidity \& inflexibility.
\end{itemize}
Possible Constitutional Factors:
\begin{itemize}
	\item Hypersensitivity to feelings of guilt.
	\item Excessive anxiety related to initiative vs. guilt behaviors.
	\item High limbic arousal (heightened capacity to experience both fear \& anger).
\end{itemize}
Possible Learning Factors:
\begin{itemize}
	\item Very \textit{conditionally}-approving, \& possibly overly punitive \& overly-controlling parents.
	\item Overly learned to reduce fear \& release anger through the exercise of rigid control.'' -- \cite[pp. 64--66]{Simon2011}
\end{itemize}

\subsubsection{The Passive-Aggressive Pattern}
``This is an often misunderstood \& mislabeled personality type. The official psychiatric manual doesn't even recognize this as a personality pattern anymore.\footnote{\textit{Diagnostic \& Statistical Manual of Mental Disorders}, 4th Edition-Text Revision, (American Psychiatric Association, 2000), pp. 789--791.} 1 of the reasons: the confusion that's always existed w.r.t. adequately defining this personality type. In the deepest recesses of their psyches, these individuals are very bit as ambivalent as their obsessive-compulsive counterparts about whether to function in an autonomous, independent manner or to rely on others. The difference: These personalities perpetuate this ambivalence very actively in the way they conduct their interpersonal relations.

There is no escaping the ambivalence of the passive-aggressive. They might appeal to another for support, but when the support is offered they will typically reject it or stymie it. They will ask for another to take the lead, \& then resist cooperating. The label passive-aggressive was applied to these individuals early on because of the extent they displayed passive resistance to cooperation in their relationships. But over the years, the term passive-aggressive also came to be commonly, but erroneously, used by professionals \& lay persons alike to describe a very different personality type.

Life with a genuinely passive-aggressive personality is always difficult \& engenders considerable frustration. Consider the following example: A husband asks his wife where she wants to go for dinner. She replies, ``I don't know, honey, you decide.'' He says: ``Let's go to the Chinese place.'' She replies: ``Why the Chinese place? You know the last time we went there I didn't like it that much.'' He then says: ``We'll go wherever you want. Where would you like to go?'' She replies: ``I've got my hands full. You decide.'' ``Okay, let's go to Smith's Steakhouse.'' She replies: ``Now you know how that will stretch our budget, \& how that would wreck our promise to eat more healthily.'' \& on \& on it goes for the passive-aggressive personality. Equally desiring to be taken care of \& utterly resenting the idea of following someone else's lead, they actively vacillate between crying out to others for direction \& then thwarting others' attempts to take charge \& resisting the perceived demand to fall in line.

Passive-Aggressive personalities have been labeled by Millon as \textit{negativistic}\footnote{Erikson, Erik H., \textit{Childhood \& Society}, (Norton, 1950).} because of the distinctively negative character of their ongoing internal conflict, \& the whininess, poutiness, contrariness, \& infuriating uncooperativeness they display in a variety of overt as well as subtle ways. There are minor variations of this personality type, \& in each variation different aspects or traits tend to dominate. There are those whose: (1) characteristic fence-sitting \& indecisiveness are more prominent, (2) complaining \& negative mood is more pronounced, (3) negativism takes on a harsh, critical \& biting edge, \& (4) penchant for uncooperativeness is reflected in their not-so-accidental forgetfulness, dawdling, \& foot-dragging.

Unfortunately, clinicians \& lay persons alike erroneously use the term passive-aggressive when they're trying to describe deliberate (\textit{active}) but subtle, underhanded, \& otherwise \textit{covert} attempts to dominate, exploit, manipulate \& control. What's worse, there is a personality type (to be discussed later) best defined by their extraordinarily manipulative (i.e. covert-aggressive) character. Such individuals, who are not at all ambivalent about whether they want to dominate, frequently engage in crafty, hard-to-detect, ``gotcha'' behaviors \& back-stabbing. This personality type has been also erroneously labeled passive-aggressive by many. These underhanded connivers are better labeled differently, \& will be discussed at length later on. But it's important to remember that there's absolutely nothing passive about their manner of relating to others. Besides, such connivers are among the most character-disturbed of all personality types, whereas the passive-aggressive (perhaps the better descriptor would be \textit{recalcitrant}) personality is among the most neurotic.

Passive-aggressive personalities \& obsessive-compulsive personalities are similar in their deep-seated ambivalence about whether to function independently or depend upon others. They're also similar in the degree to which they are neurotic. But the ambivalence, anxiety, \& neurosis they experience have different origins. The obsessive-compulsive personality is driven by a hypersensitivity to guilt \& the desire to avoid at all costs doing something which might lead to feeling guilty. In contrast, the passive-aggressive personality is driven by an excessive sensitivity to shame. Passive-aggressives appear to have failed to master the developmental task, outlined by Erikson,\footnote{Erikson, Erik H., \textit{Childhood \& Society}, (Norton, 1950).} of autonomy vs. shame \& doubt. These individuals are overly sensitive to what appears to be a lack of unconditional approval -- not of their behavior so much -- but of themselves as persons of worth. They are deeply ambivalent about taking charge of their own lives, as opposed to relying on the approval of others. They want to act in an autonomous fashion, but they also don't want to risk the potential for self-blame should they fail. Similarly, putting themselves in a position to follow the lead of others only invites them to feel weak \& ineffectual. They are constantly in a real blind. They want others to take charge, but resent acceding to demands placed upon them. They can't seem to find the balance between doing as they wish \& relying on others. They are proverbially \& perennially caught between a rock \& a hard place.

As mentioned before, passive-aggressive personalities are, on balance, more neurotic than character disordered. However, their characteristic obstinacy, deficient capacity for autonomy, \& sometimes abrasive negativism all reflect poorly on their character.

The core characteristics of the passive-aggressive personality are:
\begin{itemize}
	\item Pervasive negativism \& complaining.
	\item Expression of anger through passive resistance (not talking, pouting, not-so-accidental ``forgetting.'')
	\item Frequent refusal to meet perceived demands of others even when self-defeating.	
\end{itemize}
Possible Constitutional Factors:
\begin{itemize}
	\item Hypersensitivity to shame. Inordinate desire for unconditional love \& acceptance.
	\item Difficulty expressing anger openly \& directly.
\end{itemize}
Possible Learning Factors:
\begin{itemize}
	\item Mixed messages about self-worth in childhood. Sometimes ``schismatic'' families in which some members are overly doting \& unconditionally regarding while others are critical, demanding, shaming, \& rejecting.
	\item Mixed messages about whether greater reinforcement comes from functioning primarily on their own vs. relying on others for direction, approval, \& support.'' -- \cite[pp. 66--69]{Simon2011}
\end{itemize}

\subsubsection{The Assertive Pattern}
``Generally, this personality is fairly balanced w.r.t. the neurotic vs. character-disturbed dimension. This is an ``actively independent'' personality type. I.e., this personality actively seeks to maintain control over his\texttt{/}her life \& actively attends to getting his\texttt{/}her needs met without reliance on others. Whereas most healthy personalities have achieved an adaptive balance between the degree to which they need \& depend on others \& the degree to which they rely solely upon themselves, this personality is unabashedly self-reliant. But unlike their aggressive-personality counterparts (to be discussed a bit later), these personalities are not driven toward independence by a fierce desire to dominate or exert power over others. Rather, they simply appreciate the benefits of not depending on others. Further, in their pursuit of self-advancement, they impose limits \& boundaries upon themselves, taking care not to impinge upon or violate the rights of others. Sometimes, this is out of a neurotic sense of undue guilt or shame for doing otherwise. Sometimes the motivations are more altruistic. Most of the times, the motivation is purely pragmatic because the last thing the independent personality wants is to be at others' mercy (which they would be if caught \& sanctioned for injurious acts to others). On balance, the assertive personality is arguably among the most psychologically healthy of all personality types.'' -- \cite[p. 70]{Simon2011}

\subsubsection{Predominately Character-Disordered Personalities}
``These personality types tend to lie much further toward the character-disturbed as opposed to neurotic end of the spectrum. So it's appropriate to examine them with greater depths than the other types we have discussed. Although instances occur in which their dysfunctional interpersonal ``style'' outwardly manifest underlying neurosis, for the most part these personalities are not the way they are primarily because of their unconscious fears, insecurities, \& defenses. Rather it's mostly because of their conscious yet dysfunctional \& irresponsible choices, as well as their innate predispositions about how to view the world \& interact with it. While the various character-disordered personality types have some unique characteristics, they also have some features in common. Those are:
\begin{itemize}
	\item \textbf{Problematic Attitudes \& Thinking Patterns.} Predominantly character-disturbed personalities tend to think in ways that impair healthy interpersonal relations. Sometimes, their thinking reflects deeply-rooted but erroneous \& dysfunctional beliefs about the nature of the world, their place in society, \& requirements for conducting healthy human relationships. Such erroneous core beliefs \& thinking patterns foster markedly antisocial attitudes. These in turn predispose them to engage in some of the most problematic social behaviors. At other times, the problematic ways they think reflect their persistent disdain for the truth (more about this later), their refusal to reckon honestly with themselves, \& their resistance to conform their thinking (as well as their behavior) to society's expectations.
	\item \textbf{Problematic sense of regard for self \& others.} Whether they are inordinately egocentric, harbor attitudes of entitlement, or have compete disdain for others, disturbed characters possess a distorted \& dysfunctional sense of both their own worth \& other people's value \& dignity. Because of this, their relationships tend to be shallow \& superficial at best, \& abusive \& exploitative at worst.
	\item \textbf{Disregard for the Truth.} Disturbed characters often ignore the reality of circumstances. They act in indifference to the truth about themselves \& their behaviors. Some engage in such expansive \& unbridled fantasy that truth for them is only what they imagine it to be. Others are at outright war with the truth. They know the truth, but because it might challenge their core beliefs or interfere with their various agendas, they refuse to acknowledge or accept it. Their disregard for the truth predispose the most disturbed characters to lie to themselves as well as others. To a great degree, it also predisposes them to engage in deceitful impression management (more about this later). The truth could potentially ``level the playing field'' in their relations with others, exposing character features that might give others pause. The more disordered characters so disregard truth -- with a penchant for lying so deeply ingrained \& habitual -- that they lie even when telling the truth would have no perceivable adverse consequence (i.e. the truth would do just fine).
	\item \textbf{Responsibility-Resistance Behaviors \& Manipulation Tactics.} The more character-disturbed personalities frequently engage in fairly ``automatic'' behaviors that, by their nature, obstruct internalization of pro-social values, principles, \& inner controls. These same behaviors often serve as ``tactics'' to help the disturbed character manipulate \& gain advantage over others. The tactics also reinforce the disturbed character's attempts to manage the impressions of others. We'll be exploring these responsibility-resistance behaviors \& power tactics in substantial detail.
	\item \textbf{Impression Management.} Some disturbed characters frequently engage in managing the impression others might form of them. For some, it's a matter of keeping an unrealistically inflated self-image. For others, it's more a matter of keeping others in the dark about the kind of person they're dealing with. Without exception, however, impression management primarily serves to help the disordered character maintain a position of advantage over others. If you don't really know exactly who you're dealing with, \& what their real intentions are, in their relationship with you, you're at a distinct disadvantage \& ripe for their exploitation.
	\item \textbf{Impaired Capacity for Empathy \& Contrition.} There is a big difference between regretting adverse consequences to oneself for bad behavior (e.g., getting ``caught,'' paying fines, receiving other social sanctions), \& experiencing genuine, empathy-based remorse for injury caused to others. For a person to experience any degree of genuine ``contrition,'' prompting them to change their ways, 2 things must occur: (1) they not only have to feel genuinely bad about what they have done (i.e. guilty), but (2) they must also be internally unnerved about the kind of persons they've become (i.e. shameful) through acting so irresponsibly. Their shame \& guilt can propel them to make amends to the best of their ability, to work very hard not to engage in the same misconduct again, \& to make themselves better persons. True contrition also always involves what the Greek philosophers termed ``metanoia'' or a ``change of heart.'' \& a true change of heart always involves correcting prior dysfunctional beliefs, attitudes, \& ways of doing things.

	Disturbed characters will often scream loudly how sorry they are; but their behavior patterns rarely reflect any genuine remorse. Sure, they can feel sorry for themselves, especially when they're caught, \& have to pay the occasional very high price for a serious misdeed. They will often protest something akin to: ``I have to live with my mistake every day,'' inviting you to pity them, view them as a victim in some way, \& imagine that they must be experiencing all sorts of inner emotional pain. Meanwhile, their tightly-held attitudes, thinking \& behavior patterns give few indications that they either appreciate or feel badly about the impact of their irresponsible conduct on others.
	
	3 real-life examples of the most disordered characters' deficient capacity for genuine empathy-based remorse remain seared into my mind. All 3 examples come from counseling sessions with sexual offenders:
	\begin{enumerate}
		\item A young man who had sexually molested his younger sister finally decided he wanted to talk about the abuse. As he began to describe in detail the things he had done to force his victim's compliance, tears began rolling down his cheeks. As a young \& naive therapist, I prepared myself for what I anticipated would be an emotional flood of regret \& remorse for the pain he had inflicted. My co-therapist (herself also crying) wanted to interrupt the session, assuming the same he was undoubtedly feeling might be ``too much'' for him to bear. But what eventually became clear was that he was feeling sorry only for himself. You see, it turned out that his tears were all about this fact: His victim had put up much less overt resistance to other (older \& physically stronger) members of her extended family who had also abused her. The resistance she put up invited him to feel like he wasn't as ``worthy'' as the others, \& this made him feel rejected \& inferior. He didn't even regret the intense brutality he used to get her to comply, but only that her ``rejection'' of him drove him to take her by such force, \& made him ``look like a bad guy.'' He admitted he was still angry with her for this.
		\item A middle-aged man convicted of participating in a brutal gang-rape sobbed uncontrollably when telling me about the crime. When talking about the source of his pain, he revealed he still believed he was the least culpable of his accomplices. He only ``went along with'' the caper whereas others had done the initial planning. He further complained that he was the last to assault his victim, barely penetrating her at the very moment police burst in on the scene, so he ```didn't even get a chance to bust a nut.'' Not only did this upset him, but he also cast himself as a ``victim'' of corrupt system: The court had ordered him to serve the same amount of time as his comrades.
		\item A white-collar rapist used date-rape drugs to facilitate his victimizations of several women. He expressed outrage at the fact that I didn't appear to believe he was sorry for what he had done. He protested that ``not a day goes by'' that he doesn't have to ``live with the consequences'' of what he did. His later enumeration of these consequences centered mostly on 3 losses: his freedom, his very profitable business, \& his stature in the community. He never expressed any appreciation for the nature of the injury he inflicted on his victims, or 1 iota of genuine sorrow for the pain he inflicted on them \& their families.
	\end{enumerate}
	Now, we can spend a lot of time examining all of the disturbingly pathological thinking at work here, but I'm trying to make 2 main points with these illustrations. 1st, not everything is as it outwardly appears. We err greatly when we assume that words of regret or even crocodile tears are necessarily prompted by genuine remorse. Some characters are so deeply disturbed that they can even feign remorse. So, when it comes to disturbed characters, you must be careful not to assume anything. \& we have to be particularly careful about traditional assumptions we've tended to make (most promoted by classical psychology paradigms) about the kind of wounded soul we've long believed must lie beneath disturbed characters. 2nd, a huge difference exists between the pain of self-pity \& genuine, empathy-based contrition. In the end, actions speak louder than words or even emotional expressions. It's so easy to say you're sorry. It's another thing to act like you're sorry \& be willing to make amends. All of us have transgressed in 1 way or another. But when people have true contrition, their greatest pain is for the injury they caused someone else; \& their actions reflect a sincere effort, not only to repair the damage, but also to change their ways. So, when people show some sign of emotion related to a terrible event, it's wrong to jump to the conclusion that they must necessarily be experiencing genuine remorse or empathy for the injury caused another. It's also important to remember that, in the case of the most severely disordered characters, the very capacity for empathy is non-existent.
	\item \textbf{Problematic temperament, mood, \& disposition.} Generally speaking, disturbed characters have abnormal \& problematic aspects of their temperament, mood, \& general disposition. These play key roles in their dysfunctional interpersonal styles. The more disordered characters possess irascible temperaments, low frustration tolerance, \& high reactivity, making them ``walking time bomb.'' Others are so emotionally labile that living with them is like riding a roller coaster of passions \& sentiments. Still others are so disagreeable that working or living with them is often an exasperating experience. But all of the most disordered characters have aspects of their temperament, mood, \& general disposition that not only contribute to the habitual ways they tend to interact, but also to the difficulty they have maintaining stable, healthy relationships.
	\item \textbf{Deficient Impulse Control.} Disturbed characters notoriously lack self-control. They act without thinking or with indifference to the potential consequences. They do things that hurt people \& (sometimes) feel bad about it afterward. They don't stop to think 1st how their actions might impact others or what consequences might occur. They have a deficient capacity to delay gratification \& lack internal ``brakes.'' Sometimes, their aggressive predispositions are so strong that they overwhelm their weak braking system \& cause a multitude of problems.
	\item \textbf{Failure to Suitably Profit From Experience.} Disturbed characters are not incapable of learning, but they frequently fail to learn what most of us hope they might from their experiences. I.e., most of us hope they would stop \& reconsider their ways of doing things, since those ways seem to invite so many problems. Instead, disturbed characters not only persist in their ways, but often solidify or even intensify them. This occurs despite ample, obvious evidence that their manner of doing things is tragically flawed.
	\item \textbf{Impaired Conscience.} Disturbed characters' consciences are not sufficiently developed to either ``push'' them to do what they should, or ``restrain'' them from doing what they shouldn't. In some of the most seriously disordered characters, conscience is virtually non-existent or absent altogether.
\end{itemize}
Although the various disturbed characters have traits in common, certain traits cluster together in some very distinct ways. This allows us to differentiate some major disturbed character types. Sometimes, the distinctions can become a little blurred when a particular disturbed character shares more than a few of the traits. But it's nonetheless helpful to categorize these personalities. So let's call attention to the primary traits they possess that cause problems in their interpersonal relations.'' -- \cite[pp. 70--77]{Simon2011}

\subsubsection{The Egotistic Pattern}
``Some individuals simply cannot see themselves as anything else but the very center of the universe. Preoccupied with their own desires, concerns, \& image, their self-focus makes it difficult to give attention to or even recognize others' rights, needs, or concerns. They harbor a completely unrealistic sense of self-worth. These egotistic or narcissistic characters see themselves as inherently superior to others, \& believe they rightfully enjoy ``special'' status \& privilege. As a result, they easily come to feel \textit{entitled} to things they want \& to do whatever they wish. After all, in their minds, others \& their needs don't really count. Only they -- \& what they desire -- matters.

There are some prevailing notions about the underlying dynamics associated with this personality type. Many of these notions are not as well-founded on traditional principles as they purport to be, \& some are totally without foundation. The most popular prevailing notion: The ego-inflation that characterizes narcissism is a ``compensation'' for underlying feelings of low self-worth. This idea is generally attributed to Alfred Adler,\footnote{Adler, A., \textit{Understanding Human Nature}, (Fawcett World Library, 1954).} whose individual psychology was heavily dominated by this theory: People are in natural competition with one another for social status, \& compensate for their natural deficiencies by various mechanisms which produce ``complexes.''

I have come across a few narcissistic personalities whose apparent ego-inflation was rooted in an unresolved neurosis. For these few -- \& I do mean \textit{few} -- their displays of grandiosity \& haughtiness represented a true false self-presentation or fa\c{c}ade, masking deep feelings of inferiority \& fears of rejection. 1 theorist\footnote{Millon, T., \textit{Personality Disorders in Modern Life}, (Wiley, 2000, p. 278).} categorizes this type of individual as a narcissistic personality subtype: the \textit{compensatory} narcissist. So, neurotic narcissists do indeed exist. But \textit{the vast majority of egotistic individuals I've counseled over the years have been far more character disturbed than neurotic}. As such these individuals have displayed a sincere \& deep conviction about their superiority to others, whether or not such a belief is based on any kind of rational or solid foundation. They're not compensating for anything. They \textbf{\textit{really do think}} ``they're all that!'' Stanton Samenow has written about this\footnote{Samenow, S., \textit{Inside The Criminal Mind}, (Random House, 1984).} \& sometimes described these individuals as ``legends in their own minds.'' In fact, the shaky foundation sometimes seen for their inflated opinion of self is precisely \textit{because} they chronically overestimate their power \& worth. So, it's not that they start out with a weak foundation \& compensate for it. Rather, they spend their lives constructing the ``house'' of their self-image ``out of a deck of cards,'' unrealistically assessed regarding their strength \& integrity.

The histories of character-disturbed \& disordered narcissists are often quite different from those of neurotics, too. Sometimes they had good reason to believe they were the most powerful, important or capable persons in their immediate environment. Their mothers may have doted on them fro day 1. Their fathers may have abandoned them or abdicated all responsibility early on. Their principal caretakers might have been so inadequate that they were led to feel like the ``masters'' of their home. They may have been, in fact, the most functional \& capable person in their extended family system, with every reason to feel special or superior. If they were blessed with abundant natural gifts (e.g., physical attractiveness, technical or artistic talent, high intellectual capacity, etc.) -- \& \textit{most especially if they received much praise, adulation, \& social reinforcement from others \textbf{simply because they possessed these gifts}} -- they might have easily come by the notion that they were indeed ``special'' individuals, \& by right should have the world by the tail.

The narcissistic character is defined by the following traits:
\begin{itemize}
	\item \textbf{Inflated self-image \& sense of self-worth.} The narcissistic character sees himself as ``special'' \& more important than others. The neurotic narcissist might ``compensate'' for an impaired sense of self-worth, \& a ``fear'' of rejection for being anything less than perfect, by putting forth a false pretentiousness. But the narcissistic disturbed character genuinely believes (sometimes despite ample evidence to the contrary) that he is a most unique \& \textit{superior} creature. In 1 of the most seriously disordered character types (the psychopath), this trait is magnified to a chillingly pathological level (more about this later).
	
	The narcissistic character's self-appraisal is almost always out of whack with reality. This doesn't mean the individual's necessarily compensating for feelings of inferiority. It simply means the person's opinion of self often exceeds the situation's objective reality. Sometimes, these individuals possess very few genuine talents \& few remarkable accomplishments in their histories. Folks can be strongly tempted to erroneously presume that their inflated self-perceptions are a form of compensation. But many times they've been blessed with positives that contribute to very strong self-esteem or appraisal (e.g., talent, looks, brains, etc.). Sometimes, they've also experienced a fair degree of occupational \& vocational success. Still, it's not unusual for them to overplay \& overstate their accomplishments, \& overvalue their own worth.
	
	Sometimes professionals working with narcissistic characters -- whose histories of accomplishment are relatively lacking -- simply assume that their ego inflation must be a compensation for underlying low self-esteem. But they fail to consider this: There is a very significant differences between \textit{self-esteem} \& \textit{self-respect}. Sometimes these terms are used interchangeably. However, I find it crucial to distinguish between these very important concepts, especially because issues related to a healthy balance of self-esteem \& self-respect play pivotal roles in shaping several character types.
	
	Self-esteem literally means to estimate worth. It arises out of a person's intuitive assessment of what he has going for himself in the way of talents, abilities, etc. Self-respect, on the other hand, comes from the Latin \textit{respectere} (spectere being the root of the word spectacles) which means to look back. It arises from a retrospective assessment a person makes about what he has done with the gifts he has been given. In healthy societies, the most favorable retrospective assessments belong to those who have made meritorious efforts \& attained achievements that benefit the common good.
	
	All disturbed characters, but most especially narcissists, chronically overvalue \& claim ``ownership'' of the desirable but accidental attributes (i.e. gifts of nature or blessings of God) that foster a sense of high self-esteem. What's more, many times they get reinforcing messages from others like: ``You're so smart,'' or ``You're so talented.'' In short, they both receive \& are readily willing to claim credit for things they can't genuinely attribute to their own doing. They know what they have going for themselves, \& they equate their endowments with their identity. This inflates their egos.
	
	Narcissists \& the other disturbed characters often lack self-respect. That's because they know that, with their gifts, they haven't done enough good for others to merit a positive appraisal of social worth. In short, they lack respect because they haven't earned it. Many societies \& cultures inadequately recognize \& reward \textit{meritorious} conduct. Even some of our major religions \& philosophical schools of thought unwittingly downplay the value or even existence of human merit. Merit has to do with the manner in which a human being exercises the ultimate human power, the power to choose. Human beings are endowed with a free will. Making the meritorious choice is never easy, yet it is the essence of character. When a soldier enters a minefield knowing full well he could die, but seeks to rescue a fallen comrade, he commits a meritorious act. A father \& husband turns down a flagrant offer to engage in a tryst with an attractive co-worker. He does so out of concern for the solidity of his material commitment, the stability of his family, \& the welfare of his children. He performs an equally meritorious act. Doing the right thing is never easy. The problem: Within modern culture, \& even within major schools of religious \& philosophical belief, the value placed on such conduct is minimal. Often we expect good people to do right. Teachers \& parents rarely ``catch'' \& reward children for making the right choices, but we're quick to chastise when they choose wrongly. Jesus said: ``Render to Caesar the things that are Caesar's \& to God the things that are God's.'' If we ever want to reverse our cultural nightmare of inflated self-esteem \& deficient self-respect, we're going to have to do much better. We need to stop reinforcing people for the things which only God (or nature, if you will) can rightfully claim credit, such as their looks, their brains, their talents \& abilities; \& we must reinforce people for the truly meritorious, principled exercising of their wills, \& their willingness to subordinate their baser instincts in the service of the common good.
	
	Sometimes professionals, as well as others, are acutely aware of the disturbed character's deficiencies of self-respect. However, they often then confuse this with ``underlying feelings of low self-worth'' that they presume are being ``compensated for'' with the disturbed character's displays of grandiosity. They need to be more mindful of the distinction between self-esteem \& self-respect. A person can have too much of one \& not enough of the other. They also need to be more acutely aware of the factors that contribute to both, so they can help the person achieve a better balance. Most importantly, they need to avoid automatic presumptions about ``compensations'' that neurotic personalities engage in when they're dealing with disturbed characters.
	
	Narcissists' lack of humility w.r.t. both their natural endowments \& their success resulting from their endowments, as well as good fortune, inflates their sense of self-worth. They don't have room in their heart to acknowledge the roles of any outside factors or a higher power. They don't consider their native talents \& abilities as gifts, but rather as their defining characteristics. Humble, religious persons attribute their talents \& abilities to God or ``grace.'' Humble, non-religious persons attribute those things to a fortuitous accident of nature. The humble person also recognizes that all the raw talent in the world can't guarantee success. That, too, is dependent upon the ``grace'' of God, a certain amount of ``luck,'' opportunity, personal effort, \& working with others. Narcissists think that every blessing in their lives is the logical result of their own greatness. Even when they engage in acts of philanthropy or ``giving back'' to the community, it's generally done with a lot of fanfare. They want to further aggrandize themselves \& receive adulation. They lack the humble perspective that might foster the creed: ``To whom much is given, much is expected.''
	\item \textbf{Attitudes of entitlement.} Narcissistic characters' belief that they are \textit{entitled} to have the things they want derives from their credence that they are special \& superior individuals. Preoccupied with fantasies of unbridled success \& prestige, they believe they deserve by natural \textit{right} all the valuable things in life that most of us have to \textit{earn}. They think that, because of who they are \& their natural endowments, the world \textit{owes} them; \& they expect to be showered with recognition, adulation, \& reward. Because they believe the world owes them everything in the 1st place, they easily justify taking what they want, without feeling obligated to really pay for it through some kind of pro-social labor. They believe they have a right to anything they want simply because \textit{they} want it, \& their special nature \& status entitle them to it.
	\item \textbf{No concept of a higher power.} Narcissistic characters can't conceive of anything more important, more capable, or more potent than themselves. This leaves no room in their hearts for any concept of a higher power or authority. Now, I'm not necessarily talking about the concept of a God in a Judeo-Christian sense. As adherents to A.A. precepts know, there are many ways to conceive of a higher power. For the humanist, one needs to respect the collective ``greater good'' of society. For the non-deist scientist it might be recognizing the nature \& complexity of the physical universe that leads to humbly identifying oneself as a relatively insignificant character in a grand cosmic drama. Most of us have a deep, abiding sens that there is \textit{something} bigger than us.
	
	In the social world, most of us both recognize \& feel the need at times to pay deference to authority figures. Most of us recognize that there are people who know more than we do, possess competencies we don't, \& have been entrusted with decision-making authority over us. Narcissists have a hard time with this. They might \textit{feign} paying some deference for practical reasons, but in their heart of hearts they can't make themselves believe that anyone is in any way their superior. Lacking the humility to see themselves as inferior to anything or anyone, they don't recognize that any powers or entities exist; nor that these powers not only play a role in their success, but also place demands on them for gratitude \& a felt obligation to give back. They might give lip-service to the notion of a higher power, but they almost always leave its consideration absent from their hearts when they reflect on their lives \& their successes.
	
	As was discussed earlier, true core beliefs naturally beget certain kinds of actions. It's impossible to really believe in a higher power, subordinate your will to that power, \& then act in exploitive \& entitled ways towards others. For most of us, a belief in something that is not only greater than us, but also puts ethical demands on us, keeps us morally grounded.
	\item \textbf{Expansive fantasy.} The narcissistic character sees no reason to place limits on his abilities, or even his thoughts or desires. For him, imagining that he is something, or that he can have or do something, is akin to making it so. He finds no limits on what he can conceive or do. Similarly, he passively disregards the limitations of objective reality or truth. Reality is what the narcissistic character says it is. He is not influenced or swayed by others' judgments or opinions. The more character disturbed he is, the more he does not seek consensual validation. Rather, he finds validation in his own beliefs. As Samenow has commented, such individuals harbor the sincere belief that ``thinking makes it so.'' If they want something to be, in their mind, it is.
	
	These days, it's fairly common for some clinicians to attribute this characteristic of narcissists to a tendency toward bipolar disorder, hypomania, or mania. They then attempt to validate this view by pointing out that sometimes giving such an individual mood stabilizing medication will dampen the tendency to engage in this unrealistic thinking. However, this could also be argued: The tendency of narcissists to repeatedly engage in their unbridled fantasies \& passive disregard for objective reality might put them at increased risk for eventually developing more serious conditions such as hypomania, mania, or a bipolar disturbance.
	
	Now, I know that what I just said is likely to be quite controversial. But I don't think it's a notion that should be summarily dismissed. Ample research suggests that the link between biochemistry \& behavior is not just a 1-way street. We know, e.g., that a person who repetitively eats foods with a high glycemic index puts himself at increased risk for developing type II diabetes. We've also long known that strictly behavioral \& cognitive-behavioral treatments for obsessive-compulsive disorder result in the same types of biochemical changes in the brain's same areas as do purely medication-based treatments. Now, I'm not insinuating that t5rue Bipolar Disorder isn't a genuine condition in its own right with its own etiological factors. It can occur at times without warning in almost any personality type. What I am saying, however, is that certain personalities appear to have developed a style of relating that includes habitual ego-inflation. It's entirely possible (\& should be subjected to empirical test) that, when the pattern goes unchecked, it can eventually provide a segue into a much more serious condition.
	\item \textbf{Pathological desire for adulation \& admiration.} A narcissist can never get people to fawn over him or praise him enough. This character thrives on people being enamored with him. It's not so much that he values their opinions or that he \textit{depends} on them for a solid sense of self. Rather, he seeks to use others to fuel his massive appetite for self-aggrandizement by manipulating their attention \& praise.
	
	I had a conversation with a college classmate in which he told me of a disappointing sexual experience with a woman he had dated. Sexual conquest was a very big interest for this individual. Here was his main complaint about the particular experience he shared with me: He couldn't really enjoy himself because the woman didn't seem to be ``really getting off on'' being in bed with him. To add insult to narcissistic injury, she appeared to be merely using him to gratify herself. She was, in fact, very beautiful; but even an encounter with such an attractive person didn't really excite him. He wanted her to be enthralled with \textit{him. That} would have excited him. So, despite her stunning beauty, he couldn't perform. The frequency of his sexual exploits, the shallowness of those liaisons, \& the nature of what he was really looking for in those encounters -- these revealed just about everything one needed to know to understand his character.
	\item \textbf{\textit{Passive disregard} for the rights, needs, or concerns of others.} No one else really matters. It's all about the narcissist. It is a truly malignant egocentrism that makes it virtually impossible for the narcissistic character to develop or maintain genuine empathy. He is often completely oblivious to the emotional injury he inflicts on others. He doesn't set out to do harm, but so lacks conscientiousness about anyone else's welfare that he doesn't stop to consider the potential impact of what he says or does. The narcissist is so self-centered \& so self-absorbed, he doesn't really recognize others as independent entities with their own, wants, needs, desires, \& concerns. Sometimes he views others merely as extensions of himself. He can also view them as objects to bring him pleasure. He doesn't view them as persons of value in themselves to be respected or cherished.
\end{itemize}
The core characteristics of the Narcissistic Personality can be summarized as follows:
\begin{itemize}
	\item Unrealistic (grandiose) sense of self-worth.
	\item Preoccupation with power, brilliance, or appearance.
	\item Excessive desire for admiration.
	\item Impaired empathy \& regard for others.
	\item Excessive self-focus.
	\item Sense of entitlement.
	\item Oblivious to the wants or need of others -- exploitive.
	\item Haughtiness, arrogance toward those regarded as inferior.	
\end{itemize}
Possible Constitutional Factors:
\begin{itemize}
	\item Imperturbability. Doesn't get shaken by adverse events or challenges to perceived power \& greatness.
	\item Active imagination \& penchant for fantasy. Not constrained by the demands or confines of reality.
	\item These individuals appear to be moderately responsive to external praise but highly unresponsive to external condemnation.
\end{itemize}
Possible Learning Factors:
\begin{itemize}
	\item Received too much unconditional positive regard from others.
	\item Got too much praise for what they accidentally \textit{are} (e.g., talented, smart, physically attractive) \& not enough contingent attention or reinforcement for what they voluntarily \textit{did} to earn respect or admiration.
	\item Raised in an environment in which they were arguably the most powerful, capable, or reliable procurer of their wants \& needs.'' -- \cite[pp. 77--86]{Simon2011}
\end{itemize}

%------------------------------------------------------------------------------%

\section{The Aggressive Pattern}
``Some individuals are inherently warriors. They have a ``me against the world'' mentality, \& often pit themselves against others \& society's major rules \& authority structures. They don't passively disregard or place themselves above the rules like narcissists do; they actively challenge \& defy them. They have a pathological level of disgust for \textit{submissive} behavior of any kind. It riles them to think of themselves as weak or powerless, having to acquiesce or \textit{subordinate} themselves \& their wills to a higher authority. They strive to be ``on top'' \& ``in control'' (i.e. in a position of \textit{dominance}) at all times. They want to define the rules \& call the shots. They're also willing to do whatever it takes to satisfy their desires, even step all over the rights, boundaries, \& feelings of others. These are \textit{the aggressive personalities}. We'll soon be examining several subtypes of these individuals in greater detail. But before taking an in-depth look at the various aggressive personality subtypes, it will be necessary to introduce some concepts \& define some terms, especially with regard to the nature of aggression in humans.'' -- \cite[p. 87]{Simon2011}

\subsection{The Nature of Human Aggression}
``Aggression in human beings is not synonymous with violence. Human aggression is the forceful energy we all expend to survive, prosper, \& secure the things we want or needs. We reflect a deep-seated awareness of this fact in ur linguistics: We say things like, ``If you want something, you have to \textit{fight} for it.'' We encourage those who are sick or infirmed to rally their resources \& \textit{do battle} with their cancers, infections, other diseases. As a society, we launched a ``war on poverty.''

Humans have always done a lot of fighting. Although for years both mental health professionals \& lay persons have engaged in much denial about this most important fact, fighting is a huge part of life. It's fair to say that in the arena of human relations, when we're not making some kind of love, we're waging some kind of war (\& sometimes the distinction between the 2 activities is not all that clear). \textit{How} we fight is another matter entirely. The following diagram helps illustrate the difference between necessary, disciplined, \& potentially \textit{constructive, assertion}, \& undisciplined, \textit{destructive aggression}.'' -- \cite[pp. 87--88]{Simon2011}

\subsubsection{Assertion vs. Aggression}

\begin{table}[H]
	\centering
	\begin{tabular}{|p{.48\textwidth}|p{.48\textwidth}|}
		\hline
		\textbf{Assertive Behavior} & \textbf{Aggressive Behavior} \\
		\hline
		Fair fighting (i.e. without putting the other at a disadvantage). & Seeking unfair advantage -- attempting to victimize another. \\
		\hline
		Fighting for a legitimate purpose. & Fighting for a self-serving \& possibly immoral purpose. \\
		\hline
		Fighting disciplined by self-imposed limits designed to prevent undue harm to another. & Fighting without limits or with poor limits on what you're willing to say \&\texttt{/}or do. \\
		\hline
		Always non-violent. & Sometimes violent. \\
		\hline
		Fighting constructively (i.e. with the goal of improving a situation for all concerned). & Fighting in a destructive manner (i.e. in a manner that destroys opportunities to improve a situation for all concerned). \\
		\hline
	\end{tabular}
\end{table}
There are also several major subtypes of aggression. 2 of the more important subtypes, \textit{reactive} \& \textit{predatory} (some theorists prefer \textit{instrumental}) aggression are contrasted in the next diagram.'' -- \cite[p. 88]{Simon2011}

\subsubsection{Types of Aggression}

\begin{table}[H]
	\centering
	\begin{tabular}{|l|l|}
		\hline
		\textbf{Reactive} & \textbf{Predatory} \\
		\hline
		Spontaneous. & Premeditated, calculated. \\
		\hline
		Prompted by fear. & Prompted by \textit{desire} \\
		\hline
		Mostly \textit{defensive} character. &  Strictly \textit{offensive} character. \\
		\hline
		Goal is self-preservation. & Goal is \textit{victimization}. \\
		\hline
	\end{tabular}
\end{table}

\subsection{Reactive Aggression}
``All creatures display reactive aggression in response to a perceived threat to life or well-being. It's the kind of aggression a cat displays when it sits on the front porch \& witnesses a pit bull rounding the corner, slowly approaching, \& licking its chops. Frightened that the pit bull might be planning to make it his lunch, the cat is likely to engage in several characteristic behaviors. Its hair will stand on end. It may arch its back. It may hiss. It will brandish its claws. In a variety of ways \& in an obvious manner, it will signal its readiness to aggress. Does it want to aggress? No. Was its original intention to victimize? No. Is it primarily angry \& looking to pick a fight? No. In fact, it's primarily frightened. The last thing it wants to do is fight. Besides, it's no match for the pit bull \& would probably lose the contest. It might engage in some aggression-threatening behavior such as reaching out with its claw-displayed paws in a motion to scratch or slash. But it will only do so in response to threatening moves on the pit bull's part. All of its actions are of a strictly \textit{defense} character, \& its sincerest hope is that its potential adversary will change its mind \& move on. If the dog appears to be considering backing off, the cat will not antagonize it or otherwise provoke it into a more combative stance.

Reactive aggression is a purely \textit{spontaneous}, unplanned response to an unanticipated \& potentially serious threat to life \& well-being. It's prompted by \textit{fear}. That fear then triggers a very innate \& basic flight or fight response. The response is mostly \textit{defensive} in character, \& the purpose of any aggression-like behavior displayed is strictly to ward off or diminish the likelihood of any potential victimization.'' -- \cite[pp. 88--89]{Simon2011}

\subsection{Predatory (Instrumental) Aggression}
``Contrast the above scenario with that of another cat that spots a mouse in a room's corner \& fancies the rodent for lunch. It probably will not arch its back, but instead might put its belly low to the ground. Its hair does not stand on end out of fright. Instead, its hair lies normally. The cat is not terrified but both calm \& calculating. It won't hiss or make noise. \& it won't signal potential aggression by showing claws or making clawing motions. It will stead stealthily, \& as unobtrusively as possible, sneak up on the mouse \& pounce when reasonably sure of success. Its goal is not the avoidance of its own victimization but successful victimization of the mouse. It isn't praying that the mouse will react to signals \& go away. In fact, it wants the mouse to be right where it is, or in a hard-to-escape situation. Most importantly, it's not prompted to act because it's frightened. Its aggression is neither spontaneous nor based in fear, but rather carefully premeditated, prompted primarily by \textit{desire}. It's also \textit{not motivated by anger}. It's not mad at the mouse; it just wants to have it for lunch, an act that necessarily involves the mouse's destruction.

It's fairly common for mental health professionals \& lay persons alike to lack awareness about predatory aggression \& the many ways people can display it. Unfortunately, \textit{it's a common but erroneous belief that all aggression is always a defensive response to a perceived threat}. Some folks simply can't imagine why someone would aggress unless they felt threatened in some way. Other folks presume that people only aggress when they're angry. The presumption that anger is always the precipitant of aggression is the guiding philosophy behind many of the anger management programs so popular these days. Let's not go into all of the many misconceptions about aggression, \& especially the misconceptions about predatory or instrumental aggression. Instead, let's go back to the example of the cat \& the mouse. I'll ask some questions to examplify the absurdity of some of the most widely-held notions about why \& how humans aggress. Would you say that if the cat chose to eat the mouse, the cat must have felt ``threatened'' in some way? Would you say that the cat probably had ``anger issues,'' simply didn't know how to manage \& express its anger appropriately, \& ended up taking out its anger on the mouse? Would you say that the cat probably had a very traumatic history involving mice, probably in childhood, leading it to have ``trust issues'' with mice in general? That the cat anticipated maltreatment by mice, thus justifying a ``get them before they get you philosophy?'' Hopefully, by now you're having a bit of a laugh. What's not funny, however, is this: Often folks make assumptions very similar to the ones described above when trying to understand the motivations for human aggression, especially aggression of the predatory or instrumental variety.

Predatory aggression is motivated by desire. It's as simple as that. At 1 of my workshops, an attendee posed the question: ``Well, couldn't you still say that the cat was fearful that if it didn't eat the mouse it might have to go hungry? So, isn't it really still \textit{fear} after all -- the fear of not being able to satisfy this most basic need -- that motivates the cat? My reply was that one certainly \textit{could} frame things in that manner. \& such a perspective would once again reinforce the notion that fear is always the motivation for aggression. But what benefit is derived from such a perspective? Why stretch an inadequate metaphor to such an absurd degree that it finally fits when an alternative one better describes the situation's reality? Sometimes I think we get so married to our preferred ways of understanding the nature of the world around us that we simply cannot entertain a more sensible perspective. Reasonable scientists have long accepted the law of parsimony (also sometimes referred to as ``Occam's Razor''). The simplest adequate explanation of a phenomenon most often turns out to be the best explanation.'' -- \cite[pp. 89--91]{Simon2011}

\subsection{Other Major Types of Aggression}
``In addition to the reactive or predatory (instrumental) variety, there are some other major types of aggression:
\begin{itemize}
    \item Overt Aggression -- Open attempts to win, dominate or control.
    \item Covert Aggression -- Subtle or concealed attempts to win, dominate or control.
    \item Active Aggression -- Trying to get something you want by actively doing things \& employing tactics to victimize others.
    \item Passive Aggression -- Trying to avoid things you don't want by resisting cooperation with others.
\end{itemize}
People display \textit{overt} aggression when their bid for dominance is open \& obvious. A prize fighter in the ring attempting to knock out an opponent is displaying overt aggression. So is a corporate CEO when he openly lays out a plan to ``destroy'' the competition \& emerge as the dominant player in a particular sector of the economy. Aggression can also be \textit{covert}. I.e., a person can do his or her best to conceal any aggressive intent toward another. Keeping aggression under cover serves a dual purpose: (1) possibly being more successful in the aggressive quest (by catching the other person unaware); (2) effectively managing others' impressions \& preserving a positive image (thus keeping others from rightly discerning the aggressor's character). Besides, persons who intuitively sense the aggression, but can't objectively verify it, are quite prone to being manipulated \& controlled.

Aggression can also be \textit{active} or \textit{passive}. People \textit{actively} aggress when they \textit{do} some things deliberately to get the better of another, or to forcefully take the things they want. Conversely, they aggress \textit{passively} when they won't do things that others want them to do. Persons using the technique of a sit-down strike to protest against injustice employ passive aggression or resistance. So does a partner in a relationship when he or she won't answer, pouts, or resists cooperation with the other partner. There are still more types of aggression, several of which will become apparent as we discuss the various attributes of the aggressive personality subtypes.'' -- \cite[pp. 91--92]{Simon2011}

\subsection{Understanding the Aggressive Personality ``Styles''}
``As mentioned earlier, some personalities tend to view the world as a combat stage. Every situation they encounter is a contest in which they must emerge as the victor. We best describe such personalities as fundamentally \textit{aggressive} in their styles of interpersonal interaction. They are among the most disturbed in character of all the various personality types. Some aggressive characters are so wanton in their disregard for society's dictates that they frequently run afoul of the law, engage in criminal conduct, \& spend a fair portion of their lives incarcerated. These ``criminal personalities''\footnote{Yochelson, S., Samenow, S., \textit{The Criminal Personality, Vol. I: A Profile for Change}, Aronson, 1976).} have been the subject of much study over the years. They have often been given the label ``antisocial personality.'' But it's important to recognize that many other aggressive personalities exist other than the career criminal. Most of us have encountered 1 or more of these types in our daily lives. Extremely difficult to deal with, they are among the most character-disordered personalities you will ever encounter.

The various aggressive personality subtypes have many characteristics in common. They all display a pervasive style of doing battle with others \& the world at large. All the aggressive personalities also share traits common to the narcissistic personality. Some theorists tend to view the aggressive personalities as merely an aggressive variant of the narcissistic personality. What's more, 1 of the aggressive personalities is principally defined by the term \textit{malignant narcissism} (all the attributes of the narcissist carried to the most pathological extreme.) The principal distinguishing characteristic of the aggressive personalities, however, is not their narcissism. It's their penchant for aggression.

The various aggressive personality sub-types have more in common than they have differences. But there are several major aggressive personality subtypes, each defined by some fairly unique traits. Before we take a look at the major subtypes, let's outline what the aggressive personalities have in common:

All of the aggressive personalities:
\begin{itemize}
    \item \textbf{Actively seek the superior or dominant position in any relationship or interpersonal encounter.} There's a saying in the real estate business that 3 things really matter: location, location, \& $\ldots$ location. With aggressive personalities, 3 things really matter: position, position, \&, of course, position! The determination to seek this position cuts across a wide variety of situations \& circumstances.
    
    Primary interpersonal agenda for aggressive \& other character-disordered personalities:
    \begin{enumerate}
        \item Position (seek the upper hand\texttt{/}advantage\texttt{/}superior location).
        \item Position (seek to dominate, control, or win).
        \item Position (doesn't recognize or actively resists submission\texttt{/}subordination to a higher power.)
    \end{enumerate}
    Aggressive personalities strive for the dominant position at all times \& in all circumstances. This premise is very hard for the average person \& especially the neurotic individual to understand, let alone accept. It's incomprehensible for most of us to conceive that in every situation, every encounter, every engagement, the aggressive personality is predisposed to jockey with us for the superior position, even in situations with no recognizable need to do so. The failure to understand \& accept this, however, is how aggressive personalities so often succeed in their quest to gain advantage over others.
    
    More than once I've attended professional workshops in which the presenter advised that a therapist should refrain from asking probing, intimidating, or challenging questions. Why? Because that would necessarily put the client ``on the defensive'' \& prompt him to engage in acts of resistance, as opposed to allaying his ``anxieties,'' thus making it safe for him to ``open up.'' Of course, such concerns come from the notion that people simply won't fight unless they have to or feel threatened in some way. I used to observe the caveat when working with aggressive characters, until I realized it was unnecessary \& often even counterproductive. I eventually came to realize that, whether I wanted to admit it or not, aggressive personality clients had begun resisting any intervention efforts long before my 1st encounter with them. As mentioned before, neurotics troubled by their circumstances both seek \& appreciate therapeutic ``help.'' Disturbed characters are most often reluctantly pressured into the therapeutic process by an outside agent. They don't want any part of it from the beginning. They're happy with who they are \& their way of doing things. They might be superficially cordial, but make no mistake, if they're dragged into therapy, they plan from early on to fight any efforts encouraging them to change. So, the ``fight'' starts long before they 1st arrive at the office. If a therapist is not inclined to confront it directly \& set limits \& contingencies immediately, he or she is wasting the person's as well as his or her own time.
    
    My eyes were really opened to shortcomings of traditional therapeutic approaches as soon as I began confronting character-disturbed clients. 1stly, they seemed to show at least a superficial level of interest \& rudimentary respect for the challenge they knew they faced, trying to manipulate my impressions of them. 2ndly, they seemed to appreciate the cut-to-the-chase approach in which all the cards were on the table \& everything was out in the open. Lastly, I implicitly conveyed messages in my benign but firm confrontation of the issues, especially messages like: (a) ``It's my opinion that no matter what tactics you use to convince me of the contrary, you \textit{do} have a problem with your character''; (b) ``I have some techniques you can use to change the kind of person you are for the better''; \& (c) ``I'm going to waste time listening to you blame others, justify your way of doing things, or trivializing the wreck you've made of your life \& the lives of others; but I will only engage with you if you are at least to some degree willing to be open to guidance \& accepting of the need for change.'' These emerged as very powerful messages to send. Such messages led them eventually to conclude that: (1) I could not be manipulated; (2) I actually had principles (i.e. willingly subordinated myself to a higher authority) \& was willing to adhere to them; (3) I could be trusted (this is so critical because it takes away the most common excuse aggressive personalities give for their conduct). The main reason they eventually knew they could trust me is because I didn't \textit{need} them to need me or like me. I wasn't going to prostitute myself to the person or entity that referred them to me. I wasn't going to try \& manipulate these aggressive clients into tolerating me, or even eventually liking me a little through subtle seduction \& attempts to curry favor. \& I certainly wasn't going to ``frame'' their serious psychological issues in more palatable or politically correct terms to avoid confronting character concerns head-on.
    
    Many of my colleagues do not share my belief in focusing on the person himself, as opposed to solely the behaviors displayed. They think, of course, that confronting the person about his or her character instead of simply the behaviors is potentially damaging to self-esteem \& necessarily invites resistance. These critics would have a valid point if it were really true that \textit{everyone} struggles with low self-esteem like neurotics do, \& that people fight \& resist only when they feel attacked. But I have not found these assumptions to be valid. While disturbed characters' behaviors are problematic, equally if not more problematic is this sobering fact: They have chosen to define themselves as being comfortable with their negative behavior patterns, at peace with their antisocial attitudes, \& unashamed of their repeated abdications of responsibility. Not only that, but they also choose to define themselves as retaining an unwarrantedly high opinion of themselves (ego inflation) despite the big mess these patterns have created. So, who they \textit{are} -- not just what they do -- is a major issue needing attention \& confrontation. Their \textit{character} needs to be a center of focus. I also don't pretend that I'm their unconditional friend or ally. If they want my respect \& support, they'll soon learn that they earn it only by making some pretty hard choices w.r.t. changing their usual ways.
    \item \textbf{Abhorrence of submission to any entity viewed as a higher power or authority.} The aggression personalities are fundamentally at war with anything blocking unrestrained pursuit of their desires. Unfortunately, this often means society's rules, dictates, \& moral obligations, \& the expectations imposed by authority. Some aggressive personalities will accede to, or give \textit{assent} to, demands placed on them when it is expedient or self-serving to do so; but in their heart of hearts they never truly submit or subordinate themselves to a higher power or authority. It is anathema to think of themselves as under anyone else's influence, power, or control. So, they resist subordinating themselves. Their innate aversion to any kind of submissive behavior plays a critical role in their problems with forming consciences in their early development, \& profiting from experience (We'll take a closer look at this aspect of the aggressive character a bit later.)
    \item \textbf{Ruthless self-advancement, most often at others' expense.} The aggressive character is an unscrupulous competitor. He will lie, cheat, steal, ``con,'' manipulate, or do whatever he must to seek the position he wants over someone else. The end almost always justifies the means. Actively \& deliberately, aggressive personalities seek to exploit \& victimize others. Whereas the narcissist simply doesn't pay attention to others' rights or needs (because he doesn't consider them important), the aggressive character tramples their rights \& needs to satisfy his own desires. When the aggressive character wants something from you, he will take it by whatever means he finds necessary.
    \item \textbf{A pathological disdain \& disregard for truth.} Aggressive characters don't just disregard the truth, they're at war with it. Truth is the great equalizer, \& aggressive personalities always want to maintain a position of advantage. So, they deliberately play very fast \& loose with the truth when they're not flat out flying. They don't want you to ``have their number.'' That upsets the balance of power. So, they're usually about the business of conning \& duping you. \& because they want to have advantage over you, they often lie in subtle \& sophisticated ways, carefully managing your impression of them \& manipulating you through deception. Their lying is so pervasive \& automatic, they will lie even when the truth would do just fine; except lying keeps the con game going, which they perceive as maintaining the position of advantage. Also, the lying takes so many forms it's almost impossible to count them all. Nonetheless, we'll take a look at some of the major ways when we talk about manipulation tactics later on.
    \item \textbf{Defective internal ``brakes'' (lack of inhibitory controls).} Aggressive personalities are like a runaway train with no means to stop. When they're on a mission, they can't or won't put on the brakes, even when it's in the best interest of all to do so. They appear to have a constitutionally-based deficiency in their neurological inhibitory networks. But since they are also hardwired to fight so intensively \& frequently, installing such controls presents an almost insurmountable challenge during their development.
    
    When most of us are fed up with someone \& have the urge to punch their lights out, we exercise restraint, contemplating the possible negative consequences. In short, we think 1st \& act later, usually with some moderation. Because they lack inhibitory control, aggressive characters act 1st \& think or reflect later. They might even have some genuine after-the-fact regret (although the most severely disordered of these personalities lack regret or remorse); but such regret is usually too little \& too late in coming. Their lack of inhibitory control, as well as their overly aggressive predisposition, leads to their \textit{impaired ability to delay gratification}. They want what they want, \& they usually want it NOW.
    \item \textbf{Irascible Temperament.} Aggressive characters are often quick to react to any environmental event, \& their reactions are often quite intense. Someone will say something that most people would simply ignore, whereas they overreact. They have a remarkably low frustration tolerance, \& become easily upset when things don't go as they want them to go. As a result, in their development they cultivated little will to bear discomfort. It doesn't take much to set them off. They might become enraged at the slightest provocation. Say just the wrong thing, or look at them in the wrong way, \& you might find yourself facing their wrath.
    \item \textbf{Lack of Adaptive Fearfulness.} Although this characteristic is common to other disturbed characters, aggressive personalities are the most lacking in what researchers sometimes call adaptive fearfulness. They lack the squeamishness or apprehension a person should ideally have when contemplating doing something they probably shouldn't do. They neither weigh or fear the potential consequences. They frequently engage in unhealthy risk-taking, daredevil-type behavior, \& are notoriously sensation-seeking.
    
    When most folks experience a negative consequence of their behavior, they at least consider modifying that behavior. But aggressive personalities' imperturbability allows them to remain unshaken, \& therefore undeterred w.r.t. their destructive behavior patterns. In fact, adversity often prompts them to solidify their combative stance against the world. They tend to view adverse consequences as proof that the world is a cold, cruel place in which they must fight to survive. They tend to regard their tenacious facing of adversity as a noble character trait. They not only value it, but also sometimes try to intensify it with every adverse consequence they experience.'' -- \cite[pp. 93--99]{Simon2011}
\end{itemize}

\subsection{The World According to Aggressive Characters}
``Aggressive personalities believe every situation has only 4 possible outcomes:
\begin{enumerate}
    \item I win, you lose.
    \item You win, I lose.
    \item I win, you win.
    \item I lose, you lose.
\end{enumerate}
Naturally, they prefer the 1st possible outcome. They like it best when they win \& you lose. For them, this is the clearest indication they've emerged the victor in a contest, securing the dominant position. Contrarily, they abhor the notion that you might win \& they will lose. They resist this potential outcome with passion. It casts them in the inferior or subordinate position, which they detest. Aggressive personalities will reluctantly (although not usually graciously) accept win-win outcomes. I.e., they'll stop warning with you if they think they've secured at least some modicum of victory, even if you also end up with something you want. Tragically, if it becomes clear they're certainly headed for defeat, aggressive characters often won't go down easily. They want to take someone else with them, lessening the sting of defeat. Sadly, this scenario sometimes plays out in the tragic murder-suicides that occasionally make headlines.

To summarize, the following are the core characteristics of the aggressive personality:
\begin{itemize}
    \item Problems with authority \& societal expectations.
    \item Reckless trampling of the rights\texttt{/}needs of others.
    \item High-risk behaviors \& sensation-seeking.
    \item Problems with self-control \& delay of gratification.
    \item Frequent, sometimes flagrant, lying.
\end{itemize}
Possible Constitutional Factors:
\begin{itemize}
	\item High reactivity \& aggressive predisposition.
	\item Low anger threshold \& frustration tolerance.
	\item Deficient inhibitory capacity.
	\item Lack of adaptive fearfulness.
	\item Innate revulsion to submissive behavior modalities.
\end{itemize}
Possible Learning Factors:
\begin{itemize}
	\item \textit{Sometimes} raised in hostile, abusive, severely neglecting environments.
	\item Experienced excessive success (reinforcement) in securing goals aggressively.
	\item Insufficient experience with correctly administered punishment.
	\item Failed to learn the long-term benefits of short-term self-denial.'' -- \cite[pp. 99--100]{Simon2011}
\end{itemize}

\subsection{Traditional Paradigms' Failure to Accurately Understand \& Address Aggressive Personalities' Problems}
``Traditional thinking about antisocial characters has always been this: Abuse \& neglect in their histories led them to deeply mistrust the world \& others' motivations. As a result, Traditionalists thought disturbed characters found it too anxiety-evoking to ``bond'' emotionally with their primary caretakers as well as others. Within this model, aggressors' offensive postures are perceived to be an underlying \& rational ``defense'' against anticipated injury. They are seen as having adopted an ``I'll get you before you get me'' style of coping with life's challenges. Traditionalists have adhered to the notion that all human beings would naturally bond with others, behaving in pro-social ways unless severely traumatized \& conditioned to believe that others are untrustworthy. So traditional theories propose that, when they engage in their hostile acts, antisocial personalities are ``acting-out'' deeply unconscious conflicts about their safety. They perceive a cold, hostile, \& rejecting world. As a result, they reduce their internal pain \& anxiety with antisocial conduct, minimizing feelings of guilt or shame through unconscious mechanisms such as denial, projection, \& rationalization. Traditional beliefs about antisocial personalities have often been viewed as validated by the fact that such individuals frequently cast themselves as victims of mistreatment. They claim that they rightfully don't trust others, \& insist that their behavior was a matter of self-defense. Traditional perspectives have some value if, in fact, the aggressive personality is mostly neurotic. But this is actually fairly rare. The aggressive personalities are among the most frequently \& severely character-disordered individuals.

In my work over the years with disturbed characters, I have found that \textit{most} of the time, when aggressive personalities cast themselves as victims, they don't really believe they are the injured party. Rather, they want \textit{you} to believe that they think that way. They know full well that \textit{they} are the victimizers who injured others for no other reason than wanting to take something, regardless of the damage they inflicted on others. But if they can convince you that their actions weren't purely maliciously motivated, they can possibly evoke more sympathy from you, keeping you in the dark about their true character. Knowing what they're really all about would put them in a position of disadvantage with those they seek to manipulate (Remember: position, position, position!). A ``level playing field'' is something they always seek to avoid. So, they engage in deceitful games of impression management, working hard to have you see them as a victim in some way.

A few of the many aggressive personalities I've counseled over the years actually experienced significant abuse, neglect, \& trauma. But I've encountered many more who greatly exaggerated how badly they were victimized while simultaneously minimizing their history of victimizing others. I've also encountered many personalities whose childhood environments were unremarkable or as nurturing \& supportive as any -- sometimes even better than most. Yet, these individuals appear to have been ``fighters'' from the start, actively resisting the socialization process during their formative years, \& showing an inherent lack of empathy \& capacity to bond with others.

Any reasonable theory of personality formation simply has to take this into account: Countless individuals have experienced great hardship, abuse, \& trauma in their formative years; but they did not become aggressive personalities or any other type of disturbed characters. It's dangerous to make blanket assumptions about what must underlie the aggressive predispositions of these personalities. It's also dangerous to accept their stories without scrutiny, skepticism, \& objective verification. My experience with these individuals has shown me how problematic it can be to blindly accept traditional perspectives on what makes these disturbed individuals ``tick.''

In recent years, research has pointed to errors of many traditional assumptions about aggressive character formation. For many years, therapists generally accepted that ``bullies'' were in fact cowards who ``compensated'' for feelings of insecurity \& low self-esteem. They were seen as displaying bravado to mask feelings of inadequacy. They often picked on those weaker, which was regarded as ``proof'' that they inwardly felt incompetent. But some solid research has demonstrated that most bullies aren't cowards at all. They're not struggling with low self-esteem. They're merely brutes with outrageously inflated egos \& a huge sense of entitlement. We also now know that starting off as an egomaniacal brute early in life very strongly predicts all kinds of social maladjustment later in life, including possibly becoming a full-fledged antisocial personality.

Contrary to popular belief, substantial evidence shows that a history of trauma, especially abuse \& neglect, does not necessarily predispose a person toward antisocial personality characteristics. In fact, many individuals who lead antisocial lifestyles were raised in supportive \& nurturing environments. Evidence also exists that genetic factors play a significant role in whether someone will eventually display antisocial characteristics. Yet myths persist that such characters simply must have been subjected to abuse \& neglect, \& that their early trauma explains why they behave as they do.

Disturbed characters understand keenly what kinds of attitudes, beliefs, \& biases most people hold. They know full well that neurotics are quick to believe no one would engage in aggressive conduct unless they come from a disadvantaged or traumatic background. They often use this knowledge to play on others' sympathies, avoiding responsibility for their choices \& actions. But when one carefully examines their entire history \& seeks corroboration from reliable sources, here's what generally emerges: Antisocial characters (especially psychopaths) over-represent their degree of being victimized by others, \& under-represent how seriously they have intentionally \& wantonly victimized others. I usually treat their initial claims with some healthy skepticism, or possess contradictory documentation. Then such characters will often admit a much more lengthy history of doing horrible things to others. They'll also acknowledge that very few, if any, horrible things ever really happened to themselves. But I've also observed just the opposite in 2 situations: (1) when such individuals are questioned by a naive interviewer; or (2) when the disordered character senses the interviewer believes that people only do bad things because of traumas they've experienced in the past.

With all disordered characters, here's 1 very important thing to remember: Mounting evidence shows it isn't so much what bad events happened early on in their lives that shaped them. It's what \textit{didn't happen} w.r.t. the kinds of influences people need to adequately shape their characters. In some cases, the shaping influences were simply inadequate (i.e., discipline was absent, inconsistent, or improperly administered). Or the child's constitutional predispositions overwhelmed or stymied caretakers' attempts at providing \& promoting positive-shaping influences.'' -- \cite[pp. 100--104]{Simon2011}

\subsection{Major Aggressive Personality Subtypes}
``Both Simon\footnote{Simon, G., \textit{In Sheep's Clothing: Understanding \& Dealing with Manipulative People}, (A.J. Christopher \& Co., 1996, pp. 24--30).} \& Millon\footnote{Millon, T., \textit{Personality Disorders in Modern Life}, (Wiley, 2000, pp. 110--113).} suggest several major variations of the aggressive personality type exist. Although they share many traits in common, some important characteristics differentiate some of the aggressive personalities. As a result of my experience working with such individuals over the years, I find the patterns described below to represent the major aggressive personality subtypes:

\subsubsection{Unbridled Aggressive (Antisocial) Pattern}
This aggressive personality is best distinguished by frequent, wanton violations of major societal norms \& refusal to subordinate his will in service of the greater good. When the unbridled aggressive sees something he wants, he takes it, regardless of whether his actions run afoul of the law or otherwise transgress others' rights. Unbridled aggressive personalities often have a lengthy history of criminal conduct. Many have spent a considerable amount of their lives incarcerated (repeat incarcerations are also common).

The unbridled aggressive has often traditionally been labeled the \textit{antisocial} personality. As the term implies (literally, ``against society''), this personality makes himself the adversary of the prevailing social order, engaging in a perpetual contest with others \& the world at large, seeking personal gain no matter the cost or impact on others. Some individuals use the term ``antisocial'' inappropriately to describe the social aloofness \& loner status of shy individuals, as well as the avoidant \& asocial (schizoid) personalities. But such personality types differ radically in character from the unbridled aggressive. Unbridled aggressive personalities are not afraid to engage you; they're not at all shy or lacking in the normal human urge to interact with others (although they might at times deliberately act indifferent \& aloof). Make no mistake. These folks want to engage you, but usually to get something from you, taking advantage of you, defeating you, exerting power \& dominance over you, or inflicting injury upon you.

The core characteristics of the Unbridled Aggressive Pattern are:
\begin{itemize}
	\item Brazen defiance of major social norms of conduct.
	\item Frequent engagement in behaviors that could result in social sanction if detected.
	\item Frequent brushes with the law\texttt{/}history of criminal offenses\texttt{/}convictions.
	\item History of overtly aggressive \& sometimes violent conduct toward others.
	\item Persistence in an aggressive pattern of conduct despite adverse consequences.
	\item Emboldened in the aggressive behavior pattern when they manage to ``beat the system.''
	\item Frequent parasitic lifestyle.
\end{itemize}

\subsubsection{The Channeled-Aggressive Pattern}
Not all aggressive personalities are flagrant lawbreakers. 1 aggressive personality type generally refrains from allowing his overtly aggressive interpersonal style to lead to frequent \& brazen violations of the major rules. Channeled-Aggressive personalities place some limits on their obtrusive modus operandi. They generally confine their aggression to social pursuits in which others not only tolerate but often highly value the will to win at all costs. We've all encountered these tough-minded, callous, \textit{driven} people. They're determined to prosper, generally at someone else's expense. For them, all that matters is taking care of themselves. Stay out of their way, \& you might never have a problem with them. Get in their way, \& you're probably ``toast.'' Insensitivity, disregard for boundaries, extreme competitiveness, \& intolerance for weakness are their core characteristics. These personalities want to win at all costs, \& don't mind others knowing it. They display their aggressive tendencies openly \& proudly. They want all to know, see, \& respect their willingness to do whatever it takes to get what they want, \& to deal resolutely with those who would oppose them. They don't mind if others fear them or are intimidated by them. In fact, they might regard it a perverted indication of respect that they are a force with which to be contended. They are proud of their tenacity \& lack of apprehension when tackling the challenges of life. They rigidly adhere to the belief that the spoils of life's conflicts rightfully belong to those willing to fight for \& take them.

Several personality theorists have long equated aggressive personality disturbance with severe anti-sociality \& a history of criminal conduct. Indeed, the mental health community's official diagnostic manual pretty much defines the antisocial personality as a career criminal. Some researchers, however, have always recognized that only a certain sub-group of aggressive personalities can be defined by their habitual criminal lifestyles.\footnote{Millon, T., \textit{Personality Disorders in Modern Life}, (Wiley, 2000, pp. 107--108).} Channeled-aggressive personalities are very much like their antisocial (i.e. unbridled aggressive) counterparts, except they don't habitually lead lives of crime.

Channeled-aggressive personalities gravitate toward situations in which they can amass power, exert control, \& satisfy their insatiable appetites to dominate or win. They are often found in the ranks of law enforcement, military command, \& professional sports. They sometimes become heads of corporations, policy makers, administrators, \& leaders of various types. Some have the manipulative skill to favorably manage others' impressions, appearing less ruthless or even as team players. In reality, however, they are a team unto themselves, always looking out for number 1. Many times, they are respected for their ambition, drive, capability, \& tenacity.

Natural leaders, these personalities know how to get any job done. They are radically different from assertive personalities, however, because they don't pay much heed to how their actions might negatively affect others. \& they don't temper their actions to display care for others' rights, needs, boundaries, \& concerns. They are individuals who are hell to work with \& work for.

Channeled-aggressive personalities generally adhere to major social norms, but not out of a neurotic sense of guilt or shame, nor a truly altruistic regard for the greater good. A well-developed conscience is not what holds them back. Rather, it's the practical reality that they run the risk of social sanction, potentially restricting their freedom if they violate major rules. Not wanting to be hampered, they take some pains to see that nothing interferes with their quests for power \& success. They don't willingly subordinate their wills to a ``higher power'' or authority. They detest the notion that ``the man'' or ``the system'' might actually exert power over them if they've caught \& sanctioned for antisocial acts. So they do their best to avoid that possibility. But, like all aggressive personalities, they lack mature conscience \& a sense of moral obligation. So they won't hesitate to engage in minor transgressions unlikely to be discovered; or which, if discovered might result in significant social sanction. Nor will they refrain from major transgressions, including criminal behavior, when reasonably convinced (through the use of power, money, or influence) that they can get away with it.

The character deficiencies of the Channeled-Aggressive Personality become all too evident when they're convinced they won't be caught or sanctioned for breaking the rules. Believing their latest laser \& radar detectors are the best on the market, they take to the highway with reckless abandon, weave between cars, \& prove to the world they can shave at least 4 minutes off the time other hapless commuters spend getting to work. Assured that their corporate books have been ``cooked'' well enough that no oversight agency could possibly detect deception, they'll raid their company's coffers, line their pockets, \& swindle their investors. Confident they can successfully buy off corrupt officials, they'll engage in shady business deals. They're as much at odds with the rules as any other aggressive personality. \& it's not devotion to the greater good that leads them to avoid the life of the common criminal.

The principal features of the Channeled-Aggressive personality are:
\begin{itemize}
	\item Interpersonal ruthlessness \& heartlessness. Channeled-Aggressives are as disregarding of others' rights \& needs as any aggressive personality. This trait differentiates them from another actively ``independent'' type: the ``assertive'' personality.
	\item General confinement \& channeling of aggressive interpersonal conduct to non-criminal activity \& relatively socially acceptable outlets.
	\item Abdication of all controls when they're convinced they can successfully avoid detection or sanction.
\end{itemize}

\subsubsection{Covert-Aggression Pattern}
Covert-Aggressive Personalities are the archetypal manipulators I 1st described in my book, \textit{In Sheep's Clothing},\footnote{Simon, G., \textit{In Sheep's Clothing: Understanding \& Dealing with Manipulative People}, (A.J. Christopher \& Co., 1996).} which, after almost 16 years in print, has been revised a 3rd time \& re-released. These personalities are not openly aggressive in their interpersonal style. In fact, they do their best to keep their aggressive intentions \& behaviors carefully cloaked. They can appear charming \& amiable on the surface, but are just as ruthless as any other aggressive personality underneath their fa\c{c}ade. They are devious, underhanded, \& subtle in the ways they abuse \& exploit others. They are power-oriented individuals who do their best to look anyway but overtly power- \& dominance-seeking. They are generally equipped with an arsenal of interpersonal maneuvers \& tactics to manipulate, exert power over, \& control others in relationships. Their tactics are generally effective because they simultaneously accomplish 2 objectives:
\begin{enumerate}
	\item The tactics effectively play on the sensitivities, vulnerabilities, \& conscientiousness of others (especially neurotic individuals). The other persons then go unconsciously on the defensive (i.e. retreating mode). This quashes all potential resistance.
	\item The tactics conceal obvious aggressive intent. The other persons have little \textit{objective} evidence that the covert-aggressive is intending to take advantage of them. Instead of trusting their ``gut'' instincts, the other persons question themselves \& get hoodwinked.
\end{enumerate}
Covert-Aggressive personalities don't proudly broadcast their combative style as do their unbridled \& channeled-aggressive counterparts. They have found that the most effective way to advance their self-serving agendas is to keep them, as well as the nature of their true characters, carefully veiled. They gain advantage over others \& catch them unaware. They've learned from experience that overtly self-serving acts are likely to invite resistance. They'd rather not have to ``fight'' openly, fairly, or constructively, because they might actually lose. So, they fight surreptitiously, hoping that by the time their victims realize they've been taken advantage of, it will be too late. They're also confident in their ability to craftily ``reframe'' the nature of events, looking good despite the damage they're doing. This makes others feel like the bad guys for doubting them, \& ``justifies'' a whole host of behaviors that negatively impact the others' lives. This manipulative skill (sometimes referred to as ``impression management'') is developed to its most pathological extreme in 1 of the other aggressive personalities, which we'll discuss later.

Also, many of the tactics covert-aggressives use are also employed by other disturbed characters. We'll present these in a later chapter.

Some covert-aggressive personalities are remarkably skilled at hiding their quests for dominance while simultaneously wielding incredible power \& influence over others. Some cult leaders, as well as masterminds of radical religious \& political movements who occasionally make headlines have this personality type. While their devoted minions might truly believe the ideology they espouse, the master-manipulators know full well that their true agenda is power only. They know all combinations of tactics to bring others under their influence.

The cardinal characteristics of the covert-aggressive personality are:
\begin{itemize}
	\item Distinctly \textit{active} yet carefully \textit{veiled} aggressive style of interpersonal relating, resulting in the abuse, exploitation, \& victimization of others.
	\item Frequent use of a variety of tactics that simultaneously conceal their aggressive intent while putting conscientious others on the defensive, thus manipulating them.
	\item Skilled at impression management. Covert-aggressives ``frame'' events to maintain a favorable image, making others doubt their own intuitive mistrust of motives.
\end{itemize}

\subsubsection{The Sadistic Pattern}
Any aggressive personality is capable of causing great harm to others. However, for most of the aggressive personalities, inflicting injury or pain upon others is not their primary objective. Aggressive personalities simply want what they want \& fight tenaciously to get it. If they have to run roughshod over others to reach their goals, they will. They'll do whatever it takes to emerge victorious in an interpersonal encounter or secure the dominant or controlling position. When others either deliberately or inadvertently interfere with their quests, they attempt to remove or obliterate that interference. Sometimes their effort is violent in character, but most of the time it is simply forceful yet non-violent. They generally don't inflict any pain on others for its own sake. Rather, it's most often a secondary by-product or consequence of the overly aggressive quest to win.

Sadistic personalities differ from the other aggressive personality subtypes because inflicting pain is a primary objective of their interpersonal modus operandi. The sadist relishes hurting others, \& derives genuine pleasure (sometimes even sexual pleasure) from others' suffering \& humiliation. The sadist especially seeks to put or keep others in a degrading \& humiliating position. He wants to feel in such total control that his victims are truly helpless \& at his mercy. Sadistic personalities take the aggressive personality's quest for domination to its most pathological extreme.

Now, traditional thinking has often postulated that no one would inflict such cruelty unless they themselves had been subjected to such treatment; \& they were ``acting-out'' some sort of psychological ``payback'' against the world, or seeking a perverted sense of redress. But there's no solid, reliable, \& objective evidence for it. In fact, some studies have shown that although some highly disturbed characters \textit{report} mistreatment \& abuse in their early histories (i.e. give the ``abuse excuse'') to justify their cruel actions, such claims are often exaggerated \& sometimes bogus.

Sadistic personalities are often found in the most extreme examples of abusive relationships. Such individuals delight in physically, emotionally, \& psychologically battering \& demeaning those unfortunate souls involved with them. \&, contrary to popular belief, their victims don't usually stay with them because they are ``weak'' \& emotionally ``dependent.'' Rather, they often stay because they realize the perverted reality that they are actually safer caving into the abuser's wishes (to feel powerful \& dominant) than they might be if they were to declare their independence.

The core characteristics of the Sadistic Personality are:
\begin{itemize}
	\item Sadists actively \& intensely seek positions of dominance \& control over others.
	\item Sadists derive satisfaction (\& in some bizarre instances even experience sexual arousal) from exerting a level of control over others, leaving the victim cowering in fear, feeling helpless, utterly dependent, degraded, \& humiliated.
	\item Sadists view those perceived as weak with callousness \& disdain.
\end{itemize}

\subsubsection{The Predatory Aggressive Pattern}
Among all the aggressive personalities, by far the most pathological subtype is the one I prefer to label the predatory aggressive personality. These individuals are capable of the most heinous acts. The seemingly senseless brutality they engage in sometimes makes headlines. They are the true predators among us.

This personality type has been given many different labels in the past. The term \textit{psychopathic} was 1 of the 1st labels given to this deeply disordered character. The dominant thinking at the time was that every aspect of the human personality (including antisocial characteristics) could be explained by the theory of neurosis. That theory also proposed that when neurosis became extreme, or the inner conflicts underlying it were so intense that they caused a breakdown of the more common \& sophisticated ``defenses'' that made ``normal'' neurotic functioning impossible, the result was ``psychosis.'' Cleckley\footnote{Cleckley, H., \textit{The Mask of Sanity} (4e, Mosby, 1964).} \& others conceptualized the psychopath as a personality who appeared to be sane on the surface, but whose level of sociopathic neurosis bordered on insanity. Psychopathy, therefore, was conceptualized as a nearly insane level of antisociality.

Cleckley noted that psychopaths could be superficially charming, \& appear to the casual observer as quite benign \& ordinary characters. But they could also entertain almost unthinkable beliefs \& attitudes (while not being truly delusional), \& were prone to the vilest \& seemingly senseless antisocial acts. These characteristics would lead almost anyone to question their sanity. He also incorporated into his thinking Pinnel's observations in the early 19th Century \& even some of the ancient philosophers such as Aristotle. Cleckley determined that these individuals \textit{know} their behavior would be rightfully regarded by almost anyone as irrational, yet they persist in it. This fact appeared even more irrational \&, therefore, nearly insane. But, although they know well the vileness of their behavior, psychopaths persist in it \textit{voluntarily}. So, by current definition, they are not insane. People who are \textit{psychotic} (this term is still confused by some with the term \textit{psychopathic}) are indeed sometimes capable of heinous behavior, but it is not voluntary because their brain dysfunction impairs their capacity to reason logically \& deduce right from wrong.

The term \textit{sociopath} has also been used to describe this personality type, \& has recently again come into vogue.\footnote{Stout, M., \textit{The Sociopath Next Door}, (Random House, 2006).} The psychopath label focuses more on the uniquely abnormal mental processes of this disordered personality, whereas the sociopath label focuses more on the pattern of severe social dysfunction. Over the past several decades, the terms psychopath \& sociopath have alternatively enjoyed varying degrees of popularity among clinicians \& researchers.

Some suggest that the terms psychopath \& sociopath should not be used synonymously; that each term describes a slightly different \& uniquely extreme pathological variation of the antisocial personality. In my experience, I find some merit to this notion. I have met individuals who justify their unconscionable behavior by espousing beliefs that are truly chilling but seem crazy. Although such individuals are rarely truly delusional, they appear to really hold deeply troubling beliefs \& attitudes that make them a constant danger to others. This contrasts with other equally dangerous persons who espouse beliefs that they really don't hold; but they want others to believe that they believe, thus conferring a certain degree of bizarre rationally or justification to their heinous acts. So, even though they don't believe the ``B.S.'' they sling, they \textit{say} they do. It's their way of manipulating others \& conducting their con game of impression management.

In recent times, the eminent Canadian psychologist Robert Hare has re-popularized the label of psychopathy, \& defined its characteristics with a good deal of clarity. His arduous research also helped identify the 2 principal traits (i.e. factors) that accompany this severe disturbance of character: (1) Hare indicates this is a \textit{necessary condition} for someone to be considered a psychopath. It's what others have called a viciously \textit{malignant narcissism} exemplified by the psychopath's callous, senseless, \& remorseless use \& abuse of others.\footnote{Hare, R., \textit{Without Conscience: The Disturbing World of the Psychopaths Among Us}, (Guilford Press, 1991).} Their callousness as well as their lack of capacity for remorse is, within Hare's model, due to the absence of any conscience. This results from the psychopath's non-existent capacity to bond emotionally to the human race \& to have \textit{empathy} for others. (2) This often accompanying trait is not an essential feature. It's the socially parasitic lifestyle (e.g., criminal activity, checkered work history, abusive, \& exploitive relationships) common to antisocial personalities.

The nature of the 1st factor Hare describes, \& what gives rise to it in the very disordered characters, has always intrigued me. I've strived to better understand this factor in my work over the years with individuals in the prison population. There, estimates of the percentage of persons who score highly on ``psychopathy'' rating instruments range as high as approximately 1 in 4 or 5. Such individuals are also found among the ranks of true sexual predators. I have long observed these personalities to consider themselves superior creatures in comparison to ``common'' human beings. As they see it, amoebae, plankton, \& paramecia are at the bottom of the ``food chain''; ordinary humans are higher up, to be sure; but those like themselves are definitely at the top. Viewing themselves so pathologically superior, they tend to regard every creature ``beneath'' them (in status as well as worth) as \textit{rightful} prey. So, their seriously malignant narcissism reveals an extremely pathological kind of grandiosity, as well as an extraordinary attitude of entitlement. Just as most humans don't consider it a particularly heinous act to pluck an apple off a tree \& devour it, psychopaths don't even think twice when they use, abuse, or exploit ``ordinary'' humans in whatever ways they see fit to gratify their desires. For them, \textit{ordinary} humans -- those unfortunate creatures with fears, insecurities, emotional vulnerabilities \& sensitivities, \& especially compunctions about their actions that arise from their consciences -- are innately weak, inferior, defective, \& relatively worthless beings. They're not as entitled to survival as are those like themselves, whom they regard as the \textit{fittest} of all creation. So they prey at will on those they see as beneath them, \& with great self-edifying pleasure. That's why I think the label ``predatory aggressive'' more fully \& accurately captures the pathology of these highly disordered characters.

It's natural to wonder what factors could possibly make a person become so disordered in the 1st place. \&, of course traditional frameworks have always assumed that such individuals must necessarily have been exposed to the most extreme cruelty \& neglect; that failure to bond with others \& their hostile style of relating to the world resulted from the kinds of ``defenses'' they had to mount to deal with their cold \& hostile environment. But there's no solid evidence that any of these traditional notions are valid. Ever-mounting evidence shows that the backgrounds of these individuals do no differ all that significantly from those of other personalities, including ``normal'' or relatively well-adjusted personalities. Also, there appears to be some very unique constitutionally-based differences in these individuals that predispose their unusual character development, regardless of the environments to which they've been exposed.

Some research studies have indicated that the brains of psychopaths operate differently from those of normal individuals. The differences are most striking in the brain regions that integrate our emotions with our experiences. Such findings might help explain how psychopaths are able to carry out the most heinous acts without any apparent emotion or later remorse. A constitutional predisposition to lack genuine empathy also helps explain why individuals like Scott Peterson, Ted Bundy, Bernie Madoff \& countless others have made headlines over the years. They could be so superficially charming \& normal-appearing, could come from such relatively benign backgrounds, \& yet be capable of such predatory \& heartless behavior toward others.

In my work with predatory aggressive personalities, I have encountered many who, as Hare notes, lack any modicum of conscience or empathy for others. Such individuals are truly incapable of what psychologists call ``emotional bonding'' in a relationship. They experience absolutely no remorse when they commit acts of unspeakable horror against others. But a few that I've encountered actually appear to have \textit{some} capacity for these things. Unfortunately, they also have an extraordinary capacity to mentally wall-off or ``compartmentalize'' feelings they might have -- emotions which might otherwise unnerve them or interfere with their predatory agendas. This explains, e.g., why a psychopathic child sexual predator can feel genuine hurt when his own child is injured in an accident, or bristle at the thought of sexually offending one of his own, yet be capable of the completely cold \& heartless kidnapping, brutal rape, \& murder of a child across town when he feels the need to satisfy his craving. In my experience, this capacity for compartmentalization is more disturbing. Those individuals can appear so normal much of the time that, when others get a clue about just how dangerous they might be, it's far too late.

Some people report that juts being in the presence of this kind of personality makes the hair on the back of their necks stand up. A few researchers regard this intuitive sense of danger as a ``gift'' from our ancestors, letting us know we're in the presence of a predator.\footnote{De Becker, G., \textit{The Gift of Fear: Survival Signs that Protect Us from Violence}, (Little Brown \& Co., 1997).} But not all predators give off warning ``signals,'' \& not everyone has the intuitive capacity to pick up on those signals. Some predatory aggressives appear to have an uncanny ability to charm \& manage the impressions of others, a trait that goes beyond their typical superficial glibness. Such individuals could be considered a more extreme \& predatory variation of the covert-aggressive or manipulative personality.

As a result of my work over the years with both predatory aggressives \& their victims, I have concluded the main reason these predators are so successful in manipulating others: It lies not so much in their highly effective knowledge \& use of manipulation tactics; but rather in the reluctance of normal ``neurotic'' individuals to make harsh judgments about others, or to trust their gut instincts about the kind of person they're probably dealing with. They don't attach enough significance to the ``gift of fear,'' \& mistrust their instincts. On top of it all, they're also often blinded by the notion promoted by traditional psychology theories over the years that everyone is basically good (\& most especially, just like them underneath their wall of ``defenses''). So they allow themselves to believe that a person will only behave badly when hurting, frightened, or in some kind of inner pain. Such beliefs allow them to be easily victimized by the truly heartless. By now it should be evident that entertaining traditional notions about human behavior can easily be \textit{fatal} when it comes to dealing with a psychopath.

Like all the other aggressive personalities, predatory aggressives share some characteristics of the other aggressive personality subtypes. Predatory aggressives who have significant sadistic traits are among the most notorious serial killers you've ever read or heard about. In addition to callously preying on others, they often delight in watching their victims suffer, squirm, beg for mercy, or grovel in abject humiliation. \& predatory aggressives are often noted for having incredible manipulation skills. As the most pathological variation of the aggressive personality, it's no surprise that these individuals can often have other aggressive personality traits, each of which might exist at a disturbingly pathological level of intensity.

Some researchers say that these severely disordered personalities are untreatable; that exposing them to the kinds of treatment most commonly employed actually enable them to become more skilled as predators. Why? Because in typical treatment modalities where feelings \& unconscious conflicts are still given prominence, these predators gain greater \& more intimate knowledge of others' emotional concerns \& vulnerabilities. Presently, it's rare for treatment programs or providers to actually employ models that have a genuine chance of impacting severely disordered characters. So it will be some time before reliable data emerges on whether psychopaths can, in fact, be effectively treated. The emerging new models need much more development, \& must be more widely employed before the research becomes clear.

Through my years of experience, I believe I have witnessed significant changes take place in some predatory aggressive personalities who were confined long enough, exposed to sufficient ongoing ``corrective emotional \& behavioral experience,'' \& who showed some degree of ``internalization'' of appropriate values \& standards. More work \& research need to be done. But it is becoming increasingly evident that traditional interventions are virtually useless, \& may actually make matters worse.

To summarize, the core characteristics of the predatory aggressive personality are:
\begin{itemize}
	\item Lack of empathy that severely impairs conscience formation, begetting a callous disregard \& remorseless use \& abuse of others.
	\item Persistent ``conning'' \& predatory behavior.
	\item Glibness \& capacity for charm, often superficial, but sometimes quite convincing impression management.
	\item Sometimes parasitic lifestyle.
\end{itemize}

\subsubsection{The Mistrusting (Paranoid) Pattern}
Again, contrary to traditional notions, most aggressive personalities aren't motivated to take their power-oriented stance toward the world because of a deep-rooted mistrust of others, nor because they want to ``pay back'' the world for mistreatment they once suffered. Although they might claim so as a manipulation tactic, most don't really anticipate mistreatment by others nor believe the best ``defense'' is a strong ``offense.'' Rather, their offensive posture is simply a preferred way of doing things. \& often, their experience successfully rolling over others has convinced them that their strategy works. However, 1 aggressive personality ``style'' \textit{is} primarily influenced by a deep \& irrational mistrust of others.

Mistrusting Aggressive Personalities (alt. Paranoid Personalities) have a pervasive sense of wariness about others' intentions \& motivations. They are innately cynical \& vigilant, respecting others only for their perceived capacity to do harm if one's guard is dropped. They ardently strive to amass as much power as possible, concerned that others' power will inevitably be used against them. They can be extraordinarily spiteful \& vengeful; but their aggression is most often muted, subtly expressed, or carefully channeled. Fortunately, it's rarely expressed in violent ways.

Traditionalists view this ``paranoid'' personality style as deeply neurotic \& bordering on ``psychotic.'' The suspicious character of their cognitions, though not technically psychotic delusions, are most often quite irrational. On the dimension of neurosis vs. character disturbance, these individuals are equally neurotic \& character disordered. They're in part neurotic because their style of coping is truly fueled by deep inner conflicts of which they are largely unaware. But they are also very comfortable with their approach to dealing with the world. They're undeterred in their style by the adverse consequences of their typical behaviors, have distorted thinking processes \& an inflated sense of self. So they are also by definition to a great extent character-disordered.

Constitutional factors appear to play a significant role in this personality style. Their maladaptive wariness appears much less chosen as opposed to hardwired. Further, this trait appears to play a significant role in how the personality style develops in the 1st place.

The core characteristics of this personality style are:
\begin{itemize}
	\item Persistent, pervasive mistrust of others' intentions \& motivations.
	\item Absence of truly delusional (psychotic) thinking.
	\item Intensely hostile feelings, most often controlled, leading to holding grudges, unrealistic jealousy or suspiciousness, \& excessive vigilance.
\end{itemize}
Constitutional factors:
\begin{itemize}
	\item Excessive wariness, vigilance, \& suspiciousness.
	\item Tendency to interpret the nature of events, circumstances, \& others' behavior in a manner that easily ascribes malevolent, hostile intent to them.
\end{itemize}
Learning factors:
\begin{itemize}
	\item Provocative behaviors toward others invite hostile responses, thus reinforcing distorted perceptions.
	\item Hostile responses to others prompt hostile, defensive reactions from others, thus reinforcing distorted perceptions.
\end{itemize}

\subsubsection{Borderline Styles}
Probably no personality pattern has been the focus of as much study \& debate as the Borderline Personality. Originally the term ``borderline'' was intended to describe personalities held together with fragile, primitive, \& ineffective ego defenses. They literally straddled the border between extreme neurosis \& outright psychosis. Indeed, some borderline personalities do experience brief but reversible psychotic episodes.

As the The Diagnostic \& Statistical Manual of Mental Disorders (DSM) notes, some clinicians tend to view the borderline style as a distinctive personality pattern (i.e. the ``unstable'' pattern). Others tend to view the phenomenon as basically representing degrees of deficiency in personality organization \& solidification. Still others assert that at heart, the borderline pattern represents a disturbance or disorder of the ``self'': The individual was never able to master the principle task of adolescence. He never developed a stable, functional identity. (Indeed 1 of the previous editions of the official diagnostic manual required that the label ``identity disorder'' be given to individuals under age 18 showing borderline characteristics unless their symptoms far exceeded the criteria for an identity disturbance.) This is compatible with the notion that the resulting instability can easily become a ``style'' of its own.

We can view the borderline personality's disturbance as a failure of varying degrees to arrive at an integrated \& stable sense of self. This helps explain the wide variety of experience people have with borderline personalities, \& how any 2 borderlines can be so very different in character. That's because all of us have a variety of different personality \textit{traits} in us. Most of us form personalities that are fairly balanced, even though some of our tendencies or traits end to be more dominant. In the case of disordered personalities, dysfunctional traits are so intensely present, unyielding, or overtly dominant that they predispose a person to problems with functioning in an adaptive manner. In the case of disturbances of character, dysfunctional traits predispose individuals to chronic problems with functioning in a socially responsible manner. In the borderline personality, part of the disturbance's nature is this: Overall personality organization is so weak that we see behavior patterns associated with many different personality styles glaringly present in the same person. So, the defining overall ``style'' of interaction for the borderline is often a multitude of shifting styles (\& some propose that, in the most extreme form, it's a multiple personality). Nonetheless, some traits tend to be more dominant than others, making for a wide degree of variability in borderline personalities. A borderline personality with prominent passive-dependent traits is a remarkably different individual from a borderline personality with prominent antisocial traits. \& a borderline personality with prominent histrionic traits is very different from one with prominent obsessive-compulsive traits, etc.

Some borderline personalities had problems successfully mastering crucial psychosocial developmental stages. They experienced so much chaos, instability, abuse, \& trauma that safe, reliable role-models were not available to them. They endured such high levels of anxiety that the all-important task of identity solidification was overwhelmingly difficult, if not impossible for them to resolve adaptively. Other borderlines seemed naturally predisposed to personality difficulty in solidifying a stable sense of self. This seems due to their inherent \& chronic instability of mood, their legendary capacity for dialectical thinking (\& therefore considerable ambivalence w.r.t. ``resolving'' any developmental conflict), \& their gross deficiencies for managing anger.

Their lack of a stable sense of self \& balanced management of various traits \& tendencies cause a pattern of instability, impulsivity, \& severely self-defeating behavior. Not knowing how to cope, \& lacking internal controls, borderline personalities are notorious for attention-seeking, reckless, \& self-damaging acts when under duress. The more dependent borderlines are inordinately clingy \& have an almost irrational fear of abandonment. The more antisocial borderlines tend to enter numerous intensely-charged but remarkably shallow, volatile, \& short-lived relationships. In some borderlines, the lack of a stable sense of self leads to a lifetime of sexual-identity confusion \& fluidity. In others, the marked tendency toward dialectical thinking creates such intense ambivalence about almost every life issue, so emotional growth \& maturation is chronically impaired. In recent years, Dialectical Behavior Therapy has demonstrated some efficacy in the treatment of this personality disturbance.

The core characteristics of the borderline personality style are:
\begin{itemize}
	\item Extreme emotional instability or lability.
	\item Intense uncertainty, instability, \& variability w.r.t. personal identity \& style of inter-relating.
	\item Intense \& contrary (dialectical) urges, impulses, thinking patterns.
	\item Erratic \& impulsive patterns of behavior including episodic acts of self-harm \& irrational displays of anger toward others.
\end{itemize}
Possible Constitutional Factors:
\begin{itemize}
	\item Deficient emotional self-regulation.
	\item Excessive tendency toward dialectical thinking.
	\item Deficient impulse control capacity.
\end{itemize}
Possible Learning Factors:
\begin{itemize}
	\item Experience of severe trauma, chaos, instability, \& chronic anxiety during formative years hampering many aspects of development.
	\item Absence of stable, reliable role models sometimes a factor impairing identity formation.
\end{itemize}
Borderlines whose dominant personality traits are more closely associated with neurosis (e.g., dependent, avoidant) tend to be more neurotic than character-disturbed. Borderlines whose dominant personality traits are more closely associated with character disturbance (e.g., narcissistic, antisocial) tend to be more character-disturbed than neurotic.

\subsubsection{The Eccentric Style}
Many labels have been applied to individuals whose interpersonal style of relating is best described as ``odd'' or eccentric. These individuals appear to have such an unconventional view of the world, entertain such unusual beliefs about themselves, \& behave in such a ``weird'' or peculiar manner that they appear nearly insane. However, by definition, they are not psychotic. They do not suffer from the kind of brain dysfunction that makes rational thought impossible. They don't dysfunction is a truly ``psychotic'' manner. Their thinking is organized \& \textit{logical} in the strict sense of the word, even if it is \textit{unconventional}. They are also not plagued by hallucinations or true delusions. The term ``schizoid'' has once used to describe such individuals, but we now associate that label with the ``asocial'' or socially aloof style. In more recent times, the label ``schizotypal'' has been applied to these rare \& strangely interesting personalities.

I was once asked to interview a woman who spoke with a distinctly British accent. She carried herself in a manner that suggested she came from a long line of aristocrats. She spoke of her delight at inviting her fashionable neighbors for tea, \& was very prim \& proper in her manner. In fact, this woman was of humble origin, raised in the back woods of a southern American state. She was a perfectly delightful conversationalist \& displayed none of the cardinal signs of psychosis. She did not operate under any true delusions. Her eccentricities did not rise to the level of intensity, pervasiveness, or inflexibility to significantly impair her ability to function socially or occupationally, or cause anyone any appreciable distress. Therefore, she would not have even qualified for a diagnosis of a personality disorder. Still, she displayed many prominent traits of the schizotypal personality.

On balance, such personalities are much more neurotic than they ar character disturbed. Here's 1 of the few similarities they share with disturbed characters: They are usually quite comfortable with their style of inter-relating (i.e. their traits are ego-syntonic). However, their comfort with their personality characteristics does not fly in the face of extreme discomfort caused to others, as is the case with disturbed characters.

The core characteristics of the eccentric or schizotypal personality are:
\begin{itemize}
	\item Odd or unusual patterns of thinking, speaking, \& behavior,
	\item Impaired intimate relations except with close family members.
\end{itemize}
Possible Constitutional Factors:
\begin{itemize}
	\item Predisposed to thinking that is bizarre or seemingly irrational.
	\item Impaired ability to experience \& express normal emotions.
\end{itemize}
Possible Learning Factors:
\begin{itemize}
	\item Odd beliefs \& manner alienate others, thus reinforcing social interaction deficits.
	\item Difficulty relating normally increases anxiety which in turn fuels increased social avoidance.'' -- \cite[pp. 104--124]{Simon2011}
\end{itemize}

%------------------------------------------------------------------------------%

\section{The Process of Character Development}

\subsection{Conscience \& Character Development}
``I use catchy little rhyming saying at workshops. It sums up what must happen for values \& standards to become firmly embedded in a person's psyche. It also explains why so many disordered characters (especially the aggressive \& narcissistic personalities) don't form good consciences. They ditty goes like this: ``Internalizing a societal prohibition is at heart an act of submission.'' That's right. \textit{Internalizing} a \textit{prohibition} is most fundamentally an act of \textit{submission} to a higher power or authority. A person who truly accepts a principle of conduct, \& actively incorporates it into their conscience, willingly subordinates his or her individual will to serve the greater good. This is the very essence of healthy conscience formation.

Of all the disturbed characters, those with narcissists \&\texttt{/}or aggressive personality traits have the hardest time developing a healthy conscience. Narcissists struggle with forming good consciences because they find it difficult to even conceive of a power or cause greater than themselves. Further, their excessive sense of self-importance \& attitude of entitlement make it difficult for them to entertain the notion that anything besides what they want has any real value or importance. So, they pay little attention or heed to the wants \& needs of others.

The aggressive personalities despise the idea that anyone or anything might exert power, influence, or authority over them. They experience an innate revulsion to engaging in behaviors that even resemble acts of submission. So they wage a constant war with the potentially civilizing influences they encounter in their lives. It's not that they don't know what values \& principles others want them to adopt. They know these very well. But even though they know that society wants \& expects from them, they simply won't allow themselves to serve any master other than themselves.

Predatory aggressive (psychopathic) personalities are distinguished by their extreme lack of conscience. Their severely impaired capacity for empathy \& their extraordinary capacity to compartmentalize any sensitivity they might have make it virtually impossible for them to identify with \& care about others in the ways most of us can. Their malignant narcissism (i.e. extreme sense of superiority) prevents them from recognizing any higher power \& allows them to feel entitled to dominate \& prey upon those they view as inferior. Their deficient capacity for empathy, ability to wall-off emotion, pathological sense of superiority, \& extreme predisposition to aggressively prey all combine to make it virtually impossible to form a normal, healthy conscience.'' -- \cite[pp. 125--126]{Simon2011}

\subsection{Socialization is a \textit{Process}}

%------------------------------------------------------------------------------%

\section{Thinking Patterns \& Attitudes Predisposing Character Disturbance}

%------------------------------------------------------------------------------%

\section{Habitual Behavior Patterns Fostering \& Perpetuating Character Disturbance}

%------------------------------------------------------------------------------%

\section{Engaging Effectively \& Intervening Therapeutically with Disturbed Characters}

%------------------------------------------------------------------------------%

\section{Epilogue: Neurosis \& Character Disturbance}

%------------------------------------------------------------------------------%

\section*{Endnotes}

%------------------------------------------------------------------------------%

\printbibliography[heading=bibintoc]
	
\end{document}