\documentclass{article}
\usepackage[backend=biber,natbib=true,style=alphabetic,maxbibnames=50]{biblatex}
\addbibresource{/home/nqbh/reference/bib.bib}
\usepackage{tocloft}
\renewcommand{\cftsecleader}{\cftdotfill{\cftdotsep}}
\usepackage[colorlinks=true,linkcolor=blue,urlcolor=red,citecolor=magenta]{hyperref}
\usepackage{algorithm,algpseudocode,amsmath,amssymb,amsthm,float,graphicx,mathtools}
\allowdisplaybreaks
\numberwithin{equation}{section}
\newtheorem{assumption}{Assumption}[section]
\newtheorem{conjecture}{Conjecture}[section]
\newtheorem{corollary}{Corollary}[section]
\newtheorem{definition}{Definition}[section]
\newtheorem{example}{Example}[section]
\newtheorem{lemma}{Lemma}[section]
\newtheorem{notation}{Notation}[section]
\newtheorem{principle}{Principle}[section]
\newtheorem{problem}{Problem}[section]
\newtheorem{proposition}{Proposition}[section]
\newtheorem{question}{Question}[section]
\newtheorem{remark}{Remark}[section]
\newtheorem{theorem}{Theorem}[section]
\usepackage[left=1cm,right=1cm,top=5mm,bottom=5mm,footskip=4mm]{geometry}
\def\labelitemii{$\circ$}

\title{Character Disturbance: The Phenomenon of Our Age}
\author{George K. Simon Jr.}
\date{\today}

\begin{document}
\maketitle
\tableofcontents
\begin{quotation}
	``\textit{For Sherry}, whose heart is truer than any I know, \& who inspires me daily to be a better person. \& for all those silent but committed souls of noble character, upon whom the very survival of freedom depends.'' -- \cite{Simon2011}
\end{quotation}

%------------------------------------------------------------------------------%

\section*{Preface}
``Imagine you recently read a newspaper article about a young girl who suddenly \& inexplicably lost her eyesight. Her frantic parents took her from doctor to doctor, specialist to specialist, \& clinic to clinic, yet no one could find the reason for her blindness. In desperation, 1 day they decided to take her to a psychologist. After months of traditional psychoanalysis, the therapist revealed that the child's blindness resulted from severe emotional trauma. It seems that several months before, while riding on the school bus, this young lady just happened to glance at a boy seated with some friends across the aisle. She thought to herself: ``This guy is really cute.'' Before long, she also began thinking things like: ``I wonder what it would be like to kiss him.'' But almost immediately after having these thoughts, she started to feel badly. She began fretting about what kind of horrible person she must be to entertain such ``impure'' thoughts. She worried that they could only lead to other impure urges \& temptations, \& even perhaps some impure action on her part. Eventually, she became consumed with guilt \& shame. She remembered times in the past when she had looked at boys, \& how hard it was to resist impure thoughts \& urges. Surely worse would follow, she feared, if she didn't keep herself in check. Shortly after this incident, she lost her vision.

When she 1st saw the psychologist, this troubled little girl didn't even remember the bus incident. She certainly didn't remember what she was thinking or feeling at the time. She'd even forgotten how deeply the incident unnerved her, \& how she dealt with her anxiety over the situation. The psychologist helped her see that she had ``repeated'' her memoires as a way of easing the intensity of her emotional pain. Her lengthy analysis eventually helped her not only recover her memory of that fateful day's events, but also enabled her to reconnect with her conflicted emotions. She came to realize this: She was so deeply distressed by what she thought were her unforgivable, impure desires that she actually believed it was better not to see at all than risk having such thoughts about boys again.

Once she had confessed her sins to the doctor \& he did not condemn her, the young girl slowly began to feel better. She took heart in the notion that he appeared to accept her just as she was. She slowly began to feel that she wasn't such a horrible person after all merely for the kinds of thoughts she sometimes had about boys. In time, she came to believe that the level of her fear, guilt, \& shame was excessive \& unwarranted, given the nature of the situation. No longer believing that simply having thoughts about kissing boys was as evil as she once did, she allowed herself to see again.

Now, I would pose to you, the reader, the same question I ask of professionals \& non-professionals alike at every 1 of the hundreds of workshops I've given over the past 25 years. How many scenarios similar to the one I described have you read or heard about in the last year? How about the past 5 years? How about the past, 10--20--30 years? It should come as no surprise that the answer I get is always the same: \textbf{\textit{zero}}. What should shock you, however (\& always gets the attention of my audience), is this: \textit{Almost all of the principles of classical-psychology paradigms stemmed from various theorists' attempts to explain this \& similar phenomena} (sometimes referred to by adherents to these theories as ``hysterical'' blindness).

You see, in Sigmund Freud's day, \textit{some} individuals actually suffered from such strange maladies. It's important to note, however, that these extreme psychological illnesses were \textit{never} widespread phenomena. But in the intensely socially repressive Victorian era, cases appeared of persons experiencing extraordinary levels of guilt or shame merely for being tempted to act on a primal human instinct, \& displaying pathological symptoms as a result. If there were a motto or saying that might best describe the ``zeitgeist'' (i.e. social or cultural milieu) of that time it would be: ``Don't even \textit{think} about it!'' So, some individuals were quite unnecessarily consumed with excessive guilt \& shame about their most basic human urges. Freud treated some of these individuals, most of whom were women, typically subjected to more oppression than men. He eventually coined the term ``neurosis'' to describe the internal struggle he believed went on between a person's instinctual urges (i.e. \textit{id}) \& their conscience (i.e. \textit{superego}), the excessive \textit{anxiety} or nervous tension that often accompanies this internal war, \& the unusual psychological \textit{symptoms} a person can develop when attempting to mitigate (i.e. through the use of marginally effective \textit{defense mechanisms}) the intense emotional pain associated with these inner conflicts. He then developed a set of theories \& constructs that appeared to adequately explain how his patients developed their bizarre maladies. In the process, however, \textit{he also came to believe that he had discovered universal, fundamental principles that explained personality formation \& the entire spectrum of human psychological functioning}.

Many of the constructs \& terms 1st articulated by Freud, \& several other ``psychodynamic'' theorists who followed him, found their way into common parlance over the years. For a significant period of time, these tenets also gained widespread acceptance by mental health professionals. \& many of the principles of what I refer to in this book as ``traditional psychology'' still enjoy a fair degree of acceptance, not only among professionals but also among lay persons. This is true despite the fact that, in recent years, several of the most important assumptions \& doctrines have been proven completely false or significantly flawed.

Not only are some of traditional psychology's central tenets of questionable validity, but also times have changed dramatically since the days of Sigmund Freud. If a motto or saying best befits our modern era's zeitgeist, it would be much like the once popular commercial that urged: ``Just \textit{do} it!'' As a result, highly pathological levels of neurosis (as exemplified by the girl who went blind simply because she ``lusted'' after a boy she found attractive) have all but completely disappeared, especially in industrialized free societies. Instead of being dominated by individuals overly riddled with unfounded guilt \& shame (i.e., ``hung-up'' as children of the `60s used to say), modern western culture has produced increasing numbers of individuals who aren't ``hung-up'' enough about the things they let themselves do.

So, today we are facing a near epidemic of what some theorists refer to as \textit{character disturbance}. Neurosis is still with us, but for the most part at \textit{functional} as opposed to pathological levels. I.e., most people today experience just enough apprehension \& internal turmoil when it comes to simply acting on their primal urges, that they don't in fact ``just do it.'' Instead, they experience sufficient anticipatory guilt or shame to restrain their impulses \& conform their conduct to more socially acceptable standards. So, one can easily say that their neurosis is \textit{functional}. It's largely what makes society work.

Freud used to say that civilization is the cause of neurosis, \& given the climate of his time, it's easy to see how he came to that conclusion. In a sense, he had a point; but his observation was more than a bit narrow-sighted. True, the prohibitions any society imposes on the unrestrained expression of primal urges can give rise to a fair degree of anxiety in some of us. \&, a brutally oppressive culture can breed excessive degrees of neurosis. But in large measure, \textit{it's most people's \textbf{capacity} to become unnerved when contemplating acting like an animal (i.e. their capacity to be ``neurotic'' to some degree) that makes civilization itself possible}. So, it's precisely because significant numbers of people still get a little ``hung up'' when they contemplate punching someone's lights out, or have some apprehension \& qualms of conscience when they blink about taking something that doesn't belong to them, that there's any degree of civilization left at all.

Today many types of professionals span a wide variety of disciplines that deal with mental health issues \& personal problems of 1 variety or another. Most of these professionals have \textit{never} encountered -- let alone treated -- a case of ``hysterical blindness,'' pseudo-paralysis, or any similar phenomenon. In fact, it's becoming increasingly rare for professionals to encounter a case of neurosis at a highly pathological level of intensity. Therapists rarely deal with problems that stem from a conscience so overactive or oppressive that it causes a person to develop bizarre or severely debilitating psychosomatic or other pathological symptoms. Instead, mental health clinicians in all disciplines increasingly find themselves intervening with individuals whose problems are related to their dysfunctional \textit{attitudes} \& \textit{thinking patterns}, their shallow, self-centered relationships, their moral immaturity \& social irresponsibility, \& their habitual, \textit{dysfunctional behavior patterns}. All of these stem from an \textit{underdeveloped conscience} \& reflect significant \textbf{\textit{deficiencies or disturbances of character}}.

This wouldn't be so much of a problem if it weren't for the fact that many mental health professionals not only are trained primarily in classical theories of human behavior, but also cling to beliefs about human nature \& the underpinnings of psycho-social dysfunction that originally emanated from these theories. For this reason, they often attempt to use the tenets \& the principles that flow from more traditional paradigms to guide them in their efforts to solve today's very different kinds of psychological problems. In short, they attempt to understand \& treat character disturbance with approaches \& methods originally designed to treat extreme levels of neurosis.

Retaining outdated notions about why people do the things they do can put anyone -- lay persons \& mental health professionals alike -- at a great disadvantage when it comes to understanding \& dealing with the disturbed character. Even some of our more modern frameworks for understanding human behavior are inadequate to address the phenomenon. The problem is compounded by this fact: It's not ``politically correct'' to consider people's emotional \& behavioral problems as stemming from or reflecting deficiencies in their character. So, sometimes problems actually rooted in character pathology might be framed as almost anything else (e.g., an ``addiction,'' ADHD, Bipolar Disorder, a ``chemical imbalance,'' etc.). Their symptoms can be somewhat managed with medication or other forms of treatment. Then, if at least some degree of change is observed, that mere fact validates the perspective that disease of some sort -- as opposed to character -- caused the problem. Sometimes, professionals actually do recognize personality or character disturbances, but regard them as unchangeable or untreatable. So, they target issues other than character concerns in therapy. At other times a professional's over-immersion in traditional paradigms will prompt them to view everyone -- even the most severely character-disturbed individual -- as neurotic, at least to some degree, \& then attempt to treat their supported neurosis. But the reality is this: \textbf{\textit{character disturbance is 1 of the most pressing psychological realities of our age; it's becoming increasingly prevalent; \& it's an entirely different phenomenon from neurosis, requiring a different perspective to adequately understand \& treat}.}

Understanding character disturbance requires viewing human beings, \& the reasons they do some of the things they do, in a very different light. Further, character disturbance simply can't be dealt with effectively using traditional approaches. The tools \& techniques that have proven effective in treating character disturbance are radically different from those originally developed to treat neurosis.

Over 14 years ago, I wrote \textit{In Sheep's Clothing: Understanding \& Dealing with Manipulative People} to help both therapists \& average folks understand what certain people are really like, \& how they manage to manipulate \& control others. The book has grown increasingly popular every year since its 1st printing, has been translated into several foreign languages, \& consistently draws highly laudatory reviews from professionals as well as lay persons. The comments are generally of a similar nature. Readers report that once they cast off old notions about the nature \& behavior of the manipulative persons in their lives, ``got it'' w.r.t. what really makes such individuals ``tick,'' \& adopted not only a new perspective for understanding such persons but also a new set of rules \& principles to guide their relationships with them, their circumstances dramatically improved. This kind of feedback has been more than edifying for me, \& inspired me to write this book. \textit{In Sheep's Clothing} was about 1 type of disturbed character. This book is about all of the various disturbed character types you are likely to encounter in your life. \textit{In Sheep's Clothing} briefly delved into the general topic of character disturbance. This book takes a much deeper look at this significant \& disturbing phenomenon, its growing prevalence, \& some of the socio-cultural features of the current era responsible for promoting it.

My primary purpose in writing this book is to help you: (1) cast off faulty assumptions that can place you at a distinct disadvantage, \& (2) understand the real reasons disturbed characters behave the way they do. It's important to recognize that disturbed characters differ dramatically from neurotics on almost every imaginable dimension of interpersonal functioning. They don't hold the same values, believe the same things, harbor the same attitudes, think the same way, or behave in the same manner as neurotics. This book will help you understand why you might have had such a difficult time dealing with the disturbed characters in your life, what makes these individuals so different, \& why aid you might have sought, hoping to deal with problems involving them, proved inadequate.

As with \textit{In Sheep's Clothing}, I've written this book in a concise manner, easy to read \& understand. It should be equally helpful to the layperson as well as the professional who is primarily versed in or aligned with traditional perspectives. I have deliberately not included mounds of difficult-to-understand scientific research data \& have attempted to translate sophisticated \& highly technical material into simpler, yet reasonably accurate language. My intention was not to craft an authoritative \& comprehensive textbook-style treatise on human nature, personality, or psychopathology. Rather, I present principles \& perspectives derived from years of experience working with disturbed characters \& their victims, illustrate those principles with real-life examples, \& support the most important contentions \& perspectives with relevant research findings when appropriate or necessary.

My secondary purpose in writing this book is to expose character disturbance for the significant social problem that it is, \& address some of the socio-cultural influences responsible for it. I am not the 1st to sound the alarm on the issue, \& I hope I will not be the last. In their book on character lengths \& virtues, the eminent psychology pioneers \& researchers Christopher Peterson \& Martin E.P. Seligman address the issue directly:

``After a detour through the hedonism of the 1960's, the narcissism of the 1970's, the materialism of the 1980's \& the apathy of the 1990's, most everyone today seems to believe that character is important after all \& that the United States is facing a character crisis on many fronts, from the playground to the classroom to the sports arena to the Hollywood screen to business corporations to politics.''\footnote{Peterson, C., \& Seligman, M., \textit{Character Strengths \& Values: A Handbook \& Classification}, (American Psychological Association \& Oxford University Press, 2004), p. 5.}

But no problem can be resolved until it's fully acknowledged \& adequately defined. That's the 1st step. The 2nd step is to take a serious look at the characteristics of our social milieu that contribute to the problem. This book will do both.

I have been working with disturbed characters for over 25 years now. Yet I know that much of what I assert in this book is likely to be controversial. When I 1st started doing workshops on manipulators \& other problem characters, especially with professionals, several attendees were uncomfortable with my perspective. Some even walked out! That's because a lot of what I had to say challenged some longstanding notions \& deeply-held beliefs about why people experience psychological problems in the 1st place, \& how professionals must assist them in achieving mental \& emotional health. But time \& recent research has validated much about the perspective I introduced those many years ago. These days, it's very common for me to see many heads in the audience frequently nod in approval as I articulate the principles I've adopted to understand \& deal with disturbed characters. It's also common for individuals who have already been to 1 of my workshops to come back again several times for ``refreshers'' on the tenets I will outline in this book. 1 of the more frequent comments I get from professionals involves how much more satisfying working with disturbed characters has become after they adopted some of these principles.

'' -- \cite[pp. 8--]{Simon2011}

%------------------------------------------------------------------------------%

\section*{Introduction}

%------------------------------------------------------------------------------%

\section{Neurosis \& Character Disturbance}

%------------------------------------------------------------------------------%

\section{Major Disturbances of Personality \& Character}

%------------------------------------------------------------------------------%

\section{The Aggressive Pattern}

%------------------------------------------------------------------------------%

\section{The Process of Character Development}

%------------------------------------------------------------------------------%

\section{Thinking Patterns \& Attitudes Predisposing Character Disturbance}

%------------------------------------------------------------------------------%

\section{Habitual Behavior Patterns Fostering \& Perpetuating Character Disturbance}

%------------------------------------------------------------------------------%

\section{Engaging Effectively \& Intervening Therapeutically with Disturbed Characters}

%------------------------------------------------------------------------------%

\section{Epilogue: Neurosis \& Character Disturbance}

%------------------------------------------------------------------------------%

\section*{Endnotes}

%------------------------------------------------------------------------------%

\printbibliography[heading=bibintoc]
	
\end{document}