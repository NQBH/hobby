\documentclass[oneside]{book}
\usepackage[backend=biber,natbib=true,style=authoryear]{biblatex}
\addbibresource{/home/hong/1_NQBH/reference/bib.bib}
\usepackage[vietnamese,english]{babel}
\usepackage{tocloft}
\renewcommand{\cftsecleader}{\cftdotfill{\cftdotsep}}
\usepackage[colorlinks=true,linkcolor=blue,urlcolor=red,citecolor=magenta]{hyperref}
\usepackage{amsmath,amssymb,amsthm,mathtools,float,graphicx,algpseudocode,algorithm,tcolorbox,tikz,tkz-tab}
\DeclareMathOperator{\arccot}{arccot}
\usepackage[inline]{enumitem}
\allowdisplaybreaks
\numberwithin{equation}{section}
\newtheorem{assumption}{Assumption}[section]
\newtheorem{nhanxet}{Nhận xét}[section]
\newtheorem{conjecture}{Conjecture}[section]
\newtheorem{corollary}{Corollary}[section]
\newtheorem{hequa}{Hệ quả}[section]
\newtheorem{definition}{Definition}[section]
\newtheorem{dinhnghia}{Định nghĩa}[section]
\newtheorem{example}{Example}[section]
\newtheorem{vidu}{Ví dụ}[section]
\newtheorem{lemma}{Lemma}[section]
\newtheorem{notation}{Notation}[section]
\newtheorem{principle}{Principle}[section]
\newtheorem{problem}{Problem}[section]
\newtheorem{baitoan}{Bài toán}[section]
\newtheorem{proposition}{Proposition}[section]
\newtheorem{menhde}{Mệnh đề}[section]
\newtheorem{question}{Question}[section]
\newtheorem{cauhoi}{Câu hỏi}[section]
\newtheorem{remark}{Remark}[section]
\newtheorem{luuy}{Lưu ý}[section]
\newtheorem{theorem}{Theorem}[section]
\newtheorem{dinhly}{Định lý}[section]
\usepackage[left=0.5in,right=0.5in,top=1.5cm,bottom=1.5cm]{geometry}
\usepackage{fancyhdr}
\pagestyle{fancy}
\fancyhf{}
\lhead{\small \textsc{Sect.} ~\thesection}
\rhead{\small \nouppercase{\leftmark}}
\renewcommand{\sectionmark}[1]{\markboth{#1}{}}
\cfoot{\thepage}
\def\labelitemii{$\circ$}

\title{Quotes from The Books}
\author{\selectlanguage{vietnamese} Nguyễn Quản Bá Hồng\footnote{Independent Researcher, Ben Tre City, Vietnam\\e-mail: \texttt{nguyenquanbahong@gmail.com}; website: \url{https://nqbh.github.io}.}}
\date{\today}

\begin{document}
\frontmatter
\maketitle
\setcounter{secnumdepth}{4}
\setcounter{tocdepth}{3}
\tableofcontents
\newpage

%------------------------------------------------------------------------------%

\chapter*{Preface}

The title of this text is inspired by that of the classical elementary mathematical book \cite{Andreescu_Dospinescu2010}.

%------------------------------------------------------------------------------%

\mainmatter

\chapter{\cite{Chi2022}. A Book on Minimalism}

%------------------------------------------------------------------------------%

\chapter{\cite{Frankl2013, Frankl2017, Frankl2022}. Man's Search for Meaning}

\section{About the Book}
``A prominent\footnote{\textbf{prominent} [a] \textbf{1.} important or well known; \textbf{2.} easily seen; \textsc{synonym}: \textbf{noticeable}; \textbf{3.} sticking out from something.} Viennese psychiatrist before the war, Viktor Frankl was uniquely able to observe the way that both he \& others in Auschwitz coped (or didn't) with the experience. He noticed that it was the men who comforted others \& who gave away their last piece of bread who survived the longest -- \& who offered proof that everything can be taken away from us except the ability to choose our attitude in any given set of circumstances. The sort of person the concentration camp prisoner became was the result of an inner decision \& not of camp influence alone. Frankl came to believe man's deepest desire is to search for meaning \& purpose. This outstanding work offers us all a way to transcend suffering \& find significance in the art of living.'' -- \cite[About the Book, p. 3]{Frankl2013}

\section{About the Author: Viktor Emil Frankl}
``Viktor E. Frankl was Professor of Neurology \& Psychiatry at the University of Vienna Medical School. He was the founder of what has come to be called the 3rd Viennese school of Psychotherapy (after Freud's psychoanalysis \& Adler's individual psychology) -- the school of logotherapy. His writings has been called ``the most important contributions in the field of Psychotherapy since the days of Freud, Adler, \& Jung'' by Sir Cyril Burt, ex-President of the British Psychological Society.'' -- \cite[p. 4]{Frankl2013}

\section{Quotes from \cite{Frankl2013, Frankl2017, Frankl2022}}
\selectlanguage{vietnamese} 
\begin{enumerate}[leftmargin=0mm]
	\item ``Làm thế nào để có thể tìm được ý nghĩa cuộc sống? Theo liệu pháp ý nghĩa thì có 3 con đường nhờ đó con người sẽ đạt đến ý nghĩa cuộc sống của mình. 1 là làm việc; 2 là trải nghiệm; \& 3 là vượt lên số phận \& thay đổi bản thân.''
	
	``Tình yêu là cách duy nhất để thấu hiểu đến tận cùng 1 con người. Không ai có thể nhận thức đầy đủ về bản chất 1 người trừ khi đã đem lòng yêu thương người ấy.''
	
	``Nếu 1 người không thể thay đổi hoàn cảnh khiến mình đau khổ thì người đó vẫn có thể chọn cho mình 1 thái độ sống.'' -- \cite[Bookcover]{Frankl2022}
	\item \textit{``Người nào có \textit{lý do} để sống thì có thể tồn tại trong \textit{mọi} nghịch cảnh.''} \texttt{[translated]} -- Friedrich Wilhelm Nietzsche\footnote{\textsc{Friedrich Wilhelm Nietzsche} (1844--1900): Nhà triết học Đức, có ảnh hưởng lớn trong triết học hiện đại, nhất là đối với \textit{chủ nghĩa hiện sinh} \& \textit{chủ nghĩa hậu hiện đại}.}
	
	``Several times in the course of the book, Frankl approvingly\footnote{\textbf{approving} [a] showing that you believe that somebody\texttt{/}something is good or acceptable, \textsc{opposite}: \textbf{disapproving}.} quotes the words of Nietzsche, \textit{``He who has a Why to live for can bear almost any How.''}\,'' -- \cite[Preface by \textsc{Harold S. Kushner}, p. 9]{Frankl2013}
	\item ``[Viktor Emil Frankl] chua xót kể về những tù nhân đã đầu hàng cuộc sống, mất hết hy vọng ở tương lai \& chắc hẳn là những người đầu tiên sẽ chết; ít người chết vì thiếu thức ăn \& thuốc men, mà phần lớn họ chết vì thiếu hy vọng, thiếu 1 lẽ sống.'' -- \cite[p. 6]{Frankl2022}
	
	``He describes poignantly\footnote{\textbf{poignantly} [adv] in a way that has a strong effect on your feelings, especially when it makes you feel sad, \textsc{synonym}: \textbf{movingly}.} the prisoners who gave up on life, who had lost all hope for a future \& were inevitably\footnote{\textbf{inevitably} [adv] as is certain to happen.} the 1st to die. They died less from lack of food or lack of medicine than from lack of hope, lack of something to live for.'' -- \cite[Preface by \textsc{Harold S. Kushner}, p. 9]{Frankl2013}
	\item ``Những trải nghiệm kinh hoàng của tác giả [Viktor Emil Frankl] trong trại tập trung Auschwitz đã củng cố 1 trong những quan điểm chính của ông: Cuộc sống không phải chỉ là tìm kiếm khoái lạc, như Freud\footnote{\textsc{Sigmund Freud} (1856--1939): Bác sĩ về thần kinh \& là nhà tâm lý học người Áo, người đặt nền móng \& phát triển ngành \textit{phân tâm học}.} tin tưởng, hoặc tìm kiếm quyền lực, như Alfred Adler\footnote{\textsc{Alfred Adler} (1870--1937): Nhà tâm lý học người Áo, người sáng lập ra \textit{Tâm lý học Cá nhân}.} giảng dạy, mà là đi tìm ý nghĩa cuộc sống. Nhiệm vụ lớn lao nhất của mỗi người là tìm ra ý nghĩa trong cuộc sống của mình. Frankl đã nhìn thấy \textit{3 nguồn ý nghĩa cơ bản của đời người}: thành tựu trong công việc, sự quan tâm chăm sóc đối với những người thân yêu \& lòng can đảm khi đối mặt với những thời khắc gay go của cuộc sống. Đau khổ tự bản thân nó không có ý nghĩa gì cả, chính cách phản ứng của chúng ta mới khoác lên cho chúng ý nghĩa. Frankl đã viết rằng 1 người ``có thể giữ vững lòng quả cảm, phẩm giá \& sự bao dung, hoặc người ấy có thể quên mất phẩm giá của con người \& tự đặt mình ngang hàng loài cầm thú trong cuộc đấu tranh khắc nghiệt để sinh tồn''. Frankl thừa nhận rằng chỉ có 1 số ít tù nhân của Đức quốc xã là có thể giữ được những phẩm chất ấy, nhưng ``chỉ cần 1 ví dụ như thế cũng đủ chứng minh rằng sức mạnh bên trong của con người có thể đưa người ấy vượt lên số phận nghiệt ngã của mình''.
	
	Frankl luôn trung thành với quan điểm: Những thế lực vượt quá khả năng kiểm soát của bạn có thể lấy đi mọi thứ mà bạn có, chỉ trừ 1 thứ, đó là \textit{sự tự do chọn lựa cách bạn phản ứng trước hoàn cảnh}.'' [$\ldots$] ``Bạn không thể kiểm soát điều gì sẽ xảy ra trong đời mình, nhưng bạn luôn có thể kiểm soát cách đón nhận cũng như cách phản ứng trước mọi tình huống của cuộc sống.'' -- \cite[pp. 6--7]{Frankl2022}
	
	``His experience in Auschwitz, terrible as it was, reinforced\footnote{\textbf{reinforce} [v] \textbf{1.} \textbf{reinforce something} to make a feeling, idea, habit or tendency stronger; \textbf{2.} \textbf{reinforce something} to make a structure or material stronger, especially by adding another material to it; \textbf{3.} \textbf{reinforce something} to send more people or equipment in order to make an army, etc. stronger.} what was already 1 of his key ideas. Life is not primarily\footnote{\textbf{primarily} [adv] mainly, \textsc{synonym}: \textbf{chiefly}.} a quest\footnote{\textbf{quest} [n] a long or difficult search for something, especially for a quality such as knowledge or truth.} for pleasure, as Freud believed, or a quest for power, as Alfred Adler taught, but a quest for meaning. The great task for any person is to find meaning in his or her life. Frankl saw 3 possible sources for meaning: in work (doing something significant), in love (caring for another person, as Frankl held on to the image of his wife through the darkest days in Auschwitz), \& in courage in difficult times. Suffering in \& of itself is meaningless\footnote{\textbf{meaningless} [a] \textbf{1.} not having a meaning that is easy to understand; \textbf{2.} without any purpos or reason \& therefore not worth doing or having; \textbf{3.} \textbf{meaningless (to somebody\texttt{/}something)} not considered important, \textsc{synonym}: \textbf{irrelevant}.}; we give our suffering meaning by the way in which we respond to it. At 1 point, he writes that a person ``may remain brave, dignified\footnote{\textbf{dignified} [a] calm \& serious \& deserving respect, \textsc{opposite}: \textbf{undignified}.} \& unselfish\footnote{\textbf{unselfish} [a] giving more time or importance to other people's needs, wishes, etc. than to your own, \textsc{synonym}: \textbf{selfless}, \textsc{opposite}: \textbf{selfish}.}, or in the bitter fight or self-preservation\footnote{\textbf{self-preservation} [n] [uncountable] the fact of protecting yourself in a dangerous or difficult situation.} he may forget his human dignity \& become no more than an animal.'' He concedes that only a few prisoners of the Nazis were able to do the former, ``but even 1 such example is sufficient proof that man's inner strength may raise him above his outward fate.''
	
	Finally, Frankl's most enduring\footnote{\textbf{enduring} [a] lasting for a long time.} insight, one that I have called on often in my own life \& in countless counseling situations: forces beyond your control can take away everything you possess except 1 thing, your freedom to choose how you will respond to the situation. You cannot control what happens to you in life, but you can always control what you will feel \& do about what happens to you.'' -- \cite[Preface by \textsc{Harold S. Kushner}, p. 9]{Frankl2013}
	\item ``Frankl đã nói rằng chúng ta sẽ không bao giờ mất đi tất cả nếu chúng ta vẫn còn được tự do lựa chọn cách phản ứng với sự việc.'' -- \cite[p. 8]{Frankl2022}
	
	``Frankl would have argued that we are never left with nothing as long as we retain\footnote{\textbf{retain} [v] \textbf{1.} \textbf{retain somebody\texttt{/}something} to keep somebody\texttt{/}something; to continue to have something \& not lose it or get rid of it; \textbf{2.} \textbf{retain something}to take in a substance \& keep holding it; \textbf{3.} \textbf{retain something} to remember or continue to hold something; \textbf{4.} \textbf{retain somebody\texttt{/}something} (\textit{law}) to employ a professional person such as a lawyer or doctor; to make regular payments to such a person in order to keep their services.} the freedom to choose how we will respond.'' -- \cite[Preface by \textsc{Harold S. Kushner}, p. 10]{Frankl2013}
	\item ``$\ldots$ có nhiều doanh nhân thành công, sau khi nghỉ hưu, đã mất hết mọi nhiệt huyết với cuộc sống. Với họ, dường như ý nghĩa cuộc sống chỉ xoay quanh 2 chữ ``công việc''. Vì vậy, khi không có việc làm nữa, học cảm thấy cuộc sống trở nên vô vị\footnote{NQBH: The Emptiness in the Soul. The Nothingness in the Heart \& the Mind. The Void in the whole Life.}. Mỗi ngày qua đi, họ lặng lẽ ngồi trong nhà băn khoăn thấy mình ``không biết làm gì''.'' [$\ldots$] ``$\ldots$ nhiều người đã trưởng thành từ khả năng chịu đựng phi thường của họ 1 khi họ có lòng tin rằng những tai họa, nghịch cảnh họ đang chịu đựng không vô ích. Đa phần đối với mọi người, việc có 1 \textit{Lý do} để sống khiến cho họ có thể chịu được \textit{Mọi} hoàn cảnh, có thể đó là mong muốn được cùng mọi người trong gia đình gánh vác 1 trọng trách nào đó hoặc hy vọng các bác sĩ sẽ tìm ra phương thuốc điều trị căn bệnh của họ.'' -- \cite[p. 8]{Frankl2022}
	
	``I have known successful businessmen who, upon retirement, lost all zest\footnote{\textbf{zest} [n] \textbf{1.} [singular, uncountable] \textbf{zest (for something)} pleasure \& enthusiasm, \textsc{synonym}: \textbf{appetite}; \textbf{2.} [uncountable, singular] the quality of being exciting, interesting \& fun; \textbf{3.} [uncountable] the outer skin of an orange, a lemon, etc., when it is used in cooking.} for life. Their work had given their lives meaning. Often it was the only thing that gave their lives meaning, \& without it, they spent day after day sitting at home depressed, ``with nothing to do.'' I have known people who rose to the challenge of enduring the most terrible of afflictions\footnote{\textbf{affliction} [n] [uncountable, countable] (\textit{formal}) pain \& difficulty or something that causes it.} \& situations as long as they believed there was a point to their suffering. Whether it was a family milestone they wanted to live long enough to share or the prospect of doctors finding a cure by studying their affliction, having a Why to live for enabled them to bear the How.'' -- \cite[Preface by \textsc{Harold S. Kushner}, p. 10]{Frankl2013}
	\item ``$\ldots$ \textit{học thuyết về liệu pháp ý nghĩa} của Frankl trình bày phương thức chữa trị vết thương cho tâm hồn bằng cách dẫn dắt nó đi tìm ý nghĩa cuộc sống, những trang viết được đúc kết từ trải nghiệm mà tác giả phải chịu ở Auschwitz đã lập tức nhận được sự đồng cảm \& ủng hộ của độc giả.'' -- \cite[pp. 8--9]{Frankl2022}
	
	``Just as the ideas in my book \textit{When Bad things Happen to Good People} gained power \& credibility\footnote{\textbf{credibility} [n] [uncountable] the quality that somebody\texttt{/}something has that makes people believe or trust them\texttt{/}it.} because they were offered in the context of my struggle to understand the illness \& death of our son, Frankl's doctrine\footnote{\textbf{doctrine} [n] \textbf{1.} [countable, uncountable] \textbf{doctrine (of something)} a belief or principle, or set of beliefs or principles, held by a religion, a political party or a legal system; \textbf{2.} (\textbf{Doctrine}) [countable] (US) a statement of government policy, especially foreign policy.} of logotherapy, curing the soul by leading it to find meaning in life, gains credibility against the background of his anguish\footnote{\textbf{anguish} [n] [uncountable] (\textit{formal}) severe physical or mental pain, difficulty or unhappiness.} in Auschwitz.'' -- \cite[Preface by \textsc{Harold S. Kushner}, p. 10]{Frankl2013}
	\item ``Đây là 1 quyển sách có bàn về những nội dung rất sâu sắc liên quan đến lĩnh vực tôn giáo. Nó thể hiện tư tưởng chủ đạo của tác giả [Viktor Emil Frankl] là \textit{cuộc sống có ý nghĩa \& chúng ta phải học cách nhìn cuộc sống theo hướng có ý nghĩa bất kể hoàn cảnh nào đi nữa}. Tác phẩm cũng nhấn mạnh 1 điều: \textit{Cuộc sống này có 1 ý nghĩa tối hậu}.'' -- \cite[p. 9]{Frankl2022}
	
	``We have come to recognize that this is a profoundly religious book. It insists that life is meaningful \& that we must learn to see life as meaningful despite our circumstances. It emphasizes that there is an ultimate purpose to life.'' -- \cite[Preface by \textsc{Harold S. Kushner}, p. 10]{Frankl2013}
	\item ``Chúng ta đã tiến đến chỗ biết được Con người thực sự là gì. Rốt cuộc, con người là hữu thể đã tạo ra phòng hơi ngạt ở Auschwitz; nhưng cũng là hữu thể hiên ngang bước vào phòng hơi ngạt với kinh Lạy Cha hoặc câu kinh Shema Yisrael\footnote{``Kinh Shema Yisrael laf kinh nguyện hằng ngày của tín đồ Do Thái giáo, \& là trích đoạn của sách Đệ Nhị Luật, chương 6:4--9, tạm dịch ``Này hỡi dân Israel! Thiên Chúa là Chúa chúng ta! Là Thiên Chúa duy nhất!''.''} trên môi.'' -- \cite[p. 9]{Frankl2022}
	
	``Our generation is realistic, for we have come to know man as he really is. After all, man is that being who invented the gas chambers of Auschwitz; however, he is also that being who entered those gas chambers upright, with the Lord's Prayer or the \textit{Shema Yisrael} on his lips.'' -- \cite[p. 10]{Frankl2013}
	\item ````Thưa TS. Frankl, cuốn sách của ông luôn nằm trong danh sách những ấn phẩm được đông đảo độc giả đánh giá cao. Ông cảm giác thế nào về thành công này?''. Lúc ấy tôi luôn trả lời rằng trước hết, tôi không hề xem tình trạng ăn khách của cuốn sách là 1 thành tựu hay thành công gì về phía bản thân tôi cho bằng đó là biểu hiện cho nỗi bất hạnh của thời đại: nếu hàng trăm ngàn người tìm kiếm 1 quyển sách mà tiêu đề của nó hứa hẹn sẽ giải quyết vấn đề về ý nghĩa cuộc sống, thì đây chắc hẳn là 1 vấn đề gai góc với họ.'' -- \cite[pp. 11--12]{Frankl2022}
	
	````Dr. Frankl, your book has become a true bestseller\footnote{\textbf{bestseller} [n] a product, usually a book, which is bought by large numbers of people.} -- how do you feel about such a success?'' Whereupon I react by reporting that in the 1st place I do not at all see in the bestseller status of my book an achievement \& accomplishment\footnote{\textbf{accomplishment} [n] \textbf{1.} [countable] an impressive thing that is done or achieved after a lot of work, \textsc{synonym}: \textbf{achievement}; \textbf{2.} [uncountable] the fact of successfully completing something, \textsc{synonym}: \textbf{achievement}; \textbf{3.} [countable, uncountable] a skill or special ability.} on my part but rather an expression of the misery\footnote{\textbf{misery} [n] (plural \textbf{miseries}) \textbf{1.} [uncountable] great physical or mental pain, \textsc{synonym}: \textbf{distress}; \textbf{2.} [uncountable] very poor living conditions, \textsc{synonym}: \textbf{poverty}; \textbf{3.} [countable] something that causes great physical or mental pain; \textbf{4.} [countable] (BE, \textit{informal}) a person who is always unhappy \& complaining.} of our time: if hundreds of thousands of people reach out for a book whose very title promises to deal with the question of a meaning to life, it must be a question that burns under their fingernails.'' -- \cite[Preface to the 1992 Edition, p. 12]{Frankl2013}
	\item ``Trong tôi luôn đinh ninh 1 điều rằng 1 tác phẩm khuyết danh sẽ không thể nào đem lại tên tuổi cho tác giả. Mục đích của tôi đơn thuần chỉ là muốn chuyển tải đến độc giả 1 ví dụ cụ thể, cho thấy rằng trong mọi hoàn cảnh, cuộc sống bao giờ cũng chứa đựng 1 ý nghĩa nào đó, cho dù đó là hoàn cảnh khắc nghiệt nhất.'' -- \cite[pp. 12--13]{Frankl2022}
	
	``At 1st, however, it had been written with the absolute conviction\footnote{\textbf{conviction} [n] \textbf{1.} [countable, uncountable] the act of finding somebody guilty of a crime in court; the fact of having been found guilty; \textbf{2.} [countable, uncountable] a strong opinion or belief; \textbf{3.} [uncountable] the feeling of believing something strongly \& of being sure about it.} that, as an anonymous opus\footnote{\textbf{opus} [n] (plural \textbf{opera}) [usually singular] \textbf{1.} (abbr. \textbf{op.}) a piece of music written by a famous composer \& usually followed by a number that shows when it was written; \textbf{2.} (\textit{formal}) an important piece of literature, etc., especially one that is on a large scale, \textsc{synonym}: \textbf{work}.}, it could never earn its author literary fame. I had wanted simply to convey\footnote{\textbf{convey} [v] \textbf{1.} to communicate information, a message, an idea or a feeling; \textbf{2.} to take carry or transport somebody\texttt{/}something from 1 place to another; \textbf{3.} (\textit{law}) to change the legal owner of a property or piece of land, \textsc{synonym}: \textbf{transfer}.} to the reader by way of a concrete example that life holds a potential meaning conditions, even the most miserable\footnote{\textbf{miserable} [a] \textbf{1.} very unhappy or uncomfortable; \textbf{2.} making you feel very unhappy or uncomfortable, \textsc{synonym}: \textbf{depressing}; \textbf{3.} [only before noun] (\textit{disapproving}) (of a person) always unhappy, unfriendly \& in a bad mood, \textsc{synonym}: \textbf{grumpy}; \textbf{4.} too small in quantity.} ones.'' -- \cite[Preface to the 1992 Edition, p. 12]{Frankl2013}
	\item ``$\ldots$ tôi cảm thấy có trách nhiệm viết ra những gì mình đã trải qua, bởi vì tôi nghĩ rằng nó có thể giúp được những người đang trên bờ tuyệt vọng.'' -- \cite[p. 13]{Frankl2022}
	
	``I therefore felt responsible for writing down what I had gone through, for I thought it might be helpful to people who are prone\footnote{\textbf{prone} [a] \textbf{1.} likely to suffer from something or to do something bad, \textsc{synonym}: \textbf{liable}; \textbf{2.} (\textbf{-prone}) (in adjectives) likely to suffer or do the thing mentioned; \textbf{3.} lying flat with the front of your body touching the ground.} to despair\footnote{\textbf{despair} [n] [uncountable] the feeling of having lost all hope; [v] [intransitive] to stop having any hope that a situation will change or improve.}.'' -- \cite[Preface to the 1992 Edition, p. 12]{Frankl2013}
	\item ``$\ldots$ rất nhiều lần, tôi đã răng bảo các học trò của mình ở châu Âu \& ở Mỹ rằng: ``Đừng nhắm vào thành công -- vì các em càng nhắm vào nó, \& muốn đạt tới nó, thì các em càng dễ trượt qua nó. Vì thành công, cũng giống như hạnh phúc, không thể tìm kiếm mà có; nó phải tự sản sinh ra, \& chỉ có thể xuất hiện khi 1 người cống hiến hết mình, hoặc sống vì người khác hơn là vì bản thân mình. Hạnh phúc sẽ đến, \& thành công cũng sẽ xuất hiện: các em phải để nó diễn ra bằng cách đừng quan tâm đến nó. Tôi muốn các em lắng nghe những gì mà lương tâm của các em ra lệnh phải làm \& tiếp tục thực hiện hết mình. \& các em sẽ thấy rằng về lâu dài -- tôi nhấn mạnh là về \textit{lâu dài} -- thành công sẽ đến với các em bởi vì các em đã \textit{quên} nghĩ về nó!''.'' -- \cite[p. 13]{Frankl2022}
	
	``Again \& again I therefore admonish\footnote{\textbf{admonish} [v] (\textit{formal}) \textbf{1.} \textbf{admonish somebody (for something\texttt{/}for doing something) $|$ $+$ speech} to tell somebody strongly \& clearly that you do not approve of something that they have done, \textsc{synonym}: \textbf{reprove}; \textbf{2.} \textbf{admonish somebody (to do something)} to strongly advise somebody to do something.} my students both in Europe \& in America: ``Don't aim at success -- the more you aim at it \& make it a target, the more you are going to miss it. For success, like happiness, cannot be pursued; it must ensue\footnote{\textbf{ensue} [v] [intransitive] to happen after or as a result of another event, \textsc{synonym}: \textbf{follow}.}, \& it only does so as the unintended\footnote{\textbf{unintended} [a] an unintended effect, result or meaning is one that you did not plan or intend to happen.} side-effect\footnote{\textbf{side effect} [n] [usually plural] \textbf{1.} \textbf{side effect (of something)} an extra \& usually bad effect that a drug has on somebody, as well as curing illness, relieving pain, etc.; \textbf{2.} an unexpected result of a situation or course of action that happens as well as the result you were aiming for.} of one's dedication\footnote{\textbf{dedication} [n] \textbf{1.} [uncountable] the hard work \& effort that somebody puts into an activity or purpose because they think it is important, \textsc{synonym}: \textbf{commitment}; \textbf{2.} [countable] a ceremony that is held to show that a building or an object has a special purpose, especially a religious one; \textbf{3.} [countable] the words that are used at the beginning of a book, piece of music, a performance, etc. to offer it to somebody as a sign of thanks or respect.} to a cause greater than oneself or as the by-product\footnote{\textbf{by-product} [n] \textbf{1.} a substance that is produced during the process of making or destroying something else; \textbf{2.} \textbf{by-product (of something)} a thing that happens, often in an unexpected way, as the result of something else.} of one's surrender\footnote{\textbf{surrender} [v] \textbf{1.} [intransitive, transitive] to admit that you have been defeated \& want to stop fighting; to allow yourself to be caught, taken prisoner, etc; \textbf{2.} [transitive] to give up something\texttt{/}somebody when you are forced to, \textsc{synonym}: \textbf{relinquish}; \textbf{surrender to something $|$ surrender yourself to something} [phrasal verb] to give in to something, such as a strong feeling or an influence.} to a person other than oneself. Happiness must happen, \& the same holds for success: you have to let it happen by not caring about it. I want you to listen to what your conscience commands you to do \& go on to carry it out to the best of your knowledge. Then you will live to see that in the long run -- in the long run, I say! -- success will follow you precisely because you had \textit{forgotten} to think of it.'''' -- \cite[Preface to the 1992 Edition, p. 12]{Frankl2013}
	\item ``Ngay trước khi nước Mỹ tham chiến, tôi nhận được thư mời đến Lãnh sự quán Mỹ đặt tại Vienna để nhận giấy thị thực nhập cảnh đến Mỹ. Cha mẹ tôi rất vui mừng vì họ muốn tôi nhanh chóng rời khỏi nước Áo. Thế nhưng tôi bỗng nhiên do dự: Liệu tôi có thể để cha mẹ ở lại 1 mình đối mặt với số phận, \& chẳng sớm thì muộn, họ cũng sẽ bị đưa đến trại tập trung, hoặc thậm chí bị đưa đến trại hủy diệt? Lẽ nào tôi là 1 người vô trách nhiệm? Liệu tôi có nên đi đến 1 vùng đất trù phú, có những điều kiện thuận lợi giúp tôi nuôi dưỡng đứa con tinh thần của mình -- \textit{Liệu pháp ý nghĩa} (logotherapy)? Hay tôi nên nhận lãnh trách nhiệm của 1 người con hiếu thuận với cha mẹ -- những người đã không tiếc cả mạng sống của mình để bảo vệ tôi? Tôi đã đắn đo, trăn trở rất nhiều nhưng vẫn không thể đưa ra quyết định cuối cùng. Đây đúng là tình huống khó khăn như khi người ta chỉ được ước 1 lần.'' -- \cite[p. 14]{Frankl2022}
	
	``Shortly before the United States entered WWII, I received an invitation to come to the American Consulate in Vienna to pick up my immigration visa. My old parents were overjoyed\footnote{\textbf{overjoyed} [a] [not before noun] extremely happy or pleased, \textsc{synonym}: \textbf{delighted}.} because they expected that I would soon be allowed to leave Austria. I suddenly hesitated, however. The question beset\footnote{\textbf{beset} [v] [usually passive] (\textit{formal}) to affect somebody\texttt{/}something in an unpleasant or harmful way.} me: could I really afford to leave my parents alone to face their fate, to be sent, sooner or later, to a concentration camp, or even to a so-called extermination camp? Where did my responsibility lie? Should I foster\footnote{\textbf{foster} [v] \textbf{1.} \textbf{foster something} to encourage something to develop, \textsc{synonym}: \textbf{promote}; \textbf{2.} \textbf{foster somebody} (\textit{especially British English}) to take another person's child into your home for a period of time, without becoming the child's legal parent; [a] [only before noun] used with some nouns in connection with the fostering of a child.} my brain child, logotherapy, by emigrating\footnote{\textbf{emigrate} [v] [intransitive] to leave your own country to go \& live permanently in another country.} to fertile\footnote{\textbf{fertile} [a] \textbf{1.} (of land or soil) that plants grow well in, \textsc{opposite}: \textbf{infertile}; \textbf{2.} (of people, animals or plants) that can produce babies, young animals, fruit or new plants, \textsc{opposite}: \textbf{infertile}; \textbf{3.} [usually before noun] that encourages activity; that produces results.} soil\footnote{\textbf{soil} [n] [uncountable, countable] the top layer of the earth in which plants grow.} where I could write my books? Or should I concentrate on my duties as a real child, the child of my parents who had to do whatever he could to protect them? I pondered\footnote{\textbf{ponder} [v] [intransitive, transitive] (\textit{formal}) to think about something carefully for a period of time, \textsc{synonym}: \textbf{consider}.} the problem this way \& that but could not arrive at a solution; this was the type of dilemma that made one wish for ``a hint from Heaven,'' as the phrase goes.'' -- \cite[Preface to the 1992 Edition, p. 13]{Frankl2013}
\end{enumerate}

%------------------------------------------------------------------------------%

\chapter{\cite{Grant2013, Grant2022}. Give and Take: Why Helping Others Drives Our Success, A Revolutionary Approach to Success}

%------------------------------------------------------------------------------%

\chapter{\cite{Ruiz2011}. The Four Agreements: A Practical Guide to Personal Freedom}

%------------------------------------------------------------------------------%

\printbibliography[heading=bibintoc]
	
\end{document}