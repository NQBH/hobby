\documentclass{article}
\usepackage[backend=biber,natbib=true,style=authoryear]{biblatex}
\addbibresource{/home/nqbh/reference/bib.bib}
\usepackage{tocloft}
\renewcommand{\cftsecleader}{\cftdotfill{\cftdotsep}}
\usepackage[colorlinks=true,linkcolor=blue,urlcolor=red,citecolor=magenta]{hyperref}
\usepackage{algorithm,algpseudocode,amsmath,amssymb,amsthm,float,graphicx,mathtools}
\allowdisplaybreaks
\numberwithin{equation}{section}
\newtheorem{assumption}{Assumption}[section]
\newtheorem{conjecture}{Conjecture}[section]
\newtheorem{corollary}{Corollary}[section]
\newtheorem{definition}{Definition}[section]
\newtheorem{example}{Example}[section]
\newtheorem{lemma}{Lemma}[section]
\newtheorem{notation}{Notation}[section]
\newtheorem{principle}{Principle}[section]
\newtheorem{problem}{Problem}[section]
\newtheorem{proposition}{Proposition}[section]
\newtheorem{question}{Question}[section]
\newtheorem{remark}{Remark}[section]
\newtheorem{theorem}{Theorem}[section]
\usepackage[left=0.5in,right=0.5in,top=1.5cm,bottom=1.5cm]{geometry}
\usepackage{fancyhdr}
\pagestyle{fancy}
\fancyhf{}
\lhead{\small Sect.~\thesection}
\rhead{\small\nouppercase{\leftmark}}
\renewcommand{\sectionmark}[1]{\markboth{#1}{}}
\cfoot{\thepage}
\def\labelitemii{$\circ$}

\title{A PhD Is Not Enough! A Guide to Survival in Science}
\author{Peter J. Feibelman}
\date{\today}

\begin{document}
\maketitle
\tableofcontents
\vspace{5mm}
\begin{quotation}
	\textit{``It took me over 40 years to learn from experience what can be learned in 1 hour from this guide.''} -- Carl Djerassi
\end{quotation}

%------------------------------------------------------------------------------%

\section*{Preface: What This Book Is About}
``

'' -- \cite{Feibelman2011}

%------------------------------------------------------------------------------%

\section{Do You See Yourself in This Picture?}

\begin{center}
	\textsf{\textbf{A set of nonfiction vignettes illustrating some of the ways that young scientists make their lives more unpleasant than necessary or fail entirely to establish themselves in a research career.}}
\end{center}

%------------------------------------------------------------------------------%

\section{Advice from a Dinosaur?}

\begin{center}
	\textsf{\textbf{Can you expect someone to be an effective mentor who emerged into the scientific marketplace in a world that looked very different?}}
\end{center}

%------------------------------------------------------------------------------%

\section{Importance Choices: A Thesis Adviser, a Postdoctoral Job}

\begin{center}
	\textsf{\textbf{A discussion of what to consider: young adviser versus an older one, a superstar versus a journeyman, a small group versus a ``factory.'' Understanding \& attending to your interests as a postdoc.}}
\end{center}

%------------------------------------------------------------------------------%

\section{Giving Talks}

\begin{center}
	\textsf{\textbf{Preparing talks that will make people want to hire \& keep you \& that will make the information you present easy to assimilate.}}
\end{center}

%------------------------------------------------------------------------------%

\section{Writing Papers: Publishing Without Perishing}

\begin{center}
	\textsf{\textbf{Why it is important to write good papers. When to write up your work, how to draw the reader in, how to draw attention to your results.}}
\end{center}
``The negative connotation of the clich\'e  \textit{publish or perish} is seriously misplaced. Publication is a key component of your research efforts. It is widely accepted that a scientific endeavor is not complete until it has been written up. The exercise of putting your reasoning down on paper will frequently lead you to refine your thoughts, to detect flaws in your arguments, \& perhaps to realize that your work has wide significance than you had originally imagined. Publication also has strategic significance. As a beginning scientist, not only do you work long hours for long pay, but your job security is anything but assured. To succeed, you must make your talents well known \& widely appreciated. Publishing provides you with an important way to accomplish that. Your papers, on public view around the world, represent not only your product but also your r\'esum\'e. Compelling, thoughtful, well-written articles are timeless advertisements for yourself. You can imagine that a sloppy r\'esum\'e is not worth preparing. A premature or slapdash publication is far worse. It will remain available to readers indefinitely. These thoughts raise the 2 basic questions addressed in the present chapters: \textit{When} should one write a paper, \& \textit{how} should one write it?'' -- \cite[pp. 53--54]{Feibelman2011}

\subsection{Timing}
``Generally, articles are written too soon in response to the fear that one's competitors will publish 1st or as a result of intellectual laziness (i.e., inattention to important details). Papers are written too late because of the fear of publishing a blunder or because of writer's block. Overcoming these fears \& frailties is necessary for \textit{everyone} in science. At the very least, the knowledge that they are not yours alone may help you deal with them. (Read Carl Djerassi's novel \textit{Cantor's Dilemma} [New York: Penguin Books, 1991] for a poignant exposition of the problem of when and what to publish.)

Planning your research as a series of relatively short, complete projects (cf. Chap. 9) is the best way to achieve a disciplined publication schedule, one that serves your interests in scientific priority, self-advertisement, \& job security. Even though you are working toward an important long-term goal, you report each project as an independent piece of work that has produced a new kernel of knowledge (only half-jokingly a ``publon,'' a quantum of publication\footnote{The concept of the ``publon'' emerged from the graduate student minds of M. J. Weber, now at the University of Virginia, \& W. Eckhart, now at the Salk Institute.}). In the introduction to each paper of a series, you place the work reported in the context of the long-term goal, to which you thereby lay claim, \& you explain how the present results take you a step closer. If your project turns out to be as significant as you had hoped, after you have published several papers in the series, no doubt you will be asked to write a review. \textit{This} will provide you with an appropriate forum for a long, definitive article, one that will be widely referred to \& will help to make your name in science.

There are many advantages to writing up your work as a series of short papers. Managers \& funding agencies need concrete evidence that they have hired personnel \& spent money wisely. Nothing is more helpful in this regard than the list of publications their wisdom has fostered. Of course, they will be pleased if you eventually realize a long-term research goal. However, funding cycles are typically 2 or 3 years (cf. Chap. 8), \& renewal of junior scientific positions occurs on a similar time scale. Therefore, deans, research directors, \& contract managers cannot wait for your long-term dreams to come true. They need published evidence of your progress on an ongoing basis.

By writing numerous, relatively short articles, you can keep your name in the spotlight. The titles, abstracts, \& authorship of your new papers will show up in electronic databases, generally updated weekly. Such search engines as \url{scholar.google.com}, \url{www.osti.gov/eprints},
\& \url{www.scirus.com} will readily lead the community to manuscripts you have posted on \url{arXiv.org}, \url{precedings.nature.com}, or any of a host of other preprint servers. The number of citations of a long publication list increases more rapidly than that of a short list.

You mustn't be overly cynical about these facts of scientific life. If you attempt to achieve name-recognition by padding your publication list with repetitive papers, your efforts will soon reap scorn rather than admiration. Still, the little admiration you gain for publishing an awesome magnum opus in a single paper is surely not worth the risk that this publication strategy poses to your job security.

If you publish frequently, you are less likely to be ``scooped.'' The longer you hold back reporting your results, particularly if they are important, the greater the chance some other group will beat you into print. You do need to develop an appreciation for when a piece of work is complete enough to be written up. If the logic of a manuscript is clearly missing an important piece of confirmatory evidence, submitting it to a journal is likely to cause you endless, painful interactions with referees. This is the time to hold back. (Among other problems, the referees may very well be your competitors. Their own publication strategy is likely to be affected by their appreciation of where your incomplete work stands.) On the other hand, if you \textit{have} completed a project, the sooner you get it into the hands of a journal, the better the chances are that you will get credit for your accomplishment.

Writing a paper that presents 1 new idea or result is much easier than writing a long, complex article. This is a reasonable way to address the problem of writer's block. Much of the introduction to a shorter paper can be prepared, at least mentally, when the long-term research project is originally proposed. The organization of a paper is simpler if there is not so much material to present, \& it is also relatively easy to explain the conclusions in that case.

Referees are generally busy people \& prefer to review short papers. You are likely to receive a more thoughtful \& positive report on a short manuscript than on a long one. Shorter papers are of course not only easier on referees. They also can be read \& assimilated more easily by the scientific community at large.

Writing up individual kernels of new research should have some appeal for the perfectionist. It is easier to get everything right when one is dealing with a small project than when publishing the results of a major, complex effort.

Eventually, of course, all the significant details of a research project need to be reported in an archival journal so that others may repeat \& confirm the validity of the new science. Writing such technical papers is an important exercise, \& one that will win you credit from your peers if you do it well. On the other hand, in most cases the writing of such papers can be carried out at leisure.'' -- \cite[pp. 54--58]{Feibelman2011}

\subsection{Writing Compelling Papers}

%------------------------------------------------------------------------------%

\section{From Here to Tenure: Choosing a Career Path}

\begin{center}
	\textsf{\textbf{An unsentimental comparison of the merits of jobs in academia, industry, \& in government laboratories.}}
\end{center}

%------------------------------------------------------------------------------%

\section{Job Interviews}

\begin{center}
	\textsf{\textbf{What will happen on your interview trip; the questions you had better be prepared to answer.}}
\end{center}

%------------------------------------------------------------------------------%

\section{Getting Funded}

\begin{center}
	\textsf{\textbf{What goes into an effective grant proposal; how \& when to start writing one.}}
\end{center}

%------------------------------------------------------------------------------%

\section{Establishing a Research Program}

\begin{center}
	\textsf{\textbf{Tuning your research efforts to your own capabilities \& your situation in life; e.g., why not to start a 5-year project when you have a 2-year postdoctoral appointment.}}
\end{center}

%------------------------------------------------------------------------------%

\section{A Survival Checklist}

\begin{center}
	\textsf{\textbf{Do not attempt a takeoff before being sure the flaps are down.}}
\end{center}

%------------------------------------------------------------------------------%

\section{Afterthoughts}

\begin{center}
	\textsf{\textbf{A behaviorist approach to professional success.}}
\end{center}

%------------------------------------------------------------------------------%

\printbibliography[heading=bibintoc]
	
\end{document}