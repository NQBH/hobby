\documentclass{article}
\usepackage[backend=biber,natbib=true,style=authoryear]{biblatex}
\addbibresource{/home/nqbh/reference/bib.bib}
\usepackage{tocloft}
\renewcommand{\cftsecleader}{\cftdotfill{\cftdotsep}}
\usepackage[colorlinks=true,linkcolor=blue,urlcolor=red,citecolor=magenta]{hyperref}
\usepackage{algorithm,algpseudocode,amsmath,amssymb,amsthm,float,graphicx,mathtools}
\allowdisplaybreaks
\numberwithin{equation}{section}
\newtheorem{assumption}{Assumption}[section]
\newtheorem{conjecture}{Conjecture}[section]
\newtheorem{corollary}{Corollary}[section]
\newtheorem{definition}{Definition}[section]
\newtheorem{example}{Example}[section]
\newtheorem{lemma}{Lemma}[section]
\newtheorem{notation}{Notation}[section]
\newtheorem{principle}{Principle}[section]
\newtheorem{problem}{Problem}[section]
\newtheorem{proposition}{Proposition}[section]
\newtheorem{question}{Question}[section]
\newtheorem{remark}{Remark}[section]
\newtheorem{theorem}{Theorem}[section]
\usepackage[left=0.5in,right=0.5in,top=1.5cm,bottom=1.5cm]{geometry}
\usepackage{fancyhdr}
\pagestyle{fancy}
\fancyhf{}
\lhead{\small Sect.~\thesection}
\rhead{\small\nouppercase{\leftmark}}
\renewcommand{\sectionmark}[1]{\markboth{#1}{}}
\cfoot{\thepage}
\def\labelitemii{$\circ$}

\title{A PhD Is Not Enough! A Guide to Survival in Science}
\author{Peter J. Feibelman}
\date{\today}

\begin{document}
\maketitle
\tableofcontents
\vspace{5mm}
\begin{quotation}
	\textit{``It took me over 40 years to learn from experience what can be learned in 1 hour from this guide.''} -- Carl Djerassi
\end{quotation}

%------------------------------------------------------------------------------%

\section*{Preface: What This Book Is About}
``

'' -- \cite{Feibelman2011}

%------------------------------------------------------------------------------%

\section{Do You See Yourself in This Picture?}

\begin{center}
	\textsf{\textbf{A set of nonfiction vignettes illustrating some of the ways that young scientists make their lives more unpleasant than necessary or fail entirely to establish themselves in a research career.}}
\end{center}

%------------------------------------------------------------------------------%

\section{Advice from a Dinosaur?}

\begin{center}
	\textsf{\textbf{Can you expect someone to be an effective mentor who emerged into the scientific marketplace in a world that looked very different?}}
\end{center}

%------------------------------------------------------------------------------%

\section{Importance Choices: A Thesis Adviser, a Postdoctoral Job}

\begin{center}
	\textsf{\textbf{A discussion of what to consider: young adviser versus an older one, a superstar versus a journeyman, a small group versus a ``factory.'' Understanding \& attending to your interests as a postdoc.}}
\end{center}

%------------------------------------------------------------------------------%

\section{Giving Talks}

\begin{center}
	\textsf{\textbf{Preparing talks that will make people want to hire \& keep you \& that will make the information you present easy to assimilate.}}
\end{center}

%------------------------------------------------------------------------------%

\section{Writing Papers: Publishing Without Perishing}

\begin{center}
	\textsf{\textbf{Why it is important to write good papers. When to write up your work, how to draw the reader in, how to draw attention to your results.}}
\end{center}
``The negative connotation of the clich\'e  \textit{publish or perish} is seriously misplaced. Publication is a key component of your research efforts. It is widely accepted that a scientific endeavor is not complete until it has been written up. The exercise of putting your reasoning down on paper will frequently lead you to refine your thoughts, to detect flaws in your arguments, \& perhaps to realize that your work has wide significance than you had originally imagined. Publication also has strategic significance. As a beginning scientist, not only do you work long hours for long pay, but your job security is anything but assured. To succeed, you must make your talents well known \& widely appreciated. Publishing provides you with an important way to accomplish that. Your papers, on public view around the world, represent not only your product but also your r\'esum\'e. Compelling, thoughtful, well-written articles are timeless advertisements for yourself. You can imagine that a sloppy r\'esum\'e is not worth preparing. A premature or slapdash publication is far worse. It will remain available to readers indefinitely. These thoughts raise the 2 basic questions addressed in the present chapters: \textit{When} should one write a paper, \& \textit{how} should one write it?'' -- \cite[pp. 53--54]{Feibelman2011}

\subsection{Timing}
``Generally, articles are written too soon in response to the fear that one's competitors will publish 1st or as a result of intellectual laziness (i.e., inattention to important details). Papers are written too late because of the fear of publishing a blunder or because of writer's block. Overcoming these fears \& frailties is necessary for \textit{everyone} in science. At the very least, the knowledge that they are not yours alone may help you deal with them. (Read Carl Djerassi's novel \textit{Cantor's Dilemma} [New York: Penguin Books, 1991] for a poignant exposition of the problem of when and what to publish.)

Planning your research as a series of relatively short, complete projects (cf. Chap. 9) is the best way to achieve a disciplined publication schedule, one that serves your interests in scientific priority, self-advertisement, \& job security. Even though you are working toward an important long-term goal, you report each project as an independent piece of work that has produced a new kernel of knowledge (only half-jokingly a ``publon,'' a quantum of publication\footnote{The concept of the ``publon'' emerged from the graduate student minds of M. J. Weber, now at the University of Virginia, \& W. Eckhart, now at the Salk Institute.}). In the introduction to each paper of a series, you place the work reported in the context of the long-term goal, to which you thereby lay claim, \& you explain how the present results take you a step closer. If your project turns out to be as significant as you had hoped, after you have published several papers in the series, no doubt you will be asked to write a review. \textit{This} will provide you with an appropriate forum for a long, definitive article, one that will be widely referred to \& will help to make your name in science.

There are many advantages to writing up your work as a series of short papers. Managers \& funding agencies need concrete evidence that they have hired personnel \& spent money wisely. Nothing is more helpful in this regard than the list of publications their wisdom has fostered. Of course, they will be pleased if you eventually realize a long-term research goal. However, funding cycles are typically 2 or 3 years (cf. Chap. 8), \& renewal of junior scientific positions occurs on a similar time scale. Therefore, deans, research directors, \& contract managers cannot wait for your long-term dreams to come true. They need published evidence of your progress on an ongoing basis.

By writing numerous, relatively short articles, you can keep your name in the spotlight. The titles, abstracts, \& authorship of your new papers will show up in electronic databases, generally updated weekly. Such search engines as \url{scholar.google.com}, \url{www.osti.gov/eprints},
\& \url{www.scirus.com} will readily lead the community to manuscripts you have posted on \url{arXiv.org}, \url{precedings.nature.com}, or any of a host of other preprint servers. The number of citations of a long publication list increases more rapidly than that of a short list.

You mustn't be overly cynical about these facts of scientific life. If you attempt to achieve name-recognition by padding your publication list with repetitive papers, your efforts will soon reap scorn rather than admiration. Still, the little admiration you gain for publishing an awesome magnum opus in a single paper is surely not worth the risk that this publication strategy poses to your job security.

If you publish frequently, you are less likely to be ``scooped.'' The longer you hold back reporting your results, particularly if they are important, the greater the chance some other group will beat you into print. You do need to develop an appreciation for when a piece of work is complete enough to be written up. If the logic of a manuscript is clearly missing an important piece of confirmatory evidence, submitting it to a journal is likely to cause you endless, painful interactions with referees. This is the time to hold back. (Among other problems, the referees may very well be your competitors. Their own publication strategy is likely to be affected by their appreciation of where your incomplete work stands.) On the other hand, if you \textit{have} completed a project, the sooner you get it into the hands of a journal, the better the chances are that you will get credit for your accomplishment.

Writing a paper that presents 1 new idea or result is much easier than writing a long, complex article. This is a reasonable way to address the problem of writer's block. Much of the introduction to a shorter paper can be prepared, at least mentally, when the long-term research project is originally proposed. The organization of a paper is simpler if there is not so much material to present, \& it is also relatively easy to explain the conclusions in that case.

Referees are generally busy people \& prefer to review short papers. You are likely to receive a more thoughtful \& positive report on a short manuscript than on a long one. Shorter papers are of course not only easier on referees. They also can be read \& assimilated more easily by the scientific community at large.

Writing up individual kernels of new research should have some appeal for the perfectionist. It is easier to get everything right when one is dealing with a small project than when publishing the results of a major, complex effort.

Eventually, of course, all the significant details of a research project need to be reported in an archival journal so that others may repeat \& confirm the validity of the new science. Writing such technical papers is an important exercise, \& one that will win you credit from your peers if you do it well. On the other hand, in most cases the writing of such papers can be carried out at leisure.'' -- \cite[pp. 54--58]{Feibelman2011}

\subsection{Writing Compelling Papers}
``A journal article should present a careful \& relatively complete account of your research. However, it is all too easy to write an accurate description of your work that attracts no attention \& that adds little to your scientific reputation, \textit{even when your results are significant}. Learning to write articles that people will read \& remember will make you a more effective scientist. It will also enhance your chances for survival as a researcher.

The structure of a news article is a good model to follow in preparing a scientific publication. Newspaper readers, like your research colleagues, rarely have much time for acquiring new information. This is just the reason that news articles present a story several times, in increasing levels of detail. Their headlines, equivalent to the titles of your scientific papers, are there to draw readers in by providing a succinct description of what is noteworthy. Scientists attempting to keep up in a world of information overload often do no more than skim the tables of contents of the leading journals in their field or conduct electronic keyword searches. You can help direct them to your new paper by taking the time to prepare an accurate \& compelling title, concise yet incorporating the most important keywords. (``Cute'' should be avoided, as a rule.)

The abstract of a paper corresponds to the 1st paragraph of a news item. It summarizes the main information, what the important results are, \& what methods you used to obtain them. Numerous journals place a word limit (e.g., 75 words) on the abstract. It is a good idea to impose such a limit on yourself whether or not the journal does. An abstract that is brief \& to the point has a better chance of being read. A wordy one which reads like the introduction to or the body of a paper, will lose readers.

As in the case of titles, it is worth remembering that abstracts circulate more widely than the papers they summarize. They are the 1st item to pop up when one searches journal content \& generally available without charge, even when seeing a full article requires a subscription. A well-written abstract may thus make the difference between someone's downloading your full text or emailing you for a copy, rather than just moving on.

The introduction to a paper is where you tell your story, possibly illustrating the text with an important figure or some key results, but without going into great detail. Here is where you want to explain why your project was an important one to undertake \& how your results make a difference to the way we understand the world. Many busy scientists read only the introduction \& conclusion sections of papers, leaving the technical details for another time. Therefore, it is a good idea to highlight your results -- e.g., by placing your most important figure in the introduction. Even if your readers never take the time to plow through the complete description of your work in the body of your paper, they may think enough of the information in your introduction to make sure to catch your talk at the next scientific meeting.

Virtually everyone finds that writing the introduction to a paper is the most difficult task. It is easy to report the procedures you followed \& to describe the data you obtained. The hard part of paper writing is drawing the reader in. My solution to this problem is to start thinking about the 1st paragraph of an article \textit{when I begin a project rather than when I complete it}. I would not embark on a scientific effort if I didn't think it was important \& that my work would answer a question of rather wide interest. The reasons that I found the project in question interesting enough to work on provide half the material I need for my introduction. The remainder is a summary of my key results. The decision to start writing a paper is generally based on recognizing that a kernel of knowledge has been produced. In my introduction, I want to let my reader know what this new information is, in a nutshell, \& why it is worth reading about. Sitting at the word processor, I imagine I am on the phone with a scientist friend whom I haven't spoken to in sometime. He asks me what I have been doing recently. I write down my imagined response. If, when you try this, you feel an attack of writer's block coming on, turn on a recording device \& actually call a friend. It works.

Incidentally, if you know why you have carried out a scientific project \& what makes your results interesting, there is no reason that your paper should start with an inane clich\'e, such as, ``Recently there has been a resurgence of interest in $\ldots$ (whatever the topic),'' which bothers me every time I see it. If you have been working on a project for several months or a year solely because \textit{other} people are interested in it, you have a lot to learn about problem selection. (In this case, see Chap. 9 for some help. Do not pass go. Do not collect your next paycheck.) Before you start on a research effort, you must understand why it is important, \& in the introduction to your publication on the subject, this is just what you need to explain.

In writing your introduction, as well as the body of your paper, it is essential to place your work in context, not only by explaining what you did \& why but also by citing the relevant literature. This is important, not only to provide your readers with a way of understanding your area of research, but also because your scientific colleagues are very eager to get credit for their achievements. (This is not just vanity. Scientists' careers are built on the perceived importance or usefulness of their research results.) You have much to gain \& little to lose by scrupulously citing your competitor's work. I said above that many busy scientists read only the introduction \& conclusion sections of papers. Even more move directly from the title \& abstract to the references, to see if their work is cited. If someone's papers are not mentioned there but should be, you risk losing a potential friend or at least some respect.

I would add that an excellent way to keep up with developments in your field is to check, from time to time, who is citing your own papers. A ``citation index,'' such as is available on the ISI Web of Science${}^{\rm SM}$, makes this an easy task. Bear in mind as you do this that if checking citations is how people in your field keep up, an article you have written that fails to cite their work is more likely to go unnoticed.

In revising \& editing your article before submitting it, you should constantly be asking yourself if you have dealt with all the loose ends in your logic. \textit{Are there arguments you have thought about \& used but not written into your text? Are you wishy-washy about inferences you have drawn, instead of forceful, because there are missing links in the logic?} If so, you either need to work a little longer before writing your paper, or be forthright about what is conjecture \& what has actually been proven. Even if the referee does not catch the weak points of your manuscript, you must not forget that your paper will be on public view for a long time. Intellectual honesty is accordingly a very good policy. This is not to say you should be such a perfectionist that you never feel comfortable declaring a project done \& ready to be published, but rather that you should own up, in print, to what you think might be weak links in your reasoning. This is a service to the community, in that it points to further research directions. It shows the world that you are a thoughtful \& forthright individual. Importantly, it also provides you an out if your reasoning is later shown to be incorrect.

The format of the body of a paper is often dictated by the journal where it will be submitted. Within the journal's constraints, however, the key to organizing your work is to make your text read like a story. Often it is a good idea to relegate detailed discussion of a technical aspect of the work to an appendix. That way, experts or interested parties can try to understand your arguments in full detail, whereas others do not have to guess how much of the text to skip to move on to the next idea.

Keep in mind that the function of a journal article is to communicate, not simply to indicate how wonderful your results are. In principle, a paper should provide enough information that an interested reader would be able to reproduce your work. It is your responsibility to ensure that the necessary information is made available, at the same time as you try to make your paper as snappy \& readable as you can.'' -- \cite[pp. 54--64]{Feibelman2011}

\subsection{Snappy Papers}
``In archaic times, say 30 years ago, you generally had to write your papers as though the work had actually been done by someone else. You were discouraged from using the personal pronoun ``I'' in favor of ``we'' or, even worse, ``one.'' Journals seemed to require writing papers in the passive mood, as in ``the data were obtained using the following novel method'' rather than ``I developed the following novel method to obtain the data.'' More recently, it has become possible to drop the phoniness of this style \& to reveal in your writing that \textit{you} actually did the work you are reporting. I greatly prefer the more straightforward style \& recommend that you use it.

People of a mathematical bent often connect the sentences in their papers with such words as \textit{now, then, thus, however, therefore, whence, hence}, \& so forth. If you want your text to be readable to the non-pedantic, you should be very sparing in using them. Go over your 1st draft \& challenge yourself to see how many of these connectives you can remove without undermining the logic of your argument.

In this era of speedy desktop computers \& full-featured graphics programs, there are few excuses for omitting evocative figures from a paper. A picture may be worth more than a thousand words in a scientific article, particularly if the thousand words are not read, but the thoughtfully prepared figure is examined \& the information it reports absorbed. This does mean it is important not to prepare figures that are too cluttered. If they offend the eye, they may be ignored along with the thousands of words.

Some journals restrict the length of articles. This typically forces one to go back through the 1st draft of a manuscript to rewrite more economically. In preparing the 1st draft, it is a good idea to be as generous as possible with words. You should write down everything that comes to mind as relevant. This may not be easy but helps get all the logic on paper. (Again, get out the voice recorder if you tend to be stingy with words.) If you have written a copious text, the exercise of cutting back may be more difficult but is less likely to lead to a paper whose flow is compromised by the absence of something important. I recommend the approach of writing generously \& then editing severely in all cases -- i.e., whether or not the journal in question imposes restrictions on manuscript length. The exercise of rewriting as concisely as possible leads to more readable text \& thus to text that is read more widely.

As in the preparation of a seminar, the last section of a paper should provide not just a summary of the results reported but also some idea of how they might affect the direction of future research. The goal of the conclusions section is to leave your reader thinking about how your work affects his or her own research plans. Good science opens new doors.'' -- \cite[pp. 64--66]{Feibelman2011}

\subsection{Referees}
``Last, because arguments with journal referees can take many months to settle, \& can be very frustrating, it is a good idea to forestall them by having your manuscripts reviewed locally, by 1 or 2 of your colleagues, before submission. If you have chosen your local reviewer well, you may discover the weak points in your article in a matter of days rather than months. If English is not your mother tongue (\& if you are writing for an English-language journal) it is even more important to have your paper reviewed \& edited by a colleague, one whose English is near perfect. Your readers, including your journal's referees, are human \& thus impatient to some degree. The easier you can make their task, the better will be their response to your efforts.

Incidentally, as one who referees many papers, I much prefer receiving a cogent, well-written manuscript that I can learn from than the other kind. A paper that I enjoy reading disposes me favorably toward the author. Your referee may be your paper's most careful reader ever. Making a good impression on this anonymous potential employer is not a bad idea!

If your referee does have serious complaints about your article, getting angry is not a productive response. A better idea is to consider why this thoughtful expert did not follow your argument \& agree with it. If on reflection you believe that your results are correct \& that the referee has simply misunderstood them, it is likely that spending some time revising your text will not only persuade the referee to recommend that your paper be published but will also ultimately make your ideas less confusing to your journal's general readership.'' -- \cite[pp. 66--68]{Feibelman2011}

\subsection{Additional Reading}

\begin{itemize}
	\item Carter, Sylvester P. \textit{Writing for Your Peers: The Primary Journal Paper}. New York: Praeger, 1987.
	\item Alley, Michael. \textit{The Craft of Scientific Writing}. 3rd ed. New York: Springer Science \& Business Media, 1996.
	\item Booth, Vernon. \textit{Communicating in Science: Writing a Scientific Paper \& Speaking at Scientific Meetings}. 2nd ed. New York: Cambridge University Press, 1993.
\end{itemize}

%------------------------------------------------------------------------------%

\section{From Here to Tenure: Choosing a Career Path}

\begin{center}
	\textsf{\textbf{An unsentimental comparison of the merits of jobs in academia, industry, \& in government laboratories.}}
\end{center}
``As a scientist, your goals are to make exciting discoveries, to change the way your colleagues \& maybe even the public at large view the world, \& generally to improve people's lives. However, need I remind you, you will remain a human being, with human needs, even while you are pushing back the frontiers of ignorance. No matter how romantically you view your role in research, you will not be happy without a secure, well-paid job. You will want help in accomplishing your research goals \& recognition for your achievements. You will probably want to see your family on a regular basis \&, more generally, to have enough free time to engage in activities outside your professional life.

It is all too easy to lock yourself into a situation where 1 or more of such basic desires will not be satisfied. This may adversely affect your productivity, your family life, \& your ability to enjoy yourself. Thus it is important to consider rationally, \& in advance, not only the benefits \& disadvantages of the various kinds of scientific positions -- academic, industrial, \& governmental -- but also the merits of the different roads to permanent employment.

Economic conditions may limit your choices, but if you are fortunate enough to have more than 1 job possibility, this exercise will save you considerable stress. It may have a significant effect on your financial well-being. It may save your marriage. I harbor a secret hope: If enough of you start to act rationally, the system may eventually be rationalized.

It is only natural to adopt as role models the people one encounters in one's formative years. For this reason, many -- perhaps most of us -- finish graduate school dreaming of an academic career. For some, the academic life may be ideal. For many, it is not. Even if being a professor \textit{is} the right goal, however, it is far from clear that rising up the academic ladder is the most desirable way to get there. My recommendations \& the reasons for them are the subject of what follows.'' -- \cite[pp. 69--71]{Feibelman2011}

\subsection{The Pluses \& Minuses of a Job in Academia}
``The idea that a university is an ivory tower is commonplace. The academic freedom embodied in the granting of tenure was originally supposed to protect the professoriat from political repercussions against expressions of minority views of the world. However, tenure is in itself a uniquely desirable \& economically significant benefit.

\textit{Who wouldn't want the ultimate in job security?} As a tenured professor, if you fulfill minimal performance requirements (e.g., teaching a class every semester) \& maintain at least minimal moral standards (love affairs with your students are sometimes frowned upon), \& if your university doesn't shut down your department entirely in response to severe economic stress, you have a guaranteed paycheck. In face, universities have long since recognized the economic significance of tenure. University salaries would certainly have to be higher if professors were subject to being laid off.

Tenure is a form of financial independence \& thus conveys corollary benefits. A university professor chooses research topics \& collaborators at will. No boss is empowered to say what to work on or to decide who will work with whom. In principle, the pace of research is also up to the professor. If energetic \& ambitious, an established professor, together with a group of students \& postdocs, may produce a dozen publications a year, or more. A ``scholar'' may publish many fewer, might be poorly funded, \& may not have much of a group. The department chair or the dean may complain, but the scholarly professor will still receive a paycheck.

Although tenure \& its corollaries are the unique benefits of a professorship, they are far from the only attractive features of the job. Professors can anticipate the respect not only of class after class of students, who pay a great deal of money to be exposed to what they have to say, but also of the community at large.

Typically, professors are free to sell their services as consultants, perhaps 1 day per week, to supplement their salary. Many science professors find private companies to develop the fruits of their research \& sell them for their own profit. Others write textbooks on university time \& pay, \& then are allowed to reap the royalties for themselves.

Because classes are held only 9 months of the year, the remaining 3 are in principle a very long annual vacation or at worst, unprogrammed time. Sabbaticals are typically part of a university contract. Every several years, professors can look forward to 6 months or a year at a distant \& often exciting location where they can recharge their intellectual batteries, learn a new field, write a book, or basically do what they please -- \& get paid for it!

Given that the job has all these wonderful benefits, you might be surprised that many professors complain about the demands of their work \& that many scientists are happy not to be members of the professoriat. What, then, are the disadvantages of living in the ivory tower?

Probably the most widespread complaint is that a professor rarely has time to set foot in the lab \& to do the scientific research that used to be so much fun. Professors have so many responsibilities \& have to work so hard to fulfill them that their scientific work is mostly vicarious -- it's the students \& postdocs who do the hands-on research. To say the least, professors end up with little time for themselves. There are thankfully few tenured individuals who cynically view their permanent slot as an opportunity to do nothing (although there is generally more than enough ``dead wood'' in a department to embitter the assistant professor not promoted to tenure.) The professors I know work many more than 8 hours a day \& rarely take more than a week or 2 of vacation each year, even though in principle they could take much more.

A professorship is effectively several jobs rolled into one. A professor is of course a teacher. Although there are many stories of professors whose lecture noes are yellowed with age, taking the job of teaching seriously means devoting considerable effort to making classes coherent, informative, \& up-to-date. One needs to prepare homework sets \& exams \& to develop meaningful lab exercises. One must also spend time with students during office hours. A professor is expected to be a good departmental citizen. This means attending a significant number of meetings to decide policies \& to discuss hiring \& promotions. The ambitious professor spends a great deal of time as a manager. This means writing grant proposals, traveling to Washington to meet with grant administrators, fighting for lab space, hiring \& firing students \& postdocs, \& so forth. Being an active scientific citizen, which includes refereeing manuscripts \& grant proposals, preparing \& giving lectures at other institutions, \& attending conferences, also absorbs hours. Consulting \& textbook writing come on top of that. It does not take a genius to see what professors have little time for reading a novel or playing with the kids.

A job with many demands provides many opportunities for frustration. When economic times are tough, the chances of getting a proposal funded or renewed are reduced. If you have no grant money, you cannot afford to pay students \& postdocs. If you cannot spare much time to do research yourself, this means your research program will grind to a halt. Your ensuing lack of productivity will then make it harder for you to acquire funding in the future, a most unpleasant feedback mechanism. Apart from keeping yourself alive as a researcher, if your funding dries up, you may find yourself struggling to make ends meet. Typically, a university salary is only paid during the academic year, \& if you are not bringing in substantial outside money, your 9 months' pay will not be particularly generous. (The university reasons that you are unlikely to give up your sinecure for less than a major pay increase, something a poorly funded professor is unlikely to be offered elsewhere.) Your application for a research contract will therefore generally include a request for ``summer salary''; most universities allow you to receive 2 months' pay from grants. This makes getting funded intensely important to your pocketbook. If you succeed, your annual pay can increase by better than 20\%. If you don't, you may wonder why you are working so hard.

Interacting with students can be a great pleasure but is often very stressful. As a teacher, you will have to deal with insistent people who want to know why their exam grades were so poor \& who want private help to understand the material you have been presenting. You will have to deal with students who cheat on tests \& with premeds who have no interest in anything but grades. Only some of your graduate students will really contribute to your research. Others will break your equipment, contaminate your samples, \& install bugs in your computer programs. Some postdocs (particularly those who haven't read this book!) will flounder for a year or 2, will be bitter about their inability to find a job, \& will complain publicly about your guidance.

\textit{Your} academic freedom is certainly a great benefit, but what about that of your colleagues? In some departments, the various groups talk to each other. However, this situation is far from guaranteed. Because there is effectively no management in a university, professors tend to work independently. There is no particular reward for collaboration. This is very different from a national or industrial lab, where the job description includes helping to promote the efforts of one's professional colleagues.

\textit{Assistant professorhood:} If after this litany of disadvantages, you still want to be a tenured professor, there remains the question of how to attain such a position. The most direct route is to work your way up from the bottom, i.e., to start as an assistant professor \& to be promoted. I heartily recommend that you avoid this path if at all possible.

As an assistant professor, you suffer most of the disadvantages \& have few of the benefits of a tenured academic position. Not only do you have to teach, but unlike your senior colleagues, you haven't got sheaves of lecture notes from yesteryear. You start from scratch -- which means devoting many, many hours of preparation for each hour you spend in the classroom. The same is true when it comes to preparing homework assignments \& exam questions.

Although being responsible about your teaching duties is necessary for you to win promotion to tenure, at a research-oriented university, it is far from sufficient. You will certainly be judged on your ability to bring in grant money. Although you will have to publish to avoid perishing, you will also have to get funded to survive. This means you will be learning the ropes of grant writing at the same time as you are trying to establish a research effort \& desperately need to produce some results.

Your salary as an assistant professor, as for all professors, will not only reflect your seniority, or in this case your lack of it, but also your success at bringing in outside money. Since you are just starting out, you will have had no such success. Therefore, your salary will be miserly to poor. If you are such an exciting prospect that you have managed to land an assistant professorship at a major private university with a fancy reputation, your salary may be even worse. Such a university can expect you to accept lower pay in return for the snob appeal of its name on your r\'esum\'e. It can also offer significantly reduced opportunity if any for promotion to tenure, on the perhaps correct assumption that its name is worth more to you than job security.

Unhappily, whereas full professors might accept lower pay in return for the grant of tenure, assistant professors are expected to take the low pay without the compensation of a secure position. Responding to the American Association of University Professors' (AAUP) efforts to protect you against exploitation, most schools adhere to the policy that an assistant professor who hasn't been granted tenure after 7 years must be fired. Thus, ironically, thanks to a labor organization that purports to represent your interests, you will lose your job if you are not promoted!

There \textit{are} pleasures to working as an assistant professor. Teaching \& interacting with students can be exciting. The university environment is in itself very stimulating. There are certainly more kinds of people with more diverse interests than in any industrial lab. You do get respect from the community. On the other hand, the price of being an assistant professor is much too high. The hours are long, the pay is terrible, \& the job security is bad. After your years of study for a PhD \& further years as a postdoctoral apprentice, you will probably be about 30 years old. You'll probably be starting a family. Your former colleagues who went to engineering or business school will be making their way in the world, earning good salaries, \& having time to participate in activities outside their jobs. Do you want to be working 16 hours a day for half what they are earning, on the chance that after 5 or 6 years your department may give you tenure? If enough of you answers ``no,'' maybe the job conditions will improve. Until then, I recommend that you find a position in an industrial or government research lab. There you can establish a reputation with much less pain, as discussed below, \&, reputation in hand, can start at the top in a university job, if that is still what you want.'' -- \cite[pp. 71--79]{Feibelman2011}

\subsection{Industrial \& Government Research Positions}
``Research jobs in industry or at government labs have some serious disadvantages but many benefits relative to university professorships. At some of the national labs, there are tenured research positions, but for the most part tenure is not offered outside the framework of a university. You can be laid off for a variety of reasons if you work for private industry, of course, but also if you are employed at a government lab.

There is no doubt that tenure is a valuable benefit. However, you should remember that your real job security as a scientist is the recognition \& approval of your peers around the world. If your published research is admired \& used by fellow scientists everywhere, you have little to fear. 1 day you may have to change job locations, but unemployment should not be a worry. Industrial \& government labs provide an environment where it is relatively easy to establish a scientific r\'esum\'e. Thus, if you are competent, the issue of tenure ends up being relatively insignificant. (Incidentally, the reluctance of the managers who hired you to admit that they made a mistake provides an additional, if melancholy, form of job security at a research lab. Firing you after 6 or 7 years if you are not promoted is not built into the system as at a university.)

The most important advantage of working in a research lab, whether industrial or governmental, is that your job description is relatively simple. You are expected to be a scientific leader, to advance knowledge in 1 or more areas of importance to your employer, \& to make yourself useful to your fellow employees. The modern world being what it is, you can also anticipate being asked to help bring in funding. Because your main task is to produce results that will sooner or later benefit stockholders or the taxpayer, your lab will \textit{want} to provide you with the necessary hardware (within budgetary constraints, of course), \& if your work has a high priority, this hardware will be in the form of the latest \& highest power models. E.g., while your university colleagues are writing lengthy proposals to buy a work station, at a research lab you will be struggling to keep up with the latest upgrade to the multiteraflop, massively parallel processor. You get the idea.

Because your job description at a research lab is simple, you can perform up to expectations without working unusually long hours. As a professional, you will certainly find yourself working long days occasionally, when you are on the threshold of an exciting result, or when you have to submit an article by a certain deadline. However, you will not be spending half your time doing work that is necessary but not sufficient for your survival (i.e., teaching, explaining to students why they got a D on your last exam, etc.). You will therefore have time to help your spouse with dinner, to read a novel, to see your kids' school play, or to be a soccer coach. You won't have historians, specialists in Russian literature, or bassoon professors for colleagues, so you will have to make more effort to enhance your cultural life than at a university. On the other hand, you will have more time to spend with friends from outside the workplace.

A research lab is a \textit{managed} environment. We'll consider the downside of living with managers momentarily. The advantages are that management monitors the functioning of the lab \& has the power to make it work better, \& also that management is paid to do bureaucratic dirty work that would otherwise find its way to your in-box. At a government or industrial lab, significant portions of annual pay raises are awarded for merit rather than for having been employed 1 more year. There is unavoidably some arbitrariness \& subjectivity in the annual performance reviews by which merit pay is determined. Nevertheless, the fact that a group seriously considers whether your work is achieving recognition \& deserves a special reward, whether you \& your colleagues are interactive, \& whether support personnel are doing their jobs makes the atmosphere at an industrial or government lab enormously different from a university's. Employees who know that their attitudes \& performance will make a difference to their paychecks take collaboration more seriously. At a research lab, you will find librarians who offer to photocopy articles for you \& who will do electronic literature searches; you will find computer support personnel who want to advance their own careers by helping you make your computer programs more efficient, \& who will hold your hand while you are learning a new system. You will find groups of professional scientists addressing the same complex problem from several different perspectives, groups who meet to share new results \& think up succeeding experiments. At a university, such collegiality is rarer.

There are many ways that management can make your life less rather than more pleasant. Abrupt changes in corporate or congressional priorities may be imposed on you if you work at a commercial or government lab. You may have to redirect your research plans, or even terminate a project before it is completed, because of your company's poor earnings or because of political changes in Washington. Your research progress may be impeded by incessant demands to take Internet or live courses -- on protection of intellectual property, ``export control,'' shop safety, types of fire extinguishers, \& you name it. Heavy-handed scientific managers may insist that it is more important for you to work on their latest (harebrained?) idea than your own. They may reinforce this by refusing to buy the equipment you want for your own purposes. They may insist that you put their name on your papers or patent applications. Or, conversely, your supervisor may have little knowledge of your field \& try to compensate by requiring you to write reports on a too-frequent basis. Management may badger you with the latest buzzwords or theories to emerge from business schools\footnote{``Empowerment,'' interpreted by many in the trenches as the ability to be blamed rather than heard, \& ``thinking outside the box'' are recent ones. Howe often do managers who never take risks themselves or think outside whatever box, urge their technical staff to do just that? \&, when a high-risk, outside-the-box project proves fruitless, who do you suppose suffers the consequences?} instead of inspiring you with rewards in the form of new instruments for your lab \& more money in your bank account. Lastly, personality conflicts with someone who has the power to fire you, to determine whether you can give an invited paper in a faraway place, \& to control the size of your paycheck can cause you plenty of grief.

Obviously, if you work in a managed lab, you need to have some feeling that you will not be subject to a too-heavy hand. A bigger lab, e.g., will provide you more freedom to correct a bad situation than a smaller one would. At a large lab, if you just can't get along with your supervisor, there may be several other groups who would be happy to benefit from your wisdom \& whose supervisors would be easier to deal with. As you reputation grows, of course, your management will look to you for new ideas \& be less likely to suggest that you change directions. In a sense, this is another aspect of the reward system in a managed environment. The more credibly you play the role of a scientific leader, the more freedom you will have to follow your own research ideas. This is a real incentive, I can assure you.

Management suggestions of an important research project or area, incidentally, need not always be bad. Michelangelo was asked by the pope to paint the Sistine Chapel. He didn't write his own proposal to an ``Arts Council of Rome.'' Although research driven by applications is often viewed with some disdain, the desire to fulfill a real need can \& has led to extremely important basic science -- e.g., the Nobel prize -- winning invention of the transistor -- \& has changed the world. You can \& should judge your superiors' suggested research ideas thoughtfully \& on a case-by-case basis.

If you are considering a job in a commercial or government lab with the idea in mind that you will make a name for yourself \& then return in style to academic life, you must be careful to determine whether your projected position \& laboratory policies are consistent with your plan. If the research group you are considering works in an area that is important to the company in question but is of little basic scientific significance, you will very likely not be a viable competitor for an academic position several years down the track. You will have attended the wrong meetings, \& your papers will not have been read in the academic world. If your scientific results are going to be treated as proprietary information, i.e., are not going to be published, to protect commercial advantage, or if they are going to be hidden from the outside world as ``classified data,'' you will not be able to achieve recognition comparable to that of many of your contemporaries. Thus, even though their scientific competence may be no greater than yours, many of your peers will have a significant advantage over you in the competition for tenured academic positions.

Apart from problems in dealing with management, 1 of the worst features of scientific life in many industrial \& government labs is a lack of helpers. Whereas a well-funded university professor can enlist an army of students \& postdocs to bring projects to fruition faster, a staff member at a research lab is lucky to have a technician \& an occasional postdoc. (This is much less of a problem in the biotech industry than in companies that perform physical research, according to my sources.) There are opportunities to alleviate such a shortage, e.g., by collaborating with a university research group. However, such opportunities must be aggressively pursued \& are unlikely in unfavorable geographic situations. Scientists who have dreams of attacking a problem from many sides at once will not be able to fulfill them at a government or industrial lab unless they can persuade colleagues to help.'' -- \cite[pp. 79--87]{Feibelman2011}

\subsection{Money}
``In deciding what kind of scientific position to aim for, you will certainly want to consider relatively pay scales. There are dramatic differences between universities \& research labs in this regard. Whereas the salary distribution for government or commercial labs is a relatively narrow bell curve whose peak is in the realm of the upper-middle class, the histogram for the professoriat is much broader.\footnote{At state-funded universities, salaries are typically public information, making it possible to compile a histogram in the campus library. In some cases, publication on the Internet makes life easier. E.g., an Internet search for ``faculty salaries cavalier'' turns up a list compiled by \textit{The Cavalier Daily} of faculty members \& their salaries at the University of Virginia at Charlottesville. In 2008, the ratio of highest to lowest pay among biologists, chemists, \& physicists on the list was about 6:1.} The university pay scale starts lower than in industry, \& the median university salary is also lower. On the other hand, the incentives for senior scientists at a university are substantially greater than at a national or commercial lab. If as a professor you bring in substantial grant money, you are very valuable to your university \&, not surprisingly, you reap big rewards. The ratio of highest to lowest salaries in a physics department might be 3:1 or 4:1, or more. In an industrial lab it is likely to be less than 2:1. In addition, at a university you can supplement your income by consulting \& by writing textbooks on university time.

Financial priorities thus dictate the same career path as the scientific ones. Entry-level salaries are better in the research labs, \& the merit pay increases they provide can keep you earning more than your university colleagues until you reach the somewhat poorly defined level of ``senior scientists.'' After that, if you want to maximize your salary in industry or in a government lab, there is no alternative but to move into a management position. (1 thing managers seem to do very well is reward themselves.) If you want a high salary while keeping a hand in research, the best alternative is a full professorship. Having established an outstanding scientific reputation working 8 hours a day at a commercial or government lab, you will know what a good contract proposal looks like; you will be relatively successful at bringing in money; \& so you will have a good salary, many students \& postdocs, \& all the good things a university has to offer.

Circumstances -- economic, family, or other -- may prevent you from following the optimal career trajectory. But at least I hope you will now go into the job market with a clear idea of how you would like to arrange your career \& why.'' -- \cite[pp. 87--]{Feibelman2011}

\subsection{Additional Reading}
Browse \url{sciencecareers.sciencemag.org}, the careers website of Science magazine.

%------------------------------------------------------------------------------%

\section{Job Interviews}

\begin{center}
	\textsf{\textbf{What will happen on your interview trip; the questions you had better be prepared to answer.}}
\end{center}

%------------------------------------------------------------------------------%

\section{Getting Funded}

\begin{center}
	\textsf{\textbf{What goes into an effective grant proposal; how \& when to start writing one.}}
\end{center}

%------------------------------------------------------------------------------%

\section{Establishing a Research Program}

\begin{center}
	\textsf{\textbf{Tuning your research efforts to your own capabilities \& your situation in life; e.g., why not to start a 5-year project when you have a 2-year postdoctoral appointment.}}
\end{center}

%------------------------------------------------------------------------------%

\section{A Survival Checklist}

\begin{center}
	\textsf{\textbf{Do not attempt a takeoff before being sure the flaps are down.}}
\end{center}

%------------------------------------------------------------------------------%

\section{Afterthoughts}

\begin{center}
	\textsf{\textbf{A behaviorist approach to professional success.}}
\end{center}

%------------------------------------------------------------------------------%

\printbibliography[heading=bibintoc]
	
\end{document}