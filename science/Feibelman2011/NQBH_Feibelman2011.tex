\documentclass{article}
\usepackage[backend=biber,natbib=true,style=authoryear]{biblatex}
\addbibresource{/home/nqbh/reference/bib.bib}
\usepackage{tocloft}
\renewcommand{\cftsecleader}{\cftdotfill{\cftdotsep}}
\usepackage[colorlinks=true,linkcolor=blue,urlcolor=red,citecolor=magenta]{hyperref}
\usepackage{algorithm,algpseudocode,amsmath,amssymb,amsthm,float,graphicx,mathtools}
\allowdisplaybreaks
\numberwithin{equation}{section}
\newtheorem{assumption}{Assumption}[section]
\newtheorem{conjecture}{Conjecture}[section]
\newtheorem{corollary}{Corollary}[section]
\newtheorem{definition}{Definition}[section]
\newtheorem{example}{Example}[section]
\newtheorem{lemma}{Lemma}[section]
\newtheorem{notation}{Notation}[section]
\newtheorem{principle}{Principle}[section]
\newtheorem{problem}{Problem}[section]
\newtheorem{proposition}{Proposition}[section]
\newtheorem{question}{Question}[section]
\newtheorem{remark}{Remark}[section]
\newtheorem{theorem}{Theorem}[section]
\usepackage[left=0.5in,right=0.5in,top=1.5cm,bottom=1.5cm]{geometry}
\usepackage{fancyhdr}
\pagestyle{fancy}
\fancyhf{}
\lhead{\small Sect.~\thesection}
\rhead{\small\nouppercase{\leftmark}}
\renewcommand{\sectionmark}[1]{\markboth{#1}{}}
\cfoot{\thepage}
\def\labelitemii{$\circ$}

\title{A PhD Is Not Enough! A Guide to Survival in Science}
\author{Peter J. Feibelman}
\date{\today}

\begin{document}
\maketitle
\tableofcontents
\vspace{5mm}
\begin{quotation}
	\textit{``It took me over 40 years to learn from experience what can be learned in 1 hour from this guide.''} -- Carl Djerassi
\end{quotation}

%------------------------------------------------------------------------------%

\section*{Preface: What This Book Is About}
``

'' -- \cite{Feibelman2011}

%------------------------------------------------------------------------------%

\section{Do You See Yourself in This Picture?}

\begin{center}
	\textsf{\textbf{A set of nonfiction vignettes illustrating some of the ways that young scientists make their lives more unpleasant than necessary or fail entirely to establish themselves in a research career.}}
\end{center}

%------------------------------------------------------------------------------%

\section{Advice from a Dinosaur?}

\begin{center}
	\textsf{\textbf{Can you expect someone to be an effective mentor who emerged into the scientific marketplace in a world that looked very different?}}
\end{center}

%------------------------------------------------------------------------------%

\section{Importance Choices: A Thesis Adviser, a Postdoctoral Job}

\begin{center}
	\textsf{\textbf{A discussion of what to consider: young adviser versus an older one, a superstar versus a journeyman, a small group versus a ``factory.'' Understanding \& attending to your interests as a postdoc.}}
\end{center}

%------------------------------------------------------------------------------%

\section{Giving Talks}

\begin{center}
	\textsf{\textbf{Preparing talks that will make people want to hire \& keep you \& that will make the information you present easy to assimilate.}}
\end{center}

%------------------------------------------------------------------------------%

\section{Writing Papers: Publishing Without Perishing}

\begin{center}
	\textsf{\textbf{Why it is important to write good papers. When to write up your work, how to draw the reader in, how to draw attention to your results.}}
\end{center}
``The negative connotation of the clich\'e  \textit{publish or perish} is seriously misplaced. Publication is a key component of your research efforts. It is widely accepted that a scientific endeavor is not complete until it has been written up. The exercise of putting your reasoning down on paper will frequently lead you to refine your thoughts, to detect flaws in your arguments, \& perhaps to realize that your work has wide significance than you had originally imagined. Publication also has strategic significance. As a beginning scientist, not only do you work long hours for long pay, but your job security is anything but assured. To succeed, you must make your talents well known \& widely appreciated. Publishing provides you with an important way to accomplish that. Your papers, on public view around the world, represent not only your product but also your r\'esum\'e. Compelling, thoughtful, well-written articles are timeless advertisements for yourself. You can imagine that a sloppy r\'esum\'e is not worth preparing. A premature or slapdash publication is far worse. It will remain available to readers indefinitely. These thoughts raise the 2 basic questions addressed in the present chapters: \textit{When} should one write a paper, \& \textit{how} should one write it?'' -- \cite[pp. 53--54]{Feibelman2011}

\subsection{Timing}
``Generally, articles are written too soon in response to the fear that one's competitors will publish 1st or as a result of intellectual laziness (i.e., inattention to important details). Papers are written too late because of the fear of publishing a blunder or because of writer's block. Overcoming these fears \& frailties is necessary for \textit{everyone} in science. At the very least, the knowledge that they are not yours alone may help you deal with them. (Read Carl Djerassi's novel \textit{Cantor's Dilemma} [New York: Penguin Books, 1991] for a poignant exposition of the problem of when and what to publish.)

Planning your research as a series of relatively short, complete projects (cf. Chap. 9) is the best way to achieve a disciplined publication schedule, one that serves your interests in scientific priority, self-advertisement, \& job security. Even though you are working toward an important long-term goal, you report each project as an independent piece of work that has produced a new kernel of knowledge (only half-jokingly a ``publon,'' a quantum of publication\footnote{The concept of the ``publon'' emerged from the graduate student minds of M. J. Weber, now at the University of Virginia, \& W. Eckhart, now at the Salk Institute.}). In the introduction to each paper of a series, you place the work reported in the context of the long-term goal, to which you thereby lay claim, \& you explain how the present results take you a step closer. If your project turns out to be as significant as you had hoped, after you have published several papers in the series, no doubt you will be asked to write a review. \textit{This} will provide you with an appropriate forum for a long, definitive article, one that will be widely referred to \& will help to make your name in science.

There are many advantages to writing up your work as a series of short papers. Managers \& funding agencies need concrete evidence that they have hired personnel \& spent money wisely. Nothing is more helpful in this regard than the list of publications their wisdom has fostered. Of course, they will be pleased if you eventually realize a long-term research goal. However, funding cycles are typically 2 or 3 years (cf. Chap. 8), \& renewal of junior scientific positions occurs on a similar time scale. Therefore, deans, research directors, \& contract managers cannot wait for your long-term dreams to come true. They need published evidence of your progress on an ongoing basis.

By writing numerous, relatively short articles, you can keep your name in the spotlight. The titles, abstracts, \& authorship of your new papers will show up in electronic databases, generally updated weekly. Such search engines as \url{scholar.google.com}, \url{www.osti.gov/eprints},
\& \url{www.scirus.com} will readily lead the community to manuscripts you have posted on \url{arXiv.org}, \url{precedings.nature.com}, or any of a host of other preprint servers. The number of citations of a long publication list increases more rapidly than that of a short list.

You mustn't be overly cynical about these facts of scientific life. If you attempt to achieve name-recognition by padding your publication list with repetitive papers, your efforts will soon reap scorn rather than admiration. Still, the little admiration you gain for publishing an awesome magnum opus in a single paper is surely not worth the risk that this publication strategy poses to your job security.

If you publish frequently, you are less likely to be ``scooped.'' The longer you hold back reporting your results, particularly if they are important, the greater the chance some other group will beat you into print. You do need to develop an appreciation for when a piece of work is complete enough to be written up. If the logic of a manuscript is clearly missing an important piece of confirmatory evidence, submitting it to a journal is likely to cause you endless, painful interactions with referees. This is the time to hold back. (Among other problems, the referees may very well be your competitors. Their own publication strategy is likely to be affected by their appreciation of where your incomplete work stands.) On the other hand, if you \textit{have} completed a project, the sooner you get it into the hands of a journal, the better the chances are that you will get credit for your accomplishment.

Writing a paper that presents 1 new idea or result is much easier than writing a long, complex article. This is a reasonable way to address the problem of writer's block. Much of the introduction to a shorter paper can be prepared, at least mentally, when the long-term research project is originally proposed. The organization of a paper is simpler if there is not so much material to present, \& it is also relatively easy to explain the conclusions in that case.

Referees are generally busy people \& prefer to review short papers. You are likely to receive a more thoughtful \& positive report on a short manuscript than on a long one. Shorter papers are of course not only easier on referees. They also can be read \& assimilated more easily by the scientific community at large.

Writing up individual kernels of new research should have some appeal for the perfectionist. It is easier to get everything right when one is dealing with a small project than when publishing the results of a major, complex effort.

Eventually, of course, all the significant details of a research project need to be reported in an archival journal so that others may repeat \& confirm the validity of the new science. Writing such technical papers is an important exercise, \& one that will win you credit from your peers if you do it well. On the other hand, in most cases the writing of such papers can be carried out at leisure.'' -- \cite[pp. 54--58]{Feibelman2011}

\subsection{Writing Compelling Papers}
``A journal article should present a careful \& relatively complete account of your research. However, it is all too easy to write an accurate description of your work that attracts no attention \& that adds little to your scientific reputation, \textit{even when your results are significant}. Learning to write articles that people will read \& remember will make you a more effective scientist. It will also enhance your chances for survival as a researcher.

The structure of a news article is a good model to follow in preparing a scientific publication. Newspaper readers, like your research colleagues, rarely have much time for acquiring new information. This is just the reason that news articles present a story several times, in increasing levels of detail. Their headlines, equivalent to the titles of your scientific papers, are there to draw readers in by providing a succinct description of what is noteworthy. Scientists attempting to keep up in a world of information overload often do no more than skim the tables of contents of the leading journals in their field or conduct electronic keyword searches. You can help direct them to your new paper by taking the time to prepare an accurate \& compelling title, concise yet incorporating the most important keywords. (``Cute'' should be avoided, as a rule.)

The abstract of a paper corresponds to the 1st paragraph of a news item. It summarizes the main information, what the important results are, \& what methods you used to obtain them. Numerous journals place a word limit (e.g., 75 words) on the abstract. It is a good idea to impose such a limit on yourself whether or not the journal does. An abstract that is brief \& to the point has a better chance of being read. A wordy one which reads like the introduction to or the body of a paper, will lose readers.

As in the case of titles, it is worth remembering that abstracts circulate more widely than the papers they summarize. They are the 1st item to pop up when one searches journal content \& generally available without charge, even when seeing a full article requires a subscription. A well-written abstract may thus make the difference between someone's downloading your full text or emailing you for a copy, rather than just moving on.

The introduction to a paper is where you tell your story, possibly illustrating the text with an important figure or some key results, but without going into great detail. Here is where you want to explain why your project was an important one to undertake \& how your results make a difference to the way we understand the world. Many busy scientists read only the introduction \& conclusion sections of papers, leaving the technical details for another time. Therefore, it is a good idea to highlight your results -- e.g., by placing your most important figure in the introduction. Even if your readers never take the time to plow through the complete description of your work in the body of your paper, they may think enough of the information in your introduction to make sure to catch your talk at the next scientific meeting.

Virtually everyone finds that writing the introduction to a paper is the most difficult task. It is easy to report the procedures you followed \& to describe the data you obtained. The hard part of paper writing is drawing the reader in. My solution to this problem is to start thinking about the 1st paragraph of an article \textit{when I begin a project rather than when I complete it}. I would not embark on a scientific effort if I didn't think it was important \& that my work would answer a question of rather wide interest. The reasons that I found the project in question interesting enough to work on provide half the material I need for my introduction. The remainder is a summary of my key results. The decision to start writing a paper is generally based on recognizing that a kernel of knowledge has been produced. In my introduction, I want to let my reader know what this new information is, in a nutshell, \& why it is worth reading about. Sitting at the word processor, I imagine I am on the phone with a scientist friend whom I haven't spoken to in sometime. He asks me what I have been doing recently. I write down my imagined response. If, when you try this, you feel an attack of writer's block coming on, turn on a recording device \& actually call a friend. It works.

Incidentally, if you know why you have carried out a scientific project \& what makes your results interesting, there is no reason that your paper should start with an inane clich\'e, such as, ``Recently there has been a resurgence of interest in $\ldots$ (whatever the topic),'' which bothers me every time I see it. If you have been working on a project for several months or a year solely because \textit{other} people are interested in it, you have a lot to learn about problem selection. (In this case, see Chap. 9 for some help. Do not pass go. Do not collect your next paycheck.) Before you start on a research effort, you must understand why it is important, \& in the introduction to your publication on the subject, this is just what you need to explain.

In writing your introduction, as well as the body of your paper, it is essential to place your work in context, not only by explaining what you did \& why but also by citing the relevant literature. This is important, not only to provide your readers with a way of understanding your area of research, but also because your scientific colleagues are very eager to get credit for their achievements. (This is not just vanity. Scientists' careers are built on the perceived importance or usefulness of their research results.) You have much to gain \& little to lose by scrupulously citing your competitor's work. I said above that many busy scientists read only the introduction \& conclusion sections of papers. Even more move directly from the title \& abstract to the references, to see if their work is cited. If someone's papers are not mentioned there but should be, you risk losing a potential friend or at least some respect.

I would add that an excellent way to keep up with developments in your field is to check, from time to time, who is citing your own papers. A ``citation index,'' such as is available on the ISI Web of Science${}^{\rm SM}$, makes this an easy task. Bear in mind as you do this that if checking citations is how people in your field keep up, an article you have written that fails to cite their work is more likely to go unnoticed.

In revising \& editing your article before submitting it, you should constantly be asking yourself if you have dealt with all the loose ends in your logic. \textit{Are there arguments you have thought about \& used but not written into your text? Are you wishy-washy about inferences you have drawn, instead of forceful, because there are missing links in the logic?} If so, you either need to work a little longer before writing your paper, or be forthright about what is conjecture \& what has actually been proven. Even if the referee does not catch the weak points of your manuscript, you must not forget that your paper will be on public view for a long time. Intellectual honesty is accordingly a very good policy. This is not to say you should be such a perfectionist that you never feel comfortable declaring a project done \& ready to be published, but rather that you should own up, in print, to what you think might be weak links in your reasoning. This is a service to the community, in that it points to further research directions. It shows the world that you are a thoughtful \& forthright individual. Importantly, it also provides you an out if your reasoning is later shown to be incorrect.

The format of the body of a paper is often dictated by the journal where it will be submitted. Within the journal's constraints, however, the key to organizing your work is to make your text read like a story. Often it is a good idea to relegate detailed discussion of a technical aspect of the work to an appendix. That way, experts or interested parties can try to understand your arguments in full detail, whereas others do not have to guess how much of the text to skip to move on to the next idea.

Keep in mind that the function of a journal article is to communicate, not simply to indicate how wonderful your results are. In principle, a paper should provide enough information that an interested reader would be able to reproduce your work. It is your responsibility to ensure that the necessary information is made available, at the same time as you try to make your paper as snappy \& readable as you can.'' -- \cite[pp. 54--64]{Feibelman2011}

\subsection{Snappy Papers}
``In archaic times, say 30 years ago, you generally had to write your papers as though the work had actually been done by someone else. You were discouraged from using the personal pronoun ``I'' in favor of ``we'' or, even worse, ``one.'' Journals seemed to require writing papers in the passive mood, as in ``the data were obtained using the following novel method'' rather than ``I developed the following novel method to obtain the data.'' More recently, it has become possible to drop the phoniness of this style \& to reveal in your writing that \textit{you} actually did the work you are reporting. I greatly prefer the more straightforward style \& recommend that you use it.

People of a mathematical bent often connect the sentences in their papers with such words as \textit{now, then, thus, however, therefore, whence, hence}, \& so forth. If you want your text to be readable to the non-pedantic, you should be very sparing in using them. Go over your 1st draft \& challenge yourself to see how many of these connectives you can remove without undermining the logic of your argument.

In this era of speedy desktop computers \& full-featured graphics programs, there are few excuses for omitting evocative figures from a paper. A picture may be worth more than a thousand words in a scientific article, particularly if the thousand words are not read, but the thoughtfully prepared figure is examined \& the information it reports absorbed. This does mean it is important not to prepare figures that are too cluttered. If they offend the eye, they may be ignored along with the thousands of words.

Some journals restrict the length of articles. This typically forces one to go back through the 1st draft of a manuscript to rewrite more economically. In preparing the 1st draft, it is a good idea to be as generous as possible with words. You should write down everything that comes to mind as relevant. This may not be easy but helps get all the logic on paper. (Again, get out the voice recorder if you tend to be stingy with words.) If you have written a copious text, the exercise of cutting back may be more difficult but is less likely to lead to a paper whose flow is compromised by the absence of something important. I recommend the approach of writing generously \& then editing severely in all cases -- i.e., whether or not the journal in question imposes restrictions on manuscript length. The exercise of rewriting as concisely as possible leads to more readable text \& thus to text that is read more widely.

As in the preparation of a seminar, the last section of a paper should provide not just a summary of the results reported but also some idea of how they might affect the direction of future research. The goal of the conclusions section is to leave your reader thinking about how your work affects his or her own research plans. Good science opens new doors.'' -- \cite[pp. 64--66]{Feibelman2011}

\subsection{Referees}
``Last, because arguments with journal referees can take many months to settle, \& can be very frustrating, it is a good idea to forestall them by having your manuscripts reviewed locally, by 1 or 2 of your colleagues, before submission. If you have chosen your local reviewer well, you may discover the weak points in your article in a matter of days rather than months. If English is not your mother tongue (\& if you are writing for an English-language journal) it is even more important to have your paper reviewed \& edited by a colleague, one whose English is near perfect. Your readers, including your journal's referees, are human \& thus impatient to some degree. The easier you can make their task, the better will be their response to your efforts.

Incidentally, as one who referees many papers, I much prefer receiving a cogent, well-written manuscript that I can learn from than the other kind. A paper that I enjoy reading disposes me favorably toward the author. Your referee may be your paper's most careful reader ever. Making a good impression on this anonymous potential employer is not a bad idea!

If your referee does have serious complaints about your article, getting angry is not a productive response. A better idea is to consider why this thoughtful expert did not follow your argument \& agree with it. If on reflection you believe that your results are correct \& that the referee has simply misunderstood them, it is likely that spending some time revising your text will not only persuade the referee to recommend that your paper be published but will also ultimately make your ideas less confusing to your journal's general readership.'' -- \cite[pp. 66--68]{Feibelman2011}

\subsection{Additional Reading}

\begin{itemize}
	\item Carter, Sylvester P. \textit{Writing for Your Peers: The Primary Journal Paper}. New York: Praeger, 1987.
	\item Alley, Michael. \textit{The Craft of Scientific Writing}. 3rd ed. New York: Springer Science \& Business Media, 1996.
	\item Booth, Vernon. \textit{Communicating in Science: Writing a Scientific Paper \& Speaking at Scientific Meetings}. 2nd ed. New York: Cambridge University Press, 1993.
\end{itemize}

%------------------------------------------------------------------------------%

\section{From Here to Tenure: Choosing a Career Path}

\begin{center}
	\textsf{\textbf{An unsentimental comparison of the merits of jobs in academia, industry, \& in government laboratories.}}
\end{center}

%------------------------------------------------------------------------------%

\section{Job Interviews}

\begin{center}
	\textsf{\textbf{What will happen on your interview trip; the questions you had better be prepared to answer.}}
\end{center}

%------------------------------------------------------------------------------%

\section{Getting Funded}

\begin{center}
	\textsf{\textbf{What goes into an effective grant proposal; how \& when to start writing one.}}
\end{center}

%------------------------------------------------------------------------------%

\section{Establishing a Research Program}

\begin{center}
	\textsf{\textbf{Tuning your research efforts to your own capabilities \& your situation in life; e.g., why not to start a 5-year project when you have a 2-year postdoctoral appointment.}}
\end{center}

%------------------------------------------------------------------------------%

\section{A Survival Checklist}

\begin{center}
	\textsf{\textbf{Do not attempt a takeoff before being sure the flaps are down.}}
\end{center}

%------------------------------------------------------------------------------%

\section{Afterthoughts}

\begin{center}
	\textsf{\textbf{A behaviorist approach to professional success.}}
\end{center}

%------------------------------------------------------------------------------%

\printbibliography[heading=bibintoc]
	
\end{document}