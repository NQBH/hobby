\documentclass{article}
\usepackage[backend=biber,natbib=true,style=authoryear]{biblatex}
\addbibresource{/home/hong/1_NQBH/reference/bib.bib}
\usepackage[vietnamese,english]{babel}
\usepackage{tocloft}
\renewcommand{\cftsecleader}{\cftdotfill{\cftdotsep}}
\usepackage[colorlinks=true,linkcolor=blue,urlcolor=red,citecolor=magenta]{hyperref}
\usepackage{algorithm,algpseudocode,amsmath,amssymb,amsthm,float,graphicx,mathtools,multicol}
\allowdisplaybreaks
\numberwithin{equation}{section}
\newtheorem{assumption}{Assumption}[section]
\newtheorem{conjecture}{Conjecture}[section]
\newtheorem{corollary}{Corollary}[section]
\newtheorem{definition}{Definition}[section]
\newtheorem{example}{Example}[section]
\newtheorem{lemma}{Lemma}[section]
\newtheorem{notation}{Notation}[section]
\newtheorem{principle}{Principle}[section]
\newtheorem{problem}{Problem}[section]
\newtheorem{proposition}{Proposition}[section]
\newtheorem{question}{Question}[section]
\newtheorem{remark}{Remark}[section]
\newtheorem{theorem}{Theorem}[section]
\usepackage[left=0.5in,right=0.5in,top=1.5cm,bottom=1.5cm]{geometry}
\usepackage{fancyhdr}
\pagestyle{fancy}
\fancyhf{}
\lhead{\small \textsc{Sect.} ~\thesection}
\rhead{\small \nouppercase{\leftmark}}
\renewcommand{\sectionmark}[1]{\markboth{#1}{}}
\cfoot{\thepage}
\def\labelitemii{$\circ$}

\title{\underline{\textbf{Series}}\\A Personal Journey to Realms of Knowledge:\\Wonderland or The Promised Neverland?\\\vspace{5mm}\textsc{Miscellaneous}}
\author{\selectlanguage{vietnamese} Nguyễn Quản Bá Hồng\footnote{Independent Researcher, Ben Tre City, Vietnam\\e-mail: \texttt{nguyenquanbahong@gmail.com}; website: \url{https://nqbh.github.io}.}}
\date{\today}

\begin{document}
\maketitle
\selectlanguage{english}
\begin{abstract}
	This text is the last section titled \textit{Miscellaneous} of the \textit{Series: A Personal Journey to Realms of Knowledge: Wonderland or The Promised Neverland?}, \& its last updated version will be stored at the following \textsc{url}: \href{https://github.com/NQBH/hobby/blob/master/series_a_personal_journey_to_realms_of_knowledge/miscellaneous/NQBH_series_a_personal_journey_to_realms_of_knowledge_miscellaneous.pdf}{GitHub\texttt{/}NQBH\texttt{/}hobby\texttt{/}series a personal journey to realms of knowledge\texttt{/}miscellaneous}\footnote{Explicitly, \url{https://github.com/NQBH/hobby/blob/master/series_a_personal_journey_to_realms_of_knowledge/miscellaneous/NQBH_series_a_personal_journey_to_realms_of_knowledge_miscellaneous.pdf}.}.
\end{abstract}

\tableofcontents
\selectlanguage{vietnamese}

%------------------------------------------------------------------------------%

\section{Some Choices of Words}
\begin{itemize}
	\item \textsc{[Miscellanea\footnote{\textbf{miscellanea} [n] [plural] (\textit{formal}) various things that have been collected together, especially pieces of literature, poems, letters, etc.} vs. Miscellaneous\footnote{\textbf{miscellaneous} [a] [usually before noun] (abbr., \textbf{misc.}) consisting of many different kinds of things that are not connected \& do not easily form a group, \textsc{synonym}: \textbf{diverse, various}.} vs. Miscellany\footnote{\textbf{miscellany} [n] [singular] (\textit{formal}) a group or collection of different kinds of things, \textsc{synonym}: \textbf{assortment}.}]}: I chose the adjective \textit{miscellaneous} since Donald E. Knuth used it in his masterpiece -- \textit{The Art of Computer Programming}, e.g., \cite{Knuth1997}.
	\item \textsc{[Journey vs. Path]}: I chose the noun \textit{journey} instead of \textit{path} as in the title of the book: \textit{A Path to Combinatorics for Undergraduates: Counting Strategies}, by Titu Andreescu \& Zuming Feng. 
\end{itemize}

\section{A Journey to Literature}
Thuở đi học, mình dốt văn cực. Cấp 1 \& cấp 2 cha mẹ cho mình học ở 1 trường tiểu học \& 1 trường trung học cơ sở gần nhà cho tiện. Giờ nghĩ lại đúng là chả có lựa chọn nào khác. Nhà mình ở 1 xã nghèo trong huyện, 1 huyện nghèo trong tỉnh, 1 tỉnh nghèo trong \href{https://vi.wikipedia.org/wiki/%C4%90%E1%BB%93ng_b%E1%BA%B1ng_s%C3%B4ng_C%E1%BB%ADu_Long}{Đồng Bằng Sông Cửu Long}\footnote{``\textit{Vùng đồng bằng sông Cửu Long} (còn được gọi là \textit{Vùng đồng bằng Nam Bộ, Vùng Tây Nam Bộ, Cửu Long} hoặc \textit{Miền Tây Nam Bộ}) là vùng cực nam của Việt Nam, 1 trong 2 phần của Nam Bộ. Khu vực này có 1 thành phố trực thuộc trung ương là thành phố Cần Thơ và 12 tỉnh: Long An, Tiền Giang, Bến Tre, Vĩnh Long, Trà Vinh, Hậu Giang, Sóc Trăng, Đồng Tháp, An Giang, Kiên Giang, Bạc Liêu \& Cà Mau.'' -- Wikipedia\texttt{/}đồng bằng sông Cửu Long.}, \& đọc phần giới thiệu của \href{https://vi.wikipedia.org/wiki/%C4%90%E1%BB%93ng_b%E1%BA%B1ng_s%C3%B4ng_C%E1%BB%ADu_Long}{Wikipedia\texttt{/}đồng bằng sông Cửu Long} tới đoạn ``Tuy nhiên, Đồng bằng sông Cửu Long đứng về phương diện thu nhập vẫn còn thấp hơn cả nước: thu nhập bình quân đầu người với mức 54 triệu đồng (cả nước là 64 triệu đồng\texttt{/}người\texttt{/}năm).'' thì cái cụm `1 vùng nghèo trong nước' thêm tiếp vô cái chuỗi tu từ trước đó cũng hoàn toàn hợp lý, để làm bật lên cái lũy thừa của cơ số ``nghèo''.\footnote{À mình là dân Toán nên thích đùa\texttt{/}ví von những câu\texttt{/}phép so sánh kiểu vầy. Nếu thấy phiền, mong bạn thông cảm.} Chả biết diễn tả độ nghèo như thế nào, chỉ biết nếu xét mặt bằng chung thì xã mình thua xa tất cả các xã giáp quanh. Về mặt toán học\texttt{/}mathematically (speaking), điều đó có nghĩa là xã mình là 1 cực tiểu địa phương chặt\texttt{/}ngặt\footnote{Sách Toán, cả sơ cấp lẫn cao cấp, của miền Bắc của Việt Nam sử dụng tính từ ``ngặt'', trong khi sách miền Nam sử dụng tính từ ``chặt''. Cũng không quan trọng lắm vì đều là bản dịch của adjective ``strict'' trong tiếng Anh.} của hàm đo mức thu nhập, tức hàm đo độ giàu (có thể gọi là \textit{rich-function}\texttt{/}\textit{rich-measure}), \& đương nhiên\footnote{Cụm ``đương nhiên'' ở đây ám chỉ mệnh đề toán học: ``$x$ là 1 cực tiểu (tương ứng, cực đại) địa phương\texttt{/}toàn cục chặt\texttt{/}không chặt của hàm $f(x)$ khi \& chỉ khi $x$ cũng là cực đại (tương ứng, cực tiểu) địa phương\texttt{/}toàn cục chặt\texttt{/}không chặt của hàm $-f(x)$.'' In English, ``$x$ is a strictly\texttt{/}non-strictly local\texttt{/}global minimizer (maximizer, resp.) of a function $f(x)$ iff $x$ is a strictly\texttt{/}non-strictly local\texttt{/}global maximizer (minimizer, resp.) of the function $-f(x)$.}, cũng là cực đại địa phương chặt\texttt{/}ngặt của hàm đo độ nghèo (có thể gọi là \textit{poor-function}\texttt{/}\textit{poor-measure})! Cụ thể, nếu 1 xã $X$ thỏa mãn tính chất $\operatorname{poor}(X) > \operatorname{poor}(Y)$, $\forall Y\in\operatorname{neighborhood}(X)$ thì xã đó được gọi là cực đại địa phương của hàm nghèo.


Ờ thì -- 1 cách tiêu biểu -- nhìn cái kênh của Truyền Hình Bến Tre so với mấy kênh của Truyền Hình Tiền Giang, Cần Thơ, đặc biệt là Truyền Hình Vĩnh Long thì tự hiểu.
 
J.K. Rowling depression.
 
văn học nghệ thuật ở ĐB SCL.
 
 

Mình có học thêm môn Toán, Anh văn. Sau này lên cấp 3 thì mình không học thêm môn nào nữa, 

Nhớ hồi cấp 1 \& cấp 2 có vài lần hên hên mình được học sinh giỏi nhất Khối, hơn mấy bạn nữ chung khối học trâu chó, nhưng là nhờ điểm các môn Tự nhiên trội, chứ điểm các môn Xã hội cũng tàn tàn. Lớp 7 có lần thi cuối học kỳ mình được 10 điểm văn thiệt. Giật nảy mình. Nhưng đơn giản vì, chả hiểu làm sao, mình viết đủ các ý trong thang chấm điểm của các thầy cô.

có nhiều kiểu dốt. 1 cách ví von, kiểu dốt văn có mình là kiểu mà khi giả sử bạn có trong tay rất nhiều nguyên liệu ngon, thậm chí quý, nhưng không biết chế biến, \& tạo thành 1 đám xà bần, \& khi điều giả sử ấy là sai, tức bạn chả có nguyên liệu nào ngon cả. -> Mặc dù bổ nhưng không ngon.

Không kể xuất thân từ đâu, học thức cơ bản thế nào, con người vẫn quan trọng là ở sức rướn. -> Water quote by Bruce Lee.

\newpage

\section{A Journey to Psychology}
Đã bao lần kể từ cái ngày mình rời khỏi đất Pháp, mình luôn dằn vặt \& tự hỏi mình \textit{``Tại sao mày không chịu tự học tâm lý sớm hơn?''} Đúng. Nếu học tâm lý sớm hơn, thì nhiều chuyện buồn đã không xảy ra.


- Cái câu hỏi đó cứ vang lên trong đầu. Cả lúc tỉnh lẫn mê. Có cái gì đó thúc đẩy mình đọc về nó nhiều hơn, 

\section{Aggressiveness}
\begin{quotation}
	- Này, mày nhìn chị ta đi. Thú thật đi, mày có \textit{cảm giác gì đầu tiên}?
	
	- Chị nào cơ? Bà chị \textit{đã giúp tao} hay bà chị \textit{tao đang giúp}?
	
	- Cả 2. Họ có nhiều \textit{điểm chung} lắm. Mày suy nghĩ kỹ rồi đoán xem?
	
	- Gương mặt dễ thương, cùng là gái chuyên Toán?
	
	- Kỹ hơn nữa!
	
	- [Chép nhẹ miệng] Lùn, cận, mông hình trái xoan, mẩy, chân nhỏ nhắn, đùi mập nhưng ngực hơi nhỏ\texttt{/}lép?
	
	- \textit{Làm gì có mông hình trái xoan hả cái thằng dâm này?} Chỉ có 5 loại hình dáng mông trên thế giới\footnote{See, e.g., \href{https://metro.co.uk/2016/04/15/there-are-five-types-of-bums-in-the-world-which-one-do-you-have-5819927/}{Metro\texttt{/}there are 5 types of bums in the world -- which one do you have?}. ``there are 5 specific types of bums in the world -- \& every woman has a specific one.'' [$\ldots$] ``The shape your butt is depends on the placement of your pelvis \& hip bones, the distribution of fat, the size of your glutes, \& the way your muscles are attached to the thigh bone.'', i.e., hình dạng mông của bạn phụ thuộc vào vị trí của xương chậu \& xương hông, sự phân bố chất béo, kích thức cơ mông \& các cơ liên kết với xương đùi. See also \href{https://www.cosmopolitan.com/health-fitness/a56412/there-are-5-different-types-of-butts-in-the-world/}{Cosmopolitan\texttt{/}there are  5 different types of butts in the world \& there's a \textit{proper} underwear style for each of them.}
	
	\begin{quotation}
		``This is very educational.'' -- Hannibal Lecter, \textit{Hannibal} (2013--2015), S1.E7: \textit{Sorbet}
	\end{quotation}}: V-shape, A-shape\texttt{/}trapezoid, square, round, \& upside down heart.\footnote{Inspired by \href{https://www.youtube.com/watch?v=JGwWNGJdvx8}{Ed Sheeran\texttt{/}\textit{Shape of You}}, another song named \textit{Shape of Ass}\texttt{/}\textit{Bum}\texttt{/}\textit{Butt} should be composed.} Nhưng vấn đề ở đây không phải hình dáng bên ngoài, tao hỏi về bên trong kìa.

	- Nói chuyện \& hành xử \textit{không khéo, hơi vô duyên}? \textit{Không để ý tới cảm xúc của người khác} khi nói?
	
	- Bingo!
	
	\begin{figure}[H]
		\centering
		\includegraphics[scale=0.3]{bingo}
		\caption{``Oooh, that's a bingo.'' -- Col. Hans Landa, \href{https://www.imdb.com/title/tt0361748}{Inglourious Basterds} (2009).}
	\end{figure}
\end{quotation}
Chẳng qua 1 đoạn đối thoại tầm\texttt{/}bình thường giữa 2 người, mà trong đó có ít nhất 1 người là con trai, hoặc đúng hơn nếu xem xét xu hướng giới tính hiện tại, thì có ít nhất 1 người thích con gái -- có lẽ bạn nghĩ vậy. Nhưng thực ra có vài điều thú vị ẩn chứa ở đây -- nếu bạn \textit{tinh ý} -- đặc biệt là những từ được in nghiêng:
\begin{itemize}
	\item \textit{Cảm giác đầu tiên}\texttt{/}\textit{1st impression}: Dịch đúng hơn là \textit{ấn tượng đầu tiên} khi lần đầu gặp nhau. See, e.g., \href{https://en.wikipedia.org/wiki/First_impression_(psychology)}{Wikipedia\texttt{/}1st impression (psychology)}.
	\item \textit{Đã giúp \texttt{<người nào đó>} vs. \texttt{<người nào đó>} đang giúp}: Ở đây có 2 tình huống: chủ động giúp người khác \& được người khác giúp 1 cách bị động. Nếu chia con người thành 3 loại: giver, taker, \& matcher, see \cite{Grant2013}, thì sẽ có $3^2 = 9$ tổ hợp (combination) ở đây trong mô hình ``someone helps someone else''. Đến đây thì 1 câu hỏi xuất hiện: \textit{Takers cũng giúp người khác cơ á?} Có chứ, \& điều đặc biệt là những kẻ takers giỏi thao túng nhất sẽ bóc lột đến kiệt sức nạn nhân chỉ bằng vài hành vi giúp đỡ đúng lúc nạn nhân cần.
	\begin{quotation}
		``Easy come, easy go, that's just how you live, oh
		
		Take, take, take it all, but you never give
		
		Should have known you was trouble from the first kiss
		
		Had your eyes wide open
		
		Why were they open? (Ooh)\\
		
		Gave you all I had and you tossed it in the trash
		
		You tossed it in the trash, you did
		
		To give me all your love is all I ever ask
		
		`Cause what you don't understand is'' -- Bruno Mars, \href{https://www.youtube.com/watch?v=SR6iYWJxHqs}{\textit{Grenade}}
	\end{quotation}
	Điều này sẽ được bàn kỹ sau.
	\item \textit{Điểm chung}: 1 số người có cùng đặc điểm tính cách nào đấy thường sẽ có nhiều điểm chung. Cho nên từ \textit{điểm chung} được chú ý ở đây theo khía cạnh tâm lý học.
	\item \textit{Dâm}: Đàn ông, con trai khi tiếp xúc với phụ nữ, con gái đẹp thường ít khi dùng cái đầu (bự) để suy nghĩ. Máu thường dồn xuống chỗ thấp hơn, \& dây thần kinh thường nhạy hơn ở những cái đầu nhỏ hơn của cơ thể, e.g., đầu ngón tay\texttt{/}chân do run vì phấn khích trước sắc đẹp. Có vẻ cánh mày râu nên ăn rau râm để bớt dâm nếu muốn suy luận 1 cách tỉnh táo trước phụ nữ đẹp.
	\item \textit{Không khéo, hơi vô duyên; không để ý tới cảm xúc của người khác}: Con người cần nhiều thời gian để trau dồi kỹ năng sống, đặc biệt là trong cách tương tác với người khác. Nhưng trong 1 số trường hợp, biện hiện \textit{không khéo, vô duyên} trong lời nói \& hành vi có thể là dấu hiệu nhận biết sớm để nhận diện người đối diện là 1 kẻ có chứng rối loạn nhân cách, \& 3 trường hợp của chứng rối loạn nhân cách bao gồm: rối loạn nhân cách ái kỷ, rối loạn nhân cách chống đối xã hội \& thái nhân cách.
\end{itemize} 
\textit{Nhưng lép thì có liên quan gì? \href{https://en.wikipedia.org/wiki/Body_shaming}{Body shaming} à?} Những người phụ nữ tài năng, sắc sảo, có thể có sở thích trên cơ\texttt{/}thống trị cánh đàn ông, có quá nhiều cái giỏi đến nổi làm ông trời đố kỵ. Mà đố kỵ quá thì ổng bèn làm cho họ lép để bù trừ lại -- lép 1 cách sang trọng, e.g., Rosé:
\begin{figure}[H]
	\centering
	\includegraphics[scale=0.2]{Rose_boob}
	\caption{\textsc{eRosé, BlackPink}, Saint Laurent, 2022.}
\end{figure}
\textit{Chứ cái gì cũng giỏi thì ai chịu nổi?} Đây đơn thuần là 1 lời đùa cợt, nhưng vô tình lại đúng với nhiều trường hợp.

Quay trở lại đoạn hội thoại trên, 2 bà chị này có 1 đặc điểm chung khác khá hay, đó là khi tiếp xúc họ, với 2 khung thời gian hoàn toàn khác nhau, mình đang phải chịu ảnh hưởng trực tiếp của 1 kẻ lạm dụng\texttt{/}thao túng tâm lý mạnh, trong 1 kẻ là 1 người ái kỷ thích controlling, \& kẻ còn lại là 1 người thái nhân cách. Bà chị đầu tiên \textit{có nhiều biểu hiện} của \textit{hành vi hiếu chiến công khai} (\textit{overt aggression}), trong khi bà chị thứ 2 là 1 người bị rối loạn nhân cách \& có nhiều biểu hiện của \textit{nhân cách hiếu chiến ngầm} (\textit{covert aggression}) \& nhiều đặc trưng của 1 người có nhiều biểu hiện của chứng rối loạn nhân cách theo khuynh hướng ái kỷ. See, e.g., \cite{Simon2010}.

Bạn hoàn toàn có thể xem những câu chuyện này là tưởng tượng. Mục tiêu chính của mình là đang muốn xây dựng vài \textit{mô hình tâm lý} để hợp lý hóa những trải nghiệm bản thân. Riêng bà chị đầu tiên thì khỏi phải lo -- [em biết chị đang đọc -- em hoàn toàn không có ác ý]. Mục tiêu chính của mình vẫn là những kẻ lạm dụng\texttt{/}thao túng tâm lý.

Đoạn dạo đầu phía trên thực ra là 1 đoạn hội thoại nội tâm, giữa các bản thể khác nhau trong cùng 1 người. Mình dùng từ \textit{bản thể}, không phải \textit{nhân cách} như kiểu \textit{1 man army} \& hy vọng không tạo ra cảm giác về 1 kẻ đa nhân cách. Tự nói chuyện \& phản biện với bản thân là 1 điều thú vị, các bạn có thể xem video của Prof. Jorden Peterson tự tranh luận với chính bản thân mình, e.g., \href{https://www.youtube.com/watch?v=buD2RM0xChM}{YouTube\texttt{/}Jordan Peterson vs. Peter Jordanson}. 1 câu hỏi khá hay cũng xuất hiện khi vô tình nhắc đến đa nhân cách: \textit{Nếu 1 người sở hữu $n$ nhân cách ($n\in\mathbb{N}$, $n\ge 2$) \& nếu tất cả các nhân cách đó điều tốt, thì bạn có thấy sợ hãi khi đối diện với họ không? Nếu có thì có sợ hãi hơn 1 người chỉ có duy nhất 1 nhân cách nhưng bị rối loạn hoặc gặp các vấn đề về tâm thần hay không?}

Tiếp theo mình sẽ sử dụng \& xây dựng 1 vài mẫu chuyện, để làm rõ các khái niệm tâm lý được nêu trên.

\newpage

\section{Miscellaneous\texttt{/}Miscellaneous}

\subsection{Dumbphone vs. Smartphone}
$\ldots$ hoặc, \textit{Điện thoại ngu vs. Điện thoại khôn}, hoặc, \textit{Điện thoại cùi bắp vs. Điện thoại thông minh}.

\begin{definition}[Điện thoại di động]
	\label{def: điện thoại di động}
	``\emph{Điện thoại di động} (\emph{ĐTDĐ}), còn gọi là \emph{điện thoại cầm tay}, là loại \href{https://vi.wikipedia.org/wiki/%C4%90i%E1%BB%87n_tho%E1%BA%A1i}{điện thoại} có thể thực hiện \& nhận cuộc gọi thoại thông qua kết nối dựa trên \href{https://vi.wikipedia.org/wiki/T%E1%BA%A7n_s%E1%BB%91_v%C3%B4_tuy%E1%BA%BFn}{tần số vô tuyến} vào mạng \href{https://vi.wikipedia.org/wiki/Vi%E1%BB%85n_th%C3%B4ng}{viễn thông} trong khi người dùng đang di chuyển trong khu vực dịch vụ. Kết  nối vô tuyến thiết lập kết nối với các hệ thống chuyển mạch của \href{https://vi.wikipedia.org/wiki/Nh%C3%A0_m%E1%BA%A1ng}{nhà khai thác mạng di động}, cung cấp quyền truy cập vào \href{https://vi.wikipedia.org/wiki/M%E1%BA%A1ng_%C4%91i%E1%BB%87n_tho%E1%BA%A1i_chuy%E1%BB%83n_m%E1%BA%A1ch_c%C3%B4ng_c%E1%BB%99ng}{mạng điện thoại chuyển mạch công cộng} (PSTN). Các dịch vụ điện thoại di động hiện đại sử dụng kiến trúc \href{https://vi.wikipedia.org/wiki/M%E1%BA%A1ng_thi%E1%BA%BFt_b%E1%BB%8B_di_%C4%91%E1%BB%99ng}{mạng tế bào} (cellular network) \& do đó, điện thoại di động được gọi là \emph{cellular telephones} hay \emph{cell phones}, tại Bắc Mỹ. Ngoài dịch vụ thoại, điện thoại di động từ những năm 2000 còn hỗ trợ nhiều dịch vụ khác, e.g., \href{https://vi.wikipedia.org/wiki/SMS}{SMS}, \href{https://vi.wikipedia.org/wiki/D%E1%BB%8Bch_v%E1%BB%A5_nh%E1%BA%AFn_tin_%C4%91a_ph%C6%B0%C6%A1ng_ti%E1%BB%87n}{MMS}, \href{https://vi.wikipedia.org/wiki/Email}{email}, \href{https://vi.wikipedia.org/wiki/Truy_c%E1%BA%ADp_Internet}{truy cập Internet}, liên lạc không dây tầm ngắn (hồng ngoại, \href{https://vi.wikipedia.org/wiki/Bluetooth}{Bluetooth}), ứng dụng doanh nghiệp, \href{https://vi.wikipedia.org/wiki/Video_game}{video game}, \& \href{https://vi.wikipedia.org/wiki/Ch%E1%BB%A5p_%E1%BA%A3nh_k%E1%BB%B9_thu%E1%BA%ADt_s%E1%BB%91}{chụp ảnh kỹ thuật số}. Điện thoại di động chỉ cung cấp các khả năng đó được gọi là \href{https://vi.wikipedia.org/wiki/%C4%90i%E1%BB%87n_tho%E1%BA%A1i_ph%E1%BB%95_th%C3%B4ng}{feature phone}; điện thoại di động cung cấp khả năng tính toán tiên tiến rất lớn được gọi là \href{https://vi.wikipedia.org/wiki/Smartphone}{smartphone}.'' -- \href{https://vi.wikipedia.org/wiki/%C4%90i%E1%BB%87n_tho%E1%BA%A1i_di_%C4%91%E1%BB%99ng}{Wikipedia\emph{\texttt{/}}điện thoại di động}
\end{definition}

\begin{definition}[Dumbphone\texttt{/}feature phone\texttt{/}điện thoại phổ thông]
	\label{def: dumbphone}
	``\emph{Điện thoại phổ thông} (tiếng Anh: \emph{feature phone}), còn gọi là điện thoại ``cục gạch'', điện thoại cơ bản hay điện thoại ``ngu'' (\emph{dumbphone}, để phân biệt với \href{https://vi.wikipedia.org/wiki/%C4%90i%E1%BB%87n_tho%E1%BA%A1i_th%C3%B4ng_minh}{điện thoại thông minh}) trong văn nói, là 1 \href{https://vi.wikipedia.org/wiki/%C4%90i%E1%BB%87n_tho%E1%BA%A1i_di_%C4%91%E1%BB%99ng}{điện thoại di động} tại thời điểm sản xuất \& do giới hạn công nghệ thời đó nên không được coi là \href{https://vi.wikipedia.org/wiki/%C4%90i%E1%BB%87n_tho%E1%BA%A1i_th%C3%B4ng_minh}{điện thoại thông minh}. Tuy nhiên, nó có chức năng bổ sung \& các dịch vụ di động. Nó được dành cho người dùng muốn có 1 chiếc điện thoại giá thành thấp hơn \& đơn giản hơn so với \href{https://vi.wikipedia.org/wiki/%C4%90i%E1%BB%87n_tho%E1%BA%A1i_th%C3%B4ng_minh}{điện thoại thông minh}.'' -- \href{https://vi.wikipedia.org/wiki/%C4%90i%E1%BB%87n_tho%E1%BA%A1i_ph%E1%BB%95_th%C3%B4ng}{Wikipedia\emph{\texttt{/}}điện thoại phổ thông}
\end{definition}

\begin{definition}[Smartphone\texttt{/}điện thoại thông minh]
	\label{def: smartphone}
	``\emph{Điện thoại thông minh} hay \emph{smartphone} là khái niệm để chỉ các loại thiết bị di động kết hợp \href{https://vi.wikipedia.org/wiki/%C4%90i%E1%BB%87n_tho%E1%BA%A1i_di_%C4%91%E1%BB%99ng}{điện thoại di động} \& các chức năng \href{https://vi.wikipedia.org/wiki/%C4%90i%E1%BB%87n_to%C3%A1n_di_%C4%91%E1%BB%99ng}{điện toán di động} vào 1 thiết bị. Chúng được phân biệt với \href{https://vi.wikipedia.org/wiki/%C4%90i%E1%BB%87n_tho%E1%BA%A1i_ph%E1%BB%95_th%C3%B4ng}{điện thoại phổ thông} bởi khả năng phần cứng mạnh hơn \& \href{https://vi.wikipedia.org/wiki/H%E1%BB%87_%C4%91i%E1%BB%81u_h%C3%A0nh_di_%C4%91%E1%BB%99ng}{hệ điều hành di động} mở rộng, tạo điều kiện cho phần mềm rộng hơn, internet (bao gồm duyệt web qua \href{https://vi.wikipedia.org/wiki/B%C4%83ng_th%C3%B4ng_r%E1%BB%99ng}{băng thông rộng di động}) \& chức năng \href{https://vi.wikipedia[11pt].org/wiki/%C4%90a_ph%C6%B0%C6%A1ng_ti%E1%BB%87n}{đa phương tiện} (bao gồm âm nhạc, video, máy ảnh \& chơi game), cùng với các chức năng chính của điện thoại như cuộc gọi thoại \& nhắn tin văn bản. Điện thoại thông minh thường chứa 1 số chip \href{https://vi.wikipedia.org/wiki/Vi_m%E1%BA%A1ch}{IC} \href{https://vi.wikipedia.org/wiki/MOSFET}{kim loại-oxit-bán dẫn} (MOS), bao gồm các cảm biến khác nhau có thể được tận dụng bởi phần mềm của chúng (e.g., từ kế, cảm biến tiệm cận, phong vũ biểu, \href{https://vi.wikipedia.org/wiki/Con_quay_h%E1%BB%93i_chuy%E1%BB%83n}{con quay hồi chuyển} hoặc gia tốc kế) \& hỗ trợ giao thức truyền thông không dây (e.g., \href{https://vi.wikipedia.org/wiki/Bluetooth}{Bluetooth}, \href{https://vi.wikipedia.org/wiki/Wi-Fi}{Wi-Fi} hoặc \href{https://vi.wikipedia.org/wiki/GNSS}{định vị vệ tinh}).''
	
	``Định nghĩa công nghiệp về smartphone là 1 thiết bị điện thoại thông minh có 1 màn hình cảm ứng với kích thước \& độ phân giải cao hơn so với điện thoại truyền thống. Điện thoại thông minh được coi như 1 máy tính di động kết hợp với \href{https://vi.wikipedia.org/wiki/M%C3%A1y_%E1%BA%A3nh_s%E1%BB%91}{máy ảnh kỹ thuật số} \& thiết bị chơi game cầm tay, vì nó có 1 hệ điều hành riêng biệt được thiết kế để hiển thị phù hợp các \href{https://vi.wikipedia.org/wiki/Website}{website} 1 cách bình thường cùng nhiều chức năng khác của máy tính như thiết kế, đồ họa, \href{https://vi.wikipedia.org/wiki/Tr%C3%B2_ch%C6%A1i_video}{video game}, cũng như chụp ảnh \& quay phim.'' -- \href{https://vi.wikipedia.org/wiki/%C4%90i%E1%BB%87n_tho%E1%BA%A1i_th%C3%B4ng_minh}{Wikipedia\emph{\texttt{/}}điện thoại thông minh}
\end{definition}
Hồi cấp 2 (2007--2011), trong lớp mình chỉ vài bạn có gia đình khá giả có smartphone (Def. \ref{def: smartphone}) xài, trong khi lúc ấy mình chưa đụng tới cái điện thoại di động nào (Def. \ref{def: điện thoại di động}), dù  chỉ là cùi bắp (Def. \ref{def: dumbphone}), chả biết tính năng nó ra sao -- tò mò xíu nhưng cũng không quan trọng lắm. Lúc đó, nhà mình chỉ có mỗi 1 chiếc điện thoại bàn tổ chảng hiệu Viettel cha mình tậu, \& mình cũng chẳng có lý do gì để xài tới. Lý do khá đơn giản: Trường thì gần nhà, tối ngày mình chỉ đạp xe đi học rồi về, cùng lắm là chiều học về ghé xem mấy lớp học võ hoặc rong ruổi lòng vòng trong cái xóm nghèo ấy.

Mấy bạn có smartphone thì ``sướng'' rồi. Tối ngày nhắn tin yêu đương, ít khi chịu học. Sợ nhất là vài thằng khối mình cua mấy bà chị cách mấy lớp rồi xưng ``anh--em'' ngọt xớt. Lúc đó mình cực nhát gái \& chưa biết khái niệm \textit{phi công trẻ--máy bay bà già} nên không hiểu sao khẩu vị mấy thằng bạn mình mặn vậy. Trong số mấy bạn đó, vẫn ớn nhất là thằng ngồi kế bên mình ở bàn chót lớp -- last bench students. Anh bạn này khá gầy, có khiếu thể thao nổi trội: mấy môn đá banh, bóng chuyền, đá cầu môn này cũng khéo, được thầy thể dục hú vô đội tuyển thể thao của trường. Điều dị thường là mỗi lần có 1 mẫu xe đạp thể thao cao cấp hoặc 1 mẫu smartphone mới, anh bạn này đều tậu được 1 cái mới toanh vào lớp để khoe. Chả hiểu sao mà gia đình ảnh có sổ hộ nghèo \& được hưởng trợ cấp, \& phần tiền trợ cấp chắc 1 phần được dùng để mua mấy thứ xa xỉ đắt tiền đó, trong khi thằng ngồi kế bên thì không có tất cả những thứ đó, đến nỗi mua sách thì phải nhịn ăn sáng vài bữa thì mới đủ tiền mua 1 quyển ưng ý đã lâu. Đời nhiều cái khôi hài!\footnote{``\textit{Đến ly thứ 3, mọi câu nói của anh đều xuôi tai. Tình yêu thật khôi hài, thảo nào anh nhìn em là cười hoài (ah ha ha ha)} $\ldots$'' -- B Ray, Sofia, Châu Đăng Khoa, \href{https://www.youtube.com/watch?v=cfbNtHNCMBo}{\textit{Thiêu Thân}}.} Khôi hài từ lúc sinh ra, trưởng thành, tới khi chết đi. \textit{Nhưng cái gì cũng có phần được \& phần mất}. Mình cũng hiểu ra phần nào cái bài viết mang tên ``\textit{Sinh ra từ gia đình không khá giả, bạn học được gì?}'', \& đương nhiên, câu hỏi song hành với nó: \textit{Sinh trong gia đình khá giả, bạn không học được điều gì?} Vì quá mê điện thoại \& chưng diện bản thân (vuốt keo, nhuộm tóc như bad boy), nên thằng bạn mình học rất dốt, toàn chép bài mình. Nhiều thầy cô cấp 2 ghét \& đì mình cũng vì vụ chép bài này.

Cuối năm lớp 9, ôn thi chuyển cấp thì mình định thi sang Trường Chuyên Bến Tre. Dù là học ở 1 trường làng nghèo, nhưng mình khá tự tin vì đã đoạt nhiều giải học sinh giỏi. Vả lại là đứa đầu tiên sau nhiều năm thành lập trường thi đến vòng Quốc gia -- dù chỉ đậu giải Khuyến khích giải toán trên máy tính Casio cấp trung học cơ sở, do đa phần là tự học nên chỉ biết làm cho ra đáp số nhưng chả biết cách trình bày. Khoảng thời gian ôn thi chuyển cấp khá mệt mỏi. Thầy cô trường mình bảo trước lớp rằng bạn nào không đăng ký học lớp chuyển cấp của trường thì sẽ không phát bằng Tốt nghiệp trung học cơ sở -- dù họ thừa biết là mình phải đi qua trường chuyên để ôn thi. Lúc đấy thì mình tin soái cổ, chả hỏi rõ tại sao. Cha mẹ mình lo làm đầu tất mặt tối cũng không quan tâm tới. Sinh ra trong 1 gia đình nghèo, bọn trẻ thường có xu hướng chịu lép vế \& dẫu có bị bắt nạt bởi những người lớn cũng không dám kêu la, nguyên do là nhận thức kém, bám víu vào niềm tin ``cứ nhịn là mọi chuyện sẽ suông sẻ'', nên không dám phản kháng. \& thế là mọi chuyện diễn ra như thế này: Mỗi sáng mình phải dậy tầm 5:30, chạy đến trường để học cái lớp luyện thi do trường mình tổ chức từ lúc 6:30--11:00. Sau đó thì mình ăn trưa 1 cách vội vã rồi rồi đạp xe khoảng 20 cây số (km) (nhà mình cách xa tuyến xe bus duy nhất) giữa trưa nắng rực lửa qua trường chuyên để tham gia mấy lớp học cấp tốc với giá 600,000 đồng cho 3 môn Toán, Văn, Anh. Đến tối thì lại lủi thủi đạp xe 20 km về nhà. Tắm rửa ở cái hàng nước trong vườn rồi ăn vội để đi ngủ. Cứ thế lặp lại 1 tháng mấy gần 2 tháng. Đến cái lớp luyện thi của trường làng cấp 2 thì mình mệt rã rời, nhiều lúc che sách lên để kịp ngủ lấy lại sức. Thằng bạn kế bên với cái smartphone đời mới thì xem phim sex. Chả hiểu cái thằng nam chính là kiểu gì mà điện thoại cứ rung lên bần bật, bành bạch.

- nếu thầy Quang còn làm hiệu trưởng trường cấp 2 đó, mình nghĩ \& tin là thầy sẽ không để chuyện đó xảy ra.

- Trời nhiều mây đen \& mưa cực -- ắt hẳn là 1 ngày lý tưởng để mây mưa thuận lợi.

- mưa như trút đến nỗi rách cả áo mưa -- à ở đây ``rách áo mưa'' được dùng theo nghĩa đen.

- Phone of Warren Buffet.

- Phone of a man watching sex on train.

- Payphone Maroon 5

- luân phiên xem ai bắn bi\footnote{không có bất cứ dấy phẩy, ngăn cách từ nào giữa 2 từ này.} tốt hơn

- về nghĩa topo thì những hình dáng mông này được xem là 1, trừ khi có ai đấy chơi trội khuyết thêm 1 lỗ nữa khiến cho số lỗ \& quai của vật thể này tăng lên, khiến chúng không thể nào đồng nhất với nhau qua 1 phép biến đổi topo! Rose lép. insert hình.

- sự thức tỉnh -- cảm giác trồi lên mặt nước, do bị dìm bởi takers.

- nhưng mày làm 1 kẻ thấu cảm Empath

- micro-managing

- giống con trai. Sai, giống con nít.

- go above to Science, Morty

- đánh dấu giai đoạn dạy thì \& trưởng thành về tâm lý trễ của bản thân.

%------------------------------------------------------------------------------%

%\selectlanguage{english}
%\begin{thebibliography}{99}
%	\bibitem[]{}
%\end{thebibliography}

%------------------------------------------------------------------------------%

\selectlanguage{english}
\printbibliography[heading=bibintoc]
	
\end{document}