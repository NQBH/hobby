\documentclass[oneside]{book}
\usepackage[backend=biber,natbib=true,style=authoryear]{biblatex}
\addbibresource{/home/hong/1_NQBH/reference/bib.bib}
\usepackage[vietnamese,english]{babel}
\usepackage{tocloft}
\renewcommand{\cftsecleader}{\cftdotfill{\cftdotsep}}
\usepackage[colorlinks=true,linkcolor=blue,urlcolor=red,citecolor=magenta]{hyperref}
\usepackage{amsmath,amssymb,amsthm,mathtools,float,graphicx}
\allowdisplaybreaks
\numberwithin{equation}{section}
\newtheorem{assumption}{Assumption}[chapter]
\newtheorem{conjecture}{Conjecture}[chapter]
\newtheorem{corollary}{Corollary}[chapter]
\newtheorem{definition}{Definition}[chapter]
\newtheorem{example}{Example}[chapter]
\newtheorem{lemma}{Lemma}[chapter]
\newtheorem{notation}{Notation}[chapter]
\newtheorem{principle}{Principle}[chapter]
\newtheorem{problem}{Problem}[chapter]
\newtheorem{proposition}{Proposition}[chapter]
\newtheorem{question}{Question}[chapter]
\newtheorem{remark}{Remark}[chapter]
\newtheorem{theorem}{Theorem}[chapter]
\usepackage[left=0.5in,right=0.5in,top=1.5cm,bottom=1.5cm]{geometry}
\usepackage{fancyhdr}
\pagestyle{fancy}
\fancyhf{}
\lhead{\small \textsc{Sect.} ~\thesection}
\rhead{\small \nouppercase{\leftmark}}
\renewcommand{\sectionmark}[1]{\markboth{#1}{}}
\cfoot{\thepage}
\def\labelitemii{$\circ$}

\title{Spirituality}
\author{\selectlanguage{vietnamese} Nguyễn Quản Bá Hồng\footnote{Independent Researcher, Ben Tre City, Vietnam\\e-mail: \texttt{nguyenquanbahong@gmail.com}}}
\date{\today}

\begin{document}
\maketitle
\setcounter{secnumdepth}{4}
\setcounter{tocdepth}{4}
\tableofcontents

%------------------------------------------------------------------------------%

\chapter{Wikipedia's}

\section{\href{https://en.wikipedia.org/wiki/Spirituality}{Wikipedia\texttt{/}Spirituality}}
``The meaning of \textit{spirituality} has developed \& expanded over time, \& various meanings can be found alongside each other. Traditionally, spirituality referred to a \href{https://en.wikipedia.org/wiki/Religion}{religious} process of re-information which ``aims to recover the original shape of man'', oriented at ``the \href{https://en.wikipedia.org/wiki/Image_of_God}{image of God}'' as exemplified by the founders \& sacred texts of the religions of the world. The term was used within early \href{https://en.wikipedia.org/wiki/Christianity}{Christianity} to refer to a life oriented toward the \href{https://en.wikipedia.org/wiki/Holy_Spirit_(Christianity)}{Holy Spirit} \& broadened during the \href{https://en.wikipedia.org/wiki/Late_Middle_Ages}{Late Middle Ages} to include mental aspects of life.

In modern times, the term both spread to other religious traditions \& broadened to refer to a wide range of experience, including a range of \href{https://en.wikipedia.org/wiki/Western_esotericism}{esoteric traditions} \& religious traditions. Modern usages tend to refer to a subjective experience of a sacred dimension \& the ``deepest values \& meanings by which people live'', often in a context separate from organized religious institutions. This may involve belief in a \href{https://en.wikipedia.org/wiki/Supernatural}{supernatural} realm beyond the ordinarily observable world, \href{https://en.wikipedia.org/wiki/Personal_growth}{personal growth}, a quest for an ultimate or sacred \href{https://en.wikipedia.org/wiki/Meaning_of_life}{meaning}, \href{https://en.wikipedia.org/wiki/Religious_experience}{religious experience}, or an encounter with one's own ``inner dimension''.

\subsection{Etymology}

\subsection{Definition}

\subsection{Development of the meaning of spirituality}

\subsubsection{Classical, medieval \& early modern periods}

\subsubsection{Modern spirituality}

\paragraph{Transcendentalism \& Unitarian Universalism.}

\paragraph{Theosophy, anthroposophy, \& the perennial philosophy.}

\paragraph{Neo-Vedanta.}

\paragraph{``Spiritual but not religious''.}

\subsection{Traditional spirituality}

\subsubsection{Abrahamic faiths}

\paragraph{Judaism.}

\paragraph{Christianity.}

\paragraph{Islam.}

\subparagraph{Sufism.}

\subsubsection{Asian traditions}

\paragraph{Buddhism.}

\paragraph{Hinduism.}

\subparagraph{4 paths.}

\subparagraph{Schools \& spirituality.}

\paragraph{Jainism.}

\paragraph{Sikhism.}

\subsubsection{African spirituality}

\subsection{Contemporary spirituality}

\subsubsection{Characteristics}

\subsubsection{Spiritual experience}

\subsubsection{Spiritual practices}

\subsection{Science}

\subsubsection{Relation to science}

\subsubsection{Quantum mysticism}

\subsubsection{Scientific research}

\paragraph{Health \& well-being.}

\paragraph{Intercessionary prayer.}

\paragraph{Spiritual care in health care professions.}

\paragraph{Spiritual experiences.}

\paragraph{Measurement.}

%------------------------------------------------------------------------------%

\part{A Toltec Wisdom Book}

\chapter{\cite{Ruiz2011}. The 4 Agreements: A Practical Guide to Personal Freedom}

``To the \textit{Circle of Fire}; those who have gone before, those who are present, \& those who have yet to come.''

\section*{The Toltec}
``Thousands of years ago, the Toltec were known throughout southern Mexico as ``women \& men of knowledge.'' Anthropologists\footnote{\textbf{anthropologist} [n] a person who studies anthropology}\,\footnote{\textbf{anthropology} [n] [uncountable] \textbf{1.} (also \textbf{cultural anthropology} or \textbf{social anthropology}) the study of the human race by comparing human societies \& cultures \& how they have developed; \textbf{2.} (also \textbf{physical anthropology}) the study of the human race by examining how humans behave \& how their bodies work \& have changed during their development.} have spoken of the Toltec as a nation or a race, but, in fact, the Toltec were scientists \& artists who formed a society to explore \& conserve the spiritual knowledge \& practices of the ancient ones. They came together as masters (\textit{naguals}) \& students at Teotihuacan, the ancient city of pyramids outside Mexico City known as the place where ``Man Becomes God.''

Over the millennia\footnote{\textbf{millennium} [n] (plural \textbf{millennia} or \textbf{millenniums}) \textbf{1.} a period of 1,000 years, especially as calculated before or after the birth of Christ; \textbf{2.} (\textbf{the millennium}) the time when 1 period of 1,000 years ends \& another begins.}, the \textit{naguals} were forced to conceal\footnote{\textbf{conceal} [v] to hide something.} the ancestral\footnote{\textbf{ancestral} [a] connected with or belonging to earlier members of a family, race of people or species.} wisdom\footnote{\textbf{wisdom} [n] \textbf{1.} [uncountable, singular] the ability to make sensible decisions \& give good advice, because of the experience \& knowledge that you have; \textbf{2.} [uncountable, countable] the knowledge \& experience that develops within a particular society or group of people. \textbf{(The) conventional\texttt{/}received wisdom} is what most people believe to be true. \textbf{Common, popular} \& \textbf{traditional} are also used in this way; \textbf{3.} [singular] \textbf{the wisdom of (doing) something} how sensible something is.} \& maintain its existence in obscurity\footnote{\textbf{obscurity} [n] (plural \textbf{obscurities}) \textbf{1.} [uncountable] the state in which somebody\texttt{/}something is not well known or has been forgotten; \textbf{2.} [uncountable, countable, usually plural] \textbf{obscurity (of something)} the quality of being difficult to understand; something that is difficult to understand.}. European conquest\footnote{\textbf{conquest} [n] \textbf{1.} [uncountable, countable] the act of taking control of a country, city, etc. by force; \textbf{2.} [countable] an area of land taken by force; \textbf{3.} [uncountable] \textbf{conquest of something} the act of gaining control over something that is difficult or dangerous.}, coupled with rampant\footnote{\textbf{rampant} [a] \textbf{1.} (of something bad) existing or spreading everywhere in a way that cannot be controlled, \textsc{synonym}: \textbf{unchecked}; \textbf{2.} (of plants) growing thickly \& very fast in a way that cannot be controlled.} misuse\footnote{\textbf{misuse} [n] [uncountable, countable, usually singular] the act of using something in a dishonest way or for the wrong purpose, \textsc{synonym}: \textbf{abuse}; \textbf{misuse something} to use something in the wrong way or for the wrong purpose, \textsc{synonym}: \textbf{abuse}.} of personal power by a few of the apprentices\footnote{\textbf{apprentice} [n] a young person who works for an employer for a fixed period of time in order to learn the particular skills needed in their job; [v] [usually passive] (\textit{old-fashioned}) to make somebody an apprentice.}, made it necessary to shield the knowledge from those who were not prepared to use it wisely or who might intentionally\footnote{\textbf{intentional} [a] done deliberately, \textsc{synonym}: \textbf{deliberate, intended}, \textsc{opposite}: \textbf{unintentional}.} misuse it for personal again.

Fortunately, the esoteric\footnote{\textbf{esoteric} [a] (\textit{formal}) likely to be understood or enjoyed by only a few people with a special knowledge or interest.} Toltec knowledge was embodied\footnote{\textbf{embody} [v] \textbf{1.} to express or represent an idea or a quality, \textsc{synonym}: \textbf{represent}; \textbf{2.} \textbf{embody something} (\textit{formal}) to include or contain something.} \& passed on through generations by different lineages\footnote{\textbf{lineage} [n] [uncountable, countable] \textbf{1.} (\textit{formal}) the series of families that somebody comes from originally, \textsc{synonym}: \textbf{ancestry}; \textbf{2.} (\textit{biology}) a set of species, each member of which is considered to have evolved from the one before. In biology, \textbf{lineage} is also used to talk about a set of cells which develop from a common cell.} of \textit{naguals}. Though it remained veiled\footnote{\textbf{veil} [v] \textbf{1.} \textbf{veil something\texttt{/}yourself} to cover your face with a veil; \textbf{2.} \textbf{veil something} (\textit{literary}) to cover something with something that hides it partly or completely, \textsc{synonym}: \textbf{shroud}.}\,\footnote{\textbf{veiled} [a] \textbf{1.} a veiled threat, warning, etc. is not expressed directly or clearly because you do not want your meaning to be too obvious; \textbf{2.} wearing a veil ($=$ a piece of cloth worn to cover the face, hair, or head).} in secrecy\footnote{\textbf{secrecy} [n] [uncountable] the fact of making sure that nothing is known about something; the state of being secret.} for hundreds of years, ancient\footnote{\textbf{ancient} [a] \textbf{1.} belonging to a period of history that is thousands of years in the past, \textsc{opposite}: \textbf{modern}; \textbf{2.} very old; having existed for a very long time; \textbf{3.} (\textbf{the ancients}) [n] [plural] the people who lived in ancient times, especially the Egyptians, Greeks \& Romans.} prophecies\footnote{\textbf{prophecy} [n] (plural \textbf{prophecies}) \textbf{1.} [countable] a statement that something will happen in the future, especially one made by somebody who claims religious or magic powers. A \textbf{self-fulfilling prophecy} is a statement or theory about something that will happen in the future that itself causes the thing to happen.; \textbf{2.} [uncountable] the power of being able to say what will happen in the future.} foretold\footnote{\textbf{foretell} [v] (\textit{literary}) to know or say what will happen in the future, especially by using magic powers.} the coming of an age when it would be necessary to return the wisdom to the people. Now, don Miguel Ruiz, a \textit{nagual} from the Eagle Knight lineage, has been guided to share with us the powerful teachings of the Toltec.

Toltec knowledge arises from the same essential unity\footnote{\textbf{unity} [n] (plural \textbf{unities}) \textbf{1.} [uncountable, singular] the state of being joined together to form 1 unit; the state of being in agreement \& working together; \textbf{2.} [singular] a single thing that may consist of a number of different parts; \textbf{3.} [uncountable] (in art, literature, etc.) the state of looking or being complete in a natural \& pleasing way; \textbf{4.} [uncountable] (\textit{mathematics}) the number 1.} of truth as all the sacred\footnote{\textbf{sacred} [a] \textbf{1.} connected with God or a god \& thought to deserve special respect, \textsc{synonym}: \textbf{holy}; \textbf{2.} very important \& treated with great respect.} esoteric traditions found around the world. Though it is not a religion, it honors all the spiritual\footnote{\textbf{spiritual} [a] [usually before noun] \textbf{1.} connected with the human spirit, rather than the body or physical things, \textsc{opposite}: \textbf{material}; \textbf{2.} connected with religion.} masters who have taught on the earth. While it does embrace\footnote{\textbf{embrace} [v] \textbf{1.} \textbf{embrace something} to accept an idea, a proposal, a set of beliefs, etc., especially when it is done with enthusiasm; \textbf{2.} \textbf{embrace something} to include something; \textbf{3.} \textbf{embrace somebody} to put your arms around somebody as a sign of love or friendship.} spirit\footnote{\textbf{spirit} [n] \textbf{1.} [uncountable, countable] the part of a person that includes their mind, feelings \& character rather than their body; \textbf{2.} [singular, uncountable] an attitude or way of thinking; \textbf{3.} [uncountable, singular] loyal feelings towrads a group, team or society; \textbf{4.} [singular] \textbf{spirit (of something)} the typical or most important quality or mood of something; \textbf{5.} [uncountable] \textbf{spirit (of something)} the real or intended meaning or purpose of something; \textbf{6.} [uncountable] courage, determination or energy; \textbf{7.} [countable] \textbf{spirit (of somebody)} the part of a person that many people believe still exists after their body is dead; \textbf{8.} [countable] an imaginary creature with magic powers; \textbf{9.} [countable, usually plural] (\textit{especially British English}) a strong alcoholic drink.}, it is most accurately\footnote{\textbf{accurately} [adv] in a way that is true \& exact, \textsc{opposite}: \textbf{inaccurately}.} described as a way of life, distinguished by the ready accessibility\footnote{\textbf{accessibility} [n] [uncountable] the quality of being easy to reach, enter, use or obtain.} of happiness \& love.'' -- \cite[The Toltec]{Ruiz2011}

\section*{Introduction}

\subsection*{The Smokey Mirror}
``3000 years ago, there was a human just like you \& me who lived near a city surrounded by mountains. The human was studying to become a medicine man, to learn the knowledge of his ancestors\footnote{\textbf{ancestor} [n] \textbf{1.} \textbf{ancestor (of somebody)} a person in your family who lived a long time ago; \textbf{2.} \textbf{ancestor (of something)} an animal or plant that lived or grew in the past which a modern animal or plant has developed from; \textbf{3.} \textbf{ancestor (of something)} an early form of something which later became more developed.}, but he didn't completely agree with everything he was learning. In his heart, he felt there must be something more.

1 day, as he slept in a cave, he dreamed that he saw his own body sleeping. He came out of the cave on the night of a new moon. The sky was clear, \& he could see millions of stars. Then something \fbox{happened inside of him} that transformed his life forever. He looked at his hands, he felt his body, \& he heard his own voice say, ``I am made of light; I am made of stars.''

He looked at the stars again, \& he realized that it's not the stars that create light, but rather light that creates the stars. ``Everything is made of light,'' he said, ``\& the space in-between isn't empty.'' \& he knew that everything that exists is 1 living being, \& that light is the messenger of life, because it is alive \& contains all information.

Then he realized that although he was made of stars, he was not those stars. ``I am in-between the stars,'' he thought. So he called the stars the \textit{tonal}\footnote{\textbf{tonal} [a] \textbf{1.} (\textit{specialist}) relating to tones of sound or color; \textbf{2.} (\textit{music}) having a particular key, \textsc{opposite}: \textbf{atonal}.} \& the light between the stars the \textit{nagual}, \& he knew that what created the harmony\footnote{\textbf{harmony} [n] [uncountable] a state of peaceful existence \& agreement.} \& space between the 2 is Life or Intent\footnote{\textbf{intent} [n] [uncountable] (\textit{formal} or \textit{law}) what you intend to do, \textsc{synonym}: \textbf{intention}; \textbf{to all intents \& purposes} (\textit{British English}) (\textit{North American English} \textbf{for all intents \& purposes}) [idiom] in the effects that something has, if not officially; almost completely.}. Without Life, the \textit{tonal} \& the \textit{nagual} could not exist. Life is the force of the absolute\footnote{\textbf{absolute} [a] \textbf{1.} total; not limited in any way; \textbf{2.} existing or measured independently \& not in relation to something else, \textsc{opposite}: \textbf{relative}; [n] an idea or a principle that is believed to be true or valid in all circumstances.}, the supreme\footnote{\textbf{supreme} [a] [usually before noun] \textbf{1.} highest in rank or position; \textbf{2.} very great or the greatest in degree.}, the Creator\footnote{\textbf{creator} [n] \textbf{1.} [countable] a person, organization or quality that makes or produces a particular thing; \textbf{2.} (\textbf{the Creator}) [singular] God.} who creates everything.

This is what he discovered: Everything in existence is a manifestation\footnote{\textbf{manifestation} [n] [countable, uncountable] \textbf{manifestation (of something)} (\textit{formal}) an event, action or thing that is a sign that something exists or is happening; the act of appearing as a sign that something exists or is happening.} of the 1 living being we call God. \textit{Everything is God}. \& he came to the conclusion that \fbox{human perception}\footnote{\textbf{perception} [n] \textbf{1.} [uncountable, countable] an idea, a belief or an image you have as a result of how you see or understand something; \textbf{2.} [uncountable] the way you notice things or the ability to notice things with the senses. In biology, \textbf{perception} refers to the processes in the nervous system by which a living thing becomes aware of events \& things outside itself.; \textbf{3.} [uncountable] the ability to understand the true nature of something, \textsc{synonym}: \textbf{insight}.} is \fbox{merely light perceiving light}. He also saw that matter is a mirror\footnote{\textbf{mirror} [n] \textbf{1.} a piece of special glass that reflects images \& light; \textbf{2.} [usually singular] \textbf{mirror of something} a thing that shows what something else is like. To \textbf{hold a mirror up to something} is to examine it or show what it is like; [v] to have features that are similar to something else, especially in a way that clearly shows what the other thing is like, \textsc{synonym}: \textbf{reflect}.} -- everything is a mirror that reflects\footnote{\textbf{reflect} [v] \textbf{1.} [transitive] to show or be a sign of what something is like or how somebody thinks or feels; \textbf{2.} [transitive] to throw back light, heat, sound, etc. from a surface; \textbf{3.} [intransitive, transitive] to think carefully \& deeply about something; \textbf{reflect well, badly, etc. on somebody\texttt{/}something} [idiom] to make somebody\texttt{/}something appear to be good, bad, etc. to other people.} light \& creates images of that light -- \& the world of illusion\footnote{\textbf{illusion} [n] \textbf{1.} [countable, uncountable] a false idea or belief; \textbf{2.} [countable] something that seems to exist but in fact does not, or seems to be something that it is not.}, the \textit{Dream}, is just like smoke which doesn't allow us to see what we really are. ``\fbox{The real us is pure love, pure light},'' he said.

This realization\footnote{\textbf{realization} [n] (\textit{British English also} \textbf{realisation}) \textbf{1.} [uncountable, singular] \textbf{realization (that) $\ldots$} the process of becoming aware of something, \textsc{synonym}: \textbf{awareness}; \textbf{2.} [uncountable] \textbf{realization (of something)} the process of achieving a particular aim, etc., \textsc{synonym}: \textbf{achievement}; \textbf{3.} [uncountable, countable] \textbf{realization (of something)} (\textit{formal}) the act of producing something in an actual or physical form; the thing that is produced.} changed his life. Once he knew what he really was, he looked around at other humans \& the rest of nature, \& he was amazed at what he saw. He saw himself in everything -- in every human, in every animal, in every tree, in the water, in the rain, in the clouds, in the earth. \& he saw that Life mixed the \textit{tonal} \& the \textit{nagual} in different ways to create billions of manifestations of Life.

In those few moments he comprehended\footnote{\textbf{comprehend} [v] (often used in negative sentences) to understand something fully.} everything. He was \fbox{very excited, \& his heart was filled with peace}. He could hardly\footnote{\textbf{hardly} [adv] \textbf{1.} used to suggest that something is not likely or not reasonable; \textbf{2.} almost no; almost not; almost none; \textbf{3.} used especially after `can' or `could' \& before the main verb, to emphasize that is difficult to do something.} wait to tell his people what he had discovered. But there were no words to explain it. He tried to tell the others, but they could not understand. They could see that he had changed, that something beautiful was radiating\footnote{\textbf{radiate} [v] \textbf{1.} [transitive] \textbf{radiate something} to send out energy, especially light or heat in all directions, \textsc{synonym}: \textbf{give off something}; \textbf{2.} [intransitive] \textbf{$+$ adv.\texttt{/}prep.} (of energy, especially light or heat) to be sent out in all directions; \textbf{3.} [intransitive] \textbf{$+$ adv.\texttt{/}prep.} (of lines, etc.) to spread out in all directions from a central point; \textbf{4.} [transitive] \textbf{radiate something} (of a person) to show clearly that you have a strong feeling or quality through your expression, attitude or behavior.} from his eyes \& his voice. They noticed that he no longer had judgment\footnote{\textbf{judgement} [n] (also \textbf{judgment} \textit{especially in North American English}) \textbf{1.} [countable, uncountable] an opinion that you form about something after thinking about it carefully; the act of making this opinion known to others; \textbf{2.} [uncountable] the ability to make sensible decisions after carefully considering the best thing to do; \textbf{3.} (usually \textbf{judgment}) [countable, uncountable] the decision of a court or a judge.} about anything or anyone. He was no longer like anyone else.

He could understand everyone very well, but no one could understand him. They believed that he was an incarnation\footnote{\textbf{incarnation} [n] \textbf{1.} [countable] a period of life in a particular form; \textbf{2.} [countable] a person who represents a particular quality, e.g., in human form, \textsc{synonym}: \textbf{embodiment}; \textbf{3.} [singular, uncountable] (also \textbf{the Incarnation}) (in Christianity) the act of God coming to earth in human form as Jesus.} of God, \& he smiled when he heard this \& he said, ``It is true. I am God. But you are also God. We are the same, you \& I. We are images of light. We are God.'' But still the people didn't understand him.

He had discovered that he was a mirror for the rest of the people, a mirror in which he could see himself. ``Everyone is a mirror,'' he said. He saw himself in everyone, but nobody saw him as themselves. \& he realized that everyone was dreaming, but without awareness, without knowing what they really are. They couldn't see him as themselves because they was a wall of fog or smoke between the mirrors. \& that wall of fog was made by the interpretation of images of light -- the \textit{Dream} of humans.

Then he knew that he would soon forget all that he had learned. He wanted to remember all the visions he had had, so he decided to call himself the Smokey Mirror so that he would always know that matter is a mirror \& the smoke in-between is what keeps us from knowing what we are. He said, ``I am the Smokey Mirror, because I am looking at myself in all of you, but we don't recognize each other because of the smoke in-between us. That smoke is the \textit{Dream}, \& the mirror is you, the dreamer.''
\begin{quotation}
	``Living is easy with eyes closed, misunderstanding all you see $\ldots$'' -- John Lennon
\end{quotation}
'' -- \cite[Introduction\texttt{/}The Smokey Mirror]{Ruiz2011}

\subsection*{Domestication \& the Dream of the Planet}

\section{Domestication \& the Dream of the Planet}

\section{\textsc{The 1st Agreement}: Be Impeccable with Your Word}

\section{\textsc{The 2nd Agreement}: Don't Take Anything Personally}

\section{\textsc{The 3rd Agreement}: Don't Make Assumptions}

\section{\textsc{The 4th Agreement}: Always Do Your Best}

\section{\textsc{The Toltec Path to Freedom}: Breaking Old Agreements}

\section{\textsc{The New Dream}: Heaven on Earth Prayers}

%------------------------------------------------------------------------------%

%\selectlanguage{english}
%\begin{thebibliography}{99}
%	\bibitem[]{}
%\end{thebibliography}

%------------------------------------------------------------------------------%

\printbibliography[heading=bibintoc]
	
\end{document}