\documentclass[oneside]{book}
\usepackage[backend=biber,natbib=true,style=authoryear]{biblatex}
\addbibresource{/home/hong/1_NQBH/reference/bib.bib}
\usepackage[vietnamese,english]{babel}
\usepackage{tocloft}
\renewcommand{\cftsecleader}{\cftdotfill{\cftdotsep}}
\usepackage[colorlinks=true,linkcolor=blue,urlcolor=red,citecolor=magenta]{hyperref}
\usepackage{amsmath,amssymb,amsthm,mathtools,float,graphicx}
\allowdisplaybreaks
\numberwithin{equation}{section}
\newtheorem{assumption}{Assumption}[chapter]
\newtheorem{conjecture}{Conjecture}[chapter]
\newtheorem{corollary}{Corollary}[chapter]
\newtheorem{definition}{Definition}[chapter]
\newtheorem{example}{Example}[chapter]
\newtheorem{lemma}{Lemma}[chapter]
\newtheorem{notation}{Notation}[chapter]
\newtheorem{principle}{Principle}[chapter]
\newtheorem{problem}{Problem}[chapter]
\newtheorem{proposition}{Proposition}[chapter]
\newtheorem{question}{Question}[chapter]
\newtheorem{remark}{Remark}[chapter]
\newtheorem{theorem}{Theorem}[chapter]
\usepackage[left=0.5in,right=0.5in,top=1.5cm,bottom=1.5cm]{geometry}
\usepackage{fancyhdr}
\pagestyle{fancy}
\fancyhf{}
\lhead{\small \textsc{Sect.} ~\thesection}
\rhead{\small \nouppercase{\leftmark}}
\renewcommand{\sectionmark}[1]{\markboth{#1}{}}
\cfoot{\thepage}
\def\labelitemii{$\circ$}

\title{Spirituality}
\author{\selectlanguage{vietnamese} Nguyễn Quản Bá Hồng\footnote{Independent Researcher, Ben Tre City, Vietnam\\e-mail: \texttt{nguyenquanbahong@gmail.com}}}
\date{\today}

\begin{document}
\maketitle
\setcounter{secnumdepth}{4}
\setcounter{tocdepth}{4}
\tableofcontents

%------------------------------------------------------------------------------%

\chapter{Wikipedia's}

\section{\href{https://en.wikipedia.org/wiki/Spirituality}{Wikipedia\texttt{/}Spirituality}}
``The meaning of \textit{spirituality} has developed \& expanded over time, \& various meanings can be found alongside each other. Traditionally, spirituality referred to a \href{https://en.wikipedia.org/wiki/Religion}{religious} process of re-information which ``aims to recover the original shape of man'', oriented at ``the \href{https://en.wikipedia.org/wiki/Image_of_God}{image of God}'' as exemplified by the founders \& sacred texts of the religions of the world. The term was used within early \href{https://en.wikipedia.org/wiki/Christianity}{Christianity} to refer to a life oriented toward the \href{https://en.wikipedia.org/wiki/Holy_Spirit_(Christianity)}{Holy Spirit} \& broadened during the \href{https://en.wikipedia.org/wiki/Late_Middle_Ages}{Late Middle Ages} to include mental aspects of life.

In modern times, the term both spread to other religious traditions \& broadened to refer to a wide range of experience, including a range of \href{https://en.wikipedia.org/wiki/Western_esotericism}{esoteric traditions} \& religious traditions. Modern usages tend to refer to a subjective experience of a sacred dimension \& the ``deepest values \& meanings by which people live'', often in a context separate from organized religious institutions. This may involve belief in a \href{https://en.wikipedia.org/wiki/Supernatural}{supernatural} realm beyond the ordinarily observable world, \href{https://en.wikipedia.org/wiki/Personal_growth}{personal growth}, a quest for an ultimate or sacred \href{https://en.wikipedia.org/wiki/Meaning_of_life}{meaning}, \href{https://en.wikipedia.org/wiki/Religious_experience}{religious experience}, or an encounter with one's own ``inner dimension''.

\subsection{Etymology}

\subsection{Definition}

\subsection{Development of the meaning of spirituality}

\subsubsection{Classical, medieval \& early modern periods}

\subsubsection{Modern spirituality}

\paragraph{Transcendentalism \& Unitarian Universalism.}

\paragraph{Theosophy, anthroposophy, \& the perennial philosophy.}

\paragraph{Neo-Vedanta.}

\paragraph{``Spiritual but not religious''.}

\subsection{Traditional spirituality}

\subsubsection{Abrahamic faiths}

\paragraph{Judaism.}

\paragraph{Christianity.}

\paragraph{Islam.}

\subparagraph{Sufism.}

\subsubsection{Asian traditions}

\paragraph{Buddhism.}

\paragraph{Hinduism.}

\subparagraph{4 paths.}

\subparagraph{Schools \& spirituality.}

\paragraph{Jainism.}

\paragraph{Sikhism.}

\subsubsection{African spirituality}

\subsection{Contemporary spirituality}

\subsubsection{Characteristics}

\subsubsection{Spiritual experience}

\subsubsection{Spiritual practices}

\subsection{Science}

\subsubsection{Relation to science}

\subsubsection{Quantum mysticism}

\subsubsection{Scientific research}

\paragraph{Health \& well-being.}

\paragraph{Intercessionary prayer.}

\paragraph{Spiritual care in health care professions.}

\paragraph{Spiritual experiences.}

\paragraph{Measurement.}

%------------------------------------------------------------------------------%

\part{A Toltec Wisdom Book}

\chapter{\cite{Ruiz2011}. The 4 Agreements: A Practical Guide to Personal Freedom}

``To the \textit{Circle of Fire}; those who have gone before, those who are present, \& those who have yet to come.''

\section*{The Toltec}
``Thousands of years ago, the Toltec were known throughout southern Mexico as ``women \& men of knowledge.'' Anthropologists\footnote{\textbf{anthropologist} [n] a person who studies anthropology}\,\footnote{\textbf{anthropology} [n] [uncountable] \textbf{1.} (also \textbf{cultural anthropology} or \textbf{social anthropology}) the study of the human race by comparing human societies \& cultures \& how they have developed; \textbf{2.} (also \textbf{physical anthropology}) the study of the human race by examining how humans behave \& how their bodies work \& have changed during their development.} have spoken of the Toltec as a nation or a race, but, in fact, the Toltec were scientists \& artists who formed a society to explore \& conserve the spiritual knowledge \& practices of the ancient ones. They came together as masters (\textit{naguals}) \& students at Teotihuacan, the ancient city of pyramids outside Mexico City known as the place where ``Man Becomes God.''

Over the millennia\footnote{\textbf{millennium} [n] (plural \textbf{millennia} or \textbf{millenniums}) \textbf{1.} a period of 1,000 years, especially as calculated before or after the birth of Christ; \textbf{2.} (\textbf{the millennium}) the time when 1 period of 1,000 years ends \& another begins.}, the \textit{naguals} were forced to conceal\footnote{\textbf{conceal} [v] to hide something.} the ancestral\footnote{\textbf{ancestral} [a] connected with or belonging to earlier members of a family, race of people or species.} wisdom\footnote{\textbf{wisdom} [n] \textbf{1.} [uncountable, singular] the ability to make sensible decisions \& give good advice, because of the experience \& knowledge that you have; \textbf{2.} [uncountable, countable] the knowledge \& experience that develops within a particular society or group of people. \textbf{(The) conventional\texttt{/}received wisdom} is what most people believe to be true. \textbf{Common, popular} \& \textbf{traditional} are also used in this way; \textbf{3.} [singular] \textbf{the wisdom of (doing) something} how sensible something is.} \& maintain its existence in obscurity\footnote{\textbf{obscurity} [n] (plural \textbf{obscurities}) \textbf{1.} [uncountable] the state in which somebody\texttt{/}something is not well known or has been forgotten; \textbf{2.} [uncountable, countable, usually plural] \textbf{obscurity (of something)} the quality of being difficult to understand; something that is difficult to understand.}. European conquest\footnote{\textbf{conquest} [n] \textbf{1.} [uncountable, countable] the act of taking control of a country, city, etc. by force; \textbf{2.} [countable] an area of land taken by force; \textbf{3.} [uncountable] \textbf{conquest of something} the act of gaining control over something that is difficult or dangerous.}, coupled with rampant\footnote{\textbf{rampant} [a] \textbf{1.} (of something bad) existing or spreading everywhere in a way that cannot be controlled, \textsc{synonym}: \textbf{unchecked}; \textbf{2.} (of plants) growing thickly \& very fast in a way that cannot be controlled.} misuse\footnote{\textbf{misuse} [n] [uncountable, countable, usually singular] the act of using something in a dishonest way or for the wrong purpose, \textsc{synonym}: \textbf{abuse}; \textbf{misuse something} to use something in the wrong way or for the wrong purpose, \textsc{synonym}: \textbf{abuse}.} of personal power by a few of the apprentices\footnote{\textbf{apprentice} [n] a young person who works for an employer for a fixed period of time in order to learn the particular skills needed in their job; [v] [usually passive] (\textit{old-fashioned}) to make somebody an apprentice.}, made it necessary to shield the knowledge from those who were not prepared to use it wisely or who might intentionally\footnote{\textbf{intentional} [a] done deliberately, \textsc{synonym}: \textbf{deliberate, intended}, \textsc{opposite}: \textbf{unintentional}.} misuse it for personal again.

Fortunately, the esoteric\footnote{\textbf{esoteric} [a] (\textit{formal}) likely to be understood or enjoyed by only a few people with a special knowledge or interest.} Toltec knowledge was embodied\footnote{\textbf{embody} [v] \textbf{1.} to express or represent an idea or a quality, \textsc{synonym}: \textbf{represent}; \textbf{2.} \textbf{embody something} (\textit{formal}) to include or contain something.} \& passed on through generations by different lineages\footnote{\textbf{lineage} [n] [uncountable, countable] \textbf{1.} (\textit{formal}) the series of families that somebody comes from originally, \textsc{synonym}: \textbf{ancestry}; \textbf{2.} (\textit{biology}) a set of species, each member of which is considered to have evolved from the one before. In biology, \textbf{lineage} is also used to talk about a set of cells which develop from a common cell.} of \textit{naguals}. Though it remained veiled\footnote{\textbf{veil} [v] \textbf{1.} \textbf{veil something\texttt{/}yourself} to cover your face with a veil; \textbf{2.} \textbf{veil something} (\textit{literary}) to cover something with something that hides it partly or completely, \textsc{synonym}: \textbf{shroud}.}\,\footnote{\textbf{veiled} [a] \textbf{1.} a veiled threat, warning, etc. is not expressed directly or clearly because you do not want your meaning to be too obvious; \textbf{2.} wearing a veil ($=$ a piece of cloth worn to cover the face, hair, or head).} in secrecy\footnote{\textbf{secrecy} [n] [uncountable] the fact of making sure that nothing is known about something; the state of being secret.} for hundreds of years, ancient\footnote{\textbf{ancient} [a] \textbf{1.} belonging to a period of history that is thousands of years in the past, \textsc{opposite}: \textbf{modern}; \textbf{2.} very old; having existed for a very long time; \textbf{3.} (\textbf{the ancients}) [n] [plural] the people who lived in ancient times, especially the Egyptians, Greeks \& Romans.} prophecies\footnote{\textbf{prophecy} [n] (plural \textbf{prophecies}) \textbf{1.} [countable] a statement that something will happen in the future, especially one made by somebody who claims religious or magic powers. A \textbf{self-fulfilling prophecy} is a statement or theory about something that will happen in the future that itself causes the thing to happen.; \textbf{2.} [uncountable] the power of being able to say what will happen in the future.} foretold\footnote{\textbf{foretell} [v] (\textit{literary}) to know or say what will happen in the future, especially by using magic powers.} the coming of an age when it would be necessary to return the wisdom to the people. Now, don Miguel Ruiz, a \textit{nagual} from the Eagle Knight lineage, has been guided to share with us the powerful teachings of the Toltec.

Toltec knowledge arises from the same essential unity\footnote{\textbf{unity} [n] (plural \textbf{unities}) \textbf{1.} [uncountable, singular] the state of being joined together to form 1 unit; the state of being in agreement \& working together; \textbf{2.} [singular] a single thing that may consist of a number of different parts; \textbf{3.} [uncountable] (in art, literature, etc.) the state of looking or being complete in a natural \& pleasing way; \textbf{4.} [uncountable] (\textit{mathematics}) the number 1.} of truth as all the sacred\footnote{\textbf{sacred} [a] \textbf{1.} connected with God or a god \& thought to deserve special respect, \textsc{synonym}: \textbf{holy}; \textbf{2.} very important \& treated with great respect.} esoteric traditions found around the world. Though it is not a religion, it honors all the spiritual\footnote{\textbf{spiritual} [a] [usually before noun] \textbf{1.} connected with the human spirit, rather than the body or physical things, \textsc{opposite}: \textbf{material}; \textbf{2.} connected with religion.} masters who have taught on the earth. While it does embrace\footnote{\textbf{embrace} [v] \textbf{1.} \textbf{embrace something} to accept an idea, a proposal, a set of beliefs, etc., especially when it is done with enthusiasm; \textbf{2.} \textbf{embrace something} to include something; \textbf{3.} \textbf{embrace somebody} to put your arms around somebody as a sign of love or friendship.} spirit\footnote{\textbf{spirit} [n] \textbf{1.} [uncountable, countable] the part of a person that includes their mind, feelings \& character rather than their body; \textbf{2.} [singular, uncountable] an attitude or way of thinking; \textbf{3.} [uncountable, singular] loyal feelings towrads a group, team or society; \textbf{4.} [singular] \textbf{spirit (of something)} the typical or most important quality or mood of something; \textbf{5.} [uncountable] \textbf{spirit (of something)} the real or intended meaning or purpose of something; \textbf{6.} [uncountable] courage, determination or energy; \textbf{7.} [countable] \textbf{spirit (of somebody)} the part of a person that many people believe still exists after their body is dead; \textbf{8.} [countable] an imaginary creature with magic powers; \textbf{9.} [countable, usually plural] (\textit{especially British English}) a strong alcoholic drink.}, it is most accurately\footnote{\textbf{accurately} [adv] in a way that is true \& exact, \textsc{opposite}: \textbf{inaccurately}.} described as a way of life, distinguished by the ready accessibility\footnote{\textbf{accessibility} [n] [uncountable] the quality of being easy to reach, enter, use or obtain.} of happiness \& love.'' -- \cite[The Toltec]{Ruiz2011}

\section*{Introduction}

\subsection*{The Smokey Mirror}
``3000 years ago, there was a human just like you \& me who lived near a city surrounded by mountains. The human was studying to become a medicine man, to learn the knowledge of his ancestors\footnote{\textbf{ancestor} [n] \textbf{1.} \textbf{ancestor (of somebody)} a person in your family who lived a long time ago; \textbf{2.} \textbf{ancestor (of something)} an animal or plant that lived or grew in the past which a modern animal or plant has developed from; \textbf{3.} \textbf{ancestor (of something)} an early form of something which later became more developed.}, but he didn't completely agree with everything he was learning. In his heart, he felt there must be something more.

1 day, as he slept in a cave, he dreamed that he saw his own body sleeping. He came out of the cave on the night of a new moon. The sky was clear, \& he could see millions of stars. Then something \fbox{happened inside of him} that transformed his life forever. He looked at his hands, he felt his body, \& he heard his own voice say, ``I am made of light; I am made of stars.''

He looked at the stars again, \& he realized that it's not the stars that create light, but rather light that creates the stars. ``Everything is made of light,'' he said, ``\& the space in-between isn't empty.'' \& he knew that everything that exists is 1 living being, \& that light is the messenger of life, because it is alive \& contains all information.

Then he realized that although he was made of stars, he was not those stars. ``I am in-between the stars,'' he thought. So he called the stars the \textit{tonal}\footnote{\textbf{tonal} [a] \textbf{1.} (\textit{specialist}) relating to tones of sound or color; \textbf{2.} (\textit{music}) having a particular key, \textsc{opposite}: \textbf{atonal}.} \& the light between the stars the \textit{nagual}, \& he knew that what created the harmony\footnote{\textbf{harmony} [n] [uncountable] a state of peaceful existence \& agreement.} \& space between the 2 is Life or Intent\footnote{\textbf{intent} [n] [uncountable] (\textit{formal} or \textit{law}) what you intend to do, \textsc{synonym}: \textbf{intention}; \textbf{to all intents \& purposes} (\textit{British English}) (\textit{North American English} \textbf{for all intents \& purposes}) [idiom] in the effects that something has, if not officially; almost completely.}. Without Life, the \textit{tonal} \& the \textit{nagual} could not exist. Life is the force of the absolute\footnote{\textbf{absolute} [a] \textbf{1.} total; not limited in any way; \textbf{2.} existing or measured independently \& not in relation to something else, \textsc{opposite}: \textbf{relative}; [n] an idea or a principle that is believed to be true or valid in all circumstances.}, the supreme\footnote{\textbf{supreme} [a] [usually before noun] \textbf{1.} highest in rank or position; \textbf{2.} very great or the greatest in degree.}, the Creator\footnote{\textbf{creator} [n] \textbf{1.} [countable] a person, organization or quality that makes or produces a particular thing; \textbf{2.} (\textbf{the Creator}) [singular] God.} who creates everything.

This is what he discovered: Everything in existence is a manifestation\footnote{\textbf{manifestation} [n] [countable, uncountable] \textbf{manifestation (of something)} (\textit{formal}) an event, action or thing that is a sign that something exists or is happening; the act of appearing as a sign that something exists or is happening.} of the 1 living being we call God. \textit{Everything is God}. \& he came to the conclusion that \fbox{human perception}\footnote{\textbf{perception} [n] \textbf{1.} [uncountable, countable] an idea, a belief or an image you have as a result of how you see or understand something; \textbf{2.} [uncountable] the way you notice things or the ability to notice things with the senses. In biology, \textbf{perception} refers to the processes in the nervous system by which a living thing becomes aware of events \& things outside itself.; \textbf{3.} [uncountable] the ability to understand the true nature of something, \textsc{synonym}: \textbf{insight}.} is \fbox{merely light perceiving light}. He also saw that matter is a mirror\footnote{\textbf{mirror} [n] \textbf{1.} a piece of special glass that reflects images \& light; \textbf{2.} [usually singular] \textbf{mirror of something} a thing that shows what something else is like. To \textbf{hold a mirror up to something} is to examine it or show what it is like; [v] to have features that are similar to something else, especially in a way that clearly shows what the other thing is like, \textsc{synonym}: \textbf{reflect}.} -- everything is a mirror that reflects\footnote{\textbf{reflect} [v] \textbf{1.} [transitive] to show or be a sign of what something is like or how somebody thinks or feels; \textbf{2.} [transitive] to throw back light, heat, sound, etc. from a surface; \textbf{3.} [intransitive, transitive] to think carefully \& deeply about something; \textbf{reflect well, badly, etc. on somebody\texttt{/}something} [idiom] to make somebody\texttt{/}something appear to be good, bad, etc. to other people.} light \& creates images of that light -- \& the world of illusion\footnote{\textbf{illusion} [n] \textbf{1.} [countable, uncountable] a false idea or belief; \textbf{2.} [countable] something that seems to exist but in fact does not, or seems to be something that it is not.}, the \textit{Dream}, is just like smoke which doesn't allow us to see what we really are. ``\fbox{The real us is pure love, pure light},'' he said.

This realization\footnote{\textbf{realization} [n] (\textit{British English also} \textbf{realisation}) \textbf{1.} [uncountable, singular] \textbf{realization (that) $\ldots$} the process of becoming aware of something, \textsc{synonym}: \textbf{awareness}; \textbf{2.} [uncountable] \textbf{realization (of something)} the process of achieving a particular aim, etc., \textsc{synonym}: \textbf{achievement}; \textbf{3.} [uncountable, countable] \textbf{realization (of something)} (\textit{formal}) the act of producing something in an actual or physical form; the thing that is produced.} changed his life. Once he knew what he really was, he looked around at other humans \& the rest of nature, \& he was amazed at what he saw. He saw himself in everything -- in every human, in every animal, in every tree, in the water, in the rain, in the clouds, in the earth. \& he saw that Life mixed the \textit{tonal} \& the \textit{nagual} in different ways to create billions of manifestations of Life.

In those few moments he comprehended\footnote{\textbf{comprehend} [v] (often used in negative sentences) to understand something fully.} everything. He was \fbox{very excited, \& his heart was filled with peace}. He could hardly\footnote{\textbf{hardly} [adv] \textbf{1.} used to suggest that something is not likely or not reasonable; \textbf{2.} almost no; almost not; almost none; \textbf{3.} used especially after `can' or `could' \& before the main verb, to emphasize that is difficult to do something.} wait to tell his people what he had discovered. But there were no words to explain it. He tried to tell the others, but they could not understand. They could see that he had changed, that something beautiful was radiating\footnote{\textbf{radiate} [v] \textbf{1.} [transitive] \textbf{radiate something} to send out energy, especially light or heat in all directions, \textsc{synonym}: \textbf{give off something}; \textbf{2.} [intransitive] \textbf{$+$ adv.\texttt{/}prep.} (of energy, especially light or heat) to be sent out in all directions; \textbf{3.} [intransitive] \textbf{$+$ adv.\texttt{/}prep.} (of lines, etc.) to spread out in all directions from a central point; \textbf{4.} [transitive] \textbf{radiate something} (of a person) to show clearly that you have a strong feeling or quality through your expression, attitude or behavior.} from his eyes \& his voice. They noticed that he no longer had judgment\footnote{\textbf{judgement} [n] (also \textbf{judgment} \textit{especially in North American English}) \textbf{1.} [countable, uncountable] an opinion that you form about something after thinking about it carefully; the act of making this opinion known to others; \textbf{2.} [uncountable] the ability to make sensible decisions after carefully considering the best thing to do; \textbf{3.} (usually \textbf{judgment}) [countable, uncountable] the decision of a court or a judge.} about anything or anyone. He was no longer like anyone else.

He could understand everyone very well, but no one could understand him. They believed that he was an incarnation\footnote{\textbf{incarnation} [n] \textbf{1.} [countable] a period of life in a particular form; \textbf{2.} [countable] a person who represents a particular quality, e.g., in human form, \textsc{synonym}: \textbf{embodiment}; \textbf{3.} [singular, uncountable] (also \textbf{the Incarnation}) (in Christianity) the act of God coming to earth in human form as Jesus.} of God, \& he smiled when he heard this \& he said, ``It is true. I am God. But you are also God. We are the same, you \& I. We are images of light. We are God.'' But still the people didn't understand him.

He had discovered that he was a mirror for the rest of the people, a mirror in which he could see himself. ``Everyone is a mirror,'' he said. He saw himself in everyone, but nobody saw him as themselves. \& he realized that everyone was dreaming, but without awareness, without knowing what they really are. They couldn't see him as themselves because they was a wall of fog or smoke between the mirrors. \& that wall of fog was made by the interpretation of images of light -- the \textit{Dream} of humans.

Then he knew that he would soon forget all that he had learned. He wanted to remember all the visions he had had, so he decided to call himself the Smokey Mirror so that he would always know that matter is a mirror \& the smoke in-between is what keeps us from knowing what we are. He said, ``I am the Smokey Mirror, because I am looking at myself in all of you, but we don't recognize each other because of the smoke in-between us. That smoke is the \textit{Dream}, \& the mirror is you, the dreamer.''
\begin{quotation}
	``Living is easy with eyes closed, misunderstanding all you see $\ldots$'' -- John Lennon
\end{quotation}
'' -- \cite[Introduction\texttt{/}The Smokey Mirror]{Ruiz2011}

\section{Domestication \& the Dream of the Planet}
``What you are seeing \& hearing right now is nothing but a dream. You are dreaming right now in this moment. You are dreaming with the brain awake.

\fbox{Dreaming is the main function of the mind, \& the mind dreams 24 hours a day.} It dreams when the brain is awake\footnote{\textbf{awake} [a] [not before noun] not sleeping, \textsc{opposite}: \textbf{asleep}.}, \& it also dreams when the brain is asleep\footnote{\textbf{asleep} [a] [not before noun] sleeping, \textsc{opposite}: \textbf{awake}.}. The difference is that when the brain is awake, there is a material frame\footnote{\textbf{frame} [n] \textbf{1.} [countable] a strong border or structure of wood, metal, etc. that holds a picture, door, piece of glass, etc. in position; \textbf{2.} [countable] the supporting structure of a piece of furniture, a building, a vehicle, etc. that gives it its shape; \textbf{3.} [singular] \textbf{frame (of something)} the general ideas or structure that form the background to something; \textbf{4.} [countable] 1 of the single photographs that a film or video is made of; \textbf{5.} [countable] $=$ \textbf{frame of reference}; \textbf{frame of mind} [idiom] a particular state or condition of your feelings; [v] \textbf{1.} [usually passive] to put or make a frame or border around something; \textbf{2.} \textbf{frame something} to create \& develop something such as a plan, a system or a set of rules; \textbf{3.} \textbf{frame something} t use a set of ideas or beliefs as the background to a discussion or examination of a subject; \textbf{4.} \textbf{frame something} to express something in a particular way.} that makes us perceive\footnote{\textbf{perceive} [v] \textbf{1.} to notice or become aware of something, \textsc{synonym}: \textbf{notice}; \textbf{2.} to be aware of or experience something using the senses; \textbf{3.} [often passive] to understand or think of somebody\texttt{/}something in a particular way; to believe that a particular thing is true, \textsc{synonym}: \textbf{see}.} things in a linear\footnote{\textbf{linear} [a] \textbf{1.} of or in lines; \textbf{2.} going from 1 thing to another in a single series of stages, \textsc{opposite}: \textbf{non-linear}; \textbf{3.} (\textit{mathematics}) involving 1 dimension; that can be represented by a straight line on a graph; of an equation in which the highest power of its terms is 1, \textsc{opposite}: \textbf{non-linear}; \textbf{4.} (\textit{mathematics}) involving or showing a relationship between quantities in which their rates of change are equal, \textsc{opposite}: \textbf{non-linear}.} way. When we go to sleep we do not have the frame, \& the dream has the tendency\footnote{\textbf{tendency} [n] (plural \textbf{tendencies}) \textbf{1.} [countable] if somebody\texttt{/}something has a particular tendency, they are likely to behave or act in a particular way; \textbf{2.} [countable] a new custom that is starting to develop, \textsc{synonym}: \textbf{trend}; \textbf{3.} [countable $+$ singular or plural verb] (\textit{British English}) a group within a larger political group, whose views are more extreme than those of the rest of the group.} to change constantly\footnote{\textbf{constantly} [adv] all the time.}.

Humans are dreaming all the time. Before we were born the humans before us created a big outside dream that we will call society's dream or \textit{the dream of the planet}. The dream of the planet is the collective dream of that we will call society's dream or \textit{the dream of the planet}. The dream of the planet is the collective dream of billions of smaller, personal dreams, which together create a dream of a family, a dream of a community, a dream of a city, a dream of a country, \& finally a dream of the whole humanity. The dream of the planet includes all of society's rules, its beliefs, its laws, its religions, its different cultures \& ways to be, its governments, schools, social events, \& holidays.

We are born with the capacity\footnote{\textbf{capacity} [n] (plural \textbf{capacities}) \textbf{1.} [countable, uncountable] the ability to understand or to do something; \textbf{2.} [uncountable, countable, usually singular] the number of things or people that a container or space can hold; \textbf{3.} [singular, uncountable] the quantity that a factory, machine, etc. can produce; \textbf{4.} [countable, usually singular] the official position or function that somebody has, \textsc{synonym}: \textbf{role}; \textbf{5.} [countable, uncountable] the size or power of a piece of equipment, especially the engine of a vehicle.} to learn how to dream, \& the humans who live before us teach us how to dream the way society dreams. The outside dream has so many rules that when a new human is born, we hook the child's attention \& introduce these rules into his or her mind. The outside dream uses Mom \& Dad, the schools, \& religion to teach us how to dream.

\textit{Attention} is the ability we have to discriminate\footnote{\textbf{discriminate} [v] \textbf{1.} [intransitive, transitive] to recognize that there is a difference between people or things; to show a difference between people or things, \textsc{synonym}: \textbf{differentiate, distinguish}; \textbf{2.} [intransitive] to treat 1 person or group worse\texttt{/}better than another in an unfair way.} \& to focus only on that which we want to perceive. We can perceive millions of things simultaneously, but using our attention, we can hold whatever we want to perceive in the foreground\footnote{\textbf{foreground} [n] (\textbf{the foreground}) \textbf{1.} [countable, usually singular] the part of a view, picture, etc. that is nearest to you when you look at it; \textbf{2.} [singular] an important position that is noticed by people; [v] \textbf{foreground something} to give particular importance to something.} of our mind. The adults around us hooked our attention \& put information into our minds through repetition\footnote{\textbf{repetition} [n] \textbf{1.} [uncountable, countable] the act of repeating something that has already been done, said or written; \textbf{2.} [countable] \textbf{repetition of something} the fact of an action or event happening again in the same way as before.}. That is the way we learned everything we know.

By using our attention we learned a whole reality, a whole dream. We learned how to behave in society: what to believe \& what not to believe; what is acceptable\footnote{\textbf{acceptable} [a] \textbf{1.} that people can agree on or approve of, \textsc{opposite}: \textbf{unacceptable}; \textbf{2.} that can be allowed, \textsc{opposite}: \textbf{unacceptable}.} \& what is not acceptable; what is good \& what is bad; what is beautiful \& what is ugly; what is right \& what is wrong. It was all there already -- all that knowledge, all those rules \& concepts about how to behave in the world.

When you were in school, you sat in a little chair \& put your attention on what the teacher was teaching you. When you went to church, you put your attention on what the priest\footnote{\textbf{priest} [n] a preson who is qualified to perform religious duties \& ceremonies in some religions.} or minister\footnote{\textbf{minister} [n] \textbf{1.} (often \textbf{Minister}) (in the UK \& some other countries) a senior member of the government who is in charge of a government department or a branch of 1; \textbf{2.} a trained religious leader, especially in some Protestant Christian churches.} was telling you. It is the same dynamic\footnote{\textbf{dynamic} [n] \textbf{1.} (\textbf{dynamics}) [plural] the way in which people or things behave, develop or react to each other in a particular situation; \textbf{2.} (\textbf{dynamics}) [uncountable] the science of the forces involved in movement; \textbf{3.} [singular] \textbf{dynamic (of something)} a force that produces change, action or effects; [a] \textbf{1.} (of a process) always active, changing or making progress, \textsc{opposite}: \textbf{static}; \textbf{2.} (\textit{physics}) (of a force) producing movement; \textbf{3.} (\textit{approving}) (of a person) full of energy \& ideas.} with Mom \& Dad, brothers \& sisters: They were all trying to hook your attention. We also learn to hook the attention of other humans, \& we develop a need for attention which can become very competitive. Children compete for the attention of their parents, their teachers, their friends. ``Look at me! Look at what I'm doing! Hey, I'm here.'' The need for attention becomes very strong \& continues into adulthood.

The outside dream hooks our attention \& teaches us what to believe, beginning with the language that we speak. Language is the code for understanding \& communication between humans. Every letter, every word in each language is an agreement\footnote{\textbf{agreement} [n] \textbf{1.} [countable] an arrangement, a promise or a contract made with somebody; \textbf{2.} [uncountable] the state of sharing the same opinion or feeling, \textsc{opposite}: \textbf{disagreement}; \textbf{3.} [uncountable] \textbf{agreement (of somebody\texttt{/}something)} the fact of somebody approving of something \& allowing it to happen; \textbf{4.} [uncountable] the state of matching something else \& not showing it to be wrong; \textbf{5.} [uncountable] \textbf{agreement (with something)} (\textit{grammar}) (of words in a phrase) the state of having the same number, gender or person.}. We call this a page in a book; the word \textit{page} is an agreement that we understand. Once we understand the code, our attention is hooked \& the energy is transferred from 1 person to another.

It was not your choice to speak English. You didn't choose your religion or your moral values -- they were already there before you were born. We never had the opportunity to choose what to believe or what not to believe. We never chose even the smallest of these agreements. We didn't even choose our own name.

As children, we didn't have the opportunity to choose our beliefs, but we \textit{agreed} with the information that was passed to us from the dream of the planet via other humans. The only way to store information is by agreement. The outside dream may hook our attention, but if we don't agree, we don't store that information. As soon as we agree, we \textit{believe} it, \& this is called faith\footnote{\textbf{faith} [n] \textbf{1.} [uncountable] \textbf{faith (in somebody\texttt{/}something)} trust in somebody's ability or knowledge; trust that somebody\texttt{/}something will do what has been promised; \textbf{2.} [uncountable, singular] strong religious belief; \textbf{3.} [countable] a particular religion; \textbf{4.} [uncountable] \textbf{good faith} the intention to behave in an honest way.}. To have faith is to believe unconditionally\footnote{\textbf{unconditional} [a] without any conditions or limits, \textsc{opposite}: \textbf{conditional}.}.

That's how we learn as children. Children believe everything adults say. We agree with them, \& our faith is so strong that the belief system controls our whole dream of life. We didn't choose these beliefs, \& we may have rebelled\footnote{\textbf{rebel} [n] \textbf{1.} a person who fights against the government of their country; \textbf{2.} a person who opposes somebody in authority over them within an organization such as a political party; \textbf{3.} a person who does not like to obey rules or who does not accept normal standards of behavior or appearance; [v] to fight against or refuse to obey an authority, e.g. a government, a system of your parents.} against them, but we were not strong enough to win the rebellion\footnote{\textbf{rebellion} [n] \textbf{1.} [countable, uncountable] \textbf{rebellion (against somebody\texttt{/}something)} an attempt by some of the people in a country to change their government, using violence, \textsc{synonym}: \textbf{uprising}; \textbf{2.} [uncountable, countable] \textbf{rebellion (against somebody\texttt{/}something)} opposition to authority within an organization such as a company or a political party; \textbf{3.} [uncountable] opposition to authority; being unwilling to obey rules or accept normal standards of behavior or appearance.}. The result is surrender\footnote{\textbf{surrender} [v] \textbf{1.} [intransitive, transitive] to admit than you have been defeated \& want to stop fighting; to allow yourself to be caught, taken prisoner, etc.; \textbf{2.} [transitive] to give up something\texttt{/}somebody when you are forced to, \textsc{synonym}: \textbf{relinquish}; \textbf{surrender to something $|$ surrender yourself to something} [phrasal verb] to give in to something, such as a strong feeling or an influence; [n] [uncountable, singular] \textbf{1.} \textbf{surrender (to somebody\texttt{/}something)} an act of admitting that you have been defeated \& want to stop fighting; \textbf{2.} \textbf{surrender of something (to somebody)} an act of giving something to somebody else even though you do not want to, especially after a battle, etc.} to the beliefs with our \textit{agreement}.

I call this process \textit{the domestication\footnote{\textbf{domestication} [n] [uncountable] \textbf{1.} the process of making a wild animal used to living with or working for humans; \textbf{2.} the process of making a plant or crop suitable to grow for human use; \textbf{3.} (\textit{often humorous}) the process of making somebody good at cooking, caring for a house, etc. \& of making them enjoy home life.} of humans}. \& through this domestication we learn how to live \& how to dream. In human domestication, the information from the outside dream is conveyed to the inside dream, creating our whole belief system. 1st the child is taught the names of things: Mom, Dad, milk, bottle. Day by day, at home, at school, at church, \& from television, we are told how to live, what kind of behavior is acceptable. The outside dream teaches us how to be a human. We have a whole concept of what a ``woman'' is \& what a ``man'' is. \& we also learn to judge: We judge ourselves, judge other people, judge the neighbors.

Children are domesticated\footnote{\textbf{domesticate} [v] \textbf{1.} \textbf{domesticate something} to make a wild animal used to living with or working for humans; \textbf{2.} \textbf{domesticate something} to grow plants or crops for human use, \textsc{synonym}: \textbf{cultivate}.} the same way that we domesticate a dog, a cat, or any other animal. In order to teach a dog we punish\footnote{\textbf{punish} [v] \textbf{1.} to make somebody suffer because they have broken the law or done something wrong; \textbf{2.} \textbf{punish something (by\texttt{/}with something)} to set the punishment for a particular crime.} the dog \& we give it rewards. We train our children whom we love so much the same way that we train any domesticated animal: with a system of punishment\footnote{\textbf{punishment} [n] [uncountable, countable] an act or a way of punishing somebody.} \& reward\footnote{\textbf{reward} [n] \textbf{1.} [countable, uncountable] a thing that you are given, or something good that happens, because you have done something good, worked hard, etc.; \textbf{2.} [countable] an amount of money that is offered to encourage people to do something, such as help the police find a criminal, find something that is lost, etc.; [v] [often passive] to give something to somebody because they have done something good, worked hard, etc.}. We are told, ``You're a good boy,'' or ``You're a good girl,'' when we do what Mom \& Dad want us to do. When we don't, we are ``a bad girl'' or ``a bad boy.''

When we went against the rules we were punished; when we went along with the rules we got a reward. We were punished many times a day, \& we were also rewarded many times a day. Soon we became afraid of being punished \& also afraid of not receiving the reward. The reward is the attention that we got from our parents or from other people like siblings, teachers, \& friends. We soon develop a need to hook other people's attention in order to get the reward.

The reward feels good, \& we keep doing what others want us to do in order to get the reward. With that fear of being punished \& that fear of not getting the reward, we start pretending to be what we are not, just to please others, just to be good enough for someone else. We try to please Mom \& Dad, we try to please the teachers at school, we try to please the church, \& \fbox{so we start acting}. \fbox{We pretend to be what we are not because we are afraid of being rejected.} The fear of being rejected becomes the fear of not being good enough. Eventually we become someone that we are not. We become a copy of Mamma's beliefs, Daddy's beliefs, society's beliefs, \& a religion's beliefs.

All our normal tendencies are lost in the process of domestication. \& when we are old enough for our mind to understand, we learn  the word \textit{no}. The adults say, ``Don't do this \& don't do that.'' We rebel \& say, ``No!'' We rebel because we are defending our freedom. We want to be ourselves, but we are very little, \& the adults are big \& strong. After a certain time we are afraid because we know that every time we do something wrong we are going to be punished.

The domestication is so strong that at a certain point in our lives we no longer need anyone to domesticate us. We don't need Mom or Dad, the school or the church to domesticate us. We are so well trained that we are our own domesticator. \fbox{We are an autodomesticated animal.} We can now domesticate ourselves according to the same belief system we were given, \& using the same system of punishment \& reward. We punish ourselves when we don't follow the rules according to our belief system; we reward ourselves when we are the ``good boy'' or ``good girl.''

The belief system is like a Book of Law that rules our mind. Without question, whatever is in that Book of law, is our truth. We base all of our judgments according to the Book of Law, even if these judgments go against our own inner nature. Even moral laws like the 10 Commandments are programmed into our mind in the process of domestication. 1 by 1, all these agreements go into the Book of Law, \& these agreements rule our dream.

There is something in our minds that judges everybody \& everything, including the weather, the dog, the cat -- everything. The inner Judge uses what is in our Book of Law to judge everything we do \& don't do, everything we think \& don't think, \& everything we feel \& don't feel. Everything lives under the tyranny\footnote{\textbf{tyranny} [n] [uncountable, countable] (plural \textbf{tyrannies}) \textbf{1.} unfair or cruel use of power or authority; \textbf{2.} the rule of a tyrant; a country under this rule, \textsc{synonym}: \textbf{dictatorship}.} of this Judge. Every time we do something that goes against the Book of Law, the Judge says we are guilty, we need to be punished, we should be ashamed\footnote{\textbf{ashamed} [a] [not before noun] \textbf{1.} \textbf{ashamed (of somebody\texttt{/}something\texttt{/}yourself)} feeling shame or embarrassment about somebody\texttt{/}something or because of something you have done; \textbf{2.} \textbf{ashamed to do something} unwilling to do something because of shame or embarrassment.}. This happens many times a day, day after day, for all the years of our lives.

There is another part of us that receives the judgments, \& this part is called the Victim. The Victim carries the blame, the guilt, \& the shame. It is the part of us that says, ``Poor me, I'm not good enough, I'm not intelligent enough, I'm not attractive enough, I'm not worthy of love, poor me.'' The big Judge agrees \& says, ``Yes, you are not good enough.'' \& this is all based on a belief system that we never chose to believe. These beliefs are so strong, that even years later when we are exposed to new concepts \& try to make our own decisions, we find that these beliefs still control our lives.

Whatever goes against the Book of Law will make you feel a funny sensation\footnote{\textbf{sensation} [n] \textbf{1.} [countable] a feeling that you get when something affects your body; \textbf{2.} [uncountable] the ability to feel through your sense of touch, \textsc{synonym}: \textbf{feeling}; \textbf{3.} [countable, usually singular] \textbf{sensation (of something)} a general feeling or impression that is difficult to explain; an experience or a memory; \textbf{4.} [countable, usually singular] very great surprise, excitement or interest among a lot of people.} in your solar plexus\footnote{\textbf{solar plexus} [n] [singular] \textbf{1.} (\textit{anatomy}) a system of nerves at the base of the stomach; \textbf{2.} (\textit{informal}) the part of the body at the top of the stomach, below the ribs.}, \& it's called fear. Breaking the rules in the Book of Law opens your emotional wounds\footnote{\textbf{wound} [n] \textbf{1.} an injury to part of the body, especially one in which a hole is made in the skin; \textbf{2.} mental or emotional pain caused by something unpleasant that has been said or done to you; [v] [often passive] \textbf{1.} \textbf{wound somebody\texttt{/}something} to injure part of the body, especially by making a hole in the skin using a weapon; \textbf{2.} \textbf{wound somebody\texttt{/}something} to hurt somebody's feelings.}, \& your reaction is to create emotional poison\footnote{\textbf{poison} [n] [uncountable, countable] \textbf{1.} a substance that causes death or harm if it is swallowed or absorbed by a living thing; \textbf{2.} \textbf{poison (of something)} an idea, a feeling, etc. that is extremely harmful; [v] \textbf{1.} \textbf{poison somebody\texttt{/}yourself (with something)} to harm or kill a person or an animal by giving them poison either deliberately or by accident; \textbf{2.} \textbf{poison something} to put poison in or on something; \textbf{3.} \textbf{poison something} to have an extremely harmful effect on something.}. Because everything that is in the Book of Law has to be true, anything that challenges what you believe is going to make you feel unsafe. Even if the Book of Law is wrong, it makes you \textit{feel safe}.

That is why we need a great deal of courage to challenge our own beliefs. Because even if we know we didn't choose all these beliefs, it is also true that we agreed to all of them. The agreement is so strong that even if we understand the concept of it not being true, we feel the blame, the guilt, \& the shame that occur if we go against these rules.

Just as the government has a book of laws that rule the society's dream, our belief system is the Book of Laws that rules our personal dream. All these laws exist in our mind, we believe them, \& the Judge inside us bases everything on these rules. The Judge decrees\footnote{\textbf{decree} [n] \textbf{1.} [countable, uncountable] an official order from a ruler or government that becomes the law; \textbf{2.} [countable] a decision that is made in court; [v] to decide, judge or order something officially.}, \& the Victim suffers the guilt \& punishment. But who says there is justice\footnote{\textbf{justice} [n] \textbf{1.} [uncountable] the fair treatment of people. In English law, \textbf{natural justice} is a term that includes the right to a fair hearing \& the right to be judged without bias. \textsc{opposite}: \textbf{injustice}; \textbf{2.} [uncountable] the quality of being fair or reasonable, \textsc{opposite}: \textbf{injustice}; \textbf{3.} [uncountable] the legal system used to punish people who have committed crimes; \textbf{4.} [uncountable] a fair punishment for a crime, especially from the legal system; \textbf{5.} (\textbf{Justice}) [countable] a judge or magistrate in a court.} in this dream? True justice is paying only once for each mistake. True \textit{injustice}\footnote{\textbf{injustice} [n] [uncountable, countable] \textbf{injustice (of something\texttt{/}doing something)} the fact of a situation being unfair \& of people not being treated equally; an unfair act or an example of unfair treatment, \textsc{opposite}: \textbf{justice}.} is paying more than once for each mistake.

How many times do we pay for 1 mistake? The answer is thousands of times. The human is the only animal on earth that pays a thousand times for the same mistake. The rest of the animals pay once for every mistake they make. But not us. \fbox{We have a powerful memory.} We make a mistake, we judge ourselves, we find ourselves guilty, \& we punish ourselves. If justice exists, then that was enough; we don't need to do it again. But every time we remember, we judge ourselves again, we are guilty again, \& we punish ourselves again, \& again, \& again. If we have a wife or husband he or she also reminds us of the mistake, so we can judge ourselves again, punish ourselves again, \& find ourselves guilty again. Is this fair?

How many times do we make our spouse, our children, or our parents pay for the same mistake? Every time we remember the mistake, we blame them again \& send them all the emotional poison we feel at the injustice, \& then we make them pay again for the same mistake. Is that justice? The Judge in the mind is wrong because the belief system, the Book of Law, is wrong. The whole dream is based on false law. 95\% of the beliefs we have stored in our minds are nothing but lies, \& we suffer because we believe all these lies.

\fbox{In the dream of the planet it is normal for humans to suffer, to live in fear, \& to create emotional dramas.} The outside dream is not a pleasant dream; it is a dream of violence, a dream of fear, a dream of war, a dream of injustice. The personal dream of humans will vary, but globally\footnote{\textbf{globally} [adv] in a way that involves the whole world.} it is mostly a nightmare\footnote{\textbf{nightmare} [n] \textbf{1.} a dream that is very frightening or unpleasant; \textbf{2.} (\textit{rather informal}) an experience that is very frightening \& unpleasant, or very difficult to deal with.}. If we look at human society we see a place so difficult to live in because it is ruled by fear. Throughout the world we see human suffering, anger, revenge\footnote{\textbf{revenge} [n] [uncountable] something that you do in order to make somebody suffer because they have made you suffer.}, addictions\footnote{\textbf{addiction} [n] [uncountable, countable] the condition of being addicted to something.}, violence\footnote{\textbf{violence} [n] [uncountable] \textbf{1.} behavior involving physical force that is intended to hurt, damage or kill somebody\texttt{/}something; \textbf{2.} \textbf{violence (of something)} great force or strength; \textbf{do violence to something} [idiom] to damage or have a harmful effect on something; to go against something.} in the street, \& tremendous injustice. It may exist at different levels in different countries around the world, but fear is controlling the outside dream.

If we compare the dream of human society with the description of hell that religions all around the world have promulgated\footnote{\textbf{promulgate} [v] (\textit{formal}) \textbf{1.} [usually passive] \textbf{promulgate something} to spread an idea, a belief, etc. among many people; \textbf{2.} \textbf{promulgate something} to announce a new law or system officially or publicly.}, we find they are exactly the same. Religions say that hell is a place of punishment, a place of fear, pain, \& suffering, a place where the fire burns you. Fire is generated by emotions that come from fear. Whenever we feel the emotions of anger\footnote{\textbf{anger} [n] [uncountable] the strong feeling that you have when something has happened that you think is bad \& unfair; [v] [often passive] \textbf{anger somebody} to make somebody angry.}, jealousy\footnote{\textbf{jealousy} [n] (plural \textbf{jealousies}) \textbf{1.} [uncountable] a feeling of being jealous; \textbf{2.} [countable] an action or a remark that shows that a person is jealous.}, envy\footnote{\textbf{envy} [n] [uncountable] the feeling or wanting to be in the same situation as somebody else; the feeling of wanting something that somebody else has, \textsc{synonym}: \textbf{jealousy}; \textbf{be the envy of somebody\texttt{/}something} [idiom] to be a person or thing that other people admire \& that causes feelings of envy; [v] \textbf{1.} to wish you had the same qualities, possessions, opportunities, etc. as somebody else; \textbf{2.} to be glad that you do not have to do what somebody else has to do.}, or hate, we experience a fire burning within us. We are living in a dream of hell.

\fbox{If you consider hell as a state of mind, then hell is all around us.} Others may warn us that if we don't do what they say we should do, we will go to hell. Bad news! We are already in hell, including the people who tell us that. No human can condemn\footnote{\textbf{condemn} [v] \textbf{1.} \textbf{condemn somebody\texttt{/}something} to express very strong disapproval of somebody\texttt{/}something, usually for moral reasons; \textbf{2.} [usually passive] \textbf{condemn somebody (to something)} to say what somebody's punishment will be, \textsc{synonym}: \textbf{sentence}; \textbf{3.} [usually passive] \textbf{condemn somebody to something} to force somebody to accept a difficult or unpleasant situation; \textbf{4.} [usually passive] \textbf{condemn something (as something)} to say officially that something is not safe enough to be used.} another to hell because we are already there. Others can put us into a deeper hell, true. But only if we allow this to happen.

Every human has his or her own personal dream, \& just like the society dream, it is often ruled by fear. We learn to dream hell in our own life, in our personal dream. The same fears manifest\footnote{\textbf{manifest} [v] \textbf{1.} [transitive] \textbf{manifest something (in something)} to show something clearly, especially a feeling, attitude or quality, \textsc{synonym}: \textbf{demonstrate}; \textbf{2.} [intransitive, transitive] to appear or become easy to notice; [a] easy to see or understand, \textsc{synonym}: \textbf{clear}.} in different ways for each person, of course, but we experience anger, jealousy, hate, envy, \& other negative emotions. Our personal dream can also become an ongoing\footnote{\textbf{ongoing} [a] [usually before noun] continuing to exist or develop.} nightmare where we suffer \& live in a state of fear. But we don't need to dream a nightmare. It is possible to enjoy a pleasant dream.

\fbox{All of humanity is searching for truth, justice, \& beauty.} We are on an eternal\footnote{\textbf{eternal} [a] without an end; existing or continuing forever.} search for the truth because we only believe in the lies we have stored in our mind. We are searching for justice because in the belief system we have, there is no justice. We search for beauty because it doesn't matter how beautiful a person is, we don't believe that person has beauty. We keep searching \& searching, when everything is already within us. There is no truth to find. Wherever we turn our heads, all we see is the truth, but with the agreements \& beliefs we have stored in our mind, we have no eyes for this truth.

We don't see the truth because we are blind. What blinds us are all those false beliefs we have in our mind. We have the need to be right \& to make others wrong. We trust what we believe, \& our beliefs set us up for suffering. It is as if we live in the middle of a fog that doesn't let us see any further than our own nose. We live in a fog that is not even real. This fog is a dream, your personal dream of life -- what you believe, all the concepts you have about what you are, all the agreements you have made with others, with yourself, \& even with God.

Your whole mind is a fog which the Toltecs called a \textit{mitote} (pronounced MIH-TOE '-TAY). Your mind is a dream where a thousand people talk at the same time, \& nobody understands each other. This is the condition of the human mind -- a big \textit{mitote}, \& with that big \textit{mitote} you cannot see what you really are. In India they call the \textit{mitote maya}, which means illusion. It is the personality's notion of ``I am.'' Everything you believe about yourself \& the world, all the concepts \& programming you have in your mind, are all the \textit{mitote}. We cannot see who we truly are; we cannot see that we are not free.

That is why humans resist\footnote{\textbf{resist} [v] \textbf{1.} [transitive, intransitive] to refuse to accept something \& try to stop it from happening, \textsc{synonym}: \textbf{oppose}; \textbf{2.} [transitive] \textbf{resist something} to not be harmed, damaged or changed by something; \textbf{3.} [transitive, intransitive] (often in negative sentences) to stop yourself from having something you like or doing something you very much want to do; \textbf{4.} [intransitive, transitive] to fight back when attacked; to use force to stop something from happening.} life. \fbox{To be alive is the biggest fear humans have.} \fbox{Death is not the biggest fear we have}; \fbox{our biggest fear is taking the risk to be alive -- the risk to be alive \& express what we really are}. Just being ourselves is the biggest fear of humans. We have learned to live our lives trying to satisfy other people's demands. We have learned to live by other people's points of view because of the fear of not being accepted \& of not being good enough for someone else.

During the process of domestication, we form an image of what perfection is in order to try to be good enough. We create an image of how we should be in order to be accepted by everybody. We specially try to please the ones who love us, like Mom \& Dad, big brothers \& sisters, the priests \& the teacher. Trying to be good enough for them, we create an image of perfection, but we don't fit this image. We create this image, but this image is not real. We are never going to be perfect from this point of view. Never!

\fbox{Not being perfect, we reject ourselves.} \& the level of self-rejection depends upon how effective the adults were in breaking our integrity\footnote{\textbf{integrity} [n] [uncountable] \textbf{1.} the quality of being honest \& having strong moral principles; \textbf{2.} \textbf{integrity of something} the state of being whole \& not divided, \textsc{synonym}: \textbf{unity}; \textbf{3.} \textbf{integrity of something} the state of not being spoilt, or of not having mistakes.}. After domestication it is no longer about being good enough for anybody else. We are not good enough for ourselves because we don't fit with our own image of perfection. We cannot forgive ourselves for not being what we wish to be, or rather what we \textit{believe} we should be. We cannot forgive ourselves for not being perfect.

We know we are not what we believe we are supposed to be \& so we feel false, frustrated\footnote{\textbf{frustrated} [a] \textbf{1.} feeling impatient \& slightly angry because you cannot do or achieve what you want; \textbf{2.} (of an emotion) having no effect; not being satisfied.}, \& dishonest\footnote{\textbf{dishonest} [a] not honest; intending to deceive people, \textsc{opposite}: \textbf{honest}.}. We try to hide ourselves, \& we pretend to be what we are not. The result is that we feel unauthentic\footnote{\textbf{inauthentic} [a] not what somebody claims it is; that you cannot believe or rely on, \textsc{opposite}: \textbf{authentic}.} \& wear social masks to keep others from noticing this. We are so afraid that somebody else will notice that we are not what we pretend to be. We judge others according to our image of perfection as well, \& naturally they fall short of our expectations.

We dishonor\footnote{\textbf{dishonour} [n] (also \textbf{dishonor}) [uncountable] (\textit{formal}) a loss of honor or respect because you have done something unacceptable or morally wrong; [v] (\textit{formal}) \textbf{1.} \textbf{dishonor somebody\texttt{/}something} to make somebody\texttt{/}something lose the respect of other people; \textbf{2.} \textbf{dishonor something} to refuse to keep an agreement or a promise, \textsc{opposite}: \textbf{honor}.} ourselves just to please other people. We even do harm to our physical bodies just to be accepted by others. You see teenagers taking drugs just to avoid being rejected by other teenagers. They are not aware that the problem is that they don't accept themselves. They reject themselves because they are not what they pretend to be. They wish to be a certain way, but they are not, \& for this they carry shame \& guilt. Humans punish themselves endlessly for not being what they believe they should be. They become very self-abusive\footnote{\textbf{self-abuse} [n] [uncountable] \textbf{1.} behavior by which a person does harm to himself or herself; \textbf{2.} (\textit{old-fashioned}) masturbation ($=$ the act of giving yourself sexual pleasure by rubbing your sexual organs).}, \& they use other people to abuse themselves as well.

But nobody abuses us more than we abuse ourselves, \& it is the Judge, the Victim, \& the belief system that make us do this. True, we find people who say their husband or wife, or mother or father, abused them, but you know that we abuse ourselves much more than that. The way we judge ourselves is the worst judge that ever existed. If we make a mistake in front of people, we try to deny the mistake \& cover it up. But as soon as we are alone, the Judge becomes so strong, the guilt is so strong, \& we feel so stupid, or so bad, or so unworthy\footnote{\textbf{unworthy} [a] (\textit{formal}) \textbf{1.} \textbf{unworthy (of something)} not having the necessary qualities to deserve something, especially respect, \textsc{opposite}: \textbf{worthy}; \textbf{2.} \textbf{unworthy (of somebody)} not accepted from somebody, especially somebody has an important job or high social position, \textsc{synonym}: \textbf{unbefitting}.}.

\fbox{In your whole life nobody has ever abused you more than you have abused yourself.} \& the limit of your self-abuse is exactly the limit that you will tolerate\footnote{\textbf{tolerate} [v] \textbf{1.} \textbf{tolerate something\texttt{/}somebody} to allow the existence of something that you do not agree with or do not like, \& not take action against it, \textsc{synonym}: \textbf{put up with somebody\texttt{/}something}; \textbf{2.} \textbf{tolerate something\texttt{/}somebody} to accept something\texttt{/}somebody that is unpleasant without complaining, \textsc{synonym}: \textbf{put up with somebody\texttt{/}something}; \textbf{3.} \textbf{tolerate something} to be able to be affected by a drug, difficult conditions, etc. without being harmed.} from someone else. If someone abuses you a little more than you abuse yourself, you will probably walk away from that person. But if someone abuses you a little less than you abuse yourself, you will probably stay in the relationship \& tolerate it endlessly\footnote{\textbf{endless} [a] \textbf{1.} very large in size or amount \& seeming to have no end; \textbf{2.} continuing for a long time \& seeing to have no end.}.

If you abuse yourself very badly, you can even tolerate someone who beats you up, humiliates\footnote{\textbf{humiliate} [v] \textbf{humiliate somebody\texttt{/}yourself\texttt{/}something} to make somebody feel ashamed or stupid \& lose the respect of other people.} you, \& treats you like dirt\footnote{\textbf{dirt} [n] [uncountable] \textbf{1.} any substance that makes something dirty, e.g. dust, soil or mud; \textbf{2.} (\textit{especially North American English}) loose earth or soil.}. Why? Because in your belief system you say, ``I deserve it. This person is doing me a favor by being with me. I'm not worthy of love \& respect. I'm not good enough.''

We have the need to be accepted \& to be loved by others, but we cannot accept \& love ourselves. The more self-love we have, the less we will experience self-abuse. Self-abuse comes from self-rejection, \& self-rejection comes from having an image of what it means to be perfect \& never measuring up to that ideal. Our image of perfection is the reason we reject ourselves; it is why we don't accept ourselves the way we are, \& why we don't accept others the way they are.''

\subsection{Prelude to a new dream}
p. 19

\section{\textsc{The 1st Agreement}: Be Impeccable with Your Word}

\section{\textsc{The 2nd Agreement}: Don't Take Anything Personally}

\section{\textsc{The 3rd Agreement}: Don't Make Assumptions}

\section{\textsc{The 4th Agreement}: Always Do Your Best}

\section{\textsc{The Toltec Path to Freedom}: Breaking Old Agreements}

\section{\textsc{The New Dream}: Heaven on Earth Prayers}

%------------------------------------------------------------------------------%

%\selectlanguage{english}
%\begin{thebibliography}{99}
%	\bibitem[]{}
%\end{thebibliography}

%------------------------------------------------------------------------------%

\printbibliography[heading=bibintoc]
	
\end{document}