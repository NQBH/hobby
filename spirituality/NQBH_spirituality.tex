\documentclass[oneside]{book}
\usepackage[backend=biber,natbib=true,style=authoryear]{biblatex}
\addbibresource{/home/hong/1_NQBH/reference/bib.bib}
\usepackage[vietnamese,english]{babel}
\usepackage{tocloft}
\renewcommand{\cftsecleader}{\cftdotfill{\cftdotsep}}
\usepackage[colorlinks=true,linkcolor=blue,urlcolor=red,citecolor=magenta]{hyperref}
\usepackage{amsmath,amssymb,amsthm,mathtools,float,graphicx}
\allowdisplaybreaks
\numberwithin{equation}{section}
\newtheorem{assumption}{Assumption}[chapter]
\newtheorem{conjecture}{Conjecture}[chapter]
\newtheorem{corollary}{Corollary}[chapter]
\newtheorem{definition}{Definition}[chapter]
\newtheorem{example}{Example}[chapter]
\newtheorem{lemma}{Lemma}[chapter]
\newtheorem{notation}{Notation}[chapter]
\newtheorem{principle}{Principle}[chapter]
\newtheorem{problem}{Problem}[chapter]
\newtheorem{proposition}{Proposition}[chapter]
\newtheorem{question}{Question}[chapter]
\newtheorem{remark}{Remark}[chapter]
\newtheorem{theorem}{Theorem}[chapter]
\usepackage[left=0.5in,right=0.5in,top=1.5cm,bottom=1.5cm]{geometry}
\usepackage{fancyhdr}
\pagestyle{fancy}
\fancyhf{}
\lhead{\small \textsc{Sect.} ~\thesection}
\rhead{\small \nouppercase{\leftmark}}
\renewcommand{\sectionmark}[1]{\markboth{#1}{}}
\cfoot{\thepage}
\def\labelitemii{$\circ$}

\title{Spirituality}
\author{\selectlanguage{vietnamese} Nguyễn Quản Bá Hồng\footnote{Independent Researcher, Ben Tre City, Vietnam\\e-mail: \texttt{nguyenquanbahong@gmail.com}}}
\date{\today}

\begin{document}
\maketitle
\setcounter{secnumdepth}{4}
\setcounter{tocdepth}{4}
\tableofcontents

%------------------------------------------------------------------------------%

\chapter{Wikipedia's}

\section{\href{https://en.wikipedia.org/wiki/Spirituality}{Wikipedia\texttt{/}Spirituality}}
``The meaning of \textit{spirituality} has developed \& expanded over time, \& various meanings can be found alongside each other. Traditionally, spirituality referred to a \href{https://en.wikipedia.org/wiki/Religion}{religious} process of re-information which ``aims to recover the original shape of man'', oriented at ``the \href{https://en.wikipedia.org/wiki/Image_of_God}{image of God}'' as exemplified by the founders \& sacred texts of the religions of the world. The term was used within early \href{https://en.wikipedia.org/wiki/Christianity}{Christianity} to refer to a life oriented toward the \href{https://en.wikipedia.org/wiki/Holy_Spirit_(Christianity)}{Holy Spirit} \& broadened during the \href{https://en.wikipedia.org/wiki/Late_Middle_Ages}{Late Middle Ages} to include mental aspects of life.

In modern times, the term both spread to other religious traditions \& broadened to refer to a wide range of experience, including a range of \href{https://en.wikipedia.org/wiki/Western_esotericism}{esoteric traditions} \& religious traditions. Modern usages tend to refer to a subjective experience of a sacred dimension \& the ``deepest values \& meanings by which people live'', often in a context separate from organized religious institutions. This may involve belief in a \href{https://en.wikipedia.org/wiki/Supernatural}{supernatural} realm beyond the ordinarily observable world, \href{https://en.wikipedia.org/wiki/Personal_growth}{personal growth}, a quest for an ultimate or sacred \href{https://en.wikipedia.org/wiki/Meaning_of_life}{meaning}, \href{https://en.wikipedia.org/wiki/Religious_experience}{religious experience}, or an encounter with one's own ``inner dimension''.

\subsection{Etymology}

\subsection{Definition}

\subsection{Development of the meaning of spirituality}

\subsubsection{Classical, medieval \& early modern periods}

\subsubsection{Modern spirituality}

\paragraph{Transcendentalism \& Unitarian Universalism.}

\paragraph{Theosophy, anthroposophy, \& the perennial philosophy.}

\paragraph{Neo-Vedanta.}

\paragraph{``Spiritual but not religious''.}

\subsection{Traditional spirituality}

\subsubsection{Abrahamic faiths}

\paragraph{Judaism.}

\paragraph{Christianity.}

\paragraph{Islam.}

\subparagraph{Sufism.}

\subsubsection{Asian traditions}

\paragraph{Buddhism.}

\paragraph{Hinduism.}

\subparagraph{4 paths.}

\subparagraph{Schools \& spirituality.}

\paragraph{Jainism.}

\paragraph{Sikhism.}

\subsubsection{African spirituality}

\subsection{Contemporary spirituality}

\subsubsection{Characteristics}

\subsubsection{Spiritual experience}

\subsubsection{Spiritual practices}

\subsection{Science}

\subsubsection{Relation to science}

\subsubsection{Quantum mysticism}

\subsubsection{Scientific research}

\paragraph{Health \& well-being.}

\paragraph{Intercessionary prayer.}

\paragraph{Spiritual care in health care professions.}

\paragraph{Spiritual experiences.}

\paragraph{Measurement.}

%------------------------------------------------------------------------------%

%\selectlanguage{english}
%\begin{thebibliography}{99}
%	\bibitem[]{}
%\end{thebibliography}

%------------------------------------------------------------------------------%

\printbibliography[heading=bibintoc]
	
\end{document}