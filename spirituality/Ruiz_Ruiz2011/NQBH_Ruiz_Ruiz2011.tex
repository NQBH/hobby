\documentclass{article}
\usepackage[backend=biber,natbib=true,style=authoryear]{biblatex}
\addbibresource{/home/nqbh/reference/bib.bib}
\usepackage{tocloft}
\renewcommand{\cftsecleader}{\cftdotfill{\cftdotsep}}
\usepackage[colorlinks=true,linkcolor=blue,urlcolor=red,citecolor=magenta]{hyperref}
\usepackage{algorithm,algpseudocode,amsmath,amssymb,amsthm,float,graphicx,mathtools}
\allowdisplaybreaks
\numberwithin{equation}{section}
\newtheorem{assumption}{Assumption}[section]
\newtheorem{conjecture}{Conjecture}[section]
\newtheorem{corollary}{Corollary}[section]
\newtheorem{definition}{Definition}[section]
\newtheorem{example}{Example}[section]
\newtheorem{lemma}{Lemma}[section]
\newtheorem{notation}{Notation}[section]
\newtheorem{principle}{Principle}[section]
\newtheorem{problem}{Problem}[section]
\newtheorem{proposition}{Proposition}[section]
\newtheorem{question}{Question}[section]
\newtheorem{remark}{Remark}[section]
\newtheorem{theorem}{Theorem}[section]
\usepackage[left=0.5in,right=0.5in,top=1.5cm,bottom=1.5cm]{geometry}
\usepackage{fancyhdr}
\pagestyle{fancy}
\fancyhf{}
\lhead{\small Sect.~\thesection}
\rhead{\small\nouppercase{\leftmark}}
\renewcommand{\sectionmark}[1]{\markboth{#1}{}}
\cfoot{\thepage}
\def\labelitemii{$\circ$}

\title{The 5th Agreement: A Practical Guide to Self-Mastery (Toltec Wisdom)}
\author{Don Miguel Ruiz, Don Jose Ruiz, Janet Mills}
\date{\today}

\begin{document}
\maketitle
\tableofcontents
\vspace{5mm}
\begin{quotation}
	To every human who lives on this beautiful planet, \& to the generations to come.
\end{quotation}

%------------------------------------------------------------------------------%

\section*{The Toltec}
``Thousands of years ago, the Toltec were known throughout southern Mexico as ``women \& men of knowledge.'' Anthropologists have spoken of the Toltec as a nation or a race, but, in fact, the Toltec were scientists \& artists who formed a society to explore \& conserve the spiritual knowledge \& practices of the ancient ones. They came together as masters (\textit{naguals}) \& students at Teotihuacan, the ancient city of pyramids outside Mexico City known as the place where ``Man becomes God.'' Over the millennia, the \textit{naguals} were forced to conceal the ancestral wisdom \& maintain its existence in obscurity. European conquest, coupled with rampant misuse of personal power by a few of the apprentices, made it necessary to shield the knowledge from those were not prepared to use it widely or who might intentionally misuse it for personal gain.

Fortunately, the esoteric Toltec knowledge was embodied \& passed on through generations by different lineagues of \textit{naguals}. Though it remained veiled in secrecy for hundreds of years, ancient prophecies foretold the coming of an age when it would be necessary to return the wisdom to the people. Now, don Miguel Ruiz \& don Jose Ruiz (\textit{naguals} from the Eagle Knight lineague) have been guided to share with us the powerful teachings of the Toltec.

Toltec wisdom arises from the same essential unity of truth as all the sacred esoteric traditions found around the world. Though it is not a religion, it honors all the spiritual masters who have taught on the earth. While it does embrace spirit, it is most accurately described as a way of life, distinguished by the ready accessibility of happiness \& love.'' -- \cite[p. 12]{Ruiz_Ruiz2011}

%------------------------------------------------------------------------------%

\section*{Introduction}

\begin{flushright}
	by \textsc{don Miguel Ruiz}
\end{flushright}
``The 4 Agreements was published many years ago. If you have read the book, you already know what these agreements can do. They have the ability to transform your life by breaking thousands of limiting agreements you have made with yourself, with other people, with \textit{life} itself.

The 1st time you read \textit{The 4 Agreements}, it begins to work its magic. It goes much deeper than the words you are reading. You feel that you already know every word in the book. You feel it, but perhaps you never put it into words. When you read the book the 1st time, it challenges what you believe, \& takes you to the limit of your comprehension. You break many limiting agreements, \& overcome many challenges, but then you see new challenges. When you read the book a 2nd time, it feels as if you're reading a completely different book because the limits of your comprehension have already grown. Once again, it takes you into a deeper awareness of yourself, \& you reach the limit that you can reach in that moment. \& when you read the book a 3rd time, it's just as if you're reading another book.

Just like magic, because they \textit{are} magic, the 4 Agreements slowly help you to recover your authentic self. With practice, these 4 simple agreements take you to what you \textit{really} are, not what you pretend to be, \& this is exactly where you want to be: what you really are.

The principles in \textit{The 4 Agreements} speak to the heart of all human beings, from the young to the old. They speak to people of different cultures all around the world -- people who speak different languages, people whose religious \& philosophical beliefs are vastly different. They have been taught at different kinds of schools, from elementary schools to secondary schools to universities. The principles in \textit{The 4 Agreements} reach everyone because they are pure common sense.

Now it's time to give another gift: \textit{The 5th Agreement}. The 5th agreement wasn't included in my 1st book because the 1st 4 agreements were enough of a challenge at that time. The 5th agreement is made with words, of course, but its meaning \& intent is beyond the words. The 5th agreement is ultimately about seeing your whole reality with the eyes of truth, \textit{without} words. The result of practicing the 5th agreement is the complete acceptance of yourself just the way you are, \& the complete acceptance of everybody else just the way they are. The reward is your eternal happiness.

Many years ago, I began teaching some of the concepts in this book to my apprentices, but then I stopped because nobody seemed to understand what I was trying to say. Though I had shared the 5th agreement with my apprentices, I discovered that nobody was ready to learn the teachings that underlie this agreement. Years later, my son, don Jose, started to share the same teachings with a group of students, \& he succeeded where I had failed. Maybe the reason don Jose was successful was because he had complete faith in sharing the message. His very presence spoke the truth \& challenged the beliefs of the people who attended his classes. He made a huge difference in their lives.

Don Jose Ruiz has been my apprentice since he was a child, since he learned to speak. In this book, I am honored to introduce my son, \& to present the essence of the teachings we delivered together over a period of 7 years.

To keep the messages as personal as possible, \& keeping with the 1st-person voice of prior books in the Toltec Wisdom series, we have opted to present \textit{The 5th Agreement} in the same 1st-person style of writing. In this book, we speak to the reader with 1 voice, \& with 1 heart.'' -- \cite[pp. 14--15]{Ruiz_Ruiz2011}

%------------------------------------------------------------------------------%

\begin{center}\Large\bf
	Part I: The Power of Symbols
\end{center}

%------------------------------------------------------------------------------%

\section{In The Beginning. It's All in the Program}
``From the moment you are born, you deliver a message to the world. \textit{What is th message?} The message is \textit{you}, that child. It's the presence of an \textit{angel}, a messenger from the infinite in a human body. The infinite, a total power, creates a program just for you, \& everything you need to be what you are is in the program. You are born, you grow up, you mate, you grow old, \& in the end you return to the infinite. Every cell in your body is a universe of its own. It's intelligent, it's complete, \& it's programmed to be whatever it is.

You are programmed to be \textit{you}, whatever you are, \& it makes no difference to the program what your mind \textit{thinks} you are. The program is not in the thinking mind. It's in the body, in what we call the \textit{DNA}, \& in the beginning, you instinctively follow its wisdom. As a very young child, you know what you like, what you don't like, when you like it, when you don't. You follow what you like, \& you try to avoid what you don't like. You follow your instincts, \& those instincts guide you to be happy, to enjoy life, to play, to love, to fulfill your needs. Then what happens?

Your body begins to develop, your mind begins to mature, \& you begin to use symbols to deliver your message. Just as the birds understand the birds, \& the cats understand the cats, the human understand the humans through a symbology. If you were born on an island \& then lived all alone, it might take you 10 years, but you would give a name to everything that you see, \& you would use that language to communicate a message, even if it was only to yourself. \textit{Why would you do this?} Well, it's easy to understand, \& it's not because humans are so intelligent. It's because we are programmed to create a language, to invent an entire symbology for ourselves.

As you know, all around the world humans speak \& write in thousands of different languages. Humans have invented all kinds of symbols to communicate not only with other humans but more importantly with ourselves. The symbols can be sounds that we speak, motions that we make, or handwriting \& signs that are graphic in nature. There are symbols for objects, ideas, music, \& mathematics, but the introduction of sounds is the very 1st step, which means we learn to use symbols to speak.

The humans who come before us already have names for everything that exists, \& they teach us the meaning of sounds. They call this a \textit{table}; they call that a \textit{chair}. They also have names for things that only exist in our imagination, like mermaids \& unicorns. Every word that we learn is a symbol for something real or imagined, \& there are thousands of words to learn. If we observe children who are 1--4 years old, we can see the effort they make trying to learn an entire symbology. It's a big effort that we usually don't remember because our mind is not yet mature, but with repetition \& practice we finally learn to speak.

Once we learn to speak, the humans who take care of us teach us what they know, which means they program us with knowledge. The humans we live with have lots of knowledge, including all the social, religious, \& moral rules of their culture. They hook our attention, pass on the information, \& teach us to be like them. We learn how to be a man or a woman according to the society in which we are born. We learn how to behave the ``right'' way in our society, which means how to be a ``good'' human.

In truth, we are domesticated the same way that a dog, a cat, or any animal is domesticated: through a system of punishment \& reward. We are told that we're a \textit{good boy} or a \textit{good girl} when we do what the grown-ups want us to do; we're a \textit{bad boy} or a \textit{bad girl} when we don't do what they want us to do. Sometimes we are punished without being bad, \& sometimes we are rewarded without being good. Out of fear of being punished \& fear of not getting a reward, we start trying to please other people. We try to be good, because bad people don't receive rewards; they are punished.

In human domestication, all the rules \& values of our family \& society are imposed on us. We don't have the opportunity to choose our beliefs; we are told what to believe, \& what not to believe. The people we live with tell us their opinions: what is good \& what is bad, what is right \& what is wrong, what is beautiful \& what is ugly. Just like a computer, all that information is downloaded into our head. We are innocent; we \textit{believe} what our parents or other grown-ups tell us; we \textit{agree}, \& the information is stored in our memory. Everything we learn goes into our mind by agreement, \& it stays in our mind by agreement, but 1st it goes through the attention.

The attention is very important in humans because it's the part of the mind that makes it possible for us to concentrate on a single object or thought out of a whole range of possibilities. Through the attention, information from the outside is conveyed to the inside \& vice versa. The attention is the channel we use to send \& receive messages from human to human. It's like a bridge from 1 mind to another mind; we open the bridge with sounds, signs, symbols, touch -- with any event that hooks the attention. This is how we teach, \& this is how we learn. We cannot teach anything if we don't have someone's attention; we cannot learn anything if we don't pay attention.

Using the attention, the grown-ups teach us how to create an entire reality in our mind with the use of symbols. After they teach us symbology by sound, the grown-ups drill us with our ABCs, \& we learn the same language, but graphically. Our imagination begins to develop, our curiosity grows stronger, \& we start to ask questions. We ask \& ask, \& we keep asking questions; we gather information from everywhere. \& we know that we've finally mastered a language when we are able to use the symbols to talk to ourselves in our head. This is when we learn to \textit{think}. Before that, we don't think; we mimic sounds \& use symbols to communicate, but life is simple before we attach any meaning or emotion to the symbols.

Once we give meaning to the symbols, we begin to use them to try to make sense of everything that happens in our lives. We use the symbols to think about things that are real, \& to think about things that aren't real, but that we start to imagine are real, like beautiful \& ugly, skinny \& fat, smart \& stupid. \& if you notice, we can only think in a language that we master. For many years, I spoke only Spanish, \& it took a long time for me to master enough symbols in English to think in English. To master a language is not easy, but at a certain point, we find ourselves \textit{thinking} with the symbols we learn.

By the time we go to school, when we are 5 or 6 years old, we understand the meaning of abstract concepts like right \& wrong, winner \& loser, perfect \& imperfect. In school, we continue to learn how to read \& write the symbols we already know, \& the written language makes it possible for us to accumulate more knowledge. We continue to give meaning to more \& more symbols, \& thinking becomes not only effortless but automatic.

Now the symbols that we learned are hooking our attention all by themselves. It's what we know that's talking to us, \& we are listening to what our knowledge says. I call it \textit{the voice of knowledge} because knowledge is talking in our head. Many times we hear the voice with different tonalities; we hear the voice of our mother other, our father, our brothers \& sisters, \& the voice never stops talking. The voice isn't real; it's our creation. But we \textit{believe} that it's real because we give it life through the power of our faith, which means we believe \textit{without a doubt} what that voice is telling us. This is when the opinions of the humans around us start taking over our mind.

Everybody has an opinion of us, \& they tell us what we are. As very young children, we don't know what we are. The only way we can see ourselves is through a mirror, \& other people act as that mirror. Our mother tells us what we are, \& we believe her. It's completely different from what our father tells us, or what our brothers \& sisters tell us, but we agree with them, too. People tell us how we look, \& it's especially true when we are little children. ``Look, you have the eyes of your mother, the nose of your grandfather.'' We listen to all the opinions of our family, our teachers, \& the big children at school. We see our image in those mirrors, we agree that this is what we are, \& as soon as we agree, that opinion becomes a part of our belief system. Little by little, all these opinions modify our behavior, \& in our mind we form an image of ourselves according to what other people say we are: ``I'm beautiful; I'm not so beautiful. I'm smart; I'm not so smart. I'm a winner; I'm a loser. I'm good at this; I'm bad at that.''

At a certain point, all the opinions of our parents \& teachers, religion \& society, make us believe that we need to be a certain way in order to be accepted. They tell us the way we \textit{should be}, the way we \textit{should} look, the way we \textit{should} behave. We need to be \textit{this} way; we shouldn't be \textit{that} way -- \& because it's not okay for us to be what we are, we start pretending to be what we are not. The fear of being rejected becomes the fear of not being good enough, \& we start searching for something that we call \textit{perfection}. In our search, we form an image of perfection, the way we wish to be, but we know that we are not, \& we begin to judge ourselves for that. We don't like ourselves, \& we tell ourselves, ``Look how silly you look, how ugly you are. Look how fat, how short, how weak, how stupid you are.'' This is when the drama begins, because now the symbols are going against us. We don't even notice that we've learned to use the symbols to reject ourselves.

Before domestication, we don't care what we are or what we look like. Our tendency is to explore, to express our creativity, to seek pleasure \& avoid pain. As little children, we are wild \& free; we run around naked without self-consciousness or selfjudgment. We speak the truth because we live in truth. Our attention is in the moment; we are not afraid of the future or ashamed of the past. After domestication, we try to be good enough for everybody else, but we are no longer good enough for ourselves, because we can never live up to our image of perfection.

All of our normal human tendencies are lost in the process of domestication, \& we begin to search for what we have lost. We start searching for freedom because we are no longer free to be what we really are; we start searching for happiness because we are no longer happy; we start searching for beauty because we no longer believe that we are beautiful.

We continue to grow, \& in our adolescence, our body is programmed to introduce a substance we call \textit{hormones}. Our physical body is no longer a child's, \& we don't fit in with the way of life we lived before. We don't want to hear our parents tell us what to do \& what not to do. We want our freedom; we want to be ourselves, but we are also afraid to be by ourselves. People tell us, ``You're not a child anymore,'' but we're not an adult either, \& it's a difficult time for most humans. By the time we are teenagers, we don't need anymore to domesticate us; we have learned to judge ourselves, punish ourselves, \& reward ourselves according to the same belief system we were given, \& using the same system of punishment \& reward. The domestication may be easier for people in some places in the world, \& harder for people in other places, but in general none of us has a chance of escaping the domestication. None of us.

Finally, the body matures \& everything changes again. We start searching once again, but now, more \& more, what we are searching for is our \textit{self}. We are searching for love because we have learned to believe that love is somewhere outside of us; we are searching for justice because there is no justice in the belief system we were taught; we are searching for truth because we only believe in the knowledge we have stored in our mind. \& of course, we're still searching for perfection because now we agree with the rest of the humans that ``nobody's perfect.'''' -- \cite[pp. 19--24]{Ruiz_Ruiz2011}

%------------------------------------------------------------------------------%

\section{Symbols \& Agreements. The Art of Humans}

%------------------------------------------------------------------------------%

\section{The Story of You. The 1st Agreement: Be Impeccable with Your Word}

%------------------------------------------------------------------------------%

\section{Every Mind is a World. The 2nd Agreement: Don't Take Anything Personally}

%------------------------------------------------------------------------------%

\section{Truth or Fiction. The 3rd Agreement: Don't Make Assumptions}

%------------------------------------------------------------------------------%

\section{The Power of Belief. The Symbol of Santa Claus}

%------------------------------------------------------------------------------%

\section{Practice makes The Master. The 4th Agreement: Always Do Your Best}

%------------------------------------------------------------------------------%

\begin{center}\Large\bf
	Part II: The Power of Doubt
\end{center}

\section{The Power of Doubt. The 5th Agreement: Be Skeptical, but Learn to Listen}

%------------------------------------------------------------------------------%

\section{The Dream of The 1st Attention. The Victims}

%------------------------------------------------------------------------------%

\section{The Dream of the 2nd Attention. The Warriors}

%------------------------------------------------------------------------------%

\section{The Dream of the 3rd Attention. The Masters}

%------------------------------------------------------------------------------%

\section{Becoming a Seer. A New Point of View}

%------------------------------------------------------------------------------%

\section{The 3 Languages. What Kind of Messenger Are You?}

%------------------------------------------------------------------------------%

\section{Epilogue: Help Me to Change the World}

%------------------------------------------------------------------------------%

\printbibliography[heading=bibintoc]
	
\end{document}