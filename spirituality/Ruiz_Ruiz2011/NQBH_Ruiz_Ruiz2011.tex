\documentclass{article}
\usepackage[backend=biber,natbib=true,style=authoryear]{biblatex}
\addbibresource{/home/nqbh/reference/bib.bib}
\usepackage{tocloft}
\renewcommand{\cftsecleader}{\cftdotfill{\cftdotsep}}
\usepackage[colorlinks=true,linkcolor=blue,urlcolor=red,citecolor=magenta]{hyperref}
\usepackage{algorithm,algpseudocode,amsmath,amssymb,amsthm,float,graphicx,mathtools}
\allowdisplaybreaks
\numberwithin{equation}{section}
\newtheorem{assumption}{Assumption}[section]
\newtheorem{conjecture}{Conjecture}[section]
\newtheorem{corollary}{Corollary}[section]
\newtheorem{definition}{Definition}[section]
\newtheorem{example}{Example}[section]
\newtheorem{lemma}{Lemma}[section]
\newtheorem{notation}{Notation}[section]
\newtheorem{principle}{Principle}[section]
\newtheorem{problem}{Problem}[section]
\newtheorem{proposition}{Proposition}[section]
\newtheorem{question}{Question}[section]
\newtheorem{remark}{Remark}[section]
\newtheorem{theorem}{Theorem}[section]
\usepackage[left=0.5in,right=0.5in,top=1.5cm,bottom=1.5cm]{geometry}
\usepackage{fancyhdr}
\pagestyle{fancy}
\fancyhf{}
\lhead{\small Sect.~\thesection}
\rhead{\small\nouppercase{\leftmark}}
\renewcommand{\sectionmark}[1]{\markboth{#1}{}}
\cfoot{\thepage}
\def\labelitemii{$\circ$}

\title{The 5th Agreement: A Practical Guide to Self-Mastery (Toltec Wisdom)}
\author{Don Miguel Ruiz, Don Jose Ruiz, Janet Mills}
\date{\today}

\begin{document}
\maketitle
\tableofcontents
\vspace{5mm}
\begin{quotation}
	To every human who lives on this beautiful planet, \& to the generations to come.
\end{quotation}

%------------------------------------------------------------------------------%

\section*{The Toltec}
``Thousands of years ago, the Toltec were known throughout southern Mexico as ``women \& men of knowledge.'' Anthropologists have spoken of the Toltec as a nation or a race, but, in fact, the Toltec were scientists \& artists who formed a society to explore \& conserve the spiritual knowledge \& practices of the ancient ones. They came together as masters (\textit{naguals}) \& students at Teotihuacan, the ancient city of pyramids outside Mexico City known as the place where ``Man becomes God.'' Over the millennia, the \textit{naguals} were forced to conceal the ancestral wisdom \& maintain its existence in obscurity. European conquest, coupled with rampant misuse of personal power by a few of the apprentices, made it necessary to shield the knowledge from those were not prepared to use it widely or who might intentionally misuse it for personal gain.

Fortunately, the esoteric Toltec knowledge was embodied \& passed on through generations by different lineagues of \textit{naguals}. Though it remained veiled in secrecy for hundreds of years, ancient prophecies foretold the coming of an age when it would be necessary to return the wisdom to the people. Now, don Miguel Ruiz \& don Jose Ruiz (\textit{naguals} from the Eagle Knight lineague) have been guided to share with us the powerful teachings of the Toltec.

Toltec wisdom arises from the same essential unity of truth as all the sacred esoteric traditions found around the world. Though it is not a religion, it honors all the spiritual masters who have taught on the earth. While it does embrace spirit, it is most accurately described as a way of life, distinguished by the ready accessibility of happiness \& love.'' -- \cite[p. 12]{Ruiz_Ruiz2011}

%------------------------------------------------------------------------------%

\section*{Introduction}

\begin{flushright}
	by \textsc{don Miguel Ruiz}
\end{flushright}
``The 4 Agreements was published many years ago. If you have read the book, you already know what these agreements can do. They have the ability to transform your life by breaking thousands of limiting agreements you have made with yourself, with other people, with \textit{life} itself.

The 1st time you read \textit{The 4 Agreements}, it begins to work its magic. It goes much deeper than the words you are reading. You feel that you already know every word in the book. You feel it, but perhaps you never put it into words. When you read the book the 1st time, it challenges what you believe, \& takes you to the limit of your comprehension. You break many limiting agreements, \& overcome many challenges, but then you see new challenges. When you read the book a 2nd time, it feels as if you're reading a completely different book because the limits of your comprehension have already grown. Once again, it takes you into a deeper awareness of yourself, \& you reach the limit that you can reach in that moment. \& when you read the book a 3rd time, it's just as if you're reading another book.

Just like magic, because they \textit{are} magic, the 4 Agreements slowly help you to recover your authentic self. With practice, these 4 simple agreements take you to what you \textit{really} are, not what you pretend to be, \& this is exactly where you want to be: what you really are.

The principles in \textit{The 4 Agreements} speak to the heart of all human beings, from the young to the old. They speak to people of different cultures all around the world -- people who speak different languages, people whose religious \& philosophical beliefs are vastly different. They have been taught at different kinds of schools, from elementary schools to secondary schools to universities. The principles in \textit{The 4 Agreements} reach everyone because they are pure common sense.

Now it's time to give another gift: \textit{The 5th Agreement}. The 5th agreement wasn't included in my 1st book because the 1st 4 agreements were enough of a challenge at that time. The 5th agreement is made with words, of course, but its meaning \& intent is beyond the words. The 5th agreement is ultimately about seeing your whole reality with the eyes of truth, \textit{without} words. The result of practicing the 5th agreement is the complete acceptance of yourself just the way you are, \& the complete acceptance of everybody else just the way they are. The reward is your eternal happiness.

Many years ago, I began teaching some of the concepts in this book to my apprentices, but then I stopped because nobody seemed to understand what I was trying to say. Though I had shared the 5th agreement with my apprentices, I discovered that nobody was ready to learn the teachings that underlie this agreement. Years later, my son, don Jose, started to share the same teachings with a group of students, \& he succeeded where I had failed. Maybe the reason don Jose was successful was because he had complete faith in sharing the message. His very presence spoke the truth \& challenged the beliefs of the people who attended his classes. He made a huge difference in their lives.

Don Jose Ruiz has been my apprentice since he was a child, since he learned to speak. In this book, I am honored to introduce my son, \& to present the essence of the teachings we delivered together over a period of 7 years.

To keep the messages as personal as possible, \& keeping with the 1st-person voice of prior books in the Toltec Wisdom series, we have opted to present \textit{The 5th Agreement} in the same 1st-person style of writing. In this book, we speak to the reader with 1 voice, \& with 1 heart.'' -- \cite[pp. 14--15]{Ruiz_Ruiz2011}

%------------------------------------------------------------------------------%

\begin{center}\Large\bf
	Part I: The Power of Symbols
\end{center}

%------------------------------------------------------------------------------%

\section{In The Beginning. It's All in the Program}
``From the moment you are born, you deliver a message to the world. \textit{What is th message?} The message is \textit{you}, that child. It's the presence of an \textit{angel}, a messenger from the infinite in a human body. The infinite, a total power, creates a program just for you, \& everything you need to be what you are is in the program. You are born, you grow up, you mate, you grow old, \& in the end you return to the infinite. Every cell in your body is a universe of its own. It's intelligent, it's complete, \& it's programmed to be whatever it is.

You are programmed to be \textit{you}, whatever you are, \& it makes no difference to the program what your mind \textit{thinks} you are. The program is not in the thinking mind. It's in the body, in what we call the \textit{DNA}, \& in the beginning, you instinctively follow its wisdom. As a very young child, you know what you like, what you don't like, when you like it, when you don't. You follow what you like, \& you try to avoid what you don't like. You follow your instincts, \& those instincts guide you to be happy, to enjoy life, to play, to love, to fulfill your needs. Then what happens?

Your body begins to develop, your mind begins to mature, \& you begin to use symbols to deliver your message. Just as the birds understand the birds, \& the cats understand the cats, the human understand the humans through a symbology. If you were born on an island \& then lived all alone, it might take you 10 years, but you would give a name to everything that you see, \& you would use that language to communicate a message, even if it was only to yourself. \textit{Why would you do this?} Well, it's easy to understand, \& it's not because humans are so intelligent. It's because we are programmed to create a language, to invent an entire symbology for ourselves.

As you know, all around the world humans speak \& write in thousands of different languages. Humans have invented all kinds of symbols to communicate not only with other humans but more importantly with ourselves. The symbols can be sounds that we speak, motions that we make, or handwriting \& signs that are graphic in nature. There are symbols for objects, ideas, music, \& mathematics, but the introduction of sounds is the very 1st step, which means we learn to use symbols to speak.

The humans who come before us already have names for everything that exists, \& they teach us the meaning of sounds. They call this a \textit{table}; they call that a \textit{chair}. They also have names for things that only exist in our imagination, like mermaids \& unicorns. Every word that we learn is a symbol for something real or imagined, \& there are thousands of words to learn. If we observe children who are 1--4 years old, we can see the effort they make trying to learn an entire symbology. It's a big effort that we usually don't remember because our mind is not yet mature, but with repetition \& practice we finally learn to speak.

Once we learn to speak, the humans who take care of us teach us what they know, which means they program us with knowledge. The humans we live with have lots of knowledge, including all the social, religious, \& moral rules of their culture. They hook our attention, pass on the information, \& teach us to be like them. We learn how to be a man or a woman according to the society in which we are born. We learn how to behave the ``right'' way in our society, which means how to be a ``good'' human.

In truth, we are domesticated the same way that a dog, a cat, or any animal is domesticated: through a system of punishment \& reward. We are told that we're a \textit{good boy} or a \textit{good girl} when we do what the grown-ups want us to do; we're a \textit{bad boy} or a \textit{bad girl} when we don't do what they want us to do. Sometimes we are punished without being bad, \& sometimes we are rewarded without being good. Out of fear of being punished \& fear of not getting a reward, we start trying to please other people. We try to be good, because bad people don't receive rewards; they are punished.

In human domestication, all the rules \& values of our family \& society are imposed on us. We don't have the opportunity to choose our beliefs; we are told what to believe, \& what not to believe. The people we live with tell us their opinions: what is good \& what is bad, what is right \& what is wrong, what is beautiful \& what is ugly. Just like a computer, all that information is downloaded into our head. We are innocent; we \textit{believe} what our parents or other grown-ups tell us; we \textit{agree}, \& the information is stored in our memory. Everything we learn goes into our mind by agreement, \& it stays in our mind by agreement, but 1st it goes through the attention.

The attention is very important in humans because it's the part of the mind that makes it possible for us to concentrate on a single object or thought out of a whole range of possibilities. Through the attention, information from the outside is conveyed to the inside \& vice versa. The attention is the channel we use to send \& receive messages from human to human. It's like a bridge from 1 mind to another mind; we open the bridge with sounds, signs, symbols, touch -- with any event that hooks the attention. This is how we teach, \& this is how we learn. We cannot teach anything if we don't have someone's attention; we cannot learn anything if we don't pay attention.

Using the attention, the grown-ups teach us how to create an entire reality in our mind with the use of symbols. After they teach us symbology by sound, the grown-ups drill us with our ABCs, \& we learn the same language, but graphically. Our imagination begins to develop, our curiosity grows stronger, \& we start to ask questions. We ask \& ask, \& we keep asking questions; we gather information from everywhere. \& we know that we've finally mastered a language when we are able to use the symbols to talk to ourselves in our head. This is when we learn to \textit{think}. Before that, we don't think; we mimic sounds \& use symbols to communicate, but life is simple before we attach any meaning or emotion to the symbols.

Once we give meaning to the symbols, we begin to use them to try to make sense of everything that happens in our lives. We use the symbols to think about things that are real, \& to think about things that aren't real, but that we start to imagine are real, like beautiful \& ugly, skinny \& fat, smart \& stupid. \& if you notice, we can only think in a language that we master. For many years, I spoke only Spanish, \& it took a long time for me to master enough symbols in English to think in English. To master a language is not easy, but at a certain point, we find ourselves \textit{thinking} with the symbols we learn.

By the time we go to school, when we are 5 or 6 years old, we understand the meaning of abstract concepts like right \& wrong, winner \& loser, perfect \& imperfect. In school, we continue to learn how to read \& write the symbols we already know, \& the written language makes it possible for us to accumulate more knowledge. We continue to give meaning to more \& more symbols, \& thinking becomes not only effortless but automatic.

Now the symbols that we learned are hooking our attention all by themselves. It's what we know that's talking to us, \& we are listening to what our knowledge says. I call it \textit{the voice of knowledge} because knowledge is talking in our head. Many times we hear the voice with different tonalities; we hear the voice of our mother other, our father, our brothers \& sisters, \& the voice never stops talking. The voice isn't real; it's our creation. But we \textit{believe} that it's real because we give it life through the power of our faith, which means we believe \textit{without a doubt} what that voice is telling us. This is when the opinions of the humans around us start taking over our mind.

Everybody has an opinion of us, \& they tell us what we are. As very young children, we don't know what we are. The only way we can see ourselves is through a mirror, \& other people act as that mirror. Our mother tells us what we are, \& we believe her. It's completely different from what our father tells us, or what our brothers \& sisters tell us, but we agree with them, too. People tell us how we look, \& it's especially true when we are little children. ``Look, you have the eyes of your mother, the nose of your grandfather.'' We listen to all the opinions of our family, our teachers, \& the big children at school. We see our image in those mirrors, we agree that this is what we are, \& as soon as we agree, that opinion becomes a part of our belief system. Little by little, all these opinions modify our behavior, \& in our mind we form an image of ourselves according to what other people say we are: ``I'm beautiful; I'm not so beautiful. I'm smart; I'm not so smart. I'm a winner; I'm a loser. I'm good at this; I'm bad at that.''

At a certain point, all the opinions of our parents \& teachers, religion \& society, make us believe that we need to be a certain way in order to be accepted. They tell us the way we \textit{should be}, the way we \textit{should} look, the way we \textit{should} behave. We need to be \textit{this} way; we shouldn't be \textit{that} way -- \& because it's not okay for us to be what we are, we start pretending to be what we are not. The fear of being rejected becomes the fear of not being good enough, \& we start searching for something that we call \textit{perfection}. In our search, we form an image of perfection, the way we wish to be, but we know that we are not, \& we begin to judge ourselves for that. We don't like ourselves, \& we tell ourselves, ``Look how silly you look, how ugly you are. Look how fat, how short, how weak, how stupid you are.'' This is when the drama begins, because now the symbols are going against us. We don't even notice that we've learned to use the symbols to reject ourselves.

Before domestication, we don't care what we are or what we look like. Our tendency is to explore, to express our creativity, to seek pleasure \& avoid pain. As little children, we are wild \& free; we run around naked without self-consciousness or selfjudgment. We speak the truth because we live in truth. Our attention is in the moment; we are not afraid of the future or ashamed of the past. After domestication, we try to be good enough for everybody else, but we are no longer good enough for ourselves, because we can never live up to our image of perfection.

All of our normal human tendencies are lost in the process of domestication, \& we begin to search for what we have lost. We start searching for freedom because we are no longer free to be what we really are; we start searching for happiness because we are no longer happy; we start searching for beauty because we no longer believe that we are beautiful.

We continue to grow, \& in our adolescence, our body is programmed to introduce a substance we call \textit{hormones}. Our physical body is no longer a child's, \& we don't fit in with the way of life we lived before. We don't want to hear our parents tell us what to do \& what not to do. We want our freedom; we want to be ourselves, but we are also afraid to be by ourselves. People tell us, ``You're not a child anymore,'' but we're not an adult either, \& it's a difficult time for most humans. By the time we are teenagers, we don't need anymore to domesticate us; we have learned to judge ourselves, punish ourselves, \& reward ourselves according to the same belief system we were given, \& using the same system of punishment \& reward. The domestication may be easier for people in some places in the world, \& harder for people in other places, but in general none of us has a chance of escaping the domestication. None of us.

Finally, the body matures \& everything changes again. We start searching once again, but now, more \& more, what we are searching for is our \textit{self}. We are searching for love because we have learned to believe that love is somewhere outside of us; we are searching for justice because there is no justice in the belief system we were taught; we are searching for truth because we only believe in the knowledge we have stored in our mind. \& of course, we're still searching for perfection because now we agree with the rest of the humans that ``nobody's perfect.'''' -- \cite[pp. 19--24]{Ruiz_Ruiz2011}

%------------------------------------------------------------------------------%

\section{Symbols \& Agreements. The Art of Humans}
``During all the years that we grow up, we make countless agreements with ourselves, with society, with everybody around us. But the most important agreements are the ones we make with ourselves by understanding the symbols we learned. The symbols are telling us what we believe about ourselves; they're telling us what we are \& what we are not, what is possible \& what is not possible. The voice of knowledge is telling us everything that we know, but \textit{who tells us if what we know is the truth?}

When we go to grammar school, high school, \& college, we acquire a lot of knowledge, but \textit{what do we really know? Do we master the truth?} No, we master a language, a symbology, \& that symbology is only the truth because we \textit{agree}, not because it's \textit{really} the truth. Wherever we are born, whatever language we learn to speak, we find that almost everything we know is really about agreements, beginning with the symbols that we learn.

If we are born in England, we learn English symbols. If we are born in China, we learn Chinese symbols. But whether we learn English, Chinese, Spanish, German, Russian, or any other language, the symbols only have value because we assign them a value \& agree on their meaning. If we don't agree, the symbols are meaningless. The word \textit{tree}, e.g., is meaningful for people who speak English, but ``tree'' doesn't mean anything unless we \textit{believe} that it means something, unless we \textit{agree}. What it means for you, it means for me, \& that's why we understand one another. What I'm saying right now you understand because we agree with the meaning of every word that was programmed in our mind. But this doesn't mean that we completely agree. Each of us gives a meaning to every word, \& it's not exactly the same for everyone.

If we focus our attention on the way any word is created, we find that whatever meaning we give to a word is there for no real reason. We put words together from nowhere; we make them up. Humans invent every sound, every letter, every graphic symbol. We hear a sound like ``A'' \& we say, ``This is the symbol for that sound.'' We draw a symbol to represent the sound, we put the symbol \& the sound together, \& we give it a meaning. Then every word in our mind has a meaning, but not because it's real, not because it's truth. It's just an agreement with ourselves, \& with everybody else who learns the same symbology.

If we travel to a country where people speak a different language, we suddenly realize the importance \& power of agreement. Un árbol es sólo un árbol, el sol es sólo el sol, la tierra es sólo la tierra si estamos de acuerdo. Ein Baum ist nur ein Baum, die Sonne ist nur die Sonne, die Erde ist nur die Erde wenn wir uns darauf verständigt haben. A tree is only a tree, the sun is only the sun, the earth is only the earth if we agree. These symbols have no meaning in France, Russia, Turkey, Sweden, or in any other place where the agreements are different.

If we learn to speak English \& we go to China, we hear people talking, but we don't understand a word they are saying. Nothing makes sense to us because it's not the symbology that we learned. Many things are foreign to us; it's like being in another world. If we visit their places of worship, we find that their beliefs are completely different, their rituals are completely different, their mythologies have nothing to do with what we learned. 1 way to understand their culture is to learn the symbols they use, which means their language, but if we learn a new way of being, a new religion or philosophy, this may create a conflict with what we learned before. New beliefs clash with old beliefs, \& the doubt comes right away: \textit{``What is right \& what is wrong? Is it true what I learned before? Is it true what I'm learning right now? What is the truth?''}

The truth is that all of our knowledge, 100\% of it, is nothing more than symbolism or words that we invent for the need to understand \& express what we perceive. Every word in our mind \& on this page is just a symbol, but every word has the power of our faith because we \textit{believe} in its meaning without a doubt. Humans construct an entire belief system made up of symbols; we build an entire edifice of knowledge. Then we use everything we know, which is nothing but symbology, to justify what we believe, to try to explain 1st to ourselves, then to everybody around us, the way we perceive ourselves, the way we perceive the entire universe.

If we are aware of this, then it's easy to understand that all of the different mythologies, religions, \& philosophies around the world, all of the different beliefs \& ways of thinking, are nothing but agreements with ourselves \& with other humans. \textit{They're our creation, but are they true?} Everything that exists is true: the earth is true, the stars are true, the entire universe has always been true. But the symbols that we use to construct what we know are only true because we say so.

There's a beautiful story in the Bible that illustrates the relationship between God \& humans. In this story, Adam \& God are walking together around the world, \& God asks Adam what he wants to name everything. 1 by 1, Adam gives a name to everything he perceives. ``Let's call this a \textit{tree}. Let's call this a \textit{bird}. Let's call this a \textit{flower} $\ldots$'' \& God agrees with Adam. The story is about the creation of symbols, the creation of an entire language, \& it works by agreement.

It's like 2 sides of the same coin: We can say that 1 side is pure perception, what Adam perceives; the other side is the meaning that Adam gives to whatever he perceives. There's the object of perception, which is the truth, \& there's our interpretation of the truth, which is just a point of view. The truth is objective, \& we call it \textit{science}. Our interpretation of the truth is subjective, \& we call it \textit{art}. Science \& art; the truth, \& our interpretation of the truth. The real truth is life's creation, \& it's the absolute truth because it's truth for everyone. Our interpretation of the truth is our creation, \& it's a relative truth because it's only truth by agreement. With this awareness, we can begin to understand the human mind.

All humans are programmed to perceive the truth, \& we don't need a language to do this. But in order to \textit{express} the truth, we need to use a language, \& that expression is our art. It's no longer the truth because words are symbols, \& symbols can only represent or ``symbolize'' the truth. E.g., we can see a tree even if we don't know the symbol ``tree.'' Without the symbol, we just see an object. The object is real, it is truth, \& we perceive it. Once we call it a \textit{tree}, we are using art to express a point of view. By using more symbols, we can describe the tree -- every leaf, every color. We can say it's a big tree, a small tree, a beautiful tree, an ugly tree, but is it the truth? No, the tree is still the same tree.

Our interpretation of the tree will depend on our emotional reaction to the tree, \& our emotional reaction will depend on the symbols that we use to recreate the tree in our mind. As you can see, our interpretation of the tree is not exactly the truth. But our interpretation is a \textit{reflection} of the truth, \& that reflection is what we call the \textit{human mind}. The human mind is nothing but a virtual reality. It isn't real. What's real is truth. What's truth is truth for everyone. But the virtual reality is our personal creation; it's our art, \& it's only ``truth'' for each one of us.

All humans are artists, \textit{all} of us. Every symbol, every word, is a little piece of art. From my point of view, \& thanks to our programming, our greatest masterpiece of art is the use of a language to create an entire virtual reality within our mind. The virtual reality we create could be a clear reflection of the truth, or it could be completely distorted. Either way, it's art. Our creation could be our personal heaven, or it could be our personal hell. It doesn't matter; it's art. But what we can do with the awareness of what is truth \& what is virtual is endless. The truth leads to self-mastery, to a life that's very easy; our distortion of the truth often leads to needless conflict \& human suffering. Awareness makes all the difference.

Humans are born with awareness; we are born to perceive the truth, but we accumulate knowledge, \& we learn to deny what we perceive. We practice not being aware, \& we master not being aware. The word is pure magic, \& we learn to use our magic against ourselves, against creation, against our own kind. To be aware means to open our eyes to see the truth. When we see the truth, we see everything just as it is, not the way we believe it is, not the way we wish it to be. Awareness opens the door to millions of possibilities, \& if we know that we are the artist of our own life, we can make a choice from all those possibilities.

What I'm sharing with you comes from my personal training, which I call \textit{Toltec Wisdom}. \textit{Toltec} is a N\'ahuatl word meaning \textit{artist}. From my point of view, to be a \textit{Toltec} has nothing to do with any philosophy or place in the world. To be a Toltec is just to be an artist. A Toltec is an artist of the spirit, \& as artists we like beauty; we don't like what is not beauty. If we become better artists, our virtual reality becomes a better reflection of the truth, \& we can create a masterpiece of heaven with our art.

Thousands of years ago, the Toltec created 3 masterpieces of the artist: \textit{the mastery of awareness, the mastery of transformation}, \& \textit{the mastery of love, intent}, or \textit{faith}. The separation is just for our understanding, because the 3 masteries become only one. The truth is only one, \& the truth is what we are talking about. These 3 masteries guide us out of suffering \& return us to our true nature, which is happiness, freedom, \& love.

The Toltec understood that we are going to create a virtual reality with or without awareness. If it's with awareness, we're going to enjoy our creation. \& whether we facilitate the transformation or resist it, our virtual reality is always transforming. If we practice the art of transformation, soon we're facilitating the transformation, \& instead of using our magic against ourselves, we are using our magic for the expression of our happiness \& our love. When we master love, intent, or faith, we master the dream of our life, \& when all 3 masteries are accomplished, we reclaim our divinity \& become one with God. This is the goal of the Toltec.

The Toltec didn't have the technology that we have at the present time; they didn't know about the virtual reality of computers, but they knew how to master the virtual reality of the human mind. The master of the human mind requires complete control of the attention -- the way we interpret \& react to information we perceive from inside of us \& outside of us. The Toltec understood that each one of us is just like God, but instead of creating, we re-create. \& \textit{what do we re-create?} What we perceive. That is what becomes the human mind.

If we can understand what the human mind is, \& what the human mind does, we can begin to separate reality from virtual reality, or pure perception, which is truth, from symbology, which is art. Self-mastery is all about awareness, \& it begins with self-awareness. 1st to be aware of what is real, \& then to be aware of what is virtual, which means what we believe about what is real. With this awareness, we know that we can change what is virtual by changing what we believe. What is real we cannot change, \& it doesn't matter what we believe.'' -- \cite[pp. 27--32]{Ruiz_Ruiz2011}

%------------------------------------------------------------------------------%

\section{The Story of You. The 1st Agreement: Be Impeccable with Your Word}
``For thousands of years humans have tried to understand the universe, nature, \& mainly \textit{human} nature. It's amazing to observe humans in action all around the world, in all the different places \& cultures that exist on this beautiful planet Earth. Humans make a lot of effort to understand, but in doing so we also make a lot of assumptions. As artists, we distort the truth \& create the most amazing theories; we create entire philosophies \& the most amazing religions; we create stories \& superstitions about everything, including ourselves. \& this is exactly the main point: \textit{We create them}.

Humans are born with the power of creation, \& we are constantly creating stories with the words that we learned. Every one us us uses the word to form our opinions, to express our point of view. Countless events are happening all around us, \& using the attention, we have the capacity to put all these events together in a story. We create the story of our own life, the story of our family, the story of our community, the story of our country, the story of humanity, the story of the entire world. Every one of us has a story that we share, a message that we deliver to ourselves, \& to everyone \& everything around us.

You were programmed to deliver a message, \& the creation of that message is your greatest art. \textit{What is the message?} Your \textit{life}. With that message, you create mainly the story of you, \& then a story about everything you perceive. You create an entire virtual reality in your mind, \& you live in that reality. When you think, you're thinking in your language; you're repeating in your mind all those symbols that mean something to you. You're giving yourself a message, \& that message is the truth for you because you believe that it's the truth.

The story of you is everything that you know about you, \& when I say this, I'm talking to you, knowledge, what you believe you are, not to \textit{you}, the human, what you \textit{really} are. As you can see, I make a distinction between you \& \textit{you} because 1 of you is real, \& 1 of you is not real. \textit{You}, the physical human, are real; \textit{you} are the truth. You, knowledge -- you're not real; you're virtual. You only exist because of the agreements you made with yourself \& with the other humans around you. You, knowledge, come from the symbols you hear in your head, from all the opinions of the people you love, the people you don't love, the people you know, \& mostly the people you'll never know.

\textit{Who is talking in your head?} You make the assumption that it's you. But  if you are the one who is talking, then \textit{who is listening?} You, knowledge, are the one who is talking in your head, telling you what you are. \textit{You}, the human, are listening, but \textit{you}, the human, existed long before you had knowledge. You existed long before you understood all those symbols, before you learned to speak, \& just like any child before he or she learns to speak, you were completely authentic. You didn't pretend to be what you are not. Without even knowing it, you trusted yourself completely; you loved yourself completely. Before you learned knowledge, you were totally free to be what you really are because all those opinions \& stories from other humans were not in your head already.

Your mind is full of knowledge, but how are you \textit{using} that knowledge? How are you using the word when it comes to describing yourself? When you look at yourself in a mirror, do you like what you see, or do you judge your body \& use all those symbols to tell yourself lies? Is it \textit{really} true that you are too short or too tall, too heavy or too thin? Is it \textit{really} true that you are not beautiful? Is it \textit{really} true that you're not perfect just the way you are?

\textit{Can you see all the judgments that you have about yourself?} Every judgment is just an opinion -- it's just a point of view -- \& that point of view wasn't there when you were born. Everything you think about yourself, everything you believe about yourself, is because you learned it. You learned the opinions from Mom, Dad, siblings, \& society. They sent all those images of how a body should look; they expressed all those opinions about the way you are, the way you are not, the way you \textit{should} be. They delivered a message, \& you agreed with that message. \& now you think so many things about what you are, but \textit{are they the truth?}

You see, the problem is not really knowledge; the problem is believing in a \textit{distortion} of knowledge -- \& that is what we call a \textit{lie}. What is the truth, \& what is the lie? What is real, \& what is virtual? Can you see the difference, or do you believe that voice in your head every time it speaks \& distorts the  truth while assuring you that what you believe is the way things really are? Is it \textit{really} true that you're not a good human, \& that you'll never be good enough? Is it \textit{really} true that you don't deserve to be happy? Is it \textit{really} true that you're not worthy of love?

Remember when a tree was no longer just a tree? Once you learn a language, you interpret a tree \& judge a tree according to everything that you know. That's when a tree becomes the beautiful tree, the ugly tree, the scary tree, the wonderful tree. Well, you do the same thing with yourself. You interpret yourself \& judge yourself according to everything that you know. That's when you become the good human, the bad human, the guilty one, the crazy one, the powerful one, the weak one, the beautiful one, the ugly one. You are what you believe you are. Then the 1st question is: ``What do you believe you are?''

If you use your awareness, you will see everything you believe, \& this is how you live your life. Your lief is totally dominated by the system of beliefs that you learned. Whatever you believe is creating the story that you're experiencing; whatever you believe is creating the emotions that you're experiencing. \& you may really want to believe that you \textit{are} what you believe, but the image is completely false. It's not \textit{you}.

The real you is unique \& it's beyond everything that you know, because the real you is the truth. You, the human, are the truth. Your physical presence is real. What you believe about yourself is not real, \& it's not important unless you want to create a better story for yourself. Truth or fiction; either way, the story that you're creating is a work of art. It's a wonderful story, a beautiful story, but it's just a story, \& it's as close to the truth as you can get by using symbols.

As an artist, there is no right or wrong way to create your art; there's just beauty or there's not beauty; there's happiness or there's not happiness. If you believe yourself to be an artist, then everything becomes possible again. Words are your paintbrush, \& your lief is the canvas. You can paint whatever you want to paint; you can even copy another artist's work -- but what you express with your paintbrush is the way you see yourself, the way you see the entire reality. What you paint is your lief, \& how it looks will depend on how you are using the word. When you realize this, it may dawn on you that the word is a powerful tool for creation. When you learn to use that tool with awareness, you can make history with the word. What history? Your life's history, of course. The story of \textit{you}.'' -- \cite[pp. 35--37]{Ruiz_Ruiz2011}

\subsection{The 1st Agreement: Be Impeccable with Your Word}
``This brings us to the 1st \& most important of the 4 Agreements: \textit{Be impeccable with your word}. The word is your power of creation, \& that power can be used in more than 1 direction. 1 direction is impecability, where the word creates a beautiful story -- your personal heaven on earth. The other direction is misuse of the word, where the word destroys everything around you, \& creates your personal hell.

The word, as a symbol, has the magic \& power of creation because it can reproduce an image, an idea, a feeling, or an entire story in your imagination. Just hearing the word \textit{horse} can reproduce an entire image in your mind. That's the power of a symbol. But it can even be more powerful than that. Just by saying 2 words, \textit{The Godfather}, a whole movie can appear in your mind. This is your magic, your power of creation, \& it begins with the word.

Perhaps you can understand why the Bible says, ``In the beginning was the word, \& the word was with God, \& the word was God.'' According to many religions, in the beginning, nothing existed, \& the very 1st thing that God created was the messenger, the angel who delivers a message. You may understand the need for something that could transfer information from 1 place to another place. Of course, from nowhere to nowhere seems a little complicated, but it's very simple at the same time. In the very beginning, God created the word, \& the \textit{word} is a messenger. Then if God created the word to deliver a message, \& if the word is a messenger, then that is what you are: a messenger, an angel.

The word exists because of a force that we call \textit{life, intent}, or \textit{God}. The word \textit{is} the force; it \textit{is} the intent, \& that's why our intent manifests through the word no matter what language we speak. The word is so important in the creation of everything, because the messenger starts to deliver messages, \& the entire creation appears out of nowhere.

Remember God \& Adam talking \& walking together? God creates reality, \& we recreate reality with the word. The virtual reality that we create is a reflection of reality; it's our interpretation of reality by use of the word. Nothing can exist without the word, because the word is what we use to create everything that we know.

If you notice, I'm changing all the symbols on purpose, so you can see that the different expressions mean the exact same thing. The symbols can change, but the meaning is the same in all the different traditions around the world. If you listen to the intent \textit{behind} the symbols, you will understand what I'm trying to say. Impeccability of the word is so important because the word is \textit{you}, the messenger. The word is all about the message you deliver, not just to everyone \& everything around you, but the message you deliver to yourself.

You're telling yourself a story, but is it the truth? If you're using the word to create a story with self-judgment \& self-rejection, then you're using the word against yourself, \& you're not being impeccable. When you're impeccable, you're not going to tell yourself, ``I'm old. I'm ugly. I'm fat. I'm not good enough. I'm not strong enough. I'm never going to make it in life.'' You're not going to use your knowledge against yourself, which means your voice of knowledge is not going to use the word to judge you, find you guilty, \& punish you. Your mind is so powerful that it perceives the story that you create. If you create self-judgment, you create inner conflict that's nothing but a nightmare.

Your happiness is up to you, \& it depends on how you use the word. If you get angry \& use the word to send emotional poison to someone else, it appears that you're using the word against that person, but you're really using the word against yourself. That action is going to create a like reaction, \& that person is going to go against \textit{you}. If you insult someone, that person may even harm you in response. If you use the word to create a conflict in which your body may be injured, of course it's against you.

\textit{Be impeccable with your word} really means never use the power of the word against \textit{yourself}. When you're impeccable with your word, you never betray yourself. You never use the word to gossip about yourself or to spread emotional poison by gossiping about other people. Gossiping is the main form of communication in human society, \& we learn how to gossip by agreement. When we are children, we hear the adults around us gossiping about themselves, \& giving their opinions about other people, including people they don't even know. But now you are aware that our opinions are not the truth; they're just a point of view.

Remember, you are the creator of your own life story. If you use the word impeccably, just imagine the story that you are going to create for yourself. You're going to use the word in the direction of truth \& love for yourself. You're going to use the word to express the truth in every thought, in every action, in every word you use to describe yourself, to describe your own life story. \textit{\& what will be the result?} An extraordinarily beautiful life. In other words, you are going to be happy.

As you can see, impeccability of the word goes much deeper than it seems. The word is pure magic, \& when you adopt the 1st agreement, magic just happens in your life. Your intentions \& desires come easily because there is no resistance, there is no fear; there is only love. You are at peace, \& you create a life of freedom \& fulfillment in every way. Just this 1 agreement is enough to completely transform your life into your personal heaven. Always be aware of how you are using the word, \& \textit{be impeccable with your word}.'' -- \cite[pp. 37--40]{Ruiz_Ruiz2011}

%------------------------------------------------------------------------------%

\section{Every Mind is a World. The 2nd Agreement: Don't Take Anything Personally}
``When we are born, there are no symbols in our mind, but we have a brain \& we have eyes, \& our brain is already capturing images that come from light. We start perceiving light, we become familiar with light, \& the creation of our brain to light is an endless play of images in our imagination, in our mind. We are \textit{dreaming}. From the Toltec point of view, our whole life is a dream because the brain is programmed to dream 24 hours a day.

When the brain is awake, there is a material frame that makes us perceive things in a linear way; when the brain is asleep, there is no frame, \& the dream has the tendency to change constantly. Even with the brain awake, we have the tendency to daydream, \& the dream is constantly changing. The imagination is so powerful that it takes us to many places. We see things in our imagination that other people don't see; we hear things that other people don't hear, or maybe we don't, depending on the way we dream. Imagination gives movement to the images we see, but the images only exist in the mind, in the dream.

Light, images, imagination, dreaming, $\ldots$ You are dreaming right now, \& this is something that you can easily verify. Perhaps you never noticed that your mind is always dreaming, but if you use your imagination for just a moment, you will understand what I'm trying to explain to you. Imagine that you are looking into a mirror. Inside the mirror is a whole world of objects, but you know that what you see is just a reflection of what is real. It looks like it's real, it looks like the truth, but it's not real \& it's not the truth. If you try to touch the objects inside the mirror, you only touch the surface of the mirror.

What you see inside the mirror is just an \textit{image} of reality, which means it's a \textit{virtual} reality; it's a dream. \& it's the same kind of dream that humans dream with the brain awake. Why? Because what you see inside the mirror is a copy of reality that you create with the capacity of your eyes \& your brain. It's an \textit{image} of the world that you construct within your mind, which means it's how your own mind perceives reality. What a dog sees in the mirror is how the dog's brain perceives reality. What an eagle sees in that same mirror is how the eagle's brain perceives reality, \& it's different from your own.

Now imagine looking into your eyes instead of a mirror. Your eyes perceive light that's being reflected from millions of objects outside of your eyes. The sun sends light all around the world, \& every object reflects light. Billions of rays of light come from everywhere, go inside your eyes, \& project images of objects in your eyes. You think you are seeing all these objects, but the only thing you are \textit{really} seeing is light that's being reflected.

Everything you perceive is a reflection of what is real, just like the reflections in a mirror, except for 1 important difference. Behind the mirror there is nothing, but behind your eyes is a brain that tries to make sense of everything. Your brain is interpreting everything you perceive according to the meaning you give to every symbol, according to the structure of your language, according to all of the knowledge that was programmed in your mind. Everything you perceive is being filtered through your entire belief system. \& the result of interpreting everything you perceive by using everything you believe is your personal dream. This is how you create an entire virtual reality in your mind.

Perhaps you can see how easy it is for humans to distort what we perceive. Light reproduces a perfect image of what is real, but we distort the image by creating a story with all those symbols \& opinions that we learned. We dream about it with our imagination, \& by agreement we think that our dream is the absolute truth, when the real truth is that our dream is a relative truth, a \textit{reflection} of the truth that is always going to be distorted by all the knowledge we have stored in our memory.

Many masters have said that every mind is a world, \& it's true. The world we think we see outside of us is actually \textit{inside} of us. It's just \textit{images} in our imagination. It's a \textit{dream}. We are dreaming constantly, \& this has been known for centuries, not only in Mexico by the Toltec, but in Greece, in Rome, in India, in Egypt. People all over the world have said, ``Life is a dream.'' The question is, are we aware of it?

When we aren't aware that our mind is always dreaming, it's easy to blame everyone \& everything outside of us for all the distortions in our personal dream, for anything that makes us suffer in life. When we become aware that we are living in a dream that we artists are creating, we take a big step in our own evolution because now we can take responsibility for our creation. To realize that our mind is always dreaming gives us the key to changing our dream if we're not enjoying it.

\textit{Who is dreaming the story of your life?} You are. If you don't like your life, if you don't like what you believe about yourself, you are the only one who can change it. It's your world; it's your dream. If you're enjoying your dream, that's wonderful; then continue to enjoy each \& every moment. If your dream is a nightmare, if there's drama \& suffering, \& you're not enjoying your creation, then you can change it. As I'm sure you are aware, there are millions of books in this world written by millions of dreamers with different points of view. The story of you is as interesting as any of those books, \& it's even more interesting because your story continues to change. The way you dream when you are 10 years old is completely different from the way you dream when you are 15 or when you are 20, or 30, or 40, or the way you dream now.

The story you're dreaming today is not the same story that you were dreaming yesterday, or even half an hour ago. Every time you talk about your story, it changes depending on who you're telling the story to, depending on your physical \& emotional state at the time, depending on your beliefs at the time. Even if you try to tell the same story, your story is always changing. At a certain point, you find out that it's nothing but a story. It isn't reality; it's a virtual reality. It's nothing but a dream. \& it's a shared dream because all humans are dreaming at the same time. The shared dream of humanity, \textit{the dream of the planet}, was there before you were born, \& this is how you learned to create your own art, the story of you.'' -- \cite[pp. 43--45]{Ruiz_Ruiz2011}

\subsection{The 2nd Agreement: Don't Take Anything Personally}
``Let's use the power of our imagination to create a dream together, knowing that it's a dream. Imagine that you are in a gigantic mall where there are hundreds of movie theaters. You look around to see what's playing, \& you notice a movie that has your name. Amazing! You go inside the theater, \& it's empty except for 1 person. Very quietly, trying not to interrupt, you sit behind that person, who doesn't even notice you; all that person's attention is on the movie.

You look at the screen, \& what a big surprise! You recognize every character in the movie -- your mother, your father, your brothers \& sisters, your beloved, your children, your friends. Then you see the main character of the movie, \& it's you! You are the star of the movie \& it's the story of you. \& that person in front of you, well, it's also you, watching yourself act in the movie. Of course, the main character is just the way you believe you are, \& so are all the secondary characters because you know they story of you. After a while, you feel a little overwhelmed by everything you just witnessed, \& you decide to go to another theater.

In this theater there is also just 1 person watching a movie, \& she doesn't even notice when you sit beside her. You start watching the movie, \& you recognize all the characters, but now you're just a secondary character. This is the story of your mother's life, \& she is the one who is watching the movie with all her attention. Then you realize that your mother is not the same person who was in your movie. The way she projects herself is completely different in her movie. It's the way your mother wants everyone to perceive her. You know that it's not authentic. She's just acting. But then you begin to realize that it's the way she perceives \textit{herself}, \& it's kind of a shock.

Then you notice that the character who has your face is not the same person who was in your movie. You say to yourself, ``Ah, this isn't me,'' but now you can see how your mother perceives you, what she believes about you, \& it's far from what you believe about yourself. Then you see the character of your father, the way your mother perceives him, \& it's not at all the way you perceive him. It's completely distorted, \& so is her perception of all the other characters. You see the way your mother perceives your beloved, \& you even get a little upset with your mom. ``How dare she!'' You stand up \& get out of there.

You go to the next theater, \& it's the story of your beloved. Now you can see the way your beloved perceives you, \& the character is completely different from the one who was in your movie \& the one who was in your mother's movie. You can see the way your beloved perceives your children, your family, your friends. You can see the way you beloved wants to project him- or herself, \& it's not the way you perceive your beloved at all. Then you decide to leave that movie, \& go to your children's movie. You see the way your children see you, the way they see Grandpa, Grandma, \& you can hardly believe it. Then you watch the movies of your brothers \& sisters, of your friends, \& you find out that everybody is distorting all the characters in their movie.

After seeing all these movies, you decide to return to the 1st theater to see your own movie once again. You look at yourself acting in your movie, but you no longer believe anything you're watching; you no longer believe your own story because you can see that it's just a story. Now you know that all the acting you did your whole life was really for nothing because nobody perceives you the way you want to be perceived. You can see that all the drama that happens in your movie isn't really noticed by anybody around you. It's obvious that everybody's attention is focused on their own movie. They don't even notice when you're sitting right beside them in their theater! The actors have all their attention on their story, \& that is the only reality they live in. Their attention is so hooked by their own creation that they don't even notice their \textit{own} presence -- the one who is observing their movie.

In that moment, everything changes for you. Nothing is the same anymore, because now you see what's really happening. People live in their own world, in their own movie, in their own story. They invest all their faith in that story, \& that story is truth for them, but it's a relative truth, because it's not truth for you. Now you can see that all their opinions about you really concern the character who lives in their movie, not in yours. The one who they are judging in your name is a character they create. Whatever people think of you is really about the \textit{image} they have of you, \& that image isn't you.

At this point, it's clear that the people you love the most don't really know you, \& you don't know them either. The only thing you know about them is what you believe about them. You only know the image you created for them, \& that image has nothing to do with the real people. You thought that you knew your parents, your spouse, your children, \& your friends very well. The truth is you have no idea what is going on in their world -- what they are thinking, what they are feeling, what they are dreaming. What is even more surprising is that you thought you knew \textit{yourself}. Then you come to the conclusion that you don't even know yourself, because you've been acting for so long that you've mastered pretending to be what you are not.

With this awareness, you realize how ridiculous it is to say, ``My beloved doesn't understand me. Nobody understands me.'' Of course they don't. You don't even understand yourself. Your personality is always changing from 1 moment to the next, according to the role you are playing, according to the secondary characters in your story, according to the way you are dreaming at that time. At home, you have a certain personality. At work, your personality is completely different. With your female friends, it's 1 way; with your male friends, it's another way. But all your life you made the assumption that other people knew you so well, \& when they didn't do what you expected them to do, you took it personally, reacted with anger, \& used the word to create a lot of conflict \& drama for nothing.

Now it's easy to understand why there is so much conflict between humans. The world is populated by billions of dreamers who aren't aware that people are living in their own world, dreaming their own dream. From the point of view of the main character, which is their \textit{only} point of view, everything is all about them. When the secondary characters say something that doesn't agree with their point of view, they get angry, \& try to defend their position. They want the secondary characters to be the way they want them to be, \& if they are not, they feel very hurt. They take \textit{everything} personally. With this awareness, you can also understand the solution, \& it's something so simple \& logical: \textit{Don't take anything personality}.

Now the meaning of the 2nd agreement is profoundly clear. This agreement gives you immunity in the interaction you have with the secondary characters in your story. You don't have to concern yourself with other people points of view. Once you can see that nothing others say or do is about you, it doesn't matter who gossips about you, who blames you, who rejects you, who disagrees with your point of view. All the gossip doesn't affect you. You don't even bother to defend your point of view. You just let the dogs bark, \& surely they will bark, \& bark, \& bark. \textit{So what?} Whatever people say doesn't affect you because you are immune to their opinions \& their emotional poison. You are immune from the predators, the ones who use gossip to hurt other people, the ones who want to use other people to hurt themselves.

\textit{Don't take anything personally} is a beautiful tool of interaction with your own kind, human to human. \& it's a big ticket to personal freedom because you no longer have to rule your life according to other people's opinions. This really frees you! You can do whatever you want to do, knowing that whatever you do has nothing to do with anyone but you. The only person who needs to be concerned about the story of you is \textit{you}. This awareness changes everything. Remember, awareness of the truth is the 1st step to self-mastery, \& that is what you're doing right now. You're being reminded of the truth.

Now that you understand this truth, now that you are aware, \textit{how can you take anything personally anymore?} Once you understand that all humans live in their own world, in their own movie, in their own dream, the 2nd agreement is pure common sense: \textit{Don't take anything personally}.'' -- \cite[pp. 45--49]{Ruiz_Ruiz2011}

%------------------------------------------------------------------------------%

\section{Truth or Fiction. The 3rd Agreement: Don't Make Assumptions}
``For centuries, even millennia, humans have believed that a conflict exists in the human mind -- a conflict between good \& evil. But this isn't true. Good \& evil are just the result of the conflict, because the \textit{real} conflict is between the truth \& lies. Perhaps we should say that \textit{all} conflict is the result of lies, because the truth has no conflict at all. The truth doesn't need to prove itself; it exists whether we believe in it or not. Lies only exist if we create them, \& they only survive if we believe in them. Lies are just a distortion of the word, a distortion of the meaning of a message, \& that distortion is in the reflection, the human mind. Lies aren't real -- they're our creation -- but we give them life \& make them real in the virtual reality of our mind.

When I was a teenager, my grandfather told me this simple truth, but it took years for me to really understand it because I was always thinking ``How can we know the truth?'' I was using symbols to try to understand the truth, when the real truth is that the symbols have nothing to say about the truth. The truth existed long before humans created symbols.

As artists, we're always distorting the truth with symbols, but that's not the problem. As we said before, the problem is when we \textit{believe} that distortion, because some lies are innocent, \& others are deadly. Let's consider how we can use the word to create a story, a \textit{superstition}, about a chair. What do we know about a chair? We can say that a chair is made of wood, or metal, or cloth, but we're just using symbols to express a point of view. The truth is that we don't really know what the object is. But we can use the word with all of our authority to deliver a message to ourselves \& to every around us: ``This chair is ugly. I hate this chair.''

The message is already distorted, but this is just the beginning. We can say, ``It's a stupid chair, \& I think that whoever sits in the chair might become stupid also. I think we have to destroy the chair because if someone sits in the chair \& it falls apart, that person will fall \& break a hip. Oh yes, the chair is evil! Let's create a law against the chair so that everybody knows that it's a danger to society. From now on, it's prohibited to get near the evil chair!''

If we deliver this message, then whoever receives the message \& agrees with the message starts to become afraid of the evil chair. Very soon, there are people who are so afraid of the chair that they start having nightmares about it. They become obsessed with the evil chair, \& of course they have to destroy the chair before it destroys them.

\textit{Do you see what we can do with the word?} The chair is just an object. It exists, \& that's the truth. But the story we create about the chair is not the truth; it's a superstition. It's distorted message, \& that message is the lie. If we don't believe the lie, no problem. If we believe the lie \& try to impose that lie on other people, it can lead to what we call \textit{evil}. Of course, what we call \textit{evil} has many levels, depending on our personal power. Some people can lead the whole world into a great war where millions of people die. There are tyrants all around the world who invade other countries \& destroy their people because the tyrants believe in lies.

Now we can easily understand why there is a conflict in the human mind, \& only in the \textit{human} mind -- the virtual reality -- because it doesn't exist in the rest of nature. There are billions of humans who distort all those symbols in their heads \& deliver distorted messages. That's what really happened to humanity. I think that answers why all the wars exist, why all the injustice \& abuse exist, why the dream that we call \textit{hell} exists in the world of humans. Hell is nothing but a dream full of lies.

Remember, our dream is controlled by what we believe, \& what we believe could be truth, or could be fiction. The truth leads us to our authenticity, to happiness. Lies lead us to limitations in our lives, to suffering \& drama. Whoever believes in truth, lives in heaven. Whoever believes in lies, sooner or later lives in hell. We don't have to die to go on heaven or hell. Heaven is all around us, just as hell is all around us. Heaven is a point of view, a state of mind, \& so is hell. It's obvious that lies have been running every show in our head. Humans create the lies, \& then the lies control the humans. But sooner or later the truth arrives, \& the lies cannot survive the presence of the truth.

Centuries ago, people believed that the earth was flat. Some said that elephants were supporting the earth, \& that made them feel safe. ``Good, now we know that the earth is flat.'' Well, now we know that it isn't flat! The belief that the earth was flat was considered the truth, \& \textit{almost everybody agreed, but did that make it true?}

1 of the biggest lies we hear at the present time is: ``Nobody's perfect.'' It's a great excuse for our behavior, \& almost everybody agrees, but is it true? On the contrary, every human in this world is perfect, but we've been hearing this lie since we were children, \& as a consequence, we keep judging ourselves against an \textit{image} of perfection. We keep searching for perfection, \& in our search we find that everything in the universe is perfect except the humans. The sun is perfect, the stars are perfect, the planets are perfect, but when it comes to the humans, ``Nobody's perfect.'' The truth is that everything in creation is perfect, including the humans.

If we don't have the awareness to see this truth, it's because we are blinded by the lie. You may say, ``What about someone who is physically disabled? Is that person perfect?'' Well, according to what you know, that person may be imperfect, but is what you know the truth? Who says that what we call a \textit{disability} or even a \textit{disease} isn't perfect?

Everything about us is perfect, including any disability or disease that we may have. Someone with a learning difficulty is perfect; someone born without a finger or an arm or an ear is perfect; someone with a disease is perfect. Only perfection exists, \& that awareness is another important step in our evolution. To say otherwise is not to have the awareness of what we are. \& it's not enough to \textit{say} that we're perfect; we need to \textit{believe} that we're perfect. If we believe ourselves to be imperfect, that lie gathers more lies for support, \& together all those lies repress the truth \& guide the dream that we're creating for ourselves. Lies are nothing but superstitions, \& I can assure you that we live in a world of superstition. But again, \textit{are we aware of it?}

Just imagine waking up tomorrow morning in 14th-century Europe, knowing what you presently know, believing what you believe today. Imagine what those people would think of you, how they would judge you. They would put you on trial for taking a bath everyday. Everything you believe would threaten what they believe. How long would it take before they accused you of being a witch? They would torture you, make you confess to being a witch, \& finally kill you because of their fear of your beliefs. You can easily see that those people lived their lives immersed in superstition. Hardly anything they believed was true, \& you can easily see that because of what you believe today. But those people were not aware of their superstitions. Their way of life was completely normal for them; they didn't know any better because they never learned anything else.

Then perhaps what you believe about yourself is just as full of superstition as the beliefs of those people long ago. Just imagine if humans from 7 or 8 centuries in the future could see what most of us believe about ourselves today. The way that most of us relate to our own body is still barbarian, though not as much as 700 years ago. Our body is completely loyal to us, but we judge our body \& abuse our body; we treat it as if it's the enemy when it's our ally. Our society places a lot of importance on being attractive according to the images we see in the media -- on television, in movies, in fashion magazines. If we believe that we are not attractive enough according to these images, then we believe a lie, \& we are using the word against ourselves, against the truth.

The people in control of the media tell us what to believe, how to dress, what to eat, \& they manipulate humans like puppets, which means in whatever way they want. If they want us to hate someone, they spread gossip all around, \& the lies work their magic. When we stop being puppets, it's obvious that our lives have been guided by lies \& superstitions. Imagine what future humans would think of our superstitions. If they believed in the perfection of everything in creation, including every human, would we crucify them for their beliefs?

\textit{What is the truth \& what is the lie?} Once again, awareness is so important, because the truth doesn't come with words, with knowledge. But lies do, \& there are billions of lies. Humans believe so many lies because we aren't aware. We ignore the truth or we just don't see the truth. When we are domesticated, we accumulate a lot of knowledge, \& all that knowledge is just like a wall of fog that doesn't allow us to perceive the truth, what really \textit{is}. We only see what we want to see; we only hear what we want to hear. Our belief system is just like a mirror that only shows us what we believe.

In our development, as we grow throughout our lives, we learn so many lies that the whole structure of our lies becomes very complicated. \& we make it even more complicated because we \textit{think}, \& we \textit{believe} in what we think. We make the assumption that what we believe is the absolute truth, \& we never stop to consider that our truth is a relative truth, a virtual truth. Usually, it's not even close to any kind of truth, but it's the closest we can get without awareness.'' -- \cite[pp. 52--55]{Ruiz_Ruiz2011}

\subsection{The 3rd Agreement: Don't Make Assumptions}
``This takes us to the 3rd agreement: \textit{Don't make assumptions}. Making assumptions is just looking for trouble, because most assumptions are not the truth; they're fiction. 1 big assumption we make is that everything in our virtual reality is the truth. Another big assumption we make is that everything in everyone else's virtual reality is the truth. Well, now you know that none of the virtual realities are the truth!

Using our awareness, we can easily see all the assumptions we make, \& we can see how easy it is to make them. Humans have a powerful imagination, very powerful, \& there are so many ideas \& stories that we can imagine. We listen to the symbols talking in our head. We start imagining what other people are doing, what they're thinking, what they're saying about us, \& we dream things up in our imagination. We invent a whole story that's only truth for us, but we believe it. 1 assumption leads to another assumption; we jump to conclusions, \& we take our story very personally. Then we blame other people, \& we usually start gossiping to try to justify our assumptions. Of course, by gossiping, a distorted message becomes even more distorted.

Making assumptions \& then taking them personally is the beginning of hell in this world. Almost all of our conflicts are based on this, \& it's easy to understand why. Assumptions are nothing more than lies that we are telling ourselves. This creates a big drama for nothing, because we don't really know if something is true or not. Making assumptions is just looking for drama when there's no drama happening. \& if drama is happening in someone else's story, so what? It's not your story; it's someone else's story.

Be aware that almost everything you tell yourself is an assumption. If you're a parent, you know how easy it is to make assumptions about your children. It's midnight, \& your daughter isn't home yet. She went out to dance, \& you thought she would be home by now. You start imagining the worst; you start making assumptions: ``Oh, what if something happened to her? Maybe I should call the police.'' There are so many things you can imagine, \& you create a whole drama of possibilities in your head. 10 minutes later your daughter arrives home with a big smile. When the truth arrives \& all the lies are dispelled, you realize that you were simply torturing yourself for nothing. \textit{Don't make assumptions}.

If not taking anything personally gives you immunity in the interaction that you have with other people, then not making assumptions gives you immunity in the interaction that you have with yourself, with your voice of knowledge, or what we call \textit{thinking}. Making assumptions is all about thinking. We think too much, \& thinking leads to assumptions. Just thinking ``What if?'' can create a huge drama in our lives. Every human can think a lot, \& thinking brings fear. We have no control over all that thinking, all those symbols that we distort in our head. If we just stop thinking, we no longer try to explain anything to ourselves, \& this keeps us from making assumptions.

Humans have a need to explain \& justify everything; we have a need for knowledge, \& we make assumptions to fulfill our need to \textit{know}. We don't care whether the knowledge is true or not. Truth or fiction, we believe 100\% in what we believe, \& we go on believing it, because just having knowledge makes us feel safe. There are so many things that the mind cannot explain; we have all these questions that need answers. But instead of asking questions when we don't know something, we make all sorts of assumptions. If we just ask questions, we won't have to make assumptions. It's always better to ask \& be clear.

If we don't make assumptions, we can focus our attention on the truth, not on what we \textit{think} is the truth. Then we see life the way it is, not the way we want to see it. As we shall see, when we don't believe our own assumptions, the power of our belief that we invested in them returns to us. When we recover all the energy that we invested in making assumptions, we can use that energy to create a new dream: our personal heaven. \textit{Don't make assumptions}.'' -- \cite[pp. 55--57]{Ruiz_Ruiz2011}

%------------------------------------------------------------------------------%

\section{The Power of Belief. The Symbol of Santa Claus}
``There was a time in your life when you completely owned the power of your belief, but when you were educated to be a part of humanity, the power of your belief went into all those symbols that you learned, \& at a certain point the symbols gained power over you. In truth, the power of your belief went into \textit{everything} that you know, \& since then everything that you know has ruled your life. Obviously, when we are little children, we are overcome by the power of everyone else's beliefs. The symbols are a wonderful invention, but we are introduced to the symbols with the opinions \& beliefs already there. We ingest every opinion without questioning if it's truth or not. \& the problem is that by the time we master a language with all the opinions that we hear growing up, the symbols already have the power of our belief.

This isn't good or bad or right or wrong. It's just the way it is, \& it happens to all of us. We are learning to be a member of our society. We learn a language, we learn a religion or philosophy, we learn a way of being, \& we structure our whole belief system based on everything we are told. We have no reason to doubt what other people tell us until the 1st heartbreak happens, \& we find out that something they told us isn't true.

We go to school, \& we hear older kids talking. Referring to us, they say, ``You see that kid? He still believes in Santa Claus.'' Sooner or later, we find out that Santa Claus doesn't exist. Can you remember your reaction, how you \textit{felt} when you found out that Santa Claus was not the truth? I don't think your parents had bad intentions. Believing in Santa Claus is a wonderful tradition for millions of people. The lyrics of 1 song describe what we're told about the symbol we know as \textit{Santa}: ``You better watch out, you better not cry, you better not pout, I'm telling you why. Santa Claus is coming to town!'' We're told that Santa knows everything we do or don't do; he knows when we've been bad or good; he knows when we don't brush our teeth. \& we \textit{believe} this.

Christmas comes, \& we see a huge difference in the gifts that children receive. Let's say you ask Santa for a bicycle \& you were good the whole year. Your family is very poor. You open your gifts, \& you don't receive a bicycle. Your neighbor, who was very bad -- \& you know what \textit{very bad} means -- receives a bicycle. You say, ``I was good, this boy was bad, how come I didn't receive a bicycle? If Santa Claus really knows everything that I do, for sure he knows everything that my neighbor did. Why would Santa bring a bicycle to my neighbor \& not to me?''

It's just not fair, \& you don't understand why. Your emotional reaction is envy, anger, even sadness. You see the other little guy riding his bicycle very happily all around, behaving worse than he did before, \& you want to go \& hit him or break the bicycle. \textit{Injustice}. \& that sense of injustice is because you believe in a lie. It's an innocent lie, with no bad intention, of course, but you \textit{believe} it, \& you make an agreement with yourself: ``From now on, I won't be good. I'm going to be bad, like my neighbor.'' Later, you discover that Santa Claus is not true; he isn't real. But it's too late. You already released all the emotional poison; you already suffered the anger, the jealousy, the sadness. You already suffered from making an agreement that was based on a lie.

This is just 1 example of how we invest our faith in a symbol. There are hundreds, even thousands, of symbols, stories, \& superstitions that we learn. The symbol of Santa Claus demonstrates how even believing in an innocent lie can bring up emotions that feel like a fire burning inside us. They feel like poison -- they're hurting us, they're hurting our body -- \& we suffer from a story that isn't real. The emotions are real; they are part of the truth, but the reason we are feeling them is not real. It's not truth; it's fiction.

If you're asking yourself why you're so miserable at times, it's because you're telling yourself a story that isn't true, but you believe it. The truth is that your dream has become distorted, but that's not good or bad or right or wrong, because it's happening to billions of other people. You're not the only one in that situation, \& that's the good news.

The world of the symbols is extremely powerful because we make every symbol powerful with that force that comes from deep inside us -- that force that we call \textit{life, faith}, or \textit{intent}. We don't even realize that it's happening, but together all of the symbols form a whole structure made by agreements, \& we call it a \textit{belief system}. From a single letter to a word, from a single story to an entire philosophy, everything we agree to believe goes into that structure.

The belief system gives form \& structure to our virtual reality, \& with every agreement we make, the structure grows stronger \& gains more power until it becomes nearly as rigid as a brick building. If we imagine every symbol, every concept, every agreement as a brick, then our faith is the mortar that holds the bricks together. As we continue to learn throughout our lives, we mix the symbols in many directions, \& the concepts interact with themselves to create more complex concepts. The abstract mind becomes organized in a more complicated way, \& the structure keeps growing \& growing, until we have a totality of everything that we know.

This structure is what the Toltec call the \textit{human form}. The human form is not the form of the physical body; it's the form that our mind takes. It's the structure of our beliefs about ourselves, about everything that helps us make sense of our dream. The human form gives us our identity, but it's not the same as the frame of the dream. The frame of the dream is the material world as it is, which is truth. The human form is the belief system with all the elements of judgment. Everything in that belief system is our personal truth, \& we judge everything according to those beliefs, even if those beliefs go against our own inner nature.

In the process of domestication, the belief system becomes the \textit{book of law} that rules our lives. When we follow the rules according to our book of law, we reward ourselves; when we don't follow the rules, we punish ourselves. The belief system becomes the big judge in our mind, \& also the greatest victim because 1st it judges us, then it punishes us. The big judge is made by symbols, \& it works with symbols to judge everything we perceive, including the symbols! The victim is the part of us that receives the judgment \& suffers the punishment. \& when we interact with the outside dream, we judge \& punish everyone \& everything else according to our personal book of law.

The big judge is doing a perfect job, of course, because we agree with all those laws. The problem is that the belief system comes to life within us, \& uses our knowledge against us. It uses everything we know, all of our rules about how we have to live our lives, to punish the victim, which is the human. It uses our language to create the self-judgment, the self-rejection, the guilt, the shame. It verbally abuses us \& makes us miserable by creating our personal demons \& our personal dream of hell. There are so many symbols that we can use to say the same thing.

The belief system rules the human life like a tyrant. It takes our freedom away from us \& makes us its slave. It takes power over the \textit{real} us, the human life, \& it isn't even real! The real us stays hidden someplace in the mind, \& the one who controls the mind at that point is everything we know, everything we agreed to believe. The human body, which is beautiful \& perfect, becomes the victim of all the judgment \& punishment; it becomes just a vehicle where the mind acts \& projects itself through the body.

The belief system is in the realm of the mind; we cannot see it or measure it, but we know that it exists. Perhaps what we don't know is that this structure only exists because we create it. Our creation is completely attached to us; it follows us wherever we go. We've been living this way for so long that we don't even notice that we live in this structure. \& even though the mind isn't real -- it's virtual -- it's also \textit{total power} because it's also created by life.

Then something very important in the mastery of awareness is to be aware of our own creation, to be aware that it's alive. Every one of our beliefs, from a minimal one like the sound of a letter to a whole philosophy, is using our life force to survive. If we could see our mind in action, we would see millions of life forms, \& we would see that we are giving life to our creation by giving it the power of our faith, by giving it all of our attention. We are using our life force to support the whole structure. Without us, these ideas could not exist; without us, the whole structure would collapse.

If we use the power of our imagination, we can see the creation of our ``personal mythology,'' the construction of our belief system, \& the beginning of investing our faith in lies. In the process of all that construction -- all that learning we do -- there are many concepts that contradict other concepts. There are so many different dreams that we built, \& when we create so many structures, they go against one another \& annul the power of our word. In that moment, our word is almost nothing, because when there are 2 forces going in opposite directions, the result is zero. When there is only 1 force going in 1 direction, the power is immense, \& our intentions manifest just because we say so, just because our word has all the power of our faith.

As children, we invest our faith in almost everything that we learn, \& this is how we lose power over our own lives. By the time we grow up, our faith is already invested in so many lies that we hardly have any power left to create the dream that we want to create. The belief system has all the power of our faith, \& by the end of the equation we remain with almost zero faith, zero power. \& it's easy to see how we invest our faith in a symbol like Santa Claus, but it's not as easy to see how we do the very same thing with every symbol, every story, \& every opinion that we learn about ourselves, about everything.

I think this is very important to understand, \& the only way to understand it is by being aware that this is what we are doing. If we have the awareness that we invest our personal power in everything that we believe, perhaps it will be easy to take our power back from the symbols, \& those symbols will no longer have any power over us. If we take the power out of every symbol, the symbols become just symbols. Then they will obey the creator, which means the \textit{real} us, \& they will serve their \textit{real} purpose: to be a tool that we can use to communicate.

When we find out that Santa Claus isn't the truth, we no longer believe in Santa Claus, \& the power we invested in that symbol returns to us. This is when we become aware that we are the one who agreed to believe in Santa Claus. When we recover our awareness, we can see that we are the one who agreed to believe in the entire symbology. \& if we are the one who put the power of our faith in every symbol, then we are the only one who can take that power back.

If we have this awareness, I think we can recover the power over everything that we believe \& never lose control over our own creation. Once we can see that we are the one who creates the structure of our beliefs, this helps us to recover faith in ourselves. When we have faith in ourselves instead of the belief system, we have no doubt where that power comes from, \& we start to dismantle the structure.

Once the structure of our belief system is no longer there, we become very flexible. We can create anything we want to create; we can do anything we want to do. We can invest our faith in anything we want to believe. It's our choice. If we no longer believe in all that we know that makes us suffer, then just like magic, our suffering disappears. \& we don't need a lot of thinking; we need action. It's action that is going to make the difference.'' -- \cite[pp. 60--64]{Ruiz_Ruiz2011}

%------------------------------------------------------------------------------%

\section{Practice makes The Master. The 4th Agreement: Always Do Your Best}

%------------------------------------------------------------------------------%

\begin{center}\Large\bf
	Part II: The Power of Doubt
\end{center}

\section{The Power of Doubt. The 5th Agreement: Be Skeptical, but Learn to Listen}

%------------------------------------------------------------------------------%

\section{The Dream of The 1st Attention. The Victims}

%------------------------------------------------------------------------------%

\section{The Dream of the 2nd Attention. The Warriors}

%------------------------------------------------------------------------------%

\section{The Dream of the 3rd Attention. The Masters}

%------------------------------------------------------------------------------%

\section{Becoming a Seer. A New Point of View}

%------------------------------------------------------------------------------%

\section{The 3 Languages. What Kind of Messenger Are You?}

%------------------------------------------------------------------------------%

\section{Epilogue: Help Me to Change the World}

%------------------------------------------------------------------------------%

\printbibliography[heading=bibintoc]
	
\end{document}