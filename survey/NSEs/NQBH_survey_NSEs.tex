\documentclass{article}
\usepackage[backend=biber,natbib=true,style=authoryear]{biblatex}
\addbibresource{/home/hong/1_NQBH/reference/bib.bib}
\usepackage[utf8]{inputenc}
\usepackage{tocloft}
\renewcommand{\cftsecleader}{\cftdotfill{\cftdotsep}}
\usepackage[colorlinks=true,linkcolor=blue,urlcolor=red,citecolor=magenta]{hyperref}
\usepackage{amsmath,amssymb,amsthm,mathtools,float,graphicx,algpseudocode,algorithm}
\allowdisplaybreaks
\numberwithin{equation}{section}
\newtheorem{assumption}{Assumption}[section]
\newtheorem{lemma}{Lemma}[section]
\newtheorem{corollary}{Corollary}[section]
\newtheorem{definition}{Definition}[section]
\newtheorem{proposition}{Proposition}[section]
\newtheorem{theorem}{Theorem}[section]
\newtheorem{notation}{Notation}[section]
\newtheorem{remark}{Remark}[section]
\newtheorem{example}{Example}[section]
\newtheorem{ques}{Question}[section]
\newtheorem{problem}{Problem}[section]
\newtheorem{conjecture}{Conjecture}[section]
\usepackage[left=0.5in,right=0.5in,top=1.5cm,bottom=1.5cm]{geometry}
\usepackage{fancyhdr}
\pagestyle{fancy}
\fancyhf{}
\lhead{\small \textsc{Sect.} ~\thesection}
\rhead{\small \nouppercase{\leftmark}}
\renewcommand{\sectionmark}[1]{\markboth{#1}{}}
\cfoot{\thepage}
\def\labelitemii{$\circ$}

\title{A Survey on Navier--Stokes Equations}
\author{Collector: Nguyen Quan Ba Hong\footnote{Independent Researcher, Ben Tre City, Vietnam.\\\textit{Email.} \texttt{nguyenquanbahong@gmail.com}.}}
\date{\today}

\begin{document}
\maketitle

\begin{abstract}
	A personal survey on Navier--Stokes equations (NSEs), especially its regularity and turbulence models.
\end{abstract}
\textbf{Keywords.} Navier--Stokes equations; turbulence models.

\tableofcontents

%------------------------------------------------------------------------------%
\vspace{5mm}
\textbf{Disclaimer.} This survey is used mainly for NQBH's personal purposes. Hence, several paragraphs are quoted without quotation marks. The main purpose here is to collect and to add personal remarks.
\section*{Quick notes}
\begin{enumerate}
	\item The 4 formulations appearing in the \textit{Clay Millennium Prize} formulation \cite{Fefferman2006} of NSEs.
	\item ``The regularity theory for subcritical PDEs is often standard. That for critical PDEs is quite tricky but sometimes possible. The regularity theory for supercritical equations is usually very difficult, and known examples often require other properties such as a monotonicity formula.'' -- \cite[p. 7]{Tsai2018}
\end{enumerate}


\section{Incompressible NSEs}
\begin{quotation}
	``The Navier--Stokes equations are among the very few equations of mathematical physics for which the nonlinearity arises not from the physical attributes of the system but rather from the mathematical (kinematical) aspects of the problem.'' -- \cite[p. 2]{Foias_Manley_Rosa_Temam2001}
	
	``The no-slip boundary condition (flow past a rigid boundary) is one of the few that correspond to a physically accessible boundary condition. Another physically accessible boundary condition is the open boundary (i.e., an open surface of a flowing fluid for part or all of the boundary)\footnote{In the language of PDEs, this corresponds to a Neumann-type boundary condition; the no-slip case corresponds to a Dirichlet-type boundary condition.}'' -- \cite[p. 45]{Foias_Manley_Rosa_Temam2001}
\end{quotation}

\subsection{Various concepts of solutions to NSEs}
To describe various formulations for NSEs, we must first define properly the concept of a solution to NSEs, including, e.g., \textit{periodic solutions, finite energy solutions, $H^1$ solutions}, and \textit{smooth solutions}, etc.

\subsubsection{Smooth solutions of NSEs}
\begin{quotation}
	``Note that even within the category of smooth solutions, there is some choice in what decay hypotheses to place on the initial data and solution; for instance, one can require that the initial velocity ${\bf u}_0$ be Schwartz class, or merely smooth with finite energy. Intermediate between these two will be data which is smooth and in $H^1$.'' -- \cite{Tao2013}
\end{quotation}
Recall \cite[Def. 1.1]{Tao2013}:
\begin{definition}[Smooth solutions to NSEs]
	A \emph{smooth set of data} for NSEs up to time $T$ is a triplet $({\bf u}_0,{\bf f},T)$, where $0 < T < \infty$ is a time, the initial velocity vector field ${\bf u}_0:\mathbb{R}^3\to\mathbb{R}^3$ and the forcing term ${\bf f}:[0,T]\times\mathbb{R}^3\to\mathbb{R}^3$ are assumed to be smooth on $\mathbb{R}^3$ and $[0,T]\times\mathbb{R}^3$, respectively, (thus, ${\bf u}_0$ is infinitely differentiable in space, and ${\bf f}$ is infinitely differentiable in spacetime), and ${\bf u}_0$ is furthermore required to be divergence-free:
	\begin{align}
		\label{divergence-free initial velocity}
		\nabla\cdot{\bf u}_0 = 0,\mbox{ in }\mathbb{R}^3.
	\end{align}
	If ${\bf f} = {\bf 0}$, we say that the data is \emph{homogeneous}.
	
	The \emph{total energy} $E({\bf u}_0,{\bf f},T)$ of a smooth set of data $({\bf u}_0,{\bf f},T)$ is defined by the quantity
	\begin{align}
		\label{total energy}
		\tag{iiNS\texttt{/}$E$}
		E({\bf u}_0,{\bf f},T)\coloneqq\frac{1}{2}\left(\|{\bf u}_0\|_{L_{\bf x}^2(\mathbb{R}^3)} + \|{\bf f}\|_{L_t^1L_{\bf x}^2([0,T]\times\mathbb{R}^3)}\right)^2,
	\end{align}
	and $({\bf u}_0,{\bf f},T)$ is said to have \emph{finite energy} if $E({\bf u}_0,{\bf f},T) < \infty$. We define the \emph{$H^1$ norm} $\mathcal{H}^1({\bf u}_0,{\bf f},T)$ of the data to be the quantity
	\begin{align}
		\label{H^1 norm of data}
		\tag{iiNS\texttt{/}$\mathcal{H}^1$}
		\mathcal{H}^1({\bf u}_0,{\bf f},T)\coloneqq\|{\bf u}_0\|_{H_x^1(\mathbb{R}^3)} + \|{\bf f}\|_{L_t^\infty H_x^1(\mathbb{R}^3)} < \infty,
	\end{align}
	and say that \emph{$({\bf u}_0,{\bf f},T)$ is $H^1$} if $\mathcal{H}^1({\bf u}_0,{\bf f},T) < \infty$; note that the $H^1$ regularity is essentially 1 derivative higher than the energy regularity, which is at the level of $L^2$, and instead matches the regularity of the \emph{initial enstrophy} $\frac{1}{2}\int_{\mathbb{R}^3} \|\boldsymbol{\omega}_0(t,{\bf x})\|^2\,{\rm d}{\bf x}$, where $\omega_0\coloneqq\nabla\times{\bf u}_0$ is the \emph{initial vorticity}. We say that a smooth set of data $({\bf u}_0,{\bf f},T)$ is \emph{Schwartz} if, for all integers $\alpha,m,k\ge 0$, one has
	\begin{align}
		\sup_{{\bf x}\in\mathbb{R}^3} (1 + \|{\bf x}\|)^k\|\nabla_{\bf x}^\alpha{\bf u}_0({\bf x})\| < \infty\mbox{ and }\sup_{(t,{\bf x})\in[0,T]\times\mathbb{R}^3} (1 + \|{\bf x}\|)^k\|\nabla_{\bf x}^\alpha\partial_t^m{\bf f}({\bf x})\| < \infty.
	\end{align}
	Thus, e.g., the Schwartz property implies $H^1$, which in turn implies finite energy. We also say that $({\bf u}_0,{\bf f},T)$ is \emph{periodic} with some period $L > 0$ if one has ${\bf u}_0({\bf x} + L{\bf k}) = {\bf u}_0({\bf x})$ and ${\bf f}(t,{\bf x} + L{\bf k}) = {\bf f}(t,{\bf x})$ for all $t\in[0,T]$, ${\bf x}\in\mathbb{R}^3$, and ${\bf k}\in\mathbb{Z}^3$. Of course, periodicity is incompatible with the Schwartz, $H^1$, or finite energy properties, unless the data is zero. To emphasize the periodicity, we will sometimes write a periodic set of data $({\bf u}_0,{\bf f},T)$ as $({\bf u}_0,{\bf f},T,L)$.
	
	A \emph{smooth solution to the NSEs}, or a \emph{smooth solution}, is a quintuplet $({\bf u},p,{\bf u}_0,{\bf f},T)$, where $({\bf u}_0,{\bf f},T)$ is a smooth set of data, and the velocity vector field ${\bf u}:[0,T]\times\mathbb{R}^3\to\mathbb{R}^3$ and pressure field $p:[0,T]\times\mathbb{R}^3\to\mathbb{R}$ are smooth functions on $[0,T]\times\mathbb{R}^3$ that obey the NSE:\footnote{NQBH: Why no viscosity $\nu$? Any major differences in their mathematical analysis, especially the case $\nu = \nu(t,{\bf x},{\bf u},p)$ in turbulence models?}
	\begin{align}
		\partial_t{\bf u} + ({\bf u}\cdot\nabla){\bf u} = \Delta{\bf u} - \nabla p + {\bf f},
	\end{align}
	and the incompressibility property
	\begin{align}
		\nabla\cdot{\bf u} = 0,
	\end{align}
	on all of $[0,T]\times\mathbb{R}^3$,\footnote{NQBH: NSEs on the whole domain, hence useless for shape and topology optimizations, but useful for applying harmonic and Fourier analysis.} and also the initial condition
	\begin{align}
		{\bf u}(0,{\bf x}) = {\bf u}_0({\bf x}),\ \forall{\bf x}\in\mathbb{R}^3.
	\end{align}
	We say that a smooth solution $({\bf u},p,{\bf u}_0,{\bf f},T)$ has \emph{finite energy} if the associated data $({\bf u}_0,{\bf f},t)$ has finite energy, and in addition one has\footnote{Following \cite{Fefferman2006}, Terence Tao omitted the \textit{finite energy dissipation condition} $\nabla{\bf u}\in L_t^2L_{\bf x}^2([0,T]\times\mathbb{R}^3)$ often appearing in the literature, particular when discussing \textit{Leray--Hopf weak solutions}. However, it turns out that this condition is actually automatic from \eqref{Tao2013 Eqn. (6)} and smoothness; see \cite[Lem. 8.1]{Tao2013}. Similarly, from \cite[Corollary 11.1]{Tao2013}, the $L_t^2H_{\bf x}^2$ condition in \eqref{Tao2013 Eqn. (7)} is redundant.}
	\begin{align}
		\label{Tao2013 Eqn. (6)}
		\|{\bf u}\|_{L_t^\infty L_{\bf x}^2([0,T]\times\mathbb{R}^3)} < \infty.
	\end{align}
	Similarly, we say that \emph{$({\bf u},p,{\bf u}_0,{\bf f},T)$ is $H^1$} if the associated data $({\bf u}_0,{\bf f},T)$ is $H^1$, and in addition one has
	\begin{align}
		\label{Tao2013 Eqn. (7)}
		\|{\bf u}\|_{L_t^\infty H_{\bf x}^1([0,T]\times\mathbb{R}^3)} + \|{\bf u}\|_{L_t^2H_{\bf x}^2([0,T]\times\mathbb{R}^3)} < \infty.
	\end{align}
	We say instead that a smooth solution $({\bf u},p,{\bf u}_0,{\bf f},T)$ is \emph{periodic} with period $L > 0$ if the associated data $({\bf u}_0,{\bf f},T) = ({\bf u}_0,{\bf f},T,L)$ is periodic with period $L$, and if ${\bf u}(t,{\bf x} + L{\bf k}) = {\bf u}(t,{\bf x})$ for all $t\in[0,T]$, ${\bf x}\in\mathbb{R}^3$, and ${\bf k}\in\mathbb{Z}^3$. (Following \cite{Fefferman2006}, however, we will not initially directly require any periodicity properties on the pressure.) As before, we will sometimes write a periodic solution $({\bf u},p,{\bf u}_0,{\bf f},T)$ as $({\bf u},p,{\bf u}_0,{\bf f},T,L)$ to emphasize the periodicity.
\end{definition}
Terence Tao [but I will never] sometimes abuse notation and refer to a solution $({\bf u},p,{\bf u}_0,{\bf f},T)$ simply as $({\bf u},p)$ or even ${\bf u}$. Similarly, we will sometimes abbreviate a set of data $({\bf u}_0,{\bf f},T)$ as $({\bf u}_0,{\bf f})$ or even ${\bf u}_0$ (in the homogeneous case ${\bf f} = {\bf 0}$).

\begin{remark}
	In \cite{Fefferman2006}, one considered\footnote{The viscosity parameter $\nu$ was not normalized in \cite{Fefferman2006} to 1, as Terence Tao did in \cite{Tao2013}, but one can easily reduce to the $\nu = 1$ case by a simple rescaling.
	
	NQBH: only true for the case $\nu = {\rm const} > 0$, rescaling fails when $\nu$ varies, e.g., depends on ${\bf x}$ and\texttt{/}or ${\bf u}$, $p$.} smooth finite energy solutions associated to Schwartz data, as well as periodic smooth solutions associated to periodic smooth data. In the latter case, one can of course normalize the period $L$ to equal 1 by a simple scaling argument. In \cite{Tao2013}, Terence Tao focused on the case when the data $({\bf u}_0,{\bf f},T)$ is large, although he did not study the asymptotic regime when $T\to\infty$.
\end{remark}
Recall 2 standard \textit{global regularity} conjectures for NSEs, using the formulation in \cite{Fefferman2006}:

\begin{conjecture}[Global regularity for homogeneous Schwartz data]
	Let $({\bf u}_0,0,T)$ be a homogeneous Schwartz set of data. Then there exists a smooth finite energy solution $({\bf u},p,{\bf u}_0,0,T)$ with the indicated data.
\end{conjecture}

\begin{conjecture}[Global regularity for homogeneous periodic data]
	Let $({\bf u}_0,0,T)$ be a smooth homogeneous periodic set of data. Then there exists a smooth periodic solution $({\bf u},p,{\bf u}_0,0,T)$ with the indicated data.
\end{conjecture}
Extend these conjectures to the inhomogeneous case as follows:

\begin{conjecture}[Global regularity for Schwartz data]
	Let $({\bf u}_0,{\bf f},T)$ be a Schwartz set of data. Then there exists a smooth finite energy solution $({\bf u},p,{\bf u}_0,{\bf f},T)$ with the indicated data.
\end{conjecture}

\begin{conjecture}[Global regularity for periodic data]
	Let $({\bf u}_0,{\bf f},T)$ be a smooth periodic set of data. Then there exists a smooth periodic solution $({\bf u},p,{\bf u}_0,{\bf f},T)$ with the indicated data.
\end{conjecture}
As described in \cite{Fefferman2006}, a positive answer to either Conjecture 1.3 or 1.4, or a negative answer to Conjecture 1.5 or 1.6, would qualify for the Clay Millennium Prize.

However, Conjecture 1.6 is not quite the ``right'' extension of Conjecture 1.4 to the inhomogeneous setting, and needs correcting slightly. This is because there is a technical quirk in the inhomogeneous periodic problem as formulated in Conjecture 1.6, due to the fact that $p$ is not required to be periodic. This opens up a \textit{Galilean invariance} in the problem which allows one to homogenize away the role of the forcing term. More precisely (established in \cite[Sect. 6]{Tao2013}):

\begin{proposition}[Elimination of forcing term]
	Conjecture 1.6 $\Leftrightarrow$ Conjecture 1.4.
\end{proposition}
Terence Tao remarks that this is the only implication we know of that can deduce a global regularity result for the inhomogenous Navier--Stokes problem from a global regularity result for the homogeneous Navier--Stokes problem.

\cite[Prop. 1.7]{Tao2013} exploits the technical loophole of nonperiodic pressure. The same loophole can also be used to easily demonstrate failure of uniqueness for the periodic Navier--Stokes problem (although this can also be done by the much simpler expedient of noting that one can adjust the pressure by an arbitrary constant without affecting the momentum equation). This suggests that in the nonhomogeneous case ${\bf f}\ne{\bf 0}$, one needs an additional normalization to ``fix'' the periodic Navier--Stokes problem to avoid such loopholes. This can be done in the standard way, as follows: see \cite[p. 28]{Tao2013}$\ldots$

\subsection{OpenFOAM Solvers}

\subsubsection{OpenFOAM Solver \texttt{simpleFoam}}
The \texttt{simpleFoam} solver employs the SIMPLE algorithm to solve:
\begin{equation}
	\left\{\begin{split}
		\nabla\cdot({\bf u}\otimes{\bf u}) - \nabla\cdot{\bf R} &= -\nabla p + {\bf S}_{\bf u},&&\mbox{ in }\Omega,\\
		\nabla\cdot{\bf u} &= 0,&&\mbox{ in }\Omega,
	\end{split}\right.
\end{equation}
where ${\bf R}$ is the stress tensor, and ${\bf S}_{\bf u}$ is the momentum source.

\section{Compressible NSEs}

\subsection{OpenFOAM Solvers}

\section{\href{https://www.openfoam.com/documentation/guides/latest/doc/guide-turbulence.html}{OpenFOAM\texttt{/}Physical Modeling\texttt{/}Turbulence}}
OpenFOAM includes support for the following types of turbulence modeling: \href{https://www.openfoam.com/documentation/guides/latest/doc/guide-turbulence-ras.html}{Reynolds Averaged Simulation (RAS)}, \href{https://www.openfoam.com/documentation/guides/latest/doc/guide-turbulence-des.html}{Detached Eddy Simulation (DES)}, and \href{https://www.openfoam.com/documentation/guides/latest/doc/guide-turbulence-les.html}{Large Eddy Simulation (LES)}.

\paragraph{Usage.} Turbulence models are specified in the \verb|$FOAM_CASE/constant/turbulenceProperties| file, taking the form, e.g., when specifying the \href{https://www.openfoam.com/documentation/guides/latest/doc/guide-turbulence-ras.html}{RAS} \href{https://www.openfoam.com/documentation/guides/latest/doc/guide-turbulence-ras-k-omega-sst.html}{kOmegaSST} model:
\begin{verbatim}
simulationType      RAS;

RAS
{
    RASModel                kOmegaSST;

    // On/off switch
    turbulence              on

    // Optionally write the model coefficients at run-time
    printCoeffs             no;

    // Optionally override default model coefficients
    kOmegaSSTCoeffs
    {
        ...
    }
}
\end{verbatim}
Default coefficients are available for all models, based on their reference literature. Optionally users may override the default values by specifying a \verb|<model>Coeffs| sub-dictionary. Coefficient names can be found by observing the solver output when setting the \texttt{printCoeffs} to \texttt{yes}.

\paragraph{Mesh requirements.} Effective use of turbulence models requires close attention to their respective meshing requirements, particularly in near-wall regions. \textsf{Fig. Near wall velocity profile.}
\begin{itemize}
	\item DNS data from \cite{Lee_Moser2015}.
	\item RAS:
	\begin{itemize}
		\item high  Reynolds number: 1st cell height should be in the region of $30 < y^+ < 200$. Note that the upper limit is imposed by the location of the outer layer, which depends on the Reynolds number
		\item low Reynolds number: mesh required to resolve the viscous sub-layer, typically using 10-20 layers
	\end{itemize}
	\item LES:
	\begin{itemize}
		\item mesh required to resolve the viscous sub-layer
		\item requires high order schemes to adequately resolve the high-energy containing eddies
		\item preferably isotropic mesh		
	\end{itemize}
\end{itemize}

\paragraph{Numerical settings.} Turbulence generation is driven by the $\nabla{\bf u}$. Errors arising from the \href{https://www.openfoam.com/documentation/guides/latest/doc/guide-schemes-gradient.html}{gradient calculation}, e.g., due to poor quality meshes, can lead to spurious turbulence predictions and solver instability. This effect can be partly compensated by the application of \href{https://www.openfoam.com/documentation/guides/latest/doc/guide-schemes-gradient.html#sec-schemes-gradient-options-gradient-limiters}{limited schemes}.

\subsection{\href{https://www.openfoam.com/documentation/guides/latest/doc/guide-turbulence-ras.html}{Reynolds Averaged Simulation (RAS)}}

\paragraph{Options.} OpenFOAM includes Reynolds Averaged Simulation turbulence closures based on linear and non-linear eddy viscosity models, and Reynolds stress transport models: \href{https://www.openfoam.com/documentation/guides/latest/doc/guide-turbulence-ras-linear-eddy-viscosity-models.html}{Linear eddy viscosity models}, \href{https://www.openfoam.com/documentation/guides/latest/doc/guide-turbulence-ras-non-linear-eddy-viscosity-models.html}{Nonlinear eddy viscosity models}, and \href{https://www.openfoam.com/documentation/guides/latest/doc/guide-turbulence-ras-reynolds-stress-transport-models.html}{Reynolds stress transport models}.

\paragraph{Usage.} RAS is selected by setting the \texttt{simulationType} entry 
\begin{verbatim}
	simulationType  RAS;
	
	RAS
	{
        \\ Model input parameters
	}
\end{verbatim}
Suitable for:
\begin{itemize}
	\item 1D, 2D, and 3D cases
	\item steady-state or transient
\end{itemize}

\paragraph{Initialization.} Care should be taken to provide an adequate initialization of turbulence fields. For open systems, the flow can evolve via the inlets. However, note that the initial field values should be approximate for the flow medium, where non-physical levels will result in non-physical turbulence viscosity predictions and case instability.

\paragraph{Background.} Reynolds decomposition of the velocity into its mean and fluctuating contributions takes the form ${\bf u}(t,{\bf x}) = \overline{\bf u}(t,{\bf x}) + {\bf u}'(t,{\bf x})$ where the mean of the fluctuating component is defined as zero: $\overline{{\bf u}'} = 0$. Applied to NSEs, this leads to equations for the mean velocity and pressure:
\begin{equation*}
	\left\{\begin{split}
		\partial_t\rho + \nabla\cdot(\rho\overline{\bf u}) &= 0,\\
		\partial_t(\rho\overline{\bf u}) + \nabla\cdot(\rho\overline{\bf u}\otimes\overline{\bf u}) &= {\bf g} + \nabla\cdot\overline{\boldsymbol{\tau}} - \nabla\cdot(\rho{\bf R}),
	\end{split}\right.
\end{equation*}
where the \textit{averaged stress tensor}, $\overline{\boldsymbol{\tau}}$, for Newtonian fluids is given by:
\begin{align*}
	\overline{\boldsymbol{\tau}}\coloneqq-\left(p + \frac{2}{3}\mu\nabla\cdot\overline{\bf u}\right){\bf I} + \mu\left(\nabla\overline{\bf u} + (\nabla\overline{\bf u})^\top\right).
\end{align*}
Using the relationship: $\nabla\cdot\overline{\bf u} = \operatorname{tr}(\nabla\overline{\bf u}) = \operatorname{tr}\left((\nabla\overline{\bf u})^\top\right)$ the stress tensor becomes:
\begin{align*}
	\overline{\boldsymbol{\tau}} = -p{\bf I} + \mu\left[\nabla\overline{\bf u} + (\nabla\overline{\bf u})^\top - \frac{2}{3}\operatorname{tr}\left((\nabla\overline{\bf u})^\top\right){\bf I}\right] = -p{\bf I} + \mu\left[\nabla\overline{\bf u} + \operatorname{dev2}\left((\nabla\overline{\bf u})^\top\right)\right],
\end{align*}
and the $\operatorname{dev2}$ operator is defined as:
\begin{align*}
	\operatorname{dev2}(\phi)\coloneqq\phi - \frac{2}{3}\operatorname{tr}(\phi){\bf I}.
\end{align*}
\texttt{continue to read RAS}

\subsection{Detached Eddy Simulation (DES)}

\subsection{Large Eddy Simulation (LES)}

%------------------------------------------------------------------------------%

\begin{thebibliography}{99}
	\bibitem[TerryTao]{TerryTao} \href{https://terrytao.wordpress.com/tag/navier-stokes-equations/}{Terence Tao's blog\texttt{/}Navier--Stokes equations}.
	\begin{itemize}
		\item Terence Tao. \href{https://terrytao.wordpress.com/2011/08/04/localisation-and-compactness-properties-of-the-navier-stokes-global-regularity-problem}{\textit{Localisation and compactness properties of the Navier-Stokes global regularity problem}}. Aug 4, 2011.
		
		\item Terence Tao. \href{https://terrytao.wordpress.com/2018/09/03/254a-notes-0-physical-derivation-of-the-incompressible-euler-and-navier-stokes-equations}{\textit{254A, Notes 0: Physical derivation of the incompressible Euler and Navier-Stokes equations}}. Sep 3, 2018.
		
		\item Terence Tao. \href{https://terrytao.wordpress.com/2019/08/15/quantitative-bounds-for-critically-bounded-solutions-to-the-navier-stokes-equations/}{\textit{Quantitative bounds for critically bounded solutions to the Navier-Stokes equation}s}. Aug 15, 2019.
	\end{itemize}
	
	\bibitem[Wikipedia]{Wikipedia} \href{https://en.wikipedia.org}{Wikipedia.org}
	\begin{itemize}
		\item \href{https://en.wikipedia.org/wiki/Enstrophy}{Wikipedia\texttt{/}enstrophy}.
		\item \href{https://en.wikipedia.org/wiki/Viscosity}{Wikipedia\texttt{/}viscosity}.
	\end{itemize}
\end{thebibliography}

\printbibliography[heading=bibintoc]
	
\end{document}