\documentclass{article}
\usepackage[backend=biber,natbib=true,style=authoryear]{biblatex}
\addbibresource{/home/hong/1_NQBH/reference/bib.bib}
\usepackage{tocloft}
\renewcommand{\cftsecleader}{\cftdotfill{\cftdotsep}}
\usepackage[colorlinks=true,linkcolor=blue,urlcolor=red,citecolor=magenta]{hyperref}
\usepackage{algorithm,algpseudocode,amsmath,amssymb,amsthm,float,graphicx,mathtools}
\allowdisplaybreaks
\numberwithin{equation}{section}
\newtheorem{assumption}{Assumption}[section]
\newtheorem{conjecture}{Conjecture}[section]
\newtheorem{corollary}{Corollary}[section]
\newtheorem{definition}{Definition}[section]
\newtheorem{example}{Example}[section]
\newtheorem{lemma}{Lemma}[section]
\newtheorem{notation}{Notation}[section]
\newtheorem{principle}{Principle}[section]
\newtheorem{problem}{Problem}[section]
\newtheorem{proposition}{Proposition}[section]
\newtheorem{question}{Question}[section]
\newtheorem{remark}{Remark}[section]
\newtheorem{theorem}{Theorem}[section]
\usepackage[left=0.5in,right=0.5in,top=1.5cm,bottom=1.5cm]{geometry}
\usepackage{fancyhdr}
\pagestyle{fancy}
\fancyhf{}
\lhead{\small \textsc{Sect.} ~\thesection}
\rhead{\small \nouppercase{\leftmark}}
\renewcommand{\sectionmark}[1]{\markboth{#1}{}}
\cfoot{\thepage}
\def\labelitemii{$\circ$}

\title{Deep Learning}
\author{Collector: Nguyen Quan Ba Hong\footnote{Independent Researcher, Ben Tre City, Vietnam\\e-mail: \texttt{nguyenquanbahong@gmail.com}}}
\date{\today}

\begin{document}
\maketitle
\begin{abstract}
	A personal survey on Deep Learning.
\end{abstract}

\textbf{Keywords.} Deep learning.
\tableofcontents

%------------------------------------------------------------------------------%

\section{Review: Deep Learning}
``Deep learning allows computational models that are \fbox{composed of multiple processing layers\footnote{\textbf{layer} [n] \textbf{1.} a quantity of something that lies over a surface or between surfaces; \textbf{2.} \textbf{layer (of something)} a level or part within an organization, a society or a set of ideas.}} to learn representations of data with multiple levels of abstraction. These models have dramatically improved the state-of-the-art in speech recognition\footnote{\textbf{recognition} [n] \textbf{1.} [uncountable] the act of remembering who somebody is when you see them, or of identifying what something is; \textbf{2.} [uncountable, singular] the act of accepting that something exists, is true or is official, \textsc{synonym}: \textbf{acknowledgment}; \textbf{3.} [uncountable] public praise \& reward for somebody's work, achievements or actions.}, visual object recognition, object detection \& many other domains e.g. drug discovery \& genomics. Deep learning discovers intricate structure in large data sets by using the backpropagation algorithm to indicate how a machine should change its internal parameters that are used to compute the representation in each layer from the representation in the previous layer. Deep convolutional nets have brought about breakthroughs in processing images, video, speech \& audio, whereas recurrent nets have shone light on sequential data e.g. text \& speech.'' -- \cite[Abstract, p. 436]{LeCun_Bengio_Hinton2015}

\begin{question}
	Why Machine Learning?
\end{question}
``Machine-learning technology powers many aspects of modern society: from web searches to content filtering on social networks to recommendations on e-commerce websites, \& it is increasingly present in consumer products e.g. cameras \& smartphones. Machine-learning systems are used to identify objects in images, transcribe speech into text, match news items, posts or products with users' interests, \& select relevant results of search. Increasingly, these applications make use of a class of techniques called \textit{deep learning}.

\begin{question}
	How should I organize: AI, Machine Learning, Deep Learning (Data Mining also, if possible \& reasonable to fit in)? In a single file or multiple ones?
\end{question}

%------------------------------------------------------------------------------%

\printbibliography[heading=bibintoc]
	
\end{document}