\documentclass{article}
\usepackage[backend=biber,natbib=true,style=alphabetic]{biblatex}
\addbibresource{/home/nqbh/reference/bib.bib}
\usepackage{tocloft}
\renewcommand{\cftsecleader}{\cftdotfill{\cftdotsep}}
\usepackage[colorlinks=true,linkcolor=blue,urlcolor=red,citecolor=magenta]{hyperref}
\usepackage{algorithm,algpseudocode,amsmath,amssymb,amsthm,float,graphicx,mathtools}
\allowdisplaybreaks
\numberwithin{equation}{section}
\newtheorem{assumption}{Assumption}[section]
\newtheorem{conjecture}{Conjecture}[section]
\newtheorem{corollary}{Corollary}[section]
\newtheorem{definition}{Definition}[section]
\newtheorem{example}{Example}[section]
\newtheorem{lemma}{Lemma}[section]
\newtheorem{notation}{Notation}[section]
\newtheorem{principle}{Principle}[section]
\newtheorem{problem}{Problem}[section]
\newtheorem{proposition}{Proposition}[section]
\newtheorem{question}{Question}[section]
\newtheorem{remark}{Remark}[section]
\newtheorem{theorem}{Theorem}[section]
\usepackage[left=1cm,right=1cm,top=5mm,bottom=5mm,footskip=4mm]{geometry}
\def\labelitemii{$\circ$}

\title{On Writing: A Memoir of the Craft}
\author{Stephen King}
\date{\today}

\begin{document}
\maketitle
\tableofcontents
\vspace{5mm}
\begin{quotation}
	\textit{``Honesty's the best policy.''} -- Miguel de Cervantes
	
	\textit{``Liars prosper.''} -- Anonymous
\end{quotation}

%------------------------------------------------------------------------------%

\section*{1st Foreword}
``In the early 90s (it might have been 1992, but it's hard to remember when you're having a good time) I joined a rock-\&-roll band composed mostly of writers. The Rock Bottom Remainders were the brainchild of Kathi Kamen Goldmark, a book publicist \& musician from San Francisco. The group included Dave Barry on lead guitar, Ridley Pearson on bass, Barbara Kingsolver on keyboards, Robert Fulghum on mandolin, \& me on rhythm guitar. There was also a trio of ``chick singer,'' \textit{\`a la} the Dixie cups, made up (usually) of Kathi, Tad, Bartimus, \& Amy Tan.

The group was intended as a 1-shot deal -- we would play 2 shows at the American Booksellers Convention, get a few laughs, recapture our misspent youth for 3 or 4 hours, then go our separate ways.

It didn't happen that way, because the group never quite broke up. We found that we liked playing together too much to quit, \& with a couple of ``ringer'' musicians on sax \& drums (plus, in the early days, our musical guru, Al Kooper, at the heart of the group), we sounded pretty good. You'd pay to hear us. Not a lot, not U2 or E Street Bans prices, but maybe what the oldtimers call ``roadhouse money.'' We took the group on tour, wrote a book about it (my wife took the photos \& danced whenever the spirit took her, which was quite often), \& continue to play now \& then, sometimes as The Remainders, sometimes as Raymond Burr's Legs. The personnel comes \& goes -- columnist Mitch Albom has replaced Barbara on keyboards, \& Al doesn't play with the group anymore `cause he \& Kathi don't get along -- but the core has remained Kathi, Amy, Ridley, Dave, Mitch Albom, \& me $\ldots$ plus Josh Kelly on drums \& Erasmo Paolo on sax.

We do it for the music, but we also do it for the companionship. We like each other, \& we like having a chance to talk sometimes about the real job, the day job people are always telling us not to quit. We are writers, \& we never ask one another where we get our ideas; we know we don't know.

1 night while we were eating Chinese before a big in Miami Beach, I asked Amy if there was any 1 question she was \textit{never} asked during the Q-\&-A that follows almost every writer's talk -- that question you never get to answer when you're standing in front of a group of author-struck fans \& pretending you don't put your pants on 1 leg at a time like everyone else. Amy paused, thinking it over very carefully, \& then said: ``No one ever asks about the language.''

I owe an immense debt of gratitude to her for saying that. I had been playing with the idea of writing a little book about writing for a year or more at that time, but had held back because I didn't trust my own motivations -- \textit{why} did I want to write about writing? What made me think I had anything worth saying?

The easy answer is that someone who has sold as many books of fiction as I have must have \textit{something} worthwhile to say about writing it, but the easy answer isn't always the truth. Colonel Sanders sold a hell of a lot of fried chicken, but I'm not sure anyone wants to know how he made it. If I was going to be presumptuous enough to tell people how to write, I felt there had to be a better reason than my popular success. Put another way, I didn't want to write a book, even a short one like this, that would leave me feeling like either a literary gasbag or a transcendental asshole. There are enough of those books -- \& those writers -- on the market already, thanks.

But Amy was right: nobody ever asks about the language. They ask the DeLillos \& the Updikes \& the Styrons, but they don't ask popular novelists. Yet many of us proles also care about the language, in our humble way, \& care passionately about the art \& craft of telling stories on paper. What follows is an attempt to put down, briefly \& simply, how I came to the craft, what I know about it now, \& how it's done. It's about the day job; it's about the language.

This book is dedicated to Amy Tan, who told me in a very simple \& direct way that it was okay to write it.'' -- \cite[p. 9]{King2010}

%------------------------------------------------------------------------------%

\section*{2nd Foreword}
``This is a short book because most books about writing are filled with bullshit. Fiction writers, present company included, don't understand very much about what they do -- not why it works when it's good, not why it doesn't when it's bad. I figured the shorter the book, the less the bullshit.

1 notable exception to the bullshit rule is \textit{The Elements of Style}, by William Strunk Jr. \& E. B. White. There is little or no detectable bullshit in that book. (Of course it's short; at 85 pages it's much shorter than this one.) I'll tell you right now that every aspiring writer should read \textit{The Elements of Style}. Rule 17 in the chapter titled Principles of Composition is ``Omit needless words.'' I will try to do that here.'' -- \cite[p. 10]{King2010}

%------------------------------------------------------------------------------%

\section*{3rd Foreword}
``1 rule of the road not directly stated elsewhere in this book: ``The editor is always right.'' The corollary is that no writer will take all of his or her editor's advice; for all have sinned \& fallen short of editorial perfection. Put another way, to write is human, to edit is divine. Chuck Verrill edited this book, as he has so many of my novels. And as usual, Chuck, you were divine. -- Steve'' -- \cite[p. 11]{King2010}

%------------------------------------------------------------------------------%

\section{C.V.}
``I was stunned by Mary Karr's memoir, \textit{The Liar's Club}. Not just by its ferocity, its beauty, \& by her delightful grasp of the vernacular, but by its \textit{totality} -- she is a woman who remembers \textit{everything} about her early years.

I'm not that way. I lived an odd, herky-jerky childhood, raised by a single parent who moved around a lot in my earliest years \& who -- I am not completely sure of this -- may have farmed my brother \& me out to 1 of her sisters for awhile because she was economically or emotionally unable to cope with us for a time. Perhaps she was only chasing our father, who piled up all sorts of bills \& then did a runout when I was 2 \& my brother David was 4. If so, she never succeeded in finding him. My mom, Nellie Ruth Pillsbury King, was 1 of America's early liberated women, but not by choice.

Mary Karr presents her childhood in an almost unbroken panorama. Mine is a fogged-out landscape from which occasional memories appear like isolated trees $\ldots$ the kind that look as if they might like to grab \& eat you.

What follows are some of those memories, plus assorted snapshots from the somewhat more coherent days of my adolescence \& young manhood. This is not an autobiography. It is, rather, a kind of \textit{curriculum vitae} -- my attempt to show ow 1 writer was formed. Not how 1 writer was \textit{made}; I don't believe writers \textit{can} be made, either by circumstances or by self-will (although I did believe those things once). The equipment comes with the original package. Yet it is by no means unusual equipment; I believe large numbers of people have at least some talent as writers \& storytellers, \& that those talents can be strengthened \& sharpened. If I didn't believe that, writing a book like this would be a waste of time.

This is how it was for me, that's all -- a disjointed growth process in which ambition, desire, luck, \& a little talent all played a part. Don't bother trying to read between the lines, \& don't look for a through-line. There are \textit{no} lines -- only snapshots, most out of focus.

\fbox{\bf1} My earliest memory is of imaging I was someone else -- imagining that I was, in fact, the Ringing Brothers Circus Strongboy. This was at my Aunt Ethelyn \& Uncle Oren's house in Durham, Maine. My aunt remembers this quite clearly, \& says I was 2 \& a half or maybe 3 years old.

I had found a cement cinderblock in a corner of the garage \& had managed to pick it up. I carried it slowly across the garage's smooth cement floor, except in my mind I was dressed in an animal skin singlet (probably a leopard skin) \& carrying the cinderblock across the center ring. The vast crowd was silent. A brilliant bluewhite spotlight marked my remarkable progress. Their wondering faces told the story: never had they seen such an incredibly strong kid. ``\& he's only \textit{2}!'' someone muttered in disbelief.

Unknown to me, wasps had constructed a small nest in the lower half of the cinderblock. 1 of them, perhaps pissed off at being relocated, flew out \& stung me on the ear. The pain was brilliant, like a poisonous inspiration. It was the worst pain I had ever suffered in my short life, but it only held the top spot for a few seconds. When I dropped the cinderblock on 1 bare foot, mashing all 5 toes, I forgot all about the wasp. I can't remember if I was taken to the doctor, \& neither can my Aunt Ethelyn (Uncle Oren, to whom the Evil Cinderblock surely belonged, is almost 20 years dead), but she remembers the sting, the mashed toes, \& my reaction. ``How you howled, Stephen!'' she said. ``You were certainly in fine voice that day.''

\fbox{\bf2} A year or so later, my mother, my brother, \& I were in West De Pere, Wisconsin. I don't know why. Another of my mother's sisters, Cal (a WAAC beauty queen during World War II), lived in Wisconsin with her convivial beer-drinking husband, \& maybe Mom had moved to be near them. If so, I don't remember seeing much of the Weimers. \textit{Any} of them, actually. My mother was working, but I can't remember what her job was, either. I want to say it was a bakery she worked in, but I think that came later, when we moved to Connecticut to live near her sister Lois \& \textit{her} husband (no beer for Fred, \& not much in the way of conviviality, either; he was a crewcut daddy who was proud of driving his convertible with the top \textit{up}, God knows why).

There was a stream of babysitters during our Wisconsin period. I don't know if they left because David \& I were a handful, or because they found better-paying jobs, or because my mother insisted on higher standards than they were willing to rise to; all I know is that there were a lot of them. The only one I remember with any clarity is Eula, or maybe she was Beulah. She was a teenager, she was as big as a house, \& she laughed a lot. Eula-Beulah had a wonderful sense of humor, even at 4 I could recognize that, but it was a \textit{dangerous} sense of humor -- there seemed to be a potential thunderclap hidden inside each hand-patting, butt-rocking, head-tossing outburst of glee. When I see those hidden-camera sequences where real-life babysitters \& nannies just all of a sudden wind up \& clout the kids, it's my days with Eula-Beulah I always think of.

Was she as hard on my brother David as she was on me? I don't know. He's not in any of these pictures. Besides, he would have been less at risk from Hurricane Eula-Beulah's dangerous winds; at 6, he would have been in the 1st grade \& off the gunnery range for most of the day.

Eula-Beulah would be on the phone, laughing with someone, \& beckon me over. She would hug me, tickle me, get me laughing, \& then, still laughing, go upside my head enough to knock me down. Then she would tickle me with her bare feet until we were both laughing again.

Eula-Beulah was prone to farts -- the kind that are both loud \& smelly. Sometimes when she was so afflicted, she would throw me on the couch, drop her wool-skirted butt on my face, \& let loose. ``Pow!'' she'd cry in high glee. It was like being buried in marshgas fireworks. I remember the dark, the sense that I was suffocating, \& I remember laughing. Because, while what was happening was sort of horrible, it was also sort of funny. In many ways, Eula-Beulah prepared me for literary criticism. After having a 200-pound babysitter fart on your face \& yell \textit{Pow!, The Village Voice} holds few terrors.

I don't know what happened to the other sitters, but Eula-Beulah was fired. It was because of the eggs. 1 morning Eula-Beulah fried me an egg for breakfast. I ate it \& asked for another one. Eula-Beulah fried me a 2nd egg, then asked if I wanted another one. \& another one. \& so on. I stopped after 7, I think -- 7 is the number that sticks in my mind, \& quite clearly. Maybe we ran out of eggs. Maybe I cried off. Or maybe Eula-Beulah got scared. I don't know, but probably it was good that the game ended at 7. 7 eggs is quite a few for a 4-year-old.

I felt all right for awhile, \& then I yarked all over the floor. Eula-Beulah laughed, then went upside my head, then shoved me into the closet \& locked the door. Pow. If she'd locked me in the bathroom, she might saved her job, but she didn't. As for me, I didn't really mind being in the closet. It was dark, but it smelled of my mother's Coty perfume, \& there was a comforting line of light under the door.

I crawled to the back of the closet, Mom's coats \& dresses brushing along my back. I began to belch -- long loud belches that burned like fire. I don't remember being sick to my stomach but I must have been, because when I opened my mouth to let out another burning belch, I yarked again instead. All over my mother's shoes. That was the end for Eula-Beulah. When my mother came home from work that day, the babysitter was fast asleep on the couch \& little Stevie was locked in the closet, fast asleep with half-digested fried eggs drying in his hair.

\fbox{\bf3} Our stay in West De Pere was neither long nor successful. We were evicted from our 3rd-floor apartment when a neighbor spotted my 6-year-old brother crawling around on the roof \& called the police. I don't know where my mother was when this happened. I don't know where the babysitter of the week was, either. I only know that I was in the bathroom, standing with my bare feet on the heater, watching to see if my brother would fall off the roof or make it back into the bathroom okay. He made it back. He is now 55 \& living in New Hampshire.

\fbox{\bf4} When I was 5 or 6, I asked my mother if she had ever seen anyone die. Yes, she said, she had been 1 person die \& had heard another one. I asked how you could hear a person die \& she told me that it was a girl who had drowned off Prout's Neck in the 1920s. She said the girl swam out past the rip, couldn't get back in, \& began screaming for help. Several men tried to reach her, but that day's rip had developed a vicious undertow, \& they were all forced back. In the end they could only stand around, tourists \& townies, the teenager who became my mother among them, waiting for a rescue boat that never came \& listening to that girl scream until her strength gave out \& she went under. Her body washed up in New Hampshire, my mother said. I asked how old the girl was. Mom said she was 14, then read me a comic book \& packed me off to bed. On some other day she told me about the one she saw -- a sailor who jumped off the roof of the Graymore Hotel in Portland, Maine, \& landed in the street.

``He splattered,'' my mother said in her most matter-of-fact tone. She paused, \& added, ``The stuff that came out of him was green. I have never forgotten it.''

That makes 2 of us, Mom.

\fbox{\bf5} Most of the 9 months I should have spent in the 1st grade I spent in bed. My problems started with the measles -- a perfectly ordinary case -- \& then got steadily worse. I had bout after bout of what I mistakenly thought was called ``stripe throat''; I lay in bed drinking cold water \& imagining my throat in alternating stripes of red \& white (this was probably not so far wrong).

At some point my ears become involved, \& 1 day my mother called a taxi (she did not drive) \& took me to a doctor too important to make house calls -- an ear specialist. (For some reason I got the idea that this sort of doctor was called an otiologist.) I didn't care whether he specialized in ears or assholes. I had a fever of 104 degrees, \& each time I swallowed, pain lit up the sides of my face like a jukebox.

The doctor looked in my ears, spending most of his time (I think) on the left one. Then he laid me down on his examining table. ``Lift up a minute, Stevie,'' his nurse said, \& put a large absorbent cloth -- it might have been a diaper -- under my head, so that my cheek rested on it when I lay back down. I should have guessed that something was rotten in Denmark. Who knows, maybe I did.

There was a sharp smell of alcohol. A clank as the ear doctor opened his sterilizer. I saw the needle in his hand -- it looked as long as the ruler in my school pencil-box -- \& tensed. The ear doctor smiled reassuringly \& spoke the lie for which doctors should be immediately jailed (time of incarceration to be doubled when the lie is told to a child): ``Relax, Stevie, this won't hurt.'' I believed him.

He slid the needle into my ear \& punctured my eardrum with it. The pain was beyond anything I have ever felt since -- the only thing close was the 1st month of recovery after being struck by a van in the summer 1999. That pain was longer in duration but not so intense. The puncturing of my eardrum was pain beyond the world. I screamed. There was a sound inside my head -- a loud kissing sound. Hot fluid ran out of my ear -- it was as if I had started to cry out of the wrong hole. God knows I was crying enough out of the right ones by then. I raised my streaming face \& looked unbelieving at the ear doctor \& the ear doctor's nurse. Then I looked at the cloth the nurse had spread over the top 3rd of the exam table. It had a big wet patch on it. There were fine tendrils of yellow pus on it as well.

``There,'' the ear doctor said, patting my shoulder. ``You were very brave, Stevie, \& it's all over.''

The next week my mother called another taxi, we went back to the ear doctor's, \& I found myself once more lying on my side with the absorbent square of cloth under my head. The ear doctor once again produced the smell of alcohol -- a smell I still associate, as I suppose many people do, with pain \& sickness \& terror -- \& with it, the long needle. He once more assured me that it wouldn't hurt, \& I once more believed him. Not completely, but enough to be quiet while the needle slid into my ear.

It \textit{did} hurt. Almost as much as the 1st time, in fact. The smoothing sound in my head was louder, too; this time it was giants kissing (``suckin' face \& rotatin' tongues,'' as we used to say). ``There,'' the ear doctor's nurse said when it was over \& I lay there crying in a puddle of watery pus. ``It only hurts a little, \& you don't want to be deaf, do you? Besides, it's all over.''

I believed that for about 5 days, \& then another taxi came. We went back to the ear doctor's. I remember the cab driver telling my mother that he was going to pull over \& let us out if she couldn't shut that kid up.

Once again it was me on the exam table with the diaper under my head \& my mom out in the waiting room with a magazine she was probably incapable of reading (or so I like to imagine), Once again the pungent smell of alcohol \& the doctor turning to me with a needle that looked as long as my school ruler. Once more the smile, the approach, the assurance that \textit{this} time it wouldn't hurt.

Since the repeated eardrum-lancings when I was 6, 1 of my life's firmest principles has been this: Fool me once, shame on you. Fool me twice, shame on me. Fool me 3 times, shame on both of us. The 3rd time on the ear doctor's table I struggled \& screamed \& thrashed \& fought. Each time the needle came near the side of my face, I knocked it away. Finally the nurse called my mother in from the waiting room, \& the 2 of them managed to hold me long enough for the doctor to get his needle in. I screamed so long \& so loud that I can still hear it. In fact, I think that in some deep valley of my head that last scream is still echoing.

\fbox{\bf6} In a dull cold month not too long after that -- it would have been Jan or Feb of 1954, if I've got the sequence right -- the taxi came again. This time the specialist wasn't the ear doctor but a throat doctor. Once again my mother sat in the waiting room, once again I sat on the examining table with a nurse hovering nearby, \& once again there was that sharp smell of alcohol, an aroma that still has the power to double my heartbeat in the space of 5 seconds.

All that appeared this time, however, was some sort of throat swab. It stung, \& it tasted awful, but after the ear doctor's long needle it was a walk in the park. The throat doctor donned an interesting gadget that went around his head on a strap. It had a mirror in the middle, \& a bright fierce light that shone out of it like a 3rd eye. He looked down my gullet for a long time, urging me to open wider until my jaws creaked, but he did not put needles into me \& so I loved him. After awhile he allowed me to close my mouth \& summoned my mother.

``The problem is his tonsils,'' the doctor said. ``They look like a cat clawed them. They'll have to come out.''

At some point after that, I remember being wheeled under bright lights. A man in a white mask bent over me. He was standing at the head of the table I was lying on (1953 \& 1954 were my years for lying on tables), \& to me he looked upside down.

``Stephen,'' he said. ``Can you hear me?''

I said I could.

``I want you to breathe deep,'' he said. ``When you wake up, you can have all the ice cream you want.''

He lowered a gadget over my face. In the eye of my memory, it looks like an outboard motor. I took a deep breath, \& everything went black. When I woke up I was indeed allowed all the ice cream I wanted, which was a fine joke on me because I didn't want any. My throat felt swollen \& fat. But it was better than the old needle-in-the-ear trick. Oh yes. \textit{Anything} would have been better than the old needle-in-the-ear trick. Take my tonsils if you have to, put a steel birdcage on my leg if you must, but God save me from the otiologist.

\fbox{\bf7} That year my brother David jumped ahead to the 4th grade \& I was pulled out of school entirely. I had missed too much of the 1st grade, my mother \& the school agreed; I could start it fresh in the fall of the year, if my health was good.

Most of that year I spent either in bed or housebound. I read my way through approximately 6 tons of comic books, progressed to Tom Swift \& Dave Dawson (a heroic World War II pilot whose various planes were always ``prop-clawing for altitude''), then moved on to Jack London's bloodcurdling animal tales. At some point I began to write my own stories. Imitation preceded creation; I would copy \textit{Combat Casey} comics word for word in my Blue Horse tablet, sometimes adding my own descriptions where they seemed appropriate. ``They were camped in a big dratty farmhouse room,'' I might write; it was another year or 2 before I discovered that \textit{drat} \& \textit{draft} were different words. During that same period I remember believing that \textit{details} were \textit{dentals} \& that a bitch was an extremely tall woman. A son of a bitch was apt to be a basketball player. When you're 6, most of your Bingo balls are still floating around in the draw-tank.

Eventually I showed 1 of these copycat hybrids to my mother, \& she was charmed -- I remember he slightly amazed smile, as if she was unable to believe a kid of hers could be so smart -- practically a damned prodigy, for God's sake. I had never seen that look on her face before -- not on my account, anyway -- \& I absolutely loved it.

She asked me if I had made the story up myself, \& I was forced to admit that I had copied most of it out of a funnybook. She seemed disappointed, \& that drained away much of my pleasure. At last she handed back my tablet. ``Write 1 of your own, Stevie,'' she said. ``Those \textit{Combat Casey} funny-books are just junk -- he's always knocking someone's teeth out. I bet you could do better. Write 1 of your own.''

I remember an immense feeling of \textit{possibility} at the idea, as if I had been ushered into a vast building filled with closed doors \& had been given leave to open any I liked. There were more doors than 1 person could ever open in a lifetime, I thought (\& still think).

I eventually wrote a story about 4 magic animals who rode around in an old car, helping out little kids. Their leader was a large white bunny named Mr. Rabbit Trick. He got to drive the car. The story was 4 pages long, laboriously printed in pencil. No one in it, so far as I can remember, jumped from the roof of the Graymore Hotel. When I finished, I gave it to my mother, who sat down in the living room, put her pocketbook on the floor beside her, \& read it all at once. I could tell she liked it -- she laughed in all the right places -- but I couldn't tell if that was because she liked me \& wanted me to feel good or because it really \textit{was} good.

``You didn't copy this one?'' she asked when she had finished. I said no, I hadn't. She said it was good enough to be in a book. Nothing anyone has said to me since has made me feel any happier. I wrote 4 more stories about Mr. Rabbit Trick \& his friends. She gave me a quarter apiece for them \& sent them around to her 4 sisters, who pitied her a little, I think. \textit{They} were all still married, after all; their men had stuck. It was true that Uncle Fred didn't have much sense of humor \& was stubborn about keeping the top of his convertible up, it was also true that Uncle Oren drank quite a bit \& had dark theories about how the Jews were running the world, but they were \textit{there}. Ruth, on the other hand, had been left holding the baby when Don ran out. She wanted them to see that he was a talented baby, at least.

4 stories. A quarter apiece. That was the 1st buck I made in this business.

\fbox{\bf9} We moved to Stratford, Connecticut. By then I was in the 2nd grade \& stone in love with the pretty teenage girl who lived next door. She never looked twice at me in the daytime, but at night, as I lay in bed \& drifted toward sleep, we ran away from the cruel world of reality again \& again. My new teacher was Mrs. Taylor, a kind lady with gray Elsa Lanchester -- \textit{Bride of Frankenstein} hair \& protruding eyes. ``When we're talking I always want to cup my hands under Mrs. Taylor's peepers in case they fall out,'' my mom said.

Our new 3rd-floor apartment was on West Broad Street. A block down the hill, not far from Teddy's Market \& across from Burrets Building Materials, was a huge tangled wilderness area with a junkyard on the far side \& a train track running through the middle. This is 1 of the places I keep returning to in my imagination; it turns up in my books \& stories again \& again, under a variety of names. The kids in \textit{It} called it the Barrens; we called it the jungle. Dave \& I explored it for the 1st time not long after we had moved into our new place. It was summer. It was hot. It was great. We were deep into the green mysteries of this cool new playground when I was struck by an urgent need to move my bowels.

``Dave,'' I said. ``Take me home! I have to push!'' (This was the word we were given for this particular function.)

David didn't want to hear it. ``Go do it in the woods,'' he said. It would take at least half an hour to walk me home, \& he had no intention of giving up such a shining stretch of time just because his little brother had to take a dump.

``I can't!'' I said, shocked by the idea. ``I won't be able to wipe!''

``Sure you will,'' Dave said. ``Wipe yourself with some leaves. That's how the cowboys \& Indians did it.''

By then it was probably too late to get home, anyway; I have an idea I was out of options. Besides, I was enchanted by the idea of shitting like a cowboy. I pretended I was Hopalong Cassidy, squatting in the underbrush with my gun drawn, not to be caught unawares even at such a personal moment. I did my business, \& took care of the cleanup as my older brother had suggested, carefully wiping my ass with big handfuls of shiny green leaves. These turned out to be poison ivy.

2 days later I was bright red from the backs of my knees to my shoulderblades. My penis was spared, but my testicles turned into stoplights. My ass itched all the way up to my ribcage, it seemed. Yet most of all was the hand I had wiped with; it swelled to the size of Mickey Mouse's after Donald Duck had bopped it with a hammer, \& gigantic blisters formed at the places where the fingers rubbed together. When they burst they left deep divots of raw pink flesh. For 6 weeks I sat in lukewarm starch baths, feeling miserable \& humiliated \& stupid, listening through the open door as my mother \& brother laughed \& listened to Peter Tripp's countdown on the radio \& played Crazy 8s.

\fbox{\bf 10} Dave was a great brother, but too smart for a 10-year-old. His brains were always getting him in trouble, \& he learned at some point (probably after I had wiped my ass with poison ivy) that it was usually possible to get Brother Stevie to join him in the point position when trouble was in the wind. Dave never asked me to shoulder \textit{all} the blame for his often brilliant fuck-ups -- he was neither a sneak nor a coward -- but on several occasions I was asked to share it. Which was, I think, why we both got in trouble when Dave dammed up the stream running through the jungle \& flooded much of lower West Broad Street. Sharing the blame was also the reason we both ran the risk of getting killed while implementing his potentially lethal school science project.

This was probably 1958. I was at Center Grammar School; Dave was at Stratford Junior High. Mom was working at the Stratford Laundry, where she was the only white lady on the mangle crew. That's what she was doing -- feeling sheets into the mangle -- while Dave constructed his Science Fair project. My big brother wasn't the sort of boy to content himself drawing frog-diagrams on construction paper or making The House of the Future out of plastic Tyco bricks \& pained toilet-tissue rolls; Dave aimed for the stars. His project that year was Dave's Super Duper Electromagnet. My brother had great affection for things which were super duper \& things which began with his own name; this latter habit culminated with \textit{Dave's Rag}, which we will come to shortly.

His 1st stab at the Super Duper Electromagnet wasn't very super duper; in fact, it may not have worked at all -- I don't remember for sure. It \textit{did} come out of an actual book, rather than Dave's head, however. The idea was this: you magnetized a spike nail by rubbing it against a regular magnet. The magnetic charge imparted to the spike would be weak, the book said, but enough to pick up a few iron fillings. After trying this, you were supposed to wrap a length of copper wire around the barrel of the spike, \& attach the ends of the wire to the terminals of a dry-cell battery. According to the book, the electricity would strengthen the magnetism, \& you could pick up a lot more iron filings.

Dave didn't just want to pick up a stupid pile of metal flakes, though; Dave wanted to pick up Buicks, railroad boxcars, possibly Army transport planes. Dave wanted to turn on the juice \& move the world in its orbit.

Pow! Super!

We each had our part to play in creating the Super Duper Electromagnet. Dave's part was to build it. My part would be to test it. Little Stevie King, Stratford's answer to Chuck Yeager.


'' -- \cite[pp. 13--]{King2010}

%------------------------------------------------------------------------------%

\section{What Writing Is}
``Telepathy, of course. It's amusing when you stop to think about it -- for years people have argued about whether or not such a thing exists, folks like J. B. Rhine have busted their brains trying to create a valid testing process to isolate it, \& all the time it's been right there, lying out in the open like Mr. Poe's Purloined Letter. All the arts depend upon telepathy to some degree, but I believe that writing offers the purest distillation. Perhaps I'm prejudiced, but even if I am we may as well stick with writing, since it's what we came here to think \& talk about.

My name is Stephen King. I'm writing the 1st draft of this part at my desk (the one under the eave) on a snowy morning in December of 1997. There are things on my mind. Some are worries (bad eyes, Christmas shopping not even started, wife under the weather with a virus), some are good things (our younger son made a surprise visit home from college, I got to play Vince Taylor's ``Brand New Cadillac'' with The Wallflowers at a concert), but right now all that stuff is up top. I'm in another place, a basement place where there are lots of bright lights \& clear images. This is a place I've built for myself over the years. It's far-seeing place. I know it's a little strange, a little bit of a contradiction, that a far-seeing place should also be a basement place, but that's how it is with me. If you construct your own far-seeing place, you might put it in a treetop or on the roof of the World Trade Center or on the edge of the Grand Canyon. That's your little red wagon, as Robert McCammon says in 1 of his novels.

This book is scheduled to be published in the late summer or early fall of 2000. If that's how things work out, then you are somewhere downstream on the timeline from me $\ldots$ but you're quite likely in your own far-seeing place, the one where you go to receive telepathic messages. Not that you \textit{have} to be there; books are a uniquely portable magic. I usually listen to one in the car (always unabridged; I think abridged audiobooks are the pits), \& carry another wherever I go. You just never know when you'll want an escape hatch: mile-long lines at tollbooth plazas, the 15 minutes you have to spend in the hall of some boring college building waiting for your advisor (who's got some yank-off in there threatening to commit suicide because he\texttt{/}she is flunking Custom Kurmfurling 101) to come out so you can get his signature on a drop-card, airport boarding lounges, laundromats on rainy afternoons, \& the absolute worst, which is the doctor's office when the guy is running late \& you have to wait half an hour in order to have something sensitive mauled. At such times I find a book vital. If I have to spend time in purgatory before going to 1 place or the other, I guess I'll be all right as long as there's a lending library (if there is it's probably stocked with nothing but novels by Danielle Steel \& \textit{Chicken Soup} books, ha-ha, joke's on you, Steve).

So I read where I can, but I have a favorite place \& probably you do, too -- a place where the light is good \& the vibe is usually strong. For me it's the blue chair in my study. For you it might be the couch on the sun-porch, the rocker in the kitchen, or maybe it's propped up in your bed -- reading in bed can be heaven, assuming you can get just the right amount of light on the page \& aren't prone to spilling your coffee or cognac on the sheets.

So let's assume that you're in your favorite receiving place just as I am in the place where I do my best transmitting. We'll have to perform our mentalist routine not just over distance but over time as well, yet that presents no real problem; if we can still read Dickens, Shakespeare, \& (with the help of a footnote or 2) Herodotus, I think we can manage the gap between 1997 \& 2000. \& here we go -- actually telepathy in action. You'll notice I have nothing up my sleeves \& that my lips never move. Neither, most likely, do yours.

Look -- here's a table covered with a red cloth. On it is a cage the size of a small fish aquarium. In the cage is a white rabbit with a pink nose \& pink-rimmed eyes. In its front paws is a carrot-stub upon which it is contentedly munching. On its back, clearly marked in blue ink, is the numeral 8.

Do we see the same thing? We'd have to get together \& compare notes to make absolutely sure, but I think we do. There will be necessary variations, of course: some receivers will see a cloth which is turkey red, some will see one that's scarlet, while others may see still other shades. (To color-blind receivers, the red tablecloth is the dark gray of cigar ashes.) Some may see scalloped edges, some may see straight ones. Decorative souls may add a little lace, \& welcome -- my tablecloth is your tablecloth, knock yourself out.

Likewise, the matter of the cage leaves quite a lot of room for individual interpretation. For 1 thing, it is described in terms of \textit{rough comparison}, which is useful only if you \& I see the world \& measure the things in it with similar eyes. It's easy to become careless when making rough comparisons, but the alternative is a prissy attention to detail that takes all the fun out of writing. What am I going to say, ``on the table is a cage 3 feet, 6 inches in length, 2 feet in width, \& 14 inches high''? That's not prose, that's an instruction manual. The paragraph also doesn't tell us what sort of material the cage is made of -- wire mesh? steel rods? glass? -- but does it really matter? We all understand the cage is a see-through medium; beyond that, we don't care. The most interesting thing here isn't even the carrot-munching rabbit in the cage, but the number on its back. Not a 6, not a 4, not 19.5. It's an 8. This is what we're looking at, \& we all see it. I didn't tell you. You didn't ask me. I never opened my mouth \& you never opened yours. We're not even in the same \textit{year} together, let alone the same room $\ldots$ except we \textit{are} together. We're close.

We're having a meeting of the minds.

I sent you a table with a red cloth on it, a cage, a rabbit, \& the number 8 in blue ink. You got them all, especially that blue 8. We've engaged in an act of telepathy. No mythy-mountain shit; real telepathy. I'm not going to belabor the point, but before we go any further you have to understand that I'm not trying to be cute; there \textit{is} a point to be made.

You can approach the act of writing with nervousness, excitement, hopefulness, or even despair -- the sense that you can never completely put on the page what's in your mind \& heart. You can come to the act with your fists clenched \& your eyes narrowed, ready to kick ass \& take down names. You can come to it because you want a girl to marry you or because you want to change the world. Come to it any way but lightly. Let me say it again: \textit{you must not come lightly to the blank page}.

I'm not asking you to come reverently or unquestioningly; I'm not asking you to be politically correct or cast aside your sense of humor (please God you have one). This isn't a popularity contest, it's not the moral Olympics, \& it's not church. But it's \textit{writing}, damn it, not washing the car or putting an eyeliner. If you can take it seriously, we can do business. If you can't or won't, it's time for you to close the book \& do something else.

Wash the car, maybe.'' -- \cite[pp. 83--85]{King2010}

%------------------------------------------------------------------------------%

\section{TOOLBOX}

%------------------------------------------------------------------------------%

\section{ON WRITING}

%------------------------------------------------------------------------------%

\section{ON LIVING: A POSTSCRIPT}

%------------------------------------------------------------------------------%

\section{\& Furthermore, Part I: Door Shut, Door Open}

%------------------------------------------------------------------------------%

\section{\& Furthermore, Part II: A Booklist}

%------------------------------------------------------------------------------%

\section{Further to Furthermore, Part III}

%------------------------------------------------------------------------------%

\section{About Stephen King}

%------------------------------------------------------------------------------%

\printbibliography[heading=bibintoc]
	
\end{document}