\documentclass{article}
\usepackage[backend=biber,natbib=true,style=alphabetic]{biblatex}
\addbibresource{/home/nqbh/reference/bib.bib}
\usepackage{tocloft}
\renewcommand{\cftsecleader}{\cftdotfill{\cftdotsep}}
\usepackage[colorlinks=true,linkcolor=blue,urlcolor=red,citecolor=magenta]{hyperref}
\usepackage{algorithm,algpseudocode,amsmath,amssymb,amsthm,float,graphicx,mathtools}
\allowdisplaybreaks
\numberwithin{equation}{section}
\newtheorem{assumption}{Assumption}[section]
\newtheorem{conjecture}{Conjecture}[section]
\newtheorem{corollary}{Corollary}[section]
\newtheorem{definition}{Definition}[section]
\newtheorem{example}{Example}[section]
\newtheorem{lemma}{Lemma}[section]
\newtheorem{notation}{Notation}[section]
\newtheorem{principle}{Principle}[section]
\newtheorem{problem}{Problem}[section]
\newtheorem{proposition}{Proposition}[section]
\newtheorem{question}{Question}[section]
\newtheorem{remark}{Remark}[section]
\newtheorem{theorem}{Theorem}[section]
\usepackage[left=1cm,right=1cm,top=5mm,bottom=5mm,footskip=4mm]{geometry}
\def\labelitemii{$\circ$}

\title{On Writing: A Memoir of the Craft}
\author{Stephen King}
\date{\today}

\begin{document}
\maketitle
\tableofcontents
\vspace{5mm}
\begin{quotation}
	\textit{``Honesty's the best policy.''} -- Miguel de Cervantes
	
	\textit{``Liars prosper.''} -- Anonymous
\end{quotation}

%------------------------------------------------------------------------------%

\section*{1st Foreword}
``In the early 90s (it might have been 1992, but it's hard to remember when you're having a good time) I joined a rock-\&-roll band composed mostly of writers. The Rock Bottom Remainders were the brainchild of Kathi Kamen Goldmark, a book publicist \& musician from San Francisco. The group included Dave Barry on lead guitar, Ridley Pearson on bass, Barbara Kingsolver on keyboards, Robert Fulghum on mandolin, \& me on rhythm guitar. There was also a trio of ``chick singer,'' \textit{\`a la} the Dixie cups, made up (usually) of Kathi, Tad, Bartimus, \& Amy Tan.

The group was intended as a 1-shot deal -- we would play 2 shows at the American Booksellers Convention, get a few laughs, recapture our misspent youth for 3 or 4 hours, then go our separate ways.

It didn't happen that way, because the group never quite broke up. We found that we liked playing together too much to quit, \& with a couple of ``ringer'' musicians on sax \& drums (plus, in the early days, our musical guru, Al Kooper, at the heart of the group), we sounded pretty good. You'd pay to hear us. Not a lot, not U2 or E Street Bans prices, but maybe what the oldtimers call ``roadhouse money.'' We took the group on tour, wrote a book about it (my wife took the photos \& danced whenever the spirit took her, which was quite often), \& continue to play now \& then, sometimes as The Remainders, sometimes as Raymond Burr's Legs. The personnel comes \& goes -- columnist Mitch Albom has replaced Barbara on keyboards, \& Al doesn't play with the group anymore `cause he \& Kathi don't get along -- but the core has remained Kathi, Amy, Ridley, Dave, Mitch Albom, \& me $\ldots$ plus Josh Kelly on drums \& Erasmo Paolo on sax.

We do it for the music, but we also do it for the companionship. We like each other, \& we like having a chance to talk sometimes about the real job, the day job people are always telling us not to quit. We are writers, \& we never ask one another where we get our ideas; we know we don't know.

1 night while we were eating Chinese before a big in Miami Beach, I asked Amy if there was any 1 question she was \textit{never} asked during the Q-\&-A that follows almost every writer's talk -- that question you never get to answer when you're standing in front of a group of author-struck fans \& pretending you don't put your pants on 1 leg at a time like everyone else. Amy paused, thinking it over very carefully, \& then said: ``No one ever asks about the language.''

I owe an immense debt of gratitude to her for saying that. I had been playing with the idea of writing a little book about writing for a year or more at that time, but had held back because I didn't trust my own motivations -- \textit{why} did I want to write about writing? What made me think I had anything worth saying?

The easy answer is that someone who has sold as many books of fiction as I have must have \textit{something} worthwhile to say about writing it, but the easy answer isn't always the truth. Colonel Sanders sold a hell of a lot of fried chicken, but I'm not sure anyone wants to know how he made it. If I was going to be presumptuous enough to tell people how to write, I felt there had to be a better reason than my popular success. Put another way, I didn't want to write a book, even a short one like this, that would leave me feeling like either a literary gasbag or a transcendental asshole. There are enough of those books -- \& those writers -- on the market already, thanks.

But Amy was right: nobody ever asks about the language. They ask the DeLillos \& the Updikes \& the Styrons, but they don't ask popular novelists. Yet many of us proles also care about the language, in our humble way, \& care passionately about the art \& craft of telling stories on paper. What follows is an attempt to put down, briefly \& simply, how I came to the craft, what I know about it now, \& how it's done. It's about the day job; it's about the language.

This book is dedicated to Amy Tan, who told me in a very simple \& direct way that it was okay to write it.'' -- \cite[p. 9]{King2010}

%------------------------------------------------------------------------------%

\section*{2nd Foreword}
``This is a short book because most books about writing are filled with bullshit. Fiction writers, present company included, don't understand very much about what they do -- not why it works when it's good, not why it doesn't when it's bad. I figured the shorter the book, the less the bullshit.

1 notable exception to the bullshit rule is \textit{The Elements of Style}, by William Strunk Jr. \& E. B. White. There is little or no detectable bullshit in that book. (Of course it's short; at 85 pages it's much shorter than this one.) I'll tell you right now that every aspiring writer should read \textit{The Elements of Style}. Rule 17 in the chapter titled Principles of Composition is ``Omit needless words.'' I will try to do that here.'' -- \cite[p. 10]{King2010}

%------------------------------------------------------------------------------%

\section*{3rd Foreword}
``1 rule of the road not directly stated elsewhere in this book: ``The editor is always right.'' The corollary is that no writer will take all of his or her editor's advice; for all have sinned \& fallen short of editorial perfection. Put another way, to write is human, to edit is divine. Chuck Verrill edited this book, as he has so many of my novels. And as usual, Chuck, you were divine. -- Steve'' -- \cite[p. 11]{King2010}

%------------------------------------------------------------------------------%

\section{C.V.}
``I was stunned by Mary Karr's memoir, \textit{The Liar's Club}. Not just by its ferocity, its beauty, \& by her delightful grasp of the vernacular, but by its \textit{totality} -- she is a woman who remembers \textit{everything} about her early years.

I'm not that way. I lived an odd, herky-jerky childhood, raised by a single parent who moved around a lot in my earliest years \& who -- I am not completely sure of this -- may have farmed my brother \& me out to 1 of her sisters for awhile because she was economically or emotionally unable to cope with us for a time. Perhaps she was only chasing our father, who piled up all sorts of bills \& then did a runout when I was 2 \& my brother David was 4. If so, she never succeeded in finding him. My mom, Nellie Ruth Pillsbury King, was 1 of America's early liberated women, but not by choice.

Mary Karr presents her childhood in an almost unbroken panorama. Mine is a fogged-out landscape from which occasional memories appear like isolated trees $\ldots$ the kind that look as if they might like to grab \& eat you.

What follows are some of those memories, plus assorted snapshots from the somewhat more coherent days of my adolescence \& young manhood. This is not an autobiography. It is, rather, a kind of \textit{curriculum vitae} -- my attempt to show ow 1 writer was formed. Not how 1 writer was \textit{made}; I don't believe writers \textit{can} be made, either by circumstances or by self-will (although I did believe those things once). The equipment comes with the original package. Yet it is by no means unusual equipment; I believe large numbers of people have at least some talent as writers \& storytellers, \& that those talents can be strengthened \& sharpened. If I didn't believe that, writing a book like this would be a waste of time.

This is how it was for me, that's all -- a disjointed growth process in which ambition, desire, luck, \& a little talent all played a part. Don't bother trying to read between the lines, \& don't look for a through-line. There are \textit{no} lines -- only snapshots, most out of focus.

\fbox{\bf1} My earliest memory is of imaging I was someone else -- imagining that I was, in fact, the Ringing Brothers Circus Strongboy. This was at my Aunt Ethelyn \& Uncle Oren's house in Durham, Maine. My aunt remembers this quite clearly, \& says I was 2 \& a half or maybe 3 years old.

I had found a cement cinderblock in a corner of the garage \& had managed to pick it up. I carried it slowly across the garage's smooth cement floor, except in my mind I was dressed in an animal skin singlet (probably a leopard skin) \& carrying the cinderblock across the center ring. The vast crowd was silent. A brilliant bluewhite spotlight marked my remarkable progress. Their wondering faces told the story: never had they seen such an incredibly strong kid. ``\& he's only \textit{2}!'' someone muttered in disbelief.

Unknown to me, wasps had constructed a small nest in the lower half of the cinderblock. 1 of them, perhaps pissed off at being relocated, flew out \& stung me on the ear. The pain was brilliant, like a poisonous inspiration. It was the worst pain I had ever suffered in my short life, but it only held the top spot for a few seconds. When I dropped the cinderblock on 1 bare foot, mashing all 5 toes, I forgot all about the wasp. I can't remember if I was taken to the doctor, \& neither can my Aunt Ethelyn (Uncle Oren, to whom the Evil Cinderblock surely belonged, is almost 20 years dead), but she remembers the sting, the mashed toes, \& my reaction. ``How you howled, Stephen!'' she said. ``You were certainly in fine voice that day.''

\fbox{\bf2} A year or so later, my mother, my brother, \& I were in West De Pere, Wisconsin. I don't know why. Another of my mother's sisters, Cal (a WAAC beauty queen during World War II), lived in Wisconsin with her convivial beer-drinking husband, \& maybe Mom had moved to be near them. If so, I don't remember seeing much of the Weimers. \textit{Any} of them, actually. My mother was working, but I can't remember what her job was, either. I want to say it was a bakery she worked in, but I think that came later, when we moved to Connecticut to live near her sister Lois \& \textit{her} husband (no beer for Fred, \& not much in the way of conviviality, either; he was a crewcut daddy who was proud of driving his convertible with the top \textit{up}, God knows why).

There was a stream of babysitters during our Wisconsin period. I don't know if they left because David \& I were a handful, or because they found better-paying jobs, or because my mother insisted on higher standards than they were willing to rise to; all I know is that there were a lot of them. The only one I remember with any clarity is Eula, or maybe she was Beulah. She was a teenager, she was as big as a house, \& she laughed a lot. Eula-Beulah had a wonderful sense of humor, even at 4 I could recognize that, but it was a \textit{dangerous} sense of humor -- there seemed to be a potential thunderclap hidden inside each hand-patting, butt-rocking, head-tossing outburst of glee. When I see those hidden-camera sequences where real-life babysitters \& nannies just all of a sudden wind up \& clout the kids, it's my days with Eula-Beulah I always think of.

Was she as hard on my brother David as she was on me? I don't know. He's not in any of these pictures. Besides, he would have been less at risk from Hurricane Eula-Beulah's dangerous winds; at 6, he would have been in the 1st grade \& off the gunnery range for most of the day.

Eula-Beulah would be on the phone, laughing with someone, \& beckon me over. She would hug me, tickle me, get me laughing, \& then, still laughing, go upside my head enough to knock me down. Then she would tickle me with her bare feet until we were both laughing again.

Eula-Beulah was prone to farts -- the kind that are both loud \& smelly. Sometimes when she was so afflicted, she would throw me on the couch, drop her wool-skirted butt on my face, \& let loose. ``Pow!'' she'd cry in high glee. It was like being buried in marshgas fireworks. I remember the dark, the sense that I was suffocating, \& I remember laughing. Because, while what was happening was sort of horrible, it was also sort of funny. In many ways, Eula-Beulah prepared me for literary criticism. After having a 200-pound babysitter fart on your face \& yell \textit{Pow!, The Village Voice} holds few terrors.

I don't know what happened to the other sitters, but Eula-Beulah was fired. It was because of the eggs. 1 morning Eula-Beulah fried me an egg for breakfast. I ate it \& asked for another one. Eula-Beulah fried me a 2nd egg, then asked if I wanted another one. \& another one. \& so on. I stopped after 7, I think -- 7 is the number that sticks in my mind, \& quite clearly. Maybe we ran out of eggs. Maybe I cried off. Or maybe Eula-Beulah got scared. I don't know, but probably it was good that the game ended at 7. 7 eggs is quite a few for a 4-year-old.

I felt all right for awhile, \& then I yarked all over the floor. Eula-Beulah laughed, then went upside my head, then shoved me into the closet \& locked the door. Pow. If she'd locked me in the bathroom, she might saved her job, but she didn't. As for me, I didn't really mind being in the closet. It was dark, but it smelled of my mother's Coty perfume, \& there was a comforting line of light under the door.

I crawled to the back of the closet, Mom's coats \& dresses brushing along my back. I began to belch -- long loud belches that burned like fire. I don't remember being sick to my stomach but I must have been, because when I opened my mouth to let out another burning belch, I yarked again instead. All over my mother's shoes. That was the end for Eula-Beulah. When my mother came home from work that day, the babysitter was fast asleep on the couch \& little Stevie was locked in the closet, fast asleep with half-digested fried eggs drying in his hair.

\fbox{\bf3} Our stay in West De Pere was neither long nor successful. We were evicted from our 3rd-floor apartment when a neighbor spotted my 6-year-old brother crawling around on the roof \& called the police. I don't know where my mother was when this happened. I don't know where the babysitter of the week was, either. I only know that I was in the bathroom, standing with my bare feet on the heater, watching to see if my brother would fall off the roof or make it back into the bathroom okay. He made it back. He is now 55 \& living in New Hampshire.

\fbox{\bf4} When I was 5 or 6, I asked my mother if she had ever seen anyone die. Yes, she said, she had been 1 person die \& had heard another one. I asked how you could hear a person die \& she told me that it was a girl who had drowned off Prout's Neck in the 1920s. She said the girl swam out past the rip, couldn't get back in, \& began screaming for help. Several men tried to reach her, but that day's rip had developed a vicious undertow, \& they were all forced back. In the end they could only stand around, tourists \& townies, the teenager who became my mother among them, waiting for a rescue boat that never came \& listening to that girl scream until her strength gave out \& she went under. Her body washed up in New Hampshire, my mother said. I asked how old the girl was. Mom said she was 14, then read me a comic book \& packed me off to bed. On some other day she told me about the one she saw -- a sailor who jumped off the roof of the Graymore Hotel in Portland, Maine, \& landed in the street.

``He splattered,'' my mother said in her most matter-of-fact tone. She paused, \& added, ``The stuff that came out of him was green. I have never forgotten it.''

That makes 2 of us, Mom.

\fbox{\bf5} Most of the 9 months I should have spent in the 1st grade I spent in bed. My problems started with the measles -- a perfectly ordinary case -- \& then got steadily worse. I had bout after bout of what I mistakenly thought was called ``stripe throat''; I lay in bed drinking cold water \& imagining my throat in alternating stripes of red \& white (this was probably not so far wrong).

At some point my ears become involved, \& 1 day my mother called a taxi (she did not drive) \& took me to a doctor too important to make house calls -- an ear specialist. (For some reason I got the idea that this sort of doctor was called an otiologist.) I didn't care whether he specialized in ears or assholes. I had a fever of 104 degrees, \& each time I swallowed, pain lit up the sides of my face like a jukebox.

The doctor looked in my ears, spending most of his time (I think) on the left one. Then he laid me down on his examining table. ``Lift up a minute, Stevie,'' his nurse said, \& put a large absorbent cloth -- it might have been a diaper -- under my head, so that my cheek rested on it when I lay back down. I should have guessed that something was rotten in Denmark. Who knows, maybe I did.

There was a sharp smell of alcohol. A clank as the ear doctor opened his sterilizer. I saw the needle in his hand -- it looked as long as the ruler in my school pencil-box -- \& tensed. The ear doctor smiled reassuringly \& spoke the lie for which doctors should be immediately jailed (time of incarceration to be doubled when the lie is told to a child): ``Relax, Stevie, this won't hurt.'' I believed him.

He slid the needle into my ear \& punctured my eardrum with it. The pain was beyond anything I have ever felt since -- the only thing close was the 1st month of recovery after being struck by a van in the summer 1999. That pain was longer in duration but not so intense. The puncturing of my eardrum was pain beyond the world. I screamed. There was a sound inside my head -- a loud kissing sound. Hot fluid ran out of my ear -- it was as if I had started to cry out of the wrong hole. God knows I was crying enough out of the right ones by then. I raised my streaming face \& looked unbelieving at the ear doctor \& the ear doctor's nurse. Then I looked at the cloth the nurse had spread over the top 3rd of the exam table. It had a big wet patch on it. There were fine tendrils of yellow pus on it as well.

``There,'' the ear doctor said, patting my shoulder. ``You were very brave, Stevie, \& it's all over.''

The next week my mother called another taxi, we went back to the ear doctor's, \& I found myself once more lying on my side with the absorbent square of cloth under my head. The ear doctor once again produced the smell of alcohol -- a smell I still associate, as I suppose many people do, with pain \& sickness \& terror -- \& with it, the long needle. He once more assured me that it wouldn't hurt, \& I once more believed him. Not completely, but enough to be quiet while the needle slid into my ear.

It \textit{did} hurt. Almost as much as the 1st time, in fact. The smoothing sound in my head was louder, too; this time it was giants kissing (``suckin' face \& rotatin' tongues,'' as we used to say). ``There,'' the ear doctor's nurse said when it was over \& I lay there crying in a puddle of watery pus. ``It only hurts a little, \& you don't want to be deaf, do you? Besides, it's all over.''

I believed that for about 5 days, \& then another taxi came. We went back to the ear doctor's. I remember the cab driver telling my mother that he was going to pull over \& let us out if she couldn't shut that kid up.

Once again it was me on the exam table with the diaper under my head \& my mom out in the waiting room with a magazine she was probably incapable of reading (or so I like to imagine), Once again the pungent smell of alcohol \& the doctor turning to me with a needle that looked as long as my school ruler. Once more the smile, the approach, the assurance that \textit{this} time it wouldn't hurt.

Since the repeated eardrum-lancings when I was 6, 1 of my life's firmest principles has been this: Fool me once, shame on you. Fool me twice, shame on me. Fool me 3 times, shame on both of us. The 3rd time on the ear doctor's table I struggled \& screamed \& thrashed \& fought. Each time the needle came near the side of my face, I knocked it away. Finally the nurse called my mother in from the waiting room, \& the 2 of them managed to hold me long enough for the doctor to get his needle in. I screamed so long \& so loud that I can still hear it. In fact, I think that in some deep valley of my head that last scream is still echoing.

\fbox{\bf6} In a dull cold month not too long after that -- it would have been Jan or Feb of 1954, if I've got the sequence right -- the taxi came again. This time the specialist wasn't the ear doctor but a throat doctor. Once again my mother sat in the waiting room, once again I sat on the examining table with a nurse hovering nearby, \& once again there was that sharp smell of alcohol, an aroma that still has the power to double my heartbeat in the space of 5 seconds.

All that appeared this time, however, was some sort of throat swab. It stung, \& it tasted awful, but after the ear doctor's long needle it was a walk in the park. The throat doctor donned an interesting gadget that went around his head on a strap. It had a mirror in the middle, \& a bright fierce light that shone out of it like a 3rd eye. He looked down my gullet for a long time, urging me to open wider until my jaws creaked, but he did not put needles into me \& so I loved him. After awhile he allowed me to close my mouth \& summoned my mother.

``The problem is his tonsils,'' the doctor said. ``They look like a cat clawed them. They'll have to come out.''

At some point after that, I remember being wheeled under bright lights. A man in a white mask bent over me. He was standing at the head of the table I was lying on (1953 \& 1954 were my years for lying on tables), \& to me he looked upside down.

``Stephen,'' he said. ``Can you hear me?''

I said I could.

``I want you to breathe deep,'' he said. ``When you wake up, you can have all the ice cream you want.''

He lowered a gadget over my face. In the eye of my memory, it looks like an outboard motor. I took a deep breath, \& everything went black. When I woke up I was indeed allowed all the ice cream I wanted, which was a fine joke on me because I didn't want any. My throat felt swollen \& fat. But it was better than the old needle-in-the-ear trick. Oh yes. \textit{Anything} would have been better than the old needle-in-the-ear trick. Take my tonsils if you have to, put a steel birdcage on my leg if you must, but God save me from the otiologist.

\fbox{\bf7} That year my brother David jumped ahead to the 4th grade \& I was pulled out of school entirely. I had missed too much of the 1st grade, my mother \& the school agreed; I could start it fresh in the fall of the year, if my health was good.

Most of that year I spent either in bed or housebound. I read my way through approximately 6 tons of comic books, progressed to Tom Swift \& Dave Dawson (a heroic World War II pilot whose various planes were always ``prop-clawing for altitude''), then moved on to Jack London's bloodcurdling animal tales. At some point I began to write my own stories. Imitation preceded creation; I would copy \textit{Combat Casey} comics word for word in my Blue Horse tablet, sometimes adding my own descriptions where they seemed appropriate. ``They were camped in a big dratty farmhouse room,'' I might write; it was another year or 2 before I discovered that \textit{drat} \& \textit{draft} were different words. During that same period I remember believing that \textit{details} were \textit{dentals} \& that a bitch was an extremely tall woman. A son of a bitch was apt to be a basketball player. When you're 6, most of your Bingo balls are still floating around in the draw-tank.

Eventually I showed 1 of these copycat hybrids to my mother, \& she was charmed -- I remember he slightly amazed smile, as if she was unable to believe a kid of hers could be so smart -- practically a damned prodigy, for God's sake. I had never seen that look on her face before -- not on my account, anyway -- \& I absolutely loved it.

She asked me if I had made the story up myself, \& I was forced to admit that I had copied most of it out of a funnybook. She seemed disappointed, \& that drained away much of my pleasure. At last she handed back my tablet. ``Write 1 of your own, Stevie,'' she said. ``Those \textit{Combat Casey} funny-books are just junk -- he's always knocking someone's teeth out. I bet you could do better. Write 1 of your own.''

I remember an immense feeling of \textit{possibility} at the idea, as if I had been ushered into a vast building filled with closed doors \& had been given leave to open any I liked. There were more doors than 1 person could ever open in a lifetime, I thought (\& still think).

I eventually wrote a story about 4 magic animals who rode around in an old car, helping out little kids. Their leader was a large white bunny named Mr. Rabbit Trick. He got to drive the car. The story was 4 pages long, laboriously printed in pencil. No one in it, so far as I can remember, jumped from the roof of the Graymore Hotel. When I finished, I gave it to my mother, who sat down in the living room, put her pocketbook on the floor beside her, \& read it all at once. I could tell she liked it -- she laughed in all the right places -- but I couldn't tell if that was because she liked me \& wanted me to feel good or because it really \textit{was} good.

``You didn't copy this one?'' she asked when she had finished. I said no, I hadn't. She said it was good enough to be in a book. Nothing anyone has said to me since has made me feel any happier. I wrote 4 more stories about Mr. Rabbit Trick \& his friends. She gave me a quarter apiece for them \& sent them around to her 4 sisters, who pitied her a little, I think. \textit{They} were all still married, after all; their men had stuck. It was true that Uncle Fred didn't have much sense of humor \& was stubborn about keeping the top of his convertible up, it was also true that Uncle Oren drank quite a bit \& had dark theories about how the Jews were running the world, but they were \textit{there}. Ruth, on the other hand, had been left holding the baby when Don ran out. She wanted them to see that he was a talented baby, at least.

4 stories. A quarter apiece. That was the 1st buck I made in this business.

\fbox{\bf9} We moved to Stratford, Connecticut. By then I was in the 2nd grade \& stone in love with the pretty teenage girl who lived next door. She never looked twice at me in the daytime, but at night, as I lay in bed \& drifted toward sleep, we ran away from the cruel world of reality again \& again. My new teacher was Mrs. Taylor, a kind lady with gray Elsa Lanchester -- \textit{Bride of Frankenstein} hair \& protruding eyes. ``When we're talking I always want to cup my hands under Mrs. Taylor's peepers in case they fall out,'' my mom said.

Our new 3rd-floor apartment was on West Broad Street. A block down the hill, not far from Teddy's Market \& across from Burrets Building Materials, was a huge tangled wilderness area with a junkyard on the far side \& a train track running through the middle. This is 1 of the places I keep returning to in my imagination; it turns up in my books \& stories again \& again, under a variety of names. The kids in \textit{It} called it the Barrens; we called it the jungle. Dave \& I explored it for the 1st time not long after we had moved into our new place. It was summer. It was hot. It was great. We were deep into the green mysteries of this cool new playground when I was struck by an urgent need to move my bowels.

``Dave,'' I said. ``Take me home! I have to push!'' (This was the word we were given for this particular function.)

David didn't want to hear it. ``Go do it in the woods,'' he said. It would take at least half an hour to walk me home, \& he had no intention of giving up such a shining stretch of time just because his little brother had to take a dump.

``I can't!'' I said, shocked by the idea. ``I won't be able to wipe!''

``Sure you will,'' Dave said. ``Wipe yourself with some leaves. That's how the cowboys \& Indians did it.''

By then it was probably too late to get home, anyway; I have an idea I was out of options. Besides, I was enchanted by the idea of shitting like a cowboy. I pretended I was Hopalong Cassidy, squatting in the underbrush with my gun drawn, not to be caught unawares even at such a personal moment. I did my business, \& took care of the cleanup as my older brother had suggested, carefully wiping my ass with big handfuls of shiny green leaves. These turned out to be poison ivy.

2 days later I was bright red from the backs of my knees to my shoulderblades. My penis was spared, but my testicles turned into stoplights. My ass itched all the way up to my ribcage, it seemed. Yet most of all was the hand I had wiped with; it swelled to the size of Mickey Mouse's after Donald Duck had bopped it with a hammer, \& gigantic blisters formed at the places where the fingers rubbed together. When they burst they left deep divots of raw pink flesh. For 6 weeks I sat in lukewarm starch baths, feeling miserable \& humiliated \& stupid, listening through the open door as my mother \& brother laughed \& listened to Peter Tripp's countdown on the radio \& played Crazy 8s.

\fbox{\bf 10} Dave was a great brother, but too smart for a 10-year-old. His brains were always getting him in trouble, \& he learned at some point (probably after I had wiped my ass with poison ivy) that it was usually possible to get Brother Stevie to join him in the point position when trouble was in the wind. Dave never asked me to shoulder \textit{all} the blame for his often brilliant fuck-ups -- he was neither a sneak nor a coward -- but on several occasions I was asked to share it. Which was, I think, why we both got in trouble when Dave dammed up the stream running through the jungle \& flooded much of lower West Broad Street. Sharing the blame was also the reason we both ran the risk of getting killed while implementing his potentially lethal school science project.

This was probably 1958. I was at Center Grammar School; Dave was at Stratford Junior High. Mom was working at the Stratford Laundry, where she was the only white lady on the mangle crew. That's what she was doing -- feeling sheets into the mangle -- while Dave constructed his Science Fair project. My big brother wasn't the sort of boy to content himself drawing frog-diagrams on construction paper or making The House of the Future out of plastic Tyco bricks \& pained toilet-tissue rolls; Dave aimed for the stars. His project that year was Dave's Super Duper Electromagnet. My brother had great affection for things which were super duper \& things which began with his own name; this latter habit culminated with \textit{Dave's Rag}, which we will come to shortly.

His 1st stab at the Super Duper Electromagnet wasn't very super duper; in fact, it may not have worked at all -- I don't remember for sure. It \textit{did} come out of an actual book, rather than Dave's head, however. The idea was this: you magnetized a spike nail by rubbing it against a regular magnet. The magnetic charge imparted to the spike would be weak, the book said, but enough to pick up a few iron fillings. After trying this, you were supposed to wrap a length of copper wire around the barrel of the spike, \& attach the ends of the wire to the terminals of a dry-cell battery. According to the book, the electricity would strengthen the magnetism, \& you could pick up a lot more iron filings.

Dave didn't just want to pick up a stupid pile of metal flakes, though; Dave wanted to pick up Buicks, railroad boxcars, possibly Army transport planes. Dave wanted to turn on the juice \& move the world in its orbit.

Pow! Super!

We each had our part to play in creating the Super Duper Electromagnet. Dave's part was to build it. My part would be to test it. Little Stevie King, Stratford's answer to Chuck Yeager.

Dave's new version of the experiment bypassed the pokey old dry cell (which was probably flat anyway when we bought it at the hardware store, he reasoned) in favor of actual wall-current. Dave cut the electrical cord off an old lamp someone had put out on the curb with the trash, stripped the coating all the way down to the plug, then wrapped his magnetized spike in spirals of bare wire. Then, sitting on the floor in the kitchen of our West Broad Street apartment, he offered me the Super Duper Electromagnet \& bade me do my part \& plug it in.

I hesitated -- give me at least that much credit -- but in the end, Dave's manic enthusiasm was too much to withstand. I plugged it in. There was no noticeable magnetism, but the gadget \textit{did} blow out every light \& electrical appliance in our apartment, every light \& electrical appliance in the building, \& every light \& electrical appliance in the building next door (where my dream-girl lived in the ground-floor apartment). Something popped in the electrical transformer out front, \& some cops came. Dave \& I spent a horrible hour watching from out mother's bedroom window, the only one that looked out on the street (all the others had a good view of the grassless, turd-studded yard behind us, where the only living thing was a mangy canine named Roop-Roop). When the cops left, a power truck arrived. A man in spiked shoes climbed the pole between the 2 apartment houses to examine the transformer. Under other circumstances, this would have absorbed us completely, but not that day. That day we could only wonder if our mother would come \& see us in reform school. Eventually, the lights came back on \& the power truck went away. We were not caught \& lived to fight another day. Dave decided he might build a Super Duper Glider instead of a Super Duper Electromagnet for his science project. I, he told me, would get to take the 1st ride. Wouldn't that be great?

\fbox{\bf 11} I was born in 1947 \& we didn't get our 1st television until 1958. The 1st thing I remember watching on it was \textit{Robot Monster}, a film in which a guy dressed in an ape-suit with a goldfish bowl on his head -- Ro-Man, he was called -- ran around trying to kill the last survivors of a nuclear war. I felt this was art of quite a high nature.

I also watched \textit{Highway Patrol} with Broderick Crawford as the fearless Dan Matthews, \& \textit{1 Step Beyond}, hosted by John Newland, the man with the world's spookiest eyes. There was \textit{Cheyenne} \& \textit{Sea Hunt, Your Hit Parade} \& \textit{Annie Oakley}; there was Tommy Rettig as the 1st of Lassie's many friends, Jock Mahoney as \textit{The Range Rider}, \& Andy Devine yowling, ``Hey, Wild Bill, wait for me!'' in his odd, high voice. There was a whole world of vicarious adventure which came packaged in black-\&-white, 14 inches across \& sponsored by brand names which still sound like poetry to me. I loved it all.

But TV came relatively late to the King household, \& I'm glad. I am, when you stop to think of it, a member of a fairly select group: the final handful of American novelists who learned to read \& write before they learned to eat a daily helping of video bullshit. This might not be important. On the other hand, if you're just starting out as a writer, you could do worse than strip your television's electric plug-wire, wrap a spike around it, \& then stick it back into the wall. See what blows, \& how far.

Just an idea.

\fbox{\bf12} In the late 1950s, a literary agent \& compulsive science fiction memorabilia collector named Forrest J. Ackerman changed the lives of thousands of kids -- I was one -- when he began editing a magazine called \textit{Famous Monsters of Filmland}. Ask anyone who has been associated with the fantasy-horror-science fiction genres in the last 30 years about this magazine, \& you'll get a laugh, a flash of the eyes, \& a stream of bright memories -- I practically guarantee it.

Around 1960, Forry (who sometimes referred to himself as ``the Ackermonster'') spun off the short-lived by interesting \textit{Spacemen}, a magazine which covered science fiction films. In 1960, I sent a story to \textit{Spacemen}. It was, as well as I can remember, the 1st story I ever submitted for publication. I don't recall the title, but I was still in the Ro-Man phase of my development, \& this particular tale undoubtedly owed a great deal to the killer ape with the goldfish bowl on his head.

My story was rejected, but Forry kept it. (Forry keeps \textit{everything}, which anyone who has ever toured his house -- the Ackermansion -- will tell you.) About 20 years later, while I was signing autographs at a Los Angeles bookstore, Forry turned up in line $\ldots$ with my story, single-spaced \& typed with the long-vanished Royal typewriter my mom gave me for Christmas the year I was 11. He wanted me to sign it to him, \& I guess I did, although the whole encounter was so surreal I can't be completely sure. Talk about your ghosts. Man oh man.

\fbox{\bf13} The 1st story I did actually publish was in a horror fanzine issued by Mike Garrett of Birmingham, Alabama (Mike is still around, \& still in the biz). He published this novella under the title ``In a Half-World of Terror,'' but I still like my title much better. Mine was ``I Was a Teen-Age Grave-robber.'' Super Duper! Pow!

\fbox{\bf14} My 1st really original story idea -- you always know the 1st one, I think -- came near the end of Ike's 8-year reign of benignity. I was sitting at the kitchen table of our house in Durham, Maine, \& watching my mother stick sheets of S\&H Green Stamps into a book. (For more colorful stories about Green Stamps, see \textit{The Liar's Club}.) Our little family troika had moved back to Maine so our mom could take care of her parents in their declining years. Mama was about 80 at that time, obese \& hypertensive \& mostly blind; Daddy Guy was 82, scrawny, morose, \& prone to the occasional Donald Duck outburst which only my mother could understand. Mom called Daddy Guy ``Fazza.''

My mother's sisters had gotten my mom this job, perhaps thinking they could kill 2 birds with 1 stone -- the aged Ps would be taken care of in a homey environment by a loving daughter, \& The Nagging Problem of Ruth would be solved. She would no longer be adrift, trying to take care of 2 boys while she floated almost aimlessly from Indiana to Wisconsin to Connecticut, baking cookies at 5 in the morning or pressing sheets in a laundry where the temperatures often soared to 110 in the summer \& the foreman gave out salt pills at 1 \& 3 every afternoon from July to the end of September.

She hated her new job, I think -- in their effort to take care of her, her sisters turned our self-sufficient, funny, slightly nutty mother into a sharecropper living a largely cashless existence. The money the sisters sent her each month covered the groceries but little else. They sent boxes of clothes for us. Toward the end of each summer, Uncle Clayt \& Aunt Ella (who were not, I think, real relatives at all) would bring cartons of canned vegetables \& preserves. The house we lived in belonged to Aunt Ethelyn \& Uncle Oren. \& once she was there, Mom was caught. She got another actual job after the old folks died, but she lived in that house until the cancer got her. When she left Durham for the last time -- David \& his wife Linda cared for her during the final weeks of her final illness -- I have an idea she was probably more than ready to go.

\fbox{\bf15} Let's get 1 thing clear right now, shall we? There is no idea Dump, no Story Central, no Island of the Buried Bestsellers; good story ideas seem to come quite literally from nowhere, sailing at you right out of the empty sky: 2 previously unrelated ideas come together \& making something new under the sun. Your job isn't to find these ideas but to recognize them when they show up.

On the day this particular idea -- the 1st really good one -- came sailing at me, my mother remarked that she needed 6 more books of stamps to get a lamp she wanted to give her sister Molly for Christmas, \& she didn't think she would make it in time. ``I guess it will have to be for her birthday, instead,'' she said. ``These cussed things always look like a lot until you stick them in a book.'' Then she crossed her eyes \& ran her tongue out at me. When she did, I saw her tongue was S\&H green. I thought how nice it would be if you could make those damned stamps in your basement, \& in that instant a story called ``Happy Stamps'' was born. The concept of counterfeiting Green Stamps \& the sight of my mother's green tongue created it in an instant.

The hero of my story was your classic Poor Schmuck, a guy named Roger who had done jail time twice for counterfeiting money -- 1 more burst could make him a 3-time loser. Instead of money, he began to counterfeit Happy Stamps $\ldots$ except, he discovered, the design of Happy Stamps was so moronically simple that he wasn't really counterfeiting at all; he was creating reams of the actual article. In a funny scene -- probably the 1st really competent scene I ever wrote -- Roger sits in the living room with his old mom, the 2 of them mooning over the Happy Stamps catalogue while the printing press runs downstairs, ejecting bale after bale of those same trading stamps.

``Great Scott!'' Mom says. ``According to the fine print, you can get \textit{anything} with Happy Stamps, Roger -- you tell them what you want, \& they figure out how many books you need to get it. Why, for 6 or 7 million books, we could probably get a Happy Stamps house in the suburbs!''

Roger discovers, however, that although the \textit{stamps} are perfect, the \textit{glue} is defective. If you lap the stamps \& stick them in the book they're fine, but if you send them through a mechanical licker, the pink Happy Stamps turn blue. At the end of the story, Roger is in the basement, standing in front of a mirror. Behind him, on the table, are roughly 90 books of Happy Stamps, each book filled with individually licked sheets of stamps. Our hero's lips are pink. He runs out his tongue; that's even pinker. Even his teeth are turning pink. Mom calls cheerily down the stairs, saying she has just gotten off the phone with the Happy Stamps National Redemption Center in Terre Haute, \& the lady said they could probably get a nice Tudor home in Weston for only 11 600 000 books of Happy Stamps.

``That's nice, Mom,'' Roger says. He looks at himself a moment longer in the mirror, lips pink \& eyes bleak, then slowly returns to the table. Behind him, billions of Happy Stamps are stuffed into basement storage bins. Slowly, our hero opens a fresh stamp-book, then begins to lick sheets \& stick them in. Only 11 590 000 books to go, he thinks as the story ends, \& Mom can have her Tudor.

There were things wrong with this story (the biggest hole was probably Roger's failure simply to start over with a different glue), but it was cute, it was fairly original, \& I knew I had done some pretty good writing. After a long time spent studying the markets in my beat-up \textit{Writer's Digest}, I sent ``Happy Stamps'' off to \textit{Alfred Hitchcock's Mystery Magazine}. It came back 3 weeks later with a form rejection slip attached. This slip bore Alfred Hitchcock's unmistakable profile in red rink \& wished me good luck with my story. At the bottom was an unsigned jotted message, the only personal response I got from \textit{AHMM} over 8 years of periodic submissions. ``Don't staple manuscripts,'' the postscript read. ``Loose pages plus paperclip equal correct way to submit copy.'' This was pretty cold advice, I thought, but useful in its way. I have never stapled a manuscript since.

\fbox{\bf16} My room in our Durham house was upstairs, under the eaves. At night I could lie in bed beneath 1 of these eaves -- if I sat up suddenly, I was apt to whack my head a good one -- \& read by the light of a gooseneck lamp that put an amusing boa constrictor of shadow on the ceiling. Sometimes the house was quiet except for the whoosh of the furnace \& the patter of rats in the attic; sometimes my grandmother would spend an hour or so around midnight yelling for someone to check Dick -- she was afraid he hadn't been fed. Dick, a horse she'd had in her days as a schoolteacher, was at least 40 years dead. I had a desk beneath the room's other eave, my old Royal typewriter, \& 100 or so paperback books, mostly science fiction, which I lined up along the baseboard. On my bureau was a Bible won for memorizing verses in Methodist Youth Fellowship \& a Webcor phonograph with an automatic changer \& a turntable covered in soft green velvet. On it I played my records, mostly 45s by Elvis, Chuck Berry, Freddy Cannon, \& Fats Domino. I liked Fats; he knew how to rock, \& you could tell he was having fun.

When I got the rejection slip from \textit{AHMM}, I pounded a nail into the wall above the Webcor, wrote ``Happy Stamps'' on the rejection slip, \& poked it onto the nail. Then I sat on my bed \& listened to Fats using ``I'm Ready.'' I felt pretty good, actually. When you're still too young to shave, optimism is a perfectly legitimate response to failure.

By the time I was 14 (\& shaving twice a week whether I needed to or not) the nail in my wall would no longer support the weight of the rejection slips impaled upon it. I replaced the nail with a spike \& went on writing. By the time I was 16 I'd begun to get rejection slips with handwritten notes a little more encouraging than the advice to stop using staples \& using paperclips. The 1st of these hopeful notes was from Algis Budrys, then the editor of \textit{Fantasy \& Science Fiction}, who read a story of mine called ``The Night of the Tiger'' (the inspiration was, I think, an episode of \textit{The Fugitive} in which Dr. Richard Kimble worked as an attendant cleaning out cages in a zoo or a circus) \& wrote: ``This is good. Not for us, but good. You have talent. Submit again.''

Those 4 brief sentences, scribbled by a fountain pen that left big ragged blotches in its wake, brightened the dismal winter of my 16th year. 10 years or so later, after I'd sold a couple of novels, I discovered ``The Night of the Tiger'' in a box of old manuscripts \& thought it was still a perfectly respectable tale, albeit one obviously written by a guy who had only begun to learn his chops. I rewrote it \& on a whim resubmitted it to \textit{F\&SF}. This time they bought it. 1 thing I've noticed is that when you've had a little success, magazines are a lot less apt to use that phrase, ``Not for us.''

\fbox{\bf17} Although he was a year younger than his classmates, my big brother was bored with high school. Some of this had to do with his intellect -- Dave's IQ tested in the 150s or 160s -- but I think it was mostly his restless nature. For Dave, high school just wasn't super duper enough -- there was no pow, no wham, no \textit{fun}. He solved the problem, at least temporarily, by creating a newspaper which he called \textit{Dave's Rag}.

The \textit{Rag}'s office was a table located in the dirt-floored, rock-walled, spider-infested confines of our basement, somewhere north of the furnace \& east of the root-cellar, where Clayt \& Ella's endless cartons of preserves \& canned vegetables were kept. The \textit{Rag} was an odd combination of family newsletter \& small-town biweekly. Sometimes it was a monthly, if Dave got sidetracked by other interests (maple-sugaring, cider-making, rocket-building, \& car-customizing, just to name a few), \& then there would be jokes I didn't understand about how Dave's \textit{Rag} was a little late this month or how we shouldn't bother Dave, because he was down in the basement, on the \textit{Rag}.

Jokes or no jokes, circulation rose slowly from about 5 copies per issue (sold to nearby family members) to something like 50 or 60, with our relatives \& the relatives of neighbors in our small town (Durham's population in 1962 was about 900) eagerly awaiting each new edition. A typical number would let people know how Charley Harrington's broken leg was mending, what guest speakers might be coming to the West Durham Methodist Church, how much water the King boys were hauling from the town pump to keep from draining the well behind the house (of course it went dry every fucking summer no matter how much water we hauled), who was visiting the Browns or the Halls on the other side of Methodist Corners, \& whose relatives were due to hit town each summer. Dave also included sports, word-games, weather reports (``It's been pretty dry, but local farmer Harold Davis says if we don't have at least 1 good rain in August he will smile \& kiss a pig''), recipes, a continuing story (I wrote that), \& Dave's Jokes \& Humor, which included nuggets like these:

Stand: ``What did the beaver say to the oak tree?''

Jan: ``It was nice gnawing you!''

1st Beatnik: ``How do you get to Carnegie Hall?''

2nd Beatnik: ``Practice man practice!''

During the \textit{Rag}'s 1st year, the print was purple -- those issues were produced on a flat plate of jelly called a hectograph. My brother quickly decided the hectograph was a pain in the butt. It was just too slow for him. Even as a kid in short pants, Dave hated to be halted. Whenever Milt, our mom's boyfriend (``Sweeter than smart,'' Mom said to me 1 day a few months after she dropped him), got stuck in traffic or at a stoplight, Dave would lean over from the back seat of Milt's Buick \& yell, ``Drive over em, Uncle Milt! Drive over em!''

As a teenager, waiting for the hectograph to ``freshen'' between pages printed (while ``freshening,'' the print would melt into a vague purple membrane which hung in the jelly like a manatee's shadow) drove David all but insane with impatience. Also, he badly wanted to add photographs to the newspaper. He took good ones, \& by age 16 he was developing them, as well. He rigged a darkroom in a closet \& from its tiny, chemical-stinking confines produced pictures which were often startling in their clarity \& composition (the photo on the back of \textit{The Regulators}, showing me with a copy of the magazine containing my 1st published story, was taken by Dave with an old Kodak \& developed in his closet darkroom).

In addition to these frustrations, the flats of hectograph jelly had a tendency to incubate \& support colonies of strange, sporelike growths in the unsavory atmosphere of our basement, no matter how meticulous we were about covering the damned old slowcoach thing once the day's printing chores were done. What looked fairly ordinary on Monday sometimes looked like something out of an H.P. Lovecraft horror tale by the weekend.

In Brunswick, where he went to high school, Dave found a shop with a small drum printing press for sale. It worked -- barely. You typed up your copy on stencils which could be purchased in a local office-supply store for 19 cents apiece -- my brother called this chore ``cutting stencil,'' \& it was usually my job, as I was less prone to make typing errors. The stencils were attached to the drum of the press, lathered up with the world's stinkiest, oogiest ink, \& then you were off to the races -- crank `til your arm falls off, son. We were able to put together in 2 nights what had previously taken a week with the hectograph, \& while the drum-press was messy, it did not look inflected with a potentially fatal disease. \textit{Dave's Rag} entered its brief golden age.

\fbox{\bf18} I wasn't much interested in the printing process, \& I wasn't interested at all in the arcana of 1st developing \& then reproducing photographs. I didn't care about putting Hearst shifters in cars, making cider, or seeing if a certain formula would send a plastic rocket into the stratosphere (usually they didn't even make it over the house). What I cared about most between 1958 \& 1966 was movies.

As the 50s gave way to the 60s, there was only 2 movie theaters in the area, both in Lewiston. The Empire was the 1st-run house, showing Disney pictures, Bible epics, \& musicals in which widescreen ensembles of well-scrubbed folks danced \& sang. I went to these if I had a ride -- a movie was a movie, after all -- but I didn't like them very much. They were boringly wholesome. They were predictable. During \textit{The Parent Trap}, I kept hoping Hayley Mills would run into Vic Morrow from \textit{The Blackboard Jungle}. That would have livened things up a little, by God. I felt that 1 look at Vic's switchblade knife \& gimlet gaze would have put Hayley's piddling domestic problems in some kind of reasonable perspective. \& when I lay in bed at night under my eave, listening to the wind in the trees or the rats in the attic, it was not Debbie Reynolds as Tammy or Sandra Dee as Gidget that I dreamed of, but Yvette Vickers from \textit{Attack of the Giant Leeches} or Luana Anders from \textit{Dementia 13}. Never mind sweet; never mind uplifting; never mind Snow White \& the 7 Goddam Dwarfs. At 13 I wanted monsters that ate whole cities, radioactive corpses that came out of the ocean \& ate surfers, \& girls in black bras who looked like trailer trash.

Horror movies, science fiction movies, movies about teenage gangs on the prowl, movies about losers on motorcycles -- this was the stuff that turned my dials up to 10. The place to get all of this was not at the Empire, on the upper end of Lisbon Street, but at the Ritz, down at the lower end, amid the pawnshops \& not far from Louie's Clothing, where in 1964 I bought my 1st pair of Beatle boots. The distance from my house to the Ritz was 14 miles, \& I hitchhiked there almost every weekend during the 8 years between 1958 \& 1966, when I finally got my driver's license. Sometimes I went with my friend Chris Chesley, sometimes I went alone, but unless I was sick or something, I always went. It was at the Ritz that I saw \textit{I Married a Monster from Outer Space}. with Tom Tryon; \textit{The Haunting}, with Claire Bloom \& Julie Harris; \textit{The Wild Angels}, with Peter Fonda \& Nancy Sinatra. I saw Olivia de Havilland put out James Caan's eyes with makeshift knives in \textit{Lady in a Cage}, saw Joseph Cotton come back from the dead in \textit{Hush $\ldots$ Hush, Sweet Charlotte}, \& watched with held breath (\& not a little prurient interest) to see if Allison Hayes would grow all the way out of her clothes in \textit{Attack of the 50 Ft. Woman}. At the Ritz, all the finer things in life were available $\ldots$ or \textit{might be} available, if you only sat in the 3rd row, paid close attention, \& did not blink at the wrong moment.

Chris \& I liked just about any horror movie, but our faves were the string of American-International films, most directed by Roger Corman, with titles cribbed from Edgar Allan Poe. I wouldn't say \textit{based upon} the works of Edgar Allan Poe, because there is little in any of them which has anything to do with Poe's actual stories \& poems (\textit{The Raven} was filmed as a comedy -- no kidding). \& yet the best of them -- \textit{The Haunted Palace, The Conqueror Worm, The Masque of the Red Death} -- achieved a hallucinatory eeriness that made the special. Chris \& I had our own name for these films, one that made them into a separate genre. There were westerns, there were love stories, there were war stories $\ldots$ \& there were Poepictures.

``Wanna hitch to the show Saturday afternoon?'' Chris would ask. ``Go to the Ritz?''

``What's on?'' I'd ask.

``A motorcycle picture \& a Poepicture,'' he'd say. I, of course, was on that combo like white on rice. Bruce Dern going batshit on a Harley \& Vincent Price going batshit in a haunted castle overlooking a restless ocean: who could ask for me? You might even get Hazel Court wandering around in a lacy low-cut nightgown, if you were lucky.

Of all the Poepictures, the one that affected Chris \& me the most deeply was \textit{The Pit \& the Pendulum}. Written by Richard Matheson \& filmed in both widescreen \& Technicolor (color horror pictures were still a rarity in 1961, when this one came out), \textit{Pit} took a bunch of standard gothic ingredients \& turned them into something special. It might have been the last day really great studio horror picture before George Romero's ferocious indie \textit{The Night of the Living Dead} came along \& changed everything forever (in some few cases for the better, in most for the worse). The best scene -- the one which froze Chris \& me into our seats -- depicted John Kerr digging into a castle wall \& discovering the corpse of his sister, who was obviously buried alive. I have never forgotten the corpse's close-up, shot through a red filter \& a distorting lens which elongated the face into a huge silent scream.

On the long hitch home that night (if rides were slow in coming, you might end up walking 4 or 5 miles \& not get home until well after dark) I had a wonderful idea: I would turn \textit{The Pit \& the Pendulum} into a book! Would novelize it, as Monarch Books had novelized such undying film classics as \textit{Jack the Ripper, Gorgo}, \& Konga. But I wouldn't just write this masterpiece; I would also print it, using the drum-press in our basement, \& sell copies at school! Zap! Ka-pow!

As it was conceived, so was it done. Working with the care \& deliberation for which I would later be critically acclaimed, I turned out my ``novel version'' of \textit{The Pit \& the Pendulum} in 2 days, composing directly onto the stencils from which I'd print. Although no copies of that particular masterpiece survive (at least to my knowledge), I believe it was 8 pages long, each page single-spaced \& paragraph breaks kept to an absolute minimum (each stencil cost 19 cents, remember). I printed sheets on both sides, just as in a standard book, \& added a title page on which I drew a rudimentary pendulum dripping small black blotches which I hoped would look like blood. At the last moment I realized I had forgotten to identify the publishing house. After a half-hour or so of pleasant mulling, I typed the words \textbf{A V.I.B. BOOK} in the upper right corner of my title page. V.I.B. stood for Very Important Book.

I ran off about 40 copies of \textit{The Pit \& the Pendulum}, blissfully unaware that I was in violation of every plagiarism \& copyright statue in the history of the world; my thoughts were focused almost entirely on how much money I might make if my story was a hit at school. The stencils had cost me \$1.71 (having to use up 1 whole stencil for the title page seemed a hideous waste of money, but you had to look good, I'd reluctantly decided; you had to go out there with a bit of the old attitude), the paper cost another 2 bits or so, the staples were free, cribbed from my brother (you might have to paperclip stories you were sending out to magazines, but this was a \textit{book}, this was the bigtime). After some further thought, I priced V.I.B. \#1, \textit{The Pit \& the Pendulum} by Steve King, at a quarter a copy. I thought I might be able to sell 10 (my mother would buy 1 to get me started; she could always be counted on), \& that would add up to \$2.50. I'd make about 40 cents, which would be enough to finance another educational trip to the Ritz. If I sold 2 more, I could get a big sack of popcorn \& a Coke, as well.

\textit{The Pit \& the Pendulum} turned out to be my 1st best-seller. I took the entire print-run to school in my book-bag (in 1961 I would have been an 8th-grader at Durham's newly built 4-room elementary school), \& by noon that day I had sold 2 dozen. By the end of lunch hour, when word had gotten around about the lady buried in the wall (``They stared with horror at the bones sticking out from the ends of her fingers, realizing she had died scratcheing madley for escape''), I had sold 3 dozen. I had 9 dollars in change weighing down the bottom of my book-bag (upon which Durham's answer to Daddy Cool had carefully printed most of the lyrics to ``The Lion Sleeps Tonight'') \& was walking around in a kind of dream, unable to believe my sudden ascension to previously unsuspected realms of wealth. It all seemed too good to be true.

It was. When the school day ended at 2 o'clock, I was summoned to the principle's office, where I was told I couldn't turn the school into a marketplace, especially not, Miss Hisler said, to sell such trash as \textit{The Pit \& the Pendulum}. Her attitude didn't much surprise me. Miss Hisler had been the teacher at my previous school, the 1-roomer at Methodist Corners, where I went to the 5th \& 6th grades. During that time she had spied me reading a rather sensational ``teenage rumble'' novel (\textit{The Amboy Dukes}, by Irving Shulman), \& had taken it away. This was just more of the same, \& I was disgusted with myself for not seeing the outcome in advance. In those days we called someone who did an idiotic thing a dubber (pronounced \textit{dubba} if you were from Maine). I had just dubbed up bigtime.

``What I don't understand, Stevie,'' she said, ``is why you'd write junk like this in the 1st place. You're talented. Why do you want to waste your abilities?'' She had rolled up a copy of V.I.B. \#1 \& was brandishing it at me the way a person might brandish a rolled-up newspaper at a dog that has piddled on the rug. She waited for me to answer -- to her credit, the question was not entirely rhetorical -- but I had no answer to give. I was ashamed. I have spent a good many years since -- too many, I think -- being ashamed about what I write. I think I was 40 before I realized that almost every writer of fiction \& poetry who has ever published a line has been accused by someone of wasting his or her God-given talent. If you write (or paint or dance or sculpt or sing, I suppose), someone will try to make you feel lousy about it, that's all. I'm not editorializing, just trying to give you the facts as I see them.

Miss Hisler told me I would have to give everyone's money back. I did so with no argument, even to those kids (\& there were a quite a few, I'm happy to say) who insisted on keeping their copies of V.I.B. \#1. I ended up losing money on the deal after all, but when summer vacation came I printed 4 dozen copies of a new story, an original called \textit{The Invasion of the Star-Creatures}, \& sold all but 4 or 5. I guess that means I won in the end, at least in a financial sense. But in my heart I stayed ashamed. I kept hearing Miss Hisler asking why I wanted to waste my talent, why I wanted to waste my time, why I wanted to write junk.

\fbox{\bf19} Doing a serial story for \textit{Dave's Rag} was fun, but my other journalistic duties bored me. Still, I had worked for a newspaper of sorts, word got around, \& during my sophomore year at Lisbon High I became editor of our school newspaper, \textit{The Drum}. I don't recall being given any choice in this matter; I think I was simply appointed. My 2nd-in-command, Danny Emond, had even less interest in the paper than I did. Danny just liked the idea that Room 4, where we did our work, was near the girls' bathroom. ``Someday I'll just go crazy \& hack my way in there, Steve,;; he told me on more than 1 occasion. ``Hack, hack, hack.'' Once he added, perhaps in an effort to justify himself: ``The prettiest girls in school pull up their skirts in there.'' This struck me as so fundamentally stupid it might actually be wise, like a Zen koan or an early story by John Updike.

\textit{The Drum} did not prosper under my editorship. Then as now, I tend to go through periods of idleness followed by periods of workaholic frenzy. In the schoolyear 1963--1964. \textit{The Drum} published just 1 issue, but that one was a monster thicker than the Lisbon Falls telephone book. 1 night -- sick to death of Class Reports, Cheerleading Updates, \& some lamebrain's efforts to write a school poem -- I created a satiric high school newspaper of my own when I should have been captioning photographs for \textit{The Drum}. What resulted was a 4-sheet which I called \textit{The Village Vomit}. The boxed motto in the upper lefthand corner was not ``All the News That's Fit to Print'' but ``All the Shit That Will Stick.'' That piece of dimwit humor got me into the only real trouble of my high school career. It also led me to the most useful writing lesson I ever got.

In typical \textit{Mad} magazine style (``What, me worry?''), I filled the \textit{Vomit} with fictional tidbits about the LHS faculty, using teacher nicknames the student body would immediately recognize. Thus Miss Raypach, the study-hall monitor, became Miss Rat Pack; Mr. Ricker, the college-track English teacher (\& the school's most urbane faculty member -- he looked quite a bit like Craig Stevens in \textit{Peter Gunn}), became Cow Man because his family owned Ricker Dairy; Mr. Diehl, the earth-science teacher, became Old Raw Diehl.

As all sophomoric humorists must be, I was totally blown away by my own wit. What a funny fellow I as! A regular mill-town H. L. Mencken! I simply must take the \textit{Vomit} to school \& show all my friends! They would bust a collective gut!

As a matter of fact, they \textit{did} bust a collective gut; I had some good ideas about what tickled the funnybones of high school kids, \& most of them were showcased in \textit{The Village Vomit}. In 1 article, Cow Man's prize Jersey won a livestock farting contest at Topsham Fair; in another, Old Raw Diehl was fired for sticking the eyeballs of specimen fetal pigs up his nostrils. Humor in the grand Swiftian manner, you see. Pretty sophisticated, eh?

During period 4, 3 of my friends were laughing so hard in the back of study-hall that Miss Raypach (Rat Pack to you, chum) crept up on them to see what was so funny. She confiscated \textit{The Village Vomit}, on which I had, either out of overweening pride or almost unbelievable naivet\'e, put my name as Editor in Chief \& Grand High Poobah, \& at the close of school I was for the 2nd time in my student career summoned to the office on account of something I had written.

This time the trouble was a good deal more serious. Most of the teachers were inclined to be good sports about my teasing -- even Old Raw Diehl was willing to let bygones by bygones concerning the pigs' eyeballs -- but one was not. This was Miss Margitan, who taught shorthand \& typing to the girls in the business courses. She commanded both respect \& fear; in the tradition of teachers from an earlier era, Miss Margitan did not want to be your pal, your psychologist, or your inspiration. She was there to teach business skills, \& she wanted all learning to be done by the rules. \textit{Her} rules. Girls in Miss Margitan's classes were sometimes asked to kneel on the floor, \& if the hems of their skirts didn't touch the linoleum, they were sent home to change. No amount of tearful begging could soften her, no reasoning could modify her view of the world. Her detention lists were the longest of any teacher in the school, but her girls were routinely selected as valedictorians or salutatorians \& usually went on to good jobs. Many came to love her. Others loathed her then \& likely still do now, all these years later. These later girls called her ``Maggot'' Margitan, as their mothers had no doubt before them. \& in \textit{The Village Vomit} I had an item which began, ``Miss Margitan, known affectionately to Lisbonians everywhere as Maggot $\ldots$''

Mr. Higgins, our bald principal (breezily referred to in the \textit{Vomit} as Old Cue-Ball), told me that Miss Margitan had been very hurt \& very upset by what I had written. She was apparently not too hurt to remember that old scriptural admonition which goes ``Vengeance is mine, saith the shorthand teacher,'' however; Mr. Higgins said she wanted me suspended from school.

In my character, a kind of wildness \& a deep conservatism are wound together like hair in a braid. It was the crazy part of me that had 1st written \textit{The Village Vomit} \& then carried it to school; now that troublesome Mr. Hyde had dubbed up \& slunk out the back door. Dr. Jekyll was left to consider how my mom would look at me if she found out I had been suspended -- her hurt eyes. I had to put thoughts of her out of my mind, \& fast. I was a sophomore, I was a year older than most others in my class, \& at 6.2 feet I was 1 of the bigger boys in school. I desperately didn't want to cry in Mr. Higgin's office -- not with kids surging through the halls \& looking curiously in the window at us: Mr. Higgins behind his desk, me in the Bad Boy Seat.

In the end, Miss Margitan settled for a formal apology \& 2 weeks of detention for the bad boy who had dared call her Maggot in print. It was bad, but what in high school is not? At the time we're stuck in it, like hostages locked in a Turkish bath, high school seems the most serious business in the world to just about all of us. It's not until the 2nd or 3rd class reunion that we start realizing how absurd the whole thing was.

A day or 2 later I was ushered into Mr. Higgins's office \& made to stand in front of her. Miss Margitan sat ramrod-straight with her arthritic hands folded in her lap \& here gray eyes fixed unflinchingly on my face, \& I realized that something about her was different from any other adult I had ever met. I didn't pinpoint that difference at once, but I knew that there would be no charming this lady, no winning her over. Later, while I was flying paper planes with the other bad boys \& bad girls in detention hall (detention turned out to be not so bad), I decided that it was pretty simple: Miss Margitan didn't like boys. She was the 1st woman I ever met in my life who didn't like boys, not even 1 little bit.

If it makes any difference, my apology was heartfelt. Miss Margitan really had been hurt by what I wrote, \& that much I could understand. I doubt that she hated me -- she was probably too busy -- but she was the National Honor Society advisor at LHS, \& when my name showed up on the candidate list 2 years later, she vetoed me. The Honor Society did not need boys ``of his type,'' she said. I have come to believe she was right. A boy who once wiped his ass with poison ivy probably doesn't belong in a smart people's club.

I haven't trucked much with satire since then.

\fbox{\bf20} Hardly a week after being sprung from detention hall, I was once more invited to step down to the principal's office. I went with a sinking heart, wondering what new shit I'd stepped in.

It wasn't Mr. Higgins who wanted to see me, at least; this time the school guidance counselor had issued the summons. There had been discussions about me, he said, \& how to turn my ``restless pen'' into more constructive channels. He had enquired of John Gould, editor of Lisbon's weekly newspaper, \& had discovered Gould had an opening for a sports reporter. While the school couldn't \textit{insist} that I take this job, everyone in the front office felt it would be a good idea. \textit{Do it or die}, the G.C.'s eyes suggested. Maybe that was just paranoia, but even now, almost 40 years later, I don't think so.

I groaned inside. I was shut of \textit{Dave's Rag}, almost shut of \textit{The Drum}, \& now here was the Lisbon \textit{Weekly Enterprise}. Instead of being haunted by waters, like Norman Maclean in \textit{A River Runs Through It}, I was as a teenager haunted by newspapers. Still, what could I do? I rechecked the look in the guidance counselor's eyes \& said I would be delighted to interview for the job.

Gould -- not the well-known New England humorist or the novelist who wrote \textit{The Greenleaf Fires} but a relation of both, I think -- greeted me warily but with some interest. We would try each other out, he said, if that suited me.

Now that I was away from the administrative offices of Lisbon High, I felt able to muster a little honesty. I told Mr. Gould that I didn't know much about sports. Gould said, ``These are games people understand when they're watching them drunk in bars. You'll learn if you try.''

He gave me a huge roll of yellow paper on which to type me copy -- I think I still have it somewhere -- \& promised me a wage of half a cent a word. It was the 1st time someone had promised me wages for writing.

The 1st 2 pieces I turned in had to do with a basketball game in which an LHS player broke the school scoring record. One was a straight piece of reporting. The other was a sidebar about Robert Ransom's record-breaking performance. I brought both on Gould the day after the game so he'd have them for Friday, which was when the paper came out. He read the game piece, made 2 minor corrections, \& spiked it. Then he started in on the feature piece with a large black pen.

I took my fair share of English Lit classes in my 2 remaining years at Lisbon, \& my fair share of composition, fiction, \& poetry classes in college, but John Gould taught me more than any of them, \& in no more than 10 minutes. I wish I still had the piece -- it deserves to be framed, editorial corrections \& all -- but I can remember pretty well how it went \& how it looked after Gould had combed through it with that black pen of his. Here's an example: {\sf[excerpt]}

Gould stopped at ``the years of Korea'' \& looked up at me. ``What year was the last record made?'' he asked.

Luckily, I had my notes. ``1953,'' I said. Gould grunted \& went back to work. When he finished marking my copy in the manner indicated above, he looked up \& saw something on my face. I think he must have mistaken it for horror. It wasn't; it was pure revelation. Why, I wondered, didn't English teachers ever do this? It was like the Visible Man Old Raw Diehl had on his desk in the biology room.

``I only took out the bad parts, you know,'' Gould said. ``Most of it's pretty good.''

``I know,'' I said, meaning both things: yes, most of it was good -- okay anyway, serviceable -- \& yes, he had only taken out the bad parts. ``I won't do it again.''

He laughed. ``If that's true, you'll never have to work for a living. You can do \textit{this} instead. Do I have to explain any of these marks?''

``No,'' I said.

``When you write a story, you're telling yourself the story,'' he said. ``When you rewrite, your main job is taking out all the things that are \textit{not} the story.''

Gould said something else that was interesting on the day I turned in my 1st 2 pieces: write with the door closed, rewrite with the door open. Your stuff starts out being just for you, in other words, but then it goes out. Once you know what the story is \& get it right -- as right as you can, anyway -- it belongs to anyone who wants to read it. Or criticize it. If you're very lucky (this is my idea, not John Gould's, but I believe he would have subscribed to the notion), more will want to do the former than the latter.

\fbox{\bf21}

'' -- \cite[pp. 13--]{King2010}

%------------------------------------------------------------------------------%

\section{What Writing Is}
``Telepathy, of course. It's amusing when you stop to think about it -- for years people have argued about whether or not such a thing exists, folks like J. B. Rhine have busted their brains trying to create a valid testing process to isolate it, \& all the time it's been right there, lying out in the open like Mr. Poe's Purloined Letter. All the arts depend upon telepathy to some degree, but I believe that writing offers the purest distillation. Perhaps I'm prejudiced, but even if I am we may as well stick with writing, since it's what we came here to think \& talk about.

My name is Stephen King. I'm writing the 1st draft of this part at my desk (the one under the eave) on a snowy morning in December of 1997. There are things on my mind. Some are worries (bad eyes, Christmas shopping not even started, wife under the weather with a virus), some are good things (our younger son made a surprise visit home from college, I got to play Vince Taylor's ``Brand New Cadillac'' with The Wallflowers at a concert), but right now all that stuff is up top. I'm in another place, a basement place where there are lots of bright lights \& clear images. This is a place I've built for myself over the years. It's far-seeing place. I know it's a little strange, a little bit of a contradiction, that a far-seeing place should also be a basement place, but that's how it is with me. If you construct your own far-seeing place, you might put it in a treetop or on the roof of the World Trade Center or on the edge of the Grand Canyon. That's your little red wagon, as Robert McCammon says in 1 of his novels.

This book is scheduled to be published in the late summer or early fall of 2000. If that's how things work out, then you are somewhere downstream on the timeline from me $\ldots$ but you're quite likely in your own far-seeing place, the one where you go to receive telepathic messages. Not that you \textit{have} to be there; books are a uniquely portable magic. I usually listen to one in the car (always unabridged; I think abridged audiobooks are the pits), \& carry another wherever I go. You just never know when you'll want an escape hatch: mile-long lines at tollbooth plazas, the 15 minutes you have to spend in the hall of some boring college building waiting for your advisor (who's got some yank-off in there threatening to commit suicide because he\texttt{/}she is flunking Custom Kurmfurling 101) to come out so you can get his signature on a drop-card, airport boarding lounges, laundromats on rainy afternoons, \& the absolute worst, which is the doctor's office when the guy is running late \& you have to wait half an hour in order to have something sensitive mauled. At such times I find a book vital. If I have to spend time in purgatory before going to 1 place or the other, I guess I'll be all right as long as there's a lending library (if there is it's probably stocked with nothing but novels by Danielle Steel \& \textit{Chicken Soup} books, ha-ha, joke's on you, Steve).

So I read where I can, but I have a favorite place \& probably you do, too -- a place where the light is good \& the vibe is usually strong. For me it's the blue chair in my study. For you it might be the couch on the sun-porch, the rocker in the kitchen, or maybe it's propped up in your bed -- reading in bed can be heaven, assuming you can get just the right amount of light on the page \& aren't prone to spilling your coffee or cognac on the sheets.

So let's assume that you're in your favorite receiving place just as I am in the place where I do my best transmitting. We'll have to perform our mentalist routine not just over distance but over time as well, yet that presents no real problem; if we can still read Dickens, Shakespeare, \& (with the help of a footnote or 2) Herodotus, I think we can manage the gap between 1997 \& 2000. \& here we go -- actually telepathy in action. You'll notice I have nothing up my sleeves \& that my lips never move. Neither, most likely, do yours.

Look -- here's a table covered with a red cloth. On it is a cage the size of a small fish aquarium. In the cage is a white rabbit with a pink nose \& pink-rimmed eyes. In its front paws is a carrot-stub upon which it is contentedly munching. On its back, clearly marked in blue ink, is the numeral 8.

Do we see the same thing? We'd have to get together \& compare notes to make absolutely sure, but I think we do. There will be necessary variations, of course: some receivers will see a cloth which is turkey red, some will see one that's scarlet, while others may see still other shades. (To color-blind receivers, the red tablecloth is the dark gray of cigar ashes.) Some may see scalloped edges, some may see straight ones. Decorative souls may add a little lace, \& welcome -- my tablecloth is your tablecloth, knock yourself out.

Likewise, the matter of the cage leaves quite a lot of room for individual interpretation. For 1 thing, it is described in terms of \textit{rough comparison}, which is useful only if you \& I see the world \& measure the things in it with similar eyes. It's easy to become careless when making rough comparisons, but the alternative is a prissy attention to detail that takes all the fun out of writing. What am I going to say, ``on the table is a cage 3 feet, 6 inches in length, 2 feet in width, \& 14 inches high''? That's not prose, that's an instruction manual. The paragraph also doesn't tell us what sort of material the cage is made of -- wire mesh? steel rods? glass? -- but does it really matter? We all understand the cage is a see-through medium; beyond that, we don't care. The most interesting thing here isn't even the carrot-munching rabbit in the cage, but the number on its back. Not a 6, not a 4, not 19.5. It's an 8. This is what we're looking at, \& we all see it. I didn't tell you. You didn't ask me. I never opened my mouth \& you never opened yours. We're not even in the same \textit{year} together, let alone the same room $\ldots$ except we \textit{are} together. We're close.

We're having a meeting of the minds.

I sent you a table with a red cloth on it, a cage, a rabbit, \& the number 8 in blue ink. You got them all, especially that blue 8. We've engaged in an act of telepathy. No mythy-mountain shit; real telepathy. I'm not going to belabor the point, but before we go any further you have to understand that I'm not trying to be cute; there \textit{is} a point to be made.

You can approach the act of writing with nervousness, excitement, hopefulness, or even despair -- the sense that you can never completely put on the page what's in your mind \& heart. You can come to the act with your fists clenched \& your eyes narrowed, ready to kick ass \& take down names. You can come to it because you want a girl to marry you or because you want to change the world. Come to it any way but lightly. Let me say it again: \textit{you must not come lightly to the blank page}.

I'm not asking you to come reverently or unquestioningly; I'm not asking you to be politically correct or cast aside your sense of humor (please God you have one). This isn't a popularity contest, it's not the moral Olympics, \& it's not church. But it's \textit{writing}, damn it, not washing the car or putting an eyeliner. If you can take it seriously, we can do business. If you can't or won't, it's time for you to close the book \& do something else.

Wash the car, maybe.'' -- \cite[pp. 83--85]{King2010}

%------------------------------------------------------------------------------%

\section{TOOLBOX}

%------------------------------------------------------------------------------%

\section{ON WRITING}

%------------------------------------------------------------------------------%

\section{ON LIVING: A POSTSCRIPT}

%------------------------------------------------------------------------------%

\section{\& Furthermore, Part I: Door Shut, Door Open}

%------------------------------------------------------------------------------%

\section{\& Furthermore, Part II: A Booklist}

%------------------------------------------------------------------------------%

\section{Further to Furthermore, Part III}

%------------------------------------------------------------------------------%

\section{About Stephen King}

%------------------------------------------------------------------------------%

\printbibliography[heading=bibintoc]
	
\end{document}