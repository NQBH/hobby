\documentclass{article}
\usepackage[backend=biber,natbib=true,style=authoryear]{biblatex}
\addbibresource{/home/nqbh/reference/bib.bib}
\usepackage{tocloft}
\renewcommand{\cftsecleader}{\cftdotfill{\cftdotsep}}
\usepackage[colorlinks=true,linkcolor=blue,urlcolor=red,citecolor=magenta]{hyperref}
\usepackage{algorithm,algpseudocode,amsmath,amssymb,amsthm,float,graphicx,mathtools}
\allowdisplaybreaks
\numberwithin{equation}{section}
\newtheorem{assumption}{Assumption}[section]
\newtheorem{conjecture}{Conjecture}[section]
\newtheorem{corollary}{Corollary}[section]
\newtheorem{definition}{Definition}[section]
\newtheorem{example}{Example}[section]
\newtheorem{lemma}{Lemma}[section]
\newtheorem{notation}{Notation}[section]
\newtheorem{principle}{Principle}[section]
\newtheorem{problem}{Problem}[section]
\newtheorem{proposition}{Proposition}[section]
\newtheorem{question}{Question}[section]
\newtheorem{remark}{Remark}[section]
\newtheorem{theorem}{Theorem}[section]
\usepackage[left=0.5in,right=0.5in,top=1.5cm,bottom=1.5cm]{geometry}
\usepackage{fancyhdr}
\pagestyle{fancy}
\fancyhf{}
\lhead{\small Sect.~\thesection}
\rhead{\small\nouppercase{\leftmark}}
\renewcommand{\sectionmark}[1]{\markboth{#1}{}}
\cfoot{\thepage}
\def\labelitemii{$\circ$}

\title{On Writing: A Memoir of the Craft}
\author{Stephen King}
\date{\today}

\begin{document}
\maketitle
\tableofcontents

%------------------------------------------------------------------------------%

\section*{1st Foreword}

%------------------------------------------------------------------------------%

\section*{2nd Foreword}

%------------------------------------------------------------------------------%

\section*{3rd Foreword}

%------------------------------------------------------------------------------%

\section{C.V.}

%------------------------------------------------------------------------------%

\section{What Writing Is}
``Telepathy, of course. It's amusing when you stop to think about it -- for years people have argued about whether or not such a thing exists, folks like J. B. Rhine have busted their brains trying to create a valid testing process to isolate it, \& all the time it's been right there, lying out in the open like Mr. Poe's Purloined Letter. All the arts depend upon telepathy to some degree, but I believe that writing offers the purest distillation. Perhaps I'm prejudiced, but even if I am we may as well stick with writing, since it's what we came here to think \& talk about.

My name is Stephen King. I'm writing the 1st draft of this part at my desk (the one under the eave) on a snowy morning in December of 1997. There are things on my mind. Some are worries (bad eyes, Christmas shopping not even started, wife under the weather with a virus), some are good things (our younger son made a surprise visit home from college, I got to play Vince Taylor's ``Brand New Cadillac'' with The Wallflowers at a concert), but right now all that stuff is up top. I'm in another place, a basement place where there are lots of bright lights \& clear images. This is a place I've built for myself over the years. It's far-seeing place. I know it's a little strange, a little bit of a contradiction, that a far-seeing place should also be a basement place, but that's how it is with me. If you construct your own far-seeing place, you might put it in a treetop or on the roof of the World Trade Center or on the edge of the Grand Canyon. That's your little red wagon, as Robert McCammon says in 1 of his novels.

This book is scheduled to be published in the late summer or early fall of 2000. If that's how things work out, then you are somewhere downstream on the timeline from me $\ldots$ but you're quite likely in your own far-seeing place, the one where you go to receive telepathic messages. Not that you \textit{have} to be there; books are a uniquely portable magic. I usually listen to one in the car (always unabridged; I think abridged audiobooks are the pits), \& carry another wherever I go. You just never know when you'll want an escape hatch: mile-long lines at tollbooth plazas, the 15 minutes you have to spend in the hall of some boring college building waiting for your advisor (who's got some yank-off in there threatening to commit suicide because he\texttt{/}she is flunking Custom Kurmfurling 101) to come out so you can get his signature on a drop-card, airport boarding lounges, laundromats on rainy afternoons, \& the absolute worst, which is the doctor's office when the guy is running late \& you have to wait half an hour in order to have something sensitive mauled. At such times I find a book vital. If I have to spend time in purgatory before going to 1 place or the other, I guess I'll be all right as long as there's a lending library (if there is it's probably stocked with nothing but novels by Danielle Steel \& \textit{Chicken Soup} books, ha-ha, joke's on you, Steve).

So I read where I can, but I have a favorite place \& probably you do, too -- a place where the light is good \& the vibe is usually strong. For me it's the blue chair in my study. For you it might be the couch on the sun-porch, the rocker in the kitchen, or maybe it's propped up in your bed -- reading in bed can be heaven, assuming you can get just the right amount of light on the page \& aren't prone to spilling your coffee or cognac on the sheets.

So let's assume that you're in your favorite receiving place just as I am in the place where I do my best transmitting. We'll have to perform our mentalist routine not just over distance but over time as well, yet that presents no real problem; if we can still read Dickens, Shakespeare, \& (with the help of a footnote or 2) Herodotus, I think we can manage the gap between 1997 \& 2000. \& here we go -- actually telepathy in action. You'll notice I have nothing up my sleeves \& that my lips never move. Neither, most likely, do yours.

Look -- here's a table covered with a red cloth. On it is a cage the size of a small fish aquarium. In the cage is a white rabbit with a pink nose \& pink-rimmed eyes. In its front paws is a carrot-stub upon which it is contentedly munching. On its back, clearly marked in blue ink, is the numeral 8.

Do we see the same thing? We'd have to get together \& compare notes to make absolutely sure, but I think we do. There will be necessary variations, of course: some receivers will see a cloth which is turkey red, some will see one that's scarlet, while others may see still other shades. (To color-blind receivers, the red tablecloth is the dark gray of cigar ashes.) Some may see scalloped edges, some may see straight ones. Decorative souls may add a little lace, \& welcome -- my tablecloth is your tablecloth, knock yourself out.

Likewise, the matter of the cage leaves quite a lot of room for individual interpretation. For 1 thing, it is described in terms of \textit{rough comparison}, which is useful only if you \& I see the world \& measure the things in it with similar eyes. It's easy to become careless when making rough comparisons, but the alternative is a prissy attention to detail that takes all the fun out of writing. What am I going to say, ``on the table is a cage 3 feet, 6 inches in length, 2 feet in width, \& 14 inches high''? That's not prose, that's an instruction manual. The paragraph also doesn't tell us what sort of material the cage is made of -- wire mesh? steel rods? glass? -- but does it really matter? We all understand the cage is a see-through medium; beyond that, we don't care. The most interesting thing here isn't even the carrot-munching rabbit in the cage, but the number on its back. Not a 6, not a 4, not 19.5. It's an 8. This is what we're looking at, \& we all see it. I didn't tell you. You didn't ask me. I never opened my mouth \& you never opened yours. We're not even in the same \textit{year} together, let alone the same room $\ldots$ except we \textit{are} together. We're close.

We're having a meeting of the minds.

I sent you a table with a red cloth on it, a cage, a rabbit, \& the number 8 in blue ink. You got them all, especially that blue 8. We've engaged in an act of telepathy. No mythy-mountain shit; real telepathy. I'm not going to belabor the point, but before we go any further you have to understand that I'm not trying to be cute; there \textit{is} a point to be made.

You can approach the act of writing with nervousness, excitement, hopefulness, or even despair -- the sense that you can never completely put on the page what's in your mind \& heart. You can come to the act with your fists clenched \& your eyes narrowed, ready to kick ass \& take down names. You can come to it because you want a girl to marry you or because you want to change the world. Come to it any way but lightly. Let me say it again: \textit{you must not come lightly to the blank page}.

I'm not asking you to come reverently or unquestioningly; I'm not asking you to be politically correct or cast aside your sense of humor (please God you have one). This isn't a popularity contest, it's not the moral Olympics, \& it's not church. But it's \textit{writing}, damn it, not washing the car or putting an eyeliner. If you can take it seriously, we can do business. If you can't or won't, it's time for you to close the book \& do something else.

Wash the car, maybe.'' -- \cite[pp. 83--85]{King2010}

%------------------------------------------------------------------------------%

\section{TOOLBOX}

%------------------------------------------------------------------------------%

\section{ON WRITING}

%------------------------------------------------------------------------------%

\section{ON LIVING: A POSTSCRIPT}

%------------------------------------------------------------------------------%

\section{\& Furthermore, Part I: Door Shut, Door Open}

%------------------------------------------------------------------------------%

\section{\& Furthermore, Part II: A Booklist}

%------------------------------------------------------------------------------%

\section{Further to Furthermore, Part III}

%------------------------------------------------------------------------------%

\section{About Stephen King}

%------------------------------------------------------------------------------%

\printbibliography[heading=bibintoc]
	
\end{document}