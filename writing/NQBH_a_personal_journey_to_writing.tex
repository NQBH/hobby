\documentclass[oneside]{book}
\usepackage[backend=biber,natbib=true,style=authoryear]{biblatex}
\addbibresource{/home/hong/1_NQBH/reference/bib.bib}
\usepackage[vietnamese,english]{babel}
\usepackage{tocloft}
\renewcommand{\cftsecleader}{\cftdotfill{\cftdotsep}}
\usepackage[colorlinks=true,linkcolor=blue,urlcolor=red,citecolor=magenta]{hyperref}
\usepackage{amsmath,amssymb,amsthm,mathtools,float,graphicx}
\allowdisplaybreaks
\numberwithin{equation}{section}
\newtheorem{assumption}{Assumption}[chapter]
\newtheorem{lemma}{Lemma}[chapter]
\newtheorem{corollary}{Corollary}[chapter]
\newtheorem{definition}{Definition}[chapter]
\newtheorem{proposition}{Proposition}[chapter]
\newtheorem{theorem}{Theorem}[chapter]
\newtheorem{notation}{Notation}[chapter]
\newtheorem{remark}{Remark}[chapter]
\newtheorem{example}{Example}[chapter]
\newtheorem{question}{Question}[chapter]
\newtheorem{problem}{Problem}[chapter]
\newtheorem{conjecture}{Conjecture}[chapter]
\usepackage[left=0.5in,right=0.5in,top=1.5cm,bottom=1.5cm]{geometry}
\usepackage{fancyhdr}
\pagestyle{fancy}
\fancyhf{}
\lhead{\small \textsc{Sect.} ~\thesection}
\rhead{\small \nouppercase{\leftmark}}
\renewcommand{\sectionmark}[1]{\markboth{#1}{}}
\cfoot{\thepage}
\def\labelitemii{$\circ$}

\title{A Personal Journey to Writing}
\author{\selectlanguage{vietnamese} Nguyễn Quản Bá Hồng}
\date{\today}

\begin{document}
\maketitle
\tableofcontents

%---------------------------------------------------------------------- --------%

\chapter*{Foreword}
My personal journey to ``The Garden of Words''\footnote{\href{https://www.imdb.com/title/tt2591814/}{IMDb\texttt{/}\textsc{The Garden of Words} (2013)}, original title: \textsc{Koto no ha no niwa}.} -- the world of writings. \textit{Why writing?} Because: Instead of provoking a weak \& poor defense mechanism passively \& unconsciously, you should make your verbal enemies take a step back 1st: Words are weapons.

%---------------------------------------------------------------------- --------%

\section{Dictionary}
To read \& write well, the 1st concern is, obviously, to choose the right dictionary\texttt{/}dictionaries.

\begin{question}
	Which dictionary\emph{\texttt{/}}dictionaries should I use?
\end{question}

\begin{itemize}
	\item \href{https://dictionary.cambridge.org/}{Cambridge Dictionary}: ``Make your words meaningful''
	\item \href{https://www.collinsdictionary.com/us/}{Collins Dictionary}
	\item \href{https://www.merriam-webster.com/}{Merriam-Webster Dictionary}
	\item \href{https://www.oxfordlearnersdictionaries.com/}{Oxford Learner's Dicitonaries}
\end{itemize}
I choose \href{https://www.oxfordlearnersdictionaries.com/}{Oxford Learner's Dicitonaries}. Then the next question is:

\begin{question}
	Should I buy Oxford Learner's Dictionary of Academic English?
\end{question}
\pounds 5.5\texttt{/}year though. Bought: Seem worth it (?).

\begin{remark}[Personal style]
	I do not like to write the term ``and'', \selectlanguage{vietnamese}``và'', or ``or'', ``hoặc''. I write the symbols ``\&'' \& ``\emph{\texttt{/}}'', respectively, instead.
\end{remark}

%---------------------------------------------------------------------- --------%

\chapter{Literary Writings}

\section{Linguistics}
See, e.g., \href{https://en.wikipedia.org/wiki/Linguistics}{Wikipedia\texttt{/}linguistics}\footnote{\textbf{linguistics} [n] [uncountable] the scientific study of language or of particular languages.}.

\section{The Elements of Style}

\paragraph*{Content.} See \href{https://en.wikipedia.org/wiki/The_Elements_of_Style}{Wikipedia\texttt{/}The Elements of Style}. ``Strunk concentrated on the cultivation of good writing \& composition; the original 1918 edition exhorted writers to ``omit needless words'', use the \href{https://en.wikipedia.org/wiki/Active_voice}{active voice}, \& employ \href{https://en.wikipedia.org/wiki/Parallelism_(grammar)}{parallelism} appropriately.'' [$\ldots$] ``The 3rd edition of \textit{The Elements of Style} (1979) features 54 points: a list of common word-usage errors; 11 rules of punctuation \& grammar; 11 principles of writing; 11 matters of form; \&, in Chap. V, 21 reminders for better style. The final reminder, the 21st, ``Prefer the standard to the offbeat\footnote{\textbf{offbeat} [a] [usually before noun] (\textit{informal}) different from what most people expect, \textsc{synonym}: \textbf{unconventional}.}'', is thematically integral\footnote{\textbf{integral} [a] \textbf{1.} being an essential part of something; \textbf{2.} [usually before noun] included as part of something, rather than supplied separately; \textbf{3.} [usually before noun] having all the parts that are necessary for something to be complete.} to the subject of \textit{The Elements of Style}, yet does stand as a discrete\footnote{\textbf{discrete} [a] (\textit{formal or specialist}) independent of other things of the same type, \textsc{synonym}: \textbf{separate}.} essay about writing lucid\footnote{\textbf{lucid} [a] \textbf{1.} clearly expressed; easy to understand, \textsc{synonym}: clear; \textbf{2.} able to think clearly, especially when somebody cannot usually do this.} prose\footnote{\textbf{prose} [n] [uncountable] writing that is not poetry.}. To write well, White advises writers to have the proper\footnote{\textbf{proper} [a] \textbf{1.} [only before noun] (\textit{especially British English}) right, appropriate or correct; according to the rules, \textsc{opposite}: \textbf{improper}; \textbf{2.} [only before noun] \textit{British English}) considered to be real \& of a good enough standard; \textbf{3.} socially \& morally acceptable, \textsc{opposite}: \textbf{improper}; \textbf{4.} [after noun] according to the most exact meaning of the word; \textbf{5.} \textbf{proper to somebody\texttt{/}something} belonging to a particular type of person or thing; natural in a particular situation or place.} mind-set, that they write to please themselves, \& that they aim for ``1 moment of felicity\footnote{\textbf{felicity} [n] \textbf{1.} [uncountable] great happiness; \textbf{2.} [uncountable] the quality of being well chosen or suitable; \textbf{3.} \textbf{felicities} [plural] well-chosen or successful features, especially in a speech or piece of writing.}'', a phrase by \href{https://en.wikipedia.org/wiki/Robert_Louis_Stevenson}{Robert Louis Stevenson}. Thus Strunk's 1918 recommendation:
\begin{quotation}
	``Vigorous\footnote{\textbf{vigorous} [a] \textbf{1.} involving physical strength, effort or energy; \textbf{2.} done with determination, energy or enthusiasm; \textbf{3.} strong \& healthy.} writing is concise\footnote{\textbf{concise} [a] giving only the information that is necessary \& important, using few words.}. A sentence should contain no unnecessary words, a paragraph no unnecessary sentences, for the same reason that a drawing should have no unnecessary lines \& a machine no unnecessary parts. This requires not that the writer make all his sentences short, or that he avoid all detail \& treat his subjects only in outline, but that he make every word tell.'' -- ``Elementary Principles of Composition'', \textit{The Element of Style} \cite{Strunk1918}''
\end{quotation}
[$\ldots$] ``The 4th edition of \textit{The Elements of Style} (2000), published 54 years after Strunk's death, omits his stylistic\footnote{\textbf{stylistic} [a] [only before noun] connected with the style that a writer, artist or musician uses.} advice about masculine\footnote{\textbf{masculine} [a] \textbf{1.} having the qualities or appearance considered to be typical of men; connected with or like men; \textbf{2.} (in some languages) belonging to a class of nouns, pronouns or adjectives that have masculine gender, not feminine or neuter.} pronouns: ``unless the antecedent\footnote{\textbf{antecedent} [n] a thing or an event that exists or comes before something else \& has an influence on it; [a] existing or coming before something else, \& having an influence on it.} is or must be feminine''. In its place, the following sentence has been added: ``many writers find the use of the generic \textit{he} or \textit{his} to rename indefinite antecedents limiting or offensive.'' Further, the retitled entry ``They. He or she'', in Chap. IV: \textit{Misused Words \& Expressions}, advises the writer to avoid an ``unintentional emphasis on the masculine''.'' -- \href{https://en.wikipedia.org/wiki/The_Elements_of_Style#Content}{Wikipedia\texttt{/}The Element of Style\texttt{/}content}

\paragraph*{Reception.} ``\textit{The Elements of Style} was listed as 1 of the 100 best \& most influential\footnote{\textbf{influential} [a] having a lot of influence on the way that somebody\texttt{/}something behaves or develops, or on the way that somebody thinks.} books written in English since 1923 by \textit{Time} in its 2011 list. Upon its release, Charles Poor, writing for \href{https://en.wikipedia.org/wiki/The_New_York_Times}{\textit{The New York Times}}, called it ``a splendid\footnote{\textbf{splendid} [a] (\textit{especially British English}) \textbf{1.} very impressive; very beautiful; \textbf{2.} (\textit{old-fashioned}) excellent; very good, \textsc{synonym}: great.} trophy for all who are interested in reading \& writing.'' American poet \href{https://en.wikipedia.org/wiki/Dorothy_Parker}{Dorothy Parker} has, regarding the book, said:
\begin{quotation}
	``If you have any young friends who aspire to become writers, the 2nd-greatest favor you can do them is to present them with copies of \textit{The Elements of Style}. The 1st-greatest, of course, is to shoot them now, while they're happy.''
\end{quotation}
Criticism\footnote{\textbf{criticism} [n] \textbf{1.} [uncountable, countable] the act of expressing disapproval of somebody\texttt{/}something \& opinions about their faults or bad qualities; a statement showing disapproval; \textbf{2.} [uncountable] the work or activity of analyzing \& making fair, careful judgments about somebody\texttt{/}something, especially books, music, etc.} of \textit{Strunk \& White} has largely focused on claims that it has a \href{https://en.wikipedia.org/wiki/Linguistic_prescriptivism}{prescriptivist}\footnote{\textbf{prescriptive} [a] \textbf{1.} telling people what should be done or how something should be done; \textbf{2.} (\textit{linguistics}) telling people how a language should be used, rather than describing how it is used, \textsc{opposite}: \textbf{descriptive}.} nature, or that it has become a general \href{https://en.wikipedia.org/wiki/Anachronism}{anachronism}\footnote{\textbf{anachronism} [n] \textbf{1.} [countable] a person, a custom or an idea that seems old-fashioned \& does not belong to the present; \textbf{2.} [countable, uncountable] something that is placed, e.g., in a book or play, in the wrong period of history; the fact of placing something in the wrong period of history.} in the face of modern English usage.

In criticizing \textit{The Elements of Style}, \href{https://en.wikipedia.org/wiki/Geoffrey_Pullum}{Geoffrey Pullum}, professor of \href{https://en.wikipedia.org/wiki/Linguistics}{linguistics} at the \href{https://en.wikipedia.org/wiki/University_of_Edinburgh}{University of Edinburgh}, \& co-author of \href{https://en.wikipedia.org/wiki/The_Cambridge_Grammar_of_the_English_Language}{\textit{The Cambridge Grammar of the English Language}} (2002), said that:
\begin{quotation}
	``The book's toxic mix of \href{https://en.wikipedia.org/wiki/Linguistic_purism}{purism}\footnote{\textbf{purism} [n] [uncountable] the belief that things should be done in the traditional way \& that there are correct forms in languages, art, etc. that should be followed.}, \href{https://en.wikipedia.org/wiki/Atavism}{atavism}, \& personal \href{https://en.wikipedia.org/wiki/Eccentricity_(behavior)}{eccentricity}\footnote{\textbf{eccentricity} [n] \textbf{1.} [uncountable] behavior that people think is strange or unusual; the quality of being unusual \& different from other people; \textbf{2.} [countable, usually plural] an unusual act or habit.} is not underpinned\footnote{\textbf{underpin} [v] to support or form the basis of something.} by a proper grounding\footnote{\textbf{grounding} [n] [singular, uncountable] knowledge \& understanding of the basic parts of a subject; a basis for something.} in English grammar. It is often so misguided that the authors appear not to notice their own egregious\footnote{\textbf{egregious} [a] (\textit{formal}) extremely bad.} flouting\footnote{\textbf{flout} [v] \textbf{flout something} to show that you have no respect for a law, etc. by openly not obeying it, \textsc{synonym}: \textbf{defy}.} of its own rules $\ldots$ It's sad. Several generations of college students learned their grammar from the uninformed\footnote{\textbf{uninformed} [a] having or showing a lack of knowledge or information about something, \textsc{opposite}: informed.} bossiness\footnote{\textbf{bossiness} [n] [uncountable] (\textit{disapproving}) bossy behavior.} of \textit{Strunk \& White}, \& the result is a nation of educated people who know they feel vaguely\footnote{\textbf{vaguely} [adv] \textbf{1.} in a way that is not detailed or exact; \textbf{2.} slightly.} anxious\footnote{\textbf{anxious} [a] \textbf{1.} \textbf{anxious (about something)} feeling worried or nervous; \textbf{2.} wanting something very much.} \& insecure\footnote{\textbf{insecure} [a] \textbf{1.} not confident, especially about yourself or your abilities, \textsc{opposite}: \textbf{secure}; \textbf{2.} not safe or protected, \textsc{opposite}: \textbf{secure}.} whenever they write \textit{however} or \textit{than me} or \textit{was} or \textit{which}, but can't tell you why.''
\end{quotation}
Pullum has argued, e.g., that the authors misunderstood what constitutes the \href{https://en.wikipedia.org/wiki/English_passive_voice}{passive voice}\footnote{NQBH: Personally, I prefer the passive voice to the active one.}, \& he criticized their proscription\footnote{\textbf{proscription} [n] [countable, uncountable] (\textit{formal}) \textbf{proscription (against\texttt{/}on something)} the act of saying officially that something is banned; the stat of being banned.} of established \& unproblematic\footnote{\textbf{unproblematic} [a] not having or causing problems, \textsc{opposite}: \textbf{problematic}.} English usages, e.g. the \href{https://en.wikipedia.org/wiki/Split_infinitive}{split infinitive} \& the use of \textit{which} in a restrictive \href{https://en.wikipedia.org/wiki/English_relative_clause#That_or_which}{relative clause}. On \href{https://en.wikipedia.org/wiki/Language_Log}{Language Log}, a blog about language written by \href{https://en.wikipedia.org/wiki/Linguists}{linguists}, he further criticized \textit{The Elements of Style} for promoting \href{https://en.wikipedia.org/wiki/Linguistic_prescriptivism}{linguistic precriptivism} \& \href{https://en.wikipedia.org/wiki/Hypercorrection}{hypercorrection} among \href{https://en.wikipedia.org/wiki/Anglophones}{Anglophones}, \& called it ``the book that ate American's brain''.

\href{https://en.wikipedia.org/wiki/The_Boston_Globe}{\textit{The Boston Globe}}'s review described \textit{The Elements of Style Illustrated} (2005), with illustrations by Maira Kalman, as an ``aging zombie of a book $\ldots$ a hodgepodge\footnote{\textbf{hodgepodge} [n] (\textit{North American English}) (also \textbf{hotchpotch}, \textit{especially in British English}) [singular] (\textit{informal}) a number of things mixed together without any particular order or reason.}, its now-antiquated\footnote{\textbf{antiquated} [a] (\textit{usually disapproving}) (of things or ideas) old-fashioned \& no longer suitable for modern conditions, \textsc{synonym}: \textbf{outdated}.} \href{https://en.wikipedia.org/wiki/Pet_peeve}{pet peeves} jostling for\footnote{\textbf{jostle for} [phrasal verb] \textbf{jostle for something} to compete strongly \& with force for something.} space with 1970s taboos\footnote{\textbf{taboo} [n] \textbf{1.} \textbf{taboo (against\texttt{/}on something)} a cultural or religious custom that does not allow people to do, use or talk about a particular thing; \textbf{2.} \textbf{taboo (against\texttt{/}on something)} a general agreement not to do something or talk about something.} \& 1990s computer advice''.

Nevertheless, many contemporary\footnote{\textbf{contemporary} [a] \textbf{1.} belonging to the present time, \textsc{synonym} \textbf{modern}; \textbf{2.} (especially of people \& society) belonging to the same time as somebody\texttt{/}something else.} authors still recommend it highly. Their praise\footnote{\textbf{praise} [v] \textbf{1.} to express your approval or admiration for somebody\texttt{/}something; \textbf{2.} \textbf{praise God} to express your thanks to or your respect for God.} tends to focus on its characterization\footnote{\textbf{characterization} [n] [uncountable, countable] \textbf{1.} \textbf{characterization (of something)} the process of discovering or describing the qualities or features of something; the result of this process; \textbf{2.} the way in which the characters in a story, play or film are made to seem real.} of \fbox{good writing \& how to achieve it}, grammar being just 1 element of that purpose. In \href{https://en.wikipedia.org/wiki/On_Writing:_A_Memoir_of_the_Craft}{On writing} (2000, p. 11), \href{https://en.wikipedia.org/wiki/Stephen_King}{Stephen King} writes:
\begin{quotation}
	``There is little or no detectable \href{https://en.wikipedia.org/wiki/Bullshit}{bullshit} in that book. (Of course, it's short; at 85 pages it's much shorter than this one.) I'll tell you right now that every aspiring writer should read \textit{The Elements of Style}. Rule 17 in the chapter titled \textit{Principles of Composition} is `Omit needless words.' I will try to do that here.''
\end{quotation}
In 2011, Tim Skern remarked that \textit{The Elements of Style} ``remains the best book available on writing good English.''

In 2013, \href{https://en.wikipedia.org/wiki/Nevile_Gwynne}{Nevile Gwynne} reproduced \textit{The Elements of Style} in his work \href{https://en.wikipedia.org/wiki/Gwynne%27s_Grammar}{\textit{Gwynne's Grammar}}. Britt Peterson of the \href{https://en.wikipedia.org/wiki/Boston_Globe}{\textit{Boston Globe}} wrote that his inclusion of the book was a ``curious\footnote{\textbf{curious} [a] \textbf{1.} having a strong desire to know about something; \textbf{2.} strange \& unusual.} addition''.

In 2016, the Open Syllabus Project lists \textit{The Elements of Style} as the most frequently assigned text in US academic \href{https://en.wikipedia.org/wiki/Syllabus}{syllabuses}, based on an analysis of 933,635 texts appearing in over 1 million syllabuses.'' -- \href{https://en.wikipedia.org/wiki/The_Elements_of_Style#Reception}{Wikipedia\texttt{/}The Elements of Style\texttt{/}reception}

``The 1st writer I watched at work was my stepfather, E. B. White.\footnote{\selectlanguage{vietnamese} Sự ảnh hưởng, đặc biệt đến nhân cách \& việc lựa chọn nghề nghiệp, của những hình mẫu đầu tiên mà ta, 1 cách tình cờ hay được số phận sắp đặt, gặp gỡ trong cuộc đời.} Each Tuesday morning, he would close his study door \& sit down to write the ``Notes \& Comment'' page for \textit{The New Yorker}. The task was familiar to him -- he was required to file a few hundred words of editorial\footnote{\textbf{editorial} [a] [usually before noun] connected with the task of preparing something e.g. a newspaper, a book, or a television or radio programme, to be published or broadcast; [n] an important article in a journal or a newspaper, that expresses the editor's opinion about an issue.} of personal commentary on some topic in or out of the news that week -- but the sounds of his typewriter\footnote{\textbf{typewriter} [n] a machine that produces writing similar to print. It has keys that you press to make metal letters or signs hit a piece of paper through a long, narrow piece of cloth covered with ink ($=$ colored liquid).} \footnote{NQBH: I like the term ``typewriter'' in any literary scene., which sounds traditional \& sexy, opposite to personal notebooks\texttt{/}laptop now: modern \& robust.} from his room came in hesitant\footnote{\textbf{hesitant} [a] slow to speak or act because you feel uncertain, embarrassed or unwilling.} bursts\footnote{\textbf{burst} [v] \textbf{1.} [intransitive, transitive] to break open or apart, especially because of pressure from inside; to make something break in this way; \textbf{2.} [intransitive] \textbf{$+$ adv.\texttt{/}prep.} to go or come from somewhere suddenly; \textbf{burst into something} [phrasal verb] to start producing something suddenly \& with great force; [n] a short period of a particular activity or strong emotion that often starts suddenly.}, with long silences in between. Hours went by. Summoned at last for lunch, he was silent \& preoccupied\footnote{\textbf{preoccupied} [a] thinking \&\texttt{/}or worrying continuously about something so that you do not pay attention to other things.}, \& soon excused himself to get back to the job. When the copy went off at last, in the afternoon RFD pouch\footnote{\textbf{pouch} [n] \textbf{1.} a small bag, usually made of leather, \& often carried in a pocket or attached to a belt; \textbf{2.} a large bag for carrying letters, especially official ones; \textbf{3.} a pocket of skin on the stomach of some female marsupial animals, e.g. kangaroos, in which they carry their young; \textbf{4.} a pocket of skin in the cheeks of some animals, e.g. hamsters, in which they store food.} -- we were in Maine, a day's mail away from New York -- he rarely seemed satisfied. \fbox{``It isn't good enough.''}\footnote{``The quest for perfection can never end.''} he said sometimes, \fbox{``I wish it were better.''}

\fbox{Writing is hard}, even for authors who do it all the time. Less frequent practitioners -- the job applicant; the business executive with an annual report to get out; the high school senior with a Faulkner assignment; the graduate-school student with her thesis proposal; the writer of a letter of condolence\footnote{\textbf{condolence} [n] [countable, usually plural, uncountable] sympathy that you feel for somebody when a person in their family or that they know well has died; an expression of this sympathy.} -- often get stuck in an awkward\footnote{\textbf{awkward} [a] \textbf{1.} embarrassed; making you feel embarrassed; \textbf{2.} difficult to deal with, \textsc{synonym}: \textbf{difficult}; \textbf{3.} not convenient; \textbf{4.} difficult because of its shape or design; \textbf{5.} not moving in an easy way; not comfortable or elegant.} passage or find a muddle\footnote{\textbf{muddle} [v] (\textit{especially British English}) \textbf{1.} to put things in the wrong order or mix them up; \textbf{2.} muddle somebody (up) to confuse somebody; \textbf{3.} muddle somebody\texttt{/}something (up)$|$ \textbf{muddle A (up) with B} to confuse 1 person or thing with another, \textsc{synonym}: \textbf{mix up}.} on their screens, \& then blame themselves. What should be easy \& flowing looks tangled\footnote{\textbf{tangled} [a] \textbf{1.} twisted together in an untidy way; \textbf{2.} complicated, \& not easy to understand.} or feeble\footnote{\textbf{feeble} [a] \textbf{1.} very weak; \textbf{2.} not effective; not showing energy or effort.} or overblown\footnote{\textbf{overblown} [a] \textbf{1.} that is made to seem larger, more impressive or more important than it really is, \textsc{synonym}: \textbf{exaggerated}; \textbf{2.} (of flowers) past the best, most beautiful stage.} -- not what was meant at all. \fbox{What's wrong with me}, each one thinks. \fbox{Why can't I get this right?}''

[$\ldots$] White knew that a compendium\footnote{\textbf{compendium} [n] (plural \textbf{compendia, compendiums}) a collection of facts, drawings \& photographs on a particular subject, especially in a book.} of specific tips -- about singular \& plural verbs, parentheses, the ``that'' -- ``which'' scuffle\footnote{\textbf{scuffle} [n] \textbf{scuffle (with somebody) $|$ scuffle (between A \& B)} a short \& not very violent fight or struggle; [v] \textbf{1.} [intransitive] \textbf{scuffle (with somebody)} (of 2 or more people) to fight or struggle with each other for a short time, in a way that is not very serious; \textbf{2.} [intransitive] \textbf{$+$ adv.\texttt{/}prep.} to move quickly making a quiet rubbing noise.}, \& many others -- could clear up a recalcitrant\footnote{\textbf{recalcitrant} [a] (\textit{formal}) unwilling to obey rules or follow instructions; difficult to control.} sentence or subclause when quickly reconsulted\footnote{\textbf{consult} [v] \textbf{1.} [transitive, intransitive] to discuss something with somebody to get their permission for something, or to help you make a decision; \textbf{2.} [transitive, intransitive] to go to somebody for information or advice, especially an expert e.g. a doctor or lawyer; \textbf{3.} [transitive] \textbf{consult something} to look in or at something to get information, \textsc{synonym}: \textbf{refer to something}.}, \& that the larger principles needed to be kept in plain sight, like a wall sampler.

How simple they look, set down here in White's last chapter: ``\fbox{Write in a way that comes naturally},'' ``\fbox{Revise \& rewrite},'' ``\fbox{Do not explain too much},'' \& the rest; above all, the cleansing\footnote{\textbf{cleanse} [v] \textbf{1.} [transitive, intransitive] \textbf{cleanse (something)} to clean your skin or a wound; \textbf{2.} [transitive] \textbf{cleanse somebody (of\texttt{/}from something}) (\textit{literary}) to take away somebody's guilty feelings or sin.}, clarion\footnote{\textbf{clarion} [n] \textbf{1.} a medieval trumpet with clear shrill tones; \textbf{2.} the sound of or as if of a clarion' [a] brilliantly clear; loud \& clear.} ``Be clear.'' How often I have turned to them, in the book or in my mind, while trying to start or unblock or revise some piece of my own writing! They help -- they really do. They work. They are the way.

E. B. White's prose is celebrated for its ease\footnote{\textbf{ease} [n] [uncountable] \textbf{1.} lack of difficulty or effort, \textsc{opposite}: \textbf{difficulty}; \textbf{2.} the state of feeling relaxed or comfortable, without anxiety, problems or pain.} \& clarity\footnote{\textbf{clarity} [n] [uncountable] \textbf{1.} the quality of being expressed clearly; \textbf{2.} the ability to think about or understand something clearly; \textbf{3.} if a picture, substance or sound has clarity, you can see or hear it very clearly, or see through it easily.} -- just think of \textit{Charlotte's Web} -- but maintaining this standard required endless attention. When the new issue of \textit{The New Yorker} turned up in Maine, I sometimes saw him reading his ``Comment'' piece over to himself, with only a slightly different expression than the one he'd worn on the day it went off. Well, O.K., he seemed to be saying. \fbox{At least I got the elements right.}

This edition has been modestly\footnote{\textbf{modest} [a] \textbf{1.} fairly limited or small in amout; \textbf{2.} not expensive, rich or impressive; \textbf{3.} (of people, especially women, or their clothes) not showing too much of the body; not intended to attract attention, especially in a sexual way; \textbf{4.} (\textit{approving}) not talking much about your own abilities or possessions.} updated, with word processors \& air conditioners making their 1st appearance among White's references, \& with a light redistribution of genders to permit a feminine pronoun or female farmer to take their places among the males who once innocently\footnote{\textbf{innocent} [a] \textbf{1.} not guilty of a crime, etc.; not having done something wrong, \textsc{opposite}: \textbf{guilty}; \textbf{2.} [only before noun] suffering harm or being killed because of a crime, war, etc. although not directly involved in it; \textbf{3.} having little experience of evil or unpleasant things, or of sexual matters; \textbf{4.} not intended to cause harm or upset somebody, \textsc{synonym}: \textbf{harmless}.} served him.'' [$\ldots$] ``What is not here is anything about E-mail -- the rules-free, lower-case flow that cheerfully keeps us in touch these days. E-mail is conversation, \& it may be replacing the sweet \& endless talking we once sustained\footnote{\textbf{sustain} [v] \textbf{1.} \textbf{sustain somebody\texttt{/}something} to provide enough of what somebody\texttt{/}something needs in order to live or exist; \textbf{2.} to make something continue for some time without becoming less, \textsc{synonym}: \textbf{maintain}; \textbf{3.} \textbf{sustain something} (\textit{formal}) to experience something bad, \textsc{synonym}: \textbf{suffer}; \textbf{4.} \textbf{sustain something} to provide evidence to support an opinion, a theory, etc., \textsc{synonym}: \textbf{uphold}; \textbf{5.} \textbf{sustain something} (\textit{law}) to decide that a claim, etc. is valid, \textsc{synonym}: \textbf{uphold}.} (\& tucked away\footnote{\textbf{tuck away} [phrasal verb] \textbf{tuck something $\leftrightarrow$ away} \textbf{1.} \textbf{be tucked away} to be located in a quiet place, where not many people go; \textbf{2.} to hide something somewhere or keep it in a safe place; \textbf{3.} (\textit{British English, informal}) to eat a lot of food.}) within the informal letter. But we are all writers \& readers as well as communicators, with \fbox{the need at times to please \& satisfy ourselves} (as White put it) with the \fbox{clear \& almost perfect thought}.'' -- \cite[\textit{Foreword} by Roger Angell]{Strunk_White2019}

``I [E. B. White] passed the course, graduated from the university, \& \fbox{forgot the book but not the professor}.'' [$\ldots$]

``\textit{The Elements of Style}, when I [E. B. White] reexamined it in 1957, seemed to me to contain \fbox{rich deposits\footnote{\textbf{deposit} [n] \textbf{1.} a layer of a substance that has been left somewhere, especially by a river or flood, or is found at the bottom of a liquid; \textbf{2.} a layer of a substance that has formed naturally underground; \textbf{3.} [usually singular] \textbf{a deposit (on something)} a sum of money that is given as the 1st part of a larger payment; \textbf{4.} (in the British political system) the amount of money that a candidate in an election to Parliament has to pay, \& that is returned if they get enough votes.} of gold}. It was Will Strunk's \textit{parvum opus}\footnote{\textbf{parvum opus} [from Latin] [n] a little work, a small but meaningful work of an artist or writer.}, his attempt to cut the vast tangle\footnote{\textbf{tangle} [n] \textbf{1.} a twisted mass of threads, hair, etc. that cannot be easily separated; \textbf{2.} a lack of order; a confused state; \textbf{3.} (\textit{informal}) a disagreement or fight; [v] [transitive, intransitive] \textbf{tangle (something) up} to twist something into an untidy mass; to become twisted in this way.} of English rhetoric\footnote{\textbf{rhetoric} [n] [uncountable] \textbf{1.} (\textit{often disapproving} speech or writing that is intended to influence people, but that is not completely honest or sincere; \textbf{2.} the skill of using language in speech or writing in a special way that influences or entertains people.)} down to size \& write its rules \& principles on the head of a pin\footnote{\textbf{pin} [n] \textbf{1.} a short thin piece of stiff wire with a sharp point at 1 end \& a round head at the other, used to hold or attach things; \textbf{2.} a short piece of metal or other material, used to hold things together; \textbf{3.} a piece of metal with a sharp point, worn for decoration; \textbf{4.} 1 of the metal parts that stick out of an electric plug \& fit into a socket; [v] \textbf{pin something ($+$ adv.\texttt{/}prep.)} to attach something onto another thing or join things together with a pin, etc.; \textbf{pin something down} [phrasal verb] to explain or understand something exactly.}. Will himself had hung the tag ``little'' on the book; he referred to it sardonically\footnote{\textbf{sardonically} [adv] (\textit{disapproving}) in a way that shows that you think that you are better than other people \& do not take them seriously, \textsc{synonym}: \textbf{mockingly}.} \& with secret pride as ``the \textit{little book},'' always giving the word ``little'' a special twist, as though he were putting a spin on a ball. In its original form, it was a 43 page summation of the case for cleanliness, accuracy\footnote{\textbf{accuracy} [n] \textbf{1.} [uncountable] the state of being exact or correct, \textsc{opposite}: \textbf{inaccuracy}; \textbf{2.} [uncountable, countable] (\textit{specialist}) the degree to which the result of a measurement or calculation matches the correct value or a standard, \textsc{opposite}: \textbf{inaccuracy}.}, \& brevity\footnote{\textbf{brevity} [n] [uncountable] \textbf{1.} the quality of using few words when speaking or writing; \textbf{2.} \textbf{brevity (of something)} the fact of lasting a short time.} in the use of English. Today, 52 years later, its vigor\footnote{\textbf{vigor} [n] [uncountable] \textbf{1.} effort, energy, \& enthusiasm; \textbf{2.} \textbf{vigor (of something)} physical strength; good health.} is unimpaired\footnote{\textbf{unimpaired} [a] (\textit{formal}) not damaged or made less good, \textsc{opposite}: \textbf{impaired}.}, \& for sheer\footnote{\textbf{sheer} [a] \textbf{1.} [only before noun] used to emphasize the size, degree or amount of something; nothing but; \textbf{2.} very steep.} pith\footnote{\textbf{pith} [n] [uncountable] \textbf{1.} a soft dry white substance inside the skin of oranges \& some other fruits; \textbf{2.} the essential or most important part of something.} I think it probably sets a record that is not likely to be broken. Even after I got through tampering with\footnote{\textbf{tamper with} [phrasal verb] \textbf{tamper with something} to make changes to something without permission, especially in order to damage it, \textsc{synonym}: interfere with.} it, it was still a tiny thing, \fbox{a barely tarnished\footnote{\textbf{tarnished} [v] \textbf{1.} [intransitive, transitive] if mental tarnishes or something tarnishes it, it no longer looks bright \& shiny; \textbf{2.} [transitive, often passive] to damage the good opinion people have of somebody\texttt{/}something, \textsc{synonym}: \textbf{taint}; [n] [singular, uncountable] a thin layer on the surface of a metal that makes it look darker \& less bright.} gem\footnote{\textbf{gem} [n] \textbf{1.} (also less frequent \textbf{gemstone}) a precious stone that has been cut \& polished \& is used in jewellery, \textsc{synonym}: \textbf{jewel, precious stone}; \textbf{2.} a person, place or thing that is especially good.}}. 7 rules of usage, 11 principles of composition\footnote{\textbf{composition} [n] \textbf{1.} [uncountable] the different parts that something is made of; the way in which the different parts are organized; \textbf{2.} [countable] a piece of music or a poem; \textbf{3.} [uncountable] the act of writing a piece of music or a poem; \textbf{4.} [uncountable] (\textit{art}) the arrangement of people of objects in a painting, photograph or scene of a film.}, a few matters of form, \& a list of words \& expressions commonly misused -- that was the sum \& substance\footnote{\textbf{substance} [n] \textbf{1.} a type of solid, liquid or gas that has particular qualities; \textbf{2.} [countable] a drug or chemical, especially an illegal one, that has a particular effect on the mind or body; \textbf{3.} [uncountable] the most important or main part of something; \textbf{4.} [uncountable] (\textit{formal}) importance; \textbf{5.} [uncountable] the quality of being based on facts or the truth.} of Prof. Strunk's work. Somewhat audaciously\footnote{\textbf{audaciously} [adv] (\textit{formal}) in a way that shows you are willing to take risks or to do something that shocks people.}, \& in an attempt to give my publisher his money's worth, I [E. B. White] added a chapter called ``An Approach to Style,'' setting forth my own prejudices\footnote{\textbf{prejudice} [n] [uncountable, countable] an unreasonable dislike of a person, group, etc., especially when it is based on their race, religion, sex, etc.}, my notions of error, my articles of faith. This chapter (Chap. V) is addressed particularly to those who feel that English prose composition is not only a necessary skill but a sensible pursuit as well -- a way to spend one's days. I think Prof. Strunk would not object to that.''

[$\ldots$] ``I have now completed a 3rd revision. Chap. IV has been refurbished\footnote{\textbf{refurbish} [v] \textbf{refurbish something} to clean \& decorate a room, building, etc. in order to make it more attractive, more useful, etc.} with words \& expressions of a recent vintage\footnote{\textbf{vintage} [n] \textbf{1.} the wine that was produced in a particular year or place; the year in which it was produced; \textbf{2.} [usually singular] the period or season of gathering grapes for making wine; [a] [only before noun] \textbf{1.} \textbf{vintage} wine is of very good quality \& has been stored for several years; \textbf{2.} (British English) (of a vehicle) made between 1919 \& 1930 \& admired for its style \& interest; \textbf{3.} typical of a period in the past \& of high quality; the best work of the particular person; \textbf{4.} \textbf{vintage year} a particular good \& successful year.}; 4 rules of usage have been added to Chap. I. Fresh examples have been added to some of the rules \& principles, amplification\footnote{\textbf{amplification} [n] [uncountable] \textbf{1.} \textbf{amplification (of something)} the process of increasing the amplitude of an electrical signal; \textbf{2.} (biochemistry) \textbf{amplification (of something)} the process by which many copies of something, e.g. a gene, are made; \textbf{3.} \textbf{amplification (of something)} the action of making something greater or easier to notice; \textbf{4.} the action of adding details to a story, statement, etc.; details added to a story, statement, etc.} has reared\footnote{\textbf{rear} [v] \textbf{1.} \textbf{rear somebody\texttt{/}something} [often passive] to care for young children or animals until they are fully grown, \textsc{synonym}: \textbf{raise}; \textbf{2.} \textbf{rear something} to breed or keep animals or birds, e.g. on a farm; \textbf{something rears its head} [idiom] (of something unpleasant) to appear or happen; [n] (usually \textbf{the rear}) [singular] the back part of something; [a] [only before noun] at or near the back of something.} its head in a few places in the text where I felt an assault\footnote{\textbf{assault} [n] \textbf{1.} [uncountable, countable] the crime of attacking somebody physically; in law, \textbf{assault} is an act that threatens physical harm to somebody, whether or not actual harm is done: \textit{to commit}\texttt{/}\textit{be charged with assault}; \textbf{2.} [countable] (by an army, etc.) the act of attacking somebody\texttt{/}something, \textsc{synonym}: \textbf{attack}; \textbf{3.} [countable, usually singular, uncountable] an act of criticizing or attacking somebody\texttt{/}something severely; [v] \textbf{assault somebody} to attack somebody physically.} could successfully be made on the bastions\footnote{\textbf{bastion} [n] \textbf{1.} (\textit{formal}) a group of people or a system that protects a way of life or a belief when it seems that it may disappear; \textbf{2.} a place that military forces are defending.} of its brevity, \& in general the book has received a thorough overhaul\footnote{\textbf{overhaul} [n] an examination of a machine or system, including doing repairs on it or making changes to it; [v] \textbf{1.} \textbf{overhaul something} to examine every part of a machine, system, etc. \& make any necessary changes or repairs; \textbf{2.} \textbf{overhaul somebody} to come from behind a person you are competing against in a race \& go past them, \textsc{synonym}: \textbf{overtake}.} -- to correct errors, delete bewhiskered\footnote{\textbf{bewhiskered} [a] \textbf{1.} having whiskers; bearded; \textbf{2.} ancient, as a witticism, expression, etc.; pass\'e; hoary.} entries, \& enliven\footnote{\textbf{enliven} [v] (\textit{formal}) \textbf{enliven something} to make something more interesting or more fun.} the argument.

Prof. Strunk was a positive man. His book contains rules of grammar phrased as direct orders. In the main I [E. B. White] have not tried to soften his commands, or modify his pronouncements\footnote{\textbf{pronouncement} [n] a formal public statement.}, or remove the special objects of his scorn\footnote{\textbf{scorn} [n] [uncountable] a strong feeling that somebody\texttt{/}something is stupid or not good enough, usually shown by the way you speak, \textsc{synonym}: \textbf{contempt}; [v] \textbf{1.} \textbf{scorn somebody\texttt{/}something} to feel or show that you think somebody\texttt{/}something is stupid \& you do not respect them or it, \textsc{synonym}: \textbf{dismiss}; \textbf{2.} (\textit{formal}) to refuse to have or do something because you are too proud.}. I have tried, instead, to preserve\footnote{\textbf{preserve} [v] \textbf{1.} \textbf{preserve something} to keep a particular quality or feature; \textbf{2.} to keep something safe from harm, in good condition or in its original state; \textbf{3.} to prevent something from decaying, by treating it in a particular way; [n] [singular] an activity, job or interest that is thought to be suitable for 1 particular person or group of people.} the flavor\footnote{\textbf{flavor} [n] \textbf{1.} [uncountable] \textbf{flavor (of something)} how food or drink tastes, \textsc{synonym}: \textbf{taste}; \textbf{2.} [countable] a particular type of taste; \textbf{3.} [singular] a particular quality or atmosphere; \textbf{4.} [singular] \textbf{a\texttt{/}the flavor of something} an idea of what something is like.} of his discontent\footnote{\textbf{discontent} [n] (also \textbf{discontentment}) \textbf{1.} [uncountable] a feeling of being unhappy because you are not satisfied with a particular situation, \textsc{synonym}: \textbf{dissatisfaction}; \textbf{2.} [countable] \textbf{discontent (of somebody)} a thing that makes you feel unhappy \& not satisfied with a particular situation, \textsc{synonym}: \textbf{dissatisfaction}.} while slightly enlarging the scope of the discussion. \textit{The Elements of Style} does not pretend\footnote{\textbf{pretend} [v] \textbf{1.} to behave in a particular way, in order to make other people believe something that is not true; \textbf{2.} (usually used in negative sentences \& questions) to claim to be, do or have something, especially when this is not true.} to survey\footnote{\textbf{survey} [n] \textbf{1.} \textbf{survey} (of somebody\texttt{/}something) an investigation of the opinions, behavior, etc. of a particular group of people, which is usually done by asking them questions; \textbf{2.} an act of examining \& recording the measurements, features, etc. of an area of land in order to make a map or plan of it; \textbf{3.} \textbf{survey (of something)} a general study, view or description of something; [v] \textbf{1.} \textbf{survey somebody\texttt{/}something} to investigate the opinions or behavior of a group of people by asking them a series of questions; \textbf{2.} \textbf{survey something} to study \& give a general description of something; \textbf{3.} \textbf{survey something} to measure \& record the features of an area of land, e.g. in order to make a map or in preparation for building; \textbf{4.} \textbf{survey something} to look carefully at the whole of something, especially in order to get a general impression of it, \textsc{synonym}: \textbf{inspect}.} the whole field. Rather it proposes\footnote{\textbf{propose} [v] \textbf{1.} to suggest a plan or an idea for people to consider \& decide on; \textbf{2.} to suggest an explanation of something for people to consider.} to give in brief space the principal\footnote{\textbf{principal} [a] [only before noun] main; most important.} requirements of plain\footnote{\textbf{plain} [a] \textbf{1.} easy to see or understand, \textsc{synonym}: \textbf{clear}; \textbf{2.} [only before noun] expressed in a clear \& simple way, without using technical language; \textbf{3.} not trying to deceive anyone; honest \& direct; \textbf{4.} not decorated or complicated; simple; in computing, \textbf{plain text} is data representing text that is not written in code or using special formatting \& can be read, displayed or printed without much processing: \textit{Mathematical formulae are an example of content that cannot be represented satisfactorily via plain text.}; \textbf{5.} without marks or a pattern on it; \textbf{6.} [only before noun] (used for emphasis) simple; nothing but. \textsc{synonym}: \textbf{sheer}.} English style. It concentrates\footnote{\textbf{concentrate} [v] \textbf{1.} [transitive, often passive] \textbf{concentrate something $+$ adv.\texttt{/}prep.} to bring something together in 1 place; \textbf{2.} [intransitive, transitive] to give all your attention to something \& not think about anything else; \textbf{3.} [transitive] \textbf{concentrate something} to increase the strength of a substance by reducing its volume, e.g. by boiling it; \textbf{concentrate on something} [phrasal verb] to spend more time doing 1 particular thing than others; [n] [countable, uncountable] \textbf{concentrate (of something)} a substance that is made stronger because water or other substances have been removed.} on fundamentals\footnote{\textbf{fundamentals} [n] [plural] \textbf{fundamentals (of something)} the basic \& most important parts of something.}: the rules of usage \& principles of composition most commonly violated\footnote{\textbf{violet} [v] \textbf{1.} \textbf{violate something} to go against or refuse to obey a law, an agreement, etc.; \textbf{2.} \textbf{violate something} to not treat something with respect.}. [\texttt{inserting here ...}]'' -- \cite[Introduction]{Strunk_White2019}

%---------------------------------------------------------------------- --------%

\chapter{Scientific\texttt{/}Mathematical Writings}

\section{Luc Tartar's Writing Styles}

\section{Terence Tao\texttt{/}\href{https://terrytao.wordpress.com/advice-on-writing-papers/}{On Writing}}
\begin{quotation}
	``There are three rules for writing the novel. Unfortunately, no one knows what they are.'' -- W. Somerset Maugham
\end{quotation}
``Everyone has to \href{https://terrytao.wordpress.com/advice-on-writing-papers/write-in-your-own-voice/}{develop their own writing style}, based on their own strengths and weaknesses, on the subject matter, on the target audience, and sometimes on the target medium. As such, it is virtually impossible to prescribe rigid rules for writing that encompass all conceivable situations and styles.

Nevertheless, I do have some general advice on these topics:
\begin{itemize}
	\item Writing a paper
	\begin{itemize}
		\item ``Use the introduction to \href{https://terrytao.wordpress.com/advice-on-writing-papers/use-the-introduction-to-%E2%80%9Csell%E2%80%9D-the-key-points-of-your-paper/}{``sell'' the key points of your paper}; the results should \href{https://terrytao.wordpress.com/advice-on-writing-papers/describe-the-results-accurately/}{be described accurately}. One should also invest some effort in both \href{https://terrytao.wordpress.com/advice-on-writing-papers/organise-the-paper/}{organizing} and \href{https://terrytao.wordpress.com/advice-on-writing-papers/motivate-the-paper/}{motivating} the paper, and in particular in \href{https://terrytao.wordpress.com/advice-on-writing-papers/use-good-notation/}{selecting good notation} and \href{https://terrytao.wordpress.com/advice-on-writing-papers/give-appropriate-amounts-of-detail/}{giving appropriate amounts of detail}. But \href{https://terrytao.wordpress.com/advice-on-writing-papers/dont-overoptimise/}{one should not over-optimize} the paper.
		\item It also assists readability if you factor the paper into smaller pieces, e.g., by \href{https://terrytao.wordpress.com/advice-on-writing-papers/create-lemmas/}{making plenty of lemmas}.
		\item To reduce the time needed to write and organize a paper, I recommend \href{https://terrytao.wordpress.com/advice-on-writing-papers/write-a-rapid-prototype-first/}{writing a rapid prototype 1st}.
		\item For 1st time authors especially, it is important to try to \href{https://terrytao.wordpress.com/advice-on-writing-papers/write-professionally/}{write professionally}, and in \href{https://terrytao.wordpress.com/advice-on-writing-papers/write-in-your-own-voice/}{one's own voice}. One should \href{https://terrytao.wordpress.com/advice-on-writing-papers/take-advantage-of-the-english-language/}{take advantage of the English language}, and not just rely purely on mathematical symbols.
		\item The \href{https://terrytao.wordpress.com/advice-on-writing-papers/maximising-the-results-to-effort-ratio/}{ratio between results and effort in one's paper should be at a local maximum}.
	\end{itemize}
	\item Submitting a paper
	\begin{itemize}
		\item \href{https://terrytao.wordpress.com/advice-on-writing-papers/proofread-and-double-check-your-paper-before-submission/}{Proofread and double-check your article before submission}; you should be \href{https://terrytao.wordpress.com/advice-on-writing-papers/submit-a-final-draft-not-a-first-draft/}{submitting a final draft, not a 1st draft}
		\item \href{https://terrytao.wordpress.com/advice-on-writing-papers/submit-to-an-appropriate-journal/}{Subset to an appropriate journal}
	\end{itemize}
\end{itemize}
I should point out, of course, that my own writing style is not perfect, and I myself don't always adhere to the above rules, often to my own detriment. If some of these suggestions seem too unsuitable for your particular paper, use common sense.

Dual to the art of \textit{writing} a paper well, is the art of \textit{reading} a paper well.\footnote{NQBH: In mathematical notation: \begin{align*}
		\mbox{(Art of writing a paper well)} = \mbox{(Art of reading a paper well)}^\star,\ \mbox{(Art of reading a paper well)} = \mbox{(Art of writing a paper well)}^\star.
\end{align*}}\footnote{NQBH: In linguistic, \textit{reading} and \textit{writing skills} usually come together, so do \textit{listening} and \textit{speaking skills}. I.e., if one wants to master 1 of these 4 skills, then that person has to master its companion parallelly:
\begin{align*}
	(\mbox{reading}\land\mbox{writing})\lor(\mbox{speaking}\land\mbox{listening}).
\end{align*}} Here is some commentary of mine on this topic:
\begin{itemize}
	\item \href{https://terrytao.wordpress.com/advice-on-writing-papers/on-compilation-errors-in-mathematical-reading-and-how-to-resolve-them/}{On ``compilation errors'' in mathematical reading, and how to resolve them}.
	\item \href{https://terrytao.wordpress.com/advice-on-writing-papers/implicit-notational-conventions/}{On the use of implicit mathematical notational conventions to provide contextual clues when reading}.
	\item \href{https://terrytao.wordpress.com/advice-on-writing-papers/on-the-strength-of-theorems/}{On key ``jumps in difficulty'' in a mathematical argument, and how finding and understanding them is often key to understanding the argument as a whole}.
	\item \href{https://terrytao.wordpress.com/advice-on-writing-papers/on-local-and-global-errors-in-mathematical-papers-and-how-to-detect-them/}{On ``local'' and ``global'' errors in mathematical papers, and how to detect them}.
\end{itemize}
Some further advice on mathematical exposition: [$\ldots$]''

\subsection{Terence Tao\texttt{/}\href{https://terrytao.wordpress.com/advice-on-writing-papers/describe-the-results-accurately/}{On Writing\texttt{/}Describe the Results Accurately}}
\begin{quotation}
	``10,000 fools proclaim themselves into obscurity, while 1 wise man forgets himself into immortality.'' -- Martin Luther King Jr.
\end{quotation}
``\fbox{A paper should neither understate nor overstate its main results.}

If the main result is very surprising or a substantial breakthrough compared with the previous literature, these facts should be noted (and justified in detail, e.g., by explicit comparison with prior results, examples, and conjectures).

Conversely, if there are unsatisfactory aspects to the result (e.g., hypotheses too strong, or conclusions a little weaker than expected) these should also be stated honestly and openly, e.g., ``We do not know if hypothesis H is actually necessary''. Similarly, it is worth noting down any interesting open questions remaining after your result.

If you are using a famous unsolved conjecture to motivate your own work, one should give a candid evaluation of the extent to which your work truly represents progress towards that conjecture, so as to avoid the impression of ``false advertising'' or ``name-dropping''.

If for some reason you need to assert a non-trivial statement without proof or citation, it should be made clear that you are doing so (e.g., ``It can be shown that $\ldots$'' or ``Although we will not need or prove this fact here $\ldots$''), so that the reader does not then hunt through the rest of your paper for the non-existent justification of that statement.

\fbox{Titles of sections should be descriptive} (e.g., ``proof of the decomposition lemma'' or ``An orthogonality argument''), as opposed to uninformative (e.g., ``Step 2'' or ``Some technicalities'').

\subsection{Terence Tao\texttt{/}\href{https://terrytao.wordpress.com/advice-on-writing-papers/give-appropriate-amounts-of-detail/}{On Writing\texttt{/}Give Appropriate Amounts of Detail}}
\begin{quotation}
	``In presenting a mathematical argument the great thing is to give the educated reader the chance to catch on at once to the momentary point and take details for granted: his successive mouthfuls should be such as can be swallowed at sight; in case of accidents, or in case he wishes for once to check in detail, he should have only a clearly circumscribed little problem to solve (e.g., to check an identity: 2 trivialities omitted can add up to an impasse). The unpracticed writer, even after the dawn of a conscience, gives him no such chance; before he can spot the point he has to tease his way through a maze of symbols of which not the tiniest suffix can be skipped.'' -- John Littlewood, \textit{``A Mathematician's Miscellany''}
\end{quotation}
A paper should dwell at length (using \href{https://terrytao.wordpress.com/advice-on-writing-papers/take-advantage-of-the-english-language/}{plenty of English}) on the most important, innovative, and crucial components of the paper, and be brief on the routine, expected, and standard components of the paper.

In particular, \fbox{a paper should identity which of its components are the most interesting}. Note that this means interesting to \textit{experts in the field}, and not just interesting to \textit{yourself}; e.g., if you have just learnt how to prove a standard lemma which is well known to the experts and already in the literature, this does not mean that you should provide the standard proof of this standard lemma, unless this serves some greater purpose in the paper (e.g., by motivating a less standard lemma).

Conversely, some computations, definitions, or notational conventions which you are very familiar with, but are not widely known in the field, should be expounded on in detail, even if these details are ``obvious'' to you due to your extensive work in this area. Even a brief sentence of explanation is much better than none at all.

For a similar reason, if you are using a relatively obscure lemma from, say, 1 of your own papers, you should not assume that every reader of your current article is intimately familiar with your previous paper. In such cases it is worth stating the lemma in full, with a precise citation (as opposed to casually using phrases e.g., ``by a lemma in [my previous 100-page paper], we have $\ldots$''). When the lemma is particularly crucial, it is sometimes also worth spending a paragraph to sketch out a proof, or to otherwise remark on the significance of this lemma and its connections to other, more well known results.''

\subsection{Terence Tao\texttt{/}\href{https://terrytao.wordpress.com/advice-on-writing-papers/take-advantage-of-the-english-language/}{On Writing\texttt{/}Take Advantage of the English Language}}
\begin{quotation}
	``Use soft words and hard arguments.'' -- Proverbial
\end{quotation}
``\href{https://terrytao.wordpress.com/advice-on-writing-papers/use-good-notation/}{Mathematical notation} is a wonderfully useful tool, and it can be exciting to learn for the first time the meaning of mysterious and arcane symbols e.g., $\forall,\exists,\emptyset,\Rightarrow$, etc. However, just because you \textit{can} write statements in purely mathematical notation doesn't mean that you necessarily \textit{should}. In many cases, it is in fact far more informative and readable to use liberal amounts of plain English; if used correctly and thoughtfully, the English language can communicate to the reader on many more levels than a mathematical expression, \fbox{without sacrificing any precision or rigor}. In particular, by subtly modulating the emphasis of one's text, one can convey valuable contextual cues as to how a statement interacts with the rest of one's argument.

An example should serve to illustrate this point. Suppose for instance that $P$ and $Q$ are properties that can apply to mathematical objects $x$ and $y$. The mathematical statements $P(x)\land Q(y)$ which asserts that $x$ satisfies $P$ and $y$ satisfies $Q$, is a well-formed and precise mathematical statement. But there are many possible ways one could express that mathematical statement in English, e.g.,: $\ldots$'' \texttt{[Skip 27 items]}

``From the viewpoint of formal mathematical logic, each of these English statement is logically equivalent to the mathematical sentence $P(x)\land Q(y)$. However, each of the above English statements also provides additional useful and informative cues for the reader regarding the relative importance, non-triviality, and causal relationship of the component statements $P(x)$ and $Q(y)$, or of the component symbols $P,x,Q$, and $y$. E.g., in some of these sentences $P(x)$ and $Q(y)$ are given equal importance (being complementary or somehow in opposition to each other), whereas in others $P(x)$ is only an auxiliary statement whose only purpose is to derive $Q(y)$ (or vice versa), and in yet others, $P(x)$ and $Q(y)$ are deemed to be analogous, even if one is not formally deducible from the other. In some sentences, it is the objects $x$ and $y$ which are indicated to be the primary actors; in other sentences, it is the properties $P$ and $Q$; and in yet other sentences, it is the combined statements $P(x)$ and $Q(y)$ which are the most central.

Thus we see that English sentences can be considerably more expressive than their formal mathematical counterparts, while still retaining the precision and rigor that mathematical exposition demands. By using such humble English words as ``also'', ``but'', ``since'', etc., a sentence conveys not only its semantic content, but also how it is going to fit in with the rest of one's argument (or in the wider theory of the object), giving the reader more insight as to the overall structure of that argument. \fbox{The paper may become slightly longer because of this, but this is a small price to pay for readability} (which is \textit{not} the same as brevity!).

On the other hand, one should \href{https://terrytao.wordpress.com/advice-on-writing-papers/dont-overoptimise/}{not try to excessively ``improve''} the paper by using overly fancy or obscure words (from English or any other language), especially since such words can be mistaken for some sort of technical mathematical terminology. In many cases, one can replace complicated words by plainer equivalents, thus increasing the readability of one’s text without compromising the message. The primary purpose of mathematical writing is to \textit{communicate} and \textit{inform}, not to \textit{impress}.

Finally, there is 1 situation in which it does make sense to use the terse language of mathematical notation rather than a more leisurely English equivalent, and that is when you are performing a tedious and standard formal computation. In those cases, the reader should already know in general terms what is going to happen (especially if you flag the computation as being standard beforehand), and will only be distracted by superfluous explanation or digression. (See also ``\href{https://terrytao.wordpress.com/advice-on-writing-papers/give-appropriate-amounts-of-detail/}{give appropriate amounts of detail}''.)

Naturellement, la discussion ci-dessus s'applique également à d'autres langues, telles que la langue française.''\footnote{Of course, the above discussion also applies to other languages, such as the French language.}

\subsection{Terence Tao\texttt{/}\href{https://terrytao.wordpress.com/advice-on-writing-papers/use-good-notation/}{On Writing\texttt{/}Use Good Notation}}
\begin{quotation}
	``By relieving the brain of all unnecessary work, a good notation sets it free to concentrate on more advanced problems, and, in effect, increases the mental power of the race.'' -- Alfred North Whitehead, ``An Introduction to Mathematics''
\end{quotation}
``\fbox{Good notation can make the difference between a readable paper and an unreadable one.}

Ideally, notation should emphasize the most important parameters and features of a mathematical expression or statement, while downplaying the routine or uninteresting parameters and features. For instance, if one does not care much about the exact values of constants in estimates, then notation which conceals these constants (e.g., $\ll$, $\lesssim$, or $O(\cdot)$) are useful; conversely, these notations should be avoided if the precise values of these constants are of importance to the paper.

\fbox{Notation which is used globally should be defined in a notation section near the front of the paper, or in the introduction;} notation which is only used locally (e.g., within a single section, or within a proof of a single lemma) should be defined close to where it is used (possibly with a reminder that this notation is not used elsewhere in the paper); this is helpful when there are many sections, each with their own extensive notation.

Note that notation or statements which are introduced within a proof of a lemma are already understood to be localized to that lemma; it is bad form to then recall that notation or statement outside of that lemma, except perhaps as a remark or as motivation. In some cases it is worthwhile to define the notation once near the start of the paper, and then recall it whenever necessary.

One should strive to make one's choices of notation compatible and consistent with notation already in the literature, so that the readers who are already familiar with prior notation will adapt easily to your paper and will not be confused.

\fbox{Try to avoid notation which is overly ``cute'' or ``clever''.} This can be distracting or appear \href{https://terrytao.wordpress.com/career-advice/be-professional-in-your-work/}{unprofessional}. In particular, the notation should not be cleverer than the actual substance of the paper.

One should \textbf{definitely} avoid naming new terms after yourself (or after your family members, your pets, etc.), for the obvious reasons. If other authors name the concepts you introduce after yourself, and that appellation becomes common usage, then you may use that term as well, but in all other cases it gives the rather \fbox{blatant impression of vanity or narcissism}.

There is an issue of where to strike the balance between too little notation and too much notation. A good rule of thumb is that any expression or concept which is used 3 or more times will probably benefit from introducing some notation to capture that expression or concept; conversely, an expression which is only used once probably does not need its own special notation. (An exception would be for particularly crucial theorems or propositions in the paper; here it might be worthwhile to invest in some notation in order to make the statement of those theorems clean and readable. Conversely, if an expression only appears in multiple locations of the paper because of coincidences of no significance, then it may be better to avoid introducing notation that gives the false impression of a connection between these appearances.)

If one needs to name a certain property or class of objects, one should generally use very bland names (e.g., ``good'', ``bad'', ``Type I'', ``Type II'', etc.) for peripheral or technical terms; colorful terms should be used sparingly, and only for those concepts that are quite central to the paper, lest they distract from the main points of that paper. (This is analogous to how, in film and literature, the main characters generally tend to have more memorable names than the secondary ones.)

Sometimes one is unsure what notation to use for a particular concept, because of potential conflicts with other notation in other (as yet unwritten) parts of a paper. One solution here is to introduce a \TeX\ \href{http://en.wikipedia.org/wiki/Macro}{macro} for that notation, and force yourself to use that macro exclusively whenever that notation is used. (E.g., if you have a group which you are tentatively naming $G$, you could define a macro \verb|\grp| that is set to G, and use \verb|\grp| instead of G throughout the paper.) That way, if you find a notational conflict later on (e.g., if you discover that you really need G to denote a graph instead), then you only need to change \textit{1 line} in your \TeX\ file -- the line that defines the macro -- to resolve the notational conflict, rather than to do a tedious (and error-prone) search-and-replace.

For any rigorous component of the paper, the notation should be precise and unambiguous (and for non-rigorous components, ambiguous notation should be pointed out with ``scare quotes'' or other cautionary phrases such as ``roughly speaking'' or ``essentially''). A certain amount of abuse of notation is permitted, though, as long as this is properly pointed out.'' \texttt{[Skip the common example of \textit{division}, i.e., $a/bc$ means either $(a/b)c$  or $a/(bc)$; or use $\frac{a}{b}c$ and $\frac{a}{bc}$ instead].}

``It is also worthwhile to quietly reinforce one's notational conventions when given the opportunity. E.g., suppose in one's argument one has a vector space, which one has decided to call $V$. When referring back to this object, one could say ``the vector space'', or ``V'', but if the reader does not remember what vector space is being discussed, or what $V$ is, the reader will have to take a minute or so to flip back and figure this out. But if instead you refer to this object consistently as ``the vector space $V$'', then the notational convention is reinforced, and the reader can continue reading without breaking rhythm. (One can also modulate the choice of terminology used here to emphasize different aspects of the object being referred to. If e.g., it is the additive structure of $V$ which is currently relevant, you can instead say ``the additive group $V$''; if, later, it is the topological structure which is the most important, one can say ``the topological vector space $V$'', and so forth. This allows one to subtly draw attention to the most important features of the object under consideration, without distracting the reader from the main body of the argument.)''

See also \href{https://mathoverflow.net/questions/366070/what-are-the-benefits-of-writing-vector-inner-products-as-langle-u-v-rangle/366118#366118}{Terence Tao's answer to MathOverflow question: What are the benefits of writing vector inner products as $\langle{\bf u},{\bf v}\rangle$ as opposed to ${\bf u}^\top{\bf v}$?}

\subsection{Terence Tao\texttt{/}\href{https://terrytao.wordpress.com/advice-on-writing-papers/write-in-your-own-voice/}{On Writing\texttt{/}Write in Your Own Voice}}

\begin{quotation}
	``While one should always study the method of a great artist, one should never imitate his manner. The manner of an artist is essentially individual, the method of an artist is absolutely universal. The first is personality, which no one should copy; the second is perfection, which all should aim at.'' -- Oscar Wilde, \textit{A Critic in Pall Mall}, p. 195
\end{quotation}
``When, as a graduate student, one is starting out one's research in a mathematical subject, one usually begins by reading the papers of the current and past leaders of the field. Initially, one's understanding of the subject is fairly limited, and so it is natural to view these papers as being authoritative, especially if their authors are well-known.

Eventually, though, one requires a fair fraction of the insights and understanding conveyed by the existing literature, and is able to apply it to produce a new result or observation that goes beyond that literature (or, at least, makes explicit what was only implicitly buried in previous papers). When the ramifications and extensions of these new advances have been explored to their natural extent, it then becomes time to write up these results as a research paper.

Of course, as your work is almost certainly based in part on the previous literature, one should cite that literature whenever appropriate, and compare and contrast your own work with that literature in an \href{https://terrytao.wordpress.com/advice-on-writing-papers/describe-the-results-accurately/}{accurate}, \href{https://terrytao.wordpress.com/advice-on-writing-papers/write-professionally/}{professional}, and informative manner. Also, one should try to \href{https://terrytao.wordpress.com/advice-on-writing-papers/use-good-notation/}{maintain some level of notational consistency} with the previous literature, such as using the same fundamental definitions and to use similar notation, so that expert readers who are already familiar with that literature can quickly get up to speed on your work. And if 1 of the arguments in your work is standard in the literature, it certainly makes sense to structure the argument in a standard fashion if possible, again to assist the experts reading your paper.

\textbf{However}, one should \textbf{not} go so far as to copy entire paragraphs or more of text from a prior paper, except when used sa a direct quotation to illustrate some historical point. First of all, if one does not properly attribute that text (e.g., ``As Bourbaki [17, p. 146] writes,'', or, for that matter, the Oscar Wilde quote above), then one runs the risk of committing \href{http://en.wikipedia.org/wiki/Plagiarism}{plagiarism}. But even if the text is properly attributed, copying the text verbatim, without updating it to reflect more recent developments (including that in the paper being written) and to add your own simplifications and insights, is a redundant waste of space and a lost opportunity to advance the subject. If one is tempted to copy a significant portion of text from a prior reference without adding anything significantly new, one should instead simply cite the previous reference appropriately, e.g., ``See [27, Section 4] for further discussion.'' or ``A proof can be found in [9, Lemma 2.4].'' (cf. ``\href{https://terrytao.wordpress.com/advice-on-writing-papers/give-appropriate-amounts-of-detail/}{Give appropriate amounts of details}'').

Of course, there \textit{are} reasons to duplicate to some extent some discussion or argument that was present in a previous paper:
\begin{itemize}
	\item As mentioned earlier, one may wish to make some historical point, e.g., to track the development of a mathematical idea over time.
	\item If the paper is obscure and not widely available, reproducing a key argument from that paper may serve as a convenience to the reader.
	\item Also, if the \textit{form} of that argument can be used to \href{https://terrytao.wordpress.com/advice-on-writing-papers/motivate-the-paper/}{motivate} other arguments in your paper, then it can be worth putting in that argument so that it can be referred to later in the paper.
	\item The precise result needed for your paper may differ slightly from what is already established in the literature, and so one needs to either write out a modified version of the proof, or else point to the original proof but indicate what modifications need to be made. (The latter is suitable if the changes are particularly minor in nature.)
	\item The existing paper may have an argument which can be updated, simplified, modernized, or otherwise improved thanks to more recent advances or insights in the area (including your own). It can then be a service to the field to place an updated version of the argument in the literature (with full citations to the paper containing the original argument, of course).
\end{itemize}
However, when one is not simply quoting the prior text for historical or archival purposes, it is best to \textit{paraphrase} and \textit{interpret} the previous text rather than to copy that text verbatim. This is for a number of reasons:
\begin{itemize}
	\item One wants to avoid conveying any impression to readers, referees, or editors of plagiarism, padding, or intellectual laziness in one's papers. (Note that the latter is a danger even if one is copying from one's own work, rather than that of others.)
	\item The prior work may be dated in view of more recent developments and insights, as mentioned above.
	\item If you are copying or adapted a piece of text from another author that you do not fully understand yourself, then it may end up being inappropriate or incongruous for your intended purpose, and may convey the impression of superficiality or being ill-informed. If the text becomes inaccurate due to this adaptation, then this can also cause some embarrassment and annoyance for the original author of that text.
	\item Excessive use of quotation from famous mathematicians to make one's own work look more impressive is the mathematical equivalent of name-dropping, and should be avoided. Appeal to authority should not be the primary basis for motivating a paper; a handful of citations to demonstrate the depth of interest in the problem being studied is usually sufficient.
	\item \textit{But most importantly of all, for one's further mathematical development and career, one needs to develop one's own consistent mathematical ``voice'' and style, and to avoid the impression of simply imitating the voices of other authors}. There is no need in this subject for the mathematical equivalent of a parrot, and a text which is a mix of the author's voice and the voice of others can read very strangely.
\end{itemize}
Of course, if one is paraphrasing a previous work, one should cite that work appropriately (e.g., ``The proof here is loosely based on that in [5].'' or ``This discussion is inspired by a related discussion in [10].'').

In some cases, the imitation of a previous author's style and text is intended as a sign of respect or flattery for that author. \textbf{This is misguided}; an author will in fact often find such mimicry to actually be somewhat offensive. If one wants to truly respect a mathematician, then understand that mathematician's methods, results, and exposition, and improve, update, adapt, and advance all 3. Even the greatest mathematician's contributions should advance with the field, rather than being worshiped and preserved in some supposed state of perfection; the latter is mostly suitable only for historical purposes.

Another possible reason for copying the style of a more senior mathematician is that one does not yet have the self-confidence to write in one's own style and voice. While this is justifiable to some extent when one is just starting one's career, it becomes less excusable as one continues one's research. If one is hesitant to state things in one's own fashion, it is perfectly acceptable to couch such text with the appropriate caveats (e.g., ``to the author's knowledge, this observation is new'' or ``While Lemma 2.5 is usually phrased in a topological fashion, we found the following, more geometric, formulation to be more convenient for our applications''). And if one does not feel confident enough in one's understanding of a subject to explain it in any other way than copying from a previous paper, then this should be taken as a sign that one still needs to \href{https://terrytao.wordpress.com/career-advice/learn-and-relearn-your-field/}{internalize the subject futher}.

When writing a paper with 1 or more coauthors, there will inevitably be distinctions in style,\footnote{NQBH: a reasonable justification for loneliness and\texttt{/}in solo (academic) writing.} and so initially different sections may have sharply different tones due to their being largely written by different subsets of coauthors; but I usually find that after a few rounds of editing, the voices are harmonized into a style which is clearly derived from, but distinct from, each of the individual styles. Ideally, one should understand and respect the underlying stylistic decisions of one's coauthors, but at the same time be willing to take the initiative and find ways to formulate the text and arrangement to smoothly reconcile the coauthor's preferences with one's own; if all goes well, this can lead to a level of exposition and presentation that is superior to what each of the individual authors could separately achieve. (Of course, if you are to perform major edits on a coauthor's contribution, some consultation with that coauthor is presumably desirable). This process can be quite educational; my own writing style has definitely been influenced in a positive fashion by those of my coauthors.

Developing one's own style is, by definition, a very personal process; while external advice or role models can certainly be of some influence, they are of limited utility after a certain point. But finding an individual style which is comfortable and effective for both you and your readers is an important mark of one's \textit{mathematical maturity}, and is a goal that is definitely worth pursuing.''
%------------------------------------------------------------------------------%

\section*{Quick notes}
``when possible and\texttt{/}or necessary'' -- \cite[p. 47]{Rebollo_Lewandowski2014}

%---------------------------------------------------------------------- --------%

\begin{thebibliography}{99}
	\bibitem[TerryTao]{TerryTao} \href{https://terrytao.wordpress.com}{Terence Tao's blog}.
	\begin{itemize}
		\item Terence Tao. \href{https://terrytao.wordpress.com/advice-on-writing-papers/}{\textit{On writing}} (or in full: \textit{Advice on writing papers}).
		\begin{itemize}
			\item Terence Tao. \href{https://terrytao.wordpress.com/advice-on-writing-papers/describe-the-results-accurately/}{\textit{On writing}\texttt{/}\textit{Describe the results accurately}}.
			\item Terence Tao. \href{https://terrytao.wordpress.com/advice-on-writing-papers/give-appropriate-amounts-of-detail/}{\textit{On writing}\texttt{/}\textit{Give appropriate amounts of detail}}.
			\item Terence Tao. \href{https://terrytao.wordpress.com/advice-on-writing-papers/use-good-notation/}{\textit{On writing}\texttt{/}\textit{Use good notation}}.
			\item Terence Tao. \href{https://terrytao.wordpress.com/advice-on-writing-papers/write-in-your-own-voice/}{\textit{On writing}\texttt{/}\textit{Write in your own voice}}.
		\end{itemize}
	\end{itemize}
\end{thebibliography}

%---------------------------------------------------------------------- --------%

\printbibliography[heading=bibintoc]
	
\end{document}