\documentclass[oneside]{book}
\usepackage[backend=biber,natbib=true,style=authoryear]{biblatex}
\addbibresource{/home/hong/1_NQBH/reference/bib.bib}
\usepackage[vietnamese,english]{babel}
\usepackage{tocloft}
\renewcommand{\cftsecleader}{\cftdotfill{\cftdotsep}}
\usepackage[colorlinks=true,linkcolor=blue,urlcolor=red,citecolor=magenta]{hyperref}
\usepackage{amsmath,amssymb,amsthm,mathtools,float,graphicx}
\allowdisplaybreaks
\numberwithin{equation}{section}
\newtheorem{assumption}{Assumption}[chapter]
\newtheorem{conjecture}{Conjecture}[chapter]
\newtheorem{corollary}{Corollary}[chapter]
\newtheorem{definition}{Definition}[chapter]
\newtheorem{example}{Example}[chapter]
\newtheorem{lemma}{Lemma}[chapter]
\newtheorem{notation}{Notation}[chapter]
\newtheorem{principle}{Principle}[chapter]
\newtheorem{problem}{Problem}[chapter]
\newtheorem{proposition}{Proposition}[chapter]
\newtheorem{question}{Question}[chapter]
\newtheorem{remark}{Remark}[chapter]
\newtheorem{theorem}{Theorem}[chapter]
\usepackage[left=0.5in,right=0.5in,top=1.5cm,bottom=1.5cm]{geometry}
\usepackage{fancyhdr}
\pagestyle{fancy}
\fancyhf{}
\lhead{\small \textsc{Sect.} ~\thesection}
\rhead{\small \nouppercase{\leftmark}}
\renewcommand{\sectionmark}[1]{\markboth{#1}{}}
\cfoot{\thepage}
\def\labelitemii{$\circ$}

\title{A Personal Journey to Writing}
\author{\selectlanguage{vietnamese} Nguyễn Quản Bá Hồng}
\date{\today}

\begin{document}
\maketitle
\tableofcontents

%---------------------------------------------------------------------- --------%

\chapter*{Foreword}
My personal journey to ``The Garden of Words''\footnote{\href{https://www.imdb.com/title/tt2591814/}{IMDb\texttt{/}\textsc{The Garden of Words} (2013)}, original title: \textsc{Koto no ha no niwa}.} -- the world of writings. \textit{Why writing?} Because: Instead of provoking a weak \& poor defense mechanism passively \& unconsciously, you should make your verbal enemies take a step back 1st: Words are weapons.

Although I have studied my Master in France, \& later worked in Germany \& Austria, I only know some Frenches \& German words. I prefer to use Vietnamese \& English, even if I have a chance to learn a 3rd one comprehensively. The main purpose of writing is to express ideas, thoughts, emotions, etc., not to show off someone's linguistic ability, especially the wide range of languages they can use.

%---------------------------------------------------------------------- --------%

\section{Dictionary}
To read \& write well, the 1st concern is, obviously, to choose the right dictionary\texttt{/}dictionaries.

\begin{question}
	Which dictionary\emph{\texttt{/}}dictionaries should I use?
\end{question}

\begin{itemize}
	\item \href{https://dictionary.cambridge.org/}{Cambridge Dictionary}: ``Make your words meaningful''
	\item \href{https://www.collinsdictionary.com/us/}{Collins Dictionary}
	\item \href{https://www.merriam-webster.com/}{Merriam-Webster Dictionary}
	\item \href{https://www.oxfordlearnersdictionaries.com/}{Oxford Learner's Dicitonaries}
\end{itemize}
I choose \href{https://www.oxfordlearnersdictionaries.com/}{Oxford Learner's Dicitonaries}. Then the next question is:

\begin{question}
	Should I buy Oxford Learner's Dictionary of Academic English?
\end{question}
\pounds 5.5\texttt{/}year though. Bought: Seem worth it (?).

\begin{remark}[Personal style]
	I do not like to write the terms ``and'', \selectlanguage{vietnamese}``và'', or ``or'', ``hoặc''. I write the symbols ``\&'' \& ``\emph{\texttt{/}}'', respectively, instead.
\end{remark}

%---------------------------------------------------------------------- --------%

\part{Literary Writings}

\section{Linguistics}
See, e.g., \href{https://en.wikipedia.org/wiki/Linguistics}{Wikipedia\texttt{/}linguistics}\footnote{\textbf{linguistics} [n] [uncountable] the scientific study of language or of particular languages.}.

%---------------------------------------------------------------------- --------%

\chapter{The Elements of Style}

\paragraph*{Content.} See \href{https://en.wikipedia.org/wiki/The_Elements_of_Style}{Wikipedia\texttt{/}The Elements of Style}. ``Strunk concentrated on the cultivation of good writing \& composition; the original 1918 edition exhorted writers to ``omit needless words'', use the \href{https://en.wikipedia.org/wiki/Active_voice}{active voice}, \& employ \href{https://en.wikipedia.org/wiki/Parallelism_(grammar)}{parallelism} appropriately.'' [$\ldots$] ``The 3rd edition of \textit{The Elements of Style} (1979) features 54 points: a list of common word-usage errors; 11 rules of punctuation \& grammar; 11 principles of writing; 11 matters of form; \&, in Chap. V, 21 reminders for better style. The final reminder, the 21st, ``Prefer the standard to the offbeat\footnote{\textbf{offbeat} [a] [usually before noun] (\textit{informal}) different from what most people expect, \textsc{synonym}: \textbf{unconventional}.}'', is thematically integral\footnote{\textbf{integral} [a] \textbf{1.} being an essential part of something; \textbf{2.} [usually before noun] included as part of something, rather than supplied separately; \textbf{3.} [usually before noun] having all the parts that are necessary for something to be complete.} to the subject of \textit{The Elements of Style}, yet does stand as a discrete\footnote{\textbf{discrete} [a] (\textit{formal or specialist}) independent of other things of the same type, \textsc{synonym}: \textbf{separate}.} essay about writing lucid\footnote{\textbf{lucid} [a] \textbf{1.} clearly expressed; easy to understand, \textsc{synonym}: clear; \textbf{2.} able to think clearly, especially when somebody cannot usually do this.} prose\footnote{\textbf{prose} [n] [uncountable] writing that is not poetry.}. To write well, White advises writers to have the proper\footnote{\textbf{proper} [a] \textbf{1.} [only before noun] (\textit{especially British English}) right, appropriate or correct; according to the rules, \textsc{opposite}: \textbf{improper}; \textbf{2.} [only before noun] \textit{British English}) considered to be real \& of a good enough standard; \textbf{3.} socially \& morally acceptable, \textsc{opposite}: \textbf{improper}; \textbf{4.} [after noun] according to the most exact meaning of the word; \textbf{5.} \textbf{proper to somebody\texttt{/}something} belonging to a particular type of person or thing; natural in a particular situation or place.} mind-set, that they write to please themselves, \& that they aim for ``1 moment of felicity\footnote{\textbf{felicity} [n] \textbf{1.} [uncountable] great happiness; \textbf{2.} [uncountable] the quality of being well chosen or suitable; \textbf{3.} \textbf{felicities} [plural] well-chosen or successful features, especially in a speech or piece of writing.}'', a phrase by \href{https://en.wikipedia.org/wiki/Robert_Louis_Stevenson}{Robert Louis Stevenson}. Thus Strunk's 1918 recommendation:
\begin{quotation}
	``Vigorous\footnote{\textbf{vigorous} [a] \textbf{1.} involving physical strength, effort or energy; \textbf{2.} done with determination, energy or enthusiasm; \textbf{3.} strong \& healthy.} writing is concise\footnote{\textbf{concise} [a] giving only the information that is necessary \& important, using few words.}. A sentence should contain no unnecessary words, a paragraph no unnecessary sentences, for the same reason that a drawing should have no unnecessary lines \& a machine no unnecessary parts. This requires not that the writer make all his sentences short, or that he avoid all detail \& treat his subjects only in outline, but that he make every word tell.'' -- ``Elementary Principles of Composition'', \textit{The Element of Style} \cite{Strunk1918}''
\end{quotation}
[$\ldots$] ``The 4th edition of \textit{The Elements of Style} (2000), published 54 years after Strunk's death, omits his stylistic\footnote{\textbf{stylistic} [a] [only before noun] connected with the style that a writer, artist or musician uses.} advice about masculine\footnote{\textbf{masculine} [a] \textbf{1.} having the qualities or appearance considered to be typical of men; connected with or like men; \textbf{2.} (in some languages) belonging to a class of nouns, pronouns or adjectives that have masculine gender, not feminine or neuter.} pronouns: ``unless the antecedent\footnote{\textbf{antecedent} [n] a thing or an event that exists or comes before something else \& has an influence on it; [a] existing or coming before something else, \& having an influence on it.} is or must be feminine''. In its place, the following sentence has been added: ``many writers find the use of the generic \textit{he} or \textit{his} to rename indefinite antecedents limiting or offensive.'' Further, the retitled entry ``They. He or she'', in Chap. IV: \textit{Misused Words \& Expressions}, advises the writer to avoid an ``unintentional emphasis on the masculine''.'' -- \href{https://en.wikipedia.org/wiki/The_Elements_of_Style#Content}{Wikipedia\texttt{/}The Element of Style\texttt{/}content}

\paragraph*{Reception.} ``\textit{The Elements of Style} was listed as 1 of the 100 best \& most influential\footnote{\textbf{influential} [a] having a lot of influence on the way that somebody\texttt{/}something behaves or develops, or on the way that somebody thinks.} books written in English since 1923 by \textit{Time} in its 2011 list. Upon its release, Charles Poor, writing for \href{https://en.wikipedia.org/wiki/The_New_York_Times}{\textit{The New York Times}}, called it ``a splendid\footnote{\textbf{splendid} [a] (\textit{especially British English}) \textbf{1.} very impressive; very beautiful; \textbf{2.} (\textit{old-fashioned}) excellent; very good, \textsc{synonym}: great.} trophy for all who are interested in reading \& writing.'' American poet \href{https://en.wikipedia.org/wiki/Dorothy_Parker}{Dorothy Parker} has, regarding the book, said:
\begin{quotation}
	``If you have any young friends who aspire to become writers, the 2nd-greatest favor you can do them is to present them with copies of \textit{The Elements of Style}. The 1st-greatest, of course, is to shoot them now, while they're happy.''
\end{quotation}
Criticism\footnote{\textbf{criticism} [n] \textbf{1.} [uncountable, countable] the act of expressing disapproval of somebody\texttt{/}something \& opinions about their faults or bad qualities; a statement showing disapproval; \textbf{2.} [uncountable] the work or activity of analyzing \& making fair, careful judgments about somebody\texttt{/}something, especially books, music, etc.} of \textit{Strunk \& White} has largely focused on claims that it has a \href{https://en.wikipedia.org/wiki/Linguistic_prescriptivism}{prescriptivist}\footnote{\textbf{prescriptive} [a] \textbf{1.} telling people what should be done or how something should be done; \textbf{2.} (\textit{linguistics}) telling people how a language should be used, rather than describing how it is used, \textsc{opposite}: \textbf{descriptive}.} nature, or that it has become a general \href{https://en.wikipedia.org/wiki/Anachronism}{anachronism}\footnote{\textbf{anachronism} [n] \textbf{1.} [countable] a person, a custom or an idea that seems old-fashioned \& does not belong to the present; \textbf{2.} [countable, uncountable] something that is placed, e.g., in a book or play, in the wrong period of history; the fact of placing something in the wrong period of history.} in the face of modern English usage.

In criticizing \textit{The Elements of Style}, \href{https://en.wikipedia.org/wiki/Geoffrey_Pullum}{Geoffrey Pullum}, professor of \href{https://en.wikipedia.org/wiki/Linguistics}{linguistics} at the \href{https://en.wikipedia.org/wiki/University_of_Edinburgh}{University of Edinburgh}, \& co-author of \href{https://en.wikipedia.org/wiki/The_Cambridge_Grammar_of_the_English_Language}{\textit{The Cambridge Grammar of the English Language}} (2002), said that:
\begin{quotation}
	``The book's toxic mix of \href{https://en.wikipedia.org/wiki/Linguistic_purism}{purism}\footnote{\textbf{purism} [n] [uncountable] the belief that things should be done in the traditional way \& that there are correct forms in languages, art, etc. that should be followed.}, \href{https://en.wikipedia.org/wiki/Atavism}{atavism}, \& personal \href{https://en.wikipedia.org/wiki/Eccentricity_(behavior)}{eccentricity}\footnote{\textbf{eccentricity} [n] \textbf{1.} [uncountable] behavior that people think is strange or unusual; the quality of being unusual \& different from other people; \textbf{2.} [countable, usually plural] an unusual act or habit.} is not underpinned\footnote{\textbf{underpin} [v] to support or form the basis of something.} by a proper grounding\footnote{\textbf{grounding} [n] [singular, uncountable] knowledge \& understanding of the basic parts of a subject; a basis for something.} in English grammar. It is often so misguided that the authors appear not to notice their own egregious\footnote{\textbf{egregious} [a] (\textit{formal}) extremely bad.} flouting\footnote{\textbf{flout} [v] \textbf{flout something} to show that you have no respect for a law, etc. by openly not obeying it, \textsc{synonym}: \textbf{defy}.} of its own rules $\ldots$ It's sad. Several generations of college students learned their grammar from the uninformed\footnote{\textbf{uninformed} [a] having or showing a lack of knowledge or information about something, \textsc{opposite}: informed.} bossiness\footnote{\textbf{bossiness} [n] [uncountable] (\textit{disapproving}) bossy behavior.} of \textit{Strunk \& White}, \& the result is a nation of educated people who know they feel vaguely\footnote{\textbf{vaguely} [adv] \textbf{1.} in a way that is not detailed or exact; \textbf{2.} slightly.} anxious\footnote{\textbf{anxious} [a] \textbf{1.} \textbf{anxious (about something)} feeling worried or nervous; \textbf{2.} wanting something very much.} \& insecure\footnote{\textbf{insecure} [a] \textbf{1.} not confident, especially about yourself or your abilities, \textsc{opposite}: \textbf{secure}; \textbf{2.} not safe or protected, \textsc{opposite}: \textbf{secure}.} whenever they write \textit{however} or \textit{than me} or \textit{was} or \textit{which}, but can't tell you why.''
\end{quotation}
Pullum has argued, e.g., that the authors misunderstood what constitutes the \href{https://en.wikipedia.org/wiki/English_passive_voice}{passive voice}\footnote{NQBH: Personally, I prefer the passive voice to the active one.}, \& he criticized their proscription\footnote{\textbf{proscription} [n] [countable, uncountable] (\textit{formal}) \textbf{proscription (against\texttt{/}on something)} the act of saying officially that something is banned; the stat of being banned.} of established \& unproblematic\footnote{\textbf{unproblematic} [a] not having or causing problems, \textsc{opposite}: \textbf{problematic}.} English usages, e.g. the \href{https://en.wikipedia.org/wiki/Split_infinitive}{split infinitive} \& the use of \textit{which} in a restrictive \href{https://en.wikipedia.org/wiki/English_relative_clause#That_or_which}{relative clause}. On \href{https://en.wikipedia.org/wiki/Language_Log}{Language Log}, a blog about language written by \href{https://en.wikipedia.org/wiki/Linguists}{linguists}, he further criticized \textit{The Elements of Style} for promoting \href{https://en.wikipedia.org/wiki/Linguistic_prescriptivism}{linguistic precriptivism} \& \href{https://en.wikipedia.org/wiki/Hypercorrection}{hypercorrection} among \href{https://en.wikipedia.org/wiki/Anglophones}{Anglophones}, \& called it ``the book that ate American's brain''.

\href{https://en.wikipedia.org/wiki/The_Boston_Globe}{\textit{The Boston Globe}}'s review described \textit{The Elements of Style Illustrated} (2005), with illustrations by Maira Kalman, as an ``aging zombie of a book $\ldots$ a hodgepodge\footnote{\textbf{hodgepodge} [n] (\textit{North American English}) (also \textbf{hotchpotch}, \textit{especially in British English}) [singular] (\textit{informal}) a number of things mixed together without any particular order or reason.}, its now-antiquated\footnote{\textbf{antiquated} [a] (\textit{usually disapproving}) (of things or ideas) old-fashioned \& no longer suitable for modern conditions, \textsc{synonym}: \textbf{outdated}.} \href{https://en.wikipedia.org/wiki/Pet_peeve}{pet peeves} jostling for\footnote{\textbf{jostle for} [phrasal verb] \textbf{jostle for something} to compete strongly \& with force for something.} space with 1970s taboos\footnote{\textbf{taboo} [n] \textbf{1.} \textbf{taboo (against\texttt{/}on something)} a cultural or religious custom that does not allow people to do, use or talk about a particular thing; \textbf{2.} \textbf{taboo (against\texttt{/}on something)} a general agreement not to do something or talk about something.} \& 1990s computer advice''.

Nevertheless, many contemporary\footnote{\textbf{contemporary} [a] \textbf{1.} belonging to the present time, \textsc{synonym} \textbf{modern}; \textbf{2.} (especially of people \& society) belonging to the same time as somebody\texttt{/}something else.} authors still recommend it highly. Their praise\footnote{\textbf{praise} [v] \textbf{1.} to express your approval or admiration for somebody\texttt{/}something; \textbf{2.} \textbf{praise God} to express your thanks to or your respect for God.} tends to focus on its characterization\footnote{\textbf{characterization} [n] [uncountable, countable] \textbf{1.} \textbf{characterization (of something)} the process of discovering or describing the qualities or features of something; the result of this process; \textbf{2.} the way in which the characters in a story, play or film are made to seem real.} of \fbox{good writing \& how to achieve it}, grammar being just 1 element of that purpose. In \href{https://en.wikipedia.org/wiki/On_Writing:_A_Memoir_of_the_Craft}{On writing} (2000, p. 11), \href{https://en.wikipedia.org/wiki/Stephen_King}{Stephen King} writes:
\begin{quotation}
	``There is little or no detectable \href{https://en.wikipedia.org/wiki/Bullshit}{bullshit} in that book. (Of course, it's short; at 85 pages it's much shorter than this one.) I'll tell you right now that every aspiring writer should read \textit{The Elements of Style}. Rule 17 in the chapter titled \textit{Principles of Composition} is `Omit needless words.' I will try to do that here.''
\end{quotation}
In 2011, Tim Skern remarked that \textit{The Elements of Style} ``remains the best book available on writing good English.''

In 2013, \href{https://en.wikipedia.org/wiki/Nevile_Gwynne}{Nevile Gwynne} reproduced \textit{The Elements of Style} in his work \href{https://en.wikipedia.org/wiki/Gwynne%27s_Grammar}{\textit{Gwynne's Grammar}}. Britt Peterson of the \href{https://en.wikipedia.org/wiki/Boston_Globe}{\textit{Boston Globe}} wrote that his inclusion of the book was a ``curious\footnote{\textbf{curious} [a] \textbf{1.} having a strong desire to know about something; \textbf{2.} strange \& unusual.} addition''.

In 2016, the Open Syllabus Project lists \textit{The Elements of Style} as the most frequently assigned text in US academic \href{https://en.wikipedia.org/wiki/Syllabus}{syllabuses}, based on an analysis of 933,635 texts appearing in over 1 million syllabuses.'' -- \href{https://en.wikipedia.org/wiki/The_Elements_of_Style#Reception}{Wikipedia\texttt{/}The Elements of Style\texttt{/}reception}

``The 1st writer I watched at work was my stepfather, E. B. White.\footnote{\selectlanguage{vietnamese} Sự ảnh hưởng, đặc biệt đến nhân cách \& việc lựa chọn nghề nghiệp, của những hình mẫu đầu tiên mà ta, 1 cách tình cờ hay được số phận sắp đặt, gặp gỡ trong cuộc đời.} Each Tuesday morning, he would close his study door \& sit down to write the ``Notes \& Comment'' page for \textit{The New Yorker}. The task was familiar to him -- he was required to file a few hundred words of editorial\footnote{\textbf{editorial} [a] [usually before noun] connected with the task of preparing something e.g. a newspaper, a book, or a television or radio programme, to be published or broadcast; [n] an important article in a journal or a newspaper, that expresses the editor's opinion about an issue.} of personal commentary on some topic in or out of the news that week -- but the sounds of his typewriter\footnote{\textbf{typewriter} [n] a machine that produces writing similar to print. It has keys that you press to make metal letters or signs hit a piece of paper through a long, narrow piece of cloth covered with ink ($=$ colored liquid).} \footnote{NQBH: I like the term ``typewriter'' in any literary scene., which sounds traditional \& sexy, opposite to personal notebooks\texttt{/}laptop now: modern \& robust.} from his room came in hesitant\footnote{\textbf{hesitant} [a] slow to speak or act because you feel uncertain, embarrassed or unwilling.} bursts\footnote{\textbf{burst} [v] \textbf{1.} [intransitive, transitive] to break open or apart, especially because of pressure from inside; to make something break in this way; \textbf{2.} [intransitive] \textbf{$+$ adv.\texttt{/}prep.} to go or come from somewhere suddenly; \textbf{burst into something} [phrasal verb] to start producing something suddenly \& with great force; [n] a short period of a particular activity or strong emotion that often starts suddenly.}, with long silences in between. Hours went by. Summoned at last for lunch, he was silent \& preoccupied\footnote{\textbf{preoccupied} [a] thinking \&\texttt{/}or worrying continuously about something so that you do not pay attention to other things.}, \& soon excused himself to get back to the job. When the copy went off at last, in the afternoon RFD pouch\footnote{\textbf{pouch} [n] \textbf{1.} a small bag, usually made of leather, \& often carried in a pocket or attached to a belt; \textbf{2.} a large bag for carrying letters, especially official ones; \textbf{3.} a pocket of skin on the stomach of some female marsupial animals, e.g. kangaroos, in which they carry their young; \textbf{4.} a pocket of skin in the cheeks of some animals, e.g. hamsters, in which they store food.} -- we were in Maine, a day's mail away from New York -- he rarely seemed satisfied. \fbox{``It isn't good enough.''}\footnote{``The quest for perfection can never end.''} he said sometimes, \fbox{``I wish it were better.''}

\fbox{Writing is hard}, even for authors who do it all the time. Less frequent practitioners -- the job applicant; the business executive with an annual report to get out; the high school senior with a Faulkner assignment; the graduate-school student with her thesis proposal; the writer of a letter of condolence\footnote{\textbf{condolence} [n] [countable, usually plural, uncountable] sympathy that you feel for somebody when a person in their family or that they know well has died; an expression of this sympathy.} -- often get stuck in an awkward\footnote{\textbf{awkward} [a] \textbf{1.} embarrassed; making you feel embarrassed; \textbf{2.} difficult to deal with, \textsc{synonym}: \textbf{difficult}; \textbf{3.} not convenient; \textbf{4.} difficult because of its shape or design; \textbf{5.} not moving in an easy way; not comfortable or elegant.} passage or find a muddle\footnote{\textbf{muddle} [v] (\textit{especially British English}) \textbf{1.} to put things in the wrong order or mix them up; \textbf{2.} muddle somebody (up) to confuse somebody; \textbf{3.} muddle somebody\texttt{/}something (up)$|$ \textbf{muddle A (up) with B} to confuse 1 person or thing with another, \textsc{synonym}: \textbf{mix up}.} on their screens, \& then blame themselves. What should be easy \& flowing looks tangled\footnote{\textbf{tangled} [a] \textbf{1.} twisted together in an untidy way; \textbf{2.} complicated, \& not easy to understand.} or feeble\footnote{\textbf{feeble} [a] \textbf{1.} very weak; \textbf{2.} not effective; not showing energy or effort.} or overblown\footnote{\textbf{overblown} [a] \textbf{1.} that is made to seem larger, more impressive or more important than it really is, \textsc{synonym}: \textbf{exaggerated}; \textbf{2.} (of flowers) past the best, most beautiful stage.} -- not what was meant at all. \fbox{What's wrong with me}, each one thinks. \fbox{Why can't I get this right?}''

[$\ldots$] White knew that a compendium\footnote{\textbf{compendium} [n] (plural \textbf{compendia, compendiums}) a collection of facts, drawings \& photographs on a particular subject, especially in a book.} of specific tips -- about singular \& plural verbs, parentheses, the ``that'' -- ``which'' scuffle\footnote{\textbf{scuffle} [n] \textbf{scuffle (with somebody) $|$ scuffle (between A \& B)} a short \& not very violent fight or struggle; [v] \textbf{1.} [intransitive] \textbf{scuffle (with somebody)} (of 2 or more people) to fight or struggle with each other for a short time, in a way that is not very serious; \textbf{2.} [intransitive] \textbf{$+$ adv.\texttt{/}prep.} to move quickly making a quiet rubbing noise.}, \& many others -- could clear up a recalcitrant\footnote{\textbf{recalcitrant} [a] (\textit{formal}) unwilling to obey rules or follow instructions; difficult to control.} sentence or subclause when quickly reconsulted\footnote{\textbf{consult} [v] \textbf{1.} [transitive, intransitive] to discuss something with somebody to get their permission for something, or to help you make a decision; \textbf{2.} [transitive, intransitive] to go to somebody for information or advice, especially an expert e.g. a doctor or lawyer; \textbf{3.} [transitive] \textbf{consult something} to look in or at something to get information, \textsc{synonym}: \textbf{refer to something}.}, \& that the larger principles needed to be kept in plain sight, like a wall sampler.

How simple they look, set down here in White's last chapter: ``\fbox{Write in a way that comes naturally},'' ``\fbox{Revise \& rewrite},'' ``\fbox{Do not explain too much},'' \& the rest; above all, the cleansing\footnote{\textbf{cleanse} [v] \textbf{1.} [transitive, intransitive] \textbf{cleanse (something)} to clean your skin or a wound; \textbf{2.} [transitive] \textbf{cleanse somebody (of\texttt{/}from something}) (\textit{literary}) to take away somebody's guilty feelings or sin.}, clarion\footnote{\textbf{clarion} [n] \textbf{1.} a medieval trumpet with clear shrill tones; \textbf{2.} the sound of or as if of a clarion' [a] brilliantly clear; loud \& clear.} ``Be clear.'' How often I have turned to them, in the book or in my mind, while trying to start or unblock or revise some piece of my own writing! They help -- they really do. They work. They are the way.

E. B. White's prose is celebrated for its ease\footnote{\textbf{ease} [n] [uncountable] \textbf{1.} lack of difficulty or effort, \textsc{opposite}: \textbf{difficulty}; \textbf{2.} the state of feeling relaxed or comfortable, without anxiety, problems or pain.} \& clarity\footnote{\textbf{clarity} [n] [uncountable] \textbf{1.} the quality of being expressed clearly; \textbf{2.} the ability to think about or understand something clearly; \textbf{3.} if a picture, substance or sound has clarity, you can see or hear it very clearly, or see through it easily.} -- just think of \textit{Charlotte's Web} -- but maintaining this standard required endless attention. When the new issue of \textit{The New Yorker} turned up in Maine, I sometimes saw him reading his ``Comment'' piece over to himself, with only a slightly different expression than the one he'd worn on the day it went off. Well, O.K., he seemed to be saying. \fbox{At least I got the elements right.}

This edition has been modestly\footnote{\textbf{modest} [a] \textbf{1.} fairly limited or small in amout; \textbf{2.} not expensive, rich or impressive; \textbf{3.} (of people, especially women, or their clothes) not showing too much of the body; not intended to attract attention, especially in a sexual way; \textbf{4.} (\textit{approving}) not talking much about your own abilities or possessions.} updated, with word processors \& air conditioners making their 1st appearance among White's references, \& with a light redistribution of genders to permit a feminine pronoun or female farmer to take their places among the males who once innocently\footnote{\textbf{innocent} [a] \textbf{1.} not guilty of a crime, etc.; not having done something wrong, \textsc{opposite}: \textbf{guilty}; \textbf{2.} [only before noun] suffering harm or being killed because of a crime, war, etc. although not directly involved in it; \textbf{3.} having little experience of evil or unpleasant things, or of sexual matters; \textbf{4.} not intended to cause harm or upset somebody, \textsc{synonym}: \textbf{harmless}.} served him.'' [$\ldots$] ``What is not here is anything about E-mail -- the rules-free, lower-case flow that cheerfully keeps us in touch these days. E-mail is conversation, \& it may be replacing the sweet \& endless talking we once sustained\footnote{\textbf{sustain} [v] \textbf{1.} \textbf{sustain somebody\texttt{/}something} to provide enough of what somebody\texttt{/}something needs in order to live or exist; \textbf{2.} to make something continue for some time without becoming less, \textsc{synonym}: \textbf{maintain}; \textbf{3.} \textbf{sustain something} (\textit{formal}) to experience something bad, \textsc{synonym}: \textbf{suffer}; \textbf{4.} \textbf{sustain something} to provide evidence to support an opinion, a theory, etc., \textsc{synonym}: \textbf{uphold}; \textbf{5.} \textbf{sustain something} (\textit{law}) to decide that a claim, etc. is valid, \textsc{synonym}: \textbf{uphold}.} (\& tucked away\footnote{\textbf{tuck away} [phrasal verb] \textbf{tuck something $\leftrightarrow$ away} \textbf{1.} \textbf{be tucked away} to be located in a quiet place, where not many people go; \textbf{2.} to hide something somewhere or keep it in a safe place; \textbf{3.} (\textit{British English, informal}) to eat a lot of food.}) within the informal letter. But we are all writers \& readers as well as communicators, with \fbox{the need at times to please \& satisfy ourselves} (as White put it) with the \fbox{clear \& almost perfect thought}.'' -- \cite[\textit{Foreword} by Roger Angell]{Strunk_White2019}

``I [E. B. White] passed the course, graduated from the university, \& \fbox{forgot the book but not the professor}.'' [$\ldots$]

``\textit{The Elements of Style}, when I [E. B. White] reexamined it in 1957, seemed to me to contain \fbox{rich deposits\footnote{\textbf{deposit} [n] \textbf{1.} a layer of a substance that has been left somewhere, especially by a river or flood, or is found at the bottom of a liquid; \textbf{2.} a layer of a substance that has formed naturally underground; \textbf{3.} [usually singular] \textbf{a deposit (on something)} a sum of money that is given as the 1st part of a larger payment; \textbf{4.} (in the British political system) the amount of money that a candidate in an election to Parliament has to pay, \& that is returned if they get enough votes.} of gold}. It was Will Strunk's \textit{parvum opus}\footnote{\textbf{parvum opus} [from Latin] [n] a little work, a small but meaningful work of an artist or writer.}, his attempt to cut the vast tangle\footnote{\textbf{tangle} [n] \textbf{1.} a twisted mass of threads, hair, etc. that cannot be easily separated; \textbf{2.} a lack of order; a confused state; \textbf{3.} (\textit{informal}) a disagreement or fight; [v] [transitive, intransitive] \textbf{tangle (something) up} to twist something into an untidy mass; to become twisted in this way.} of English rhetoric\footnote{\textbf{rhetoric} [n] [uncountable] \textbf{1.} (\textit{often disapproving} speech or writing that is intended to influence people, but that is not completely honest or sincere; \textbf{2.} the skill of using language in speech or writing in a special way that influences or entertains people.)} down to size \& write its rules \& principles on the head of a pin\footnote{\textbf{pin} [n] \textbf{1.} a short thin piece of stiff wire with a sharp point at 1 end \& a round head at the other, used to hold or attach things; \textbf{2.} a short piece of metal or other material, used to hold things together; \textbf{3.} a piece of metal with a sharp point, worn for decoration; \textbf{4.} 1 of the metal parts that stick out of an electric plug \& fit into a socket; [v] \textbf{pin something ($+$ adv.\texttt{/}prep.)} to attach something onto another thing or join things together with a pin, etc.; \textbf{pin something down} [phrasal verb] to explain or understand something exactly.}. Will himself had hung the tag ``little'' on the book; he referred to it sardonically\footnote{\textbf{sardonically} [adv] (\textit{disapproving}) in a way that shows that you think that you are better than other people \& do not take them seriously, \textsc{synonym}: \textbf{mockingly}.} \& with secret pride as ``the \textit{little book},'' always giving the word ``little'' a special twist, as though he were putting a spin on a ball. In its original form, it was a 43 page summation of the case for cleanliness, accuracy\footnote{\textbf{accuracy} [n] \textbf{1.} [uncountable] the state of being exact or correct, \textsc{opposite}: \textbf{inaccuracy}; \textbf{2.} [uncountable, countable] (\textit{specialist}) the degree to which the result of a measurement or calculation matches the correct value or a standard, \textsc{opposite}: \textbf{inaccuracy}.}, \& brevity\footnote{\textbf{brevity} [n] [uncountable] \textbf{1.} the quality of using few words when speaking or writing; \textbf{2.} \textbf{brevity (of something)} the fact of lasting a short time.} in the use of English. Today, 52 years later, its vigor\footnote{\textbf{vigor} [n] [uncountable] \textbf{1.} effort, energy, \& enthusiasm; \textbf{2.} \textbf{vigor (of something)} physical strength; good health.} is unimpaired\footnote{\textbf{unimpaired} [a] (\textit{formal}) not damaged or made less good, \textsc{opposite}: \textbf{impaired}.}, \& for sheer\footnote{\textbf{sheer} [a] \textbf{1.} [only before noun] used to emphasize the size, degree or amount of something; nothing but; \textbf{2.} very steep.} pith\footnote{\textbf{pith} [n] [uncountable] \textbf{1.} a soft dry white substance inside the skin of oranges \& some other fruits; \textbf{2.} the essential or most important part of something.} I think it probably sets a record that is not likely to be broken. Even after I got through tampering with\footnote{\textbf{tamper with} [phrasal verb] \textbf{tamper with something} to make changes to something without permission, especially in order to damage it, \textsc{synonym}: interfere with.} it, it was still a tiny thing, \fbox{a barely tarnished\footnote{\textbf{tarnished} [v] \textbf{1.} [intransitive, transitive] if mental tarnishes or something tarnishes it, it no longer looks bright \& shiny; \textbf{2.} [transitive, often passive] to damage the good opinion people have of somebody\texttt{/}something, \textsc{synonym}: \textbf{taint}; [n] [singular, uncountable] a thin layer on the surface of a metal that makes it look darker \& less bright.} gem\footnote{\textbf{gem} [n] \textbf{1.} (also less frequent \textbf{gemstone}) a precious stone that has been cut \& polished \& is used in jewellery, \textsc{synonym}: \textbf{jewel, precious stone}; \textbf{2.} a person, place or thing that is especially good.}}. 7 rules of usage, 11 principles of composition\footnote{\textbf{composition} [n] \textbf{1.} [uncountable] the different parts that something is made of; the way in which the different parts are organized; \textbf{2.} [countable] a piece of music or a poem; \textbf{3.} [uncountable] the act of writing a piece of music or a poem; \textbf{4.} [uncountable] (\textit{art}) the arrangement of people of objects in a painting, photograph or scene of a film.}, a few matters of form, \& a list of words \& expressions commonly misused -- that was the sum \& substance\footnote{\textbf{substance} [n] \textbf{1.} a type of solid, liquid or gas that has particular qualities; \textbf{2.} [countable] a drug or chemical, especially an illegal one, that has a particular effect on the mind or body; \textbf{3.} [uncountable] the most important or main part of something; \textbf{4.} [uncountable] (\textit{formal}) importance; \textbf{5.} [uncountable] the quality of being based on facts or the truth.} of Prof. Strunk's work. Somewhat audaciously\footnote{\textbf{audaciously} [adv] (\textit{formal}) in a way that shows you are willing to take risks or to do something that shocks people.}, \& in an attempt to give my publisher his money's worth, I [E. B. White] added a chapter called ``An Approach to Style,'' setting forth my own prejudices\footnote{\textbf{prejudice} [n] [uncountable, countable] an unreasonable dislike of a person, group, etc., especially when it is based on their race, religion, sex, etc.}, my notions of error, my articles of faith. This chapter (Chap. V) is addressed particularly to those who feel that English prose composition is not only a necessary skill but a sensible pursuit as well -- a way to spend one's days. I think Prof. Strunk would not object to that.''

[$\ldots$] ``I have now completed a 3rd revision. Chap. IV has been refurbished\footnote{\textbf{refurbish} [v] \textbf{refurbish something} to clean \& decorate a room, building, etc. in order to make it more attractive, more useful, etc.} with words \& expressions of a recent vintage\footnote{\textbf{vintage} [n] \textbf{1.} the wine that was produced in a particular year or place; the year in which it was produced; \textbf{2.} [usually singular] the period or season of gathering grapes for making wine; [a] [only before noun] \textbf{1.} \textbf{vintage} wine is of very good quality \& has been stored for several years; \textbf{2.} (British English) (of a vehicle) made between 1919 \& 1930 \& admired for its style \& interest; \textbf{3.} typical of a period in the past \& of high quality; the best work of the particular person; \textbf{4.} \textbf{vintage year} a particular good \& successful year.}; 4 rules of usage have been added to Chap. I. Fresh examples have been added to some of the rules \& principles, amplification\footnote{\textbf{amplification} [n] [uncountable] \textbf{1.} \textbf{amplification (of something)} the process of increasing the amplitude of an electrical signal; \textbf{2.} (biochemistry) \textbf{amplification (of something)} the process by which many copies of something, e.g. a gene, are made; \textbf{3.} \textbf{amplification (of something)} the action of making something greater or easier to notice; \textbf{4.} the action of adding details to a story, statement, etc.; details added to a story, statement, etc.} has reared\footnote{\textbf{rear} [v] \textbf{1.} \textbf{rear somebody\texttt{/}something} [often passive] to care for young children or animals until they are fully grown, \textsc{synonym}: \textbf{raise}; \textbf{2.} \textbf{rear something} to breed or keep animals or birds, e.g. on a farm; \textbf{something rears its head} [idiom] (of something unpleasant) to appear or happen; [n] (usually \textbf{the rear}) [singular] the back part of something; [a] [only before noun] at or near the back of something.} its head in a few places in the text where I felt an assault\footnote{\textbf{assault} [n] \textbf{1.} [uncountable, countable] the crime of attacking somebody physically; in law, \textbf{assault} is an act that threatens physical harm to somebody, whether or not actual harm is done: \textit{to commit}\texttt{/}\textit{be charged with assault}; \textbf{2.} [countable] (by an army, etc.) the act of attacking somebody\texttt{/}something, \textsc{synonym}: \textbf{attack}; \textbf{3.} [countable, usually singular, uncountable] an act of criticizing or attacking somebody\texttt{/}something severely; [v] \textbf{assault somebody} to attack somebody physically.} could successfully be made on the bastions\footnote{\textbf{bastion} [n] \textbf{1.} (\textit{formal}) a group of people or a system that protects a way of life or a belief when it seems that it may disappear; \textbf{2.} a place that military forces are defending.} of its brevity, \& in general the book has received a thorough overhaul\footnote{\textbf{overhaul} [n] an examination of a machine or system, including doing repairs on it or making changes to it; [v] \textbf{1.} \textbf{overhaul something} to examine every part of a machine, system, etc. \& make any necessary changes or repairs; \textbf{2.} \textbf{overhaul somebody} to come from behind a person you are competing against in a race \& go past them, \textsc{synonym}: \textbf{overtake}.} -- to correct errors, delete bewhiskered\footnote{\textbf{bewhiskered} [a] \textbf{1.} having whiskers; bearded; \textbf{2.} ancient, as a witticism, expression, etc.; pass\'e; hoary.} entries, \& enliven\footnote{\textbf{enliven} [v] (\textit{formal}) \textbf{enliven something} to make something more interesting or more fun.} the argument.

Prof. Strunk was a positive man. His book contains rules of grammar phrased as direct orders. In the main I [E. B. White] have not tried to soften his commands, or modify his pronouncements\footnote{\textbf{pronouncement} [n] a formal public statement.}, or remove the special objects of his scorn\footnote{\textbf{scorn} [n] [uncountable] a strong feeling that somebody\texttt{/}something is stupid or not good enough, usually shown by the way you speak, \textsc{synonym}: \textbf{contempt}; [v] \textbf{1.} \textbf{scorn somebody\texttt{/}something} to feel or show that you think somebody\texttt{/}something is stupid \& you do not respect them or it, \textsc{synonym}: \textbf{dismiss}; \textbf{2.} (\textit{formal}) to refuse to have or do something because you are too proud.}. I have tried, instead, to preserve\footnote{\textbf{preserve} [v] \textbf{1.} \textbf{preserve something} to keep a particular quality or feature; \textbf{2.} to keep something safe from harm, in good condition or in its original state; \textbf{3.} to prevent something from decaying, by treating it in a particular way; [n] [singular] an activity, job or interest that is thought to be suitable for 1 particular person or group of people.} the flavor\footnote{\textbf{flavor} [n] \textbf{1.} [uncountable] \textbf{flavor (of something)} how food or drink tastes, \textsc{synonym}: \textbf{taste}; \textbf{2.} [countable] a particular type of taste; \textbf{3.} [singular] a particular quality or atmosphere; \textbf{4.} [singular] \textbf{a\texttt{/}the flavor of something} an idea of what something is like.} of his discontent\footnote{\textbf{discontent} [n] (also \textbf{discontentment}) \textbf{1.} [uncountable] a feeling of being unhappy because you are not satisfied with a particular situation, \textsc{synonym}: \textbf{dissatisfaction}; \textbf{2.} [countable] \textbf{discontent (of somebody)} a thing that makes you feel unhappy \& not satisfied with a particular situation, \textsc{synonym}: \textbf{dissatisfaction}.} while slightly enlarging the scope of the discussion. \textit{The Elements of Style} does not pretend\footnote{\textbf{pretend} [v] \textbf{1.} to behave in a particular way, in order to make other people believe something that is not true; \textbf{2.} (usually used in negative sentences \& questions) to claim to be, do or have something, especially when this is not true.} to survey\footnote{\textbf{survey} [n] \textbf{1.} \textbf{survey} (of somebody\texttt{/}something) an investigation of the opinions, behavior, etc. of a particular group of people, which is usually done by asking them questions; \textbf{2.} an act of examining \& recording the measurements, features, etc. of an area of land in order to make a map or plan of it; \textbf{3.} \textbf{survey (of something)} a general study, view or description of something; [v] \textbf{1.} \textbf{survey somebody\texttt{/}something} to investigate the opinions or behavior of a group of people by asking them a series of questions; \textbf{2.} \textbf{survey something} to study \& give a general description of something; \textbf{3.} \textbf{survey something} to measure \& record the features of an area of land, e.g. in order to make a map or in preparation for building; \textbf{4.} \textbf{survey something} to look carefully at the whole of something, especially in order to get a general impression of it, \textsc{synonym}: \textbf{inspect}.} the whole field. Rather it proposes\footnote{\textbf{propose} [v] \textbf{1.} to suggest a plan or an idea for people to consider \& decide on; \textbf{2.} to suggest an explanation of something for people to consider.} to give in brief space the principal\footnote{\textbf{principal} [a] [only before noun] main; most important.} requirements of plain\footnote{\textbf{plain} [a] \textbf{1.} easy to see or understand, \textsc{synonym}: \textbf{clear}; \textbf{2.} [only before noun] expressed in a clear \& simple way, without using technical language; \textbf{3.} not trying to deceive anyone; honest \& direct; \textbf{4.} not decorated or complicated; simple; in computing, \textbf{plain text} is data representing text that is not written in code or using special formatting \& can be read, displayed or printed without much processing: \textit{Mathematical formulae are an example of content that cannot be represented satisfactorily via plain text.}; \textbf{5.} without marks or a pattern on it; \textbf{6.} [only before noun] (used for emphasis) simple; nothing but. \textsc{synonym}: \textbf{sheer}.} English style. It concentrates\footnote{\textbf{concentrate} [v] \textbf{1.} [transitive, often passive] \textbf{concentrate something $+$ adv.\texttt{/}prep.} to bring something together in 1 place; \textbf{2.} [intransitive, transitive] to give all your attention to something \& not think about anything else; \textbf{3.} [transitive] \textbf{concentrate something} to increase the strength of a substance by reducing its volume, e.g. by boiling it; \textbf{concentrate on something} [phrasal verb] to spend more time doing 1 particular thing than others; [n] [countable, uncountable] \textbf{concentrate (of something)} a substance that is made stronger because water or other substances have been removed.} on fundamentals\footnote{\textbf{fundamentals} [n] [plural] \textbf{fundamentals (of something)} the basic \& most important parts of something.}: the rules of usage \& principles of composition most commonly violated\footnote{\textbf{violet} [v] \textbf{1.} \textbf{violate something} to go against or refuse to obey a law, an agreement, etc.; \textbf{2.} \textbf{violate something} to not treat something with respect.}.

The reader will soon discover that these rules \& principles are in the form of sharp commands, Sergeant\footnote{\textbf{sergeant} [n] (abbr., \textbf{Sergt, Sgt}) \textbf{1.} a member of 1 of the middle ranks in the army \& the air force, below an officer; \textbf{2.} (in a UK) a police officer just below the rank of an inspector; \textbf{3.} (in the US) a police officer just below the rank of a lieutenant or caption.} Strunk snapping\footnote{\textbf{snap} [v] \textit{break} \textbf{1.} [transitive, intransitive] to break something suddenly with a sharp noise; to be broken in this way; \textit{take photograph} \textbf{2.} [transitive, intransitive] (\textit{informal}) to take a photograph; \textit{open}\texttt{/}\textit{close}\texttt{/}\textit{move into position} \textbf{3.} [intransitive, transitive] to move, or to move something, into a particular position quickly, especially with a sudden sharp noise; \textit{speak impatiently} \textbf{4.} [transitive, intransitive] to speak or say something in an impatient, usually angry, voice; \textit{of animal} \textbf{5.} [intransitive] \textbf{snap (at somebody\texttt{/}something)} to try to bite somebody\texttt{/}something, \textsc{synonym}: \textbf{nip}; \textit{lose control} \textbf{6.} [intransitive] to suddenly be unable to control your feelings any longer because the situation has become too difficult; \textit{fasten clothing} \textbf{7.} [intransitive, transitive] \textbf{snap (something)} (\textit{North American English}) to fasten a piece of clothing with a snap; \textit{in American football} \textbf{8.} [transitive] \textbf{snap something} (\textit{sport}) (in American football) to start play by passing the ball back between your legs.} orders to his platoon\footnote{\textbf{platoon} [n] a small group of soldiers that is part of a company \& commanded by a lieutenant.}. ``Do not join independent clauses with a comma.'' (Rule 5.) ``Do not break sentences in 2.'' (Rule 6.) ``Use the active voice.'' (Rule 14.) ``Omit\footnote{\textbf{omit} [v] \textbf{1.} to not include something\texttt{/}somebody, either deliberately or because you have forgotten it\texttt{/}them, \textsc{synonym}: \textbf{leave somebody\texttt{/}something out (of something)}; \textbf{2.} \textbf{omit to do something} to not do or fail to do something.} needless\footnote{\textbf{needless} [a] (of something bad) not necessary; that could be avoided, \textsc{synonym}: unnecessary.} words.'' (Rule 17.) ``Avoid a succession\footnote{\textbf{succession} [n] \textbf{1.} [countable, usually singular] a number of things or people that follow each other in time or order, \textsc{synonym}: \textbf{series}; \textbf{2.} [uncountable] the act of taking over an official position or title; \textbf{3.} [uncountable] the right to take over an official position or title, especially to become the king or queen of a country.} of loose\footnote{\textbf{loose} [a] \textbf{1.} not firmly fixed where it should be; that can become separated from something; \textbf{2.} not tightly packed together; not solid or hard; \textbf{3.} not strictly organized or controlled; \textbf{4.} not exact; not very careful; \textbf{5.} (of clothes) not fitting closely, \textsc{opposite}: \textbf{tight}; \textbf{6.} not tied together; not held in position by anything or contained in anything; \textbf{7.} (\textit{medical}) (of body waste) having too much liquid in it.} sentences.'' (Rule 18.) ``In summaries, keep to 1 tense.'' (Rule 21.) Each rule or principle is followed by a short hortatory\footnote{\textbf{hortatory} [a] trying to strongly encourage or persuade someone to do something.} essay, \& usually the exhortation\footnote{\textbf{exhortation} [n] [countable, uncountable] (\textit{formal}) \textbf{exhortation (to do something)} an act of trying very hard to persuade somebody to do something.} is followed by, or interlarded\footnote{\textbf{interlard} [v] (used with object) \textbf{1.} to diversify by adding or interjecting something unique, striking, or contrasting (usually followed by \textit{with}); \textbf{2.} (of things) to be intermixed in.} with, examples in parallel columns -- the true vs. the false, the right vs. the wrong, the timid\footnote{\textbf{timid} [a] shy \& nervous; not brave.} vs. the bold, the ragged\footnote{\textbf{ragged} [a] \textbf{1.} (of clothes) old \& torn, \textsc{synonym}: \textbf{shabby}; \textbf{2.} (of people) wearing old or torn clothes; \textbf{3.} having an outline, an edge or a surface that is not straight or even; \textbf{4.} not smooth or regular; not showing control or careful preparation; \textbf{5.} (\textit{informal}) very tired, especially after physical effort.} vs. the trim\footnote{\textbf{trim} [v] \textbf{1.} \textbf{trim something} to make something neater, smaller, better, etc., by cutting parts from it; \textbf{2.} to cut away unnecessary parts from something; \textbf{3.} [usually passive] \textbf{trim something (with something)} to decorate something, especially around its edges.}. From every line there peers out at me the puckish\footnote{\textbf{puckish} [a] [usually before noun] (\textit{literary}) enjoying playing tricks on other people, \textsc{synonym}: \textbf{mischievous}.} face of my professor, his short hair parted neatly\footnote{\textbf{neat} [a] \textbf{1.} in good order; carefully done or arranged; \textbf{2.} simple but clever; \textbf{3.} containing or made out of just 1 substance; not mixed with anything else.} in the middle \& combed down over his forehead, his eyes blinking incessantly\footnote{\textbf{incessantly} [adv] (\textit{usually disapproving}) without stopping, \textsc{synonym}: \textbf{constantly}.} behind steel-rimmed spectacles\footnote{\textbf{spectacle} [n] \textbf{1.} [countable, uncountable] \textbf{spectacle (of something)} a performance or an event that is very impressive \& exciting to look at; \textbf{2.} [singular] \textbf{spectacle (of something)} an unusual, embarrassing or sad sight or situation that attracts a lot of attention; \textbf{3.} (\textbf{spectacles}) [plural] [\textit{formal}] $=$ \textbf{glass}.} as though he had just emerged into strong light, his lips nibbling each other like nervous horses, his smile shuttling to and fro under a carefully edged mustache.

``Omit needless words!'' cries the author on p. 23, \& into that imperative\footnote{\textbf{imperative} [n] a thing that is very important \& needs immediate attention or action; [a] [not usually before noun] very important \& needing immediate attention or action, \textsc{synonym}: \textbf{vital}.} Will Strunk \fbox{really put his heart \& soul}. In the days when I was sitting in his class, he omitted so many needless words, \& omitted them so forcibly\footnote{\textbf{forcibly} [adv] \textbf{1.} in a way that involves the use of physical force; \textbf{2.} in a way that makes something very clear.} \& with such eagerness\footnote{\textbf{eager} [a] very interested \& excited by something that is going to happen or about something that you want to do, \textsc{synonym}: \textbf{keen}.} \& obvious relish\footnote{\textbf{relish} [v] to get great pleasure from something; to want very much to do or have something, \textsc{synonym}: \textbf{enjoy}; [n] \textbf{1.} [uncountable] great pleasure; \textbf{2.} [uncountable, countable] a cold, thick, spicy sauce made from fruit \& vegetables that have been boiled, that is served with meat, cheese, etc.}, that he often seemed in the position of having shortchanged\footnote{\textbf{short-change} [v] [often passive] \textbf{1.} \textbf{short-change somebody} to give back less than the correct amount of money to somebody who has paid for something with more than the exact price; \textbf{2.} \textbf{short-change somebody} to treat somebody unfairly by not giving them what they have earned or deserve.} himself -- a man left with nothing more to say yet with time to fill, a radio prophet who had outdistanced\footnote{\textbf{outdistance} [v] \textbf{outdistance somebody\texttt{/}something} to leave somebody\texttt{/}something behind by going faster, further, etc.; to be better than somebody\texttt{/}something, \textsc{synonym}: \textbf{outstrip}.} the clock. Will Strunk got out of this predicament\footnote{\textbf{predicament} [n] a difficult or an unpleasant situation, especially one where it is difficult to know what to do, \textsc{synonym}: \textbf{quandary}.} by a simple trick: he uttered\footnote{\textbf{utter} [v] \textbf{utter something} to make a sound with your voice; to say something.} every sentence 3 times. When he delivered his oration\footnote{\textbf{oration} [n] (\textit{formal}) a formal speech made on a public occasion, especially as part of a ceremony.} on brevity to the class, he leaned forward over his desk, grasped his coat lapels\footnote{\textbf{lapel} [n] 1 of the 2 front parts of the top of a coat or jacket that are joined to the collar \& are folded back.} in his hands, \&, in a husky\footnote{\textbf{husky} [a] \textbf{1.} (of a person of their voice) sounding deep, quiet \& rough, sometimes in an attractive way; \textbf{2.} (\textit{North American English}) with a large, strong body; [n] (North American English also \textbf{huskie}) a large strong dog with thick hair, used for pulling sledges across snow.}, conspiratorial\footnote{\textbf{conspiratorial} [a] \textbf{1.} connected with, or making you think of, a conspiracy ($=$ a secret plan to do something illegal); \textbf{2.} (of a person's behavior) suggesting that a secret is being shared.} voice, said, ``Rule 17. Omit needless words! Omit needless words! Omit needless word!''

He was a memorable\footnote{\textbf{memorable} [a] special, good or unusual \& therefore worth remembering; easy to remember.} man, friendly \& funny. Under the remembered sting of his kindly lash\footnote{\textbf{lash} [v] \textbf{1.} [intransitive, transitive] to hit somebody\texttt{/}something with great force, \textsc{synonym}: \textbf{pound}; \textbf{2.} [transitive] \textbf{lash somebody\texttt{/}something} to hit a person or an animal with a whip, rope, stick, etc., \textsc{synonym}: \textbf{beat}.}, I have been trying to omit needless words since 1919, \& although there are still many words that cry for omission \& the huge task will never be accomplished, it is exciting to me to reread to masterly Strunkian elaboration\footnote{\textbf{elaboration} [n] [uncountable, countable] \textbf{1.} the act of explaining or describing something in a more detailed way; \textbf{2.} the process of developing a plan, an idea, etc. \& making it complicated or detailed; \textbf{3.} \textbf{elaboration (of something)} (\textit{biology}) the production of a substance or structure from elements or simpler constituents in a natural process.} of this noble\footnote{\textbf{noble} [a] \textbf{1.} belonging to a family of high social rank, \textsc{synonym}: \textbf{aristocratic}; \textbf{2.} having or showing fine personal qualities that people admire, e.g. courage, honesty \& care for others; [n] a person who comes from a family of high social rank; a member of the nobility, \textsc{synonym}: \textbf{aristocratic}.} theme\footnote{\textbf{theme} [n] the subject of a talk, piece of writing, exhibition, etc.; an idea that keeps returning in a piece of research or a work of art or literature.}. It goes:
\begin{quotation}
	\textit{Vigorous writing is concise. A sentence should contain no unnecessary words, a paragraph no unnecessary sentences, for the same reason that a drawing should have no unnecessary lines \& a machine no unnecessary parts. This requires not that the writer make all sentences short or avoid all detail \& treat subjects only in outline, but that every word tell.}
\end{quotation}
There you have a short, valuable essay on the nature \& beauty of brevity -- 59 words that could change the world. Having recovered from his adventure in prolixity\footnote{\textbf{prolixity} [n] [uncountable] (\textit{formal}) the fact of using too many words \& therefore creating a piece of writing, a speech, etc., that is boring.} (59 words were a lot of words in the tight world of William Strunk Jr.), the professor proceeds to give a few quick lessons in pruning\footnote{\textbf{pruning} [n] [uncountable] \textbf{1.} the activity of cutting off some of the branches from a tree, bush, etc. so that it will grow better \& stronger; \textbf{2.} the act of making something smaller by removing parts; the act of cutting out parts of something.}. Students learn to cut the dead-wood from ``this is a subject that,'' reducing it to ``this subject,'' a saving of 3 words. They learn to trim\footnote{\textbf{trim} [v] \textbf{1.} \textbf{trim something} to make something neater, smaller, better, etc., by cutting parts from it; \textbf{2.} to cut away unnecessary parts from something; \textbf{3.} [usually passive] \textbf{trim something (with something)} to decorate something, especially around its edges.} ``used for fuel purposes'' down to ``used for fuel.'' They learn that they are being chatterboxes\footnote{\textbf{chatterbox} [n] (\textit{informal}) a person who talks a lot, especially a child.} when they say ``the question as to whether'' \& that they should just say ``whether'' -- a saving of 4 words out of a possible 5.

The professor devotes\footnote{\textbf{devote} [v] \textbf{devote yourself to somebody\texttt{/}something} to give most of your time, energy or attention to somebody\texttt{/}something, \textsc{synonym}: \textbf{dedicate}; \textbf{devote something to something}: to give an amount of time, attention or resources to something.} a special paragraph to the vile\footnote{\textbf{vile} [a] \textbf{1.} (\textit{informal}) extremely unpleasant or bad, \textsc{synonym}: \textbf{disgusting}; \textbf{2.} (\textit{formal}) morally bad; completely unacceptable, \textsc{synonym}: \textbf{wicked}.} expression \textit{the fact that}, a phrase that causes him to quiver\footnote{\textbf{quiver} [v] to shake slightly; to make a slight movement, \textsc{synonym}: \textbf{tremble}; [n] \textbf{1.} an emotion that has an effect on your body; a slight movement in part of your body; \textbf{2.} a case for carrying arrows.} with revulsion\footnote{\textbf{revulsion} [n] [uncountable, singular] (\textit{formal}) a strong feeling of horror, \textsc{synonym}: \textbf{disgust, repugnance}.}. The expression, he says, should be ``revised out of every sentence in which it occurs.'' But a shadow\footnote{\textbf{shadow} [n] \textbf{1.} [countable] the dark area or shape produced by somebody\texttt{/}something coming between light \& a surface; \textbf{2.} [uncountable] (\textbf{shadows} [plural]) darkness, especially that produced by somebody\texttt{/}something coming between light \& a surface; \textbf{3.} [singular] the strong (usually bad) influence of somebody\texttt{/}something.} of gloom\footnote{\textbf{gloom} [n] \textbf{1.} [uncountable, singular] a feeling of being sad \& without hope, \textsc{synonym}: \textbf{depression}; \textbf{2.} [uncountable] (\textit{literary}) almost total darkness.} seems to hang over the page, \& you feel that he knows how hopeless his cause is. I suppose I have written \textit{the fact that} a thousand times in the heat of composition, revised it out maybe 500 times in the cool aftermath\footnote{\textbf{aftermath} [n] [usually singular] the situation that exists as a result of an important (\& usually unpleasant) event, especially a war, an accident, etc.}. To be batting only .500 this late in the season, to fail half the time to connect with this fat pitch, saddens me, for it seems a betrayal of the man who showed me how to swing\footnote{\textbf{swing} [v] \textbf{1.} [intransitive, transitive] to change to make somebody\texttt{/}something change from 1 opinion or mood to another; \textbf{2.} [intransitive, transitive] to turn or change direction suddenly; to make something do this; \textbf{3.} [intransitive, transitive] to move backwards or forwards or from side to side while hanging from a fixed point; to make something do this; \textbf{4.} [intransitive, transitive] to move or make something move with a wide curved movement; [n] a change from 1 opinion or situation to another; the amount by which something changes.} at it \& made the swinging seem worthwhile.

I treasure\footnote{\textbf{treasure} [n] \textbf{1.} [uncountable] a collection of valuable things e.g. gold, silver \& jewelery; \textbf{2.} [countable, usually plural] a highly valued object; \textbf{3.} [singular] a person who is much loved or valued; [v] \textbf{treasure something} to have or keep something that you love \& that is extremely valuable to you, \textsc{synonym}: \textbf{cherish}.} \textit{The Elements of Style} for its sharp\footnote{\textbf{sharp} [a] \textbf{1.} [usually before noun] (especially of a change in something) sudden \& fast; \textbf{2.} [usually before noun] (especially of a difference in something) clear \& definite; \textbf{3.} (especially of something that can cut or make a hole in something) having a fine edge or point, \textsc{opposite}: \textbf{blunt}; \textbf{4.} (of a person or what they say) critical or severe; \textbf{5.} (of a physical feeling or an emotion) very strong \& sudden, often like being cut or wounded, \textsc{synonym}: \textbf{intense}; \textbf{6.} changing direction suddenly; \textbf{7.} (of people or their minds or eyes) quick to notice or understand things or to react.} advice, but I treasure it even more for the \fbox{audacity}\footnote{\textbf{audacity} [n] [uncountable] behavior that is brave but likely to shock or offend people, \textsc{synonym}: \textbf{nerve}.} \& self-confidence\footnote{\textbf{self-confidence} [n] [uncountable] confidence in yourself \& your abilities, \textsc{synonym}: \textbf{self-assurance, confidence}.} of its author. \fbox{Will knew where he stood.} He was so sure of where he stood, \& made his position so clear \& so plausible, that his peculiar\footnote{\textbf{peculiar} [a] belonging to or connected with 1 particular place, situation, person, etc., \& not others.} stance\footnote{\textbf{stance} [n] the opinions that somebody has about something \& expresses publicly, \textsc{synonym}: \textbf{position}.} has continued to invigorate\footnote{\textbf{invigorate} [v] \textbf{1.} \textbf{invigorate somebody} to make somebody feel healthy \& full of energy; \textbf{2.} \textbf{invigorate something} to make a situation, an organization, etc. efficient \& successful.} me -- \&, I am sure, thousands of other ex-students -- during the years that have intervened\footnote{\textbf{intervene} [v] \textbf{1.} [intransitive] to become involved in a situation in order to improve it or stop it from getting worse; \textbf{2.} [intransitive] to happen in the time between events; \textbf{3.} [intransitive] to exist or be found in the space between things; \textbf{4.} [intransitive] to happen in a way  that delays something or prevents it from happening.} since our 1st encounter. He had a number of likes \& dislikes that were almost as whimsical\footnote{\textbf{whimsical} [a] unusual \& not serious in a way that is either funny or annoying.} as the choice of a necktie, yet he made them seem utterly\footnote{\textbf{utter} [a] [only before noun] used to emphasize how complete something is, \textsc{synonym}: \textbf{total}; [v] \textbf{utter something} to make a sound with your voice; to say something.} convincing. He disliked the word \textit{forceful}\footnote{\textbf{forceful} [a] \textbf{1.} (of people) expressing opinion firmly \& clearly in a way that persuades other people to believe them, \textsc{synonym}: \textbf{assertive}; \textbf{2.} (of opinions, etc.) expressed firmly \& clearly so that other people believe them; \textbf{3.} using force; \textbf{4.} (of action) strong \& effective.} \& advised us to use \textit{forcible}\footnote{\textbf{forcible} [a] [only before noun] involving the use of physical force.} instead. He felt that the word \textit{clever}\footnote{\textbf{clever} [a] \textbf{1.} (especially British English) quick at learning \& understanding things, \textsc{synonym}: \textbf{intelligent}; \textbf{2.} \textbf{clever (at something\texttt{/}doing somethign)} (especially British English) skillful; \textbf{3.} showing intelligence or skill, e.g. in the design of an object, in an idea or somebody's actions.} was greatly overused: ``It is best restricted to ingenuity\footnote{\textbf{ingenuity} [n] [uncountable] the ability to invent things or solve problems in clever new ways, \textsc{synonym}: \textbf{inventiveness}.} displayed in small matters.'' He despised\footnote{\textbf{despise} [v] (not used in the progressive tenses) to dislike \& have no respect for somebody\texttt{/}something.} the expression \textit{student body}, which he termed gruesome\footnote{\textbf{gruesome} [a] very unpleasant \& filling you with horror, usually because it is connected with death or injury.}, \& made a special trip downtown to the \textit{Alumni News} office 1 day to protest\footnote{\textbf{protest} [n] [uncountable, countable] the expression of strong disagreement with or opposition to something; a statement or an action that shows this.} the expression \& suggest that \textit{studentry} be substituted\footnote{\textbf{substitute} [v] [intransitive, transitive] to take the place of somebody\texttt{/}something else; to use somebody\texttt{/}something instead of somebody\texttt{/}something else; [n] a person or thing that you use or have instead of the usual one.} -- a coinage\footnote{\textbf{coinage} [n] \textbf{1.} [uncountable] the coins used in a particular place or at a particular time; coins of a particular type; \textbf{2.} [countable, uncountable] a word or phrase that has been invented recently; the process of inventing a word or phrase.} of his own, which he felt was similar to \textit{citizenry}\footnote{\textbf{citizenry} [n] [singular $+$ singular or plural verb] (\textit{formal}) all the citizens of a particular town, country, etc.}. I am told that the \textit{News} editor was so charmed by the visit, if not by the word, that he ordered the student body buried, never to rise again. \textit{Studentry} has taken its place. It's not much of an improvement, but it does sound less cadaverous\footnote{\textbf{cadaverous} [a] (\textit{literary}) (of a person) extremely pale, thin \& looking ill.}, \& it made Will Strunk quite happy.

Some years ago, when the heir\footnote{\textbf{heir} [n] \textbf{1.} a person who has the legal right to receive somebody's property, money or title when that person dies; \textbf{2.} a person who is thought to continue the work or a tradition started by somebody else.} to the throne of England was a child, I noticed a headline in the \textit{Times} about Bonnie Prince Charlie: ``CHARLES' TONSILS OOUT.'' Immediately Rule 1 leapt to mind.
\begin{quotation}
	\textbf{1.} Form the possessive singular of nouns by adding \textit{'s}. Follow this rule whatever the final consonant\footnote{\textbf{consonant} [n] \textbf{1.} (phonetics) a speech sound made by completely or partly stopping the flow of air being breathed out through the mouth; \textbf{2.} a letter of the alphabet that represents a consonant sound.}. Thus write, \textit{Charles's friend, Burns's poems, the witch's malice\footnote{\textbf{malice} [n] [uncountable] a desire to harm somebody caused by a feeling of hate.}}.
\end{quotation}
Clearly, Will Strunk had foreseen\footnote{\textbf{foreseen} [v] to know about something before it happens.}, as far back as 1918, the dangerous tonsillectomy\footnote{\textbf{tonsillectomy} [n] (\textit{medical}) a medical operation to remove the tonsils.} of a prince, in which the surgeon removes the tonsils \& the \textit{Times} copy desk removes the final \textit{s}. He started his book with it. I commend Rule 1 to the \textit{Times}, \& I trust that Charles's throat, not Charles' throat, is in fine shape today.

Style rules of this sort are, of course, somewhat a matter of individual preference\footnote{\textbf{preference} [n] \textbf{1.} [countable, usually singular, uncountable] a greater interest in or desire for somebody\texttt{/}something than somebody\texttt{/}something else; \textbf{2.} [countable] a thing that is liked better or best.}, \& even the established rules of grammar are open to challenge. Prof. Strunk, although 1 of the most inflexible\footnote{\textbf{inflexible} [a] \textbf{1.} (\textit{disapproving}) that cannot be changed or made more suitable for a particular situation, \textsc{synonym}: \textbf{rigid}; \textbf{2.} (\textit{disapproving}) (of people or organizations) unwilling to change their opinions, decision or behavior.} \& choosy\footnote{\textbf{choosy} [a] (\textit{informal}) careful in choosing; difficult to please, \textsc{synonym}: \textbf{fussy, picky}.} of men, was quick to acknowledge\footnote{\textbf{acknowledge} [v]  \textbf{1.} to accept that something is true or exists; \textbf{2.} to accept that somebody\texttt{/}something has a particular quality, importance or status, \textsc{synonym}: \textbf{recognize}; \textbf{3.} \textbf{acknowledge somebody\texttt{/}something} to publicly express thanks fo help or inspiration; \textbf{4.} \textbf{acknowledge something} to tell somebody that you have received something that they sent to you.} the fallacy\footnote{\textbf{fallacy} [n] \textbf{1.} [countable] a false idea that many people believe is true; \textbf{2.} [uncountable, countable] a false way of thinking about something.} of inflexibility \& the danger of doctrine\footnote{\textbf{doctrine} [n] \textbf{1.} [countable, uncountable] \textbf{doctrine (of something)} a belief or principle, or set of beliefs or principles, held by a religion, a political party or a legal system; \textbf{2.} (\textbf{Doctrine}) [countable] (US) a statement of government policy, especially foreign policy.}. ``It is an old observation,'' he wrote, ``that the best writers sometimes disregard\footnote{\textbf{disregard} [v] \textbf{disregard something} to not consider something; to treat something as unimportant, \textsc{synonym}: \textbf{ignore}.} the rules of rhetoric\footnote{\textbf{rhetoric} [n] [uncountable] \textbf{1.} (\textit{often disapproving}) speech or writing that is intended to influence people, but that is not completely honest or sincere; \textbf{2.} the skill of using language in speech or writing in a special way that influences or entertains people.}. \texttt{[stop translating here]} When they do so, however, the reader will usually find in the sentence some compensating merit, attained at the cost of the violation. Unless he is certain of doing as well, he will probably do best to follow the rules.''

It is encouraging to see how perfectly a book, even a dusty rule book, perpetuates \& extends the spirit of a man. Will Strunk loved the clear, the brief, the bold, \& his book is clear, brief, bold. Boldness is perhaps its chief distinguishing mark. On p. 26, explaining 1 of his parallels, he says, ``The lefthand version gives the impression that the writer is undecided or timid, apparently unable or afraid to choose 1 form of expression \& hold to it.'' \& his original Rule 11 was ``Make definite assertions.'' That was Will all over. He scorned the vague, the tame, the colorless, the irresolute. He felt it was worse to be irresolute than to be wrong. I remember a day in class when he leaned far forward, in his characteristic pose -- the pose of a man about to impart a secret -- \& croaked, ``If you don't know how to pronounce a word, say it loud! If you don't know how to pronounce a word, say it loud!'' This comical piece of advice struck me as sound at the time, \& I still respect it.\fbox{ Why compound ignorance with inaudibility?} \fbox{Why run \& hide?}

All through \textit{The Elements of Style} one finds evidence of the author's deep sympathy for the reader. Will felt that the reader was in serious trouble most of the time, floundering in a swamp, \& that it was the duty of anyone attempting to write English to drain this swamp quickly \& get the reader up on dry ground, or at least to throw a rope. In revising the text, I have tried to hold steadily in mind this belief of his, this concern for the bewildered reader.

In the English classes of today, ``the little book'' is surrounded by longer, lower textbooks -- books with permissive steering \& automatic transitions. Perhaps the book has become something of a curiosity. To me, it still seems to maintain its original poise, standing, in a drafty time, erect, resolute, \& assured. I still find the Strunkian wisdom a comfort, the Strunkian humor a delight, \& the Strunkian attitude forward right-\&-wrong a blessing undisguised.'' -- \cite[Introduction (by E. B. White)]{Strunk_White2019}

\section{Elementary Rules of Usage}
This section is devoted to study \cite[Chap. 1]{Strunk_White2019}.

\subsection{Form the possessive singular of nouns by adding \textit{'s}}
``Follow this rule whatever the final consonant. Thus write \textit{Charles's friend, Burns's poems, the witch's malice}. Exceptions are the possessive of ancient proper names in \textit{-es} \& \textit{-is}, the possessive \textit{Jesus'}, \& such forms as \textit{for conscience' sake, for righteousness' sake}. But such forms as \textit{Achilles' heel, Moses' laws, Isis' temple} are commonly replaced by: \textit{the laws of Moses, the temple of Isis}. The pronominal possessives \textit{hers, its, theirs, yours, \& ours} have no apostrophe. Indefinite pronouns, however, use the apostrophe to show possession: \textit{one's rights, somebody else's umbrella}. A common error is to write \textit{it's} for \textit{its}, or vice versa. The 1st is a contraction, meaning ``it is.'' The 2nd is a possessive.

\begin{example}
	It's a wise dog that scratches its own fleas.''
\end{example}
-- \cite[Chap. 1, Sect. 1, p. 14]{Strunk_White2019}

\subsection{In a series of $\ge 3$ terms with a single conjunction, use a comma after each term except the last}
``Thus write,

\begin{example}
	red, white, \& blue; gold, silver, or copper
	
	He opened the letter, read it, \& made a note of its contents.
\end{example}
This comma is often referred to as the ``serial'' comma. In the names of business firms the last comma is usually omitted. Follow the usage of the individual firm.

\begin{example}
	Little, Brown \& Company; Donaldson, Lufkin \& Jenrette''
\end{example}
-- \cite[Chap. 1, Sect. 2, p. 15]{Strunk_White2019}

\subsection{Enclose parenthetic expressions between commas}

\begin{example}
	``The best way to see a country, unless you are pressed for time, is to travel on foot.
\end{example}
This rule is difficult to apply; it is frequently hard to decide whether a single word, e.g. \textit{however}, or a brief phrase is or is not parenthetic. If the interruption to the flow of the sentence is but slight, the commas may be safely omitted. But whether the interruption is slight or considerable, never omit 1 comma \& leave the other. There is no defense for such punctuation as

\begin{example}
	Marijories husband, Colonel Nelson paid us a visit yesterday.
	
	My brother you will be pleased to hear, is now in perfect health.
\end{example}
Dates usually contain parenthetic words or figures. Punctuate as follows:

\begin{example}
	February to July, 1992; April 6, 1985; Wednesday, November 14, 1990
\end{example}
Note that it is customary to omit the comma in \textit{6 April 1988}. The last form is an excellent way to write a date; the figures are separated by a word \& are, for that reason, quickly grasped.

A name or a title in direct address is parenthetic.

\begin{example}
	If, Sir, you refuse, I cannot predict what will happen.
	
	Well, Susan, this is a fine mess you are in.
\end{example}
The abbreviations \textit{etc., i.e.,} \& \textit{e.g.,} the abbreviations for academic degrees, \& titles that follow a name are parenthetic \& should be punctuated accordingly.

\begin{example}
	Letters, packages, etc., should go here.
	
	Horace Fulsome, Ph.D., presided.
	
	Rachel Simonds, Attorney
	
	The Reverend Harry Lang, S.J.
\end{example}
No comma, however, should separate a noun from a restrictive term of identification.

\begin{example}
	Billy the Kid; The novelist Jane Austen; William the Conqueror; The poet Sappho
\end{example}
Although \textit{Junior}, with its abbreviation \textit{Jr.}, has commonly been regarded as parenthetic, logic suggests that it is, in fact, restrictive \& therefore not i need of a comma, e.g., \textit{James Wright Jr}.

Nonrestrictive relative clauses are parenthetic, as are similar clauses introduced by conjunctions indicating time or place. Commas are therefore needed. A nonrestrictive clause is one that does not serve to identify or define the antecedent noun.

\begin{example}
	The audience, which had at 1st been indifferent, became more \& more interested.
	
	In 1769, when Napoleon was born, Corsica had but recently been acquired by France.
	
	Nether Stowey, where Coleridge wrote The Rime of the Ancient Mariner, is a few miles from Bridgewater.
\end{example}
In these sentences, the clauses introduced by \textit{which, when, \& where} are nonrestrictive; they do not limit or define, they merely add something. In the 1st example, the clause introduced by \textit{which} does not sever to tell which of several possible audiences is meant; the reader presumably knows that already. The clause adds, parenthetically, a statement supplementing that in the main clause. Each of the 3 sentences is a combination of 2 statements that might have been made independently.

\begin{example}
	The audience was at 1st indifferent. Later it became more \& more interested.
	
	Napoleon was born in 1769. At that time Corsica had but recently been acquired by France.
	
	Coleridge wrote The Time of the Ancient Mariner at Nether Stowey. Nether Stowey is a few miles from Bridgewater.
\end{example}
Restrictive clauses, by contrast, are not parenthetic \& are not set off by commas. Thus

\begin{example}
	People who live in glass houses shouldn't throw stones.
\end{example}
Here the clause introduced by \textit{who} does serve to tell which people are meant; the sentence, unlike the sentences above, cannot be split into 2 independent statements. The same principle of comma use applies to participial phrases \& to appositives.

\begin{example}
	People sitting in the rear couldn't hear, \emph{(restrictive)}
	
	Uncle Bert, being slightly deaf, moved forward, \emph{(non-restrictive)}
	
	My cousin Bob is a talented harpist, \emph{(restrictive)}
	
	Our oldest daughter, Mary, sings, \emph{(nonrestrictive)}
\end{example}
When the main clause of a sentence is preceded by a phrase or a subordinate clause, use a comma to set off these elements.

\begin{example}
	Partly by hard fighting, partly by diplomatic skill, they enlarged their dominions to the east \& rose to royal rank with the possession of Sicily.''
\end{example}
-- \cite[Chap. 1, Sect. 3, pp. 16--17]{Strunk_White2019}

\subsection{Place a comma before a conjunction introducing an independent clause}

\begin{example}
	``The early records of the city have disappeared, \& the story of its 1st years can no longer be reconstructed.
	
	The situation is perilous, but there is still 1 chance of escape.
\end{example}
2-part sentences of which the 2nd member is introduced by \textit{as} (in the sense of ``because''), \textit{for, or, nor}, or \textit{while} (in the sense o ``\& at the same time'') likewise require a comma before the conjunction.

If a dependent clause, or an introductory phrase requiring to be set off by a comma, precedes the 2nd independent clause, no comma is needed after the conjunction.

\begin{example}
	The situation is perilous, but if we are prepared to act promptly, there is still 1 chance of escape.
\end{example}
When the subject is the same for both clauses \& is expressed only once, a comma is useful if the connective is \textit{but}. When the connective is \textit{and}, the comma should be omitted if the relation between the 2 statements is close or immediate.

\begin{example}
	I have heard the arguments, but am still unconvinced.
	
	He has had several years' experience \& is thoroughly competent.''
\end{example}
-- \cite[Chap. 1, Sect. 4, p. 18]{Strunk_White2019}

\subsection{Do not join independent clauses with a comma}
``If 2 or more clauses grammatically complete \& not joined by a conjunction are to form a single compound sentence, the proper mark of punctuation is a semicolon.

\begin{example}
	Mary Shelley's works are entertaining; they are full of engaging ideas.
	
	It is nearly half past 5; we cannot reach town before dark.
\end{example}
It is, of course, equally correct to write each of these as 2 sentences, replacing the semicolons with periods.

\begin{example}
	Mary Shelley's works are entertaining. They are full of engaging ideas.
	
	It is nearly half past 5. We cannot reach town before dark.
\end{example}
If a conjunction is inserted, the proper mark is a comma. (Rule 4.)

\begin{example}
	Mary Shelley's works are entertaining, for they are full of engaging ideas.
	
	It is nearly half past 5, \& we cannot reach town before dark.
\end{example}
A comparison of the 3 forms given above will show clearly the advantage of the 1st. It is, at least the examples given, better than the 2nd form because it suggests the close relationship between the 2 statements in a way that the 2nd does not attempt, \& better than the 3rd because it is briefer \& therefore more forcible. Indeed, this simple method of indicating relationship between statements is 1 of the most useful devices of composition. The relationship, as above, is commonly 1 of cause \& consequence.

Note that if the 2nd clause is preceded by an adverb, e.g., \textit{accordingly, besides, then, therefore}, or \textit{thus}, \& not by a conjunction, the semicolon is still required.

\begin{example}
	I had never been in the place before; besides, it was dark as a tomb.
\end{example}
An exception to the semicolon rule is worth noting here. A comma is preferable when the clauses are very short \& alike in form, or when the tone of the sentence is easy \& conversational.

\begin{example}
	Man proposes, God disposes.
	
	The gates swung apart, the bridge fell, the portcullis was drawn up.
	
	I hardly knew him, he was so changed.
	
	Here today, gone tomorrow.''
\end{example}
-- \cite[Chap. 1, Sect. 5, p. 19]{Strunk_White2019}

\subsection{Do not break sentences in 2}
``In other words, do not use periods for commas.

\begin{example}
	I met them on a Cunard liner many years ago. Coming home from Liverpool to New York.
	
	She was an interesting talker. A woman who had traveled all over the world \& lived in half a dozen countries.
\end{example}
In both these examples, the 1st period should be replaced by a comma \& the following word begun with a small letter.

It is permissible to make an emphatic word or expression serve the purpose of a sentence \& to punctuate it accordingly:

\begin{example}
	Again \& again he called out. No reply.
\end{example}
The writer must, however, be certain that the emphasis is warranted, lest a clipped sentence seem merely a blunder in syntax or in punctuation. Generally speaking, the place for broken sentences is in dialogue, when a character happens to speak in a clipped or fragmentary way.

Rules 3, 4, 5, \& 6 cover the most important principles that govern punctuation. They should be so thoroughly mastered that their application becomes 2nd nature.'' -- \cite[Chap. 1, Sect. 6, p. 20]{Strunk_White2019}

\subsection{Use a colon after an independent clause to introduce a list of particulars, an appositive, an amplification, or an illustrative quotation}
``A colon tells the reader that what follows is closely related to the preceding clause. The colon has more effect than the comma, less power to separate than the semicolon, \& more formality than the dash. It usually follows an independent clause \& should not separate a verb from its complement or a preposition from its object. The examples in the lefthand column, below, are wrong; they should be rewritten as in the righthand column.

\begin{example}
	Your dedicated whittler requires: a knife, a piece of wood, \& a back porch.
	
	$\hookrightarrow$ Your dedicated whittler requires 3 props: a knife, a piece of wood, \& a back porch.
	
	Understanding is that penetrating quality of knowledge that grows from: theory, practice, conviction, assertion, error, \& humiliation.
	
	$\hookrightarrow$ Understanding is that penetrating quality of knowledge that grows from theory, practice, conviction, assertion, error, \& humiliation.
\end{example}
Join 2 independent clauses with a colon if the 2nd interprets or amplifies the 1st.

\begin{example}
	But even so, there was a directness \& dispatch about animal burial: there was no stopover in the undertaker's foul parlor, no wreath or spray.
\end{example}
A colon may introduce a quotation that supports or contributes to the preceding clause.

\begin{example}
	The squalor of the streets reminded her of a line from Oscar Wilde: ``We are all in the gutter, but some of us are looking at the stars.''
\end{example}
The colon also has certain functions of form: to follow the salutation of a formal letter, to separate hour from minute in a notation of time, \& to separate the title of a work from its subtitle or a Bible chapter from a verse.

\begin{example}
	Dear Mr. Montague:
	
	departs at 10:48 P.M.
	
	Practical Calligraphy: An Introduction to Italic Script
	
	Nehemiah 11:7''
\end{example}
-- \cite[Chap. 1, Sect. 7, p. 21]{Strunk_White2019}

\subsection{Use a dash to set off an abrupt break or interruption \& to announce a long appositive or summary}
``A dash is a mark of separation stronger than a comma, less formal than a colon, \& more relaxed than parentheses.

\begin{example}
	His 1st thought on getting out of bed -- if he had any thought at all -- was to get back in again.
	
	The rear axle began to make a noise -- a grinding, chattering, teeth-gritting rasp.
	
	The increasing reluctance of the sun to rise, the extra nip in the breeze, the patter of shed leaves dropping -- all the evidences of fall drifting into winter were clearer each day.
\end{example}
Use a dash only when a more common mark of punctuation seems inadequate.

\begin{example}
	Her father's suspicions proved well-founded -- it was not Edward she cared for -- it was San Francisco.
	
	$\hookrightarrow$ Her father's suspicions proved well-founded. It was not Edward she cared for, it was San Francisco.
	
	Violence -- the kind you see on television -- is not honestly violent -- there lies its harm.
	
	$\hookrightarrow$ Violence, the kind you see on television, is not honestly violent. There lies its harm.''
\end{example}
-- \cite[Chap. 1, Sect. 8, p. 22]{Strunk_White2019}

\subsection{The number of the subject determines the number of the verb}
``Words that intervene between subject \& verb do not affect the number of the verb.

\begin{example}
	The bittersweet flavor of youth -- its trials, its joys, its adventures, its challenges -- are not soon forgotten.
	
	$\hookrightarrow$ The bittersweet flavor of youth -- its trials, its joys, its adventures, its challenges -- is not soon forgotten.
\end{example}
A common blunder is the use of a singular verb form in a relative clause following ``1 of $\ldots$'' or a similar expression when the relative is the subject.

\begin{example}
	1 of the ablest scientists who has attacked this problema $\to$ 1 of the ablest scientists who have attacked this problem
	
	1 of those people who is never ready on time $\to$ 1 of those people who are never ready on time
\end{example}
Use a singular verb form after \textit{each, either, everyone, everybody, neither, nobody, someone}.

\begin{example}
	Everybody thinks he has a unique sense of humor.
	
	Although both clocks strike cheerfully, neither keeps good time.
\end{example}
With \textit{none}, use the singular verb when the word means ``no one'' or ``not one.''

\begin{example}
	None of us are perfect. $\to$ None of us is perfect.
\end{example}
A plural verb is commonly used when \textit{none} suggests more than 1 thing or person.

\begin{example}
	None are so fallible as those who are sure they're right.
\end{example}
A compound subject formed of 2 or more nouns joined by \textit{\&} almost always requires a plural verb.

\begin{example}
	The walrus \& the carpenter were walking close at hand.
\end{example}
But certain compounds, often cliches, are so inseparable they are considered a unit \& so take a singular verb, as do compound subjects qualified by \textit{each} or \textit{every}.

\begin{example}
	The long \& the short of it is $\ldots$
	
	Bread \& butter was all she served.
	
	Give \& take is essential to a happy household.
	
	Every window, picture, \& mirror was smashed.
\end{example}
A singular subject remains singular even if other nouns are connected to it by \textit{with, as well as, in addition to, except, together with}, \& \textit{no less than}.

\begin{example}
	His speech as well as his manner is objectionable.
\end{example}
A linking verb agrees with the number of its subject.

\begin{example}
	What is wanted is a few more pairs of hands.
	
	The trouble with truth is its many varieties.
\end{example}
Some nouns that appear to be plural are usually construed as singular \& given a singular verb.

\begin{example}
	Politics is an art, not a science.
	
	The Republican Head quarters is on this side of the tracks.
\end{example}
But
\begin{example}
	The general's quarters are across the river.
\end{example}
In these cases the writer must simply learn the idioms. The content of a book is singular. The contents of a jar may be either singular or plural, depending on what's in the jar -- jam or marbles.'' -- \cite[Chap. 1, Sect. 9, pp. 23--24]{Strunk_White2019}

\subsection{Use the proper case of pronoun}
``The personal pronouns, as well as the pronoun \textit{who}, change form as they function as subject or object.

\begin{example}
	Will Jane or he be hired, do you think?
	
	The culprit, it turned out, was he.
	
	We heavy eaters would rather walk than ride.
	
	Who knocks?
	
	Give this work to whoever looks idle.
\end{example}
In the last example, \textit{whoever} is the subject of \textit{looks idle}; the object of the preposition \textit{to} is the entire clause \textit{whoever looks idle}. When \textit{who} introduces a subordinate clause, its case depends on its function in that clause.

\begin{example}
	Virgil Soames is the candidate whom we think will win. $\to$ Virgil Soames is the candidate who we think will win. [We think \emph{he} will win.]
	
	Virgil Soames is the candidate who we hope to elect. $\to$ Virgil Soames is the candidate whom we hope to elect. [We hope to elect \emph{him}.]
\end{example}
A pronoun in a comparison is nominative if it is the subject of a stated or understood verb.

\begin{example}
	Sandy writes better than I. (Than I write).
\end{example}
In general, avoid ``understood'' verbs by supplying them.

\begin{example}
	I think Horace admires Jessica more than I. $\to$ I think Horace admires Jessica more than I do.
	
	Polly loves cake more than me. $\to$ Polly loves cake more than she loves me.
\end{example}
The objective case is correct in the following examples.

\begin{example}
	The ranger offered Shirley \& him some advice on campsites.
	
	They came to meet the Baldwins \& us.
	
	Let's talk it over between us, then, you \& me.
	
	Whom should I ask?
	
	A group of us taxpayers protested.
\end{example}
\textit{Us} in the last example is in apposition to taxpayers, the object of the preposition \textit{of}. The wording, although grammatically defensible, is rarely apt. ``A group of us protested as taxpayers.'' is better, if not exactly equivalent.

Use the simple personal pronoun as a subject.

\begin{example}
	Blake \& myself stayed home. $\to$ Blake \& I stayed home.
	
	Howawrd \& yourself brought the lunch, I thought. $\to$ Howard \& you brought the lunch, I thought.
\end{example}
The possession case of pronouns is used to show ownership. It has 2 forms: the adjectival modifier, \textit{your} hat, \& the noun form, a hat of \textit{yours}.

\begin{example}
	The dog has buried 1 of your gloves \& 1 of mine in the flower bed.
\end{example}
Gerunds usually require the possessive case.

\begin{example}
	Mother objected to our driving on the icy roads.
\end{example}
A present participle as a verbal, on the other hand, takes the objective case.

\begin{example}
	They heard him singing in the shower.
\end{example}
The difference between a verbal participle \& a gerund is not always obvious, but not what is really said in each of the following.

\begin{example}
	Do you mind me asking a question?
	
	Do you mind my asking a question?
\end{example}
In the 1st sentence, the queried objection is to \textit{me}, as opposed to other members of group, asking a question. In the 2nd example, the issue is whether a question may be asked at all.'' -- \cite[Chap. 1, Sect. 10, pp. 25--26]{Strunk_White2019}

\subsection{A participial phrase at the beginning of a sentence must refer to the grammatical subject}

\begin{example}
	Walking slowly down the road, he saw a woman accompanied by 2 children.
\end{example}
The word \textit{walking} refers to the subject of the sentence, not to the woman. To make it refer to the woman, the writer must recast the sentence.

\begin{example}
	He saw a woman, accompanied by 2 children, walking slowly down the road.
\end{example}
Participial phrases preceded by a conjunction or by a preposition, nouns in apposition, adjectives, \& adjective phrases come under the same rule if they begin the sentence.

\begin{example}
	On arriving in Chicago, his friends met him at the station. $\to$ On arriving in Chicago, he was met at the station by his friends.
	
	A soldier of proved valor, they entrusted him with the defense of the city. $\to$ A soldier of proved valor, he was entrusted with the defense of the city.
	
	Young \& inexperienced, the task seemed easy to me. $\to$ Young \& inexperienced, I thought the task easy.
	
	Without a friend to counsel him, the temptation proved irresistible. $\to$ Without a friend to counsel him, he found the temptation irresistible.
\end{example}
Sentences violating Rule 11 are often ludicrous:

\begin{example}
	Being in a dilapidated condition, I was able to buy the house very cheap.
	
	Wondering irresolutely what to do next, the clock struck 12.'' -- \cite[Chap. 1, Sect. 11, p. 27]{Strunk_White2019}
\end{example}

\section{Elementary Principles of Composition}
This section is devoted to study \cite[Chap. 2]{Strunk_White2019}.

\subsection{Choose a suitable design \& hold to it}
``A basic structural design underlies every kind of writing. Writers will in part follow this design, in part deviate from it, according to their skills, their needs, \& the unexpected events that accompany the act of composition. Writing, to be effective, must follow closely the thoughts of the writer, but not necessarily in the order in which those  thoughts occur. This calls for a scheme of procedure. In some cases, the best design is no design, as with a love letter, which is simply an outpouring, or with a casual essay, which is a ramble. But in most cases, planning must be a deliberate prelude to writing. The 1st principle of composition, therefore, is to foresee or determine the shape of what is to come \& pursue that shape.

A sonnet is built on a 14-line frame, each line containing 5 feet. Hence, sonneteers know exactly where they are headed, although they may not know how to get there. Most forms of composition are less clearly defined, more flexible, but all have skeletons to which the writer will bring the flesh \& the blood. The more clearly the writer perceives the shape, the better are the chances of success.'' --  \cite[Chap. 2, Sect. 12, p. 29]{Strunk_White2019}

\subsection{Make the paragraph the unit of composition: 1 paragraph to each topic}
``The paragraph is a convenient unit; it serves all forms of literary work. As long as it holds together, a paragraph may be of any length -- a single, short sentence or a passage of great duration.

If the subject on which you are writing is of slight extent, or if you intend to treat it briefly, there may be no need to divide it into topics. Thus, a brief description, a brief book review, a brief account of a single incident, a narrative merely outlining an action, the setting forth of a single idea -- any 1 of these is best writing in a single paragraph. After the paragraph has been written, examine it to see whether division will improve it.

Ordinarily, however, a subject requires division into topics, each of which should be dealt with in a paragraph. The object of treating each topic in a paragraph by itself, of course, to aid the reader. The beginning of each paragraph is a signal that a new step in the development of the subject has been reached.

As a rule, single sentences should not be written or printed as paragraphs. An exception may be made of sentences of transition, indicating the relation between the parts of an exposition or argument.

In dialogue, each speech, even if only a single word, is usually a paragraph by itself; i.e., a new paragraph begins with each change of speaker. The application of this rule when dialogue \& narrative are combined is best learned from examples in well-edited works of fiction. Sometimes a writer, seeking to create an effect of rapid talk or for some other reason, will elect not to set off each speech in a separate paragraph \& instead will run speeches together. The common practice, however, \& the one that serves best in most instances, is to give each speech a paragraph of its own.

As a rule, begin each paragraph either with a sentence that suggests the topic or with a sentence that helps the transition. If a paragraph forms part of a larger composition, its relation to what precedes, or its function as a part of the whole, may need to be expressed. This can sometimes be done by a mere word or phrase (\textit{again, therefore, for the same reason}) in the 1st sentence. Sometimes, however, it is expedient to get into the topic slowly, by way of a sentence or 2 of introduction or translation.

In narration \& description, the paragraph sometimes begins with a concise, comprehensive statement serving to hold together the details that follow.

\begin{example}
	The breeze served us admirably.
	
	The campaign opened with a series of reverses.
	
	The next 10 or 12 pages were filled with a curious set of entries.
\end{example}
\fbox{But when this device, or any device, is too often used, it becomes a mannerism.} More commonly, the opening sentence simply indicates by its subject the direction the paragraph is to take.

\begin{example}
	At length I thought I might return toward the stockade.
	
	He picked up the heavy lamp from the table \& began to explore.
	
	Another flight of steps, \& they emerged on the roof.
\end{example}
In animated narrative, the paragraphs are likely to be short \& without any semblance of a topic sentence, the writer rushing headlong, event following event in rapid succession. The break between such paragraphs merely serves the purpose of a rhetorical pause, throwing into prominence some detail of the action.

In general, remember that paragraphing calls for a good eye as well as a logical mind. Enormous blocks of print look formidable to readers, who are often reluctant to tackle them. Therefore, breaking long paragraphs in 2, even if it is not necessary to do so for sense, meaning, or logical development, is often a visual help. But remember, too, that firing off many short paragraphs in quick succession can be distracting. Paragraph breaks used only for show read like the writing of commerce or of display advertising. Moderation \& a sense of order should be the main considerations in paragraphing.'' --  \cite[Chap. 2, Sect. 13, pp. 30--31]{Strunk_White2019}

\subsection{Use the active voice}
``\fbox{The active voice is usually more direct \& vigorous than the passive}:

\begin{example}
	I shall always remember my 1st visit to Boston.
\end{example}
This is much better than:

\begin{example}
	My 1st visit to Boston will always be remembered by me.
\end{example}
The latter sentence is less direct, less bold, \& less concise. If the writer tries to make it more concise by omitting ``by me,'': \textit{My 1st visit to Boston will always be remembered}, it becomes indefinite: is it the writer or some undisclosed person or the world at large that will always remember this visit?

This rule does not, of course, mean that the writer should entirely discard the passive voice, which is frequently convenient \& sometimes necessary.

\begin{example}
	The dramatists of the Restoration are little esteemed today.
	
	Modern readers have little esteem for the dramatists of the Restoration.
\end{example}
The 1st would be the preferred form in a paragraph on the dramatists of the Restoration, the 2nd in a paragraph on the tastes of modern readers. The need to make a particular word the subject of the sentence will often, as in these examples, determine which voice is to be used.

The habitual use of the active voice, however, makes for forcible writing. This is true not only in narrative concerned principally with action but in writing of any kind. Many a tame sentence of description or exposition can be made lively \& emphatic by substituting a transitive in the active voice for some such perfunctory expression as \textit{there is} or \textit{could be heard}.

\begin{example}
	There were a great number of dead leaves lying on the ground. $\to$ Dead leaves covered the ground.
	
	At dawn the crowing of a rooster could be heard. $\to$ The cock's crow came with dawn.
	
	The reason he left college was that his health became impaired. $\to$ Failing health compelled him to leave college.
	
	It was not long before she was very sorry that she had said what she had. $\to$ She soon repented her words.
\end{example}
Note, in the examples above, that when a sentence is made stronger, it usually becomes shorter.

Thus, \fbox{brevity is a by-product of vigor}.'' --  \cite[Chap. 2, Sect. 14, p. 32]{Strunk_White2019}

\subsection{Put statements in positive form}
``Make definite assertions. Avoid tame, colorless, hesitating, noncommittal language. Use the word \textit{not} as a means of denial or in antithesis, never as a means of evasion.

\begin{example}
	He was not very often on time. $\to$ He usually came late.
	
	She did not think that studying Latin was a sensible way to use one's time. $\to$ She thought the study of Latin a waste of time.
	
	\emph{The Taming of the Shrew} is rather weak in spots. Shakespeare does not portray Katharine as a very admirable character, nor does Bianca remain long in memory as an important character in Shakespeare's works.
	
	$\hookrightarrow$ The women in \emph{The Taming of the Shrew} are unattractive. Katharine is disagreeable, Bianca insignificant.
\end{example}
The last example, before correction, is indefinite as well as negative. The corrected version, consequently, is simply a guess at the writer's intention.

All 3 examples show the weakness inherent in the word \textit{not}. Consciously or unconsciously, the reader is dissatisfied with being told only what is not; the reader wishes to be told what is. Hence, as a rule, it is better to express even a negative in positive form.

\begin{example}
	not honest $\to$ dishonest; not important $\to$ trifling; did not remember $\to$ forgot; did not pay any attention to $\to$ ignored; $\to$ did not have much confidence in $\to$ distrusted
\end{example}
Placing negative \& positive in opposition makes for a stronger structure.

\begin{example}
	Not charity, but simple justice.
	
	Not that I loved Caesar less, but that I loved Rome more.
	
	Ask not what your country can do for you -- ask what you can do for your country.\footnote{\selectlanguage{vietnamese} ``Đừng hỏi Tổ quốc đã làm gì cho ta mà phải hỏi ta đã làm gì cho Tổ quốc hôm nay'' -- \textit{Khát Vọng Tuổi Trẻ} (1995), sáng tác: Vũ Hoàng.}
\end{example}
Negative words other than \textit{not} are usually strong.

\begin{example}
	Her loveliness I never knew\emph{\texttt{/}}Until she smiled on me.
\end{example}
Statements qualified with unnecessary auxiliaries or conditionals sound irresolute.

\begin{example}
	If you would let us know the time of your arrival, we would be happy to arrange your transportation from the airport.
	
	$\hookrightarrow$ If you will let us know the time of your arrival, we shall be happy to arrange your transportation from the airport.
	
	Applicants can make a good impression by being neat \& punctual. $\to$ Applicants will make a good impression if they are neat \& punctual.
	
	Plath may be ranked among those modem poets who died young. $\to$ Plath was 1 of those modern poets who died young.
\end{example}
\fbox{If your every sentence admits a doubt, your writing will lack authority.} Save the auxiliaries \textit{would, should, could, may, might}, \& \textit{can} for situations involving real uncertainty.'' --  \cite[Chap. 2, Sect. 15, pp. 33--34]{Strunk_White2019}

\subsection{Use definite, specific, concrete language}
``Prefer the specific to the general, the definite to the vague, the concrete to the abstract.

\begin{example}
	A period of unfavorable weather set in. $\to$ It rained every day for a week.
	
	He showed satisfaction as he took possession of his well-earned reward. $\to$ He grinned as he pocketed the coin.
\end{example}
If those who have studied the art of writing are in accord on any 1 point, it is this: the surest way to arouse \& hold the readers attention is by being specific, definite, \& concrete. The greatest writers -- Homer, Dante, Shakespeare -- are effective largely because they deal in particulars \& report the details that matter. Their words call up pictures.

Jean Stafford, to cite a more modern author, demonstrates in her short story ``In the Zoo'' how prose is made vivid by the use of words that evoke images \& sensations:

\begin{example}
	$\ldots$ Daisy \& I in time found asylum in a small menagerie down by the railroad tracks. It belonged to a gentle alcoholic ne'er-do- well, who did nothing all day long but drink bathtub gin in rickeys \& play solitaire \& smile to himself \& talk to his animals. He had a little, stunted red vixen \& a deodorized skunk, a parrot from Tahiti that spoke Parisian French, a woebegone coyote, \& 2 capuchin monkeys, so serious \& humanized, so small \& sad \& sweet, \& so religious-looking with their tonsured heads that it was impossible not to think their gibberish was really an ordered language with a grammar that somebody some philologist wound understand.
	
	Gran knew about our visits to Mr. Murphy \& she did not object, for it gave her keen pleasure to excoriate him when we came home. His vice was not a matter of guesswork; it was an established fact that he was half-seas over from dawn till midnight. ``With the black Irish,'' said Gran, ``the taste for drink is taken in with the mother's milk \& is never mastered. Oh, I know all about those promises to join the temperance movement \& not to touch another drop. \fbox{The way to Hell is paved with good intentions.}'' -- Excerpt from \textit{``In the Zoo''} from Bad Characters by Jean Stafford.
\end{example}
If the experiences of Walter Mitty, of Molly Bloom, of Rabbit Angstrom have seemed for the moment real to countless readers, if in reading Faulkner we have almost the sense of inhabiting Yoknapatawpha County during the decline of the South, it is because the details used are definite, the terms concrete. It is not that every detail is given -- that would be impossible, as well as to no purpose -- but that all the significant details are given, \& with such accuracy \& vigor that readers, in imagination, can project themselves into the scene.

In exposition \& in argument, the writer must likewise never lose hold of the concrete; \& even when dealing with general principles, the writer must furnish particular instances of their application.

In his \textit{Philosophy of Style}, Herbert Spencer gives 2 sentences to illustrate how the vague \& general can be turned into the vivid \& particular:

\begin{example}
	In proportion as the manners, customs, \& amusements of a nation are cruel \& barbarous, the regulations of their penal code will be severe.
	
	$\hookrightarrow$ In proportion as men delight in battles, bullfights, \& combats of gladiators, will they punish by hanging, burning, \& the rack.
\end{example}
To show what happens when strong writing is deprived of its vigor, George Orwell once took a passage from the Bible \& drained it of its blood. On the left, below, is Orwell's translation; on the right, the verse from Ecclesiastes (King James Version).

\begin{example}
	Objective consideration of contemporary phenomena compels the conclusion that success or failure in competitive activities exhibits no tendency to be commensurate with innate capacity, but that a considerable element of the unpredictable must inevitably be taken into account.
	
	$\hookrightarrow$ I returned, \& saw under the sun, that the race is not to the swift, nor the battle to the strong, neither yet bread to the wise, nor yet riches to men of understanding, nor yet favor to men of skill; but time \& chance happeneth to them all.''
\end{example}
--  \cite[Chap. 2, Sect. 16, pp. 35--36]{Strunk_White2019}

\subsection{Omit needless words}
``\fbox{Vigorous writing is concise.} A sentence should contain no unnecessary words, a paragraph no unnecessary sentences, for the same reason that a drawing should have no unnecessary lines \& a machine no unnecessary parts. This requires not that the writer make all sentences short, or avoid all detail \& treat subjects only in outline, but that every word tell.

Many expressions in common use violate this principle.

\begin{example}
	the question as to whether $\to$ whether (the question whether)
	
	there is no doubt but that $\to$ no doubt (doubtless)
	
	used for fuel purposes $\to$ used for fuel
	
	he is a man who $\to$ he
	
	in a hasty manner $\to$ hastily
	
	this is a subject that $\to$ this subject
	
	Her story is a strange one. $\to$ Her story is strange.
	
	the reason why is that $\to$ because
\end{example}
\textit{The fact that} is an especially debilitating expression. It should be revised out of every sentence in which it occurs.

\begin{example}
	owing to the fact that $\to$ since (because)
	
	in spite of the fact that $\to$ though (although)
	
	call your attention to the fact that $\to$ remind you (notify you)
	
	I was unaware of the fact that $\to$ I was unaware that (did not know)
	
	the fact that he had not succeeded $\to$ his failure
	
	the fact that I had arrived $\to$ my arrival
\end{example}
See also the words \textit{case, character, nature} in Chap. IV. \textit{Who is, which was}, \& the like are often superfluous.

\begin{example}
	His cousin, who is a member of the same firm $\to$ His cousin, a member of the same firm
	
	Trafalgar, which was Nelson's last battle $\to$ Trafalgar, Nelson's last battle
\end{example}
As the active voice is more concise than the passive, \& a positive statement more concise than a negative one, many of the examples given under Rules 14 \& 15 illustrate this rule as well.

A common way to fall into wordiness is to present a single complex idea, step by step, in a series of sentences that might to advantage be combined into one.

\begin{example}
	Macbeth was very ambitious. This led him to wish to become king of Scotland. The witches told him that this wish of his would come true. The king of Scotland at this time was Duncan. Encouraged by his wife, Macbeth murdered Duncan. He was thus enabled to succeed Duncan as king. (51 words)
	
	$\hookrightarrow$ Encouraged by his wife, Macbeth achieved his ambition \& realized the prediction of the witches by murdering Duncan \& becoming king of Scotland in his place. (26 words)''
\end{example}
--  \cite[Chap. 2, Sect. 17, pp. 37--38]{Strunk_White2019}

\subsection{Avoid a succession of loose sentences}
``This rule refers especially to loose sentences of a particular type: those consisting of 2 clauses, the 2nd introduced by a conjunction or relative. A writer may err by making sentences too compact \& periodic. An occasional loose sentence prevents the style from becoming too formal \& gives the reader a certain relief. Consequently, loose sentences are common in easy, unstudied writing. The danger is that there may be too many of them.

An unskilled writer will sometimes construct a whole paragraph of sentences of this kind, using as connectives \textit{\&, but}, \&, less frequently, \textit{who, which, when, where}, \& \textit{while}, these last in nonrestrictive senses. (See Rule 3.)

\begin{example}
	The 3rd concert of the subscription series was given last evening, \& a large audience was in attendance. Mr. Edward Appleton was the soloist, \& the Boston Symphony Orchestra furnished the instrumental music. The former showed himself to be an artist of the 1st rank, while the latter proved itself fully deserving of its high reputation. The interest aroused by the series has been very gratifying to the Committee, \& it is planned to give a similar series annually hereafter. The 4th concert will be given on Tuesday, May 10, when an equally attractive program will be presented.
\end{example}
Apart from its triteness \& emptiness, the paragraph above is bad because of the structure of its sentences, with their mechanical symmetry \& singsong. Compare these sentences from the chapter ``What I Believe'' in E. M. Forster's \textit{2 Cheers for Democracy}:

\begin{example}
	I believe in aristocracy, though -- if that is the right word, \& if a democrat may use it. Not an aristocracy of power, based upon rank \& influence, but an aristocracy of the sensitive, the considerate \& the plucky. Its members are to be found in all nations \& classes, \& all through the ages, \& there is a secret understanding between them when they meet. They represent the true human tradition, the 1 permanent victory of our queer race over cruelty \& chaos. Thousands of them perish in obscurity, a few are great names. They are sensitive for others as well as for themselves, they are considerate without being fussy, their pluck is not swankiness but the power to endure, \& they can take a joke.
\end{example}
A writer who has written a series of loose sentences should recast enough of them to remove the monotony, replacing them with simple sentences, sentences of 2 clauses joined by a semicolon, periodic sentences of 2 clauses, or sentences (loose or periodic) of 3 clauses -- whichever best represent the real relations of the thought.'' -- \cite[Chap. 2, Sect. 18, pp. 39--40]{Strunk_White2019}

\subsection{Express coordinate ideas in similar form}
``This principle, that of parallel construction, requires that expressions similar in content \& function be outwardly similar. The likeness of form enables the reader to recognize more readily the likeness of content \& function. The familiar Beautitudes exemplify the virtue of parallel construction.

\begin{example}
	Blessed are the poor in spirit: for theirs is the kingdom of heaven.
	
	Blessed are they that mourn: for they shall be comforted.
	
	Blessed are the meek: for they shall inherit the earth.
	
	Blessed are they which do hunger \& thirst after righteousness: for they shall be filled.
\end{example}
The unskilled writer often violates this principle, mistakenly believing in the value of constantly varying the form of expression. When repeating a statement to emphasize it, the writer may need to vary its form. Otherwise, the writer should follow the principle of parallel construction.

\begin{example}
	Formerly, science was taught by the textbook method, while now the laboratory method is employed.
	
	$\hookrightarrow$ Formerly, science was taught by the textbook method; now it is taught by the laboratory method.
\end{example}
The lefthand version gives the impression that the writer is undecided or timid, apparently unable or afraid to choose 1 form of expression \& hold to it. The right hand version shows that the writer has at least made a choice \& abided by it.

By this principle, an article or a preposition applying to all the members of a series must either be used only before the 1st term or else be repeated before each term.

\begin{example}
	The French, the Italians, Spanish, \& Portuguese $\to$ The French, the Italians, the Spanish, \& the Portuguese
	
	In spring, summer, or in winter $\to$ In spring, summer, or winter (In spring, in summer, or in winter)
\end{example}
Some words require a particular preposition in certain idiomatic uses. When such words are joined in a compound construction, all the appropriate prepositions must be included, unless they are the same.

\begin{example}
	His speech was marked by disagreement \& scorn for his opponent's position. $\to$ His speech was marked by disagreement with \& scorn for his opponent's position.
\end{example}
Correlative expressions (\textit{both, and; not, but; not only, but also; either, or; 1st, 2nd, 3rd}; \& the like) should be followed by the same grammatical construction. Many violations of this rule can be corrected by rearranging the sentence.

\begin{example}
	It was both a long ceremony \& very tedious. $\to$ The ceremony was both long \& tedious.
	
	A time not for words but action. $\to$ A time not for words but for action.
	
	Either you must grant his request or incur his ill will. $\to$ You must either grant his request or incur his ill will.
	
	My objections are, 1st, the injustice of the measure; 2nd, that it is unconstitutional. $\to$ My objections are, 1st,  that the measure is unjust; 2nd, that it is unconstitutional.
\end{example}
It may be asked, what if you need to express a rather large number of similar ideas -- say, 20? Must you write 20 consecutive sentences of the same pattern? On closer examination, you will probably find that the difficulty is imaginary -- that these 20 ideas can be classified in groups, \& that you need apply the principle only within each group. Otherwise, it is best to avoid the difficulty by putting statements in the form of a table.'' -- \cite[Chap. 2, Sect. 19, pp. 41--42]{Strunk_White2019}

\subsection{Keep related words together}
``\fbox{The position of the words in a sentence is the principal means of showing their relationship.} Confusion \& ambiguity result when words are badly placed. The writer must, therefore, bring together the words \& groups of words that are related in thought \& keep apart those that are not so related.

\begin{example}
	He noticed a large stain in the rug that was right in the center. $\to$ He noticed a large stain right in the center of the rug.
	
	You can call your mother in London \& tell her about George's taking you out to dinner for just 2 dollars. $\to$ For just 2 dollars you can call your mother in London \& tell her all about George's taking you out to dinner.
	
	New York's 1st commercial human-sperm bank opened Friday with semen samples from 18 men frozen in a stainless steel tank. $\to$ New York's 1st commercial human-sperm bank opened Friday when semen samples were taken from 18 men. The samples were then frozen \& stored in a stainless steel tank.
\end{example}
In the lefthand version of the 1st example, the reader has no way of knowing whether the stain was in the center of the rug or the rug was in the center of the room. In the lefthand version of the 2nd example, the reader may well wonder which cost 2 dollars -- the phone call or the dinner. In the lefthand version of the 3rd example, the reader's heart goes out to those 18 poor fellows frozen in a steel tank.

The subject of a sentence \& the principal verb should not, as a rule, be separated by a phrase or clause that can be transferred to the beginning.

\begin{example}
	Toni Morrison, in \emph{Beloved}, writes about characters who have escaped from slavery but are haunted by its heritage. $\to$ In \emph{Beloved}, Toni Morrison writes about characters who have escaped from slavery but are haunted by its heritage.
	
	A dog, if you fail to discipline him, becomes a household pest. $\to$ Unless disciplined, a dog becomes a household pest.
\end{example}
Interposing a phrase or a clause, as in the lefthand examples above, interrupts the flow of the main clause. This interruption, however, is not usually bothersome when the flow is checked only by a relative clause or by an expression in apposition. Sometimes, in periodic sentences, the interruption is a deliberate device for creating suspense\footnote{\textbf{suspense} [n] [uncountable] a feeling of worry or excitement that you have when you feel that something is going to happen, somebody is going to tell you some news, etc.}. (See examples under Rule 22.)

The relative pronoun should come, in most instances, immediately after its antecedent.

\begin{example}
	There was a stir in the audience that suggested disapproval. $\to$ A stir that suggested disapproval swept the audience.
	
	He wrote 3 articles about his adventures in Spain, which were published in \emph{Harper's Magazine}.
	
	$\hookrightarrow$ He published 3 articles in \emph{Harper's Magazine} about his adventures in Spain.
	
	This is a portrait of Benjamin Harrison, who became President in 1889. He was the grandson of William Henry Harrison.
	
	$\hookrightarrow$ This is a portrait of Benjamin Harrison, grandson of William Henry Harrison, who became President in 1889.
\end{example}
If the antecedent consists of a group of words, the relative comes at the end of the group, unless this would cause ambiguity.

\begin{example}
	The Superintendent of the Chicago Division, who
\end{example}
No ambiguity results from the above. But

\begin{example}
	A proposal to amend the Sherman Act, which has been variously judged
\end{example}
leaves the reader wondering whether it is the proposal or the Act that has been various judged. The relative clause must be moved forward, to read, ``A proposal, which has been variously judged, to amend the Sherman Act $\ldots$'' Similarly

\begin{example}
	The grandson of William Henry Harrison, who $\to$ William Henry Harrison's grandson, Benjamin Harrison, who
\end{example}
A noun in apposition may come between antecedent \& relative, because in such a combination no real ambiguity can arise.

\begin{example}
	The Duke of York, his brother, who was regarded with hostility by the Whigs
\end{example}
Modifiers should come, if possible, next to the words they modify. If several expressions modify the same word, they should be arranged so that no wrong relation is suggested.

\begin{example}
	All the members were not present. $\to$ Not all the members were present.
	
	She only found 2 mistakes. $\to$ She found only 2 mistakes.
	
	The director said he hoped all members would give generously to the Fund at a meeting of the committee yesterday.
	
	$\hookrightarrow$ At a meeting of the committee yesterday, the director said he hoped all members would give generously to the Fund.
	
	Major R. E. Joyce will give a lecture on Tuesday evening in Bailey Hall, to which the public is invited on ``My Experiences in Mesopotamia'' at 8:00 P.M.
	
	$\hookrightarrow$ On Tuesday evening at 8, Major R. E. Joyce will give a lecture in Bailey Hall on ``My Experiences in Mesopotamia.'' The public is invited.
\end{example}
Note, in the last lefthand example, how swiftly meaning departs when words are wrongly juxtaposed.'' -- \cite[Chap. 2, Sect. 20, pp. 43--45]{Strunk_White2019}

\subsection{In summaries, keep to 1 tense}
``In summarizing the action of a drama, use the present tense. In summarizing a poem, story, or novel, also use the present, though you may use the past if it seems more natural to do so. If the summary is in the present tense, antecedent action should be expressed by the perfect; if in the past, by the past perfect.

\begin{example}
	Chance prevents Friar John from delivering Friar Lawrence's letter to Romeo. Meanwhile, owning to her father's arbitrary change of the day set for her wedding, Juliet has been compelled to drink the potion on Tuesday night, with the result that Balthasar informs Romeo of her supposed death before Friar Lawrence learns of the nondelivery of the letter.
\end{example}
But whichever tense is used in the summary, a past tense in indirect discourse or in indirect question remains unchanged.

\begin{example}
	The Friar confesses that it was he who married them.
\end{example}
Apart from the exceptions noted, the writer should use the same tense throughout. Shifting from 1 tense to another gives the appearance of uncertainty \& irresolution.

In presenting the statements or the thought of someone else, as in summarizing an essay or reporting a speech, do not overwork such expressions as ``he said,'' ``she stated,'' ``the speaker added,'' ``the speaker then went on to say,'' ``the author also thinks.'' Indicate clearly at the outset, once for all, that what follows is summary, \& then waste no words in repeating the notification.

In notebooks, in newspapers, in handbooks of literature, summaries of 1 kind or another may be indispensable\footnote{\textbf{indispensable} [a] too important to be without, \textsc{synonym}: \textbf{essential}.}, \& for children in primary schools retelling a story in their own words is a useful exercise. But in the criticism or interpretation of literature, be careful to avoid dropping into summary. It may be necessary to devote 1 or 2 sentences to indicating the subject, or the opening situation, of the work being discussed, or to cite numerous details to illustrate its qualities. But you should aim at writing an orderly discussion supported by evidence, not a summary with occasional comment. Similarly, if the scope of the discussion includes a number of works, as a rule it is better not to take them up singly in chronological order but to aim from the beginning at establishing general conclusions.'' -- \cite[Chap. 2, Sect. 21, pp. 46--47]{Strunk_White2019}

\subsection{Place the emphatic words of a sentence at the end}
``The proper place in the sentence for the word or group of words that the writer desires to make most prominent is usually the end.

\begin{example}
	Humanity has hardly advanced in fortitude since that time, though it has advanced in many other ways.
	
	$\hookrightarrow$ Since that time, humanity has advanced in many ways, but it has hardly advanced in fortitude.
	
	This steel is principally used for making razors, because of its hardness.
	
	$\hookrightarrow$ Because of its hardness, this steel is used primarily for making razors.
\end{example}
The word or group of words entitled to this position of prominence is usually the logical predicate -- i.e., the \textit{new} element in the sentence, as it is in the 2nd example. The effectiveness of the periodic sentence arises from the prominence it gives to the main statement.

\begin{example}
	4 centuries ago, Christopher Columbus, 1 of the Italian mariners whom the decline of their own republics had put at the service of the world \& of adventure, seeking for Spain a westward passage to the Indies to offset the achievement of Portuguese discoverers, lighted on America.
	
	With these hopes \& in this belief I would urge you, laying aside all hindrance, thrusting away all private aims, to devote yourself unswervingly \& unflinchingly to the vigorous \& successful prosecution of this war.
\end{example}
The other prominent position in the sentence is the beginning. Any element in the sentence other than the subject becomes emphatic when placed 1st.

\begin{example}
	Deceit or treachery she could never forgive.
	
	Vast \& rude, fretted by the action of nearly 3000 years, the fragments of this architecture may often seem, at 1st sight, like works of nature.
	
	Home is the sailor.
\end{example}
A subject coming 1st in its sentence may be emphatic, but hardly by its position alone. In the sentence

\begin{example}
	Great kings worshiped at his shrine
\end{example}
the emphasis upon \textit{kings} arises largely from its meaning \& from the context. To receive special emphasis, the subject of a sentence must take the position of the predicate.

\begin{example}
	Through the middle of the valley flowed a winding stream.
\end{example}
The principle that the proper place for what is to be made most prominent is the end applies equally to the words of a sentence, to the sentences of a paragraph, \& to the paragraphs of a composition.'' -- \cite[Chap. 2, Sect. 22, pp. 48--49]{Strunk_White2019}

\section{A Few Matters of Form}
This section is devoted to study \cite[Chap. 3]{Strunk_White2019}.

\paragraph*{Colloquialisms.} ``If you use a colloquialism or a slang word or phrase, simply use it; do not draw attention to it by enclosing it in quotation marks. To do so is to put on airs, as though you were inviting the reader to join you in a select society of those who know better.

\paragraph*{Exclamations.} Do not attempt to emphasize simple statements by using a mark of exclamation.

\begin{example}
	It was a wonderful show! $\to$ It was a wonderful show.
\end{example}
The exclamation mark is to be reserved for use after true exclamations or commands.

\begin{example}
	What a wonderful show!
	
	Halt!
\end{example}

\paragraph*{Headings.} If a manuscript is to be submitted for publication, leave plenty of space at the top of p. 1. The editor will need this space to write directions to the compositor. Place the heading, or title, at least a 4th of the way down the page. Leave a blank line, or its equivalent in space, after the heading. On succeeding pages, begin near the top, but not so near as to give a crowded appearance. Omit the period after a title or heading. A question mark or an exclamation point may be used if the heading calls for it.

\paragraph*{Hyphen.} When 2 or more words are combined to form a compound adjective, a hyphen is usually required.

\begin{example}
	He belonged to the leisure class \& enjoyed leisure-class pursuits.
	
	She entered her boat in the round-the-island race.
\end{example}
Do not use a hyphen between words that can better be written as 1 word: \textit{water-fowl, waterfowl}. Common sense will aid you in the decision, but a dictionary is more reliable. The steady evolution of the language seems to favor union: 2 words eventually become 1, usually after a period of hyphenation.

\begin{example}
	bed chamber $\to$ bed-chamber $\to$ bedchamber; wild life $\to$ wild-life $\to$ wildlife; bell boy $\to$ bell-boy $\to$ bellboy
\end{example}
The hyphen can play tricks on the unwary\footnote{\textbf{unwary} [a] \textbf{1.} [only before noun] not aware of the possible dangers or problems of a situation \& therefore likely to be harmed in some way; \textbf{2.} \textbf{the unwary} [n] [plural] people who are unwary.}, as it did in Chattanooga when 2 newspapers merged -- the \textit{News} \& the \textit{Free Press}. Someone introduced a hyphen into the merger, \& the paper become \textit{The Chattanooga News-Free Press}, which sounds as though the paper were news-free, or devoid\footnote{\textbf{devoid} [a] \textbf{devoid of something} completely lacking in something.} of news. Obviously, we ask too much of a hyphen when we ask it to cast its spell over words it does not adjoin.

\paragraph*{Margins.} Keep righthand \& lefthand margins roughly the same width. Exception: If a great deal of annotating or editing is anticipated, the lefthand margin should be roomy enough to accommodate this work.

\paragraph*{Numerals.} Do not spell out dates or other serial numbers. Write them in figures or in Roman notation, as appropriate.

\begin{example}
	August 9, 1988; Part XII; Rule 3; 352d Infantry
\end{example}
Exception: When they occur in dialogue, most dates \& numbers are best spelled out.

\begin{example}
	``I arrived home on August 9th.''; ``In the year 1990, I turned 21.'' ``Read Chap. 12.''
\end{example}

\paragraph*{Parentheses.} A sentence containing an expression in parentheses is punctuated outside the last mark of parenthesis exactly as if the parenthetical expression were absent. The expression within the marks is punctuated as if it stood by itself, except that the final stop is omitted unless it is a question mark or an exclamation point.

\begin{example}
	I went to her house yesterday (my 3rd attempt to see her), but she had left town.
	
	He declares (\& why should we doubt his good faith?) that he is now certain of success.
\end{example}
(When a wholly detached expression or sentence is parenthesized, the final stop comes before the last mark of parenthesis.)

\paragraph*{Quotations.} Formal quotations cited as documentary evidence are introduced by a colon \& enclosed in quotation marks.

\begin{example}
	The United States Coast Pilot has this to say of the place: ``Bracy Cove, 0.5 mile eastward of Bear Island, is exposed to southeast winds, has a rocky \& uneven bottom, \& is unfit for anchorage.''
\end{example}
A quotation grammatically in apposition or the direct object of a verb is preceded by a comma \& enclosed in quotation marks.

\begin{example}
	I am reminded of the advice of my neighbor, ``Never worry about your heart till it stops beating.''
	
	Mark Twain says, ``A classic is something that everybody wants to have read \& nobody wants to read.''
\end{example}
When a quotation is followed by an attributive phrase, the comma is enclosed within the quotation marks.

\begin{example}
	``I can't attend,'' she said.
\end{example}
Typographical usage dictates that the comma be inside the marks, though logically it often seems not to belong there.

\begin{example}
	``The Fish,'' ``Poetry,'' \& ``The Monkeys'' are in Marianne Moore's Selected Poems.
\end{example}
When quotations of an entire line, or more, of either verse or prose are to be distinguished typographically from text matter, as are the quotations in this book, begin on a fresh line \& indent. Quotation marks should not be used unless they appear in the original, as in dialogue.

\begin{example}
	Worldworth's enthusiasm for the French Revolution was at 1st unbounded:
	
	Bliss was it in that dawn to be alive,
	
	But to be young was very heaven!
\end{example}
Quotations introduced by \textit{that} are indirect discourse \& not enclosed in quotation marks.

\begin{example}
	Keats declares that beauty is truth, truth beauty.
	
	Dickinson states that a coffin is a small domain.
\end{example}
Proverbial expressions \& familiar phrases of literary origin require no quotation marks.

\begin{example}
	These are the times that try men's souls.
	
	He lives far from the madding crowd.
\end{example}

\paragraph*{References.} In scholarly work requiring exact references, abbreviate titles that occur frequently, giving the full forms in an alphabetical list at the end. As a general practice, give the reference in parentheses or in footnotes, not in the body of the sentence. Omit the words \textit{act, scene, line, book, volume, page}, except when referring to only 1 of them. Punctuate as indicated below.

\begin{example}
	in the 2nd scene of the 3rd act $\to$ in III.ii (Better still, simply insert III.ii in parentheses at the proper place in the sentence.)
	
	After the killing of Polonius, Hamlet is placed under guard (IV.ii.14)
	
	2 Samuel i: 17--27
	
	Othello II.iii. 264--267, III.iii. 155--161
\end{example}

\paragraph*{Syllabication.} When a word must be divided at the end of a line, consult a dictionary to learn the syllables between which division should be made. The student will do well to examine the syllable division in a number of pages of any carefully printed book.

\paragraph*{Titles.} For the titles of literary works, scholarly usage prefers italics with capitalized initials. The usage of editors \& publishers varies, some using italics with capitalized initials, others using Roman with capitalized initials \& with or without quotation marks. Use italics (indicated in manuscript by underscoring) except in writing for a periodical that follows a different practice. Omit initial \textit{A} or \textit{The} from titles when you place the possessive before them.

\begin{example}
	A Tale of 2 Cities; \emph{Dickens's} Tale of 2 Cities.
	
	The Age of Innocence; \emph{Wharton's} Age of Innocence.''
\end{example}
-- \cite[Chap. 3, pp. 50--53]{Strunk_White2019}

\section{Words \& Expressions Commonly Misused}
This section is devoted to study \cite[Chap. 4]{Strunk_White2019}.

``Many of the words \& expressions listed here are not so much bad English as bad style, the common places of careless writing. As illustrated under \textit{Feature}, the proper correction is likely to be not the replacement of 1 word or set of words by another but the replacement of vague generality by definite statement.

The shape of our language is not rigid; in questions of usage we have no lawgiver whose word is final. Students whose curiosity is aroused by the interpretations that follow, or whose doubts are raised, will wish to pursue their investigations further. Books useful in such pursuits are \textit{Merriam Webster's Collegiate Dictionary}, 10th Edition; \textit{The American Heritage Dictionary of the English Language}, 3rd Edition; \textit{Webster's 3rd New International Dictionary; The New Fowler's Modern English Usage}, 3rd Edition, edited by R. W. Burchfield; \textit{Modern American Usage: A Guide} by Wilson Follett \& Erik Wensberg; \& \textit{The Careful Writer} by Theodore M. Bernstein.

\begin{enumerate}
	\item \textbf{Aggravate. Irritate.} The 1st means ``to add to'' an already troublesome or vexing matter or condition. The 2nd means ``to vex'' or ``to annoy'' or ``to chafe.''
	\item \textbf{All right.} Idiomatic in familiar speech as a detached phrase in the sense ``Agreed,'' or ``Go ahead,'' or ``O.K.'' Properly written as 2 words -- \textit{all right}.
	\item \textbf{Allude.} Do not confuse with \textit{elude}. You \textit{allude} to a book; you \textit{elude} a pursuer. Note, too, that \textit{allude} is not synonymous with \textit{refer}. An allusion is an indirect mention, a reference is a specific one.
	\item \textbf{Allusion.} Easily confused with \textit{illusion}. The 1st means ``an indirect reference''; the 2nd means ``an unreal image'' or ``a false impression.''
	\item \textbf{Alternate. Alternative.} The words are not always interchangeable as nouns or adjectives. The 1st means every other one in a series; the 2nd, 1 of 2 possibilities. As the other one of a series of 2, an \textit{alternate} may stand for ``a substitute,'' but an \textit{alternative}, although used in a similar sense, connotes a matter of choice that is never present with \textit{alternate}.
	
	\begin{example}
		As the flooded road left them no alternative, they took the alternate route.
	\end{example}
	\item \textbf{Among. Between.} When $\ge 2$ things or persons are involved, \textit{among} is usually called for: ``The money was divided among the 4 players.'' When, however, $\ge 2$ are involved but each is considered individually, \textit{between} is preferred: ``an agreement between the 6 heirs.''
	\item \textbf{And\texttt{/}or.} A device, or shortcut, that damages a sentence \& often leads to confusion or ambiguity.
	
	\begin{example}
		1st of all, would an honor system successfully cut down on the amount of stealing \&\texttt{/}or cheating?
		
		$\hookrightarrow$ 1st of all, would an honor system reduce the incidence of stealing or cheating or both?
	\end{example}
	\item \textbf{Anticipate.} Use \textit{expect} in the sense of simple expectation.
	
	\begin{example}
		I anticipated that he would look older. $\to$ I expected that he would look older.
		
		My brother anticipated the upturn in the market. $\to$ My brother expected the upturn in the market.
	\end{example}
	In the 2nd example, the word \textit{anticipated} is ambiguous. It could mean simply that the brother believed the upturn would occur, or it could mean that he acted in advance of the expected upturn -- by buying stock, perhaps.
	\item \textbf{Anybody.} In the sense of ``any person,'' not to be written as 2 words. \textit{Any body} means ``any corpse,'' or ``any human form,'' or ``any group.'' The rule holds equally for \textit{everybody, nobody, \& somebody}.
	\item \textbf{Anyone.} In the sense of ``anybody,'' written as 1 word. \textit{Any one} means ``any single person'' or ``any single thing.''
	\item \textbf{As good or better than.} Expressions of this type should be corrected by rearranging the sentences.
	
	\begin{example}
		My opinion is as good or better than his. $\to$ My opinion is as good as his, or better (if not better).
	\end{example}
	\item \textbf{As to whether.} \textit{Whether} is sufficient.
	\item \textbf{As yet.} \textit{Yet} nearly always is as good, if not better.
	
	\begin{example}
		No agreement has been reached as yet. $\to$ No agreement has yet been reached.
	\end{example}
	The chief exception is at the beginning of a sentence, where \textit{yet} means something different.
	
	\begin{example}
		Yet (\emph{or} despite everything) he has not succeeded.
		
		As yet (\emph{or} so far) he has not succeeded.
	\end{example}
	\item \textbf{Being.} Not appropriate after \textit{regard $\ldots$ as}.
	
	\begin{example}
		He is regarded as being the best dancer in the club. $\to$ He is regarded as the best dancer in the club.
	\end{example}
	\item \textbf{But.} Unnecessary after \textit{doubt} \& \textit{help}.
	
	\begin{example}
		I have no doubt but that $\to$ I have no doubt that.
		
		He could not help but see that. $\to$ He could not help seeing that.
	\end{example}
	The too-frequent use of \textit{but} as a conjunction leads to the fault discussed under Rule 18. A loose sentence formed with \textit{but} can usually be converted into a periodic sentence formed with \textit{although}.
	
	Particularly awkward is one \textit{but} closely following another, thus making a contrast to a contrast, or a reservation to a reservation. This is easily corrected by rearrangement.
	
	\begin{example}
		Our country has vast resources but seemed almost wholly unprepared for war. But within a year it had created an army of 4 million.
		
		$\hookrightarrow$ Our country seemed almost wholly unprepared for war, but it had vast resources. Within a year it had created an army of 4 million.
	\end{example}
	\item \textbf{Can.} Means ``am (is, are) able.'' Not to be used as a substitute for \textit{may}.
	\item \textbf{Care less.} The dismissive ``I couldn't care less'' is often used with the shortened ``not'' mistakenly (\& mysteriously) omitted: ``I could care less.'' The error destroys the meaning of the sentence \& is careless indeed.
	\item \textbf{Case.} Often unnecessary.
	
	\begin{example}
		In many cases, the rooms lacked air conditioning. $\to$ Many of the rooms lacked air conditioning.
		
		It has rarely been the case that any mistake has been made. $\to$ Few mistakes have been made.
	\end{example}
	\item \textbf{Certainly.} Used indiscriminately\footnote{\textbf{indiscriminately} [adv] \textbf{1.} without thinking about what the result of your actions may be, especially when this causes people to be harmed; \textbf{2.} without careful judgment.} by some speakers, much as others use \textit{very}, in an attempt to intensify any \& every statement. A mannerism of this kind, bad in speech, is even worse in writing.
	\item \textbf{Character.} Often simply redundant, used from a mere habit of wordiness.
	
	\begin{example}
		acts of a hostile character $\to$ hostile acts
	\end{example}
	\item \textbf{Claim.} [v] With object-noun, means ``lay claim to.'' May be used with a dependent clause if this sense is clearly intended: ``She claimed that she was the sole heir.'' (But even here \textit{claimed to be} would be better.) Not to be used as a substitute for \textit{declare, maintain}, or \textit{charge}.
	
	\begin{example}
		He claimed he knew know. $\to$ He declared he knew how.
	\end{example}
	\item \textbf{Clever.} Note that the word means 1 thing when applied to people, another when applied to horses. A clever horse is a good-natured one, not an ingenious one.
	\item \textbf{Compare.} To \textit{compare to} is to point out or imply resemblances between objects regarded as essentially of a different order; to \textit{compare with} is mainly to point out differences between objects regarded as essentially of the same order. Thus, life has been \textit{compared to} a pilgrimage, \textit{to} a drama, \textit{to} a battle; Congress may be \textit{compared with} the British Parliament. Paris has been \textit{compared to} ancient Athens; it may be \textit{compared with} modern London.
	\item \textbf{Comprise.} Literally, ``embrace'': A zoo comprises mammals, reptiles, \& birds (because it ``embraces,'' or ``includes,'' them). But animals do not comprise (``embrace'') a zoo -- they constitute a zoo.
	\item \textbf{Consider.} Not followed by \textit{as} when it means ``believe to be.''
	
	\begin{example}
		I consider him as competent. $\to$ I consider him competent.
	\end{example}
	When \textit{considered} means ``examined'' or ``discussed,'' it is followed by \textit{as}:
	\begin{example}
		The lecturer considered Eisenhower 1st as soldier \& 2nd as administrator.
	\end{example}
	\item \textbf{Contact.} As a transitive verb, the word is vague \& self-important. Do not \textit{contact} people; get in touch with them, look them up, phone them, find them, or meet them.
	\item \textbf{Cope.} An intransitive verb used with \textit{with}. In formal writing, one doesn't ``cope,'' one ``copes with'' something or somebody.
	
	\begin{example}
		I knew they'd cope. (jocular\footnote{\textbf{jocular} [a] (\textit{formal}) \textbf{1.} humorous; \textbf{2.} (of a person) enjoying making people laugh, \textsc{synonym}: \textbf{jolly}.}) $\to$ I knew they would cope with the situation.
	\end{example}
	\item \textbf{Currently.} In the sense of \textit{now} with a verb in the present tense, \textit{currently} is usually redundant; emphasis is better achieved through a more precise reference to time.
	
	\begin{example}
		We are currently reviewing your application. $\to$ We are at this moment reviewing your application.
	\end{example}
	\item \textbf{Data.} Like \textit{strata, phenomena, \& media}, \textit{data} is a plural \& is best used with a plural verb. The word, however, is slowly gaining acceptance as a singular.
	
	\begin{example}
		The data is misleading. $\to$ These data are misleading.
	\end{example}
	\item \textbf{Different than.} Here logic supports established usage: 1 thing differs \textit{from} another, hence, \textit{different from}. Or, \textit{other than, unlike}.
	\item \textbf{Disinterested.} Means ``impartial.'' Do not confuse it with \textit{uninterested}, which means ``not interested in.''
	
	\begin{example}
		Let a disinterested person judge our dispute, (an impartial person)
		
		This man is obviously uninterested in our dispute, (couldn't care less)
	\end{example}
	\item \textbf{Divided into.} Not to be misused for \textit{composed of}. The time is sometimes difficult to draw; doubtless plays are divided into acts, but poems are composed of stanzas. An apple, halved, is divided into sections, but an apple is composed of seeds, flesh, \& skin.
	\item \textbf{Due to.} Loosely used for \textit{through, because of}, or \textit{owning to}, in adverbial phrases.
	
	\begin{example}
		He lost the 1st game due to carelessness. $\to$ He lost the 1st game because of carelessness.
	\end{example}
	In correct use, synonymous with \textit{attributable to}: ``The accident was due to bad weather''; ``losses due to preventable fires.''
	\item \textbf{Each \& every one.} Pitchman's jargon\footnote{\textbf{jargon} [n] [uncountable] (\textit{often disapproving}) words or expressions that are used by a particular profession or group of people, \& are difficult for others to understand.}. Avoid, except in dialogue.
	
	\begin{example}
		It should be a lesson to each \& every one of us. $\to$ It should be a lesson to every one of us (to us all).
	\end{example}
	\item \textbf{Effect.} As a noun, means ``result''; as a verb, means ``to bring about,'' ``to accomplish'' (not to be confused with \textit{affect}, which means ``to influence'').
	
	As a noun, often loosely used in perfunctory writing about fashions, music, painting, \& other arts: ``a Southwestern effect''; ``effects in pale green''; ``very delicate effects''; ``subtle effects''; ``a charming effect was produced.'' The writer who has a definite meaning to express will not take refuge in such vagueness.
	\item \textbf{Enormity.} Use only in the sense of ``monstrous wickedness.'' Misleading, if not wrong, when used to express bigness.
	\item \textbf{Enthuse.} An annoying verb growing out of the noun \textit{enthusiasm}. Not recommended.
	
	\begin{example}
		She was enthused about her new car. $\to$ She was enthusiastic about her new car.
		
		She enthused about her new car. $\to$ She talked enthusiastically (expressed enthusiasm) about her new car.
	\end{example}
	\item \textbf{Etc.} Literally, ``\& other things''; somethings loosely used to mean ``\& other persons.'' The phrase is equivalent to \textit{\& the erst, \& so forth}, \& hence is not to be used if 1 of these would be insufficient -- i.e., if the reader would be left in doubt as to any important particulars. Least open to objection when it represents the last terms of a list already given almost in full, or immaterial words at the end of a quotation.
	
	At the end of a list introduced by \textit{such as, for example}, or any similar expression, \textit{etc.} is incorrect. In formal writing, \textit{etc.} is a misfit. An item important enough to call for \textit{etc.} is probably important enough to be named.
	\item \textbf{Fact.} Use this word only of matters capable of direct verification, not of matters of judgment. That a particular event happened on a given date \& that lead melts at a certain temperature are facts. But such conclusions as that Napoleon was the greatest of modern generals or that the climate of California is delightful, however defensible they may be, are not properly called facts.
	\item \textbf{Facility.} Why must jails, hospitals, \& schools suddenly become ``facilities''?
	
	\begin{example}
		Parents complained bitterly about the fire hazard in the wooden facility.
		
		$\hookrightarrow$ Parents complained bitterly about the fire hazard in the wooden schoolhouse.
		
		He has been appointed warden of the new facility. $\to$ He has been appointed warden of the new prison.
	\end{example}
	\item \textbf{Factor.} A hackneyed\footnote{\textbf{hackneyed} [a] used too often \& therefore boring, \textsc{synonym}: clich\'ed.} word; the expressions of which it is a part can usually be replaced by something more direct \& idiomatic.
	
	\begin{example}
		Her superior training was the great factor in her winning the match. $\to$ She won the match by being better trained.
		
		Air power is becoming an increasingly important factor in deciding battles. $\to$ Air power is playing a larger \& larger part in deciding battles.
	\end{example}
	\item \textbf{Farther. Further.} The 2 words are commonly interchanged, but there is a distinction worth observing: \textit{farther} serves best as a distance word, \textit{further} as a time or quantity word. You chase a ball \textit{farther} than the other fellow; you pursue a subject \textit{further}.
	\item \textbf{Feature.} Another hackneyed word; like \textit{factor}, it usually adds nothing to the sentence in which it occurs.
	
	\begin{example}
		A feature of the entertainment especially worthy of mention was the singing of Allison Jones.
		
		$\hookrightarrow$ (Better use the same number of words to tell what Allison Jones sang \& how she sang it.)
	\end{example}
	As a verb, in the sense of ``offer as a special attraction,'' it is to be avoided.
	\item \textbf{Finalize.} A pompous\footnote{\textbf{pompous} [a] (\textit{disapproving}) showing that you think you are more important than other people, especially by using long \& formal words, \textsc{synonym}: \textbf{pretentious}.}, ambiguous verb. (See Chap. V, Reminder 21.)
	\item \textbf{Fix.} Colloquial in America for \textit{arrange, prepare, mend}. The usage is well established. But bear in mind that this verb is from \textit{figere}: ``to make firm,'' ``to place definitely.'' These are the preferred meanings of the word.
	\item \textbf{Flammable.} \texttt{[Pause at p. 60, move to Chap. 5]}
\end{enumerate}
-- \cite[Chap. 4, pp. 54--73]{Strunk_White2019}

\section{An Approach to Style (With a List of Reminders)}
This section is devoted to study \cite[Chap. 5]{Strunk_White2019}.

``Up to this point, the book has been concerned with what is correct, or acceptable, in the use of English. In this final chapter, we approach style in its broader meaning: style in the sense of what is distinguished \& distinguishing. Here we leave solid ground. Who can confidently say what ignites a certain combination of words, causing them to explode in the mind? Who knows why certain notes in music are capable of stirring the listener deeply, though the same notes slightly rearranged are impotent\footnote{\textbf{impotent} [a] \textbf{1.} having no power to change things or to influence a situation, \textsc{synonym}: \textbf{powerless}; \textbf{2.} (of a man) unable to achieve an erection \& therefore unable to have full sex.}? These are high mysteries, \& this chapter is a mystery story, thinly disguised\footnote{\textbf{disguise} [v] \textbf{1.} to hide the true nature of something so that it cannot be recognized, \textsc{synonym}: conceal; \textbf{2.} disguise somebody\texttt{/}yourself (as somebody\texttt{/}something) to change your appearance so that people cannot recognize you; [n] [countable, uncountable] a thing that you wear or use to change your appearance so that people do not recognize you.}. There is no satisfactory explanation of style, no infallible guide to good writing, no assurance that a person who thinks clearly will be able to write clearly, no key that unlocks the door, no inflexible rule by which writers may shape their course. Writers will often find themselves steering by stars that are disturbingly in motion.

The preceding chapters contain instruction drawn from established English usage; this one contains advice drawn from a writer's experience of writing. Since the book is a rule book, these cautionary remarks, these subtly dangerous hints, are presented in the form of rules, but they are, in essence, mere gentle reminders: they state what most of us know \& at times forget.

Style is an increment\footnote{\textbf{increment} [n] \textbf{1.} an increase or addition, especially 1 of a series; \textbf{2.} a regular increase in salary.} in writing. When we speak of Fitzgerald's style, we don't mean his command of the relative pronoun, we mean the sound his words make on paper. All writers, by the way they use the language, reveal something of their spirits, their habits, their capacities, \& their biases\footnote{\textbf{bias} [n] \textbf{1.} [uncountable, countable] the fact that the results of research or an experiment are not accurate because a particular factor has not been considered when collecting the information; \textbf{2.} [uncountable, countable, usually singular] a strong feeling in favor of or against 1 group of people, or 1 side in an argument, in a way that influences your decisions in an unfair way; \textbf{3.} [countable, usually singular] an interest in 1 area or subject more than others; [v] \textbf{1.} \textbf{bias something} to have an effect on the results of research or an experiment so that they do not show the real situation; \textbf{2.} \textbf{bias somebody\texttt{/}something} to influence somebody's opinions or decisions, sometimes in an unfair way.}. \fbox{This is inevitable as well as enjoyable.} \fbox{All writing is communication}; \fbox{creative writing is communication through revelation -- it is the Self escaping into the open}. \fbox{No writer long remains incognito.}

If you doubt that style is something of a mystery, try rewriting a familiar sentence \& see what happens. Any much-quoted sentence will do. Suppose we take ``These are the times that try men's souls.'' Here we have 8 short, easy words, forming a simple declarative sentence. The sentence contains no flashy ingredient such ass ``Damn the torpedoes\footnote{\textbf{torpedo} [n] a long, narrow bomb that is fired under the water from a ship or submarine \& that explodes when it hits a ship, etc.}!'' \& the words, as you see, are ordinary. Yet in that arrangement, they have shown great durability\footnote{\textbf{durability} [n] [uncountable] the quality of being able to last for a long time without breaking or getting weaker.}; the sentence is into its 3rd century. Now compare a few variations:

\begin{example}
	Times like these try men's souls.
	
	How trying it is to live in these times!
	
	These are trying times for men's souls.
	
	Soulwise, these are trying times.
\end{example}
It seems unlikely that Thomas Paine could have made his sentiment\footnote{\textbf{sentiment} [n] \textbf{1.} [countable, uncountable] a feeling or an opinion; \textbf{2.} [uncountable] (\textit{sometimes disapproving}) feelings of romantic love, sadness, etc. which may be too strong or not appropriate.} stick if he had couched it in any of these forms. But why not? No fault of grammar can be detected in them, \& in every case the meaning is clear. Each version is correct, \& each, for some reason that we can't readily put our finger on, is marked for oblivion\footnote{\textbf{oblivion} [n] [uncountable] \textbf{1.} a state in which you are not aware of what is happening around you, usually because you are unconscious or asleep; \textbf{2.} the state in which somebody\texttt{/}something has been forgotten \& is no longer famous or important, \textsc{synonym}: \textbf{obscurity}; \textbf{3.} a state in which something has been completely destroyed.}. We could, of course, talk about ``rhythm\footnote{\textbf{rhythm} [n] \textbf{1.} [countable] \textbf{rhythm (of something)} a regular pattern of changes or events; \textbf{2.} [countable, uncountable] a strong regular repeated pattern of sounds or movements.}'' \& ``cadence\footnote{\textbf{cadence} [n] \textbf{1.} (\textit{formal}) the rise \& fall of the voice in speaking; \textbf{2.} the end of a musical phrase.},'' but the talk would be vague \& unconvincing\footnote{\textbf{unconvincing} [a] not seeming true or real; not making you believe that something is true, \textsc{opposite}: \textbf{convincing}.}. We could declare \textit{soulwise} to be a silly word, inappropriate to the occasion; but even that won't do -- it does not answer the main question. Are we even sure \textit{soulwise} is silly? If \textit{otherwise} is a serviceable\footnote{\textbf{serviceable} [a] of good enough quality to be used.} word, what's the matter with \textit{soulwise}?

Here is another sentence, this one by a later Tom. It is not a famous sentence, although its author (Thomas Wolfe) is well known. ``Quick are the mouths of earth, \& quick the teeth that fed upon this loveliness.'' The sentence would not take a prize for clarity, \& rhetorically it is at the opposite pole from ``These are the times.'' Try it in a different form, without the inversions:

\begin{example}
	The mouths of earth are quick, \& the teeth that fed upon this loveliness are quick, too.
\end{example}
The author's meaning is still intact, but not his overpowering emotion. What was poetical \& sensuous has become prosy \& wooden; instead of the secret sounds of beauty, we are left with the simple crunch of mastication. (Whether Mr. Wolfe was guilty of overwriting is, of course, another question -- one that is not pertinent here.)

With some writers, style not only reveals the spirit of the man but reveals his identity, as surely as would his fingerprints. Here, following, are 2 brief passages from the works of 2 American novelists. The subject in each case is languor\footnote{\textbf{languor} [n] [uncountable, singular] (\textit{literary}) the pleasant state of feeling lazy \& without energy.}. In both, the words used are ordinary, \& there is nothing eccentric about the construction.

\begin{example}
	He did not still feel weak, he was merely luxuriating in that supremely gutful lassitude of convalescence in which time, hurry, doing, did not exist, the accumulating seconds \& minutes \& hours to which in its well state the body is slave both waking \& sleeping, now reversed \& time now the lip-server \& mendicant to the body's pleasure instead of the body thrall to time's headlong course.
	
	Manuel drank his brandy. He felt sleepy himself. It was too hot to go out into the town. Besides there was nothing to do. He wanted to see Zurito. He would go to sleep while he waited.
\end{example}
Anyone acquainted with Faulkner \& Hemingway will have recognized them in these passages \& perceived which was which. How different are their languors!

Or take 2 American poets, stopping at evening. One stops by woods, the other by laughing flesh.

\begin{example}
	My little horse must think it queer
	
	To stop without a farmhouse near
	
	Between the woods \& frozen lake
	
	The darkest evening of the year.
\end{example}

\begin{example}
	I have perceived that to be with those I like is enough,
	
	To stop in company with the rest at evening is enough,
	
	To be surrounded by beautiful, curious, breathing,
	
	laughing flesh is enough $\ldots$
\end{example}
Because of the characteristic styles, there is little question about identity here, \& if the situations were reversed, with Whitman stopping by woods \& Frost by laughing flesh (not 1 of his regularly scheduled stops), the reader would know who was who.

Young writers often suppose that style is a garnish\footnote{\textbf{garnish} [v] \textbf{garnish something (with something)} to decorate a dish of food with a small amount of another food; [n] [countable, uncountable] a small amount of food that is used to decorate a larger dish of food.} for the meat of prose, a sauce by which a dull dish is made palatable\footnote{\textbf{palatable} [a] \textbf{1.} (of food or drink) having a pleasant or acceptable taste; \textbf{2.} \textbf{palatable (to somebody)} pleasant or acceptable to somebody, \textsc{opposite}: \textbf{unpalatable}.}. Style has no such separate entity; it is nondetachable, unfilterable. The beginner should approach style warily, realizing that it is an expression of self, \& should turn resolutely away from all devices that are popularly believed to indicate style -- all mannerisms, tricks, adornments\footnote{\textbf{adornment} [n] \textbf{1.} [countable] something that you wear to make yourself look more attractive; something used to decorate a place or an object; \textbf{2.} [uncountable] the action of making something\texttt{/}somebody look more attractive by decorating it or them with something.}. The approach to style is by way of plainness, simplicity, orderliness, sincerity.

Writing is, for most, laborious \& slow. The mind travels faster than the pen; consequently, writing becomes a question of learning to make occasional wing shots, bringing down the bird of thought as it flashes by. A writer is a gunner, sometimes waiting in the blind for something to come in, sometimes roaming the countryside hoping to scare something up. Like other gunners, the writer must cultivate patience, working many covers to bring down 1 partridge\footnote{\textbf{partridge} [n] [countable, uncountable] a brown bird with a round body \& a short tail, that people hunt for sport or food; the meat of this bird.}. Here, following, are some suggestions \& cautionary hints that may help the beginner find the way to a satisfactory style.'' -- \cite[Chap. 5, pp. 75--77]{Strunk_White2019}

\subsection{Place yourself in the background}
``Write in a way that draws the reader's attention to the sense \& substance of the writing, rather than to the mood \& temper of the author. If the writing is solid \& good, the mood \& temper of the writer will eventually be revealed \& not at the expense of the work. Therefore, the 1st piece of advice is this: to achieve style, begin by affecting none -- i.e., place yourself in the background. A careful \& honest writer does not need to worry about style. As you become proficient in the use of language, your style will emerge, because you yourself will emerge, \& when this happens you will find it increasingly easy to break through the barriers that separate you from other minds, other hearts -- which is, of course, the purpose of writing, as well as its \fbox{principal reward}. Fortunately, the act of composition, or creation, disciplines the mind; writing is 1 way to go about thinking, \& the practice \& habit of writing not only drain the mind but supply it, too.'' -- \cite[Chap. 5, Sect. 1, p. 78]{Strunk_White2019}

\subsection{Write in a way that comes naturally}
``Write in a way that comes easily \& naturally to you, using words \& phrases that come readily to hand. But do not assume that because you have acted naturally your product is without law.

\fbox{The use of language begins with imitation.} The infant imitates the sounds made by its parents; the child imitates 1st the spoken language, then the stuff of books. The imitative life continues long after the writer is secure in the language, for it is almost impossible to avoid imitating what one admires. Never imitate consciously, but do not worry about being an imitator; take pains instead to admire what is good. Then when you write in a way that comes naturally, you will echo the halloos that bear repeating.''  -- \cite[Chap. 5, Sect. 2, p. 79]{Strunk_White2019}

\subsection{Work from a suitable design}
``Before beginning to compose something, gauge the nature \& extent of the enterprise \& work from a suitable design. (See Chap. II, Rule 12.) Design informs even the simplest structure, whether of brick \& steel or of prose. You raise a pup tent from 1 sort of vision, a cathedral\footnote{\textbf{cathedral} [n] the main church of a district, under the care of a bishop ($=$ a priest of high rank).} from another. This does not a mean that you must sit with a blueprint always in front of you, merely that you had best anticipate wha tyou are getting into. To compose a laundry list, you can work directly from the pile of soiled garments, ticking them off 1 by 1. But to write a biography, you will need at least a rough scheme; you cannot plunge in blindly \& start ticking off fact after fact about your subject, lest you miss the forest for the trees \& there be no end to your labors.

Sometimes, of course, impulse \& emotion are more compelling than design. If you are deeply troubled \& are composing a letter appealing for mercy or for love, you had best not attempt to organize your emotions; the prose will have a better chance if the emotions are left in disarray -- which you'll probably have to do anyway, since feelings do not usually lend themselves to rearrangement. But even the kind of writing that is essentially adventurous \& impetuous will on examination be found to have a secret plan: Columbus didn't just sail, he sailed west, \& the New World took shape from this simple \&, we now think, sensible design.'' -- \cite[Chap. 5, Sect. 3, p. 80]{Strunk_White2019}

\subsection{Write with nouns \& verbs}
``Write with nouns \& verbs, not with adjectives \& adverbs. The adjective hasn't been built that can pull a weak or inaccurate noun out of a tight place. This is not to disparage adjectives \& adverbs; they are indispensable parts of speech. Occasionally they surprise us with their power, as in

\begin{example}
	Up the airy mountain,
	
	Down the rushy glen,
	
	We daren't go a-hunting
	
	For fear of little men $\ldots$
\end{example}
The nouns \textit{mountain} \& \textit{glen} are accurate enough, but had the mountain not become airy, the glen rushy, William Ailing-ham might never have got off the ground with his poem. In general, however, it is nouns \& verbs, not their assistants, that give good writing its toughness \& color.'' -- \cite[Chap. 5, Sect. 4, p. 81]{Strunk_White2019}

\subsection{Revise \& rewrite}
``Revising is part of writing. Few writers are so expert that they can produce what they are after on the 1st try. Quite often you will discover, on examining the completed work, that there are serious flaws in the arrangement of the material, calling for transpositions. When this is the case, a word processor can save you time \& labor as you rearrange the manuscript. You can select material on your screen \& move it to  a more appropriate spot, or, if you cannot find the right spot, you can move the material to the end of the manuscript until you decide whether to delete it. Some writers find that working with a printed copy of the manuscript helps them to visualize the process of change; others prefer to revise entirely on screen. Above all, do not be afraid to experiment with what you have written. Save both the original \& the revised versions; you can always use the computer to restore the manuscript to its original condition, should that course seem best. Remember, it is no sign of weakness or defeat that your manuscript ends up in need of major surgery. This is a common occurrence in all writing, \& among the best writers.'' -- \cite[Chap. 5, Sect. 5, p. 82]{Strunk_White2019}

\subsection{Do not overwrite}
``Rich, ornate\footnote{\textbf{ornate} [a] covered with a lot of decoration, especially when this involves very small or complicated designs.} prose is hard to digest, generally unwholesome, \& sometimes nauseating. If the sickly-sweet word, the overblown phrase are your natural form of expression, as is sometimes the case, you will have to compensate for it by a show of vigor, \& by writing something as meritorious as the Song of Songs, which is Solomon's.

When writing with a computer, you must guard against wordiness. The click \& flow of a word processor can be seductive\footnote{\textbf{seductive} [a] \textbf{1.} sexually attractive; \textbf{2.} attractive in a way that makes you want to have or do something, \textsc{synonym}: \textbf{tempting}.}, \& you may find yourself adding a few unnecessary words or even a whole passage just to experience the pleasure of running your fingers over the keyboard \& watching your words appear on the screen. It is always a good idea to reread your writing later \& ruthlessly delete the excess.'' -- \cite[Chap. 5, Sect. 6, p. 83]{Strunk_White2019}

\subsection{Do not overstate}
``When you overstate, readers will be instantly on guard, \& everything that has preceded your overstatement as well as everything that follows it will be suspect in their minds because they have lost confidence in your judgment on your poise. Overstatement is 1 of the common faults. A single overstatement, wherever or however it occurs, diminishes the whole, \& a single carefree superlative has the power to destroy, for readers, the object of your enthusiasm.'' -- \cite[Chap. 5, Sect. 7, p. 84]{Strunk_White2019}

\subsection{Avoid the use of qualifiers}
``\textit{Rather, very, little, pretty} -- these are the leeches that infest the pond of prose, sucking the blood of words. The constant use of the adjective \textit{little} (except to indicate size) is particularly debilitating; we should all try to do a little better, we should all be very watchful of this rule, for it is a rather important one, \& we are pretty sure to violate it now \& then.'' -- \cite[Chap. 5, Sect. 8, p. 85]{Strunk_White2019}

\subsection{Do not affect a breezy manner}
``The volume of writing is enormous\footnote{\textbf{enormous} [a] extremely large, \textsc{synonym}: \textbf{huge, immense}.}, these days, \& much of it has a short of windiness\footnote{\textbf{windy} [a] \textbf{1.} (of weather, etc.) with a lot of wind; \textbf{2.} (of a place) getting a lot of wind; \textbf{3.} (\textit{informal, disapproving}) (of speech) involving speaking for longer than necessary \& in a way that is complicated \& not clear.} about it, almost as though the author were in a state of euphoria\footnote{\textbf{euphoria} [n] [uncountable] an extremely strong feeling of happiness \& excitement that usually lasts only a short time.}. ``Spontaneous\footnote{\textbf{spontaneous} [a] \textbf{1.} happening naturally, without being made to happen; \textbf{2.} not planned but done because you suddenly want to do it.} me,'' sang Whitman, \&, \&, in his innocence\footnote{\textbf{innocence} [n] [uncountable] \textbf{1.} the fact of not being guilty of a crime, etc., \textsc{opposite}: \textbf{guilt}; \textbf{2.} lack of knowledge \& experience of the world, especially of evil or unpleasant things.}, let loose the hordes of uninspired scribblers who would 1 day confuse spontaneity with genius.

The breezy\footnote{\textbf{breezy} [a] \textbf{1.} with the wind blowing quite strongly; \textbf{2.} having or showing a cheerful \& relaxed manner.}\,\footnote{``Easy breezy'' -- Windranger, DotA 2.} style is often the work of an egocentric\footnote{\textbf{egocentric} [a] thinking only about yourself \& not about what other people need or want, \textsc{synonym}: \textbf{selfish}.}, the person who imagines that everything that comes to mind is of general interest \& that uninhibited prose creates high spirits \& carries the day. Open any alumni\footnote{\textbf{alumni} [n] [plural] (\textit{especially North American English}) the former male \& female students of a school, college or university.} magazine, turn to the class notes, \& you are quite likely to encounter old Spontaneous Me at work -- an aging collegian who writes something like this:

\begin{example}
	Well, guys, here I am again dishing the dirt about your disorderly classmates, after passing a weekend in the Big Apple trying to catch the Columbia hoops tilt \& then a cab-ride from hell through the West Side casbah. \& speaking of news, howzabout tossing a few primo items this way?
\end{example}
This is an extreme example, but the same wind blows, at lesser velocities, across vast expanses of journalistic prose. The author in this case has managed in 2 sentences to commit most of the unpardonable sins: he obviously has nothing to say, he is showing off \& directing the attention of the reader to himself, he is using slang with neither provocation nor ingenuity, he adopts a patronizing air by throwing in the word \textit{primo}, he is humorless (though full of fun), dull, \& empty. He has not done his work. Compare his opening remarks with the following -- a plunge directly into the news:

\begin{example}
	Clyde Crawford, who stroked the varsity shell in 1958, is swinging an oar again after a lapse of 40 years. Clyde resigned last spring as executive sales manager  of the Indiana Flotex Company \& is now a gondolier in Venice.
\end{example}
This, although conventional\footnote{\textbf{conventional} [a] \textbf{1.} [usually before noun] based on what is generally believed; following the way something is usually done; \textbf{2.} (\textit{often disapproving}) tending to follow what is done or considered acceptable by society in general; normal \& ordinary, \& perhaps not very interesting, \textsc{opposite}: \textbf{unconventional}; \textbf{3.} [usually before noun] (especially of weapons) not nuclear; \textbf{4.} (of literature, art or the theater) using a traditional style or method.}, is compact\footnote{\textbf{compact} [a] \textbf{1.} closely \& firmly packed together; \textbf{2.} smaller than is usual for things of the same kind; \textbf{3.} using or filling only a small amount of space; \textbf{4.} (of speech or writing) giving the information that is important using few words or symbols.}, informative\footnote{\textbf{informative} [a] giving useful information.}, unpretentious\footnote{\textbf{unpretentious} [a] (\textit{approving}) not trying to appear more special, intelligent, important, etc. than you really are\texttt{/}it really it, \textsc{opposite}: \textbf{pretentious}.}. The writer has dug up an item of news \& presented it in a straightforward manner. What the 1st writer tried to accomplish by cutting rhetorical capers\footnote{\textbf{caper} [n] \textbf{1.} [usually plural] the small green flower bud of a Mediterranean bush, preserved in vinegar \& used in preparing sauces \& other dishes; \textbf{2.} (\textit{informal}) an activity, especially one that is illegal or dangerous; \textbf{3.} a humorous film that contains a lot of action; \textbf{4.} a short jumping or dancing movement; [v] [intransitive] (\textit{formal}) \textbf{($+$ adv.\texttt{/}prep.)} to run or jump around in a happy \& excited way.} \& by breeziness\footnote{\textbf{breeziness} [n] [uncountable] a cheerful \& relaxed way of behaving.}, the 2nd writer managed to achieve by good reporting, by keeping a tight rein\footnote{\textbf{rein} [n] \textbf{(the reins)} [plural] \textbf{rein (of something)} the state of being in control or the leader of something.} on his material, \& by staying out of the act.'' -- \cite[Chap. 5, Sect. 9, pp. 86--87]{Strunk_White2019}

\subsection{Use orthodox spelling}
``In ordinary composition, use orthodox spelling. Do not write \textit{nite} for \textit{night}, \textit{thru} for \textit{through}, \textit{pleez} for \textit{please}, unless you plan to introduce a complete system of simplified spelling \& are prepared to take the consequences.

In the original edition of \textit{The Elements of Style}, there was a chapter on spelling. In it, the author had this to say:

\begin{quotation}\it
	The spelling of English words is not fixed \& invariable\footnote{\textbf{invariable} [a] always the same; never changing, \textsc{synonym}: \textbf{unchanging}.}, nor does it depend on any other authority\footnote{\textbf{authority} [n] \textbf{1.} [uncountable] the power to give orders to people or to say how things should be done; \textbf{2.} [uncountable] official permission or the right to do something; \textbf{3.} [countable] an organization that has the power to make decisions or that has a particular area of responsibility in a country or region; \textbf{4.} [uncountable] the power to influence people because they respect your knowledge or official position; \textbf{5.} [countable] \textbf{authority (on something)} a person with special knowledge, \textsc{synonym}: \textbf{specialist}.} than general statement. At the present day there is practically\footnote{\textbf{practically} [adv] \textbf{1.} almost; very nearly, \textsc{synonym}: \textbf{virtually}; \textbf{2.} in a realistic or sensible way; in real situations.} unanimous\footnote{\textbf{unanimous} [a] \textbf{1.} if a decision or an opinion is \textbf{unanimous}, it is agreed or shared by everyone in a group; \textbf{2.} \textbf{unanimous (in something)} if a group of people are \textbf{unanimous}, they all agree about something.} agreement as to the spelling of most words $\ldots$ At any given moment, however, a relatively small number of words may be spelled in more than 1 way. Gradually, as a rule, 1 of these forms comes to be generally preferred, \& the less customary form comes to look obsolete \& is discarded. From time to time new forms, mostly simplifications, are introduced by innovators, \& either win their place or die of neglect.
	
	The practical objection to unaccepted \& oversimplified spellings is the disfavor with which they are received by the reader. They distract his attention \& exhaust his patience. He reads the form though automatically, without thought of its needless complexity; he reads the abbreviation tho \& mentally supplies the missing letters, at the cost of a fraction of his attention. The writer has defeated his own purpose.
\end{quotation}
The language manages somehow to keep pace with events. A word that has taken hold in our century is \textit{thru-way}; it was born of necessity \& is apparently here to stay. In combination with \textit{way, thru} is more serviceable than \textit{through}; it is a high-speed word for readers who are going 65. \textit{Throughway} would be too long to fit on a road sign, too slow to serve the speeding eye. It is conceivable that because of our thruways, \textit{through} will eventually become \textit{thru} -- after many more thousands of miles of travel.'' -- \cite[Chap. 5, Sect. 10, p. 88]{Strunk_White2019}

\subsection{Do not explain too much}
``It is seldom advisable to tell all. Be sparing, for instance, in the use of adverbs after ``he said,'' ``she replied,'' \& the like: ``he said consolingly''; ``she replied grumblingly.'' Let the conversation itself disclose the speaker's manner or condition. Dialogue heavily weighted with adverbs after the attributive verb is cluttery\footnote{\textbf{clutter} [v] \textbf{clutter something (up) (with something\texttt{/}somebody)} to fill a place or area with too many things, so that it is untidy; [n] [uncountable, singular] (\textit{disapproving}) a lot of things in an untidy state, especially things that are not necessary or are not being used; a lack of order, \textsc{synonym}: \textbf{mess}.}\,\footnote{\textbf{cluttered} [a] \textbf{cluttered (up) (with somebody\texttt{/}something)} covered with, or full of, a lot of things or people, in a way that is untidy, \textsc{opposite}: \textbf{uncluttered}.} \& annoying. Inexperienced writers not only overwork their adverbs but load their attributives with explanatory verbs: ``he consoled,'' ``she congratulated.'' They do this, apparently, in the belief that the word \textit{said} is always in need of support, or because they have been told to do it by experts in the art of bad writing.'' -- \cite[Chap. 5, Sect. 11, p. 89]{Strunk_White2019}

\subsection{Do not construct awkward adverbs}
``Adverbs are easy to build. Take an adjective or a participle, add \textit{-ly}, \& behold\footnote{\textbf{behold} [v] (\textit{old use or literary}) \textbf{behold somebody\texttt{/}something} to look at or see somebody\texttt{/}something.}! you have an adverb. But you'd probably be better off without it. Do not write \textit{tangledly}. \fbox{The word itself is a tangle.} Do not even write \textit{tiredly}. Nobody says \textit{tangledly} \& not many people say \textit{tiredly}. Words that are not used orally\footnote{\textbf{oral} [a] \textbf{1.} [usually before noun] spoken rather than writing, \textsc{opposite}: \textbf{written}; \textbf{2.} [only before noun] connected with the mouth.} are seldom the ones to put on paper.

\begin{example}
	He climbed tiredly to bed. $\to$ He climbed wearily\footnote{\textbf{wearily} [adv] (\textit{formal}) \textbf{1.} in a way that shows somebody is very tired; \textbf{2.} in a way that shows somebody is annoyed \& bored because they have had to do something, hear something, explian something, etc. many times.} to bed.
	
	The lamp cord lay tangledly beneath her chair. $\to$ The lamp cord lay in tangles beneath her chair.
\end{example}
Do not dress words up by adding \textit{-ly} to them, as though putting a hat on a horse.

\begin{example}
	overly $\to$ over; muchly $\to$ much; thusly $\to$ thus''
\end{example}
-- \cite[Chap. 5, Sect. 12, p. 90]{Strunk_White2019}

\subsection{Make sure the reader knows who is speaking}
``Dialogue is a total loss unless you indicate who the speaker is. In long dialogue passages containing no attributives, the reader may become lost \& be compelled to go back \& reread in order to puzzle the thing out. Obscurity is an imposition on the reader, to say nothing of its damage to the work.

In dialogue, make sure that your attributives do not awkwardly interrupt a spoken sentence. Place them where the break would come naturally in speech -- i.e., where the speaker would pause for emphasis, or take a breath. The best test for locating an attributive is to speak the sentence aloud.

\begin{example}
	``Now, my boy, we shall see,'' he said, ``how well you have learned your lesson.'' $\to$ ``Now, my boy,'' he said, ``we shall see how well you have learned your lesson.''
	
	``What's more, they would never,'' she added, ``consent to the plan.'' $\to$ ``What's more,'' she added, ``they would never consent to the plan.''''
\end{example}
-- \cite[Chap. 5, Sect. 13, p. 91]{Strunk_White2019}

\subsection{Avoid fancy words}
``Avoid the elaborate, the pretentious, the coy, \& the cute. Do not be tempted by a 20-dollar word when there is a 10-center handy, ready \& able. Anglo-Saxon is a livelier tongue than Latin, so use Anglo-Saxon words. In this, as in so many matters pertaining to style, \fbox{one's ear must be one's guide}: \textit{gut} is lustier noun than \textit{intestine}, but the 2 words are not interchangeable, because \textit{gut} is often inappropriate, being too coarse for the context. Never call a stomach a tummy without good reason.

If you admire fancy words, if every sky is \textit{beauteous}, every blonde \textit{curvaceous}, everyone intelligent child prodigious, if you are tickled by \textit{discombobulate}, you will have a bad time with Reminder 14. What is wrong, you ask, with \textit{beauteous}? No one knows, for sure. There is nothing wrong, really, with any word -- all are good, but some are better than others. A matter of ear, a matter of reading the books that sharpen the ear.

The line between the fancy \& the plain, between the atrocious\footnote{\textbf{atrocious} [a] \textbf{1.} very bad or unpleasant, \textsc{synonym}: \textbf{terrible}; \textbf{2.} very cruel \& making you feel shocked.} \& the felicitous\footnote{\textbf{felicitous} [a] (\textit{formal or literary}) chosen well; very suitable; giving a good result, \textsc{synonym}: \textbf{apt, happy}.}, is sometimes alarmingly\footnote{\textbf{alarming} [a] causing worry \& fear.} fine. The opening phrase of the Gettysburg address is close to the line, at least by our standards today, \& Mr. Lincoln, knowingly or unknowingly, was flirting with disaster when he wrote ``4 score \& 7 years ago.'' The President could have got into his sentence with plain ``87'' -- a saving of 2 words \& less of a strain on the listeners' powers of multiplication. But Lincoln's ear must have told him to go ahead with 4 score \& 7. By doing so, he achieved cadence\footnote{\textbf{cadence} [n] \textbf{1.} (\textit{formal}) the ries \& fall of the voice in speaking; \textbf{2.} the end of a musical phrase.} while skirting the edge of fanciness. Suppose he had blundered over the line \& written, ``In the year of our Lord seventeen hundred \& seventy-six.'' His speech would have sustained\footnote{\textbf{sustain} [v] \textbf{1.} \textbf{sustain somebody\texttt{/}something} to provide enough of what somebody\texttt{/}something needs in order to live or exist; \textbf{2.} to make something continue for some time without becoming less, \textsc{synonym}: \textbf{maintain}; \textbf{3.} \textbf{sustain something} (\textit{formal}) to experience something bad, \textsc{synonym}: \textbf{suffer}; \textbf{4.} \textbf{sustain something} to provide evidence to support an opinion, a theory, etc., \textsc{synonym}: \textbf{uphold}; \textbf{5.} \textbf{sustain something} (\textit{law}) to decide that a claim, etc. is valid, \textsc{synonym}: \textbf{uphold}.} a heavy blow. Or suppose he had settle for ``87.'' In that case he would have got into his introductory sentence too quickly; the timing would have been bad.

\fbox{The question of ear is vital.}\footnote{\textbf{vital} [a] \textbf{1.} necessary or essential in order for something to succeed or exist; \textbf{2.} [only before noun] connected with or necessary for staying alive.} Only the writer whose ear is reliable is in a position to use bad grammar deliberately; this writer knows for sure when a colloquialism \footnote{\textbf{colloquialism} [n] a word or phrase that is used in conversation but not in formal speech or writing.} is better formal phrasing \& is able to sustain the work at a level of good taste. So cock\footnote{\textbf{cock} [n] \textbf{1.} (\textit{British English}) (also \textbf{rooster} \textit{North American English, British English}) [countable] an adult male chicken; \textbf{2.} [countable] (especially in compounds) a male of any other bird; \textbf{3.} [countable] (taboo, slang) a penis; \textbf{4.} [countable] (also \textit{stopcock}) a tap that controls the flow of liquid or gas through a pipe; \textbf{5.} [singular] (\textit{British English, old-fashioned, slang}) used as a friendly form of address between men.} your ear. Years ago, students were warned not to end a sentence with a preposition; time, of course, has softened that rigid decree\footnote{\textbf{decree} [n] \textbf{1.} [countable, uncountable] an official order from a ruler or government that becomes the law; \textbf{2.} [countable] a decision that is made in court; [v] to decide, judge or order something officially.}. Not only is the preposition acceptable at the end, sometimes it is more effective in that spot than anywhere else. ``A claw hammer, not an ax, was the tool he murdered her with.'' This is preferable to ``A claw hammer, not an ax, was the tool with which he murdered her.'' Why? Because it sounds more violent, more like murder. \fbox{A matter of ear.}

And would you write ``The worst tennis player around here is I'' or ``The worst tennis player around here is me''? The 1st is good grammar, the 2nd is good judgment -- although the \textit{me} might not do in all contexts.

The split infinitive is another trick of rhetoric in which the ear must be quicker than the handbook. Some infinitives seem to improve on being split, just as a stick of round stovewood does. ``I cannot bring myself to really like the fellow.'' The sentence is relaxed, the meaning is clear, the violation is harmless \& scarcely perceptible\footnote{\textbf{perceptible} [a] \textbf{1.} great enough to be able to be noticed, \textsc{synonym}: \textbf{noticeable}; \textbf{2.} that can be noticed or felt with the senses.}. Put the other way, the sentence becomes stiff, needlessly formal. A matter of ear.

There are times when the ear not only guides us through difficult situations but also saves us from minor or major embarrassments of prose. The ear, e.g., must decide when to omit \textit{that} from a sentence, when to retain it. ``She knew she could do it'' is preferable to ``She knew that she could do it'' -- simpler \& just as clear. But in many cases the \textit{that} is needed. ``He felt that his big nose, which was sunburned, made him look ridiculous.'' Omit the \textit{that} \& you have ``He felt his big nose $\ldots$'''' -- \cite[Chap. 5, Sect. 14, pp. 92--93]{Strunk_White2019}

\subsection{Do not use dialect unless your ear is good}
``Do not attempt to use dialect\footnote{\textbf{dialect} [n] [countable, uncountable] the form of a language that is spoken in 1 area with grammar, words \& pronunciation that may be different from other forms of the same language.} unless you are a devoted student of the tongue\footnote{\textbf{tongue} [n] \textbf{1.} the soft part in the mouth that moves around, used for tasting, swallowing, speaking, etc.; \textbf{2.} (\textit{formal} or \textit{literary}) a language.} you hope to reproduce\footnote{\textbf{reproduce} [v] \textbf{1.} [transitive] \textbf{reproduce something} to produce something again; to make something happen again in the same way; \textbf{2.} [transitive] \textbf{reproduce something} to make a copy of a picture, piece of text, etc.; to include a copy of a picture, etc.; \textbf{3.} [intransitive, transitive] (of people, animals, plants, etc.) to produce young.}. If you use dialect, be consistent. The reader will become impatient or confused upon finding 2 or more versions of the same word or expression. In dialect it is necessary to spell phonetically, or at least ingeniously, to capture unusual inflections. Take, e.g., the word \textit{once}. If often appears in dialect writing as \textit{oncet}, but \textit{oncet} looks as though it should be pronounced ``onset.'' A better spelling would be \textit{wunst}. But if you write it \textit{oncet} once, write it that way throughout. The best dialect writers, by \& large, are economical\footnote{\textbf{economical} [a] \textbf{1.} providing good service or value in relation to the amount of time or money spent; \textbf{2.} using no more of something than is necessary.} of their talents; they use the minimum, not the maximum, of deviation\footnote{\textbf{deviation} [n] \textbf{1.} [uncountable, countable] \textbf{deviation (from something)} a difference from what is expected or usual; \textbf{2.} [countable] \textbf{deviation (from something)} (\textit{statistics}) the amount by which a single measurement is different from a fixed value such as the mean; \textbf{3.} [uncountable] \textbf{deviation (from something)} behavior that is different from what most people consider normal or acceptable.} from the norm, thus sparing their readers as well as convincing them.'' -- \cite[Chap. 5, Sect. 15, p. 94]{Strunk_White2019}

\subsection{Be clear}
``Clarity is not the prize in writing, nor is it always the principal mark of a good style. There are occasions when obscurity\footnote{\textbf{obscurity} [n] \textbf{1.} [uncountable] the state in which somebody\texttt{/}something is not well known or has been forgotten; \textbf{2.} [uncountable, countable, usually plural] \textbf{obscurity (of something)} the quality of being difficult to understand; something that is difficult to understand.} serves a literary yearning\footnote{\textbf{yearning} [n] [countable, uncountable] (\textit{formal}) a strong \& emotional desire, \textsc{synonym}: \textbf{longing}.}, if not a literary purpose, \& there are writers whose mien\footnote{\textbf{mien} [n] [singular] (\textit{formal or literary}) a person's appearance or manner that shows how they are feeling.} is more overcast\footnote{\textbf{overcast} [a] covered with clouds; not bright.} than clear. But since writing is communication, clarity can only be a virtue. \& although there is no substitute for merit in writing, clarity comes closest to being one. Even to a writer who is being intentionally obscure or wild of tongue we can say, ``Be obscure clearly! Be wild of tongue in a way we can understand!'' Even to writers of market letters, telling us (but not telling us) which securities are promising, we can say, ``Be cagey plainly! be elliptical in a straightforward fashion!''

Clarity, clarity, clarity. When you become hopelessly mired in a sentence, it is best to start fresh; do not try to fight your way through against the terrible odds of syntax. Usually what is wrong is that the construction has become too involved at some point; the sentence needs to be broken apart \& replaced by 2 or more shorter sentences.

Muddiness is not merely a disturber of prose, it is also a destroyer of life, of hope: death on the highway caused by a badly worded road sign, heartbreak among lovers caused by a misplaced phrase in a well-intentioned letter, anguish of a traveler expecting to be met at a railroad station \& not being met because of a sliphod telegram. Think of the tragedies that are rooted in ambiguity, \& be clear! When you say something, make sure you have said it. The chances of your having said it are only fair.'' -- \cite[Chap. 5, Sect. 16, p. 95]{Strunk_White2019}

\subsection{Do not inject opinion}
``Unless there is a good reason for its being there, do not inject\footnote{\textbf{inject} [v] \textbf{1.} [transitive, intransitive] to put a drug or another substance into a person's or an animal's body using a syringe; \textbf{2.} [transitive] to put a liquid into something using a syringe or similar equipment; \textbf{3.} [transitive] \textbf{inject something (into something)} to add a particular quality to something; \textbf{4.} [transitive] \textbf{inject something (into something)} to give money to an organization or a project so that it can function.} opinion into a piece of writing. We all have opinions about almost everything, \& the temptation to toss them in is great. To air one's views gratuitously\footnote{\textbf{gratuitously} [adv] (\textit{disapproving}) without any good reason or purpose, in a way that may have harmful effects, \textsc{synonym}: \textbf{unnecessarily}.}, however, is to imply that the demand for them is brisk, which may not be the case, \& which, in any event, may not be relevant to the discussion. Opinions scattered indiscriminately about leave the mark of egotism\footnote{\textbf{egoism} [n] (also \textbf{egotism}) [uncountable] (\textit{disapproving}) the fact of thinking that you are better or more important than anyone else.} on a work. Similarly, to air one's views at an improper time may be in bad taste. If you have received a letter inviting you to speak at the dedication of a new cat hospital, \& you hate cats, your reply, declining the invitation, does not necessarily have to cover the full range of your emotions. You must make it clear that you will not attend, but you do not have to let fly at cats. The writer of the letter asked a civil question; attack cats, then, only if you can do so with good humor, good taste, \& in such a way that your answer will be courteous\footnote{\textbf{courteous} [a] polite, especially in a way that shows respect, \textsc{opposite}: \textbf{discourteous}.} as well as responsive. Since you are out of sympathy with cats, you may quite properly give this as a reason for not appearing at the dedicatory ceremonies of a cat hospital. But bear in mind that your opinion of cats was not sought, only your services as a speaker. Try to keep things straight.'' -- \cite[Chap. 5, Sect. 17, p. 96]{Strunk_White2019}

\subsection{Use figures of speech sparingly}
``The simile\footnote{\textbf{simile} [n] [countable, uncountable] a word or phrase that compares something to something else, using the words \textit{like} or \textit{as}, e.g., \textit{a face like a mask} or \textit{as white as snow}; the use of such words \& phrases.} is a common device \& a useful one, but similes coming in rapid fire, one right on top of another, are more distracting than illuminating. Readers need time to catch their breath; they can't be expected to compare everything with something else, \& no relief in sight.

When you use metaphor\footnote{\textbf{metaphor} [n] [countable, uncountable] \textbf{1.} \textbf{metaphor (for something)} something that represents another situation or idea; \textbf{2.} a word or phrase used to describe somebody\texttt{/}something else, in a way that is different from its normal use, in order to show that the 2 things have the same qualities \& to make the description more powerful; the use of such words \& phrases.}, do not mix it up. I.e., don't start by calling something a swordfish\footnote{\textbf{swordfish} [n] [countable, uncountable] (plural \textbf{swordfish}) a large sea fish with a very long, thin, pointed upper jaw.} \& end by calling it an hourglass\footnote{\textbf{hourglass} [n] a glass container holding sand that takes exactly an hour to pass through a small opening between the top \& bottom sections; [a] [only before noun] a woman who has an \textbf{hourglass} figure, shape, etc. has large breasts \& hips \& a small waist.}.'' -- \cite[Chap. 5, Sect. 18, p. 97]{Strunk_White2019}

\subsection{Do not take shortcuts at the cost of clarity}
``Do not use initials for the names of organizations or movements unless you are certain the initials will be readily understood. Write things out. Not everyone knows that MADD means Mothers Against Drunk Driving, \& even if everyone did, there are babies being born every minute who will someday encounter the name for the 1st time. They deserve to see the words, not simply the initials. A good rule is to start your article by writing out names in full, \& then, later, when you readers have got their bearings, to shorten them.

Many shortcuts are self-defeating\footnote{\textbf{self-defeating} [a] causing more problems \& difficulties instead of solving them; not achieving what you wanted to achieve but having an opposite effect.}; they waste the reader's time instead of conserving it. There are all sorts of rhetorical stratagems\footnote{\textbf{stratagem} [n] (\textit{formal}) a trick or plan that you use to gain an advantage or to trick an opponent.} \& devices that attract writers who hope to be pithy\footnote{\textbf{pithy} [a] (\textit{approving}) (of a comment, piece of writing, etc.) short but expressed well \& full of meaning.}, but most of them are simply bothersome. The longest way round is usually the shortest way home, \& the one truly reliable shortcut in writing is to choose words that are strong \& surefooted\footnote{\textbf{sure-footed} [a] \textbf{1.} not likely to fall when walking or climbing on rough ground; \textbf{2.} confident \& unlikely to make mistakes, especially in difficult situations.} to carry readers on their way.'' -- \cite[Chap. 5, Sect. 19, p. 98]{Strunk_White2019}

\subsection{Avoid foreign languages}
``The writer will occasionally find it convenient or necessary to borrow from other languages. Some writers, however, from sheer exuberance\footnote{\textbf{exuberance} [n] [uncountable] the quality of being full of energy, excitement \& happiness.} or a desire to show off, sprinkle\footnote{\textbf{sprinkle} [v] \textbf{1.} [transitive] to shake small pieces of something or drops of a liquid on something; \textbf{2.} [transitive, usually passive] \textbf{sprinkle something with something} to include a few of something in something else, \textsc{synonym}: \textbf{strew}; \textbf{3.} [intransitive] (\textit{North American English}) if it sprinkles, it rains lightly, \textsc{synonym}: \textbf{drizzle}; [n] \textbf{1.} (also \textbf{sprinkling}) [usually singular] a small amount of a substance that is dropped somewhere, or a number of things or people that are spread or included somewhere; \textbf{2.} (\textit{especially North American English}) light rain.} their work liberally\footnote{\textbf{liberally} [adv] \textbf{1.} in large amounts, \textsc{synonym}: \textbf{freely}; \textbf{2.} in a way that is not completely accurate or exact.} with foreign expressions, with no regard for the reader's comfort. It is a bad habit. Write in English.'' -- \cite[Chap. 5, Sect. 20, p. 99]{Strunk_White2019}

\subsection{Prefer the standard to the offbeat}
``Young writers will be drawn at every turn toward eccentricities\footnote{\textbf{eccentricity} [n] \textbf{1.} [uncountable] behavior that people think is strange or unusual; the quality of being unusual \& different from other people; \textbf{2.} [countable, usually plural] an unusual act or habit.}  in language. They will hear the beat of new vocabularies, the exciting rhythms of special segments\footnote{\textbf{segment} [n] \textbf{1.} a part of something that is separate from the other parts or can be considered separately; \textbf{2.} (\textit{geometry}) part of a shape separated from the rest by at least 1 line or plane; the line between 2 points.} of their society, each speaking a language of its own. All of us come under the spell of these unsettling\footnote{\textbf{unsettling} [a] making you feel upset, nervous or worried.} drums; the problem for beginners is to listen to them, learn the words, feel the vibrations\footnote{\textbf{vibration} [n] [countable, uncountable] \textbf{1.} \textbf{vibration (of something)} a continuous shaking movement; \textbf{2.} \textbf{vibration (of something)} (\textit{physics}) oscillation in a substance about its equilibrium state.}, \& not be carried away.

Youths invariably\footnote{\textbf{invariably} [adv] in every case; every time, \textsc{synonym}: \textbf{always}.} speak to other youths in a tongue of their own devising: they renovate\footnote{\textbf{renovate} [v] \textbf{renovate something} to repair \& paint an old building, a piece of furniture, etc. so that it is in good conditions again.} the language with a wild vigor, as they would a basement apartment. By the time this paragraphs sees print, \textit{psyched, nerd, ripoff, dude, geek, \& funky} will be the words of yesteryear, \& we will be fielding more recent ones that have come bouncing into our speech -- some of them into our dictionary as well. A new word is always up for survival. Many do survive. Others grow stale\footnote{\textbf{stale} [a] \textbf{1.} (of food, especially bread \& cake) no longer fresh \& therefore unpleasant to eat; \textbf{2.} (of air, smoke, etc.) no longer fresh; smelling unpleasant; \textbf{3.} something that is \textbf{stale} has been said or done too many times before \& is no longer interesting or exciting; \textbf{4.} a person who is \textbf{stale} has done the same thing for too long \& so is unable to do it well or produce any new ideas.} \& disappear. Most are, at least in their infancy\footnote{\textbf{infancy} [n] [uncountable] \textbf{1.} the time when a child is a baby or very young; \textbf{2.} the early development of something.}, more appropriate to conversation than to composition.

Today, the language of advertising enjoys an enormous circulation\footnote{\textbf{circulation} [n] \textbf{1.} [uncountable] the movement of blood around the body; \textbf{2.} [uncountable] the movement of something (e.g. air, water or gas) around an area or inside a system or machine; \textbf{3.} [uncountable] the fact of goods, information or ideas passing from 1 person or place to another; \textbf{4.} [countable, usually singular] the usual number of copies of a newspaper or magazine that are sold each day, week, etc.}. With its deliberate\footnote{\textbf{deliberate} [a] done on purpose rather than by accident, \textsc{synonym}: \textbf{intentional}, \textsc{opposite}: \textbf{unintentional}; [v] [intransitive, transitive] to think very carefully about something, usually before making a decision.} infractions\footnote{\textbf{infraction} [n] [countable, uncountable] (\textit{formal}) an act of breaking a rule or law, \textsc{synonym}: \textbf{infringement}.} of grammatical rules \& its crossbreeding\footnote{\textbf{cross-breeding} [n] [uncountable] the activity of making an animal or plant breed ($=$ produce young animals\texttt{/}new plants) with a different type.} of the parts of speech, it profoundly\footnote{\textbf{profoundly} [adv] \textbf{1.} in a way that has a very great effect on somebody\texttt{/}something; \textbf{2.} extremely; \textbf{3.} (\textit{medical}) very seriously; completely.} influences the tongues \& pens of children \& adults. Your new kitchen range is so revolutionary it \textit{obsoletes} all other ranges. Your counter top is beautiful because it is \textit{accessorized} with gold-plated faucets. Your cigarette tastes good \textit{like} a cigarette should. \&, \textit{like the man says}, you will want to try one. You will also, in all probability, want to try writing that way, using that language. You do so at your peril\footnote{\textbf{peril} [n] (\textit{formal or literary}) \textbf{1.} [uncountable] serious danger; \textbf{2.} [countable, usually plural] \textbf{peril (of something)} the fact of something being dangerous or harmful.}, for it is the language of mutilation\footnote{\textbf{mutilation} [n] [uncountable, countable] \textbf{1.} severe damage to somebody's body, especially when part of it is cut or torn off; the act of causing such damage; \textbf{2.} severe damage to something; the act of causing severe damage to something.}.

Advertisers are quite understandably\footnote{\textbf{understandably} [adv] in a way that seems normal \& reasonable in a particular situation, \textsc{synonym}: \textbf{naturally}.} interested in what they call ``attention getting.'' The man photographed must have lost an eye or grown a pink beard, or he must have 3 arms or be sitting wrong-end-to on a horse. This technique is proper in its place, which is the world of selling, but the young writer had best not adopt the device of mutilation in ordinary composition, whose purpose is to engage, not paralyze\footnote{\textbf{paralyse} [v] (\textit{British English}) (\textit{North American English} \textbf{paralyze}) [often passive] \textbf{1.} \textbf{paralyze somebody} to make somebody unable to feel or move all or part of their body; \textbf{2.} \textbf{paralyze something} to prevent something from functioning normally.}, the readers senses. Buy the old-plated faucets if you will, but do not accessorize\footnote{\textbf{accessorize} [v] (\textit{British English also} \textbf{accessorise}) \textbf{accessorize something} to add fashionable items or extra decorations to something, especially to your clothes.} your prose. To use the language well, do not begin by hacking it to bits; accept the whole body of it, cherish its classic form, its variety, \& its richness.

Another segment of society that has constructed a language of its own business. People in business say that toner\footnote{\textbf{toner} [n] [uncountable, countable] \textbf{1.} a type of ink ($=$ colored liquid) used in machines that print or photocopy; \textbf{2.} a liquid or cream used for making the skin on your face tighter \& smoother.} cartridges\footnote{\textbf{cartridges} [n] \textbf{1.} (\textit{North American English also} \textbf{shell}) a tube or case containing explosive \& a bullet or shot, for shooting from a gun; \textbf{2.} a case containing something that is used in a machine, e.g. ink for a printer, film for a camera, etc. Cartridges are put into the machine \& can be removed \& replaced when they are finished or empty; \textbf{3.} a thin tube containing ink ($=$ colored liquid for writing) which you put inside a pen.} are \textit{in short supply}, that they have \textit{updated} the next shipment of these cartridges, \& they they will \textit{finalize} their recommendations at the next meeting of the board. They are speaking a language familiar \& dear to them. Its portentous\footnote{\textbf{portentous} [a] \textbf{1.} (\textit{literary}) important as a sign or a warning of something that is going to happen in the future, especially when it is something unpleasant; \textbf{2.} (\textit{formal, disapproving}) very serious \& intended to impress people, \textsc{synonym}: \textbf{pompous}.} nouns \& verbs invest ordinary events with high adventure; executives walk among toner cartridges, caparisoned\footnote{\textbf{caparisoned} [a] in the past a \textbf{caparisoned} horse or other animal was one covered with a decorated cloth.} like knights. We should tolerate them -- every person of spirit wants to ride a white horse. The only question is whether business vocabulary is helpful to ordinary prose. Usually, the same ideas can be expressed less formidably\footnote{\textbf{formidably} [adv] in a way that makes you feel fear \&\texttt{/}or respect, because something is impressive or powerful or seems very difficult.}, if one makes the effort. A good many of the special words of business seem designed more to express the user's dreams than to express a precise meaning. Not all such words, of course, can be dismissed summarily\footnote{\textbf{summarily} [adv] immediately, without paying attention to the normal process that should be followed.}; indeed, no word in the language can be dismissed offhand by anyone who has a healthy curiosity. \textit{Update} isn't a bad word; in the right setting it is useful. In the wrong setting, though, it is destructive, \& the trouble with adopting coinages too quickly is that they will bedevil\footnote{\textbf{bedevil} [v] (\textit{formal}) \textbf{bedevil somebody\texttt{/}something} to cause a lot of problems for somebody\texttt{/}something over a long period of time, \textsc{synonym}: \textbf{beset}.} one by insinuating\footnote{\textbf{insinuate} [v] \textbf{1.} (\textit{usually disapproving}) to suggest directly that something unpleasant is true, \textsc{synonym}: \textbf{imply}; \textbf{2.} \textbf{insinuate yourself into something} (\textit{formal, disapproving}) to succeed in gaining somebody's respect, trust, etc. so that you can use the situation to your own advantage; \textbf{3.} \textbf{insinuate yourself\texttt{/}something $+$ adv.\texttt{/}prep.} (\textit{formal}) to slowly move yourself or a part of your body into a particular position or place.} themselves where they do not belong. This may sound like rhetorical snobbery\footnote{\textbf{snobbery} [n] the attitudes \& behavior of people who are snobs}, or plain stuffiness\footnote{\textbf{stuffiness} [n] [uncountable] \textbf{1.} (\textit{informal, disapproving}) the fact of being very serious, formal, boring or old-fashioned; \textbf{2.} the fact of being warm in an unpleasant way \& without enough fresh air; \textbf{3.} (\textit{especially North American English}) the fact of having a blocked nose because you have a cold.}; but you will discover, in the course of your work, that the setting of a word is just as restrictive as the setting of a jewel. The general rule here is to prefer the standard. \textit{Finalize}, for instance, is not standard; it is special, \& it is a peculiarly fuzzy \& silly word. Does it mean ``terminate,'' or does it mean ``put into final form''? One can't be sure, really, what it means, \& one gets the impression that the person using it doesn't know, either, \& doesn't want to know.

The special vocabularies of the law, of the military, of government are familiar to most of us. Even the world of criticism has a modest pouch of private words (\textit{luminous, taut}), whose only virtue is that they are exceptionally nimble\footnote{\textbf{nimble} [a] \textbf{1.} able to move quickly \& easily, \textsc{synonym}: \textbf{agile}; \textbf{2.} able to think, react \& adapt quickly.} \& can escape from the garden of meaning over the wall. Of these critical words, Wolcott Gibbs once wrote, ``$\ldots$ they are detached from the language \& inflated like little balloons.'' The young writer should learn to spot them -- words that at 1st glance seem freighted with delicious meaning but that soon burst in air, leaving nothing but a memory of bright sound.

The language is perpetually\footnote{\textbf{perpetual} [a] [usually before noun] \textbf{1.} continuing for a long period of time without interruption, \textsc{synonym}: \textbf{continuous}; \textbf{2.} frequently repeated, \textsc{synonym}: \textbf{continual}.} in flux\footnote{\textbf{flux} [n] \textbf{1.} [countable, uncountable] \textbf{flux (of something)} (\textit{specialist}) a flow; an act of flowing; in physics, \textbf{flux} can be the rate of flow of a liquid, a gas, energy or particles across a particular area; or the total electric or magnetic field passing through a surface; \textbf{2.} [uncountable] continuous movement \& change.}: it is a living stream, shifting, changing, receiving new strength from a thousand tributaries\footnote{\textbf{tributary} [n] a river or stream that flows into a larger river or a lake.}, losing old forms in the backwaters\footnote{\textbf{backwater} [n] \textbf{1.} a part of a river away from the main part, where the water only moves slowly; \textbf{2.} (\textit{often disapproving}) a place that is away from the places where most things happen, \& is therefore not affected by events, progress, new ideas, etc.} of time. To suggest that a young writer not swim in the main stream of this \fbox{turbulence} would be foolish indeed, \& such is not the intent of these cautionary remarks. The intent is to suggest that in choosing between the formal \& the informal, the regular \& the offbeat\footnote{\textbf{offbeat} [a] [usually before noun] (\textit{informal}) different from what most people expect, \textsc{synonym}: \textbf{unconventional}.}, the general \& the special, the orthodox\footnote{\textbf{orthodox} [a] \textbf{1.} (especially of beliefs or behavior) generally accepted or approved of; following generally accepted beliefs, \textsc{synonym}: \textbf{traditional}; \textbf{2.} following closely the traditional beliefs \& practices of a religion; \textbf{3.} (\textbf{Orthodox}) belonging to or connected with the Orthodox Church.} \& the heretical\footnote{\textbf{heretical} [a] \textbf{1.} (of a religious belief or opinion) against the principles of a particular religion; \textbf{2.} (of a belief or opinion) disagreeing strongly with what most people believe.}, the beginner err\footnote{\textbf{err} [v] [intransitive] to make a mistake; \textbf{err on the side of something} [idiom] to show too much of a good quality.} on the side of conservatism\footnote{\textbf{conservatism} [n] [uncountable] \textbf{1.} the tendency to resist great or sudden change; \textbf{2.} the belief that society should change as little as possible; \textbf{3.} (\textbf{Conservatism}) the beliefs of a political party that has traditional ideas about society \& that favors businesses that are privately owned \& that operate with little government control.}, on the side of established usage. No idiom is taboo\footnote{\textbf{taboo} [a] considered so offensive or embarrassing that people must not mention it; [n] \textbf{1.} \textbf{taboo (against\texttt{/}on something)} a cultural or religious custom that does not allow people to do, use or talk about a particular thing; \textbf{2.} \textbf{taboo (against\texttt{/}on something)} a general agreement not to do something or talk about something.}, no accent forbidden\footnote{\textbf{forbidden} [a] not allowed.}; there is simply a better chance of doing well if the writer holds a steady course, enters the stream of English quietly, \& does not thrash\footnote{\textbf{thrash} [v] \textbf{1.} [transitive] \textbf{thrash somebody\texttt{/}something} to hit a person or an animal many times with a stick, etc. as a punishment; \textbf{2.} [intransitive, transitive] to move or make something move in a way that is violent or show a loss of control; \textbf{3.} [transitive] \textbf{thrash somebody\texttt{/}something} (\textit{informal, especially British English}) to defeat somebody very easily in a game.} about.

``But,'' you may ask, ``what if it comes natural to me to experiment rather than conform\footnote{\textbf{conform} [v] \textbf{1.} [intransitive] to behave \& think in the same way as most other people in a group or society; \textbf{2.} [intransitive] to obey a rule or law, \textsc{synonym}: \textbf{comply}; \textbf{conform to something} [phrasal verb] to agree with or match something.}? What if I am a pioneer, or even a genius?'' Answer: then be one. But do not forget that what may seem like pioneering may be merely evasion\footnote{\textbf{evasion} [n] \textbf{1.} [uncountable] the act of not doing something, especially something that legally or morally you should do; \textbf{2.} [countable] a statement that somebody makes that avoids dealing with something or talking about something honestly \& directly; \textbf{3.} [uncountable] \textbf{evasion (of something)} the act of escaping or avoiding somebody\texttt{/}something.}, or laziness -- the disinclination\footnote{\textbf{disinclination} [n] [singular, uncountable] (\textit{formal}) a lack of desire to do something; a lack of enthusiasm for something.} to submit to discipline. Writing good standard English is no cinch\footnote{\textbf{cinch} [n] [singular] (\textit{formal}) \textbf{1.} something that is very easy, \textsc{synonym}: \textbf{doddle}; \textbf{2.} (\textit{especially North American English}) a thing that is certain to happen; a person who is certain to do something; [v] \textbf{1.} \textbf{cinch something} (\textit{especially North American English}) to fasten something tightly around the middle part of your body; to be fastened around the middle part of somebody's body; \textbf{2.} \textbf{cinch something} (\textit{North American English}) to fasten a girth around  a horse; \textbf{3.} \textbf{cinch something} (\textit{North American English, informal}) to make something certain.}, \& before you have managed it you will have encountered enough rough country to satisfy even the most adventurous\footnote{\textbf{adventurous} [a] \textbf{1.} (\textit{North American English also} \textbf{adventuresome}) (of a person) willing to take risks \& try new ideas; enjoying being in new, exciting situations; \textbf{2.} including new \& interesting things, methods \& ideas; \textbf{3.} full of new, exciting or dangerous experiences, \textsc{opposite}: \textbf{unadventurous}.} spirit.

Style takes its final shape more from attitudes of mind than from principles of composition, for, as an elderly practitioner once remarked, \fbox{``Writing is an act of faith, not a trick of grammar.''} This moral observation would have no place in a rule book were it not that style \textit{is} the writer, \& therefore what you are, rather than what you know, will at last determine your style. If you write, you must believe -- in the truth \& worth of the scrawl, in the ability of the reader to receive \& decode the message. No one can write decently\footnote{\textbf{decently} [adv] \textbf{1.} well enough; to a good enough standard or quality; \textbf{2.} honestly \& fairly; in a way that involves treating people with respect; \textbf{3.} in a way that is acceptable in a particular situation.} who is distrustful of the reader's intelligence, or whose attitude is patronizing\footnote{\textbf{patronizing} [a] (\textit{British English also} \textbf{patronising}) (\textit{disapproving}) showing that you think you are better or more intelligent than somebody else, \textsc{synonym}: \textbf{superior}.}.

Many references have been made in this book to ``the reader,'' who has been much in the news. It is now necessary to warn you that your concern for the reader must be pure: you must sympathize\footnote{\textbf{sympathize} [v] (\textit{British English also} \textbf{sympathise}) \textbf{1.} [intransitive] \textbf{sympathize (with somebody\texttt{/}something)} to feel sorry for somebody; to show that you understand \& feel sorry about somebody's problems; \textbf{2.} [intransitive] \textbf{sympathize with somebody\texttt{/}something} to support or approve of somebody\texttt{/}something.} with the reader's plight\footnote{\textbf{plight} [n] [singular] a difficult \& sad situation.} (most readers are in trouble about half the time) but never seek to know the reader's wants. Your whole duty as a writer is to please \& satisfy yourself, \& the true writer always plays to an audience of one. Start sniffing the air, or glancing at the Trend Machine, \& you are as good as dead, although you may make a nice living.

Full of belief, sustained \& elevated\footnote{\textbf{elevated} [a] [usually before noun] \textbf{1.} higher than normal; \textbf{2.} high in rank; \textbf{3.} higher than the area around; above the level of the ground; \textbf{4.} having a high moral or intellectual level.} by the power of purpose, armed with the rules of grammar, you are ready for exposure. At this point, you may well pattern yourself on the fully exposed cow of Robert Louis Stevenson's rhyme\footnote{\textbf{rhyme} [n] \textbf{1.} [uncountable] the use of words in a poem or song that have the same sound, especially at the ends of lines; \textbf{2.} [countable] a word that has the same sound or ends with the same sound as another word; \textbf{3.} [countable] a short poem in which the last word in the line has the same sound as the last word in another line, especially the next one; [v] [intransitive, transitive] (of 2 words of syllables) to have or end with the same sound; to put words that sound the same together, e.g. when writing poetry.}. This friendly \& commendable\footnote{\textbf{commendable} [a] (\textit{formal}) deserving praise \& approval.} animal, you may recall, was ``blown by all the winds that pass\texttt{/}\^ wet with all the showers.'' \& so must you as a young writer be. In our modern idiom, we would say that you must get wet all over. Mr. Stevenson, working in a plainer style, said it with felicity, \& suddenly 1 cow, out of so many, received the gift of immortality. Like the steadfast\footnote{\textbf{steadfast} [a] (\textit{literary, approving}) not changing in your attitudes or aims, \textsc{synonym}: \textbf{firm}.} writer, she is at home in the wind \& the rain; \&, thanks to 1 moment of felicity, she will live on \& on \& on.'' -- \cite[Chap. 5, Sect. 21, pp. 100--103]{Strunk_White2019}

\subsection{Afterword}
``Will Strunk \& E. B. White were unique collaborators\footnote{\textbf{collaborator} [n] \textbf{1.} a person who works with another person to create or produce something such as book; \textbf{2.} \textbf{collaborator (with somebody\texttt{/}something)} a person who helps the enemy in a war, when they have taken control of the person's country.}. Unlike Gilbert \& Sullivan, or Woodward \& Bernstein, they worked separately \& decades apart.

We have no way of knowing whether Prof. Strunk took particular notice of Elwyn Brooks White, a student of his at Cornell University in 19191. Neither teacher nor pupil could have realized that their names would be linked as they now are. Nor could they have imagined that 38 years after they met, White would take this little gem of a textbook that Strunk had written for his students, polish it, expand it, \& transform it into a classic.

E. B. White shared Strunk's sympathy for the reader. To Strunk's do's \& don'ts he added passages about the power of words \& the clear expression of thoughts \& feelings. To the nuts\footnote{\textbf{nut} [n] a small hard fruit with a very hard shell that grows on some trees.} \& bolts\footnote{\textbf{bolt} [n] \textbf{1.} a long, narrow piece of metal that you slide across the inside of a door or window in order to lock it; \textbf{2.} a piece of metal like a thick nail without a point which is used with a circle of metal ($=$ a nut) to fasten things together; \textbf{3.} \textbf{bolt of lightning} a sudden flash of lightning in the sky, appearing as a line; \textbf{4.} a short heavy arrow shot from a crossbow; \textbf{5.} a long piece of cloth wound in a roll around a piece of cardboard.} of grammar he added a rhetorical\footnote{\textbf{rhetorical} [a] \textbf{1.} connected with the art of rhetoric; \textbf{2.} (\textit{often disapproving}) (of a speech or piece of writing) intended to influence people, but not completely honest or sincere; \textbf{3.} (of a question) asked only to make a statement or to produce an effect rather than to get an answer.} dimension.

The editors of this edition have followed in White's footsteps, once again providing fresh examples \& modernizing usage where appropriate. \textit{The Elements of Style} is still a little book, small enough \& important enough to carry in your pocket, as I carry mine. It has helped me to write better. I believe it can do the same for you.'' -- \cite[Afterword by Charles Osgood, p. 104]{Strunk_White2019}

%---------------------------------------------------------------------- --------%

\part{Scientific\texttt{/}Mathematical Writings}

\chapter{Luc Tartar's Writing Styles}

%---------------------------------------------------------------------- --------%

\chapter{Terence Tao\texttt{/}\href{https://terrytao.wordpress.com/advice-on-writing-papers/}{On Writing}}
\begin{quotation}
	``There are three rules for writing the novel. Unfortunately, no one knows what they are.'' -- W. Somerset Maugham
\end{quotation}
``Everyone has to \href{https://terrytao.wordpress.com/advice-on-writing-papers/write-in-your-own-voice/}{develop their own writing style}, based on their own strengths and weaknesses, on the subject matter, on the target audience, and sometimes on the target medium. As such, it is virtually impossible to prescribe rigid rules for writing that encompass all conceivable situations and styles.

Nevertheless, I do have some general advice on these topics:
\begin{itemize}
	\item Writing a paper
	\begin{itemize}
		\item ``Use the introduction to \href{https://terrytao.wordpress.com/advice-on-writing-papers/use-the-introduction-to-%E2%80%9Csell%E2%80%9D-the-key-points-of-your-paper/}{``sell'' the key points of your paper}; the results should \href{https://terrytao.wordpress.com/advice-on-writing-papers/describe-the-results-accurately/}{be described accurately}. One should also invest some effort in both \href{https://terrytao.wordpress.com/advice-on-writing-papers/organise-the-paper/}{organizing} and \href{https://terrytao.wordpress.com/advice-on-writing-papers/motivate-the-paper/}{motivating} the paper, and in particular in \href{https://terrytao.wordpress.com/advice-on-writing-papers/use-good-notation/}{selecting good notation} and \href{https://terrytao.wordpress.com/advice-on-writing-papers/give-appropriate-amounts-of-detail/}{giving appropriate amounts of detail}. But \href{https://terrytao.wordpress.com/advice-on-writing-papers/dont-overoptimise/}{one should not over-optimize} the paper.
		\item It also assists readability if you factor the paper into smaller pieces, e.g., by \href{https://terrytao.wordpress.com/advice-on-writing-papers/create-lemmas/}{making plenty of lemmas}.
		\item To reduce the time needed to write and organize a paper, I recommend \href{https://terrytao.wordpress.com/advice-on-writing-papers/write-a-rapid-prototype-first/}{writing a rapid prototype 1st}.
		\item For 1st time authors especially, it is important to try to \href{https://terrytao.wordpress.com/advice-on-writing-papers/write-professionally/}{write professionally}, and in \href{https://terrytao.wordpress.com/advice-on-writing-papers/write-in-your-own-voice/}{one's own voice}. One should \href{https://terrytao.wordpress.com/advice-on-writing-papers/take-advantage-of-the-english-language/}{take advantage of the English language}, and not just rely purely on mathematical symbols.
		\item The \href{https://terrytao.wordpress.com/advice-on-writing-papers/maximising-the-results-to-effort-ratio/}{ratio between results and effort in one's paper should be at a local maximum}.
	\end{itemize}
	\item Submitting a paper
	\begin{itemize}
		\item \href{https://terrytao.wordpress.com/advice-on-writing-papers/proofread-and-double-check-your-paper-before-submission/}{Proofread and double-check your article before submission}; you should be \href{https://terrytao.wordpress.com/advice-on-writing-papers/submit-a-final-draft-not-a-first-draft/}{submitting a final draft, not a 1st draft}
		\item \href{https://terrytao.wordpress.com/advice-on-writing-papers/submit-to-an-appropriate-journal/}{Subset to an appropriate journal}
	\end{itemize}
\end{itemize}
I should point out, of course, that my own writing style is not perfect, and I myself don't always adhere to the above rules, often to my own detriment. If some of these suggestions seem too unsuitable for your particular paper, use common sense.

Dual to the art of \textit{writing} a paper well, is the art of \textit{reading} a paper well.\footnote{NQBH: In mathematical notation: \begin{align*}
		\mbox{(Art of writing a paper well)} = \mbox{(Art of reading a paper well)}^\star,\ \mbox{(Art of reading a paper well)} = \mbox{(Art of writing a paper well)}^\star.
\end{align*}}\footnote{NQBH: In linguistic, \textit{reading} and \textit{writing skills} usually come together, so do \textit{listening} and \textit{speaking skills}. I.e., if one wants to master 1 of these 4 skills, then that person has to master its companion parallelly:
\begin{align*}
	(\mbox{reading}\land\mbox{writing})\lor(\mbox{speaking}\land\mbox{listening}).
\end{align*}} Here is some commentary of mine on this topic:
\begin{itemize}
	\item \href{https://terrytao.wordpress.com/advice-on-writing-papers/on-compilation-errors-in-mathematical-reading-and-how-to-resolve-them/}{On ``compilation errors'' in mathematical reading, and how to resolve them}.
	\item \href{https://terrytao.wordpress.com/advice-on-writing-papers/implicit-notational-conventions/}{On the use of implicit mathematical notational conventions to provide contextual clues when reading}.
	\item \href{https://terrytao.wordpress.com/advice-on-writing-papers/on-the-strength-of-theorems/}{On key ``jumps in difficulty'' in a mathematical argument, and how finding and understanding them is often key to understanding the argument as a whole}.
	\item \href{https://terrytao.wordpress.com/advice-on-writing-papers/on-local-and-global-errors-in-mathematical-papers-and-how-to-detect-them/}{On ``local'' and ``global'' errors in mathematical papers, and how to detect them}.
\end{itemize}
Some further advice on mathematical exposition: [$\ldots$]''

\section{Terence Tao\texttt{/}\href{https://terrytao.wordpress.com/advice-on-writing-papers/describe-the-results-accurately/}{On Writing\texttt{/}Describe the Results Accurately}}
\begin{quotation}
	``10,000 fools proclaim themselves into obscurity, while 1 wise man forgets himself into immortality.'' -- Martin Luther King Jr.
\end{quotation}
``\fbox{A paper should neither understate nor overstate its main results.}

If the main result is very surprising or a substantial breakthrough compared with the previous literature, these facts should be noted (and justified in detail, e.g., by explicit comparison with prior results, examples, and conjectures).

Conversely, if there are unsatisfactory aspects to the result (e.g., hypotheses too strong, or conclusions a little weaker than expected) these should also be stated honestly and openly, e.g., ``We do not know if hypothesis H is actually necessary''. Similarly, it is worth noting down any interesting open questions remaining after your result.

If you are using a famous unsolved conjecture to motivate your own work, one should give a candid evaluation of the extent to which your work truly represents progress towards that conjecture, so as to avoid the impression of ``false advertising'' or ``name-dropping''.

If for some reason you need to assert a non-trivial statement without proof or citation, it should be made clear that you are doing so (e.g., ``It can be shown that $\ldots$'' or ``Although we will not need or prove this fact here $\ldots$''), so that the reader does not then hunt through the rest of your paper for the non-existent justification of that statement.

\fbox{Titles of sections should be descriptive} (e.g., ``proof of the decomposition lemma'' or ``An orthogonality argument''), as opposed to uninformative (e.g., ``Step 2'' or ``Some technicalities'').

\section{Terence Tao\texttt{/}\href{https://terrytao.wordpress.com/advice-on-writing-papers/give-appropriate-amounts-of-detail/}{On Writing\texttt{/}Give Appropriate Amounts of Detail}}
\begin{quotation}
	``In presenting a mathematical argument the great thing is to give the educated reader the chance to catch on at once to the momentary point and take details for granted: his successive mouthfuls should be such as can be swallowed at sight; in case of accidents, or in case he wishes for once to check in detail, he should have only a clearly circumscribed little problem to solve (e.g., to check an identity: 2 trivialities omitted can add up to an impasse). The unpracticed writer, even after the dawn of a conscience, gives him no such chance; before he can spot the point he has to tease his way through a maze of symbols of which not the tiniest suffix can be skipped.'' -- John Littlewood, \textit{``A Mathematician's Miscellany''}
\end{quotation}
A paper should dwell at length (using \href{https://terrytao.wordpress.com/advice-on-writing-papers/take-advantage-of-the-english-language/}{plenty of English}) on the most important, innovative, and crucial components of the paper, and be brief on the routine, expected, and standard components of the paper.

In particular, \fbox{a paper should identity which of its components are the most interesting}. Note that this means interesting to \textit{experts in the field}, and not just interesting to \textit{yourself}; e.g., if you have just learnt how to prove a standard lemma which is well known to the experts and already in the literature, this does not mean that you should provide the standard proof of this standard lemma, unless this serves some greater purpose in the paper (e.g., by motivating a less standard lemma).

Conversely, some computations, definitions, or notational conventions which you are very familiar with, but are not widely known in the field, should be expounded on in detail, even if these details are ``obvious'' to you due to your extensive work in this area. Even a brief sentence of explanation is much better than none at all.

For a similar reason, if you are using a relatively obscure lemma from, say, 1 of your own papers, you should not assume that every reader of your current article is intimately familiar with your previous paper. In such cases it is worth stating the lemma in full, with a precise citation (as opposed to casually using phrases e.g., ``by a lemma in [my previous 100-page paper], we have $\ldots$''). When the lemma is particularly crucial, it is sometimes also worth spending a paragraph to sketch out a proof, or to otherwise remark on the significance of this lemma and its connections to other, more well known results.''

\section{Terence Tao\texttt{/}\href{https://terrytao.wordpress.com/advice-on-writing-papers/take-advantage-of-the-english-language/}{On Writing\texttt{/}Take Advantage of the English Language}}
\begin{quotation}
	``Use soft words and hard arguments.'' -- Proverbial
\end{quotation}
``\href{https://terrytao.wordpress.com/advice-on-writing-papers/use-good-notation/}{Mathematical notation} is a wonderfully useful tool, and it can be exciting to learn for the first time the meaning of mysterious and arcane symbols e.g., $\forall,\exists,\emptyset,\Rightarrow$, etc. However, just because you \textit{can} write statements in purely mathematical notation doesn't mean that you necessarily \textit{should}. In many cases, it is in fact far more informative and readable to use liberal amounts of plain English; if used correctly and thoughtfully, the English language can communicate to the reader on many more levels than a mathematical expression, \fbox{without sacrificing any precision or rigor}. In particular, by subtly modulating the emphasis of one's text, one can convey valuable contextual cues as to how a statement interacts with the rest of one's argument.

An example should serve to illustrate this point. Suppose for instance that $P$ and $Q$ are properties that can apply to mathematical objects $x$ and $y$. The mathematical statements $P(x)\land Q(y)$ which asserts that $x$ satisfies $P$ and $y$ satisfies $Q$, is a well-formed and precise mathematical statement. But there are many possible ways one could express that mathematical statement in English, e.g.,: $\ldots$'' \texttt{[Skip 27 items]}

``From the viewpoint of formal mathematical logic, each of these English statement is logically equivalent to the mathematical sentence $P(x)\land Q(y)$. However, each of the above English statements also provides additional useful and informative cues for the reader regarding the relative importance, non-triviality, and causal relationship of the component statements $P(x)$ and $Q(y)$, or of the component symbols $P,x,Q$, and $y$. E.g., in some of these sentences $P(x)$ and $Q(y)$ are given equal importance (being complementary or somehow in opposition to each other), whereas in others $P(x)$ is only an auxiliary statement whose only purpose is to derive $Q(y)$ (or vice versa), and in yet others, $P(x)$ and $Q(y)$ are deemed to be analogous, even if one is not formally deducible from the other. In some sentences, it is the objects $x$ and $y$ which are indicated to be the primary actors; in other sentences, it is the properties $P$ and $Q$; and in yet other sentences, it is the combined statements $P(x)$ and $Q(y)$ which are the most central.

Thus we see that English sentences can be considerably more expressive than their formal mathematical counterparts, while still retaining the precision and rigor that mathematical exposition demands. By using such humble English words as ``also'', ``but'', ``since'', etc., a sentence conveys not only its semantic content, but also how it is going to fit in with the rest of one's argument (or in the wider theory of the object), giving the reader more insight as to the overall structure of that argument. \fbox{The paper may become slightly longer because of this, but this is a small price to pay for readability} (which is \textit{not} the same as brevity!).

On the other hand, one should \href{https://terrytao.wordpress.com/advice-on-writing-papers/dont-overoptimise/}{not try to excessively ``improve''} the paper by using overly fancy or obscure words (from English or any other language), especially since such words can be mistaken for some sort of technical mathematical terminology. In many cases, one can replace complicated words by plainer equivalents, thus increasing the readability of one’s text without compromising the message. The primary purpose of mathematical writing is to \textit{communicate} and \textit{inform}, not to \textit{impress}.

Finally, there is 1 situation in which it does make sense to use the terse language of mathematical notation rather than a more leisurely English equivalent, and that is when you are performing a tedious and standard formal computation. In those cases, the reader should already know in general terms what is going to happen (especially if you flag the computation as being standard beforehand), and will only be distracted by superfluous explanation or digression. (See also ``\href{https://terrytao.wordpress.com/advice-on-writing-papers/give-appropriate-amounts-of-detail/}{give appropriate amounts of detail}''.)

Naturellement, la discussion ci-dessus s'applique également à d'autres langues, telles que la langue française.''\footnote{Of course, the above discussion also applies to other languages, such as the French language.}

\section{Terence Tao\texttt{/}\href{https://terrytao.wordpress.com/advice-on-writing-papers/use-good-notation/}{On Writing\texttt{/}Use Good Notation}}
\begin{quotation}
	``By relieving the brain of all unnecessary work, a good notation sets it free to concentrate on more advanced problems, and, in effect, increases the mental power of the race.'' -- Alfred North Whitehead, ``An Introduction to Mathematics''
\end{quotation}
``\fbox{Good notation can make the difference between a readable paper and an unreadable one.}

Ideally, notation should emphasize the most important parameters and features of a mathematical expression or statement, while downplaying the routine or uninteresting parameters and features. For instance, if one does not care much about the exact values of constants in estimates, then notation which conceals these constants (e.g., $\ll$, $\lesssim$, or $O(\cdot)$) are useful; conversely, these notations should be avoided if the precise values of these constants are of importance to the paper.

\fbox{Notation which is used globally should be defined in a notation section near the front of the paper, or in the introduction;} notation which is only used locally (e.g., within a single section, or within a proof of a single lemma) should be defined close to where it is used (possibly with a reminder that this notation is not used elsewhere in the paper); this is helpful when there are many sections, each with their own extensive notation.

Note that notation or statements which are introduced within a proof of a lemma are already understood to be localized to that lemma; it is bad form to then recall that notation or statement outside of that lemma, except perhaps as a remark or as motivation. In some cases it is worthwhile to define the notation once near the start of the paper, and then recall it whenever necessary.

One should strive to make one's choices of notation compatible and consistent with notation already in the literature, so that the readers who are already familiar with prior notation will adapt easily to your paper and will not be confused.

\fbox{Try to avoid notation which is overly ``cute'' or ``clever''.} This can be distracting or appear \href{https://terrytao.wordpress.com/career-advice/be-professional-in-your-work/}{unprofessional}. In particular, the notation should not be cleverer than the actual substance of the paper.

One should \textbf{definitely} avoid naming new terms after yourself (or after your family members, your pets, etc.), for the obvious reasons. If other authors name the concepts you introduce after yourself, and that appellation becomes common usage, then you may use that term as well, but in all other cases it gives the rather \fbox{blatant impression of vanity or narcissism}.

There is an issue of where to strike the balance between too little notation and too much notation. A good rule of thumb is that any expression or concept which is used 3 or more times will probably benefit from introducing some notation to capture that expression or concept; conversely, an expression which is only used once probably does not need its own special notation. (An exception would be for particularly crucial theorems or propositions in the paper; here it might be worthwhile to invest in some notation in order to make the statement of those theorems clean and readable. Conversely, if an expression only appears in multiple locations of the paper because of coincidences of no significance, then it may be better to avoid introducing notation that gives the false impression of a connection between these appearances.)

If one needs to name a certain property or class of objects, one should generally use very bland names (e.g., ``good'', ``bad'', ``Type I'', ``Type II'', etc.) for peripheral or technical terms; colorful terms should be used sparingly, and only for those concepts that are quite central to the paper, lest they distract from the main points of that paper. (This is analogous to how, in film and literature, the main characters generally tend to have more memorable names than the secondary ones.)

Sometimes one is unsure what notation to use for a particular concept, because of potential conflicts with other notation in other (as yet unwritten) parts of a paper. One solution here is to introduce a \TeX\ \href{http://en.wikipedia.org/wiki/Macro}{macro} for that notation, and force yourself to use that macro exclusively whenever that notation is used. (E.g., if you have a group which you are tentatively naming $G$, you could define a macro \verb|\grp| that is set to G, and use \verb|\grp| instead of G throughout the paper.) That way, if you find a notational conflict later on (e.g., if you discover that you really need G to denote a graph instead), then you only need to change \textit{1 line} in your \TeX\ file -- the line that defines the macro -- to resolve the notational conflict, rather than to do a tedious (and error-prone) search-and-replace.

For any rigorous component of the paper, the notation should be precise and unambiguous (and for non-rigorous components, ambiguous notation should be pointed out with ``scare quotes'' or other cautionary phrases such as ``roughly speaking'' or ``essentially''). A certain amount of abuse of notation is permitted, though, as long as this is properly pointed out.'' \texttt{[Skip the common example of \textit{division}, i.e., $a/bc$ means either $(a/b)c$  or $a/(bc)$; or use $\frac{a}{b}c$ and $\frac{a}{bc}$ instead].}

``It is also worthwhile to quietly reinforce one's notational conventions when given the opportunity. E.g., suppose in one's argument one has a vector space, which one has decided to call $V$. When referring back to this object, one could say ``the vector space'', or ``V'', but if the reader does not remember what vector space is being discussed, or what $V$ is, the reader will have to take a minute or so to flip back and figure this out. But if instead you refer to this object consistently as ``the vector space $V$'', then the notational convention is reinforced, and the reader can continue reading without breaking rhythm. (One can also modulate the choice of terminology used here to emphasize different aspects of the object being referred to. If e.g., it is the additive structure of $V$ which is currently relevant, you can instead say ``the additive group $V$''; if, later, it is the topological structure which is the most important, one can say ``the topological vector space $V$'', and so forth. This allows one to subtly draw attention to the most important features of the object under consideration, without distracting the reader from the main body of the argument.)''

See also \href{https://mathoverflow.net/questions/366070/what-are-the-benefits-of-writing-vector-inner-products-as-langle-u-v-rangle/366118#366118}{Terence Tao's answer to MathOverflow question: What are the benefits of writing vector inner products as $\langle{\bf u},{\bf v}\rangle$ as opposed to ${\bf u}^\top{\bf v}$?}

\section{Terence Tao\texttt{/}\href{https://terrytao.wordpress.com/advice-on-writing-papers/write-in-your-own-voice/}{On Writing\texttt{/}Write in Your Own Voice}}

\begin{quotation}
	``While one should always study the method of a great artist, one should never imitate his manner. The manner of an artist is essentially individual, the method of an artist is absolutely universal. The first is personality, which no one should copy; the second is perfection, which all should aim at.'' -- Oscar Wilde, \textit{A Critic in Pall Mall}, p. 195
\end{quotation}
``When, as a graduate student, one is starting out one's research in a mathematical subject, one usually begins by reading the papers of the current and past leaders of the field. Initially, one's understanding of the subject is fairly limited, and so it is natural to view these papers as being authoritative, especially if their authors are well-known.

Eventually, though, one requires a fair fraction of the insights and understanding conveyed by the existing literature, and is able to apply it to produce a new result or observation that goes beyond that literature (or, at least, makes explicit what was only implicitly buried in previous papers). When the ramifications and extensions of these new advances have been explored to their natural extent, it then becomes time to write up these results as a research paper.

Of course, as your work is almost certainly based in part on the previous literature, one should cite that literature whenever appropriate, and compare and contrast your own work with that literature in an \href{https://terrytao.wordpress.com/advice-on-writing-papers/describe-the-results-accurately/}{accurate}, \href{https://terrytao.wordpress.com/advice-on-writing-papers/write-professionally/}{professional}, and informative manner. Also, one should try to \href{https://terrytao.wordpress.com/advice-on-writing-papers/use-good-notation/}{maintain some level of notational consistency} with the previous literature, such as using the same fundamental definitions and to use similar notation, so that expert readers who are already familiar with that literature can quickly get up to speed on your work. And if 1 of the arguments in your work is standard in the literature, it certainly makes sense to structure the argument in a standard fashion if possible, again to assist the experts reading your paper.

\textbf{However}, one should \textbf{not} go so far as to copy entire paragraphs or more of text from a prior paper, except when used sa a direct quotation to illustrate some historical point. First of all, if one does not properly attribute that text (e.g., ``As Bourbaki [17, p. 146] writes,'', or, for that matter, the Oscar Wilde quote above), then one runs the risk of committing \href{http://en.wikipedia.org/wiki/Plagiarism}{plagiarism}. But even if the text is properly attributed, copying the text verbatim, without updating it to reflect more recent developments (including that in the paper being written) and to add your own simplifications and insights, is a redundant waste of space and a lost opportunity to advance the subject. If one is tempted to copy a significant portion of text from a prior reference without adding anything significantly new, one should instead simply cite the previous reference appropriately, e.g., ``See [27, Section 4] for further discussion.'' or ``A proof can be found in [9, Lemma 2.4].'' (cf. ``\href{https://terrytao.wordpress.com/advice-on-writing-papers/give-appropriate-amounts-of-detail/}{Give appropriate amounts of details}'').

Of course, there \textit{are} reasons to duplicate to some extent some discussion or argument that was present in a previous paper:
\begin{itemize}
	\item As mentioned earlier, one may wish to make some historical point, e.g., to track the development of a mathematical idea over time.
	\item If the paper is obscure and not widely available, reproducing a key argument from that paper may serve as a convenience to the reader.
	\item Also, if the \textit{form} of that argument can be used to \href{https://terrytao.wordpress.com/advice-on-writing-papers/motivate-the-paper/}{motivate} other arguments in your paper, then it can be worth putting in that argument so that it can be referred to later in the paper.
	\item The precise result needed for your paper may differ slightly from what is already established in the literature, and so one needs to either write out a modified version of the proof, or else point to the original proof but indicate what modifications need to be made. (The latter is suitable if the changes are particularly minor in nature.)
	\item The existing paper may have an argument which can be updated, simplified, modernized, or otherwise improved thanks to more recent advances or insights in the area (including your own). It can then be a service to the field to place an updated version of the argument in the literature (with full citations to the paper containing the original argument, of course).
\end{itemize}
However, when one is not simply quoting the prior text for historical or archival purposes, it is best to \textit{paraphrase} and \textit{interpret} the previous text rather than to copy that text verbatim. This is for a number of reasons:
\begin{itemize}
	\item One wants to avoid conveying any impression to readers, referees, or editors of plagiarism, padding, or intellectual laziness in one's papers. (Note that the latter is a danger even if one is copying from one's own work, rather than that of others.)
	\item The prior work may be dated in view of more recent developments and insights, as mentioned above.
	\item If you are copying or adapted a piece of text from another author that you do not fully understand yourself, then it may end up being inappropriate or incongruous for your intended purpose, and may convey the impression of superficiality or being ill-informed. If the text becomes inaccurate due to this adaptation, then this can also cause some embarrassment and annoyance for the original author of that text.
	\item Excessive use of quotation from famous mathematicians to make one's own work look more impressive is the mathematical equivalent of name-dropping, and should be avoided. Appeal to authority should not be the primary basis for motivating a paper; a handful of citations to demonstrate the depth of interest in the problem being studied is usually sufficient.
	\item \textit{But most importantly of all, for one's further mathematical development and career, one needs to develop one's own consistent mathematical ``voice'' and style, and to avoid the impression of simply imitating the voices of other authors}. There is no need in this subject for the mathematical equivalent of a parrot, and a text which is a mix of the author's voice and the voice of others can read very strangely.
\end{itemize}
Of course, if one is paraphrasing a previous work, one should cite that work appropriately (e.g., ``The proof here is loosely based on that in [5].'' or ``This discussion is inspired by a related discussion in [10].'').

In some cases, the imitation of a previous author's style and text is intended as a sign of respect or flattery for that author. \textbf{This is misguided}; an author will in fact often find such mimicry to actually be somewhat offensive. If one wants to truly respect a mathematician, then understand that mathematician's methods, results, and exposition, and improve, update, adapt, and advance all 3. Even the greatest mathematician's contributions should advance with the field, rather than being worshiped and preserved in some supposed state of perfection; the latter is mostly suitable only for historical purposes.

Another possible reason for copying the style of a more senior mathematician is that one does not yet have the self-confidence to write in one's own style and voice. While this is justifiable to some extent when one is just starting one's career, it becomes less excusable as one continues one's research. If one is hesitant to state things in one's own fashion, it is perfectly acceptable to couch such text with the appropriate caveats (e.g., ``to the author's knowledge, this observation is new'' or ``While Lemma 2.5 is usually phrased in a topological fashion, we found the following, more geometric, formulation to be more convenient for our applications''). And if one does not feel confident enough in one's understanding of a subject to explain it in any other way than copying from a previous paper, then this should be taken as a sign that one still needs to \href{https://terrytao.wordpress.com/career-advice/learn-and-relearn-your-field/}{internalize the subject futher}.

When writing a paper with 1 or more coauthors, there will inevitably be distinctions in style,\footnote{NQBH: a reasonable justification for loneliness and\texttt{/}in solo (academic) writing.} and so initially different sections may have sharply different tones due to their being largely written by different subsets of coauthors; but I usually find that after a few rounds of editing, the voices are harmonized into a style which is clearly derived from, but distinct from, each of the individual styles. Ideally, one should understand and respect the underlying stylistic decisions of one's coauthors, but at the same time be willing to take the initiative and find ways to formulate the text and arrangement to smoothly reconcile the coauthor's preferences with one's own; if all goes well, this can lead to a level of exposition and presentation that is superior to what each of the individual authors could separately achieve. (Of course, if you are to perform major edits on a coauthor's contribution, some consultation with that coauthor is presumably desirable). This process can be quite educational; my own writing style has definitely been influenced in a positive fashion by those of my coauthors.

Developing one's own style is, by definition, a very personal process; while external advice or role models can certainly be of some influence, they are of limited utility after a certain point. But finding an individual style which is comfortable and effective for both you and your readers is an important mark of one's \textit{mathematical maturity}, and is a goal that is definitely worth pursuing.''
%------------------------------------------------------------------------------%

\section*{Quick notes}
``when possible and\texttt{/}or necessary'' -- \cite[p. 47]{Rebollo_Lewandowski2014}

%---------------------------------------------------------------------- --------%

\begin{thebibliography}{99}
	\bibitem[TerryTao]{TerryTao} \href{https://terrytao.wordpress.com}{Terence Tao's blog}.
	\begin{itemize}
		\item Terence Tao. \href{https://terrytao.wordpress.com/advice-on-writing-papers/}{\textit{On writing}} (or in full: \textit{Advice on writing papers}).
		\begin{itemize}
			\item Terence Tao. \href{https://terrytao.wordpress.com/advice-on-writing-papers/describe-the-results-accurately/}{\textit{On writing}\texttt{/}\textit{Describe the results accurately}}.
			\item Terence Tao. \href{https://terrytao.wordpress.com/advice-on-writing-papers/give-appropriate-amounts-of-detail/}{\textit{On writing}\texttt{/}\textit{Give appropriate amounts of detail}}.
			\item Terence Tao. \href{https://terrytao.wordpress.com/advice-on-writing-papers/use-good-notation/}{\textit{On writing}\texttt{/}\textit{Use good notation}}.
			\item Terence Tao. \href{https://terrytao.wordpress.com/advice-on-writing-papers/write-in-your-own-voice/}{\textit{On writing}\texttt{/}\textit{Write in your own voice}}.
		\end{itemize}
	\end{itemize}
\end{thebibliography}

%---------------------------------------------------------------------- --------%

\printbibliography[heading=bibintoc]
	
\end{document}