\documentclass{article}
\usepackage[backend=biber,natbib=true,style=authoryear]{biblatex}
\addbibresource{/home/hong/1_NQBH/reference/bib.bib}
\usepackage[utf8]{vietnam}
\usepackage{tocloft}
\renewcommand{\cftsecleader}{\cftdotfill{\cftdotsep}}
\usepackage[colorlinks=true,linkcolor=blue,urlcolor=red,citecolor=magenta]{hyperref}
\usepackage{amsmath,amssymb,amsthm,mathtools,float,graphicx}
\allowdisplaybreaks
\numberwithin{equation}{section}
\newtheorem{assumption}{Assumption}[section]
\newtheorem{lemma}{Lemma}[section]
\newtheorem{corollary}{Corollary}[section]
\newtheorem{definition}{Definition}[section]
\newtheorem{proposition}{Proposition}[section]
\newtheorem{theorem}{Theorem}[section]
\newtheorem{notation}{Notation}[section]
\newtheorem{remark}{Remark}[section]
\newtheorem{example}{Example}[section]
\newtheorem{ques}{Question}[section]
\newtheorem{problem}{Problem}[section]
\newtheorem{conjecture}{Conjecture}[section]
\usepackage[left=0.5in,right=0.5in,top=1.5cm,bottom=1.5cm]{geometry}
\usepackage{fancyhdr}
\pagestyle{fancy}
\fancyhf{}
\lhead{\small \textsc{Sect.} ~\thesection}
\rhead{\small \nouppercase{\leftmark}}
\renewcommand{\sectionmark}[1]{\markboth{#1}{}}
\cfoot{\thepage}
\def\labelitemii{$\circ$}

\title{A Personal Journey to Writing}
\author{Nguyễn Quản Bá Hồng}
\date{\today}

\begin{document}
\maketitle
\begin{abstract}
	My personal journey to ``The Garden of Words''\footnote{\href{https://www.imdb.com/title/tt2591814/}{IMDb\texttt{/}The Garden of Words (2013)}, original title: Koto no ha no niwa.} -- the world of writings.
\end{abstract}
\tableofcontents

%------------------------------------------------------------------------------%

\section{Literary Writings}

\section{Scientific\texttt{/}Mathematical Writings}

\subsection{Luc Tartar's Writing Styles}

\subsection{Terence Tao\texttt{/}\href{https://terrytao.wordpress.com/advice-on-writing-papers/}{On Writing}}
\begin{quotation}
	``There are three rules for writing the novel. Unfortunately, no one knows what they are.'' -- W. Somerset Maugham
\end{quotation}
``Everyone has to \href{https://terrytao.wordpress.com/advice-on-writing-papers/write-in-your-own-voice/}{develop their own writing style}, based on their own strengths and weaknesses, on the subject matter, on the target audience, and sometimes on the target medium. As such, it is virtually impossible to prescribe rigid rules for writing that encompass all conceivable situations and styles.

Nevertheless, I do have some general advice on these topics:
\begin{itemize}
	\item Writing a paper
	\begin{itemize}
		\item ``Use the introduction to \href{https://terrytao.wordpress.com/advice-on-writing-papers/use-the-introduction-to-%E2%80%9Csell%E2%80%9D-the-key-points-of-your-paper/}{``sell'' the key points of your paper}; the results should \href{https://terrytao.wordpress.com/advice-on-writing-papers/describe-the-results-accurately/}{be described accurately}. One should also invest some effort in both \href{https://terrytao.wordpress.com/advice-on-writing-papers/organise-the-paper/}{organizing} and \href{https://terrytao.wordpress.com/advice-on-writing-papers/motivate-the-paper/}{motivating} the paper, and in particular in \href{https://terrytao.wordpress.com/advice-on-writing-papers/use-good-notation/}{selecting good notation} and \href{https://terrytao.wordpress.com/advice-on-writing-papers/give-appropriate-amounts-of-detail/}{giving appropriate amounts of detail}. But \href{https://terrytao.wordpress.com/advice-on-writing-papers/dont-overoptimise/}{one should not over-optimize} the paper.
		\item It also assists readability if you factor the paper into smaller pieces, e.g. by \href{v}{making plenty of lemmas}.
		\item To reduce the time needed to write and organize a paper, I recommend \href{https://terrytao.wordpress.com/advice-on-writing-papers/write-a-rapid-prototype-first/}{writing a rapid prototype 1st}.
		\item For 1st time authors especially, it is important to try to \href{https://terrytao.wordpress.com/advice-on-writing-papers/write-professionally/}{write professionally}, and in \href{https://terrytao.wordpress.com/advice-on-writing-papers/write-in-your-own-voice/}{one's own voice}. One should \href{https://terrytao.wordpress.com/advice-on-writing-papers/take-advantage-of-the-english-language/}{take advantage of the English language}, and not just rely purely on mathematical symbols.
		\item The \href{https://terrytao.wordpress.com/advice-on-writing-papers/maximising-the-results-to-effort-ratio/}{ratio between results and effort in one's paper should be at a local maximum}.
	\end{itemize}
	\item Submitting a paper
	\begin{itemize}
		\item \href{https://terrytao.wordpress.com/advice-on-writing-papers/proofread-and-double-check-your-paper-before-submission/}{Proofread and double-check your article before submission}; you should be \href{https://terrytao.wordpress.com/advice-on-writing-papers/submit-a-final-draft-not-a-first-draft/}{submitting a final draft, not a 1st draft}
		\item \href{https://terrytao.wordpress.com/advice-on-writing-papers/submit-to-an-appropriate-journal/}{Subset to an appropriate journal}
	\end{itemize}
\end{itemize}
I should point out, of course, that my own writing style is not perfect, and I myself don't always adhere to the above rules, often to my own detriment. If some of these suggestions seem too unsuitable for your particular paper, use common sense.

Dual to the art of \textit{writing} a paper well, is the art of \textit{reading} a paper well.\footnote{NQBH: In mathematical notation: \begin{align*}
		\mbox{(Art of writing a paper well)} = \mbox{(Art of reading a paper well)}^\star,\ \mbox{(Art of reading a paper well)} = \mbox{(Art of writing a paper well)}^\star.
\end{align*}}\footnote{NQBH: In linguistic, \textit{reading} and \textit{writing skills} usually come together, so do \textit{listening} and \textit{speaking skills}. I.e., if one wants to master 1 of these 4 skills, then that person has to master the other one parallelly of its couple!
\begin{align*}
	(\mbox{reading}\land\mbox{writing})\lor(\mbox{speaking}\land\mbox{listening}).
\end{align*}} Here is some commentary of mine on this topic:
\begin{itemize}
	\item \href{https://terrytao.wordpress.com/advice-on-writing-papers/on-compilation-errors-in-mathematical-reading-and-how-to-resolve-them/}{On ``compilation errors'' in mathematical reading, and how to resolve them}.
	\item \href{https://terrytao.wordpress.com/advice-on-writing-papers/implicit-notational-conventions/}{On the use of implicit mathematical notational conventions to provide contextual clues when reading}.
	\item \href{https://terrytao.wordpress.com/advice-on-writing-papers/on-the-strength-of-theorems/}{On key ``jumps in difficulty'' in a mathematical argument, and how finding and understanding them is often key to understanding the argument as a whole}.
	\item \href{https://terrytao.wordpress.com/advice-on-writing-papers/on-local-and-global-errors-in-mathematical-papers-and-how-to-detect-them/}{On ``local'' and ``global'' errors in mathematical papers, and how to detect them}.
\end{itemize}
Some further advice on mathematical exposition: [$\ldots$]''

\subsubsection{Terence Tao\texttt{/}\href{https://terrytao.wordpress.com/advice-on-writing-papers/describe-the-results-accurately/}{On Writing\texttt{/}Describe the Results Accurately}}
\begin{quotation}
	``10,000 fools proclaim themselves into obscurity, while 1 wise man forgets himself into immortality.'' -- Martin Luther King Jr.
\end{quotation}
``\fbox{A paper should neither understate nor overstate its main results.}

If the main result is very surprising or a substantial breakthrough compared with the previous literature, these facts should be noted (and justified in detail, e.g., by explicit comparison with prior results, examples, and conjectures).

Conversely, if there are unsatisfactory aspects to the result (e.g. hypotheses too strong, or conclusions a little weaker than expected) these should also be stated honestly and openly, e.g. ``We do not know if hypothesis H is actually necessary''. Similarly, it is worth noting down any interesting open questions remaining after your result.

If you are using a famous unsolved conjecture to motivate your own work, one should give a candid evaluation of the extent to which your work truly represents progress towards that conjecture, so as to avoid the impression of ``false advertising'' or ``name-dropping''.

If for some reason you need to assert a non-trivial statement without proof or citation, it should be made clear that you are doing so (e.g., ``It can be shown that $\ldots$'' or ``Although we will not need or prove this fact here $\ldots$''), so that the reader does not then hunt through the rest of your paper for the non-existent justification of that statement.

\fbox{Titles of sections should be descriptive} (e.g., ``proof of the decomposition lemma'' or ``An orthogonality argument''), as opposed to uninformative (e.g., ``Step 2'' or ``Some technicalities'').

\subsubsection{Terence Tao\texttt{/}\href{https://terrytao.wordpress.com/advice-on-writing-papers/give-appropriate-amounts-of-detail/}{On Writing\texttt{/}Give Appropriate Amounts of Detail}}
\begin{quotation}
	``In presenting a mathematical argument the great thing is to give the educated reader the chance to catch on at once to the momentary point and take details for granted: his successive mouthfuls should be such as can be swallowed at sight; in case of accidents, or in case he wishes for once to check in detail, he should have only a clearly circumscribed little problem to solve (e.g. to check an identity: 2 trivialities omitted can add up to an impasse). The unpracticed writer, even after the dawn of a conscience, gives him no such chance; before he can spot the point he has to tease his way through a maze of symbols of which not the tiniest suffix can be skipped.'' -- John Littlewood, \textit{``A Mathematician's Miscellany''}
\end{quotation}
A paper should dwell at length (using \href{https://terrytao.wordpress.com/advice-on-writing-papers/take-advantage-of-the-english-language/}{plenty of English}) on the most important, innovative, and crucial components of the paper, and be brief on the routine, expected, and standard components of the paper.

In particular, \fbox{a paper should identity which of its components are the most interesting}. Note that this means interesting to \textit{experts in the field}, and not just interesting to \textit{yourself}; e.g., if you have just learnt how to prove a standard lemma which is well known to the experts and already in the literature, this does not mean that you should provide the standard proof of this standard lemma, unless this serves some greater purpose in the paper (e.g. by motivating a less standard lemma).

Conversely, some computations, definitions, or notational conventions which you are very familiar with, but are not widely known in the field, should be expounded on in detail, even if these details are ``obvious'' to you due to your extensive work in this area. Even a brief sentence of explanation is much better than none at all.

For a similar reason, if you are using a relatively obscure lemma from, say, 1 of your own papers, you should not assume that every reader of your current article is intimately familiar with your previous paper. In such cases it is worth stating the lemma in full, with a precise citation (as opposed to casually using phrases e.g. ``by a lemma in [my previous 100-page paper], we have $\ldots$''). When the lemma is particularly crucial, it is sometimes also worth spending a paragraph to sketch out a proof, or to otherwise remark on the significance of this lemma and its connections to other, more well known results.''

\subsubsection{Terence Tao\texttt{/}\href{https://terrytao.wordpress.com/advice-on-writing-papers/write-in-your-own-voice/}{On Writing\texttt{/}Write in Your Own Voice}}

\begin{quotation}
	``While one should always study the method of a great artist, one should never imitate his manner. The manner of an artist is essentially individual, the method of an artist is absolutely universal. The first is personality, which no one should copy; the second is perfection, which all should aim at.'' -- Oscar Wilde, \textit{A Critic in Pall Mall}, p. 195
\end{quotation}
``When, as a graduate student, one is starting out one's research in a mathematical subject, one usually begins by reading the papers of the current and past leaders of the field. Initially, one's understanding of the subject is fairly limited, and so it is natural to view these papers as being authoritative, especially if their authors are well-known.

Eventually, though, one requires a fair fraction of the insights and understanding conveyed by the existing literature, and is able to apply it to produce a new result or observation that goes beyond that literature (or, at least, makes explicit what was only implicitly buried in previous papers). When the ramifications and extensions of these new advances have been explored to their natural extent, it then becomes time to write up these results as a research paper.

Of course, as your work is almost certainly based in part on the previous literature, one should cite that literature whenever appropriate, and compare and contrast your own work with that literature in an \href{https://terrytao.wordpress.com/advice-on-writing-papers/describe-the-results-accurately/}{accurate}, \href{https://terrytao.wordpress.com/advice-on-writing-papers/write-professionally/}{professional}, and informative manner. Also, one should try to \href{https://terrytao.wordpress.com/advice-on-writing-papers/use-good-notation/}{maintain some level of notational consistency} with the previous literature, such as using the same fundamental definitions and to use similar notation, so that expert readers who are already familiar with that literature can quickly get up to speed on your work. And if 1 of the arguments in your work is standard in the literature, it certainly makes sense to structure the argument in a standard fashion if possible, again to assist the experts reading your paper.

\textbf{However}, one should \textbf{not} go so far as to copy entire paragraphs or more of text from a prior paper, except when used sa a direct quotation to illustrate some historical point. First of all, if one does not properly attribute that text (e.g. ``As Bourbaki [17, p. 146] writes,'', or, for that matter, the Oscar Wilde quote above), then one runs the risk of committing \href{http://en.wikipedia.org/wiki/Plagiarism}{plagiarism}. But even if the text is properly attributed, copying the text verbatim, without updating it to reflect more recent developments (including that in the paper being written) and to add your own simplifications and insights, is a redundant waste of space and a lost opportunity to advance the subject. If one is tempted to copy a significant portion of text from a prior reference without adding anything significantly new, one should instead simply cite the previous reference appropriately, e.g. ``See [27, Section 4] for further discussion.'' or ``A proof can be found in [9, Lemma 2.4].'' (cf. ``\href{https://terrytao.wordpress.com/advice-on-writing-papers/give-appropriate-amounts-of-detail/}{Give appropriate amounts of details}'').

Of course, there \textit{are} reasons to duplicate to some extent some discussion or argument that was present in a previous paper:
\begin{itemize}
	\item As mentioned earlier, one may wish to make some historical point, e.g. to track the development of a mathematical idea over time.
	\item If the paper is obscure and not widely available, reproducing a key argument from that paper may serve as a convenience to the reader.
	\item Also, if the \textit{form} of that argument can be used to \href{https://terrytao.wordpress.com/advice-on-writing-papers/motivate-the-paper/}{motivate} other arguments in your paper, then it can be worth putting in that argument so that it can be referred to later in the paper.
	\item The precise result needed for your paper may differ slightly from what is already established in the literature, and so one needs to either write out a modified version of the proof, or else point to the original proof but indicate what modifications need to be made. (The latter is suitable if the changes are particularly minor in nature.)
	\item The existing paper may have an argument which can be updated, simplified, modernized, or otherwise improved thanks to more recent advances or insights in the area (including your own). It can then be a service to the field to place an updated version of the argument in the literature (with full citations to the paper containing the original argument, of course).
\end{itemize}
However, when one is not simply quoting the prior text for historical or archival purposes, it is best to \textit{paraphrase} and \textit{interpret} the previous text rather than to copy that text verbatim. This is for a number of reasons:
\begin{itemize}
	\item One wants to avoid conveying any impression to readers, referees, or editors of plagiarism, padding, or intellectual laziness in one's papers. (Note that the latter is a danger even if one is copying from one's own work, rather than that of others.)
	\item The prior work may be dated in view of more recent developments and insights, as mentioned above.
	\item If you are copying or adapted a piece of text from another author that you do not fully understand yourself, then it may end up being inappropriate or incongruous for your intended purpose, and may convey the impression of superficiality or being ill-informed. If the text becomes inaccurate due to this adaptation, then this can also cause some embarrassment and annoyance for the original author of that text.
	\item Excessive use of quotation from famous mathematicians to make one's own work look more impressive is the mathematical equivalent of name-dropping, and should be avoided. Appeal to authority should not be the primary basis for motivating a paper; a handful of citations to demonstrate the depth of interest in the problem being studied is usually sufficient.
	\item \textit{But most importantly of all, for one's further mathematical development and career, one needs to develop one's own consistent mathematical ``voice'' and style, and to avoid the impression of simply imitating the voices of other authors}. There is no need in this subject for the mathematical equivalent of a parrot, and a text which is a mix of the author's voice and the voice of others can read very strangely.
\end{itemize}
Of course, if one is paraphrasing a previous work, one should cite that work appropriately (e.g. ``The proof here is loosely based on that in [5].'' or ``This discussion is inspired by a related discussion in [10].'').

In some cases, the imitation of a previous author's style and text is intended as a sign of respect or flattery for that author. \textbf{This is misguided}; an author will in fact often find such mimicry to actually be somewhat offensive. If one wants to truly respect a mathematician, then understand that mathematician's methods, results, and exposition, and improve, update, adapt, and advance all 3. Even the greatest mathematician's contributions should advance with the field, rather than being worshiped and preserved in some supposed state of perfection; the latter is mostly suitable only for historical purposes.

Another possible reason for copying the style of a more senior mathematician is that one does not yet have the self-confidence to write in one's own style and voice. While this is justifiable to some extent when one is just starting one's career, it becomes less excusable as one continues one's research. If one is hesitant to state things in one's own fashion, it is perfectly acceptable to couch such text with the appropriate caveats (e.g. ``to the author's knowledge, this observation is new'' or ``While Lemma 2.5 is usually phrased in a topological fashion, we found the following, more geometric, formulation to be more convenient for our applications''). And if one does not feel confident enough in one's understanding of a subject to explain it in any other way than copying from a previous paper, then this should be taken as a sign that one still needs to \href{https://terrytao.wordpress.com/career-advice/learn-and-relearn-your-field/}{internalize the subject futher}.

When writing a paper with 1 or more coauthors, there will inevitably be distinctions in style,\footnote{NQBH: a reasonable justification for loneliness and\texttt{/}in solo (academic) writing.} and so initially different sections may have sharply different tones due to their being largely written by different subsets of coauthors; but I usually find that after a few rounds of editing, the voices are harmonized into a style which is clearly derived from, but distinct from, each of the individual styles. Ideally, one should understand and respect the underlying stylistic decisions of one's coauthors, but at the same time be willing to take the initiative and find ways to formulate the text and arrangement to smoothly reconcile the coauthor's preferences with one's own; if all goes well, this can lead to a level of exposition and presentation that is superior to what each of the individual authors could separately achieve. (Of course, if you are to perform major edits on a coauthor's contribution, some consultation with that coauthor is presumably desirable). This process can be quite educational; my own writing style has definitely been influenced in a positive fashion by those of my coauthors.

Developing one's own style is, by definition, a very personal process; while external advice or role models can certainly be of some influence, they are of limited utility after a certain point. But finding an individual style which is comfortable and effective for both you and your readers is an important mark of one's \textit{mathematical maturity}, and is a goal that is definitely worth pursuing.''
%------------------------------------------------------------------------------%

\begin{thebibliography}{99}
	\bibitem[TerryTao]{TerryTao} \href{https://terrytao.wordpress.com}{Terence Tao's blog}.
	\begin{itemize}
		\item Terence Tao. \href{https://terrytao.wordpress.com/advice-on-writing-papers/}{\textit{On writing}} (or in full: \textit{Advice on writing papers}).
		\begin{itemize}
			\item Terence Tao. \href{https://terrytao.wordpress.com/advice-on-writing-papers/describe-the-results-accurately/}{\textit{On writing}\texttt{/}\textit{Describe the results accurately}}.
			\item Terence Tao. \href{https://terrytao.wordpress.com/advice-on-writing-papers/give-appropriate-amounts-of-detail/}{\textit{On writing}\texttt{/}\textit{Give appropriate amounts of detail}}.
			\item Terence Tao. \href{https://terrytao.wordpress.com/advice-on-writing-papers/write-in-your-own-voice/}{\textit{On writing}\texttt{/}\textit{Write in your own voice}}.
		\end{itemize}
	\end{itemize}
\end{thebibliography}

\printbibliography[heading=bibintoc]
	
\end{document}