\documentclass[oneside]{book}
\usepackage[backend=biber,natbib=true,style=authoryear]{biblatex}
\addbibresource{/home/hong/1_NQBH/reference/bib.bib}
\usepackage[vietnamese,english]{babel}
\usepackage{tocloft}
\renewcommand{\cftsecleader}{\cftdotfill{\cftdotsep}}
\usepackage[colorlinks=true,linkcolor=blue,urlcolor=red,citecolor=magenta]{hyperref}
\usepackage{amsmath,amssymb,amsthm,mathtools,float,graphicx}
\allowdisplaybreaks
\numberwithin{equation}{section}
\newtheorem{assumption}{Assumption}[chapter]
\newtheorem{conjecture}{Conjecture}[chapter]
\newtheorem{corollary}{Corollary}[chapter]
\newtheorem{definition}{Definition}[chapter]
\newtheorem{example}{Example}[chapter]
\newtheorem{lemma}{Lemma}[chapter]
\newtheorem{notation}{Notation}[chapter]
\newtheorem{principle}{Principle}[chapter]
\newtheorem{problem}{Problem}[chapter]
\newtheorem{proposition}{Proposition}[chapter]
\newtheorem{question}{Question}[chapter]
\newtheorem{remark}{Remark}[chapter]
\newtheorem{theorem}{Theorem}[chapter]
\usepackage[left=0.5in,right=0.5in,top=1.5cm,bottom=1.5cm]{geometry}
\usepackage{fancyhdr}
\pagestyle{fancy}
\fancyhf{}
\lhead{\small \textsc{Sect.} ~\thesection}
\rhead{\small \nouppercase{\leftmark}}
\renewcommand{\sectionmark}[1]{\markboth{#1}{}}
\cfoot{\thepage}
\def\labelitemii{$\circ$}

\title{A Personal Journey to Writing}
\author{\selectlanguage{vietnamese} Nguyễn Quản Bá Hồng\footnote{Independent Researcher, Ben Tre City, Vietnam\\e-mail: \texttt{nguyenquanbahong@gmail.com}}}
\date{\today}

\begin{document}
\maketitle
\setcounter{tocdepth}{3}
\setcounter{secnumdepth}{3}
\tableofcontents

%-----------------------------------------------------------------------------%

\chapter*{Foreword}
My personal journey to ``The Garden of Words''\footnote{\href{https://www.imdb.com/title/tt2591814/}{IMDb\texttt{/}\textsc{The Garden of Words} (2013)}, original title: \textsc{Koto no ha no niwa}.} -- the world of writings. \textit{Why writing?} Because: Instead of provoking a weak \& poor defense mechanism passively \& unconsciously, you should make your verbal enemies take a step back 1st: Words are weapons.

Although I have studied my Master in France, \& later worked in Germany \& Austria, I only know some Frenches \& German words. I prefer to use Vietnamese \& English, even if I have a chance to learn a 3rd one comprehensively. The main purpose of writing is to express ideas, thoughts, emotions, etc., not to show off someone's linguistic ability, especially the wide range of languages they can use.

\section{Dictionary}
To read \& write well, the 1st concern is, obviously, to choose the right dictionary\texttt{/}dictionaries.

\begin{question}
	Which dictionary\emph{\texttt{/}}dictionaries should I use?
\end{question}

\begin{itemize}
	\item \href{https://dictionary.cambridge.org/}{Cambridge Dictionary}: ``Make your words meaningful''
	\item \href{https://www.collinsdictionary.com/us/}{Collins Dictionary}
	\item \href{https://www.merriam-webster.com/}{Merriam-Webster Dictionary}
	\item \href{https://www.oxfordlearnersdictionaries.com/}{Oxford Learner's Dicitonaries}
\end{itemize}
I choose \href{https://www.oxfordlearnersdictionaries.com/}{Oxford Learner's Dicitonaries}. Then the next question is:

\begin{question}
	Should I buy Oxford Learner's Dictionary of Academic English?
\end{question}
\pounds 5.5\texttt{/}year though. Bought: Seem worth it (?).

\begin{remark}[Personal style]
	I do not like to write the terms ``and'', \selectlanguage{vietnamese}``và'', or ``or'', ``hoặc''. I write the symbols ``\&'' \& ``\emph{\texttt{/}}'', respectively, instead.
\end{remark}

%-----------------------------------------------------------------------------%

\chapter{Wikipedia's}

\section{\href{https://en.wikipedia.org/wiki/Literature}{Wikipedia\texttt{/}Literature}}
\textsf{Fig. \href{https://en.wikipedia.org/wiki/The_Adventures_of_Pinocchio}{\textit{The Adventures of Pinocchio}} (1883) is a canonical piece of children's literature \& \href{https://en.wikipedia.org/wiki/List_of_best-selling_books}{1 of the best-selling books} ever published.}

``\textit{Literature} broadly is any collection of \href{https://en.wikipedia.org/wiki/Writing}{written} work, but it is also used more narrowly for writings specifically considered to be an \href{https://en.wikipedia.org/wiki/Art}{art} form, especially \href{https://en.wikipedia.org/wiki/Prose}{prose} \href{https://en.wikipedia.org/wiki/Fiction}{fiction}, \href{https://en.wikipedia.org/wiki/Drama}{drama}, \& \href{https://en.wikipedia.org/wiki/Poetry}{poetry}. In recent centuries, the definition has expanded to include \href{https://en.wikipedia.org/wiki/Oral_literature}{oral literature}, much of which has been transcribed. Literature is a method of recording, preserving, \& transmitting knowledge \& entertainment, \& can also have a social, psychological, spiritual, or political role.

Literature, as an art form, can also include works in various non-fiction genres, such as \href{https://en.wikipedia.org/wiki/Biography}{biography}, \href{https://en.wikipedia.org/wiki/Diary}{diaries}, \href{https://en.wikipedia.org/wiki/Memoir}{memoir}, \href{https://en.wikipedia.org/wiki/Letter_(message)}{letters}, \& the \href{https://en.wikipedia.org/wiki/Essay}{essay}. Within its broad definition, literature includes non-fictional books, articles or other printed information on a particular subject.

\href{https://en.wikipedia.org/wiki/Etymology}{Etymologically}, the term derives from \href{https://en.wikipedia.org/wiki/Latin_language}{Latin} \textit{literatura}\texttt{/}\textit{litteratura} ``learning, a writing, grammar,'' originally ``writing formed with letters,'' from \textit{litera}\texttt{/}\textit{littera} ``letter''. In spite of this, the term has also been applied to \href{https://en.wikipedia.org/wiki/Oral_literature}{spoken or sung texts}. \href{https://en.wikipedia.org/wiki/History_of_printing}{Developments in print technology} have allowed an ever-growing distribution \& proliferation of written works, which now includes \href{https://en.wikipedia.org/wiki/Electronic_literature}{electronic literature}.

Literature is classified according to whether it is poetry, prose or drama, \& such works are categorized according to historical periods, or their adherence to certain \href{https://en.wikipedia.org/wiki/Aesthetics}{aesthetic} features, or \href{https://en.wikipedia.org/wiki/Genre}{genre}.'' -- \href{https://en.wikipedia.org/wiki/Literature}{Wikipedia\texttt{/}literature}

\subsection{Definitions}
``Definitions of literature have varied over time. In \href{https://en.wikipedia.org/wiki/Western_Europe}{Western Europe}, prior to the 18th century, literature denoted all books \& writing literature can be seen as returning to older, more inclusive notions, so that \href{https://en.wikipedia.org/wiki/Cultural_studies}{cultural studies}, e.g., include, in addition to \href{https://en.wikipedia.org/wiki/Western_canon}{canonical works}, \href{https://en.wikipedia.org/wiki/Genre_fiction}{popular \& minority genres}. The word is also used in reference non-written works: to ``\href{https://en.wikipedia.org/wiki/Oral_literature}{oral literature}'' \& ``the literature of \href{https://en.wikipedia.org/wiki/Preliterate}{preliterate} culture''.

A \href{https://en.wikipedia.org/wiki/Value_judgment}{value judgment} definition of literature considers it as consisting solely of high quality writing that forms part of the \href{https://en.wikipedia.org/wiki/Belles-lettres}{\textit{belles-lettres}} (``fine writing'') tradition. An example of this in the (1910--11) \href{https://en.wikipedia.org/wiki/Encyclopedia_Britannica_Eleventh_Edition}{\textit{Encyclop\ae dia Britannica}} that classified literature as ``the best expression of the best thought reduced to writing''.'' -- \href{https://en.wikipedia.org/wiki/Literature#Definitions}{Wikipedia\texttt{/}literature\texttt{/}definitions}

\subsection{History}
Main article: \href{https://en.wikipedia.org/wiki/History_of_literature}{Wikiepdia\texttt{/}history of literature}.

\subsubsection{Oral literature}
\textsf{Fig. A traditional \href{https://en.wikipedia.org/wiki/Kyrgyz_people}{Kyrgyz} \href{https://en.wikipedia.org/wiki/Manaschi}{manaschi} performing part of the \href{https://en.wikipedia.org/wiki/Epic_of_Manas}{Epic of Manas} at a \href{https://en.wikipedia.org/wiki/Yurt}{yurt} camp in \href{https://en.wikipedia.org/wiki/Karakol}{Karakol}, \href{https://en.wikipedia.org/wiki/Kyrgyzstan}{Kyrgyzstan}.}

``The use of the term ``literature'' here is a little problematic because of its origins in the Latin \textit{littera}, ``letter,'' essentially writing. Alternatives such as ``oral forms'' \& ``oral genres'' have been suggested but the word literature is widely used.

\href{https://en.wikipedia.org/wiki/Oral_literature}{Oral literature} is an \href{https://en.wikipedia.org/wiki/Oral_tradition}{ancient human tradition} found in ``all corners of the world''. Modern archaeology has been unveiling evidence of the human efforts to preserve \& transmit arts \& knowledge that depended completely or partially on an oral tradition, across various cultures:
\begin{quotation}
	The Judeo--Christian Bible reveals its oral traditional roots; medieval European manuscripts are penned by performing scribes; geometric vases from archaic Greece mirror Homer's oral style. ($\ldots$) Indeed, if these final decades of the millennium have taught us anything, it must be that oral tradition never was the other we accused it of being; it never was the primitive, preliminary technology of communication we thought it to be. Rather, if the whole truth is told, oral tradition stands out as the single most dominant communicative technology of our species as both a historical fact \&, in many areas still, a contemporary reality.
\end{quotation}
The earliest \href{https://en.wikipedia.org/wiki/Poetry}{poetry} is believed to have been recited or sung, employed as a way of remembering \href{https://en.wikipedia.org/wiki/Oral_history}{history}, \href{https://en.wikipedia.org/wiki/Genealogy}{genealogy}, \& \href{https://en.wikipedia.org/wiki/Law}{law}.

In Asia, the transmission of folklore, mythologies as well as scriptures in ancient India, in different Indian religions, was by oral tradition, preserved with precision with the help of elaborate \href{https://en.wikipedia.org/wiki/Vedic_chant}{mnemonic techniques}.

The early Buddhist texts are also generally believed to be of oral tradition, with the 1st by comparing inconsistencies in the transmitted versions of literature from various oral societies such as the Greek, Serbia \& other cultures, then noting that the Vedic literature is too consistent \& vast to have been composed \& transmitted orally across generations, without being written down. According to Goody, the Vedic texts likely involved both a written \& oral tradition, calling it a ``parallel products of a literate society''.

\href{https://en.wikipedia.org/wiki/Australian_Aboriginal_culture}{Australian Aboriginal culture} has thrived on oral traditions \& oral histories passed down through thousands of years. In a study published in Feb 2020, new evidence showed that both \href{https://en.wikipedia.org/wiki/Budj_Bim}{Budj Bim} \& \href{https://en.wikipedia.org/wiki/Tower_Hill_(volcano)}{Tower Hill} volcanoes erupted between 34,000 \& 40,000 years ago. Significantly, this is a ``minimum age constraint for human presence in \href{https://en.wikipedia.org/wiki/Victoria,_Australia}{Victoria}'', \& also could be interpreted as evidence for the oral histories of the \href{https://en.wikipedia.org/wiki/Gunditjmara}{Gunditjmara} people, an \href{https://en.wikipedia.org/wiki/Aboriginal_Australian}{Aboriginal Australian} people of south-western Victoria, which tell of volcanic eruptions being some of the oldest oral traditions in existence. An axe found underneath \href{https://en.wikipedia.org/wiki/Volcanic_ash}{volcanic ash} in 1947 had already proven that humans inhabited the region before the eruption of Tower Hill.

All ancient Greek literature was to some degree oral in nature, \& the earliest literature completely so. \href{https://en.wikipedia.org/wiki/Homer}{Homer}'s epic poetry, states Michael Gagarin, was largely composed, performed \& transmitted orally. As folklores \& legends were performed in front of distant audiences, the singers would substitute the names in the stories with local characters or rulers to give the stories a local flavor \& thus connect with the audience, but making the historicity embedded in the oral tradition as unreliable. The lack of surviving texts about the Greek \& Roman religious traditions have led scholars to presume that these were ritualistic \& transmitted as oral traditions, but some scholars disagree that the complex rituals in the ancient Greek \& Roman civilizations were an exclusive product of an oral tradition.

Writing systems are not known to have existed among \href{https://en.wikipedia.org/wiki/Native_North_Americans}{Native North Americans} before contact with Europeans. Oral storytelling traditions flourished in a context without the use of writing to record \& preserve history, scientific knowledge, \& social practices. While some stories were told for amusement \& leisure, most functioned as practical lessons from tribal experience applied to immediate moral, social, psychological, \& environmental issues. Stories fuse fictional, supernatural, or otherwise exaggerated characters \& circumstances with real emotions \& morals as a means of teaching. Plots often reflect real life situations \& may be aimed at particular people known by the story's audience. In this way, social pressure could be exerted without directly causing embarrassment or social exclusion. E.g., rather than yelling, \href{https://en.wikipedia.org/wiki/Inuit_culture}{Inuit} parents might deter their children from wandering too close to the water's edge by telling a story about a sea monster with a pouch for children within its reach. See also \href{https://en.wikipedia.org/wiki/African_literature#Oral_literature}{Wikipedia\texttt{/}African literature\texttt{/}oral literature}

\paragraph{Oratory.} \href{https://en.wikipedia.org/wiki/Rhetoric}{Oratory} or the art of \href{https://en.wikipedia.org/wiki/Public_speaking}{public speaking} ``was for long considered a literary art''. From \href{https://en.wikipedia.org/wiki/Ancient_Greece}{Ancient Greence} to the late 19th century, \href{https://en.wikipedia.org/wiki/Rhetoric}{rhetoric} played a central role in Western education in training orators, lawyers, counselors, historians, statesmen, \& poets.'' -- \href{https://en.wikipedia.org/wiki/Literature#Oral_literature}{Wikipedia\texttt{/}literature\texttt{/}history\texttt{/}oral literature}

\subsubsection{Writing}
\textsf{Fig. Limestone \href{https://en.wikipedia.org/wiki/Kish_tablet}{Kish tablet} from \href{https://en.wikipedia.org/wiki/Sumer}{Summer} with pictographic writing; may be the earliest known writing, 3500 BC. \href{https://en.wikipedia.org/wiki/Ashmolean_Museum}{Ashmolean Museum}.}

``Further information: \href{https://en.wikipedia.org/wiki/History_of_writing}{Wikipedia\texttt{/}history of writing}. Around the 4th millennium BC, the complexity of trade \& administration in \href{https://en.wikipedia.org/wiki/Mesopotamia}{Mesopotamia} outgrew human memory, \& \href{https://en.wikipedia.org/wiki/Writing}{writing} became a more dependable method of recording \& presenting transactions in a permanent form. Though in both \href{https://en.wikipedia.org/wiki/Ancient_Egypt}{ancient Egypt} \& \href{https://en.wikipedia.org/wiki/Mesoamerica}{Mesoamerica}, writing may have already emerged because of the need to record historical \& environmental events. Subsequent innovations included more uniform, predictable, \href{https://en.wikipedia.org/wiki/List_of_national_legal_systems}{legal systems}, \href{https://en.wikipedia.org/wiki/Religious_text}{sacred texts}, \& the origins of modern practices of \href{https://en.wikipedia.org/wiki/Models_of_scientific_inquiry}{scientific inquiry} \& \href{https://en.wikipedia.org/wiki/Knowledge_management}{knowledge-consolidation}, all largely reliant on portable \& easily reproducible forms of writing.'' -- \href{https://en.wikipedia.org/wiki/Literature#Writing}{Wikipedia\texttt{/}literature\texttt{/}history\texttt{/}writing}

\subsubsection{Early written literature}
``Main articles: \href{https://en.wikipedia.org/wiki/History_of_literature}{Wikipedia\texttt{/}history of literature}, \href{https://en.wikipedia.org/wiki/Ancient_literature}{Wikipedia\texttt{/}ancient literature}, \& \href{https://en.wikipedia.org/wiki/History_of_books}{Wikipedia\texttt{/}history of books}. \href{https://en.wikipedia.org/wiki/Ancient_Egyptian_literature}{Ancient Egyptian literature}, along with \href{https://en.wikipedia.org/wiki/Sumerian_literature}{Sumerian literature}, are considered the world's \href{https://en.wikipedia.org/wiki/Ancient_literature}{oldest literatures}. The primary \href{https://en.wikipedia.org/wiki/Genre}{genres} of the literature of \href{https://en.wikipedia.org/wiki/Ancient_Egypt}{ancient Egypt}--\href{https://en.wikipedia.org/wiki/Didacticism}{didactic} texts, hymns \& prayers, \& tales -- were written almost entirely in verse; By the \href{https://en.wikipedia.org/wiki/Old_Kingdom}{Old Kingdom} (26th century BC--22nd century BC), literary works included \href{https://en.wikipedia.org/wiki/Ancient_Egyptian_funerary_texts}{funerary texts}, \href{https://en.wikipedia.org/wiki/Epistle}{epistles} \& letters, \href{https://en.wikipedia.org/wiki/Hymns}{hymns} \& poems, \& commemorative \href{https://en.wikipedia.org/wiki/Autobiography}{autobiographical} texts recounting the careers of prominent administrative officials. It was not until the early \href{https://en.wikipedia.org/wiki/Middle_Kingdom_of_Egypt}{Middible Kingdom} (21st century BC--17th century BC) that a narrative Egyptian literature was created.

Many works of early periods, even in narrative form, had a covert moral or didactic purpose, such as the Sanskrit \href{https://en.wikipedia.org/wiki/Panchatantra}{\textit{Panchatantra}} 200 BC--300 AD, based on older oral tradition. \href{https://en.wikipedia.org/wiki/Drama}{Drama} \& \href{https://en.wikipedia.org/wiki/Satire}{satire} also developed as urban culture provided a larger public audience, \& later readership, for literary production. \href{https://en.wikipedia.org/wiki/Lyric_poetry}{Lyric poetry} (as opposed to epic poetry) was often the specialty of courts \& aristocratic circles, particularly in East Asia where songs were collected by the Chinese aristocracy as poems, the most notable being the \textit{Shijing} or \href{https://en.wikipedia.org/wiki/Book_of_Songs_(Chinese)}{Book of Songs} (1046--c.600 BC).

In \href{https://en.wikipedia.org/wiki/Chinese_classics}{ancient China}, early literature was primarily focused on philosophy, \href{https://en.wikipedia.org/wiki/Historiography}{historiography}, \href{https://en.wikipedia.org/wiki/Military_science}{military science}, agriculture, \& \href{https://en.wikipedia.org/wiki/Chinese_poetry}{poetry}. China, the origin of modern \href{https://en.wikipedia.org/wiki/Paper_making}{paper making} \& \href{https://en.wikipedia.org/wiki/Woodblock_printing}{woodblock printing}, produced the world's 1st \href{https://en.wikipedia.org/wiki/Print_culture}{print cultures}. Much of Chinese literature originates with the \href{https://en.wikipedia.org/wiki/Hundred_Schools_of_Thought}{Hundred Schools of Thought} period that occurred during the \href{https://en.wikipedia.org/wiki/Zhou_Dynasty}{Eastern Zhou Dynasty} (769--269 BC). The most important of these include the Classics of \href{https://en.wikipedia.org/wiki/Confucianism}{Confucianism}, of \href{https://en.wikipedia.org/wiki/Taoism}{Daoism}, of \href{https://en.wikipedia.org/wiki/Mohism}{MMohism}, of \href{https://en.wikipedia.org/wiki/Legalism_(Chinese_philosophy)}{Legalism}, as well sa works of military science (e.g., \href{https://en.wikipedia.org/wiki/Sun_Tzu}{Sun Tzu}'s \href{https://en.wikipedia.org/wiki/The_Art_of_War}{\textit{The Art of War}}, c.5th century BC) \& \href{https://en.wikipedia.org/wiki/History_of_China}{Chinese history} (e.g. \href{https://en.wikipedia.org/wiki/Sima_Qian}{Sima Qian}'s \href{https://en.wikipedia.org/wiki/Records_of_the_Grand_Historian}{\textit{Records of the Grand Historian}}, c.94 BC). Ancient Chinese literature had a heavy emphasis on historiography, with often very detailed court records. An exemplary piece of \href{https://en.wikipedia.org/wiki/Narrative_history}{narrative history} of ancient China was the \href{https://en.wikipedia.org/wiki/Zuo_Zhuan}{\textit{Zuo Zhuan}}, which was compiled no later than 389 BC, \& attributed to the blind 5th-century BC historian \href{https://en.wikipedia.org/wiki/Zuo_Qiuming}{Zuo Qiuming}.

In \href{https://en.wikipedia.org/wiki/Indian_literature#In_archaic_Indian_languages}{ancient India}, literature originated from stories that were originally orally transmitted. Early genres included \href{https://en.wikipedia.org/wiki/Sanskrit_drama}{drama}, \href{https://en.wikipedia.org/wiki/Panchatantra}{fables}, \href{https://en.wikipedia.org/wiki/S%C5%ABtra}{sutras} \& \href{https://en.wikipedia.org/wiki/Indian_epic_poetry}{epic poetry}. \href{https://en.wikipedia.org/wiki/Sanskrit_literature}{Sanskrit literature} begins with the \href{https://en.wikipedia.org/wiki/Vedas}{Vedas}, dating back to 1500--1000 BC, \& continues with the \href{https://en.wikipedia.org/wiki/Sanskrit_Epics}{Sanskrit Epics} of \href{https://en.wikipedia.org/wiki/Iron_Age_India}{Iron Age India}. The Vedas are among the \href{https://en.wikipedia.org/wiki/Ancient_literature}{oldest sacred texts}. The Samhitas (vedic collections) date to roughly 1500--1000 BC, \& the ``circum-Vedic'' texts, as well as the \href{https://en.wikipedia.org/wiki/Shakha}{redaction} of the Samhitas, date to c. 1000--500 BC, resulting in a \href{https://en.wikipedia.org/wiki/Vedic_period}{Vedic period}, spanning the mid-2nd to mid 1st millennium BC, or the \href{https://en.wikipedia.org/wiki/Bronze_Age}{Late Bronze Age} \& the \href{https://en.wikipedia.org/wiki/Iron_Age_India}{Iron Age}. The period between approximately the 6th to 1st centuries BC saw the composition \& redaction of the 2 most influential Indian epics, the \href{https://en.wikipedia.org/wiki/Mahabharata}{\textit{Mahabharata}} \& the \href{https://en.wikipedia.org/wiki/Ramayana}{Ramayana}, with subsequent redaction progressing down to the 4th century AD. Other major literary works are \href{https://en.wikipedia.org/wiki/Ramcharitmanas}{Ramcharitmanas} \& Krishnacharitmanas.

The earliest known Greek writings are \href{https://en.wikipedia.org/wiki/Mycenaean_language}{Mycenaean} (c.1600--1100 BC), written in the \href{https://en.wikipedia.org/wiki/Linear_B}{Linear B} syllabary on clay tablets. These documents contain prosaic records largely concerned with trade (lists, inventories, receipts, etc.); no real literature has been discovered. \href{https://en.wikipedia.org/wiki/Michael_Ventris}{Michael Ventris} \& \href{https://en.wikipedia.org/wiki/John_Chadwick}{John Chadwick}, the original decipherers of Linear B, state that literature almost certainly existed in \href{https://en.wikipedia.org/wiki/Mycenaean_Greece}{Mycenaean Greece}, but it was either not written down or, if it was, it was on parchment or wooden tablets, which did not survive the \href{https://en.wikipedia.org/wiki/Late_Bronze_Age_collapse#Greece}{destruction of the Mycenaean palaces in the 12th century BC}. \href{https://en.wikipedia.org/wiki/Homer}{Homer}'s, \href{https://en.wikipedia.org/wiki/Epic_poems}{epic poems} the \href{https://en.wikipedia.org/wiki/Iliad}{\textit{Iliad}} \& the \href{https://en.wikipedia.org/wiki/Odyssey}{Odyssey}, are central works of \href{https://en.wikipedia.org/wiki/Ancient_Greek_literature}{ancient Greek literature}. It is generally accepted that the poems were composed at some point around the late 8th or early 7th century BC. Modern scholars consider these accounts \href{https://en.wikipedia.org/wiki/Legend}{legendary}. Most researchers believe that the poems were originally \href{https://en.wikipedia.org/wiki/Oral_tradition}{transmitted orally}. From \href{https://en.wikipedia.org/wiki/Classical_antiquity}{antiquity} until the present day, the influence of Homeric epic on \href{https://en.wikipedia.org/wiki/Western_culture}{Western civilization} has been great, inspiring many of its most famous works of literature, music, art \& film. The Homeric epics were the greatest influence on ancient Greek culture \& education; to \href{https://en.wikipedia.org/wiki/Plato}{Plato}, Homer was simply the one who ``has taught Greece'' -- \textit{10 Hellada pepaideuken}. \href{https://en.wikipedia.org/wiki/Hesiod}{Hesiod}'s \href{https://en.wikipedia.org/wiki/Works_and_Days}{Works \& Days} (c.700 BC) \& \href{https://en.wikipedia.org/wiki/Theogony}{Theogony}, are some of the earliest, \& most influential, of ancient Greek literature. Classical Greek genres included philosophy, \href{https://en.wikipedia.org/wiki/Poetry}{poetry}, historiography, \href{https://en.wikipedia.org/wiki/Comedies}{comedies} \& \href{https://en.wikipedia.org/wiki/Drama}{dramas}. \href{https://en.wikipedia.org/wiki/Plato}{Plato} (428\texttt{/}427 or 424\texttt{/}423--348\texttt{/}347 BC) \& \href{https://en.wikipedia.org/wiki/Aristotle}{Aristotle} (384--322 BC) authored philosophical texts that are the foundation of \href{https://en.wikipedia.org/wiki/Western_philosophy}{Western philosophy}, \href{https://en.wikipedia.org/wiki/Sappho}{Sappho} (c.630--c.570 BC) \& \href{https://en.wikipedia.org/wiki/Pindar}{Pindar} were influential \href{https://en.wikipedia.org/wiki/Lyric_poetry}{lyric poets}, \& \href{https://en.wikipedia.org/wiki/Herodotus}{Herodotus} (c.484--c.425 BC) \& \href{https://en.wikipedia.org/wiki/Thucydides}{Thucydides} were early Greek historians. Although drama was popular in ancient Greece, of the hundreds of \href{https://en.wikipedia.org/wiki/Tragedy}{tragedies} written \& performed during the \href{https://en.wikipedia.org/wiki/Classical_age}{classical age}, only a limited number of plays by 3 authors still exist: \href{https://en.wikipedia.org/wiki/Aeschylus}{Aeschylus}, \href{https://en.wikipedia.org/wiki/Sophocles}{Sophocles}, \& \href{https://en.wikipedia.org/wiki/Euripides}{Euripides}. The plays of \href{https://en.wikipedia.org/wiki/Aristophanes}{Aristophanes} (c.446--c.386 BC) provide the only real examples of a genre of comic drama known as \href{https://en.wikipedia.org/wiki/Ancient_Greek_comedy}{Old Comedy}, the earliest form of Greek Comedy, \& are in fact used to define the genre.

The \href{https://en.wikipedia.org/wiki/Hebrew}{Hebrew} religious text, the \href{https://en.wikipedia.org/wiki/Torah}{Torah}, is widely seen as a product of the \href{https://en.wikipedia.org/wiki/Persian_period}{Persian period} (539--333 BC, probably 450--350 BC). This consensus echoes a traditional Jewish view which gives \href{https://en.wikipedia.org/wiki/Ezra}{Ezra}, the leader of the Jewish community on its return from Babylon, a pivotal role in its promulgation. This represents a major source of Christianity's \href{https://en.wikipedia.org/wiki/Bible}{Bible}, which has been a major influence on Western literature.

The beginning of \href{https://en.wikipedia.org/wiki/Roman_literature}{Roman literature} dates to 240 BC, when a Roman audience saw a Latin version of a Greek play. Literature in \href{https://en.wikipedia.org/wiki/Latin}{Latin} would flourish for the next 6 centuries, \& includes essays, histories, poems, plays, \& other writings.

The \href{https://en.wikipedia.org/wiki/Qur%27an}{Qur'an} (610 AD--632 AD), the main \href{https://en.wikipedia.org/wiki/Religious_text}{holy book} of \href{https://en.wikipedia.org/wiki/Islam}{Islam}, had a significant influence on the Arab language, \& marked the beginning of \href{https://en.wikipedia.org/wiki/Islamic_literature}{Islamic literature}. Muslims believe it was transcribed in the Arabic dialect of the \href{https://en.wikipedia.org/wiki/Quraysh}{uraysh}, the tribe of \href{https://en.wikipedia.org/wiki/Muhammad}{Muhammad}. As Islam spread, the Quran had the effect of unifying \& standardizing Arabic.

Theological works in Latin were the dominant form of \href{https://en.wikipedia.org/wiki/Mediaeval_literature#Types_of_writing}{literature} in Europe typically found in libraries during the \href{https://en.wikipedia.org/wiki/Middle_Ages}{Middle Ages}. \href{https://en.wikipedia.org/wiki/Western_culture}{Western} \href{https://en.wikipedia.org/wiki/Vernacular_literature}{Vernacular literature} includes the \href{https://en.wikipedia.org/wiki/Poetic_Edda}{Poetic Edda} \& the \href{https://en.wikipedia.org/wiki/Sagas}{sagas}, or heroic epics, of Iceland, the Anglo-Saxon \href{https://en.wikipedia.org/wiki/Beowulf}{iBeowulf}, \& the German \href{https://en.wikipedia.org/wiki/Song_of_Hildebrandt}{\textit{Song of Hildebrandt}}. A later form of \href{https://en.wikipedia.org/wiki/Mediaeval_literature}{medieval fiction} was the \href{https://en.wikipedia.org/wiki/Romance_(heroic_literature)}{romance}, an adventurous \& sometimes magical narrative with strong popular appeal.

Controversial, religious, political \& instructional literature proliferated during the European \href{https://en.wikipedia.org/wiki/Renaissance}{Renaissance} as a result of the \href{https://en.wikipedia.org/wiki/Johannes_Gutenberg}{Johannes Gutenberg}'s invention of the \href{https://en.wikipedia.org/wiki/Printing_press}{printing press} around 1440, while the \href{https://en.wikipedia.org/wiki/Medieval_romance}{Medieval romance} developed into the \href{https://en.wikipedia.org/wiki/Novel}{novel}.'' -- \href{https://en.wikipedia.org/wiki/Literature#Early_written_literature}{Wikipedia\texttt{/}literature\texttt{/}history\texttt{/}early written literature}

\subsubsection{Publishing}
\textsf{Fig. the intricate frontispiece of the \href{https://en.wikipedia.org/wiki/Diamond_Sutra}{Diamond Sutra} from \href{https://en.wikipedia.org/wiki/Tang_dynasty}{Tang dynasty} China, the world's earliest dated printed book, AD 868 (\href{https://en.wikipedia.org/wiki/British_Library}{British Library}).}

``Publishing became possible with the \href{https://en.wikipedia.org/wiki/History_of_writing}{inventing of writing}, but became more practical with the \href{https://en.wikipedia.org/wiki/History_of_printing}{invention of printing}. Prior to printing, distributed works were copied manually, by \href{https://en.wikipedia.org/wiki/Scribe}{scribes}.

The Chinese inventor \href{https://en.wikipedia.org/wiki/Bi_Sheng}{Bi Sheng} made \href{https://en.wikipedia.org/wiki/Movable_type}{movable type} of earthenware c.1045. Then c.1450, separately \href{https://en.wikipedia.org/wiki/Johannes_Gutenberg}{Johannes Gutenberg} invented movable type in Europe. This invention gradually made books less expensive to produce \& more widely available.

Early printed books, single sheets \& images which were created before 1501 in Europe are known as \href{https://en.wikipedia.org/wiki/Incunable}{incunables} or \textit{incunabula}. ``A man born in 1453, the year of the fall of Constantinople, could look back from his 50th year on a lifetime in which about 8 million books had been printed, more perhaps than all the scribes of Europe had produced since Constantine founded his city in A.D. 330.''

Eventually, printing enabled other forms of publishing besides books. The \href{https://en.wikipedia.org/wiki/History_of_newspaper_publishing}{history of modern newspaper publishing} started in Germany in 1609, with \href{https://en.wikipedia.org/wiki/Magazine#History}{publishing of magaiznes} following in 1663.'' -- \href{https://en.wikipedia.org/wiki/Literature#Publishing}{Wikipedia\texttt{/}literature\texttt{/}history\texttt{/}publishing}

\subsubsection{University discipline}

\paragraph{In England.} ``Main article: \href{https://en.wikipedia.org/wiki/English_studies}{English studies}. In England in the late 1820s, growing political \& social awareness, ``particularly among the \href{https://en.wikipedia.org/wiki/Utilitarians}{utilitarians} \& \href{https://en.wikipedia.org/wiki/Jeremy_Bentham}{Benthamites}, promoted the possibility of including courses in English literary study in the newly formed \href{https://en.wikipedia.org/wiki/London_University}{London University}''. This further developed into the idea of the study of literature being ``the ideal carrier for the propagation of the humanist cultural myth of a well-educated, culturally harmonious nation''.

\paragraph{America.} \href{https://en.wikipedia.org/wiki/American_Literature_(academic_discipline)}{Wikipedia\texttt{/}American Literature (academic discipline)}.'' -- \href{https://en.wikipedia.org/wiki/Literature#University_discipline}{Wikipedia\texttt{/}literature\texttt{/}history\texttt{/}university discipline}

\subsubsection{Women \& literature}
``Further information: \href{https://en.wikipedia.org/wiki/French_literature}{French literature}, \href{https://en.wikipedia.org/wiki/German_literature}{German literature}, \href{https://en.wikipedia.org/wiki/Russian_literature}{Russian literature}, \& \href{https://en.wikipedia.org/wiki/English_poetry#Women_poets_in_the_18th_century}{Wikipedia\texttt{/}English poetry\texttt{/}Women poets in the 18th century}. The widespread education of women was not common until the 19th century, \& because of this literature until recently was mostly \href{https://en.wikipedia.org/wiki/Western_canon#Historical_exclusion_of_women}{male dominated}.

There are few women poets writing in English, whose names are remembered, until the 20th century. In the \href{https://en.wikipedia.org/wiki/English_poetry#Victorian_poetry}{19th century} some names that stand out are \href{https://en.wikipedia.org/wiki/Emily_Bront%C3%AB}{Emily Bront\"e}, \href{https://en.wikipedia.org/wiki/Elizabeth_Barrett_Browning}{Elizabeth Barrett Browning}, \& \href{https://en.wikipedia.org/wiki/Emily_Dickinson}{Emily Dickinson} (see \href{https://en.wikipedia.org/wiki/American_poetry}{American poetry}). But while generally women are absent from the European cannon of \href{https://en.wikipedia.org/wiki/Romantic_poetry}{Romantic literature}, there is 1 notable exception, the French novelist \& memoirist Amantine Dupin (1804--1876) best known by her pen name \href{https://en.wikipedia.org/wiki/George_Sand}{George Sand}. 1 of the more popular writers in Europe in her lifetime, being more renowned than both \href{https://en.wikipedia.org/wiki/Victor_Hugo}{Victor Hugo} \& \href{https://en.wikipedia.org/wiki/Honor%C3%A9_de_Balzac}{Honor\'e de Balzac} in England in the 1830s \& 1840s, Sand is recognized as 1 of the most notable writers of the European Romantic era. \href{https://en.wikipedia.org/wiki/Jane_Austen}{Jane Austen} (1775--1817) is the 1st major English woman novelist, while \href{https://en.wikipedia.org/wiki/Aphra_Behn}{Aphra Behn} is an early female dramatist.
\begin{quotation}
	``George Sand was an idea. She has a unique place in our age. Others are great men $\ldots$ she was a great woman.'' -- \href{https://en.wikipedia.org/wiki/Victor_Hugo}{Victor Hugo}, \textit{Les fun\'erailles de George Sand}
\end{quotation}
\href{https://en.wikipedia.org/wiki/Nobel_Prize_in_Literature}{Nobel Prizes in Literature} have been awarded between 1901 \& 2020 to 117 individuals: 101 men \& 16 women. \href{https://en.wikipedia.org/wiki/Selma_Lagerl%C3%B6f}{Selma Lagerl\"of} (1858--1940) was the 1st women to win the \href{https://en.wikipedia.org/wiki/Nobel_Prize_in_Literature}{Nobel Prizes in Literature}, which she was awarded in 1909. Additionally, she was the 1st woman to be granted a membership in The \href{https://en.wikipedia.org/wiki/Swedish_Academy}{Swedish Academy} in 1914.

\href{https://en.wikipedia.org/wiki/Feminism}{Feminist scholars} have since the 20th century sought \href{https://en.wikipedia.org/wiki/Women%27s_writing_(literary_category)#Rediscovering_ignored_works_from_the_past}{expand the literary canon} to include more women writers.'' -- \href{https://en.wikipedia.org/wiki/Literature#Women_and_literature}{Wikipedia\texttt{/}literature\texttt{/}history\texttt{/}women \& literature}

\subsubsection{Children's literature}
``A separate genre of \href{https://en.wikipedia.org/wiki/Children%27s_literature}{children's literature} only began to emerge in the 18th century, with the development of the concept of \href{https://en.wikipedia.org/wiki/Childhood}{childhood}. The earliest of these books were educational books, books on conduct, \& simple ABCs -- often decorated with animals, plants, \& anthropomorphic letters.'' -- \href{https://en.wikipedia.org/wiki/Literature#Children's_literature}{Wikipedia\texttt{/}literature\texttt{/}history\texttt{/}children's literature}

\subsection{Aesthetics}
Further information: \href{https://en.wikipedia.org/wiki/Aesthetic_judgment}{Wikipedia\texttt{/}aesthetic judgment} \& \href{https://en.wikipedia.org/wiki/Value_judgment}{Wikipedia\texttt{/}value judgment}.

\subsubsection{Literary theory}
``Further information: \href{https://en.wikipedia.org/wiki/Literary_theory}{Wikipedia\texttt{/}literary theory} \& \href{https://en.wikipedia.org/wiki/Philosophy_and_literature#The_philosophy_of_literature}{Wikipedia\texttt{/}philosophy \& literature\texttt{/}the philosophy of literature}. A fundamental question of \href{https://en.wikipedia.org/wiki/Literary_theory}{literary theory} is ``what is literature?'' -- although many contemporary theorists \& literary scholars believe either that ``literature'' cannot be defined or that it can refer to any use of language.'' -- \href{https://en.wikipedia.org/wiki/Literature#Literary_theory}{Wikipedia\texttt{/}literature\texttt{/}aesthetics\texttt{/}literary theory}

\subsubsection{Literary fiction}
``Further information: \href{https://en.wikipedia.org/wiki/Western_canon#Literary_canon}{Wikipedia\texttt{/}Western canon\texttt{/}literary canon}. \href{https://en.wikipedia.org/wiki/Literary_fiction}{Literary fiction} is a term used to describe \href{https://en.wikipedia.org/wiki/Fiction}{fiction} that explores any facet of the \href{https://en.wikipedia.org/wiki/Human_condition}{human condition}, \& may involve \href{https://en.wikipedia.org/wiki/Social_commentary}{social commentary}. It is often regarded as having more artistic merit than \href{https://en.wikipedia.org/wiki/Genre_fiction}{genre fiction}, especially the most commercially oriented types, but this has been contested in recent years, with the serious study of genre fiction within universities.

The following, by the award-winning British author \href{https://en.wikipedia.org/wiki/William_Boyd_(writer)}{William Boyd} on the short story, might be applied to all prose fiction:
\begin{quotation}
	[short stories] seem to answer something very deep in our nature as if, for the duration of its telling, something special has been created, some essence of our experience extrapolated, some temporary sense has been made of our common, turbulent journey towards the grave \& oblivion.
\end{quotation}
The very best in literature is annually recognized by the \href{https://en.wikipedia.org/wiki/Nobel_Prize_in_Literature}{Nobel Prize in Literature}, which is awarded to an author from any country who has, in the words of the will of Swedish industrialist \href{https://en.wikipedia.org/wiki/Alfred_Nobel}{Alfred Nobel}, produced ``in the field of literature the most outstanding work in an ideal direction'' (original Swedish: \textit{den som inom litteraturen har producerat det mest framstående verket i en idealisk riktning}).'' -- \href{https://en.wikipedia.org/wiki/Literature#Literary_fiction}{Wikipedia\texttt{/}literature\texttt{/}aesthetics\texttt{/}literary fiction}

\subsubsection{The value of imaginative literature}
``Some researchers suggest that literary fiction can play a role in an individual's psychological development. Psychologists have also been using literature as a therapeutic tool. Psychologist Hogan argues for the value of the time \& emotion that a person devotes to understanding a character's situation in literature; that it can unite a large community by provoking universal emotions, as well as allowing readers access to different cultures, \& new emotional experiences. 1 study, e.g., suggested that the presence of familiar cultural values in literary texts played an important impact on the performance of minority students.

Psychologist \href{https://en.wikipedia.org/wiki/Abraham_Maslow}{Maslow}'s ideas help literary critics understand how characters in literature reflect their personal culture \& the history. The theory suggests that literature helps an individual's struggle for self-fulfillment.'' -- \href{https://en.wikipedia.org/wiki/Literature#The_value_of_imaginative_literature}{Wikipedia\texttt{/}literature\texttt{/}aesthetics\texttt{/}the value of imaginative literature}

\subsection{The influence of religious texts}
``Further information: \href{https://en.wikipedia.org/wiki/Islamic_literature}{Wikipedia\texttt{/}Islamic literature} \& \href{https://en.wikipedia.org/wiki/King_James_Version#Influence}{Wikipedia\texttt{/}King James Version\texttt{/}influence}. Religion has had a major influence on literature, through works like the \href{https://en.wikipedia.org/wiki/Vedas}{Vedas}, the \href{https://en.wikipedia.org/wiki/Torah}{Torah}, the \href{https://en.wikipedia.org/wiki/Bible}{Bible}, \& the \href{https://en.wikipedia.org/wiki/Qur%27an}{Qur'an}.

The \href{https://en.wikipedia.org/wiki/King_James_Version}{King James Version} of the Bible has been called ``the most influential version of the most influential book in the world, in what is now its most influential language'', ``the most important book in English religion \& culture'', \& ``the most celebrated book in the \href{https://en.wikipedia.org/wiki/English-speaking_world}{English-speaking world}'' -- principally because of its literary style \& widespread distribution. Prominent \href{https://en.wikipedia.org/wiki/Atheism}{atheist} figures such as the late \href{https://en.wikipedia.org/wiki/Christopher_Hitchens}{Christopher Hitchens} \& \href{https://en.wikipedia.org/wiki/Richard_Dawkins}{Richard Dawkins} have praised the King James Version as being ``a giant step in the maturing of English literature'' \& ``a great work of literature'', respectively, with Dawkins then adding, ``A native speaker of English who has never read a word of the King James Bible is verging on the barbarian''.

Societies in which \href{https://en.wikipedia.org/wiki/Preaching}{preaching} has great importance, \& those in which religious structures \& \href{https://en.wikipedia.org/wiki/Clergy}{authorities} have a near-monopoly of \href{https://en.wikipedia.org/wiki/Literacy}{reading \& writing} \&\texttt{/}or a censorship role, may impart a religious gloss to much of the literature those societies produce or retain -- as e.g. in the \href{https://en.wikipedia.org/wiki/European_Middle_Ages}{European Middle Ages}. The traditions of \href{https://en.wikipedia.org/wiki/Textual_criticism}{close study} of religious texts has furthered the development of techniques \& theories in \href{https://en.wikipedia.org/wiki/Literary_studies}{literary studies}.'' -- \href{https://en.wikipedia.org/wiki/Literature#The_influence_of_religious_texts}{Wikipedia\texttt{/}literature\texttt{/}the influence of religious texts}

\subsection{Types of literature}

\subsubsection{Poetry}
``\href{https://en.wikipedia.org/wiki/Poetry}{Poetry} has traditionally been distinguished from \href{https://en.wikipedia.org/wiki/Prose}{prose} by its greater use of the \href{https://en.wikipedia.org/wiki/Aesthetics}{aesthetic} qualities of language, including \href{https://en.wikipedia.org/wiki/Music}{musical} devices such as \href{https://en.wikipedia.org/wiki/Assonance}{assonance}, \href{https://en.wikipedia.org/wiki/Alliteration}{alliteration}, \href{https://en.wikipedia.org/wiki/Rhyme}{rhyme}, \& \href{https://en.wikipedia.org/wiki/Rhythm}{rhythm}, \& by being set in \href{https://en.wikipedia.org/wiki/Line_(poetry)}{lines} \& \href{https://en.wikipedia.org/wiki/Verse_(poetry)}{verses} rather than paragraphs, \& more recently its use of other \href{https://en.wikipedia.org/wiki/Typography}{typographical} elements. This distinction is complicated by various hybrid forms such as \href{https://en.wikipedia.org/wiki/Sound_poetry}{sound poetry}, \href{https://en.wikipedia.org/wiki/Concrete_poetry}{concrete poetry} \& \href{https://en.wikipedia.org/wiki/Prose_poem}{prose poem}, \& more generally by the fact that prose possesses rhythm. Abram Lipsky refers to it as an ``open secret'' that ``prose is not distinguished from poetry by lack of rhythm''.

Prior to the 19th century, poetry was commonly understood to be something set in metrical lines: ``any kind of subject consisting of Rhythm or Verses''. Possibly as a result of \href{https://en.wikipedia.org/wiki/Aristotle}{Aristotle}'s influence (his \href{https://en.wikipedia.org/wiki/Poetics_(Aristotle)}{\textit{Poetics}}), ``poetry'' before the 19th century was usually less a technical designation for verse than a normative category of fictive or rhetorical art. As a form it may pre-date \href{https://en.wikipedia.org/wiki/Literacy}{literacy}, with the earliest works being composed within \& sustained by an oral tradition; hence it constitutes the earliest example of literature.'' -- \href{https://en.wikipedia.org/wiki/Literature#Poetry}{Wikipedia\texttt{/}literature\texttt{/}types of literature\texttt{/}poetry}

\subsubsection{Prose}
``As noted above, \href{https://en.wikipedia.org/wiki/Prose}{prose} generally makes far less use of the aesthetic qualities of language than poetry. However, developments in modern literature, including \href{https://en.wikipedia.org/wiki/Free_verse}{free verse} \& \href{https://en.wikipedia.org/wiki/Prose_poetry}{prose poetry} have tended to blur the differences, \& American poet \href{https://en.wikipedia.org/wiki/T.S._Eliot}{T.S. Eliot} suggested that while: ``the distinction between \href{https://en.wikipedia.org/wiki/Verse_(poetry)}{verse} \& prose is clear, the distinction between \href{https://en.wikipedia.org/wiki/Poetry}{poetry} is obscure''. There are \href{https://en.wikipedia.org/wiki/Verse_novel}{verse novels}, a type of narrative poetry in which a novel-length narrative is told through the medium of poetry rather than prose. \href{https://en.wikipedia.org/wiki/Eugene_Onegin}{\textit{Eugene Onegin}} (1831) by \href{https://en.wikipedia.org/wiki/Alexander_Pushkin}{Alexander Pushkin} is the most famous example.

On the historical development of prose, Richard Graff notes that ``[In the case of \href{https://en.wikipedia.org/wiki/Ancient_Greece}{ancient Greece}] recent scholarship has emphasized the fact that formal prose was a comparatively late development, an ``invention'' properly associated with the \href{https://en.wikipedia.org/wiki/Classical_antiquity}{classical period}''.

\href{https://en.wikipedia.org/wiki/Latin}{Latin} was a major influence on the development of prose in many European countries. Especially important was the great Roman orator \href{https://en.wikipedia.org/wiki/Cicero}{Cicero}. It was the \textit{lingua franca} among literate Europeans until quite recent times, \& the great works of \href{https://en.wikipedia.org/wiki/Descartes}{Descartes} (1596--1650), \href{https://en.wikipedia.org/wiki/Francis_Bacon}{Francis Bacon} (1561--1626), \& \href{https://en.wikipedia.org/wiki/Baruch_Spinoza}{Baruch Spinoza} (1632--1677) were published in Latin. Among the last important books written primarily in Latin prose were the works of \href{https://en.wikipedia.org/wiki/Emanuel_Swedenborg}{Swedenborg} (d. 1772), \href{https://en.wikipedia.org/wiki/Carl_Linnaeus}{Linnaeus} (d. 1778), \href{https://en.wikipedia.org/wiki/Leonhard_Euler}{Euler} (d. 1783), \href{https://en.wikipedia.org/wiki/Carl_Friedrich_Gauss}{Gauss} (d. 1855), \& \href{https://en.wikipedia.org/wiki/Isaac_Newton}{Isaac Newton} (d. 1727).

\paragraph{Novel.} \textsf{Fig. Sculpture in Berlin depicting a stack of books on which are inscribed the names of great German writers.} See also: \href{https://en.wikipedia.org/wiki/Genre_fiction}{Genre fiction}. A \href{https://en.wikipedia.org/wiki/Novel}{novel} is a long \href{https://en.wikipedia.org/wiki/Fictional}{fictional} prose narrative. In English, the term emerged from the \href{https://en.wikipedia.org/wiki/Romance_language}{Romance languages} in the late 15th century, with the meaning of ``news''; it came to indicate something new, without a distinction between fact or fiction. The romance is a closely related long prose narrative. \href{https://en.wikipedia.org/wiki/Walter_Scott}{Walter Scott} defined it as ``a fictitious narrative in prose or verse; the interest of which turns upon marvelous \& uncommon incidents'', whereas in the novel ``the events are accommodated to the ordinary train of human events \& the modern state of society''. Other European languages do not distinguish between romance \& novel: ``a novel is \textit{le roman, der Roman, il romanzo}'', indicates the proximity of the forms.

Although there are many historical prototypes, so-called ``novels before the novel'', the modern novel form emerges late in cultural history -- roughly during the 18th century. Initially subject to much criticism, the novel has acquired a dominant position amongst literary forms, both popularly \& critically.

\paragraph{Novella.} The publisher \href{https://en.wikipedia.org/wiki/Melville_House_Publishing}{Melville House} classifies the \href{https://en.wikipedia.org/wiki/Novella}{novella} as ``too short to be a novel, too long to be a short story''. Publishers \& literary award societies typically consider a novella to be between 17,000 \& 40,000 words.

\paragraph{Short story.} A dilemma in defining the ``\href{https://en.wikipedia.org/wiki/Short_story}{short story}'' as a literary form is how to, or whether one should, distinguish it from any short narrative \& its contested origin, that include the \href{https://en.wikipedia.org/wiki/Bible}{Bible}, \& \href{https://en.wikipedia.org/wiki/Edgar_Allan_Poe}{Edgar Allan Poe}.

\paragraph{Graphic novel.} \href{https://en.wikipedia.org/wiki/Graphic_novel}{Graphic novels} \& \href{https://en.wikipedia.org/wiki/Comic_book}{comic books} present stories told in a combination of artwork, dialogue, \& text.

\paragraph{Electronic literature.} \href{https://en.wikipedia.org/wiki/Electronic_literature}{Electronic literature} is a literary genre consisting of works created exclusively on \& for \href{https://en.wikipedia.org/wiki/Digital_devices}{digital devices}.

\paragraph{Nonfiction.} Common literary examples of \href{https://en.wikipedia.org/wiki/Nonfiction}{nonfiction} include, the \href{https://en.wikipedia.org/wiki/Essay}{essay}; \href{https://en.wikipedia.org/wiki/Travel_literature}{travel literature} \& \href{https://en.wikipedia.org/wiki/Nature_writing}{nature writing}; \href{https://en.wikipedia.org/wiki/Biography}{biography}, \href{https://en.wikipedia.org/wiki/Autobiography}{autobiography} \& \href{https://en.wikipedia.org/wiki/Memoir}{memoir}; \href{https://en.wikipedia.org/wiki/Journalism}{journalism}; \href{https://en.wikipedia.org/wiki/Letter_(message)}{letters}; \href{https://en.wikipedia.org/wiki/Diary}{journals}; history, \href{https://en.wikipedia.org/wiki/Philosophy_and_literature#Philosophical_writing_as_literature}{philosophy}, economics; \href{https://en.wikipedia.org/wiki/Scientific_writing}{scientific}, \& \href{https://en.wikipedia.org/wiki/Technical_writing}{technical} writings.

Nonfiction can fall within the broad category of literature as ``any collection of written work'', but some works fall within the narrower definition ``by virtue of the excellence of their writing, their originality \& their general aesthetic \& artistic merits''.'' -- \href{https://en.wikipedia.org/wiki/Literature#Prose}{Wikipedia\texttt{/}literature\texttt{/}types of literature\texttt{/}prose}

\subsubsection{Drama}
\textsf{Fig. Cover of a 1921 libretto for \href{https://en.wikipedia.org/wiki/Umberto_Giordano}{Giordano}'s opera \href{https://en.wikipedia.org/wiki/Andrea_Chenier}{\textit{Andrea Ch\'enier}}.}

``\href{https://en.wikipedia.org/wiki/Drama}{Drama} is literature intended for \href{https://en.wikipedia.org/wiki/Performance}{performance}. The form is combined with music \& dance in \href{https://en.wikipedia.org/wiki/Opera}{opera} \& \href{https://en.wikipedia.org/wiki/Musical_theatre}{musical theater} (see \href{https://en.wikipedia.org/wiki/Libretto}{libretto}). A \href{https://en.wikipedia.org/wiki/Play_(theatre)}{play} is a written dramatic work by a \href{https://en.wikipedia.org/wiki/Playwright}{playwright} that is intended for performance in a theater; it comprises chiefly \href{https://en.wikipedia.org/wiki/Dialogue}{dialogue} between \href{https://en.wikipedia.org/wiki/Fictional_character}{characters}. A \href{https://en.wikipedia.org/wiki/Closet_drama}{closet drama}, by contrast, is written to be read rather than to be performed; the meaning of which can be realized fully on the page. Nearly all drama took verse form until comparatively recently.

The earliest form of which there exists substantial knowledge is \href{https://en.wikipedia.org/wiki/Greek_theatre}{Greek drama}. This developed as a performance associated with \href{https://en.wikipedia.org/wiki/Religion}{religious} \& civic \href{https://en.wikipedia.org/wiki/Festival}{festivals}, typically enacting for developing upon well-known \href{https://en.wikipedia.org/wiki/History}{historical}, or \href{https://en.wikipedia.org/wiki/Mythology}{mythological} themes.

In the 20th century \href{https://en.wikipedia.org/wiki/Screenplay}{scripts} written for non-stage media have been added to this form, including \href{https://en.wikipedia.org/wiki/Radio_drama}{radio}, television \& film.'' -- \href{https://en.wikipedia.org/wiki/Literature#Drama}{Wikipedia\texttt{/}literature\texttt{/}types of literature\texttt{/}drama}

\subsection{Law}

\subsubsection{Law \& literature}
``The \href{https://en.wikipedia.org/wiki/Law_and_literature}{law \& literature} movement focuses on the interdisciplinary connection between law \& literature.''\\-- \href{https://en.wikipedia.org/wiki/Literature#Law_and_literature}{Wikipedia\texttt{/}literature\texttt{/}law\texttt{/}law \& literature}

\subsubsection{Copyright}
\textsf{Fig. The \href{https://en.wikipedia.org/wiki/Palais_Bourbon\#The_Library}{Library} of the \href{https://en.wikipedia.org/wiki/Palais_Bourbon}{Palais Bourbon} in Paris.}

``Further information: \href{https://en.wikipedia.org/wiki/History_of_copyright}{Wikipedia\texttt{/}history of copyright}. \href{https://en.wikipedia.org/wiki/Copyright}{Copyright} is a type of \href{https://en.wikipedia.org/wiki/Intellectual_property}{intellectual property} that gives its owner the exclusive right to make copies of a \href{https://en.wikipedia.org/wiki/Creative_work}{creative work}, usually for a limited time. The creative work may be in a literary, artistic, educational, or musical form. Copyright is intended to protect the original expression of an idea in the form of a creative work, but not the idea itself.

\paragraph{United Kingdom.} Literary works have been protected by copyright law from unauthorized reproduction since at least 1710. Literary works are defined by copyright law to mean ``any work, other than a dramatic or musical work, which is written, spoken or sung, \& accordingly includes
\begin{itemize}
	\item[(a)] a table or compilation (other than a database),
	\item[(b)] a computer program,
	\item[(c)] preparatory design material for a computer program, \&
	\item[(d)] a database.''
\end{itemize}
Literary works are all works of literature; that is all works expressed in print or writing (other than dramatic or musical works).

\paragraph{United States.} The \href{https://en.wikipedia.org/wiki/Copyright_law_of_the_United_States}{copyright law of the United States} has a long \& complicated history, dating back to colonial times. It was established as federal law with the Copyright Act of 1790. This act was updated many times, including a \href{https://en.wikipedia.org/wiki/Copyright_Act_of_1976}{major revision in 1976}.

\paragraph{European Union.} The \href{https://en.wikipedia.org/wiki/Copyright_law_of_the_European_Union}{copyright law of the European Union} is the copyright law applicable within the \href{https://en.wikipedia.org/wiki/European_Union}{European Union}. Copyright law is largely harmonized in the Union, although country to country differences exist. The body of law was implemented in the EU through a number of \href{https://en.wikipedia.org/wiki/Directive_(European_Union)}{directives}, which the member states need to enact into their national law. The main copyright directives are the \href{https://en.wikipedia.org/wiki/Copyright_Term_Directive}{Copyright Term Directive}, the \href{https://en.wikipedia.org/wiki/Information_Society_Directive}{Information Society Directive} \& the \href{https://en.wikipedia.org/wiki/Directive_on_Copyright_in_the_Digital_Single_Market}{Directive on Copyright in the Digital Single Market}. Copyright in the Union is furthermore dependent on international conventions to which the European Union is a member (such as the \href{https://en.wikipedia.org/wiki/TRIPS_Agreement}{TRIPS Agreement} \& conventions to which all Member States are parties (such as the \href{https://en.wikipedia.org/wiki/Berne\_Convention}{Berne Convention})).

\paragraph{Copyright in communist countries.} Further information: \href{https://en.wikipedia.org/wiki/Copyright_in_Russia}{Wikipedia\texttt{/}copyright in Russia}, \href{https://en.wikipedia.org/wiki/Copyright_law_of_the_Soviet_Union}{Wikipedia\texttt{/}copyright law of the Soviet Union}, \& \href{https://en.wikipedia.org/wiki/Intellectual_property_in_China}{Wikipedia\texttt{/}intellectual property in China}.

\paragraph{Copyright in Japan.} \href{https://en.wikipedia.org/wiki/Copyright_in_Japan}{Japan} was a party to the original \href{https://en.wikipedia.org/wiki/Berne_convention}{Berne convention} in 1899, so its copyright law is in sync with most international regulations. The convention protected copyrighted works for 50 years after the author's death (or 50 years after publication for unknown authors \& corporations). However, in 2004 Japan extended the copyright term to 70 years for cinematographic works. At the end of 2018, as a result of the \href{https://en.wikipedia.org/wiki/Trans-Pacific_Partnership}{Trans-Pacific Partnership} negotiations, the 70 year term was applied to all works. This new term is not applied retroactively; works that had entered the public domain between 1999 \& 2018 by expiration would remain in the public domain.'' -- \href{https://en.wikipedia.org/wiki/Literature#Copyright}{Wikipedia\texttt{/}literature\texttt{/}law\texttt{/}copyright}

\subsubsection{Censorship}
\textsf{Fig. \href{https://en.wikipedia.org/wiki/Soviet}{Soviet} poet \href{https://en.wikipedia.org/wiki/Anna_Akhmatova}{Anna Akhmatova} (1922), whose works were condemned \& censored by the \href{https://en.wikipedia.org/wiki/Stalin}{Stalinist} authorities.}

``Further information: \href{https://en.wikipedia.org/wiki/Book_censorship}{Wikipedia\texttt{/}book censorship}, \href{https://en.wikipedia.org/wiki/Theatre_censorship}{Wikipedia\texttt{/}theater censorship}, \& \href{https://en.wikipedia.org/wiki/Film_censorship}{Wikipedia\texttt{/}film censorship}. Is a means employed by states, religious organizations, educational institutions, etc., to control what can be portrayed, spoken, performed, or written. Generally such bodies attempt to ban works for \href{https://en.wikipedia.org/wiki/Sedition}{poligical reasons}, or because they deal with other controversial matters such as race, or \href{https://en.wikipedia.org/wiki/Obscenity}{sex}.

A notorious example of censorship is \href{https://en.wikipedia.org/wiki/James_Joyce}{James Joyce}'s novel \href{https://en.wikipedia.org/wiki/Ulysses_(novel)}{Ulysses}, which has been described by Russian-American novelist \href{https://en.wikipedia.org/wiki/Vladimir_Nabokov}{Vladimir Nabokov} as a ``divine work of art'' \& the greatest masterpiece of 20th century prose. It was \href{https://en.wikipedia.org/wiki/Obscenity_trial_of_Ulysses_in_The_Little_Review}{banned in the United States from 1921 until 1933} on the grounds of obscenity. Nowadays it is a central literary text in English literature courses, throughout the world.'' -- \href{https://en.wikipedia.org/wiki/Literature#Censorship}{Wikipedia\texttt{/}literature\texttt{/}law\texttt{/}censorship}

\subsection{Awards}
``There are numerous \href{https://en.wikipedia.org/wiki/Literary_award}{awards} recognizing achievement \& contribution in literature. Given the diversity of the field, awards are typically limited in scope, usually on: form, \href{https://en.wikipedia.org/wiki/Genre}{genre}, language, nationality \& output (e.g., for 1st-time writers or \href{https://en.wikipedia.org/wiki/Debut_novel}{debut novels}).

The \href{https://en.wikipedia.org/wiki/Nobel_Prize_in_Literature}{Nobel Prize in Literature} was 1 of the 6 \href{https://en.wikipedia.org/wiki/Nobel_Prizes}{Nobel Prizes} established by the will of \href{https://en.wikipedia.org/wiki/Alfred_Nobel}{Alfred Nobel} in 1895, \& is awarded to an author on the basis of their body of work, rather than to, or for, a particular work itself. Other literary prizes for which all nationalities are eligible include: the \href{https://en.wikipedia.org/wiki/Neustadt_International_Prize_for_Literature}{Neustadt International Prize for Literature}, the \href{https://en.wikipedia.org/wiki/Man_Booker_International_Prize}{Man Booker International Prize}, \href{https://en.wikipedia.org/wiki/Pulitzer_Prize_for_Drama}{Pulitzer Prize}, \href{https://en.wikipedia.org/wiki/Hugo_Award_for_Best_Novel}{Hugo Award}, \href{https://en.wikipedia.org/wiki/Guardian_First_Book_Award}{Guardian 1st Book Award} \& the \href{https://en.wikipedia.org/wiki/Franz_Kafka_Prize}{Franz Kafka Prize}.'' -- \href{https://en.wikipedia.org/wiki/Literature#Awards}{Wikipedia\texttt{/}literature\texttt{/}awards}

%-----------------------------------------------------------------------------%

\section{\href{https://en.wikipedia.org/wiki/Neverland}{Wikipedia\texttt{/}Neverland}}
\textbf{Neverland.} \href{https://en.wikipedia.org/wiki/Peter_and_Wendy}{Peter Pan} location. \textsf{Fig. Peter Pan playing the \href{https://en.wikipedia.org/wiki/Pan_flute}{pipes}, with Neverland in the background (illustration by \href{https://en.wikipedia.org/wiki/Francis_Donkin_Bedford}{F. D. Bedford}, \href{https://en.wikipedia.org/wiki/Peter_and_Wendy}{Peter \& Wendy} (1911)).}
\begin{itemize}
	\item \textbf{1st appearance.} \href{https://en.wikipedia.org/wiki/Peter_and_Wendy}{Peter Pan or the Boy Who Would Not Grow Up} (1904)
	\item \textbf{Created by.} \href{https://en.wikipedia.org/wiki/J._M._Barrie}{J. M. Barrie}
	\item \textbf{Other name(s).} Never Never Land
	\item \textbf{Type.} Fictional imaginary island
	\item \textbf{Locations.} Neverwood, Mermaids' Lagoon, Marooners' Rock
	\item \textbf{Characters.} \href{https://en.wikipedia.org/wiki/Peter_Pan}{Peter Pan}, \href{https://en.wikipedia.org/wiki/Captain_Hook}{Captain Hook}, \href{https://en.wikipedia.org/wiki/Tinker_Bell}{Tinker Bell}, \href{https://en.wikipedia.org/wiki/Tiger_Lily_(Peter_Pan)}{Tiger Lily}
	\item \textbf{Population.} \href{https://en.wikipedia.org/wiki/Lost_Boys_(Peter_Pan)}{Lost Boys}, pirates, Fairies, Native Americans
\end{itemize}
``\textit{Neverland} is a fictional island featured in the works of \href{https://en.wikipedia.org/wiki/J._M._Barrie}{J. M. Barrie} \& those based on them. It is an imaginary faraway place where \href{https://en.wikipedia.org/wiki/Peter_Pan}{Peter Pan}, \href{https://en.wikipedia.org/wiki/Tinker_Bell}{Tinker Bell}, \href{https://en.wikipedia.org/wiki/Captain_Hook}{Captain Hook}, the \href{https://en.wikipedia.org/wiki/Lost_Boys_(Peter_Pan)}{Lost Boys}, \& some other imaginary beings \& creatures live.

Although not all people who come to Neverland cease to age, its best-known resident famously refused to grow up. Thus, the term is often used as a metaphor for \href{https://en.wikipedia.org/wiki/Peter_Pan_syndrome}{eternal childhood} (\& childishness), as well as \href{https://en.wikipedia.org/wiki/Immortality}{immortality} \& \href{https://en.wikipedia.org/wiki/Escapism}{escapism}.

The concept was 1st introduced as ``the \textit{Never Never Land}'' in Barrie's \href{https://en.wikipedia.org/wiki/Theatre\_play}{theater play} \href{https://en.wikipedia.org/wiki/Peter_Pan,_or_The_Boy_Who_Wouldn%27t_Grow_Up}{Peter Pan, or The Boy Who Wouldn't Grow Up}, 1st staged in 1904. In the earliest drafts of the play, the island was called ``\textit{Peter's Never Never Never Land}'', a name possibly influenced by the `\href{https://en.wikipedia.org/wiki/Never_Never_(Australian_outback)}{Never Never}', a contemporary term for \href{https://en.wikipedia.org/wiki/Outback}{outback} Australia. In the 1928 published version of the play's script, the name was shortened to ``the Never Land''. Although the caption to 1 of \href{https://en.wikipedia.org/wiki/Francis_Donkin_Bedford}{F. D. Bedford}'s illustrations also calls it ``The Never Never Land,'' Barrie's 1911 novelization \href{https://en.wikipedia.org/wiki/Peter_and_Wendy}{\textit{Peter \& Wendy}} simply refers to it as ``the Neverland,'' \& its many variations ``the Neverlands.''

Neverland has been featured prominently in subsequent works that either adapted Barrie's works or expanded upon them. These Neverlands sometimes vary in nature from the original.'' -- \href{https://en.wikipedia.org/wiki/Neverland}{Wikipedia\texttt{/}Neverland}

\subsection{Description}

\subsubsection{Location}
``The novel says that the Neverlands are compact enough that adventures are never far between, \& that a map of a child's mind would resemble a map of Neverland, with no boundaries at all. Accordingly, Barrie explains that the Neverlands are found in the minds of children; although each is ``always more or less an island'' as well as having a family resemblance, they are not the same from 1 child to the next. E.g., \href{https://en.wikipedia.org/wiki/John_Darling_(Peter_Pan)}{John Darling}'s Neverland had ``a lagoon with flamingos flying over it,'' while his little brother \href{https://en.wikipedia.org/wiki/Michael_Darling_(Peter_Pan)}{Michael}'s had ``a flamingo with lagoons flying over it.''

The exact situation of Neverland is ambiguous \& vague. In Barrie's original tale, the name for the real world is the Mainland, which suggests Neverland is a small island, reached by flight. Peter -- who is described as saying ``anything that came into his head'' -- tells Wendy the way to Neverland is ``2nd to the right, \& straight on till morning.'' In the novel, the children are said to have found the island only because it was ``out looking for them.'' Barrie additionally writes that Neverland is near the ``stars of the milky way'' \& it is reached ``always at the time of sunrise.''

In Barrie's \href{https://en.wikipedia.org/wiki/Peter_Pan_in_Kensington_Gardens}{\textit{Peter Pan in Kensington Gardens}} (1906), a proto-version of Neverland, located in the \href{https://en.wikipedia.org/wiki/The_Serpentine}{Serpentine} in \href{https://en.wikipedia.org/wiki/Kensington_Gardens}{Kensington Gardens}, is called the \textbf{Birds' Island}, where baby Peter reaches by flight, or by sailing in a paper boat or \href{https://en.wikipedia.org/wiki/Thrush_(bird)}{thrush}'s nest.

\href{https://en.wikipedia.org/wiki/Walt_Disney}{Walt Disney}'s 1953 \href{https://en.wikipedia.org/wiki/Peter_Pan_(1953_film)}{\textit{Peter Pan}} suggests Neverland is located in outer space, adding a ``star'' to Peter's directions: ``2nd star to the right, \& straight on till morning.'' From afar, these stars depict Neverland in the distance. The \href{https://en.wikipedia.org/wiki/Peter_Pan_(2003_film)}{2003 live-action film} (produced by \href{https://en.wikipedia.org/wiki/Universal_Pictures}{Universal Pictures}, \href{https://en.wikipedia.org/wiki/Columbia_Pictures}{Columbia Pictures}, \href{https://en.wikipedia.org/wiki/Revolution_Studios}{Revolution Studios}, \href{https://en.wikipedia.org/wiki/Red_Wagon_Entertainment}{Red Wagon Entertainment} \& \href{https://en.wikipedia.org/wiki/Allied_Stars_Ltd}{Allied Stars Ltd}) repeats this representation, as the Darling children are flown through the solar system to reach Neverland. In the 1991 film \href{https://en.wikipedia.org/wiki/Hook_(film)}{\textit{Hook}} (produced by \href{https://en.wikipedia.org/wiki/TriStar_Pictures}{TriStar Pictures} \& \href{https://en.wikipedia.org/wiki/Amblin_Entertainment}{Amblin Entertainment}), Neverland is shown to be located in the same way as the 1953 Disney film. While flying is the only way to reach it, the film does not show exactly how Captain Hook manages to get from Neverland to London in order to kidnap Peter's children, Jack \& Maggie. In \href{https://en.wikipedia.org/wiki/Peter_Pan_in_Scarlet}{\textit{Peter Pan in Scarlet}} (2006), by \href{https://en.wikipedia.org/wiki/Geraldine_McCaughrean}{Geraldine McCaughrean}, Neverland is located in waters known as the `Sea of One Thousand Islands'. The children get to the island by flying on a road called \textit{the High Way}.

In \href{https://en.wikipedia.org/wiki/Peter_David}{Peter David}'s 2009 novel \textit{Tigerheart}, Neverland is renamed the \textbf{Anyplace} \& is described as being both a physical place \& a dream land where human adults \& children go when they dream. Additionally, there is a location called the \textbf{Noplace} which is cold \& devoid of color where people in a coma \& those who are ``lost'' live. In the 2011 miniseries \href{https://en.wikipedia.org/wiki/Neverland_(miniseries)}{\textit{Neverland}}, inspired by Barrie's works, the titular place is said to be another planet existing at the center of the universe. It is accessible only via a magic portal generated by a strange sphere. In the 2015 American film \href{https://en.wikipedia.org/wiki/Pan_(2015_film)}{\textit{Pan}}, Neverland is a floating island in a sky-like dimension.'' -- \href{https://en.wikipedia.org/wiki/Neverland#Location}{Wikipedia\texttt{/}Neverland\texttt{/}description\texttt{/}location}

\subsubsection{Time}
``The passage of time in Neverland is similarly ambiguous. The novel \href{https://en.wikipedia.org/wiki/Peter_and_Wendy}{\textit{Peter \& Wendy}} mentions that in Neverland there are many more suns \& moons than on the Mainland, making time difficult to track. 1 way to tell the time is to find the crocodile, \& wait until the clock inside it strikes the hour. Although Neverland is widely thought of as a place where children don't grow up, Barrie wrote that the \href{https://en.wikipedia.org/wiki/Lost_Boys_(Peter_Pan)}{Lost Boys} eventually do grow up, having to leave, \& fairies there lived typically short lifespans.

In \href{https://en.wikipedia.org/wiki/Peter_Pan_in_Scarlet}{\textit{Peter Pan in Scarlet}} (2006), by \href{https://en.wikipedia.org/wiki/Geraldine_McCaughrean}{Geraldine McCaughrean}, time freezes as soon as the children arrived in Neverland. In the 2011 miniseries \href{https://en.wikipedia.org/wiki/Neverland_(miniseries)}{\textit{Neverland}}, in which Neverland is said to be another planet entirely, time has frozen due to external cosmic forces converging on the planet, preventing anyone living there from ageing.'' -- \href{https://en.wikipedia.org/wiki/Neverland#Time}{Wikipedia\texttt{/}Neverland\texttt{/}description\texttt{/}time}

\subsection{Locations within Neverland}

\subsubsection{Canon}
``In J. M. Barrie's play \& novel, most of the adventures in the stories take place in the \textbf{Neverwood}, where the \href{https://en.wikipedia.org/wiki/Lost_Boys_(Peter_Pan)}{Lost Boys} hunt \& fight the pirates \& Native Americans.

Peter \& the Lost Boys live in the \textbf{Home Under The Ground}, which also contains Tinker Bell's ``private apartment.'' The Home is accessed by sliding down hollowed tree trunks, one for each boy.
\begin{quotation}
	It consisted of 1 large room, $\ldots$ with a floor in which you could dig if you wanted to go fishing, \& in this floor grew stout mushrooms of a charming color, which were used as stools. A Never tree tried hard to grow in the center of the room, but every morning they sawed the trunk through, level with the floor.''
\end{quotation}
The \textbf{Little House} is built from branches by the Lost Boys for Wendy after she is hit by \href{https://en.wikipedia.org/wiki/Tootles}{Tootles}' arrow. At the end of the play, 1 year after the main events in the story, the house appears in different spots every night, but always on some tree-tops. The Little House is the original ``\href{https://en.wikipedia.org/wiki/Wendy_house}{Wendy house},'' now the name of a children's playhouse.

The \textbf{Jolly Roger} is the pirates' \href{https://en.wikipedia.org/wiki/Brig}{brig}, described by Barrie as ``a rakish-looking craft foul to the hull.''

The mermaids live in the \textbf{Mermaids' Lagoon}, which is also the location of \textbf{Marooners' Rock}, the most dangerous place in Neverland. Trapped on Marooners' Rock in the lagoon just offshore, Peter faced impending death by drowning, as he could not swim or fly from it to safety. The mermaids made no attempt to rescue him, but he was saved by the Never bird.'' -- \href{https://en.wikipedia.org/wiki/Neverland#Canon}{Wikipedia\texttt{/}Neverland\texttt{/}locations within Neverland\texttt{/}canon}

\subsubsection{Non-canon}
``In the many film, television, \& video game adaptations of \textit{Peter Pan}, adventures that originally take place in either the Mermaids' Lagoon, the Neverwood forest, or on the pirates' ship are played out in a greater number of more elaborate locations.

\paragraph{Disney.} \textsf{Fig. Map of Neverland created by \href{https://en.wikipedia.org/wiki/Walt_Disney_Productions}{Walt Disney Productions} as a promotion for its 1953 film \href{https://en.wikipedia.org/wiki/Peter_Pan_(1953_film)}{\textit{Peter Pan}}. Users of \href{https://en.wikipedia.org/wiki/Colgate-Palmolive}{Colgate-Palmolive}'s ``Peter Pan Beauty Bar with Chlorophyll'' could obtain the map by mailing in 3 soap wrappers \& 15 cents.}

In the \href{https://en.wikipedia.org/wiki/Peter_Pan_(franchise)}{Disney-franchise version} of Neverland, many non-canon locales are added which appear variously throughout different installments, as well as adding or giving names to implied locations within Barrie's original Neverland. These locales include:
\begin{itemize}
	\item \textbf{Cannibal Cove\texttt{/}Tiki Forest} -- A jungle environment, original to the Disney franchise, filled with  monkeys, parrots, boars, cobras, bees \& a ``host of evil traps.'' It is occupied by a tribe reminiscent of both African \& indigenous Pacific-Islander cultures. This location appears regularly in \href{https://en.wikipedia.org/wiki/Disney_Channel}{Disney Channel}'s animated series \href{https://en.wikipedia.org/wiki/Jake_and_the_Neverland_Pirates}{Jake \& the Neverland Pirates}.
	\item The \textbf{Neverseas} are the seas around Neverland in Disney's \href{https://en.wikipedia.org/wiki/Tinker_Bell_(film_series)}{\textit{Tinker Bell} films}. Some small islands can be found in it, \& it seems that it can communicate with the real seas, as a normal ship comes across the path of a young James Hook in \href{https://en.wikipedia.org/wiki/The_Pirate_Fairy}{\textit{The Pirate Fairy}}.
	\item \href{https://en.wikipedia.org/wiki/Pixie_Hollow}{\textbf{Pixie Hollow}} is where Tinker Bell \& her tiny fairy friends live \& dwell in Disney's \href{https://en.wikipedia.org/wiki/Disney_Fairies}{\textit{Tinker Bell} franchise}.
	\item \textbf{Never Land Plains} -- A location where the Indians reside.
	\item \textbf{Skull Rock} -- A location where the ``pirates are said to hide their booty.''
	\item \textbf{Crocodile Creek} -- A swamp environment where the Crocodile lives.
\end{itemize}

\paragraph{\textit{Hook} (1991).} In \href{https://en.wikipedia.org/wiki/Steven_Spielberg}{Steven Spielberg}'s 1991 film \href{https://en.wikipedia.org/wiki/Hook_(film)}{\textit{Hook}}, the pirates occupy a small port town peppered with merchant shopfronts, warehouses, hotels, pubs, \& an improvised baseball field, \& many ships \& boats of varying sizes \& kinds fill the harbor. The Home Underground has also been replaced by an intricate tree house structure, which is home to a larger number of Lost Boys. In certain areas, the territory surrounding the tree house has its own unique weather (i.e. spring, summer, autumn, winter).

The Mermaids' Lagoon is directly connected to the Lost Boys' tree house structure by a giant clam-shell pulley system. The Home Underground is discovered buried \& forgotten by an adult Peter in the film, underneath the new home of the Lost Boys. Neither the redskins nor their territory appear in the film, though they are mentioned by Hook during a conversation with Smee.

\paragraph{Other.} The \textbf{Black Castle}, which is referred to in \href{https://en.wikipedia.org/wiki/Peter_Pan_(2003_film)}{the 2003 film}, is an old ruined \& abandoned castle, decorated with stone \href{https://en.wikipedia.org/wiki/Dragon}{dragons} \& \href{https://en.wikipedia.org/wiki/Gargoyle}{gargoyles}. It is 1 of the places where \href{https://en.wikipedia.org/wiki/Tiger_Lily_(Peter_Pan)}{Tiger Lily} is taken by Captain James Hook. This sequence is based on the Marooners' Rock sequence in the original play \& book: like Disney's non-canon `Skull Rock', Black Castle replaces Marooners's Rock in this film.

\textbf{Neverpeak Mountain} is the huge mountain that is right in the middle of Neverland. According to \href{https://en.wikipedia.org/wiki/Peter_Pan_in_Scarlet}{\textit{Peter Pan in Scarlet}}, when a child is on top of Neverpeak Mountain, he or she can see over anyone \& anything \& can see beyond belief.

\textbf{The Maze of Regrets} is a maze in \textit{Peter Pan in Scarlet} where all the mothers of the Lost Boys go to find their boys.'' -- \href{https://en.wikipedia.org/wiki/Neverland#Non-canon}{Wikipedia\texttt{/}Neverland\texttt{/}locations within Neverland\texttt{/}non-canon}

\subsection{Inhabitants}

\subsubsection{Fairies}
``See also: \href{https://en.wikipedia.org/wiki/Tinker_Bell}{Tinker Bell} \& \href{https://en.wikipedia.org/wiki/Disney_Fairies}{Disney Fairies}. \href{https://en.wikipedia.org/wiki/Fairy}{Fairies} are arguably the most important magical inhabitants of the Neverland, \& its primary \href{https://en.wikipedia.org/wiki/Magic_(paranormal)}{magic} users. A property of their nature is the production \& possession of fairy dust, the magic material which enables flying for all characters except Peter, who was taught to fly by the birds, \& later by the fairies in Kensington Gardens. The only-named fairy is \href{https://en.wikipedia.org/wiki/Tinker_Bell}{Tinker Bell}, \href{https://en.wikipedia.org/wiki/Peter_Pan}{Peter Pan}'s companion, whose name alludes to her profession as a `\href{https://en.wikipedia.org/wiki/Tinker}{tinker}', or fixer of pots \& pans. Tinker Bell is essentially a household fairy, but far from benign. Her exotic, fiery nature, \& capacity for evil \& mischief, due to fairies being too small to feel more than 1 type of emotion at any 1 time, is reminiscent of the more hostile fairies encountered by Peter in Kensington Gardens.

In Barrie's play \& novel, the roles of fairies are brief: they are allies to the \href{https://en.wikipedia.org/wiki/Lost_Boys_(Peter_Pan)}{Lost Boys} against the pirates, the source of fairy dust \& where they act as ``guides'' for parties traveling to \& from Neverland. They are also responsible for the collection of abandoned or lost babies from the Mainland to the Neverland. The roles \& activities of the fairies are more elaborate in \href{https://en.wikipedia.org/wiki/Peter_Pan_in_Kensington_Gardens}{\textit{Peter Pan in Kensington Gardens}} (1906): they occupy kingdoms in the Gardens \& at night ``mischief children who are locked in after dark'' to their deaths or entertain them before they return to their parents the following day; \& they guard the paths to a ``Proto-Neverland'' called the \textit{birds' island}. These fairies are more regal \& engage in a variety of human activities in a magical fashion. They have courts; can grant wishes to children; \& have a practical relationship with the birds, which is however ``strained by differences.'' They are portrayed as dangerous, whimsical \& extremely clever but quite \href{https://en.wikipedia.org/wiki/Hedonism}{hedonistic}. After forgetting how to fly, unable to be taught by the birds, Peter is given the power to fly again by the fairies.

Barrie writes that ``when the 1st baby laughed for the 1st time, its laugh broke into a thousand pieces, $\ldots$ \& that was the beginning of fairies.'' Neverland's fairies can be killed whenever someone says they don't believe in fairies, suggesting that the race of fairies is finite \& exhaustible. When dying from Hook's poison, Tinker Bell is saved when Peter \& other children \& adults across the Neverlands \& Mainland call out ``I do believe in fairies, I do, I do,'' so their deaths are not necessarily permanent. At the end of Barrie's novel Wendy asks Peter about Tinker Bell, whom he has forgotten \& he answers, ``I expect she is no more.''

The \href{https://en.wikipedia.org/wiki/Disney_Fairies}{\textit{Disney Fairies}}--\href{https://en.wikipedia.org/wiki/Peter_Pan_(franchise)}{Peter Pan} franchise has elaborated on aspects of Barrie's fairy mythology. The \textbf{Never Fairies} (\& associated sparrow men) live in \href{https://en.wikipedia.org/wiki/Pixie_Hollow}{Pixie Hollow}, located in the heart of Neverland. As stated in the \href{https://en.wikipedia.org/wiki/Tinker_Bell_(film)}{\textit{Tinker Bell} film}, after the baby's 1st laugh enters a flower, it breaks the flower into numerous pieces (the seeds), any piece that can blow with the wind \& survive the trip to Pixie Hollow becomes a fairy, who then learns his\texttt{/}her specific talent.'' -- \href{https://en.wikipedia.org/wiki/Neverland#Fairies}{Wikipedia\texttt{/}Neverland\texttt{/}inhabitants\texttt{/}fairies}

\subsubsection{Birds}
``In the novel \& the play, between the flight from the Mainland (reality) \& the Neverland, they are relatively simple animals which provide entertainment, instruction \& some limited guidance to flyers. These birds are described as unable to sight its shores, ``even, carrying maps \& consulting them at windy corners.''

The \textbf{Never Bird} saves Peter from drowning when he is stranded on Marooners' Rock, by giving him her nest which he uses as a sailing vessel.

In Barrie's \href{https://en.wikipedia.org/wiki/Peter_Pan_in_Kensington_Gardens}{\textit{Peter Pan in Kensington Gardens}}, birds have a far more prominent role on a proto-Neverland called the Birds' Island. On the island, the various birds speak bird-language, described as being related to fairy language which can be understood by young humans, who used to be birds. The birds are responsible for bringing human babies into the Mainland, whose human parents send folded paper boats along the serpentine ``with `boy' or `girl' \& `thin' or `fat' (\& so on) written'', indicating to the official birds which species to send back to transform into human children, who are described as having an ``itch on their backs where their wings used to be'' \& that their warbles are fairy\texttt{/}bird talk.'' -- \href{https://en.wikipedia.org/wiki/Neverland#Birds}{Wikipedia\texttt{/}Neverland\texttt{/}inhabitants\texttt{/}birds}

\subsubsection{Lost Boys}
``Main article: \href{https://en.wikipedia.org/wiki/Lost_Boys_(Peter_Pan)}{Lost Boys (Peter Pan)}. The \textbf{Lost Boys} are a tribe of ``children who fall out of their prams when the nurse is not looking;'' having not been claimed by humans in 7 days, they were collected by the fairies \& flown to the Neverland. There are no `lost girls' because, as Peter explains, girls are much too clever to fall out of their prams \& be lost in this manner.

There are 6 Lost Boys: Tootles, Nibs, Slightly, Curly \& the Twins. They are not permitted to fly by Peter, as it is a sign of his authority \& uniqueness. They live in \href{https://en.wikipedia.org/wiki/Tree_house}{tree houses} \& \href{https://en.wikipedia.org/wiki/Cave}{caves}, wear animal skins, hear spears \& bows \& arrows, \& live for adventure. They are a formidable fighting force despite their youth \& they make war with the pirates, although they seem to enjoy a harmonious existence with the other inhabitants of Neverland.'' -- \href{https://en.wikipedia.org/wiki/Neverland#Lost_Boys}{Wikipedia\texttt{/}Neverland\texttt{/}inhabitants\texttt{/}Lost Boys}

\subsubsection{Pirates}
``The crew of the \href{https://en.wikipedia.org/wiki/Pirate_ship}{Pirate ship} \textit{Jolly Roger} have taken up residence off-shore, \& are widely feared throughout Neverland. How they came to be in Neverland is unclear. Their captain is the ruthless \href{https://en.wikipedia.org/wiki/Captain_Hook}{James Hook}, named after the hook in place of his right hand.'' -- \href{https://en.wikipedia.org/wiki/Neverland#Pirates}{Wikipedia\texttt{/}Neverland\texttt{/}inhabitants\texttt{/}pirates}

\subsubsection{``Redskins''}
``There is a tribe of \href{https://en.wikipedia.org/wiki/Wigwam}{wigwam}-dwelling \href{https://en.wikipedia.org/wiki/Native_Americans_in_the_United_States}{Native Americans} who live on the island, referred to by Barrie as ``\href{https://en.wikipedia.org/wiki/Redskin}{Redskin}'' or as the \href{https://en.wikipedia.org/wiki/Pickaninny}{Piccaninny} tribe. Their \href{https://en.wikipedia.org/wiki/Tribal_chief}{chief} is Great Big Little Panther, whose daughter \href{https://en.wikipedia.org/wiki/Tiger_Lily_(Peter_Pan)}{Tiger Lily} has a crush on Peter Pan. The Piccaninny tribe are known to make ferocious \& deadly war against Captain Hook \& his pirates, but their connection with the Lost Boys is more lighthearted. For ``many moons'' the 2 groups have captured each other, only to promptly release the captives, as though it were a game.'' -- \href{https://en.wikipedia.org/wiki/Neverland#%22Redskins%22}{Wikipedia\texttt{/}Neverland\texttt{/}inhabitants\texttt{/}``Redskins''}

\subsubsection{Mermaids}
``\href{https://en.wikipedia.org/wiki/Mermaid}{Mermaids} live in the \href{https://en.wikipedia.org/wiki/Lagoon}{lagoon}. They enjoy the company of \href{https://en.wikipedia.org/wiki/Peter_Pan}{Peter Pan} but keep their distance from everyone else on the island, including the fairies. They are not sociable creatures \& do not speak nor interact with outsiders. They are malevolent, hedonistic \& frivolous; yet they sing \& play ``mermaid games'' in which they ``rise to the surface in extraordinary numbers to play with their bubbles,'' ``made in \href{https://en.wikipedia.org/wiki/Rainbow}{rainbow} water.'' They also ``love to bash out on Marooners' Rock, combing their hair in a lazy way.''

At 1st glance, Wendy is enchanted by their beauty, but finds them vain \& irritating, as they would ``splash her with their tails, not accidentally, but intentionally'' when she attempted to steal a closer look. Their homes are ``coral caves underneath the waves'' to which they retire at sunset \& rising tide, as well as in anticipation of storms. When 1 mermaid tries to pull Wendy into the water \& drown her, Peter intervenes \& hisses -- rather than crows -- at them \& they quickly dive into the water \& disappear. Barrie describes the mermaids' ``haunting'' transformation at the ``turn of the moon'' while ``uttering strange wailing cries'' at night as the lagoon becomes a very ``dangerous place for mortals''. The Mermaids' Lagoon is a favorite ``adventure'' for the children, \& where they take their ``midday meal''. Peter gives Wendy 1 of the mermaid's combs as a gift.

The \href{https://en.wikipedia.org/wiki/Peter_Pan_(2003_film)}{2003 \textit{Peter Pan film}} briefly describes mermaids as different from those in traditional story books, but as ``dark creatures in touch with all things mysterious,'' \& who will drown humans who get too close, but do not harm Peter who seems to be the only one who can speak the mermaid's language. They always seem to know Hook's whereabouts on the island at any given time \& tell Peter.'' -- \href{https://en.wikipedia.org/wiki/Neverland#Mermaids}{Wikipedia\texttt{/}Neverland\texttt{/}inhabitants\texttt{/}mermaids}

\subsubsection{Animals}
``Animals (referred to as \textbf{beasts}) live throughout Neverland, such as bears, tigers, lions, wolves, flamingos \& crocodiles. In Barrie's original novel, these ``beasts'' hunt the Piccaninny tribe, who hunt the Pirates, who are themselves hunting the Lost Boys, who in turn hunt the beasts, creating a chain of prey \& murder in the Neverland that only ends when 1 party stops or slows down, or when peter redirects the Lost Boys to other tasks \& activities. Like all the agencies of the Neverland, the animals do not need to eat, nor are they eaten when killed, nor do they reproduce (as they enjoy the same immortality as all other inhabitants), so their presence is a paradox. There are also a variety of birds, whose societies are present in the proto-Neverland described in Barrie's \href{https://en.wikipedia.org/wiki/Peter_Pan_in_Kensington_Gardens}{\textit{Peter Pan in Kensington Gardens}}.'' - \href{https://en.wikipedia.org/wiki/Neverland#Animals}{Wikipedia\texttt{/}Neverland\texttt{/}inhabitants\texttt{/}animals}

\subsubsection{Other residents}
``Other inhabitants of Neverland are suggested by Barrie in his original novel, such as a ``small old lady with a hooked nose,'' ``\href{https://en.wikipedia.org/wiki/Gnome}{gnomes} who are mostly tailors,'' \& \href{https://en.wikipedia.org/wiki/Prince}{princes} ``with 6 elder brothers'' -- reminiscent of European \href{https://en.wikipedia.org/wiki/Fairy_tale}{fairy tales}. There are also some briefly described locations without inhabitants, but the narrator hints at their former presence, such as a ``hut fast going to decay.''

In the 1989 Japanese anime series, \href{https://en.wikipedia.org/wiki/Peter_Pan:_The_Animated_Series}{\textit{The Adventures of Peter Pan}}, the individual characters of the pirates, ``redskins,'' \& mermaids are explained, \& new characters such as the \href{https://en.wikipedia.org/wiki/Schizophrenia}{schizophrenic} spellcaster princess Luna \& the \href{https://en.wikipedia.org/wiki/Witchcraft}{witch} Sinistra are added.'' -- \href{https://en.wikipedia.org/wiki/Neverland#Other_residents}{Wikipedia\texttt{/}Neverland\texttt{/}inhabitants\texttt{/}other residents}

%-----------------------------------------------------------------------------%

\section{\href{https://en.wikipedia.org/wiki/Persuasive_writing}{Wikipedia\texttt{/}Persuasive Writing}}
``\textit{Persuasive writing} is any written communication with the intention to convince or influence readers to believe in an idea or opinion \& to do an action. Many writings such as \href{https://en.wikipedia.org/wiki/Criticism}{criticisms}, \href{https://en.wikipedia.org/wiki/Review}{reviews}, reactions papers, \href{https://en.wikipedia.org/wiki/Editorial}{editorials}, proposals, \href{https://en.wikipedia.org/wiki/Advertising}{advertisements}, \& \href{https://en.wikipedia.org/wiki/Brochure}{brochures} use different ways of \href{https://en.wikipedia.org/wiki/Persuasion}{persuasion} to influence readers. Persuasive writing can also be used in \href{https://en.wikipedia.org/wiki/Indoctrination}{indoctrination}.

It is a form of \href{https://en.wikipedia.org/wiki/Non-fiction}{non-fiction} writing the writer uses to develop \href{https://en.wikipedia.org/wiki/Argument}{logical arguments}, making use of carefully chosen words \& phrases. But, it's believed that some literature rooted in a \href{https://en.wikipedia.org/wiki/Fiction}{fiction} genre could also be intended as persuasive writings.

E.g., a 3-part science-fiction novel series named DEUS MODUS, according to its author Lux Cutus Umbra, was originally intended to persuade the general public to inform themselves \& become more actively involved in the supervised development of Artificial Intelligence.'' -- \href{https://en.wikipedia.org/wiki/Persuasive_writing}{Wikipedia\texttt{/}persuasive writing}

\subsection{Common techniques in persuasive writing}
``Presenting strong \href{https://en.wikipedia.org/wiki/Evidence}{evidence}, such as \href{https://en.wikipedia.org/wiki/Fact}{facts} \& \href{https://en.wikipedia.org/wiki/Statistics}{statistics}, statements of expert authorities, \& research findings, establishes \href{https://en.wikipedia.org/wiki/Credibility}{credibility} \& \href{https://en.wikipedia.org/wiki/Authenticity_(philosophy)}{authenticity}. Readers will more likely be convinced to side with the writer's position or agree with their opinion if it is backed up by verifiable evidence. Concrete, relevant, \& reasonable examples or anecdotes can enhance the writer's idea or opinion. They can be based on observations or from the writer's personal experience. Accurate, current, \& balanced information adds to the credibility of persuasive writing. The writer does not only present evidence that favors their ideas, but they also acknowledge some evidence that opposes their own -- this has been proven in psychology to have the greatest influence upon the reader. In the writing, though, their ideas would be sounder.'' -- \href{https://en.wikipedia.org/wiki/Persuasive_writing#Common_techniques_in_persuasive_writing}{Wikipedia\texttt{/}persuasive writing\texttt{/}common techniques in persuasive writing}

\subsection{Ethos, Logos, \& Pathos}
``There are 3 \href{https://en.wikipedia.org/wiki/Aesthetics}{aesthetic features} to persuasive writing. \href{https://en.wikipedia.org/wiki/Ethos}{Ethos} is the appeal to credibility. It convinces the audience of the credibility of the writer. The writer's expertise on their subject matter lends to such credibility. The level of education \& profession of the writer also come into play. \href{https://en.wikipedia.org/wiki/Logos}{Logos} is the appeal to \href{https://en.wikipedia.org/wiki/Logic}{logic} \& reason. It is the most commonly accepted mode in persuasion because it aims to be scientific in its approach to argumentation. In writing, facts are presented logically \& faulty logic is avoided. \href{https://en.wikipedia.org/wiki/Pathos}{Pathos} is the appeal to \href{https://en.wikipedia.org/wiki/Emotion}{emotion}. This aims to convince the audience by appealing to human emotions. Emotions such as \href{https://en.wikipedia.org/wiki/Sympathy}{sympathy}, \href{https://en.wikipedia.org/wiki/Anger}{anger}, \& \href{https://en.wikipedia.org/wiki/Sadness}{sadness} motivate humans; using pathos will get the audience emotionally invested in the subject of the writing.'' -- \href{https://en.wikipedia.org/wiki/Persuasive_writing#Ethos,_Logos,_and_Pathos}{Wikipedia\texttt{/}persuasive writing\texttt{/}ethos, logos, \& pathos}

%-----------------------------------------------------------------------------%

\section{\href{https://en.wikipedia.org/wiki/Wonderland_(fictional_country)}{Wikipedia\texttt{/}Wonderland (fictional country)}}
\textbf{Wonderland.} \href{https://en.wikipedia.org/wiki/Alice%27s_Adventures_in_Wonderland}{Alice's Adventures in Wonderland} location. \textsf{Fig. The royal garden in Wonderland.} Coat of Arms of Wonderland.
\begin{itemize}
	\item \textbf{1st appearance.} \href{https://en.wikipedia.org/wiki/Alice%27s_Adventures_in_Wonderland}{\textit{Alice's Adventures in Wonderland}}
	\item \textbf{Created by.} \href{https://en.wikipedia.org/wiki/Lewis_Carroll}{Lewis Carroll}
	\item \textbf{Genre.} Fantasy
	\item \textbf{Other name(s).} Underland
	\item \textbf{Type.} Monarchy
	\item \textbf{Ruler.} \href{https://en.wikipedia.org/wiki/Queen_of_Hearts_(Alice%27s_Adventures_in_Wonderland)}{Queen of Hearts}
	\item \textbf{Locations.} \href{https://en.wikipedia.org/wiki/Burrow}{Rabbit hole}, March Hare's house, Queen's Croquet Ground
	\item \textbf{Characters.} \href{https://en.wikipedia.org/wiki/White_Rabbit}{White Rabbit}, \href{https://en.wikipedia.org/wiki/Duchess_(Alice%27s_Adventures_in_Wonderland)}{Duchess}, \href{https://en.wikipedia.org/wiki/Cheshire_Cat}{Cheshire Cat}, \href{https://en.wikipedia.org/wiki/The_Hatter}{the Hatter}, \href{https://en.wikipedia.org/wiki/March_Hare}{March Hare}, \href{https://en.wikipedia.org/wiki/Mock_Turtle}{Mock Turtle}, \href{https://en.wikipedia.org/wiki/Queen_of_Hearts_(Alice%27s_Adventures_in_Wonderland)}{Queen of Hearts}
\end{itemize}
\textit{Wonderland} is the setting for \href{https://en.wikipedia.org/wiki/Lewis_Carroll}{Lewis Carroll}'s 1865 children's novel \href{https://en.wikipedia.org/wiki/Alice%27s_Adventures_in_Wonderland}{\textit{Alice's Adventures in Wonderland}}.

\subsection{Geography}
``In the story, Wonderland is accessed by an underground passage, \& Alice reaches it by traveling down a \href{https://en.wikipedia.org/wiki/Burrow}{rabbit hole}. While the location is apparently somewhere beneath \href{https://en.wikipedia.org/wiki/Oxfordshire}{Oxfordshier}, Carroll does not specify how far down it is, \& he has Alice speculate whether it is near the \href{https://en.wikipedia.org/wiki/Center_of_the_Earth}{center of the Earth} or even at the \href{https://en.wikipedia.org/wiki/Antipodes}{Antipodes}.

The land is heavily \href{https://en.wikipedia.org/wiki/Woodland}{wooded} \& grows \href{https://en.wikipedia.org/wiki/Mushroom}{mushrooms}. There are well-kept gardens \& substantial houses, such as those of the \href{https://en.wikipedia.org/wiki/Duchess_(Alice%27s_Adventures_in_Wonderland)}{Duchess} \& the \href{https://en.wikipedia.org/wiki/White_Rabbit}{White Rabbit}. Wonderland has a \href{https://en.wikipedia.org/wiki/Coast}{seacoast}, where the \href{https://en.wikipedia.org/wiki/Mock_Turtle}{Mock Turtle} lives.'' -- \href{https://en.wikipedia.org/wiki/Wonderland_(fictional_country)#Geography}{Wikipedia\texttt{/}Wonderland (fictional country)\texttt{/}geography}

\subsection{Government}
``The land is nominally ruled by the \href{https://en.wikipedia.org/wiki/Queen_of_Hearts_(Alice%27s_Adventures_in_Wonderland)}{Queen of Hearts}, whose whimsical decrees of \href{https://en.wikipedia.org/wiki/Capital_punishment}{capital punishment} are routinely nullified by the \href{https://en.wikipedia.org/wiki/King_of_Hearts_(Alice%27s_Adventures_in_Wonderland)}{King of Hearts}. Other kings \& queens are mentioned as their guests, \& are implied to be the kings \& queens of the other \href{https://en.wikipedia.org/wiki/Playing_card_suit}{card suits}. There is at least 1 Duchess.'' -- \href{https://en.wikipedia.org/wiki/Wonderland_(fictional_country)#Government}{Wikipedia\texttt{/}Wonderland (fictional country)\texttt{/}government}

\subsection{Inhabitants}
``The main population consists of animated \href{https://en.wikipedia.org/wiki/Playing_cards}{playing cards}: the royal family (\href{https://en.wikipedia.org/wiki/Hearts_(suit)}{hearts}), courtiers (\href{https://en.wikipedia.org/wiki/Diamonds_(suit)}{diamonds}), soldiers (\href{https://en.wikipedia.org/wiki/Clubs_(suit)}{clubs}), \& servants (\href{https://en.wikipedia.org/wiki/Spades_(suit)}{spades}). In addition, there are many \href{https://en.wikipedia.org/wiki/Talking_animals_in_fiction}{talking animals}.

Among the characters Alice meets are: \href{https://en.wikipedia.org/wiki/Bill_the_Lizard}{Bill the Lizard}, \href{https://en.wikipedia.org/wiki/Caterpillar_(Alice%27s_Adventures_in_Wonderland)}{Caterpilla}, \href{https://en.wikipedia.org/wiki/Cheshire_Cat}{Cheshire Cat}, \href{https://en.wikipedia.org/wiki/Dodo_(Alice%27s_Adventures_in_Wonderland)}{Dodo}, \href{https://en.wikipedia.org/wiki/Dormouse_(Alice%27s_Adventures_in_Wonderland)}{Dormouse}, \href{https://en.wikipedia.org/wiki/Duchess_(Alice%27s_Adventures_in_Wonderland)}{Duchess}, \href{https://en.wikipedia.org/wiki/Gryphon_(Alice%27s_Adventures_in_Wonderland)}{Gryphon},  \href{https://en.wikipedia.org/wiki/King_of_Hearts_(Alice%27s_Adventures_in_Wonderland)}{king of Hearts}, \href{https://en.wikipedia.org/wiki/Knave_of_Hearts_(Alice%27s_Adventures_in_Wonderland)}{Knave of Hearts}, \href{https://en.wikipedia.org/wiki/Hatter_(Alice%27s_Adventures_in_Wonderland)}{Mad Hatter}, \href{https://en.wikipedia.org/wiki/March_Hare}{March Hare}, \href{https://en.wikipedia.org/wiki/Mock_Turtle}{Mock Turtle}, \href{https://en.wikipedia.org/wiki/Pat_(Alice%27s_Adventures_in_Wonderland)}{Pat}, \href{https://en.wikipedia.org/wiki/Queen_of_Hearts_(Alice%27s_Adventures_in_Wonderland)}{Queen of Hearts}, \href{https://en.wikipedia.org/wiki/White_Rabbit}{White Rabbit}.'' -- \href{https://en.wikipedia.org/wiki/Wonderland_(fictional_country)#Inhabitants}{Wikipedia\texttt{/}Wonderland (fictional country)\texttt{/}inhabitants}

\subsection{In other media}
\begin{itemize}
	\item ``Wonderland is featured in \href{https://en.wikipedia.org/wiki/Walt_Disney}{Walt Disney}'s 1951 animated film \href{https://en.wikipedia.org/wiki/Alice_in_Wonderland_(1951_film)}{\textit{Alice in Wonderland}}.
	\item Wonderland is featured in \href{https://en.wikipedia.org/wiki/Tim_Burton}{Tim Burton}'s 2010 film \href{https://en.wikipedia.org/wiki/Alice_in_Wonderland_(2010_film)}{Alice in Wonderland}. Here, it is actually named \textbf{Underland}; Alice misheard the name as a child, believing it to be ``Wonderland.'' Alice returns to Wonderful when the \href{https://en.wikipedia.org/wiki/White_Queen_(Through_the_Looking-Glass)}{White Queen} is challenging her tyrannical sister, the \href{https://en.wikipedia.org/wiki/Red_Queen_(Through_the_Looking-Glass)}{Red Queen}, for the crown of Underland.
	\item In the 3rd volume of \href{https://en.wikipedia.org/wiki/Captain_Marvel_(DC_Comics)}{\textit{Shazam!}}, the Magiclands location of the Wozenderlands is the result of \href{https://en.wikipedia.org/wiki/Dorothy_Gale}{Dorothy Gale} \& Alice uniting the \href{https://en.wikipedia.org/wiki/Land_of_Oz}{Land of Oz} \& Wonderland against the threat from the Monsterlands.'' -- \href{https://en.wikipedia.org/wiki/Wonderland_(fictional_country)#In_other_media}{Wikipedia\texttt{/}Wonderland (fictional country)\texttt{/}in other media}
\end{itemize}

\subsubsection{\textit{Once Upon a Time}}
``Wonderland is featured in \href{https://en.wikipedia.org/wiki/Once_Upon_a_Time_(TV_series)}{\textit{Once Upon a Time}} \& its spin-off \href{https://en.wikipedia.org/wiki/Once_Upon_a_Time_in_Wonderland}{\textit{Once Upon a Time in Wonderland}}. In this series, there are 2 iterations of Wonderland.

In the 1st iteration, the realm is ruled by the Queen of Hearts, the \href{https://en.wikipedia.org/wiki/Red_King_(Through_the_Looking-Glass)}{Red King} \& Queen, the \href{https://en.wikipedia.org/wiki/White_King_(Through_the_Looking-Glass)}{White King} \& Queen, \& the Caterpillar. Some of the known locations in Wonderland include the Black Forrest (a dark forest where no light shines through) \& its Boro Grove (where those affected by the scent of the perfume flowers are mesmerized \& slowly turned into trees), the Boiling Sea (which is a sea of boiling water), Jafar's Lair (a floating landmass where \href{https://en.wikipedia.org/wiki/Jafar_(Disney)}{Jafar} lives \& keeps his prisoners), Mallow Marsh (a marsh that consists of sticky \href{https://en.wikipedia.org/wiki/Marshmallow}{marshmallo}-like substances), Mimsy Meadows (where Alice \& Cyrus buried Cyrus' lamp until it was excavated by the White Rabbit under the Red Queen's orders), the Outlands (the outskirts of Wonderland where Alice \& Cyrus planted an invisible tent given to Cyrus by the Caterpillar), the Queen of Heart's Palace, Tulgey Woods (a forest where the Mad Hatter's house resides), Underland (which serves as a lair for the Caterpillar \& his Collectors), Whispering Woods (where a deformed man named Grendel resided until he was killed by Jafar), \& Wonderland Castle (where the Red Queen resides).

Not much is known on the 2nd iteration called \textbf{New Wonderland} in the series's 7th season except it is the home of its local Jabberwocky \& has an Infinite Maze.'' -- \href{https://en.wikipedia.org/wiki/Wonderland_(fictional_country)#Once_Upon_a_Time}{Wikipedia\texttt{/}Wonderland (fictional country)\texttt{/}in other media\texttt{/}\textit{Once Upon a Time}}

%-----------------------------------------------------------------------------%

\part{Literary Writings}

\section{Linguistics}
See, e.g., \href{https://en.wikipedia.org/wiki/Linguistics}{Wikipedia\texttt{/}linguistics}\footnote{\textbf{linguistics} [n] [uncountable] the scientific study of language or of particular languages.}.

%-----------------------------------------------------------------------------%

\chapter{William Strunk Jr.,. The Elements of Style}

\paragraph*{Content.} See \href{https://en.wikipedia.org/wiki/The_Elements_of_Style}{Wikipedia\texttt{/}The Elements of Style}. ``Strunk concentrated on the cultivation of good writing \& composition; the original 1918 edition exhorted writers to ``omit needless words'', use the \href{https://en.wikipedia.org/wiki/Active_voice}{active voice}, \& employ \href{https://en.wikipedia.org/wiki/Parallelism_(grammar)}{parallelism} appropriately.'' [$\ldots$] ``The 3rd edition of \textit{The Elements of Style} (1979) features 54 points: a list of common word-usage errors; 11 rules of punctuation \& grammar; 11 principles of writing; 11 matters of form; \&, in Chap. V, 21 reminders for better style. The final reminder, the 21st, ``Prefer the standard to the offbeat\footnote{\textbf{offbeat} [a] [usually before noun] (\textit{informal}) different from what most people expect, \textsc{synonym}: \textbf{unconventional}.}'', is thematically integral\footnote{\textbf{integral} [a] \textbf{1.} being an essential part of something; \textbf{2.} [usually before noun] included as part of something, rather than supplied separately; \textbf{3.} [usually before noun] having all the parts that are necessary for something to be complete.} to the subject of \textit{The Elements of Style}, yet does stand as a discrete\footnote{\textbf{discrete} [a] (\textit{formal or specialist}) independent of other things of the same type, \textsc{synonym}: \textbf{separate}.} essay about writing lucid\footnote{\textbf{lucid} [a] \textbf{1.} clearly expressed; easy to understand, \textsc{synonym}: clear; \textbf{2.} able to think clearly, especially when somebody cannot usually do this.} prose\footnote{\textbf{prose} [n] [uncountable] writing that is not poetry.}. To write well, White advises writers to have the proper\footnote{\textbf{proper} [a] \textbf{1.} [only before noun] (\textit{especially British English}) right, appropriate or correct; according to the rules, \textsc{opposite}: \textbf{improper}; \textbf{2.} [only before noun] \textit{British English}) considered to be real \& of a good enough standard; \textbf{3.} socially \& morally acceptable, \textsc{opposite}: \textbf{improper}; \textbf{4.} [after noun] according to the most exact meaning of the word; \textbf{5.} \textbf{proper to somebody\texttt{/}something} belonging to a particular type of person or thing; natural in a particular situation or place.} mind-set, that they write to please themselves, \& that they aim for ``1 moment of felicity\footnote{\textbf{felicity} [n] \textbf{1.} [uncountable] great happiness; \textbf{2.} [uncountable] the quality of being well chosen or suitable; \textbf{3.} \textbf{felicities} [plural] well-chosen or successful features, especially in a speech or piece of writing.}'', a phrase by \href{https://en.wikipedia.org/wiki/Robert_Louis_Stevenson}{Robert Louis Stevenson}. Thus Strunk's 1918 recommendation:
\begin{quotation}
	``Vigorous\footnote{\textbf{vigorous} [a] \textbf{1.} involving physical strength, effort or energy; \textbf{2.} done with determination, energy or enthusiasm; \textbf{3.} strong \& healthy.} writing is concise\footnote{\textbf{concise} [a] giving only the information that is necessary \& important, using few words.}. A sentence should contain no unnecessary words, a paragraph no unnecessary sentences, for the same reason that a drawing should have no unnecessary lines \& a machine no unnecessary parts. This requires not that the writer make all his sentences short, or that he avoid all detail \& treat his subjects only in outline, but that he make every word tell.'' -- ``Elementary Principles of Composition'', \textit{The Element of Style} \cite{Strunk1918}''
\end{quotation}
[$\ldots$] ``The 4th edition of \textit{The Elements of Style} (2000), published 54 years after Strunk's death, omits his stylistic\footnote{\textbf{stylistic} [a] [only before noun] connected with the style that a writer, artist or musician uses.} advice about masculine\footnote{\textbf{masculine} [a] \textbf{1.} having the qualities or appearance considered to be typical of men; connected with or like men; \textbf{2.} (in some languages) belonging to a class of nouns, pronouns or adjectives that have masculine gender, not feminine or neuter.} pronouns: ``unless the antecedent\footnote{\textbf{antecedent} [n] a thing or an event that exists or comes before something else \& has an influence on it; [a] existing or coming before something else, \& having an influence on it.} is or must be feminine''. In its place, the following sentence has been added: ``many writers find the use of the generic \textit{he} or \textit{his} to rename indefinite antecedents limiting or offensive.'' Further, the retitled entry ``They. He or she'', in Chap. IV: \textit{Misused Words \& Expressions}, advises the writer to avoid an ``unintentional emphasis on the masculine''.'' -- \href{https://en.wikipedia.org/wiki/The_Elements_of_Style#Content}{Wikipedia\texttt{/}The Element of Style\texttt{/}content}

\paragraph*{Reception.} ``\textit{The Elements of Style} was listed as 1 of the 100 best \& most influential\footnote{\textbf{influential} [a] having a lot of influence on the way that somebody\texttt{/}something behaves or develops, or on the way that somebody thinks.} books written in English since 1923 by \textit{Time} in its 2011 list. Upon its release, Charles Poor, writing for \href{https://en.wikipedia.org/wiki/The_New_York_Times}{\textit{The New York Times}}, called it ``a splendid\footnote{\textbf{splendid} [a] (\textit{especially British English}) \textbf{1.} very impressive; very beautiful; \textbf{2.} (\textit{old-fashioned}) excellent; very good, \textsc{synonym}: great.} trophy for all who are interested in reading \& writing.'' American poet \href{https://en.wikipedia.org/wiki/Dorothy_Parker}{Dorothy Parker} has, regarding the book, said:
\begin{quotation}
	``If you have any young friends who aspire to become writers, the 2nd-greatest favor you can do them is to present them with copies of \textit{The Elements of Style}. The 1st-greatest, of course, is to shoot them now, while they're happy.''
\end{quotation}
Criticism\footnote{\textbf{criticism} [n] \textbf{1.} [uncountable, countable] the act of expressing disapproval of somebody\texttt{/}something \& opinions about their faults or bad qualities; a statement showing disapproval; \textbf{2.} [uncountable] the work or activity of analyzing \& making fair, careful judgments about somebody\texttt{/}something, especially books, music, etc.} of \textit{Strunk \& White} has largely focused on claims that it has a \href{https://en.wikipedia.org/wiki/Linguistic_prescriptivism}{prescriptivist}\footnote{\textbf{prescriptive} [a] \textbf{1.} telling people what should be done or how something should be done; \textbf{2.} (\textit{linguistics}) telling people how a language should be used, rather than describing how it is used, \textsc{opposite}: \textbf{descriptive}.} nature, or that it has become a general \href{https://en.wikipedia.org/wiki/Anachronism}{anachronism}\footnote{\textbf{anachronism} [n] \textbf{1.} [countable] a person, a custom or an idea that seems old-fashioned \& does not belong to the present; \textbf{2.} [countable, uncountable] something that is placed, e.g., in a book or play, in the wrong period of history; the fact of placing something in the wrong period of history.} in the face of modern English usage.

In criticizing \textit{The Elements of Style}, \href{https://en.wikipedia.org/wiki/Geoffrey_Pullum}{Geoffrey Pullum}, professor of \href{https://en.wikipedia.org/wiki/Linguistics}{linguistics} at the \href{https://en.wikipedia.org/wiki/University_of_Edinburgh}{University of Edinburgh}, \& co-author of \href{https://en.wikipedia.org/wiki/The_Cambridge_Grammar_of_the_English_Language}{\textit{The Cambridge Grammar of the English Language}} (2002), said that:
\begin{quotation}
	``The book's toxic mix of \href{https://en.wikipedia.org/wiki/Linguistic_purism}{purism}\footnote{\textbf{purism} [n] [uncountable] the belief that things should be done in the traditional way \& that there are correct forms in languages, art, etc. that should be followed.}, \href{https://en.wikipedia.org/wiki/Atavism}{atavism}, \& personal \href{https://en.wikipedia.org/wiki/Eccentricity_(behavior)}{eccentricity}\footnote{\textbf{eccentricity} [n] \textbf{1.} [uncountable] behavior that people think is strange or unusual; the quality of being unusual \& different from other people; \textbf{2.} [countable, usually plural] an unusual act or habit.} is not underpinned\footnote{\textbf{underpin} [v] to support or form the basis of something.} by a proper grounding\footnote{\textbf{grounding} [n] [singular, uncountable] knowledge \& understanding of the basic parts of a subject; a basis for something.} in English grammar. It is often so misguided that the authors appear not to notice their own egregious\footnote{\textbf{egregious} [a] (\textit{formal}) extremely bad.} flouting\footnote{\textbf{flout} [v] \textbf{flout something} to show that you have no respect for a law, etc. by openly not obeying it, \textsc{synonym}: \textbf{defy}.} of its own rules $\ldots$ It's sad. Several generations of college students learned their grammar from the uninformed\footnote{\textbf{uninformed} [a] having or showing a lack of knowledge or information about something, \textsc{opposite}: informed.} bossiness\footnote{\textbf{bossiness} [n] [uncountable] (\textit{disapproving}) bossy behavior.} of \textit{Strunk \& White}, \& the result is a nation of educated people who know they feel vaguely\footnote{\textbf{vaguely} [adv] \textbf{1.} in a way that is not detailed or exact; \textbf{2.} slightly.} anxious\footnote{\textbf{anxious} [a] \textbf{1.} \textbf{anxious (about something)} feeling worried or nervous; \textbf{2.} wanting something very much.} \& insecure\footnote{\textbf{insecure} [a] \textbf{1.} not confident, especially about yourself or your abilities, \textsc{opposite}: \textbf{secure}; \textbf{2.} not safe or protected, \textsc{opposite}: \textbf{secure}.} whenever they write \textit{however} or \textit{than me} or \textit{was} or \textit{which}, but can't tell you why.''
\end{quotation}
Pullum has argued, e.g., that the authors misunderstood what constitutes the \href{https://en.wikipedia.org/wiki/English_passive_voice}{passive voice}\footnote{NQBH: Personally, I prefer the passive voice to the active one.}, \& he criticized their proscription\footnote{\textbf{proscription} [n] [countable, uncountable] (\textit{formal}) \textbf{proscription (against\texttt{/}on something)} the act of saying officially that something is banned; the stat of being banned.} of established \& unproblematic\footnote{\textbf{unproblematic} [a] not having or causing problems, \textsc{opposite}: \textbf{problematic}.} English usages, e.g. the \href{https://en.wikipedia.org/wiki/Split_infinitive}{split infinitive} \& the use of \textit{which} in a restrictive \href{https://en.wikipedia.org/wiki/English_relative_clause#That_or_which}{relative clause}. On \href{https://en.wikipedia.org/wiki/Language_Log}{Language Log}, a blog about language written by \href{https://en.wikipedia.org/wiki/Linguists}{linguists}, he further criticized \textit{The Elements of Style} for promoting \href{https://en.wikipedia.org/wiki/Linguistic_prescriptivism}{linguistic precriptivism} \& \href{https://en.wikipedia.org/wiki/Hypercorrection}{hypercorrection} among \href{https://en.wikipedia.org/wiki/Anglophones}{Anglophones}, \& called it ``the book that ate American's brain''.

\href{https://en.wikipedia.org/wiki/The_Boston_Globe}{\textit{The Boston Globe}}'s review described \textit{The Elements of Style Illustrated} (2005), with illustrations by Maira Kalman, as an ``aging zombie of a book $\ldots$ a hodgepodge\footnote{\textbf{hodgepodge} [n] (\textit{North American English}) (also \textbf{hotchpotch}, \textit{especially in British English}) [singular] (\textit{informal}) a number of things mixed together without any particular order or reason.}, its now-antiquated\footnote{\textbf{antiquated} [a] (\textit{usually disapproving}) (of things or ideas) old-fashioned \& no longer suitable for modern conditions, \textsc{synonym}: \textbf{outdated}.} \href{https://en.wikipedia.org/wiki/Pet_peeve}{pet peeves} jostling for\footnote{\textbf{jostle for} [phrasal verb] \textbf{jostle for something} to compete strongly \& with force for something.} space with 1970s taboos\footnote{\textbf{taboo} [n] \textbf{1.} \textbf{taboo (against\texttt{/}on something)} a cultural or religious custom that does not allow people to do, use or talk about a particular thing; \textbf{2.} \textbf{taboo (against\texttt{/}on something)} a general agreement not to do something or talk about something.} \& 1990s computer advice''.

Nevertheless, many contemporary\footnote{\textbf{contemporary} [a] \textbf{1.} belonging to the present time, \textsc{synonym} \textbf{modern}; \textbf{2.} (especially of people \& society) belonging to the same time as somebody\texttt{/}something else.} authors still recommend it highly. Their praise\footnote{\textbf{praise} [v] \textbf{1.} to express your approval or admiration for somebody\texttt{/}something; \textbf{2.} \textbf{praise God} to express your thanks to or your respect for God.} tends to focus on its characterization\footnote{\textbf{characterization} [n] [uncountable, countable] \textbf{1.} \textbf{characterization (of something)} the process of discovering or describing the qualities or features of something; the result of this process; \textbf{2.} the way in which the characters in a story, play or film are made to seem real.} of \fbox{good writing \& how to achieve it}, grammar being just 1 element of that purpose. In \href{https://en.wikipedia.org/wiki/On_Writing:_A_Memoir_of_the_Craft}{On writing} (2000, p. 11), \href{https://en.wikipedia.org/wiki/Stephen_King}{Stephen King} writes:
\begin{quotation}
	``There is little or no detectable \href{https://en.wikipedia.org/wiki/Bullshit}{bullshit} in that book. (Of course, it's short; at 85 pages it's much shorter than this one.) I'll tell you right now that every aspiring writer should read \textit{The Elements of Style}. Rule 17 in the chapter titled \textit{Principles of Composition} is `Omit needless words.' I will try to do that here.''
\end{quotation}
In 2011, Tim Skern remarked that \textit{The Elements of Style} ``remains the best book available on writing good English.''

In 2013, \href{https://en.wikipedia.org/wiki/Nevile_Gwynne}{Nevile Gwynne} reproduced \textit{The Elements of Style} in his work \href{https://en.wikipedia.org/wiki/Gwynne%27s_Grammar}{\textit{Gwynne's Grammar}}. Britt Peterson of the \href{https://en.wikipedia.org/wiki/Boston_Globe}{\textit{Boston Globe}} wrote that his inclusion of the book was a ``curious\footnote{\textbf{curious} [a] \textbf{1.} having a strong desire to know about something; \textbf{2.} strange \& unusual.} addition''.

In 2016, the Open Syllabus Project lists \textit{The Elements of Style} as the most frequently assigned text in US academic \href{https://en.wikipedia.org/wiki/Syllabus}{syllabuses}, based on an analysis of 933,635 texts appearing in over 1 million syllabuses.'' -- \href{https://en.wikipedia.org/wiki/The_Elements_of_Style#Reception}{Wikipedia\texttt{/}The Elements of Style\texttt{/}reception}

``The 1st writer I watched at work was my stepfather, E. B. White.\footnote{\selectlanguage{vietnamese} Sự ảnh hưởng, đặc biệt đến nhân cách \& việc lựa chọn nghề nghiệp, của những hình mẫu đầu tiên mà ta, 1 cách tình cờ hay được số phận sắp đặt, gặp gỡ trong cuộc đời.} Each Tuesday morning, he would close his study door \& sit down to write the ``Notes \& Comment'' page for \textit{The New Yorker}. The task was familiar to him -- he was required to file a few hundred words of editorial\footnote{\textbf{editorial} [a] [usually before noun] connected with the task of preparing something e.g. a newspaper, a book, or a television or radio programme, to be published or broadcast; [n] an important article in a journal or a newspaper, that expresses the editor's opinion about an issue.} of personal commentary on some topic in or out of the news that week -- but the sounds of his typewriter\footnote{\textbf{typewriter} [n] a machine that produces writing similar to print. It has keys that you press to make metal letters or signs hit a piece of paper through a long, narrow piece of cloth covered with ink ($=$ colored liquid).} \footnote{NQBH: I like the term ``typewriter'' in any literary scene., which sounds traditional \& sexy, opposite to personal notebooks\texttt{/}laptop now: modern \& robust.} from his room came in hesitant\footnote{\textbf{hesitant} [a] slow to speak or act because you feel uncertain, embarrassed or unwilling.} bursts\footnote{\textbf{burst} [v] \textbf{1.} [intransitive, transitive] to break open or apart, especially because of pressure from inside; to make something break in this way; \textbf{2.} [intransitive] \textbf{$+$ adv.\texttt{/}prep.} to go or come from somewhere suddenly; \textbf{burst into something} [phrasal verb] to start producing something suddenly \& with great force; [n] a short period of a particular activity or strong emotion that often starts suddenly.}, with long silences in between. Hours went by. Summoned at last for lunch, he was silent \& preoccupied\footnote{\textbf{preoccupied} [a] thinking \&\texttt{/}or worrying continuously about something so that you do not pay attention to other things.}, \& soon excused himself to get back to the job. When the copy went off at last, in the afternoon RFD pouch\footnote{\textbf{pouch} [n] \textbf{1.} a small bag, usually made of leather, \& often carried in a pocket or attached to a belt; \textbf{2.} a large bag for carrying letters, especially official ones; \textbf{3.} a pocket of skin on the stomach of some female marsupial animals, e.g. kangaroos, in which they carry their young; \textbf{4.} a pocket of skin in the cheeks of some animals, e.g. hamsters, in which they store food.} -- we were in Maine, a day's mail away from New York -- he rarely seemed satisfied. \fbox{``It isn't good enough.''}\footnote{``The quest for perfection can never end.''} he said sometimes, \fbox{``I wish it were better.''}

\fbox{Writing is hard}, even for authors who do it all the time. Less frequent practitioners -- the job applicant; the business executive with an annual report to get out; the high school senior with a Faulkner assignment; the graduate-school student with her thesis proposal; the writer of a letter of condolence\footnote{\textbf{condolence} [n] [countable, usually plural, uncountable] sympathy that you feel for somebody when a person in their family or that they know well has died; an expression of this sympathy.} -- often get stuck in an awkward\footnote{\textbf{awkward} [a] \textbf{1.} embarrassed; making you feel embarrassed; \textbf{2.} difficult to deal with, \textsc{synonym}: \textbf{difficult}; \textbf{3.} not convenient; \textbf{4.} difficult because of its shape or design; \textbf{5.} not moving in an easy way; not comfortable or elegant.} passage or find a muddle\footnote{\textbf{muddle} [v] (\textit{especially British English}) \textbf{1.} to put things in the wrong order or mix them up; \textbf{2.} muddle somebody (up) to confuse somebody; \textbf{3.} muddle somebody\texttt{/}something (up)$|$ \textbf{muddle A (up) with B} to confuse 1 person or thing with another, \textsc{synonym}: \textbf{mix up}.} on their screens, \& then blame themselves. What should be easy \& flowing looks tangled\footnote{\textbf{tangled} [a] \textbf{1.} twisted together in an untidy way; \textbf{2.} complicated, \& not easy to understand.} or feeble\footnote{\textbf{feeble} [a] \textbf{1.} very weak; \textbf{2.} not effective; not showing energy or effort.} or overblown\footnote{\textbf{overblown} [a] \textbf{1.} that is made to seem larger, more impressive or more important than it really is, \textsc{synonym}: \textbf{exaggerated}; \textbf{2.} (of flowers) past the best, most beautiful stage.} -- not what was meant at all. \fbox{What's wrong with me}, each one thinks. \fbox{Why can't I get this right?}''

[$\ldots$] White knew that a compendium\footnote{\textbf{compendium} [n] (plural \textbf{compendia, compendiums}) a collection of facts, drawings \& photographs on a particular subject, especially in a book.} of specific tips -- about singular \& plural verbs, parentheses, the ``that'' -- ``which'' scuffle\footnote{\textbf{scuffle} [n] \textbf{scuffle (with somebody) $|$ scuffle (between A \& B)} a short \& not very violent fight or struggle; [v] \textbf{1.} [intransitive] \textbf{scuffle (with somebody)} (of 2 or more people) to fight or struggle with each other for a short time, in a way that is not very serious; \textbf{2.} [intransitive] \textbf{$+$ adv.\texttt{/}prep.} to move quickly making a quiet rubbing noise.}, \& many others -- could clear up a recalcitrant\footnote{\textbf{recalcitrant} [a] (\textit{formal}) unwilling to obey rules or follow instructions; difficult to control.} sentence or subclause when quickly reconsulted\footnote{\textbf{consult} [v] \textbf{1.} [transitive, intransitive] to discuss something with somebody to get their permission for something, or to help you make a decision; \textbf{2.} [transitive, intransitive] to go to somebody for information or advice, especially an expert e.g. a doctor or lawyer; \textbf{3.} [transitive] \textbf{consult something} to look in or at something to get information, \textsc{synonym}: \textbf{refer to something}.}, \& that the larger principles needed to be kept in plain sight, like a wall sampler.

How simple they look, set down here in White's last chapter: ``\fbox{Write in a way that comes naturally},'' ``\fbox{Revise \& rewrite},'' ``\fbox{Do not explain too much},'' \& the rest; above all, the cleansing\footnote{\textbf{cleanse} [v] \textbf{1.} [transitive, intransitive] \textbf{cleanse (something)} to clean your skin or a wound; \textbf{2.} [transitive] \textbf{cleanse somebody (of\texttt{/}from something}) (\textit{literary}) to take away somebody's guilty feelings or sin.}, clarion\footnote{\textbf{clarion} [n] \textbf{1.} a medieval trumpet with clear shrill tones; \textbf{2.} the sound of or as if of a clarion' [a] brilliantly clear; loud \& clear.} ``Be clear.'' How often I have turned to them, in the book or in my mind, while trying to start or unblock or revise some piece of my own writing! They help -- they really do. They work. They are the way.

E. B. White's prose is celebrated for its ease\footnote{\textbf{ease} [n] [uncountable] \textbf{1.} lack of difficulty or effort, \textsc{opposite}: \textbf{difficulty}; \textbf{2.} the state of feeling relaxed or comfortable, without anxiety, problems or pain.} \& clarity\footnote{\textbf{clarity} [n] [uncountable] \textbf{1.} the quality of being expressed clearly; \textbf{2.} the ability to think about or understand something clearly; \textbf{3.} if a picture, substance or sound has clarity, you can see or hear it very clearly, or see through it easily.} -- just think of \textit{Charlotte's Web} -- but maintaining this standard required endless attention. When the new issue of \textit{The New Yorker} turned up in Maine, I sometimes saw him reading his ``Comment'' piece over to himself, with only a slightly different expression than the one he'd worn on the day it went off. Well, O.K., he seemed to be saying. \fbox{At least I got the elements right.}

This edition has been modestly\footnote{\textbf{modest} [a] \textbf{1.} fairly limited or small in amout; \textbf{2.} not expensive, rich or impressive; \textbf{3.} (of people, especially women, or their clothes) not showing too much of the body; not intended to attract attention, especially in a sexual way; \textbf{4.} (\textit{approving}) not talking much about your own abilities or possessions.} updated, with word processors \& air conditioners making their 1st appearance among White's references, \& with a light redistribution of genders to permit a feminine pronoun or female farmer to take their places among the males who once innocently\footnote{\textbf{innocent} [a] \textbf{1.} not guilty of a crime, etc.; not having done something wrong, \textsc{opposite}: \textbf{guilty}; \textbf{2.} [only before noun] suffering harm or being killed because of a crime, war, etc. although not directly involved in it; \textbf{3.} having little experience of evil or unpleasant things, or of sexual matters; \textbf{4.} not intended to cause harm or upset somebody, \textsc{synonym}: \textbf{harmless}.} served him.'' [$\ldots$] ``What is not here is anything about E-mail -- the rules-free, lower-case flow that cheerfully keeps us in touch these days. E-mail is conversation, \& it may be replacing the sweet \& endless talking we once sustained\footnote{\textbf{sustain} [v] \textbf{1.} \textbf{sustain somebody\texttt{/}something} to provide enough of what somebody\texttt{/}something needs in order to live or exist; \textbf{2.} to make something continue for some time without becoming less, \textsc{synonym}: \textbf{maintain}; \textbf{3.} \textbf{sustain something} (\textit{formal}) to experience something bad, \textsc{synonym}: \textbf{suffer}; \textbf{4.} \textbf{sustain something} to provide evidence to support an opinion, a theory, etc., \textsc{synonym}: \textbf{uphold}; \textbf{5.} \textbf{sustain something} (\textit{law}) to decide that a claim, etc. is valid, \textsc{synonym}: \textbf{uphold}.} (\& tucked away\footnote{\textbf{tuck away} [phrasal verb] \textbf{tuck something $\leftrightarrow$ away} \textbf{1.} \textbf{be tucked away} to be located in a quiet place, where not many people go; \textbf{2.} to hide something somewhere or keep it in a safe place; \textbf{3.} (\textit{British English, informal}) to eat a lot of food.}) within the informal letter. But we are all writers \& readers as well as communicators, with \fbox{the need at times to please \& satisfy ourselves} (as White put it) with the \fbox{clear \& almost perfect thought}.'' -- \cite[\textit{Foreword} by Roger Angell]{Strunk_White2019}

``I [E. B. White] passed the course, graduated from the university, \& \fbox{forgot the book but not the professor}.'' [$\ldots$]

``\textit{The Elements of Style}, when I [E. B. White] reexamined it in 1957, seemed to me to contain \fbox{rich deposits\footnote{\textbf{deposit} [n] \textbf{1.} a layer of a substance that has been left somewhere, especially by a river or flood, or is found at the bottom of a liquid; \textbf{2.} a layer of a substance that has formed naturally underground; \textbf{3.} [usually singular] \textbf{a deposit (on something)} a sum of money that is given as the 1st part of a larger payment; \textbf{4.} (in the British political system) the amount of money that a candidate in an election to Parliament has to pay, \& that is returned if they get enough votes.} of gold}. It was Will Strunk's \textit{parvum opus}\footnote{\textbf{parvum opus} [from Latin] [n] a little work, a small but meaningful work of an artist or writer.}, his attempt to cut the vast tangle\footnote{\textbf{tangle} [n] \textbf{1.} a twisted mass of threads, hair, etc. that cannot be easily separated; \textbf{2.} a lack of order; a confused state; \textbf{3.} (\textit{informal}) a disagreement or fight; [v] [transitive, intransitive] \textbf{tangle (something) up} to twist something into an untidy mass; to become twisted in this way.} of English rhetoric\footnote{\textbf{rhetoric} [n] [uncountable] \textbf{1.} (\textit{often disapproving} speech or writing that is intended to influence people, but that is not completely honest or sincere; \textbf{2.} the skill of using language in speech or writing in a special way that influences or entertains people.)} down to size \& write its rules \& principles on the head of a pin\footnote{\textbf{pin} [n] \textbf{1.} a short thin piece of stiff wire with a sharp point at 1 end \& a round head at the other, used to hold or attach things; \textbf{2.} a short piece of metal or other material, used to hold things together; \textbf{3.} a piece of metal with a sharp point, worn for decoration; \textbf{4.} 1 of the metal parts that stick out of an electric plug \& fit into a socket; [v] \textbf{pin something ($+$ adv.\texttt{/}prep.)} to attach something onto another thing or join things together with a pin, etc.; \textbf{pin something down} [phrasal verb] to explain or understand something exactly.}. Will himself had hung the tag ``little'' on the book; he referred to it sardonically\footnote{\textbf{sardonically} [adv] (\textit{disapproving}) in a way that shows that you think that you are better than other people \& do not take them seriously, \textsc{synonym}: \textbf{mockingly}.} \& with secret pride as ``the \textit{little book},'' always giving the word ``little'' a special twist, as though he were putting a spin on a ball. In its original form, it was a 43 page summation of the case for cleanliness, accuracy\footnote{\textbf{accuracy} [n] \textbf{1.} [uncountable] the state of being exact or correct, \textsc{opposite}: \textbf{inaccuracy}; \textbf{2.} [uncountable, countable] (\textit{specialist}) the degree to which the result of a measurement or calculation matches the correct value or a standard, \textsc{opposite}: \textbf{inaccuracy}.}, \& brevity\footnote{\textbf{brevity} [n] [uncountable] \textbf{1.} the quality of using few words when speaking or writing; \textbf{2.} \textbf{brevity (of something)} the fact of lasting a short time.} in the use of English. Today, 52 years later, its vigor\footnote{\textbf{vigor} [n] [uncountable] \textbf{1.} effort, energy, \& enthusiasm; \textbf{2.} \textbf{vigor (of something)} physical strength; good health.} is unimpaired\footnote{\textbf{unimpaired} [a] (\textit{formal}) not damaged or made less good, \textsc{opposite}: \textbf{impaired}.}, \& for sheer\footnote{\textbf{sheer} [a] \textbf{1.} [only before noun] used to emphasize the size, degree or amount of something; nothing but; \textbf{2.} very steep.} pith\footnote{\textbf{pith} [n] [uncountable] \textbf{1.} a soft dry white substance inside the skin of oranges \& some other fruits; \textbf{2.} the essential or most important part of something.} I think it probably sets a record that is not likely to be broken. Even after I got through tampering with\footnote{\textbf{tamper with} [phrasal verb] \textbf{tamper with something} to make changes to something without permission, especially in order to damage it, \textsc{synonym}: interfere with.} it, it was still a tiny thing, \fbox{a barely tarnished\footnote{\textbf{tarnished} [v] \textbf{1.} [intransitive, transitive] if mental tarnishes or something tarnishes it, it no longer looks bright \& shiny; \textbf{2.} [transitive, often passive] to damage the good opinion people have of somebody\texttt{/}something, \textsc{synonym}: \textbf{taint}; [n] [singular, uncountable] a thin layer on the surface of a metal that makes it look darker \& less bright.} gem\footnote{\textbf{gem} [n] \textbf{1.} (also less frequent \textbf{gemstone}) a precious stone that has been cut \& polished \& is used in jewellery, \textsc{synonym}: \textbf{jewel, precious stone}; \textbf{2.} a person, place or thing that is especially good.}}. 7 rules of usage, 11 principles of composition\footnote{\textbf{composition} [n] \textbf{1.} [uncountable] the different parts that something is made of; the way in which the different parts are organized; \textbf{2.} [countable] a piece of music or a poem; \textbf{3.} [uncountable] the act of writing a piece of music or a poem; \textbf{4.} [uncountable] (\textit{art}) the arrangement of people of objects in a painting, photograph or scene of a film.}, a few matters of form, \& a list of words \& expressions commonly misused -- that was the sum \& substance\footnote{\textbf{substance} [n] \textbf{1.} a type of solid, liquid or gas that has particular qualities; \textbf{2.} [countable] a drug or chemical, especially an illegal one, that has a particular effect on the mind or body; \textbf{3.} [uncountable] the most important or main part of something; \textbf{4.} [uncountable] (\textit{formal}) importance; \textbf{5.} [uncountable] the quality of being based on facts or the truth.} of Prof. Strunk's work. Somewhat audaciously\footnote{\textbf{audaciously} [adv] (\textit{formal}) in a way that shows you are willing to take risks or to do something that shocks people.}, \& in an attempt to give my publisher his money's worth, I [E. B. White] added a chapter called ``An Approach to Style,'' setting forth my own prejudices\footnote{\textbf{prejudice} [n] [uncountable, countable] an unreasonable dislike of a person, group, etc., especially when it is based on their race, religion, sex, etc.}, my notions of error, my articles of faith. This chapter (Chap. V) is addressed particularly to those who feel that English prose composition is not only a necessary skill but a sensible pursuit as well -- a way to spend one's days. I think Prof. Strunk would not object to that.''

[$\ldots$] ``I have now completed a 3rd revision. Chap. IV has been refurbished\footnote{\textbf{refurbish} [v] \textbf{refurbish something} to clean \& decorate a room, building, etc. in order to make it more attractive, more useful, etc.} with words \& expressions of a recent vintage\footnote{\textbf{vintage} [n] \textbf{1.} the wine that was produced in a particular year or place; the year in which it was produced; \textbf{2.} [usually singular] the period or season of gathering grapes for making wine; [a] [only before noun] \textbf{1.} \textbf{vintage} wine is of very good quality \& has been stored for several years; \textbf{2.} (British English) (of a vehicle) made between 1919 \& 1930 \& admired for its style \& interest; \textbf{3.} typical of a period in the past \& of high quality; the best work of the particular person; \textbf{4.} \textbf{vintage year} a particular good \& successful year.}; 4 rules of usage have been added to Chap. I. Fresh examples have been added to some of the rules \& principles, amplification\footnote{\textbf{amplification} [n] [uncountable] \textbf{1.} \textbf{amplification (of something)} the process of increasing the amplitude of an electrical signal; \textbf{2.} (biochemistry) \textbf{amplification (of something)} the process by which many copies of something, e.g. a gene, are made; \textbf{3.} \textbf{amplification (of something)} the action of making something greater or easier to notice; \textbf{4.} the action of adding details to a story, statement, etc.; details added to a story, statement, etc.} has reared\footnote{\textbf{rear} [v] \textbf{1.} \textbf{rear somebody\texttt{/}something} [often passive] to care for young children or animals until they are fully grown, \textsc{synonym}: \textbf{raise}; \textbf{2.} \textbf{rear something} to breed or keep animals or birds, e.g. on a farm; \textbf{something rears its head} [idiom] (of something unpleasant) to appear or happen; [n] (usually \textbf{the rear}) [singular] the back part of something; [a] [only before noun] at or near the back of something.} its head in a few places in the text where I felt an assault\footnote{\textbf{assault} [n] \textbf{1.} [uncountable, countable] the crime of attacking somebody physically; in law, \textbf{assault} is an act that threatens physical harm to somebody, whether or not actual harm is done: \textit{to commit}\texttt{/}\textit{be charged with assault}; \textbf{2.} [countable] (by an army, etc.) the act of attacking somebody\texttt{/}something, \textsc{synonym}: \textbf{attack}; \textbf{3.} [countable, usually singular, uncountable] an act of criticizing or attacking somebody\texttt{/}something severely; [v] \textbf{assault somebody} to attack somebody physically.} could successfully be made on the bastions\footnote{\textbf{bastion} [n] \textbf{1.} (\textit{formal}) a group of people or a system that protects a way of life or a belief when it seems that it may disappear; \textbf{2.} a place that military forces are defending.} of its brevity, \& in general the book has received a thorough overhaul\footnote{\textbf{overhaul} [n] an examination of a machine or system, including doing repairs on it or making changes to it; [v] \textbf{1.} \textbf{overhaul something} to examine every part of a machine, system, etc. \& make any necessary changes or repairs; \textbf{2.} \textbf{overhaul somebody} to come from behind a person you are competing against in a race \& go past them, \textsc{synonym}: \textbf{overtake}.} -- to correct errors, delete bewhiskered\footnote{\textbf{bewhiskered} [a] \textbf{1.} having whiskers; bearded; \textbf{2.} ancient, as a witticism, expression, etc.; pass\'e; hoary.} entries, \& enliven\footnote{\textbf{enliven} [v] (\textit{formal}) \textbf{enliven something} to make something more interesting or more fun.} the argument.

Prof. Strunk was a positive man. His book contains rules of grammar phrased as direct orders. In the main I [E. B. White] have not tried to soften his commands, or modify his pronouncements\footnote{\textbf{pronouncement} [n] a formal public statement.}, or remove the special objects of his scorn\footnote{\textbf{scorn} [n] [uncountable] a strong feeling that somebody\texttt{/}something is stupid or not good enough, usually shown by the way you speak, \textsc{synonym}: \textbf{contempt}; [v] \textbf{1.} \textbf{scorn somebody\texttt{/}something} to feel or show that you think somebody\texttt{/}something is stupid \& you do not respect them or it, \textsc{synonym}: \textbf{dismiss}; \textbf{2.} (\textit{formal}) to refuse to have or do something because you are too proud.}. I have tried, instead, to preserve\footnote{\textbf{preserve} [v] \textbf{1.} \textbf{preserve something} to keep a particular quality or feature; \textbf{2.} to keep something safe from harm, in good condition or in its original state; \textbf{3.} to prevent something from decaying, by treating it in a particular way; [n] [singular] an activity, job or interest that is thought to be suitable for 1 particular person or group of people.} the flavor\footnote{\textbf{flavor} [n] \textbf{1.} [uncountable] \textbf{flavor (of something)} how food or drink tastes, \textsc{synonym}: \textbf{taste}; \textbf{2.} [countable] a particular type of taste; \textbf{3.} [singular] a particular quality or atmosphere; \textbf{4.} [singular] \textbf{a\texttt{/}the flavor of something} an idea of what something is like.} of his discontent\footnote{\textbf{discontent} [n] (also \textbf{discontentment}) \textbf{1.} [uncountable] a feeling of being unhappy because you are not satisfied with a particular situation, \textsc{synonym}: \textbf{dissatisfaction}; \textbf{2.} [countable] \textbf{discontent (of somebody)} a thing that makes you feel unhappy \& not satisfied with a particular situation, \textsc{synonym}: \textbf{dissatisfaction}.} while slightly enlarging the scope of the discussion. \textit{The Elements of Style} does not pretend\footnote{\textbf{pretend} [v] \textbf{1.} to behave in a particular way, in order to make other people believe something that is not true; \textbf{2.} (usually used in negative sentences \& questions) to claim to be, do or have something, especially when this is not true.} to survey\footnote{\textbf{survey} [n] \textbf{1.} \textbf{survey} (of somebody\texttt{/}something) an investigation of the opinions, behavior, etc. of a particular group of people, which is usually done by asking them questions; \textbf{2.} an act of examining \& recording the measurements, features, etc. of an area of land in order to make a map or plan of it; \textbf{3.} \textbf{survey (of something)} a general study, view or description of something; [v] \textbf{1.} \textbf{survey somebody\texttt{/}something} to investigate the opinions or behavior of a group of people by asking them a series of questions; \textbf{2.} \textbf{survey something} to study \& give a general description of something; \textbf{3.} \textbf{survey something} to measure \& record the features of an area of land, e.g. in order to make a map or in preparation for building; \textbf{4.} \textbf{survey something} to look carefully at the whole of something, especially in order to get a general impression of it, \textsc{synonym}: \textbf{inspect}.} the whole field. Rather it proposes\footnote{\textbf{propose} [v] \textbf{1.} to suggest a plan or an idea for people to consider \& decide on; \textbf{2.} to suggest an explanation of something for people to consider.} to give in brief space the principal\footnote{\textbf{principal} [a] [only before noun] main; most important.} requirements of plain\footnote{\textbf{plain} [a] \textbf{1.} easy to see or understand, \textsc{synonym}: \textbf{clear}; \textbf{2.} [only before noun] expressed in a clear \& simple way, without using technical language; \textbf{3.} not trying to deceive anyone; honest \& direct; \textbf{4.} not decorated or complicated; simple; in computing, \textbf{plain text} is data representing text that is not written in code or using special formatting \& can be read, displayed or printed without much processing: \textit{Mathematical formulae are an example of content that cannot be represented satisfactorily via plain text.}; \textbf{5.} without marks or a pattern on it; \textbf{6.} [only before noun] (used for emphasis) simple; nothing but. \textsc{synonym}: \textbf{sheer}.} English style. It concentrates\footnote{\textbf{concentrate} [v] \textbf{1.} [transitive, often passive] \textbf{concentrate something $+$ adv.\texttt{/}prep.} to bring something together in 1 place; \textbf{2.} [intransitive, transitive] to give all your attention to something \& not think about anything else; \textbf{3.} [transitive] \textbf{concentrate something} to increase the strength of a substance by reducing its volume, e.g. by boiling it; \textbf{concentrate on something} [phrasal verb] to spend more time doing 1 particular thing than others; [n] [countable, uncountable] \textbf{concentrate (of something)} a substance that is made stronger because water or other substances have been removed.} on fundamentals\footnote{\textbf{fundamentals} [n] [plural] \textbf{fundamentals (of something)} the basic \& most important parts of something.}: the rules of usage \& principles of composition most commonly violated\footnote{\textbf{violet} [v] \textbf{1.} \textbf{violate something} to go against or refuse to obey a law, an agreement, etc.; \textbf{2.} \textbf{violate something} to not treat something with respect.}.

The reader will soon discover that these rules \& principles are in the form of sharp commands, Sergeant\footnote{\textbf{sergeant} [n] (abbr., \textbf{Sergt, Sgt}) \textbf{1.} a member of 1 of the middle ranks in the army \& the air force, below an officer; \textbf{2.} (in a UK) a police officer just below the rank of an inspector; \textbf{3.} (in the US) a police officer just below the rank of a lieutenant or caption.} Strunk snapping\footnote{\textbf{snap} [v] \textit{break} \textbf{1.} [transitive, intransitive] to break something suddenly with a sharp noise; to be broken in this way; \textit{take photograph} \textbf{2.} [transitive, intransitive] (\textit{informal}) to take a photograph; \textit{open}\texttt{/}\textit{close}\texttt{/}\textit{move into position} \textbf{3.} [intransitive, transitive] to move, or to move something, into a particular position quickly, especially with a sudden sharp noise; \textit{speak impatiently} \textbf{4.} [transitive, intransitive] to speak or say something in an impatient, usually angry, voice; \textit{of animal} \textbf{5.} [intransitive] \textbf{snap (at somebody\texttt{/}something)} to try to bite somebody\texttt{/}something, \textsc{synonym}: \textbf{nip}; \textit{lose control} \textbf{6.} [intransitive] to suddenly be unable to control your feelings any longer because the situation has become too difficult; \textit{fasten clothing} \textbf{7.} [intransitive, transitive] \textbf{snap (something)} (\textit{North American English}) to fasten a piece of clothing with a snap; \textit{in American football} \textbf{8.} [transitive] \textbf{snap something} (\textit{sport}) (in American football) to start play by passing the ball back between your legs.} orders to his platoon\footnote{\textbf{platoon} [n] a small group of soldiers that is part of a company \& commanded by a lieutenant.}. ``Do not join independent clauses with a comma.'' (Rule 5.) ``Do not break sentences in 2.'' (Rule 6.) ``Use the active voice.'' (Rule 14.) ``Omit\footnote{\textbf{omit} [v] \textbf{1.} to not include something\texttt{/}somebody, either deliberately or because you have forgotten it\texttt{/}them, \textsc{synonym}: \textbf{leave somebody\texttt{/}something out (of something)}; \textbf{2.} \textbf{omit to do something} to not do or fail to do something.} needless\footnote{\textbf{needless} [a] (of something bad) not necessary; that could be avoided, \textsc{synonym}: unnecessary.} words.'' (Rule 17.) ``Avoid a succession\footnote{\textbf{succession} [n] \textbf{1.} [countable, usually singular] a number of things or people that follow each other in time or order, \textsc{synonym}: \textbf{series}; \textbf{2.} [uncountable] the act of taking over an official position or title; \textbf{3.} [uncountable] the right to take over an official position or title, especially to become the king or queen of a country.} of loose\footnote{\textbf{loose} [a] \textbf{1.} not firmly fixed where it should be; that can become separated from something; \textbf{2.} not tightly packed together; not solid or hard; \textbf{3.} not strictly organized or controlled; \textbf{4.} not exact; not very careful; \textbf{5.} (of clothes) not fitting closely, \textsc{opposite}: \textbf{tight}; \textbf{6.} not tied together; not held in position by anything or contained in anything; \textbf{7.} (\textit{medical}) (of body waste) having too much liquid in it.} sentences.'' (Rule 18.) ``In summaries, keep to 1 tense.'' (Rule 21.) Each rule or principle is followed by a short hortatory\footnote{\textbf{hortatory} [a] trying to strongly encourage or persuade someone to do something.} essay, \& usually the exhortation\footnote{\textbf{exhortation} [n] [countable, uncountable] (\textit{formal}) \textbf{exhortation (to do something)} an act of trying very hard to persuade somebody to do something.} is followed by, or interlarded\footnote{\textbf{interlard} [v] (used with object) \textbf{1.} to diversify by adding or interjecting something unique, striking, or contrasting (usually followed by \textit{with}); \textbf{2.} (of things) to be intermixed in.} with, examples in parallel columns -- the true vs. the false, the right vs. the wrong, the timid\footnote{\textbf{timid} [a] shy \& nervous; not brave.} vs. the bold, the ragged\footnote{\textbf{ragged} [a] \textbf{1.} (of clothes) old \& torn, \textsc{synonym}: \textbf{shabby}; \textbf{2.} (of people) wearing old or torn clothes; \textbf{3.} having an outline, an edge or a surface that is not straight or even; \textbf{4.} not smooth or regular; not showing control or careful preparation; \textbf{5.} (\textit{informal}) very tired, especially after physical effort.} vs. the trim\footnote{\textbf{trim} [v] \textbf{1.} \textbf{trim something} to make something neater, smaller, better, etc., by cutting parts from it; \textbf{2.} to cut away unnecessary parts from something; \textbf{3.} [usually passive] \textbf{trim something (with something)} to decorate something, especially around its edges.}. From every line there peers out at me the puckish\footnote{\textbf{puckish} [a] [usually before noun] (\textit{literary}) enjoying playing tricks on other people, \textsc{synonym}: \textbf{mischievous}.} face of my professor, his short hair parted neatly\footnote{\textbf{neat} [a] \textbf{1.} in good order; carefully done or arranged; \textbf{2.} simple but clever; \textbf{3.} containing or made out of just 1 substance; not mixed with anything else.} in the middle \& combed down over his forehead, his eyes blinking incessantly\footnote{\textbf{incessantly} [adv] (\textit{usually disapproving}) without stopping, \textsc{synonym}: \textbf{constantly}.} behind steel-rimmed spectacles\footnote{\textbf{spectacle} [n] \textbf{1.} [countable, uncountable] \textbf{spectacle (of something)} a performance or an event that is very impressive \& exciting to look at; \textbf{2.} [singular] \textbf{spectacle (of something)} an unusual, embarrassing or sad sight or situation that attracts a lot of attention; \textbf{3.} (\textbf{spectacles}) [plural] [\textit{formal}] $=$ \textbf{glass}.} as though he had just emerged into strong light, his lips nibbling each other like nervous horses, his smile shuttling to \& fro under a carefully edged mustache.

``Omit needless words!'' cries the author on p. 23, \& into that imperative\footnote{\textbf{imperative} [n] a thing that is very important \& needs immediate attention or action; [a] [not usually before noun] very important \& needing immediate attention or action, \textsc{synonym}: \textbf{vital}.} Will Strunk \fbox{really put his heart \& soul}. In the days when I was sitting in his class, he omitted so many needless words, \& omitted them so forcibly\footnote{\textbf{forcibly} [adv] \textbf{1.} in a way that involves the use of physical force; \textbf{2.} in a way that makes something very clear.} \& with such eagerness\footnote{\textbf{eager} [a] very interested \& excited by something that is going to happen or about something that you want to do, \textsc{synonym}: \textbf{keen}.} \& obvious relish\footnote{\textbf{relish} [v] to get great pleasure from something; to want very much to do or have something, \textsc{synonym}: \textbf{enjoy}; [n] \textbf{1.} [uncountable] great pleasure; \textbf{2.} [uncountable, countable] a cold, thick, spicy sauce made from fruit \& vegetables that have been boiled, that is served with meat, cheese, etc.}, that he often seemed in the position of having shortchanged\footnote{\textbf{short-change} [v] [often passive] \textbf{1.} \textbf{short-change somebody} to give back less than the correct amount of money to somebody who has paid for something with more than the exact price; \textbf{2.} \textbf{short-change somebody} to treat somebody unfairly by not giving them what they have earned or deserve.} himself -- a man left with nothing more to say yet with time to fill, a radio prophet who had outdistanced\footnote{\textbf{outdistance} [v] \textbf{outdistance somebody\texttt{/}something} to leave somebody\texttt{/}something behind by going faster, further, etc.; to be better than somebody\texttt{/}something, \textsc{synonym}: \textbf{outstrip}.} the clock. Will Strunk got out of this predicament\footnote{\textbf{predicament} [n] a difficult or an unpleasant situation, especially one where it is difficult to know what to do, \textsc{synonym}: \textbf{quandary}.} by a simple trick: he uttered\footnote{\textbf{utter} [v] \textbf{utter something} to make a sound with your voice; to say something.} every sentence 3 times. When he delivered his oration\footnote{\textbf{oration} [n] (\textit{formal}) a formal speech made on a public occasion, especially as part of a ceremony.} on brevity to the class, he leaned forward over his desk, grasped his coat lapels\footnote{\textbf{lapel} [n] 1 of the 2 front parts of the top of a coat or jacket that are joined to the collar \& are folded back.} in his hands, \&, in a husky\footnote{\textbf{husky} [a] \textbf{1.} (of a person of their voice) sounding deep, quiet \& rough, sometimes in an attractive way; \textbf{2.} (\textit{North American English}) with a large, strong body; [n] (North American English also \textbf{huskie}) a large strong dog with thick hair, used for pulling sledges across snow.}, conspiratorial\footnote{\textbf{conspiratorial} [a] \textbf{1.} connected with, or making you think of, a conspiracy ($=$ a secret plan to do something illegal); \textbf{2.} (of a person's behavior) suggesting that a secret is being shared.} voice, said, ``Rule 17. Omit needless words! Omit needless words! Omit needless word!''

He was a memorable\footnote{\textbf{memorable} [a] special, good or unusual \& therefore worth remembering; easy to remember.} man, friendly \& funny. Under the remembered sting of his kindly lash\footnote{\textbf{lash} [v] \textbf{1.} [intransitive, transitive] to hit somebody\texttt{/}something with great force, \textsc{synonym}: \textbf{pound}; \textbf{2.} [transitive] \textbf{lash somebody\texttt{/}something} to hit a person or an animal with a whip, rope, stick, etc., \textsc{synonym}: \textbf{beat}.}, I have been trying to omit needless words since 1919, \& although there are still many words that cry for omission \& the huge task will never be accomplished, it is exciting to me to reread to masterly Strunkian elaboration\footnote{\textbf{elaboration} [n] [uncountable, countable] \textbf{1.} the act of explaining or describing something in a more detailed way; \textbf{2.} the process of developing a plan, an idea, etc. \& making it complicated or detailed; \textbf{3.} \textbf{elaboration (of something)} (\textit{biology}) the production of a substance or structure from elements or simpler constituents in a natural process.} of this noble\footnote{\textbf{noble} [a] \textbf{1.} belonging to a family of high social rank, \textsc{synonym}: \textbf{aristocratic}; \textbf{2.} having or showing fine personal qualities that people admire, e.g. courage, honesty \& care for others; [n] a person who comes from a family of high social rank; a member of the nobility, \textsc{synonym}: \textbf{aristocratic}.} theme\footnote{\textbf{theme} [n] the subject of a talk, piece of writing, exhibition, etc.; an idea that keeps returning in a piece of research or a work of art or literature.}. It goes:
\begin{quotation}
	\textit{Vigorous writing is concise. A sentence should contain no unnecessary words, a paragraph no unnecessary sentences, for the same reason that a drawing should have no unnecessary lines \& a machine no unnecessary parts. This requires not that the writer make all sentences short or avoid all detail \& treat subjects only in outline, but that every word tell.}
\end{quotation}
There you have a short, valuable essay on the nature \& beauty of brevity -- 59 words that could change the world. Having recovered from his adventure in prolixity\footnote{\textbf{prolixity} [n] [uncountable] (\textit{formal}) the fact of using too many words \& therefore creating a piece of writing, a speech, etc., that is boring.} (59 words were a lot of words in the tight world of William Strunk Jr.), the professor proceeds to give a few quick lessons in pruning\footnote{\textbf{pruning} [n] [uncountable] \textbf{1.} the activity of cutting off some of the branches from a tree, bush, etc. so that it will grow better \& stronger; \textbf{2.} the act of making something smaller by removing parts; the act of cutting out parts of something.}. Students learn to cut the dead-wood from ``this is a subject that,'' reducing it to ``this subject,'' a saving of 3 words. They learn to trim\footnote{\textbf{trim} [v] \textbf{1.} \textbf{trim something} to make something neater, smaller, better, etc., by cutting parts from it; \textbf{2.} to cut away unnecessary parts from something; \textbf{3.} [usually passive] \textbf{trim something (with something)} to decorate something, especially around its edges.} ``used for fuel purposes'' down to ``used for fuel.'' They learn that they are being chatterboxes\footnote{\textbf{chatterbox} [n] (\textit{informal}) a person who talks a lot, especially a child.} when they say ``the question as to whether'' \& that they should just say ``whether'' -- a saving of 4 words out of a possible 5.

The professor devotes\footnote{\textbf{devote} [v] \textbf{devote yourself to somebody\texttt{/}something} to give most of your time, energy or attention to somebody\texttt{/}something, \textsc{synonym}: \textbf{dedicate}; \textbf{devote something to something}: to give an amount of time, attention or resources to something.} a special paragraph to the vile\footnote{\textbf{vile} [a] \textbf{1.} (\textit{informal}) extremely unpleasant or bad, \textsc{synonym}: \textbf{disgusting}; \textbf{2.} (\textit{formal}) morally bad; completely unacceptable, \textsc{synonym}: \textbf{wicked}.} expression \textit{the fact that}, a phrase that causes him to quiver\footnote{\textbf{quiver} [v] to shake slightly; to make a slight movement, \textsc{synonym}: \textbf{tremble}; [n] \textbf{1.} an emotion that has an effect on your body; a slight movement in part of your body; \textbf{2.} a case for carrying arrows.} with revulsion\footnote{\textbf{revulsion} [n] [uncountable, singular] (\textit{formal}) a strong feeling of horror, \textsc{synonym}: \textbf{disgust, repugnance}.}. The expression, he says, should be ``revised out of every sentence in which it occurs.'' But a shadow\footnote{\textbf{shadow} [n] \textbf{1.} [countable] the dark area or shape produced by somebody\texttt{/}something coming between light \& a surface; \textbf{2.} [uncountable] (\textbf{shadows} [plural]) darkness, especially that produced by somebody\texttt{/}something coming between light \& a surface; \textbf{3.} [singular] the strong (usually bad) influence of somebody\texttt{/}something.} of gloom\footnote{\textbf{gloom} [n] \textbf{1.} [uncountable, singular] a feeling of being sad \& without hope, \textsc{synonym}: \textbf{depression}; \textbf{2.} [uncountable] (\textit{literary}) almost total darkness.} seems to hang over the page, \& you feel that he knows how hopeless his cause is. I suppose I have written \textit{the fact that} a thousand times in the heat of composition, revised it out maybe 500 times in the cool aftermath\footnote{\textbf{aftermath} [n] [usually singular] the situation that exists as a result of an important (\& usually unpleasant) event, especially a war, an accident, etc.}. To be batting only .500 this late in the season, to fail half the time to connect with this fat pitch, saddens me, for it seems a betrayal of the man who showed me how to swing\footnote{\textbf{swing} [v] \textbf{1.} [intransitive, transitive] to change to make somebody\texttt{/}something change from 1 opinion or mood to another; \textbf{2.} [intransitive, transitive] to turn or change direction suddenly; to make something do this; \textbf{3.} [intransitive, transitive] to move backwards or forwards or from side to side while hanging from a fixed point; to make something do this; \textbf{4.} [intransitive, transitive] to move or make something move with a wide curved movement; [n] a change from 1 opinion or situation to another; the amount by which something changes.} at it \& made the swinging seem worthwhile.

I treasure\footnote{\textbf{treasure} [n] \textbf{1.} [uncountable] a collection of valuable things e.g. gold, silver \& jewelery; \textbf{2.} [countable, usually plural] a highly valued object; \textbf{3.} [singular] a person who is much loved or valued; [v] \textbf{treasure something} to have or keep something that you love \& that is extremely valuable to you, \textsc{synonym}: \textbf{cherish}.} \textit{The Elements of Style} for its sharp\footnote{\textbf{sharp} [a] \textbf{1.} [usually before noun] (especially of a change in something) sudden \& fast; \textbf{2.} [usually before noun] (especially of a difference in something) clear \& definite; \textbf{3.} (especially of something that can cut or make a hole in something) having a fine edge or point, \textsc{opposite}: \textbf{blunt}; \textbf{4.} (of a person or what they say) critical or severe; \textbf{5.} (of a physical feeling or an emotion) very strong \& sudden, often like being cut or wounded, \textsc{synonym}: \textbf{intense}; \textbf{6.} changing direction suddenly; \textbf{7.} (of people or their minds or eyes) quick to notice or understand things or to react.} advice, but I treasure it even more for the \fbox{audacity}\footnote{\textbf{audacity} [n] [uncountable] behavior that is brave but likely to shock or offend people, \textsc{synonym}: \textbf{nerve}.} \& self-confidence\footnote{\textbf{self-confidence} [n] [uncountable] confidence in yourself \& your abilities, \textsc{synonym}: \textbf{self-assurance, confidence}.} of its author. \fbox{Will knew where he stood.} He was so sure of where he stood, \& made his position so clear \& so plausible, that his peculiar\footnote{\textbf{peculiar} [a] belonging to or connected with 1 particular place, situation, person, etc., \& not others.} stance\footnote{\textbf{stance} [n] the opinions that somebody has about something \& expresses publicly, \textsc{synonym}: \textbf{position}.} has continued to invigorate\footnote{\textbf{invigorate} [v] \textbf{1.} \textbf{invigorate somebody} to make somebody feel healthy \& full of energy; \textbf{2.} \textbf{invigorate something} to make a situation, an organization, etc. efficient \& successful.} me -- \&, I am sure, thousands of other ex-students -- during the years that have intervened\footnote{\textbf{intervene} [v] \textbf{1.} [intransitive] to become involved in a situation in order to improve it or stop it from getting worse; \textbf{2.} [intransitive] to happen in the time between events; \textbf{3.} [intransitive] to exist or be found in the space between things; \textbf{4.} [intransitive] to happen in a way  that delays something or prevents it from happening.} since our 1st encounter\footnote{\textbf{encounter} [v] \textbf{1.} \textbf{encounter something} to experience something, especially something unpleasant or difficult, while you are trying to do something else, \textsc{synonym}: \textbf{run into something}; \textbf{2.} \textbf{encounter something\texttt{/}somebody} to discover or experience something, or meet somebody, especially something\texttt{/}somebody new, unusual or unexpected, \textsc{synonym}: \textbf{come across somebody\texttt{/}something}; [n] a meeting, especially one that is sudden or unexpected.}. He had a number of likes \& dislikes that were almost as whimsical\footnote{\textbf{whimsical} [a] unusual \& not serious in a way that is either funny or annoying.} as the choice of a necktie, yet he made them seem utterly\footnote{\textbf{utter} [a] [only before noun] used to emphasize how complete something is, \textsc{synonym}: \textbf{total}; [v] \textbf{utter something} to make a sound with your voice; to say something.} convincing. He disliked the word \textit{forceful}\footnote{\textbf{forceful} [a] \textbf{1.} (of people) expressing opinion firmly \& clearly in a way that persuades other people to believe them, \textsc{synonym}: \textbf{assertive}; \textbf{2.} (of opinions, etc.) expressed firmly \& clearly so that other people believe them; \textbf{3.} using force; \textbf{4.} (of action) strong \& effective.} \& advised us to use \textit{forcible}\footnote{\textbf{forcible} [a] [only before noun] involving the use of physical force.} instead. He felt that the word \textit{clever}\footnote{\textbf{clever} [a] \textbf{1.} (especially British English) quick at learning \& understanding things, \textsc{synonym}: \textbf{intelligent}; \textbf{2.} \textbf{clever (at something\texttt{/}doing somethign)} (especially British English) skillful; \textbf{3.} showing intelligence or skill, e.g. in the design of an object, in an idea or somebody's actions.} was greatly overused: ``It is best restricted to ingenuity\footnote{\textbf{ingenuity} [n] [uncountable] the ability to invent things or solve problems in clever new ways, \textsc{synonym}: \textbf{inventiveness}.} displayed in small matters.'' He despised\footnote{\textbf{despise} [v] (not used in the progressive tenses) to dislike \& have no respect for somebody\texttt{/}something.} the expression \textit{student body}, which he termed gruesome\footnote{\textbf{gruesome} [a] very unpleasant \& filling you with horror, usually because it is connected with death or injury.}, \& made a special trip downtown to the \textit{Alumni News} office 1 day to protest\footnote{\textbf{protest} [n] [uncountable, countable] the expression of strong disagreement with or opposition to something; a statement or an action that shows this.} the expression \& suggest that \textit{studentry} be substituted\footnote{\textbf{substitute} [v] [intransitive, transitive] to take the place of somebody\texttt{/}something else; to use somebody\texttt{/}something instead of somebody\texttt{/}something else; [n] a person or thing that you use or have instead of the usual one.} -- a coinage\footnote{\textbf{coinage} [n] \textbf{1.} [uncountable] the coins used in a particular place or at a particular time; coins of a particular type; \textbf{2.} [countable, uncountable] a word or phrase that has been invented recently; the process of inventing a word or phrase.} of his own, which he felt was similar to \textit{citizenry}\footnote{\textbf{citizenry} [n] [singular $+$ singular or plural verb] (\textit{formal}) all the citizens of a particular town, country, etc.}. I am told that the \textit{News} editor was so charmed by the visit, if not by the word, that he ordered the student body buried, never to rise again. \textit{Studentry} has taken its place. It's not much of an improvement, but it does sound less cadaverous\footnote{\textbf{cadaverous} [a] (\textit{literary}) (of a person) extremely pale, thin \& looking ill.}, \& it made Will Strunk quite happy.

Some years ago, when the heir\footnote{\textbf{heir} [n] \textbf{1.} a person who has the legal right to receive somebody's property, money or title when that person dies; \textbf{2.} a person who is thought to continue the work or a tradition started by somebody else.} to the throne of England was a child, I noticed a headline in the \textit{Times} about Bonnie Prince Charlie: ``CHARLES' TONSILS OOUT.'' Immediately Rule 1 leapt to mind.
\begin{quotation}
	\textbf{1.} Form the possessive singular of nouns by adding \textit{'s}. Follow this rule whatever the final consonant\footnote{\textbf{consonant} [n] \textbf{1.} (phonetics) a speech sound made by completely or partly stopping the flow of air being breathed out through the mouth; \textbf{2.} a letter of the alphabet that represents a consonant sound.}. Thus write, \textit{Charles's friend, Burns's poems, the witch's malice\footnote{\textbf{malice} [n] [uncountable] a desire to harm somebody caused by a feeling of hate.}}.
\end{quotation}
Clearly, Will Strunk had foreseen\footnote{\textbf{foreseen} [v] to know about something before it happens.}, as far back as 1918, the dangerous tonsillectomy\footnote{\textbf{tonsillectomy} [n] (\textit{medical}) a medical operation to remove the tonsils.} of a prince, in which the surgeon removes the tonsils \& the \textit{Times} copy desk removes the final \textit{s}. He started his book with it. I commend Rule 1 to the \textit{Times}, \& I trust that Charles's throat, not Charles' throat, is in fine shape today.

Style rules of this sort are, of course, somewhat a matter of individual preference\footnote{\textbf{preference} [n] \textbf{1.} [countable, usually singular, uncountable] a greater interest in or desire for somebody\texttt{/}something than somebody\texttt{/}something else; \textbf{2.} [countable] a thing that is liked better or best.}, \& even the established rules of grammar are open to challenge. Prof. Strunk, although 1 of the most inflexible\footnote{\textbf{inflexible} [a] \textbf{1.} (\textit{disapproving}) that cannot be changed or made more suitable for a particular situation, \textsc{synonym}: \textbf{rigid}; \textbf{2.} (\textit{disapproving}) (of people or organizations) unwilling to change their opinions, decision or behavior.} \& choosy\footnote{\textbf{choosy} [a] (\textit{informal}) careful in choosing; difficult to please, \textsc{synonym}: \textbf{fussy, picky}.} of men, was quick to acknowledge\footnote{\textbf{acknowledge} [v]  \textbf{1.} to accept that something is true or exists; \textbf{2.} to accept that somebody\texttt{/}something has a particular quality, importance or status, \textsc{synonym}: \textbf{recognize}; \textbf{3.} \textbf{acknowledge somebody\texttt{/}something} to publicly express thanks fo help or inspiration; \textbf{4.} \textbf{acknowledge something} to tell somebody that you have received something that they sent to you.} the fallacy\footnote{\textbf{fallacy} [n] \textbf{1.} [countable] a false idea that many people believe is true; \textbf{2.} [uncountable, countable] a false way of thinking about something.} of inflexibility \& the danger of doctrine\footnote{\textbf{doctrine} [n] \textbf{1.} [countable, uncountable] \textbf{doctrine (of something)} a belief or principle, or set of beliefs or principles, held by a religion, a political party or a legal system; \textbf{2.} (\textbf{Doctrine}) [countable] (US) a statement of government policy, especially foreign policy.}. ``It is an old observation,'' he wrote, ``that the best writers sometimes disregard\footnote{\textbf{disregard} [v] \textbf{disregard something} to not consider something; to treat something as unimportant, \textsc{synonym}: \textbf{ignore}.} the rules of rhetoric\footnote{\textbf{rhetoric} [n] [uncountable] \textbf{1.} (\textit{often disapproving}) speech or writing that is intended to influence people, but that is not completely honest or sincere; \textbf{2.} the skill of using language in speech or writing in a special way that influences or entertains people.}. \texttt{[stop translating here]} When they do so, however, the reader will usually find in the sentence some compensating merit, attained at the cost of the violation. Unless he is certain of doing as well, he will probably do best to follow the rules.''

It is encouraging to see how perfectly a book, even a dusty rule book, perpetuates \& extends the spirit of a man. Will Strunk loved the clear, the brief, the bold, \& his book is clear, brief, bold. Boldness is perhaps its chief distinguishing mark. On p. 26, explaining 1 of his parallels, he says, ``The lefthand version gives the impression that the writer is undecided or timid, apparently unable or afraid to choose 1 form of expression \& hold to it.'' \& his original Rule 11 was ``Make definite assertions.'' That was Will all over. He scorned the vague, the tame, the colorless, the irresolute. He felt it was worse to be irresolute than to be wrong. I remember a day in class when he leaned far forward, in his characteristic pose -- the pose of a man about to impart a secret -- \& croaked, ``If you don't know how to pronounce a word, say it loud! If you don't know how to pronounce a word, say it loud!'' This comical piece of advice struck me as sound at the time, \& I still respect it.\fbox{ Why compound ignorance with inaudibility?} \fbox{Why run \& hide?}

All through \textit{The Elements of Style} one finds evidence of the author's deep sympathy for the reader. Will felt that the reader was in serious trouble most of the time, floundering in a swamp, \& that it was the duty of anyone attempting to write English to drain this swamp quickly \& get the reader up on dry ground, or at least to throw a rope. In revising the text, I have tried to hold steadily in mind this belief of his, this concern for the bewildered reader.

In the English classes of today, ``the little book'' is surrounded by longer, lower textbooks -- books with permissive steering \& automatic transitions. Perhaps the book has become something of a curiosity. To me, it still seems to maintain its original poise, standing, in a drafty time, erect, resolute, \& assured. I still find the Strunkian wisdom a comfort, the Strunkian humor a delight, \& the Strunkian attitude forward right-\&-wrong a blessing undisguised.'' -- \cite[Introduction (by E. B. White)]{Strunk_White2019}

\section{Elementary Rules of Usage}
This section is devoted to study \cite[Chap. 1]{Strunk_White2019}.

\subsection{Form the possessive singular of nouns by adding \textit{'s}}
``Follow this rule whatever the final consonant. Thus write \textit{Charles's friend, Burns's poems, the witch's malice}. Exceptions are the possessive of ancient proper names in \textit{-es} \& \textit{-is}, the possessive \textit{Jesus'}, \& such forms as \textit{for conscience' sake, for righteousness' sake}. But such forms as \textit{Achilles' heel, Moses' laws, Isis' temple} are commonly replaced by: \textit{the laws of Moses, the temple of Isis}. The pronominal possessives \textit{hers, its, theirs, yours, \& ours} have no apostrophe. Indefinite pronouns, however, use the apostrophe to show possession: \textit{one's rights, somebody else's umbrella}. A common error is to write \textit{it's} for \textit{its}, or vice versa. The 1st is a contraction, meaning ``it is.'' The 2nd is a possessive.

\begin{example}
	It's a wise dog that scratches its own fleas.''
\end{example}
-- \cite[Chap. 1, Sect. 1, p. 14]{Strunk_White2019}

\subsection{In a series of $\ge 3$ terms with a single conjunction, use a comma after each term except the last}
``Thus write,

\begin{example}
	red, white, \& blue; gold, silver, or copper
	
	He opened the letter, read it, \& made a note of its contents.
\end{example}
This comma is often referred to as the ``serial'' comma. In the names of business firms the last comma is usually omitted. Follow the usage of the individual firm.

\begin{example}
	Little, Brown \& Company; Donaldson, Lufkin \& Jenrette''
\end{example}
-- \cite[Chap. 1, Sect. 2, p. 15]{Strunk_White2019}

\subsection{Enclose parenthetic expressions between commas}

\begin{example}
	``The best way to see a country, unless you are pressed for time, is to travel on foot.
\end{example}
This rule is difficult to apply; it is frequently hard to decide whether a single word, e.g. \textit{however}, or a brief phrase is or is not parenthetic. If the interruption to the flow of the sentence is but slight, the commas may be safely omitted. But whether the interruption is slight or considerable, never omit 1 comma \& leave the other. There is no defense for such punctuation as

\begin{example}
	Marijories husband, Colonel Nelson paid us a visit yesterday.
	
	My brother you will be pleased to hear, is now in perfect health.
\end{example}
Dates usually contain parenthetic words or figures. Punctuate as follows:

\begin{example}
	February to July, 1992; April 6, 1985; Wednesday, November 14, 1990
\end{example}
Note that it is customary to omit the comma in \textit{6 April 1988}. The last form is an excellent way to write a date; the figures are separated by a word \& are, for that reason, quickly grasped.

A name or a title in direct address is parenthetic.

\begin{example}
	If, Sir, you refuse, I cannot predict what will happen.
	
	Well, Susan, this is a fine mess you are in.
\end{example}
The abbreviations \textit{etc., i.e.,} \& \textit{e.g.,} the abbreviations for academic degrees, \& titles that follow a name are parenthetic \& should be punctuated accordingly.

\begin{example}
	Letters, packages, etc., should go here.
	
	Horace Fulsome, Ph.D., presided.
	
	Rachel Simonds, Attorney
	
	The Reverend Harry Lang, S.J.
\end{example}
No comma, however, should separate a noun from a restrictive term of identification.

\begin{example}
	Billy the Kid; The novelist Jane Austen; William the Conqueror; The poet Sappho
\end{example}
Although \textit{Junior}, with its abbreviation \textit{Jr.}, has commonly been regarded as parenthetic, logic suggests that it is, in fact, restrictive \& therefore not i need of a comma, e.g., \textit{James Wright Jr}.

Nonrestrictive relative clauses are parenthetic, as are similar clauses introduced by conjunctions indicating time or place. Commas are therefore needed. A nonrestrictive clause is one that does not serve to identify or define the antecedent noun.

\begin{example}
	The audience, which had at 1st been indifferent, became more \& more interested.
	
	In 1769, when Napoleon was born, Corsica had but recently been acquired by France.
	
	Nether Stowey, where Coleridge wrote The Rime of the Ancient Mariner, is a few miles from Bridgewater.
\end{example}
In these sentences, the clauses introduced by \textit{which, when, \& where} are nonrestrictive; they do not limit or define, they merely add something. In the 1st example, the clause introduced by \textit{which} does not sever to tell which of several possible audiences is meant; the reader presumably knows that already. The clause adds, parenthetically, a statement supplementing that in the main clause. Each of the 3 sentences is a combination of 2 statements that might have been made independently.

\begin{example}
	The audience was at 1st indifferent. Later it became more \& more interested.
	
	Napoleon was born in 1769. At that time Corsica had but recently been acquired by France.
	
	Coleridge wrote The Time of the Ancient Mariner at Nether Stowey. Nether Stowey is a few miles from Bridgewater.
\end{example}
Restrictive clauses, by contrast, are not parenthetic \& are not set off by commas. Thus

\begin{example}
	People who live in glass houses shouldn't throw stones.
\end{example}
Here the clause introduced by \textit{who} does serve to tell which people are meant; the sentence, unlike the sentences above, cannot be split into 2 independent statements. The same principle of comma use applies to participial phrases \& to appositives.

\begin{example}
	People sitting in the rear couldn't hear, \emph{(restrictive)}
	
	Uncle Bert, being slightly deaf, moved forward, \emph{(non-restrictive)}
	
	My cousin Bob is a talented harpist, \emph{(restrictive)}
	
	Our oldest daughter, Mary, sings, \emph{(nonrestrictive)}
\end{example}
When the main clause of a sentence is preceded by a phrase or a subordinate clause, use a comma to set off these elements.

\begin{example}
	Partly by hard fighting, partly by diplomatic skill, they enlarged their dominions to the east \& rose to royal rank with the possession of Sicily.''
\end{example}
-- \cite[Chap. 1, Sect. 3, pp. 16--17]{Strunk_White2019}

\subsection{Place a comma before a conjunction introducing an independent clause}

\begin{example}
	``The early records of the city have disappeared, \& the story of its 1st years can no longer be reconstructed.
	
	The situation is perilous, but there is still 1 chance of escape.
\end{example}
2-part sentences of which the 2nd member is introduced by \textit{as} (in the sense of ``because''), \textit{for, or, nor}, or \textit{while} (in the sense o ``\& at the same time'') likewise require a comma before the conjunction.

If a dependent clause, or an introductory phrase requiring to be set off by a comma, precedes the 2nd independent clause, no comma is needed after the conjunction.

\begin{example}
	The situation is perilous, but if we are prepared to act promptly, there is still 1 chance of escape.
\end{example}
When the subject is the same for both clauses \& is expressed only once, a comma is useful if the connective is \textit{but}. When the connective is \textit{and}, the comma should be omitted if the relation between the 2 statements is close or immediate.

\begin{example}
	I have heard the arguments, but am still unconvinced.
	
	He has had several years' experience \& is thoroughly competent.''
\end{example}
-- \cite[Chap. 1, Sect. 4, p. 18]{Strunk_White2019}

\subsection{Do not join independent clauses with a comma}
``If 2 or more clauses grammatically complete \& not joined by a conjunction are to form a single compound sentence, the proper mark of punctuation is a semicolon.

\begin{example}
	Mary Shelley's works are entertaining; they are full of engaging ideas.
	
	It is nearly half past 5; we cannot reach town before dark.
\end{example}
It is, of course, equally correct to write each of these as 2 sentences, replacing the semicolons with periods.

\begin{example}
	Mary Shelley's works are entertaining. They are full of engaging ideas.
	
	It is nearly half past 5. We cannot reach town before dark.
\end{example}
If a conjunction is inserted, the proper mark is a comma. (Rule 4.)

\begin{example}
	Mary Shelley's works are entertaining, for they are full of engaging ideas.
	
	It is nearly half past 5, \& we cannot reach town before dark.
\end{example}
A comparison of the 3 forms given above will show clearly the advantage of the 1st. It is, at least the examples given, better than the 2nd form because it suggests the close relationship between the 2 statements in a way that the 2nd does not attempt, \& better than the 3rd because it is briefer \& therefore more forcible. Indeed, this simple method of indicating relationship between statements is 1 of the most useful devices of composition. The relationship, as above, is commonly 1 of cause \& consequence.

Note that if the 2nd clause is preceded by an adverb, e.g., \textit{accordingly, besides, then, therefore}, or \textit{thus}, \& not by a conjunction, the semicolon is still required.

\begin{example}
	I had never been in the place before; besides, it was dark as a tomb.
\end{example}
An exception to the semicolon rule is worth noting here. A comma is preferable when the clauses are very short \& alike in form, or when the tone of the sentence is easy \& conversational.

\begin{example}
	Man proposes, God disposes.
	
	The gates swung apart, the bridge fell, the portcullis was drawn up.
	
	I hardly knew him, he was so changed.
	
	Here today, gone tomorrow.''
\end{example}
-- \cite[Chap. 1, Sect. 5, p. 19]{Strunk_White2019}

\subsection{Do not break sentences in 2}
``In other words, do not use periods for commas.

\begin{example}
	I met them on a Cunard liner many years ago. Coming home from Liverpool to New York.
	
	She was an interesting talker. A woman who had traveled all over the world \& lived in half a dozen countries.
\end{example}
In both these examples, the 1st period should be replaced by a comma \& the following word begun with a small letter.

It is permissible to make an emphatic word or expression serve the purpose of a sentence \& to punctuate it accordingly:

\begin{example}
	Again \& again he called out. No reply.
\end{example}
The writer must, however, be certain that the emphasis is warranted, lest a clipped sentence seem merely a blunder in syntax or in punctuation. Generally speaking, the place for broken sentences is in dialogue, when a character happens to speak in a clipped or fragmentary way.

Rules 3, 4, 5, \& 6 cover the most important principles that govern punctuation. They should be so thoroughly mastered that their application becomes 2nd nature.'' -- \cite[Chap. 1, Sect. 6, p. 20]{Strunk_White2019}

\subsection{Use a colon after an independent clause to introduce a list of particulars, an appositive, an amplification, or an illustrative quotation}
``A colon tells the reader that what follows is closely related to the preceding clause. The colon has more effect than the comma, less power to separate than the semicolon, \& more formality than the dash. It usually follows an independent clause \& should not separate a verb from its complement or a preposition from its object. The examples in the lefthand column, below, are wrong; they should be rewritten as in the righthand column.

\begin{example}
	Your dedicated whittler requires: a knife, a piece of wood, \& a back porch.
	
	$\hookrightarrow$ Your dedicated whittler requires 3 props: a knife, a piece of wood, \& a back porch.
	
	Understanding is that penetrating quality of knowledge that grows from: theory, practice, conviction, assertion, error, \& humiliation.
	
	$\hookrightarrow$ Understanding is that penetrating quality of knowledge that grows from theory, practice, conviction, assertion, error, \& humiliation.
\end{example}
Join 2 independent clauses with a colon if the 2nd interprets or amplifies the 1st.

\begin{example}
	But even so, there was a directness \& dispatch about animal burial: there was no stopover in the undertaker's foul parlor, no wreath or spray.
\end{example}
A colon may introduce a quotation that supports or contributes to the preceding clause.

\begin{example}
	The squalor of the streets reminded her of a line from Oscar Wilde: ``We are all in the gutter, but some of us are looking at the stars.''
\end{example}
The colon also has certain functions of form: to follow the salutation of a formal letter, to separate hour from minute in a notation of time, \& to separate the title of a work from its subtitle or a Bible chapter from a verse.

\begin{example}
	Dear Mr. Montague:
	
	departs at 10:48 P.M.
	
	Practical Calligraphy: An Introduction to Italic Script
	
	Nehemiah 11:7''
\end{example}
-- \cite[Chap. 1, Sect. 7, p. 21]{Strunk_White2019}

\subsection{Use a dash to set off an abrupt break or interruption \& to announce a long appositive or summary}
``A dash is a mark of separation stronger than a comma, less formal than a colon, \& more relaxed than parentheses.

\begin{example}
	His 1st thought on getting out of bed -- if he had any thought at all -- was to get back in again.
	
	The rear axle began to make a noise -- a grinding, chattering, teeth-gritting rasp.
	
	The increasing reluctance of the sun to rise, the extra nip in the breeze, the patter of shed leaves dropping -- all the evidences of fall drifting into winter were clearer each day.
\end{example}
Use a dash only when a more common mark of punctuation seems inadequate.

\begin{example}
	Her father's suspicions proved well-founded -- it was not Edward she cared for -- it was San Francisco.
	
	$\hookrightarrow$ Her father's suspicions proved well-founded. It was not Edward she cared for, it was San Francisco.
	
	Violence -- the kind you see on television -- is not honestly violent -- there lies its harm.
	
	$\hookrightarrow$ Violence, the kind you see on television, is not honestly violent. There lies its harm.''
\end{example}
-- \cite[Chap. 1, Sect. 8, p. 22]{Strunk_White2019}

\subsection{The number of the subject determines the number of the verb}
``Words that intervene between subject \& verb do not affect the number of the verb.

\begin{example}
	The bittersweet flavor of youth -- its trials, its joys, its adventures, its challenges -- are not soon forgotten.
	
	$\hookrightarrow$ The bittersweet flavor of youth -- its trials, its joys, its adventures, its challenges -- is not soon forgotten.
\end{example}
A common blunder is the use of a singular verb form in a relative clause following ``1 of $\ldots$'' or a similar expression when the relative is the subject.

\begin{example}
	1 of the ablest scientists who has attacked this problema $\to$ 1 of the ablest scientists who have attacked this problem
	
	1 of those people who is never ready on time $\to$ 1 of those people who are never ready on time
\end{example}
Use a singular verb form after \textit{each, either, everyone, everybody, neither, nobody, someone}.

\begin{example}
	Everybody thinks he has a unique sense of humor.
	
	Although both clocks strike cheerfully, neither keeps good time.
\end{example}
With \textit{none}, use the singular verb when the word means ``no one'' or ``not one.''

\begin{example}
	None of us are perfect. $\to$ None of us is perfect.
\end{example}
A plural verb is commonly used when \textit{none} suggests more than 1 thing or person.

\begin{example}
	None are so fallible as those who are sure they're right.
\end{example}
A compound subject formed of 2 or more nouns joined by \textit{\&} almost always requires a plural verb.

\begin{example}
	The walrus \& the carpenter were walking close at hand.
\end{example}
But certain compounds, often cliches, are so inseparable they are considered a unit \& so take a singular verb, as do compound subjects qualified by \textit{each} or \textit{every}.

\begin{example}
	The long \& the short of it is $\ldots$
	
	Bread \& butter was all she served.
	
	Give \& take is essential to a happy household.
	
	Every window, picture, \& mirror was smashed.
\end{example}
A singular subject remains singular even if other nouns are connected to it by \textit{with, as well as, in addition to, except, together with}, \& \textit{no less than}.

\begin{example}
	His speech as well as his manner is objectionable.
\end{example}
A linking verb agrees with the number of its subject.

\begin{example}
	What is wanted is a few more pairs of hands.
	
	The trouble with truth is its many varieties.
\end{example}
Some nouns that appear to be plural are usually construed as singular \& given a singular verb.

\begin{example}
	Politics is an art, not a science.
	
	The Republican Head quarters is on this side of the tracks.
\end{example}
But
\begin{example}
	The general's quarters are across the river.
\end{example}
In these cases the writer must simply learn the idioms. The content of a book is singular. The contents of a jar may be either singular or plural, depending on what's in the jar -- jam or marbles.'' -- \cite[Chap. 1, Sect. 9, pp. 23--24]{Strunk_White2019}

\subsection{Use the proper case of pronoun}
``The personal pronouns, as well as the pronoun \textit{who}, change form as they function as subject or object.

\begin{example}
	Will Jane or he be hired, do you think?
	
	The culprit, it turned out, was he.
	
	We heavy eaters would rather walk than ride.
	
	Who knocks?
	
	Give this work to whoever looks idle.
\end{example}
In the last example, \textit{whoever} is the subject of \textit{looks idle}; the object of the preposition \textit{to} is the entire clause \textit{whoever looks idle}. When \textit{who} introduces a subordinate clause, its case depends on its function in that clause.

\begin{example}
	Virgil Soames is the candidate whom we think will win. $\to$ Virgil Soames is the candidate who we think will win. [We think \emph{he} will win.]
	
	Virgil Soames is the candidate who we hope to elect. $\to$ Virgil Soames is the candidate whom we hope to elect. [We hope to elect \emph{him}.]
\end{example}
A pronoun in a comparison is nominative if it is the subject of a stated or understood verb.

\begin{example}
	Sandy writes better than I. (Than I write).
\end{example}
In general, avoid ``understood'' verbs by supplying them.

\begin{example}
	I think Horace admires Jessica more than I. $\to$ I think Horace admires Jessica more than I do.
	
	Polly loves cake more than me. $\to$ Polly loves cake more than she loves me.
\end{example}
The objective case is correct in the following examples.

\begin{example}
	The ranger offered Shirley \& him some advice on campsites.
	
	They came to meet the Baldwins \& us.
	
	Let's talk it over between us, then, you \& me.
	
	Whom should I ask?
	
	A group of us taxpayers protested.
\end{example}
\textit{Us} in the last example is in apposition to taxpayers, the object of the preposition \textit{of}. The wording, although grammatically defensible, is rarely apt. ``A group of us protested as taxpayers.'' is better, if not exactly equivalent.

Use the simple personal pronoun as a subject.

\begin{example}
	Blake \& myself stayed home. $\to$ Blake \& I stayed home.
	
	Howawrd \& yourself brought the lunch, I thought. $\to$ Howard \& you brought the lunch, I thought.
\end{example}
The possession case of pronouns is used to show ownership. It has 2 forms: the adjectival modifier, \textit{your} hat, \& the noun form, a hat of \textit{yours}.

\begin{example}
	The dog has buried 1 of your gloves \& 1 of mine in the flower bed.
\end{example}
Gerunds usually require the possessive case.

\begin{example}
	Mother objected to our driving on the icy roads.
\end{example}
A present participle as a verbal, on the other hand, takes the objective case.

\begin{example}
	They heard him singing in the shower.
\end{example}
The difference between a verbal participle \& a gerund is not always obvious, but not what is really said in each of the following.

\begin{example}
	Do you mind me asking a question?
	
	Do you mind my asking a question?
\end{example}
In the 1st sentence, the queried objection is to \textit{me}, as opposed to other members of group, asking a question. In the 2nd example, the issue is whether a question may be asked at all.'' -- \cite[Chap. 1, Sect. 10, pp. 25--26]{Strunk_White2019}

\subsection{A participial phrase at the beginning of a sentence must refer to the grammatical subject}

\begin{example}
	Walking slowly down the road, he saw a woman accompanied by 2 children.
\end{example}
The word \textit{walking} refers to the subject of the sentence, not to the woman. To make it refer to the woman, the writer must recast the sentence.

\begin{example}
	He saw a woman, accompanied by 2 children, walking slowly down the road.
\end{example}
Participial phrases preceded by a conjunction or by a preposition, nouns in apposition, adjectives, \& adjective phrases come under the same rule if they begin the sentence.

\begin{example}
	On arriving in Chicago, his friends met him at the station. $\to$ On arriving in Chicago, he was met at the station by his friends.
	
	A soldier of proved valor, they entrusted him with the defense of the city. $\to$ A soldier of proved valor, he was entrusted with the defense of the city.
	
	Young \& inexperienced, the task seemed easy to me. $\to$ Young \& inexperienced, I thought the task easy.
	
	Without a friend to counsel him, the temptation proved irresistible. $\to$ Without a friend to counsel him, he found the temptation irresistible.
\end{example}
Sentences violating Rule 11 are often ludicrous:

\begin{example}
	Being in a dilapidated condition, I was able to buy the house very cheap.
	
	Wondering irresolutely what to do next, the clock struck 12.'' -- \cite[Chap. 1, Sect. 11, p. 27]{Strunk_White2019}
\end{example}

\section{Elementary Principles of Composition}
This section is devoted to study \cite[Chap. 2]{Strunk_White2019}.

\subsection{Choose a suitable design \& hold to it}
``A basic structural design underlies every kind of writing. Writers will in part follow this design, in part deviate from it, according to their skills, their needs, \& the unexpected events that accompany the act of composition. Writing, to be effective, must follow closely the thoughts of the writer, but not necessarily in the order in which those  thoughts occur. This calls for a scheme of procedure. In some cases, the best design is no design, as with a love letter, which is simply an outpouring, or with a casual essay, which is a ramble. But in most cases, planning must be a deliberate prelude to writing. The 1st principle of composition, therefore, is to foresee or determine the shape of what is to come \& pursue that shape.

A sonnet is built on a 14-line frame, each line containing 5 feet. Hence, sonneteers know exactly where they are headed, although they may not know how to get there. Most forms of composition are less clearly defined, more flexible, but all have skeletons to which the writer will bring the flesh \& the blood. The more clearly the writer perceives the shape, the better are the chances of success.'' --  \cite[Chap. 2, Sect. 12, p. 29]{Strunk_White2019}

\subsection{Make the paragraph the unit of composition: 1 paragraph to each topic}
``The paragraph is a convenient unit; it serves all forms of literary work. As long as it holds together, a paragraph may be of any length -- a single, short sentence or a passage of great duration.

If the subject on which you are writing is of slight extent, or if you intend to treat it briefly, there may be no need to divide it into topics. Thus, a brief description, a brief book review, a brief account of a single incident, a narrative merely outlining an action, the setting forth of a single idea -- any 1 of these is best writing in a single paragraph. After the paragraph has been written, examine it to see whether division will improve it.

Ordinarily, however, a subject requires division into topics, each of which should be dealt with in a paragraph. The object of treating each topic in a paragraph by itself, of course, to aid the reader. The beginning of each paragraph is a signal that a new step in the development of the subject has been reached.

As a rule, single sentences should not be written or printed as paragraphs. An exception may be made of sentences of transition, indicating the relation between the parts of an exposition or argument.

In dialogue, each speech, even if only a single word, is usually a paragraph by itself; i.e., a new paragraph begins with each change of speaker. The application of this rule when dialogue \& narrative are combined is best learned from examples in well-edited works of fiction. Sometimes a writer, seeking to create an effect of rapid talk or for some other reason, will elect not to set off each speech in a separate paragraph \& instead will run speeches together. The common practice, however, \& the one that serves best in most instances, is to give each speech a paragraph of its own.

As a rule, begin each paragraph either with a sentence that suggests the topic or with a sentence that helps the transition. If a paragraph forms part of a larger composition, its relation to what precedes, or its function as a part of the whole, may need to be expressed. This can sometimes be done by a mere word or phrase (\textit{again, therefore, for the same reason}) in the 1st sentence. Sometimes, however, it is expedient to get into the topic slowly, by way of a sentence or 2 of introduction or translation.

In narration \& description, the paragraph sometimes begins with a concise, comprehensive statement serving to hold together the details that follow.

\begin{example}
	The breeze served us admirably.
	
	The campaign opened with a series of reverses.
	
	The next 10 or 12 pages were filled with a curious set of entries.
\end{example}
\fbox{But when this device, or any device, is too often used, it becomes a mannerism.} More commonly, the opening sentence simply indicates by its subject the direction the paragraph is to take.

\begin{example}
	At length I thought I might return toward the stockade.
	
	He picked up the heavy lamp from the table \& began to explore.
	
	Another flight of steps, \& they emerged on the roof.
\end{example}
In animated narrative, the paragraphs are likely to be short \& without any semblance of a topic sentence, the writer rushing headlong, event following event in rapid succession. The break between such paragraphs merely serves the purpose of a rhetorical pause, throwing into prominence some detail of the action.

In general, remember that paragraphing calls for a good eye as well as a logical mind. Enormous blocks of print look formidable to readers, who are often reluctant to tackle them. Therefore, breaking long paragraphs in 2, even if it is not necessary to do so for sense, meaning, or logical development, is often a visual help. But remember, too, that firing off many short paragraphs in quick succession can be distracting. Paragraph breaks used only for show read like the writing of commerce or of display advertising. Moderation \& a sense of order should be the main considerations in paragraphing.'' --  \cite[Chap. 2, Sect. 13, pp. 30--31]{Strunk_White2019}

\subsection{Use the active voice}
``\fbox{The active voice is usually more direct \& vigorous than the passive}:

\begin{example}
	I shall always remember my 1st visit to Boston.
\end{example}
This is much better than:

\begin{example}
	My 1st visit to Boston will always be remembered by me.
\end{example}
The latter sentence is less direct, less bold, \& less concise. If the writer tries to make it more concise by omitting ``by me,'': \textit{My 1st visit to Boston will always be remembered}, it becomes indefinite: is it the writer or some undisclosed person or the world at large that will always remember this visit?

This rule does not, of course, mean that the writer should entirely discard the passive voice, which is frequently convenient \& sometimes necessary.

\begin{example}
	The dramatists of the Restoration are little esteemed today.
	
	Modern readers have little esteem for the dramatists of the Restoration.
\end{example}
The 1st would be the preferred form in a paragraph on the dramatists of the Restoration, the 2nd in a paragraph on the tastes of modern readers. The need to make a particular word the subject of the sentence will often, as in these examples, determine which voice is to be used.

The habitual use of the active voice, however, makes for forcible writing. This is true not only in narrative concerned principally with action but in writing of any kind. Many a tame sentence of description or exposition can be made lively \& emphatic by substituting a transitive in the active voice for some such perfunctory expression as \textit{there is} or \textit{could be heard}.

\begin{example}
	There were a great number of dead leaves lying on the ground. $\to$ Dead leaves covered the ground.
	
	At dawn the crowing of a rooster could be heard. $\to$ The cock's crow came with dawn.
	
	The reason he left college was that his health became impaired. $\to$ Failing health compelled him to leave college.
	
	It was not long before she was very sorry that she had said what she had. $\to$ She soon repented her words.
\end{example}
Note, in the examples above, that when a sentence is made stronger, it usually becomes shorter.

Thus, \fbox{brevity is a by-product of vigor}.'' --  \cite[Chap. 2, Sect. 14, p. 32]{Strunk_White2019}

\subsection{Put statements in positive form}
``Make definite assertions. Avoid tame, colorless, hesitating, noncommittal language. Use the word \textit{not} as a means of denial or in antithesis, never as a means of evasion.

\begin{example}
	He was not very often on time. $\to$ He usually came late.
	
	She did not think that studying Latin was a sensible way to use one's time. $\to$ She thought the study of Latin a waste of time.
	
	\emph{The Taming of the Shrew} is rather weak in spots. Shakespeare does not portray Katharine as a very admirable character, nor does Bianca remain long in memory as an important character in Shakespeare's works.
	
	$\hookrightarrow$ The women in \emph{The Taming of the Shrew} are unattractive. Katharine is disagreeable, Bianca insignificant.
\end{example}
The last example, before correction, is indefinite as well as negative. The corrected version, consequently, is simply a guess at the writer's intention.

All 3 examples show the weakness inherent in the word \textit{not}. Consciously or unconsciously, the reader is dissatisfied with being told only what is not; the reader wishes to be told what is. Hence, as a rule, it is better to express even a negative in positive form.

\begin{example}
	not honest $\to$ dishonest; not important $\to$ trifling; did not remember $\to$ forgot; did not pay any attention to $\to$ ignored; $\to$ did not have much confidence in $\to$ distrusted
\end{example}
Placing negative \& positive in opposition makes for a stronger structure.

\begin{example}
	Not charity, but simple justice.
	
	Not that I loved Caesar less, but that I loved Rome more.
	
	Ask not what your country can do for you -- ask what you can do for your country.\footnote{\selectlanguage{vietnamese} ``Đừng hỏi Tổ quốc đã làm gì cho ta mà phải hỏi ta đã làm gì cho Tổ quốc hôm nay'' -- \textit{Khát Vọng Tuổi Trẻ} (1995), sáng tác: Vũ Hoàng.}
\end{example}
Negative words other than \textit{not} are usually strong.

\begin{example}
	Her loveliness I never knew\emph{\texttt{/}}Until she smiled on me.
\end{example}
Statements qualified with unnecessary auxiliaries or conditionals sound irresolute.

\begin{example}
	If you would let us know the time of your arrival, we would be happy to arrange your transportation from the airport.
	
	$\hookrightarrow$ If you will let us know the time of your arrival, we shall be happy to arrange your transportation from the airport.
	
	Applicants can make a good impression by being neat \& punctual. $\to$ Applicants will make a good impression if they are neat \& punctual.
	
	Plath may be ranked among those modem poets who died young. $\to$ Plath was 1 of those modern poets who died young.
\end{example}
\fbox{If your every sentence admits a doubt, your writing will lack authority.} Save the auxiliaries \textit{would, should, could, may, might}, \& \textit{can} for situations involving real uncertainty.'' --  \cite[Chap. 2, Sect. 15, pp. 33--34]{Strunk_White2019}

\subsection{Use definite, specific, concrete language}
``Prefer the specific to the general, the definite to the vague, the concrete to the abstract.

\begin{example}
	A period of unfavorable weather set in. $\to$ It rained every day for a week.
	
	He showed satisfaction as he took possession of his well-earned reward. $\to$ He grinned as he pocketed the coin.
\end{example}
If those who have studied the art of writing are in accord on any 1 point, it is this: the surest way to arouse \& hold the readers attention is by being specific, definite, \& concrete. The greatest writers -- Homer, Dante, Shakespeare -- are effective largely because they deal in particulars \& report the details that matter. Their words call up pictures.

Jean Stafford, to cite a more modern author, demonstrates in her short story ``In the Zoo'' how prose is made vivid by the use of words that evoke images \& sensations:

\begin{example}
	$\ldots$ Daisy \& I in time found asylum in a small menagerie down by the railroad tracks. It belonged to a gentle alcoholic ne'er-do- well, who did nothing all day long but drink bathtub gin in rickeys \& play solitaire \& smile to himself \& talk to his animals. He had a little, stunted red vixen \& a deodorized skunk, a parrot from Tahiti that spoke Parisian French, a woebegone coyote, \& 2 capuchin monkeys, so serious \& humanized, so small \& sad \& sweet, \& so religious-looking with their tonsured heads that it was impossible not to think their gibberish was really an ordered language with a grammar that somebody some philologist wound understand.
	
	Gran knew about our visits to Mr. Murphy \& she did not object, for it gave her keen pleasure to excoriate him when we came home. His vice was not a matter of guesswork; it was an established fact that he was half-seas over from dawn till midnight. ``With the black Irish,'' said Gran, ``the taste for drink is taken in with the mother's milk \& is never mastered. Oh, I know all about those promises to join the temperance movement \& not to touch another drop. \fbox{The way to Hell is paved with good intentions.}'' -- Excerpt from \textit{``In the Zoo''} from Bad Characters by Jean Stafford.
\end{example}
If the experiences of Walter Mitty, of Molly Bloom, of Rabbit Angstrom have seemed for the moment real to countless readers, if in reading Faulkner we have almost the sense of inhabiting Yoknapatawpha County during the decline of the South, it is because the details used are definite, the terms concrete. It is not that every detail is given -- that would be impossible, as well as to no purpose -- but that all the significant details are given, \& with such accuracy \& vigor that readers, in imagination, can project themselves into the scene.

In exposition \& in argument, the writer must likewise never lose hold of the concrete; \& even when dealing with general principles, the writer must furnish particular instances of their application.

In his \textit{Philosophy of Style}, Herbert Spencer gives 2 sentences to illustrate how the vague \& general can be turned into the vivid \& particular:

\begin{example}
	In proportion as the manners, customs, \& amusements of a nation are cruel \& barbarous, the regulations of their penal code will be severe.
	
	$\hookrightarrow$ In proportion as men delight in battles, bullfights, \& combats of gladiators, will they punish by hanging, burning, \& the rack.
\end{example}
To show what happens when strong writing is deprived of its vigor, George Orwell once took a passage from the Bible \& drained it of its blood. On the left, below, is Orwell's translation; on the right, the verse from Ecclesiastes (King James Version).

\begin{example}
	Objective consideration of contemporary phenomena compels the conclusion that success or failure in competitive activities exhibits no tendency to be commensurate with innate capacity, but that a considerable element of the unpredictable must inevitably be taken into account.
	
	$\hookrightarrow$ I returned, \& saw under the sun, that the race is not to the swift, nor the battle to the strong, neither yet bread to the wise, nor yet riches to men of understanding, nor yet favor to men of skill; but time \& chance happeneth to them all.''
\end{example}
--  \cite[Chap. 2, Sect. 16, pp. 35--36]{Strunk_White2019}

\subsection{Omit needless words}
``\fbox{Vigorous writing is concise.} A sentence should contain no unnecessary words, a paragraph no unnecessary sentences, for the same reason that a drawing should have no unnecessary lines \& a machine no unnecessary parts. This requires not that the writer make all sentences short, or avoid all detail \& treat subjects only in outline, but that every word tell.

Many expressions in common use violate this principle.

\begin{example}
	the question as to whether $\to$ whether (the question whether)
	
	there is no doubt but that $\to$ no doubt (doubtless)
	
	used for fuel purposes $\to$ used for fuel
	
	he is a man who $\to$ he
	
	in a hasty manner $\to$ hastily
	
	this is a subject that $\to$ this subject
	
	Her story is a strange one. $\to$ Her story is strange.
	
	the reason why is that $\to$ because
\end{example}
\textit{The fact that} is an especially debilitating expression. It should be revised out of every sentence in which it occurs.

\begin{example}
	owing to the fact that $\to$ since (because)
	
	in spite of the fact that $\to$ though (although)
	
	call your attention to the fact that $\to$ remind you (notify you)
	
	I was unaware of the fact that $\to$ I was unaware that (did not know)
	
	the fact that he had not succeeded $\to$ his failure
	
	the fact that I had arrived $\to$ my arrival
\end{example}
See also the words \textit{case, character, nature} in Chap. IV. \textit{Who is, which was}, \& the like are often superfluous.

\begin{example}
	His cousin, who is a member of the same firm $\to$ His cousin, a member of the same firm
	
	Trafalgar, which was Nelson's last battle $\to$ Trafalgar, Nelson's last battle
\end{example}
As the active voice is more concise than the passive, \& a positive statement more concise than a negative one, many of the examples given under Rules 14 \& 15 illustrate this rule as well.

A common way to fall into wordiness is to present a single complex idea, step by step, in a series of sentences that might to advantage be combined into one.

\begin{example}
	Macbeth was very ambitious. This led him to wish to become king of Scotland. The witches told him that this wish of his would come true. The king of Scotland at this time was Duncan. Encouraged by his wife, Macbeth murdered Duncan. He was thus enabled to succeed Duncan as king. (51 words)
	
	$\hookrightarrow$ Encouraged by his wife, Macbeth achieved his ambition \& realized the prediction of the witches by murdering Duncan \& becoming king of Scotland in his place. (26 words)''
\end{example}
--  \cite[Chap. 2, Sect. 17, pp. 37--38]{Strunk_White2019}

\subsection{Avoid a succession of loose sentences}
``This rule refers especially to loose sentences of a particular type: those consisting of 2 clauses, the 2nd introduced by a conjunction or relative. A writer may err by making sentences too compact \& periodic. An occasional loose sentence prevents the style from becoming too formal \& gives the reader a certain relief. Consequently, loose sentences are common in easy, unstudied writing. The danger is that there may be too many of them.

An unskilled writer will sometimes construct a whole paragraph of sentences of this kind, using as connectives \textit{\&, but}, \&, less frequently, \textit{who, which, when, where}, \& \textit{while}, these last in nonrestrictive senses. (See Rule 3.)

\begin{example}
	The 3rd concert of the subscription series was given last evening, \& a large audience was in attendance. Mr. Edward Appleton was the soloist, \& the Boston Symphony Orchestra furnished the instrumental music. The former showed himself to be an artist of the 1st rank, while the latter proved itself fully deserving of its high reputation. The interest aroused by the series has been very gratifying to the Committee, \& it is planned to give a similar series annually hereafter. The 4th concert will be given on Tuesday, May 10, when an equally attractive program will be presented.
\end{example}
Apart from its triteness \& emptiness, the paragraph above is bad because of the structure of its sentences, with their mechanical symmetry \& singsong. Compare these sentences from the chapter ``What I Believe'' in E. M. Forster's \textit{2 Cheers for Democracy}:

\begin{example}
	I believe in aristocracy, though -- if that is the right word, \& if a democrat may use it. Not an aristocracy of power, based upon rank \& influence, but an aristocracy of the sensitive, the considerate \& the plucky. Its members are to be found in all nations \& classes, \& all through the ages, \& there is a secret understanding between them when they meet. They represent the true human tradition, the 1 permanent victory of our queer race over cruelty \& chaos. Thousands of them perish in obscurity, a few are great names. They are sensitive for others as well as for themselves, they are considerate without being fussy, their pluck is not swankiness but the power to endure, \& they can take a joke.
\end{example}
A writer who has written a series of loose sentences should recast enough of them to remove the monotony, replacing them with simple sentences, sentences of 2 clauses joined by a semicolon, periodic sentences of 2 clauses, or sentences (loose or periodic) of 3 clauses -- whichever best represent the real relations of the thought.'' -- \cite[Chap. 2, Sect. 18, pp. 39--40]{Strunk_White2019}

\subsection{Express coordinate ideas in similar form}
``This principle, that of parallel construction, requires that expressions similar in content \& function be outwardly similar. The likeness of form enables the reader to recognize more readily the likeness of content \& function. The familiar Beautitudes exemplify the virtue of parallel construction.

\begin{example}
	Blessed are the poor in spirit: for theirs is the kingdom of heaven.
	
	Blessed are they that mourn: for they shall be comforted.
	
	Blessed are the meek: for they shall inherit the earth.
	
	Blessed are they which do hunger \& thirst after righteousness: for they shall be filled.
\end{example}
The unskilled writer often violates this principle, mistakenly believing in the value of constantly varying the form of expression. When repeating a statement to emphasize it, the writer may need to vary its form. Otherwise, the writer should follow the principle of parallel construction.

\begin{example}
	Formerly, science was taught by the textbook method, while now the laboratory method is employed.
	
	$\hookrightarrow$ Formerly, science was taught by the textbook method; now it is taught by the laboratory method.
\end{example}
The lefthand version gives the impression that the writer is undecided or timid, apparently unable or afraid to choose 1 form of expression \& hold to it. The right hand version shows that the writer has at least made a choice \& abided by it.

By this principle, an article or a preposition applying to all the members of a series must either be used only before the 1st term or else be repeated before each term.

\begin{example}
	The French, the Italians, Spanish, \& Portuguese $\to$ The French, the Italians, the Spanish, \& the Portuguese
	
	In spring, summer, or in winter $\to$ In spring, summer, or winter (In spring, in summer, or in winter)
\end{example}
Some words require a particular preposition in certain idiomatic uses. When such words are joined in a compound construction, all the appropriate prepositions must be included, unless they are the same.

\begin{example}
	His speech was marked by disagreement \& scorn for his opponent's position. $\to$ His speech was marked by disagreement with \& scorn for his opponent's position.
\end{example}
Correlative expressions (\textit{both, and; not, but; not only, but also; either, or; 1st, 2nd, 3rd}; \& the like) should be followed by the same grammatical construction. Many violations of this rule can be corrected by rearranging the sentence.

\begin{example}
	It was both a long ceremony \& very tedious. $\to$ The ceremony was both long \& tedious.
	
	A time not for words but action. $\to$ A time not for words but for action.
	
	Either you must grant his request or incur his ill will. $\to$ You must either grant his request or incur his ill will.
	
	My objections are, 1st, the injustice of the measure; 2nd, that it is unconstitutional. $\to$ My objections are, 1st,  that the measure is unjust; 2nd, that it is unconstitutional.
\end{example}
It may be asked, what if you need to express a rather large number of similar ideas -- say, 20? Must you write 20 consecutive sentences of the same pattern? On closer examination, you will probably find that the difficulty is imaginary -- that these 20 ideas can be classified in groups, \& that you need apply the principle only within each group. Otherwise, it is best to avoid the difficulty by putting statements in the form of a table.'' -- \cite[Chap. 2, Sect. 19, pp. 41--42]{Strunk_White2019}

\subsection{Keep related words together}
``\fbox{The position of the words in a sentence is the principal means of showing their relationship.} Confusion \& ambiguity result when words are badly placed. The writer must, therefore, bring together the words \& groups of words that are related in thought \& keep apart those that are not so related.

\begin{example}
	He noticed a large stain in the rug that was right in the center. $\to$ He noticed a large stain right in the center of the rug.
	
	You can call your mother in London \& tell her about George's taking you out to dinner for just 2 dollars. $\to$ For just 2 dollars you can call your mother in London \& tell her all about George's taking you out to dinner.
	
	New York's 1st commercial human-sperm bank opened Friday with semen samples from 18 men frozen in a stainless steel tank. $\to$ New York's 1st commercial human-sperm bank opened Friday when semen samples were taken from 18 men. The samples were then frozen \& stored in a stainless steel tank.
\end{example}
In the lefthand version of the 1st example, the reader has no way of knowing whether the stain was in the center of the rug or the rug was in the center of the room. In the lefthand version of the 2nd example, the reader may well wonder which cost 2 dollars -- the phone call or the dinner. In the lefthand version of the 3rd example, the reader's heart goes out to those 18 poor fellows frozen in a steel tank.

The subject of a sentence \& the principal verb should not, as a rule, be separated by a phrase or clause that can be transferred to the beginning.

\begin{example}
	Toni Morrison, in \emph{Beloved}, writes about characters who have escaped from slavery but are haunted by its heritage. $\to$ In \emph{Beloved}, Toni Morrison writes about characters who have escaped from slavery but are haunted by its heritage.
	
	A dog, if you fail to discipline him, becomes a household pest. $\to$ Unless disciplined, a dog becomes a household pest.
\end{example}
Interposing a phrase or a clause, as in the lefthand examples above, interrupts the flow of the main clause. This interruption, however, is not usually bothersome when the flow is checked only by a relative clause or by an expression in apposition. Sometimes, in periodic sentences, the interruption is a deliberate device for creating suspense\footnote{\textbf{suspense} [n] [uncountable] a feeling of worry or excitement that you have when you feel that something is going to happen, somebody is going to tell you some news, etc.}. (See examples under Rule 22.)

The relative pronoun should come, in most instances, immediately after its antecedent.

\begin{example}
	There was a stir in the audience that suggested disapproval. $\to$ A stir that suggested disapproval swept the audience.
	
	He wrote 3 articles about his adventures in Spain, which were published in \emph{Harper's Magazine}.
	
	$\hookrightarrow$ He published 3 articles in \emph{Harper's Magazine} about his adventures in Spain.
	
	This is a portrait of Benjamin Harrison, who became President in 1889. He was the grandson of William Henry Harrison.
	
	$\hookrightarrow$ This is a portrait of Benjamin Harrison, grandson of William Henry Harrison, who became President in 1889.
\end{example}
If the antecedent consists of a group of words, the relative comes at the end of the group, unless this would cause ambiguity.

\begin{example}
	The Superintendent of the Chicago Division, who
\end{example}
No ambiguity results from the above. But

\begin{example}
	A proposal to amend the Sherman Act, which has been variously judged
\end{example}
leaves the reader wondering whether it is the proposal or the Act that has been various judged. The relative clause must be moved forward, to read, ``A proposal, which has been variously judged, to amend the Sherman Act $\ldots$'' Similarly

\begin{example}
	The grandson of William Henry Harrison, who $\to$ William Henry Harrison's grandson, Benjamin Harrison, who
\end{example}
A noun in apposition may come between antecedent \& relative, because in such a combination no real ambiguity can arise.

\begin{example}
	The Duke of York, his brother, who was regarded with hostility by the Whigs
\end{example}
Modifiers should come, if possible, next to the words they modify. If several expressions modify the same word, they should be arranged so that no wrong relation is suggested.

\begin{example}
	All the members were not present. $\to$ Not all the members were present.
	
	She only found 2 mistakes. $\to$ She found only 2 mistakes.
	
	The director said he hoped all members would give generously to the Fund at a meeting of the committee yesterday.
	
	$\hookrightarrow$ At a meeting of the committee yesterday, the director said he hoped all members would give generously to the Fund.
	
	Major R. E. Joyce will give a lecture on Tuesday evening in Bailey Hall, to which the public is invited on ``My Experiences in Mesopotamia'' at 8:00 P.M.
	
	$\hookrightarrow$ On Tuesday evening at 8, Major R. E. Joyce will give a lecture in Bailey Hall on ``My Experiences in Mesopotamia.'' The public is invited.
\end{example}
Note, in the last lefthand example, how swiftly meaning departs when words are wrongly juxtaposed.'' -- \cite[Chap. 2, Sect. 20, pp. 43--45]{Strunk_White2019}

\subsection{In summaries, keep to 1 tense}
``In summarizing the action of a drama, use the present tense. In summarizing a poem, story, or novel, also use the present, though you may use the past if it seems more natural to do so. If the summary is in the present tense, antecedent action should be expressed by the perfect; if in the past, by the past perfect.

\begin{example}
	Chance prevents Friar John from delivering Friar Lawrence's letter to Romeo. Meanwhile, owning to her father's arbitrary change of the day set for her wedding, Juliet has been compelled to drink the potion on Tuesday night, with the result that Balthasar informs Romeo of her supposed death before Friar Lawrence learns of the nondelivery of the letter.
\end{example}
But whichever tense is used in the summary, a past tense in indirect discourse or in indirect question remains unchanged.

\begin{example}
	The Friar confesses that it was he who married them.
\end{example}
Apart from the exceptions noted, the writer should use the same tense throughout. Shifting from 1 tense to another gives the appearance of uncertainty \& irresolution.

In presenting the statements or the thought of someone else, as in summarizing an essay or reporting a speech, do not overwork such expressions as ``he said,'' ``she stated,'' ``the speaker added,'' ``the speaker then went on to say,'' ``the author also thinks.'' Indicate clearly at the outset, once for all, that what follows is summary, \& then waste no words in repeating the notification.

In notebooks, in newspapers, in handbooks of literature, summaries of 1 kind or another may be indispensable\footnote{\textbf{indispensable} [a] too important to be without, \textsc{synonym}: \textbf{essential}.}, \& for children in primary schools retelling a story in their own words is a useful exercise. But in the criticism or interpretation of literature, be careful to avoid dropping into summary. It may be necessary to devote 1 or 2 sentences to indicating the subject, or the opening situation, of the work being discussed, or to cite numerous details to illustrate its qualities. But you should aim at writing an orderly discussion supported by evidence, not a summary with occasional comment. Similarly, if the scope of the discussion includes a number of works, as a rule it is better not to take them up singly in chronological order but to aim from the beginning at establishing general conclusions.'' -- \cite[Chap. 2, Sect. 21, pp. 46--47]{Strunk_White2019}

\subsection{Place the emphatic words of a sentence at the end}
``The proper place in the sentence for the word or group of words that the writer desires to make most prominent is usually the end.

\begin{example}
	Humanity has hardly advanced in fortitude since that time, though it has advanced in many other ways.
	
	$\hookrightarrow$ Since that time, humanity has advanced in many ways, but it has hardly advanced in fortitude.
	
	This steel is principally used for making razors, because of its hardness.
	
	$\hookrightarrow$ Because of its hardness, this steel is used primarily for making razors.
\end{example}
The word or group of words entitled to this position of prominence is usually the logical predicate -- i.e., the \textit{new} element in the sentence, as it is in the 2nd example. The effectiveness of the periodic sentence arises from the prominence it gives to the main statement.

\begin{example}
	4 centuries ago, Christopher Columbus, 1 of the Italian mariners whom the decline of their own republics had put at the service of the world \& of adventure, seeking for Spain a westward passage to the Indies to offset the achievement of Portuguese discoverers, lighted on America.
	
	With these hopes \& in this belief I would urge you, laying aside all hindrance, thrusting away all private aims, to devote yourself unswervingly \& unflinchingly to the vigorous \& successful prosecution of this war.
\end{example}
The other prominent position in the sentence is the beginning. Any element in the sentence other than the subject becomes emphatic when placed 1st.

\begin{example}
	Deceit or treachery she could never forgive.
	
	Vast \& rude, fretted by the action of nearly 3000 years, the fragments of this architecture may often seem, at 1st sight, like works of nature.
	
	Home is the sailor.
\end{example}
A subject coming 1st in its sentence may be emphatic, but hardly by its position alone. In the sentence

\begin{example}
	Great kings worshiped at his shrine
\end{example}
the emphasis upon \textit{kings} arises largely from its meaning \& from the context. To receive special emphasis, the subject of a sentence must take the position of the predicate.

\begin{example}
	Through the middle of the valley flowed a winding stream.
\end{example}
The principle that the proper place for what is to be made most prominent is the end applies equally to the words of a sentence, to the sentences of a paragraph, \& to the paragraphs of a composition.'' -- \cite[Chap. 2, Sect. 22, pp. 48--49]{Strunk_White2019}

\section{A Few Matters of Form}
This section is devoted to study \cite[Chap. 3]{Strunk_White2019}.

\paragraph*{Colloquialisms.} ``If you use a colloquialism or a slang word or phrase, simply use it; do not draw attention to it by enclosing it in quotation marks. To do so is to put on airs, as though you were inviting the reader to join you in a select society of those who know better.

\paragraph*{Exclamations.} Do not attempt to emphasize simple statements by using a mark of exclamation.

\begin{example}
	It was a wonderful show! $\to$ It was a wonderful show.
\end{example}
The exclamation mark is to be reserved for use after true exclamations or commands.

\begin{example}
	What a wonderful show!
	
	Halt!
\end{example}

\paragraph*{Headings.} If a manuscript is to be submitted for publication, leave plenty of space at the top of p. 1. The editor will need this space to write directions to the compositor. Place the heading, or title, at least a 4th of the way down the page. Leave a blank line, or its equivalent in space, after the heading. On succeeding pages, begin near the top, but not so near as to give a crowded appearance. Omit the period after a title or heading. A question mark or an exclamation point may be used if the heading calls for it.

\paragraph*{Hyphen.} When 2 or more words are combined to form a compound adjective, a hyphen is usually required.

\begin{example}
	He belonged to the leisure class \& enjoyed leisure-class pursuits.
	
	She entered her boat in the round-the-island race.
\end{example}
Do not use a hyphen between words that can better be written as 1 word: \textit{water-fowl, waterfowl}. Common sense will aid you in the decision, but a dictionary is more reliable. The steady evolution of the language seems to favor union: 2 words eventually become 1, usually after a period of hyphenation.

\begin{example}
	bed chamber $\to$ bed-chamber $\to$ bedchamber; wild life $\to$ wild-life $\to$ wildlife; bell boy $\to$ bell-boy $\to$ bellboy
\end{example}
The hyphen can play tricks on the unwary\footnote{\textbf{unwary} [a] \textbf{1.} [only before noun] not aware of the possible dangers or problems of a situation \& therefore likely to be harmed in some way; \textbf{2.} \textbf{the unwary} [n] [plural] people who are unwary.}, as it did in Chattanooga when 2 newspapers merged -- the \textit{News} \& the \textit{Free Press}. Someone introduced a hyphen into the merger, \& the paper become \textit{The Chattanooga News-Free Press}, which sounds as though the paper were news-free, or devoid\footnote{\textbf{devoid} [a] \textbf{devoid of something} completely lacking in something.} of news. Obviously, we ask too much of a hyphen when we ask it to cast its spell over words it does not adjoin.

\paragraph*{Margins.} Keep righthand \& lefthand margins roughly the same width. Exception: If a great deal of annotating or editing is anticipated, the lefthand margin should be roomy enough to accommodate this work.

\paragraph*{Numerals.} Do not spell out dates or other serial numbers. Write them in figures or in Roman notation, as appropriate.

\begin{example}
	August 9, 1988; Part XII; Rule 3; 352d Infantry
\end{example}
Exception: When they occur in dialogue, most dates \& numbers are best spelled out.

\begin{example}
	``I arrived home on August 9th.''; ``In the year 1990, I turned 21.'' ``Read Chap. 12.''
\end{example}

\paragraph*{Parentheses.} A sentence containing an expression in parentheses is punctuated outside the last mark of parenthesis exactly as if the parenthetical expression were absent. The expression within the marks is punctuated as if it stood by itself, except that the final stop is omitted unless it is a question mark or an exclamation point.

\begin{example}
	I went to her house yesterday (my 3rd attempt to see her), but she had left town.
	
	He declares (\& why should we doubt his good faith?) that he is now certain of success.
\end{example}
(When a wholly detached expression or sentence is parenthesized, the final stop comes before the last mark of parenthesis.)

\paragraph*{Quotations.} Formal quotations cited as documentary evidence are introduced by a colon \& enclosed in quotation marks.

\begin{example}
	The United States Coast Pilot has this to say of the place: ``Bracy Cove, 0.5 mile eastward of Bear Island, is exposed to southeast winds, has a rocky \& uneven bottom, \& is unfit for anchorage.''
\end{example}
A quotation grammatically in apposition or the direct object of a verb is preceded by a comma \& enclosed in quotation marks.

\begin{example}
	I am reminded of the advice of my neighbor, ``Never worry about your heart till it stops beating.''
	
	Mark Twain says, ``A classic is something that everybody wants to have read \& nobody wants to read.''
\end{example}
When a quotation is followed by an attributive phrase, the comma is enclosed within the quotation marks.

\begin{example}
	``I can't attend,'' she said.
\end{example}
Typographical usage dictates that the comma be inside the marks, though logically it often seems not to belong there.

\begin{example}
	``The Fish,'' ``Poetry,'' \& ``The Monkeys'' are in Marianne Moore's Selected Poems.
\end{example}
When quotations of an entire line, or more, of either verse or prose are to be distinguished typographically from text matter, as are the quotations in this book, begin on a fresh line \& indent. Quotation marks should not be used unless they appear in the original, as in dialogue.

\begin{example}
	Worldworth's enthusiasm for the French Revolution was at 1st unbounded:
	
	Bliss was it in that dawn to be alive,
	
	But to be young was very heaven!
\end{example}
Quotations introduced by \textit{that} are indirect discourse \& not enclosed in quotation marks.

\begin{example}
	Keats declares that beauty is truth, truth beauty.
	
	Dickinson states that a coffin is a small domain.
\end{example}
Proverbial expressions \& familiar phrases of literary origin require no quotation marks.

\begin{example}
	These are the times that try men's souls.
	
	He lives far from the madding crowd.
\end{example}

\paragraph*{References.} In scholarly work requiring exact references, abbreviate titles that occur frequently, giving the full forms in an alphabetical list at the end. As a general practice, give the reference in parentheses or in footnotes, not in the body of the sentence. Omit the words \textit{act, scene, line, book, volume, page}, except when referring to only 1 of them. Punctuate as indicated below.

\begin{example}
	in the 2nd scene of the 3rd act $\to$ in III.ii (Better still, simply insert III.ii in parentheses at the proper place in the sentence.)
	
	After the killing of Polonius, Hamlet is placed under guard (IV.ii.14)
	
	2 Samuel i: 17--27
	
	Othello II.iii. 264--267, III.iii. 155--161
\end{example}

\paragraph*{Syllabication.} When a word must be divided at the end of a line, consult a dictionary to learn the syllables between which division should be made. The student will do well to examine the syllable division in a number of pages of any carefully printed book.

\paragraph*{Titles.} For the titles of literary works, scholarly usage prefers italics with capitalized initials. The usage of editors \& publishers varies, some using italics with capitalized initials, others using Roman with capitalized initials \& with or without quotation marks. Use italics (indicated in manuscript by underscoring) except in writing for a periodical that follows a different practice. Omit initial \textit{A} or \textit{The} from titles when you place the possessive before them.

\begin{example}
	A Tale of 2 Cities; \emph{Dickens's} Tale of 2 Cities.
	
	The Age of Innocence; \emph{Wharton's} Age of Innocence.''
\end{example}
-- \cite[Chap. 3, pp. 50--53]{Strunk_White2019}

\section{Words \& Expressions Commonly Misused}
This section is devoted to study \cite[Chap. 4]{Strunk_White2019}.

``Many of the words \& expressions listed here are not so much bad English as bad style, the common places of careless writing. As illustrated under \textit{Feature}, the proper correction is likely to be not the replacement of 1 word or set of words by another but the replacement of vague generality by definite statement.

The shape of our language is not rigid; in questions of usage we have no lawgiver whose word is final. Students whose curiosity is aroused by the interpretations that follow, or whose doubts are raised, will wish to pursue their investigations further. Books useful in such pursuits are \textit{Merriam Webster's Collegiate Dictionary}, 10th Edition; \textit{The American Heritage Dictionary of the English Language}, 3rd Edition; \textit{Webster's 3rd New International Dictionary; The New Fowler's Modern English Usage}, 3rd Edition, edited by R. W. Burchfield; \textit{Modern American Usage: A Guide} by Wilson Follett \& Erik Wensberg; \& \textit{The Careful Writer} by Theodore M. Bernstein.

\begin{enumerate}
	\item \textbf{Aggravate. Irritate.} The 1st means ``to add to'' an already troublesome or vexing matter or condition. The 2nd means ``to vex'' or ``to annoy'' or ``to chafe.''
	\item \textbf{All right.} Idiomatic in familiar speech as a detached phrase in the sense ``Agreed,'' or ``Go ahead,'' or ``O.K.'' Properly written as 2 words -- \textit{all right}.
	\item \textbf{Allude.} Do not confuse with \textit{elude}. You \textit{allude} to a book; you \textit{elude} a pursuer. Note, too, that \textit{allude} is not synonymous with \textit{refer}. An allusion is an indirect mention, a reference is a specific one.
	\item \textbf{Allusion.} Easily confused with \textit{illusion}. The 1st means ``an indirect reference''; the 2nd means ``an unreal image'' or ``a false impression.''
	\item \textbf{Alternate. Alternative.} The words are not always interchangeable as nouns or adjectives. The 1st means every other one in a series; the 2nd, 1 of 2 possibilities. As the other one of a series of 2, an \textit{alternate} may stand for ``a substitute,'' but an \textit{alternative}, although used in a similar sense, connotes a matter of choice that is never present with \textit{alternate}.
	
	\begin{example}
		As the flooded road left them no alternative, they took the alternate route.
	\end{example}
	\item \textbf{Among. Between.} When $\ge 2$ things or persons are involved, \textit{among} is usually called for: ``The money was divided among the 4 players.'' When, however, $\ge 2$ are involved but each is considered individually, \textit{between} is preferred: ``an agreement between the 6 heirs.''
	\item \textbf{And\texttt{/}or.} A device, or shortcut, that damages a sentence \& often leads to confusion or ambiguity.
	
	\begin{example}
		1st of all, would an honor system successfully cut down on the amount of stealing \&\texttt{/}or cheating?
		
		$\hookrightarrow$ 1st of all, would an honor system reduce the incidence of stealing or cheating or both?
	\end{example}
	\item \textbf{Anticipate.} Use \textit{expect} in the sense of simple expectation.
	
	\begin{example}
		I anticipated that he would look older. $\to$ I expected that he would look older.
		
		My brother anticipated the upturn in the market. $\to$ My brother expected the upturn in the market.
	\end{example}
	In the 2nd example, the word \textit{anticipated} is ambiguous. It could mean simply that the brother believed the upturn would occur, or it could mean that he acted in advance of the expected upturn -- by buying stock, perhaps.
	\item \textbf{Anybody.} In the sense of ``any person,'' not to be written as 2 words. \textit{Any body} means ``any corpse,'' or ``any human form,'' or ``any group.'' The rule holds equally for \textit{everybody, nobody, \& somebody}.
	\item \textbf{Anyone.} In the sense of ``anybody,'' written as 1 word. \textit{Any one} means ``any single person'' or ``any single thing.''
	\item \textbf{As good or better than.} Expressions of this type should be corrected by rearranging the sentences.
	
	\begin{example}
		My opinion is as good or better than his. $\to$ My opinion is as good as his, or better (if not better).
	\end{example}
	\item \textbf{As to whether.} \textit{Whether} is sufficient.
	\item \textbf{As yet.} \textit{Yet} nearly always is as good, if not better.
	
	\begin{example}
		No agreement has been reached as yet. $\to$ No agreement has yet been reached.
	\end{example}
	The chief exception is at the beginning of a sentence, where \textit{yet} means something different.
	
	\begin{example}
		Yet (\emph{or} despite everything) he has not succeeded.
		
		As yet (\emph{or} so far) he has not succeeded.
	\end{example}
	\item \textbf{Being.} Not appropriate after \textit{regard $\ldots$ as}.
	
	\begin{example}
		He is regarded as being the best dancer in the club. $\to$ He is regarded as the best dancer in the club.
	\end{example}
	\item \textbf{But.} Unnecessary after \textit{doubt} \& \textit{help}.
	
	\begin{example}
		I have no doubt but that $\to$ I have no doubt that.
		
		He could not help but see that. $\to$ He could not help seeing that.
	\end{example}
	The too-frequent use of \textit{but} as a conjunction leads to the fault discussed under Rule 18. A loose sentence formed with \textit{but} can usually be converted into a periodic sentence formed with \textit{although}.
	
	Particularly awkward is one \textit{but} closely following another, thus making a contrast to a contrast, or a reservation to a reservation. This is easily corrected by rearrangement.
	
	\begin{example}
		Our country has vast resources but seemed almost wholly unprepared for war. But within a year it had created an army of 4 million.
		
		$\hookrightarrow$ Our country seemed almost wholly unprepared for war, but it had vast resources. Within a year it had created an army of 4 million.
	\end{example}
	\item \textbf{Can.} Means ``am (is, are) able.'' Not to be used as a substitute for \textit{may}.
	\item \textbf{Care less.} The dismissive ``I couldn't care less'' is often used with the shortened ``not'' mistakenly (\& mysteriously) omitted: ``I could care less.'' The error destroys the meaning of the sentence \& is careless indeed.
	\item \textbf{Case.} Often unnecessary.
	
	\begin{example}
		In many cases, the rooms lacked air conditioning. $\to$ Many of the rooms lacked air conditioning.
		
		It has rarely been the case that any mistake has been made. $\to$ Few mistakes have been made.
	\end{example}
	\item \textbf{Certainly.} Used indiscriminately\footnote{\textbf{indiscriminately} [adv] \textbf{1.} without thinking about what the result of your actions may be, especially when this causes people to be harmed; \textbf{2.} without careful judgment.} by some speakers, much as others use \textit{very}, in an attempt to intensify any \& every statement. A mannerism of this kind, bad in speech, is even worse in writing.
	\item \textbf{Character.} Often simply redundant, used from a mere habit of wordiness.
	
	\begin{example}
		acts of a hostile character $\to$ hostile acts
	\end{example}
	\item \textbf{Claim.} [v] With object-noun, means ``lay claim to.'' May be used with a dependent clause if this sense is clearly intended: ``She claimed that she was the sole heir.'' (But even here \textit{claimed to be} would be better.) Not to be used as a substitute for \textit{declare, maintain}, or \textit{charge}.
	
	\begin{example}
		He claimed he knew know. $\to$ He declared he knew how.
	\end{example}
	\item \textbf{Clever.} Note that the word means 1 thing when applied to people, another when applied to horses. A clever horse is a good-natured one, not an ingenious one.
	\item \textbf{Compare.} To \textit{compare to} is to point out or imply resemblances between objects regarded as essentially of a different order; to \textit{compare with} is mainly to point out differences between objects regarded as essentially of the same order. Thus, life has been \textit{compared to} a pilgrimage, \textit{to} a drama, \textit{to} a battle; Congress may be \textit{compared with} the British Parliament. Paris has been \textit{compared to} ancient Athens; it may be \textit{compared with} modern London.
	\item \textbf{Comprise.} Literally, ``embrace'': A zoo comprises mammals, reptiles, \& birds (because it ``embraces,'' or ``includes,'' them). But animals do not comprise (``embrace'') a zoo -- they constitute a zoo.
	\item \textbf{Consider.} Not followed by \textit{as} when it means ``believe to be.''
	
	\begin{example}
		I consider him as competent. $\to$ I consider him competent.
	\end{example}
	When \textit{considered} means ``examined'' or ``discussed,'' it is followed by \textit{as}:
	\begin{example}
		The lecturer considered Eisenhower 1st as soldier \& 2nd as administrator.
	\end{example}
	\item \textbf{Contact.} As a transitive verb, the word is vague \& self-important. Do not \textit{contact} people; get in touch with them, look them up, phone them, find them, or meet them.
	\item \textbf{Cope.} An intransitive verb used with \textit{with}. In formal writing, one doesn't ``cope,'' one ``copes with'' something or somebody.
	
	\begin{example}
		I knew they'd cope. (jocular\footnote{\textbf{jocular} [a] (\textit{formal}) \textbf{1.} humorous; \textbf{2.} (of a person) enjoying making people laugh, \textsc{synonym}: \textbf{jolly}.}) $\to$ I knew they would cope with the situation.
	\end{example}
	\item \textbf{Currently.} In the sense of \textit{now} with a verb in the present tense, \textit{currently} is usually redundant; emphasis is better achieved through a more precise reference to time.
	
	\begin{example}
		We are currently reviewing your application. $\to$ We are at this moment reviewing your application.
	\end{example}
	\item \textbf{Data.} Like \textit{strata, phenomena, \& media}, \textit{data} is a plural \& is best used with a plural verb. The word, however, is slowly gaining acceptance as a singular.
	
	\begin{example}
		The data is misleading. $\to$ These data are misleading.
	\end{example}
	\item \textbf{Different than.} Here logic supports established usage: 1 thing differs \textit{from} another, hence, \textit{different from}. Or, \textit{other than, unlike}.
	\item \textbf{Disinterested.} Means ``impartial.'' Do not confuse it with \textit{uninterested}, which means ``not interested in.''
	
	\begin{example}
		Let a disinterested person judge our dispute, (an impartial person)
		
		This man is obviously uninterested in our dispute, (couldn't care less)
	\end{example}
	\item \textbf{Divided into.} Not to be misused for \textit{composed of}. The time is sometimes difficult to draw; doubtless plays are divided into acts, but poems are composed of stanzas. An apple, halved, is divided into sections, but an apple is composed of seeds, flesh, \& skin.
	\item \textbf{Due to.} Loosely used for \textit{through, because of}, or \textit{owning to}, in adverbial phrases.
	
	\begin{example}
		He lost the 1st game due to carelessness. $\to$ He lost the 1st game because of carelessness.
	\end{example}
	In correct use, synonymous with \textit{attributable to}: ``The accident was due to bad weather''; ``losses due to preventable fires.''
	\item \textbf{Each \& every one.} Pitchman's jargon\footnote{\textbf{jargon} [n] [uncountable] (\textit{often disapproving}) words or expressions that are used by a particular profession or group of people, \& are difficult for others to understand.}. Avoid, except in dialogue.
	
	\begin{example}
		It should be a lesson to each \& every one of us. $\to$ It should be a lesson to every one of us (to us all).
	\end{example}
	\item \textbf{Effect.} As a noun, means ``result''; as a verb, means ``to bring about,'' ``to accomplish'' (not to be confused with \textit{affect}, which means ``to influence'').
	
	As a noun, often loosely used in perfunctory writing about fashions, music, painting, \& other arts: ``a Southwestern effect''; ``effects in pale green''; ``very delicate effects''; ``subtle effects''; ``a charming effect was produced.'' The writer who has a definite meaning to express will not take refuge in such vagueness.
	\item \textbf{Enormity.} Use only in the sense of ``monstrous wickedness.'' Misleading, if not wrong, when used to express bigness.
	\item \textbf{Enthuse.} An annoying verb growing out of the noun \textit{enthusiasm}. Not recommended.
	
	\begin{example}
		She was enthused about her new car. $\to$ She was enthusiastic about her new car.
		
		She enthused about her new car. $\to$ She talked enthusiastically (expressed enthusiasm) about her new car.
	\end{example}
	\item \textbf{Etc.} Literally, ``\& other things''; somethings loosely used to mean ``\& other persons.'' The phrase is equivalent to \textit{\& the erst, \& so forth}, \& hence is not to be used if 1 of these would be insufficient -- i.e., if the reader would be left in doubt as to any important particulars. Least open to objection when it represents the last terms of a list already given almost in full, or immaterial words at the end of a quotation.
	
	At the end of a list introduced by \textit{such as, for example}, or any similar expression, \textit{etc.} is incorrect. In formal writing, \textit{etc.} is a misfit. An item important enough to call for \textit{etc.} is probably important enough to be named.
	\item \textbf{Fact.} Use this word only of matters capable of direct verification, not of matters of judgment. That a particular event happened on a given date \& that lead melts at a certain temperature are facts. But such conclusions as that Napoleon was the greatest of modern generals or that the climate of California is delightful, however defensible they may be, are not properly called facts.
	\item \textbf{Facility.} Why must jails, hospitals, \& schools suddenly become ``facilities''?
	
	\begin{example}
		Parents complained bitterly about the fire hazard in the wooden facility.
		
		$\hookrightarrow$ Parents complained bitterly about the fire hazard in the wooden schoolhouse.
		
		He has been appointed warden of the new facility. $\to$ He has been appointed warden of the new prison.
	\end{example}
	\item \textbf{Factor.} A hackneyed\footnote{\textbf{hackneyed} [a] used too often \& therefore boring, \textsc{synonym}: clich\'ed.} word; the expressions of which it is a part can usually be replaced by something more direct \& idiomatic.
	
	\begin{example}
		Her superior training was the great factor in her winning the match. $\to$ She won the match by being better trained.
		
		Air power is becoming an increasingly important factor in deciding battles. $\to$ Air power is playing a larger \& larger part in deciding battles.
	\end{example}
	\item \textbf{Farther. Further.} The 2 words are commonly interchanged, but there is a distinction worth observing: \textit{farther} serves best as a distance word, \textit{further} as a time or quantity word. You chase a ball \textit{farther} than the other fellow; you pursue a subject \textit{further}.
	\item \textbf{Feature.} Another hackneyed word; like \textit{factor}, it usually adds nothing to the sentence in which it occurs.
	
	\begin{example}
		A feature of the entertainment especially worthy of mention was the singing of Allison Jones.
		
		$\hookrightarrow$ (Better use the same number of words to tell what Allison Jones sang \& how she sang it.)
	\end{example}
	As a verb, in the sense of ``offer as a special attraction,'' it is to be avoided.
	\item \textbf{Finalize.} A pompous\footnote{\textbf{pompous} [a] (\textit{disapproving}) showing that you think you are more important than other people, especially by using long \& formal words, \textsc{synonym}: \textbf{pretentious}.}, ambiguous verb. (See Chap. V, Reminder 21.)
	\item \textbf{Fix.} Colloquial in America for \textit{arrange, prepare, mend}. The usage is well established. But bear in mind that this verb is from \textit{figere}: ``to make firm,'' ``to place definitely.'' These are the preferred meanings of the word.
	\item \textbf{Flammable.} \texttt{[Pause at p. 60, move to Chap. 5]}
\end{enumerate}
-- \cite[Chap. 4, pp. 54--73]{Strunk_White2019}

\section{An Approach to Style (With a List of Reminders)}
This section is devoted to study \cite[Chap. 5]{Strunk_White2019}.

``Up to this point, the book has been concerned with what is correct, or acceptable, in the use of English. In this final chapter, we approach style in its broader meaning: style in the sense of what is distinguished \& distinguishing. Here we leave solid ground. Who can confidently say what ignites a certain combination of words, causing them to explode in the mind? Who knows why certain notes in music are capable of stirring the listener deeply, though the same notes slightly rearranged are impotent\footnote{\textbf{impotent} [a] \textbf{1.} having no power to change things or to influence a situation, \textsc{synonym}: \textbf{powerless}; \textbf{2.} (of a man) unable to achieve an erection \& therefore unable to have full sex.}? These are high mysteries, \& this chapter is a mystery story, thinly disguised\footnote{\textbf{disguise} [v] \textbf{1.} to hide the true nature of something so that it cannot be recognized, \textsc{synonym}: conceal; \textbf{2.} disguise somebody\texttt{/}yourself (as somebody\texttt{/}something) to change your appearance so that people cannot recognize you; [n] [countable, uncountable] a thing that you wear or use to change your appearance so that people do not recognize you.}. There is no satisfactory explanation of style, no infallible guide to good writing, no assurance that a person who thinks clearly will be able to write clearly, no key that unlocks the door, no inflexible rule by which writers may shape their course. Writers will often find themselves steering by stars that are disturbingly in motion.

The preceding chapters contain instruction drawn from established English usage; this one contains advice drawn from a writer's experience of writing. Since the book is a rule book, these cautionary remarks, these subtly dangerous hints, are presented in the form of rules, but they are, in essence, mere gentle reminders: they state what most of us know \& at times forget.

Style is an increment\footnote{\textbf{increment} [n] \textbf{1.} an increase or addition, especially 1 of a series; \textbf{2.} a regular increase in salary.} in writing. When we speak of Fitzgerald's style, we don't mean his command of the relative pronoun, we mean the sound his words make on paper. All writers, by the way they use the language, reveal something of their spirits, their habits, their capacities, \& their biases\footnote{\textbf{bias} [n] \textbf{1.} [uncountable, countable] the fact that the results of research or an experiment are not accurate because a particular factor has not been considered when collecting the information; \textbf{2.} [uncountable, countable, usually singular] a strong feeling in favor of or against 1 group of people, or 1 side in an argument, in a way that influences your decisions in an unfair way; \textbf{3.} [countable, usually singular] an interest in 1 area or subject more than others; [v] \textbf{1.} \textbf{bias something} to have an effect on the results of research or an experiment so that they do not show the real situation; \textbf{2.} \textbf{bias somebody\texttt{/}something} to influence somebody's opinions or decisions, sometimes in an unfair way.}. \fbox{This is inevitable as well as enjoyable.} \fbox{All writing is communication}; \fbox{creative writing is communication through revelation -- it is the Self escaping into the open}. \fbox{No writer long remains incognito.}

If you doubt that style is something of a mystery, try rewriting a familiar sentence \& see what happens. Any much-quoted sentence will do. Suppose we take ``These are the times that try men's souls.'' Here we have 8 short, easy words, forming a simple declarative sentence. The sentence contains no flashy ingredient such ass ``Damn the torpedoes\footnote{\textbf{torpedo} [n] a long, narrow bomb that is fired under the water from a ship or submarine \& that explodes when it hits a ship, etc.}!'' \& the words, as you see, are ordinary. Yet in that arrangement, they have shown great durability\footnote{\textbf{durability} [n] [uncountable] the quality of being able to last for a long time without breaking or getting weaker.}; the sentence is into its 3rd century. Now compare a few variations:

\begin{example}
	Times like these try men's souls.
	
	How trying it is to live in these times!
	
	These are trying times for men's souls.
	
	Soulwise, these are trying times.
\end{example}
It seems unlikely that Thomas Paine could have made his sentiment\footnote{\textbf{sentiment} [n] \textbf{1.} [countable, uncountable] a feeling or an opinion; \textbf{2.} [uncountable] (\textit{sometimes disapproving}) feelings of romantic love, sadness, etc. which may be too strong or not appropriate.} stick if he had couched it in any of these forms. But why not? No fault of grammar can be detected in them, \& in every case the meaning is clear. Each version is correct, \& each, for some reason that we can't readily put our finger on, is marked for oblivion\footnote{\textbf{oblivion} [n] [uncountable] \textbf{1.} a state in which you are not aware of what is happening around you, usually because you are unconscious or asleep; \textbf{2.} the state in which somebody\texttt{/}something has been forgotten \& is no longer famous or important, \textsc{synonym}: \textbf{obscurity}; \textbf{3.} a state in which something has been completely destroyed.}. We could, of course, talk about ``rhythm\footnote{\textbf{rhythm} [n] \textbf{1.} [countable] \textbf{rhythm (of something)} a regular pattern of changes or events; \textbf{2.} [countable, uncountable] a strong regular repeated pattern of sounds or movements.}'' \& ``cadence\footnote{\textbf{cadence} [n] \textbf{1.} (\textit{formal}) the rise \& fall of the voice in speaking; \textbf{2.} the end of a musical phrase.},'' but the talk would be vague \& unconvincing\footnote{\textbf{unconvincing} [a] not seeming true or real; not making you believe that something is true, \textsc{opposite}: \textbf{convincing}.}. We could declare \textit{soulwise} to be a silly word, inappropriate to the occasion; but even that won't do -- it does not answer the main question. Are we even sure \textit{soulwise} is silly? If \textit{otherwise} is a serviceable\footnote{\textbf{serviceable} [a] of good enough quality to be used.} word, what's the matter with \textit{soulwise}?

Here is another sentence, this one by a later Tom. It is not a famous sentence, although its author (Thomas Wolfe) is well known. ``Quick are the mouths of earth, \& quick the teeth that fed upon this loveliness.'' The sentence would not take a prize for clarity, \& rhetorically it is at the opposite pole from ``These are the times.'' Try it in a different form, without the inversions:

\begin{example}
	The mouths of earth are quick, \& the teeth that fed upon this loveliness are quick, too.
\end{example}
The author's meaning is still intact, but not his overpowering emotion. What was poetical \& sensuous has become prosy \& wooden; instead of the secret sounds of beauty, we are left with the simple crunch of mastication. (Whether Mr. Wolfe was guilty of overwriting is, of course, another question -- one that is not pertinent here.)

With some writers, style not only reveals the spirit of the man but reveals his identity, as surely as would his fingerprints. Here, following, are 2 brief passages from the works of 2 American novelists. The subject in each case is languor\footnote{\textbf{languor} [n] [uncountable, singular] (\textit{literary}) the pleasant state of feeling lazy \& without energy.}. In both, the words used are ordinary, \& there is nothing eccentric about the construction.

\begin{example}
	He did not still feel weak, he was merely luxuriating in that supremely gutful lassitude of convalescence in which time, hurry, doing, did not exist, the accumulating seconds \& minutes \& hours to which in its well state the body is slave both waking \& sleeping, now reversed \& time now the lip-server \& mendicant to the body's pleasure instead of the body thrall to time's headlong course.
	
	Manuel drank his brandy. He felt sleepy himself. It was too hot to go out into the town. Besides there was nothing to do. He wanted to see Zurito. He would go to sleep while he waited.
\end{example}
Anyone acquainted with Faulkner \& Hemingway will have recognized them in these passages \& perceived which was which. How different are their languors!

Or take 2 American poets, stopping at evening. One stops by woods, the other by laughing flesh.

\begin{example}
	My little horse must think it queer
	
	To stop without a farmhouse near
	
	Between the woods \& frozen lake
	
	The darkest evening of the year.
\end{example}

\begin{example}
	I have perceived that to be with those I like is enough,
	
	To stop in company with the rest at evening is enough,
	
	To be surrounded by beautiful, curious, breathing,
	
	laughing flesh is enough $\ldots$
\end{example}
Because of the characteristic styles, there is little question about identity here, \& if the situations were reversed, with Whitman stopping by woods \& Frost by laughing flesh (not 1 of his regularly scheduled stops), the reader would know who was who.

Young writers often suppose that style is a garnish\footnote{\textbf{garnish} [v] \textbf{garnish something (with something)} to decorate a dish of food with a small amount of another food; [n] [countable, uncountable] a small amount of food that is used to decorate a larger dish of food.} for the meat of prose, a sauce by which a dull dish is made palatable\footnote{\textbf{palatable} [a] \textbf{1.} (of food or drink) having a pleasant or acceptable taste; \textbf{2.} \textbf{palatable (to somebody)} pleasant or acceptable to somebody, \textsc{opposite}: \textbf{unpalatable}.}. Style has no such separate entity; it is nondetachable, unfilterable. The beginner should approach style warily, realizing that it is an expression of self, \& should turn resolutely away from all devices that are popularly believed to indicate style -- all mannerisms, tricks, adornments\footnote{\textbf{adornment} [n] \textbf{1.} [countable] something that you wear to make yourself look more attractive; something used to decorate a place or an object; \textbf{2.} [uncountable] the action of making something\texttt{/}somebody look more attractive by decorating it or them with something.}. The approach to style is by way of plainness, simplicity, orderliness, sincerity.

Writing is, for most, laborious \& slow. The mind travels faster than the pen; consequently, writing becomes a question of learning to make occasional wing shots, bringing down the bird of thought as it flashes by. A writer is a gunner, sometimes waiting in the blind for something to come in, sometimes roaming the countryside hoping to scare something up. Like other gunners, the writer must cultivate patience, working many covers to bring down 1 partridge\footnote{\textbf{partridge} [n] [countable, uncountable] a brown bird with a round body \& a short tail, that people hunt for sport or food; the meat of this bird.}. Here, following, are some suggestions \& cautionary hints that may help the beginner find the way to a satisfactory style.'' -- \cite[Chap. 5, pp. 75--77]{Strunk_White2019}

\subsection{Place yourself in the background}
``Write in a way that draws the reader's attention to the sense \& substance of the writing, rather than to the mood \& temper of the author. If the writing is solid \& good, the mood \& temper of the writer will eventually be revealed \& not at the expense of the work. Therefore, the 1st piece of advice is this: to achieve style, begin by affecting none -- i.e., place yourself in the background. A careful \& honest writer does not need to worry about style. As you become proficient in the use of language, your style will emerge, because you yourself will emerge, \& when this happens you will find it increasingly easy to break through the barriers that separate you from other minds, other hearts -- which is, of course, the purpose of writing, as well as its \fbox{principal reward}. Fortunately, the act of composition, or creation, disciplines the mind; writing is 1 way to go about thinking, \& the practice \& habit of writing not only drain the mind but supply it, too.'' -- \cite[Chap. 5, Sect. 1, p. 78]{Strunk_White2019}

\subsection{Write in a way that comes naturally}
``Write in a way that comes easily \& naturally to you, using words \& phrases that come readily to hand. But do not assume that because you have acted naturally your product is without law.

\fbox{The use of language begins with imitation.} The infant imitates the sounds made by its parents; the child imitates 1st the spoken language, then the stuff of books. The imitative life continues long after the writer is secure in the language, for it is almost impossible to avoid imitating what one admires. Never imitate consciously, but do not worry about being an imitator; take pains instead to admire what is good. Then when you write in a way that comes naturally, you will echo the halloos that bear repeating.''  -- \cite[Chap. 5, Sect. 2, p. 79]{Strunk_White2019}

\subsection{Work from a suitable design}
``Before beginning to compose something, gauge the nature \& extent of the enterprise \& work from a suitable design. (See Chap. II, Rule 12.) Design informs even the simplest structure, whether of brick \& steel or of prose. You raise a pup tent from 1 sort of vision, a cathedral\footnote{\textbf{cathedral} [n] the main church of a district, under the care of a bishop ($=$ a priest of high rank).} from another. This does not a mean that you must sit with a blueprint always in front of you, merely that you had best anticipate wha tyou are getting into. To compose a laundry list, you can work directly from the pile of soiled garments, ticking them off 1 by 1. But to write a biography, you will need at least a rough scheme; you cannot plunge in blindly \& start ticking off fact after fact about your subject, lest you miss the forest for the trees \& there be no end to your labors.

Sometimes, of course, impulse \& emotion are more compelling than design. If you are deeply troubled \& are composing a letter appealing for mercy or for love, you had best not attempt to organize your emotions; the prose will have a better chance if the emotions are left in disarray -- which you'll probably have to do anyway, since feelings do not usually lend themselves to rearrangement. But even the kind of writing that is essentially adventurous \& impetuous will on examination be found to have a secret plan: Columbus didn't just sail, he sailed west, \& the New World took shape from this simple \&, we now think, sensible design.'' -- \cite[Chap. 5, Sect. 3, p. 80]{Strunk_White2019}

\subsection{Write with nouns \& verbs}
``Write with nouns \& verbs, not with adjectives \& adverbs. The adjective hasn't been built that can pull a weak or inaccurate noun out of a tight place. This is not to disparage adjectives \& adverbs; they are indispensable parts of speech. Occasionally they surprise us with their power, as in

\begin{example}
	Up the airy mountain,
	
	Down the rushy glen,
	
	We daren't go a-hunting
	
	For fear of little men $\ldots$
\end{example}
The nouns \textit{mountain} \& \textit{glen} are accurate enough, but had the mountain not become airy, the glen rushy, William Ailing-ham might never have got off the ground with his poem. In general, however, it is nouns \& verbs, not their assistants, that give good writing its toughness \& color.'' -- \cite[Chap. 5, Sect. 4, p. 81]{Strunk_White2019}

\subsection{Revise \& rewrite}
``Revising is part of writing. Few writers are so expert that they can produce what they are after on the 1st try. Quite often you will discover, on examining the completed work, that there are serious flaws in the arrangement of the material, calling for transpositions. When this is the case, a word processor can save you time \& labor as you rearrange the manuscript. You can select material on your screen \& move it to  a more appropriate spot, or, if you cannot find the right spot, you can move the material to the end of the manuscript until you decide whether to delete it. Some writers find that working with a printed copy of the manuscript helps them to visualize the process of change; others prefer to revise entirely on screen. Above all, do not be afraid to experiment with what you have written. Save both the original \& the revised versions; you can always use the computer to restore the manuscript to its original condition, should that course seem best. Remember, it is no sign of weakness or defeat that your manuscript ends up in need of major surgery. This is a common occurrence in all writing, \& among the best writers.'' -- \cite[Chap. 5, Sect. 5, p. 82]{Strunk_White2019}

\subsection{Do not overwrite}
``Rich, ornate\footnote{\textbf{ornate} [a] covered with a lot of decoration, especially when this involves very small or complicated designs.} prose is hard to digest, generally unwholesome, \& sometimes nauseating. If the sickly-sweet word, the overblown phrase are your natural form of expression, as is sometimes the case, you will have to compensate for it by a show of vigor, \& by writing something as meritorious as the Song of Songs, which is Solomon's.

When writing with a computer, you must guard against wordiness. The click \& flow of a word processor can be seductive\footnote{\textbf{seductive} [a] \textbf{1.} sexually attractive; \textbf{2.} attractive in a way that makes you want to have or do something, \textsc{synonym}: \textbf{tempting}.}, \& you may find yourself adding a few unnecessary words or even a whole passage just to experience the pleasure of running your fingers over the keyboard \& watching your words appear on the screen. It is always a good idea to reread your writing later \& ruthlessly delete the excess.'' -- \cite[Chap. 5, Sect. 6, p. 83]{Strunk_White2019}

\subsection{Do not overstate}
``When you overstate, readers will be instantly on guard, \& everything that has preceded your overstatement as well as everything that follows it will be suspect in their minds because they have lost confidence in your judgment on your poise. Overstatement is 1 of the common faults. A single overstatement, wherever or however it occurs, diminishes the whole, \& a single carefree superlative has the power to destroy, for readers, the object of your enthusiasm.'' -- \cite[Chap. 5, Sect. 7, p. 84]{Strunk_White2019}

\subsection{Avoid the use of qualifiers}
``\textit{Rather, very, little, pretty} -- these are the leeches that infest the pond of prose, sucking the blood of words. The constant use of the adjective \textit{little} (except to indicate size) is particularly debilitating; we should all try to do a little better, we should all be very watchful of this rule, for it is a rather important one, \& we are pretty sure to violate it now \& then.'' -- \cite[Chap. 5, Sect. 8, p. 85]{Strunk_White2019}

\subsection{Do not affect a breezy manner}
``The volume of writing is enormous\footnote{\textbf{enormous} [a] extremely large, \textsc{synonym}: \textbf{huge, immense}.}, these days, \& much of it has a short of windiness\footnote{\textbf{windy} [a] \textbf{1.} (of weather, etc.) with a lot of wind; \textbf{2.} (of a place) getting a lot of wind; \textbf{3.} (\textit{informal, disapproving}) (of speech) involving speaking for longer than necessary \& in a way that is complicated \& not clear.} about it, almost as though the author were in a state of euphoria\footnote{\textbf{euphoria} [n] [uncountable] an extremely strong feeling of happiness \& excitement that usually lasts only a short time.}. ``Spontaneous\footnote{\textbf{spontaneous} [a] \textbf{1.} happening naturally, without being made to happen; \textbf{2.} not planned but done because you suddenly want to do it.} me,'' sang Whitman, \&, \&, in his innocence\footnote{\textbf{innocence} [n] [uncountable] \textbf{1.} the fact of not being guilty of a crime, etc., \textsc{opposite}: \textbf{guilt}; \textbf{2.} lack of knowledge \& experience of the world, especially of evil or unpleasant things.}, let loose the hordes of uninspired scribblers who would 1 day confuse spontaneity with genius.

The breezy\footnote{\textbf{breezy} [a] \textbf{1.} with the wind blowing quite strongly; \textbf{2.} having or showing a cheerful \& relaxed manner.}\,\footnote{``Easy breezy'' -- Windranger, DotA 2.} style is often the work of an egocentric\footnote{\textbf{egocentric} [a] thinking only about yourself \& not about what other people need or want, \textsc{synonym}: \textbf{selfish}.}, the person who imagines that everything that comes to mind is of general interest \& that uninhibited prose creates high spirits \& carries the day. Open any alumni\footnote{\textbf{alumni} [n] [plural] (\textit{especially North American English}) the former male \& female students of a school, college or university.} magazine, turn to the class notes, \& you are quite likely to encounter old Spontaneous Me at work -- an aging collegian who writes something like this:

\begin{example}
	Well, guys, here I am again dishing the dirt about your disorderly classmates, after passing a weekend in the Big Apple trying to catch the Columbia hoops tilt \& then a cab-ride from hell through the West Side casbah. \& speaking of news, howzabout tossing a few primo items this way?
\end{example}
This is an extreme example, but the same wind blows, at lesser velocities, across vast expanses of journalistic prose. The author in this case has managed in 2 sentences to commit most of the unpardonable sins: he obviously has nothing to say, he is showing off \& directing the attention of the reader to himself, he is using slang with neither provocation nor ingenuity, he adopts a patronizing air by throwing in the word \textit{primo}, he is humorless (though full of fun), dull, \& empty. He has not done his work. Compare his opening remarks with the following -- a plunge directly into the news:

\begin{example}
	Clyde Crawford, who stroked the varsity shell in 1958, is swinging an oar again after a lapse of 40 years. Clyde resigned last spring as executive sales manager  of the Indiana Flotex Company \& is now a gondolier in Venice.
\end{example}
This, although conventional\footnote{\textbf{conventional} [a] \textbf{1.} [usually before noun] based on what is generally believed; following the way something is usually done; \textbf{2.} (\textit{often disapproving}) tending to follow what is done or considered acceptable by society in general; normal \& ordinary, \& perhaps not very interesting, \textsc{opposite}: \textbf{unconventional}; \textbf{3.} [usually before noun] (especially of weapons) not nuclear; \textbf{4.} (of literature, art or the theater) using a traditional style or method.}, is compact\footnote{\textbf{compact} [a] \textbf{1.} closely \& firmly packed together; \textbf{2.} smaller than is usual for things of the same kind; \textbf{3.} using or filling only a small amount of space; \textbf{4.} (of speech or writing) giving the information that is important using few words or symbols.}, informative\footnote{\textbf{informative} [a] giving useful information.}, unpretentious\footnote{\textbf{unpretentious} [a] (\textit{approving}) not trying to appear more special, intelligent, important, etc. than you really are\texttt{/}it really it, \textsc{opposite}: \textbf{pretentious}.}. The writer has dug up an item of news \& presented it in a straightforward manner. What the 1st writer tried to accomplish by cutting rhetorical capers\footnote{\textbf{caper} [n] \textbf{1.} [usually plural] the small green flower bud of a Mediterranean bush, preserved in vinegar \& used in preparing sauces \& other dishes; \textbf{2.} (\textit{informal}) an activity, especially one that is illegal or dangerous; \textbf{3.} a humorous film that contains a lot of action; \textbf{4.} a short jumping or dancing movement; [v] [intransitive] (\textit{formal}) \textbf{($+$ adv.\texttt{/}prep.)} to run or jump around in a happy \& excited way.} \& by breeziness\footnote{\textbf{breeziness} [n] [uncountable] a cheerful \& relaxed way of behaving.}, the 2nd writer managed to achieve by good reporting, by keeping a tight rein\footnote{\textbf{rein} [n] \textbf{(the reins)} [plural] \textbf{rein (of something)} the state of being in control or the leader of something.} on his material, \& by staying out of the act.'' -- \cite[Chap. 5, Sect. 9, pp. 86--87]{Strunk_White2019}

\subsection{Use orthodox spelling}
``In ordinary composition, use orthodox spelling. Do not write \textit{nite} for \textit{night}, \textit{thru} for \textit{through}, \textit{pleez} for \textit{please}, unless you plan to introduce a complete system of simplified spelling \& are prepared to take the consequences.

In the original edition of \textit{The Elements of Style}, there was a chapter on spelling. In it, the author had this to say:

\begin{quotation}\it
	The spelling of English words is not fixed \& invariable\footnote{\textbf{invariable} [a] always the same; never changing, \textsc{synonym}: \textbf{unchanging}.}, nor does it depend on any other authority\footnote{\textbf{authority} [n] \textbf{1.} [uncountable] the power to give orders to people or to say how things should be done; \textbf{2.} [uncountable] official permission or the right to do something; \textbf{3.} [countable] an organization that has the power to make decisions or that has a particular area of responsibility in a country or region; \textbf{4.} [uncountable] the power to influence people because they respect your knowledge or official position; \textbf{5.} [countable] \textbf{authority (on something)} a person with special knowledge, \textsc{synonym}: \textbf{specialist}.} than general statement. At the present day there is practically\footnote{\textbf{practically} [adv] \textbf{1.} almost; very nearly, \textsc{synonym}: \textbf{virtually}; \textbf{2.} in a realistic or sensible way; in real situations.} unanimous\footnote{\textbf{unanimous} [a] \textbf{1.} if a decision or an opinion is \textbf{unanimous}, it is agreed or shared by everyone in a group; \textbf{2.} \textbf{unanimous (in something)} if a group of people are \textbf{unanimous}, they all agree about something.} agreement as to the spelling of most words $\ldots$ At any given moment, however, a relatively small number of words may be spelled in more than 1 way. Gradually, as a rule, 1 of these forms comes to be generally preferred, \& the less customary form comes to look obsolete \& is discarded. From time to time new forms, mostly simplifications, are introduced by innovators, \& either win their place or die of neglect.
	
	The practical objection to unaccepted \& oversimplified spellings is the disfavor with which they are received by the reader. They distract his attention \& exhaust his patience. He reads the form though automatically, without thought of its needless complexity; he reads the abbreviation tho \& mentally supplies the missing letters, at the cost of a fraction of his attention. The writer has defeated his own purpose.
\end{quotation}
The language manages somehow to keep pace with events. A word that has taken hold in our century is \textit{thru-way}; it was born of necessity \& is apparently here to stay. In combination with \textit{way, thru} is more serviceable than \textit{through}; it is a high-speed word for readers who are going 65. \textit{Throughway} would be too long to fit on a road sign, too slow to serve the speeding eye. It is conceivable that because of our thruways, \textit{through} will eventually become \textit{thru} -- after many more thousands of miles of travel.'' -- \cite[Chap. 5, Sect. 10, p. 88]{Strunk_White2019}

\subsection{Do not explain too much}
``It is seldom advisable to tell all. Be sparing, for instance, in the use of adverbs after ``he said,'' ``she replied,'' \& the like: ``he said consolingly''; ``she replied grumblingly.'' Let the conversation itself disclose the speaker's manner or condition. Dialogue heavily weighted with adverbs after the attributive verb is cluttery\footnote{\textbf{clutter} [v] \textbf{clutter something (up) (with something\texttt{/}somebody)} to fill a place or area with too many things, so that it is untidy; [n] [uncountable, singular] (\textit{disapproving}) a lot of things in an untidy state, especially things that are not necessary or are not being used; a lack of order, \textsc{synonym}: \textbf{mess}.}\,\footnote{\textbf{cluttered} [a] \textbf{cluttered (up) (with somebody\texttt{/}something)} covered with, or full of, a lot of things or people, in a way that is untidy, \textsc{opposite}: \textbf{uncluttered}.} \& annoying. Inexperienced writers not only overwork their adverbs but load their attributives with explanatory verbs: ``he consoled,'' ``she congratulated.'' They do this, apparently, in the belief that the word \textit{said} is always in need of support, or because they have been told to do it by experts in the art of bad writing.'' -- \cite[Chap. 5, Sect. 11, p. 89]{Strunk_White2019}

\subsection{Do not construct awkward adverbs}
``Adverbs are easy to build. Take an adjective or a participle, add \textit{-ly}, \& behold\footnote{\textbf{behold} [v] (\textit{old use or literary}) \textbf{behold somebody\texttt{/}something} to look at or see somebody\texttt{/}something.}! you have an adverb. But you'd probably be better off without it. Do not write \textit{tangledly}. \fbox{The word itself is a tangle.} Do not even write \textit{tiredly}. Nobody says \textit{tangledly} \& not many people say \textit{tiredly}. Words that are not used orally\footnote{\textbf{oral} [a] \textbf{1.} [usually before noun] spoken rather than writing, \textsc{opposite}: \textbf{written}; \textbf{2.} [only before noun] connected with the mouth.} are seldom the ones to put on paper.

\begin{example}
	He climbed tiredly to bed. $\to$ He climbed wearily\footnote{\textbf{wearily} [adv] (\textit{formal}) \textbf{1.} in a way that shows somebody is very tired; \textbf{2.} in a way that shows somebody is annoyed \& bored because they have had to do something, hear something, explian something, etc. many times.} to bed.
	
	The lamp cord lay tangledly beneath her chair. $\to$ The lamp cord lay in tangles beneath her chair.
\end{example}
Do not dress words up by adding \textit{-ly} to them, as though putting a hat on a horse.

\begin{example}
	overly $\to$ over; muchly $\to$ much; thusly $\to$ thus''
\end{example}
-- \cite[Chap. 5, Sect. 12, p. 90]{Strunk_White2019}

\subsection{Make sure the reader knows who is speaking}
``Dialogue is a total loss unless you indicate who the speaker is. In long dialogue passages containing no attributives, the reader may become lost \& be compelled to go back \& reread in order to puzzle the thing out. Obscurity is an imposition on the reader, to say nothing of its damage to the work.

In dialogue, make sure that your attributives do not awkwardly interrupt a spoken sentence. Place them where the break would come naturally in speech -- i.e., where the speaker would pause for emphasis, or take a breath. The best test for locating an attributive is to speak the sentence aloud.

\begin{example}
	``Now, my boy, we shall see,'' he said, ``how well you have learned your lesson.'' $\to$ ``Now, my boy,'' he said, ``we shall see how well you have learned your lesson.''
	
	``What's more, they would never,'' she added, ``consent to the plan.'' $\to$ ``What's more,'' she added, ``they would never consent to the plan.''''
\end{example}
-- \cite[Chap. 5, Sect. 13, p. 91]{Strunk_White2019}

\subsection{Avoid fancy words}
``Avoid the elaborate, the pretentious, the coy, \& the cute. Do not be tempted by a 20-dollar word when there is a 10-center handy, ready \& able. Anglo-Saxon is a livelier tongue than Latin, so use Anglo-Saxon words. In this, as in so many matters pertaining to style, \fbox{one's ear must be one's guide}: \textit{gut} is lustier noun than \textit{intestine}, but the 2 words are not interchangeable, because \textit{gut} is often inappropriate, being too coarse for the context. Never call a stomach a tummy without good reason.

If you admire fancy words, if every sky is \textit{beauteous}, every blonde \textit{curvaceous}, everyone intelligent child prodigious, if you are tickled by \textit{discombobulate}, you will have a bad time with Reminder 14. What is wrong, you ask, with \textit{beauteous}? No one knows, for sure. There is nothing wrong, really, with any word -- all are good, but some are better than others. A matter of ear, a matter of reading the books that sharpen the ear.

The line between the fancy \& the plain, between the atrocious\footnote{\textbf{atrocious} [a] \textbf{1.} very bad or unpleasant, \textsc{synonym}: \textbf{terrible}; \textbf{2.} very cruel \& making you feel shocked.} \& the felicitous\footnote{\textbf{felicitous} [a] (\textit{formal or literary}) chosen well; very suitable; giving a good result, \textsc{synonym}: \textbf{apt, happy}.}, is sometimes alarmingly\footnote{\textbf{alarming} [a] causing worry \& fear.} fine. The opening phrase of the Gettysburg address is close to the line, at least by our standards today, \& Mr. Lincoln, knowingly or unknowingly, was flirting with disaster when he wrote ``4 score \& 7 years ago.'' The President could have got into his sentence with plain ``87'' -- a saving of 2 words \& less of a strain on the listeners' powers of multiplication. But Lincoln's ear must have told him to go ahead with 4 score \& 7. By doing so, he achieved cadence\footnote{\textbf{cadence} [n] \textbf{1.} (\textit{formal}) the ries \& fall of the voice in speaking; \textbf{2.} the end of a musical phrase.} while skirting the edge of fanciness. Suppose he had blundered over the line \& written, ``In the year of our Lord seventeen hundred \& seventy-six.'' His speech would have sustained\footnote{\textbf{sustain} [v] \textbf{1.} \textbf{sustain somebody\texttt{/}something} to provide enough of what somebody\texttt{/}something needs in order to live or exist; \textbf{2.} to make something continue for some time without becoming less, \textsc{synonym}: \textbf{maintain}; \textbf{3.} \textbf{sustain something} (\textit{formal}) to experience something bad, \textsc{synonym}: \textbf{suffer}; \textbf{4.} \textbf{sustain something} to provide evidence to support an opinion, a theory, etc., \textsc{synonym}: \textbf{uphold}; \textbf{5.} \textbf{sustain something} (\textit{law}) to decide that a claim, etc. is valid, \textsc{synonym}: \textbf{uphold}.} a heavy blow. Or suppose he had settle for ``87.'' In that case he would have got into his introductory sentence too quickly; the timing would have been bad.

\fbox{The question of ear is vital.}\footnote{\textbf{vital} [a] \textbf{1.} necessary or essential in order for something to succeed or exist; \textbf{2.} [only before noun] connected with or necessary for staying alive.} Only the writer whose ear is reliable is in a position to use bad grammar deliberately; this writer knows for sure when a colloquialism \footnote{\textbf{colloquialism} [n] a word or phrase that is used in conversation but not in formal speech or writing.} is better formal phrasing \& is able to sustain the work at a level of good taste. So cock\footnote{\textbf{cock} [n] \textbf{1.} (\textit{British English}) (also \textbf{rooster} \textit{North American English, British English}) [countable] an adult male chicken; \textbf{2.} [countable] (especially in compounds) a male of any other bird; \textbf{3.} [countable] (taboo, slang) a penis; \textbf{4.} [countable] (also \textit{stopcock}) a tap that controls the flow of liquid or gas through a pipe; \textbf{5.} [singular] (\textit{British English, old-fashioned, slang}) used as a friendly form of address between men.} your ear. Years ago, students were warned not to end a sentence with a preposition; time, of course, has softened that rigid decree\footnote{\textbf{decree} [n] \textbf{1.} [countable, uncountable] an official order from a ruler or government that becomes the law; \textbf{2.} [countable] a decision that is made in court; [v] to decide, judge or order something officially.}. Not only is the preposition acceptable at the end, sometimes it is more effective in that spot than anywhere else. ``A claw hammer, not an ax, was the tool he murdered her with.'' This is preferable to ``A claw hammer, not an ax, was the tool with which he murdered her.'' Why? Because it sounds more violent, more like murder. \fbox{A matter of ear.}

And would you write ``The worst tennis player around here is I'' or ``The worst tennis player around here is me''? The 1st is good grammar, the 2nd is good judgment -- although the \textit{me} might not do in all contexts.

The split infinitive is another trick of rhetoric in which the ear must be quicker than the handbook. Some infinitives seem to improve on being split, just as a stick of round stovewood does. ``I cannot bring myself to really like the fellow.'' The sentence is relaxed, the meaning is clear, the violation is harmless \& scarcely perceptible\footnote{\textbf{perceptible} [a] \textbf{1.} great enough to be able to be noticed, \textsc{synonym}: \textbf{noticeable}; \textbf{2.} that can be noticed or felt with the senses.}. Put the other way, the sentence becomes stiff, needlessly formal. A matter of ear.

There are times when the ear not only guides us through difficult situations but also saves us from minor or major embarrassments of prose. The ear, e.g., must decide when to omit \textit{that} from a sentence, when to retain it. ``She knew she could do it'' is preferable to ``She knew that she could do it'' -- simpler \& just as clear. But in many cases the \textit{that} is needed. ``He felt that his big nose, which was sunburned, made him look ridiculous.'' Omit the \textit{that} \& you have ``He felt his big nose $\ldots$'''' -- \cite[Chap. 5, Sect. 14, pp. 92--93]{Strunk_White2019}

\subsection{Do not use dialect unless your ear is good}
``Do not attempt to use dialect\footnote{\textbf{dialect} [n] [countable, uncountable] the form of a language that is spoken in 1 area with grammar, words \& pronunciation that may be different from other forms of the same language.} unless you are a devoted student of the tongue\footnote{\textbf{tongue} [n] \textbf{1.} the soft part in the mouth that moves around, used for tasting, swallowing, speaking, etc.; \textbf{2.} (\textit{formal} or \textit{literary}) a language.} you hope to reproduce\footnote{\textbf{reproduce} [v] \textbf{1.} [transitive] \textbf{reproduce something} to produce something again; to make something happen again in the same way; \textbf{2.} [transitive] \textbf{reproduce something} to make a copy of a picture, piece of text, etc.; to include a copy of a picture, etc.; \textbf{3.} [intransitive, transitive] (of people, animals, plants, etc.) to produce young.}. If you use dialect, be consistent. The reader will become impatient or confused upon finding 2 or more versions of the same word or expression. In dialect it is necessary to spell phonetically, or at least ingeniously, to capture unusual inflections. Take, e.g., the word \textit{once}. If often appears in dialect writing as \textit{oncet}, but \textit{oncet} looks as though it should be pronounced ``onset.'' A better spelling would be \textit{wunst}. But if you write it \textit{oncet} once, write it that way throughout. The best dialect writers, by \& large, are economical\footnote{\textbf{economical} [a] \textbf{1.} providing good service or value in relation to the amount of time or money spent; \textbf{2.} using no more of something than is necessary.} of their talents; they use the minimum, not the maximum, of deviation\footnote{\textbf{deviation} [n] \textbf{1.} [uncountable, countable] \textbf{deviation (from something)} a difference from what is expected or usual; \textbf{2.} [countable] \textbf{deviation (from something)} (\textit{statistics}) the amount by which a single measurement is different from a fixed value such as the mean; \textbf{3.} [uncountable] \textbf{deviation (from something)} behavior that is different from what most people consider normal or acceptable.} from the norm, thus sparing their readers as well as convincing them.'' -- \cite[Chap. 5, Sect. 15, p. 94]{Strunk_White2019}

\subsection{Be clear}
``Clarity is not the prize in writing, nor is it always the principal mark of a good style. There are occasions when obscurity\footnote{\textbf{obscurity} [n] \textbf{1.} [uncountable] the state in which somebody\texttt{/}something is not well known or has been forgotten; \textbf{2.} [uncountable, countable, usually plural] \textbf{obscurity (of something)} the quality of being difficult to understand; something that is difficult to understand.} serves a literary yearning\footnote{\textbf{yearning} [n] [countable, uncountable] (\textit{formal}) a strong \& emotional desire, \textsc{synonym}: \textbf{longing}.}, if not a literary purpose, \& there are writers whose mien\footnote{\textbf{mien} [n] [singular] (\textit{formal or literary}) a person's appearance or manner that shows how they are feeling.} is more overcast\footnote{\textbf{overcast} [a] covered with clouds; not bright.} than clear. But since writing is communication, clarity can only be a virtue. \& although there is no substitute for merit in writing, clarity comes closest to being one. Even to a writer who is being intentionally obscure or wild of tongue we can say, ``Be obscure clearly! Be wild of tongue in a way we can understand!'' Even to writers of market letters, telling us (but not telling us) which securities are promising, we can say, ``Be cagey plainly! be elliptical in a straightforward fashion!''

Clarity, clarity, clarity. When you become hopelessly mired in a sentence, it is best to start fresh; do not try to fight your way through against the terrible odds of syntax. Usually what is wrong is that the construction has become too involved at some point; the sentence needs to be broken apart \& replaced by 2 or more shorter sentences.

Muddiness is not merely a disturber of prose, it is also a destroyer of life, of hope: death on the highway caused by a badly worded road sign, heartbreak among lovers caused by a misplaced phrase in a well-intentioned letter, anguish of a traveler expecting to be met at a railroad station \& not being met because of a sliphod telegram. Think of the tragedies that are rooted in ambiguity, \& be clear! When you say something, make sure you have said it. The chances of your having said it are only fair.'' -- \cite[Chap. 5, Sect. 16, p. 95]{Strunk_White2019}

\subsection{Do not inject opinion}
``Unless there is a good reason for its being there, do not inject\footnote{\textbf{inject} [v] \textbf{1.} [transitive, intransitive] to put a drug or another substance into a person's or an animal's body using a syringe; \textbf{2.} [transitive] to put a liquid into something using a syringe or similar equipment; \textbf{3.} [transitive] \textbf{inject something (into something)} to add a particular quality to something; \textbf{4.} [transitive] \textbf{inject something (into something)} to give money to an organization or a project so that it can function.} opinion into a piece of writing. We all have opinions about almost everything, \& the temptation to toss them in is great. To air one's views gratuitously\footnote{\textbf{gratuitously} [adv] (\textit{disapproving}) without any good reason or purpose, in a way that may have harmful effects, \textsc{synonym}: \textbf{unnecessarily}.}, however, is to imply that the demand for them is brisk, which may not be the case, \& which, in any event, may not be relevant to the discussion. Opinions scattered indiscriminately about leave the mark of egotism\footnote{\textbf{egoism} [n] (also \textbf{egotism}) [uncountable] (\textit{disapproving}) the fact of thinking that you are better or more important than anyone else.} on a work. Similarly, to air one's views at an improper time may be in bad taste. If you have received a letter inviting you to speak at the dedication of a new cat hospital, \& you hate cats, your reply, declining the invitation, does not necessarily have to cover the full range of your emotions. You must make it clear that you will not attend, but you do not have to let fly at cats. The writer of the letter asked a civil question; attack cats, then, only if you can do so with good humor, good taste, \& in such a way that your answer will be courteous\footnote{\textbf{courteous} [a] polite, especially in a way that shows respect, \textsc{opposite}: \textbf{discourteous}.} as well as responsive. Since you are out of sympathy with cats, you may quite properly give this as a reason for not appearing at the dedicatory ceremonies of a cat hospital. But bear in mind that your opinion of cats was not sought, only your services as a speaker. Try to keep things straight.'' -- \cite[Chap. 5, Sect. 17, p. 96]{Strunk_White2019}

\subsection{Use figures of speech sparingly}
``The simile\footnote{\textbf{simile} [n] [countable, uncountable] a word or phrase that compares something to something else, using the words \textit{like} or \textit{as}, e.g., \textit{a face like a mask} or \textit{as white as snow}; the use of such words \& phrases.} is a common device \& a useful one, but similes coming in rapid fire, one right on top of another, are more distracting than illuminating. Readers need time to catch their breath; they can't be expected to compare everything with something else, \& no relief in sight.

When you use metaphor\footnote{\textbf{metaphor} [n] [countable, uncountable] \textbf{1.} \textbf{metaphor (for something)} something that represents another situation or idea; \textbf{2.} a word or phrase used to describe somebody\texttt{/}something else, in a way that is different from its normal use, in order to show that the 2 things have the same qualities \& to make the description more powerful; the use of such words \& phrases.}, do not mix it up. I.e., don't start by calling something a swordfish\footnote{\textbf{swordfish} [n] [countable, uncountable] (plural \textbf{swordfish}) a large sea fish with a very long, thin, pointed upper jaw.} \& end by calling it an hourglass\footnote{\textbf{hourglass} [n] a glass container holding sand that takes exactly an hour to pass through a small opening between the top \& bottom sections; [a] [only before noun] a woman who has an \textbf{hourglass} figure, shape, etc. has large breasts \& hips \& a small waist.}.'' -- \cite[Chap. 5, Sect. 18, p. 97]{Strunk_White2019}

\subsection{Do not take shortcuts at the cost of clarity}
``Do not use initials for the names of organizations or movements unless you are certain the initials will be readily understood. Write things out. Not everyone knows that MADD means Mothers Against Drunk Driving, \& even if everyone did, there are babies being born every minute who will someday encounter the name for the 1st time. They deserve to see the words, not simply the initials. A good rule is to start your article by writing out names in full, \& then, later, when you readers have got their bearings, to shorten them.

Many shortcuts are self-defeating\footnote{\textbf{self-defeating} [a] causing more problems \& difficulties instead of solving them; not achieving what you wanted to achieve but having an opposite effect.}; they waste the reader's time instead of conserving it. There are all sorts of rhetorical stratagems\footnote{\textbf{stratagem} [n] (\textit{formal}) a trick or plan that you use to gain an advantage or to trick an opponent.} \& devices that attract writers who hope to be pithy\footnote{\textbf{pithy} [a] (\textit{approving}) (of a comment, piece of writing, etc.) short but expressed well \& full of meaning.}, but most of them are simply bothersome. The longest way round is usually the shortest way home, \& the one truly reliable shortcut in writing is to choose words that are strong \& surefooted\footnote{\textbf{sure-footed} [a] \textbf{1.} not likely to fall when walking or climbing on rough ground; \textbf{2.} confident \& unlikely to make mistakes, especially in difficult situations.} to carry readers on their way.'' -- \cite[Chap. 5, Sect. 19, p. 98]{Strunk_White2019}

\subsection{Avoid foreign languages}
``The writer will occasionally find it convenient or necessary to borrow from other languages. Some writers, however, from sheer exuberance\footnote{\textbf{exuberance} [n] [uncountable] the quality of being full of energy, excitement \& happiness.} or a desire to show off, sprinkle\footnote{\textbf{sprinkle} [v] \textbf{1.} [transitive] to shake small pieces of something or drops of a liquid on something; \textbf{2.} [transitive, usually passive] \textbf{sprinkle something with something} to include a few of something in something else, \textsc{synonym}: \textbf{strew}; \textbf{3.} [intransitive] (\textit{North American English}) if it sprinkles, it rains lightly, \textsc{synonym}: \textbf{drizzle}; [n] \textbf{1.} (also \textbf{sprinkling}) [usually singular] a small amount of a substance that is dropped somewhere, or a number of things or people that are spread or included somewhere; \textbf{2.} (\textit{especially North American English}) light rain.} their work liberally\footnote{\textbf{liberally} [adv] \textbf{1.} in large amounts, \textsc{synonym}: \textbf{freely}; \textbf{2.} in a way that is not completely accurate or exact.} with foreign expressions, with no regard for the reader's comfort. It is a bad habit. Write in English.'' -- \cite[Chap. 5, Sect. 20, p. 99]{Strunk_White2019}

\subsection{Prefer the standard to the offbeat}
``Young writers will be drawn at every turn toward eccentricities\footnote{\textbf{eccentricity} [n] \textbf{1.} [uncountable] behavior that people think is strange or unusual; the quality of being unusual \& different from other people; \textbf{2.} [countable, usually plural] an unusual act or habit.}  in language. They will hear the beat of new vocabularies, the exciting rhythms of special segments\footnote{\textbf{segment} [n] \textbf{1.} a part of something that is separate from the other parts or can be considered separately; \textbf{2.} (\textit{geometry}) part of a shape separated from the rest by at least 1 line or plane; the line between 2 points.} of their society, each speaking a language of its own. All of us come under the spell of these unsettling\footnote{\textbf{unsettling} [a] making you feel upset, nervous or worried.} drums; the problem for beginners is to listen to them, learn the words, feel the vibrations\footnote{\textbf{vibration} [n] [countable, uncountable] \textbf{1.} \textbf{vibration (of something)} a continuous shaking movement; \textbf{2.} \textbf{vibration (of something)} (\textit{physics}) oscillation in a substance about its equilibrium state.}, \& not be carried away.

Youths invariably\footnote{\textbf{invariably} [adv] in every case; every time, \textsc{synonym}: \textbf{always}.} speak to other youths in a tongue of their own devising: they renovate\footnote{\textbf{renovate} [v] \textbf{renovate something} to repair \& paint an old building, a piece of furniture, etc. so that it is in good conditions again.} the language with a wild vigor, as they would a basement apartment. By the time this paragraphs sees print, \textit{psyched, nerd, ripoff, dude, geek, \& funky} will be the words of yesteryear, \& we will be fielding more recent ones that have come bouncing into our speech -- some of them into our dictionary as well. A new word is always up for survival. Many do survive. Others grow stale\footnote{\textbf{stale} [a] \textbf{1.} (of food, especially bread \& cake) no longer fresh \& therefore unpleasant to eat; \textbf{2.} (of air, smoke, etc.) no longer fresh; smelling unpleasant; \textbf{3.} something that is \textbf{stale} has been said or done too many times before \& is no longer interesting or exciting; \textbf{4.} a person who is \textbf{stale} has done the same thing for too long \& so is unable to do it well or produce any new ideas.} \& disappear. Most are, at least in their infancy\footnote{\textbf{infancy} [n] [uncountable] \textbf{1.} the time when a child is a baby or very young; \textbf{2.} the early development of something.}, more appropriate to conversation than to composition.

Today, the language of advertising enjoys an enormous circulation\footnote{\textbf{circulation} [n] \textbf{1.} [uncountable] the movement of blood around the body; \textbf{2.} [uncountable] the movement of something (e.g. air, water or gas) around an area or inside a system or machine; \textbf{3.} [uncountable] the fact of goods, information or ideas passing from 1 person or place to another; \textbf{4.} [countable, usually singular] the usual number of copies of a newspaper or magazine that are sold each day, week, etc.}. With its deliberate\footnote{\textbf{deliberate} [a] done on purpose rather than by accident, \textsc{synonym}: \textbf{intentional}, \textsc{opposite}: \textbf{unintentional}; [v] [intransitive, transitive] to think very carefully about something, usually before making a decision.} infractions\footnote{\textbf{infraction} [n] [countable, uncountable] (\textit{formal}) an act of breaking a rule or law, \textsc{synonym}: \textbf{infringement}.} of grammatical rules \& its crossbreeding\footnote{\textbf{cross-breeding} [n] [uncountable] the activity of making an animal or plant breed ($=$ produce young animals\texttt{/}new plants) with a different type.} of the parts of speech, it profoundly\footnote{\textbf{profoundly} [adv] \textbf{1.} in a way that has a very great effect on somebody\texttt{/}something; \textbf{2.} extremely; \textbf{3.} (\textit{medical}) very seriously; completely.} influences the tongues \& pens of children \& adults. Your new kitchen range is so revolutionary it \textit{obsoletes} all other ranges. Your counter top is beautiful because it is \textit{accessorized} with gold-plated faucets. Your cigarette tastes good \textit{like} a cigarette should. \&, \textit{like the man says}, you will want to try one. You will also, in all probability, want to try writing that way, using that language. You do so at your peril\footnote{\textbf{peril} [n] (\textit{formal or literary}) \textbf{1.} [uncountable] serious danger; \textbf{2.} [countable, usually plural] \textbf{peril (of something)} the fact of something being dangerous or harmful.}, for it is the language of mutilation\footnote{\textbf{mutilation} [n] [uncountable, countable] \textbf{1.} severe damage to somebody's body, especially when part of it is cut or torn off; the act of causing such damage; \textbf{2.} severe damage to something; the act of causing severe damage to something.}.

Advertisers are quite understandably\footnote{\textbf{understandably} [adv] in a way that seems normal \& reasonable in a particular situation, \textsc{synonym}: \textbf{naturally}.} interested in what they call ``attention getting.'' The man photographed must have lost an eye or grown a pink beard, or he must have 3 arms or be sitting wrong-end-to on a horse. This technique is proper in its place, which is the world of selling, but the young writer had best not adopt the device of mutilation in ordinary composition, whose purpose is to engage, not paralyze\footnote{\textbf{paralyse} [v] (\textit{British English}) (\textit{North American English} \textbf{paralyze}) [often passive] \textbf{1.} \textbf{paralyze somebody} to make somebody unable to feel or move all or part of their body; \textbf{2.} \textbf{paralyze something} to prevent something from functioning normally.}, the readers senses. Buy the old-plated faucets if you will, but do not accessorize\footnote{\textbf{accessorize} [v] (\textit{British English also} \textbf{accessorise}) \textbf{accessorize something} to add fashionable items or extra decorations to something, especially to your clothes.} your prose. To use the language well, do not begin by hacking it to bits; accept the whole body of it, cherish its classic form, its variety, \& its richness.

Another segment of society that has constructed a language of its own business. People in business say that toner\footnote{\textbf{toner} [n] [uncountable, countable] \textbf{1.} a type of ink ($=$ colored liquid) used in machines that print or photocopy; \textbf{2.} a liquid or cream used for making the skin on your face tighter \& smoother.} cartridges\footnote{\textbf{cartridges} [n] \textbf{1.} (\textit{North American English also} \textbf{shell}) a tube or case containing explosive \& a bullet or shot, for shooting from a gun; \textbf{2.} a case containing something that is used in a machine, e.g. ink for a printer, film for a camera, etc. Cartridges are put into the machine \& can be removed \& replaced when they are finished or empty; \textbf{3.} a thin tube containing ink ($=$ colored liquid for writing) which you put inside a pen.} are \textit{in short supply}, that they have \textit{updated} the next shipment of these cartridges, \& they they will \textit{finalize} their recommendations at the next meeting of the board. They are speaking a language familiar \& dear to them. Its portentous\footnote{\textbf{portentous} [a] \textbf{1.} (\textit{literary}) important as a sign or a warning of something that is going to happen in the future, especially when it is something unpleasant; \textbf{2.} (\textit{formal, disapproving}) very serious \& intended to impress people, \textsc{synonym}: \textbf{pompous}.} nouns \& verbs invest ordinary events with high adventure; executives walk among toner cartridges, caparisoned\footnote{\textbf{caparisoned} [a] in the past a \textbf{caparisoned} horse or other animal was one covered with a decorated cloth.} like knights. We should tolerate them -- every person of spirit wants to ride a white horse. The only question is whether business vocabulary is helpful to ordinary prose. Usually, the same ideas can be expressed less formidably\footnote{\textbf{formidably} [adv] in a way that makes you feel fear \&\texttt{/}or respect, because something is impressive or powerful or seems very difficult.}, if one makes the effort. A good many of the special words of business seem designed more to express the user's dreams than to express a precise meaning. Not all such words, of course, can be dismissed summarily\footnote{\textbf{summarily} [adv] immediately, without paying attention to the normal process that should be followed.}; indeed, no word in the language can be dismissed offhand by anyone who has a healthy curiosity. \textit{Update} isn't a bad word; in the right setting it is useful. In the wrong setting, though, it is destructive, \& the trouble with adopting coinages too quickly is that they will bedevil\footnote{\textbf{bedevil} [v] (\textit{formal}) \textbf{bedevil somebody\texttt{/}something} to cause a lot of problems for somebody\texttt{/}something over a long period of time, \textsc{synonym}: \textbf{beset}.} one by insinuating\footnote{\textbf{insinuate} [v] \textbf{1.} (\textit{usually disapproving}) to suggest directly that something unpleasant is true, \textsc{synonym}: \textbf{imply}; \textbf{2.} \textbf{insinuate yourself into something} (\textit{formal, disapproving}) to succeed in gaining somebody's respect, trust, etc. so that you can use the situation to your own advantage; \textbf{3.} \textbf{insinuate yourself\texttt{/}something $+$ adv.\texttt{/}prep.} (\textit{formal}) to slowly move yourself or a part of your body into a particular position or place.} themselves where they do not belong. This may sound like rhetorical snobbery\footnote{\textbf{snobbery} [n] the attitudes \& behavior of people who are snobs}, or plain stuffiness\footnote{\textbf{stuffiness} [n] [uncountable] \textbf{1.} (\textit{informal, disapproving}) the fact of being very serious, formal, boring or old-fashioned; \textbf{2.} the fact of being warm in an unpleasant way \& without enough fresh air; \textbf{3.} (\textit{especially North American English}) the fact of having a blocked nose because you have a cold.}; but you will discover, in the course of your work, that the setting of a word is just as restrictive as the setting of a jewel. The general rule here is to prefer the standard. \textit{Finalize}, for instance, is not standard; it is special, \& it is a peculiarly fuzzy \& silly word. Does it mean ``terminate,'' or does it mean ``put into final form''? One can't be sure, really, what it means, \& one gets the impression that the person using it doesn't know, either, \& doesn't want to know.

The special vocabularies of the law, of the military, of government are familiar to most of us. Even the world of criticism has a modest pouch of private words (\textit{luminous, taut}), whose only virtue is that they are exceptionally nimble\footnote{\textbf{nimble} [a] \textbf{1.} able to move quickly \& easily, \textsc{synonym}: \textbf{agile}; \textbf{2.} able to think, react \& adapt quickly.} \& can escape from the garden of meaning over the wall. Of these critical words, Wolcott Gibbs once wrote, ``$\ldots$ they are detached from the language \& inflated like little balloons.'' The young writer should learn to spot them -- words that at 1st glance seem freighted with delicious meaning but that soon burst in air, leaving nothing but a memory of bright sound.

The language is perpetually\footnote{\textbf{perpetual} [a] [usually before noun] \textbf{1.} continuing for a long period of time without interruption, \textsc{synonym}: \textbf{continuous}; \textbf{2.} frequently repeated, \textsc{synonym}: \textbf{continual}.} in flux\footnote{\textbf{flux} [n] \textbf{1.} [countable, uncountable] \textbf{flux (of something)} (\textit{specialist}) a flow; an act of flowing; in physics, \textbf{flux} can be the rate of flow of a liquid, a gas, energy or particles across a particular area; or the total electric or magnetic field passing through a surface; \textbf{2.} [uncountable] continuous movement \& change.}: it is a living stream, shifting, changing, receiving new strength from a thousand tributaries\footnote{\textbf{tributary} [n] a river or stream that flows into a larger river or a lake.}, losing old forms in the backwaters\footnote{\textbf{backwater} [n] \textbf{1.} a part of a river away from the main part, where the water only moves slowly; \textbf{2.} (\textit{often disapproving}) a place that is away from the places where most things happen, \& is therefore not affected by events, progress, new ideas, etc.} of time. To suggest that a young writer not swim in the main stream of this \fbox{turbulence} would be foolish indeed, \& such is not the intent of these cautionary remarks. The intent is to suggest that in choosing between the formal \& the informal, the regular \& the offbeat\footnote{\textbf{offbeat} [a] [usually before noun] (\textit{informal}) different from what most people expect, \textsc{synonym}: \textbf{unconventional}.}, the general \& the special, the orthodox\footnote{\textbf{orthodox} [a] \textbf{1.} (especially of beliefs or behavior) generally accepted or approved of; following generally accepted beliefs, \textsc{synonym}: \textbf{traditional}; \textbf{2.} following closely the traditional beliefs \& practices of a religion; \textbf{3.} (\textbf{Orthodox}) belonging to or connected with the Orthodox Church.} \& the heretical\footnote{\textbf{heretical} [a] \textbf{1.} (of a religious belief or opinion) against the principles of a particular religion; \textbf{2.} (of a belief or opinion) disagreeing strongly with what most people believe.}, the beginner err\footnote{\textbf{err} [v] [intransitive] to make a mistake; \textbf{err on the side of something} [idiom] to show too much of a good quality.} on the side of conservatism\footnote{\textbf{conservatism} [n] [uncountable] \textbf{1.} the tendency to resist great or sudden change; \textbf{2.} the belief that society should change as little as possible; \textbf{3.} (\textbf{Conservatism}) the beliefs of a political party that has traditional ideas about society \& that favors businesses that are privately owned \& that operate with little government control.}, on the side of established usage. No idiom is taboo\footnote{\textbf{taboo} [a] considered so offensive or embarrassing that people must not mention it; [n] \textbf{1.} \textbf{taboo (against\texttt{/}on something)} a cultural or religious custom that does not allow people to do, use or talk about a particular thing; \textbf{2.} \textbf{taboo (against\texttt{/}on something)} a general agreement not to do something or talk about something.}, no accent forbidden\footnote{\textbf{forbidden} [a] not allowed.}; there is simply a better chance of doing well if the writer holds a steady course, enters the stream of English quietly, \& does not thrash\footnote{\textbf{thrash} [v] \textbf{1.} [transitive] \textbf{thrash somebody\texttt{/}something} to hit a person or an animal many times with a stick, etc. as a punishment; \textbf{2.} [intransitive, transitive] to move or make something move in a way that is violent or show a loss of control; \textbf{3.} [transitive] \textbf{thrash somebody\texttt{/}something} (\textit{informal, especially British English}) to defeat somebody very easily in a game.} about.

``But,'' you may ask, ``what if it comes natural to me to experiment rather than conform\footnote{\textbf{conform} [v] \textbf{1.} [intransitive] to behave \& think in the same way as most other people in a group or society; \textbf{2.} [intransitive] to obey a rule or law, \textsc{synonym}: \textbf{comply}; \textbf{conform to something} [phrasal verb] to agree with or match something.}? What if I am a pioneer, or even a genius?'' Answer: then be one. But do not forget that what may seem like pioneering may be merely evasion\footnote{\textbf{evasion} [n] \textbf{1.} [uncountable] the act of not doing something, especially something that legally or morally you should do; \textbf{2.} [countable] a statement that somebody makes that avoids dealing with something or talking about something honestly \& directly; \textbf{3.} [uncountable] \textbf{evasion (of something)} the act of escaping or avoiding somebody\texttt{/}something.}, or laziness -- the disinclination\footnote{\textbf{disinclination} [n] [singular, uncountable] (\textit{formal}) a lack of desire to do something; a lack of enthusiasm for something.} to submit to discipline. Writing good standard English is no cinch\footnote{\textbf{cinch} [n] [singular] (\textit{formal}) \textbf{1.} something that is very easy, \textsc{synonym}: \textbf{doddle}; \textbf{2.} (\textit{especially North American English}) a thing that is certain to happen; a person who is certain to do something; [v] \textbf{1.} \textbf{cinch something} (\textit{especially North American English}) to fasten something tightly around the middle part of your body; to be fastened around the middle part of somebody's body; \textbf{2.} \textbf{cinch something} (\textit{North American English}) to fasten a girth around  a horse; \textbf{3.} \textbf{cinch something} (\textit{North American English, informal}) to make something certain.}, \& before you have managed it you will have encountered enough rough country to satisfy even the most adventurous\footnote{\textbf{adventurous} [a] \textbf{1.} (\textit{North American English also} \textbf{adventuresome}) (of a person) willing to take risks \& try new ideas; enjoying being in new, exciting situations; \textbf{2.} including new \& interesting things, methods \& ideas; \textbf{3.} full of new, exciting or dangerous experiences, \textsc{opposite}: \textbf{unadventurous}.} spirit.

Style takes its final shape more from attitudes of mind than from principles of composition, for, as an elderly practitioner once remarked, \fbox{``Writing is an act of faith, not a trick of grammar.''} This moral observation would have no place in a rule book were it not that style \textit{is} the writer, \& therefore what you are, rather than what you know, will at last determine your style. If you write, you must believe -- in the truth \& worth of the scrawl, in the ability of the reader to receive \& decode the message. No one can write decently\footnote{\textbf{decently} [adv] \textbf{1.} well enough; to a good enough standard or quality; \textbf{2.} honestly \& fairly; in a way that involves treating people with respect; \textbf{3.} in a way that is acceptable in a particular situation.} who is distrustful of the reader's intelligence, or whose attitude is patronizing\footnote{\textbf{patronizing} [a] (\textit{British English also} \textbf{patronising}) (\textit{disapproving}) showing that you think you are better or more intelligent than somebody else, \textsc{synonym}: \textbf{superior}.}.

Many references have been made in this book to ``the reader,'' who has been much in the news. It is now necessary to warn you that your concern for the reader must be pure: you must sympathize\footnote{\textbf{sympathize} [v] (\textit{British English also} \textbf{sympathise}) \textbf{1.} [intransitive] \textbf{sympathize (with somebody\texttt{/}something)} to feel sorry for somebody; to show that you understand \& feel sorry about somebody's problems; \textbf{2.} [intransitive] \textbf{sympathize with somebody\texttt{/}something} to support or approve of somebody\texttt{/}something.} with the reader's plight\footnote{\textbf{plight} [n] [singular] a difficult \& sad situation.} (most readers are in trouble about half the time) but never seek to know the reader's wants. Your whole duty as a writer is to please \& satisfy yourself, \& the true writer always plays to an audience of one. Start sniffing the air, or glancing at the Trend Machine, \& you are as good as dead, although you may make a nice living.

Full of belief, sustained \& elevated\footnote{\textbf{elevated} [a] [usually before noun] \textbf{1.} higher than normal; \textbf{2.} high in rank; \textbf{3.} higher than the area around; above the level of the ground; \textbf{4.} having a high moral or intellectual level.} by the power of purpose, armed with the rules of grammar, you are ready for exposure. At this point, you may well pattern yourself on the fully exposed cow of Robert Louis Stevenson's rhyme\footnote{\textbf{rhyme} [n] \textbf{1.} [uncountable] the use of words in a poem or song that have the same sound, especially at the ends of lines; \textbf{2.} [countable] a word that has the same sound or ends with the same sound as another word; \textbf{3.} [countable] a short poem in which the last word in the line has the same sound as the last word in another line, especially the next one; [v] [intransitive, transitive] (of 2 words of syllables) to have or end with the same sound; to put words that sound the same together, e.g. when writing poetry.}. This friendly \& commendable\footnote{\textbf{commendable} [a] (\textit{formal}) deserving praise \& approval.} animal, you may recall, was ``blown by all the winds that pass\texttt{/}\^ wet with all the showers.'' \& so must you as a young writer be. In our modern idiom, we would say that you must get wet all over. Mr. Stevenson, working in a plainer style, said it with felicity, \& suddenly 1 cow, out of so many, received the gift of immortality. Like the steadfast\footnote{\textbf{steadfast} [a] (\textit{literary, approving}) not changing in your attitudes or aims, \textsc{synonym}: \textbf{firm}.} writer, she is at home in the wind \& the rain; \&, thanks to 1 moment of felicity, she will live on \& on \& on.'' -- \cite[Chap. 5, Sect. 21, pp. 100--103]{Strunk_White2019}

\subsection{Afterword}
``Will Strunk \& E. B. White were unique collaborators\footnote{\textbf{collaborator} [n] \textbf{1.} a person who works with another person to create or produce something such as book; \textbf{2.} \textbf{collaborator (with somebody\texttt{/}something)} a person who helps the enemy in a war, when they have taken control of the person's country.}. Unlike Gilbert \& Sullivan, or Woodward \& Bernstein, they worked separately \& decades apart.

We have no way of knowing whether Prof. Strunk took particular notice of Elwyn Brooks White, a student of his at Cornell University in 19191. Neither teacher nor pupil could have realized that their names would be linked as they now are. Nor could they have imagined that 38 years after they met, White would take this little gem of a textbook that Strunk had written for his students, polish it, expand it, \& transform it into a classic.

E. B. White shared Strunk's sympathy for the reader. To Strunk's do's \& don'ts he added passages about the power of words \& the clear expression of thoughts \& feelings. To the nuts\footnote{\textbf{nut} [n] a small hard fruit with a very hard shell that grows on some trees.} \& bolts\footnote{\textbf{bolt} [n] \textbf{1.} a long, narrow piece of metal that you slide across the inside of a door or window in order to lock it; \textbf{2.} a piece of metal like a thick nail without a point which is used with a circle of metal ($=$ a nut) to fasten things together; \textbf{3.} \textbf{bolt of lightning} a sudden flash of lightning in the sky, appearing as a line; \textbf{4.} a short heavy arrow shot from a crossbow; \textbf{5.} a long piece of cloth wound in a roll around a piece of cardboard.} of grammar he added a rhetorical\footnote{\textbf{rhetorical} [a] \textbf{1.} connected with the art of rhetoric; \textbf{2.} (\textit{often disapproving}) (of a speech or piece of writing) intended to influence people, but not completely honest or sincere; \textbf{3.} (of a question) asked only to make a statement or to produce an effect rather than to get an answer.} dimension.

The editors of this edition have followed in White's footsteps, once again providing fresh examples \& modernizing usage where appropriate. \textit{The Elements of Style} is still a little book, small enough \& important enough to carry in your pocket, as I carry mine. It has helped me to write better. I believe it can do the same for you.'' -- \cite[Afterword by Charles Osgood, p. 104]{Strunk_White2019}

%-----------------------------------------------------------------------------%

\chapter{William Zinsser. \textit{On Writing Well: The Classic Guide to Writing Nonfiction}}

\section*{\href{https://www.amazon.com/Writing-Well-Classic-Guide-Nonfiction/dp/0060891548}{Amazon\texttt{/}On Writing Well: The Classic Guide to Writing Nonfiction by William Zinsser}}
``\textit{On Writing Well} has been praised for its sound advice, its clarity \& the warmth of its style. It is a book for everybody who wants to learn how to write or who needs to do some writing to get through the day, as almost everybody does in the age of e-mail \& the Internet.

Whether you want to write about people or places, science \& technology, business, sports, the arts or about yourself in the increasingly popular memoir genre, \textit{On Writing Well} offers you fundamental principles as well as the insights of a distinguished writer \& teacher. With $>10^6$ copies sold, this volume has stood the test of time \& remains a valuable resource for writers \& would-be writers.''

\subsection*{Popular Highlights in \cite{Zinsser2016}}
\begin{itemize}
	\item ``But the secret of good writing is to strip every sentence to its cleanest components.''
	\item ``Clear thinking becomes clear writing; one can't exist without the other.''
	\item ``Writers must therefore constantly ask: what am I trying to say? Surprisingly often they don't know.''
\end{itemize}

\subsection*{Editorial Reviews}

\subsubsection*{Review}
\begin{itemize}
	\item ``\textit{On Writing Well} belongs on any shelf of serious reference works for writers.'' -- New York Times
	\item ``Not since \textit{The Elements of Style} has there been a guide to writing as well presented \& readable as this one. A love \& respect for the language is evident on every page.'' -- Library Journal
\end{itemize}

\subsubsection*{About William Zinsser}
``\textsc{William Zinsser} is a writer, editor \& teacher. He began his career on the New York \textit{Herald Tribune} \& has since written regularly for leading magazines. During the 1970s he was master of Branford College at Yale. His 17 books, ranging from baseball to music to American travel, include the influential \textit{Writing to Learn} \& \textit{Writing About Your Life}. He teaches at the New School in New York.''

\section*{Introduction}
``1 of the pictures hanging in my office in mid-Manhattan is a photograph of the writer \textsc{E. B. White}. It was taken by \textsc{Jill Krementz} when White was 77 years old, at his home in North Brooklin, Maine. A white-haired man is sitting on a plain\footnote{\textbf{plain} [a] (\textbf{plainer, plainest}) \textbf{1.} easy to see or understand, \textsc{synonym}: \textbf{clear}; \textbf{2.} [only before noun] expressed in a clear \& simple way, without using technical language; \textbf{3.} not trying to deceive anyone; honest \& direct; \textbf{4.} not decorated or complicated; simple. In computing, \textbf{plain text} is data representing text that is not written in code or using special formatting \& can be read, displayed or printed without much processing.; \textbf{5.} without marks or a patter on it; \textbf{6.} [only before noun] (used for emphasis) simple; nothing but, \textsc{synonym}: \textbf{sheer}; [n] (\textbf{plains} [plural]) a large area of flat land.} wooden\footnote{\textbf{wooden} [a] [usually before noun] made of wood.} bench\footnote{\textbf{bench} [n] \textbf{1.} [countable] a long seat for $\ge 2$ people, usually made of wood; \textbf{2.} \textbf{the bench} [singular] (\textit{law}) a judge in court or the seat where he\texttt{/}she sits; the position of being a judge or magistrate; \textbf{3.} [countable, usually plural] (in British parliament) a seat where a particular group of politicians sit; \textbf{4.} \textbf{the bench} [singular] (\textit{sport}) the seats where players sit when they are not playing in the game; \textbf{5.} (also \textbf{workbench}) [countable] a long heavy table used for doing practical jobs, working with tools, etc.} at a plain wooden table -- 3 boards\footnote{\textbf{board} [n] \textbf{1.} [countable $+$ singular or plural verb] \textbf{board (of somebody\texttt{/}something)} a group of people who have power to make decisions \& control a company or other organization; \textbf{2.} [countable] used in the name of some organizations; \textbf{3.} [countable] vertical surface on which to write or attach notices; \textbf{4.} [countable] a thin, flat piece of wood or other stiff material on which to cut things, play games or perform other activities; \textbf{5.} [countable, uncountable] a long thin piece of strong hard material, especially wood, used, e.g., for making floors, building walls \& roofs, \& making boats; \textbf{6.} [uncountable] the meals that are provided when you stay in a place such as a hotel; \textbf{across the board} [idiom] involving or applying to everyone or everything.} nailed\footnote{\textbf{nail} [n] \textbf{1.} the thin hard layer covering the outer tip of the fingers or toes; \textbf{2.} a small thin pointed piece of metal with a flat head, used for joining pieces of wood together or hanging things on a wall; [v] \textbf{1.} \textbf{nail something ($+$ adv.\texttt{/}prep.\texttt{/}adj.)} to fasten something to something with a nail or nails; \textbf{2.} \textbf{nail somebody} (\textit{informal}) to catch somebody \& prove they are guilty of a crime or of doing something bad; \textbf{3.} \textbf{nail a lie, myth, etc.} (\textit{informal}) to prove that something is not true; \textbf{4.} \textbf{nail something} (\textit{informal}) to achieve something or something right, especially in sport.} to 4 legs -- in a small boathouse\footnote{\textbf{boathouse} [n] a building next to a river or lake for keeping a boat in.}. The window is open to a view across the water. White is typing on a manual\footnote{\textbf{manual} [a] \textbf{1.} [only before noun] manual work involves using mainly physical strength; a manual worker has a job that involves mainly physical work; \textbf{2.} [usually before noun] done by somebody with their hands rather than using a machine; \textbf{3.} [only before noun] connected with the hands; [n] \textbf{1.} \textbf{manual (of something)} a book that tells you how to do something; \textbf{2.} a book that describes the parts of a machine \& explains how to operate it.} typewriter\footnote{\textbf{typewriter} [n] a machine that produces writing similar to print. It has keys that you press to make metal letters or signs hit a piece of paper through a long, narrow piece of cloth covered with ink ($=$ colored liquid).}, \& the only other objects are an ashtray\footnote{\textbf{ashtray} [n] a container into which people who smoke put ash, cigarette ends, etc.} \& a nail keg\footnote{\textbf{keg} [n] \textbf{1.} [countable] a round wooden or metal container with a flat top \& bottom, used especially for storing beer, like a barrel but smaller; \textbf{2.} (\textit{British English}) (also \textbf{keg beer} \textit{British \& North American English}) [uncountable] (in the UK) beer served from metal containers, using gas pressure.}. The keg, I don't have to be told, is his wastebasket\footnote{\textbf{wastebasket} [n] (\textit{North American English}) (\textit{British English} \textbf{wastepaper basket}) a basket or other container for waste paper, etc.}.

Many people from many corners\footnote{\textbf{corner} [n] \textbf{1.} a part of something where 2 or more sides, lines or edges join; \textbf{2.} the area inside a room or other space near the place where 2 walls or other surfaces meet; \textbf{3.} the part at the end of the mouth or an eye; \textbf{4.} a place where 2 streets or roads join; \textbf{5.} \textbf{corner of something} a region or an area of a place (sometimes used for one that is far away or difficult to reach).} of my life -- writers \& aspiring\footnote{\textbf{aspire} [v] [intransitive] to have a strong desire to achieve or to become something.} writers, students \& former students -- have seen that picture. They come to talk through a writing problem or to catch me up on their lives. But usually it doesn't take more than a few minutes for their eye to be drawn to the old man sitting at the typewriter. What gets their intention is the simplicity of the process. White has everything he needs: a writing implement\footnote{\textbf{implement} [v] \textbf{implement something} to start to use a new plan, system or law, \textsc{synonym}: \textbf{carry something out}.}, a piece of paper, \& a receptacle\footnote{\textbf{receptacle} [n] \textbf{1.} \textbf{receptacle (for something)} (\textit{formal}) a container for putting something in; \textbf{2.} (also \textbf{outlet} (\textit{both North American English})) (also \textbf{socket} \textit{British \& North American English}) (\textit{British English also} \textbf{power point}) a device in a wall that you put a plug into ($=$ a small plastic object with 2 or 3 metal pins) in order to connect electrical equipment to the power supply of a building.} for all the sentences that didn't come out the way he wanted them to.

Since then writing has gone electronic\footnote{\textbf{electronic} [a] [usually before noun] \textbf{1.} (of a device) having or using many small parts, such as microchips, that control \& direct a small electric current; \textbf{2.} done by means of a computer or other electronic device, especially over a network; \textbf{3.} connected with electronic equipment; \textbf{4.} connected with electron; \textbf{5.} connected with electronics.}. Computers have replaced the typewriter, the delete key has replaced the wastebasket, \& various other keys insert, move \& rearrange whole chunks\footnote{\textbf{chunk} [n] \textbf{1.} a thick, solid piece that has been cut or broken off something; \textbf{2.} (\textit{informal}) a fairly large amount of something; \textbf{3.} (\textit{linguistics}) a phrase or group of words that can be learnt as a unit by somebody who is learning a language.} of text. But nothing has replaced the writer. He or she is still stuck with the same old job of saying something that other people will want to read. That's the point of the photograph, \& it's still the point -- 30 years later -- of this book.

I 1st wrote \textit{On Writing Well} in an outbuilding\footnote{\textbf{outbuilding} [n] [usually plural] a building such as a shed or stable that is built near to, but separate from, a main building.} in Connecticut that was as small \& as crude\footnote{\textbf{crude} [a] (\textbf{cruder, crudest}) \textbf{1.} [usually before noun] (of oil or another natural substance) in its natural state, before it has been treated with chemicals; \textbf{2.} (of figures) not adjusted or corrected; \textbf{3.} (of an estimate or guess) simple \& not very accurate but giving a general idea of something; \textbf{4.} (of an object, machine, etc.) simple \& basic; not showing much skill or attention to detail; \textbf{5.} (of people or the way they behave) offensive or rude, especially about sex.} as White's boathouse. My tools were a dangling\footnote{\textbf{dangle} [v] \textbf{1.} [intransitive, transitive] to hang or move freely; to hold something so that it hangs or moves freely; \textbf{2.} [transitive] \textbf{dangle something (before\texttt{/}in front of somebody)} to offer somebody something good in order to persuade them to do something.} lightbulb\footnote{\textbf{light bulb} [n] (also \textbf{bulb}) the glass part that fits into an electric lamp, etc. to give light when it is switched on.}, an Underwood standard typewriter, a ream\footnote{\textbf{ream} [n] \textbf{1.} \textbf{reams} [plural] (\textit{informal}) a large quantity of writing; \textbf{2.} [countable] (\textit{specialist}) 500 sheets of paper; [v] (\textit{North American English, informal}) \textbf{ream somebody} to treat somebody unfairly or cheat them.} of yellow copy paper \& a wire\footnote{\textbf{wire} [n] \textbf{1.} [uncountable, countable] metal in the form of thin thread; a piece of this; \textbf{2.} [countable, uncountable] a piece of wire that is used to carry an electric current or signal.} wastebasket. I had then been teaching my nonfiction\footnote{\textbf{nonfiction} [n] [uncountable] books, articles, or texts about real facts, people, \& events, \textsc{opposite}: \textbf{fiction}.} writing course at Yale for 5 years, \& I wanted to use the summer of 1975 to try to put the course into a book.

E. B. White, as it happened, was very much on my mind. I had long considered him my model as a writer. His was the seemingly\footnote{\textbf{seemingly} [adv] in a way that appears to be true but many in fact not be, \textsc{synonym}: \textbf{apparently}.} effortless\footnote{\textbf{effortless} [a] needing little or no effect, so that it seems easy.} style -- achieved\footnote{\textbf{achieve} [v] to succeed in reaching a particular goal or result, especially by effort or skill, \textsc{synonym}: \textbf{attain}.}, I knew, with great effort - that I wanted to emulate\footnote{\textbf{emulate} [v] \textbf{1.} \textbf{emulate somebody\texttt{/}something} to try to do something as well as somebody else because you admire them; \textbf{2.} \textbf{emulate something} (\textit{computing}) (of a computer program, etc.) to work in the same way as another computer, etc. \& perform the same tasks.}, \& whenever I began a new project I would 1st read some White to get his cadences\footnote{\textbf{cadence} [n] \textbf{1.} (\textit{formal}) the rise \& fall of the voice in speaking; \textbf{2.} the end of a musical phrase.} into my ear. But now I also had a pedagogical\footnote{\textbf{pedagogic} [a] (also \textbf{pedagogical} BrE) concerning methods of teaching.} interest: White was the reigning\footnote{\textbf{reign} [n] \textbf{1.} \textbf{reign (of somebody)} the period during which a king, queen, emperor, etc. rules; \textbf{2.} \textbf{reign of something} the period during which an idea, a system, etc. has a lot of influence or control; [v] [intransitive] \textbf{1.} to rule as king, queen, emperor, etc.; \textbf{2.} to be the best or most important in a particular situation or area of skill.} champ\footnote{\textbf{champ} [v] [intransitive, transitive] \textbf{champ (something)} (especially of horses) to bite or eat something noisily; [n] an informal way of referring to a champion, often used in newspapers.} of the arena\footnote{\textbf{arena} [n] \textbf{1.} an area of activity, especially one where there is a lot of discussion or argument; \textbf{2.} a place with a flat open area in the middle \& seats around it where people can watch sports \& entertainment.} I was trying to enter. \textit{The Elements of Style}, his updating of the book that had most influenced \textit{him}, written in 1919 by his English professor at Cornell, \textsc{William Strunk Jr.,} was the dominant\footnote{\textbf{dominant} [a] \textbf{1.} stronger, \& having more power \& influence than other things or people, \textsc{synonym}: \textbf{predominant}; \textbf{2.} more common, easier to notice, or more important than other things, \textsc{synonym}: \textbf{predominant}; \textbf{3.} (\textit{ecology}) (of a type of plant or animal) more common in a place than other types of plant or animal; \textbf{4.} (\textit{biology}) connected with a characteristic that appears in an individual even if it only has 1 gene for this characteristic, passed on by only 1 of its parents.} how-to\footnote{\textbf{how-to} [a] [only before noun] providing detailed instructions or advice on how to do something; [n] (plural \textbf{how-tos}) a guide providing detailed instructions or advice on how to do something.} manual for writers. Tough\footnote{\textbf{tough} [a] (\textbf{tougher, toughest}) \textbf{1.} (of a thing) not easily damaged; strong; \textbf{2.} (\textit{rather informal}) having or causing problems, \textsc{synonym}: \textbf{difficult}; \textbf{3.} (\textit{rather informal}) demanding that laws be obeyed, \& not accepting any reasons for not obeying them, \textsc{opposite}: \textbf{soft}; \textbf{4.} (\textit{rather informal}) (of a person) strong enough to deal successfully with difficult conditions or situations. \textbf{Tough} can sometimes suggest that somebody may be violent. The more formal word \textbf{resilient} does not suggest this.} competition\footnote{\textbf{competition} [n] \textbf{1.} [uncountable] (used especially about the world of business) a situation in which somebody\texttt{/}something tries to be more successful than somebody\texttt{/}something else, or tries to get something rather than let somebody\texttt{/}something else get it; \textbf{2.} (\textbf{the competition}) [singular] a person or business that is trying to be more successful than others; goods or services that are intended to be more successful than others; \textbf{3.} [uncountable, countable] (\textit{ecology}) a situation in which animals, plants or other living things try to get resources, with the result that other animals, plants, etc. may not be able to get them; \textbf{4.} [countable] a contest to find out who is the best at something.}.

Instead of competing with the Strunk \& White book I decided to complement\footnote{\textbf{complement} [n] \textbf{1.} something that provides extra qualities, so that it improves or completes something else; \textbf{2.} [usually singular] the complete number or quantity that is possible or normal; \textbf{3.} (\textit{grammar}) a word or phrase, especially an adjective or a noun phrase, that is used after a linking verb such as \textit{be} or \textit{become}, \& describes the subject of the verb. In some descriptions of grammar, a \textbf{complement} is any word or phrase which is governed by a verb, usually coming after the verb in a sentence.; \textbf{4.} \textbf{complement (of something)} (\textit{mathematics}) the members of a set that are not members of a particular subject; [v] to add to something in a way that improves it or completes it.} it. \textit{The Elements of Style} was a book of pointers\footnote{\textbf{pointer} [n] \textbf{1.} a sign that something exists; a sign that shows how something may develop in the future; \textbf{2.} (\textit{rather informal}) a piece of advice; \textbf{3.} (\textit{computing}) a small symbol that marks a point on a computer screen.} \& admonitions\footnote{\textbf{admonition} [n] (\textit{also less frequent} \textbf{admonishment}) [countable, uncountable] (\textit{formal}) a warning to somebody about their behavior.}: do this, don't do that. What it \textit{didn't} address was how to apply those principles to the various forms that nonfiction writing \& journalism\footnote{\textbf{journalism} [n] [uncountable] the work of collecting \& writing new stories for newspapers, magazines, radio or television.} can take. That's what I taught in my course, \& it's what I would teach in my book: how to write about people \& places, science \& technology, history \& medicine, business \& education, sports \& the arts \& everything else under the sun that's waiting to be written about.

So \textit{On Writing Well} was born, in 1976, \& it's now in its 3rd generation of readers, its sales well over a million. Today I often meet young newspaper reporters who were given the book by the editor who hired them, just as those editors were 1st given the book by the editor who hired \textit{them}. I also often meet gray-haired matrons\footnote{\textbf{matron} [n] \textbf{1.} (\textit{British English}) a woman who works as a nurse in a school; \textbf{2.} (\textit{British English}) a senior nurse in charge of the other nurses in a hospital.} who remember being assigned the book in college \& not finding it the horrible medicine they expected. Sometimes they bring that early edition for me to sign, its sentences highlighted in yellow. They apologize for the mess. \fbox{I love the mess.}

As America has steadily\footnote{\textbf{steadily} [adv] \textbf{1.} gradually \& in an even \& regular way; \textbf{2.} without any change or interruption; \textbf{slowly but surely\texttt{/}steadily} [idiom] making slow but definite progress.} changed in 30 years, so has the book. I've revised\footnote{\textbf{revise} [v] \textbf{1.} [transitive] \textbf{revise something} to change something, such as book, process or rule, in order to improve it or make it more suitable; \textbf{2.} [transitive] \textbf{revise something} to change an opinion or a plan, usually because of new information; \textbf{3.} [transitive] \textbf{revise something ($+$ adv.\texttt{/}prep.)} to change something, such as estimate or price, in order to correct or improve it; \textbf{4.} [intransitive] \textbf{revise (for something)} (\textit{British English}) to prepare for an exam by looking again at work that you have done.} it 6 times to keep pace\footnote{\textbf{pace} [n] \textbf{1.} [uncountable, singular] \textbf{pace (of something)} the speed at which something happens; \textbf{2.} [singular, uncountable] the speed at which somebody\texttt{/}something walks, runs or moves; \textbf{3.} [countable] an act of stepping once when walking or running; the distance traveled when doing this.} with new social trends\footnote{\textbf{trend} [n] a general direction in which a situation is changing or developing.} (more interest in memoir\footnote{\textbf{memoir} [n] \textbf{1.} (\textbf{memoirs}) [plural] an account written by somebody, especially somebody famous, about their life \& experiences; \textbf{2.} [countable] \textbf{memoir (of somebody\texttt{/}something)} a written account of somebody's life, a place or an event, written by somebody who knows it well.}, business, science \& sports), new literary trends (more women writing nonfiction\footnote{\textbf{nonfiction} [n] [uncountable] books, articles, or texts about real facts, people, \& events.}), new demographic\footnote{\textbf{demographic} [a] connected with the population \& different groups within it; [n] \textbf{1.} (\textbf{demographics}) [plural] data about the population \& different groups within it; \textbf{2.} [singular] a particular group of people within the population who have a common characteristic.} patterns\footnote{\textbf{pattern} [n] \textbf{1.} the regular way in which something happens or is done; \textbf{2.} a regular arrangement of lines, shapes, colors, etc. found in similar objects or as a design of material, etc. In science, \textbf{pattern formation} is the scientific study of patterns in nature: \textit{Pattern formation is a central process in the study of development}; \textbf{3.} [usually singular] \textbf{pattern (for something)} an example for others to copy; [v] \textbf{pattern something} (\textit{specialist}) to give a clear or regular form to something in nature or society.} (more writers from other cultural\footnote{\textbf{cultural} [a] [usually before noun] \textbf{1.} connected with the customs, beliefs, art, way of life or social organization of a particular country or group; \textbf{2.} connected with activities such as film, literature, music \& art, thought of as a group.} traditions\footnote{\textbf{tradition} [n] [countable, uncountable]} a belief, custom, story of way of doing something that has existed for a long time among a particular group of people; a set of these beliefs, etc.), new technologies\footnote{\textbf{technology} [n] (plural \textbf{technologies}) [uncountable, countable, usually plural] equipment, machines \& processes that are developed using knowledge of engineering \& science; the knowledge used in developing them.} (the computer) \& new words \& usages\footnote{\textbf{usage} [n] \textbf{1.} [uncountable] the fact of something being used; how much something is used; \textbf{2.} [uncountable, countable] the way in which words are used in a language; \textbf{3.} [countable] a custom, practice or habit that people have.}. I've also incorporated\footnote{\textbf{incorporate} [v] \textbf{1.} to include something so that it forms a part of something; \textbf{2.} [usually passive] (\textit{business}) to create a legally recognized company.} lessons I learned by continuing to wrestle\footnote{\textbf{wrestle} [v] \textbf{1.} [intransitive, transitive] to fight somebody by holding them \& trying to throw or force them to the ground, sometimes as a sport; \textbf{2.} [intransitive, transitive] to struggle physically to move or manage something; \textbf{3.} [intransitive] to struggle to deal with something that is difficult, \textsc{synonym}: \textbf{battle, grapple}.} with the craft\footnote{\textbf{craft} [n] \textbf{1.} [countable, uncountable] an activity involving a special skill at making things with your hands; \textbf{2.} [singular] all the skills needed for a particular activity; \textbf{3.} (plural \textbf{craft}) [countable] a boat or ship; [v] [usually passive] \textbf{craft something} to make something using a special skill, \textsc{synonym}: \textbf{fashion}.} myself, writing books on subjects I hadn't tried before: baseball \& music \& American history. My purpose\footnote{\textbf{purpose} [n] \textbf{1.} [countable, uncountable] the aim, intention or function of something; the thing that something is supposed to achieve; \textbf{2.} (\textbf{purposes}) [plural] what is needed or being considered in a particular situation; \textbf{3.} [uncountable, countable] the feeling that what you are doing is valuable; something important that you want to achieve.} is to make myself \& my experience available. If readers connect with my book it's because they don't think they're hearing from an English professor. They're hearing from a working writer.

My concerns as a teacher have also shifted. I'm more interested in the intangibles\footnote{\textbf{intangible} [a] \textbf{1.} that exists but that is difficult to describe, understand or measure, \textsc{opposite}: \textbf{tangible}; \textbf{2.} (\textit{business} that does not exist as a physical thing but is still valuable to a company.)} that produce good writing -- confidence\footnote{\textbf{confidence} [n] [uncountable] \textbf{1.} the feeling that you can trust, believe in \& be sure about the abilities or good qualities of somebody\texttt{/}something; \textbf{2.} a belief in your own ability to do things \& be successful; \textbf{3.} the feeling that you are certain about something; \textbf{4.} \textbf{(in) confidence} a feeling of trust that somebody will keep information private.}, enjoyment\footnote{\textbf{enjoyment} [n] [uncountable] \textbf{1.} \textbf{enjoyment of something} the fact of having \& using something; \textbf{2.} \textbf{enjoyment (of something)} the pleasure that you get from something.}, intention\footnote{\textbf{intention} [n] [countable, uncountable] what you intend or plan to do; your aim.}, integrity\footnote{\textbf{integrity} [n] [uncountable] \textbf{1.} the quality of being honest \& having strong moral principles; \textbf{2.} \textbf{integrity of something} the state of being whole \& not divided, \textsc{synonym}: \textbf{unity}; \textbf{3.} \textbf{integrity of something} the state of not being spoilt, or of not having mistakes.} -- \& I've written new chapters on those values. Since the 1990s I've also taught an adult course on memoir \& family story at the New School. My students are men \& women who want to use writing to try to understand who they are \& what heritage\footnote{\textbf{heritage} [n] [usually singular] \textbf{1.} the history, traditions \& qualities that a country or society has had for many years \& that are considered an important part of its character; \textbf{2.} the country or part of the world where somebody's family originally came from; \textbf{3.} \textbf{$+$ noun} used to describe things of special historical or natural value that are preserved for future generations of a country.} they were born into. Year after year their stories take me deeply into their lives \& into their yearning\footnote{\textbf{yearning} [n] [countable, uncountable] (\textit{formal}) a strong \& emotional desire, \textsc{synonym}: \textbf{longing}.} to leave a record of what they have done \& thought \& felt. Half the people in America, it seems, are writing a memoir.

The bad news is that most of them are \fbox{paralyzed by the size of the task}. How can they even begin to impose\footnote{\textbf{impose} [v] \textbf{1.} to introduce something such as a new law, tax or system; to order that a law or punishment be used; \textbf{2.} \textbf{impose something (on\texttt{/}upon somebody\texttt{/}something)} to make somebody accept or follow the same opinions or beliefs as your own; \textbf{3.} \textbf{impose something (on\texttt{/}upon somebody\texttt{/}something)} to give something that is difficult or unpleasant to somebody\texttt{/}something; \textbf{4.} \textbf{impose yourself (on\texttt{/}upon somebody\texttt{/}something)} to make somebody\texttt{/}something accept you or your ideas.} a coherent\footnote{\textbf{coherent} [a] \textbf{1.} (of an argument, theory, statement or policy) logical \& well organized; easy to understand \& clear, \textsc{opposite}: \textbf{incoherent}; \textbf{2.} (of a person) able to talk \& express yourself clearly; showing this, \textsc{opposite}: \textbf{incoherent}; \textbf{3.} made up of different parts that fit or work well together; \textbf{4.} (\textit{physics}) (of waves) in phase with each other, \textsc{opposite}: \textbf{incoherent}.} shape on the past -- that vast\footnote{\textbf{vast} [a] extremely large in area, size or amount, \textsc{synonym}: \textbf{huge}.} sprawl\footnote{\textbf{sprawl} [v] \textbf{1.} [intransitive] \textbf{($+$ adv.\texttt{/}prep.)} to sit, lie or fall with your arms \& legs spread out in a relaxed or careless way; \textbf{2.} [intransitive] \textbf{$+$ adv.\texttt{/}prep.} to spread in an untidy way; to cover a large area; [n] [uncountable, countable, usually singular] a large area covered with buildings that spreads from the city into the countryside in an ugly way.} of half-remembered people \& events \& emotions? Many are near despair\footnote{\textbf{despair} [n] [uncountable] the feeling of having lost all hope; [v] to stop having any hope that a situation will change or improve.}. To offer some help \& comfort\footnote{\textbf{comfort} [n] \textbf{1.} [uncountable] the state of being physically relaxed \& free from pain; \textbf{2.} [uncountable] the state of having pleasant life, with everything that you need; \textbf{3.} [uncountable] a feeling of not suffering or worrying so much; a feeling of being less unhappy; \textbf{4.} [singular] \textbf{comfort (to somebody)} a person or thing that helps you when you are suffering, worried or unhappy; \textbf{5.} [countable, usually plural] \textbf{comfort (of something)} a thing that makes your life easier or more comfortable; [v] \textbf{comfort somebody} to make somebody who is worried or unhappy feel better by being kind \& sympathetic towards them.} I wrote a book in 2004 called \textit{Writing About Your Life}. It's a memoir of various events in my own life, but it's also a teaching book: along the way I explain the writing decisions I made. They are the same decisions\footnote{\textbf{decision} [n] \textbf{1.} [countable] a choice or judgment that you make after thinking \& talking about what is the best thing to do; \textbf{2.} [uncountable] the process of deciding something.} that confront\footnote{\textbf{confront} [v] \textbf{1.} (of problems or a difficult situation) to appear \& need to be dealt with by somebody, \textsc{synonym}: \textbf{face}; \textbf{2.} \textbf{confront something} to deal with a problem or difficult situation, \textsc{synonym}: \textbf{face up to something}; \textbf{3.} \textbf{confront somebody} to face somebody so that they cannot avoid seeing \& hearing you, especially in an unfriendly or dangerous situation; \textbf{4.} \textbf{confront somebody with somebody\texttt{/}something} to make somebody face or deal with an unpleasant or difficult person or situation.} every writer going in search of his or her past: matters\footnote{\textbf{matter} [n] \textbf{1.} [uncountable] a substance of a particular sort; \textbf{2.} [uncountable] physical substance in general that everything in the world consist of; \textbf{3.} [countable] a subject or situation that you must consider or deal with; \textbf{4.} (\textbf{matters}) [plural] the present situation; the situation that you are talking about, \textsc{synonym}: \textbf{thing}; \textbf{5.} [singular] \textbf{matter of something} a situation that involves something or depends on something, \textsc{synonym}: \textbf{question}; \textbf{6.} [uncountable] written or printed material; [v] [intransitive, transitive] (not used in the progressive tenses) to be important or have an important effect on somebody\texttt{/}something.} of selection\footnote{\textbf{selection} [n] \textbf{1.} [uncountable, countable] the process of choosing somebody\texttt{/}something from a group of people or things, usually according to a system; \textbf{2.} [countable] a number of people or things that have been chosen from a larger group; \textbf{3.} [countable] \textbf{selection (of something)} a collection of things from which something can be chosen, \textsc{synonym}: \textbf{range}; \textbf{4.} [uncountable] (\textit{biology}) (in evolution) a process in which environmental or genetic factors influence which types of living thing are more successful than others.}, reduction\footnote{\textbf{reduction} [n] \textbf{1.} [countable, uncountable] the action or fact of making something smaller or less in amount, size or degree; \textbf{2.} [uncountable] \textbf{reduction of something to something} an explanation of a subject or problem in terms of another simpler or more basic one; \textbf{3.} [countable] an amount of money by which something is made cheaper; \textbf{4.} [uncountable, countable] \textbf{reduction (of something)} (\textit{chemistry}) the fact of removing oxygen from a substance or adding hydrogen to a substance, \textsc{opposite}: \textbf{oxidation}; \textbf{5.} [uncountable, countable] (\textit{chemistry}) the fact of adding 1  or more electrons to a substance.}, organization\footnote{\textbf{organization} [n] (\textit{British English also} \textbf{organisation}) \textbf{1.} [countable] an organized group of people with a particular purpose, such as a business or government department; \textbf{2.} [uncountable] the way in which the different parts of something are arranged, \textsc{synonym}: \textbf{structure}; \textbf{3.} [uncountable] the act of making arrangements or preparations for something, \textsc{synonym}: \textbf{planning}; \textbf{4.} [uncountable] the quality of being arranged in a neat, careful \& logical way; the ability to plan your work or life well \& in an efficient way.} \& tone\footnote{\textbf{tone} [n] \textbf{1.} [singular] the general character \& attitude of something such as piece of writing; the atmosphere of an event; \textbf{2.} [countable] the quality of somebody's voice, especially expressing a particular emotion; \textbf{3.} [countable] the quality of a sound, especially the sound of a musical instrument or one produced by electronic equipment; \textbf{4.} [countable] \textbf{tone (of something)} the extent to which a particular form of a color is light or dark; \textbf{5.} [uncountable] how strong \& firm your muscles or skin are.}. Now, for this 7th edition, I've put the lessons I learned into a new chapter called ``Writing Family History \& Memoir.''

When I 1st wrote \textit{On Writing Well}, the readers I had in mind were a small segment\footnote{\textbf{segment} [n] \textbf{1.} a part of something that is separate from the other parts or can be considered separately; \textbf{2.} (\textit{geometry}) part of a shape separated from the rest by at least 1 line or plane; the line between 2 points; [v] [often passive] \textbf{segment something} (\textit{specialist}) to divide something into different parts.} of the population: students, writers, editors, teachers \& people who wanted to learn how to write. I had no inkling\footnote{\textbf{inkling} [n] [usually singular] a slight knowledge of something that is happening or about to happen, \textsc{synonym}: \textbf{suspicion}.} of the electronic marvels\footnote{\textbf{marvel} [n] \textbf{1.} a wonderful \& surprising person or thing, \textsc{synonym}: \textbf{wonder}; \textbf{2.} \textbf{marvels} [plural] wonderful results or things that have been achieved, \textsc{synonym}: \textbf{wonders}; [v] [intransitive, transitive] \textbf{marvel (at something) $|$ marvel that $\ldots$ $|$ $+$ speech} to be very surprised or impressed by something.} that would soon revolutionize\footnote{\textbf{revolutionize} [v] (\textit{British English also} \textbf{revolutionise}) to completely change the way that something is done.} the act of writing. 1st came the word processor, in the 1990s, which made the computer an everyday tool for people who had never thought of themselves as writers. Then came the Internet \& e-mail, in the 1990s, which continued the revolution\footnote{\textbf{revolution} [n] \textbf{1.} [countable, uncountable] a great change in conditions, ways of working, beliefs, etc. that affects large numbers of people; \textbf{2.} [countable, uncountable] an attempt, by a large number of people, to change the government of a country, especially by violent action; \textbf{3.} [countable, uncountable] a complete circular movement around a point.}. Today everybody in the world is writing to everybody else, making instant\footnote{\textbf{instant} [a] [usually before noun] happening immediately, \textsc{synonym}: \textbf{immediate}; [n] [usually singular] \textbf{1.} a particular point in time, \textsc{synonym}: \textbf{moment}; \textbf{2.} a very short period of time, \textsc{synonym}: \textbf{moment}.} contact\footnote{\textbf{contact} [n] \textbf{1.} [uncountable] the act of communicating with somebody, especially regularly; \textbf{2.} [uncountable] the state of touching something\texttt{/}somebody; \textbf{3.} [uncountable] the state of meeting somebody or experiencing something; \textbf{4.} [countable, usually plural] a person that you know, especially somebody who can be helpful to you in your work; a meeting, communication or relationship with somebody; \textbf{5.} [countable] an electrical connection; \textbf{6.} a person who may be infectious because they have recently been near to somebody with an infectious disease; [v] to communicate with somebody, e.g. by telephone, letter or email.} across\footnote{\textbf{across} [prep] \textbf{1.} from 1 side to the other side of something; \textbf{2.} on the other side of something; \textbf{3.} on or over a part of the body; \textbf{4.} in every part of a place, group of people, etc., \textsc{synonym}: \textbf{throughout}; [adv] from 1 side to the other side; \textbf{across from somebody\texttt{/}something} [idiom] opposite somebody\texttt{/}something.} every border\footnote{\textbf{border} [n] \textbf{1.} the line that divides 2 countries or areas; the land near this line; \textbf{2.} the point at which a quality, subject, type of object, etc. is separated from another; \textbf{3.} \textbf{border (of something)} the edge of something; [v] \textbf{1.} \textbf{border something} (of a country or an area) to be next to or share a border with another country or area; \textbf{2.} \textbf{border something} to form a line along or around the edge of something.} \& across every time zone. Bloggers\footnote{\textbf{blogger} [n] a person who writes a blog.} are saturating\footnote{\textbf{saturate} [v] \textbf{1.} \textbf{saturate something (with something)} to make something completely wet; \textbf{2.} [often passive] \textbf{saturate something (with something)} to fill something completely with something so that it is impossible to add any more.} the globe\footnote{\textbf{globe} [n] \textbf{1.} (\textbf{the globe}) [singular] the world (used especially to emphasize its size); \textbf{2.} [countable] an object shaped like a ball with a map of the world on its surface.}.

On 1 level the new torrent\footnote{\textbf{torrent} [n] \textbf{1.} a large amount of water moving very quickly; \textbf{2.} a large amount of something that comes suddenly \& violently, \textsc{synonym}: \textbf{deluge}.} is good news. Any invention\footnote{\textbf{invention} [n] \textbf{1.} [countable] something that has been created or designed that has not existed before; \textbf{2.} [uncountable] \textbf{invention of something} the act of creating or designing something that has not existed before; \textbf{3.} [countable, uncountable] the act of saying or describing something, \& pretending that it is true, especially in order to deceive people; something that is said or described in this way; \textbf{4.} [uncountable] the ability to have new \& interesting ideas.} that reduces\footnote{\textbf{reduce} [v] \textbf{1.} [transitive, often passive] to make something less or smaller in size, amount or degree; \textbf{2.} [intransitive] \textbf{reduce (from something) (to something)} to become less or smaller sin size, amount or degree; \textbf{3.} [transitive, often passive] \textbf{reduce something (to something)} (\textit{chemistry}) to remove oxygen from a substance or add hydrogen to a substance; \textbf{4.} [transitive, often passive] \textbf{reduce something (to something)} (\textit{chemistry}) to add 1 or more electrons to a substance; to have 1 or more electrons added; \textbf{reduce somebody\texttt{/}something to something} [phrasal verb] [usually passive] to force somebody\texttt{/}something into a worse state or condition; \textbf{reduce something to something $|$ reduce to something} [phrasal verb] to change something to a more general or more simple form; to be changed in this way.} the fear\footnote{\textbf{fear} [n] [uncountable, countable] the bad feeling that you have when you are in danger, when something bad might happen, or when a particular thing frightens you.} of writing is up there with air-conditioning \& the lightbulb. \fbox{But, as always, there's a catch.}\footnote{\textbf{catch} [v] \textbf{1.} \textbf{catch somebody\texttt{/}something} to capture a person or an animal that tries or would try to escape; \textbf{2.} \textbf{catch something} to stop \& hold a moving object, especially in your hands; \textbf{3.} [often passive] to cause somebody to be in a difficult \& usually unexpected situation; \textbf{4.} to find or discover somebody doing something, especially something wrong; \textbf{5.} \textbf{catch somebody's attention, imagination, etc.} if something catches your attention, imagination, etc. you notice it \& feel interested in it, \textsc{synonym}: \textbf{capture}; \textbf{6.} \textbf{catch something} to show or describe something accurately, \textsc{synonym}: \textbf{capture}; \textbf{7.} \textbf{catch sight\texttt{/}a glimpse of somebody\texttt{/}something} to notice somebody\texttt{/}something, if only for a moment; \textbf{8.} \textbf{catch something (from somebody\texttt{/}something)} to get an illness; \textbf{9.} \textbf{catch fire} to begin to burn; \textbf{10.} \textbf{catch something} to be in time for a bus, train, plane, etc. \& get on it; [n] \textbf{1.} \textbf{catch (of something)} an act of catching something; \textbf{2.} an amount of fish that are caught.} Nobody told all the new computer writers that\\\fbox{the essence of writing is rewriting}\footnote{\textbf{essence} [n] [uncountable] \textbf{essence (of something)} the most important quality or feature of something, that makes it what it is; \textbf{in essence} [idiom] in the most important \& basic ways, without considering things that are less important; \textbf{of the essence} [idiom] necessary \& very important.}. Just because they're writing fluently\footnote{\textbf{fluently} [adv] \textbf{1.} if you speak a language or read fluently, you speak or read easily \& well; \textbf{2.} in a way that is smooth \& shows skill.} doesn't mean they're writing well.

That condition was 1st revealed\footnote{\textbf{reveal} [v] \textbf{1.} to make something known to somebody, \textsc{synonym}: \textbf{disclose}; \textbf{2.} to show something that previously could not be seen.} with the arrival\footnote{\textbf{arrival} [n] \textbf{1.} [uncountable, countable] an act of coming or being brought to a place, \textsc{opposite}: \textbf{departure}; \textbf{2.} [countable] a person or thing that comes to a place; \textbf{3.} [uncountable] \textbf{arrival of something} the time when a new technology or idea is introduced.} of the word processor\footnote{\textbf{word processor} [n] a program or machine used to create, edit \& store text documents, usually typed from a keyboard.}. 2 opposite\footnote{\textbf{opposite} [a] \textbf{1.} [usually before noun] as different as possible from something; involving 2 different extremes; \textbf{2.} [usually before noun] on the other side of something or facing something; [n] \textbf{1.} (\textbf{the opposite}) [singular] the situation, idea or activity that is as different from another situation, etc. as it is possible to be, \textsc{synonym}: \textbf{the reverse}; \textbf{2.} (\textbf{opposites}) [plural] people, ideas or situations that are as different as possible from each other; \textbf{the exact opposite} [idiom] a person or thing that is as different as possible from somebody\texttt{/}something else; [prep] on the other side of a particular area from somebody\texttt{/}something, \& usually facing them.} things happened: good writers got better \& bad writers got worse. Good writers welcomed\footnote{\textbf{welcome} [v] \textbf{1.} \textbf{welcome somebody (to something)} to greet somebody in a friendly way when they arrive somewhere; \textbf{2.} \textbf{welcome somebody ($+$ adv.\texttt{/}prep.)} to be pleased that somebody has come or has joined an organization, activity, etc.; \textbf{3.} \textbf{welcome something} to be pleased to receive or accept something; [a] \textbf{1.} that you are pleased to have or receive; \textbf{2.} (of people) accepted or wanted somewhere.} the gift\footnote{\textbf{gift} [n] \textbf{1.} something that you give to somebody without payment, \textsc{synonym}: \textbf{present}; \textbf{2.} a natural ability; \textbf{3.} \textbf{gift (of something)} something that is freely available to somebody \& is good to have; \textbf{4.} [usually singular] \textbf{gift (to\texttt{/}for somebody)} a thing that is very easy to do or an opportunity that somebody should not miss, e.g. because it gives them an advantage.} of being able to fuss\footnote{\textbf{fuss} [n] \textbf{1.} [uncountable, singular] unnecessary excitement, worry or activity; \textbf{2.} [singular] anger or complaints about something, especially something that is not important; [v] \textbf{1.} [intransitive] to do things, or pay too much attention to things, that are not important or necessary; \textbf{2.} [intransitive] \textbf{fuss (about something)} to worry about things that are not very important.} endlessly\footnote{\textbf{endlessly} [adv] in a way that continues for a long time \& seems to have no end.} with their sentences -- pruning\footnote{\textbf{pruning} [n] [uncountable] \textbf{1.} the activity of cutting off some of the branches from a tree, bush, etc. so that it will grow better \& stronger; \textbf{2.} the act of making something smaller by removing parts; the act of cutting out parts of something.} \& revising \& reshaping\footnote{\textbf{reshape} [v] \textbf{reshape something} to change the shape or structure of something.} -- without the drudgery\footnote{\textbf{drudgery} [n] [uncountable] hard boring work.} of retyping. Bad writers became even more verbose because writing was suddenly so easy \& their sentences looked so pretty\footnote{\textbf{pretty} [adv] (with adjectives \& adverbs) (\textit{rather informal}) \textbf{1.} to some extent; fairly; \textbf{2.} very; \textbf{pretty much\texttt{/}well} [idiom] (\textit{rather informal}) almost; almost completely; [a] (\textbf{prettier, prettiest}) \textbf{1.} (used most often about a women or girl) attractive without being very beautiful; \textbf{2.} (of places or things) attractive \& pleasant to look at or to listen to without being large, beautiful or impressive.} on the screen. How could such beautiful sentences not be perfect?

E-mail is an impromptu\footnote{\textbf{impromptu} [adv] without preparation or planning; [a] done without preparation or planning, \textsc{synonym}: \textbf{improvised}.} medium\footnote{\textbf{medium} [n] (plural \textbf{media, mediums}) In academic writing, the plural is usually \textbf{media}; \textbf{1.} a way of communicating information to people; \textbf{2.} something that is used for a particular purpose; \textbf{3.} (\textit{biology}) a substance that something exists in or grows in or that it travels through; [a] [usually before noun] (abbr., \textbf{M}) in the middle between 2 sizes, amounts, times, temperatures, etc.; \textbf{in the long\texttt{/}short\texttt{/}medium run} [idiom] used to describe what will happen a long, short, etc. time in the future; \textbf{in the long\texttt{/}short\texttt{/}medium term} [idiom] used to describe what will happen a long, short, etc. time in the future.}, not conductive\footnote{\textbf{conductive} [a] (\textit{physics}) able to conduct electricity, heat, etc.} to slowing down or looking back. It's ideal\footnote{\textbf{ideal} [a] \textbf{1.} perfect; most suitable; \textbf{2.} [only before noun] the best that can be imagined, but not likely to become real; \textbf{in an ideal\texttt{/}a perfect world} [idiom] used to say that something is what you would like to happen or what should happen, but you know it cannot; [n] \textbf{1.} \textbf{ideal (of somebody\texttt{/}something)} an idea or a standard that seems perfect \& worth trying to achieve; \textbf{2.} [usually singular] \textbf{ideal (of something)} a person or thing considered as perfect.} for the never-ending\footnote{\textbf{never-ending} [a] seeming to last forever, \textsc{synonym}: \textbf{endless, interminable}.} upkeep\footnote{\textbf{upkeep} [n] [uncountable] \textbf{1.} \textbf{upkeep (of something)} the cost or process of keeping something in good condition, \textsc{synonym}: \textbf{maintenance}; \textbf{2.} \textbf{upkeep (of somebody\texttt{/}something)} the cost or process of giving a child or an animal the things that they need.} of daily life. If the writing is disorderly\footnote{\textbf{disorderly} [a] [usually before noun] (\textit{formal}) \textbf{1.} (of people or behavior) showing lack of control; publicly noisy or violent; \textbf{2.} untidy, \textsc{opposite}: \textbf{orderly}.}, no real harm is done. But e-mail is also where much of the world's business is now conducted\footnote{\textbf{conduct} [v] \textbf{1.} \textbf{conduct something} to organize \&\texttt{/}or do a particular activity; \textbf{2.} \textbf{conduct something} (of a substance) to allow heat or electricity to pass along or through it; \textbf{3.} \textbf{conduct yourself $+$ adv.\texttt{/}prep.} (\textit{formal}) to behave in a particular way; [n] [uncountable] (\textit{formal}) \textbf{1.} a person's behavior; \textbf{2.} \textbf{conduct of something} the way in which a business or an activity is organized \& managed.}. Millions of e-mail messages every day give people the information they need to do their job, \& a badly written message can do a lot of damage. So can a badly written Web site. The new age, for all its electronic wizardry\footnote{\textbf{wizardry} [n] [uncountable] a very impressive \& clever achievement; great skill.}, is still writing-based.

\textit{On Writing Well} is a craft book, \& its principles haven't changed since it was written 30 years ago. I don't know what still newer marvels will make writing twice as easy in the next 30 years. But I do know they won't making writing twice as good. That will \fbox{still require plain old hard thinking} -- what E. B. White was doing in his boathouse -- \& the plain old tools of the English language.
\begin{flushright}
	\textsc{William Zinsser}, Apr 2006'' -- \cite[pp. 5--8]{Zinsser2016}
\end{flushright}

\begin{center}
	\LARGE PART I: Principles.
\end{center}

\section{The Transaction}
``A school in Connecticut once held ``a day devoted to the arts,'' \& I was asked if I would come \& talk about writing as a vocation\footnote{\textbf{vocation} [n] \textbf{1.} [countable] \textbf{vocation (as something)} a type of work or way of life that you believe is especially suitable for you, \textsc{synonym}: \textbf{calling}; \textbf{2.} [countable, uncountable] a belief that a particular type of work or way of life is especially suitable for you; a belief that you have been chosen by God to be a priest or nun, \textsc{synonym}: \textbf{calling}.}. When I arrived i found that a 2nd speaker had been invited -- Dr. Brock (as I'll call him), a surgeon\footnote{\textbf{surgeon} [n] a doctor who is trained to perform surgery.} who had recently\footnote{\textbf{recently} [adv] not long ago.} begun to write \& had sold some stories to magazines. He was going to talk about writing as an avocation\footnote{\textbf{avocation} [n] (\textit{formal}) a hobby or other activity that you do for interest \& pleasure.}. That made us a panel\footnote{\textbf{panel} [n] \textbf{1.} [countable $+$ singular or plural verb] a group of experts who give their advice or opinion about something; \textbf{2.} [countable] a square or rectangular piece of wood, glass or metal that forms part of a larger surface such as a door or wall. A \textbf{solar panel} uses light \& heat energy from the sun to produce electricity or heat water.; \textbf{3.} [countable] a flat board in a vehicle or on a piece of machinery where the controls \& instruments are fixed; \textbf{4.} [countable] a section of a page that shows a particular piece of information.}, \& we sat own to face a crowd of students \& teachers \& parents, all eager\footnote{\textbf{eager} [a] very interested \& excited by something that is going to happen or about something that you want to do, \textsc{synonym}: \textbf{keen}.} to learn the secrets\footnote{\textbf{secret} [a] \textbf{1.} known about by only a few people \& kept hidden from others; not done in the presence of other people; \textbf{2.} [only before noun] used to describe actions \& behavior that you do not want other people to know about; \textbf{3.} [only before noun] working secretly against a government's political opponents; [n] \textbf{1.} [countable] something that is known about by only a few people \& not told to others; \textbf{2.} [countable, usually singular] (usually \textbf{the secret}) the best or only way to achieve something; the way a particular person achieves something; \textbf{3.} [countable, usually plural] \textbf{secret (of something)} a thing that is not yet fully understood or that is difficult to understand; \textbf{in secret} [idiom] without other people knowing about it.} of our glamorous\footnote{\textbf{glamorous} [a] (\textit{also informal} \textbf{glam}) especially attractive \& exciting, \& different from ordinary things or people.} work.

Dr. Brock was dressed in a bright red jacket, looking vaguely\footnote{\textbf{vaguely} [adv] \textbf{1.} in a way that is not detailed or exact; \textbf{2.} slightly.} bohemian\footnote{\textbf{bohemian} [n] a person, often somebody who is involved with the arts, who lives in a very informal way without following accepted rules of behavior; [a] living in a very informal way without following accepted rules of behavior, \& often involved in the arts.}, as authors are supposed to look, \& the 1st question went to him. What was it like to be a writer?

He said it was tremendous\footnote{\textbf{tremendous} [a] (\textit{rather informal}) very great, \textsc{synonym}: \textbf{huge}.} fun. Coming home from an arduous\footnote{\textbf{arduous} [a] involving a lot of effort \& energy, especially over a period of time.} day at the hospital, he would go straight to his yellow pad\footnote{\textbf{pad} [n] \textsf{of soft material} \textbf{1.} a thick piece of soft material that is used, e.g., for cleaning or protecting something or for holding liquid; \textsf{of paper} \textbf{2.} a number of pieces of paper for writing or drawing on, that are fastened together at 1 edge; \textsf{of animal's foot} \textbf{3.} the soft part under the foot of a cat, dog, etc.; \textsf{for cleaning} \textbf{4.} a small piece of rough material used for cleaning pans, surfaces, etc.; \textsf{for spacecraft\texttt{/}helicopter} \textbf{5.} a flat surface where a spacecraft or a helicopter takes off \& lands; \textsf{for protection} \textbf{6.} [usually plural] a piece of thick material that you wear in some sports, e.g. football \& cricket, to protect parts of your body; \textsf{of water plants} \textbf{7.} the large flat leaf of some water plants, especially the water lily; \textsf{flat\texttt{/}apartment} \textbf{8.} [usually singular] (\textit{informal}) the place where somebody lives, especially a flat; [v] \textsf{add soft material} \textbf{1.} [transitive, often passive] to put a layer of soft material in or on something in order to protect it, make it thicker or change its shape; \textsf{walk quietly} \textbf{2.} [intransitive] \textbf{$+$ adv.\texttt{/}prep.} to walk with quiet steps; \textsf{bills} \textbf{3.} [transitive] \textbf{pad something} (\textit{North American English}) to dishonestly add items to bills to obtain more money.} \& write his tensions\footnote{\textbf{tension} [n] \textbf{1.} [uncountable, countable, usually plural] a situation in which people do not trust each other, or feel unfriendly towards each other, \& which may cause them to attack each other; \textbf{2.} [countable, uncountable] \textbf{tension (between A \& B)} a situation in which the fact that there are different needs or interests causes difficulties; \textbf{3.} [uncountable] a feeling of anxiety \& stress that makes it impossible to relax; \textbf{4.} [uncountable] the feeling of fear \& excitement that is created by a writer or a film director; \textbf{5.} [uncountable] the state of being stretched tight; the extent to which something is stretched tight.} away. The words just flowed. It was easy. I then said that \fbox{writing wasn't easy \& wasn't fun}. It was \fbox{hard \& lonely, \& the words seldom just flowed}.

Next Dr. Brock was asked if it was important to rewrite. Absolutely not, he said. ``Let it all hang out,'' he told us, \& whatever form the sentences take will reflect the writer at his most natural. I then said that \fbox{rewriting is the essence of writing}. I pointed out that professional\footnote{\textbf{professional} [a] \textbf{1.} [only before noun] connected with a job that needs special training or skill, especially one that needs a high level of education; \textbf{2.} (of people) having a job that needs special training \& a high level of education; \textbf{3.} showing that somebody is well trained \& extremely skilled, \textsc{synonym}: \textbf{competent}; \textbf{4.} suitable or appropriate for somebody working in a particular profession; \textbf{5.} doing something as a paid job rather than just for pleasure; [n] a person who does a job that needs special training \& a high level of education.} writers rewrite their sentences over \& over \& then rewrite what they have rewritten.

``What do you do on days when it isn't going well?'' Dr. Brock was asked. He said he just stopped writing \& put the work aside for a day when it would go better. I then said that the professional writer must establish\footnote{\textbf{establish} [v] \textbf{1.} \textbf{establish something} to start or create an organization, system or practice that will last for a long time, \textsc{synonym}: \textbf{set something up}; \textbf{2.} \textbf{establish something} to start having a relationship, especially a formal one, with another person, group or country; \textbf{3.} to discover or find proof of the facts of a situation, \textsc{synonym}: \textbf{ascertain}; \textbf{4.} \textbf{establish something} to make people accept a principle, claim or custom; \textbf{5.} \textbf{establish somebody\texttt{/}something\texttt{/}yourself (in something) (as something)} to succeed in something well enough to make people accept or respect you or make your future safe.} a daily schedule\footnote{\textbf{schedule} [n] \textbf{1.} [countable, uncountable] a plan for doing something, giving times when events should happen; \textbf{2.} [countable] a plan for the things that a particular person has to do, \& the times when they have to do them, \textsc{synonym}: \textbf{timetable}; \textbf{3.} [countable] a written list of things, e.g. prices of questions; \textbf{4.} (\textit{North American English}) $=$ \textbf{timetable}; \textbf{5.} [countable] (\textit{law}) an addition to a formal document, especially one that is in the form of a list; \textbf{6.} [countable] a list of the television \& radio programmes that are on a particular channel \& the times that they start; [v] [often passive] to arrange for something to happen at a particular time.} \& stick\footnote{\textbf{stick} [v] \textbf{1.} [transitive, intransitive] to fix something to something else, usually with a sticky substance; to become fixed to something in this way; \textbf{2.} [intransitive] (\textit{rather informal}) to become accepted; \textbf{stick out $|$ stick something out} [phrasal verb] to be further out than something else or come through a hole; to push something further out than something else or through a hole; \textbf{stick to something} [phrasal verb] \textbf{1.} to act according to an argument or decision that you have made; \textbf{2.} to continue doing or using something \& not change it; \textbf{stick with something} [phrasal verb] [no passive] to continue with something or continue doing something.} to it. I said that writing is a craft, not an art, \& that the man who runs away from this craft because he lacks inspiration\footnote{\textbf{inspiration} [n] \textbf{1.} [uncountable] the experience of being made to feel confident \& excited about doing something; \textbf{2.} [countable, usually singular] \textbf{inspiration (to somebody)} a person or thing that makes you feel confident \& excited about doing something; \textbf{3.} [uncountable, countable, usually singular] the idea of doing something or the reason for doing something; the person or thing that provides this.} is fooling himself. He is also going broke.

``What if you're feeling depressed\footnote{\textbf{depressed} [a] \textbf{1.} suffering from the medical condition of depression. In non-academic English, \textbf{depression} is often used to refer to a less serious feeling of sadness, which is not considered to be a medical condition.; \textbf{2.} [usually before noun] (of a place or an industry) without enough economic activity or jobs for people; \textbf{3.} having a lower amount or level than usual.} or unhappy\footnote{\textbf{unhappy} [a] (\textbf{unhappier, unhappiest}) (You can also use \textbf{more unhappy} \& \textbf{most unhappy}) \textbf{1.} not happy; sad; \textbf{2.} not pleased or satisfied with something; \textbf{3.} [only before noun] unfortunate or not suitable.}?'' a student asked. ``Won't that affect your writing?''

Probably\footnote{\textbf{probably} [adv] used to say that something is likely to be true or to happen.} it will, Dr. Brock replied. Go fishing. Take a walk. Probably it won't, I said. If your job is to write every day, you learn to do it like any other job.

A student asked if we found it useful to circulate\footnote{\textbf{circulate} [v] \textbf{1.} [intransitive, transitive] to move continuously or freely around a system or area; to cause something to move in this way; \textbf{2.} [intransitive, transitive] (of ideas or information) to pass from place to place or person to person; to pass on ideas or information, especially to all the members of a group.} in the literary world. Dr. Brock said he was greatly enjoying his new life as \fbox{a man of letters}, \& he told several\footnote{\textbf{several} [determiner, pronoun] more than 2 but not very many.} stories of being taken to lunch by his publisher\footnote{\textbf{publisher} [n] a person or company that prepares \& prints books, magazines, newspapers or electronic products \& makes them available to the public.} \& his agent\footnote{\textbf{agent} [n] \textbf{1.} a person whose job is to act for, or manage the affairs of, other people or organizations in business, politics, etc.; \textbf{2.} \textbf{agent for\texttt{/}of something} a person or thing that has an important effect on a situation; \textbf{3.} (\textit{specialist}) a chemical or a substance that produces an effect or a change or is used for a particular purpose; \textbf{4.} (\textit{grammar}) the person or thing that does an action (expressed as the subject of an active verb, or in a `by' phrase with a passive verb).} at Manhattan restaurants where writers \& editors\footnote{\textbf{editor} [n] (abbr., \textbf{ed.}) \textbf{1.} \textbf{editor (of something)} a person who chooses texts written by 1 or by several writers \& prepares them to be published in a book or journal; \textbf{2.} a person who is in charge of a newspaper or magazine, or part of one, \& who decides what should be included; \textbf{3.} \textbf{editor (of something)} a person who prepares a book to be published, e.g. by checking \& correcting the text \& making improvements.} gather\footnote{\textbf{gather} [v] \textbf{1.} [transitive] \textbf{gather something} to collect information from different sources; \textbf{2.} [intransitive, transitive] \textbf{$+$ adv.\texttt{/}prep.} to come together in 1 place to form a group; to bring people together in this way; \textbf{3.} [transitive] \textbf{gather something (together\texttt{/}up)} to bring things together; \textbf{4.} [transitive] \textbf{gather something} to collect plants, fruit, etc. from a wide area; \textbf{5.} [transitive] \textbf{gather something (up)} to pick or cut \& collect crops to be stored; \textbf{6.} [transitive] \textbf{gather something} to increase in speed, force, etc., \textsc{synonym}: \textbf{gain}; \textbf{7.} [transitive] (not used in the progressive tenses) to believe or understand that something is true because of information or evidence that you have, although you have not been told directly.}. I said that\\\fbox{professional writers are solitary drudges who seldom see other writers}\footnote{\textbf{solitary} [a] \textbf{1.} [usually before noun] done alone; without other people; \textbf{2.} (of a person or an animal) usually or frequently spending time alone; \textbf{3.} (of a person, thing or place) alone, with no other people or things around, \textsc{synonym}: \textbf{single}.}\,\footnote{\textbf{drudge} [n] a person who has to do long hard boring jobs.}\,\footnote{\textbf{seldom} [adv] not often, \textsc{synonym}: \textbf{rarely}.}.

``Do you put symbolism\footnote{\textbf{symbolism} [n] [uncountable] \textbf{1.} the use of symbols to represent ideas or qualities; \textbf{2.} \textbf{symbolism (of something)} the symbolic meaning attached to objects or facts.} in your writing?'' a student asked me.

``Not if I can help it,'' I replied. I have an \fbox{unbroken record} of missing the deeper meaning in any story, play or movie, \& as for dance \& mime\footnote{\textbf{mime} [n] (\textit{also less frequent} \textbf{dumbshow}) [uncountable, countable] (especially in the theater) the use of movements of your hands or body \& the expressions on your face to tell a story or to act something without speaking; a performance using this method of acting; [v] \textbf{1.} [transitive, intransitive] to act, tell a story, etc. by moving your body \& face but without speaking; \textbf{2.} [intransitive, transitive] \textbf{mime (to something) $|$ mime (something)} to pretend to sing a song that is actually being sung by somebody else on a recording.}, I have never had any idea of what is being conveyed\footnote{\textbf{convey} [v] \textbf{1.} to communicate information, a message, an idea or a feeling; \textbf{2.} to take, carry or transport somebody\texttt{/}something from 1 place to another; \textbf{3.} (\textit{law}) to change the legal owner of a property or piece of land, \textsc{synonym}: \textbf{transfer}.}.

``I \textit{love} symbols!'' Dr. Brock exclaimed\footnote{\textbf{exclaim} [v] [intransitive, transitive] to say something suddenly \& loudly, especially because of strong emotion or pain.}, \& he described\footnote{\textbf{describe} [v] \textbf{1.} [often passive] to give an account of something in words. \textbf{Described} is often used after a noun phrase, without \textit{that is}\texttt{/}\textit{was}, etc.; \textbf{2.} [often passive] to say what somebody\texttt{/}something is like; to say what somebody\texttt{/}something is; \textbf{3.} \textbf{describe something} to make a movement which has a particular shape; to form a particular shape; \textbf{4.} \textbf{describe something} (\textit{specialist}) (of a diagram of calculation) to represent something.} with gusto\footnote{\textbf{gusto} [n] [uncountable] enthusiasm \& energy in doing something.} the joys\footnote{\textbf{joy} [n] \textbf{1.} [uncountable] a feeling of great happiness, \textsc{synonym}: \textbf{delight}; \textbf{2.} [countable] a person or thing that causes you to feel very happy; \textbf{3.} [uncountable] (\textit{British English, informal}) (in questions \& negative sentences) success or satisfaction.} of weaving\footnote{\textbf{weave} [v] \textbf{1.} [transitive, intransitive] to make cloth by crossing threads or strips across, over \& under each other by hand or by machine; \textbf{2.} [transitive] to put facts, events, details, etc. together to make a story or a closely connected whole.} them through his work.

So the morning went, \& it was a revelation\footnote{\textbf{revelation} [n] \textbf{1.} [countable] a fact that people are made aware of, especially one that has been secret \& is surprising, \textsc{synonym}: \textbf{disclosure}; \textbf{2.} [uncountable] \textbf{revelation (of something)} the act of making people aware of something that has been secret, \textsc{synonym}: \textbf{disclosure}; \textbf{3.} [countable, uncountable] something that is considered to be a sign or message from God.} to all of us. At the end Dr. Brock told me he was enormously\footnote{\textbf{enormously} [adv] very; very much.} interested in my answers -- it had never occurred to him that writing could be hard. I told him I was just as interested in \textit{his} answers -- it had never occurred to me that writing could be easy. Maybe I should take up surgery\footnote{\textbf{surgery} [n] (plural \textbf{surgeries}) \textbf{1.} [uncountable, countable] medical treatment of injuries or diseases that involves cutting open a person's body, sewing up wounds, etc.; \textbf{2.} [countable] (\textit{British English}) a place where a doctor sees patients; \textbf{3.} [countable] (\textit{British English}) a time during which a doctor, an MP or another professional person is available to see people.} on the side.

As for the students, anyone might think we left them bewildered\footnote{\textbf{bewildered} [a] confused.}. But in fact we gave them a broader\footnote{\textbf{broad} [a] (\textbf{broader, broadest}) \textbf{1.} wide; \textbf{2.} including a great variety of things, \textsc{opposite}: \textbf{narrow}; \textbf{3.} [only before noun] general; not detailed; \textbf{4.} with most people agreeing about something in a general way; \textbf{5.} covering a wide area.} glimpse\footnote{\textbf{glimpse} [n] [usually singular] \textbf{1.} a sight of somebody\texttt{/}something for a very short time, when you do not see the person or thing completely; \textbf{2.} a short experience of something that helps you to understand it; [v] \textbf{1.} \textbf{glimpse somebody\texttt{/}something} to see somebody\texttt{/}something for a moment, but not very clearly, \textsc{synonym}: \textbf{catch, spot}; \textbf{2.} \textbf{glimpse something} to start to understand something.} of the writing process than if only 1 of us had talked. For there isn't any ``right'' way to do such personal work. There are all kinds of writers \& all kinds of methods, \& any method that helps you to say what you want to say is the right method for you. Some people write by day, others by night. Some people need silence\footnote{\textbf{silence} [n] \textbf{1.} [countable, uncountable] a situation when no one is speaking; \textbf{2.} [uncountable, singular] a situation in which somebody refuses to talk about something or to answer questions; \textbf{3.} [uncountable] a complete lack of noise or sound; [v] \textbf{silence somebody\texttt{/}something} to make somebody stop expressing opinions that are opposed to yours.}, others turn on the radio. Some write by hand, some by computer, some by talking into a tape recorder. Some people write their 1st draft in 1 long bust\footnote{\textbf{bust} [n] (\textit{rather informal}) a period of economic difficulty or depression.} \& then revise; others can't write the 2nd paragraph until they have fiddled\footnote{\textbf{fiddle} [v] \textbf{1.} [intransitive] \textbf{fiddle (with something)} to keep touching or moving something with your hands, especially because you are bored or nervous; \textbf{2.} [transitive] \textbf{fiddle something} (\textit{informal}) to change the details or figures of something in order to try to get money dishonestly or gain an advantage; \textbf{3.} [intransitive] (\textit{informal}) to play music on the violin; \textbf{fiddle with} [phrasal verb] \textbf{fiddle around with something $|$ fiddle with something} (also \textbf{fiddle about with something} \textit{especially in British English}) \textbf{1.} to keep touching something or making small changes to something because you are not satisfied with it; \textbf{2.} to touch or move the parts of something in order to try to change it or repair it; [n] (\textit{informal}) \textbf{1.} (also \textbf{violin}) [countable] a musical instrument with strings, that you hold under your chin \& play with a bow; \textbf{2.} [countable] (\textit{British English}) something that is done dishonestly to get money, \textsc{synonym}: \textbf{fraud}; \textbf{3.} [singular] (\textit{British English}) something that is difficult to do.} endlessly with the 1st.

But all of them are vulnerable\footnote{\textbf{vulnerable} [a] \textbf{vulnerable (to somebody\texttt{/}something)} weak \& easily hurt physically or emotionally.} \& all of them are tense\footnote{\textbf{tense} [n] any of the forms of a verb that may be used to show the time of the action or state expressed by the verb; [a] \textbf{1.} (of a situation, an event, a period of time, etc.) making people have strong feelings such as anger or anxiety that often cannot be expressed; \textbf{2.} (of a person) nervous or anxious \& unable to relax; \textbf{3.} (of a muscle or other part of the body) stretched tight rather than relaxed.}. They are driven by a compulsion\footnote{\textbf{compulsion} [n] \textbf{1.} [uncountable, countable] strong pressure that makes somebody do something that they do not want to do; \textbf{2.} [countable] \textbf{compulsion to do something} a strong desire to do something, especially something that is dangerous or wrong, \textsc{synonym}: \textbf{urge}.} to put some part of themselves on paper, \& yet they don't just write what comes naturally\footnote{\textbf{naturally} [adv] \textbf{1.} existing or happening as a normal part of nature, without special help, treatment or action by somebody; \textbf{2.} in a way that you would expect, \textsc{synonym}: \textbf{of course}; \textbf{3.} as a normal, logical result of something; \textbf{4.} in a way that shows or uses abilities or qualities that a person or an animal is born with; \textbf{come naturally (to somebody\texttt{/}something)} [idiom] if something comes naturally to you, you are able to do it very easily \& very well.}. They sit down to commit\footnote{\textbf{commit} [v] \textbf{1.} [transitive] \textbf{commit a crime, etc.} to do something wrong or illegal; \textbf{2.} [transitive] \textbf{commit suicide} to kill yourself deliberately; \textbf{3.} [transitive, often passive] to promise that you will definitely do something or keep to an agreement or arrangement; \textbf{4.} [transitive] \textbf{commit yourself (to something)} to give an opinion or make a decision publicly so that it is then difficult to change it; \textbf{5.} [intransitive] \textbf{commit (to somebody\texttt{/}something)} to be completely loyal to 1 person or organization or give your time \& effort to your work or an activity; \textbf{6.} [transitive] \textbf{commit somebody\texttt{/}something (to something)} to spend resources such as money, people or time on something\texttt{/}somebody; \textbf{7.} [transitive, usually passive] \textbf{commit somebody to something} to order somebody to be sent to a hospital; \textbf{8.} [transitive] \textbf{commit something to memory} to learn something well enough to remember it exactly.} an act of literature, \& the self who emerges on paper is far stiffer\footnote{\textbf{stiff} [a] (\textbf{stiffer, stiffest}) \textbf{1.} firm \& difficult to bend or move; \textbf{2.} when a person or part of their body is stiff, their muscles hurt when they move them; \textbf{3.} more difficult or severe than usual; \textbf{4.} (of a person or their behavior) not friendly or relaxed.} than the person who sat down to write. The problem is to \fbox{find the real man or woman behind the tension}.

Ultimately\footnote{\textbf{ultimately} [adv] \textbf{1.} in the end; finally; \textbf{2.} at the most basic \& important level, \textsc{synonym}: \textbf{basically, essentially}.} the product that any writer has to sell is not the subject being written about, but who he or she is. I often find myself reading with interest about a topic I never thought would interest me -- some scientific quest, perhaps. What holds me is the \fbox{enthusiasm of the writer for his field}. How was he drawn into it? What emotional baggage\footnote{\textbf{baggage} [n] [uncountable] \textbf{1.} bags, cases, etc. that contain somebody's clothes \& things when they are traveling, \textsc{synonym}: \textbf{luggage}; \textbf{2.} the equipment that an army carries with it; \textbf{3.} the beliefs \& attitudes that somebody has as a result of their past experiences.} did he bring along? How did it change his life? It's not necessary to want to spend a year alone at Walden Pond to become involved with a writer who did.

This is the personal transaction\footnote{\textbf{transaction} [n] \textbf{1.} [countable] a piece of business that is done between people, especially an act of buying or selling, \textsc{synonym}: \textbf{deal}; \textbf{2.} [uncountable] \textbf{transaction of something} (\textit{formal}) the process of doing something.} that's at the heart of good nonfiction writing. Out of it come 2 of the most important qualities that this book will go in search of: humanity \& warmth\footnote{\textbf{warmth} [n] [uncountable] \textbf{1.} the state or quality of being warm, rather than hot or cold; \textbf{2.} enthusiasm, affection or kindness.}. Good writing has an aliveness that keeps the reader reading from 1 paragraph to the next, \& it's not a question of gimmicks\footnote{\textbf{gimmick} [n] (\textit{often disapproving}) an unusual trick or unnecessary device that is intended to attract attention or to persuade people to buy something.} to ``personalize\footnote{\textbf{personalize} [v] (\textit{British English also} \textbf{personalise}) \textbf{1.} \textbf{personalize something} to cause an issue, argument, etc. to become concerned with particular people or feelings rather than with general matters; \textbf{2.} \textbf{personalize something} to design or change something so that it is suitable for the needs of a particular person; \textbf{3.} \textbf{personalize something} to mark something in some way to show that it belongs to a particular person.}'' the author. It's a question of using the English language in a way that will \fbox{achieve the greatest clarity \& strength}.

Can such principles be taught? Maybe not. But most of them can be learned.'' -- \cite[pp. 12--13]{Zinsser2016}

\section{Simplicity}
\footnote{\textbf{simplicity} [n] [uncountable] \textbf{1.} the quality of being easy to understand or use; \textbf{2.} (\textit{approving}) the quality of being natural \& plain or not complicated in design.} ``Clutter\footnote{\textbf{clutter} [v] \textbf{clutter something (up) (with something\texttt{/}somebody)} to fill a place or area with too many things, so that it is untidy; [n] [uncountable, singular] (\textit{disapproving}) a lot of things in an untidy state, especially things that are not necessary or are not being used; a lack of order, \textsc{synonym}: \textbf{mess}.} is the disease\footnote{\textbf{disease} [n] [uncountable, countable] an illness of the body in humans, animals or plants.} of American writing. we are a society strangling\footnote{\textbf{strangle} [v] \textbf{1.} \textbf{strangle somebody} to kill somebody by pressing their throat \& neck hard, especially with your fingers; \textbf{2.} \textbf{strangle something} to prevent something from growing or developing.} in unnecessary\footnote{\textbf{unnecessary} [a] \textbf{1.} not needed, \textsc{opposite}; \textbf{2.} more than is needed, \textsc{synonym}: \textbf{excessive, unjustified}.} words, circular\footnote{\textbf{circular} [a] \textbf{1.} shaped like a circle; round; \textbf{2.} moving around in a circle; \textbf{3.} (of an argument or a theory) using an idea or a statement to prove something which is then used to prove the idea or statement at the beginning.} constructions\footnote{\textbf{construction} [n] \textbf{1.} [uncountable] the process or method of building or making something, especially roads, bridges, buildings or machines; \textbf{2.} [uncountable, countable] \textbf{construction (of something)} the act or process of creating something from ideas, opinions \& knowledge; something created in this way; \textbf{3.} [uncountable] the way that something has been built or made; \textbf{4.} [countable] the way in which words are used together \& arranged to form a sentence or a phrase; \textbf{5.} [countable] (\textit{formal}) the way in which words, statements or actions are understood by somebody, \textsc{synonym}: \textbf{interpretation}.}, pompous\footnote{\textbf{pompous} [a] (\textit{disapproving}) showing that you think you are more important than other people, especially by using long \& formal words, \textsc{synonym}: \textbf{pretentious}.} frills\footnote{\textbf{frill} [n] \textbf{1.} a narrow piece of cloth with a lot of folds that is attached to the edge of a dress, curtain, etc. to decorate it, \textsc{synonym}: \textbf{ruffle}; \textbf{2.} [usually plural] things that are not necessary but are added to make something more attractive or interesting.} \& meaningless\footnote{\textbf{meaningless} [a] \textbf{1.} not having a meaning that is easy to understand; \textbf{2.} without any purpose or reason \& therefore not worth doing or having; \textbf{3.} \textbf{meaningless (to somebody\texttt{/}something)} not considered important, \textsc{synonym}: \textbf{irrelevant}.} jargon\footnote{\textbf{jargon} [n] [uncountable] (\textit{disapproving}) words or expressions that are used by a particular profession or group of people, \& are difficult for others to understand.}.

Who can understand the clotted\footnote{\textbf{clot} [n] a lump that is formed when a liquid, especially blood, dries or becomes thicker; [v] [intransitive] when blood clots, it forms thick lumps.} language of everyday American commerce\footnote{\textbf{commerce} [n] [uncountable] trade, especially between countries; the buying \& selling of goods \& services.}: the memo\footnote{\textbf{memo} [n] (plural \textbf{memos}) (\textit{also formal} \textbf{memorandum}) an official note from 1 person to another in the same organization.}, the corporation\footnote{\textbf{corporation} [n] (abbr., \textbf{Corp.}) a large business company, or a group of companies that is recognized by law as a single unit.} report\footnote{\textbf{report} [v] \textbf{1.} [transitive, often passive, intransitive] to tell people that something has happened or exists, or to provide other information about something; \textbf{2.} [transitive, often passive, intransitive] to present information in a newspaper, on television, etc. about something that has happened; \textbf{3.} [transitive] to tell a person in authority about a crime or about something else that is wrong; \textbf{4.} [intransitive] to go somewhere \& tell somebody in authority that you have arrived; \textbf{report back (to somebody\texttt{/}something) (on something)} [phrasal verb] to provide information for others, after doing something or after returning somewhere; \textbf{report to somebody} (not used in the progressive tenses) (\textit{business}) [phrasal verb] if you report to somebody, they are in charge of you or are responsible for your work; [n] \textbf{1.} a written document in which a particular situation or subject is examined or discussed; \textbf{2.} a statement that something has happened or exists; a piece of information about something; \textbf{3.} information that is presented in a newspaper, on television, etc. about something that has happened; \textbf{4.} (\textit{British English}) (\textit{North American English} \textbf{report card}) a written document about a student's progress at school.}, the business\footnote{\textbf{business} [n] \textbf{1.} [uncountable] the activity of making, buying, selling or supplying goods or services for money; \textbf{2.} [countable] a commercial organization such as a company, shop or factory. In this meaning, the word \textbf{business} often describes a small or medium-sized organization; the word \textbf{company} can be used for both small \& large organizations.; \textbf{3.} [uncountable] work or another activity that is part of your job \& not done for pleasure or for any other reason; \textbf{4.} [uncountable] the amount of work done by a company, etc.; the rate, volume, value or quality of this work; \textbf{5.} [countable] a particular area of commercial activity; \textbf{6.} [uncountable] the fact of a person or people buying goods or services from a business; \textbf{7.} [uncountable] something that concerns a particular person or organization; \textbf{8.} [uncountable] important matters that need to be dealt with or discussed; \textbf{9.} [singular] (usually with an adjective) \textbf{business (of something\texttt{/}of doing something)} a situation or a series of events; \textbf{go about your business} [idiom] to do the things that you normally do; \textbf{out of business} [idiom] having stopped operating as a business because there is no more money or work available.} letter\footnote{\textbf{letter} [n] \textbf{1.} [countable] a message that i written down or printed on paper \& usually put in an envelope \& sent to somebody; \textbf{2.} [countable] a written or printed sign representing a sound used in speech; \textbf{3.} (\textbf{the letter of something}) [singular] the exact words of a rule or statement rather than its general meaning; \textbf{to the letter} [idiom] doing exactly what somebody\texttt{/}something says, paying attention to every detail.}, the notice\footnote{\textbf{notice} [v] \textbf{1.} (not usually used in the progressive tenses) to see or hear somebody\texttt{/}something; to become aware of somebody\texttt{/}something; \textbf{2.} (not usually used in the progressive tenses) to pay attention to somebody\texttt{/}something; [n] \textbf{1.} [uncountable] the fact of somebody paying attention to somebody\texttt{/}something or knowing about something; \textbf{2.} [uncountable] information or a warning given in advance of something that is going to happen; \textbf{3.} [uncountable, countable] a formal letter or statement saying that you will or must do something, e.g. leave your job at the end of a particular period of time; \textbf{4.} [countable] a small advertisement or announcement in a newspaper or magazine, or on a website; \textbf{5.} [countable] a sheet of paper or an email giving written information about an event, etc.; \textbf{6.} [countable] a board or sign giving information, an instruction or a warning.} from the bank explaining its latest ``simplified\footnote{\textbf{simplify} [v] \textbf{1.} \textbf{simplify something} to make something less complicated, or easier to do or understand; \textbf{2.} \textbf{simplify something (to something)} (\textit{mathematics}) to rewrite an equation in its simplest form by, e.g., gathering common terms together \& canceling repeated terms where appropriate.}'' statement? What member of an insurance\footnote{\textbf{insurance} [n] \textbf{1.} [uncountable, countable] an arrangement with a company in which you pay them regular amounts of money \& they agree to pay the costs, e.g., if you die or are ill, or if you lose or damage something; \textbf{2.} [uncountable] the business of providing people with insurance; \textbf{3.} [uncountable] money paid by or to an insurance company; \textbf{4.} [uncountable, countable] \textbf{insurance (against something)} something you do to protect yourself against something bad happening in the future.} or medical\footnote{\textbf{medical} [a] [usually before noun] \textbf{1.} connected with the science or practice of medicine; \textbf{2.} connected with medicine as opposed to surgery, psychiatry, etc.} plan can decipher\footnote{\textbf{decipher} [v] \textbf{1.} \textbf{decipher something} to convert something written in code into normal language; \textbf{2.} \textbf{decipher something} to succeed in finding the meaning of something that is difficult to read or understand.} the brochure\footnote{\textbf{brochure} [n] a small magazine or book containing pictures \& information about something or advertising something.} explaining his costs \& benefits\footnote{\textbf{benefit} [n] \textbf{1.} [countable, uncountable] a helpful \& useful effect that something has; an advantage that something provides; \textbf{2.} [uncountable, countable] (\textit{British English}) money provided by the government to people who need financial help because they are unemployed, sick, etc.; [v] \textbf{1.} [intransitive] to be in a better position because of something; \textbf{2.} [transitive] \textbf{benefit somebody\texttt{/}something} to be useful or provide an advantage or somebody\texttt{/}something.}? What father or mother can put together a child's toy from the instructions\footnote{\textbf{instruction} [n] \textbf{1.} (\textbf{instructions}) [plural] detailed information on how to do or use something, \textsc{synonym}: \textbf{direction}; \textbf{2.} [countable, usually plural] something that somebody tells you to do, \textsc{synonym}: \textbf{order}; \textbf{3.} [countable] (\textit{computing}) a code in a program that tells a computer to perform a particular operation; \textbf{4.} [uncountable] the act of teaching something to somebody.} on the box? Our national\footnote{\textbf{national} [a] [usually before noun] \textbf{1.} connected with a particular nation; shared by a whole nation; \textbf{2.} owned, controlled or paid for by the government; [n] a citizen of a particular country.}\footnote{NQBH: `natural' should be used here instead?} tendency\footnote{\textbf{tendency} [n] (plural \textbf{tendencies}) \textbf{1.} [countable] if somebody\texttt{/}something has a particular tendency, they are likely to behave or act in a particular way; \textbf{2.} [countable] a new custom that is starting to develop, \textsc{synonym}: \textbf{trend}; \textbf{3.} [countable $+$ singular or plural verb] (\textit{British English}) a group within a larger political group, whose views are more extreme than those of the rest of the group.} is to inflate\footnote{\textbf{inflate} [v] \textbf{1.} [transitive] \textbf{inflate something} to make something appear to be more important or impressive than it really is; \textbf{2.} [transitive] \textbf{inflate something} to increase the amount of something; \textbf{3.} [transitive, intransitive] \textbf{inflate (something)} to fill something with gas or air; to become filled with gas or air, \textsc{opposite}: \textbf{deflate}.} \& thereby\footnote{\textbf{thereby} [adv] (\textit{formal}) used to introduce the result of the action or situation mentioned.} sound important. The airline\footnote{\textbf{airline}  [n] [countable $+$ singular or plural verb] a company that provides regular flights to atke passengers \& goods to different places.} pilot\footnote{\textbf{pilot} [n] a person who operates the controls of an aircraft, especially as a job; [a] [only before noun] done on a small scale in order to see if something is successful enough to do on a large scale; [v] \textbf{pilot something} to test a new product, idea, etc. with a few people or in a small area before it is introduced everywhere.} who announces\footnote{\textbf{announce} [v] \textbf{1.} to make a formal public statement about a fact, event or intention; \textbf{2.} to say something in a loud \&\texttt{/}or serious way.} that he is presently\footnote{\textbf{presently} [adv] \textbf{1.} (usually used before the word or sentence that it refers to) at the time you are speaking or writing; now, \textsc{synonym}: \textbf{currently}; \textbf{2.} (usually used at the end of a sentence or clause) at a later time, e.g. at a later point in the text that you are writing.} anticipating\footnote{\textbf{anticipate} [v] \textbf{1.} to expect or predict something; \textbf{2.} to see what might happen in the future \& take action to prepare for it; \textbf{3.} \textbf{anticipate something} to think with pleasure \& excitement about something that is going to happen; \textbf{4.} \textbf{anticipate something} to come before \& influence something else that is similar; to be a sign of what is going to happen.} experiencing\footnote{\textbf{experience} [n] \textbf{1.} [uncountable] the knowledge \& skill that you have gained through doing something for a period of time; the process of gaining this; \textbf{2.} [uncountable] the things that have happened to you that affect the way you think \& behave; \textbf{3.} [countable] an event or activity that affects you in some way; \textbf{4.} (\textbf{the $\ldots$ experience}) [singular] events or knowledge shared by all the members of a particular group in society, that affects the way they think \& behave; [v] \textbf{1.} \textbf{experience something} to have a particular situation affect you or happen to you; \textbf{2.} \textbf{experience something} to have a particular emotion or physical feeling.} considerable\footnote{\textbf{considerable} [a] great in amount, size or importance.} precipitation\footnote{\textbf{precipitation} [n] \textbf{1.} [uncountable] (\textit{specialist}) rain, snow, etc. that falls; the amount of this that falls; \textbf{2.} [uncountable, countable] \textbf{precipitation (of something)} (\textit{chemistry}) a chemical process in which solid material is separated from a liquid.} wouldn't think of saying it may rain. The sentence is too simple -- there must be something wrong with it.

But the secret of good writing is to strip\footnote{\textbf{strip} [n] \textbf{1.} a long narrow piece of paper, metal, cloth, etc.; \textbf{2.} a long narrow area of land, sea, etc.; [v] \textbf{1.} [transitive] to remove all of a particular type of thing, person or quality from a structure, place, organization, etc.; \textbf{2.} [transitive] \textbf{strip somebody of something} to take away property, honors or rights from somebody, as a punishment; \textbf{3.} [transitive] \textbf{strip A (off\texttt{/}from B)} to remove a layer from something, especially so that it is completely exposed; \textbf{4.} [intransitive, transitive] to take off all or most of your clothes or another person's clothes; \textbf{strip something away} [phrasal verb] \textbf{1.} to remove a layer from something; \textbf{2.} to remove anything that is not true or necessary.} every sentence to its cleanest\footnote{\textbf{clean} [a] \textbf{cleaner, cleanest} \textbf{1.} not dirty; \textbf{2.} not containing or producing harmful or unpleasant substances; [v] [transitive, intransitive] \textbf{clean (something)} to make something free from dirt or dust; \textbf{clean something up $|$ clean up} [phrasal verb] to remove dirt or pollution from somewhere or something.} components\footnote{\textbf{component} [n] \textbf{1.} 1 or several parts that combine together to make a system, machine or substance; \textbf{2.} a necessary feature or part of something.}. Every\footnote{\textbf{every} [determiner] \textbf{1.} used with singular nouns to refer to all the members of a group of things or people; \textbf{2.} all possible; \textbf{3.} used to say how often something happens or is done or how common something is; \textbf{every other} [idiom] if something happens every other day, night, etc. it happens on 1 day, etc. but not the next, \textsc{synonym}: \textbf{alternate}.} word\footnote{\textbf{word} [n] \textbf{1.} [countable] a single unit of language which means something \& can be spoken or written; \textbf{2.} [countable] a thing that you say; a remark or statement; \textbf{3.} [singular] a promise that you will do something or that something will happen or is true; \textbf{4.} [singular] a piece of information or news; [v] [often passive] \textbf{word something}  to write or say something using particular words.} that serves\footnote{\textbf{serve} [v] \textbf{1.} [intransitive, transitive] to have a particular effect, use or result; \textbf{2.} [transitive] to be useful to somebody in achieving something; \textbf{3.} [transitive] to provide an area or a group of people with a product or service; \textbf{4.} [intransitive, transitive] to work or perform duties for a person, an organization, a country, etc.; \textbf{5.} [transitive] \textbf{serve something} to spend a period of time in prison; \textbf{6.} [transitive] to give somebody food or drink, e.g. at a restaurant or during a meal; \textbf{7.} [transitive] (\textit{law}) to give or send somebody an official document, especially one that orders them to appear in court; \textbf{serve something up} [phrasal verb] to give, offer or provide something.} no function\footnote{\textbf{function} [n] \textbf{1.} [countable, uncountable] the action or purpose that somebody\texttt{/}something has in a particular situation; the ability that somebody\texttt{/}something has to perform a particular job or role; \textbf{2.} [countable] \textbf{function (of something)} (\textit{mathematics}) a quantity whose value depends on the varying values of others; \textbf{3.} [countable] a part of a computer program or system that performs a basic operation; \textbf{4.} [countable] a social event or official ceremony; \textbf{be a function of something} [idiom] to be something that depends on something else; [v] [intransitive] to work in the correct way; to work in a particular way, \textsc{synonym}: \textbf{operate}; \textbf{function as somebody\texttt{/}something} [phrasal verb] to perform the action or the job of the thing or person mentioned.}, every long word that could be a short word, every adverb\footnote{\textbf{adverb} [n] (\textit{grammar}) a word that adds more information about place, time, manner, cause or degree to a verb, an adjective, a phrase or another adverb.} that carries\footnote{\textbf{carry} [v] \textbf{1.} to support the weight of somebody\texttt{/}something \& take them\texttt{/}it from place to place; to take somebody\texttt{/}something from 1 place to another; \textbf{2.} \textbf{carry something} to have something with you \& take it wherever you go; \textbf{3.} to contain \& direct the flow of water, electricity, etc.; \textbf{4.} \textbf{carry something} to contain something such as information, a message or a signal \& be able to pass it from 1 place, person, etc. to another; \textbf{5.} \textbf{carry something} if a person, an animal, etc. carries a disease, they are infected with it \& might spread it to others although they might not become ill themselves; \textbf{6.} \textbf{carry something} to support the weight of something; \textbf{7.} \textbf{carry something} to have some as a quality, feature or possible result; \textbf{8.} \textbf{carry something} to accept responsibility for something; to suffer the results of something; \textbf{9.} \textbf{carry something\texttt{/}somebody $+$ adv.\texttt{/}prep.} to take something\texttt{/}somebody to a particular point or in a particular direction; \textbf{10.} [usually passive] to approve of something by more people voting for it than against it; \textbf{11.} \textbf{carry something} (of a newspaper or broadcast) to publish or broadcast a particular story.} the same meaning that's already in the verb, every passive\footnote{\textbf{passive} [a] \textbf{1.} accepting what happens or what people do without trying to change anything or oppose them, \textsc{opposite}: \textbf{active}; \textbf{2.} (\textit{grammar}) connected with the form of a verb used when the subject is affected by the action of the verb; [n] (also \textbf{passive voice}) (often \textbf{the passive (voice)}) [singular] (\textit{grammar}) the form of a verb used when the subject is affected by the action of the verb.} construction that leaves the reader unsure\footnote{\textbf{unsure} [a] [not before noun] not certain of something; having doubts.} of who is doing what -- these are the thousand \& 1 adulterants\footnote{\textbf{adulterate} [v] [often passive] to make food or drink less pure by adding another substance to it, \textsc{synonym}: \textbf{contaminate}.}\,\footnote{\textbf{adulteration} [n] [uncountable] the action of making food or drink less pure by adding another substance to it, \textsc{synonym}: \textbf{contamination}.} that weaken\footnote{\textbf{weaken} [v] \textbf{1.} [transitive, intransitive] \textbf{weaken (somebody\texttt{/}something)} to make somebody\texttt{/}something less strong or powerful; to become less strong or powerful, \textsc{opposite}: \textbf{strengthen}; \textbf{2.} [transitive, intransitive] \textbf{weaken (something)} to make something less physically strong; to become less physically strong, \textsc{opposite}: \textbf{strengthen}; \textbf{3.} [intransitive, transitive] to become less determined or certain about something; to make somebody less determined or certain, \textsc{opposite}: \textbf{strengthen}.} the strength\footnote{\textbf{strength} [n] \textbf{1.} [uncountable, singular] the quality that a person or animal has of being physically strong, \textsc{opposite}: \textbf{weakness}; \textbf{2.} [uncountable] the quality that an object or substance has of being strong \& not easily broken or damaged, \textsc{opposite}: \textbf{weakness}; \textbf{3.} [countable] a quality or ability that a person or thing has that gives them an advantage,\textsc{opposite}: \textbf{weakness, limitation}; \textbf{4.} [uncountable] the power \& influence that somebody\texttt{/}something has; \textbf{5.} [uncountable] how strong a natural force is; \textbf{6.} [uncountable, countable] how strong a drug, chemical or drink is, \textsc{synonym}: \textbf{concentration}; \textbf{7.} [uncountable] \textbf{strength (of something)} how clear \& reliable an argument, evidence or connection is; \textbf{8.} [uncountable, singular] the quality of being brave \& determined in a difficult situation; \textbf{9.} [uncountable] \textbf{strength (of something)} how strong or deeply felt an opinion or feeling is; \textbf{10.} [uncountable] the number of people in a group, a team or an organization; \textbf{on the strength of something} [idiom] because somebody has been influenced or persuaded by something.} of a sentence\footnote{\textbf{sentence} [n] \textbf{1.} [countable] (\textit{grammar}) a set of words expressing a statement, a question or an order, usually containing a subject \& a verb. In written English, sentences begin with a capital letter \& end with a full stop (.) or a question mark (?).; \textbf{2.} [countable, uncountable] the punishment given by a court; [v] [often passive] to say officially in court that somebody is to receive a particular punishment.}. \& they usually occur\footnote{\textbf{occur} [v] \textbf{1.} [intransitive] to happen; \textbf{2.} [intransitive] \textbf{$+$ adv.\texttt{/}prep.} to exist or be found somewhere; \textbf{occur to somebody} (of an idea or thought) [phrasal verb] to come into your mind.} in proportion\footnote{\textbf{proportion} [n] \textbf{1.} [countable $+$ singular or plural verb] a part or share of a whole; \textbf{2.} [uncountable] the relationship of 1 thing to another in size, amount or number, \textsc{synonym}: \textbf{ratio}; \textbf{3.} (\textbf{proportions}) [plural] the size \& shape of 1 part of something in relation to the other parts; \textbf{4.} [uncountable] the correct relationship in size between 1 thing \& another or between the parts of a whole.} to education\footnote{\textbf{education} [n] \textbf{1.} [uncountable, singular] a process of teaching, training \& learning, especially in schools or colleges, to improve knowledge \& develop skills; \textbf{2.} [uncountable] a particular kind of teaching or training; \textbf{3.} (\textbf{Education}) [uncountable] the institutions or people involved in teaching \& training; \textbf{4.} (usually \textbf{Education}) [uncountable] the subject of study that deals with how to teach.} \& rank\footnote{\textbf{rank} [n] \textbf{1.} [countable, usually plural, uncountable] the position that somebody has in a particular organization or society; \textbf{2.} (\textbf{the ranks}) [plural] the members of particular group or organization; \textbf{3.} [countable] \textbf{rank (of something)} the position that somebody has in the army, navy, police, etc.; \textbf{4.} [singular] the degree to which somebody\texttt{/}something is higher or lower on a scale of quality, important, success, etc.; [v] [transitive, intransitive] (not used in the progressive tenses) to give somebody\texttt{/}something a particular position on a scale according to quality, importance, success, etc.; to have a position of this kind.}.

During\footnote{\textbf{during} [prep] \textbf{1.} all through a period of time; \textbf{2.} at some point in a period of time. \textbf{During} is used to say when something happens; \textbf{for} answers the question `how long?'.} the 1960s the president\footnote{\textbf{president} [n] \textbf{1.} (\textbf{President}) the elected leader of a republic; \textbf{2.} (\textbf{President}) \textbf{president (of something)} the person in charge of some organizations, clubs, colleges, etc.; \textbf{3.} \textbf{president (of something)} (\textit{especially North American English}) the person in charge of a bank or commercial organization.} of my university\footnote{\textbf{university} [n] (plural \textbf{universities}) [countable, uncountable] (abbr., \textbf{Univ.}) an institution at the highest level of education where you can study for a degree or do research.} wrote a letter to mollify\footnote{\textbf{mollify} [v] (\textit{formal}) \textbf{mollify somebody} to make somebody feel less angry or upset, \textsc{synonym}: \textbf{placate}.} the alumni\footnote{\textbf{alumni} [n] [plural] (\textit{especially North American English}) the former male \& female students of a school, college or university.} after a spell\footnote{\textbf{spell} [v] \textbf{1.} [transitive, intransitive] \textbf{spell (something)} to write or say the letters of a word in the correct order; to form words correctly from individual letters; \textbf{2.} [transitive] \textbf{spell something} (of letters) to form a word when they are put together in a particular order; \textbf{3.} [transitive] \textbf{spell something (for somebody\texttt{/}something)} to have something, usually something bad, as a result; to mean something, usually something bad; \textbf{spell something out} [phrasal verb] to explain something in a simple, clear way; [n] \textbf{1.} [countable] a short period of time during which something lasts for somebody does something; \textbf{2.} [singular] a quality that somebody\texttt{/}something has that gives them control or influence over people as if in a magical way; \textbf{cast a spell (on\texttt{/}over somebody\texttt{/}something)} [idiom] to use words that are thought to be magic \& have the power to change somebody\texttt{/}something; to have a powerful influence over somebody\texttt{/}something.} of campus\footnote{\textbf{campus} [n] the buildings of a university or college \& the land around them.} unrest\footnote{\textbf{unrest} [n] [uncountable] a situation in which people are angry \& likely to protest against the government or their employers, \textsc{synonym}: \textbf{disorder}.}. ``You are probably aware\footnote{\textbf{aware} [a] \textbf{1.} [not before noun] knowing or realizing that something is true or exists, \textsc{opposite}: \textbf{unaware}; \textbf{2.} (used with an adverb) concerned \& knowing a lot about a particular situation or development.},'' he began, ``that we have been experiencing very considerable potentially\footnote{\textbf{potentially} [adv] possibly going to develop or be developed into something, especially something bad.} explosive\footnote{\textbf{explosive} [a] \textbf{1.} exploding; easily able or likely to explode; \textbf{2.} likely to cause violence or strong feelings of anger; \textbf{3.} increasing suddenly \& quickly.} expressions\footnote{\textbf{expression} [n] \textbf{1.} [countable, uncountable] things that people say, write or do in order to show their feelings, opinions \& ideas; \textbf{2.} [countable, uncountable] a look on a person's face that shows their thoughts or feelings; \textbf{3.} [countable] a word or phrase; \textbf{4.} [countable] (\textit{mathematics}) a group of signs that represent an idea or a quantity; \textbf{5.} [uncountable] (\textit{biochemistry}) the presence of a gene product in a cell, which shows the gene is there.} of dissatisfaction\footnote{\textbf{dissatisfaction} [n] [uncountable, countable] a feeling that you are not pleased or satisfied, because something is not as good as you expected, \textsc{opposite}: \textbf{satisfaction}.} on issues\footnote{\textbf{issue} [n] \textbf{1.} [countable] an important topic that people are discussing or arguing about; \textbf{2.} [countable] (often \textbf{issues} [plural]) a problem, concern or difficulty; \textbf{3.} [countable] 1 of a regular series of magazines or newspapers; \textbf{4.} [countable, uncountable] something that is supplied or made available for people to buy or use; the act of supplying or making available things for people to buy or use; \textbf{5.} [uncountable] (\textit{law}) children of your own; [v] \textbf{1.} to make something known formally; to make something available publicly; \textbf{2.} [often passive] to give something to somebody, especially officially; \textbf{3.} \textbf{issue something} to start a legal process against somebody, especially by means of an official document; \textbf{4.} \textbf{issue something} to produce new stamps, coins, shares, etc. for sale to the public; \textbf{issue from something} [phrasal verb] to come out of something.} only partially\footnote{\textbf{partially} [adv] partly; not completely.} related\footnote{\textbf{related} [a] \textbf{1.} connected with something\texttt{/}somebody in some way, \textsc{opposite}: \textbf{unrelated}; \textbf{2.} belonging to the same group, \textsc{opposite}: \textbf{unrelated}; \textbf{3.} \textbf{related (to something\texttt{/}somebody)} connected by a family relationship or by marriage, \textsc{opposite}: \textbf{unrelated}.}.'' He meant that the students had been hassling\footnote{\textbf{hassle} [v] (\textit{informal}) \textbf{hassle somebody (for something\texttt{/}to do something)} to annoy somebody or cause them trouble, especially by asking them to do something many times, \textsc{synonym}: \textbf{bother}.} them about different things. I was far more upset by the president's English than by the students' potentially explosive expressions of dissatisfaction. I would have preferred the presidential\footnote{\textbf{presidential} [a] connected with a president.} approach taken by \textsc{Franklin D. Roosevelt} when he tried to convert\footnote{\textbf{convert} [v] \textbf{1.} [transitive, intransitive] to change the form, use or character of something; to change from 1 form, purposes or system to another; \textbf{2.} [intransitive] \textbf{convert into\texttt{/}to something} to be able to change or be changed from 1 form or purpose to another; \textbf{3.} [intransitive, transitive] to change or make somebody change their religion, beliefs or way of life; \textbf{convert somebody to something} [phrasal verb] to persuade somebody to support a particular idea; [n] a person who has changed their religion, beliefs or way of life.} into English his own government's\footnote{\textbf{government} [n] \textbf{1.} [countable $+$ singular or plural verb, uncountable] (often \textbf{the Government}) (abbr., \textbf{govt}) the group of people \& the institutions connected with them that are responsible for controlling a country or state; \textbf{2.} [uncountable] a particular system or method of controlling a country; \textbf{3.} [uncountable] the activity or manner of controlling a country.} memos\footnote{\textbf{memo} [n] (plural \textbf{memos}) (\textit{also formal} \textbf{memorandum}) an official note from 1 person to another in the same organization.}, such as this blackout\footnote{\textbf{blackout} [n] \textbf{1.} a period when there is no light as a result of an electrical power failure; \textbf{2.} a situation when the government or the police will not allow any news or information on a particular subject to be given to the public; \textbf{3.} [usually singular] a period of time during a war when all lights must be put out or covered at night, so that they cannot be seen by an enemy attacking by air; \textbf{4.} [usually plural] (\textit{British English}) a piece of material that covers windows to stop light being seen from outside, or light from outside from coming into a room; \textbf{5.} a temporary loss of consciousness, sight or memory.} order\footnote{\textbf{order} [n] \textbf{1.} [uncountable, countable] the way in which people or things are placed or arranged in relation to each other; \textbf{2.} [uncountable] the state in which everything is in the right place or something is as it should be, \textsc{opposite}: \textbf{disorder}; \textbf{3.} [uncountable] the state that exists when people obey laws, rules or authority; \textbf{4.} [countable] something that somebody is told to do by somebody in authority; \textbf{5.} [countable] a written instruction by a court or judge; \textbf{6.} [countable, uncountable] a request to make or supply goods; \textbf{7.} [countable, usually singular] the way that a society, the world, etc. is arranged, with its system of rules \& customs; \textbf{8.} [singular] a particular quality or degree; \textbf{9.} [countable] \textbf{order (of something)} (\textit{biology}) a group into which animals, plants, etc. that are related are divided, smaller than a class \& larger than a family; [v] \textbf{1.} to use your position of authority to tell somebody to do something or say that something must happen; \textbf{2.} \textbf{order something (from somebody\texttt{/}something)} to ask for goods to be made or supplied; to ask for a service to be provided; \textbf{3.} \textbf{order something} to organize or arrange something.} of 1942:

\begin{example}
	Such preparations\footnote{\textbf{preparation} [n] \textbf{1.} [uncountable] the act or process of making something\texttt{/}somebody ready or of getting ready for something; \textbf{2.} [countable, usually plural] things that you do not get ready for something or to make something ready; \textbf{3.} [countable] a substance that has been specially prepared for use as a medicine, cosmetic, etc.} shall be made as will completely\footnote{\textbf{completely} [adv] (used to emphasize the following word or phrase) in every way possible, \textsc{synonym}: \textbf{totally}.} obscure\footnote{\textbf{obscure} [v] to cover something; to make it difficult to see, hear or understand something; [a] \textbf{1.} not well known, \textsc{synonym}: \textbf{unknown}; \textbf{2.} difficult to understand.} all Federal\footnote{\textbf{federal} [a] \textbf{1.} having a system of government in which the individual states of a country have control over their own affairs, but are controlled by a central government for national decisions; \textbf{2.} (within a federal system, e.g. the US \& Canada) connected with national government rather than the local government of an individual state.} buildings\footnote{\textbf{building} [n] \textbf{1.} [countable] a structure with a roof \& walls, such as a house or factory; \textbf{2.} [uncountable] the process \& work of building.} \& non-Federal buildings occupied\footnote{\textbf{occupied} [a] \textbf{1.} [not before noun] busy; \textbf{2.} [not before noun] being used, \textsc{opposite}: \textbf{unoccupied}; \textbf{3.} (of a country, etc.) controlled by people from another country, etc., using military force.}\,\footnote{\textbf{occupy} [v] \textbf{1.} \textbf{occupy something} to fill or use a space, area or amount of time, \textsc{synonym}: \textbf{take up something}; \textbf{2.} \textbf{occupy something} to live or work in a room, house or building; \textbf{3.} \textbf{occupy something} to enter a place in a large group \& take control of it, especially by military force; \textbf{4.} \textbf{occupy something} to have an official job or position, \textsc{synonym}: \textbf{hold}; \textbf{5.} \textbf{occupy something} to be in or at a particular position in a system, \textsc{synonym}: \textbf{hold}; \textbf{6.} to fill your time or keep you busy doing something.} by the Federal government during an air raid\footnote{\textbf{air raid} [n] an attack by a number of aircraft dropping many bombs on a place.} for any period of time from visibility\footnote{\textbf{visibility} [n] [uncountable] \textbf{1.} how far or well you can see, especially as affected by the light or the weather; \textbf{2.} the degree to which something attracts attention; the fact or state of being easy to see.} by reason of internal\footnote{\textbf{internal} [a] \textbf{1.} [usually before noun] connected with the inside of something, \textsc{opposite}: \textbf{external}; \textbf{2.} [only before noun] connected with the inside of a person's or animal's body, \textsc{opposite}: \textbf{external}; \textbf{3.} involving or concerning only the people who are part of a particular organization rather than people from outside it, \textsc{opposite}: \textbf{external}; \textbf{4.} [usually before noun] happening or existing within a country or region rather than involving other countries or regions, \textsc{synonym}: \textbf{domestic}, \textsc{opposite}: \textbf{external}; \textbf{5.} [only before noun] coming from within a thing itself rather than from outside it, \textsc{opposite}: \textbf{external}; \textbf{6.} happening or existing in a person's mind.} or external\footnote{\textbf{external} [a] \textbf{1.} coming from outside the place, organization or situation that is affected. The \textbf{external validity} of a study is the degree to which its findings apply beyond its own research context. \textsc{opposite}: \textbf{internal}; \textbf{2.} existing outside a place, an organization or a particular situation; connected with the outside of something, \textsc{opposite}: \textbf{internal}; \textbf{3.} connected with foreign countries, \textsc{opposite}: \textbf{internal}.} illumination\footnote{\textbf{illumination} [n] \textbf{1.} [uncountable, countable] light or a place that might comes from; \textbf{2.} [uncountable] understanding or explanation of something.}.
	
	``Tell them,'' Roosevelt said, ``that in buildings where they have to keep the work going to put something across the windows.''
\end{example}
Simplify\footnote{\textbf{simplify} [v] \textbf{1.} \textbf{simplify something} to make something less complicated, or easier to do or understand; \textbf{2.} \textbf{simplify something (to something)} (\textit{mathematics}) to rewrite an equation in its simplest form by, e.g., gathering common terms together \& canceling repeated terms where appropriate.}, simplify. Thoreau said it, as we are so often reminded\footnote{\textbf{remind} [v] to help somebody remember something, especially something important that they must do; \textbf{remind somebody of somebody\texttt{/}something} [phrasal verb] to make somebody remember or think about another person, place or thing by being similar to them in some way.}, \& no American writer more consistently\footnote{\textbf{consistently} [adv] always in the same way; the following the same pattern or standard.} practiced what he preached\footnote{\textbf{preach} [v] \textbf{1.} [intransitive, transitive] to give a religious talk in a public place, especially in a church during a service; \textbf{2.} [transitive, intransitive] to tell people about a particular religion, way of life, system, etc. in order to persuade them to accept it; \textbf{3.} [intransitive] (\textit{disapproving}) to give somebody advice on moral standards, behavior, etc., especially in a way that they find annoying or boring; \textbf{practice what you preach} [idiom] to do the things yourself that you tell other people to do.}. Open \textit{Walden}\footnote{\textbf{Walden} the best-known book by the US writer \textsc{Henry David Thoreau}. Its full title is \textit{Walden, or Life in the Woods}.} to any page \& you will find a man saying in a plain \& orderly\footnote{\textbf{orderly} [a] \textbf{1.} arranged or organized in a neat, careful \& logical way; \textbf{2.} behaving well; peaceful.} way what is on his mind:

\begin{example}
	I went to the woods because I wished to live deliberately\footnote{\textbf{deliberately} [adv] on purpose rather than by accident, \textsc{synonym}: \textbf{intentionally}.}, to front only the essential\footnote{\textbf{essential} [a] \textbf{1.} completely necessary; extremely important in a particular situation or for a particular activity, \textsc{synonym}: \textbf{vital}; \textbf{2.} [only before noun] connected with the most important aspect or basic nature of somebody\texttt{/}something, \textsc{synonym}: \textbf{fundamental}; \textbf{3.} (of an amino acid or fatty acid) required for normal growth but not produced in the body, \& therefore necessary in the diet; [n] [usually plural] \textbf{1.} something that is needed in a particular situation or in order to do a particular thing; \textbf{2.} \textbf{essential (of something)} an important basic fact or piece of knowledge about a subject.} facts of life, \& see if I could not learn what it had to teach, \& not, when I came to die, discover that I had not lived.
\end{example}
How can the rest of us achieve such enviable\footnote{\textbf{enviable} [a] something that is enviable is the sort of thing that is good \& that other people want to have too, \textsc{opposite}: \textbf{unenviable}.} freedom\footnote{\textbf{freedom} [n] \textbf{1.} [uncountable, countable] \textbf{freedom (of something)} the right to do or say what you want without anyone stopping you; \textbf{2.} [uncountable] the state of being able to do what you want, without anything stopping you; \textbf{3.} [uncountable] the state of not being a prisoner or slave; \textbf{4.} [uncountable] \textbf{freedom (from something)} the state of not being ruled by a foreign country; \textbf{5.} [uncountable] \textbf{freedom from something} the state of not being affected by the thing mentioned; \textbf{freedom of\texttt{/}room for manoeuvre} [idiom] the chance to change the way that something happens \& influence decisions that are made.} from clutter\footnote{\textbf{clutter} [v] \textbf{clutter something (up) (with something\texttt{/}somebody)} to fill a place or area with too many things, so that it is untidy; [n] [uncountable, singular] (\textit{disapproving}) a lot of things in an untidy state, especially things that are not necessary or are not being used; a lack of order, \textsc{synonym}: \textbf{mess}.}? \fbox{The answer is to clear our heads of clutter.} \fbox{Clear thinking becomes clear writing; one can't exist without the other.} It's impossible\footnote{\textbf{impossible} [a] \textbf{1.} that cannot exist or be done; not possible; \textbf{2.} (\textbf{the impossible}) [n] [singular] a thing that is or seems impossible; \textbf{3.} very difficult to deal with.} for a muddy\footnote{\textbf{muddy} [a] (\textbf{muddier, muddiest}) \textbf{1.} full of or covered in mud; \textbf{2.} (of a liquid) containing mud; not clear; \textbf{3.} (of colors) not clear or bright; [v] \textbf{muddy something} to make something muddy.} thinker\footnote{\textbf{thinker} [n] \textbf{1.} a person who thinks seriously, \& often writes about about important things, such as philosophy or science; \textbf{2.} a person who thinks in a particular way.} to write good English. He may get away with it for a paragraph or 2, but soon the reader will be lost, \& there's no sin so grave\footnote{\textbf{grave} [n] \textbf{1.} a place in the ground where a dead person is buried; \textbf{2.} [singular] (often \textbf{the grave}) death; a person's death; [a] [usually before noun] (\textbf{graver, gravest}) very serious; causing great worry.}, for the reader will not easily be lured\footnote{\textbf{lure} [v] (\textit{disapproving}) \textbf{lure somebdoy ($+$ adv.\texttt{/}prep.)} to persuade or trick somebody or go somewhere or to do something by promising them a reward, \textsc{synonym}: \textbf{entice}; [n] \textbf{1.} [usually singular] \textbf{the lure of something} the attractive qualities of something; \textbf{2.} a thing that is used to attract fish or animals, so they they can be caught.} back.

Who is this elusive\footnote{\textbf{elusive} [a] difficult to find, define or achieve.} creature\footnote{\textbf{creature} [n] \textbf{1.} a living thing, real or imaginary, that can move around, such as an animal; \textbf{2.} a person, considered in a particular way; \textbf{3.} \textbf{creature of something\texttt{/}somebody} a person or organization that is considered to be under the complete control of another.}, the reader? The reader is someone with an attention span of about 30 seconds -- a person assailed\footnote{\textbf{assail} [v] (\textit{formal}) \textbf{1.} \textbf{assail somebody\texttt{/}something (with something)} to attack somebody\texttt{/}something violently, either physically or with words; \textbf{2.} [usually passive] to worry or upset somebody severely.} by many forces competing for attention. At 1 time those forces were relatively\footnote{\textbf{relatively} [adv] to a fairly large degree, especially in comparison with something else; \textbf{relatively speaking} [idiom] used when you are comparing something with all similar things.} few: newspapers, magazines, radio, spouse, children, pets. Today they also include a galaxy\footnote{\textbf{galaxy} [n] (plural \textbf{galaxies}) any of the very large systems of stars, planets, gas \& dust in space. \textbf{The Galaxy} refers to our own galaxy, containing our sun \& its planets, seen as a bright band in the night sky, \& also known as \textbf{the Milky Way}.} of electronic devices for receiving\footnote{\textbf{receive} [v] \textbf{1.} to get or accept something that is sent or given to you; \textbf{2.} to experience, suffer or be given a particular type of attention or treatment; \textbf{3.} [usually passive] to react to something new, in a particular way; \textbf{4.} to change broadcast signals into sounds or pictures on a television or other equipment; \textbf{5.} \textbf{receive somebody} to welcome or entertain a visitor; \textbf{6.} \textbf{receive somebody (into something)} (\textit{formal}) to officially recognize \& accept somebody as a member of a group.} entertainment\footnote{\textbf{entertainment} [n] \textbf{1.} [uncountable, countable] films, music, etc. used to entertain people; an example of this; \textbf{2.} [uncountable] \textbf{entertainment (of somebody)} the act of entertaining somebody.} \& information -- television\footnote{\textbf{television} [n] (abbr., \textbf{TV}) \textbf{1.} [uncountable] the system \& business of broadcasting pictures \& sounds using electronic signals \& creating programmes for people to watch; \textbf{2.} [uncountable] the programmes that are broadcast on television; \textbf{3.} (also \textbf{television set}) [countable] a piece of electrical equipment with a screen on which people watch programmes that are broadcast.}, VCRs\footnote{\textbf{VCR} [n] (especially North American English) a machine that was used, especially in the past, to play videos or to record programmes from a television (abbr. for `video cassette recorder').}, DVDs\footnote{\textbf{DVD} [n] [countable, uncountable] a disk on which data, especially photographs \& video, can be stored, for use on a computer or \textbf{DVD player} (abbr. for `digital versatile disc' or, originally, `digital videodisc').}, CDs\footnote{\textbf{CD} [n] a small disc on which sound or information is recorded. CDs can be played or read on various types of machines, including CD players \& computers (abbr. for `compact disc').}, video games, the Internet, e-mail, cell phones, BlackBerries, iPods\footnote{\textbf{iPod} [n] a brand o MP3 player that can store information taken from the Internet \& that you carry with you, e.g. so that you can listen to music.} -- as well as a fitness\footnote{\textbf{fitness} [n] [uncountable] \textbf{1.} the state of being physically healthy \& strong; \textbf{2.} the state of being suitable or good enough for something.} program, a pool\footnote{\textbf{pool} [n] \textbf{1.} a small area of still water, especially one that has formed naturally. A \textbf{swimming pool} is an area of water that has been created for people to swim in.; \textbf{2.} \textbf{pool (of something)} an amount of something that is available \& can be used when needed. In biology, a \textbf{gene pool} is all of the genes that are available within breeding populations of a particular species of animal or plant.; \textbf{3.} \textbf{pool (of somebody\texttt{/}something)} a group of people, especially of people who are available for work when needed; [v] \textbf{1.} to put together information from different sources so that it can be considered together; \textbf{2.} \textbf{pool something} to put together money, resources, etc. from different people so that all of them can use it.}, a lawn\footnote{\textbf{lawn} [n] \textbf{1.} [countable] an area of ground covered in short grass in a garden or park, or used for playing a game on; \textbf{2.} [uncountable] a type of fine cotton or linen cloth used for making clothes.} \& that most potent\footnote{\textbf{potent} [a] having great power or influence; having a strong effect on your body or mind, \textsc{synonym}: \textbf{powerful}.} of competitors\footnote{\textbf{competitor} [n] \textbf{1.} a person or business that is competing to be more successful than another person or business; \textbf{2.} \textbf{competitor for something} a person, animal or organization that is competing to get something, with the result that somebody\texttt{/}something else may not be able to get it.}, sleep. The man or woman snoozing\footnote{\textbf{snooze} [v] [intransitive] (\textit{informal}) to have a short, light sleep, especially during the way \& usually not in bed; [n] \textbf{1.} [countable, usually singular] (\textit{informal}) a short, light sleep, especially during the day \& usually not in bed; \textbf{2.} [uncountable] (also \textbf{snooze button} [countable]) a control on a clock or phone that you press when you wake up, so that you can sleep a little longer \& be woken up again after a short time.} in a chair with a magazine or a book is a person who was being given too much unnecessary trouble\footnote{\textbf{trouble} [n] \textbf{1.} [uncountable, countable] a problem, worry, difficulty, etc.; a situation causing this; \textbf{2.} [uncountable] something that is wrong with a part of the body, machine, vehicle, etc.; \textbf{3.} [uncountable] a situation that is difficult or dangerous; a situation in which you can be criticized or punished; \textbf{4.} [uncountable] an angry or violent situation; \textbf{5.} [uncountable] extra work or work; [v] [often passive] \textbf{1.} \textbf{trouble somebody} to make somebody worried or unhappy; \textbf{2.} \textbf{trouble somebody} (of a medical problem) to cause pain or problems.} by the writer.

It won't do to say that the reader is too dumb\footnote{\textbf{dumb} [a] (\textbf{dumber, dumbest}) \textbf{1.} (\textit{especially North American English, informal}) stupid; \textbf{2.} (\textit{old-fashioned, offensive}) unable to speak; \textbf{3.} temporarily not speaking or refusing to speak; [v] \textbf{dumb down} [phrasal verb] \textbf{dumb down $|$ dumb something $\leftrightarrow$ down} (\textit{disapproving}) to make something less accurate or educational, \& of worse quality, by trying to make it easier for people to understand.} or too lazy\footnote{\textbf{lazy} [a] (\textbf{lazier, laziest}) \textbf{1.} (\textit{disapproving}) unwilling to work or be active; doing as little as possible, \textsc{synonym}: \textbf{idle}; \textbf{2.} not involving much energy or activity; slow \& relaxed; \textbf{3.} (\textit{disapproving}) showing a lack of effort or care; \textbf{4.} (\textit{literary}) moving slowly, \textsc{synonym}: \textbf{torpid}.} to keep pace with the \fbox{train of thought}. If the reader is lost, it's usually because the writer hasn't been careful enough. That carelessness\footnote{\textbf{carelessness} [n] [uncountable] lack of attention \& thought about what you are doing.} can take any number of forms. Perhaps a sentence is so excessively\footnote{\textbf{excessively} [adv] to a much greater level or degree than seems reasonable or appropriate.} cluttered that the reader, hacking\footnote{\textbf{hacking} [n] [uncountable] the activity of using computers to get access to data in somebody else's computer or phone system without permission.} through the verbiage\footnote{\textbf{verbiage} [n] [uncountable] (\textit{formal, disapproving}) the use of too many words, or of more difficult words than are needed, to express an idea.}, simply doesn't know what it means. Perhaps a sentence has been so shoddily\footnote{\textbf{shoddily} [adv] \textbf{1.} \textbf{shoddily built, constructed, made, designed, etc.} built, made or designed badly \& with not enough care; \textbf{2.} in a dishonest or unfair way.} constructed that the reader could read it in several ways. Perhaps the writer has switched\footnote{\textbf{switch} [v] \textbf{1.} [intransitive, transitive] to change from 1 thing to another; to make something do this; \textbf{2.} [transitive] to exchange 1 thing for another; \textbf{switch off\texttt{/}on $|$ switch something off\texttt{/}on} [phrasal verb] to turn a light, machine, etc. off\texttt{/}on by pressing a button or switch; [n] \textbf{1.} a small device that you press or move up \& down in order to turn a piece of electrical equipment on \& off; \textbf{2.} \textbf{switch (in\texttt{/}of something) (from A to B)} a change from 1 thing to another, especially when this is sudden \& complete.} pronouns in midsentence, or has switched tenses, so the reader loses track of who is talking or when the action took place. Perhaps Sentence B is not a logical\footnote{\textbf{logical} [a] \textbf{1.} following or able to follow the rules of logic in which ideas or facts are based on other true ideas or facts; \textbf{2.} (of an action or event) seeming natural, reasonable or sensible, \textsc{opposite}: \textbf{illogical}; \textbf{3.} (\textit{computing}) connected to the system or set of principles used in preparing a computer to perform a particular task.} sequel\footnote{\textbf{sequel} [n] \textbf{1.} \textbf{sequel (to something)} a book, film, play, etc. that continues the story of an earlier one; \textbf{2.} [usually singular] \textbf{sequel (to something)} something that happens after an earlier event or as a result of an earlier event.} to Sentence A; the writer, in whose head the connection is clear, hasn't bothered\footnote{\textbf{bother} [v] \textbf{1.} [intransitive, transitive] (often used in negative sentences \& questions) to spend time \&\texttt{/}or energy doing something; \textbf{2.} [transitive] to annoy, worry or upset somebody; to cause somebody trouble or pain; \textbf{3.} [transitive] to interrupt somebody; to talk to somebody when they do not want to talk to you; \textbf{be bothered (about somebody\texttt{/}something)} [idiom] (\textit{especially British English, informal}) to think that somebody\texttt{/}something is important; [n] \textbf{1.} [uncountable] trouble or difficulty; \textbf{2.} \textbf{a bother} [singular] an annoying situation, thing or person, \textsc{synonym}: \textbf{nuisance}; [exclamation] (\textit{British English, informal}) used to express the fact that you are annoyed about something\texttt{/}somebody.} to provide the missing\footnote{\textbf{missing} [a] \textbf{1.} not available, e.g. because it has been removed, lost or destroyed; not included; \textbf{2.} used when it is not known where somebody is, or whether somebody is alive.} link. Perhaps the writer has used a word incorrectly\footnote{\textbf{incorrectly} [adv] \textbf{1.} in a way that is not accurate or true, \textsc{opposite}: \textbf{correctly}; \textbf{2.} in the wrong way; not as it should be, \textsc{opposite}: \textbf{correctly}.} by not taking the trouble to look it up.

Faced with such obstacles\footnote{\textbf{obstacle} [n] \textbf{1.} a situation, event or fact that makes it difficult for you to do or achieve something; \textbf{2.} an object that is in your way \& that makes it difficult for you to move forward.}, readers are at 1st tenacious\footnote{\textbf{tenacious} [a] (\textit{formal}) \textbf{1.} that does not stop holding something or give up something easily; determined; \textbf{2.} continuing to exist, have influence, etc. for longer than you might expect, \textsc{synonym}: \textbf{persistent}.}. They blame themselves -- they obviously\footnote{\textbf{obviously} [adv] \textbf{1.} used when giving information that you expect other people to know already or agree with, \textsc{synonym}: \textbf{clearly}; \textbf{2.} in a way that is easy to see or understand, \textsc{synonym}: \textbf{clearly}.} missed something, \& they go back over the mystifying\footnote{\textbf{mystifying} [a] making somebody confused because they do not understand something, \textsc{synonym}: \textbf{baffling}.} sentence, or over the whole paragraph, piecing\footnote{\textbf{piece} [v] \textbf{piece together} [phrasal verb] \textbf{piece something $\leftrightarrow$ together} \textbf{1.} to understand a story, situation, etc. by taking all the facts \& details about it \& putting them together; \textbf{2.} to put all the separate parts of something together to make a complete whole, \textsc{synonym}: \textbf{assemble}.} it out like an ancient\footnote{\textbf{ancient} [a] \textbf{1.} belonging to a period of history that is thousands of years in the past, \textsc{opposite}: \textbf{modern}; \textbf{2.} very old; having existed for a very long time; \textbf{3.} (\textbf{the ancients}) [n] [plural] the people who lived in ancient times, especially the Egyptians, Geeks \& Romans.} rune\footnote{\textbf{rune} [n] \textbf{1.} 1 of the letters in an alphabet that people in northern Europe used in ancient times \& cut into wood or stone; \textbf{2.} a symbol that has a mysterious or magic meaning.}, making guesses \& moving on. But they won't do that for long. The writer is making them work too hard, \& they will look for one who is better at the craft.

Writers must therefore constantly\footnote{\textbf{constantly} [adv] all the time.} ask: what am I trying to say? Surprisingly\footnote{\textbf{surprisingly} [adv] in a way that causes surprise.} often they don't know. Then they must look at what they have written \& ask: have I said it? Is it clear to someone encountering the subject for the 1st time? If it's not, some fuzz\footnote{\textbf{fuzz} [n] \textbf{1.} [uncountable] short soft fine hair or fur that covers something, especially a person's face or arms, \textsc{synonym}: \textbf{down}; \textbf{2.} [singular] a mass of curly hair; \textbf{3.} \textbf{the fuzz} [singular $+$ singular or plural verb] (\textit{old-fashioned, slang}) the police; \textbf{4.} something that you cannot see clearly, \textsc{synonym}: \textbf{blur}.} has worked its way into the machinery\footnote{\textbf{machinery} [n] \textbf{1.} [uncountable] machines as a group, especially large ones; \textbf{2.} [uncountable, singular] the parts of a living thing that are involved in a particular process; \textbf{3.} [uncountable, singular] the organization or structure of something; the system for doing something.}. The \fbox{clear writer} is someone clearheaded\footnote{\textbf{clear-headed} [a] able to think in a clear \& sensible way, especially in a difficult situation.} enough to see this stuff\footnote{\textbf{stuff} [n] [uncountable] \textbf{1.} something that something else is based on or is made from; the most important feature of something; \textbf{2.} (\textit{informal}) used to refer to a substance, a material, a group of objects, some information, etc. when you do not know the name, when the name is not important or when it is obvious what you are talking about.} or what it is: fuzz.

I don't mean that some people are born clearheaded \& are therefore natural\footnote{\textbf{natural} [a] \textbf{1.} [only before noun] existing in nature; not made or caused by humans; \textbf{2.} normal; as you would expect. If somebody dies of \textbf{natural causes}, they die of old age or disease rather than violence.; \textbf{3.} (of behavioral or an ability) part of the character that a person or an animal was born with; \textbf{4.} [only before noun] having an ability that you were born with; \textbf{5.} [only before noun] based on a sense of what is right \& wrong; \textbf{6.} [only before noun] (of parents or their children) related by blood.} writers, whereas others are naturally\footnote{\textbf{naturally} [adv] \textbf{1.} existing or happening as a normal part of nature, without special help, treatment or action by somebody; \textbf{2.} in a way that you would expect, \textsc{synonym}: \textbf{of course}; \textbf{3.} as a normal, logical result of something; \textbf{4.} in a way that shows or uses abilities or qualities that a person or an animal is born with; \textbf{come naturally (to somebody\texttt{/}something)} [idiom] if something comes naturally to you, you are able to do it very easily \& very well.} fuzzy\footnote{\textbf{fuzzy} [a] (\textbf{fuzzier, fuzziest}) \textbf{1.} covered with short soft fine hair or fur, \textsc{synonym}: \textbf{downy}; \textbf{2.} (of hair) in a mass of tight curls; \textbf{3.} not clear in shape or sound, \textsc{synonym}: \textbf{blurred}; \textbf{4.} confused \& not expressed clearly.} \& will never write well. Thinking clearly is a conscious\footnote{\textbf{conscious} [a] \textbf{1.} [not before noun] aware of something; noticing something, \textsc{opposite}: \textbf{unconscious}; \textbf{2.} able to use your senses \& mental powers to understand what is happening, \textsc{opposite}: \textbf{unconscious}; \textbf{3.} (of actions, feelings, etc.) deliberate or controlled, \textsc{opposite}: \textbf{unconscious}; \textbf{4.} being particularly interested in something.} act that writers must force on themselves, as if they were working on any other project that requires logic: making a shopping list or doing an algebra problem. Good writing doesn't come naturally, though most people seem to think it does. Professional writers are constantly bearded\footnote{\textbf{bearded} [a] having a beard.} by people who say they'd like to ``try a little writing sometime'' -- meaning when they retire\footnote{\textbf{retire} [v] [intransitive] to stop doing your job, especially because you have reached a particular age.} from their real profession\footnote{\textbf{profession} [n] \textbf{1.} [countable] a type of job that needs special training or skill, especially one that needs a high level of education; \textbf{2.} (\textbf{the profession}) [singular $+$ singular or plural verb] all the people who work in a particular profession; \textbf{3.} (\textbf{the professions}) [plural] the traditional jobs that need a high level of education \& training, such as being a doctor or lawyer; \textbf{4.} [countable] \textbf{profession of something} a statement about what you believe, feel or think about something, that is sometimes made publicly, \textsc{synonym}: \textbf{declaration}.}, like insurance\footnote{\textbf{insurance} [n] \textbf{1.} [uncountable, countable] an agreement with a company in which you pay them regular amounts of money \& they agree to pay the costs, e.g., if you die or are ill, or if you lose or damage something; \textbf{2.} [uncountable] the business of providing people with insurance; \textbf{3.} [uncountable] money paid by or to an insurance company; \textbf{4.} [uncountable, countable] \textbf{insurance (against something)} something you do to protect yourself against something bad happening in the future.} or real estate\footnote{\textbf{real estate} [n] [uncountable] (\textit{especially North American English}) \textbf{1.} property in the form of land or buildings; \textbf{2.} the business of selling houses or land for building.}, which is hard. Or they say, ``I could write a book about that.'' I doubt\footnote{\textbf{doubt} [n] [uncountable, countable] a feeling of not being sure about something or not believing something; [v] \textbf{1.} to not feel sure about something; to feel that something is not true or will probably not happen; \textbf{2.} \textbf{doubt somebody\texttt{/}something} to not trust somebody\texttt{/}something; to not believe somebody.} it.

\fbox{Writing is hard work.} \fbox{A clear sentence is no accident.} Very few sentences come out right the 1st time, or even the 3rd time. Remember this in moments of despair. If you find that writing is hard, it's because it \textit{is} hard.

\textsf{Figures. 2 pages of the final manuscript of this chapter from the 1st Edition of \textit{On Writing Well}. Although they look like a 1st draft\footnote{\textbf{draft} [n] \textbf{1.} [countable] a rough written version of something that is not yet in its final form; \textbf{2.} (\textbf{the draft}) [singular] (\textit{US}) $=$ \textbf{conscription}; [v] (also \textbf{draught} \textit{especially in British English}) \textbf{1.} \textbf{draft something} to write the 1st rough version of something such as a letter, speech or book; \textbf{2.} [usually passive] \textbf{be drafted $+$ adv.\texttt{/}prep.} to choose people \& send them somewhere for a special task; \textbf{3.} [usually passive] \textbf{be drafted (into something)} (\textit{US}) $=$ \textbf{conscript}.}, they had already been rewritten\footnote{\textbf{rewrite} [v] to write something again in a different way, e.g. in order to improve it or because there is some new information.} \& retyped -- like almost every other page -- 4 or 5 times. With each rewrite I try to make what I have written tighter\footnote{\textbf{tight} [a] (\textbf{tighter, tightest}) \textbf{1.} held, fixed or closed firmly; difficult to move or open; \textbf{2.} (of a rule or a form of control) very strict \& firm; \textbf{3.} (of clothes, etc.) fitting closely to the body \& sometimes uncomfortable, \textsc{opposite}: \textbf{loose}; \textbf{4.} stretched or pulled so that it cannot stretch much further; \textbf{5.} [usually before noun] with things or people packed closely together, leaving little space between them; \textbf{6.} (of a piece of writing, an argument, etc.) well organized, giving only the information that is important, \textsc{synonym}: \textbf{concise}; \textbf{7.} (of time, money, etc.) difficult to manage with because there is not enough; \textbf{8.} (of part of the body) feeling painful or uncomfortable because of illness or emotion; \textbf{9.} having a close relationship with somebody else or with other people, \textsc{synonym}: \textbf{close}; \textbf{10.} curving suddenly or in a small circle rather than gradually, \textsc{synonym}: \textbf{sharp}.}, stronger \& more precise\footnote{\textbf{precise} [a] \textbf{1.} clear \& accurate; \textbf{2.} [only before noun] used to emphasize that something happens at a particular time or in a particular way; \textbf{to be (more) precise} [idiom] used to show that you are giving more detailed \& accurate information about something you have just mentioned.}, eliminating\footnote{\textbf{eliminate} [v] \textbf{1.} to remove or get rid of something\texttt{/}somebody; \textbf{2.} \textbf{eliminate somebody} to kill somebody, especially an enemy or opponent; \textbf{3.} \textbf{eliminate something} (\textit{mathematics}) to remove a variable from an equation, typically by substituting another which is shown by another equation to have the same value; \textbf{4.} \textbf{eliminate something} (\textit{chemistry}) to produce a simple substance such as water in addition to a more complex substance as a result of a chemical reaction involving larger organic molecules.} every element\footnote{\textbf{element} [n] \textbf{1.} [countable] a necessary or typical part of something; \textbf{2.} [countable] a simple chemical substance that consists o atoms of only 1 type \& cannot be split by chemical means into a simpler substance. Gold, oxygen \& carbon are all elements.; \textbf{3.} [countable, usually singular] \textbf{element of risk, truth, surprise, etc.} a small amount of a quality or feeling; \textbf{4.} [countable, usually plural] \textbf{element (of something)} a group of people who form a part of a larger group or society; \textbf{5.} [countable] (\textit{mathematics}) a member of a set or group; \textbf{6.} [countable] the part of a piece of electrical equipment that gives out heat; \textbf{7.} [countable] 1 of the 4 substances (earth, air, fire \& water) which people used to believe everything else was made of; \textbf{8.} (\textbf{the elements}) [plural] the weather, especially bad weather; \textbf{in your element} [idiom] doing what you are good at \& enjoy.} that's not doing useful work. Then I go over it once more, reading it aloud, \& am always amazed at how much clutter can still be cut. (In later editions I eliminated the sexist\footnote{\textbf{sexist} [a] (\textit{disapproving}) treating other people, especially women, unfairly because of their sex or making offensive remarks about them; [n] (\textit{disapproving}) a person who treats other people, especially women, unfairly because of their sex or who makes offensive remarks about them.} pronoun ``he'' denoting ``the writer'' \& ``the reader.'')}'' -- \cite[pp. 15--18]{Zinsser2016}

\section{Clutter}
``Fighting clutter is like fighting weeds\footnote{\textbf{weed} [n] \textbf{1.} [countable] a wild plant growing where it is not wanted, especially among crops or garden plants; \textbf{2.} [uncountable] any wild plant without flowers that grows in water \& forms a green floating mass; \textbf{3.} \textbf{the weed} [singular] (\textit{humorous}) tobacco or cigarettes; \textbf{4.} [uncountable] (\textit{informal}) the drug cannabis; \textbf{5.} [countable] (\textit{British English, informal, disapproving}) a person with a weak character or body; [v] [transitive, intransitive] \textbf{weed (something)} to take out weeds from the ground; \textbf{weed out} [phrasal verb] \textbf{weed something\texttt{/}somebody $\leftrightarrow$ out} to remove or get rid of people or things from a group because they are not wanted or are less good than the rest.} -- the writer is always slightly\footnote{\textbf{slightly} [adv] \textbf{1.} a little; \textbf{2.} a \textbf{slight built} person is small \& thin.} behind. New varieties\footnote{\textbf{variety} [n] (plural \textbf{varieties}) \textbf{1.} [singular] \textbf{variety (of something)} a number or range of different things of the same general type; \textbf{2.} [uncountable] the quality of not being the same in all parts or not doing the same thing all the time, \textsc{synonym}: \textbf{diversity}; \textbf{3.} [countable] a type of thing, e.g. a plant or language, that is different from others in the same general group. In biology, a \textbf{variety} is a category below a \textbf{species} \& \textbf{subspecies}, used especially to describe plants.} sprout\footnote{\textbf{sprout} [v] \textbf{1.} [intransitive] (of plants or seeds) to produce new leaves or buds; to start to grow; \textbf{2.} [intransitive, transitive] to appear; to develop something, especially in large numbers; \textbf{3.} [transitive, intransitive] to start to grow something; to start to grow on somebody\texttt{/}something; [n] \textbf{1.} (also \textbf{Brussels sprout, Brussel sprout}) a small round green vegetable like a very small cabbage; \textbf{2.} a new part growing on a plant.} overnight\footnote{\textbf{overnight} [adv] \textbf{1.} during or for the night; \textbf{2.} suddenly or quickly.}, \& by noon they are part of American speech. Consider what President Nixon's aide\footnote{\textbf{aide} [n] a person who helps another person, especially a politician, in their job.} Join Dean accomplished\footnote{\textbf{accomplish} [v] to succeed in doing or completing something, \textsc{synonym}: \textbf{achieve}.} in just 1 day of testimony\footnote{\textbf{testimony} [n] (plural \textbf{testimonies}) \textbf{1.} [uncountable, countable] a formal written or spoken statement saying what you know to be true, usually in court; \textbf{2.} [uncountable, singular] \textbf{testimony (to something)} a thing that shows that something else exists or is true; \textbf{bear witness\texttt{/}testimony to something} [idiom] to provide evidence of the truth of something.} on television during the Watergate\footnote{\textbf{Watergate} the US political scandal that forced President Richard Nixon to leave office in 1974. It involved Republican Party members who in 1972 tried to steal information from the offices of the Democratic party in the Watergate building in Washington, DC. Nixon said he did not know about this, but The Washing Post \& tapes of his phone conversations proved he did. He resigned as Congress was about to begin impeachment, \& several important government officials were sent to prison for illegally trying to keep the affair secret. The Watergate incident made the role of the President weaker for several years \& many people were shocked that people in power that behaved so badly. The word ending \textit{-gate} has since been used to create names for other scandals, e.g., \textbf{Irangate}.} hearings\footnote{\textbf{hearing} [n] \textbf{1.} [uncountable] the ability to hear; \textbf{2.} [countable] an official meeting at which the facts about a crime or complaint are presented to the person or group of people who will have to decide what action to take; \textbf{3.} [singular] an opportunity to explain your actions, ideas or opinions.}. The next day everyone in America was saying ``at this point in time'' instead of ``now.''

Consider all the prepositions\footnote{\textbf{preposition} [n] (\textit{grammar}) a word or group of words, such as \textit{in, from, to, out of} \& \textit{on behalf of}, used before a noun or pronoun to show place, position, time or method. The preposition is sometimes placed at the end of the clause. In written academic language, this is sometimes considered incorrect, even though it may sound more natural. If in doubt, put the preposition before the noun or pronoun to which it relates.} that are draped onto\footnote{\textbf{drape} [v] \textbf{1.} [transitive] \textbf{drape something around\texttt{/}over\texttt{/}across, etc. something} to hang clothes, materials, etc. loosely on somebody\texttt{/}something; \textbf{2.} [intransitive] (of clothes or materials) to hang loosely; \textbf{3.} [transitive] \textbf{drape somebody\texttt{/}something in \texttt{/}with something} to cover or decorate somebody\texttt{/}something with material; \textbf{4.} [transitive] \textbf{drape something around\texttt{/}round\texttt{/}over, etc. something} to allow part of your body to rest on something in a relaxed way; [n] (\textit{especially North American English}) (\textit{North American English also} \textbf{drapery}) [usually plural] a long thick curtain.} verbs that don't need any help. We no longer head committees\footnote{\textbf{committee} [n] [countable $+$ singular or plural verb] a group of people who are chosen, usually by a larger group, to make decisions or deal with a particular subject. A committee typically consists of members of the larger group.}. We head them up. We don't face problems anymore. We face up to them when we can free up a few minutes. A small detail\footnote{\textbf{detail} [n] \textbf{1.} [uncountable] exact information about something; \textbf{2.} [countable, usually plural] a small individual fact or item of information; \textbf{3.} [countable, uncountable] a small part of something that can be looked at; 1 or more of these taken together; \textbf{4.} [plural] information about somebody such as their name, address age, etc.; [v] \textbf{1.} to give full information about a subject; \textbf{2.} \textbf{detail something} to give a list of facts or items.}, you may say -- not worth bothering about. It \textit{is} worth bothering about. Writing improves\footnote{\textbf{improve} [v] [intransitive, transitive] to become better than before; to make something\texttt{/}somebody better than before; \textbf{improve on\texttt{/}upon something} [phrasal verb] to achieve or produce something that is of a better quality than something else.} in direct\footnote{\textbf{direct} [a] \textbf{1.} [usually before noun] happening or done without involving other people or factors; having no one or nothing in between. \textbf{Direct cost} describes the costs involved in producing a product, such as raw materials \& wages. \textbf{Direct taxation} describes tax that is paid on a person's income or a company's profits, rather than on the goods \& services they buy., \textsc{opposite}: \textbf{indirect}; \textbf{2.} [only before noun] with nothing between something\texttt{/}somebody \& the source of light or heat; \textbf{3.} [usually before noun] clear \& able to be understood in only 1 way, \textsc{opposite}: \textbf{indirect}; \textbf{4.} (\textit{sometimes disapproving}) (of a person or their behavior) saying what you mean in a way that is honest \& that can be understood in only 1 way, \textsc{synonym}: \textbf{frank}, \textsc{opposite}: \textbf{indirect}; \textbf{5.} going in the straightest line between 2 places without stopping or changing direction, \textsc{opposite}: \textbf{indirect}; \textbf{6.} [only before noun] related through parents \& children rather than brothers or sisters; \textbf{7.} [only before noun] \textbf{direct quotation} taken from somebody's words without being changed; [v] \textbf{1.} [transitive] to give attention or effort something; \textbf{2.} [transitive] to aim something in a particular direction or at a particular person or object; \textbf{3.} [transitive] \textbf{direct something at somebody\texttt{/}something} to aim a comment or criticism at somebody\texttt{/}something; \textbf{4.} [transitive] \textbf{direct something} to control or manage something; \textbf{5.} [transitive, intransitive] \textbf{direct (somebody\texttt{/}something)} to be in charge of actors in a play or film, a group of musicians, etc.; \textbf{6.} [transitive] \textbf{direct somebody to something} to tell or show somebody how to reach a place or access something; \textbf{7.} [transitive] to give an official order or instruction.} ratio\footnote{\textbf{ratio} [n] (plural \textbf{ratios}) the relationship between 2 groups of people or things that is represented by 2 numbers showing how much larger 1 group is than the other.} to the number of things we can keep out of it that shouldn't be there. ``Up'' in ``free up'' shouldn't be there. Examine every word you put on paper. You'll find a surprising number that don't serve any purpose\footnote{\textbf{purpose} [n] \textbf{1.} [countable, uncountable] the aim, intention or function of something; the thing that something is supposed to achieve; \textbf{2.} (\textbf{purpose}) [plural] what is needed or being considered in a particular situation; \textbf{3.} [uncountable, countable] the feeling that what you are doing is valuable; something important that you want to achieve.}.

Take the adjective ``personal\footnote{\textbf{personal} [a] \textbf{1.} [only before noun] your own; not belonging to or connected with anyone else; \textbf{2.} [only before noun] connected with individual people, especially their feelings, characters \& relationships; \textbf{3.} not connected with a person's job or official position; \textbf{4.} [only before noun] done by a particular person rather than by somebody who is acting for them; \textbf{5.} [only before noun] made or done for a particular person rather than for a large group of people or people in general; \textbf{6.} [only before noun] connected with a person's body; \textbf{7.} connected with a particular person's character, appearance or private life in a way that is offensive.},'' as in ``a personal friend of mine,'' ``his personal feeling\footnote{\textbf{feeling} [n] \textbf{1.} [countable] \textbf{feeling (of something)} something that you feel through the mind or the senses; \textbf{2.} [singular] the idea or belief that a particular thing is true or a particular situation is likely to happen; \textbf{3.} [countable, uncountable] an attitude or opinion about something; \textbf{4.} (\textbf{feelings}) [plural] a person's emotions rather than their thoughts or ideas; \textbf{5.} [uncountable] \textbf{feeling for somebody\texttt{/}something} the ability to understand somebody\texttt{/}something or to do something in a sensitive way; \textbf{6.} [plural, uncountable] \textbf{feeling (for somebody\texttt{/}something)} sympathy or love for somebody\texttt{/}something; \textbf{7.} [uncountable] strong emotion; \textbf{8.} [uncountable] the ability to feel physically, \textsc{synonym}: \textbf{sensation}; \textbf{9.} [singular] \textbf{feeling (of something)} the atmosphere of a place or situation.}'' or ``her personal physician\footnote{\textbf{physician} [n] a doctor, especially one who is a specialist in general medicine \& not surgery. In British English, \textbf{physician} is used mainly in academic \& formal English. In general English, \textbf{doctor} or \textbf{GP} is used instead. In American English, however, \textbf{physician} is used in both academic \& general English.}.'' It's typical\footnote{\textbf{typical} [a] \textbf{1.} having the usual qualities or features of a particular type of a person, thing or group, \textsc{opposite}: \textbf{atypical}; \textbf{2.} happening in the usual way; showing what something is usually like; \textbf{3.} behaving in the way that you expect.} of hundreds of words that can be eliminated. The personal friend has come into the language to distinguish\footnote{\textbf{distinguish} [v] \textbf{1.} [intransitive, transitive] to recognize or show the difference between 2 people or things, \textsc{synonym}: \textbf{differentiate}; \textbf{2.} [transitive] (not used in the progressive tenses) to be a characteristic that makes 2 people, animals or things different, \textsc{synonym}: \textbf{differentiate}; \textbf{3.} [transitive] \textbf{distinguish A (from B)} to make something different or seem different from other similar things, \textsc{synonym}: \textbf{differentiate}; \textbf{4.} [transitive] to do something so well that people notice \& admire you; \textbf{5.} [transitive] (not used in the progressive tenses) \textbf{distinguish something} to be able to see or hear something, \textsc{synonym}: \textbf{make somebody\texttt{/}something out}.} him or her from the business\footnote{\textbf{business} [n] \textbf{1.} [uncountable] the activity of making, buying, selling or supplying goods or services for money; \textbf{2.} [countable] a commercial organization such as a company, shop or factory. In this meaning, the word \textbf{business} often describes a small or medium-sized organization; the word \textbf{company} can be used for both small \& large organization.; \textbf{3.} [uncountable] work or another activity that is part of your job \& not done for pleasure or for any other reason; \textbf{4.} [uncountable] the amount of work done by a company, etc,; the rate, volume, value or quality of this work; \textbf{5.} [countable] a particular area of commercial activity; \textbf{6.} [uncountable] the fact of a person or people buying goods or services from a business; \textbf{7.} [uncountable] something that concerns a particular person or organization; \textbf{8.} [uncountable] important matters that need to be dealt with or discussed; \textbf{9.} [singular] (usually with an adjective) \textbf{business (of something\texttt{/}of doing something)} a situation or a series of events.} friend, thereby debasing\footnote{\textbf{debase} [v] \textbf{debase somebody\texttt{/}something} to make somebody\texttt{/}something less valuable or respected, \textsc{synonym}: \textbf{devalue}.} both language \& friendship\footnote{\textbf{friendship} [n] \textbf{1.} [countable] a relationship between friends; \textbf{2.} [uncountable] the feeling or relationship that friends have; the state of being friends.}. Someone's feeling \textit{is} that person's personal feeling -- that's what ``his'' means. As for the personal physician, that's the man or woman summoned\footnote{\textbf{summon} [v] \textbf{1.} to order somebody to appear in court; \textbf{2.} \textbf{summon somebody (to something) (to do something)} to order somebody to come to you; \textbf{3.} \textbf{summon something} to arrange an official meeting, \textsc{synonym}: \textbf{convene}; \textbf{4.} \textbf{summon something} to call for or try to obtain something; \textbf{5.} \textbf{summon something (up)} to make an effort to produce a particular quality in yourself, especially when you find it difficult; \textbf{summon something up} [phrasal verb] to make a feeling, an idea, a memory, etc. come into your mind, \textsc{synonym}: \textbf{evoke}.} to the dressing room of a stricken\footnote{\textbf{stricken} [a] (\textit{formal}) \textbf{1.} seriously affected by an unpleasant feeling or disease or by a difficult situation; \textbf{2.} (in compounds) seriously affected by the thing mentioned.} actress\footnote{\textbf{actress} [n] \textbf{1.} a woman who performs on the stage, on television or in films, especially as a profession; \textbf{2.} a woman who plays a part, pretending by her behavior to be a particular kind of person.} so she won't have to be treated by the impersonal\footnote{\textbf{impersonal} [a] \textbf{1.} not referring to any particular person; \textbf{2.} lacking friendly human feelings or atmosphere.} physician assigned\footnote{\textbf{assign} [v] \textbf{1.} to give somebody something that they can use, or some work or a duty, \textsc{synonym}: \textbf{allocate}; \textbf{2.} to say that somebody\texttt{/}something is responsible for something; \textbf{3.} to say that something has a particular value or function, or happens at a particular time or place; \textbf{4.} to choose somebody for a particular task, position or purpose; \textbf{5.} [usually passive] \textbf{assign somebody to somebody\texttt{/}something} to send a person to work or live under the authority of somebody or in a particular group or place; \textbf{6.} \textbf{assign something to somebody} (\textit{law}) to say that your property or rights now belong to somebody else.} to the theater\footnote{\textbf{movie theater} [n] (also \textbf{theater}) (\textit{North American English}) $=$ \textbf{cinema}.}. Someday I'd like to see that person identified as ``her doctor.'' Physicians are physicians, friends are friends. The rest is clutter.

Clutter is the laborious\footnote{\textbf{laborious} [a] taking a lot of time \& effort.} phrase\footnote{\textbf{phrase} [n] \textbf{1.} (\textit{grammar}) a group of words without a finite verb, especially one that forms part of a sentence; \textbf{2.} a group of words that have a particular meaning when used together; [v] [often passive] to say or write something in a particular way.} that has pushed out the short word that means the same thing. Even before John Dean, people \& business had stopped saying ``now.'' They were saying ``currently\footnote{\textbf{currently} [adv] at the present time.}'' (``all our operators\footnote{\textbf{operator} [n] \textbf{1.} (often in compounds) a person or company that runs a particular business; \textbf{2.} (often in compounds) a person who operates equipment or a machine; \textbf{3.} (\textit{mathematics}) a symbol or function which represents an operation in mathematics, e.g., $\times,+$.} are currently assisting\footnote{\textbf{assist} [v] \textbf{1.} [intransitive, transitive] to help somebody to do something; \textbf{2.} [transitive] \textbf{assist something} to help something to happen more easily, \textsc{synonym}: \textbf{facilitate}.} other customers\footnote{\textbf{customer} [n] a person or an organization that buys goods \& services.}''), or ``at the present time,'' or ``presently\footnote{\textbf{presently} [adv] \textbf{1.} (usually used before the word or sentence that it refers to) at the time you are speaking or writing; now, \textsc{synonym}: \textbf{currently}; \textbf{2.} (usually used at the end of a sentence or clause) at a later time, e.g. at a later point in the text that you are writing.}'' (which means ``soon''). Yet the idea can always be expressed by ``now'' to mean the immediate\footnote{\textbf{immediate} [a] \textbf{1.} happening or done without delay, \textsc{synonym}: \textbf{instant}; \textbf{2.} [usually before noun] existing now \& needing urgent attention; \textbf{3.} [only before noun] next to or very close to a particular place or time; \textbf{4.} [only before noun] nearest in relationship or rank; \textbf{5.} [only before noun] having a direct effect; \textbf{with immediate effect, with effect from }[idiom] starting now; starting from $\ldots$.} moment\footnote{\textbf{moment} [n] \textbf{1.} [singular] a particular occasion or exact point in time; \textbf{2.} [countable] \textbf{moment (of something)} a very short period of time; \textbf{at the\texttt{/}this moment} [idiom] now; at the present time; \textbf{for the moment} [idiom] for now. \textbf{For the moment} is used in writing to ask the reader to consider 1 particular aspect or possibility 1st, before going on to consider other aspects or possibilities.} (``Now I can see him''), or by ``today'' to mean the historical\footnote{\textbf{historical} [a] [usually before noun] \textbf{1.} connected with the past; \textbf{2.} connected with the study of history; \textbf{3.} (of a book or film) about people \& events in the past.} present (``Today prices are high''), or simply by the verb ``to be'' (``It is raining''). There's no need to say, ``At the present time we are experiencing precipitation\footnote{\textbf{precipitation} [n] \textbf{1.} [uncountable] (\textit{specialist}) rain, snow, etc. that falls; the amount of this that falls; \textbf{2.} [uncountable, countable] \textbf{precipitation (of something)} (\textit{chemistry}) a chemical process in which solid material is separated from a liquid.}.''

\fbox{``Experiencing'' is 1 of the worst clutterers.} Even your dentist\footnote{\textbf{dentist} [n] a person whose job is to take care of people's teeth.} will ask if you are experiencing any pain\footnote{\textbf{pain} [n] [uncountable, countable] \textbf{1.} the feelings that somebody has in their body when they have been hurt or when they are ill; \textbf{2.} \textbf{pain (of something)} mental or emotional suffering; \textbf{on pain of something} [idiom] with the threat of having something done to you as a punishment if you do not obey.}. If he had his own kid in the chair he would say, ``Does it hurt?'' He would, in short, be himself. By using a more pompous\footnote{\textbf{pompous} [a] (\textit{disapproving}) showing that you think you are more important than other people, especially by using long \& formal words, \textsc{synonym}: \textbf{pretentious}.} phrase in his professional role\footnote{\textbf{role} [n] \textbf{1.} the function that somebody\texttt{/}something has or the part somebody\texttt{/}something plays in a particular situation; \textbf{2.} an actor's part in a play, film, etc.} he not only sounds more important; he blunts\footnote{\textbf{blunt} [v] \textbf{1.} \textbf{blunt something} to make something weaker or less effective; \textbf{2.} \textbf{blunt something} to make a point or an edge less sharp.} the painful\footnote{\textbf{painful} [a] \textbf{1.} causing pain; \textbf{2.} causing you to feel upset or embarrassed; \textbf{3.} unpleasant or difficult to do.} edge\footnote{\textbf{edge} [n] \textbf{1.} [countable] the outside limit of an object, a surface or an area; the part furthest from the center; \textbf{2.} (usually \textbf{the edge}) [singular] the point at which something is unpleasant or important may begin to happen; \textbf{3.} [singular] a slight advantage over somebody\texttt{/}something; \textbf{4.} [singular] a strong, often exciting, quality.} of truth\footnote{\textbf{truth} [n] \textbf{1.} [uncountable] the quality or state of being true, \textsc{opposite}: \textbf{falsity}; \textbf{2.} (\textbf{the truth}) [singular] the true facts about something, rather than the things that have been invented or guessed; \textbf{3.} [countable] a fact that is believed by most people to be true; \textbf{in truth} [idiom] used to emphasize the true facts about a situation; \textbf{to tell\texttt{/}speak the truth} [idiom] to say things that are true.}. It's the language of the flight attendant\footnote{\textbf{flight attendant} [n] a person whose job is to serve \& take care of passengers on an aircraft.} demonstrating\footnote{\textbf{demonstrate} [v] \textbf{1.} [transitive] to show something clearly by giving proof or evidence; \textbf{2.} [transitive] \textbf{demonstrate something} to show by your actions that you have a particularly quality, feeling or opinion, \textsc{synonym}: \textbf{display}; \textbf{3.} [transitive] \textbf{demonstrate something} to show \& explain how something works or how to do something; \textbf{4.} [intransitive] \textbf{demonstrate (against somebody\texttt{/}something)} to take part in a public meeting or march, usually as a protest or to show support for something.} the oxygen mask\footnote{\textbf{oxygen mask} [n] a device placed over the nose \& mouth through which a person can breathe oxygen, e.g. in an aircraft or a hospital.} that will drop down if the plane should run out of air. ``In the unlikely\footnote{\textbf{unlikely} [a] (\textbf{unlikelier, unlikeliest}) (\textbf{more unlikely} \& \textbf{most unlikely} are the usual forms.) \textbf{1.} not likely to happen, \textsc{opposite}: \textbf{likely}; \textbf{2.} [only before noun] not the person, thing or place that you would normally think of or expect; \textbf{3.} [only before noun] not likely to be true, \textsc{synonym}: \textbf{implausible}.} possibility\footnote{\textbf{possibility} [n] (plural \textbf{possibilities}) \textbf{1.} [uncountable, countable] the fact that something might exist, happen, or be true, but is not certain; \textbf{2.} [countable, usually plural] 1 of the different things that you can do in a particular situation.} that the aircraft\footnote{\textbf{aircraft} [n] (plural \textbf{aircraft}) any vehicle that can fly \& carry goods or passengers.} should experience such an eventuality\footnote{\textbf{eventuality} [n] (plural \textbf{eventualities}) something that many possibly happen, especially something unpleasant.},'' she begins -- a phrase so oxygen-depriving\footnote{\textbf{deprive} [v] \textbf{deprive of} [phrasal verb] \textbf{deprive somebody\texttt{/}something of something} to prevent somebody from having or doing something, especially something important.} in itself that we are prepared for any disaster\footnote{\textbf{disaster} [n] \textbf{1.} [countable] an unexpected event that kills a lot of people or causes a lot of damage, \textsc{synonym}: \textbf{catastrophe}; \textbf{2.} [countable, uncountable] \textbf{disaster (for somebody\texttt{/}something)} a very bad situation that causes problems.}.

Clutter is the ponderous\footnote{\textbf{ponderous} [a] (\textit{formal}) \textbf{1.} (\textit{disapproving}) (of speech \& writing) too slow \& careful; serious \& boring, \textsc{synonym}: \textbf{tedious}; \textbf{2.} moving slowly \& heavily; able to move only slowly, \textsc{synonym}: \textbf{labored}.} euphemism\footnote{\textbf{euphemism} [n] \textbf{euphemism (for something)} an indirect word or phrase that people often use to refer to something embarrassing or unpleasant, sometimes to make it seem more acceptable than it really is.} that turns a slum\footnote{\textbf{slum} [n] an area of a city that is very poor \& where the houses are dirty \& in bad condition.} into a depressed\footnote{\textbf{depressed} [a] \textbf{1.} suffering from the medical condition of depression. In non-academic English, \textbf{depressed} is often used to refer to a less serious feeling of sadness, which is not considered to be a medical condition; \textbf{2.} [usually before noun] (of a place or an industry) without enough economic activity or jobs for people; \textbf{3.} having a lower amount or level than usual.} socioeconomic\footnote{\textbf{socio-economic} [a] connected with society \& economics.} area\footnote{\textbf{area} [n] \textbf{1.} [countable] part of a town, a country or the world; \textbf{2.} [countable] a particular subject or activity, or an aspect of it; \textbf{3.} [countable, uncountable] \textbf{area (of something)} the amount of space covered by a surface or piece of land, described as a measurement; \textbf{4.} [countable] \textbf{area (of something)} a particular place on an object or a diagram.}, garbage\footnote{\textbf{garbage} [n] [uncountable] (\textit{especially North American English}) waste food, paper, etc. that you throw away; a place or container where waste food, paper, etc. can be placed, \textsc{synonym}: \textbf{rubbish}.} collectors\footnote{\textbf{collector} [n] \textbf{1.} (especially in compounds) a person who collects things, either as a hobby or as a job; \textbf{2.} the leader of a district in some South Asian countries.}\,\footnote{\textbf{garbage man} [n] (also formal \textbf{garbage collector}) (\textit{both North American English}) (\textit{British English} \textbf{dustman}, \textbf{informal} \textbf{binman}, \textit{formal} \textbf{refuse collector}) a person whose job is to remove waste from outside houses, etc.} into waste-disposal\footnote{\textbf{waste} [n] \textbf{1.} [uncountable] (\textbf{wastes}) [plural] materials that are no longer needed \& are thrown away; \textbf{2.} [uncountable] material that the body gets rid of as solid or liquid material; \textbf{3.} [uncountable, singular] the act of using something in a careless or unnecessary way, causing it to be lost or destroyed; \textbf{4.} [singular] \textbf{waste of something} a situation in which it is not worth spending time, money, etc. on something; [v] \textbf{1.} to use more of something than is necessary or useful; \textbf{2.} to give, say, use, etc. something where it is not valued or used to its full or best effect; [a] [usually before noun] \textbf{1.} no longer needed for a particular process or remaining after a process has finished, \& therefore thrown away; \textbf{2.} (of land) not suitable for building or growing things on \& therefore not used.}\,\footnote{\textbf{disposal} [n] \textbf{1.} [uncountable] the act of getting rid of something; \textbf{2.} [countable] (\textit{business}) the sale of part of a business, property, etc.; \textbf{3.} [countable] $=$ \textbf{garbage disposal.}}\,\footnote{\textbf{garbage disposal} [n] (also \textbf{disposal}) (\textit{both North American English}) \textit{British English} \textbf{waste disposal unit, waste disposer} a machine connected to the waste pipe of a kitchen sink, for cutting food waste into small pieces.} personnel\footnote{\textbf{personnel} [n] [plural] the people who work for a business, an organization or 1 of the armed forces.} \& the town\footnote{\textbf{town} [n] [countable, uncountable] a place with many houses, shops, etc. where people live \& work. It is larger than a village but smaller than a city.} dump\footnote{\textbf{dump} [v] \textsf{get rid of} \textbf{1.} \textbf{dump something} to get rid of something you do not want, especially in a place that is not suitable; \textbf{2.} \textbf{dump somebody\texttt{/}something (on somebody)} (\textit{informal}) to get rid of; \textbf{3.} \textbf{dump something} (\textit{business}) to get rid of goods by selling them at a very low price, often in another country; \textsf{put down} \textbf{4.} \textbf{dump something\texttt{/}somebody} (\textit{informal}) to put something\texttt{/}somebody down in a careless or untidy way; \textsf{end relationship} \textbf{5.} \textbf{dump somebody} (\textit{informal}) to end a romantic relationship with somebody; \textsf{computing} \textbf{6.} \textbf{dump something} to copy information \& move it somewhere to store it; [n] \textsf{for waste} \textbf{1.} a place where waste or rubbish is taken \& left; \textbf{2.} (also \textbf{mine dump}) (\textit{South American English}) a hill that is formed when waste sand from the mining of gold is piled in 1 place over a period of time; \textsf{dirty place} \textbf{3.} (\textit{informal, disapproving}) a dirty or unpleasant place; \textsf{for weapons} \textbf{4.} a temporary store for military supplies; \textsf{computing} \textbf{5.} an act of copying data stored in a computer; a copy or list of the contents of this data; \textsf{waste from body} \textbf{6.} [countable] (\textit{slang}) an act of passing waste matter from the body through the bowels.} into the volume\footnote{\textbf{volume} [n] \textbf{1.} [uncountable, countable] \textbf{volume (of something)} the amount of space that an object or a substance fills; the amount of space that a container has; \textbf{2.} [uncountable, countable] \textbf{volume (of something)} the amount of something; \textbf{3.} [uncountable] the amount of sound that is produced by a television, radio, etc.; \textbf{4.} [countable] (abbr., \textbf{vol.}) a book that is part of a series of books; \textbf{5.} [countable] \textbf{volume (of something)} (\textit{formal}) a book; \textbf{6.} (abbr., \textbf{vol.}) \textbf{volume (of something)} a series of different issues of the same magazine, especially all the issues for 1 year.} reduction\footnote{\textbf{reduction} [n] \textbf{1.} [countable, uncountable] the action or fact of making something smaller or less in amount, size or degree; \textbf{2.} [uncountable] \textbf{reduction of something to something} an explanation of a subject or problem in terms of another simpler or more basic one; \textbf{3.} [countable] an amount of money by which something is made cheaper; \textbf{4.} [uncountable, countable] \textbf{reduction (of something)} (\textit{chemistry}) the fact of removing oxygen from a substance or adding hydrogen to a substance, \textsc{opposite}: \textbf{oxidation}; \textbf{5.} [uncountable, countable] (\textit{chemistry}) the fact of adding 1 or more electrons to a substance.} unit\footnote{\textbf{unit} [n] \textbf{1.} a single thing, person or group that is complete by itself but can also form part of something larger; \textbf{2.} (\textit{business}) a single item of the type of product that a company sells; \textbf{3.} a group of people who work or live together, especially for a particular purpose; \textbf{4.} a department, especially in a hospital, that provides a particular type of care or treatment; \textbf{5.} a fixed quantity that is used as a standard measurement; \textbf{6.} a small machine that has a particular purpose or is part of a larger machine; \textbf{7.} 1 of the parts into which a textbook or a series of lessons is divided.}. I think of Bill Mauldin's cartoon\footnote{\textbf{cartoon} [n] \textbf{1.} a humorous drawing in a newspaper or magazine, especially one about politics or events in the news; \textbf{2.} (also \textbf{comic strip}, \textit{British English} \textbf{strip cartoon}) a series of drawings inside boxes that tells a story \& are often printed in newspapers. A \textbf{cartoon} is really made using the old technique of photographing drawings or models; a film made with modern computer techniques (CGI) is usually called an \textbf{animation}, not a cartoon. However, some people might use the word \textbf{cartoon} for both kinds of film.; \textbf{4.} (\textit{specialist}) a drawing made by an artist as a preparation for a painting.} of 2 hoboes\footnote{\textbf{hobo} (plural \textbf{hobos} or \textbf{hoboes}) (\textit{especially North American English, old-fashioned}) \textbf{1.} a person who travels from place to place looking for work, especially on farms; \textbf{2.} (also \textbf{tramp}) a person with no home or job who travels from place to place, usually asking people in the street for food or money.} riding a freight\footnote{\textbf{freight} [n] [uncountable] goods that are transported by ships, planes, trains or lorries; the system of transporting goods in this way; [v] \textbf{1.} \textbf{freight something} to send or carry goods by air, sea or train; \textbf{2.} [usually passive] (\textit{literary}) to fill something with a particular mood or tone.} car\footnote{\textbf{freight car} [n] (\textit{North American English}) (\textit{British English} \textbf{wagon}) a railway truck for carrying goods.}. 1 of them says, ``I started as a simple bum\footnote{\textbf{bum} [n] (\textit{informal}) \textbf{1.} (\textit{British English}) the part of the body that you sit on, \textsc{synonym}: \textbf{backside, behind, bottom}; \textbf{2.} (\textit{especially North American English}) a person who has no home or job \& who asks other people for money or food; \textbf{3.} a lazy person who does nothing for other people or for society; [v] \textbf{1.} \textbf{bum something (from\texttt{/}off somebody)} (\textit{informal}) to get something from somebody by asking, \textsc{synonym}: \textbf{cadge}; \textbf{2.} \textbf{bum somebody (out)} (\textit{North American English, informal}) to make somebody feel upset or disappointed; [a] [only before noun] (\textit{informal}) of bad quality; wrong or not worth anything.}, but now I'm hard-core\footnote{\textbf{hard-core} [a] \textbf{1.} having a belief or a way of behaving that will not change; \textbf{2.} showing or describing sexual activity in a detailed or violent way.}\,\footnote{\textbf{hardcore} [n] [uncountable] \textbf{1.} a type of electronic pop music that became popular in the UK in the early 1990s. It is similar to techno, with a very fast beat \& few words; \textbf{2.} (\textit{British English}) small pieces of stone, brick, etc. used as a base for building roads on}\,\footnote{\textbf{hard core} [n] [singular $+$ singular or plural verb] the small central group in an organization, or in a particular group of people, who are the most active or who will not change their beliefs or behavior.} unemployed\footnote{\textbf{unemployed} [a] without a job, although able to work \& actively looking for work.}.'' Clutter is political\footnote{\textbf{political} [a] \textbf{1.} connected with the state, government or public affairs; \textbf{2.} connected with the different groups working in politics, especially their policies \& the competition between them; \textbf{3.} (of people) interested in or active in politics; \textbf{4.} concerned with the competition for power within an organization, rather than with matters of principle.} correctness\footnote{\textbf{correctness} [n] [uncountable] \textbf{1.} the quality of being accurate or true, without any mistakes, \textsc{synonym}: \textbf{accuracy}, \textsc{opposite}: \textbf{incorrectness}; \textbf{2.} the quality of being right \& suitable, so that something is done as it should be done, \textsc{synonym}: \textbf{appropriateness}; \textbf{3.} care in the way that you speak or behave so that you follow the accepted standards or rules, \textsc{opposite}: \textbf{incorrectness}.}\,\footnote{\textbf{political correctness} [n] [uncountable] (\textit{usually disapproving}) the principle of avoiding language \& behavior that may offend particular groups of people. This term is usually used by people who do not agree with this principle, or think it has been used in ways that are not reasonable.} gone amok\footnote{\textbf{amok} [adv] \textbf{run amok} [idiom] to suddenly become very angry or excited \& start behaving violently, especially in a public place.}. I saw an ad for a boys' camp designed to provide ``individual\footnote{\textbf{individual} [n] \textbf{1.} a person considered separately rather than as part of a group; \textbf{2.} a single member of a group or class; \textbf{3.} a person who is very different from others \& has lots of new \& interesting ideas; [a] \textbf{1.} [only before noun] considered separately rather than as part of a group; \textbf{2.} [only before noun] of or for a particular person; \textbf{3.} [only before noun] designed for use by 1 person; \textbf{4.} characteristic of a particular person or thing; \textbf{5.} (\textit{usually approving}) having an unusual character, \textsc{synonym}: \textbf{distinctive, original}.} attention\footnote{\textbf{attention} [n] \textbf{1.} [uncountable] the act of listening to, looking at or thinking about something\texttt{/}somebody carefully; interest that people show in somebody\texttt{/}something; \textbf{2.} [uncountable] special care, action or treatment; \textbf{3.} [countable, usually plural] things done to try to please somebody or to show an interest in them.} for the minimally\footnote{\textbf{minimally} [adv] in a way that involves the least possible amount of activity or content.} exceptional\footnote{\textbf{exceptional} [a] \textbf{1.} unusually good, \textsc{synonym}: \textbf{outstanding}; \textbf{2.} very unusual, \textsc{opposite}: \textbf{unexceptional}.}.''

Clutter is the official\footnote{\textbf{official} [n] (often in compounds) a person who is in a position of authority in a large organization; [a] \textbf{1.} [only before noun] connected with the job of somebody who is in a position of authority, \textsc{opposite}: \textbf{unofficial}; \textbf{2.} [usually before noun] agreed to, said or done by somebody who is in a position of authority, \textsc{opposite}: \textbf{unofficial}; \textbf{3.} [only before noun] that is told to be public but may not be true.} language used by corporations\footnote{\textbf{corporation} [n] (abbr., \textbf{Corp.}) a large business company, or a group of companies that is recognized by law as a single unit.} to hide\footnote{\textbf{hide} [v] \textbf{1.} [transitive] \textbf{hide something} to keep something secret, \textsc{synonym}: \textbf{conceal}; \textbf{2.} [transitive] to put or keep somebody\texttt{/}something in a place where they\texttt{/}it cannot be seen or found, \textsc{synonym}: \textbf{conceal}; \textbf{3.} [intransitive, transitive] to go somewhere where you hope you will not be seen or found; \textbf{4.} [transitive] \textbf{$+$ adv.\texttt{/}prep.} to cover something so that it cannot be seen, \textsc{synonym}: \textbf{conceal}; \textbf{hide behind something} [phrasal verb] to use something to stop people from finding out the truth or information about you.} their mistakes\footnote{\textbf{mistake} [n] \textbf{1.} an action or opinion that is not correct, or that produces a result that is not wanted. \textbf{Make no mistake} is sometimes used in less formal writing to emphasize that what you are saying is correct. This use is best avoided in more formal writing. \textbf{2.} a word, figure, fact, etc. that is not correct, \textsc{synonym}: \textbf{error}; \textbf{by mistake} [idiom] by accident; without intending to. \textbf{By mistake} is very common in general English, \& is also used in academic English. However, in academic English, the more formal \textbf{accidentally} \& \textbf{unintentionally} are more common; [v] \textbf{mistake something\texttt{/}somebody for something\texttt{/}somebody} [often passive] to think wrongly that something\texttt{/}somebody is something\texttt{/}somebody else.}. When the Digital\footnote{\textbf{digital} [a] \textbf{1.} using a system of receiving \& sending information as a series of the numbers 1 \& 0, showing that an electronic signal is there or is not there; connected with computer technology; \textbf{2.} (of clocks, watches, etc.) displaying only the appropriate numbers, rather than pointing to numbers from a larger set of numbers; other information displayed in this way; \textbf{3.} connected with a finger or the fingers of the hand.} Equipment\footnote{\textbf{equipment} [n] [uncountable] the necessary items for a particular purpose or activity.} Corporation eliminated 3,000 jobs its statement didn't mention\footnote{\textbf{mention} [v] to write or speak about something\texttt{/}somebody, especially without giving much information; \textbf{not to mention} [idiom] used to introduce extra information \& emphasize what you are saying; [n] [uncountable, countable, usually singular] an act of referring to somebody\texttt{/}something in speech or writing.} layoffs\footnote{\textbf{lay-off} [n] \textbf{1.} an act of making people unemployed because there is no more work left for them to do; \textbf{2.} a period of time when somebody is not working or not doing something that they normally do regularly.}; those were ``involuntary\footnote{\textbf{involuntary} [a] \textbf{1.} happening without the person concerned wanting it to; \textbf{2.} an involuntary movement, etc. is made suddenly, without you intending it or being able to control it, \textsc{opposite}: \textbf{voluntary}.} methodologies\footnote{\textbf{methodology} [n] (plural \textbf{methodologies}) [countable, uncountable] a set of methods \& principles used to perform a particular activity.}.'' When an Air Force\footnote{\textbf{air force} [n] [countable $+$ singular or plural verb] the part of a country's armed forces that fights using aircraft.} missile\footnote{\textbf{missile} [n] \textbf{1.} a weapon that is sent through the air \& that explodes when it hits the thing that it is aimed at; \textbf{2.} an object that is thrown at somebody to hurt them, \textsc{synonym}: \textbf{projectile}.} crashed\footnote{\textbf{crash} [n] \textbf{1.} an accident in which a vehicle hits something, e.g. another vehicle, causing damage \& often injury or death; \textbf{2.} a sudden serious fall in the price or value of something; the occasion when a business suddenly fails, \textsc{synonym}: \textbf{collapse}; \textbf{3.} a sudden failure of a machine, especially of a computer system; [v] \textbf{1.} [intransitive, transitive] to hit something hard while moving, causing noise \&\texttt{/}or damage; to make something hit somebody\texttt{/}something in this way; \textbf{2.} [intransitive] (of prices, shares, a business, etc.) to lose value or fail suddenly \& quickly; \textbf{3.} [intransitive] (of a computer) to stop working suddenly.}, it ``impacted\footnote{\textbf{impact} [n] [countable, usually singular, uncountable] \textbf{1.} the powerful effect that something has on somebody\texttt{/}something; \textbf{2.} the act of 1 object hitting another; the force with which this happens; [v] [transitive, intransitive] to have an effect on something.} with the ground prematurely\footnote{\textbf{premature} [a] \textbf{1.} happening before the normal or expected time; \textbf{2.} (of a birth or a baby) happening or being born before the normal length of pregnancy has been completed; \textbf{3.} happening or made too soon.}.'' When General\footnote{\textbf{general} [a] \textbf{1.} affecting or including all or most people, places or things; \textbf{2.} [usually before noun] normal; usual; true in most cases; \textbf{3.} including the most important aspects of something; not exact or detailed, \textsc{synonym}: \textbf{broad}, \textsc{opposite}: \textbf{specific}; \textbf{4.} \textbf{the general direction\texttt{/}area} used to describe the approximate, but not exact, direction or area mentioned; \textbf{5.} not limited to a particular subject, use or activity; \textbf{6.} not limited to 1 part or aspect of a person or thing; \textbf{7.} [only before noun] highest in rank; \textbf{as a general rule} [idiom] usually; \textbf{of general interest} [idiom] of interest to most people; [n] (abbr., \textbf{Gen.}) an officer of very high rank in the army or the US air force; the commander of an army.} Motors\footnote{\textbf{motor} [n] \textbf{1.} a device that uses a source of power, such as electricity or fuel, to make a vehicle or part of a machine move; \textbf{2.} (\textit{biology}) a part of a living creature, such as bacteria, that is involved in moving it around; [a] [only before noun] \textbf{1.} having an engine; using the power of an engine; \textbf{2.} (\textit{especially British English}) connected with vehicles that have engines; \textbf{3.} (\textit{specialist}) connected with movement of the body that is produced by muscles; connected with the nerves that control movement.} had a plant\footnote{\textbf{plant} [n] \textbf{1.} [countable] a living thing that grows in the earth \& usually has a stem, leaves \& roots. Plants typically take in water \& other substances through their roots, \& make food in their leaves by photosynthesis.; \textbf{2.} [countable] a factory or place where power is produced or an industrial process takes place; \textbf{3.} [uncountable] the large machinery that is used in industrial processes; [v] \textbf{1.} \textbf{plant something} to put plants or seeds in the ground to grow; \textbf{2.} to cover or supply an area of land with plants; \textbf{3.} \textbf{plant something (in something)} to make an idea become established in a person's mind or people's minds; \textbf{4.} \textbf{plant something ($+$ adv.\texttt{/}prep.)} to hide something such as a bomb in a place where it will not be found.} shutdown\footnote{\textbf{shutdown} [n] the act of closing a factory or business or stopping a large machine from working, either temporarily or permanently.}, that was a ``volume-related production-schedule\footnote{\textbf{production} [n] \textbf{1.} [uncountable] the process of growing or making food, goods or materials, especially in large quantities; \textbf{2.} [uncountable] the total quantity of goods or materials that is produced; \textbf{3.} [uncountable] the act or process of making something naturally; \textbf{4.} [countable, uncountable] \textbf{production (of something)} a film, play or broadcast that is prepared for the public; the act of preparing a film, play, etc.; \textbf{5.} [uncountable] \textbf{production of something} the act of making or supplying something for somebody to use, use or consider; \textbf{6.} [uncountable] (\textit{linguistics}) the ability to produce language or information that you have learned in speech or writing; the process of doing this.} adjustment\footnote{\textbf{adjustment} [n] [countable, uncountable] \textbf{1.} a small change made to something in order to correct or improve it; the act of making a small change; \textbf{2.} a change in the way a person behaves or thinks.}.'' Companies\footnote{\textbf{company} [n] (plural \textbf{companies}) \textbf{1.} [countable] (abbr., \textbf{Co.}) (often in names) a business organization that makes money by producing or selling goods or services; \textbf{2.} (often in names) [countable $+$ singular or plural verb] a group of people, especially actors or dancers, who work or perform together; \textbf{3.} [uncountable] the fact of being with somebody else \& not alone, especially in a way that provides friendship or enjoyment.} that go belly-up\footnote{\textbf{belly} [n] (plural \textbf{bellies}) \textbf{1.} the part of the body below the chest, \textsc{synonym}: \textbf{stomach, gut}; \textbf{2.} (\textit{literary}) the round or curved part of an object; \textbf{3.} \textbf{-bellied} (in adjectives) having the type of belly mentioned; \textbf{fire in the\texttt{/}your belly} [idiom] a very strong desire to achieve something; \textbf{go belly up} [idiom] (\textit{informal}) to fail completely; [v] \textbf{belly (out)} (especially of sails) to fill with air \& become rounder.} have ``a negative\footnote{\textbf{negative} [a] \textbf{1.} bad or harmful, \textsc{opposite}: \textbf{positive}; \textbf{2.} considering only the bad side of something\texttt{/}somebody; lacking enthusiasm or hope, \textsc{opposite}: \textbf{positive}; \textbf{3.} expressing the answer `no', \textsc{opposite}: \textbf{affirmative}; \textbf{4.} containing a word such as `no', `not', `never', etc.; \textbf{5.} (abbr., \textbf{neg.}) not showing any evidence of a particular substance or medical condition, \textsc{opposite}: \textbf{positive}; \textbf{6.} $<0$, \textsc{opposite}: \textbf{positive}; \textbf{7.} (of a relationship between 2 amounts or events) related in such a way that, when one increases, the other decreases, \textsc{opposite}: \textbf{positive}; \textbf{8.} containing or producing the type of electricity that is carried by an electron, \textsc{opposite}: \textbf{positive}; [n] \textbf{1.} a word or statement that means `no' or `not'; \textbf{2.} the result of a test or an experiment that shows that a substance or condition is not present, \textsc{opposite}: \textbf{positive}; \textbf{3.} a disadvantage or problem, \textsc{opposite}: \textbf{positive}.} cash-flow\footnote{\textbf{cash flow} [n] [countable, uncountable] the movement of money into \& out of a business as goods are bought \& sold.} position\footnote{\textbf{position} [n] \textbf{1.} [countable] the place where somebody\texttt{/}something is located; \textbf{2.} [uncountable] the place where somebody\texttt{/}something is meant to be; the correct place; \textbf{3.} [countable, uncountable] \textbf{(in\texttt{/}into a) position} the way in which somebody is sitting or standing; the way in which something is arranged; \textbf{4.} [countable, usually singular] the situation that somebody is in, especially when it affects what they can \& cannot do, \textsc{synonym}: \textbf{situation}; \textbf{5.} [countable] an opinion on or attitude towards a particular subject; \textbf{6.} [countable, uncountable] s\textbf{position (of somebody\texttt{/}something)} the level of importance of a person, organization or thing when compared with others, \textsc{synonym}: \textbf{status}; \textbf{7.} [countable] a job, \textsc{synonym}: \textbf{post}; \textbf{8.} [countable] a place in a race, competition or test, when compared with others; [v] \textbf{1.} to put somebody\texttt{/}something in a particular position; \textbf{2.} [often passive] to put somebody\texttt{/}something in a particular situation, especially when it affects what they can \& cannot do; \textbf{3.} to advertise a product, service or business as satisfying the needs of a particular group of customers.}.''

Clutter is the language of the Pentagon\footnote{\textbf{pentagon} [n] \textbf{1.} [countable] (\textit{geometry}) a flat shape with 5 straight sides \& 5 angles; \textbf{2.} \textbf{the Pentagon} [singular] the building near Washington DC that is the headquarters of the US Department of Defense \& the military leaders.} calling an invasion\footnote{\textbf{invasion} [n] [countable, uncountable] \textbf{1.} the act of an army entering another country by force in order to take control of it; \textbf{2.} the fact of a large number of people or things entering a place, situation or area of activity, especially people or things that are considered unpleasant; \textbf{3.} \textbf{invasion of something} an act of process that disturbs somebody\texttt{/}something or has an unpleasant effect on it; \textbf{4.} the act of spreading into a living thing or body part; \textbf{5.} the fact of a large number of animals or plants moving or spreading into a place.} a ``reinforced\footnote{\textbf{reinforce} [v] \textbf{1.} \textbf{reinforce something} to make a feeling, idea, habit or tendency stronger; \textbf{2.} \textbf{reinforce something} to make a structure or material stronger, especially by adding another material to it; \textbf{3.} \textbf{reinforce something} to send more people or equipment in order to make an army, etc. stronger. } protective\footnote{\textbf{protective} [a] \textbf{1.} [only before noun] providing or intend to provide protection; \textbf{2.} \textbf{protective (of somebody\texttt{/}something)} having or showing a wish to protect somebody\texttt{/}something; \textbf{3.} intended to give an advantage to your own country's industry.} reaction\footnote{\textbf{reaction} [n] \textbf{1.} [countable, uncountable] what you do, say or think as a result of something that has happened; \textbf{2.} [countable] (\textit{chemistry}) a chemical change produced by 2 or more substances acting on each other; \textbf{3.} [countable, uncountable] (\textit{medical}) a response by the body, usually a bad one, to something such as a drug or a chemical substance; \textbf{4.} [uncountable, countable] (\textit{physics}) a force shown by showing in response to another force, which is of equal strength \& acts in the opposite direction; \textbf{5.} [countable, usually singular] \textbf{reaction (against something)} a change in people's attitudes or behavior caused by strong disapproval of other very different attitudes; \textbf{6.} [uncountable] opposition to social or political progress or change; \textbf{7.} (\textbf{reactions}) [plural] the ability to move quickly in response to something, especially if in danger.} strike\footnote{\textbf{strike} [v] \textbf{1.} [transitive] \textbf{strike somebody\texttt{/}something} to hit somebody\texttt{/}something hard or with force; \textbf{2.} [transitive] \textbf{strike somebody\texttt{/}something} to hit somebody\texttt{/}something with your hand or a weapon; \textbf{3.} [intransitive, transitive] to attack somebody\texttt{/}something, especially suddenly; \textbf{4.} [intransitive, transitive] to happen suddenly \& have a harmful or damaging effect on somebody\texttt{/}something; \textbf{5.} [intransitive, transitive] (of lighting) to hit \& hurt or damage somebody\texttt{/}something on the ground; \textbf{6.} [transitive] \textbf{strike something} (of light) to fall on a surface; \textbf{7.} [transitive, often passive] to cause somebody to notice or be interested; to make a particular impression on somebody; \textbf{8.} [intransitive] to refuse to work, because of a disagreement over pay or conditions; [n] \textbf{1.} [countable, uncountable] a period of time when an organized group of employees of a company stops working because of a disagreement over pay or conditions; \textbf{2.} [countable] a military attack, especially by aircraft dropping bombs; \textbf{3.} [countable] an act of hitting something\texttt{/}somebody.}'' \& justifying\footnote{\textbf{justify} [v] to give an explanation or excuse for something or for doing something; to show that somebody\texttt{/}something is right or reasonable; \textbf{the end justifies the means} [idiom] used to say that bad or unfair methods of doing something are acceptable if the result of that action is good or positive.} its vast budgets\footnote{\textbf{budget} [n] \textbf{1.} [countable, uncountable] the money that is available to a person or an organization \& a plan of how it will be spent over a period of time; \textbf{2.} (\textit{British English also} \textbf{Budget}) [countable, usually singular] an official statement by the government of a country's income from taxes, etc. \& how it will be spent; [v] [intransitive, transitive] to allow or provide a particular amount of money for a particular purpose.} on the need for ``counterforce\footnote{\textbf{counterforce} [n] (plural \textbf{counterforces} or \textbf{counter-forces}) a force that opposes another force; [a] being or relating to military activity that is focused on reducing the fighting capability of the opponent's forces (as by destroying military bases or weapons) while attempting to minimize civilian casualities.} deterrence\footnote{\textbf{deterrent} [n] \textbf{1.} a thing that makes somebody less likely to do something; \textbf{2.} \textbf{deterrent (against somebody\texttt{/}something)} a weapon or weapons system that makes others less likely to attack you.}.'' As \textsc{George Orwell} pointed\footnote{\textbf{point} [n] \textbf{1.} [countable] a thing that somebody says or writes giving their opinion or stating a fact; \textbf{2.} [countable] (usually \textbf{the point}) the main or most important idea in something that is said or done; \textbf{3.} [uncountable, singular] the purpose or aim of something; \textbf{4.} [countable] a particular detail or fact; \textbf{5.} [countable] \textbf{point (that $\ldots$)} a particular quality or feature that somebody\texttt{/}something has; \textbf{6.} [countable] a particular time or stage of development; \textbf{7.} [countable] a particular place or area; \textbf{8.} [countable] a mark or unit on a scale of measurement; \textbf{9.} [countable] a unit of credit towards a score, award or benefit; \textbf{10.} [countable] the sharp thin end of something; \textbf{11.} [countable] a small dot used in writing, especially the dot that separates a whole number from the part that comes after it; \textbf{12.} [countable] \textbf{point (of something)} a very small dot of light or color; [v] \textbf{1.} [intransitive, transitive] to lead to or suggest a particular development or logical argument; \textbf{2.} [intransitive] \textbf{$+$ adv.\texttt{/}prep.} to face in or be directed towards a particular direction; \textbf{3.} [transitive] \textbf{point something (at somebody\texttt{/}something)} to aim something at somebody\texttt{/}something; \textbf{4.} [intransitive] \textbf{point (at\texttt{/}to\texttt{/}towards somebody\texttt{/}something)} to stretch out your finger towards somebody\texttt{/}something in order to show somebody where that somebody\texttt{/}something is; \textbf{point out (to somebody) $|$ point something out (to somebody)} [phrasal verb] to mention something in order to provide information or an opinion about it; to mention something in order to make somebody notice it.} out in ``Politics \& the English Language,'' an essay\footnote{\textbf{essay} [n] \textbf{1.} \textbf{essay (on something)} a short piece of writing by a student as part of a course of study; \textbf{2.} \textbf{essay (on something)} a short piece of writing on a particular subject, written in order to be published; \textbf{3.} \textbf{essay (in something)} an attempt to do something.} written in 1946 but often cited during the wards in Cambodia, Vietnam \& Iraq, ``political speech \& writing are largely\footnote{\textbf{largely} [adv] to a great extent; mostly or mainly.} the defense\footnote{\textbf{defence} [n] (\textit{US} \textbf{defense}) \textbf{1.} [countable, uncountable] support for somebody\texttt{/}something that has been criticized, \textsc{opposite}: \textbf{attack}; \textbf{2.} [uncountable, countable] the action of protecting somebody\texttt{/}something from attack, \textsc{opposite}: \textbf{attack}; \textbf{3.} [countable, uncountable] something that provides protection against attack from enemies, the weather, illness, etc.; \textbf{4.} [uncountable] military measures or resources for protecting a country from attack; \textbf{5.} [countable] a set of facts or arguments presented in court to support a person who has been accused of committing a crime, or who is being sued; \textbf{6.} (\textbf{the defence}) [singular $+$ singular or plural verb] the lawyer or lawyers whose job is to represent in court a person who has been accused of committing a crime, or who is being sued.} of the indefensible\footnote{\textbf{indefensible} [a] \textbf{1.} (of an action or opinion) impossible to excuse or defend because it is morally wrong, \textsc{opposite}: \textbf{defensible}; \textbf{2.} (of a place) impossible to defend from an attack, \textsc{opposite}: \textbf{defensible}.} $\ldots$. Thus political language has to consist\footnote{\textbf{consist} [v] (not used in the progressive tenses) to have something as the main or only part or feature; \textbf{consist of somebody\texttt{/}something} to be formed from the people or things mentioned. Although \textbf{consist} is not used in the progressive tenses, \textbf{consisting of} is common after a noun.} largely of euphemism\footnote{\textbf{euphemism} [n] \textbf{euphemism (for something)} an indirect word or phrase that people often use to refer to something embarrassing or unpleasant, sometimes to make it seem more acceptable than it really is.}, question-begging \& sheer\footnote{\textbf{sheer} [a] \textbf{1.} [only before noun] used to emphasize the size, degree or amount of something; nothing but; \textbf{2.} very steep.} cloudy\footnote{\textbf{cloudy} [a] (\textbf{cloudier, cloudiest}) \textbf{1.} (of the sky or the weather) covered with clouds; with a lot of clouds, \textsc{opposite}: \textbf{clear}; \textbf{2.} (of liquids) not clear or transparent.} vagueness\footnote{\textbf{vagueness} [n] [uncountable] \textbf{1.} the fact of not being clear in person's mind; \textbf{2.} the fact of not having or giving enough information or details about something.}\,\footnote{\textbf{vague} [n] (\textbf{vaguer, vaguest}) not having or giving enough information or details about something.}.'' Orwell's warning that clutter is not just a nuisance\footnote{\textbf{nuisance} [n] \textbf{1.} [countable, usually singular, uncountable] a thing, person or situation that causes trouble or problems; \textbf{2.} [countable, uncountable] (\textit{law}) behavior by somebody that harms or offends other people \& that a court can order the person to stop.} but a deadly\footnote{\textbf{deadly} [a] (\textbf{deadlier, deadliest}) (\textbf{More deadly} \& \textbf{deadliest} are the usual forms. You can also use \textbf{most deadly}) causing or likely to cause death, \textsc{synonym}: \textbf{lethal}.} tool\footnote{\textbf{tool} [n] \textbf{1.} a thing that helps somebody to do a job or to achieve something; \textbf{2.} a piece of equipment held in the hand, that is used for making things or repairing things.} has come true in the recent decades\footnote{\textbf{decade} [n] a period of 10 years, especially 1 beginning with a year ending in 0.} of American military\footnote{\textbf{military} [a] [usually before noun] connected with soldiers or the armed forces; [n] (\textbf{the military}) [singular $+$ singular or plural verb] soldiers; the armed forces.} adventurism\footnote{\textbf{adventurism} [n] [uncountable] (\textit{disapproving}) the fact of being willing to take risks in business or politics in order to gain something for yourself.}. It was during George W. Bush's presidency\footnote{\textbf{presidency} [n] (plural \textbf{presidencies}) [usually singular] the job of being president of a country or an organization; the period of time somebody holds this job. The \textbf{presidency} of an organization is sometimes held by a country rather than a single person.} that ``civilian\footnote{\textbf{civilian} [n] a person who is not a member of the armed forces or the police; [a] [usually before noun] connected with people who are not members of the armed forces or the police.} casualties\footnote{\textbf{casualty} [n] (plural \textbf{casualties}) \textbf{1.} a person who is killed or injured in war or in an accident; \textbf{2.} \textbf{casualty (of something)} a person who suffers or a thing that is destroyed when something else takes place.}'  in Iraq become ``collateral\footnote{\textbf{collateral} [a] connected with something else, but in addition to it \& less important.} damage\footnote{\textbf{damage} [n] \textbf{1.} [uncountable] \textbf{damage (to something)} physical harm caused to something; \textbf{2.} \textbf{damage (to somebody\texttt{/}something)} harmful effects on somebody\texttt{/}something; \textbf{3.} (\textbf{damages}) [plural] an amount of money that a court decides should be paid to somebody by somebody\texttt{/}something that has caused them harm or injury; [v] \textbf{1.} \textbf{damage something} to cause physical harm to something; \textbf{2.} \textbf{damage something} to have a harmful effect on something.}.''

Verbal\footnote{\textbf{verbal} [a] \textbf{1.} involving words, \textsc{non-opposite}: \textbf{non-verbal}; \textbf{2.} spoken, not written.} camouflage\footnote{\textbf{camouflage} [n] \textbf{1.} [uncountable] a way of hiding soldiers \& military equipment, using paint, leaves or nets, so that they look like part o what is around or near them; \textbf{2.} [uncountable, singular] the way in which an animal's color or shape matches what is around or near it \& makes it difficult to see; \textbf{3.} [uncountable, singular] behavior that is deliberately meant to hide the truth; [v] \textbf{camouflage something\texttt{/}yourself (with something)} to hide somebody\texttt{/}something\texttt{/}yourself by making them\texttt{/}it\texttt{/}yourself look like the things around, or like something else.} reached\footnote{\textbf{reach} [v] \textbf{1.} [transitive] \textbf{reach something} to achieve a particular aim, \textsc{synonym}: \textbf{arrive at something}; \textbf{2.} [transitive] \textbf{reach something} to increase or decrease to a particular level, over a period of time; \textbf{3.} [transitive] \textbf{reach something} to arrive at a particular point or stage of something after a period of time; \textbf{4.} [transitive] \textbf{reach something} to arrive at the place that somebody\texttt{/}something has been traveling to; \textbf{5.} [intransitive, transitive] to be big or long enough to go to a particular point or place; \textbf{6.} [transitive] \textbf{reach somebody} to be seen or heard by somebody; \textbf{7.} [transitive] \textbf{reach somebody} to communicate with somebody, especially by telephone; \textbf{8.} [transitive] \textbf{reach somebody} to come to somebody's attention; \textbf{reach back to something} [phrasal verb] to go back from the present to a particular earlier time; \textbf{reach out to somebody} [phrasal verb] to show somebody that you are interested in them \&\texttt{/}or want to have contact with them or help them; [n] \textbf{1.} [uncountable, singular] the limit to which somebody\texttt{/}something has the power or influence to do something; \textbf{2.} [uncountable] the distance over which you can stretch your arms to touch something; the distance over which a particular object can be used to touch something else; \textbf{3.} [countable, usually plural] a straight section of water between 2 bends on a river; \textbf{4.} (\textbf{reaches}) [plural] \textbf{the outer, further, etc. reach of something} the parts of an area or a place that are a long way from the center.} new heights during General Alexander Haig's tenure\footnote{\textbf{tenure} [n] \textbf{1.} [countable] \textbf{tenure (as something)} the period of time when somebody holds a job; \textbf{2.} [uncountable] the right to stay permanently in your job, especially as a teacher at a university or in a public position; \textbf{3.} [uncountable] \textbf{tenure (over something)} the legal right to live in a house or use a piece of land.} as President Reagan's secretary\footnote{\textbf{secretary} [n] (plural \textbf{secretaries}) \textbf{1.} a person who works in an office, working for another person, dealing with letters \& telephone calls, keeping records, arranging meetings, etc.; \textbf{2.} (\textbf{Secretary}) \textbf{secretary (of something)} the head of a government department. In the UK the title \textbf{Secretary} or \textbf{Secretary of State} is used for particular, important government departments. In the US, \textbf{secretary} is the usual term for the head of a government department, chosen by the president; the \textbf{Secretary of State} is specifically the head of the government department that deals with foreign affairs.; \textbf{3.} an assistant of a UK government minister or ambassador.} of state\footnote{\textbf{state} [n] \textbf{1.} [countable] the mental, emotional or physical condition that a person or thing is in. In physics \& chemistry, the \textbf{state} of a substance is whether it is a solid, liquid or gas.; \textbf{2.} (\textbf{State}) [countable] a country considered as an organized political community controlled by 1 government; \textbf{3.} (\textbf{State}) [countable] (abbr., \textbf{St.}) \textbf{state (of something)} an organized political community forming part of a country; \textbf{4.} (\textbf{the State}) [singular, uncountable] the government of a country; \textbf{5.} [uncountable] the formal ceremonies connected with high levels of government or with kinds \& queens; \textbf{a state of affairs} [idiom] a situation; \textbf{state of the art} [idiom] the most modern or advanced techniques or methods in a particular field; [a] (also \textbf{State}) [only before noun] \textbf{1.} provided by, controlled by or belonging to the government of a country; \textbf{2.} connected with the leader of a country attending an official ceremony; \textbf{3.} connected with a particular state of a country, especially in the US; [v] \textbf{1.} to formally write or say something, especially in a careful \& clear way; \textbf{2.} [usually passive] to fix or announce the details of something, especially on a written document; \textbf{put\texttt{/}stated differently} [idiom] in other words; used to introduce an explanation of something. Note that this expression is usually used at the start of a sentence.}. Before Haig nobody had thought of saying ``at this juncture\footnote{\textbf{juncture} [n] (\textit{formal}) a particular point or state in an activity or a series of events.} of maturization\footnote{See, e.g., \href{https://ell.stackexchange.com/questions/94449/alexander-haig-s-cluttered-langugae}{StackExchange\texttt{/}English Language Learners\texttt{/}Alexander Haig's cluttered language}.}\,\footnote{\textbf{maturation} [n] [uncountable] (\textit{formal}) \textbf{1.} the process of becoming or being made mature ($=$ ready to eat or drink after being left for a period of time); \textbf{2.} the process of becoming adult.}'' to mean ``now.'' He told the American people that terrorism\footnote{\textbf{terrorism} [n] [uncountable] the use of violent action in order to achieve political aims or to force a government to act.} could be fought with ``meaningful\footnote{\textbf{meaningful} [a] \textbf{1.} serious, useful or important; \textbf{2.} clearly showing the information that is required.} sanctionary\footnote{See \href{https://en.wiktionary.org/wiki/sanctionary}{Wikipedia\texttt{/}sanctionary}: \textbf{sanctionary} [a] of, pertaining to, or giving sanction.} teeth'' \& that intermediate\footnote{\textbf{intermediate} [a] \textbf{1.} [usually before noun] coming between 2 times, states, places or things. In business, \textbf{intermediate goods} are materials, such as cotton, or partly finished goods, such as car engines, which are used to produce \textbf{final goods} which are sold to the public.; \textbf{2.} having more than a basic knowledge of something but not yet advanced; suitable for somebody who is at this level.} nuclear\footnote{\textbf{nuclear} [a] [usually before noun] \textbf{1.} of the nucleus ($=$ central part) of an atom; \textbf{2.} using, producing or resulting from energy that is produced by splitting the nucleus of atoms; \textbf{3.} connected with weapons that use energy produced by splitting atoms; \textbf{4.} (\textit{biology}) of the nucleus ($=$ central part) of a cell.} missiles\footnote{\textbf{missile} [n] \textbf{1.} a weapon that is sent through the air \& that explodes when it hits the thing that it is aimed at; \textbf{2.} an object that is thrown at somebody to hurt them, \textsc{synonym}: \textbf{projectile}.} were ``at the vortex\footnote{\textbf{vortex} [n] (plural \textbf{vortexes, vortices}) \textbf{1.} (\textit{specialist}) a mass of air, water, etc. that turns round \& round very fast \& pull things into its center, \textsc{synonym}: \textbf{whirlpool, whirlwind}; \textbf{2.} (\textit{literary}) a very powerful feeling or situation that you cannot avoid or escape from.} of cruciality\footnote{\textbf{cruciality} [n] the quality or state of being crucial.}.'' As for any worries\footnote{\textbf{worry} [v] \textbf{1.} [intransitive] to keep thinking about unpleasant things that might happen or about problems that you have; \textbf{2.} [transitive] to make somebody\texttt{/}yourself anxious about somebody\texttt{/}something; [n] (plural \textbf{worries}) \textbf{1.} [countable] something that worries you; \textbf{2.} [uncountable] the state of worrying about something, \textsc{synonym}: \textbf{anxiety}.} that the public\footnote{\textbf{public} [a] \textbf{1.} [only before noun] connected with ordinary people in society in general; \textbf{2.} [only before noun] provided or available for the use of people in general, \textsc{opposite}: \textbf{private}; \textbf{3.} [only before noun] paid for or connected with the government as opposed to a private company or individual, \textsc{opposite}: \textbf{private}; \textbf{4.} involving activities \& people not connected to somebody's home, family or private life, \textsc{opposite}: \textbf{private}; \textbf{5.} able to intended to be seen, heard or known by people in general, \textsc{opposite}: \textbf{private}; [n] \textbf{1.} (\textbf{the public}) [singular $+$ singular or plural verb] ordinary people in society in general; \textbf{2.} [countable $+$ singular or plural verb] a group of people who share a particular interest or who are involved in the same activity; \textbf{in public} [idiom] in a place when other people, especially people you do not know, are present.} might harbor\footnote{\textbf{harbor} [n] (\textit{US English} \textbf{harbor}) [countable, uncountable] an area of water on the coast, protected from the open sea by strong walls, where ships can shelter; [v] (\textit{US English} \textbf{harbor}) \textbf{1.} \textbf{harbor somebody} to hide \& protect somebody who is hiding from the police; \textbf{2.} \textbf{harbor something} to keep feelings or thoughts, especially negative ones, in your mind for a long time; \textbf{3.} \textbf{harbor something} to contain something \& allow it to develop.}, his message\footnote{\textbf{messenger}  [n] \textbf{1.} a piece of information sent in electronic form, e.g. by email or mobile phone; \textbf{2.} \textbf{message (of something)} a written or spoken piece of information that you send to somebody or leave for somebody when you cannot speak to them yourself; \textbf{3.} [usually singular] an important idea that something such as a book, speech or company is trying to communicate; \textbf{4.} [singular] \textbf{message (from something)} something important that should be learned from an event; \textbf{5.} a piece of information or an instruction that is sent from 1 part of the body to another.} was ``leave it to AI\footnote{\textbf{artificial intelligence} [n] [uncountable] (abbr., \textbf{AI} \textit{computing}) an area of study concerned with making computes copy intelligent human behavior.},'' thought what he actually\footnote{\textbf{actually} [adv] \textbf{1.} used to emphasize a fact or the truth about a situation; \textbf{2.} used to show a contrast between what is true \& what somebody believes, \& to show surprise about this contrast.} said was: ``We must push this to a lower decibel\footnote{\textbf{decibel} [n] a unit for measuring how loud a sound is.} of public fixation\footnote{\textbf{fixation} [n] \textbf{1.} [countable] \textbf{fixation (with\texttt{/}on something\texttt{/}somebody)} an extreme interest or belief in something\texttt{/}somebody, that does not match the truth or is not normal or natural; \textbf{2.} [uncountable] (\textit{biology}) the process by which a microorganism or plant takes in \& transforms nitrogen or carbon dioxide.}. I don't think there's much of a learning\footnote{\textbf{learning} [n] [uncountable] \textbf{1.} the process of learning something; \textbf{2.} knowledge that you get from reading, study \& experience.} curve\footnote{\textbf{curve} [n] \textbf{1.} (\textit{statistics}) a line on a graph (either straight or curved) showing how 1 quantity varies w.r.t. another; \textbf{2.} \textbf{curve (of something)} a line or surface that bends gradually; a smooth bend; [v] [intransitive, transitive] to form the shape of a curve; to make something form the shape of a curve.}\,\footnote{\textbf{learning curve} [n] the rate at which you learn a new subject or a new skill; the process of learning from the mistakes you make.} to be achieved in this area of content\footnote{\textbf{content} [n] \textbf{1.} (\textbf{contents}) [plural] \textbf{content (of something)} the things that are contained in something; \textbf{2.} (\textbf{contents}) [plural] the different sections that are contained in a book, magazine, journal or website; a list of these sections; \textbf{3.} [singular] the subject matter of a book, speech, programme, etc. \textbf{4.} [singular] (following a noun or an adjective) the amount of a substance that is contained in something else; \textbf{5.} [uncountable] the information or other material contained on a website, CD-ROM, etc.; [a] [not before noun] satisfied \& happy with what you have; willing to do or accept something; [v] \textbf{content yourself with something} to accept \& be satisfied with something \& not try to have or do something better.}.''

I could go on quoting\footnote{\textbf{quote} [v] \textbf{1.} [transitive, intransitive] to use words from another speaker or writer in your writing or speech, showing clearly that the words are from this source; \textbf{2.} [transitive, often passive] to mention somebody\texttt{/}something in a particular way; to give something as an example; \textbf{3.} [transitive, often passive] to give a price for something. If shares or companies \textbf{are quoted} on a stock exchange, the prices of the shares are listed there; [n] (\textit{rather informal}) a quotation.} examples\footnote{\textbf{example} [n] \textbf{1.} something such as a fact or a situation that shows, explains or supports what you say; \textbf{2.} a thing that is typical of or represents a particular group or set; \textbf{3.} a person or thing that is a good or bad model for others to copy; \textbf{for example} (abbr., \textbf{e.g.}), \textbf{by way of example} [idiom] used to emphasize something that explains or supports what you are saying; used to give an example of what you are saying, \textsc{synonym}: \textbf{for instance}.} from various\footnote{\textbf{various} [a] several different.} fields\footnote{\textbf{field} [n] \textbf{1.} [countable] a particular subject or activity that somebody works in or is interested in, \textsc{synonym}: \textbf{area}; \textbf{2.} [countable] an area of land in the country used for growing crops or keeping animals in, usually surrounded by a fence, etc.; \textbf{3.} [countable] (usually in compounds) a large area of land covered with the thing mentioned; an area from which the thing mentioned is obtained; \textbf{4.} [countable] (usually in compounds) an area of land used for the purpose mentioned; \textbf{5.} [countable] (usually used as an adjective) the fact of people doing practical work or study, rather than working in a library or laboratory; \textbf{6.} [singular $+$ singular or plural verb] all the people or products competing in a particular area of business or activity; \textbf{7.} [countable] (usually in compounds) an area within which the force mentioned has an effect; \textbf{8.} [countable] the area within which objects can be seen from a particular point; \textbf{9.} [countable] part of a record that is a separate item of data.} -- every profession\footnote{\textbf{profession} [n] \textbf{1.} [countable] a type of job that needs special training or skill, especially one that needs a high level of education; \textbf{2.} (\textbf{the profession}) [singular $+$ singular or plural verb] all the people who work in a particular profession; \textbf{3.} (\textbf{the professions}) [plural] the traditional jobs that need a high level of education \& training, such as being a doctor or lawyer; \textbf{4.} [countable] \textbf{profession of something} a statement about what you believe, feel or think about something, that is sometimes made publicly, \textsc{synonym}: \textbf{declaration}.} has its growing\footnote{\textbf{growing} [a] [only before noun] increasing in size, amount or degree.} arsenal\footnote{\textbf{arsenal} [n] \textbf{1.} a collection of weapons such as guns \& explosives; \textbf{2.} a building where military weapons \& explosives are made or stored.} of jargon\footnote{\textbf{jargon} [n] [uncountable] (\textit{often disapproving}) words or expressions that are used by a particular profession or group of people, \& are difficult for others to understand.} to throw\footnote{\textbf{throw} [v] \textbf{1.} [transitive, intransitive] to send something from your hand through the air by moving your hand or arm quickly; \textbf{2.} [transitive] \textbf{throw something\texttt{/}somebody\texttt{/}yourself $+$ adv.\texttt{/}prep.} to move something\texttt{/}somebody suddenly \& with force; \textbf{3.} [transitive] \textbf{throw something $+$ adv.\texttt{/}prep.} to put something in a particular place quickly \& carelessly; \textbf{4.} [transitive, usually passive] to make somebody\texttt{/}something be in a particular bad state; \textbf{5.} [transitive] \textbf{throw something on\texttt{/}at somebody\texttt{/}something} to direct something at somebody\texttt{/}something.} dust\footnote{\textbf{dust} [n] [uncountable] \textbf{1.} a fine powder that consists of very small pieces of sand, earth, etc.; \textbf{2.} the fine powder of dirt that forms in buildings, on furniture, floors, etc.; \textbf{3.} a fine powder that consists of very small pieces of a particular substance.} in the eyes\footnote{\textbf{eye} [n] \textbf{1.} [countable] either of the 2 organs on the face that you see with. \textbf{Eye} is also used to describe the organ of vision in many invertebrates.; \textbf{2.} (\textbf{-eyed}) (in adjectives) having the type or number of eyes mentioned; \textbf{3.} [countable] used to refer to somebody's opinion or attitude towards something; \textbf{4.} [singular] \textbf{eye (for something)} the ability to see; \textbf{in the public eye} [idiom] well known to many people, e.g. through newspaper \& television.} of the populace\footnote{\textbf{populace} [n] (usually \textbf{the populace}) [singular $=$ singular or plural verb] (\textit{formal}) all the ordinary people of a particular country or area.}. But the list would be tedious\footnote{\textbf{tedious} [a] lasting or taking too long \& not interesting, \textsc{synonym}: \textbf{boring}.}. The point of raising it now is to serve notice that \fbox{clutter is the enemy}\footnote{\textbf{enemy} [n] (plural \textbf{enemies}) \textbf{1.} [countable $+$ singular or plural verb] a country that you are fighting a war against; the soldiers, etc. of this country; \textbf{2.} [countable] a person who hates somebody or who acts or speaks against somebody\texttt{/}something; \textbf{3.} [countable] anything that harms something or prevents it from being successful.}. Beware\footnote{\textbf{beware} [v] [intransitive, transitive] (used only in infinitives \& in orders) if you tell somebody to \textbf{beware}, you are warning them that somebody\texttt{/}something is dangerous \& that they should be careful.}, then, of \fbox{the long word that's no better than the short word}: ``assistance\footnote{\textbf{assistance} [n] [uncountable] help or support in the form of money, resources, information or practical action.}'' (help), ``numerous'' (many), ``facilitate'' (ease), ``individual'' (man or woman), ``remainder'' (rest), ``initial'' (1st), ``implement'' (do), ``sufficient'' (enough), ``attempt'' (try), ``referred to as'' (called) \& hundreds more.

\texttt{[pause here for reading chapter on memoir \& a book on PhD $\ldots$]}

'' -- \cite[pp. 19--]{Zinsser2016}

\section{Style}

\section{The Audience}

\section{Words}

\section{Usage}

\begin{center}
	\LARGE PART II: Methods
\end{center}

\section{Unity}

\section{The Lead \& the Ending}

\section{Bits \& Pieces}

\begin{center}
	\LARGE PART III: Forms
\end{center}

\section{Nonfiction as Literature}

\section{Writing About People: The Interview}

\section{writing About Places: The Travel Article}

\section{Writing About Yourself: The Memoir}

\section{Science \& Technology}

\section{Business Writing: Writing in Your Job}

\section{Sports}

\section{Writing About the Arts: Critics \& Columnists}

\section{Humor}

\begin{center}
	\LARGE PART IV: Attitudes
\end{center}

\section{The Sound of Your Voice}

\section{Enjoyment, Fear \& Confidence}

\section{The Tyranny of the Final Product}

\section{A Writer's Decisions}

\section{Writing Family History \& Memoir}

\section{Write as Well as You Can}

%------------------------------------------------------------------------------%

\chapter{\href{https://proofed.co.uk/}{Proofed}}

``Professional editing \& proofreading services at your fingertips.''

\section{\href{https://proofed.co.uk/writing-tips/how-to-write-a-preface-for-a-book/}{Proofed\texttt{/}How to Write a Preface for a Book}}
``A preface explains the writing of a \href{https://proofed.co.uk/author/book-editing-and-proofreading/}{book} in the author's own words. Typically, it will include how the book was conceived, researched \& written, the author's credentials \& any changes made to updated versions of the book.

\textit{But how do you write a preface?} In this post, we cover the following tips:
\begin{enumerate}
	\item Use your preface to share the book's origin story.
	\item If relevant, explain your credentials for writing the book.
	\item Make it compelling so you can grab the reader's attention.
	\item Keep it concise (ideally, no more than a page or 2).
	\item Edit \& proofread your preface to make sure it is error free.''
\end{enumerate}
[$\ldots$] ``1st, though, we'll look at whether \& when a book needs a preface.''

\subsection*{Does Your Book Need a Preface?}
``Not all books need a preface. They are most commonly found in \href{https://en.wikipedia.org/wiki/Nonfiction}{non-fiction} \& academic writing, but they may occasionally be encountered in fiction, too. \textit{So, how do you know if your book needs a preface?}

The purpose of a preface is to prime the reader for what follows. E.g., in the case of \href{https://en.wikipedia.org/wiki/Historical_fiction}{historical fiction}, the writer may provide historical detail in the preface to explain the context of the fictional work. In an academic book, meanwhile, the writer may explain the background of their research.

The key question, then, is whether there is something you need to share with the reader before they start reading the main text. \textit{Will knowing the story of how \& why you wrote it add something to the reading experience? Is there information that readers will need to know to make sense of the book overall?} If so, a preface is a great place to include this kind of material. If not, you're probably fine without one!

If you do decide to write a preface for your book, though, you'll want to make sure you get it right! \& that's where the following 5 top tips can come in handy.''

\subsection{Share Your Book's Origin Story}
``1 approach to writing a preface is to explain the background of the book:
\begin{itemize}
	\item \textit{Who or what inspired you to write it?}
	\item \textit{How did you assemble the story (e.g., what research methods, historical context \& personal, noteworthy experiences contributed to the book)?}
	\item \textit{What were the challenges of writing it?}
	\item \textit{What is the main purpose of the book?}
	\item \textit{\& what, if any, changes have been made to updated versions of the book?}
\end{itemize}
You may also want to acknowledge others who have helped you in your book's creation, although this is usually handled in a separate \href{https://www.linkedin.com/pulse/write-your-book-acknowledgments-without-stressing-over-tucker-max/}{acknowledgments section}.''

\subsection{Justify Your Role as Author}
``Another way to use a preface is to explain your credentials for writing the book. This can be especially useful in academic writing \& other forms of non-fiction.

If you decide to do this, highlight any relevant qualifications or experience you have. You can also explain why you care about the subject. Letting your enthusiasm shine through at this early stage is also a helpful way of engaging your readers.''

\subsection{Make It Compelling}
``Speaking of engaging readers, a good preface should be compelling enough to grab people's attention \& make them want to keep reading.

1 way to do this is to tease what is yet to come, sharing a few interesting details or insights that will pique the reader's curiosity. E.g., in \href{https://www.britannica.com/topic/Adventures-of-Huckleberry-Finn-novel-by-Twain}{\textit{The Adventures of Huckleberry Finn}}, Mark Twain describes how he tried to create a dialect for each character based on where they were from. Meanwhile, in the preface to \href{https://www.amazon.co.uk/dp/B002RI90B8/ref=dp-kindle-redirect?_encoding=UTF8&btkr=1}{\textit{Churchill's Wizards: The British Genius for Deception, 1914--1945}}, Nicholas Rankin devotes a section to Churchill's eclectic choice of headgear.''

\subsection{Keep It Concise}
``Typically, a good preface should be no more than 1--2 pages, covering only the essential points you want the reader to know before the start reading.

\fbox{Planning is key} here. Think about how much the reader really needs to know. If something isn't essential or especially compelling, cut it. Likewise, try to \href{https://proofed.co.uk/writing-tips/how-to-write-concisely/}{write concisely} (though don't vary your authorial voice too much to achieve this).''

\subsection{Edit \& Proofread Your Preface}
``The preface might be the last part of your book that you write, as it is usually easier to sum up the writing process, etc., once you have a strong draft of the rest of your manuscript ready. But this doesn't mean it warrants any less attention!

Once you have a 1st draft of your premise, then, set it aside for a day or 2, then come back to it with fresh eyes. You can then tweak or refine it as required. \& it's always a good idea to ask for feedback on your preface at this point.

Finally, remember to proofread your preface. \& to find out how Proofed can help with this process, you can \href{https://proofed.co.uk/author/book-editing-and-proofreading/}{submit a trial document} for proofreading today.''

%------------------------------------------------------------------------------%

\chapter{\href{https://lithub.com/}{Literary Hub}}

\section{\href{https://lithub.com/kazuo-ishiguro-write-what-you-know-is-the-stupidest-thing-ive-ever-heard/}{Literary Hub\texttt{/}Kazuo Ishiguro: `Write What You Know' is The Stupidest Thing I've Ever Heard \& Other Hot Tips from the Newly Crowned Nobel Laureate}}
\hfill By \textsc{Emily Temple}, Nov 8, 2017

``\textsc{Kazuo Ishiguro}, author of \textit{The Remains of the Day, Never Let Me Go}, \& almost recently \textit{The Buried Giant}, \& oh, also our newest Nobel Laureate\footnote{\textbf{laureate} [n] a person who has been given an official honor or prize for something important they have achieved.} in Literature\footnote{\textbf{literature} [n] \textbf{1.} [uncountable] pieces of writing that are considered to be works of art, especially novels, plays \& poems (in contrast to technical books \& newspapers, magazines, etc.); \textbf{2.} [uncountable, countable] pieces of writing or printed information on a particular subject.}, turns 63 today. I have long admired\footnote{\textbf{admire} [v] \textbf{1.} to respect somebody\texttt{/}something for what they are or what they have done; \textbf{2.} \textbf{admire somebody\texttt{/}something} to look at somebody\texttt{/}something \& think that they are attractive \&\texttt{/}or impressive.} Ishiguro's work, which seems almost effortless\footnote{\textbf{effortless} [a] needing little or no effort, so that it seems easy.}, presenting its multi-faceted\footnote{\textbf{multifaceted} [a] (\textit{formal}) having many different aspects to be considered.} subjects\footnote{\textbf{subject} [n] \textbf{1.} a thing or person that is being discussed, described or dealt with; \textbf{2.} a person being used to study something, especially in an experiment; \textbf{3.} an area of knowledge studied in a school, college, etc.; \textbf{4.} a person or thing that is the main feature or a picture or photograph, or that a work of art is based on; \textbf{5.} a member of a state who is not its ruler, especially when that ruler is a king or queen; \textbf{6.} (\textit{grammar}) a noun, nun phrase or pronoun representing the person or thing that performs the action of the verb, about which something is stated or, in a passive sentence, that is affected by the action of the verb; [a] \textbf{1.} \textbf{subject to something} affected or likely to be affected by an event, action or process; \textbf{2.} \textbf{subject to something} depending on something in order to be completed or agreed; \textbf{3.} \textbf{subject to something\texttt{/}somebody} under the authority of something\texttt{/}somebody; \textbf{4.} [only before noun] under the control of another ruler, country or government; [v] to bring a country or group of people under your control or rule, especially by using force; \textbf{subject somebody\texttt{/}something to something} [phrasal verb] [often passive] to make somebody\texttt{/}something experience or suffer something, usually something unpleasant; \textbf{subject something to something} [phrasal verb] [usually passive] to make something go through or be affected by a particular process, often as part of an experiment.} with cold-water clarity\footnote{\textbf{clarity} [n] [uncountable] \textbf{1.} the quality of being expressed clearly; \textbf{2.} the ability to think about or understand something clearly; \textbf{3.} if a picture, substance or sound has clarity, you can see or hear it very clearly, or see through it easily.} while nimbly\footnote{\textbf{nimbly} [adv] \textbf{1.} with quick \& easy movements; \textbf{2.} in a way that shows you are able to think \& understand quickly.} experimenting\footnote{\textbf{experiment} [n] \textbf{1.} [countable] a scientist test that is done in order to study what happens \& to gain new knowledge; \textbf{2.} [countable] a new activity, idea or method that you try out to see what happens or what effect it has; \textbf{3.} [uncountable] the process of testing something to study what happens or to see what effect it has; [v] \textbf{1.} [intransitive] to try or test new ideas or methods to find out what effect they have; \textbf{2.} [intransitive] to do a scientific experiment or experiments.} with genre\footnote{\textbf{genre} [n] a particular type or style of literature, art, film or music that you can recognize because of its special features.}, style, \& subject. He also has the distinction\footnote{\textbf{distinction} [n] \textbf{1.} [countable] a clear difference, especially between people or things that are similar or related; \textbf{2.} [uncountable] the division of people or things into different groups; \textbf{3.} [singular] \textbf{distinction of being\texttt{/}something} the quality of being something that is special; \textbf{4.} [uncountable, countable] a special mark, grade or award that is given to somebody, especially a student, for excellent work.} of being among the small class of authors whose work is critically\footnote{\textbf{critically} [adv] \textbf{1.} in a way that carefully analyzes what is good or bad about something; \textbf{2.} extremely; in a way that is important; \textbf{3.} seriously.} lauded\footnote{\textbf{lauded} [v] (\textit{formal}) \textbf{laud somebody\texttt{/}something} to praise somebody\texttt{/}something.} \& commercially\footnote{\textbf{commercially} [adv] \textbf{1.} in a way that is intended to make a profit for a business; \textbf{2.} in order to be sold.} successful, which is no small feat\footnote{\textbf{feat} [n] (\textit{approving}) an action or a piece of work that needs skill, strength or courage.}. So in case you would like to follow\footnote{\textbf{follow} [v] \textbf{1.} [transitive, intransitive] to come after something\texttt{/}somebody else in time or order; to happen as a result of something else; \textbf{2.} [intransitive, transitive] (not usually used in the progressive tenses) to be the logical result of something; \textbf{3.} [transitive] \textbf{follow something} to develop or happen in a particular way or according to a particular pattern; \textbf{4.} [transitive] \textbf{follow something} to take a particular course of action; \textbf{5.} [transitive] \textbf{follow something} to act according to advice, instructions or rules; \textbf{6.} [transitive] to act according to the example of somebody; to do something in the same way as somebody\texttt{/}something else; \textbf{7.} [transitive] to understand the meaning of an argument, or what somebody is saying; \textbf{8.} [transitive] \textbf{follow something} to take an active interest in something \& be aware of what is happening; \textbf{9.} [transitive] \textbf{follow somebody\texttt{/}something} (of a book, film, programme, etc.) to be concerned with the life or development of somebody\texttt{/}something; \textbf{10.} [transitive] \textbf{follow something} to go along a road, path, etc.; \textbf{11.} [transitive] \textbf{follow something} (of a road, path, etc.) to go in the same direction as something; \textbf{12.} [transitive] to come or go after or behind somebody\texttt{/}something; \textbf{follow in somebody's footsteps} [idiom] \textbf{1.} to use the work or example of somebody as a basis for further progress; \textbf{2.} to do the same job, have the same style of life, etc. as somebody else, especially somebody in your family.} in his footsteps\footnote{\textbf{footstep} [n] [usually plural] the sound or mark made each time your foot touches the ground when you are walking or running.} -- who wouldn't -- here's Ishiguro on his process, what he likes (\& hates) to see in literature, \& some advice for young writers.

\begin{enumerate}
	\item \textbf{Don't write what you know.} ``Write about what you know'' is the most stupid thing I've heard. It encourages\footnote{\textbf{encourage} [v] \textbf{1.} to make something more likely to happen or develop, \textsc{opposite}: \textbf{discourage}; \textbf{2.} to persuade somebody to do something by making it easier for them \& making them believe it is a good thing to do, \textsc{opposite}: \textbf{discourage}; \textbf{3.} \textbf{encourage somebody} to give somebody support or hope, \textsc{opposite}: \textbf{discourage}.} people to write a dull\footnote{\textbf{dull} [a] (\textbf{duller, dullest}) \textbf{1.} (of pain) not very severe, but continuous; \textbf{2.} not interesting or exciting; \textbf{3.} not bright; \textbf{4.} (of sound) not clear or loud; \textbf{5.} slow to understand things; not intelligent.} autobiography\footnote{\textbf{autobiography} [n] (plural \textbf{autobiographies}) [countable, uncountable] the story of a person's life, written by that person; this type of writing.}. It's the reverse\footnote{\textbf{reverse} [v] \textbf{1.} [transitive, intransitive] to change something completely so that it is the opposite of what it was before; to change in this way; \textbf{2.} [transitive] \textbf{reverse something} to change a previous decision, law, etc. to the opposite one, \textsc{synonym}: \textbf{revoke}; \textbf{3.} [transitive] \textbf{reverse something} to exchange the positions or functions of 2 things; [a] [only before noun] opposite to what has been mentioned; [n] \textbf{1.} (\textbf{the reverse}) [singular] the opposite of what has just been mentioned; \textbf{2.} a loss or defeat; a change from success to failure, \textsc{synonym}: \textbf{setback}; \textbf{3.} (\textbf{the reverse}) [singular] \textbf{reverse (of something)} the back of a coin, piece of material, piece of paper, etc.; \textbf{in reverse} [idiom] in the opposite order or way, \textsc{synonym}: \textbf{backwards}.} of firing the imagination \& potential of writers. -- from an interview with \href{https://www.shortlist.com/entertainment/books/kazuo-ishiguro-talks-zuckerberg-game-of-thrones-and-his-new-novel/97003}{ShortList}
	\item \textbf{Let go of genre boundaries.} Is it possible that what we think of as genre boundaries are things that have been invented fairly recently by the publishing industry? I can see there's a case for saying there are certain patterns, \& you can divide up stories according to these patterns, perhaps usually. But I get worried when readers \& writers take these boundaries too seriously, \& think that something strange happens when you cross them, \& that you should think very carefully before doing so $\ldots$ I would like to see things breaking down a lot more. I suppose my essential position is that I'm against any kind of imagination police, whether they're coming from marketing reasons or from class snobbery. -- from a conversation with Neil Gaiman in the \href{https://www.newstatesman.com/2015/05/neil-gaiman-kazuo-ishiguro-interview-literature-genre-machines-can-toil-they-can-t-imagine}{New Statesman}
	\item \textbf{Write towards emotions, not morals.} I'm not looking for any kind of clear moral, \& I never do in my novels. I like to highlight some aspect of being human. I'm not really trying to say, so don't do this, or do that. I'm saying, this is how it feels to me. Emotions are very important to me in a novel. -- from an interview at \href{https://www.huffingtonpost.com/2015/03/03/kazuo-ishiguro-interview_n_6785824.html}{HufPost}
	\item \textbf{In fact, start with the relationships.} I used to think in terms of characters, how to develop their eccentricities \& quirks. Then I realized that it's better to focus on the relationships instead, \& then the characters develop naturally.
	
	Relationship have to be natural, to be authentic human drama. I'm a little suspicious of stories that have an intellectual theme bolted, when the characters stop \& debate before they carry on.
	
	I ask myself: What is an interesting relationship? Is the relationship a journey? Is it standard, clich\'e, or something deeper, more subtle, more surprising? People talk about flat versus 3D characters; you can talk about relationships the same way. -- from an interview with \href{https://holdenlee.wordpress.com/2014/02/18/kazuo-ishiguro-on-writing/}{Richard Beard}
	\item \textbf{To eliminate distractions, try a ``Crash.''} Many people have to work long hours. When it comes to the writing of novels, however, the consensus seems to be that after 4 hours or so of continuous writing, diminishing returns set in. I'd always more or less gone along with this view, but as the summer of 1987 approached I became convinced a drastic approach was needed. Lorna, my wife, agreed $\ldots$. So Lorna \& I came up with a plan. I would, for a 4-week period, ruthlessly clear my diary \& go on what we somewhat mysteriously called a ``Crash''. During the Crash, I would do nothing but write from 9am to 10:30pm, Monday through Saturday. I'd get 1 hour off for lunch \& 2 for dinner. I'd not see, let alone answer, any mail, \& would not go near the phone. No one would come to the house. Lorna, despite her own busy schedule, would for this period do my share of the cooking \& housework. In this way, so we hoped, I'd not only complete more work quantitively, but reach a mental state in which my fictional world was more real to me than the actual one. $\ldots$ This, fundamentally, was how The Remains of the Day was written. Throughout the Crash, I wrote free-hand, not caring about the style or if something I wrote in the afternoon contradicted something I'd established in the story that morning. The priority was simply to get the ideas surfacing \& growing. Awful sentences, hideous dialogue, scenes that went nowhere -- I let them remain \& ploughed on. -- from ``How I Wrote The Remains of the Day in 4 Weeks,'' as published in \href{https://www.theguardian.com/books/2014/dec/06/kazuo-ishiguro-the-remains-of-the-day-guardian-book-club}{The Guardian}
	\item \textbf{Embrace the ``down draft.''} I have 2 desks. One has a writing slope \& the other has a computer on it. The computer dates from 1996. It's not connected to the Internet. I prefer to work by pen on my writing slope for the initial drafts. I want it to be more or less illegible to anyone apart from myself. The rough draft is a big mess. I pay no attention to anything to do with style or coherence. I just need to get everything own on paper. If I'm suddenly struck by a new idea that doesn't fit with what's gone before, I'll still put it in. I just make a note to go back \& sort it all out later. Then I plan the whole thing out from that. I number sections \& move them around. By the time I write my next draft, I have a clearer idea of where I'm going. This time round, I write much more carefully$\ldots$. I rarely go beyond the 3rd draft. Having said that, there are individual passages that I've had to write over \& over again. -- from an interview with \href{https://www.theparisreview.org/interviews/5829/kazuo-ishiguro-the-art-of-fiction-no-196-kazuo-ishiguro}{The Paris Review}
	\item \textbf{Protect your mind from unwanted influences.} I find that when you're writing, it becomes quite a battle to keep your fictional world intact. In fact, as i write, I almost deliberately avoid anything in the realm of what I'm working on. For instance, [while working on \textit{The Buried Giant}] I hadn't seen a single episode of \textit{Game of Thrones}. That whole thing happened when I was quite deep into the writing, \& I thought, `If I watch something like that, it might influence the way I visualize a scene or tamper with the world that I've set up.' -- from an interview with \href{https://electricliterature.com/a-language-that-conceals-an-interview-with-kazuo-ishiguro-author-of-the-buried-giant-9673849885c7}{Electric Literature}
	\item \textbf{Make deliberate choices.} Most writers have certain things that they decide quite consciously, \& other things they decide less consciously. In my case, the choice of narrator \& setting are deliberate. You do have to choose a setting with great care, because with a setting come all kinds of emotional \& historical reverberations. But I leave quite a large area for improvisation after that. -- from an interview with \href{https://www.theparisreview.org/interviews/5829/kazuo-ishiguro-the-art-of-fiction-no-196-kazuo-ishiguro}{The Paris Review}
	\item \textbf{Be sparing your allusions.} I don't really like to work with literary allusions very much. I never want to be in a position where I'm saying, ``You've got to read a lot of other stuff'' or ``You've got to have had a good education in literature to fully appreciate what I'm doing.'' $\ldots$ I actually dislike, more than many people, working through literary allusion. I just feel that there's something a bit snobbish or elitist about that. I don't like it as a reader, when I'm reading something. It's not just the elitism of it; it jolts me ou of the mode in which I'm reading. I've immersed myself in the world \& then when the light goes on I'm supposed to be making some kind of literary comparison to another text. I find I'm pulled out of my kind of fictional world, I'm asked to use my brain in a different kind of way. I don't like that. -- from an interview with \href{https://www.guernicamag.com/mythic-retreat/}{Guernica}
	\item \textbf{Be careful what you start.} Everything is built on the early part of the process. It's important to be careful about what projects you take on, in the same way that you should examine someone you want to get married to. It's different for everyone: should it be based on your experience, or do you write better at greatest distance, do you write best in a genre? Don't take on a creative project lightly. -- from an interview with \href{https://holdenlee.wordpress.com/2014/02/18/kazuo-ishiguro-on-writing/}{Richard Beard}''
\end{enumerate}

%-----------------------------------------------------------------------------%

\chapter{\href{https://medium.com/}{Medium}}

\section{\href{https://medium.com/personal-growth/george-orwell-why-your-writing-must-have-purpose-77a3e94d6692}{Medium\texttt{/}Harry J. Stead. The True Purpose of Writing: Orwell on the purpose behind art \& artists}}
``The Spanish Civil war broke out in the summer of 1936 \& Orwell set out for Barcelona to join the Republican forces. Het met with James McNair, a British socialist politician, in Barcelona to hear about the ongoing political crisis in Spain. McNair was confused as to why Orwell was so keen to fight in Spain. He quoted Orwell as saying, `I've come to fight against Fascism' \& that, `he would like to write about the situation \& endeavor to stir working class opinion in Britain \& France.' Clearly, Orwell saw the war in Spain as a great adventure.

George Orwell was, rather hastily, assigned as a corporal \& sent to the Aragon front. He saw snippets of intense action, but the front was relatively quiet. However, in mid-1937, Orwell was shot through the throat by a sniper, the bullet missing the major artery only by a whisker.

He wrote in `Homage to Catalonia' that people would often tell him that a man who is hit through the neck \& survives is the luckiest creature alive, but that he personally thought `it would be even luckier not to be hit at all.'

It was around this time that internal conflicts were heating up within the Republican side \& the political situation became unpredictable. The Communists had, for many political reasons, outlawed all organizations that differed from their ideology. This made Orwell, who had joined a Trotskyist battalion, a fugitive \& he was forced to flee.

Orwell served on the Aragon front for 115 days. It was not until the end of April 1937 that he was granted leave \& was able to see his wife, Eileen, in Barcelona again. Eileen wrote on 1 May that she found her husband, `a little lousy, dark brown, \& looking really very well.'

In 1946, George Orwell published an essay titled `Why I Write', detailing his journey to becoming a writer. In the essay, Orwell wrote about how he became the political writer that people had come to know him for.

Orwell lists \textbf{`4 great motives for writing'} which he feels exist in every writer. He explains that all are present, but in different proportions, \& that these proportions vary from time to time. \& the varying weight of these motives often determines how meaningful the writer's words are.

Orwell believed that a writer writes from a \textbf{`desire to seem clever, to be talked about, to be remembered after death, to get your own back on grown-ups who snubbed you in childhood.'} This is a trait that the writer shares with scientists, artists, lawyers -- \textbf{`the whole top crust of humanity'}. Indeed, after the age of 30, the majority abandon individual ambition \& work to serve those around them -- family, friends, colleagues. But, a minority, artists mostly, remains keen \textbf{`to live their own lives to the end'}.

\begin{quotation}
	``I do not think on can assess a writer's motives without knowing something of his early development. His subject matter will be determined by the age he lives in $\ldots$ but before he ever begins to write he will have acquired an emotional attitude from which he will never completely escape.''
\end{quotation}
Likewise, writers are those who have fallen in love with the rhyme \& fluidity of words, who have \textbf{`pleasure in the impact of 1 sound on another, in the firmness of good prose or the rhythm of a good story.'} They want their writing to look beautiful, to sing like poetry \& flow like music.

If it was not for the artistic experience, few people would choose to write at all. Orwell fell in love with the sound of words when he was just a child; he never abandoned this early worldview \& continued to adore the tune of his sentences throughout his life.

3rdly, writers desire, \& indeed all artists too, to be remembered after death. They want to be known as someone who left behind something of value for the world, someone who revealed a truth, someone who expanded the universe \& allowed the audience to see a reality they did not now existed before. This is the great attraction behind the written word for words stand firm against time while the author cannot.

But, most importantly, Orwell argued, artists are those who embrace a strong purpose, a desire \textbf{`to push the world in a certain direction'.}

Orwell considered himself a political writer, but he admitted that it is in his nature to be a person in whom the desire to appear clever \& to write poetically outweighed the desire to write from purpose \& clarity.

He may have even remained unaware of his political loyalties \& had he not lived during a period of great political turmoil. \& he cited the Spanish Civil War as the defining event that moulded the political slant of his writing:

\begin{quotation}
	`The Spanish war \& other events in 1936--37 turned the scale \& thereafter I knew where I stood. Every line of serious work that I have written since 1936 has been written, directly or indirectly, against totalitarianism \& for democratic socialism, as I understand it.'
\end{quotation}
The war in Spain showed Orwell his own purpose \& truth that his works so far had only lightly skimmed against.

During the decade after the Spanish Civil War, Orwell felt compelled to turn \textbf{`political writing into an art'}. \&, indeed, he fulfilled this purpose through the publication of both `Animal Farm' \& `Nineteen Eighty-Four', his most famous works, after the 2nd World War.

The 1st 3 motives of the artist are wholly selfish. \& rightly so. Because the artist is he who follows passion, all that which raptures \& excites him. He acts on behalf of his true `self' with few disturbances, hoping to discover his own kingdom, a place \& time where his soul can be at peace.

But, the artist who balances the expressions of his heart with a purpose not only serves himself, but the entire world. By expressing his own authentic message, the lyrics of his quiet life, he nudges upon a universal truth that belongs with everyone. \& it is the artist's responsibility to guide this truth to brighten the dark corners of his city.

Here, Orwell showed how he balanced the seething energies within him that compelled him to write:

\begin{quotation}
	`When I sit down to write a book, I do not say to myself, `I am going to produce a work of art.' I write it because there is some lie that I want to expose, some fact to which I want to draw attention, \& my initial concern is to get a hearing. But I could not do the work of writing a book, or even a long magazine article if it were not also an aesthetic experience.' -- p. 8, \textit{`Why I Write'}.
\end{quotation}
He concluded the essay by explaining that \textbf{`it is invariably where I lacked a political purpose that I wrote lifeless books \& was betrayed into purple passages, sentences without meaning, decorative adjectives \& humbug generally.'}

No writer knows which of their motives are the strongest, but all should, Orwell argued, understand which of them deserve to be followed. Orwell believed that when his writing lacked a purpose or a passion, his words were bland \& idle. \& so, he decided after the war in Spain that his writing would now be wielded as a weapon against the greatest problem of his time -- totalitarianism.

Orwell remained dedicated to the defence of democratic socialism throughout his life -- he followed this purpose even when it was irrational \& senseless to do so. It saw him explore the darkest streets of Paris \& London, submerge himself into poverty in Northern England \& fight fascism in Spain.

Indeed, Orwell was tough for an intellectual sort; he never feared a plunge into the cold, nor a long stare into the face of bitter truth. It is hear, as Orwell discovered, steeped in mud, lost in the trenches \& fighting for the truth, where all exceptional stories are born. This blood-thirsty persistence for uncomfortable facts is the reason the world continues to celebrate Orwell's novels.'' -- \textsc{Harry J. Stead}

%-----------------------------------------------------------------------------%

\chapter{\href{https://www.rollingstone.com/}{Rolling Stone}}

\section{\href{https://en.wikipedia.org/wiki/Rolling_Stone}{Wikipedia\texttt{/}Rolling Stone}}
``\textit{Rolling Stone} is an American monthly magazine that focuses on music, politics, \& popular culture. It was founded in \href{https://en.wikipedia.org/wiki/San_Francisco}{San Francisco, California}, in 1967 by \href{https://en.wikipedia.org/wiki/Jann_Wenner}{Jann Wenner}, \& the music critic \href{https://en.wikipedia.org/wiki/Ralph_J._Gleason}{Ralph J. Gleason}. It was 1st known for its coverage of rock music \& for political reporting by \href{https://en.wikipedia.org/wiki/Ralph_J._Gleason}{Hunter S. Thompson}. In the 1990s, the magazine broadened \& shifted its focus to a younger readership interested in youth-oriented television shows, film actors, \& popular music. It has since returned to its traditional mix of content, including music, entertainment, \& politics.

The 1st magazine was released in 1967 \& featured \href{https://en.wikipedia.org/wiki/John_Lennon}{John Lennon} on the cover \& was published every 2 weeks. It is known for provocative photography \& its cover photos, featuring musicians, politicians, athletes, \& actors. In addition to its print version in the \href{https://en.wikipedia.org/wiki/United_States}{United States}, it publishes content through \url{rollingstone.com} \& numerous international editions.

\href{https://en.wikipedia.org/wiki/Penske_Media_Corporation}{Penske Media Corporation} is the current owner of \textit{Rolling Stone}, having purchased 51\% of the magazine in 2017 \& the remaining 49\% in 2020. \href{https://en.wikipedia.org/wiki/Noah_Shachtman}{Noah Shachtman} become the editor-in-chief in 2021.'' -- \href{https://en.wikipedia.org/wiki/Rolling_Stone}{Wikipedia\texttt{/}Rolling Stone}

\section{\href{https://www.rollingstone.com/music/music-news/taylor-swift-nyu-speech-1355121/}{Rolling Stone\texttt{/}Read Taylor Swift's Inspiring Speech for NYU's Class of '22}}
\begin{flushright}
	By \textsc{Brittany Spanos}
\end{flushright}
``To day, you leave New York University \& then you go out into the world searching for what's next. \& so will I,'' superstar says at Yankee Stadium.

\textsf{Figure: Taylor Swift arrives to deliver the New York University 2022 Commencement Address at Yankee Stadium on May 18, 2022 in New York City.}

``\textsc{Taylor Swift} received a warm welcome to New York University on Wednesday when she accepted an honorary Doctor of Fine Arts, honoris causa, from the institution. Swift attended the 2022 all-school commencement ceremony at Yankee Stadium \& delivered a rousing speech for this year's graduates.



''


%-----------------------------------------------------------------------------%

\chapter{\href{http://www.tramdoc.vn/}{Trạm Đọc}}

\selectlanguage{vietnamese}

\section{\href{http://www.tramdoc.vn/tin-tuc/muoi-loi-khuyen-viet-lach-tu-nha-van-doat-giai-nobel-kazuo-ishiguro-n95AxW.html?fbclid=IwAR0rfLQU9Y0lB9vRdZ7LpsKl5z_l_2SMS5i5bLBxOzWn6hIr8yMWNMnFM00}{Trạm Đọc\texttt{/}10 Lời Khuyên Viết Lách Từ Nhà Văn Đoạt Giải Nobel Kazuo Ishiguro}}
``\textsc{Kazuo Ishiguro} là tác giả của ``Tàn ngày để lại'', ``Mãi đừng xa tôi'', $\ldots$, là chủ nhân của giải Nobel danh giá năm 2017. Các tác phẩm của ông thể hiện sự điêu luyện về thể loại, phong cách \& đề tài. Ngoài ra ông còn nằm trong danh sách những nhà văn hiếm có viết ra tác phẩm vừa có giá trị văn chương lại vừa có giá trị thương mại. Dưới đây là những gì Ishiguro nói về việc viết lách, về những gì ông thích (hoặc ghét) trong văn học, cùng 1 vài lời khuyên cho các nhà văn trẻ.

\begin{enumerate}
	\item \textbf{Đừng viết những gì bạn biết.} ````Viết về những gì bạn biết'' là điều ngu ngốc nhất mà tôi từng nghe. Nó khuyến khích mọi người viết 1 cuốn tự truyện buồn tẻ. Đó là mặt trái của việc khơi dậy trí tưởng tượng \& tiềm năng của các nhà văn.'' -- Từ cuộc phỏng vấn với \href{https://www.shortlist.com/entertainment/books/kazuo-ishiguro-talks-zuckerberg-game-of-thrones-and-his-new-novel/97003}{ShortList}
	\item \textbf{Bỏ qua ranh giới thể loại.} ``Có thể nào những gì chúng ta nghĩ về ranh giới thể loại là những thứ đã được phát minh khá gần đây bởi ngành xuất bản hay không? Tôi thấy có 1 trường hợp nói rằng có 1 số mẫu nhất định \& bạn có thể chia các câu chuyện theo các mẫu này, có thể sẽ hữu ích. Nhưng tôi lo lắng khi người đọc \& người viết quá coi trọng những ranh giới này, \& nghĩ rằng sẽ có điều gì đó kỳ lạ xảy ra khi bạn vượt qua chúng \& bạn phải suy nghĩ thật kỹ trước khi làm như vậy $\ldots$ Tôi muốn thấy các rào cản đổ vỡ nhiều hơn nữa. Tôi cho rằng vị trí quan trọng của tôi là chống lại bất kỳ thứ gì cảnh sát canh giữ cho trí tưởng tượng, cho dù họ đến từ lý do tiếp thị hay từ sự hơm hĩnh của giai cấp.'' -- Từ cuộc trò chuyện với Neil Gaiman trong \href{https://www.newstatesman.com/2015/05/neil-gaiman-kazuo-ishiguro-interview-literature-genre-machines-can-toil-they-can-t-imagine}{New Statesman}
	\item \textbf{Viết hướng tới cảm xúc chứ không phải đạo đức.} ``Tôi không tìm kiếm bất kỳ loại đạo đức rõ ràng nào, \& tôi không bao giờ làm như vậy trong tiểu thuyết của mình. Tôi muốn làm nổi bật 1 số khía cạnh của con người. Tôi không thực sự cố gắng thể hiện, như vậy đừng làm điều này, hoặc làm điều kia. Tôi chỉ muốn nói đây là cảm giác của tôi. Cảm xúc rất quan trọng đối với tôi trong 1 cuốn tiểu thuyết.'' -- Từ một cuộc phỏng vấn tại \href{https://www.huffingtonpost.com/2015/03/03/kazuo-ishiguro-interview_n_6785824.html}{HuffPost}
	\item \textbf{Trên thực tế, hãy bắt đầu từ các mối quan hệ.} `` Tôi đã từng nghĩ về các nhân vật, làm thế nào để phát triển tính cách lập dị \& kỳ quặc của họ. Sau đó, tôi nhận ra rằng tốt hơn là nên tập trung vào các mối quan hệ \& sau đó các nhân vật phát triển 1 cách tự nhiên.
	
	Các mối quan hệ phải tự nhiên, phải là 1 vở kịch đích thực của con người. Tôi hơi nghi ngờ về những câu chuyện có chủ đề trí tuệ, khi các nhân vật dừng lại \& tranh luận với nhau trước khi tiếp tục.
	
	Tôi tự hỏi mình: 1 mối quan hệ thú vị là gì? Mối quan hệ của phải là 1 cuộc hành trình? Nó là tiêu chuẩn, khuôn sáo, hay 1 cái gì đó sâu sắc hơn, tinh tế hơn, đáng ngạc nhiên hơn? Mọi người thường nói về các nhân vật của 2 chiều hoặc có chiều sâu; bạn cũng có thể hiểu về các mối quan hệ theo cách tương tự.'' -- Từ cuộc phỏng vấn với \href{https://holdenlee.wordpress.com/2014/02/18/kazuo-ishiguro-on-writing/}{Richard Beard}
	\item \textbf{Để loại bỏ những phiền nhiễu, hãy thử ``Crash''.} ``Nhiều người phải làm việc nhiều giờ. Tuy nhiên, khi nói đến việc viết tiểu thuyết, dường như ai cũng đồng ý rằng sau độ khoảng 4 giờ đồng hồ liên tục viết, thì thành quả thu được sẽ bắt đầu không tương xứng với công sức bỏ ra. Tôi ít nhiều cũng đồng tình với quan điểm này, nhưng kể từ mùa hè năm 1987 đến này, tôi tin rằng cần phải có 1 cách tiếp cận quyết liệt hơn. Lorna, vợ tôi cũng đồng ý với điều đó. Vì vậy, Lorna \& tôi đã nghỉ ra 1 kế hoạch. Trong khoảng thời gian 4 tuần, tôi sẽ thẳng tay xóa những dự định của mình \& tiếp tục làm 1 thứ bí ẩn gọi là ``Crash''. Trong thời gian đó, tôi sẽ không làm gì khác ngoài việc viết từ 9:00 sáng đến 10:30 tối, từ Thứ 2 đến thứ 7. Tôi sẽ được nghỉ1 giờ cho bữa trưa \& 2 giờ cho bữa tối. Tôi không quan tâm \& không trả lời thư \& sẽ không đến gần điện thoại. Không ai đến nhà. Lorna, mặc dù lịch trình bận rộn của riêng mình nhưng trong giai đoạn này sẽ đảm nhiệm tất cả việc nấu nướng \& nội trợ. Bằng cách này, chúng tôi hy vọng, tôi không chỉ hoàn thành nhiều công việc hơn về mặt định lượng, mà còn \textit{đạt đến trạng thái tinh thần trong đó thế giới hư cấu của tôi đối với tôi thực hơn thế giới thực} $\ldots$ Về cơ bản, đây là accsh viết ``The Remains of the Day''. Trong suốt quá trình Crash, tôi viết tự do, không quan tâm đến văn phong hoặc có điều gì đó tôi viết vào buổi chiều mâu thuẫn với những thứ tôi đã xây dựng trong câu chuyện sáng hôm đó hay không. Ưu tiên chỉ đơn giản là làm cho các ý tưởng xuất hiện \& phát triển. Những câu kinh khủng, những phân cảnh chẳng đi đâu đến đâu -- tôi cứ để chúng ở đó \& tiếp tục.'' -- Từ ``Cách Tôi Viết ``Tàn ngày để lại'' Trong 4 Tuần'', được đăng trên \href{https://www.theguardian.com/books/2014/dec/06/kazuo-ishiguro-the-remains-of-the-day-guardian-book-club}{The Guardian}
	\item \textbf{Hãy giữ lấy ``bản nháp''.} ``Tôi có 2 bàn làm việc. 1 cái bàn nghiêng để viết \& cái bàn còn lại có máy tính trên đó. Máy tính đời 1996. Nó không được kết nối với Internet. Tôi thích viết bằng bút trên mặt bàn nghiêng cho các bản nháp ban đầu. Tôi muốn nó ít nhiều không thể đọc được đối với bất kỳ ai ngoài bản thân tôi. Bảo thảo thô là 1 mớ hỗn độn lớn. Tôi không chú ý đến bất cứ điều gì liên quan đến văn phong hay sự mạch lạc. Tôi chỉ cần ghi lại mọi thứ trên giấy. Nếu tôi đột nhiên nghĩ ra 1 ý tưởng mới không ăn nhập với những gì đã có trước đó, tôi sẽ vẫn tiếp tục đưa nó vào. Tôi chỉ cần ghi chú để quay lại \& sắp xếp lại sau. Sau đó, tôi lên kế hoạch toàn bộ cho cuốn sách. Tôi đánh số các phần \& di chuyển vị trí của chúng. Vào thời điểm tôi viết bản nháp tiếp theo của mình, tôi đã có 1 ý tưởng rõ ràng hơn về nơi tôi sẽ đi. Lần này, tôi viết cẩn thận hơn nhiều $\ldots$ Tôi hiếm khi viết đến bản nháp thứ 3. Nhưng có những phần tôi đã phải viết đi viết lại nhiều lần.'' -- Từ cuộc phỏng vấn với The Paris Review.
	\item \textbf{Bảo vệ tâm trí của bạn khỏi những ảnh hưởng không mong muốn.} ``Tôi thấy rằng khi bạn đang viết, việc giữ nguyên vẹn thế giới hư cấu của bạn sẽ trở thành 1 trận chiến. Trên thực tế, khi tôi viết, tôi gần như cố ý tránh bất cứ điều gì liên quan đến lĩnh vực mà tôi đang viết. Ví dụ, trong khi viết ``Người khổng lồ ngủ quên'', tôi đã không xem 1 tập nào của ``Trò chơi vương quyền''. Bộ phim được chiếu khi tôi đã khá sâu vào cuốn sách của mình \& tôi nghĩ, ``Nếu tôi xem 1 cái gì đó như vậy, nó có thể ảnh hưởng đến cách tôi hình dung 1 phân đoạn hoặc làm xáo trộn thế giới mà tôi đã thiết lập nên''. -- Từ cuộc phỏng vấn với \href{https://electricliterature.com/a-language-that-conceals-an-interview-with-kazuo-ishiguro-author-of-the-buried-giant-9673849885c7}{Electric Văn học}
	\item \textbf{Thực hiện các lựa chọn có chủ ý.} ``Hầu hết các nhà văn có 1 số việc mà họ quyết định 1 cách khá tỉnh táo, \& những thứ khác họ quyết định ít có ý thức hơn. Trong trường hợp của tôi, việc lựa chọn người kể chuyện \& bối cảnh là có chủ ý. Bạn phải chọn 1 bối cảnh thật cẩn thận, bởi vì với 1 bối cảnh bao gồm tất cả các loại âm vang lịch sử \& cảm xúc. Nhưng tôi để lại 1 khoảng đất khá rộng để ngẫu hứng sau đó.'' -- Từ cuộc phỏng vấn với \href{https://www.theparisreview.org/interviews/5829/kazuo-ishiguro-the-art-of-fiction-no-196-kazuo-ishiguro}{The Paris Review}
	\item \textbf{Sử dụng ít những điển tích văn học.} ``Tôi không thực sự thích những ám chỉ văn học cho lắm. Tôi không bao giờ muốn ở vào vị trí mà tôi sẽ nói rằng: ``Bạn phải đọc nhiều thứ khác'' hoặc ``Bạn phải có 1 nền giáo dục tốt về văn học để đánh giá đầy đủ những gì tôi viết.'' $\ldots$ Tôi thực sự không thích, hơn nhiều người khác, viết bằng những điển tích văn chương. Tôi chỉ cảm thấy rằng có điều gì đó hơi hợm hĩnh hoặc tinh hoa về điều đó. Tôi không thích nó với tư cách là 1 người đọc, khi tôi đang đọc 1 thứ gì đó. Đó không chỉ là chủ nghĩa tinh hoa; nó đẩy tôi ra khỏi tâm trạng đọc. Tôi đắm mình trong thế giới của tôi \& rồi khi ánh sáng bật lên, tôi phải so sánh văn học với 1 văn bản khác. Tôi thấy mình bị kéo ra khỏi thế giới trong tưởng tượng của mình, tôi được yêu cầu sử dụng bộ não của mình theo 1 cách khác. Tôi không thích điều đó.'' -- Từ cuộc phỏng vấn với \href{https://www.guernicamag.com/mythic-retreat/}{Guernica}
	\item \textbf{Hãy cẩn thận những gì bạn bắt đầu.} ``Mọi thứ đều được xây dựng trên phần đầu của quá trình. Điều quan trọng là phải cẩn thận về những dự định bạn thực hiện, giống như cách bạn nên xem xét người mà bạn muốn kết hôn. Nó khác với mọi người: nó nên dựa trên kinh nghiệm của bạn, hay bạn viết tốt hơn ở khoảng cách xa nhất, bạn viết tốt nhất ở thể loại nào? Đừng xem nhẹ 1 dự án sáng tạo.'' -- Từ cuộc phỏng vấn với \href{https://holdenlee.wordpress.com/2014/02/18/kazuo-ishiguro-on-writing/}{Richard Beard}
\end{enumerate}

\begin{flushright}
	\textbf{Trạm Đọc} $|$ Theo \href{https://lithub.com/kazuo-ishiguro-write-what-you-know-is-the-stupidest-thing-ive-ever-heard/}{Lithub}
\end{flushright}

\section{\href{http://www.tramdoc.vn/tin-tuc/dau-moi-la-muc-dich-thuc-su-cua-viec-viet-lach-nnOqWW.html}{Trạm Đọc\texttt{/}Đâu Mới Là Mục Đích Thực Sự Của Việc Viết Lách?}}
``\textbf{Orwell bàn về động cơ thực sự của nghệ thuật \& của các nghệ sĩ.}

Nội chiến Tây Ban Nha bùng nổ vào mùa hè năm 1936 \& Orwell đã lên đường tới thành phố Barcelona để gia nhập phe Cộng hòa. Tại đó, Orwell gặp nhà chính trị xã hội người Anh, James McNair \& nghe ông ấy kể về cuộc khủng hoảng chính trị đang tiếp diễn ở quốc gia này. McNair thực sự cảm thấy bối rối trước lý do khiến Orwell đặc biệt quan tâm đến cuộc chiến ở Tây Ban Nha. Ông trích dẫn lời của Orwell rằng: ``Tôi đến để chiến đấu chống lại chủ nghĩa phát xít'' \& rằng ``ông ấy sẽ viết về tình hình \& nỗ lực khuấy động tinh thần đấu tranh của tầng lớp lao động ở Anh \& Pháp''. Rõ ràng, Orwell đã xem chiến tranh ở Tây Ban Nha là 1 cuộc phiêu lưu vĩ đại. George Orwell nhanh chóng được phong làm hạ sĩ quan \& được điều đến mặt trận Argagon. Tại đây, ông đã nhìn thấy những cảnh tượng căng thẳng nhưng chiến trường lại chẳng có động tĩnh gì. Tuy nhiên, vào giữa năm 1937, cổ họng Orwell bị trúng đạn. Rất may, viên đạn không xuyên qua động mạch chủ của ông.

Ông viết trong tác phẩm ``Lòng kính trọng dành cho Catalonia'' rằng, mọi người thường kể với ông rằng, 1 người đàn ông bị bắn vào cổ mà vẫn sống sót là người may mắn nhất nhưng cá nhân ông lại nghĩ rằng, ``Sẽ là may mắn hơn nếu ông không bị trúng phát đạn nào''.

Sự việc đó xảy đến đúng vào lúc các cuộc xung đột trong nội bộ phe Cộng hòa ở Tây Ban Nha diễn ra ngày càng ác liệt \& các diễn biến ỏ chiến trường luôn có vẻ khó lường. Vì nhiều lý do, những người cộng sản cấm tất cả các tổ chức khác hệ tư tưởng với mình. Do tham gia tiểu đoàn Trotskyist, Orwell trở thành kẻ lánh nạn \& buộc phải chạy trốn.

Khi Orwell phục vụ ở mặt trận Aragon được 115 ngày, mãi đến cuối tháng 4 năm 1937, ông mới được phép xuất ngũ \& được gặp vợ mình, Eileen ở thành phố Barcelona. Vào 1.5 năm đó, Eileen viết rằng, bà đã tìm thấy chồng, ``1 người đàn ông nhếch nhác, làn da nâu sậm nhưng nhìn vẫn rất khỏe mạnh''.

Vào năm 1946, George Orwell xuất bản 1 bài luận với tựa đề ``Vì sao tôi viết'' miêu tả chi tiết hành trình trở thành 1 nhà văn của mình.

Orwell đã liệt kê ra ``4 loại động cơ tuyệt vời thôi thúc ông cầm bút viết'' \& tin rằng, nhiều tác giả khác cũng cảm thấy như vậy. Ông giải thích rằng, hiện tại, ông vẫn chịu ảnh hưởng của cả 4 động lực đó nhưng theo các tỷ lệ khác nhau \& rằng những tỷ lệ đó dẫn thay đổi theo thời gian. \& mức độ nặng nhẹ khác nhau của các động cơ đó thường quyết định giá trị của tác phẩm.

Orwell tin rằng, mục đích cao cả mà 1 tác giả hướng đến là họ sẽ viết để trở nên thông tuệ, để được mọi người bàn luận về tác phẩm của họ, để được thế hệ sau nhắc đến sau khi họ qua đời, để nhận được sự ủng hộ từ những người đã từng bắt nạt họ khi họ ở tuổi ấu thơ. Các nhà khoa học, luật sư cũng có chung suy nghĩ này. Trên thực tế, sau 30 tuổi, đa phần mọi người đều rời bỏ tham vọng cá nhân \& làm việc để phục vụ những người xung quanh họ: gia đình, bạn bè, đồng nghiệp. Tuy nhiên, 1 bộ phận nhỏ, chủ yếu là các nghệ sĩ vẫn ưa thích cuộc sống cá nhân cho đến cùng.

\begin{quotation}
	``Tôi không nghĩ mọi người có thể đánh giá động lực viết lách của 1 tác giả mà không biết gì về thành tựu trước đó của anh ta. Độ tuổi cũng sẽ quyết định vấn đề anh ta muốn viết nhưng trước khi cầm bút lên, anh ta chắc chắn từng chìm vào cảm xúc mà cho đến mãi sau này, thậm chí là tương lai, anh ta vẫn chưa thể hoàn toàn thoát ra được.''
\end{quotation}
Tương tự như vậy, nhà văn bao giờ cũng bị cuốn hút bởi nhịp ngắt nghỉ sự lưu loát của câu văn. Họ muốn nội dung tác phẩm của mình hiển hiện sống động trong trí tưởng tượng của độc giả, người đọc có thể phổ nhạc hoặc ngân nga những câu đó.

Nếu không có những trải nghiệm nghệ sĩ như vậy, gần như chẳng có ai đi theo nghiệp viết lách cả. Orwell thích thú với âm thanh của từ ngữ khi ông mới chỉ là 1 đứa trẻ; ông chưa bao giờ bỏ rơi thế giới quan \& không ngừng tôn thờ giai điệu của câu văn.

Điều thứ 3, các nhà văn luôn khao khát \& giới nghệ sĩ cũng đều như vậy. Họ ao ước người đời sau sẽ vẫn nhớ đến họ ngay cả khi họ đã qua đời. Họ muốn mình được biết đến với tư cách là người đã để lại điều gì đó có giá trị cho thế giới, người đã tiết lộ sự thật, người đã mở rộng vũ trụ \& cho phép khán giả nhìn thấy 1 thực tế mà họ không hề hay biết trước đó. Đây là điểm thu hút đằng sau các câu chữ được viết ra. Từ ngữ luôn tồn tại vững chãi qua thời gian còn tác giả thì không thể.

Nhưng điều quan trọng nhất, Orwell lập luận rằng, các nghệ sĩ là những người luôn mang trong mình 1 mục tiêu cao cả, 1 khao khát ``thúc đẩy thế giới đi theo 1 hướng nhất định''.

Orwell tự xem mình là nhà văn chính trị nhưng ông cũng thừa nhận rằng, về bản chất, ông muốn viết để trở nên thông minh \& dể viết 1 cách thi vị hơn.

\begin{quotation}
	``Chiến tranh Tây Ban Nha \& các sự kiện diễn ra vào khoảng những năm 1936--1937 đã làm thay đổi quy mô cuộc chiến \& sau đó tôi biết vị trí của mình nằm ở đâu. Mỗi dòng trong tác phẩm tôi viết ra từ năm 1936 đã trực tiếp hoặc gián tiếp chống lại chủ nghĩa toàn trị \& ủng hộ xã hội dân chủ.''
\end{quotation}
Chiến tranh ở Tây Ban Nha đã giúp Orwell nhận ra mục đích viết của chính ông \& sự thật là các tác phẩm của ông mới chỉ thể hiện khá hời hợt sự phản kháng.

Trong suốt thế kỷ sau nội chiến Tây Ban Nha, Orwell buộc phải chuyển các bài viết bàn về chính trị thành các tác phẩm viết về nghệ thuật. Ông đã cho ra đời 2 tác phẩm ``Trang trại động vật'' \& ``18 -- 84'', tác phẩm nổi tiếng nhất của ông sau chiến tranh thế giới thứ 2.

3 mục tiêu đầu tiên của người nghệ sĩ mới chỉ hướng đến bản thân họ. Đó là vì người nghệ sĩ chính là bản thân anh ta, người theo đuổi đam mê, \& đam mê ấy khiến anh ta cảm thấy vui \& hào hứng. Anh ta hy vọng khám phá ra vương quốc của mình, điểm đến \& thời gian cho tâm hồn muốn được an yên của mình.

Tuy vậy, người nghệ sĩ nên cân bằng cảm xúc với mục tiêu viết không phải chỉ vì mỗi mình anh ta mà còn vì cả thế giới. Bởi qua việc thể hiện thông điệp thực sự của chính mình, giai điệu của cuộc đời lăng yên, anh ta đang thúc đẩy 1 sự thật phổ quát thuộc về mọi người. \& đó chính là trách nhiệm của người nghệ sĩ. Họ đưa quần chúng đến sự thật, làm sáng tỏ các góc tối trong thành phố anh ta.

\begin{quotation}
	``Orwell từng viết rằng: ``Khi tôi ngồi viết sách, tôi không nói với bản thân mình rằng, ``Tôi sẽ sản sinh ra 1 tác phẩm nghệ thuật''. Tôi viết vì tôi phát hiện ra điều gì đó giả dối \& tôi muốn phơi bày sự thật, chứng cứ khiến tôi quan tâm \& điều đầu tiên tôi cần là sự lắng nghe của mọi người. Nhưng tôi không thể viết sách hay viết 1 bài tạp chí dài nếu nó không phải là 1 trải nghiệm có tính thẩm mỹ.'' (p. 8, trang tác phẩm ``Tại sao tôi viết'')
\end{quotation}
Ông kết thúc bài luận bằng việc giải tích rằng, ``đó luôn là nơi tôi thiếu 1 mục tiêu chính trị, tôi viết các cuốn sách vô hồn, các câu văn vô nghĩa, các tính từ chỉ mang tính trang trí, chung chung.''

Viết lách là 1 cuộc chiến làm kiệt sức, 1  sự hỗn loạn của các khả năng \& cảm xúc. Mọi người sẽ không thể chịu đựng được rào cản này nếu không có 1 góc tối nào đó trong tâm hồn thôi thúc họ. Orwell cho rằng, không tác giả nào hiểu được động cơ nào mạnh mẽ nhất thúc đẩy họ sáng tác nhưng họ nên hiểu điều gì đáng để họ theo đuổi. Orwell tin rằng, khi tác phẩm của ông không có mục đích nào để hướng đến hoặc thể hiện 1 đam mê nào đó thì những gì ông viết ra thật sự nhạt nhẽo \& vu vơ. \& vì vậy, sau khi chiến tranh ở Tây Ban Nha chấm dứt, ông đã quyết định rằng, tác phẩm của ông sẽ được sử dụng như 1 thứ vũ khí chống lại vấn đề nhức nhối nhất trong thời đại ông sống: chủ nghĩa toàn trị.

Trong suốt cuộc đời viết lách của mình, ông vẫn luôn ủng hộ chủ nghĩa xã hội dân chủ. Điều này chứng tỏ, ông từng khám phá ra những con phố tăm tối ở Paris \& London, nhấm chìm bản thân trong sự cơ cực, nghèo khổ ở miền Bắc nước Anh \& chiến đấu chống lại chủ nghĩa phát xít ở Tây Ban Nha.

Thực vậy, Orwell đã phải trải qua nhiều chông gai mới lĩnh hội được những tri thức đó. Ông chưa bao giờ ngần ngại lao mình vào nơi lạnh giá hay nhìn chằm chằm vào sự thật cay đắng. Chính tại những thời khắc khi Orwell bị kẻ địch phát hiện ra, phải dìm mình trong bùn, bị mất chiến hào \& đấu tranh cho chính nghĩa là lúc ông có nguyên liệu viết ra những câu chuyện hùng hồn. Sự kiên nhẫn phơi bày những thực tế không mấy thoải mái đó là lý do vì sao thế giới tiếp tục ca ngợi các cuốn tiểu thuyết của Orwell.'' -- Jenny, theo bài viết \href{https://medium.com/personal-growth/george-orwell-why-your-writing-must-have-purpose-77a3e94d6692}{Medium\texttt{/}The True Purpose Of Writing}.

%-----------------------------------------------------------------------------%

\selectlanguage{english}
\chapter{\href{https://zingnews.vn}{Zing News}}
\selectlanguage{vietnamese}

\section{\href{https://zingnews.vn/cac-nha-van-au-my-viet-gi-ve-tuoi-tre-post1170767.html}{Zing News\texttt{/}Các Nhà Văn Âu Mỹ Viết Gì Về Tuổi Trẻ?}}
``\textbf{Tuổi thanh xuân luôn được ca ngợi là quãng thời gian đáng quý trong đời người bởi nhiều lý do. Thế nên, mỗi nhà văn đã tìm được 1 cái cớ riêng để viết về nó.} Với cái nhìn đa dạng, các nhà văn đã kể những câu chuyện khác nhau về thời thanh xuân. Hãy cùng điểm qua 1 vài cuốn tiểu thuyết đặc sắc của văn chương Âu Mỹ viết về tuổi trẻ.''

\subsection{\textit{Nhà Giả Kim} của Paulo Coelho}
``Khi nói tới tiểu thuyết có nhân vật chính là những người trẻ tuổi, không thể bỏ qua \textit{Nhà giả kim}. Tác phẩm này đã đưa tên tuổi của nhà văn người Brazil Paulo Coelho đến với độc giả khắp thế giới. Tác phẩm khiến ông trở thành ``nhà văn viết bằng tiếng Bồ Đào Nha bán chạy nhất mọi thời đại''. 1 số nhà phê bình cho rằng \textit{Nhà giả kim} là sự pha trộn tuyệt vời giữa thi ca \& 1 câu chuyện ý nghĩa về lý tưởng sống.

Nhân vật chính của cuốn tiểu thuyết là chàng trai trẻ Santiago thích phiêu lưu với bao hoài bão. Gia đình muốn anh trở thành linh mục. Nhưng Santiago lại muốn chu du đây đó, nên anh chàng này đã thuyết phục cha để trở thành người chăn cừu. Từ đó, Santiago cùng bầy cừu lang thang khắp vùng Andalusia.

Số mệnh \& những hoài bão của tuổi trẻ đã đa Santiago tới những quyết định táo bạo. Chàng bán bầy cừu \& bắt đầu cuộc hành trình mới. Santiago muốn tới Ai Cập để tìm kho báu, nhưng rồi chàng nhận ra kho báu quan trọng nhất nằm trên chính quê hương mình.

Không chỉ là cuốn tiểu thuyết giàu chất thơ, Paulo Coelho còn mang đến cho người đọc nhiều bài học sâu sắc về cuộc sống qua tiểu thuyết \textit{Nhà giả kim}. Tác phẩm này đã trở thành cuốn sách tâm đắc của nhiều người nổi tiếng.''

\subsection{\textit{Ở Quán Cà Phê Của Tuổi Trẻ Lạc Lối} của Patrick Modiano}
``Patrick Modiano được mệnh danh là ``nhà văn của ký ức''. Trong hành trình văn chương, ông mang đến cho người đọc những áng văn đẹp, mang màu sắc hoài niệm, nơi người viết luôn chìm đắm trong quá khứ.

Trong \textit{Quán cà phê của tuổi trẻ lạc lối}, nhà văn để các nhân vật của mình mải miết đi tìm ý nghĩa của tuổi trẻ \& tự vấn về những gì đã qua.

Bối cảnh của cuốn tiểu thuyết là quán cà phê Le Condé. Những vị khách tới đây đều nằm trong độ tuổi từ 19--25, đó là những năm tháng đẹp nhất của đời.

Nhưng sự trầm tư của họ khiến cho chúng ta hoài nghi về điều đó. 1 chàng sinh viên bỏ học, 1 cô gái trẻ \& người tình, 1 thám tử tư đang mông lung không biết mình tìm kiếm điều gì. Mỗi người theo đuổi những ý nghĩ của riêng mình, sự cô đơn trở thành sợi dây vô hình kết nối họ.

Trong dòng chảy ký ức của mỗi nhân vật, người đọc cảm nhận được sự bất an cũng những lo sợ sâu thẳm bên trong họ. Tất cả đều hoang mang trước cuộc sống này, họ tiếc nuối những gì đã qua \& lo sợ trước tương lai như ảo ảnh.''

\subsection{\textit{La Mã Sụp Đổ} của J\'er\^ome Ferrari}
``Rời quán cà phê Le Condé của Patrick Modiano, 1 nhà văn người Pháp khác là J\'er\^ome Ferrari sẽ đưa đọc giả tới 1 quán bar nhỏ trên đảo Corse để kể chuyện về anh chàng giàu có, mơ mộng.

Matthieu \& Libero bỏ học đại học \& nuôi giấc mộng kiến tạo thế giới cho riêng mình. Họ đã tìm thấy quán bar nhỏ trên đảo Corse để dừng chân trước khi bắt đầu hành trình mới.

Ở đây, họ nhận ra mộng tưởng \& thực tế là những câu chuyện hoàn toàn khác nhau. Tưởng tượng thì dễ, nhưng bắt tay để thực hiện mọi thứ là cả 1 vấn đề. Trong cuộc đời này, không có gì là hoàn hảo. Điều quan trọng nằm ở chỗ, con người ta biết chấp nhận những thứ chưa hoàn hảo \& khiến nó trở nên tốt đẹp hơn.

Trước khi bắt đầu sự nghiệp viết lách, nhà văn J\'er\^ome Ferrari đã có nhiều năm nghiên cứu về triết học. Trong cuốn tiểu thuyết của mình, ông đã khéo léo gửi gắm tới độc giả trẻ tuổi những lời khuyên hữu ích về nhiều vấn đề cuộc sống. Trốn tránh chỉ là giải pháp tạm thời, muốn sống 1 cách than thản, chúng ta phải học cách đối mặt với những khó khăn.''

\subsection{\textit{Ở Giữa Thanh Xuân Trống Rỗng} của Bret Easton Ellis}
``Với cuốn tiểu thuyết này, nhà văn Bret Easton Ellis mang đến 1 bức tranh sống động về mặt trái của sự giàu sang.

3 nhân vật chính Lauren, Sean \& Paul là sinh viên của 1 đại học tư thục hạng sang. Họ lớn lên trong nhung lụa \& thoải mái chi tiêu mà chưa bao giờ phải nghĩ đến chuyện tiền bạc. Thế nhưng, họ có thực sự hạnh phúc không?

Cuộc sống của cả 3 là những ngày tháng chìm đắm trong những bữa tiệc xa hoa với rượu \& chất kích thích. Không 1 ai trong số họ để ý đến chuyện học hành. Có những kẻ khốn khổ vì không có tiền, còn 3 cậu ấm, cô chiêu trong tác phẩm đau khổ vì không có nổi 1 giấc mơ.

Bret Easton Ellis đã mang tới 1 câu chuyện chân thực, đậm chất đời sống. Góc tối của ``những con người sinh ra ở vạch đích'' được tác giả bóc trần. Mọi sai lầm đều có nguyên do của nó. Ẩn sau những con người bất cần \& trụy lạc kia là trái tim đầy ắp vết thương.'' -- Thụy Oanh, Jan 7, 2021

%-----------------------------------------------------------------------------%

\section{\href{https://zingnews.vn/cac-nha-van-chau-a-viet-gi-ve-tuoi-tre-post1126430.html}{Zing News\texttt{/}Các Nhà Văn Châu Á Viết Gì Về Tuổi Trẻ?}}
``\textbf{Những năm tháng đôi mươi hiện lên trong văn chương của các cây bút châu Á với những màu sắc riêng. Ở đó, bạn đọc thấy nụ cười, nước mắt \& cả sự nuối tiếc khôn nguôi.} Thanh xuân luôn là quãng thời gian tươi đẹp nhất của đời người. Sự bướng bỉnh, đôi lúc có chút ngông cuồng của tuổi trẻ lại chính là nguồn sức mạnh để kẻ yếu đuối cũng có thể làm nên kỳ tích.

Chỉ có những năm tháng hoa niên tươi đẹp ấy, con người mới dám sống hết mình cho tình yêu \& lý tưởng của bản thân, dám liều lĩnh mà không hề sợ hãi.

Đó cũng là lý do khiến bao thăng trầm của tuổi trẻ trở thành đề tài giành được sự quan tâm đặc biệt của các nhà văn trên thế giới nói chung \& ở châu Á nói riêng.''

\subsection{\textit{Rừng Na Uy} -- Haruki Murakami}
``Hơn 3 thập kỷ qua, \textit{Rừng Na Uy} vẫn là cuốn tiểu thuyết được nhắc đến nhiều nhất của Haruki Murakami. Dù gia tài sáng tác của nhà văn người Nhật bản vẫn dày lên theo năm tháng, nhưng mỗi khi nhắc tới ông, người đọc liền nhớ về câu chuyện đầy khắc khoải của anh chàng Toru Watanbe. Nỗi buồn, sự cô độc \& chán chường của tuổi trẻ đã được nhà văn xứ phù tang khắc họa thật tinh tế.

3 nhân vật Toru, Naoko \& Midori đều khao khát yêu thương. Ái tình giống như ngọn lửa, giúp họ xua tan bóng tối của sự cô đơn đang ngự trị nơi tâm hồn.

Trong \textit{Rừng Na Uy}, tình duc được sử dụng như 1 thứ ngôn ngữ để biểu đạt tình yêu. Haruki Murakami đã có 1 cái nhìn thẳng thắn khi viết về người trẻ. Những con người đang bế tắc \& luôn muốn tìm cho mình 1 lối thoát.

Bên cạnh đó, \textit{Rừng Na Uy} cũng chứng minh được khả năng làm chủ \& kiểm soát ngôn từ 1 cách tinh tế của nhà văn người Nhật Bản. Tác phẩm cho độc giả được thưởng thức sự biến hóa kỳ ảo của ngôn từ, trần trụi, đôi khi có phần sắc lạnh, nhưng có lúc, lại dịu dàng \& đẹp như 1 bài thơ. Không chỉ nổi tiếng ở Nhật, cuốn tiểu thuyết này đã gây tiếng vang trên toàn châu Á.

\textsf{Fig. Tiểu thuyết \textit{Rừng Na Uy} đã được đạo diễn Trần Anh Hùng chuyển thể thành phim.}

\subsection{\textit{3 Chàng Ngốc} -- Chetan Bhagat}
``Tiểu thuyết \textit{3 chàng ngốc} đã mang cái tên Chetan Bhagat đến gần hơn với độc giả châu Á. Cuốn tiểu thuyết là câu chuyện dí dỏm \& hài hước về 3 anh chàng: Hari, Ryan \& Alok.

Vừa thi đỗ vào Học viện Kỹ Thuật Ấn Độ (IIT), các tân sinh viên vô cùng hào hứng với cuộc sống mới. Họ cho rằng đây chính là quãng thời gian tươi đẹp để tận hưởng tuổi trẻ. Thế nhưng, cuộc đời vốn không đẹp như những giấc mơ.

Kết quả học tập thảm hại, chuyện tình yêu không như mong muốn, cùng những rắc rối ập đến 1 cách bất ngờ khiến 3 chàng trai hiểu ra rằng không thể nhìn cuộc đời 1 cách đơn giản.

Tuổi trẻ là quãng thời gian để rèn luyện cho bản thân thật kiên cường. Nhưng quan trọng hơn hết vẫn là việc được sống cuộc đời mình mơ ước.

Khác với văn phong trầm lắng \& dịu dàng của Haruki Murakami, Chetan Bhagat luôn mang tới cho độc giả cuốn tiểu thuyết hài hước \& dí dỏm, với lối dẫn truyện đầy tự nhiên. Thưởng thức tác phẩm của anh, người đọc có cảm giác như được nghe 1 anh chàng vui tính kể câu chuyện đời mình.''

\subsection{\textit{Chai Thời Gian} -- Prabhassorn Sevikul}
``Cùng \textit{Nghiệt duyên} của Thommayanti, \textit{Chai thời gian} là 1 trong số ít những cuốn tiểu thuyết được nhiều thế hệ thanh thiếu niên Thái Lan yêu thích. Tác phẩm này đã được tái bản hơn 20 lần trong 10 năm.

Bối cảnh của câu chuyện là những năm 1970, nhiều thanh niên Thái Lan đã phải trưởng thành trong bất ổn không lường của thời cuộc.

Nhân vật chính của tác phẩm là Nat, thiếu niên lớn lên trong gia đình không hạnh phúc. Cậu hiểu rằng bản thân mình phải trở thành chỗ dựa cho em gái tội nghiệp. Tình bạn đã trở thành điểm tựa cho những chàng trai, cô gái trước giông bão của cuộc đời. Họ học cách kiên cường vượt qua những thử thách của số phận.

Trong \textit{Chai thời gian}, niềm vui \& nỗi buồn đan xen được 1 cách hài hòa. Nụ cười \& những giọt nước mắt của các nhân vật trong tác phẩm giúp người đọc cảm nhận vẻ đẹp đa sắc của tuổi trẻ.''

\subsection{\textit{Cá Thu} -- Gong Ji Yong}
``\textit{Cá thu} là bức tranh sống động về cuộc sống, lý tưởng \& tình yêu của lớp thanh niên Hàn Quốc những năm 80 của thế kỷ trước. Khi đó, các tập đoàn tư bản bắt đầu bành trướng nền kinh tế \& tạo ảnh hưởng đến chính trị.

Những cuộc đình công, bãi công \& các phòng trào biểu tình của học sinh, sinh viên Hàn Quốc diễn ra ở khắp nơi. Đấu tranh hay thỏa hiệp, 2 con đường đó khiến số phận của bao người thay đổi.

Nhân vật trung tâm của cuốn tiểu thuyết là No Eun Rim, cô gái thông minh nhưng yếu đuối. Giữa thời kỳ đầy biến động, Eun Rim chỉ biết sống 1 cách bản năng \& đấu tranh hết mình cho lý tưởng mà cô đã chọn.

Dù đã có gia đình, cô lại đem lòng yêu Myeong Woo. Tình yêu lầm lạc ấy được cô gái trẻ tôn thờ, nhưng cuối cùng, thứ mà cô nhận được chỉ là sự phản bội.

Gong Ji Yong là tác giả đương đại, được yêu thích ở Hàn Quốc. Bà biến những cuốn tiểu thuyết của mình thành tiếng nói phản biện xã hội cảm động \& sâu sắc.

\textit{Cá thu} không chỉ là cuốn tiểu thuyết, mà còn là 1 phần ký ức tuổi trẻ của Gong Ji Yong. Vào những năm 80, nhà văn này đã tham gia nhiều phong trào sinh viên của trường Đại học Yonsei, nơi bà từng theo học.'' -- Thụy Oanh, Sep 1, 2020

%-----------------------------------------------------------------------------%

\section{\href{https://zingnews.vn/hanh-trinh-tu-hoc-cua-tac-gia-rosie-nguyen-post1251308.html}{Zing News\texttt{/}Hành Trình Tự Học của Tác Giả Rosie Nguyễn}}
``\textbf{Sau thành công của ``Tuổi trẻ đáng giá bao nhiêu'', Rosie Nguyễn lên đường du học. Cô nhìn lại quá trình học tập của mình trong cuốn ``Trên hành trình tự học''.} Rosie Nguyễn là tác giả được nhiều bạn trẻ yêu thích. Cho đến cuối năm 2020, riêng cuốn \textit{Tuổi trẻ đáng giá bao nhiêu} của cô đã bán ra thị trường 350,000 bản. Các sách khác của nữ tác giả như \textit{Ta balo trên đất Á, Mình nói gì khi nói về hạnh phúc?} đều có con số phát hành ấn tượng.

Hiện, cô theo học thạc sĩ nghiên cứu ngành truyeen thông báo chí tại Đại học Wisconsin, Mỹ. Chủ đề tự học là niềm hứng thú của Rosie.

Ngay từ \textit{Tuổi trẻ đáng giá bao nhiêu}, Rosie Nguyễn đã nhắc về việc tự học. Nhưng cho đến nay, những suy ngẫm về hành trình tự giáo dục mới được cô chắt lọc, trở thành chủ đề bao trùm cuốn sách \textit{Trên hành trình tự học}.''

\subsection{``Xấu hổ với các bậc tiền bối, tôi bước vào hành trình tự học''}
``Trong buổi giao lưu qua mạng cùng bạn đọc tối 14.8, tác giả Rosie Nguyễn chia sẻ về nguồn cảm hứng khiến cô bắt đầu hành trình tự học. Điểm khởi đầu của hành trình ấy bắt nguồn từ khoảng thời gian mà cô cho là khó khăn, lạc lối của tuổi trẻ.

Ra trường, Rosie Nguyễn có việc làm văn phòng trong 1 khu công nghiệp. Hàng ngày, cô dậy từ 5h sáng, chuẩn bị rồi lên xe bus đi làm, 19h mới về. Rosie bắt đầu cuộc sống của người trưởng thành, không thể tránh khỏi vòng xoáy cơm áo gạo tiền.

Sau 1 thời gian như vậy, cô thấy cuộc sống của mình bị bó hẹp trong 4 bức tường văn phòng \& 4 bức tường ngôi nhà mình ở. 1 đồng nghiệp hơn tuổi đã nói với Rosie Nguyễn về cuộc sống: Ra trường đi làm, tiết kiệm tiền, mua chung cư, cưới chồng, sinh con, ổn định cuộc sống.

``Khi nghe chị đồng nghiệp nói vậy, tôi thấy không thỏa mãn, tự nhủ cuộc sống này chỉ lặp đi lặp lại như vậy thôi sao?'', Rosie Nguyễn kể.

Có những người làm 1 công việc trong 10--20 năm vẫn hạnh phúc, nhưng Rosie Nguyễn không phải như vậy. Cô thấy \fbox{nếu lặp đi lặp lại 1 việc làm, 1 vòng ổn định, bản thân sẽ già đi mà không lớn lên}. Cô nhìn xung quanh, thấy nhiều bạn học xong đi làm thì vứt hết sách vở để lao vào kiếm tiền \& vòng quay cuộc sống.

``Tùy bản thân, mỗi người có những mục đích khác nhau. Tôi không muốn quay cuồng để tìm kiếm tiền. Tôi thấy mình phải tìm 1 lối thoát khác, kiếm tìm những điều mà bản thân mình cho là hạnh phúc'', Rosie Nguyễn kể lại.

Cô bắt đầu đọc lại những cuốn sách như \textit{Tôi tự học} (Thu Giang, Nguyễn Duy Cần), \textit{Tự học -- 1 nhu cầu thời đại} (Nguyễn Hiến Lê), các sách học làm người của Hoàng Xuân Việt $\ldots$

Đọc sách của các học giả, Rosie Nguyễn nhận thấy bậc tiền bối ít có điều kiện học tập như mình, không có Internet, thiếu công cụ tra cứu, nhưng kiến thức của họ đầy đặn, thông kim bác cổ, hiểu Đông, Tây.

``Nhìn lại tôi xấu hổ vì mình không biết bên ngoài, không hiểu rõ nơi mình sinh ra, văn hóa nước mình. Lúc đố, tôi mới bước vào hành trình tự học'', nữ tác giả tâm sự.

\textit{Khuyến học} -- cuốn sách gối đầu giường của người Nhật 1 thời -- cũng là sách gối đầu giường trong hành trình tự học của Rosie. Trong cuốn sách ấy, Rosie tâm đắc với luận điểm của tác giả Fukuzawa Yukichi, đại ý Nhật Bản khi ấy có nhiều kiểu thanh niên. Trong đó, thanh niên đi làm kiếm tiền, lấy vợ, xây tổ ấm được nhiều người ngưỡng mộ. Nhưng nếu chỉ như vậy thì đến con sâu, cái kiến cũng tự kiếm ăn, tự xây tổ của mình. Tác giả khuyến khích mỗi người tự vươn lên, hướng tới cái cao đẹp trong mình.

Đó là 1 trong những động lực để Rosie Nguyễn bước vào hành trình tự học, tự giáo dục để phát triển bản thân.

Hành trình đó biến Rosie Nguyễn từ nhân viên văn phòng sang tác giả sách, rồi làm công việc phát triển thanh niên, giờ đây là tự học, \& nghiên cứu.

Không thể biết tương lai hành trình này sẽ đưa Rosie đến đâu, làm nghề gì, cô luôn tin khi có kỹ năng tự học tốt, có thể sống được cuộc đời mình mơ ước.''

\subsection{Vươn Tới Điều Đẹp Đẽ Bên Trong Mình}
``\textit{Trên hành trình tự học} không phải là cẩm nang về tự học. Nó không có chỉ dẫn, các bước 1,2,3,4 để tự học. Sách là những chiêm nghiệm, phản tư của Rosie Nguyễn về cốt lõi của tự giáo dục.

``Tôi viết quyển sách này muốn mở ra 1 không gian, diễn đàn, 1 khu vườn nhỏ để người thích tự học thì cùng nhau bàn luận'', tác giả nói.

Cuốn sách có cấu trúc 4 phần: Học để biết, Học để làm, Học để chuyển mình, Học để chung sống. Mỗi phần đều được viết dựa trên kinh nghiệm của tác giả \& từ những câu chuyện của những người bạn mà tác giả có dịp gặp gỡ, chia sẻ.

Rosie Nguyễn cùng bàn về các khía cạnh khác nhau của sự học: Cách học trực tuyến, học từ trường, từ gia đình, $\ldots$

Tác giả cùng đưa ra nhận định tổng quan về giá trị chung của việc học.

Trong thời gian viết sách, Rosie Nguyễn tìm hiểu \& tâm đắc với triết lý ``Giáo dục toàn diện''.

\fbox{Triết lý Giáo dục toàn diện cho rằng mỗi người có thể tìm thấy căn tính bản thân mình thông qua kết nối}: Kết nối với bản thân mình, kết nối với cộng đồng xung quanh mình, kết nối với thế giới tự nhiên, kết nối với các giá trị nhân văn (như lòng trắc ẩn, tình yêu thương, tri thức) $\ldots$

Triết lý giáo dục này hướng tới niềm say mê với cuộc sống, tri thức \& học hỏi. Mục đích của nó là trở thành những gì mà 1 con người mong muốn trở thành, phát triển sức khỏe, cảm xúc, sáng tạo, $\ldots$

Triết lý này giúp cho mỗi người tự khai sáng bản thân mình, thấy được ánh sáng, niềm tin, cảm nhận được sự kết nối giữa mình với mình, mình với thế giới bên ngoài, vươn tới khao khát cao cả của con người.

``Tự học cho ta công cụ để vươn tới phần đẹp đẽ bên trong mình. Viết cuốn sách này là 1 phần trải nghiệm đó. Tôi cảm thấy mình đã có được niềm vui, 1 phần quả ngọt của hành trình tự học đó'', Rosie Nguyễn thổ lộ.

Qua cuốn sách, Rosie Nguyễn gửi gắm thông điệp về học tập suốt đời, xem cuộc sống là trường học vĩ đại, nơi mỗi trải nghiệm là 1 cơ hội học hỏi để không ngừng lớn lên, trở thành phiên bản tốt hơn của chính mình.

Tác giả nói: ``Khi viết cuốn sách này, tôi vẫn là 1 người đang trong hành trình tự học. Có những lúc tôi thấy mình mất động lực, học chưa đủ, làm chưa đủ $\ldots$ đó đều là phần tự nhiên của quá trình tự học mà thôi''.'' -- Y Nguyên, Aug 17, 2021

%-----------------------------------------------------------------------------%

\section{\href{https://zingnews.vn/khuc-tu-tinh-cua-nhung-nguoi-tre-bat-an-post1322529.html}{Zing News\texttt{/}Khúc Tự Tình Của Những Người Trẻ Bất An}}
``\textbf{Tuổi thanh xuân được Nguyễn Phương Văn cảm nhận bằng nhãn quan khác biệt. Khi đôi mươi, người ta có hoài bão, nhiệt huyết \& tình yêu nhưng phải đối mặt với đầy rẫy áp lực.} Tập truyện \textit{Mặt trời trong suối lạnh} giống như 1 bức tranh về cuộc sống của những người trẻ. Ở đó, gam màu trầm đóng vai trò chủ đạo. Xuyên suốt tác phẩm là những câu chuyện được kể bằng giọng văn nhẹ nhàng, đầy tình tứ nhưng man mác buồn \& trĩu nặng suy tư. Đằng sau những trang viết đầy chất thơ ấy là \fbox{tâm sự của bạn trẻ đang loay hoay tìm cách khẳng định chính mình}.''

\subsection{Tìm Chính Mình Trong Thế Giới Rộng Lớn}
``Nguyễn Phương Văn đã mang đến cho người đọc 1 tập truyện dung dị, nhưng đầy cảm xúc. Tác giả không cầu kỳ trong cách lựa chọn nhân vật \& xây dựng tình huống truyện. Các sáng tác đều được viết bằng lối văn tự nhiên, với câu từ đơn giản nhưng giàu cảm xúc, khiến người đọc có cảm giác đang ngồi kề bên nhân vật, lắng nghe họ chuyện trò.

Các nhân vật của Nguyễn Phương Văn mộc mạc \& bình dị như vừa bước ra từ đời thực. Đó có thể là 1 cô gái tuổi đôi mươi, hoang mang không biết làm gì với tương lai, nghe theo sự sắp đặt của gia đình, hay cố sức vượt khỏi sự an bài đầy yên ổn để theo đuổi hoài bão của bản thân.

Những gì mà nhân vật trải qua trong tác phẩm là cảm giác đời thường mà chính mỗi người trong chúng ta từng gặp phải, đó là sự bất lực khi phải từ bỏ ước mơ, cảm giác bế tắc khi loay hoay giữa các lựa chọn, không biết phải làm gì để có được kết quả tốt nhất.

Trong cuộc sống, đúng \& sai đôi khi thật mơ hồ, nó không rõ ràng giống như phép toán trong bài kiểm tra. Trước khi thử, người ta mường tượng ra kết quả tốt nhất, nhưng thực tế không dễ dàng như vậy. Sẽ có lúc, người đọc cảm thấy nuối tiếc cho nhân vật, bởi họ đã kiên trì theo đuổi giấc mơ nhưng kết quả họ nhận được chỉ là thất bại mà thôi.

Tác giả khá táo bạo khi cho nhân vật sống thật với những cảm xúc của mình. Thay vì ``cổ tích hóa'' cuộc sống, Nguyễn Phương Văn lại mang tới cho người đọc những câu chuyện trần trụi \& khốc liệt về hành trình trưởng thành của người trẻ. Trong cuốn sách này, đã có những nhân vật gục ngã \& quyết định buông bỏ cuộc sống.

Họ tìm tới cái chết, coi đó như lối thoát duy nhất cho những vấn đề mà mình gặp phải. Trải qua khoảnh khắc cận kề sinh tử, họ dần nhận ra giá trị của cuộc sống. Với tác giả, \fbox{trưởng thành giống như đi bộ đường trường}, hành trang quan trọng nhất chính là sự kiên nhẫn.

\fbox{Khi vừa trải qua 1 biến cố lớn, muốn lấy lại sự cân bằng cần phải có thời gian.} Hiểu được điều đó, nên tác giả tôn trọng những cảm xúc rất con người của mỗi nhân vật \& đưa chúng vào trong các truyện ngắn 1 cách trọn vẹn.''

\subsection{Bản Hòa Âm Mang Hơi Thở Cuộc Sống}
``\textit{Mặt trời trong suối lạnh} đem đến những cảm nhận đa chiều về người trẻ trong những năm tháng đầy biến động. Trong tập truyện là sự hiện diện \& đan cài của hàng loạt cảm xúc khác nhau: sợ hãi, do dự, mông lung, cố chấp, $\ldots$ Trải qua hết những hỉ nộ rất đời thường ấy, người ta nhận ra mình đã trưởng thành.

Các nhận vật: Giang, Zi, Quyên, Sang, $\ldots$ không có ai hoàn hảo. Mỗi người đều có những khiếm khuyết riêng, với những vấn đề khác nhau trong quá trình hoàn thiện bản thân \& tìm thấy ý nghĩa của cuộc đời.

Chủ đề xuyên suốt của tập truyện là hành trình khẳng định bản thân \& đi tìm ý nghĩa cuộc sống của những người trẻ. Nhưng tác giả khá tài tình khi xây dựng 1 hệ thống nhân vật đa sắc, có cá tính riêng, khiến cho mỗi truyện ngắn mang 1 màu sắc riêng biệt.

Viết về tình yêu, Nguyễn Phương Văn đã để lại trong lòng người đọc nhiều nuối tiếc. Các mối tính trong tập truyện này đều dang dở \& kết thúc trong sự day dứt của nhân vật. Những năm tháng tuổi trẻ, người ta dễ phải lòng nhau \& hay mơ mộng, nhưng mấy ai đủ kiên trì \& bao dung để nuôi dưỡng tình yêu tới ngày đơm hoa kết trái.

Phần lớn sáng tác trong tập truyện này được lấy bối cảnh vào cuối những năm 1990, giới trẻ khi ấy chìm đắm trong những giai điệu âm nhạc của Âu Mỹ, đi đâu người ta cũng có thể nghe thấy những câu từ đầy da diết của U2. Nỗi hoài nhớ quá khứ được tác giả gợi lên tinh tế qua những tình tiết nhỏ.

3 truyện ngắn \textit{Em xóa nó đi, được không?, Thức giấc nghe nước xa chảy xiết, Sài Gòn 68} \& tiểu thuyết mini \textit{3 mùa yêu} có sự kết nối tự nhiên, tạo nên 1 chỉnh thể thống nhất cho cuốn sách.

Nguyễn Phương Văn đã kết nối các nhân vật của mình 1 cách tài tình. Nhân vật phụ của truyện ngắn này, lại trở thành nhân vật chính của 1 truyện ngắn khác. Các yếu tố này tạo nên chuỗi móc xích, kết nối các sáng tác trong tác phẩm, khiến chúng không rời rạc mà nâng đỡ cho nhau.

Bên cạnh đó, tập truyện này còn hấp dẫn người đọc ở lối viết tình cảm, giàu chất thơ. Ngoài cốt truyện lôi cuốn, tác giả còn phối hợp các yếu tố liên văn bản để tạo sự mới lạ, hấp dẫn cho tác phẩm. Hiện thực \& những suy tưởng của nhận vật được đan xen 1 cách hợp lý khiến cho tác phẩm không nhàm chán.'' -- Thụy Oanh, Jun 9, 2022

%-----------------------------------------------------------------------------%

\selectlanguage{english}
\part{Scientific\texttt{/}Mathematical Writings}

\chapter{Luc Tartar's Writing Styles}

%-----------------------------------------------------------------------------%

\chapter{Terence Tao\texttt{/}\href{https://terrytao.wordpress.com/advice-on-writing-papers/}{On Writing}}
\begin{quotation}
	``There are three rules for writing the novel. Unfortunately, no one knows what they are.'' -- W. Somerset Maugham
\end{quotation}
``Everyone has to \href{https://terrytao.wordpress.com/advice-on-writing-papers/write-in-your-own-voice/}{develop their own writing style}, based on their own strengths \& weaknesses, on the subject matter, on the target audience, \& sometimes on the target medium. As such, it is virtually impossible to prescribe rigid rules for writing that encompass all conceivable situations \& styles.

Nevertheless, I do have some general advice on these topics:
\begin{itemize}
	\item Writing a paper
	\begin{itemize}
		\item ``Use the introduction to \href{https://terrytao.wordpress.com/advice-on-writing-papers/use-the-introduction-to-%E2%80%9Csell%E2%80%9D-the-key-points-of-your-paper/}{``sell'' the key points of your paper}; the results should \href{https://terrytao.wordpress.com/advice-on-writing-papers/describe-the-results-accurately/}{be described accurately}. One should also invest some effort in both \href{https://terrytao.wordpress.com/advice-on-writing-papers/organise-the-paper/}{organizing} \& \href{https://terrytao.wordpress.com/advice-on-writing-papers/motivate-the-paper/}{motivating} the paper, \& in particular in \href{https://terrytao.wordpress.com/advice-on-writing-papers/use-good-notation/}{selecting good notation} \& \href{https://terrytao.wordpress.com/advice-on-writing-papers/give-appropriate-amounts-of-detail/}{giving appropriate amounts of detail}. But \href{https://terrytao.wordpress.com/advice-on-writing-papers/dont-overoptimise/}{one should not over-optimize} the paper.
		\item It also assists readability if you factor the paper into smaller pieces, e.g., by \href{https://terrytao.wordpress.com/advice-on-writing-papers/create-lemmas/}{making plenty of lemmas}.
		\item To reduce the time needed to write \& organize a paper, I recommend \href{https://terrytao.wordpress.com/advice-on-writing-papers/write-a-rapid-prototype-first/}{writing a rapid prototype 1st}.
		\item For 1st time authors especially, it is important to try to \href{https://terrytao.wordpress.com/advice-on-writing-papers/write-professionally/}{write professionally}, \& in \href{https://terrytao.wordpress.com/advice-on-writing-papers/write-in-your-own-voice/}{one's own voice}. One should \href{https://terrytao.wordpress.com/advice-on-writing-papers/take-advantage-of-the-english-language/}{take advantage of the English language}, \& not just rely purely on mathematical symbols.
		\item The \href{https://terrytao.wordpress.com/advice-on-writing-papers/maximising-the-results-to-effort-ratio/}{ratio between results \& effort in one's paper should be at a local maximum}.
	\end{itemize}
	\item Submitting a paper
	\begin{itemize}
		\item \href{https://terrytao.wordpress.com/advice-on-writing-papers/proofread-and-double-check-your-paper-before-submission/}{Proofread \& double-check your article before submission}; you should be \href{https://terrytao.wordpress.com/advice-on-writing-papers/submit-a-final-draft-not-a-first-draft/}{submitting a final draft, not a 1st draft}
		\item \href{https://terrytao.wordpress.com/advice-on-writing-papers/submit-to-an-appropriate-journal/}{Subset to an appropriate journal}
	\end{itemize}
\end{itemize}
I should point out, of course, that my own writing style is not perfect, \& I myself don't always adhere to the above rules, often to my own detriment. If some of these suggestions seem too unsuitable for your particular paper, use common sense.

Dual to the art of \textit{writing} a paper well, is the art of \textit{reading} a paper well.\footnote{NQBH: In mathematical notation: \begin{align*}
		\mbox{(Art of writing a paper well)} = \mbox{(Art of reading a paper well)}^\star,\ \mbox{(Art of reading a paper well)} = \mbox{(Art of writing a paper well)}^\star.
\end{align*}}\footnote{NQBH: In linguistic, \textit{reading} \& \textit{writing skills} usually come together, so do \textit{listening} \& \textit{speaking skills}. I.e., if one wants to master 1 of these 4 skills, then that person has to master its companion parallelly:
\begin{align*}
	(\mbox{reading}\land\mbox{writing})\lor(\mbox{speaking}\land\mbox{listening}).
\end{align*}} Here is some commentary of mine on this topic:
\begin{itemize}
	\item \href{https://terrytao.wordpress.com/advice-on-writing-papers/on-compilation-errors-in-mathematical-reading-and-how-to-resolve-them/}{On ``compilation errors'' in mathematical reading, \& how to resolve them}.
	\item \href{https://terrytao.wordpress.com/advice-on-writing-papers/implicit-notational-conventions/}{On the use of implicit mathematical notational conventions to provide contextual clues when reading}.
	\item \href{https://terrytao.wordpress.com/advice-on-writing-papers/on-the-strength-of-theorems/}{On key ``jumps in difficulty'' in a mathematical argument, \& how finding \& understanding them is often key to understanding the argument as a whole}.
	\item \href{https://terrytao.wordpress.com/advice-on-writing-papers/on-local-and-global-errors-in-mathematical-papers-and-how-to-detect-them/}{On ``local'' \& ``global'' errors in mathematical papers, \& how to detect them}.
\end{itemize}
Some further advice on mathematical exposition: [$\ldots$]''

\section{Terence Tao\texttt{/}\href{https://terrytao.wordpress.com/advice-on-writing-papers/describe-the-results-accurately/}{On Writing\texttt{/}Describe the Results Accurately}}
\begin{quotation}
	``10,000 fools proclaim themselves into obscurity, while 1 wise man forgets himself into immortality.'' -- Martin Luther King Jr.
\end{quotation}
``\fbox{A paper should neither understate nor overstate its main results.}

If the main result is very surprising or a substantial breakthrough compared with the previous literature, these facts should be noted (and justified in detail, e.g., by explicit comparison with prior results, examples, \& conjectures).

Conversely, if there are unsatisfactory aspects to the result (e.g., hypotheses too strong, or conclusions a little weaker than expected) these should also be stated honestly \& openly, e.g., ``We do not know if hypothesis H is actually necessary''. Similarly, it is worth noting down any interesting open questions remaining after your result.

If you are using a famous unsolved conjecture to motivate your own work, one should give a candid evaluation of the extent to which your work truly represents progress towards that conjecture, so as to avoid the impression of ``false advertising'' or ``name-dropping''.

If for some reason you need to assert a non-trivial statement without proof or citation, it should be made clear that you are doing so (e.g., ``It can be shown that $\ldots$'' or ``Although we will not need or prove this fact here $\ldots$''), so that the reader does not then hunt through the rest of your paper for the non-existent justification of that statement.

\fbox{Titles of sections should be descriptive} (e.g., ``proof of the decomposition lemma'' or ``An orthogonality argument''), as opposed to uninformative (e.g., ``Step 2'' or ``Some technicalities'').

\section{Terence Tao\texttt{/}\href{https://terrytao.wordpress.com/advice-on-writing-papers/give-appropriate-amounts-of-detail/}{On Writing\texttt{/}Give Appropriate Amounts of Detail}}
\begin{quotation}
	``In presenting a mathematical argument the great thing is to give the educated reader the chance to catch on at once to the momentary point \& take details for granted: his successive mouthfuls should be such as can be swallowed at sight; in case of accidents, or in case he wishes for once to check in detail, he should have only a clearly circumscribed little problem to solve (e.g., to check an identity: 2 trivialities omitted can add up to an impasse). The unpracticed writer, even after the dawn of a conscience, gives him no such chance; before he can spot the point he has to tease his way through a maze of symbols of which not the tiniest suffix can be skipped.'' -- John Littlewood, \textit{``A Mathematician's Miscellany''}
\end{quotation}
A paper should dwell at length (using \href{https://terrytao.wordpress.com/advice-on-writing-papers/take-advantage-of-the-english-language/}{plenty of English}) on the most important, innovative, \& crucial components of the paper, \& be brief on the routine, expected, \& standard components of the paper.

In particular, \fbox{a paper should identity which of its components are the most interesting}. Note that this means interesting to \textit{experts in the field}, \& not just interesting to \textit{yourself}; e.g., if you have just learnt how to prove a standard lemma which is well known to the experts \& already in the literature, this does not mean that you should provide the standard proof of this standard lemma, unless this serves some greater purpose in the paper (e.g., by motivating a less standard lemma).

Conversely, some computations, definitions, or notational conventions which you are very familiar with, but are not widely known in the field, should be expounded on in detail, even if these details are ``obvious'' to you due to your extensive work in this area. Even a brief sentence of explanation is much better than none at all.

For a similar reason, if you are using a relatively obscure lemma from, say, 1 of your own papers, you should not assume that every reader of your current article is intimately familiar with your previous paper. In such cases it is worth stating the lemma in full, with a precise citation (as opposed to casually using phrases e.g., ``by a lemma in [my previous 100-page paper], we have $\ldots$''). When the lemma is particularly crucial, it is sometimes also worth spending a paragraph to sketch out a proof, or to otherwise remark on the significance of this lemma \& its connections to other, more well known results.''

\section{Terence Tao\texttt{/}\href{https://terrytao.wordpress.com/advice-on-writing-papers/take-advantage-of-the-english-language/}{On Writing\texttt{/}Take Advantage of the English Language}}
\begin{quotation}
	``Use soft words \& hard arguments.'' -- Proverbial
\end{quotation}
``\href{https://terrytao.wordpress.com/advice-on-writing-papers/use-good-notation/}{Mathematical notation} is a wonderfully useful tool, \& it can be exciting to learn for the first time the meaning of mysterious \& arcane symbols e.g., $\forall,\exists,\emptyset,\Rightarrow$, etc. However, just because you \textit{can} write statements in purely mathematical notation doesn't mean that you necessarily \textit{should}. In many cases, it is in fact far more informative \& readable to use liberal amounts of plain English; if used correctly \& thoughtfully, the English language can communicate to the reader on many more levels than a mathematical expression, \fbox{without sacrificing any precision or rigor}. In particular, by subtly modulating the emphasis of one's text, one can convey valuable contextual cues as to how a statement interacts with the rest of one's argument.

An example should serve to illustrate this point. Suppose for instance that $P$ \& $Q$ are properties that can apply to mathematical objects $x$ \& $y$. The mathematical statements $P(x)\land Q(y)$ which asserts that $x$ satisfies $P$ \& $y$ satisfies $Q$, is a well-formed \& precise mathematical statement. But there are many possible ways one could express that mathematical statement in English, e.g.,: $\ldots$'' \texttt{[Skip 27 items]}

``From the viewpoint of formal mathematical logic, each of these English statement is logically equivalent to the mathematical sentence $P(x)\land Q(y)$. However, each of the above English statements also provides additional useful \& informative cues for the reader regarding the relative importance, non-triviality, \& causal relationship of the component statements $P(x)$ \& $Q(y)$, or of the component symbols $P,x,Q$, \& $y$. E.g., in some of these sentences $P(x)$ \& $Q(y)$ are given equal importance (being complementary or somehow in opposition to each other), whereas in others $P(x)$ is only an auxiliary statement whose only purpose is to derive $Q(y)$ (or vice versa), \& in yet others, $P(x)$ \& $Q(y)$ are deemed to be analogous, even if one is not formally deducible from the other. In some sentences, it is the objects $x$ \& $y$ which are indicated to be the primary actors; in other sentences, it is the properties $P$ \& $Q$; \& in yet other sentences, it is the combined statements $P(x)$ \& $Q(y)$ which are the most central.

Thus we see that English sentences can be considerably more expressive than their formal mathematical counterparts, while still retaining the precision \& rigor that mathematical exposition demands. By using such humble English words as ``also'', ``but'', ``since'', etc., a sentence conveys not only its semantic content, but also how it is going to fit in with the rest of one's argument (or in the wider theory of the object), giving the reader more insight as to the overall structure of that argument. \fbox{The paper may become slightly longer because of this, but this is a small price to pay for readability} (which is \textit{not} the same as brevity!).

On the other hand, one should \href{https://terrytao.wordpress.com/advice-on-writing-papers/dont-overoptimise/}{not try to excessively ``improve''} the paper by using overly fancy or obscure words (from English or any other language), especially since such words can be mistaken for some sort of technical mathematical terminology. In many cases, one can replace complicated words by plainer equivalents, thus increasing the readability of one’s text without compromising the message. The primary purpose of mathematical writing is to \textit{communicate} \& \textit{inform}, not to \textit{impress}.

Finally, there is 1 situation in which it does make sense to use the terse language of mathematical notation rather than a more leisurely English equivalent, \& that is when you are performing a tedious \& standard formal computation. In those cases, the reader should already know in general terms what is going to happen (especially if you flag the computation as being standard beforehand), \& will only be distracted by superfluous explanation or digression. (See also ``\href{https://terrytao.wordpress.com/advice-on-writing-papers/give-appropriate-amounts-of-detail/}{give appropriate amounts of detail}''.)

Naturellement, la discussion ci-dessus s'applique également à d'autres langues, telles que la langue française.''\footnote{Of course, the above discussion also applies to other languages, such as the French language.}

\section{Terence Tao\texttt{/}\href{https://terrytao.wordpress.com/advice-on-writing-papers/use-good-notation/}{On Writing\texttt{/}Use Good Notation}}
\begin{quotation}
	``By relieving the brain of all unnecessary work, a good notation sets it free to concentrate on more advanced problems, and, in effect, increases the mental power of the race.'' -- Alfred North Whitehead, ``An Introduction to Mathematics''
\end{quotation}
``\fbox{Good notation can make the difference between a readable paper \& an unreadable one.}

Ideally, notation should emphasize the most important parameters \& features of a mathematical expression or statement, while downplaying the routine or uninteresting parameters \& features. For instance, if one does not care much about the exact values of constants in estimates, then notation which conceals these constants (e.g., $\ll$, $\lesssim$, or $O(\cdot)$) are useful; conversely, these notations should be avoided if the precise values of these constants are of importance to the paper.

\fbox{Notation which is used globally should be defined in a notation section near the front of the paper, or in the introduction;} notation which is only used locally (e.g., within a single section, or within a proof of a single lemma) should be defined close to where it is used (possibly with a reminder that this notation is not used elsewhere in the paper); this is helpful when there are many sections, each with their own extensive notation.

Note that notation or statements which are introduced within a proof of a lemma are already understood to be localized to that lemma; it is bad form to then recall that notation or statement outside of that lemma, except perhaps as a remark or as motivation. In some cases it is worthwhile to define the notation once near the start of the paper, \& then recall it whenever necessary.

One should strive to make one's choices of notation compatible \& consistent with notation already in the literature, so that the readers who are already familiar with prior notation will adapt easily to your paper \& will not be confused.

\fbox{Try to avoid notation which is overly ``cute'' or ``clever''.} This can be distracting or appear \href{https://terrytao.wordpress.com/career-advice/be-professional-in-your-work/}{unprofessional}. In particular, the notation should not be cleverer than the actual substance of the paper.

One should \textbf{definitely} avoid naming new terms after yourself (or after your family members, your pets, etc.), for the obvious reasons. If other authors name the concepts you introduce after yourself, \& that appellation becomes common usage, then you may use that term as well, but in all other cases it gives the rather \fbox{blatant impression of vanity or narcissism}.

There is an issue of where to strike the balance between too little notation \& too much notation. A good rule of thumb is that any expression or concept which is used 3 or more times will probably benefit from introducing some notation to capture that expression or concept; conversely, an expression which is only used once probably does not need its own special notation. (An exception would be for particularly crucial theorems or propositions in the paper; here it might be worthwhile to invest in some notation in order to make the statement of those theorems clean \& readable. Conversely, if an expression only appears in multiple locations of the paper because of coincidences of no significance, then it may be better to avoid introducing notation that gives the false impression of a connection between these appearances.)

If one needs to name a certain property or class of objects, one should generally use very bland names (e.g., ``good'', ``bad'', ``Type I'', ``Type II'', etc.) for peripheral or technical terms; colorful terms should be used sparingly, \& only for those concepts that are quite central to the paper, lest they distract from the main points of that paper. (This is analogous to how, in film \& literature, the main characters generally tend to have more memorable names than the secondary ones.)

Sometimes one is unsure what notation to use for a particular concept, because of potential conflicts with other notation in other (as yet unwritten) parts of a paper. One solution here is to introduce a \TeX\ \href{http://en.wikipedia.org/wiki/Macro}{macro} for that notation, \& force yourself to use that macro exclusively whenever that notation is used. (E.g., if you have a group which you are tentatively naming $G$, you could define a macro \verb|\grp| that is set to G, \& use \verb|\grp| instead of G throughout the paper.) That way, if you find a notational conflict later on (e.g., if you discover that you really need G to denote a graph instead), then you only need to change \textit{1 line} in your \TeX\ file -- the line that defines the macro -- to resolve the notational conflict, rather than to do a tedious (and error-prone) search-and-replace.

For any rigorous component of the paper, the notation should be precise \& unambiguous (and for non-rigorous components, ambiguous notation should be pointed out with ``scare quotes'' or other cautionary phrases such as ``roughly speaking'' or ``essentially''). A certain amount of abuse of notation is permitted, though, as long as this is properly pointed out.'' \texttt{[Skip the common example of \textit{division}, i.e., $a/bc$ means either $(a/b)c$  or $a/(bc)$; or use $\frac{a}{b}c$ \& $\frac{a}{bc}$ instead].}

``It is also worthwhile to quietly reinforce one's notational conventions when given the opportunity. E.g., suppose in one's argument one has a vector space, which one has decided to call $V$. When referring back to this object, one could say ``the vector space'', or ``V'', but if the reader does not remember what vector space is being discussed, or what $V$ is, the reader will have to take a minute or so to flip back \& figure this out. But if instead you refer to this object consistently as ``the vector space $V$'', then the notational convention is reinforced, \& the reader can continue reading without breaking rhythm. (One can also modulate the choice of terminology used here to emphasize different aspects of the object being referred to. If e.g., it is the additive structure of $V$ which is currently relevant, you can instead say ``the additive group $V$''; if, later, it is the topological structure which is the most important, one can say ``the topological vector space $V$'', \& so forth. This allows one to subtly draw attention to the most important features of the object under consideration, without distracting the reader from the main body of the argument.)''

See also \href{https://mathoverflow.net/questions/366070/what-are-the-benefits-of-writing-vector-inner-products-as-langle-u-v-rangle/366118#366118}{Terence Tao's answer to MathOverflow question: What are the benefits of writing vector inner products as $\langle{\bf u},{\bf v}\rangle$ as opposed to ${\bf u}^\top{\bf v}$?}

\section{Terence Tao\texttt{/}\href{https://terrytao.wordpress.com/advice-on-writing-papers/write-in-your-own-voice/}{On Writing\texttt{/}Write in Your Own Voice}}

\begin{quotation}
	``While one should always study the method of a great artist, one should never imitate his manner. The manner of an artist is essentially individual, the method of an artist is absolutely universal. The first is personality, which no one should copy; the second is perfection, which all should aim at.'' -- Oscar Wilde, \textit{A Critic in Pall Mall}, p. 195
\end{quotation}
``When, as a graduate student, one is starting out one's research in a mathematical subject, one usually begins by reading the papers of the current \& past leaders of the field. Initially, one's understanding of the subject is fairly limited, \& so it is natural to view these papers as being authoritative, especially if their authors are well-known.

Eventually, though, one requires a fair fraction of the insights \& understanding conveyed by the existing literature, \& is able to apply it to produce a new result or observation that goes beyond that literature (or, at least, makes explicit what was only implicitly buried in previous papers). When the ramifications \& extensions of these new advances have been explored to their natural extent, it then becomes time to write up these results as a research paper.

Of course, as your work is almost certainly based in part on the previous literature, one should cite that literature whenever appropriate, \& compare \& contrast your own work with that literature in an \href{https://terrytao.wordpress.com/advice-on-writing-papers/describe-the-results-accurately/}{accurate}, \href{https://terrytao.wordpress.com/advice-on-writing-papers/write-professionally/}{professional}, \& informative manner. Also, one should try to \href{https://terrytao.wordpress.com/advice-on-writing-papers/use-good-notation/}{maintain some level of notational consistency} with the previous literature, such as using the same fundamental definitions \& to use similar notation, so that expert readers who are already familiar with that literature can quickly get up to speed on your work. And if 1 of the arguments in your work is standard in the literature, it certainly makes sense to structure the argument in a standard fashion if possible, again to assist the experts reading your paper.

\textbf{However}, one should \textbf{not} go so far as to copy entire paragraphs or more of text from a prior paper, except when used sa a direct quotation to illustrate some historical point. First of all, if one does not properly attribute that text (e.g., ``As Bourbaki [17, p. 146] writes,'', or, for that matter, the Oscar Wilde quote above), then one runs the risk of committing \href{http://en.wikipedia.org/wiki/Plagiarism}{plagiarism}. But even if the text is properly attributed, copying the text verbatim, without updating it to reflect more recent developments (including that in the paper being written) \& to add your own simplifications \& insights, is a redundant waste of space \& a lost opportunity to advance the subject. If one is tempted to copy a significant portion of text from a prior reference without adding anything significantly new, one should instead simply cite the previous reference appropriately, e.g., ``See [27, Section 4] for further discussion.'' or ``A proof can be found in [9, Lemma 2.4].'' (cf. ``\href{https://terrytao.wordpress.com/advice-on-writing-papers/give-appropriate-amounts-of-detail/}{Give appropriate amounts of details}'').

Of course, there \textit{are} reasons to duplicate to some extent some discussion or argument that was present in a previous paper:
\begin{itemize}
	\item As mentioned earlier, one may wish to make some historical point, e.g., to track the development of a mathematical idea over time.
	\item If the paper is obscure \& not widely available, reproducing a key argument from that paper may serve as a convenience to the reader.
	\item Also, if the \textit{form} of that argument can be used to \href{https://terrytao.wordpress.com/advice-on-writing-papers/motivate-the-paper/}{motivate} other arguments in your paper, then it can be worth putting in that argument so that it can be referred to later in the paper.
	\item The precise result needed for your paper may differ slightly from what is already established in the literature, \& so one needs to either write out a modified version of the proof, or else point to the original proof but indicate what modifications need to be made. (The latter is suitable if the changes are particularly minor in nature.)
	\item The existing paper may have an argument which can be updated, simplified, modernized, or otherwise improved thanks to more recent advances or insights in the area (including your own). It can then be a service to the field to place an updated version of the argument in the literature (with full citations to the paper containing the original argument, of course).
\end{itemize}
However, when one is not simply quoting the prior text for historical or archival purposes, it is best to \textit{paraphrase} \& \textit{interpret} the previous text rather than to copy that text verbatim. This is for a number of reasons:
\begin{itemize}
	\item One wants to avoid conveying any impression to readers, referees, or editors of plagiarism, padding, or intellectual laziness in one's papers. (Note that the latter is a danger even if one is copying from one's own work, rather than that of others.)
	\item The prior work may be dated in view of more recent developments \& insights, as mentioned above.
	\item If you are copying or adapted a piece of text from another author that you do not fully understand yourself, then it may end up being inappropriate or incongruous for your intended purpose, \& may convey the impression of superficiality or being ill-informed. If the text becomes inaccurate due to this adaptation, then this can also cause some embarrassment \& annoyance for the original author of that text.
	\item Excessive use of quotation from famous mathematicians to make one's own work look more impressive is the mathematical equivalent of name-dropping, \& should be avoided. Appeal to authority should not be the primary basis for motivating a paper; a handful of citations to demonstrate the depth of interest in the problem being studied is usually sufficient.
	\item \textit{But most importantly of all, for one's further mathematical development \& career, one needs to develop one's own consistent mathematical ``voice'' \& style, \& to avoid the impression of simply imitating the voices of other authors}. There is no need in this subject for the mathematical equivalent of a parrot, \& a text which is a mix of the author's voice \& the voice of others can read very strangely.
\end{itemize}
Of course, if one is paraphrasing a previous work, one should cite that work appropriately (e.g., ``The proof here is loosely based on that in [5].'' or ``This discussion is inspired by a related discussion in [10].'').

In some cases, the imitation of a previous author's style \& text is intended as a sign of respect or flattery for that author. \textbf{This is misguided}; an author will in fact often find such mimicry to actually be somewhat offensive. If one wants to truly respect a mathematician, then understand that mathematician's methods, results, \& exposition, \& improve, update, adapt, \& advance all 3. Even the greatest mathematician's contributions should advance with the field, rather than being worshiped \& preserved in some supposed state of perfection; the latter is mostly suitable only for historical purposes.

Another possible reason for copying the style of a more senior mathematician is that one does not yet have the self-confidence to write in one's own style \& voice. While this is justifiable to some extent when one is just starting one's career, it becomes less excusable as one continues one's research. If one is hesitant to state things in one's own fashion, it is perfectly acceptable to couch such text with the appropriate caveats (e.g., ``to the author's knowledge, this observation is new'' or ``While Lemma 2.5 is usually phrased in a topological fashion, we found the following, more geometric, formulation to be more convenient for our applications''). And if one does not feel confident enough in one's understanding of a subject to explain it in any other way than copying from a previous paper, then this should be taken as a sign that one still needs to \href{https://terrytao.wordpress.com/career-advice/learn-and-relearn-your-field/}{internalize the subject futher}.

When writing a paper with 1 or more coauthors, there will inevitably be distinctions in style,\footnote{NQBH: a reasonable justification for loneliness and\texttt{/}in solo (academic) writing.} \& so initially different sections may have sharply different tones due to their being largely written by different subsets of coauthors; but I usually find that after a few rounds of editing, the voices are harmonized into a style which is clearly derived from, but distinct from, each of the individual styles. Ideally, one should understand \& respect the underlying stylistic decisions of one's coauthors, but at the same time be willing to take the initiative \& find ways to formulate the text \& arrangement to smoothly reconcile the coauthor's preferences with one's own; if all goes well, this can lead to a level of exposition \& presentation that is superior to what each of the individual authors could separately achieve. (Of course, if you are to perform major edits on a coauthor's contribution, some consultation with that coauthor is presumably desirable). This process can be quite educational; my own writing style has definitely been influenced in a positive fashion by those of my coauthors.

Developing one's own style is, by definition, a very personal process; while external advice or role models can certainly be of some influence, they are of limited utility after a certain point. But finding an individual style which is comfortable \& effective for both you \& your readers is an important mark of one's \textit{mathematical maturity}, \& is a goal that is definitely worth pursuing.''

%-----------------------------------------------------------------------------%

\appendix

\section*{Quick notes}
``when possible and\texttt{/}or necessary'' -- \cite[p. 47]{Rebollo_Lewandowski2014}

\selectlanguage{vietnamese}

Cấu trúc ``Đấy là [event 1], [event 2] gay cấn hơn'' -- Vũ Hà Văn, \textit{Giáo sư phiêu lưu ký: Tản mạn với 1 nhà toán học}, p. 14

%-----------------------------------------------------------------------------%

\selectlanguage{english}

\begin{thebibliography}{99}
	\bibitem[TerryTao]{TerryTao} \href{https://terrytao.wordpress.com}{Terence Tao's blog}.
	\begin{itemize}
		\item Terence Tao. \href{https://terrytao.wordpress.com/advice-on-writing-papers/}{\textit{On writing}} (or in full: \textit{Advice on writing papers}).
		\begin{itemize}
			\item Terence Tao. \href{https://terrytao.wordpress.com/advice-on-writing-papers/describe-the-results-accurately/}{\textit{On writing}\texttt{/}\textit{Describe the results accurately}}.
			\item Terence Tao. \href{https://terrytao.wordpress.com/advice-on-writing-papers/give-appropriate-amounts-of-detail/}{\textit{On writing}\texttt{/}\textit{Give appropriate amounts of detail}}.
			\item Terence Tao. \href{https://terrytao.wordpress.com/advice-on-writing-papers/use-good-notation/}{\textit{On writing}\texttt{/}\textit{Use good notation}}.
			\item Terence Tao. \href{https://terrytao.wordpress.com/advice-on-writing-papers/write-in-your-own-voice/}{\textit{On writing}\texttt{/}\textit{Write in your own voice}}.
		\end{itemize}
	\end{itemize}
\end{thebibliography}

%-----------------------------------------------------------------------------%

\printbibliography[heading=bibintoc]
	
\end{document}