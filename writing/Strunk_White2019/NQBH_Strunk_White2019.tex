\documentclass{article}
\usepackage[backend=biber,natbib=true,style=authoryear]{biblatex}
\addbibresource{/home/nqbh/reference/bib.bib}
\usepackage{tocloft}
\renewcommand{\cftsecleader}{\cftdotfill{\cftdotsep}}
\usepackage[colorlinks=true,linkcolor=blue,urlcolor=red,citecolor=magenta]{hyperref}
\usepackage{algorithm,algpseudocode,amsmath,amssymb,amsthm,float,graphicx,mathtools}
\allowdisplaybreaks
\numberwithin{equation}{section}
\newtheorem{assumption}{Assumption}[section]
\newtheorem{conjecture}{Conjecture}[section]
\newtheorem{corollary}{Corollary}[section]
\newtheorem{definition}{Definition}[section]
\newtheorem{example}{Example}[section]
\newtheorem{lemma}{Lemma}[section]
\newtheorem{notation}{Notation}[section]
\newtheorem{principle}{Principle}[section]
\newtheorem{problem}{Problem}[section]
\newtheorem{proposition}{Proposition}[section]
\newtheorem{question}{Question}[section]
\newtheorem{remark}{Remark}[section]
\newtheorem{theorem}{Theorem}[section]
\usepackage[left=0.5in,right=0.5in,top=1.5cm,bottom=1.5cm]{geometry}
\usepackage{fancyhdr}
\pagestyle{fancy}
\fancyhf{}
\lhead{\small Sect.~\thesection}
\rhead{\small\nouppercase{\leftmark}}
\renewcommand{\sectionmark}[1]{\markboth{#1}{}}
\cfoot{\thepage}
\def\labelitemii{$\circ$}

\title{The Elements of Style}
\author{William Strunk, Jr. \and E. B. White}
\date{\today}

\begin{document}
\maketitle
\tableofcontents

%------------------------------------------------------------------------------%

\section*{Foreword}

%------------------------------------------------------------------------------%

\section*{Introduction}

%------------------------------------------------------------------------------%

\section{Elementary Rules of Usage}

\subsection{Form the possessive singular of nouns by adding 's.}

%------------------------------------------------------------------------------%

\subsection{In a series of 3 or more terms with a single conjunction, use a comma after each term except the last.}

%------------------------------------------------------------------------------%

\subsection{Enclose parenthetic expressions between commas.}

%------------------------------------------------------------------------------%

\subsection{Place a comma before a conjunction introducing an independent clause.}

%------------------------------------------------------------------------------%

\subsection{Do not join independent clauses with a comma.}

%------------------------------------------------------------------------------%

\subsection{Do not break sentences in 2.}

%------------------------------------------------------------------------------%

\subsection{Use a colon after an independent clause to introduce a list of particulars, an appositive, an amplification, or an illustrative quotation.}

%------------------------------------------------------------------------------%

\subsection{Use a dash to set off an abrupt break or interruption \& to announce a long appositive or summary.}

%------------------------------------------------------------------------------%

\subsection{The number of the subject determines the number of the verb.}

%------------------------------------------------------------------------------%

\subsection{Use the proper case of pronoun.}

%------------------------------------------------------------------------------%

\subsection{A participial phrase at the beginning of a sentence must refer to the grammatical subject.}

%------------------------------------------------------------------------------%

\section{Elementary Principles of Composition}

\subsection{Choose a suitable design \& hold to it.}

%------------------------------------------------------------------------------%

\subsection{Make the paragraph the unit of composition: 1 paragraph to each topic.}

%------------------------------------------------------------------------------%

\subsection{Use the active voice.}

%------------------------------------------------------------------------------%

\subsection{Put statements in positive form.}

%------------------------------------------------------------------------------%

\subsection{Use definite, specific, concrete  language.}

%------------------------------------------------------------------------------%

\subsection{Omit needless words.}

%------------------------------------------------------------------------------%

\subsection{Avoid a succession of loose sentences.}

%------------------------------------------------------------------------------%

\subsection{Express coordinate ideas in similar form.}

%------------------------------------------------------------------------------%

\subsection{Keep related words together.}

%------------------------------------------------------------------------------%

\subsection{In summaries, keep to 1 tense.}

%------------------------------------------------------------------------------%

\subsection{Place the emphatic words of a sentence at the end.}

%------------------------------------------------------------------------------%

\section{A Few Matters of Form}

%------------------------------------------------------------------------------%

\section{Words \& Expressions Commonly Misused}

%------------------------------------------------------------------------------%

\section{An Approach to Style (With a List of Reminders)}

\subsection{Place yourself in the background.}
``Write in a way that draws the reader's attention to the sense \& substance of the writing, rather than to the mood \& temper of the author. If the writing is solid \& good, the mood \& temper of the writer will eventually be revealed \& not at the expense of the work. Therefore, the 1st piece of advice is this: to achieve style, begin by affecting none -- i.e., place yourself in the background. A careful \& honest writer does not need to worry about style. As you become proficient in the use of language, your style will emerge, because you yourself will emerge, \& when this happens you will find it increasingly easy to break through the barriers that separate you from other minds, other hearts -- which is, of course, the purpose of writing, as well as its principal reward. Fortunately, the act of composition, or creation, disciplines the mind; writing is 1 way to go about thinking, \& the practice \& habit of writing not only drain the mind but supply it, too.'' -- \cite[p. 78]{Strunk_White2019}

%------------------------------------------------------------------------------%

\subsection{Write in a way that comes naturally.}
``Write in a way that comes easily \& naturally to you, using words \& phrases that come readily to hand. But do not assume that becaues you have acted naturally your product is without flaw.

The use of language begins with imitation. The infant imitates the sounds made by its parents; the child imitates 1st the spoken language, then the stuff of books. The imitative life continues long after the writer is secure in the language, for it is almost impossible to avoid imitating what one admires. Never imitate consciously, but do not worry about being an imitator; take pains instead to admire what is good. Then when you write in a way that comes naturally, you will echo the halloos that bear repeating.'' -- \cite[p. 79]{Strunk_White2019}

%------------------------------------------------------------------------------%

\subsection{Work from a suitable design.}
``Before beginning to compose something, gauge the nature \& extent of the enterprise \& work from a suitable design. (See Chap. II, Rule 12.) Design informs even the simplest structure, whether of brick \& steel or of prose. You raise a pup tent from 1 sort of vision, a cathedral from another. This does not mean that you must sit with a blueprint always in front of you, merely that you had best anticipate what you are getting into. To compose a laundry list, you can work directly from the pile of soiled garments, ticking them off 1 by 1. By to write a biography, you will need at least a rough scheme; you cannot plunge in blindly \& start ticking off fact after fact about your subject, lest you miss the forest for the trees \& there be no end to your labors.

Sometimes, of course, impulse \& emotion are more compelling than design. If you are deeply troubled \& are composing a letter appealing for mercy or for love, you had best not attempt to organize your emotions; the prose will have a better chance if the emotions are left in disarray -- which you'll probably have to do anyway, since feelings do not usually lend themselves to rearrangement. But even the kind of writing that is essentially adventurous \& impetuous will on examination be found to have a secret plan: Columbus didn't just sail, he sailed west, \& the New World took shape from this simple \&, we now think, sensible design.'' -- \cite[p. 80]{Strunk_White2019}

%------------------------------------------------------------------------------%

\subsection{Write with nouns \& verbs.}
``Write with nouns \& verbs, not with adjectives \& adverbs. The adjective hasn't been built that can pull a weak or inaccurate noun out of a tight place. This is not to disparage adjectives \& adverbs; they are indispensable parts of speech. Occasionally they surprise us with their power, as in
\begin{quotation}\it
	Up the airy mountain,
	
	Down the rushy glen,
	
	We daren't go a-hunting
	
	For fear of little men $\ldots$
\end{quotation}
The nouns \textit{mountain} \& \textit{glen} are accurate enough, but had the mountain not become airy, the glen rushy, William Ailing-ham might never have got off the ground with this poem. In general, however, it is nouns \& verbs, not their assistants, that give good writing its toughness \& color.'' -- \cite[p. 81]{Strunk_White2019}

%------------------------------------------------------------------------------%

\subsection{Revise \& rewrite.}
``Revising is part of writing. Few writers are so expert that they can produce what they are after on the 1st try. Quite often you will discover, on examining the completed work, that there are serious flaws in the arrangement of the material, calling for transpositions. When this is the case, a word processor can save you time \& labor as you rearrange the manuscript. You can select material on your screen \& move it to a more appropriate spot, or, if you cannot find the right spot, you can move the material to the end of the manuscript until you decide whether to delete it. Some writers find that working with a printed copy of the manuscript helps them to visualize the process of change; others prefer to revise entirely on screen. Above all, do not be afraid to experiment with what you have written. Save both the original \& the revised versions; you can always use the computer to restore the manuscript to its original condition, should that course seem best. Remember, it is no sign of weakness or defeat that your manuscript ends up in need of major surgery. This is a common occurrence in all writing, \& among the best writers.'' -- \cite[p. 82]{Strunk_White2019}

%------------------------------------------------------------------------------%

\subsection{Do not overwrite.}
``Rich, ornate prose is hard to digest, generally unwholesome, \& sometimes nauseating. If the sickly-sweet word, the overblown phrase are your natural form of expression, as is sometimes the case, you will have to compensate for it by a show of vigor, \& by writing something as meritorious as the Songs of Songs, which is Solomon's.

When writing with a computer, you must guard against wordiness. The click \& flow of a word processor can be seductive, \& you may find yourself adding a few unnecessary words or even a whole passage just to experience the pleasure of running your fingers over the keyboard \& watching your words appear on the screen. It is always a good idea to reread your writing later \& ruthlessly delete the excess.'' -- \cite[p. 83]{Strunk_White2019}

%------------------------------------------------------------------------------%

\subsection{Do not overstate.}
``When you overstate, readers will be instantly on guard, \& everything that has preceded your overstatement as well as everything that follows it will be suspect in their minds because they have lost confidence in your judgment or your poise. Overstatement is 1 of the common faults. A single overstatement, wherever or however it occurs, diminishes the whole, \& a single carefree superlative has the power to destroy, for readers, the object of your enthusiasm.'' -- \cite[p. 84]{Strunk_White2019}

%------------------------------------------------------------------------------%

\subsection{Avoid the use of qualifiers.}
``\textit{Rather, very, little, pretty} -- these are the leeches that infest the pond of prose, sucking the blood of words. The constant use of the adjective \textit{little} (except to indicate size) is particularly debilitating; we should all try to do a little better, we should all be very watchful of this rule, for it is a rather important one, \& we are pretty sure to violate it now \& then.'' -- \cite[p. 85]{Strunk_White2019}

%------------------------------------------------------------------------------%

\subsection{Do not affect a breezy manner.}
``The volume of writing is enormous, these days, \& much of it has a sort of windiness about it, almost as though the author were in a state of euphoria. ``Spontaneous me,'' say Whitman, \&, in his innocence, let loose the hordes of uninspired scribblers who would 1 day confuse spontaneity with genius.

The breezy style is often the work of an egocentric, the person who imagines that everything that comes to mind is of general interest \& that uninhibited prose creates high spirits \& carries the day. Open any alumni magazine, turn to the class notes, \& you are quite likely to encounter old Spontaneous Me at work -- an aging collegian who writes something like this:
\begin{quotation}\it
	Well, guys, here I am again dishing the dirt about your disorderly classmates, after passing a week in the Big Apple trying to catch the Columbia hoops tilt \& then a cab-ride from hell through the West Side casbah. \& speaking of news, howzabout tossing a few primo items this way?
\end{quotation}
This is an extreme example, but the same wind blows, at lesser velocities, across vast expanses of journalistic prose. The author in this case has managed in 2 sentences to commit most of the unpardonable sins: he obviously has nothing to say, he is showing off \& directing the attention of the reader to himself, he is using slang with neither provocation nor ingenuity, he adopts a patronizing air by throwing in the word \textit{primo}, he is humorless (though full of fun), dull, \& empty. He has not done his work. Compare his opening remarks with the following -- a plunge directly into the news:
\begin{quotation}\it
	Clyde Crawford, who stroked the varsity shell in 1958, is swinging an oar again after a lapse of 40 years. Clyde resigned last spring as executive sales manager of the Indiana Flotex Company \& is now a gondolier in Venice.
\end{quotation}
This, although conventional, is compact, informative, unpretentious. The writer has dug up an item of news \& presented it in a straightforward manner. What the 1st writer tried to accomplish by cutting rhetorical capers \& by breeziness, the 2nd writer managed to achieve by good reporting, by keeping a tight rein on his material, \& by staying out of the act.'' -- \cite[p. 87]{Strunk_White2019}

%------------------------------------------------------------------------------%

\subsection{Use orthodox spelling.}
``In ordinary composition, use orthodox spelling. Do not write \textit{nite} for \textit{night, thru} for \textit{through, pleez} for \textit{please}, unless you plan to introduce a complete system of simplified spelling \& are prepared to take the consequences.

In the original edition of \textit{The Elements of Style}, there was a chapter on spelling. In it, the author had this to say:
\begin{quotation}\it
	The spelling of English words is not fixed \& invariable, nor does it depend on any other authority than general agreement. At the present day there is practically unanimous agreement as to the spelling of most words $\ldots$ At any given moment, however, a relatively small number of words may be spelled in more than 1 way. Gradually, as a rule, 1 of these forms comes to be generally preferred, \& the less customary form comes to look obsolete \& is discarded. From time to time new forms, mostly simplifications, are introduced by innovators, \& either win their place or die of neglect.
	
	The practical objection to unaccepted \& oversimplified spellings is the disfavor with which they are received by the reader. They distract his attention \& exhaust his patience. He reads the form though automatically, without thought of its needless complexity; he reads the abbreviation tho \& mentally supplies the missing letters, at the cost of a fraction of his attention. The writer has defeated his own purposed.
\end{quotation}
The language manages somehow to keep pace with events. A word that has taken hold in our century is \textit{thru-way}; it was born of necessity \& is apparently here to stay. In combination with \textit{way, thru} is more serviceable than \textit{through}; it is a high-speed word for readers who are going 65. \textit{Throughway} would be too long to fit on a road sign, too slow to serve the speeding eye. It is conceivable that because of our thruways, \textit{through} will eventually become \textit{thru} -- after many more thousands of miles of travel.'' -- \cite[p. 88]{Strunk_White2019}

%------------------------------------------------------------------------------%

\subsection{Do not explain too much.}
``It is seldom advisable to tell all. Be sparing, e.g., in the use of adverbs after ``he said,'' ``she replied,'' \& the like: ``he said consolingly''; ``she replied grumblingly.'' Let the conversation itself disclose the speaker's manner of condition. Dialogue heavily weighted with adverbs after the attributive verb is cluttery \& annoying. Inexperienced writers not only overwork their adverbs but load their attributives with explanatory verbs: ``he consoled,'' ``she congratulated.'' They do this, apparently, in the belief that the word \textit{said} is always in need of support, or because they have been told to do it by experts in the art of bad writing.'' -- \cite[p. 89]{Strunk_White2019}

%------------------------------------------------------------------------------%

\subsection{Do not construct awkward adverbs.}
``Adverbs are easy to build. Take an adjective or a participle, add \textit{-ly}, \& behold! you have an adverb. But you'd probably be better off without it. Do not write \textit{tangledly}. The word itself is a tangle. Do not even write \textit{tiredly}. Nobody says \textit{tangledly} \& not many people say \textit{tiredly}. Words that are not used orally are seldom the ones to put on paper.
\begin{example}
	He climbed tiredly to bed. $\to$ He climbed wearily to bed.
	
	The lamp cord lay tangledly beneath her chair. $\to$ The lamp cord lay in tangles beneath her chair.
\end{example}
Do not dress words up by adding \textit{-ly} to them, as though putting a hat on a horse.
\begin{example}
	overly $\to$ over, muchly $\to$ much, thusly $\to$ thus.'' -- \cite[p. 90]{Strunk_White2019}
\end{example}


%------------------------------------------------------------------------------%

\subsection{Make sure the reader knows who is speaking.}
``Dialogue is a total loss unless you indicate who the speaker is. In long dialogue passages containing no attributives, the reader may become lost \& be compelled to go back \& reread in order to puzzle the thing out. Obscurity is an imposition on the reader, to say nothing of its damage to the work.

In dialogue, make sure that your attributives do not awkwardly interrupt a spoken sentence. Place them where the break would come naturally in speech -- i.e., where the speaker would pause for emphasis, or take a breath. The best test for locating an attributive is to speak the sentence aloud.
\begin{example}
	``Now, my boy, we shall see,'' he said, ``how well you have learned your lesson.'' $\to$ ``Now, my boy,'' he said, ``we shall see how well you have learned your lesson.''
	
	``What's more, they would never,'' she added, ``consent to the plan.'' $\to$  ``What's more,'' she added, ``they would never consent to the plan.'''' -- \cite[p. 91]{Strunk_White2019}
\end{example}

%------------------------------------------------------------------------------%

\subsection{Avoid fancy words.}
``Avoid the elaborate, the pretentious, the coy, \& the cute. Do not be tempted by a 20-dollar word when there is a 10-center handy, ready \& able. Anglo-Saxon is a livelier tongue than Latin, so use Anglo-Saxon words. In this, as in so many matters pertaining to style, one's ear must be one's guide: \textit{gut} is a lustier noun than \textit{intestine}, but the 2 words are not interchangeable, because \textit{gut} is often inappropriate, being too coarse for the context. Never call a stomach a tummy without good reason.

If you admire fancy words, if every sky is \textit{beauteous}, every blonde \textit{curvaceous}, every intelligent child prodigious, if you are tickled by \textit{discombobulate}, you will have a bad time with Reminder 14. What is wrong, you ask, with \textit{beauteous?} No one knows, for sure. There is nothing wrong, really, with any word -- all are good, but some are better than others. A matter of ear, a matter of reading the books that sharpen the ear.

The line between the fancy \& the plain, between the atrocious \& the felicitous, is sometimes alarmingly fine. The opening phrase of the Gettysburg address is close to the line, at least by our standards today, \& Mr. Lincoln, knowingly or unknowingly, was flirting with disaster when he wrote ``4 score \& 7 years ago.'' The President could have got into his sentence with plain ``87'' -- a saving of 2 words \& less of a strain on the listeners' powers of multiplication. But Lincoln's ear must have told him to go ahead with 4 score \& 7. By doing so, he achieved cadence while skirting the edge of fanciness. Suppose he had blundered over the line \& written, ``In the year of our Lord seventeen hundred \& seventy-six.'' His speech would have sustained a heavy blow. Or suppose he had settle for ``87.'' In that case he would have got into his introductory sentence too quickly; the timing would have been bad.

The question of ear is vital. Only the writer whose ear is reliable is in a position to use bad grammar deliberately; this writer knows for sure when a colloquialism is better than formal phrasing \& is able to sustain the work at a level of good taste. So cock your ear. Years ago, students were warned not to end a sentence with a preposition; time, of course, has softened that rigid decree. Not only is the preposition acceptable at the end, sometimes it is more effective in that spot than anywhere else. ``A claw hammer, not an ax, was the tool he murdered her with.'' This is preferable to ``A claw hammer, not an ax, was the tool with which he murdered her.'' Why? Because it sounds more violent, more like murder. A matter of ear.

\& would you write ``The worst tennis player around here is I'' or ``The The worst tennis player around here is me''? The 1st is good grammar, the 2nd is good judgment -- although the \textit{me} might not do in all contexts.

The split infinitive is another trick of rhetoric in which the ear must be quicker than the handbook. Some infinitives seem to improve on being split, just as a stick of round stovewood does. ``I cannot bring myself to really like the fellow.'' The sentence is relaxed, the meaning is clear, the violation is harmless \& scarcely perceptible. Put the other way, the sentence becomes stiff, needlessly formal. A matter of ear.

There are times when the ear not only guides us through difficult situations but also saves us from minor or major embarrassments of prose. The ear, e.g., must decide when to omit \textit{that} from a sentence, when to retain it. ``She knew she could do it'' is preferable to ``She knew that she could do it'' -- simpler \& just as clear. Bu tin many cases the \textit{that} is needed. ``He felt that his big nose, which was sunburned, made him look ridiculous.'' Omit the \textit{that} \& you have ``He felt his big nose $\ldots$'''' -- \cite[p. 93]{Strunk_White2019}

%------------------------------------------------------------------------------%

\subsection{Do not use dialect unless your ear is good.}

%------------------------------------------------------------------------------%

\subsection{Be clear.}

%------------------------------------------------------------------------------%

\subsection{Do not inject opinion.}

%------------------------------------------------------------------------------%

\subsection{Use figures of speech sparingly.}

%------------------------------------------------------------------------------%

\subsection{Do not take shortcuts at the cost of clarity.}

%------------------------------------------------------------------------------%

\subsection{Avoid foreign languages.}

%------------------------------------------------------------------------------%

\subsection{Prefer the standard to the offbeat.}

%------------------------------------------------------------------------------%

\section{Afterword}

%------------------------------------------------------------------------------%

\printbibliography[heading=bibintoc]
	
\end{document}