\documentclass{article}
\usepackage[backend=biber,natbib=true,style=authoryear]{biblatex}
\addbibresource{/home/nqbh/reference/bib.bib}
\usepackage{tocloft}
\renewcommand{\cftsecleader}{\cftdotfill{\cftdotsep}}
\usepackage[colorlinks=true,linkcolor=blue,urlcolor=red,citecolor=magenta]{hyperref}
\usepackage{algorithm,algpseudocode,amsmath,amssymb,amsthm,float,graphicx,mathtools}
\allowdisplaybreaks
\numberwithin{equation}{section}
\newtheorem{assumption}{Assumption}[section]
\newtheorem{conjecture}{Conjecture}[section]
\newtheorem{corollary}{Corollary}[section]
\newtheorem{definition}{Definition}[section]
\newtheorem{example}{Example}[section]
\newtheorem{lemma}{Lemma}[section]
\newtheorem{notation}{Notation}[section]
\newtheorem{principle}{Principle}[section]
\newtheorem{problem}{Problem}[section]
\newtheorem{proposition}{Proposition}[section]
\newtheorem{question}{Question}[section]
\newtheorem{remark}{Remark}[section]
\newtheorem{theorem}{Theorem}[section]
\usepackage[left=0.5in,right=0.5in,top=1.5cm,bottom=1.5cm]{geometry}
\usepackage{fancyhdr}
\pagestyle{fancy}
\fancyhf{}
\lhead{\small Sect.~\thesection}
\rhead{\small\nouppercase{\leftmark}}
\renewcommand{\sectionmark}[1]{\markboth{#1}{}}
\cfoot{\thepage}
\def\labelitemii{$\circ$}

\title{The Elements of Style}
\author{William Strunk, Jr. \and E. B. White}
\date{\today}

\begin{document}
\maketitle
\tableofcontents

%------------------------------------------------------------------------------%

\section*{Foreword}

%------------------------------------------------------------------------------%

\section*{Introduction}

%------------------------------------------------------------------------------%

\section{Elementary Rules of Usage}

\subsection{Form the possessive singular of nouns by adding 's.}

%------------------------------------------------------------------------------%

\subsection{In a series of 3 or more terms with a single conjunction, use a comma after each term except the last.}

%------------------------------------------------------------------------------%

\subsection{Enclose parenthetic expressions between commas.}

%------------------------------------------------------------------------------%

\subsection{Place a comma before a conjunction introducing an independent clause.}

%------------------------------------------------------------------------------%

\subsection{Do not join independent clauses with a comma.}

%------------------------------------------------------------------------------%

\subsection{Do not break sentences in 2.}

%------------------------------------------------------------------------------%

\subsection{Use a colon after an independent clause to introduce a list of particulars, an appositive, an amplification, or an illustrative quotation.}

%------------------------------------------------------------------------------%

\subsection{Use a dash to set off an abrupt break or interruption \& to announce a long appositive or summary.}

%------------------------------------------------------------------------------%

\subsection{The number of the subject determines the number of the verb.}

%------------------------------------------------------------------------------%

\subsection{Use the proper case of pronoun.}

%------------------------------------------------------------------------------%

\subsection{A participial phrase at the beginning of a sentence must refer to the grammatical subject.}

%------------------------------------------------------------------------------%

\section{Elementary Principles of Composition}

\subsection{Choose a suitable design \& hold to it.}

%------------------------------------------------------------------------------%

\subsection{Make the paragraph the unit of composition: 1 paragraph to each topic.}

%------------------------------------------------------------------------------%

\subsection{Use the active voice.}

%------------------------------------------------------------------------------%

\subsection{Put statements in positive form.}

%------------------------------------------------------------------------------%

\subsection{Use definite, specific, concrete  language.}

%------------------------------------------------------------------------------%

\subsection{Omit needless words.}

%------------------------------------------------------------------------------%

\subsection{Avoid a succession of loose sentences.}

%------------------------------------------------------------------------------%

\subsection{Express coordinate ideas in similar form.}

%------------------------------------------------------------------------------%

\subsection{Keep related words together.}

%------------------------------------------------------------------------------%

\subsection{In summaries, keep to 1 tense.}

%------------------------------------------------------------------------------%

\subsection{Place the emphatic words of a sentence at the end.}

%------------------------------------------------------------------------------%

\section{A Few Matters of Form}

%------------------------------------------------------------------------------%

\section{Words \& Expressions Commonly Misused}

%------------------------------------------------------------------------------%

\section{An Approach to Style (With a List of Reminders)}

\subsection{Place yourself in the background.}
``Write in a way that draws the reader's attention to the sense \& substance of the writing, rather than to the mood \& temper of the author. If the writing is solid \& good, the mood \& temper of the writer will eventually be revealed \& not at the expense of the work. Therefore, the 1st piece of advice is this: to achieve style, begin by affecting none -- i.e., place yourself in the background. A careful \& honest writer does not need to worry about style. As you become proficient in the use of language, your style will emerge, because you yourself will emerge, \& when this happens you will find it increasingly easy to break through the barriers that separate you from other minds, other hearts -- which is, of course, the purpose of writing, as well as its principal reward. Fortunately, the act of composition, or creation, disciplines the mind; writing is 1 way to go about thinking, \& the practice \& habit of writing not only drain the mind but supply it, too.'' -- \cite[p. 78]{Strunk_White2019}

%------------------------------------------------------------------------------%

\subsection{Write in a way that comes naturally.}

%------------------------------------------------------------------------------%

\subsection{Work from a suitable design.}

%------------------------------------------------------------------------------%

\subsection{Write with nouns \& verbs.}

%------------------------------------------------------------------------------%

\subsection{Revise \& rewrite.}

%------------------------------------------------------------------------------%

\subsection{Do not overwrite.}

%------------------------------------------------------------------------------%

\subsection{Do not overstate.}

%------------------------------------------------------------------------------%

\subsection{Avoid the use of qualifiers.}

%------------------------------------------------------------------------------%

\subsection{Do not affect a breezy manner.}

%------------------------------------------------------------------------------%

\subsection{Use orthodox spelling.}

%------------------------------------------------------------------------------%

\subsection{Do not explain too much.}

%------------------------------------------------------------------------------%

\subsection{Do not construct awkward adverbs.}

%------------------------------------------------------------------------------%

\subsection{Make sure the reader knows who is speaking.}

%------------------------------------------------------------------------------%

\subsection{Avoid fancy words.}

%------------------------------------------------------------------------------%

\subsection{Do not use dialect unless your ear is good.}

%------------------------------------------------------------------------------%

\subsection{Be clear.}

%------------------------------------------------------------------------------%

\subsection{Do not inject opinion.}

%------------------------------------------------------------------------------%

\subsection{Use figures of speech sparingly.}

%------------------------------------------------------------------------------%

\subsection{Do not take shortcuts at the cost of clarity.}

%------------------------------------------------------------------------------%

\subsection{Avoid foreign languages.}

%------------------------------------------------------------------------------%

\subsection{Prefer the standard to the offbeat.}

%------------------------------------------------------------------------------%

\section{Afterword}

%------------------------------------------------------------------------------%

\printbibliography[heading=bibintoc]
	
\end{document}