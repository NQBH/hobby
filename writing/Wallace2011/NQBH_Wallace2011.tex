\documentclass{article}
\usepackage[backend=biber,natbib=true,style=authoryear]{biblatex}
\addbibresource{/home/nqbh/reference/bib.bib}
\usepackage{tocloft}
\renewcommand{\cftsecleader}{\cftdotfill{\cftdotsep}}
\usepackage[colorlinks=true,linkcolor=blue,urlcolor=red,citecolor=magenta]{hyperref}
\usepackage{algorithm,algpseudocode,amsmath,amssymb,amsthm,float,graphicx,mathtools}
\usepackage{enumitem}
\setlist{leftmargin=4mm}
\allowdisplaybreaks
\numberwithin{equation}{section}
\newtheorem{assumption}{Assumption}[section]
\newtheorem{conjecture}{Conjecture}[section]
\newtheorem{corollary}{Corollary}[section]
\newtheorem{definition}{Definition}[section]
\newtheorem{example}{Example}[section]
\newtheorem{lemma}{Lemma}[section]
\newtheorem{notation}{Notation}[section]
\newtheorem{principle}{Principle}[section]
\newtheorem{problem}{Problem}[section]
\newtheorem{proposition}{Proposition}[section]
\newtheorem{question}{Question}[section]
\newtheorem{remark}{Remark}[section]
\newtheorem{theorem}{Theorem}[section]
\usepackage[left=1cm,right=1cm,top=5mm,bottom=5mm,footskip=4mm]{geometry}
\def\labelitemii{$\circ$}

\title{Infinite Jest: A Novel}
\author{David Foster Wallace}
\date{\today}

\begin{document}
\maketitle
\tableofcontents

%------------------------------------------------------------------------------%

\section{Quotes}

\begin{enumerate}
	\item ``Everybody is identical in their secret unspoken belief that way deep down they are different from everyone else.''
	\item ``I do things like get in a taxi \& say, ``The library, \& step on it.''''
	\item ``The truth will set you free. But not until it is finished with you.''
	\item ``You will become way less concerned with what other people think of you when you realize how seldom they do.''
	\item ``Mario, what do you get when you cross an insomniac, \& unwilling agnostic \& a dyslexic?''
	
	``I give.''
	
	``You get someone who stays up all night torturing himself mentally over the question of whether or not there's a dog.''
	\item ``It's weird to feel like you miss someone you're not even sure you know.''
	\item ``Try to learn to let what is unfair teach you.''
	\item ``What passes for hip cynical transcendence of sentiment is really some kind of fear of being really human, since to be really human [$\ldots$] is probably to be unavoidably sentimental \& naive \& goo-prone \& generally pathetic.''
	\item ``It did what all ads are supposed to do: create an anxiety relievable by purchase.''
	\item ``That sometimes human beings have to just sit in 1 place \&, like, hurt. That you will become way less concerned with what other people think of you when you realize how seldom they do. That there is such a thing as raw, unalloyed, agendaless kindness. That it is possible to fall asleep during an anxiety attack. That concentrating on anything is very hard work.''
	\item ``[$\ldots$] almost nothing important that ever happens to you happens because you engineer it. Destiny has no beeper; destiny always leans trenchcoated out of an alley with some sort of `psst' that you usually can't even hear because you're in such a rush to or from something important you've tried to engineer.''
	\item ``$\ldots$ logical validity is not a guarantee of truth.';
	\item ``Te Occidere Possunt Sed Te Edere Non Possunt Nefas Est'' (``They can kill you, but the legalities of eating you are quite a bit dicier'').''
	\item ``I think there must be probably different types of suicides. I'm not one of the self-hating ones. The type of like ``I'm shit \& the world'd be better off without poor me'' type that says that but also imagines what everybody'll say at their funeral. I've met types like that on wards. Poor-me-I-hate-me-punish-me-come-to-my-funeral. Then they show you a $20\times25$ glossy of their dead cat. It's all self-pity bullshit. It's bullshit. I didn't have any special grudges. I didn't fail an exam or get dumped by anybody. All these types. Hurt themselves. I didn't want to especially hurt myself. Or like punish. I don't hate myself. I just wanted out. I didn't want to play anymore is all. I wanted to just stop being conscious. I'm a whole different type. I wanted to stop feeling this way. If I could have just put myself in a really long coma I would have done that. Or given myself shock I would have done that. Instead.''
	\item ``\& lo, for the Earth was empty of Form, \& avoid. \& Darkness was all over the Face of the Deep. \& We said: `Look at that fucker Dance.''
	\item ``You can be shaped, or you can be broken. There is not much in between. Try to learn. Be coachable. Try to learn from everybody, especially those who fail. This is hard. $\ldots$ How promising you are as a Student of the Game is a function of what you can pay attention to without running away.''
	\item ``I'll say god seems to have a kind of laid-back management style I'm not crazy about. I'm pretty much anti-death. God looks by all accounts to be pro-death. I'm not seeing how we can get together on this issue, he \& I $\ldots$''
	\item ``sarcasm \& jokes were often the bottle in which clinical depressives sent out their most plangent screams for someone to care \& help them.''
	\item ``$\ldots$ That no single, individual moment is in \& of itself unendurable.''
	\item ``$\ldots$ most Substance-addicted people are also addicted to thinking, meaning they have a compulsive \& unhealthy relationship with their own thinking.''
\end{enumerate}

%------------------------------------------------------------------------------%

\begin{quotation}
	``Uproarious $\ldots$ \textit{Infinite Jest} shows off Wallace as 1 of the big talents of his generation, a writer of virtuosic talents who can seemingly do anything.'' -- New York Times
\end{quotation}
``Where the names of real places, corporations, institutions, \& public figures are projected onto made-up stuff, they are intended to denote only made-up stuff, not anything presently real.''

%------------------------------------------------------------------------------%

\section*{Foreword}
``In recent years, there have been a few literary dustups -- how insane is it that such a thing exists in a world at war? -- about readability in contemporary fiction. In essence, there are some people who feel that fiction should be easy to read, that it's a popular medium that should communicate on a somewhat conversational wavelength. On the other hand, there are those who feel that fiction can be challenging, generally \& thematically, \& even on a sentence-by-sentence basis -- that it's okay if a person needs to work a bit while reading, for the rewards can be that much greater when one's mind has been exercised \& thus (presumably) expanded.

Much in the way that would-be civilized debates are polarized by extreme thinkers on either side, this debate has been made to seem like an either{\tt/}or proposition, that the world has room for only 1 kind of fiction, \& that the other kind should be banned \& its proponents hunted down \&, why not, dismembered.

But while the polarizers have been going at it, there has existed a silent legion of readers, perhaps the majority of readers of literary fiction, who don't mind a little of both. They believe, though not too vocally, that so-called difficult books can exist next to, can even rub bindings suggestively with, more welcoming fiction. These readers might actually read \textit{both} kinds of fiction themselves, \textit{sometimes in the same week}. There might even be -- though it's impossible to prove -- readers who find it possible to enjoy Thomas Pynchon 1 day \& Elmore Leonard the next. Or even: readers who can have fun with Jonathan Franzen in the morning while wrestling with William Gaddis at night.

David Foster Wallace has long straddle the worlds of difficult \& not-as-difficult, with most readers agreeing that his essays are easier to read than his fiction, \& his journalism most accessible of all. But while much of his work is challenging, his tone, in whatever form he's exploring, is rigorously unpretentious. A Wallace reader gets the impression of being in a room with a very talkative \& brilliant uncle or cousin who, just when he's about to push it too far, to try our patience with too much detail, has the good sense to throw in a good lowbrow joke. Wallace, like many other writers who could be otherwise considered too smart for their own good -- Bellow comes to mind -- is, like Bellow, always aware of the reader, of the idea that books are essentially meant to entertain, \& so almost unerringly balances his prose to suit. This had been Wallace's hallmark for years before this book, of course. He was already known as a very smart \& challenging \& funny \& preternaturally gifted writer when \textit{Infinite Jest} was released in 1996, \& thereafter his reputation included all the adjectives mentioned just now, \& also this one: Holy shit.

Not, that isn't an adjective in the strictest sense. But you get the idea. The book is 1,079 pages long \& there is not 1 lazy sentence. The book is drum-tight \& relentlessly smart, \& though it does not wear its heart on its sleeve, it's deeply felt \& incredibly moving. That it was written in 3 years by a writer under 35 is very painful to think about. So let's not think about that. The point is that it's for all these reasons -- acclaimed, daunting, not-lazy, drum-tight, very funny (we didn't mention that yet but yes) -- that you picked up this book. Now the question is this: Will you actually read it?

In commissioning this foreword, the publisher wanted a very brief \& breezy essay that might convince a new reader of \textit{Infinite Jest} that the book is approachable, effortless even -- a barrel of monkey's worth of fun to read. Well. It's easy to agree with the former, more difficult to advocate the latter. The book is approachable, yes, because it doesn't include complex scientific or historical content, nor does it require any particular expertise or erudition. As verbose as it is, \& as long as it is, it never wants to punish you for some knowledge you lack, nor does it want to send you to the dictionary every few pages. \& yet, while it uses a familiar enough vocabulary, make no mistake that \textit{Infinite Jest} is something \textit{other}. I.e., it bears little resemblance to anything before it, \& comparisons to anything since are desperate \& hollow. It appeared in 1996, \textit{sui generis}, very different from virtually anything before it. It defied categorization \& thwarted efforts to take it apart \& explain it.

It's possible, with most contemporary novels, for astute readers, if they are wont, to break it down into its parts, to take it apart as one would a car or Ikea shelving unit. I.e., let's say a reader is a sort of mechanic. \& let's say this particular reader-mechanic has worked on lots of books, \& after a few hundred contemporary novels, the mechanic feels like he can take apart just about any book \& put it back together again. I.e., the mechanic recognizes the components of modern fiction \& can say, e.g., \textit{I've seen this part before, so I know why it's there \& what it does. \& this one, too -- I recognize it. This part connects to this \& performs this function. This one usually goes here, \& does that. All of this is familiar enough}. That's no knock on the contemporary fiction that is recognizable \& breakdownable. This includes about 98\% of the fiction we know \& love.

But this is not possible with \textit{Infinite Jest}. This book is like a spaceship with no recognizable components, no rivets or bolts, no entry points, no way to take it apart. It is very shiny, \& it has no discernible flaws. If you could somehow smash it into smaller pieces, there would certainly be no way to put it back together again. It simply \textit{is}. Page by page, line by line, it is probably the strangest, most distinctive, \& most involved work of fiction by an American in the last 20 years. At no time while reading \textit{Infinite Jest} are you are unaware that this is a work of complete obsession, of a stretching of the mind of a young writer to the point of, we assume, near madness.

Which isn't to say it's madness in the way that Burroughs or even Fred Exley used a type of madness with which to create. Exley, like many writers of his generation \& the few before it, drank to excess, \& Burroughs ingested every controlled substance he could buy or borrow. But Wallace is a different sort of madman, one in full control of his tools, one who instead of teetering on the edge of this precipice or that, under the influence of drugs or alcohol, seems to be heading ever-inward, into the depths of memory \& the relentless conjuring of a certain time \& place in a way that evokes -- it seems so wrong to type this name but then again, so right! -- Marcel Proust. There is the same sort of obsessiveness, the same incredible precision \& focus, \& the same sense that the writer wanted (\& arguably succeeds at) nailing the consciousness of an age.

Let's talk about age, the more pedestrian meaning of the word. It's to be expected that the average age of the new \textit{Infinite Jest} reader would be about 25. There are certainly many collegians among you, probably, \& there may be an equal number of 30-year-olds or 50-year-olds who have for whatever reason reached a point in their lives where they have determined themselves finally ready to tackle the book, which this or that friend has urged upon them. The point is that the average age is appropriate enough. I was 25 myself when I 1st read it. I had known it was coming for about a year, because the publisher, Little, Brown, had been very clever about building anticipation for it, with monthly postcards, bearing teasing phrases \& hints, sent to every media outlet in the country. When the book was finally released, I started in on it almost immediately.

\& thus I spent a month of my young life. I did little else. \& I can't say it was always a barrel of monkeys. It was occasionally trying. It demands your full attention. It can't be read at a crowded caf\'e, or with a child on one's lap. It was frustrating that the footnotes were at the end of the book, rather than on the bottom of the page, as they had been in Wallace's essays \& journalism. There were times, reading a very exhaustive account of a tennis match, say, when I thought, well, okay. I like tennis as much as the next guy, but enough already.

\& yet the time spent in this book, in this world of language, is absolutely rewarded. When you exit these pages after that month of reading, you are a better person. It's insane, but also hard to deny. Your brain is stronger because it's been given a monthlong workout, \& more importantly, your heart is sturdier, for there has scarcely been written a more moving account of desperation, depression, addiction, generational stasis \& yearning, or the obsession with human expectations, with artistic \& athletic \& intellectual possibility. The themes here are big, \& the emotions (guarded as they are) are very real, \& the cumulative effect of the book is, you could say, seismic. It would be very unlikely that you would find a reader who, after finishing the book, would shrug \& say, ``Eh.''

Here's a question once posed to me, by a large, baseball cap-wearing English major at a medium-size western college: Is it our duty to read \textit{Infinite Jest}? This is a good question, \& one that many people, particularly literary-minded people, ask themselves. The answer is: Maybe. Sort of. Probably, in some way. If we think it's our duty to read this book, it's because we're interested in genius. We're interested in epic writerly ambition. We're fascinated with what can be made by a person with enough time \& focus \& caffeine \&, in Wallace's case, chewing tobacco. If we are drawn to \textit{Infinite Jest}, we're also drawn to the Magnetic Field's \textit{69 Songs}, for which Stephin Merritt wrote that many songs, all fo them above love, in about 2 years. \& we're drawn to the 10000 paintings of folk artist Howard Finster. Or the work of Sufjan Stevens, who is on a mission to create an album about each state in the union. He's currently at State No. 2, but if he reaches his goal, it will approach what Wallace did with the book in your hands. The point is that if we are interested in human possibility, \& we are able to cheer each other on to leaps in science \& athletics \& art \& thought, we must admire the work that our peers have managed to create. We have an obligation, to ourselves, chiefly, to see what a brain, \& particularly a brain like our own -- i.e., using the same effluvium we, too, swim through - - is capable of. It's why we watch \textit{Shoah}, or visit the unending scroll on which Jack Kerouac wrote (in a fever of days) \textit{On the Road}, or William T. Vollmann's 3,300-page \textit{Rising Up \& Rising Down}, or Michael Apted's \textit{7-Up, 28-Up, 42-Up} series of films, or $\ldots$ well, the list goes on.

\& now, unfortunately, we're back to the impression that this book is daunting. Which it isn't, really. It's long, but there are pleasures everywhere. There is humor everywhere. There is also a very quiet but very sturdy \& constant tragic undercurrent that concerns a people who are completely lost, who are lost within their families \& lost within their nation, \& lost within their time, \& who only want some sort of direction or purpose or sense of community or love. Which is, after all \& conveniently enough for the end of this introduction, what an author is seeking when he sets out to write a book -- any book, but particularly a book like this, a book that gives so much, that required such sacrifice \& dedication. Who would do such a thing if not for want of connection \& thus of love?

Last thing: In attempting to persuade you to buy this book, or check it out of your library, it's useful to tell you that the author is a normal person. Dave Wallace -- \& he is commonly known as such -- keeps big sloppy dogs \& has never dressed them in taffeta or made them wear raincoats. He has complained often about sweating too much when he gives public readings, so much so that he wears a bandanna to keep the perspiration from soaking the pages below him. He was once a nationally ranked tennis player, \& he cares about good government. He is from the Midwest -- east-central Illinois, to be specific, which is an intensely normal part of the country (not far, in fact, from a city, no joke, named Normal). So he is normal, \& regular, \& ordinary, \& this is his extraordinary, \& irregular, \& not-normal achievement, a thing that will outlast him \& you \& me, but will help future people understand us -- how we felt, how we lived, what we gave to each other \& why. -- Dave Eggers, Sep 2006'' -- \cite[pp. xi--xvi]{Wallace2011}

%------------------------------------------------------------------------------%

\section{Year of Glad}
``I am seated in an office, surrounded by heads \& bodies. My posture is consciously congruent to the shape of my hard chair. This is a cold room in University Administration, wood-walled, Remington-hung, double-windowed against the Nov heat, insulated from Administrative sounds by the reception area outside, at which Uncle Charles, Mr. deLint \& I were lately received.

I am in here.

3 faces have resolved into place above summer-weight sportcoats \& half-Windsors across a polished pine conference table shiny with the spidered light of an Arizona noon. These are 3 Deans -- of Admissions, Academic Affairs, Athletic Affairs. I do not know which face belongs to whom.

I believe I appear neutral, maybe even pleasant, though I've been coached to err on the side of neutrality \& not attempt what would feel to me like a pleasant expression or smile.

I have committed to crossing my legs I hope carefully, ankle on knee, hands together in the lap of my slacks. My fingers are mated into a mirrored series of what manifests, to me, as the letter X. The interview room's other personnel include: the University's Director of Composition, its varsity tennis coach, \& Academy prorector Mr. A. deLint. C.T. is beside me; the others sit, stand \& stand, respectively, at the periphery of my focus. The tennis coach jingles pocket-change. There is something vaguely digestive about the room's odor. The high-traction sole of my complimentary Nike sneaker runs parallel to the wobbling loafer of my mother's half-brother, here in his capacity as Headmaster, sitting in the chair to what I hope is my immediate right, also facing Deans.

The Dean at left, a lean yellowish man whose fixed smile nevertheless has the impermanent quality of something stamped into uncooperative material, is a personality-type I've come lately to appreciate, the type who delays need of any response from me by relating my side of the story for me, to me. Passed a packet of computer-sheets by the shaggy lion of a Dean at center, he is speaking more or less to these pages, smiling down.

`You are Harold Incandenza, 18, date of secondary-school graduation approximately 1 month from now, attending the Enfield Tennis Academy, Enfield, Massachusetts, a boarding school, where you reside.' His reading glasses are rectangular, court-shaped, the sidelines at top \& bottom. `You are, according to Coach White \& Dean [unintelligible], a regionally, nationally, \& continentally ranked junior tennis player, a potential O.N.A.N.C.A.A. athlete of substantial promise, recruited by Coach White via correspondence with Dr. Tavis here commencing $\ldots$ Feb of of this year.' The top page is removed \& brought around neatly to the bottom of the sheaf, at intervals. `You have been in residence at the Enfield Tennis Academy since age 7.'

I am debating whether to risk scratching the right side of my jaw, where there is a wen.

`Coach White informs out offices that he holds the Enfield Tennis Academy's program \& achievements in high regard, that the University of Arizona tennis squad has profited from the major matriculation of several former E.T.A. alumni, 1 of whom was 1 Mr. Aubrey F. deLint, who appears also to be with you here today. Coach White \& his staff have given us --'

The yellow administrator's usage is on the whole undistinguished, though I have to admit he's made himself understood. The Director of Composition seems to have more than the normal number of eyebrows. The Dean at right is looking at my face a bit strangely.

Uncle Charles is saying that though he can anticipate that the Deans might be predisposed to weigh what he avers as coming from his possible appearance as a kind of cheerleader for E.T.A., he can assure the assembled Deans that all this is true, \& that the Academy has presently in residence no fewer than $\frac{1}{3}$ of the continent's top 30 juniors, in age brackets all across the board, \& that I here, why go by `Hal,' usually, am `right up there among the very cream.' Right \& center Deans smile professionally; the heads of deLint \& the coach incline as the Dean at left clears this throat:

`-- belief that you could well make, even as a freshman, a real contribution to this University's varsity tennis program. We are pleased,' he either says or reads, removing a page, `that a competition of some major sort here has brought you down \& given us the chance to sit down \& chat together about your application \& potential recruitment \& matriculation \& scholarship.'

`I've been asked to add that Hal here is seeded 3rd, Boys' 18-\&-Under Singles, in the prestigious WhataBurger Southwest Junior Invitational out at the Randolph Tennis Center --' says what I infer is Athletic Affairs, his cocked head showing a freckled scalp.

`Out at Randolph Park, near the outstanding El Con Marriott,' C.T. inserts, `a venue the whole contingent's been vocal about finding absolutely top-hole thus far, which --'

`Just so, Chuck, \& that according to Chuck here Hall has already justified his seed, he's reached the semifinals as of this morning's apparently impressive win, \& that he'll be playing out at the Center again tomorrow, against the winner of a quarterfinal game tonight, \& so will be playing tomorrow at I believe scheduled for 0830 --'

`Try to get under way before the godawful heat out there. Though of course a dry heat.'

`-- \& has apparently already qualified for this winter's Continental Indoors, up in Edmonton, Kirk tells me --' cocking further to look up \& left at the varsity coach, whose smile's teeth are radiant against a violent sunburn -- `Which is something indeed.' He smiles, looking at me. `Did we get all that right Hal.'

C.T. has crossed his arms casually; their triceps' flesh is webbed with mottle in the air-conditioned sunlight. `You sure did. Bill.' He smiles. The 2 halves of his mustache never quite match. `\& let me say if I may that Hal's excited, excited to be invited for the 3rd year running to the Invitational again, to be back here in a community he has real affection for, to visit with your alumni \& coaching staff, to have already justified his high seed in this week's not unstiff competition, to as they say still be in it without the fat woman in the Viking that having sung, so to speak, but of course most of all to have a chance to meet you gentlemen \& have a look at the facilities here. Everything here is absolutely top-slot, from what he's seen.'

There is a silence. DeLint shifts his back against the room's paneling \& recenters his weight. My uncle beams \& straightens a straight watchband. 62.5\% of the room's faces are directed my way, pleasantly expectant. My chest bumps like a dryer with shoes in it. I compose what I project will be seen as a smile. I turn this way \& that, slightly, sort of directing the expression to everyone in the room.

There is a new silence. The yellow Dean's eyebrows go circumflex. The 2 other Deans look to the Director of Composition. The tennis coach has moved to stand at the broad window, feeling at the back of his crewcut. Uncle Charles strokes the forearm above his watch. Sharp curved palm-shadows move slightly over the pine table's shine, the one head's shadow a black moon.

`Is Hal all right, Chuck?' Athletic Affairs asks. `Hal just seemed to $\ldots$ well, grimace. Is he in pain? Are you in pain, son?'

`Hal's right as rain,' smiles my uncle, soothing the air with a casual hand. `Just a bit of a let's call it maybe a facial tic, slightly, at all the adrenaline of being here on your impressive campus, justifying his seed so far without dropping a set, receiving that official written offer of not only waivers but a living allowance from Coach White here, on Pac 10 letterhead, being ready in all probability to sign a National Letter of Intent right here \& now this very day, he's indicated to me.' C.T. looks to me, his look horribly mild. I do the safe thing, relaxing every muscles in my face, emptying out all expression. I stare carefully into the Kekul\'ean knot of the middle Dean's necktie.

My silent response to the expectant silence begins to affect the air of the room, the bits of dust \& sportcoat-lint stirred around by the AC's vents dancing jaggedly in the slanted plane of windowlight, the air over the table like the sparkling space just above a fresh-poured seltzer. The coach, in a slight accent neither British or Australian, is telling C.T. that the whole application-interface process, while usually just a pleasant formality, is probably best accentuated by letting the applicant speak up for himself. Right \& center Deans have inclined together in soft conference, forming a kind of tepee of skin \& hair. I presume it's probably \textit{facilitate} that the tennis coach mistook for \textit{accentuate}, though \textit{accelerate}, while clunkier than \textit{facilitate}, is from a phonetic perspective more sensible, as a mistake. The Dean with the flat yellow face has leaned forward, his lips drawn back from his teeth in what I see as concern. His hands come together on the conference table's surface. His own fingers look like they mate as my own 4-X series dissolves \& I hold tight to the sides of my chair.

We need candidly to chat re potential problems with my application, they \& I, he is beginning to say. He makes a reference to candor \& its value.

`The issues my office faces with the application materials on file from you, Hal, involve some test scores.' He glances down at a colorful sheet of standardized scores in the trench his arms have made. `The Admissions staff is looking at standardized test scores from you that are, as I'm sure you know \& can explain, are, shall we say $\ldots$ subnormal.' I'm to explain.

It's clear that this really pretty sincere yellow Dean at left is Admissions. \& surely the little aviarian figure at right is Athletics, then, because the facial creases of the shaggy middle Dean are now pursed in a kind of distanced affront, an I'm-eating-something-that-makes-me-really-appreciate-the-presence-of-whatever-I'm-drinking-along-with-it look that spells professionally Academic reservations. An uncomplicated loyalty to standards, then, at center. My uncle looks to Athletics as if puzzled. He shifts slightly in his chair.

The incongruity between Admissions's hand- \& face-color is almost wild. `-- verbal scores that are just quite a bit closer to 0 than we're comfortable with, as against a secondary-school transcript from the institution where both your mother \& her brother are administrators --' reading directly out of the sheaf inside his arms' ellipse -- `that this past year, yes, has fallen off a bit, but by the word I mean ``fallen off'' to outstanding from 3 previous years of frankly incredible.'

`Off the charts.'

`Most institutions do not even \textit{have} grades of A with multiple pluses after it,' says the Director of Composition, his expression impossible to interpret.

`This kind of $\ldots$ how shall I put it $\ldots$ incongruity,' Admissions says, his expression frank \& concerned, `I've got to tell you sends up a red flag of potential concern during the admissions process.'

`We thus invite you to explain the appearance of incongruity if not outright shenanigans.' Students has a tiny piping voice that's absurd coming out of a face this big.

`Surely by \textit{incredible} you meant very very very impressive, as opposed to literally quote ``incredible,'' surely,' says C.T., seeming to watch the coach at the window massaging the back of his neck. The huge window gives out on nothing more than dazzling sunlight \& cracked earth with heat-shimmers over it.

`Then there is before us the matter of not the required 2 but \textit{9} separate application essays, some of which of nearly monograph-length, each without exception being --' different sheet -- `the adjective various evaluators used was quote ``stellar'' --'

Dir. of Comp.: `I made in my assessment deliberate use of \textit{lapidary} \& \textit{effete}.'

7
'' -- \cite[pp. 3--]{Wallace2011}

%------------------------------------------------------------------------------%

\printbibliography[heading=bibintoc]
	
\end{document}