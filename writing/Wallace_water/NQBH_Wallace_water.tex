\documentclass{article}
\usepackage[backend=biber,natbib=true,style=alphabetic,maxbibnames=50]{biblatex}
\addbibresource{/home/nqbh/reference/bib.bib}
\usepackage{tocloft}
\renewcommand{\cftsecleader}{\cftdotfill{\cftdotsep}}
\usepackage[colorlinks=true,linkcolor=blue,urlcolor=red,citecolor=magenta]{hyperref}
\usepackage{amsmath,amssymb,amsthm,caption,float,graphicx,mathtools,subcaption}
\allowdisplaybreaks
\numberwithin{equation}{section}
\newtheorem{assumption}{Assumption}[section]
\newtheorem{conjecture}{Conjecture}[section]
\newtheorem{corollary}{Corollary}[section]
\newtheorem{definition}{Definition}[section]
\newtheorem{example}{Example}[section]
\newtheorem{lemma}{Lemma}[section]
\newtheorem{notation}{Notation}[section]
\newtheorem{principle}{Principle}[section]
\newtheorem{problem}{Problem}[section]
\newtheorem{proposition}{Proposition}[section]
\newtheorem{question}{Question}[section]
\newtheorem{remark}{Remark}[section]
\newtheorem{theorem}{Theorem}[section]
\usepackage[left=1cm,right=1cm,top=5mm,bottom=5mm,footskip=4mm]{geometry}
\def\labelitemii{$\circ$}

\title{This Is Water: Some Thoughts, Delivered on a Significant Occasion, about Living a Compassionate Life}
\author{David Foster Wallace}
\date{\today}

\begin{document}
\maketitle

``This Is Water

There are these 2 young fish swimming along \& they happen to meet an older fish swimming the other way, who nods at them \& says, ``Morning, boys, How's the water?''

\& the 2 young fish swim on for a bit, \& then eventually 1 of them looks at the other \& goes, ``What the hell is water?''

This is a standard requirement of US commencement speeches, the deployment of didactic little parable-ish stories.

The story thing turns out to be 1 of the better, less bullshitty conventions of the genre $\ldots$ but if you're worried that I plan to present myself here as the wise old fish explaining what water is to you younger fish, please don't be.

I am not the wise old fish.

The immediate point of the fish story is merely that the most obvious, ubiquitous, important realities are often the ones that are hardest to see \& talk about.

Stated as an English sentence, of course, this is just a banal platitude -- but the fact is that, in the day to day trenches of adult existence, banal platitudes can have a life-or-death importance.

Or so I wish to suggest to you on this dry \& lovely morning.

Of course the main requirement of speeches like this is that I'm supposed to talk about your liberal arts education's meaning, to try to explain why the degree you're about to receive has actual human value instead of just a material payoff.

So let's talk about the single most pervasive clich\'e in the commencement speech genre, which is that a liberal arts education is not so much about filling you up with knowledge as it is about, quote, ``teaching you how to think.''

If you're like me as a college student, you've never liked hearing this, \& you tend to feel a bit insulted by the claim that you've needed anybody to teach you how to think, since the fact that you even got admitted to a college this good seems like proof that you already know how to think.

But I'm going to posit to you that the liberal arts clich\'e turns out not to be insulting at all, because the really significant education in thinking that we are supposed to get in a place like this isn't really about the capacity to think, but rather about the choice of what to think about.

If your complete freedom of choice regarding what to think about seems too obvious to waste time talking about, I'd ask you to think about fish \& water, \& to bracket, for just a few minutes, your skepticism about the value of the totally obvious.

Here's another didactic little story.

There are these 2 guys sitting together in a bar in the remote Alaskan wilderness.

1 of the guys is religious, the other's an atheist, \& they're arguing about the existence of God with that special intensity that comes after abut the 4th beer.

\& the atheist says, ``Look, it's not like I don't have actual reasons for not believing in God.

It's not like I haven't ever experimented with the whole God-\&-prayer thing.

Just last month, I got caught off away from the camp in that terrible blizzard, \& I couldn't see a thing, \& I was totally lost, \& it was 50 below, \& so I did, I tried it: I fell to my knees in the snow \& cried out, `God, if there is a God, I'm lost in this blizzard, \& I'm gonna die if you don't help me!''

\& now, in the bar, the religious guy looks at the atheist all puzzled: ``Well then, you must believe now,'' he says. ``After all, here you are, alive.''

The atheist rolls his eyes like the religious guy is a total simp: ``No, man, all that happened was that a couple Eskimos just happened to come wandering by, \& they showed me the way back to the camp.''

It's easy to run this story through a kind of standard liberal arts analysis: The exact same experience can mean 2 completely different things to 2 different people, given those people's 2 different belief templates \& 2 different ways of constructing meaning from experience.

Because we prize tolerance \& diversity of belief, nowhere in our liberal arts analysis do we want to claim that one guy's interpretation is true \& the other guy's is false or bad.

Which is fine, except we also never end up talking about just where these individual templates \& beliefs come from, meaning, where they come from \textit{inside} the 2 guys.

As if a person's most basic orientation toward the world \& the meaning of his experience were somehow automatically hardwired, like height or shoe size, or absorbed from the culture, like language.

As if how we construct meaning were not actually a matter of personal, intentional choice, of conscious decision.

Plus, there's the matter of arrogance.

The nonreligious guy is so totally, obnoxiously confident in his dismissal of the possibility that the Eskimos had anything to do with his prayer for help.

True, there are plenty of religious people who seem arrogantly certain of their own interpretations, too.

They're probably even more repulsive than atheists, at least to most of us here, but the fact is that religious dogmatists' problem is exactly the same as the story's atheist's -- arrogance, blind certainty, a closed-mindedness that's like an imprisonment so complete that the prisoner doesn't even know he's locked up.

The point here is that I think this is 1 part of what the liberal arts mantra of ``teaching me how to think'' is really supposed to mean: To be just a little less arrogant, to have some ``critical awareness'' abut myself \& my certainties $\ldots$ because a huge percentage of the stuff that I tend to be automatically certain of is, it turns out, totally wrong \& deluded.

I have learned this the hard way, as I predict you graduates will, too.

Here's 1 example of the utter wrongness of something I tend to be automatically sure of.

Everything in my own immediate experience supports my deep belief that I am the absolute center of the universe, the realest, most vivid \& important person in existence.

We rarely think about this sort of natural, basic self-centeredness, because it's so socially repulsive, but it's pretty much the same for all of us, deep down.

It is our default setting, hardwired into our boards at birth.

Think about it: There is no experience you've had that you were not at the absolute center of.

The world as you experience it is there in front of you, or behind you, to the left or right of you, on your TV, on your monitor, or whatever.

Other people's thoughts \& feelings have to be communicated to you somehow, but your own are so immediate, urgent, \textit{real}.

You get the idea.

But please don't worry that I'm getting ready to preach to you about compassion or other-directedness or all the so-called ``virtues.''

This is not a matter of virtue -- It's a matter of my choosing to do the work of somehow altering or getting free of my natural, hardwired default setting, which is to be deeply \& literally self-centered, \& to see \& interpret everything through this lens of self.

People who \textit{can} adjust their natural default setting this way are often described as being, quote, ``well-adjusted,'' which I suggest to you is not an accidental term.

Given the academic setting here, an obvious question is how much of this work of adjusting our default setting involves actual knowledge or intellect.

The answer, not surprisingly, is that it depends on what kind of knowledge we're talking about.

Probably the most dangerous thing about an academic education, at least in my own case, is that it enables my tendency to over-intellectualize stuff, to get lost in abstract thinking instead of simply paying attention to what's going on in front of me.

Instead of paying attention to what's going on \textit{inside} me.

As I'm sure you guys know by now, it is extremely difficult to stay alert \& attentive instead of getting hypnotized by the constant monologue inside your head.

What you don't yet know are the stakes of this struggle.

In the 20 years since my own graduation, I have come gradually to understand these stakes, \& to see that the liberal arts clich\'e about ``teaching you how to think'' was actually shorthand for a very deep \& important truth.

``Learning how to think'' really means learning how to exercise some control over \textit{how} \& \textit{what} you think.

It means being conscious \& aware enough to \textit{choose} what you pay attention to \& to \textit{choose} how you construct meaning from experience.

Because if you cannot or will not exercise this kind of choice in adult life, you will be totally hosed.

Think of the old clich\'e about the mind being ``an excellent servant but a terrible master.''

This, like many clich\'es, so lame \& banal on the surface, actually expresses a great \& terrible truth.

It is not the least bit coincidental that adults who commit suicide with firearms nearly always shoot themselves in $\ldots$ the \textit{head}.

\& the truth is that most of these suicides are actually dead long before they pull the trigger.

\& I submit that this is what the real, no-shit value of your liberal arts education is supposed to be about: How to keep from going through your comfortable, prosperous, respectable adult life dead, unconscious, a slave to your head \& to your natural default setting of being uniquely, completely, imperially alone, day in \& day out.

That may sound like hyperbole, or abstract nonsense.

So let's get concrete.

The plain fact is that you graduating seniors do not yet have any clue what ``day in, day out'' really means.

There happen to be whole large parts of adult American life that nobody talks about in commencement speeches.

1 such part involves \fbox{boredom, routine, \& petty frustration.}

The parents \& older folks here will know all too well what I am talking about.

By way of example, let's say it's an average adult day, \& you get up in the morning, go to your challenging, white-collar college-graduate job, \& you work hard for 9--10 hours, \& at the end of the day you are tired, \& you're stressed out, \& all you want is to go home \& have a good supper \& maybe unwind for a couple hours \& then hit the rack early because you have to get up the next day \& do it all again.

But then you remember there's no food at home -- you haven't had time to shop this week because of your challenging job -- \& so now after work you have to get in your car \& drive to the supermarket.

It's the end of the workday, \& the traffic's very bad, so getting to the store takes way longer than it should, \& when you finally get there, the supermarket is very crowded, because of course it's the time of day when all the other people with jobs also try to squeeze in some grocery shopping, \& the store is hideously, fluorescently lit, \& infused with soul-killing Muzak or corporate pop, \& it's pretty much the last place you want to be, but you can't just get in \& quickly out.

You have to wander all over the huge, overlit store's crowded aisles to find the stuff you want, \& you have to maneuver your junky cart through all these other tired, hurried people with carts, \& of course there are also the glacially slow old people \& the spacey people \& the ADHD kids who all block the aisle, \& you have to grit your teeth \& try to be polite as you ask them to let you by, \& eventually, finally, you get all your supper supplies, except now it turns out there aren't enough checkout lanes open even though it's the end-of-the-day rush, so the checkout line is incredibly long.

Which is stupid \& infuriating, but you can't take your fury out on the frantic lady working the register, who is overworked at a job whose daily tedium \& meaninglessness surpass the imagination of any of us here at a prestigious college $\ldots$ but anyway, you finally get to the checkout line's front, \& you pay for your food, \& wait to get your check or card authenticated by a machine, \& you get told to ``Have a nice day'' in a voice that is the absolute voice of \textit{death}.

'' -- \cite{Wallace2009}

%------------------------------------------------------------------------------%

\printbibliography[heading=bibintoc]
	
\end{document}