\documentclass{article}
\usepackage[backend=biber,natbib=true,style=alphabetic]{biblatex}
\addbibresource{/home/nqbh/reference/bib.bib}
\usepackage{tocloft}
\renewcommand{\cftsecleader}{\cftdotfill{\cftdotsep}}
\usepackage[colorlinks=true,linkcolor=blue,urlcolor=red,citecolor=magenta]{hyperref}
\usepackage{algorithm,algpseudocode,amsmath,amssymb,amsthm,float,graphicx,mathtools}
\usepackage{enumitem}
\setlist{leftmargin=4mm}
\allowdisplaybreaks
\usepackage[left=1cm,right=1cm,top=5mm,bottom=5mm,footskip=4mm]{geometry}
\def\labelitemii{$\circ$}

\title{On Writing Well: The Classic Guide to Writing Nonfiction}
\author{William Zinsser}
\date{\today}

\begin{document}
\maketitle
\tableofcontents
\subsection*{Reference{\tt/}Writing Skills}

\begin{quotation}
	``\textit{On Writing Well} \cite{Zinsser2001, Zinsser2016} is a bible for a generation of writers looking for clues to clean, compelling prose.'' -- \textit{New York Times}
\end{quotation}
``\textit{On Writing Well} has been praised for its sound avice, its clarity \& the warmth of its style. It is a book for everybody who wants to learn how to write or who needs to do some writing to get through the day, as almost everybody does in the age of e-mail \& the Internet.

Whether you want to write about people or places, science \& technology, business, sports, the arts or about yourself in the increasingly popular memoir genre, \textit{On Writing Well} offers you fundamental principles as well as the insights of a distinguished writer \& teacher. With $>10^6$ copies sold, this volume has stood the test of time \& remains a valuable resource for writers \& would-be writers.''
\begin{quotation}
	``Not since \textit{The Elements of Style} has there been a guide to writing as well presented \& readable as this one. A love \& respect for the language is evident on every page.'' -- \textit{Library Journal}
\end{quotation}
\textsc{Books by William Zinsser.} \textit{Any Old Place With You. Seen Any Good Movies Lately? The City Dwellers. Weekend Guests. The Haircurl Papers. Pop Goes America. The Paradise Bit. The Lunacy Boom. On Writing Well. Writing With a Word Processor. Willie \& Dwike} (republished as \textit{Mitchell \& Ruff}). \textit{Writing to Learn. Spring Training. American Places. Speaking of Journalism. Easy to Remember}.

\noindent\textsc{Audio Books by William Zinsser.} \textit{On Writing Well. How to Write a Memoir}.

\noindent\textsc{Books Edited by William Zinsser.} \textit{Extraordinary Lives: The Art \& Craft of American Biography. Inventing the Truth: The Art \& Craft of Memoir. Spiritual Quests: The Art \& Craft of Religious Writing. Paths of Resistance: The Art \& Craft of the Political Novel. Worlds of Childhood: The Art \& Craft of Writing for Children. They Went: The Art \& Craft of Travel Writing. Going on Faith: Writing as a Spiritual Quest}.

%------------------------------------------------------------------------------%

\section*{Introduction}
``When I 1st wrote this book, in 1976, the readers I had in mind were a relatively small segment of the population: students, writers, editors, \& people who wanted to learn to write. I wrote it on a typewriter, the highest technology then available. I had no inkling of the electronic marvels just around the corner that were about to revolutionize the act of writing. 1st came the word processor, in the 1980s, which made the computer an everyday tool for people who had never thought of themselves as writers. Then came the Internet \& e-mail, in the 1990s, which completed the revolution. Today everybody in the world is writing to everybody else, keeping in touch \& doing business across every border \& time zone.

To me this is nothing less than a miracle, curing overnight what appeared to be a deep American disorder. I've been repeated told by people in nonwriting occupations -- especially people in science, technology, medicine, business, \& finance -- that they hate writing \& can't write \& don't want to be made to write. 1 thing they particularly didn't want to write was letters. Just getting started on a letter loomed as a chore with so many formalities -- Where's the stationery? Where's the envelope? Where's the stamp? -- that they would keep putting it off, \& when they finally did sit down to write they would spend the entire 1st paragraph explaining why they hadn't written sooner. In the 2nd paragraph they would describe the weather in their part of the country -- a subject of no interest anywhere else. Only in the 3rd paragraph would they begin to relax \& say what they wanted to say.

Then along came e-mail \& all the formalities went away. E-mail has no etiquette. It doesn't require stationery, or neatness, or proper spelling, or preliminary chitchat. E-mail writers are like people who stop a friend on the sidewalk \& say, ``Did you see the game last night?'' WHAP! No amenities. They just start typing at full speed. So here's the miracle: All those people who said they hate writing \& can't write \& don't want to write \textit{can} write \& \textit{do} want to write. In fact, they can't be turned off. Never have so many Americans written so profusely \& with so few inhibitions. Which means that it wasn't a cognitive problem after all. It was a cultural problem, rooted in that old bugaboo of American education: fear.



'' -- \cite[pp. ix--]{Zinsser2001}

%------------------------------------------------------------------------------%

\begin{center}\huge
	Part I: Principles
\end{center}

%------------------------------------------------------------------------------%

\section{The Transaction}

%------------------------------------------------------------------------------%

\section{Simplicity}

%------------------------------------------------------------------------------%

\section{Clutter}

%------------------------------------------------------------------------------%

\section{Style}

%------------------------------------------------------------------------------%

\section{The Audience}

%------------------------------------------------------------------------------%

\section{Words}

%------------------------------------------------------------------------------%

\section{Usage}

%------------------------------------------------------------------------------%

\begin{center}\huge
	Part II: Methods
\end{center}

%------------------------------------------------------------------------------%

\section{Unity}

%------------------------------------------------------------------------------%

\section{The Lead \& the Ending}

%------------------------------------------------------------------------------%

\section{Bits \& Pieces}

%------------------------------------------------------------------------------%

\begin{center}\huge
	Part III: Forms
\end{center}

%------------------------------------------------------------------------------%

\section{Nonfiction as Literature}

%------------------------------------------------------------------------------%

\section{Writing About People: The Interview}

%------------------------------------------------------------------------------%

\section{Writing About Places: The Travel Article}

%------------------------------------------------------------------------------%

\section{Writing About Yourself: The Memoir}

%------------------------------------------------------------------------------%

\section{Science \& Technology}

%------------------------------------------------------------------------------%

\section{Business Writing: Writing in Your Job}

%------------------------------------------------------------------------------%

\section{Sports}

%------------------------------------------------------------------------------%

\section{Writing About the Arts: Critics \& Columnists}

%------------------------------------------------------------------------------%

\section{Humor}

%------------------------------------------------------------------------------%

\section{The Sound of Your Voice}

%------------------------------------------------------------------------------%

\section{Enjoyment, Fear, \& Confidence}

%------------------------------------------------------------------------------%

\section{The Tyranny of the Final Product}

%------------------------------------------------------------------------------%

\section{A Writer's Decisions}

%------------------------------------------------------------------------------%

\section{Write as Well as You Can}

%------------------------------------------------------------------------------%

\printbibliography[heading=bibintoc]

\end{document}