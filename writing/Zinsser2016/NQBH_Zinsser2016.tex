\documentclass{article}
\usepackage[backend=biber,natbib=true,style=authoryear]{biblatex}
\addbibresource{/home/nqbh/reference/bib.bib}
\usepackage{tocloft}
\renewcommand{\cftsecleader}{\cftdotfill{\cftdotsep}}
\usepackage[colorlinks=true,linkcolor=blue,urlcolor=red,citecolor=magenta]{hyperref}
\usepackage{algorithm,algpseudocode,amsmath,amssymb,amsthm,float,graphicx,mathtools}
\allowdisplaybreaks
\numberwithin{equation}{section}
\newtheorem{assumption}{Assumption}[section]
\newtheorem{conjecture}{Conjecture}[section]
\newtheorem{corollary}{Corollary}[section]
\newtheorem{definition}{Definition}[section]
\newtheorem{example}{Example}[section]
\newtheorem{lemma}{Lemma}[section]
\newtheorem{notation}{Notation}[section]
\newtheorem{principle}{Principle}[section]
\newtheorem{problem}{Problem}[section]
\newtheorem{proposition}{Proposition}[section]
\newtheorem{question}{Question}[section]
\newtheorem{remark}{Remark}[section]
\newtheorem{theorem}{Theorem}[section]
\usepackage[left=0.5in,right=0.5in,top=1.5cm,bottom=1.5cm]{geometry}
\usepackage{fancyhdr}
\pagestyle{fancy}
\fancyhf{}
\lhead{\small Sect.~\thesection}
\rhead{\small\nouppercase{\leftmark}}
\renewcommand{\sectionmark}[1]{\markboth{#1}{}}
\cfoot{\thepage}
\def\labelitemii{$\circ$}

\title{On Writing Well: The Classic Guide to Writing Nonfiction}
\author{William Zinsser}
\date{\today}

\begin{document}
\maketitle
\tableofcontents
\vspace{5mm}

%------------------------------------------------------------------------------%

\section*{Introduction}

\begin{center}\LARGE\sf
	\textbf{Part I: Principles}
\end{center}

\section{The Transaction}

%------------------------------------------------------------------------------%

\section{Simplicity}

%------------------------------------------------------------------------------%

\section{Clutter}

%------------------------------------------------------------------------------%

\section{Style}

%------------------------------------------------------------------------------%

\section{The Audience}

%------------------------------------------------------------------------------%

\section{Words}

%------------------------------------------------------------------------------%

\section{Usage}

%------------------------------------------------------------------------------%

\begin{center}\LARGE\sf
	\textbf{Part II: Methods}
\end{center}

\section{Unity}

%------------------------------------------------------------------------------%

\section{The Lead \& the Ending}

%------------------------------------------------------------------------------%

\section{Bits \& Pieces}

%------------------------------------------------------------------------------%

\begin{center}\LARGE\sf
	\textbf{Part III: Forms}
\end{center}

\section{Nonfiction as Literature}

%------------------------------------------------------------------------------%

\section{Writing About People: The Interview}

%------------------------------------------------------------------------------%

\section{Writing About Places: The Travel Article}

%------------------------------------------------------------------------------%

\section{Writing About Yourself: The Memoir}

%------------------------------------------------------------------------------%

\section{Science \& Technology}

%------------------------------------------------------------------------------%

\section{Business Writing: Writing in Your Job}

%------------------------------------------------------------------------------%

\section{Sports}

%------------------------------------------------------------------------------%

\section{Writing About the Arts: Critics \& Columnists}

%------------------------------------------------------------------------------%

\section{Humor}
``Humor is the secret weapon of the nonfiction writer. It's secret because so few writers realize that humor is often their best tool -- \& sometimes their only tool -- for making an important point.

If this strikes you as a paradox, you're not alone. Writers of humor live with the knowledge that many of their readers don't know what they are trying to do. I remember a reporter calling to ask how I happened to write a certain parody in \textit{Life}. At the end he said, ``Should I refer to you as a humorist? Or have you also written anything serious?''

The answers is that if you're trying to write humor, almost everything you do is serious. Few Americans understand this. We dismiss our humorists as triflers because they never settled down to ``real'' work. The Pulitzer Prizes go to authors like Ernest Hemingway \& William Faulkner, who are (God knows) serious \& therefore certified as men of literature. The prizes seldom go to people like George Ade, H. L. Mencken, Ring Lardner, S. J. Perelman, Art Buchwald, Jules Feiffer, Woody Allen \& Garrison Keillor, who seem to be just fooling around.

They're just just fooling around. They are as serious in purpose as Hemingway or Faulkner -- a national asset in forcing the country to see itself clearly. Humor, to them, is urgent work. It's an attempt to say important things in a special way that regular writers aren't getting said in a regular way -- or if they are, it's so regular that nobody is reading it.

1 strong editorial cartoon is worth a hundred solemn editorials. 1 \textit{Doonesbury} comic strip by Garry Trudeau is worth a thousand words of moralizing. 1 \textit{Catch-22} or \textit{Dr. Strangelove} is more powerful than all the books \& movies that try to show war ``as it is.'' Those 2 works of comic invention are still standard points of reference for anyone trying to warn us about the military mentality that could blow us all up tomorrow. Joseph Heller \& Stanley Kubrick heightened the truth about war just enough to catch its lunacy, \& we recognize it as lunacy. The joke is no joke.

This heightening of some crazy truth -- to a level where it will be seen as crazy -- is the essence of what serious humorists are trying to do. Here's 1 example of how they go about their mysterious work.

1 day in the 1960s I realized that half the girls \& women in America were suddenly wearing hair curlers. It was a weird new blight, all the more puzzling because I couldn't understand when the women took the curlers out. There was no evidence that they ever did -- they wore them to the supermarket \& to church \& on dates. So what was the wonderful event they were saving the wonderful hairdo for?

I tried for a year to think of a way to write about this phenomenon. I could have said ```It's an outrage'' or ``Have these women no pride?'' But that would have been a sermon, \& sermons are the death of humor. The writer must find some comic device -- satire, parody, irony, lampoon, nonsense -- that he can use to disguise his serious point. Very often he never finds it, \& the point doesn't get made.

Luckily, my vigil was at last rewarded. I was browsing at a newsstand \& saw 4 magazines side by side: \textit{Hairdo, Celebrity Hairdo, Combout} \& \textit{Pouf}. I bought all 4 -- to the alarm of the newsdealer -- \& found a whole world of journalism devoted solely to hair: life from the neck up, but not including the brain. The magazines had diagrams of elaborate roller positions \& columns in which a girl could send her roller problem to the editors for their advice. That was what I needed. I invented a magazine called \textit{Haircurl} \& wrote a series of parody letters \& replies. The piece ran in \textit{Life} \& it began like this:

'' -- \cite[pp. 195--]{Zinsser2016}

%------------------------------------------------------------------------------%

\begin{center}\LARGE\sf
	\textbf{Part IV: Attitudes}
\end{center}

\section{The Sound of Your Voice}
``I wrote 1 book about baseball \& one about jazz. But it never occurred to me to write 1 of them in sports English \& the other in jazz English. I tried to write them both in the best English I could, in my usual style. Though the books were widely different in subject, I wanted readers to know that they were hearing from the same person. It was \textit{my} book about baseball \& \textit{my} book about jazz. Other writers would write \textit{their} book. My commodity as a writer, whatever I'm writing about, is me. \& your commodity is you. Don't alter your voice to fit your subject. Develop 1 voice that readers will recognize when they hear it on the page, a voice that's enjoyable not only in its musical line but in its avoidance of sounds that would cheapen its tone: breeziness \& condescension \& clich\'es.

Let's start with breeziness.

There is a kind of writing that sounds so relaxed that you think you hear the author talking to you. E. B. White was probably its best practitioner, though many other masters of the style -- James Thurber, V. S. Pritchett, Lewis Thomas -- come to mind. I'm partial to it because it's a style I've always tried to write. The common assumption is that the style is effortless. In fact the opposite is true: the effortless style is achieved by strenuous effort \& constant refining. The nails of grammar \& syntax are in place \& the English is as good as the writer can make it.

Here's how a typical piece by E. B. White begins:
\begin{quotation}
	I spent several days \& night in mid-September with an ailing pig \& I feel driven to account for this stretch of time, more particularly since the pig died at last, \& I lived, \& things might easily have gone the other way around \& none left to do the accounting.
\end{quotation}
The sentence is so folksy that we imagine ourselves sitting on the porch of White's house in Maine. White is in a rocking chair, puffing on a pipe, \& the words just tumble out in his storyteller's voice. But look at the sentence again. Nothing about it is accidental. It's a disciplined act of writing. The grammar is formal, the words are plain \& precise, \& the cadences are those of a poet. That's the effortless style at its best: a methodical act of composition that disarms us with its generated warmth. The writer sounds confident; he's not trying to ingratiate himself with the reader.

Inexperienced writers miss this point. They think that all they have to do to achieve a casual effect is to be ``just folks'' -- good old Betty or Bob chatting over the back fence. They want to be a pal to the reader. They're so eager not to appear formal that they don't even try to write good English. What they write is the breezy style.

How would a breezy writer handle E. B. White's vigil with the pig? He might sound like this:
\begin{quotation}
	Every stay up late babysitting for a sick porker? Believe you me, a guy can lose a heckuva lot of shut-eye. I did that gig for 3 nights back in September \& my better half thought I'd lost my marbles. (Just kidding, Pam!) Frankly, the whole deal kind of bummed me out. Because, you see, the pig up \& died on me. To tell you the truth, I wasn't feeling in the pink myself, so I suppose it could have been yours truly \& not old Porky who kicked the bucket. \& you can bet your bottom dollar Mr. Pig wasn't going to write a book about it!
\end{quotation}
There's no need to labor all the reasons why this stuff is so terrible. It's crude. It's corny. It's verbose. It's contemptuous of the English language. It's condescending. (I stop reading writers who say ``You see.'') But the most pathetic thing about the breezy style is that it's harder to read than good English. In the writer's attempt to ease the reader's journey he has littered the path with obstacles: cheap slang, shoddy sentences, windy philosophizing. E. B. White's style is much easier to read. He knows that the tools of grammar haven't survived for so many centuries by chance; they are props the reader needs \& subconsciously wants. Nobody ever stopped reading E. B. White or V. S. Pritchett because the writing was too good. But readers will stop reading you if they think you are talking down to them. Nobody wants to be patronized.

Write with respect for the English language at its best -- \& for readers at their best. If you're smitten by the urge to try the breezy style, read what you've written aloud \& see if you like the sound of your voice.

Finding a voice that your readers will enjoy is largely a matter of taste. Saying that isn't much help -- taste is a quality so intangible that it can't even be defined. But we know it when we meet it. A woman with taste in clothes delights us with her ability to turn herself out in a combination that's not only stylish \& surprising, but exactly right. She knows what works \& what doesn't.

For writers \& other creative artists, knowing what \textit{not} to do is a major component of taste. 2 jazz pianists may be equally proficient. The one with taste will put every note to useful work in telling his or her story; the one without taste will drench us in ripples \& other unnecessary ornaments. Painters with taste will trust their eye to tell them what needs to be on the canvas \& what doesn't; a painter without taste will give us a landscape that's too pretty, or too cluttered, or too gaudy -- anyway, too something. A graphic designer with taste knows that less is more: that design is the servant of the written word. A designer without taste will smother the writing in background tints \& swirls \& decorative frills.

I realize I'm trying to pin down a matter that's subjective; 1 person's beautiful object is somebody else's kitsch. Taste can also change from 1 decade to another -- yesterday's charm is derided today as junk, but it will be back in vogue tomorrow, certified again as charming. \textit{So why do I even raise the issue?} Just to remind you that it exists. Taste is an invisible current that runs through writing, \& you should be aware of it.

Sometimes, in fact, it's visible. Every art form has a core of verities that survive the fickleness of time. There must be something innately pleasing in the proportions of the Parthenon; Western man continues to let the Greeks of 2000 years ago design his public buildings, as anyone walking around Washington, D.C., soon discovers. The fugues of Bach have a timeless elegance that's rooted in the timeless laws of mathematics.

Does writing have any such guideposts for us? Not many; writing is the expression of every person's individuality, \& we know that we like when it comes along. Again, however, much can be gained by knowing what to omit. Clich\'es, for instance. If a writer lives in blissful ignorance that clich\'es are the kiss of death, if in the final analysis he leaves no stone unturned to use them, we can infer that he lacks an instinct for what gives language its freshness. Faced with a choice between the novel \& the banal, he goes unerringly for the banal. His voice is the voice of a hack.

Not that clich\'es are easy to stamp out. They are everywhere in the air around us, familiar friends just waiting to be helpful, ready to express complex ideas for us in the shorthand form of metaphor. That's how they become clich\'es in the 1st place, \& even careful writers use quite a few on their 1st draft. But after that we are given a chance to clean them out. Clich\'es are 1 of the things you should keep listening for when you rewrite \& read your successive drafts aloud. Notice how incriminating they sound, convicting you of being satisfied to use the same old chestnuts instead of making an effort to replace them with fresh phrases of your own. Clich\'es are the enemy of taste.

Extend the point beyond individual clich\'es to your larger use of language. Again, freshness is crucial. Taste chooses words that have surprise, strength \& precision. Non-taste slips into the breezy vernacular of the alumni magazine's class notes -- a world where people in authority are the top brass or the powers that be. What exactly is wrong with ``the top brass''? Nothing -- \& everything. Taste knows that it's better to call people in authority what they are: officials, executives, chairmen, presidents, directors, managers. Non-taste reaches for the corny synonym, which has the further disadvantage of being imprecise; exactly \textit{which} company officers are the top brass? Non-taste uses ``umpteenth.'' \& ``zillions.'' Non-taste uses ``period'': ``She said she didn't want to hear any more about it. Period.''

But finally taste is a mixture of qualities that are beyond analyzing: an ear that can hear the difference between a sentence that limps \& a sentence that lilts, an intuition that knows when a casual or a vernacular phrase dropped into a formal sentence will not only sound right but will seem to be the inevitable choice. (E. B. White was a master of that balancing act.) Does this man that taste can be learned? Yes \& no. Perfect taste, like perfect pitch, is a gift from God. But a certain amount can be acquired. The trick is to study writers who have it.

Never hesitate to imitate another writer. Imitation is part of the creative process for anyone learning an art or a craft. Bach \& Picasso didn't spring full-blown as Bach \& Picasso; they needed models. This is especially true of writing. Find the best writers in the fields that interest you \& read their work aloud. Get their voice \& their taste into your ear -- their attitude toward language. Don't worry that by imitating them you'll lose your own voice \& your own identity. Soon enough you will shed those skins \& become who you are supposed to become.

By reading other writers you also plug yourself into a longer tradition that enriches you. Sometimes you will tap a vein of eloquence or racial memory that gives your writing a depth it could never attain on its own. Let me illustrate what I mean by a roundabout route.

Ordinarily I don't read the proclamations issued by state officials to designate important days of the year as important days of the year. But in 1976, when I was teaching at Yale, the governor of Connecticut, Ella Grasso, had the pleasant idea of reissuing the Thanksgiving Proclamation written 40 years earlier by Governor Wilbur Cross, which she called ``a masterpiece of eloquence.'' I often wonder whether eloquence has vanished from American life, or whether we even still consider it a goal worth striving for. So I studied Governor Cross's words to see how they had weathered the passage of time, that cruel judge of the rhetoric of earlier generations. I was delighted to find that I agreed with Governor Grasso. It was a piece written by a master:
\begin{quotation}
	Time out of mind at this turn of the seasons when the hardy oak leaves rustle in the wind \& the frost gives a tang to the air \& the dusk falls early \& the friendly evenings lengthen under the heel or Orion, it has seemed good to our people to join together in praising the Creator \& Preserver, who has brought us by a way that we did not know to the end of another year. In observance of this custom, I appoint Thursday, the 26th of November, as a day of Public Thanksgiving for the blessings that have been our common lot \& have placed our beloved stat with the favored regions of earth -- for all the creature comforts: the yield of the soil that has fed us \& the richer yield from labor of every kind that has sustained our lives -- \& for all those things, as dear as breath to the body, that quicken man's faith in his manhood, that nourish \& strengthen his word \& act; for honor held above price; for steadfast courage \& zeal in the long, long search after truth; for liberty \& for justice freely granted by each to his fellow \& so as freely enjoyed; \& for the crowning glory \& mercy of peace upon our land -- that we may humbly take heart of these blessings as we gather once again with solemn \& festive rites to keep our Harvest Home.
\end{quotation}
Governor Grasso added a postscript urging the citizens of Connecticut ``to renew their dedication to the spirit of sacrifice \& commitment which the Pilgrims invoked during their 1st harsh winter in the New World,'' \& I made a mental note to look at Orion that night. I was glad to be reminded that I was living in 1 of the favored regions of earth. I was also glad to be reminded that peace is not the only crowning glory to be thankful for. So is the English language when it is gracefully used for the public good. The cadences of Jefferson, Lincoln, Churchill, Roosevelt \& Adlai Stevenson came rolling down to me. (The cadences of Eisenhower, Nixon \& the 2 Bushes did not.)

I posted the Thanksgiving Proclamation on a bulletin board for my students to enjoy. From their comments I realized that several of them thought I was being facetious. Knowing my obsession with simplicity, they assumed that I regarded Governor Cross's message as florid excess.

The incident left me with several questions. Had I sprung Wilbur Cross's prose on a generation that had never been exposed to nobility of language as a means of addressing the populace? I couldn't recall a single attempt since John F. Kennedy's inaugural speech in 1961. (Mario Cuomo \& Jesse Jackson would partly restore my faith.) This was a generation reared on television, where the picture is valued more than the word -- where the word, in fact, is devalued, used as mere chatter \& often misused \& mispronounced. It was also a generation reared on music -- songs \& rhythms meant primarily to be heard \& felt. With so much noise in the air, was any American child being trained to listen? Was anyone calling attention to the majesty of a well-constructed sentence?

My other question raised a more subtle mystery: what is th line that separates eloquence from bombast? Why are we exalted by the words of Wilbur Cross \& anesthetized by the speeches of most politicians \& public officials who ply us with oratorical ruffles \& flourishes?

Part of the answer takes us back to taste. A writer with an ear for language will reach for fresh imagery \& avoid phrases that are trite. The hack will reach for those very clich\'es, thinking he will enrich his thoughts with currency that is, as he would put it, tried \& true. Another part of the answer lies in simplicity. Writing that will endure tends to consist of words that are short \& strong; words that sedate are words of 3, 4 \& 5  syllables, mostly of Latin origin, many of them ending in ``ion'' \& embodying a vague concept. In Wilbur Cross's Thanksgiving Proclamation there are no 4-syllable words \& only 10 3-syllable words, 3 of which are proper nouns he was stuck with. Notice how many of the governor's words are anything but vague: leaves, wind, frost, air, evening, earth, comforts, soil, labor, breath, body, justice, courage, peace, land, rites, home. They are homely words in the best sense -- they catch the rhythm of the seasons \& the dailiness of life. Also notice that all of them are nouns. After verbs, plain nouns are your strongest tools; they resonate with emotion.

But ultimately eloquence runs on a deeper current. It moves us with what it leaves unsaid, touching off echoes in what we already know from our reading, our religion \& our heritage. Eloquence invites us to bring some part of ourselves to the transaction. It was no accident that Lincoln's speeches resounded with echoes of the King James Bible; he knew it almost by heart from his boyhood, \& he had so soaked himself in its sonorities that his formal English was more Elizabethan than American. The 2nd Inaugural Address reverberates with Biblical phrases \& paraphrases: ``It may seem strange that any men should dare to ask a just God's assistance in wringing their bread from the sweat of other men's faces, but let us judge not, that we be not judged.'' The 1st half of the sentence borrows a metaphor from Genesis, the 2nd half reshapes a famous command in Matthew, \& ``a just God'' is from Isaiah.

If this speech affects me more than any other American document, it's not only because I know that Lincoln was killed 5 weeks later, or because I'm moved by all the pain that culminated in this plea for a reconciliation that would have malice toward none \& charity for all. It's also because Lincoln tapped some of Western man's oldest teachings about slavery, clemency \& judgment. His words carried stern overtones for the men \& women who heard him in 1865, reared, as he was, on the Bible. But even in the 21st century it's hard not to feel a wrath almost too ancient to grasp in Lincoln's notion that God might will the Civil War to continue ``until all the wealth piled by the bondsman's 250 years of unrequited toil shall be sunk, \& until every drop of blood drawn with the lash shall be paid by another drawn with the sword, as was said 3000 years ago, so still it must be said `the judgments of the Lord are true \& righteous altogether.''

Wilbur Cross's Thanksgiving Proclamation also echoes with trusts that we know in our bones. To such mysteries as the changing of the seasons \& the bounty of the earth we bring strong emotions of our own. Who hasn't looked with awe at Orion? To such democratic processes as ``the long search after truth'' \& ``liberty \& justice freely granted'' we bring fragments of our own searches after truth, our own grantings \& receivings, in a nation where so many human rights have been won \& so many still elude us. Governor Cross doesn't take our time to explain these processes, \& I'm grateful to him for that. I hate to think how many clich\'es a hack orator would marshal to tell us far more -- \& nourish us far less.

Therefore remember the uses of the past when you tell your story. What moves us in writing that has regional or ethnic roots -- Southern writing, African-American writing, Jewish-American writing -- is the sound of voices far older than the narrator's, talking in cadences that are more than ordinarily rich. Toni Morrison, 1 of the most eloquent of black writers, once said: ``I remember the language of the people I grew up with. Language was so important to them. All that power was in it. \& grace \& metaphor. Some of it was very formal \& Biblical, because the habit is that when you have something important to say you go into parable, if you're from Africa, or you go into another level of language. I wanted to use language that way, because my feeling was that a black novel was not black because I wrote it, or because there were black people in it, or because it was about black things. It was the style. It had a certain style. It was inevitable. I couldn't describe it, but I could produce it.''

Go with what seems inevitable in your own heritage. Embrace it \& it may lead you to eloquence.'' -- \cite[pp. 217--225]{Zinsser2016}

%------------------------------------------------------------------------------%

\section{Enjoyment, Fear, \& Confidence}
``As a boy I never wanted to grow up to be a writer, or -- God forbid -- an author. I wanted to be a newspaperman, \& the newspaper I wanted to be a man on was the \textit{New York Herald Tribune}. Reading it every morning, I loved the sense of enjoyment it conveyed. Everyone who worked on the papers -- editors, writers, photographers, make-up men -- was having a wonderful time. The articles usually had an extra touch of gracefulness, or humanity, or humor -- some gift of themselves that the writers \& editors enjoyed making to their readers. I thought they were putting out the paper just for me. To be 1 of those editors \& writers was my idea of the ultimate American dream.

The dream came true when I returned home from World War II \& got a job on the \textit{Herald Tribune} staff. I brought with me my belief that a sense of enjoyment is a priceless attribute for a writer or for a publication, \& I was now in the same room with the men \& women who had 1st put that idea in my head. The great reporters wrote with warmth \& gusto, \& the great critics \& columnists like Virgil Thomson \& Red Smith wrote with elegance \& with a mirthful confidence in their opinions. On the ``split page'' -- as the 1st of the 2nd section was called, when papers only had 2 sections -- the political column of Walter Lippmann, America's most venerated pundit, ran above the 1-panel cartoon by H. T. Webster, creator of ``The Timid Soul,'' who was also an American institution. I liked the insouciance that presented on the same page 2 features so different in gravity. Nobody thought of hustling Webster off to the comics section. Both men were giants, part of the same equation.

Among those blithe souls a city-desk reporter named John O'Reilly, who was admired for his deadpan coverage of human-interest \& animal-interest stories, managed to make whimsy a serious beat. I remember his annual article about the woolly bear, the caterpillar whose brown \& black stripes are said to foretell by their width whether the coming winter will be harsh or mild. Every fall O'Reilly would drive to Bear Mountain Park with the photographer Nat Fein, best known for his Pulitzer Prize-winning shot of Babe Ruth's farewell at Yankee Stadium, to observe a sample of woolly bears crossing the road, \& his article was written in mock-scientific museum-expedition style, duly portentous. The paper always ran the story at the bottom of page 1 under a 3-column head, along with a cut of a woolly bear, its stripes none too distinct. In the spring O'Reilly would write a follow-up piece telling his readers whether the woolly bears had been right, \& nobody blamed him -- or them -- if they hadn't. The point was to give everybody a good time.

Since then I've made that sense of enjoyment my credo as a writer \& an editor. Writing is such lonely work that I try to keep myself cheered up. If something strikes me as funny in the act of writing, I throw it in just to amuse myself. If I think it's funny I assume a few other people will find it funny, \& that seems to me to be a good day's work. It doesn't bother me that a certain number of readers will not be amused; I know that a fair chunk of the population has no sense of humor -- no idea that there are people in the world trying to entertain them.

When I was teaching at Yale I invited the humorist S. J. Perelman to talk to my students, \& 1 of them asked him, ``What does it take to be a comic writer?'' He said, ``It takes audacity \& exuberance \& gaiety, \& the most important one is audacity.'' Then he said: ``The reader has to feel that the writer is feeling good.'' The sentence went off in my head like a Roman candle: it started the entire case for enjoyment. Then he added: ``Even if he isn't.'' That sentence hit me almost as hard, because I knew that Perelman's life contained more than the usual share of depression \& travail. Yet he went to his typewriter every day \& made the English language dance. How could he not be feeling good? He cranked it up.

Writers have to jump-start themselves at the moment of performance, no less than actors \& dancers \& painters \& musicians. There are some writers who sweep us along so strongly in the current of their energy -- Norman Mailer, Tom Wolfe, Toni Morrison, William F. Buckley, Jr., Hunter Thompson, David Foster Wallace, Dave Eggers -- that we assume that when they go to work the words just flow. Nobody thinks of the effort they made every morning to turn on the switch.

You also have to turn on the switch. Nobody is going to do it for you.

Unfortunately, an equally strong negative current -- fear -- is at work. Fear of writing gets planted in most Americans at an early age, usually at school, \& it never entirely goes away. The blank piece of paper or the blank computer screen, waiting to be filled with our wonderful words, can freeze us into not writing any words at all, or writing words that are less than wonderful. I'm often dismayed by the sludge I see appearing on my screen if I approach writing as a task -- the day's work -- \& not with some enjoyment. My only consolation is that I'll get another shot at those dismal sentences tomorrow \& the next day \& the day after. With each rewrite I try to force my personality onto the material.

Probably the biggest fear for nonfiction writers is the fear of not being able to bring off their assignment. With fiction it's a different situation. Because authors of fiction are writing about a world of their own invention, often in an allusive style that thye have also invented (Thomas Pynchon, Don DeLillo), we have no right to tell them, ``That's wrong.'' We can only say, ``It doesn't work for me.'' Nonfiction writers get no such break. They are infinitely accountable: to the facts, to the people they interviewed, to the locale of their story \& to the events that happened there. They are also accountable to their craft \& all  its perils of excess \& disorder: losing the reader, confusing the reader, boring the reader, not keeping the reader engaged from beginning to end. With every inaccuracy of reporting \& every misstep of craft we can say, ``That's wrong.''

How can you fight off all those fears of disapproval \& failure? 1 way to generate confidence is to write about subjects that interest you \& that you care about. The poet Allen Ginsberg, another writer who came to Yale to talk to my students, was asked if there was a moment when he consciously decided to become a poet. Ginsberg said, ``It wasn't quite a choice -- it was a realization. I was 28 \& I had a job as a market researcher. 1 day I told my psychiatrist that what I really wanted to do was to quit my job \& just write poetry. \& the psychiatrist said, `Why not?' \& I said, `What would the American Psychoanalytical Association say?' \& he said, `There's no party line.' So I did.''

We'll never know how big a loss that was for the field of market research. But it was a big moment for poetry. There's no party line: good advice for writers. You can be your own party line. Red Smith, delivering the eulogy at the funeral of a fellow sports-writer, said, ``Dying is no big deal. Living is the trick.'' 1 of the reasons I admired Red Smith was that he wrote about sports for 55 years, with grace \& humor, without succumbing to the pressure, which was the ruin of many sportswriters, that he ought to be writing about something ``serious.'' He found in sportswriting what he wanted to do \& what he loved doing, \& because it was right for him he said more important things about American values than many writers  who wrote about serious subjects -- so seriously that nobody could read them.

Living is the trick. Writers who write interestingly tend to be men \& women who keep themselves interested. That's almost the whole point of becoming a writer. I've used writing to give myself an interesting life \& a continuing education. If you write about subjects you think you would enjoy knowing about, your enjoyment will show in what you write. Learning is a tonic.

That doesn't mean you won't be nervous when you go forth into unfamiliar terrain. As a nonfiction writer you'll be thrown again \& again into specialized worlds, \& you'll worry that you're not qualified to bring the story back. I feel that anxiety every time I embark on a new project. I felt it when I went to Bradenton to write my baseball book, \textit{Spring Training}. Although I've been a baseball fan all my life, I have never done any sports reporting, never interviewed a professional athlete. Strictly, I had no credentials; any of the men I approached with my notebook -- managers, coaches, players, umpires, scouts -- could have asked, ``What else have you written about baseball?'' But nobody did. They didn't ask because I had another kind of credential: sincerity. It was obvious to those men that I really wanted to know how they did their work. Remember this when you enter new territory \& need a shot of confidence. Your best credential is yourself.

Also remember that your assignment may not be as narrow as you think. Often it will turn out to touch some unexpected corner of your experience or your education, enabling you to broaden the story with strengths of your own. Every such reduction of the unfamiliar will reduce your fear.

That lesson was brought home to me in 1992 when I got a call from an editor at \textit{Audubon} asking if I could write an article for the magazine. I said I wouldn't. I'm a 4th-generation New Yorker, my roots deep in the cement. ``That wouldn't be right for me, or for you, or for \textit{Audubon},'' I told the editor. I've never accepted an assignment I didn't think I was suited for, \& I'm quick to tell editors that they should look for someone else. The \textit{Audubon} editor replied -- as good editors should -- that he was sure we could come up with something, \& a few weeks later he called to say that the magazine had decided it was time for a new article on Roger Tory Peterson, the man who made America a nation of birdwatchers, his \textit{Field Guide to the Birds} a best-seller since 1934. Was I interested? I said I didn't know enough about birds. The only one I can identify for sure is the pigeon, a frequent caller at my Manhattan windowsill.

I need to feel a certain rapport with the person I'll be writing about. The Peterson assignment wasn't one that I originated; it came looking for me. Almost every profile I've written has been of someone whose work I knew \& had an affection for: such creative souls as the cartoonist Chic Young (\textit{Blondie}), the songwriter Harold Arlen, the British actor Peter Sellers, the pianist Dick Hyman \& the British travel writer Norman Lewis. My gratitude for the pleasure of their company over the years was a source of energy when I sat down to write. If you want your writing to convey enjoyment, write about people you respect. Writing to destroy \& to scandalize can be as destructive to the writer as it is to the subject.

Something came up, however, that changed my mind about the \textit{Audubon} offer. I happened to see a PBS television documentary called \textit{A Celebration of Birds}, which summed up Roger Tory Peterson's life \& work. The film had so much beauty that I wanted to know more about him. What caught my attention was that Peterson was still going at full momentum at 84 -- painting 4 hours a day \& photographing birds in habitats all over the world. That \textit{did} interest me. Birds aren't my subject, but survivors are: how old people keep going. I remembered that Peterson lived in a Connecticut town not far from where our family goes in the summer. I could just drive over \& meet him; if the vibrations weren't right, nothing would be lost except a gallon of gas. I told the \textit{Audubon} editor I would try something informal -- ``a visit with Roger Tory Peterson,'' not a major profile.

Of course it did turn into a major profile, 4000 words long, because as soon as I saw Peterson's studio I realized that to think of him as an ornithologist, as I always had, was to miss the point of his life. He was above all an artist. It was his skill as a painter that had made his knowledge of birds accessible to millions \& had given him his authority as a writer, editor \& conversationist. I asked him about his early teachers \& mentors -- major American artists like John Sloan \& Edwin Dickinson -- \& about the influence of the great bird painters James Audubon \& Louis Agassiz Fuertes, \& my story became an art story \& a teaching story as well as a bird story, engaging many of my interests. It was also a survivor story; in his mid-80s Peterson was on a schedule that would tax a man of 50.

The moral for nonfiction writers is: think broadly about your assignment. Don't assume that an article for \textit{Audubon} has to be strictly about nature, or an article for \textit{Car \& Driver} strictly about cars. Push the boundaries of your subject \& see where it takes you. Bring some part of your own life to it; it's not your version of the story until you write it.

As for \textit{my} version of the Peterson story, not long after it ran in \textit{Audubon} my wife found a message on our home answering machine that said, ``Is this the William Zinsser who writes about nature?'' She thought it was hilarious, \& it was. But in fact my article was received by the birding community as a definitive portrait of Peterson. I mention this to give confidence to all nonfiction writers: a point of craft. If you master the tools of the trade -- the fundamentals of interviewing \& of orderly construction -- \& if you bring to the assignment your general intelligence \& your humanity, you can write about \textit{any} subject. That's your ticket to an interesting life.

Still, it's hard not to be intimidated by the expertise of the expert. You think, ``This man knows so much about his field, I'm too dumb to interview him. He'll think I'm stupid.'' The reason he knows so much about his field is because it's his field; you're a generalist trying to make his work accessible to the public. That means prodding him to clarify statements that are so obvious to him that he assumes they are obvious to everyone else. Trust your common sense to figure out what you need to know, \& don't be afraid to ask a dumb question. If the expert thinks you're dumb, that's his problem.

Your test should be: is the expert's 1st answer sufficient? Usually it's not. I learned that when I signed up for a 2nd expedition into Peterson territory. An editor at Rizzoli, the publisher of art books, called to say that the firm was preparing a coffee-table volume on ``The Art \& Photography of Roger Tory Peterson,'' with hundreds of color plates. An 8000-word text was needed, \& as the new Peterson authority I was asked to write it. Talk about hilarious.

I told the editor that I made it a point never to write the same story twice. I had written my \textit{Audubon} article as carefully as I could the 1st time \& wouldn't be able to rework it. He would be welcome, however, to acquire \& reprint my article in his book. He agreed to that if I would write an additional 4000 words -- invisible weaving -- that would deal mainly with Peterson's methods as an artist \& a photographer.

That sounded interesting, \& I went back to Peterson with a new set of questions, more technical than the ones I had put to him for \textit{Audubon}. That audience had wanted to hear about a life. Now I was writing for readers who wanted to know how the artist created his art, \& my questions got right down to process \& technique. We began with painting.

``I call my work `mixed media,''' Peterson told me, ``because my main purpose is to instruct. I may start with transparent watercolors, then I go to gouache, then I give it a protective coat of acrylic, then I go over that with acrylics or a touch of pastel, or colored pencil, or pencil, or ink -- anything that will do what I want.''

I knew from my earlier interview that Peterson's 1st answer was seldom sufficient. He was a taciturn man, the son of Swedish immigrants, not given to amplitude. I asked him how his present technique differed from his previous methods.

``Right now I'm straddling,'' he said. ``I'm trying to add more detail without losing the simplified effect.'' Then he stopped again.

But why did he feel that he needed more detail at this late point in his life?

``Over the years so many people have become familiar with the straight profile of my birds,'' he said, ``that they've begun to want something more: the look of feathers, or a more 3D feeling.''

After we got through with painting we moved on to photography. Peterson recalled every bird-shooting camera he ever owned, starting at age 13 with a Primo \#9, which used glass plates \& had bellows, \& ended with praise for such modern technology as auto-focus \& the fill-in flash. Not being a photographer, I had never heard of auto-focus or the fill-in flash, but I only had to reveal my stupidity to learn why they are so helpful. Auto-focus: ``If you can get the bird in your viewfinder the camera will do the rest.'' Fill-in flash: ``Film never sees as much as you see. The human eye sees detail in the shadows, but the fill-in flash enables the camera to pick up that detail.''

Technology, however, is only technology, Peterson reminded me. ``Many people think good equipment makes it easy,'' he said. ``They're deceived into thinking the camera does it all.'' He knew what he meant by that, but \textit{I} need to know why the camera doesn't do it all. When I pressed him with my ``Why not?'' \& my ``What else?'' I got not just 1 answer but 3:

``As a photographer, you bring your eye \& a sense of composition to the process, \& also warmth -- you don't shoot pictures at high noon, e.g., or at the beginning or the end of the day. You're also mindful of the quality of light; a thin overcast can do nice things. Knowledge of the animal is also a tremendous help: anticipating what a bird will do. You can anticipate such activities as a feeding frenzy, when birds feed on fish traveling in small groups. Feeding frenzies are important to a photographer because 1 of the basic things birds do is eat, \& they'll put up with you a lot longer if they're eating. In fact, they'll often ignore you.''

So we proceed, Mr. Expert \& Mr. Stupid, until I had extracted many ideas that I found interesting. ``I go halfway back to Audubon,'' Peterson said -- \textit{that} was interesting -- ``so I have a feeling for the changes that have taken place because of the environmental movement.'' In his boyhood, he recalled, every kid with a slingshot would shoot birds, \& many species had been killed off or brought close to extermination by hunters who slaughtered them for their plumes, or to sell to restaurants, or for sport. The good news, which he had lived long enough to see, was that many species had made a comeback from their narrow escape, helped by a citizenry that now takes an active role in protecting birds \& their habitats. Then he said: ``The attitude of people towards birds has changed the attitude of birds towards people.''

\textit{That} was interesting. I'm struck by how often as a writer I say to myself, ``That's interesting.'' If you find yourself saying it, pay attention \& follow your nose. Trust your curiosity to connect with the curiosity of your readers.

What did Peterson mean about birds changing their attitudes?

``Crows are becoming tamer,'' he said. ``Gulls have increased -- they're the cleanup crew at garbage dumps. The Least Tern has taken to nesting on top of shopping malls; a few years ago there were a thousand pair on the roof of the Singing River Mall in Gautier, Mississippi. Mockingbirds are particularly fond of malls -- they like the planting, especially the multiflora rose; its tiny hips are small enough for them to swallow. They also enjoy the bustle of shopping malls -- they sit there \& direct the traffic.''

We have been talking for several hours in Peterson's studio. The studio was a small outpost of the arts \& sciences -- easels, paints, paintbrushes, paintings, prints, maps, cameras, photographic equipment, tribal masks, \& shelves of reference books \& journals -- \& at the end of my visit, as he was walking me out, I said, ``Have I seen everything?'' Often you'll get your best material after you put your pencil away, in the chitchat of leave-taking. The person being interviewed, off the hook after the hard work of making his or her life presentable to a stranger, thinks of a few important afterthoughts.

When I asked whether I had seen everything, Peterson said, ``Would you like to see my collection of birds?'' I said I certainly would. He led me down an outside staircase to a cellar door, which he unlocked, ushering me into a basement full of cabinets \& drawers -- the familiar furniture of scientific storage, reminiscent of every small college museum that never got modernized. Darwin might have used such drawers.

``I've got 2000 specimens down here that I use for research,'' he told me. ``Most of them are around 100 years old, \& they're still useful.'' He pulled open a drawer \& took out a bird \& showed me the tag, which said \textsc{Acorn Woodpecker, Apr 10, 1882}. ``Think of it! This bird is 112 years old,'' he said. He opened some other drawers \& gently held several other late Victorians for me to ponder -- a link to the presidency of Grover Cleveland.

The Rizzoli book, with its stunning paintings \& photographs, was published in 1995, \& Peterson died a year later, his quest finally over, having sighted ``scarcely more than 4500'' of the world's 9000 species of birds. Did I enjoy the time I spent on the 2 articles? I can't really say I did; Peterson was too dour for that, not much fun. But I enjoyed having brought off a complicated story that took me outside my normal experience. I also had bagged a rare bird of my own, \& when I put Peterson away in a drawer with my other collected specimens, I thought: that was interesting.'' -- \cite[pp. 227--236]{Zinsser2016}

%------------------------------------------------------------------------------%

\section{The Tyranny of the Final Product}

%------------------------------------------------------------------------------%

\section{A Writer's Decisions}

%------------------------------------------------------------------------------%

\section{Writing Family History \& Memoir}

%------------------------------------------------------------------------------%

\section{Write as Well as You Can}

%------------------------------------------------------------------------------%

\printbibliography[heading=bibintoc]
	
\end{document}