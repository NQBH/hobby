\documentclass{article}
\usepackage[backend=biber,natbib=true,style=authoryear]{biblatex}
\addbibresource{/home/nqbh/reference/bib.bib}
\usepackage{tocloft}
\renewcommand{\cftsecleader}{\cftdotfill{\cftdotsep}}
\usepackage[colorlinks=true,linkcolor=blue,urlcolor=red,citecolor=magenta]{hyperref}
\usepackage{algorithm,algpseudocode,amsmath,amssymb,amsthm,float,graphicx,mathtools}
\allowdisplaybreaks
\numberwithin{equation}{section}
\newtheorem{assumption}{Assumption}[section]
\newtheorem{conjecture}{Conjecture}[section]
\newtheorem{corollary}{Corollary}[section]
\newtheorem{definition}{Definition}[section]
\newtheorem{example}{Example}[section]
\newtheorem{lemma}{Lemma}[section]
\newtheorem{notation}{Notation}[section]
\newtheorem{principle}{Principle}[section]
\newtheorem{problem}{Problem}[section]
\newtheorem{proposition}{Proposition}[section]
\newtheorem{question}{Question}[section]
\newtheorem{remark}{Remark}[section]
\newtheorem{theorem}{Theorem}[section]
\usepackage[left=0.5in,right=0.5in,top=1.5cm,bottom=1.5cm]{geometry}
\usepackage{fancyhdr}
\pagestyle{fancy}
\fancyhf{}
\lhead{\small Sect.~\thesection}
\rhead{\small\nouppercase{\leftmark}}
\renewcommand{\sectionmark}[1]{\markboth{#1}{}}
\cfoot{\thepage}
\def\labelitemii{$\circ$}

\title{On Writing Well: The Classic Guide to Writing Nonfiction}
\author{William Zinsser}
\date{\today}

\begin{document}
\maketitle
\tableofcontents
\vspace{5mm}

%------------------------------------------------------------------------------%

\section*{Introduction}

\begin{center}\LARGE\sf
	\textbf{Part I: Principles}
\end{center}

\section{The Transaction}

%------------------------------------------------------------------------------%

\section{Simplicity}

%------------------------------------------------------------------------------%

\section{Clutter}

%------------------------------------------------------------------------------%

\section{Style}

%------------------------------------------------------------------------------%

\section{The Audience}

%------------------------------------------------------------------------------%

\section{Words}

%------------------------------------------------------------------------------%

\section{Usage}

%------------------------------------------------------------------------------%

\begin{center}\LARGE\sf
	\textbf{Part II: Methods}
\end{center}

\section{Unity}

%------------------------------------------------------------------------------%

\section{The Lead \& the Ending}

%------------------------------------------------------------------------------%

\section{Bits \& Pieces}

%------------------------------------------------------------------------------%

\begin{center}\LARGE\sf
	\textbf{Part III: Forms}
\end{center}

\section{Nonfiction as Literature}

%------------------------------------------------------------------------------%

\section{Writing About People: The Interview}

%------------------------------------------------------------------------------%

\section{Writing About Places: The Travel Article}

%------------------------------------------------------------------------------%

\section{Writing About Yourself: The Memoir}

%------------------------------------------------------------------------------%

\section{Science \& Technology}

%------------------------------------------------------------------------------%

\section{Business Writing: Writing in Your Job}

%------------------------------------------------------------------------------%

\section{Sports}

%------------------------------------------------------------------------------%

\section{Writing About the Arts: Critics \& Columnists}

%------------------------------------------------------------------------------%

\section{Humor}
``Humor is the secret weapon of the nonfiction writer. It's secret because so few writers realize that humor is often their best tool -- \& sometimes their only tool -- for making an important point.

If this strikes you as a paradox, you're not alone. Writers of humor live with the knowledge that many of their readers don't know what they are trying to do. I remember a reporter calling to ask how I happened to write a certain parody in \textit{Life}. At the end he said, ``Should I refer to you as a humorist? Or have you also written anything serious?''

The answers is that if you're trying to write humor, almost everything you do is serious. Few Americans understand this. We dismiss our humorists as triflers because they never settled down to ``real'' work. The Pulitzer Prizes go to authors like Ernest Hemingway \& William Faulkner, who are (God knows) serious \& therefore certified as men of literature. The prizes seldom go to people like George Ade, H. L. Mencken, Ring Lardner, S. J. Perelman, Art Buchwald, Jules Feiffer, Woody Allen \& Garrison Keillor, who seem to be just fooling around.

They're just just fooling around. They are as serious in purpose as Hemingway or Faulkner -- a national asset in forcing the country to see itself clearly. Humor, to them, is urgent work. It's an attempt to say important things in a special way that regular writers aren't getting said in a regular way -- or if they are, it's so regular that nobody is reading it.

1 strong editorial cartoon is worth a hundred solemn editorials. 1 \textit{Doonesbury} comic strip by Garry Trudeau is worth a thousand words of moralizing. 1 \textit{Catch-22} or \textit{Dr. Strangelove} is more powerful than all the books \& movies that try to show war ``as it is.'' Those 2 works of comic invention are still standard points of reference for anyone trying to warn us about the military mentality that could blow us all up tomorrow. Joseph Heller \& Stanley Kubrick heightened the truth about war just enough to catch its lunacy, \& we recognize it as lunacy. The joke is no joke.

This heightening of some crazy truth -- to a level where it will be seen as crazy -- is the essence of what serious humorists are trying to do. Here's 1 example of how they go about their mysterious work.

1 day in the 1960s I realized that half the girls \& women in America were suddenly wearing hair curlers. It was a weird new blight, all the more puzzling because I couldn't understand when the women took the curlers out. There was no evidence that they ever did -- they wore them to the supermarket \& to church \& on dates. So what was the wonderful event they were saving the wonderful hairdo for?

I tried for a year to think of a way to write about this phenomenon. I could have said ```It's an outrage'' or ``Have these women no pride?'' But that would have been a sermon, \& sermons are the death of humor. The writer must find some comic device -- satire, parody, irony, lampoon, nonsense -- that he can use to disguise his serious point. Very often he never finds it, \& the point doesn't get made.

Luckily, my vigil was at last rewarded. I was browsing at a newsstand \& saw 4 magazines side by side: \textit{Hairdo, Celebrity Hairdo, Combout} \& \textit{Pouf}. I bought all 4 -- to the alarm of the newsdealer -- \& found a whole world of journalism devoted solely to hair: life from the neck up, but not including the brain. The magazines had diagrams of elaborate roller positions \& columns in which a girl could send her roller problem to the editors for their advice. That was what I needed. I invented a magazine called \textit{Haircurl} \& wrote a series of parody letters \& replies. The piece ran in \textit{Life} \& it began like this:

'' -- \cite[pp. 195--]{Zinsser2016}

%------------------------------------------------------------------------------%

\begin{center}\LARGE\sf
	\textbf{Part IV: Attitudes}
\end{center}

\section{The Sound of Your Voice}

%------------------------------------------------------------------------------%

\section{Enjoyment, Fear, \& Confidence}

%------------------------------------------------------------------------------%

\section{The Tyranny of the Final Product}

%------------------------------------------------------------------------------%

\section{A Writer's Decisions}

%------------------------------------------------------------------------------%

\section{Writing Family History \& Memoir}

%------------------------------------------------------------------------------%

\section{Write as Well as You Can}

%------------------------------------------------------------------------------%

\printbibliography[heading=bibintoc]
	
\end{document}