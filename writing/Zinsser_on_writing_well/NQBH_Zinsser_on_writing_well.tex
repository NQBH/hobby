\documentclass{article}
\usepackage[backend=biber,natbib=true,style=alphabetic]{biblatex}
\addbibresource{/home/nqbh/reference/bib.bib}
\usepackage{tocloft}
\renewcommand{\cftsecleader}{\cftdotfill{\cftdotsep}}
\usepackage[colorlinks=true,linkcolor=blue,urlcolor=red,citecolor=magenta]{hyperref}
\usepackage{algorithm,algpseudocode,amsmath,amssymb,amsthm,float,graphicx,mathtools}
\usepackage{enumitem}
\setlist{leftmargin=4mm}
\allowdisplaybreaks
\usepackage[left=1cm,right=1cm,top=5mm,bottom=5mm,footskip=4mm]{geometry}
\def\labelitemii{$\circ$}

\title{On Writing Well: The Classic Guide to Writing Nonfiction}
\author{William Zinsser}
\date{\today}

\begin{document}
\maketitle
\tableofcontents
\subsection*{Reference{\tt/}Writing Skills}

\begin{quotation}
	``\textit{On Writing Well} \cite{Zinsser2001, Zinsser2016} is a bible for a generation of writers looking for clues to clean, compelling prose.'' -- \textit{New York Times}
\end{quotation}
``\textit{On Writing Well} has been praised for its sound avice, its clarity \& the warmth of its style. It is a book for everybody who wants to learn how to write or who needs to do some writing to get through the day, as almost everybody does in the age of e-mail \& the Internet.

Whether you want to write about people or places, science \& technology, business, sports, the arts or about yourself in the increasingly popular memoir genre, \textit{On Writing Well} offers you fundamental principles as well as the insights of a distinguished writer \& teacher. With $>10^6$ copies sold, this volume has stood the test of time \& remains a valuable resource for writers \& would-be writers.''
\begin{quotation}
	``Not since \textit{The Elements of Style} has there been a guide to writing as well presented \& readable as this one. A love \& respect for the language is evident on every page.'' -- \textit{Library Journal}
\end{quotation}
\textsc{Books by William Zinsser.} \textit{Any Old Place With You. Seen Any Good Movies Lately? The City Dwellers. Weekend Guests. The Haircurl Papers. Pop Goes America. The Paradise Bit. The Lunacy Boom. On Writing Well. Writing With a Word Processor. Willie \& Dwike} (republished as \textit{Mitchell \& Ruff}). \textit{Writing to Learn. Spring Training. American Places. Speaking of Journalism. Easy to Remember}.

\noindent\textsc{Audio Books by William Zinsser.} \textit{On Writing Well. How to Write a Memoir}.

\noindent\textsc{Books Edited by William Zinsser.} \textit{Extraordinary Lives: The Art \& Craft of American Biography. Inventing the Truth: The Art \& Craft of Memoir. Spiritual Quests: The Art \& Craft of Religious Writing. Paths of Resistance: The Art \& Craft of the Political Novel. Worlds of Childhood: The Art \& Craft of Writing for Children. They Went: The Art \& Craft of Travel Writing. Going on Faith: Writing as a Spiritual Quest}.

%------------------------------------------------------------------------------%

\section*{Introduction}
``When I 1st wrote this book, in 1976, the readers I had in mind were a relatively small segment of the population: students, writers, editors, \& people who wanted to learn to write. I wrote it on a typewriter, the highest technology then available. I had no inkling of the electronic marvels just around the corner that were about to revolutionize the act of writing. 1st came the word processor, in the 1980s, which made the computer an everyday tool for people who had never thought of themselves as writers. Then came the Internet \& e-mail, in the 1990s, which completed the revolution. Today everybody in the world is writing to everybody else, keeping in touch \& doing business across every border \& time zone.

To me this is nothing less than a miracle, curing overnight what appeared to be a deep American disorder. I've been repeated told by people in nonwriting occupations -- especially people in science, technology, medicine, business, \& finance -- that they hate writing \& can't write \& don't want to be made to write. 1 thing they particularly didn't want to write was letters. Just getting started on a letter loomed as a chore with so many formalities -- Where's the stationery? Where's the envelope? Where's the stamp? -- that they would keep putting it off, \& when they finally did sit down to write they would spend the entire 1st paragraph explaining why they hadn't written sooner. In the 2nd paragraph they would describe the weather in their part of the country -- a subject of no interest anywhere else. Only in the 3rd paragraph would they begin to relax \& say what they wanted to say.

Then along came e-mail \& all the formalities went away. E-mail has no etiquette. It doesn't require stationery, or neatness, or proper spelling, or preliminary chitchat. E-mail writers are like people who stop a friend on the sidewalk \& say, ``Did you see the game last night?'' WHAP! No amenities. They just start typing at full speed. So here's the miracle: All those people who said they hate writing \& can't write \& don't want to write \textit{can} write \& \textit{do} want to write. In fact, they can't be turned off. Never have so many Americans written so profusely \& with so few inhibitions. Which means that it wasn't a cognitive problem after all. It was a cultural problem, rooted in that old bugaboo of American education: fear.

Fear of writing gets planted in American schoolchildren at an early age, especially children of scientific or technical or mechanical bent. They are led to believe that writing is a special language owned by the English teacher, available only to the humanistic few who have ``a gift for words.'' But writing isn't a skill that some people are born with \& others aren't, like a gift for art or music. Writing is talking to someone else on paper. Anybody who can think clearly can write clearly, about any subject at all. That has always been the central premise of this book.

On 1 level, therefore, the new fluency created by e-mail is terrific news. Any invention that eliminates the fear of writing is up there with air conditioning \& the lightbulb. But, as always, there's a catch. Nobody told all the new e-mail writers that the essence of writing is rewriting. Just because they are writing with ease \& enjoyment doesn't mean they are writing well.

That condition was 1st revealed in the 1980s, when people began writing on word processors. 2 opposite things happened. The word processor made good writers better \& bad writers worse. Good writers know that very few sentences come out right the 1st time or even the 3rd time or the 5th time. For them the word processor was a rare gift, enabling them to fuss endlessly with their sentences -- cutting \& revising \& reshaping -- without the drudgery of retyping. Bad writers became even more verbose because writing was suddenly so easy \& their sentences looked so pretty on the screen. How could such beautiful sentences not be perfect?

E-mail pushed that verbosity to a new extreme: chatter unlimited. It's a spontaneous medium, not conductive to slowing down or looking back. That makes it ideal for the never-ending upkeep of personal life: maintaining contact with far-flung children \& grandchildren \& friends \& long-lost classmates. If the writing is often garrulous or disorganized or not quite clear, no real harm is done.

But e-mail is also where much of the world's business is now conducted. Millions of e-mail messages every day give people the information they need to do their job, \& a badly written message can cause a lot of damage. Employers have begun to realize that they literally cannot afford to hire men \& women who can't write sentences that are tight \& logical \& clear. The new information age, for all its high-tech gadgetry, is, finally, writing-based. E-mail, the Internet \& the fax are all forms of writing, \& writing is, finally, a craft, with its own set of tools, which are words. Like all tools, they have to be used right.

\textit{On Writing Well} is a craft book. That's what I set out to write 25 years ago -- a book that would teach the craft of writing warmly \& clearly -- \& its principles have never changed; they are as valid in the digital age as they were in the age of the typewriter. I don't mean that the book itself hasn't changed. I've revised \& expanded it 5 times since 1976 to keep pace with new trends in the language \& in society: a far greater interest in memoir-writing, e.g., \& in writing about business \& science \& sports, \& in nonfiction writing by women \& by newcomers to the United States from other cultural traditions.

I'm also not the same person I was 25 years ago. Books that teach, if they have a long life, should reflect who the writer has become at later stages of his own long life -- what he has been doing \& thinking about. \textit{On Writing Well} \& I have grown older \& wiser together. In each of the 5 new editions the new material consisted of things I had learned since the previous edition by continuing to wrestle with the craft as a writer. As a teacher, I've become far more preoccupied with the intangibles of the craft -- the attitudes \& values, like enjoyment \& confidence \& intention, that keep us going \& produce our best work. But it wasn't until the 6th edition that I knew enough to write the 2 chapters (21 \& 22) that deal at proper length with those attitudes \& values.

Ultimately, however, good writing rests on craft \& always will. I don't know what still newer electronic marvels are waiting just around the corner to make writing twice as easy \& twice as fast in the next 25 years. But I do know they won't make writing twice as good. That will still require plain old hard work -- clear thinking -- \& the plain old tools of the English language. \textsc{William Zinsser}, Sep 2001.'' -- \cite[pp. ix--xii]{Zinsser2001}

``1 of the pictures hanging in my office in mid-Manhattan is a photograph of the writer E. B. White. It was taken by Jill Krementz when White was 77 years old, at his home in North Brooklin, Maine. A white-haired man is sitting on a plain wooden bench at a plain wooden table -- 3 boards nailed to 4 legs -- in a small boathouse. The window is open to a view across the water. White is typing on a manual typewriter, \& the only other objects are an ashtray \& a nail keg. The keg, I don't have to be told, is his wastebasket.

Many people from many corners of my life -- writers \& aspiring writers, students \& former students -- have seen that picture. They come to talk through a writing problem or to catch me up on their lives. But usually it doesn't take more than a few minutes for their eye to be drawn to the old man sitting at the typewriter. What gets their attention is the simplicity of the process. White has everything he needs: a writing implement, a piece of paper, \& a receptable for all the sentences that didn't come out the way he wanted them to.

Since then writing has gone electronic. Computers have replaced the typewriter, the delete key has replaced the wastebasket, \& various other keys insert, move, \& rearrange whole chunks of text. But nothing has replaced the writer. He or she is still stuck with the same old job of saying something that other people will want to read. That's the point of the photograph, \& it's still the point -- 30 years later -- of this book.

I 1st wrote \textit{On Writing Well} in an outbuilding in Connecticut that was as small \& as crude as White's boathouse. My tools were a dangling lightbulb, an Underwood standard typewriter, a ream of yellow copy paper \& a wire wastebasket. I had then been teaching my nonfiction writing course at Yale for 5 years, \& I wanted to use the summer of 1975 to try to put the course into a book.

E. B. White, as it happened, was very much on my mind. I had long considered him my model as a writer. His was the seemingly effortless style -- achieved, I knew, with great effort -- that I wanted to emulate, \& whenever I began a new project I would 1st read some White to get his cadences into my ear. But now I also had a pedagogical interest: White was the reigning champ of the arena I was trying to enter. \textit{The Elements of Style}, his updating of the book that had most influenced \textit{him}, written in 1919 by his English professor at Cornell, William Strunk, Jr., was the dominant how-to manual for writers. Tough competition.

Instead of competing with the Strunk \& White book I decided to complement it. \textit{The Elements of Style} was a book of pointers \& admonitions: do this, don't do that. What it \textit{didn't} address was how to apply those principles to the various forms that nonfiction writing \& journalism can take. That's what I taught in my course, \& it's what I would teach in my book: how to write about people \& places, science \& technology, history \& medicine, business \& education, sports \& that arts \& everything else under the sun that's waiting to be written about.

So \textit{On Writing Well} was born, in 1976, \& it's now in its 3rd generation of readers, its sales well over a million. Today I often meet young newspaper reporters who were given the book by the editor who hired them, just as those editors were 1st given the book by the editor who hired \textit{them}. I also often meet gray-haired matrons who remember being assigned the book in college \& not finding it the horrible medicine they expected. Sometimes they bring that early edition for me to sign, its sentences highlighted in yellow. They apologize for the mess. I love the mess.

As America has steadily changed in 30 years, so has the book. I've revised it 6 times to keep pace with new social trends (more interest in memoir, business, science, \& sports), new literary trends (more women writing nonfiction), new demographic patterns (more writers from other cultural traditions), new technologies (the computer) \& new words \& usages. I've also incorporated lessons I learned by continuing to wrestle with the craft myself, writing books on subjects I hadn't tried before: baseball \& music \& American history. My purpose is to make myself \& my experience available. If readers connect with my book it's because they don't think they're hearing from an English professor. They're hearing from a working writer.

My concerns as a teacher have also shifted. I'm more interested in the intangibles that produce good writing -- confidence, enjoyment, intention, integrity -- \& I've written new chapters on those values. Since the 1990s I've also taught an adult course on memoir \& family history at the New School. My students are men \& women who want to use writing to try to understand who they are \& what heritage they were born into. Year after year their stories take me deeply into their lives \& into their yearning to leave a record of what they have done \& thought \& felt. Half the people in America, it seems, are writing a memoir.

The bad news is that most of them are paralyzed by the size of the task. How can they begin to impose a coherent shape on the past -- that vast sprawl of half-remembered people \& events \& emotions? Many are near despair. To offer some help \& comfort I wrote a book in 2004 called \textit{Writing About Your Life}. It's a memoir of various events in my own life, but it's also a teaching book: along the way I explain the writing decisions I made. They are the same decisions that confront every writer going in search of his or her past: matters of selection, reduction, organization, \& tone. Now, for this 17th edition, I've put the lessons I learned into a new chapter called ``Writing Family History \& Memoir.''

When I 1st wrote \textit{On Writing Well}, the readers I had in mind were a small segment of the population: students, writers, editors, teachers, \& people who wanted to learn how to write. I had no inkling of the electronic marvels that would soon revolutionize the act of writing. 1st came the word processor, in the 1980s, which made the computer an everyday  tool for people who had never thought of themselves as writers. Then came the Internet \& e-mail, in the 1990s, which continued the revolution. Today everybody in the world is writing to everybody else, making instant contact across every border \& across every time zone. Bloggers are saturating the globe.

On 1 level the new torrent is good news. Any invention that reduces the fear of writing is up there with air-conditioning \& the lightbulb. But, as always, there's a catch. Nobody told all the new computer writers that the essence of writing is rewriting. Just because they're writing fluently doesn't mean they're writing well.

That condition was 1st revealed with the arrival of the word processor. 2 opposite things happened: good writers go better \& bad writers got worse. Good writers welcomed the gift of being able to fuss endlessly with their sentences -- pruning \& revising \& reshaping -- without the drudgery of retyping. Bad writers became even more verbose because writing was suddenly so easy \& their sentences looked so pretty on the screen. How could such beautiful sentences not be perfect?

E-mail is an impromptu medium, not conducive to slowing down or looking back. It's ideal for the never-ending upkeep of daily life. If the writing is disorderly, no real harm is done. But e-mail is also where much of the world's business is now conducted. Millions of e-mail messages every day give people the information they need to do their job, \& a badly written message can do a lot of damage. So can a badly written Web site. The new age, for all its electronic wizardry, is still writing-based.

\textit{On Writing Well} is a craft book, \& its principles haven't changed since it was written 30 years ago. I don't know what still newer marvels will make writing twice as easy in the next 30 years. But I do know they won't make writing twice as good. That will still require plain old hard thinking -- what E. B. White was doing in his boathouse -- \& the plain old tools of the English language. \textsc{William Zinsser}, Apr 2006.'' -- \cite[pp. 5--8]{Zinsser2016}

%------------------------------------------------------------------------------%

\begin{center}\huge
	Part I: Principles
\end{center}

%------------------------------------------------------------------------------%

\section{The Transaction}
``A school in Connecticut once held ``a day devoted to the arts,'' \& I was asked if I would come \& talk about writing as a vocation. When I arrived I found that a 2nd speaker had been invited -- Dr. Brock (as I'll call him), a surgeon who had recently begun to write \& had sold some stories to magazines. He was going to talk about writing as an avocation. That made us a panel, \& we sat down to face a crowd of students \& teachers \& parents, all eager to learn the secrets of our glamorous work.

Dr. Brock was dressed in a bright red jacket, looking vaguely bohemian, as authors are supposed to look, \& the 1st question went to him. What was it like to be a writer?

He said it was tremendous fun. Coming home from an arduous day at the hospital, he would go straight to his yellow pad \& write his tensions away. The words just flowed. It was easy. I then said that writing wasn't easy \& wasn't fun. It was hard \& lonely, \& the words seldom just flowed.

Next Dr. Brock was asked if it was important to rewrite. Absolutely not, he said. ``Let it all hang out,'' he told us, \& whatever form the sentences take will reflect the writer at his most natural. I then said that rewriting is the essence of writing. I pointed out that professional writers rewrite their sentences over \& over \& then rewrite what they have rewritten.

``What do you do on days when it isn't going well?'' Dr. Brock was asked. He said he just stopped writing \& put the work aside for a day when it would go better. I then said that the professional writer must establish a daily schedule \& stick to it. I said that writing is a craft, not an art, \& that the man who runs away from his craft because he lacks inspiration is fooling himself. He is also going broke.

``What if you're feeling depressed or unhappy?'' a student asked. ``Won't that affect your writing?''

Probably it will, Dr. Brock replied. Go fishing. Take a walk. Probably it won't, I said. If your job is to write every day, you learn to do it like any other job.

A student asked if we found it useful to circulate in the literary world. Dr. Brock said he was greatly enjoying his new life as a man of letters, \& he told several stories of being taken to lunch by his publisher \& his agent at Manhattan restaurants where writers \& editors gather. I said that professional writers are solitary drudges who seldom see other writers.

``Do you put symbolism in your writing?'' a student asked me.

``Not if I can help it,'' I replied. I have an unbroken record of missing the deeper meaning in any story, play or movie, \& as for dance \& mime, I have never had any idea of what is being conveyed.

``I \textit{love} symbols!'' Dr. Brock exclaimed, \& he described with gusto the joys of weaving them through his work.

So the morning went, \& it was a revelation to all of us. At the end Dr. Brock told me he was enormously interested in my answers -- it had never occurred to him that writing could be hard. I told him I was just as interested in \textit{his} answers -- it had never occurred to me that writing could be easy. Maybe I should take up surgery on the side.

As for the students, anyone might think we left them bewildered. But in fact we gave them a broader glimpse of the writing process than if only 1 of us had talked. For there isn't any ``right'' way to do such personal work. There are all kinds of writers \& all kinds of methods, \& any method that helps you to say what you want to say is the right method for you. Some people write by day, others by night. Some people need silence, others turn on the radio. Some write by hand, some by computer, some by talking into a tape recorder. Some people write their 1st draft in 1 long burst \& then revise; others can't write the 2nd paragraph until they have fiddled endlessly with the 1st.

But all of them are vulnerable \& all of them are tense. They are driven by a compulsion to put some part of themselves on paper, \& yet they don't just write what comes naturally. They sit down to commit an act of literature, \& the self who emerges on paper is far stiffer than the person who sat down to write. The problem is to find the real man or woman behind the tension.

Ultimately the product that any writer has to sell is not the subject being written about, but who he or she is. I often find myself reading with interest about a topic I never thought would interest me -- some scientific quest, perhaps. What holds me is the enthusiasm of the writer for his field. How was he drawn into it? What emotional baggage did he bring along? How did it change his life? It's not necessary to want to spend a year alone at Walden Pond to become involved with a writer who did.

This is the personal transaction that's at the heart of good nonfiction writing. Out of it come 2 of the most important qualities that this book will go in search of: humanity \& warmth. Good writing has an aliveness that keeps the reader reading from 1 paragraph to the next, \& it's not a question of gimmicks to ``personalize'' the author. It's a question of using the English language in a way that will achieve the greatest clarity \& strength.

Can such principles be taught? Maybe not. But most of them can be learned.'' -- \cite[pp. 12--13]{Zinsser2016}

%------------------------------------------------------------------------------%

\section{Simplicity}

%------------------------------------------------------------------------------%

\section{Clutter}

%------------------------------------------------------------------------------%

\section{Style}

%------------------------------------------------------------------------------%

\section{The Audience}

%------------------------------------------------------------------------------%

\section{Words}

%------------------------------------------------------------------------------%

\section{Usage}

%------------------------------------------------------------------------------%

\begin{center}\huge
	Part II: Methods
\end{center}

%------------------------------------------------------------------------------%

\section{Unity}

%------------------------------------------------------------------------------%

\section{The Lead \& the Ending}

%------------------------------------------------------------------------------%

\section{Bits \& Pieces}

%------------------------------------------------------------------------------%

\begin{center}\huge
	Part III: Forms
\end{center}

%------------------------------------------------------------------------------%

\section{Nonfiction as Literature}

%------------------------------------------------------------------------------%

\section{Writing About People: The Interview}

%------------------------------------------------------------------------------%

\section{Writing About Places: The Travel Article}

%------------------------------------------------------------------------------%

\section{Writing About Yourself: The Memoir}

%------------------------------------------------------------------------------%

\section{Science \& Technology}

%------------------------------------------------------------------------------%

\section{Business Writing: Writing in Your Job}

%------------------------------------------------------------------------------%

\section{Sports}

%------------------------------------------------------------------------------%

\section{Writing About the Arts: Critics \& Columnists}

%------------------------------------------------------------------------------%

\section{Humor}

%------------------------------------------------------------------------------%

\section{The Sound of Your Voice}

%------------------------------------------------------------------------------%

\section{Enjoyment, Fear, \& Confidence}

%------------------------------------------------------------------------------%

\section{The Tyranny of the Final Product}

%------------------------------------------------------------------------------%

\section{A Writer's Decisions}

%------------------------------------------------------------------------------%

\section{Write as Well as You Can}

%------------------------------------------------------------------------------%

\printbibliography[heading=bibintoc]

\end{document}